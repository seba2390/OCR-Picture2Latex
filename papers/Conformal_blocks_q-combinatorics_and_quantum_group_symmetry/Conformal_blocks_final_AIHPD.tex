\documentclass[oneside,english]{amsart}
\usepackage[T1]{fontenc}
\usepackage[latin9]{inputenc}
\usepackage{geometry}
\geometry{verbose,tmargin=3.00cm,bmargin=3.00cm,lmargin=2.75cm,rmargin=3.00cm,headheight=1cm,headsep=1.5cm,footskip=1.5cm}
\pagestyle{headings}
\setcounter{tocdepth}{2}%{3}
\setlength{\parskip}{\medskipamount}
\setlength{\parindent}{0pt}
\usepackage{verbatim}
\usepackage{amsthm}
\usepackage{amstext}
\usepackage{amssymb}
\usepackage{graphicx}
\usepackage{subfigure}
%\usepackage{xypic}
\usepackage{esint}
\usepackage{url}
\usepackage{yhmath}
\usepackage[all]{xy}
\usepackage{color}

\usepackage[percent]{overpic}


\usepackage[space]{grffile} % This allows file names with spaces
\makeatletter
%%%%%%%%%%%%%%%%%%%%%%%%%%%%%% Textclass specific LaTeX commands.
\numberwithin{equation}{section}
\numberwithin{figure}{section}
\theoremstyle{plain}
\newtheorem*{thm*}{\protect\theoremname}
\theoremstyle{plain}
\newtheorem{thm}{\protect\theoremname}%[section]
\renewcommand*{\thethm}{\Alph{thm}}
\theoremstyle{plain}
\newtheorem{lem}{\protect\lemmaname}
\theoremstyle{remark}
\newtheorem{rem}[lem]{\protect\remarkname}
\theoremstyle{plain}
\newtheorem{prop}[lem]{\protect\propositionname}
\theoremstyle{plain}
\newtheorem{cor}[lem]{\protect\corollaryname}
\theoremstyle{plain}
\newtheorem{conj}[lem]{\protect\conjecturename}
\theoremstyle{plain}
\newtheorem{proc}[lem]{\protect\procname}
\theoremstyle{plain}
\newtheorem{exa}[lem]{Example}
\theoremstyle{plain}
\newtheorem{defn}[lem]{Definition}
\theoremstyle{plain}
\newtheorem{notn}[lem]{Notation}
\newtheorem{algorithm}[lem]{Algorithm}
\theoremstyle{plain}
% \theoremstyle{plain}
% \newtheorem*{thm*}{\protect\theoremname}
% \theoremstyle{plain}
% \newtheorem{thm}{\protect\theoremname}[section]
% \theoremstyle{plain}
% \newtheorem{lem}[thm]{\protect\lemmaname}
% \theoremstyle{remark}
% \newtheorem{rem}[thm]{\protect\remarkname}
% \theoremstyle{plain}
% \newtheorem{prop}[thm]{\protect\propositionname}
% \theoremstyle{plain}
% \newtheorem{cor}[thm]{\protect\corollaryname}
% \theoremstyle{plain}
% \newtheorem{conj}[thm]{\protect\conjecturename}
% \theoremstyle{plain}
% \newtheorem{proc}[thm]{\protect\procname}
% \theoremstyle{plain}
% \newtheorem{exa}[thm]{Example}
% \theoremstyle{plain}
% \newtheorem{defn}[thm]{Definition}
% \theoremstyle{plain}
% \newtheorem{notn}[thm]{Notation}
% \newtheorem{algorithm}[thm]{Algorithm}
% \theoremstyle{plain}

%%%%%%%%%%%%%%%%%%%%%%%%%%%%%% User specified LaTeX commands.
\usepackage{mathrsfs}
\usepackage{subfigure}

%Notation for SLE
\newcommand{\SLE}{\mathrm{SLE}}
\newcommand{\SLEk}{\mathrm{SLE}_{\kappa}}
\newcommand{\SLEkappa}[1]{\mathrm{SLE}_{#1}}
\newcommand{\SLEkapparho}[2]{\mathrm{SLE}_{#1}(#2)}
\newcommand{\SLEmeasure}{\mathsf{P}}
\newcommand{\Primary}{\Psi}
\newcommand{\PrimaryBlock}[2]{\Psi_{#1}^{#2}}
\newcommand{\sigmaseq}{\bar{\sigma}}

%Probabilistic notation
\newcommand{\PR}{\mathsf{P}}
\newcommand{\EX}{\mathsf{E}}

%Calligraphic letters
\newcommand{\sF}{\mathcal{F}}
\newcommand{\sZ}{\mathcal{Z}}
\newcommand{\sD}{\mathcal{D}}
\newcommand{\sC}{\mathcal{C}}
\newcommand{\sL}{\mathcal{L}}
\newcommand{\sA}{\mathcal{A}}
\newcommand{\sR}{\mathcal{R}}
\newcommand{\sS}{\mathcal{S}}
\newcommand{\sP}{\mathcal{P}}

%Mathbb letters
\newcommand{\bR}{\mathbb{R}}
\newcommand{\R}{\bR}
\newcommand{\bRpos}{\mathbb{R}_{> 0}}
\newcommand{\bRnn}{\mathbb{R}_{\geq 0}}
\newcommand{\bZ}{\mathbb{Z}}
\newcommand{\Z}{\bZ}
\newcommand{\bN}{\mathbb{N}}
\newcommand{\N}{\bN}
\newcommand{\bZpos}{\mathbb{Z}_{> 0}}
\newcommand{\bZnn}{\mathbb{Z}_{\geq 0}}
\newcommand{\Zpos}{\bZpos}
\newcommand{\Znn}{\bZnn}
\newcommand{\bQ}{\mathbb{Q}}
\newcommand{\Q}{\bQ}
\newcommand{\bC}{\mathbb{C}}
\newcommand{\C}{\bC}

%Complex analysis notation
\newcommand{\Rsphere}{\overline{\bC}}
\newcommand{\bD}{\mathbb{D}}
\newcommand{\bH}{\mathbb{H}}
\newcommand{\re}{\Re\mathfrak{e}}
\newcommand{\im}{\Im\mathfrak{m}}
%\newcommand{\arg}{\mathrm{arg}}
\newcommand{\ii}{\mathfrak{i}}
\newcommand{\domain}{\Lambda}
\newcommand{\bdry}{\partial}
\newcommand{\cl}[1]{\overline{#1}}
\newcommand{\Mob}{\mu}
\newcommand{\confmap}{\phi}

%O-notation
\newcommand{\OO}{\mathcal{O}}
\newcommand{\oo}{\mathit{o}}

%Derivatives notation
\newcommand{\ud}{\mathrm{d}}
\newcommand{\der}[1]{\frac{\ud}{\ud#1}}
\newcommand{\pder}[1]{\frac{\partial}{\partial#1}}
\newcommand{\pdder}[1]{\frac{\partial^{2}}{\partial#1^{2}}}
\newcommand{\pddder}[1]{\frac{\partial^{3}}{\partial#1^{3}}}
\newcommand{\pddmix}[2]{\frac{\partial^{2}}{\partial#1 \partial#2}}

%Set theory notation
\newcommand{\set}[1]{\left\{  #1\right\}  }
\newcommand{\setcond}[2]{\left\{  #1\;\big|\;#2\right\}  }
\newcommand{\complementof}[1]{ {#1}^\compl}
\newcommand{\compl}{\mathrm{c}}


%Quantum group notation
\newcommand{\slLie}{\mathfrak{sl}}
\newcommand{\Uqsltwo}{{U}_{q}(\mathfrak{sl}_{2})}
% \newcommand{\Uqsltwo}{\mathscr{U}_{q}(\mathfrak{sl}_{2})}
% \newcommand{\Uqsltwo}{\mathcal{U}_{q}(\mathfrak{sl}_{2})}
\newcommand{\Hcp}{\Delta}
\newcommand{\qnum}[1]{\left[#1\right] }
\newcommand{\qfact}[1]{\left[#1\right]! }
\newcommand{\qbin}[2]{\left[\begin{array}{c} #1 \\ #2 \end{array}\right]}
\newcommand{\Wd}{\mathsf{M}}
\newcommand{\HWsp}{\mathsf{H}}
\newcommand{\HWspCal}{\mathcal{H}}
\newcommand{\Wbas}{\mathrm{e}}
% \newcommand{\Wbas}{e}
\newcommand{\Tbas}{\tau}
\newcommand{\MTbas}{\theta}
\newcommand{\Sbas}{\mathrm{s}}
% \newcommand{\Sbas}{s}
\newcommand{\TRbas}{\mathrm{t}}
% \newcommand{\TRbas}{t}

\newcommand{\vir}{\mathfrak{vir}}
\newcommand{\hwv}[1]{{#1}^\circ}
\newcommand{\dhwv}[1]{{#1}^\star}
\newcommand{\Qrep}{\mathcal{Q}}

\newcommand{\ldual}{\big\langle}
\newcommand{\rdual}{\big\rangle}
\newcommand{\dualbrakets}[1]{\left\langle #1 \right\rangle}
\newcommand{\dualpairing}[2]{\dualbrakets{ #1 , \, #2 }}

%\newcommand{\Puregeomtwodim}{\mathrm{v}}
\newcommand{\Puregeomtwodim}{v}
%\newcommand{\Puregeom}{\mathfrak{v}}
%\newcommand{\Bdryvec}{\Puregeom}%{\mathfrak{v}}
\newcommand{\Coblobastwodim}{\mathrm{u}}
% \newcommand{\Coblobastwodim}{u}
%\newcommand{\Coblobas}{\mathfrak{u}}

\newcommand{\Projection}{\mathfrak{p}_{L,R}}
\newcommand{\Embedding}{\mathfrak{I}_{L,R}}


%Plane partitions notation
\newcommand{\Arch}{\mathrm{LP}}
%\newcommand{\Arch{\mathfrak{A}\mathrm{rch}}
\newcommand{\LP}{\Arch}
\newcommand{\Conn}{\mathrm{Conn}}
\newcommand{\arch}[2]{[\wideparen{#1,#2}]}%\newcommand{\arch#1#2{\left[\wideparen{#1,#2}\right]}%\newcommand{\arch#1#2{\left[#1\phantom{}^{\frown}#2\right]}
\newcommand{\link}[2]{\{ #1 , #2 \}}
% \newcommand{\link}[2]{\arch{#1}{#2}}
\newcommand{\nested}{\boldsymbol{\underline{\Cap}}}
\newcommand{\unnested}{\boldsymbol{\underline{\cap\cap}}}
\newcommand{\walks}{\mathcal{W}}
\newcommand{\Catalan}{\mathrm{C}}

%Combinatorics for plane partitions notation
\newcommand{\sciOp}{\text{\Rightscissors}}
\newcommand{\tieOp}{\wp}
\newcommand{\removeArch}{/}
\newcommand{\removeLink}{\removeArch}
%\newcommand{\removeParen}{\removeArch}

%\newcommand{\KWleq{\overset{{\scriptsize ()}}{\leftarrow}}
%\newcommand{\KWleq{{\leftarrow\hspace{-2.55ex}{}^{()}\hspace{0.325ex}}}
%\newcommand{\KWleq}{{\leftharpoondown}}
\newcommand{\KWleq}{\stackrel{\scriptscriptstyle{()}}{\scriptstyle{\longleftarrow}}} % Sori, hairitsi ihan mahottomasti toi harppuunanotaatio niin vaihdoin :/ T. Alex
\newcommand{\KWgeq}{\stackrel{\scriptscriptstyle{()}}{\scriptstyle{\longrightarrow}}} % Sori, hairitsi ihan mahottomasti toi harppuunanotaatio niin vaihdoin :/ T. Alex
\newcommand{\Mmat}{\mathscr{M}}
\newcommand{\Minv}{\mathscr{M}^{-1}}
\newcommand{\genMmat}{\mathfrak{M}}
\newcommand{\genMinv}{\mathfrak{M}^{-1}}

%Algebra notation
\newcommand{\Hom}{\mathrm{Hom}}
\newcommand{\End}{\mathrm{End}}
\newcommand{\Aut}{\mathrm{Aut}}
\newcommand{\dmn}{\mathrm{dim}}
\newcommand{\spn}{\mathrm{span}}
\newcommand{\tens}{\otimes}
\newcommand{\unitmat}{\mathbb{I}}
\newcommand{\id}{\mathrm{id}}
\newcommand{\isom}{\cong}
\newcommand{\Kern}{\mathrm{Ker}}

%Integration contours, integrands etc. notation
\newcommand{\chamber}{\mathfrak{X}}
%\newcommand{\Wchamber{\mathfrak{W}}
%\newcommand{\FKcone{L\mathrm{-cone}}
%\newcommand{\CubeInt{\widetilde{\rho}}
%\newcommand{\SimplexInt{\rho}
\newcommand{\FWint}{\varphi}
%\newcommand{\AsyInt{\alpha}
%\newcommand{\DecoInt{\widetilde{\omega}}
%\newcommand{\SurfSimplex{\mathfrak{R}}
%\newcommand{\SurfCube{\SurfSimplex^{\approx}}
\newcommand{\SurfFW}{\mathfrak{L}^{\Supset}}
%\newcommand{\SurfAsy{\mathfrak{M}^{\sim\!\!\!\supset}}
%\newcommand{\SurfSimplexAnch{\mathfrak{R}^{(x_{0})}}
%\newcommand{\SurfCubeAnch{\SurfSimplex^{(x_{0});\approx}}
%\newcommand{\SurfFWAnch{\mathfrak{G}^{(x_{0});\Supset}}
%\newcommand{\fSimplex{f}
%\newcommand{\fCube{\fSimplex^{\approx}}
%\newcommand{\fFW{\fSimplex^{\Supset}}
%\newcommand{\fAsy{\fSimplex^{\sim\!\!\!\supset}}
% \newcommand{\PartF}{\mathscr{Z}}
\newcommand{\PartF}{\sZ}
\newcommand{\Sol}{\sS}
\newcommand{\ConvSet}{\sC}
\newcommand{\FKdual}{\sL}
\newcommand{\Quantumdual}{\psi}
\newcommand{\HarmMeas}{\mathsf{H}}
\newcommand{\HarmMeasH}{\mathcal{H}}
\newcommand{\ExcK}{\mathsf{K}}
\newcommand{\ExcKH}{\mathcal{K}}
\newcommand{\ExcKdom}{\mathcal{K}_\domain}
\newcommand{\FominDet}{\mathbf{\Delta}}
\newcommand{\LPdet}[2]{\FominDet_{#1}^{#2}}
%\newcommand{\LPdet#1#2{\FominDet_{#1}{#2}}
\newcommand{\ConfBlockFun}{\mathcal{U}}
\newcommand{\ConfBlockVec}{\mathrm{u}}
\newcommand{\hwvec}{\mathrm{w}}

%SLE boundary visits notation
\newcommand{\Ampl}{\zeta}
\newcommand{\Corr}{\chi}
%\newcommand{\lft{-}
%\newcommand{\rgt{+}
%\newcommand{\rgtlft{\pm}
\newcommand{\Orders}{\mathcal{O}}
\newcommand{\twoFone}{ {{}_2 F_1} }

%Miscellaneous notation
%\newcommand{\braid{\sigma}
%\newcommand{\twoFone{\,{}_{2}F_{1}}
\newcommand{\Pf}{\mathrm{Pf}}
%\newcommand{\sgn}{\mathrm{sign}}
\newcommand{\dist}{\mathrm{dist}}
\newcommand{\const}{\mathrm{const.}}
\newcommand{\eps}{\varepsilon}
\newcommand{\half}{\frac{1}{2}}
%%%%%%%%%%%%%%%%%%%%%%%

\newcommand{\Gr}{\mathcal{G}}
\renewcommand{\Vert}{\mathcal{V}}
\newcommand{\Edg}{\mathcal{E}}
\newcommand{\GrWired}{\Gr/\bdry}
\newcommand{\ein}{e_\mathrm{in}}
\newcommand{\eout}{e_\mathrm{out}}
\newcommand{\evis}{\hat{e}}
\newcommand{\evism}[2]{\evis_{#1 , #2}}
\newcommand{\GrNbhd}{\mathcal{U}}
\newcommand{\boxsize}{m}

\newcommand{\pin}{p_\mathrm{in}}
\newcommand{\pout}{p_\mathrm{out}}

\newcommand{\xin}{x_\mathrm{in}}
\newcommand{\xout}{x_\mathrm{out}}

\newcommand{\tree}{\mathcal{T}}
\newcommand{\branch}{\gamma}

\newcommand{\LE}{\mathrm{LE}}
%\newcommand{\RW{\gamma}
\newcommand{\RW}{\mathcal{X}}
%\newcommand{\LERW{\lambda}
\newcommand{\LERW}{\mathcal{L}}

\newcommand{\SymmGrp}{\mathfrak{S}}
\newcommand{\sgn}{\mathrm{sgn}}

\newcommand{\PPP}{\mathrm{PPP}}
\newcommand{\UST}{\mathcal{T}}

\newcommand{\MobF}{\mathfrak{m}}

\newcommand{\walksfromto}[2]{\mathscr{W}(#1,#2)}
\newcommand{\walkfromto}[3]{#1 \in \walksfromto{#2}{#3}}
%\newcommand{\walkfromto}[3]{#1 \text{ from } #2 \text{ to } #3}
\newcommand{\pathfromto}[2]{#1 \rightsquigarrow #2}

\newcommand{\edgeof}[2]{{\langle #1 , #2 \rangle}}

%%%%%%%%
% Some combinatorial crap added by Alex
\newcommand{\DP}{\mathrm{DP}}
\newcommand{\DPleq}{\preceq} % partial order of DPs
\newcommand{\DPgeq}{\succeq} % partial order of DPs
\newcommand{\BPE}{\mathrm{BPE}}
\newcommand{\BPEfont}[1]{\textup{\tt{#1}}}
\newcommand{\wedgeat}[1]{\lozenge_#1} % wedges of DPs
\newcommand{\upwedgeat}[1]{\wedge^#1}
\newcommand{\downwedgeat}[1]{\vee_#1}
\newcommand{\upwedge}{\wedge}
\newcommand{\downwedge}{\vee}
\newcommand{\upslope}{\nearrow}
\newcommand{\downslope}{\searrow}
\newcommand{\wedgelift}[1]{\uparrow \wedgeat{#1}} % wedge-lifting operation
\newcommand{\removewedge}[1]{\setminus \wedgeat{#1}}
\newcommand{\removeupwedge}[1]{\setminus \upwedgeat{#1}}
\newcommand{\removedownwedge}[1]{\setminus \downwedgeat{#1}}
\newcommand{\removeparen}[1]{\setminus \wedgeat{#1}}
\newcommand{\slopeat}[1]{\times_#1} % slope
\newcommand{\upslopeat}[1]{\upslope_#1} % slope
\newcommand{\downslopeat}[1]{\downslope^#1} % slope
\newcommand{\diag}{\mathrm{diag}} %diagonal matrix 
\newcommand{\indicator}{\mathbb{I}} % indicator fcn
\newcommand{\opar}{\rm{\url{(}}} % parenthesis notations have syntax {()()()}
\newcommand{\cpar}{\rm{\url{)}}}
\newcommand{\nestedtilingof}{T_0}
%\newcommand{\nestedtilingof}{\widehat{T}}
%\newcommand{\nestedtilingof}{\widehat{\hat{T}}}
%\newcommand{\nestedtilingof}{\mathcal{T}}
\newcommand{\Proba}{\mathbb{P}}
\newcommand{\CItilingsof}{\mathcal{C}}
\newcommand{\addLink}{\cup}

% Some combinatorial crap added by Eve
\newcommand{\walk}{\alpha}
\newcommand{\emptywalk}{{(0)}}
\newcommand{\nodef}{s}
\newcommand{\genDP}{\mathrm{GDP}}
\newcommand{\NormalizationConstant}{c}
\newcommand{\inversewalk}{\overset{{}_{\leftarrow}}{\walk}}

\newcommand{\power}{p}

%TODO: figure out good notation for number of links
\newcommand{\sizeofppp}[1]{||#1||}

% The below command allows making figures wider than the standard text width. with the below command, e.g., the code

%\begin{figure}
%\centerfloat
%\includegraphics[width = 1.3\textwidth ]{a very wide figure.pdf}
%\end{figure}

% will give a huge figure that still fits in the page.

\makeatletter
\newcommand*{\centerfloat}{%
  \parindent \z@
  \leftskip \z@ \@plus 1fil \@minus \textwidth
  \rightskip\leftskip
  \parfillskip \z@skip}
\makeatother



\AtBeginDocument{
  \def\labelitemii{\(\ast\)}
  \def\labelitemiii{\normalfont\bfseries{--}}
}

\makeatother

\usepackage{babel}
\providecommand{\corollaryname}{Corollary}
\providecommand{\lemmaname}{Lemma}
\providecommand{\propositionname}{Proposition}
\providecommand{\remarkname}{Remark}
\providecommand{\theoremname}{Theorem}
\providecommand{\conjecturename}{Conjecture}

\definecolor{kallecol}{rgb}{.75,.0,.55}
\newcommand{\kallemod}[1]{{\color{kallecol} #1}}

\definecolor{allucol}{rgb}{.3,.6,.2}
\newcommand{\alexmod}[1]{{\color{allucol} #1}}

\definecolor{blue}{rgb}{0,0,1}
\newcommand{\blue}[1]{{\color{blue} #1}}

\definecolor{red}{rgb}{1,0,0}
\newcommand{\red}[1]{{\color{red} #1}}

\definecolor{green}{rgb}{0,1,0}
\newcommand{\green}[1]{{\color{green} #1}}

\begin{document}

%\title{Conformal blocks and generalization of Fomin's formula}

% % % 
% % % 
\author{A.~Karrila, K.~Kytölä, and E.~Peltola}

\

\vspace{2.5cm}

\begin{center}
\LARGE \bf \scshape {
Conformal blocks, $q$-combinatorics, \\ and quantum group symmetry
% Conformal blocks, $q$-combinatorics, \\ and the quantum group $\Uqsltwo$
% % % % The $q$-combinatorics of conformal blocks
%Boundary correlations in planar LERW and UST %PDEs of CFT for LERW and UST  %Boundary correlations in planar LERW and UST
}%{Conformal blocks, pure partition functions, and parenthesis reversal relation}
\end{center}

\vspace{0.75cm}

\begin{center}
{\large \scshape Alex Karrila}\\
{\footnotesize{\tt alex.karrila@aalto.fi}}\\
{\small{Department of Mathematics and Systems Analysis}}\\
{\small{P.O. Box 11100, FI-00076 Aalto University, Finland}}\bigskip{}
% \bigskip{}
\\
{\large \scshape Kalle Kyt\"ol\"a}\\
{\footnotesize{\tt kalle.kytola@aalto.fi}}\\
{\small{Department of Mathematics and Systems Analysis}}\\
{\small{P.O. Box 11100, FI-00076 Aalto University, Finland}\\
\url{https://math.aalto.fi/~kkytola/}}\bigskip{}
% \bigskip{}
\\
{\large \scshape Eveliina Peltola}\\
{\footnotesize{\tt eveliina.peltola@unige.ch}}\\
{\small{Section de Math\'{e}matiques, Universit\'{e} de Gen\`{e}ve,}}\\
{\small{2--4 rue du Li\`{e}vre, Case Postale 64, 1211 Gen\`{e}ve 4, Switzerland}}
%1227 Les Acacias (GE), Switzerland
\end{center}

\vspace{0.75cm}

% % % 
% % % \blue{[Eve: I changed the notation $\HWsp_{2N}^{(0)}$ to $\HWspCal_N$ since it seemed unnecessarily complicated. 
% % % Is it distinguishable enough from $\HWsp_{n}^{(\nodef)}$ ?]}

\begin{center}
\begin{minipage}{0.85\textwidth} \footnotesize
{\scshape Abstract.}
In this article, we find a $q$-analogue for Fomin's formulas.
The original Fomin's formulas relate determinants of random walk
excursion kernels to loop-erased random walk partition functions,
and our formulas analogously relate conformal block functions of
conformal field theories to pure partition functions of multiple SLE
random curves.
We also provide a construction of the conformal block functions by a
method based on a quantum group, the $q$-deformation of $\slLie_2$.
The construction both highlights the representation theoretic origin of conformal
block functions and explains the appearance of $q$-combinatorial formulas.
% To explain the appearance of such $q$-combinatorial formulas,
% we also provide a construction of the conformal block functions by a
% method based on a quantum group, the $q$-deformation of $\slLie_2$.
\end{minipage}
\end{center}


%\vspace{0.75cm}
% \tableofcontents



\bigskip{}


% \input{CB-sec1-intro.tex}
% 


% % % \newpage 




\section{Introduction}

%{\color{red}[Eve: there was too much repetition of the phrase ``conformal field theory'' in the first paragraphs.
%With minor modifications, I re-wrote the first two paragraphs. The old is text is here (commented out; see tex code).]}
% %OLD VERSION HERE:
%Conformal blocks are fundamental building blocks of correlation functions of conformal field theories.
%In this article, we study the combinatorics of conformal block functions associated to 
%the simplest non-trivial primary fields in conformal field theories.
%
%
%Following the conventions in the literature about random conformally invariant curves of $\SLEk$ type, we parameterize
%the central charge of the conformal field theory via a parameter $\kappa > 0$, as
%\begin{align} \label{eq: central charge parametrization}
%c=\; & \frac{(3\kappa-8)(6-\kappa)}{2\kappa} . 
%\end{align}
%We assume $\kappa \in (0,8) \setminus \bQ$.
%The primary fields whose conformal blocks we study are of conformal weight
%\begin{align*}
%h = \frac{6 - \kappa}{2 \kappa} .
%\end{align*}
%This is the first non-trivial conformal weight in the Kac table~\cite{Kac-ICM_proceedings_Helsinki}, and
%fields of this type appear in particular as the boundary changing fields that
%create the tip of an $\SLEk$ type curve~\cite{BB-SLE_martingales_and_Virasoro_algebra, BB-CFTs_of_SLEs,
%FW-conformal_restriction_highest_weight_representations_and_SLE,
%BBK-multiple_SLEs, Dubedat-commutation, Kytola-local_mgales, KM-GFF_and_CFT,
%Dubedat-SLE_and_Virasoro_representations_localization}.
%
%
%We cover some background on conformal blocks in conformal field theories
%in Section~\ref{sec: conformal block functions background}.
%For all other parts of the article, a few key properties
%of conformal block functions can be taken as their definition.
%Namely, the partial differential equations, M\"obius covariance, and
%asymptotics given precisely in Section~\ref{sub: defining conformal block functions}
%serve as their defining properties.


% %NEW:
Conformal blocks are fundamental building blocks of correlation functions of conformal field theories.
In this article, we study the combinatorics of conformal block functions associated to 
the simplest non-trivial primary fields in conformal field theories (CFT).


Following the conventions in the literature about random conformally invariant curves of $\SLEk$ type, we parameterize
the central charge of the %conformal field theory 
CFT via a parameter $\kappa > 0$, as
\begin{align} \label{eq: central charge parametrization}
c=\; & \frac{(3\kappa-8)(6-\kappa)}{2\kappa} . 
\end{align}
We assume $\kappa \in (0,8) \setminus \bQ$.
The primary fields whose conformal blocks we study are of conformal weight
\begin{align*}
h = \frac{6 - \kappa}{2 \kappa} .
\end{align*}
This is the first non-trivial conformal weight in the Kac table~\cite{Kac-ICM_proceedings_Helsinki}, and
fields of this type appear in particular as the boundary changing fields that
create the tip of an $\SLEk$ type curve~\cite{BB-SLE_martingales_and_Virasoro_algebra, BB-CFTs_of_SLEs,
FW-conformal_restriction_highest_weight_representations_and_SLE,
BBK-multiple_SLEs, Dubedat-commutation, Kytola-local_mgales, KM-GFF_and_CFT,
Dubedat-SLE_and_Virasoro_representations_localization}.


We cover some background on conformal blocks in CFT %conformal field theories
in Section~\ref{sec: conformal block functions background}.
For all other parts of the article, a few key properties
of conformal block functions can be taken as their definition.
Namely, the partial differential equations, M\"obius covariance, and
asymptotics given precisely in Section~\ref{sub: defining conformal block functions}
serve as their defining properties. %(see Theorem~\ref{thm: change of basis theorem}).


%%%Conformal blocks are fundamental building blocks of correlation functions of conformal field theories.
%%%In this article, we study the combinatorics of conformal block functions associated to 
%%%the simplest non-trivial primary fields in conformal field theories.
%%%
%%%%\green{
%%%%A parameter $\kappa$ is fixed throughout, and it determines in particular
%%%%the central charge of the conformal field theory via
%%%%\begin{align} \label{eq: central charge parametrization}
%%%%c=\; & \frac{(3\kappa-8)(6-\kappa)}{2\kappa} . 
%%%%\end{align}
%%%%This follows the conventions in the literature about random conformally invariant curves of $\SLEk$ type.
%%%%%We will assume that 
%%%%\blue{We consider the generic case}
%%%%\begin{align*}
%%%%\kappa \in (0,8) \setminus \bQ .
%%%%\end{align*}
%%%%This assumption is made with the following motivation:
%%%%we need $\kappa \geq 0$ so that $\SLEk$ can be defined, that $\kappa \leq 8$ so that the $\SLEk$ curves have the properties of interfaces
%%%%in statistical mechanics systems, and that $\kappa$ is irrational for the underlying representation theory
%%%%of the Virasoro algebra and a hidden quantum group to be generic.
%%%%The primary fields whose conformal blocks %will be studied 
%%%%\blue{we study}
%%%%are of conformal weight
%%%%\begin{align*}
%%%%h = \frac{6 - \kappa}{2 \kappa} .
%%%%\end{align*}
%%%%This is the first non-trivial conformal weight in the Kac table~\cite{Kac-ICM_proceedings_Helsinki}\blue{[reference added]}, and
%%%%fields of this type appear in particular as the boundary changing fields that
%%%%create the tip of an $\SLEk$ type curve~\cite{BB-SLE_martingales_and_Virasoro_algebra, BB-CFTs_of_SLEs,
%%%%FW-conformal_restriction_highest_weight_representations_and_SLE, Kontsevich:CFT_SLE_and_phase_boundaries,
%%%%Friedrich-Kalkkinen:On_CFT_and_SLE,
%%%%BBK-multiple_SLEs, Dubedat-commutation, Kytola-local_mgales, KM-GFF_and_CFT,
%%%%Dubedat-SLE_and_Virasoro_representations_localization}.
%%%%}{\color{blue}[references added]}
%%%
%%%%\blue{[Eve: the above green paragraph is not very fluent... Here is a suggestion:]
%%%
%%%Following the conventions in the literature about random conformally invariant curves of $\SLEk$ type, we parameterize
%%%the central charge of the conformal field theory via a parameter $\kappa > 0$, as
%%%\begin{align} \label{eq: central charge parametrization}
%%%c=\; & \frac{(3\kappa-8)(6-\kappa)}{2\kappa} . 
%%%\end{align}
%%%We assume $\kappa \in (0,8) \setminus \bQ$.
%%%% % % % Throughout this article, the parameter $\kappa$ is assumed to take a generic value
%%%% % % % $\kappa \in (0,8) \setminus \bQ$.
%%%% % % % % \begin{align}
%%%% % % % % \kappa \in (0,8) \setminus \bQ .
%%%% % % % % \end{align}
%%%% % % % This assumption is made with the following motivation:
%%%% % % % we need $\kappa \geq 0$ so that $\SLEk$ can be defined, that $\kappa \leq 8$ so that the $\SLEk$ curves have the properties of interfaces
%%%% % % % in statistical mechanics systems, and that $\kappa$ is irrational for the underlying representation theory
%%%% % % % of the Virasoro algebra and a hidden quantum group to be generic.
%%%% % % % 
%%%The primary fields whose conformal blocks we study are of conformal weight
%%%\begin{align}
%%%h = \frac{6 - \kappa}{2 \kappa} .
%%%\end{align}
%%%This is the first non-trivial conformal weight in the Kac table~\cite{Kac-ICM_proceedings_Helsinki}, and
%%%fields of this type appear in particular as the boundary changing fields that
%%%create the tip of an $\SLEk$ type curve~\cite{BB-SLE_martingales_and_Virasoro_algebra, BB-CFTs_of_SLEs,
%%%FW-conformal_restriction_highest_weight_representations_and_SLE,
%%%BBK-multiple_SLEs, Dubedat-commutation, Kytola-local_mgales, KM-GFF_and_CFT,
%%%Dubedat-SLE_and_Virasoro_representations_localization}.
%%%%}
%%%
%%%
%%%We cover some background on conformal blocks in conformal field theories
%%%in Section~\ref{sec: conformal block functions background}.
%%%For all other parts of the article, a few key properties
%%%of conformal block functions can be taken as their definition.
%%%Namely, the partial differential equations, M\"obius covariance, and
%%%asymptotics given precisely in Section~\ref{sub: defining conformal block functions}
%%%serve as their defining properties.


\begin{figure}
\centerfloat
% \includegraphics[width = 0.2 \textwidth]{KW_matrices_printer-1.pdf} \qquad
% \includegraphics[width = 0.2 \textwidth]{numerical_weighted_KW_matrices_printer-1.pdf} \\
% \vspace{0.5 cm}
\includegraphics[width = 0.45 \textwidth]{pics-Dyck_paths_9-1.pdf} \qquad
\includegraphics[width = 0.45 \textwidth]{pics-Dyck_paths_9-3.pdf} \\
\caption{\label{fig: Dyck paths}
Examples of Dyck paths.
}
\end{figure}
The starting point for the combinatorics
is the observation that the conformal block functions are functions
\begin{align*}
\ConfBlockFun_\alpha (x_1 , \ldots , x_{2N})
\end{align*}
of an even number $n = 2N$ of variables, which are
indexed by Dyck paths $\alpha$ of length $2N$, that is,
sequences $\alpha = (\alpha(0) , \alpha(1) , \ldots , \alpha(2N))$ of non-negative integers
with $|\alpha(j) - \alpha(j-1)| = 1$ for all $j$ and $\alpha(0) = \alpha(2N) = 0$.
Figure~\ref{fig: Dyck paths} depicts examples of Dyck paths.
% Examples of Dyck paths are illustrated in Figure~\ref{fig: Dyck paths}.
% % Regarding the combinatorics of Dyck paths and related objects,
% % we follow the conventions of the article~\cite{KKP-boundary_correlations_in_planar_LERW_and_UST},
% % and we make use of combinatorial results from there as well.
% % We recall the needed notions in Section~\ref{sec: combinatorial preliminaries}.

% \blue{[Eve: give here the defining properties of conformal blocks, instead of Section~\ref{sub: defining conformal block functions} ?
% This only amounts to stating PDE, COV, ASY, right?]}

% % % % In particular, we denote the set of Dyck paths of lenght $2N$ by $\DP_N$.
% % % % Any two consecutive steps of a Dyck path $\alpha$ are said to form either a slope or a wedge,
% % % % according to the cases illustrated in Figure~\ref{fig: wedges and slopes}:
% % % % we say that $\alpha$ has a wedge at $j$ if $\alpha(j-1) = \alpha(j+1)$,
% % % % and that $\alpha$ has a slope at $j$ otherwise.
% % % % Moreover, a wedge at $j$ is called an up-wedge if $\alpha(j) = \alpha(j \pm 1) + 1$,
% % % % and correspondingly a down-wedge if $\alpha(j) = \alpha(j \pm 1) - 1$,
% % % % and in these two cases we respectively write $\upwedgeat{j} \in \alpha$ or $\downwedgeat{j} \in \alpha$.
% % % % By removing a wedge at $j$ from a Dyck path $\alpha \in \DP_N$ we obtain a shorter Dyck path
% % % % $\hat{\alpha} \in \DP_{N-1}$, namely
% % % % $\hat{\alpha} = (\alpha(0) , \alpha(1) , \ldots , \alpha(j-1) , \alpha(j+2) , \ldots , \alpha(2N))$.
% % % % According to whether the removed wedge is an up-wedge or a down-wedge, we denote
% % % % $\hat{\alpha} = \alpha \removeupwedge{j}$ or $\hat{\alpha} = \alpha \removedownwedge{j}$.
% % % % \begin{figure}
% % % % % \centerfloat
% % % % % \includegraphics[width = 0.2 \textwidth]{KW_matrices_printer-1.pdf} \qquad
% % % % % \includegraphics[width = 0.2 \textwidth]{numerical_weighted_KW_matrices_printer-1.pdf} \\
% % % % % \vspace{0.5 cm}
% % % % \centering
% % % % \subfigure[up-wedge]
% % % % {\qquad \includegraphics[width = 0.12 \textwidth]{pics-wedges_and_slopes-1.pdf} \qquad}
% % % % \subfigure[down-wedge]
% % % % {\qquad \includegraphics[width = 0.12 \textwidth]{pics-wedges_and_slopes-2.pdf} \qquad}
% % % % \subfigure[up-slope]
% % % % {\qquad \includegraphics[width = 0.12 \textwidth]{pics-wedges_and_slopes-3.pdf} \qquad}
% % % % \subfigure[down-slope]
% % % % {\qquad \includegraphics[width = 0.12 \textwidth]{pics-wedges_and_slopes-4.pdf} \qquad}
% % % % % \includegraphics[width = 0.12 \textwidth]{pics-wedges_and_slopes-1.pdf} \qquad
% % % % % \includegraphics[width = 0.12 \textwidth]{pics-wedges_and_slopes-2.pdf} \qquad
% % % % % \includegraphics[width = 0.12 \textwidth]{pics-wedges_and_slopes-3.pdf} \qquad
% % % % % \includegraphics[width = 0.12 \textwidth]{pics-wedges_and_slopes-4.pdf} \\
% % % % \caption{\label{fig: wedges and slopes}
% % % % Wedges and slopes.
% % % % }
% % % % \end{figure}
% % % % 
% % % % The defining properties of conformal block functions $\ConfBlockFun_\alpha$ are the following.
% % % % The functions $\ConfBlockFun_\alpha$ satisfy partial differential equations
% % % % \begin{align}
% % % % \label{eq: PDE for conformal blocks} \tag{PDE} 
% % % % & \left[ \frac{\kappa}{2} \pdder{x_j}
% % % %     + \sum_{i \neq j} \Big( \frac{2}{x_i-x_j} \pder{x_i} - \frac{2 h}{(x_i-x_j)^2} \Big) \right] \ConfBlockFun_\alpha (x_1 , \ldots, x_{2N}) = 0 \qquad \text{for all } j=1,\ldots,2N ,
% % % % \end{align}
% % % % M\"obius covariance
% % % % \begin{align}
% % % % \label{eq: COV for conformal blocks} \tag{COV} 
% % % % & \ConfBlockFun_\alpha(x_1 , \ldots, x_{2N}) = 
% % % %     \prod_{j=1}^{2 N} \Mob'(x_j)^{h} \times \ConfBlockFun_\alpha (\Mob(x_1) , \ldots, \Mob(x_{2N}))  \\
% % % % \nonumber
% % % % & \text{for all } \Mob(z) = \frac{a z + b}{c z + d}, \; \text{ with } a,b,c,d \in \bR, \; ad-bc > 0, 
% % % %  \text{ such that } \Mob(x_1) < \cdots < \Mob(x_{2N}) ,
% % % % \end{align}
% % % % and the following recursive asymptotics properties: for any 
% % % % $j \in \set{1, \ldots, 2N-1}$ and $\xi \in (x_{j-1}, x_{j+2})$,
% % % % \begin{align} \label{eq: ASY for conformal blocks} \tag{ASY} 
% % % % & \lim_{x_j , x_{j+1} \to \xi} 
% % % % \frac{\ConfBlockFun_\alpha^{(\kappa)} (x_1 , \ldots, x_{2N})}{(x_{j+1} - x_j)^{\Delta}}
% % % % = \begin{cases}
% % % % 0 & \text{if } \slopeat{j} \in \alpha \\
% % % %    \ConfBlockFun^{(\kappa)}_{\alpha \removeupwedge{j}} (x_1, \ldots, x_{j-1} , x_{j+2} , \ldots, x_{2N}) 
% % % % & \text{if } \upwedgeat{j} \in \alpha \\
% % % % -\frac{\qnum{\alpha(j)+1}}{\qnum{\alpha(j)+2}} \times
% % % % \ConfBlockFun^{(\kappa)}_{\alpha \removedownwedge{j}} (x_1, \ldots, x_{j-1} , x_{j+2} , \ldots, x_{2N})  & \text{if } \downwedgeat{j} \in \alpha
% % % %     \end{cases} ,
% % % % \end{align}
% % % % where the exponent of the asymptotics is $\Delta = 1-\frac{6}{\kappa}$, and the square bracket expressions are the $q$-integers
% % % % $\qnum{n} = \frac{q^n - q^{-n}}{q - q^{-1}}$
% % % % % \begin{align}\label{eq: q numbers}
% % % % % \qnum{m} = \; & \frac{q^{m}-q^{-m}}{q-q^{-1}}
% % % % %     =  q^{m-1}+q^{m-3}+\cdots+q^{3-m}+q^{1-m},
% % % % % \end{align}
% % % % with the parameter $q = e^{\ii \pi 4 / \kappa}$ depending on $\kappa$.
% % % % 

Our first main result, Theorem~\ref{thm: change of basis theorem} given in Section~\ref{sec: change of basis}, % {thm: conformal blocks and pure partition functions},
relates the conformal block functions via explicit $q$-combinatorial formulas to
another family of functions: the pure partition functions of multiple $\SLE$s~\cite{KP-pure_partition_functions_of_multiple_SLEs},
whose precise definition we recall in Section~\ref{sub: pure partition functions}.
The pure partition functions are a key ingredient in the construction of joint laws of $N$
curves of $\SLEk$ type, with deterministic
connectivity~\cite{BBK-multiple_SLEs, KP-pure_partition_functions_of_multiple_SLEs,
KKP-boundary_correlations_in_planar_LERW_and_UST, PH-Global_multiple_SLEs_and_pure_partition_functions}.
They are indexed by the planar connectivities, or equivalently, %also 
by Dyck paths.
%, and their definition will be recalled in Section~\ref{sub: pure partition functions}.
In the case $\kappa = 2$, 
a similar relation between the conformal block functions and the pure partition
functions arises as a consequence of Fomin's formulas~\cite{Fomin-LERW_and_total_positivity}
for loop-erased random walks, as explained in~\cite{KKP-boundary_correlations_in_planar_LERW_and_UST},
and our result can be seen as a $q$-analogue of Fomin's formulas.
%{\color{blue}In particular, we invite the reader to compare 
%Theorem~\ref{thm: change of basis theorem} and Equation~\eqref{eq: ASY for conformal blocks} 
%of the present article respectively with~\cite[Theorems~3.12~and~4.1]{KKP-boundary_correlations_in_planar_LERW_and_UST}
%and~\cite[Proposition~4.6]{KKP-boundary_correlations_in_planar_LERW_and_UST}.}
%%%%%, with the deformation parameter related to $\kappa$ via $q = e^{\ii 4 \pi / \kappa}$.
% The pure partition functions of multiple $\SLEk$ were constructed for $\kappa \in (0,8) \setminus \bQ$
% in~\cite{KP-pure_partition_functions_of_multiple_SLEs}:
% they are functions $\PartF_\alpha$ satisfying the same
% partial differential equations %~\eqref{eq: PDE for conformal blocks}
% and M\"obius covariance %~\eqref{eq: COV for conformal blocks}
% as the conformal block functions, but different asymptotics.
% % % % % \blue{[Eve: there are also results on $\kappa = 4$ in~\cite{PH-Global_multiple_SLEs_and_pure_partition_functions}, 
% % % % % but these are, of course, not known as Fomin's formulas. Should we still mention them?]} \alexmod{A: Then also mention the double dimer combinatorics \cite{KW-boundary_partitions_in_trees_and_dimers, KW-double_dimer_pairings_and_skew_Young_diagrams}?}

Specifically, we show that for fixed $N$,
% both families of functions $\big(\ConfBlockFun_\alpha\big)_{\alpha \in \DP_N}$
% and $\big(\PartF_\alpha\big)_{\alpha \in \DP_N}$
the conformal block functions and the multiple $\SLE$ pure partition functions %are shown to 
form two bases of the same function space of dimension given by the $N$:th Catalan number 
$\Catalan_N = \frac{1}{N+1} \binom{2N}{N}$,
and we give an explicit combinatorial formula for the change of basis matrix $\genMmat$
from the latter basis to the former, as well as for the inverse $\genMinv$.
The rows and columns of both $\genMmat$ and $\genMinv$ are indexed by Dyck paths,
and the entries are rational functions of $q = e^{\ii 4 \pi / \kappa}$.
The non-zero entries of $\genMmat$ appear where a binary relation introduced in
\cite{KW-double_dimer_pairings_and_skew_Young_diagrams, SZ-path_representations_of_maximal_paraboloc_KL_polynomials}
holds between the two Dyck paths, whereas the
non-zero entries of $\genMinv$ appear where the two Dyck paths are in the
natural partial order. Combinatorial formulas for the matrices %will be 
are
given in Section~\ref{sub: change of basis results}, but for small values of $N$
their forms are already illustrated in
Figures~\ref{fig: quantum KW matrices only} and~\ref{fig: quantum CIDT matrices only}.
% % already illustrate
% % their form. % in the cases $N=2$ and $N=3$.

% \blue{[Eve: it might be possible to give an informal statement of Theorem~\ref{thm: change of basis theorem} here.
% What do you think?]} \alexmod{[A: I vote no.]}

\begin{figure}
\centerfloat
% \includegraphics[width = 0.2 \textwidth]{KW_matrices_printer-1.pdf} \qquad
% \includegraphics[width = 0.2 \textwidth]{numerical_weighted_KW_matrices_printer-1.pdf} \\
% \vspace{0.5 cm}
% \includegraphics[width = 0.28 \textwidth]{KW_matrices_printer-2.pdf} \qquad
\includegraphics[width = 0.336 \textwidth]{q_analogues-2.pdf} %\\
\hspace{0.5 cm}
% \includegraphics[width = 0.45\textwidth]{KW_matrices_printer-3.pdf} \qquad
\includegraphics[width = 0.54\textwidth]{q_analogues-3.pdf} %\\
\caption{\label{fig: quantum KW matrices only}
The rows and columns of the matrix $\genMmat$ are indexed by Dyck
paths of $2N$ steps. The non-zero entries appear where a certain binary
relation --- the parenthesis reversal relation --- holds between the two Dyck paths.
This figure gives the explicit matrix elements
of $\genMmat$ in terms of $q = e^{\ii 4 \pi / \kappa}$
for %the cases 
$N=2$ and $N=3$.
% (Zero matrix elements are not drawn.)
}
\end{figure}
%
\begin{figure}
\centerfloat
% \includegraphics[width = 0.2 \textwidth]{KW_matrices_printer-1.pdf} \qquad
% \includegraphics[width = 0.2 \textwidth]{numerical_weighted_KW_matrices_printer-1.pdf} \\
% \vspace{0.5 cm}
% \includegraphics[width = 0.28 \textwidth]{inverse_KW_matrices_printer-2.pdf} \qquad
\includegraphics[width = 0.336 \textwidth]{q_analogues-4.pdf} %\\ 0.28
\hspace{0.5 cm}
% \includegraphics[width = 0.45\textwidth]{inverse_KW_matrices_printer-3.pdf} \qquad
\includegraphics[width = 0.54\textwidth]{q_analogues-6.pdf} %\\ 0.45
\caption{\label{fig: quantum CIDT matrices only}
The rows and columns of the matrix $\genMinv$ are indexed by Dyck
paths of $2N$ steps. The non-zero entries appear where the natural partial order
relation holds between the two Dyck paths: in particular, the matrix is upper triangular.
This figure gives the explicit matrix elements
of $\genMinv$ in terms of $q = e^{\ii 4 \pi / \kappa}$
for %the cases
$N=2$ and $N=3$.
% (Zero matrix elements are not drawn.)
% \blue{[Eve: the matrix elements in the figures are not very readable. Could we write them differently of make the figure bigger?]}
}
\end{figure}


The second main result of this article, Theorem~\ref{thm: conformal block functions}
given in Section~\ref{sec: construction of conformal block functions},
is a construction of the conformal block functions via the quantum group based method
of~\cite{KP-conformally_covariant_boundary_correlation_functions_with_a_quantum_group}.
Our construction expresses the conformal block functions as concrete linear combinations of
integrals of Coulomb gas type, similar to~\cite{DF-multipoint_correlation_functions}. 
It also reflects the underlying idea of conformal blocks, according to which
the Dyck path serves to label a sequence of intermediate representations.
% % % % % % % % % % The construction also explicitly exhibits
% % % % % % % % % % the representation theoretically fundamental property of conformal blocks that the Dyck path labels 
% % % % % % % % % % a sequence of intermediate representations.
% % % % \blue{[Eve: too vague --- might be confusing.]}
%  --- although in the construction the intermediate
% representations are of the hidden quantum group, a $q$-deformation of $\slLie_2$,
% rather than directly of the Virasoro algebra which accounts for conformal symmetry.





\subsection*{Acknowledgments}
We thank Steven Flores and David Radnell for interesting discussions.

A.K. and K.K. are supported by the Academy of Finland project
``Algebraic structures and random geometry of stochastic lattice models''. A.K. is also supported by the Vilho, Yrj\"{o} and Kalle V\"{a}is\"{a}l\"{a} Foundation.
E.P. is supported by the ERC AG COMPASP, the NCCR SwissMAP, and the Swiss NSF.




\bigskip

\section{Combinatorial preliminaries}
\label{sec: combinatorial preliminaries}

%\blue{In this section, we give some definitions and combinatorial results which are used in the subsequent sections. 
%We follow the notations and conventions of our previous article~\cite{KKP-boundary_correlations_in_planar_LERW_and_UST}.}

In this section, we recall some combinatorial definitions and  results. A complete account can be found in 
our previous article~\cite[Section~2]{KKP-boundary_correlations_in_planar_LERW_and_UST}, whose notations and conventions we follow.

\subsection{Dyck paths, skew Young diagrams, and Dyck tiles}
%\subsection{\blue{Skew Young diagrams and Dyck paths, tiles, and tilings}}

We denote by $\DP_N$ the set of \emph{Dyck paths} of $2N$ steps, i.e., %of 
sequences $\alpha = (\alpha(0), \alpha(1), \ldots, \alpha(2N))$ such that $\alpha(j) \in \Znn$ and $|\alpha(j) - \alpha(j-1)| = 1$
for all %$j=1,\ldots,2N$, 
$j \in \set{1,\ldots,2N}$, 
and $\alpha(0) = \alpha(2N)=0$. The number of such Dyck paths is a Catalan number,
\begin{align*}
\# \DP_N = \Catalan_N = \frac{1}{N+1} \binom{2N}{N} .
\end{align*}
We also denote by $\DP := \bigsqcup_{N \in \bZnn} \DP_N$ the set of
Dyck paths of arbitrary length.
% We also define the sets $\DP_0 := \set{\emptywalk}$ \alexmod{[A: should be empty walk $(0)$]} for $N = 0$,
% and $\DP := \bigsqcup_{N \in \bZnn} \DP_N$.

%The sets %$\DP_N$
%of Dyck paths have natural partial orderings:
For each $N$, the set of Dyck paths of $2N$ steps has a natural partial ordering:
for $\alpha, \beta \in \DP_N$ we denote $\alpha \DPleq \beta$
if and only if $\alpha(j) \leq \beta(j)$ for all %$j=0,1,\ldots,2N$.
$j \in \set{0,1,\ldots,2N}$.
When $\alpha \DPleq \beta$, the area between the Dyck paths $\alpha$ and $\beta$
forms a \emph{skew Young diagram}, denoted by $\alpha / \beta$.

The main combinatorial objects for the present article are certain tilings of
skew Young diagrams,
called Dyck tilings.
The tiles $t$ in these tilings are skew Young diagrams of a particular
type: namely $t = \alpha/\beta$ such that for some $0 < x_t \leq x'_t < 2N$
and $h_t \in \Zpos$ we have
\begin{align*}
& \begin{cases}
\alpha(j) = \beta(j) & \text{ for } 0 \leq j < x_t \\
\alpha(j) = \beta(j) -2 & \text{ for } x_t \leq j \leq x'_t \\
\alpha(j) = \beta(j) & \text{ for } x'_t < j \leq 2N 
\end{cases}
\end{align*}
and
\begin{align*}
%\qquad \text{ and } \qquad
& \alpha(x_t-1) = \beta(x_t-1) = \alpha(x_t+1) = \beta(x_t+1) = h_t .
\end{align*}
Such tiles $t=\alpha/\beta$ are called \emph{Dyck tiles}, the number $h_t$
% \begin{align*}
% h_t = \alpha(x_t-1) = \beta(x_t-1) = \alpha(x_t+1) = \beta(x_t+1) 
% \end{align*}
is called the \emph{height} of $t$, and the intervals $[x_t , x_t']$
and $(x_t-1, x_t+1)$ are called the \emph{horizontal extent} and \emph{shadow} of $t$,
respectively. Figure~\ref{fig: extent shadow and height} illustrates these notions.
We say that a Dyck tile $t_2 = \alpha_2 / \beta_2$ \emph{covers} another Dyck
tile $t_1 = \alpha_1 / \beta_1$ if there exists a $j$ such that
$j \in [x_{t_1} , x'_{t_1}] \cap [x_{t_2} , x'_{t_2}]$ and $\alpha_1(j) < \alpha_2(j)$.
% We say that a Dyck tile $\tilde{t} = \tilde{\alpha} / \tilde{\beta}$ \emph{covers} another Dyck
% tile $t = \alpha / \beta$ if there exists a $j$ such that
% $j \in [x_t , x'_t] \cap [x_{\tilde{t}} , x'_{\tilde{t}}]$ and $\tilde{\alpha}_j < \alpha(j)$.
% \blue{[Eve: is tilde a good notation here?]}
\begin{figure}[h!]
\includegraphics[width = 0.4\textwidth]{extent_and_shadow-2.pdf} \hspace{1cm}
\includegraphics[width = 0.4\textwidth]{extent_and_shadow-1.pdf}
\caption{\label{fig: extent shadow and height} The vertical position of a
Dyck tile $t$ is described by the integer height $h_t$.
The horizontal extent $[x_t , x'_t]$ (in red) and shadow $(x_t-1 , x'_t +1)$
(in blue) are intervals that describe the 
horizontal position.
The shape of a Dyck tile is essentially that of a %shorter
Dyck path, as illustrated by the black path drawn inside the tile.
% \blue{[Eve: is this figure used in our previous paper?]}
}
\end{figure}

In general, a \emph{Dyck tiling} $T$ of a skew Young diagram $\alpha/\beta$ is a collection
of Dyck tiles $t$ which cover the area of the skew Young diagram,
$\bigcup_{t \in T} t = \alpha/\beta$, and which have no overlap. 
%We will specifically consider 
Specifically, we consider
so called nested Dyck tilings and cover-inclusive Dyck tilings illustrated
in Figures~\ref{fig: NDT examples} and~\ref{fig: CIDT examples} %, respectively, 
and defined below 
in Sections~\ref{sub: nested  DT} and~\ref{sub: cover-inclusive DT}, respectively.



\subsection{Nested Dyck tilings and the parenthesis reversal relation}
\label{sub: nested  DT}

A Dyck tiling $T$ of a skew Young diagram $\alpha/\beta$ is said to be a \emph{nested
Dyck tiling} if the shadows of any two distinct tiles of $T$ are either disjoint
or one contained in the other, and in the latter case the tile with the larger
shadow covers the other. Figure~\ref{fig: NDT examples} exemplifies nested Dyck
tilings. It is not difficult to see that if a skew Young diagram $\alpha/\beta$ admits a
nested Dyck tiling, such a tiling is necessarily unique.
In this case, we write \[ \alpha \KWleq \beta , \] and 
we denote the nested Dyck tiling of $\alpha/\beta$ %will be denoted 
by $\nestedtilingof (\alpha / \beta)$.
This binary relation~$\KWleq$ on $\DP_N$ was first introduced
in~\cite{KW-double_dimer_pairings_and_skew_Young_diagrams, SZ-path_representations_of_maximal_paraboloc_KL_polynomials},
and we %will 
call it the \emph{parenthesis reversal relation}, because of a
convenient characterization it has in terms of balanced parenthesis expressions,
see~\cite[Lemma~2.7]{KKP-boundary_correlations_in_planar_LERW_and_UST}.
%
\begin{figure}
\includegraphics[width = 0.4\textwidth]{Dyck_tilings_nested-6.pdf} \quad
\includegraphics[width = 0.4\textwidth]{Dyck_tilings_nested-17.pdf}
%\includegraphics[width = 0.35\textwidth]{NDT_with_more_space.pdf}
\caption{\label{fig: NDT examples} Examples of nested Dyck tilings of skew Young diagrams.}
\end{figure}

\subsection{Cover-inclusive Dyck tilings}
\label{sub: cover-inclusive DT}

A Dyck tiling $T$ of a skew Young diagram $\alpha/\beta$ is said to be a \emph{cover-inclusive
Dyck tiling} if for any two distinct tiles of $T$, either the horizontal
extents are disjoint, or the tile that covers the other has horizontal extent contained in the horizontal
extent of the other. Figure~\ref{fig: CIDT examples} exemplifies cover-inclusive
Dyck tilings. In contrast with nested Dyck tilings, any skew Young diagram 
has cover-inclusive Dyck tilings.
% admits a cover-inclusive Dyck tiling, and such a tiling is not necessarily unique.
For $\alpha \DPleq \beta$, the set of cover-inclusive Dyck tilings of $\alpha/\beta$ is
denoted by $\CItilingsof (\alpha/\beta)$.
%
\begin{figure}
\includegraphics[width = 0.4\textwidth]{Dyck_tilings_CIDTa-18.pdf} \quad
% \includegraphics[width = 0.4\textwidth]{Dyck_tilings_CIDTa-6.pdf} \\
% \includegraphics[width = 0.4\textwidth]{Dyck_tilings_CIDTd-14.pdf} \quad
\includegraphics[width = 0.4\textwidth]{Dyck_tilings_CIDTf-6.pdf} \\
%\includegraphics[width = 0.35\textwidth]{NDT_with_more_space.pdf}
\caption{\label{fig: CIDT examples} Examples of cover-inclusive Dyck tilings of skew Young diagrams.}
\end{figure}

\subsection{Weighted incidence matrices and their inversion}

The incidence matrix of the binary relation~$\KWleq$ on the set $\DP_N$ of Dyck paths
plays a role in the combinatorics of dimers and groves~\cite{KW-double_dimer_pairings_and_skew_Young_diagrams},
and of uniform spanning tree boundary branches~\cite{KKP-boundary_correlations_in_planar_LERW_and_UST}.
The rows and columns of this incidence matrix are indexed by Dyck paths, and its entries
are $1$ or $0$ 
%\blue{one or zero}
according to whether 
or not
the relation $\KWleq$ holds between the two paths.
It turns out that an appropriately weighted incidence matrix is relevant for the combinatorics of conformal blocks. 
%\green{
%We %will 
%follow the convention of~\cite{KKP-boundary_correlations_in_planar_LERW_and_UST}
%of including signs in front of the weights in the weighted incidence matrix, in order to
%avoid signs in the inverse matrix.
%}
%\blue{[Eve: is this sentence useful? I would remove it.]}

Suppose that a weight $w(t) \in \bC$ has been %given 
assigned
to each %possible 
Dyck tile $t$.
% We set the weight associated to a collection of tiles %is set 
% to be the product of the weights of the tiles in it, and
We define the weighted incidence matrix by setting
for all $\alpha, \beta \in \DP_N$
\begin{align} \label{eq: def of weighted incidence matrix}
M_{\alpha, \beta} := \begin{cases}
\prod_{t \in \nestedtilingof (\alpha / \beta)} (-w(t)) & \text{if }\alpha \KWleq \beta\\
0 & \text{otherwise}, 
\end{cases} 
\end{align}
%for all $\alpha, \beta \in \DP_N$,
where $\nestedtilingof (\alpha / \beta)$ denotes the unique nested tiling of the skew Young diagram $\alpha / \beta$ when $\alpha \KWleq \beta$.
% The minus signs in the above definition are included for convenience, in order to avoid signs
% in the inverse.

% \subsubsection{\textbf{Inversion of weighted incidence matrices}}

%A key combinatorial result that we rely on 
We rely on the following combinatorial result, which
gives a formula for the inverse of the weighted incidence 
matrix~\eqref{eq: def of weighted incidence matrix} in terms of cover-inclusive Dyck tilings.
Such formulas for the inverses appear
in~\cite{KW-double_dimer_pairings_and_skew_Young_diagrams,
SZ-path_representations_of_maximal_paraboloc_KL_polynomials, KKP-boundary_correlations_in_planar_LERW_and_UST}.
\begin{prop}
\label{prop: weighted KW incidence matrix inversion}
%% Assign weights $w(t) \in \bC$ to all Dyck tiles $t$ (with shape and placement).
%% % % For each Dyck tile $t$ with a given shape and placement,
%% % % %that can appear between $2N$-step DPs, %s.t. $t = \lambda / \rho$ for some $\lambda , \rho \in \mathcal{D}_N$ with $\lambda \KWleq \rho$,
%% % % impose a weight $w(t) \in \bC$.
%Let $M \in \bC^{\DP_N \times \DP_N}$ be the weighted
%incidence matrix~\eqref{eq: def of weighted incidence matrix}
%% % collecting the weight and sign of the transitions 
%% % $\alpha \KWleq \beta$ by nested Dyck tilings, i.e.,
%% \begin{align}\label{eq: def of weighted incidence matrix}
%% M_{\alpha, \beta} := \begin{cases}
%% \prod_{t \in \nestedtilingof (\alpha / \beta)} (-w(t)) & \text{if }\alpha \KWleq \beta\\
%% 0 & \text{otherwise}
%% \end{cases}
%% \end{align}
%of the parenthesis reversal relation $\KWleq$.
The weighted incidence matrix $M \in \bC^{\DP_N \times \DP_N}$ with entries~\eqref{eq: def of weighted incidence matrix}
%Then $M$ 
is invertible, and the entries of the inverse matrix $M^{-1}$ are given by the weighted sums %the weights of the cover-inclusive tilings, i.e.,
\begin{align*} %\label{eq: inverse of weighted incidence matrix}
M^{-1}_{\alpha, \beta} = \begin{cases}
\sum_{T \in \CItilingsof (\alpha/\beta)} \prod_{t \in T} w(t) & \text{if }\alpha \DPleq \beta \\
0 & \text{otherwise} %,
\end{cases}
\end{align*}
over the sets 
%where
$\CItilingsof (\alpha/\beta)$ %denotes the family
of cover-inclusive Dyck tilings of the skew Young diagrams $\alpha/\beta$.
\end{prop}
\begin{proof}
%See~\cite[Thm~2.9]{KKP-boundary_correlations_in_planar_LERW_and_UST}.
In this form, the assertion is proved in~\cite[Theorem~2.9]{KKP-boundary_correlations_in_planar_LERW_and_UST}.
\end{proof}


\subsection{Slopes and wedges in Dyck paths and a recursion for incidence matrices}
%\subsection{\blue{A recursion for incidence matrices}}

%For the remaining results of this section, we need the following notions.
%\blue{[text removed]}
Any two consecutive steps of a Dyck path $\alpha$ are said to form either a slope or a wedge,
according to the cases illustrated in Figure~\ref{fig: wedges and slopes}:
we say that $\alpha$ has a \emph{wedge} at $j$ if $\alpha(j-1) = \alpha(j+1)$,
and that $\alpha$ has a \emph{slope} at $j$ otherwise.

A slope at $j$ %in $\alpha$ 
is called an \emph{up-slope} if $\alpha(j+1) = \alpha(j-1)+2$
and a \emph{down-slope} if $\alpha(j+1) = \alpha(j-1)-2$.
%, and in these two cases we respectively write $\upslopeat{j} \in \alpha$
%and $\downslopeat{j} \in \alpha$. 
Without specifying the type of the slope, we denote
the presence of a slope at $j$ %is denoted 
by $\slopeat{j} \in \alpha$.

% % % \blue{[Eve: the notations $\upslopeat{j} \in \alpha$ and  $\downslopeat{j} \in \alpha$ are not used. I removed them.
% % % Notations like $\upslope$, $\downslope$ are used in section~\ref{sec: conformal block functions background} but there the table indicates the notation.]} \alexmod{A: I'd keep them here.}

% The presence of a wedge at $j$ in $\alpha$ is denoted by $\wedgeat{j} \in \alpha$,
% and the presence of a slope at $j$ in $\alpha$ by $\slopeat{j} \in \alpha$.
A wedge at $j$ is called an \emph{up-wedge} if $\alpha(j) = \alpha(j \pm 1) + 1$, and %correspondingly, 
a \emph{down-wedge} if $\alpha(j) = \alpha(j \pm 1) - 1$,
and in these two cases we respectively write $\upwedgeat{j} \in \alpha$ and $\downwedgeat{j} \in \alpha$.
Without specifying the type of the wedge, we denote
the presence of a wedge at $j$ %is denoted 
by $\wedgeat{j} \in \alpha$.
By removing a wedge at $j$ from a Dyck path $\alpha \in \DP_N$ we obtain a shorter Dyck path
$\hat{\alpha} \in \DP_{N-1}$, namely
$\hat{\alpha} = (\alpha(0) , \alpha(1) , \ldots , \alpha(j-1) , \alpha(j+2) , \ldots , \alpha(2N))$.
According to whether the removed wedge is an up-wedge or a down-wedge, we %denote
write $\hat{\alpha} = \alpha \removeupwedge{j}$ or $\hat{\alpha} = \alpha \removedownwedge{j}$,
or without specifying the type of the removed wedge, we may
%denote 
write $\hat{\alpha} = \alpha \removewedge{j}$.
\begin{figure}
% \centerfloat
% \includegraphics[width = 0.2 \textwidth]{KW_matrices_printer-1.pdf} \qquad
% \includegraphics[width = 0.2 \textwidth]{numerical_weighted_KW_matrices_printer-1.pdf} \\
% \vspace{0.5 cm}
\centering
\subfigure[up-wedge]
{\qquad \includegraphics[width = 0.12 \textwidth]{pics-wedges_and_slopes-1.pdf} \qquad \label{fig: up-wedge}}
\subfigure[down-wedge]
{\qquad \includegraphics[width = 0.12 \textwidth]{pics-wedges_and_slopes-2.pdf} \qquad \label{fig: down-wedge}}
\subfigure[up-slope]
{\qquad \includegraphics[width = 0.12 \textwidth]{pics-wedges_and_slopes-3.pdf} \qquad \label{fig: up-slope}}
\subfigure[down-slope]
{\qquad \includegraphics[width = 0.12 \textwidth]{pics-wedges_and_slopes-4.pdf} \qquad \label{fig: down-slope}}
% \includegraphics[width = 0.12 \textwidth]{pics-wedges_and_slopes-1.pdf} \qquad
% \includegraphics[width = 0.12 \textwidth]{pics-wedges_and_slopes-2.pdf} \qquad
% \includegraphics[width = 0.12 \textwidth]{pics-wedges_and_slopes-3.pdf} \qquad
% \includegraphics[width = 0.12 \textwidth]{pics-wedges_and_slopes-4.pdf} \\
\caption{\label{fig: wedges and slopes}
Wedges and slopes.
% \blue{[Eve: we could make the figures a bit less high by removing some layers from the bottom. ]}
}
\end{figure}

Suppose that the weights $w(t)$ of Dyck tiles $t$ are chosen to only depend on
the height $h_t$ of the tile. %, via $w(t) = f(h_t)$.
Then, wedge removals %then 
allow for a characterization of weighted incidence matrices %of Dyck tilings ????? 
of the parenthesis reversal relation by the following recursion.
\begin{prop}\label{prop: recursion for matrix elements}
Let $f \colon \Zpos \to \bC$ be a given function. Then
the collection $(M^{(N)})_{N \in \bN}$ of weighted incidence
matrices~\eqref{eq: def of weighted incidence matrix} %of the parenthesis reversal relation %given by~\eqref{eq: def of weighted incidence matrix}
with weights of tiles determined by tile heights via $w(t) = f(h_t)$
is the unique collection of matrices $M^{(N)} \in \bC^{\DP_N \times \DP_N}$
satisfying the following recursion: we have $M^{(0)} = 1$,
and for any $N \in \bZpos$, and $\alpha, \beta \in \DP_N$, and $j \in \set{1,\ldots,2N-1}$ such that 
$\upwedgeat{j} \in \beta$, we have
\begin{align*} %\label{eq: equivalent recursion}
M_{\walk,\beta}^{(N)} = \begin{cases} 
0 & \text{if } \slopeat{j} \in \walk \\
M_{\hat{\walk},\hat{\beta}}^{(N-1)}
& \text{if } \upwedgeat{j} \in \walk \\
- f(\walk(j)+1) \times
M_{\hat{\walk},\hat{\beta}}^{(N-1)} 
& \text{if } \downwedgeat{j} \in \walk,
\end{cases} 
\end{align*}
where we denote by $\hat{\walk} = \walk \removewedge{j} \in \DP_{N-1}$ and 
$\hat{\beta} = \beta \removeupwedge{j} \in \DP_{N-1}$.
%$\hat{\beta} = \beta \removeLink \link{j}{j+1} \in \DP_{N-1}$.
\end{prop}
\begin{proof}
See~\cite[Lemmas~2.13~and~2.14]{KKP-boundary_correlations_in_planar_LERW_and_UST}.
\end{proof}

\bigskip

\section{Conformal block functions}
\label{sec: conformal block functions background}

% \blue{[Eve: Do we speak of "fields" or "field operators" ?]}

In the operator formalism of quantum field theories, fields correspond
to linear operators on the state space of the theory, and correlation
functions are written as ``vacuum expected values''. Somewhat more
concretely, $n$-point correlation functions are particular matrix
elements of a composition of $n$ linear operators on the state space.
%of the theory. Since t
Since the state space carries representations of the symmetries of the quantum field theory and
can be split into a direct sum of subrepresentations, % (usually irreducible). %(irreducible, or at least indecomposable).
% Therefore,
it is natural to split these linear operators into corresponding blocks. 
In conformal field theory (CFT), %in particular, 
the state space is a representation of the Virasoro algebra by virtue of conformal symmetry. 
The term conformal block 
% \blue{``conformal block''}
refers to the idea of splitting the field operators into pieces
that go from one Virasoro subrepresentation of the state space to another, and compositions of field operators to
pieces that pass through a given sequence of %such 
subrepresentations, see~\cite{BPZ-infinite_conformal_symmetry_in_2D_QFT, Felder-BRST_approach,
DMS-CFT, Ribault-conformal_field_theory_on_the_plane}.
% \cite{Felder-BRST_approach,FFK-braid_matrices,DMS-CFT}

The main purpose of this section is to provide background for the definition of conformal block functions
that we use in the rest of the article. 
%, which we give in Section~\ref{sub: defining conformal block functions} and then use in the rest of the article. 
This definition is given in
Section~\ref{sub: defining conformal block functions}.
The background is included 
% % % % \blue{We include some background here}
% % % %only 
% % % to provide sufficient context and main ideas, but the presentation here is not intended to be fully rigorous~--- 
% % % % \blue{nor self-contained} \alexmod{[A: Skip that.]}
% % % a complete mathematical treatment would require extensive formalism and results on vertex operator algebras.
to provide sufficient context and main ideas, but its
presentation here is not intended to be fully rigorous.
% \blue{nor self-contained} \alexmod{[A: Skip that.]}
We believe that a complete mathematical derivation of our defining properties 
is possible using the formalism of vertex operator
algebras~\cite{Lepowsky_Li-VOA}, but it is beyond the present work, and 
seems not to be readily contained in the existing literature. 


% \blue{[Eve: is this clear enough? Maybe it could be written better...]}

\subsection{Highest weight representations of Virasoro algebra}

% \blue{[Eve: we should maybe refer to \cite{Kac-ICM_proceedings_Helsinki, IK-representation_theory_of_the_Virasoro_algebra} 
% somewhere in the text.]}

Usually, in
conformal field theory the state space is assumed to split into a
direct sum of highest weight representations of the Virasoro 
algebra~\cite{Kac-ICM_proceedings_Helsinki, FF-representations, IK-representation_theory_of_the_Virasoro_algebra},
with a common central charge $c\in\bR$ and various highest weights
$h\in\bR$. We parametrize the central charges $c\leq1$ 
%(relevant to statistical physics applications of conformal field theories) 
by $\kappa>0$ via~\eqref{eq: central charge parametrization},
%$c = c(\kappa)$ 
% \begin{align*}
% c=\; & \frac{(3\kappa-8)(6-\kappa)}{2\kappa},%\label{eq: central charge parametrization}
% \end{align*}
as is relevant to the theory of $\SLEk$ type random curves.
% % % \blue{(central charges $c\leq1$ are relevant to statistical physics applications of conformal field theories).}
%~\cite{BB-SLE_martingales_and_Virasoro_algebra, BB-CFTs_of_SLEs,
% FW-conformal_restriction_highest_weight_representations_and_SLE,
% Kytola-local_mgales}
%~\cite{Kytola-local_mgales}
% {FW-conformal_restriction_highest_weight_representations_and_SLE}
A special role is played by a primary field of conformal
weight $h=h_{1, 2} = \frac{6-\kappa}{2\kappa}$, %which creates a tip of an $\SLEk$-curve, and
which through fusion generates the so called first row of the Kac table with conformal weights 
\begin{align*} %\label{eq: highest weight parametrization}
h(\lambda) = \; & h_{1,\lambda+1}=\frac{\lambda^{2}+2\lambda}{\kappa}-\frac{\lambda}{2} \qquad 
\text{for }
\lambda\in\Znn.
\end{align*}
In this article, we consider the generic case
$\kappa\notin\bQ$. %in which case 
Then, the irreducible highest weight representation with highest weight $h(\lambda)$ and
central charge $c$ is a quotient of the corresponding Verma module
by a submodule that itself is a Verma module\footnote{At rational values of $\kappa$, the structure of highest weight representations
can be more involved, see, e.g.,~\cite{IK-representation_theory_of_the_Virasoro_algebra}.}. 
We denote this irreducible
quotient %representation 
by $\Qrep_{\lambda}$ and a highest weight vector in it by $\hwvec_{\lambda}$.
% The representation $\Qrep_{\lambda}$ is $\bN$-graded by $L_0$-eigenvalues
% $\Qrep_{\lambda} = \bigoplus_{n \in \bN} \Qrep_{\lambda}^{(n)}$,
% 
The contragredient (graded dual) representation $\Qrep_{\lambda}^{*}$
(see, e.g.,~\cite{IK-representation_theory_of_the_Virasoro_algebra}) is isomorphic
to $\Qrep_{\lambda}$, and we choose a highest weight vector $\hwvec_{\lambda}^{*}$
for it so that the normalization $\dualpairing{\hwvec_{\lambda}^{*}}{\hwvec_{\lambda}}=1$ holds.
% \red{[Eve: should we define contragredient module? At least a reference should be given.]}


\subsection{Conformal blocks}

\subsubsection{\textbf{Intertwining relation for primary field operators}}
A primary field $\psi$ of conformal weight $h$ is characterized by its
transformation property
\begin{align*}
\psi(x)\rightsquigarrow\;\confmap'(x)^{h}\;\psi(\confmap(x)) 
\end{align*}
under conformal transformations $\confmap$.
According to the seminal work~\cite{BPZ-infinite_conformal_symmetry_in_2D_QFT},
more general fields can be understood in terms of these primary fields.

In the operator formalism, a primary field $\psi(x)$ is realized by a primary field
operator $\Psi(x)$. We wish to split $\Psi(x)$
into conformal blocks between various highest weight representations $\Qrep_{\lambda}$ and
$\Qrep_{\mu}$, with $\lambda,\mu\in\Znn$.
% The primary field operator $\Primary(x)$ has a dependence on the position $x$ of the field,
% and so does its block $\PrimaryBlock{\lambda}{\mu}(x)$.
The intertwining relation
\begin{align}
L_{n}\,\Primary(x)-\Primary(x)\,L_{n}=\; & x^{1+n}\pder x\Primary(x)+(1+n)x^{n}h\,\Primary(x)\label{eq: intertwining relation}
\end{align}
with the Virasoro generators $L_{n}$, $n \in \bZ$, is the infinitesimal form of
the primary field transformation property
% \begin{align*}
% \psi(x)\rightsquigarrow\;\confmap'(x)^{h}\;\psi(\confmap(x)) 
% \end{align*}
under a conformal transformation $\confmap$ obtained
by varying the identity transformation to the direction of the holomorphic vector
field $\ell_{n}=-x^{1+n}\pder x$.

% % % % % % % % 
% % % % % % % % According to the seminal work~\cite{BPZ-infinite_conformal_symmetry_in_2D_QFT},
% % % % % % % % all fields in a conformal field theory can be understood via the study of
% % % % % % % % so called primary fields,
% % % % % % % % 
% % % % % % % % Among all fields of a conformal field theory a special role is played by primary
% % % % % % % % fields $\psi$~\cite{BPZ-infinite_conformal_symmetry_in_2D_QFT},
% % % % % % % % whose transformation property under conformal transformations $\confmap$ reads
% % % % % % % % \begin{align*}
% % % % % % % % \psi(x)\rightsquigarrow\;\confmap'(x)^{h}\;\psi(\confmap(x)) .
% % % % % % % % \end{align*}
% % % % % % % % % In typical CFTs, such fields generate the set of all local fields under the Virasoro action.
% % % % % % % % 
% % % % % % % % % The seminal work~\cite{BPZ-infinite_conformal_symmetry_in_2D_QFT} raise
% % % % % % % % % 
% % % % % % % % % It was noted in the seminal article \cite{BPZ-infinite_conformal_symmetry_in_2D_QFT}
% % % % % % % % % that among all fields of a conformal field theory there are primary
% % % % % % % % % fields $\psi$, characterized by their transformation property under conformal transformations $\confmap$
% % % % % % % % % \begin{align*}
% % % % % % % % % \psi(x)\rightsquigarrow\;\confmap'(x)^{h}\;\psi(\confmap(x)) ,
% % % % % % % % % \end{align*}
% % % % % % % % % which in a certain sense generate (in typical CFTs)
% % % % % % % % % the set of all fields.


% % % We study correlation functions of \red{primary field operators} 
% % % \red{[Eve: what is the definition of a primary field operator?]}
% % % of conformal weight $h=h(1)=\frac{6-\kappa}{2\kappa}$, which we wish to split
% % % to conformal blocks between representations $\Qrep_{\lambda}$ and
% % % $\Qrep_{\mu}$, with $\lambda,\mu\in\Znn$. The primary field operator
% % % $\Primary(x)$ has a dependence on the position $x$ of the field,
% % % and so does its block $\PrimaryBlock{\lambda}{\mu}(x)$. The intertwining
% % % relation
% % % \begin{align}
% % % L_{n}\,\Primary(x)-\Primary(x)\,L_{n}=\; & x^{1+n}\pder x\Primary(x)+(1+n)x^{n}h\,\Primary(x)\label{eq: intertwining relation}
% % % \end{align}
% % % with the Virasoro generators $L_{n}$, $n \in \bZ$, is the infinitesimal form of
% % % the primary field transformation property
% % % \begin{align*}
% % % \psi(x)\rightsquigarrow\;\confmap'(x)^{h}\;\psi(\confmap(x)) 
% % % \end{align*}
% % % under a conformal transformation $\confmap$ obtained
% % % by varying the identity transformation to the direction of the holomorphic vector
% % % field $\ell_{n}=-x^{1+n}\pder x$.

\subsubsection{\textbf{Matrix elements characterizing conformal blocks}}

The single matrix element between highest weight vectors,
\begin{align} \label{eq: 3pt function}
U_{\lambda}^{\mu}(x)=\; & \dualpairing{\hwvec_{\mu}^{*}}{\PrimaryBlock{\lambda}{\mu}(x)\,\hwvec_{\lambda}}, 
\end{align}
contains sufficient information to completely
determine $\PrimaryBlock{\lambda}{\mu}(x)$.
%completely. 
% \red{[Eve: In the following, the index $n$ is very bad notation: it is used for the Virasoro generators.]}
More generally, a composition of primary field operators
splits into conformal blocks indexed by a sequence $\sigmaseq=(\sigma_{0},\sigma_{1},\ldots,\sigma_{n})$
with $\sigma_{j}\in\Znn$ %, for $j=0,1,\ldots,n$, $j \in \set{0,1,\ldots,2N}$
labeling the intermediate
representations $\Qrep_{\sigma_{0}},\Qrep_{\sigma_{1}},\ldots,\Qrep_{\sigma_{n}}$.
Now, the matrix element 
\begin{align}
U_{\sigmaseq}(x_{1},x_{2},\ldots,x_{n})=\; & \dualpairing{\hwvec_{\sigma_{n}}^{*}}{\PrimaryBlock{\sigma_{n-1}}{\sigma_{n}}(x_{n})\cdots\PrimaryBlock{\sigma_{1}}{\sigma_{2}}(x_{2})\,\PrimaryBlock{\sigma_{0}}{\sigma_{1}}(x_{1})\,\hwvec_{\sigma_{0}}}\label{eq: general conformal block function}
\end{align}
contains sufficient information to uniquely
determine the block of the composition.
%uniquely. 
Note that $U_{\lambda}^{\mu}(x)$ is a special case of $U_{\sigmaseq}(x_{1},x_{2},\ldots,x_{n})$
with $n=1$, $\sigma_{0}=\lambda$, and $\sigma_{1}=\mu$. 
We furthermore point out that $U_{\sigmaseq}(x_{1},x_{2},\ldots,x_{n})$ appears
in an actual vacuum expected value of $n$ fields if the highest weight
states on the right and left are the absolute vacua, i.e., if $\sigma_{0}=0$ and $\sigma_{n}=0$.
% % % \red{[Eve: this is not clear enough. What is the definition of the "actual vacuum expected value" ?]}

In the vertex operator algebra axiomatization of conformal field theory~\cite{Lepowsky_Li-VOA},
the block
$\PrimaryBlock{\lambda}{\mu}(x)$ is a formal power
series in $x$ with
coefficients that are linear operators, and 
the matrix elements
$U_{\lambda}^{\mu}(x)$ and $U_{\sigmaseq}(x_{1},x_{2},\ldots,x_{n})$ are formal power series with complex coefficients. For radially ordered variables
\begin{align*}
 & 0<|x_{1}|<|x_{2}|<\cdots<|x_{n}|,
\end{align*}
%however, 
the series are in fact convergent, so we may
view~\eqref{eq: 3pt function} and~\eqref{eq: general conformal block function}
as actual functions.
The fact that they determine the operators and their compositions
justifies calling them conformal block functions.




\subsection{Properties of conformal block functions}

We now review properties of the conformal block functions $U_{\sigmaseq}(x_{1}, \ldots, x_{n})$,
which in particular completely fix the form of the matrix elements $U_{\lambda}^{\mu}(x)$,
and characterize for which $\lambda$ and $\mu$ the block $\PrimaryBlock{\lambda}{\mu}(x)$
can be non-vanishing in the first place.

\subsubsection{\textbf{Covariance properties}}

The intertwining relation~\eqref{eq: intertwining relation} for $L_0$
% \red{[Eve: e.g., here the notation $n$ for both Virasoro generators and number of variables is very bad.]}
combined with the eigenvalues $L_{0} \, \hwvec_{\sigma_0} = h(\sigma_0)\,\hwvec_{\sigma_0}$
and $L_{0}^\top \, \hwvec^*_{\sigma_n} = h(\sigma_n)\,\hwvec_{\sigma_n}^*$
of the highest weight vectors gives
\begin{align*}
\big( h(\sigma_n) - h(\sigma_0) \big) \; U_{\sigmaseq}(x_{1},\ldots,x_{n})
    = \sum_{j=1}^n \Big( x_j \pder{x_j}  + h \Big) \; U_{\sigmaseq}(x_{1},\ldots,x_{n}) ,
\end{align*}
with $h=h(1)=\frac{6-\kappa}{2\kappa}$.
This infinitesimal relation can be integrated to 
obtain the homogeneity property
\begin{align} \label{eq: homogeneity of conformal block functions}
U_{\sigmaseq}(rx_{1},\ldots,rx_{n})=\; & r^{h(\sigma_{n})-h(\sigma_{0})-nh}\;U_{\sigmaseq}(x_{1},\ldots,x_{n})
    , \qquad\text{for } r>0 , 
\end{align}
of the conformal block functions. For the simplest conformal block
function $U_{\lambda}^{\mu}(x)$, this homogeneity fixes its
form
%of $U_{\lambda}^{\mu}(x)$
up to a multiplicative constant $C_{\lambda}^{\mu}$: %, namely
%\begin{align} \label{eq: the general form of 3pt function}
%U_{\lambda}^{\mu}(x)=\; & C_{\lambda}^{\mu}\;x^{h(\mu)-h(\lambda)-h} .
%\end{align}
\begin{align} \label{eq: the general form of 3pt function}
U_{\lambda}^{\mu}(x) =  \dualpairing{\hwvec_{\mu}^{*}}{\PrimaryBlock{\lambda}{\mu}(x)\,\hwvec_{\lambda}} =\; & C_{\lambda}^{\mu}\;x^{h(\mu)-h(\lambda)-h} .
\end{align}

Next, if we have $\sigma_0 = 0$, then
%there is a null vector 
$L_{-1} \, \hwvec_{\sigma_0} = 0$ 
is a null vector in the quotient representation $\Qrep_0$.
Together with the intertwining relation~\eqref{eq: intertwining relation} for $L_{-1}$, and the property $L_{-1}^\top \, \hwvec^*_{\sigma_n} = 0$,
this gives the infinitesimal form of 
the translation invariance
\begin{align*} %\label{eq: translation invariance}
U_{\sigmaseq}(x_{1} + t, \ldots, x_{n} + t)
    = U_{\sigmaseq}(x_{1}, \ldots, x_{n})
    , \qquad\text{for } t \in \bR .
\end{align*}

Likewise, if $\sigma_n = 0$, then $L_1^\top \, \hwvec_{\sigma_n}^* = 0$
is a null vector in the contragredient representation $\Qrep_0^*$. Together with the intertwining relation~\eqref{eq: intertwining relation} for $L_1$, and the property $L_{1} \, \hwvec_{\sigma_0} = 0$,
this gives the infinitesimal form of the following covariance under special conformal transformations:
\begin{align*}
U_{\sigmaseq} \Big( \frac{x_{1}}{1 - s x_1}, \ldots, \frac{x_{n}}{1 - s x_n} \Big)
    = \prod_{j=1}^n (1 - s x_j)^{2h} \; \times  U_{\sigmaseq}(x_{1}, \ldots, x_{n})
    , \qquad\text{for } s \in \bR .
\end{align*}

For the conformal blocks that contribute to the vacuum expected value, we
have %both 
$\sigma_0 = 0$ and $\sigma_n = 0$. These conformal block functions satisfy the covariance
\begin{align*} %\label{eq: Mobius covariance}
U_{\sigmaseq}(x_{1}, \ldots, x_{n})
= \prod_{j=1}^n \Mob'(x_j)^{2h} \; \times U_{\sigmaseq} \big( \Mob(x_1), \ldots, \Mob(x_n) \big)
\end{align*}
under general M\"obius transformations $\Mob(x) = \frac{a x + b}{c x + d}$
with $a,b,c,d \in \bR$ and $ad-bc>0$.
% \blue{[Eve: do we consider bulk or boundary theory? Should we specify the M\"obius transformations better? ]}

\subsubsection{\textbf{Partial differential equations}}

% \red{[Eve: degeneracy at level $\ell$ has not been defined!]}

Suppose now that the primary field $\Psi(x)$ of conformal weight $h=h(1)=\frac{6-\kappa}{2 \kappa}$ has the same 
degeneracy at grade %\blue{level}
two as the quotient representation $\Qrep_{1}$ of highest weight $h(1)$. %, as we will assume from here on, t
Then the conformal block functions satisfy partial differential equations of second order.
These PDEs obtain a more symmetric expression in terms of the shifted
versions of the conformal block functions defined by
\begin{align*}
\widetilde{U}_{\sigmaseq}(x_{0},x_{1},\ldots,x_{n}) :=\; & U_{\sigmaseq}(x_{1}-x_{0},\ldots,x_{n}-x_{0}).
\end{align*}


The %degeneracy
partial differential equation
arising from the degeneracy of $\Psi(x_j)$ at grade two takes the form
given in~\cite{BPZ-infinite_conformal_symmetry_in_2D_QFT},
\begin{align}\label{eq: PDEs for conformal block functions}
\begin{split}
\Bigg\{ & \; \frac{\kappa}{2}\pdder{x_{j}} + \frac{2}{x_{0}-x_{j}}\pder{x_{0}} - \frac{2\,h(\sigma_{0})}{(x_{0}-x_{j})^{2}} \quad 
%   \label{eq: PDEs for conformal block functions}
	  \\
& \; \qquad + \sum_{\substack{i=1,\ldots,n \\ i\neq j}} \frac{2}{x_{i}-x_{j}}\pder{x_{i}}
- \sum_{\substack{i=1,\ldots,n\\i\neq j}} \frac{2\,h}{(x_{i}-x_{j})^{2}} \Bigg\} \; \widetilde{U}_{\sigmaseq}(x_{0},x_{1},\ldots,x_{n})
       = 0 . 
% \nonumber
\end{split}
\end{align}
% The %degeneracy 
% partial differential equations 
% arising from the degeneracy at grade two take the form~\cite{BPZ-infinite_conformal_symmetry_in_2D_QFT}
% \begin{align}
% \Bigg\{ & \; \frac{\kappa}{2}\pdder{x_{j}} + \frac{2}{x_{0}-x_{j}}\pder{x_{0}} - \frac{2\,h(\sigma_{0})}{(x_{0}-x_{j})^{2}} & %\quad  
%    \label{eq: PDEs for conformal block functions} \\
% & \; \qquad + \sum_{\substack{i=1,\ldots,n \\ i\neq j}} \frac{2}{x_{i}-x_{j}}\pder{x_{i}}
% - \sum_{\substack{i=1,\ldots,n\\i\neq j}} \frac{2\,h}{(x_{i}-x_{j})^{2}} \Bigg\} \; \widetilde{U}_{\sigmaseq}(x_{0},x_{1},\ldots,x_{n})
%        = 0 %, \nonumber 
% \end{align}
% for all $j=1,\ldots,n$ %. $j \in \set{1,\ldots,n}$
% % % \blue{
% % % \begin{align}  \label{eq: PDEs for conformal block functions}
% % % \begin{split}
% % % \quad
% % % \Bigg\{ & \; \frac{\kappa}{2}\pdder{x_{j}} + \frac{2}{x_{0}-x_{j}}\pder{x_{0}} - \frac{2\,h(\sigma_{0})}{(x_{0}-x_{j})^{2}} \\
% % % & \; \qquad + \sum_{i\neq j} \frac{2}{x_{i}-x_{j}}\pder{x_{i}}
% % % - \sum_{i\neq j} \frac{2\,h}{(x_{i}-x_{j})^{2}} \Bigg\} \; \widetilde{U}_{\sigmaseq}(x_{0},x_{1},\ldots,x_{n})
% % %         = 0 .
% % % %       \qquad  \text{for all $j \in \set{1,\ldots,n}$.}   
% % % %      \text{for all $j=1,\ldots,n$.} 
% % % \end{split}
% % % \end{align}
% % % }
% % % \blue{[Eve: here I used the "split" environment and removed the index set from the sums to simplify.]}

\subsubsection{\textbf{Selection rules}}

The PDEs above in particular imply selection rules for when non-vanishing conformal
blocks can exist. Namely, for $U_{\lambda}^{\mu}(x)=C_{\lambda}^{\mu}\,x^{\Delta}$,
with $\Delta=h(\mu)-h(\lambda)-h$ as in Equation~\eqref{eq: the general form of 3pt function},
the requirement of the PDE~\eqref{eq: PDEs for conformal block functions} amounts to
\begin{align*}
C_{\lambda}^{\mu}\;\left(\frac{\kappa}{2}\Delta(\Delta-1)+2\Delta -2h(\lambda)\right)\;x^{\Delta-2}=\; & 0,
\end{align*}
which implies either the vanishing of $C_{\lambda}^{\mu}$ and therefore
of the entire conformal block, or a quadratic equation relating the
conformal weights $h(\mu)$ and $h(\lambda)$. For fixed $\lambda$,
the two solutions of this quadratic equation are obtained at $\mu=\lambda\pm1$.
One can therefore conclude that the conformal blocks take the form
\begin{align}
U_{\lambda}^{\mu}(x)=\; & \begin{cases}
C_{\lambda}^{\pm}\;x^{h(\mu)-h(\lambda)-h} & \text{ if }\mu=\lambda\pm1\\
0 & \text{ if }|\mu-\lambda|\neq1.
\end{cases}\label{eq: form of 3pt function with selection}
\end{align}
% \blue{[Eve: Is $C_{\lambda}^{\pm}$ good notation for the constants here? ]}
The normalizations $C_{\lambda}^{\pm}$ are not canonically fixed;
in fact, the space of intertwining operators forms a vector space
(in the present cases always of dimension %$1$ or $0$, 
one or zero depending on whether % or not
the selection rules are fulfilled). 
One can make any convenient choice, and we will fix our choice later.

In the case $\lambda=0$, there is one further selection rule:
%, simply from translation invariance.
% not
% yet accounted for by the above consideration. The highest weight vector
% $\hwvec_{0}$ in the quotient representation $\Qrep_{0}$ satisfies
% $L_{-1}\,\hwvec_{0}=0$. With the intertwining relation~\eqref{eq: intertwining relation}
% this implies that $\pder{x} U_{0}^{\mu}(x)=0$, i.e., $U_{0}^{\mu}(x)$
% is constant.
%Namely, 
by translation invariance, %we have $\pder{x} U_{0}^{\mu}(x)=0$, i.e.,
$U_{0}^{\mu}(x)$ is constant. This further restricts the possibilities
in~\eqref{eq: the general form of 3pt function} to $h(\mu)=h=h(1)$, i.e., $\mu=1$.

In conclusion, a non-vanishing intertwining operator from $\Qrep_{\lambda}$
to $\Qrep_{\mu}$ can only exist if $|\mu-\lambda|=1$ and $\mu\geq0$.
Consequently, the composition of intertwining operators as in the
conformal block function~\eqref{eq: general conformal block function}
can only be non-trivial if the sequence $\sigmaseq=(\sigma_{0},\sigma_{1},\ldots,\sigma_{n})$
satisfies $\sigma_{j}\in\Znn$ and $|\sigma_{j}-\sigma_{j-1}|=1$
for all $j$. This means that non-trivial conformal block functions
are indexed by nearest neighbor walks on non-negative integers. 
The conformal block functions that contribute to the actual vacuum expected
value of $n$ fields must furthermore have $\sigma_{0}=0$ and $\sigma_{n}=0$,
so they are in fact indexed by
% excursions of nearest neighbor walks on non-negative
% integers from zero to zero in $n$ steps,
Dyck paths, and in particular, $n$ must be even.
% \blue{(and in particular $n$ must be even).
% % In this case, we usually denote the sequence $\sigmaseq$
% % of intermediate dimensions by $\alpha$, as in Section~\ref{sec: combinatorial preliminaries}.


%\subsubsection{\textbf{Asymptotics of conformal block functions}}
\subsubsection{\textbf{Asymptotics}}
% \blue{[text removed from title]}

Above, in Equation~\eqref{eq: form of 3pt function with selection}, we completely
fixed the form of the simplest conformal block function $U_{\lambda}^{\mu}(x)$,
i.e., the case $n=1$. Now we consider the next case $n=2$ with $\sigmaseq=(\sigma_{0},\sigma_{1},\sigma_{2})$,
and the corresponding conformal block function
\begin{align*}
U_{\sigmaseq}(x_{1},x_{2})=\; & \dualpairing{\hwvec_{\sigma_{2}}^{*}}{\PrimaryBlock{\sigma_{1}}{\sigma_{2}}(x_{2})\,\PrimaryBlock{\sigma_{0}}{\sigma_{1}}(x_{1})\,\hwvec_{\sigma_{0}}}.
\end{align*}

%Note first that i
If $x_{1}$ is kept fixed, then the vector $\PrimaryBlock{\sigma_{0}}{\sigma_{1}}(x_{1})\,\hwvec_{\sigma_{0}}$
can be expanded in the usual basis of $\Qrep_{\sigma_{1}}$ as
\begin{align*}
\PrimaryBlock{\sigma_{0}}{\sigma_{1}}(x_{1})\,\hwvec_{\sigma_{0}}
    = & \sum_{k \in \bN} \; \sum_{n_{1},\ldots,n_{k}>0} a_{n_{1},\ldots,n_{k}}(x_{1}) \; L_{-n_{k}}\cdots L_{-n_{1}}\,\hwvec_{\sigma_{1}},
\end{align*}
%\begin{align*}
%\PrimaryBlock{\sigma_{0}}{\sigma_{1}}(x_{1})\,\hwvec_{\sigma_{0}}
%    = & \sum_{n_{1},\ldots,n_{k}>0} a^{(\sigma_0, \sigma_1)}_{n_{1},\ldots,n_{k}}(x_{1}) \; L_{-n_{k}}\cdots L_{-n_{1}}\,\hwvec_{\sigma_{1}},
%\end{align*}
where in particular the coefficient of the highest weight vector $\hwvec_{\sigma_{1}}$
% \blue{(i.e., with $k=0$)}
is picked by the projection to $\hwvec_{\sigma_{1}}^{*}$,
\begin{align*}
a_{\emptyset}(x_{1})=\; & \dualpairing{\hwvec_{\sigma_{1}}^{*}}{\PrimaryBlock{\sigma_{0}}{\sigma_{1}}(x_{1})\,\hwvec_{\sigma_{0}}}
    = U_{\sigma_{0}}^{\sigma_{1}}(x_{1}) = C_{\sigma_{0}}^{\sigma_{1}}\,x_{1}^{h(\sigma_{1})-h(\sigma_{0})-h}.
\end{align*}
Using this expansion, the conformal block function becomes
\begin{align*}
U_{\sigmaseq}(x_{1},x_{2})
    =\; & \sum_{k \in \bN} \; \sum_{n_{1},\ldots,n_{k}>0} a_{n_{1},\ldots,n_{k}}(x_{1})\;\dualpairing{\hwvec_{\sigma_{2}}^{*}}{\PrimaryBlock{\sigma_{1}}{\sigma_{2}}(x_{2})\,L_{-n_{k}}\cdots L_{-n_{1}}\,\hwvec_{\sigma_{1}}}.
\end{align*}
With the generic form~\eqref{eq: the general form of 3pt function} of the conformal blocks $U_{\sigma_{1}}^{\sigma_{2}}(x_{2})$,
%Given that $\dualpairing{\hwvec_{\sigma_{2}}^{*}}{\PrimaryBlock{\sigma_{1}}{\sigma_{2}}(x_{2})\,\hwvec_{\sigma_{1}}}\propto x_{2}^{h(\sigma_{2})-h(\sigma_{1})-h}$,
the intertwining relation~\eqref{eq: intertwining relation} implies
%for the other terms 
%$\dualpairing{\hwvec_{\sigma_{2}}^{*}}{\PrimaryBlock{\sigma_{1}}{\sigma_{2}}(x_{2})\,L_{-n_{k}}\cdots L_{-n_{1}}\hwvec_{\sigma_{1}}}\propto x_{2}^{h(\sigma_{2})-h(\sigma_{1})-h-\sum n_{j}}$,
\begin{align*}
\dualpairing{\hwvec_{\sigma_{2}}^{*}}{\PrimaryBlock{\sigma_{1}}{\sigma_{2}}(x_{2})\,L_{-n_{k}}\cdots L_{-n_{1}}\hwvec_{\sigma_{1}}} 
\propto x_{2}^{h(\sigma_{2}) - h(\sigma_{1}) - h - n_{1} - \, \cdots \, - n_k} .
\end{align*}
% Since $n_{j}>0$, %the term with $k=0$ gives
The leading contribution
%which is of smaller order 
in the limit $x_{2}\to\infty$ 
comes from the highest weight vector $\hwvec_{\sigma_1}$, since $n_j>0$.
%since $\sum_{j=1}^k n_{j}>0$
Thus, for fixed $x_{1}$ and as $x_{2}\to\infty$, %in the limit as $x_{2}\to\infty$, 
the leading asymptotics of the conformal block function is
% \blue{in the limit as $x_{2}\to\infty$ is}
%is given by the term
\begin{align}
\begin{split}
U_{\sigmaseq}(x_{1},x_{2})
    \sim\; & a_{\emptyset}(x_{1})\;\dualpairing{\hwvec_{\sigma_{2}}^{*}}{\PrimaryBlock{\sigma_{1}}{\sigma_{2}}(x_{2})\,\hwvec_{\sigma_{1}}} \\
     %\nonumber \\
    =\; & a_{\emptyset}(x_{1})\times U_{\sigma_{1}}^{\sigma_{2}}(x_{2}) \\
    %\nonumber \\
    =\; & C_{\sigma_{0}}^{\sigma_{1}}\,x_{1}^{h(\sigma_{1})-h(\sigma_{0})-h}\times C_{\sigma_{1}}^{\sigma_{2}}\,x_{2}^{h(\sigma_{2})-h(\sigma_{1})-h}. \\
    %\nonumber \\
    =\; & C_{\sigma_{0}}^{\sigma_{1}}C_{\sigma_{1}}^{\sigma_{2}}\;x_{1}^{h(\sigma_{1})-h(\sigma_{0})-h}\,x_{2}^{h(\sigma_{2})-h(\sigma_{1})-h}.
    \label{eq: asymptotics for x2}
\end{split}
\end{align}
By a similar argument, for fixed $x_{2}$ and as $x_{1}\to0$,
the leading asymptotics of the conformal block function is
% \blue{in the limit as $x_{1}\to0$ is}
%is given by the term
\begin{align} \label{eq: asymptotics for x1}
U_{\sigmaseq}(x_{1},x_{2})\sim\; & C_{\sigma_{0}}^{\sigma_{1}}C_{\sigma_{1}}^{\sigma_{2}}\;x_{2}^{h(\sigma_{2})-h(\sigma_{1})-h}\,x_{1}^{h(\sigma_{1})-h(\sigma_{0})-h}.
\end{align}
The remaining interesting asymptotics of $U_{\sigmaseq}(x_{1},x_{2})$
concerns the limit $|x_{1}-x_{2}|\to0$.
To analyze these, we resort to direct solutions of the PDEs~\eqref{eq: PDEs for conformal block functions} in the following.
% % \green{
% % We %will 
% % use direct solutions of the PDEs~\eqref{eq: PDEs for conformal block functions} to fix these.
% % }
% % \blue{[Eve: not clear]}

The homogeneity~\eqref{eq: homogeneity of conformal block functions}
can be used to cast the $n=2$ conformal block function into the form
\begin{align*}
U_{\sigmaseq}(x_{1},x_{2})=\; & x_{2}^{h(\sigma_{2})-h(\sigma_{0})-2h}\;g_{\sigmaseq}\left(\frac{x_{1}}{x_{2}}\right),
\end{align*}
and the PDEs~\eqref{eq: PDEs for conformal block functions} for $\widetilde{U}_{\sigmaseq}(x_{0},x_{1},x_{2})=(x_{2}-x_{0})^{h(\sigma_{2})-h(\sigma_{0})-2h}\;g_{\sigmaseq}\big(\frac{x_{1}-x_{0}}{x_{2}-x_{0}}\big)$
then translate to the following
second order ODEs for $g_{\sigmaseq}(z)$:
\begin{align} \label{eq: ODE}
\begin{split}
0=\; & \kappa\;z^{2}(z-1)^{2}\;g_{\sigmaseq}''(z)+8\;z(z-1)(z-\frac{1}{2})\;g_{\sigmaseq}'(z)\\
 & +4\Big(z(z-2)h-z(z-1)h(\sigma_{2})+(z-1)h(\sigma_{0})\Big)\;g_{\sigmaseq}(z)
 \end{split} \\
% \begin{split}
% 0 = \; &  \kappa\;z^{2}(z-1)^{2}\;g_{\sigmaseq}''(z) - 2 z (z-1) \Big[ \kappa \Big( h(\sigma_2) - h(\sigma_0) - 2 h -1 \Big) (z-1)  + 2 (z-2) \Big] \; g'(z) \\
% & + \Bigg[  \Bigg( \Big( h(\sigma_{2})   - h(\sigma_{0}) - 2 h \Big) \Big(4 +\kappa (h(\sigma_{2}) - h(\sigma_{0}) - 2 h  - 1) \Big)-4 h(\sigma_0) \Bigg) (z-1)^2 - 4h \Bigg] \; g(z),
%\end{split}
\label{eq: ODE2}
 \begin{split}
 0 = \; &  \kappa\;z^{2}(z-1)^{2}\;g_{\sigmaseq}''(z) - 2 z (z-1) \Big( \kappa ( \tilde{\Delta} -1 ) (z-1)  + 2 (z-2) \Big) \; g'(z) \\
 & + \Big[  \Big(\kappa \tilde{\Delta}(\tilde{\Delta} - 1) +  4 \tilde{\Delta} -4 h(\sigma_0) \Big) (z-1)^2 - 4h \Big] \; g(z),
\end{split}
\end{align}
%\begin{align} \label{eq: ODE}
% \kappa^{2}(z-1)^{2}g_{\sigmaseq}''(z)+8z(z-1)(z-\frac{1}{2})g_{\sigmaseq}'(z)
%  +4\Big(z(z-2)h-z(z-1)h(\sigma_{0})+(z-1)h(\sigma_{2})\Big)g_{\sigmaseq}(z).
%\end{align}
where we denoted the scaling exponent in $\widetilde{U}_{\sigmaseq}$ by $ \tilde{\Delta} = h(\sigma_{2})-h(\sigma_{0})-2h$. The two ODEs~\eqref{eq: ODE}~--~\eqref{eq: ODE2} coincide if $\sigma_0 = \sigma_2$. %, sharing a two-dimensional solution space. % In the complementary case, i.e., when the ODEs are distinct, it is not obvious that a nontrivial common solution exists at all.
We analyze below separately the different $\sigmaseq$ allowed by the selection rules \eqref{eq: form of 3pt function with selection}, finding that the asymptotics
\begin{align} \label{eq: required asymptotics for the ODE for 4pt conformal block}
g_{\sigmaseq}(z)\sim\; & C_{\sigma_{0}}^{\sigma_{1}}C_{\sigma_{1}}^{\sigma_{2}}\;z^{h(\sigma_{1})-h(\sigma_{0})-h} \qquad \text{ as } z \to 0 
\end{align}
obtained from~\eqref{eq: asymptotics for x2}~--~\eqref{eq: asymptotics for x1} specify a unique solution to the first ODE~\eqref{eq: ODE} in all cases. These solutions are explicit, and can be verified to also satisfy the second ODE~\eqref{eq: ODE2}.
% \red{[the solution uniquely? depending on the b.c.?]}


%Such an ODE in principle has a two-dimensional solution space, but the %above 
%asymptotics
%\begin{align} \label{eq: required asymptotics for the ODE for 4pt conformal block}
%g_{\sigmaseq}(z)\sim\; & C_{\sigma_{0}}^{\sigma_{1}}C_{\sigma_{1}}^{\sigma_{2}}\;z^{h(\sigma_{1})-h(\sigma_{0})-h} \qquad \text{ as } z \to 0 
%\end{align}
%obtained from~\eqref{eq: asymptotics for x2}~--~\eqref{eq: asymptotics for x1}
%turn out to be enough to fix the solutions. 
%% \red{[the solution uniquely? depending on the b.c.?]}
%We analyze different cases allowed by the selection rules separately.

%Let us d
Denote $\sigma_{0}=\lambda$. %, and note that b
By the selection rules, there are %then 
four possibilities when $n=2$, which we label as follows:~
\begin{center}
\begin{tabular}{c|cccc}
 &
 \textbf{up-wedge}  \qquad  &
 \textbf{down-wedge}  \qquad &
 \textbf{up-slope}  \qquad &
 \textbf{down-slope}  \qquad \\
\hline
$\sigmaseq$ &
 $(\lambda,\lambda+1,\lambda)$ &
 $(\lambda,\lambda-1,\lambda)$ &
 $(\lambda,\lambda+1,\lambda+2)$ &
 $(\lambda,\lambda-1,\lambda-2)$ \\
abbreviation &
 $\upwedge$ &
 $\downwedge$ &
 $\upslope$ &
 $\downslope$ \\
Figure &
~\ref{fig: up-wedge} &
~\ref{fig: down-wedge} &
~\ref{fig: up-slope} &
~\ref{fig: down-slope} \\
\end{tabular}
\smallskip
\end{center}
% % \begin{description}
% % \item [{up-slope}] $\sigmaseq=(\lambda,\lambda+1,\lambda+2)$, abbreviated
% % $\sigmaseq=\upslope$
% % \item [{down-slope}] $\sigmaseq=(\lambda,\lambda-1,\lambda-2)$, abbreviated
% % $\sigmaseq=\downslope$
% % \item [{up-wedge}] $\sigmaseq=(\lambda,\lambda+1,\lambda)$, abbreviated
% % $\sigmaseq=\upwedge$
% % \item [{down-wedge}] $\sigmaseq=(\lambda,\lambda-1,\lambda)$, abbreviated
% % $\sigmaseq=\downwedge$.
% % \end{description}
%
In the case of an up-slope, the only solution of the %second order
ODE~\eqref{eq: ODE} with the correct asymptotics~\eqref{eq: required asymptotics for the ODE for 4pt conformal block} is
\begin{align*}
g_{\upslope}(z)=\; & C_{\lambda}^{+}C_{\lambda+1}^{+}\times(1-z)^{\frac{2}{\kappa}}\times z^{h(\lambda+1)-h(\lambda)-h}.
\end{align*}
This also satisfies the ODE~\eqref{eq: ODE2}. A similar conclusion holds in the case of a down-slope:
\begin{align*}
g_{\downslope}(z)=\; & C_{\lambda}^{-}C_{\lambda-1}^{-}\times(1-z)^{\frac{2}{\kappa}}\times z^{h(\lambda-1)-h(\lambda)-h}.
\end{align*}
In the case of an up-wedge, the two ODEs coincide. The unique solution with the asymptotics~\eqref{eq: required asymptotics for the ODE for 4pt conformal block} is a slightly more complicated,
% \blue{we find the}
non-degenerate hypergeometric function
\begin{align*}
g_{\upwedge}(z)=\; & C_{\lambda}^{+}C_{\lambda+1}^{-}\,z^{\frac{2\lambda}{\kappa}}(1-z)^{\frac{\kappa-6}{\kappa}}\,\twoFone\Big(\frac{\kappa-4}{\kappa},\frac{4\lambda}{\kappa};\frac{4\lambda+4}{\kappa};z\Big).
\end{align*}
Similarly, in the case of a down-wedge, 
% \blue{we find} 
the solution is
\begin{align*}
g_{\downwedge}(z)=\; & C_{\lambda}^{-}C_{\lambda-1}^{+}\,z^{\frac{\kappa-2\lambda-4}{\kappa}}(1-z)^{\frac{\kappa-6}{\kappa}}\,\twoFone\Big(\frac{\kappa-4}{\kappa},\frac{2\kappa-8-4\lambda}{\kappa};\frac{2\kappa-4\lambda-4}{\kappa};z\Big).
\end{align*}

The asymptotics as $z\to1$ of such hypergeometric functions can be
obtained from the identities
$\twoFone(a,b;c;0)=1$ and~\cite[Equation~(15.3.6)]{AS-handbook}:
\begin{align*}
\twoFone(a,b;c;z)=\; & (1-z)^{c-a-b}\frac{\Gamma(c)\,\Gamma(a+b-c)}{\Gamma(a)\,\Gamma(b)}\twoFone(c-a,c-b;c-a-b+1;1-z)\\
 & +\frac{\Gamma(c)\,\Gamma(c-a-b)}{\Gamma(c-a)\,\Gamma(c-b)}\twoFone(a,b;a+b-c+1;1-z) .
\end{align*}
%and the value $\twoFone(a,b;c;0)=1$. 
Because we assume $0<\kappa<8$, the parameters $a,b,c$
of the hypergeometric functions in both $g_{\upwedge}(z)$ and $g_{\downwedge}(z)$
satisfy $c-a-b=\frac{8-\kappa}{\kappa}>0$. Thus, in the limit $z\to1$
of the hypergeometric function, the first term above vanishes and the second
term tends to $\frac{\Gamma(c)\,\Gamma(c-a-b)}{\Gamma(c-a)\,\Gamma(c-b)}$.
%In other words, 
This shows that we have
\begin{align*}
(1-z)^{\frac{6-\kappa}{\kappa}}\times g_{\upwedge}(z)\quad \longrightarrow \quad\; & C_{\lambda}^{+}C_{\lambda+1}^{-}\,\frac{\Gamma\big(\frac{4+4\lambda}{\kappa}\big)\,\Gamma\big(\frac{8-\kappa}{\kappa}\big)}{\Gamma\big(\frac{8-\kappa+4\lambda}{\kappa}\big)\,\Gamma\big(\frac{4}{\kappa}\big)}\\
(1-z)^{\frac{6-\kappa}{\kappa}}\times g_{\downwedge}(z)\quad \longrightarrow \quad\; & C_{\lambda}^{-}C_{\lambda-1}^{+}\,\frac{\Gamma\big(\frac{2\kappa-4-4\lambda}{\kappa}\big)\,\Gamma\big(\frac{8-\kappa}{\kappa}\big)}{\Gamma\big(\frac{4}{\kappa}\big)\,\Gamma\big(\frac{\kappa-4\lambda}{\kappa}\big)},
\end{align*}
and we can write down explicit asymptotics of the conformal
block functions $U_{\sigmaseq}(x_{1},x_{2})$ as %$x_{1},x_{2}\to\hat{x}$:
$x_{1},x_{2} \to \xi$:
\begin{align}
\frac{U_{\upwedge}(x_{1},x_{2})}{(x_{2}-x_{1})^{-2h}}\quad \longrightarrow \quad\; & C_{\lambda}^{+}C_{\lambda+1}^{-}\,\frac{\Gamma\big(\frac{4+4\lambda}{\kappa}\big)\,\Gamma\big(\frac{8-\kappa}{\kappa}\big)}{\Gamma\big(\frac{8-\kappa+4\lambda}{\kappa}\big)\,\Gamma\big(\frac{4}{\kappa}\big)}\label{eq: non-normalized upwedge asymptotics}\\
\frac{U_{\downwedge}(x_{1},x_{2})}{(x_{2}-x_{1})^{-2h}}\quad \longrightarrow \quad\; & C_{\lambda}^{-}C_{\lambda-1}^{+}\,\frac{\Gamma\big(\frac{2\kappa-4-4\lambda}{\kappa}\big)\,\Gamma\big(\frac{8-\kappa}{\kappa}\big)}{\Gamma\big(\frac{4}{\kappa}\big)\,\Gamma\big(\frac{\kappa-4\lambda}{\kappa}\big)},\label{eq: non-normalized downwedge asymptotics}\\
\frac{U_{\upslope}(x_{1},x_{2})}{(x_{2}-x_{1})^{h(2)-2h}}\quad \longrightarrow \quad\; & C_{\lambda}^{+}C_{\lambda+1}^{+} \; 
%\hat{x}
\xi^{h(\lambda+2)-h(\lambda)-h(2)}\label{eq: non-normalized upslope asymptotics}\\
\frac{U_{\downslope}(x_{1},x_{2})}{(x_{2}-x_{1})^{h(2)-2h}}\quad \longrightarrow \quad\; & C_{\lambda}^{-}C_{\lambda-1}^{-} \; 
%\hat{x}
\xi^{h(\lambda-2)-h(\lambda)-h(2)}.\label{eq: non-normalized downslope asymptotics}
\end{align}
The last two expressions above,~\eqref{eq: non-normalized upslope asymptotics} and~\eqref{eq: non-normalized downslope asymptotics},
are proportional to one-point conformal block functions from $\Qrep_{\lambda}$
to $\Qrep_{\lambda\pm2}$ %by 
of a primary field operator of conformal
weight $h(2)=\frac{8-\kappa}{\kappa}$, whereas the first two,
\eqref{eq: non-normalized upwedge asymptotics} and~\eqref{eq: non-normalized downwedge asymptotics}, are
proportional to an one-point conformal block function from $\Qrep_{\lambda}$ to itself %by 
of
a primary field operator of conformal weight $h(0)=0$. %---
%namely 
The latter is
the identity operator, whose one-point conformal block function is just the constant $1$.
% % % % % % % \begin{align}
% % % % % % % U_{(\lambda,\lambda)}(x)=\; & \dualpairing{\hwvec_{\lambda}^{*}}{\hwvec_{\lambda}}=1.\label{eq: identity operator normalization}
% % % % % % % \end{align}
%Since there is some freedom in the normalizations, w
% % % \blue{[text removed]}
We now choose the
normalization constants $C_{\lambda}^{\pm}$ so that the coefficient of the identity
operator in~\eqref{eq: non-normalized upwedge asymptotics} %is unity,
equals one,
i.e., we set %for all $\lambda>0$
\begin{align*}
C_{\lambda}^{-} = \; & \frac{1}{C_{\lambda-1}^{+}}\times\frac{\Gamma\big(\frac{4-\kappa+4\lambda}{\kappa}\big)\,\Gamma\big(\frac{4}{\kappa}\big)}{\Gamma\big(\frac{4\lambda}{\kappa}\big)\,\Gamma\big(\frac{8-\kappa}{\kappa}\big)} \qquad
\text{for all $\lambda>0$.}
\end{align*}
%for all $\lambda>0$. 
Then, the coefficients $C_{\lambda}^{+}$ are the remaining free parameters.
The coefficient of the identity operator in~\eqref{eq: non-normalized downwedge asymptotics}
then becomes a ratio of gamma-functions, which can be further simplified to
%using the identity $\Gamma(w)\Gamma(1-w) = \pi / \sin(\pi w)$ % $\Gamma(w)\Gamma(1-w) = \frac{\pi}{\sin(\pi w)}$
%twice as follows
\begin{align*}
C_{\lambda}^{-}C_{\lambda-1}^{+}\,\frac{\Gamma\big(\frac{2\kappa-4-4\lambda}{\kappa}\big)\,\Gamma\big(\frac{8-\kappa}{\kappa}\big)}{\Gamma\big(\frac{4}{\kappa}\big)\,\Gamma\big(\frac{\kappa-4\lambda}{\kappa}\big)}=\; & \frac{\Gamma\big(\frac{2\kappa-4-4\lambda}{\kappa}\big)\,\Gamma\big(\frac{4-\kappa+4\lambda}{\kappa}\big)}{\Gamma\big(\frac{\kappa-4\lambda}{\kappa}\big)\,\Gamma\big(\frac{4\lambda}{\kappa}\big)}\\
=\; & \frac{\sin\big(\pi\frac{4\lambda}{\kappa}\big)}{\sin\big(\pi(\frac{4\lambda+4}{\kappa}-1)\big)}=-\frac{\sin\big(\pi\frac{4\lambda}{\kappa}\big)}{\sin\big(\pi\frac{4\lambda+4}{\kappa}\big)} ,
\end{align*}
using the identity $\Gamma(w)\Gamma(1-w) = \pi / \sin(\pi w)$ twice.
Introducing the parameter $q=e^{\ii\pi4/\kappa}$ and $q$-integers
$\qnum n=\frac{q^{n}-q^{-n}}{q-q^{-1}}=\frac{\sin(4\pi n/\kappa)}{\sin(4\pi/\kappa)}$,
this takes the simple form
\begin{align*}
C_{\lambda}^{-}C_{\lambda-1}^{+}\,\frac{\Gamma\big(\frac{2\kappa-4-4\lambda}{\kappa}\big)\,\Gamma\big(\frac{8-\kappa}{\kappa}\big)}{\Gamma\big(\frac{4}{\kappa}\big)\,\Gamma\big(\frac{\kappa-4\lambda}{\kappa}\big)}=\; & -\frac{\qnum{\lambda}}{\qnum{\lambda+1}}.
\end{align*}
With the chosen normalization convention and when $0<\kappa<8$, the
leading asymptotics~\eqref{eq: non-normalized upwedge asymptotics}~--~\eqref{eq: non-normalized downslope asymptotics}
as $x_{1},x_{2}\to\xi\in(0,\infty)$ of the conformal block functions
can thus be summarized as
\begin{align*} % \label{eq: asy}
\frac{U_{\sigmaseq}(x_{1},x_{2})}{(x_{2}-x_{1})^{-2h}}\quad \longrightarrow \quad\; & \begin{cases}
1 & \text{if }\sigmaseq=\upwedge\\
-\frac{\qnum{\lambda}}{\qnum{\lambda+1}} & \text{if }\sigmaseq=\downwedge\\
0 & \text{if }\sigmaseq=\upslope\text{ or }\sigmaseq=\downslope.
\end{cases}
\end{align*}
% \blue{[Eve: Why is the exponent $\Delta$ not used here but it is used later?]}

In the case of general $n$ and $\sigmaseq=(\sigma_{0},\sigma_{1},\ldots,\sigma_{n})$,
the leading asymptotics on pairwise diagonals can be inferred recursively from the above calculation. 
% % % % % % \blue{[text removed; it was not clear nor useful.]}
%that expressed the leading asymptotics of the composition
%$\PrimaryBlock{\sigma_{j}}{\sigma_{j+1}}(x_{j+1})\,\PrimaryBlock{\sigma_{j-1}}{\sigma_{j}}(x_{j})$
%as a multiple of the identity operator in the case of a wedge, $\sigma_{j-1} = \sigma_{j+1}$.
Specifically, we get %, as $x_{j},x_{j+1}\to\xi\in(x_{j-1},x_{j+2})$
\begin{align*}
\frac{U_{\sigmaseq}(x_{1},\ldots,x_{n})}{(x_{j+1}-x_{j})^{-2h}}\quad \longrightarrow \quad\; & \begin{cases}
U_{\hat{\sigmaseq}}(x_{1},\ldots,x_{j-1},x_{j+2},\ldots,x_{n}) & \text{if $\sigma_{j-1} = \sigma_{j+1} = \sigma_j -1$}\\
-\frac{\qnum{\sigma_{j}+1}}{\qnum{\sigma_{j}+2}}\times U_{\hat{\sigmaseq}}(x_{1},\ldots,x_{j-1},x_{j+2},\ldots,x_{n}) & \text{if $\sigma_{j-1} = \sigma_{j+1} = \sigma_j +1$}\\
0 & \text{if $\sigma_{j-1} \neq \sigma_{j+1}$} ,
\end{cases}
\end{align*}
as $x_{j},x_{j+1}\to\xi\in(x_{j-1},x_{j+2})$,
where we denote
$\hat{\sigmaseq} = (\sigma_0, \sigma_1, \ldots, \sigma_{j-1}, \sigma_{j+2}, \ldots, \sigma_{2N})$.

\subsection{Defining properties of conformal block functions}
\label{sub: defining conformal block functions}

So far in this section we have provided background on conformal block functions, so as to
have a self-contained justification of their properties that we use as their
definition %of conformal block functions 
in the rest of this article.
% % % % % We now list these defining properties in the form that they will be used.
We %will 
only consider the conformal block functions that contribute to
vacuum expected values, in which case $\sigma_0 = 0$ and $\sigma_n = 0$,
and $n$ is necessarily even: $n=2N$. The sequence
$(\sigma_0 , \sigma_1 , \ldots, \sigma_n)$ then forms a Dyck path
of $n=2N$ steps, see Figure~\ref{fig: Dyck paths}.
% We denote the set of Dyck paths of $2N$ steps by $\DP_N$.
% The number of such Dyck paths is a Catalan number,
% \begin{align*}
% \# \DP_N = \Catalan_N = \frac{1}{N+1} \binom{2N}{N} .
% \end{align*}
%Dyck paths will also from here on typically be denoted by $\alpha$ %or $\beta$
%instead of the notation $\sigmaseq$, %used in the general case,
%and we use the notation
Instead of the notation $\sigmaseq$, we use the notation $\alpha \in \DP_N$ for this Dyck path, and
\begin{align*}
\ConfBlockFun_\alpha (x_1 , \ldots , x_{2N})
\end{align*}
for the corresponding conformal block function.

% The conformal block functions are functions
% \begin{align*}
% \ConfBlockFun_\alpha (x_1 , \ldots , x_{2N})
% \end{align*}
% of an even number $n = 2N$ of variables, which are
% indexed by Dyck paths $\alpha$ of length $2N$, i.e.,
% sequences $\alpha = (\alpha(0) , \alpha(1) , \ldots , \alpha(2N))$ of non-negative integers
% with $|\alpha(j) - \alpha(j-1)|=1$ for all $j$ and $\alpha(0) = \alpha(2N) = 0$.

% % % % % % \red{[Eve: tags are not good in the following. I changed them so as to be different than the late tags (PDE) and (COV),
% % % % % % but they are no good either. Feel free to invent better tags!]}

We next list the defining properties of
$\big(\ConfBlockFun_\alpha\big)_{\alpha \in \DP}$ in the form that they will be used.
% Our first main result, Theorem~\ref{thm: change of basis theorem}, contains
% in particular the statement that the functions are uniquely determined by these properties.
% % % % % % The defining properties of the functions $\ConfBlockFun_\alpha$ are the following. %\alexmod{We set $\ConfBlockFun_\emptywalk = 1$, and for all other Dyck paths $\alpha$\ldots}
% % % % % % %The functions $\ConfBlockFun_\alpha$ 
% % % % % % With $h=h(1)=\frac{6-\kappa}{2\kappa}$, they
% % % % % % satisfy the partial differential equations
Denote $h = \frac{6-\kappa}{2\kappa}$ as before.
The required properties of $\ConfBlockFun_\alpha$ for $\alpha \in \DP_N$
are the partial differential equations
% \blue{[c.f.~\eqref{eq: PDEs for conformal block functions} with~\eqref{eq: translation invariance}]}
\begin{align}
% \label{eq: PDE for conformal blocks} \tag{$\ConfBlockFun$-PDE} 
\label{eq: PDE for conformal blocks} \tag{PDE} 
\left[ \frac{\kappa}{2} \pdder{x_j}
    + \sum_{\substack{i=1,\ldots,2N \\ i\neq j}} \Big( \frac{2}{x_i-x_j} \pder{x_i} - \frac{2 h}{(x_i-x_j)^2} \Big) \right] \; & \ConfBlockFun_\alpha (x_1 , \ldots, x_{2N}) = 0 \\
\nonumber
& \text{for all } j \in \set{1,\ldots,2N} ,
    %j=1,\ldots,2N ,
\end{align}
the M\"obius covariance
% \blue{[c.f.~\eqref{eq: Mobius covariance}]}
\begin{align}
% \label{eq: COV for conformal blocks} \tag{$\ConfBlockFun$-COV} 
\label{eq: COV for conformal blocks} \tag{COV} 
& \ConfBlockFun_\alpha(x_1 , \ldots, x_{2N}) = 
    \prod_{i=1}^{2 N} \Mob'(x_i)^{h} \times \ConfBlockFun_\alpha (\Mob(x_1) , \ldots, \Mob(x_{2N}))  \\
\nonumber
& \text{for all } \Mob(z) = \frac{a z + b}{c z + d}, \; \text{ with } a,b,c,d \in \bR, \; ad-bc > 0, 
 \text{ such that } \Mob(x_1) < \cdots < \Mob(x_{2N}) ,
\end{align}
and the %following 
recursive asymptotics properties
% \blue{[c.f.~\eqref{eq: asy}]:} 
\begin{align}\label{eq: ASY for conformal blocks} \tag{$\ConfBlockFun$-ASY}
\lim_{x_j , x_{j+1} \to \xi} 
\frac{\ConfBlockFun_\alpha (x_1 , \ldots, x_{2N})}{(x_{j+1} - x_j)^{-2h}}
= \; & \begin{cases}
0 & \text{if } \slopeat{j} \in \alpha \\
   \ConfBlockFun_{\alpha \removeupwedge{j}} (x_1, \ldots, x_{j-1} , x_{j+2} , \ldots, x_{2N}) 
& \text{if } \upwedgeat{j} \in \alpha \\
-\frac{\qnum{\alpha(j)+1}}{\qnum{\alpha(j)+2}} \times
\ConfBlockFun_{\alpha \removedownwedge{j}} (x_1, \ldots, x_{j-1} , x_{j+2} , \ldots, x_{2N})  & \text{if } \downwedgeat{j} \in \alpha ,
    \end{cases}  \\
\nonumber 
& \text{for any $j \in \set{1, \ldots, 2N-1}$ and $\xi \in (x_{j-1}, x_{j+2})$,}
\end{align}
%
%for any $j \in \set{1, \ldots, 2N-1}$ and $\xi \in (x_{j-1}, x_{j+2})$,
%\begin{align} \label{eq: ASY for conformal blocks} \tag{$\ConfBlockFun$-ASY} 
%& \lim_{x_j , x_{j+1} \to \xi} 
%\frac{\ConfBlockFun_\alpha (x_1 , \ldots, x_{2N})}{(x_{j+1} - x_j)^{\Delta}}
%= \begin{cases}
%0 & \text{if } \slopeat{j} \in \alpha \\
%   \ConfBlockFun_{\alpha \removeupwedge{j}} (x_1, \ldots, x_{j-1} , x_{j+2} , \ldots, x_{2N}) 
%& \text{if } \upwedgeat{j} \in \alpha \\
%-\frac{\qnum{\alpha(j)+1}}{\qnum{\alpha(j)+2}} \times
%\ConfBlockFun_{\alpha \removedownwedge{j}} (x_1, \ldots, x_{j-1} , x_{j+2} , \ldots, x_{2N})  & \text{if } \downwedgeat{j} \in \alpha ,
%    \end{cases} 
%\end{align}
where % the exponent of the asymptotics is $\Delta = 1-\frac{6}{\kappa}$, and
the square bracket expressions are the $q$-integers
$\qnum{n} = \frac{q^n - q^{-n}}{q - q^{-1}}$
% \begin{align}\label{eq: q numbers}
% \qnum{m} = \; & \frac{q^{m}-q^{-m}}{q-q^{-1}}
%     =  q^{m-1}+q^{m-3}+\cdots+q^{3-m}+q^{1-m},
% \end{align}
with the parameter $q = e^{\ii \pi 4 / \kappa}$ depending on $\kappa$.
% \blue{[Eve: do we use the exponent $\Delta$ much? Could we just write it explicitly? This was done in~\eqref{eq: asy}.]}
Finally, the case $N=0$ fixes an overall normalization when
we require that $\ConfBlockFun_\emptywalk = 1$ for the Dyck path $\emptywalk \in \DP_0$.

% % { \color{magenta}
In one of the main results of this article, Theorem~\ref{thm: change of basis theorem} in Section~\ref{sub: change of basis results}, 
we in particular prove that the conformal block functions $\ConfBlockFun_{\alpha}$ are uniquely determined by 
the properties above.
% % % % % % % % the properties~\eqref{eq: PDE for conformal blocks}, \eqref{eq: COV for conformal blocks},
% % % % % % % % and~\eqref{eq: ASY for conformal blocks}.
% % }

\bigskip

\section{Change of basis between conformal block functions and pure partition functions}
\label{sec: change of basis}

%\subsection{Solution space to partial differential equations and covariance}

% \red{[Eve: The pure partition functions are not defined sufficiently well at this point.]}

This section contains our first main result. It states first of all that
the conformal block functions and the multiple $\SLE$ pure partition functions, 
whose definition will be recalled below in Section~\ref{sub: pure partition functions}, both
form a basis of the same solution space of the system~\eqref{eq: PDE for conformal blocks} of partial differential equations.
Moreover, it gives an explicit $q$-combinatorial formula for the change of basis matrix.
% 
% % % This section contains our first main result: that both the conformal block functions
% % % and the multiple $\SLE$ pure partition functions form a basis of the same solution space
% % % of the system~\eqref{eq: PDE for conformal blocks} of partial differential equations,
% % % and we have explicit $q$-combinatorial formulas for the change of basis matrices.
% The precise definition of multiple SLE pure partition functions is recalled
% in Section~\ref{sub: pure partition functions} below.
% 
% Section~\ref{sub: pure partition functions} below

We begin by discussing the space of functions.
Fix $N \in \bN$, and consider the system of $2N$ second order partial differential
equations %~\eqref{eq: PDE for conformal blocks}
and M\"obius covariance conditions %~\eqref{eq: COV for conformal blocks}
as in Section~\ref{sub: defining conformal block functions},
\begin{align}
\label{eq: PDE for F} \tag{PDE} 
\left[ \frac{\kappa}{2} \pdder{x_j}
    + \sum_{\substack{i=1,\ldots,2N \\ i\neq j}} \left( \frac{2}{x_i-x_j} \pder{x_i} - \frac{2 h}{(x_i-x_j)^2} \right) \right] F (x_1 , \ldots, x_{2N}) = 0 & \\
\nonumber
\text{for all } j \in \set{1,\ldots,2N} , & \\
    %j=1,\ldots,2N , \\
%\end{align}
%M\"obius covariance
%\begin{align}
%\label{eq: COV for F} \tag{COV} 
%\begin{split}
%& F (x_1 , \ldots, x_{2N}) = 
%    \prod_{j=1}^{2 N} \Mob'(x_j)^{h} \times F (\Mob(x_1) , \ldots, \Mob(x_{2N}))  \\
%\nonumber
%& \text{for all } \Mob(z) = \frac{a z + b}{c z + d}, \; \text{ with } a,b,c,d \in \bR, \; ad-bc > 0, 
% \text{ such that } \Mob(x_1) < \cdots < \Mob(x_{2N}) ,
%\end{split}
%\end{align}
%\begin{align}
\label{eq: COV for F} \tag{COV} 
F (x_1 , \ldots, x_{2N}) = 
    \prod_{i=1}^{2 N} \Mob'(x_i)^{h} \times F (\Mob(x_1) , \ldots, \Mob(x_{2N})) & \\
\nonumber
% \text{for all M\"obius maps $\Mob$ as in~\eqref{eq: COV for conformal blocks},} &
\text{for all } \Mob(z) = \frac{a z + b}{c z + d}, \; \text{ with } a,b,c,d \in \bR, \; ad-bc > 0, 
    \text{ such that } \Mob(x_1) < \cdots < \Mob(x_{2N}) , &
%    \\
%\nonumber
%& \text{for all } \Mob(z) = \frac{a z + b}{c z + d}, \; \text{ with } a,b,c,d \in \bR, \; ad-bc > 0, 
% \text{ such that } \Mob(x_1) < \cdots < \Mob(x_{2N}) ,
\end{align}
for complex valued functions $F$ defined on the set
\begin{align*}
% F \colon \; & \chamber_{2N} \to \bC \text{ , where}\\
\chamber_{2N} := \; & \set{ (x_1, x_2, \ldots, x_{2N}) \in  \R^n \; \Big| \; x_1 < x_2 < \cdots < x_{2N} }
\end{align*}
of $2N$-tuples of real variables in increasing order. %\alexmod{For technical reasons, we also have to require} 
Require moreover
that $F$ has
at most polynomial growth on pairwise diagonals and at infinity in the sense that
\begin{align} 
\label{eq: power law growth bound} \tag{GROW}
\begin{split}
& \text{there exist positive constants $C,p > 0$ such that we have }  \\
& \big| F(x_1,\ldots,x_{2N}) \big| \leq C \times
\prod_{i < j} \max\left( (x_j - x_i)^p, (x_j - x_i)^{-p} \right)
\qquad \text{for all } (x_1,\ldots,x_{2N}) \in \chamber_{2N}.
\end{split}
\end{align}
%\begin{align} 
%\label{eq: power law growth bound} \tag{GROW}
%& \text{there exist positive constants $C,p > 0$ such that we have }  \\
%& \big| F(x_1,\ldots,x_{2N}) \big| \leq C \times
%\prod_{i < j} \max\left( (x_j - x_i)^p, (x_j - x_i)^{-p} \right)
%\qquad \text{for all } (x_1,\ldots,x_{2N}) \in \chamber_{2N}.
%\nonumber
%\end{align}
%The space of solutions that we consider is
We consider the following space of solutions:
\begin{align*} %\label{eq: solution space}
\Sol_N := \set{ F \colon \chamber_{2N} \to \bC \;\big|\; F \text{ satisfies 
\eqref{eq: PDE for F},~\eqref{eq: COV for F},
and~\eqref{eq: power law growth bound} } }.
\end{align*}
The dimension of this space is known to be the $N$:th Catalan number, $\dmn \Sol_N = \Catalan_N$,
%This space is known to be of dimension given by the Catalan number $\Catalan_N$,
and %a basis for it is 
the multiple $\SLE$ pure partition functions
% % % \blue{$\big(\PartF_\alpha\big)_{\alpha \in \DP_N}$ \alexmod{[A: notn not defd]} form a basis for $\Sol_N$},
form a basis for $\Sol_N$,
as we %will 
recall precisely in Section~\ref{sub: pure partition functions}. 
% % % % % % % % % % { \color{magenta}
% % % % % % % % % % We show in Theorem~\ref{thm: change of basis theorem} in Section~\ref{sub: change of basis results}
% % % % % % % % % % %we will show 
% % % % % % % % % % that the  
% % % % % % % % % % conditions~\eqref{eq: PDE for conformal blocks}, \eqref{eq: COV for conformal blocks}, and~\eqref{eq: ASY for conformal blocks}
% % % % % % % % % % listed in Section~\ref{sub: defining conformal block functions}
% % % % % % % % % % uniquely determine the conformal block functions 
% % % % % % % % % %  $\ConfBlockFun_\alpha$, \alexmod{$\alpha \in \DP_N$}, in the spaces~\eqref{eq: solution space}
% % % % % % % % % % and that for fixed $N$, \alexmod{these conformal block functions} 
% % % % % % % % % % %\blue{ $\ConfBlockFun_\alpha \in \Sol_N$}
% % % % % % % % % % form a basis of $\Sol_N$.
% % % % % % % % % % }

%\begin{thm}{\cite{KP-pure_partition_functions_of_multiple_SLEs,FK-solution_space_for_a_system_of_null_state_PDEs_1,FK-solution_space_for_a_system_of_null_state_PDEs_2,FK-solution_space_for_a_system_of_null_state_PDEs_3}}
%\label{thm: pure partition functions for generic kappa}
%For any $\kappa \in (0,8) \setminus \bQ$, there exists a unique collection 
%$\big(\PartF_{\alpha}^{(\kappa)}\big)_{\alpha \in \DP}$ of solutions
%$\PartF_{\alpha}^{(\kappa)}  \in \Sol_{\sizeofppp{\alpha}}$
%satisfying 
%the asymptotics properties~\eqref{eq: ASY for multiple SLEs},
%such that $\PartF_{\emptywalk}^{(\kappa)} \equiv 1$.
%For any fixed $N\in\bZnn$, the set 
%$\{\PartF_{\alpha}^{(\kappa)} \;|\; \alpha \in \DP_N\}$ is a basis of the 
%$\Catalan_N$-dimensional solution space $\Sol_N$.
%\end{thm}

%The main statement of 
%Theorem~\ref{thm: conformal blocks and pure partition functions}
%is that the change of basis between the basis of multiple $\SLEk$ pure partition 
%functions and the basis of conformal blocks is established via 
%a weighted incidence matrix, where the weights of the Dyck tiles 
%are given in terms of $q$-integers, with 
%%$\kappa \in (0,8) \setminus \bQ$ and 
%$q = e^{\ii \pi 4 / \kappa}$.





\subsection{Multiple $\SLE$ pure partition functions}
\label{sub: pure partition functions}
In many situations in planar statistical physics, boundary conditions force the
existence of multiple macroscopic interfaces. In the scaling limit at criticality, such
interfaces are described by multiple $\SLEk$ curves %--- 
with $\kappa$ depending on the model. Contrary to, e.g., a single chordal $\SLEk$ curve,
the law of a multiple $\SLEk$ with fixed %$\kappa$ and
number $N$ of curves
is not unique~--- instead, the possible laws form a non-trivial convex set.
It is thus natural to express any multiple $\SLEk$ as a convex combination of
the extremal points of this convex set: the pure geometries,
in which the curves connect the starting points of the interfaces  
pairwise in a deterministic planar pair partition. 
% \blue{way in the plane.}
The pure geometries are thus
indexed by planar pair partitions, which in turn are in bijection with Dyck paths.
For background on multiple $\SLE$s, we refer 
to~\cite{BBK-multiple_SLEs,Dubedat-commutation,LK-configurational_measure,KP-pure_partition_functions_of_multiple_SLEs},
and for results on their role as scaling limits,
to~\cite{CS-universality_in_2d_Ising, Izyurov-critical_Ising_interfaces_in_multiply_connected_domains,
KKP-boundary_correlations_in_planar_LERW_and_UST,
PH-Global_multiple_SLEs_and_pure_partition_functions, Wu-convergence_of_Ising_interfaces_to_hypergeometric_SLE,
Beffara_Peltola_Wu-Uniqueness_of_global_multiple_SLEs, KS-configurations_of_FK_interfaces}.

For the purposes of this article, the important aspect of
multiple $\SLE$s is their partition functions $\PartF$, which essentially define the
multiple $\SLEk$ by giving the Radon-Nikodym density of its law with respect to independent chordal
$\SLEk$ laws, see \cite{Dubedat-commutation, KP-pure_partition_functions_of_multiple_SLEs}.
%
In particular, each pure geometry with connectivity encoded by a Dyck path $\alpha$ has a
partition function denoted by $\PartF_\alpha$.
% % % % The partition functions of the pure geometries are called pure partition functions,
% % % % and denoted by $\PartF_\alpha$, with the Dyck path $\alpha$ describing the
% % % % planar pair partition of the connectivity.
% 
% % The partition functions of the pure geometries are called pure partition functions,
% % and denoted by $\PartF_\alpha$, with the Dyck path $\alpha$ describing the
% % planar pair partition of the connectivity.
These functions satisfy the partial differential equations~\eqref{eq: PDE for F} and
M\"obius covariance~\eqref{eq: COV for F} as before, and the following recursive asymptotics:
\begin{align}
\label{eq: ASY for multiple SLEs} \tag{$\PartF$-ASY} 
\lim_{x_j , x_{j+1} \to \xi} 
\frac{\PartF_\alpha(x_1 , \ldots, x_{2N})}{(x_{j+1} - x_j)^{-2h}}
= & \begin{cases}
    \PartF_{\alpha \removeupwedge{j} } (x_1, \ldots, x_{j-1} , x_{j+2} , \ldots, x_{2N}) & \text{ if } \upwedgeat{j} \in \alpha \\
    0 & \text{ if } \upwedgeat{j} \notin \alpha
    \end{cases} \\
\nonumber
& \text{for any $j \in \set{1, \ldots, 2N-1}$, and $\xi \in (x_{j-1}, x_{j+2})$.}
\end{align}
As stated in the following proposition, these requirements together with the
normalization condition $\PartF_\emptywalk = 1$ uniquely determine the functions $\PartF_\alpha$,
called the \emph{multiple SLE pure partition functions}.
% 
% For $\kappa \in (0,8) \setminus \bQ$,
% this determines uniquely pure partition functions in $\Sol_N$:
% 
% ABOVE PARAGRAPH BEFORE MODIFICATION BY ALEX ON 290817
%%%%For the purposes of this article, the important aspect of
%%%%multiple $\SLE$s is their partition functions $\PartF$, which essentially define the
%%%%multiple $\SLEk$ by giving the Radon-Nikodym density of its law with respect to independent chordal
%%%%$\SLEk$ laws. The partition functions of the pure geometries are called pure partition functions,
%%%%and denoted by $\PartF_\alpha$, with the Dyck path $\alpha \in \DP_N$ describing the
%%%%connectivity planar pair partition.
%%%%\alexmod{
%%%%The \emph{pure partition functions} are defined as the solutions of the partial differential equations~\eqref{eq: PDE for F} that satisfy the
%%%%M\"obius covariance~\eqref{eq: COV for F} as before, and the asymptotics~\eqref{eq: ASY for multiple SLEs} below.
%%%%For $\kappa \in (0,8) \setminus \bQ$,
%%%%they are thus well defined, by an explicit construction in~\cite{KP-pure_partition_functions_of_multiple_SLEs}:

%For $\kappa \in (0,8) \setminus \bQ$,
%the pure partition functions were constructed in~\cite{KP-pure_partition_functions_of_multiple_SLEs}, and for $\kappa = 2$ in \cite{KKP-boundary_correlations_in_planar_LERW_and_UST}.
%Their defining properties are the same partial differential equations~\eqref{eq: PDE for F} and
%M\"obius covariance~\eqref{eq: COV for F} as before, and asymptotics~\eqref{eq: ASY for multiple SLEs}
%given in the following.
\begin{prop}
%[\cite{KP-pure_partition_functions_of_multiple_SLEs,
%FK-solution_space_for_a_system_of_null_state_PDEs_3}]
\label{prop: solution space with power law bound}
Let $\kappa \in (0,8) \setminus \bQ$.
There exists a unique collection of functions $\big(\PartF_\alpha\big)_{\alpha \in \DP}$,
such that $\PartF_\alpha \in \Sol_N$ when $\alpha \in \DP_N$, $\PartF_\emptywalk = 1$,
and~\eqref{eq: ASY for multiple SLEs} holds for all $\alpha$.
Moreover, for any $N \in \bZnn$, the functions $\big( \PartF_\alpha \big)_{\alpha \in \DP_N}$
form a basis of the %following $\Catalan_N$-dimensional
solution space $\Sol_N$.
\end{prop}
% ABOVE PROP BEFORE MODIFICATION BY ALEX ON 290817
%%%%%\begin{prop}
%%%%%%[\cite{KP-pure_partition_functions_of_multiple_SLEs,
%%%%%%FK-solution_space_for_a_system_of_null_state_PDEs_3}]
%%%%%\label{prop: solution space with power law bound}
%%%%%Let $\kappa \in (0,8) \setminus \bQ$.
%%%%%There exists a unique collection of functions $\big(\PartF_\alpha\big)_{\alpha \in \DP}$,
%%%%%with $\PartF_\alpha \in \Sol_N$ when $\alpha \in \DP_N$, such that \alexmod{$\PartF_\emptywalk = 1$} and
%%%%%%%%
%%%%%%%%%for any %$\alpha$
%%%%%%%%for any $N \in \bZpos$, $\alpha \in \DP_N$, $j \in \set{1, \ldots, 2N-1}$, and $\xi \in (x_{j-1}, x_{j+2})$,
%%%%%%%%the following asymptotics hold:
%%%%%%%%\begin{align}
%%%%%%%%\label{eq: ASY for multiple SLEs} \tag{$\PartF$-ASY} 
%%%%%%%%& \lim_{x_j , x_{j+1} \to \xi} 
%%%%%%%%\frac{\PartF_\alpha(x_1 , \ldots, x_{2N})}{(x_{j+1} - x_j)^{\Delta}}
%%%%%%%%= \begin{cases}
%%%%%%%%    \PartF_{\alpha \removeupwedge{j} } (x_1, \ldots, x_{j-1} , x_{j+2} , \ldots, x_{2N}) & \text{ if } \upwedgeat{j} \in \alpha \\
%%%%%%%%    0 & \text{ if } \upwedgeat{j} \notin \alpha.
%%%%%%%%    \end{cases}
%%%%%%%%\end{align}
%%%%%%%%
%%%%%\begin{align}
%%%%%\label{eq: ASY for multiple SLEs} \tag{$\PartF$-ASY} 
%%%%%\lim_{x_j , x_{j+1} \to \xi} 
%%%%%\frac{\PartF_\alpha(x_1 , \ldots, x_{2N})}{(x_{j+1} - x_j)^{-2h}}
%%%%%= & \begin{cases}
%%%%%    \PartF_{\alpha \removeupwedge{j} } (x_1, \ldots, x_{j-1} , x_{j+2} , \ldots, x_{2N}) & \text{ if } \upwedgeat{j} \in \alpha \\
%%%%%    0 & \text{ if } \upwedgeat{j} \notin \alpha
%%%%%    \end{cases} \\
%%%%%\nonumber
%%%%%& \text{for any $j \in \set{1, \ldots, 2N-1}$, and $\xi \in (x_{j-1}, x_{j+2})$.}
%%%%%\end{align}
%%%%%Moreover, for any $N \in \bZnn$, the functions $\big( \PartF_\alpha \big)_{\alpha \in \DP_N}$
%%%%%form a basis of the %following $\Catalan_N$-dimensional
%%%%%solution space $\Sol_N$.
%%%%%%%%%that is, the space of solutions to the PDE 
%%%%%%%%%system~\eqref{eq: PDE for multiple SLEs} 
%%%%%%%%%and the M\"obius covariance~\eqref{eq: COV for multiple SLEs}, 
%%%%%%%%%subject to the following power law growth bound: 
%%%%%%%%where the power law growth bound is the following condition: 
%%%%%%%%%there exist constants $C,p \in \bR$ such that we have 
%%%%%%%%\begin{align}\label{eq: power law growth bound}
%%%%%%%%& \text{ there exist constants $C,p \in \bR$ such that we have } 
%%%%%%%%\nonumber \\
%%%%%%%%& |F(x_1,\ldots,x_{2N})| \leq C \times
%%%%%%%%\prod_{i < j} \max\left( (x_j - x_i)^p, (x_j - x_i)^{-p} \right)
%%%%%%%%\qquad \text{for all } (x_1,\ldots,x_{2N}) \in \chamber_{2N}.
%%%%%%%%\end{align}
%%%%%\end{prop}
\begin{proof}
By~\cite[Theorem~8]{FK-solution_space_for_a_system_of_null_state_PDEs_3}, 
%the dimension of the solution space equals a Catalan number:
we have
$\dmn \Sol_N = \Catalan_N$. 
On the other hand,~\cite[Theorem~4.1]{KP-pure_partition_functions_of_multiple_SLEs}
shows that the pure partition functions $\big( \PartF_\alpha \big)_{\alpha \in \DP_N}$
form a linearly independent set in this space
(the power-law bound~\eqref{eq: power law growth bound} can be verified from the explicit form of 
the functions as Coulomb gas integrals, 
see~\cite{KP-pure_partition_functions_of_multiple_SLEs, KP-conformally_covariant_boundary_correlation_functions_with_a_quantum_group}). 
The assertion follows, since $\# \DP_N = \Catalan_N$.
\end{proof}


%\subsection{The change of basis results}
\subsection{The change of basis result}
\label{sub: change of basis results}

% { \color{magenta}
% We now show that the conformal block functions are uniquely determined by the properties~\eqref{eq: PDE for conformal blocks}, \eqref{eq: COV for conformal blocks}, and~\eqref{eq: ASY for conformal blocks} listed
% in Section~\ref{sub: defining conformal block functions} and that they
% also form a basis of the solution spaces
% $\Sol_N$ for $N \in \bZnn$.
% }
%~\eqref{eq: solution space}.
% % We also show that the change of basis matrices between the conformal block functions
% % and the multiple $\SLE$ pure partition functions are weighted incidence
% % matrices of the parenthesis reversal relation.
We now show how to express the conformal block functions $\ConfBlockFun_\alpha$ in the basis of
the multiple $\SLE$ pure partition functions $\PartF_\alpha$ using weighted incidence
matrices of the parenthesis reversal relation. From this, it follows that the conformal
block functions are well-defined and also form a basis.
%{\color{red}TBA}


We take the weight of a Dyck tile $t$ at height $h_t$ to be
\begin{align}\label{eq: weights in terms of q-numbers}
w(t) := \frac{\qnum{h_t}}{\qnum{h_t+1}} ,
\end{align}
where $\qnum{n} = \frac{q^n - q^{-n}}{q - q^{-1}}$ and $q = e^{\ii \pi 4/\kappa}$ as before.
Denote by $\genMmat = (\genMmat_{\alpha, \beta})$ the correspondingly weighted incidence 
matrix~\eqref{eq: def of weighted incidence matrix}: its non-zero elements are
\begin{align} \label{eq: M}
\genMmat_{\alpha, \beta} = \prod_{t \in \nestedtilingof(\alpha / \beta)} (-w(t)) , \qquad \text{ for $\alpha \KWleq \beta$,}
\end{align}
where $\nestedtilingof(\alpha / \beta)$ is the nested Dyck tiling of the skew Young diagram 
$\alpha/\beta$.
Proposition~\ref{prop: weighted KW incidence matrix inversion} shows that the matrix 
$\genMmat$ is invertible and the non-zero matrix elements of its inverse are 
\begin{align}  \label{eq: Minv}
\genMinv_{\alpha, \beta} = \sum_{T \in \CItilingsof (\alpha/\beta)} \prod_{t \in T} w(t) , \qquad \text{ for $\alpha \DPleq \beta$,}
\end{align}
with $\CItilingsof (\alpha/\beta)$ the family of cover-inclusive Dyck tilings of 
the skew Young diagram $\alpha/\beta$.
Examples of these matrices are %depicted
shown in Figures~\ref{fig: quantum KW matrices only}~and~\ref{fig: quantum CIDT matrices only}.

\begin{thm}\label{thm: change of basis theorem}
% \blue{Let $\kappa \in (0,8) \setminus \bQ$.}
There exists a unique collection $\big(\ConfBlockFun_\alpha \big)_{\alpha \in \DP}$ %of conformal block functions  
such that $\ConfBlockFun_\alpha \in \Sol_N$ when $\alpha \in \DP_N$,
$\ConfBlockFun_\emptywalk = 1$, and 
%\blue{for any $N \in \bZpos$, $\alpha \in \DP_N$, $j \in \set{1, \ldots, 2N-1}$, and $\xi \in (x_{j-1}, x_{j+2})$,}
the asymptotics~\eqref{eq: ASY for conformal blocks} hold. %for all $\alpha$ and $j$.
% the following projection properties hold:
% \begin{align}
% & \hat{\pi}_j(\Coblobastwodim_\alpha) =
% \label{eq: coblo walk projection conditions alex}
% \begin{cases} 
% 0 & \text{if } \slopeat{j} \in \alpha \\
% \Coblobastwodim_{\alpha\removeupwedge{j}} 
% & \text{if } \upwedgeat{j} \in \alpha \\
% - \frac{\qnum{\alpha(j)+1}}{\qnum{\alpha(j)+2}} \times
% \Coblobastwodim_{\alpha \removedownwedge{j}} 
% & \text{if } \downwedgeat{j} \in \alpha
% \end{cases}
% \qquad\text{for all } j \in \set{1,\ldots,2N-1} .
% \end{align}
For any $\alpha \in \DP_N$, the function $\ConfBlockFun_\alpha$ % vector $\Coblobastwodim_\alpha$
of this collection can be written in the basis $\big( \PartF_\beta \big)_{\beta \in \DP_N}$ of
Proposition~\ref{prop: solution space with power law bound} as
\begin{align*}
\ConfBlockFun_\alpha =
\sum_{\beta \in \DP_N} \genMmat_{\alpha,\beta}\; \PartF_\beta,
\end{align*}
% \begin{align*}
% \Coblobastwodim_\alpha =
% \sum_{\beta \in \DP_N} \genMmat_{\alpha,\beta}\;\Puregeomtwodim_\beta,
% \end{align*}
where $\genMmat$ is the weighted incidence matrix of the parenthesis reversal
relation with weights~\eqref{eq: weights in terms of q-numbers}.
% $t$ given in terms of its height $h_t$ as
% \begin{align}\label{eq: weights in terms of q-numbers}
% w(t) = \frac{\qnum{h_t}}{\qnum{h_t+1}}
% \end{align}
Moreover, for any $N \in \bZnn$, the functions $\big( \ConfBlockFun_\alpha \big)_{\alpha \in \DP_N}$
form a basis of the %following $\Catalan_N$-dimensional
solution space $\Sol_N$ and
%Moreover, for fixed $N$, the collection $\big(\ConfBlockFun_\alpha\big)_{\alpha \in \DP_N}$ is a basis of $\Sol_N$, and
\begin{align*}
\PartF_\alpha =
\sum_{\beta \in \DP_N} \genMinv_{\alpha,\beta}\;\ConfBlockFun_\beta .
\end{align*}
% where $\genMinv$ is the inverse matrix of $\genMmat$,
% as in Theorem~\ref{thm:}.
\end{thm}

%
%{\color{red}[Eve: we should mention here that the change of basis formulas generalize Fomin's formulas in certain sense. 
%Fomin is only mentioned in sections 1 and 5 and two times in the abstract. So the referee is correct in that it is not very well indicated what the generalization is...]}

\begin{rem}
The above change of basis formulas %in Theorem~\ref{thm: change of basis theorem} 
can be regarded as analogues of Fomin's formulas in the following sense.
In a special planar case, general Fomin's formulas~\cite{Fomin-LERW_and_total_positivity} yield
linear relationships between determinants of discrete Green's functions and 
%The general Fomin's formulas~\cite{Fomin-LERW_and_total_positivity} specialize
%in the planar case to a relation between determinants of Green's functions and 
probabilities of certain planar connectivity events for the uniform spanning tree~\cite[Section~3.4]{KKP-boundary_correlations_in_planar_LERW_and_UST}.
%see~\cite{KW-boundary_partitions_in_trees_and_dimers, KKP-boundary_correlations_in_planar_LERW_and_UST}
%and~\cite[Section~3.4]{KKP-boundary_correlations_in_planar_LERW_and_UST}.
%The structure of this linear system is encoded in an invertible matrix $(\Mmat_{\alpha, \beta})$ indexed by Dyck paths, 
%whose non-zero entries are %$\pm 1$ %$+1$ or $-1$
%of the form~\eqref{eq: M} with unit tile weights $w=1$, 
%and appear at the same positions as in~\eqref{eq: M}, %~\eqref{eq: M}~--~\eqref{eq: Minv}, %that is, respectively 
%i.e., according to the combinatorial relation $\alpha \KWleq \beta$. 
%%(for $\Mmat$) and $\alpha \DPleq \beta$ (for $\Mmat^{-1}$).
%The structure of this linear 
This linear system is encoded in an invertible matrix $(\Mmat_{\alpha, \beta})$ indexed by Dyck paths, 
whose non-zero entries are of the form~\eqref{eq: M} with unit tile weights $w=1$, 
instead of the $q$-dependent weights~\eqref{eq: weights in terms of q-numbers}.
Thus, the structure of the non-zero entries is encoded in the same combinatorial relation $\alpha \KWleq \beta$. 
%Furthermore, they appear at the same positions as in~\eqref{eq: M}, %~\eqref{eq: M}~--~\eqref{eq: Minv}, %that is, respectively 
%i.e., according to the combinatorial relation $\alpha \KWleq \beta$. 
%(for $\Mmat$) and $\alpha \DPleq \beta$ (for $\Mmat^{-1}$).
In the scaling limit as the mesh of the graph tends to zero, these planar connectivity probabilities tend to
the pure partition functions of multiple $\SLE_\kappa$ for $\kappa = 2$,
and the determinants tend to a distinguished basis of the solution space $\Sol_N$ with $\kappa = 2$,
closely related to the conformal blocks
%which in fact can be %viewed as analogues of 
%regarded as the conformal blocks for $\kappa = 2$
--- 
see~\cite[Section~5]{KW-boundary_partitions_in_trees_and_dimers}
and~\cite[Theorems~3.12~and~4.1~and~Proposition~4.6]{KKP-boundary_correlations_in_planar_LERW_and_UST}. 
%
%(compare Theorem~\ref{thm: change of basis theorem} and Equation~\eqref{eq: ASY for conformal blocks} 
%with~\cite[Proposition~4.6 and Theorems~3.12~and~4.1]{KKP-boundary_correlations_in_planar_LERW_and_UST})
%%In the article~\cite{Fomin-LERW_and_total_positivity}, S.~Fomin found a combinatorial identity 
%%that expresses certain determinants of Green's functions of walks on a graph as linear combinations of weights of loop-erased walks.
%In the planar case, Fomin's identity simplifies significantly, yielding a relation between random walk excursion kernels and probabilities of certain planar connectivity events
%in terms of, e.g., loop-erased random walks, or equivalently, branches in a uniform spanning 
%tree~\cite{KW-boundary_partitions_in_trees_and_dimers, KKP-boundary_correlations_in_planar_LERW_and_UST}.
%Furthermore, in the scaling limit as the mesh of the graph tends to zero, these connectivity probabilities correspond with 
%the pure partition functions of multiple $\SLE_\kappa$ for $\kappa = 2$ 
%--- see~\cite[Section~5]{KW-boundary_partitions_in_trees_and_dimers}
%and~\cite[Theorems~3.12~and~4.1]{KKP-boundary_correlations_in_planar_LERW_and_UST}. 
%We note that even though the analogy with Theorem~\ref{thm: change of basis theorem} is clear, 
%it is not obvious how to take the $\kappa \to 2$ (i.e., $q \to 1$) limit to explicitly recover the formulas 
%appearing in~\cite{KW-boundary_partitions_in_trees_and_dimers, KKP-boundary_correlations_in_planar_LERW_and_UST}. 
%From the point of view of the quantum group discussed in the next section, the limit $q \to 1$ would correspond to some kind of a ``classical'' limit.
\end{rem}


\begin{proof}[Proof of Theorem~\ref{thm: change of basis theorem}]
Let $\big( \ConfBlockFun_\alpha \big)_{\alpha \in \DP}$ be any collection of functions such that
$\ConfBlockFun_\alpha \in \Sol_N$ for $\alpha \in \DP_N$
and the asymptotics~\eqref{eq: ASY for conformal blocks} hold with $\ConfBlockFun_\emptywalk = 1$.
Write $\ConfBlockFun_\alpha$ in the basis $\big( \PartF_\beta \big)_{\beta \in \DP_N}$ of $\Sol_N$ as
\begin{align*}
\ConfBlockFun_\alpha = \sum_{\beta \in \DP_N} M^{(N)}_{\alpha, \beta} \; \PartF_\beta ,
\end{align*}
which, for each $N \in \bN$, defines a matrix $M^{(N)} \in \bC^{\DP_N \times \DP_N}$.
% for any $\alpha \in \DP_N$ and any $N$.
The %recurrence relations
recursive asymptotics~\eqref{eq: ASY for conformal blocks} then
equivalently require that the matrices $M^{(N)}_{\alpha, \beta}$ satisfy the initial condition $M^{(0)}=1$ and the recursion
\begin{align}
\begin{split}
%\nonumber
& \lim_{x_j , x_{j+1} \to \xi} \frac{1}{(x_{j+1} - x_j)^{-2h}}
    \sum_{\beta \in \DP_N} M^{(N)}_{\alpha, \beta} \; \PartF_\beta (x_1 , \ldots, x_{2N}) \\
= \; & \begin{cases} 
0 & \text{if } \slopeat{j} \in \alpha \\
\label{eq: coblo walk projection conditions 2}
 \sum_{\hat{\beta} \in \DP_{N-1}} M^{(N-1)}_{\hat{\alpha}, \hat{\beta} } \; \PartF_{ \hat{\beta} } (x_1 , \ldots, x_{j-1} , x_{j+2}, \ldots, x_{2N})
& \text{if } \upwedgeat{j} \in \alpha \\
- \frac{\qnum{\alpha(j)+1}}{\qnum{\alpha(j)+2}} \times
 \sum_{\hat{\beta} \in \DP_{N-1}} M^{(N-1)}_{\hat{\alpha}, \hat{\beta} } \; \PartF_{ \hat{\beta} } (x_1 , \ldots, x_{j-1} , x_{j+2}, \ldots, x_{2N})
& \text{if } \downwedgeat{j} \in \alpha
\end{cases}
\end{split}
\end{align}
% % % % % % % \blue{for all $\alpha \in \DP_N$, $j \in \set{1, \ldots, 2N-1}$, and $\xi \in (x_{j-1}, x_{j+2})$,}
% \begin{align}
% & \sum_{\beta \in \DP_N} M^{(N)}_{\alpha, \beta}  \hat{\pi}_j( \Puregeomtwodim_\beta ) =
% \label{eq: coblo walk projection conditions 2}
% \begin{cases} 
% 0 & \text{if } \slopeat{j} \in \alpha \\
%  \sum_{\hat{\beta} \in \DP_{N-1}} M^{(N-1)}_{\hat{\alpha}, \hat{\beta} } \Puregeomtwodim_{ \hat{\beta} }
% & \text{if } \upwedgeat{j} \in \alpha \\
% - \frac{\qnum{\alpha(j)+1}}{\qnum{\alpha(j)+2}} \times
%  \sum_{\hat{\beta} \in \DP_{N-1}} M^{(N-1)}_{\hat{\alpha}, \hat{\beta} } \Puregeomtwodim_{ \hat{\beta} }
% & \text{if } \downwedgeat{j} \in \alpha
% \end{cases}
% \qquad\text{for all } j \in \set{1,\ldots,2N-1},
% \end{align}
where we denote $\alpha \removewedge{j} = \hat{\alpha}$.
Now recall the asymptotics properties~\eqref{eq: ASY for multiple SLEs} of pure partition functions,
and note that each $\hat{\beta} \in \DP_{N-1}$ determines a unique 
$\beta \in \DP_N$ with $\upwedgeat{j} \in \beta$ such that $\beta \removeupwedge{j} = \hat{\beta}$.
The left-hand side of~\eqref{eq: coblo walk projection conditions 2} then becomes
\begin{align*}
& \lim_{x_j , x_{j+1} \to \xi} \frac{1}{(x_{j+1} - x_j)^{-2h}}
    \sum_{\beta \in \DP_N} M^{(N)}_{\alpha, \beta} \; \PartF_\beta (x_1 , \ldots, x_{2N}) \\
= \; & \sum_{\hat{\beta} \in \DP_{N-1}} M^{(N)}_{\alpha, \beta} \; \PartF_{ \hat{\beta} } (x_1 , \ldots, x_{j-1} , x_{j+2}, \ldots, x_{2N}) .
\end{align*}
% \begin{align*}
% \sum_{\beta \in \DP_N} M^{(N)}_{\alpha, \beta}  \hat{\pi}_j( \Puregeomtwodim_\beta ) = \sum_{\hat{\beta} \in \DP_{N-1}} M^{(N)}_{\alpha, \beta} \Puregeomtwodim_{ \hat{\beta} },
% \end{align*}
%where $\upwedgeat{j} \in \beta$ and $\beta \removeupwedge{j} = \hat{\beta}$.
Since $\big( \PartF_{ \hat{\beta} } \big)_{\hat{\beta}  \in \DP_{N - 1}}$ is a basis,
% \blue{Since $\PartF_{ \hat{\beta} }$ for $\hat{\beta}  \in \DP_{N - 1}$ form a basis for $\Sol_{N-1}$},
the recursion~\eqref{eq: coblo walk projection conditions 2} is equivalent to
% the initial condition $M^{(1)}=1$ and
the following: %linear recurrence relations: 
% % % \blue{the recursion~\eqref{eq: equivalent recursion} %of Proposition~\ref{prop: recursion for matrix elements}
% % % with weight function $f(x) = \frac{\qnum{x}}{\qnum{x+1}}$. By Proposition~\ref{prop: recursion for matrix elements}, 
% % % this recursion with the initial condition $M^{(1)}=1$ has a unique solution: the weighted incidence matrix  $M=\genMmat$. 
% % % Proposition~\ref{prop: weighted KW incidence matrix inversion} then shows that this matrix is invertible with inverse $\genMinv$.
% % % The assertion follows.}
for any $j \in \set{1,\ldots,2N-1}$ and any $\beta \in \DP_N$ such that $\upwedgeat{j} \in \beta$, we have
\begin{align}\label{eq: recursion for the change of basis matrix alex}
M_{\alpha,\beta}^{(N)} = \begin{cases} 
   0 & \text{if } \slopeat{j} \in \alpha \\
   M_{\hat{\alpha},\hat{\beta}}^{(N-1)}
   & \text{if } \upwedgeat{j} \in \alpha \\
   -\frac{\qnum{\alpha(j)+1}}{\qnum{\alpha(j)+2}} \times
   M_{\hat{\alpha},\hat{\beta}}^{(N-1)} 
   & \text{if } \downwedgeat{j} \in \alpha,
\end{cases} 
\end{align}
where we denote by $\hat{\alpha} = \alpha \removewedge{j} \in \DP_{N-1}$ and 
$\hat{\beta} = \beta \removeupwedge{j} \in \DP_{N-1}$. Finally,
Proposition~\ref{prop: recursion for matrix elements} states that the
recursion~\eqref{eq: recursion for the change of basis matrix alex} holds if
and only if the matrices  $M^{(N)}$ are, for any $N$, 
the weighted incidence matrices of the parenthesis reversal relation, $M=\genMmat$. The rest follows since
the matrix $\genMmat$ is, for any $N$, invertible by Proposition~\ref{prop: weighted KW incidence matrix inversion}.
\end{proof}


\bigskip

\section{Direct construction of conformal block functions by a quantum group method}
\label{sec: construction of conformal block functions}
%The conformal block functions $\ConfBlockFun_\alpha$ can be constructed
%as explicit linear combinations of multiple $\SLE$ pure partition functions $\PartF_\beta$,
%as stated in Theorem~\ref{thm: change of basis theorem}.
In the preceding section, we expressed the conformal block function~$\ConfBlockFun_\alpha$
as linear combinations of multiple $\SLE$ pure partition functions $\PartF_\alpha$ and vice versa, 
generalizing Fomin's formula~\cite{Fomin-LERW_and_total_positivity, KKP-boundary_correlations_in_planar_LERW_and_UST}.
These expressions can also be viewed as a construction of the conformal block functions, which
however relies on an earlier construction of the multiple SLE pure partition
functions and detailed information about the solution
space~\cite{FK-solution_space_for_a_system_of_null_state_PDEs_1,
FK-solution_space_for_a_system_of_null_state_PDEs_2, FK-solution_space_for_a_system_of_null_state_PDEs_3,
KP-pure_partition_functions_of_multiple_SLEs}.
% % With the highly nontrivial inputs of the explicit construction \cite{KP-pure_partition_functions_of_multiple_SLEs}
% % and uniqueness \cite{FK-solution_space_for_a_system_of_null_state_PDEs_1, FK-solution_space_for_a_system_of_null_state_PDEs_2, FK-solution_space_for_a_system_of_null_state_PDEs_3} of the pure partition functions $\PartF_\beta$, this provides an explicit construction and uniqueness of the conformal block functions.
In this section, we provide an alternative, more direct construction of the
conformal block functions $\ConfBlockFun_\alpha$ based on a quantum group method developed
in~\cite{KP-conformally_covariant_boundary_correlation_functions_with_a_quantum_group}.
In analogy with the core underlying idea of
conformal blocks as discussed in Section~\ref{sec: conformal block functions background},
the present construction employs the Dyck path $\alpha$
as labeling a sequence of representations of the quantum group $\Uqsltwo$.
% % % % In this construction, we in particular recover the role of the Dyck path $\alpha$
% % % % as labeling a sequence of representations, in line with the core underlying idea of
% % % % conformal blocks as discussed in Section~\ref{sec: conformal block functions background} 
% % % % --- although it is representations of a hidden quantum group rather than of Virasoro algebra that
% % % % play a role in our construction. 
This quantum group construction furthermore sheds some light on why $q$-combinatorial
formulas for conformal blocks appear in the first place.
% \blue{[Eve: not very clear...]}

Generalizations of this construction for
conformal blocks in representations of the quantum group $\Uqsltwo$ and in
relation to CFT correlation functions are used and studied in~\cite{Flores-Peltola:WJTL_algebra, Flores-Peltola:Colored_braid_representations_and_QSW,
Flores-Peltola:Monodromy_invariant_correlation_function}.



\subsection{\label{sub: quantum group}The quantum group and its representations}

We begin by introducing the needed definitions and notation
about the quantum group $\Uqsltwo$ and its representations.
For more background, see, e.g.,~\cite{Kassel-Quantum_groups} and references therein.
Let $q = e^{\ii \pi 4 / \kappa}$ as before.
As a $\bC$-algebra, $\Uqsltwo$ is generated by the elements $K,K^{-1},E,F$ subject to 
the relations
\begin{align} \label{eq: quantum group relations}
\begin{split}
 KK^{-1}=&\; 1=K^{-1}K,\qquad KE=q^{2}EK,\qquad KF=q^{-2}FK,\\%\label{eq: quantum group relations}\\
 EF-FE=&\; \frac{1}{q-q^{-1}}\left(K-K^{-1}\right) . 
\end{split}
\end{align}
It has a Hopf algebra structure, with coproduct $\Delta \colon \Uqsltwo \rightarrow \Uqsltwo\tens\Uqsltwo$ given on its generators~by
\begin{align}\label{eq: coproduct} 
%\Delta \; \colon\; & \Uqsltwo \rightarrow \Uqsltwo\tens\Uqsltwo, 
%\\
\Hcp(E) = \; & E\tens K + 1\tens E,\qquad
\Hcp(K) = \; K \tens K,\qquad
\Hcp(F) = \; F \tens 1 + K^{-1} \tens F . %\nonumber
\end{align}
The coproduct is used to define the action of %a Hopf algebra 
the Hopf algebra $\Uqsltwo$
on tensor products of representations as follows. If the coproduct of an element $X \in \Uqsltwo$ reads
\[ \Hcp(X) = \sum_i X_i' \tens X_i'' , \]
and if $V'$ and $V''$ are two representations, then
$X$ acts on a tensor $v' \tens v'' \in V' \tens V''$ by the formula
\begin{align*}
X.(v' \tens v'') = \; & \sum_i X_i'.v' \tens X_i''.v'' .
% \,\in\, V' \tens V''
\end{align*}
%for any $v' \in V'$, $v'' \in V''$.
% % % % \alexmod{
% % % % By the coassociativity property 
% % % % $(\id\tens\Hcp)\circ\Hcp = (\Hcp\tens\id)\circ\Hcp$,
% % % % we need not specify the order in which the tensor products are formed
% % % % when studying tensor product representations of multiple tensor components.
% % % % For definiteness, we will study representations
% % % % $n$ tensor components using the $(n-1)$-fold coproduct
% % % % \begin{align*}
% % % % \Hcp^{(n)} = \; & (\Hcp\tens\id^{\tens(n-2)})\circ(\Hcp\tens\id^{\tens(n-3)})\circ\cdots
% % % % \circ(\Hcp\tens\id)\circ\Hcp ,
% % % % \qquad \Hcp^{(n)} \colon\Uqsltwo\rightarrow\Big(\Uqsltwo\Big)^{\tens n} .
% % % % \end{align*}
% % % % }
Tensor product representations with 
$n$ tensor components are defined using the $(n-1)$-fold coproduct
\begin{align*}
\Hcp^{(n)} = \; & (\Hcp\tens\id^{\tens(n-2)})\circ(\Hcp\tens\id^{\tens(n-3)})\circ\cdots
\circ(\Hcp\tens\id)\circ\Hcp ,
\qquad \Hcp^{(n)} \colon\Uqsltwo\rightarrow\Big(\Uqsltwo\Big)^{\tens n} .
\end{align*}
By the coassociativity property 
$(\id\tens\Hcp)\circ\Hcp = (\Hcp\tens\id)\circ\Hcp$ 
the tensor products of representations thus defined are associative, i.e.,
there is no need to specify the order in which the tensor products are formed.

%\subsubsection{\textbf{Irreducible representations of the quantum group}}
%Like the simple Lie algebra $\mathfrak{sl}_2(\bC)$,
For each $d \in \bN$, the quantum group
$\Uqsltwo$ has an irreducible representation $\Wd_d$ of dimension $d$,
obtained by suitably $q$-deforming the $d$-dimensional irreducible representation
of the simple Lie algebra $\mathfrak{sl}_2$. Of primary importance to
us is the two-dimensional irreducible representation $\Wd_2$: it has a basis $\set{\Wbas_0 , \Wbas_1}$
on which the generators act by
\begin{align*} 
K. \Wbas_0 = q \, \Wbas_0 , \qquad
K. \Wbas_1 = q^{-1} \, \Wbas_1 , \qquad
E. \Wbas_0 = 0 , \qquad
E. \Wbas_1 = \Wbas_0 , \qquad
F. \Wbas_0 = \Wbas_1 , \qquad
F. \Wbas_1 = 0 .
\end{align*}
A similar explicit definition of the $d$-dimensional irreducible
$\Wd_d$ can be found in, e.g.,~\cite{KP-conformally_covariant_boundary_correlation_functions_with_a_quantum_group}.
The tensor product of two two-dimensional irreducibles
decomposes as a direct sum of subrepresentations,
\[ \Wd_2 \tens \Wd_2 \isom \Wd_1 \oplus \Wd_3 , \]
where $\Wd_1$ is a one-dimensional
%{\color{blue} irreducible }
subrepresentation spanned by the vector
\begin{align} \label{eq: singlet basis vector}
\Sbas %= \Tbas_{0}^{(1;2,2)}
= \frac{1}{q-q^{-1}}\left(\Wbas_{1}\tens\Wbas_{0}-q\,\Wbas_{0}\tens\Wbas_{1}\right),
\end{align}
and $\Wd_3$ is a three-dimensional irreducible subrepresentation with basis
\begin{align*}%\label{eq: triplet basis vectors}
\TRbas_+ %= \Tbas_{0}^{(3;2,2)}
= \Wbas_{0}\tens\Wbas_{0},\qquad 
\TRbas_0 %= \Tbas_{1}^{(3;2,2)} 
= q^{-1}\,\Wbas_{0}\tens\Wbas_{1} + \Wbas_{1}\tens\Wbas_{0},\qquad 
\TRbas_- %= \Tbas_{2}^{(3;2,2)}
= [2]\,\Wbas_{1}\tens\Wbas_{1} .
\end{align*}
% In the decomposition $\Wd_{2} \tens \Wd_{2} \isom \Wd_{3}\oplus\Wd_{1}$,
% we denote the projection to the singlet subspace by
We denote the projection onto the 
one-dimensional
subrepresentation
$\Wd_1 \subset \Wd_2 \tens \Wd_2$ %, the singlet subspace,
by
\begin{align*}
\pi\colon\; &\Wd_{2}\tens\Wd_{2}\to\Wd_{2}\tens\Wd_{2} , \qquad 
\pi(\Sbas)=\Sbas \qquad \text{ and } \qquad
%\pi(\Tbas^{(3;2,2)}_l) = 0  \text{ for }  l = 0,1,2 . 
\pi(\TRbas_{+}) = \pi(\TRbas_{0}) = \pi(\TRbas_{-}) = 0 .
%\pi(\Sbas)=\Sbas,\qquad\pi(\TRbas_{+,0,-}) = 0 .
\end{align*}
The one-dimensional representation $\Wd_1$ is trivial in the sense that
it is the neutral element for tensor products of representations: for any representation $V$,
we have $\Wd_1 \tens V \isom V \isom V \tens \Wd_1$, and
$\Wd_1$ can thus simply be identified with 
the scalars $\bC$.
Using the identification $\Sbas \mapsto 1 \in \bC$, we denote the
projection from $\Wd_2 \tens \Wd_2$ to $\Wd_1 \isom \bC$ by
\begin{align}\label{eq: projection to singlet}
\hat{\pi}\colon\; &\Wd_{2}\tens\Wd_{2}\to \bC , \qquad
\hat{\pi}(\Sbas) = 1 \qquad \text{ and } \qquad
%\hat{\pi}(\Tbas^{(3;2,2)}_l) = 0  \text{ for }  l = 0,1,2 , 
\hat{\pi}(\TRbas_{+}) = \hat{\pi}(\TRbas_{0}) = \hat{\pi}(\TRbas_{-}) = 0 .
%\pi(\Sbas)=\Sbas,\qquad\pi(\TRbas_{+,0,-}) = 0 .
\end{align}

More generally, we have the $q$-Clebsch-Gordan formula
% \blue{``$q$-Clebsch-Gordan formula''}
\begin{align}\label{eq: decomposition of tensor product}
\Wd_{d_{2}}\tens\Wd_{d_{1}}\isom\; &
    \Wd_{d_{1}+d_{2}-1}\oplus\Wd_{d_{1}+d_{2}-3}\oplus\cdots\oplus\Wd_{|d_{1}-d_{2}|+3}\oplus\Wd_{|d_{1}-d_{2}|+1} 
\end{align}
for the direct sum decomposition of the tensor product of 
the irreducible representations of dimensions $d_1$ and $d_2$,
see, e.g.~\cite[Lemma~2.4]{KP-conformally_covariant_boundary_correlation_functions_with_a_quantum_group}.
% Applied recursively, this
% formula can in fact be viewed as a definition of the irreducible representation $\Wd_d$.
Repeated application of the 
decomposition~\eqref{eq: decomposition of tensor product} gives
%the direct sum decomposition % to irreducibles
\begin{align}\label{eq: tensor power of two dimensionals}
\Wd_2^{\tens n} \; \isom \; \bigoplus_d m_d^{(n)} \, \Wd_d,
\end{align}
where the irreducible $\Wd_d$ of dimension $d$ appears with multiplicity
$m_d^{(n)}$, see, e.g.~\cite[Lemma~2.2]{KP-pure_partition_functions_of_multiple_SLEs}.
When $n = 2N$,
the trivial subrepresentation 
\begin{align} \label{eq: trivial submodule}
%\HWsp_{2N}^{(0)} := \set{v \in \Wd_{2}^{\tens 2N}\;\Big|\;E.v=0 ,\;K.v=v} %= m_1^{(2N)} \, \Wd_1 . %= \Catalan_N \times \Wd_1 .
% \HWspCal_N
\HWsp_{2N}^{(0)}
:= \set{v \in \Wd_{2}^{\tens 2N}\;\Big|\;E.v=0 ,\;K.v=v}
\end{align}
coincides with the sum of all %one-dimensional subrepresentations
copies of $\Wd_1$, and has dimension equal to a Catalan number
% \[ \dmn \HWsp_{2N}^{(0)} }
% = m_1^{(2N)} = \Catalan_N . \]
\[ \dmn \HWsp_{2N}^{(0)} = m_1^{(2N)} = \Catalan_N . \]

Finally,
in the tensor product $\Wd_2^{\tens n}$, we denote by $\pi_j$ and $\hat{\pi}_j$
the projections $\pi$ and $\hat{\pi}$ 
acting on the $j$:th and $(j+1)$:st tensor components counting from the right, i.e.,
\begin{align*}
\pi_j := \; & \id^{\tens (n-1-j)} \tens \pi \tens \id^{\tens (j-1)} \colon\; \Wd_{2}^{\tens n} \to \; \Wd_{2}^{\tens n} \\
\hat{\pi}_j := \; & \id^{\tens (n-1-j)} \tens \hat{\pi} \tens \id^{\tens (j-1)} \colon\; \Wd_{2}^{\tens n} \to \; \Wd_{2}^{\tens (n-2)} .
\end{align*}


\subsection{Constructing conformal blocks}
The purpose of this section is to give a construction of the conformal block functions.
% \blue{[text removed]}
Our construction relies on the method 
introduced in~\cite{KP-conformally_covariant_boundary_correlation_functions_with_a_quantum_group},
called ``spin chain~-~Coulomb gas correspondence'', %introduced in~\cite{KP-conformally_covariant_boundary_correlation_functions_with_a_quantum_group},
which is allows to solve conformal field theory PDEs with given boundary conditions
by quantum group calculations. We %will 
use the correspondence in the following form, which combines a special case of a more general theorem
in~\cite{KP-conformally_covariant_boundary_correlation_functions_with_a_quantum_group}
with additional information available in that special case~\cite{KP-pure_partition_functions_of_multiple_SLEs}.
\begin{prop} 
%[\cite{KP-conformally_covariant_boundary_correlation_functions_with_a_quantum_group,KP-pure_partition_functions_of_multiple_SLEs}]
\label{prop: SCCG correspondence map}
Let $\kappa \in (0,8) \setminus \bQ$ and $q = e^{\ii \pi 4 / \kappa}$.
There exist explicit linear isomorphisms
\[ \sF \colon \HWsp_{2N}^{(0)} \to \Sol_N ,\] 
% \[ \sF \colon 
% \red{
% \HWspCal_N
% }
% \to \Sol_N ,\] 
for all $N \in \bZnn$, with the following property.
Let $v \in \HWsp_{2N}^{(0)}$, 
% $v \in 
% \red{
% \HWspCal_N
% }
% $
and $j \in \set{1,2,\ldots,2N-1}$, and denote
$\hat{v} = \hat{\pi}_{j}(v) \in \HWsp_{2(N-1)}^{(0)}$.
% $\hat{v} = \hat{\pi}_{j}(v) \in 
% \red{
% \HWspCal_{N-1}
% }
% $.
Then, for any  $\xi \in (x_{j-1},x_{j+2})$, the function $\sF[v] \colon \chamber_{2N} \to \bC$ has the asymptotics 
\begin{align*}
\lim_{x_{j},x_{j+1}\to\xi}
\frac{\sF[v](x_{1},\ldots,x_{2N})}{(x_{j+1}-x_{j})^{-2 h}}
\; = \; \; & 
%\frac{\Gamma(1-4/\kappa)^2}{\Gamma(2-8/\kappa)} \times 
B \times 
\sF[\hat{v}](x_{1},\ldots,x_{j-1},x_{j+2},\ldots,x_{2N}) ,
\end{align*}
where $B = \frac{\Gamma(1-4/\kappa)^2}{\Gamma(2-8/\kappa)}$.
\end{prop}
\begin{proof}
Such a map $\sF$
was constructed in~\cite{KP-conformally_covariant_boundary_correlation_functions_with_a_quantum_group} 
and it follows from the explicit expressions of the functions $\sF[v]$ as Coulomb gas integrals
that $\sF[v] \in \Sol_N$ for all $v \in \HWsp_{2N}^{(0)}$.
% $v \in 
% \red{
% \HWspCal_N
% }
% $.
That $\sF$ is injective is proven in~\cite{KP-pure_partition_functions_of_multiple_SLEs},
and comparison of dimensions then shows that $\sF$ is a linear isomorphism.
% % % % follows from 
% % % % Proposition~\ref{prop: solution space with power law bound},
% % % % Lemma~\ref{thm: pure partition vectors for generic kappa},
% % % % and Equation~\eqref{eq: pure partition functions from SCCG}:
% % % % the basis vectors $\Puregeomtwodim_\alpha$ are mapped to non-zero multiples of the basis functions 
% % % % $\PartF_\alpha$ under the map $\sF$.
Finally, the asymptotics property follows immediately 
from~\cite[Theorem~4.17(ASY)]{KP-conformally_covariant_boundary_correlation_functions_with_a_quantum_group}.
\end{proof}


% \subsection{\label{sub: vectors in quantum group representations}The collections of vectors and change of basis matrix} 
% 
% In~\cite{KP-pure_partition_functions_of_multiple_SLEs}, 
% the pure partition functions $\PartF_\alpha^{(\kappa)}$
% {\color{green} , $\alpha \in \DP$, }
% {\color{blue} (remove) }
% were constructed from the following vectors $\Puregeomtwodim_\alpha$
% in the trivial subrepresentations~\eqref{eq: trivial submodule} 
% of tensor products of two-dimensional irreducibles. Their defining projection
% properties~\eqref{eq: multiple sle projection conditions} below correspond exactly
% to the asymptotic boundary conditions~\eqref{eq: ASY for multiple SLEs} 
% {\color{green} of }
% {\color{blue} for the }
% pure partition functions
% {\color{blue} $\PartF_\alpha^{(\kappa)}$ } .
% 
% {\color{red} I changed this to a lemma:}
% %\begin{thm}{\cite[Theorem~3.1]{KP-pure_partition_functions_of_multiple_SLEs}}
% \begin{lem}{\cite[Theorem~3.1]{KP-pure_partition_functions_of_multiple_SLEs}}
% \label{thm: pure partition vectors for generic kappa}
% There exists a unique collection $(\Puregeomtwodim_\alpha)_{\alpha \in \DP}$
% of vectors, $\Puregeomtwodim_\alpha \in \HWsp_{2N}^{(0)}$ for $\alpha \in \DP_N$,
% such that $\Puregeomtwodim_\emptywalk = 1$ and 
% the following projection properties hold:
% \begin{align}
% %& K.\Puregeomtwodim_\alpha = \Puregeomtwodim_\alpha
% %\label{eq: multiple sle cartan eigenvalue}\\
% %& E.\Puregeomtwodim_\alpha = 0
% %\label{eq: multiple sle highest weight vector}\\
% & \hat{\pi}_j(\Puregeomtwodim_\alpha) =
% \label{eq: multiple sle projection conditions}\begin{cases}
%     0 & \mbox{if } \upwedgeat{j} \notin\alpha \\
%     \Puregeomtwodim_{\alpha \removeupwedge{j}} & \mbox{if } \upwedgeat{j}\in\alpha
% \end{cases}
% \qquad\text{for all } j \in \set{1,\ldots,2N-1}.
% \end{align}
% Moreover, for fixed $N$, the collection $(\Puregeomtwodim_\alpha )_{\alpha \in \DP_N}$ is a
% basis of $\HWsp_{2N}^{(0)}$.
% %\end{thm}
% \end{lem}
% 
% The conformal block functions $\ConfBlockFun_\alpha^{(\kappa)}$
% {\color{green} , $\alpha \in \DP$, }
% {\color{blue}  (remove) }
% can 
% {\color{green} similarly be }
% {\color{blue} be similarly}
% constructed from suitable vectors $\Coblobastwodim_\alpha$
% {\color{green}in the trivial subrepresentations~\eqref{eq: trivial submodule}.}
% {\color{blue}  (remove, maybe save one line!) }
% Their defining projection
% properties~\eqref{eq: multiple sle projection conditions} correspond exactly
% to the asymptotic boundary conditions~\eqref{eq: ASY for multiple SLEs} of
% pure partition functions.

% {\color{magenta}
% Alex's suggestion: The following combines Thm\ref{thm: conformal block vectors}, thm~\ref{thm: change of basis between conformal blocks and pure partition functions}, Lemma~\ref{lem: recursion of change of basis matrix with generic q}, and allows in addition to skip the uniqueness part (Sec. 6.3.1) in the representation-theoretic derivation of the defining conditions~\eqref{eq: coblo walk projection conditions}. I left the old versions below.
% 
% \begin{thm}\label{thm: ...}
% There exists a unique collection $(\Coblobastwodim_\alpha)_{\alpha \in \DP}$
% of vectors $\Coblobastwodim_\alpha \in \HWsp_{2N}^{(0)}$ for $\alpha \in \DP_N$,
% such that $\Coblobastwodim_\emptywalk = 1$ and 
% the following projection properties hold:
% \begin{align}
% & \hat{\pi}_j(\Coblobastwodim_\alpha) =
% \label{eq: coblo walk projection conditions alex}
% \begin{cases} 
% 0 & \text{if } \slopeat{j} \in \alpha \\
% \Coblobastwodim_{\alpha\removeupwedge{j}} 
% & \text{if } \upwedgeat{j} \in \alpha \\
% - \frac{\qnum{\alpha(j)+1}}{\qnum{\alpha(j)+2}} \times
% \Coblobastwodim_{\alpha \removedownwedge{j}} 
% & \text{if } \downwedgeat{j} \in \alpha
% \end{cases}
% \qquad\text{for all } j \in \set{1,\ldots,2N-1}.
% \end{align}
% For any $\alpha \in \DP_N$, the vector $\Coblobastwodim_\alpha$ of this collection, is given in the basis of Lemma~\ref{thm: pure partition vectors for generic kappa} by
% \begin{align*}
% \Coblobastwodim_\alpha =
% \sum_{\beta \in \DP_N} \genMmat_{\alpha,\beta}\;\Puregeomtwodim_\beta,
% \end{align*}
% where $\genMmat$ is 
% as in~Theorem~\ref{thm: conformal blocks and pure partition functions}.
% Moreover, for fixed $N$, the collection 
% $(\Coblobastwodim_\alpha)_{\alpha \in \DP_N}$ is a
% basis of $\HWsp_{2N}^{(0)}$, and
% \begin{align*}
% \Puregeomtwodim_\alpha =
% \sum_{\beta \in \DP_N} \genMinv_{\alpha,\beta}\;\Coblobastwodim_\beta ,
% \end{align*}
% where $\genMinv$ is the inverse matrix of $\genMmat$,
% as in~Theorem~\ref{thm: conformal blocks and pure partition functions}.
% \end{thm}
% 
% \begin{proof}
% Let $(\Coblobastwodim_\alpha)_{\alpha \in \DP}$ be any collection 
% of vectors such that  $\Coblobastwodim_\alpha \in \HWsp_{2N}^{(0)}$ for $\alpha \in \DP_N$, and let $M^{(N)} \in \C^{\DP_N \times \DP_N}$ be the unique matrices representing $\Coblobastwodim_\alpha \in \HWsp_{2N}^{(0)}$ in the basis $(\Puregeomtwodim_\beta)_{\beta \in \DP_N}$:
% \begin{align*}
% \Coblobastwodim_\alpha = \sum_{\beta \in \DP_N} M^{(N)}_{\alpha, \beta} \Puregeomtwodim_\beta
% \end{align*}
% for any $\alpha \in \DP_N$ and any $N$.
% The recurrence relations~\eqref{eq: coblo walk projection conditions alex} then equivalently require that the matrices $M^{(N)}_{\alpha, \beta}$ satisfy the initial condition $M^{(1)}=1$ and
% \begin{align}
% & \sum_{\beta \in \DP_N} M^{(N)}_{\alpha, \beta}  \hat{\pi}_j( \Puregeomtwodim_\beta ) =
% \label{eq: coblo walk projection conditions 2}
% \begin{cases} 
% 0 & \text{if } \slopeat{j} \in \alpha \\
%  \sum_{\hat{\beta} \in \DP_{N-1}} M^{(N-1)}_{\hat{\alpha}, \hat{\beta} } \Puregeomtwodim_{ \hat{\beta} }
% & \text{if } \upwedgeat{j} \in \alpha \\
% - \frac{\qnum{\alpha(j)+1}}{\qnum{\alpha(j)+2}} \times
%  \sum_{\hat{\beta} \in \DP_{N-1}} M^{(N-1)}_{\hat{\alpha}, \hat{\beta} } \Puregeomtwodim_{ \hat{\beta} }
% & \text{if } \downwedgeat{j} \in \alpha
% \end{cases}
% \qquad\text{for all } j \in \set{1,\ldots,2N-1},
% \end{align}
% where we denoted $\alpha \removewedge{j} = \hat{\alpha}$. Now, recall the projection properties~\eqref{eq: multiple sle projection conditions}, and recall that each $\hat{\beta} \in \DP_{N-1}$ determines a unique 
% such $\beta \in \DP_N$ with $\upwedgeat{j} \in \beta$, related by $\beta \removeupwedge{j} = \hat{\beta}$. The left-hand side of~\eqref{eq: coblo walk projection conditions 2} then becomes 
% \begin{align*}
% \sum_{\beta \in \DP_N} M^{(N)}_{\alpha, \beta}  \hat{\pi}_j( \Puregeomtwodim_\beta ) = \sum_{\hat{\beta} \in \DP_{N-1}} M^{(N)}_{\alpha, \beta} \Puregeomtwodim_{ \hat{\beta} },
% \end{align*}
% where $\upwedgeat{j} \in \beta$ and $\beta \removeupwedge{j} = \hat{\beta}$. Since $(\Puregeomtwodim_{ \hat{\beta} })_{\hat{\beta}  \in \DP_{N - 1}}$ is a basis,~\eqref{eq: coblo walk projection conditions 2} is hence equivalent to the initial condition $M^{(1)}=1$ and the following linear recurrence relations: 
% for any $j \in \set{1,\ldots,2N-1}$ and any $\beta \in \DP_N$ such that 
% $\upwedgeat{j} \in \beta$, we have
% \begin{align}\label{eq: recursion for the change of basis matrix alex}
% M_{\alpha,\beta}^{(N)} = \begin{cases} 
% 0 & \text{if } \slopeat{j} \in \alpha \\
% M_{\hat{\alpha},\hat{\beta}}^{(N-1)}
% & \text{if } \upwedgeat{j} \in \alpha \\
% -\frac{\qnum{\alpha(j)+1}}{\qnum{\alpha(j)+2}} \times
% M_{\hat{\alpha},\hat{\beta}}^{(N-1)} 
% & \text{if } \downwedgeat{j} \in \alpha,
% \end{cases} 
% \end{align}
% where we denote by $\hat{\alpha} = \alpha \removewedge{j} \in \DP_{N-1}$ and 
% $\hat{\beta} = \beta \removeupwedge{j} \in \DP_{N-1}$. Finally, Corollary~\ref{cor: recursion for matrix elements} states that the recursion~\eqref{eq: recursion for the change of basis matrix alex} holds if and only if the matrices  $M^{(N)}$ are, for any $N$, the weighted incidence matrix of the parenthesis reversal relation, $M=\genMmat$. The rest follows since the matrix $\genMmat$ is, for any $N$, invertible by Theorem~\ref{thm: weighted KW incidence matrix inversion}.
% \end{proof}
% }

With the help of the correspondence $\sF$ of Proposition~\ref{prop: SCCG correspondence map},
the task of constructing the conformal block functions is reduced to the task of
constructing suitable vectors in the trivial subrepresentation $\HWsp_{2N}^{(0)}$ %$\red{\HWspCal_N}$
of a tensor product $\Wd_2^{\tens 2N}$ of
two-dimensional irreducible representations of the quantum group. This is achieved in the following 
proposition, which we prove in the end of this section.

\begin{prop}\label{prop: conformal block vectors}
There exists a unique collection of vectors $( \Coblobastwodim_\alpha )_{\alpha \in \DP}$, 
with $\Coblobastwodim_\alpha \in \HWsp_{2N}^{(0)}$ 
% $\Coblobastwodim_\alpha \in 
% \red{
% \HWspCal_N
% }
% $ 
when $\alpha \in \DP_N$,
such that $\Coblobastwodim_\emptywalk = 1$ and the following projection properties hold:
\begin{align}
% % % % & K.\Coblobastwodim_\alpha = \Coblobastwodim_\alpha
% % % % \label{eq: coblo walk cartan eigenvalue}\\
% % % % & E.\Coblobastwodim_\alpha = 0
% % % % \label{eq: coblo walk highest weight vector}\\
& \hat{\pi}_j(\Coblobastwodim_\alpha) =
\label{eq: coblo walk projection conditions}
\begin{cases} 
0 & \text{if } \slopeat{j} \in \alpha \\
\Coblobastwodim_{\alpha\removeupwedge{j}} 
& \text{if } \upwedgeat{j} \in \alpha \\
- \frac{\qnum{\alpha(j)+1}}{\qnum{\alpha(j)+2}} \times
\Coblobastwodim_{\alpha \removedownwedge{j}} 
& \text{if } \downwedgeat{j} \in \alpha
\end{cases}
%\begin{cases} \Coblobastwodim_{\alpha \removewedge{j}} & \mbox{if } \alpha \text{ contains an up-wedge }\upwedgeat{j} \\
%- \frac{\qnum{\alpha(j)+1}}{\qnum{\alpha(j)+2}} \times
%\Coblobastwodim_{\alpha \removewedge{j}} & \mbox{if } \alpha \text{ contains a down-wedge }\downwedgeat{j} \\
% 0 & \mbox{otherwise} \end{cases}
\qquad\text{for all } j \in \set{1,\ldots,2N-1}.
\end{align}
Moreover, for any $N \in \bZnn$, the collection $( \Coblobastwodim_\alpha )_{\alpha \in \DP_N}$ is a basis of $\HWsp_{2N}^{(0)}$.
% $\red{\HWspCal_N}$.
\end{prop}


% %\begin{proof}
% The proof of this theorem is given in Section~\ref{sss: construction of coblovec}.
% %\end{proof}
% 
% 
% The relation between multiple $\SLE$ pure partition functions and conformal block functions,
% {\color{blue} stated in }
% Theorem~\ref{thm: conformal blocks and pure partition functions},
% will be a consequence of the following analogous result on the above two
% collections of vectors.
% 
% \begin{thm}\label{thm: change of basis between conformal blocks and pure partition functions}
% We have 
% \begin{align*}
% \Coblobastwodim_\alpha =
% \sum_{\beta \in \DP_N} \genMmat_{\alpha,\beta}\;\Puregeomtwodim_\beta,
% % % % % \qquad \text{ or, equivalently, } \qquad
% % % % % \ConfBlockFun_\alpha =
% % % % % \sum_{\beta \in \DP_N} M_{\alpha,\beta}\;\PartF_\beta,
% \end{align*}
% where $\genMmat$ is the weighted incidence matrix
%~\eqref{eq: def of weighted incidence matrix} 
% with the weight $w(t) := \frac{\qnum{h_t}}{\qnum{h_t+1}}$
% assigned to any Dyck tile $t$, 
% as in~Theorem~\ref{thm: conformal blocks and pure partition functions}.
% Conversely, we have
% \begin{align*}
% \Puregeomtwodim_\alpha =
% \sum_{\beta \in \DP_N} \genMinv_{\alpha,\beta}\;\Coblobastwodim_\beta ,
% % % % % \qquad \text{ or, equivalently, } \qquad
% % % % % \PartF_\beta = \sum_{\alpha \in \DP_N} M^{-1}_{\alpha,\beta}\;\ConfBlockFun_\alpha,
% \end{align*}
% where $\genMinv$ is the inverse matrix of $\genMmat$,
% as in~Theorem~\ref{thm: conformal blocks and pure partition functions}.
% %\begin{align*}
% %M^{-1}_{\alpha, \beta} = \begin{cases}
% %\sum_{T \in \CItilingsof (\alpha/\beta)} \prod_{t \in T} w(t) & \text{if }\alpha \DPleq \beta \\
% %0 & \text{otherwise},
% %\end{cases}
% %\end{align*}
% %where $\CItilingsof (\alpha/\beta)$ denotes the family of all cover-inclusive Dyck tilings of the skew Young diagram $\alpha/\beta$.
% \end{thm}
% %The proof of this theorem is given after the next lemma,
% %which is the crucial observation for the proof.
% %The proof is given in the next subsubsection.
% 
% {\color{blue} This theorem will be proved in the end this section, 
% after establishing some auxiliary results. }
% 
% %\subsubsection{\label{sss: proof of change of basis}\textbf{Proof of Theorem~\ref{thm: change of basis between conformal blocks and pure partition functions}}}
% 
% {\color{red} I think the subsubsection 
% "Proof of Theorem~\ref{thm: change of basis between conformal blocks and pure partition functions}"
% is not necessary. I removed it.}
% 
% Let $(\Puregeomtwodim_\alpha )_{\alpha \in \DP}$ be the collection
% of vectors given by 
% {\color{green} Theorem~\ref{thm: pure partition vectors for generic kappa}, }
% {\color{blue} Lemma~\ref{thm: pure partition vectors for generic kappa}, }
% and let $(\Coblobastwodim_\alpha)_{\alpha \in \DP}$ be any collection of vectors
% $\Coblobastwodim_\alpha \in \HWsp_{2N}^{(0)}$, 
% {\color{blue} for }
% $\alpha \in \DP_N$,
% satisfying the projection properties~\eqref{eq: coblo walk projection conditions}
% --- the construction and uniqueness of the latter collection 
% (Theorem~\ref{thm: conformal block vectors})
% is the topic of Section~\ref{sss: construction of coblovec}.
% 
% We first note that, 
% because $(\Puregeomtwodim_\alpha )_{\alpha \in \DP_N}$ 
% is a basis of $\HWsp_{2N}^{(0)}$, 
% there exists a matrix $M = (M_{\alpha,\beta})$ such that
% %we can write 
% \begin{align*}
% \Coblobastwodim_\alpha =
% \sum_{\beta \in \DP_N} M_{\alpha,\beta}\;\Puregeomtwodim_\beta.
% \end{align*}
% %for some matrix $M = (M_{\alpha,\beta})$. 
% The projection properties
% of the vectors $\Coblobastwodim_\alpha$ and $\Puregeomtwodim_\beta$
% show that the matrix entries $M_{\alpha,\beta} = M_{\alpha,\beta}^{(N)}$
% %are in fact determined recursively. This will allow us to explicitly solve 
% %for $M_{\alpha,\beta}$,
% can be 
% {\color{green} in fact }
% {\color{blue} (remove)}
% solved recursively, using the Cascade Recursion 
% (Corollary~\ref{cor: recursion for matrix elements}) from 
% {\color{green} Section~\ref{sec: combinatorics} }
% {\color{blue} Section~\ref{subsec: cascades}} .
% % % % With the following auxiliary result,
% % % % the proof of Theorem~\ref{thm: change of basis between conformal blocks and pure partition functions}
% % % % will become an easy consequence of the Cascade Recursion in Section~\ref{sec: combinatorics}.
% \begin{lem}\label{lem: recursion of change of basis matrix with generic q}
% Let $\alpha \in \DP_N$. The matrix entries $M_{\alpha,\beta}^{(N)}$
% satisfy $M^{(1)} = 1$ and the following linear recurrence relations: 
% for any $j \in \set{1,\ldots,2N-1}$ and any $\beta \in \DP_N$ such that 
% $\upwedgeat{j} \in \beta$, we have
% \begin{align}\label{eq: recursion for the change of basis matrix}
% M_{\alpha,\beta}^{(N)} = \begin{cases} 
% 0 & \text{if } \slopeat{j} \in \alpha \\
% M_{\hat{\alpha},\hat{\beta}}^{(N-1)}
% & \text{if } \upwedgeat{j} \in \alpha \\
% -\frac{\qnum{\alpha(j)+1}}{\qnum{\alpha(j)+2}} \times
% M_{\hat{\alpha},\hat{\beta}}^{(N-1)} 
% & \text{if } \downwedgeat{j} \in \alpha,
% \end{cases} 
% \end{align}
% where we denote by $\hat{\alpha} = \alpha \removewedge{j} \in \DP_{N-1}$ and 
% $\hat{\beta} = \beta \removeupwedge{j} \in \DP_{N-1}$.
% \end{lem}
% \begin{proof}
% We apply the projection 
% {\color{green} properties~\eqref{eq: coblo walk projection conditions} and~\eqref{eq: multiple sle projection conditions} }
% {\color{blue} 
% properties~\eqref{eq: multiple sle projection conditions}
% and~\eqref{eq: coblo walk projection conditions} 
% }
% of the vectors 
% {\color{green} $\Coblobastwodim_\alpha$ and $\Puregeomtwodim_\alpha$, }
% {\color{blue} $\Puregeomtwodim_\alpha$ and $\Coblobastwodim_\alpha$, }
% respectively.
% First, 
% {\color{green} by the properties~\eqref{eq: multiple sle projection conditions}, we have }
% {\color{blue}~\eqref{eq: multiple sle projection conditions} shows that }
% \begin{align*}
% \hat{\pi}_j(\Puregeomtwodim_\beta) =
% \begin{cases} 0 & \mbox{if } \upwedgeat{j} \notin \beta \\ \Puregeomtwodim_{\hat{\beta}} & \mbox{if } \upwedgeat{j} \in \beta 
% \end{cases},
% \end{align*}
% for any $j \in \set{1,\ldots,2N-1}$, where $\hat{\beta} = \beta \removeupwedge{j}$.
% Now fix $j$. 
% {\color{green} Then each $\hat{\beta} \in \DP_{N-1}$ determines a unique 
% such $\beta \in \DP_N$ with $\upwedgeat{j} \in \beta$. 
% %such that $\hat{\beta} = \beta \removeupwedge{j}$.
% %Therefore, for fixed $j$, we get
% Therefore in the following, it is natural to re-index the sum by $\hat{\beta} \in \DP_{N-1}$, }
% {\color{blue} Then, each $\hat{\beta} \in \DP_{N-1}$ determines a unique $\beta \in \DP_N$ such that 
% $\upwedgeat{j} \in \beta$ and $\hat{\beta} = \beta \removeupwedge{j}$, and we can write }
% \begin{align*}
% \hat{\pi}_j(\Coblobastwodim_\alpha)
% \;=\;& \sum_{\beta\in\DP_N}M_{\alpha,\beta}^{(N)} \;\hat{\pi}_j(\Puregeomtwodim_\beta)
% = \sum_{\hat{\beta}\in\DP_{N-1}}M_{\alpha,\beta}^{(N)}\;\hat{\pi}_j(\Puregeomtwodim_{\beta})
% = \sum_{\hat{\beta}\in\DP_{N-1}}M_{\alpha,\beta}^{(N)}\;\Puregeomtwodim_{\hat{\beta}}.
% \end{align*}
% Suppose now that $\alpha$ has a slope 
% {\color{green} $\slopeat{j}$ at $j$ }
% {\color{blue} at $j$, i.e., $\slopeat{j} \in \alpha$ }.
% Then, by the properties~\eqref{eq: coblo walk projection conditions},
% we have $\hat{\pi}_j(\Coblobastwodim_\alpha)=0$, so
% \begin{align*}
% 0 = \hat{\pi}_j(\Coblobastwodim_\alpha)=
% \sum_{\hat{\beta}\in\DP_{N-1}} M_{\alpha,\beta}^{(N)}\;\Puregeomtwodim_{\hat{\beta}},
% \end{align*}
% and because the vectors $\Puregeomtwodim_{\hat{\beta}}$ in the sum
% are linearly independent
% {\color{green} (by Theorem~\ref{thm: pure partition vectors for generic kappa}), }
% {\color{blue} (by Lemma~\ref{thm: pure partition vectors for generic kappa}), }
% we get the asserted equation $M_{\alpha,\beta}^{(N)} = 0$, for all 
% $\beta \in \DP_N$ 
% {\color{green} containing the link $\upwedgeat{j}$ }
% {\color{blue} such that $\upwedgeat{j} \in \beta$ } .
% 
% Suppose then that $\alpha$ contains an up-wedge 
% {\color{green} $\upwedgeat{j}$ at $j$ }
% {\color{blue} at $j$, i.e., $\upwedgeat{j} \in \alpha$ }.
% Then, by~\eqref{eq: coblo walk projection conditions}, we have 
% $\hat{\pi}_j(\Coblobastwodim_\alpha) = \Coblobastwodim_{\hat{\alpha}}$, so
% \begin{align*}
% \sum_{\hat{\beta}\in\DP_{N-1}} M_{\hat{\alpha},\hat{\beta}}^{(N-1)}\;
% \Puregeomtwodim_{\hat{\beta}} =
% \Coblobastwodim_{\hat{\alpha}}
%  \;=\;&\hat{\pi}_j(\Coblobastwodim_\alpha)=
% \sum_{\hat{\beta}\in\DP_{N-1}}M_{\alpha,\beta}^{(N)}
% \;\Puregeomtwodim_{\hat{\beta}}.
% \end{align*}
% By the linear independence of $\Puregeomtwodim_{\hat{\beta}}$, 
% {\color{green} the desired equation }
% {\color{blue} the asserted equation~\eqref{eq: recursion for the change of basis matrix} }
% again follows: we have 
% $M_{\hat{\alpha},\hat{\beta}}^{(N-1)} = M_{\alpha,\beta}^{(N)}$,
% for all $\beta \in \DP_N$ 
% {\color{green} containing the link $\upwedgeat{j}$ }
% {\color{blue} such that $\upwedgeat{j} \in \beta$ } .
% 
% Finally, suppose that $\alpha$ contains a down-wedge 
% {\color{green} $\downwedgeat{j}$ at $j$. We }
% {\color{blue} at $j$, i.e., $\downwedgeat{j} \in \alpha$. }
% {\color{blue} By~\eqref{eq: coblo walk projection conditions}, we}
% then have 
% $\hat{\pi}_j(\Coblobastwodim_\alpha) = 
% -\frac{\qnum{\alpha(j)+1}}{\qnum{\alpha(j)+2}} 
% \times \Coblobastwodim_{\hat{\alpha}}$,
% {\color{green} by~\eqref{eq: coblo walk projection conditions}, }
% and similarly as above, we get
% {\color{blue} the asserted equation }
% $M_{\alpha,\beta}^{(N)} = -\frac{\qnum{\alpha(j)+1}}{\qnum{\alpha(j)+2}} 
% \times M_{\hat{\alpha},\hat{\beta}}^{(N-1)}$ 
% for all $\beta \in \DP_N$ 
% {\color{green} containing the link $\upwedgeat{j}$ }
% {\color{blue} such that $\upwedgeat{j} \in \beta$ } .
% %This concludes the proof.
% \end{proof}
% 
% {\color{red} I shortened this proof: }
% 
% \begin{proof}[Proof of Theorem~\ref{thm: change of basis between conformal blocks and pure partition functions}]
% {\color{blue} 
% The change of basis matrix $M = \genMmat$ satisfies the recurrence 
% relations~\eqref{eq: recursion for the change of basis matrix} given in
% Lemma~\ref{lem: recursion of change of basis matrix with generic q}. 
% This recurrence is of type~\eqref{eq: equivalent recursion} with 
% $f(n) = \frac{\qnum{n}}{\qnum{n+1}}$, so %the Cascade Recursion in the form of 
% Corollary~\ref{cor: recursion for matrix elements} shows that
% the matrix entries $\genMmat_{\alpha,\beta}$ 
% are given by the weighted incidence matrix
%~\eqref{eq: def of weighted incidence matrix} 
% with tile weight $w(t) = \frac{\qnum{h_t}}{\qnum{h_t+1}}$. 
% By Theorem~\ref{thm: weighted KW incidence matrix inversion},
% this matrix is invertible with the asserted inverse.
% }
% %By Lemma~\ref{lem: recursion of change of basis matrix with generic q}, 
% %the change of basis matrix satisfies the recurrence relations
% %\eqref{eq: recursion for the change of basis matrix}, 
% %which are of type~\eqref{eq: equivalent recursion} with 
% %$f(n) = \frac{\qnum{n}}{\qnum{n+1}}$.
% %%$f(\alpha(j)+1) = \frac{\qnum{\alpha(j)+1}}{\qnum{\alpha(j)+2}}$.
% %Therefore, the Cascade Recursion in the form of 
% %Corollary~\ref{cor: recursion for matrix elements} shows that
% %the matrix entries $M_{\alpha,\beta} = \genMmat_{\alpha,\beta}$ 
% %%$M_{\alpha,\beta}^{(N)}$ 
% %are given by the weighted incidence matrix
% %\eqref{eq: def of weighted incidence matrix} 
% %with the weight $w(t) := \frac{\qnum{h_t}}{\qnum{h_t+1}}$
% %assigned to any Dyck tile $t$
% %By Theorem~\ref{thm: weighted KW incidence matrix inversion},
% %this matrix is invertible with the asserted inverse.
% %This concludes the proof of Theorem~\ref{thm: change of basis between conformal blocks and pure partition functions}.
% \end{proof}
% 
% %\subsubsection{\label{sss: construction of coblovec}\textbf{Proof of Theorem~\ref{thm: conformal block vectors}}}
% \subsection{\label{sss: construction of coblovec}Proof of Theorem~\ref{thm: conformal block vectors}}

Once Proposition~\ref{prop: conformal block vectors} is established, the
construction %will be 
is immediate:
\begin{thm}\label{thm: conformal block functions}
Let $( \Coblobastwodim_\alpha )_{\alpha \in \DP}$ be the collection
of vectors in Proposition~\ref{prop: conformal block vectors}, and let 
$\sF \colon \HWsp_{2N}^{(0)} \to \Sol_N$ 
% $\sF \colon 
% \red{
% \HWspCal_N
% } 
% \to \Sol_N$ 
be the linear isomorphisms of Proposition~\ref{prop: SCCG correspondence map}.
Then the functions
\[ \ConfBlockFun_\alpha := \frac{1}{B^N} \times \sF[\Coblobastwodim_\alpha]  , \qquad \text{for $\alpha \in \DP_N$,}\]
%for $\alpha \in \DP_N$, 
satisfy the defining %properties 
properties~\eqref{eq: PDE for conformal blocks}, \eqref{eq: COV for conformal blocks}, and \eqref{eq: ASY for conformal blocks}
of conformal block functions.
%:~\eqref{eq: PDE for conformal blocks}, \eqref{eq: COV for conformal blocks}, and \eqref{eq: ASY for conformal blocks}.
\end{thm}
\begin{proof}
This follows by combining Propositions~\ref{prop: conformal block vectors}
and~\ref{prop: SCCG correspondence map}.
\end{proof}

\subsection{\textbf{Proof of Proposition~\ref{prop: conformal block vectors}}}

The rest of this section constitutes the proof of Proposition~\ref{prop: conformal block vectors},
%The proof is 
divided in four parts: uniqueness, construction, linear independence, and verification of
projection properties. %linear independence.
% % % $( \Coblobastwodim_\alpha )_{\alpha \in \DP_N}$ forming a basis for $\HWsp_{2N}^{(0)}$.
% % % % $\red{\HWspCal_N}$.
The uniqueness is routine by considering the corresponding homogeneous problem.
The explicit construction of the vectors $\Coblobastwodim_\alpha$ is the essence of the proof.
In the construction, each Dyck path $\alpha \in \DP_N$
specifies a sequence of representations of the quantum group,
and we recursively assemble the vector $\Coblobastwodim_\alpha$ proceeding along this sequence.
Linear independence is transparent in the construction. Finally, for the projection properties
we just have to inspect a number of cases explicitly.


%The uniqueness amounts to a routine check that the corresponding homogeneous linear
%system has no non-zero solutions, and the linear independence follows
%by a calculation entirely parallel to Theorem~\ref{thm: change of basis theorem}.
%The main task is therefore to explicitly construct the vectors $\Coblobastwodim_\alpha$.
% % % In the first two parts, the (unique) conformal block vectors 
% % % $( \Coblobastwodim_\alpha )_{\alpha \in \DP}$
% % % are constructed.
%For each link pattern $\alpha$, %vector $\Coblobastwodim_\alpha$,
%the corresponding Dyck path %associated to the link pattern $\alpha \in \DP_N$
%\blue{Each Dyck path $\alpha \in \DP_N$}
%specifies a sequence of representations of the quantum group, and our construction
%of the vector $\Coblobastwodim_\alpha$ is recursive along this sequence.


% % % This 
% % % {\color{green} construction of the vectors }
% % % {\color{blue} (remove "construction of the vectors") }
% % % is the key step 
% % % for the construction of the conformal block functions $\ConfBlockFun_\alpha$ by the quantum group method.
% The vectors are shown to belong to the trivial 
% subrepresentation~\eqref{eq: trivial submodule}
% %$\HWsp_{2N}^{(0)}$ 
% and the projection properties~\eqref{eq: coblo walk projection conditions}
% are verified. The third part concerns the collections 
% $( \Coblobastwodim_\alpha )_{\alpha \in \DP_N}$ for fixed $N$, which 
% form a basis of the trivial 
% subrepresentation~\eqref{eq: trivial submodule}. % $\HWsp_{2N}^{(0)}$.

\subsubsection{\textbf{Uniqueness}}
%\paragraph*{\emph{Uniqueness}}

Uniqueness of the collection
$( \Coblobastwodim_\alpha )_{\alpha \in \DP}$ 
of vectors satisfying the projection properties~\eqref{eq: coblo walk projection conditions}
follows from arguments exploited in similar contexts in
the articles~\cite{KP-pure_partition_functions_of_multiple_SLEs,
P-basis_for_solutions_of_BSA_PDEs}.
% and therefore, 
%we only give the idea to prove the uniqueness.
The crucial observation is the following lemma about the homogeneous problem.
%that the homogeneous system 
%\begin{align*}
%K.\Coblobastwodim_\alpha = \Coblobastwodim_\alpha, \qquad
%E.\Coblobastwodim_\alpha = 0, \qquad
%\hat{\pi}_j(\Coblobastwodim_\alpha) = 0 \quad \text{ for all } j \in \set{1,\ldots,2N-1},
%\end{align*}
%where the first two equations just state the property  
%$\Coblobastwodim_\alpha \in \HWsp_{2N}^{(0)}$,
%only admits the trivial solution. To show this, the following lemma is crucial.
%%The lemma below gives a useful characterization of the vectors of
%%highest dimensional subrepresentation $\Wd_{n+1} \subset \Wd_2^{\tens n}$
%%in terms of projections to subrepresentations in 
%%two consecutive tensor components.
%%\begin{lem}{\cite[Lemma~2.4 \& Corollary~2.5]{KP-pure_partition_functions_of_multiple_SLEs}}
%%\label{lem: all projections vanish}
%%For any $v\in\Wd_{2}^{\tens n}$, %$n= d - 1 \in \bZpos$,
%%the following conditions are equivalent.
%%\begin{description}
%%\item[(a)] $\hat{\pi}_{j}^{(1)}(v) = 0$ for all $1\leq j < n.$
%%\item[(b)] $v\in\Wd_{n+1} \subset \Wd_{2}^{\tens n}$.
%%\end{description}
%%In particular, if we have $E.v=0$, $K.v=v$, and 
%%$\hat{\pi}_{j}^{(1)}(v) = 0$ for all $1\leq j< n,$ then $v=0$.  
%%\end{lem}

\begin{lem}
%[\cite[Corollary~2.5]{KP-pure_partition_functions_of_multiple_SLEs}]
\label{lem: all projections vanish}
If a vector $v \in \HWsp_{2N}^{(0)}$ 
% $v \in
% \red{
% \HWspCal_N
% } 
% $ 
satisfies the property
$\hat{\pi}_{j}^{(1)}(v) = 0$ for all $j \in \set{1,\ldots,2N-1}$, then $v = 0$.  
\end{lem}

\begin{proof}
See, e.g.,~\cite[Corollary~2.5]{KP-pure_partition_functions_of_multiple_SLEs}. 
\end{proof}

As a corollary, the solution space of the recursive projection
properties~\eqref{eq: coblo walk projection conditions} is one-dimensional,
with initial condition $u_\emptywalk \in \Wd_2^{\tens 0} \isom \bC$ determining
the solution.
\begin{cor}\label{cor: uniqueness}
Let $( \Coblobastwodim_\alpha )_{\alpha \in \DP}$ and 
$( \Coblobastwodim'_\alpha )_{\alpha \in \DP}$ 
be two collections of vectors 
$\Coblobastwodim_\alpha,\Coblobastwodim'_\alpha \in \HWsp_{2N}^{(0)}$
% $\Coblobastwodim_\alpha,\Coblobastwodim'_\alpha \in 
% \red{
% \HWspCal_N
% } 
% $
satisfying the projection properties~\eqref{eq: coblo walk projection conditions}, 
and having same initial condition $\Coblobastwodim'_\emptywalk = \Coblobastwodim_\emptywalk$. 
Then we have
\begin{align*}
\Coblobastwodim'_\alpha = \Coblobastwodim_\alpha\qquad\text{for all }\alpha \in \DP.
\end{align*}
\end{cor}
\begin{proof}
%The proof is similar to the proof of
%\cite[Proposition~3.2]{KP-pure_partition_functions_of_multiple_SLEs},
%see also~\cite[Corollary~2.5]{KP-pure_partition_functions_of_multiple_SLEs}.
% % % % Clearly $\Coblobastwodim'_\emptywalk = c \, \Coblobastwodim_\emptywalk$ for some
% % % % $c\in\bC\setminus\{0\}$. 
Let $N \geq 1$ and suppose the condition
$\Coblobastwodim'_\beta = \Coblobastwodim_\beta$ holds 
for all $\beta \in \DP_{N-1}$.
Then, for any $\alpha \in \DP_N$, 
the difference $v = \Coblobastwodim'_\alpha - \Coblobastwodim_\alpha$
satisfies $\hat{\pi}_j(v) = 0$ for all $j \in \set{1,\ldots,2N-1}$, 
so $v = 0$ by Lemma~\ref{lem: all projections vanish}. 
The assertion follows by induction on $N$.
\end{proof}

\subsubsection{\textbf{Construction}}
%\paragraph*{\emph{Construction}}

We now construct the vectors $\Coblobastwodim_\alpha$ of Proposition~\ref{prop: conformal block vectors} and show that they lie in the correct subspaces $\HWsp_{2N}^{(0)}$.
In the intermediate steps of the construction,
we encounter vectors in the highest weight vector 
spaces
% \footnote{\blue{For simplicity, in the special case of $n = 2N$ and $\nodef = 0$, we use
% the notation $\HWspCal_N = \HWsp_{2N}^{(0)}$ as in~\eqref{eq: trivial submodule}.}
% }
\begin{align} \label{eq: highest vector space}
\HWsp_{n}^{(\nodef)} = \set{ v \in \Wd_{2}^{\tens n} \; \Big| \; E.v=0 ,\;K.v = q^{\nodef}v } . 
\end{align}
These spaces consist of generators of the $\Wd_{d}$-isotypic components in the tensor
product~\eqref{eq: tensor power of two dimensionals} with $d = \nodef + 1$:
for any non-zero $v \in \HWsp_{n}^{(\nodef)}$, the collection $(F^k.v)_{k=0}^s$ 
obtained from $v$ by the action of the generator $F$ spans a subrepresentation 
isomorphic to $\Wd_{d}$ in $\Wd_{2}^{\tens n}$.
%In particular, f
For each $d$, the dimension of the linear space 
\eqref{eq: highest vector space} equals $m_d^{(n)}$.
%Note that 
When $s \neq 0$,
the spaces~\eqref{eq: highest vector space} themselves are not %in general
representations of $\Uqsltwo$. %when $s \neq 0$. %$d \neq 1$.

For $k=1,2,\ldots,2N$,
we will first construct vectors $u_\walk^{(k)} \in \HWsp_{k}^{(\walk(k))}$, 
which can be thought of as being indexed by the first $k$ steps of the walk $\walk$.
% % % sub-walks of $\walk \in \DP_N$ 
% % % on $\bZnn$ starting from zero and ending at $\walk(k)$. %$\walk(k) \in \bZnn$.
From these vectors we will then construct the vectors $\Coblobastwodim_\alpha$ --- see Equation~\eqref{eq: coblobas vectors} below.

Let $u_\walk^{(0)} := 1 \in \bC \isom \Wd_2^{\tens 0}$. 
Define recursively
$u_\walk^{(k+1)} \in \Wd_2^{\tens (k+1)}$ in terms of $u_\walk^{(k)} \in \Wd_2^{\tens k}$ by
% Assume that the vector 
% $u_\walk^{(k)} \in \HWsp_{k}^{(\walk(k))} \subset \Wd_2^{\tens k}$
% has been constructed. Define
% % % % % %\begin{align}\label{eq: coblobas vector recursive construction}
% % % % % %u_\walk^{(k+1)} := \; & \begin{cases}
% % % % % %u_\walk^{(k)}\tens \Wbas_0 & \text{if } \walk(k+1) = \walk(k)+1 \\
% % % % % %\frac{1}{q-q^{-1}}\left(-q\,u_\walk^{(k)}\tens\Wbas_1 + \frac{1}{\qnum{\walk(k)}}\, F.u_\walk^{(k)}\tens\Wbas_0\right) 
% % % % % %& \text{if } \walk(k+1) = \walk(k)-1. \end{cases}
% % % % % %\end{align}
\begin{align}\label{eq: coblobas vector recursive construction}
u_\walk^{(k+1)} := \; & \begin{cases}
\Wbas_0 \tens u_\walk^{(k)} & \text{if }\walk(k+1) = \walk(k)+1 \\
\frac{1}{q-q^{-1}}\left( \Wbas_1 \tens u_\walk^{(k)} 
- \frac{q^{\walk(k)}}{\qnum{\walk(k)}}\,\Wbas_0 \tens F.u_\walk^{(k)} \right) 
& \text{if } \walk(k+1) = \walk(k)-1. \end{cases}
\end{align}

\begin{lem}\label{lem: recursively defined vectors}
For $k=0,1,\ldots,2N$,
the vectors $u_\walk^{(k)} \in \Wd_2^{\tens k}$
%thus defined
satisfy $u_\walk^{(k)} \in \HWsp_{k}^{(\walk(k))}$, that is, we have
\begin{align}\label{eq: hwv properties of u} 
E.u_\walk^{(k)} = 0 \qquad \text{and} \qquad K.u_\walk^{(k)} 
= q^{\walk(k)}\,u_\walk^{(k)}.
\end{align}
% The vectors $u_\walk^{(k+1)} \in \Wd_2^{\tens k+1}$ thus defined satisfy
% $u_\walk^{(k+1)} \in \HWsp_{k+1}^{(\walk(k+1))}$, that is, we have 
% \begin{align}\label{eq: hwv properties of u} 
% E.u_\walk^{(k+1)} = 0 \qquad \text{and} \qquad K.u_\walk^{(k+1)} 
% = q^{\walk(k+1)}\,u_\walk^{(k+1)}.
% \end{align}
\end{lem}
\begin{proof}
We prove the assertion by induction on $k$ relying on a direct calculation. 
The base case $k=0$ is clear. Assuming that the claim holds for $u_\walk^{(k)}$,
we verify it for $u_\walk^{(k+1)}$.
Recall that the actions of $E$ and $K$ on $\Wd_2 \tens \Wd_2^{\tens k}$ are given by the 
coproduct~\eqref{eq: coproduct}. We %will 
use the identities
\begin{align*}
E.\Wbas_0 = 0, \qquad E.\Wbas_1 = \Wbas_0, \qquad
K.\Wbas_0 = q\,\Wbas_0, \qquad K.\Wbas_1 = q^{-1}\,\Wbas_1, \qquad
E.u_\walk^{(k)} = 0, \qquad K.u_\walk^{(k)} = q^{\walk(k)}\,u_\walk^{(k)}.
\end{align*}
If $\walk(k+1) = \walk(k)+1$, we have 
$u_\walk^{(k+1)} = \Wbas_0 \tens u_\walk^{(k)}$ and we easily calculate
\begin{align*}
E.u_\walk^{(k+1)} = \; & E.\Wbas_0 \tens K.u_\walk^{(k)}
 + 1.\Wbas_0 \tens E.u_\walk^{(k)} = 0 \\
K.u_\walk^{(k+1)} = \; & K.\Wbas_0 \tens K.u_\walk^{(k)} 
= q^{1+\walk(k)} \, u_\walk^{(k+1)} = q^{\walk(k+1)}\,u_\walk^{(k+1)}.
\end{align*}

If $\walk(k+1) = \walk(k)-1$, 
then we have
$u_\walk^{(k+1)} 
= \frac{1}{q-q^{-1}}\left( \Wbas_1 \tens u_\walk^{(k)} 
- \frac{q^{\walk(k)}}{\qnum{\walk(k)}}\,\Wbas_0 \tens F.u_\walk^{(k)} \right)$, 
and we similarly~get
\begin{align*}
E.u_\walk^{(k+1)} = \; & \frac{1}{q-q^{-1}} \left(
E.\Wbas_1 \tens K.u_\walk^{(k)} 
- q^{\walk(k)}\, \frac{E.\Wbas_0 \tens KF.u_\walk^{(k)}}{\qnum{\walk(k)}}
+ 1.\Wbas_1 \tens E.u_\walk^{(k)} 
- q^{\walk(k)}\,
\frac{1.\Wbas_0 \tens EF.u_\walk^{(k)}}{\qnum{\walk(k)}} \right) \\
= \; & \frac{1}{q-q^{-1}}\left(
E.\Wbas_1 \tens K.u_\walk^{(k)} 
- \frac{q^{\walk(k)}}{\qnum{\walk(k)}}\,\Wbas_0 \tens 
\frac{(K-K^{-1}).u_\walk^{(k)}}{q-q^{-1}} \right) \\
= \; & \frac{1}{q-q^{-1}}\left(
q^{\walk(k)}\, \Wbas_0 \tens u_\walk^{(k)} 
- \frac{q^{\walk(k)}}{\qnum{\walk(k)}} \,\Wbas_0 \tens 
\frac{q^{\walk(k)}-q^{-\walk(k)}}{q-q^{-1}} \, u_\walk^{(k)} \right) \\
= \; & \frac{q^{\walk(k)}}{q-q^{-1}}\left(
\Wbas_0 \tens u_\walk^{(k)} 
- \frac{\qnum{\walk(k)}}{\qnum{\walk(k)}} \,\Wbas_0 \tens u_\walk^{(k)} \right) 
= 0,
\end{align*}
where we also used the commutation relation 
$EF-FE = \frac{1}{q-q^{-1}}\left(K-K^{-1}\right)$ from~\eqref{eq: quantum group relations}.

Finally, using the commutation relation $KF = q^{-2} FK$  from~\eqref{eq: quantum group relations}, we get
(still with $\walk(k+1) = \walk(k)-1$)
\begin{align*}
K.u_\walk^{(k+1)} = \; & \frac{1}{q-q^{-1}}\left(
K.\Wbas_1 \tens K.u_\walk^{(k)} 
- q^{\walk(k)}\, 
\frac{K.\Wbas_0 \tens KF.u_\walk^{(k)}}{\qnum{\walk(k)}} \right) \\
= \; & \frac{1}{q-q^{-1}}\left(
q^{-1+\walk(k)}\, \Wbas_1 \tens u_\walk^{(k)} 
- q^{1-2+2\walk(k)} \, \frac{\Wbas_0 \tens F.u_\walk^{(k)}}{\qnum{\walk(k)}} 
\right) \\
= \; & q^{\walk(k)-1}\,u_\walk^{(k+1)} = q^{\walk(k+1)}\,u_\walk^{(k+1)}.
\end{align*}
This concludes the proof.
\end{proof}

The vectors $\Coblobastwodim_\alpha$ corresponding to 
the conformal block functions $\ConfBlockFun_\alpha$
are obtained by taking the last
of the recursively defined vectors above, $u_\walk^{(2N)}$, and normalizing it appropriately.
Specifically, for 
$\alpha \in \DP_N$, we set
\begin{align}\label{eq: coblobas vectors}
\Coblobastwodim_\alpha %= \Coblobastwodim_\alpha^{(N)} 
:= \qnum{2}^N \NormalizationConstant_\alpha \times u_{\alpha}^{(2N)},
\qquad \text{ where } \qquad
\NormalizationConstant_\alpha:=
\Big( \prod_{\upwedgeat{i} \in \alpha} \frac{1}{\qnum{\alpha(i)+1}} \Big)
\Big( \prod_{\downwedgeat{i} \in \alpha} \qnum{\alpha(i)+1} \Big) .
\end{align}
% From Lemma~\ref{lem: recursively defined vectors} it follows that
% $\Coblobastwodim_\alpha$ belongs to the trivial
% subrepresentation~\eqref{eq: trivial submodule}.
% The projection properties~\eqref{eq: coblo walk projection conditions}
% of these conformal block vectors will be proved in
% Proposition~\ref{prop: coblo walk projection properties}.
% That they belong to the trivial
% subrepresentation~\eqref{eq: trivial submodule} is a direct consequence
% of Lemma~\ref{lem: recursively defined vectors}.
We finish this subsection by noting that these vectors indeed belong
to the trivial subrepresentation~\eqref{eq: trivial submodule}.
% % % % % % % %For a Dyck path $\walk \in \DP_N$, we now set
% % % % % % % For a link pattern (Dyck path) $\alpha \in \DP_N$, we now set
% % % % % % % \begin{align}\label{eq: coblobas vectors}
% % % % % % % \Coblobastwodim_\alpha= \Coblobastwodim_\alpha^{(N)} 
% % % % % % % := \qnum{2}^N \NormalizationConstant_\alpha \times u_{\alpha}^{(2N)},
% % % % % % % \qquad \text{ where } \qquad
% % % % % % % \NormalizationConstant_\alpha:=
% % % % % % % \Big( \prod_{\upwedgeat{i} \in \alpha} \frac{1}{\qnum{\alpha(i)+1}} \Big)
% % % % % % % \Big( \prod_{\downwedgeat{i} \in \alpha} \qnum{\alpha(i)+1} \Big) .
% % % % % % % \end{align}
% % % % % % % %and define the vectors $u_{\walk}^{(2N)} \in \HWsp_{2N}^{(0)}$ 
% % % % % % % %recursively as follows.
% % % % % % % These vectors correspond to the conformal blocks.
% % % % % % % We can easily verify that they belong to 
% % % % % % % the trivial subrepresentation~\eqref{eq: trivial submodule}.
% % % % % % % The projection properties~\eqref{eq: coblo walk projection conditions}
% % % % % % % are proved in Proposition~\ref{prop: coblo walk projection properties}.

\begin{cor} \label{cor: coblo walk hwv properties}
We have $\Coblobastwodim_\alpha \in \HWsp_{2N}^{(0)}$ 
% $\Coblobastwodim_\alpha \in 
% \red{
% \HWspCal_N = \HWsp_{2N}^{(0)}
% }
% $ 
for all $\alpha \in \DP_N$.
\end{cor}
\begin{proof}
This follows immediately from the properties
\eqref{eq: hwv properties of u} of $u_{\alpha}^{(k)}$ with $k = 2N$.
\end{proof}


\subsubsection{\label{sss: linear independence}\textbf{Linear independence}}

We now quickly verify the linear independence of the
vectors $\Coblobastwodim_\alpha$ constructed in~\eqref{eq: coblobas vector recursive construction}
and~\eqref{eq: coblobas vectors}.
Since we have $\dmn \HWsp_{2N}^{(0)} = \Catalan_N = \# \DP_N$,
linear independence also implies that the collection
$(\Coblobastwodim_\alpha)_{\alpha \in \DP_N}$ is a basis of $\HWsp_{2N}^{(0)}$.

By the recursive construction~\eqref{eq: coblobas vector recursive construction},
the first $k$ steps of $\alpha$ determine a vector $u_\alpha^{(k)} \in \Wd_2^{\tens k}$.
Inductively on $k$, it is clear that all different initial segments of $k$ steps
define linearly independent vectors. The linear independence of
$(\Coblobastwodim_\alpha)_{\alpha \in \DP_N}$ follows from the case $k=2N$.


\subsubsection{\textbf{Projection properties}}

To prove the projection properties~\eqref{eq: coblo walk projection conditions}
for the vectors $\Coblobastwodim_\alpha$ %$(\Coblobastwodim_\alpha)_{\alpha \in \DP_N}$
constructed in~\eqref{eq: coblobas vector recursive construction}
and~\eqref{eq: coblobas vectors},
we use a recursion property of the normalization coefficients
$\NormalizationConstant_\alpha$. % in~\eqref{eq: coblobas vectors}.

\begin{lem}\label{lem: recursion for normalization constants}
The coefficients $\NormalizationConstant_\alpha$ satisfy the %recurrence relations
following recursion: for any $j$, we have
\begin{align}\label{eq: recursion for coblobas normalization constants}
\NormalizationConstant_\alpha= \begin{cases}
\frac{\qnum{\alpha(j)}}{\qnum{\alpha(j)+1}} \times
 \NormalizationConstant_{\alpha \removeupwedge{j}} & \text{if } \upwedgeat{j} \in \alpha \\ %\wedgeat{j} = \upwedgeat{j}\\
\frac{\qnum{\alpha(j)+1}}{\qnum{\alpha(j)+2}}\times
 \NormalizationConstant_{\alpha \removedownwedge{j}} & \text{if } \downwedgeat{j} \in \alpha . % \wedgeat{j} = \downwedgeat{j} .
\end{cases}
\end{align}
\end{lem}
%%%{\color{red}Eve: this proof can be simplified as follows:
%%%\begin{proof}
%%%Fix a single-valued complex square root function, and consider the steps of a path as segments of a line. Then, it is easy to see that
%%%\begin{align*}
%%%\prod_{\text{all steps of the path }\alpha} \frac{\sqrt{[\text{minimum height of the step } + 1]}}{\sqrt{[\text{maximum height of the step } + 1]}} =
%%%\Big( \prod_{\upwedgeat{i} \in \alpha} \frac{1}{\qnum{\alpha(i)+1}} \Big)
%%%\Big( \prod_{\downwedgeat{i} \in \alpha} \qnum{\alpha(i)+1} \Big) = \NormalizationConstant_\alpha.
%%%\end{align*}
%%%since if there is a slope at $j$, the $[\alpha(j) + 1]$ cancel out from subsequent step factors of the product, and the endpoints $j=0, 2N$ do not affect since $[1] = 1$. The claim is obvious from this stepwise product form for $\NormalizationConstant_\alpha$.
%%%\end{proof}
%%%}
\begin{proof}
Observe that the coefficients in~\eqref{eq: coblobas vectors} can be written in the form
%(with a fixed branch of the square root function)
\begin{align*}%\label{eq: normalization constant}
\prod_{i=1}^{2N} \frac{\sqrt{\qnum{\min\{\alpha(i-1),\alpha(i)\} + 1}}}{\sqrt{\qnum{\max\{\alpha(i-1),\alpha(i)\} + 1}}}
= \Big( \prod_{\upwedgeat{i} \in \alpha} \frac{1}{\qnum{\alpha(i)+1}} \Big)
\Big( \prod_{\downwedgeat{i} \in \alpha} \qnum{\alpha(i)+1} \Big) = \NormalizationConstant_\alpha.
\end{align*}
%Now, the expression~\eqref{eq: normalization constant} 
The expression on the left clearly satisfies 
the recursion~\eqref{eq: recursion for coblobas normalization constants}.
\end{proof}
%%%{\color{blue}
%%%\begin{proof}
%%%Observe that %(with a fixed branch of the square root function), 
%%%the following product over the steps %$(\alpha(i-1),\alpha(i))$ 
%%%of the Dyck path $\alpha$, 
%%%\begin{align}\label{eq: normalization constant}
%%%\prod_{i=1}^{2N} \frac{\sqrt{\qnum{\min\{\alpha(i-1),\alpha(i)\} + 1}}}{\sqrt{\qnum{\max\{\alpha(i-1),\alpha(i)\} + 1}}} ,
%%%%= \Big( \prod_{\upwedgeat{i} \in \alpha} \frac{1}{\qnum{\alpha(i)+1}} \Big)
%%%%\Big( \prod_{\downwedgeat{i} \in \alpha} \qnum{\alpha(i)+1} \Big) = \NormalizationConstant_\alpha
%%%\end{align}
%%%satisfies 
%%%the recursion~\eqref{eq: recursion for coblobas normalization constants}.
%%%It remains to prove that in fact,~\eqref{eq: normalization constant} is equal to
%%%\begin{align}\label{eq: normalization constant again}
%%%%\prod_{i=1}^{2N} \frac{\sqrt{\qnum{\min\{\alpha(i-1),\alpha(i)\} + 1}}}{\sqrt{\qnum{\max\{\alpha(i-1),\alpha(i)\} + 1}}}
%%%%= 
%%%\Big( \prod_{\upwedgeat{i} \in \alpha} \frac{1}{\qnum{\alpha(i)+1}} \Big)
%%%\Big( \prod_{\downwedgeat{i} \in \alpha} \qnum{\alpha(i)+1} \Big) = \NormalizationConstant_\alpha.
%%%\end{align}
%%%Fix $j \in \set{0, 1, \ldots, 2N}$ and consider a term 
%%%of the form $\sqrt{\qnum{\alpha(j) + 1}}^{\;\power}$ 
%%%in~\eqref{eq: normalization constant}
%%%%containing $\alpha(j)$ 
%%%which arises from the $j$:th and $j+1$:st
%%%factor of the product~\eqref{eq: normalization constant},
%%%where the power $\power$ is to be specified.
%%%We may assume that $j \neq 0,2N$, because for $j = 0,2N$, we have 
%%%$\qnum{\alpha(j) + 1} = \qnum{1} = 1$.
%%%%Therefore, these factors 
%%%%in the product~\eqref{eq: normalization constant} are irrelevant. 
%%%We specify the power $\power$ in three cases, depending on 
%%%the $j$:th and $j+1$:st steps of $\alpha$.
%%%%the steps $j$ and $j+1$ of $\alpha$.
%%%First, if $\alpha$ has a slope $\slopeat{j}$ at $j$, then we have 
%%%$\power = 1 - 1 = 0$, because $\alpha(j)$ is the minimum height of exactly one 
%%%of the steps %$j$ and $j+1$ of $\alpha$ 
%%%%$(\alpha(i-1),\alpha(i))$ and $(\alpha(i),\alpha(i+1))$ 
%%%and the maximum of the other one. 
%%%Similarly, if $\alpha$ has an up-wedge $\upwedgeat{j}$ 
%%%(resp. down-wedge $\downwedgeat{j}$) 
%%%at $j$, then $\alpha(j)$ is the maximum (resp. minimum) height of both steps,
%%%so $\power = -1 - 1 = -2$ (resp. $\power = 1 + 1 = 2$).
%%%Therefore,~\eqref{eq: normalization constant again} is equal to
%%%\eqref{eq: normalization constant}. This concludes the proof.
%%%
%%%%Second, if $\alpha$ has an up-wedge $\upwedgeat{j}$ 
%%%%at $j$, then $\alpha(j)$ is the maximum height of both steps 
%%%%%$(\alpha(j-1),\alpha(j))$ and $(\alpha(j),\alpha(j+1))$, 
%%%%so $\power = -1 - 1 = -2$. 
%%%%Finally, if $\alpha$ has a down-wedge $\downwedgeat{j}$ at $i$, then 
%%%%$\alpha(j)$ is the minimum height of both steps 
%%%%$(\alpha(j-1),\alpha(j))$ and $(\alpha(j),\alpha(j+1))$, so $\power = 1 + 1 = 2$.
%%%%This concludes the proof.
%%%
%%%%Now, the expression~\eqref{eq: normalization constant} 
%%%%This expression clearly satisfies 
%%%%the recursion~\eqref{eq: recursion for coblobas normalization constants}.
%%%\end{proof}
%%%}

We %will next 
also make use of the following explicit formulas for the projection $\hat{\pi}$ defined in Equation~\eqref{eq: projection to singlet}.

\begin{lem}
%[{see, e.g.~\cite[Lemma~2.3]{KP-pure_partition_functions_of_multiple_SLEs}}] 
\label{lem: projection formulas}
With $\Sbas \in \Wd_1$ defined in~\eqref{eq: singlet basis vector},
we have $\pi(v)=\hat{\pi}(v) \, \Sbas$ for any $v\in\Wd_{2}\tens\Wd_{2}$, and
%The values of $\hat{\pi}$ on the tensor product basis are
\begin{align*}
&\hat{\pi}(\Wbas_{0}\tens\Wbas_{0})=0,\qquad\qquad\quad\qquad\qquad\hat{\pi}(\Wbas_{1}\tens\Wbas_{1})=0,\\
&\hat{\pi}(\Wbas_{0}\tens\Wbas_{1})=\frac{q^{-1}-q}{\qnum 2},\qquad\qquad\qquad\hat{\pi}(\Wbas_{1}\tens\Wbas_{0})=\frac{1-q^{-2}}{\qnum 2}.
\end{align*}
\end{lem}
\begin{proof}
See, e.g.,~\cite[Lemma~2.3]{KP-pure_partition_functions_of_multiple_SLEs}.
\end{proof}

%With these preparations, w
%We are now ready to prove the projection properties~\eqref{eq: coblo walk projection conditions}
%of the vectors $\Coblobastwodim_\alpha$.

\begin{prop}\label{prop: coblo walk projection properties}
The vectors $( \Coblobastwodim_\alpha )_{\alpha \in \DP}$, 
defined in~\eqref{eq: coblobas vectors}, 
satisfy the projection properties~\eqref{eq: coblo walk projection conditions}.
\end{prop}
\begin{proof}
Fix $j\in\set{1,\ldots,2N-1}$. As the projection $\pi_j$ acts locally
on the $j$:th and $(j+1)$:st tensor components, the value of 
$\hat{\pi}_j(\Coblobastwodim_\alpha)$ can be calculated using 
the explicit construction~\eqref{eq: coblobas vector recursive construction} 
and the %recurrence
recursion~\eqref{eq: recursion for coblobas normalization constants} of 
Lemma~\ref{lem: recursion for normalization constants} for the normalization 
constants appearing in the definition~\eqref{eq: coblobas vectors} 
of~$\Coblobastwodim_\alpha$.
We treat separately each possible local shape of a Dyck path $\alpha$
at $j$, i.e., the cases depicted in Figure~\ref{fig: wedges and slopes}.

Suppose first that $\alpha$ contains a slope 
at $j$, i.e., $\slopeat{j} \in \alpha$.
We need to show that in this case, we have 
$\hat{\pi}_j(\Coblobastwodim_\alpha) = 0$, or, equivalently, that
$\hat{\pi}_j(u_{\alpha}^{(j+1)}) = 0$. 
Depending whether the slope is an up-slope or a down-slope,
we study the two cases in~\eqref{eq: coblobas vector recursive construction}.

In the easiest case of an up-slope, that is, when we have $\alpha(j) = \alpha(j-1)+1$ and 
$\alpha(j+1) = \alpha(j)+1$, the tensor components $j$ and $j+1$ in 
$u_{\alpha}^{(j+1)}$ (counting from the right)
are proportional to $\Wbas_0 \tens \Wbas_0$,
and $\hat{\pi}_j$ thus annihilates the vector $u_{\alpha}^{(j+1)}$ 
by Lemma~\ref{lem: projection formulas}(a). 
Equations~\eqref{eq: coblobas vector recursive construction} 
and~\eqref{eq: coblobas vectors} then show that we also have
$\hat{\pi}_j(\Coblobastwodim_\alpha) = 0$, as asserted in~\eqref{eq: coblo walk projection conditions}.

In the case of a down-slope, that is, when we have $\alpha(j) = \alpha(j-1)-1$ and 
$\alpha(j+1) = \alpha(j)-1$, the tensor components $j$ and $j+1$ in 
$u_{\alpha}^{(j+1)}$ have several terms. To perform the calculations,
it is convenient to first write down the action of $F$ on 
$u_{\alpha}^{(j)}$.
The action is given by the coproduct~\eqref{eq: coproduct} as follows:
\begin{align*}
(q-q^{-1}) \, F.u_\alpha^{(j)} \; & = 
F. \left( \Wbas_1 \tens u_\alpha^{(j-1)} 
- \frac{q^{\alpha(j-1)}}{\qnum{\alpha(j-1)}}\,\Wbas_0 \tens F.u_\alpha^{(j-1)} \right) \\
\; & = F.\Wbas_1 \tens 1.u_\alpha^{(j-1)} 
+ K^{-1}.\Wbas_1 \tens F.u_\alpha^{(j-1)}
- q^{\alpha(j-1)}\, \frac{F.\Wbas_0 \tens F.u_\alpha^{(j-1)}
- K^{-1}.\Wbas_0 \tens F^2.u_\alpha^{(j-1)}}{\qnum{\alpha(j-1)}} \\
\; & = q\, \Wbas_1 \tens F.u_\alpha^{(j-1)} 
- \frac{q^{\alpha(j-1)}}{\qnum{\alpha(j-1)}}\left( \Wbas_1 \tens F.u_\alpha^{(j-1)}
- q^{-1} \, \Wbas_0 \tens F^2.u_\alpha^{(j-1)} \right) \\
\; & = \left( q - \frac{q^{\alpha(j-1)}}{\qnum{\alpha(j-1)}} \right) 
\Wbas_1 \tens F.u_\alpha^{(j-1)}
- \frac{q^{\alpha(j-1)-1}}{\qnum{\alpha(j-1)}}\, \Wbas_0 \tens F^2.u_\alpha^{(j-1)},
\end{align*}
where we used the identities $F.\Wbas_1 = 0$,
$F.\Wbas_0 = \Wbas_1$, $K^{-1}.\Wbas_1 = q\,\Wbas_1$, 
and $K^{-1}.\Wbas_0 = q^{-1}\,\Wbas_0$.
The vector $u_\alpha^{(j+1)}$ now reads
\begin{align*}
u_\alpha^{(j+1)} \propto \; &
\Wbas_1 \tens u_\alpha^{(j)} 
- \frac{q^{\alpha(j)}}{\qnum{\alpha(j)}}\,\Wbas_0 \tens F.u_\alpha^{(j)} \\
\propto \; &  \Wbas_1 \tens \left(
\Wbas_1 \tens u_\alpha^{(j-1)} 
- \frac{q^{\alpha(j-1)}}{\qnum{\alpha(j-1)}}\,\Wbas_0 \tens F.u_\alpha^{(j-1)} \right)  \\
\; & - \frac{q^{\alpha(j)}}{\qnum{\alpha(j)}}\,\Wbas_0 \tens \left(
\left( q - \frac{q^{\alpha(j-1)}}{\qnum{\alpha(j-1)}} \right) 
\Wbas_1 \tens F.u_\alpha^{(j-1)}
- \frac{q^{\alpha(j-1)-1}}{\qnum{\alpha(j-1)}}\, \Wbas_0 \tens F^2.u_\alpha^{(j-1)} \right).
\end{align*}

Using Lemma~\ref{lem: projection formulas}(a), the down-step
$\alpha(j) = \alpha(j-1)-1$, and the geometric sum expansion
of the $q$-integers $\qnum{n} = q^{n-1} + q^{n-3} + \cdots + q^{3-n} + q^{1-n}$, %given in~\eqref{eq: q numbers}, 
we verify that
\begin{align*}
\hat{\pi}_j(u_\alpha^{(j+1)})
\propto \; &  
\left(
- \frac{q^{\alpha(j-1)}}{\qnum{\alpha(j-1)}}\,
\hat{\pi} ( \Wbas_1 \tens \Wbas_0 )
- \frac{q^{\alpha(j)+1} - \frac{q^{\alpha(j)+\alpha(j-1)}}{\qnum{\alpha(j-1)}}}{\qnum{\alpha(j)}}\,
\hat{\pi} ( \Wbas_0 \tens \Wbas_1 ) 
\right) \tens F.u_\alpha^{(j-1)} \\
% = \; & \left(
% - \frac{q^{\alpha(j-1)}}{\qnum{\alpha(j-1)}}\,
% \frac{1-q^{-2}}{\qnum 2}
% - \frac{q^{\alpha(j)+1} - \frac{q^{\alpha(j)+\alpha(j-1)}}{\qnum{\alpha(j-1)}}}{\qnum{\alpha(j)}}\, \frac{q^{-1}-q}{\qnum 2} \right) \tens F.u_\alpha^{(j-1)} \\
= \; & \left(
- \frac{q^{\alpha(j)+1}}{\qnum{\alpha(j-1)}}\,
\frac{1-q^{-2}}{\qnum 2}
- \frac{q^{\alpha(j)+1} - \frac{q^{2\alpha(j)+1}}{\qnum{\alpha(j-1)}}}{\qnum{\alpha(j)}}\, \frac{q^{-1}-q}{\qnum 2} \right) \tens F.u_\alpha^{(j-1)} \\
= \; & \frac{q^{\alpha(j)+1}(q-q^{-1})}{\qnum 2 \qnum{\alpha(j-1)} \qnum{\alpha(j)}} \times 
\left( - q^{-1}\,\qnum{\alpha(j)}
+ \qnum{\alpha(j)+1} - q^{\alpha(j)} \right) \tens F.u_\alpha^{(j-1)}  \\
= \; & 0.
\end{align*}
%using the identity
%\begin{align*}
%- q^{-1}\,\qnum{\alpha(j)}
%+ \qnum{\alpha(j)+1} - q^{\alpha(j)} \; & = 
%- q^{-1}\, (q^{\alpha(j)-1} + q^{\alpha(j)-3} + \cdots + q^{3-\alpha(j)} 
%+ q^{1-\alpha(j)}) \\
%\; &  + (q^{\alpha(j)} + q^{\alpha(j)-2} + \cdots + q^{2-\alpha(j)} + q^{-\alpha(j)})
%- q^{\alpha(j)} = 0.
%\end{align*}
It thus follows by 
Equations~\eqref{eq: coblobas vector recursive construction} 
and~\eqref{eq: coblobas vectors} 
that the asserted property $\hat{\pi}_j(\Coblobastwodim_\alpha) = 0$ holds
also with $\alpha$ having an down-slope at $j$.

Suppose then that $\alpha$ contains an up-wedge 
at $j$, i.e., $\upwedgeat{j} \in \alpha$.
We need to show that in this case, we have 
$\hat{\pi}_j(\Coblobastwodim_\alpha) 
= \Coblobastwodim_{\alpha \removeupwedge{j}}$. 
Now $\alpha(j) = \alpha(j-1)+1$ and $\alpha(j+1) = \alpha(j)-1$
and the vector $u_\alpha^{(j+1)}$ reads
\begin{align*}
u_\alpha^{(j+1)} 
= \; & \frac{1}{q-q^{-1}}\left( 
\Wbas_1 \tens (\Wbas_0 \tens u_\alpha^{(j-1)})
- \frac{q^{\alpha(j)}}{\qnum{\alpha(j)}}\,
\Wbas_0 \tens F.(\Wbas_0 \tens u_\alpha^{(j-1)}) \right) \\
= \; & \frac{1}{q-q^{-1}}\left(
\Wbas_1 \tens (\Wbas_0 \tens u_\alpha^{(j-1)})
- \frac{q^{\alpha(j)}}{\qnum{\alpha(j)}}\,\Wbas_0 \tens 
(F.\Wbas_0 \tens 1.u_\alpha^{(j-1)} + K^{-1}.\Wbas_0 \tens F.u_\alpha^{(j-1)}) \right) \\
= \; & \frac{1}{q-q^{-1}}\left(
\Wbas_1 \tens (\Wbas_0 \tens u_\alpha^{(j-1)}) 
- \frac{q^{\alpha(j)}}{\qnum{\alpha(j)}}\,\Wbas_0 \tens (\Wbas_1 \tens u_\alpha^{(j-1)} + q^{-1}\, \Wbas_0 \tens F.u_\alpha^{(j-1)}) \right).
\end{align*}
%where we used the identity
%\begin{align*}
%F.(\Wbas_0 \tens u_\alpha^{(j-1)}) 
%= \; & F.\Wbas_0 \tens 1.u_\alpha^{(j-1)} + K^{-1}.\Wbas_0 \tens F.u_\alpha^{(j-1)}
%= \Wbas_1 \tens u_\alpha^{(j-1)} + q^{-1}\, \Wbas_0 \tens F.u_\alpha^{(j-1)}.
%\end{align*}

Applying the projection $\hat{\pi}_j$ on both sides and using
Lemma~\ref{lem: projection formulas}(a), we obtain
\begin{align*}
\hat{\pi}_j ( u_\alpha^{(j+1)} )
= \; & \frac{1}{q-q^{-1}}\left(
\hat{\pi} ( \Wbas_1 \tens \Wbas_0)
- \frac{q^{\alpha(j)}}{\qnum{\alpha(j)}}\, 
\hat{\pi} ( \Wbas_0 \tens \Wbas_1) \right) \tens u_\alpha^{(j-1)} \\
= \; & \frac{1}{q-q^{-1}} \left(
\frac{1-q^{-2}}{\qnum 2}
- \frac{q^{\alpha(j)}}{\qnum{\alpha(j)}}\,\frac{q^{-1}-q}{\qnum 2}
\right) \times u_\alpha^{(j-1)} \\
= \; & \frac{1}{\qnum 2 \qnum{\alpha(j)}} 
\left( q^{-1}\,\qnum{\alpha(j)} + q^{\alpha(j)} \right)
\times u_\alpha^{(j-1)}.
\end{align*}
Using again the geometric sum expansion
of the $q$-integers, % given in~\eqref{eq: q numbers},
we simplify
%\begin{align*}
%q^{-1}\,\qnum{\alpha(j)} + q^{\alpha(j)}
%= \; & q^{-1} \, (q^{\alpha(j)-1} + q^{\alpha(j)-3} + \cdots + q^{3-\alpha(j)} + q^{1-\alpha(j)}) + q^{\alpha(j)} \\
%= \; & (q^{\alpha(j)-2} + q^{\alpha(j)-4} + \cdots + q^{2-\alpha(j)} + q^{-\alpha(j)}) + q^{\alpha(j)} \\
%= \; & \qnum{\alpha(j)+1},
%\end{align*}
the multiplicative factor by
$q^{-1}\,\qnum{\alpha(j)} + q^{\alpha(j)} =  \qnum{\alpha(j)+1}$,
which yields
\begin{align*}
\hat{\pi}_j ( u_\alpha^{(j+1)} )
= \; & \frac{\qnum{\alpha(j)+1}}{\qnum 2 \qnum{\alpha(j)}} \times u_\alpha^{(j-1)} .
\end{align*}
By Equations~\eqref{eq: coblobas vector recursive construction} 
and~\eqref{eq: coblobas vectors} and the %recurrence
recursion~\eqref{eq: recursion for coblobas normalization constants}, the 
asserted property~\eqref{eq: coblo walk projection conditions} follows:
\begin{align*}
\hat{\pi}_j ( \Coblobastwodim_\alpha ) 
= \; & \qnum{2}^N \NormalizationConstant_\alpha \times 
\hat{\pi}_j ( u_{\alpha}^{(2N)} ) \\
= \; & \qnum{2}^N \frac{\qnum{\alpha(j)}}{\qnum{\alpha(j)+1}} \times
\NormalizationConstant_{\alpha\removeupwedge{j}} \times 
\frac{\qnum{\alpha(j)+1}}{\qnum 2 \qnum{\alpha(j)}}
\times u_{\alpha \removeupwedge{j}}^{(2N-2)} \\
= \; & \qnum{2}^{N-1} \NormalizationConstant_{\alpha\removeupwedge{j}} \times
u_{\alpha \removeupwedge{j}}^{(2N-2)} \\
= \; & \Coblobastwodim_{\alpha \removeupwedge{j}}.
\end{align*}

Finally, suppose that $\alpha$ contains a down-wedge 
at $j$, i.e., $\downwedgeat{j} \in \alpha$.
We need to show that in this case, we have 
$\hat{\pi}_j(\Coblobastwodim_\alpha) =
-\frac{\qnum{\alpha(j)+1}}{\qnum{\alpha(j)+2}} \times \Coblobastwodim_{\alpha\removedownwedge{j}}$. 
Now $\alpha(j) = \alpha(j-1)-1$ and $\alpha(j+1) = \alpha(j)+1$
and $u_\alpha^{(j+1)}$ reads
\begin{align*}
u_\alpha^{(j+1)} 
= \; & \frac{1}{q-q^{-1}} \left( 
\Wbas_0 \tens (\Wbas_1 \tens u_\alpha^{(j-1)})
- \frac{q^{\alpha(j-1)}}{\qnum{\alpha(j-1)}}\,\Wbas_0 \tens (\Wbas_0 \tens F.u_\alpha^{(j-1)}) 
\right).
\end{align*}
Applying the projection $\hat{\pi}_j$ on both sides and using
Lemma~\ref{lem: projection formulas}(a), we obtain
\begin{align*}
\hat{\pi}_j ( u_\alpha^{(j+1)} )
= \; & \frac{1}{q-q^{-1}} \left( 
\hat{\pi} (\Wbas_0 \tens \Wbas_1) \right) \tens u_\alpha^{(j-1)} 
= \frac{1}{q-q^{-1}} \frac{q^{-1}-q}{\qnum 2} \times u_\alpha^{(j-1)}
= - \frac{1}{\qnum 2} \times u_\alpha^{(j-1)},
\end{align*}
and again, by Equations~\eqref{eq: coblobas vector recursive construction} 
and~\eqref{eq: coblobas vectors} 
and the %recurrence
recursion~\eqref{eq: recursion for coblobas normalization constants}, the asserted
property~\eqref{eq: coblo walk projection conditions} follows:
\begin{align*}
\hat{\pi}_j ( \Coblobastwodim_\alpha ) 
= \; & \qnum{2}^N \NormalizationConstant_\alpha \times 
\hat{\pi}_j ( u_{\alpha}^{(2N)} ) \\
= \; & \qnum{2}^N \frac{\qnum{\alpha(j)+1}}{\qnum{\alpha(j)+2}}  \times
\NormalizationConstant_{\alpha\removedownwedge{j}} \times 
- \frac{1}{\qnum 2} 
\times u_{\alpha \removedownwedge{j}}^{(2N-2)} \\
= \; & - \frac{\qnum{\alpha(j)+1}}{\qnum{\alpha(j)+2}} 
\qnum{2}^{N-1} \NormalizationConstant_{\alpha\removedownwedge{j}} \times 
u_{\alpha \removedownwedge{j}}^{(2N-2)} \\
= \; & - \frac{\qnum{\alpha(j)+1}}{\qnum{\alpha(j)+2}} 
\times \Coblobastwodim_{\alpha \removedownwedge{j}}.
\end{align*}
This concludes the proof.
\end{proof}

% % % % % % % \blue{
% % % % % % % \subsubsection{\textbf{Basis property}}
% % % % % % % 
% % % % % % % \alexmod{
% % % % % % % I would prefer the following proof in this chapter: it is obvious from the construction that the vectors $u_\alpha^{(k)}$
% % % % % % % corresponding to the different possible Dyck path initial segments $(\alpha(0), \ldots, \alpha(k))$ are linearly independent, 
% % % % % % % for all $k$. There are $C_N = dim \HWspCal_N$ initial segments of $k=2N$, so the vectors must be a basis.
% % % % % % % }
% % % % % % % 
% % % % % % % To prove that the conformal block vectors $\Coblobastwodim_\alpha$ for $\alpha \in \DP_N$
% % % % % % % form a basis of the space %$\HWsp_{2N}^{(0)}$, 
% % % % % % % $\red{\HWspCal_N}$,
% % % % % % % we use a similar calculation as in the proof of Theorem~\ref{thm: change of basis theorem},
% % % % % % % where $\Coblobastwodim_\alpha$ play the role of the conformal block functions $\ConfBlockFun_\alpha$,
% % % % % % % and the pure partition functions $\PartF_\alpha$ are replaced by their inverse images under the mapping 
% % % % % % % %$\sF \colon \HWsp_{2N}^{(0)} \to \Sol_N$ 
% % % % % % % $\sF \colon 
% % % % % % % \red{
% % % % % % % \HWspCal_N
% % % % % % % } 
% % % % % % % \to \Sol_N$ 
% % % % % % % of proposition~\ref{prop: SCCG correspondence map}:
% % % % % % % \begin{align*}
% % % % % % % \Puregeomtwodim_\alpha = B^N \times \sF^{-1}[\PartF_\alpha] , \qquad \text{for $\alpha \in \DP_N$.}
% % % % % % % \end{align*}
% % % % % % % 
% % % % % % % \begin{lem} \label{lem: basis}
% % % % % % % For any $N \in \bZnn$, the collection $( \Coblobastwodim_\alpha )_{\alpha \in \DP_N}$ 
% % % % % % % of vectors defined in~\eqref{eq: coblobas vectors} is a basis of %$\HWsp_{2N}^{(0)}$.
% % % % % % % $\red{\HWspCal_N}$.
% % % % % % % \end{lem}
% % % % % % % \begin{proof}
% % % % % % % %Because $\dmn \HWsp_{2N}^{(0)} = \Catalan_N = \# \DP_N$, it suffices to prove that the vectors 
% % % % % % % %$\Coblobastwodim_\alpha$ are linearly independent.
% % % % % % % Combining Propositions~\ref{prop: solution space with power law bound} and~\ref{prop: SCCG correspondence map},
% % % % % % % we see that $(\Puregeomtwodim_\alpha)_{\alpha \in \DP_N}$ is a basis for %$\HWsp_{2N}^{(0)}$, 
% % % % % % % $\red{\HWspCal_N}$, 
% % % % % % % uniquely determined by the projection properties~\cite[Theorem~3.1]{KP-pure_partition_functions_of_multiple_SLEs}
% % % % % % % \begin{align}
% % % % % % % %& K.\Puregeomtwodim_\alpha = \Puregeomtwodim_\alpha
% % % % % % % %\label{eq: multiple sle cartan eigenvalue}\\
% % % % % % % %& E.\Puregeomtwodim_\alpha = 0
% % % % % % % %\label{eq: multiple sle highest weight vector}\\
% % % % % % % & \hat{\pi}_j(\Puregeomtwodim_\alpha) =
% % % % % % % \label{eq: multiple sle projection conditions}\begin{cases}
% % % % % % %     0 & \mbox{if } \upwedgeat{j} \notin\alpha \\
% % % % % % %     \Puregeomtwodim_{\alpha \removeupwedge{j}} & \mbox{if } \upwedgeat{j}\in\alpha
% % % % % % % \end{cases}
% % % % % % % \qquad\text{for all } j \in \set{1,\ldots,2N-1}.
% % % % % % % \end{align}
% % % % % % % Thanks to the isomorphism $\sF$, by an analogous calculation as in the proof of Theorem~\ref{thm: change of basis theorem}, we see that 
% % % % % % % \begin{align*}
% % % % % % % \Coblobastwodim_\alpha =
% % % % % % % \sum_{\beta \in \DP_N} \genMmat_{\alpha,\beta}\;\Puregeomtwodim_\beta ,
% % % % % % % \end{align*}
% % % % % % % where $\genMmat$ is the weighted incidence matrix~\eqref{eq: def of weighted incidence matrix}
% % % % % % % with weights~\eqref{eq: weights in terms of q-numbers}, also appearing in Theorem~\ref{thm: change of basis theorem}.
% % % % % % % By Proposition~\ref{prop: weighted KW incidence matrix inversion}, this matrix is invertible,
% % % % % % % so $( \Coblobastwodim_\alpha )_{\alpha \in \DP_N}$ is a basis for %$\HWsp_{2N}^{(0)}$.
% % % % % % % $\red{\HWspCal_N}$.
% % % % % % % \end{proof}
% % % % % % % }




\subsubsection{\textbf{Proof of Proposition~\ref{prop: conformal block vectors}}}

%\begin{proof}[Proof of Theorem~\ref{thm: conformal block vectors}]
The vectors $(\Coblobastwodim_\alpha)_{\alpha \in \DP}$ constructed
in~\eqref{eq: coblobas vector recursive construction} and~\eqref{eq: coblobas vectors}
lie in the space $\HWsp_{2N}^{(0)}$ by Corollary~\ref{cor: coblo walk hwv properties}
and satisfy the projection properties by Proposition~\ref{prop: coblo walk projection properties}.
Such a collection is unique by Corollary~\ref{cor: uniqueness}.
In Section~\ref{sss: linear independence} we verified that
$(\Coblobastwodim_\alpha)_{\alpha \in \DP_N}$ forms a basis of $\HWsp_{2N}^{(0)}$.
% $\red{\HWspCal_N}$.}
%$\Coblobastwodim_\alpha$, for all $\alpha \in \DP$.
%, which
%by Corollary~\ref{cor: coblo walk hwv properties} 
%and Proposition~\ref{prop: coblo walk projection properties}
%satisfy ???.
% % Finally, for fixed $N$, the vectors 
% % $( \Coblobastwodim_\alpha )_{\alpha \in \DP_N}$ 
% % form a basis of the trivial subrepresentation $\HWsp_{2N}^{(0)}$,
% % since the matrix $\genMmat = (\genMmat_{\alpha,\beta})$ 
% % gives the change of basis between 
% % $( \Coblobastwodim_\alpha )_{\alpha \in \DP_N}$ and 
% % $(\Puregeomtwodim_\alpha )_{\alpha \in \DP_N}$,
% % {\color{green} by Theorems~\ref{thm: change of basis between conformal blocks and pure partition functions} 
% % and~\ref{thm: pure partition vectors for generic kappa} }
% % {\color{blue} by Theorem~\ref{thm: change of basis between conformal blocks and pure partition functions} 
% % and Lemma~\ref{thm: pure partition vectors for generic kappa} } .
%The linear independence of the vectors $\Coblobastwodim_\alpha$ 
%follows from the formula 
%\begin{align*}
%\Coblobastwodim_\alpha =
%\sum_{\beta \in \DP_N} \genMmat_{\alpha,\beta}\;\Puregeomtwodim_\beta
%\end{align*}
%and the facts that, first, 
%$\left( \Puregeomtwodim_\beta \right)_{\beta \in \DP_N}$ 
%is a basis of $\HWsp_{2N}^{(0)}$
%(by Theorem~\ref{thm: pure partition vectors for generic kappa}),
%and, second, the matrix $\genMmat = (\genMmat_{\alpha,\beta})$ is invertible 
%(by Lemma~\ref{lem: recursion of change of basis matrix with generic q}).
%This proves Proposition~\ref{prop: conformal block vectors}. 
$\hfill \qed$
%\end{proof}




\bibliographystyle{annotate}
%\bibliography{/u/01/kytolak1/unix/Dropbox/Files/Databases/Library_database-Kalle,/home/kpjkytol/Dropbox/Files/Databases/Library_database-Kalle}

\newcommand{\etalchar}[1]{$^{#1}$}
%\def\cprime{$'$}

\begin{thebibliography}{CDCH{\etalchar{+}}13}

\bibitem[AS64]{AS-handbook}
M.~Abramowitz and I.~A.~Stegun (eds.).
\newblock {\em Handbook of mathematical functions}.
\newblock Dover Publications Inc., 1964.

% \bibitem[AKL12]{AKL-the_Greens_function_for_radial_SLE}
% T.~Alberts, M.~J.~Kozdron, and G.~F.~Lawler.
% \newblock The Green's function for the radial Schramm-Loewner evolution.
% \newblock {\em J.~Phys. A: Math. Theor.}, 45:494015, 2012.

% \bibitem[AS08]{AS-Hausdorff_dimension_of_SLE_real_line}
% T.~Alberts and S.~Sheffield.
% \newblock {H}ausdorff dimension of the {SLE} curve intersected with the real
%   line.
% \newblock {\em Electron. J.~Probab.}, 13(40):1166--1188, 2008.

% \bibitem[AS09]{AS-covariant_measure_of_SLE_on_boundary}
% T.~Alberts and S.~Sheffield.
% \newblock The covariant measure of $\SLE$ on the boundary.
% \newblock {\em Probab. Theory Related Fields}, 149(3-4):331--371, 2011.

% \bibitem[BB03]{BB-SLE_CFT_and_zigzag_probabilities}
% M.~Bauer and D.~Bernard.
% \newblock $\SLE$, CFT and zig-zag probabilities.
% \newblock {\em Proceedings of the conference `Conformal Invariance and Random Spatial Processes', Edinburgh}, 2003.

\bibitem[BB03a]{BB-SLE_martingales_and_Virasoro_algebra}
M.~Bauer and D.~Bernard.
\newblock $\SLE$ martingales and the Virasoro algebra.
\newblock {\em Phys. Lett.~B}, 557(3-4):309--316, 2003.

\bibitem[BB03b]{BB-CFTs_of_SLEs}
M.~Bauer and D.~Bernard.
\newblock Conformal field theories of stochastic Loewner evolutions.
\newblock {\em Comm. Math. Phys.}, 239(3):493--521, 2003.

% \bibitem[BB04]{BB-conformal_transformations_and_SLE_partition_function_martingale}
% M.~Bauer and D.~Bernard.
% \newblock Conformal transformations and the $\SLE$ partition function mar-
% tingale.
% \newblock {\em Ann. Henri Poincaré}, 5(2):289--326, 2004.

\bibitem[BBK05]{BBK-multiple_SLEs}
M.~Bauer, D.~Bernard, and K.~Kyt{\"o}l{\"a}.
\newblock Multiple {S}chramm-{L}oewner evolutions and statistical mechanics
  martingales.
\newblock {\em J.~Stat. Phys.}, 120(5-6):1125--1163, 2005.

\bibitem[BPW18]{Beffara_Peltola_Wu-Uniqueness_of_global_multiple_SLEs}
Vincent Beffara, Eveliina Peltola, and Hao Wu. 
\newblock On the uniqueness of global multiple $\SLE$s.
\newblock Preprint, \url{http://arxiv.org/abs/1801.07699}, 2018.

\bibitem[BPZ84]{BPZ-infinite_conformal_symmetry_in_2D_QFT}
A.~A.~Belavin, A.~M.~Polyakov, and A.~B.~Zamolodchikov.
\newblock Infinite conformal symmetry in two-dimensional quantum field theory.
\newblock {\em Nucl. Phys.~B}, 241(2):333--380, 1984.

%\bibitem[BPZ84b]{BPZ-infinite_conformal_symmetry_of_critical_fluctiations}
%A.~A.~Belavin, A.~M.~Polyakov, and A.~B.~Zamolodchikov.
%\newblock Infinite conformal symmetry of critical fluctuations in two dimensions.
%\newblock {\em J.~Stat. Phys.}, 34(5-6):763--774, 1984.

%\bibitem[BLV16]{BLV-scaling_limit_of_LERW_Greens_function}
%C.~Bene\v{s}, G.~F.~Lawler, and F.~Viklund.
%\newblock Scaling limit of the loop-erased random walk Green's function.
%\newblock {\em Probab. Theory Related Fields}, 166(1):271--319, 2016.

%\bibitem[BSA88]{BSA-degenerate_CFTs_and_explicit_expressions}
%L.~Benoit and Y.~Saint-Aubin.
%\newblock Degenerate conformal field theories and explicit expressions for some
%  null vectors.
%\newblock {\em Phys. Lett.}, B215(3):517--522, 1988.

%\bibitem[BMP90]{BMP-quantum_group_structure_in_the_Fock_space_resolutions_of_s
%lnhat_representations}
%P.~Bouwknegt, J.~McCarthy, and K.~Pilch.
%\newblock Quantum group structure in the Fock space resolutions of
%  {$\hat{sl}(n)$} representations.
%\newblock {\em Comm. Math. Phys.}, 131(1):125--155, 1990.

%\bibitem[BFGT09]{BFGT-Lusztig_limit_at_root_of_unity_and_fusion}
%P.~V.~Bushlanov, B.L.~Feigin, A.M.~Gainutdinov, and I.Yu.~Tipunin.
%\newblock Lusztig limit of quantum sl(2) at root of unity and fusion of (1,p)
%  {V}irasoro logarithmic minimal models.
%\newblock {\em Nucl. Phys.~B}, 818(3):179--195, 2009.

%\bibitem[CS11]{CS-proof_of_RS_conjecture}
%L.~Cantini and A.~Sportiello.
%\newblock Proof of the Razumov-Stroganov conjecture.
%\newblock {\em J.~Combin. Th., Ser.~A}, 118(5):1549--1574, 2011.

%\bibitem[Car88]{Cardy-conformal_invariance_and_statistical_mechanics}
%J.~Cardy
%\newblock Conformal invariance and statistical mechanics.
%\newblock In {\em Fields, Strings and Critical Phenomena (Les Houches 1988)},
%  Eds. E.~Br\'ezin and J.~Zinn-Justin.
%  Elsevier Science Publishers BV, 1988.

%\bibitem[CP94]{Chari_Pressley-Quantum_groups}
%V.~Chari and A.~Pressley.
%\newblock A guide to quantum groups.
%\newblock Cambridge University Press, 1994.

%\bibitem[CHI15]{CHI-conformal_invariance_of_spin_correlations_in_the_planar_Ising_model}
%D.~Chelkak, C.~Hongler, and K.~Izyurov.
%\newblock Conformal invariance of spin correlations in the planar Ising model.
%\newblock {\em Ann. Math.}, 181(3):1087--1138, 2015.

%\bibitem[CI13]{CI-holomorphic_spinor_observables_in_the_critical_Ising_model}
%D.~Chelkak and K.~Izyurov.
%\newblock Holomorphic spinor observables in the critical Ising model.
%\newblock {\em Comm. Math. Phys.}, 322(2):303--332, 2013.

%\bibitem[CDCH{\etalchar{+}}14]{CDHKS-convergence_of_Ising_interfaces_to_SLE}
%D.~Chelkak, H.~Duminil-Copin, C.~Hongler, A.~Kemppainen, and S.~Smirnov.
%\newblock Convergence of {I}sing interfaces to {SLE}.
%\newblock {\em C.~R. Acad. Sci. Paris Ser.~I}, 352(2):157--161, 2014.

%\bibitem[CS11]{CS-discrete_complex_analysis_on_isoradial}
%D.~Chelkak and S.~Smirnov.
%\newblock Discrete complex analysis on isoradial graphs.
%\newblock {\em Adv. Math.}, 228(3):1590--1630, 2011.

\bibitem[CS12]{CS-universality_in_2d_Ising}
D.~Chelkak and S.~Smirnov.
\newblock Universality in the 2D Ising model and conformal invariance of fermionic observables.
\newblock {\em Invent. Math.}, 189(3):515--580, 2012.

%\bibitem[CFL28]{CFL-uber_die_PDE_der_mathphys}
%R.~Courant, K.~Friedrichs, and H.~Lewy.
%\newblock \"Uber die partiellen Differenzengleichungen der mathematischen Physik.
%\newblock {\em Math. Ann.}, 100(1):32--74, 1928.

\bibitem[DMS97]{DMS-CFT}
P.~Di~Francesco, P.~Mathieu, and D.~S\'en\'echal.
\newblock {\em Conformal field theory}.
\newblock Springer Verlag, 1997.

\bibitem[DF84]{DF-multipoint_correlation_functions}
V.~S.~Dotsenko and V.~A.~Fateev.
\newblock {Conformal algebra and multipoint correlation functions in 2D
statistical models}.
\newblock {\em Nucl. Phys.}, B240(3):312--348, 1984.

%\bibitem[Dub06a]{Dubedat-Euler_integrals}
%J.~Dub{\'e}dat.
%\newblock Euler integrals for commuting $\SLE$s.
%\newblock {\em J.~Stat. Phys.}, 123(6):1183--1218, 2006.

%\bibitem[Dub06b]{Dubedat-excursion_decompositions_and_Watts_crossing_formula}
%J.~Dub{\'e}dat.
%\newblock Excursion decompositions for {SLE} and Watts' crossing formula.
%\newblock {\em Probab. Theory Rel. Fields}, 134(3):453--488, 2006.

\bibitem[Dub07]{Dubedat-commutation}
J.~Dub{\'e}dat.
\newblock Commutation relations for $\SLE$.
\newblock {\em Comm. Pure Appl. Math.}, 60(12):1792--1847, 2007.

\bibitem[Dub15]{Dubedat-SLE_and_Virasoro_representations_localization}
J.~Dub\'edat.
\newblock $\SLE$ and Virasoro representations: Localization.
\newblock {\em Comm. Math. Phys.}, 336(2):695--760, 2015.

%\bibitem[Dub15b]{Dubedat-SLE_and_Virasoro_representations_fusion}
%J.~Dub\'edat.
%\newblock $\SLE$ and Virasoro representations: Fusion.
%\newblock {\em Comm. Math. Phys.}, 336(2):761--809, 2015.

\bibitem[FF90]{FF-representations}
B.~L.~Fe{\u\i}gin and D.~B.~Fuchs.
\newblock Representations of the {V}irasoro algebra.
\newblock In {\em Representation of Lie groups and related topics}, volume~7 of
  {\em Adv. Stud. Contemp. Math.}, pages 465--554. Gordon and Breach, New York,
  1990.

\bibitem[Fel89]{Felder-BRST_approach}
G.~Felder.
\newblock B{RST} approach to minimal models.
\newblock {\em Nucl. Phys.~B}, 317(1):215--236, 1989.
\newblock Erratum ibid. B, 324(2):548, 1989.

%\bibitem[FFK89]{FFK-braid_matrices}
%G.~Felder, J.~Fr{\"o}hlich, and G.~Keller.
%\newblock Braid matrices and structure constants for minimal conformal models.
%\newblock {\em Comm. Math. Phys.}, 124(4):647--664, 1989.

%\bibitem[FW91]{FW-topological_representation_of_Uqsl2}
%G.~Felder and C.~Wieczerkowski.
%\newblock Topological representation of the quantum group $\Uqsltwo$. % {U}q(sl2).
%\newblock {\em Comm. Math. Phys.}, 138(3):583--605, 1991.

\bibitem[FK15a]{FK-solution_space_for_a_system_of_null_state_PDEs_1}
S.~M.~Flores and P.~Kleban.
\newblock A solution space for a system of null-state partial differential
  equations, Part I.
\newblock {\em Comm. Math. Phys.}, 333(1):389--434, 2015.

\bibitem[FK15b]{FK-solution_space_for_a_system_of_null_state_PDEs_2}
S.~M.~Flores and P.~Kleban.
\newblock A solution space for a system of null-state partial differential
  equations, Part II.
\newblock {\em Comm. Math. Phys.}, 333(1):435--481, 2015.

\bibitem[FK15c]{FK-solution_space_for_a_system_of_null_state_PDEs_3}
S.~M.~Flores and P.~Kleban.
\newblock A solution space for a system of null-state partial differential
  equations, Part III.
\newblock {\em Comm. Math. Phys.}, 333(2):597--667, 2015.

% \bibitem[FK15d]{FK-solution_space_for_a_system_of_null_state_PDEs_4}
% S.~M.~Flores and P.~Kleban.
% \newblock A solution space for a system of null-state partial differential
%   equations, Part IV.
% \newblock {\em Comm. Math. Phys.}, 333(2):669--715, 2015.

\bibitem[FP18a]{Flores-Peltola:WJTL_algebra}
S.~M.~Flores and E.~Peltola.
\newblock Standard modules, Jones-Wenzl projectors, and the valenced Temperley-Lieb algebra.
\newblock Preprint, \url{http://arxiv.org/abs/1801.10003}, 2018.

\bibitem[FP18b{\etalchar{+}}]{Flores-Peltola:Colored_braid_representations_and_QSW}
S.~M.~Flores and E.~Peltola.
\newblock Higher quantum and classical {S}chur-{W}eyl duality for {$\mathfrak{sl}(2)$}.
\newblock In preparation, 2018.

\bibitem[FP18c{\etalchar{+}}]{Flores-Peltola:Monodromy_invariant_correlation_function}
S.~M.~Flores and E.~Peltola.
\newblock Monodromy invariant {CFT} correlation functions of first column {K}ac operators.
\newblock In preparation, 2018.

%\bibitem[FP18d]{FP-monodromy_invariant_correlations}
%S.~M.~Flores and E.~Peltola.
%\newblock Monodromy invariant CFT correlation functions of first column Kac operators.
%\newblock In preparation, 2018.

\bibitem[Fom01]{Fomin-LERW_and_total_positivity}
S.~Fomin.
\newblock Loop-erased walks and total positivity.
\newblock {\em Trans. Amer. Math. Soc.}, 353(9):3363--3583, 2001.

\bibitem[FW03]{FW-conformal_restriction_highest_weight_representations_and_SLE}
R.~Friedrich and W.~Werner.
\newblock Conformal restriction, highest-weight representations and SLE.
\newblock {\em Comm. Math. Phys.}, 243(1):105--122, 2003.

%\bibitem[FK04]{Friedrich-Kalkkinen:On_CFT_and_SLE}
%Roland Friedrich and Jussi Kalkkinen.
%\newblock On conformal field theory and stochastic {L}oewner evolution.
%\newblock {\em Nuclear Phys.~B}, 687(3):279--302, 2004.
 
% \bibitem[Fuc92]{Fuchs-affine_Lie_algebras_and_quantum_groups}
% J.~Fuchs.
% \newblock {\em Affine Lie algebras and quantum groups}.
% \newblock Cambridge Monographs on Mathematical Physics. Cambridge Univ. Press,
%   1992.

%\bibitem[Fuc92]{Fuchs-Quantum_groups}
%J.~Fuchs. 
%\newblock Affine Lie algebras and quantum groups.
%\newblock Cambridge Monographs on Mathematical Physics, Cambridge University Press, 1992. 

%\bibitem[GV85]{GV-binomial_determinants_paths_and_hook_length_formulae}
%I.~Gessel and G.~Viennot.
%\newblock Binomial determinants, paths, and hook length formulae.
%\newblock {\em Adv. Math.}, 58(3):300--321, 1985.

% \bibitem[GRAS96]{GRS-quantum_groups_in_2d_physics}
% C.~G{\'o}mez, M.~Ruiz-Altaba, and G.~Sierra.
% \newblock {\em Quantum groups in two-dimensional physics}.
% \newblock Cambridge Univ. Press, 1996.

% \bibitem[GS90]{GS-quantum_group_meaning_of_the_Coulomb_gas}
% C.~G{\'o}mez and G.~Sierra.
% \newblock Quantum group meaning of the {C}oulomb gas.
% {\em Phys. Lett.}, 240B(1-2):149--157, 1990.

% \bibitem[Gra07]{Graham-multiple_SLEs}
% K.~Graham.
% \newblock On multiple Schramm-Loewner evolutions.
% \newblock {\em J.~Stat. Mech.: Theory and Exp.}, P03008, 2007.

%\bibitem[Hon10]{Hongler-conformal_invariance_of_Ising_model_correlations}
%C.~Hongler.
%\newblock {\em Conformal invariance of Ising model correlations.}
%\newblock Ph.D. thesis, Universit\'e de Gen\`eve, 2010.

%\bibitem[HS13]{HS-energy_density}
%C.~Hongler and S.~Smirnov.
%\newblock The energy density in the 2d Ising model.
%\newblock {\em Acta Math.}, 211(2):191--225, 2013.

%\bibitem[HK13]{HK-Ising_interfaces_and_free_boundary_conditions}
%C.~Hongler, and K.~Kyt{\"o}l{\"a}.
%Ising interfaces and free boundary conditions
%{\em J.~Amer. Math. Soc.}, 26(4):1107--1189, 2013.

\bibitem[IK11]{IK-representation_theory_of_the_Virasoro_algebra}
K.~Iohara and Y.~Koga.
\newblock {\em Representation theory of the Virasoro algebra}.
\newblock Springer Monographs in Mathematics. Springer, 2011.
 
%\bibitem[Izy11]{Izyurov-PhD_thesis}
%K.~Izyurov.
%\newblock Holomorphic spinor observables and interfaces in the critical Ising model.
%\newblock {\em Ph.D. thesis}, Universit\'e de Gen\`eve, 2011.

%\bibitem[Izy15]{Izyurov-Smirnovs_observable_for_free_boundary_conditions}
%K.~Izyurov.
%\newblock Smirnov's observable for free boundary conditions, interfaces and crossing probabilities.
%\newblock {\em Comm. Math. Phys.}, 337(1):225--252, 2015.

\bibitem[Izy17]{Izyurov-critical_Ising_interfaces_in_multiply_connected_domains}
K.~Izyurov.
\newblock Critical {I}sing interfaces in multiply-connected domains
\newblock {\em Probab. Theory Related Fields}, 167(1-2):379--415, 2017.

%\bibitem[JJK16]{JJK-SLE_boundary_visits}
%N.~Jokela, M.~J{\"a}rvinen, and K.~Kyt{\"o}l{\"a}.
%\newblock $\SLE$ boundary visits.
%\newblock {\em Ann. Henri Poincar\'e}, 17(6):1263--1330, 2016.

\bibitem[Kac78]{Kac-ICM_proceedings_Helsinki}
V.~Kac.
\newblock Highest weight representations of infinite dimensional Lie algebras.
\newblock {\em Proceedings of ICM, Helsinki 1978}, pages 299--304, 1980.

%\bibitem[KN04]{KN-guide_to_SLE}
%W.~Kager and B.~Nienhuis.
%\newblock A guide to stochastic L\"owner evolution and its applications.
%\newblock {\em J.~Stat. Phys.}, 115(5):1149--1229, 2004.

\bibitem[KM13]{KM-GFF_and_CFT}
N.-G.~Kang and N.~Makarov, 
\newblock Gaussian free field and conformal field theory.
\newblock {\em Ast\'erisque}, 353, 2013.

%\bibitem[KM59]{KM-coincidence_probabilities}
%S.~Karlin and J.~McGregor.
%\newblock Coincidence probabilities.
%\newblock {\em Pacific J.~Math.}, 9(4):1141--1164, 1959.

\bibitem[KKP17]{KKP-boundary_correlations_in_planar_LERW_and_UST}
A.~Karrila, K.~Kyt{\"o}l{\"a}, and E.~Peltola.
\newblock Boundary correlations in planar LERW and UST.
\newblock Preprint, \url{http://arxiv.org/abs/1702.03261}, 2017.

\bibitem[Kas95]{Kassel-Quantum_groups}
C.~Kassel.
\newblock Quantum groups.
\newblock Springer-Verlag, New York, 1995.

\bibitem[KS18]{KS-configurations_of_FK_interfaces}
A.~Kemppainen and S.~Smirnov.
\newblock Configurations of FK Ising interfaces and hypergeometric SLE.
\newblock {\em Math. Res. Lett.}, 25(3):875--889, 2018.

%\bibitem[Ken00]{Kenyon-conformal_invariance_of_the_domino_tiling}
%R.~Kenyon.
%\newblock Conformal invariance of domino tiling.
%\newblock {\em Ann. Probab.}, 28(2):759--795, 2000.

%\bibitem[Ken00]{Kenyon-the_asymptotic_determinant_of_the_discrete_Laplacian}
%R.~Kenyon.
%\newblock The asymptotic determinant of the discrete Laplacian.
%\newblock {\em Acta Math.}, 185(2):239--286, 2000.

\bibitem[KW11a]{KW-boundary_partitions_in_trees_and_dimers}
R.~W.~Kenyon and D.~B.~Wilson.
\newblock Boundary partitions in trees and dimers.
\newblock {\em Trans. Amer. Math. Soc.}, 363(3):1325--1364, 2011.

\bibitem[KW11b]{KW-double_dimer_pairings_and_skew_Young_diagrams}
R.~W.~Kenyon and D.~B.~Wilson.
\newblock Double-dimer pairings and skew Young diagrams.
\newblock {\em Electr. J.~Combinatorics}, 18(1):130--142, 2011.

%\bibitem[KW15]{KW-spanning_trees_of_graphs_on_surfaces_and_the_intensity_of_LERW}
%R.~W.~Kenyon and D.~B.~Wilson.
%\newblock Spanning trees of graphs on surfaces and the intensity of loop-erased random walk on planar graphs.
%\newblock {\em J.~Amer. Math. Soc.}, 28(4):985--1030, 2015.

%\bibitem[Kim12]{Kim-proofs_of_two_conjectures}
%J.~S.~Kim.
%\newblock Proofs of two conjectures of Kenyon and Wilson on Dyck tilings.
%\newblock {\em J.~Combin. Theory Ser.~A}, 119(8):1692--1710, 2012.

%\bibitem[KMPW14]{KMPW-Dyck_tilings_increasing_trees_and_inversions}
%J.~S.~Kim, K.~M\'esz\'aros, G.~Panova, D.~B.~Wilson.
%\newblock Dyck tilings, increasing trees, descents, and inversions.
%\newblock {\em J.~Combin. Theory Ser.~A}, 122(C):9--27, 2014.

%\bibitem[Kon03]{Kontsevich:CFT_SLE_and_phase_boundaries}
%Maxim Kontsevich.
%\newblock CFT, $\SLE$, and phase boundaries.
%\newblock {\em Oberwolfach Arbeitstagung}, 2003.

\bibitem[KL07]{LK-configurational_measure}
M.~J.~Kozdron and G.~F.~Lawler.
\newblock The configurational measure on mutually avoiding $\SLE$ paths.
\newblock In {\em Universality and Renormalization: From Stochastic Evolution
  to Renormalization of Quantum Fields}, Fields Inst. Commun. Amer.
  Math. Soc., 2007.

\bibitem[Kyt07]{Kytola-local_mgales}
K.~Kyt{\"o}l{\"a}.
\newblock Virasoro module structure of local martingales of $\SLE$ variants.
\newblock {\em Rev. Math. Phys.}, 19(5):455--509, 2007.

\bibitem[KP16]{KP-pure_partition_functions_of_multiple_SLEs}
K.~Kyt{\"o}l{\"a} and E.~Peltola.
\newblock {Pure partition functions of multiple SLEs.}
\newblock {\em Comm. Math. Phys.}, 346(1):237--292, 2016.

\bibitem[KP18]{KP-conformally_covariant_boundary_correlation_functions_with_a_quantum_group}
K.~Kyt{\"o}l{\"a} and E.~Peltola.
\newblock Conformally covariant boundary correlation functions with a quantum group.
\newblock {\em J.~Eur. Math. Soc.}, to appear, 2018.
%\newblock Preprint, \url{http://arxiv.org/abs/1408.1384}, 2014.

% \bibitem[Kyt07]{Kytola-local_mgales}
% K.~Kyt{\"o}l{\"a}.
% \newblock Virasoro module structure of local martingales of {SLE} variants.
% \newblock {\em Rev. Math. Phys.}, 19(5):455--509, 2007.

%\bibitem[Law05]{Lawler-conformally_invariant_processes_in_the_plane}
%G.~F.~Lawler.
%\newblock {\em Conformally invariant processes in the plane.}
%\newblock American Mathematical Society, 2005.

%\bibitem[Law91]{Lawler-intersections_of_random_walks}
%G.~F.~Lawler.
%\newblock {\em Intersections of random walks.}
%\newblock Birkh\"auser, 1991.

%\bibitem[Law14]{Lawler-the_probability_that_planar_LERW_uses_a_given_edge}
%G.~F.~Lawler.
%\newblock The probability that planar loop-erased random walk uses a given edge.
%\newblock {\em Electron. Commun. Probab.}, 19:1--13, 2014.

% \bibitem[Law14]{Lawler-Minkowski_SLE_real_line}
% G.~F. Lawler.
% \newblock Minkowski content of the intersection of a {S}chramm-{L}oewner
%     Evolution ($\SLE$) curve with the real line.
% \newblock {\em J.~Math. Soc. Japan}, 67(4):1631--1669, 2015.

% \bibitem[LR15]{LR-Minkowski_content_and_natural_parametrization_for_SLE}
% G.~F.~Lawler and M.~A.~Rezaei.
% \newblock Minkowski content and natural parameterization for the Schramm-Loewner evolution.
% \newblock {\em Ann. Probab.}, 43(3):1082--1120, 2015.
 
% \bibitem[LS11]{LS-natural_parametrization_of_SLE}
% G.~F. Lawler and S.~Sheffield.
% \newblock A natural parametrization for the {S}chramm-{L}oewner evolution.
% \newblock {\em Ann. Probab.}, 39(5):1896--1937, 2011.

%\bibitem[LS11]{LS-natural_parametrization_for_SLE}
%G.~F.~Lawler and S.~Sheffield.
%\newblock A natural parametrization for the Schramm-Loewner evolution.
%\newblock {\em Ann. Probab.}, 39(5):1896--1937, 2011.

%\bibitem[LSW04]{LSW-LERW_and_UST}
%G.~F.~Lawler, O.~Schramm, and W.~Werner.
%\newblock Conformal invariance of planar loop-erased random walks and uniform
%  spanning trees.
%\newblock {\em Ann. Probab.}, 32(1B):939--995, 2004.

%\bibitem[LV16a]{LV-natural_parametrization_for_SLE}
%G.~F.~Lawler and F.~Viklund.
%\newblock Convergence of loop-erased random walk in the natural parametrization.
%\newblock Preprint, \url{http://arxiv.org/abs/1603.05203}, 2016.

%\bibitem[LW13]{LW-multi_point_Greens_functions_for_SLE}
%G.~F.~Lawler and B.~M.~Werness.
%\newblock Multi-point Green's functions for $\SLE$ and an estimate of Beffara.
%\newblock {\em Ann. Probab.}, 41(3A):1513--1555, 2013.

%\bibitem[LZ13]{LZ-SLE_curves_and_natural_parametrization}
%G.~F.~Lawler and W.~Zhou.
%\newblock $\SLE$ curves and natural parametrization.
%\newblock {\em Ann. Probab.}, 41(3A):1556--1584, 2013.

%\bibitem[LV17]{LV-Coulomb_gas_for_commuting_SLEs}
%J.~Lenells and F.~Viklund.
%\newblock Coulomb gas integrals for commuting SLEs: Schramm's formula and Green's function.
%\newblock Preprint, \url{http://arxiv.org/abs/1701.03698}, 2017.

\bibitem[LL04]{Lepowsky_Li-VOA}
J.~Lepowsky and H.~Li.
\newblock Introduction to vertex operator algebras and their representations.
\newblock Birkh\"auser Boston, 2004

%\bibitem[Lin73]{Lindstrom-vector_representations_of_induced_matroids}
%B.~Lindstr\"om. 
%\newblock On the vector representations of induced matroids.
%\newblock {\em  Bull. Lond. Math. Soc.}, 5(1):85--90, 1973.

%\bibitem[MR89]{MR-comment_on_quantum_group_symmetry_in_CFT}
%G.~Moore and N.~Reshetikhin.
%\newblock A comment on quantum group symmetry in conformal field theory.
%\newblock {\em Nucl. Phys.}, B328(3):557--574, 1989.

%\bibitem[PW14]{PW-Pfaffian_formulas_for_spanning_tree_probabilities}
%G.~Panova and D.~B.~Wilson.
%\newblock Pfaffian formulas for spanning tree probabilities.
%\newblock {\em Combin. Probab. Comput.}, 26(1):118--137, 2017.

\bibitem[Pel16]{P-basis_for_solutions_of_BSA_PDEs}
E.~Peltola.
\newblock Basis for solutions of the Benoit \& Saint-Aubin PDEs with particular asymptotic properties.
\newblock {\em Ann. Inst. H.~Poincar\'e~D}, to appear, 2018.
%\newblock Preprint, \url{http://arxiv.org/abs/1605.06053}, 2016.

\bibitem[PW17]{PH-Global_multiple_SLEs_and_pure_partition_functions}
E.~Peltola and H.~Wu. 
\newblock Global and local multiple $\SLE$s for $\kappa \leq 4$ and connection probabilities for level lines of $\mathrm{GFF}$.
\newblock Preprint, \url{http://arxiv.org/abs/1703.00898}, 2017.

%\bibitem[Pem91]{Pemantle-choosing_a_spanning_tree_for_the_integer_lattice_uniformly}
%R.~Pemantle.
%\newblock Choosing a spanning tree for the integer lattice uniformly.
%\newblock {\em Ann. Probab.}, 19(4):1559--1574, 1991.

%\bibitem[RRRA91]{RRR-contour_picture_of_quantum_groups_in_CFT}
%C.~Ramirez, H.~Ruegg, and M.~Ruiz-Altaba.
%\newblock The contour picture of quantum groups in conformal field theories.
%\newblock {\em Nucl. Phys.}, B364(1):195--233, 1991.

\bibitem[Rib14]{Ribault-conformal_field_theory_on_the_plane}
S.~Ribault.
\newblock Conformal field theory on the plane.
\newblock \url{http://arxiv.org/abs/1406.4290}, 2014.

%\bibitem[RS05]{RS-basic_properties_of_SLE}
%S.~Rohde and O.~Schramm.
%\newblock Basic properties of $\SLE$.
%\newblock {\em Ann. Math.}, 161(2):883--924, 2005.

%\bibitem[Sch00]{Schramm-LERW_and_UST}
%O.~Schramm.
%\newblock Scaling limits of loop-erased random walks and uniform spanning
%  trees.
%\newblock {\em Israel J.~Math.}, 118(1):221--288, 2000.

%\bibitem[SW11]{SW-Schramms_proof_of_Watts_formula}
%S.~Sheffield and D.~B.~Wilson.
%\newblock Schramm's proof of Watts' formula.
%\newblock {\em Ann. Probab.}, 39(5):1844--1863, 2011.

\bibitem[SZ12]{SZ-path_representations_of_maximal_paraboloc_KL_polynomials}
K.~Shigechi and P.~Zinn-Justin.
\newblock Path representation of maximal parabolic Kazhdan--Lusztig polynomials.
\newblock {\em J.~Pure Appl. Algebra}, 216(11):2533--2548, 2012.

%\bibitem[Smi01]{Smirnov-critical_percolation}
%S.~Smirnov.
%\newblock Critical percolation in the plane: conformal invariance, {C}ardy's
%  formula, scaling limits.
%\newblock {\em C.~R. Acad. Sci. Paris}, 333(3):239--244, 2001.
%\newblock See also \url{http://arxiv.org/abs/0909.4499}.
%%[arXiv:0909.4499]

%\bibitem[Suz14]{Suzuki-Convergence_of_LERW_on_a_planar_graph_to_a_chordal_SLE(2)}
%H.~Suzuki.
%\newblock Convergence of loop erased random walks on a planar graph to a chordal SLE(2) curve.
%\newblock {\em Kodai Math.~J.}, 37(2):303--329, 2014.

% \bibitem[Var92]{Varchenko-ICM1990-multidimensional_hypergeometric_functions}
% A.~Varchenko.
% \newblock Multidimensional hypergeometric functions in conformal field theory,
%   algebraic {K-theory}, algebraic geometry.
% \newblock In {\em Proceedings of the {International} {Congress} of
%   {Mathematicians}, Kyoto 1990}, 1992.

% \bibitem[Var95]{Varchenko-multidimensional_hypergeometric_functions_and_repres%
% entation_theory_of_Lie_algebras_and_quantum_groups}
% A.~Varchenko.
% \newblock {\em Multidimensional Hypergeometric Functions and Representation
%   Theory of Lie Algebras and Quantum Groups}.
% \newblock Advanced Series in Mathematical Physics, Vol.~21. World Scientific,
%   1995.

%\bibitem[Wil96]{Wilson-generating_random_spanning_trees}
%D.~Wilson.
%\newblock Generating random spanning trees more quickly than the cover time.
%\newblock {\em Proc. 28th Annual ACM Symposium on the Theory of Computing}, pages 296--303, 1996.

\bibitem[Wu17]{Wu-convergence_of_Ising_interfaces_to_hypergeometric_SLE}
H.~Wu.
\newblock Hypergeometric {SLE}: conformal {M}arkov characterization and applications.
\newblock Preprint, \url{http://arxiv.org/abs/1703.02022}, 2017.

%\bibitem[YY11]{YY-Loop-erased_random_walk_and_Poisson_kernel_on_planar_graphs} 
%A.~Yadin and A.~Yehudayoff.
%\newblock Loop-erased random walk and Poisson kernel on planar graphs.
%\newblock {\em Ann. Probab.}, 39(4):1243--1285, 2011. 

%\bibitem[Zha08]{Zhan-scaling_limits_of_planar_LERW}
%D.~Zhan.
%\newblock The scaling limits of planar {LERW} in finitely connected domains.
%\newblock {\em Ann. Probab.}, 36(2):467--529, 2008.

\end{thebibliography}


\end{document}
