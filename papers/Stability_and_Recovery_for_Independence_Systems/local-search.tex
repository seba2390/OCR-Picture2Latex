\section{Local Search Performance}\label{sec:local search}


In this section we discuss {\em local search}~\cite{lenstra2003local} (described in \hyperref[sec:preliminaries]{Section~\ref{sec:preliminaries}}).
Local search often gives better results than Greedy, at the cost of a slower running time --- for example for submodular maximization subject to the intersection of $k$ matroids \cite{lee2009submodular,filmus2012power}, and for $k$-set packing \cite{SviridenkoW13,Cygan13,FurerY14}. For some interesting recent results about local search in \textit{beyond-worst-case} settings and on geometric optimization we refer the reader to~\cite{cohen2016local,cohen2014unreasonable,cohen17}.

Somewhat surprisingly, it was not known (to our knowledge) how local search performs for $p$-systems and $p$-extendible systems. (We recall that the greedy algorithm gives a factor of $1/p$ for maximization of an additive function and $1/(p+1)$ for maximization of a monotone submodular function under these constraints.)
Here, we prove that local search in fact performs worse than Greedy for these constraints. Although it gives a $1/p$-approximation for cardinality maximization under a $p$-system constraint (essentially by definition), it does not give any bounded approximation factor for additive function maximization under a $p$-system, and only a $1/p^2$-approximation under a $p$-extendible system. 


\subsection{Local search fails for $p$-systems}
We construct simple examples where local search will not recover any fraction of the maximum-weight solution for $p$-systems (even if it is arbitrarily stable, $p=2$, and even if we allow large exchange neighborhoods). In particular, consider a ground set $X = A \cup \{e^*\}$ where $|A|=n$. The independent sets of $\I$ are:
\begin{itemize}
\item any subset of $A$, or
\item $e^*$ plus any subset of at most $n/2$ elements of A. 
\end{itemize}

Note that this is a 2-system, because for $S \subseteq X$, any independent subset of $S$ can be extended to an independent set of size at least $\min \{|S|, n/2\}$, and the maximum independent subset of $S$ has size at most $\min \{|S|,n\}$. The weights could be 0 on $A$, and 1 on the special element $e^*$. So the optimum is $w(e^*)$ = 1 (observe that the optimal solution is $c$-stable for arbitrarily large $c$). However, $A$ is a local optimum, unless we are willing to swap out $n/2$ elements, which is not possible for efficient local search.



\subsection{Lower bound for $p$-extendible systems}

Let us consider the following instance. Let $X = A \cup B$ where $A, B$ are disjoint sets. We define $\I \subseteq 2^X$ as follows: $S \in \I$ iff
\begin{itemize}
\item $|S \cap A| + p |S \cap B| \leq |A|$, or
\item $p|S \cap A| + |S \cap B| \leq |B|$.
\end{itemize}

\begin{lemma}
For any $A,B$ disjoint, the above is a $p$-extendible system.
\end{lemma}

\begin{proof}
Let $S \subseteq T$ and $i \in X \setminus T$ be such that $S+i \in \I$ and $T \in \I$. We need to prove that there is $Z \subseteq T \setminus S, |Z| \leq p$ such that $(T \setminus Z) + i \in \I$. We can assume that $|T \setminus S| > p$, because otherwise we can set $Z = T \setminus S$ and obviously $(T \setminus Z) + i = S + i \in \I$.
Assuming $|T \setminus S| > p$, let $Z$ be an arbitrary set of $p$ elements from $T \setminus S$. We consider 2 cases: If $|T \cap A| + p |T \cap B| \leq |A|$, then $|(T \setminus Z) \cap A| + p |(T \setminus Z) \cap B| \leq |A| - p$. Adding the element $i$ can increase the left-hand side by at most $p$, and so $|(T \setminus Z + i) \cap A| + p |(T \setminus Z + i) \cap B| \leq |A|$. Similarly, in the second case, if $p |T \cap A| + |T \cap B| \leq |B|$, then $p |(T \setminus Z) \cap A| + |(T \setminus Z) \cap B| \leq |B| - p$. Adding the element $i$ can increase the left-hand side by at most $p$, and so $p |(T \setminus Z + i) \cap A| + |(T \setminus Z + i) \cap B| \leq |B|$. 
\end{proof}

Now we choose the cardinalities of $A$ and $B$ and the weights of their elements appropriately to get a negative result.

\begin{lemma}
For $\epsilon>0$, let $|A| = n$ and $|B| = (p - \epsilon) n$, and set the weights as $w_a = 1$ for $a \in A$ and $w_b = p - \epsilon$ for $b \in B$.
Then $A$ is a local optimum of value $w(A) = w(B) / (p - \epsilon)^2$, unless the local search explores exchanges of size at least $\frac{\epsilon}{p} n$.
\end{lemma}

\begin{proof}
Both $A$ and $B$ are independent sets. 
Note that for any $i \in B$, we need to remove $Z \subseteq A$ of cardinality at least $|Z| = p$ to obtain $S = (A \setminus Z) + i$ satisfying $|S \cap A| + p|S \cap B| \leq |A|$. More generally, for $Y \subseteq B$, we need to remove $Z \subseteq A, |Z| = p|Y|$ to obtain $S = (A \setminus Z) \cup Y$ that satisfies $|S \cap A| + p|S \cap B| \leq |A|$. Possibly, we could satisfy the second condition, $p|S \cap A| + |S \cap B| \leq |B|$, but this will not happen unless $|A \setminus Z| = |S \cap A| \leq |B| / p = (1 - \frac{\epsilon}{p}) n$. Therefore, we would need to remove $Z$ of cardinality at least $\frac{\epsilon}{p} n$. If the swaps considered are smaller than $\frac{\epsilon}{p} n$ then $A$ is a local optimum because adding $Y \subseteq B$ and removing $Z \subseteq A, |Z| = p |Y|$ results in a solution of lower weight. In conclusion, $A$ is a local optimum of value $w(A) = n$, while the optimum is $OPT = w(B) = (p-\epsilon)^2 n$.
\end{proof}


\subsection{Upper bound for $p$-extendible systems}

Here we prove that local search does in fact provide a $1/p^2$-approximation for weighted maximization under a $p$-extendible system. More generally, we prove (here, we will ignore the technicalities of stopping the local search in polynomial time as this can be handled using standard techniques, while losing $1/poly(n)$ in the approximation factor) the following: 



\begin{theorem} \label{th:LS-approx}
For any $p$-extendible system $\I \subseteq 2^X$ and a monotone submodular function $f:2^X \rightarrow \RR_+$,
local search with $(p,1)$-swaps (including at most $1$ element and removing at most $p$ elements) provides a $1/(p^2+1)$-approximation. For additive $f$, the factor is $1/p^2$.
%If $f$ is additive, the approximation factor is $1/p^2$.
\end{theorem}



\begin{proof}
Let $A$ be a local optimum under $(p,1)$-swaps, and let $B$ be an optimal solution. (For convenience, let us also assume that we always try to add elements to $A$ if possible, even if they bring zero marginal value.) We proceed in two steps, the first one inspired by the analysis of the greedy algorithm for $p$-extendible systems \cite{calinescu2011maximizing} and the second one similar to other analyses of local search.

Let $A = \{a_1,\ldots,a_k\}$ be a greedy ordering of $A$ in the sense that $a_1$ is the element of $A$ maximizing $f_\emptyset(a_1)$; given $a_1$, $a_2$ is the element of $A-a_1$ maximizing $f_{\{a_1\}}(a_2)$, $a_3$ is the element of $A-a_1-a_2$ maximizing $f_{\{a_1,a_2\}}(a_3)$, etc. Using the $p$-extendible property, there is a subset $B_1 \subseteq B, |B_1| \leq p$ such that $(B \setminus B_1) + a_1 \in \I$. Further, since $\{a_1,a_2\} \in \I$, there is a subset $B_2 \subseteq B \setminus B_1, |B_2|\leq p$ such that $(B \setminus (B_1 \cup B_2)) \cup \{a_1,a_2\} \in \I$, etc. Generally, there are disjoint subsets $B_1,\ldots,B_k \subseteq B, |B_i| \leq p$ such that $(B \setminus (B_1 \cup \ldots B_i)) \cup \{a_1,\ldots,a_i\} \in \I$. In fact, if $|A| = k$, the sets $B_1,\ldots,B_k$ form a partition of $B$. Otherwise there would be additional elements in $B \setminus (B_1 \cup \ldots \cup B_k)$ which can be added to $A$, which would contradict the local optimality of $A$.

Now, we claim that for each $b \in B_i$, we have $f_A(b) \leq p f_{\{a_1,\ldots,a_{i-1}\}}(a_i)$. If not, we would be able to add $b$ and, since $\{a_1,\ldots,a_{i-1},b\} \in \I$, we could remove at most $p$ elements $Z \subseteq A \setminus \{a_1,\ldots,a_{i-1}\}$ so that $(A \setminus Z) + b \in \I$. By submodularity and the greedy ordering, we would have $f(A \setminus Z) \geq f(A) - p f_{\{a_1,\ldots,a_{i-1}\}}(a_i)$ and again by submodularity, we would have $f((A \setminus Z) + b) \geq f(A \setminus Z) + f_A(b) > f(A \setminus Z) + p f_{\{a_1,\ldots,a_{i-1}\}}(a_i) \geq f(A)$. Therefore, this would be an improving local swap.

Since $A$ is a local optimum, we conclude that $f_A(b) \leq p f_{\{a_1,\ldots,a_{i-1}\}}(a_i)$ for each $b \in B_i$. Since $B = B_1 \cup \ldots \cup B_k$ and $|B_i| \leq p$, we have by submodularity
$$ f_A(B) \leq \sum_{i=1}^{k} \sum_{b \in B_i} f_A(b) \leq \sum_{i=1}^{k} |B_i| p f_{\{a_1,\ldots,a_{i-1}\}}(a_i)
\leq p^2 \sum_{i=1}^{k} f_{\{a_1,\ldots,a_{i-1}\}}(a_i) \leq p^2 f(A) $$
For $f$ monotone submodular, we have $f(B) \leq f(A) + f_A(B) \leq (p^2+1) f(A)$.
For $f$ additive, we have $f(B) = f_A(B) \leq p^2 f(A)$. This completes the proof.
\end{proof}

\subsection{Recovery for $p$-extendible systems}
%Note that in the proof of \hyperref[th:LS-approx]{Theorem \ref{th:LS-approx}} we can forget about $A\cap B$ and restrict our attention only to comparing the value of $A\sm B$ and $B\sm A$. This turns out to be useful for exact recovery as we can perturb only $A\sm B$. The following theorem intuitively tells us that \textit{local optima of stable instances are global optima}.
Note that in the proof of \hyperref[th:LS-approx]{Theorem \ref{th:LS-approx}}, if we focus on comparing the values of $A\sm B$ and $B\sm A$, we will be able to get exact recovery as we can perturb only $A\sm B$. The following theorem intuitively tells us that \textit{local optima of stable instances are global optima}.
\begin{theorem} \label{th:LS-recovery}
Given a $p$-extendible system $\I \subseteq 2^X$ and a monotone submodular function $f:2^X \rightarrow \RR_+\cup\{0\}$ we wish to maximize, if the optimal solution $B$ is $(p^2+1)$-stable, then local search with $(p,1)$-swaps exactly recovers it. If $f$ is additive, recovery holds if $B$ is $p^2$-stable.
\end{theorem}

\begin{proof}
The basic idea is that we can contract the elements that belong to $A\cap B$ and then use the same charging argument from above. Using the notation from the proof of \hyperref[th:LS-approx]{Theorem \ref{th:LS-approx}}, for elements $a_i\in A\cap B$ the corresponding $B_i$ block is just $\{a_i\}$. Now we can rename elements in $A\sm B=\{a_1,\dots,a_m\}$ with corresponding blocks $B_1,\dots,B_m$ such that $B\sm A = B_1 \cup \ldots \cup B_m$ and $|B_i| \leq p$. Rewriting the local search guarantee:
$$f_A(B\sm A) \leq \sum_{i=1}^{m} \sum_{b \in B_i} f_A(b) \leq \sum_{i=1}^{m} |B_i| p f_{\{a_1,\ldots,a_{i-1}\}}(a_i)
\leq p^2 \sum_{i=1}^{m} f_{\{a_1,\ldots,a_{i-1}\}}(a_i) \leq p^2 f(A\sm B)$$
Since $f_A(B\sm A)=f(B\cup A)-f(A)\ge f(B)-f(A)$, we can $(p^2+1)$-perturb the input (only the marginal of elements in $A\sm B$) and get: $\tilde{f}(B)=f(B) \le f(A)+p^2f(A\sm B)=\tilde{f}(A)$, hence contradicting the $(p^2+1)$-stability.
In the case of additive $f$, $f_A(B\sm A)= f(B\sm A)$ and $\tilde{f}(B)=f(B)=f(B\sm A)+f(B\cap A)\le p^2f(A\sm B) +f(B\cap A)\le f(A)+(p^2-1)f(A\sm B)=\tilde{f}(A)$, where we $p^2$-perturbed the instance, hence contradicting the $p^2$-stability of the instance.
\end{proof}



%For a constraint $(X,\I)$, we wish to find the $max\{f(S): S\in \I\}$ ($f$: monotone submodular) where $\I$ is the intersection of $p$ matroids: $\I=\cap_{i=1}^p \I_i$. We prove that local search with ($p,1$)-swaps exactly recovers the optimal solution if it is $(p+1)$-stable. For one matroid $p=1$ (a matroid is 1-extendible), our previous \hyperref[th:LS-recovery]{Theorem~\ref{th:LS-recovery}} implies:




\subsection{Recovery for the intersection of Matroids}
If the independence system $\I$ is the intersection of $p$ matroids: $\I=\cap_{i=1}^p \I_i$, local search with ($p,1$)-swaps recovers $(p+1)$-stable optimal solutions (for proof, see \hyperref[app:LSrecovery]{Appendix~\ref{app:LSrecovery}}). 

%Since a matroid is 1-extendible, our previous \hyperref[th:LS-recovery]{Theorem~\ref{th:LS-recovery}} implies:
%\begin{corollary}
%Given a matroid $(X,\I)$ and $f$ monotone submodular, such that the optimal solution is 2-stable, Local Search exactly recovers it.
%\end{corollary}
%\begin{proof}
%Suppose again that $A$ is the local optimum solution and $B$ is the global optimum. By the local search criterion we have that $\exists$ a bijection $\pi:A\setminus B \to B\setminus A:$
%\begin{itemize}
%\item 1. $\forall x \in A\setminus B: (A-x+\pi(x)) \in \I$
%\item 2. $\forall x \in A\setminus B: f(A-x+\pi(x))\le f(A)$
%\end{itemize}
%Since $(A-x)\subseteq (A+\pi(x)-x)$, using the submodularity for adding $x$, we get :
%\[
%-(f(A)-f(A-x))+(f(A+\pi(x))-f(A))\le 
%\]
%\[
%\le -(f(A+\pi(x))-f(A+\pi(x)-x)) + (f(A+\pi(x))-f(A))\le0
%\]
%Rearranging we get: $f(A+\pi(x))-f(A)\le f(A)-f(A-x)=f_{A-x}(x), \forall x\in A$. Let $A = \{a_1,\ldots,a_k\}$, $A_i=\{a_1,a_2,\dots a_i\}$ and let the marginal improvement be $\delta_i = f_{A_{i-1}}(a_i)= f(A_i)-f(A_{i-1})$ at the point of addition of $a_i$ (for simplicity we make abuse of notation and have $\delta_{a_i}=\delta_i$). Summing up the inequalities for all $x \in A\setminus B$, we get a telescoping sum:
%\[
%f(A\cup B)-f(A)\le \sum_{i\in A\setminus B}\delta_i \iff f(A\cup B)\le f(A)+\sum_{i\in A\setminus B}\delta_i
%\]
%We can now use our assumption about 2-stability by perturbing (here $\gamma=2$) only the marginals of the elements $x \in A\setminus B$ (this favours only our local search solution). The new value for the local search solution $A$ is: $\tilde{f}(A)=f(A)+\sum_{i\in A\setminus B}\delta_i$ and thus we get:
%\[
%\tilde{f}(B)=f(B)\le f(A\cup B)\le f(A)+\sum_{i\in A\setminus B}\delta_i=\tilde{f}(A),
%\]
%which contradicts the 2-stability of the instance.
%\end{proof}

%Generally, for $p$ matroids we have the following theorem: (for the proof, we refer the reader to \hyperref[app:LSrecovery]{Appendix~\ref{app:LSrecovery}} of the full version of this paper):
\begin{theorem}\label{th:LSmatroids}
Given $(X,\I)$, with $\I=\cap_{i=1}^p \I_i$ where each $\I_i$ is a matroid and $f$ monotone submodular, such that the optimal solution is $(p+1)$-stable, Local Search exactly recovers it.
\end{theorem}


%\begin{proof}
%We denote with $A$ our local search solution (let it be maximal, even if new elements add zero value to it) and with $B$ the global optimum. Let $Y=B\sm A=\{y_1,y_2,\dots,y_k\}$ be the elements of the optimum that local search didn't choose. By the matching property~\cite{reichel2007evolutionary} of the matroids we get:
%\[
%\exists\ X^1,X^2,\dots,X^p \subseteq A\sm B, \text{where\ } X^j=\{x_1^j,x_2^j,\dots,x_k^j\}\  \text{such that:\ }
%\]
%\[
%\forall j\in\{1,\dots,p\}: x_i^j \in C_j(A,y_i), \forall i\in \{1,2,\dots,k\},
%\]
%where $C_j(A,y_i)$ is the circuit (minimally dependent set) created in matroid $\I_j$ when adding $y_i$ in $A$.

%Using the local search (with $(p,1)$ swaps) stopping condition, we have: (for ease, we use $+,-$ instead of the more accurate $\cup, \sm$)
%\[
%f(A+y_i-x_i^1-x_i^2-\dots-x_i^p)\le f(A), \forall i\in \{1,2,\dots,k\}
%\]
%(Note that in case $f$ is additive the above inequality just becomes: $f(y_i)\le f(x_i^1)+f(x_i^2)+\dots+f(x_i^p)$).
%Since $(A-\cup_{j=1}^p x^j_i)\subseteq (A+y_i-\cup_{j=1}^p x^j_i)$, using the submodularity for adding $\cup_{j=1}^p x^j_i$, we get:
%\[
%f(A+y_i)-f(A+y_i-\cup_{j=1}^p x^j_i)\le f(A)-f(A-\cup_{j=1}^p x^j_i)
%\]
%and adding $f(A+y_i)-f(A)$ to both sides and using submodularity and the local search stopping condition, we get:
%\[
%f(A+y_i)-f(A)-f(A)+f(A-\cup_{j=1}^p x^j_i)\le f(A+y_i-\cup_{j=1}^p x^j_i)-f(A)\le 0
%\]
%We conclude: $f(A+y_i)-f(A)\le f(A)-f(A-\cup_{j=1}^p x^j_i), \forall i\in \{1,2,\dots,k\}$ and adding these inequalities ($\delta(x_i^j)$ is the marginal gain by adding $x_i^j$ at the point of addition):
%\[
%f_A(B\sm A)\le \sum_{i=1}^k\sum_{j=1}^p\delta(x_i^j) = \sum_{j=1}^p\sum_{i=1}^k\delta(x_i^j)\le \sum_{j=1}^p f(X^j)\le \sum_{j=1}^p f(A\sm B)\le pf(A\sm B)
%\]
%Now we can $(p+1)$-perturb the marginals for elements of $A\sm B$:
%\[
%\tilde{f}(B)=f(B)\le f(A\cup B)\le f(A)+f_A(B\sm A)\le f(A)+pf(A\sm B)=\tilde{f}(A)
%\]
%which contradicts the ($p+1$)-stability of the instance. Once again, for the case of additive $f$: $f_A(B\sm A)=f(B\sm A)$ and thus $p$-stability is enough to guarantee recovery. ($\tilde{f}(B)=f(B)=f(B\sm A)+f(B\cap A)\le pf(A\sm B) +f(B\cap A)\le f(A)+(p-1)f(A\sm B)=\tilde{f}(A)$, where we $p$-perturbed the instance)
%\end{proof}








