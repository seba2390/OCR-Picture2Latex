\section{Two simple motivating examples}\label{app-A}
Here we will see two motivating examples out of the many problems \cite{mestre2006greedy} that fall into the framework of $p$-extendible systems and thus, we will directly get recovery results on their stable instances. Some of the problems might be hard, like ATSP, whereas others may be easy (i.e. in $P$, like Weighted Matching), having exact algorithms, however the greedy is extremely simple and fast compared to those. 


\subsection{Motivating Examples}

To illustrate some special cases for the setting for our problem and highlight two important applications, we will focus on Matching and on Asymmetric Traveling Salesman Problem (ATSP) proving greedy recovery from scratch. We will recover 2-stable instances for Matching and 3-stable instances for ATSP. These results of course, can be directly derived from our main result for the greedy in the additive case (\hyperref[greedy-additive]{Theorem \ref{greedy-additive}}).\\
\begin{itemize}
\item Weighted Matching: Given a graph $G=(V,E)$ and weights $w_e\ge 0$ for the edges, the goal is to find a matching (i.e. a set of edges that have no common endpoints) of large
weight. The greedy algorithm sorts the edges by weight (let's suppose no ties; the recovery results remain the same) and starts with the empty matching $M$; it then adds heavy edges to
the matching $M$, in this sorted order, as long as the set remains a feasible matching. We denote the optimum matching by $M^*$.

It is a well-known fact that the greedy is a 2-approximation algorithm; the weight of the matching $M$ returned by the greedy algorithm is at least half
of the weight of any matching $M'$, thus we get $2w(M)\ge w(M^*)$. Of course, we can solve matching exactly in polynomial time, however the simplicity of greedy and the fact that it runs almost in linear time is the reason why we are interested. Our 2-stability recovery result implies that sometimes simple and fast algorithms are probably enough for practical purposes.

Now, suppose we had a 2-stable instance; the optimum matching $M^*$ remains the unique optimum even after multiplying by 2 any edge weights we want. We want to prove that in an instance like that, the greedy algorithm will in fact output the optimum solution, i.e. we will have $M\equiv M^*$.
\end{itemize}
\begin{proposition}
Greedy recovers optimum on 2-stable instances of Weighted Matching.
\end{proposition}
\begin{proof}
It is not hard to see why this is true via an inductive argument: Let's look at the first edge $e=(x,y)$ that the greedy decided to pick, but does not belong to the optimum solution. The greedy picked the edge $e=(x,y)$ and at most 2 edges $e_x, e_y \in M^*$ might have a conflict with $e$. By the greedy criterion, we know that $w_e > w_{e_x}$ and $w_e > w_{e_y}$, because otherwise we would have picked $e_x$ or $e_y$ (the feasibility follows from induction, since before picking $e$, the greedy coincides with the optimum). We get that $2\cdot w_e > w_{e_x}+w_{e_y}$ and thus the optimum solution $M^*$ would not remain the same in the perturbed instance, where we decided to multiply by 2 the edge weight $w_e$ (to see this, observe that, in the perturbed instance, the optimum improves by replacing both $e_x$ and $e_y$ with $e$, while all of its other choices remain unaffected by this exchange). This violates the fact that the input is 2-stable. We deduce that $M\setminus M^* = \emptyset$. Finally, since the greedy outputs a maximal matching, we know that $M$ is not a strict subset of $M^*$ and so we have that $M^*\setminus M=\emptyset$ and thus we get $M\equiv M^*$.
\end{proof}

We note that the corresponding set system for matching is 2-extendible (see~\hyperref[sec:preliminaries]{Section~\ref{sec:preliminaries}} for definitions) and that ($2-\epsilon$)-stability is not enough; given any small $\epsilon>0$, there exists a ($2-\epsilon$)-stable instance for which greedy fails to recover the optimum matching. The counterexample is the same that is used to show that greedy is a tight 2-approximation: the path of length 3 with weights 1, 1+$\epsilon'$, 1, where $\epsilon'<\tfrac{\epsilon}{2-\epsilon}$ is a $(2-\epsilon)$-stable instance, for which greedy fails to recover the optimum. For arbitrarily large input, we can repeat the counterexample.

\begin{itemize}
\item Asymmetric TSP: Our next example is the well-studied $NP$-hard problem of Maximum Asymmetric Travelling Salesman. We are given a complete directed
graph on $n$ nodes, with non-negative weights (again no ties, see remark  below) and we must find a maximum weight tour that visits every node exactly once. 

We can view ATSP as an independence system with its elements being the directed edges of the complete graph. In addition, a set is independent if its edges form a collection of vertex disjoint paths or a cycle that contains every vertex exactly once. The greedy, as in all problems, sorts the edges and then starts by picking the heaviest edge possible, while still remaining feasible, i.e. no smaller than Hamilton cycles, no vertex having in-degree or out-degree more than 1.
\end{itemize}
\begin{proposition}
Greedy recovers optimum on 3-stable instances of ATSP.
\end{proposition}
\begin{proof}
Let's denote with $C$ the greedy solution and with $C^*$ the optimal. We want to show that $C\equiv C^*$. We will suppose the contrary, and show that under 3-stability, $C^*$ could improve. Supposing now $C\not\equiv C^*$, let's denote $e=(x,y)$ the first (directed) edge the greedy picked that is not in $C^*$. We know that in the directed cycle $C^*$, there is exactly one edge leaving node $x$ and exactly one edge going into $y$. We denote them respectively with $e_x^{out}=(x,x^+)$ and $e_y^{in}=(y^-,y)$. Since greedy is feasible and it preferred $e$ (important: $e$ is the first disagreement) over $e_x^{out}$ and $e_y^{in}$, we deduce that $w_e > w_{e_x^{out}}$ and $w_e > w_{e_y^{in}}$. It is easy to see that $e=(x,y)$ cannot be the last choice of greedy ($C^*$ wouldn't be optimal then) and thus, in the unique directed path $P_{y\rightarrow x}^* \in C^*$ from node $y$ to node $x$, there exists an edge $e^*=(u,v)$ that the greedy didn't choose over $e=(x,y)$ (if not, greedy would close a non-Hamilton cycle which is not feasible). From this observation, we can also deduce that $w_e > w_{e^*}$. Now we are ready to put it all together. We have 
\begin{align*}
w_e > w_{e_x^{out}}, w_e > w_{e_y^{in}}, w_e > w_{e^*} \implies 3w_e > w_{e_x^{out}}+ w_{e_y^{in}}+ w_{e^*}
\end{align*}
and all edges  $e_x^{out}, e_y^{in}, e^*$ are in $C^*$, but not in greedy. We perturb the instance by a factor of 3, by multiplying $w_e$ by 3. We claim that $C^*$ in not optimum now and for that we will construct a better Hamilton cycle that differs slightly from the cycle $C^*$: We start from node $x$, take edge $e=(x,y)$, follow path $P_{y\rightarrow u}^* \subset P_{y\rightarrow x}^*$, take edge $(u,x^+)$, follow the unique directed path $P_{x^+\rightarrow y^-}^* \in C^*$, take edge $(y^-,v)$ and finally return to node $x$ through the unique path $P_{v\rightarrow x}^* \in C^*$. Note that even if the weights of newly put edges $(u,x^+)$ and $(y^-,v)$ are zero, we do not care; the new cycle, having $(x,y)$ in it, is still better in the 3-perturbed instance than the old supposedly optimum $C^*$. This violates the 3-stability property and thus, using maximality again, we get $C\equiv C^*$.
\end{proof}

We note that the corresponding set system for ATSP is 3-extendible~\cite{mestre2006greedy}, and that ($3-\epsilon$)-stability is not enough; given any small $\epsilon>0$, there exists a ($3-\epsilon$)-stable instance for which greedy fails to recover the optimum tour for ATSP. The counterexample, as is the case for matching, is the same that is used to show that greedy is a tight 3-approximation: a triangle with weights (1,1,1) for the edges going one direction and weights $(1+\epsilon'$,0,0) going the other direction, where $\epsilon'<\tfrac{\epsilon}{3-\epsilon}$.

As a remark for the analysis of both our examples, we note that the knowledge of optimum is not necessary and that in the case we have ties in some edges' weights, our results for recovery still hold, because of optimum solution's uniqueness under the stability definition.

