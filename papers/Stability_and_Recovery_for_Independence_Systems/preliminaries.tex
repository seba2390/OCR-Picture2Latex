\section{Preliminaries}\label{sec:preliminaries}
In this section we describe the notation and definitions which we use through the rest of the paper. We start by defining the family of $p$-systems and the problem of submodular maximization; then we present our two protagonist algorithms and the standard (additive) stability definition.

\begin{itemize}
\item
$p$-Systems~\cite{korte1978analysis,lee2009submodular}: Suppose we are given a (finite) ground set $X$ of $m$ elements (this could be the set of edges in a graph) and we are also given an \textit{independence family} $\I\sse 2^X$, a family of subsets that is downward closed; that is, $A \in \I$ and $B\sse A$ imply that $B\in \I$. A set $A$ is independent iff $A\in \I$. For a set $Y\sse X$, a set $J$ is called a \textit{base} of $Y$, if $J$  is a maximal independent subset of $Y$; in other words $J\in \I$ and for each $e\in Y\sm J$, $J+e\not\in \I$. Note that $Y$ may have multiple bases and that a base of $Y$ may not be a base of a superset of $Y$. $(X,\I)$ is said to be a $p$-system if for each $Y\sse X$ the following holds:
\[ \dfrac{\max_{J: J\ \text{is a base of}\ Y} |J|}{\min_{J: J\ \text{is a base of}\ Y}|J|} \le p\]
 All set systems are assumed to be down-closed. There are some interesting special cases of $p$-systems~\cite{mestre2006greedy, calinescu2011maximizing}: 
\[
\text{intersection of $p$ matroids $\sse$ $p$-circuit-bounded systems $\sse$ $p$-extendible systems $\sse$ $p$-systems} 
\]
\item $p$-extendible: An independence system $(X,\I)$ is $p$-extendible if the following holds: suppose we have $A\sse B, A,B \in \I$ and $A+e\in \I$; then there should exist a set $Z\sse B\sm A$ such that $|Z|\le p$ and $B\sm Z +e \in \I$. We note here that $p$-extendible systems make sense only for integer values of $p$, whereas $p$-systems can have $p$ being fractional and that 1-systems as well as 1-extendible systems are exactly matroids. It is a family of independence systems containing many important and seemingly unrelated problems like welfare maximization, $k$-dimensional Matching, Asymmetric Travelling Salesman Problem, weighted $\Delta$-Independent Set ($\Delta$: maximum degree) and others~\cite{mestre2006greedy}.

\item
Submodular Maximization: A set function $f : 2^X\rightarrow \RR^+\cup \{0\}$ is submodular if for every $A, B \subseteq X$, we have $f(A) + f(B) \ge f(A \cup B) +
f(A\cap B)$. Given a $p$-system $(X,\I)$ and a monotone submodular function $f$, we are interested in the problem of maximizing $f(S)$ over the
independent sets $S \in \I$; in other words we wish to find $\max_{S\in \I} f(S)$. If $f$ is additive, we can associate a weight $w_e$ with each element $e \in X$ and we want to find $\max_{S\in \I}w(S)$, where $w(S)=\sum_{e\in S}w_e$.

\item
Greedy algorithm: It starts with $S=\emptyset$ and greedily picks elements of $X$ that will increase its objective value by the most, while remaining feasible i.e. picks $e^*=\arg\max_{e\in X, S+e\in \I}(f(S+e)-f(S))$. It is a well-known fact~\cite{korte1978analysis,nemhauser1978analysis}, that for any $p$-system, the standard greedy algorithm is a ($p+1$)-approximation (if $f$ is additive, Greedy is a $p$-approximation).

\item
$(p,q)$-Local Search: It starts from a feasible solution and at each iteration seeks for an improving move. In particular, starting from any $S\in \I$, it tries to find a \textit{better} $S'\in\I$ with: $|S\sm S'|\le p$, $|S'\sm S|\le q$ and $f(S')>f(S)$. If it finds such a feasible solution $S'$, it switches to $S'$ and repeats. It stops when no improving move can be made. Note that the stopping condition and its performance depend on the size of the $(p,q)$-\textit{neighbourhood} used. We note that $(p,1)$-local search is necessary for $p$-extendible systems. For recent improvements on Local Search performance in the case of matroids, we refer the reader to \cite{lee2009submodular}.

\item
Stable instances: Stability can be defined in general for instances of weighted optimization problems~\cite{bilu2012stable}, where the objective function $w$ is additive. In our case, given a $p$-system and an \textit{additive} function $w$ we wish to maximize over the $p$-system, we call the instance $\gamma$-stable, if the optimal solution $S^* \in \mathcal{I}$ remains the unique optimum, even after assigning a new weight $\tilde{w}_e$ to an element $e$ such that $w_e\le \tilde{w}_e\le \gamma \cdot w_e$. In an extreme case, we can keep the weights of the elements in optimum the same and increase all others by a factor of $\gamma$; the optimum should remain the same. Sometimes, we say that we $\gamma$-perturb the input when we multiply some weights by at most $\gamma$. We will see in \hyperref[sec:submodular] {Section \ref{sec:submodular}} how to extend this additive stability definition to stability for submodular functions.

\end{itemize}