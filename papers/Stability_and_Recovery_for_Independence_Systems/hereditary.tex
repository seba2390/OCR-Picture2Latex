\section{Hereditary Systems}\label{sec:hereditary}

Motivated by the ``bad'' example (see \hyperref[knapsack]{Proposition \ref{knapsack}}) for the greedy algorithm, we define a new notion of an independence system that we call \textit{hereditary} $p$-system or $p$-\textit{hereditary} that as we see later is a different characterization of $p$-extendible systems. In the aforementioned example, even though we started with a $p$-system, as we progressed picking elements with the greedy algorithm, the system became a $p'$-system with $p' \gg p$, thus leading to bad performance for the greedy, even though we had the optimal solution being stable by a large amount.

The intuition behind the following definition is that we want our system to remain a $p$-system under deletions and contractions of elements.

\begin{definition}[Hereditary $p$-system]
A $p$-system $(X,\I)$ is said to be \textit{hereditary} if:
\begin{enumerate}
\item For each set $Y\sse X$, the system $(X', \I | X')$\footnote{By $(X',\I | X')$, we mean the \textit{restriction} of $\I$ to the set of elements $X'$, which is the independence system on the set $X'$, whose independent sets are the independent sets of the initial set $\I$ that are contained in $X'$.}, where $X'=X \sm Y$, is a $p$-system. This corresponds to the {\bf deletion} of the elements in $Y$ from the system.
\item For each set $Y\sse X$,  the system $(X\sm Y, \I / Y)$\footnote{By $(X\sm Y,\I / Y)$, we mean the \textit{contraction} of $\I$ by $Y$, which is the independence system on the underlying set $X\sm Y$, whose independent sets are the sets $Z \sse X\sm Y$, such that $Z\cup Y \in \I$.} is a $p$-system. This corresponds to the {\bf contraction} of the elements in $Y$.
\end{enumerate}
\end{definition}

\noindent Looking back at our ``bad'' Knapsack example we see that it is not a hereditary system since initially $p=\tfrac{2M}{M+1}\le 2$, but after we had picked all the elements in set $A$, the system on the remaining elements became an $M$-extendible system. We now prove that the family of hereditary $p$-systems coincides with the family of $p$-extendible systems.  

\begin{proposition}
A $p$-system is $p$-hereditary if and only if it is $p$-extendible.
\end{proposition}

\begin{proof}
$p$-hereditary $\implies$ $p$-extendible: Let's first think of $p$ as an integer; as we will see afterwards only this case (with integer $p$) is interesting. Suppose we had a $p$-hereditary system that was not $p$-extendible. By negating the definition of $p$-extendibility (see \hyperref[sec:preliminaries]{Preliminaries}), it follows that there exist sets $A,B \sse X$ with $A \sse B$, $A,B \in \I$ and $A\cup\{e\} \in \I$ such that for all sets $Z\sse B\sm A$ with $|Z| \le p$: $(B\sm Z)\cup \{e\} \not\in \I$. Define $Z_0 \sse B\sm A$ to be the smallest set that we need to remove from $B$ in order to have: $(B\sm Z_0)\cup \{e\} \in \I$. We know that $|Z_0|>p$ and thus, by the hereditary property, if we project the independence system on the elements $Z_0\cup \{e\}$, we get $Z_0\cup \{e\} \not\in \I$ with the ratio $\tfrac{|Z_0|}{|\{e\}|}=\tfrac{|Z_0|}{1}>p$, which contradicts the fact that we started with a $p$-hereditary system.


For $p$-extendible $\implies$ $p$-hereditary: This direction follows easily just by the definition of $p$-extendibility. To handle non-integer values of $p$, we observe that by the first argument above, a $p$-hereditary system is actually $\lfloor p\rfloor$-extendible and thus, it is $\lfloor p\rfloor$-hereditary (e.g. a 2.9-hereditary system is 2-extendible).
\end{proof}
























