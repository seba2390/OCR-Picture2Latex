\section{Warm-up: Additive Case and Greedy Recovery}\label{sec:additive}
In this section, as a warm-up, we deal with additive functions, proving the first positive recovery result for the greedy algorithm and showing that it is tight.
\subsection{Exact Recovery for $p$-extendible, $p$-stable systems}

We are given an independence set system $(X,\mathcal{I},w)$ and we want to find an independent solution $S^* \in \mathcal{I}$ with maximum weight, where for $I\in\I: w(I)=\sum_{e \in I}w(e)$. We are interested in the performance of the standard greedy algorithm and we can prove the following: 





\begin{theorem}\label{greedy-additive}
Given an instance of a $p$-extendible independence system $(X,\mathcal{I},w)$, that has a $p$-stable optimal solution $S^* = \arg \max_{I\in \I}w(I)$, the Greedy algorithm exactly recovers $S^*$.
\end{theorem}

\begin{proof}
From the definition of $p$-extendibility we know that for the system $\mathcal{I}$, the following holds: suppose $A\subseteq B, A,B \in \mathcal{I}$ and $A+e \in \mathcal{I}$, then there is a set $Z\subseteq B\setminus A$ such that $|Z|\le p$ and $B\setminus Z+e \in \mathcal{I}$. The Greedy starts from the empty set and greedily picks elements with maximum weight subject to being feasible; it finally outputs $S$ which is a maximal solution, i.e. $S\cup \{e\}\notin \I, \forall e\in X\sm S$. In order to get exact recovery, we want to show that $S\equiv S^*$. 

Let's suppose $S\sm S^*\neq \emptyset$. Then, out of all the elements of $S\sm S^*$ that the Greedy selected, let's focus on the first element $e_1\in S\sm S^*$. Let $S_{\{e_1\}}$ denote the greedy solution right before it picked element $e_1$. Note that before choosing $e_1$, greedy $S_{\{e_1\}}$ was in agreement with the optimal solution, i.e. $S_{\{e_1\}}\subseteq S^*$. Since $e_1 \not \in S^*$, we can use the $p$-extendibility, where we specify $A=S_{\{e_1\}}, B=S^*, e=e_1$ ($A+e \equiv S_{\{e_1\}} + e_1 \in \mathcal{I}$, since Greedy is always feasible) and we get, following the above definition, that there exists set of elements $Z\subseteq S^*\sm A \equiv S^*\sm S_{\{e_1\}}$, with $|Z|\le p$ and $(S^*\setminus Z)\cup\{e_1\} \in \mathcal{I}$. This intuitively means that the element $e_1$ has conflicts with the elements in $Z\subseteq S^* \setminus S_{\{e_1\}}$, but if we remove at most $|Z|\le p$ elements from $S^*\sm S_{\{e_1\}}$, we get no conflicts and thus an independent (feasible) solution according to the system $\mathcal{I}$. 

We call this solution $J$, i.e. $J=(S^*\setminus Z)\cup\{e_1\} \in \mathcal{I}$ (note $J\neq S^*$) and we will show that we can perturb the instance (new weight function $\tilde{w}$) no more than a factor of $p$, so that $J$'s weight is at least that of the optimal, i.e. $\tilde{w}(J)\ge \tilde{w}(S^*)$, which would be a contradiction to the $p$-stability of the given instance. All we have to do is perturb the instance by multiplying the weight of the element $e_1$ by $p$. By the greedy criterion for picking elements (note that all elements of $Z$ were available to Greedy at the point it chose $e_1$) and the fact that $|Z|\le p$ we get:
\begin{align} \label{eqeq1}
\forall e \in Z\subseteq \left(S^*\sm S_{\{e_1\}}\right): w(e_1) \ge w(e) \implies p\cdot w(e_1)\ge \sum_{e\in Z}w(e) = w(Z)
\end{align} which implies that the weight of the set $J$ is actually no less than the weight of $S^*$ in the aforementioned perturbed instance (weight function $\tilde{w}$). Indeed:
\[
\tilde{w}(J)=\tilde{w}(\left(S^*\setminus Z\right) \cup e_1) = \tilde{w}(S^*\setminus Z) + \tilde{w}(e_1) = w(S^*)-w(Z)+p\cdot w(e_1)\ge w(S^*)=\tilde{w}(S^*)
\]
where for the last inequality we used (\ref{eqeq1}). This is a contradiction because it violates the $p$-stability property (the optimal solution should stand out as the unique optimum for any $p$-perturbation) and thus we conclude that $S\sm S^*=\emptyset$. Since Greedy outputs a maximal solution, we conclude that $S$ coincides with $S^*$ and so Greedy exactly recovers the optimal solution.
\end{proof}

% Although the proof is fairly straightforward, it is the first time,
% to the best of our knowledge, that recovery results for many
% different problems are proven simultaneously, indicating that
% independence systems was the right notion of abstraction. As we show
% next, 
We next show that our result is tight both in terms of the stability factor and the generality of $p$-extendible systems. 

\begin{proposition}
There exist $p$-extendible systems with a $(p-\epsilon)$-stable optimal solution $S^*$, for which the Greedy fails to recover it.
\end{proposition}

\begin{proof}
%We can give a simple counterexample for a problem which is even in
%$P$. 
Take a Maximum Weight Matching instance (here $p=2$): a path of length
3 with weights (1,$1+\epsilon'$,1). The Greedy fails to recover the
optimal solution $S^*$, since it picks the $(1+\epsilon') $ edge
whereas it should have picked both the other edges. For the right
choice of $\epsilon'$ ($\epsilon'<\tfrac{\epsilon}{2-\epsilon}$), we
can make the instance arbitrarily close to $(p-\epsilon)=(2-\epsilon)$
stable.  Observe that we can give such examples for any value of $p$
(consider the $p$-dimensional Matching problem) and that the example
can be made arbitrarily large just by repeating it.  
\end{proof}

\begin{proposition} \label{knapsack}
There are $p$-systems whose optimal solution $S^*$ is $M$-stable (for arbitrary $M>1$) and for which the greedy algorithm fails to recover it.
\end{proposition}

\begin{proof}
The example is based on a knapsack constraint. Fix $M'>1$ and let the size of the knapsack $B=M'+1$. We will have elements of type $A$ ($|A|=M'$), a special element $e^*$ and elements of type $C$ ($|C|=M'$). The pair (value, size) for elements in $A,C$ is respectively: $(2,1),(1,\tfrac{1}{M'})$ and for $e^*: (1+\epsilon,1), \epsilon>0$. Note that the optimal solution $S^*$ is $A\cup C$ with total value $2M'+M'=3M'$ and size $M'+M'\cdot\tfrac{1}{M'}=M'+1$ (fits in the knapsack). However, Greedy will pick $A\cup\{e^*\}$ for a total value of $2M'+1+\epsilon$ and size $M'+1$. Note that this is a $p$-system for a value of $p<2$ since any feasible solution $S$ can be extended to a solution $S'$ with $|S'|\ge M'+1$ and the largest feasible solution has $2M'$ elements (there are only $2M'+1$ elements in total). However, this is not a $2$-extendible system (it is actually an $M'$-extendible system) and we see that even if it is ($M'-1$)-stable, Greedy still fails to recover the optimal solution $S^*$. To see why it is ($M'-1$)-stable, note that the only $\gamma$-perturbation (perturbations are allowed only on the values, not the sizes) we can make to favour the greedy solution is to the element $e^*$, thus we would need $\gamma(1+\epsilon)\ge M' \implies \gamma>M'-1$ ($\epsilon$ is small). Choose $M'=M+1$ and this concludes the proof. We also note that a variation of this counterexample would trick as well the (more natural) Greedy that sorts the elements according to value density ($\tfrac{v_i}{s_i}$) instead of just their value.
\end{proof}

We find \hyperref[knapsack]{Proposition~\ref{knapsack}} surprising, given that the greedy
algorithm is a good worst-case approximation algorithm for such
problems. The above ``bad'' example leads us to the definition of \textit{hereditary} systems; it turns out that this is another characterization of the $p$-extendible systems. Due to space constraints, we defer the definition until \hyperref[sec:hereditary]{Appendix~\ref{sec:hereditary}}.
%Hence the story of recovery is somewhat different from the story of %approximation, where having a $p$-system (generalization of $p$-extendible systems) is enough in order to get a $p$-approximation greedily, whereas it is not enough to guarantee greedy recovery. 


%\subsection{Relaxing the notion of stability (local stability)}

%We introduce a more realistic notion of stability which we call \textit{local stability}. Intuitively, an element $e\in X$


%The optimal solution of every perturbed instance of a weakly
%stable instance, is close to the optimal solution of the original instance, but may not be exactly
%the same (see Section 6 for details). We believe that $\gamma$-weak stability may be a more realistic
%assumption than $\gamma$-stability in practice. Bilu and Linial [9] mentioned weakly stable instances in
%the introduction to their paper (without formally defining them), and proposed to study them in
%the future. Our algorithms for $\gamma$-weakly stable instances are not robust.








%%% Local Variables:
%%% mode: latex
%%% TeX-master: "main"
%%% End:
