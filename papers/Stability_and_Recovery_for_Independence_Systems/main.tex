
\documentclass[11pt]{article}

 
\usepackage{microtype}%if unwanted, comment out or use option "draft"
%\usepackage{hyperref}
%\hypersetup{
%    colorlinks=true,
%    linkcolor=blue,      
%    citecolor=blue
%}
\bibliographystyle{plainurl}
\usepackage{graphicx}
\usepackage{caption}
\usepackage{subcaption}
\usepackage{theorem}
%\usepackage{subfigure}
\usepackage{epsfig}
\usepackage{amssymb}
\usepackage{fullpage}
\usepackage{amsmath}
\usepackage{ifthen}
\usepackage{verbatim}
\usepackage{mdframed}
\usepackage{xspace}
\usepackage{enumitem}
\usepackage[multiple]{footmisc}
\usepackage{url}
\usepackage[colorlinks,linkcolor=blue,citecolor=blue,urlcolor=blue]{hyperref}

\usepackage{tikz}
\usetikzlibrary{shapes,shapes.geometric,arrows,fit,calc,positioning,automata,arrows.meta}
\tikzset{>={Latex[width=2mm,length=2mm]}}
\usepackage{float}

%algorithm env customization
\usepackage[noend,noline]{algorithm2e}
\SetEndCharOfAlgoLine{}
\SetArgSty{}
\SetKwBlock{Repeat}{repeat}{}
\newtheorem{theorem}	 			{Theorem}[section]
\newtheorem{lemma}		[theorem]	{Lemma}	
\newtheorem{fact}		[theorem]	{Fact}
\newtheorem{claim}		[theorem]	{Claim}
\newtheorem{corollary}		[theorem]	{Corollary}
\newtheorem{prop}		[theorem]	{Proposition}
\newtheorem{definition}	 			{Definition}[section]
%{\theorembodyfont{\rmfamily} \newtheorem{definition}
%[theorem]	{Definition}}
{\theorembodyfont{\rmfamily} \newtheorem{remark}		[theorem]
{Remark}}
{\theorembodyfont{\rmfamily} \newtheorem{proposition}		[theorem]
{Proposition}}
{\theorembodyfont{\rmfamily} \newtheorem{example}		[theorem]
{Example}}
{\theorembodyfont{\rmfamily} \newtheorem{question}
{Open Question}}
{\theorembodyfont{\rmfamily} \newtheorem{warning}			
{Warning}}
{\theorembodyfont{\rmfamily} \newtheorem{q}			[theorem]
{Question}}
{\theorembodyfont{\rmfamily} \newtheorem{exercise}		[theorem]
{Exercise}}
{\theorembodyfont{\rmfamily} \newtheorem{coursegoal}	
{Course Goal}}
\theoremstyle{break}
{\theorembodyfont{\rmfamily} \newtheorem{remarkbreak}		[theorem]
{Remark}}

\newenvironment{proof}{\noindent {\em {Proof:}}}{$\blacksquare$\vskip
\belowdisplayskip}

\newenvironment{prevproof}[2]{\noindent {\em {Proof of
{#1}~\ref{#2}:}}}{$\blacksquare$\vskip \belowdisplayskip}

%\newcommand{\prevproof}[3]{
%{\noindent {\em Proof of {#1}~\ref{#2}.} {#3} \rule{2mm}{2mm} \vskip
%\belowdisplayskip}
%}

%%%FOOTNOTES OUTSIDE OF FRAMES

\usepackage{tablefootnote} 
\makeatletter 
\AfterEndEnvironment{mdframed}{%
 \tfn@tablefootnoteprintout% 
 \gdef\tfn@fnt{0}% 
}
\makeatother

\newlist{exlist}{enumerate}{1}
\setlist[exlist]{label=(\alph*)}

\newcommand{\floor}[1]{
{\lfloor {#1} \rfloor}
}
\newcommand{\bigfloor}[1]{
{\left\lfloor {#1} \right\rfloor}
}

\newcommand{\argmax}{\operatornamewithlimits{argmax}}
\newcommand{\argmin}{\operatornamewithlimits{argmin}}
\newcommand{\prob}[2][]{\text{\bf Pr}\ifthenelse{\not\equal{}{#1}}{_{#1}}{}\!\left[#2\right]}
\newcommand{\expect}[2][]{\text{\bf E}\ifthenelse{\not\equal{}{#1}}{_{#1}}{}\!\left[#2\right]}
\newcommand{\var}[2][]{\text{Var}\ifthenelse{\not\equal{}{#1}}{_{#1}}{}\!\left[#2\right]}
\newcommand{\dev}[2][]{\text{StdDev}\ifthenelse{\not\equal{}{#1}}{_{#1}}{}\!\left[#2\right]}
\newcommand{\ip}[2]{
{\langle {#1} , {#2} \rangle}
}

\newcommand{\subclaim}[1]{
\vskip 0.10in
{\noindent {\bf Claim: } {\em {#1}}}
\vskip 0.10in
}
\newcommand{\claimproof}[1]{
{\noindent {\it Proof of Claim.} {#1} \rule{2mm}{2mm} \vskip
\belowdisplayskip}}

\newcommand{\bid}{b}
\newcommand{\bids}{{\mathbf \bid}}
\newcommand{\bidsmi}{{\mathbf \bid}_{-i}}
\newcommand{\bidi}[1][i]{\bid_{#1}}

\newcommand{\val}{v}
\newcommand{\vals}{{\mathbf \val}}
\newcommand{\valsmi}{{\mathbf \val}_{-i}}
\newcommand{\vali}[1][i]{\val_{#1}}

\mdfsetup{frametitlealignment=\centering}
\mdfdefinestyle{offset}{backgroundcolor=white,linecolor=black,innerrightmargin=15pt,leftmargin=23pt,rightmargin=23pt,innertopmargin=.5\baselineskip,innerbottommargin=.5\baselineskip}

\def\sm{\setminus}
\def\sse{\subseteq}
\def\eps{\epsilon}
\def\RR{\mathbb{R}}
\def\P{\mathcal{P}}
\def\H{\mathcal{H}}
\def\C{\mathcal{C}}
\def\I{\mathcal{I}}
\def\p{\mathbf{p}}
\def\q{\mathbf{q}}
\def\r{\mathbf{r}}
\def\z{\mathbf{z}}
\def\a{\mathbf{a}}
\def\e{\mathbf{e}}
\def\x{\mathbf{x}}
\def\w{\mathbf{w}}
\def\y{\mathbf{y}}
\def\rows{\mathbf{x}}
\def\cols{\mathbf{y}}
\def\A{\mathbf{A}}
\def\Alg{\mathcal{A}}
\def\bfA{\mathbf{A}}
\def\Ax{\mathbf{Ax}}
\def\c{\mathbf{c}}
\def\b{\mathbf{b}}
\def\f{\mathbf{f}}
\def\u{\mathbf{u}}
\def\l{\mathbf{\ell}}
\def\ones{\mathbf{1}}
\def\var{\mbox{Var}}
\def\stddev{\mbox{StdDev}}
\def\daa{\text{dist}(\a,\a^*)}
\def\poly{\text{poly}}
\def\polylog{\text{polylog}}

\def\σ{\mathbf{\sigma}}
\def\β{\mathbf{\beta}}

\newcommand{\MaxCut }{{\sc MaxCut }}







\title{Stability and Recovery for Independence Systems}
\author{Vaggos Chatziafratis\thanks{Computer Science Department, Stanford University}\\{\tt vaggos@stanford.edu}\and Tim Roughgarden\footnotemark[1]\\{\tt tim@cs.stanford.edu}\and Jan Vondrak\thanks{Department of Mathematics, Stanford University}\\{\tt jvondrak@stanford.edu}}
\date{\today}

\begin{document}

\maketitle


\begin{abstract}
Two genres of
  heuristics that are frequently reported to perform much better on
  ``real-world'' instances than in the worst case are {\em greedy
    algorithms} and {\em local search algorithms}.  In this paper, we
  systematically study these two types of algorithms for the problem
  of maximizing a monotone submodular set function subject to
  downward-closed feasibility constraints.  We consider {\em
    perturbation-stable} instances, in the sense of Bilu and
  Linial~\cite{bilu2012stable}, and precisely identify the stability
  threshold beyond which these algorithms are guaranteed to recover
  the optimal solution.  Byproducts of our work include the first
  definition of perturbation-stability for non-additive objective
  functions, and a resolution of the worst-case approximation
  guarantee of local search in $p$-extendible systems.

\end{abstract}
\clearpage
% \leavevmode
% \\
% \\
% \\
% \\
% \\
\section{Introduction}
\label{introduction}

AutoML is the process by which machine learning models are built automatically for a new dataset. Given a dataset, AutoML systems perform a search over valid data transformations and learners, along with hyper-parameter optimization for each learner~\cite{VolcanoML}. Choosing the transformations and learners over which to search is our focus.
A significant number of systems mine from prior runs of pipelines over a set of datasets to choose transformers and learners that are effective with different types of datasets (e.g. \cite{NEURIPS2018_b59a51a3}, \cite{10.14778/3415478.3415542}, \cite{autosklearn}). Thus, they build a database by actually running different pipelines with a diverse set of datasets to estimate the accuracy of potential pipelines. Hence, they can be used to effectively reduce the search space. A new dataset, based on a set of features (meta-features) is then matched to this database to find the most plausible candidates for both learner selection and hyper-parameter tuning. This process of choosing starting points in the search space is called meta-learning for the cold start problem.  

Other meta-learning approaches include mining existing data science code and their associated datasets to learn from human expertise. The AL~\cite{al} system mined existing Kaggle notebooks using dynamic analysis, i.e., actually running the scripts, and showed that such a system has promise.  However, this meta-learning approach does not scale because it is onerous to execute a large number of pipeline scripts on datasets, preprocessing datasets is never trivial, and older scripts cease to run at all as software evolves. It is not surprising that AL therefore performed dynamic analysis on just nine datasets.

Our system, {\sysname}, provides a scalable meta-learning approach to leverage human expertise, using static analysis to mine pipelines from large repositories of scripts. Static analysis has the advantage of scaling to thousands or millions of scripts \cite{graph4code} easily, but lacks the performance data gathered by dynamic analysis. The {\sysname} meta-learning approach guides the learning process by a scalable dataset similarity search, based on dataset embeddings, to find the most similar datasets and the semantics of ML pipelines applied on them.  Many existing systems, such as Auto-Sklearn \cite{autosklearn} and AL \cite{al}, compute a set of meta-features for each dataset. We developed a deep neural network model to generate embeddings at the granularity of a dataset, e.g., a table or CSV file, to capture similarity at the level of an entire dataset rather than relying on a set of meta-features.
 
Because we use static analysis to capture the semantics of the meta-learning process, we have no mechanism to choose the \textbf{best} pipeline from many seen pipelines, unlike the dynamic execution case where one can rely on runtime to choose the best performing pipeline.  Observing that pipelines are basically workflow graphs, we use graph generator neural models to succinctly capture the statically-observed pipelines for a single dataset. In {\sysname}, we formulate learner selection as a graph generation problem to predict optimized pipelines based on pipelines seen in actual notebooks.

%. This formulation enables {\sysname} for effective pruning of the AutoML search space to predict optimized pipelines based on pipelines seen in actual notebooks.}
%We note that increasingly, state-of-the-art performance in AutoML systems is being generated by more complex pipelines such as Directed Acyclic Graphs (DAGs) \cite{piper} rather than the linear pipelines used in earlier systems.  
 
{\sysname} does learner and transformation selection, and hence is a component of an AutoML systems. To evaluate this component, we integrated it into two existing AutoML systems, FLAML \cite{flaml} and Auto-Sklearn \cite{autosklearn}.  
% We evaluate each system with and without {\sysname}.  
We chose FLAML because it does not yet have any meta-learning component for the cold start problem and instead allows user selection of learners and transformers. The authors of FLAML explicitly pointed to the fact that FLAML might benefit from a meta-learning component and pointed to it as a possibility for future work. For FLAML, if mining historical pipelines provides an advantage, we should improve its performance. We also picked Auto-Sklearn as it does have a learner selection component based on meta-features, as described earlier~\cite{autosklearn2}. For Auto-Sklearn, we should at least match performance if our static mining of pipelines can match their extensive database. For context, we also compared {\sysname} with the recent VolcanoML~\cite{VolcanoML}, which provides an efficient decomposition and execution strategy for the AutoML search space. In contrast, {\sysname} prunes the search space using our meta-learning model to perform hyperparameter optimization only for the most promising candidates. 

The contributions of this paper are the following:
\begin{itemize}
    \item Section ~\ref{sec:mining} defines a scalable meta-learning approach based on representation learning of mined ML pipeline semantics and datasets for over 100 datasets and ~11K Python scripts.  
    \newline
    \item Sections~\ref{sec:kgpipGen} formulates AutoML pipeline generation as a graph generation problem. {\sysname} predicts efficiently an optimized ML pipeline for an unseen dataset based on our meta-learning model.  To the best of our knowledge, {\sysname} is the first approach to formulate  AutoML pipeline generation in such a way.
    \newline
    \item Section~\ref{sec:eval} presents a comprehensive evaluation using a large collection of 121 datasets from major AutoML benchmarks and Kaggle. Our experimental results show that {\sysname} outperforms all existing AutoML systems and achieves state-of-the-art results on the majority of these datasets. {\sysname} significantly improves the performance of both FLAML and Auto-Sklearn in classification and regression tasks. We also outperformed AL in 75 out of 77 datasets and VolcanoML in 75  out of 121 datasets, including 44 datasets used only by VolcanoML~\cite{VolcanoML}.  On average, {\sysname} achieves scores that are statistically better than the means of all other systems. 
\end{itemize}


%This approach does not need to apply cleaning or transformation methods to handle different variances among datasets. Moreover, we do not need to deal with complex analysis, such as dynamic code analysis. Thus, our approach proved to be scalable, as discussed in Sections~\ref{sec:mining}.
%!TEX root = hopfwright.tex
%

In this section we systematically recast the Hopf bifurcation problem in Fourier space. 
We introduce appropriate scalings, sequence spaces of Fourier coefficients and convenient operators on these spaces. 
To study Equation~\eqref{eq:FourierSequenceEquation} we consider Fourier sequences $ \{a_k\}$ and fix a Banach space in which these sequences reside. It is indispensable for our analysis that this space have an algebraic structure. 
The Wiener algebra of absolutely summable Fourier series is a natural candidate, which we use with minor modifications. 
In numerical applications, weighted sequence spaces with algebraic and geometric decay have been used to great effect to study periodic solutions which are $C^k$ and analytic, respectively~\cite{lessard2010recent,hungria2016rigorous}. 
Although it follows from Lemma~\ref{l:analytic} that the Fourier coefficients of any solution decay exponentially, we choose to work in a space of less regularity. 
The reason is that by working in a space with less regularity, we are better able to connect our results with the global estimates in \cite{neumaier2014global}, see Theorem~\ref{thm:UniqunessNbd2}.


%
%
%\begin{remark}
%	Although it follows from Lemma~\ref{l:analytic} that the Fourier coefficients of any solution decay exponentially, we choose to work in a space of less regularity, namely summable Fourier coefficients. This allows us to draw SOME MORE INTERESTING CONCLUSION LATER.
%	EXPLAIN WHY WE CHOOSE A NORM WITH ALMOST NO DECAY!
%	% of s Periodic solutions to Wright's equation are known to be real analytic and so their  Fourier coefficients must decay geometrically [Nussbaum].
%	% We do not use such a strong result;  any periodic solution must be continuously differentiable, by which it follows that $ \sum | c_k| < \infty$.
%\end{remark}


\begin{remark}\label{r:a0}
There is considerable redundancy in Equation~\eqref{eq:FourierSequenceEquation}. First, since we are considering real-valued solutions $y$, we assume $\c_{-k}$ is the complex conjugate of $\c_k$. This symmetry implies it suffices to consider Equation~\eqref{eq:FourierSequenceEquation} for $k \geq 0$.
Second, we may effectively ignore the zeroth Fourier coefficient of any periodic solution \cite{jones1962existence}, since it is necessarily equal to $0$. 
%In \cite{jones1962existence}, it is shown that if $y \not\equiv -1$ is a periodic solution of~\eqref{eq:Wright} with frequency $\omega$, then $ \int_0^{2\pi/\omega} y(t) dt =0$. 
		The self contained argument is as follows. 
		As mentioned in the introduction, any periodic solution to Wright's equation must satisfy $ y(t) > -1$ for all $t$. 
	By dividing Equation~\eqref{eq:Wright} by $(1+y(t))$, which never vanishes, we obtain
	\[
	\frac{d}{dt} \log (1 + y(t)) = - \alpha y(t-1).
	\]  
	Integrating over one period $L$ we derive the condition 
	$0=\int_0^L y(t) dt $.
	Hence $a_0=0$ for any periodic solution. 
	It will be shown in Theorem~\ref{thm:FourierEquivalence1} that a related argument implies that we do not need to consider Equation~\eqref{eq:FourierSequenceEquation} for $k=0$.
\end{remark}

%%%
%%%
%%%\begin{remark}\label{r:c0} 
%%%In \cite{jones1962existence}, it is shown that if $y \not\equiv -1$ is a periodic solution of~\eqref{eq:Wright} with frequency $\omega$, then $ \int_0^{2\pi/\omega} y(t) dt =0$. 
%%%PERHAPS TOO MUCH DETAIL HERE. The self contained argument is as follows.
%%%If $y \not\equiv -1$ then $y(t) \neq -1$ for all $t$, since if $y(t_0)=-1$ for some $t_0 \in \R$ then $y'(t_0)=0$ by~\eqref{eq:Wright} and in fact by differentiating~\eqref{eq:Wright} repeatedly one obtains that all derivatives of $y$ vanish at $t_0$. Hence $y \equiv -1$ by Lemma~\ref{l:analytic}, a contradiction. Now divide~\eqref{eq:Wright} by $(1+y(t))$, which never vanishes, to obtain
%%%\[
%%%  \frac{d}{dt} \log |1 + y(t)| = - \alpha y(t-1).
%%%\]  
%%%Integrating over one period we obtain $\int_0^L y(t) dt =0$.
%%%\end{remark}



%Furthermore, the condition that $y(t)$ is real forces $\c_{-k} = \overline{\c}_{k}$.  
%
We define the spaces of absolutely summable Fourier series
\begin{alignat*}{1}
	\ell^1 &:= \left\{ \{ \c_k \}_{k \geq 1} : 
    \sum_{k \geq 1} | \c_k| < \infty  \right\} , \\
	\ell^1_\bi &:= \left\{ \{ \c_k \}_{k \in \Z} : 
    \sum_{k \in \Z} | \c_k| < \infty  \right\} .
\end{alignat*} 
We identify any semi-infinite sequence $ \{ \c_k \}_{k \geq 1} \in \ell^1$ with the bi-infinite sequence $ \{ \c_k \}_{k \in \Z} \in \ell^1_\bi$ via the conventions (see Remark~\ref{r:a0})
\begin{equation}
  \c_0=0 \qquad\text{ and }\qquad \c_{-k} = \c_{k}^*. 
\end{equation}
In other word, we identify $\ell^1$ with the set
\begin{equation*}
   \ell^1_\sym := \left\{ \c \in \ell^1_\bi : 
	\c_0=0,~\c_{-k}=\c_k^* \right\} .
\end{equation*}
On $\ell^1$ we introduce the norm
\begin{equation}\label{e:lnorm}
  \| \c \| = \| \c \|_{\ell^1} := 2 \sum_{k = 1}^\infty |\c_k|.
\end{equation}
The factor $2$ in this norm is chosen to have a Banach algebra estimate.
Indeed, for $\c, \tilde{\c} \in \ell^1 \cong \ell^1_\sym$ we define
the discrete convolution 
\[
\left[ \c * \tilde{\c} \right]_k = \sum_{\substack{k_1,k_2\in\Z\\ k_1 + k_2 = k}} \c_{k_1} \tilde{\c}_{k_2} .
\]
Although $[\c*\tilde{\c}]_0$ does not necessarily vanish, we have $\{\c*\tilde{\c}\}_{k \geq 1} \in \ell^1 $ and 
\begin{equation*}
	\| \c*\tilde{\c} \| \leq \| \c \| \cdot  \| \tilde{\c} \| 
	\qquad\text{for all } \c , \tilde{\c} \in \ell^1, 
\end{equation*}
hence $\ell^1$ with norm~\eqref{e:lnorm} is a Banach algebra.

By Lemma~\ref{l:analytic} it is clear that any periodic solution of~\eqref{eq:Wright} has a well-defined Fourier series $\c \in \ell^1_\bi$. 
The next theorem shows that in order to study periodic orbits to Wright's equation we only need to study Equation~\eqref{eq:FourierSequenceEquation} 
for $k \geq 1$. For convenience we introduce the notation 
\[
G(\alpha,\omega,\c)_k=
( i \omega k + \alpha e^{ - i \omega k}) \c_k + \alpha \sum_{k_1 + k_2 = k} e^{- i \omega k_1} \c_{k_1} \c_{k_2} \qquad \text{for } k \in \N.
\]
We note that we may interpret the trivial solution $y(t)\equiv 0$ as a periodic solution of arbitrary period.
\begin{theorem}
\label{thm:FourierEquivalence1}
Let $\alpha>0$ and $\omega>0$.
If $\c \in \ell^1 \cong \ell^1_{\sym}$ solves
$G(\alpha,\omega,\c)_k =0$  for all $k \geq 1$,
then $y(t)$ given by~\eqref{eq:FourierEquation} is a periodic solution of~\eqref{eq:Wright} with period~$2\pi/\omega$.
Vice versa, if $y(t)$ is a periodic solution of~\eqref{eq:Wright} with period~$2\pi/\omega$ then its Fourier coefficients $\c \in \ell^1_\bi$ lie in $\ell^1_\sym \cong \ell^1$ and solve $G(\alpha,\omega,\c)_k =0$ for all $k \geq 1$.
\end{theorem}

\begin{proof}	
	If $y(t)$ is a periodic solution of~\eqref{eq:Wright} then it is real analytic by Lemma~\ref{l:analytic}, hence its Fourier series $\c$ is well-defined and $\c \in \ell^1_{\sym}$ by Remark~\ref{r:a0}.
	Plugging the Fourier series~\eqref{eq:FourierEquation} into~\eqref{eq:Wright} one easily derives that $\c$ solves~\eqref{eq:FourierSequenceEquation} for all $k \geq 1$.

To prove the reverse implication, assume that $\c \in \ell^1_\sym$ solves
Equation~\eqref{eq:FourierSequenceEquation} for all $k \geq 1$. Since $\c_{-k}
= \c_k^*$, Equation \eqref{eq:FourierSequenceEquation} is also satisfied for
all $k \leq -1$. It follows from the Banach algebra property and
\eqref{eq:FourierSequenceEquation} that $\{k \c_k\}_{k \in \Z} \in \ell^1_\bi$,
hence $y$, given by~\eqref{eq:FourierEquation}, is continuously differentiable.
% (and by bootstrapping one infers that $\{k^m c_k \} \in \ell^1_\bi$, 
% hence $y \in C^m$ for any $m \geq 1$).
	Since~\eqref{eq:FourierSequenceEquation} is satisfied for all $k \in \Z \setminus \{0\}$ (but not necessarily for $k=0$) one may perform the inverse Fourier transform on~\eqref{eq:FourierSequenceEquation} to conclude that
	$y$ satisfies the delay equation 
\begin{equation}\label{eq:delaywithK}
   	y'(t) = - \alpha y(t-1) [ 1 + y(t)] + C
\end{equation}
	for some constant $C \in \R$. 
   Finally, to prove that $C=0$ we argue by contradiction.
   Suppose $C \neq 0$. Then $y(t) \neq -1$ for all $t$.
   Namely, at any point where $y(t_0) =-1$ one would have $y'(t_0) = C$
   which has fixed sign,   hence it would follow that $y$ is not periodic
   ($y$ would not be able to cross $-1$ in the opposite direction, 
   preventing $y$  from being periodic).  
  We may thus divide~\eqref{eq:delaywithK} through by $1 + y(t)$ and obtain 
\begin{equation*}
	\frac{d}{dt} \log | 1 + y(t) | = - \alpha y(t-1) + \frac{C}{1+y(t)} .
\end{equation*}
	By integrating both sides of the equation over one period $L$ and by using that $\c_0=0$, we 
	obtain
	\[
	 C \int_0^L \frac{1}{1+y(t)} dt =0.
	\]
	Since the integrand is either strictly negative or strictly positive, this implies that $C=0$. Hence~\eqref{eq:delaywithK} reduces to~\eqref{eq:Wright},
	and $y$ satisfies Wright's equation. 
\end{proof}






To efficiently study Equation~\eqref{eq:FourierSequenceEquation}, we introduce the following linear operators on $ \ell^1$:
\begin{alignat*}{1}
   [K \c ]_k &:= k^{-1} \c_k  , \\ 
   [ U_\omega \c ]_k &:= e^{-i k \omega} \c_k  .
\end{alignat*}
The map $K$ is a compact operator, and it has a densely defined inverse $K^{-1}$. The domain of $K^{-1}$ is denoted by
\[
  \ell^K := \{ \c \in \ell^1 : K^{-1} \c \in \ell^1 \}.  
\]
The map $U_{\omega}$ is a unitary operator on $\ell^1$, but
it is discontinuous in $\omega$. 
With this notation, Theorem~\ref{thm:FourierEquivalence1} implies that our problem of finding a SOPS to~\eqref{eq:Wright} is equivalent to finding an $\c \in \ell^1$ such that
\begin{equation}
\label{e:defG}
  G(\alpha,\omega,\c) :=
  \left( i \omega K^{-1} + \alpha U_\omega \right) \c + \alpha \left[U_\omega \, \c \right] * \c  = 0.
\end{equation}


%In order for the solutions of Equation \ref{eq:FHat} to be isolated we need to impose a phase condition. 
%If there is a sequence $ \{ c_k \} $ which satisfies  Equation \ref{eq:FHat}, then $ y( t + \tau) = \sum_{k \in \Z} c_k e^{ i k \omega (t + \tau)}$ satisfies Wright's equation at parameter $\alpha$. 
%Fix $ \tau = - Arg[c_1] / \omega$ so that $ c_1  e^{ i \omega \tau} $ is a nonnegative real number. 
%By Proposition \ref{thm:FourierEquivalence1} it follows that $\{ c'_k \} =  \{c_k e^{ i \omega k \tau }   \}$ is a solution to Equation \ref{eq:FHat}, and furthermore that $ c'_1 = \epsilon$ for some $ \epsilon \geq 0$. 


Periodic solutions are invariant under time translation: if $y(t)$ solves Wright's equation, then so does $ y(t+\tau)$ for any $\tau \in \R$. 
We remove this degeneracy by adding a phase condition. 
Without loss of generality, if $\c \in \ell^1$ solves Equation~\eqref{e:defG}, we may assume that $\c_1 = \epsilon$ for some 
\emph{real non-negative}~$\epsilon$:
\[
  \ell^1_{\epsilon} := \{\c \in \ell^1 : \c_1 = \epsilon \} 
  \qquad \text{where } \epsilon \in \R,  \epsilon \geq 0.
\]
In the rest of our analysis, we will split elements $\c \in \ell^1$ into two parts: $\c_1$ and $\{\c_{k}\}_{k \geq 2}$.  
We define the basis elements $\e_j \in \ell^1$ for $j=1,2,\dots$ as
\[
  [\e_j]_k = \begin{cases}
  1 & \text{if } k=j, \\
  0 & \text{if } k \neq j.
  \end{cases}
\]
We note that $\| \e_j \|=2$. 
Then we can decompose
% We define
% \[
%   \tilde{\epsilon} := (\epsilon,0,0,0,\dots) \in \ell^1
% \]
% and
% For clarity when referring to sequences $\{c_{k}\}_{k \geq 2}$, we make the following definition:
% \[
% \ell^1_0  := \{ \tc \in \ell^1 : \tc_1 = 0 \}.
% \]
% With the
any $\c \in \ell^1_\epsilon$ uniquely as
\begin{equation}\label{e:aepsc}
  \c= \epsilon \e_1 + \tc \qquad \text{with}\quad 
  \tc \in \ell^1_0 := \{ \tc \in \ell^1 : \tc_1 = 0 \}.
\end{equation}
We follow the classical approach in studying Hopf bifurcations and consider 
$\c_1 = \epsilon$ to be a parameter, and then find periodic solutions with Fourier modes in $\ell^1_{\epsilon}$.
This approach rewrites the function $G: \R^2 \times \ell^K \to \ell^1$ as a function $\tilde{F}_\epsilon : \R^2 \times \ell^K_0 \to \ell^1$, where 
we denote 
\[
\ell^K_0 := \ell^1_0 \cap \ell^K.
\]
% I AM ACTUALLY NOT SURE IF YOU WANT TO DEFINE THIS WITH RANGE IN $\ell^1$
% OR WITH DOMAIN IN $\ell^1_0$ ?? IT SEEMS TO DEPEND ON WHICH GLOBAL STATEMENT YOU WANT/NEED TO MAKE!?
\begin{definition}
We define the $\epsilon$-parameterized family of  functions $\tilde{F}_\epsilon: \R^2 \times \ell^K_0  \to \ell^1$ 
by 
\begin{equation}
\label{eq:fourieroperators}
\tilde{F}_{\epsilon}(\alpha,\omega, \tc) := 
\epsilon [i \omega + \alpha e^{-i \omega}] \e_1 + 
( i \omega K^{-1} + \alpha U_{\omega}) \tc + 
\epsilon^2 \alpha e^{-i \omega}  \e_2  +
\alpha \epsilon L_\omega \tc + 
\alpha  [ U_{\omega} \tc] * \tc ,
\end{equation}
where
$L_\omega : \ell^1_0 \to \ell^1$ is given by
\[
   L_{\omega} := \sigma^+( e^{- i \omega} I + U_{\omega}) + \sigma^-(e^{i \omega} I + U_{\omega}),
\]
with $I$ the identity and  $\sigma^\pm$ the shift operators on $\ell^1$:
\begin{alignat*}{2}
\left[ \sigma^- a \right]_k &:=  a_{k+1}  , \\
\left[ \sigma^+ a \right]_k &:=  a_{k-1}  &\qquad&\text{with the convention } \c_0=0.
\end{alignat*}
The operator $ L_\omega$ is discontinuous in $\omega$ and $ \| L_\omega \| \leq 4$. 
\end{definition} 

%The maps $ \sigma^{+}$ and $ \sigma^-$ are shift up and shift down operators respectively. 
We reformulate Theorem~\ref{thm:FourierEquivalence1}  in terms of the map  $\tilde{F}$. 
We note that it follows from Lemma~\ref{l:analytic} and 
%\marginpar{Reformulate}
%one's choice of  
Equation~\eqref{eq:FourierSequenceEquation}  
%or Equation ~\eqref{eq:fourieroperators},
that the Fourier coefficients of any periodic solution of~\eqref{eq:Wright} lie in $\ell^K$.
These observations are summarized in the following theorem.
\begin{theorem}
\label{thm:FourierEquivalence2}
	Let $ \epsilon \geq 0$,  $\tc \in \ell^K_0$, $\alpha>0$ and $ \omega >0$. 
	Define $y: \R\to \R$ as 
\begin{equation}\label{e:ytc}
	y(t) = 
	\epsilon \left( e^{i \omega t }  + e^{- i \omega t }\right) 
	+  \sum_{k = 2}^\infty   \tc_k e^{i \omega k t }  + \tc_k^* e^{- i \omega k t } .
\end{equation}
%	and suppose that $ y(t) > -1$. 
	Then $y(t)$ solves~\eqref{eq:Wright} if and only if $\tilde{F}_{\epsilon}( \alpha , \omega , \tc) = 0$. 
	Furthermore, up to time translation, any periodic solution of~\eqref{eq:Wright} with period $2\pi/\omega$ is described by a Fourier series of the form~\eqref{e:ytc} with $\epsilon \geq 0$ and $\tc \in \ell^K_0$.
\end{theorem}


%We note that for $\epsilon>0$ such solutions are truly periodic, while for $\epsilon=0$ a zero of $\tilde{F}_\epsilon$ may either correspond to a periodic solution or to the trivial solution $y(t) \equiv 0$. 



% \begin{proof}
%  By Proposition \ref{thm:FourierEquivalence1}, it suffices to show that $\tilde{F}(\alpha,\omega,c) =0$ is equivalent to Equation \ref{eq:FourierSequenceEquation} being satisfied for $k \geq 1$.
%  Since Equation \ref{eq:FourierSequenceEquation} is equivalent to Equation \ref{eq:FHat}, we expand  Equation \ref{eq:FHat} by writing $ \hat{c} = \hat{\epsilon } + c$  where $ \hat{\epsilon} := (\epsilon,0,0,\dots) \in \ell^1$ as below:
%  \begin{equation}
%  0=  \left( i \omega K^{-1} + \alpha U_\omega \right) (\hat{\epsilon}+ c) + \alpha \left[U_\omega \, (\hat{\epsilon}+ c) \right] * (\hat{\epsilon}+ c) \label{eq:Intial}
%  \end{equation}
%  The RHS of Equation \ref{eq:Intial} is $ \tilde{F}(\alpha,\omega,c)$, so the theorem is proved.
% \end{proof}



Since we want to analyze a Hopf bifurcation, we will want to solve $\tilde{F}_\epsilon = 0$ for small values of~$\epsilon$. 
However, at the bifurcation point, $ D \tilde{F}_0(\pp  ,\pp , 0)$ is not invertible.
In order for our asymptotic analysis to be non-degenerate,
we work with a rescaled version of the problem. To this end, for any $\epsilon >0$, we rescale both $\tc$ and $\tilde{F}$ as follows. Let $\tc = \epsilon c$ and 
\begin{equation}\label{e:changeofvariables}
  \tilde{F}_\epsilon (\alpha,\omega,\epsilon c) = \epsilon F_\epsilon (\alpha,\omega,c).
\end{equation}
For $\epsilon>0$ the problem then reduces to finding zeros of 
\begin{equation}
\label{eq:FDefinition}
	F_\epsilon(\alpha,\omega, c) := 
	[i \omega + \alpha e^{-i \omega}] \e_1 + 
	( i \omega K^{-1} + \alpha U_{\omega}) c + 
	\epsilon \alpha e^{-i \omega} \e_2  +
	\alpha \epsilon L_\omega c + 
	\alpha \epsilon [ U_{\omega} c] * c.
\end{equation}
We denote the triple $(\alpha,\omega,c) \in \R^2 \times \ell^1_0$ by $x$.
To pinpoint the components of $x$ we use the projection operators
\[
   \pi_\alpha x = \alpha, \quad \pi_\omega x = \omega, \quad 
  \pi_c x = c \qquad\text{for any } x=(\alpha,\omega,c).
\]

After the change of variables~\eqref{e:changeofvariables} we now have an invertible Jacobian $D F_0(\pp  ,\pp , 0)$ at the bifurcation point.
On the other hand, for $\epsilon=0$ the zero finding problems for $\tilde{F}_\epsilon$ and $F_\epsilon$ are not equivalent. 
However, it follows from the following lemma that any nontrivial periodic solution having $ \epsilon=0$ must have a relatively large size when $ \alpha $ and $ \omega $ are close to the bifurcation point. 

\begin{lemma}\label{lem:Cone}
	Fix $ \epsilon \geq 0$ and $\alpha,\omega >0$. 
	Let
	\[
	b_* :=  \frac{\omega}{\alpha} - \frac{1}{2} - \epsilon  \left(\frac{2}{3}+ \frac{1}{2}\sqrt{2 + 2 |\omega-\pp| } \right).
	\]
Assume that $b_*> \sqrt{2} \epsilon$. 
Define
% \begin{equation*}%\label{e:zstar}
% 	z^{\pm}_* :=b_* \pm \sqrt{(b_*)^2- \epsilon^2 } .
% \end{equation*}
% \note[J]{Proposed change to match Lemma E.4}
\begin{equation}\label{e:zstar}
z^{\pm}_* :=b_* \pm \sqrt{(b_*)^2- 2 \epsilon^2 } .
\end{equation}
If there exists a $\tc \in \ell^1_0$ such that $\tilde{F}_\epsilon(\alpha, \omega,\tc) = 0$, then \\
\mbox{}\quad\textup{(a)} either $ \|\tc\| \leq  z_*^-$ or $ \|\tc\| \geq z_*^+  $.\\
\mbox{}\quad\textup{(b)} 
$ \| K^{-1} \tc \| \leq (2\epsilon^2+ \|\tc\|^2) / b_*$. 
\end{lemma}
\begin{proof}
	The proof follows from Lemmas~\ref{lem:gamma} and~\ref{lem:thecone} in Appendix~\ref{appendix:aprioribounds}, combined with the observation that
$\frac{\omega}{\alpha} - \gamma \geq b_*$,
% \[
%   \frac{\omega}{\alpha} - \gamma \geq b_*
%  \qquad\text{for all }
% | \alpha - \pp| \leq r_\alpha \text{ and } 
%   | \omega - \pp| \leq r_\omega.
% \]
with $\gamma$ as defined in Lemma~\ref{lem:gamma}.
\end{proof}

\begin{remark}\label{r:smalleps}
We note that for $\alpha < 2\omega$
\begin{alignat*}{1}
z^+_* &\geq   \frac{2 \omega - \alpha}{\alpha} 
- \epsilon \left(4/3+\sqrt{2 + 2 |\omega-\pp| } \, \right) + \cO(\epsilon^2)
\\[1mm]
z^-_* & \leq   \cO(\epsilon^2)
\end{alignat*}
for small $\epsilon$. 
Hence Lemma~\ref{lem:Cone} implies that for values of $(\alpha,\omega)$ near $(\pp,\pp)$ any solution has either $\|\tc\|$ of order 1 or $\|\tc\| =  \cO(\epsilon^2)$. 
The asymptotically small term bounding $z_*^-$ is explicitly calculated in Lemma~\ref{lem:ZminusBound}. 
A related consequence is that for $\epsilon=0$ there are no nontrivial solutions 
of $\tilde{F}_0(\alpha,\omega,\tc)=0$ with 
$\| \tc \| < \frac{2 \omega - \alpha}{\alpha} $. 
\end{remark}

\begin{remark}\label{r:rhobound}
In Section~\ref{s:contraction} we will work on subsets of $\ell^K_0$ of the form
\[
  \ell_\rho := \{ c \in \ell^K_0 : \|K^{-1} c\| \leq \rho \} .
\]
Part (b) of Lemma~\ref{lem:Cone} will be used in Section~\ref{s:global} to guarantee that we are not missing any solutions by considering $\ell_\rho$ (for some specific choice of $\rho$) rather than the full space $\ell^K_0$.
In particular, we infer from Remark~\ref{r:smalleps} that  small solutions (meaning roughly that $\|\tc\| \to 0$ as $\epsilon \to 0$)
satisfy $\| K^{-1} \tc \| = \cO(\epsilon^2)$.
\end{remark}

The following theorem guarantees that near the bifurcation point the problem of finding all periodic solutions is equivalent to considering the rescaled problem $F_\epsilon(\alpha,\omega,c)=0$.
\begin{theorem}
\label{thm:FourierEquivalence3}
\textup{(a)} Let $ \epsilon > 0$,  $c \in \ell^K_0$, $\alpha>0$ and $ \omega >0$. 
	Define $y: \R\to \R$ as 
\begin{equation}\label{e:yc}
	y(t) = 
	\epsilon \left( e^{i \omega t }  + e^{- i \omega t }\right) 
	+ \epsilon  \sum_{k = 2}^\infty   c_k e^{i \omega k t }  + c_k^* e^{- i \omega k t } .
\end{equation}
%	and suppose that $ y(t) > -1$. 
	Then $y(t)$ solves~\eqref{eq:Wright} if and only if $F_{\epsilon}( \alpha , \omega , c) = 0$.\\
\textup{(b)}
Let $y(t) \not\equiv 0$ be a periodic solution of~\eqref{eq:Wright} of period $2\pi/\omega$
 with Fourier coefficients $\c$.
Suppose $\alpha < 2\omega$ and $\| \c \| < \frac{2 \omega - \alpha}{\alpha} $.
Then, up to time translation, $y(t)$ is described by a Fourier series of the form~\eqref{e:yc} with $\epsilon > 0$ and $c \in \ell^K_0$.
\end{theorem}

\begin{proof}
Part (a) follows directly from Theorem~\ref{thm:FourierEquivalence2} and the  change of variables~\eqref{e:changeofvariables}.
To prove part (b) we need to exclude the possibility that there is a nontrivial solution with $\epsilon=0$. The asserted bound on the ratio of $\alpha$ and $\omega$ guarantees, by Lemma~\ref{lem:Cone} (see also Remark~\ref{r:smalleps}), that indeed $\epsilon>0$ for any nontrivial solution. 
\end{proof}

We note that in practice (see Section~\ref{s:global}) a bound on $\| \c \|$ is derived from a bound on $y$ or $y'$ using Parseval's identity.

\begin{remark}\label{r:cone}
It follows from Theorem~\ref{thm:FourierEquivalence3} and Remark~\ref{r:smalleps} that for values of $(\alpha,\omega)$ near $(\pp,\pp)$ any reasonably bounded solution satisfies $\| c\| =  O(\epsilon)$ as well as $\|K^{-1} c \| = O(\epsilon)$ asymptotically (as $\epsilon \to 0$).
These bounds will be made explicit (and non-asymptotic) for specific choices of the parameters in Section~\ref{s:global}.
\end{remark}

% We are able to rule out such large amplitude solutions using global estimates such as those in \cite{neumaier2014global}.
% Hence, near the bifurcation point, the problem of describing periodic solutions of~\eqref{eq:Wright} reduces to studying the family of zeros finding problems $F_\epsilon=0$.





%Specifically, if a solution having $ \epsilon = 0$ does in fact correspond to a nontrivial periodic solution and $\alpha  < 2\omega $, then $ \| \tilde{c} \| > 2 \omega \alpha^{-1} -1$. 
%%PERHAPS THIS NEEDS A FORMULATION AS A THEOREM AS WELL?
%%IN OTHER WORDS: ARE WE SURE WE HAVE FOUND ALL ZEROS OF $\tilde{F}_0$, I.E. ALL SOLUTIONS WITH $\epsilon=0$ NEAR THE BIFURCATION POINT? AFTER RESCALING THESE ARE INVISIBLE?
%%THERE SHOULD BE A STATEMENT ABOUT THIS SOMEWHERE! EITHER HERE OR SOME





We finish this section by defining a curve of approximate zeros $\bx_\epsilon$ of $F_\epsilon$ 
(see \cite{chow1977integral,hassard1981theory}). 
%(see \cite{chow1977integral,morris1976perturbative,hassard1981theory}). 


\begin{definition}\label{def:xepsilon}
Let
\begin{alignat*}{1}
	\balpha_\epsilon &:= \pp + \tfrac{\epsilon^2}{5} ( \tfrac{3\pi}{2} -1)  \\
	\bomega_\epsilon &:= \pp -  \tfrac{\epsilon^2}{5} \\
	\bc_\epsilon 	 &:= \left(\tfrac{2 - i}{5}\right) \epsilon \,  \e_2 \,.
\end{alignat*}
We define the approximate solution 
$ \bx_\epsilon := \left( \balpha_\epsilon , \bomega_\epsilon  , \bc_\epsilon \right)$
for all $\epsilon \geq 0$.
\end{definition}

We leave it to the reader to verify that both 
 $F_\epsilon(\pp,\pp,\bc_{\epsilon})=\cO(\epsilon^2)$ and $F_\epsilon(\bx_\epsilon)=\cO(\epsilon^2)$.
%%%	
%%%	
%%%	}{Better like this?}
%%%\annote[J]{ $F_\epsilon(\bx_0)=\cO(\epsilon^2)$ and $F_\epsilon(\bx_\epsilon)=\cO(\epsilon^2)$.}{I think we'd still need the $ \bar{c}_\epsilon$ term in $\bar{x}_0$ to be of order $ \epsilon$.}
%%%\remove[JB]{We show in Proposition A.1
%%%%\ref{prop:ApproximateSolutionWorks} 
%%% that any $ x \in \R^2 \times \ell^1_0$ which is $ \cO(\epsilon^2)$ close to $ \bar{x}_\epsilon $ will yield the estimate $F_\epsilon(x) = \cO(\epsilon^2)$.
%%%Hence choosing $\{ \pp , \pp, \bar{c}_\epsilon\}$ as our approximate solution would also have been a natural choice for performing an $\cO(\epsilon^2)$ analysis and would have simplified several of our calculations.
%%%However,} 
%%%
We choose to use the more accurate approximation 
for the $ \alpha$ and $ \omega $ components to improve our final quantitative results. 














%
% Values for $ (\alpha, \omega,c)$ which approximately solve $\tilde{F}(\alpha,\omega,c) = 0$  are computed in  \cite{chow1977integral,morris1976perturbative,hassard1981theory} and are as follows:
%  \begin{eqnarray}
%  \tilde{\alpha}( \epsilon) &:=& \pi /2 + \tfrac{\epsilon^2}{5} ( \tfrac{3\pi}{2} -1) \nonumber \\
%  \tilde{\omega}( \epsilon) &:=& \pi /2 -  \tfrac{\epsilon^2}{5} \label{eq:ScaleApprox} \\
%  \tc(\epsilon) 	  &:=& \{ \left(\tfrac{2 - i}{5}\right)  \epsilon^2 , 0,0, \dots \} \nonumber
%  \end{eqnarray}
% In Appendix \ref{sec:OperatorNorms} we illustrate an alternative method for deriving this approximation.
%
%
%
%
% We want to solve $ \tilde{F}(\alpha , \omega, \hat{c}) =0$ for small values of $ \epsilon$.
% However $ D \tilde{F}(\alpha , \omega , c)$ is not invertible at $ ( \pp , \pp , 0)$ when $ \epsilon = 0$.
% In order for our asymptotic analysis to be non-degenerate, we need to make the change of variables $ c \mapsto \epsilon c$.
% Under this change of variables, we define the function $ F$ below so that $ \tilde{F}(\alpha , \omega , \epsilon c) =\epsilon  F( \alpha , \omega , c)$.
%
%
%
% \begin{definition}
% Construct an $\epsilon$-parameterized family of densely defined functions  $F : \R^2 \oplus \ell^1 / \C \to \ell^1$ by:
% \begin{equation}
% \label{eq:FDefinition}
% 	F(\alpha,\omega, c) :=
% 	[i \omega + \alpha e^{-i \omega}]_1 +
% 	( i \omega K^{-1} + \alpha U_{\omega}) c +
% 	[\epsilon \alpha e^{-i \omega}]_2  +
% 	\alpha \epsilon L_\omega c +
% 	\alpha \epsilon [ U_{\omega} c] * c.
% \end{equation}
% \end{definition}

%%
%%
%%\begin{corollary}
%%	\label{thm:FourierEquivalence3}
%%	Fix $ \epsilon > 0$, and $ c \in \ell^1 / \C $, and $ \omega >0$. Define $y: \R\to \R$ as 
%%	\[
%%	y(t) = 
%%	\epsilon \left( e^{i \omega t }  + e^{- i \omega t }\right) 
%%	+  \epsilon  \left( \sum_{k = 2}^\infty   c_k e^{i \omega k t }  + \overline{c}_k e^{- i \omega k t } \right) 
%%	\]
%%	and suppose that $ y(t) > -1$. 
%%	Then $y(t)$ solves Wright's equation at parameter $ \alpha > 0 $ if and only if $ F( \alpha , \omega , c) = 0$ at parameter $ \epsilon$. 
%%	
%%	
%%	
%%\end{corollary}
%%
%%
%%\begin{proof}
%%	Since $ \tilde{F}(\alpha,\omega, \epsilon c) = \epsilon F( \alpha , \omega , c)$, the result follows from Theorem \ref{thm:FourierEquivalence2}.
%%\end{proof}

% If we can find $(\alpha , \omega, c)$ for which $ F( \alpha , \omega,c)=0$ at parameter $\epsilon$, then $ \tilde{F}(\alpha ,\omega, c)=0$.
% By Theorem \ref{thm:FourierEquivalence2} this amounts to finding a periodic solution to Wright's equation.
% Lastly, because we have performed the change of variables $ c \mapsto \epsilon c$, we need to  apply this change of variables to our approximate solution as well.
%
% \begin{definition}
% 	Define the approximate solution $ x( \epsilon) = \left\{ \alpha(\epsilon ) , \omega ( \epsilon ) , c(\epsilon) \right\}$ as below,  where $c(\epsilon) = \{ c_2( \epsilon) , 0 ,0 , \dots\} $.
% 	We may also write $ x_\epsilon = x(\epsilon) $.
% 	\begin{eqnarray}
% 	\alpha( \epsilon) &:=& \pi /2 + \tfrac{\epsilon^2}{5} ( \tfrac{3\pi}{2} -1) \nonumber \\
% 	\omega( \epsilon) &:=& \pi /2 -  \tfrac{\epsilon^2}{5} \label{eq:Approx} \\
% 	c_2(\epsilon) 	  &:=& \left(\tfrac{2 - i}{5}\right) \epsilon \nonumber
% 	\end{eqnarray}
%
% \end{definition}

\subsection{Additive Valuations: Polynomial-Time Learning}\label{sec:additive}
In this section, we consider bidders with additive valuations, again sharpening our results to show polynomial-time learnability. It is known that the better of the following two mechanisms achieves at least $\frac{1}{8}$ of the optimal revenue when all bidders have additive valuations~\cite{Yao15,CaiDW16}:

\vspace{.05in}	
\noindent\textbf{Selling Separately}: the mechanism sells each item separately using Myerson's optimal auction.

\vspace{.05in}	
\noindent \textbf{VCG with Entry Fee}: the mechanism solicits bids $\bold{b}=(b_1,\cdots, b_n)$ from the bidders, then offers each bidder $i$ the option to participate for an entry fee $e_i(b_{-i},D_i)$, which is the median of the random variable $\sum_{j\in[m]}(t_{ij}-\max_{k\neq i} b_{kj})^+$, where $t_i\sim D_i$\footnote{The entry fee function defined in~\cite{Yao15,CaiDW16} is slightly different. They showed that there exists an entry fee $X_i$, such that bidder $i$ accepts the entry fee with probability at least $1/2$. Then they argued that extracting $X_i/2$ as the revenue in the VCG with entry fee mechanism is enough to obtain a factor $8$ approximation. It is not hard to observe that our entry fee is accepted with probability exactly $1/2$, thus our entry fee is at least as large as $X_i$. So our mechanism also suffices to provide a factor $8$ approximation.}. This random variable is exactly bidder $i$'s utility when her type is $t_i$ and the other bidders' are $b_{-i}$. If bidder $i$ chooses to participate, she pays the entry fee and can take any item $j$ at price $\max_{k\neq i} b_{kj}$. Notice that the mechanism never over allocate any item, as only the highest bidder for an item can afford it. %Moreover, this mechanism is DSIC, because the entry fee and item prices for bidder $i$ only depend on the other bidders' bids and bidder $i$'s type distribution but not her bid.

Indeed, only counting the revenue from the entry fee in the second mechanism and the optimal revenue from selling the items separately already suffices to provide an $8$-approximation~\cite{Yao15, CaiDW16}. 

\begin{theorem}[\cite{CaiDW16}]\label{thm:UB additive}
	Let $\srev$ be the optimal revenue for selling the items separately and $\brev$ be the expected entry fee collected from the VCG with entry fee mechanism. Then $\opt\leq 6\cdot \srev+2\cdot\brev.$ 
\end{theorem}

Goldner and Karlin~\cite{GoldnerK16} showed that one sample suffices to learn a mechanism that achieves a constant fraction of the optimal revenue when $D_{ij}$ is regular for all $i\in[n]$ and $j\in[m]$. %In the rest of the section, we discuss
We show how to learn an approximately optimal mechanism in the other two models.
% (1) all $D_{ij}$ are supported on $[0,H]$, and (2) direct access to distributions $\hat{D}_{ij}$, where $||\hat{D}_{ij}-D_{ij}||_K\leq \epsilon$ for all $i\in[n]$ and $j\in[m]$.
\begin{theorem}\label{thm:additive}
	When the bidders have additive valuations and\begin{itemize}
		\item $D_{ij}$ is supported on $[0,H]$ for all bidder $i$ and item $j$, we can learn in polynomial time a mechanism whose expected revenue is at least $\frac{\opt}{32}-{\epsilon}\cdot H$ with probability $1-\delta$ given $O\left(\left(\frac{m}{\epsilon}\right)^2 \cdot\left(n\log n\log \frac{1}{\epsilon}+\log\frac{1}{\delta} \right)\right)$ samples from $D$; or
		\item  we are only given access to distributions $\hat{D}_{ij}$ where $||\hat{D}_{ij}-D_{ij}||_K\leq \epsilon$ for all bidder $i$ and item $j$, there is a polynomial time algorithm that constructs a mechanism whose expected revenue under $D$ is at least $\frac{\opt}{266}-96\epsilon\cdot mnH$ when $\epsilon\leq \frac{1}{16\max\{m,n\}}$.
	\end{itemize}
\end{theorem}

\noindent\textbf{Sample Access to Bounded Distributions:} Goldner and Karlin's proof~\cite{GoldnerK16} can be directly applied to the bounded distributions to show a single sample suffices to learn a mechanism whose expected revenue approximates the $\brev$. Then as $\srev$ is the revenue of $m$ separate single-item auctions, we can use the result in~\cite{MorgensternR16} to approximate it. See Theorem~\ref{thm:additive bounded} in Appendix~\ref{sec:additive bounded} for further details.

\vspace{.05in}
\noindent\textbf{Direct Access to Approximate Distributions:} for each single item, we apply Theorem~\ref{thm:unit-demand} to learn an individual auction, then run these learned auctions in parallel. Clearly, the combined auction's revenue approximates $\srev$. For $\brev$, we show that for every bidder $i$ and every bid profile $b_{-i}$ of the other bidders, the event that corresponds to bidder $i$ accepting any entry fee is \emph{single-intersecting} (see Definition~\ref{def:single-intersecting}). This implies that the probability for a bidder to accept an entry fee under $\hat{D}$ and $D$ is close (Lemma~\ref{lem:Kolmogorov stable for sc}). So we can essentially use the median of $\sum_{j\in[m]}(t_{ij}-\max_{k\neq i} b_{kj})^+$ with $t_i\sim\hat{D}_i$ as the entry fee. See Theorem~\ref{thm:additive Kolmogorov} in Appendix~\ref{sec:additive Kolmogorov} for further details.
\section{Recurrent Submodular Welfare}
Let $f(S): 2^{\A} \rightarrow \mathbb{R}_{\geq 0}$ be a monotone submodular function over a universe $\A$ of $k$ elements, such that $f(\emptyset) = 0$. In the {\em blocking} setting, each element $i \in \A$ is associated with a known deterministic {\em delay} $d_i \in \mathbb{N}_{>0}$, such that once the arm is played at some round $t$, it becomes unavailable for the next $d_i-1$ rounds, namely, in the interval $\{t, \dots, t+d_i-1\}$. At each round $t \in [T]$, the player chooses a subset $\A_t$ of available (i.e., non-blocked) elements and collects a reward $f(\A_t)$. The goal is to maximize the total reward collected, i.e., $\sum_{t \in [T]} f(\A_t)$, within an unknown time horizon $T$. 

Before we present our algorithm, we provide ``bad'' instances for two natural approaches to \rsm.

\begin{remark} \label{rem:greedy}
The greedy approach of choosing $\A_t$ to be the set of all available elements at round $t \in [T]$ can be as bad as a $\frac{1}{k}$-approximation. In order to see that, consider the monotone (budget-additive) submodular function $f(S) = \min\{|S|, 1\}$. Let $k$ be the number of elements with delay $d_i = k$ for each $i \in \A$. Assuming an infinite time horizon, the optimal strategy collects an average reward of $1$, simply by choosing one element at a time in a round-robin manner. However, the average reward of the greedy approach in this case is $\frac{1}{k}$.
\end{remark}

\begin{remark}
The independent randomized sampling approach of adding each arm $i$ to $\A_t$ independently with probability $\frac{1}{d_i}$, if available, can be as bad as a $(1 - \frac{1}{\sqrt{e}} )$-approximation. Consider the same setting as in Remark \ref{rem:greedy}, where for $T \to \infty$ the optimal average reward is $1$. However, the average expected reward of the independent randomized sampling strategy is $1 - (1 - p)^k$, where $p = \frac{1}{2k-1}$ is the probability that each element is selected at each round (in stationarity). For $k \to \infty$, we have that $1 - (1 - p)^k \to 1- e^{-\frac{1}{2}} \approx 0.393$.

\end{remark}
We provide an efficient randomized $\left(1-\frac{1}{e}\right)$-approximation algorithm for \rsm. Informally, the algorithm starts by considering, for each element $i \in \A$, a sequence of rational numbers of the form $\{t\cdot \frac{1}{d_i}\}_{t \in [T]}$. Then, these sequences are {\em interleaved} by randomly adding an {\em offset} $r_i$, drawn uniformly at random from $[0,1]$, for each $i \in \A$ to the corresponding sequence. At every round $t \in [T]$, the algorithm chooses a set $\A_t$, consisting only of elements for which the (perturbed) interval $L_{i,t} = [t\cdot \frac{1}{d_i}+ r_i, (t+1)\cdot \frac{1}{d_i}+ r_i )$ contains an integer.

\begin{algorithm}[\is (\IS)]
For each element $i \in \A$, let $r_i \sim U[0,1]$ be a random {\em offset} drawn uniformly from $[0,1]$. 
At every round $t = 1, 2, \dots$,  let $\A_t \subseteq \A$ be the subset of elements such that for any $i \in \A_t$, the interval $L_{i,t} = [t\cdot \frac{1}{d_i} + r_i, (t+1) \cdot \frac{1}{d_i} + r_i)$ contains an integer. Choose the elements $\A_t$ and collect the reward $f(\A_t)$.
\end{algorithm}


\subsection{Correctness and approximation guarantee.} 
We first show the algorithm is correct, namely, that the elements chosen at each round respect the blocking constraints. The correctness is established by the following simple observation:

\begin{restatable}{fact}{restatefactalwaysavailable}\label{inter:fact:alwaysavailable}
At any $t \in [T]$, all the elements in $\A_t$ are available (i.e., not blocked).
\end{restatable}

In order to prove the competitive guarantee of our algorithm, we first construct a convex programming (CP)-based (approximate) upper bound on the optimal reward. Although our algorithm never computes an optimal solution to this CP, this step allows us to prove our guarantee, leveraging results on the correlation gap of submodular functions. For $\bm{d}^{-1} \in \mathbb{R}^k$ such that $(\bm{d}^{-1})_i = \frac{1}{d_i}, \forall i \in [k]$, consider the following formulation based on the concave closure $f^+$ of $f$:
\begin{align}
\maximize_{\z \in \mathbb{R}^k}~~ T \cdot f^+(\z)~~\textbf{s.t.}~~ \bm{0} \preceq \z \preceq \bm{d}^{-1}. \tag{\textbf{CP}} \label{cp:CP}
\end{align}

In \eqref{cp:CP}, each variable $z_{i}$ can be thought of as the fraction of rounds where element $i\in \A$ is chosen. Intuitively, the constraints indicate the fact that, due to the blocking, each element $i \in \A$ can be played at most once every $d_i$ steps. 
In order to derive \eqref{cp:CP}, we start from a non-convex integer program (IP) with 0-1 variables $\{x_{i,t}\}_{i \in \A, t \in [T]}$, each indicating whether element $i \in \A$ is used at round $t \in [T]$. The objective is to maximize $\sum_{t \in [T]} \sum_{S \subseteq \A} f(S) \prod_{i \in S} x_{i,t} \prod_{i \notin S}(1 - x_{i,t})$ subject to natural blocking constraints. For integral solutions, the above objective is equivalent to $\sum_{t \in [T]} f^+(\x_t)$ (where $(\x_t)_i = x_{i,t}$) and, thus, the above relaxation is simply the result of averaging over time the variables and constraints of this IP. By using the concavity of $f^+$, we are able to show that \eqref{cp:CP} yields an (approximate) upper bound on the optimal solution of \rsm, while the approximation becomes exact as $T$ increases.

\begin{restatable}{lemma}{restateStructuralCP}\label{lem:structural:CP}
Let $\Rew^{CP}(T)$ be the optimal solution to \eqref{cp:CP} and $\OPT(T)$ be the optimal solution over $T$ rounds. We have
$
\Rew^{CP}(T) \geq \OPT(T) - \mathcal{O}(d_{\max} f(\A)),
$ where $d_{\max} = \max_{i \in \A}\{d_i\}$.
\end{restatable}

\begin{remark}
By replacing $f^+(\z)$ in \eqref{cp:CP} with the multi-linear extension $F(\z)$, the formulation no longer yields an upper bound on the optimal reward (not even asymptotically). Indeed, consider a function $f$ over a ground set $\A=\{1,2\}$ with $d_1 = d_2 = 2$, such that $f(\emptyset) = 0$, $f(\{1\}) = f(\{2\}) = 2$ and $f(\{1,2\}) = 3$. For $T \to \infty$, the optimal average reward is $2$, simply by choosing the two elements interchangeably. However, the formulation based on $F(\z)$ in that case would be to maximize $2z_1(1-z_2) + 2z_2(1-z_1) + 3 z_1 z_2$ subject to $z_1,z_2 \leq \frac{1}{2}$, which has a global maximum of $\frac{7}{4} < 2$.
\end{remark}


Before we complete the proof of our first main result, we first compute the probability that $i \in \A_t$, i.e., an element $i \in \A$ is sampled at round $t \in [T]$:

\begin{restatable}{fact}{restatefactsampling}\label{inter:fact:sampling}
For any $i \in \A$ and $t \in [T]$, we have
$\Pro{i \in \A_t} = \Pro{L_{i,t} \cap \mathbb{N} \neq \emptyset } = \frac{1}{d_i}.
$
\end{restatable}

\noindent{\em Proof of Theorem \ref{thm:interleavedSubmodular}.} 
Let us denote by $S \sim {\bf p}$ with ${\bf p} \in [0,1]^k$ the random set $S \subseteq \A$, where each element $i$ participates in $S$ independently with probability equal to $p_i$. 
By Fact~\ref{inter:fact:sampling} and due to the randomness of the offsets $\{r_i\}_{i \in \A}$, we have that $\A_t \sim {\bf d}^{-1}$ for each $t \in [T]$. Let $\z^*$ be an optimal solution to \eqref{cp:CP}. By monotonicity of $f$ and the fact that $\z^* \preceq \bm{d}^{-1}$, for the expected value of $f(\A_t)$ at any round $t \in [T]$, we know that $\Ex{\A_t \sim \bm{d}^{-1}}{f(\A_t)} \geq \Ex{\A_t \sim \z^*}{f(\A_t)}$. Moreover, by definition of the multi-linear extension, we have that $\Ex{\A_t \sim \z^*}{f(\A_t)} = F(\z^*)$, while by Lemma~\ref{lem:correlationgap} (the correlation gap of submodular functions), we have that, $F(\z) \geq \left(1 - \frac{1}{e}\right)f^+(\z)$ for any vector $\z \in [0,1]^k$. By combining the above facts, we can see that
\begin{align*}
\Rew^{IS}(T) = \sum_{t \in [T]} \Ex{\A_t \sim \bm{d}^{-1}}{f(\A_t)} \geq 
\sum_{t \in [T]} F(\z^*) \geq \left(1 - \frac{1}{e}\right)T\cdot f^+(\z^*) = \left(1 - \frac{1}{e}\right) \Rew^{CP}(T).
\end{align*}
Therefore, by Lemma~\ref{lem:structural:CP}, we can conclude that $\Rew^{IS}(T) \geq \left(1 - \frac{1}{e}\right)\OPT(T) - \mathcal{O}(d_{\max} f(\A))$.
\qed
\newline

In Appendix \ref{appendix:hardness}, we provide a $\left(1-\frac{1}{e}\right)$-hardness result for \rsm, thus proving that the guarantee of Theorem~\ref{thm:interleavedSubmodular} is asymptotically tight. This result, which holds even for the special case where $d_{\max} = o(T)$ (that is when the delays are significantly smaller than the time horizon), is proved via a reduction from the SWM problem with identical utilities, in a way that the constructed \rsm instance accepts w.l.o.g. solutions of a simple periodic structure.

\begin{restatable}{theorem}{restateSubmodularHardness}\label{thm:submodular:hardness}
For any $\epsilon>0$, there exists no polynomial-time $\left(1-\frac{1}{e} + \epsilon \right)$-approximation algorithm for the \rsm problem, unless ${\bf P}={\bf NP}$, even in the special case where $d_{\max} = o(T)$.
\end{restatable}
\section{Local Search Performance}\label{sec:local search}


In this section we discuss {\em local search}~\cite{lenstra2003local} (described in \hyperref[sec:preliminaries]{Section~\ref{sec:preliminaries}}).
Local search often gives better results than Greedy, at the cost of a slower running time --- for example for submodular maximization subject to the intersection of $k$ matroids \cite{lee2009submodular,filmus2012power}, and for $k$-set packing \cite{SviridenkoW13,Cygan13,FurerY14}. For some interesting recent results about local search in \textit{beyond-worst-case} settings and on geometric optimization we refer the reader to~\cite{cohen2016local,cohen2014unreasonable,cohen17}.

Somewhat surprisingly, it was not known (to our knowledge) how local search performs for $p$-systems and $p$-extendible systems. (We recall that the greedy algorithm gives a factor of $1/p$ for maximization of an additive function and $1/(p+1)$ for maximization of a monotone submodular function under these constraints.)
Here, we prove that local search in fact performs worse than Greedy for these constraints. Although it gives a $1/p$-approximation for cardinality maximization under a $p$-system constraint (essentially by definition), it does not give any bounded approximation factor for additive function maximization under a $p$-system, and only a $1/p^2$-approximation under a $p$-extendible system. 


\subsection{Local search fails for $p$-systems}
We construct simple examples where local search will not recover any fraction of the maximum-weight solution for $p$-systems (even if it is arbitrarily stable, $p=2$, and even if we allow large exchange neighborhoods). In particular, consider a ground set $X = A \cup \{e^*\}$ where $|A|=n$. The independent sets of $\I$ are:
\begin{itemize}
\item any subset of $A$, or
\item $e^*$ plus any subset of at most $n/2$ elements of A. 
\end{itemize}

Note that this is a 2-system, because for $S \subseteq X$, any independent subset of $S$ can be extended to an independent set of size at least $\min \{|S|, n/2\}$, and the maximum independent subset of $S$ has size at most $\min \{|S|,n\}$. The weights could be 0 on $A$, and 1 on the special element $e^*$. So the optimum is $w(e^*)$ = 1 (observe that the optimal solution is $c$-stable for arbitrarily large $c$). However, $A$ is a local optimum, unless we are willing to swap out $n/2$ elements, which is not possible for efficient local search.



\subsection{Lower bound for $p$-extendible systems}

Let us consider the following instance. Let $X = A \cup B$ where $A, B$ are disjoint sets. We define $\I \subseteq 2^X$ as follows: $S \in \I$ iff
\begin{itemize}
\item $|S \cap A| + p |S \cap B| \leq |A|$, or
\item $p|S \cap A| + |S \cap B| \leq |B|$.
\end{itemize}

\begin{lemma}
For any $A,B$ disjoint, the above is a $p$-extendible system.
\end{lemma}

\begin{proof}
Let $S \subseteq T$ and $i \in X \setminus T$ be such that $S+i \in \I$ and $T \in \I$. We need to prove that there is $Z \subseteq T \setminus S, |Z| \leq p$ such that $(T \setminus Z) + i \in \I$. We can assume that $|T \setminus S| > p$, because otherwise we can set $Z = T \setminus S$ and obviously $(T \setminus Z) + i = S + i \in \I$.
Assuming $|T \setminus S| > p$, let $Z$ be an arbitrary set of $p$ elements from $T \setminus S$. We consider 2 cases: If $|T \cap A| + p |T \cap B| \leq |A|$, then $|(T \setminus Z) \cap A| + p |(T \setminus Z) \cap B| \leq |A| - p$. Adding the element $i$ can increase the left-hand side by at most $p$, and so $|(T \setminus Z + i) \cap A| + p |(T \setminus Z + i) \cap B| \leq |A|$. Similarly, in the second case, if $p |T \cap A| + |T \cap B| \leq |B|$, then $p |(T \setminus Z) \cap A| + |(T \setminus Z) \cap B| \leq |B| - p$. Adding the element $i$ can increase the left-hand side by at most $p$, and so $p |(T \setminus Z + i) \cap A| + |(T \setminus Z + i) \cap B| \leq |B|$. 
\end{proof}

Now we choose the cardinalities of $A$ and $B$ and the weights of their elements appropriately to get a negative result.

\begin{lemma}
For $\epsilon>0$, let $|A| = n$ and $|B| = (p - \epsilon) n$, and set the weights as $w_a = 1$ for $a \in A$ and $w_b = p - \epsilon$ for $b \in B$.
Then $A$ is a local optimum of value $w(A) = w(B) / (p - \epsilon)^2$, unless the local search explores exchanges of size at least $\frac{\epsilon}{p} n$.
\end{lemma}

\begin{proof}
Both $A$ and $B$ are independent sets. 
Note that for any $i \in B$, we need to remove $Z \subseteq A$ of cardinality at least $|Z| = p$ to obtain $S = (A \setminus Z) + i$ satisfying $|S \cap A| + p|S \cap B| \leq |A|$. More generally, for $Y \subseteq B$, we need to remove $Z \subseteq A, |Z| = p|Y|$ to obtain $S = (A \setminus Z) \cup Y$ that satisfies $|S \cap A| + p|S \cap B| \leq |A|$. Possibly, we could satisfy the second condition, $p|S \cap A| + |S \cap B| \leq |B|$, but this will not happen unless $|A \setminus Z| = |S \cap A| \leq |B| / p = (1 - \frac{\epsilon}{p}) n$. Therefore, we would need to remove $Z$ of cardinality at least $\frac{\epsilon}{p} n$. If the swaps considered are smaller than $\frac{\epsilon}{p} n$ then $A$ is a local optimum because adding $Y \subseteq B$ and removing $Z \subseteq A, |Z| = p |Y|$ results in a solution of lower weight. In conclusion, $A$ is a local optimum of value $w(A) = n$, while the optimum is $OPT = w(B) = (p-\epsilon)^2 n$.
\end{proof}


\subsection{Upper bound for $p$-extendible systems}

Here we prove that local search does in fact provide a $1/p^2$-approximation for weighted maximization under a $p$-extendible system. More generally, we prove (here, we will ignore the technicalities of stopping the local search in polynomial time as this can be handled using standard techniques, while losing $1/poly(n)$ in the approximation factor) the following: 



\begin{theorem} \label{th:LS-approx}
For any $p$-extendible system $\I \subseteq 2^X$ and a monotone submodular function $f:2^X \rightarrow \RR_+$,
local search with $(p,1)$-swaps (including at most $1$ element and removing at most $p$ elements) provides a $1/(p^2+1)$-approximation. For additive $f$, the factor is $1/p^2$.
%If $f$ is additive, the approximation factor is $1/p^2$.
\end{theorem}



\begin{proof}
Let $A$ be a local optimum under $(p,1)$-swaps, and let $B$ be an optimal solution. (For convenience, let us also assume that we always try to add elements to $A$ if possible, even if they bring zero marginal value.) We proceed in two steps, the first one inspired by the analysis of the greedy algorithm for $p$-extendible systems \cite{calinescu2011maximizing} and the second one similar to other analyses of local search.

Let $A = \{a_1,\ldots,a_k\}$ be a greedy ordering of $A$ in the sense that $a_1$ is the element of $A$ maximizing $f_\emptyset(a_1)$; given $a_1$, $a_2$ is the element of $A-a_1$ maximizing $f_{\{a_1\}}(a_2)$, $a_3$ is the element of $A-a_1-a_2$ maximizing $f_{\{a_1,a_2\}}(a_3)$, etc. Using the $p$-extendible property, there is a subset $B_1 \subseteq B, |B_1| \leq p$ such that $(B \setminus B_1) + a_1 \in \I$. Further, since $\{a_1,a_2\} \in \I$, there is a subset $B_2 \subseteq B \setminus B_1, |B_2|\leq p$ such that $(B \setminus (B_1 \cup B_2)) \cup \{a_1,a_2\} \in \I$, etc. Generally, there are disjoint subsets $B_1,\ldots,B_k \subseteq B, |B_i| \leq p$ such that $(B \setminus (B_1 \cup \ldots B_i)) \cup \{a_1,\ldots,a_i\} \in \I$. In fact, if $|A| = k$, the sets $B_1,\ldots,B_k$ form a partition of $B$. Otherwise there would be additional elements in $B \setminus (B_1 \cup \ldots \cup B_k)$ which can be added to $A$, which would contradict the local optimality of $A$.

Now, we claim that for each $b \in B_i$, we have $f_A(b) \leq p f_{\{a_1,\ldots,a_{i-1}\}}(a_i)$. If not, we would be able to add $b$ and, since $\{a_1,\ldots,a_{i-1},b\} \in \I$, we could remove at most $p$ elements $Z \subseteq A \setminus \{a_1,\ldots,a_{i-1}\}$ so that $(A \setminus Z) + b \in \I$. By submodularity and the greedy ordering, we would have $f(A \setminus Z) \geq f(A) - p f_{\{a_1,\ldots,a_{i-1}\}}(a_i)$ and again by submodularity, we would have $f((A \setminus Z) + b) \geq f(A \setminus Z) + f_A(b) > f(A \setminus Z) + p f_{\{a_1,\ldots,a_{i-1}\}}(a_i) \geq f(A)$. Therefore, this would be an improving local swap.

Since $A$ is a local optimum, we conclude that $f_A(b) \leq p f_{\{a_1,\ldots,a_{i-1}\}}(a_i)$ for each $b \in B_i$. Since $B = B_1 \cup \ldots \cup B_k$ and $|B_i| \leq p$, we have by submodularity
$$ f_A(B) \leq \sum_{i=1}^{k} \sum_{b \in B_i} f_A(b) \leq \sum_{i=1}^{k} |B_i| p f_{\{a_1,\ldots,a_{i-1}\}}(a_i)
\leq p^2 \sum_{i=1}^{k} f_{\{a_1,\ldots,a_{i-1}\}}(a_i) \leq p^2 f(A) $$
For $f$ monotone submodular, we have $f(B) \leq f(A) + f_A(B) \leq (p^2+1) f(A)$.
For $f$ additive, we have $f(B) = f_A(B) \leq p^2 f(A)$. This completes the proof.
\end{proof}

\subsection{Recovery for $p$-extendible systems}
%Note that in the proof of \hyperref[th:LS-approx]{Theorem \ref{th:LS-approx}} we can forget about $A\cap B$ and restrict our attention only to comparing the value of $A\sm B$ and $B\sm A$. This turns out to be useful for exact recovery as we can perturb only $A\sm B$. The following theorem intuitively tells us that \textit{local optima of stable instances are global optima}.
Note that in the proof of \hyperref[th:LS-approx]{Theorem \ref{th:LS-approx}}, if we focus on comparing the values of $A\sm B$ and $B\sm A$, we will be able to get exact recovery as we can perturb only $A\sm B$. The following theorem intuitively tells us that \textit{local optima of stable instances are global optima}.
\begin{theorem} \label{th:LS-recovery}
Given a $p$-extendible system $\I \subseteq 2^X$ and a monotone submodular function $f:2^X \rightarrow \RR_+\cup\{0\}$ we wish to maximize, if the optimal solution $B$ is $(p^2+1)$-stable, then local search with $(p,1)$-swaps exactly recovers it. If $f$ is additive, recovery holds if $B$ is $p^2$-stable.
\end{theorem}

\begin{proof}
The basic idea is that we can contract the elements that belong to $A\cap B$ and then use the same charging argument from above. Using the notation from the proof of \hyperref[th:LS-approx]{Theorem \ref{th:LS-approx}}, for elements $a_i\in A\cap B$ the corresponding $B_i$ block is just $\{a_i\}$. Now we can rename elements in $A\sm B=\{a_1,\dots,a_m\}$ with corresponding blocks $B_1,\dots,B_m$ such that $B\sm A = B_1 \cup \ldots \cup B_m$ and $|B_i| \leq p$. Rewriting the local search guarantee:
$$f_A(B\sm A) \leq \sum_{i=1}^{m} \sum_{b \in B_i} f_A(b) \leq \sum_{i=1}^{m} |B_i| p f_{\{a_1,\ldots,a_{i-1}\}}(a_i)
\leq p^2 \sum_{i=1}^{m} f_{\{a_1,\ldots,a_{i-1}\}}(a_i) \leq p^2 f(A\sm B)$$
Since $f_A(B\sm A)=f(B\cup A)-f(A)\ge f(B)-f(A)$, we can $(p^2+1)$-perturb the input (only the marginal of elements in $A\sm B$) and get: $\tilde{f}(B)=f(B) \le f(A)+p^2f(A\sm B)=\tilde{f}(A)$, hence contradicting the $(p^2+1)$-stability.
In the case of additive $f$, $f_A(B\sm A)= f(B\sm A)$ and $\tilde{f}(B)=f(B)=f(B\sm A)+f(B\cap A)\le p^2f(A\sm B) +f(B\cap A)\le f(A)+(p^2-1)f(A\sm B)=\tilde{f}(A)$, where we $p^2$-perturbed the instance, hence contradicting the $p^2$-stability of the instance.
\end{proof}



%For a constraint $(X,\I)$, we wish to find the $max\{f(S): S\in \I\}$ ($f$: monotone submodular) where $\I$ is the intersection of $p$ matroids: $\I=\cap_{i=1}^p \I_i$. We prove that local search with ($p,1$)-swaps exactly recovers the optimal solution if it is $(p+1)$-stable. For one matroid $p=1$ (a matroid is 1-extendible), our previous \hyperref[th:LS-recovery]{Theorem~\ref{th:LS-recovery}} implies:




\subsection{Recovery for the intersection of Matroids}
If the independence system $\I$ is the intersection of $p$ matroids: $\I=\cap_{i=1}^p \I_i$, local search with ($p,1$)-swaps recovers $(p+1)$-stable optimal solutions (for proof, see \hyperref[app:LSrecovery]{Appendix~\ref{app:LSrecovery}}). 

%Since a matroid is 1-extendible, our previous \hyperref[th:LS-recovery]{Theorem~\ref{th:LS-recovery}} implies:
%\begin{corollary}
%Given a matroid $(X,\I)$ and $f$ monotone submodular, such that the optimal solution is 2-stable, Local Search exactly recovers it.
%\end{corollary}
%\begin{proof}
%Suppose again that $A$ is the local optimum solution and $B$ is the global optimum. By the local search criterion we have that $\exists$ a bijection $\pi:A\setminus B \to B\setminus A:$
%\begin{itemize}
%\item 1. $\forall x \in A\setminus B: (A-x+\pi(x)) \in \I$
%\item 2. $\forall x \in A\setminus B: f(A-x+\pi(x))\le f(A)$
%\end{itemize}
%Since $(A-x)\subseteq (A+\pi(x)-x)$, using the submodularity for adding $x$, we get :
%\[
%-(f(A)-f(A-x))+(f(A+\pi(x))-f(A))\le 
%\]
%\[
%\le -(f(A+\pi(x))-f(A+\pi(x)-x)) + (f(A+\pi(x))-f(A))\le0
%\]
%Rearranging we get: $f(A+\pi(x))-f(A)\le f(A)-f(A-x)=f_{A-x}(x), \forall x\in A$. Let $A = \{a_1,\ldots,a_k\}$, $A_i=\{a_1,a_2,\dots a_i\}$ and let the marginal improvement be $\delta_i = f_{A_{i-1}}(a_i)= f(A_i)-f(A_{i-1})$ at the point of addition of $a_i$ (for simplicity we make abuse of notation and have $\delta_{a_i}=\delta_i$). Summing up the inequalities for all $x \in A\setminus B$, we get a telescoping sum:
%\[
%f(A\cup B)-f(A)\le \sum_{i\in A\setminus B}\delta_i \iff f(A\cup B)\le f(A)+\sum_{i\in A\setminus B}\delta_i
%\]
%We can now use our assumption about 2-stability by perturbing (here $\gamma=2$) only the marginals of the elements $x \in A\setminus B$ (this favours only our local search solution). The new value for the local search solution $A$ is: $\tilde{f}(A)=f(A)+\sum_{i\in A\setminus B}\delta_i$ and thus we get:
%\[
%\tilde{f}(B)=f(B)\le f(A\cup B)\le f(A)+\sum_{i\in A\setminus B}\delta_i=\tilde{f}(A),
%\]
%which contradicts the 2-stability of the instance.
%\end{proof}

%Generally, for $p$ matroids we have the following theorem: (for the proof, we refer the reader to \hyperref[app:LSrecovery]{Appendix~\ref{app:LSrecovery}} of the full version of this paper):
\begin{theorem}\label{th:LSmatroids}
Given $(X,\I)$, with $\I=\cap_{i=1}^p \I_i$ where each $\I_i$ is a matroid and $f$ monotone submodular, such that the optimal solution is $(p+1)$-stable, Local Search exactly recovers it.
\end{theorem}


%\begin{proof}
%We denote with $A$ our local search solution (let it be maximal, even if new elements add zero value to it) and with $B$ the global optimum. Let $Y=B\sm A=\{y_1,y_2,\dots,y_k\}$ be the elements of the optimum that local search didn't choose. By the matching property~\cite{reichel2007evolutionary} of the matroids we get:
%\[
%\exists\ X^1,X^2,\dots,X^p \subseteq A\sm B, \text{where\ } X^j=\{x_1^j,x_2^j,\dots,x_k^j\}\  \text{such that:\ }
%\]
%\[
%\forall j\in\{1,\dots,p\}: x_i^j \in C_j(A,y_i), \forall i\in \{1,2,\dots,k\},
%\]
%where $C_j(A,y_i)$ is the circuit (minimally dependent set) created in matroid $\I_j$ when adding $y_i$ in $A$.

%Using the local search (with $(p,1)$ swaps) stopping condition, we have: (for ease, we use $+,-$ instead of the more accurate $\cup, \sm$)
%\[
%f(A+y_i-x_i^1-x_i^2-\dots-x_i^p)\le f(A), \forall i\in \{1,2,\dots,k\}
%\]
%(Note that in case $f$ is additive the above inequality just becomes: $f(y_i)\le f(x_i^1)+f(x_i^2)+\dots+f(x_i^p)$).
%Since $(A-\cup_{j=1}^p x^j_i)\subseteq (A+y_i-\cup_{j=1}^p x^j_i)$, using the submodularity for adding $\cup_{j=1}^p x^j_i$, we get:
%\[
%f(A+y_i)-f(A+y_i-\cup_{j=1}^p x^j_i)\le f(A)-f(A-\cup_{j=1}^p x^j_i)
%\]
%and adding $f(A+y_i)-f(A)$ to both sides and using submodularity and the local search stopping condition, we get:
%\[
%f(A+y_i)-f(A)-f(A)+f(A-\cup_{j=1}^p x^j_i)\le f(A+y_i-\cup_{j=1}^p x^j_i)-f(A)\le 0
%\]
%We conclude: $f(A+y_i)-f(A)\le f(A)-f(A-\cup_{j=1}^p x^j_i), \forall i\in \{1,2,\dots,k\}$ and adding these inequalities ($\delta(x_i^j)$ is the marginal gain by adding $x_i^j$ at the point of addition):
%\[
%f_A(B\sm A)\le \sum_{i=1}^k\sum_{j=1}^p\delta(x_i^j) = \sum_{j=1}^p\sum_{i=1}^k\delta(x_i^j)\le \sum_{j=1}^p f(X^j)\le \sum_{j=1}^p f(A\sm B)\le pf(A\sm B)
%\]
%Now we can $(p+1)$-perturb the marginals for elements of $A\sm B$:
%\[
%\tilde{f}(B)=f(B)\le f(A\cup B)\le f(A)+f_A(B\sm A)\le f(A)+pf(A\sm B)=\tilde{f}(A)
%\]
%which contradicts the ($p+1$)-stability of the instance. Once again, for the case of additive $f$: $f_A(B\sm A)=f(B\sm A)$ and thus $p$-stability is enough to guarantee recovery. ($\tilde{f}(B)=f(B)=f(B\sm A)+f(B\cap A)\le pf(A\sm B) +f(B\cap A)\le f(A)+(p-1)f(A\sm B)=\tilde{f}(A)$, where we $p$-perturbed the instance)
%\end{proof}










\\
\stitle{Acknowledgements:} We thank Yifan Wu for the initial inspiration, Anant Bhardwaj for data collection, Laura Rettig on early formulations of the problem, and the support of NSF 1527765 and 1564049.

\bibliography{main}

\appendix



\section{Proof of \hyperref[th:submod]{Theorem} for welfare maximization}\label{app:submod}

\begin{theorem}
Let $(X,\I)$ be a matroid on the elements of $X$, let $B_1,B_2,\dots,B_k$ be a partition of $X$, $f_i: 2^{B_i}\to \RR^+\cup\{0\}$, for $i\in \{1,2,\dots,k\}$ be monotone submodular and let $f=\sum_{i=1}^kf_i$. If the optimal solution $S^*$ of $\max\{f(S): S\in \I\}$ is $2$-stable with respect only to individual perturbations of the functions $f_i$, greedy will recover $S^*$.
\end{theorem}

\begin{proof}
We note that the $B_i$'s form a partition of $X$ which is not tied to the matroid in any way. To avoid confusion, we should first emphasize the greedy algorithm in this case: It starts with the empty set $S_0=\emptyset$, at step $t$ it selects: $e=\arg\max_{x\in X}\{f(S_{t-1}+x)-f(S_{t-1})\}$ subject to the matroid constraint and it updates $S_t\leftarrow S_{t-1}+e$. This is a particular instantiation of the standard greedy algorithm in welfare maximization that first picks an item giving it to the player so that it yields maximum marginal improvement.

Suppose greedy outputs $S\neq S^*$ and that it chose elements $A_i\subseteq B_i$. Let $S_e$ be the greedy solution right before adding element $e$. Then a 2-perturbation of the individual functions is:
\[
\tilde{f_i}(A_i)=f_i(A_i)+\sum_{e\in (S\cap B_i)\sm S^*}f_i((B_i\cap S_e)+e)-f_i(B_i\cap S_e)
\]
Now coming back to the total welfare function $f$ we get:
\[
\tilde{f}(S)=\sum_{i=1}^k\tilde{f}_i(S\cap B_i)=f(S)+\sum_{e\in S\sm S^*}f_{S_e}(e)\ge f(S)+\sum_{e'\in S^*\sm S, e\leftrightarrow e'}f_{S_e}(e')\ge
\]
\[
\ge f(S)+f_S(S^*\sm S)\ge f(S^*\cup S)\ge f(S^*)=\tilde{f}(S^*)
\]
where we made use of the greedy criterion, submodularity and the matroid matching $e\leftrightarrow e'$ between elements $e\in S\sm S^*$ and $e' \in S^*\sm S$. We got $\tilde{f}(S)\ge \tilde{f}(S^*)$, hence a contradiction to the 2-stability of $S^*$ and hence $S\equiv S^*$ and greedy exactly recovers the optimal solution. 
\end{proof}
























%%% Local Variables:
%%% mode: latex
%%% TeX-master: "main"
%%% End:


\section{Hereditary Systems}\label{sec:hereditary}

Motivated by the ``bad'' example (see \hyperref[knapsack]{Proposition \ref{knapsack}}) for the greedy algorithm, we define a new notion of an independence system that we call \textit{hereditary} $p$-system or $p$-\textit{hereditary} that as we see later is a different characterization of $p$-extendible systems. In the aforementioned example, even though we started with a $p$-system, as we progressed picking elements with the greedy algorithm, the system became a $p'$-system with $p' \gg p$, thus leading to bad performance for the greedy, even though we had the optimal solution being stable by a large amount.

The intuition behind the following definition is that we want our system to remain a $p$-system under deletions and contractions of elements.

\begin{definition}[Hereditary $p$-system]
A $p$-system $(X,\I)$ is said to be \textit{hereditary} if:
\begin{enumerate}
\item For each set $Y\sse X$, the system $(X', \I | X')$\footnote{By $(X',\I | X')$, we mean the \textit{restriction} of $\I$ to the set of elements $X'$, which is the independence system on the set $X'$, whose independent sets are the independent sets of the initial set $\I$ that are contained in $X'$.}, where $X'=X \sm Y$, is a $p$-system. This corresponds to the {\bf deletion} of the elements in $Y$ from the system.
\item For each set $Y\sse X$,  the system $(X\sm Y, \I / Y)$\footnote{By $(X\sm Y,\I / Y)$, we mean the \textit{contraction} of $\I$ by $Y$, which is the independence system on the underlying set $X\sm Y$, whose independent sets are the sets $Z \sse X\sm Y$, such that $Z\cup Y \in \I$.} is a $p$-system. This corresponds to the {\bf contraction} of the elements in $Y$.
\end{enumerate}
\end{definition}

\noindent Looking back at our ``bad'' Knapsack example we see that it is not a hereditary system since initially $p=\tfrac{2M}{M+1}\le 2$, but after we had picked all the elements in set $A$, the system on the remaining elements became an $M$-extendible system. We now prove that the family of hereditary $p$-systems coincides with the family of $p$-extendible systems.  

\begin{proposition}
A $p$-system is $p$-hereditary if and only if it is $p$-extendible.
\end{proposition}

\begin{proof}
$p$-hereditary $\implies$ $p$-extendible: Let's first think of $p$ as an integer; as we will see afterwards only this case (with integer $p$) is interesting. Suppose we had a $p$-hereditary system that was not $p$-extendible. By negating the definition of $p$-extendibility (see \hyperref[sec:preliminaries]{Preliminaries}), it follows that there exist sets $A,B \sse X$ with $A \sse B$, $A,B \in \I$ and $A\cup\{e\} \in \I$ such that for all sets $Z\sse B\sm A$ with $|Z| \le p$: $(B\sm Z)\cup \{e\} \not\in \I$. Define $Z_0 \sse B\sm A$ to be the smallest set that we need to remove from $B$ in order to have: $(B\sm Z_0)\cup \{e\} \in \I$. We know that $|Z_0|>p$ and thus, by the hereditary property, if we project the independence system on the elements $Z_0\cup \{e\}$, we get $Z_0\cup \{e\} \not\in \I$ with the ratio $\tfrac{|Z_0|}{|\{e\}|}=\tfrac{|Z_0|}{1}>p$, which contradicts the fact that we started with a $p$-hereditary system.


For $p$-extendible $\implies$ $p$-hereditary: This direction follows easily just by the definition of $p$-extendibility. To handle non-integer values of $p$, we observe that by the first argument above, a $p$-hereditary system is actually $\lfloor p\rfloor$-extendible and thus, it is $\lfloor p\rfloor$-hereditary (e.g. a 2.9-hereditary system is 2-extendible).
\end{proof}

























\section{Proof of \hyperref[th:LSmatroids]{Theorem} for Intersection of Matroids and recovery}\label{app:LSrecovery}

\begin{theorem}
Given $(X,\I)$, with $\I=\cap_{i=1}^p \I_i$ where each $\I_i$ is a matroid and $f$ monotone submodular, such that the optimal solution is $(p+1)$-stable, Local Search exactly recovers it.
\end{theorem}
\begin{proof}
We denote with $A$ our local search solution (let it be maximal, even if new elements add zero value to it) and with $B$ the global optimum. Let $Y=B\sm A=\{y_1,y_2,\dots,y_k\}$ be the elements of the optimum that local search didn't choose. By the matching property~\cite{reichel2007evolutionary} of the matroids we get:
\[
\exists\ X^1,X^2,\dots,X^p \subseteq A\sm B, \text{where\ } X^j=\{x_1^j,x_2^j,\dots,x_k^j\}\  \text{such that:\ }
\]
\[
\forall j\in\{1,\dots,p\}: x_i^j \in C_j(A,y_i), \forall i\in \{1,2,\dots,k\},
\]
where $C_j(A,y_i)$ is the circuit (minimally dependent set) created in matroid $\I_j$ when adding $y_i$ in $A$.

Using the local search (with $(p,1)$ swaps) stopping condition, we have: (for ease, we use $+,-$ instead of the more accurate $\cup, \sm$)
\[
f(A+y_i-x_i^1-x_i^2-\dots-x_i^p)\le f(A), \forall i\in \{1,2,\dots,k\}
\]
(Note that in case $f$ is additive the above inequality just becomes: $f(y_i)\le f(x_i^1)+f(x_i^2)+\dots+f(x_i^p)$).
Since $(A-\cup_{j=1}^p x^j_i)\subseteq (A+y_i-\cup_{j=1}^p x^j_i)$, using the submodularity for adding $\cup_{j=1}^p x^j_i$, we get:
\[
f(A+y_i)-f(A+y_i-\cup_{j=1}^p x^j_i)\le f(A)-f(A-\cup_{j=1}^p x^j_i)
\]
and adding $f(A+y_i)-f(A)$ to both sides and using submodularity and the local search stopping condition, we get:
\[
f(A+y_i)-f(A)-f(A)+f(A-\cup_{j=1}^p x^j_i)\le f(A+y_i-\cup_{j=1}^p x^j_i)-f(A)\le 0
\]
We conclude: $f(A+y_i)-f(A)\le f(A)-f(A-\cup_{j=1}^p x^j_i), \forall i\in \{1,2,\dots,k\}$ and adding these inequalities ($\delta(x_i^j)$ is the marginal gain by adding $x_i^j$ at the point of addition):
\[
f_A(B\sm A)\le \sum_{i=1}^k\sum_{j=1}^p\delta(x_i^j) = \sum_{j=1}^p\sum_{i=1}^k\delta(x_i^j)\le \sum_{j=1}^p f(X^j)\le \sum_{j=1}^p f(A\sm B)\le pf(A\sm B)
\]
Now we can $(p+1)$-perturb the marginals for elements of $A\sm B$:
\[
\tilde{f}(B)=f(B)\le f(A\cup B)\le f(A)+f_A(B\sm A)\le f(A)+pf(A\sm B)=\tilde{f}(A)
\]
which contradicts the ($p+1$)-stability of the instance. Once again, for the case of additive $f$: $f_A(B\sm A)=f(B\sm A)$ and thus $p$-stability is enough to guarantee recovery. ($\tilde{f}(B)=f(B)=f(B\sm A)+f(B\cap A)\le pf(A\sm B) +f(B\cap A)\le f(A)+(p-1)f(A\sm B)=\tilde{f}(A)$, where we $p$-perturbed the instance)
\end{proof}




















%%% Local Variables:
%%% mode: latex
%%% TeX-master: "main"
%%% End:

\section{Counterexamples}\label{sec:counter}

Here are two simple counterexamples that prove the tightness of our Greedy recovery results and our Local Search approximation and recovery results:

\begin{itemize}
\item In the submodular case, we proved greedy recovers $(p+1)$-stable $p$-extendible systems. Here is a simple example of a matroid (1-extendible) where Greedy and Local Search fail to recover the optimal solution even though it is 2-stable (also notice that here, Greedy and Local Search give a 2-approximation):  Take $A_1=\{x,\epsilon_1\}, B_1=\{y\}, A_2=\{\epsilon_2\}, B_2=\{x\}$ as in~\cite{filmus2012power}. Assign $w(x)=w(y)=1$ and $w(\epsilon_1)=\epsilon, w(\epsilon_2)=\epsilon$ for some small $\epsilon>0$ (and so $w(A_1)=1+\epsilon$) and consider the partition matroid whose independent sets can only contain one of $\{A_i,B_i\}, i=1,2$. Observe that $\{A_1,A_2\}$ is a local optimum with value $1+2\epsilon$, whereas the global optimum is $\{B_1,B_2\}$ with value 2. Also notice that the same solution is produced by the Greedy algorithm and that the instance can be $(2-\epsilon')$-stable for any small $\epsilon'>0$.

\item Local Search is a $p^2$-approximation for $p$-extendible systems. Look at \hyperref[counter1]{Figure~\ref{counter1}} for a tight counterexample (just for simplicity, we have the $p=2$ case; it generalizes readily).
\end{itemize} 


\begin{figure}[h!]
	\centering
	\includegraphics[width=5cm,height=4cm]{IMAGE1}
        	\caption{Local Search is a 4-approximation for this 2-extendible system $(X,\I)$: Let $A=\{a_1,a_2\}$ be feasible and assign $w(a_1)=w(a_2)=1+\epsilon$ and $w(b_i)=2, \forall i\in\{1,2,3,4\}$. The constraints are: $a_1\cup B_1 \notin \I, a_2\cup B_2 \notin \I $, $a_i\cup B_j \in \I$ for $i\neq j$ and $A\cup b_i\notin \I,\forall i\in\{1,2,3,4\}$. Observe that $A$ is a local optimum ($(2,1)$-swaps) with value $2+2\epsilon$, whereas $B_1\cup B_2$ is the global optimum with value 8. Notice also that for the appropriate choice of $\epsilon$, this can be a $(4-\epsilon')$-stable instance for any small $\epsilon'$.}
	\label{counter1}
\end{figure}

\subsection{Cardinality Constraints}\label{sec:cardinality-counter}

Another interesting separation between approximation and stability happens for the case of cardinality constraints. A special case of submodular maximization on $p$-extendible systems is when we have a uniform matroid constraint where the only feasible solutions are those that have cardinality $k \ge 1$ ($\I=\{S\subseteq X: |S|\le k\}$). For this special case, recall that greedy is a $(1-\tfrac{1}{e})$-approximation (in fact, $1-(1-\tfrac{1}{k})^k$) and that this is tight \cite{feige1998threshold}. Regarding stability, we show that the stability threshold needed by greedy for recovery is at least $2-\tfrac{1}{k}$ and so $(1-\tfrac{1}{e})^{-1}$-stability is not enough, i.e. here the approximation threshold is strictly smaller than the stability threshold needed for recovery (see also \hyperref[fig:cardinality]{Figure~\ref{fig:cardinality}}).

\begin{proposition}
For submodular maximization under a uniform matroid ($\I=\{S: |S|\le k\},k\ge 1$), greedy cannot recover $\gamma$-stable instances if $\gamma<(2-\tfrac{1}{k})$.
\end{proposition}

\begin{proof}
The $(2-\tfrac{1}{k}-\delta)$-stable counterexample (for any small $\delta$) where greedy fails is the following: We have in total ($k+1$) elements: $x_1,x_2,\dots, x_k$ and a special element $e$. Denote $O=\{x_1,x_2,\dots, x_k\}$ and with $O_i$ any subset of $O$ with exactly $i$ elements. The function $f$ has: $f(O)=1, f_{O_i}(x_j)=\tfrac{1}{k}, \forall x_j\in O\setminus O_i,  f_{\{e\}\cup O_i}(x_j)=\tfrac{1}{k}(1-\tfrac{1}{k}), \forall x_j\in O\setminus O_i $ and $f(e)=\tfrac{1}{k}, f_{O_i}(e)=\tfrac{1}{k}-\tfrac{i}{k^2}$. Then Greedy first picks element $e$ (to break ties we could set $f(e)=\tfrac{1}{k}+\epsilon$) and then $k-1$ other elements $O_{k-1}\subseteq O$ (let $S=\{e\}\cup O_{k-1}$). However, the optimal solution is $O$ with $f(O)=1$ and greedy has value $1-(\tfrac{1}{k}-\tfrac{1}{k^2})$. Since $S\sm O=\{e\}$, any perturbation such that $\tilde{f}(S)\ge \tilde{f}(O)$ could only $\gamma$-perturb the value $f(e)$: $\tilde{f}(S)\ge \tilde{f}(O)\iff (\gamma-1)\tfrac{1}{k}\ge\tfrac{1}{k}-\tfrac{1}{k^2} \iff \gamma\ge (2-\tfrac{1}{k})$.
\end{proof}
\begin{figure}[h!]
	\centering
	\includegraphics[width=13cm,height=4cm]{IMAGE2}
        	\caption{This is the case for $k=2$ and $k=3$ (the area corresponds to marginal improvements). For $k=2$, there are three elements: $\{e,x_1,x_2\}$. $f(\{x_1,x_2\})=1$, so the optimal solution is $O=\{x_1,x_2\}$. We trick the greedy algorithm which first chooses $\{e\}$ that has slightly better marginal value. For exact recovery, a $\tfrac{3}{2}$-perturbation is needed, even though Greedy is a $\left(\tfrac{4}{3}\right)^{-1}$-approximation. Similarly, for $k=3$, the optimum is $O=\{x_1,x_2,x_3\}$, whereas Greedy picks $\{e,x_1,x_2\}$ and needs $\tfrac{5}{3}$-stability for recovery, even though it is $\left(\tfrac{19}{27}\right)^{-1}$-approximation. Note that stability thresholds need to be larger than the approximation factors.}
	\label{fig:cardinality}
\end{figure}




 


























\end{document}

%%% Local Variables:
%%% mode: latex
%%% TeX-master: t
%%% End:
