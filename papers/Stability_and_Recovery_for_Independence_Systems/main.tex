
\documentclass[11pt]{article}

 
\usepackage{microtype}%if unwanted, comment out or use option "draft"
%\usepackage{hyperref}
%\hypersetup{
%    colorlinks=true,
%    linkcolor=blue,      
%    citecolor=blue
%}
\bibliographystyle{plainurl}
\usepackage{graphicx}
\usepackage{caption}
\usepackage{subcaption}
\usepackage{theorem}
%\usepackage{subfigure}
\usepackage{epsfig}
\usepackage{amssymb}
\usepackage{fullpage}
\usepackage{amsmath}
\usepackage{ifthen}
\usepackage{verbatim}
\usepackage{mdframed}
\usepackage{xspace}
\usepackage{enumitem}
\usepackage[multiple]{footmisc}
\usepackage{url}
\usepackage[colorlinks,linkcolor=blue,citecolor=blue,urlcolor=blue]{hyperref}

\usepackage{tikz}
\usetikzlibrary{shapes,shapes.geometric,arrows,fit,calc,positioning,automata,arrows.meta}
\tikzset{>={Latex[width=2mm,length=2mm]}}
\usepackage{float}

%algorithm env customization
\usepackage[noend,noline]{algorithm2e}
\SetEndCharOfAlgoLine{}
\SetArgSty{}
\SetKwBlock{Repeat}{repeat}{}
\newtheorem{theorem}	 			{Theorem}[section]
\newtheorem{lemma}		[theorem]	{Lemma}	
\newtheorem{fact}		[theorem]	{Fact}
\newtheorem{claim}		[theorem]	{Claim}
\newtheorem{corollary}		[theorem]	{Corollary}
\newtheorem{prop}		[theorem]	{Proposition}
\newtheorem{definition}	 			{Definition}[section]
%{\theorembodyfont{\rmfamily} \newtheorem{definition}
%[theorem]	{Definition}}
{\theorembodyfont{\rmfamily} \newtheorem{remark}		[theorem]
{Remark}}
{\theorembodyfont{\rmfamily} \newtheorem{proposition}		[theorem]
{Proposition}}
{\theorembodyfont{\rmfamily} \newtheorem{example}		[theorem]
{Example}}
{\theorembodyfont{\rmfamily} \newtheorem{question}
{Open Question}}
{\theorembodyfont{\rmfamily} \newtheorem{warning}			
{Warning}}
{\theorembodyfont{\rmfamily} \newtheorem{q}			[theorem]
{Question}}
{\theorembodyfont{\rmfamily} \newtheorem{exercise}		[theorem]
{Exercise}}
{\theorembodyfont{\rmfamily} \newtheorem{coursegoal}	
{Course Goal}}
\theoremstyle{break}
{\theorembodyfont{\rmfamily} \newtheorem{remarkbreak}		[theorem]
{Remark}}

\newenvironment{proof}{\noindent {\em {Proof:}}}{$\blacksquare$\vskip
\belowdisplayskip}

\newenvironment{prevproof}[2]{\noindent {\em {Proof of
{#1}~\ref{#2}:}}}{$\blacksquare$\vskip \belowdisplayskip}

%\newcommand{\prevproof}[3]{
%{\noindent {\em Proof of {#1}~\ref{#2}.} {#3} \rule{2mm}{2mm} \vskip
%\belowdisplayskip}
%}

%%%FOOTNOTES OUTSIDE OF FRAMES

\usepackage{tablefootnote} 
\makeatletter 
\AfterEndEnvironment{mdframed}{%
 \tfn@tablefootnoteprintout% 
 \gdef\tfn@fnt{0}% 
}
\makeatother

\newlist{exlist}{enumerate}{1}
\setlist[exlist]{label=(\alph*)}

\newcommand{\floor}[1]{
{\lfloor {#1} \rfloor}
}
\newcommand{\bigfloor}[1]{
{\left\lfloor {#1} \right\rfloor}
}

\newcommand{\argmax}{\operatornamewithlimits{argmax}}
\newcommand{\argmin}{\operatornamewithlimits{argmin}}
\newcommand{\prob}[2][]{\text{\bf Pr}\ifthenelse{\not\equal{}{#1}}{_{#1}}{}\!\left[#2\right]}
\newcommand{\expect}[2][]{\text{\bf E}\ifthenelse{\not\equal{}{#1}}{_{#1}}{}\!\left[#2\right]}
\newcommand{\var}[2][]{\text{Var}\ifthenelse{\not\equal{}{#1}}{_{#1}}{}\!\left[#2\right]}
\newcommand{\dev}[2][]{\text{StdDev}\ifthenelse{\not\equal{}{#1}}{_{#1}}{}\!\left[#2\right]}
\newcommand{\ip}[2]{
{\langle {#1} , {#2} \rangle}
}

\newcommand{\subclaim}[1]{
\vskip 0.10in
{\noindent {\bf Claim: } {\em {#1}}}
\vskip 0.10in
}
\newcommand{\claimproof}[1]{
{\noindent {\it Proof of Claim.} {#1} \rule{2mm}{2mm} \vskip
\belowdisplayskip}}

\newcommand{\bid}{b}
\newcommand{\bids}{{\mathbf \bid}}
\newcommand{\bidsmi}{{\mathbf \bid}_{-i}}
\newcommand{\bidi}[1][i]{\bid_{#1}}

\newcommand{\val}{v}
\newcommand{\vals}{{\mathbf \val}}
\newcommand{\valsmi}{{\mathbf \val}_{-i}}
\newcommand{\vali}[1][i]{\val_{#1}}

\mdfsetup{frametitlealignment=\centering}
\mdfdefinestyle{offset}{backgroundcolor=white,linecolor=black,innerrightmargin=15pt,leftmargin=23pt,rightmargin=23pt,innertopmargin=.5\baselineskip,innerbottommargin=.5\baselineskip}

\def\sm{\setminus}
\def\sse{\subseteq}
\def\eps{\epsilon}
\def\RR{\mathbb{R}}
\def\P{\mathcal{P}}
\def\H{\mathcal{H}}
\def\C{\mathcal{C}}
\def\I{\mathcal{I}}
\def\p{\mathbf{p}}
\def\q{\mathbf{q}}
\def\r{\mathbf{r}}
\def\z{\mathbf{z}}
\def\a{\mathbf{a}}
\def\e{\mathbf{e}}
\def\x{\mathbf{x}}
\def\w{\mathbf{w}}
\def\y{\mathbf{y}}
\def\rows{\mathbf{x}}
\def\cols{\mathbf{y}}
\def\A{\mathbf{A}}
\def\Alg{\mathcal{A}}
\def\bfA{\mathbf{A}}
\def\Ax{\mathbf{Ax}}
\def\c{\mathbf{c}}
\def\b{\mathbf{b}}
\def\f{\mathbf{f}}
\def\u{\mathbf{u}}
\def\l{\mathbf{\ell}}
\def\ones{\mathbf{1}}
\def\var{\mbox{Var}}
\def\stddev{\mbox{StdDev}}
\def\daa{\text{dist}(\a,\a^*)}
\def\poly{\text{poly}}
\def\polylog{\text{polylog}}

\def\σ{\mathbf{\sigma}}
\def\β{\mathbf{\beta}}

\newcommand{\MaxCut }{{\sc MaxCut }}







\title{Stability and Recovery for Independence Systems}
\author{Vaggos Chatziafratis\thanks{Computer Science Department, Stanford University}\\{\tt vaggos@stanford.edu}\and Tim Roughgarden\footnotemark[1]\\{\tt tim@cs.stanford.edu}\and Jan Vondrak\thanks{Department of Mathematics, Stanford University}\\{\tt jvondrak@stanford.edu}}
\date{\today}

\begin{document}

\maketitle


\begin{abstract}
Two genres of
  heuristics that are frequently reported to perform much better on
  ``real-world'' instances than in the worst case are {\em greedy
    algorithms} and {\em local search algorithms}.  In this paper, we
  systematically study these two types of algorithms for the problem
  of maximizing a monotone submodular set function subject to
  downward-closed feasibility constraints.  We consider {\em
    perturbation-stable} instances, in the sense of Bilu and
  Linial~\cite{bilu2012stable}, and precisely identify the stability
  threshold beyond which these algorithms are guaranteed to recover
  the optimal solution.  Byproducts of our work include the first
  definition of perturbation-stability for non-additive objective
  functions, and a resolution of the worst-case approximation
  guarantee of local search in $p$-extendible systems.

\end{abstract}
\clearpage
\section{Introduction}  \label{sec:introduction}

\newcommand\inexpIntro[3]{#1?(#2,#3).}
\newcommand\rinexpIntro[3]{*#1?(#2,#3).}
\newcommand\outexpIntro[3]{#1!(#2,#3).}
\newcommand\outatomIntro[3]{#1!(#2,#3)}

We propose a fully automated method for proving termination of \(\pi\)-calculus processes.
Although there have been a lot of studies on termination analysis for the \(\pi\)-calculus
and related calculi~\cite{Deng06IC,Demangeon07,SangiorgiTermination,KobayashiHybrid,Yoshida04IC,DBLP:journals/jlp/DemangeonHS10,Venet98SAS}, most of them have been rather theoretical,
and there have been surprisingly little efforts in developing  fully automated termination
verification methods and tools based on them. To our knowledge,
Kobayashi's \typical{}~\cite{TyPiCal,KobayashiHybrid} is the only exception that
can prove termination of \(\pi\)-calculus processes (extended with natural numbers)
fully automatically, but its termination analysis is quite limited (see Section~\ref{sec:relatedwork}).

Our method is based on a reduction to termination analysis for sequential programs:
we translate a \(\pi\)-calculus process \(P\) to a sequential program \(S_P\), so that
if \(S_P\) is terminating, so is \(P\). The reduction allows us to use
powerful, mature methods and tools
for termination analysis of sequential programs~\cite{heizmann2016ultimate,freqterm,DBLP:conf/lics/PodelskiR04,Kuwahara2014Termination,DBLP:journals/cacm/CookPR11}.

The idea of the translation is to convert a chain of communications on replicated input
channels to a chain of recursive function calls of the target sequential program.
Let us consider the following Fibonacci process:
\begin{align*}
    & \rinexpIntro{\fib}{n}{r}
        \ifexp{n<2}{ \soutatom{r}{1} \\ &\quad}
                   { \nuexp{s_1} \nuexp{s_2} (\outatomIntro{\fib}{n-1}{s_1} \PAR \outatomIntro{\fib}{n-2}{s_2} \PAR \sinexp{s_1}{x}\sinexp{s_2}{y}\soutatom{r}{x+y}) \\}
    & \PAR \outatomIntro{\fib}{m}{r}
\end{align*}
Here, the process
$\rinexpIntro{\fib}{n}{r} \ldots$ is a function server that computes the \(n\)-th Fibonacci number
in parallel and returns the result to \(r\),
and $\outatom{\fib}{m}{r}$ sends a request for computing the \(m\)-th Fibonacci number;
those who are not familiar with the syntax of the \(\pi\)-calculus may wish to consult
Section~\ref{sec:targetlanguage} first.
To prove that the process above is terminating for any integer \(m\),
it suffices to show that there is no infinite chain of communications on $\fib$:
\[
    \fib(m,r) \to \fib(m_1,r_1) \to \fib(m_2,r_2) \to \cdots.
\]
We convert the process above to the following program:\footnote{The actual translation
  given later is a little more complex.}
\begin{verbatim}
 let rec fib(n) = if n<2 then () else (fib(n-1) [] fib(n-2)) in
 fib(m)
\end{verbatim}
Here, \texttt{[]} represents the non-deterministic choice.
Note that, although the calculation of Fibonacci numbers is not preserved,
for each chain of communications on \texttt{fib}, there is a corresponding
sequence of recursive calls:
\[
\mathtt{fib}(m) \to \mathtt{fib}(m_1) \to \mathtt{fib}(m_2) \to \cdots.
\]
Thus, the termination of the sequential program above implies the termination of
the original process.
As shown in the example above, (i) each communication on a replicated input channel
is converted to a function call, (ii) each communication on a non-replicated input
channel is just removed (or, in the actual translation, replaced by a call of
a trivial function defined by \(f(\seq{x})=(\,)\)), and (iii) parallel composition
is replaced by a non-deterministic choice.
We formalize the translation outlined above and prove its correctness.

The basic translation sketched above sometimes loses too much information.
For example, consider the following process:
\begin{align*}
    & \rinexpIntro{\pre}{n}{r} \soutatom{r}{n-1} \\
    & \PAR \rinexpIntro{f}{n}{r} \ifexp{n<0}{ \soutatom{r}{1} }
                                       { \nuexp{s} (\outatomIntro{\pre}{n}{s} \PAR \sinexp{s}{x}\outatomIntro{f}{x}{r}) } \\
    & \PAR \outatomIntro{f}{m}{r}
\end{align*}
The translation sketched above would yield:
\begin{verbatim}
  let pred(n) = n-1 in
  let rec f(n) = if n<0 then () else (pred(n) [] f(*)) in
  f(m)
\end{verbatim}
Here, \texttt{*} represents a non-deterministic integer: since we have removed
the input $\sinatom{s}{x}$, we do not have information about the value of \( x \).
As a result, the sequential program above is non-terminating, although the original
process is terminating.
To remedy this problem, we also refine the basic translation above by using a refinement
type system for the \(\pi\)-calculus. Using the refinement type system,
we can infer that the value of \(x\) in the original process is less than \(n\),
so that we can refine the definition of \texttt{f} to:
\begin{verbatim}
 let rec f(n) = ... else (pred(n) [] let x=* in assume(x<n);f(x))
\end{verbatim}
The target program is now terminating, from which
we can deduce that the original process is also terminating.
We have implemented an automated tool based on the refined translation above.

The contributions of this paper are summarized as follows.
\begin{itemize}
\item The formalization of the basic translation from the \(\pi\)-calculus
  (extended with integers) to sequential programs, and a proof of its correctness.
\item The formalization of a refined translation based on a refinement type system.
\item An implementation of the refined translation, including automated refinement type
  inference based on CHC solving, and experiments to evaluate the effectiveness of
  our method.
\end{itemize}

The rest of this paper is structured as follows.
Section~\ref{sec:targetlanguage} introduces the source and target languages
of our translation.
Section~\ref{sec:approach} 
formalizes the basic translation, and proves its correctness.
Section~\ref{sec:refinement} refines the basic translation by using a refinement type system.
Section~\ref{sec:implementation} reports an implementation and experiments.
Section~\ref{sec:relatedwork} discusses related work,
and Section~\ref{sec:conclusion} concludes the paper.

\section{Preliminaries}\label{chpt:preliminiaries}
In this chapter we will introduce some of the mathematical background and notation needed for this thesis. In particular, we will shortly introduce the differential geometric description of spacetime in Section \ref{sec:spacetime_geometry} and give an introduction to the notion of global hyperbolicity and its connection to Green- and normally-hyperbolic operators in Section \ref{sec:global_hyperbolicity}. In a bit more detail, we will introduce the notion of differential forms and give explicit definitions, also in terms of an index based notation, in Section \ref{sec:differential_forms}. For completeness, in Section \ref{sec:cat-theory}, we present basic definitions of category theory. The reader familiar with these topics can safely skip this chapter and refer to it when interested in the chosen conventions.
%
%
%
%
%%%%%%
%%SPACTIME GEOMETRY
%%%%%
%
%
%
\subsection{Spacetime geometry}\label{sec:spacetime_geometry}
In GR, the universe is mathematically described as a four dimensional \emph{spacetime}, consisting of a smooth, four dimensional manifold \gls{M} (assumed to be Hausdorff, connected, oriented, time-oriented and para-compact) and a Lorentzian metric $g$. We will assume the signature of the Lorentzian metric $g$ to be $(-,+,+,+)$. The Levi-Civita connection on $(\M,g)$ is as usual denoted by \gls{nabla}.
Throughout this thesis, we treat spacetime as fixed, implementing a gravitational background determined classically by Einstein's field equations. Hence, we neglect any back-reaction of the fields on the metric, both in the quantum and the classical case. In that sense, we treat the fields as \emph{test fields}.\par
For the basic mathematical theory regarding Lorentzian manifolds, we refer to the literature: An introduction to the topic with an emphasis on the physical application in GR is for example given in \cite{wald_GR} and \cite{carroll_spacetime-and-gr}.
Here, we will shortly recap the notion of a tangent space and tangent bundle and generalize to the notion of a vector bundle which we will use in the general description of normally hyperbolic operators and differential forms.
In the following, we generalize the setting to an arbitrary smooth manifold $\N$ of dimension $N$ with either Lorentzian or Riemannian metric $k$.\par
%
%
A \emph{tangent vector} $v_x$ at point $x \in \N$ is a linear map $v_x : C^\infty(\N , \IR) \to \IR$ that obeys the Leibniz rule, that is, for $f,g \in C^\infty (\N,\IR)$ it holds $v_x(fg) = f(x)v_x(g) + v_x(f)g(x)$.
We define the \emph{tangent space} \gls{TxN} of $\N$ at $x$ as the real $N$-dimensional vector space of all tangent vectors at point $x$.
The disjoint union of all tangent spaces is called the \emph{tangent bundle} \gls{TN} of $\N$ and is itself a manifold of dimension $2N$. A \emph{vector field} is a map $v: \N \to T\N$ such that $v(x) \in T_x\N$.
The respective dual spaces, that is the space of all linear functionals, the \emph{co-tangent space} and the \emph{co-tangent bundle}, are denoted by \gls{TsxN} and \gls{TsN} respectively.\par
%
For Lorentzian manifolds, we call a tangent vector $v$ at $x \in \N$ \emph{timelike} if $k_{\mu \nu} v^\mu v^\nu < 0$, \emph{spacelike} if $k_{\mu \nu} v^\mu v^\nu > 0$ and \emph{null} (or lightlike) if $k_{\mu \nu} v^\mu v^\nu = 0$. At every point $x \in \N$, we define the set of all \emph{causal}, that is, either timelike or null, tangent vectors in the tangent space at $x$. This set is called the \emph{light cone} at $x$ and it is split up into two distinct parts, one that we call the future light cone, and one that we call the past light cone at $x$. Since we assume the manifold to be time orientable, there exists a smooth vector field $t$ that is timelike at every $x \in \N$. Given this time orientation, we identify the future (past) light cone with the set of tangent vectors $v \in T_x\N$ such that $k_{\mu\nu} v^\mu t^\nu < 0$ (respectively $> 0$). Therefore, a tangent vector $v$ at $x$ is called \emph{future directed} (past directed) if it lies in the future (past) light cone at $x$.\\
Accordingly, a curve $\gamma : I \to \N$ is called timelike (spacelike, null, causal, future or past directed) if its tangent vector $\dot{\gamma}$ is timelike (spacelike, null, causal, future or past directed) at every $x \in \N$.  For every point $x \in \N$ we define the \emph{causal future/past} \gls{causalfuturepast} of $x$ as the set of all points $q \in \N$ that can be reached by a future directed causal curve originating in $x$. For any subset $S \in \N$ we define $J^\pm (S) = \bigcup_{x \in S} J^\pm(x)$ and $J(S) = J^+(S) \cup J^- (S)$. Finally, the future/past domain of dependence $\gls{futurepastdomainofdependence}$ of a set $S \subset \N$ is the set of all points $x \in \N$ such that every inextendible causal curve through $x$ intersects $S$. The \emph{domain of dependence} \gls{domainofdependence} of $S$ is the union of the future and past domain of dependence of the set $S$.
For more details on the causal structure of spacetime we refer to for example \cite[Chapter 8]{wald_GR}.\par
%
%
%
The notion of tangent bundles can be generalized to the notion of a vector bundle. Instead of ``attaching'' the vector spaces $T_x \N$ to every point $x$ of the manifold, we allow for the occurrence of arbitrary vector spaces, called the fibres of the vector bundle. A vector bundle then consists of the base manifold, in our case $\N$, the total space and a map $\pi$ from the total space to the base manifold, that can be locally trivialized. At each point of the base manifold, the pre-image of $\pi$ is the fibre of the vector bundle. To be precise we define, following \cite{rudolph_schmidt}:
\begin{definition}[Vector bundle]
	A smooth \emph{vector bundle} over $\N$ is a tuple $\gls{vectorbundle} = (E,\N, \pi)$, where $E$ is a smooth manifold and $\pi : E \to \N$ is a smooth surjective map satisfying:
	\begin{enumerate}
		\item For every $x \in \N$, $\pi^{-1}(x)$ is a vector space, called the fibre of the bundle at point $x$.
		\item There exists a finite dimensional vector space $F$, an open covering $\left\{ U_\alpha\right\}_\alpha$ of $\N$ and a family of diffeomorphisms $\chi_\alpha : \pi^{-1}(U_\alpha) \to U_\alpha \times F$ such that for all $\alpha$ it holds $\chi_\alpha \comp \text{pr}_1 =  \restr{\pi}{\pi^{-1}(U_\alpha)}$ and for every $x \in \N$ the map $\text{pr}_2 \comp \restr{\chi_\alpha}{\pi^{-1}(x)} : \pi^{-1}(x) \to F$ is linear.
	\end{enumerate}
\end{definition}
Here, the maps $\text{pr}_1$ and $\text{pr}_2$ denote the projection onto the first respectively second component of an element in $U_\alpha \times F$. The properties graphically mean that \emph{locally}, the vector bundle ``looks like" the product of the base manifold with the fibre. The tuples $(U_\alpha, \chi_\alpha)$ are called \emph{local trivializations} of the vector bundle. Like for vector spaces, we can define the sum and product of vector bundles, by using the according vector space definitions on the fibres of the bundle.\par
Let $\mathfrak{X}, \mathfrak{Y}$ be vector bundles over $\N$ with fibres $X_x$ and $Y_x$ at $x \in \N$. We denote by \gls{whitneysum} the \emph{Whitney sum} of the two vector bundles - the vector bundle over $\N$ whose fibres are given by the direct sum $X_x \oplus Y_x$. Similarly, one obtains the local trivializations of the Whitney sum from the trivializations of $\mathfrak{X}, \mathfrak{Y}$ and direct sums.\par
Accordingly, let $\mathfrak{X}, \mathfrak{Y}$ be vector bundles over $\N$ and $\widetilde{\N}$, with fibres $X_x$ and $Y_{\tilde{x}}$ at $x \in \N$, $\tilde{x} \in \widetilde{\N}$ respectively. We denote by \gls{outerproductbundle} the \emph{outer product} of the two vector bundles - the vector bundle over $\N \times \widetilde{\N}$ whose fibres are given by the tensor products $X_x \otimes Y_x$. Similarly, one obtains the local trivializations of the outer product from the trivializations of $\mathfrak{X}, \mathfrak{Y}$ and tensor products. \par
%
Finally, we generalize the notion of vector fields:
\begin{definition}[Sections of vector bundles]
Let $\mathfrak{X}=(E,\N,\pi)$ be a vector bundle with fibres $X_x=\pi^{-1}(x)$ at $x \in \N$. A \emph{smooth section} of the vector bundle is a smooth map $\gamma : \N \to E$ such that $\gamma(x) \in X_x$ for all $x \in \N$. The \emph{vector space of smooth sections} of $\mathfrak{X}$ is denoted by \gls{gammax}, the one with compactly supported sections is as usual denoted by \gls{gammaxzero}.
\end{definition}
In this language, a vector field $v$ is just a smooth section of the tangent bundle of a manifold, $v \in \Gamma(T\N)$. One may therefore identify the physical notion of fields with smooth sections of vector bundles. This point of view will be used to define the notion of differential forms in Section \ref{sec:differential_forms}.\par
In this thesis, we usually are interested in complex valued functions (or sections in general). Therefore, we view all occurring vector bundles as complex, in the sense that we take two distinct copies of the vector bundle, one representing the real, one the imaginary part of the bundle. A section of that complex vector bundle is just a pair of two sections of the real vector bundle under consideration. From now, if not specified explicitly, we will view all vector bundles, including the tangent bundle $T\N$, as complex vector bundles. Accordingly, smooth sections of those bundles will in general be complex valued.
%
%
%
%
%
%
%
%
%%%%%%%
%%PARTIAL DIFFERENTIAL OPERATORS AND GLOBAL HYPERBOLICITY
%%%%%%%
%
%
%
\subsection{Partial differential operators and global hyperbolicity}\label{sec:global_hyperbolicity}
When dealing with field theories, whether classical or quantum, one is, of course, interested in the dynamics of the fields. These are usually described by some partial differential equation, often of second order. In the following, we give a short introduction to the theory of certain partial differential operators acting on smooth sections of a vector bundle over the spacetime $(\M,g)$.\par
%
As we have seen, these smooth sections are generalizations of the notion of a field.  In the following, let $\mathfrak{X}$ denote a vector bundle over the manifold $\M$ and let $P: \Gamma(\mathfrak{X}) \to \Gamma(\mathfrak{X})$ be a partial differential operator acting on smooth sections of the bundle. As in the case of flat spacetime, we are interested in basic questions regarding the differential equation $Pf = j$, for example: Can we formulate a (globally) well posed initial value problem? Does the differential equation possess (unique) solutions? To answer these questions, we will now restrict to the case where $P$ is linear and of second order, as it is often the case in physical applications. One can show that for a certain class of such operators, namely normally hyperbolic partial differential operators of second order, we can rigorously treat these questions.\par
Choosing local coordinates $x=(x_\mu)$ on $\M$ and a local trivialization of $\mathfrak{X}$, a linear partial differential operator of second order is called \emph{normally hyperbolic} if it takes the form
\begin{align}
	P = - \sum_{\mu,\nu} g^{\mu \nu} \partial_\mu \partial_\nu + \sum_{\alpha} A_\alpha (x) \partial_\alpha + B(x) \formspace,
\end{align}
where $A_\alpha$ and $B$ are matrix-valued coefficients depending smoothly on the coordinate $x$ (see. \cite[Chapter 1.5]{baer_ginoux_pfaeffle}). One can also formulate a coordinate independent definition in terms of the principal symbol, which we will not present here (see for example \cite[Section 1.5]{baer_ginoux_pfaeffle} ). \par
%
Normally hyperbolic operators possess unique fundamental solutions (see for example the fundamental solutions to the wave operator as noted in Lemma \ref{lem:fundamental_solution_wave_operator}). These fundamental solutions fulfill certain physically important properties, such as a finite propagation speed smaller than the speed of light. Furthermore, specifying the initial data on some space-like hypersurface $X \in  \M$ specifies a unique solution on the domain of dependence $D(X)$ of $X$. Due to these properties, one often calls normally hyperbolic operators just \emph{wave operators}. But to state a \emph{globally} well posed initial value problem for a wave equation, we need to restrict the class of spacetimes $\M$ under consideration to those that possess space-like hypersurfaces $X$ whose domain of dependence is all of the spacetime, $D(X) = \M$. This leads to the notion of \emph{globally hyperbolic} spacetimes:
\begin{definition}[Global Hyperbolicity]
	A spacetime $\M$ is called \emph{globally hyperbolic} if there exists a Cauchy surface $\gls{sigma}$ in $\M$.
\end{definition}
\noindent Here, a Cauchy surface is a space-like hypersurface $\Sigma \subset \M$ such that every inextendible causal curve $\gamma$ intersects $\Sigma$ exactly once. One can show that Cauchy surfaces fulfill the desired property mentioned above, that is,  $D(\Sigma) = \M$. Furthermore, one can show that any globally hyperbolic spacetime $\M$ is foliated by a one-parameter family $\left\{ \Sigma_t \right\}_t$ of Cauchy surfaces (see for example \cite[Theorem 8.3.14]{wald_GR}). \par
In physical applications, one often finds the dynamics of a theory to be described by wave operators. Most prominently, the Klein-Gordon operator $(\square + m^2)$ acting on scalar fields, or its generalization, the wave operator acting on differential forms introduced in Section \ref{sec:differential_forms}, is normally hyperbolic. But there are also important physical field theories that are not described by wave operators, such as the Proca field treated in this thesis. It turns out that the Proca operator (see Definition \ref{def:proca_operator}) is a so called \emph{Green-hyperbolic} operator. These are again partial differential operators $P$ of second order acting on smooth sections of some vector bundle, such that $P$ (and its dual $P'$) posses fundamental solutions. Obviously, normally hyperbolic operators are Green-hyperbolic, but the opposite is not true. One can generalize some results obtained by studying normally hyperbolic operators to Green-hyperbolic operators. An introduction to this topic is given in \cite{baer_green-hyperbolic}, where it is also shown that the Proca operator is Green-hyperbolic but not normally hyperbolic.\par
For our application, the notion of Green-hyperbolicity is not of vast importance, but it is worth mentioning that there exists a more detailed mathematical background on the treatment of such operators.
A very detailed description of normally hyperbolic operators on Lorentzian manifolds, including proofs of the above statements regarding the initial value problem and the existence of fundamental solutions, is given in \cite{baer_ginoux_pfaeffle}, also with an overview of quantization. A shorter introduction to the topic is for example treated in \cite{baer-ginoux_classical-and-quantum-fields}, also with a description of quantization.
%
%
%
%
%
%
%%%
%
%
%
%%
%%%%%%%%%
%%%DIFFERENTIAL FORMS
%%%%%%%%
%
%
%
\subsection{Differential forms}\label{sec:differential_forms}
%
%
Differential forms provide an elegant, coordinate independent description of calculus on smooth manifolds. In particular, they generalize the notion of line- and volume-integrals that are known from analysis. Differential forms play a remarkable role in physics, as one can argue that they indeed describe fundamental physical entities. As an example, instead of viewing a classical force as a vector, one can think of it, more closely related to experiments, as a differential one-form that assigns a scalar to a tangent vector of a curve. This scalar is the (infinitesimal) work associated with the force along the curve. Also, differential forms allow for an elegant geometric description of field theories, for example the Maxwell and Proca field theories that we encounter in this thesis. In Maxwell's classical theory of electromagnetism, instead of viewing the electric and magnetic field (which are conceptually just forces) as the fundamental physical entities, one introduces the \emph{vector potential}, a one-form, consisting of the scalar electric potential and the vector potential associated with the magnet field. Experiments like the Aharonov-Bohm experiment allow for an interpretation of the vector potential as the fundamental physical object, rather than the associated electromagnetic field. \\
Even more fundamentally, the two main theories of physics, General Relativity and the Standard Model of particle physics, are field theories. They are deeply connected to a geometric interpretation and can be elegantly described using differential forms. \par
%
%
Despite of all this, differential forms are usually not part of the standard curriculum of physicists. We shall therefore introduce the basic aspects and definitions regarding differential forms that are used in this thesis. For a more detailed introduction we refer to the literature: For example \cite[Chapter 2 and 4]{rudolph_schmidt} or \cite[Appendix B]{wald_GR} provide introductions to the topic.\par
%
%
In the following, let $\N$ denote a smooth $N$-dimensional manifold, assumed to be Hausdorff, connected, oriented and para-compact, with either Lorentzian or Riemannian metric $k$ and Levi-Civita connection $\nabla$. For a Lorentzian manifold we use the sign convention $(-,+,\dots,+)$ of the metric $k$. The number of negative eigenvalues of $k$ is denoted by $s$, so $s=0$ for a Riemannian manifold and, in our convention, $s=1$ for a Lorentzian manifold.
Later, we will specify to a four dimensional (globally hyperbolic) spacetime consisting of a four dimensional manifold $\M$ with Lorentzian metric $g$ and Cauchy surface $\Sigma$ with induced Riemannian metric $h$.
%
We define:
\begin{definition}[Differential form]
	Let $p\in \{0,1,\dots,N\}$. A \emph{differential form} $\omega$ of degree $p$, or $p$-form for short, on the manifold $\N$ is an anti-symmetric tensor field of rank $(0,p)$. That is, at every point $x \in \N$, $\omega_x$ is an anti-symmetric multi-linear map
	\begin{align}
	\omega_x : \underbrace{T_x \N \times T_x \N \times \cdots \times T_x \N}_{p\text{-times}} \to \IR \formspace.
	\end{align}
	We denote the vector space\footnote{Naturally, addition and scalar multiplication are defined point-wise.} of $p$-forms on $\N$ by $\gls{omegap}$, the space with compactly supported ones by \gls{omegapz}.
\end{definition}
As an example, a zero-form $f \in \Omega^0(\N)$ is just a $C^\infty$-function from $\N$ to $\IR$, hence we can identify $\Omega^0(\N) = C^\infty (\N, \IR)$. A one-form $A \in \Omega^1(\N)$ is nothing more than a co-vector field and in a physical context usually denoted in local coordinates by $A_\mu$. Note, that alternatively one can directly define a $p$-form as a smooth section of the $p$-th exterior product of the co-tangent bundle and hence identify $\Omega^p(\N) = \Gamma \big( \largewedge^k T^*\N\big)$. As mentioned in Section \ref{sec:spacetime_geometry}, we view the tangent bundle as a complex bundle. Therefore, the sections of that bundle will be complex valued functionals. In that fashion, we will usually view the spaces $\Omega^p(\N)$ as complex valued differential forms.\par
%
Next we define the basic operations, besides addition and scalar multiplication, that one can perform on differential forms.
%
\begin{definition}[Exterior product]
	Let $A \in \Omega^p(\N)$ be a $p$-form and  $B\in \Omega^q(\N)$ a $q$-form on $\N$. \\
	The \emph{exterior product} $\gls{wedge}:\Omega^p(\N) \times \Omega^q(\N) \to \Omega^{p+q} (\N)$ is defined by
	\begin{align}
	(A \wedge B)_{\mu_1\dots\mu_p \nu_1\dots\nu_q} = \frac{(p+q)!}{p!q!}\, A_{[\mu_1 \dots \mu_p} B_{\nu_1\dots\nu_q]} \formspace,
	\end{align}
	where the anti-symmetrization of a tensor $T$ is given through
	\begin{align}
	T_{[\mu_1\dots\mu_p]} = \frac{1}{p!} \sum\limits_{\sigma\in S_N }\textrm{sgn}(\sigma) T_{\sigma(\mu_1)\dots\sigma(\mu_p)} \formspace.
	\end{align}
\end{definition}
Here, $S_N$ denotes the symmetric group\footnote{Usually the symmetric group is defined as the set of permutations of $\{1,2,\dots,N\}$ but we chose the index to run over $\{0,1,\dots,N-1\}$, identifying the time component with zero rather then one.} of degree $N$, consisting of permutations of the set $\{0,1,\dots,N-1\}$.
With this notion of multiplication, point-wise addition and scalar multiplication, the space $\gls{omega} \coloneqq \bigoplus_{p = 0}^\infty \Omega^p(\N) = \bigoplus_{p = 0}^N \Omega^p(\N)$ becomes an algebra, usually called the Grassmann- or \emph{exterior algebra} of differential forms on $\N$. We have used that obviously $\Omega^k(\N) =0$ for $k >N$ due to the anti-symmetrization.\par
Furthermore, we find a notion of how to \emph{pullback} differential forms on manifolds to another manifold, for example the pullback of a differential form on the spacetime $\M$ to differential forms on its Cauchy surface $\Sigma$. Given a $C^\infty$-map $\psi: \widetilde{\N} \to \N$, where $\N, \widetilde{\N}$ are manifolds, we can naturally define the pullback of a function $f \in \Omega^0(\N)$ to a function $(\psi^* f) \in \Omega^0(\widetilde{\N})$ by composing $f$ with $\psi$:
\begin{align}
\psi^* f \coloneqq f \comp \psi \formspace.
\end{align}
\newpage
With the pullback of functions defined, we can define how to \emph{push forward}, or carry along, vector fields on $\widetilde{\N}$ to vector fields on $\N$: Let $f\in \Omega^0(\N)$ and $\tilde{v} \in \Gamma(T\widetilde{\N})$ and $\tilde{x} \in \widetilde{\N}$. Then
\begin{align}
(\psi_* \tilde{v})_{\psi(\tilde{x})} (f) \coloneqq \tilde{v}_{\tilde{x}}(\psi^* f)
\end{align}
defines the vector field $(\psi_* v) \in \Gamma(T\N)$. With these basic operations at hand, we can generalize to define the pullback of differential forms:
\begin{definition}[Pullback]\label{def:pullback}
	Let $\N, \widetilde{\N}$ be manifolds of dimension $N,\widetilde{N}$ respectively, and let $\psi: \widetilde{\N} \to \N$ be a smooth map. Then, $\psi$ defines an algebra homomorphism $\psi^* : \Omega(\N) \to  \Omega(\widetilde{\N})$,
	called the \emph{pullback} of differential forms. For $\omega \in \Omega^p(\N)$, $\tilde{x} \in \widetilde{\N}$ and $\tilde{v}_i \in T_x \widetilde{\N}$, $i=1,2,\dots,p$, it is defined by
	\begin{align}
	\left( \psi^* \omega \right)_{\tilde{x}}  (\tilde{v}_1,\tilde{v}_2,\dots,\tilde{v}_p) \coloneqq \omega_{\psi(\tilde{x})} (\psi_* \tilde{v}_1, \dots , \psi_* \tilde{v}_p) \formspace.
	\end{align}
\end{definition}
%
%
%
%
On the exterior algebra we find a duality, provided by the Hodge operator:
\begin{definition}[Hodge dual]
	The hodge star operator $\gls{hodge}: \Omega^p(\N) \to \Omega^{N-p}(\N)$ is defined through
	\begin{align}
	B \wedge *A = \frac{1}{p!} B^{\mu_1\dots\mu_p}A_{\mu_1\dots\mu_p} \dvolk \formspace,
	\end{align}
	which yields the coordinate representation
	\begin{align}
	(*A)_{\mu_{p+1}\dots\mu_N} = \frac{\detk}{p!} \, \epsilon_{\mu_1\dots\mu_N} A^{\mu_1\dots\mu_p} \formspace.
	\end{align}
\end{definition}
Here, \gls{levicivita} denotes the fully antisymmetric tensor of rank $N$ (Levi-Civita symbol) satisfying $\epsilon_{12,\dots,N} =1$ and the \emph{volume element} \gls{dvolk} is defined by
\begin{align}
\left( \gls{dvolk} \right)_{\alpha_1\dots\alpha_N} = \detk \, \epsilon_{\alpha_1\dots\alpha_N} \formspace.
\end{align}
In a sense, the volume element describes how the curvature of the manifold deforms a unit volume.
The duality follows from the important property of the Hodge operator as stated in the following lemma:
\begin{lemma}
	Let $*$ denote the Hodge star operator on the exterior algebra $\Omega(\N) $. It holds that
	\begin{align}
	** = (-1)^{s+p(N-p)} \, \mathbbm{1} \formspace,
	\end{align}
	which is trivially equivalent to $*^{-1} = (-1)^{s+p(N-p)} \, *$.
\end{lemma}
\begin{proof}
	Let $A \in \Omega^p(\N)$ be a $p$-form on $\N$. Then:
	\begin{align}
	(*{*A})_{\mu_1 \dots \mu_p}
	&= \frac{\detk \, \detk}{p! \, (N-p)!} \; \epsilon_{\alpha_{p+1}\dots\alpha_N \mu_1 \dots \mu_p}\;\epsilon^{\alpha_{1}\dots\alpha_N}\;A_{\alpha_1\dots\alpha_p} \notag\\
	&= (-1)^{p(N-p)} \frac{\detk \, \detk}{p! \, (N-p)!} \; \epsilon_{\alpha_{p+1}\dots\alpha_N \mu_1 \dots \mu_p}\;\epsilon^{\alpha_{p+1}\dots\alpha_{N}\alpha_1\dots\alpha_p}\;A_{\alpha_1\dots\alpha_p}  \notag\\
	&= (-1)^{s+p(N-p)} \delta\indices{^{[\alpha_{1}}_{\mu_{1}}}\, \dots \, \delta\indices{^{\alpha_p ] }_{\mu_p}} \;A_{\alpha_1\dots\alpha_p} \notag\\
	&=  (-1)^{s+p(N-p)}\;A_{\mu_1\dots\mu_p} \formspace
	\end{align}
	We have used Lemma \ref{lem:epsilon_contraction} and, in the last step, that the anti-symmetrization is absorbed by contraction because $A$ is antisymmetric.
\end{proof}
%
%
%
%
%
Furthermore, we can equip the exterior algebra with a differentiable structure, introducing the notion of the exterior derivative.
\begin{definition}[Exterior derivative]
	The \emph{exterior derivative} $\gls{d}:\Omega^p(\N) \to \Omega^{p+1} (\N)$ is defined by the following properties:
	\begin{enumerate}
		\item $d$ is linear
		\item $d$ obeys a graded Leibniz rule: Let $A \in \Omega^p(\N)$ and  $B\in \Omega^q(\N)$, then
		\begin{align}
		d(A \wedge B) = dA \wedge B + (-1)^p \, A \wedge dB
		\end{align}
		\item $d$ is nilpotent, that is,  $d^2 = 0$.
	\end{enumerate}
	In local coordinates, this is equivalent to the representation
	\begin{align}
	(dA)_{\mu \alpha_1\dots\alpha_p} = (p+1)\, \nabla_{[\mu}A_{\alpha_1\dots\alpha_p]} \formspace.
	\end{align}
\end{definition}
An important property of the exterior derivative is that it commutes (or rather intertwines its action) with pullbacks (see \cite[Proposition 4.1.7]{rudolph_schmidt}).
A $p$-form $\omega \in \Omega^p(\N)$ is called \emph{exact} if there is a $(p-1)$-form $\alpha \in \Omega^{p-1}(\N)$ such that $\omega = d\alpha$. We call $\omega$ \emph{closed} if $d \omega =0$. Accordingly, the space of closed $p$-forms is denoted by \gls{omegapd}, the space of exact ones by \gls{domegap}. As usual, the ones with compact support are denoted by a subscript zero. Note, that every exact form is closed, using that $d$ is by definition nilpotent, but the reverse is in general not true. It does hold, however, on certain manifolds with trivial topology, such as Minkowski spacetime. This is expressed in the so called Poincar\'e-Lemma (see for example \cite[Chapter 4]{bott_tu}) based on the study of de Rham cohomology.\par
%
Moreover, $N$-forms can naturally be integrated. Using local coordinates and a partition of unity, we define the integral of $N$-forms via the well known integration on $\IR^N$:
\begin{definition}[Integration on manifolds]
	Let $\left\{U_\alpha, \psi_\alpha\right\}_\alpha$ be an atlas of the manifold $\N$ and $\left\{\chi_\alpha\right\}_\alpha$ a partition of unity subordinate to the locally finite open cover $\left\{U_\alpha\right\}_\alpha$. Let $x^\mu_{(\alpha)}$ be a coordinate basis of $\psi$ on $U_\alpha$. For any $N$-form $\omega \in \Omega^N_0(\M)$ we define the integral
	\begin{align}
	\int\limits_{\N} \omega &\coloneqq \sum_{\alpha} \int\limits_{\psi_\alpha (U_\alpha)} w(x_{(\alpha)}^0,\dots,x_{(\alpha)}^1)\; dx_{(\alpha)}^0 \cdots dx_{(\alpha)}^{N-1} \formspace,
	\end{align}
	where $w$ are the components of $\omega$ in the coordinates $x_{(\alpha)}^\mu$, that is $\omega = w dx_{(\alpha)}^0 \wedge \cdots \wedge dx_{(\alpha)}^{N-1}$.
	This definition is independent of the choice of the atlas and the partition of unity (see \cite[Proposition 3.3]{bott_tu}).
\end{definition}
With integration at our disposal, we present an important theorem regarding the integration of exact differential forms:
\begin{theorem}[Stoke's Theorem]\label{thm:stokes}
	Let $\N$ be an oriented manifold of dimension $N$ and let its boundary $\partial \N$ be endowed with the induced orientation. Let $\gls{inclusionmap} : \partial \N \hookrightarrow \N$ be the inclusion operator.
	Let $\omega \in \Omega^{N-1}_0(\N)$ be a compactly supported $(N-1)$-form on $\N$. Then it holds
	\begin{align}
	\int\limits_\N d\omega = \int\limits_{\partial \N} i^*\omega \formspace.
	\end{align}
\end{theorem}
\begin{proof}
	A proof is given in most of the introductory literature on differential geometry (see for example \cite[Chapter 17, Theorem 2.1]{lang}).
	Note that one can equivalently formulate Stoke's theorem on a \emph{compact} manifold but for {arbitrary} (that is, in general not compactly supported) $(N-1)$-forms on the manifold (see for example \cite[Theorem 4.2.14]{rudolph_schmidt}). This will be of importance in later calculations.
\end{proof}
%
Furthermore, we can define a bilinear map on $\Omega^p(\N)$ using the integration of $N$-forms:
\begin{definition}
	Let $A,B \in \Omega^p(\N)$ such that their supports have a compact intersection. Define the bilinear map $\gls{innerprod} : \Omega^p(\N) \times \Omega^p(\N) \to \IC$ by
	\begin{align}
	\langle A, B \rangle_\N \coloneqq  \int_{\N } A \wedge * B = \int_{\N } A_{\mu_1 \dots \mu_p}B^{\mu_1 \dots \mu_p}\,\dvolk \formspace.
	\end{align}
\end{definition}
Since by definition $A \wedge * B$ is a compactly supported $N$-form, this is well defined. We may sometimes refer to $\langle \cdot , \cdot \rangle_\N$ as an inner product for simplicity, even though it is not positive definite.
%
%
%
%
%
Using the exterior derivative, we define the interior or co-derivative:
\begin{definition}[Interior derivative]
	The \emph{interior derivative} $\gls{delta} : \Omega^p(\N) \to \Omega^{p-1}(\N)$ is defined by
	\begin{align}
	\delta \coloneqq (-1)^{s+1+N(p-1)}\, {*{d*}} \formspace.
	\end{align}
	From the defining properties of $d$ and $*$ it follows $\delta^2 =0$.
\end{definition}
Here, $s$ again denotes the number of negative eigenvalues of the metric $k$ of $\N$. In accordance with our nomenclature, we call a $p$-form $\omega$ co-exact if there exists a $\alpha \in \Omega^{p+1}(\N)$ such that $\omega = \delta \alpha$ and co-closed if $\delta \omega = 0$. Accordingly, the spaces of co-closed and co-exact $p$-forms are denoted by \gls{omegapdelta} and \gls{deltaomegap} respectively.\par
Using the exterior and interior derivative we define the partial differential operator:
\begin{definition}[D'Alembert Operator]
	The d'Alembert (or Laplace - de Rham) operator $\gls{dalembert}: \Omega^p(\N) \to \Omega^{p}(\N)$ is defined by
	\begin{align}
	\square \coloneqq \delta d +d \delta \formspace.
	\end{align}
\end{definition}
By definition of the exterior and interior derivative, it is easy to show that $\square$ commutes with both $d$ and $\delta$:
\begin{align}
\square d &= (\delta d + d \delta )d \notag \\
&= d \delta d \notag \\
&= d (\delta d + d \delta) \formspace,
\end{align}
and analogously for $\delta$.
The d'Alembert operator, and its generalization to $(\square + m^2)$ for some constant $m > 0$, are important examples for a normally hyperbolic differential operators (see Section \ref{sec:global_hyperbolicity}) and we may therefore sometimes just refer to them as \emph{wave operators}.\par
The sign convention in the definition of the exterior derivative is chosen such that on any Lorentzian or Riemannian manifold the interior derivative is formally adjoint to the exterior derivative, that is,  for $A \in \Omega^{p}(\N)$ and $B \in \Omega^{p+1}(\N)$ it holds that
\begin{align}
\langle dA , B \rangle_{\N} = \langle A , \delta B \rangle_\N \formspace,
\end{align}
which leads to a representation in local coordinates of the Manifold given by:
\begin{align}
(\delta A)_{\mu_2\dots\mu_p} = - \nabla^{\mu_1}A_{\mu_1\dots\mu_p} \formspace.
\end{align}
To see that this is consistent, let $A \in \Omega^{p-1}(\N)$ and $B \in \Omega^{p}(\N)$ such that their supports have compact intersection.
We obtain, using Stoke's Theorem \ref{thm:stokes}:
\begin{align}
0 &= \int \limits_{\partial \N} i^* (A \wedge *B) \notag\\
&= \int \limits_{\N} d(A \wedge *B)  \notag\\
&= \int \limits_{\N} dA \wedge *B + (-1)^{p-1} A \wedge d{*B} \notag\\
&= \int \limits_{\N} dA \wedge *B + (-1)^{p-1} A \wedge *{*^{-1}}\underbrace{d{*B}}_{\textrm{is a } (N-p+1) \textrm{ form.}} \notag\\
&= \int \limits_{\N} dA \wedge *B + (-1)^{p-1}(-1)^{s+(N-p+1)(N-N+p-1)} A \wedge *{*d{*B}} \notag\\
&= \int \limits_{\N} dA \wedge *B + (-1)^{p+(1-p)(p-1)} A \wedge *\delta B \formspace.
\end{align}
It can easily be proven by induction that $\big(p+(1-p)(p-1)\big)$ is odd for any $p \in \IN$, which yields the result
\begin{align}
\langle dA , B \rangle_{\N} = \langle A , \delta B \rangle_\N \formspace.
\end{align}
The definitions stated above thus fulfill the requirement of formal adjointness of the exterior and interior derivate on an arbitrary Lorentzian or Riemannian manifold $\N$.
In local coordinates we use a partial integration to obtain
\begin{align}
\langle dA , B \rangle_\N &= \int \limits_{\N} dA \wedge * B \notag\\
%&= \int \limits_{\N} \frac{1}{p!} (dA)^{\alpha_1\dots\alpha_p}\,B_{\alpha_1 \dots \alpha_p} \, \dvolk \notag\\
&= \int \limits_{\N}  \frac{p}{p!} \nabla^{[\alpha_1}A^{\alpha_2\dots\alpha_p]}\,B_{\alpha_1 \dots \alpha_p} \, \dvolk \notag\\
&= \int \limits_{\N}  \frac{1}{(p-1)!} \nabla^{\alpha_1}A^{\alpha_2\dots\alpha_p}\,B_{\alpha_1 \dots \alpha_p} \, \dvolk \notag\\
&= - \int \limits_{\N}  \frac{1}{(p-1)!} A^{\alpha_2\dots\alpha_p}\, \nabla^{\alpha_1}B_{\alpha_1 \dots \alpha_p} \, \dvolk \notag\\
&= \langle A, \delta B \rangle_\N \formspace,
\end{align}
which yields
\begin{align}
-\nabla^{\alpha_1}B_{\alpha_1 \dots \alpha p} = (\delta B)_{\alpha_2 \dots \alpha_p}\formspace.
\end{align}
On the four dimensional spacetime $(\M,g)$ the definitions of the Hodge star operator and the interior derivative simplify, such that
\begin{align}
*_{(\M)}*_{(\M)} &= (-1)^{p+1} \mathbbm{1} \\
\delta_{(\M)} &= *_{(\M)}{d_{(\M)}*_{(\M)}} \formspace ,
\end{align}
holds on the spacetime $(\M,g)$ and
\begin{align}
*_{(\Sigma)}*_{(\Sigma)} &= \mathbbm{1} \\
\delta_{(\Sigma)} &= (-1)^p *_{(\Sigma)}{d_{(\Sigma)}*_{(\Sigma)}}
\end{align}
holds on  $(\Sigma,h)$. In the following we will drop the subscript ${(\M)}$, since we will perform all the calculations on a four dimensional spacetime, except when explicitly noted (for example with a subscript $(\Sigma)$).
%
%
%
%
%
%
%
%
%%%%%%
%%CATEGORY THEORY
%%%%%%
\subsection{Category theory}\label{sec:cat-theory}
The description of Quantum Field Theory on Curved Spacetimes (QFTCS) in the framework of \name{Brunetti}, \name{Fredenhagen} and \name{Verch} \cite{Brunetti_Fredenhagen_Verch} is based on category theory. In this thesis, we will not go into detail on those categorical aspects, however we will need some basic definitions to formulate the theory rigorously, that is namely the notion of a category and that of covariant functors, since, in the used framework, the generally covariant QFTCS is a functor.\par
Here, we present definitions given in \cite[Appendix A.1]{baer_ginoux_pfaeffle} and refer to the appropriate literature for details. We define:
\begin{definition}[Category]
	A \emph{category} $\mathsf{Cat}$ consists of the following:
	\begin{enumerate}
		\item a class $\mathsf{Obj}_\mathsf{Cat}$ whose members are called \emph{objects},
		\item a set $\mathsf{Mor}_\mathsf{Cat}(A,B)$, for any two objects $A,B \in \mathsf{Obj}_\mathsf{Cat}$, whose elements are called \emph{morphisms},
		\item for any three objects $A,B,C \in \mathsf{Obj}_\mathsf{Cat}$ there is a map
		\begin{align}
\mathsf{Mor}_\mathsf{Cat}(B,C) \times \mathsf{Mor}_\mathsf{Cat}(A,B) &\to \mathsf{Mor}_\mathsf{Cat}(A,C) \notag\\
(\psi,\phi) &\mapsto \psi \comp \phi
		\end{align}
		called the composition of morphisms subject to the relations:\vspace{4mm}
		\begin{enumerate}[label=(\arabic*)]
			\item for non equal pairs $(A,B)$, $(A',B')$ of objects, the sets $\mathsf{Mor}_\mathsf{Cat}(A,B)$ and $\mathsf{Mor}_\mathsf{Cat}(A',B')$ are disjoint,
			\item for every object $A$ there exists a morphism $\text{id}_A \in \mathsf{Mor}_\mathsf{Cat}(A,A)$ such that it holds for all objects $B$, morphisms $\psi \in \mathsf{Mor}_\mathsf{Cat}(B,A)$ and $\phi \in \mathsf{Mor}_\mathsf{Cat}(A,B)$
			\begin{align}
				\text{id}_A \comp \psi &= \psi \quad \text{and}\\
				\phi \comp \text{id}_A &= \phi \quad,
			\end{align}
			\item the composition law is associative, that is for an objects $A,B,C,D$ and any morphisms $\psi \in \mathsf{Mor}_\mathsf{Cat}(A,B)$, $\phi \in \mathsf{Mor}_\mathsf{Cat}(B,C)$ and $\chi \in \mathsf{Mor}_\mathsf{Cat}(C,D)$ it holds
			\begin{align}
				(\chi \comp \phi) \comp \psi = \chi \comp (\phi \comp \psi) \formspace.
			\end{align}
		\end{enumerate}
	\end{enumerate}
\end{definition}
%
%
%
\begin{definition}[Functor]
	Let $\mathsf{Cat1}$ and $\mathsf{Cat2}$ be categories. A \emph{covariant functor} $\mathscr{A}: \mathsf{Cat1} \to \mathsf{Cat2}$ consists of the map $\mathscr{A} : \mathsf{Obj}_\mathsf{Cat1} \to \mathsf{Obj}_\mathsf{Cat2}$ and maps $\mathscr{A}: \mathsf{Mor}_\mathsf{Cat1}(A,B) \to \mathsf{Mor}_\mathsf{Cat2}\big(\mathscr{A}(A),\mathscr{A}(B)\big)$ for any two objects $A,B \in \mathsf{Obj}_\mathsf{Cat1}$ such that
	\begin{enumerate}
		\item {the composition is preserved, that is for all objects $A,B,C \in \mathsf{Obj}_\mathsf{Cat1}$ and for any morphisms $\psi \in \mathsf{Mor}_\mathsf{Cat1}(A,B)$ and $\phi \in \mathsf{Mor}_\mathsf{Cat1}(B,C)$ it holds
		\begin{align}
			\mathscr{A}(\phi \comp \psi) = \mathscr{A}(\phi) \comp \mathscr{A}(\psi) \formspace,
		\end{align}}
		\item{
			$\mathscr{A}$ maps identities to identities, that is for any object $A \in \mathsf{Obj}_\mathsf{Cat1}$ it holds
			\begin{align}
				\mathscr{A}(\text{id}_\mathsf{A}) = \text{id}_{\mathscr{A}(A)} \formspace.
			\end{align}
			}
	\end{enumerate}
\end{definition}
%
%
%
%
%
%
%
%
%
%
%
%
%%%%%%
%%SIGN CONVENTIONS
%%%%%%
%
%
\subsection{Sign conventions}\label{sec:sign_conventions}
At certain points throughout this chapter we have had a freedom of choice regarding the signs of some entities, in particular the sign of the signature of the Lorentzian metric $g$ and that of the interior derivative $\delta$. Though at this stage the choice can be made arbitrarily, we want to make it in a way that in the end allows us to make certain physical interpretations on some parameters. More precisely, we want to interpret the parameter $m$ of the Klein-Gordon equation\footnote{or its generalization on $p$-forms} $(\square + m^2) f = 0$ for a zero-form $f \in \Omega^0(\M)$ as a mass in the physical sense. With the chosen sign convention for $\delta$ we find, using ${\delta}f = 0$:
\begin{align}
	\square f
	&= (\delta d + d \delta) f \notag\\
	&= \delta d f \notag\\
	&= - \nabla^\mu \nabla_\mu f \formspace.
\end{align}
In the following heuristic (local) argument we see
\begin{align}
	\square + m^2
	&= -\nabla^\mu \nabla_\mu + m^2 \notag\\
	&\sim \partial_t^2 + \sum_i \partial_i^2 + m^2\notag\\
	&\sim -E^2 + \abs{\vector{p}}^2 + m^2
\end{align}
which yields the correct relativistic relation of energy, momentum and mass according to $E^2 = \abs{\vector{p}}^2 + m^2$.
A similar calculation holds for the Klein-Gordon operator generalized to act on one-forms. If we had found a ``wrong'' relation between energy, momentum and mass, we would have had to adapt the chosen signs. Usually one chooses the sign of the metric and the interior derivative such that they are in some sense mathematically convenient (although one might disagree with another one's choice). We have made the choice of the metric, such that the Cauchy surfaces become Riemannian rather that ``anti-Riemannian'' (with an all minus signature), which seems more natural to some. Also, a lot of the used references on spacetime geometry (in particular the book by \name{Wald} \cite{wald_GR}) use this sign convention, which makes the application of certain formulas easier. As mentioned, the sign of the interior derivative was chosen such that it is formally adjoint to the exterior derivative (with respect the specified inner product) on all Lorentzian and Riemannian manifolds. It seemed convenient for the actual calculations to fix the sign regardless of the signature of the metric of the underlying manifold. One could equivalently have fixed the opposite sign, yielding the two derivatives to be skew-adjoint, which is also done in the literature. However, in the end, one has one freedom left to make the energy-momentum-mass relation work: that is the sign in front of the mass in the Klein-Gordon equation and all other wave equations accordingly. Hence, one regularly also finds the Klein-Gordon equation to be defined with a flipped sign of the mass term. But for our case, we want the mass $m$ in any wave equation to appear with a positive sign.
%
%

\subsection{Additive Valuations: Polynomial-Time Learning}\label{sec:additive}
In this section, we consider bidders with additive valuations, again sharpening our results to show polynomial-time learnability. It is known that the better of the following two mechanisms achieves at least $\frac{1}{8}$ of the optimal revenue when all bidders have additive valuations~\cite{Yao15,CaiDW16}:

\vspace{.05in}	
\noindent\textbf{Selling Separately}: the mechanism sells each item separately using Myerson's optimal auction.

\vspace{.05in}	
\noindent \textbf{VCG with Entry Fee}: the mechanism solicits bids $\bold{b}=(b_1,\cdots, b_n)$ from the bidders, then offers each bidder $i$ the option to participate for an entry fee $e_i(b_{-i},D_i)$, which is the median of the random variable $\sum_{j\in[m]}(t_{ij}-\max_{k\neq i} b_{kj})^+$, where $t_i\sim D_i$\footnote{The entry fee function defined in~\cite{Yao15,CaiDW16} is slightly different. They showed that there exists an entry fee $X_i$, such that bidder $i$ accepts the entry fee with probability at least $1/2$. Then they argued that extracting $X_i/2$ as the revenue in the VCG with entry fee mechanism is enough to obtain a factor $8$ approximation. It is not hard to observe that our entry fee is accepted with probability exactly $1/2$, thus our entry fee is at least as large as $X_i$. So our mechanism also suffices to provide a factor $8$ approximation.}. This random variable is exactly bidder $i$'s utility when her type is $t_i$ and the other bidders' are $b_{-i}$. If bidder $i$ chooses to participate, she pays the entry fee and can take any item $j$ at price $\max_{k\neq i} b_{kj}$. Notice that the mechanism never over allocate any item, as only the highest bidder for an item can afford it. %Moreover, this mechanism is DSIC, because the entry fee and item prices for bidder $i$ only depend on the other bidders' bids and bidder $i$'s type distribution but not her bid.

Indeed, only counting the revenue from the entry fee in the second mechanism and the optimal revenue from selling the items separately already suffices to provide an $8$-approximation~\cite{Yao15, CaiDW16}. 

\begin{theorem}[\cite{CaiDW16}]\label{thm:UB additive}
	Let $\srev$ be the optimal revenue for selling the items separately and $\brev$ be the expected entry fee collected from the VCG with entry fee mechanism. Then $\opt\leq 6\cdot \srev+2\cdot\brev.$ 
\end{theorem}

Goldner and Karlin~\cite{GoldnerK16} showed that one sample suffices to learn a mechanism that achieves a constant fraction of the optimal revenue when $D_{ij}$ is regular for all $i\in[n]$ and $j\in[m]$. %In the rest of the section, we discuss
We show how to learn an approximately optimal mechanism in the other two models.
% (1) all $D_{ij}$ are supported on $[0,H]$, and (2) direct access to distributions $\hat{D}_{ij}$, where $||\hat{D}_{ij}-D_{ij}||_K\leq \epsilon$ for all $i\in[n]$ and $j\in[m]$.
\begin{theorem}\label{thm:additive}
	When the bidders have additive valuations and\begin{itemize}
		\item $D_{ij}$ is supported on $[0,H]$ for all bidder $i$ and item $j$, we can learn in polynomial time a mechanism whose expected revenue is at least $\frac{\opt}{32}-{\epsilon}\cdot H$ with probability $1-\delta$ given $O\left(\left(\frac{m}{\epsilon}\right)^2 \cdot\left(n\log n\log \frac{1}{\epsilon}+\log\frac{1}{\delta} \right)\right)$ samples from $D$; or
		\item  we are only given access to distributions $\hat{D}_{ij}$ where $||\hat{D}_{ij}-D_{ij}||_K\leq \epsilon$ for all bidder $i$ and item $j$, there is a polynomial time algorithm that constructs a mechanism whose expected revenue under $D$ is at least $\frac{\opt}{266}-96\epsilon\cdot mnH$ when $\epsilon\leq \frac{1}{16\max\{m,n\}}$.
	\end{itemize}
\end{theorem}

\noindent\textbf{Sample Access to Bounded Distributions:} Goldner and Karlin's proof~\cite{GoldnerK16} can be directly applied to the bounded distributions to show a single sample suffices to learn a mechanism whose expected revenue approximates the $\brev$. Then as $\srev$ is the revenue of $m$ separate single-item auctions, we can use the result in~\cite{MorgensternR16} to approximate it. See Theorem~\ref{thm:additive bounded} in Appendix~\ref{sec:additive bounded} for further details.

\vspace{.05in}
\noindent\textbf{Direct Access to Approximate Distributions:} for each single item, we apply Theorem~\ref{thm:unit-demand} to learn an individual auction, then run these learned auctions in parallel. Clearly, the combined auction's revenue approximates $\srev$. For $\brev$, we show that for every bidder $i$ and every bid profile $b_{-i}$ of the other bidders, the event that corresponds to bidder $i$ accepting any entry fee is \emph{single-intersecting} (see Definition~\ref{def:single-intersecting}). This implies that the probability for a bidder to accept an entry fee under $\hat{D}$ and $D$ is close (Lemma~\ref{lem:Kolmogorov stable for sc}). So we can essentially use the median of $\sum_{j\in[m]}(t_{ij}-\max_{k\neq i} b_{kj})^+$ with $t_i\sim\hat{D}_i$ as the entry fee. See Theorem~\ref{thm:additive Kolmogorov} in Appendix~\ref{sec:additive Kolmogorov} for further details.
\section{Recurrent Submodular Welfare}
Let $f(S): 2^{\A} \rightarrow \mathbb{R}_{\geq 0}$ be a monotone submodular function over a universe $\A$ of $k$ elements, such that $f(\emptyset) = 0$. In the {\em blocking} setting, each element $i \in \A$ is associated with a known deterministic {\em delay} $d_i \in \mathbb{N}_{>0}$, such that once the arm is played at some round $t$, it becomes unavailable for the next $d_i-1$ rounds, namely, in the interval $\{t, \dots, t+d_i-1\}$. At each round $t \in [T]$, the player chooses a subset $\A_t$ of available (i.e., non-blocked) elements and collects a reward $f(\A_t)$. The goal is to maximize the total reward collected, i.e., $\sum_{t \in [T]} f(\A_t)$, within an unknown time horizon $T$. 

Before we present our algorithm, we provide ``bad'' instances for two natural approaches to \rsm.

\begin{remark} \label{rem:greedy}
The greedy approach of choosing $\A_t$ to be the set of all available elements at round $t \in [T]$ can be as bad as a $\frac{1}{k}$-approximation. In order to see that, consider the monotone (budget-additive) submodular function $f(S) = \min\{|S|, 1\}$. Let $k$ be the number of elements with delay $d_i = k$ for each $i \in \A$. Assuming an infinite time horizon, the optimal strategy collects an average reward of $1$, simply by choosing one element at a time in a round-robin manner. However, the average reward of the greedy approach in this case is $\frac{1}{k}$.
\end{remark}

\begin{remark}
The independent randomized sampling approach of adding each arm $i$ to $\A_t$ independently with probability $\frac{1}{d_i}$, if available, can be as bad as a $(1 - \frac{1}{\sqrt{e}} )$-approximation. Consider the same setting as in Remark \ref{rem:greedy}, where for $T \to \infty$ the optimal average reward is $1$. However, the average expected reward of the independent randomized sampling strategy is $1 - (1 - p)^k$, where $p = \frac{1}{2k-1}$ is the probability that each element is selected at each round (in stationarity). For $k \to \infty$, we have that $1 - (1 - p)^k \to 1- e^{-\frac{1}{2}} \approx 0.393$.

\end{remark}
We provide an efficient randomized $\left(1-\frac{1}{e}\right)$-approximation algorithm for \rsm. Informally, the algorithm starts by considering, for each element $i \in \A$, a sequence of rational numbers of the form $\{t\cdot \frac{1}{d_i}\}_{t \in [T]}$. Then, these sequences are {\em interleaved} by randomly adding an {\em offset} $r_i$, drawn uniformly at random from $[0,1]$, for each $i \in \A$ to the corresponding sequence. At every round $t \in [T]$, the algorithm chooses a set $\A_t$, consisting only of elements for which the (perturbed) interval $L_{i,t} = [t\cdot \frac{1}{d_i}+ r_i, (t+1)\cdot \frac{1}{d_i}+ r_i )$ contains an integer.

\begin{algorithm}[\is (\IS)]
For each element $i \in \A$, let $r_i \sim U[0,1]$ be a random {\em offset} drawn uniformly from $[0,1]$. 
At every round $t = 1, 2, \dots$,  let $\A_t \subseteq \A$ be the subset of elements such that for any $i \in \A_t$, the interval $L_{i,t} = [t\cdot \frac{1}{d_i} + r_i, (t+1) \cdot \frac{1}{d_i} + r_i)$ contains an integer. Choose the elements $\A_t$ and collect the reward $f(\A_t)$.
\end{algorithm}


\subsection{Correctness and approximation guarantee.} 
We first show the algorithm is correct, namely, that the elements chosen at each round respect the blocking constraints. The correctness is established by the following simple observation:

\begin{restatable}{fact}{restatefactalwaysavailable}\label{inter:fact:alwaysavailable}
At any $t \in [T]$, all the elements in $\A_t$ are available (i.e., not blocked).
\end{restatable}

In order to prove the competitive guarantee of our algorithm, we first construct a convex programming (CP)-based (approximate) upper bound on the optimal reward. Although our algorithm never computes an optimal solution to this CP, this step allows us to prove our guarantee, leveraging results on the correlation gap of submodular functions. For $\bm{d}^{-1} \in \mathbb{R}^k$ such that $(\bm{d}^{-1})_i = \frac{1}{d_i}, \forall i \in [k]$, consider the following formulation based on the concave closure $f^+$ of $f$:
\begin{align}
\maximize_{\z \in \mathbb{R}^k}~~ T \cdot f^+(\z)~~\textbf{s.t.}~~ \bm{0} \preceq \z \preceq \bm{d}^{-1}. \tag{\textbf{CP}} \label{cp:CP}
\end{align}

In \eqref{cp:CP}, each variable $z_{i}$ can be thought of as the fraction of rounds where element $i\in \A$ is chosen. Intuitively, the constraints indicate the fact that, due to the blocking, each element $i \in \A$ can be played at most once every $d_i$ steps. 
In order to derive \eqref{cp:CP}, we start from a non-convex integer program (IP) with 0-1 variables $\{x_{i,t}\}_{i \in \A, t \in [T]}$, each indicating whether element $i \in \A$ is used at round $t \in [T]$. The objective is to maximize $\sum_{t \in [T]} \sum_{S \subseteq \A} f(S) \prod_{i \in S} x_{i,t} \prod_{i \notin S}(1 - x_{i,t})$ subject to natural blocking constraints. For integral solutions, the above objective is equivalent to $\sum_{t \in [T]} f^+(\x_t)$ (where $(\x_t)_i = x_{i,t}$) and, thus, the above relaxation is simply the result of averaging over time the variables and constraints of this IP. By using the concavity of $f^+$, we are able to show that \eqref{cp:CP} yields an (approximate) upper bound on the optimal solution of \rsm, while the approximation becomes exact as $T$ increases.

\begin{restatable}{lemma}{restateStructuralCP}\label{lem:structural:CP}
Let $\Rew^{CP}(T)$ be the optimal solution to \eqref{cp:CP} and $\OPT(T)$ be the optimal solution over $T$ rounds. We have
$
\Rew^{CP}(T) \geq \OPT(T) - \mathcal{O}(d_{\max} f(\A)),
$ where $d_{\max} = \max_{i \in \A}\{d_i\}$.
\end{restatable}

\begin{remark}
By replacing $f^+(\z)$ in \eqref{cp:CP} with the multi-linear extension $F(\z)$, the formulation no longer yields an upper bound on the optimal reward (not even asymptotically). Indeed, consider a function $f$ over a ground set $\A=\{1,2\}$ with $d_1 = d_2 = 2$, such that $f(\emptyset) = 0$, $f(\{1\}) = f(\{2\}) = 2$ and $f(\{1,2\}) = 3$. For $T \to \infty$, the optimal average reward is $2$, simply by choosing the two elements interchangeably. However, the formulation based on $F(\z)$ in that case would be to maximize $2z_1(1-z_2) + 2z_2(1-z_1) + 3 z_1 z_2$ subject to $z_1,z_2 \leq \frac{1}{2}$, which has a global maximum of $\frac{7}{4} < 2$.
\end{remark}


Before we complete the proof of our first main result, we first compute the probability that $i \in \A_t$, i.e., an element $i \in \A$ is sampled at round $t \in [T]$:

\begin{restatable}{fact}{restatefactsampling}\label{inter:fact:sampling}
For any $i \in \A$ and $t \in [T]$, we have
$\Pro{i \in \A_t} = \Pro{L_{i,t} \cap \mathbb{N} \neq \emptyset } = \frac{1}{d_i}.
$
\end{restatable}

\noindent{\em Proof of Theorem \ref{thm:interleavedSubmodular}.} 
Let us denote by $S \sim {\bf p}$ with ${\bf p} \in [0,1]^k$ the random set $S \subseteq \A$, where each element $i$ participates in $S$ independently with probability equal to $p_i$. 
By Fact~\ref{inter:fact:sampling} and due to the randomness of the offsets $\{r_i\}_{i \in \A}$, we have that $\A_t \sim {\bf d}^{-1}$ for each $t \in [T]$. Let $\z^*$ be an optimal solution to \eqref{cp:CP}. By monotonicity of $f$ and the fact that $\z^* \preceq \bm{d}^{-1}$, for the expected value of $f(\A_t)$ at any round $t \in [T]$, we know that $\Ex{\A_t \sim \bm{d}^{-1}}{f(\A_t)} \geq \Ex{\A_t \sim \z^*}{f(\A_t)}$. Moreover, by definition of the multi-linear extension, we have that $\Ex{\A_t \sim \z^*}{f(\A_t)} = F(\z^*)$, while by Lemma~\ref{lem:correlationgap} (the correlation gap of submodular functions), we have that, $F(\z) \geq \left(1 - \frac{1}{e}\right)f^+(\z)$ for any vector $\z \in [0,1]^k$. By combining the above facts, we can see that
\begin{align*}
\Rew^{IS}(T) = \sum_{t \in [T]} \Ex{\A_t \sim \bm{d}^{-1}}{f(\A_t)} \geq 
\sum_{t \in [T]} F(\z^*) \geq \left(1 - \frac{1}{e}\right)T\cdot f^+(\z^*) = \left(1 - \frac{1}{e}\right) \Rew^{CP}(T).
\end{align*}
Therefore, by Lemma~\ref{lem:structural:CP}, we can conclude that $\Rew^{IS}(T) \geq \left(1 - \frac{1}{e}\right)\OPT(T) - \mathcal{O}(d_{\max} f(\A))$.
\qed
\newline

In Appendix \ref{appendix:hardness}, we provide a $\left(1-\frac{1}{e}\right)$-hardness result for \rsm, thus proving that the guarantee of Theorem~\ref{thm:interleavedSubmodular} is asymptotically tight. This result, which holds even for the special case where $d_{\max} = o(T)$ (that is when the delays are significantly smaller than the time horizon), is proved via a reduction from the SWM problem with identical utilities, in a way that the constructed \rsm instance accepts w.l.o.g. solutions of a simple periodic structure.

\begin{restatable}{theorem}{restateSubmodularHardness}\label{thm:submodular:hardness}
For any $\epsilon>0$, there exists no polynomial-time $\left(1-\frac{1}{e} + \epsilon \right)$-approximation algorithm for the \rsm problem, unless ${\bf P}={\bf NP}$, even in the special case where $d_{\max} = o(T)$.
\end{restatable}
\section{Local Search Performance}\label{sec:local search}


In this section we discuss {\em local search}~\cite{lenstra2003local} (described in \hyperref[sec:preliminaries]{Section~\ref{sec:preliminaries}}).
Local search often gives better results than Greedy, at the cost of a slower running time --- for example for submodular maximization subject to the intersection of $k$ matroids \cite{lee2009submodular,filmus2012power}, and for $k$-set packing \cite{SviridenkoW13,Cygan13,FurerY14}. For some interesting recent results about local search in \textit{beyond-worst-case} settings and on geometric optimization we refer the reader to~\cite{cohen2016local,cohen2014unreasonable,cohen17}.

Somewhat surprisingly, it was not known (to our knowledge) how local search performs for $p$-systems and $p$-extendible systems. (We recall that the greedy algorithm gives a factor of $1/p$ for maximization of an additive function and $1/(p+1)$ for maximization of a monotone submodular function under these constraints.)
Here, we prove that local search in fact performs worse than Greedy for these constraints. Although it gives a $1/p$-approximation for cardinality maximization under a $p$-system constraint (essentially by definition), it does not give any bounded approximation factor for additive function maximization under a $p$-system, and only a $1/p^2$-approximation under a $p$-extendible system. 


\subsection{Local search fails for $p$-systems}
We construct simple examples where local search will not recover any fraction of the maximum-weight solution for $p$-systems (even if it is arbitrarily stable, $p=2$, and even if we allow large exchange neighborhoods). In particular, consider a ground set $X = A \cup \{e^*\}$ where $|A|=n$. The independent sets of $\I$ are:
\begin{itemize}
\item any subset of $A$, or
\item $e^*$ plus any subset of at most $n/2$ elements of A. 
\end{itemize}

Note that this is a 2-system, because for $S \subseteq X$, any independent subset of $S$ can be extended to an independent set of size at least $\min \{|S|, n/2\}$, and the maximum independent subset of $S$ has size at most $\min \{|S|,n\}$. The weights could be 0 on $A$, and 1 on the special element $e^*$. So the optimum is $w(e^*)$ = 1 (observe that the optimal solution is $c$-stable for arbitrarily large $c$). However, $A$ is a local optimum, unless we are willing to swap out $n/2$ elements, which is not possible for efficient local search.



\subsection{Lower bound for $p$-extendible systems}

Let us consider the following instance. Let $X = A \cup B$ where $A, B$ are disjoint sets. We define $\I \subseteq 2^X$ as follows: $S \in \I$ iff
\begin{itemize}
\item $|S \cap A| + p |S \cap B| \leq |A|$, or
\item $p|S \cap A| + |S \cap B| \leq |B|$.
\end{itemize}

\begin{lemma}
For any $A,B$ disjoint, the above is a $p$-extendible system.
\end{lemma}

\begin{proof}
Let $S \subseteq T$ and $i \in X \setminus T$ be such that $S+i \in \I$ and $T \in \I$. We need to prove that there is $Z \subseteq T \setminus S, |Z| \leq p$ such that $(T \setminus Z) + i \in \I$. We can assume that $|T \setminus S| > p$, because otherwise we can set $Z = T \setminus S$ and obviously $(T \setminus Z) + i = S + i \in \I$.
Assuming $|T \setminus S| > p$, let $Z$ be an arbitrary set of $p$ elements from $T \setminus S$. We consider 2 cases: If $|T \cap A| + p |T \cap B| \leq |A|$, then $|(T \setminus Z) \cap A| + p |(T \setminus Z) \cap B| \leq |A| - p$. Adding the element $i$ can increase the left-hand side by at most $p$, and so $|(T \setminus Z + i) \cap A| + p |(T \setminus Z + i) \cap B| \leq |A|$. Similarly, in the second case, if $p |T \cap A| + |T \cap B| \leq |B|$, then $p |(T \setminus Z) \cap A| + |(T \setminus Z) \cap B| \leq |B| - p$. Adding the element $i$ can increase the left-hand side by at most $p$, and so $p |(T \setminus Z + i) \cap A| + |(T \setminus Z + i) \cap B| \leq |B|$. 
\end{proof}

Now we choose the cardinalities of $A$ and $B$ and the weights of their elements appropriately to get a negative result.

\begin{lemma}
For $\epsilon>0$, let $|A| = n$ and $|B| = (p - \epsilon) n$, and set the weights as $w_a = 1$ for $a \in A$ and $w_b = p - \epsilon$ for $b \in B$.
Then $A$ is a local optimum of value $w(A) = w(B) / (p - \epsilon)^2$, unless the local search explores exchanges of size at least $\frac{\epsilon}{p} n$.
\end{lemma}

\begin{proof}
Both $A$ and $B$ are independent sets. 
Note that for any $i \in B$, we need to remove $Z \subseteq A$ of cardinality at least $|Z| = p$ to obtain $S = (A \setminus Z) + i$ satisfying $|S \cap A| + p|S \cap B| \leq |A|$. More generally, for $Y \subseteq B$, we need to remove $Z \subseteq A, |Z| = p|Y|$ to obtain $S = (A \setminus Z) \cup Y$ that satisfies $|S \cap A| + p|S \cap B| \leq |A|$. Possibly, we could satisfy the second condition, $p|S \cap A| + |S \cap B| \leq |B|$, but this will not happen unless $|A \setminus Z| = |S \cap A| \leq |B| / p = (1 - \frac{\epsilon}{p}) n$. Therefore, we would need to remove $Z$ of cardinality at least $\frac{\epsilon}{p} n$. If the swaps considered are smaller than $\frac{\epsilon}{p} n$ then $A$ is a local optimum because adding $Y \subseteq B$ and removing $Z \subseteq A, |Z| = p |Y|$ results in a solution of lower weight. In conclusion, $A$ is a local optimum of value $w(A) = n$, while the optimum is $OPT = w(B) = (p-\epsilon)^2 n$.
\end{proof}


\subsection{Upper bound for $p$-extendible systems}

Here we prove that local search does in fact provide a $1/p^2$-approximation for weighted maximization under a $p$-extendible system. More generally, we prove (here, we will ignore the technicalities of stopping the local search in polynomial time as this can be handled using standard techniques, while losing $1/poly(n)$ in the approximation factor) the following: 



\begin{theorem} \label{th:LS-approx}
For any $p$-extendible system $\I \subseteq 2^X$ and a monotone submodular function $f:2^X \rightarrow \RR_+$,
local search with $(p,1)$-swaps (including at most $1$ element and removing at most $p$ elements) provides a $1/(p^2+1)$-approximation. For additive $f$, the factor is $1/p^2$.
%If $f$ is additive, the approximation factor is $1/p^2$.
\end{theorem}



\begin{proof}
Let $A$ be a local optimum under $(p,1)$-swaps, and let $B$ be an optimal solution. (For convenience, let us also assume that we always try to add elements to $A$ if possible, even if they bring zero marginal value.) We proceed in two steps, the first one inspired by the analysis of the greedy algorithm for $p$-extendible systems \cite{calinescu2011maximizing} and the second one similar to other analyses of local search.

Let $A = \{a_1,\ldots,a_k\}$ be a greedy ordering of $A$ in the sense that $a_1$ is the element of $A$ maximizing $f_\emptyset(a_1)$; given $a_1$, $a_2$ is the element of $A-a_1$ maximizing $f_{\{a_1\}}(a_2)$, $a_3$ is the element of $A-a_1-a_2$ maximizing $f_{\{a_1,a_2\}}(a_3)$, etc. Using the $p$-extendible property, there is a subset $B_1 \subseteq B, |B_1| \leq p$ such that $(B \setminus B_1) + a_1 \in \I$. Further, since $\{a_1,a_2\} \in \I$, there is a subset $B_2 \subseteq B \setminus B_1, |B_2|\leq p$ such that $(B \setminus (B_1 \cup B_2)) \cup \{a_1,a_2\} \in \I$, etc. Generally, there are disjoint subsets $B_1,\ldots,B_k \subseteq B, |B_i| \leq p$ such that $(B \setminus (B_1 \cup \ldots B_i)) \cup \{a_1,\ldots,a_i\} \in \I$. In fact, if $|A| = k$, the sets $B_1,\ldots,B_k$ form a partition of $B$. Otherwise there would be additional elements in $B \setminus (B_1 \cup \ldots \cup B_k)$ which can be added to $A$, which would contradict the local optimality of $A$.

Now, we claim that for each $b \in B_i$, we have $f_A(b) \leq p f_{\{a_1,\ldots,a_{i-1}\}}(a_i)$. If not, we would be able to add $b$ and, since $\{a_1,\ldots,a_{i-1},b\} \in \I$, we could remove at most $p$ elements $Z \subseteq A \setminus \{a_1,\ldots,a_{i-1}\}$ so that $(A \setminus Z) + b \in \I$. By submodularity and the greedy ordering, we would have $f(A \setminus Z) \geq f(A) - p f_{\{a_1,\ldots,a_{i-1}\}}(a_i)$ and again by submodularity, we would have $f((A \setminus Z) + b) \geq f(A \setminus Z) + f_A(b) > f(A \setminus Z) + p f_{\{a_1,\ldots,a_{i-1}\}}(a_i) \geq f(A)$. Therefore, this would be an improving local swap.

Since $A$ is a local optimum, we conclude that $f_A(b) \leq p f_{\{a_1,\ldots,a_{i-1}\}}(a_i)$ for each $b \in B_i$. Since $B = B_1 \cup \ldots \cup B_k$ and $|B_i| \leq p$, we have by submodularity
$$ f_A(B) \leq \sum_{i=1}^{k} \sum_{b \in B_i} f_A(b) \leq \sum_{i=1}^{k} |B_i| p f_{\{a_1,\ldots,a_{i-1}\}}(a_i)
\leq p^2 \sum_{i=1}^{k} f_{\{a_1,\ldots,a_{i-1}\}}(a_i) \leq p^2 f(A) $$
For $f$ monotone submodular, we have $f(B) \leq f(A) + f_A(B) \leq (p^2+1) f(A)$.
For $f$ additive, we have $f(B) = f_A(B) \leq p^2 f(A)$. This completes the proof.
\end{proof}

\subsection{Recovery for $p$-extendible systems}
%Note that in the proof of \hyperref[th:LS-approx]{Theorem \ref{th:LS-approx}} we can forget about $A\cap B$ and restrict our attention only to comparing the value of $A\sm B$ and $B\sm A$. This turns out to be useful for exact recovery as we can perturb only $A\sm B$. The following theorem intuitively tells us that \textit{local optima of stable instances are global optima}.
Note that in the proof of \hyperref[th:LS-approx]{Theorem \ref{th:LS-approx}}, if we focus on comparing the values of $A\sm B$ and $B\sm A$, we will be able to get exact recovery as we can perturb only $A\sm B$. The following theorem intuitively tells us that \textit{local optima of stable instances are global optima}.
\begin{theorem} \label{th:LS-recovery}
Given a $p$-extendible system $\I \subseteq 2^X$ and a monotone submodular function $f:2^X \rightarrow \RR_+\cup\{0\}$ we wish to maximize, if the optimal solution $B$ is $(p^2+1)$-stable, then local search with $(p,1)$-swaps exactly recovers it. If $f$ is additive, recovery holds if $B$ is $p^2$-stable.
\end{theorem}

\begin{proof}
The basic idea is that we can contract the elements that belong to $A\cap B$ and then use the same charging argument from above. Using the notation from the proof of \hyperref[th:LS-approx]{Theorem \ref{th:LS-approx}}, for elements $a_i\in A\cap B$ the corresponding $B_i$ block is just $\{a_i\}$. Now we can rename elements in $A\sm B=\{a_1,\dots,a_m\}$ with corresponding blocks $B_1,\dots,B_m$ such that $B\sm A = B_1 \cup \ldots \cup B_m$ and $|B_i| \leq p$. Rewriting the local search guarantee:
$$f_A(B\sm A) \leq \sum_{i=1}^{m} \sum_{b \in B_i} f_A(b) \leq \sum_{i=1}^{m} |B_i| p f_{\{a_1,\ldots,a_{i-1}\}}(a_i)
\leq p^2 \sum_{i=1}^{m} f_{\{a_1,\ldots,a_{i-1}\}}(a_i) \leq p^2 f(A\sm B)$$
Since $f_A(B\sm A)=f(B\cup A)-f(A)\ge f(B)-f(A)$, we can $(p^2+1)$-perturb the input (only the marginal of elements in $A\sm B$) and get: $\tilde{f}(B)=f(B) \le f(A)+p^2f(A\sm B)=\tilde{f}(A)$, hence contradicting the $(p^2+1)$-stability.
In the case of additive $f$, $f_A(B\sm A)= f(B\sm A)$ and $\tilde{f}(B)=f(B)=f(B\sm A)+f(B\cap A)\le p^2f(A\sm B) +f(B\cap A)\le f(A)+(p^2-1)f(A\sm B)=\tilde{f}(A)$, where we $p^2$-perturbed the instance, hence contradicting the $p^2$-stability of the instance.
\end{proof}



%For a constraint $(X,\I)$, we wish to find the $max\{f(S): S\in \I\}$ ($f$: monotone submodular) where $\I$ is the intersection of $p$ matroids: $\I=\cap_{i=1}^p \I_i$. We prove that local search with ($p,1$)-swaps exactly recovers the optimal solution if it is $(p+1)$-stable. For one matroid $p=1$ (a matroid is 1-extendible), our previous \hyperref[th:LS-recovery]{Theorem~\ref{th:LS-recovery}} implies:




\subsection{Recovery for the intersection of Matroids}
If the independence system $\I$ is the intersection of $p$ matroids: $\I=\cap_{i=1}^p \I_i$, local search with ($p,1$)-swaps recovers $(p+1)$-stable optimal solutions (for proof, see \hyperref[app:LSrecovery]{Appendix~\ref{app:LSrecovery}}). 

%Since a matroid is 1-extendible, our previous \hyperref[th:LS-recovery]{Theorem~\ref{th:LS-recovery}} implies:
%\begin{corollary}
%Given a matroid $(X,\I)$ and $f$ monotone submodular, such that the optimal solution is 2-stable, Local Search exactly recovers it.
%\end{corollary}
%\begin{proof}
%Suppose again that $A$ is the local optimum solution and $B$ is the global optimum. By the local search criterion we have that $\exists$ a bijection $\pi:A\setminus B \to B\setminus A:$
%\begin{itemize}
%\item 1. $\forall x \in A\setminus B: (A-x+\pi(x)) \in \I$
%\item 2. $\forall x \in A\setminus B: f(A-x+\pi(x))\le f(A)$
%\end{itemize}
%Since $(A-x)\subseteq (A+\pi(x)-x)$, using the submodularity for adding $x$, we get :
%\[
%-(f(A)-f(A-x))+(f(A+\pi(x))-f(A))\le 
%\]
%\[
%\le -(f(A+\pi(x))-f(A+\pi(x)-x)) + (f(A+\pi(x))-f(A))\le0
%\]
%Rearranging we get: $f(A+\pi(x))-f(A)\le f(A)-f(A-x)=f_{A-x}(x), \forall x\in A$. Let $A = \{a_1,\ldots,a_k\}$, $A_i=\{a_1,a_2,\dots a_i\}$ and let the marginal improvement be $\delta_i = f_{A_{i-1}}(a_i)= f(A_i)-f(A_{i-1})$ at the point of addition of $a_i$ (for simplicity we make abuse of notation and have $\delta_{a_i}=\delta_i$). Summing up the inequalities for all $x \in A\setminus B$, we get a telescoping sum:
%\[
%f(A\cup B)-f(A)\le \sum_{i\in A\setminus B}\delta_i \iff f(A\cup B)\le f(A)+\sum_{i\in A\setminus B}\delta_i
%\]
%We can now use our assumption about 2-stability by perturbing (here $\gamma=2$) only the marginals of the elements $x \in A\setminus B$ (this favours only our local search solution). The new value for the local search solution $A$ is: $\tilde{f}(A)=f(A)+\sum_{i\in A\setminus B}\delta_i$ and thus we get:
%\[
%\tilde{f}(B)=f(B)\le f(A\cup B)\le f(A)+\sum_{i\in A\setminus B}\delta_i=\tilde{f}(A),
%\]
%which contradicts the 2-stability of the instance.
%\end{proof}

%Generally, for $p$ matroids we have the following theorem: (for the proof, we refer the reader to \hyperref[app:LSrecovery]{Appendix~\ref{app:LSrecovery}} of the full version of this paper):
\begin{theorem}\label{th:LSmatroids}
Given $(X,\I)$, with $\I=\cap_{i=1}^p \I_i$ where each $\I_i$ is a matroid and $f$ monotone submodular, such that the optimal solution is $(p+1)$-stable, Local Search exactly recovers it.
\end{theorem}


%\begin{proof}
%We denote with $A$ our local search solution (let it be maximal, even if new elements add zero value to it) and with $B$ the global optimum. Let $Y=B\sm A=\{y_1,y_2,\dots,y_k\}$ be the elements of the optimum that local search didn't choose. By the matching property~\cite{reichel2007evolutionary} of the matroids we get:
%\[
%\exists\ X^1,X^2,\dots,X^p \subseteq A\sm B, \text{where\ } X^j=\{x_1^j,x_2^j,\dots,x_k^j\}\  \text{such that:\ }
%\]
%\[
%\forall j\in\{1,\dots,p\}: x_i^j \in C_j(A,y_i), \forall i\in \{1,2,\dots,k\},
%\]
%where $C_j(A,y_i)$ is the circuit (minimally dependent set) created in matroid $\I_j$ when adding $y_i$ in $A$.

%Using the local search (with $(p,1)$ swaps) stopping condition, we have: (for ease, we use $+,-$ instead of the more accurate $\cup, \sm$)
%\[
%f(A+y_i-x_i^1-x_i^2-\dots-x_i^p)\le f(A), \forall i\in \{1,2,\dots,k\}
%\]
%(Note that in case $f$ is additive the above inequality just becomes: $f(y_i)\le f(x_i^1)+f(x_i^2)+\dots+f(x_i^p)$).
%Since $(A-\cup_{j=1}^p x^j_i)\subseteq (A+y_i-\cup_{j=1}^p x^j_i)$, using the submodularity for adding $\cup_{j=1}^p x^j_i$, we get:
%\[
%f(A+y_i)-f(A+y_i-\cup_{j=1}^p x^j_i)\le f(A)-f(A-\cup_{j=1}^p x^j_i)
%\]
%and adding $f(A+y_i)-f(A)$ to both sides and using submodularity and the local search stopping condition, we get:
%\[
%f(A+y_i)-f(A)-f(A)+f(A-\cup_{j=1}^p x^j_i)\le f(A+y_i-\cup_{j=1}^p x^j_i)-f(A)\le 0
%\]
%We conclude: $f(A+y_i)-f(A)\le f(A)-f(A-\cup_{j=1}^p x^j_i), \forall i\in \{1,2,\dots,k\}$ and adding these inequalities ($\delta(x_i^j)$ is the marginal gain by adding $x_i^j$ at the point of addition):
%\[
%f_A(B\sm A)\le \sum_{i=1}^k\sum_{j=1}^p\delta(x_i^j) = \sum_{j=1}^p\sum_{i=1}^k\delta(x_i^j)\le \sum_{j=1}^p f(X^j)\le \sum_{j=1}^p f(A\sm B)\le pf(A\sm B)
%\]
%Now we can $(p+1)$-perturb the marginals for elements of $A\sm B$:
%\[
%\tilde{f}(B)=f(B)\le f(A\cup B)\le f(A)+f_A(B\sm A)\le f(A)+pf(A\sm B)=\tilde{f}(A)
%\]
%which contradicts the ($p+1$)-stability of the instance. Once again, for the case of additive $f$: $f_A(B\sm A)=f(B\sm A)$ and thus $p$-stability is enough to guarantee recovery. ($\tilde{f}(B)=f(B)=f(B\sm A)+f(B\cap A)\le pf(A\sm B) +f(B\cap A)\le f(A)+(p-1)f(A\sm B)=\tilde{f}(A)$, where we $p$-perturbed the instance)
%\end{proof}










We would like to start by thanking
our sponsors: Stanford Computer Science Department and Stanford Program in AI-assisted Care
(PAC). Next, we specially thank De-An Huang, Kenji Hata, Serena Yeung, Ozan Sener and all the members of Stanford Vision and Learning Lab for their insightful discussion and feedback. Lastly, we thank all the anonymous reviewers for their valuable comments.

\bibliography{main}

\appendix



\section{Proof of \hyperref[th:submod]{Theorem} for welfare maximization}\label{app:submod}

\begin{theorem}
Let $(X,\I)$ be a matroid on the elements of $X$, let $B_1,B_2,\dots,B_k$ be a partition of $X$, $f_i: 2^{B_i}\to \RR^+\cup\{0\}$, for $i\in \{1,2,\dots,k\}$ be monotone submodular and let $f=\sum_{i=1}^kf_i$. If the optimal solution $S^*$ of $\max\{f(S): S\in \I\}$ is $2$-stable with respect only to individual perturbations of the functions $f_i$, greedy will recover $S^*$.
\end{theorem}

\begin{proof}
We note that the $B_i$'s form a partition of $X$ which is not tied to the matroid in any way. To avoid confusion, we should first emphasize the greedy algorithm in this case: It starts with the empty set $S_0=\emptyset$, at step $t$ it selects: $e=\arg\max_{x\in X}\{f(S_{t-1}+x)-f(S_{t-1})\}$ subject to the matroid constraint and it updates $S_t\leftarrow S_{t-1}+e$. This is a particular instantiation of the standard greedy algorithm in welfare maximization that first picks an item giving it to the player so that it yields maximum marginal improvement.

Suppose greedy outputs $S\neq S^*$ and that it chose elements $A_i\subseteq B_i$. Let $S_e$ be the greedy solution right before adding element $e$. Then a 2-perturbation of the individual functions is:
\[
\tilde{f_i}(A_i)=f_i(A_i)+\sum_{e\in (S\cap B_i)\sm S^*}f_i((B_i\cap S_e)+e)-f_i(B_i\cap S_e)
\]
Now coming back to the total welfare function $f$ we get:
\[
\tilde{f}(S)=\sum_{i=1}^k\tilde{f}_i(S\cap B_i)=f(S)+\sum_{e\in S\sm S^*}f_{S_e}(e)\ge f(S)+\sum_{e'\in S^*\sm S, e\leftrightarrow e'}f_{S_e}(e')\ge
\]
\[
\ge f(S)+f_S(S^*\sm S)\ge f(S^*\cup S)\ge f(S^*)=\tilde{f}(S^*)
\]
where we made use of the greedy criterion, submodularity and the matroid matching $e\leftrightarrow e'$ between elements $e\in S\sm S^*$ and $e' \in S^*\sm S$. We got $\tilde{f}(S)\ge \tilde{f}(S^*)$, hence a contradiction to the 2-stability of $S^*$ and hence $S\equiv S^*$ and greedy exactly recovers the optimal solution. 
\end{proof}
























%%% Local Variables:
%%% mode: latex
%%% TeX-master: "main"
%%% End:


\section{Hereditary Systems}\label{sec:hereditary}

Motivated by the ``bad'' example (see \hyperref[knapsack]{Proposition \ref{knapsack}}) for the greedy algorithm, we define a new notion of an independence system that we call \textit{hereditary} $p$-system or $p$-\textit{hereditary} that as we see later is a different characterization of $p$-extendible systems. In the aforementioned example, even though we started with a $p$-system, as we progressed picking elements with the greedy algorithm, the system became a $p'$-system with $p' \gg p$, thus leading to bad performance for the greedy, even though we had the optimal solution being stable by a large amount.

The intuition behind the following definition is that we want our system to remain a $p$-system under deletions and contractions of elements.

\begin{definition}[Hereditary $p$-system]
A $p$-system $(X,\I)$ is said to be \textit{hereditary} if:
\begin{enumerate}
\item For each set $Y\sse X$, the system $(X', \I | X')$\footnote{By $(X',\I | X')$, we mean the \textit{restriction} of $\I$ to the set of elements $X'$, which is the independence system on the set $X'$, whose independent sets are the independent sets of the initial set $\I$ that are contained in $X'$.}, where $X'=X \sm Y$, is a $p$-system. This corresponds to the {\bf deletion} of the elements in $Y$ from the system.
\item For each set $Y\sse X$,  the system $(X\sm Y, \I / Y)$\footnote{By $(X\sm Y,\I / Y)$, we mean the \textit{contraction} of $\I$ by $Y$, which is the independence system on the underlying set $X\sm Y$, whose independent sets are the sets $Z \sse X\sm Y$, such that $Z\cup Y \in \I$.} is a $p$-system. This corresponds to the {\bf contraction} of the elements in $Y$.
\end{enumerate}
\end{definition}

\noindent Looking back at our ``bad'' Knapsack example we see that it is not a hereditary system since initially $p=\tfrac{2M}{M+1}\le 2$, but after we had picked all the elements in set $A$, the system on the remaining elements became an $M$-extendible system. We now prove that the family of hereditary $p$-systems coincides with the family of $p$-extendible systems.  

\begin{proposition}
A $p$-system is $p$-hereditary if and only if it is $p$-extendible.
\end{proposition}

\begin{proof}
$p$-hereditary $\implies$ $p$-extendible: Let's first think of $p$ as an integer; as we will see afterwards only this case (with integer $p$) is interesting. Suppose we had a $p$-hereditary system that was not $p$-extendible. By negating the definition of $p$-extendibility (see \hyperref[sec:preliminaries]{Preliminaries}), it follows that there exist sets $A,B \sse X$ with $A \sse B$, $A,B \in \I$ and $A\cup\{e\} \in \I$ such that for all sets $Z\sse B\sm A$ with $|Z| \le p$: $(B\sm Z)\cup \{e\} \not\in \I$. Define $Z_0 \sse B\sm A$ to be the smallest set that we need to remove from $B$ in order to have: $(B\sm Z_0)\cup \{e\} \in \I$. We know that $|Z_0|>p$ and thus, by the hereditary property, if we project the independence system on the elements $Z_0\cup \{e\}$, we get $Z_0\cup \{e\} \not\in \I$ with the ratio $\tfrac{|Z_0|}{|\{e\}|}=\tfrac{|Z_0|}{1}>p$, which contradicts the fact that we started with a $p$-hereditary system.


For $p$-extendible $\implies$ $p$-hereditary: This direction follows easily just by the definition of $p$-extendibility. To handle non-integer values of $p$, we observe that by the first argument above, a $p$-hereditary system is actually $\lfloor p\rfloor$-extendible and thus, it is $\lfloor p\rfloor$-hereditary (e.g. a 2.9-hereditary system is 2-extendible).
\end{proof}

























\section{Proof of \hyperref[th:LSmatroids]{Theorem} for Intersection of Matroids and recovery}\label{app:LSrecovery}

\begin{theorem}
Given $(X,\I)$, with $\I=\cap_{i=1}^p \I_i$ where each $\I_i$ is a matroid and $f$ monotone submodular, such that the optimal solution is $(p+1)$-stable, Local Search exactly recovers it.
\end{theorem}
\begin{proof}
We denote with $A$ our local search solution (let it be maximal, even if new elements add zero value to it) and with $B$ the global optimum. Let $Y=B\sm A=\{y_1,y_2,\dots,y_k\}$ be the elements of the optimum that local search didn't choose. By the matching property~\cite{reichel2007evolutionary} of the matroids we get:
\[
\exists\ X^1,X^2,\dots,X^p \subseteq A\sm B, \text{where\ } X^j=\{x_1^j,x_2^j,\dots,x_k^j\}\  \text{such that:\ }
\]
\[
\forall j\in\{1,\dots,p\}: x_i^j \in C_j(A,y_i), \forall i\in \{1,2,\dots,k\},
\]
where $C_j(A,y_i)$ is the circuit (minimally dependent set) created in matroid $\I_j$ when adding $y_i$ in $A$.

Using the local search (with $(p,1)$ swaps) stopping condition, we have: (for ease, we use $+,-$ instead of the more accurate $\cup, \sm$)
\[
f(A+y_i-x_i^1-x_i^2-\dots-x_i^p)\le f(A), \forall i\in \{1,2,\dots,k\}
\]
(Note that in case $f$ is additive the above inequality just becomes: $f(y_i)\le f(x_i^1)+f(x_i^2)+\dots+f(x_i^p)$).
Since $(A-\cup_{j=1}^p x^j_i)\subseteq (A+y_i-\cup_{j=1}^p x^j_i)$, using the submodularity for adding $\cup_{j=1}^p x^j_i$, we get:
\[
f(A+y_i)-f(A+y_i-\cup_{j=1}^p x^j_i)\le f(A)-f(A-\cup_{j=1}^p x^j_i)
\]
and adding $f(A+y_i)-f(A)$ to both sides and using submodularity and the local search stopping condition, we get:
\[
f(A+y_i)-f(A)-f(A)+f(A-\cup_{j=1}^p x^j_i)\le f(A+y_i-\cup_{j=1}^p x^j_i)-f(A)\le 0
\]
We conclude: $f(A+y_i)-f(A)\le f(A)-f(A-\cup_{j=1}^p x^j_i), \forall i\in \{1,2,\dots,k\}$ and adding these inequalities ($\delta(x_i^j)$ is the marginal gain by adding $x_i^j$ at the point of addition):
\[
f_A(B\sm A)\le \sum_{i=1}^k\sum_{j=1}^p\delta(x_i^j) = \sum_{j=1}^p\sum_{i=1}^k\delta(x_i^j)\le \sum_{j=1}^p f(X^j)\le \sum_{j=1}^p f(A\sm B)\le pf(A\sm B)
\]
Now we can $(p+1)$-perturb the marginals for elements of $A\sm B$:
\[
\tilde{f}(B)=f(B)\le f(A\cup B)\le f(A)+f_A(B\sm A)\le f(A)+pf(A\sm B)=\tilde{f}(A)
\]
which contradicts the ($p+1$)-stability of the instance. Once again, for the case of additive $f$: $f_A(B\sm A)=f(B\sm A)$ and thus $p$-stability is enough to guarantee recovery. ($\tilde{f}(B)=f(B)=f(B\sm A)+f(B\cap A)\le pf(A\sm B) +f(B\cap A)\le f(A)+(p-1)f(A\sm B)=\tilde{f}(A)$, where we $p$-perturbed the instance)
\end{proof}




















%%% Local Variables:
%%% mode: latex
%%% TeX-master: "main"
%%% End:

\section{Counterexamples}\label{sec:counter}

Here are two simple counterexamples that prove the tightness of our Greedy recovery results and our Local Search approximation and recovery results:

\begin{itemize}
\item In the submodular case, we proved greedy recovers $(p+1)$-stable $p$-extendible systems. Here is a simple example of a matroid (1-extendible) where Greedy and Local Search fail to recover the optimal solution even though it is 2-stable (also notice that here, Greedy and Local Search give a 2-approximation):  Take $A_1=\{x,\epsilon_1\}, B_1=\{y\}, A_2=\{\epsilon_2\}, B_2=\{x\}$ as in~\cite{filmus2012power}. Assign $w(x)=w(y)=1$ and $w(\epsilon_1)=\epsilon, w(\epsilon_2)=\epsilon$ for some small $\epsilon>0$ (and so $w(A_1)=1+\epsilon$) and consider the partition matroid whose independent sets can only contain one of $\{A_i,B_i\}, i=1,2$. Observe that $\{A_1,A_2\}$ is a local optimum with value $1+2\epsilon$, whereas the global optimum is $\{B_1,B_2\}$ with value 2. Also notice that the same solution is produced by the Greedy algorithm and that the instance can be $(2-\epsilon')$-stable for any small $\epsilon'>0$.

\item Local Search is a $p^2$-approximation for $p$-extendible systems. Look at \hyperref[counter1]{Figure~\ref{counter1}} for a tight counterexample (just for simplicity, we have the $p=2$ case; it generalizes readily).
\end{itemize} 


\begin{figure}[h!]
	\centering
	\includegraphics[width=5cm,height=4cm]{IMAGE1}
        	\caption{Local Search is a 4-approximation for this 2-extendible system $(X,\I)$: Let $A=\{a_1,a_2\}$ be feasible and assign $w(a_1)=w(a_2)=1+\epsilon$ and $w(b_i)=2, \forall i\in\{1,2,3,4\}$. The constraints are: $a_1\cup B_1 \notin \I, a_2\cup B_2 \notin \I $, $a_i\cup B_j \in \I$ for $i\neq j$ and $A\cup b_i\notin \I,\forall i\in\{1,2,3,4\}$. Observe that $A$ is a local optimum ($(2,1)$-swaps) with value $2+2\epsilon$, whereas $B_1\cup B_2$ is the global optimum with value 8. Notice also that for the appropriate choice of $\epsilon$, this can be a $(4-\epsilon')$-stable instance for any small $\epsilon'$.}
	\label{counter1}
\end{figure}

\subsection{Cardinality Constraints}\label{sec:cardinality-counter}

Another interesting separation between approximation and stability happens for the case of cardinality constraints. A special case of submodular maximization on $p$-extendible systems is when we have a uniform matroid constraint where the only feasible solutions are those that have cardinality $k \ge 1$ ($\I=\{S\subseteq X: |S|\le k\}$). For this special case, recall that greedy is a $(1-\tfrac{1}{e})$-approximation (in fact, $1-(1-\tfrac{1}{k})^k$) and that this is tight \cite{feige1998threshold}. Regarding stability, we show that the stability threshold needed by greedy for recovery is at least $2-\tfrac{1}{k}$ and so $(1-\tfrac{1}{e})^{-1}$-stability is not enough, i.e. here the approximation threshold is strictly smaller than the stability threshold needed for recovery (see also \hyperref[fig:cardinality]{Figure~\ref{fig:cardinality}}).

\begin{proposition}
For submodular maximization under a uniform matroid ($\I=\{S: |S|\le k\},k\ge 1$), greedy cannot recover $\gamma$-stable instances if $\gamma<(2-\tfrac{1}{k})$.
\end{proposition}

\begin{proof}
The $(2-\tfrac{1}{k}-\delta)$-stable counterexample (for any small $\delta$) where greedy fails is the following: We have in total ($k+1$) elements: $x_1,x_2,\dots, x_k$ and a special element $e$. Denote $O=\{x_1,x_2,\dots, x_k\}$ and with $O_i$ any subset of $O$ with exactly $i$ elements. The function $f$ has: $f(O)=1, f_{O_i}(x_j)=\tfrac{1}{k}, \forall x_j\in O\setminus O_i,  f_{\{e\}\cup O_i}(x_j)=\tfrac{1}{k}(1-\tfrac{1}{k}), \forall x_j\in O\setminus O_i $ and $f(e)=\tfrac{1}{k}, f_{O_i}(e)=\tfrac{1}{k}-\tfrac{i}{k^2}$. Then Greedy first picks element $e$ (to break ties we could set $f(e)=\tfrac{1}{k}+\epsilon$) and then $k-1$ other elements $O_{k-1}\subseteq O$ (let $S=\{e\}\cup O_{k-1}$). However, the optimal solution is $O$ with $f(O)=1$ and greedy has value $1-(\tfrac{1}{k}-\tfrac{1}{k^2})$. Since $S\sm O=\{e\}$, any perturbation such that $\tilde{f}(S)\ge \tilde{f}(O)$ could only $\gamma$-perturb the value $f(e)$: $\tilde{f}(S)\ge \tilde{f}(O)\iff (\gamma-1)\tfrac{1}{k}\ge\tfrac{1}{k}-\tfrac{1}{k^2} \iff \gamma\ge (2-\tfrac{1}{k})$.
\end{proof}
\begin{figure}[h!]
	\centering
	\includegraphics[width=13cm,height=4cm]{IMAGE2}
        	\caption{This is the case for $k=2$ and $k=3$ (the area corresponds to marginal improvements). For $k=2$, there are three elements: $\{e,x_1,x_2\}$. $f(\{x_1,x_2\})=1$, so the optimal solution is $O=\{x_1,x_2\}$. We trick the greedy algorithm which first chooses $\{e\}$ that has slightly better marginal value. For exact recovery, a $\tfrac{3}{2}$-perturbation is needed, even though Greedy is a $\left(\tfrac{4}{3}\right)^{-1}$-approximation. Similarly, for $k=3$, the optimum is $O=\{x_1,x_2,x_3\}$, whereas Greedy picks $\{e,x_1,x_2\}$ and needs $\tfrac{5}{3}$-stability for recovery, even though it is $\left(\tfrac{19}{27}\right)^{-1}$-approximation. Note that stability thresholds need to be larger than the approximation factors.}
	\label{fig:cardinality}
\end{figure}




 


























\end{document}

%%% Local Variables:
%%% mode: latex
%%% TeX-master: t
%%% End:
