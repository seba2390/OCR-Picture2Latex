\section{Statistical model}
\label{sec:model}
Stars and galaxies radiate photons. An astronomical image records photons---each originating from a particular
celestial body or from background atmospheric and detector noise---that pass through a telescope's lens during an exposure.
A single image contains photons from many light sources; even a single pixel may capture photons from multiple sources.

Section~\ref{light-sources} describes our model of light sources. Quantities of interest, such as direction\footnote{A direction is a position on the celestial sphere. We use the term ``direction'', not ``location'', because the distance to a light source, unlike its direction, is not directly observable.}, color, and flux, are random variables. Section~\ref{images} describes a generative model of astronomical images: the distribution of each pixel's intensity---an observed random variable---depends on the latent variables that we aim to infer. Pixel intensities are conditionally independent given these latent random variables.
Figure~\ref{graphical_model} presents our statistical model as a graphical model.

Table~\ref{structural-table} lists the model's structural constants,
denoted by capital Roman letters.
All are positive integers.
None are estimated.

Table~\ref{rv-table} lists the model's random variables for a light source $s \in \{1,\ldots,S\}$,
an image $n \in \{1,\ldots, N\}$,
and a pixel $m \in \{1,\ldots,M\}$.
(S, N, and M appear in Table~\ref{structural-table}.)
All are denoted by lowercase Roman letters.
All are scalars except for the color vector $c_s$ and the direction vector $u_s$.
Inferring the posterior distribution of the unobserved random variables in Table~\ref{rv-table} is the primary problem addressed by this article.

Table~\ref{params-table} lists model parameters.
The first eight rows describe hyperparameters; they parameterize the prior and are distinguished by calligraphic font.
They are estimated a priori by maximum likelihood, as described in Section~\ref{prior}.
The remaining parameters, denoted by lowercase Greek letters, are set by the SDSS pipeline.

\subsection{Light sources}
\label{light-sources}

An astronomical catalog is a table with one row for each light source.
The number of light sources, $S$, is treated as a known constant here; we determine it by running existing cataloging software~\citep{bertin1996sextractor}.
Modeling $S$ as random we defer to future work.

Light sources in our model are either stars or galaxies, as are the vast majority of light sources in the universe.
Exceptions include asteroids, planets, airplanes, and man-made satellites, which also occasionally appear in astronomical images.
For light source $s=1,\ldots,S$, the latent random variable
\begin{align}
a_s &\sim \mathrm{Bernoulli}(\mathcal A)
\end{align}
indicates whether it is a star (${a_s=1}$) or a galaxy (${a_s=0}$). Here $\mathcal{A}$ is the prior probability that a light source is a star. (We discuss how we set $\mathcal A$, and all other prior hyperparameters, in Section~\ref{prior}.)


\FloatBarrier
\begin{table}
\centering
\caption{Structural constants in our model.}
\label{params}
\centering
\begin{tabular}{lll}
\toprule
\textbf{name} & \textbf{brief description}  & \textbf{SDSS value} \\
\hline
B & number of filter bands & 5\\
E & number of PSF ``eigenimages'' & 4\\
F & number of knots per PSF eigenimage & $51 \times 51$\\
H & number of rows of pixels per image & 2048 \\
I & number of source types (i.e., star, galaxy) & 2\\
J & number of components in the color prior mixture & 8\\
K & number of components in the galaxy mixture model & 8\\
L & number of parameters in a WCS header & 16\\
M & number of pixels per image & $H \times W$\\
N & number of images & 4,690,230\\
Q & number of knots for the sky background model & $192 \times 256$\\
S & number of light sources & 469,053,874\\
W & number of columns of pixels per image & 1361 \\
\hline
\end{tabular}
\label{structural-table}
\end{table}

\begin{table}
\centering
\caption{Random variables in our model.}
\centering
\begin{tabular}{llll}
\hline
\textbf{name} & \textbf{brief description} & \textbf{units} & \textbf{domain} \\
\hline
$a_s$           & galaxy / star indicator & unitless & \{0, 1\} \\
$c_s$           & colors & magnitude & $\mathbb R^{B-1}$ \\
$e_s^{angle}$   & angle of galaxy's major axis & degrees & $[0, 180)$ \\
$e_s^{radius}$  & galaxy's half-light radius & arcseconds & $(0, \infty)$ \\
$e_s^{profile}$ & galaxy's profile mixing weight & unitless & $[0, 1]$ \\
$e_s^{axis}$    & galaxy's minor-major axis ratio & unitless & $(0, 1)$ \\
$r_s$           & reference-band flux density & nanomaggies & $[0, \infty)$ \\
$u_s$           & direction (longitude, latitude) & degrees & $[0, 360) \times [-90, 90]$ \\
$x_{nm}$        & pixel intensity (observed) & photon count & $\{0,1,2,\ldots\}$ \\
\hline
\end{tabular}
\label{rv-table}
\end{table}

\begin{table}
\centering
\caption{Parameters in our model.}
\centering
\begin{tabular}{lll}
\hline
\textbf{name} & \textbf{brief description}  & \textbf{domain} \\
\hline
$\mathcal A$           & prior probability a light source is a star & $[0, 1]$\\
$\mathcal C^{weight}$  & color prior mixture weights &  $\mathbb R^{I \times J}$\\
$\mathcal C^{mean}$    & color prior mixture component means &  $\mathbb R^{I \times J \times (B - 1)}$\\
$\mathcal C^{cov}$     & color prior mixture component covariance matrices &  $\mathbb R^{I \times J \times (B - 1)  \times (B - 1)}$\\
$\mathcal E^{radius}$  & galaxy half-light radius prior parameters  & $\mathbb R^2$\\
$\mathcal E^{profile}$ & galaxy profile prior parameters & $\mathbb R^2$\\
$\mathcal E^{axis}$    & galaxy axis ratio prior parameters & $\mathbb R^2$\\
$\mathcal R$           & reference-band flux prior parameters & $\mathbb R^{I \times 2}$\\
\hline
$\sigma_{n}$           & sky background model & $\mathbb R^Q$\\
$\psi_n^{calib}$       & expected number of photons per nanomaggy & $\mathbb R^H$\\
$\psi_n^{wcs}$         & image alignment & $\mathbb R^L$\\
$\psi_n^{weight}$      & point spread function loadings & $\mathbb R^E$\\
$\psi_n^{image}$       & point spread function principal components & $\mathbb R^{E \times F}$\\
$\beta_n $             & filter band & $\{1,2,\ldots,B\}$\\
\hline
\end{tabular}
\label{params-table}
\end{table}
\FloatBarrier

\begin{figure}[ht!]
	\includegraphics[width=1.8in]{figures/celeste_graphical_model.png}
	\caption{The proposed graphical model. The shaded vertex represents observed
		random variables. Empty vertices represent latent random variables.
		Black dots represent constants, set before inference takes place.
		Edges signify conditional dependencies. Rectangles
		(``plates'') represent independent replication.
		Tables~\ref{structural-table},~\ref{rv-table}, and~\ref{params-table} summarize the variables.
	}
	\label{graphical_model}
\end{figure}

The latent random two-vector $u_s$ denotes the direction of light source $s$
in the units of J2000.0 equatorial coordinates, a latitude and longitude system relative to the Earth's equator.
Figure~\ref{sdss_coverage} illustrates this system of coordinates.
The first coordinate is longitude and the second coordinate is latitude.
Both are measured in degrees.

A priori, $u_s$ is uniformly distributed over the sphere.
Treating light sources as uniformly distributed is a simplification---some regions of the sky are known a priori to have more light sources than others, e.g., the galactic plane. This is known as directional dependence. Additionally, it is a simplification to model light sources as positioned independently of each other; gravity causes some clustering among light sources.


\subsubsection{Flux}
\label{flux}

The flux of light source $s$ is defined as its expected total radiation reaching a unit area of Earth's surface directly facing $s$, per unit of time. We can measure the flux as the portion of this radiation (per square meter per second) that passes through each filter in a standardized filter set. Such a set is called a filter system. These standardized filters are approximately band-pass: each allows most of the energy in a certain band of wavelengths through, while blocking most of the energy outside the band.
The physical filters attached to a telescope lens closely match the standardized filters of some filter systems.

The five SDSS filters are named for the wavelengths they are most likely to let pass:
ultraviolet~($u'$), green~($g'$), red~($r'$), near infrared~($i'$), and infrared~($z'$).
Figure~\ref{sdss-filter-curves} shows how likely a photon of a particular wavelength is to pass through each filter.
\cite{fukugita1996sloan} further describe the SDSS filter system.

We model flux with respect to the $B = 5$ filters of the SDSS filter system.
We designate a particular filter as the ``reference'' filter, letting the
random variable $r_s$ denote the flux of object $s$ with respect to that
filter. A priori,
\begin{align}
r_s | (a_s = i) &\sim \mathrm{LogNormal}(\mathcal R_{i1}, \mathcal R_{i2}),\,\,\, i \in \{0,1\}.
\end{align}
Our prior depends on $a_s$ to reflect that stars tend to have higher flux density than galaxies.
The flux density $r_s$ is measured in nanomaggies~\citep{sdssglossary,nanomaggies}.
One nanomaggy is equivalent to $3.631 \times 10^{-6}$ Jansky.
A nanomaggy is a linear unit; we expect to receive twice as many photons from a two-nanomaggy light source as from a one-nanomaggy light source.

The log-normal distribution reflects that flux is non-negative and that stars' fluxes often differ by orders of magnitude. Empirically,
a log-normal distribution fits the SDSS catalog better than any gamma distribution---another common model for non-negative real-valued variables.
In future work, we may also explore a power law distribution for galaxy fluxes, as there is some theoretical support for that model.

The flux of light source $s$ with respect to the remaining $B-1$ filters is encoded using colors.
The color $c_{s\beta}$ is defined as the log ratio of
fluxes with respect to filters $\beta$ and $\beta + 1$. Here, the filters are
ordered by the wavelength bands they let pass. The $B-1$ colors for
object $s$ are collectively denoted by $c_s$, a random $(B-1)$-vector.
We denote the colors as \textit{u-g, g-r, r-i}, and \textit{i-z}.
The reference-filter flux $r_s$ and the colors $c_s$ uniquely specify the flux for light source $s$ through any filter $\beta$, denoted $\ell_{s\beta}$.

Our model uses the color parameterization because stars and galaxies have very
distinct prior distributions in color space. Indeed, for idealized
stars---blackbodies---all $B - 1$ colors lie on a one-dimensional manifold
indexed by surface temperature. On the other hand, though galaxies are
composed of stars, theory does not suggest they lie near the same
manifold: the stars in a galaxy can have many different surface
temperatures, and some of the photons are re-processed to other energies
through interactions with dust and gas.
Figure~\ref{color_priors} demonstrates that stars are much closer to a
one-dimensional manifold in color space than galaxies are.

We model the prior distribution on $c_s$ as a $D$-component Gaussian mixture model (GMM):
\begin{align}
c_s | (a_s = i) &\sim \mathrm{GMM}(\mathcal C_{i}^{weight}, C_{i}^{mean}, \mathcal C_{i}^{cov}),\,\,\, i \in \{0,1\}.
\end{align}
We discuss how we set $D$ and the color priors' hyperparameters in Section~\ref{prior}.

\begin{figure}
  \includegraphics[width=3.5in]{figures/color_priors}
  \caption{Density plots for two colors, \textit{g-r} and \textit{r-i}, based on the SDSS DR10 catalog.}
  \label{color_priors}
\end{figure}


\subsubsection{Spatial extent}

\begin{figure}
\begin{floatrow}
\ffigbox{%
  \includegraphics[width=1.8in]{ngc3610}
}{%
  \caption{A schematic of the galaxy light kernel. The blue ellipse surrounds half of the light emissions of this galaxy.
  The length of the major axis is the half-light radius $e_s^{radius}$. The angle in degrees of the major axis is~${e_s^{angle} = 45}$. The ratio of the lengths of minor and major axes is~${e_s^{axis} = 1/2}$. Because this galaxy is purely elliptical,~${e_s^{profile} = 0}$.}%
  \label{cartoon-galaxy}
}
\ffigbox{%
  \includegraphics[width=1.6in]{figures/faint_galaxy.png}
}{%
  \caption{A distant galaxy approximately 20 pixels
  in height, estimated to
  have half-light radius~${e_s^{radius}=0.6}$ arcseconds, rotation angle~${e_s^{angle}=80}$ degrees, and minor-major axis ratio~${e_s^{axis}=0.17}$.}
  \label{fig:faint_galaxy}
}
\end{floatrow}
\end{figure}


\begin{figure}
\begin{subfigure}{0.45\textwidth}
\includegraphics[height=1.8in]{figures/messier_87.jpg}
\subcaption{Messier 87, a galaxy that exhibits the de Vaucouleurs profile. Credit: NASA}
\label{fig:dev_galaxy}
\end{subfigure}
\hfill
\begin{subfigure}{0.45\textwidth}
\includegraphics[height=1.8in]{figures/triangulum.jpg}
\subcaption{Triangulum, a galaxy that exhibits the exponential profile. Credit: NASA}
\label{fig:exp_galaxy}
\end{subfigure}
\caption{Extremal galaxy profiles.}
\label{galaxy-profiles}
\end{figure}

Consider a light source~$s$, centered at some direction~$u_s$.
Its flux density in filter band~$\beta$, measured at a possibly different direction~$\mu$,
is given by
\begin{align}
\varphi_{s\beta}(\mu) \coloneqq h_s(\mu) \ell_{s\beta}.
\end{align}
Here $h_s$ (a density) models the spatial characteristics of light source $s$, quantifying its relative intensity at each direction $\mu$ specified in sky coordinates (not image-specific ``pixel coordinates''). We refer to $h_s$ as the ``light kernel'' for light source $s$.

The distance from Earth to any star other than the Sun exceeds the star's radius by many orders of magnitude. Therefore, we model stars as point sources. If light source $s$ is a star (i.e., $a_s = 1$), then $h_s$ is simply a delta function: one if $\mu = u_s$, zero otherwise.

Modeling the two-dimensional appearance of galaxies as seen from Earth is more involved. If light source $s$ is a galaxy (i.e., ${a_s = 0}$), then $h_s$ is parameterized by a latent random 4-vector
\begin{align}
e_s \coloneqq (e_s^{profile}, e_s^{angle}, e_s^{radius}, e_s^{axis}).
\end{align}
We take $h_s$ to be a convex combination of two extremal profiles,
known in astronomy as ``de Vaucouleurs'' and ``exponential''
profiles:
\begin{align}
h_{s}(\mu) &= e_s^{profile} h_{s1}(\mu) + (1 - e_s^{profile}) h_{s2}(\mu).
\end{align}
The de Vaucouleurs profile is characteristic of elliptical galaxies, whose luminosities vary gradually in space (Figure~\ref{fig:dev_galaxy}), whereas the exponential profile matches spiral galaxies (Figure~\ref{fig:exp_galaxy})~\citep{feigelson2012modern}.
The profile functions $h_{s1}(\mu)$ and $h_{s2}(\mu)$ also account for additional galaxy-specific parameters illustrated in Figure~\ref{cartoon-galaxy}. In particular,
each profile function is a rotated, scaled mixture of bivariate normal distributions. Rotation angle and scale are galaxy-specific, while the remaining parameters of each mixture are not:
\begin{align}
  h_{si}(\mu)
  & = \sum_{j=1}^{J} \alpha_{ij} \phi(\mu; u_{s}, \tau_{ij} \Sigma_{s}),\,\,\, i \in \{0,1\}.\label{eq:hsi}
\end{align}
Here the $\alpha_{ij}$ and the $\tau_{ij}$ are prespecified constants that characterize the exponential and de Vaucouleurs profiles; $u_s$ is the center of the galaxy in sky coordinates; $\Sigma_s$ is a 2$\times$2-covariance matrix shared across the components; and $\phi$ is the bivariate normal density.

The light kernel $h_s(\mu)$ is a finite scale mixture of Gaussians: its mixture
components have a common mean $u_s$; the isophotes (level sets of
$h_s(\mu)$) are concentric ellipses. Although this model
prevents us from fitting individual ``arms,'' like those of the galaxy in
Figure~\ref{fig:exp_galaxy}, most galaxies
are not sufficiently resolved to see such substructures.
Figure~\ref{fig:faint_galaxy} shows a more typical galaxy image.

The spatial covariance matrix~$\Sigma_s$ is parameterized by a rotation angle~$e_s^{angle}$, an eccentricity (minor-major axis ratio)~$e_{s}^{axis}$, and an
overall size scale~$e_{s}^{radius}$:
\begin{align}
\Sigma_s \coloneqq R_s^{\top}\begin{bmatrix} [ e_{s}^{radius}]^{2} & 0\\
                            0 & [e_{s}^{axis}]^2[e_{s}^{radius}]^{2}
\end{bmatrix} R_s,
\end{align}
where the rotation matrix is given by
\begin{align}
R_s \coloneqq \begin{bmatrix}\cos e_s^{angle} & -\sin e_s^{angle}\\
                           \sin e_s^{angle} & \cos e_s^{angle}
\end{bmatrix}.
\end{align}
The scale $e_{s}^{radius}$ is specified in terms of half-light radius---the
radius of the disc that contains half of the galaxy's light emissions
before applying the eccentricity $e_s^{angle}$.

All four entries of $e_s$ are random.
%Prior parameters $\mathcal E$ are set based on pre-existing catalogs.
The mixing weight prior is given by
\begin{align}
e_s^{profile} &\sim \mathrm{Beta}(\mathcal E^{profile}_1, \mathcal E^{profile}_2).
\end{align}
Every angle is equally likely, and galaxies are symmetric, so
\begin{align}
e_s^{angle} &\sim \mathrm{Uniform}([0, 180]).
\end{align}
We found that the following half-light-radius distribution fit well empirically:
\begin{align}
e_s^{radius} &\sim \mathrm{LogNormal}(\mathcal E^{radius}_1, \mathcal E^{radius}_2).
\end{align}
The ``fatter'' tail of a log-normal distribution fits better than a gamma distribution, for example.
A priori, the minor-major axis ratio is beta distributed:
\begin{align}
e_s^{axis} &\sim \mathrm{Beta}(\mathcal E^{axis}_1, \mathcal E^{axis}_2).
\end{align}


\subsubsection{Setting the priors' parameters}
\label{prior}
The light source prior is parameterized by 1099 real-valued scalars.
All but ten are for the GMM color prior.
Empirical Bayes is an appealing way to fit this prior because the number of parameters is small relative to the number of light sources (hundreds of millions for SDSS).

Unfortunately, re-fitting the prior parameters iteratively during inference---a common way of performing empirical Bayes---is difficult in a distributed setting: fitting the global prior parameters during inference couples together numerical optimization for disparate regions of sky.
Instead, we fit the prior parameters based on existing SDSS catalogs through maximum likelihood estimation.
Because these prior parameters are fit to a catalog based on the same data we subsequently analyze, our procedure is in the spirit of empirical Bayes.
However, using maximum likelihood in this way to assign priors ignores measurement error (and classification error) and therefore will produce priors that are  overdispersed. It produces estimates that are formally inconsistent, unlike conventional empirical Bayes approaches that iteratively refit the prior.

If the depth of our catalog were much greater than existing SDSS catalogs, we too might refit these
prior parameters periodically while performing inference.
Refitting in this way could be interpreted as a block coordinate ascent scheme.
However, in our work to date, the depth of our catalog is limited by the peak-finding preprocessing routine, just as in SDSS.
Therefore, for simplicity, we hold these prior parameters fixed during inference.

Fitting the color prior warrants some additional discussion.
First, maximum-likelihood estimation for a GMM is nonconvex, so the optimization path may matter: we use the GaussianMixture.jl software~\citep{gmm}.
Second, we set the number of GMM components $D$ based on computational considerations.
In principle, $D$ could be set with a statistical model-selection criterion.
In practice, we set $D=8$ without any apparent accuracy reduction for the point estimates, which is the primary way we assess our model in Section~\ref{sec:experiments}.
Because we have so much data (millions of light sources), there is no risk of overfitting with $D=8$: held-out log-likelihood improves as $D$ increases up to $D=256$, the largest setting our hardware allowed us to test.
There is also little risk that $D=8$ underfits: setting $D=16$ does not substantively change our estimates.

Empirical Bayes seems broadly applicable to sky-survey data; the number of light sources in typical surveys is large relative to the number of hyperparameters.
But the details of our procedure (e.g., how to set $D$, or whether to update the hyperparameters iteratively during inference) may need to be tailored based on the research goals. If so, our fitted priors may be considered ``interim'' priors.


\subsection{Images}
\label{images}

\begin{figure}
\begin{floatrow}
\ffigbox{%
\includegraphics[width=.6\linewidth]{"figures/sdss_camera"}
}{%
\caption{The SDSS camera. Its CCDs---each $2048 \times 2048$ pixels---are arranged in six columns and five rows. A different filter covers each row. Credit:~\cite{sdsscamera}.}
\label{sdss-camera}
}
\ffigbox{%
\includegraphics[width=\linewidth]{"figures/sdss_filter_curves"}
}{%
\caption{SDSS filter curves. Filter response is the probability that a photon of a particular wavelength will pass through the filter. Credit:~\cite{doi2010photometric}.}
\label{sdss-filter-curves}
}
\end{floatrow}
\end{figure}


\begin{figure}[t]
\includegraphics[width=3in]{stripemap}
\caption{SDSS sky coverage map. Each monocolor arc represents the sky
photographed during a particular night. Axes' units are degrees of right ascension (longitude) and declination (latitude). Credit:~\cite{sdsscoverage}.}
\label{sdss_coverage}
\end{figure}

Astronomical images are taken through telescopes.
Photons that enter the telescope reach a camera that records the pixel each photon hits,
thus contributing an electron.
The SDSS camera (Figure~\ref{sdss-camera}) consists of 30 charge-coupled devices (CCDs) arranged in a grid of six columns and five rows.
Each row is covered by a different filter---transparent colored glass that limits which photons can pass through and potentially be recorded.
Each of the five filters selects, stochastically, for photons of different wavelengths (Figure~\ref{sdss-filter-curves}).
Multiple images of the same region of the sky with different filters reveal the colors of stars and galaxies.

The SDSS telescope collects images by drift scanning, an imaging regime where the camera reads the CCDs continuously as the photons arrive.
Each night the telescope images a contiguous ``arc'' of sky (Figure~\ref{sdss_coverage}).

Each arc is divided into multiple image files.
SDSS Data Release 13 contains $N=$ 4,690,230 of these images, each taken through one of the 30 CCDs.
For $n=1,\ldots,N$, the constant $\beta_n$ denotes the filter color for image $n$.

Each image is a grid of $M=2048 \times 1361$ pixels.
The random variable $x_{nm}$ denotes the count of photons that, during the exposure for image $n$, entered the telescope, passed through the filter, and were recorded by pixel $m$.


\subsubsection{Skyglow}

The night sky is not completely dark, even in directions without resolvable light sources. This is due to both artificial light production (e.g., light pollution from cities) and natural phenomena.
The background flux is called ``skyglow.''
Sources of natural skyglow include sunlight reflected off dust particles in the solar system, nebulosity (i.e., glowing gas---a constituent of the interstellar medium), extragalactic background light from distant unresolved galaxies, night airglow from molecules in Earth's atmosphere, and scattered starlight and moonlight.
The flux from skyglow (``sky intensity'') varies by the time of the exposure, due to changing atmospheric conditions.
It also varies with direction; for example, sky intensity is typically greater near the galactic plane.
We model skyglow as a spatial Poisson process whose rate varies gradually by pixel, independent of stars and galaxies.
For the vast majority of pixels, the skyglow is the only source of photons.

Sky intensity is estimated during preprocessing by pre-existing software \citep{bertin1996sextractor} and fixed during inference.
This software fits a smooth parametric model to the intensities of the pixels that it determines are not near any light source.
The sky intensity could, in principle, be fit within our inference procedure; we defer this idea to future work.

The sky intensity for image $n$ is stored as a grid of $Q$ intensities in the matrix $\sigma_n$.
Typically $Q \ll M$ because the sky intensity varies slowly.
To form the sky intensity for a particular pixel, $\sigma_n$ is interpolated linearly.
We denote the sky intensity for a particular pixel by $\sigma_{n}(m)$.


\subsubsection{Point-spread functions}

Astronomical images are blurred by a combination of small-angle scattering in Earth's atmosphere, the diffraction limit of the telescope, optical distortions in the camera, and charge diffusion within the silicon of the CCD detectors. Together these effects are represented by the ``point-spread function'' (PSF) of a given image. Stars are essentially point sources, but the PSF represents how their photons are spread over dozens of adjacent pixels.

The PSF is set during preprocessing by pre-existing software~\citep{lupton2001sdss}. This software fits the PSF based on several stars with extremely high flux in each image whose characteristics are well established by previous studies using different instrumentation (e.g., spectrographs).
As with sky intensity, we could fit the PSF jointly with light sources through our approximate inference procedure, but we do not pursue this idea here.

The PSF is specified through several image-specific parameters that are bundled together in $\psi_n$.
The vector $\psi_n^{calib}$ gives the expected number of photons per nanomaggy for each column of image $n$.
The vector $\psi_n^{wcs}$ specifies a mapping from sky direction to pixel coordinates. This mapping is linear---an approximation that holds up well locally.
The rows of the matrix~$\psi_n^{image}$ give the top principal components from an eigendecomposition of the PSF as rendered on a grid centered at a light source.
The vector~$\psi_n^{weight}$ gives the loading of the PSF at any point in the image. It has smooth spatial variation.

Consider a one-nanomaggy star having direction $\mu$.
We denote its expected contribution of photons to the $m$th pixel of image~$n$ as~$g_{nm}(\mu)$; this is derived as needed from the explicitly represented quantities discussed above.


\subsubsection{The likelihood}

Let $z_s \coloneqq (a_s, r_s, c_s, e_s, u_s)$ denote the latent random variables for light source $s$.
Let $z \coloneqq \{z_s\}_{s=1}^S$ denote all the latent random variables.
Then, for the number of photons received by pixel $m$ of image $n$, we take the likelihood to be
\begin{align}
x_{nm} | z \sim \mathrm{Poisson}(\lambda_{nm}).
\label{xnm}
\end{align}

The dependence of $\lambda_{nm}$ on $z$ is not notated here.
We model $x_{nm}$ as observed, though the reality is more complicated~\citep{frame}:
at the end of an exposure, the CCD readout process transfers the electrons to a small capacitor, converting the (discrete) charge to a voltage that is amplified and forms the output of the CCD chip.
The net voltage is measured and digitized by an analog-to-digital converter (ADC).
The conversion is characterized by a conversion gain.
The ADC output is an integer called a digital number (DN).
The conversion gain is specified in terms of electrons per DN.
While in our model $x_{nm}$ is the number of photons received, in practice we set $x_{nm}$ to a value determined by scaling DN according to gain and rounding it to the nearest integer.
The Poisson mass function is fairly constant across the quantization range.
\footnote{An alternate perspective is that $x_{nm}$ is not approximated in practice: rather, $x_{nm}$ is our approximation.
We do not take this perspective so that we can explain our model at a high level before introducing low-level details of CCD technology.
}

Because of these complexities, it is not clear whether a Poisson distribution is more suitable here than a Gaussian distribution with its mean equal to its variance.
We make no claims about the superiority of one or the other.
In the SDSS, the sky background is typically at least 500 electrons per pixel, so it seems unlikely that the choice of a Gaussian (with its mean equal to its variance) or a Poisson distribution would matter.
Furthermore, neither likelihood simplifies the subsequent inferential calculations.

In Equation~\ref{xnm}, the rate parameter $\lambda_{nm}$ is unique to pixel $m$ in image $n$.
It is a deterministic function of the catalog (which includes random quantities)
given by
\begin{align}
\lambda_{nm} &\coloneqq \sigma_{n}(m) + \sum_{s=1}^S \ell_{s\beta_n} \int g_{nm}(\mu) h_s(\mu) \,d\mu\,.
\end{align}
The summation over light sources reflects the assumption that light sources do not occlude one another, or the skyglow.
The integral is over all sky locations.
In practice, it can be restricted to pixels near pixel $m$---distant light sources contribute a negligible number of photons.
Our implementation bases this distance measurement on conservative estimates of light sources' extents from SExtractor~\citep{bertin1996sextractor}.
As shorthand, we denote the integral as
\begin{align}
f_{nms} \coloneqq \int g_{nm}(\mu) h_s(\mu)\,d\mu.
\end{align}
If light source $s$ is a star, then it is straightforward to express $f_{nms}$ analytically:
\begin{align}
f_{nms} = g_{nm}(u_s).
\end{align}

If light source $s$ is a galaxy, the same integral is more complex because galaxies have spatial extent.
Our approach is to approximate $g_{nm}$ with a mixure of bivariate normal densities.
Because Gaussian-Gaussian convolution is analytic, we get an analytic approximation to $f_{nms}$.

Our primary use for the model is
computing the posterior distribution of its unobserved random variables conditional on a
particular collection of astronomical images. We denote the posterior by $p(z | x)$, where
$x \coloneqq \{x_{nm}\}_{n=1,m=1}^{N,M}$ represents all the pixel intensities.
Exact posterior inference is computationally intractable for
the proposed model, as it is for most non-trivial probabilistic models.
The next two sections consider two approaches to approximate posterior inference: Markov chain Monte Carlo (MCMC) and variational inference (VI).
