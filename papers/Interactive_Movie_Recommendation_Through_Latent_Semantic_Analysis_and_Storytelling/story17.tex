% $Id: template.tex 11 2007-04-03 22:25:53Z jpeltier $

\documentclass{vgtc}                          % final (conference style)
\let\ifpdf\relax
%\documentclass[review]{vgtc}                 % review
%\documentclass[widereview]{vgtc}             % wide-spaced review
%\documentclass[preprint]{vgtc}               % preprint
%\documentclass[electronic]{vgtc}             % electronic version

%% Uncomment one of the lines above depending on where your paper is
%% in the conference process. ``review'' and ``widereview'' are for review
%% submission, ``preprint'' is for pre-publication, and the final version
%% doesn't use a specific qualifier. Further, ``electronic'' includes
%% hyperreferences for more convenient online viewing.

%% Please use one of the ``review'' options in combination with the
%% assigned online id (see below) ONLY if your paper uses a double blind
%% review process. Some conferences, like IEEE Vis and InfoVis, have NOT
%% in the past.

%% Figures should be in CMYK or Grey scale format, otherwise, colour 
%% shifting may occur during the printing process.

%% These three lines bring in essential packages: ``mathptmx'' for Type 1 
%% typefaces, ``graphicx'' for inclusion of EPS figures. and ``times''
%% for proper handling of the times font family.

\usepackage{mathptmx}
\usepackage{graphicx}
\usepackage{times}
%\usepackage{epstopdf}
\usepackage{enumitem} 

%% We encourage the use of mathptmx for consistent usage of times font
%% throughout the proceedings. However, if you encounter conflicts
%% with other math-related packages, you may want to disable it.

%% If you are submitting a paper to a conference for review with a double
%% blind reviewing process, please replace the value ``0'' below with your
%% OnlineID. Otherwise, you may safely leave it at ``0''.
%\onlineid{2913}

%% declare the category of your paper, only shown in review mode
\vgtccategory{Research}

%% allow for this line if you want the electronic option to work properly
\vgtcinsertpkg
%\graphicspath{ {image/addedImages/}{image/} }

%% In preprint mode you may define your own headline.
%\preprinttext{To appear in an IEEE VGTC sponsored conference.}

%% Paper title.

\title{Interactive Movie Recommendation Through Latent Semantic Analysis and Storytelling}

%% This is how authors are specified in the journal style

%% indicate IEEE Member or Student Member in form indicated below
%\author{Roy G. Biv, Ed Grimley, \textit{Member, IEEE}, and Martha Stewart}
\author{Kodzo Wegba$^{1}$, Aidong Lu$^{1}$, Yuemeng Li$^{1}$, and Wencheng Wang$^{2}$\\
$^1$ University of North Carolina at Charlotte, USA, \{kwegba1, aidong.lu, yli19\}@uncc.edu\\
$^2$ Chinese Academy of Sciences, China, whn@ios.ac.cn
}
%\authorfooter{
%\item
%Wegba, Lu, and Li are with University of North Carolina at Charlotte. E-mail: kwegba1, aidong.lu, yli19@uncc.edu.
%\item
%Wencheng Wang (whn@ios.ac.cn) is with Chinese Academy of Sciences.
%}
%%% insert punctuation at end of each item
% Ed Grimley is with Grimley Widgets, Inc.. E-mail: ed.grimley@aol.com.
%\item
% Martha Stewart is with Martha Stewart Enterprises at Microsoft
% Research. E-mail: martha.stewart@marthastewart.com.


%other entries to be set up for journal
%\shortauthortitle{Biv \MakeLowercase{\textit{et al.}}: Global Illumination for Fun and Profit}
%\shortauthortitle{Firstauthor \MakeLowercase{\textit{et al.}}: Paper Title}

%% Abstract section.
\abstract{
Recommendation has become one of the most important components of online services for improving sale records, however visualization work for online recommendation is still very limited.
This paper presents an interactive recommendation approach with the following two components.
First, rating records are the most widely used data for online recommendation, but they are often processed in high-dimensional spaces that can not be easily understood or interacted with. 
We propose a Latent Semantic Model (LSM) that captures the statistical features of semantic concepts on 2D domains and abstracts user preferences for personal recommendation.
Second, we propose an interactive recommendation approach through a storytelling mechanism for promoting the communication between the user and the recommendation system.
Our approach emphasizes interactivity, explicit user input, and semantic information convey; thus it can be used by general users without any knowledge of recommendation or visualization algorithms.
We validate our model with data statistics and demonstrate our approach with case studies from the MovieLens100K dataset. 
%We have also evaluated the effectiveness of interactive storytelling and received very positive results.
Our approaches of latent semantic analysis and interactive recommendation can also be extended to other network-based visualization applications, including various online recommendation systems.
} % end of abstract
%The techniques of storytelling and narrative visualization have recently raised many interests.
%This work studies the analysis capabilities of storytelling for a popular application of movie recommendation.
%Our solution integrates results from two aspects:
%a new latent semantic model for extracting recommendation story pieces from the high-dimensional latent space,
%and an interactive storytelling approach for providing user-friendly narrative visualization and interactive recommendation.
%Our approach emphasizes interactivity, explicit user input, and semantic information convey; thus it can be used by general users without any knowledge of recommendation and visualization algorithms.
%We validate our model with data statistics and demonstrate our approach on interactive recommendation and visual comparison of user tastes from the MovieLens100K dataset. 
%We have also evaluated the effectiveness of interactive storytelling and received very positive results.
%Our approaches of storytelling and latent semantic analysis can be easily extended to other network-based visualization applications, including various online recommender systems.


%% Keywords that describe your work. Will show as 'Index Terms' in journal
%% please capitalize first letter and insert punctuation after last keyword
%\keywords{Interactive recommendation, latent semantic analysis, storytelling, narrative visualization.}

%% ACM Computing Classification System (CCS). 
%% See <http://www.acm.org/class/1998/> for details.
%% The ``\CCScat'' command takes four arguments.

%\CCScatlist{ % not used in journal version
%\CCScat{I.3.6}{Computer Graphics}{Methodology and Techniques}{Interaction techniques};
%\CCScat{I.3.8}{Computer Graphics}{}{Applications}
%}

% Uncomment below to include a teaser figure.
\teaser{
\centering
\includegraphics[scale=0.65]{SVDSemanticOrig.EPS}
\caption{
%The abstraction of user taste for movie recommendation and interactive storytelling, where recommended movies are enlarged as blue circles, and other candidates are shown as purple nodes.
%Narrative visualization, generated with our latent semantic model based on the domain of recommendation degrees and example-enhanced storyline, guides users throughout interactive storytelling, recommendation, and exploration processes.
An example from our latent semantic model for interactive recommendation and abstraction of user preferences.
Our approach identifies a 2D visualization domain, where the horizontal axis layouts recommendable movies on a latent dimension between two combined movie features that are selected based on the user's watch history, and the vertical axis uses recommendation degrees to move highly recommendable movies to the top.
This example demonstrates the preference of a user on drama/documentary/biography movies (green zone toward the right) over comedy/music genres (orange zone toward the left).
The movies selected to recommend are enlarged as blue circles, recommendable movies are shown as purple nodes, watched and liked movies as green nodes, and disliked movies as orange nodes.
Two example movie posters, one liked movie ``Casino" and one disliked movie ``Airheads", are also provided to demonstrate the latent dimension.
For illustration purpose, we also add the arrowed line at the bottom and several movie titles to confirm the movie distributions on the visualization domain.
}
\label{svdSemantic}
}

%% Uncomment below to disable the manuscript note
%\renewcommand{\manuscriptnotetxt}{}

%% Copyright space is enabled by default as required by guidelines.
%% It is disabled by the 'review' option or via the following command:
% \nocopyrightspace

%%%%%%%%%%%%%%%%%%%%%%%%%%%%%%%%%%%%%%%%%%%%%%%%%%%%%%%%%%%%%%%%
%%%%%%%%%%%%%%%%%%%%%% START OF THE PAPER %%%%%%%%%%%%%%%%%%%%%%
%%%%%%%%%%%%%%%%%%%%%%%%%%%%%%%%%%%%%%%%%%%%%%%%%%%%%%%%%%%%%%%%%

\begin{document}

%% The ``\maketitle'' command must be the first command after the
%% ``\begin{document}'' command. It prepares and prints the title block.

%% the only exception to this rule is the \firstsection command
\firstsection{Introduction}

\maketitle

%% \section{Introduction} %for journal use above \firstsection{..} instead
%% \leavevmode
% \\
% \\
% \\
% \\
% \\
\section{Introduction}
\label{introduction}

AutoML is the process by which machine learning models are built automatically for a new dataset. Given a dataset, AutoML systems perform a search over valid data transformations and learners, along with hyper-parameter optimization for each learner~\cite{VolcanoML}. Choosing the transformations and learners over which to search is our focus.
A significant number of systems mine from prior runs of pipelines over a set of datasets to choose transformers and learners that are effective with different types of datasets (e.g. \cite{NEURIPS2018_b59a51a3}, \cite{10.14778/3415478.3415542}, \cite{autosklearn}). Thus, they build a database by actually running different pipelines with a diverse set of datasets to estimate the accuracy of potential pipelines. Hence, they can be used to effectively reduce the search space. A new dataset, based on a set of features (meta-features) is then matched to this database to find the most plausible candidates for both learner selection and hyper-parameter tuning. This process of choosing starting points in the search space is called meta-learning for the cold start problem.  

Other meta-learning approaches include mining existing data science code and their associated datasets to learn from human expertise. The AL~\cite{al} system mined existing Kaggle notebooks using dynamic analysis, i.e., actually running the scripts, and showed that such a system has promise.  However, this meta-learning approach does not scale because it is onerous to execute a large number of pipeline scripts on datasets, preprocessing datasets is never trivial, and older scripts cease to run at all as software evolves. It is not surprising that AL therefore performed dynamic analysis on just nine datasets.

Our system, {\sysname}, provides a scalable meta-learning approach to leverage human expertise, using static analysis to mine pipelines from large repositories of scripts. Static analysis has the advantage of scaling to thousands or millions of scripts \cite{graph4code} easily, but lacks the performance data gathered by dynamic analysis. The {\sysname} meta-learning approach guides the learning process by a scalable dataset similarity search, based on dataset embeddings, to find the most similar datasets and the semantics of ML pipelines applied on them.  Many existing systems, such as Auto-Sklearn \cite{autosklearn} and AL \cite{al}, compute a set of meta-features for each dataset. We developed a deep neural network model to generate embeddings at the granularity of a dataset, e.g., a table or CSV file, to capture similarity at the level of an entire dataset rather than relying on a set of meta-features.
 
Because we use static analysis to capture the semantics of the meta-learning process, we have no mechanism to choose the \textbf{best} pipeline from many seen pipelines, unlike the dynamic execution case where one can rely on runtime to choose the best performing pipeline.  Observing that pipelines are basically workflow graphs, we use graph generator neural models to succinctly capture the statically-observed pipelines for a single dataset. In {\sysname}, we formulate learner selection as a graph generation problem to predict optimized pipelines based on pipelines seen in actual notebooks.

%. This formulation enables {\sysname} for effective pruning of the AutoML search space to predict optimized pipelines based on pipelines seen in actual notebooks.}
%We note that increasingly, state-of-the-art performance in AutoML systems is being generated by more complex pipelines such as Directed Acyclic Graphs (DAGs) \cite{piper} rather than the linear pipelines used in earlier systems.  
 
{\sysname} does learner and transformation selection, and hence is a component of an AutoML systems. To evaluate this component, we integrated it into two existing AutoML systems, FLAML \cite{flaml} and Auto-Sklearn \cite{autosklearn}.  
% We evaluate each system with and without {\sysname}.  
We chose FLAML because it does not yet have any meta-learning component for the cold start problem and instead allows user selection of learners and transformers. The authors of FLAML explicitly pointed to the fact that FLAML might benefit from a meta-learning component and pointed to it as a possibility for future work. For FLAML, if mining historical pipelines provides an advantage, we should improve its performance. We also picked Auto-Sklearn as it does have a learner selection component based on meta-features, as described earlier~\cite{autosklearn2}. For Auto-Sklearn, we should at least match performance if our static mining of pipelines can match their extensive database. For context, we also compared {\sysname} with the recent VolcanoML~\cite{VolcanoML}, which provides an efficient decomposition and execution strategy for the AutoML search space. In contrast, {\sysname} prunes the search space using our meta-learning model to perform hyperparameter optimization only for the most promising candidates. 

The contributions of this paper are the following:
\begin{itemize}
    \item Section ~\ref{sec:mining} defines a scalable meta-learning approach based on representation learning of mined ML pipeline semantics and datasets for over 100 datasets and ~11K Python scripts.  
    \newline
    \item Sections~\ref{sec:kgpipGen} formulates AutoML pipeline generation as a graph generation problem. {\sysname} predicts efficiently an optimized ML pipeline for an unseen dataset based on our meta-learning model.  To the best of our knowledge, {\sysname} is the first approach to formulate  AutoML pipeline generation in such a way.
    \newline
    \item Section~\ref{sec:eval} presents a comprehensive evaluation using a large collection of 121 datasets from major AutoML benchmarks and Kaggle. Our experimental results show that {\sysname} outperforms all existing AutoML systems and achieves state-of-the-art results on the majority of these datasets. {\sysname} significantly improves the performance of both FLAML and Auto-Sklearn in classification and regression tasks. We also outperformed AL in 75 out of 77 datasets and VolcanoML in 75  out of 121 datasets, including 44 datasets used only by VolcanoML~\cite{VolcanoML}.  On average, {\sysname} achieves scores that are statistically better than the means of all other systems. 
\end{itemize}


%This approach does not need to apply cleaning or transformation methods to handle different variances among datasets. Moreover, we do not need to deal with complex analysis, such as dynamic code analysis. Thus, our approach proved to be scalable, as discussed in Sections~\ref{sec:mining}.
%\section{Introduction}

%Modern consumers are inundated with information and choices.

Recommendation systems have been reported as key pillars of online services~\cite{Gomez-Uribe:2015:NRS:2869770.2843948} for significantly improving sale records~\cite{1167344}.
Nowadays, online retailers and content providers offer a huge amount of products or services, which are often overwhelming for consumers.
To improve user satisfaction and loyalty, Internet leaders like Amazon, Google, and Yahoo are all using recommendation systems to provide personalized suggestions.
However, matching consumers with the most appropriate products is not trivial.
While many recommendation algorithms have been developed, including 
the prevailing top-N recommendation approaches~\cite{Deshpande:2004:ITN:963770.963776}, effective interaction mechanisms for consumers to adjust search preferences or recommendation results are still lacking~\cite{loepp2014choice}.
%The most popular approaches, item-based top-N recommendation~\cite{Deshpande:2004:ITN:963770.963776}, are often achieved through establishing connections between consumers and products from the past ratings. 

The challenges for improving user satisfaction of interactive recommendation systems come from several aspects.
First, the satisfaction of consumers vary obviously with other factors such as emotions and situations, which require explicit input from users.
Second, useful information for assisting users to find the right movies, such as the similarities between recommended and rated movies, is often completely hidden from users.
Third, complex recommendation algorithms and visualization systems are often too complicated for general consumers to use.
To the best of our knowledge, there are no previous work on addressing all above challenges.

%bridging the gap between fully automatic top-N recommendation algorithms and complex visualization techniques to develop user-friendly recommendation systems.

In this work, we present an interactive recommendation approach that simulates the scenario of an expert recommending movies to a user -- an expert generally selects movies for a user based on his or her watch history, makes recommendations, listens to the feedback, and continues to recommend more movies until the user finds a movie to watch.
Similarly, our approach adopts such a continuous and interactive recommendation process that allows the user to orientate search results actively and explicitly.
For efficiency, we recommend movies in groups, with each group 
containing movies selected from combined movie features that are extracted from watch history.
As shown in Figure~\ref{svdSemantic}, for a user who likes a number of drama, documentary, and biography movies, we visualize the movie distributions on the dimension between the liked and disliked movie features, and make recommendations based on important criteria such as user preference and variety.
%As shown in Figure~\ref{svdSemantic}, a real user may have watched movies from a diverse set of genres and the recommended movies need to be selected based on the user's watched history.
This requires us to study the personal movie features from rating statistics and present recommended movies with both useful information and connections with user-rated movies in a succinct, user-friendly manner.
%Our interactive recommendation approach continuously modifies recommendation contents and adjust story styles based on user interaction, which simulates the scenario that a user consults an expert for movie recommendation.

Our work consists of two components, a model of latent semantic analysis and an approach of interactive recommendation.
We first present a latent semantic model (LSM) built upon the high-dimensional latent space factorized from the movie rating records, as they are the most commonly used data for online recommendation.
Our model collects a set of semantic concepts, including like, dislike, familiarity, diversity, typicality, and un-typicality, that can be used for two purposes:
one is to help a user to correlate new recommended movies to the movies the user has watched, the other is to allow the user to specify the preferences of recommendation results explicitly.
We abstract the statistical features of semantic concepts among the high-dimensional latent space and automatically identify suitable 2D domains for interactive visualization.
We validate LSM with a real-life dataset to show that our model is applicable to various users for recommendation.

We then present an interactive recommendation system that recommend movies in a storytelling style to promote the communication between the user and recommendation system.
We employ the LSM to generate recommendation stories by assigning movies with suitable characters, roles, and narrative structures for describing a group of recommended movies.
The recommendation stories are constructed automatically to attract user attention, supported with multi-level visualization and animation effects, and can be adjusted flexibly during the interactive recommendation process.
Different from previous online recommendation systems, our approach reveals information that is generally hidden from users and allows interactive exploration of movie similarities.
%This is designed based on our study of the application requirements and existing recommender systems.
%Both the narrative visualization and user interactions are designed for users without any background of recommendation or visualization. 
%We have also performed a controlled user study and a questionnaire to evaluate our approach.
%The results demonstrate that interactive recommendation is very promising for engaging user participation and providing better recommendation experiences.
We also provide several case studies to demonstrate the usage of LSM and interactive recommendation on different recommendation tasks.
%, including visualization of user tastes, recommendation to individual or group users, and comparison of user tastes.

The remainder of this paper is organized as follows. 
We start with related work in Section 2.
We then describe our LSM in Section 3 and interactive recommendation approach in Section 4.
We present the recommendation system in Section 5 and results in Section 6.
Section 7 concludes the work and presents future work.

%\section{Related Work}\label{sec:related}
 
The authors in \cite{humphreys2007noncontact} showed that it is possible to extract the PPG signal from the video using a complementary metal-oxide semiconductor camera by illuminating a region of tissue using through external light-emitting diodes at dual-wavelength (760nm and 880nm).  Further, the authors of  \cite{verkruysse2008remote} demonstrated that the PPG signal can be estimated by just using ambient light as a source of illumination along with a simple digital camera.  Further in \cite{poh2011advancements}, the PPG waveform was estimated from the videos recorded using a low-cost webcam. The red, green, and blue channels of the images were decomposed into independent sources using independent component analysis. One of the independent sources was selected to estimate PPG and further calculate HR, and HRV. All these works showed the possibility of extracting PPG signals from the videos and proved the similarity of this signal with the one obtained using a contact device. Further, the authors in \cite{10.1109/CVPR.2013.440} showed that heart rate can be extracted from features from the head as well by capturing the subtle head movements that happen due to blood flow.

%
The authors of \cite{kumar2015distanceppg} proposed a methodology that overcomes a challenge in extracting PPG for people with darker skin tones. The challenge due to slight movement and low lighting conditions during recording a video was also addressed. They implemented the method where PPG signal is extracted from different regions of the face and signal from each region is combined using their weighted average making weights different for different people depending on their skin color. 
%

There are other attempts where authors of \cite{6523142,6909939, 7410772, 7412627} have introduced different methodologies to make algorithms for estimating pulse rate robust to illumination variation and motion of the subjects. The paper \cite{6523142} introduces a chrominance-based method to reduce the effect of motion in estimating pulse rate. The authors of \cite{6909939} used a technique in which face tracking and normalized least square adaptive filtering is used to counter the effects of variations due to illumination and subject movement. 
The paper \cite{7410772} resolves the issue of subject movement by choosing the rectangular ROI's on the face relative to the facial landmarks and facial landmarks are tracked in the video using pose-free facial landmark fitting tracker discussed in \cite{yu2016face} followed by the removal of noise due to illumination to extract noise-free PPG signal for estimating pulse rate. 

Recently, the use of machine learning in the prediction of health parameters have gained attention. The paper \cite{osman2015supervised} used a supervised learning methodology to predict the pulse rate from the videos taken from any off-the-shelf camera. Their model showed the possibility of using machine learning methods to estimate the pulse rate. However, our method outperforms their results when the root mean squared error of the predicted pulse rate is compared. The authors in \cite{hsu2017deep} proposed a deep learning methodology to predict the pulse rate from the facial videos. The researchers trained a convolutional neural network (CNN) on the images generated using Short-Time Fourier Transform (STFT) applied on the R, G, \& B channels from the facial region of interests.
The authors of \cite{osman2015supervised, hsu2017deep} only predicted pulse rate, and we extended our work in predicting variance in the pulse rate measurements as well.

All the related work discussed above utilizes filtering and digital signal processing to extract PPG signals from the video which is further used to estimate the PR and PRV.  %
The method proposed in \cite{kumar2015distanceppg} is person dependent since the weights will be different for people with different skin tone. In contrast, we propose a deep learning model to predict the PR which is independent of the person who is being trained. Thus, the model would work even if there is no prior training model built for that individual and hence, making our model robust. 

%
\section{Related Work}

This section presents the previous works on interactive recommendation, storytelling and narrative visualization approaches.

\subsection{Visualization for Recommendation}

Different from the topic of visualization recommendation that suggests suitable visualization formats given data or tasks, we focus on visualization approaches for recommender systems, where graph visualizations are often adopted.
%Recommender systems have been widely applied to online retailers.
%such as for images~\cite{4703261}, music~\cite{5972554}, books, arts, and movies.
%QoS-based web services~\cite{5928315}, 
%Instead of presenting recommendation algorithms from the fields of data mining and machine learning, we concentrate on related interactive approaches, where the graph visualizations are often adopted.
For example, 
Luo et al.~\cite{luo2009personalized} used hyperbolic and multi-modal view to visualize a recommendation list. 
Kermarrec et al.~\cite{kermarrec2012data} used SVD-like matrix factorization and PCA for global mapping of movie ratings from high dimensions to a two-dimensional space. 
% They used Curvilinear Component Analysis (CCA) for personalized recommender 
% list mapping for a given user.
Crnovrsanin et al.~\cite{crnovrsanin2011visual} proposed a 
task-based and information-based network representation for users to 
interact and visualize a recommendation list. 
%
Vlachos et al.~\cite{vlachos2012recommendation} used bipartite graphs and minimum spanning trees to explore and visualize recommendation results of a movie-actor dataset. 

Interactive recommendation approaches have also been developed.
Gretarsson et al.~\cite{gretarsson2010smallworlds} visualized recommended users in social networks with node-link diagram and grouped relevant nodes on 
the recommendation list in parallel layers. 
Loepp et al. presented an interactive recommendation approach by having users to choose between two sets of sample items iteratively and extracting latent factors~\cite{loepp2014choice}.
Recently, Loepp et al. ~\cite{loepp2015blended} presented MyMovieMixer for interactively expressing user preferences over the hybrid recommendation process.

%Different from existed visualization approaches for recommender systems, 
Our approach dynamically recommends suitable movies to users with the support of interactive storytelling methods.
A key feature of semantic storytelling is that users do not need to understand complex recommendation algorithms or visualization techniques.

 

\subsection{Storytelling and Narrative Visualization} 

``Storytelling" has a long history and it has become a visualization technique~\cite{Gershon:2001:SIV:381641.381653, 1626183, 6111347, 6412677, 7274435, wojtkowski2002storytelling}. 
While the term of narrative visualization is relatively new~\cite{segel2010narrative}, it also refers to using data stories to improve visual communication~\cite{hullman2011visualization, Hullman:2013:DUS:2553699.2553753, Satyanarayan:2014:ANV:2771495.2771532}.

We focus on techniques of narrative structure, which is a key concept in storytelling and narrative visualization.
It refers to ``a series of events, facts, etc., given in order and with the establishing of connections between them" from the Oxford English Dictionary and it is often simplified to structures like beginning, middle, and end in visualization systems~\cite{segel2010narrative}.
Studies from journalism~\cite{segel2010narrative}
%, videos~\cite{amini2015understanding}, 
and political messaging and decision-making~\cite{hullman2011visualization} have been performed to understand useful narrative structures for visualization.

Several interactive or automatic storytelling approaches have been developed for applications ranging from general visualization process~\cite{cruz2011generative, 4677364} 
%such as generative storytelling~\cite{cruz2011generative, 6902879}, documents~\cite{4271973}, and events~\cite{4677364}, 
to specific domains. %such as geo-visualization~\cite{Lidal:2012:GSG:2331067.2331070, 6676572}.
For example, Wohlfart and Hauser~\cite{Wohlfart:2007:STP:2384179.2384194}
used storytelling as a guided and interactive visualization presentation approach for medical visualization.
Eccles et al.~\cite{4388992} detected geo-temporal patterns and integrates story narration to increase analytic sense-making cohesion in GeoTime.
%Cruz and Machado~\cite{cruz2011generative} developed a conceptual framework to build various stories from a set of time-ordered events given a dataset.
Yu et al.~\cite{CGF:CGF1816} generated automatic animations with narrative structures extracted from event graphs for time-varying scientific visualization.
Hullman et al.~\cite{Hullman:2013:DUS:2553699.2553753} presented a graph-driven approach for automatically identifying effective sequences in a set of visualizations to be presented linearly.
Lee et al.~\cite{Lee:2013:STM:2553699.2553755} presented a storytelling process with steps involved in finding insights, turning insights into a narrative, and communicating to an audience. 
Satyanarayan and Heer~\cite{Satyanarayan:2014:ANV:2771495.2771532} developed a model of storytelling abstractions and instantiate the model in Ellipsis with a graphical interface for story authoring.
Wang et al. presented a narrative visualization system that presents literature review as interactive slides with three levels of narrative structures~\cite{Wang:2016:GTL:2968220.2968242}.
Bryan et al.~\cite{7539294} generated textual annotations with a temporal layout and comic strip-style data snapshots for visualizing multidimensional and time-varying data.


%Hullman and Diakopoulos~\cite{hullman2011visualization} demonstrated visualization rhetoric as an analytical framework for understanding the effects of design techniques on end-user interpretation.


%Andrews and Baber~\cite{andrews2014visualizing} designed a branching comic to compare how readers recall a visual narrative.
%Pschetz et al.~\cite{pschetz2014turningpoint} developed TurningPoint to investigate narrative-driven talk planning in slideware.
% and demonstrated that narrative templates allowed users to focus attention and limit experimentation at the same time.
%Spaulding and Faste used studies to prototype and build immersive design words~\cite{spaulding2013design}.
%We didn't include storytelling from videos including movies as their focus is often on imaging techniques.

Storytelling has been shown to be effective on conveying data in a number of applications~\cite{dragicevic2011temporal, spaulding2013design}.
For visualization tasks, while storytelling did not seem to increase user-engagement in exploration~\cite{boy2015storytelling}, annotated visualizations were proven to be better in balancing graphical salience and relevance~\cite{Hullman:2013:CAG:2470654.2481374} and graph comics were useful to help a general audience understand complex temporal changes quickly~\cite{bach2016telling}.
%The results are mixed~\cite{gonzalez1996does, boy2015storytelling}.
Nonetheless, it is clear that a flexible creation process should be provided and  adjusted according to application requirements~\cite{mitchell2011limits, 6902874, amini2015understanding}.

Different from other storytelling techniques, we present a highly interactive storytelling approach that simulates human communication with two features - continuous updating stories with or without user inputs and allowing interaction in all stages of exploring data, making a story, and telling a story.


%\input{narrative}
\section{Latent Semantic Model for Interactive Recommendation and Movie Exploration} 
\label{latent}

Our interactive recommendation approach for general users is consisted of two components: LSM for abstracting personal movie preferences (Section 3) and interactive recommendation with storytelling (Section 4).
In this section, we start by introducing the latent space from collaborative filtering algorithms.
We then describe our recommendation approach and measurement of recommendation degrees.
At the end, we present the LSM, which transforms high-dimensional data statistics into recommendation domains according to a set of semantic concepts. 
The LSM is also used to design the recommendation storytelling and user interaction in Section 4.
During this work, the designs of LSM and interactive recommendation system are simultaneously proceeded to ensure that the same set of semantic concepts can be used for both interactive recommendation and exploration of movies.


\subsection{Latent Space from Collaborative Filtering}

To provide an effective interactive recommendation system, we need to integrate a recommendation algorithm into the visualization mechanism.
The latent factor models are the primary approaches of Collaborative Filtering (CF) techniques, which have been successfully adopted by a number of commercial systems~\cite{Koren:2008:FMN:1401890.1401944}.
The latent factor models based on Singular Value Decomposition (SVD) establish recommendations by transforming both movies and users to the same latent factor space, thus making them directly comparable. 
We choose this latent space as it can be used to explore not only recommended movies, but also the distribution patterns of movies and users from the aspect of semantic analysis.

The latent space can be used to interpret a number of preference / relevance features. 
For example, a dimension from comedy to drama can be used to represent the taste of a user favoring these two movie genres.  
In majority of cases, the latent space captures statistical distribution of rating records that combine features from all users, which are often hard to describe or understand directly.
To distinguish users from movies, we reserve special indexing letters of $u$ and $v$ for users and $i$ and $j$ for movies. 
A rating $r_{ui}$ indicates the preference of a user $u$ on a movie $i$, where rating values are in the set of $\{1, 2, ..., 5\}$ with $1$ for no interests to $5$ for strong interests. 

We generate the latent space with the factorization of user-movie rating matrix using SVD. 
For a user-movie matrix $M$ with $m$ users and $n$ movies, the SVD algorithm factorizes $M$ into three matrices such that $M = USV^T$.
It is common to truncate these matrices to yield $U_k$, $S_k$, and $V_k$, in order to decrease the dimensionality of the vector space, and only leave the strongest effects in the model by dropping dimensions with small singular values~\cite{Ekstrand:2011:CFR:2185827.2185828}.
Specifically, the rows of the $U_k$ are the interests of users in each of the $k$ inferred features, and the columns of the matrix $V_k$ are the relevance of movies for each feature. 
The diagonal matrix $S_k$ contains the $k$ biggest singular values of $M$, which are the weights for the preferences, representing the influence of a particular topic on user-movie preferences. 


\subsection{Recommendable Movies and Degrees}

%We filter the dataset by removing users without any common ratings. 

In a recommendation approach, we are interested in two things the most: recommendable movies and recommendation degrees for interpreting how likely we would recommend a movie.
We follow a typical recommendation algorithm by adjusting rating history with the normalization of global effects. 
This step balances the tendencies that some users like to give higher ratings than others and some movies receive higher ratings than others, therefore the adjusted ratings $\hat{r_{ui}}$ are more accurate to compare different users or different movies.
Denote $a_{u}$ as the average rating given by user $u$, $a_{i}$ as the average rating of movie $i$, and $A$ and $B$ as the averages of all users and all movies respectively. 
\begin{equation}
\hat{r_{ui}} = r_{ui} - (a_{u} - A) - (a_{i} - B)
\end{equation}

Our recommendation algorithm starts by mapping each user into the latent space by multiplying the adjacency matrix $M$ and the $V_k$ component of SVD.
We denote this product as $C=M \times V_k$, where $C_u$ represents a row vector for user $u$ from the matrix $C$.
We then compute the cosine similarity $s_{uv}$ between users $u$ and $v$ %using Pearson Correlation coefficient 
with the $C_u$ and $C_v$ coordinates~\cite{leskovec2014mining}.
\begin{equation}
s_{uv} =  \frac{\displaystyle\sum_{l=1}^{k} {C_u}_l \times {C_v}_l}
  			 {\sqrt{\displaystyle\sum_{l=1}^{k} ({C_u}_l)^2 } \times 
  			  \sqrt{\displaystyle\sum_{l=1}^{k} ({C_v}_l)^2 } }    
\label{similarity}     
\end{equation}

Next, we select a list of similar users with positive similarity coefficients, as recommendable movies are generally selected based on ratings from similar users.
We denote this set as $S_u=\lbrace v | s_{uv} \geq 0 \rbrace $ for user $u$.
For a new user with no rating record, all the similarity coefficients $s_{uv}$ are zero and the set $S_u$ contains all the users.

We further select the list of recommendable movies $L_u$ for the user $u$ by choosing movies that have not been watched by $u$, but received positive ratings from similar users as follows, where the positive rating threshold $w_c$ is initialized as $3$ and can be adjusted for different numbers of recommendable movies.
\begin{equation}
L_u=\lbrace i | v \in S_u, \hat{r_{vi}} \ge w_c \ and \ r_{ui}=0 \rbrace. 
\end{equation}

In addition, we measure the recommendation degree of a movie $i$ for user $u$ by $b_{ui}$.
The movies with high $b_{ui}$ degrees are more likely to be recommended during the interactive recommendation process.
%We use it to reduce the size of $L_u$ by keeping movies with large $b_{ui}$ degrees and select movies during the interactive recommendation process.
It is calculated as the average rating from similar users with positive cosine similarities. 
For any $i \in L_u$,
\begin{equation}
b_{ui} = \sum_{v \in S_u \ and \  r_{vi} \ne 0} \{r_{vi} \times 
s_{uv}\} / \sum_{v \in S_u \ and \ r_{vi} \ne 0} \{r_{vi}\}
\label{ratingScore}
\end{equation}
We normalize the $s_{uv}$ and $b_{ui}$ values to range of $[-1, 1]$ and $[0,1]$ respectively to balance the differences among users.

%For example, when we set $w_c$ to 3 for a sample user, we have $b_{ui}$ ranging from 0.199354(for `` Ayn Rand: A Sense of Life (1997)'', a documentary movie) to 0.012575 (for`` Showgirls (1995)'' which is a drama movie).
%The full result of our model is shown in Figure~\ref{svdSemantic}.

\begin{figure}
\centering
\includegraphics[width = 3.4in]{model1v.eps}\\
LSM mode (a) -- Variations based on the Like/Dislike groups\\
\includegraphics[width = 3.4in]{model2.eps}\\
LSM mode (b)
\caption{The LSM maps the distribution of a set of semantic concepts in the latent space.
We separate the two modes of LSM for clarity -- (a) includes concepts of like, dislike, familiarity and diversity and (b) includes concepts of typicality and un-typicality.
%The two modes co-exist in all the latent dimensions and LSM chooses the right one to use automatically.
The movie nodes are colored based on the groups -- liked in green, disliked in orange, recommendable in blue, and not recommendable in black.
}
\label{svd}
\end{figure}
%(neutral in yellow), 

\subsection{Latent Semantic Model}

While the latent space has been widely used in recommendation algorithms, it is a high-dimensional space and not directly suitable for visualization or interaction. 
Our goal of the LSM is to identify semantic concepts and visualization domains that can be used in interactive recommendation and exploration of movie similarities.
%This is achieved through finding a set of semantic concepts related to recommendation from the latent space $V_k$ and studying their statistical features.
Our approach is consisted of the following three steps: separating movie groups, identifying a set of semantic concepts, and selecting suitable latent dimensions for interactive recommendation.


\subsubsection{Separating Movie Groups}

To identify recommendable movies, we separate all the movies in the database to five groups based on the rating history of the user and our estimation of recommendation degrees.
For any user $u$, each movie belongs to one and only one of the following groups.

\begin{itemize}
\vspace{-2mm}
\item Like (with positive ratings from $u$): $G_+ = \{i | {r_{ui}} \geq \tau_{+}\}$
\vspace{-2mm}
\item Dislike (with negative ratings from $u$): $G_- = \{i | 0 < {r_{ui}} < \tau_{-}\}$
\vspace{-2mm}
\item Neutral: $G_{neu} = \{i | \tau_{-} \leq r_{ui} <\tau_{+}\}$
\vspace{-2mm}
\item Recommendable (with positive recommendation degrees):\\ $G_r = \{i | b_{ui} \geq \tau_{r}\}$ and not specified as ``thumb-down" by the user during interaction recommendation process
\vspace{-2mm}
\item Not recommendable (with negative recommendation degrees): $G_n = \{i | b_{ui} < \tau_{r}\}$ or specified as ``thumb-down" by the user during interaction recommendation process
\end{itemize}

The thresholds of positive rating $\tau_{+}$ and negative rating $\tau_{-}$ are set to value $3$, and the threshold of recommendation degree $\tau_{r}$ to value $0$ initially.
The values can be adjusted to control the number of movies in each group.


\subsubsection{Semantic Concepts for Recommendation}

In everyday life, we often describe an object with several terms, such as families, friends, and enemies for a person.
Similarly, a set of semantic concepts can be used to describe movies for a quick impression, which may assist users to find movies effectively.

In addition to ``like" and ``dislike", we have identified the following four semantic concepts for recommendation based on two criteria -- concepts that are often used in recommendation for the general public; 
and concepts with clear distribution features in the latent space.
%and more importantly, concepts that can be easily understood by general users without any knowledge of recommendation or visualization techniques.

\begin{itemize}
\item Familiarity -- movie styles that the user has already watched;
%regions containing movies the user $u$ has rated with both positive and negative ratings; 
\vspace{-2mm}
\item Diversity -- movie styles that the user is not familiar with;
% other regions outside the familiar zone.
\vspace{-2mm}
\item Typicality -- movie styles that can be well defined based on combined movie features;
%regions away from the Origin of the latent space.
\vspace{-2mm}
\item Un-typicality -- movie styles that are unclear to a feature.
% the region close to Origin the latent space.
\end{itemize}

%The core of our LSM is to extract sub-latent spaces that map the distributions of recommendable movies based on the set of semantic concepts, which further enable us to construct recommendation stories and interaction mechanisms for users to adjust recommendation preferences.
Next, we describe the distribution features of semantic concepts in the latent space.
As illustrated in Figure~\ref{svd}, we can identify the familiar zone by including both the like and dislike groups -- all the movies that have been watched by the user $u$.
The diverse zones are regions outside the familiar zone and there are generally two diverse zones on each side of a latent dimension.
Figure~\ref{svdexample} uses examples from real scenarios to show the variations of the LSM mode (a).

The LSM mode (b) between typicality and un-typicality utilizes the distance of a movie to the Origin of the latent space.
Since both the $U_k$ and $V_k$ from the latent space demonstrate clustering features of similar users and movies~\cite{Pu:2013:UIR:2507157.2507178}, the distances in the latent space can interpret the similarity degrees.
For example, on a latent dimension which contains comedy movies on one side and drama movies on the other, the movies with combined comedy/drama or other genres are distributed near the Origin.
This feature is similar to spectral spaces that nodes with few connections are often located close to the Origin~\cite{6231611}.
What's important is that this feature preserves for general latent dimensions describing combined movie features or styles, with only un-typical movies corresponding to the latent dimension located close to the Origin.
As shown in Figure~\ref{svd} mode (b), any latent dimension can be modeled as typical to untypical to typical zones.
The two typical zones correspond to two opposite features represented by the latent dimension.

While the distribution features of semantic concepts on the latent dimensions are clear, the relative locations among the semantic zones 
vary depending on the Origin and the familiar zone.
For example, the Like and Dislike zones can appear on either side of the Origin or Overlap with the Origin.
%As shown in Figures~\ref{svd} and \ref{svdexample}, the zones of familiarity and diversity do not overlap, while the zones of familiarity/diversity and zones of typicality/un-typicality are independent.
Nonetheless, they all provide valuable semantic information for interactive recommendation.
We generally adjust thresholds of $\tau_{+}$, $\tau_{-}$, and $\tau_{r}$ to make sure that all the semantic zones contain recommendable movies.

\begin{figure}
\centering
%\includegraphics[width = 3in]{./image/Case1.eps}\\
%\includegraphics[width = 3in]{./image/Case1Bis.eps}\\
%\includegraphics[width = 3in]{./image/Case2.eps}\\
%\includegraphics[width = 3in]{./image/Case4.eps}
\includegraphics[width = 3.4in]{BestDim1.eps}\\
\vspace{+1mm}
\includegraphics[width = 3.4in]{BestDim2.eps}\\
\vspace{+1mm}
Good latent dimensions for visualization and interaction\\
\vspace{+1mm}
\includegraphics[width = 3.4in]{WorstDim1.eps}\\
\vspace{+1mm}
\includegraphics[width = 3.4in]{WorstDim2.eps}\\
\vspace{+1mm}
Bad latent dimensions as visualization and interaction
\caption{Good and bad example latent dimensions that are selected based on criteria described in Section 3.3.3.
The liked and disliked zones are highlighted with shaded rectangles in green and orange respectively.
The recommendable movies are colored in purple, ``liked" movies in green, and ``disliked" movies in orange.
%The first two examples are good candidates of visualization domains, as they describe different aspects of combined user tastes.
%The last two rows are not suitable for visualization, as they do not separate the like and dislike zones well.
}
\label{svdexample}
\end{figure}
%The grey column near the center marks the Origin of the latent space.
%All the examples demonstrate that our latent semantic model can represent diverse scenarios from the real datasets.

\subsubsection{Selecting Dimensions for Visualization and Interaction}

In the latent space, each dimension describes certain joint statistical features of movies and users.
%Generally the most dominant features appear first and small features later. 
We are interested in searching for dimensions that can help users understand movie relationships and are suitable for visualization and interaction.
%
We need to select not only one, but also a set of suitable dimensions, so that multiple aspects of recommendable movies can be covered.
However, not all the dimensions in the $k$-dimensional latent space are suitable for recommendation, as they may represent features of irrelevant movies or users, repeated features from other dimensions, or features that are hard to understand.
Therefore, we go through the following process of selection based upon the distribution of the semantic concepts.

The selection is based on the combined information of user history, recommendation degrees, and explicit information from user interaction -- thumb up/down for specific movies, which is described in the section 4.
We prefer to choose the dimensions that separate the semantic zones with the following three factors:
%contain suitable distribution of recommendable movies according to the first two ideal cases of our model.
%Specifically, we consider the following three factors: sizes of important groups, overlapping relationships, and the distribution of recommendable movies.
\begin{description}[style=unboxed,leftmargin=0cm]
\vspace{-2mm}
\item[Group sizes.] 
For each dimension, we measure the region of the like group as $R_+$, the dislike group as $R_-$, the overlapped region as $R_{o}$, and the combined range as $R$.
It is ideal that $R_+$ occupies a large portion of $R$, as majority recommendable movies are selected within or close to this region.
We also prefer that $R_+$ and $R_-$ do not cover the entire dimension, so that there is room for diverse zones.
This factor is simplified as $R_+ / R$.
%Assuming $R$ is the range on the dimension, we want $R_+ / R$ to be large and $R_o / R$ to be small.
\vspace{-2mm}
\item[The overlapping region.]
We try to avoid the third and fourth cases of LSM mode (a) in Figure~\ref{svd}, which happens when one group is located inside another.
They are less desirable as the meanings of such dimensions are hard to describe and understand.
We set large penalty to avoid large overlapping ratios for $R_o / R_+$ and $R_o / R_-$. 
%\vspace{-2mm}
%\item[The number of recommendable movies.]
%We prefer a dimension capturing a big portion of recommendable movies, especially on the zones of $R_+ \bigcap \overline{R_o}$.
%Denoting the numbers as $N_+$ and $N_{o}$ for groups of $R_+$ and $R_{o}$.
%This factor is also computed as the ratio $(N_+ - N_o)/N$, where $N = | L_u |$.
\vspace{-2mm}
\item[The distribution of recommendable movies.]
Inside $R_+$, it is ideal that the recommendable movies are evenly distributed. %, such as the Normal distribution.
This factor helps to remove the dimensions with many movies mapping to a small range, which indicates that these movie features and differences are not well represented on the dimensions.
%We can use 68?95?99.7 rule to measure if the distribution is normal.
%https://en.wikipedia.org/wiki/68%E2%80%9395%E2%80%9399.7_rule
We measure this factor with standard deviation $\theta_+$ of recommendable movies in $R_+$.
\end{description}

Specifically, we use the following $D_v$ equation to measure if a latent dimension $v$ is suitable as a visualization domain, where $w_+$, $w_{o}$ and $w_{\theta}$ are the weights for the three factors described above.
\begin{equation}
D_v =  (\frac{R_+}{R})^{w_+} \times (1 - \frac{R_o}{R_+})^{w_o}  \times (1 - \frac{R_o}{R_-})^{w_o} \times ( \frac{R_+}{\theta_+})^{w_{\theta}} 
\label{equ:dv}
\end{equation}
%D_v =  (\frac{R_+}{R})^{w_+} \times ( (1 - rRo) * (1 - rNo)) ^{w_o} \times (1 - \frac{R_o}{R_+})^{w_o}  \times (1 - \frac{R_o}{R_-})^{w_o} \times ( \frac{R_+}{\theta_+})^{w_d} 
%Dv = Math.Pow(rRl* rNl, 2) * Math.Pow( (1 - rRo) * (1 - rNo), 20) * Math.Pow(1 - Old / Rl, 10) * Math.Pow(1 - Old / Rd, 10)


%separate the like and dislike groups and spread out the movies in each group.
%Specifically, for each of the $k$ dimensions, we calculate the regions of the like group as $R_+$, the dislike group as $R_-$, and the overlapping regions as $R_{o}$.
%We also consider the significance of a latent dimension through $S_v$ to favor important dimensions in the latent space.
%The following $D_v$ measures if a latent dimension $v$ is suitable for a plot candidate:
%\begin{equation}
%D_v = (w_+ \times R_+ \times w_{+b} + w_- \times R_- \times (1 - w_{-b}) + w_{o} \times R_{o}) \times {|S_v|}^{w_s}
%\label{equ:dv}
%\end{equation}
%where $w_+$, $w_-$, $w_{o}$, and $w_s$ are the weights for importances of groups $R_+$, $R_-$, and $R_{o}$ and $S_v$ factor respectively. 
%Also, we use $R_{+l}$ and $R_{+h}$ to define the low and high boundaries of the like group, and similarly $R_{-l}$ and $R_{-h}$ for dislike group.
%We use large $w_+$ to emphasize the distribution features in the like group and large $w_{o}$ to reduce the overlap region between two groups.
%\begin{eqnarray}
%\nonumber
%R_{o} &=& |R_{-h} - R_{+l}| - (R_+ + R_-), if (R_{+l} \leq R_{-l})\\
%&& |R_{+h} - R_{-l}| - (R_+ + R_-), otherwise.
%\end{eqnarray}
%The $w_{+b}$ and $w_{-b}$ are the effects of recommendation degrees and detect if $b_{ui}$ distributions are consistent with the groups of $R_+$ and $R_-$.
%As in the following equation, we define $Z_+$ as the set of recommendable movies in $R_+$ and $Z_-$ as the set of recommendable movies in $R_-$ on dimension $v$.
%\begin{equation}
%\nonumber
%w_{+b} = \frac{\sum_{i \in Z_+} {b_{ui}}}{| Z_+ |}
%\ \ \ \ \ \ \ \ 
%w_{-b} = \frac{\sum_{i \in Z_-} {b_{ui}}}{| Z_- |}
%\end{equation}

Next, we filter the set $S_{d} = \{v | D_v > \tau_{v} \}$, composed of latent dimensions with high $D_v$ values, by removing similar dimensions.
This is achieved by comparing the locations of movies to the groups on the two dimensions.
Among the similar dimensions with high $D_s$ values, only the dimension with the highest $D_v$ value is kept in $S_{d}$.
\begin{equation}
%D_s = \sum_{i \in G_l \cup G_d, p, q \in G_{s}} w_i \times 
%|\frac{|l_{ip} - lc_{p}|}{R_{lp}} - \frac{|l_{iq} - lc_{q}|}{R_{lq}}|
D_s(p, q) = \sum_{i \in L_u} w_i \times 
[(i \in R_{+p}) \bigoplus (i \in R_{+q}) + (i \in R_{-p}) \bigoplus (i \in R_{-q})]
\label{equ:ds}
\end{equation}
where, $p \in S_{d}$ and $q \in S_{d}$ are latent dimensions, $w_i$ is the weight for each movie to incorporate user preference, $R_{+p}$ and $R_{-p}$ are the ranges of like and dislike groups on $p$.
%$G_l$ and $G_d$ are the set of all recommended items for which the user clicks the thumb-up/down buttons.
We set $w_i = 1$ for all movies initially and double the value when the user clicks the thumb-up/down buttons.

%We sort the set $G_{s} = \{v | D_v > th_{ds} \}$ of selected dimensions according to $D_v$ and always choose the dimension with the highest $D_v$ value.
%Then, we continue to filter the set $G_{s}$ by comparing the relative locations of movies to the group centers.
%Dimensions with low $D_s$ values are skipped, until we finish all the $k$ dimensions.

%This give us a small set of dimensions to use, each can be described using several example movies the user has watched.

%We run our model using MovieLens dataset after downloading movies' images from IMDb web site. 
%Figure~\ref{svdSemantic} shows an example of latent dimension describing user taste of movies. 
%The movies range from drama, horror, love, success, and city-dweller types (left) to drama, love, life after war, finance struggle, and country types (right). 
%Through our model, we can summarize that this user prefers the movies about success and stories in large cities over topics of poor, struggle and rural areas. 

%who is a 32 year old man working as administrator in Boston city, Massachusetts,
%the second best dimension selected based on Dv score corresponding to the 166th dimension of SVD.
%The user's liked and disliked zones are separated on this dimension. 

%\begin{figure*}
%\centering
%\includegraphics[width = 7.0in]{./image/addedImages/BestDim2Semantic.eps}
%\caption{The semantic of SVD dimensions. SVD second best dimension selected by our model.
%The leading movies in this dimension range from drama, horror, love, success, and city-dweller type of movies (on the left) to drama, love, life after war, finance struggle, and country type of movies.
%The contrast between both sides is very clear. 
%}
%\label{svdSemantic}
%\end{figure*}

\subsection{Model Validation}

We validate LSM with a real dataset from two aspects.
First, we test LSM on all the users in the MovieLens 100K dataset~\cite{MovieLens100k}.
For all the results in this paper, we use 4 for $\tau_+$, 2 for  $\tau_-$, 5 for $w_+$, and 10 for $w_o$ and $w_{\theta}$.
The best latent dimensions are automatically selected and used to measure how well LSM can distinguish the like and dislike regions.
The results show that the best dimensions for all the users contain ideal group distributions in either case 1 or case 2 of Figure~\ref{svd} (a), indicating that LSM can be applied to users with various rating histories.

Second, we observe the distributions of recommendable movies on the best latent dimensions.
Since every recommendable movie can be in the search results, we collect the total amount of $b_{ui}$ for all movies in a group. 
It is worth mentioning that the dislike region may also contain recommendable movies, as the choices of recommendation come from multiple aspects. % - different combined movie features / dimensions in our work.
Also, cases like two users rated one movie similarly while rating another movie very differently can complicate the statistical distributions.
As shown in the table~\ref{tablevalid}, the average in $R_+$ of all users is significantly higher than that of $R_-$.
If we remove the overlapping regions from $R_+$ and $R_-$, the difference is more significant. 
We also compute the Pearson correlation to compare the values from two pairs in statistics.
By removing the overlapping regions, the second pair is very close to no correlation.
This result shows that LSM captures the majority recommendable movies in the like region for making recommendations.

\vspace{-4ex}
\begin{table}[htb]
\caption{Comparison of the sum of $b_{ui}$ in different regions}
\vspace{1ex}
\label{tablevalid}
\centering
\begin{tabular}{| c | c | c | | c | c | c |}
\hline
$\sum_{i \in R_+}$ & $\sum_{i \in R_-}$ & Pearson & $\sum_{i \in R_+ \bigcap \overline{R_o}}$ & $\sum_{i \in R_- \bigcap \overline{R_o}}$ & Pearson \\
\hline
151.0 & 101.7 & 0.54 & 62.5 & 13.1 & -0.05\\
\hline
\end{tabular}
\end{table}
\vspace{-2ex}

  % Sum Bui Like: 38209.748244556 
  % Sum Bui Like Without Overlap: 15805.6755926232 
  % Sum bui Dislike: 25722.6828700504 
  % Sum Bui Dislike Without Overlap: 3318.61021811758 
  % Sum Bui Overlap: 22404.0726519328 
  % Avg Std Like: 0.0411774222792137 
  % Avg Std Dislike:  309160143995338 
  % Average Bui Like: 0.802955912706401 
  % Average Bui Dislike: 0.772311416433231

%Specifically, we measure whether the distributions of $b_{ui}$ degrees among the $R_+$ and $R_-$ zones are consistent with the like and dislike groups.
%We expect that majority recommendable movies with high $b_{ui}$ degrees should be inside $R_+$ instead of $R_-$.
%For all the users, we randomly sample dimensions with different $D_v$ values and compute the average $b_{ui}$ degrees of the top-K recommendable movies.
%As shown in Figure~\ref{validation}, the averages of $b_{ui}$ are high for $R_+$ and low for $R_-$, indicating our model is consistent with movie distributions on the latent dimensions.
%The result also shows that this feature is preserved for dimensions with high $D_v$ values, which often have small singular values.
%This is because dimensions with large singular values capture statistical features of the entire dataset; therefore features for individual users are often found in minor dimensions.

%\begin{figure}
%\caption{The correlations among the averages $b_{ui}$ of the top-K recommendable movies in $R_+$, $R_-$, $D_v$, and singular values validate our latent semantic model.}
%\label{validation}
%\end{figure}

%%The second approach is that similar users with high $s_{uv}$ have the same SVD dimensions selected (high $D_v$ values).
%We also validate the semantic features of latent dimensions for representing user features.
%For each user in the dataset, the dimension with the highest $D_v$ value is used to represent the combined features of movies for the user, including both liked and disliked movies.
%One latent dimension may be selected for multiple users.
%To judge whether the users have similar tastes of movies when they share the best latent dimension, we measure the average coefficient $s_{uv}$ in equation~\ref{similarity} of all user pairs with the same best latent dimension.
%The largest $s_{uv}$ is mapped to $1$.
%The average is ??, much higher than the average of the all $s_{uv}$ values.


%\input{visualization}
\section{Interactive Recommendation through Storytelling}

We start this section by describing how we connect the process of recommendation to storytelling.
Then, we present our strategies to construct recommendation stories automatically through the LSM from the aspects of characters, roles, user interaction and narrative structures respectively.


\subsection{Connecting Recommendation to Storytelling}

As described in the introduction, we propose interactive recommendation to simulate the scenario of an expert recommending movies to a user.
During recommendation, the expert generally presents one or several movies and provides reasons of recommendation, such as high popularity, high similarity to a movie the user likes, or special features.
The user may respond by indicating his or her preferences on the recommended movies, such as ``recommend more movies like this" or ``no more movies like that".
This process continues until the user finds an interesting movie to watch.

To simulate this communication process, we present a storytelling mechanism that treats the procedure of recommending movies as telling a story.
We design a \textbf{recommendation story} as a set of recommendable movies and brief reasons of recommendation and an \textbf{interactive recommendation} approach as a continuous storytelling process, which can be adjusted promptly with user interaction.
As shown in Figure~\ref{IT}, the interactive storytelling pipeline starts with exploring movie database for a user, selecting recommendable movies, and collecting necessary information using LSM described in Section 3.
Then, the second step of ``make a story" is to generate a recommendation story automatically with the approach described in this section.
The third step ``tell a story" is to present a story as an animation sequence described in Section 5.

The loop among ``make a story", ``tell a story", and ``user" provides the proposed continuous and interactive process of recommendation until a desirable movie is identified.
The arrow from ``user" to ``make a story" indicates that the user can provide feedback to request new recommendation stories that reflect user inputs.
The arrow from ``user" to ``tell a story" indicates that the user can  adjust the storytelling animation interactively, such as replaying a recommended movie or finishing the story immediately.
If the user does not provide an input, the loop continues to different recommendation stories to achieve the effect of ``how about some other types of movies you may like?"
The automatic switch between different recommendation stories can avoid users getting bored from similar movies.
During the process, the user can also use our interaction tools to explore additional information of movies.

Different from the previous storytelling processes~\cite{7274435} that are mainly an ordered sequence of the three components - explore data, make and tell a story, our interactive recommendation approach is supported by 1) interaction functions that allow a user to interact with the storytelling pipeline at any time during the process, 2) automatic construction of narrative structure that allows new and adjusted recommendation stories to be generated continuously.


\begin{figure}
\centering
\includegraphics[width = 3in]{flow.eps}
\caption{Pipeline of storytelling with continuous communication between the user and our interactive recommendation system.
Users can interact with the system at any time to indicate preferences and explore movie information to accelerate recommendation process.
}
\label{IT}
\end{figure}

\subsection{Character}

The character in an ordinary story is generally a person. 
In our recommendation stories, the characters are movies.
Similar to the human characters, each movie character maintains a different relationship with the user, such as a rated movie, a favorite movie, or a recommendable movie.
The movie characters are also related to each other, such as being rated by the same users or have similar rating averages.
Since our focus is recommendation, the leading characters are recommendable movies, whose relationships with the user can be represented with LSM and recommendation degrees.

\subsection{Role}

A role describes what function each character serves in the story.
The roles of recommendable movies provide a mechanism for users to explore the movie database. 
Before going through the movie details, the roles of a movie provide a quick catch of the movie features, such as a typical drama movie which is very similar to one of the user's favorite movies.
This provides the user a quick way to find several interested movies to explore.
 

In recommendation stories, we use the semantic concepts from LSM to describe the roles, such as a ``liked" movie that the user rated high or a ``typical" movie that has strong features of certain movie genres.
Each movie character can play multiple roles, such as familiarity and un-typicality, just like a human character can be both a colleague and a friend. 
The actual roles of a movie character are determined by the locations among the semantic zones on selected latent dimensions, as shown in Figure~\ref{svd}.


\subsection{User Interaction During Recommendation}

For interactive recommendation, the user interaction becomes an important component of the storytelling pipeline.
To allow active user interaction with the storytelling pipeline shown in Figure~\ref{IT}, we provide three groups of explicit interaction functions as follows:
%we provide interaction functions for adjusting the types of recommended movies with slide bars between two sets of user-friendly concepts.

The first group of interaction functions is from ``user" to "make a story". 
Corresponding to the roles of a movie, users can specify the preferred movie types between options of familiar ($f$) / diverse and typical ($t$) / untypical movies.
The parameter values become effective immediately on generating the new recommendation stories.
%We provide two slide bars for this function.
For a specific movie, the user can click thumbs-up (like) and thumbs-down (dislike) buttons, so that the specified movie is moved to the like or dislike groups (and the group of not recommendable, so that it is removed from the recommendation process).
We also increase the $w_i$ value (a parameter to control the effects of user selection) of the movie in equations~\ref{equ:dv} and ~\ref{equ:ds} for choosing latent dimensions for new stories.
For each dimension $v$, the new measurement $D'_v$ contains components from both data distribution by $D_v$ and user interaction. 
We detect if the user preference is aligned with LSM, especially if a liked movie is in the like range and if a disliked movie is in the disliked range.
\begin{equation}
%D'_v = D_v  + w_{ui} \times (\sum_{i \in liked \& i \in R_l} {b_{ui}} - \sum_{i \in liked \& i \in R_d} {b_{ui}} + \sum_{i \in disliked \& i \in R_d} {b_{ui}} - \sum_{i \in disliked \& i \in R_l} {b_{ui}})
D'_v = D_v  + w_{i} \times (\sum_{i \in (U_+ \cap R_+) \cup (U_- \cap R_-)} {b_{ui}} - \sum_{i \in (U_+ \cap R_-) \cup (U_- \cap R_+)} {b_{ui}})
\end{equation}
where $U_+$ and $U_-$ are the sets containing all liked or disliked movies specified by the user.
%a recommendable item $i$ is liked or disliked by a user (when a user clicks on like up or dislike button) by $i \in U_+$ or $i \in U_-$ respectively.

The second group of interaction functions is from ``user" to "tell a story". 
To control the animated storytelling, the users can replay, pause, and stop the current recommendation story, or play more stories (the default is continuing to recommend additional movies).

The third group is to explore movie details.
If finding an interesting movie, the user can click the movie poster or movie node on the interface to view details.
The users can also mouse over a movie node anytime to reveal a set of basic information, including user rating, average rating, popularity, title and genres.


%The importance of using latent space from SVD is that we can transfer the above user interactions into the semantic meanings related to the movie recommendation.
%Both the recommendation results and storytelling visualization are adjusted automatically during the interactive process.


\subsection{Automatic Generation of Narrative Structures}
%\subsection{Plot - Latent Dimensions}

A narrative structure refers to the sequence of events in a story.
In recommendation stories, each event is a recommendable movie and brief reasons of recommendation. 
The sequence of a set of recommendable movies in a narrative structure is crucial to improve users' understanding of the movies effectively.

Considering the short attention of users in online recommendation, we prefer simple stories that can finish in a very short duration. Therefore, we generate each recommendation story only with one latent dimension selected with LSM.
As each latent dimension reflects one combined movie / user feature, such recommendation stories simulate the effect that we recommend movies from one combined movie feature to another, such as popular drama/comedy movies to unpopular documentary movies.
Other designs of recommendation stories using our LSM are also feasible.
For example, long stories can be generated by connecting different latent dimensions. 
Due to the focus of our interactive recommendation approach, we only use short stories in this work.

Based on a latent dimension, we try to maintain smooth story transitions by generating linear narrative structures among the four semantic concepts: familiarity, diversity, typicality, and un-typicality.
This is achieved by identifying a starting point, choosing narrative structures based on user preferences, and selecting recommendable movies.

The \textbf{starting point} of a story is determined according to user preferences of $f$ and $p$.
The default values are $0.5$ for both $f$ and $p$, although we favor typical over un-typical and familiar over diverse movies, which is consistent with the preferences of majority users.
The following list is the order we set as default.

\vspace{+1mm}
\noindent {High familiarity:} starting from the like group

\vspace{+1mm}
\noindent {High diversity:} starting from the diverse zone closer to like group

\vspace{+1mm}
\noindent {High typicality:} starting from the typical zone closer to like group

\vspace{+1mm}
\noindent {High un-typicality:} starting from the un-typical zone
\vspace{+1mm}

The \textbf{order} of recommended movies also considers user preferences.
As shown in Figure~\ref{path}, we use the four narrative structures
to cover all combinations of the two user preferences of $f$ and $t$.
The narrative structures are designed to be linear sequences, so that users can expect very similar narrative visualization during the interactive recommendation process.
Since the narrative structure is on one latent dimension each time, two zones from the LSM are involved.
The ranges of the latent dimension to select recommended movies are also shown in Figure~\ref{path}.
During the interactive storytelling process, we switch narrative structures between the two options randomly to avoid simply repeated stories.

%There is a transition when switching dimensions.

\begin{figure}
\centering
\includegraphics[width = 2.5in]{path1.eps}
\includegraphics[width = 2.5in]{path2.eps}\\
\caption{The four linear narrative structures based on the user preferences of t -- typical, u=1-t -- un-typical, f -- familiar, and d=1-f -- diverse. 
%The ranges of latent dimensions are marked with the blue parentheses. 
%For narrative structures for typical/un-typical preference, we use the half of latent dimension that overlaps with majority of the liked group.
}
\label{path}
\end{figure}

The \textbf{selection of recommended movies} is based on the sample rates computed according to user preferences of $f$ and $t$.
Assume a set $G_T$ of $T$ recommended movies is selected for each narrative structure.
When the structure is between typicality and un-typicality, we determine the number of recommended movies from the typical zone as $s_t$ and un-typical zone as $s_{u}$:
\begin{equation}
\centering
s_t = \lceil t * T \rceil; \ \ \ \ \ 
s_{u} = T - s_t
\end{equation}
Similarly, when the structure is between familiarity and diversity, the sampling numbers for familiar zone $s_f$ and diverse zone $s_d$ are: 
\begin{equation}
\centering
s_f = \lceil f * T \rceil; \ \ \ \ \ 
s_d = T - s_f
\end{equation}

Inside each zone, we select recommended movies with the following procedure that is composed of a local sampling and a random test procedure.
We first randomly pick a location $l$ and choose the best candidate within a local window by combining all three factors: the recommendation degree $b_{ui}$ of a candidate movie $i$, the distance of movie $i$ to location $l$, and the similarity of movie $i$ to the set of $q$ movies that are specified by the user.
On the latent dimension $V_p$, assume the location of movie $i$ is $V_p(i)$.
The effect of a user specified movie is set to be within a location window $\delta$ with a truncated Gaussian function $G_q()$, with high weights for thumb-up movies and low weights for thumb-down movies.
%The $D_u$ is a field generated according to user interaction.
%It is for emphasizing the results made during the interaction. 
%We use two shapes to differentiate movies that are liked or disliked by the user.
The movie $i$ with the highest value from the following combined measurement is selected as the best candidate.
\begin{equation}
b_{ui} \times \overbrace{ G_1(| V_p(u_1) - V_p(i) |) \times ... \times G_q(| V_p(u_q) - V_p(i) |)}^{q = | U_+ \cup U_- |} / | V_p(i)- l |
\end{equation}

%\begin{figure}
%\centering
%\includegraphics[width = 1.3in, height = 0.4in]{./structure/like.eps} \ \ \ \ \
%\includegraphics[width = 1.3in, height = 0.4in]{./structure/dislike.eps}
%\caption{Functions used to emphasize user inputs for thumb-up and thumb-down.}
%\label{up-down}
%\end{figure}

The purpose of an additional random test is to ensure that our selections of recommendable movies are consistent with both user preferences of $f$ and $t$, although only one factor is used to determine the narrative order.
If the selected movie $i$ passes the random test, we add it to the selected set $G_T$; otherwise we pick another random location and perform the local sampling again.
For example, for the narrative structure between familiarity and diversity, we try to maintain an average typicality value close to the user preference $t$.
We measure the typicality value of a movie $i$ as $t(i) = | V_p(i) |$.
The test is determined by if the movie $i$ can make the average typicality value closer to $t$.
\begin{equation}
| \frac{(\sum_{i \in G_T}{t(i)}) + t(i)} {| G_T | + 1} - t | < | \frac{\sum_{i \in G_T}{t(i)}} {| G_T |} - t |
\end{equation}
Similarly, for the narrative structure between typicality and un-typicality, we measure the familiarity value of a movie $i$ given the center location of the like group $c_+$ as $f(i) = | V_p(i) - c_+ |$.
The random test is performed by replacing the $t(i)$ with $f(i)$ in the above equation.

Since the movie database is generally large, we can assume that there are always enough movies to recommend.
In the cases that the recommendable movies run out, we can adjust the parameters $\tau_+$ and $\tau_-$ to include additional movies.

%%!TEX root = ../main.tex

\section{System Overview} \label{sec:system}
\begin{figure}
% \hspace{-.8cm}
    \center{\includegraphics[scale=0.19]
          {figures/architecture.pdf}}
            %   \vspace{-.2cm}
    \caption{The overall workflow of \system.}
    \label{fig:architecture}
\end{figure}
Figure~\ref{fig:architecture} depicts the abstract workflow of our proposed system, \system. %It consists of two main phases: the {\em offline} phase and the {\em online} phase.
In the offline phase, \system builds an inverted index structure for the tables in the corpus. 
In the online phase, i.e. discovery phase, \system uses the inverted index to find the top-k joinable tables among all candidate tables for a given input table. 

% \vspace{0.1cm}
\noindent\textbf{Indexing step (Offline).} To efficiently discover tables that are joinable to a given table, we propose an extension to the state-of-the-art single-value inverted index to efficiently serve the multi-attribute applications.
We introduce an additional index element, which is an aggregated hash value called \emph{super key}.
The \emph{super key} is space-efficient and does not change the nature of the single-attribute inverted index while serving the purpose of multi-attribute join discovery.
The super key is used to prune irrelevant tables and rows to reduce the post-processing overhead. It aggregates the values inside each table-row into a fixed-size entry.

It is worth noting that the super key element does not require any knowledge of the actual keys of a table and serves all possible key combinations inside a table.
This way, for a given query dataset and composite key, the system can decide in a single operation if the row is a candidate joinable row or not.
Furthermore, the filter does not result in any false negatives.
To generate the super key entries, \system leverages a novel hash function \hash that encodes each row value in a highly disseminative way and aggregates them to the super key. The hash function, as well as the super key, will be discussed in more detail in Section~\ref{sec:index}.

% \vspace{0.1cm}
\noindent\textbf{Discovery step (Online).} In the actual data discovery scenario, the user provides a dataset with a composite key and a parameter $k$, expecting the system to find top-$k$ tables with the highest number of equi-joinable rows to the given input dataset.
\system enables the efficient n-ary key joins by using the super key entry to prune as many irrelevant tables as possible before the joinability calculation.
This process undergoes four phases: \textit{(i)}~initialization, \textit{(ii)}~table filtering, \textit{(iii)}~ row filtering, and \textit{(iv)}~the joinability calculation.

In the initialization step, first, the system selects an initial query column from the composite key \textit{Q} based on a simple cardinality-based heuristic. The goal of the initial query column is to reduce the number of fetched tables to the minimum by picking the column that leads to the smaller number of PL items from the corpus. 
That is why \system selects a single column to fetch the initial set of candidate tables from the corpus.
The column selection can be supervised and preempted by the user.
Then, the system generates the super key entry for the join columns of the input dataset, i.e.,~generating a hash-code for each key value combination and aggregating them into a single hash. 
In the filtering step, \system applies two levels of pruning for each table: table-level and row-level. 
With the table-level pruning, \system decides whether the current table is a promising table to be one of the top-k tables.
With the row-level pruning, \system checks for non-joinable rows in the candidate table. It compares the super keys of the input dataset with those inside the candidate table to drop irrelevant rows from further joinability verification.
Finally, \system retrieves the exact values from the table corpus to compute the final joinability score $\jmath$ for the remainig tables and rows. After calculating the $\jmath$, the set of top-$k$ tables is updated and the next candidate table undergoes the pruning steps.
We detail the online phase in Section~\ref{sec:index_applicaiton}.


\section{Interactive Recommendation System}

The interface of our system is designed to be consistent with popular commercial recommendation systems, such as Youtube, Netflix and Amazon Movie. 
As shown in Figure~\ref{interface}, our system contains three common components -- an enlarged movie poster with information (a), the list of top-N recommended movies (c), and additional information at the bottom (e).
We add a small area for user interaction (b) and a visualization domain (d) to support interactive recommendation and exploration of movie database.
Our system also supports the following features of interactive recommendation.
%In the blue component of additional information, we show the watch history as it is a useful hidden function in the example systems.

%Although only the Amazon video website provides watch history (at the end of the page), the watch history can always be found with the ``back" button during web browsing.
%We can also add additional information in the order of importance to the blue components for details.
%Note that the interaction functions provided by Amazon is for purchasing and adding to wishlist; while we provide interaction functions for adjusting the types of recommended movies with slide bars between two sets of user-friendly concepts.

\begin{figure}[htb]
\centering
\includegraphics[scale=0.28]{Interface.EPS}
\caption{The interface of our recommendation system.
Similar to online movie recommendation systems, our interface includes (a) basic information of a selected movie, (c) a list of recommendable movies, and (e) watch history.
We also add (b) for explicit input of user preferences and (d) visualization for exploring recommendable movies.
%The top band with black background shows detailed information (a) about the current recommended movie (the current story piece of the recommended movies) that is describing to the user. On the same band, we have an interactive panel (b) that allows the user to adjust his or her recommendation preferences. The recommended list (c) and the storyline (d) are used to narrate and explain to the user why a movie is recommended based on his or her rating history. 
%This storyline (d) describes a user taste from drama and romance genres that the user likes toward the left to comedy type that the user dislikes toward the right. (e) shows the recommended movies the user selected from the recommended list.
}
\label{interface}
\end{figure}
%we can describe the combined feature on the best dimension as from comedy/romance (on the left) to drama/biography/music (on the right). 



%\subsection{Narrative visualization}


%The narrative visualization illustrates a recommendation story with the order of recommended movies and the role of each movie on the plot is indicated by locations.
%We use the locations of movies on the latent dimensions to determine the order and role of each movie.

\subsection{Example-based Narrative Structure}

The visualization of the latent dimension plays an important role in illustrating the recommendable movies and the movie distributions.
To describe the narrative structure visually, we design an example-based approach shown in Figures~\ref{svdSemantic}, \ref{svdexample}, \ref{interface}, which uses two movies that have been watched by the user, one on each side of the familiar zone, to describe the combined features of that latent dimension.
Since the combined features are generated simply from data statistics, they generally cannot be described with languages or equations.
The example movies help the users to understand the overall trend of the latent dimension and provide a quick impression of a new movie by the distances to the two examples.
All the recommendable movies are located on the 2D domain of latent dimension and recommendation degree, providing additional visual cues for recommendation reasons.


\subsection{Multi-Level Visualization}

The reasons to recommend a movie can be multifold. 
While the reasons supported by LSM mainly come from the aspects of statistical similarity and semantic analysis, we can provide a variety of information as the ``brief" reasons of a recommendation, such as a typical drama movie in the familiar zone of a user, which is similar to an example movie shown with the poster.
%Theoretically we can add additional information to the system, especially after the watch history on the detailed panel (e), if we allow long recommendation stories.
We organize the information on the visualization domain at three levels, so that the storytelling animation can follow the levels to achieve the effect that additional information is introduced gradually.
Users can stop at any time if they are not interested in the movie and continue to the next recommendation.

As shown in image (i) of Figure~\ref{animation}, the level one provides the basic information of a latent dimension with the liked and disliked regions, where the nodes of movies being recommended are drawn as circles.
The nodes of movies that the user has watched are colored green and the nodes of recommendable movies are colored based on their genres.
Two example movies are provided for illustrating the latent dimension and making correlations to other movies.

%the relative locations of the recommended movies with the others.  This level provides the SVD semantic clustering of the movies. The closer a movie is to others, the more similar they are. The SVD clusters are then grouped based on the user rating history. For example, the SVD clusters from which the user liked movies are grouped together by shading their location background in green in the storyline. This gives the user the ability to link the recommended movies to those he or she already liked.

The level two introduces the recommendation degrees to the vertical locations of recommendable movies. As shown in images (j) and (k) of Figure~\ref{animation}, the movies with higher recommendation degrees are placed on the top to attract user attention. 
The scaling can be automatically done in an animation or interactively adjusted by the user when exploring the movie database.


The level three provides the richest information for exploring movies.
For the movie being recommended, we reveal the top four similar movies that the user has watched and liked to strengthen the reasons of recommendation, as shown in images (l) and (m)  of Figure~\ref{animation}.
All the four movies have high user ratings and they are close to the movie being recommended on the latent dimension.
We also color all the movie nodes based on the genres and add colored links to connect posters to the movie nodes.


%Should the links to the five movies be provided on all the levels? They are not just about the current movie.



\begin{figure*}[htb]
%\centering
\includegraphics[width = 1.64in]{componentAnimation1.EPS} \ \ \
\includegraphics[width = 1.64in]{componentAnimation2.EPS} \ \ \
\includegraphics[width = 1.64in]{componentAnimation3.EPS} \ \ \
\includegraphics[width = 1.64in]{componentAnimation4.EPS}\\ 
 \ \ \ \ \ \ (a) \ \ \ \ \ \ \ \ \ \ \ \ \ \ \ \ \ \ \ \ \ \ \ \ \ \ \ \ \ \ \ \ \ \ \ \ \ \ \ \  \ \ \ \ \ \ \ \ \ \ \ \ (b) \ \ \ \ \ \ \ \ \ \ \ \ \ \ \ \ \ \ \ \ \ \ \ \ \ \ \ \ \ \ \ \ \ \ \ \ \ \ \ \  \ \ \ \ \ \ \ \ \ \ \ \ (c) \ \ \ \ \ \ \ \ \ \ \ \ \ \ \ \ \ \ \ \ \ \ \ \ \ \ \ \ \ \ \ \ \ \ \ \ \ \ \ \  \ \ \ \ \ \ \ \ \ \ \ \ (d) \\ 
\includegraphics[width = 1.64in]{dimAnimation1.EPS} \ \ \
\includegraphics[width = 1.64in]{dimAnimation2.EPS} \ \ \
\includegraphics[width = 1.64in]{dimAnimation3.EPS} \ \ \
\includegraphics[width = 1.64in]{dimAnimation4.EPS}\\
  \ \ \ \ \ \ (e) \ \ \ \ \ \ \ \ \ \ \ \ \ \ \ \ \ \ \ \ \ \ \ \ \ \ \ \ \ \ \ \ \ \ \ \ \ \ \ \  \ \ \ \ \ \ \ \ \ \ \ \ (f) \ \ \ \ \ \ \ \ \ \ \ \ \ \ \ \ \ \ \ \ \ \ \ \ \ \ \ \ \ \ \ \ \ \ \ \ \ \ \ \  \ \ \ \ \ \ \ \ \ \ \ \ (g) \ \ \ \ \ \ \ \ \ \ \ \ \ \ \ \ \ \ \ \ \ \ \ \ \ \ \ \ \ \ \ \ \ \ \ \ \ \ \ \  \ \ \ \ \ \ \ \ \ \ \ \ (h) \\ 
\includegraphics[width = 1.64in]{recAnimation10.EPS} \ \ \
\includegraphics[width = 1.64in]{recAnimation11.EPS} \ \ \
\includegraphics[width = 1.64in]{recAnimation12.EPS} \ \ \
%\includegraphics[width = 1.64in]{recAnimation13.EPS}\\
\includegraphics[width = 1.64in]{recAnimation14.EPS}\\
 \ \ \ \ \ \ (i)  \ \ \ \ \ \ \ \ \ \ \ \ \ \ \ \ \ \ \ \ \ \ \ \ \ \ \ \ \ \ \ \ \ \ \ \ \ \ \ \  \ \ \ \ \ \ \ \ \ \ \ \ (j) \ \ \ \ \ \ \ \ \ \ \ \ \ \ \ \ \ \ \ \ \ \ \ \ \ \ \ \ \ \ \ \ \ \ \ \ \ \ \ \  \ \ \ \ \ \ \ \ \ \ \ \ (k) \ \ \ \ \ \ \ \ \ \ \ \ \ \ \ \ \ \ \ \ \ \ \ \ \ \ \ \ \ \ \ \ \ \ \ \ \ \ \ \  \ \ \ \ \ \ \ \ \ \ \ \ (l) \\ 
%\includegraphics[width = 1.64in]{recAnimation14.EPS} \ \ \
\includegraphics[width = 1.64in]{recAnimation15.EPS} \ \ \
\includegraphics[width = 1.64in]{recAnimation21.EPS} \ \ \
\includegraphics[width = 1.64in]{recAnimation51.EPS}\\
  \ \ \ \ \ \ (m) \ \ \ \ \ \ \ \ \ \ \ \ \ \ \ \ \ \ \ \ \ \ \ \ \ \ \ \ \ \ \ \ \ \ \ \ \ \ \ \  \ \ \ \ \ \ \ \ \ \ \ \ (n) \ \ \ \ \ \ \ \ \ \ \ \ \ \ \ \ \ \ \ \ \ \ \ \ \ \ \ \ \ \ \ \ \ \ \ \ \ \ \ \  \ \ \ \ \ \ \ \ \ \ \ \ (o)\\ 
\caption{Example snapshots of the storytelling animation. 
We start by introducing the system interface to a new user, such as the use of rating history (a), color code (b), recommendation degree (c) and the liked zone (d).
We also use example movies to illustrate the latent dimension and attract user attention with poster transitions (e-h).
Next, we introduce the first recommended movie by animating the movie node and showing a green line under its poster (i).
The visualization domain is animated from level one (i), level two (j-k), to level three (l-m) gradually.
The same procedure (i-m) is used to animate the second (n) to the last (o) recommended movies respectively.
%The third and forth rows images (i-n) show animated visualization of the first recommended movie. Its is highlighted by a green line under its poster while its node height is gradually animated from zero to its recommendation degree (i-k). Its node is then enlarged (l). (m-n) show the animation of its top four related movies. (o) and (p) show the end of the story of the second and fifth recommended movies respectively.
}

%The visualization of latent dimension is described with  movie ``Airhead'' the user liked (on the left) and movie ``Weekend at Bernie's'' the use disliked (on the right) (e-h).


%(a) Zone highlighting. Like zone is highlighted accompanied by a descriptive message, (b) Interpreting the Storyline Semantic using animation. Example movie node and poster animating (on the right) to describe the storyline with movies familiar to the user. and (c)  Explaining a recommended movie with user liked top four movies, the related movie nodes are enlarged while their posters are rotated click-wise.}
\label{animation}
\end{figure*} 





\subsection{Animation Effects}

Our system provides fully automatic animations to ``tell" a recommendation story with the following three sets of animation effects.

The first animation set is to introduce the system components. 
Step by step, each component is highlighted with a brief description as shown in images (a-d) in Figure~\ref{animation}. 
%The visualization of latent dimension is also presented in details.
%Each zone is highlighted with a boundary and its meaning is described with a short message. 
%For example, image (d) of Figure~\ref{animation} shows the liked zone highlighted.
The user can replay or skip to the next animation set anytime.
 
The second animation set is to present the narrative structure of a recommendation story.
We use our example-based approach to animate the recommended movies for user attention and the two example movies from left to right for providing the impression of dimension meanings.
Specifically, each example movie poster is animated by changing its size and the movie node is also animated simultaneously, as shown in  Figure~\ref{animation} (e-h).

%Specifically, the storyline is described by flashing nodes that represent the example movies in the storyline visualization, while their posters are briefly animated by reducing their sizes.
%One of the drawbacks of animation in visualization is the human difficulty to focus on many moving things simultaneously. For this reason, we used a sequential ordering for the sets of animation without overlapping them. 

The third animation set is to present a recommendation story, as shown in images (i-o) in Figure~\ref{animation}.
We start to highlight the selected movie by flashing the movie node in the visualization panel when its poster in the recommendation list is animated.  
A green line under the movie poster also indicates the focused movie. 
Then, we switch the visualization from level one to level three gradually to provide detailed information of a selected movie.
Specifically, after the level one is shown, the movie nodes in the visualization panel are progressively scaled to their maximum recommendation degree in the level two.
From level two to level three, the set of similar movies are animated.
We use the same animation procedure for all the selected movies in the story to avoid confusion.
The user can observe all the information or switch anytime to the next recommended movie.

When the animation of a recommendation story ends or the user selects a movie for further exploration, a new story is generated automatically.
The system re-introduces the new narrative structure and repeats the animation sequences to convey the story behind the new selected movies.

%Similar to the interaction mantra - overview first and details on demand, we start with providing an overview of the group of recommended movies with examples. We call this presentation phases.  


%Additional visual indicators for improving user understanding are also provided. 
%First, we highlight the movie under recommendation with a blue window and demonstrate its location on the story outline with a flash effect.
%Second, a progress bar is provided so that users know exactly where they are during a story.
%The progress bar is located right between the recommendation list in green and the storyline panel on the interface.

% generate more steps to provide a more complete view of the animation
% if out of space, the multilevel figure can be combined here.

%% obs-noise = 0.05, derivative-obs-noise = 0.2
\begin{tabular}{llll}
\toprule
            & HIP-GP & SVGP   & Exact GP \\
\midrule
RMSE        & 0.0192 & 0.0192 & 0.0192 \\
Uncertainty & 0.0198 & 0.0206 & 0.0198   \\
\bottomrule
\end{tabular}


\iffalse
% obs-noise = 0.05, derivative-obs-noise = 0.03
\begin{tabular}{llll}
\toprule
            & HIP-GP & SVGP   & Exact GP \\
\midrule
RMSE        & 0.0165 & 0.0165 & 0.0165   \\
Uncertainty & 0.0167 & 0.0175 & 0.0167   \\
\bottomrule
\end{tabular}


% obs-noise = 0.05, derivative-obs-noise = 0.1
\begin{tabular}{llll}
\toprule
            & HIP-GP & SVGP   & Exact GP \\
\midrule
RMSE        & 0.0173 & 0.0172 & 0.0173  \\
Uncertainty & 0.0181 & 0.0189 & 0.0181   \\
\bottomrule
\end{tabular}
\fi



\section{Results and Case Studies}

\subsection{Case Studies}

This section describes three case studies using the MovieLens100K dataset~\cite{MovieLens100k}. 
The dataset has 100K ratings from 1 to 5 and 1682 movies from different categories rated by 943 users.
We have select users with different backgrounds and rating histories to demonstrate our approach.

%We use examples to show that the recommendation results in all the case studies match the backgrounds of users.
%All the results match the backgrounds of users, such as the first user working in entertainment likes drama movies, the second user as a student is familiar with adventure movies, and the third user as an executive is picky on movie selections.

%For each user, we alter the interaction of selecting movies and adjusting preference parameters.
%For each case study, we list the movie orders in the storytelling during the interaction. 
%We also justify the selected movies and orders with description and numbers.

%\subsubsection{Users who seems to like certain genres of movies}

% Figure 1

%We identified two users that seem to like certain genres of movies and one user that rated lot movies but only liked few of them.

% User - 16: 21 year old male working in entaintainment. He lives in Staten Island, NY
\textbf{The first user} is a 21 year old male working in entertainment.
As shown in Figure~\ref{interface}, a latent dimension is identified between comedy movies toward the right and drama/romance/biography genres toward the left.
As the default preference settings of 0.5/0.5 for typicality/un-typicality and familiarity/diversity, our recommendation system starts with movies of drama genre in the familiar zone and continues to drama and romance types for diversity. 
The recommended movies are: When a Man Loves a Woman (1994, Drama and Romance), Meet John Doe (1941, Drama, Romance and Comedy), Nobody’s Fool (1994, Drama and Comedy), Ref The (1994, Comedy and Drama), and Rebel Without a Cause (1995, Drama).

We compare the recommendation results with the user's watch history.
%His rating records show a clear preference on drama or drama-romance movies over comedy types.
The user has rated 44 movies.
Among which, about 68\% of the movies are drama or a combination of drama and romance genres. Their average rating is high (4.30).
The movies of other genres received lower ratings, especially comedy movies.
Therefore, our result captures the user's preference of drama movies over comedy types.

%The story is told in the sequence of above, from left to right, from movies in the familiar zone to the movies in the variance zone.

%Most of the movies he watched (30 out of 34, about 88\%) are comedy or romance genre with 3.41 average ratings.
%The other 4 movies he rated were either a composite genre with comedy or romance, such as comedy-horror or comedy-drama-romance. 

%user 764
%The second user has rat•	When a Man Loves a Woman (1994), Drama-Romanceed 39 movies of several genres, including comedy, drama, horror, and romance. 
%His overall average rating is 3.80 and his ratings suggest that he does not like comedy and romance movies. 
%Figure~\ref{case1RecommendedList} shows our storytelling interface with several movies labeled according to the genres.
%On the storyline, the familiar zone is delimited by a comedy-romance movie he does not likes,``Pallbearer (1996)'', and the movie he likes, ``Shine (1996)'', which is of genres music, biography, and drama.
%Between the two movies example that delimit the user familiar zone, we have some mixed genres of movies such as comedy-romance-drama as indicated by the composite color of the movies title. 
%Our system automatically plays the recommendation story of a set of five movies, House Arrest (1996), The Crucible (1996), Carrie (1976), Nobody's Fool (1994), and Career Girls (1997), from familiar zone to diverse zone. 
%Without user input, our system continues to an additional story with a new set of movies as shown in the first row of Figure~\ref{case1InteractionsResults}.
%The user can adjust the preferences of typicality and familiarity or click on a recommended movie for more information (see Figure~\ref{case1InteractionsResults}).  

% %User 476: 28 year old male student living in Bolingbrook, IL


\textbf{The second user} is a 28 year old male student.
Figure~\ref{case1RecommendedList} shows the first recommendation results for the user. 
The latent dimension spreads out movies in drama, documentary and comedy genres, with the combined features of positive influence (such as love and drama) toward the left direction and negative influence (such as horror) toward the right. 
%On the storyline, the familiar zone is delimited by two comedy/drama movies ``When Harry Met Sally (1989)'' and ``Son in Law (1993)''.
Five movies are recommended from diverse to familiar zones: Wonderland (1997, Documentary), Fearless (1993, Drama), Man Without a Face (The 1993, Drama), Everest (1998, Documentary), and Miracle on 34th Street (1994, Drama).

Without user input, our system continues to an additional story with a new set of movies as shown in the first row of Figure~\ref{case1InteractionsResults}.
The user can also adjust the recommendation preferences and select their interested movies, shown on the second and third rows of Figure~\ref{case1InteractionsResults}.  

We compare the recommendation results with the user's watch history.
He has an average rating of 3.64 for 34 movies from comedy, drama, horror, and romance genres.
The rating records suggest that he likes drama and romance movies. 
Our recommendation results recommend majority movies in the familiar zone of the user and also introduce mixed adventure/documentary/drama movies for diversity.

\begin{figure}[htb]
\centering
\includegraphics[width = 3.3in]{case1RecommendedList1.EPS}
%\includegraphics[width = 7in]{./image/addedImages/case1RecommendedList2.eps}
\caption{The result of the second user describes the user preference of love/drama movies he liked toward the left to horror/comedy movies he disliked toward the right.}
\label{case1RecommendedList}
\end{figure}
%we can describe the combined feature on the best dimension as from comedy/romance (on the left) to drama/biography/music (on the right). 


\begin{figure}[htb]
\centering
\includegraphics[width = 3.3in]{case1RecommendedList2.EPS}\\
\vspace{+1mm}
\includegraphics[width = 3.3in]{case1FamiliarRecommendedList.EPS}\\
\vspace{+1mm}
\includegraphics[width = 3.3in]{case1TypicalDiverseRecommendedList.EPS}
\caption{Interaction examples for the second user.
The first row shows another story piece describing a different aspect of user's preference on Groundhog Day (1993, Comedy and Romance) over Son in Law (1993, Comedy).
The second row shows that all recommended movies come from the familiar zone when the user switch the familiar preference to 1.
The bottom row shows the result when both the typical and diverse preferences are set to 1 and the movie ``One Fine Day (1996)'' from the recommended list is selected.
%and select the movie ``Dangerous Minds (1995)'' from the recommended list in the result of the second row image.  
}
\label{case1InteractionsResults}
\end{figure}

%\subsubsection{User who rated lot movies and like very few of them}
% User 181: 26 year old male executive living in Baltimore, MD
\textbf{The third user} is a 26 year old male executive.
%His average rating in the dataset is 1.71. 
%The only movies he rated 5 starts are the comedy movie ``Birdcage, The 1996'' and the drama-romance movie ``Jerry Maguire (1996)''.
As shown in Figure~\ref{case2RecommendedList}, the latent dimension is delimited by drama-comedy movie ``Sabrina (1996)'' with a high rating on the left and the comedy movie ``Down Periscope (1996)'' with a low rating on the right. 
Our system recommends five movies from drama movies combined with other genres in the familiar zone, such as Cool Hand Luke (1967, Comedy and Drama), Philadelphia Story (1940, comedy romance), Private Part (1997, Comedy and Drama), Sophie’s Choice (1982, Drama and Romance) and Friday (1995, Comedy and Drama).
%familiar to d Storyiverse types - Now and Then (1995), Dangerous Minds (1995), Higher Leaning (1995), Quiz Show (1994), and Philadelphia (1993). 

We compare the recommendation results with the user's watch history.
He has rated 120 movies of several genres, such as comedy, drama, and horror.
Most of his ratings were 1 or 2 stars and he only rated six movies with 4 or 5 stars.
Our result reflects his low rating records with a large dislike region and a small like region.
The latent dimension describes movie types from drama-comedy movies toward the left, to various other movie genres that the user has rated low toward the right.
Such users are generally picky on movie selections, but our model still captures his favorite movie types and creates a matching recommendation story.


\begin{figure}[htb]
\centering
\includegraphics[width = 3.3in]{case2RecommendedList.EPS}
\caption{The result of the third user demonstrates a picky user who has rated many movies low.
Our approach identifies the user's preference on comedy-drama movies (toward the left) over the other genres (toward the right).
%The preferences are set to 0.5 for typical and 0.5 for familiar. 
%The story is told from familiar zone of the combined feature the user likes toward the extreme end (typical zone) of the distribution of the movies on the dimension.
}
\label{case2RecommendedList}
\end{figure}

\subsection{System Performance}

The preprocess of our system includes the computation described in Section 3, including process of rating records, construction of SVD space, selection of recommendable movies and latent dimensions for the user. 
The performance of this stage is depended on the sizes of movie database and rating records.
The preprocess takes 3-10 seconds for 30,000 ratings from 940 users for 370 movies and several minutes for the Movie100K dataset on a desktop computer with Intel Core i7 2.93 GHz processor.

During interactive recommendation stage, all the visualization, interaction, and storytelling processes described in Section 4 and 5 are interactive. 
This is essential for providing smooth user interaction in an online system.
It is achieved by only keeping the relevant data to the user during run time.
The performance is the main reason that rating records are used in our approach and many other popular online systems.
%In case additional data is included into consideration, such as the movie contents, the interactive performance of the system still needs to be maintained.

%\section{Evaluation}
\label{sec:evaluation}
\begin{table*}[!t]
\begin{center}
%\small
\caption {Benchmarks and applications for the study of the application-level resilience}
\vspace{-5pt}
\label{tab:benchmark}
\tiny
\begin{tabular}{|p{1.7cm}|p{7.5cm}|p{4cm}|p{2.5cm}|}
\hline
\textbf{Name} 	& \textbf{Benchmark description} 		& \textbf{Execution phase for evaluation}  			& \textbf{Target data objects}             \\ \hline \hline
CG (NPB)             & Conjugate Gradient, irregular memory access (input class S)   & The routine conj\_grad in the main computation loop  & The arrays $r$ and $colidx$     \\\hline
MG (NPB)    	       & Multi-Grid on a sequence of meshes (input class S)             & The routine mg3P in the main computation loop & The arrays $u$ and $r$ 	\\ \hline
FT (NPB)             & Discrete 3D fast Fourier Transform (input class S)            & The routine fftXYZ in the main computation loop  & The arrays $plane$ and $exp1$    \\ \hline
BT (NPB)             & Block Tri-diagonal solver (input class S)         		& The routine x\_solve in the main computation loop & The arrays $grid\_points$ and $u$	\\ \hline
SP (NPB)             & Scalar Penta-diagonal solver (input class S)         		& The routine x\_solve in the main computation loop & The arrays $rhoi$ and $grid\_points$  \\ \hline
LU (NPB)            & Lower-Upper Gauss-Seidel solver (input class S)        	& The routine ssor 	& The arrays $u$ and $rsd$ \\ \hline \hline
LULESH~\cite{IPDPS13:LULESH} & Unstructured Lagrangian explicit shock hydrodynamics (input 5x5x5) & 
The routine CalcMonotonicQRegionForElems 
& The arrays $m\_elemBC$ and $m\_delv\_zeta$ \\ \hline
AMG2013~\cite{anm02:amg} & An algebraic multigrid solver for linear systems arising from problems on unstructured grids (we use  GMRES(10) with AMG preconditioner). We use a compact version from LLNL with input matrix $aniso$. & The routine hypre\_GMRESSolve & The arrays $ipiv$ and $A$   \\ \hline
%$hierarchy.levels[0].R.V$ \\ \hline
\end{tabular}
\end{center}
\vspace{-5pt}
\end{table*}

%We evaluate the effectiveness of ARAT, and 
%We use ARAT to study the application-level resilience.
%The goal is to demonstrate 
%that aDVF can be a very useful metric to quantify the resilience of data objects
%at the application level. 
We study 12 data objects from six benchmarks of the NAS parallel benchmark (NPB) suite (we use SNU\_NPB-1.0.3) and 4 data objects from two scientific applications. 
%which is a c version of NPB 3.3, but ARAT can work for Fortran.
Those data objects are chosen to be representative: they have various data access patterns and participate in various execution phases.  
%For the benchmarks, we use CLASS S as the input problems and use the default compiler options of NPB.
For those benchmarks and applications, we use their default compiler options, and use gcc 4.7.3 and LLVM 3.4.2 for trace generation.
To count the algorithm-level fault masking, we use the default convergence thresholds (or the fault tolerance levels) for those benchmarks.
Table~\ref{tab:benchmark} gives 
%for->on by anzheng
detailed information on the benchmarks and applications.
The maximum fault propagation path for aDVF analysis is set to 10 by default.
%the value shadowing threshold is set as 0.01 (except for BT, we use $1 \times 10^{-6}$).
%These value shadowing thresholds are chosen such that any error corruption
%that results in the operand's value variance less than 1\% (for the threshold 0.01) or 0.0001\% (for the threshold $1 \times 10^{-6}$) during the 
%trace analysis does not impact the outcome correctness of six benchmarks.
%LU: check the newton-iteration residuals against the tolerance levels
%SP: check the newton-iteration residuals against the tolerance levels
%BT: check the newton-iteration residuals against the tolerance levels

\subsection{Resilience Modeling Results}
%We use ARAT to calculate aDVF values of 16 data objects. 
Figure~\ref{fig:aDVF_3tiers_profiling}
shows the aDVF results and breaks them down into the three levels 
(i.e., the operation-level, fault propagation level, and algorithm-level).
Figure~\ref{fig:aDVF_3classes_profiling} shows the 
%for->of by anzheng
results for the analyses at the levels of the operation and fault propagation,
and further breaks down the results into 
the three classes (i.e., the value overwriting, logical and comparison operations,
and value shadowing). %based on the reasons of the fault masking.
We have multiple interesting findings from the results.

\begin{figure*}
	\centering
        \includegraphics[width=0.8\textwidth]{three_tiers_gray.pdf}
% * <azguolu@gmail.com> 2017-03-23T03:20:28.808Z:
%
% ^.
        \vspace{-5pt}
        \caption{The breakdown of aDVF results based on the three level analysis. The $x$ axis is the data object name.}
        \vspace{-8pt}
        \label{fig:aDVF_3tiers_profiling}
\end{figure*}


\begin{figure*}
	\centering
	\includegraphics[width=0.8\textwidth]{three_types_gray.pdf}
	\vspace{-5pt}
	\caption{The breakdown of aDVF results based on the three classes of fault masking. The $x$ axis is the data object name. \textit{zeta} and \textit{elemBC} in LULESH are \textit{m\_delv\_zeta} and \textit{m\_elemBC} respectively.} % Anzheng
	\vspace{-5pt}
	\label{fig:aDVF_3classes_profiling}
    %\vspace{-5pt}
\end{figure*}

(1) Fault masking is common across benchmarks and applications.
Several data objects (e.g., $r$ in CG, and $exp1$ and $plane$ in FT)
have aDVF values close to 1 in Figure~\ref{fig:aDVF_3tiers_profiling}, 
which indicates that most of operations working on these data objects
have fault masking.
However, a couple of data objects have much less intensive fault masking.
For example, the aDVF value of $colidx$ in CG is 0.28 (Figure~\ref{fig:aDVF_3tiers_profiling}). 
Further study reveals that $colidx$ is an array to store column indexes of sparse matrices, and there is few operation-level or fault propagation-level fault masking  (Figure~\ref{fig:aDVF_3classes_profiling}).
The corruption of it can easily cause segmentation fault caught by the
algorithm-level analysis. 
$grid\_points$ in SP and BT also have a relatively small aDVF value (0.14 and 0.38 for SP and BT respectively in Figure~\ref{fig:aDVF_3tiers_profiling}).
Further study reveals that $grid\_points$ defines input problems for SP and BT. 
A small corruption of $grid\_points$ 
%change->changes by anzheng
can easily cause major changes in computation
caught by the fault propagation analysis. 

The data object $u$ in BT also has a relatively small aDVF value (0.82 in Figure~\ref{fig:aDVF_3tiers_profiling}).
Further study reveals that $u$ is read-only in our target code region
for matrix factorization and Jacobian, neither of which is friendly
for fault masking.
Furthermore, the major fault masking for $u$ comes from value shadowing,
and value shadowing only happens in a couple of the least significant bits 
of the operands that reference $u$, which further reduces the value of aDVF.
%also reduces fault masking.

(2) The data type is strongly correlated with fault masking.
Figure~\ref{fig:aDVF_3tiers_profiling} reveals that the integer data objects ($colidx$ in CG, $grid\_points$ in BT and SP, $m\_elemBC$ in LULESH) appear to be 
more sensitive to faults than the floating point data objects 
($u$ and $r$ in MG, $exp1$ and $plane$ in FT, $u$ and $rsd$ in LU, $m\_delv\_zeta$ in LULESH, and $rhoi$ in SP).
In HPC applications, the integer data objects are commonly employed to
define input problems and bound computation boundaries (e.g., $colidx$ in CG and $grid\_points$ in BT), 
or track computation status (e.g., $m\_elemBC$ in LULESH). Their corruption 
%these integer data objects
is very detrimental to the application correctness. 

(3) Operation-level fault masking is very common.
For many data objects, the operation-level fault masking contributes 
more than 70\% of the aDVF values. For $r$ in CG, $exp1$ in FT, and $rhoi$ in SP,
the contribution of the operation-level fault masking is close to 99\% (Figure~\ref{fig:aDVF_3tiers_profiling}).

Furthermore, the value shadowing is a very common operation level fault masking,
especially for floating point data objects (e.g., $u$ and $r$ in BT, $m\_delv\_zeta$ in LULESH, and $rhoi$ in SP in Figure~\ref{fig:aDVF_3classes_profiling}).
This finding has a very important indication for studying the application resilience.
In particular, the values of a data object can be different across different input problems. If the values of the data object are different, 
then the number of fault masking events due to the value shadowing will be different. 
Hence, we deduce that the application resilience
can be correlated with the input problems,
because of the correlation between the value shadowing and input problems. 
We must consider the input problems when studying the application resilience.
This conclusion is consistent with a very recent work~\cite{sc16:guo}.

(4) The contribution of the algorithm-level fault masking to the application resilience can be nontrivial.
For example, the algorithm-level fault masking contributes 19\% of the aDVF value for $u$ in MG and 27\% for $plane$ in FT (Figure~\ref{fig:aDVF_3tiers_profiling}).
The large contribution of algorithm-level fault masking in MG is consistent with
the results of existing work~\cite{mg_ics12}. 
For FT (particularly 3D FFT), the large contribution of algorithm-level fault masking in $plane$ (Figure~\ref{fig:aDVF_3tiers_profiling})
comes from frequent transpose and 1D FFT computations that average out 
or overwrite the data corruption.
CG, as an iterative solver, is known to have the algorithm-level fault masking
because of the iterative nature~\cite{2-shantharam2011characterizing}.
Interestingly, the algorithm-level fault masking in CG contributes most to the resilience of $colidx$ which is a vulnerable integer data object (Figure~\ref{fig:aDVF_3tiers_profiling}).

%Our study reveals the algorithm-level fault masking of CG from
%two perspectives. First, $a$ in CG, which is an array for intermediate results,
%has few algorithm-level fault masking (0.008\%);
%Second, $x$ in CG, which is a result vector, has 5.4\% of the aDVF value coming from the algorithm-level fault masking.
%This result indicates that the effects of the algorithm-level fault masking
%are not uniform across data objects. 

(5) Fault masking at the fault propagation level is small.
For all data objects, the contribution of the fault masking at the level of fault propagation is less than 5\% (Figure~\ref{fig:aDVF_3tiers_profiling}).
For 6 data objects ($r$ and $colidx$ in CG, $grid\_points$ and $u$ in BT, and 
$grid\_points$ and $rhoi$ in SP),  there is no fault masking at the level of fault propagation.
In combination with the finding 4, we conclude that once the fault
is propagated, it is difficult to mask it because of the contamination of
more data objects after fault propagation, and only the algorithm semantics can tolerate  propagated faults well. 
%This finding is consistent with our sensitivity analysis. 

(6) Fault masking by logical and comparison operations is small,
%For all data objects, the fault masking contributions due to logical and comparison operations are very small, 
comparing with the contributions of value shadowing and overwriting (Figure~\ref{fig:aDVF_3classes_profiling}). 
Among all data objects, 
the logical and comparison operations in $grid\_points$ in BT contribute the most (25\% contribution in Figure~\ref{fig:aDVF_fine_profiling}), 
because of intensive ICmp operations (integer comparison). %logical OR and SHL (left shifting).


(7) The resilience varies across data objects. %within the same application.
This fact is especially pronounced in two data objects $colidx$ and $r$ in CG (Figure~\ref{fig:aDVF_3tiers_profiling}).
 $colidx$ has aDVF much smaller than $r$, which means $colidx$ is much less resilient than $r$ (see finding 1 for a detailed analysis on $colidx$). 
Furthermore, $colidx$ and $r$ have different algorithm-level
fault masking (see finding 4 for a detailed analysis).

\begin{comment}
\textbf{Finding 7: The resilience of the same data objects varies across different applications.}
This fact is especially pronounced in BT and SP.
BT and SP address the same numerical problem but with different algorithms.
BT and SP have the same data objects, $qs$ and $rhoi$, but
$qs$ manifests different resilience in BT and SP.
This result is interesting, because it indicates that by using
different algorithms, we have opportunities to
improve the resilience of data objects.
\end{comment}

To further investigate the reasons for fault masking, 
we break down the aDVF results at the granularity of LLVM instructions,
based on the analyses at the levels of operation and fault propagation.
The results are shown in Figure~\ref{fig:aDVF_fine_profiling}.
%Because of the space limitation, 
%we only show one data object per benchmark, but each selected data object has the most diverse fault masking events within the corresponding benchmark.
%Based on Figure~\ref{fig:aDVF_fine_profiling}, we have another interesting finding.

(8) Arithmetic operations make a lot of contributions to fault masking.
%For $r$ in CG, $r$ in MG, $exp1$ in FT, $u$ in BT, $qs$ in SP, and $u$ in LU,
%the arithmetic operations, FMul (100\%), Add (16\%), FMul (85\%), 
%FMul (94\%), FMul (28\%), and FAdd (50\%)
For $r$ in CG, $u$ in BT, $plane$ and $exp1$ in FT, $m\_elemBC$ in LULESH, 
arithmetic operations (addition, multiplication, and division) contribute to almost 100\% of the fault masking (Figure~\ref{fig:aDVF_fine_profiling}).  
%(at the operation level and the fault propagation level).
%For $qs$ in SP and $u$ in LU, the store operation also makes
%important contributions as the arithmetic operations because of value overwriting.

\begin{figure*}
	\centering
	\includegraphics[width=0.77\textheight, height=0.23\textheight]{pie_chart.pdf}
	\vspace{-10pt}
	\caption{Breakdown of the aDVF results based on the analyses at the levels of operation and fault propagation}
    \vspace{-10pt}
	\label{fig:aDVF_fine_profiling}
\end{figure*}


\subsection{Sensitivity Study}
\label{sec:eval_sen}
%\textbf{change the fault propagation threshold and study the sensitivity of analysis to the threshold}
ARAT uses 10 as the default fault propagation analysis threshold. 
The fault propagation analysis will not go beyond 10 operations. Instead,
we will use deterministic fault injection after 10 operations. 
In this section, we study the impact of this threshold on the modeling accuracy. We use a range of threshold values and examine how the aDVF value varies and whether
the identification of fault masking varies. 
Figure~\ref{fig:sensitivity_error_propagation} shows the results for 
%add , after BT by anzheng
multiple data objects in CG, BT, and SP.
We perform the sensitivity study for all 16 data objects.
%in six benchmarks and two applications.
Due to the page space limitation, we only show the results for three data objects,
but we summarize the sensitivity study results for all data objects in this section.
%but other data objects in all benchmarks have the same trend.

Our results reveal that the identification of fault masking by tracking fault propagation is not significantly 
affected by the fault propagation analysis threshold. Even if we use a rather large threshold (50), 
the variation of aDVF values is 4.48\% on average among all data objects,
and the variation at each of the three levels of analysis (the operation level, fault propagation level,  and algorithm level) is less than 5.2\% on average. 
In fact, using a threshold value of 5 is sufficiently accurate in most of the cases (14 out of 16 data objects).
This result is consistent with our finding 5 (i.e., fault masking at the fault propagation level is small). %in most benchmarks).
However, we do find a data object ($m\_elementBC$ in LULESH) %and $exp1$ in FT) 
showing relatively high-sensitive (up to 15\% variation) to the threshold. For this uncommon data object, using 50 as the fault propagation path is sufficient. 

%In other words, even though using a larger threshold value can identify more error masking by tracking error 
%propagation, the implicit error masking induced by the error propagation is very limited.

\begin{figure}
		\begin{center}
		\includegraphics[width=0.48\textwidth,height=0.11\textheight]{sensi_study_gray.pdf}
		\vspace{-15pt}
		\caption{Sensitivity study for fault propagation threshold}
		\label{fig:sensitivity_error_propagation}
		\end{center}
\vspace{-15pt}
\end{figure}


\begin{comment}
\subsection{Comparison with the Traditional Random Fault Injection}
%\textbf{compare with the traditional fault injection to verify accuracy}
To show the effectiveness of our resilience modeling, we compare traditional random fault injection
and our analytical modeling. Figure~\ref{fig:comparison_fi} and Table~\ref{tab:comparison} show the results.
The figure shows the success rate of all random fault injection. The ``success'' means the application
outcome is verified successfully by the benchmarks and the execution does not have any segfault. The success rate is used as a metric
to evaluate the application resilience.

We use a data-oriented approach to perform random fault injection.
In particular, given a data object, for each fault injection test we trigger a bit flip
in an operand of a random instruction, and this operand must be a reference to the
target data object. We develop a tool based on PIN~\cite{pintool} to implement the above fault injection functionality.
For each data object, we conduct five sets of random fault injection tests, 
and each set has 200 tests (in total 1000 tests per data object). 
We show the results for CG and FT in this section, but we find that
the conclusions we draw from CG and FT are also valid for the other four benchmarks.


%\begin{table*}
%\label{tab:success_rate}
%\begin{centering}
%\renewcommand\arraystretch{1.1}
%\begin{tabular}{|c|c|c|c|c|c|c|}
%\hline 
%Success Rate (Difference) & Test set 1 & Test set 2 & Test set 3 & Test set 4 & Test set 5 & Average\tabularnewline
%\hline 
%\hline 
%CG-a & 66.1\% (11.7\%) & 68.5\% (15.7\%) & 56.7\% (4.21\%) & 61.3\% (3.57\%) & 43.3\% (26.8\%) & 59.2\%\tabularnewline
%\hline 
%CG-x & 99.2\% (2.2\%) & 98.6\% (1.5\%) & 96.5\% (0.63\%) & 97.8\% (0.64\%) & 93.6\% (3.7\%) & 97.1\%\tabularnewline
%\hline 
%CG-colidx & 36.8\% (12.7\%) & 49.6\% (17.8\%) & 40.2\% (4.6\%) & 52.6\% (24.9\%) & 31.4\% (25.4\%) & 42.1\%\tabularnewline
%\hline 
%FT-exp1 & 52.7\% (1.4\%) & 22.6\% (56.5\%) & 78.5\% (51.0\%) & 60.7\% (16.7\%) & 45.4\% (12.7\%) & 51.9\%\tabularnewline
%\hline 
%FT-plane & 82.1\% (2.5\%) & 79.3\% (5.6\%) & 99.5\% (18.2\%) & 93.2\% (10.7\%) & 66.8\% (20.6\%) & 84.2\%\tabularnewline
%\hline 
%\end{tabular}
%\par\end{centering}
%\caption{XXXXX}
%\end{table*}


\begin{table*}
\begin{centering}
\caption{\small The results for random fault injection. The numbers in parentheses for each set of tests (200 tests per set) are the success rate difference from the average success rate of 1000 fault injection tests.}
\label{tab:comparison}
\renewcommand\arraystretch{1.1}
\begin{tabular}{|c|p{2.2cm}|p{2.2cm}|p{2.2cm}|p{2.2cm}|p{2.2cm}|p{1.8cm}|}
\hline 
       %& Test set 1 & Test set 2 & Test set 3 & Test set 4 & Test set 5 & Average\tabularnewline
       & \hspace{13pt} Test set 1 \hspace{1pt}/  & \hspace{13pt} Test set 2 \hspace{1pt}/ & \hspace{13pt} Test set 3 \hspace{1pt}/ & \hspace{13pt} Test set 4 \hspace{1pt}/ & \hspace{13pt} Test set 5 \hspace{1pt}/ & Ave. of all test / \\
       & success rate (diff.) & success rate (diff.) & success rate (diff.) & success rate (diff.) & success rate (diff.) & \hspace{5pt} success rate \\
\hline 
\hline 
CG-a & 66.1\% (6.9\%) & 68.5\% (9.3\%) & 56.7\% (-2.5\%) & 61.3\% (2.1\%) & 43.3\% (-15.9\%) & 59.2\%\tabularnewline
\hline 
CG-x & 99.2\% (2.1\%) & 98.6\% (1.5\%) & 96.5\% (-0.6\%) & 97.8\% (0.7\%) & 93.6\% (-3.5\%) & 97.1\%\tabularnewline
\hline 
CG-colidx & 36.8\% (-5.3\%) & 49.6\% (7.5\%) & 40.2\% (-2.0\%) & 52.6\% (10.5\%) & 31.4\% (-10.7\%) & 42.1\%\tabularnewline
\hline 
FT-exp1 & 52.7\% (0.8\%) & 22.6\% (-29.3\%) & 78.5\% (26.6\%) & 60.7\% (8.8\%) & 45.4\% (-6.5\%) & 51.9\%\tabularnewline
\hline 
FT-plane & 82.1\% (-2.1\%) & 79.3\% (-4.9\%) & 99.5\% (15.3\%) & 93.2\% (9.0\%) & 66.8\% (-17.4\%) & 84.2\%\tabularnewline
\hline 
\end{tabular}
\par\end{centering}
\vspace{-0.4cm}
\end{table*}

\begin{figure}
	\begin{center}
		\includegraphics[width=0.48\textwidth,keepaspectratio]{verifi-study.png}
		\caption{The traditional random fault injection vs. ARAT}
		\label{fig:comparison_fi}
	\end{center}
\vspace{-0.7cm}
\end{figure}


We first notice from Table~\ref{tab:comparison} that 
%across 5 sets of random fault injection tests, there are big variances (up to 55.9\% in $exp1$ of FT) in terms of the success rate. 
the results of 5 test sets can be quite different from each other and from 1000 random fault inject tests (up to 29.3\%).
1000 fault injection tests provide better statistical significance than 200 fault injection tests.
We expect 1000 fault injection tests potentially provide higher accuracy to quantify the application resilience.
The above result difference is clearly an indication to the randomness of fault injection, and there
is no guarantee on the random fault injection accuracy.

%In Figure~\ref{fig:comparison_fi}, 
We compare the success rate of 1000 fault inject tests with the aDVF value (Fig.~\ref{fig:comparison_fi}). 
We find that the order of the success rate of the three data objects in CG (i.e., $colidx < a < x$) and the two data objects in FT 
(i.e., $exp1 < plane$) is the same as the order of the aDVF values of these data objects. 
%In fact, 1000 fault injection tests
%account for \textcolor{blue}{\textbf{xxx\%}} of total memory references to the data object,
%and provide better resilience quantification than 200 fault injection tests.
The same order (or the same resilience trend)
%between our approach and the random fault injection based on a large number of tests 
is a demonstration of the effectiveness of our approach.
Note that the values of the aDVF and success rate %for a data object
cannot be exactly the same (even if we have sufficiently large numbers of random fault injection), 
because aDVF and random fault injection quantify
the resilience based on different metrics.
Also, the random fault injection can miss some fault masking events that can be captured by our approach.

\end{comment}
%\mySection{Related Works and Discussion}{}
\label{chap3:sec:discussion}

In this section we briefly discuss the similarities and differences of the model presented in this chapter, comparing it with some related work presented earlier (Chapter \ref{chap1:artifact-centric-bpm}). We will mention a few related studies and discuss directly; a more formal comparative study using qualitative and quantitative metrics should be the subject of future work.

Hull et al. \citeyearpar{hull2009facilitating} provide an interoperation framework in which, data are hosted on central infrastructures named \textit{artifact-centric hubs}. As in the work presented in this chapter, they propose mechanisms (including user views) for controlling access to these data. Compared to choreography-like approach as the one presented in this chapter, their settings has the advantage of providing a conceptual rendezvous point to exchange status information. The same purpose can be replicated in this chapter's approach by introducing a new type of agent called "\textit{monitor}", which will serve as a rendezvous point; the behaviour of the agents will therefore have to be slightly adapted to take into account the monitor and to preserve as much as possible the autonomy of agents.

Lohmann and Wolf \citeyearpar{lohmann2010artifact} abandon the concept of having a single artifact hub \cite{hull2009facilitating} and they introduce the idea of having several agents which operate on artifacts. Some of those artifacts are mobile; thus, the authors provide a systematic approach for modelling artifact location and its impact on the accessibility of actions using a Petri net. Even though we also manipulate mobile artifacts, we do not model artifact location; rather, our agents are equipped with capabilities that allow them to manipulate the artifacts appropriately (taking into account their location). Moreover, our approach considers that artifacts can not be remotely accessed, this increases the autonomy of agents.

The process design approach presented in this chapter, has some conceptual similarities with the concept of \textit{proclets} proposed by Wil M. P. van der Aalst et al. \citeyearpar{van2001proclets, van2009workflow}: they both split the process when designing it. In the model presented in this chapter, the process is split into execution scenarios and its specification consists in the diagramming of each of them. Proclets \cite{van2001proclets, van2009workflow} uses the concept of \textit{proclet-class} to model different levels of granularity and cardinality of processes. Additionally, proclets act like agents and are autonomous enough to decide how to interact with each other.

The model presented in this chapter uses an attributed grammar as its mathematical foundation. This is also the case of the AWGAG model by Badouel et al. \citeyearpar{badouel14, badouel2015active}. However, their model puts stress on modelling process data and users as first class citizens and it is designed for Adaptive Case Management.

To summarise, the proposed approach in this chapter allows the modelling and decentralized execution of administrative processes using autonomous agents. In it, process management is very simply done in two steps. The designer only needs to focus on modelling the artifacts in the form of task trees and the rest is easily deduced. Moreover, we propose a simple but powerful mechanism for securing data based on the notion of accreditation; this mechanism is perfectly composed with that of artifacts. The main strengths of our model are therefore : 
\begin{itemize}
	\item The simplicity of its syntax (process specification language), which moreover (well helped by the accreditation model), is suitable for administrative processes;
	\item The simplicity of its execution model; the latter is very close to the blockchain's execution model \cite{hull2017blockchain, mendling2018blockchains}. On condition of a formal study, the latter could possess the same qualities (fault tolerance, distributivity, security, peer autonomy, etc.) that emanate from the blockchain;
	\item Its formal character, which makes it verifiable using appropriate mathematical tools;
	\item The conformity of its execution model with the agent paradigm and service technology.
\end{itemize}
In view of all these benefits, we can say that the objectives set for this thesis have indeed been achieved. However, the proposed model is perfectible. For example, it can be modified to permit agents to respond incrementally to incoming requests as soon as any prefix of the extension of a bud is produced. This makes it possible to avoid the situation observed on figure \ref{chap3:fig:execution-figure-4} where the associated editor is informed of the evolution of the subtree resulting from $C$ only when this one is closed. All the criticisms we can make of the proposed model in particular, and of this thesis in general, have been introduced in the general conclusion (page \pageref{chap5:general-conclusion}) of this manuscript.





%\section{Activation During Perception of Noisy Speech}\label{sec:Apps}
The dataset, provided as  {\tt data6} in the AFNI tutorial~\citet{cox96},
is originally from an fMRI study~\citet{nathandbeauchamp11} where
\begin{figure}[h]
\subfloat[]{\includegraphics[width=0.5\columnwidth]{figs/AR-FAST-001-Visual-crop}}
\subfloat[]{\includegraphics[width=0.5\columnwidth]{figs/AR-FAST-001-Audio-crop}}
%\subfloat[]{\includegraphics[width=0.33\textwidth]{figs/AM-FAST-005-diff}}
\caption{ AR-FAST-identified activation regions on SPMs obtained by
  fitting ~\ref{eq:lm} with   AR($\hat{p}$) to AFNI's {\tt data6} for
  (a) visual-reliable stimulus and (b) audio-reliable
  stimulus.}
\label{fig:AMSmoothingAFNI}
\end{figure}
\begin{comment}
\begin{figure*}[h]
\subfloat[]{\includegraphics[width=0.25\textwidth]{figures/Visual_AM-crop}}
\subfloat[]{\includegraphics[width=0.25\textwidth]{figures/Audio_AM-crop}}
\subfloat[]{\includegraphics[width=0.25\textwidth]{figures/Visual_Audio_AM-crop}}
\caption{Activation areas obtained using AR-FAST with in {\it AFNI data6} on the SPM obtained
  after fitting AR($\hat{p}$) of the (a) Visual-reliable, (b) Audio-reliable and (c) the difference contrast between Visual-reliable and Audio-reliable.}
\label{fig:AMSmoothingAFNI}
\end{figure*}
\end{comment}
a subject heard and saw a female volunteer speak words, separately, in
two different formats. The audio-reliable setting had the subject
clearly hear the spoken word but see a degraded image of the speaker
while the visual-reliable case had the subject clearly see the speaker
vocalize the word but the audio was of reduced quality.  There were
three experimental runs, each 
consisting of a randomized design of 10 blocks, equally divided into blocks of
audio-reliable and visual-reliable stimuli. %An echo-planar imaging
% sequence (TR=2s) was used to obtain
$\mbox{T}_2^*$-weighted images with volumes of $80 \times 80 \times
33$ (with voxels of dimension $2.75 \times  2.75 \times 3.0\  mm^3$)
from  echo-planar sequences (TR=2s) 
were obtained  over $152$ time-points. Our interest was in determining 
activation corresponding to the audio
($H_0:\beta_{a}=0$) and visual
($H_0:\beta_{v}=0$) tasks.
%, as well as their contrast  ($H_0:\beta_{v} - \beta_{a}=0$). The
%first two cases have one-sided alternatives while the contrast in
%activation corresponds to a two-sided alternative.
At each voxel, we fitted AR models for 
$p=0,1,2,3,4,5$ and chose $p$ with the highest BIC. 
Figure \ref{fig:AMSmoothingAFNI} uses AFNI and Surface Mapping (SUMA)
to display activated regions obtained using AR-FAST on the SPM:
see  Figure~\ref{fig:Visual-Audio} for  maps drawn from ALL-FAST, AS,
AWS and CT. We used $\alpha = 0.01$ because of the high (greater than
4) upper percentile of the voxel-wise estimated CNRs. Most of the activation 
occurs in Brodmann areas 18 and 19 (BA18 and BA19)
which comprise the
occipital cortex %in the human brain,  accounting for the bulk of the
                 %volume of the occipital lobe. Both areas form part
                 %of the visual association area while BA 19, also the occipital lobe cortex as well. Along with BA18, it comprises
and the extrastriate (or peristriate) cortex. In humans with normal
sight,  this area is for visual association where 
feature-extraction, shape recognition, attentional and multimodal
integrating functions occur. We also see increased activation in
the STS, which recent
studies~\citep{grossman2001brain} have related to  distinguishing
voices from environmental sounds, 
stories versus nonsensical speech, moving faces versus moving objects,
biological motion and so on. ALL-FAST performs similarly as AR-FAST,
while the other methods also identify the same regions but they identify
a lot more activated 
voxels, some of which appear to be false positives. Although a 
detailed analysis of the results of this study is beyond the purview
of this paper, we note that AR-FAST 
finds interpretable results even when applied to a single
subject high-level cognition experiment. 
\begin{comment}
\begin{table}[h!]
\centering
\caption{Coordinates of the maximum $t$-statistic and its corresponding value}\label{tab:maxt}
\begin{tabular}{c|c}
Task & $(x,y,z) mm$  \\
\hline
Visual-Audio & (-30.162,86.221,6.349) \\
Audio & (-27.412,75.221,-5.651)  \\
 Visual & (-30.162, 80.721, 15.349) \\
\hline
\end{tabular}
\end{table}
\end{comment}


\section{Conclusions and Future Work}

This paper presents an interactive recommendation approach for the popular application of online movie recommendation with the general public as end users.
We have studied LSM, which enables us to translate abstract data and complex recommendation algorithms to a set of semantic concepts, provide explicit interaction of search preferences, and construct meaningful recommendation stories.
The LSM can be easily combined with other recommendation algorithms, as we separate the estimation of recommendation degrees and the latent space.
The interactive recommendation approach automatically generates storytelling animations for recommendation, with the supports of several interactive exploration functions for users to adjust the search results explicitly. 
Different from traditional recommendation algorithms, the interactive recommendation approach emphasizes the visual communication between users and recommendation systems for engaging users and improving search experiences.
Both of our results can also be extended to recommend other online products or services.

As future work, we plan to perform formal evaluations on the effectiveness of interactive recommendation.
We are interested in studying the suitable amount of information for different users, such as the general public and movie experts, so that different versions of narrative visualization can be developed to suit for different needs. 
We also plan to develop variations of recommendation stories, such as long versions that can combine multiple latent dimensions, for different recommendation tasks.
At the end, we are interested in integrating other techniques for movie recommendation, such as text analysis approaches for mining useful information from the movie reviews.
The results will enrich the contents of narrative visualization and may provide better search experiences.


%At the end, we need to study the state-of-the-art of recommendation systems to ensure that our approach can be combined for end users smoothly.

%we are interested in continuing to study the capabilities of interactive storytelling as a visual analysis tool for various visualization applications.
%For movie recommendation, we expect to improve user satisfaction by refining our approach to accommodate subtle differences of user tastes.

%% if specified like this the section will be committed in review mode
%\acknowledgments{
%The authors wish to thank A, B, C. This work was supported in part by
%a grant from XYZ.}


\bibliographystyle{abbrv}
%%use following if all content of bibtex file should be shown
%\nocite{*}
\bibliography{story}
\end{document}

up to ten (10) pages

Please provide supplemental videos in QuickTime MPEG-4 or DivX version 5, and use TIFF, JPEG, or PNG for supplemental images.

Technique papers introduce novel techniques or algorithms that have not previously appeared in the literature, or that significantly extend known techniques or algorithms, for example by scaling to datasets of much larger size than before or by generalizing a technique to a larger class of uses. The technique or algorithm description provided in the paper should be complete enough that a competent graduate student in visualization could implement the work, and the authors should create a prototype implementation of the methods. Relevant previous work must be referenced, and the advantage of the new methods over it should be clearly demonstrated. There should be a discussion of the tasks and datasets for which this new method is appropriate, and its limitations. Evaluation through informal or formal user studies, or other methods, will often serve to strengthen the paper, but are not mandatory.

System papers present a blend of algorithms, technical requirements, user requirements, and design that solves a major problem. The system that is described is both novel and important, and has been implemented. The rationale for significant design decisions is provided, and the system is compared to documented, best-of-breed systems already in use. The comparison includes specific discussion of how the described system differs from and is, in some significant respects, superior to those systems. For example, the described system may offer substantial advancements in the performance or usability of visualization systems, or novel capabilities. Every effort should be made to eliminate external factors (such as advances in processor performance, memory sizes or operating system features) that would affect this comparison. For further suggestions, please review "How (and How Not) to Write a Good Systems Paper" by Roy Levin and David Redell, and "Empirical Methods in CS and AI" by Toby Walsh.

Application / Design Study papers explore the choices made when applying visualization and visual analytics techniques in an application area, for example relating the visual encodings and interaction techniques to the requirements of the target task. Similarly, Application papers have been the norm when researchers describe the use of visualization techniques to glean insights from problems in engineering and science. Although a significant amount of application domain background information can be useful to provide a framing context in which to discuss the specifics of the target task, the primary focus of the case study must be the visualization content. The results of the Application / Design Study, including insights generated in the application domain, should be clearly conveyed. Describing new techniques and algorithms developed to solve the target problem will strengthen a design study paper, but the requirements for novelty are less stringent than in a Technique paper. Where necessary, the identification of the underlying parametric space and its efficient search must be aptly described. The work will be judged by the design lessons learned or insights gleaned, on which future contributors can build. We invite submissions on any application area.
