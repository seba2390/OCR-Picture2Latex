% $Id: template.tex 11 2007-04-03 22:25:53Z jpeltier $

\documentclass{vgtc}                          % final (conference style)
\let\ifpdf\relax
%\documentclass[review]{vgtc}                 % review
%\documentclass[widereview]{vgtc}             % wide-spaced review
%\documentclass[preprint]{vgtc}               % preprint
%\documentclass[electronic]{vgtc}             % electronic version

%% Uncomment one of the lines above depending on where your paper is
%% in the conference process. ``review'' and ``widereview'' are for review
%% submission, ``preprint'' is for pre-publication, and the final version
%% doesn't use a specific qualifier. Further, ``electronic'' includes
%% hyperreferences for more convenient online viewing.

%% Please use one of the ``review'' options in combination with the
%% assigned online id (see below) ONLY if your paper uses a double blind
%% review process. Some conferences, like IEEE Vis and InfoVis, have NOT
%% in the past.

%% Figures should be in CMYK or Grey scale format, otherwise, colour 
%% shifting may occur during the printing process.

%% These three lines bring in essential packages: ``mathptmx'' for Type 1 
%% typefaces, ``graphicx'' for inclusion of EPS figures. and ``times''
%% for proper handling of the times font family.

\usepackage{mathptmx}
\usepackage{graphicx}
\usepackage{times}
%\usepackage{epstopdf}
\usepackage{enumitem} 

%% We encourage the use of mathptmx for consistent usage of times font
%% throughout the proceedings. However, if you encounter conflicts
%% with other math-related packages, you may want to disable it.

%% If you are submitting a paper to a conference for review with a double
%% blind reviewing process, please replace the value ``0'' below with your
%% OnlineID. Otherwise, you may safely leave it at ``0''.
%\onlineid{2913}

%% declare the category of your paper, only shown in review mode
\vgtccategory{Research}

%% allow for this line if you want the electronic option to work properly
\vgtcinsertpkg
%\graphicspath{ {image/addedImages/}{image/} }

%% In preprint mode you may define your own headline.
%\preprinttext{To appear in an IEEE VGTC sponsored conference.}

%% Paper title.

\title{Interactive Movie Recommendation Through Latent Semantic Analysis and Storytelling}

%% This is how authors are specified in the journal style

%% indicate IEEE Member or Student Member in form indicated below
%\author{Roy G. Biv, Ed Grimley, \textit{Member, IEEE}, and Martha Stewart}
\author{Kodzo Wegba$^{1}$, Aidong Lu$^{1}$, Yuemeng Li$^{1}$, and Wencheng Wang$^{2}$\\
$^1$ University of North Carolina at Charlotte, USA, \{kwegba1, aidong.lu, yli19\}@uncc.edu\\
$^2$ Chinese Academy of Sciences, China, whn@ios.ac.cn
}
%\authorfooter{
%\item
%Wegba, Lu, and Li are with University of North Carolina at Charlotte. E-mail: kwegba1, aidong.lu, yli19@uncc.edu.
%\item
%Wencheng Wang (whn@ios.ac.cn) is with Chinese Academy of Sciences.
%}
%%% insert punctuation at end of each item
% Ed Grimley is with Grimley Widgets, Inc.. E-mail: ed.grimley@aol.com.
%\item
% Martha Stewart is with Martha Stewart Enterprises at Microsoft
% Research. E-mail: martha.stewart@marthastewart.com.


%other entries to be set up for journal
%\shortauthortitle{Biv \MakeLowercase{\textit{et al.}}: Global Illumination for Fun and Profit}
%\shortauthortitle{Firstauthor \MakeLowercase{\textit{et al.}}: Paper Title}

%% Abstract section.
\abstract{
Recommendation has become one of the most important components of online services for improving sale records, however visualization work for online recommendation is still very limited.
This paper presents an interactive recommendation approach with the following two components.
First, rating records are the most widely used data for online recommendation, but they are often processed in high-dimensional spaces that can not be easily understood or interacted with. 
We propose a Latent Semantic Model (LSM) that captures the statistical features of semantic concepts on 2D domains and abstracts user preferences for personal recommendation.
Second, we propose an interactive recommendation approach through a storytelling mechanism for promoting the communication between the user and the recommendation system.
Our approach emphasizes interactivity, explicit user input, and semantic information convey; thus it can be used by general users without any knowledge of recommendation or visualization algorithms.
We validate our model with data statistics and demonstrate our approach with case studies from the MovieLens100K dataset. 
%We have also evaluated the effectiveness of interactive storytelling and received very positive results.
Our approaches of latent semantic analysis and interactive recommendation can also be extended to other network-based visualization applications, including various online recommendation systems.
} % end of abstract
%The techniques of storytelling and narrative visualization have recently raised many interests.
%This work studies the analysis capabilities of storytelling for a popular application of movie recommendation.
%Our solution integrates results from two aspects:
%a new latent semantic model for extracting recommendation story pieces from the high-dimensional latent space,
%and an interactive storytelling approach for providing user-friendly narrative visualization and interactive recommendation.
%Our approach emphasizes interactivity, explicit user input, and semantic information convey; thus it can be used by general users without any knowledge of recommendation and visualization algorithms.
%We validate our model with data statistics and demonstrate our approach on interactive recommendation and visual comparison of user tastes from the MovieLens100K dataset. 
%We have also evaluated the effectiveness of interactive storytelling and received very positive results.
%Our approaches of storytelling and latent semantic analysis can be easily extended to other network-based visualization applications, including various online recommender systems.


%% Keywords that describe your work. Will show as 'Index Terms' in journal
%% please capitalize first letter and insert punctuation after last keyword
%\keywords{Interactive recommendation, latent semantic analysis, storytelling, narrative visualization.}

%% ACM Computing Classification System (CCS). 
%% See <http://www.acm.org/class/1998/> for details.
%% The ``\CCScat'' command takes four arguments.

%\CCScatlist{ % not used in journal version
%\CCScat{I.3.6}{Computer Graphics}{Methodology and Techniques}{Interaction techniques};
%\CCScat{I.3.8}{Computer Graphics}{}{Applications}
%}

% Uncomment below to include a teaser figure.
\teaser{
\centering
\includegraphics[scale=0.65]{SVDSemanticOrig.EPS}
\caption{
%The abstraction of user taste for movie recommendation and interactive storytelling, where recommended movies are enlarged as blue circles, and other candidates are shown as purple nodes.
%Narrative visualization, generated with our latent semantic model based on the domain of recommendation degrees and example-enhanced storyline, guides users throughout interactive storytelling, recommendation, and exploration processes.
An example from our latent semantic model for interactive recommendation and abstraction of user preferences.
Our approach identifies a 2D visualization domain, where the horizontal axis layouts recommendable movies on a latent dimension between two combined movie features that are selected based on the user's watch history, and the vertical axis uses recommendation degrees to move highly recommendable movies to the top.
This example demonstrates the preference of a user on drama/documentary/biography movies (green zone toward the right) over comedy/music genres (orange zone toward the left).
The movies selected to recommend are enlarged as blue circles, recommendable movies are shown as purple nodes, watched and liked movies as green nodes, and disliked movies as orange nodes.
Two example movie posters, one liked movie ``Casino" and one disliked movie ``Airheads", are also provided to demonstrate the latent dimension.
For illustration purpose, we also add the arrowed line at the bottom and several movie titles to confirm the movie distributions on the visualization domain.
}
\label{svdSemantic}
}

%% Uncomment below to disable the manuscript note
%\renewcommand{\manuscriptnotetxt}{}

%% Copyright space is enabled by default as required by guidelines.
%% It is disabled by the 'review' option or via the following command:
% \nocopyrightspace

%%%%%%%%%%%%%%%%%%%%%%%%%%%%%%%%%%%%%%%%%%%%%%%%%%%%%%%%%%%%%%%%
%%%%%%%%%%%%%%%%%%%%%% START OF THE PAPER %%%%%%%%%%%%%%%%%%%%%%
%%%%%%%%%%%%%%%%%%%%%%%%%%%%%%%%%%%%%%%%%%%%%%%%%%%%%%%%%%%%%%%%%

\begin{document}

%% The ``\maketitle'' command must be the first command after the
%% ``\begin{document}'' command. It prepares and prints the title block.

%% the only exception to this rule is the \firstsection command
\firstsection{Introduction}

\maketitle

%% \section{Introduction} %for journal use above \firstsection{..} instead
%\section{Introduction}  \label{sec:introduction}

\newcommand\inexpIntro[3]{#1?(#2,#3).}
\newcommand\rinexpIntro[3]{*#1?(#2,#3).}
\newcommand\outexpIntro[3]{#1!(#2,#3).}
\newcommand\outatomIntro[3]{#1!(#2,#3)}

We propose a fully automated method for proving termination of \(\pi\)-calculus processes.
Although there have been a lot of studies on termination analysis for the \(\pi\)-calculus
and related calculi~\cite{Deng06IC,Demangeon07,SangiorgiTermination,KobayashiHybrid,Yoshida04IC,DBLP:journals/jlp/DemangeonHS10,Venet98SAS}, most of them have been rather theoretical,
and there have been surprisingly little efforts in developing  fully automated termination
verification methods and tools based on them. To our knowledge,
Kobayashi's \typical{}~\cite{TyPiCal,KobayashiHybrid} is the only exception that
can prove termination of \(\pi\)-calculus processes (extended with natural numbers)
fully automatically, but its termination analysis is quite limited (see Section~\ref{sec:relatedwork}).

Our method is based on a reduction to termination analysis for sequential programs:
we translate a \(\pi\)-calculus process \(P\) to a sequential program \(S_P\), so that
if \(S_P\) is terminating, so is \(P\). The reduction allows us to use
powerful, mature methods and tools
for termination analysis of sequential programs~\cite{heizmann2016ultimate,freqterm,DBLP:conf/lics/PodelskiR04,Kuwahara2014Termination,DBLP:journals/cacm/CookPR11}.

The idea of the translation is to convert a chain of communications on replicated input
channels to a chain of recursive function calls of the target sequential program.
Let us consider the following Fibonacci process:
\begin{align*}
    & \rinexpIntro{\fib}{n}{r}
        \ifexp{n<2}{ \soutatom{r}{1} \\ &\quad}
                   { \nuexp{s_1} \nuexp{s_2} (\outatomIntro{\fib}{n-1}{s_1} \PAR \outatomIntro{\fib}{n-2}{s_2} \PAR \sinexp{s_1}{x}\sinexp{s_2}{y}\soutatom{r}{x+y}) \\}
    & \PAR \outatomIntro{\fib}{m}{r}
\end{align*}
Here, the process
$\rinexpIntro{\fib}{n}{r} \ldots$ is a function server that computes the \(n\)-th Fibonacci number
in parallel and returns the result to \(r\),
and $\outatom{\fib}{m}{r}$ sends a request for computing the \(m\)-th Fibonacci number;
those who are not familiar with the syntax of the \(\pi\)-calculus may wish to consult
Section~\ref{sec:targetlanguage} first.
To prove that the process above is terminating for any integer \(m\),
it suffices to show that there is no infinite chain of communications on $\fib$:
\[
    \fib(m,r) \to \fib(m_1,r_1) \to \fib(m_2,r_2) \to \cdots.
\]
We convert the process above to the following program:\footnote{The actual translation
  given later is a little more complex.}
\begin{verbatim}
 let rec fib(n) = if n<2 then () else (fib(n-1) [] fib(n-2)) in
 fib(m)
\end{verbatim}
Here, \texttt{[]} represents the non-deterministic choice.
Note that, although the calculation of Fibonacci numbers is not preserved,
for each chain of communications on \texttt{fib}, there is a corresponding
sequence of recursive calls:
\[
\mathtt{fib}(m) \to \mathtt{fib}(m_1) \to \mathtt{fib}(m_2) \to \cdots.
\]
Thus, the termination of the sequential program above implies the termination of
the original process.
As shown in the example above, (i) each communication on a replicated input channel
is converted to a function call, (ii) each communication on a non-replicated input
channel is just removed (or, in the actual translation, replaced by a call of
a trivial function defined by \(f(\seq{x})=(\,)\)), and (iii) parallel composition
is replaced by a non-deterministic choice.
We formalize the translation outlined above and prove its correctness.

The basic translation sketched above sometimes loses too much information.
For example, consider the following process:
\begin{align*}
    & \rinexpIntro{\pre}{n}{r} \soutatom{r}{n-1} \\
    & \PAR \rinexpIntro{f}{n}{r} \ifexp{n<0}{ \soutatom{r}{1} }
                                       { \nuexp{s} (\outatomIntro{\pre}{n}{s} \PAR \sinexp{s}{x}\outatomIntro{f}{x}{r}) } \\
    & \PAR \outatomIntro{f}{m}{r}
\end{align*}
The translation sketched above would yield:
\begin{verbatim}
  let pred(n) = n-1 in
  let rec f(n) = if n<0 then () else (pred(n) [] f(*)) in
  f(m)
\end{verbatim}
Here, \texttt{*} represents a non-deterministic integer: since we have removed
the input $\sinatom{s}{x}$, we do not have information about the value of \( x \).
As a result, the sequential program above is non-terminating, although the original
process is terminating.
To remedy this problem, we also refine the basic translation above by using a refinement
type system for the \(\pi\)-calculus. Using the refinement type system,
we can infer that the value of \(x\) in the original process is less than \(n\),
so that we can refine the definition of \texttt{f} to:
\begin{verbatim}
 let rec f(n) = ... else (pred(n) [] let x=* in assume(x<n);f(x))
\end{verbatim}
The target program is now terminating, from which
we can deduce that the original process is also terminating.
We have implemented an automated tool based on the refined translation above.

The contributions of this paper are summarized as follows.
\begin{itemize}
\item The formalization of the basic translation from the \(\pi\)-calculus
  (extended with integers) to sequential programs, and a proof of its correctness.
\item The formalization of a refined translation based on a refinement type system.
\item An implementation of the refined translation, including automated refinement type
  inference based on CHC solving, and experiments to evaluate the effectiveness of
  our method.
\end{itemize}

The rest of this paper is structured as follows.
Section~\ref{sec:targetlanguage} introduces the source and target languages
of our translation.
Section~\ref{sec:approach} 
formalizes the basic translation, and proves its correctness.
Section~\ref{sec:refinement} refines the basic translation by using a refinement type system.
Section~\ref{sec:implementation} reports an implementation and experiments.
Section~\ref{sec:relatedwork} discusses related work,
and Section~\ref{sec:conclusion} concludes the paper.

%\section{Introduction}

%Modern consumers are inundated with information and choices.

Recommendation systems have been reported as key pillars of online services~\cite{Gomez-Uribe:2015:NRS:2869770.2843948} for significantly improving sale records~\cite{1167344}.
Nowadays, online retailers and content providers offer a huge amount of products or services, which are often overwhelming for consumers.
To improve user satisfaction and loyalty, Internet leaders like Amazon, Google, and Yahoo are all using recommendation systems to provide personalized suggestions.
However, matching consumers with the most appropriate products is not trivial.
While many recommendation algorithms have been developed, including 
the prevailing top-N recommendation approaches~\cite{Deshpande:2004:ITN:963770.963776}, effective interaction mechanisms for consumers to adjust search preferences or recommendation results are still lacking~\cite{loepp2014choice}.
%The most popular approaches, item-based top-N recommendation~\cite{Deshpande:2004:ITN:963770.963776}, are often achieved through establishing connections between consumers and products from the past ratings. 

The challenges for improving user satisfaction of interactive recommendation systems come from several aspects.
First, the satisfaction of consumers vary obviously with other factors such as emotions and situations, which require explicit input from users.
Second, useful information for assisting users to find the right movies, such as the similarities between recommended and rated movies, is often completely hidden from users.
Third, complex recommendation algorithms and visualization systems are often too complicated for general consumers to use.
To the best of our knowledge, there are no previous work on addressing all above challenges.

%bridging the gap between fully automatic top-N recommendation algorithms and complex visualization techniques to develop user-friendly recommendation systems.

In this work, we present an interactive recommendation approach that simulates the scenario of an expert recommending movies to a user -- an expert generally selects movies for a user based on his or her watch history, makes recommendations, listens to the feedback, and continues to recommend more movies until the user finds a movie to watch.
Similarly, our approach adopts such a continuous and interactive recommendation process that allows the user to orientate search results actively and explicitly.
For efficiency, we recommend movies in groups, with each group 
containing movies selected from combined movie features that are extracted from watch history.
As shown in Figure~\ref{svdSemantic}, for a user who likes a number of drama, documentary, and biography movies, we visualize the movie distributions on the dimension between the liked and disliked movie features, and make recommendations based on important criteria such as user preference and variety.
%As shown in Figure~\ref{svdSemantic}, a real user may have watched movies from a diverse set of genres and the recommended movies need to be selected based on the user's watched history.
This requires us to study the personal movie features from rating statistics and present recommended movies with both useful information and connections with user-rated movies in a succinct, user-friendly manner.
%Our interactive recommendation approach continuously modifies recommendation contents and adjust story styles based on user interaction, which simulates the scenario that a user consults an expert for movie recommendation.

Our work consists of two components, a model of latent semantic analysis and an approach of interactive recommendation.
We first present a latent semantic model (LSM) built upon the high-dimensional latent space factorized from the movie rating records, as they are the most commonly used data for online recommendation.
Our model collects a set of semantic concepts, including like, dislike, familiarity, diversity, typicality, and un-typicality, that can be used for two purposes:
one is to help a user to correlate new recommended movies to the movies the user has watched, the other is to allow the user to specify the preferences of recommendation results explicitly.
We abstract the statistical features of semantic concepts among the high-dimensional latent space and automatically identify suitable 2D domains for interactive visualization.
We validate LSM with a real-life dataset to show that our model is applicable to various users for recommendation.

We then present an interactive recommendation system that recommend movies in a storytelling style to promote the communication between the user and recommendation system.
We employ the LSM to generate recommendation stories by assigning movies with suitable characters, roles, and narrative structures for describing a group of recommended movies.
The recommendation stories are constructed automatically to attract user attention, supported with multi-level visualization and animation effects, and can be adjusted flexibly during the interactive recommendation process.
Different from previous online recommendation systems, our approach reveals information that is generally hidden from users and allows interactive exploration of movie similarities.
%This is designed based on our study of the application requirements and existing recommender systems.
%Both the narrative visualization and user interactions are designed for users without any background of recommendation or visualization. 
%We have also performed a controlled user study and a questionnaire to evaluate our approach.
%The results demonstrate that interactive recommendation is very promising for engaging user participation and providing better recommendation experiences.
We also provide several case studies to demonstrate the usage of LSM and interactive recommendation on different recommendation tasks.
%, including visualization of user tastes, recommendation to individual or group users, and comparison of user tastes.

The remainder of this paper is organized as follows. 
We start with related work in Section 2.
We then describe our LSM in Section 3 and interactive recommendation approach in Section 4.
We present the recommendation system in Section 5 and results in Section 6.
Section 7 concludes the work and presents future work.

%\textbf{Related work}:
% Object detection related datasets/algo in non-medical domain
% Locally labeled CXR dataset
A few CXR datasets have localized abnormality annotations \cite{shih2019augmenting,filice2020crowdsourcing,jaeger2014two} that are curated manually. These are high quality gold standard ground truth datasets but tend to be smaller in scale (< 30,000 images) and have a narrow coverage, with typically only 1-2 labels. In addition, since most labeling efforts only have abnormality semantics attached, no direct relationships with the affected anatomical locations are available. 

%MEHDI: repeated concepts from above. I am removing the following: 

%The lack of anatomic semantics in the annotation is a limitation for complex multi-modal clinical reasoning work, e.g., differential diagnosis, since clinicians often integrate information along anatomical lines, and for downstream report generation tasks, which often requires describing not only the abnormality but also correctly communicate the location of the abnormalities (and medical devices) to the receiving clinicians. 

Two recent CXR datasets have labels for anatomies described in the reports. In \cite{datta2020dataset}, a small manually annotated dataset (2000 reports) included 10 abnormalities that are individually associated with 29 unique spatial locations (anatomies) at the report level. Another CXR dataset has automatically extracted abnormality and anatomy labels as disconnected concepts that are only correlated at the study level from  160,000 reports using a supervised NLP algorithm \cite{bustos2020padchest}. This was trained on a smaller set of manually annotated data. Neither datasets contain localized annotations for the associated CXR images, nor any comparison relation annotations between sequential exams, both of which are available in the Chest ImaGenome dataset. In Table \ref{tab:related}, we present a comparison of our Chest ImagGenome dataset with other datasets available in the literature.

% Table -- Kashyap

% MEdical imaging datasets to go here: Discussed that we will only focus on cxr datasets that are available for this paper. 
% \caption{\color{red} Kashyap, feel free to continue with the table. We should remove the questionmarks and add a line for our dataset (since all others are not graph). For longer text, using abbreviations and explaining them in the caption often works better. If fill in the values is not possible, it is better to remove the table altogether.}


\begin{table}[t!]
\caption{Summary of existing chest X-ray datasets}
\resizebox{\textwidth}{!}{%
\begin{tabular}{@{}lllllllll@{}}
\toprule
\textbf{Dataset} & \textbf{Annotation Level} & \textbf{Annotation Method} & \textbf{Num Labels} & \textbf{Anatomy Labeled} & \textbf{Graph} & \textbf{Dataset Size} & \textbf{Temporal Labels} & \textbf{Reports} \\ \midrule
SIIM-ACR Pneumothorax Segmentation \cite{filice2020crowdsourcing} & Segmentation & Manual + augmented & 1 & No & No & 12,047 & No & No \\
RSNA Pneumonia Detection Challenge   \cite{shih2019augmenting} & Bounding Boxes & Manual & 1 & No & No & 30,000 & No & No \\
Indiana University Chest X-ray collection \cite{demner2016preparing} & Global & Automated & 10 & No & No & 3,813 & No & Yes \\
NIH CXR dataset \cite{wang2017chestx} & Global & Automated & 14 & No & No & 112,120 & No & No \\
PLCO \cite{team2000prostate} & Global & Automated & 24 & Yes & No & 236,000 & Yes & No \\
Stanford CheXpert \cite{irvin2019chexpert} & Global & Automated & 14 & No & No & 224,316 & No & No \\
MIMIC-CXR \cite{johnson2019mimic} & Global & Automated & 14 & No & No & 377,110 & No & Yes \\
Dutta \cite{datta2020dataset} & Global & Manual & 10 & Yes & Yes & 2,000 & No & Yes \\
PadChest \cite{bustos2020padchest} & Global & Manual + automated & 297 & Yes & No & 160,868 & No & Yes \\
Montgomery County Chest X-ray   \cite{jaeger2014two} & Segmentation & Manual & 1 & Yes & No & 138 & No & No \\
Shenzen Hospital Chest X-ray   \cite{jaeger2014two} & Segmentation & Manual & 1 & Yes & No & 662 & No & No \\  \hline \hline
\textbf{Chest ImaGenome} & Bounding Boxes & Automated & 131 & Yes & Yes & 242,072 & Yes & Yes \\
\bottomrule
\end{tabular}%
}
\label{tab:related}
\vspace{-0.4cm}
\end{table}
% removed (Derived from MIMIC-CXR \cite{johnson2019mimic}) % makes table really small

\section{Related Work}

This section presents the previous works on interactive recommendation, storytelling and narrative visualization approaches.

\subsection{Visualization for Recommendation}

Different from the topic of visualization recommendation that suggests suitable visualization formats given data or tasks, we focus on visualization approaches for recommender systems, where graph visualizations are often adopted.
%Recommender systems have been widely applied to online retailers.
%such as for images~\cite{4703261}, music~\cite{5972554}, books, arts, and movies.
%QoS-based web services~\cite{5928315}, 
%Instead of presenting recommendation algorithms from the fields of data mining and machine learning, we concentrate on related interactive approaches, where the graph visualizations are often adopted.
For example, 
Luo et al.~\cite{luo2009personalized} used hyperbolic and multi-modal view to visualize a recommendation list. 
Kermarrec et al.~\cite{kermarrec2012data} used SVD-like matrix factorization and PCA for global mapping of movie ratings from high dimensions to a two-dimensional space. 
% They used Curvilinear Component Analysis (CCA) for personalized recommender 
% list mapping for a given user.
Crnovrsanin et al.~\cite{crnovrsanin2011visual} proposed a 
task-based and information-based network representation for users to 
interact and visualize a recommendation list. 
%
Vlachos et al.~\cite{vlachos2012recommendation} used bipartite graphs and minimum spanning trees to explore and visualize recommendation results of a movie-actor dataset. 

Interactive recommendation approaches have also been developed.
Gretarsson et al.~\cite{gretarsson2010smallworlds} visualized recommended users in social networks with node-link diagram and grouped relevant nodes on 
the recommendation list in parallel layers. 
Loepp et al. presented an interactive recommendation approach by having users to choose between two sets of sample items iteratively and extracting latent factors~\cite{loepp2014choice}.
Recently, Loepp et al. ~\cite{loepp2015blended} presented MyMovieMixer for interactively expressing user preferences over the hybrid recommendation process.

%Different from existed visualization approaches for recommender systems, 
Our approach dynamically recommends suitable movies to users with the support of interactive storytelling methods.
A key feature of semantic storytelling is that users do not need to understand complex recommendation algorithms or visualization techniques.

 

\subsection{Storytelling and Narrative Visualization} 

``Storytelling" has a long history and it has become a visualization technique~\cite{Gershon:2001:SIV:381641.381653, 1626183, 6111347, 6412677, 7274435, wojtkowski2002storytelling}. 
While the term of narrative visualization is relatively new~\cite{segel2010narrative}, it also refers to using data stories to improve visual communication~\cite{hullman2011visualization, Hullman:2013:DUS:2553699.2553753, Satyanarayan:2014:ANV:2771495.2771532}.

We focus on techniques of narrative structure, which is a key concept in storytelling and narrative visualization.
It refers to ``a series of events, facts, etc., given in order and with the establishing of connections between them" from the Oxford English Dictionary and it is often simplified to structures like beginning, middle, and end in visualization systems~\cite{segel2010narrative}.
Studies from journalism~\cite{segel2010narrative}
%, videos~\cite{amini2015understanding}, 
and political messaging and decision-making~\cite{hullman2011visualization} have been performed to understand useful narrative structures for visualization.

Several interactive or automatic storytelling approaches have been developed for applications ranging from general visualization process~\cite{cruz2011generative, 4677364} 
%such as generative storytelling~\cite{cruz2011generative, 6902879}, documents~\cite{4271973}, and events~\cite{4677364}, 
to specific domains. %such as geo-visualization~\cite{Lidal:2012:GSG:2331067.2331070, 6676572}.
For example, Wohlfart and Hauser~\cite{Wohlfart:2007:STP:2384179.2384194}
used storytelling as a guided and interactive visualization presentation approach for medical visualization.
Eccles et al.~\cite{4388992} detected geo-temporal patterns and integrates story narration to increase analytic sense-making cohesion in GeoTime.
%Cruz and Machado~\cite{cruz2011generative} developed a conceptual framework to build various stories from a set of time-ordered events given a dataset.
Yu et al.~\cite{CGF:CGF1816} generated automatic animations with narrative structures extracted from event graphs for time-varying scientific visualization.
Hullman et al.~\cite{Hullman:2013:DUS:2553699.2553753} presented a graph-driven approach for automatically identifying effective sequences in a set of visualizations to be presented linearly.
Lee et al.~\cite{Lee:2013:STM:2553699.2553755} presented a storytelling process with steps involved in finding insights, turning insights into a narrative, and communicating to an audience. 
Satyanarayan and Heer~\cite{Satyanarayan:2014:ANV:2771495.2771532} developed a model of storytelling abstractions and instantiate the model in Ellipsis with a graphical interface for story authoring.
Wang et al. presented a narrative visualization system that presents literature review as interactive slides with three levels of narrative structures~\cite{Wang:2016:GTL:2968220.2968242}.
Bryan et al.~\cite{7539294} generated textual annotations with a temporal layout and comic strip-style data snapshots for visualizing multidimensional and time-varying data.


%Hullman and Diakopoulos~\cite{hullman2011visualization} demonstrated visualization rhetoric as an analytical framework for understanding the effects of design techniques on end-user interpretation.


%Andrews and Baber~\cite{andrews2014visualizing} designed a branching comic to compare how readers recall a visual narrative.
%Pschetz et al.~\cite{pschetz2014turningpoint} developed TurningPoint to investigate narrative-driven talk planning in slideware.
% and demonstrated that narrative templates allowed users to focus attention and limit experimentation at the same time.
%Spaulding and Faste used studies to prototype and build immersive design words~\cite{spaulding2013design}.
%We didn't include storytelling from videos including movies as their focus is often on imaging techniques.

Storytelling has been shown to be effective on conveying data in a number of applications~\cite{dragicevic2011temporal, spaulding2013design}.
For visualization tasks, while storytelling did not seem to increase user-engagement in exploration~\cite{boy2015storytelling}, annotated visualizations were proven to be better in balancing graphical salience and relevance~\cite{Hullman:2013:CAG:2470654.2481374} and graph comics were useful to help a general audience understand complex temporal changes quickly~\cite{bach2016telling}.
%The results are mixed~\cite{gonzalez1996does, boy2015storytelling}.
Nonetheless, it is clear that a flexible creation process should be provided and  adjusted according to application requirements~\cite{mitchell2011limits, 6902874, amini2015understanding}.

Different from other storytelling techniques, we present a highly interactive storytelling approach that simulates human communication with two features - continuous updating stories with or without user inputs and allowing interaction in all stages of exploring data, making a story, and telling a story.


%\input{narrative}
\section{Latent Semantic Model for Interactive Recommendation and Movie Exploration} 
\label{latent}

Our interactive recommendation approach for general users is consisted of two components: LSM for abstracting personal movie preferences (Section 3) and interactive recommendation with storytelling (Section 4).
In this section, we start by introducing the latent space from collaborative filtering algorithms.
We then describe our recommendation approach and measurement of recommendation degrees.
At the end, we present the LSM, which transforms high-dimensional data statistics into recommendation domains according to a set of semantic concepts. 
The LSM is also used to design the recommendation storytelling and user interaction in Section 4.
During this work, the designs of LSM and interactive recommendation system are simultaneously proceeded to ensure that the same set of semantic concepts can be used for both interactive recommendation and exploration of movies.


\subsection{Latent Space from Collaborative Filtering}

To provide an effective interactive recommendation system, we need to integrate a recommendation algorithm into the visualization mechanism.
The latent factor models are the primary approaches of Collaborative Filtering (CF) techniques, which have been successfully adopted by a number of commercial systems~\cite{Koren:2008:FMN:1401890.1401944}.
The latent factor models based on Singular Value Decomposition (SVD) establish recommendations by transforming both movies and users to the same latent factor space, thus making them directly comparable. 
We choose this latent space as it can be used to explore not only recommended movies, but also the distribution patterns of movies and users from the aspect of semantic analysis.

The latent space can be used to interpret a number of preference / relevance features. 
For example, a dimension from comedy to drama can be used to represent the taste of a user favoring these two movie genres.  
In majority of cases, the latent space captures statistical distribution of rating records that combine features from all users, which are often hard to describe or understand directly.
To distinguish users from movies, we reserve special indexing letters of $u$ and $v$ for users and $i$ and $j$ for movies. 
A rating $r_{ui}$ indicates the preference of a user $u$ on a movie $i$, where rating values are in the set of $\{1, 2, ..., 5\}$ with $1$ for no interests to $5$ for strong interests. 

We generate the latent space with the factorization of user-movie rating matrix using SVD. 
For a user-movie matrix $M$ with $m$ users and $n$ movies, the SVD algorithm factorizes $M$ into three matrices such that $M = USV^T$.
It is common to truncate these matrices to yield $U_k$, $S_k$, and $V_k$, in order to decrease the dimensionality of the vector space, and only leave the strongest effects in the model by dropping dimensions with small singular values~\cite{Ekstrand:2011:CFR:2185827.2185828}.
Specifically, the rows of the $U_k$ are the interests of users in each of the $k$ inferred features, and the columns of the matrix $V_k$ are the relevance of movies for each feature. 
The diagonal matrix $S_k$ contains the $k$ biggest singular values of $M$, which are the weights for the preferences, representing the influence of a particular topic on user-movie preferences. 


\subsection{Recommendable Movies and Degrees}

%We filter the dataset by removing users without any common ratings. 

In a recommendation approach, we are interested in two things the most: recommendable movies and recommendation degrees for interpreting how likely we would recommend a movie.
We follow a typical recommendation algorithm by adjusting rating history with the normalization of global effects. 
This step balances the tendencies that some users like to give higher ratings than others and some movies receive higher ratings than others, therefore the adjusted ratings $\hat{r_{ui}}$ are more accurate to compare different users or different movies.
Denote $a_{u}$ as the average rating given by user $u$, $a_{i}$ as the average rating of movie $i$, and $A$ and $B$ as the averages of all users and all movies respectively. 
\begin{equation}
\hat{r_{ui}} = r_{ui} - (a_{u} - A) - (a_{i} - B)
\end{equation}

Our recommendation algorithm starts by mapping each user into the latent space by multiplying the adjacency matrix $M$ and the $V_k$ component of SVD.
We denote this product as $C=M \times V_k$, where $C_u$ represents a row vector for user $u$ from the matrix $C$.
We then compute the cosine similarity $s_{uv}$ between users $u$ and $v$ %using Pearson Correlation coefficient 
with the $C_u$ and $C_v$ coordinates~\cite{leskovec2014mining}.
\begin{equation}
s_{uv} =  \frac{\displaystyle\sum_{l=1}^{k} {C_u}_l \times {C_v}_l}
  			 {\sqrt{\displaystyle\sum_{l=1}^{k} ({C_u}_l)^2 } \times 
  			  \sqrt{\displaystyle\sum_{l=1}^{k} ({C_v}_l)^2 } }    
\label{similarity}     
\end{equation}

Next, we select a list of similar users with positive similarity coefficients, as recommendable movies are generally selected based on ratings from similar users.
We denote this set as $S_u=\lbrace v | s_{uv} \geq 0 \rbrace $ for user $u$.
For a new user with no rating record, all the similarity coefficients $s_{uv}$ are zero and the set $S_u$ contains all the users.

We further select the list of recommendable movies $L_u$ for the user $u$ by choosing movies that have not been watched by $u$, but received positive ratings from similar users as follows, where the positive rating threshold $w_c$ is initialized as $3$ and can be adjusted for different numbers of recommendable movies.
\begin{equation}
L_u=\lbrace i | v \in S_u, \hat{r_{vi}} \ge w_c \ and \ r_{ui}=0 \rbrace. 
\end{equation}

In addition, we measure the recommendation degree of a movie $i$ for user $u$ by $b_{ui}$.
The movies with high $b_{ui}$ degrees are more likely to be recommended during the interactive recommendation process.
%We use it to reduce the size of $L_u$ by keeping movies with large $b_{ui}$ degrees and select movies during the interactive recommendation process.
It is calculated as the average rating from similar users with positive cosine similarities. 
For any $i \in L_u$,
\begin{equation}
b_{ui} = \sum_{v \in S_u \ and \  r_{vi} \ne 0} \{r_{vi} \times 
s_{uv}\} / \sum_{v \in S_u \ and \ r_{vi} \ne 0} \{r_{vi}\}
\label{ratingScore}
\end{equation}
We normalize the $s_{uv}$ and $b_{ui}$ values to range of $[-1, 1]$ and $[0,1]$ respectively to balance the differences among users.

%For example, when we set $w_c$ to 3 for a sample user, we have $b_{ui}$ ranging from 0.199354(for `` Ayn Rand: A Sense of Life (1997)'', a documentary movie) to 0.012575 (for`` Showgirls (1995)'' which is a drama movie).
%The full result of our model is shown in Figure~\ref{svdSemantic}.

\begin{figure}
\centering
\includegraphics[width = 3.4in]{model1v.eps}\\
LSM mode (a) -- Variations based on the Like/Dislike groups\\
\includegraphics[width = 3.4in]{model2.eps}\\
LSM mode (b)
\caption{The LSM maps the distribution of a set of semantic concepts in the latent space.
We separate the two modes of LSM for clarity -- (a) includes concepts of like, dislike, familiarity and diversity and (b) includes concepts of typicality and un-typicality.
%The two modes co-exist in all the latent dimensions and LSM chooses the right one to use automatically.
The movie nodes are colored based on the groups -- liked in green, disliked in orange, recommendable in blue, and not recommendable in black.
}
\label{svd}
\end{figure}
%(neutral in yellow), 

\subsection{Latent Semantic Model}

While the latent space has been widely used in recommendation algorithms, it is a high-dimensional space and not directly suitable for visualization or interaction. 
Our goal of the LSM is to identify semantic concepts and visualization domains that can be used in interactive recommendation and exploration of movie similarities.
%This is achieved through finding a set of semantic concepts related to recommendation from the latent space $V_k$ and studying their statistical features.
Our approach is consisted of the following three steps: separating movie groups, identifying a set of semantic concepts, and selecting suitable latent dimensions for interactive recommendation.


\subsubsection{Separating Movie Groups}

To identify recommendable movies, we separate all the movies in the database to five groups based on the rating history of the user and our estimation of recommendation degrees.
For any user $u$, each movie belongs to one and only one of the following groups.

\begin{itemize}
\vspace{-2mm}
\item Like (with positive ratings from $u$): $G_+ = \{i | {r_{ui}} \geq \tau_{+}\}$
\vspace{-2mm}
\item Dislike (with negative ratings from $u$): $G_- = \{i | 0 < {r_{ui}} < \tau_{-}\}$
\vspace{-2mm}
\item Neutral: $G_{neu} = \{i | \tau_{-} \leq r_{ui} <\tau_{+}\}$
\vspace{-2mm}
\item Recommendable (with positive recommendation degrees):\\ $G_r = \{i | b_{ui} \geq \tau_{r}\}$ and not specified as ``thumb-down" by the user during interaction recommendation process
\vspace{-2mm}
\item Not recommendable (with negative recommendation degrees): $G_n = \{i | b_{ui} < \tau_{r}\}$ or specified as ``thumb-down" by the user during interaction recommendation process
\end{itemize}

The thresholds of positive rating $\tau_{+}$ and negative rating $\tau_{-}$ are set to value $3$, and the threshold of recommendation degree $\tau_{r}$ to value $0$ initially.
The values can be adjusted to control the number of movies in each group.


\subsubsection{Semantic Concepts for Recommendation}

In everyday life, we often describe an object with several terms, such as families, friends, and enemies for a person.
Similarly, a set of semantic concepts can be used to describe movies for a quick impression, which may assist users to find movies effectively.

In addition to ``like" and ``dislike", we have identified the following four semantic concepts for recommendation based on two criteria -- concepts that are often used in recommendation for the general public; 
and concepts with clear distribution features in the latent space.
%and more importantly, concepts that can be easily understood by general users without any knowledge of recommendation or visualization techniques.

\begin{itemize}
\item Familiarity -- movie styles that the user has already watched;
%regions containing movies the user $u$ has rated with both positive and negative ratings; 
\vspace{-2mm}
\item Diversity -- movie styles that the user is not familiar with;
% other regions outside the familiar zone.
\vspace{-2mm}
\item Typicality -- movie styles that can be well defined based on combined movie features;
%regions away from the Origin of the latent space.
\vspace{-2mm}
\item Un-typicality -- movie styles that are unclear to a feature.
% the region close to Origin the latent space.
\end{itemize}

%The core of our LSM is to extract sub-latent spaces that map the distributions of recommendable movies based on the set of semantic concepts, which further enable us to construct recommendation stories and interaction mechanisms for users to adjust recommendation preferences.
Next, we describe the distribution features of semantic concepts in the latent space.
As illustrated in Figure~\ref{svd}, we can identify the familiar zone by including both the like and dislike groups -- all the movies that have been watched by the user $u$.
The diverse zones are regions outside the familiar zone and there are generally two diverse zones on each side of a latent dimension.
Figure~\ref{svdexample} uses examples from real scenarios to show the variations of the LSM mode (a).

The LSM mode (b) between typicality and un-typicality utilizes the distance of a movie to the Origin of the latent space.
Since both the $U_k$ and $V_k$ from the latent space demonstrate clustering features of similar users and movies~\cite{Pu:2013:UIR:2507157.2507178}, the distances in the latent space can interpret the similarity degrees.
For example, on a latent dimension which contains comedy movies on one side and drama movies on the other, the movies with combined comedy/drama or other genres are distributed near the Origin.
This feature is similar to spectral spaces that nodes with few connections are often located close to the Origin~\cite{6231611}.
What's important is that this feature preserves for general latent dimensions describing combined movie features or styles, with only un-typical movies corresponding to the latent dimension located close to the Origin.
As shown in Figure~\ref{svd} mode (b), any latent dimension can be modeled as typical to untypical to typical zones.
The two typical zones correspond to two opposite features represented by the latent dimension.

While the distribution features of semantic concepts on the latent dimensions are clear, the relative locations among the semantic zones 
vary depending on the Origin and the familiar zone.
For example, the Like and Dislike zones can appear on either side of the Origin or Overlap with the Origin.
%As shown in Figures~\ref{svd} and \ref{svdexample}, the zones of familiarity and diversity do not overlap, while the zones of familiarity/diversity and zones of typicality/un-typicality are independent.
Nonetheless, they all provide valuable semantic information for interactive recommendation.
We generally adjust thresholds of $\tau_{+}$, $\tau_{-}$, and $\tau_{r}$ to make sure that all the semantic zones contain recommendable movies.

\begin{figure}
\centering
%\includegraphics[width = 3in]{./image/Case1.eps}\\
%\includegraphics[width = 3in]{./image/Case1Bis.eps}\\
%\includegraphics[width = 3in]{./image/Case2.eps}\\
%\includegraphics[width = 3in]{./image/Case4.eps}
\includegraphics[width = 3.4in]{BestDim1.eps}\\
\vspace{+1mm}
\includegraphics[width = 3.4in]{BestDim2.eps}\\
\vspace{+1mm}
Good latent dimensions for visualization and interaction\\
\vspace{+1mm}
\includegraphics[width = 3.4in]{WorstDim1.eps}\\
\vspace{+1mm}
\includegraphics[width = 3.4in]{WorstDim2.eps}\\
\vspace{+1mm}
Bad latent dimensions as visualization and interaction
\caption{Good and bad example latent dimensions that are selected based on criteria described in Section 3.3.3.
The liked and disliked zones are highlighted with shaded rectangles in green and orange respectively.
The recommendable movies are colored in purple, ``liked" movies in green, and ``disliked" movies in orange.
%The first two examples are good candidates of visualization domains, as they describe different aspects of combined user tastes.
%The last two rows are not suitable for visualization, as they do not separate the like and dislike zones well.
}
\label{svdexample}
\end{figure}
%The grey column near the center marks the Origin of the latent space.
%All the examples demonstrate that our latent semantic model can represent diverse scenarios from the real datasets.

\subsubsection{Selecting Dimensions for Visualization and Interaction}

In the latent space, each dimension describes certain joint statistical features of movies and users.
%Generally the most dominant features appear first and small features later. 
We are interested in searching for dimensions that can help users understand movie relationships and are suitable for visualization and interaction.
%
We need to select not only one, but also a set of suitable dimensions, so that multiple aspects of recommendable movies can be covered.
However, not all the dimensions in the $k$-dimensional latent space are suitable for recommendation, as they may represent features of irrelevant movies or users, repeated features from other dimensions, or features that are hard to understand.
Therefore, we go through the following process of selection based upon the distribution of the semantic concepts.

The selection is based on the combined information of user history, recommendation degrees, and explicit information from user interaction -- thumb up/down for specific movies, which is described in the section 4.
We prefer to choose the dimensions that separate the semantic zones with the following three factors:
%contain suitable distribution of recommendable movies according to the first two ideal cases of our model.
%Specifically, we consider the following three factors: sizes of important groups, overlapping relationships, and the distribution of recommendable movies.
\begin{description}[style=unboxed,leftmargin=0cm]
\vspace{-2mm}
\item[Group sizes.] 
For each dimension, we measure the region of the like group as $R_+$, the dislike group as $R_-$, the overlapped region as $R_{o}$, and the combined range as $R$.
It is ideal that $R_+$ occupies a large portion of $R$, as majority recommendable movies are selected within or close to this region.
We also prefer that $R_+$ and $R_-$ do not cover the entire dimension, so that there is room for diverse zones.
This factor is simplified as $R_+ / R$.
%Assuming $R$ is the range on the dimension, we want $R_+ / R$ to be large and $R_o / R$ to be small.
\vspace{-2mm}
\item[The overlapping region.]
We try to avoid the third and fourth cases of LSM mode (a) in Figure~\ref{svd}, which happens when one group is located inside another.
They are less desirable as the meanings of such dimensions are hard to describe and understand.
We set large penalty to avoid large overlapping ratios for $R_o / R_+$ and $R_o / R_-$. 
%\vspace{-2mm}
%\item[The number of recommendable movies.]
%We prefer a dimension capturing a big portion of recommendable movies, especially on the zones of $R_+ \bigcap \overline{R_o}$.
%Denoting the numbers as $N_+$ and $N_{o}$ for groups of $R_+$ and $R_{o}$.
%This factor is also computed as the ratio $(N_+ - N_o)/N$, where $N = | L_u |$.
\vspace{-2mm}
\item[The distribution of recommendable movies.]
Inside $R_+$, it is ideal that the recommendable movies are evenly distributed. %, such as the Normal distribution.
This factor helps to remove the dimensions with many movies mapping to a small range, which indicates that these movie features and differences are not well represented on the dimensions.
%We can use 68?95?99.7 rule to measure if the distribution is normal.
%https://en.wikipedia.org/wiki/68%E2%80%9395%E2%80%9399.7_rule
We measure this factor with standard deviation $\theta_+$ of recommendable movies in $R_+$.
\end{description}

Specifically, we use the following $D_v$ equation to measure if a latent dimension $v$ is suitable as a visualization domain, where $w_+$, $w_{o}$ and $w_{\theta}$ are the weights for the three factors described above.
\begin{equation}
D_v =  (\frac{R_+}{R})^{w_+} \times (1 - \frac{R_o}{R_+})^{w_o}  \times (1 - \frac{R_o}{R_-})^{w_o} \times ( \frac{R_+}{\theta_+})^{w_{\theta}} 
\label{equ:dv}
\end{equation}
%D_v =  (\frac{R_+}{R})^{w_+} \times ( (1 - rRo) * (1 - rNo)) ^{w_o} \times (1 - \frac{R_o}{R_+})^{w_o}  \times (1 - \frac{R_o}{R_-})^{w_o} \times ( \frac{R_+}{\theta_+})^{w_d} 
%Dv = Math.Pow(rRl* rNl, 2) * Math.Pow( (1 - rRo) * (1 - rNo), 20) * Math.Pow(1 - Old / Rl, 10) * Math.Pow(1 - Old / Rd, 10)


%separate the like and dislike groups and spread out the movies in each group.
%Specifically, for each of the $k$ dimensions, we calculate the regions of the like group as $R_+$, the dislike group as $R_-$, and the overlapping regions as $R_{o}$.
%We also consider the significance of a latent dimension through $S_v$ to favor important dimensions in the latent space.
%The following $D_v$ measures if a latent dimension $v$ is suitable for a plot candidate:
%\begin{equation}
%D_v = (w_+ \times R_+ \times w_{+b} + w_- \times R_- \times (1 - w_{-b}) + w_{o} \times R_{o}) \times {|S_v|}^{w_s}
%\label{equ:dv}
%\end{equation}
%where $w_+$, $w_-$, $w_{o}$, and $w_s$ are the weights for importances of groups $R_+$, $R_-$, and $R_{o}$ and $S_v$ factor respectively. 
%Also, we use $R_{+l}$ and $R_{+h}$ to define the low and high boundaries of the like group, and similarly $R_{-l}$ and $R_{-h}$ for dislike group.
%We use large $w_+$ to emphasize the distribution features in the like group and large $w_{o}$ to reduce the overlap region between two groups.
%\begin{eqnarray}
%\nonumber
%R_{o} &=& |R_{-h} - R_{+l}| - (R_+ + R_-), if (R_{+l} \leq R_{-l})\\
%&& |R_{+h} - R_{-l}| - (R_+ + R_-), otherwise.
%\end{eqnarray}
%The $w_{+b}$ and $w_{-b}$ are the effects of recommendation degrees and detect if $b_{ui}$ distributions are consistent with the groups of $R_+$ and $R_-$.
%As in the following equation, we define $Z_+$ as the set of recommendable movies in $R_+$ and $Z_-$ as the set of recommendable movies in $R_-$ on dimension $v$.
%\begin{equation}
%\nonumber
%w_{+b} = \frac{\sum_{i \in Z_+} {b_{ui}}}{| Z_+ |}
%\ \ \ \ \ \ \ \ 
%w_{-b} = \frac{\sum_{i \in Z_-} {b_{ui}}}{| Z_- |}
%\end{equation}

Next, we filter the set $S_{d} = \{v | D_v > \tau_{v} \}$, composed of latent dimensions with high $D_v$ values, by removing similar dimensions.
This is achieved by comparing the locations of movies to the groups on the two dimensions.
Among the similar dimensions with high $D_s$ values, only the dimension with the highest $D_v$ value is kept in $S_{d}$.
\begin{equation}
%D_s = \sum_{i \in G_l \cup G_d, p, q \in G_{s}} w_i \times 
%|\frac{|l_{ip} - lc_{p}|}{R_{lp}} - \frac{|l_{iq} - lc_{q}|}{R_{lq}}|
D_s(p, q) = \sum_{i \in L_u} w_i \times 
[(i \in R_{+p}) \bigoplus (i \in R_{+q}) + (i \in R_{-p}) \bigoplus (i \in R_{-q})]
\label{equ:ds}
\end{equation}
where, $p \in S_{d}$ and $q \in S_{d}$ are latent dimensions, $w_i$ is the weight for each movie to incorporate user preference, $R_{+p}$ and $R_{-p}$ are the ranges of like and dislike groups on $p$.
%$G_l$ and $G_d$ are the set of all recommended items for which the user clicks the thumb-up/down buttons.
We set $w_i = 1$ for all movies initially and double the value when the user clicks the thumb-up/down buttons.

%We sort the set $G_{s} = \{v | D_v > th_{ds} \}$ of selected dimensions according to $D_v$ and always choose the dimension with the highest $D_v$ value.
%Then, we continue to filter the set $G_{s}$ by comparing the relative locations of movies to the group centers.
%Dimensions with low $D_s$ values are skipped, until we finish all the $k$ dimensions.

%This give us a small set of dimensions to use, each can be described using several example movies the user has watched.

%We run our model using MovieLens dataset after downloading movies' images from IMDb web site. 
%Figure~\ref{svdSemantic} shows an example of latent dimension describing user taste of movies. 
%The movies range from drama, horror, love, success, and city-dweller types (left) to drama, love, life after war, finance struggle, and country types (right). 
%Through our model, we can summarize that this user prefers the movies about success and stories in large cities over topics of poor, struggle and rural areas. 

%who is a 32 year old man working as administrator in Boston city, Massachusetts,
%the second best dimension selected based on Dv score corresponding to the 166th dimension of SVD.
%The user's liked and disliked zones are separated on this dimension. 

%\begin{figure*}
%\centering
%\includegraphics[width = 7.0in]{./image/addedImages/BestDim2Semantic.eps}
%\caption{The semantic of SVD dimensions. SVD second best dimension selected by our model.
%The leading movies in this dimension range from drama, horror, love, success, and city-dweller type of movies (on the left) to drama, love, life after war, finance struggle, and country type of movies.
%The contrast between both sides is very clear. 
%}
%\label{svdSemantic}
%\end{figure*}

\subsection{Model Validation}

We validate LSM with a real dataset from two aspects.
First, we test LSM on all the users in the MovieLens 100K dataset~\cite{MovieLens100k}.
For all the results in this paper, we use 4 for $\tau_+$, 2 for  $\tau_-$, 5 for $w_+$, and 10 for $w_o$ and $w_{\theta}$.
The best latent dimensions are automatically selected and used to measure how well LSM can distinguish the like and dislike regions.
The results show that the best dimensions for all the users contain ideal group distributions in either case 1 or case 2 of Figure~\ref{svd} (a), indicating that LSM can be applied to users with various rating histories.

Second, we observe the distributions of recommendable movies on the best latent dimensions.
Since every recommendable movie can be in the search results, we collect the total amount of $b_{ui}$ for all movies in a group. 
It is worth mentioning that the dislike region may also contain recommendable movies, as the choices of recommendation come from multiple aspects. % - different combined movie features / dimensions in our work.
Also, cases like two users rated one movie similarly while rating another movie very differently can complicate the statistical distributions.
As shown in the table~\ref{tablevalid}, the average in $R_+$ of all users is significantly higher than that of $R_-$.
If we remove the overlapping regions from $R_+$ and $R_-$, the difference is more significant. 
We also compute the Pearson correlation to compare the values from two pairs in statistics.
By removing the overlapping regions, the second pair is very close to no correlation.
This result shows that LSM captures the majority recommendable movies in the like region for making recommendations.

\vspace{-4ex}
\begin{table}[htb]
\caption{Comparison of the sum of $b_{ui}$ in different regions}
\vspace{1ex}
\label{tablevalid}
\centering
\begin{tabular}{| c | c | c | | c | c | c |}
\hline
$\sum_{i \in R_+}$ & $\sum_{i \in R_-}$ & Pearson & $\sum_{i \in R_+ \bigcap \overline{R_o}}$ & $\sum_{i \in R_- \bigcap \overline{R_o}}$ & Pearson \\
\hline
151.0 & 101.7 & 0.54 & 62.5 & 13.1 & -0.05\\
\hline
\end{tabular}
\end{table}
\vspace{-2ex}

  % Sum Bui Like: 38209.748244556 
  % Sum Bui Like Without Overlap: 15805.6755926232 
  % Sum bui Dislike: 25722.6828700504 
  % Sum Bui Dislike Without Overlap: 3318.61021811758 
  % Sum Bui Overlap: 22404.0726519328 
  % Avg Std Like: 0.0411774222792137 
  % Avg Std Dislike:  309160143995338 
  % Average Bui Like: 0.802955912706401 
  % Average Bui Dislike: 0.772311416433231

%Specifically, we measure whether the distributions of $b_{ui}$ degrees among the $R_+$ and $R_-$ zones are consistent with the like and dislike groups.
%We expect that majority recommendable movies with high $b_{ui}$ degrees should be inside $R_+$ instead of $R_-$.
%For all the users, we randomly sample dimensions with different $D_v$ values and compute the average $b_{ui}$ degrees of the top-K recommendable movies.
%As shown in Figure~\ref{validation}, the averages of $b_{ui}$ are high for $R_+$ and low for $R_-$, indicating our model is consistent with movie distributions on the latent dimensions.
%The result also shows that this feature is preserved for dimensions with high $D_v$ values, which often have small singular values.
%This is because dimensions with large singular values capture statistical features of the entire dataset; therefore features for individual users are often found in minor dimensions.

%\begin{figure}
%\caption{The correlations among the averages $b_{ui}$ of the top-K recommendable movies in $R_+$, $R_-$, $D_v$, and singular values validate our latent semantic model.}
%\label{validation}
%\end{figure}

%%The second approach is that similar users with high $s_{uv}$ have the same SVD dimensions selected (high $D_v$ values).
%We also validate the semantic features of latent dimensions for representing user features.
%For each user in the dataset, the dimension with the highest $D_v$ value is used to represent the combined features of movies for the user, including both liked and disliked movies.
%One latent dimension may be selected for multiple users.
%To judge whether the users have similar tastes of movies when they share the best latent dimension, we measure the average coefficient $s_{uv}$ in equation~\ref{similarity} of all user pairs with the same best latent dimension.
%The largest $s_{uv}$ is mapped to $1$.
%The average is ??, much higher than the average of the all $s_{uv}$ values.


%\input{visualization}
\section{Interactive Recommendation through Storytelling}

We start this section by describing how we connect the process of recommendation to storytelling.
Then, we present our strategies to construct recommendation stories automatically through the LSM from the aspects of characters, roles, user interaction and narrative structures respectively.


\subsection{Connecting Recommendation to Storytelling}

As described in the introduction, we propose interactive recommendation to simulate the scenario of an expert recommending movies to a user.
During recommendation, the expert generally presents one or several movies and provides reasons of recommendation, such as high popularity, high similarity to a movie the user likes, or special features.
The user may respond by indicating his or her preferences on the recommended movies, such as ``recommend more movies like this" or ``no more movies like that".
This process continues until the user finds an interesting movie to watch.

To simulate this communication process, we present a storytelling mechanism that treats the procedure of recommending movies as telling a story.
We design a \textbf{recommendation story} as a set of recommendable movies and brief reasons of recommendation and an \textbf{interactive recommendation} approach as a continuous storytelling process, which can be adjusted promptly with user interaction.
As shown in Figure~\ref{IT}, the interactive storytelling pipeline starts with exploring movie database for a user, selecting recommendable movies, and collecting necessary information using LSM described in Section 3.
Then, the second step of ``make a story" is to generate a recommendation story automatically with the approach described in this section.
The third step ``tell a story" is to present a story as an animation sequence described in Section 5.

The loop among ``make a story", ``tell a story", and ``user" provides the proposed continuous and interactive process of recommendation until a desirable movie is identified.
The arrow from ``user" to ``make a story" indicates that the user can provide feedback to request new recommendation stories that reflect user inputs.
The arrow from ``user" to ``tell a story" indicates that the user can  adjust the storytelling animation interactively, such as replaying a recommended movie or finishing the story immediately.
If the user does not provide an input, the loop continues to different recommendation stories to achieve the effect of ``how about some other types of movies you may like?"
The automatic switch between different recommendation stories can avoid users getting bored from similar movies.
During the process, the user can also use our interaction tools to explore additional information of movies.

Different from the previous storytelling processes~\cite{7274435} that are mainly an ordered sequence of the three components - explore data, make and tell a story, our interactive recommendation approach is supported by 1) interaction functions that allow a user to interact with the storytelling pipeline at any time during the process, 2) automatic construction of narrative structure that allows new and adjusted recommendation stories to be generated continuously.


\begin{figure}
\centering
\includegraphics[width = 3in]{flow.eps}
\caption{Pipeline of storytelling with continuous communication between the user and our interactive recommendation system.
Users can interact with the system at any time to indicate preferences and explore movie information to accelerate recommendation process.
}
\label{IT}
\end{figure}

\subsection{Character}

The character in an ordinary story is generally a person. 
In our recommendation stories, the characters are movies.
Similar to the human characters, each movie character maintains a different relationship with the user, such as a rated movie, a favorite movie, or a recommendable movie.
The movie characters are also related to each other, such as being rated by the same users or have similar rating averages.
Since our focus is recommendation, the leading characters are recommendable movies, whose relationships with the user can be represented with LSM and recommendation degrees.

\subsection{Role}

A role describes what function each character serves in the story.
The roles of recommendable movies provide a mechanism for users to explore the movie database. 
Before going through the movie details, the roles of a movie provide a quick catch of the movie features, such as a typical drama movie which is very similar to one of the user's favorite movies.
This provides the user a quick way to find several interested movies to explore.
 

In recommendation stories, we use the semantic concepts from LSM to describe the roles, such as a ``liked" movie that the user rated high or a ``typical" movie that has strong features of certain movie genres.
Each movie character can play multiple roles, such as familiarity and un-typicality, just like a human character can be both a colleague and a friend. 
The actual roles of a movie character are determined by the locations among the semantic zones on selected latent dimensions, as shown in Figure~\ref{svd}.


\subsection{User Interaction During Recommendation}

For interactive recommendation, the user interaction becomes an important component of the storytelling pipeline.
To allow active user interaction with the storytelling pipeline shown in Figure~\ref{IT}, we provide three groups of explicit interaction functions as follows:
%we provide interaction functions for adjusting the types of recommended movies with slide bars between two sets of user-friendly concepts.

The first group of interaction functions is from ``user" to "make a story". 
Corresponding to the roles of a movie, users can specify the preferred movie types between options of familiar ($f$) / diverse and typical ($t$) / untypical movies.
The parameter values become effective immediately on generating the new recommendation stories.
%We provide two slide bars for this function.
For a specific movie, the user can click thumbs-up (like) and thumbs-down (dislike) buttons, so that the specified movie is moved to the like or dislike groups (and the group of not recommendable, so that it is removed from the recommendation process).
We also increase the $w_i$ value (a parameter to control the effects of user selection) of the movie in equations~\ref{equ:dv} and ~\ref{equ:ds} for choosing latent dimensions for new stories.
For each dimension $v$, the new measurement $D'_v$ contains components from both data distribution by $D_v$ and user interaction. 
We detect if the user preference is aligned with LSM, especially if a liked movie is in the like range and if a disliked movie is in the disliked range.
\begin{equation}
%D'_v = D_v  + w_{ui} \times (\sum_{i \in liked \& i \in R_l} {b_{ui}} - \sum_{i \in liked \& i \in R_d} {b_{ui}} + \sum_{i \in disliked \& i \in R_d} {b_{ui}} - \sum_{i \in disliked \& i \in R_l} {b_{ui}})
D'_v = D_v  + w_{i} \times (\sum_{i \in (U_+ \cap R_+) \cup (U_- \cap R_-)} {b_{ui}} - \sum_{i \in (U_+ \cap R_-) \cup (U_- \cap R_+)} {b_{ui}})
\end{equation}
where $U_+$ and $U_-$ are the sets containing all liked or disliked movies specified by the user.
%a recommendable item $i$ is liked or disliked by a user (when a user clicks on like up or dislike button) by $i \in U_+$ or $i \in U_-$ respectively.

The second group of interaction functions is from ``user" to "tell a story". 
To control the animated storytelling, the users can replay, pause, and stop the current recommendation story, or play more stories (the default is continuing to recommend additional movies).

The third group is to explore movie details.
If finding an interesting movie, the user can click the movie poster or movie node on the interface to view details.
The users can also mouse over a movie node anytime to reveal a set of basic information, including user rating, average rating, popularity, title and genres.


%The importance of using latent space from SVD is that we can transfer the above user interactions into the semantic meanings related to the movie recommendation.
%Both the recommendation results and storytelling visualization are adjusted automatically during the interactive process.


\subsection{Automatic Generation of Narrative Structures}
%\subsection{Plot - Latent Dimensions}

A narrative structure refers to the sequence of events in a story.
In recommendation stories, each event is a recommendable movie and brief reasons of recommendation. 
The sequence of a set of recommendable movies in a narrative structure is crucial to improve users' understanding of the movies effectively.

Considering the short attention of users in online recommendation, we prefer simple stories that can finish in a very short duration. Therefore, we generate each recommendation story only with one latent dimension selected with LSM.
As each latent dimension reflects one combined movie / user feature, such recommendation stories simulate the effect that we recommend movies from one combined movie feature to another, such as popular drama/comedy movies to unpopular documentary movies.
Other designs of recommendation stories using our LSM are also feasible.
For example, long stories can be generated by connecting different latent dimensions. 
Due to the focus of our interactive recommendation approach, we only use short stories in this work.

Based on a latent dimension, we try to maintain smooth story transitions by generating linear narrative structures among the four semantic concepts: familiarity, diversity, typicality, and un-typicality.
This is achieved by identifying a starting point, choosing narrative structures based on user preferences, and selecting recommendable movies.

The \textbf{starting point} of a story is determined according to user preferences of $f$ and $p$.
The default values are $0.5$ for both $f$ and $p$, although we favor typical over un-typical and familiar over diverse movies, which is consistent with the preferences of majority users.
The following list is the order we set as default.

\vspace{+1mm}
\noindent {High familiarity:} starting from the like group

\vspace{+1mm}
\noindent {High diversity:} starting from the diverse zone closer to like group

\vspace{+1mm}
\noindent {High typicality:} starting from the typical zone closer to like group

\vspace{+1mm}
\noindent {High un-typicality:} starting from the un-typical zone
\vspace{+1mm}

The \textbf{order} of recommended movies also considers user preferences.
As shown in Figure~\ref{path}, we use the four narrative structures
to cover all combinations of the two user preferences of $f$ and $t$.
The narrative structures are designed to be linear sequences, so that users can expect very similar narrative visualization during the interactive recommendation process.
Since the narrative structure is on one latent dimension each time, two zones from the LSM are involved.
The ranges of the latent dimension to select recommended movies are also shown in Figure~\ref{path}.
During the interactive storytelling process, we switch narrative structures between the two options randomly to avoid simply repeated stories.

%There is a transition when switching dimensions.

\begin{figure}
\centering
\includegraphics[width = 2.5in]{path1.eps}
\includegraphics[width = 2.5in]{path2.eps}\\
\caption{The four linear narrative structures based on the user preferences of t -- typical, u=1-t -- un-typical, f -- familiar, and d=1-f -- diverse. 
%The ranges of latent dimensions are marked with the blue parentheses. 
%For narrative structures for typical/un-typical preference, we use the half of latent dimension that overlaps with majority of the liked group.
}
\label{path}
\end{figure}

The \textbf{selection of recommended movies} is based on the sample rates computed according to user preferences of $f$ and $t$.
Assume a set $G_T$ of $T$ recommended movies is selected for each narrative structure.
When the structure is between typicality and un-typicality, we determine the number of recommended movies from the typical zone as $s_t$ and un-typical zone as $s_{u}$:
\begin{equation}
\centering
s_t = \lceil t * T \rceil; \ \ \ \ \ 
s_{u} = T - s_t
\end{equation}
Similarly, when the structure is between familiarity and diversity, the sampling numbers for familiar zone $s_f$ and diverse zone $s_d$ are: 
\begin{equation}
\centering
s_f = \lceil f * T \rceil; \ \ \ \ \ 
s_d = T - s_f
\end{equation}

Inside each zone, we select recommended movies with the following procedure that is composed of a local sampling and a random test procedure.
We first randomly pick a location $l$ and choose the best candidate within a local window by combining all three factors: the recommendation degree $b_{ui}$ of a candidate movie $i$, the distance of movie $i$ to location $l$, and the similarity of movie $i$ to the set of $q$ movies that are specified by the user.
On the latent dimension $V_p$, assume the location of movie $i$ is $V_p(i)$.
The effect of a user specified movie is set to be within a location window $\delta$ with a truncated Gaussian function $G_q()$, with high weights for thumb-up movies and low weights for thumb-down movies.
%The $D_u$ is a field generated according to user interaction.
%It is for emphasizing the results made during the interaction. 
%We use two shapes to differentiate movies that are liked or disliked by the user.
The movie $i$ with the highest value from the following combined measurement is selected as the best candidate.
\begin{equation}
b_{ui} \times \overbrace{ G_1(| V_p(u_1) - V_p(i) |) \times ... \times G_q(| V_p(u_q) - V_p(i) |)}^{q = | U_+ \cup U_- |} / | V_p(i)- l |
\end{equation}

%\begin{figure}
%\centering
%\includegraphics[width = 1.3in, height = 0.4in]{./structure/like.eps} \ \ \ \ \
%\includegraphics[width = 1.3in, height = 0.4in]{./structure/dislike.eps}
%\caption{Functions used to emphasize user inputs for thumb-up and thumb-down.}
%\label{up-down}
%\end{figure}

The purpose of an additional random test is to ensure that our selections of recommendable movies are consistent with both user preferences of $f$ and $t$, although only one factor is used to determine the narrative order.
If the selected movie $i$ passes the random test, we add it to the selected set $G_T$; otherwise we pick another random location and perform the local sampling again.
For example, for the narrative structure between familiarity and diversity, we try to maintain an average typicality value close to the user preference $t$.
We measure the typicality value of a movie $i$ as $t(i) = | V_p(i) |$.
The test is determined by if the movie $i$ can make the average typicality value closer to $t$.
\begin{equation}
| \frac{(\sum_{i \in G_T}{t(i)}) + t(i)} {| G_T | + 1} - t | < | \frac{\sum_{i \in G_T}{t(i)}} {| G_T |} - t |
\end{equation}
Similarly, for the narrative structure between typicality and un-typicality, we measure the familiarity value of a movie $i$ given the center location of the like group $c_+$ as $f(i) = | V_p(i) - c_+ |$.
The random test is performed by replacing the $t(i)$ with $f(i)$ in the above equation.

Since the movie database is generally large, we can assume that there are always enough movies to recommend.
In the cases that the recommendable movies run out, we can adjust the parameters $\tau_+$ and $\tau_-$ to include additional movies.

%%!TEX root = main.tex
\section{The Story Forest System}
\label{sec:system}


\begin{figure*}
\includegraphics[width=6.7in]{figure/System}
\caption{An overview of the system architecture of \textit{Story Forest}.}
\label{fig:system}
\vspace{-0mm}
\end{figure*}

%In this section, we provide an overview of the proposed \textit{Story Forest} system.
%Then we describe our detailed procedures of extracting events from a news corpus of large amounts of real-world text data in each time period, organizing related events into stories, and modeling stories' evolutionary structure by story trees.



%\subsection{System Overview}
%\label{subsec:system-overview}

An overview of our \textit{Story Forest} system is shown in Fig.~\ref{fig:system}, which mainly consists of three components: preprocessing, document clustering and story tree update, divided into 5 steps. First, the input news document stream will be processed by a variety of NLP and machine learning tools, mainly including document filtering, word segmentation and keyword extraction. Second, steps 2--3 will cluster documents into events in a novel 2-layer procedure as follows.
For news corpus $\mathcal D_t$ in each time period $t$, we form a keyword graph \cite{sayyadi2013graph} from these documents based on keyword co-occurrence, and extract topics as subgraphs from the keyword graph using community detection algorithms. The topics with few keywords will be discarded. After each topic is found, we find all the documents associated with the topic, and further cluster these documents into events through a semi-supervised document clustering procedure aided by a pre-trained document-pair relationship classifier.
Finally, in steps 4--5 we update the story trees (formed previously) by either inserting each discovered event into an existing story tree at the right place, or creating a new story tree if the event does not belong to any existing story. Note that each topic may contain multiple story trees and each story tree consists of logically connected events.
We will explain the design choices of each component in detail in the following.


\subsection{Preprocessing}
\label{subsec:preprocessing}
When a new set of news documents arrives,  we need to clean, filter documents, and extract features that will be helpful to the steps that follow. 
Our preprocessing module mainly includes the following three steps, which are critical to the overall system performance:

\textbf{Document filtering}: unimportant documents with content length smaller than a threshold (20 characters) will be discarded.

\textbf{Word segmentation}: we segment the title and body of each document using Stanford Chinese Word Segmenter \textit{Version 3.6.0} \cite{chang2008optimizing}, which has proved to yield excellent performance on Chinese word segmentation tasks. Note that for data in a different language, the corresponding word segmentation tool in that language can be used instead. 

% \textbf{Document topic classification}: we trained SVM classifiers to classify each document into one of $30$ different topic categories,  including politics, sociology, entertainment, finance, etc., based on the document's TF-IDF feature. The one with the maximum classification score will be selected.

\textbf{Keyword extraction}: extracting keywords from each document to represent the main concepts of the document is quite critical to the performance and efficiency of the entire system. We found that traditional keyword extraction approaches, such as TF-IDF based keyword extraction and TextRank \cite{mihalcea2004textrank}, are not sufficient to achieve good performance for real-world news data. For example, the TF-IDF based method measures each word's importance by frequency information; it cannot detect keywords that yet have a relatively low frequency. The TextRank algorithm utilizes the word co-occurrence information and is able to handle such cases. However, its efficiency is relatively low, with time cost increasing significantly as the document length increases.
% Alternatively, we may manually design a fine-tuned rule-based system to combine different strategies based on the observed results. However, such type of systems highly relies on the rule design, and often introduces other systematic errors.


\begin{table}
  \caption{Features for the classifier to extract keywords.}
  \label{tab:features}
  \begin{tabular}{lp{5.5cm}}
    \toprule
    Type & Features\\
    \midrule
    Word feature & Named entity or not, location name or not, contains angle brackets or not. \\
    Structural feature & TF-IDF, whether appear in title, first occurrence position in document, average occurrence position in document, distance between first and last occurrence positions, average distance between word adjacent occurrences, percentage of sentences that contains the word, TextRank score.\\
    Semantic feature & LDA\tablefootnote{We trained a $1000$-dimensional LDA model based on news data collected from January 1, 2016 to May 31, 2016 that contains $300,000+$ documents.
    % The training process costs $30$ hours.
    }\\
    \bottomrule
  \end{tabular}
  \vspace{-3mm}
\end{table}

\begin{figure}
\includegraphics[width=3.0in]{figure/KeywordClassify}
\caption{The classifier to extract keywords.}
\vspace{-1mm}
\label{fig:keywordClassify}
\vspace{-3mm}
\end{figure}

To efficiently and accurately extract keywords, we constructed a supervised learning system to classify whether a word is a keyword or not for a document.
% In particular, we trained an $1000$-dimensional LDA model based on 6 months of news documents (the dataset here is not the one we used in the evaluation in Sec.~\ref{sec:eval}).
In particular, we manually labeled the keywords of $10,000+$ documents, including $20,000+$ positive keyword samples and $350,000+$ negative samples.
Table~\ref{tab:features} lists the main features that we found critical to the binary classifier.

%As we combined different types of features, feature preprocessing must be carefully designed to improve the performance of classifiers such as Logistic Regression (LR) or Support Vector Machine (SVM).
A straightforward idea is to input the raw features listed above to a Logistic Regression (LR). However, as a linear classifier, LR relies on careful feature engineering.
To reduce the impact of human judgement in feature engineering, we combine a Gradient Boosting Decision Tree (GBDT) with the LR classifier to get the binary yes/no classification result, as shown in Fig. \ref{fig:keywordClassify}. GBDT, as a nonlinear model, can automatically discover useful cross features or feature combinations from raw features and discretize continuous features. 
The output of the GBDT will serve as the input of the LR classifier. Finally, the LR classifier will determine whether a word is a keyword or not for the document in question. We also tried SVM as the classifier in the second layer instead of LR and observed similar performance. Our final keyword extraction precision and recall rate are $0.83$ and $0.76$, while they are $0.72$ and $0.76$ respectively if we don't add the GBDT component.

\subsection{Document Clustering and Event Extraction}
\label{subsec:eventClustering}

% \begin{algorithm}
% \caption{Graph-based Document Clustering to Obtain Events}\label{alg:graph-cluster}
% \KwIn{A set of news documents $\mathcal{D} = \{ d_1, d_2, ..., d_{|\mathcal{D}|}\}$, with extracted features described in Sec. \ref{subsec:preprocessing}. }
% \KwOut{A set of events $E = \{ \mathcal{E}_1, \mathcal{E}_2, ..., \mathcal{E}_{|E|} \}$.}

% \begin{algorithmic}[1]
% 	\STATE Construct a keyword co-occurrence graph $\mathcal{G}$ of all documents' keywords $w_i$. There is an edge $e_{i,j} = <w_i, w_j>$ if the times that the keywords $w_i$ and $w_j$ co-occur exceed a certain threshold, and $\Pr\{w_j | w_i\},\ \Pr\{w_i | w_j\}$ are bigger than another threshold. 
	
% 	\STATE Split $\mathcal{G}$ into a set of small and strongly connected keyword communities $C = \{\mathcal{C}_1, \mathcal{C}_2, ..., \mathcal{C}_{|C|} \}$, based on the community detection algorithm \cite{ohsawa1998keygraph}. The algorithm keeps splitting a graph by iteratively delete edges with high betweenness centrality score, until a stop condition is satisfied. 

% 	\FOR{each keyword community $\mathcal{C}_i,\ i= 1,\ldots,|C|$}
% 		\STATE Retrieve a subset of documents $D_i$ which is highly related to this keyword community by calculating the cosine similarity between the TF-IDF vector of each document and that of the keyword community, and comparing it to a  threshold.

% 		\STATE Divide $D_i$ into clusters according to the document topics.

% 		\STATE Further split each cluster into events by comparing the titles of each pair of documents, after word segmentation and dropping stop words.
		
% 		%each pair of documents' title keywords. %Until this step, each document cluster is an event $\mathcal{E}$.
% 	\ENDFOR

% \end{algorithmic}
% \end{algorithm}

After document preprocessing, we need to extract events. Event extraction here is essentially a fine-tuned document clustering procedure to group conceptually similar documents into events. Although clustering studies are often subjective in nature, we show that our carefully designed procedure can significantly improve the accuracy of event clustering, conforming to human understanding, based on a manually labeled news dataset.
% Algorithm~\ref{alg:graph-cluster} shows the detailed steps of such document clustering.
To handle the high accuracy requirement for long news text clustering, we propose a $2$-layer clustering approach based on both keyword graphs and document graphs.

\textit{First}, we construct a large keyword co-occurrence graph \cite{sayyadi2013graph} $\mathcal{G}$. Each node in $\mathcal{G}$ is a keyword $w$ extracted by the scheme described in Sec.~\ref{subsec:preprocessing}, and each undirected edge $e_{i,j}$ indicates that $w_i$ and $w_j$ have ever co-occured in a same document. 
Edges that satisfy two conditions will be kept and other edges will be dropped: the times of co-occurrence shall be above a minimum threshold (we use $3$ in our system), and the conditional probabilities of the occurrence $\Pr\{w_j| w_i\}$ and $\Pr\{w_i | w_j\}$ also need to be bigger than a predefined threshold (we use $0.15$), where the conditional probability $\Pr\{w_j| w_i\}$ represents the probability that $w_j$ occurs in a document if the document contains word $w_i$.

\textit{Second}, we perform community detection in the constructed keyword graph. This step aims to split the whole keyword graph $\mathcal{G}$ into communities $C = \{\mathcal{C}_1, \mathcal{C}_1, ..., \mathcal{C}_{|C|}\}$, where each community $\mathcal{C}_i$ contains the keywords for a certain topic (to which multiple stories may be associated). 
The benefit of using community detection in the keyword graph is that each keyword can appear in multiple communities, which makes sense in reality. 
We also tried another method of clustering keywords by \textit{Word2Vec}.
However, the performance is worse than community detection based on co-occurrence graphs. The reason is that using word vectors tends to cluster the words with similar semantic meanings. However, unlike articles in a specialized domain, in long news documents in the open domain, it is highly possible that keywords with different semantic meanings can co-occur in the same event.

To detect keyword communities, we utilize the \emph{betweenness centrality score} \cite{sayyadi2013graph} of edges to measure the strength of each edge in the keyword graph. An edge's betweenness score is defined as the number of shortest paths between all pairs of nodes that pass through it. An edge between two communities is expected to achieve a high betweenness score. Edges with high betweenness score will be removed iteratively to extract communities. The iterative splitting process will stop until the number of nodes in each sub-graph is smaller than a predefined threshold, or until the maximum betweenness score of all edges in the sub-graph is smaller than a threshold that depends on the sub-graph's size. We refer interested readers to \cite{sayyadi2013graph} for more details about community detection.

After we obtain the keyword communities, we calculate the cosine similarity between each document and a keyword community.  The documents are represented by TF-IDF vectors. As a keyword community is a bag of words, it can also be considered as a document. We assign each document to the keyword community which gives the highest similarity and the similarity is above a predefined threshold. Up to now, we have finished document clustering in the first layer, i.e., the documents are grouped according to topics. 

\textit{Third}, we further perform the second-layer document clustering within each topic to obtain fine-grained events. We also call this process \emph{event clustering}. An event only contains documents that talk about the same semantic event. To yield fine-grained event clustering, unsupervised learning is not sufficient. 
Instead, we adopt a supervised-learning-guided clustering procedure in the second layer.


Specifically, we train an SVM classifier to determine whether a pair of documents are talking about the same event or not using a bunch of document-pair features as the input, including the cosine similarities of content TF-IDF and TF vectors, the cosine similarities of title TF-IDF and TF vectors, the similarity of the first sentences in the two documents, etc.

For each pair of documents within a same topic, we decide whether to connect them or not according to the prediction made by the document-pair relationship classifier mentioned above. Hence, the documents in each topic will form a document graph. We then apply the same community detection algorithm mention above to such document graphs. 
Note that the graph-based clustering on the second layer is highly efficient, since the number of documents contained in each topic is significantly smaller after the first-layer document clustering. 

In a nutshell, our 2-layer scheme groups documents into topics based on keyword community detection and further groups the documents within each topic into fine-grained events. For each event $\mathcal{E}$, we also record the set of keywords $\mathcal{C}_{\mathcal{E}}$ of the topic (keyword community) which it belongs to, which will be helpful in the subsequent story tree development.


%!TEX root = main.tex
\subsection{Growing Story Trees Online}
\label{sec:tree}


% According to our observations on real-world news data, a tree structure is sufficient to capture the evolving structures of most stories for breaking news and trending topics which often last for a constrained time period. Besides, as the number of event nodes increase, a graph structure will be too complex to clearly reveal the main logic flows in the stories, while a tree structure provides a clearer view of different story paths, including branches and the main thread. Furthermore, when grow trees in an online manner, it will be easy for users to identify the online change of a story, where a tree update operation is simply inserting a new event node on the right branch.


% \begin{algorithm}
% \caption{Online Story Forest Growing}\label{alg:story-structure}
% \KwIn{A stream of documents $D = \{ \mathcal{D}_1, \mathcal{D}_2, ..., \mathcal{D}_T\}$ incoming by time. Event compatibility threshold $\delta$.}
% \KwOut{A story forest $\mathcal{F} = \{\mathcal{S}_1, \mathcal{S}_2, ..., \mathcal{S}_{|\mathcal{F}|}\}$ that dynamically changes with time.}

% \begin{algorithmic}[1]
% 	\STATE Perform event clustering on the first batch of documents $\mathcal{D}_1$ to get events $E_1 = \{\mathcal{E}_1, \mathcal{E}_2, ..., \mathcal{E}_{|E_1|}\}$. 

% 	\STATE Create a story tree $\mathcal{S}_1$ from $\mathcal{E}_1$. Iteratively processing other events in $E_1$ using the same steps as described below. Then we get initial story forest $\mathcal{F}_1 = \{\mathcal{S}_1, \mathcal{S}_2, ..., \mathcal{S}_{|\mathcal{F}_1|}\}$  

% 	\WHILE{a new batch of documents $\mathcal{D}_t$ comes at time slot $t$}
% 		\STATE Perform event clustering and get a set of events $E_t = \{\mathcal{E}_1, \mathcal{E}_2, ..., \mathcal{E}_{|E_t|}\}$.

% 		\FOR{each event $\mathcal{E} \in E_t$}

% 			\STATE Match $\mathcal{E}$ with each story $\mathcal{S} \in \mathcal{F}_{t-1}$.

% 			\IF{$\mathcal{E}$ belongs to story $\mathcal{S}$}
% 				\STATE Compare $\mathcal{E}$ with each story node $\mathcal{E}_{\mathcal{S}, i} \in \mathcal{S}$ to check whether they are describing the same event. If yes, $\mathcal{E}_{\mathcal{S}, i} = merge(\mathcal{E}_{\mathcal{S}, i},\ \mathcal{E})$, and continue to process the next event. Otherwise, continue the following steps.

% 				\STATE Set $matchIdx \leftarrow -1,\ \text{linkScore}_{max} \leftarrow -1$

% 				\FOR{Each event $\mathcal{E}_{\mathcal{S}, j} \in \mathcal{S}$}

% 					\STATE Calculate $\text{linkScore}(\mathcal{E}, \mathcal{E}_{\mathcal{S}, j})$ according to Equation \ref{eqn:linkScore}.

% 					\IF{$\text{linkScore}_{max} < \text{linkScore}(\mathcal{E}, \mathcal{E}_{\mathcal{S}, j})$}

%                     	\STATE $\text{linkScore}_{max} \leftarrow \text{linkScore}(\mathcal{E}, \mathcal{E}_{\mathcal{S}, j}),\ matchIdx \leftarrow j$

% 					\ENDIF

% 				\ENDFOR

% 				 \IF{$\text{linkScore}_{max} \geq \delta$}
% 				 	\STATE Insert $\mathcal{E}$ into $\mathcal{S}$, and add link $<E_{\mathcal{S}, matchIdx},\ E>$ to $\mathcal{S}$.
% 				 \ELSE

% 				 	\STATE Insert $\mathcal{E}$ into $\mathcal{S}$, and add link $<ROOT,\ E>$ to $\mathcal{S}$.

% 				 \ENDIF
% 			\ELSE
% 				\STATE Create a new story tree $\mathcal{S}$ from $\mathcal{E}$ and add it to story forest $\mathcal{F}_t$.
% 			\ENDIF
% 		\ENDFOR
% 	\ENDWHILE

% \end{algorithmic}
% \end{algorithm}

Given the set of extracted events for a particular topic, we further organize these events into multiple stories under this topic in an online manner. Each story is represented by a \textit{Story Tree} to characterize the evolving structure of that story.
% Algorithm \ref{alg:story-structure} describes how we organize events into story trees in an online manner.
Upon the arrival of a new event and given an existing story forest, our online algorithm to grow the story forest mainly involves two steps: a) identifying the story tree to which the event belongs; b) updating the found story tree by inserting the new event at the right place. 
If this event does not belong to any existing story, we create a new story tree.


{\bf a) Identifying the related story tree.} 
Given a set of new events $E_t = \{\mathcal{E}_1, \mathcal{E}_2, ..., \mathcal{E}_{|E_t|}\}$ at time period $t$ and an existing story forest $\mathcal{F}_{t-1} = \{ \mathcal{S}_1, \mathcal{S}_2, ..., \mathcal{S}_{|\mathcal{F}_{t-1}|}\}$ that has been formed during previous $t-1$ time periods, our objective is to assign each new event $\mathcal{E} \in E_t$ to an existing story tree $\mathcal{S} \in \mathcal{F}_{t-1}$. If no story in the current story forest matches that event, a new story tree will be created and added to the story forest. %otherwise, we continue the steps after matching stories.
%also explain same event filtering.

We apply a two-step strategy to decide whether a new event $\mathcal{E}$ belongs to an existing story tree $\mathcal{S}$ formed previously.
% Specifically, we measure 1) keyword graph similarity, 2) document similarity and 3) title similarity sequentially to make a final decision about whether an event matches a certain story tree.
First, as described at the end of Sec. \ref{subsec:eventClustering}, event $\mathcal{E}$ has its own keyword set $\mathcal{C}_{\mathcal{E}}$.
Similarly, for the existing story tree $\mathcal{S}$, there is an associated keyword set $\mathcal{C}_{\mathcal{S}}$ that is a union of all the keyword sets of the events in that tree.

Then, we can calculate the compatibility between event $\mathcal{E}$ and story tree $\mathcal{S}$ as the Jaccard similarity coefficient between $\mathcal{C}_{\mathcal{S}}$ and $\mathcal{C}_{\mathcal{E}}$: 
$
  \text{compatibility}(\mathcal{C}_{\mathcal{S}}, \mathcal{C}_{\mathcal{E}}) = \frac{|\mathcal{C}_{\mathcal{S}} \cap \mathcal{C}_{\mathcal{E}}|}{|\mathcal{C}_{\mathcal{S}} \cup \mathcal{C}_{\mathcal{E}}|}.
$
If the compatibility is bigger than a threshold, we further check whether at least a document in event $\mathcal{E}$ and at least a document in story tree $\mathcal{S}$ share $n$ or more common words in their titles (with stop words removed). If yes, we assign event $\mathcal{E}$ to story tree $\mathcal{S}$. Otherwise, they are not related. In our experiments, we set $n=1$. 
% Notice that the document frequencies (DFs) of words change while news documents keep arriving. We maintain a time window of length $T_{df}$ days to update the document frequencies of words. In our case, we set $T_{df}=2$.
If the event $\mathcal{E}$ is not related to any existing story tree, a new story tree will be created.


\begin{figure}
\includegraphics[width=3.3in]{figure/NodeOperation}
\caption{Three types of operations to place a new event into its related story tree.}
\label{fig:nodeOperations}
\vspace{-2mm}
\end{figure}

{\bf b) Updating the related story tree.} After a related story tree $\mathcal{S}$ has been identified for the incoming event $\mathcal{E}$, we perform one of the 3 types of operations to place event $\mathcal{E}$ in the tree: \textit{merge}, \textit{extend} or \textit{insert}, as shown in Fig.~\ref{fig:nodeOperations}.
The \textit{merge} operation merges the new event $\mathcal{E}$ into an existing event node in the tree. The \textit{extend} operation will append event $\mathcal{E}$ as a child node to an existing event node in the tree. Finally, the \textit{insert} operation directly appends event $\mathcal{E}$ to the root node of story tree $\mathcal{S}$. Our system chooses the most appropriate operation to process the incoming event based on the following procedures.

{\bf \emph{Merge}}: we merge $\mathcal{E}$ with an existing event in the tree, if they essentially talk about the same event.
This can be achieved by checking whether the centroid documents of the two events are talking about the same thing using the document-pair relationship classifier described in Sec.~\ref{subsec:eventClustering}. The centroid document of an event is simply the concatenation of all the documents in the event.
% If yes, we merge the new incoming event with the existing event node. Otherwise, we continue the procedures below.

{\bf\emph{Extend}} \emph{and} {\bf \emph{Insert}}: if event $\mathcal{E}$ does not overlap with any existing event, we will find the parent event node in $\mathcal{S}$ to which it should be appended.
We calculate the \emph{connection strength} between the new event $\mathcal{E}$ and each existing event $\mathcal{E}_j \in \mathcal{S}$ based on three factors: 1) the time distance between $\mathcal{E}$ and $\mathcal{E}_j$, 2) the compatibility of the two events, and 3) the \emph{storyline coherence} if $\mathcal{E}$ is appended to $\mathcal{E}_j$ in the tree, i.e., 
\begin{align}
\label{eqn:linkScore} 
\begin{split}
  \text{ConnectionStrength}(\mathcal{E}_j, \mathcal{E})  :=\ \text{compatibility}(\mathcal{E}_j, \mathcal{E}) \times \\
  \text{coherence}(\mathcal{L}_{\mathcal{S}\to\mathcal{E}_j\to \mathcal{E}}) \times \text{timePenalty}(\mathcal{E}_j, \mathcal{E}).
\end{split}
\end{align}

Now we explain the three components in the above equation one by one. \emph{First}, the compatibility between two events $\mathcal{E}_i$ and $\mathcal{E}_j$ is given by
\begin{equation}
  \text{compatibility}(\mathcal{E}_i, \mathcal{E}_j) = \frac{\text{TF}(d_{c_i}) \cdot \text{TF}(d_{c_{j}})}{\|\text{TF}(d_{c_i})\| \cdot \|\text{TF}(d_{c_{j}})\|},
\end{equation}
where $d_{c_i}$ is the centroid document of event $\mathcal{E}_i$.
% Notice that here we use the term frequency (TF) vector of each document rather than TF-IDF, since this choice leads to better performance in practice.

Furthermore, the storyline of $\mathcal{E}_j$ is defined as the path in $\mathcal{S}$ starting from the root node of $\mathcal{S}$ ending at $\mathcal{E}_j$ itself, denoted by $\mathcal{L}_{\mathcal{S}\rightarrow \mathcal{E}_j}$. Similarly, the storyline of $\mathcal{E}$ appended to $\mathcal{E}_j$ is denoted by $\mathcal{L}_{\mathcal{S}\rightarrow \mathcal{E}_j\rightarrow\mathcal{E}}$.
% Previous works on online event threading \cite{wang2016socially} usually just measure the similarities between event pairs to determine whether a event belongs to an existing story line, and then attach into the line if some criteria is satisfied. However, such kind of approaches doesn't consider the \textit{coherence} \cite{xu2013summarizing} of the whole story line.
For a storyline $\mathcal{L}$ represented by a path
$\mathcal{E}^0\to \ldots \to \mathcal{E}^{|\mathcal{L}|}$, where $\mathcal{E}^0 := \mathcal S$, its \textit{coherence} \cite{xu2013summarizing} measures the theme consistency along the storyline, and is defined as
\begin{equation}
  \text{coherence}(\mathcal{L}) = \frac{1}{|\mathcal{L}|}\sum_{i=0}^{|\mathcal{L}|-1} \text{compatibility}(\mathcal{E}^i, \mathcal{E}^{i+1}),
\end{equation}

Finally, the bigger the time gap between two events, the less possible that the two events are connected. We thus calculate time penalty by
\begin{align}
  \text{timePenalty}(\mathcal{E}_j, \mathcal{E}) = \begin{cases}
  e^{\delta \cdot (t_{\mathcal{E}_j} - t_{\mathcal{E}})}
   &\ \text{if } t_{\mathcal{E}_j} - t_{\mathcal{E}} < 0\\
  0 &\ \text{otherwise}\\
  \end{cases}
\end{align}
where $t_{\mathcal{E}_j}$ and  $t_{\mathcal{E}}$ are the timestamps of event $\mathcal{E}_j$ and $\mathcal{E}$ respectively. The timestamp of an event is the minimum timestamp of all the documents in the event.

We calculate the connection strength between the new event $\mathcal{E}$ and every event node $\mathcal{E}_j \in \mathcal{S}$ using \eqref{eqn:linkScore}, and append event $\mathcal{E}$ to the existing $\mathcal{E}_j$ that leads to the maximum connection strength. 
If the maximum connection strength is lower than a threshold value, we \textit{insert} $\mathcal{E}$ into story tree $\mathcal{S}$ by directly appending it to the root node of $\mathcal{S}$. In other words, \emph{insert} is a special case of \emph{extend}.



\section{Interactive Recommendation System}

The interface of our system is designed to be consistent with popular commercial recommendation systems, such as Youtube, Netflix and Amazon Movie. 
As shown in Figure~\ref{interface}, our system contains three common components -- an enlarged movie poster with information (a), the list of top-N recommended movies (c), and additional information at the bottom (e).
We add a small area for user interaction (b) and a visualization domain (d) to support interactive recommendation and exploration of movie database.
Our system also supports the following features of interactive recommendation.
%In the blue component of additional information, we show the watch history as it is a useful hidden function in the example systems.

%Although only the Amazon video website provides watch history (at the end of the page), the watch history can always be found with the ``back" button during web browsing.
%We can also add additional information in the order of importance to the blue components for details.
%Note that the interaction functions provided by Amazon is for purchasing and adding to wishlist; while we provide interaction functions for adjusting the types of recommended movies with slide bars between two sets of user-friendly concepts.

\begin{figure}[htb]
\centering
\includegraphics[scale=0.28]{Interface.EPS}
\caption{The interface of our recommendation system.
Similar to online movie recommendation systems, our interface includes (a) basic information of a selected movie, (c) a list of recommendable movies, and (e) watch history.
We also add (b) for explicit input of user preferences and (d) visualization for exploring recommendable movies.
%The top band with black background shows detailed information (a) about the current recommended movie (the current story piece of the recommended movies) that is describing to the user. On the same band, we have an interactive panel (b) that allows the user to adjust his or her recommendation preferences. The recommended list (c) and the storyline (d) are used to narrate and explain to the user why a movie is recommended based on his or her rating history. 
%This storyline (d) describes a user taste from drama and romance genres that the user likes toward the left to comedy type that the user dislikes toward the right. (e) shows the recommended movies the user selected from the recommended list.
}
\label{interface}
\end{figure}
%we can describe the combined feature on the best dimension as from comedy/romance (on the left) to drama/biography/music (on the right). 



%\subsection{Narrative visualization}


%The narrative visualization illustrates a recommendation story with the order of recommended movies and the role of each movie on the plot is indicated by locations.
%We use the locations of movies on the latent dimensions to determine the order and role of each movie.

\subsection{Example-based Narrative Structure}

The visualization of the latent dimension plays an important role in illustrating the recommendable movies and the movie distributions.
To describe the narrative structure visually, we design an example-based approach shown in Figures~\ref{svdSemantic}, \ref{svdexample}, \ref{interface}, which uses two movies that have been watched by the user, one on each side of the familiar zone, to describe the combined features of that latent dimension.
Since the combined features are generated simply from data statistics, they generally cannot be described with languages or equations.
The example movies help the users to understand the overall trend of the latent dimension and provide a quick impression of a new movie by the distances to the two examples.
All the recommendable movies are located on the 2D domain of latent dimension and recommendation degree, providing additional visual cues for recommendation reasons.


\subsection{Multi-Level Visualization}

The reasons to recommend a movie can be multifold. 
While the reasons supported by LSM mainly come from the aspects of statistical similarity and semantic analysis, we can provide a variety of information as the ``brief" reasons of a recommendation, such as a typical drama movie in the familiar zone of a user, which is similar to an example movie shown with the poster.
%Theoretically we can add additional information to the system, especially after the watch history on the detailed panel (e), if we allow long recommendation stories.
We organize the information on the visualization domain at three levels, so that the storytelling animation can follow the levels to achieve the effect that additional information is introduced gradually.
Users can stop at any time if they are not interested in the movie and continue to the next recommendation.

As shown in image (i) of Figure~\ref{animation}, the level one provides the basic information of a latent dimension with the liked and disliked regions, where the nodes of movies being recommended are drawn as circles.
The nodes of movies that the user has watched are colored green and the nodes of recommendable movies are colored based on their genres.
Two example movies are provided for illustrating the latent dimension and making correlations to other movies.

%the relative locations of the recommended movies with the others.  This level provides the SVD semantic clustering of the movies. The closer a movie is to others, the more similar they are. The SVD clusters are then grouped based on the user rating history. For example, the SVD clusters from which the user liked movies are grouped together by shading their location background in green in the storyline. This gives the user the ability to link the recommended movies to those he or she already liked.

The level two introduces the recommendation degrees to the vertical locations of recommendable movies. As shown in images (j) and (k) of Figure~\ref{animation}, the movies with higher recommendation degrees are placed on the top to attract user attention. 
The scaling can be automatically done in an animation or interactively adjusted by the user when exploring the movie database.


The level three provides the richest information for exploring movies.
For the movie being recommended, we reveal the top four similar movies that the user has watched and liked to strengthen the reasons of recommendation, as shown in images (l) and (m)  of Figure~\ref{animation}.
All the four movies have high user ratings and they are close to the movie being recommended on the latent dimension.
We also color all the movie nodes based on the genres and add colored links to connect posters to the movie nodes.


%Should the links to the five movies be provided on all the levels? They are not just about the current movie.



\begin{figure*}[htb]
%\centering
\includegraphics[width = 1.64in]{componentAnimation1.EPS} \ \ \
\includegraphics[width = 1.64in]{componentAnimation2.EPS} \ \ \
\includegraphics[width = 1.64in]{componentAnimation3.EPS} \ \ \
\includegraphics[width = 1.64in]{componentAnimation4.EPS}\\ 
 \ \ \ \ \ \ (a) \ \ \ \ \ \ \ \ \ \ \ \ \ \ \ \ \ \ \ \ \ \ \ \ \ \ \ \ \ \ \ \ \ \ \ \ \ \ \ \  \ \ \ \ \ \ \ \ \ \ \ \ (b) \ \ \ \ \ \ \ \ \ \ \ \ \ \ \ \ \ \ \ \ \ \ \ \ \ \ \ \ \ \ \ \ \ \ \ \ \ \ \ \  \ \ \ \ \ \ \ \ \ \ \ \ (c) \ \ \ \ \ \ \ \ \ \ \ \ \ \ \ \ \ \ \ \ \ \ \ \ \ \ \ \ \ \ \ \ \ \ \ \ \ \ \ \  \ \ \ \ \ \ \ \ \ \ \ \ (d) \\ 
\includegraphics[width = 1.64in]{dimAnimation1.EPS} \ \ \
\includegraphics[width = 1.64in]{dimAnimation2.EPS} \ \ \
\includegraphics[width = 1.64in]{dimAnimation3.EPS} \ \ \
\includegraphics[width = 1.64in]{dimAnimation4.EPS}\\
  \ \ \ \ \ \ (e) \ \ \ \ \ \ \ \ \ \ \ \ \ \ \ \ \ \ \ \ \ \ \ \ \ \ \ \ \ \ \ \ \ \ \ \ \ \ \ \  \ \ \ \ \ \ \ \ \ \ \ \ (f) \ \ \ \ \ \ \ \ \ \ \ \ \ \ \ \ \ \ \ \ \ \ \ \ \ \ \ \ \ \ \ \ \ \ \ \ \ \ \ \  \ \ \ \ \ \ \ \ \ \ \ \ (g) \ \ \ \ \ \ \ \ \ \ \ \ \ \ \ \ \ \ \ \ \ \ \ \ \ \ \ \ \ \ \ \ \ \ \ \ \ \ \ \  \ \ \ \ \ \ \ \ \ \ \ \ (h) \\ 
\includegraphics[width = 1.64in]{recAnimation10.EPS} \ \ \
\includegraphics[width = 1.64in]{recAnimation11.EPS} \ \ \
\includegraphics[width = 1.64in]{recAnimation12.EPS} \ \ \
%\includegraphics[width = 1.64in]{recAnimation13.EPS}\\
\includegraphics[width = 1.64in]{recAnimation14.EPS}\\
 \ \ \ \ \ \ (i)  \ \ \ \ \ \ \ \ \ \ \ \ \ \ \ \ \ \ \ \ \ \ \ \ \ \ \ \ \ \ \ \ \ \ \ \ \ \ \ \  \ \ \ \ \ \ \ \ \ \ \ \ (j) \ \ \ \ \ \ \ \ \ \ \ \ \ \ \ \ \ \ \ \ \ \ \ \ \ \ \ \ \ \ \ \ \ \ \ \ \ \ \ \  \ \ \ \ \ \ \ \ \ \ \ \ (k) \ \ \ \ \ \ \ \ \ \ \ \ \ \ \ \ \ \ \ \ \ \ \ \ \ \ \ \ \ \ \ \ \ \ \ \ \ \ \ \  \ \ \ \ \ \ \ \ \ \ \ \ (l) \\ 
%\includegraphics[width = 1.64in]{recAnimation14.EPS} \ \ \
\includegraphics[width = 1.64in]{recAnimation15.EPS} \ \ \
\includegraphics[width = 1.64in]{recAnimation21.EPS} \ \ \
\includegraphics[width = 1.64in]{recAnimation51.EPS}\\
  \ \ \ \ \ \ (m) \ \ \ \ \ \ \ \ \ \ \ \ \ \ \ \ \ \ \ \ \ \ \ \ \ \ \ \ \ \ \ \ \ \ \ \ \ \ \ \  \ \ \ \ \ \ \ \ \ \ \ \ (n) \ \ \ \ \ \ \ \ \ \ \ \ \ \ \ \ \ \ \ \ \ \ \ \ \ \ \ \ \ \ \ \ \ \ \ \ \ \ \ \  \ \ \ \ \ \ \ \ \ \ \ \ (o)\\ 
\caption{Example snapshots of the storytelling animation. 
We start by introducing the system interface to a new user, such as the use of rating history (a), color code (b), recommendation degree (c) and the liked zone (d).
We also use example movies to illustrate the latent dimension and attract user attention with poster transitions (e-h).
Next, we introduce the first recommended movie by animating the movie node and showing a green line under its poster (i).
The visualization domain is animated from level one (i), level two (j-k), to level three (l-m) gradually.
The same procedure (i-m) is used to animate the second (n) to the last (o) recommended movies respectively.
%The third and forth rows images (i-n) show animated visualization of the first recommended movie. Its is highlighted by a green line under its poster while its node height is gradually animated from zero to its recommendation degree (i-k). Its node is then enlarged (l). (m-n) show the animation of its top four related movies. (o) and (p) show the end of the story of the second and fifth recommended movies respectively.
}

%The visualization of latent dimension is described with  movie ``Airhead'' the user liked (on the left) and movie ``Weekend at Bernie's'' the use disliked (on the right) (e-h).


%(a) Zone highlighting. Like zone is highlighted accompanied by a descriptive message, (b) Interpreting the Storyline Semantic using animation. Example movie node and poster animating (on the right) to describe the storyline with movies familiar to the user. and (c)  Explaining a recommended movie with user liked top four movies, the related movie nodes are enlarged while their posters are rotated click-wise.}
\label{animation}
\end{figure*} 





\subsection{Animation Effects}

Our system provides fully automatic animations to ``tell" a recommendation story with the following three sets of animation effects.

The first animation set is to introduce the system components. 
Step by step, each component is highlighted with a brief description as shown in images (a-d) in Figure~\ref{animation}. 
%The visualization of latent dimension is also presented in details.
%Each zone is highlighted with a boundary and its meaning is described with a short message. 
%For example, image (d) of Figure~\ref{animation} shows the liked zone highlighted.
The user can replay or skip to the next animation set anytime.
 
The second animation set is to present the narrative structure of a recommendation story.
We use our example-based approach to animate the recommended movies for user attention and the two example movies from left to right for providing the impression of dimension meanings.
Specifically, each example movie poster is animated by changing its size and the movie node is also animated simultaneously, as shown in  Figure~\ref{animation} (e-h).

%Specifically, the storyline is described by flashing nodes that represent the example movies in the storyline visualization, while their posters are briefly animated by reducing their sizes.
%One of the drawbacks of animation in visualization is the human difficulty to focus on many moving things simultaneously. For this reason, we used a sequential ordering for the sets of animation without overlapping them. 

The third animation set is to present a recommendation story, as shown in images (i-o) in Figure~\ref{animation}.
We start to highlight the selected movie by flashing the movie node in the visualization panel when its poster in the recommendation list is animated.  
A green line under the movie poster also indicates the focused movie. 
Then, we switch the visualization from level one to level three gradually to provide detailed information of a selected movie.
Specifically, after the level one is shown, the movie nodes in the visualization panel are progressively scaled to their maximum recommendation degree in the level two.
From level two to level three, the set of similar movies are animated.
We use the same animation procedure for all the selected movies in the story to avoid confusion.
The user can observe all the information or switch anytime to the next recommended movie.

When the animation of a recommendation story ends or the user selects a movie for further exploration, a new story is generated automatically.
The system re-introduces the new narrative structure and repeats the animation sequences to convey the story behind the new selected movies.

%Similar to the interaction mantra - overview first and details on demand, we start with providing an overview of the group of recommended movies with examples. We call this presentation phases.  


%Additional visual indicators for improving user understanding are also provided. 
%First, we highlight the movie under recommendation with a blue window and demonstrate its location on the story outline with a flash effect.
%Second, a progress bar is provided so that users know exactly where they are during a story.
%The progress bar is located right between the recommendation list in green and the storyline panel on the interface.

% generate more steps to provide a more complete view of the animation
% if out of space, the multilevel figure can be combined here.

%\section{Experiments}

In this section, we demonstrate the effectiveness of our approach at exploiting dyadic interactions. To this end, we first introduce our \lindyhop{} dataset depicting couples that perform lindy hop dance movements.

\subsection{LindyHop600K}
Lindy hop is a type of swing dance with fast-paced steps synchronized with the music. It constitutes a good example of motions with strong mutual dependencies between the subjects, who are engaged in close interactions. To build this dataset, we filmed three men and four women dancers paired up in different combinations. Overall, \lindyhop{} contains nine dance sequences, each two to three minutes long, with a maximum of eight cameras at 60 fps. We use the shortest two sequences as validation and test sets. Table~\ref{table:seq_lhop} shows the details of the dataset organization. Our dataset displays standard lindy hop dancer positions and steps, such as the so-called open, closed, side and behind positions. In the open and closed positions, the dancers are facing each other with a varying distance between them. In the side position, both are facing the same direction, and in the behind position, the leader stands directly behind the follower, both facing the same direction. In each position, the dancers communicate through hand and shoulder grips. To the best of our knowledge, \lindyhop{} is the first large dance dataset involving the videos and 3D ground-truth poses of dancers.

\begin{table}[t] 
	\centering 
	\scalebox{0.9}{
		
		\begin{tabular}{ c|c|c|c|c } 
			\hline
			Sequence & Couple & Frames & Cameras & Split \\
			\hline
			{1} & A1 & 10152 & 5 & Train  \\ 
			{2} & B2 & 8819 & 8 & Train  \\ 
			{3} & C3 & 6519 & 8 & Validation  \\ 
			{4} & A4 & 7687 & 8 & Test \\ 
			{5} & B1 & 9977 & 8 & Train \\ 
			{6} & C2 & 9636 & 8 & Train\\ 
			{7} & A3 & 8930 & 7 & Train \\ 
			{8} & B4 & 9027 & 8 & Train \\ 
			{9} & C1 & 9635 & 8 & Train \\ 
			\hline
		\end{tabular}
	}
	\caption[\lindyhop{} dataset structure]{ \textbf{\lindyhop{} dataset structure.}}
	\label{table:seq_lhop}
	\vspace{-4mm}
\end{table}

To obtain the 3D poses of the dancers, we first extract 2D pixel locations of the visible joints using OpenPose~\cite{Cao17}. Because our dataset was captured with multiple cameras, this lets us obtain the  3D joint coordinates by performing a bundle adjustment using the 2D joint locations in all the views. However, this process comes with several problems because it requires annotating the poses of both subjects together. The major issues encompass body part confusions, missing 2D annotations and tracking errors in the OpenPose predictions, which occur when two people are very close to each other or wear similar garments. An example of this is shown in Fig.~\ref{fig:optimizing_3dposes}. To remedy this, we adopt a solution based on temporal smoothness. Specifically, we assign manually the 2D joint locations to each dancer in the first frame of each sequence. For the subsequent frames, the low confidence joint detections are replaced with ones interpolated using the high confidence joints from the neighboring frames. Despite these 2D joint corrections, the 3D locations extracted from the bundle adjustment procedure can still be very noisy. Thus, we employ a third degree spline interpolation across 30 frames coupled with an optimization scheme to generate the final 3D poses. Since the spline interpolation is done separately for each dimension of each joint, the length of each limb varies from one frame to another. To tackle this problem, we implement an optimization scheme which minimizes the squared difference between the length of a limb $c$ in the current frame and the average length of  limb $c$. We combine this loss function with additional regularizers penalizing feet from sliding on the floor, constraining the shape of the hips and shoulders, and preventing the optimization to the initial 3D pose estimates. For more detail, we refer the reader to the supplementary material.


\begin{figure}
	\centering
	\begin{tabular}{c}
		
		\includegraphics[width=0.67\linewidth]{figures/lindyhop_failure.pdf} \\
		(a) \footnotesize OpenPose 2D detection failure and the optimized 3D poses \\ \\
		\includegraphics[width=0.67\linewidth]{figures/lindyhop_success.pdf} \\
		(b) \footnotesize Correct OpenPose detections and the optimized 3D poses\\
	\end{tabular}
	\caption[Optimizing 3D poses in the \lindyhop{} dataset]{\textbf{Optimizing 3D poses in the \lindyhop{} dataset.} (a) Example of OpenPose 2D detection failure. The left leg of the woman is mapped to the left leg of the man. Our multi-view footage and refinement strategy allow us to obtain accurate 3D poses of the dancers despite the mismatch in the 2D detections. (b) Example of correct OpenPose detections and the optimized 3D ground truth poses.}
	\label{fig:optimizing_3dposes}
	\vspace{-4mm}
\end{figure}

\subsection{Data Pre-processing}
Each video sequence is first downsampled to 30 fps. The human body skeleton in the \lindyhop{} dataset originally comprises of $25$ body joints. We remove some of the facial, hand and foot joints and train our models with a skeleton of $19$ joints. The 3D joint locations are represented in the world coordinates. Since the position and orientation of the dancers change from one frame to another, we apply a rigid transformation to the poses.  We first subtract the global position of the hip center joint from every joint coordinate in every frame. Then, for each sequence, we take the first pose as  reference and rotate it such that the unit vector from the left to right shoulder is aligned with the $x$-axis and the unit vector from the center hip joint to the neck is aligned with the $z$-axis. We apply the same rotation to all the other poses in the sequence. 

\subsection{Results}

In this section, we evaluate our approach depicted by Fig.~\ref{fig:overview_3dmotion_forecasting} on our new \lindyhop{} dataset. We compare our method with the state-of-the-art single person approaches. They include HRI~\cite{Mao20}, which relies on an attention mechanism and a GCN decoder~\cite{Mao19} to predict the future poses of the individuals in isolation; HRI-Itr, which uses the output of the predictor as input and predicts the future motion recursively; TIM~\cite{Lebailly20}, which extends~\cite{Mao19} by combining it with a temporal inception layer to process the input at different subsequence lengths; and MSR-GCN~\cite{Lingwei21}, the most recent method, which extracts features from the human body at different scales by grouping the joints in close proximity. All the baselines rely on a GCN architecture that is trained and tested according to the data split shown in Table~\ref{table:seq_lhop}. They take as input a sequence of $60$ poses as  past motion. Except for HRI-Itr that recursively predicts $10$ poses at a time, all the baselines predict $30$ poses in the future. 

In Table~\ref{table:sota_lhop}, we report the MPJPE for short-term ($<$ 500ms) and long-term ($>$ 500ms) motion prediction in mm. Our method outperforms the baselines by a large margin. Fig.~\ref{fig:qualitative_lhop_sota} depicts qualitative results of our approach and the best performing three baselines for the \lindyhop{} test subjects with the corresponding follower and leader roles in the top two and bottom two portions, respectively. In contrast to the baselines, our method accurately predicts moves that are hard to anticipate in the long term, such as fast changing feet movements and less frequent arm openings. Although the observed motion of the primary subject does not include sufficient clues for such moves, the second person provides a useful prior so that our model can learn to predict the motion complementary or symmetric to that of the auxiliary subject. Therefore, we attribute this performance to our modeling of the motion dependencies via our pairwise attention mechanism. We provide additional qualitative results and further analysis on the learned pairwise attention scores in the supplementary material.

\begin{figure*}
	\vspace{-4mm}
	\centering
	\begin{tabular}{c}
		\includegraphics[width=0.93\linewidth]{figures/sota_qual_lhop_two_people.pdf} \\
	\end{tabular}
	\vspace{-4mm}
	\caption[Qualitative 3D motion prediction results on the \lindyhop{} test subjects]{\textbf{Qualitative evaluation of our results on the LindyHop600K test subjects compared to the state-of-the-art methods.} Black: Ground truth, green: TIM~\cite{Lebailly20}, blue: MSR-GCN~\cite{Lingwei21}, violet: HRI~\cite{Mao20}, red: Ours-Dyadic. Top two portions show the predictions for dancer with the follower role. Bottom two portions show the predictions for the dancer with the leader role. The left side of the vertical bar in the black row depicts the sampled input to our model and the right side shows the ground truth future poses. The colored rows correspond to the predictions of the state-of-the-art single person approaches. The red row depicts the output of our model shown in Fig.~\ref{fig:overview_3dmotion_forecasting}. The numbers at the top indicate the timestamp in milliseconds and the green region highlights the long-term predictions.}
	\label{fig:qualitative_lhop_sota}
\end{figure*}




\begin{table*}[t]
	%\vspace{0.2cm}
	\centering
	\scalebox{1.0}{
		\begin{tabular}{lccccccccccc}
			\toprule
			milliseconds											&100	&200	&300	&400	&500	&600 &700 &800 &900 &1000 &Average  \\ 
			\midrule
			
			
			{TIM~\cite{Lebailly20}}				   &6.06 &12.39 &19.83 &29.35 &41.80 &56.91 &73.17 &89.23 &104.31 &118.20    &51.13 \\
			{MSR-GCN~\cite{Lingwei21}}		&9.02 &17.02 &24.79 &33.26 &43.69 &56.34 &70.49 &85.00  &98.37   &109.73 &51.11  \\	
			{HRI-Itr~\cite{Mao20}}				   &2.21 &4.94 &9.51 &17.71 &30.93 &49.66 &72.95 &98.39 &122.93 &144.24  &50.41\\
			{HRI~\cite{Mao20}}						&5.34 &9.95 &15.08 &22.19 &32.45 &45.82 &61.29 &77.40 &92.47 &105.15    &43.17 \\
			{Ours}		&\textbf{1.31} &\textbf{4.31} &\textbf{9.49} &\textbf{17.33} &\textbf{27.42} &\textbf{39.85} &\textbf{54.22} &\textbf{70.20} &\textbf{86.23} &\textbf{100.09} &\textbf{37.57}\\	
			\bottomrule 
		\end{tabular}
		
	}  \\
	\caption[Comparison of our dyadic motion prediction approach with the state-of-the-art methods on the \lindyhop{} dataset]{\textbf{Comparison of our dyadic motion prediction approach with the state-of-the-art single person methods on the \lindyhop{} dataset.} We present the MPJPE for short-term ($<$ 500ms) and long-term ($>$ 500ms) motion prediction in mm. Despite the fast-paced and nonrepetitive nature of the dance moves, our method outperforms all the baselines for both short-term and long-term prediction. The best results in each column are shown in bold.}
	\label{table:sota_lhop}
\end{table*}


\subsection{Ablation Study}

\begin{table*}
	%\vspace{0.2cm}
	\centering
	\scalebox{1.0}{
		\renewcommand{\tabcolsep}{1.5mm}
		\begin{tabular}{lccccccccccc}
			\toprule
			milliseconds											&100	&200	&300	&400	&500	&600 &700 &800 &900 &1000 &Average  \\ 
			\midrule
			
			
			{HRI-Concat}	   &17.13 &33.99 &51.32 &69.89 &90.67 &113.41 &136.00 &156.10 &172.06 &183.40 &96.34\\	
			{Ours-SumPooling} &5.77&10.78&16.07&22.86&32.41&45.17&60.63&77.40&93.45&106.94&43.54\\
			{Ours-AvgPooling} &5.66&10.47&15.90&23.53&34.46&48.68&65.13&82.19&97.99&111.02&45.77\\
			{Ours-MaxPooling} &5.07&9.50&14.57&21.65&31.79&44.89&60.13&76.26&91.61&104.72&42.48\\
			{Ours-w/oPairwiseAtt} &3.60 &11.48 &25.08 &43.00 &62.22  &81.41 &100.25 &118.70 &135.48 &149.39 &68.04 \\	
			{Ours-w/o$\Delta$Pose}		&3.28 &8.36 &16.84 &23.87 &36.77 &52.22 &68.67 &85.02 &100.02 &112.07 &46.33\\	
			{Ours-EarlyMerge}		 &4.25 &8.11 &12.78 &19.25 &28.45 &40.84 &56.05 &73.11 &90.27 &105.40 &40.27\\		
			{Ours-w/SelfAttAux} &1.30 &5.04 &10.47 &18.12 &28.95 &42.41 &57.89 &74.52 &90.47 &104.09 &39.76\\
			{Ours-PairwiseAtt$\textbf{U}^{12}$ } &\textbf{1.17}&4.48&9.74&17.82&28.35&41.27&56.25&72.32&88.09&101.77&38.66\\
			{Ours}	&1.31 &\textbf{4.31} &\textbf{9.49} &\textbf{17.33} &\textbf{27.42} &\textbf{39.85} &\textbf{54.22} &\textbf{70.20} &\textbf{86.23} &\textbf{100.09} &\textbf{37.57}\\	\\	
			\bottomrule 
		\end{tabular}
		
	}  \\
	\caption[Ablation study for incorporating interactions]{\textbf{Ablation study for incorporating interactions.} We present the MPJPE for short-term ($<$ 500ms) and long-term ($>$ 500ms) motion prediction in mm. Here, we analyze different ways of incorporating interactions. HRI-Concat concatenates the motion history of the primary and auxiliary subject to treat them as one person. Ours-SumPooling, Ours-AvgPooling and Ours-MaxPooling use the social pooling layers from~\cite{Adeli20}. The remaining baselines show the benefits of the different components in our approach. Ours, depicted in Fig.~\ref{fig:overview_3dmotion_forecasting}, outperforms all other baselines and poses an effective way of handling coupled motion. The best results in each column are shown in bold.}
	\label{table:ablation_study_lhop}
	\vspace{-3mm}
\end{table*}

We evaluate the effect of modeling interactions via different strategies: \\
\textit{HRI-Concat} concatenates the motion history of the primary and auxiliary subject to treat them as one person. \\
\textit{Ours-SumPooling}, \textit{Ours-AvgPooling} and \textit{Ours-MaxPooling} discard the pairwise attention module, apply self-attention on the sequences of both subjects independently and combines the individual embeddings using the different pooling strategies proposed by~\cite{Adeli20}. The resulting vector is fed to the GCN decoder to predict the future poses of the primary subject. \\
\textit{Ours-w/oPairwiseAtt} excludes the pairwise attention module, applies self-attention and the GCN decoder on the sequences of both subjects independently and merges the GCN outputs from the two people to predict the future poses of the primary subject. \\
 \textit{Ours-w/o$\Delta$Pose} is our model which takes as input the past motion of the auxiliary subject directly instead of their relative motion to the primary subject.\\
 \textit{Ours-EarlyMerge} merges the pairwise embeddings $\textbf{U}^{12}$ and $\textbf{U}^{21}$ with the self-attention embedding of the primary subject $\textbf{U}^{1}$ before feeding them to the GCN module. \\
\textit{Ours-w/SelfAttAux} applies self-attention also on the sequence of the auxiliary subject and merges the result with the pairwise embeddings $\textbf{U}^{12}$ and $\textbf{U}^{21}$. \\
\textit{Ours-PairwiseAtt$\textbf{U}^{12}$ } excludes the pairwise attention that takes the keys and values from the auxiliary and the query from the primary subject. 
 

As can be seen in Table~\ref{table:ablation_study_lhop}, our method achieves the highest MPJPE in all timestamps. The comparison with \textit{HRI-Concat} shows that the naive way of combining the motion of the subjects is not an effective strategy to model their dependencies. The results of \textit{Ours-SumPooling}, \textit{Ours-AvgPooling} and \textit{Ours-MaxPooling} show that the social pooling layers proposed by~\cite{Adeli20} are suboptimal in the presence of strong interactions. The comparison to the remaining baselines evidence the benefits of the different components in our approach, which all contribute to the final results. 

\subsection{Limitations}
In Fig.~\ref{fig:qualitative_lhop_sota} and in the additional qualitative results, we observe that the lower arms and feet joints are usually difficult to predict and deviate the most from the ground-truth positions. Although Lindy Hop is a structured dance with highly correlated coupled motion, the dancers have their own styles. Therefore, predicting a single future is likely not to accurately match the body extremities which undergo the largest motion. This, however, can be overcome performing multiple diverse motion prediction, following a similar strategy to that used in~\cite{Yuan20,Aliakbarian21,Mao21b} for single-person motion prediction.

Another limitation of our model and many other motion prediction works in general is its use of complete sequences of ground-truth 3D poses as input. This may make our model sensitive to missing or faulty observations. To remedy this, as future work, we aim to incorporate the 3D poses obtained from the input images into our forecasting network and handle incomplete or noisy sequences to predict realistic future 3D poses for the interacting people.




\section{Results and Case Studies}

\subsection{Case Studies}

This section describes three case studies using the MovieLens100K dataset~\cite{MovieLens100k}. 
The dataset has 100K ratings from 1 to 5 and 1682 movies from different categories rated by 943 users.
We have select users with different backgrounds and rating histories to demonstrate our approach.

%We use examples to show that the recommendation results in all the case studies match the backgrounds of users.
%All the results match the backgrounds of users, such as the first user working in entertainment likes drama movies, the second user as a student is familiar with adventure movies, and the third user as an executive is picky on movie selections.

%For each user, we alter the interaction of selecting movies and adjusting preference parameters.
%For each case study, we list the movie orders in the storytelling during the interaction. 
%We also justify the selected movies and orders with description and numbers.

%\subsubsection{Users who seems to like certain genres of movies}

% Figure 1

%We identified two users that seem to like certain genres of movies and one user that rated lot movies but only liked few of them.

% User - 16: 21 year old male working in entaintainment. He lives in Staten Island, NY
\textbf{The first user} is a 21 year old male working in entertainment.
As shown in Figure~\ref{interface}, a latent dimension is identified between comedy movies toward the right and drama/romance/biography genres toward the left.
As the default preference settings of 0.5/0.5 for typicality/un-typicality and familiarity/diversity, our recommendation system starts with movies of drama genre in the familiar zone and continues to drama and romance types for diversity. 
The recommended movies are: When a Man Loves a Woman (1994, Drama and Romance), Meet John Doe (1941, Drama, Romance and Comedy), Nobody’s Fool (1994, Drama and Comedy), Ref The (1994, Comedy and Drama), and Rebel Without a Cause (1995, Drama).

We compare the recommendation results with the user's watch history.
%His rating records show a clear preference on drama or drama-romance movies over comedy types.
The user has rated 44 movies.
Among which, about 68\% of the movies are drama or a combination of drama and romance genres. Their average rating is high (4.30).
The movies of other genres received lower ratings, especially comedy movies.
Therefore, our result captures the user's preference of drama movies over comedy types.

%The story is told in the sequence of above, from left to right, from movies in the familiar zone to the movies in the variance zone.

%Most of the movies he watched (30 out of 34, about 88\%) are comedy or romance genre with 3.41 average ratings.
%The other 4 movies he rated were either a composite genre with comedy or romance, such as comedy-horror or comedy-drama-romance. 

%user 764
%The second user has rat•	When a Man Loves a Woman (1994), Drama-Romanceed 39 movies of several genres, including comedy, drama, horror, and romance. 
%His overall average rating is 3.80 and his ratings suggest that he does not like comedy and romance movies. 
%Figure~\ref{case1RecommendedList} shows our storytelling interface with several movies labeled according to the genres.
%On the storyline, the familiar zone is delimited by a comedy-romance movie he does not likes,``Pallbearer (1996)'', and the movie he likes, ``Shine (1996)'', which is of genres music, biography, and drama.
%Between the two movies example that delimit the user familiar zone, we have some mixed genres of movies such as comedy-romance-drama as indicated by the composite color of the movies title. 
%Our system automatically plays the recommendation story of a set of five movies, House Arrest (1996), The Crucible (1996), Carrie (1976), Nobody's Fool (1994), and Career Girls (1997), from familiar zone to diverse zone. 
%Without user input, our system continues to an additional story with a new set of movies as shown in the first row of Figure~\ref{case1InteractionsResults}.
%The user can adjust the preferences of typicality and familiarity or click on a recommended movie for more information (see Figure~\ref{case1InteractionsResults}).  

% %User 476: 28 year old male student living in Bolingbrook, IL


\textbf{The second user} is a 28 year old male student.
Figure~\ref{case1RecommendedList} shows the first recommendation results for the user. 
The latent dimension spreads out movies in drama, documentary and comedy genres, with the combined features of positive influence (such as love and drama) toward the left direction and negative influence (such as horror) toward the right. 
%On the storyline, the familiar zone is delimited by two comedy/drama movies ``When Harry Met Sally (1989)'' and ``Son in Law (1993)''.
Five movies are recommended from diverse to familiar zones: Wonderland (1997, Documentary), Fearless (1993, Drama), Man Without a Face (The 1993, Drama), Everest (1998, Documentary), and Miracle on 34th Street (1994, Drama).

Without user input, our system continues to an additional story with a new set of movies as shown in the first row of Figure~\ref{case1InteractionsResults}.
The user can also adjust the recommendation preferences and select their interested movies, shown on the second and third rows of Figure~\ref{case1InteractionsResults}.  

We compare the recommendation results with the user's watch history.
He has an average rating of 3.64 for 34 movies from comedy, drama, horror, and romance genres.
The rating records suggest that he likes drama and romance movies. 
Our recommendation results recommend majority movies in the familiar zone of the user and also introduce mixed adventure/documentary/drama movies for diversity.

\begin{figure}[htb]
\centering
\includegraphics[width = 3.3in]{case1RecommendedList1.EPS}
%\includegraphics[width = 7in]{./image/addedImages/case1RecommendedList2.eps}
\caption{The result of the second user describes the user preference of love/drama movies he liked toward the left to horror/comedy movies he disliked toward the right.}
\label{case1RecommendedList}
\end{figure}
%we can describe the combined feature on the best dimension as from comedy/romance (on the left) to drama/biography/music (on the right). 


\begin{figure}[htb]
\centering
\includegraphics[width = 3.3in]{case1RecommendedList2.EPS}\\
\vspace{+1mm}
\includegraphics[width = 3.3in]{case1FamiliarRecommendedList.EPS}\\
\vspace{+1mm}
\includegraphics[width = 3.3in]{case1TypicalDiverseRecommendedList.EPS}
\caption{Interaction examples for the second user.
The first row shows another story piece describing a different aspect of user's preference on Groundhog Day (1993, Comedy and Romance) over Son in Law (1993, Comedy).
The second row shows that all recommended movies come from the familiar zone when the user switch the familiar preference to 1.
The bottom row shows the result when both the typical and diverse preferences are set to 1 and the movie ``One Fine Day (1996)'' from the recommended list is selected.
%and select the movie ``Dangerous Minds (1995)'' from the recommended list in the result of the second row image.  
}
\label{case1InteractionsResults}
\end{figure}

%\subsubsection{User who rated lot movies and like very few of them}
% User 181: 26 year old male executive living in Baltimore, MD
\textbf{The third user} is a 26 year old male executive.
%His average rating in the dataset is 1.71. 
%The only movies he rated 5 starts are the comedy movie ``Birdcage, The 1996'' and the drama-romance movie ``Jerry Maguire (1996)''.
As shown in Figure~\ref{case2RecommendedList}, the latent dimension is delimited by drama-comedy movie ``Sabrina (1996)'' with a high rating on the left and the comedy movie ``Down Periscope (1996)'' with a low rating on the right. 
Our system recommends five movies from drama movies combined with other genres in the familiar zone, such as Cool Hand Luke (1967, Comedy and Drama), Philadelphia Story (1940, comedy romance), Private Part (1997, Comedy and Drama), Sophie’s Choice (1982, Drama and Romance) and Friday (1995, Comedy and Drama).
%familiar to d Storyiverse types - Now and Then (1995), Dangerous Minds (1995), Higher Leaning (1995), Quiz Show (1994), and Philadelphia (1993). 

We compare the recommendation results with the user's watch history.
He has rated 120 movies of several genres, such as comedy, drama, and horror.
Most of his ratings were 1 or 2 stars and he only rated six movies with 4 or 5 stars.
Our result reflects his low rating records with a large dislike region and a small like region.
The latent dimension describes movie types from drama-comedy movies toward the left, to various other movie genres that the user has rated low toward the right.
Such users are generally picky on movie selections, but our model still captures his favorite movie types and creates a matching recommendation story.


\begin{figure}[htb]
\centering
\includegraphics[width = 3.3in]{case2RecommendedList.EPS}
\caption{The result of the third user demonstrates a picky user who has rated many movies low.
Our approach identifies the user's preference on comedy-drama movies (toward the left) over the other genres (toward the right).
%The preferences are set to 0.5 for typical and 0.5 for familiar. 
%The story is told from familiar zone of the combined feature the user likes toward the extreme end (typical zone) of the distribution of the movies on the dimension.
}
\label{case2RecommendedList}
\end{figure}

\subsection{System Performance}

The preprocess of our system includes the computation described in Section 3, including process of rating records, construction of SVD space, selection of recommendable movies and latent dimensions for the user. 
The performance of this stage is depended on the sizes of movie database and rating records.
The preprocess takes 3-10 seconds for 30,000 ratings from 940 users for 370 movies and several minutes for the Movie100K dataset on a desktop computer with Intel Core i7 2.93 GHz processor.

During interactive recommendation stage, all the visualization, interaction, and storytelling processes described in Section 4 and 5 are interactive. 
This is essential for providing smooth user interaction in an online system.
It is achieved by only keeping the relevant data to the user during run time.
The performance is the main reason that rating records are used in our approach and many other popular online systems.
%In case additional data is included into consideration, such as the movie contents, the interactive performance of the system still needs to be maintained.

%%!TEX root = main.tex
\section{Evaluation}
\label{sec:eval}

In this section, we evaluate the performance of our unsupervised Ordered Word Mover's Distance metric and supervised Multi-scale Sentence Matching model with factorized sentences as input. We apply our algorithms to semantic textual similarity estimation tasks and sentence pair paraphrase identification tasks, based on four datasets: STSbenchmark, SICK, MSRP and MSRvid. 

\subsection{Experimental Setup}
\label{subsec:setup}


\begin{table}[tb]
  \caption{Description of evaluation datasets.}
  \label{tab:datasets}
  \begin{tabular}{lllll}
    \toprule
    Dataset & Task & Train & Dev & Test\\
    \midrule
    STSbenchmark & Similarity scoring & $5748$ & $1500$ & $1378$ \\
    SICK & Similarity scoring & $4500$ & $500$ & $4927$ \\
    MSRP & Paraphrase identification & $4076$ & - & $1725$ \\
    MSRvid & Similarity scoring & $750$ & - & $750$ \\
    \bottomrule
  \end{tabular}
  \vspace{-2mm}
\end{table}

We will start with a brief description for each dataset:
\begin{itemize}
\item \textbf{STSbenchmark}\cite{cer2017semeval}: it is a dataset for semantic textual similarity (STS) estimation. The task is to assign a similarity score to each sentence pair on a scale of 0.0 to 5.0, with 5.0 being the most similar.

\item \textbf{SICK}\cite{marelli2014sick}: it is another STS dataset from the SemEval 2014 task 1. It has the same scoring mechanism as STSbenchmark, where 0.0 represents the least amount of relatedness and 5.0 represents the most.

\item \textbf{MSRvid}: the Microsoft Research Video Description Corpus contains 1500 sentences that are concise summaries on the content of a short video. Each pair of sentences is also assigned a semantic similarity score between 0.0 and 5.0. 

\item \textbf{MSRP}\cite{quirk2004monolingual}: the Microsoft Research Paraphrase Corpus is a set of 5800 sentence pairs collected from news articles on the Internet. Each sentence pair is labeled 0 or 1, with 1 indicating that the two sentences are paraphrases of each other.
\end{itemize}

Table \ref{tab:datasets} shows a detailed breakdown of the datasets used in evaluation.
For STSbenchmark dataset we use the provided train/dev/test split.
The SICK dataset does not provide development set out of the box, so we extracted 500 instances from the training set as the development set.
For MSRP and MSRvid, since their sizes are relatively small to begin with, we did not create any development set for them.

One metric we used to evaluate the performance of our proposed models on the task of semantic textual similarity estimation is the Pearson Correlation coefficient, commonly denoted by $r$. Pearson Correlation is defined as:
\begin{equation}
\label{eq:pearson}
 r = cov(X,Y) /( \sigma_X \sigma_Y),
\end{equation}
where $cov(X,Y)$ is the co-variance between distributions X and Y, and $\sigma_X$, $\sigma_Y$ are the standard deviations of X and Y.
The Pearson Correlation coefficient can be thought as a measure of how well two distributions fit on a straight line. Its value has range [-1, 1], where a value of 1 indicates that data points from two distribution lie on the same line with a positive slope.
% Due to this unique property, we believe the Pearson Correlation coefficient is a strong indicator of the performance of our metric. 

Another metric we utilized is the Spearman's Rank Correlation coefficient. Commonly denoted by $r_s$, the Spearman's Rank Correlation coefficient shares a similar mathematical expression with the Pearson Correlation coefficient, but it is applied to ranked variables.
Formally it is defined as \cite{wiki:spearman}:
\begin{equation}
\label{eq:spearman}
 \rho = cov(rg_X, rg_Y) / (\sigma_{rg_X} \sigma_{rg_Y}),
\end{equation}
where $rg_X$, $rg_Y$ denotes the ranked variables derived from $X$ and $Y$. $cov(rg_X,rg_Y)$, $\sigma_{rg_X}$, $\sigma_{rg_Y}$ corresponds to the co-variance and standard deviations of the rank variables. The term ranked simply means that each instance in X is ranked higher or lower against every other instances in X and the same for Y. We then compare the rank values of X and Y with \ref{eq:spearman}. Like the Pearson Correlation coefficient, the Spearman's Rank Correlation coefficient has an output range of [-1, 1], and it measures the monotonic relationship between X and Y. A Spearman's Rank Correlation value of 1 implies that as X increases, Y is guaranteed to increase as well.
The Spearman's Rank Correlation is also less sensitive to noise created by outliers compared to the Pearson Correlation.

For the task of paraphrase identification, the classification accuracy of label $1$ and the F1 score are used as metrics. 

In the supervised learning portion, we conduct the experiments on the aforementioned four datasets. We use training sets to train the models, development set to tune the hyper-parameters and each test set is only used once in the final evaluation. For datasets without any development set, we will use cross-validation in the training process to prevent overfitting, that is, use $10\%$ of the training data for validation and the rest is used in training. For each model, we carry out training for 10 epochs. We then choose the model with the best validation performance to be evaluated on the test set.  


\subsection{Unsupervised Matching with OWMD}
\label{subsec:eval-owmd}

To evaluate the effectiveness of our Ordered Word Mover's Distance metric, we first take an unsupervised approach towards the similarity estimation task on the STSbenchmark, SICK and MSRvid datasets. Using the distance metrics listed in Table \ref{tab:compare-pearson} and \ref{tab:compare-spearman}, we first computed the distance between two sentences, then calculated the Pearson Correlation coefficients and the Spearman's Rank Correlation coefficients between all pair's distances and their labeled scores. We did not use the MSRP dataset since it is a binary classification problem.


In our proposed Ordered Word Mover's Distance metric, distance between two sentences is calculated using the order preserving Word Mover's Distance algorithm. For all three datasets, we performed hyper-parameter tuning using the training set and calculated the Pearson Correlation coefficients on the test and development set. We found that for the STSbenchmark dataset, setting $\lambda_1=10$, $\lambda_2=0.03$ produces the most optimal result. For the SICK dataset, a combination of $\lambda_1=3.5$, $\lambda_2=0.015$ works best. And for the MSRvid dataset, the highest Pearson Correlation is attained when $\lambda_1=0.01$, $\lambda_2=0.02$.
We maintain a max iteration of 20 since in our experiments we found that it is sufficient for the correlation result to converge.
During hyper-parameter tuning we discovered that using the Euclidean metric along with $\sigma=10$ produces better results, so all OWMD results summarized in Table \ref{tab:compare-pearson} and \ref{tab:compare-spearman} are acquired under these parameter settings. Finally, it is worth mentioning that our OWMD metric calculates the distances using factorized versions of sentences, while all other metrics use the original sentences. Sentence factorization is a necessary preprocessing step for the OWMD metric.


We compared the performance of Ordered Word Mover's Distance metric with the following methods:

\begin{itemize}
\item \textbf{Bag-of-Words (BoW)}: in the Bag-of-Words metric, distance between two sentences is computed as the cosine similarity between the word counts of the sentences.

\item \textbf{LexVec}~\cite{salle2016enhancing}: calculate the cosine similarity between the  averaged 300-dimensional LexVec word embedding of the two sentences. 

\item \textbf{GloVe}~\cite{pennington2014glove}: calculate the cosine similarity between the averaged 300-dimensional GloVe 6B word embedding of the two sentences. 

\item \textbf{Fastext}~\cite{joulin2016bag}: calculate the cosine similarity between the averaged 300-dimensional Fastext word embedding of the two sentences. 

\item \textbf{Word2vec}~\cite{mikolov2013efficient}: calculate the cosine similarity between the averaged 300-dimensional Word2vec word embedding of the two sentences.

\item \textbf{Word Mover's Distance (WMD)}~\cite{kusner2015word}: estimating the semantic distance between two sentences by WMD introduced in Sec.~\ref{sec:owmd}.
\end{itemize} 


\begin{table}[tb]
  \caption{Pearson Correlation results on different distance metrics.}
  \label{tab:compare-pearson}
  \begin{tabular}{c|cc|cc|c}
    \toprule
    \multirow{2}{*}{Algorithm} & \multicolumn{2}{c}{STSbenchmark} & \multicolumn{2}{c}{SICK} & MSRvid\\ 
     & Test & Dev & Test & Dev & Test\\
    \midrule
    BoW & $0.5705$ & $0.6561$ & $0.6114$ & $0.6087$ & $0.5044$ \\
    LexVec & $0.5759$ & $0.6852$ & $0.6948$ & $\mathbf{0.6811}$ & $0.7318$\\
    GloVe & $0.4064$ & $0.5207$ & $0.6297$ & $0.5892$  & $0.5481$ \\
    Fastext & $0.5079$ & $0.6247$ & $0.6517$ & $0.6421$  & $0.5517$  \\
    Word2vec & $0.5550$ & $0.6911$ & $\mathbf{0.7021}$ & $0.6730$  & $0.7209$  \\
    WMD & $0.4241$ & $0.5679$ & $0.5962$ & $0.5953$  & $0.3430$  \\
    OWMD & $\mathbf{0.6144}$ & $\mathbf{0.7240}$ & $0.6797$ & $0.6772$  & $\mathbf{0.7519}$  \\
    \bottomrule
  \end{tabular}
  \vspace{-4mm}
\end{table}

\begin{table}[tb]
  \caption{Spearman's Rank Correlation results on different distance metrics.}
  \label{tab:compare-spearman}
  \begin{tabular}{c|cc|cc|c}
    \toprule
    \multirow{2}{*}{Algorithm} & \multicolumn{2}{c}{STSbenchmark} & \multicolumn{2}{c}{SICK} & MSRvid\\ 
     & Test & Dev & Test & Dev & Test\\
    \midrule
    BoW & $0.5592$ & $0.6572$ & $0.5727$ & $0.5894$ & $0.5233$ \\
    LexVec & $0.5472$ & $0.7032$ & $0.5872$ & $0.5879$ & $0.7311$\\
    GloVe & $0.4268$ & $0.5862$ & $0.5505$ & $0.5490$  & $0.5828$ \\
    Fastext & $0.4874$ & $0.6424$ & $0.5739$ & $0.5941$  & $0.5634$  \\
    Word2vec & $0.5184$ & $0.7021$ & $0.6082$ & $0.6056$  & $0.7175$  \\
    WMD & $0.4270$ & $0.5781$ & $0.5488$ & $0.5612$  & $0.3699$  \\
    OWMD & $\mathbf{0.5855}$ & $\mathbf{0.7253}$ & $\mathbf{0.6133}$ & $\mathbf{0.6188}$  & $\mathbf{0.7543}$  \\
    \bottomrule
  \end{tabular}
  \vspace{-2mm}
\end{table}


Table \ref{tab:compare-pearson} and Table \ref{tab:compare-spearman} compare the performance of different metrics in terms of the Pearson Correlation coefficients and the Spearman's Rank Correlation coefficients.
We can see that the result of our OWMD metric achieves the best performance on all the datasets in terms of the Spearman's Rank Correlation coefficients.
It also produced the best Pearson Correlation results on the STSbenchmark and the MSRvid dataset, while the performance on SICK datasets are close to the best.
This can be attributed to the two characteristics of OWMD. First, the input sentence is re-organized into a predicate-argument structure using the sentence factorization tree. Therefore, corresponding semantic units in the two sentences will be aligned roughly in order. Second, our OWMD metric takes word positions into consideration and penalizes disordered matches. Therefore, it will produce less mismatches compared with the WMD metric.

% On the SICK dataset, although the result of our metric falls slightly behind Word2vec, LexVec on the test set and Word2vec on the development set, we still believe that it is a superior metric because it produced competitive results across multiple datasets. 

% Table \ref{tab:compare-spearman} presents the Spearman's Rank Correlation coefficients acquired with the same distance metrics. We can observe that our OWMD metric achieves the highest correlation scores on all three datasets. Which proves once again that OWMD is a better distance metric for the task of semantic similarity detection.

\subsection{Supervised Multi-scale Semantic Matching}
\label{subsec:eval-multilayer}

\begin{table*}[tb]
  \caption{A comparison among different supervised learning models in terms of accuracy, F1 score, Pearson's $r$ and Spearman's $\rho$ on various test sets.}
  \label{tab:sts}
  \begin{tabular}{c|cc|cc|cc|cc}
    \toprule
    \multirow{2}{*}{Model} & \multicolumn{2}{c}{MSRP} & \multicolumn{2}{c}{SICK} & \multicolumn{2}{c}{MSRvid} & \multicolumn{2}{c}{STSbenchmark}\\ 
     & Acc.(\%) & F1(\%) & $r$ & $\rho$ & $r$ & $\rho$ & $r$ & $\rho$ \\
    \midrule
    MaLSTM & $66.95$ & $73.95$ & $0.7824$ & $0.71843$ & $0.7325$ & $0.7193$ & $0.5739$ & $0.5558$\\
    Multi-scale MaLSTM & $\mathbf{74.09}$ & $\mathbf{82.18}$ & $\mathbf{0.8168}$ & $\mathbf{0.74226}$ & $\mathbf{0.8236}$ & $\mathbf{0.8188}$ & $\mathbf{0.6839}$ & $\mathbf{0.6575}$\\
    \midrule
    HCTI & $73.80$ & $80.85$ & $0.8408$ & $0.7698$ & $\mathbf{0.8848}$ & $\mathbf{0.8763}$  & $\mathbf{0.7697}$ & $\mathbf{0.7549}$ \\
    Multi-scale HCTI & $\mathbf{74.03}$ & $\mathbf{81.76}$ & $\mathbf{0.8437}$ & $\mathbf{0.7729}$ & $0.8763$ & $0.8686$  & $0.7269$ & $0.7033$  \\
    \bottomrule
  \end{tabular}
  \vspace{-2mm}
\end{table*}

The use of sentence factorization can improve both existing unsupervised metrics and existing supervised models. 
% We extend the normal Siamese model to Fig. \ref{fig:network} to take advantage of different level of information in the factorized sentence. 
To evaluate how the performance of existing Siamese neural networks can be improved by our sentence factorization technique and the multi-scale Siamese architecture, we implemented two types of Siamese sentence matching models, HCTI \cite{mueller2016siamese} and MaLSTM \cite{shao2017hcti}. HCTI is a Convolutional Neural Network (CNN) based Siamese model, which achieves the best Pearson Correlation coefficient on STSbenchmark dataset in SemEval2017 competition (compared with all the other neural network models). MaLSTM is a Siamese adaptation of the Long Short-Term Memory (LSTM) network for learning sentence similarity. As the source code of HCTI is not released in public, we implemented it according to \cite{shao2017hcti} by Keras \cite{chollet2015keras}. With the same parameter settings listed in paper \cite{shao2017hcti} and tried our best to optimize the model, we got a Pearson correlation of 0.7697 (0.7833 in paper \cite{shao2017hcti}) in STSbencmark test dataset.

We extended HCTI and MaLSTM to our proposed Siamese architecture in Fig. \ref{fig:network}, namely the Multi-scale MaLSTM and the Multi-scale HCTI. To evaluate the performance of our models, the experiment is conducted on two tasks: 1) semantic textual similarity estimation based on the STSbenchmark, MSRvid, and SICK2014 datasets; 2) paraphrase identification based on the MSRP dataset.

Table \ref{tab:sts} shows the results of HCTI, MaLSTM and our multi-scale models on different datasets. Compared with the original models, our models with multi-scale semantic units of the input sentences as network inputs significantly improved the performance on most datasets. 
Furthermore, the improvements on different tasks and datasets also proved the general applicability of our proposed architecture.

Compared with MaLSTM, our multi-scaled Siamese models with factorized sentences as input perform much better on each dataset. For MSRvid and STSbenmark dataset, both Pearson's $r$ and Spearman's $\rho$ increase about $10\%$ with Multi-scale MaLSTM. Moreover, the Multi-scale MaLSTM achieves the highest accuracy and F1 score on the MSRP dataset compared with other models listed in Table \ref{tab:sts}.

There are two reasons why our Multi-scale MaLSTM significantly outperforms MaLSTM model. First, for an input sentence pair, 
we explicitly model their semantic units with the factorization algorithm.
%we explicitly model the different scales of semantics of them with the semantic units produced by our sentence factorization algorithm. 
Second, our multi-scaled network architecture is 
specifically designed
%specially adapted to 
for multi-scaled sentences representations. Therefore, it is able to explicitly match a pair of sentences at different granularities.

We also report the results of HCTI and Multi-scale HCTI in Table \ref{tab:sts}. For the paraphrase identification task, our model shows better accuracy and F1 score on MSRP dataset. For the semantic textual similarity estimation task, the performance varies across datasets. On the SICK dataset, the performance of Multi-scale HCTI is close to HCTI with slightly better Pearson' $r$ and Spearman's $\rho$. However, the Multi-scale HCTI is not able to outperform HCTI on MSRvid and STSbenchmark. HCTI is still the best neural network model on the STSbenchmark dataset, and the MSRvid dataset is a subset of STSbenchmark.
Although HCTI has strong performance on these two datasets, it performs worse than our model on other datasets.
% Overall, the experimental results demonstrated the superior applicability and generalizability of our proposed models.
Overall, the experimental results demonstrated the general applicability of our proposed model architecture, which performs well on various semantic matching tasks.

% \begin{table}[tb]
%   \caption{Results of Accuracy and F1 score on MSRP test dataset.}
%   \label{tab:MSRP result}
%   \begin{tabular}{lllll}
%     \toprule
%     Model & Acc.(\%) & F1(\%)  \\
%     \midrule
%     MaLSTM & $66.95$ & $73.95$ \\
%     Factorized MaLSTM & $\mathbf{74.09}$ & $\mathbf{82.18}$ \\
%     HCTI & $73.80$ & $80.85$ \\
%     Factorized HCTI & $\mathbf{74.03}$ & $\mathbf{81.76}$ \\
%     \bottomrule
%   \end{tabular}
%   \vspace{0mm}
% \end{table}


% \begin{table}[tb]
%   \caption{Results of Pearson's $r$ and Spearman's $\rho$ on SICK test dataset.}
%   \label{tab:SICK result}
%   \begin{tabular}{lllll}
%     \toprule
%     Model & r & $\rho$ \\
%     \midrule
%     MaLSTM & $0.7824$ & $0.71843$ \\
%     Factorized MaLSTM & $\mathbf{0.8168}$ & $\mathbf{0.74226}$ \\
%     HCTI & $0.8408$ & $\mathbf{0.7698}$ \\
%     Factorized HCTI & $\mathbf{0.8429}$ & $0.7676$ \\
%     \bottomrule
%   \end{tabular}
%   \vspace{0mm}
% \end{table}


% \begin{table}[tb]
%   \caption{Results of Pearson's $r$ and Spearman's $\rho$ on MSRvid test dataset.}
%   \label{tab:MSRvid result}
%   \begin{tabular}{lll}
%     \toprule
%     Model & r & $\rho$  \\
%     \midrule
%     MaLSTM & $0.7325$ & $0.7193$ \\
%     Factorized MaLSTM & $\mathbf{0.8236}$ & $\mathbf{0.8188}$ \\
%     HCTI & $\mathbf{0.8848}$ & $\mathbf{0.8763}$ \\
%     Factorized HCTI & $0.8763$ & $0.8686$ \\
%     \bottomrule
%   \end{tabular}
%   \vspace{0mm}
% \end{table}



% \begin{table}[tb]
%   \caption{Results of Pearson's $r$ and Spearman's $\rho$ on STSbenchmark test dataset.}
%   \label{tab:STSbenchmark result}
%   \begin{tabular}{lllll}
%     \toprule
%     Model & r & $\rho$ \\
%     \midrule
%     MaLSTM & $0.5739$ & $0.5558$ \\
%     Factorized MaLSTM & $\mathbf{0.6839}$ & $\mathbf{0.6575}$ \\
%     HCTI & $\mathbf{0.7697}$ & $\mathbf{0.7549}$ \\
%     Factorized HCTI & $0.7269$ & $0.7033$ \\
%     \bottomrule
%   \end{tabular}
%   \vspace{0mm}
% \end{table}




%In this paper, 2D and 3D CNN models were used to generate pelvic sCTs from T1-weighted MR images. Our sCT generation methods were fully automated, requiring no deformable registration or manual segmentation of bone tissues. As shown in Figure~\ref{fig3}, the 2D and 3D CNN models generated high quality sCTs. MAE curves shown in Figure~\ref{fig4} indicated that both models could precisely estimate soft-tissue HU values but had difficulty in reproducing air and high-density bone tissues. 

The MAEs within the body contour across all patients were 40.5 $\pm$ 5.4 HU and 37.6 $\pm$ 5.1 HU for the 2D and 3D models, respectively. The time required for generating a pelvic sCT using our CNN models was about 5.5 s. Our MAE results are comparable to previous studies. Kim $et \ al.$\cite{RN41} presented a voxel-based weighted summation method that produced an MAE of 74.3 $\pm$ 3.9 HU. However, manual contouring of bone tissues required for this method can be tedious and time-consuming. An MAE of 40.5 $\pm$ 8.2 HU was achieved by Dowling $et \ al.$\cite{RN11} using an average MRI-CT atlas from 38 patients. Andreasen $et \ al.$\cite{RN42} reported an MAE of 54 $\pm$ 8 HU using an atlas-based method with pattern recognition, and its prediction time was about 20.8 min. Another random forest model proposed by Andreasen $et \ al.$\cite{RN43} generated sCTs with an MAE of 58 $pm$ 9 HU. A hybrid method suggested by Siversson $et \ al.$ \cite{RN45} obtained an MAE of 36.5 $\pm$ 4.1 HU when ignoring errors introduced by gas cavities. This hybrid method was implemented in the cloud-based commercial software MriPlanner (Spectronic Medical AB, Helsingborg, Sweden), which required 50 to 80 min to generate a sCT.\cite{RN45} The patch-based 3D context-aware generative adversarial network presented by Nie $et \ al.$\cite{RN26} achieved an MAE of 39.0 $\pm$ 4.6 HU. 

Our CNN models reproduced low-density bone as shown in Figure ~\ref{fig4}. The bone-region DSCs were 0.81 $\pm$ 0.04 and 0.82 $\pm$ 0.04 from the 2D and 3D models, respectively. These results are comparable to reported DSC results of 0.79 $\pm$ 0.12\cite{RN10} and 0.91$\pm$0.03{\cite{RN11}}, where the authors compared bone contours manually drawn on the sCT and CT.

It was feasible to train the proposed 3D model with 16 image volumes from scratch. Results of the Wilcoxon signed-rank tests shown in Table~\ref{tab1} demonstrated a statistically significant improvement in overall MAE, bone DSC, and bone precision of the 3D model compared to the 2D model. However, as shown in Figure~\ref{fig4}, the 2D model seemed to perform better in estimating the high-density bone HU values. It should be noted that smaller overall MAEs do not guarantee improved sCT dose calculation and patient positioning performance. While the models performed well, we will continue to acquire more patient data to potentially improve model accuracy and further test model differences.

As this was a retrospective study, the MR image voxel sizes were not matched, resulting in different voxel intensities between images. This may have affected the sCT generation accuracy although we applied intensity normalization. A potential study could examine how voxel size variations affects sCT estimation. 

The proposed 3D model can be implemented on a 12 GB GPU to process volumetric images with dimensions of 256 $\times$ 256 $\times$ 30. More GPU memory would be required to process higher resolution 3D images. Considering the limited access to multi-GPU systems, a 3D architecture with fewer convolutional layers could be considered to deal with higher resolutions. However, the performance could be affected by the reduced parameters and smaller receptive fields of the less complex model. Another approach would be to extract 30-slice sub-volumes from CT and MR images for training the 3D model. The sCT could then be generated by averaging 30-slice sCT sub-volumes produced by the model. 

A number of techniques could be investigated for improving model performance.  Nie $et \ al.$\cite{RN26} showed that introducing an additional adversarial discriminator improved overall sCT quality. The same approach could be adapted in our proposed 2D and 3D CNN models.  Non-rigid deformation\cite{RN44} could also be applied to both CT and MR images in the process of the on-the-fly data augmentation to produce more training pairs. Multiple MR images acquired with different sequences could be fed into models to provide more information for distinguishing different tissues. Multi-GPU systems with more memory would enable the exploration of larger batch sizes for training CNN models, which could reduce variances in gradient estimation and accelerate the training. 



%\section{Activation During Perception of Noisy Speech}\label{sec:Apps}
The dataset, provided as  {\tt data6} in the AFNI tutorial~\citet{cox96},
is originally from an fMRI study~\citet{nathandbeauchamp11} where
\begin{figure}[h]
\subfloat[]{\includegraphics[width=0.5\columnwidth]{figs/AR-FAST-001-Visual-crop}}
\subfloat[]{\includegraphics[width=0.5\columnwidth]{figs/AR-FAST-001-Audio-crop}}
%\subfloat[]{\includegraphics[width=0.33\textwidth]{figs/AM-FAST-005-diff}}
\caption{ AR-FAST-identified activation regions on SPMs obtained by
  fitting ~\ref{eq:lm} with   AR($\hat{p}$) to AFNI's {\tt data6} for
  (a) visual-reliable stimulus and (b) audio-reliable
  stimulus.}
\label{fig:AMSmoothingAFNI}
\end{figure}
\begin{comment}
\begin{figure*}[h]
\subfloat[]{\includegraphics[width=0.25\textwidth]{figures/Visual_AM-crop}}
\subfloat[]{\includegraphics[width=0.25\textwidth]{figures/Audio_AM-crop}}
\subfloat[]{\includegraphics[width=0.25\textwidth]{figures/Visual_Audio_AM-crop}}
\caption{Activation areas obtained using AR-FAST with in {\it AFNI data6} on the SPM obtained
  after fitting AR($\hat{p}$) of the (a) Visual-reliable, (b) Audio-reliable and (c) the difference contrast between Visual-reliable and Audio-reliable.}
\label{fig:AMSmoothingAFNI}
\end{figure*}
\end{comment}
a subject heard and saw a female volunteer speak words, separately, in
two different formats. The audio-reliable setting had the subject
clearly hear the spoken word but see a degraded image of the speaker
while the visual-reliable case had the subject clearly see the speaker
vocalize the word but the audio was of reduced quality.  There were
three experimental runs, each 
consisting of a randomized design of 10 blocks, equally divided into blocks of
audio-reliable and visual-reliable stimuli. %An echo-planar imaging
% sequence (TR=2s) was used to obtain
$\mbox{T}_2^*$-weighted images with volumes of $80 \times 80 \times
33$ (with voxels of dimension $2.75 \times  2.75 \times 3.0\  mm^3$)
from  echo-planar sequences (TR=2s) 
were obtained  over $152$ time-points. Our interest was in determining 
activation corresponding to the audio
($H_0:\beta_{a}=0$) and visual
($H_0:\beta_{v}=0$) tasks.
%, as well as their contrast  ($H_0:\beta_{v} - \beta_{a}=0$). The
%first two cases have one-sided alternatives while the contrast in
%activation corresponds to a two-sided alternative.
At each voxel, we fitted AR models for 
$p=0,1,2,3,4,5$ and chose $p$ with the highest BIC. 
Figure \ref{fig:AMSmoothingAFNI} uses AFNI and Surface Mapping (SUMA)
to display activated regions obtained using AR-FAST on the SPM:
see  Figure~\ref{fig:Visual-Audio} for  maps drawn from ALL-FAST, AS,
AWS and CT. We used $\alpha = 0.01$ because of the high (greater than
4) upper percentile of the voxel-wise estimated CNRs. Most of the activation 
occurs in Brodmann areas 18 and 19 (BA18 and BA19)
which comprise the
occipital cortex %in the human brain,  accounting for the bulk of the
                 %volume of the occipital lobe. Both areas form part
                 %of the visual association area while BA 19, also the occipital lobe cortex as well. Along with BA18, it comprises
and the extrastriate (or peristriate) cortex. In humans with normal
sight,  this area is for visual association where 
feature-extraction, shape recognition, attentional and multimodal
integrating functions occur. We also see increased activation in
the STS, which recent
studies~\citep{grossman2001brain} have related to  distinguishing
voices from environmental sounds, 
stories versus nonsensical speech, moving faces versus moving objects,
biological motion and so on. ALL-FAST performs similarly as AR-FAST,
while the other methods also identify the same regions but they identify
a lot more activated 
voxels, some of which appear to be false positives. Although a 
detailed analysis of the results of this study is beyond the purview
of this paper, we note that AR-FAST 
finds interpretable results even when applied to a single
subject high-level cognition experiment. 
\begin{comment}
\begin{table}[h!]
\centering
\caption{Coordinates of the maximum $t$-statistic and its corresponding value}\label{tab:maxt}
\begin{tabular}{c|c}
Task & $(x,y,z) mm$  \\
\hline
Visual-Audio & (-30.162,86.221,6.349) \\
Audio & (-27.412,75.221,-5.651)  \\
 Visual & (-30.162, 80.721, 15.349) \\
\hline
\end{tabular}
\end{table}
\end{comment}


\section{Conclusions and Future Work}

This paper presents an interactive recommendation approach for the popular application of online movie recommendation with the general public as end users.
We have studied LSM, which enables us to translate abstract data and complex recommendation algorithms to a set of semantic concepts, provide explicit interaction of search preferences, and construct meaningful recommendation stories.
The LSM can be easily combined with other recommendation algorithms, as we separate the estimation of recommendation degrees and the latent space.
The interactive recommendation approach automatically generates storytelling animations for recommendation, with the supports of several interactive exploration functions for users to adjust the search results explicitly. 
Different from traditional recommendation algorithms, the interactive recommendation approach emphasizes the visual communication between users and recommendation systems for engaging users and improving search experiences.
Both of our results can also be extended to recommend other online products or services.

As future work, we plan to perform formal evaluations on the effectiveness of interactive recommendation.
We are interested in studying the suitable amount of information for different users, such as the general public and movie experts, so that different versions of narrative visualization can be developed to suit for different needs. 
We also plan to develop variations of recommendation stories, such as long versions that can combine multiple latent dimensions, for different recommendation tasks.
At the end, we are interested in integrating other techniques for movie recommendation, such as text analysis approaches for mining useful information from the movie reviews.
The results will enrich the contents of narrative visualization and may provide better search experiences.


%At the end, we need to study the state-of-the-art of recommendation systems to ensure that our approach can be combined for end users smoothly.

%we are interested in continuing to study the capabilities of interactive storytelling as a visual analysis tool for various visualization applications.
%For movie recommendation, we expect to improve user satisfaction by refining our approach to accommodate subtle differences of user tastes.

%% if specified like this the section will be committed in review mode
%\acknowledgments{
%The authors wish to thank A, B, C. This work was supported in part by
%a grant from XYZ.}


\bibliographystyle{abbrv}
%%use following if all content of bibtex file should be shown
%\nocite{*}
\bibliography{story}
\end{document}

up to ten (10) pages

Please provide supplemental videos in QuickTime MPEG-4 or DivX version 5, and use TIFF, JPEG, or PNG for supplemental images.

Technique papers introduce novel techniques or algorithms that have not previously appeared in the literature, or that significantly extend known techniques or algorithms, for example by scaling to datasets of much larger size than before or by generalizing a technique to a larger class of uses. The technique or algorithm description provided in the paper should be complete enough that a competent graduate student in visualization could implement the work, and the authors should create a prototype implementation of the methods. Relevant previous work must be referenced, and the advantage of the new methods over it should be clearly demonstrated. There should be a discussion of the tasks and datasets for which this new method is appropriate, and its limitations. Evaluation through informal or formal user studies, or other methods, will often serve to strengthen the paper, but are not mandatory.

System papers present a blend of algorithms, technical requirements, user requirements, and design that solves a major problem. The system that is described is both novel and important, and has been implemented. The rationale for significant design decisions is provided, and the system is compared to documented, best-of-breed systems already in use. The comparison includes specific discussion of how the described system differs from and is, in some significant respects, superior to those systems. For example, the described system may offer substantial advancements in the performance or usability of visualization systems, or novel capabilities. Every effort should be made to eliminate external factors (such as advances in processor performance, memory sizes or operating system features) that would affect this comparison. For further suggestions, please review "How (and How Not) to Write a Good Systems Paper" by Roy Levin and David Redell, and "Empirical Methods in CS and AI" by Toby Walsh.

Application / Design Study papers explore the choices made when applying visualization and visual analytics techniques in an application area, for example relating the visual encodings and interaction techniques to the requirements of the target task. Similarly, Application papers have been the norm when researchers describe the use of visualization techniques to glean insights from problems in engineering and science. Although a significant amount of application domain background information can be useful to provide a framing context in which to discuss the specifics of the target task, the primary focus of the case study must be the visualization content. The results of the Application / Design Study, including insights generated in the application domain, should be clearly conveyed. Describing new techniques and algorithms developed to solve the target problem will strengthen a design study paper, but the requirements for novelty are less stringent than in a Technique paper. Where necessary, the identification of the underlying parametric space and its efficient search must be aptly described. The work will be judged by the design lessons learned or insights gleaned, on which future contributors can build. We invite submissions on any application area.
