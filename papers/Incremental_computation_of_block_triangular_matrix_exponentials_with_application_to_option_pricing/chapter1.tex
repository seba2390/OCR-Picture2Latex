\section{Introduction}

%\fbox{\parbox{\textwidth}{ State the mathematical problem first:
%\begin{itemize} \item Show 2-by-2 block triangular matrix. Explain
%why this is the building for incrementally computing expm for block
%triangular matrices.  Important: Why can't we just use expm(G)?
%Daniel: Need more information than just expm(G) to compute
%off-diagonal block.  \item Top level explanation where this appears
%in applications. You have a nested basis, discretize an operator in
%that basis, which preserves the nestedness. Incremetally refining the
%basis, one wants to update the exponential. Important example:
%polynomial diffusion models. (last paragraph of the current intro)
%\item Basic idea of the paper: Store intermediate quantities produced
%by scaling-and-squaring and update this quantities when going from G
%to $\tilde G$.  \end{itemize}}}

We study the problem of computing the matrix exponential for a sequence of nested block
triangular matrices. In order to give a precise problem formulation, consider a sequence of
block upper triangular matrices $G_0, G_1, G_2, \dotsc$ of the form
\begin{equation}
    \label{eq:G_n}
    G_n =
    \begin{bmatrix}
        G_{0,0} & G_{0,1}  & \cdots & G_{0,n} \\
                & G_{1,1}  & \cdots & G_{1,n} \\
                &          & \ddots & \vdots  \\
                &          &        & G_{n,n} \\
    \end{bmatrix}
    \in \R^{d_n \times d_n},
\end{equation}
where all diagonal blocks $G_{n,n}$ are square.  In other words, the
matrix $G_i$ arises from $G_{i-1}$ by appending a block column
(and adjusting the size).  We aim at computing the sequence of
matrix exponentials
\begin{equation}
    \label{eq:exp_sequence}
    \exp(G_0),\; \exp(G_1),\; \exp(G_2),\; \dotsc .
\end{equation}

One could, of course, simply compute each of the exponentials~\eqref{eq:exp_sequence} individually using standard techniques (see~\cite{Moler2003} for an overview).
However, the sequence of matrix exponentials~\eqref{eq:exp_sequence}
inherits the nested structure from the matrices $G_n$ in~\eqref{eq:G_n},
i.e., $\exp(G_{n-1})$ is a leading principle submatrix of $\exp(G_n)$.
In effect only the last block column of $\exp(G_n)$ needs to be computed and the goal of this paper is to explain how this can be achieved in a numerically safe manner.

In the special case where the spectra of the diagonal blocks $G_{n,n}$
are separated, Parlett's method~\cite{Parlett1976} yields -- in principle -- an efficient computational scheme:  Compute $F_{0,0} \defby
\exp(G_{0,0})$ and $F_{1,1} \defby \exp(G_{1,1})$ separately, then the
missing (1,2) block of $\exp(G_1)$ is given as the unique solution $X$ to
the Sylvester equation
\begin{equation*}
    G_{0,0} X - X G_{1,1} = F_{0,0} G_{0,1} - G_{0,1} F_{1,1}.
\end{equation*}
Continuing in this manner all the off-diagonal blocks required to compute~\eqref{eq:exp_sequence} could be
obtained from solving Sylvester equations. However, it is well known (see chapter 9 in~\cite{Higham2008}) that Parlett's method is numerically safe only when the spectra of the diagonal blocks are well separated, in the sense that all involved Sylvester equations are well conditioned. Since we consider the block structure as fixed, imposing such a condition would severely limit the scope of applications; it is certainly not met by the application we discuss below. 

A general class of applications for the described incremental
computation of exponentials arises from the matrix representations of
a linear operator $\mathcal{G} : V \to V$ restricted to a sequence of nested, finite
dimensional subspaces of a given infinite dimensional vector space $V$.
More precisely, one starts with a
finite dimensional subspace $V_0$ of $V$ with a basis $\mathcal{B}_0$.
Successively, the vector space $V_0$ is extended to $V_1 \subseteq V_2
\subseteq \cdots \subseteq V$ by generating a sequence of nested bases
$\mathcal{B}_0 \subseteq \mathcal{B}_1 \subseteq
\mathcal{B}_2\subseteq \cdots$. Assume that $\mathcal{G} V_n \subseteq V_n$ for all
$n=0,1,\dotsc$, and consider the sequence of matrix representations
$G_{n}$ of $\mathcal{G}$ with respect to $\mathcal{B}_n$.  Due to the
nestedness of the bases, $G_n$ is constructed from
$G_{n-1}$ by adding the columns representing the action of
$\mathcal{G}$ to $\mathcal{B}_n \setminus \mathcal{B}_{n-1}$.
As a
result, we obtain a sequence of matrices structured as
in~\eqref{eq:G_n}. 
%Note that an incremental computation of the matrix
%exponential is useful only if an appropriate size of the basis
%$\mathcal{B}_n$ is {not known a priori}.  Otherwise, one would
%obviously only compute $\exp(G_n)$, by standard methods, for an
%appropriate value of~$n$.

A specific example for the scenario outlined above arises in computational finance, when pricing options
based on polynomial diffusion models; see~\cite{filipovic2016polynomial}. As we explain in more detail in section~\ref{sct3}, in this setting $\mathcal{G}$ is the generator of a stochastic differential equation (SDE), and $V_n$ are nested subspaces of multivariate polynomials.
Some pricing techniques require the computation of certain conditional moments that can be extracted from the matrix exponentials~\eqref{eq:exp_sequence}.
While increasing $n$ allows for a better approximation of the option price, the value of $n$ required to attain a desired accuracy is usually not known a priori. Algorithms that choose $n$ adaptively can be expected to rely on the incremental computation  of the whole sequence~\eqref{eq:exp_sequence}. 

Exponentials of block triangular matrices have also been studied in other contexts.  For two-by-two block triangular matrices, Dieci and
Papini study conditioning issues in~\cite{Dieci2001}, and a discussion
on the choice of scaling parameters for using Pad\'e approximants to
exponential function in~\cite{Dieci2000}.  In the case where the
matrix is also block-Toeplitz, a fast exponentiation algorithm is
developed in~\cite{Bini2016}.

The rest of this paper is organized as follows. In
section~\ref{sec:scaling_and_squaring} we give a detailed
description of our algorithm for incrementally computing exponentials
of block triangular matrices as in~\eqref{eq:G_n}. In section 3 we
discuss polynomial diffusion models, and some pricing
techniques which necessitate the use of such an incremental algorithm. Finally, 
numerical results are presented in section 4.
