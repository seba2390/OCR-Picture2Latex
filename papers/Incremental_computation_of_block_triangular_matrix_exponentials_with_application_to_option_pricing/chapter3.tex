\section{Option pricing in polynomial models}\label{sct3}

The main purpose of this section is to explain how certain option pricing techniques require the sequential computation of matrix exponentials for block triangular matrices. The description will necessarily be rather brief; we refer to, e.g., the textbook~\cite{Elliot2005} for more details.

Because we are evaluating at initial time $t=0$, the price of a certain option expiring at time $\tau>0$ consists of computing an expression of the form
\begin{equation}\label{priceexpectation}
e^{-r\tau} \mathbb{E}[f(X_\tau)],
\end{equation}
where $(X)_{0 \leq t \leq \tau}$ is a $d$-dimensional stochastic process modelling the price of financial assets over the time interval $[0,\tau]$, $f : \mathbb{R}^d \to \mathbb{R}$ is the so-called payoff function and $r$ represents a fixed interest rate.  In the following, we consider stochastic processes described by an SDE of the form
\begin{align}\label{SDE}
dX_t=b(X_t)dt+\Sigma(X_t)dW_t,
\end{align}
where $W$ denotes a $d$-dimensional Brownian motion, $b : \mathbb{R}^d \mapsto \mathbb{R}^{d}$, and $\Sigma : \mathbb{R}^d \mapsto \mathbb{R}^{d \times d}$. 



\subsection{Polynomial diffusion models}

During the last years, polynomial diffusion models have become a versatile tool in financial applications, including option pricing. In the following, we provide a short summary and refer to the paper by Filipovi\'c and Larsson \cite{filipovic2016polynomial} for the mathematical foundations. 

For a polynomial diffusion process one assumes that the coefficients of the vector $b$ in~\eqref{SDE} and the matrix $A : = \Sigma \Sigma^T$ satisfy
\begin{equation}\label{polynomial}
A_{ij} \in \text{Pol}_2(\mathbb{R}^d), \qquad b_i \in \text{Pol}_1(\mathbb{R}^d)  \quad \text{for} \quad i,j = 1,\ldots,d.
\end{equation}
Here, $\text{Pol}_n(\mathbb{R}^d)$ represents the set of $d$-variate polynomials of total degree at most $n$, that is,
\begin{equation*}
\text{Pol}_n(\mathbb{R}^d):= \Big\{\sum_{0 \le |\textbf{k}|\le n} \alpha_{\textbf{k}} x^{\bf{k}}| x \in \mathbb{R}^d, \alpha_{\textbf{k}} \in \mathbb{R}\Big\},
\end{equation*}
where we use multi-index notation: $\mathbf{k}=(k_1, \dots, k_d) \in \mathbb{N}_0^d$, $|\mathbf{k}|:=k_1+\dots+k_d$ and $x^{\bf{k}}:=x_1^{k_1}\dots x_d^{k_d}$. In the following, $\text{Pol}(\mathbb{R}^d)$ represents the set of all multivariate polynomials on $\mathbb{R}^d$.\

Associated with $A$ and $b$ we define the partial differential operator $\mathcal{G}$ by 
\begin{equation}\label{generator}
\mathcal{G}f=\frac{1}{2}\tr(A \nabla^2 f)+b^T \nabla f.
\end{equation}
which represents the so called generator
%\footnote{
%The generator of a stochastic process $(X_t)_{0 \leq t \leq T}$ is defined as $\mathcal{G}f(x):= \lim_{t \to 0}\frac{\mathbb{E}^x[f(X_t)]-f(x)}{t}$, where the expectation is taken with respect to the law $\mathbb{P}^x$ of $X$ given $X_0=x$. See for more details.}
for~\eqref{SDE}, see~\cite{Oksendal2003}. It can be directly verified that~\eqref{polynomial} implies that $\text{Pol}_n(\mathbb{R}^d$) is invariant under $\mathcal{G}$  for any $n \in \mathbb{N}$, that is, 
\begin{equation}\label{PreservingProperty}
\mathcal{G}\text{Pol}_n(\mathbb{R}^d) \subseteq \text{Pol}_n(\mathbb{R}^d).
\end{equation}
\begin{remark}
    \begin{rm}
In many applications, one is interested in solutions to $\eqref{SDE}$ that lie on a state space $E \subseteq \mathbb{R}^d$ to incorporate, for example, nonnegativity. This problem is largely studied in~\cite{filipovic2016polynomial}, where existence and uniqueness of solutions to \eqref{SDE} on several types of state spaces $E \subseteq \mathbb{R}^d$ and for large classes of $A$ and $b$ is shown.
    \end{rm}
\end{remark}
Let us now fix a basis of polynomials $\mathcal{H}_n=\{h_1, \dots, h_N\}$ for $\text{Pol}_n(\mathbb{R}^d)$, where $N= \dim \text{Pol}_n(\mathbb{R}^d) = \binom{n+d}{n}$, and write
\begin{equation*}
H_n(x)=(h_1(x), \dots, h_N(x))^T.
\end{equation*}
Let $G_n$ denote the matrix representation with respect to $\mathcal{H}$ of the linear operator $\mathcal{G}$ restricted to $\text{Pol}_n(\mathbb{R}^d)$. By definition,
\begin{equation*}
\mathcal{G}p(x)=H_n(x)^T G_n \vec{p}.
\end{equation*}
for any $p\in \text{Pol}_n(\mathbb{R}^d)$ with coordinate vector $\vec{p} \in \R^N$ with respect to $\mathcal{H}_n$.
By Theorem 3.1 in \cite{filipovic2016polynomial}, the  corresponding polynomial moment can be computed from
\begin{equation}\label{condmoments}
\mathbb{E}[p(X_\tau)]=H_n(X_0)^Te^{\tau G_n} \vec{p}.
\end{equation}
% \begin{remark}
% To be more precise, Theorem 3.1 in~\cite{filipovic2016polynomial} provides a more general formula for the computation of conditional polynomial moments at any time $t \leq T$. However, in our applications we are always interested in the special case $t=0$ and we use \eqref{condmoments}. 
% \end{remark}

The setting discussed above corresponds to the scenario described in the introduction. We have a sequence of subspaces
\[ \text{Pol}_0(\mathbb{R}^d) \subseteq \text{Pol}_1(\mathbb{R}^d) \subseteq \text{Pol}_2(\mathbb{R}^d) \subseteq \cdots \subseteq \text{Pol}(\mathbb{R}^d)\]
and the polynomial preserving property~\eqref{PreservingProperty} implies that the matrix representation $G_n$ is block upper triangular with $n+1$ square diagonal blocks of size \[1, d, \binom{1+d}{2}, \ldots, \binom{n+d-1}{n}.\]

In the rest of this section we introduce two different pricing techniques that require the incremental computation of polynomial moments of the form~\eqref{condmoments}. %This needs the iterative extension of the vector space $\text{Pol}_n(\mathbb{R}^d)$ (increasing $n$) and the computation of the sequence of matrix exponentials \eqref{eq:exp_sequence}.
%Our algorithm described in chapter 2 can be used in this context.
%%%%%%%%%%%%%%%%%%%%%%%%%%%%%%%%%%%%%%%%%%%%%%%%%%%%%%%%%%%%

\subsection{Moment-based option pricing for Jacobi models} \label{sec:jacobix}
The Jacobi stochastic volatility model is a special case of a polynomial diffusion model and it is characterized by the SDE
\begin{align*}
&dY_t=(r-V_t/2)dt + \rho \sqrt{Q(V_t)} dW_{1t}+\sqrt{V_t-\rho^2Q(V_t)}dW_{2t},\\
&dV_t=\kappa(\theta -V_t)dt+\sigma \sqrt{Q(V_t)}dW_{1t},
\end{align*}
where 
\begin{equation*}
Q(v)=\frac{(v-v_{\min})(v_{\max}-v)}{(\sqrt{v_{\max}}-\sqrt{v_{\min}})^2},
\end{equation*}
for some $0 \leq v_{\min} < v_{\max}$. Here, $W_{1t}$ and $W_{2t}$ are independent standard Brownian motions and the model parameters satisfy the conditions $\kappa \geq 0$, $\theta \in [v_{\min},v_{\max}]$, $\sigma >0$, $r \geq 0$, $\rho \in [-1,1]$.\
In their paper, Ackerer et al.~\cite{ackerer2016jacobi} use this model in the context of option pricing where the price of the asset is specified by $S_t \defby e^{Y_t}$ and $V_t$ represents the squared stochastic volatility. In the following, we briefly introduce the pricing technique they propose and  explain how it involves the incremental computation of polynomial moments. 

Under the Jacobi model with the discounted payoff function $f$ of an European claim, 
the option price~\eqref{priceexpectation} at initial time $t=0$ can be expressed as 
\begin{equation}\label{price}
\sum_{n \geq 0} f_n l_n,
\end{equation}
where $\{f_n, n\geq 0\}$ are the Fourier coefficients of $f$ and $\{l_n, n\geq 0\}$ are Hermite moments. As explained in~\cite{ackerer2016jacobi}, the Fourier coefficients can be conveniently computed in a recursive manner. The Hermite moments are computed using~\eqref{condmoments}. Specifically,
consider the monomial basis of $\text{Pol}_n(\R^2)$:
\begin{equation}\label{eq:basisJacobi}
H_n(y,v) \defby (1,y,v,y^2,yv,v^2,\dots,y^n,y^{n-1}v,\dots,v^n)^T.
\end{equation}
Then 
\begin{equation} \label{eq:hermitemoments}
l_n = H_n(Y_0,V_0)^Te^{\tau G_n} \vec{h}_n,
\end{equation}
where $\vec{h}_n$ contains the coordinates with respect to~\eqref{eq:basisJacobi} of 
\begin{equation*}
 \frac{1}{\sqrt{n!}} h_n \left(\frac{y-\mu_w}{\sigma_w}\right),
\end{equation*}
with real parameters $\sigma_w, \mu_w$ and the $n$th Hermite polynomial ${h}_n$.

Truncating the sum \eqref{price} after a finite number of terms allows
one to obtain an approximation of the option price.
Algorithm~\ref{alg:jacobi} describes a heuristic to selecting the truncation based on the absolute value of the summands, using Algorithm~\ref{alg:full} for computing the required moments incrementally. 
\begin{algorithm}[ht]
\caption{Option pricing for the European call option under the Jacobi stochastic volatility model}
\begin{algorithmic}[1]\label{alg:jacobi}
\REQUIRE Model and payoff parameters, tolerance $\epsilon$
\ENSURE Approximate option price
\STATE $n=0$
\STATE Compute $l_0$, $f_0$; set $\text{Price} = l_0 f_0$.
\WHILE{$|l_n f_n| > \epsilon \cdot \text{Price}$}
\STATE $n=n+1$
\STATE \label{line:jacobiexp} Compute $\exp(\tau G_n)$ using Algorithm~\ref{alg:step}.
\STATE \text{Compute Hermite moment $l_n$ using~\eqref{eq:hermitemoments}. }
\STATE \text{Compute Fourier coefficient $f_n$ as described in~\cite{ackerer2016jacobi}.}
\STATE $\text{Price}=\text{Price}+l_n f_n$;
\ENDWHILE
\end{algorithmic}
\end{algorithm}
%In chapter 4 we will show some numerical examples based on algorithm \ref{algJacobi} where we consider the European call option.\

As discussed in Section~\ref{sec:scaling_and_squaring}, a norm estimate for $G_n$ is instrumental for choosing a priori the scaling parameter in the scaling and squaring method. The following lemma provides such an estimate for the model under consideration. 
\begin{lemma} \label{lemmanormJ}
Let $G_n$ be the matrix representation of the operator $\mathcal{G}$ defined in~\eqref{generator}, with respect to the basis~\eqref{eq:basisJacobi} of $\mathrm{Pol}_{n}(\mathbb{R}^2)$.  Define 
\begin{equation*}
\alpha:=\frac{\sigma (1+v_{\min}v_{\max}+v_{\max}+v_{\min})}{2(\sqrt{v_{\max}}-\sqrt{v_{\min}})^2}.
\end{equation*}
Then the matrix 1-norm of $G_n$ is bounded by
\begin{equation*}
n( r + \kappa + \kappa\theta - \sigma \alpha ) + \frac12 n^2 ( 1 + |\rho| \alpha + 2 \sigma \alpha).
\end{equation*}
\begin{proof}
The operator $\mathcal{G}$ in the Jacobi model takes the form
\begin{equation*}
\mathcal{G}f(y,v)=\frac{1}{2}\tr(A(v) \nabla^2 f(y,v))+b(v)^\top \nabla f(y,v),
\end{equation*}
where 
\begin{equation*}
b(v)=\begin{bmatrix}
  r-v/2 \\
  \kappa(\theta -v)
\end{bmatrix},
 \quad 
A(v)=\begin{bmatrix}
  v & \rho \sigma Q(v)\\
 \rho \sigma Q(v) &  \sigma^2 Q(v)
\end{bmatrix}.
\end{equation*} 
Setting $S\defby (\sqrt{v_{\max}}-\sqrt{v_{\min}})^2$, we consider the action of the generator $\mathcal{G}$ on a basis element $y^pv^q$:
\begin{align*}
\mathcal{G} y^p v^q=&y^{p-2} v^{q+1}p \frac{p-1}{2} - y^{p-1}v^{q+1} p \Big(\frac{1}{2}+\frac{q\rho \sigma}{S}\Big)+y^{p-1} v^qp \Big(r+q \rho \sigma \frac{v_{\max} + v_{\min}}{S}\Big)\\
& -y^{p-1}v^{q-1}\frac{pq \rho \sigma v_{\max} v_{\min}}{S} - y^p v^q q \Big(\kappa + \frac{q-1}{2} \frac{\sigma^2}{S} \Big) \\
& - y^p v^{q-2} q \frac{q-1}{2} \frac{\sigma^2 v_{\max}v_{\min}}{S} +y^p v^{q-1} q \Big(\kappa \theta + \frac{q-1}{2} \sigma^2 \frac{ v_{\max} + v_{\min}}{S} \Big). 
\end{align*}
For the matrix 1-norm of $G_n$, one needs to determine the values of $(p,q) \in \mathcal{M}:= \{(p,q) \in \mathbb{N}_0 \times \mathbb{N}_0 | p+q \leq n\}$ for which the $1$-norm of the coordinate vector of $\mathcal{G} y^p v^q$ becomes maximal. Taking into account the nonnegativity of the involved model parameters and replacing $\rho$ by $|\rho|$, we obtain an upper bound as follows:
\begin{align*}
& p \frac{p-1}{2} + p \Big(\frac{1}{2}+\frac{q|\rho|\sigma}{S}\Big)+ p \Big(r+q |\rho|\sigma \frac{v_{\max} + v_{\min}}{S}\Big)+ \frac{pq |\rho|\sigma v_{\max} v_{\min}}{S}\\
&+ q \Big(\kappa + \frac{q-1}{2} \frac{\sigma^2}{S} \Big) + q \frac{q-1}{2} \frac{\sigma^2 v_{\max}v_{\min}}{S} + q \Big(\kappa \theta + \frac{q-1}{2} \sigma^2 \frac{ v_{\max} + v_{\min}}{S} \Big) \\
=& pr + q\kappa(\theta+1) + \frac12 p^2 + 2pq|\rho| \alpha +  q (q-1) \sigma \alpha  \\
\le & n( r + \kappa + \kappa\theta ) + \frac12 n^2 + 2pq|\rho| \alpha + n(n-1)\sigma \alpha. 
\end{align*}
This completes the proof, noting that the maximum of $pq$ on $\mathcal{M}$ is bounded by $n^2 / 4$
over $\mathcal{M}$.
\end{proof}
\end{lemma}

The result of Lemma~\ref{lemmanormJ} predicts that the norm of $G_n$ grows, in general, quadratically. This prediction is confirmed numerically for parameter settings of practical relevance.
% If the scaling parameter is to be chosen by using the 2-norm, above lemma allows us to estimate $\norm{G_n}_2$ by means of the norm inequality
% \begin{equation}\label{estimation}
% \norm{G_n}_2 \leq \sqrt{M} \norm{G_n}_{1},
% \end{equation}
% where $M$ is the size of the matrix $G_n$. Consider for example the set of parameters
% \begin{equation*}
% \kappa=0.5, \quad \theta=0.04, \quad  \sigma=0.15, \quad  \rho=-0.5, \quad  v_{\min}=0.01,\quad  v_{\max}=1,\quad  r=0.
% \end{equation*}
% Figure \ref{fig:normJacobi} shows (left) the norm of the matrices $G_l, l=1,\cdots,n$ and the estimation performed by using the inequality 
% \eqref{estimation}. On the right side, one can see the scaling parameter we would choose for all different values of $l$.
% \begin{figure}[t]
% \centering 
% \includegraphics[height=4.8 cm]{Figures/Norm_estimation_Jacobi}
% \includegraphics[height=4.8 cm]{Figures/scaling_parameters_Jacobi}
% \caption{\textsl{Left:} Norm estimations \eqref{estimation}. \textsl{Right:} Corresponding scaling parameters.}
% \label{fig:normJacobi}
% \end{figure}
%%%%%%%%%%%%%%%%%%%%%%%%%%%%%%%%%%%%%%%%%%%%%%%%%%%%%%%%%%%%
\subsection{Moment-based option pricing for Heston models} \label{sec:hestonx}

The Heston model is another special case of a polynomial diffusion model, characterized by the SDE 
\begin{align*}
&dY_t=(r-V_t/2)d_t + \rho \sqrt{V_t} dW_{1t}+\sqrt{V_t}\sqrt{1-\rho^2}dW_{2t},\\
&dV_t=\kappa(\theta -V_t)dt+\sigma \sqrt{V_t}dW_{1t},
\end{align*}
with model parameters satisfying the conditions $\kappa \geq 0$, $\theta \geq 0$, $\sigma >0$, $r \geq 0$, $\rho \in [-1,1]$. As before, the asset price is modeled via $S_t \defby e^{Y_t}$, while $V_t$ represents the squared stochastic volatility.

Lasserre et al.~\cite{Lasserre2006} developed a general option pricing technique based on moments and semidefinite programming (SDP). In the following we briefly explain the main steps and in which context an incremental computation of moments is needed. In doing so, we restrict ourselves to the specific case of the Heston model and European call options.

Consider the payoff function $f(y) \defby (e^{y}-e^K)^+$ for a certain log strike value $K$. Let $\nu(dy)$ be the $Y_\tau$-marginal distribution of the joint distribution of the random variable $(Y_\tau,V_\tau)$. Define the restricted measures $\nu_1$ and $\nu_2$ as $\nu_1=\nu |_{(- \infty , K]}$ and $\nu_2=\nu |_{[K, \infty)}$. By approximating the exponential in the payoff function with a Taylor series truncated after $n$ terms, the option price~\eqref{priceexpectation} can be written as a certain linear function $L$ in the moments of $\nu_1$ and $\nu_2$, i.e., 
\begin{equation*}
\mathbb{E}[f(Y_\tau)]= L(n,\nu_1^0, \cdots, \nu_1^n,\nu_2^0, \cdots, \nu_2^n),
\end{equation*}
where $\nu_i^m$ represents the $m$th moment of the $i$th measure.\

A lower / upper bound of the option price can then be computed by solving the optimization problems 
 \begin{align} \label{optiproblem}
 SDP_n \defby \left\{
                \begin{array}{ll}
                  \min/ \max \hspace{0.1 cm}&L(n,\nu_1^0, \cdots, \nu_1^n,\nu_2^0, \cdots, \nu_2^n)\\
\text{subject to   }\hspace{0.1 cm}  &\nu_1^j+\nu_2^j=\nu^j, \quad j=0, \cdots, n\\ 
                  &\nu_1 \text{ is a Borel measure on } (- \infty , K],\\
                  &\nu_2 \text{ is a Borel measure on } [K,\infty).\\
		\end{array}
              \right.
\end{align}
Two SDP arise when writing the last two conditions in \eqref{optiproblem} via moment and localizing matrices, corresponding to the so-called truncated Stieltjes moment problem.

Formula \eqref{condmoments} is used in this setting to compute the moments $\nu^j$. Increasing the relaxation order $n$ iteratively allows us to find sharper bounds (this is trivial because increasing $n$ adds more constraints). One stops as soon as the bounds are sufficiently close. Algorithm~\ref{algoSDP} summarizes the resulting pricing algorithm.
\begin{algorithm}[ht]
\caption{Option pricing for European options based on SDP and moments relaxation}
\begin{algorithmic}[1]\label{algoSDP}
\REQUIRE Model and payoff parameters, tolerance $\epsilon$
\ENSURE Approximate option price
\STATE   $n=1$, $\mathrm{gap}=1$
\WHILE{\text{$\mathrm{gap} > \epsilon$}}
\STATE Compute $\exp(\tau G_n)$ using Algorithm~\ref{alg:step}
\STATE Compute moments of order $n$ using \eqref{condmoments}
\STATE Solve corresponding $SDP_n$ to get $LowerBound$ and $UpperBound$ 
\STATE  $\mathrm{gap} = |UpperBound - LowerBound|$
\STATE  $n=n+1$
\ENDWHILE
\end{algorithmic}
\end{algorithm}

The following lemma extends the result of Lemma~\ref{lemmanormJ} to the Heston model.
\begin{lemma}
    \label{lem:heston_est}
Let $G_n$ be the matrix representation of the operator $\mathcal{G}$ introduced above with respect to the basis~\eqref{eq:basisJacobi} of $\mathrm{Pol}_{n}(\mathbb{R}^2)$.  
Then the matrix 1-norm of $G_n$ is bounded by
\begin{equation*}
n( r + \kappa + \kappa\theta - \frac{ \sigma^2}{2} ) + \frac12 n^2 ( 1 + |\rho| \frac{\sigma}{2} + \sigma^2).
\end{equation*}
\begin{proof}
Similar to the proof of Lemma \ref{lemmanormJ}.
\end{proof}
\end{lemma}

%TODO:
%\begin{itemize}
% \item Explain Heston model as special case of polynomial diffusion.
% \item Short paragraph indicating how the polynomial moments are used in option pricing.
% \item Move algorithm from numerical experiments here
% \item Include lemma on block norms of $G$
%\end{itemize}

