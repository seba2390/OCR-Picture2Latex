%% This is file `elsarticle-template-1-num.tex',
%%
%% Copyright 2009 Elsevier Ltd
%%
%% This file is part of the 'Elsarticle Bundle'.
%% ---------------------------------------------
%%
%% It may be distributed under the conditions of the LaTeX Project Public
%% License, either version 1.2 of this license or (at your option) any
%% later version.  The latest version of this license is in
%%    http://www.latex-project.org/lppl.txt
%% and version 1.2 or later is part of all distributions of LaTeX
%% version 1999/12/01 or later.
%%
%% The list of all files belonging to the 'Elsarticle Bundle' is
%% given in the file `manifest.txt'.
%%
%% Template article for Elsevier's document class `elsarticle'
%% with numbered style bibliographic references
%%
%% $Id: elsarticle-template-1-num.tex 149 2009-10-08 05:01:15Z rishi $
%% $URL: http://lenova.river-valley.com/svn/elsbst/trunk/elsarticle-template-1-num.tex $
%%
\documentclass[preprint,12pt]{elsarticle}

%% Use the option review to obtain double line spacing
%% \documentclass[preprint,review,12pt]{elsarticle}

%% Use the options 1p,twocolumn; 3p; 3p,twocolumn; 5p; or 5p,twocolumn
%% for a journal layout:
%% \documentclass[final,1p,times]{elsarticle}
%% \documentclass[final,1p,times,twocolumn]{elsarticle}
%% \documentclass[final,3p,times]{elsarticle}
%% \documentclass[final,3p,times,twocolumn]{elsarticle}
%% \documentclass[final,5p,times]{elsarticle}
%% \documentclass[final,5p,times,twocolumn]{elsarticle}

%% if you use PostScript figures in your article
%% use the graphics package for simple commands
%% \usepackage{graphics}
%% or use the graphicx package for more complicated commands
%% \usepackage{graphicx}
%% or use the epsfig package if you prefer to use the old commands
%% \usepackage{epsfig}

%% The amssymb package provides various useful mathematical symbols
\usepackage{amssymb}
\usepackage{bm}
\usepackage{amsmath}
\usepackage{cleveref}
\usepackage{extarrows}
\usepackage{subfigure}
\usepackage{color} 
\usepackage{boxedminipage}
\usepackage{diagbox} % 加载宏包
\usepackage{multirow}
\usepackage[sc]{mathpazo}
\newcommand{\tabincell}[2]{\begin{tabular}{@{}#1@{}}#2\end{tabular}}
%%--------------------------------------------------------------------------------------------------------------
\usepackage{nomencl}
\makenomenclature
%%--------------------------------------------------------------------------------------------------------------


%\usepackage[dvipdfmx]{graphicx}
%\usepackage{bmpsize}
%% The amsthm package provides extended theorem environments     
%% \usepackage{amsthm}

%% The lineno packages adds line numbers. Start line numbering with
%% \begin{linenumbers}, end it with \end{linenumbers}. Or switch it on
%% for the whole article with \linenumbers after \end{frontmatter}.
%% \usepackage{lineno}

%% natbib.sty is loaded by default. However, natbib options can be
%% provided with \biboptions{...} command. Following options are
%% valid:

%%   round  -  round parentheses are used (default)
%%   square -  square brackets are used   [option]
%%   curly  -  curly braces are used      {option}
%%   angle  -  angle brackets are used    <option>
%%   semicolon  -  multiple citations separated by semi-colon
%%   colon  - same as semicolon, an earlier confusion
%%   comma  -  separated by comma
%%   numbers-  selects numerical citations
%%   super  -  numerical citations as superscripts
%%   sort   -  sorts multiple citations according to order in ref. list
%%   sort&compress   -  like sort, but also compresses numerical citations
%%   compress - compresses without sorting
%%
%% \biboptions{comma,round}

% \biboptions{}

\journal{}
%\journal{Nuclear Physics B}

\begin{document}

\begin{frontmatter}

%% Title, authors and addresses

%% use the tnoteref command within \title for footnotes;
%% use the tnotetext command for the associated footnote;
%% use the fnref command within \author or \address for footnotes;
%% use the fntext command for the associated footnote;
%% use the corref command within \author for corresponding author footnotes;
%% use the cortext command for the associated footnote;
%% use the ead command for the email address,
%% and the form \ead[url] for the home page:
%%
%% \title{Title\tnoteref{label1}}
%% \tnotetext[label1]{}
%% \author{Name\corref{cor1}\fnref{label2}}
%% \ead{email address}
%% \ead[url]{home page}
%% \fntext[label2]{}
%% \cortext[cor1]{}
%% \address{Address\fnref{label3}}
%% \fntext[label3]{}

\title{Efficient calculation of the integral equation for simulating 2D TE scattering in a homogeneous medium using the Ewald method and a Gabor frame discretization}

%% use optional labels to link authors explicitly to addresses:
%% \author[label1,label2]{<author name>}
%% \address[label1]{<address>}
%% \address[label2]{<address>}

\author[label1]{Xinyang Lu}
      \address[label1]{School of Mathematical Sciences, Zhejiang University, Hangzhou 310027, Zhejiang, China}
       \address[label2]{Department of Electrical Engineering, Eindhoven University of Technology, P.O. Box 513, 5600 MB Eindhoven, The Netherlands}
\author[label2]{M. C. van Beurden}
\author[label1]{Qingbiao Wu}
      

\begin{abstract}
%% Text of abstract
We utilize the domain integral equation formulation to simulate two-dimensional transverse electric scattering in a homogeneous medium and a summation of modulated Gaussian functions to approximate the dual Gabor window.
% In the spatial representation, we employ the piecewise-linear functions along the $x$ direction and Gabor frame with the Gaussian window along the $z$ direction. Meanwhile, the dual Gabor window is approximated by a summation of modulated Gaussian functions. 
Then we apply Ewald Green function transformation to separate the integrals related to $x$ and $z$ in the integral equation, which %also 
produce Gaussian functions. These Gaussian functions in the integrands can be integrated analytically, which greatly simplifies the calculation process. Finally, we discuss the convergence and the selection of the Ewald splitting parameter $\mathcal{E}$.
\end{abstract}

\begin{keyword}
%% keywords here, in the form: keyword \sep keyword

%% MSC codes here, in the form: \MSC code \sep code
%% or \MSC[2008] code \sep code (2000 is the default)
scattering \sep domain integral equation \sep Green function \sep Ewald method  \sep Gabor frame 
\end{keyword}

\end{frontmatter}

%%
%% Start line numbering here if you want
%%
% \linenumbers
%In the following sections, a large number of symbols will be introduced. As a summary, the interpretation of most symbols are listed as follows,
%\begin{center}
%\begin{boxedminipage}[b]{1\linewidth}

\printnomenclature[1in]
%makeindex elsarticle-template-1-num.nlo -s nomencl.ist -o elsarticle-template-1-num.nls

%\end{boxedminipage}
%\end{center}
%% main text
%%------------------------------------------------------------------------------------------------
%%------------------------------------------------------------------------------------------------
\section{Introduction}
\label{introduction}
 \text{The} domain integral equation approach {has been used} to solve the electromagnetic scattering problems in recent decades \cite{PP1991Aweak, AIM1996EB, Lalanne2007Numerical}. {The Conjugate Gradient Fast Fourier Transform (CGFFT) \cite{Zwamborn1992The} and pre-corrected FFT \cite{Phillips1994Proceedings} have been efficient ways to solve the integral equation.}
 The Green function play{s} an important role in this method. How to overcome the singularity and coupling of the variables in the Green function is the key to solve the integral equation. Dilz and {Van Beurden} \cite{Dilz2016The} proposed 
the mixed spatial and spectral approach with Gabor frame discretization. It utilizes the Fourier transformation to eliminate the coupling and deal with the singularity in spectral domain by scaling coordinates. The numerical results demonstrate that it
yields an efficient and accurate way to numerically simulate the scattering from a finite dielectric object in a homogeneous background medium. A Gabor frame \cite{Bastiaans1981A, Bastiaans1995Gabor} is rapidly convergent both in spatial and spectral domain, which makes it possible for us to use the mixed spatial and spectral method to solve the domain integral function efficiently. 

The Gabor frame is {presented} in \cite{Feichtinger1998Gabor} and there is a brief introduction in Section 2.2, in which the formulas show that the dual window $\eta(x)$ plays an important role in the calculation of Gabor coefficients. Since the Gabor coefficients are not uniquely determined because of oversampling, the choice for the dual window function is not unique. Zak transform with generalized Moore-Penrose pseudo inverse is {a} popular way to calculate $\eta(x)$ \cite{Bastiaans1995Gabor, Werther2005Dual} and $\eta$ obtained via this method  is the optimum solution in the sense of minimum $\mathcal{L}^2$ norm \cite{Feichtinger1998Gabor, Janssen1994Signal}. However we can only obtain discrete values of $\eta(x)$, which makes it difficult to derive analytical results of these integrals related to $\eta$.  Here, we assume that 
{the dual window is represented via the original Gabor frame of the expansion.} Then we can simplify the integral containing $\eta$ by using the property of Gaussian functions.

In the integral equation, we must deal with the singularity problem of {the} Green's function.  Although a coordinate scaling to smoothen the branchcut of the Green function in the spectral domain works well \cite{Dilz2016The}, we want {a} faster and more accurate means for evaluating the Green function in the spatial domain. Ewald Green's function transformation, proposed by P. P. Ewald and described in \cite{Ewald1921Die, Ewald1970On},  provides the integral formula of the {free-space} Green's function for {three-dimensional} (3D) problems \cite{Jordan1986An} and 2D problems \cite{Mathis1996A}. Applying this formula to the integral equation, the integrals related to $x$ and $z$ are separated, which greatly reduces the difficulty of the integral. Meanwhile, the integral related to $x$ also has the Gaussian integral form and it can be combined with $\eta$ to get some analytic results. 

To {handle} the singularity of Green function, we use the Ewald method \cite{Capolino2005Efficient, Capolino2007Efficient,  Komanduri20161} that splits the integral {representation} of Green function into two parts. The selection of the Ewald splitting parameter $\mathcal{E}$ {is} discussed. % to find the optimum value. 
Only $G_{spectral}$ suffers from the singularity in spatial domain and we utilize the Fourier {transformation} to solve $G_{spectral}$ in spectral domain. Finally, what we need to calculate is several {one-dimensional} integrals for the products of the continuous functions. The numerical examples indicate that the results obtained from our proposed method match well that in \cite{Dilz2016The} with less computation and storage.
%%------------------------------------------------------------------------------------------------
%%------------------------------------------------------------------------------------------------
\section{Statement of the problem}
\label{Statement of the problem}
The derivation of the problem formulation can be found in  \cite{Dilz2016The} and \cite{RJDilz2017spatialspectral}. We will only present the main formulas used in the calculation and more details can be found in Chapter 2 and Chapter 5 in \cite{RJDilz2017spatialspectral}. 

%%%------------------------------------------------------------------------------------------------
\subsection{Problem formulation}
For two dimensional transverse electric scattering in a homogeneous isotropic dielectric medium, the final equation we will solve is as follows:
 \begin{small}
 \begin{equation}
 \begin{split}
 &\qquad \chi(x,z)E^s(x,z) = \\
 & k^2_0\varepsilon_{rb}\chi(x,z)\int \limits_x\int \limits_z G(x,z|x',z')\cdot\chi(x',z') \cdot  [E^i(x',z') + E^s(x',z')]dx'dz',
 \end{split}
 \end{equation}
 \end{small}
 with the scattered electric field $E^s(x,z) $, the wavenumber in vacuum $k^2_0 = \omega^2 \mu_0 \varepsilon_0$, the relative permittivity of the background medium $\varepsilon_{rb}$, and  the Green function $G(x,z|x',z')$ corresponding to the Helmholtz equation. The contrast function $\chi(x,z)$ is defined as:
 \begin{equation}
 \chi (x,z) = \frac{\varepsilon_r(x,z)}{\varepsilon_{rb}} - 1,
\end{equation}
 where $\varepsilon_r(x,z)$ denotes the relative permittivity containing the scattering object.
 
 For the sake of simplicity, we assume $\varepsilon_{rb} = 1$.
 
%%%------------------------------------------------------------------------------------------------
\subsection{Spatial discretization}
The same discretization in $x$ and $z$ as in \cite{Dilz2016The} and \cite{RJDilz2017spatialspectral} is employed here. Below, we give a brief summary of the discretization.

% From the previous section, we know that the support set of $\chi (x,z)$ is finite. Without loss of generality, we assume the support set is contained in the region bounded by $x \in [-W,W]$ , $z\in [z_{min}, z_{max}]$.

Along the $z$ direction, piecewise-linear functions $\Lambda_k$ are employed as expansion functions. The definition of $\Lambda_k$ is given by
\begin{small}
\begin{equation}
\Lambda_k (z) = \left \{
\begin{array}{ll} 
1 - \frac{|z - k\Delta - z_{min}|}{\Delta} \qquad \text{if} \qquad |z - k\Delta - z_{min}| < \Delta , \\
0 \qquad \text{if} \qquad |z  - k\Delta - z_{min}| > \Delta.
\end{array}\right. 
\end{equation}
\end{small}
Along the $x$ direction, a Gabor frame is employed to approximate $E(x,z)$ for fixed $z$. We have
\begin{small}
\begin{subequations}
\begin{equation}
\begin{split}
\chi (x, z_l) E^s (x, z_l)& = k^2_0  \chi(x,z_l)\int \limits^{z_{max}}_{z_{min}} \int \limits^{\infty}_{-\infty} G(x,z_l|x',z')\chi(x',z') E(x',z') dx' dz' \\
% &\triangleq \sum_{m,n} f_{mn,l}g_{mn}(x) ,
\end{split}
\end{equation}
\begin{equation}
\chi(x,z)E(x,z) = \chi E^i + \chi E^s = \sum^{N_k}_{k=0}\sum_{m,n} J_{mn,k}{g^w_{mn}}(x)\Lambda_k(z), 
\end{equation}
\label{chi_E}
\end{subequations}
\end{small}

where $N_k$ is the total number of expansion functions in the $z$ direction and 
\begin{small}
\begin{equation}
g^w_{mn}(x) = g^w(x - \alpha m X)e^{j\beta K n x}, \qquad {g^w}(x) = 2^{\frac{1}{4}}\exp \left( -\pi\frac{x^2}{X^2}\right),
\end{equation}
\end{small}
with the spatial window width $X$ and $K = 2\pi/X$. $\alpha$ and $\beta$ are the spatial and spectral oversampling rates respectively, such that $\alpha \beta  < 1$ . 
It is convenient to constrain $\alpha \beta$ to be a rational number in practice \cite{Bastiaans1995Gabor}, that is, $\alpha \beta =\frac{q}{p}$ with $q,p \in \mathbb{Z}^{+}$. And according to the calculation process of {the} Gabor dual window based on Zak transform, we need to choose smaller $q$ and $p$ to reduce the computation. In the following, we will find that the computation has no relation with $p$ and $q$ in our proposed method, thus we can choose oversampling parameters more freely.

According to the definition of the dual Gabor window, if $f(x)$ can be represented as
\begin{small}
$$ f(x) = \sum^{\infty}_{m =-\infty} \sum^{\infty}_{n =-\infty}f_{mn} {g^w_{mn}}(x),$$
\end{small}
then, the Gabor coefficients $\{f_{mn}\}$ can be obtained via
\begin{small}
\begin{equation}
f_{mn} = \int \limits^{\infty}_{-\infty} f(x) \eta^*_{mn}(x) dx ,
\label{gabor_coefficient_via_eta}
 \end{equation}
 \end{small}
where $ \eta_{mn}(x) = \eta (x - \alpha m X)e^{j\beta K n x}$ is the dual window and the asterisk denotes the conjugation operation.
%%%------------------------------------------------------------------------------------------------
%%%------------------------------------------------------------------------------------------------
\section{The Ewald Green function transformation}
\label{the_ewald_transformation}
For the Helmholtz equation in $\mathbb{R}^2$, i.e.
 \begin{small}
 \begin{equation}
 \Delta \phi + k^2_0 \phi = f,
 \end{equation}
 \end{small}
 the Green function as the fundamental solution that satisfies the radiation conditions at infinity for the time convention $\exp(j\omega t)$ of  the above equation is  
 \begin{small}
 \begin{equation}
G(x,z|x',z') = \frac{1}{4j}H^{(2)}_0 (k_0R),
 \end{equation}
 \end{small}
 where $H^{(2)}_0$ denotes the zeroth order Hankel function of the second kind \cite{Capolino2005Efficient} and  $R = \sqrt{(x-x')^2 + (z - z')^2}$.

 From \cite{Capolino2005Efficient}, we also obtain the Ewald transformation of the above Green function,
 \begin{small}
 \begin{equation}
 G(x,z|x',z')  = \frac{1}{2\pi}\int \limits^{\infty}_0\; \frac{\exp \left( -R^2 \xi^2 + \frac{k_0^2}{4\xi^2} \right)}{\xi} \; d\xi .
 \label{green_ewald}
 \end{equation}
 \end{small}
 According to the constraint conditions for the integration path of the Ewald transformation described in \cite{Capolino2005Efficient,Arens2013Analysing}, we utilize the integration contour, as shown in Fig.~\ref{integral_path}.
 \begin{figure}[!h]
  \centering\includegraphics[width=3.2in]{picture/pdflatex_fig/ulitized_integral_path2.png}
 \caption{Contour path we employed in the Ewald transformation.}
 \label{integral_path}
 \end{figure}

 The definition of the integration path is 
 \begin{small}
 \begin{equation}
 \xi(w) = \left\{\begin{array}{ll}
 (1 + j)w \qquad 0 <w< \mathcal{E}/{2}, \\
w + (\mathcal{E} - w) j \qquad \mathcal{E}/{2} < w< \mathcal{E}, \\
w  \qquad w > \mathcal{E} ,\end{array} \right. 
 \label{integral_path_p}
 \end{equation}   
 \end{small}
 where $ w \in (0, \infty)$. 

In the Ewald transformation, the integral in Eq.~\eqref{green_ewald} is split in two parts and we denote them as 
\begin{small}
\begin{subequations}
\begin{equation}
G_{spectral}(x,z|x',z') = \frac{1}{2\pi}\int \limits^{\mathcal{E}}_0 
\; \frac{\exp \left( -R^2\xi^2 + \frac{k^2_0}{4\xi^2} \right)}{\xi} \; d\xi, \\
\end{equation}
\begin{equation}
 G_{spatial}(x,z|x',z') = \frac{1}{2\pi}  \int \limits^{\infty}_\mathcal{E}
 \; \frac{\exp \left( -R^2\xi^2 + \frac{k^2_0}{4\xi^2} \right)}{\xi} \; d\xi .
 \end{equation}
 \end{subequations}
 \end{small}
%%------------------------------------------------------------------------------------------------
\subsection{The spatial part of the Green function}
The spatial part of the Green function is
\begin{small}
\begin{equation}
G_{spatial}(x,z|x',z') = \frac{1}{2\pi} \int \limits^{\infty}_\mathcal{E} 
\;\frac{\exp \left( -R^2\xi^2 + \frac{k^2_0}{4\xi^2} \right)}{\xi} \; d\xi,
\end{equation}
\end{small}
where $\xi \in \mathbb{R}$ and $\xi \geq \mathcal{E}$. That means the integral path in the above equation is the real axis from $\mathcal{E}$ to positive infinity. The coordinates are now separable, owing to the square of $R$.
%%------------------------------------------------------------------------------------------------
\subsection{The spectral part of the Green function}
The spectral part of the Green function is
\begin{small}
\begin{equation}
G_{spectral}(x,z|x',z') = \frac{1}{2\pi} \int \limits^{\mathcal{E}}_0 
\;\frac{\exp \left( -R^2\xi^2 + \frac{k^2_0}{4\xi^2} \right)}{\xi} \; d\xi .
\label{G_spectral_2.2}
\end{equation}
\end{small}
Its Fourier transform is 
\begin{small}
\begin{equation}
\label{G_spectral_original}
\widehat{G}_{spectral}(k_x,z|z')
= \frac{1}{2\pi} \int \limits^{\mathcal{E}}_0  \;\frac{1}{\xi} e^{-(z-z')^2\xi^2 + \frac{k_0^2}{4\xi^2}} \left[  \; \int \limits^{\infty}_{-\infty} \; e^{-x^2 \xi^2 - jk_x x} \; dx\right] d\xi.
\end{equation}
\end{small}
We require that $\mathrm{Re}\{\xi^2\} > 0$ to guarantee the convergence of the $\xi$-integral in Eq.~\eqref{G_spectral_2.2}, which is automatically satisfied for the chosen path as shown in Fig.~\ref{integral_path}. Once this is ensured, the result is not affected by the interchange of the order of integration. According to  the closed-form evaluation of the $\xi$-integral using the formula (2.33) in \cite{2014Table}
\begin{small}
 \begin{equation}
\int \limits^{\infty}_{-\infty} e^{-a\xi^2 + b\xi}\; d\xi = \sqrt{\frac{\pi}{a}}e^{\frac{b^2}{4a}},
 \label{xi_integral}
 \end{equation}
 \end{small}
 we get
 \begin{small}
\begin{equation}
\widehat{G}_{spectral}(k_x,z|z') \xlongequal{\zeta = 1/\xi} \frac{1}{2\sqrt{\pi}}\int \limits^{\infty}_{1/\mathcal{E}} \;  e^{-(z-z')^2/\zeta^2+ (k_0^2 - k_x^2)\zeta^2/4} \; d\zeta ,
\end{equation}
\end{small}
where the integral path $\xi$ is defined by Eq.~\eqref{integral_path_p} and
$ \zeta = \frac{1}{\xi(w)}$. Replacing $w$ by $\frac{1}{w}$ does not change the path $\zeta$, thus we rewrite $\zeta$ as
\begin{small}
\begin{equation}
\zeta 
=\frac{1}{\xi(\frac{1}{w}) } 
=
 \left\{ \begin{array}{ll}
\frac{w - (\mathcal{E}w^2 -w)j}{1 + (\mathcal{E}w - 1)^2} \qquad {1}/\mathcal{E} < w < {2}/\mathcal{E}, \\
\\
\frac{1-j}{2} w \qquad {2}/{\mathcal{E}} < w < \infty .\\
\end{array}\right.  
\label{integral_path_q}
\end{equation}
\end{small}
%%------------------------------------------------------------------------------------------------
%%------------------------------------------------------------------------------------------------
\section{Calculation of the domain integral representation}
 Substituting the Ewald transformation Eq.~\eqref{green_ewald} in Eq.~\eqref{chi_E}, we obtain
 \begin{small}
 \begin{equation}
\begin{split}
 &\chi(x,z_l) E^s(x,z_l)  
 = \frac{k^2_0}{2\pi} \sum_{m,n,k}  J_{mn,k}\cdot  \\
& \quad \int \limits^{\infty}_0    \; 
 \frac{ e^{\frac{k_0^2}{4\xi^2}}}{\xi}  \cdot 
  \left[ \; \int \limits^{\infty}_{-\infty} dx' \; e^{-(x-x')^2\xi^2} \cdot  {g^w_{mn}}(x') \right]\cdot  
 \left[\; \int \limits^{z_{max}}_{z_{min}} dz' \; \Lambda_k(z')\cdot e^{-(z' - z_l)^2\xi^2} \right] d\xi
\end{split}
 \end{equation}
 \end{small}
Owing to the Ewald transformation, the integrals related to $x'$ and $z'$ are separable. We split the above integral in two parts,
 \begin{small}
 \begin{subequations}
 \begin{equation}
\begin{split}
\chi E_{spatial}(x,z_k) &= \int \limits^{z_{max}}_{z_{min}} \int \limits^{\infty}_{-\infty} G_{spatial}(x,z_k|x',z') \cdot \chi E(x',z')\; dx'dz' \\
&= \sum_{m,n} f^{spatial}_{mn,k} {g^w_{mn}}(x),
\end{split}
\label{chi_E_spatial}
\end{equation}
\begin{equation}
\begin{split}
\chi E_{spectral}(x,z_k) &= \int \limits^{z_{max}}_{z_{min}} \int \limits^{\infty}_{-\infty} G_{spectral}(x,z_k|x',z') \cdot \chi E(x',z')\; dx'dz' \\
&=  \sum_{m,n} f^{spectral}_{mn,k} {g^w_{mn}}(x).
\end{split}
\label{chi_E_spectral}
 \end{equation}
 \end{subequations}
 \end{small}
 %Then we calculate them respectively.
In the following, we further simplify these two representations by working out the integrals over $x'$ and $z'$.
%%------------------------------------------------------------------------------------------------
\subsection{Spatial part of the electric field}
According to Eq.~\eqref{chi_E_spatial} and Eq.~\eqref{gabor_coefficient_via_eta}, we have
\begin{small}
\begin{equation}
\begin{split}
f^{spatial}_{st,l} &= \int \limits^{\infty}_{-\infty} \chi E_{spatial}(x,z_l) \cdot \eta^*_{st}(x) dx \\
&= \frac{1}{2\pi}\sum_{m,n,k}J_{mn,k} \int \limits^{\infty}_{\mathcal{E}}  \; \frac{e^{\frac{k_0^2}{4\xi^2}}}{\xi} \cdot \left[ \; \int \limits^{z_{max}}_{z_{min}}\; \Lambda_k(z') \cdot e^{-(z_l-z')^2\xi^2 } \;  dz'\right]
\cdot \\
&\qquad  \quad 
 \Bigg{\{}  \int \limits^{\infty}_{-\infty}  \; \eta^*_{st}(x) \left[\; \int \limits^{\infty}_{-\infty} \; e^{-(x-x')^2\xi^2} \cdot {g^w_{mn}}(x') \; dx'\right] \; dx\Bigg{\}} d\xi .
\end{split}
\label{f_spatial_mnl}
\end{equation}
\end{small}


\subsubsection{The integral related to $x$ and $x'$}
\label{integral_x_spatial}
The integral related to  $x$ and $x'$ in Eq.~\eqref{f_spatial_mnl} is 
\begin{small}
\begin{equation}
\begin{split}
& \quad  \int \limits^{\infty}_{-\infty} \; \eta^*_{st}(x) \left[ \; \int \limits^{\infty}_{-\infty} \; e^{-(x-x')^2\xi^2} \cdot g_{mn}(x') \; dx'\right] \; dx \\
&=  \int \limits^{\infty}_{-\infty} \int \limits^{\infty}_{-\infty}  \; e^{-(x-x')^2\xi^2} \cdot {g^w}(x' - \alpha mX) \cdot e^{j\beta K n x'} \cdot \eta^* (x - \alpha s X) \cdot e^{-j\beta K t x} \; dx'dx .\\
\end{split}
\label{integral_eta_g_e}
\end{equation}
\end{small}
We assume that the dual Gabor window $\eta(x)$ can be represented by a weighted sum of modulated Gaussian functions of the Gabor frame used for the expansion as follows,
\begin{equation}
\eta (x) = \sum^{N_u}_{u} \sum^{N_v}_{v} a_{u v}\cdot {g^w_{uv}}(x).
\end{equation}
By substituting the above equation in Eq.~\eqref{integral_eta_g_e}, we obtain
\begin{small}
\begin{equation}
\begin{split}
& \quad  \int \limits^{\infty}_{-\infty} \; \eta^*_{st}(x) \left[ \;  \int \limits^{\infty}_{-\infty} \; e^{-(x-x')^2\xi^2} \cdot {g^w_{mn}}(x') \; dx'\right]dx \\
&=\sqrt{2}\sum^{N_{u}}_{u} \sum^{N_{v}}_{v} a^*_{u v} \cdot e^{j2\pi\alpha \beta s v} \cdot  \\
&\qquad \int \limits^{\infty}_{-\infty} \int \limits^{\infty}_{-\infty} \; e^{-(x-x')^2\xi^2 - \frac{\pi}{X^2} (x' - \alpha m X)^2 - \frac{\pi}{X^2}(x - \alpha s X - \alpha u X)^2 + j\beta K n x' -j\beta K (v + t) x } \; dx'  dx\\
&\xlongequal[q = m-s-u]{p = v + n + t}\sum^{N_{u}}_{u} \sum^{N_{v}}_{v}2\sqrt{\pi}X^2 a^*_{u v} \cdot e^{j2\pi\alpha \beta [s v + mn - (s+u)(t+v)] - \frac{\pi}{2}\beta^2(v + t -n)^2} \cdot f(q,p,\xi),
\end{split}
\end{equation}
\end{small}
where
\begin{small}
\begin{equation}
 f (q,p,\xi) = \frac{1}{\sqrt{4X^2\xi^2 + 2\pi}} \cdot \exp \left( - \frac{\pi}{2 + \frac{\pi}{X^2\xi^2}} \cdot (\alpha q + j\beta p)^2 - \frac{\pi}{2} \beta^2p^2  \right).
\end{equation}
 \end{small}
The term $- \frac{\pi}{2}\beta^2 p^2 $ in the argument of the exponential function ensures that the real part of the  exponent 
\begin{small}
 $$ - \frac{\pi}{2 + \frac{\pi}{X^2\xi^2}} \cdot (\alpha q + j\beta p)^2 - \frac{\pi}{2}\beta^2 p^2 $$
 \end{small}
 is always negative. Thus, for any fixed $q$ and $p$, $f(q,p,\xi)$ %%is not divergent.
remains bounded.
 
 
 
 \subsubsection{The integral related to $z'$}
 \label{integral_z_spatial}
 After mathematical deduction we obtain the analytical results of the integral related to $z'$. Let
 \begin{small}
 \begin{equation}
 \begin{split}
 g({k-l},\xi) &= ({k-l}+1)\cdot \frac{\sqrt{\pi}}{2\xi} \left[ \text{erf}(\xi ({k-l}+1)\Delta) - \text{erf}(\xi {k-l} \Delta) \right]  +  \\
& \quad  \frac{1}{2\Delta \xi^2} \left[ e^{-({k-l}+1)^2\Delta^2 \xi^2} - e^{-({k-l})^2\Delta^2 \xi^2} \right], 
\end{split}
 \end{equation}
 \end{small}
 where $\text{erf} (z)$ is the error function defined as \cite{M1972Handbook}
 \begin{equation}
 \text{erf}(\xi) = \frac{2}{\sqrt{\pi}} \int \limits^\xi_{-\infty} e^{-t^2} dt.
 \end{equation}
 Then we denote
 %\begin{small}
 %\begin{equation}
 %h(w,\xi) \xlongequal[w = k-l]{\triangle}\int \limits^{z_{max}}_{z_{min}} \Lambda_k(z') \cdot e^{-(z_l-z')^2\xi^2}dz' =
% \left\{ \begin{array}{ll}
% g(w,\xi) \quad k = 0, \\ 
% g(w,\xi) + g(-w,\xi) \quad 0 < k < N_k, \\ 
% g(-w,\xi) \quad k = N_k .
% \end{array}\right.
% \end{equation}
% \end{small}
 \begin{small}
 \begin{equation}
 h(k-l,\xi)  \triangleq \int \limits^{z_{max}}_{z_{min}} \Lambda_k(z') \cdot e^{-(z_l-z')^2\xi^2}dz' =
 \left\{ \begin{array}{ll}
 g(k-l,\xi) \quad k = 0, \\ 
 g(k-l,\xi) + g(l-k,\xi) \quad 0 < k < N_k, \\ 
 g(l-k,\xi) \quad k = N_k .
 \end{array}\right.
 \end{equation}
 \end{small}


 \subsubsection{The entire integral of the spatial part}
We denote
 \begin{small}
 \begin{equation}
 \begin{split}
 &\quad f^{spatial}_{st,l}(m,n,k) \\
 &=  \frac{1}{2\pi} \int \limits^{\infty}_{\mathcal{E}}  \frac{e^{\frac{k^2_0}{4\xi^2}}}{\xi} \cdot \Bigg{\{} \; \int \limits^{\infty}_{-\infty}  \; \eta^*_{st}(x') \left[ \; \int \limits^{\infty}_{-\infty} \; e^{-(x-x')^2\xi^2} \cdot g_{mn}(x) dx \right] dx' \Bigg{\}} \cdot  \\
 &\qquad  \quad \left[ \; \int \limits^{z_{max}}_{z_{min}} \; \Lambda_k(z') \cdot e^{-(z_l-z')^2\xi^2}  dz'\right] d\xi \\ 
&= \frac{X^2}{\sqrt{\pi}} \sum_{u,v} a^*_{u v} \cdot e^{j2\pi\alpha\beta[sv +mn - (s+u)(t+v)] - \frac{\pi}{2} \beta^2 (v + n -n)^2}\cdot  \\
 &\qquad \quad \int \limits^{\infty}_{\mathcal{E}} \frac{e^{\frac{k^2_0}{4\xi^2}}}{\xi} \cdot f(m-s-u, n + t + v,\xi) \cdot h(k-l,\xi)\; d\xi . \\
 \end{split}
 \label{f_spatial_mnl_stk}
 \end{equation}
 \end{small}
  The definitions of $f(q,p,\xi)$ and $h(k-l,\xi)$ can be found in Section \ref{integral_x_spatial} and \ref{integral_z_spatial}, respectively. Consider the following integral :
  \begin{small}
  \begin{equation}
 \begin{split}
 & \quad \int \limits^{\infty}_{\mathcal{E}} d\xi  \cdot \frac{e^{\frac{k^2_0}{4\xi^2}}}{\xi} \cdot f(q, p,\xi) \cdot h(k-l ,\xi)\; d\xi  \\
 &= h_1 \int \limits^{\infty}_{\mathcal{E}}\frac{e^{\frac{k^2_0}{4\xi^2}}}{\xi} \cdot f(q, p,\xi) \cdot  g(k-l,\xi) \;  d\xi  +
 h_2 \int \limits^{\infty}_{\mathcal{E}} \frac{e^{\frac{k^2_0}{4\xi^2}}}{\xi} \cdot f(q, p,\xi) \cdot  g(l-k ,\xi) \; d\xi,\\
 \end{split}
 \end{equation}
 \end{small}
 where $h_1, h_2 \in \{0,1\}$. Thus, we can study the following alternative \mbox{integral:}
 \begin{small}
 \begin{equation}
   \int \limits^{\infty}_{\mathcal{E}} \;\frac{e^{\frac{k^2_0}{4\xi^2}}}{\xi} \cdot f(q, p,\xi) \cdot  g(k-l ,\xi) \; d\xi .
   \label{integral_spatial}
 \end{equation}
 \end{small}
 If we want to calculate $f^{spatial}_{st,l}(m,n,k)$ for every $m,s \in \{-N_m,\ldots ,N_m\}$, $n,t \in \{-N_n,\ldots,N_n\}$ and  $k,l \in \{0,\ldots, N_k\}$, we need to calculate Eq.~\eqref{integral_spatial}
 for every $q$, $p$ and $k-l$. 

%%------------------------------------------------------------------------------------------------
\subsection{Spectral part of electric field}
 Based on the convolution property, the Fourier transform of the spectral part, i.e. Eq.~\eqref{chi_E_spectral} can be simplified as
 \begin{small}
 \begin{equation}
 \begin{split}
 \widehat{\chi E}_{spectral}(k_{x},z_l) &=  \sum_{m,n,k}J_{mn,k} \mathcal{F} \left(  \; \int \limits^{z_{max}}_{z_{min}}\int \limits^{\infty}_{-\infty} G_{spectral}(x,z_l|x',z')\cdot {g^w_{mn}}(x') \cdot \Lambda_k (z')  dx' dz'  \right) \\
 &=\frac{1}{2\sqrt{\pi}} \sum_{m,n,k}J_{mn,k}\cdot {\widehat{g^w}_{nm}}(k_{x}) \cdot e^{2\pi j \alpha \beta mn} \cdot \\
 &\qquad \quad \int \limits^{\infty}_{1/\mathcal{E}}\;  e^{(k_0^2 - k_{x}^2)\zeta^2/4} \cdot  \left[ \; \int \limits^{z_{max}}_{z_{min}} \; \Lambda_k(z') \cdot e^{-(z_l - z')^2/\zeta^2} dz'\right] d\zeta,
 \end{split}
 \end{equation}
 \end{small}
 where $\zeta = q(w)$ and the definition of $q(w)$ is shown in \eqref{integral_path_q}.
Let
 \begin{small}
 \begin{equation}
% \widehat{\chi E}_{spectral}(k_{x},z') = \sum^{N_k}_{l=0} \sum_{t,s} \widetilde{E}_{ts,l}\widehat{g}_{ts}(k_{x})\Lambda_l(z'),
  \widehat{\chi E}_{spectral}(k_{x},z') = \sum^{N_k}_{k=0} \sum_{n,m} \widetilde{E}_{nm,k}{\widehat{g^w}_{nm}}(k_{x})\Lambda_k(z'),
 \end{equation}
 \end{small}
 then we have
 \begin{small}
 \begin{equation}
 \begin{split}
 \widetilde{E}_{ts,l} &= \int^{\infty}_{-\infty} \widehat{\chi E}_{spectral}(k_{x},z_l) \cdot \widehat{\eta}^*_{ts}(k_{x}) dk_{x}\\
 &= \frac{1}{2\sqrt{\pi}}\sum^{N_k}_{k=0}\sum_{m,n}J_{mn,k}\cdot e^{2\pi j \alpha \beta mn} 
 \int \limits^{\infty}_{1/\mathcal{E}} \; e^{k_0^2 \zeta^2/4}\cdot \left[ \; \int \limits^{z_{max}}_{z_{min}}\; \Lambda_k(z') \cdot e^{-(z_l- z')^2/\zeta^2}dz' \right]
  \cdot  \\
&\qquad \quad \left[\;  \int \limits^{\infty}_{-\infty} e^{-k^2_{x}\zeta^2/4}\cdot {\widehat{g^w}_{nm}}(k_{x})\cdot \widehat{\eta}^*_{ts}(k_{x})dk_{x}\right]d\zeta 
\end{split}
\label{E_spectral_nml}
\end{equation}
\end{small}
where 
\begin{small}
\begin{equation}
\begin{array}{ll}
{\widehat{g^w}} (k_{x}) = 2^{\frac{1}{4}} X e^{- \frac{\pi}{K^2}k^2_{x}}, \qquad  
{\widehat{g^w}_{nm}}(k_{x}) ={ \widehat{g^w}}(k_{x} - n\beta K)e^{-jm\alpha Xk_{x}}, \\
\widehat{\eta}(k_{x}) = \sum \limits^{N_{u}}_{u}  \sum \limits^{N_{v}}_{v} \widehat{a}_{u v} {\widehat{g^w}_{u v}}(k_{x}),
 \qquad
\widehat{\eta}_{ts}(k_{x}) = \widehat{\eta}(k_{x} - t\beta K)e^{-js\alpha X k_{x}}.
\end{array}
\end{equation}
\end{small}
%%--------------------------------------------------------------------------------
\subsubsection{The integral related to $k_{x}$}
\label{integral_kx_spectral}
 The integral related to $k_{x}$ in Eq.~\eqref{E_spectral_nml} can be solved analytically as follows, 
 \begin{small}
 \begin{equation}
 \begin{split}
 &\quad   \int \limits^{\infty}_{-\infty} dk_{x} \; e^{-k^2_{x}\zeta^2/4} \cdot  {\widehat{g^w}_{nm}}(k_{x}) \cdot \widehat{\eta}^*_{ts}(k_{x})  \\
 &\triangleq 2^{\frac{3}{2}}X^2K \sum^{N_{u}}_{u} \sum^{N_{v}}_{v} \widehat{a}^*_{uv} \cdot e^{-j2\pi \alpha \beta t v} \cdot e^{-\frac{\pi}{2}\beta^2 (n-t -u)^2} \cdot \widetilde{f}(q,p,\zeta),
 \end{split}
 \end{equation}
 \end{small}
 where
 \begin{small}
 \begin{equation}
 \left\{
 \begin{array}{ll}
 q = s + v - m ,\\
 p = t + u + n , \\
 \widetilde{f}(q, p, \zeta) = \sqrt{ \frac{\pi}{K^2\zeta^2 + 8\pi}}\cdot  \exp \left( \frac{4\pi^2}{K^2\zeta^2+8\pi}(\beta p + j\alpha q)^2 - \frac{\pi}{2}\beta^2p^2 \right) .
 \end{array}\right.
 \end{equation}
 \end{small}
 

%%------------------------------------------------------------------------------------------
\subsubsection{The integral related to $z'$}
\label{integral_z_spectral}
 Similar to the spatial part, we have
 \begin{small}
 \begin{equation}
 \widetilde{h}(k-l,\zeta) \triangleq \int \limits^{z_{max}}_{z_{min}}\; \Lambda_k(z') \cdot e^{-( z_l - z')^2/\zeta^2} \; dz'
 = \left\{ \begin{array}{ll} 
\widetilde{g}(k-l,\zeta) \quad k = 0,\\
\widetilde{g}(k-l,\zeta) + \widetilde{g}(l-k,\zeta) \quad 0 < k< N_k ,\\
\widetilde{g}(l-k,\zeta) \quad k = N_k ,
\end{array}\right.
\end{equation}
\end{small}
and 
\begin{small}
\begin{equation}
\begin{split}
\widetilde{g}(k-l,\zeta) &= \frac{\sqrt{\pi}(k-l+1) }{2} \zeta \cdot \left[ \text{erf}\left( \frac{(k-l+1)\Delta}{\zeta} \right) - \text{erf}\left( \frac{(k-l)\Delta}{\zeta} \right) \right] + \\&\qquad \quad \frac{\zeta^2}{2\Delta}\cdot  \left[ \exp \left( -\frac{(k-l+1)^2\Delta^2}{\zeta^2} \right) - \exp \left( -\frac{(k-l)^2\Delta^2}{\zeta^2} \right) \right] .
\end{split}
\label{widetilde_g}
\end{equation}
\end{small}



\subsubsection{The entire integral of the spectral part}
 We denote
 \begin{small}
 \begin{equation}
 \begin{split}
 \widetilde{E}_{ts,l}(m,n,k) 
 &= \frac{1}{2\sqrt{\pi}}\cdot e^{2\pi j \alpha \beta mn} 
\int \limits^{\infty}_{1/\mathcal{E}} \; e^{k_0^2 \zeta^2/4} \left[ \; \int \limits^{z_{max}}_{z_{min}}\; \Lambda_k(z) \cdot e^{-(z_l- z')^2/\zeta^2} dz'\right]\ \cdot  \\
 &\qquad  \qquad \left[ \int \limits^{\infty}_{-\infty} \; e^{-k^2_{x}\zeta^2/4} \cdot  {\widehat{g^w}_{nm}}(k_{x}) \cdot \widehat{\eta}^*_{ts}(k_{x})dk_{x} \right] d\zeta \\
 &= 2X\sqrt{2\pi}\sum_{uv}\widehat{a}^*_{uv}\cdot e^{j2\pi\alpha\beta(mn - tv) -\frac{\pi}{2}\beta^2(n-t-u)^2}\cdot \\
 &\quad \int \limits^{\infty}_{1/\mathcal{E}} e^{k_0^2\zeta^2/4} \cdot \widetilde{f}(s+v-m,t+u+n,\zeta) \cdot \widetilde{h}(k-l,\zeta) d\zeta .
 \end{split}
 \label{E_spectral_nml_stk}
 \end{equation}
 \end{small}
 The definitions of $\widetilde{f}(q,p,\zeta)$ and $\widetilde{h}(k-l, \zeta)$ can be found in Section \ref{integral_kx_spectral} and Section \ref{integral_z_spectral}, respectively.

 Considering that $\widetilde{h}(k-l,\zeta)$ can be directly expressed in $\widetilde{g}(k-l,\zeta)$, we can study the following alternative integral :
 \begin{small}
 \begin{equation}
\int \limits^{\infty}_{1/\mathcal{E}} \; e^{k_0^2\zeta^2/4} \cdot \widetilde{f}(q,p,\zeta) \cdot \widetilde{g}(k-l,\zeta) \; d\zeta.
 \label{integral_spectral}
 \end{equation}
 \end{small}
%%------------------------------------------------------------------------------------------------
%%------------------------------------------------------------------------------------------------
\section{Splitting parameter $\mathcal{E}$}
\label{optimum_splitting_parameter}
\subsection{Asymptotic {c}onvergence of the integrands}
Here we analyze the convergence properties of the integrals related to $G_{spatial}$ and $G_{spectral}$ given by Eq.~\eqref{integral_spatial} and Eq.~\eqref{integral_spectral}, respectively.

\subsubsection{Spatial part}
 For large $\xi$, it is easy to obtain the asymptotic convergences of $\frac{e^{\frac{k^2_0}{4\xi^2}}}{\xi}$ {and} $f(q,p,\xi)${,} respectively,
\begin{small}
 \begin{equation}
\frac{e^{\frac{k^2_0}{4\xi^2}}}{\xi} \sim \frac{1}{\xi},
 \end{equation}
\begin{equation}
 \big| f(q,p,\xi) \big| \sim  \frac{1}{\xi} \cdot e^{ - \frac{\pi}{2}\alpha^2 p^2}.
 \end{equation}
\end{small}
From \cite{Capolino2005Efficient}, we obtain the asymptotic expansion for the error function for large argument, $\text{erf}(\xi) \sim 1 - \frac{e^{-\xi^2}}{\sqrt{\pi}\xi}$. Therefore the asymptotic behavior of $g(k-l,\xi)$ for large $\xi$ {is}:
 \begin{small}
 \begin{equation}
 g(k-l,\xi) \sim \left\{ \begin{array}{ll}
 \frac{1}{2\Delta \xi^2} \qquad k = l-1, \\
 \frac{\sqrt{\pi}}{2\xi} \qquad   k = l, \\
 \frac{1}{2(k-l) \Delta \xi^2}e^{-(k-l)^2\Delta^2 \xi^2} \qquad \text{other}.
 \end{array}\right.
 \end{equation}
 \end{small} 
 As a consequence, we have
 \begin{small}
 \begin{equation}
 \frac{e^{\frac{k_0^2}{4\xi^2}}}{\xi} \cdot f(q,p,\xi) \cdot g(k-l,\xi) \sim 
 \left\{\begin{array}{ll}
  \frac{1}{4X\Delta \xi^4} \cdot \exp \left( -\frac{\pi \alpha^2 q^2}{2} \right) \quad  k=l -1 ,\\
 \frac{\sqrt{\pi}}{4X\xi^3} \cdot \exp \left( -\frac{\pi \alpha^2 q^2}{2} \right) \quad  k= l ,\\
 \frac{1}{4(k-l)X\Delta \xi^4} \cdot \exp \left( - (k-l)^2\Delta^2 \xi^2 - \frac{\pi \alpha^2 q^2}{2} \right) \quad \text{other}.
 \end{array}\right.
 \label{limit_spatial}
 \end{equation}
 \end{small}
 when $\xi$ approaches infinity.
 If $\frac{\pi\alpha^2q^2}{2}$ is large enough, $\frac{e^{\frac{k_0^2}{4\xi^2}}}{\xi} \cdot f(q,p,\xi) \cdot g(k-l,\xi)$ is close to zero.
 

 \subsubsection{Spectral part}
 When $w$ approaches infinity $(> \frac{2}{\mathcal{E}})$, we have
 \begin{small}
 \begin{equation}
  \big| e^{-k^2_x\zeta(w)^2/4} \big|  = \big | \exp \left( -j k^2_x w^2/8 \right) \big| \sim 1
 \end{equation}
 \begin{equation}
 \big| \widetilde{f}(q, p, \zeta(w))  \big| \sim 
 \sqrt{2\pi}\cdot \frac{1}{Kw} \cdot \exp\left( -\frac{\pi}{2}\beta^2 p^2 \right).
 \end{equation}
\end{small}
By expanding the integrand $e^{-\zeta^2}$ into its Maclaurin series and integrating term by term, we obtain the error function's Maclaurin series as
\begin{small}
\begin{equation}
\text{erf}(\zeta) = \frac{2}{\sqrt{\pi}} \sum^{\infty}_{n=0} \frac{(-1)^n \zeta^{2n+1}}{n!(2n+1)}.
\label{err_Maclaurin}
\end{equation}
\end{small}
By substituting Eq.~\eqref{err_Maclaurin} {in} Eq.~\eqref{widetilde_g} and letting $w\rightarrow \infty$ (thus $\zeta$ {goes  to $(1-j)\infty$}), we can obtain the asymptotic behavior of $\widetilde{g}$ as follows,
\begin{small}
\begin{equation}
\widetilde{g}(k-l,\zeta) \sim \frac{\Delta}{2}.
\end{equation}
\end{small}

 As a consequence, when $w \rightarrow \infty$,
 \begin{small}
 \begin{equation}
 \big|  e^{k_0^2\zeta^2(w)/4} \cdot \widetilde{f}(q,p,\zeta(w)) \cdot \widetilde{g}(k-l,\zeta(w)) \cdot \zeta'(w) \big|
 \sim
 \frac{\sqrt{2\pi}\Delta (1 -j)}{4Kw} \cdot \exp \left( -\frac{\pi}{2}\beta^2 p^2\right) .
 \label{limit_spectral}
 \end{equation}
 \end{small}

\subsection{Discussion on the choice of $\mathcal{E}$}
 In the $spatial$ part, for large $\xi$,  the asymptotic behavior of the integrand exhibits Gaussian convergence $e^{-(k-l)^2\Delta^2\xi^2}$ if $k \neq l-1, l$. In the $spectral$ part, the large argument ${w}> \frac{2}{\mathcal{E}}$ leads to {a} rapidly oscillatory behavior in the integrand, $e^{k_0^2 \zeta^2(w)/4} = \exp \left( - j\frac{k^2_0 w^2}{8}\right)$. If $k^2_0 w^2$ is large enough, the infinite integration interval can be truncated. To minimize the calculation {while} holding high accuracy, we would like to maximize the lower bounds of $(k-l)^2\Delta^2\xi^2$ and $\frac{k^2_0 w^2}{8}$. Because
 \begin{small}
 \begin{equation}
 (k-l)^2\Delta^2\xi^2 \geq \Delta^2 \mathcal{E}^2  \qquad  \text{for} \; k \neq l-1, l\; \text{and}\;\xi \geq \mathcal{E}  ,
 \end{equation}
 \begin{equation}
 \frac{k^2_0 w^2}{8} \geq \frac{k^2_0}{2\mathcal{E}^2} \qquad \text{for} \; w \geq \frac{2}{\mathcal{E}}.
 \end{equation}
 \end{small}
 Thus, we let
 \begin{small}
 \begin{equation}
 \Delta^2 \mathcal{E}^2 = \frac{k^2_0}{2\mathcal{E}^2},
 \end{equation}
 \end{small}
 to get the optimum splitting parameter 
 \begin{small}
 \begin{equation}
 \mathcal{E} = 2^{-\frac{1}{4}}\sqrt{\frac{k_0}{\Delta}}.
 \end{equation}
 \end{small}
%%------------------------------------------------------------------------------------------------
%%------------------------------------------------------------------------------------------------
\section{Numerical examples}
\label{numerical_examples}
\subsection{Validation and accuracy}
We employ the same scatters described in \cite{Dilz2016The} to test the accuracy of our algorithm. The parameters of the three scatters are listed in Table \ref{scatters_parameters}.

$E^i(x,z) = E_0 \exp (jk_0 (x \cos \theta + z \sin \theta))$ with $E_0 = 1V/m$. {After several numerical experiments, we construct the approximation of the dual window by choosing $u=2$ and $v=3$.}
% In the $x$-direction, the discretization through Gabor frame is set as $X = 0.5$, $M = 6$, $N = 3$ and $\alpha =\beta = \sqrt{2/3}$. It means that there are $13$ ($|m| \leq M, m \in \mathbb{Z}$) spatial windows with $0.5m$ width and $7$ ($|n| \leq N, n \in \mathbb{Z}$) spectral windows. The oversampling rates are set the same both in spatial and spectral domain .  In the $z$-direction, the step size $\Delta$ is determined as $\Delta = 0.05m$. For the approximation of the dual window, we choose $u = 2$ and $v = 3$ after several attempts.

Furthermore, the parameters of the discretization in the $x$ and $z$ direction are shown in Table \ref{discretization_parameters}. As a consequence, there are $13$ ($|m| \leq M, m \in \mathbb{Z}$) spatial windows with $0.5m$ width and $7$ ($|n| \leq N, n \in \mathbb{Z}$) spectral windows.
\setcounter{table}{0}

\begin{small}
\begin{table}[h]
\centering
\begin{tabular}{|c|c|c|c|}
\hline
Parameters & Circle & Rectangle & Grating  \\\hline 
$\varepsilon_r$ of object & 2 & 2 & 2\\\hline
size of object (m) & $r = 1.35$ & $2.0 \times 5.0$ & 5 blocks, $1\times 1.4$, spacing $2$ \\\hline
wavenumber $k_0$ (m$^{-1}$) & 1.45 & 0.8388 & 1.5 \\\hline
angle of incidence $\theta$ & $0^{\circ}$ & $90^{\circ}$ & $45^{\circ}$ \\\hline
\end{tabular}
\caption{The parameters of the three {scattering} examples.}
\label{scatters_parameters}
\end{table}
\end{small}

\begin{small}
\begin{table}[h]
\centering
\begin{tabular}{|c|c|c|c|c|c|c|}
\hline
 Parameters of the discretization & $X$ & $M$ & $N$ & $\alpha (=\beta)$ & $\Delta $ & $N_k$ \\\hline
 $x$-direction &  $0.5$ & $6$ & $3$ & $\sqrt{2/3}$ &   \diagbox[dir=SW]{}{\textcolor{white}{.}}  &   \diagbox[dir=SW]{}{\textcolor{white}{.}} 
  \\\hline
  $z$-direction  &  \diagbox[dir=SW]{\textcolor{white}{.}}{} & \diagbox[dir=SW]{}{\textcolor{white}{.}}  &   \diagbox[dir=SW]{}{\textcolor{white}{.}}  &   \diagbox[dir=SW]{}{\textcolor{white}{.}}  & 0.05 & 56
\\\hline
\end{tabular}
\caption{The parameters of the discretization in the $x$ and $z$ direction.}
\label{discretization_parameters}
\end{table}
\end{small}


\begin{figure}[htb]
\centering 
\subfigure[]{ 
\label{example1_realpart} 
\includegraphics[height=4.5cm]{picture/pdflatex_fig/example1_realpart}} 
\subfigure[]{ 
\label{example2_realpart}
\includegraphics[height=4.5cm]{picture/pdflatex_fig/example2_realpart}} 
\caption{The real parts of $\chi E^s (x,z)$ for the circular and rectangular scatterers.} 
\label{realpart}
\end{figure}

 \begin{figure}[htb]
\centering \subfigure[]{ 
\label{example1_error} 
\includegraphics[height=4cm]{picture/pdflatex_fig/example1_error}} 
\subfigure[]{ 
\label{example2_error}
 \includegraphics[height=4cm]{picture/pdflatex_fig/example2_error}} 
\caption{The absolute value of the error between $\chi E^s$ from our simulation results and validation $E^s$ from $JCMWave$ for the circle and rectangle, respectively.}
\label{error}
 \end{figure}
 
\begin{figure}[htb]
 \centering \subfigure[]{ 
 \label{example1_real_z_neg02} 
  \includegraphics[height=4cm]{picture/pdflatex_fig/example1_real_z_neg02}} 
 \subfigure[]{ 
 \label{example1_imag_z_neg02}
 \includegraphics[height=4cm]{picture/pdflatex_fig/example1_imag_z_neg02}}  
 \caption{The real (shown in (a)) and imaginary (shown in (b)) part of $E^s$ from validation (black solid) and $\chi E^s$ from the algorithm in \cite{Dilz2016The} (blue dashed) and our simulation (red dashdot) at $z = -0.2$ respectively for the circle.}
 \label{example1_z_neg02}
 \end{figure}
%%-------------------------------------------------------------------------------------------
 \begin{figure}[htb]
 \centering \subfigure[]{ 
 \label{example2_real_z_0} 
  \includegraphics[height=4cm]{picture/pdflatex_fig/example2_real_z_0}} 
 \subfigure[]{ 
 \label{example2_imag_z_0}
  \includegraphics[height=4cm]{picture/pdflatex_fig/example2_imag_z_0}} 
 \caption{The real (shown in (a)) and imaginary (shown in (b)) part of $E^s$ from validation (black solid) and $\chi E^s$ from the algorithm in \cite{Dilz2016The} (blue dashed) and our simulation (red dashdot) at $z = 0$ respectively for the rectangle.}
 \label{example2_z_0}
 \end{figure}

 The real part of the scattered field calculated by our method is shown in Fig.~\ref{realpart}. 
 To validate our results, we utilize the numerical solutions calculated by the JCMWave software package \cite{Burger2013Finite} as the benchmarks. Fig.~\ref{error} presents the difference between the scattered electric field $E^s$ from JCMWave and $\chi E^s$ from our simulations. The large error around the discontinuity is due to the Gibbs phenomenon. Additionally, outside the object, the scattered electric field $E^s$ and the contrast source $\chi E^s$ have a large deviation since the contrast source is zero outside the support of the objects, whereas the scattered electric field is nonzero, which explains the large error outside the objects. It can also be observed in Fig.~\ref{example1_z_neg02} and Fig.~\ref{example2_z_0}. To show the details in the comparison, we plot the real and imaginary part of $E^s$ from JCMWave and $\chi E^s$ from the algorithm described in \cite{Dilz2016The} and our simulation at $z = -0.2$ for the circle in Fig.~\ref{example1_z_neg02} and {at} $z = 0$ for the rectangle in Fig.~\ref{example2_z_0}. 
 These error figures illustrate that our simulations {are} reliable and very {close} to the results obtained by the algorithm in \cite{Dilz2016The}. 


 We have also tested our code on the finite grating structure shown in Table \ref{scatters_parameters}. The real part of the scattered field and the error are shown in Fig.~\ref{example3_realpart} and Fig.~\ref{example3_error}. The detailed comparisons of the real and imaginary part of the scattered field at $z=0$ can be seen in Fig.~\ref{example3_real_z_0} and Fig.~\ref{example3_imag_z_0}. 
 \begin{figure}[htb]
 \centering
  \includegraphics[height=2.5cm]{picture/pdflatex_fig/example3_realpart}
 \caption{The real parts of $\chi E^s (x,z)$ for the finite grating structure.}
 \label{example3_realpart}
 \end{figure}

 \begin{figure}[!h]
 \centering
  \includegraphics[height=2.5cm]{picture/pdflatex_fig/example3_error}
 \caption{The absolute value of the error between $\chi E^s$ from our simulation results and validation $E^s$ from JCMWave for the finite grating structure.}
 \label{example3_error}
 \end{figure}

 \begin{figure}[!h]
 \centering \subfigure[]{ 
 \label{example3_real_z_0} 
 \includegraphics[height=4cm]{picture/pdflatex_fig/example3_real_z_0}} 
%%\hspace{0.5in} 
 \subfigure[]{ 
 \label{example3_imag_z_0}
 \includegraphics[height=4cm]{picture/pdflatex_fig/example3_imag_z_0}} 
 \caption{The real (shown in (a)) and imaginary (shown in (b)) part of $E^s$ from validation (black solid) and $\chi E^s$ from the algorithm in \cite{Dilz2016The} (blue dashed) and our simulation (red dashdot) at $z = 0$ for the finite grating structure.}
 \label{example3_z_0}
 \end{figure}


%%--------------------------------------------------------------------------------------------------------
\subsection{Computational costs}
 To demonstrate the efficiency of our algorithm, we compare the computation times with the algorithm described in \cite{Dilz2016The}. The two algorithms  are both implemented {in} MATLAB R2017b. The parameters of the discretizations for these scatterers are presented in {Table \ref{discretization_parameters}},
 \begin{small}
 \begin{table} [!h]
 \centering
 \begin{tabular}{|c|c|c|c|c|c|c|}
 \hline
 Parameters of the discretization &$X$ & $M$ &$N$  & $\alpha (=\beta)$ & $\Delta$ & $N_k$ \\\hline
  Circle &0.5& 6 & 3 & $\sqrt{2/3}$ & 0.05 & 56 \\\hline
  Rectangle &0.5& 6 & 3 & $\sqrt{2/3}$ & 0.05 & 56\\\hline
   Grating  & 0.5 & 11 & 7 & $\sqrt{2/3}$ & 0.05 & 33  \\\hline
 \end{tabular}
 \caption{The parameters of the discretization for circle, rectangle, and grating, respectively.}
 \label{discretization_parameters}
 \end{table}
 \end{small}

% \begin{small}
% \begin{table}[!h]
% \centering
% \begin{tabular}{|c|c|c|c|}
% \hline
% Computation times (minute) & Circle & Rectangle & Grating  \\\hline
% Algorithm Ref \cite{Dilz2016The} & 8.89 & 8.95 & 95.98\\\hline
% Our proposed method & 4.45 & 4.50 & 19.43 \\\hline
% \end{tabular}
% \caption{The computation times of the two algorithms calculating the three scatterers.}
% \label{CPUtimes}
% \end{table}
% \end{small}
 
 \begin{small}
 \begin{table}[!htb]
  \centering
    \begin{tabular}{|p{2.7cm}|p{1.1cm}|p{1.1cm}|p{1.8cm}|p{1.1cm}|p{1.1cm}|p{1.8cm}|}
    \hline
    \multirow{2}{*}{ \tabincell{l}{Computation\\time (seconds)}  } &
    \multicolumn{3}{c|}{Algorithm Ref \cite{Dilz2016The}} &
    \multicolumn{3}{c|}{ Our proposed method}  \\
    \cline{2-7}
    & MSetT & MSolT  & total time  &   
     MSetT & MSolT  & total time   \\
    \hline
    Circle &532 &0.411 & 533 &265 &0.443  & 267 \\
    \hline
    Rectangle & 536 & 0.353 & 537 & 269  & 0.340 &270 \\
    \hline
     Grating&5755 & 3.138 & 5759 & 1161 & 3.268 & 1165\\
    \hline
  \end{tabular}
  \caption{The computation times of the two algorithms calculating the three scatterers.}
 \label{CPUtimes}
\end{table}
 \end{small}
\noindent where $MSetT$ {indicates} the time for the final coefficient matrix setup, while $MSolT$ means the time for solving the linear equation. 
 
 Table \ref{CPUtimes} {demonstrates} that 
 the ratio of the {computation times} of our simulation and the algorithm in \cite{Dilz2016The} is about $1/2$. As the parameters $M$ and $N$ increase, this ratio drops to $1/4$. {This is mainly because} the inverse matrix calculation of a $(2M+1)(2N+1)(N_k+1)$ matrix when deriving the relation between the Gabor coefficients of the scattered electric field $E^s$ and that of the incoming electric field $E^i$. As a consequence, our proposed method has a greater advantage when simulating larger structures. 

 Next, we briefly discuss the storage requirements in computation. According to Eq.~\eqref{f_spatial_mnl_stk} and Eq.~\eqref{E_spectral_nml_stk}, we utilize two three-dimensional {arrays} to {store} the integrals in {the} spatial and spectral domain. These integrands are continuous and their convergence rates are shown in Eq.~\eqref{limit_spatial} and Eq.~\eqref{limit_spectral}. By observing the integrands, we see that the integral matrices can be just calculated once if the window width $X$ and the oversampling rates $\alpha$, $\beta$ remain unchanged. Then, we can assemble the coefficient matrices of the spatial and spectral part via simple linear computations of the two integral arrays. The scaling of the number of coefficients in these arrays is $O(MNN_k)$. 


%%------------------------------------------------------------------------------------------------
%%------------------------------------------------------------------------------------------------
\section{Conclusions}
\label{conclusions}
 In this paper, we proposed an efficient method to solve the domain integral equation for simulating 2D transverse electric scattering of a finite size object in a homogeneous medium using the Ewald transformation and Gabor frame discretization. The Ewald Green function transformation separates the integrals related to $x$ and $z$, and the Ewald method splits the integral formula of the Green function in two parts, $G_{spatial}$ and $G_{spectral}$. Therefore, the coefficient matrices can be obtained through integral and linear operations without matrix inversion, {which} is the main reason why our proposed method can save computational time. Next we utilize the Fourier {transformation} to deal with the singularity in $G_{spectral}$.  The Gabor frame ensures fast conversion between {the} spatial domain and spectral domain. In addition, we use the summation of modulated Gaussian functions to approximate the dual Gabor window $\eta (x)$ instead of discrete numerical values. As a consequence, we are faced with integrands composed of modulated Gaussian functions and this kind of integrals has an analytic solution. Consequently, the three-dimensional integrals in Eq.~\eqref{f_spatial_mnl_stk} and Eq.~\eqref{E_spectral_nml_stk} {are reduced to one-dimensional integrals.}

Our work in the future is to extend the method to 2D TE scattering in {a} layered medium and the {three-dimensional} case.
%%------------------------------------------------------------------------------------------------
%%------------------------------------------------------------------------------------------------
%\section*{Acknowledgment}
%This work is supported by National Natural Science Foundation of China (Grant Nos.11771393,11632015).

%% The Appendices part is started with the command \appendix;
%% appendix sections are then done as normal sections
%% \appendix

%% \section{}
%% \label{}

%% References
%%
%% Following citation commands can be used in the body text:
%% Usage of \cite is as follows:
%%   \cite{key}          ==>>  [#]
%%   \cite[chap. 2]{key} ==>>  [#, chap. 2]
%%   \citet{key}         ==>>  Author [#]

%% References with bibTeX database:

\section*{Reference}
\bibliographystyle{model1-num-names}
\bibliography{sample}

%% Authors are advised to submit their bibtex database files. They are
%% requested to list a bibtex style file in the manuscript if they do
%% not want to use model1-num-names.bst.

%% References without bibTeX database:

 % \begin{thebibliography}{00}

%% \bibitem must have the following form:
%%   \bibitem{key}...
%%

% \bibitem{}

% \end{thebibliography}

\nomenclature{$X,K,\alpha,\beta$}{Gabor frame parameters (constants)}
\nomenclature{$x,z,x',z'$}{Coordinate variables}
\nomenclature{$k_0$}{Wavenumber in vacuum}
\nomenclature{$k_x$}{Wavenumber in $x$-direction}
\nomenclature{$R$}{The distance between observation point $(x,z)$ and the source element $(x',z')$}
\nomenclature{$w$}{Continuous real variable}
\nomenclature{$\xi,\zeta$}{Complex variables}
\nomenclature{$\phi,f,g,h$}{functions in the spatial domain}
\nomenclature{$\widetilde{f},\widetilde{g},\widetilde{h}$}{functions in the spectral domain}
\nomenclature{$\begin{array}{cc} m,n,l,s,t,k, \\ p,q,u,v \end{array}$}{Integer variables as the index of Gabor coefficients}
\nomenclature{$N_\theta $}{The upper limit of absolute value of $\theta$, where $\theta$ can be any one in  $\{m,n,l,s,t,k,p,q,u,v\}$}


\end{document}
%%
%% End of file `elsarticle-template-1-num.tex'.
