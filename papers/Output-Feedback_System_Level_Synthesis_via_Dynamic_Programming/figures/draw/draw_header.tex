\documentclass{IEEEtran}

\usepackage{tikz}
\usepackage{pgfplots}
\usepackage{datatool}

\pgfplotsset{compat=newest}
\usepackage{bm}  % bold greek letters

\usepackage[margin=0in]{geometry}
\pagenumbering{gobble}

\newlength{\origwidth}
\newlength{\origheight}
\newlength{\origcolumnwidth}

\setlength{\origwidth}{12.8cm}
\setlength{\origheight}{9.6cm}
\setlength{\origcolumnwidth}{20pc}
%\setlength{\paperwidth}{\origwidth}
%\setlength{\paperheight}{\origheight}

\setlength\parindent{0pt}

\usepgfplotslibrary{statistics}
\usepgfplotslibrary[statistics]
\usetikzlibrary{pgfplots.statistics}
\usetikzlibrary[pgfplots.statistics]
\usetikzlibrary{patterns}

% the command uses macro keys
\def\mackey#1{\edef\TEMP{\noexpand#1}\TEMP}

\newcommand{\percentileBar}[7][1]{
\draw[color_#1, thick] ([yshift=-2pt]axis cs:#3,#2)--([yshift=2pt]axis cs:#3,#2);
\draw[color_#1, thick] (axis cs:#3,#2)--(axis cs:#4,#2);
\draw[color_#1, pattern=pattern_#1, pattern color=color_#1, thick] ([yshift=-2pt]axis cs:#4,#2) rectangle([yshift=2pt]axis cs:#6,#2);
\draw[color_#1, thick] ([yshift=-2pt]axis cs:#5,#2) rectangle([yshift=2pt]axis cs:#5,#2);
\draw[color_#1, thick] (axis cs:#6,#2)--(axis cs:#7,#2);
\draw[color_#1, thick] ([yshift=-2pt]axis cs:#7,#2)--([yshift=2pt]axis cs:#7,#2);
}

\newcommand{\autoboundingby}[1]{\setbox0=\vbox{\hbox{%
\begin{tikzpicture}
#1
\end{tikzpicture}%
}}
\setlength{\paperwidth}{\wd0}
\setlength{\paperheight}{\ht0}}


% firstpage, following pages
\newcommand{\autoboundingmultipage}[2]{\autoboundingby{#1%
\coordinate (AUTOBOUNDING-NE) at (current bounding box.north east);%
\coordinate (AUTOBOUNDING-SW) at (current bounding box.south west);%
}\begin{document}
\box0
#2
\end{document}}

\newcommand{\tikzpage}[1]{\clearpage\newpage%
\begin{tikzpicture}%
\useasboundingbox (AUTOBOUNDING-SW) rectangle (AUTOBOUNDING-NE);
#1
\end{tikzpicture}%
}

\newcommand{\autobounding}[1]{\autoboundingmultipage{#1}{}}

\newcommand{\tikzlegend}[2]{\autobounding{%
\begin{axis}[
    hide axis,
    xmin=0,xmax=1,
    ymin=0,ymax=1,
	legend style={
		fill=none,%
		#1
	}
]
#2
\end{axis}}}

\newcommand{\tikzcolorbar}[2]{\autobounding{%
\begin{axis}[
    hide axis,
    xmin=0,xmax=1,
    ymin=0,ymax=1,
    colormap={map}{
		#1
    },
    colorbar horizontal,
	#2
]
\end{axis}}}

\usetikzlibrary{calc}
\tikzset{
jump/.style={
     to path={
         let \p1=(\tikztostart),\p2=(\tikztotarget),\n1={atan2(\y2-\y1,\x2-\x1)} in
         (\tikztostart) -- ($($(\tikztostart)!#1!(\tikztotarget)$)!0.1cm!(\tikztostart)$)
         arc[start angle=\n1+180,end angle=\n1,radius=0.1cm] -- (\tikztotarget)}
},
jump/.default={0.5},
jumpat/.style={
     to path={
         let \p1=(\tikztostart),\p2=(\tikztotarget),\n1={atan2(\y2-\y1,\x2-\x1)} in
         (\tikztostart) -- ($($#1$)!0.1cm!(\tikztostart)$)
         arc[start angle=\n1+180,end angle=\n1,radius=0.1cm] -- (\tikztotarget)}
},
}

\newcommand{\coordbase}[5]{
% scale (0,0)--(1,1) to the figure size
\begin{scope}[xscale=#1,yscale=#2]
% scale (0,0)--(\xmax,\ymax) to (0,0)--(1,1) 
\begin{scope}[xscale=1/#3,yscale=1/#4]
#5
\end{scope}
\end{scope}
}

\newcommand{\defaultcoordbase}[1]{%
\coordbase{\xspan}{\yspan}{\xmax}{\ymax}{%
#1
}}

\newcommand{\coordpage}[1]{\tikzpage{%
\defaultcoordbase{%
#1
}}}