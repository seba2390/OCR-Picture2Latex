\section{Evaluation}\label{sec:evaluation}
% timing
To evaluate the performance of our algorithm, we compare the time required to run our nominal algorithm and
approximation algorithm with the solution given by CVX. We construct a set of stochastic chain matrices and time
 the computations for the $\mathcal{H}_2$ and quadratic objectives for the CVX algorithm, dynamic programming (\alg{H2} and \alg{LQ}), and their corresponding approximation algorithm (in the form of \alg{approxDP}). 
Additionally, we determine in which scenarios the approximation algorithm produces transfer functions that successfully attenuate disturbances. 
The data was generated on a desktop with an AMD Ryzen 7 3700X processor (16 logical cores) and 32 GB DDR4 memory; the CVX code was taken from SLSpy \cite{SLSpy}.

\subsection{Evaluation Setup}
The test system is a stochastic chain with state dimension $N_x$, input dimension $N_u$, and observation dimension $N_y$, which takes the form
\begin{align*}
    A = \begin{bmatrix}
        1-\alpha & \alpha & & & \\
        \alpha & 1-2\alpha & \alpha & & \\
        & \ddots & \ddots & \ddots & \\
        & & \alpha & 1-2\alpha & \alpha \\
        & & & \alpha & 1-\alpha
    \end{bmatrix}\\
    B = \begin{bmatrix}
        I_{N_u} \\
        0
    \end{bmatrix}\ \ 
    C = \begin{bmatrix}
        I_{N_y} & 0
    \end{bmatrix}
\end{align*}
where $\alpha \in (0,1)$ and $I_{N\cdot}$ is the identity matix of size $N\cdot$. We set $D$ to zero and do not use a constraint set. 
We construct the test cases to span the size ranges $N_x \in [5,20]$ with $N_u = N_y = N_x$ and $T=10$. 
Let $T$ be the length of the finite impulse response (FIR) horizon; we run scenarios with $T \in [10,25]$ with $N_x=N_u=N_y=10$. For both tests, we run 50 simulations per data point.
The computing time is defined as the time required to 
compute \alg{H2} or \alg{LQ}, which includes constructing the reformulated matrices $\tilde{A}$, $\tilde{B}$ and new cost function $h_{\tau}$. Analogously, the
computing time for CVX is the time required to synthesize the stablizing controller under \alg{H2} or \alg{LQ}.
% ,such that it takes the form 
% \begin{align*}
%     \begin{gathered}
%         A = D + D_1 + D_2 + D_{-1} + D_{-2}
%     \end{gathered}
% \end{align*}
% where $D$ is a diagonal matrix whose values are chosen from the uniform integer distribution $u[-2,2]$, and $D_k$ is a matrix with non-zero values on
% the $k^{th}$ diagonal, themselves chosen to be zero or nonzero with equal probability, and if nonzero, then chosen from the normal distribution. The matrix $B$ 
% has at most one nonzero value per row, with the nonzero value being one, and chosen such that each input maps to exactly one state.
% A = \begin{bmatrix}
%     a_{0}    & k_{1,0}  & k_{2,0} & & & & & & \\
%     k_{-1,0} & a_{1}    & k_{1,1} & k_{2,1} & & & & & \\
%     k_{-2,0} & k_{-1,1} & a_{2} & k_{1,2} & k_{2,2} & &\\
%              & k_{-2,1} & k_{-1,2} & a_{3} & k_{1,3} & k_{2,3} & & &\\
%              &          &          &       & \ddots  &         & & & \\
%              &          &          &       &         & k_{-2,N_x-4} & k_{-1,N_x-3} & a_{N_x-1} & k_{1,N_x-2}\\
%              &          &          &       &         &         & k_{-2,N_x-3} & k_{-1,N_x-2} & a_{N_x-1}
% \end{bmatrix}
\subsection{Computation Time and Scalability}
% FIR horizon
We plot the computing time as a function of each problem dimension ($N_x$ and $T$) for the $\mathcal{H}_2$ and quadratic objectives
to compare the algorithms' scalability. \fig{timing_all} illustrates how the DP algorithm reduces the computation time compared with
using CVX; for large values of $N_x$, DP is more than 5 times faster than CVX and for large $T$ it is at least 6 time faster. The greatest improvement is computed 
as over 7 times faster for $N_x$ for the quadratic objective. We also observe that the time increases more slowly with increasing system dimensions
compared with CVX for larger system dimensions, showing the scalability of our methods.


\begin{figure}
    \includegraphics[scale=0.5]{figures/scalability_N.png}
    \includegraphics[scale=0.5]{figures/scalability_T.png}
    \caption{The dymanic programming algorithms outperforms the CVX algorithms for both the $\mathcal{H}_2$ and quadratic objectives.}
    \label{fig:timing_all}
\end{figure}



\subsection{Convergence of Approximation Algorithm}
The approximation algorithm partially removes the constraint requiring that the transfer functions at time $T+1$ are zero, which removes the guarantee 
that disturbances will be killed off in exchange for skipping expensive nullspace computations. We test scenarios with varying values of the allowance $T_a$, which is the
number of nullspace computations skipped, and record whether the transfer functions properly terminate disturbances within the finite impulse response time. 
In \fig{time_saved_skipped_iterations}, we see that avoiding computations of $\Acal_u$ for every time step in the FIR horizon significantly
improves the overall computation time relative to the plain DP algorithm; run time decrease by 42\% to 68\% for the largest allowances, and we see the largest percentage improvements for systems with the largest FIR horizon.
For all $T\in[10,25]$, for all $T - T_a \leq 3$, and for each of the 50 test simulations, we verified that the solution was in fact optimal.



\begin{figure}
    \includegraphics[scale=0.5]{figures/LQ_approx.png}
    \includegraphics[scale=0.5]{figures/H2_approx.png}
    \caption{As the number of steps at which $H_X$ and $H_{\Lambda}$ are not recomputed increases, the time needed to compute the solution decreases.
    The dashed lines show the average computation time for the full SLS DP method.}
    \label{fig:time_saved_skipped_iterations}
\end{figure}

