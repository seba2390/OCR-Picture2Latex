\section{Introduction}\label{sec:introduction}
% what is the question, why it is important


In classical control theory, transfer functions and controllers are constructed to minimize a variety of cost functions, primarily functions of the state and input \cite{doyle1992feedback} \cite{zhou1996robust}. These 
celebrated techniques have been proven useful for state-feedback small-scale systems. However, in modern applications of linear control, especially for large-scale systems, the full state of a system is often unknown or too expensive to acquire within a reasonable amount of time. 
As a result, we rely on partial observations of the state to control the system in an output-feedback fashion. 
Not only does such a proxied information structure make output-feedback control difficult, but it also hinders one in incorporating additional system level constraints.
To address these challenges, system level synthesis (SLS) \cite{anderson2019system, wang2019system} 
derives its system level parameterization of output-feedback controllers as a convex optimization problem that
 maintains system-level constraint enforcement. This formulation can be used to solve the linear-quadratic Gaussian problem and optimization under the $\Hcal_2$ norm as shown in \cite{anderson2019system}. With SLS, we can 
 solve these traditional control problems with added system-level constraints, which is not possible with the Youla or Input-Output parameterizations \cite{furieri2019input} \cite{sabau2014youla} \cite{youla1976modern2}.
 % the output-feedback control problems as convex optimizations,
\iffalse
%is a discrete-time frequency-based control method that allows large-scale problems to be solved in a distributed fashion while using local information. 
%However, SLS problems are difficult to solve 
quickly because finding a controller involves solving a large, slow optimization problem. The output feedback 
case poses a greater challenge than state feedback because it requires optimizing over twice the number of transfer
functions, which increases the size of the optimization problem.
\fi

Although convex optimization problems are in general tractable, it is still slow to solve the SLS problems for output-feedback settings.
Existing work proposes to accelerate the SLS computation by exploiting the structure of underlying systems, and an early work \cite{wang2018separable} has established that an SLS problem can be solved by ADMM in theory if the problem is partially separable -- for both the constraints and the objective. In practice, most work primarily focuses on state-feedback settings \cite{anderson2017structured,matni2017scalable,amo-alonso2020distributed,amo-alonsosubeffective} since the state-feedback SLS constraints are column-wise separable and we only need to ensure the objective is also column-wise separable. Output-feedback SLS is much more complicated as its constraints consist of both column-wise and row-wise separable parts, which require the objective to be both column-wise and row-wise separable for ADMM to work. As a consequence, we end up relying on the solver to deal with general non-separable output-feedback problems.

% challenges!
Some recent work provides an alternative approach to the SLS computation. \cite{tseng2020system} shows that a state-feedback SLS problem can be reformulated as a discrete-time control problem, which admits efficient computation via dynamic programming (DP). This approach could potentially be applied to output-feedback SLS as long as we could address the following discrepancy between the state-feedback and output-feedback SLS. First, the state-feedback SLS system response comprises only two transfer matrices, which are then conveniently set as the state and the control in \cite{tseng2020system}, while the output-feedback version has four matrices to handle. Also, the system dynamics matrices are multiplied at the left in state-feedback constraints, while the output-feedback formulation have both left and right matrix multiplications in its constraints. Further, we have only one initial condition in the state-feedback SLS while the initial condition has a much more complicated structure in the output-feedback setting. 

\iffalse
An alternative method for solving this optimization problem is to reformulate it as a control problem and use dynamic programming \cite{bertsekas2005dynamic1}.
We build on Tseng's 2020 \cite{tseng2020system} paper which develops a dynamic programming algorithm for state feedback by 
considering the two transfer functions as the state and control input. We select one of the four transfer
functions in the output feedback case as the control input, and combine the other three transfer functions
as the state. This results in a state feedback problem, which can be solved using dynamic programming. 

We evaluate our algorithm, along with an approximation algorithm, by comparing the compute time for 
solving the nominal problem with our dynamic programming algorithms. Our algorithms perform up to fifteen times faster than a standard
convex optimization solver, and we use the software developed by the authors of \cite{SLSpy} as the baseline for comparison.
% what is hte problem - output feedabck is improtatn becasue can't see state 
% most of the time
% SLS allows us to solve large distributed problem
% SLS itself is hard to solve quickly
% special things - treat optimization as a control problem, then we can use DP 
% to solve problem

% what's good about ours - ours is faster! (lol)
% cite SLS papers (Nick, etc), shih-hao's state feedback paper, explain that 
% output feedback is harder
% 
\fi

\subsection{Contributions and Organization}
% contribution!
We address the aforementioned challenges and extend the approach in \cite{tseng2020system} to derive a DP algorithm for output-feedback SLS by reformulating it as a discrete-time control problem. By our choice of the corresponding state and control variables, the control problem consists of a new transition constraint and initial condition compared to the state-feedback scenario. By vectorizing the multi-sided matrix multiplications, we can solve the control problem by a two-stage DP procedure. We develop the full DP and an approximation algorithm by skipping the constraint enforcement up to some \emph{allowance}. Explicitly, we derive the algorithm for $\Hcal_2$ and quadratic objectives. By comparing the computing time with the CVX-based solver \cite{SLSpy}, we show that DP perform up to $7$ times faster. Also, the approximation algorithm can further shorten the computing time by $42\%$ to $68\%$ with mild accuracy degradation.

% organization
The paper is organized as follows. We begin with brief introductions of output-feedback SLS and DP in \sec{preliminaries}. 
Then, we present our DP algorithms to solve and approximate output-feedback SLS in \sec{DP_algorithm}. In \sec{examples}, 
the $\Hcal_2$ and quadratic costs are explicity derived, and we numerically evaluate our algorithms in \sec{evaluation}. %results?
We conclude the paper in \sec{conclusion}.

\section{Notation}
\label{sec:notation}

Let bold functions $\mathbf{f}$ represent the row-wise vectorized versions of their scalar counterparts $f$: $\mathbf{f}(\mathbf{x}) = [f(x_1), f(x_2) \dots f(x_n)]^T$. 

Let $I_a(z)$ be the indicator function for $z=a$.

Let $P(Z^*)$ be a distribution that places probability $\frac{1}{n}$ on each element of  a set of samples $\{z_1, z_2 \dots z_n\}$. Let $z^*$ denote a sample from $Z^*$ and $\bm{z}^*$ denote a set of $n$ such samples (a bootstrap sample).

\subsection{Notation for Observational data}

Observational datasets are represented by a set of $n$ tuples of observations $(x_i,w_i,y_i)$. For each subject $i$, $x_i \in \mathcal{R}^{p}$ is a vector of observed pre-treatment covariates, $w_i \in \{0,1\}^n$ is a binary indicator of treatment status, and $y_i \in \mathcal{R}^n$ is a real-valued outcome. Let capital letters $X$, $W$, and $Y$ represent the corresponding random variables. Let $\mathcal{S}_0$ be the set of indices of untreated subjects, and $\mathcal{S}_1$ the set of indices of treated subjects: $\mathcal{S}_w = \{i | w_i = w\}$.  Denote a series of I.I.D. realizations of a random variable $z \sim P(Z)$ as $\bm{z}$ so that the full observational dataset can be written as $d = (\mathbf{x}, \mathbf{w}, \mathbf{y})$ where $(x_i,w_i,y_i) \overset{\text{I.I.D.}}{\sim} P(X,W,Y)$. 

The average treatment effect is defined as 

\begin{equation}
\tau = E_{X,Y}[Y|X,W=1] - E_{X,Y}[Y|X,W=0]
\label{eq:effect}
\end{equation}

which is the expected difference between what the outcome would have been had a subject received the treatment and what the outcome would have been had a subject not received the treatment, averaged over all subjects in the population. %The outcomes under each of these conditions are referred to as potential outcomes. Let those be denoted as $f_w(x,u) = f(x,u,w) = E[Y_w | X, U]$. %Note that $f_w(\mathbf{x}, \mathbf{u}) + \bm{\eta} \in \mathcal{R}^n$, but $\mathbf{y}_w \in \mathcal{R}^{n_w}$. The latter does not include potential outcomes under treatment $w$ that were not observed.