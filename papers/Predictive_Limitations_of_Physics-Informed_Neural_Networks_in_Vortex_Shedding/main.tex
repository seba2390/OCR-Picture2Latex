
\documentclass[5p, twocolumn, times, sort&compress]{elsarticle}
\usepackage{url}  % to fix underscores in URLs in BibTex
\usepackage{amsmath}
\usepackage{amsthm}
\usepackage{amssymb}
\usepackage{amsfonts}
\usepackage{mathtools}
\usepackage[overload]{empheq}
\usepackage[section]{placeins} % limit floats (figs, tables, etc.) to the same sections
\usepackage{tikz}  % visualizing computational graphs
\usepackage{relsize}  % setting font sizes relative to the current font size
\usepackage{makecell}  % create a multi-line cell
\usepackage{stfloats}  % for positioning of figure* on the same page
\usepackage{booktabs}  % table separation lines
\usepackage{threeparttable}  % three part table
\usepackage[group-separator={,}]{siunitx}  % unifying the styles of numbers, units, and quantities
\usepackage[multidot]{grffile}

% temporary; can be deleted
\usepackage{lipsum}  % generate random text as place holders

% helpful macros
\DeclareMathOperator*{\argmin}{arg\,min}  % thin space, limits underneath in displays
\RenewDocumentCommand{\vec}{m}{\mathbf{#1}}  % ISO-style vector notation
\NewDocumentCommand{\mat}{m}{\boldsymbol{\mathit{#1}}}  % ISO-style matrix notation
\NewDocumentCommand{\diff}{}{\mathop{}\!\mathrm{d}}  % ISO-style upright ordinary differentiation
\NewDocumentCommand{\pdiff}{mm}{\frac{\partial #1}{\partial #2}}  % partial differential terms

% tikz setup
\input{tikzit.tikzstyles}

% figure-related
\graphicspath{{figs/}}  % default figure search path
\DeclareGraphicsExtensions{.pdf,.png}  % priority of the formats of figures

% bibliography style
\bibliographystyle{elsarticle-num}

% journal name
\journal{an Elsevier journal}

% main body
\begin{document}

    % front matter
    \begin{frontmatter}
        % title
        \title{%
        Predictive Limitations of Physics-Informed Neural Networks in Vortex Shedding%
        }

        % author list
        \author[1]{Pi-Yueh Chuang}
        \ead{pychuang@gwu.edu}
        \author[1]{Lorena A. Barba\corref{cor1}}
        \ead{labarba@gwu.edu}
        \cortext[cor1]{Corresponding author}
        \affiliation[1]{%
            organization={%
                Department of Mechanical and Aerospace Engineering, %
                The George Washington University%
            }, %
            city={Washington}, %
            state={DC 20052}, %
            country={USA}%
        }

        % abstract
        \begin{abstract}
            The recent surge of interest in physics-informed neural network (PINN) methods has led to a wave of studies that attest to their potential for solving partial differential equations (PDEs) and predicting the dynamics of physical systems. However, the predictive limitations of PINNs have not been thoroughly investigated. We look at the flow around a 2D cylinder and find that data-free PINNs are unable to predict vortex shedding. Data-driven PINN exhibits vortex shedding only while the training data (from a traditional CFD solver) is available, but reverts to the steady state solution when the data flow stops. We conducted dynamic mode decomposition and analyze the Koopman modes in the solutions obtained with PINNs versus a traditional fluid solver (PetIBM). The distribution of the Koopman eigenvalues on the complex plane suggests that PINN is numerically dispersive and diffusive. The PINN method  reverts to the steady solution possibly as a consequence of spectral bias. This case study reaises concerns about the ability of PINNs to predict flows with instabilities, specifically vortex shedding. Our computational study supports the need for more theoretical work to analyze the numerical properties of PINN methods. The results in this paper are transparent and reproducible, with all data and code available in public repositories and persistent archives; links are provided in the paper repository at \url{https://github.com/barbagroup/jcs_paper_pinn}, and a Reproducibility Statement within the paper.
        \end{abstract}

        % keywords
        \begin{keyword}
            computational fluid dynamics \sep
            physics-informed neural networks \sep
            dynamic mode analysis \sep
            Koopman analysis \sep
            vortex shedding
        \end{keyword}
    \end{frontmatter}

    \section{Introduction}
    % \leavevmode
% \\
% \\
% \\
% \\
% \\
\section{Introduction}
\label{introduction}

AutoML is the process by which machine learning models are built automatically for a new dataset. Given a dataset, AutoML systems perform a search over valid data transformations and learners, along with hyper-parameter optimization for each learner~\cite{VolcanoML}. Choosing the transformations and learners over which to search is our focus.
A significant number of systems mine from prior runs of pipelines over a set of datasets to choose transformers and learners that are effective with different types of datasets (e.g. \cite{NEURIPS2018_b59a51a3}, \cite{10.14778/3415478.3415542}, \cite{autosklearn}). Thus, they build a database by actually running different pipelines with a diverse set of datasets to estimate the accuracy of potential pipelines. Hence, they can be used to effectively reduce the search space. A new dataset, based on a set of features (meta-features) is then matched to this database to find the most plausible candidates for both learner selection and hyper-parameter tuning. This process of choosing starting points in the search space is called meta-learning for the cold start problem.  

Other meta-learning approaches include mining existing data science code and their associated datasets to learn from human expertise. The AL~\cite{al} system mined existing Kaggle notebooks using dynamic analysis, i.e., actually running the scripts, and showed that such a system has promise.  However, this meta-learning approach does not scale because it is onerous to execute a large number of pipeline scripts on datasets, preprocessing datasets is never trivial, and older scripts cease to run at all as software evolves. It is not surprising that AL therefore performed dynamic analysis on just nine datasets.

Our system, {\sysname}, provides a scalable meta-learning approach to leverage human expertise, using static analysis to mine pipelines from large repositories of scripts. Static analysis has the advantage of scaling to thousands or millions of scripts \cite{graph4code} easily, but lacks the performance data gathered by dynamic analysis. The {\sysname} meta-learning approach guides the learning process by a scalable dataset similarity search, based on dataset embeddings, to find the most similar datasets and the semantics of ML pipelines applied on them.  Many existing systems, such as Auto-Sklearn \cite{autosklearn} and AL \cite{al}, compute a set of meta-features for each dataset. We developed a deep neural network model to generate embeddings at the granularity of a dataset, e.g., a table or CSV file, to capture similarity at the level of an entire dataset rather than relying on a set of meta-features.
 
Because we use static analysis to capture the semantics of the meta-learning process, we have no mechanism to choose the \textbf{best} pipeline from many seen pipelines, unlike the dynamic execution case where one can rely on runtime to choose the best performing pipeline.  Observing that pipelines are basically workflow graphs, we use graph generator neural models to succinctly capture the statically-observed pipelines for a single dataset. In {\sysname}, we formulate learner selection as a graph generation problem to predict optimized pipelines based on pipelines seen in actual notebooks.

%. This formulation enables {\sysname} for effective pruning of the AutoML search space to predict optimized pipelines based on pipelines seen in actual notebooks.}
%We note that increasingly, state-of-the-art performance in AutoML systems is being generated by more complex pipelines such as Directed Acyclic Graphs (DAGs) \cite{piper} rather than the linear pipelines used in earlier systems.  
 
{\sysname} does learner and transformation selection, and hence is a component of an AutoML systems. To evaluate this component, we integrated it into two existing AutoML systems, FLAML \cite{flaml} and Auto-Sklearn \cite{autosklearn}.  
% We evaluate each system with and without {\sysname}.  
We chose FLAML because it does not yet have any meta-learning component for the cold start problem and instead allows user selection of learners and transformers. The authors of FLAML explicitly pointed to the fact that FLAML might benefit from a meta-learning component and pointed to it as a possibility for future work. For FLAML, if mining historical pipelines provides an advantage, we should improve its performance. We also picked Auto-Sklearn as it does have a learner selection component based on meta-features, as described earlier~\cite{autosklearn2}. For Auto-Sklearn, we should at least match performance if our static mining of pipelines can match their extensive database. For context, we also compared {\sysname} with the recent VolcanoML~\cite{VolcanoML}, which provides an efficient decomposition and execution strategy for the AutoML search space. In contrast, {\sysname} prunes the search space using our meta-learning model to perform hyperparameter optimization only for the most promising candidates. 

The contributions of this paper are the following:
\begin{itemize}
    \item Section ~\ref{sec:mining} defines a scalable meta-learning approach based on representation learning of mined ML pipeline semantics and datasets for over 100 datasets and ~11K Python scripts.  
    \newline
    \item Sections~\ref{sec:kgpipGen} formulates AutoML pipeline generation as a graph generation problem. {\sysname} predicts efficiently an optimized ML pipeline for an unseen dataset based on our meta-learning model.  To the best of our knowledge, {\sysname} is the first approach to formulate  AutoML pipeline generation in such a way.
    \newline
    \item Section~\ref{sec:eval} presents a comprehensive evaluation using a large collection of 121 datasets from major AutoML benchmarks and Kaggle. Our experimental results show that {\sysname} outperforms all existing AutoML systems and achieves state-of-the-art results on the majority of these datasets. {\sysname} significantly improves the performance of both FLAML and Auto-Sklearn in classification and regression tasks. We also outperformed AL in 75 out of 77 datasets and VolcanoML in 75  out of 121 datasets, including 44 datasets used only by VolcanoML~\cite{VolcanoML}.  On average, {\sysname} achieves scores that are statistically better than the means of all other systems. 
\end{itemize}


%This approach does not need to apply cleaning or transformation methods to handle different variances among datasets. Moreover, we do not need to deal with complex analysis, such as dynamic code analysis. Thus, our approach proved to be scalable, as discussed in Sections~\ref{sec:mining}.

    \section{Method}
    \section{SYSTEM OVERVIEW}
\begin{figure}
\centering

\def\picScale{0.08}    % define variable for scaling all pictures evenly
\def\colWidth{0.5\linewidth}

\begin{tikzpicture}
\matrix [row sep=0.25cm, column sep=0cm, style={align=center}] (my matrix) at (0,0) %(2,1)
{
\node[style={anchor=center}] (FREEhand) {\includegraphics[width=0.85\linewidth]{figures/FREEhand.pdf}}; %\fill[blue] (0,0) circle (2pt);
\\
\node[style={anchor=center}] (rigid_v_soft) {\includegraphics[width=0.75\linewidth]{figures/FREE_vs_rigid-v8.pdf}}; %\fill[blue] (0,0) circle (2pt);
\\
};
\node[above] (FREEhand) at ($ (FREEhand.south west)  !0.05! (FREEhand.south east) + (0, 0.1)$) {(a)};
\node[below] (a) at ($ (rigid_v_soft.south west) !0.20! (rigid_v_soft.south east) $) {(b)};
\node[below] (b) at ($ (rigid_v_soft.south west) !0.75! (rigid_v_soft.south east) $) {(c)};
\end{tikzpicture}


% \begin{tikzpicture} %[every node/.style={draw=black}]
% % \draw[help lines] (0,0) grid (4,2);
% \matrix [row sep=0cm, column sep=0cm, style={align=center}] (my matrix) at (0,0) %(2,1)
% {
% \node[style={anchor=center}] {\includegraphics[width=\colWidth]{figures/photos/labFREEs3.jpg}}; %\fill[blue] (0,0) circle (2pt)
% &
% \node[style={anchor=center}] {\includegraphics[width=\colWidth, height=160pt]{figures/stewartRender.png}}; %\fill[blue] (0,0) circle (2pt);
% \\
% };

% %\node[style={anchor=center}] at (0,-5) (FREEstate) {\includegraphics[width=0.7\linewidth]{figures/FREEstate_noLabels2.pdf}};

% \end{tikzpicture}

\caption{\revcomment{2.3}{(a) A fiber-reinforced elastomerc enclosure (FREE) is a soft fluid-driven actuator composed of an elastomer tube with fibers wound around it to impose specific deformations under an increase in volume, such as extension and torsion. (b) A linear actuator and motor combined in \emph{series} has the ability to generate 2 dimensional forces at the end effector (shown in red), but is constrained to motions only in the directions of these forces. (b) Three FREEs combined in \emph{parallel} can generate the same 2 dimensional forces at the end effector (shown in red), without imposing kinematic constraints that prohibit motion in other directions (shown in blue).}}

% \caption{A fiber-reinforced elastomeric enclosure (FREE) (top) is a soft fluid-driven actuator composed of an elastomer tube with fibers wound around it to impose deformation in specific directions upon pressurization, such as extension and torsion. \revcomment{2.3}{In this paper we explore the potential of combining multiple FREEs in parallel to generate fully controllable multi-dimensional spacial forces}, such as in a parallel arrangement around a flexible spine element (bottom-left), or a Stewart Platform arrangement (bottom-right).}

\label{fig:overview}
\end{figure}


We now give an overview of our learning framework as illustrated in Figure~\ref{fig:overview}. Our framework splits athletic jumps into two phases: a run-up phase and a jump phase. The {\em take-off state} marks the transition between these two phases, and consists of a time instant midway through the last support phase before becoming airborne. The take-off state is key to our exploration strategy, as it is a strong determinant of the resulting jump strategy. We characterize the take-off state by a feature vector that captures key aspects of the state, such as the net angular velocity and body orientation. This defines a low-dimensional take-off feature space that we can sample in order to explore and evaluate a variety of motion strategies. While random sampling of take-off state features is straightforward, it is computationally impractical as evaluating one sample involves an expensive DRL learning process that takes hours even on modern machines. Therefore, we introduce a sample-efficient Bayesian Diversity Search (BDS) algorithm as a key part of our Stage~1 optimization process.

Given a specific sampled take-off state, we then need to produce an optimized run-up controller and a jump controller that result in the best possible corresponding jumps. This process has several steps. We first train a {\em }run-up controller, using deep reinforcement learning, that imitates a single generic run-up motion capture clip while also targeting the desired take-off state. For simplicity, the run-up controller and its training are not shown in Figure~\ref{fig:overview}. These are discussed in Section~\ref{sec:Experiments-Runup}. The main challenge lies with the synthesis of the actual jump controller which governs the remainder of the motion, and for which we wish to discover strategies without any recourse to known solutions.

The jump controller begins from the take-off state and needs to control the body during take-off, over the bar, and to prepare for landing. This poses a challenging learning problem because of the demanding nature of the task, the sparse fail/success rewards, and the difficulty of also achieving natural human-like movement. We apply two key insights to make this task learnable using deep reinforcement learning. First, we employ an action space defined by a subspace of natural human poses as modeled with a Pose Variational Autoencoder (P-VAE). Given an action parameterized as a target body pose, individual joint torques are then realized using PD-controllers. We additionally allow for regularized {\em offset} PD-targets that are added to the P-VAE targets to enable strong takeoff forces. Second, we employ a curriculum that progressively increases the task difficulty, i.e., the height of the bar, based on current performance.

A diverse set of strategies can already emerge after the Stage 1 BDS optimization. To achieve further strategy variations, we reuse the take-off states of the existing discovered strategies for another stage of optimization. The diversity is explicitly incentivized during this Stage 2 optimization via a novelty reward, which is focused specifically on features of the body pose at the peak height of the jump. As shown in Figure~\ref{fig:overview}, Stage~2 makes use of the same overall DRL learning procedure as in Stage~1, albeit with a slightly different reward structure.




    \section{Verification and Validation}
    %! TEX root = main.tex

This section presents the verification and validation (V\&V) of our PINN solvers and PetIBM, an in-house CFD solver \cite{chuang_petibm_2018}.
V\&V results are necessary to build confidence in our case study described later in section \ref{sec:case-study}.
For verification, we solved a 2D Taylor-Green vortex (TGV) at Reynolds number $Re=\num{100}$, which has a known analytical solution.
For validation, on the other hand, we use 2D cylinder flow at $Re=40$, which exhibits a well-known steady state solution with plenty of experimental data available in the published literature.

\subsection{Verification: 2D Taylor-Green Vortex (TGV), $Re=\num{100}$}\label{sec:verification}

Two-dimensional Taylor-Green vortices with periodic boundary conditions have closed-form analytical solutions,
and serve as standard verification cases for CFD solvers.
We used the following 2D TGV configuration, wih $Re=\num{100}$, to verify both the PINN solvers and PetIBM:
\begin{equation}\label{eq:tgv}
    \left\{
        \begin{aligned}
            u(x, y, t) &= \cos(x)\sin(y)\exp(-2 \nu t) \\
            v(x, y, t) &= - \sin(x)\cos(y)\exp(-2 \nu t) \\
            p(x, y, t) &= -\frac{\rho}{4}\left(\cos(2x) + \cos(2y)\right)\exp(-4 \nu t)
        \end{aligned}
    \right.
\end{equation}
where $\nu=\num{0.01}$ and $\rho=\num{1}$ are the kinematic viscosity and the density, respectively.
The spatial and temporal domains are $x, y \in \left[-\pi, \pi\right]$ and $t \in [0, 100]$.
Periodic conditions are applied at all boundaries.

In PetIBM, we used the Adams-Bashforth and the Crank-Nicolson schemes for the temporal discretization of convection and diffusion terms, respectively.
The spatial discretization is central difference for all terms.
Theoretically, we expect to see second-order convergence in both time and space for this 2D TGV problem in PetIBM.
We used the following $L_2$ spatial-temporal error to examine the convergence:
\begin{equation}\label{eq:spt-err-def}
    \begin{aligned}
    L_{2,sp-t} \equiv &\sqrt{
        \frac{1}{L_x L_y T}
        \iiint\limits_{x, y, t} \lVert f - f_{ref} \rVert^2 \diff x \diff y \diff t
    } \\
    = &
    \sqrt{\frac{1}{N_x N_y N_t}\sum\limits_{i=1}^{N_x}\sum\limits_{j=1}^{N_y}\sum\limits_{k=1}^{N_t}\left(f^{(i, j, k)} - f_{ref}^{(i, j, k)}\right)^2}
    \end{aligned}
\end{equation}
Here, $N_x$, $N_y$, and $N_t$ represent the number of solution points in $x$, $y$, and $t$;
$L_x$ and $L_y$ are the domain lengths in $x$ and $y$;
$T$ is the total simulation time;
$f$ is the flow quantity of interest, while $f_{ref}$ is the corresponding analytical solution.
The superscript $(i, j, k)$ denotes the value at the $(i, j, k)$ solution point in the discretized spatial-temporal space.
We used Cartesian grids with $2^{n} \times 2^{n}$ cells for $i=4$, $5$, $\dots$, $10$.
The time step size $\Delta t$ does not follow a fixed refinement ratio, and takes the values $\Delta t = \num{1.25e-1}$, $\num{8e-2}$, $\num{4e-2}$, $\num{2e-2}$, $\num{1e-2}$, $\num{5e-3}$, and $\num{1.25e-3}$, respectively.
$\Delta t$ was determined based on the maximum allowed CFL number and whether it was a factor of $2$ to output transient results every $\num{2}$ simulation seconds.
The velocity and pressure linear systems were both solved with BiCGSTAB (bi-conjugate gradient stabilized method).
The preconditioners of the two systems are the block Jacobi preconditioner and the algebraic multigrid preconditioner from NIVIDA's AmgX library.
At each time step, both linear solvers stop when the preconditioned residual reaches \num{e-14}.
The hardware used for PetIBM simulations contains 5 physical cores of Intel E5-2698 v4 and 1 NVIDIA V100 GPU.

Figure \ref{fig:tgv-petibm-convergence} shows the spatial-temporal convergence results of PetIBM.
\begin{figure}
    \centering%
    \includegraphics[width=\columnwidth]{tgv-2d-re100/petibm-tgv-2d-re100-convergence}%
    \caption{%
        Grid-convergence test and verification of PetIBM using 2D TGV at $Re=\num{100}$.
        The spatial-temporal $L_2$ error is defined in equation \eqref{eq:spt-err-def}.
        Taking the cubic root of the total spatial-temporal solution points gives the characteristic cell size.
        Both $u$ and $v$ velocities follow second-order convergence, while the pressure $p$ follows the trend with some fluctuation.
    }
    \label{fig:tgv-petibm-convergence}%
\end{figure}
Both $u$ and $v$ follow an expected second-order convergence before the machine round-off errors start to dominate on the $1024 \times 1024$ grid.
The pressure follows the expected convergence rate with some fluctuations.
Further scrutiny revealed that the AmgX library was not solving the pressure system to the desired tolerance.
The AmgX library has a hard-coded stop mechanism when the relative residual (relative to the initial residual) reaches machine precision.
So while we configured the absolute tolerance to be \num{e-14}, the final preconditioned residuals of the pressure systems did not match this value.
On the other hand, the velocity systems were solved to the desired tolerance.
With this minor caveat, we consider the verification of PetIBM to be successful, as the minor issue in the convergence of pressure is irrelevant to the code implementation in PetIBM.

Next, we solved this same TGV problem using an unsteady PINN solver.
For the optimization, we used PyTorch's Adam optimizer
with the following parameters: $\beta_1=\num{0.9}$, $\beta_2=\num{0.999}$, and $\epsilon=\num{e-8}$.
The total iteration number in the optimization is \num{400000}.
Two learning-rate schedulers were tested: the exponential learning rate and the cyclical learning rate.
Both learning rates are from PyTorch and were used merely to satisfy our curiosity.
The exponential scheduler has only one parameter in PyTorch: $\gamma=0.95^{\frac{1}{5000}}$.
The cyclical scheduler has the following parameters: $\eta_{low}=\num{1.5e-5}$, $\eta_{high}=\num{1.5e-3}$, $N_c=\num{5000}$, and $\gamma=\num{9.99989e-1}$.
These values were chosen so that the peak learning rate at each cycle is slightly higher than the exponential rates.
Figure \ref{fig:tgv-learning-rate-hist} shows a comparison of the two schedulers.

\begin{figure}
    \centering%
    \includegraphics[width=\columnwidth]{tgv-2d-re100/learning-rate-hist}%
    \caption{%
        Learning-rate history of 2D TGV $Re=\num{100}$ w/ PINN
        The exponential learning rate scheduler is denoted as {\it Exponential}, and the cyclical scheduler is denoted as {\it Cyclical}.
    }
    \label{fig:tgv-learning-rate-hist}%
\end{figure}

The MLP network used consisted of \num{3} hidden layers and \num{128} neurons per layer.
\num{8192e4} spatial-temporal points were used to evaluate the PDE losses (i.e., the $L_1$, $L_2$, and $L_3$ in figure \ref{fig:pinn-workflow}).
We randomly sampled these spatial-temporal points from the spatial-temporal domain$\left[-\pi, \pi\right] \times \left[-\pi, \pi\right] \times \left(0, 100\right]$.
During each optimization iteration, however, we only used \num{8192} points to evaluate the PDE losses.
It means the optimizer sees each point \num{40} times on average because we have a total of \num{4e5} iterations.
Similarly, \num{8192e4} spatial-temporal points were sampled from $x,y \in \left[-\pi, \pi\right] ] \times \left[-\pi, \pi\right]$ and $t=0$ for the initial condition loss (i.e., $L_4$ to $L_6$).
And the same number of points were sampled from each domain boundary ($x=\pm\pi$ and $y=\pm\pi$) and $t\in\left(0, 100\right]$ for boundary-condition losses ($L_7$ to $L_{10}$).
A total of \num{8192} points were used in each iteration for these losses as well.

We used one NVIDIA V100 GPU to run the unsteady PINN solver for the TGV problem.
Note that the PINN solver used single-precision floats, which is the default in PyTorch.

After training, we evaluated the PINN solver's errors at cell centers in a $512$ $\times$ $512$ Cartesian grid and at $t=0$, $2$, $4$, $\cdots$, $100$.
Figure \ref{fig:tgv-pinn-loss} shows the histories of the optimization loss and the $L_2$ errors at $t=0$ and $t=40$ of the $u$ velocity on the left vertical axis.
\begin{figure}
    \centering%
    \includegraphics[width=\columnwidth]{tgv-2d-re100/pinn-nl3-nn128-npts8192-convergence}%
    \caption{%
        Histories with respect to optimization iterations for the total loss and the $L_2$ errors of $u$ at $t=0$ and $40$ in the TGV verification of the unsteady PINN solver.
        The left vertical axis corresponds to the total loss and the errors.
        The right vertical axis corresponds to the run time.
        The cyclical scheduler has a slightly better accuracy at $t=40$ with a slightly more time cost, though its total loss is higher.
    }
    \label{fig:tgv-pinn-loss}%
\end{figure}
The same figure also shows the run time (wall time) on the right vertical axis.
The total loss converges to an order of magnitude of \num{e-6}, which may reflect the fact that PyTorch uses single-precision floats.
The errors at $t=0$ and $t=40$ converge to the orders of \num{e-4} and \num{e-2}, respectively.
This observation is reasonable because the net errors over the whole temporal domain is, by definition, similar to the square root of the total, which is \num{e-3}.
The PINN solver got exact initial conditions for training (i.e., $L_4$ to $L_6$), so it is reasonable to see a better prediction accuracy at $t=0$ than later times.
Finally, though the computational performance is not the focus of this paper, for the interested reader's benefit we would like to point out that the PINN solver took about 6 hours to converge with a V100 GPU, while PetIBM needed less than 10 seconds to get an error level of \num{e-2} using a very dated K40 GPU (and most of the time was overhead to initialize the solver).

In sum, we determined the PINN solution to be verified, although the accuracy and the computational cost were not satisfying.
The relatively low accuracy is likely a consequence of the use of single-precision floats and the intrinsic properties of PINNs, rather than implementation errors.
Figure \ref{fig:tgv-pinn-contours} shows the contours of the PINN solver's predictions at $t=40$ for reference.

\begin{figure}[t]
    \centering%
    \includegraphics[width=0.95\columnwidth]{tgv-2d-re100/pinn-nl3-nn256-npts4096-contours.png}
    \caption{%
        Contours at $t=40$ of 2D TGV at $Re=\num{100}$ primitive variables and errors using the unsteady PINN solver.
        Roughly speaking, the absolute errors are at the level of \num{e-3} for primitive variables ($u$, $v$, and $p$), which corresponds to the square root of the total loss.
        The vorticity was obtained from post-processing and hence was augmented in terms of errors.
    }
    \label{fig:tgv-pinn-contours}%
\end{figure}

% vim:ft=tex:
    %! TEX root = main.tex

\subsection{Validation: 2D Cylinder, $Re=\num{40}$}\label{sec:val_2d_cylinder_re40}

We used 2D cylinder flow at $Re=40$ to validate the solvers because it has a similar configuration with the $Re=200$ case that we will study later.
The $Re=40$ flow, however, does not exhibit vortex shedding and reaches a steady-state solution, making it suitable for validating the core functionality of the code.
Experimental data for this flow configuration is also widely available.

The spatial and temporal computational domains are $[-10$, $30]$ $\times$ $[-10$, $10]$ and $t \in [0, 20]$.
A cylinder with a nondimensional radius of $0.5$ sits at $x=y=0$.
Density is $\rho=1$, and kinematic viscosity is $\nu=0.025$.
The initial conditions are $u=1$ and $v=0$ everywhere in the spatial domain.
The boundary conditions are $u=1$ and $v=0$ on $x=-10$ and $y=\pm 10$.
At the outlet, i.e., $x=30$, the boundary conditions are set to 1D convective conditions:
\begin{equation}\label{eq:convec-bc}
    \pdiff{}{t}\begin{bmatrix} u \\ v \end{bmatrix}
    +
    c\pdiff{}{\vec{n}}\begin{bmatrix} u \\ v \end{bmatrix} = 0,
\end{equation}
where $\vec{n}$ is the normal vector of the boundary (pointing outward), and $c=1$ is the convection speed.

We ran the PetIBM validation case on a workstation with one (very old) NVIDIA K40 GPU and 6 CPU cores of the Intel i7-5930K processor.
The grid resolution is $562 \times 447$ with $\Delta t=\num{e-2}$.
The tolerance for all linear solvers in PetIBM was $\num{e-14}$.
We used the same linear solver configurations as those in the TGV verification case.

We validated two implementations of the PINN method with this cylinder flow because both codes were used in the $Re=200$ case (section \ref{sec:case-study}).
The first implementation is an unsteady PINN solver, which is the same piece of code used in the verification case (section \ref{sec:verification}).
It solves the unsteady Navier-Stokes equations as shown in figure \ref{fig:pinn-workflow}.
The second one is a steady PINN solver, which solves the steady Navier-Stokes equations.
The workflow of the steady PINN solver works similar to that in figure \ref{fig:pinn-workflow} except that all time-related terms and losses are dropped.

Both PINN solvers used MLP networks with \num{6} hidden layers and \num{512} neurons each.
The Adam optimizer configuration is the same as that in section \ref{sec:verification}.
The learning rate scheduler is a cyclical learning rate with $\eta_{low}=\num{e-6}$, $\eta_{high}=\num{e-2}$, $N_c=\num{5000}$, and $\gamma={9.9998e-1}$.
We ran all PINN-related validations with one NVIDIA A100 GPU,
all using single-precision floats.

To evaluate PDE losses, \num{256000000} spatial-temporal points were randomly sampled from the computational domain and the desired simulation time range.
The PDE losses were evaluated on \num{25600} points in each iteration, so the Adam optimizer would see each point \num{40} times on average during the \num{400000}-iteration optimization.
On the boundaries, \num{25600000} points were sampled  at $y=\pm 10$, and \num{12800000} at $x=-10$ and $x=30$.
On the cylinder surface, the number of spatial-temporal points were \num{5120000}.
In each iteration, \num{2560}, \num{1280}, and \num{512} points were used, respectively.

Figure \ref{fig:cylinder-re40-pinn-loss} shows the training history of the PINN solvers.
The total loss of the steady PINN solver converged to around \num{e-4}, while that of the unsteady PINN solver converged to around \num{e-2} after about 26 hours of training.
Readers should be aware that the configuration of the PINN solvers might not be optimal, so the accuracy and the computational cost shown in this figure should not be treated as an indication of PINNs' general performance.
In our experience, it is possible to reduce the run time in half but obtain the same level of accuracy by adjusting the number of spatial-temporal points used per iteration.

\begin{figure}
    \centering%
    \includegraphics[width=\columnwidth]{cylinder-2d-re40/loss-hist.png}%
    \caption{%
        Training convergence history of 2D cylinder flow at $Re=\num{40}$ for both steady and unsteady PINN solvers.
    }
    \label{fig:cylinder-re40-pinn-loss}%
\end{figure}


\begin{figure}
    \centering%
    \includegraphics[width=\columnwidth]{cylinder-2d-re40/drag-lift-coeffs}%
    \caption{%
        Drag and lift coefficients of 2D cylinder flow at $Re=\num{40}$ w/ PINNs.
    }
    \label{fig:cylinder-re40-drag-lift}%
\end{figure}

Figure \ref{fig:cylinder-re40-drag-lift} gives the drag and lift coefficients ($C_D$ and $C_L$) with respect to simulation time, where PINN and PetIBM visually agree.
Table \ref{table:cylinder-re40-cd-comparison} compares the values of $C_D$ against experimental data and simulation data from the literature.
Values from different works in the literature do not closely agree with each other.
Though there is not a single value to compare against, at least the $C_D$ from the PINN solvers and PetIBM fall into the range of other published works.
We consider the results of $C_D$ validated for the PINN solvers and PetIBM.


\begin{figure}
    \centering%
    \includegraphics[width=0.95\columnwidth]{cylinder-2d-re40/surface-pressure}%
    \caption{%
        Surface pressure distribution of 2D cylinder flow at $Re=\num{40}$
    }
    \label{fig:cylinder-re40-pinn-surfp}%
\end{figure}


\begin{figure*}
    \centering%
    \includegraphics{cylinder-2d-re40/contour-comparison}%
    \caption{%
        Contour plots for 2D cylinder flow at $Re=\num{40}$
    }
    \label{fig:cylinder-re40-contours}%
\end{figure*}

Figure \ref{fig:cylinder-re40-pinn-surfp} shows the pressure distribution on the cylinder surface.
Again, though there is not a single solution that all works agree upon, the results from PetIBM and the PINN solvers visually agree with the published literature.
We consider PetIBM and both PINN solvers validated.
Finally, figure \ref{fig:cylinder-re40-contours} compares the steady-state flow fields (i.e., the snapshots at $t=20$ for PetIBM and the unsteady PINN solver).
The PINN solvers' results visually agree with PetIBM's.
The variation in the vorticity of PINNs only happens at the contour line of \num{0}, so it is likely caused by trivial rounding errors.
Note that vorticity is obtained by post-processing for all solvers.
PetIBM used central difference to calculate the vorticity, while the PINN solvers used automatic differentiation to obtain it.


\begin{table}[h]
    \centering%
    \begin{threeparttable}
        \begin{tabular}{lccc}
            \toprule
            & $C_D$ & $C_{D_p}$ & $C_{D_f}$ \\
            \midrule
            Steady PINN & 1.62 & 1.06 & 0.55 \\
            Unsteady PINN & 1.60 & 1.06 & 0.55 \\
            PetIBM & 1.63 & 1.02 & 0.61 \\
            Rosetti et al., 2012\cite{rosetti_urans_2012}\tnote{1} & \num{1.74+-0.09} & n/a & n/a \\
            Rosetti et al., 2012\cite{rosetti_urans_2012}\tnote{2} & 1.61 & n/a & n/a \\
            Sen et al., 2009\cite{sen_steady_2009}\tnote{2} & 1.51 & n/a & n/a \\
            Park et al., 1988\cite{park_numerical_1998}\tnote{2} & 1.51 & 0.99 & 0.53 \\
            Tritton, 1959\cite{tritton_experiments_1959}\tnote{1} & 1.48--1.65 & n/a & n/a \\
            Grove et al., 1964\cite{grove_experimental_1964}\tnote{1} & n/a & 0.94 & n/a \\
            \bottomrule
        \end{tabular}%
        \begin{tablenotes}
            \footnotesize
            \item [1] Experimental result
            \item [2] Simulation result
        \end{tablenotes}
        \caption{%
            Validation of drag coefficients. %
            $C_D$, $C_{D_p}$, and $C_{D_f}$ denote the coefficients of total drag, pressure drag, %
            and friction drag, respectively.%
        }%
        \label{table:cylinder-re40-cd-comparison}
    \end{threeparttable}
\end{table}%



% vim:ft=tex:


    \section{Case Study: 2D Cylinder Flow at $Re=\num{200}$}\label{sec:case-study}
    %! TEX root = main.tex

The previous section presented successful verification with a Taylor-Green vortex having an analytical solution, and validation of the solvers with the $Re=40$ cylinder flow, which exhibits a steady state solution.
Those results give confidence that the solvers are correctly solving the Navier-Stokes equations, and able to model vortical flow. In this section, we study the case of cylinder flow at $Re=200$, exhibiting vortex shedding.

\subsection{Case configurations}

The computational domain is $[-8$, $25]$ $\times$ $[-8$, $8]$ for $x$ and $y$, and $t\in[0$, $200]$.
Other boundary conditions, initial conditions, and density were the same as those in section \ref{sec:val_2d_cylinder_re40}.
The non-dimensional kinematic viscosity is set to $0.005$ to make the Reynolds number $200$.

The PetIBM simulation was done with a grid resolution of $1485$ $\times$ $720$ and $\Delta t = \num{5e-3}$.
The hardware used and the configurations of the linear solvers were the same as described in section \ref{sec:val_2d_cylinder_re40}.

As for the PINN solvers, in addition to the steady and unsteady solvers, a third PINN solver was used: a data-driven PINN.
The data-driven PINN solver is the same as the unsteady PINN solver but replaces the three initial condition losses ($L_4$ to $L_6$) with:

\begin{equation}\label{eq:data-driven-loss}
    \left\{
        \begin{array}{l}
            L_4 = u - u_{data}\\
            L_5 = v - v_{data}\\
            L_6 = p - p_{data}\\
        \end{array}
    \right.
    ,\text{ if }
    \begin{array}{l}
        \vec{x} \in \Omega \\
        t \in T_{data}
    \end{array}
\end{equation}
where subscript $data$ denotes data from a PetIBM simulation.
$T_{data}$ denotes the time range for which we feed the PetIBM simulation data to the data-driven PINN solver.
In this case, $T_{data} \equiv \left[125, 140\right]$.
The PetIBM simulation outputted transient snapshots every 1 second in simulation time, hence the data fed to the data-driven PINN solver consisted of 16 snapshots.
These snapshots contain around $3$ full periods of vortex shedding.
The total number of spatial-temporal points in these snapshots is around $\num{17000000}$, and we only used $\num{6400}$ every iteration, meaning each data batch was repeated approximately every $\num{2650}$ iterations.
Except for replacing the IC losses with a data-driven approach, all other loss terms and the code in the data-driven PINN solver remain the same as the unsteady PINN solver.

Note that for the data-driven PINN solver, the PDE and boundary condition losses were evaluated only in $t\in[125$, $200]$ because we treated the PetIBM snapshots as if they were initial conditions.
Another note is the use of steady PINN solver.
The $Re=200$ cylinder flow is not expected to have a steady-state solution.
However, it is not uncommon to see a steady-state flow solver used for unsteady flows for engineering purposes, especially two or three decades ago when computing power was much more restricted.

The MLP network used on all PINN solvers has 6 hidden layers and 512 neurons per layer.
The configurations of spatial-temporal points are the same as those in section \ref{sec:val_2d_cylinder_re40}.
The Adam optimizer is also the same, except that now we ran for \num{1000000} optimization iterations.
The parameters of the cyclical learning rate scheduler are now: $\eta_{low}=\num{1e-6}$, $\eta_{high}=\num{1e-2}$, $N_c=5000$, and $\gamma=\num{0.9999915}$.
The hardware used was one NVIDIA A100 GPU for all PINN runs.

\subsection{Results}\label{sec:cylinder-re200-results}

The overall run times for the steady, unsteady, and data-driven PINN solvers were about 28 hours, 31 hours, and 33.5 hours using one A100 GPU.
The PetIBM simulation, on the other hand, took around 1.7 hours with a K40 GPU, which is 5-generation-behind in terms of the computing technology.

Figure \ref{fig:cylinder-re200-pinn-loss} shows the aggregated loss convergence history of all cases.
\begin{figure}
    \centering%
    \includegraphics[width=0.95\columnwidth]{cylinder-2d-re200/loss-hist}%
    \caption{%
        Training convergence history of 2D cylinder flow at $Re=\num{200}$ w/ PINNs
    }
    \label{fig:cylinder-re200-pinn-loss}%
\end{figure}
It shows both the losses and the run times of the three PINN solvers.
As seen in section \ref{sec:val_2d_cylinder_re40}, the unsteady PINN solver converges to a higher total loss than the steady PINN solver does.
Also, the data-driven PINN solver converges to an even higher total loss.
However, it is unclear at this point if having a higher loss means a higher prediction error in data-driven PINN because we replaced the initial condition losses with 16 snapshots from PetIBM and only ran the data-driven PINN solver for $t\in[125, 200]$.

Figure \ref{fig:cylinder-re200-drag-lift} shows the drag and lift coefficients versus simulation time.
\begin{figure}[t]
    \centering%
    \includegraphics[width=0.95\columnwidth]{cylinder-2d-re200/drag-lift-coeffs}%
    \caption{%
        Drag and lift coefficients of 2D cylinder flow at $Re=\num{200}$ w/ PINNs
    }
    \label{fig:cylinder-re200-drag-lift}%
\end{figure}
The coefficients from the steady case are shown as just a horizontal line since there is no time variable in this case.
The unsteady case, to our surprise, does not exhibit oscillations, meaning it results on no vortex shedding, even though it fits well with the PetIBM result before vortex shedding starts (at about $t=75$).
Comparing the coefficients between the steady, unsteady, and PetIBM's values before shedding, we believe the unsteady PINN in this particular case behaves just like a steady solver.
This is supported by the values in table \ref{table:cylinder-2d-re200-cd}, in which we compare $C_D$ against published values in the literature of both unsteady and steady CFD simulations.
\begin{table}
    \centering%
    \begin{threeparttable}[b]
        \begin{tabular}{lcc}
            \toprule
            & $C_D$ \\
            \midrule
            PetIBM & 1.38   \\
            Steady PINN & 0.95 \\
            Unsteady PINN & 0.95 \\
            Deng et al., 2007\cite{deng_hydrodynamic_2007}\tnote{1} & 1.25 \\
            Rajani et al., 2009\cite{Rajani2009}\tnote{1} & 1.34 \\
            Gushchin \& Shchennikov, 1974\cite{gushchin_numerical_1974}\tnote{2} & 0.97 \\
            Fornberg, 1980\cite{fornberg_numerical_1980}\tnote{2} & 0.83 \\
            \bottomrule
        \end{tabular}%
        \begin{tablenotes}
            \footnotesize
            \item [1] Unsteady simulations.
            \item [2] Steady simulations.
        \end{tablenotes}
        \caption{%
            PINNs, 2D Cylinder, $Re=200$: validation of drag coefficients.%
            The data-driven case is excluded because it does not have an obvious periodic state nor a steady-state solution.%
        }%
        \label{table:cylinder-2d-re200-cd}
    \end{threeparttable}
\end{table}%
The $C_D$ obtained from the unsteady PINN is the same as the steady PINN and close to those steady CFD simulations.

As for the data-driven case, its temporal domain is $t\in[125$, $200]$, so the coefficients' trajectories start from $t=125$.
The result, again unexpected to us, only exhibits shedding in the timeframe with PetIBM data, i.e., $t\in[125$, $140]$.
This result also implies that data-driven PINNs may be more difficult to train, compared to data-free PINNs and regular data-only model fitting.
Even in the time range with PetIBM data, the data-driven PINN solver is not able to reach the given maximal $C_L$, and the $C_D$ is obviously off from the given data.
After $t=140$, the trajectories quickly fall back to the no-shedding pattern, though it still deviates from the trajectories of the steady and unsteady PINNs.
Combining the loss magnitude shown in figure \ref{fig:cylinder-re200-pinn-loss}, the deviation of $C_D$ and $C_L$ from the data-driven PINN may be caused by not enough training.
As figure \ref{fig:cylinder-re200-pinn-loss} shows data-driven PINN had already converged, other optimization techniques or hyperparameter tuning may be required to further reduce the loss.
Insufficient training only explains why the data-driven case deviates from the PetIBM's data in $t \in [125, 140]$ and from the other two PINNs for $t > 140$.
Even with a better optimization and eventually a lower loss, based on the trajectories, we do not believe the shedding will continue after $t=140$.

To examine how the transient flow develops, we visually compared several snapshots of the flow fields from PetIBM, unsteady PINN, and the data-driven PINN, shown in figures \ref{fig:cylinder-re200-pinn-contours-u}, \ref{fig:cylinder-re200-pinn-contours-v}, \ref{fig:cylinder-re200-pinn-contours-p}, and \ref{fig:cylinder-re200-pinn-contours-omega_z}.
We also present the flow contours from the steady PINN in figure \ref{fig:cylinder-re200-steady-pinn-contours} for reference.

\begin{figure*}
    \centering%
    \includegraphics[width=0.95\textwidth]{cylinder-2d-re200/contour-comparison-u}%
    \caption{%
        $u$-velocity comparison of 2D cylinder flow of $Re=\num{200}$ between PetIBM, unsteady PINN, and data-driven PINN.
    }
    \label{fig:cylinder-re200-pinn-contours-u}%
\end{figure*}

\begin{figure*}
    \centering%
    \includegraphics[width=0.95\textwidth]{cylinder-2d-re200/contour-comparison-v}%
    \caption{%
        $v$-velocity comparison of 2D cylinder flow of $Re=\num{200}$ between PetIBM, unsteady PINN, and data-driven PINN.
    }
    \label{fig:cylinder-re200-pinn-contours-v}%
\end{figure*}

\begin{figure*}
    \centering%
    \includegraphics[width=0.95\textwidth]{cylinder-2d-re200/contour-comparison-p}%
    \caption{%
        Pressure comparison of 2D cylinder flow of $Re=\num{200}$ between PetIBM, unsteady PINN, and data-driven PINN.
    }
    \label{fig:cylinder-re200-pinn-contours-p}%
\end{figure*}

\begin{figure*}
    \centering%
    \includegraphics[width=0.95\textwidth]{cylinder-2d-re200/contour-comparison-omega_z}%
    \caption{%
        Vorticity ($\omega_z$) comparison of 2D cylinder flow of $Re=\num{200}$ between PetIBM, unsteady PINN, and data-driven PINN.
    }
    \label{fig:cylinder-re200-pinn-contours-omega_z}%
\end{figure*}

\begin{figure}[t]
    \centering%
    \includegraphics[width=0.95\columnwidth]{cylinder-2d-re200/contour-comparison-steady}%
    \caption{%
        Contours of 2D cylinder flow at $Re=\num{200}$ w/ steady PINN.
    }
    \label{fig:cylinder-re200-steady-pinn-contours}%
\end{figure}

At $t=10$, we can see the wake is still developing, and the unsteady PINN visually matches PetIBM.
At $t=50$, the contours again show the unsteady PINN matching the PetIBM simulation before shedding.
These observations verify that the unsteady PINN is indeed solving unsteady governing equations.
The data-driven PINN does not show meaningful results because $t=10$ and $50$ are out of the data-driven PINN's temporal domain.
These results also indicate that the data-driven PINN is not capable of extrapolating backward in time in dynamical systems.

At $t=140$, vortex shedding has already happened.
However, the unsteady PINN solution does not show any shedding.
Moreover, the unsteady PINN's contour plot is similar to the steady case in figure \ref{fig:cylinder-re200-steady-pinn-contours}.
$t=140$ is also the last snapshot we fed into the data-driven PINN for training.
The contour of the data-driven PINN at this time shows that it at least could qualitatively capture the shedding, which is expected.
At $t=144$, it is just $4$ time units from the last snapshot being fed to the data-driven PINN.
At such time, the data-driven PINN has already stopped generating new vortices.
The existing vortex can be seen moving toward the boundary, and the wake is gradually restoring to the steady state wake.
Flow at $t=190$ further confirms that the data-driven PINN's behavior is tending toward that of the unsteady PINN, which behaves like a steady state solver.
On the other hand, the solutions from the unsteady PINN for these times remain steady.

Figure \ref{fig:cylinder-re200-pinn-vort-gen} shows the vorticity from PetIBM and the data-driven PINN in the vicinity of the cylinder in $t \in [140, 142.5]$, which contains a half cycle of vortex shedding.
\begin{figure}
    \centering%
    \includegraphics[width=\columnwidth]{cylinder-2d-re200/vorticity_z}%
    \caption{%
        Vorticity generation near the cylinder for 2D cylinder flow of $Re=\num{200}$ at $t=140$, $141$, $142$, and $142.5$ w/ data-driven PINNs.
    }
    \label{fig:cylinder-re200-pinn-vort-gen}%
\end{figure}
These contours compare how vorticity was generated right after we stopped feeding PetIBM data into the data-driven PINN.
These comparisons might shed some light on why the data-free PINN cannot generate vortex shedding and why the data-driven PINN stops to do so after $t=140$.

At $t=140$, PetIBM and the data-driven PINN show visually indistinguishable vorticity contours.
This is expected as the data-driven PINN has training data from PetIBM at this time.
At $t=141$ in PetIBM's results, the main clockwise vortex (the blue circular pattern in the domain of $[1, 2]\times[-0.5, 0.5]$) moves downstream.
It slows down the downstream $u$ velocity and accelerates the $v$ velocity in $y<0$.
Intuitively, we can treat the main clockwise vortex as a blocker that blocks the flow in $y<0$ and forces the flow to move upward.
The net effect is the generation of a counterclockwise vortex at around $x\approx 1$ and $y \in [-0.5, 0]$.
This new counterclockwise vortex further generates a small but strong secondary clockwise vortex on the cylinder surface in $y\in[-0.5, 0]$.
On the other hand, the result of the data-driven PINN at $t=141$ shows that the main clockwise vortex becomes more dissipated and weaker, compared to that in PetIBM.
It is possible that the main clockwise vortex is not strong enough to slow down the flow in $y<0$ nor to bring the flow upward.
The downstream flow in $y<0$ (the red arm-like pattern below the cylinder) thus does not change its direction and keeps going straight down in the $x$ direction.
In the results of $t=142$ and $t=142.5$ from PetIBM, the flow completes a half cycle.
That is, the flow pattern at $t=142.5$ is an upside down version of that at $t=140$.
The results from the data-driven PINN, however, do not have any new vortices and the wake becomes more like steady flow.
These observations might indicate that the PINN is either diffusive or dissipative (or both).

Next, we examined the Q-criterion in the same vicinity of the cylinder in $t\in[140, 142.5]$, shown in figure \ref{fig:cylinder-re200-pinn-qcriterion}.
\begin{figure}
    \centering%
    \includegraphics[width=\columnwidth]{cylinder-2d-re200/qcriterion}%
    \caption{%
        Q-criterion generation near the cylinder for 2D cylinder flow of $Re=\num{200}$ at $t=140$, $141$, $142$, and $142.5$ w/ data-driven PINNs.
    }
    \label{fig:cylinder-re200-pinn-qcriterion}%
\end{figure}
The Q-criterion is defined as follows \cite{jeong_identification_1995}:
\begin{equation}
    Q \equiv \frac{1}{2}\left(\lVert \Omega \rVert^2 - \lVert S \rVert^2\right),
\end{equation}
where $\Omega\equiv\frac{1}{2}\left(\nabla\vec{u}-\nabla\vec{u}^\mathsf{T}\right)$ is the vorticity tensor;
$S\equiv\frac{1}{2}\left(\nabla\vec{u}+\nabla\vec{u}^\mathsf{T}\right)$ is the strain rate tensor;
and $\nabla\vec{u}$ is the velocity gradient tensor.
A criterion $Q > 0$ identifies a vortex structure in the fluid flow, that is, where the rotation rate is greater than the strain rate.

We observe that vortices in the data-driven PINN are diffusive and could be dissipative.
Moreover, judging by the locations of vortex centers, vortices also move slower in the PINN solution than with PetIBM.
The edges of the vortices move at a different speed from that of the vortex centers in the PINN case.
This might be hinting at the existence of numerical dispersion in the PINN solver.

\subsection{Dynamical Modes and Koopman Analysis}\label{sec:cylinder-re200-koopman}

We conducted spectral analysis on the cylinder flow to extract frequencies embedded in the simulation results.
Fluid flow is a dynamical system, and how information (or signals) propagates in time plays an important role.
Information with different frequencies advances at different speeds in the temporal direction, and the superposition of information forms complicated flow patterns over time.
Spectral analysis reveals a set of modes, each associated with a fixed oscillation frequencies and decay or growth rate, called {\it dynamic modes} in fluid dynamics.
By comparing the dynamic modes in the solutions obtained with PINNs and PetIBM, we may examine how well or how badly the data-driven PINN learned information with different frequencies.
Koopman analysis is a method to achieve such spectral analysis for dynamical systems.
Please refer to {\it the method of snapshots} in reference \cite{chen_variants_2012} and reference \cite{rowley_spectral_2009} for the theory and the algorithms used in this work.

We analyzed the results from PetIBM and the data-driven PINN in $t\in$$[125$, $140]$, which contains about three full cycles of vortex shedding.
A total of $76$ solution snapshots were used from each solver.
The time spacing is $\Delta t = 0.2$.
The Koopman analysis would result in $75$ modes.
Since the snapshots cover three full cycles, we would expect only $25$ distinct frequencies and $25$ nontrivial modes---only $25$ out of $76$ snapshots are distinct.
However, this expectation only happens when the data are free from noise and numerical errors and when the number of three periods is exact.
We would see more than $25$ distinct frequencies and modes if the data were not ideal.
In $t \in [125, 140]$, the data-driven PINN was trained against PetIBM's data, so we expected to see similar spectral results between the two solvers.

To put it simply, each dynamic mode is identified by a complex number.
Taking logarithm on the complex number's absolute value gives a mode's growth rate, and the angle of the complex number corresponds to a mode's frequency.
Figure \ref{fig:cylinder-re200-koopman-eig-dist} shows the distributions of the dynamic modes on the complex plane.
\begin{figure}
    \centering%
    \includegraphics[width=\columnwidth]{cylinder-2d-re200/koopman_eigenvalues_complex}%
    \caption{%
        Distribution of the Koopman eigenvalues on the complex plane for 2D cylinder flow at $Re=\num{200}$ obteined with PetIBM and with data-driven PINN.
    }
    \label{fig:cylinder-re200-koopman-eig-dist}%
\end{figure}
The color of each dot represents the normalized strength of the corresponding mode, which is also obtained from the Koopman analysis.
The star marker denotes the mode with a frequency of zero, i.e., a steady or time-independent mode.
This mode usually has much higher strength than others, so we excluded it from the color map and annotated its strength directly.
Koopman analysis delivers dynamical modes with complex conjugate pairs, so the modes are symmetric with respect to the real-number axis. 
A conjugate pair has an opposite sign in the frequencies mathematically but has the same physical frequency.

We also plotted a circle with a radius of one on each figure.
As flow has already reached a fully periodic regime, the growth rates should be zero because no mode becomes stronger nor weaker.
In other words, all modes were expected to fall on this circle on the complex plane.
If a mode falls inside the circle, it has a negative growth rate, and its contribution to the solution diminishes over time.
Similarly, a mode falling outside the circle has a positive growth rate and becomes stronger over time.

On the complex plane, all the modes captured by PetIBM (the left plot in figure \ref{fig:cylinder-re200-koopman-eig-dist}) fall onto the circle or very close to the circle.
The plot shows $75$---rather than $25$---distinct $\lambda_j$ and modes, but the modes are evenly clustered into 25 groups.
Each group has three modes, among which one or two modes fall on the circle, while the remaining one(s) falls inside but very close to the circle.
Modes within each group have a similar frequency, and the one precisely on the circle has significantly higher strength than other modes (if not all modes in the group are weak).
Due to the numerical errors in PetIBM's solutions, data in a vortex period are similar to but not exactly the same as those in another period.
The strong modes falling precisely on the circle may represent the period-averaged flow patterns and are the $25$ modes we expected earlier. 
The effect of numerical errors was filtered out from these modes.
We call these 25 modes primary modes and all other modes secondary modes.
Secondary modes are mostly weak and may come from the numerical errors in the PetIBM simulation.
The plot shows these secondary modes are slightly dispersive but non-increasing over time, which is reasonable because the numerical schemes in PetIBM are stable.

As for the PINN result (the right pane in figure \ref{fig:cylinder-re200-koopman-eig-dist}), the mode distribution is not as structured as with PetIBM.
It is hard to distinguish if all 25 expected modes also exist in this plot.
However, we observe that at least the top 7 primary modes (the steady mode, two purple and 4 orange dots on the circle) also exist in the PINN case.
Secondary modes spread out more widely on and inside the circle, compared to the clustered modes in PetIBM.
We believe this means that PINN is more numerically dispersive and noisy.
The frequencies of many of these secondary modes do not exist in PetIBM.
So one possible source of these additional frequencies and modes may be the PINN method itself.
It could be insufficient training or that the neural network itself inherently is dispersive. 
However, secondary modes on the circle are weak.
We suspect that their contribution to the solution may be trivial.

A more concerning observation is the presence of damped modes (modes that fall inside the circle). 
These modes have negative growth rates and hence are damped over time.
We believe these modes contribute significantly to the solution because their strengths are substantial.
The existence of the damped modes also means that PINN's predictions have more important discrepancies from one vortex period to another vortex period, compared to the PetIBM simulation.
In addition, the flow pattern in PINN would keep changing after $t=140$.
They may be the culprits causing the PINN solution to quickly fall back to a non-oscillating flow pattern for $t>140$.
We may consider these errors as numerical dissipation.
However, whether these errors came from insufficient training or were inherent in the PINN is unclear.

Note that the spectral analysis was done against data in $t\in[125, 140]$.
It does not mean the solutions in $t>140$ also have the same spectral characteristics: the flow system is nonlinear, but the Koopman analysis uses linear approximations \cite{rowley_spectral_2009}.

Figure \ref{fig:cylinder-re200-koopman-mode-strength} shows mode strengths versus frequencies.
\begin{figure}
    \centering%
    \includegraphics[width=\columnwidth]{cylinder-2d-re200/koopman_mode_strength}%
    \caption{%
        Mode strengths versus mode frequencies for 2D cylinder flow at $Re=\num{200}$.
        Note that we use a log scale for the vertical axis.
    }
    \label{fig:cylinder-re200-koopman-mode-strength}%
\end{figure}
The plots use nondimensional frequency, i.e., Strouhal number, in the horizontal axes. 
We only plotted modes with positive numerical frequencies for a concise visualization.
Plots in this figure also show the same observations as in the previous paragraphs: the data-driven PINN is more dispersive and dissipative.

An observation that is now clearer from figure \ref{fig:cylinder-re200-koopman-mode-strength} is the strength distribution.
In PetIBM's case, strengths decrease exponentially from the steady mode (i.e., $St=0$) to high-frequency modes.
One can deduce a similar conclusion from PetIBM's simulation result.
The vortex shedding is dominated by a single frequency (this frequency is $St\approx 0.2$ because $t\in[125, 140]$ contains three periods).
Therefore, the flow should be dominated by the steady mode and a mode with a frequency close to $St=0.2$.
We can indeed verify this statement for PetIBM's case in figure \ref{fig:cylinder-re200-koopman-mode-strength}: the primary modes of $St=0$ and $St\approx 0.2$ are much stronger than others.
The strength of the immediately next mode, i.e., $St\approx 0.4$, drops by an order of magnitude. 
Note the use of a logarithmic scale.
If we re-plot the figure using a regular scale, only $St=0$ and $St=0.2$ would be visible in the figure.

The strength distribution in the case of PINN also shows that $St=0$ and $St\approx 0.2$ are strong.
However, they are not the only dominating modes.  
Some other modes also have strengths at around \num{e-1}.
As discussed in the previous paragraphs, these additional strong modes are damped modes.
We also observed that some damped modes have the same frequencies as primary modes.
For example, the secondary modes at $St=0$ and $St=0.2$ are damped modes.
Note that for $St=0$, if a mode is damped, then it is not a steady mode anymore because its magnitude changes with time, though it is still non-oscillating.

Table \ref{table:koopman-petibm} summarizes the top 4 modes (ranked by their strengths) in PetIBM's spectral result.
\begin{table}
    \begin{threeparttable}[b]
        \begin{tabular}{ccccc}
            \toprule
            $St$ & Strength & Growth Rate & Contours \\
            \midrule
            0     & 0.96 & 1.3e-7  & Figure \ref{fig:cylinder-re200-koopman-petibm-1st}\\
            0.201 & 0.20 & -4.3e-7 & Figure \ref{fig:cylinder-re200-koopman-petibm-2nd}\\
            0.403 & 0.04 & 1.7e-6  & Figure \ref{fig:cylinder-re200-koopman-petibm-3rd}\\
            0.604 & 0.03 & 2.7e-6  & Figure \ref{fig:cylinder-re200-koopman-petibm-4th}\\
            \bottomrule
        \end{tabular}%
        \caption{%
            2D Cylinder, $Re=200$: top 4 primary dynamic modes (sorted by strengths) for PetIBM%
        }%
        \label{table:koopman-petibm}
    \end{threeparttable}
\end{table}%
For reference, these modes' contours are provided in the appendex as denoted in the table.
The dynamic modes are complex-valued, and the contours include both the real and the imaginary parts.
Note the growth rates of these 4 modes are not exactly zero but around \num{e-6} and \num{e-7}.
We were unsure if we could treat them as zero at these orders of magnitude.
If not, and if they do cause the primary modes to be slightly damped or augmented over time, then we believe they also serve as a reasonable explanation for the existence of the other 50 non-primary modes in PetIBM---to compensate for the loss or the gain in the primary modes.

Table \ref{table:koopman-pinn-primary} lists the PINN solution's top 4 primary modes, which are the same as those in table \ref{table:koopman-petibm}.
\begin{table}
    \begin{threeparttable}[b]
        \begin{tabular}{ccccc}
            \toprule
            $St$ & Strength & Growth Rate & Contours \\
            \midrule
            0     & 0.97 & -2.2e-6  & Figure \ref{fig:cylinder-re200-koopman-pinn-primary-1st}\\
            0.201 & 0.18 & -9.4e-6  & Figure \ref{fig:cylinder-re200-koopman-pinn-primary-2nd}\\
            0.403 & 0.03 &  2.3e-5  & Figure \ref{fig:cylinder-re200-koopman-pinn-primary-3rd}\\
            0.604 & 0.03 & -8.6e-5  & Figure \ref{fig:cylinder-re200-koopman-pinn-primary-4th}\\
            \bottomrule
        \end{tabular}%
        \caption{%
            2D Cylinder, $Re=200$: top 4 primary dynamic modes (sorted by strengths) for PINN%
        }%
        \label{table:koopman-pinn-primary}
    \end{threeparttable}
\end{table}%
Table \ref{table:koopman-pinn-damped} shows the top 4 secondary modes in the PINN method's result.
\begin{table}
    \begin{threeparttable}[b]
        \begin{tabular}{ccccc}
            \toprule
            $St$ & Strength & Growth Rate & Contours \\
            \midrule
            1.142 & 0.12 & -0.24 & Figure \ref{fig:cylinder-re200-koopman-pinn-damped-1st}\\
            1.253 & 0.08 & -0.22 & Figure \ref{fig:cylinder-re200-koopman-pinn-damped-2nd}\\
            0.633 & 0.05 & -0.14 & Figure \ref{fig:cylinder-re200-koopman-pinn-damped-3rd}\\
            0.761 & 0.04 & -0.13 & Figure \ref{fig:cylinder-re200-koopman-pinn-damped-4th}\\
            \bottomrule
        \end{tabular}%
        \caption{%
            2D Cylinder, $Re=200$: top 4 damped dynamic modes (sorted by strengths) for PINN%
        }%
        \label{table:koopman-pinn-damped}
    \end{threeparttable}
\end{table}%
Corresponding contours are also included in the appendix and denoted in the tables for readers' reference.
The growth rates of the primary modes in the PINN method's result are around \num{e-5} and \num{e-6}, slightly larger than those of PetIBM.
If these orders of magnitude can not be deemed as zero, then these primary modes are slightly damped and dissipative, though the major source of the numerical dissipation may still be the secondary modes in table \ref{table:koopman-pinn-damped}.

% vim:ft=tex:


    \section{Discussion}
    \mySection{Related Works and Discussion}{}
\label{chap3:sec:discussion}

In this section we briefly discuss the similarities and differences of the model presented in this chapter, comparing it with some related work presented earlier (Chapter \ref{chap1:artifact-centric-bpm}). We will mention a few related studies and discuss directly; a more formal comparative study using qualitative and quantitative metrics should be the subject of future work.

Hull et al. \citeyearpar{hull2009facilitating} provide an interoperation framework in which, data are hosted on central infrastructures named \textit{artifact-centric hubs}. As in the work presented in this chapter, they propose mechanisms (including user views) for controlling access to these data. Compared to choreography-like approach as the one presented in this chapter, their settings has the advantage of providing a conceptual rendezvous point to exchange status information. The same purpose can be replicated in this chapter's approach by introducing a new type of agent called "\textit{monitor}", which will serve as a rendezvous point; the behaviour of the agents will therefore have to be slightly adapted to take into account the monitor and to preserve as much as possible the autonomy of agents.

Lohmann and Wolf \citeyearpar{lohmann2010artifact} abandon the concept of having a single artifact hub \cite{hull2009facilitating} and they introduce the idea of having several agents which operate on artifacts. Some of those artifacts are mobile; thus, the authors provide a systematic approach for modelling artifact location and its impact on the accessibility of actions using a Petri net. Even though we also manipulate mobile artifacts, we do not model artifact location; rather, our agents are equipped with capabilities that allow them to manipulate the artifacts appropriately (taking into account their location). Moreover, our approach considers that artifacts can not be remotely accessed, this increases the autonomy of agents.

The process design approach presented in this chapter, has some conceptual similarities with the concept of \textit{proclets} proposed by Wil M. P. van der Aalst et al. \citeyearpar{van2001proclets, van2009workflow}: they both split the process when designing it. In the model presented in this chapter, the process is split into execution scenarios and its specification consists in the diagramming of each of them. Proclets \cite{van2001proclets, van2009workflow} uses the concept of \textit{proclet-class} to model different levels of granularity and cardinality of processes. Additionally, proclets act like agents and are autonomous enough to decide how to interact with each other.

The model presented in this chapter uses an attributed grammar as its mathematical foundation. This is also the case of the AWGAG model by Badouel et al. \citeyearpar{badouel14, badouel2015active}. However, their model puts stress on modelling process data and users as first class citizens and it is designed for Adaptive Case Management.

To summarise, the proposed approach in this chapter allows the modelling and decentralized execution of administrative processes using autonomous agents. In it, process management is very simply done in two steps. The designer only needs to focus on modelling the artifacts in the form of task trees and the rest is easily deduced. Moreover, we propose a simple but powerful mechanism for securing data based on the notion of accreditation; this mechanism is perfectly composed with that of artifacts. The main strengths of our model are therefore : 
\begin{itemize}
	\item The simplicity of its syntax (process specification language), which moreover (well helped by the accreditation model), is suitable for administrative processes;
	\item The simplicity of its execution model; the latter is very close to the blockchain's execution model \cite{hull2017blockchain, mendling2018blockchains}. On condition of a formal study, the latter could possess the same qualities (fault tolerance, distributivity, security, peer autonomy, etc.) that emanate from the blockchain;
	\item Its formal character, which makes it verifiable using appropriate mathematical tools;
	\item The conformity of its execution model with the agent paradigm and service technology.
\end{itemize}
In view of all these benefits, we can say that the objectives set for this thesis have indeed been achieved. However, the proposed model is perfectible. For example, it can be modified to permit agents to respond incrementally to incoming requests as soon as any prefix of the extension of a bud is produced. This makes it possible to avoid the situation observed on figure \ref{chap3:fig:execution-figure-4} where the associated editor is informed of the evolution of the subtree resulting from $C$ only when this one is closed. All the criticisms we can make of the proposed model in particular, and of this thesis in general, have been introduced in the general conclusion (page \pageref{chap5:general-conclusion}) of this manuscript.





    \section{Conclusion}
    % \vspace{-0.5em}
\section{Conclusion}
% \vspace{-0.5em}
Recent advances in multimodal single-cell technology have enabled the simultaneous profiling of the transcriptome alongside other cellular modalities, leading to an increase in the availability of multimodal single-cell data. In this paper, we present \method{}, a multimodal transformer model for single-cell surface protein abundance from gene expression measurements. We combined the data with prior biological interaction knowledge from the STRING database into a richly connected heterogeneous graph and leveraged the transformer architectures to learn an accurate mapping between gene expression and surface protein abundance. Remarkably, \method{} achieves superior and more stable performance than other baselines on both 2021 and 2022 NeurIPS single-cell datasets.

\noindent\textbf{Future Work.}
% Our work is an extension of the model we implemented in the NeurIPS 2022 competition. 
Our framework of multimodal transformers with the cross-modality heterogeneous graph goes far beyond the specific downstream task of modality prediction, and there are lots of potentials to be further explored. Our graph contains three types of nodes. While the cell embeddings are used for predictions, the remaining protein embeddings and gene embeddings may be further interpreted for other tasks. The similarities between proteins may show data-specific protein-protein relationships, while the attention matrix of the gene transformer may help to identify marker genes of each cell type. Additionally, we may achieve gene interaction prediction using the attention mechanism.
% under adequate regulations. 
% We expect \method{} to be capable of much more than just modality prediction. Note that currently, we fuse information from different transformers with message-passing GNNs. 
To extend more on transformers, a potential next step is implementing cross-attention cross-modalities. Ideally, all three types of nodes, namely genes, proteins, and cells, would be jointly modeled using a large transformer that includes specific regulations for each modality. 

% insight of protein and gene embedding (diff task)

% all in one transformer

% \noindent\textbf{Limitations and future work}
% Despite the noticeable performance improvement by utilizing transformers with the cross-modality heterogeneous graph, there are still bottlenecks in the current settings. To begin with, we noticed that the performance variations of all methods are consistently higher in the ``CITE'' dataset compared to the ``GEX2ADT'' dataset. We hypothesized that the increased variability in ``CITE'' was due to both less number of training samples (43k vs. 66k cells) and a significantly more number of testing samples used (28k vs. 1k cells). One straightforward solution to alleviate the high variation issue is to include more training samples, which is not always possible given the training data availability. Nevertheless, publicly available single-cell datasets have been accumulated over the past decades and are still being collected on an ever-increasing scale. Taking advantage of these large-scale atlases is the key to a more stable and well-performing model, as some of the intra-cell variations could be common across different datasets. For example, reference-based methods are commonly used to identify the cell identity of a single cell, or cell-type compositions of a mixture of cells. (other examples for pretrained, e.g., scbert)


%\noindent\textbf{Future work.}
% Our work is an extension of the model we implemented in the NeurIPS 2022 competition. Now our framework of multimodal transformers with the cross-modality heterogeneous graph goes far beyond the specific downstream task of modality prediction, and there are lots of potentials to be further explored. Our graph contains three types of nodes. while the cell embeddings are used for predictions, the remaining protein embeddings and gene embeddings may be further interpreted for other tasks. The similarities between proteins may show data-specific protein-protein relationships, while the attention matrix of the gene transformer may help to identify marker genes of each cell type. Additionally, we may achieve gene interaction prediction using the attention mechanism under adequate regulations. We expect \method{} to be capable of much more than just modality prediction. Note that currently, we fuse information from different transformers with message-passing GNNs. To extend more on transformers, a potential next step is implementing cross-attention cross-modalities. Ideally, all three types of nodes, namely genes, proteins, and cells, would be jointly modeled using a large transformer that includes specific regulations for each modality. The self-attention within each modality would reconstruct the prior interaction network, while the cross-attention between modalities would be supervised by the data observations. Then, The attention matrix will provide insights into all the internal interactions and cross-relationships. With the linearized transformer, this idea would be both practical and versatile.

% \begin{acks}
% This research is supported by the National Science Foundation (NSF) and Johnson \& Johnson.
% \end{acks}

    \section{Reproducibility statement}
    %! TEX root = main.tex

In our work, we strive for achieving reproducibility of the results, and all the code we developed for this research is available publicly on GitHub under an open-source license, while all the data is available in open archival repositories.
PetIBM is an open-source CFD library based on the immersed boundary method, and is available at \url{https://github.com/barbagroup/PetIBM} under the permissive BSD-3 license. 
The software was peer reviewed and published in the Journal of Open Source Software \cite{chuang_petibm_2018}. 
Our PINN solvers based on the NVIDIA \emph{Modulus} toolkit can be found following the links in the GitHub repository for this paper, located at \url{https://github.com/barbagroup/jcs_paper_pinn/}. 
There, the folder prefixed by \texttt{repro-pack} corresponds to a git submodule pointing to the relevant commit on a branch of the repository for the full reproducibility package of the first author's PhD dissertation \cite{chuang_thesis_2023}.
The branch named \texttt{jcs-paper} contains the modified plotting scripts to produce the publication-quality figures in this paper.    
A snapshot of the repro-pack is archived on Zenodo, and the DOI is 10.5281/zenodo.7988067.
As described in the README of the repro-pack, readers can use pre-generated data for plotting the figures in this paper, or they can re-run the solutions using the code and data available in the repro-pack.
The latter option is of course limited by the computational resources available to the reader.
For the first option, the reader can find the raw data in a Zenodo archive, with DOI: 10.5281/zenodo.7988106.
To facilitate reproducibility of the computational environment, we execute all cases using Singularity/Apptainer images for both the PetIBM and PINN cases. 
All the container recipes are included in the repro-pack under the \texttt{resources} folder. 
The \emph{Modulus} toolkit was open-sourced by NVIDIA in March 2023,\footnote{\url{https://developer.nvidia.com/blog/physics-ml-platform-modulus-is-now-open-source/}} under the Apache License 2.0.
This is a permissive license that requires preservation of copyright and license notices and provides an express grant of patent rights. 
When we started this research, \emph{Modulus} was not yet open-source, but it was publicly available through the conditions of an End User Agreement. 
Documentation of those conditions can be found via the May 21, 2022, snapshot of the \emph{Modulus} developer website on the Internet Archive Wayback Machine.\footnote{\url{https://web.archive.org/web/20220521223413/https://catalog.ngc.nvidia.com/orgs/nvidia/teams/modulus/containers/modulus}}
We are confident that following the best practices of open science described in this statement provides good conditions for reproducibility of our results. 
Readers can inspect the code if any detail is unclear in the paper narrative, and they can re-analyze our data or re-run the computational experiments.
We spared no effort to document, organize, and preserve all the digital artifacts for this work.


    \section*{Acknowledgement}
    We appreciate the support by NVIDIA, through sponsoring the access to its high-performance computing cluster.

    % bibliography
    \bibliography{references}

  
    \appendix
    \section{Supplement}
    \appendix 
\section*{Supplementary Materials}
\section{Background: Standard ADMM Training of DNNs} \label{sec:admm_nn}

Alternating Direction Method of Multipliers (ADMM) \cite{gabay1975dual,boyd2011distributed} is a class of optimization methods belonging to  \textit{operator splitting techniques} which borrows benefits from both dual decomposition and augmented Lagrangian methods for constrained optimization. %To show the potentials of standard ADMM, we first revisit a general formulation of ADMM in DNN training, similar to those used in prior work. Then, we propose our stochastic block-ADMM in the next subsection.

To formulate training an $L$-layer DNN in a general supervised setting, we would have the following non-convex constrained optimization problem \cite{zeng2018global}:
% \vspace{-0.1in}
\begin{align} \label{eq:obj}
	\minimize_{ \mathcal{W}, \mathcal{A}, \mathcal{Z}} \quad &\mathcal{J}\left(\mY, \mZ_{L} \right) + \sum_{\ell = 1}^{L} \lambda_{\ell}  {\bf r}_{\ell} (\mW_{\ell}) \\
	 {\rm subject~to} \quad & \mA_{\ell} - {\bm \phi}_{\ell } \left( \mZ_{\ell} \right) = {\bf 0}, \quad \ell = 1,\dots, L-1   \nonumber \\
	 {\rm subject~to} \quad & \mZ_{\ell} - \mW_{\ell} \mA_{\ell-1} = {\bf 0}, \quad \ell = 1, \dots , L \nonumber 
\end{align}
where $\mathcal{J}$ is the main objective (\textit{e.g.}, cross-entropy, mean-squared-error loss functions) that needs to be minimized. The subscript $\ell$ denotes the $\ell$-th layer in the network. The optimization variables are $\mathcal{W} = \{ \mW_\ell\}_{\ell=1}^{L}$, $\mathcal{A} = \{ \mA_{\ell}\}_{\ell=1}^{L-1}$, and $\mathcal{Z} = \{ \mZ_{\ell}\}_{\ell=1}^{L}$ where $\mW_\ell$, $\mZ_{\ell}$, $\mA_\ell$, and ${\bm \phi}_\ell (.)$ are the weight matrix, output matrix, activation matrix, and the activation function (\textit{e.g.}, ReLU) at the $\ell$-th layer, respectively. Note that $\mA_{0} = \mX$ where $\mX = \{ \vx_1,\dots, \vx_N \} \in  \R^{M \times N}$ is the input data matrix containing $N$ samples with input dimensionality $M$; $\mY = \{\vy_1,\dots, \vy_N \} \in \R^{C \times N}$ is the target matrix pair comprised of $N$ one-hot vector label of dimension $C$, representing number of prediction classes. Also, ${\bf r(.)}$ is the regularization term with (\textit{e.g.}, Frobenius norm $\|.\|_F^2$) corresponding penalty weight $\lambda_{\ell}$. Note that the regularization term can be simply ignored by setting $\lambda_\ell$ to zero. In this formulation, the intercept in each layer is ignored for simplicity as it can be simply be added by slightly modifying the $\mW_\ell$ and the input to each layer. The formulation in Eq. (\ref{eq:obj}) breaks the the conventional multi-layer backpropagation optimization of DNNs into simpler sub-problems that can be solved efficiently (e.g. reducing to least-squares problem). This also facilitates training in a distributed manner --- as the layers of the DNN are decoupled and the variables can be updated in parallel across layers ($\mW_\ell$) and data points (\ $\mW_\ell, \mZ_\ell, \mA_\ell$).



To enforce the constraints in problem (\ref{eq:obj}) and solve the optimization using ADMM, we would have the following augmented Lagrangian problem:

\begin{eqnarray} \label{eq:augmented}
	\minimize_{ \mathcal{W}, \mathcal{A}, \mathcal{Z}} \quad &\mathcal{J}\left(\mY, \mZ_{L} \right) + \sum_{\ell = 1}^{L} \lambda_{\ell}  {\bf r}_{\ell} (\mW_{\ell}) \\
	& + \sum_{\ell=1}^{L} \frac{\beta_{\ell}}{2} \| \mZ_{\ell} - \mW_{\ell} \mA_{\ell-1} + \mU_{\ell}\|_{F}^{2} \nonumber\\
	& + \sum_{\ell=1}^{L-1} \frac{\gamma_{\ell}}{2} \| \mA_{\ell} - {\bm \phi}_{\ell}(\mZ_{\ell}) + \mV_{\ell}\|_{F}^{2}\nonumber
\end{eqnarray}
where $\beta_{\ell}, \gamma_\ell >0$ are the step sizes, $\mU_{\ell}$ and $\mV_{\ell}$ are the \textit{(scaled) dual variables} \cite{boyd2011distributed} for the equality constraint at the layer $\ell$. 
Algorithm \ref{alg:admm} shows a standard ADMM scheme for optimizing Eq. (\ref{eq:augmented}). Note, the parameters are updated in a closed-form as analytical solution can be simply derived. For simplicity of the equations, we denote $\gP_\ell (.) = \frac{\beta_{\ell}}{2} \| \mZ_{\ell} - \mW_{\ell} \mA_{\ell-1} + \mU_{\ell}\|_{F}^{2} $ and $\gQ_\ell (.) = \frac{\gamma_{\ell}}{2} \| \mA_{\ell} - {\bm \phi}_{\ell}(\mZ_{\ell}) + \mV_{\ell}\|_{F}^{2}$. This algorithm is similar to \cite{taylor2016training,wang2019admm} with the difference that all the equality constraints in problem (\ref{eq:obj}) are enforced using multipliers, while previous work only enforced the constraints on the last layer $L$ while other constraints were only loosely enforced using quadratic penalty. 

\begin{algorithm}[htb]
  \caption{Standard ADMM for DNN Training}
  \label{alg:admm}
\begin{algorithmic}
  {\STATE \scalebox{1}{\bfseries Input:} data $\mX$, labels $\mY$}
  \STATE  \scalebox{1}{{\bfseries Params:} $\beta_\ell >0, \gamma_\ell >0,\lambda_\ell > 0$ }
  \STATE  \scalebox{0.8}{{\bfseries Initialize:} $\{\mW_\ell^0\}_{\ell=1}^{L}, \{ \mU_\ell^0\}_{\ell=1}^{L}, \{ \mV_\ell^0\}_{\ell=1}^{L-1}, \{\mZ^0_\ell\}_{\ell=1}^{L}, \{\mA^0_\ell\}_{\ell=1}^{L-1}\; k \leftarrow 0$ }
  \REPEAT
  \FOR{$\ell=1$ {\bfseries to} $L$}
  \STATE \scalebox{1}{$\mW_\ell^{k+1} \leftarrow \argmin\; \{ \gP_\ell (.) +  \lambda_{\ell}  {\bf r}_{\ell} (\mW_{\ell}^{k})\}$}
  \ENDFOR
  \FOR{$\ell=1$ {\bfseries to} $L-1$}
  \STATE \scalebox{1}{ $\mZ_\ell^{k+1} \leftarrow \argmin\; \{ \gP_\ell (.) +  \gQ_\ell (.) \}$ }
  \STATE \scalebox{1}{$\mA_\ell^{k+1} \leftarrow \argmin\; \{ \gP_{\ell+1} (.) +  \gQ_\ell (.) \} $}
  \ENDFOR
    \STATE \scalebox{1}{ $\mZ_{L}^{k+1} \leftarrow \argmin\; \{ \mathcal{J}\left(\mY, \mZ_{L}^{k} \right) + \gP_L (.) \}$ }
  \FOR{$\ell=1$ {\bfseries to} $L-1$}
  \STATE \scalebox{1}{$\mU_\ell^{k+1} \leftarrow \mU_\ell^{k} + \mZ_{\ell}^{k+1} - \mW_{\ell}^{k+1} \mA_{\ell-1}^{k+1}$}
  \STATE \scalebox{1}{$\mV_\ell^{k+1} \leftarrow \mV_\ell^{k} + \mA_{\ell}^{k+1} - {\bm \phi}_{\ell}(\mZ_{\ell}^{k+1})$}
  \ENDFOR
  \STATE \scalebox{1}{$\mU_L^{k+1} \leftarrow \mU_L^{k} + \mZ_{L}^{k+1} - \mW_{L}^{k+1} \mA_{L-1}^{k+1}$}
  \UNTIL{some stopping criterion is reached.}
\end{algorithmic}
\end{algorithm}


While the standard ADMM Algorithm \ref{alg:admm} has potentials in training (simple) DNNs \cite{taylor2016training}, there exists hurdles that confines extending ADMM to more complex problems --- the global convergence proof of the ADMM \cite{deng2016global} assumes that $\mathcal{J}$ is deterministic and the global solution is calculated at each iteration of the cyclic parameter updates.
% and during each iteration of the cyclic parameter updates, all the data samples are visited.
This makes standard ADMM computationally expensive thus impractical for training of many large-scale optimization problems. Specifically, for  deep learning, this would impose a severe restriction on training set size when limited computational resources are available. In addition, since the variable updates in standard ADMM are analytically driven, the extent of its applications is limit to trivial tasks \cite{taylor2016training}, making it incompetent to perform on par with the recent complex architectures introduced in deep learning (e.g. \cite{he2016deep}).


\section{Proof for Proposition 1}\label{sec:proof}

We follow the steps in the proof for similar problems in \cite{fu2018anchor} and \cite{shi2017penalty} with deterministic primal updates. Proper modifications are made to cover the stochastic primal update in our proof.


Note that we have
              \[     \nabla{\cal L}_{\rho_k}(\X^k)= \nabla f(\X^k) + \nabla h(\X^k)^T\bm \mu^k,          \]
              where 
              \[      \bm \mu^k = (1/\rho_k)h(\bm X^k)+\bm \lambda^k.   
              \]
              Our first step is to show that $\{\bm \mu^k\}$ is a convergent sequence. To see this, we define 
              \[ \bm \bar{\bm \mu}^k = \frac{\bm \mu^k}{\|{\bm \mu}^k\|}. \]
              Since $\bm \bar{\bm \mu}^k$ is bounded, it converges to a limit point $\bm \bar{\bm \mu}$. Also let $\x^\star$ be a limit point of $\x^k$.
              Because we have assumed that 
              $$\varepsilon_k\rightarrow 0,\quad \sigma_k^2\rightarrow 0,$$ 
              it means that the mean and variance of the stochastic gradient of our primal update goes to zero.
              Since our stochastic gradient is unbiased, we have
              \[       {\cal G}(\X^k) \rightarrow \nabla {\cal L}_{\rho_k}(\X^\star). \]  
              This also means that  we must have ${\cal G}(\x^k)\rightarrow \bm 0$ and $$\nabla L_{\rho_k}(\bm x^k)\rightarrow \bm 0.$$
     Hence, the following holds when $k\rightarrow \infty$:
              \begin{equation}\label{eq:approxkkt}
                 \nabla L_{\rho_k}(\bm X^\star)=\nabla f(\X^\star)+\nabla h(\X^\star)^T\bm {\bm \mu}^\infty = 0,
              \end{equation}           
               
               
              Suppose $\bm \mu^k$ is unbounded. By dividing \eqref{eq:approxkkt} by the above $\|\bm \mu^k\|$ and considering $k\rightarrow \infty$, we must have 
              \begin{equation}\label{eq:key}
                \nabla h(\X^\star)^T\bm \bar{\bm \mu}= 0,\quad \forall \X.    
              \end{equation}               
              The term $\nabla f(\bm X^\star)/\|\bm \mu\|$ is zero since we assumed $\bar{\bm \mu}$ is unbounded.
              Since $h(\bm X)=\bm 0$ satisfies the Robinson's condition, then, for any $\bm w$, there exists $\beta>0$ and $\bm x$ such that
              \[      \bm w = \beta \nabla h(\X^\star)(\X-\X^\star).        \]
              This together with \eqref{eq:key} says that $\bar{\bm \mu}=\bm 0$. This contradicts to the fact $\|\bar{\bm \mu}\|=1$. Hence, $\{ \bm \mu^k \}$ must be a bounded sequence and thus admits a limit point. Denote $\bm \mu^\star$ as this limit point, and take limit of both sides of \eqref{eq:approxkkt}. We have:
              \begin{equation}
              \nabla f(\X^\star)+\nabla h(\X^\star)^T\bm \mu^\star= \bm 0,\quad \forall \X.
              \end{equation}
               
              In addition, since $$\rho_k(\bm \mu^k-\bm \lambda^k) = h(\mathbf{\X^k})$$ with $\rho_k \rightarrow 0$ or $\bm \mu_k-\bm \lambda_k \rightarrow 0$ (per our updating rule and $\eta_k\rightarrow 0$), the constraints will be enforced in the limit.      $\mbox{     } \square$   \\
              

% \subsection*{{\uppercase\expandafter{\romannumeral D}. Supervised training on Fashion-Mnist}}\label{fmnist}


% To compare our method with dlADMM \citet{wang2019admm}, we evaluated the performance of our method on the Fashion-MNIST dataset \citep{xiao2017/online} with 60,000 training samples and 10,000 testing samples. We followed the settings in \citet{wang2019admm} by having 2 hidden layers with 1000 neurons each, and Cross-Entropy loss at the final layer. Also, the batch size is set to 128, $\beta_t = 1$, and the updates for $\mZ_t$ and $\Theta_t$ (eq. 6a) are performed 3 times at each epoch. Figure \ref{fig:fmnist_acc} shows the test set accuracy results over 200 epochs of training. It can be noticed that Stochastic Block ADMM is converging at lower epochs and reaching a higher test accuracy while performing efficient mini-batch updates. Further, in section C., it will be demonstrated that Stochastic Block ADMM converges drastically faster than dlADMM in terms of wall clock time.

   
% \begin{figure}[ht]
% \begin{center}
% \centerline{
% \includesvg[width=\columnwidth]{img/fmnist_acc.svg}
% }
% \caption{Test accuracy comparison of Stochastic Block ADMM and dlADMM \citep{wang2019admm} on Fashion-MNIST dataset using a network with 3 fully-connected layers: $784-1000-1000-10$. Final test accuracy: "Stochastic Block ADMM": $\bf 90.39\%$, "Wang \etal":$84.67 \%$ (averaged over 5 runs).}
% \vskip -0.25in
% \label{fig:fmnist_acc}
% \end{center}
% \end{figure}


\begin{figure}[ht]
\begin{center}
\centerline{
\includegraphics[width=\columnwidth]{imgs/fmnist_acc.pdf}
}
\caption{Test accuracy comparison of Stochastic Block ADMM and dlADMM on Fashion-MNIST dataset using a network with 3 fully-connected layers: $784-1000-1000-10$. Final test accuracy: "Stochastic Block ADMM": $\bf 90.39\%$, "Wang \textit{et al.}":$84.67 \%$ (averaged over 5 runs).}
% \vskip -0.25in
\label{fig:fmnist_acc}
\end{center}
\end{figure}




%----------------------------
\section{Supervised training of DNNs}\label{sec:sup_train}

\textbf{Fashion-MNIST.}
To compare our method with dlADMM \cite{wang2019admm}, we evaluated the performance of our method on the Fashion-MNIST dataset \cite{xiao2017/online} with 60,000 training samples and 10,000 testing samples. We followed the settings in \cite{wang2019admm} by having 2 hidden layers with 1000 neurons each, and Cross-Entropy loss at the final layer. Also, the batch size is set to 128, $\beta_t = 1$, and the updates for $\mZ_t$ and $\Theta_t$ (eq. 6a) are performed 3 times at each epoch. Figure \ref{fig:fmnist_acc} shows the test set accuracy results over 200 epochs of training. It can be noticed that Stochastic Block ADMM is converging at lower epochs and reaching a higher test accuracy while performing efficient mini-batch updates. Further, in section C., it will be demonstrated that Stochastic Block ADMM converges drastically faster than dlADMM in terms of wall clock time.



\textbf{CIFAR-10.}
The previous works on training deep netowrks using ADMM have been limited to trivial networks and datasets (e.g. MNIST) \cite{taylor2016training,wang2019admm}. However, our proposed method does not have many of the existing restrictions and assumptions in the network architecture, as in previous works do, and can easily be extended to train non-trivial applications. It is critical to validate stochastic block-ADMM in settings where deep and modern architectures such as deep residual networks, convolutional layers, cross-entropy loss function, etc., are used. To that end, we validate the ability of our method is a supervised setting (image classification) on the CIFAR-10 dataset \cite{cifar} using ResNet-18 \cite{he2016deep}. To best of our knowledge, this is the first attempt of using ADMM for training complex networks such as ResNets. 


For this purpose, we used 50,000 samples for training and the remaining 10,000 for evaluation. 
To have a fair comparison, we followed the configuration suggested in \cite{gotmare2018decoupling} by converting Resnet-18 network into two blocks $(T=2)$, with the splitting point located at the end of {\sc conv3\_x} layer. We used the Adam optimizer to update both the blocks and the decoupling variables with the learning rates of $\eta_t = 5e^{-3}$ and $\zeta_t = 0.5$. We noted since the auxiliary variables $\mZ_t$ are not "shared parameters" across data samples, they usually require a higher learning rate in Algorithm \ref{alg:blockadmm}. Also, we found the ADMM step size $\beta_t = 1$ to be sufficient for enforcing the block's coupling. 


Figure. \ref{fig:cifar} shows the results from our method compared with two baselines: \cite{gotmare2018decoupling}, and conventional end-to-end neural network training using back-propagation and SGD. Our algorithm consistently outperformed ~\cite{gotmare2018decoupling} however cannot match the conventional SGD results. There are several factors that we hypothesize that might have contributed to the performance difference: 1) in a ResNet the residual structure already partially solved the vanishing gradient problem, hence SGD/Adam performs significantly better than a fully-connected version; 
% 2) The common data augmentation in CIFAR will end up sending a different training example to the optimization algorithm at each iteration, which does not seem to affect SGD but seem to affect ADMM convergence somewhat; 
2) we noticed decreasing the learning rate for $\Theta_t$ updates does not impact the performance as it does for an end-to-end back-propagation using SGD. Still, we obtained the best performance of ADMM-type methods on both MNIST and CIFAR datasets, showing the promise of our approach.
% As illustrated, ADMM gets to a good performance fast and then slowly progress to higher accuracy..


%---------------------------- fig cifar  ------------------------------

\begin{figure}[htb]
% \vskip 0.15in
\begin{center}
\centerline{
\includegraphics[width=\columnwidth]{imgs/cifar.pdf}
}
% \vskip -0.05in
\caption{Test set accuracy on CIFAR-10 dataset. Final accuracy "Block ADMM": $89.66\%$, "Gotmare \etal":$87.12 \%$, "SGD": $\bf 92.70\%$. (Best viewed in color.)}
\label{fig:cifar}
\end{center}
% \vskip -0.2in
\end{figure}

 
 
 
% \subsection*{{\uppercase\expandafter{\romannumeral C}. Wall Clock Time Comparison}} \label{time_cmp}

% In this section, we setup a experiment to further analyse the efficiency of Stochastic Block ADMM and compare its training wall clock time against other baselines: \citet{gotmare2018decoupling,zeng2018global} (BCD), and \citet{wang2019admm} (ADMM). 
% For this purpose, we follow the similar settings as in section 4.1 for a supervised Deep Neural Network (DNN) training over MNIST dataset. Figure \ref{fig:time} shows the test set accuracy v.s. the training wall clock time from different methods. All the experiments are run on a machine with a single NVIDIA GeForce RTX 2080 Ti GPU. The methods are implemented in PyTorch framework -- except for dlADMM \citep{wang2019admm} that is implemented\footnote{code taken from \url{https://github.com/xianggebenben/dlADMM}} in "cupy", a NumPy-compatible matrix library accelerated by CUDA. \citet{gotmare2018decoupling} and Stochastic Block ADMM are trained with a mini-batch size of 128 and \citet{zeng2018global,wang2019admm} are trained in a batch setting. Note that in Figure \ref{fig:time}, the time recorded merely shows the \emph{training time} and excludes the time taken for initialization, data loading, etc. It can be observed that \citet{gotmare2018decoupling} and dlADMM are showing much slower convergence behaviors than Stochastic Block ADMM. We speculate that enforcing all the constraints by dual variables along with the efficient and cheap mini-batch updates in our method highly contributes to the convergence speed as well as its performance superiority over the other methods, including \citet{zeng2018global}.


% \begin{figure}[ht]
% \begin{center}
% \centerline{
% \includesvg[width=\columnwidth]{img/time_comparison.svg}
% }
% \caption{Test set accuracy v.s. training wall clock time comparison of different alternating optimization methods for training DNNs on MNIST dataset. Our method (blue) shows superior performance while presenting comparable convergence speed against \citet{zeng2018global} (green).}
% \vskip - 0.15in
% \label{fig:time}
% \end{center}
% \end{figure}


\begin{table*}[htb]
\caption{Prediction accuracy (\%) of individual attributes in LFWA dataset. DeepFacto with other weakly-supervised and supervised baselines.}
\label{table:attr_lfw}
\vskip 0.15in
\begin{center}
\begin{small}
\begin{sc}
\begin{tabular}{lcccccc}
\toprule
{Attributes} & \multicolumn{3}{c}{\small DeepFacto} & \small \cite{liu2015deep} & \small \cite{liu2018exploring} & \small \cite{zhang2014panda}\\
 {} & \multicolumn{3}{c}{\tiny (Weakly-Supervised)} & {\tiny (Weakly-Supervised)} & {\tiny (Supervised)} & {\tiny (Supervised)} \\
 {} & $r= $256 & 32 & 4 \\
\midrule
‘5 o Clock Shadow’ & 83.3 & 80.0 & 68.7 & 78.8 & \bf84 & \bf84\\
‘Arched Eyebrows’ & \bf86.6 & 83.9 & 79.2 & 78.1 & 82 & 79\\
‘Attractive’ & \bf84.3 & 79.8 & 73.3 & 79.2 & 83 & 81\\
‘Bags Under Eyes’ & \bf83.9 & 72.5 & 64.5 & 83.1 & 83 & 80 \\
‘Bald’ & \bf94.3 & 93.3 & 89.3 & 84.8 & 88 & 84\\
‘Bangs’ & \bf93.2 & 88.4 & 84.4 & 86.5 & 88 & 84\\
‘Big Lips’ & \bf83.2 & 77.0 & 71.9 & 75.2 & 75 & 73\\
‘Big Nose' & 80.1 & 68.7 & 61.4 & \bf81.3 & 81 & 79\\
‘Black Hair’ & \bf92.7 & 91.4 & 87.4 & 87.4 & 90 & 87\\
‘Blond Hair’ & \bf97.9 & 97.3 & 93.2 & 94.2 & 97 & 94\\
‘Blurry’ & \bf90.4 & 90.5 & 86.5 & 78.4 & 74 & 74\\
‘Brown Hair’ & \bf78.4 & 74.4 & 70.2 & 72.9 & 77 & 74\\
‘Bushy Eyebrows’ & \bf84.0 & 78.6 & 63.4 & 83.0 & 82 & 79\\
‘Chubby’ & \bf80.5 & 75.2 & 71.1 & 74.6 & 73 & 69\\
‘Double Chin’ & \bf86.0 & 77.9 & 72.3 & 80.2 & 78 & 75\\
‘Eyeglasses’ & 94.3 & 89.6 & 84.8 & 89.5 & \bf95 & 89\\
‘Goatee’ & \bf89.1 & 85.4 & 80.0 & 78.6 & 78 & 75\\
‘Gray Hair’ & \bf91.9 & 90 & 85.6 & 86.9 & 84 & 81\\
‘Heavy Makeup’ & \bf96.3 & 91.5 & 87.4 & 94.5 & 95 & 93\\
‘High Cheekbones’ & \bf90.4 & 79.0 & 72.1 & 88.8 & 88 & 86\\
‘Male’ & 81.3 & 76.6 & 70.5 & \bf94.3 & 94 & 92\\
‘Mouth Slightly Open’ & \bf85.4 & 78.0 & 73.3 & 81.7 & 82 & 78 \\
‘Mustache’ & \bf96.6 & 93.2 & 91.3 & 83.3 & 92 & 87\\
‘Narrow Eyes’ & \bf78.3 & 69.3 & 58.4 & 77.5 & 81 & 73\\
‘No Beard’ & \bf79.5 & 73.0 & 65.5 & 77.7 & 79 & 75\\
‘Oval Face’ & \bf80.6 & 73.2 & 66.1 & 78.7 & 74 & 72\\
‘Pale Skin’ & 75.1 & 66.7 & 60.6 & \bf89.8 & 84 & 84\\
‘Pointy Nose'& \bf81.6 & 73.7 & 62.2 & 79.8 & 80 & 76\\
‘Receding Hairline’ & 84.0 & 80.9 & 73.8 & \bf88.0 & 85 & 84 \\
‘Rosy Cheeks’ & \bf87.3 & 87.4 & 83.4 & 79.9 & 78 & 73\\
‘Sideburns’ & \bf85.4 & 81.5 & 75.8 & 80.5 & 77 & 76\\
‘Smiling’ & \bf92.6 & 78.7 & 69.8 & 92.2 & 91 & 89\\
‘Straight Hair’ & \bf82.8 & 77.0 & 72.1 &  73.6 & 76 & 73\\
‘Wavy Hair’ & 80.4 & 77.0 & 68.3 & \bf81.7 & 76 & 75\\
‘Wearing Earrings’ & \bf95.4 & 91.6 & 87.1 & 89.7 & 94 & 92\\
‘Wearing Hat’ & \bf93.0 & 90.2 & 87.0 & 80.5 & 88 & 82\\
‘Wearing Lipstick’ & \bf95.8 & 92.8 & 89.0 & 91.4 & 95 & 93\\
‘Wearing Necklace’ & \bf93.0 & 89.8 & 85.1 & 84.0 & 88 & 86\\
‘Wearing Necktie’ & \bf79.8 & 75.2 & 70.6 & 78.7 & 79 & 79\\
‘Young’ & \bf91.0 & 88.4 & 84.4 & 79.2 & 86 & 82\\
\midrule
Average & \bf87.0 & 81.4 & 74.8 &  83.1 & 84 & 81\\
\bottomrule
\end{tabular}
\end{sc}
\end{small}
\end{center}
\vskip -0.25in
\end{table*}




\section{Weakly Supervised Attribute Prediction}\label{sec:weakly_sup}


\subsection*{Factorizing the activations}\label{sec:factor_layer} 

With the assumption that the observations are formed by a linear combination of few basis vectors, one can approximate a given matrix $\mX \in \R^{m \times n}$ into a \textit{basis} matrix $\mM \in \R^{m \times r}$ and an \textit{score} matrix $\mS \in \R^{r \times n}$ such that $\mX \approx \mM \mS$ where $r$ is the (reduced) \textit{rank} of the factorized matrices -- commonly $r \ll \min(m, n)$.
Methods such as NMF would restrict the entries of $\mM$ and $\mS$ to be non-negative $(\forall i,j \;  \mM_{ij} \ge 0,\; \mS_{ij} \ge 0)$ which forces the decomposition to be only \textit{additive}. This has been shown to result in a parts-based representation that is intuitively more close to human perception. It is also worth mentioning that obviously, the matrix $\mX$ needs to be positive $({\forall i,j} \;  \mX_{ij} \ge 0)$. For non-negative factorization on the activations of the DNNS, due to the common use of activation functions such as \textit{ReLU}, this would not impose any constraints in most of the problems.

Activations of the CNN networks are generally tensors of the shape $\tZ_{\ell} \in \R^{(N, C, H, W)}$ which namely represent the batch size of the input, the number of the channels, the height of each channel, and the corresponding width. To adapt such tensors for the NMF problem, we reshape the tensor into the matrix $\mZ_{\ell} \in \R^{ C \times (N * H * W)}$ by stacking it over its channels while flattening the other dimensions. This way, the channels would be embedded into a pre-defined small dimension $r$ while keeping each sample and pixels information. For the weakly-supervised problem of attribute classification using DeepFacto, we attached the DeepFacto module to the last convolutional layer of the Inception-Resnet-V1 architecture followed by a \emph{ReLU}. This layer has 1792 channels and, for a given input of the size $160 \times 160$ pixels (the original input size from the LFWA dataset), the height and the width are both equal to 3. 

\begin{figure}[htb]
\vskip -0.05in
\begin{center}
\centerline{
\includegraphics[width=\columnwidth]{imgs/heatmap.jpg}
}
\caption{Heat map visualizations from three different dimensions of the score matrix $\mS$ (rows) trained by DeepFacto-32 over different samples (columns) in LFWA dataset. These dimensions can capture interpretable representations over different faces identities: \emph{eyes} (top), \emph{forehead} (middle), and \emph{nose} (bottom).}
\label{fig:heatmap}
\end{center}
\vskip -0.15in
\end{figure}

% Table \ref{table:attr_lfw} shows the prediction accuracy of each attribute in LFWA dataset and compares DeepFacto with different ranks ($r=4,32,256$) against other supervised and weakly-supervised baselines. It can be noted that our method can generate highly informative representation of the LFWA attributes without accessing their labels. This supports our conjecture that DeepFacto, by non-negatively factorizing the activations of the DNNs in and end-to-end training, can lead to an interpretable decomposition of the DNN activations.



\subsection*{Heat maps}\label{sec:heatmap}
To qualitatively investigate the interpretability of the factorized representations learned from DeepFacto, similar to \cite{collins2018deep}, one can visualize the score matrix $\mS$. Each dimension of the score matrix $\mS$ can be reshaped back to the original activation size and be up-sampled to the size of the input using bi-linear interpolation. In Figure \ref{fig:heatmap}, the score matrix learned form the DeepFacto with $r=32$ (average attribute prediction of 81.4\%) is used where three different heat maps (out of 32) are depicted over different samples from LFWA dataset. We have found $r=4$ to be very low to represent interpretable heat maps for the attributes and $r=256$ to contain redundant heat maps. It can be seen, that the heat maps can show local and persistent attention over different face identities: \emph{eyes}, \emph{forehead}, \emph{nose}, etc.




\end{document}
% vim:ft=tex:
