% Created 2013-10-07 月 18:04
% aaai
% \documentclass[10pt,letterpaper]{article}
% \usepackage{aaai18}
% ecai
% \documentclass{ecai}
% \usepackage{times}
% jair
\documentclass[11pt]{article}
\usepackage{jair}
\usepackage{named}
%%%%%%%%%%%%
% \usepackage{hyperref}
\usepackage{tabularx}
\usepackage{xparse}
\usepackage{times}
\usepackage{helvet}
\usepackage{courier}
\usepackage{amsmath}
\usepackage{amsthm}
\usepackage{amssymb}
\usepackage{xspace}
\usepackage{relsize}
\usepackage{aaai_my}
\usepackage{graphicx}
\usepackage{abbrev}
\usepackage{multirow}
\usepackage{comment}
\usepackage[normalem]{ulem}
\usepackage[frozencache]{minted}
\usepackage{algorithm}
\usepackage[noend]{algorithmic}
\usepackage{url}

%% jair?
\usepackage{fancyhdr}
\fancypagestyle{plain}{%
\fancyhf{} % clear all header and footer fields
\fancyfoot[C]{\iffloatpage{}{\thepage}} % except the center
\renewcommand{\headrulewidth}{0pt}
\renewcommand{\footrulewidth}{0pt}}
\pagestyle{plain}
\usepackage{longtable}
\setlength{\LTcapwidth}{\textwidth}
\setcounter{LTchunksize}{10}

\frenchspacing
\setlength{\pdfpagewidth}{8.5in}
\setlength{\pdfpageheight}{11in}
\pdfinfo{
/Title (Classical Planning in Deep Latent Space: Bridging the Subsymbolic-Symbolic Boundary)
/Keywords (Classical Planning, Deep Learning, Symbol Grounding)
/Author (Masataro Asai, Alex Fukunaga)
}

%%%%%%%%%%%%%%%%%%%%%%%%%
% % forbid the direct use of \\ref --- it results in inconsistency in styles, e.g., Figure,Fig., etc.
% % Macros are for styles!
% % !!!WARN!!! this should be removed in the camera-ready version.
% \def\ref{\todo{Do not use ``ref'' directly!}}
% %% journal, wide margin
% % \renewcommand{\subsubparagraph}[1]{%
% % \vspace{3pt}%
% % \hrule width \linewidth%
% % \marginpar[\hfill\relsize{-1} \protect{{#1}} $\triangleright$]{\relsize{-1} $\triangleleft$ \protect{{#1}}}%
% % \vspace{3pt}%
% % }
% % \setlength{\marginparwidth}{3.3cm}
% % \setlength{\marginparsep}{5pt}
% %% conference, smaller margin
% \renewcommand{\subsubparagraph}[1]{%
% \hrule width \linewidth%
% \marginpar[\hfill\relsize{-1} \protect{\rotatebox[origin=c]{90}{#1}} $\triangleright$]{\relsize{-1} $\triangleleft$ \protect{\rotatebox[origin=c]{-90}{#1}}}}
% % \usepackage{soul}
\makeatletter
\renewcommand{\@todons}[2]{%
  \uline{#1}%
  \marginpar[%
    \hfill\relsize{-1}%
    \protect{%
      \rotatebox[origin=c]{90}{\bf todo: #2}}%
    $\triangleright$]{%
    \relsize{-1} $\triangleleft$%
    \protect{%
      \rotatebox[origin=c]{-90}{\bf todo: #2}}}}
\renewcommand{\@todos}[2]{%
  \uline{#1}%
  \marginpar[%
    \hfill\relsize{-1}%
    \protect{%
      \rotatebox[origin=c]{90}{done \it #2}}%
    $\triangleright$]{%
    \relsize{-1} $\triangleleft$%
    \protect{%
      \rotatebox[origin=c]{-90}{done \it #2}}}}
\makeatother
% \setlength{\columnsep}{0.5em}
%%%%%%%%%%%%%%%%%%%%%%%%%

\hyphenation{bar-man air-port cyber-sec driver-log floor-tile free-cell
no-mystery open-stacks parc-printer path-ways peg-sol pipes-world
no-tankage scan-alyzer soko-ban tidy-bot visit-all zeno-travel tie-break-ing}

%% figure/table caption for journal
% \makeatletter
%   \renewcommand{\thefigure}{%
%   \thesection.\arabic{figure}}
%   \@addtoreset{figure}{section}
%   \renewcommand{\thetable}{%
%   \thesection.\arabic{table}}
%  \@addtoreset{table}{section}
% \makeatother

%% journal title / author
\usepackage{array}
% hidden column type
\newcolumntype{H}{>{\setbox0=\hbox\bgroup}c<{\egroup}@{}}
\newcolumntype{E}{r!{$\pm$}l}

% \jairheading{1}{1993}{1-15}{6/91}{9/91}
\ShortHeadings{Classical Planning in Deep Latent Space (Accepted in AAAI-18, Extended Manuscript)}{Asai \& Fukunaga (Accepted in AAAI-18, Extended Manuscript)}
\firstpageno{1}
% 
\author{
\name Masataro Asai \email guicho2.71828\textcircled{$\alpha$}gmail.com\\
\name Alex Fukunaga \\
\addr Graduate School of Arts and Sciences, University of Tokyo}

%% conference title / author
\title{Classical Planning in Deep Latent Space:\\ Bridging the Subsymbolic-Symbolic Boundary} %title targeting ICAPS-learning people % temporarily removing "deep" to prevent  DL image processing expert reviewers from being  attracted

\begin{document}
% \nocopyright % XXXTODO: camera verision needs copyright -- this is to save some space in the submission.
\maketitle

%%%%%%%%%%%%%%%%%%%%%%%%%%%%%%%%%%%%%%%%%%%%%%%%%%%%%%%%%%%%%%%%
\begin{hidden}

* too much ``''s make the sentence look scattered and visually less recognizable. ``e.g.'' also.

* \em, \bf, \it are all obsolete \TeX primitives, and it does not take effect properly --- for example, {\bf {\it aaa}} shows ``aaa'' in italic but NOT IN BOLD. Use \emph{}, \textit{}, \textbf{} and so on.

* always use \ff, \fd, \cea, \pr, \mv , and do not use it directly, e.g. FF, FD/LAMA2011, etc. 

* use of footnotes should be minimized.

* IPC2011 should always be \ipc . The definition can later be modified in abbrev.sty .

* prefer separated words over hyphened words. domain
  independent>domain-independent, planner independent >
  planner-independent.

* Table, Figure, Fig., should not be used directly. Always use \refig and \reftbl. When the development flag is enabled, direct use of \ref signals an error.

* Caption ends with a period.
\end{hidden}

%%%%%%%%%%%%%%%%%%%%%%%%%%%%%%%%%%%%%%%%%%%%%%%%%%%%%%%%%%%%%%%%

\begin{abstract}
Current domain-independent, classical planners require symbolic models of the problem domain and instance as input, resulting in a knowledge acquisition bottleneck.
Meanwhile, although deep learning has achieved significant success in many fields, the knowledge is encoded in a subsymbolic representation which is incompatible with symbolic systems such as planners.
We propose \latentplanner, an unsupervised architecture combining deep learning and classical planning.
% 
Given only an unlabeled set of image pairs showing a subset of transitions allowed in the environment (training inputs),
and a pair of images representing the initial and the goal states (planning inputs),
\latentplanner finds a plan to the goal state in a symbolic latent space and returns a visualized plan execution.
The contribution of this paper is twofold:
(1) State Autoencoder, which finds a propositional state representation of the environment using a Variational Autoencoder.
It generates a discrete latent vector from the images, based on which a PDDL model can be constructed and then solved by an off-the-shelf planner.
(2) Action Autoencoder / Discriminator, a neural architecture which jointly finds the action symbols and the implicit action models (preconditions/effects),
and provides a successor function for the implicit graph search.
We evaluate \latentplanner using image-based versions of 3 planning domains: 8-puzzle, Towers of Hanoi and LightsOut.
\end{abstract}


\section*{Note}

\textbf{This is an extended manuscript of the paper accepted in AAAI-18.
The contents of AAAI-18 submission itself is significantly extended from what has been published in
Arxiv, KEPS-17, NeSy-17 or Cognitum-17 workshops.
Over half of the paper describing (2) is new.
Additionally, this manuscript contains the contents in the supplemental material of AAAI-18 submission.
% This contains a section (\refsec{sec:heuristics-evaluation}) that was later regarded by the reviewers as of significant value.
These implementation/experimental details are moved to the Appendix.
}

\subsection*{Note to the ML / deep learning researchers}

This article
combines the Machine Learning systems and the classical, logic-based symbolic systems.
Some readers may not be familiar with NNs and related fields like you are, thus
we include very basic description of the architectures and the training methods.

\section{Introduction}

\label{sec:introduction}

% An intro for IJCAI targeting ICAPS people.
Recent advances in domain-independent planning have greatly enhanced their capabilities.
%Using admissible heuristics, search-based planners can find cost-optimal solutions to many problems.
However, planning problems need to be provided to the planner in a structured, symbolic representation such as PDDL \cite{McDermott00}, and in general, such symbolic models need to be provided by a human, either directly in a modeling language such as PDDL, or via a compiler which transforms some other symbolic problem representation into PDDL.
This results in the {\it knowledge-acquisition bottleneck}, where the modeling step is sometimes the bottleneck in the problem-solving cycle.
In addition, the requirement for symbolic input poses a significant obstacle to applying planning in {\it new, unforeseen} situations where no human is available to create such a model or a generator, e.g., autonomous spacecraft exploration.
In particular this first requires generating symbols from raw sensor input, i.e., the {\it symbol grounding problem} \cite{Steels2008}.

Recently,  significant advances have been made in neural network (NN) deep learning approaches for perceptually-based cognitive tasks including image classification \cite{deng2009imagenet}, object recognition \cite{ren2015faster}, speech recognition \cite{deng2013new}, machine translation
 % game playing \cite{dqn} and human-competitive Go-playing. % try deferring mentioning these later
as well as  NN-based problem-solving systems \cite{dqn,neuraltm}.
However, the current state-of-the-art, pure NN-based systems do not yet provide guarantees provided by symbolic planning systems, such as deterministic completeness and solution optimality.

% \cite{bieszczad1998neurosolver, bieszczad2015neurosolver}

Using a NN-based perceptual system to 
{\it automatically} provide input models for domain-independent planners could greatly expand the applicability of planning technology and offer the benefits of both paradigms.
\emph{We consider the problem of robustly,  automatically bridging the gap between such subsymbolic representations and the symbolic representations required by domain-independent planners}.

\refig{fig:mandrill-intro} (left) shows a scrambled, 3x3 tiled version of the photograph on the right, i.e., an image-based instance of the 8-puzzle.
Even for humans, this photograph-based task is arguably more difficult to solve than the standard 8-puzzle because of the distracting visual aspects.
We seek a domain-independent system which, given only a set of unlabeled images showing the valid moves for this image-based puzzle, finds an optimal solution to the puzzle.
Although the 8-puzzle is trivial for symbolic planners, solving this image-based problem with 
a domain-independent system which (1)  \emph{has no prior assumptions/knowledge}
 (e.g., ``sliding objects'', ``tile arrangement''), and (2) \emph{must acquire all knowledge from the images}, is nontrivial.
Such a system should not make assumptions about the image (e.g., ``a grid-like structure'').
The only assumption allowed about the nature of the task is that it can be modeled as a classical planning problem (deterministic and fully observable).


\begin{figure}[tbp]
 \centering
 \includegraphics[width=\linewidth]{img/mandrill-intro-new.pdf}
 \caption{An image-based 8-puzzle.}
% Example planning input to the \latentplanner system, and its visual depiction for readers. Note that there are no explicit labels that indicate the size or the position of each ``tile'', nor the very notion that the image is divided into such tiles.}
 \label{fig:mandrill-intro}
\end{figure}


We propose Latent-space Planner (\latentplanner), an architecture which completely automatically generates a 
symbolic problem representation from the subsymbolic input, which can be used as the input for a classical planner.
%\todo{``implementation'' vs. ``system'' vs. ``Architecture'':  It's always tricky deciding when to use ``system'' vs. ``architecture''.``Architecture'' is a high levelset of constraints upon a system.  and an ``implementation'' is the concrete program which implements a system. Architectures is unambiguously higher-level than implementation. In this paper, we should emphasize that the architecture is quite novel and powerful,  although the current implementation has some limitations.}
\latentplanner consists of 3 components: (1) a NN-based {\it State Autoencoder} (SAE), which provides a bidirectional mapping between the raw images of the environment and its propositional representation, 
(2) an {\it action model acquisition} (AMA) system which grounds the action symbols and learns the action model,
and (3) a symbolic planner. 
Given only a set of {\it unlabeled images} of the environment, and in an unsupervised manner,
we train the SAE and AMA to generate its symbolic representation.
Then, given a planning problem instance as a pair of initial and goal images such as \refig{fig:mandrill-intro}, \latentplanner 
uses the SAE to map the problem to a symbolic planning instance, invokes a planner, then visualizes the plan execution.
We evaluate \latentplanner using image-based versions of the 8-puzzle, LightsOut, and Towers of Hanoi domains.

\section{Background}
\label{sec:background}

% As this work bridges the gap between the symbolic methods and sub-symbolic methods, we provide both backgrounds for the sake of both symbolic and subsymbolic sides of researchers.
% 


\subsection{Classical Planning}

Classical Planning is achieving a significant advance in the recent years due to the success of heuristic search.
The input problem to a Classical Planning solver (a \emph{planner}) is a 5-tuple 
$\Pi=\brackets{P,O,I,G,A}$ where $P$ defines a set of first-order predicates, $O$ is a set of symbols called \emph{objects}, $I$ is the initial state, $G$ is a set of goal conditions, and $A$ is a set of actions which defines the state transitions in the search space.
A state is an assignment of boolean values to the set of propositional variables, while a condition is a partial assignment that assigns values only to a subset of propositions.
Each proposition is an instantiation of a predicate with objects.
% 
% 
Lifted action schema $a \in A$ is a 5-tuple
$\brackets{\mbox{\textit{params}}, \mbox{\textit{pre}}, e^+, e^-, c}$
where each element means the set of parameters, preconditions,
add-effects, delete-effects and the cost, respectively.
% Precondition defines a situation when the action is applicable. Add/delete-effects define how the action modifies the state by adding/deleting propositions.
% In most cases, an action and its elements are \emph{lifted} (parameterized) by \textit{params}.
Parameter substitution using objects in $O$ instantiates \emph{ground actions}.
When $c$ is not specified, it is usually assumed $c=1$.
These inputs are described in a PDDL modeling language \cite{bacchus2000} and its extensions.

\refig{8puzzle-pddl} shows one possible representation of a state in
3x3 sliding tile puzzle (8-puzzle) in the First Order Logic formula, and the
representation of the same state using PDDL.

\begin{figure}[htb]
\centering
\begin{minipage}[c]{0.2\linewidth}
 \begin{align*}
        & Empty(x_0, y_0)          \\
  \land & At   (x_1, y_0, panel_6) \\
  \land & Up   (y_0, y_1)          \\
  \land & Down (y_1, y_0)          \\
  \land & Right(x_0, x_1)          \\
  \land & Left (x_1, x_0)       \ldots 
 \end{align*}
\end{minipage}
\begin{minipage}[c]{0.4\linewidth}
 \begin{minted}[]{common-lisp}
  (empty x0 y0)
  (at    x1 y0 panel6)
  (up    y0 y1)
  (down  y1 y0)
  (right x0 x1)
  (left  x1 x0)...
 \end{minted}
\end{minipage}
\begin{minipage}[c]{0.3\linewidth}
 \includegraphics{img/pddl/8puzzle-standard.pdf}
\end{minipage}
\caption{One possible state representation of a
3x3 sliding tile puzzle (8-puzzle) in the first order logic formula and its
corresponding PDDL notation. It contains predicate symbols 
\pddl{empty}, \pddl{up}, \pddl{down}, \pddl{left}, \pddl{right}, \pddl{at} as well as
object symbols such as \pddl{x}$_i$, \pddl{y}$_i$, \pddl{panel}$_j$ for $i \in \braces{0..3}$ and $j\in \braces{1..8}$.
}
\label{8puzzle-pddl}
\end{figure}

\begin{figure}[htb]
\begin{minipage}[c]{0.19\linewidth}
 When $Empty(x, y_{old}) \land at(x, y_{new}, p) \land up(y_{new}, y_{old})$ ;
 
 then $\lnot Empty(x,y_{old}) \land Empty(x,y_{new}) \land \lnot at(x, y_{new}, p) \land at(x, y_{old}, p)$
\end{minipage}
\begin{minipage}[c]{0.56\linewidth}
 \begin{minted}[]{common-lisp}
 ;; Translates to a PDDL model below:
 (:action slide-up ...
  :precondition
  (and (empty ?x ?y-old)
       (at ?x ?y-new ?p) ...)
  :effects
  (and (not (empty ?x ?y-old))
       (empty ?x ?y-new)
       (not (at ?x ?y-new ?p))
       (at ?x ?y-old ?p)))
 \end{minted}
\end{minipage}
\begin{minipage}[c]{0.24\linewidth}
 \includegraphics{img/pddl/8puzzle-standard-tile7.pdf}
\end{minipage}
\caption{One possible action representation of sliding up a tile in 3x3
sliding tile puzzle in (left) the first order logic formula and (middle)
its corresponding PDDL notation. In addition to
\refig{8puzzle-pddl}, it further contains an action symbol
\pddl{slide-up}.  } \label{8puzzle-action-pddl}
\end{figure}


The task of a planning problem is to find a path from the initial state
$I$ to some goal state $s^*\supseteq G$, using the state transition
rules in $A$. A state $s$ can be transformed into a new state $t$ by
applying a ground action $a$ when $s\supseteq \mbox{\textit{pre}}$, and
then $t=(s\setminus e^-)\cup e^+$ \cite{bacchus2000}. This transition
can also be viewed as applying a state transition function $a$ to $s$,
which can be written as $t=a(s)$.

% Finally, a \emph{plan}  $P$ is a sequence of ground actions
% $\brackets{a_0,a_1\ldots a_n}$, which, when applied to $I$, results in a
% goal state $s^*\supseteq G$.\todo*{seems that defining ``path'' was unnecessary(?)}
% Finally, a path is a sequence of ground actions $\brackets{a_0,a_1\ldots
% a_n}$. When a path leads to some goal $s^*$ from $I$, it is called a
% \emph{plan} of $\Pi$.

\sota planners solve this problem as a path finding problem on a
implicit graph defined by the state transition rules. They usually
employ forward state space heuristic search, such as \astar (for finding
the shortest path) or Greedy Best-First Search (for finding a suboptimal
path more quickly). Thanks to the variety of successful
domain-independent heuristic functions
\cite{Helmert2009,sievers2012efficient,helmert2007flexible,bonet2013admissible,hoffmann01,Helmert04,richter2008landmarks},
current \lsota planners can scale to larger problems which requires to
find a plan consisting of more than 1000 steps \cite{Asai2015}.

% To name a few heuristics, they are (admissible) LM-cut heuristics \cite{Helmert2009}, Pattern Database (PDB) heuristics \cite{sievers2012efficient}, Merge-and-shrink heuristics \cite{helmert2007flexible} or LP-relaxation heuristics \cite{bonet2013admissible}, or (inadmissible) FF heuristics \cite{hoffmann01}, Causal Graph heuristic \cite{Helmert04} and Landmark-Count heuristics \cite{richter2008landmarks}.



\subsection{Knowledge Acquisition Bottleneck}

%The utiolity of the symbolic models in a language such as PDDL has been under debate. % "utility" is usefulness, and I don't think we're arguing the models are useful or not, but instead whether it's practical to assume such models can be generated
While ideally, symbolic models like \refig{8puzzle-pddl} should be learned/generated by the machine itself,
% Researchers longed for such a system but currently no such practical systems exist.
in practice, they must be hand-coded by a human, resulting in the 
% , which harms the applicability of the system
% 
so-called Knowledge Acquisition Bottleneck \cite{cullen88}, 
which refers to the excessive cost of human involvement in converting real-world problems into inputs for symbolic AI systems.

In order to fully automatically acquire symbolic models for Classical Planning, 
\textbf{Symbol Grounding} and \textbf{Action Model Acquisition} (AMA) are necessary.
%
\textbf{Symbol Grounding} is an unsupervised process of establishing a mapping
from huge, noisy, continuous, unstructured inputs
to a set of compact, % clean
discrete, identifiable (structured) entities, i.e., symbols.
For example, PDDL has six kinds of symbols: Objects, predicates, propositions, actions, problems and domains (\reftbl{tab:type-of-symbols}).
Each type of symbol requires its own mechanism for grounding.
For example, the large body of work in the image processing community on recognizing 
objects (e.g. faces) and their attributes (male, female) in images, or scenes in videos (e.g. cooking)
can be viewed as corresponding to grounding the object, predicate and action symbols, respectively.

\begin{table}[tbp] 
\centering
\begin{tabular}{ll}
Types of symbols & \\
\hline
Object symbols    & \textbf{panel7, x\(_{\text{0}}\), y\(_{\text{0}}\)} \ldots{}               \\
Predicate symbols & (\textbf{empty} ?x ?y) (\textbf{up} ?y\(_{\text{0}}\) ?y\(_{\text{1}}\))   \\
Propositions      & \textbf{empty\(_{\text{5}}\)} = (empty x\(_{\text{2}}\) y\(_{\text{1}}\)) (6th application) \\
Action symbols    & (\textbf{slide-up} panel\(_{\text{7}}\) x\(_{\text{0}}\) y\(_{\text{1}}\)) \\
Problem symbols   & \textbf{eight-puzzle-instance1504}, etc.                                   \\
Domain  symbols   & \textbf{eight-puzzle}, \textbf{hanoi}                                      \\
\hline
\end{tabular}
\caption{6 types of symbols in a PDDL definition.}
\label{tab:type-of-symbols}
\end{table}

In contrast, an \textbf{Action Model} is a % description of  %%--- dangerous --- we did not describe AMA2
symbolic/subsymbolic data structure representing the causality in the transitions of the world,
% is limited to a single action step and a single agent, and 
which, in PDDL, consists of preconditions and effects (\refig{8puzzle-pddl}). %, e.g. when / how cooking happens.
%In order to obtain the PDDL model automatically, the system should perform both.
In this paper, we focus on propositional and action symbols, as well as AMA, leaving first-order symbols (predicates, objects) as future work.
% None of these symbols are lifted i.e. we do not derive \emph{1st-order logic} symbols.


\subsection{Action Model Acquisition (AMA) Methods}

Existing methods require symbolic
or near-symbolic, structured inputs. ARMS \cite{YangWJ07}, LOCM
\cite{CresswellMW13}, and \citeauthor{MouraoZPS12} (\citeyear{MouraoZPS12})
assume the action, object, predicate symbols.
% 
Framer \cite{lindsay2017framer} parses natural language texts and emits PDDL,
but requires a clear grammatical structure and word consistency. %, i.e. more symbolic.

\citeauthor{KonidarisKL14} generated PDDL from
% a low-level, sensor actuator space of an agent characterized as a
semi-MDP (\citeyear{KonidarisKL14}).
% The semi-MDP input $\brackets{S,O,R,P}$ consists of a state space $S$, a finite set of \emph{options} $O$,
% the reward $R$ and the probability $P$ of executing an option $o\in O$.
% Options in semi-MDP are ``temporally extended actions''\shortcite{KonidarisKL14},
% and are analogous to actions in planning.
% 
They convert a probabilistic \emph{model} into a propositional \emph{model},
i.e., they do not generate a model from unstructured inputs.
In fact, options ($\approx$ actions) in their semi-MDP have names assigned by a human (\pddl{move}/\pddl{interact}),
and state variables are identifiable entities
(x/y distances toward objects, light level, state of a switch) i.e. already symbolic.
% For example, in their experimental evaluation,
% $S$ has 33 low-level variables representing
% structured sensor input (e.g., x/y distances between each effector and each object, light level)
% and categorical states (the on/off state of a button, whether the monkey has cried out)
% while there are with little or no entanglements between sensors.

%% this could be in the action symbol grounding paper.
% In various model-free Reinforcement Learning settings such as DQN
% \cite{dqn}, the system avoids the symbol grounding problem by directly
% using the subsymbolic state encoding obtained from a NN.
% However, in these systems, typically the number of the actions and their labels are known \emph{apriori}:
% For example, DQN knows that there are 8-directional joystick and a button.

Previous work in Learning from Observation, which could take images (unstructured input),
typically assume domain-dependent hand-coded symbol extractors,
such as \emph{ellipse detectors} for tic-tac-toe board data
which immediately obtains propositions \cite{BarbuNS10}.
\citeauthor{Kaiser12} (\citeyear{Kaiser12}) similarly assumes grids and pieces
to obtain the relational structures in the board image.

% The fact that all of these AMA methods require the (near) symbolic inputs indicates
% that \emph{their utility will be greatly increased by a robust symbol grounding method}, which we propose in this paper.




\subsection{Autoencoders and Latent Representations}
An Autoencoder (AE) is a type of feed-forward neural network that learns an identity function whose output matches the input \cite{hinton2006reducing}.
% It is a form of unsupervised learning as it does not require the labels.
Its intermediate layer (typically smaller than the input) has a compressed, \emph{latent representation} of the input.
% AEs are commonly used for pretraining a NN.
AEs are trained by backpropagation (BP) to minimize the reconstruction loss, the distance between the input and the output according to a distance function such as Euclidean distance.
NNs, including AEs, typically have continuous activations and integrating them with propositional reasoners is not straightforward.


% -*- truncate-lines : nil -*-

\section{\latentplanner:  System Architecture}
\label{sec:overview}

\begin{figure}[tbp]
 \centering
 \includegraphics[width=\linewidth]{img/planning.pdf}
 \caption{Classical planning in latent space:
We use the learned State Autoencoder (\refsec{sec:state-autoencoder}) to convert pairs of images $(\before,\after)$ first to symbolic transitions, from which the AMA component generates an action model.
We also encode the initial and goal state images into symbolic initial/goal states.
A classical planner finds the symbolic solution plan.
Finally, intermediate states in the plan are decoded back to a human-comprehensible image sequence.}
\label{fig:overview}
\end{figure}

This section describes the high-level architecture of \latentplanner (\refig{fig:overview}).
\latentplanner takes two inputs.
The first input is the \emph{transition input} $Tr$, a set of pairs of raw data.
Each pair $tr_i=(\before_i, \after_i) \in Tr$ represents a transition of the environment before and after some action is executed.
The second input is the \emph{planning input} $(i, g)$, a pair of raw data, which corresponds to the initial and the goal state of the environment.
The output of \latentplanner is a data sequence representing the plan execution that reaches $g$ from $i$.
While we present an image-based implementation (``data'' = raw images),
the architecture itself does not make such assumptions
and could be applied to the other data formats e.g. audio/text.

% Emphasize the type of symbol, as readers are not familier with or haven't deeply thought about symbols
% \subsection{SAE for Propositional Symbol Grounding}

\latentplanner works in 3 phases.
In Phase 1, a \emph{State Autoencoder} (SAE) learns a bidirectional mapping between raw data (e.g., images)
 and propositional states from a set of unlabeled, random snapshots of the environment.
The $Encode$ function maps images to propositional states, and $Decode$ function maps the propositional states back to images.
After training the SAE from $\braces{\before_i, \after_i\ldots}$,
we apply $Encode$ to each $tr_i \in Tr$ and obtain $(Encode(\before_i),$ $Encode(\after_i))=$ $(s_i,t_i)=$ $\overline{tr}_i\in \overline{Tr}$,
the symbolic representations (latent space vectors) of the transitions.

% \subsection{Action Symbol Grounding and Action Model Acquisition}

% be careful about the distinction between ``action'' and ``transition''.

In Phase 2, an AMA method identifies the action symbols from $\overline{Tr}$ and learns an action model, both in an unsupervised manner.
We propose two approaches: AMA$_1$ directly generates a PDDL and AMA$_2$ produces a successor function (implicit model).
Both methods have advantages and drawbacks.
AMA$_1$ is a trivial AMA method designed to show the feasibility of SAE-generated propositional symbols. It does not learn/generalize from examples, instead requiring all valid state transitions. However, since AMA$_1$ directly produces a PDDL model, it serves as a demonstration that in principle, the approach is compatible with existing planners.
AMA$_2$ is a novel NN architecture which jointly learns action symbols and action models from a small subset of transitions in an unsupervised manner. Unlike existing methods, AMA$_2$ does not require action symbols. Since it does not produce PDDL, it needs a search algorithm (such as A*) for AMA$_2$, or semi-declarative symbolic planners \cite{frances2017purely}, instead of PDDL-based solvers.

In Phase 3, a planning problem instance is generated from the planning input $(i,g)$.
These are converted to symbolic states by the SAE, and the symbolic planner solves the problem.
For example, an 8-puzzle problem instance consists of an image of the start (scrambled) configuration of the puzzle ($i$), and an image of the solved state ($g$).

Since the intermediate states comprising the plan are SAE-generated latent bit vectors, the ``meaning'' of each state (and thus the plan) is not necessarily clear to a human observer.
However, in the final step, we obtain a step-by-step visualization of the plan execution (e.g. \refig{fig:mnist})
by $Decode$'ing the latent bit vectors for each intermediate state.

% the following is to pre-empt questions/attacks about the fact that the other half of symbol grounding, symbol->actuation is not addressed at all yet.
In this paper, we evaluate \latentplanner as a high-level planner using puzzle domains such as the 8-puzzle.
Mapping a high-level action to low-level actuation sequences via a motion planner is beyond the scope of this paper.

% Thus, physically ``executing'' the plan is not necessary, as finding the solution to the puzzles is the objective, so a ``mental image'' of the solution (i.e., the image sequence visualization) is sufficient.
% In domains where actions have effects in the world, it will be necessary to consider how actions found by \latentplanner (transitions between latent bit vector pairs) can be mapped to actuation (future work).

\section{SAE as a Gumbel-Softmax VAE}
\label{sec:state-autoencoder}

First, note that a direct 1-to-1 mapping from images to propositions can be trivially obtained from
the array of discretized pixel values or an image hash function.
% The model generation method of \refsec{sec:ama1-overview} could be applied to such ``propositions''.
However, such a trivial SAE lacks the crucial properties of
% "unforseen" means, completely unanticipated" but here I think we mean "never seen before"
% pixel values can ``map'' any image to symbols. but it uses different symbols for different pixels,
% i.e. the description lacks the consistency.
\emph{generalization} -- ability to describe unseen world states with the same symbols --
\emph{robustness} -- two similar images that represent ``the same world state'' should map to the same representation --
and \emph{bijection} -- ability to map symbolic states to real-world images.
We need a bidirectional mapping where the symbolic representation captures the ``essence'' of the image, not merely the literal, raw pixel vector.

The first technical contribution of this paper is the proposal of a SAE which is implemented as 
a Variational Autoencoder \cite{kingma2014semi} with a Gumbel-Softmax (GS) activation \cite{jang2016categorical} (\refig{fig:sae}).
% 

A Variational Autoencoder (VAE) \cite{kingma2013auto} is a type of AE that forces the \emph{latent layer} (the most compressed layer in the AE) to follow a certain distribution (e.g., Gaussian).
% While initially proposed for enforcing Gaussian distributions, VAEs have been used to enforce arbitrary types of distribution (notably by Generative Adversarial Network \cite{goodfellow2014generative,makhzani2015adversarial}). 
Since the random distribution is not differentiable (BP is not applicable), VAEs use \emph{reparametrization tricks}, which decompose the target distribution into a differentiable and a purely random distribution (the latter does not require the gradient).
For example, the Gaussian $N(\sigma,\mu)$ is decomposed to $\mu+\sigma N(1,0)$, where $\mu,\sigma$ are learned.
In addition to the reconstruction loss, VAE should also minimize the variational loss (the difference between the learned and the target distributions) measured by, e.g.,  KL divergence.

Gumbel-Softmax (GS) is a recently proposed re\-para\-metri\-zation trick \cite{jang2016categorical} for categorical distribution.
It continuously approximates Gumbel-Max \cite{maddison2014sampling}, a method for drawing categorical samples.
Assume the output $z$ is a one-hot vector, e.g. if the domain is $D=\braces{a,b,c}$, $\brackets{0,1,0}$ represents ``b''.
The input is a class probability vector $\pi$, e.g. $\brackets{.1,.1,.8}$.
Gumbel-Max draws samples from $D$ following $\pi$:
% \[
 $z_i \equiv [ i = \arg \max_j (g_j+\log \pi_j)\ ?\ 1 : 0 ]$
% \]
where $g_j$ are i.i.d samples drawn from Gumbel$(0,1)$ \cite{gumbel1954statistical}.
Gumbel-Softmax approximates argmax with softmax to make it differentiable:
% \[
$z_i = \text{Softmax}((g_i+\log \pi_i)/\tau)$.
% \]
``Temperature'' $\tau$ controls the magnitude of approximation, which is annealed to 0 by a certain schedule.
The output of GS converges to a discrete one-hot vector when $\tau\approx 0$.






{\it Our key observation is that these categorical variables can be used directly as propositional symbols by a symbolic reasoning system, i.e., this gives a solution to the propositional symbol grounding in our architecture}.
In the SAE, we use GS in the latent layer. Its input is connected to the encoder network. The output is an $(N, M)$ matrix where $N$ is the number of categorical variables and $M$ is the number of categories.
% The input is fed to a fully connected layer of size $N\times M$, reshaped to a $(N, M)$ matrix and processed by the GS activation function.
We specify $M=2$, effectively obtaining $N$ propositional state variables. It is possible to specify different $M$ for each variable and represent the world using multi-valued representation as in SAS+ \cite{backstrom1995complexity}, but we use $M=2$ for all variables for simplicity.
% and also for leveraging GPU parallelism by running the computation as a matrix operation.
This does not affect the expressiveness because bitstrings of sufficient length can represent arbitrary integers.

The trained SAE provides a bidirectional mapping between the raw inputs (subsymbolic representation) and their symbolic representations:
\begin{itemize} %this is crucial to understanding the rest of the paper, so making an itemize to highlight it
\setlength{\itemsep}{-0.3em}
\item $b=Encode(r)$ maps an image  $r$ to a boolean vector $b$.
\item $\tilde{r}=Decode(b)$ maps a boolean vector $b$ to an image $\tilde{r}$.
\end{itemize}
$Encode(r)$ maps raw input $r$ to a symbolic representation by feeding the raw input to the encoder network, extract the activation in the GS layer,
 and take the first row in the $N \times 2$ matrix, resulting in a binary vector of length $N$. Similarly, $Decode(b)$ maps a binary vector $b$  back to an image by concatenating $b$ and its complement $\bar{b}$ to obtain a $N \times 2$ matrix and feeding it to the decoder.
These are lossy compression/decompression functions, so in general, $\tilde{r}=Decode(Encode(r))$ may have an acceptable amount of errors from $r$ for visualization.

\begin{figure}[tbp]
 \centering
 \includegraphics[width=\linewidth]{img/train-state-ae.pdf}
 \caption{Step 1:
Train the State Autoencoder by
 minimizing the sum of the reconstruction loss and the variational loss of Gumbel-Softmax.
As the training continues, the output of the network converges to the input images.
Also, as the Gumbel-Softmax temperature $\tau$ decreases during training,
the latent values approach either 0 or 1.}
 %  (KL divergence between the actual latent distribution and the random categorical distribution as the target)
 % (binary cross-entropy between the input and the output)
\label{fig:sae}
\end{figure}

%% we can add when they ask.
% To map an image to a latent state in a 1-to-1 manner,
% the latent layer requires certain capacity.
% The lower bound of the number of propositional variables $N$ is
% the encoding length of the states, i.e. $N \geq \log_2 \bars{S}$ for a state space $S$.

It is {\it not} sufficient to simply use traditional activation functions such as sigmoid or softmax and round the continuous activation values in the latent layer to obtain discrete 0/1 values.
In order to map the propositional states back to images,
we need a decoding network trained for 0/1 values.
A rounding-based scheme would be unable to restore the images because the decoder is not trained with inputs near 0/1 values.
Also, embedding the rounding operation as a layer of the network is infeasible because rounding is non-differentiable, precluding BP-based training of the network.

% This is almost 
% \emph{In some domains, an SAE trained on a small fraction of the possible states successfully generalizes so that it can $Encode$ and $Decode$ every possible state in that domain.}
% In all our experiments below, on each domain, we train the SAE using randomly selected images from the domain. For example, on  the 8-puzzle, the SAE trained on 12000 randomly generated configurations out of 362880 possible
% configurations is used by the action model generator (\refsec{sec:ama1-overview}) to $Encode$ every 8-puzzle state.

SAE implementation can easily and largely benefit from the progress in the image processing community.
% In addition to GS,
% denoising AE is necessary for SD mixed example convergence in the later section.
We implemented SAE as a denoising autoencoder \cite{vincent2008extracting} to add noise robustness,
with some techniques which improve the accuracy (see Appendix \refsec{sec:SAE-detail}).


\section{AMA$_1$: Oracular PDDL Generator}
\label{sec:ama1-overview}

% avoid ``file'' --- actually two files.
In AMA$_1$, our first AMA method, the output is a PDDL definition for a grounded unit-cost STRIPS planning problem.
AMA$_1$ is a trivial, \emph{oracular} strategy which generates a model based on \emph{all} transitions, i.e., 
$Tr$ contains image pairs representing all transitions that are possible in this domain, and
$\overline{Tr}$ contains all corresponding symbolic transitions.
The images are generated by an external, domain-specific image generator.
It is important to note that while $Tr$ for AMA$_1$ contains all transitions, the SAE is trained using only a subset of state images.
% small subset of $Tr$: small is subjective. Also SAE learns from each state, not transitions.
Although ideally an AMA component should induce a complete action model from a limited set of transitions,
AMA$_1$ is intended to demonstrate the overall feasibility of SAE-produced propositions and the overall \latentplanner architecture.


AMA$_1$ compiles $\overline{Tr}$ directly into a PDDL model as follows.
Each transition $\overline{tr}_i \in \overline{Tr}$ directly maps to an action $a_i$.
Each bit $b_j (1 \leq j \leq N)$ in boolean vectors $s_i$ and $t_i$ is mapped to propositions \texttt{(b$_j$-true)} and \texttt{(b$_j$-false)} when the encoded value is 1 and 0 (resp.). 
$s_i$ is directly used as the preconditions of action $a_i$.
The add/delete effects of action $i$ are computed by taking the bitwise difference between $s_i$ and $t_i$.
For example, when $b_j$ changes from 1 to 0, the effect compiles to \texttt{(and (b$_j$-false) (not (b$_j$-true)))}.
The initial and the goal states are similarly created by applying the SAE to the initial and goal images.

The PDDL instance output by AMA$_1$ can be solved by an off-the-shelf planner.
We use a modified version of Fast Downward \cite{Helmert2006} (see Appendix \refsec{sec:ama1-planner}).
\latentplanner inherits all of the search-related properties of the planner which is used. 
For example, if the planner is complete and optimal, \latentplanner will find an optimal plan for the given problem (if one exists), with respect to the portion of the state-space graph captured by the Action Model.
% Domain-independent heuristics developed in the planning literature are designed to exploit structure in the action model.
% Although the structure in models acquired by \latentplanner may not directly correspond to those in hand-coded models, 
% intuitively, there should be some exploitable structure. 
% The search results in \refsec{sec:heuristics-evaluation}  suggest that the domain-independent heuristics can reduce the search effort.

% indeed, latent space has some exploitable structure. 



\subsection{Evaluating AMA$_1$ on Various Puzzles}

\label{sec:ama1-experiments}

We evaluated \latentplanner with AMA$_1$ on several puzzle domains.
Resulting plans are shown in \refigs{fig:mnist}{fig:mnist2}.
See Appendix \refsec{sec:domain-details} for further details of the network, training and inputs.

% the shortened domain description
\textbf{MNIST 8-puzzle}
is an image-based version of the 8-puzzle, where tiles contain hand-written digits (0-9) from the  MNIST database \cite{lecun1998gradient}.
Valid moves in this domain swap the ``0'' tile  with a neighboring tile, i.e., the ``0'' serves as the ``blank'' tile in the classic 8-puzzle. 
The \textbf{Scrambled Photograph 8-puzzle (Mandrill, Spider)} cuts and scrambles real photographs, similar to the puzzles sold in stores).
These differ from the MNIST 8-puzzle in that ``tiles'' are \textit{not} cleanly separated by black regions
(we re-emphasize that \latentplanner has no built-in notion of square or movable region).
In \textbf{Towers of Hanoi (ToH)},
we generated the 4 disks instances.
4-disk ToH resulted in a 15-step optimal plan.
\textbf{LightsOut} is
a video game where a grid of lights is in some on/off configuration ($+$: On),
and pressing a light toggles its state as well as the states of its neighbors.
The goal is all lights Off.
Unlike previous puzzles, a single operator can flip 5/16 locations at once and
removes some ``objects'' (lights).
This demonstrates that \latentplanner is not limited to domains with highly local effects and static objects.
\textbf{Twisted LightsOut} distorts the original LightsOut game image by a swirl effect, 
showing that \latentplanner is not limited to handling rectangular ``objects''/regions.

\begin{figure}[tbp]
 \centering
 \includegraphics[width=\linewidth]{img/mnist-plan-new.pdf}
 \includegraphics[width=\linewidth]{img/mandrill-plan-new.pdf}
 \includegraphics[width=\linewidth]{img/spider-plan-new.pdf}
 \caption{
Output of \latentplanner + AMA$_1$ solving the MNIST/Mandrill/Spider 8-puzzle instance
with the longest (31 steps) optimal plan (Reinefeld 1993).
This shows that \latentplanner finds an optimal solution
given a correct model by AMA$_1$ and an admissible search algorithm.
\latentplanner has no notion of ``slide'' or ``tiles'',
making MNIST, Mandrill and Spider entirely distinct domains.
SAEs are trained from scratch without knowledge transfer.
}
 \label{fig:mnist}
\end{figure}

\begin{figure}[tbp]
 \includegraphics[width=\linewidth]{img/hanoi4-new.pdf}
 \includegraphics[width=\linewidth]{img/lightsout_new4x4.pdf}
 \includegraphics[width=\linewidth]{img/lightsout_twisted_new4x4.pdf}
 \caption{
(1) Output of solving ToH with 4 disks.
(2-3) Output of solving 4x4 LightsOut and Twisted LightsOut.
The blurs in the goal states are simply the noise that was normalized and enhanced by the plotting library.
}
 \label{fig:mnist2}
\end{figure}

\subsection{Robustness to Noisy Input}

\label{sec:noise-experiments}

\refig{fig:noise} demonstrates the robustness of the system vs. input noise.
We corrupted the initial/goal state inputs by adding Gaussian or salt/pepper noise.
The system is robust enough to successfully solve the problem
because of the Denoising AE \cite{vincent2008extracting}.
% which has an internal \emph{GaussianNoise layer} which adds a Gaussian noise to the inputs (only during training)
% which reconstructs the original image from a corrupted image.
% This demonstrates the benefit of exploiting existing Deep Learning techniques,
% and there are plenty of potentials in applying \emph{more} techniques to increase the robustness.

\begin{figure}[tbp]
 \centering
 \includegraphics[width=\linewidth]{img/noise-new.pdf}
 \caption{
SAE robustness vs noise:
  Corrupted initial state image $r$ and its reconstruction $Decode(Encode(r))$.
 Images are corrupted by Gaussian noise of $\sigma$ up to $0.3$ and by salt/pepper noise up to $p=0.06$.
 \latentplanner successfully solved the problems.
 The SAE maps the noisy image to the correct symbolic vector, finds a plan, then
 maps the plan back to the denoised images.
 }
  \label{fig:noise}
\end{figure}

% \input{unused/heuristics.tex}


\section{AMA$_2$: Action Symbol Grounding}
\label{sec:ama2-overview}

\latentplanner + AMA$_1$ shows that (1) the SAE can robustly learn image $\leftrightarrow$ propositional vector mappings from examples, and that (2) if all valid image-image transitions (i.e., the entire state space) is given, \latentplanner can correctly generate optimal plans.
However, AMA$_1$ is clearly not practical due to the requirement that it uses the entire state space as input, and lacks the ability to learn/generalize an action model from a small subset of valid action transitions (image pairs).
Next, we propose AMA$_2$, a novel neural architecture  which jointly grounds the action symbols and acquires the action model from the subset of examples, in an unsupervised manner.

Acquiring a descriptive action model (e.g., PDDL) from a set of unlabeled propositional state transitions consists of three steps.
% 
(Step 1) Identify the ``types'' of transitions, where each ``type'' is an identifiable, \emph{action symbol}.
For example, a  hand-coded ``slide-up-8-at-1-2'' in 8-puzzle is an example of action symbols, but note that an AMA system should ground anonymous symbols without human-provided labels.
% 
While they are not lifted/parameterized, they still provide abstraction. For example, the same ``slide-up-8-at-1-2'' action, which slides the tile 8 at position $(x,y)=(1,2)$ upward, applies to many states (each state being a permutation of tiles 1-7).
% 
(Step 2) Identify the preconditions and the effects of each action and store the information in an action model.
(Step 3) Represent the model in a modeling language (e.g., PDDL) as in \refig{8puzzle-pddl}.

Addressing this entire process is a daunting task. 
Existing AMA methods typically handle only Steps 2 and 3, skipping Step 1.
Without step 1, however, an agent lacks the ability to learn in an unknown environment where it does not know \emph{what is even possible}.
Note that even if the agent has the full knowledge of its low-level actuator capabilities, it does not know its own high-level capabilities e.g. sliding a tile.
Note that AMA$_1$ handles only Step 3, as providing all valid transitions is equivalent to skipping Step 1/2.

On the other hand, search on a state space graph in an unknown environment is \textit{feasible} even if Step 3 is missing.
PDDL provides two elements, a \emph{successor function} and its \emph{description}.
While ideally both are available, the description is not the \emph{essential} requirement.
The description may increase the explainability of the system in a language such as PDDL,
but such explainability may be lost anyway when the propositional symbols are identified by SAE, as the meanings of such propositions are unclear to humans (\refsec{sec:overview}).
The description is also useful for constructing the heuristic functions, but
the recent success of simulator-based planning \cite{frances2017purely}
shows that, in some application, efficient search is possible without action descriptions.
 
The new method, AMA$_2$, thus focuses on Steps 1 and 2.
It grounds the action symbols (Step 1) and finds a successor function that can be used for forward state space search (Step 2), but maintains its implicit representation.
% 
AMA$_2$ comprises two networks: an \emph{Action Autoencoder} (AAE) and an \emph{Action Discriminator} (AD). The AAE jointly learns the action symbols and the action effects, and provides the ability to enumerate the candidates of the successors of a given state. The AD learns which transitions are valid, i.e. preconditions. Using the enumeration \& filtering approach, the AAE and the AD provides a successor function that returns a list of valid successors of the current state. Both networks are trained unsupervised, and operate in the symbolic latent space, i.e. both the input and output are SAE-generated bitvectors. This keeps the network small and easy to train.


\subsection{Action Autoencoder}

Consider a simple, linear search space with no branches.
In this case, grounding the action symbol is not necessary and
% (there is only a single action) --- this may be inaccurate. eat, sleep, eat, sleep... linear, but two actions.
the AMA task reduces to predicting the next state $t$ from the current state $s$.
% Assuming that NN can learn arbitrary functions, % 
A NN $a'$ could be trained for a successor function $a(s)=t$, minimizing the loss $|t-a'(s)|$.
This applies to much of the work on scene prediction from videos such as \cite{srivastava2015unsupervised}. % in this context "much"=="a lot", and safer than while "most" (>= "majority") when making broad but ambiguous claims like this.

%% invalid statement
% This scenario also applies to the reinforcement learning setting, as the only task of RL is to learn the single best next state that the agent should reach in the next step. Typically, an RL system knows how many actions are available, e.g. the number of buttons, or the low-level actuation for each motor.

However, when the current state has multiple successors, as in planning problems, such a network cannot be applied.
One might consider training a separate NN for each action, but
(1) it is unknown how many types of transitions are available,
(2) the number of transitions depends on the current state, and
(3) it does not know which transition belongs to which action.
Although a single NN could learn a multi-modal distribution,
it lacks the ability to \emph{enumerate} the successors,
a crucial requirement for a search algorithm.

\begin{figure}[tbp]
 \centering
 \includegraphics[width=\linewidth]{img/aae/aae.pdf}
 \caption{Action Autoencoder.}
 \label{fig:aae}
\end{figure}

To solve this, we propose an Action Autoencoder (AAE, \refig{fig:aae}). % made it more clear that the AAE is your idea, not something that's already known
The key idea of AAE is to reformulate the transitions as $apply(a,s)=t$, which lifts the action symbol and makes it trainable,
and to realize that $s$ is the \emph{background information} of the state transition function.
The AAE has $s,t$ as inputs and reconstructs $t$ as $\tilde{t}$ whose error $|t-\tilde{t}|$ is minimized.
The main difference from a typical AE is:
(1) The latent layer is a Gumbel-Softmax one-hot vector indicating the \textbf{action label} $a$. %added bf to make this def noticeable so that nobody gets confused and thinks that an action label is a human-assigned label
(2) Every layer is concatenated with $s$.
The latter conditions the entire network by $s$,
which makes the 128 action labels (7bit) represent only the \emph{conditional information} (difference) necessary to ``reconstruct $t$ \emph{given} $s$'',
unlike typical AEs which encode the \emph{entire} information of the input.
% In a typical AE, the \emph{entire} information of the input is maintained in the latent space because it can restore the original input.
% %ensuring that the information is not lost during the network flow. % "information not lost" too strong; "network flow" is a term strongly associated with combinatorial optimization
% In contrast, in our AAE, 128 action labels (7bit) represent only the \emph{conditional information} (difference) of $t$ \emph{given} $s$, as $s$ is necessary to reconstruct $t$ from $a$.
% AE learns the identity function $x=\textit{Id}(x)$, but AAE learns the \emph{conditional} identity function $t=\textit{Id}(t,s)$.
% Thus, in order to reconstruct the successor state $\tilde{t}$, it requires both the latent layer (action label) $a$ as well as the before-state $s$ acting as a conditional prior.
% 
% The before-state is only fed to the network and is not learned; 
% This means that the differences in the valid transitions can be effectively compressed into the 7-bits represented by the action labels.
As a result, the AAE learns the bidirectional mapping between $t$ and $a$, both conditioned by $s$:
\begin{itemize}
\setlength{\itemsep}{-0.3em}
 \item $Action(t,s)=a$ returns the action label from $t$.
 \item $Apply(a,s)=\tilde{t}$ applies $a$ to $s$ and returns a successor $\tilde{t}$. 
\end{itemize}

The number of labels serves as the upper bound on  the number of action symbols learned by the network.
Too few labels make AAE reconstruction loss fail to converge to zero.
After training, some labels may not be mapped to by any of the example transitions.
In the later phases of \latentplanner, these unused labels are ignored.
Since we obtain a limited number of action labels,
we can enumerate the candidates of the successor states of the given current state in constant time.
Without AAE, all $2^N$ states would be enumerated as the potential successors, which is clearly impractical.

\subsection{Action Discriminator}

An AAE identifies the number of actions and learns the effects of actions, but does not address the applicability (preconditions) of actions.
Preconditions are necessary to avoid invalid moves (e.g. swapping 3 tiles at once) or invalid states (e.g. having duplicated tiles), as shown in \refig{fig:aae-mixed}.
% 
Thus we need an \textit{Action Discriminator} (AD, \refig{fig:ad}) which learns the 0/1 mapping for each transition indicating whether it is valid, i.e., the ``preconditions''. This is a standard binary classification function which takes $s,t$ as inputs and returns a probability that $(s,t)$ is valid.

\begin{figure}[tbp]
 \centering
 \includegraphics[height=0.7\paperheight]{img/aae/aae_mixed_states.pdf}
\caption{The successors of a state $s$ (bottom-right),
  generated by applying all 98 actions identified by the AAE.
 A valid successor is marked by the red border.}
 \label{fig:aae-mixed}
\end{figure}

\begin{figure}[tbp]
 \centering
 \includegraphics[width=0.5\linewidth]{img/aae/ad.pdf}
 \caption{Action Discriminator.}
 \label{fig:ad}
\end{figure}

One technical problem in training the AD is that explicit \emph{invalid} transitions are unavailable.
This is not just a matter of insufficient data, but rather a fundamental constraint in an image-based system operating in the physical environment: Invalid transitions which violate the laws of physics (e.g. teleportation) are \emph{never} observed (because they never happens).
We then might consider ``imagining/generating'' the negative examples, as humans do in a thought experiment, but it is also impossible due to the lack of specification of \emph{what} is invalid.

To overcome this issue, we use the PU-Learning framework \cite{elkan2008learning}, which can learn a positive/negative classifier from the positive and \emph{mixed} examples that may contain both positive and negative examples.
% 
We used $\overline{Tr}$ as the positive examples (they are all valid).
The mixed, i.e. possibly invalid, examples are generated by
applying each action $a$ (except unused ones) on each before-state $s$ in $\overline{Tr}$, and
removing the known positive examples from the generated pairs $(s,\tilde{t})$.

\subsection{PU-learning}
\label{sec:pu-learning}

The implementation of PU-learning is quite simple, following \cite{elkan2008learning}.
Given a positive ($p$) and a mixed ($m$) dataset,
 $p$ and $m$ are first arbitrarily divided into a training set ($p_1$ and $m_1$)
and validation set ($p_2$ and $m_2$), as usual.

Then, a binary classifier for $p_1$ (true) and $m_1$ (false) is trained.
As a result, we obtain a positive/mixed classifier $d_1(x)$ which is a function which returns a probability
that a data $x$ belongs to $p_1$.
% 
After the training has finished,
the positive examples in the validation set ($p_2$) are classified, and
the  probability of $p_2$ belonging to $p_1$ are averaged to obtain a scalar $c = average(d_1(p_2))$.
% 
As the final step, the true positive/negative classifier $d_2(x)$,
which is a function which returns a probability that a data $x$ is positive,
is defined as $d_2(x) = c \cdot d_1(x)$. 


\subsection{State Discriminator}

As a performance improvement, we also trained a State Discriminator (SD) which is a binary classifier for a single state $s$ and detects the invalid states, e.g. states with duplicated tiles in 8-puzzles. Again, we use PU-learning. Positive examples are the before/after states in $\overline{Tr}$ (all valid). Mixed examples are generated from the random bit vectors $\rho$ (may be invalid):
Many of the images $Decode$'ed from $\rho$ are blurry and do not represent autoencodable, meaningful real-world images.
However, when they are repeatedly encoded/decoded (\refig{fig:random-bit}), they converge to the clear, autoencodable invalid states because of the denoising AE \cite{vincent2008extracting}, and we used the results as the mixed examples.
If $Decode(\rho)$ results in a blurry image,
this ``blur'' is recognized as a noise and reduced in each autoencoding step,
finally resulting in a clean, reconstructable invalid image.
We use the SD to prune some mixed action examples for the AD training so that they contain only the valid successors.
This improves the AD accuracy significantly.

\begin{figure}[tbp]
 \centering
 \includegraphics[width=0.5\linewidth]{img/aae/random-convergence.pdf}
 \caption{
(top, left) Random bit vector $\rho$,
(bottom, left) $Decode(\rho)$,
(top, middle)  $Encode(Decode(\rho))=\rho_2$,
(bottom, middle) $Decode(\rho_2)$,
(top, right)  $Encode(Decode(\rho_2))=\rho_3$,
(bottom, right) $Decode(\rho_3)$.
As more autoencoding is performed, images become less blurry,
 although they are still invalid (two 1-tiles).
}
\label{fig:random-bit}
\end{figure}

\subsection{Additional Pruning Methods}

Additionally, we have two more pruning methods:
The first one ensures that the SAE successfully reconstructs the successor state $t$, i.e. $t$ and $Encode(Decode(t))$ are identical.
The second one ensures that the AAE reconstructs $t$, i.e. $t$ and $Apply(Action(t,s),s)$ are identical.
Successors which failed to be reconstructed are removed from the consideration.

\subsection{Planning in \latentplanner using AMA$_2$}

In the case of AMA$_2$, we can not use an off-the-shelf PDDL-based planner because the action model is embedded in
the AAE, AD, and SD neural networks.
However, they allow us to implement a successor function which can be used in any symbolic, 
forward state space search algorithm such as \astar \cite{hart1968formal}.
The AAE generates the (potentially invalid) successors and
 the AD and SD filter the invalid states:
\begin{align*}
  Succ(s) &= \{t = apply(a,s) \; | \; a \in \braces{0\ldots 127} \setminus \textit{unused},\\
          & \qquad \land AD(s,t) \geq 0.5 \\
          & \qquad \land SD(t) \geq 0.5 \\
          & \qquad \land Encode(Decode(s)) \equiv s \\
          & \qquad \land Apply(Action(t,s),s) \equiv t \}
\end{align*}
We implemented A* in which states are latent-space (propositional) vectors, and the above $Succ$ function is used to generate successors of states.
A simple goal-count heuristic is used.
As the goal-count heuristics is inadmissible, the results could be suboptimal.
However, the purpose of implementing this planner is to see the feasibility of the action model.


% Extracting the symbolic description from each action is future work. % <-- should be in conclusion.

% Combination with semi-declarative solvers \cite{frances2017purely} is an interesting avenue for future work.




\subsection{Evaluation}

\label{sec:ama2-experiments}

We evaluate the feasibility of the action symbols and the action models learned by AAE and AD.
We tested 8-puzzle (mnist, mandrill, spider), LightsOut (+ Twisted). % and ToH.
% ToH was given a special treatment of using 90\% of whole transitions because of the smaller search space.
We generated 100 instances for each domain and for each noise type (std, gaussian noise, salt/pepper noise)
 by  7-step (benchmark A)or  14-step (benchmark B) self-avoiding random walks from the goal state,
and evaluated the planner with the 180 sec.\ time limit.
We verified the resulting image plans with domain-specific validators.
\reftbl{tab:aae-results} shows that the \latentplanner achieves a high success rate.
% In 8-puzzle/LightsOut domains,
The failures are due to timeouts
(the successor function requires many calls to the feedforward neural nets,
 resulting in a very slow node generation).
% In Hanoi domains, failures are due to disconnected search graph.

We next examine the accuracy of the AD and SD (\reftbl{tab:aae-results}).
We measured the type-1/2 errors for the valid and invalid transitions (AD) and states (SD).
Low errors show that our networks successfully learned the action models.

\begin{table}[tbp]
\centering
% \setlength{\tabcolsep}{0.1em}
% \relsize{-1}
\begin{tabular}{|l|r|r|r|r|r|r||l|l|l|l|l|l|}
\hline
domain   & \multicolumn{3}{c|}{A:step=7}& \multicolumn{3}{c||}{B:step=14} &  \multicolumn{2}{c|}{SD error (\%)} &  \multicolumn{4}{c|}{AD error (in \%)} \\
                     & std & G   & s/p & std & G & s/p & type1   & type2   & type1  & type2  & 2/SD & 2/V  \\ \hline
MNIST                & 72  & 64  & 64  &6    &4  &3    & 0.09    & $<$0.01 & 1.55   & 14.9   & 6.15 & 6.20 \\ 
Mandrill             & 100 & 100 & 100 &9    &14 &14   & $<$0.01 & $<$0.01 & 1.10   & 16.6   & 2.93 & 2.94 \\ 
Spider               & 94  & 99  & 98  &29   &36 &38   & $<$0.01 & $<$0.01 & 1.22   & 17.7   & 4.97 & 4.91 \\ 
L. Out               & 100 & 99  & 100 &59   &60 &51   & $<$0.01 & N/A     & 0.03   & 1.64   & 1.64 & 1.64 \\
Twisted              & 96  & 65  & 98  &75   &68 &72   & $<$0.01 & N/A     & 0.02   & 1.82   & 1.82 & 1.82 \\
% Hanoi                & 37  & 44  & 39  &15   &18 &17   & 0.03    & $<$0.01 & 0.25   & 3.50   & 3.79 & 4.07 \\
\hline
\end{tabular}
\caption{
AMA$_2$ results: (\textbf{left}) Number of solved instances out of 100 within 3 min. time limit.
% All failures are due to timeouts (OPEN never exhausted; graph never disconnected).
% Note: not on hanoi.
The 2nd/3rd columns show the results when the input is corrupted by G(aussian) or s(alt)/p(epper) noise.
In benchmark A (created with 7-step random walks),
\latentplanner solved the majority of instances even under the input noise.
In the harder instances (benchmark B: 14-steps),
many instances were still solved. 
(\textbf{right}) Misclassification by SD and AD in \%, measured as:
(SD type-1) Generate all valid states and count the states misclassified as invalid.
(type-2) Generate reconstructable states, remove the valid states (w/ validator),
sample 30k states, and count the states misclassified as valid.
N/A means all reconstructable states were valid.
(AD type-1) Generate all valid transitions and count the number of misclassification.
(type-2) For 1000 randomly selected valid states, generate all successors, remove the valid transitions (w/ validator), then count the transitions misclassified as valid.
(2/SD, 2/V) Same as Type-2, but ignore the transitions whose successors are invalid according to SD or the validator.
Relatively large AD errors explain the increased number of failures in MNIST 8-puzzles.
}
\label{tab:aae-results}
\end{table}




\section{Related Work}
\label{sec:related}

% This section reviews and compares previous work with our proposed architecture.
% A more complete, recent survey of other machine learning methods for planning can be found in \cite{JimenezRFFB12}.

Compared to the work by \citeauthor{KonidarisKL14} (\citeyear{KonidarisKL14}),
the inputs to \latentplanner are unstructured (42x42=1764-dimensional arrays for 8-puzzle);
each pixel does not carry a meaning and the boundary between ``identifiable entities'' is unknown.
% vs. 33 symbolic continuous/discrete variables in their work
% Symbols found by the SAE are formed by the entanglements among multiple pixels which are hard to model explicitly.
% The only specification to the propositional symbols are the upper bound of the number of state variables.
% SAE thus correctly addresses the grounding of propositional symbols.
Also, AMA$_2$ automatically grounds action symbols, while they rely on human-assigned symbols (\pddl{move, interact}).
Furthermore, they do not explicitly deal with robustness to noisy input, while we implemented SAE as a denoising AE.
% and further generalization can be expected from deep learning techniques (e.g. translational/rotational invariance).
However, effects/preconditions in AMA$_2$ is implicit in the network, and their approach could be utilized to extract PDDL from AAE/AD (future work).

% \subsection{Learning from Observation}

% Our approach differs from the
% work on learning from observation (LfO) in the robotics literature \cite{ArgallCVB09} in that:
% (1) \latentplanner is trained based on image pairs showing before/after images of valid individual actions, while LfO work is largely based on longer sequence of actions (e.g., of videos);
% (2) \latentplanner generates PDDL suited for high-level (puzzle-like) tasks, while LfO focuses on motion planning tasks. 
%
% A closely related line of work in LfO is learning of board game play from observation of video/images \cite{BarbuNS10,Kaiser12,kirk2016learning}.
% These work made relatively strong assumptions about the environment, e.g., that there is a grid-like environment with ``piece''-like objects.
% In contrast, as shown in \refsec{sec:experiments}, \latentplanner does not make assumptions about the contents of the images.

% \subsection{Neural-Symbolic Hybrid Approaches}
There is a large body of work using NNs to directly solve combinatorial tasks,
starting with the well-known TSP solver \cite{hopfield1985neural}.
% With respect to state-space search similar to what we consider,
Neurosolver represents a search state as a node in NN 
and solved ToH \cite{bieszczad2015neurosolver}. %bieszczad1998neurosolver --- not much space in page 8
However, they assume a symbolic input.
\begin{comment}
%very short version of related work on NN+search
Previous work combining NNs and symbolic search algorithms embedded NNs {\it inside} a search algorithm
to provide search control knowledge \cite{alphago,ArfaeeZH11} % TODO: add ,SatzgerK13 later in longer version.
In contrast, we use a NN-based SAE for symbol grounding, not for search control.
\end{comment}

Previous work combining symbolic search and NNs embedded NNs {\it inside} a search
to provide the search control knowledge,
% AlphaGo  uses a NN (learned from traces of expert Go players as well as self-play) as a heuristic evaluation function
% to guide search in a Monte-Carlo tree search algorithm \cite{alphago}.
e.g., domain-specific heuristic functions for
the sliding-tile puzzle and Rubik's Cube \cite{ArfaeeZH11},
classical planning \cite{SatzgerK13},
or the game of Go \cite{alphago}.
% We use NNs for symbol grounding/AMA, not for search control.
% We do not use NNs for search control.
%ALE lipovetzky
% -- input is raw state vector, which is  ``unstructured'', but arguable whether it is   ``subsymbolic'' .
% must be extremely careful in pointing out limitations of DRL because many people are interested in it 
Deep Reinforcement Learning (DRL) has solved complex problems,
including video games where it communicates to a simulator through images \cite[DQN]{dqn}.
%
In contrast, \latentplanner only requires a set of unlabeled image pairs (transitions), and 
does not require a reward function for unit-action-cost planning,
nor expert solution traces (AlphaGo),
nor a  simulator (DQN), nor predetermined action symbols (``hands'', control levers/buttons).
% Access to expert traces or simulators is a significant limitation of RL.
% as they may not be available nor be obtained by random exploration.
% In contrast, the only requirement of \latentplanner is to collect the transition data which are not necessarily expert or ``good'' in any sense, and can be collected mechanically by a random walk.
% Finally, RL lacks the completeness/optimality guaranteed by classical planning, which is mandatory for critical applications.
Extending \latentplanner to symbolic POMDP planning is an interesting avenue for future work.

\begin{comment}
 An Autoencoder is a nonlinear generalization of Principal Component Analysis
 \cite{bourlard1988auto}.
 Therefore, our approach is somewhat similar to an approach that uses PCA
 to run RL in the continuous latent space of the configuration space of a high-DOF
 robot \cite{luck2014latent}.
 Since the latent representation is more compact than the original configuration space of a robot, they can run reinforcement learning more efficiently.
\end{comment}

% PAIR community (Plan/Activitity/Intent Recognition) work. -- peripherally related

A significant difference between \latentplanner and learning from observation (LfO) in the robotics literature \cite{ArgallCVB09} is that
\latentplanner is trained based on individual transitions
while LfO work is largely based on the longer sequence of transitions (e.g. videos)
and should identify the start/end of actions (\emph{action segmentation}).
% 
Action segmentation would not be an issue in 
an implementation of autonomous \latentplanner-based agent
because it has the full control over its low-level actuators and
initiates/terminates its own action for the data collection.


 % moving here because it's hard to compare vs related work until \latentplanner has been fully defined

\section{Discussion and Conclusion}

\label{sec:discussion}

%* CAUTION:  avoid being too aggressive with claims, e.g., 
%  ``completely new'' --   there may be natural language processing or video-to-text systems which perform very impressive tasks -- safest to focus on novelty with regard to planning/logic/reasonining community framework.

We proposed  \latentplanner, an integrated architecture for learning and planning which,
given only a set of unlabeled images and no prior knowledge, generates a classical planning problem,
solves it with a symbolic planner,
and presents the plan as a human-comprehensible sequence of images.
%After training the SAE for bidirectional conversion between raw data representation to symbols, \latentplanner can solve planning problem instances presented as a pair of initial state and goal images.
%
We demonstrated its feasibility using image-based versions of planning/state-space-search problems (8-puzzle, Towers of Hanoi,  Lights Out).
% "technical contribution", as opposed to the conceptual contribution of the architecture
Our key technical contributions are
(1) \emph{SAE, which leverages the Gumbel-Softmax to learn a bidirectional mapping between raw images and propositional symbols compatible to symbolic planners}.
%This also allows the symbolic plans in the latent vector space to be mapped back to a comprehensible sequence of images.
On 8-puzzle, the ``gist'' of  42x42 training images are robustly compressed into propositions, capturing the essence of the images.
% as shown in \refsec{sec:experiments}, 
(2) \emph{AMA$_2$, which jointly grounds action symbols and learns the preconditions/effects.}
It identifies which transitions are ``same'' wrto the state changes and when they are allowed.

The only key assumptions about the input domain we make are that
(1) it is fully observable and deterministic and (2) NNs can learn from the available data.
% a small nuance: this simultaneously implies that we can train/tune the NN, as well as we have a sufficent amount of data.
Thus, we have shown that different domains can all be solved by the same system,
without modifying any code or the NN architecture.
In other words, \emph{\latentplanner is a domain-independent, image-based classical planner}.
% 
To our knowledge, this is the first system which completely automatically constructs a logical representation
\emph{directly usable by a symbolic planner} from a set of unlabeled images for a diverse set of problems.

% NO is too strong
% with no explicit assumptions or knowledge about the nature of the domains.

% Let's pre-empt criticisms about cherry-picking easy domains, possibly limited scope of current implementation, possible lack of robustness.
% However, as a proof-of-concept first implementation, it has significant limitations to be addressed in future work.

% \subsection{Action Learning}

%% obsolete question!
% The SAE provides a method of automatically extracting propositional representations from raw images.
% Thus, we believe that it should be possible to adapt and apply action model acquisition methods which have been developed for environments with deterministic effects and fully observable states  \cite[Sec.3,Fig.2]{JimenezRFFB12} to the propositional representations generated using the SAE in \latentplanner.
% Thus, replacing this primitive generator in \currentlatplan with a more sophisticated generator \cite{CresswellMW13,KonidarisKL14,MouraoZPS12,YangWJ07} is an important direction for future work.

%% probably doable by learning AD on goal states, but it needs goal state images
% Related question for future work is how to specify the goal condition for \latentplanner. 
% Since \currentlatplan assumes a single goal state as an input,
% developing a method for specifying a set of goal states with a partial goal specification as in IPC domains is an interesting future work.
% For example, one may want to tell the planner ``the goal states must have tiles 0,1,2 in the correct places'' in a MNIST 8-puzzle instance.


%Image-based action learning provides a new platform for investigating/extending previous work on action model acquisition,
% and we believe that \latentplanner opens many avenues for future work in this area.

%Furthermore, the explicit graph can potentially have many states which are redundant in the sense that
% two states which are conceptually the ``same'' symbolic state might be represented by different latent space vectors
% because the SAE fails to generalize. %TODO -- if/how to mention this for IJCAI?

% Review of actoin model learning:
% This work corresponds to action model acquisition in
% ``deterministic effects, full-state observability'' in Jiminez et al classification \cite[Sec3,Fig2]{JimenezRFFB12}

%A review of action model acquisition is in \cite[Section3.1]{JimenezRFFB12}.
%\citeauthor{MouraoZPS12} have a line of work on learning STRIPS operators \cite{MouraoZPS12}.
%Tends to be based on plan examples \cite{YangWJ07} -- we provide only positive, primitive action examples.

We demonstrated the feasibility of leveraging deep learning in order to enable 
symbolic planning using classical search algorithms such as A*, when only image pairs representing action start/end states are available,
and there is no simulator, no expert solution traces, and no reward function.
Although much work is required to determine the applicability and scalability of this approach,
we believe this is an important first step in bridging the gap between symbolic and subsymbolic reasoning and opens many avenues for future research.


\begin{comment}
% \subsection{Improving the SAE}
Although we showed that \latentplanner works on several domains,
we did not provide the minor implementation details of the networks (see Appendix), nor claimed that it works on \emph{all} data.
A truly robust learner is {\it not} a problem unique to \latentplanner, but a fundamental problem in deep learning,
and future work will seek to leverage further improvements.
% , such as 
% adopting more image processing techniques for further generalization,
% or extending the system for natural language audio or text data.

%Also, although we demonstrated that the SAE is somewhat robust to variations/noise, 
%it is not, able to, for example, solve an instance (initial state image) of a  sliding-tile puzzle instance scrawled on a napkin by an arbitrary person.
%\currentlatplan  will fail if, for example, some of the numbers in the initial state image for the 8-puzzle were rotated or translated, or were written by a different person than the person who wrote the digits used  in the training instance.
%An ideal system would be robust enough to solve an instance (initial state image) of a  sliding-tile puzzle instance scrawled on a napkin by an arbitrary person.
%It may be possible to achieve this level of robustness by making a state encoder which is robust to styles, rotation, translation, etc.

% Convolutional Networks \cite{le1989handwritten}.
% , Inception module \cite{szegedy2015going} or Residual Networks \cite{he2016deep}.

% Another direction for modifying the SAE is to combine the techniques in the autoencoders for the other types of inputs: Namely unstructured texts \cite{li2015hierarchical} and audio data \cite{deng2010binary}. Applying Gumbel-Softmax to these techniques, it could be possible for LatPlan to perform language-based or voice-based reasoning.

%* Completely automated link/interface between subsymbolic representations (e.g., images) and symbolic representations.
%       other subsymbolic representations: audio, video
%        similarly, ``unstructured symbolic representations'' natural language text
%       symbolic representation: here, we specifically mean ``logic'' (in the broadest sense of the term).



Finally, \emph{our conceptual contribution is the first demonstration that it is
possible to leverage deep learning quite effectively for classical planning,
% from AIMA, p396
which ``has been central to AI research since its inception.''} \cite{russell1995artificial}
%  purest form of symbolic GOFAI for deliberating automated agent
\end{comment}

\clearpage
\appendix

\section{State AutoEncoder}

\label{sec:SAE-detail}

All of the SAE networks used in the evaluation have the same network
topology for each domain, except the input layer which should fit the size of the input
images. They are implemented with TensorFlow and Keras libraries in under
5k lines of code.
We used a trivial, custom-made random grid search for automated tuning.
All layers except Gumbel-Softmax in the network are the very basic ones introduced in a standard tutorial.

% The network for AMA$_1$ experiments is a 6-layer autoencoder:
% [Input($input$), GaussianNoise(0.1),
% fc(4000), relu, bn, dropout(0.4),
% fc(4000), relu, bn, dropout(0.4),
% fc(98), reshape(49x2), GumbelSoftmax, dropout(0.4),
% fc(4000), relu, bn, dropout(0.4),
% fc(4000), relu, bn, dropout(0.4), fc($input$), sigmoid].
% Here, Input layer has the same dimension as the image size.
% fc = fully connected layer, bn = Batch Normalization,
% and tensors are reshaped accordingly.
% 
% The network is trained to minimize the sum of the variational loss and
% the reconstruction loss (binary cross-entropy) using Adam optimizer,
% learning rate (lr) 0.001, 1000 epochs, batch-size 1000.
% Training takes about 40 minutes with 1000 epochs on a single NVIDIA GTX-1070. %mid-range GPU.
% 
% The network tuned for hanoi with 4 disks in AMA$_1$ experiments consists of the following layers.
% Same training method was applied except the epoch is increased to 10000.
% 
% [Input($input$), GaussianNoise(0.1),
% fc(6000), relu, bn, dropout(0.4),
% fc(6000), relu, bn, dropout(0.4),
% fc(58), reshape(29x2), GumbelSoftmax, dropout(0.4),
% fc(6000), relu, bn, dropout(0.4),
% fc(6000), relu, bn, dropout(0.4), fc($input$), sigmoid].
% 
% 
% The network for AMA$_2$ experiments uses a convolutional network in the
% encoder, and fc layers in the decoder. The latent layer is reduced to 36
% bits and FC layer has 1000 nodes. The network was trained using
% lr:0.001, Adam optimizer, batch size 1000, and 300 epochs.
% Training takes about 20 minutes on the same hardware.
% 
% [Input($input$), GaussianNoise(0.1),
% conv(3,3,16), relu, bn, dropout(0.4),
% conv(3,3,16), relu, bn, dropout(0.4),
% fc(72), reshape(36x2), GumbelSoftmax,
% fc(1000), relu, bn, dropout(0.4),
% fc(1000), relu, bn, dropout(0.4), fc($input$), sigmoid].


The network uses a convolutional network in the encoder, and fc layers
in the decoder (\refig{fig:sae-detail}). The latent layer has 36 bits.
Input layer has the same dimension as the image size.
The network was trained using the Adam optimizer \cite{kingma2014adam}.
Learning rate (lr) starts at 0.001, and is decreased to 0.0001 at the half of the entire epoch.
In 8-puzzle domains, we used 150 epochs and batch-size 4000.
In LightsOut domains, we used 100 epochs and batch-size 2000, due to the larger size of the image.
In Hanoi, we used a channel size of 12 instead of 16 for convolutions, dropout 0.6, and batch-size 500.
Training takes about 15 minutes on a single NVIDIA GTX-1070.

\begin{figure}[htb]
\centering
\begin{tabular}{|l|}
 Input($input$),\\
 GaussianNoise(0.4),\\
 conv(3,3,16), tanh, bn, dropout(0.4),\\
 conv(3,3,16), tanh, bn, dropout(0.4),\\
 fc(72), reshape(36x2), GumbelSoftmax,\\
 fc(1000), relu, bn, dropout(0.4),\\
 fc(1000), relu, bn, dropout(0.4),\\
 fc($input$), sigmoid.
\end{tabular}
\caption{SAE implementation.
 Here, fc = fully connected layer, conv = convolutional layer, 
relu = Rectified Linear Unit,
bn = Batch Normalization, % \cite{ioffe2015batch} ,
and tensors are reshaped accordingly.}
\label{fig:sae-detail}
\end{figure}

% The last layers can be replaced with [fc($input\times 2$), GumbelSoftmax, TakeFirstRow]
% for better reconstruction when we can assume that the input image is binarized.

In all experiments, 
the annealing schedule of Gumbel-Softmax is $\tau \leftarrow \max (0.7, \tau_0\exp(-rt))$ where
 $t$ is the current training epoch, $\tau_0$ is the initial temperature and $r$ is an annealing ratio.
We chose $\tau_0,r$ so that $\tau = 5.0$ when the training starts and $\tau = 0.7$ when the training finishes.
The above schedule is similar to the original schedule in  \citeauthor{jang2016categorical} (\citeyear{jang2016categorical}).

\subsection{State Augmentation}

As mentioned in the paper, the number of bits should be larger than the minimum encoding length $\log_2 |S|$
of the entire state space $S$.
36 bits in the latent layer sufficiently covers
the total number of states in any of the problems that are used in the experiments.

However, excessive latent space capacity (number of bits) is also harmful.
Due to the nature of Gumbel-Softmax, which uses Gumbel random distribution,
excessive number of bits results in meaningless bits that does not affect the decoder output.
These bits act like purely random variables and cause multiple symbolic states to represent the same image.
This causes an undesirable behavior in the latent space,
since it could make the search graph disconnected.

One way to obtain a connected search graph under this condition is
what we call \emph{state augmentation},
which encodes the same image several times and simply sample the bitvectors for an image.
This technique is used in the Towers of Hanoi (ToH) AMA$_1$ experiments, as ToH has the small search space.

In general, there is a tradeoff: The larger the latent space capacity, the easier it is to train the SAE,
but the latent space becomes more stochastic.
Thus, it is desirable to reduce the latent capacity with further engineering,
while trying to connect the search graph with sampling.

%While this means that there are some redundancy in the encoding,
% further reducing the length of latent vector requires the further tuning \& engineering to
% improve network performance.

\section{Action AutoEncoder}


\begin{figure}[htb]
\centering
\begin{tabular}{|l|}
 Input(36),                                                  \\
 concatenate(s), fc(400), relu, bn, dropbout(0.4),           \\
 concatenate(s), fc(400), relu, bn, dropbout(0.4),           \\
 concatenate(s), fc(128), reshape(1x128), GumbelSoftmax,     \\
 concatenate(s), fc(400), relu, bn, dropbout(0.4),           \\
 concatenate(s), fc(400), relu, bn, dropbout(0.4),           \\
 fc(36), sigmoid
\end{tabular}
\caption{AAE implementation. Here, fc = fully connected layer, bn = Batch Normalization,
and tensors are reshaped accordingly.}
\label{fig:aae-detail}
\end{figure}

AAE consists of the layers as shown in \refig{fig:aae-detail}.
The input takes the successor state $t$ (36bit) and concatenate(s) concatenate the input with the before-state $s$ (36bit).
The output of the network is $\tilde{t}$, a 36bit reconstruction of $t$.
The network was trained with lr:0.001, Adam, batch size 2000, 1000 epochs,
to minimize the reconstruction loss $|t-\tilde{t}|$ in terms of binary cross-entropy.
Training takes about 5 min.

In all experiments, the annealing schedule of Gumbel-Softmax is $\tau \leftarrow \max (0.1, \tau_0\exp(-rt))$.
We chose $\tau_0, r$ so that $\tau = 5.0$ and $\tau = 0.1$ when the training finishes.


\section{Action Discriminator (AD)}

The Action Discriminator uses PU-learning framework \cite{elkan2008learning}
to learn a positive/negative binary classifier from a positive/mixed
dataset.

We first concatenate $s$ and $t$, resulting in a set of 72 bit binary vectors.
We prepare a vector whose length is the same as the number of data, and assign 1 to the positive data,
and 0 to the mixed data.
$p$ and $m$ are concatenated, shuffled, and divided into training set (90\%) and the validation set (10\%).

To classify $p_1$ and $m_1$, we trained several networks shown in \refig{fig:ad-detail} and
chose the one which achieved the best accuracy.
The network is trained using Adam, lr:0.001, 3000 epochs with early stopping, batch size 1000,
using binary cross-entropy loss. Each training takes 2-10 minutes depending on the domain.

\begin{figure}[htb]
\centering
\begin{tabular}{|l|}
 Input(72), \\
 \ [bn, fc(300), relu, dropout($X$)] $\times Y$, \\
 fc(1), sigmoid.
\end{tabular}
\caption{AD implementation. It has metaparameters $X,Y$, where
$X\in \braces{0.5,0.8}$, $Y\in \braces{1,2}$, resulting in 4 configurations in total.
Depending on the value of $Y$, it becomes a single-layer or a two-layer perceptron.
}
\label{fig:ad-detail}
\end{figure}

\section{State Discriminator}

The State Discriminator also uses the PU-learning framework \cite{elkan2008learning}.
The dataset is prepared as described in the paper, and divided into training and validation set (90\% and 10\%).
We use the following single layer perceptron:
[Input(36), bn, fc(50), relu, dropout(0.8), fc(1), sigmoid].
% 
The network is trained using Adam, lr:0.0001, 3000 epochs with early stopping, batch size 1000,
using binary cross-entropy loss.

\section{Domain Details}

\label{sec:domain-details}

\subsection{MNIST 8-puzzle}

This is an image-based version of the 8-puzzle, where tiles contain
hand-written digits (0-9) from the MNIST database
\cite{lecun1998gradient}. Each digit is shrunk to 14x14 pixels, so each
state of the puzzle is a 42x42 image.  Valid moves in this domain swap
the ``0'' tile with a neighboring tile, i.e., the ``0'' serves as the
``blank'' tile in the classic 8-puzzle.  The entire state space consists
of 362880 states ($9!$) and 967680 actions.  From any specific goal state, the reachable
number of states is 181440 ($9!/2$).  Note that the same image is used
for each digit in all states, e.g., the tile for the ``1'' digit is the
same image in all states.

We provide 20000 random transition images as $Tr$.
This contains 2x20000 images including the duplicates.
SAE learns from these 40000 images.
Next, SAE generates latent vectors of 2x20000 images, then use them as the input to AAE, AD and SD.
In all cases training:validation ratio 9:1 is maintained
 (i.e. only 36000 images and 18000 transitions are used for training).

\subsection{Mandrill, Spider 8-Puzzle}

These are 8-puzzles generated by cutting and scrambling real photographs
(similar to sliding tile puzzle toys sold in stores). We used the
``Mandrill'' and ``Spider'' images, two of the standard benchmark in the image processing
literature.  The image was first converted to greyscale and then
% rounded to black/white (0/1) values
histogram-normalization and contrast enhancement was applied.
The same number of transitions as in the MNIST-8puzzle experiments are used.

\subsection{LightsOut}

A video game where a grid of lights is in some on/off configuration ($+$: On),
and pressing a light toggles its state (On/Off) as well as the state of all of its neighbors.
The goal is all lights Off.
(cf. \url{https://en.wikipedia.org/wiki/Lights_Out_(game)})
Unlike the 8-puzzle where each move affects only two adjacent tiles, a single operator in 4x4 LightsOut  can simultaneously flip 5/16 locations.
Also, unlike 8-puzzle and ToH, the LightsOut game allows some ``objects'' (lights) to disappear.
This demonstrates that \latentplanner is not limited to domains with highly local effects and static objects.

The image dimension is 36x36 and the size of each button ($+$ button) is 9x9.
4x4 LightsOut has $2^{16}=65536$ states and $16\times 2^{16}=1048576$ transitions.
Similar to the 8-puzzle instances, we used 20000 transitions.
Training:validation ratio 9:1 is maintained (i.e. only 36000 images and 18000 transitions are used for training).

\subsection{Twisted LightsOut}

The images have the same structure as LightsOut, but 
we additionally applied a swirl effect available in scikit-image package.
The effect is applied to the center, with strength=3, linear interpolation, 
and radius equal to 0.75 times the dimension of the image.

The image dimension is 36x36. 
Before the swirl effect is applied,
the size of each button ($+$ button) was 9x9.

\subsection{Towers of Hanoi}

Disks of various sizes must be moved from one peg to another, with the
constraint that a larger disk can never be placed on top of a
smaller disk.
Each input image has a dimension of $16\times 60$ (resp.),
where each disk is presented as a 4px line segment.

Due to the smaller state space ($3^d$ states for $d$ disks: 81 states, 240 transitions for 4 disks)
compared to the other domains tested in this paper,
we used 90\% of states as the training examples in AMA$_1$ experiments,
and verified on the 10\% validation set that the network is generalizing.

We also applied the state augmentation technique described in \refsec{sec:SAE-detail},
as the detrimental effect of excessive number of bits in the latent space becomes more obvious in this domain.

\section{Planner details}

\subsection{Planner in AMA$_1$ experiments}
\label{sec:ama1-planner}

In the AMA$_1$ experiments (\refsec{sec:ama1-experiments}),
we found that the invariant detection
routines in the Fast Downward PDDL to SAS translator (translate.py)
became a bottleneck.
This is because the PDDL represent individual transitions as ground actions, whose number is very large.
In order to run the experiments in \refsec{sec:ama1-experiments} % and \refsec{sec:heuristics-evaluation}
,
we wrote a trivial, replacement PDDL to SAS converter without the invariant detection.
Still, each experiment may require more than 7GB memory and 4 hours on a Xeon E6-2676 CPU.
Most of the runtime was spent on the preprocessing, and the search takes only a few seconds.

\subsection{Planner in AMA$_2$ experiments}

In the AMA$_2$ experiments (\refsec{sec:ama2-experiments}), we implemented a trivial A* planner in python.
Although this implementation could be hugely inefficient compared to the traditional native-complied solvers,
the performance is not our concern.
% 
In fact, the most time-consuming step is the generation and the filtering of the successor states using AAE, AD etc.,
and the low-level implementation detail is not the bottleneck.

The goal-count heuristics is based on the bitwise difference between the
latent representation of the goal image and the current state.

\section{Statement on Reproducibility}

To facilitate reproducibility of the experiments, the entire source code of the system and the pre-trained network weights
 will be made public on Github (\url{https://github.com/guicho271828/latplan}).



% not necessary in journal
\fontsize{9.5pt}{10.5pt}
\selectfont
 
% \bibliographystyle{aaai} %XXXTODO: for compression, the .bst was hacked so that 3 authors is displayed at "Firstauthor et al."   For the final version, needs to be reverted to the official bst so that 3 authors showsas "Firstauthor, 2ndauthor, and 3rdauthor"
\begin{thebibliography}{}

\bibitem[\protect\citeauthoryear{Arfaee \bgroup \em et al.\egroup
  }{2011}]{ArfaeeZH11}
Shahab~Jabbari Arfaee, Sandra Zilles, and Robert~C. Holte.
\newblock {Learning Heuristic Functions for Large State Spaces}.
\newblock {\em {Artificial Intelligence}}, 175(16-17):2075--2098, 2011.

\bibitem[\protect\citeauthoryear{Argall \bgroup \em et al.\egroup
  }{2009}]{ArgallCVB09}
Brenna Argall, Sonia Chernova, Manuela~M. Veloso, and Brett Browning.
\newblock {A Survey of Robot Learning from Demonstration}.
\newblock {\em Robotics and Autonomous Systems}, 57(5):469--483, 2009.

\bibitem[\protect\citeauthoryear{Asai and Fukunaga}{2015}]{Asai2015}
Masataro Asai and Alex Fukunaga.
\newblock {Solving Large-Scale Planning Problems by Decomposition and Macro
  Generation}.
\newblock In {\em {ICAPS}}, Jerusalem, Israel, June 2015.

\bibitem[\protect\citeauthoryear{Bacchus}{2000}]{bacchus2000}
Fahiem Bacchus.
\newblock {Subset of {PDDL} for the {AIPS2000} Planning Competition}.
\newblock In {\em {IPC}}, 2000.

\bibitem[\protect\citeauthoryear{B{\"a}ckstr{\"o}m and
  Nebel}{1995}]{backstrom1995complexity}
Christer B{\"a}ckstr{\"o}m and Bernhard Nebel.
\newblock {Complexity Results for SAS+ Planning}.
\newblock {\em Computational Intelligence}, 11(4):625--655, 1995.

\bibitem[\protect\citeauthoryear{Barbu \bgroup \em et al.\egroup
  }{2010}]{BarbuNS10}
Andrei Barbu, Siddharth Narayanaswamy, and Jeffrey~Mark Siskind.
\newblock {Learning Physically-Instantiated Game Play through Visual
  Observation}.
\newblock In {\em {ICRA}}, pages 1879--1886, 2010.

\bibitem[\protect\citeauthoryear{Bieszczad and
  Kuchar}{2015}]{bieszczad2015neurosolver}
Andrzej Bieszczad and Skyler Kuchar.
\newblock {Neurosolver Learning to Solve Towers of Hanoi Puzzles}.
\newblock In {\em {IJCCI}}, volume~3, pages 28--38. IEEE, 2015.

\bibitem[\protect\citeauthoryear{Bonet}{2013}]{bonet2013admissible}
Blai Bonet.
\newblock {An Admissible Heuristic for SAS+ Planning Obtained from the State
  Equation.}
\newblock In {\em {IJCAI}}, 2013.

\bibitem[\protect\citeauthoryear{Cresswell \bgroup \em et al.\egroup
  }{2013}]{CresswellMW13}
Stephen Cresswell, Thomas~Leo McCluskey, and Margaret~Mary West.
\newblock Acquiring planning domain models using \emph{LOCM}.
\newblock {\em Knowledge Eng. Review}, 28(2):195--213, 2013.

\bibitem[\protect\citeauthoryear{Cullen and Bryman}{1988}]{cullen88}
J~Cullen and A~Bryman.
\newblock The knowledge acquisition bottleneck: Time for reassessment?
\newblock {\em Expert Systems}, 5(3), August 1988.

\bibitem[\protect\citeauthoryear{Deng \bgroup \em et al.\egroup
  }{2009}]{deng2009imagenet}
Jia Deng, Wei Dong, Richard Socher, Li-Jia Li, Kai Li, and Li~Fei-Fei.
\newblock {ImageNet: A Large-Scale Hierarchical Image Database}.
\newblock In {\em {CVPR}}, pages 248--255. IEEE, 2009.

\bibitem[\protect\citeauthoryear{Deng \bgroup \em et al.\egroup
  }{2013}]{deng2013new}
Li~Deng, Geoffrey Hinton, and Brian Kingsbury.
\newblock {New Types of Deep Neural Network Learning for Speech Recognition and
  Related Applications: An Overview}.
\newblock In {\em {ICASSP}}, pages 8599--8603. IEEE, 2013.

\bibitem[\protect\citeauthoryear{Elkan and Noto}{2008}]{elkan2008learning}
Charles Elkan and Keith Noto.
\newblock {Learning Classifiers from Only Positive and Unlabeled Data}.
\newblock In {\em Proceedings of the 14th ACM SIGKDD international conference
  on Knowledge discovery and data mining}, pages 213--220. ACM, 2008.

\bibitem[\protect\citeauthoryear{Frances \bgroup \em et al.\egroup
  }{2017}]{frances2017purely}
Guillem Frances, Miquel Ram{\i}rez, Nir Lipovetzky, and Hector Geffner.
\newblock {Purely Declarative Action Representations are Overrated: Classical
  Planning with Simulators}.
\newblock In {\em {IJCAI}}, pages 4294--4301, 2017.

\bibitem[\protect\citeauthoryear{Graves \bgroup \em et al.\egroup
  }{2016}]{neuraltm}
Alex Graves, Greg Wayne, Malcolm Reynolds, Tim Harley, Ivo Danihelka, Agnieszka
  Grabska-Barwi{\'n}ska, Sergio~G{\'o}mez Colmenarejo, Edward Grefenstette,
  Tiago Ramalho, John Agapiou, et~al.
\newblock {Hybrid Computing using a Neural Network with Dynamic External
  Memory}.
\newblock {\em Nature}, 538(7626):471--476, 2016.

\bibitem[\protect\citeauthoryear{Gumbel and
  Lieblein}{1954}]{gumbel1954statistical}
Emil~Julius Gumbel and Julius Lieblein.
\newblock Statistical theory of extreme values and some practical applications:
  a series of lectures.
\newblock 1954.

\bibitem[\protect\citeauthoryear{Hart \bgroup \em et al.\egroup
  }{1968}]{hart1968formal}
Peter~E. Hart, Nils~J. Nilsson, and Bertram Raphael.
\newblock {A Formal Basis for the Heuristic Determination of Minimum Cost
  Paths}.
\newblock {\em Systems Science and Cybernetics, IEEE Transactions on},
  4(2):100--107, 1968.

\bibitem[\protect\citeauthoryear{Helmert and Domshlak}{2009}]{Helmert2009}
Malte Helmert and Carmel Domshlak.
\newblock {Landmarks, Critical Paths and Abstractions: What's the Difference
  Anyway?}
\newblock In {\em {ICAPS}}, 2009.

\bibitem[\protect\citeauthoryear{Helmert \bgroup \em et al.\egroup
  }{2007}]{helmert2007flexible}
Malte Helmert, Patrik Haslum, and J{\"o}rg Hoffmann.
\newblock {Flexible Abstraction Heuristics for Optimal Sequential Planning}.
\newblock In {\em {ICAPS}}, pages 176--183, 2007.

\bibitem[\protect\citeauthoryear{Helmert}{2004}]{Helmert04}
Malte Helmert.
\newblock {A Planning Heuristic Based on Causal Graph Analysis}.
\newblock In {\em {ICAPS}}, pages 161--170, 2004.

\bibitem[\protect\citeauthoryear{Helmert}{2006}]{Helmert2006}
Malte Helmert.
\newblock {The Fast Downward Planning System}.
\newblock {\em {J. Artif. Intell. Res.(JAIR)}}, 26:191--246, 2006.

\bibitem[\protect\citeauthoryear{Hinton and
  Salakhutdinov}{2006}]{hinton2006reducing}
Geoffrey~E Hinton and Ruslan~R Salakhutdinov.
\newblock {Reducing the Dimensionality of Data with Neural Networks}.
\newblock {\em Science}, 313(5786):504--507, 2006.

\bibitem[\protect\citeauthoryear{Hoffmann and Nebel}{2001}]{hoffmann01}
J{\"o}rg Hoffmann and Bernhard Nebel.
\newblock {The FF Planning System: Fast Plan Generation through Heuristic
  Search}.
\newblock {\em {J. Artif. Intell. Res.(JAIR)}}, 14:253--302, 2001.

\bibitem[\protect\citeauthoryear{Hopfield and Tank}{1985}]{hopfield1985neural}
John~J Hopfield and David~W Tank.
\newblock {"Neural" Computation of Decisions in Optimization Problems}.
\newblock {\em Biological Cybernetics}, 52(3):141--152, 1985.

\bibitem[\protect\citeauthoryear{Jang \bgroup \em et al.\egroup
  }{2017}]{jang2016categorical}
Eric Jang, Shixiang Gu, and Ben Poole.
\newblock {Categorical Reparameterization with Gumbel-Softmax}.
\newblock In {\em {ICLR}}, 2017.

\bibitem[\protect\citeauthoryear{Kaiser}{2012}]{Kaiser12}
Lukasz Kaiser.
\newblock {Learning Games from Videos Guided by Descriptive Complexity}.
\newblock In {\em {AAAI}}, 2012.

\bibitem[\protect\citeauthoryear{Kingma and Ba}{2014}]{kingma2014adam}
Diederik Kingma and Jimmy Ba.
\newblock {Adam: A Method for Stochastic Optimization}.
\newblock {\em arXiv preprint arXiv:1412.6980}, 2014.

\bibitem[\protect\citeauthoryear{Kingma and Welling}{2013}]{kingma2013auto}
Diederik~P Kingma and Max Welling.
\newblock {Auto-Encoding Variational Bayes}.
\newblock In {\em {ICLR}}, 2013.

\bibitem[\protect\citeauthoryear{Kingma \bgroup \em et al.\egroup
  }{2014}]{kingma2014semi}
Diederik~P Kingma, Shakir Mohamed, Danilo~Jimenez Rezende, and Max Welling.
\newblock {Semi-Supervised Learning with Deep Generative Models}.
\newblock In {\em {NIPS}}, pages 3581--3589, 2014.

\bibitem[\protect\citeauthoryear{Konidaris \bgroup \em et al.\egroup
  }{2014}]{KonidarisKL14}
George Konidaris, Leslie~Pack Kaelbling, and Tom{\'a}s Lozano{-}P{\'e}rez.
\newblock {Constructing Symbolic Representations for High-Level Planning}.
\newblock In {\em {AAAI}}, pages 1932--1938, 2014.

\bibitem[\protect\citeauthoryear{LeCun \bgroup \em et al.\egroup
  }{1998}]{lecun1998gradient}
Yann LeCun, L{\'e}on Bottou, Yoshua Bengio, and Patrick Haffner.
\newblock {Gradient-Based Learning Applied to Document Recognition}.
\newblock {\em {Proc. of the IEEE}}, 86(11):2278--2324, 1998.

\bibitem[\protect\citeauthoryear{Lindsay \bgroup \em et al.\egroup
  }{2017}]{lindsay2017framer}
Alan Lindsay, Jonathon Read, Joao~F Ferreira, Thomas Hayton, Julie Porteous,
  and Peter~J Gregory.
\newblock {Framer: Planning Models from Natural Language Action Descriptions}.
\newblock In {\em {ICAPS}}, 2017.

\bibitem[\protect\citeauthoryear{Maddison \bgroup \em et al.\egroup
  }{2014}]{maddison2014sampling}
Chris~J Maddison, Daniel Tarlow, and Tom Minka.
\newblock A* sampling.
\newblock In {\em {NIPS}}, pages 3086--3094, 2014.

\bibitem[\protect\citeauthoryear{McDermott}{2000}]{McDermott00}
Drew~V. McDermott.
\newblock {The 1998 {AI} Planning Systems Competition}.
\newblock {\em {AI} Magazine}, 21(2):35--55, 2000.

\bibitem[\protect\citeauthoryear{Mnih \bgroup \em et al.\egroup }{2015}]{dqn}
Volodymyr Mnih, Koray Kavukcuoglu, David Silver, Andrei~A Rusu, Joel Veness,
  Marc~G Bellemare, Alex Graves, Martin Riedmiller, Andreas~K Fidjeland, Georg
  Ostrovski, et~al.
\newblock {Human-Level Control through Deep Reinforcement Learning}.
\newblock {\em Nature}, 518(7540):529--533, 2015.

\bibitem[\protect\citeauthoryear{Mour{\~a}o \bgroup \em et al.\egroup
  }{2012}]{MouraoZPS12}
Kira Mour{\~a}o, Luke~S. Zettlemoyer, Ronald P.~A. Petrick, and Mark Steedman.
\newblock {Learning {STRIPS} Operators from Noisy and Incomplete Observations}.
\newblock In {\em {UAI}}, pages 614--623, 2012.

\bibitem[\protect\citeauthoryear{Ren \bgroup \em et al.\egroup
  }{2015}]{ren2015faster}
Shaoqing Ren, Kaiming He, Ross Girshick, and Jian Sun.
\newblock {Faster R-CNN: Towards Real-time Object Detection with Region
  Proposal Networks}.
\newblock In {\em {NIPS}}, pages 91--99, 2015.

\bibitem[\protect\citeauthoryear{Richter \bgroup \em et al.\egroup
  }{2008}]{richter2008landmarks}
Silvia Richter, Malte Helmert, and Matthias Westphal.
\newblock {Landmarks Revisited}.
\newblock In {\em {AAAI}}, 2008.

\bibitem[\protect\citeauthoryear{Satzger and Kramer}{2013}]{SatzgerK13}
Benjamin Satzger and Oliver Kramer.
\newblock {Goal Distance Estimation for Automated Planning using Neural
  Networks and Support Vector Machines}.
\newblock {\em Natural Computing}, 12(1):87--100, 2013.

\bibitem[\protect\citeauthoryear{Sievers \bgroup \em et al.\egroup
  }{2012}]{sievers2012efficient}
Silvan Sievers, Manuela Ortlieb, and Malte Helmert.
\newblock {Efficient Implementation of Pattern Database Heuristics for
  Classical Planning.}
\newblock In {\em {SOCS}}, 2012.

\bibitem[\protect\citeauthoryear{Silver \bgroup \em et al.\egroup
  }{2016}]{alphago}
David Silver, Aja Huang, Chris~J Maddison, Arthur Guez, Laurent Sifre, George
  {Van Den Driessche}, Julian Schrittwieser, Ioannis Antonoglou, Veda
  Panneershelvam, Marc Lanctot, et~al.
\newblock {Mastering the Game of Go with Deep Neural Networks and Tree Search}.
\newblock {\em Nature}, 529(7587):484--489, 2016.

\bibitem[\protect\citeauthoryear{Srivastava \bgroup \em et al.\egroup
  }{2015}]{srivastava2015unsupervised}
Nitish Srivastava, Elman Mansimov, and Ruslan Salakhudinov.
\newblock {Unsupervised Learning of Video Representations using LSTMs}.
\newblock In {\em {ICML}}, pages 843--852, 2015.

\bibitem[\protect\citeauthoryear{Steels}{2008}]{Steels2008}
Luc Steels.
\newblock {The Symbol Grounding Problem has been Solved. So What's Next?}
\newblock In Manuel de~Vega, Arthur Glenberg, and Arthur Graesser, editors,
  {\em {Symbols and Embodiment}}. Oxford University Press, 2008.

\bibitem[\protect\citeauthoryear{Vincent \bgroup \em et al.\egroup
  }{2008}]{vincent2008extracting}
Pascal Vincent, Hugo Larochelle, Yoshua Bengio, and Pierre-Antoine Manzagol.
\newblock {Extracting and Composing Robust Features with Denoising
  Autoencoders}.
\newblock In {\em {ICML}}, pages 1096--1103. ACM, 2008.

\bibitem[\protect\citeauthoryear{Yang \bgroup \em et al.\egroup
  }{2007}]{YangWJ07}
Qiang Yang, Kangheng Wu, and Yunfei Jiang.
\newblock {Learning Action Models from Plan Examples using Weighted {MAX-SAT}}.
\newblock {\em {Artificial Intelligence}}, 171(2-3):107--143, 2007.

\end{thebibliography}

% ijcai
% \bibliographystyle{named}
% jsai
% \bibliographystyle{jsai}
% ecai
% \bibliographystyle{ecai}
% jair
\bibliographystyle{named}

\end{document}
