\begin{table}[!th]
\caption{
\footnotesize
Stability of the proposed pointwise QPP metric APAE with respect to listwise approach, across different pairs of IR metrics and IR models. Red cells indicate the lowest value in each group, while the lowest values along each column are bold-faced.
\label{tab:sdres}
}
\vskip 0.5em
\begin{subtable}[t]{.48\linewidth}
\centering
\begin{adjustbox}{width=\textwidth}
%\begin{tabular}{@{}l@{~~}lcccccc@{}}
\begin{tabular}{@{}l@{~~}l@{~~}c@{~~}c@{~~}c@{~~}c@{~~}c@{~~}c@{}}

\toprule

Model & Metric & 
AP@100 & 
R@10 & R@100 & 
nDCG@10 & nDCG@100 \\

% & &
% @100 &
% @10 & @100 &
% @10 & @100 \\

\midrule

\lmjm & \multirow{3}{*}{AP@10} & 
0.497 & 
0.813 & \cellcolor{melon}0.429 &
0.783 & \cellcolor{melon}0.429 \\

BM25 & & 
0.897 & 
0.722 & 0.722 &
0.793 & 0.793 \\

\lmdr & & 
0.897 & 
0.786 & 0.786 &
0.823 & 0.905 \\

\cmidrule{1-2}

\lmjm & \multirow{3}{*}{AP@100} & \nores & 
\cellcolor{melon}\textbf{0.328} & 0.811 &
0.363 & 0.783 \\

BM25 & & 
\nores & 
0.783 & 0.784 &
0.714 & 0.642 \\

\lmdr & & 
\nores & 
0.823 & 0.901 &
0.834 & 0.789 \\

\cmidrule{1-2}

% \lmjm & AP & 
% \nores & \nores & 
% \cellcolor{melon}0.346 & 0.655 &
% 0.389 & 0.733 \\

% BM25 & $@1000$ & 
% \nores & \nores & 
% 0.6731 & 0.6731 & 0.5321 &
% 0.8643 & 0.8522 & 0.7095 \\

% \lmdr & & 
% \nores & \nores & 
% 0.8962 & 0.7845 & 0.4539 &
% 0.8790 & 0.8235 & 0.8790 \\

% \cmidrule{1-2}

\lmjm & \multirow{3}{*}{R@10} & 
\nores & 
\nores & 0.624 &
0.893 & \cellcolor{melon}0.503 \\

BM25 & & 
\nores & 
\nores & 0.803 &
0.982 & 0.894 \\

\lmdr & & 
\nores & 
\nores & 0.903 &
0.864 & 0.864 \\

\cmidrule{1-2}

\lmjm & \multirow{3}{*}{R@100} & 
\nores & \nores & 
\nores &
0.852 & 0.804 \\

BM25 & & 
\nores & \nores & 
\nores &
0.786 & 0.890 \\

\lmdr & & 
\nores & \nores & 
\nores &
\cellcolor{melon}0.738 & \cellcolor{melon}0.738 \\

\cmidrule{1-2}

% \lmjm & R & 
% \nores & \nores & 
% \nores & \nores & \nores &
% \cellcolor{melon}0.4269 & 0.9032 & 0.7809 \\ 

% BM25 & $@1000$ & 
% \nores & \nores & 
% \nores & \nores & \nores &
% 0.9078 & 0.8231 & 0.7238 \\

% \lmdr & & 
% \nores & \nores & 
% \nores & \nores & \nores &
% 0.9048 & 0.9048 & 0.7238 \\

% \cmidrule{1-2}

\lmjm & \multirow{3}{*}{nDCG@10} & 
\nores & \nores & 
\nores & \nores& \cellcolor{melon}0.537 \\

BM25 & & 
\nores & \nores & 
\nores &
\nores & 0.904 \\

\lmdr & & 
\nores & \nores & 
\nores &
\nores & 0.868 \\

% \cmidrule{1-2}

% \lmjm & nDCG & 
% \nores & \nores & 
% \nores & \nores & \nores &
% \nores & \nores & 0.9043 \\

% BM25 & $@100$ & 
% \nores & \nores & 
% \nores & \nores & \nores &
% \nores & \nores & \cellcolor{melon}0.8642 \\

% \lmdr & & 
% \nores & \nores & 
% \nores & \nores & \nores &
% \nores & \nores & 0.8956 \\

\bottomrule
\label{tab:metric_tau_tau}
\end{tabular}
\end{adjustbox}
\caption{\footnotesize Correlations between the relative ranks of 7 different QPP systems across different pairs of IR target metrics. QPP systems were evaluated with the baseline listwise metric - Kendall's $\tau$. 
\label{stab:tab1}
}
\end{subtable}%
\quad
\begin{subtable}[t]{.48\linewidth}
\centering
\begin{adjustbox}{width=\textwidth}
%\begin{tabular}{@{}l@{~~}lccccc@{}}
\begin{tabular}{@{}l@{~~}l@{~~}c@{~~}c@{~~}c@{~~}c@{~~}c@{}}
\toprule

Model & Metric & 
AP@100 & 
R@10 & R@100 & 
nDCG@10 & nDCG@100 \\

% & &
% @100 &
% @10 & @100 &
% @10 & @100 \\

\midrule

\lmjm & \multirow{3}{*}{AP@10} & 
0.904 & 
1.000 & \cellcolor{melon}0.715 &
1.000 & 0.792 \\

BM25 & & 
1.000 & 
1.000 & 1.000 &
1.000 & 1.000 \\

\lmdr & & 
1.000 & 
1.000 & 1.000 &
1.000 & 1.000 \\

\cmidrule{1-2}

\lmjm & \multirow{3}{*}{AP@100} & 
\nores & 
0.905 & 0.811 &
\cellcolor{melon}0.669 & 1.000 \\

BM25 & & 
\nores & 
1.000 & 1.000 &
1.000 & 1.000 \\

\lmdr & & 
\nores & 
1.000 & 1.000 &
1.000 & 1.000 \\

% \cmidrule{1-2}

% \lmjm & AP & 
% \nores & \nores & 
% 0.723 & 0.811 & 0.811 &
% 0.573 & 1.000 & 1.000 \\

% BM25 & $@1000$ & 
% \nores & \nores & 
% 1.000 & 1.000 & 0.635 &
% 1.000 & 1.000 & 0.811 \\

% \lmdr & & 
% \nores & \nores & 
% 0.905 & 0.811 & \cellcolor{melon}\textbf{0.524} &
% 0.905 & 0.905 & 0.811 \\

\cmidrule{1-2}

\lmjm & \multirow{3}{*}{R@10} & 
\nores & 
\nores & 0.603 &
0.905 & \cellcolor{melon}\textbf{0.542} \\

BM25 & & 
\nores & 
\nores & 1.000 &
1.000 & 1.000 \\

\lmdr & & 
\nores & 
\nores & 1.000 &
1.000 & 1.000 \\

\cmidrule{1-2}

\lmjm & \multirow{3}{*}{R@100} & 
\nores & \nores & 
\nores &
\cellcolor{melon}0.654 & 1.000 \\

BM25 & & 
\nores & \nores & 
\nores &
1.000 & 1.000 \\

\lmdr & & 
\nores & \nores & 
\nores &
1.000 & 1.000 \\

\cmidrule{1-2}

% \lmjm & R & 
% \nores & \nores & 
% \nores & \nores & \nores &
% \cellcolor{melon}0.627 & 0.903 & 0.781 \\ 

% BM25 & $@1000$ & 
% \nores & \nores & 
% \nores & \nores & \nores &
% 0.908 & 0.823 & 0.724 \\

% \lmdr & & 
% \nores & \nores & 
% \nores & \nores & \nores &
% 0.905 & 0.905 & 0.823 \\

% \cmidrule{1-2}

\lmjm & \multirow{3}{*}{nDCG@10} & 
\nores & \nores & 
\nores & \nores &
\cellcolor{melon}0.649 \\

BM25 & & 
\nores & \nores & 
\nores & \nores &
1.000 \\

\lmdr & & 
\nores & \nores & 
\nores & \nores &
1.000 \\

% \cmidrule{1-2}

% \lmjm & nDCG & 
% \nores & \nores & 
% \nores & \nores & \nores &
% \nores & \nores & 1.000 \\

% BM25 & $@100$ & 
% \nores & \nores & 
% \nores & \nores & \nores &
% \nores & \nores & 1.000 \\

% \lmdr & & 
% \nores & \nores & 
% \nores & \nores & \nores &
% \nores & \nores & 1.000 \\

\bottomrule
\label{tab:metric_apae_tau}
\end{tabular}
\end{adjustbox}
\caption{\footnotesize Similar to Table \ref{stab:tab1}, except QPP performance was evaluated with the pointwise approach \proposed. A comparison with Table \ref{stab:tab1} indicates a better consistency in the relative ranks of QPP systems for variations in the IR metrics.
\label{stab:tab2}
}
\end{subtable}
%
\begin{subtable}[t]{.48\linewidth}
\centering
\begin{adjustbox}{width=\textwidth}
%\begin{tabular}{@{}l@{~~}cccccccc@{}}
\begin{tabular}{@{}l@{~~}c@{~~}c@{~~}c@{~~}c@{~~}c@{~~}c@{~~}c@{~~}c@{}}

\toprule

\multirow{2}{*}{Metric} & \multirow{2}{*}{Model} & 
\lmjm &
BM25 & BM25 & 
\lmdr & \lmdr \\

& & ($0.6$) &
($0.7, 0.3$) & ($0.3, 0.7$) &
($500$) & ($1000$) \\

\midrule

AP@100 & & 
0.826 & 
0.904 & 0.819 & 
0.714 & 0.895 \\

nDCG@100 & \lmjm & 
0.780 & 
\cellcolor{melon}0.694 & 0.695 & 
0.759 & 0.759 \\

R@100 & ($0.3$) & 
0.824 & 
0.769 & 0.782 & 
0.904 & 0.904 \\

% P@10 & & 
% 0.883 & 
% 0.762 & 0.792 & 
% 0.711 & 0.674 \\

\cmidrule{1-2}

AP@100 & &  
\nores & 
0.703 & 0.712 & 
0.904 & 0.823 \\

nDCG@100 & \lmjm & 
\nores & 
0.781 & 0.827 & 
0.811 & 0.811 \\

R@100 & ($0.6$) & 
\nores & 
0.813 & 0.725 & 
0.731 & \cellcolor{melon}\textbf{0.675} \\

% P@10 & & 
% \nores & 
% 0.813 & 0.806 & 
% 0.710 & 0.706 \\

\cmidrule{1-2}

AP@100 & &
\nores & 
\nores & 0.903 & 
0.785 & 0.785 \\

nDCG@100 & BM25 &
\nores & 
\nores & 0.897 & 
0.786 & 0.786 \\

R@100 & ($0.7, 0.3$) &
\nores & 
\nores & 0.812 & 
\cellcolor{melon}0.752 & 0.779 \\

% P@10 & & 
% \nores & 
% \nores & 0.884 & 
% 0.813 & \cellcolor{melon}0.731 \\

% \cmidrule{1-2}

% AP@100 & & 
% \nores & 
% \nores & \nores & 0.811 & 
% 0.884 & 0.732 & 0.884 \\

% nDCG@100 & BM25 &
% \nores & 
% \nores & \nores & 0.903 & 
% 0.782 & 0.884 & 0.812 \\

% R@100 & ($1.0, 1.0$) & 
% \nores & 
% \nores & \nores & 0.833 & 
% 0.855 & 0.853 & 0.851 \\

% P@10 & & 
% \nores & 
% \nores & \nores & 0.739 & 
% \cellcolor{melon}0.713 & 0.798 & 0.798 \\

\cmidrule{1-2}

AP@100 & &  
\nores & 
\nores & \nores & 
0.887 & \cellcolor{melon}0.882 \\

nDCG@100 & BM25 & 
\nores & 
\nores & \nores & 
0.901 & 0.895 \\

R@100 & ($0.3, 0.7$) & 
\nores & 
\nores & \nores & 
0.889 & 0.901 \\

% P@10 & & 
% \nores & 
% \nores & \nores & 
% \cellcolor{melon}0.794 & 0.843 \\

% \cmidrule{1-2}

% AP@100 & & 
% \nores & 
% \nores & \nores & \nores & 
% \nores & 0.912 & 0.904 \\

% nDCG@100 & \lmdr &
% \nores & 
% \nores & \nores & \nores & 
% \nores & 0.901 & 0.925 \\

% R@100 & ($100$) & 
% \nores & 
% \nores & \nores & \nores & 
% \nores & 0.895 & 0.913 \\

% P@10 & &  
% \nores & 
% \nores & \nores & \nores & 
% \nores & \cellcolor{melon}0.888 & 0.911 \\

\cmidrule{1-2}

AP@100 & &  
\nores & 
\nores & 
\nores & \nores & 0.901 \\

nDCG@100 & \lmdr & 
\nores & 
\nores & 
\nores & \nores & \cellcolor{melon}0.893 \\

R@100 & ($500$) &  
\nores & 
\nores & 
\nores & \nores & 0.903 \\

% P@10 & & 
% \nores & 
% \nores & 
% \nores & \nores & \cellcolor{melon}0.879 \\

\bottomrule
\end{tabular}
\end{adjustbox}
%\caption{\small Unlike Table \ref{stab:tab1}, the rank correlations between the relative ranks of QPP systems are measured across IR model pairs. As in Table \ref{stab:tab1}, QPP systems were evaluated with $\tau$. The numbers alongside the IR models denote their respective parameter values, e.g. $\lambda$ for \lmjm.
\caption{\footnotesize Here rank correlations between the relative ranks of QPP systems are measured across IR model pairs. As in Table \ref{stab:tab1}, QPP systems were evaluated with $\tau$. The numbers alongside the IR models denote their respective parameters.
% 
\label{stab:tab3}
}
\end{subtable}%
\quad
\begin{subtable}[t]{.48\linewidth}
\centering
\begin{adjustbox}{width=\textwidth}
%\begin{tabular}{@{}l@{~~}cccccc@{}}
\begin{tabular}{@{}l@{~~}c@{~~}c@{~~}c@{~~}c@{~~}c@{~~}c@{}}

\toprule

\multirow{2}{*}{Metric} & \multirow{2}{*}{Model} & 
\lmjm &
BM25 & BM25 & 
\lmdr & \lmdr \\

& & ($0.6$) &
($0.7, 0.3$) & ($0.3, 0.7$) &
($500$) & ($1000$) \\

\midrule

AP@100 & & 
1.000 & 
1.000 & 1.000 & 
1.000 & 1.000 \\

nDCG@100 & \lmjm & 
1.000 & 
0.864 & 1.000 & 
\cellcolor{melon}0.843 & 0.864 \\

R@100 & ($0.3$) & 
1.000 & 
0.864 & 1.000 & 
1.000 & 1.000 \\

% P@10 & & 
% 1.000 & 
% 0.811 & 0.811 & 
% \cellcolor{melon}0.803 & 1.000 \\

\cmidrule{1-2}

AP@100 & &  
\nores & 
1.000 & 1.000 & 
1.000 & 1.000 \\

nDCG@100 & \lmjm & 
\nores & 
0.914 & 1.000 & 
\cellcolor{melon}\textbf{0.813} & 0.914 \\

R@100 & ($0.6$) & 
\nores & 
1.000 & 1.000 & 
1.000 & 1.000 \\

% P@10 & & 
% \nores & 
% \cellcolor{melon}0.801 & 0.928 & 
% 0.824 & 1.000 \\

\cmidrule{1-2}

AP@100 & &
\nores & 
\nores & 1.000 & 
1.000 & 1.000 \\

nDCG@100 & BM25 &
\nores & 
\nores & 1.000 & 
1.000 & 1.000 \\

R@100 & ($0.7, 0.3$) &
\nores & 
\nores & \cellcolor{melon}0.812 & 
0.905 & 1.000 \\

% P@10 & & 
% \nores & 
% \nores & 1.000 & 
% 1.000 & \cellcolor{melon}\textbf{0.799} \\

% \cmidrule{1-2}

% AP@100 & & 
% \nores & 
% \nores & \nores & 1.000 & 
% 1.000 & 1.000 \\

% nDCG@100 & BM25 &
% \nores & 
% \nores & \nores & 1.000 & 
% 1.000 & 1.000 \\

% R@100 & ($1.0, 1.0$) & 
% \nores & 
% \nores & \nores & 1.000 & 
% 1.000 & 1.000 \\

% P@10 & & 
% \nores & 
% \nores & \nores & \cellcolor{melon}0.811 & 
% 0.923 & 1.000 \\

\cmidrule{1-2}

AP@100 & &  
\nores & 
\nores & \nores & 
1.000 & 1.000 \\

nDCG@100 & BM25 & 
\nores & 
\nores & \nores & 
1.000 & 1.000 \\

R@100 & ($0.3, 0.7$) & 
\nores & 
\nores & \nores & 
1.000 & 1.000 \\

% P@10 & & 
% \nores & 
% \nores & \nores & 
% 1.000 & \cellcolor{melon}0.843 \\

% \cmidrule{1-2}

% AP@100 & & 
% \nores & 
% \nores & \nores & \nores & 
% \nores & 1.000 & 1.000 \\

% nDCG@100 & \lmdr &
% \nores & 
% \nores & \nores & \nores & 
% \nores & 1.000 & 1.000 \\

% R@100 & ($100$) & 
% \nores & 
% \nores & \nores & \nores & 
% \nores & 1.000 & 1.000 \\

% P@10 & &  
% \nores & 
% \nores & \nores & \nores & 
% \nores & \cellcolor{melon}0.811 & \cellcolor{melon}0.811 \\

\cmidrule{1-2}

AP@100 & &  
\nores & 
\nores & 
\nores & \nores & 1.000 \\

nDCG@100 & \lmdr & 
\nores & 
\nores &  
\nores & \nores & 1.000 \\

R@100 & ($500$) &  
\nores & 
\nores & 
\nores & \nores & 1.000 \\

% P@10 & & 
% \nores & 
% \nores & 
% \nores & \nores & 1.000 \\

\bottomrule
\end{tabular}
\end{adjustbox}
\caption{\footnotesize Unlike Table \ref{stab:tab3}, here the QPP outcomes were evaluated by \proposed~(instead of $\tau$).
\label{stab:tab4}
}
\end{subtable}
%
\end{table}