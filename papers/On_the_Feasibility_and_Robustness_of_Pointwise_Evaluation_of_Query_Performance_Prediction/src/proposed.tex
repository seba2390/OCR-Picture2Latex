\label{sec:propose}

\paragraph{Correlation with listwise ground-truth}
Before describing our new QPP evaluation framework \proposed, we begin by introducing the required notation. Formally, a QPP estimate is a function of the form $\phi(Q, M_k(Q)) \mapsto \mathbb{R}$, where $M_k(Q)$ is the set of top-$k$ ranked documents retrieved by an IR model $M$ for a query $Q \in \mathcal{Q}$, a benchmark set of queries.

For the purpose of listwise evaluation, for each $Q\in \mathcal{Q}$, we first compute the value of a target IR evaluation metric, $\mu(Q)$ that reflects the quality of the retrieved list $M_k(Q)$. The next step uses these $\mu(Q)$ scores to induce a \textit{ground-truth ranking} of the set $\mathcal{Q}$, or in other words, arrange the queries by their decreasing (or increasing) $\mu(Q)$ values, i.e., 
\begin{equation}
\mathcal{Q}_\mu = \{Q_i \in \mathcal{Q}: \mu(Q_i) > \mu(Q_{i+1}),
\, \forall i=1,\ldots,|\mathcal{Q}|-1\}  \}
\end{equation}
Similarly, the evaluation framework also yields a \emph{predicted ranking} of the queries, where this time the queries are sorted by the QPP estimated scores, i.e.,
\begin{equation}
\mathcal{Q}_\phi = \{Q_i \in \mathcal{Q}: \phi(Q_i) > \phi(Q_{i+1}),
\, \forall i=1,\ldots,|\mathcal{Q}|-1 \} 
\label{qpp_listwise_pred}
\end{equation}
A listwise evaluation framework then computes the rank correlation between these two ordered sets
$\gamma(\mathcal{Q}_\mu, \mathcal{Q}_\phi),\,\,\text{where}\,\, \gamma: \mathbb{R}^{|\mathcal{Q}|}\times\mathbb{R}^{|\mathcal{Q}|} \mapsto [0,1]$ is a correlation measure, such as Kendall's $\tau$.

\paragraph{Individual ground-truth}
In contrast to listwise evaluations, where the ground-truth takes the form of an ordered set of queries, pointwise QPP evaluation involves making $|\mathcal{Q}|$ \textit{independent comparisons}. Each comparison is made between a query $Q$'s predicted QPP score $\phi(Q)$ and its retrieval effectiveness measure $\mu(Q)$, i.e.,
\begin{equation}
\eta(\mathcal{Q}, \mu, \phi) \defas \frac{1}{|\mathcal{Q}|}\sum_{Q \in \mathcal{Q}}\eta(\mu(Q), \phi(Q))
\label{eq:pwcorr}  
\end{equation}
Unlike the rank correlation $\gamma$, 
here $\eta$ is a pointwise correlation function of the form $\eta:\mathbb{R}\times \mathbb{R}\mapsto\mathbb{R}$.
It is often convenient to think of $\eta$ as the inverse of a \emph{distance} function that measures the extent to which a predicted value deviates from the corresponding true value.
In contrast to ground-truth evaluation metrics, most QPP estimates (e.g., NQC, WIG etc.) are not bounded within $[0, 1]$. Therefore, to employ a distance measure, each QPP estimate $\phi(Q)$ must be normalized to the unit interval. Subsequently, $\eta$ can be defined as
$\eta(\mu(Q), \phi(Q)) \defas 1-|\mu(Q) - \phi(Q)/\aleph|$,
where $\aleph$ is a normalization constant which is sufficiently large to ensure that the denominator is positive.

\paragraph{Selecting an IR metric for pointwise QPP evaluation}

In general, an unsupervised QPP estimator will be agnostic with respect to the target IR metric $\mu$. For instance, NQC scores can be seen as being approximations of AP@100 values, but can also be interpreted as approximating any other metric, such as nDCG@20 or P@10. Therefore, a question arises around which metric should be used to compute the individual correlations in Equation \ref{eq:pwcorr}. Of course, the results can differ substantially for different choices of $\mu$, e.g., AP or nDCG. This is also the case for listwise QPP evaluation, as reported in \cite{dg22ecir}. To reduce the effect of such variations, we now propose a simple yet effective solution.

\paragraph{Metric-agnostic pointwise QPP evaluation}
For a set of evaluation functions
$\mu \in \mathcal{M}$ (e.g., $\mathcal{M} = \{\text{AP@100}, \text{nDCG@20},\ldots\}$), we employ an aggregation function to compute the overall pointwise correlation (Equation \ref{eq:pwcorr}) of a QPP estimate with respect to each metric.
Formally,
\begin{equation}
\eta(Q,\mathcal{M},\phi) = \Sigma_{\mu \in \mathcal{M}} 
(1-|\mu(Q) - \phi(Q)/\aleph|), \label{eq:avgpwcorr}
\end{equation}
where $\Sigma$ denotes an aggregation function (it does not indicate summation). In particular, we use the most commonly-used such functions as choices for $\Sigma$: `minimum', `maximum', and `average' -- i.e., $\Sigma \in \{\text{avg}, \text{min}, \text{max}\}$.
Next, we find the average over these values computed for a given set of queries $\mathcal{Q}$, i.e., we substitute $\eta(Q,\mathcal{M},\phi)$ from Equation \ref{eq:avgpwcorr} into the summation of Equation \ref{eq:pwcorr}.