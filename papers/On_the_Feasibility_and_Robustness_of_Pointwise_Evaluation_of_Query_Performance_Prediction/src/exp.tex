\label{sec:exp}
\begin{table}[!t]
\caption{\small
QPP configurations - (QPP method, IR metric, and models) used to measure variations.}
% \vskip 0.2em
\centering
\begin{adjustbox}{width=.85\columnwidth}
%\begin{tabularx}{1\columnwidth}{l|X}
\begin{tabular}{@{}l@{~~~~}|l@{}}
\toprule
QPP Methods & AvgIDF \cite{survey_preret_qpp}, Clarity \cite{croft_qpp_sigir02}, NQC \cite{kurland_tois12}, WIG \cite{wig_croft_SIGIR07}, UEF(Clarity), UEF(NQC), UEF(WIG) \cite{uef_kurland_sigir10} \\
\midrule
IR Metrics & AP@100, nDCG@100, P@10, Recall@100 \\
\midrule
IR Models & \lmjm~($\lambda=0.6$), \lmdr~($\mu=1000$), BM25 $(k,b)=(0.7,0.3)$ \\
\bottomrule
\end{tabular}
\end{adjustbox}

\label{tab:models_and_metrics}
\end{table}


A QPP experiment context \cite{dg22ecir} involves three configuration choices: i) the \textbf{QPP method} itself that is used to predict the relative performance of queries; ii) the \textbf{IR metric} that is used to obtain a ground-truth ordering of the query performances as measured on a set of top-$k$ ($k=100$ in our experiments) documents retrieved by iii) a specific \textbf{IR model}. Table \ref{tab:models_and_metrics} summarizes the IR models and metrics used in our experiments, along with the relevant hyper-parameter values. The objective of our experiments is to investigate the following two key research questions:

\begin{compactitem}
    \item \RQ{1}: Does \proposed~\textit{agree} with the standard listwise correlation metrics?
    \item \RQ{2}: How \textit{robust} is \proposed~with respect to changes in the QPP experiment context?
\end{compactitem}
\section{Theoretical Results}
\label{sec:stability}

In this section, we present a generalization bound on the true risk induced by our algorithm using the theoretical framework of the Uniform Stability~\cite{bousquet2002stability}.
We will see that this theoretical analysis gives some insights about the number of landmarks to select in practice. 

\subsection{\landSVM's uniform stability}

The idea of Uniform Stability is to check if an algorithm produces similar solutions from datasets that are slightly different. Let $S$ be the original dataset and $S^i$ the set obtained after having replaced the $i^{th}$ example of $S$ by a new sample $z_i'$ drawn according to ${\cal D}$.
We will say that an algorithm is uniformly stable if the difference between the loss suffered (on a new instance) by the hypothesis $f$ learned from $S$ and the loss suffered by the hypothesis $f^i$ learned from $S^i$ converges in $O(\frac 1m)$.

For the following analysis, we introduce a new notation that allows us to simplify the derivations. We rewrite

$$ f(x,k) = \theta \, \mul{x}^T $$

with $\theta = [\theta_{0.},...,\theta_{k.},...,\theta_{K.},b]$ and $\mul{x} = [\bm{0},...,\mul{x},\bm{0},...,\bm{0},1]$ (that implicitly depends on $k$) both of size $KL+1$ and

$$ F(f) = \frac{1}{2} \norm{\theta}^2 + \frac{c}{m}\sumi \max(0,1-y_i (\theta \mul{x_i}^T)).$$

\begin{defn}{(\bf{Uniform Stability})}
    A learning algorithm A has uniform stability $2 \frac{\beta}{m}$ \wrt the loss function $\ell$ with $\beta \in \mathbb{R}^{+}$ if

    $$ \sup_{z \sim \mathcal{D}} \abs{\ell(f,z) - \ell(f^{i},z)} \leq 2 \frac{\beta}{m} .$$ 
\end{defn}

The uniform stability is directly implied if 

$$ \forall z \in \mathcal{D}, \;\; \abs{\ell(f,z) - \ell(f^{\setminus i},z)} \leq \frac{\beta}{m}$$

where $f^{\setminus i}$ is the hypothesis learned on $S^{\setminus i}$, the set $S$ without the $i^{th}$ instance $z_i$, which follows from

$$\abs{\ell(f^{i},z) - \ell(f,z)} \leq \abs{\ell(f^{i},z) - \ell(f^{\setminus i},z)} + \abs{\ell(f^{\setminus i},z) - \ell(f,z)}  \leq 2 \abs{\ell(f^{\setminus i},z) - \ell(f,z)}$$

that uses the triangular inequality (at worse, altering a point is like removing a point and adding another one).

In order to study the uniform stability of an algorithm, it is required to prove the  $\sigma$-admissibility of the loss function.

\begin{defn}{(\bf{$\sigma$-admissibility})}
    A loss function $\ell(f,z)$ is $\sigma$-admissible \wrt $f$ if it is convex \wrt its first argument and $\forall f_1,f_2 $ and $\forall z \in \mathcal{Z}$:

    $$ \abs{\ell(f_1,z) - l(f_2,z)} \leq \sigma \abs{f_1(x,k)-f_2(x,k)} .$$
\end{defn}

Following~\cite{bousquet2002stability}, we know that the hinge loss is $1$-admissible.

We can now present the main result about our algorithm \landSVM.

\begin{thm}{\bf{\landSVM Uniform Stability}}
  Assuming that  $ \forall x \in \mathcal{X}, \norm{x} \leq c$,  \landSVM has uniform stability  $ \frac{c L M^2}{m}$,
where $M = \max(c^2,1)$ if  $\mu$ is the dot product and $M = 1$ if $\mu$ uses the RBF kernel.
\end{thm}

\begin{proof}

    As $\ell(f,z)$ is $1$-admissible, $\forall z=(x,y,k) \in \mathcal{Z}$,

    %      \abs{\ell(f^{\setminus i},z) - \ell(f,z)} \! \leq \! \abs{f^{\setminus i}\!(x,k)-f(x,k)} \!=\! \abs{\Delta f (x,k)} \label{testit}$$

    \begin{small}
    \begin{align}
      \abs{\ell(f^{\setminus i},z) - \ell(f,z)} &\leq \abs{f^{\setminus i}\!(x,k)-f(x,k)} = \abs{\Delta f (x,k)} \label{lin:lossdiffabs}
    \end{align}
    \end{small}

    with $ \Delta f = f^{\setminus i} - f$.
    By denoting $\Delta\theta = \theta^{\setminus i} - \theta$, we can derive, $\forall z=(x,y,k) \in \mathcal{Z}$,

    
    \begin{align}
        \abs{\Delta f (x,k)} &= \abs{\theta^{\setminus i} \mul{x}^T - \theta \mul{x}^T} \nonumber \\
        &= \abs{(\theta^{\setminus i}- \theta)\mul{x}^T} \nonumber \\
        & \leq \normf{\theta^{\setminus i} - \theta} \norm{\mul{x}} \label{lin:cauchy} \\
        & \leq \normf{\Delta\theta} \norm{\mul{x}} \nonumber \\
        & \leq \normf{\Delta\theta} \sqrt{L} \norm{\mul{x}}_\infty \label{lin:inf} \\
        & \leq \normf{\Delta\theta} \sqrt{L} \max_l(\mu(x,l)) \nonumber \\
        & \leq \normf{\Delta\theta} \sqrt{L} M \label{lin:dthetasqlm}
    \end{align}

    Eq.~\eqref{lin:cauchy} is due to the Cauchy-Swartz inequality,
    % Eq.~\eqref{lin:theta} is because $ \norm{\Delta f (.,k_i)} = \norm{\theta^{\setminus i} - \theta}$ ($\Delta f$ is linear in it's first parameter)
    and Eq.~\eqref{lin:inf} is because $ \norm{\mul{x}} \leq \sqrt{L} \norm{\mul{x}}_\infty$ recalling that $\mul{x} \in \mathbb{R}^{(1 \times L)}$.

    The value of $M$ depends on the chosen function $\mu$. For instance, if $\mu$ is the dot product, $M = \max(C^2,1)$ and if it uses the RBF kernel, $M = 1$.

    From Lemma 21 of \cite{bousquet2002stability}:

    $$ 2 \normf{\Delta\theta}^2 \leq \frac{c}{m} \abs{\Delta f(x_i,k_i)}.$$

    Then, by instantiating Eq.~\eqref{lin:dthetasqlm} for $z = z_i$, we get

    $$\normf{\Delta\theta}^2 \leq \frac{c}{2m} \abs{\Delta f(x_i,k_i)} \leq \frac{c}{2m} \normf{\Delta\theta} \sqrt{L} M$$

    and as $\normf{\Delta\theta} > 0$, we obtain

    $$ \normf{\Delta\theta} \leq \frac{c}{2m} \sqrt{L} M $$

    so, from the previous bound on $\abs{\Delta f(x,k)}$, we get

    $$ \forall z=(x,y,k), \;\; \abs{\Delta f(x,k)} \leq \normf{\Delta\theta} \sqrt{L} M \leq \frac{c L M^2}{2m}$$

    which, with Eq.~\eqref{lin:lossdiffabs} gives the $\frac{c L M^2}{m}$ uniform stability.


\end{proof}

Note that the stability of the algorithm depends on the number of selected landmarks. \landSVM is stable only if $L \ll m$, which is not a strict condition considering that, in practice, we select $L = O(n)$ landmarks (with $n$ the size of the input space $\mathcal{X}$) and that, for learning in general, $n \ll m$.

\begin{thm}{\cite{bousquet2002stability}}
Let A be an algorithm with uniform stability $\frac{2\beta}{m}$ \wrt a loss $\ell$ such that $0 \leq \ell(f,z) \leq E$, $\forall z \in \mathcal{Z}$. Then, for any i.i.d. sample $S$ of size $m$ and for any $\delta \in (0,1)$, with probability $1- \delta$:

$$ R_{\mathcal{D}}(f) \leq \hat{R}_{S}(f) + \frac{2\beta}{m} + \big( 4\beta + E \big) \sqrt{\frac{\ln \frac{1}{\delta}}{2m}}$$

where $R_{\mathcal{D}}(f)$ is the true risk and $\hat{R}_{S}(f)$ is the empirical risk on sample $S$. 

\end{thm}

Before deriving the generalization bound, we need to prove that our loss $\ell$ is bounded by a constant $E$ when evaluated at the optimal solution of $F$. Let $f$ be the minimizer of $F$. We deduce that:

\begin{gather}
    F(f) \leq F(\bm{0}) \nonumber \\
    \frac{1}{2} \norm{\theta}^2 + \frac{c}{m} \sumi \max(0,1-y_i (\theta \mul{x_i}^T)) \leq \frac{1}{2} \norm{\bm{0}}^2 + \frac{c}{m} \sumi \max(0,1-y_i (\bm{0} \mul{x_i}^T)) \nonumber \\
    \frac{1}{2} \norm{\theta}^2  \leq c \label{eq:sum} \\ 
    \norm{\theta}^2  \leq 2c \nonumber
\end{gather}

Eq.~\eqref{eq:sum} is because $ \forall a,b,c \in \mathbb{R}^{+}$, $ a + b \leq c $ implies that $ b \leq c $. Thus,

\begin{align}
    \ell(f,z) &= \max(0,1-y \theta \mul{x}^T) \nonumber \\
    & \leq 1 + \abs{\theta \mul{x}^T} \nonumber \\
    & \leq 1 + \norm{\theta} \norm{\mul{x}^T}  \label{eq:cauchy2}\\
    & \leq 1 + 2c \sqrt{L} M = E \nonumber
\end{align}

Eq.~\eqref{eq:cauchy2} comes again from the Cauchy-Swartz inequality.

\begin{cor}
    The generalization bound of \landSVM derived using the Uniform Stability framework is as follows:

    \small{
    $$ R_{\mathcal{D}}(f)\! \leq \!\hat{R}_{S}(f) + \frac{c L M^2}{m} + \left( \frac{2c L M^2}{m} \!+ \!1 \!+\! 2c \sqrt{L} M \! \right)\!\!\sqrt{\frac{\ln \frac{1}{\delta}}{2m}}.$$}

\end{cor}

%In other words, as the size of the sample increases, the true risk tends to be smaller or equal to the empirical risk, which implies that the algorithm generalizes well on unseen data.



An affirmative answer to \RQ{1} would indicate that our proposed metric \proposed~is \textit{consistent} with existing metrics used for QPP evaluation, while an affirmative answer to \RQ{2} would suggest that \proposed~is preferable to existing methods due to its higher stability with respect to different experimental settings.

\begin{table}[!th]
\caption{
\footnotesize
Stability of the proposed pointwise QPP metric APAE with respect to listwise approach, across different pairs of IR metrics and IR models. Red cells indicate the lowest value in each group, while the lowest values along each column are bold-faced.
\label{tab:sdres}
}
\vskip 0.5em
\begin{subtable}[t]{.48\linewidth}
\centering
\begin{adjustbox}{width=\textwidth}
%\begin{tabular}{@{}l@{~~}lcccccc@{}}
\begin{tabular}{@{}l@{~~}l@{~~}c@{~~}c@{~~}c@{~~}c@{~~}c@{~~}c@{}}

\toprule

Model & Metric & 
AP@100 & 
R@10 & R@100 & 
nDCG@10 & nDCG@100 \\

% & &
% @100 &
% @10 & @100 &
% @10 & @100 \\

\midrule

\lmjm & \multirow{3}{*}{AP@10} & 
0.497 & 
0.813 & \cellcolor{melon}0.429 &
0.783 & \cellcolor{melon}0.429 \\

BM25 & & 
0.897 & 
0.722 & 0.722 &
0.793 & 0.793 \\

\lmdr & & 
0.897 & 
0.786 & 0.786 &
0.823 & 0.905 \\

\cmidrule{1-2}

\lmjm & \multirow{3}{*}{AP@100} & \nores & 
\cellcolor{melon}\textbf{0.328} & 0.811 &
0.363 & 0.783 \\

BM25 & & 
\nores & 
0.783 & 0.784 &
0.714 & 0.642 \\

\lmdr & & 
\nores & 
0.823 & 0.901 &
0.834 & 0.789 \\

\cmidrule{1-2}

% \lmjm & AP & 
% \nores & \nores & 
% \cellcolor{melon}0.346 & 0.655 &
% 0.389 & 0.733 \\

% BM25 & $@1000$ & 
% \nores & \nores & 
% 0.6731 & 0.6731 & 0.5321 &
% 0.8643 & 0.8522 & 0.7095 \\

% \lmdr & & 
% \nores & \nores & 
% 0.8962 & 0.7845 & 0.4539 &
% 0.8790 & 0.8235 & 0.8790 \\

% \cmidrule{1-2}

\lmjm & \multirow{3}{*}{R@10} & 
\nores & 
\nores & 0.624 &
0.893 & \cellcolor{melon}0.503 \\

BM25 & & 
\nores & 
\nores & 0.803 &
0.982 & 0.894 \\

\lmdr & & 
\nores & 
\nores & 0.903 &
0.864 & 0.864 \\

\cmidrule{1-2}

\lmjm & \multirow{3}{*}{R@100} & 
\nores & \nores & 
\nores &
0.852 & 0.804 \\

BM25 & & 
\nores & \nores & 
\nores &
0.786 & 0.890 \\

\lmdr & & 
\nores & \nores & 
\nores &
\cellcolor{melon}0.738 & \cellcolor{melon}0.738 \\

\cmidrule{1-2}

% \lmjm & R & 
% \nores & \nores & 
% \nores & \nores & \nores &
% \cellcolor{melon}0.4269 & 0.9032 & 0.7809 \\ 

% BM25 & $@1000$ & 
% \nores & \nores & 
% \nores & \nores & \nores &
% 0.9078 & 0.8231 & 0.7238 \\

% \lmdr & & 
% \nores & \nores & 
% \nores & \nores & \nores &
% 0.9048 & 0.9048 & 0.7238 \\

% \cmidrule{1-2}

\lmjm & \multirow{3}{*}{nDCG@10} & 
\nores & \nores & 
\nores & \nores& \cellcolor{melon}0.537 \\

BM25 & & 
\nores & \nores & 
\nores &
\nores & 0.904 \\

\lmdr & & 
\nores & \nores & 
\nores &
\nores & 0.868 \\

% \cmidrule{1-2}

% \lmjm & nDCG & 
% \nores & \nores & 
% \nores & \nores & \nores &
% \nores & \nores & 0.9043 \\

% BM25 & $@100$ & 
% \nores & \nores & 
% \nores & \nores & \nores &
% \nores & \nores & \cellcolor{melon}0.8642 \\

% \lmdr & & 
% \nores & \nores & 
% \nores & \nores & \nores &
% \nores & \nores & 0.8956 \\

\bottomrule
\label{tab:metric_tau_tau}
\end{tabular}
\end{adjustbox}
\caption{\footnotesize Correlations between the relative ranks of 7 different QPP systems across different pairs of IR target metrics. QPP systems were evaluated with the baseline listwise metric - Kendall's $\tau$. 
\label{stab:tab1}
}
\end{subtable}%
\quad
\begin{subtable}[t]{.48\linewidth}
\centering
\begin{adjustbox}{width=\textwidth}
%\begin{tabular}{@{}l@{~~}lccccc@{}}
\begin{tabular}{@{}l@{~~}l@{~~}c@{~~}c@{~~}c@{~~}c@{~~}c@{}}
\toprule

Model & Metric & 
AP@100 & 
R@10 & R@100 & 
nDCG@10 & nDCG@100 \\

% & &
% @100 &
% @10 & @100 &
% @10 & @100 \\

\midrule

\lmjm & \multirow{3}{*}{AP@10} & 
0.904 & 
1.000 & \cellcolor{melon}0.715 &
1.000 & 0.792 \\

BM25 & & 
1.000 & 
1.000 & 1.000 &
1.000 & 1.000 \\

\lmdr & & 
1.000 & 
1.000 & 1.000 &
1.000 & 1.000 \\

\cmidrule{1-2}

\lmjm & \multirow{3}{*}{AP@100} & 
\nores & 
0.905 & 0.811 &
\cellcolor{melon}0.669 & 1.000 \\

BM25 & & 
\nores & 
1.000 & 1.000 &
1.000 & 1.000 \\

\lmdr & & 
\nores & 
1.000 & 1.000 &
1.000 & 1.000 \\

% \cmidrule{1-2}

% \lmjm & AP & 
% \nores & \nores & 
% 0.723 & 0.811 & 0.811 &
% 0.573 & 1.000 & 1.000 \\

% BM25 & $@1000$ & 
% \nores & \nores & 
% 1.000 & 1.000 & 0.635 &
% 1.000 & 1.000 & 0.811 \\

% \lmdr & & 
% \nores & \nores & 
% 0.905 & 0.811 & \cellcolor{melon}\textbf{0.524} &
% 0.905 & 0.905 & 0.811 \\

\cmidrule{1-2}

\lmjm & \multirow{3}{*}{R@10} & 
\nores & 
\nores & 0.603 &
0.905 & \cellcolor{melon}\textbf{0.542} \\

BM25 & & 
\nores & 
\nores & 1.000 &
1.000 & 1.000 \\

\lmdr & & 
\nores & 
\nores & 1.000 &
1.000 & 1.000 \\

\cmidrule{1-2}

\lmjm & \multirow{3}{*}{R@100} & 
\nores & \nores & 
\nores &
\cellcolor{melon}0.654 & 1.000 \\

BM25 & & 
\nores & \nores & 
\nores &
1.000 & 1.000 \\

\lmdr & & 
\nores & \nores & 
\nores &
1.000 & 1.000 \\

\cmidrule{1-2}

% \lmjm & R & 
% \nores & \nores & 
% \nores & \nores & \nores &
% \cellcolor{melon}0.627 & 0.903 & 0.781 \\ 

% BM25 & $@1000$ & 
% \nores & \nores & 
% \nores & \nores & \nores &
% 0.908 & 0.823 & 0.724 \\

% \lmdr & & 
% \nores & \nores & 
% \nores & \nores & \nores &
% 0.905 & 0.905 & 0.823 \\

% \cmidrule{1-2}

\lmjm & \multirow{3}{*}{nDCG@10} & 
\nores & \nores & 
\nores & \nores &
\cellcolor{melon}0.649 \\

BM25 & & 
\nores & \nores & 
\nores & \nores &
1.000 \\

\lmdr & & 
\nores & \nores & 
\nores & \nores &
1.000 \\

% \cmidrule{1-2}

% \lmjm & nDCG & 
% \nores & \nores & 
% \nores & \nores & \nores &
% \nores & \nores & 1.000 \\

% BM25 & $@100$ & 
% \nores & \nores & 
% \nores & \nores & \nores &
% \nores & \nores & 1.000 \\

% \lmdr & & 
% \nores & \nores & 
% \nores & \nores & \nores &
% \nores & \nores & 1.000 \\

\bottomrule
\label{tab:metric_apae_tau}
\end{tabular}
\end{adjustbox}
\caption{\footnotesize Similar to Table \ref{stab:tab1}, except QPP performance was evaluated with the pointwise approach \proposed. A comparison with Table \ref{stab:tab1} indicates a better consistency in the relative ranks of QPP systems for variations in the IR metrics.
\label{stab:tab2}
}
\end{subtable}
%
\begin{subtable}[t]{.48\linewidth}
\centering
\begin{adjustbox}{width=\textwidth}
%\begin{tabular}{@{}l@{~~}cccccccc@{}}
\begin{tabular}{@{}l@{~~}c@{~~}c@{~~}c@{~~}c@{~~}c@{~~}c@{~~}c@{~~}c@{}}

\toprule

\multirow{2}{*}{Metric} & \multirow{2}{*}{Model} & 
\lmjm &
BM25 & BM25 & 
\lmdr & \lmdr \\

& & ($0.6$) &
($0.7, 0.3$) & ($0.3, 0.7$) &
($500$) & ($1000$) \\

\midrule

AP@100 & & 
0.826 & 
0.904 & 0.819 & 
0.714 & 0.895 \\

nDCG@100 & \lmjm & 
0.780 & 
\cellcolor{melon}0.694 & 0.695 & 
0.759 & 0.759 \\

R@100 & ($0.3$) & 
0.824 & 
0.769 & 0.782 & 
0.904 & 0.904 \\

% P@10 & & 
% 0.883 & 
% 0.762 & 0.792 & 
% 0.711 & 0.674 \\

\cmidrule{1-2}

AP@100 & &  
\nores & 
0.703 & 0.712 & 
0.904 & 0.823 \\

nDCG@100 & \lmjm & 
\nores & 
0.781 & 0.827 & 
0.811 & 0.811 \\

R@100 & ($0.6$) & 
\nores & 
0.813 & 0.725 & 
0.731 & \cellcolor{melon}\textbf{0.675} \\

% P@10 & & 
% \nores & 
% 0.813 & 0.806 & 
% 0.710 & 0.706 \\

\cmidrule{1-2}

AP@100 & &
\nores & 
\nores & 0.903 & 
0.785 & 0.785 \\

nDCG@100 & BM25 &
\nores & 
\nores & 0.897 & 
0.786 & 0.786 \\

R@100 & ($0.7, 0.3$) &
\nores & 
\nores & 0.812 & 
\cellcolor{melon}0.752 & 0.779 \\

% P@10 & & 
% \nores & 
% \nores & 0.884 & 
% 0.813 & \cellcolor{melon}0.731 \\

% \cmidrule{1-2}

% AP@100 & & 
% \nores & 
% \nores & \nores & 0.811 & 
% 0.884 & 0.732 & 0.884 \\

% nDCG@100 & BM25 &
% \nores & 
% \nores & \nores & 0.903 & 
% 0.782 & 0.884 & 0.812 \\

% R@100 & ($1.0, 1.0$) & 
% \nores & 
% \nores & \nores & 0.833 & 
% 0.855 & 0.853 & 0.851 \\

% P@10 & & 
% \nores & 
% \nores & \nores & 0.739 & 
% \cellcolor{melon}0.713 & 0.798 & 0.798 \\

\cmidrule{1-2}

AP@100 & &  
\nores & 
\nores & \nores & 
0.887 & \cellcolor{melon}0.882 \\

nDCG@100 & BM25 & 
\nores & 
\nores & \nores & 
0.901 & 0.895 \\

R@100 & ($0.3, 0.7$) & 
\nores & 
\nores & \nores & 
0.889 & 0.901 \\

% P@10 & & 
% \nores & 
% \nores & \nores & 
% \cellcolor{melon}0.794 & 0.843 \\

% \cmidrule{1-2}

% AP@100 & & 
% \nores & 
% \nores & \nores & \nores & 
% \nores & 0.912 & 0.904 \\

% nDCG@100 & \lmdr &
% \nores & 
% \nores & \nores & \nores & 
% \nores & 0.901 & 0.925 \\

% R@100 & ($100$) & 
% \nores & 
% \nores & \nores & \nores & 
% \nores & 0.895 & 0.913 \\

% P@10 & &  
% \nores & 
% \nores & \nores & \nores & 
% \nores & \cellcolor{melon}0.888 & 0.911 \\

\cmidrule{1-2}

AP@100 & &  
\nores & 
\nores & 
\nores & \nores & 0.901 \\

nDCG@100 & \lmdr & 
\nores & 
\nores & 
\nores & \nores & \cellcolor{melon}0.893 \\

R@100 & ($500$) &  
\nores & 
\nores & 
\nores & \nores & 0.903 \\

% P@10 & & 
% \nores & 
% \nores & 
% \nores & \nores & \cellcolor{melon}0.879 \\

\bottomrule
\end{tabular}
\end{adjustbox}
%\caption{\small Unlike Table \ref{stab:tab1}, the rank correlations between the relative ranks of QPP systems are measured across IR model pairs. As in Table \ref{stab:tab1}, QPP systems were evaluated with $\tau$. The numbers alongside the IR models denote their respective parameter values, e.g. $\lambda$ for \lmjm.
\caption{\footnotesize Here rank correlations between the relative ranks of QPP systems are measured across IR model pairs. As in Table \ref{stab:tab1}, QPP systems were evaluated with $\tau$. The numbers alongside the IR models denote their respective parameters.
% 
\label{stab:tab3}
}
\end{subtable}%
\quad
\begin{subtable}[t]{.48\linewidth}
\centering
\begin{adjustbox}{width=\textwidth}
%\begin{tabular}{@{}l@{~~}cccccc@{}}
\begin{tabular}{@{}l@{~~}c@{~~}c@{~~}c@{~~}c@{~~}c@{~~}c@{}}

\toprule

\multirow{2}{*}{Metric} & \multirow{2}{*}{Model} & 
\lmjm &
BM25 & BM25 & 
\lmdr & \lmdr \\

& & ($0.6$) &
($0.7, 0.3$) & ($0.3, 0.7$) &
($500$) & ($1000$) \\

\midrule

AP@100 & & 
1.000 & 
1.000 & 1.000 & 
1.000 & 1.000 \\

nDCG@100 & \lmjm & 
1.000 & 
0.864 & 1.000 & 
\cellcolor{melon}0.843 & 0.864 \\

R@100 & ($0.3$) & 
1.000 & 
0.864 & 1.000 & 
1.000 & 1.000 \\

% P@10 & & 
% 1.000 & 
% 0.811 & 0.811 & 
% \cellcolor{melon}0.803 & 1.000 \\

\cmidrule{1-2}

AP@100 & &  
\nores & 
1.000 & 1.000 & 
1.000 & 1.000 \\

nDCG@100 & \lmjm & 
\nores & 
0.914 & 1.000 & 
\cellcolor{melon}\textbf{0.813} & 0.914 \\

R@100 & ($0.6$) & 
\nores & 
1.000 & 1.000 & 
1.000 & 1.000 \\

% P@10 & & 
% \nores & 
% \cellcolor{melon}0.801 & 0.928 & 
% 0.824 & 1.000 \\

\cmidrule{1-2}

AP@100 & &
\nores & 
\nores & 1.000 & 
1.000 & 1.000 \\

nDCG@100 & BM25 &
\nores & 
\nores & 1.000 & 
1.000 & 1.000 \\

R@100 & ($0.7, 0.3$) &
\nores & 
\nores & \cellcolor{melon}0.812 & 
0.905 & 1.000 \\

% P@10 & & 
% \nores & 
% \nores & 1.000 & 
% 1.000 & \cellcolor{melon}\textbf{0.799} \\

% \cmidrule{1-2}

% AP@100 & & 
% \nores & 
% \nores & \nores & 1.000 & 
% 1.000 & 1.000 \\

% nDCG@100 & BM25 &
% \nores & 
% \nores & \nores & 1.000 & 
% 1.000 & 1.000 \\

% R@100 & ($1.0, 1.0$) & 
% \nores & 
% \nores & \nores & 1.000 & 
% 1.000 & 1.000 \\

% P@10 & & 
% \nores & 
% \nores & \nores & \cellcolor{melon}0.811 & 
% 0.923 & 1.000 \\

\cmidrule{1-2}

AP@100 & &  
\nores & 
\nores & \nores & 
1.000 & 1.000 \\

nDCG@100 & BM25 & 
\nores & 
\nores & \nores & 
1.000 & 1.000 \\

R@100 & ($0.3, 0.7$) & 
\nores & 
\nores & \nores & 
1.000 & 1.000 \\

% P@10 & & 
% \nores & 
% \nores & \nores & 
% 1.000 & \cellcolor{melon}0.843 \\

% \cmidrule{1-2}

% AP@100 & & 
% \nores & 
% \nores & \nores & \nores & 
% \nores & 1.000 & 1.000 \\

% nDCG@100 & \lmdr &
% \nores & 
% \nores & \nores & \nores & 
% \nores & 1.000 & 1.000 \\

% R@100 & ($100$) & 
% \nores & 
% \nores & \nores & \nores & 
% \nores & 1.000 & 1.000 \\

% P@10 & &  
% \nores & 
% \nores & \nores & \nores & 
% \nores & \cellcolor{melon}0.811 & \cellcolor{melon}0.811 \\

\cmidrule{1-2}

AP@100 & &  
\nores & 
\nores & 
\nores & \nores & 1.000 \\

nDCG@100 & \lmdr & 
\nores & 
\nores &  
\nores & \nores & 1.000 \\

R@100 & ($500$) &  
\nores & 
\nores & 
\nores & \nores & 1.000 \\

% P@10 & & 
% \nores & 
% \nores & 
% \nores & \nores & 1.000 \\

\bottomrule
\end{tabular}
\end{adjustbox}
\caption{\footnotesize Unlike Table \ref{stab:tab3}, here the QPP outcomes were evaluated by \proposed~(instead of $\tau$).
\label{stab:tab4}
}
\end{subtable}
%
\end{table}

We conduct our QPP experiments on the TREC Robust dataset, which consists of $249$ topics. Following the standard practice for QPP experiments \cite{hamed_neuralqpp,query_variants_kurland}, we report results aggregated over a total of 30 randomly chosen equal-sized train-test splits of the data. The training split of each partition was used for tuning the hyper-parameters for the QPP method.

\paragraph{Agreement between listwise and pointwise evaluation}
Firstly, we investigate the consistency of \proposed~with respect to three standard listwise QPP evaluation metrics: Pearson's $r$, Spearman's $\rho$ and Kendall's $\tau$; and a pointwise approach, scaled Absolute Rank Error (sARE) \cite{sare}. Since sARE is an error measure, we measure correlations of \proposed~with $1-\text{sARE}$ measures (which for the sake of simplicity, we refer to as sARE in Table \ref{tab:stable}). We experiment with three different instances of \proposed~obtained by substituting the aggregation functions -- avg, min and max as $\Sigma$ in Equation \ref{eq:avgpwcorr}, denoted respectively as $\pwua{\text{avg}}$, $\pwua{\text{min}}$ and $\pwua{\text{max}}$.

The results presented in Table \ref{tab:stable} answer \RQ{1} in the affirmative. Each reported value here corresponds to the rank correlation (Kendall's $\tau$) between the relative ranks of the QPP systems ordered by their effectiveness as computed via one of the standard metrics (one of $r$, $\rho$, $\tau$ or sARE) and \proposed, i.e., one of $\pwua{\text{avg}}$, $\pwua{\text{min}}$ and $\pwua{\text{max}}$). The high correlation values between the standard listwise and the proposed pointwise metrics show that \proposed~can be used as a substitute for the standard listwise evaluation. Notably, we see that the average aggregate function yields the best results, and hence for the subsequent experiments we use $\pwua{\text{avg}}$ as the pointwise evaluation metric.

\paragraph{Variances in relative effectiveness of QPP methods}

To investigate \RQ{2}, we consider the relative stability of QPP system ranks for variations in QPP contexts (i.e., different IR models and target metrics), comparing both listwise and pointwise approaches (see Table \ref{tab:sdres}). To clarify with an example, if working with three QPP methods, say AvgIDF, NQC, WIG, we observe that $\tau$(NQC) $>$ $\tau$(WIG) $>$ $\tau$(AvgIDF) for LMDir as measured relative to AP@100. We expect to observe a similar ordering for a different choice of the IR model and target IR metric, say BM25 with nDCG@100. As in our previous experiments, here we measure the rank correlations between a total of seven QPP systems (see Table \ref{tab:models_and_metrics}) via Kendall's $\tau$.