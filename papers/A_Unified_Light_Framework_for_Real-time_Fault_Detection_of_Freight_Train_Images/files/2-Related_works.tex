\section{Related works}
\label{relatedworks}
\textbf{Fault Detection for Freight Train Images.}
Some of the recent researches for fault detection of freight train images are listed as follows. Liu \emph{et al}.~\cite{Liu2015Automated} proposed a hierarchical fault inspection framework to detect the missing of bogie block key on freight trains with high speed and accuracy. Zheng \emph{et al}.~\cite{Zheng2016Automatic} proposed an automatic image inspection system to inspect coupler yokes by a linear support vector machine (SVM) classifier for localization and Adaboost decision trees for recognition. However, these methods only detect one type of faults, which greatly influence their effectiveness. In addition, Sun \emph{et al}.~\cite{Sun2018Railway} proposed a fast adaptive Markov random field (FAMRF) for image segmentation and an exact height function (EHF) for shape matching of fault region. This method solves the problem of multi-fault detection, but it is too complex to achieve enough accuracy and fast speed. Differing from the conventional techniques, deep learning methods deal with more complex and difficult problems in machine vision field. Sun \emph{et al}.~\cite{Sun2017Automatic} presented a CNN-based system consisting of two complex models for target region detection and fault recognition, respectively. Pahwa \emph{et al}.~\cite{8917062} performed a two-step high resolution segmentation of the train valves and use image processing techniques to identify faulty valves. Fu \emph{et al}.~\cite{FU2020212} proposed a two-stage method cascading a bearing localization stage and a defection segmentation stage to recognize the defect areas in a coarse-to-fine manner. However, these methods have high computational cost, which are insufficient to meet actual requirements of fault detection like real-time and versatility.

\textbf{Object Detection.}
As the basis of fault detection, object detectors based on CNNs have been developed rapidly over the years, which are widely used in actual applications due to their powerful capacity~\cite{lu2018estimation}. These CNN-based detectors can be divided into two parts: one-stage and two-stage. One-stage detectors directly predict object classes and locations without region proposal generation, such as you only look once (YOLO)~\cite{redmon2018yolov3}, single shot multibox detector (SSD)~\cite{LiuAESRFB16}, reverse connection with objectness prior network (RON)~\cite{kong2017ron}, RefineDet~\cite{zhang2018single}, and deeply supervised object detector (DSOD)~\cite{shen2017dsod}. Based on one-stage strategy, these detectors can obtain fast speed, which are suited for limited computing condition, but usually sacrifices accuracy. Two-stage detectors firstly generate a set of region proposals, and then classify whether they are background or foreground. These detectors achieve accurate and effective object detection such as Faster R-CNN~\cite{RenHGS15}, R-FCN~\cite{DaiLHS16}, multi-scale location-aware kernel representation (MLKP)~\cite{wang2018multi}, and Cascade R-CNN~\cite{cai2018cascade}. Compared with one-stage detectors, these two-stage detectors have higher performance but need more computations. Hence, they are usually incapable of coping with practical application for real-time demand.

\textbf{Light-weight Neural Networks.}
During past years, many efforts are devoted to designing light-weight backbones for object detection task in the resource-restricted conditions. There are some light-weight architecture designs which achieve better speed-accuracy trade-offs, including SqueezeNet~\cite{iandola2016squeezenet}, MobileNet~\cite{sandler2018inverted}, and ShuffleNet~\cite{ma2018shufflenet}, etc. Compared with the superior performance models like ResNet~\cite{He2016Deep}, these networks have fewer parameters with approximate precision. Moreover, there has been a growing interest in incorporating light-weight networks into CNNs for object detection task. For example, Tiny-DSOD~\cite{yuxi2018tinydsod} consists of depthwise dense block based backbone and depthwise feature pyramid network, achieving a better trade-off between resources and accuracy. Pelee~\cite{wang2018pelee} combines a PeleeNet with SSD to keep detection accuracy for mobile applications with fast speed. Each of these light-weight neural networks has a small model size, but its accuracy still has a large room for improvement.

As for real-time fault detection task, both accuracy and computation complexity are important considerations under stringent resource requirements in field environment. To obtain an outstanding balance between accuracy and computational cost, we take inspiration from both the incredible efficiency of the Fire modules introduced in SqueezeNet~\cite{iandola2016squeezenet} and the powerful detection performance demonstrated by the R-FCN~\cite{DaiLHS16}. In addition, the multi-level feature fusion strategy in~\cite{8911418} can be used to further improve the detection accuracy notably without a lot redundant computations.
