\section{Introduction}
\label{intro}
\IEEEPARstart{F}{ault} detection is a vital routine maintenance work with regard to railway system~\cite{8673908,8692707,8667891}. For the freight trains, vehicle braking and steering systems contain many important parts that need to be carefully detected, because the loss or displacement of these components will seriously affect the driving safety. Such detection task applies the vision-based methods to replace the manual detection with many advantages such as high efficiency and accuracy. However, the image acquisition devices are installed outdoors as shown in Fig.~\ref{fig:detection}(a). The illumination variation always impacts the quality of acquired images as shown in Fig.~\ref{fig:detection}(b). It is difficult to possess sufficient features on account of that various parts are usually small, polluted, or obscure. These parts usually contain too much structural information, and the textures are similar to the backgrounds. All these problems always lead to failure in fault detection for freight train images. Moreover, the only resource-constrained devices are available in practical applications due to field environmental limitations.

In general, vision-based fault detection can be considered as a special type of object detection task in computer vision~\cite{8911418}. Recently, the rapid development in deep learning techniques can provide a robust solution for object detection, because deep networks especially the convolutional neural networks (CNNs) actually implement the functions of higher complexity. Deep learning-based object detection methods can detect different objects simultaneously with higher accuracy, even in complicated and changeable environments. To get better results, researchers have designed deeper, broader and more complex networks such as faster region-based CNN (Faster R-CNN)~\cite{RenHGS15}, and region-based fully convolutional network (R-FCN)~\cite{DaiLHS16}. However, these superior CNN-based detectors face many difficulties when they are applied into real-time fault detection. For example, Zhang~\emph{et al}.~\cite{zhang2018} proposed a unified framework for fault detection of freight train images (FTI-FDet) based on Faster R-CNN~\cite{RenHGS15}. But it is insufficient to achieve fast speed, and its model size is huge ($>$550 MB). Based on the above FTI-FDet, Zhang \emph{et al}.~\cite{8911418} proposed a specialized light fault detector (Light FTI-FDet) which pursues a balance between accuracy and speed. However, its model size is still over 89 MB. Such approaches~\cite{zhang2018,8911418} have been proved to be accurate enough to meet the actual needs, but the effectiveness of CNN-based detectors especially the illumination invariance, is missing to be analyzed in principle. More importantly, all previous studies~\cite{Sun2017Automatic,zhang2018,8911418,RenHGS15,DaiLHS16} have been unable to meet a real-time detection speed of above 30 frames per second (fps)~\cite{Gaussian-YOLOv3}, which is a prerequisite for vision-based fault detection. In addition, the model size of these detectors is also too large for strict memory and computational budget constraints. Therefore, a real-time fault detector for freight train images should first be robust to the illumination variation in the field environment. Then it needs to achieve an outstanding trade-off between speed and accuracy under stringent resource requirements which are not only computational cost for speed, but also memory resources on hardware.

\begin{figure*}[!t]
 \centering
 \subfloat[]{
 \includegraphics[height=1.5in]{Figures/Figure1a.jpg}}
 \hspace{1em}
 \subfloat[]{
 \includegraphics[height=2.2in]{Figures/Figure1b.jpg}}
 \caption{Real-time fault detection for freight train images. (a) Image acquisition devices contain high speed cameras and auxiliary lights, which are installed on both sides and in the middle of railway tracks. (b) Some typical samples of freight train images. Some samples of freight train images are used to train fault detector. The final detection results include the location of the fault and its type.}
 \label{fig:detection}
\end{figure*}

To solve these problems, we propose a light and accurate framework to fulfill real-time fault detection task for freight train images. Over the years, many innovative light-weight networks have been proposed such as SqueezeNet~\cite{iandola2016squeezenet} and MobileNet~\cite{sandler2018inverted}, etc. Meanwhile, many researches are devoted to putting these light-weight backbones into practice~\cite{muhammad2019efficient, 8673908}. Inspired by the SqueezeNet, we firstly design a novel real-time fault detection network (RFDNet) as a backbone to improve accuracy while optimizing the network to meet resource requirement (hardware). It is also proved that our proposed RFDNet is robust to the illumination variation in freight train images. To improve the detection performance, we introduce a fault multi region proposal network (multi-RPN) by fusing the features from multiple layers in RFDNet. Unlike superior detectors such as Faster R-CNN and R-FCN only perform region of interest (RoI) pooling, we present multi level position-sensitive (MLPS) score maps and RoI pooling by using multi scale features for fault region detection. Experimental results on ImageNet, PASCAL VOC, and MS COCO datasets demonstrate that our RFDNet achieves much better performance than SqueezeNet. In addition, the extensive experiments on six fault datasets show that our framework can be effectively applied to achieve real-time fault detection at over 38 fps. Our framework achieves competitive accuracy and lower resource requirements such as 19.6 MB model size, compared with the state-of-the-art detectors.


In summary, this work makes the following contributions.
\begin{itemize}
  \item We design a light and accurate framework to achieve real-time fault detection for freight train images under stringent resource requirements.
  \item We propose a light-weight backbone RFDNet to significantly improve detection accuracy and reduce computational cost, which is confirmed to be robust to illumination changes.
  \item We introduce multi-RPN and MLPS by using multi-scale feature maps generated from RFDNet to improve the detection performance with few redundant computations.
  \item We validate the effectiveness of our RFDNet on public benchmark datasets and our method on six fault datasets with thorough ablation studies. Compared with the state-of-the-art methods, our framework achieves real-time detection at over 38 fps with competitive accuracy and lower resource requirements.
\end{itemize}

The rest of this paper is organized as follows. Section II presents some related works about fault detection for freight train images, object detection, and light-weight neural networks. Section III describes our framework and each important module. Comprehensive experiments are shown in Section IV to validate the superiority of our method and finally Section V concludes this paper.

\begin{figure*}[!t]
 \centering
 \includegraphics[width=6.3in]{Figures/Figure2.jpg}
 \caption{Pipeline of our proposed framework for real-time fault detection of freight train images. Our framework consists of three parts: real-time fault detection network (RFDNet), multi region proposal network, as well as multi level position-sensitive score maps and RoI pooling. The proposed framework takes an image as input, generates hundreds of fault region proposals via multi region proposal network from RFDNet, and then scores each proposal using multi level position-sensitive score maps and RoI pooling.}
 \label{fig:pipeline}
\end{figure*} 