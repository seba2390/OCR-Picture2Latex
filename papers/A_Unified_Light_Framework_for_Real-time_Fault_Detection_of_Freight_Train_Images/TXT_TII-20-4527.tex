
%% bare_jrnl.tex
%% V1.4b
%% 2015/08/26
%% by Michael Shell
%% see http://www.michaelshell.org/
%% for current contact information.
%%
%% This is a skeleton file demonstrating the use of IEEEtran.cls
%% (requires IEEEtran.cls version 1.8b or later) with an IEEE
%% journal paper.
%%
%% Support sites:
%% http://www.michaelshell.org/tex/ieeetran/
%% http://www.ctan.org/pkg/ieeetran
%% and
%% http://www.ieee.org/

%%*************************************************************************
%% Legal Notice:
%% This code is offered as-is without any warranty either expressed or
%% implied; without even the implied warranty of MERCHANTABILITY or
%% FITNESS FOR A PARTICULAR PURPOSE!
%% User assumes all risk.
%% In no event shall the IEEE or any contributor to this code be liable for
%% any damages or losses, including, but not limited to, incidental,
%% consequential, or any other damages, resulting from the use or misuse
%% of any information contained here.
%%
%% All comments are the opinions of their respective authors and are not
%% necessarily endorsed by the IEEE.
%%
%% This work is distributed under the LaTeX Project Public License (LPPL)
%% ( http://www.latex-project.org/ ) version 1.3, and may be freely used,
%% distributed and modified. A copy of the LPPL, version 1.3, is included
%% in the base LaTeX documentation of all distributions of LaTeX released
%% 2003/12/01 or later.
%% Retain all contribution notices and credits.
%% ** Modified files should be clearly indicated as such, including  **
%% ** renaming them and changing author support contact information. **
%%*************************************************************************


% *** Authors should verify (and, if needed, correct) their LaTeX system  ***
% *** with the testflow diagnostic prior to trusting their LaTeX platform ***
% *** with production work. The IEEE's font choices and paper sizes can   ***
% *** trigger bugs that do not appear when using other class files.       ***                          ***
% The testflow support page is at:
% http://www.michaelshell.org/tex/testflow/



\documentclass[journal]{IEEEtran}
%
% If IEEEtran.cls has not been installed into the LaTeX system files,
% manually specify the path to it like:
% \documentclass[journal]{../sty/IEEEtran}





% Some very useful LaTeX packages include:
% (uncomment the ones you want to load)


% *** MISC UTILITY PACKAGES ***
%
%\usepackage{ifpdf}
% Heiko Oberdiek's ifpdf.sty is very useful if you need conditional
% compilation based on whether the output is pdf or dvi.
% usage:
% \ifpdf
%   % pdf code
% \else
%   % dvi code
% \fi
% The latest version of ifpdf.sty can be obtained from:
% http://www.ctan.org/pkg/ifpdf
% Also, note that IEEEtran.cls V1.7 and later provides a builtin
% \ifCLASSINFOpdf conditional that works the same way.
% When switching from latex to pdflatex and vice-versa, the compiler may
% have to be run twice to clear warning/error messages.






% *** CITATION PACKAGES ***
%
\usepackage{cite}
% cite.sty was written by Donald Arseneau
% V1.6 and later of IEEEtran pre-defines the format of the cite.sty package
% \cite{} output to follow that of the IEEE. Loading the cite package will
% result in citation numbers being automatically sorted and properly
% "compressed/ranged". e.g., [1], [9], [2], [7], [5], [6] without using
% cite.sty will become [1], [2], [5]--[7], [9] using cite.sty. cite.sty's
% \cite will automatically add leading space, if needed. Use cite.sty's
% noadjust option (cite.sty V3.8 and later) if you want to turn this off
% such as if a citation ever needs to be enclosed in parenthesis.
% cite.sty is already installed on most LaTeX systems. Be sure and use
% version 5.0 (2009-03-20) and later if using hyperref.sty.
% The latest version can be obtained at:
% http://www.ctan.org/pkg/cite
% The documentation is contained in the cite.sty file itself.






% *** GRAPHICS RELATED PACKAGES ***
%
\ifCLASSINFOpdf
   \usepackage[pdftex]{graphicx}
  % declare the path(s) where your graphic files are
   \graphicspath{{../pdf/}{../jpeg/}}
  % and their extensions so you won't have to specify these with
  % every instance of \includegraphics
   \DeclareGraphicsExtensions{.pdf,.jpeg,.png}
\else
  % or other class option (dvipsone, dvipdf, if not using dvips). graphicx
  % will default to the driver specified in the system graphics.cfg if no
  % driver is specified.
   \usepackage[dvips]{graphicx}
  % declare the path(s) where your graphic files are
   \graphicspath{{../eps/}}
  % and their extensions so you won't have to specify these with
  % every instance of \includegraphics
   \DeclareGraphicsExtensions{.eps}
\fi
% graphicx was written by David Carlisle and Sebastian Rahtz. It is
% required if you want graphics, photos, etc. graphicx.sty is already
% installed on most LaTeX systems. The latest version and documentation
% can be obtained at:
% http://www.ctan.org/pkg/graphicx
% Another good source of documentation is "Using Imported Graphics in
% LaTeX2e" by Keith Reckdahl which can be found at:
% http://www.ctan.org/pkg/epslatex
%
% latex, and pdflatex in dvi mode, support graphics in encapsulated
% postscript (.eps) format. pdflatex in pdf mode supports graphics
% in .pdf, .jpeg, .png and .mps (metapost) formats. Users should ensure
% that all non-photo figures use a vector format (.eps, .pdf, .mps) and
% not a bitmapped formats (.jpeg, .png). The IEEE frowns on bitmapped formats
% which can result in "jaggedy"/blurry rendering of lines and letters as
% well as large increases in file sizes.
%
% You can find documentation about the pdfTeX application at:
% http://www.tug.org/applications/pdftex





% *** MATH PACKAGES ***
%
\usepackage{amsmath}
% A popular package from the American Mathematical Society that provides
% many useful and powerful commands for dealing with mathematics.
%
% Note that the amsmath package sets \interdisplaylinepenalty to 10000
% thus preventing page breaks from occurring within multiline equations. Use:
%\interdisplaylinepenalty=2500
% after loading amsmath to restore such page breaks as IEEEtran.cls normally
% does. amsmath.sty is already installed on most LaTeX systems. The latest
% version and documentation can be obtained at:
% http://www.ctan.org/pkg/amsmath





% *** SPECIALIZED LIST PACKAGES ***
%
\usepackage{algorithmic}
% algorithmic.sty was written by Peter Williams and Rogerio Brito.
% This package provides an algorithmic environment fo describing algorithms.
% You can use the algorithmic environment in-text or within a figure
% environment to provide for a floating algorithm. Do NOT use the algorithm
% floating environment provided by algorithm.sty (by the same authors) or
% algorithm2e.sty (by Christophe Fiorio) as the IEEE does not use dedicated
% algorithm float types and packages that provide these will not provide
% correct IEEE style captions. The latest version and documentation of
% algorithmic.sty can be obtained at:
% http://www.ctan.org/pkg/algorithms
% Also of interest may be the (relatively newer and more customizable)
% algorithmicx.sty package by Szasz Janos:
% http://www.ctan.org/pkg/algorithmicx




% *** ALIGNMENT PACKAGES ***
%
\usepackage{array}
% Frank Mittelbach's and David Carlisle's array.sty patches and improves
% the standard LaTeX2e array and tabular environments to provide better
% appearance and additional user controls. As the default LaTeX2e table
% generation code is lacking to the point of almost being broken with
% respect to the quality of the end results, all users are strongly
% advised to use an enhanced (at the very least that provided by array.sty)
% set of table tools. array.sty is already installed on most systems. The
% latest version and documentation can be obtained at:
% http://www.ctan.org/pkg/array


% IEEEtran contains the IEEEeqnarray family of commands that can be used to
% generate multiline equations as well as matrices, tables, etc., of high
% quality.




% *** SUBFIGURE PACKAGES ***
\ifCLASSOPTIONcompsoc
  \usepackage[caption=false,font=normalsize,labelfont=sf,textfont=sf]{subfig}
\else
  \usepackage[caption=false,font=footnotesize]{subfig}
\fi
% subfig.sty, written by Steven Douglas Cochran, is the modern replacement
% for subfigure.sty, the latter of which is no longer maintained and is
% incompatible with some LaTeX packages including fixltx2e. However,
% subfig.sty requires and automatically loads Axel Sommerfeldt's caption.sty
% which will override IEEEtran.cls' handling of captions and this will result
% in non-IEEE style figure/table captions. To prevent this problem, be sure
% and invoke subfig.sty's "caption=false" package option (available since
% subfig.sty version 1.3, 2005/06/28) as this is will preserve IEEEtran.cls
% handling of captions.
% Note that the Computer Society format requires a larger sans serif font
% than the serif footnote size font used in traditional IEEE formatting
% and thus the need to invoke different subfig.sty package options depending
% on whether compsoc mode has been enabled.
%
% The latest version and documentation of subfig.sty can be obtained at:
% http://www.ctan.org/pkg/subfig




% *** FLOAT PACKAGES ***
%
%\usepackage{fixltx2e}
% fixltx2e, the successor to the earlier fix2col.sty, was written by
% Frank Mittelbach and David Carlisle. This package corrects a few problems
% in the LaTeX2e kernel, the most notable of which is that in current
% LaTeX2e releases, the ordering of single and double column floats is not
% guaranteed to be preserved. Thus, an unpatched LaTeX2e can allow a
% single column figure to be placed prior to an earlier double column
% figure.
% Be aware that LaTeX2e kernels dated 2015 and later have fixltx2e.sty's
% corrections already built into the system in which case a warning will
% be issued if an attempt is made to load fixltx2e.sty as it is no longer
% needed.
% The latest version and documentation can be found at:
% http://www.ctan.org/pkg/fixltx2e


\usepackage{stfloats}
% stfloats.sty was written by Sigitas Tolusis. This package gives LaTeX2e
% the ability to do double column floats at the bottom of the page as well
% as the top. (e.g., "\begin{figure*}[!b]" is not normally possible in
% LaTeX2e). It also provides a command:
%\fnbelowfloat
% to enable the placement of footnotes below bottom floats (the standard
% LaTeX2e kernel puts them above bottom floats). This is an invasive package
% which rewrites many portions of the LaTeX2e float routines. It may not work
% with other packages that modify the LaTeX2e float routines. The latest
% version and documentation can be obtained at:
% http://www.ctan.org/pkg/stfloats
% Do not use the stfloats baselinefloat ability as the IEEE does not allow
% \baselineskip to stretch. Authors submitting work to the IEEE should note
% that the IEEE rarely uses double column equations and that authors should try
% to avoid such use. Do not be tempted to use the cuted.sty or midfloat.sty
% packages (also by Sigitas Tolusis) as the IEEE does not format its papers in
% such ways.
% Do not attempt to use stfloats with fixltx2e as they are incompatible.
% Instead, use Morten Hogholm'a dblfloatfix which combines the features
% of both fixltx2e and stfloats:
%
% \usepackage{dblfloatfix}
% The latest version can be found at:
% http://www.ctan.org/pkg/dblfloatfix




%\ifCLASSOPTIONcaptionsoff
%  \usepackage[nomarkers]{endfloat}
% \let\MYoriglatexcaption\caption
% \renewcommand{\caption}[2][\relax]{\MYoriglatexcaption[#2]{#2}}
%\fi
% endfloat.sty was written by James Darrell McCauley, Jeff Goldberg and
% Axel Sommerfeldt. This package may be useful when used in conjunction with
% IEEEtran.cls'  captionsoff option. Some IEEE journals/societies require that
% submissions have lists of figures/tables at the end of the paper and that
% figures/tables without any captions are placed on a page by themselves at
% the end of the document. If needed, the draftcls IEEEtran class option or
% \CLASSINPUTbaselinestretch interface can be used to increase the line
% spacing as well. Be sure and use the nomarkers option of endfloat to
% prevent endfloat from "marking" where the figures would have been placed
% in the text. The two hack lines of code above are a slight modification of
% that suggested by in the endfloat docs (section 8.4.1) to ensure that
% the full captions always appear in the list of figures/tables - even if
% the user used the short optional argument of \caption[]{}.
% IEEE papers do not typically make use of \caption[]'s optional argument,
% so this should not be an issue. A similar trick can be used to disable
% captions of packages such as subfig.sty that lack options to turn off
% the subcaptions:
% For subfig.sty:
% \let\MYorigsubfloat\subfloat
% \renewcommand{\subfloat}[2][\relax]{\MYorigsubfloat[]{#2}}
% However, the above trick will not work if both optional arguments of
% the \subfloat command are used. Furthermore, there needs to be a
% description of each subfigure *somewhere* and endfloat does not add
% subfigure captions to its list of figures. Thus, the best approach is to
% avoid the use of subfigure captions (many IEEE journals avoid them anyway)
% and instead reference/explain all the subfigures within the main caption.
% The latest version of endfloat.sty and its documentation can obtained at:
% http://www.ctan.org/pkg/endfloat
%
% The IEEEtran \ifCLASSOPTIONcaptionsoff conditional can also be used
% later in the document, say, to conditionally put the References on a
% page by themselves.




% *** PDF, URL AND HYPERLINK PACKAGES ***
%
\usepackage{url}
% url.sty was written by Donald Arseneau. It provides better support for
% handling and breaking URLs. url.sty is already installed on most LaTeX
% systems. The latest version and documentation can be obtained at:
% http://www.ctan.org/pkg/url
% Basically, \url{my_url_here}.




% *** Do not adjust lengths that control margins, column widths, etc. ***
% *** Do not use packages that alter fonts (such as pslatex).         ***
% There should be no need to do such things with IEEEtran.cls V1.6 and later.
% (Unless specifically asked to do so by the journal or conference you plan
% to submit to, of course. )


\usepackage{booktabs}       % professional-quality tables
\usepackage{amsfonts}       % blackboard math symbols
\usepackage{multirow}
\usepackage{amsmath}

%\usepackage[switch]{lineno}
% correct bad hyphenation here
\hyphenation{op-tical net-works semi-conduc-tor}


\begin{document}
%\linenumbers	
% paper title
% Titles are generally capitalized except for words such as a, an, and, as,
% at, but, by, for, in, nor, of, on, or, the, to and up, which are usually
% not capitalized unless they are the first or last word of the title.
% Linebreaks \\ can be used within to get better formatting as desired.
% Do not put math or special symbols in the title.
\title{A Unified Light Framework for Real-time Fault Detection of Freight Train Images}
% author names and IEEE memberships
% note positions of commas and nonbreaking spaces ( ~ ) LaTeX will not break
% a structure at a ~ so this keeps an author's name from being broken across
% two lines.
% use \thanks{} to gain access to the first footnote area
% a separate \thanks must be used for each paragraph as LaTeX2e's \thanks
% was not built to handle multiple paragraphs
%
%
\author{Yang~Zhang,~Moyun~Liu,~Yang~Yang,~Yanwen~Guo,~and~Huiming~Zhang
%\thanks{Manuscript received ** **, 2020; revised ** **, 2020. accepted ** **, 2021.
%Date of publication ** **, 2021; date of current version ** **, 2021. This work was
%supported in part by the National Natural Science Foundation of China
%(Grant 62032011, 61772257, 51675166, and 61672279), and the program B for Outstanding PhD
%candidate  of  Nanjing University (Grant 202001B054). Paper no. TII-20-4527. 
%\emph{(Corresponding author: Moyun~Liu.)}}
\thanks{Y.~Zhang is with the School of Mechanical Engineering, Hubei University of 
Technology, Wuhan 430068, China, and also with the National Key
Laboratory for Novel Software Technology, Nanjing University, Nanjing 210023,
China (e-mail: yzhangcst@smail.nju.edu.cn)}
\thanks{M. Liu is with the School of Mechanical Science and Engineering,
Huazhong University of Science and Technology, Wuhan 430074, China
(e-mail: lmomoy8@gmail.com)}
\thanks{Y.~Yang, Y.~Guo, H.~Zhang and are with the National Key
Laboratory for Novel Software Technology, Nanjing University, Nanjing 210023,
China (e-mail: yyang\_nju@outlook.com; ywguo@nju.edu.cn; zhanghmcst@163.com).}% <-this % stops a space
%\thanks{Color versions of one or more of the figures in this article are 
%available online at http://ieeexplore.ieee.org.}
%\thanks{Digital Object Identifier}
}

% note the % following the last \IEEEmembership and also \thanks -
% these prevent an unwanted space from occurring between the last author name
% and the end of the author line. i.e., if you had this:
%
% \author{....lastname \thanks{...} \thanks{...} }
%                     ^------------^------------^----Do not want these spaces!
%
% a space would be appended to the last name and could cause every name on that
% line to be shifted left slightly. This is one of those "LaTeX things". For
% instance, "\textbf{A} \textbf{B}" will typeset as "A B" not "AB". To get
% "AB" then you have to do: "\textbf{A}\textbf{B}"
% \thanks is no different in this regard, so shield the last } of each \thanks
% that ends a line with a % and do not let a space in before the next \thanks.
% Spaces after \IEEEmembership other than the last one are OK (and needed) as
% you are supposed to have spaces between the names. For what it is worth,
% this is a minor point as most people would not even notice if the said evil
% space somehow managed to creep in.



% The paper headers
\markboth{IEEE TRANSACTIONS ON INDUSTRIAL INFORMATICS,~Vol.~**, No.~*, ***~****}%
{Shell \MakeLowercase{\textit{et al.}}: Bare Demo of IEEEtran.cls for IEEE Journals}
% The only time the second header will appear is for the odd numbered pages
% after the title page when using the twoside option.
%
% *** Note that you probably will NOT want to include the author's ***
% *** name in the headers of peer review papers.                   ***
% You can use \ifCLASSOPTIONpeerreview for conditional compilation here if
% you desire.




% If you want to put a publisher's ID mark on the page you can do it like
% this:
%\IEEEpubid{0000--0000/00\$00.00~\copyright~2015 IEEE}
% Remember, if you use this you must call \IEEEpubidadjcol in the second
% column for its text to clear the IEEEpubid mark.



% use for special paper notices
%\IEEEspecialpapernotice{(Invited Paper)}




% make the title area
\maketitle

% As a general rule, do not put math, special symbols or citations
% in the abstract or keywords.
\begin{abstract}
  Real-time fault detection for freight trains plays a vital role in guaranteeing the security and optimal operation of railway transportation under stringent resource requirements. Despite the promising results for deep learning based approaches, the performance of these fault detectors on freight train images, are far from satisfactory in both accuracy and efficiency. This paper proposes a unified light framework to improve detection accuracy while supporting a real-time operation with a low resource requirement. We firstly design a novel light-weight backbone  (RFDNet) to improve the accuracy and reduce computational cost. Then, we propose a multi region proposal network using multi-scale feature maps generated from RFDNet to improve the detection performance. Finally, we present multi level position-sensitive score maps and region of interest pooling to further improve accuracy with few redundant computations. Extensive experimental results on public benchmark datasets suggest that our RFDNet can significantly improve the performance of baseline network with higher accuracy and efficiency. Experiments on six fault datasets show that our method is capable of real-time detection at over 38 frames per second and achieves competitive accuracy and lower computation than the state-of-the-art detectors.
\end{abstract}

% Note that keywords are not normally used for peerreview papers.
\begin{IEEEkeywords}
Real-time, fault detection, freight train images, light-weight, multi-scale.
\end{IEEEkeywords}

% For peer review papers, you can put extra information on the cover
% page as needed:
% \ifCLASSOPTIONpeerreview
% \begin{center} \bfseries EDICS Category: 3-BBND \end{center}
% \fi
%
% For peerreview papers, this IEEEtran command inserts a page break and
% creates the second title. It will be ignored for other modes.
\IEEEpeerreviewmaketitle


\section{Introduction}
A classical result, due to M. Eichler and G. Shimura, states that the first cohomology of the complex modular curve with coefficients in $\mathrm{Sym}^{k-2}\Q^2$, after scalar extension to $\C$, admits a Hecke-equivariant decomposition as the direct sum of the space of weight $k$ holomorphic modular forms and the space of weight $k$ anti-holomorphic cusp forms.

G. Faltings establishes a $p$-adic analogue of the Eichler--Shimura decomposition. Let $p$ be a prime number and let $\C_p$ be the completion of a fixed algebraic closure $\overline{\Q}_p$ of $\Q_p$. Suppose $X$ is the modular curve (of some tame level $N$) over $\Q_p$ and let $\overline{X}$ be its compactification. Let $\pi: E\rightarrow \overline{X}$ be the universal generalised elliptic curve and let $\underline{\omega}:=\pi_*\Omega^1_{E/\overline{X}}$. In \cite{FaltingsHT}, Faltings constructs a Hecke- and $\mathrm{Gal}(\overline{\Q}_p/\Q_p)$-equivariant decomposition
\[
H_{\et}^1(X_{\overline{\Q}_p},\mathrm{Sym}^{k}\Q_p^2)\otimes_{\Q_p} \C_p(1) \simeq \big(H^0(\overline{X},\underline{\omega}^{k+2})\otimes_{\Q_p} \C_p\big) \oplus \big(H^1(\overline{X},\underline{\omega}^{-k})\otimes_{\Q_p} \C_p (k+1)\big) 
\]
where the Galois actions on the coherent cohomology groups are trivial. 

This type of results have been generalised to families of overconvergent modular forms by F. Andreatta, A. Iovita and G. Stevens in \cite{AIS-2015} and later extended to the case of compact Shimura curves over $\Q$ by P. Chojecki, D. Hansen and C. Johansson in \cite{CHJ-2017}. The novelty of the second work consists of a ``perfectoid construction'' of overconvergent families of automorphic forms as well as a succinct calculation of overconvergent cohomology groups on the pro-\'etale site. The same method has also been adapted to the case of elliptic and Hilbert modular forms in \cite{BHW-2019}.

There are two main goals we would like to achieve in this paper. We first present a construction of automorphic sheaves for overconvergent Siegel modular forms (of genus $g$) using the perfectoid method. Then we follow the spirit of Chojecki--Hansen--Johansson to establish an overconvergent Eichler--Shimura morphism for Siegel modular forms. 

One key ingredient for such a perfectoid construction is the (toroidally compactified) perfectoid Siegel modular variety $\overline{\calX}_{\Gamma(p^{\infty})}$ studied in \cite{Pilloni-Stroh-CoherentCohomologyandGaloisRepresentations}. This perfectoid space comes equipped with the \emph{Hodge--Tate period map} $\pi_{\HT}: \overline{\calX}_{\Gamma(p^{\infty})}\rightarrow \adicFL$ where the adic flag variety $\adicFL$ parameterises maximal lagrangian subspaces of a fixed symplectic space of rank $2g$. We also consider the Siegel modular variety $\overline{\calX}_{\Iw^+}$ for the finite \emph{strict Iwahori level} (see Definition \ref{Definition: Siegel modular varieties of (strict) Iwahoris level} for details), viewed as an adic space. Roughly speaking, the strict Iwahori level serves as a deeper level compared with the usual Iwahori level. There is a natural projection $\overline{\calX}_{\Gamma(p^{\infty})}\rightarrow \overline{\calX}_{\Iw^+}$ from the Siegel modular variety at infinite level to the one at finite level.

In order to investigate the overconvergent Siegel modular forms, we restrict our attentions to certain open subspaces $\overline{\calX}_{\Gamma(p^{\infty}), w}$ and $\overline{\calX}_{\Iw^+, w}$ of $\overline{\calX}_{\Gamma(p^{\infty})}$ and $\overline{\calX}_{\Iw^+}$, respectively. They are referred as the \emph{$w$-ordinary loci} (see Definition \ref{Definition: w-ordinary}) and they contain the usual ordinary locus.

Let $(R_{\calU}, \kappa_{\calU})$ be a weight (see Definition \ref{Definition: weights}) and let $r_{\calU}$ be the integer as defined in Definition \ref{Definition: w-analytic weight}. For any $w\in \Q_{>0}$ with $w>1+r_{\calU}$, we construct a sheaf $\underline{\omega}_w^{\kappa_{\calU}}$ over $\overline{\calX}_{\Iw^+, w}$ whose global sections are exactly the $w$-overconvergent Siegel modular forms of strict Iwahori level and weight $\kappa_{\calU}$. Roughly speaking, sections of $\underline{\omega}_w^{\kappa_{\calU}}$ consist of functions $f$ on $\overline{\calX}_{\Gamma(p^{\infty}),w}$ which 
\begin{enumerate}
    \item[$\bullet$] take value in a certain weight-$\kappa_{\calU}$ analytic representation $C_{\kappa_{\calU}}^{w-\an}(\Iw_{\GL_g}, \C_p\widehat{\otimes}R_{\calU})$ of the Iwahori subgroup of $\GL_g(\Z_p)$, and
    \item[$\bullet$] satisfy the following formula regarding the natural action of the stricit Iwahori subgroup $\Iw_{\GSp_{2g}}^+$ of $\GSp_{2g}(\Z_p)$:
    \[
        \bfgamma^* f=\rho_{\kappa_{\calU}}(\bfgamma_a+\frakz\bfgamma_c)^{-1}f\qquad\text{for any} \qquad\bfgamma=\begin{pmatrix}\bfgamma_a & \bfgamma_b\\ \bfgamma_c & \bfgamma_d\end{pmatrix}\in \Iw^+_{\GSp_{2g}},
    \]
    where $\frakz$ stands for the pullback of the coordinate function on the flag variety and  $\rho_{\kappa_{\calU}}(\bfgamma_a+\frakz\bfgamma_c)$ stands for a certain automorphism on $C_{\kappa_{\calU}}^{w-\an}(\Iw_{\GL_g}, \C_p\widehat{\otimes}R_{\calU})$. 
\end{enumerate}
When $p>2g$, this sheaf coincides with (the pullback to the strict Iwahori level of) the automorphic sheaf constructed in \cite{AIP-2015}. See \S \ref{section:constructionsheaf} for a complete story.

The next step is to look at the overconvergent cohomology groups. Suppose the weight $(R_{\calU}, \kappa_{\calU})$ is \emph{small} in the sense of Definition \ref{Definition: weights}. A typical example of a small weight is a \emph{wide-open} disc whose ring of functions is isomorphic to $\Z_p\llbrack  T_1, \ldots, T_g \rrbrack$. For such a weight, we consider a space $D^r_{\kappa}(\T_0, R_{\calU})$ of analytic distributions on certain $p$-adic manifold $\T_0$. (See \S \ref{subsection:continuousfunctions} for the precise definitions.) This space of distributions is equipped with a natural action of the strict Iwahori subgroup and hence induces a sheaf $\scrD_{\kappa_{\calU}}^r$ on the Siegel modular variety $\overline{\calX}_{\Iw^+}$. The cohomology groups of $\scrD_{\kappa_{\calU}}^r$ are precisely the \emph{overconvergent modular symbols} studied by A. Ash and G. Stevens, among others (see for example \cite{Ash-Stevens, Hansen-PhD, Johansson-Newton}).

Our goal is to explicitly construct a morphism that relates the automorphic sheaves with the overconvergent cohomology groups. To this end, we will compute the objects on both sides on the \emph{pro-Kummer \'etale site} $\overline{\calX}_{\Iw^+, \proket}$. On one hand, the overconvergent cohomology groups can be calculated in terms of the cohomology groups of some sheaf of distributions $\sheafOD_{\kappa_{\calU}}^r$ over the pro-Kummer \'etale site. On the other hand, let $\widehat{\underline{\omega}}_{w}^{\kappa_{\calU}}$ be the ``completed'' pull-back of the automorphic sheaf $\underline{\omega}_w^{\kappa_{\calU}}$ to this site. The key step is then Proposition \ref{Proposition: OES for sheaves on the pro-Kummer etale site} where we construct a morphism of sheaves
\[
\sheafOD_{\kappa_{\calU}}^r\rightarrow \widehat{\underline{\omega}}_{w}^{\kappa_{\calU}}.
\]
This morphism can be thought of as an analytic analogue of the ``projection onto the highest weight vector''. We remark that the construction involves taking transposes of matrices while the usual Iwahori subgroup is not closed under taking transposes. This is why we have to work with a smaller subgroup---the strict Iwahori subgroup---as a compromise.

Consequently, we obtain the \emph{overconvergent Eichler-Shimura morphism}. It is a natural Hecke- and Galois-equivariant morphism from the (family of) oveconvergent cohomology groups to the (family of) overconvergent Siegel modular forms
\[
    \ES_{\kappa_{\calU}}: H_{\proket}^{n_0}(\overline{\calX}_{\Iw^+}, \sheafOD_{\kappa_{\calU}}^r) \rightarrow H^0(\overline{\calX}_{\Iw^+, w}, \,\underline{\omega}_w^{\kappa_{\calU}+g+1})(-n_0),
\] where $n_0 = \dim_{\C_p} \overline{\calX}_{\Iw^+}$ and ``$(-n_0)$'' stands for the Tate twist.

We list some properties of this map. First of all, we show that it is compatible with base change on the weights. Secondly, we are able to control its image when specialising to a dominant algebraic weight $k\in \Z_{\geq 0}^{g}$. In particular, we show in Theorem \ref{thm:imageclassicalweight} that the image of $\ES_{k}$ is contained in the space of classical algebraic Siegel modular forms. The proof uses the fact that when the weight is a dominant algebraic weight, the ``highest weight vector'' is an algebraic function. Lastly, we conclude the paper by showing that $\ES_{\kappa_{\calU}}$ can be glued to a morphism of sheaves on a suitable cuspidal eigenvariety.

There are many things we don't do: we don't control the cokernel of the map $\ES_{\kappa_{\calU}}$ in general and we do not investigate the kernel of this map at all. A natural expectation is that one could construct a full filtration of $H_{\proket}^{n_0}(\overline{\calX}_{\Iw^+}, \sheafOD_{\kappa_{\calU}}^r) $ using higher Coleman theory, recently developed by G. Boxer and V. Pilloni \cite{Boxer--Pilloni--higherColeman}, in terms of information on suitable strata of $\overline{\calX}_{\Iw^+}$. When $g=1$, such a result is recently announced by J. E. Rodr\'iguez Camargo (\cite{Rodriguez-dualOES}).

We also remark that while preparing this paper, Andreatta and Iovita have announced the article \cite{AI-2020}, in which they upgrade their previous work \cite{AIS-2015} to an ``overconvergent de Rham Eichler--Shimura morphism'', meaning that their Eichler--Shimura map has as source the overconvergent cohomology groups tensored with $\bbB_{\mathrm{dR}}$.
They have also announced upcoming work concerning this type of de Rham Eichler--Shimura morphisms for overconvergent Siegel modular forms of genus $2$. This suggests that the results in this paper can be upgraded to investigate finer $p$-adic Hodge theoretic properties of overconvergent cohomology groups; for example, the construction of de Rham (or even crystalline) periods in $p$-adic families. We shall leave this to future studies. 

The paper is organised as follows. In \S \ref{section:PerfectoidSMV}, we define the main geometric objects of interest, including the Siegel modular varieties for various level structures, the perfectoid Siegel modular variety at infinite level, the flag variety, and the Hodge--Tate period map. The next section, \S \ref{section:constructionsheaf}, contributes to the construction of the overconvergent automorphic sheaves. In particular, when $p>2g$, we show that our construction coincides with the (pullback to the strict Iwahori level of the) automorphic sheaves of Andreatta--Iovita--Pilloni. We warn the reader that \S \ref{subsection: admissibility}-\ref{subsection:comparison sheaf aip} is the most technical part of the paper. These subsections can be skipped on a first reading. In \S \ref{section:overconvergentcohomologies}, we study the space of analytic distributions and the overconvergent cohomology groups. Finally, in \S \ref{section:EichlerShimura} and \S \ref{section:oncuspidaleigenvariety}, we construct the Eichler--Shimura morphism as well as its upgraded version on the cuspidal eigenvariety. 

In \S \ref{Section: Kummer etale and pro-Kummer etale sites of log adic spaces}, we recall the basics of logarithmic adic spaces and their Kummer pro-\'etale topology following \cite{Diao}. Also in the same section, we include some technical calculations of the derived functor $R^i \nu_*$ and a generalised projection formula, both of which are used in the main text. In \S \ref{section: boundary}, we recall the basics of the toroidal compactifications of Siegel modular varieties. We also review the ``modified integral structures'' studied in \cite{Pilloni-Stroh-CoherentCohomologyandGaloisRepresentations} as well as the construction of the perfectoid Siegel modular variety at the infinite level.

\paragraph{Acknowledgement.} The authors would like to express their gratitude to Przemy{\l}aw Chojecki, David Hansen, and Christian Johansson for discussions on the perfectoid construction of sheaves of families of Siegel modular forms and for pointing out an inaccuracy in the first version of this paper. G.R. would like to thank Daniel Barrera-Salazar and Riccardo Brasca. J.-F. W. would like to thank Ulrich G\"ortz for helpful conversations about toroidal compactifications of Siegel modular varieties and $p$-divisible groups. He would also like to thank Marc Levine for a succinct introduction on $1$-motives during a casual conversation. 


\paragraph{Conventions and notations.} Through out this paper, we fix the following: \begin{enumerate}
    \item[$\bullet$] $g\in \Z_{\geq 1}$.
    \item[$\bullet$] $p\in \Z_{> 0}$ is an odd prime number. Due to certain technicality, we will have to assume $p>2g$ at some places. Such an assumption shall be clear in the context. 
    \item[$\bullet$] $N\in \Z_{\geq 3}$ is an integer coprime to $p$.
    \item[$\bullet$] We fix once and forever an algebraic closure $\overline{\Q}_p$ of $\Q_p$ and an algebraic isomorphism $\C_p\simeq \C$, where $\C_p$ is the $p$-adic completion of $\overline{\Q}_p$. We write $G_{\Q_p}$ for the absolute Galois group $\Gal(\overline{\Q}_p/\Q_p)$. We also fix the $p$-adic absolute value on $\C_p$ so that $|p|=p^{-1}$.
    \item[$\bullet$] For any $w\in \Q_{>0}$, we denote by ``$p^w$'' an element in $\C_p$ with absolute value $p^{-w}$. All constructions in the paper will not depend on such choices.
    \item[$\bullet$] We adopt the language of almost mathematics. In particular, for an $\calO_{\C_p}$-module $M$, we denote by $M^a$ for the associated almost $\calO_{\C_p}$-module.
    \item[$\bullet$] For $n\in \Z_{\geq 1}$ and any set $R$, we denote by $M_n(R)$ the set of $n$ by $n$ matrices with coefficients in $R$.
    \item[$\bullet$] The transpose of a matrix $\bfalpha$ is denoted by $\trans\bfalpha$.
    \item[$\bullet$] For any $n\in \Z_{\geq 1}$, we denote by $\one_n$ the $n\times n$ identity matrix and denote by $\oneanti_n$ the $n\times n$ anti-diagonal matrix whose non-zero entries are $1$; \emph{i.e.,} $$\one_n=\begin{pmatrix} 1& & \\ & \ddots & \\ & &1\end{pmatrix}\quad\text{ and }\quad\oneanti_n=\begin{pmatrix} & & 1\\ & \iddots & \\ 1 & &\end{pmatrix}$$
    \item[$\bullet$] We use $\cong$ to denote canonical isomorphisms and $\simeq$ to denote non-canonical ones.
    \item[$\bullet$] In principle (except for \S \ref{Section: Kummer etale and pro-Kummer etale sites of log adic spaces}), symbols in Gothic font (\emph{e.g.} $\frakX, \frakY, \frakZ$) stand for formal schemes; symbols in calligraphic font (\emph{e.g.} $\calX, \calY, \calZ$) stand for adic spaces; and symbols in script font (\emph{e.g.} $\scrO, \scrF, \scrE$) stand for sheaves (over various geometric objects). 
\end{enumerate}
\section{Related works}
\label{relatedworks}
\textbf{Fault Detection for Freight Train Images.}
Some of the recent researches for fault detection of freight train images are listed as follows. Liu \emph{et al}.~\cite{Liu2015Automated} proposed a hierarchical fault inspection framework to detect the missing of bogie block key on freight trains with high speed and accuracy. Zheng \emph{et al}.~\cite{Zheng2016Automatic} proposed an automatic image inspection system to inspect coupler yokes by a linear support vector machine (SVM) classifier for localization and Adaboost decision trees for recognition. However, these methods only detect one type of faults, which greatly influence their effectiveness. In addition, Sun \emph{et al}.~\cite{Sun2018Railway} proposed a fast adaptive Markov random field (FAMRF) for image segmentation and an exact height function (EHF) for shape matching of fault region. This method solves the problem of multi-fault detection, but it is too complex to achieve enough accuracy and fast speed. Differing from the conventional techniques, deep learning methods deal with more complex and difficult problems in machine vision field. Sun \emph{et al}.~\cite{Sun2017Automatic} presented a CNN-based system consisting of two complex models for target region detection and fault recognition, respectively. Pahwa \emph{et al}.~\cite{8917062} performed a two-step high resolution segmentation of the train valves and use image processing techniques to identify faulty valves. Fu \emph{et al}.~\cite{FU2020212} proposed a two-stage method cascading a bearing localization stage and a defection segmentation stage to recognize the defect areas in a coarse-to-fine manner. However, these methods have high computational cost, which are insufficient to meet actual requirements of fault detection like real-time and versatility.

\textbf{Object Detection.}
As the basis of fault detection, object detectors based on CNNs have been developed rapidly over the years, which are widely used in actual applications due to their powerful capacity~\cite{lu2018estimation}. These CNN-based detectors can be divided into two parts: one-stage and two-stage. One-stage detectors directly predict object classes and locations without region proposal generation, such as you only look once (YOLO)~\cite{redmon2018yolov3}, single shot multibox detector (SSD)~\cite{LiuAESRFB16}, reverse connection with objectness prior network (RON)~\cite{kong2017ron}, RefineDet~\cite{zhang2018single}, and deeply supervised object detector (DSOD)~\cite{shen2017dsod}. Based on one-stage strategy, these detectors can obtain fast speed, which are suited for limited computing condition, but usually sacrifices accuracy. Two-stage detectors firstly generate a set of region proposals, and then classify whether they are background or foreground. These detectors achieve accurate and effective object detection such as Faster R-CNN~\cite{RenHGS15}, R-FCN~\cite{DaiLHS16}, multi-scale location-aware kernel representation (MLKP)~\cite{wang2018multi}, and Cascade R-CNN~\cite{cai2018cascade}. Compared with one-stage detectors, these two-stage detectors have higher performance but need more computations. Hence, they are usually incapable of coping with practical application for real-time demand.

\textbf{Light-weight Neural Networks.}
During past years, many efforts are devoted to designing light-weight backbones for object detection task in the resource-restricted conditions. There are some light-weight architecture designs which achieve better speed-accuracy trade-offs, including SqueezeNet~\cite{iandola2016squeezenet}, MobileNet~\cite{sandler2018inverted}, and ShuffleNet~\cite{ma2018shufflenet}, etc. Compared with the superior performance models like ResNet~\cite{He2016Deep}, these networks have fewer parameters with approximate precision. Moreover, there has been a growing interest in incorporating light-weight networks into CNNs for object detection task. For example, Tiny-DSOD~\cite{yuxi2018tinydsod} consists of depthwise dense block based backbone and depthwise feature pyramid network, achieving a better trade-off between resources and accuracy. Pelee~\cite{wang2018pelee} combines a PeleeNet with SSD to keep detection accuracy for mobile applications with fast speed. Each of these light-weight neural networks has a small model size, but its accuracy still has a large room for improvement.

As for real-time fault detection task, both accuracy and computation complexity are important considerations under stringent resource requirements in field environment. To obtain an outstanding balance between accuracy and computational cost, we take inspiration from both the incredible efficiency of the Fire modules introduced in SqueezeNet~\cite{iandola2016squeezenet} and the powerful detection performance demonstrated by the R-FCN~\cite{DaiLHS16}. In addition, the multi-level feature fusion strategy in~\cite{8911418} can be used to further improve the detection accuracy notably without a lot redundant computations.

\section{The Proposed Framework}
\label{method}
In this section, we first introduce our proposed light-weight backbone RFDNet in Section~\ref{sec:DS-SqueezeNet}. To enrich features, we present a multi-RPN to combine different level feature maps, as introduced in Section~\ref{MRPN}. Then, we propose MLPS score maps and RoI pooling to better use the multi-level feature maps, as introduced in Section~\ref{MLPS}. The detailed architecture of the proposed framework is depicted in Fig.~\ref{fig:pipeline}. The proposed framework takes an image as input, generates hundreds of fault region proposals via multi-RPN from RFDNet, and then scores each proposal using MLPS score maps and RoI pooling.

\subsection{Light-weight Backbone}
\label{sec:DS-SqueezeNet}
In original R-FCN, the backbone (\textit{i.e.} ResNet~\cite{He2016Deep}) needs a large number of parameters and floating point operations (FLOPs) to achieve a satisfactory accuracy, thus requiring a huge amount of computations. However, as discussed in the previous section, fault detection task cannot meet the demand of tremendous computing power in the wild. Compared to the ResNet, a light-weight network usually has fewer parameters and lower computations with approximate precision, which is more suitable for real-time fault detection. SqueezeNet~\cite{iandola2016squeezenet} is an efficient network which uses a bottleneck approach to design a small-size network. Nevertheless, there is still a large accuracy gap between these networks and those of full-sized counterparts for detection~\cite{yuxi2018tinydsod}. For traditional SqueezeNet, it is a challenge to increase accuracy while reducing computing cost. In Fig.~\ref{fig:pipeline}, the core of SqueezeNet is Fire module which consists of a squeeze layer and an expand layer. The expand layer contains two layers: 3$\times$3 convolutional layer and 1$\times$1 convolutional layer. However, it is unreliable to preset the kernel numbers of 1$\times$1 and 3$\times$3 convolutional layers in expand layers. As an alternative, we remove the 1$\times$1 convolutional and retain the 3$\times$3 convolutional, and the network is adjusted to be a streamline.
In general, we define a $K_c\times K_c\times P\times Q$ convolutional (Conv.) kernel $K(\cdot)$, where $K_c$ is the spatial dimension of the kernel assumed to be square. \textit{P} is the number of input channels, and \textit{Q} is the number of output channels. The kernel $K(\cdot)$ slides on an input feature map $F(\cdot)$ to extract output features maps $G(\cdot)$ as follows~\cite{sandler2018inverted}:
\begin{equation}
G_{m, n, q}=\sum_{i,j,p}K_{i, j, p, q}\cdot F_{m+i-1, n+j-1, p} \;.
\end{equation}

\begin{table}[!t]
\renewcommand{\arraystretch}{1.03}
\centering
\caption{RFDNet detailed architecture}
\begin{tabular}{l|l|l}
\toprule
Layer name & Type / Stride & Filter shape \\
\midrule
Conv 1                                                   &Conv / s2  &3$\times$3$\times$3$\times$64   \\
\hline
MP 1                                             &MaxPooling / s2  &Pool 3$\times$3    \\
\hline
\multirow{3}{*}{{\begin{tabular}[l]{@{}c@{}} DSF 2 / 3 \end{tabular}}}    &Conv / s1   & 1$\times$1$\times$64$\times$64   \\ \cline{2-3}
           &Dw-Conv / s1 & 3$\times$3$\times$64 dw         \\ \cline{2-3}
           &Conv / s1    & 1$\times$1$\times$64$\times$64  \\
\hline
MP 3                                             &MaxPooling / s2  & Pool 3$\times$3    \\
\hline
\multirow{3}{*}{{\begin{tabular}[l]{@{}c@{}} DSF 4 \end{tabular}}}    &Conv / s1    & 1$\times$1$\times$64$\times$128  \\ \cline{2-3}
           &Dw-Conv / s1 & 3$\times$3$\times$128 dw         \\ \cline{2-3}
           &Conv / s1  & 1$\times$1$\times$128$\times$128   \\
\hline
\multirow{3}{*}{{\begin{tabular}[l]{@{}c@{}} DSF 5 \end{tabular}}}    &Conv / s1    & 1$\times$1$\times$128$\times$128  \\ \cline{2-3}
           &Dw-Conv / s1 & 3$\times$3$\times$128 dw         \\ \cline{2-3}
           &Conv / s1  & 1$\times$1$\times$128$\times$128   \\
\hline
MP 5                                             &MaxPooling / s2  & Pool 3$\times$3    \\
\hline
\multirow{3}{*}{{\begin{tabular}[l]{@{}c@{}} DSF 6 \end{tabular}}} &Conv / s1  & 1$\times$1$\times$128$\times$256      \\ \cline{2-3}
           &Dw-Conv / s1 & 3$\times$3$\times$256 dw         \\ \cline{2-3}
           &Conv / s1  & 1$\times$1$\times$256$\times$256   \\
\hline
\multirow{3}{*}{{\begin{tabular}[l]{@{}c@{}} DSF 7 \end{tabular}}} &Conv / s1  & 1$\times$1$\times$256$\times$256      \\ \cline{2-3}
           &Dw-Conv / s1 & 3$\times$3$\times$256 dw         \\ \cline{2-3}
           &Conv / s1  & 1$\times$1$\times$256$\times$256   \\
\hline
\multirow{3}{*}{{\begin{tabular}[l]{@{}c@{}} DSF 8 \end{tabular}}}  &Conv / s1  & 1$\times$1$\times$256$\times$512     \\ \cline{2-3}
           &Dw-Conv / s1 & 3$\times$3$\times$512 dw         \\ \cline{2-3}
           &Conv / s1  & 1$\times$1$\times$512$\times$512   \\
\hline
\multirow{3}{*}{{\begin{tabular}[l]{@{}c@{}} DSF 9 \end{tabular}}}  &Conv / s1  & 1$\times$1$\times$512$\times$512     \\ \cline{2-3}
           &Dw-Conv / s1 & 3$\times$3$\times$512 dw         \\ \cline{2-3}
           &Conv / s1  & 1$\times$1$\times$512$\times$512   \\
\hline
Conv 10                                                  &Conv / s1  & 1$\times$1$\times$512$\times$1000   \\
\hline
Avgpooling                                              &Average Pooling  / s1  & Pool 14$\times$14  \\
\hline
SoftmaxWithLoss                                         &Softmax / s1  & Classifier  \\
\bottomrule
\end{tabular}
\label{Dsqunetarchitecture}
\end{table}

Moreover, depthwise separable Conv.~\cite{sandler2018inverted} has shown computing efficiency in generic image classification tasks, drastically reducing computational cost and model size. In RFDNet, we use depthwise separable Conv. to improve the performance of Fire module, called as depthwise separable Fire (DSF) module. The proposed DSF contains a depthwise Conv. (Dw-Conv) and a pointwise Conv. layer. We use 3$\times$3 Dw-Conv to replace original expand layers of each Fire module in SqueezeNet. The Dw-Conv~\cite{sandler2018inverted} is defined as:
\begin{equation}
\hat{G}_{m, n, p}=\sum_{i,j} \hat{K}_{i, j, p}\cdot F_{m+i-1, n+j-1, p} \;,
\end{equation}
where $\hat{K}(\cdot)$ is the Dw-Conv kernel of size $K_c\times K_c\times P$, and $\hat{G}(\cdot)$ is the filtered output feature map. Such an approach achieves 8$\times$ less computation than standard Conv.~\cite{sandler2018inverted}. Pointwise Conv., namely a simple 1$\times$1 Conv., is then applied to create a linear combination of the output of depthwise layer. Both batch normalization (BN)~\cite{Ioffe2015Batch} and rectified linear unit (ReLU) nonlinearities are used for all layers in the DSF. The detailed architecture can be found in Table~\ref{Dsqunetarchitecture}. The RFDNet\footnote{For real-time fault detection, we remove the average pooling and the final layer, and only use the Conv.1 layer and DSF2$\sim$9 as the backbone.} begins with a standalone Conv. layer (Conv1) and eight DSF modules (DSF2$\sim$9), followed by a global average pooling, ending with a 1000-d 1$\times$1 Conv. layer.

To verify the advantage of our DSF intuitively, we calculate the average feature maps over all channels from Fire3, Fire5, and Fire7 in SqueezeNet, and the corresponding DSF3, DSF5, and DSF7 in our RFDNet, respectively. The visualization comparisons of extracted feature maps are demonstrated in Fig.~\ref{fig:RFDNet_vis}. The results indicate that DSF produces more salient features, while Fire misses some valuable information. In addition, the feature maps from DSF remain more textural property in the low-level layer. In this case, our DSF can show better fine-grained object details. Furthermore, our proposed DSF can capture more semantic cues in the high-level layer. The effectiveness of our DSF in RFDNet will be further described in the following Section~\ref{analysisframework}.

\begin{figure}[!t]
	\centering
    \includegraphics[width=3.3in]{Figures/Figure3.jpg}
	\caption{Visualization comparison of average feature maps extracted by Fire modules in SqueezeNet (top) and by the corresponding DSF modules in RFDNet (bottom). The average feature maps over all channels from Fire3, Fire5, and Fire7 are shown from left to right, respectively. The average feature maps over all channels from DSF3, DSF5, and DSF7 are shown from left to right, respectively. DSF modules remain more textural property in the low-level layer which present better fine-grained object details. DSF modules also capture more semantic cues in the high-level layer.}
	\label{fig:RFDNet_vis}
\end{figure}

Besides the rich salient features, our proposed RFDNet can generate feature description which is invariant to illumination. Aiming at the main variation in illumination for freight train images, its robustness can be proved by feature maps extracted from different layers in this paper. In Fig.~\ref{fig:conv}, we firstly obtain the images under different illumination intensity. The feature maps are then extracted from different intermediate layers (MP1, MP3 and MP5) of RFDNet. We can observe that feature maps derived from different inputs are similar at the same stage of networks. This means that RFDNet is robust to the change of illumination for freight train images. We attribute this success to abundant input data and self-learning capacity of RFDNet, which can automatically learn to obtain better feature maps. Therefore, the introduction of RFDNet is advisable, because the weather and sunlight would make a great difference in light intensity for freight train images, which is common in practice. %Moreover, some modules in CNNs such as batch normalization (BN)~\cite{Ioffe2015Batch} can enhance generalization ability of network by normalizing samples to a certain distribution range.

\begin{figure}[!t]
 \centering
 \subfloat[]{
 \includegraphics[width=1.05in]{Figures/Figure4a.jpg}}
 \subfloat[]{
 \includegraphics[width=1.05in]{Figures/Figure4b.jpg}}
 \subfloat[]{
 \includegraphics[width=1.05in]{Figures/Figure4c.jpg}}
 \caption{Illustration of RFDNet for freight train images in varying light conditions. (a) Input image in low illumination with average feature maps of MP1, MP3 and MP5 from RFDNet. (b) Normal illumination image with the corresponding average feature maps. (c) High illumination image with the corresponding average feature maps. The feature maps derived from different inputs are similar at the same stage of networks, which means that RFDNet is robust to the change of illumination for fault detection.}
 \label{fig:conv}
\end{figure}

\subsection{Multi-scale Feature Utilization}
Lower-level to higher-level layers in CNNs usually possess diverse distinguishing features for different size of objects. It can be seen from Figs.~\ref{fig:RFDNet_vis} and~\ref{fig:conv} that lower-level layers with higher resolution can capture more fine-grained information, which is helpful for recognizing small objects. Higher-level layers are more sensitive to semantic cues than lower-level layers. Therefore, multi-scale features can better represent all objects by incorporating multiple spatial resolutions in images. For freight train images, the detected parts have a range in size so that a single feature map cannot support for a satisfactory detection performance. So, we apply a multi-scale feature to produce more powerful feature maps of fault region, which can help to detect different size of objects.

\subsubsection{Multi-RPN}
\label{MRPN}
The function of RPN is to quickly select some candidate regions for target objects, which can greatly decrease the computation burden for inference process. A set of rectangular object proposals are usually generated by a fully Conv. network on feature maps. How to build an accurate RPN is important for two-stage detectors, and one potential way to improve its performance is employing multi-scale features.

We propose a novel RPN using a multi-scale feature fusion (MFF) block (see details of MFF\_1 in Fig.~\ref{fig:pipeline}) to apply multi-scale sliding windows over multi-level DSFs, which associates a set of prior anchors with each sliding position to generate fault region proposals. Specially, according to the size of fault regions, we use a 3$\times$3 sliding window who carries 9 anchors with 3 scales and 3 aspect ratios over the MFF\_1 block to produce multiple spatial features. To adjust multi-level feature maps to the same resolution for combination, different DSFs are processed by different sampling strategies. For DSF4, a 2$\times$2 max pooling layer is added to carry out subsampling. Then, we use 192-d 1$\times$1 Conv. to extract local feature over the above processed DSF4, DSF7, and DSF9, respectively. We normalize multiple feature maps using BN and then concatenate them. We encode the above concatenated feature maps using a 512-d 3$\times$3 Conv. layer which not only extracts more semantic features but also compresses them into a uniform space. The 512-d feature is then entered into two output layers: a classification layer that predicts the score of fault region, and a regression layer that refines the location for each prior anchor.
We define a bounding box as $t=(t_x,t_y,t_w,t_h)$ with the score $s$, and our loss function defined on each RoI is the summation of cross-entropy loss $L_{cls}$ and box regression loss $L_{reg}$~\cite{RenHGS15}:
\begin{equation}
\begin{split}
L(s,t_{x,y,w,h})&=L_{cls}(s_{c^*})+\lambda[c^*>0]L_{reg}(t,t^*) \\
&=-log(s_{c^*})+\lambda[c^*>0]L_{reg}(t,t^*),
\end{split}
\end{equation}
where $c^*$ denotes the ground-truth label of a RoI, and $t^*$ is the ground-truth bounding box. $\lambda$ is a balance weight which is set as 1. $[c^*>0]$ is an indicator that equals to 1 if the argument is true and 0 otherwise. Besides that, all local features are pre-computed before multi-RPN and detection without redundant computation~\cite{8911418}. The effectiveness of multi-RPN will be further described in the following Section~\ref{analysisframework}. %In addition, we apply a linear nonmaximum suppression (NMS)~\cite{BodlaSCD17} with a threshold 0.7 to rapidly suppress the lower score boxes and retain the highest scoring anchor in the neighborhood.

\subsubsection{MLPS}
\label{MLPS}
To better use the multi-level features and enrich the different information of each anchor, we perform position-sensitive RoI pooling over MLPS score maps. Before encoding position information into each RoI, we use another MFF block (see details of MFF\_2 in Fig.~\ref{fig:pipeline}) and encode the concatenated feature with a 512-d 1$\times$1 Conv. layer to combine the multi-level features. We then attach a 256-d 1$\times$1 Conv. layer for reducing dimension.
After that, the multi-level weighted fusion feature is accessed to produce $k^2$ position-sensitive score maps for each of the $C$ categories ($k$ is set to 7 in practice~\cite{RenHGS15}), correspondingly all RoIs also are evenly divided into $k^2$ grid areas. The MLPS scores vote on the RoI by averaging the scores, which is MLPS RoI pooling that can be denoted as:
\begin{equation}
P_{c}|\mathcal{A}_{(m,n)}=\sum_{i=1}^{N} \frac{1}{N}p_{(i)}^{(c)}|L_{m,n,c} \quad p\in \mathcal{A}_{(m,n)},
\end{equation}
where the $\mathcal{A}_{(m,n)}$ is an area within $k^2$ grids in each RoI, and it represents the location for specified area (0$\leq$ $m,n$ $\leq$  $k-1$). $P_{c}|\mathcal{A}_{(m,n)}$ is the pooling result for category $C$ at $\mathcal{A}_{(m,n)}$, and $p$ is the pixel in $\mathcal{A}_{(m,n)}$. $N$ denotes total number of pixels in $\mathcal{A}_{(m,n)}$, while $L_{m,n,c}$ is one of score maps that corresponds to $\mathcal{A}_{(m,n)}$ in $k^2$ score maps for $C$.

Finally, a ($C+1$)-d vector is produced for classification, and an average vote is used over the vector as follows
\begin{equation}
P_{c}= \sum _{m,n} P_{c}|\mathcal{A}_{(m,n)},
\end{equation}
where $P_{c}$ is the final score for category $C$, and we then calculate the softmax responses across categories:
\begin{equation}
s_c=\frac{e^{P_{c}}}{\sum_{c'=0}^{C} e^{P_{c'}}}.
\end{equation}
These are used for computing the cross-entropy loss $L_{cls}$ during training and for ranking the RoIs during inference.

Aiming at achieving bounding box regression, a sibling 4$k^2$-d Conv. layer is then appended for bounding box regression. The MLPS RoI pooling is performed on this bank of 4$k^2$ maps as well. Then, it is aggregated into a 4-d vector by average voting which is used to parameterize a bounding box. There is no learnable layer after the RoI, enabling nearly cost-free region-wise computation and speeding up both training and inference~\cite{RenHGS15}. The visualization results of the multi-level feature concatenation are demonstrated in Fig.~\ref{fig:vis_MLPS}. The average feature maps of DSF9 are extremely scarce for different freight train images, which only contain semantic cues with low resolution. The MLPS score maps and RoI Pooling will be unreliable to detect faults only based on the feature maps processed by DSF9. Nevertheless, the multi-level fusion feature has rich object characteristic such as shape and contour, which is helpful to improve the detection accuracy. The applicability of MLPS score maps and RoI Pooling will be further described in the following Section~\ref{analysisframework}.

\begin{figure}[!t]
	\centering
    \subfloat[]{\includegraphics[width=1.05in]{Figures/Figure5a.jpg}}\hspace{0.1em}
    \subfloat[]{\includegraphics[width=1.05in]{Figures/Figure5b.jpg}}\hspace{0.1em}
    \subfloat[]{\includegraphics[width=1.05in]{Figures/Figure5c.jpg}}
	\caption{Visualization results of the multi-level feature concatenation. (a) Input images; (b) Average feature maps extracted by DSF9; (c) Average feature maps concatenated by DSF4, DSF7, and DSF9. The multi-level fusion feature has richer object characteristic than a single, such as shape and contour.}
	\label{fig:vis_MLPS}
\end{figure} 
\section{Experiments and Analysis}
\label{experiment}
In this section, we evaluate the effectiveness of our framework on the problem of real-time fault detection for freight train images. To this end, we firstly evaluate our proposed light-weight backbone on three datasets, ImageNet ILSVRC 2012~\cite{SimonyanZ14a}, PASCAL visual object classes (VOC) 2007~\cite{EveringhamEGWWZ15} and MS COCO~\cite{2014Microsoft}. Then we compare the proposed framework with state-of-the-art fault detectors and well-known object detection methods on six fault datasets~\cite{zhang2018,8911418}. We conduct all of our experiments using Caffe~\cite{JiaSDKLGGD14} on a single NVIDIA GeForce GTX1080Ti GPU.
%\begin{table}[!t]
%\renewcommand{\arraystretch}{1.1}
%\centering
%\caption{The datasets of fault detection for freight train images}
%\begin{tabular}{lcccc}
%\toprule
%\multirow{2}{*}{Datasets}  & \multirow{2}{*}{Training set} & \multicolumn{3}{c}{Testing set}
%\\ \cline{3-5}
%                    &              & Non-fault   & Fault & Total \\
%\midrule
%Angle cock          & 2002         & 1049        & 975   & 2024  \\
%Bogie block key     & 5440         & 2530        & 367   & 2897  \\
%Brake shoe key      & 5636         & 2000        & 2000  & 4000  \\
%Cut-out cock        & 815          & 671         & 179   & 850   \\
%Dust collector      & 815          & 798         & 52    & 850   \\
%Fastening bolt      & 1724         & 445         & 1257  & 1902  \\
%\bottomrule
%\end{tabular}
%\label{database}
%\end{table}

\subsection{Experimental Setup}
\subsubsection{Implementation Details}
In ImageNet experiments, to make a fair comparison, all the hyper-parameters follow SqueezeNet~\cite{iandola2016squeezenet}. We use BN after each Conv. layer before ReLU activation, and the initial learning rate is set to 0.04. We use the polynomial decay learning rate scheduling strategy in the batch size of 32. The momentum and weight decay are set as 0.9 and 0.0002, respectively. Finally, we use the validation set of ImageNet ILSVRC 2012 to validate our backbone.

In PASCAL VOC experiments, we use the same hyper-parameters as SqueezeNet to make a fair comparison. Based on a pre-trained model from the ImageNet experiments, we fine-tune the resulting model using RMSProp with 0.0001 initial learning rate, 0.9 momentum, and 0.0005 weight decay. We set 120K training steps and execute multi-scale training in the batch size of 64. We use the step decay learning rate scheduling strategy and multiply with a factor 0.1 at the 20K, 50K, and 100K steps, respectively. Finally, the VOC 2007 test set is used to verify our RFDNet following the protocol in~\cite{RenHGS15}.

In MS COCO experiments, we also use the same hyper-parameters as SqueezeNet for fair comparison. Based on a pre-trained model from the ImageNet experiments, we fine-tune the resulting model using SGD with 0.001 initial learning rate, 0.9 momentum, and 0.0005 weight decay. We set 480K training steps and execute multi-scale training in the batch size of 56. We use the step decay learning rate scheduling strategy and multiply with a factor 0.1 at the 280K, and 360K steps, respectively. Finally, the COCO minival set is adopted to evaluate our backbone following the standard protocol.

In fault detection experiments, our method is trained via back-propagation and stochastic gradient descent (SGD). We use a basic learning rate of 0.001 and it is divided by 10 for each 40K mini-batch until convergence. The batch sizes of  multi-RPN and MLPS RoI are 256 and 512, respectively. A pre-trained RFDNet model for ImageNet is first used to initialize shared Conv. layers of our backbone network, and then the new layers are initialized with a zero mean and a standard deviation of 0.01 Gaussian distribution. We train the network with 70K iterations in total. The momentum and weight decay are set as 0.9 and 0.0005, respectively. The confidence score in the detecting stage is 0.9.

\subsubsection{Fault Datasets} To evaluate the performance of our method, six fault datasets~\cite{zhang2018,8911418} for freight train images are directly used in this study, including angle cock, bogie block key, brake shoe key, cut-out cock, dust collector, and fastening bolt on brake beam. Some typical samples of freight train images are shown in Fig~\ref{fig:detection}(b).
\begin{itemize}
  \item \textbf{Angle cock} is a key component of the air brake system of freight trains, and its role is to ensure the smooth flow of air in the main pipeline. For this dataset, training and evaluation are performed on the 2002 images in the trainval and the 2024 images in the test, respectively.
  \item \textbf{Bogie block key} is a very small part used to prevent the wheel set from getting out of the bogie. This dataset is divided into two sets, training and testing with 5440 and 2897 images, respectively.
  \item \textbf{Brake shoe key} is also a small component equipped in brake shoe, which is vital for safe operation of braking system. The dataset provides more than 5600 images for training, and 4000 images for its test set.
  \item \textbf{Cut-out cock} is a key part that cuts off the air from main reservoir to the brake pipe, which is used to shut down the brake pipe. The images are divided into a train set of 815 images and a test set of 850 images.
  \item \textbf{Dust collector} is usually installed next to the cut-out cock and its role is to filter impurities towards compressed air. So, the images in this dataset are annotated directly on the images in cut-out cock dataset.
  \item\textbf{Fastening bolt} is an important part for train brake. When the train brakes, the fastening bolts may break or fall off because of a large horizontal force generated from brake beam. There are 1724 images in the train set and another 1902 images in the test set.
\end{itemize}

\begin{table}[!t]
\renewcommand{\arraystretch}{1}
\centering
\caption{Classification results on ImageNet ILSVRC 2012}
\begin{tabular}{lcccc}
\toprule
 \multirow{2}{*}{{\begin{tabular}[l]{@{}c@{}} Model \end{tabular}}}  & \multirow{2}{*}{{\begin{tabular}[c]{@{}c@{}} Computational \\ cost (FLOPs) \end{tabular}}}
& \multirow{2}{*}{{\begin{tabular}[c]{@{}c@{}} Model size\\(Parameters)\end{tabular}}} & \multirow{2}{*}{Top-1} & \multirow{2}{*}{Top-5}\\
&&&&\\
\midrule
SqueezeNet                   & 833M  & 4.8MB  & 57.5\% & 80.3\% \\
RFDNet                       & 580M  & 7.1MB  & 64.4\% & 85.8\% \\
\bottomrule
\end{tabular}
\label{imagenet}
\end{table}

\begin{table}[!t]
\renewcommand{\arraystretch}{1}
\centering
\caption{Detection results on PASCAL VOC 2007 dataset. The “07+12” means VOC07 trainval union with VOC12 trainval}
\begin{tabular}{lcccc}
\toprule
\multirow{2}{*}{{\begin{tabular}[l]{@{}c@{}} Model \end{tabular}}}  & \multirow{2}{*}{{\begin{tabular}[l]{@{}c@{}} Training \\ data \end{tabular}}}  & \multirow{2}{*}{{\begin{tabular}[c]{@{}c@{}} Input \\dimension \end{tabular}}}
& \multirow{2}{*}{{\begin{tabular}[c]{@{}c@{}} Model size\\(Parameters)\end{tabular}}} & \multirow{2}{*}{{\begin{tabular}[l]{@{}c@{}} mAP \end{tabular}}} \\
&&&& \\
\midrule
SqueezeNet-SSD  & 07+12  & 300$\times$300  & 21.1MB  & 64.3 \\
RFDNet-SSD      & 07+12  & 300$\times$300  & 17.2MB  & 70.1 \\
\bottomrule
\end{tabular}
\label{voc2007}
\end{table}

\begin{table}[!t]
\renewcommand{\arraystretch}{1}
\centering
\caption{Detection results on MS COCO dataset}
\setlength{\tabcolsep}{1.8mm}{
\begin{tabular}{lccccc}
\toprule
\multirow{2}{*}{{\begin{tabular}[l]{@{}c@{}} Model \end{tabular}}}     & \multirow{2}{*}{{\begin{tabular}[c]{@{}c@{}} Input \\dimension \end{tabular}}}
& \multirow{2}{*}{{\begin{tabular}[c]{@{}c@{}} Model size\\(Parameters)\end{tabular}}} & \multicolumn{3}{c}{Avg. Precision, IoU:} \\
&&&0.5:0.95&0.5&0.75 \\
\midrule
SqueezeNet-SSD    & 300$\times$300  & 55.4MB  & 8.4  & 15.2 & 8.2\\
RFDNet-SSD        & 300$\times$300  & 40.1MB  & 11.7 & 19.7 & 12.1\\
\bottomrule
\end{tabular}}
\label{coco2015}
\end{table}

\begin{table}[!t]
	\renewcommand{\arraystretch}{1}
	\centering
	\caption{Detection results of different DSFs on six datasets}
	\begin{tabular}{lcccc}
		\toprule
		Modules  & Width & mCDR/\%$\uparrow$      & mMDR/\%$\downarrow$     & mFDR/\%$\downarrow$ \\
		\midrule
		DSF9      &512$\times$1   & 98.09  & 1.26   & 0.65  \\
		DSF(4,9)    &256$\times$2   & 98.39  & 0.92   & 0.69  \\
		DSF(5,9)    &256$\times$2   & 94.13  & 4.27   & 1.60  \\
		DSF(6,9)    &256$\times$2   & 96.64  & 2.88   & 0.48  \\
		DSF(7,9)    &256$\times$2   & 97.90  & 1.64   & 1.46  \\
		DSF(8,9)    &256$\times$2   & 98.37  & 1.40   & 0.23  \\
		DSF(4,6,9)  &192$\times$3   & 97.99  & 1.64   & 0.37  \\
		DSF(4,7,9)  &192$\times$3   & 98.60  & 0.94   & 0.46  \\
		DSF(4,8,9)  &192$\times$3   & 98.51  & 0.86   & 0.63  \\
		DSF(5,6,9)  &192$\times$3   & 89.90  & 5.22   & 4.88  \\
		DSF(5,7,9)  &192$\times$3   & 95.15  & 3.44   & 1.41  \\
		\bottomrule
	\end{tabular}
	\label{diffmod}
\end{table}

\begin{table*}[!t]
	\renewcommand{\arraystretch}{1}
	\centering
	\caption{Detection results of connecting different modules, including SqueezeNet, RFDNet, MRPN, and MLPS}
	\setlength{\tabcolsep}{2.9mm}{
		\begin{tabular}{cccccccccccc}
			\toprule
			\multirow{2}{*}{SqueezeNet} & \multirow{2}{*}{RFDNet} & \multirow{2}{*}{MRPN} & \multirow{2}{*}{MLPS}    & \multirow{2}{*}{mCDR/\%$\uparrow$}  & \multirow{2}{*}{mMDR/\%$\downarrow$} & \multirow{2}{*}{mFDR/\%$\downarrow$} &
			\multirow{2}{*}{{\begin{tabular}[c]{@{}c@{}} Training \\ speed/s\end{tabular}}}   &
			\multirow{2}{*}{{\begin{tabular}[c]{@{}c@{}} Testing \\ speed/s\end{tabular}}}   &
			\multirow{2}{*}{{\begin{tabular}[c]{@{}c@{}} Memory \\ usage/MB\end{tabular}}}  &
			\multirow{2}{*}{{\begin{tabular}[c]{@{}c@{}} Model \\size/MB\end{tabular}}} \\
			& & & & & & & & & &  \\
			\midrule
			$\surd$ &--      &--      &--       & 97.12 & 1.15 & 1.73  & 0.085 & 0.026 & 745 & 20.7  \\
			--      &$\surd$ &--      &--       & 98.09 & 1.26 & 0.65  & 0.105 & 0.024 & 683 & 13.8  \\
			$\surd$ &--      &$\surd$ &--       & 97.90 & 0.91 & 1.19  & 0.114 & 0.027 & 770 & 21.6  \\
			--      &$\surd$ &$\surd$ &--       & 98.45 & 1.14 & 0.41  & 0.126 & 0.025 & 698 & 17.4  \\
			$\surd$ &--      &$\surd$ &$\surd$  & 98.36 & 0.70 & 0.94  & 0.118 & 0.028 & 795 & 25.1  \\
			--      &$\surd$ &$\surd$ &$\surd$  & 98.60 & 0.94 & 0.46  & 0.135 & 0.026 & 713 & 19.6  \\
			\bottomrule
	\end{tabular}}
	\label{modules}
\end{table*}


\begin{table*}[!t]
	\renewcommand{\arraystretch}{1}
	\caption{Detection results of six typical faults in comparison with state-of-the-art methods}
	\centering
	\setlength{\tabcolsep}{3.5mm}{
		%\resizebox{\textwidth}{!}{
		\begin{tabular}{lccccccccccccccc}
			\toprule
			\multirow{2}{*}{Methods}    & \multirow{2}{*}{mCDR/\%$\uparrow$}  & \multirow{2}{*}{mMDR/\%$\downarrow$} & \multirow{2}{*}{mFDR/\%$\downarrow$} &
			\multirow{2}{*}{{\begin{tabular}[c]{@{}c@{}} Training \\ speed/s\end{tabular}}}   &
			\multirow{2}{*}{{\begin{tabular}[c]{@{}c@{}} Testing \\ speed/s\end{tabular}}}   &
			\multirow{2}{*}{{\begin{tabular}[c]{@{}c@{}} Batch \\ size\end{tabular}}} &
            \multirow{2}{*}{{\begin{tabular}[c]{@{}c@{}} Model\\size/MB\end{tabular}}}&
            \multicolumn{2}{c}{\multirow{2}{*}{{\begin{tabular}[c]{@{}c@{}} Memory \\ usage/MB\end{tabular}}}} \\
			& & & & & & & &\\
			\midrule
			Cascade detector(LBP)       & 87.55   & 6.33   & 6.12   & --    & 0.048 & --  & 0.12  & \multicolumn{2}{c}{--}   \\
			HOG+Adaboost+SVM            & 93.32   & 3.25   & 3.43   & --    & 0.049 & --  & 0.11  & \multicolumn{2}{c}{--}   \\
			FAMRF+EHF                   & 94.96   & 1.00   & 4.04   & --    & 0.725 & --  & --    & \multicolumn{2}{c}{--}   \\
			\midrule
			SSD(VGG16)                  & 96.32   & 0.88   & 2.80   & 0.747 & 0.047 & 16  & 95.5  & \multicolumn{2}{c}{1173} \\
			YOLOv3                      & 88.85   & 2.58   & 8.57   & 3.537 & 0.026 & 64  & 246.3 & \multicolumn{2}{c}{1501} \\
			RefineDet(VGG16)            & 96.06   & 0.74   & 3.20   & 1.742 & 0.056 & 16  & 135.8 & \multicolumn{2}{c}{1415} \\
			RON(VGG16)                  & 98.15   & 0.47   & 1.38   & 0.892 & 0.029 & 32  & 157.9 & \multicolumn{2}{c}{1143} \\
			DSOD(DenseNet)              & 95.62   & 2.13   & 2.25   & 0.517 & 0.109 & 2   & 50.8  & \multicolumn{2}{c}{4429} \\
			\midrule
			MLKP(VGG16)                 & 98.21   & 0.68   & 1.11   & 0.722 & 0.147 & 128 & 596.1 & \multicolumn{2}{c}{3711} \\
			Faster R-CNN(VGG16)         & 98.19   & 0.96   & 0.85   & 0.289 & 0.065 & 128 & 546.8 & \multicolumn{2}{c}{1817} \\
			%CoupleNet(ResNet101)        & 97.51   & 0.22   & 2.27   & 0.851 & 0.112 & 128 & 409.6 & \multicolumn{2}{c}{3443} \\
			R-FCN(ResNet101)            & 94.68   & 1.71   & 3.61   & 0.524 & 0.096 & 128 & 199.9 & \multicolumn{2}{c}{3114} \\
			FTI-FDet(VGG16)             & 99.41   & 0.37   & 0.22   & 0.336 & 0.071 & 128 & 557.3 & \multicolumn{2}{c}{1823} \\
			Light FTI-FDet(VGG16)       & 99.22   & 0.32   & 0.46   & 0.318 & 0.058 & 128 & 89.7  & \multicolumn{2}{c}{1533} \\
            Cascade R-CNN(ResNet101)    & 97.34   & 0.96   & 1.70   & 0.615 & 0.203 & 2   & 220.8 & \multicolumn{2}{c}{3818} \\
			\midrule
			MobileNetV2-SSD             & 97.97   & 0.58   & 1.45   & 0.561 & 0.034 & 8   & 15.2  & \multicolumn{2}{c}{1343} \\
			MobileNetV2-SSDLite         & 94.65   & 0.29   & 5.06   & 0.101 & 0.018 & 16  & 12.3  & \multicolumn{2}{c}{827}  \\
			ShuffleNetV2-SSD            & 96.24   & 0.51   & 3.25   & 0.254 & 0.028 & 16  & 11.8  & \multicolumn{2}{c}{850}  \\
			Tiny-DSOD                   & 95.74   & 0.31   & 3.95   & 0.467 & 0.057 & 4   & 3.5   & \multicolumn{2}{c}{1469} \\
			Pelee(PeleeNet)             & 96.34   & 0.92   & 2.74   & 0.757 & 0.051 & 16  & 20.2  & \multicolumn{2}{c}{1412} \\
			\midrule
            Light FTI-FDet(RFDNet)      & 98.17   & 0.94   & 0.89   & 0.178 & 0.034 & 128 & 27.8  & \multicolumn{2}{c}{857}  \\
			RFDNet-SSD                  & 97.98   & 0.32   & 1.70   & 0.809 & 0.036 & 24  & 11.4  & \multicolumn{2}{c}{905}  \\
			LR FTI-FDet(RFDNet)         & 98.60   & 0.94   & 0.46   & 0.135 & 0.026 & 256 & 19.6  & \multicolumn{2}{c}{713}  \\
			\bottomrule
	\end{tabular}}
	\label{meanaccuray}
\end{table*}


\subsubsection{Evaluation Metrics} There are seven indexes: correct detection rate (CDR), missing detection rate (MDR), false detection rate (FDR), training speed, testing speed, test memory usage and model size (parameters) to evaluate the effectiveness of fault detectors. The indexes of CDR, MDR, and FDR are all used to measure the accuracy of detectors, which are calculated based on the method directly from~\cite{8911418}. For example, there is a test set which contains $m$ fault images and $n$ normal (non-fault) images, through the work of the detector, $a$ images are detected as fault, among them $b$ images are detected by error, meanwhile, $c$ images are detected as normal, among them $d$ images are detected by error. In this case, the indexes will be defined as:
\begin{equation}
  CDR = \frac{a+c}{m+n},\ MDR = \frac{b}{m+n},\ FDR = \frac{d}{m+n}.
\end{equation}
The mean value of CDRs, MDRs, and FDRs are calculated as mCDR, mMDR, and mFDR respectively to indicate the accuracy of fault detection for different datasets. Both model size and accuracy report the impact of CNN architectural designs~\cite{iandola2016squeezenet} on fault detectors. Both memory usage and training/testing speed reflect the dependence of detectors on hardware. Especially, we use the computational time for each iteration in training and testing phase for each image as training and testing speed, respectively. Memory usage is collected from a detector's memory usage on a single GPU in the testing phase.

\subsection{Performance Analysis}
\label{analysisframework}

\subsubsection{Backbone}
To verify the effectiveness of our RFDNet, we give a detailed discussion on the performance of RFDNet in comparison with the baseline light-weight network SqueezeNet. It can be seen from Table~\ref{imagenet} that RFDNet achieves a baseline of 64.4\% top-1 and 85.8\% top-5 accuracy on ImageNet, which is 6.9\% and 5.5\% higher than SqueezeNet with 1.4$\times$ less computation at the same size. The proposed RFDNet can also be deployed as an effective base network in object detection. We then perform experiments on VOC 2007 and MS COCO for detailed analysis of our RFDNet based on the SSD. Detection accuracy is measured by mean Average Precision (mAP) with 300 input resolutions. The experimental results on VOC2007 test set are summarized in Table~\ref{voc2007}. Our RFDNet achieves 70.1\% mAP, and its accuracy is higher than that of SqueezeNet by 5.8\% at only 81.2\% of model size. Moreover, the results on COCO minival set are summarized in Table~\ref{coco2015}. Our proposed RFDNet achieves 19.7\%/12.1\% with 0.5/0.75 IoU, which outperforms the SqueezeNet with a large margin. We observe that our [0.5:0.95] result is 3.3\% higher than the SqueezeNet at 72.4\% of model size. This indicates that our predicted locations are more accurate than the SqueezeNet with lower computational cost.

\subsubsection{Multi-scale Feature Utilization}
An important property of our method is that it combines coarse-to-fine information across deep CNN models. As an example, we compare different Conv. feature maps on six datasets to illustrate the superiority of the proposed multi-scale feature utilization (MFF\_1 and MFF\_2). Table~\ref{diffmod} shows the detection performance for connecting different DSF modules. “DSF(4,9)" means connecting DSF4 and DSF9 in both multi-RPN and MLPS score maps. “192$\times$3" means that we apply the 192-d 1$\times$1 Conv. layer on each of three DSF modules, respectively. In Table~\ref{diffmod}, the combination of DSF4, DSF7, and DSF9 works the best. The results indicate that the multi-layer combination performs roughly better than a single layer, and further verify the effectiveness of low-to-high combination strategy.

\subsubsection{Different Modules}
We analyze RFDNet, multi-RPN, and MLPS score maps by conducting experiments on six datasets. With the aforementioned computer, we only change the configuration of modules for a fair comparison. In Table~\ref{modules}, our RFDNet has higher accuracy and less computation than SqueezeNet. The combination of three modules in our framework can achieve the best performance. The index of mCDR significantly improves from 97.12\% to 98.60\%, and the testing speed is 0.026s. The results reveal that both MRPN and MLPS score maps can improve detection performance with few redundant computations. These two modules are able to learn more effective and comprehensive features than a single DSF for distinguishing faults from complex backgrounds.

\subsection{Comparison with State-of-the-art Methods}
%To illustrate the superiority of our method, we compare our framework called as Light-weight Real-time FTI-FDet (LR FTI-FDet) with traditional detectors (Cascade detector with local binary pattern (LBP)~\cite{Sun2018Railway}, FAMRF + EHF~\cite{Sun2018Railway}, histogram of oriented gradient (HOG) + Adaboost + SVM\footnote{https://github.com/pdollar/toolbox}~\cite{DollarABP14}), one-stage detectors (YOLOv3\footnote{https://pjreddie.com/darknet/}~\cite{redmon2018yolov3}, SSD\footnote{https://github.com/weiliu89/caffe/tree/ssd}~\cite{LiuAESRFB16}, RefineDet\footnote{https://github.com/sfzhang15/RefineDet}~\cite{zhang2018single}, RON\footnote{https://github.com/taokong/RON}~\cite{kong2017ron}, DSOD\footnote{https://github.com/szq0214/DSOD}~\cite{shen2017dsod}), two-stage detectors (Faster R-CNN\footnote{https://github.com/rbgirshick/py-faster-rcnn}~\cite{RenHGS15}, MLKP\footnote{https://github.com/Hwang64/MLKP}~\cite{wang2018multi}, R-FCN\footnote{https://github.com/daijifeng001/R-FCN}~\cite{DaiLHS16}, Cascade R-CNN\footnote{https://github.com/zhaoweicai/cascade-rcnn}~\cite{cai2018cascade}, FTI-FDet~\cite{zhang2018}, Light FTI-FDet\footnote{https://github.com/Yangzhangcst/Light-FTI-FDet}~\cite{8911418}), and light-weight detectors (MobileNetV2-SSD~\cite{sandler2018inverted}, MobileNetV2-SSDLite\footnote{https://github.com/chuanqi305/MobileNetv2-SSDLite}~\cite{sandler2018inverted}, ShuffleNetV2-SSD~\cite{ma2018shufflenet}, Tiny-DSOD\footnote{https://github.com/lyxok1/Tiny-DSOD}~\cite{yuxi2018tinydsod}, Pelee\footnote{https://github.com/Robert-JunWang/Pelee}~\cite{wang2018pelee}).
To illustrate the superiority of our method, we compare our framework called as Light-weight Real-time FTI-FDet (LR FTI-FDet) with traditional detectors (Cascade detector with local binary pattern (LBP)~\cite{Sun2018Railway}, FAMRF + EHF~\cite{Sun2018Railway}, histogram of oriented gradient (HOG) + Adaboost + SVM~\cite{DollarABP14}), one-stage detectors (YOLOv3~\cite{redmon2018yolov3}, SSD~\cite{LiuAESRFB16}, RefineDet~\cite{zhang2018single}, RON~\cite{kong2017ron}, DSOD), two-stage detectors (Faster R-CNN~\cite{RenHGS15}, MLKP~\cite{wang2018multi}, R-FCN~\cite{DaiLHS16}, Cascade R-CNN~\cite{cai2018cascade}, FTI-FDet~\cite{zhang2018}, Light FTI-FDet~\cite{8911418}), and light-weight detectors (MobileNetV2-SSD~\cite{sandler2018inverted}, MobileNetV2-SSDLite~\cite{sandler2018inverted}, ShuffleNetV2-SSD~\cite{ma2018shufflenet}, Tiny-DSOD~\cite{yuxi2018tinydsod}, Pelee~\cite{wang2018pelee}).  In addition, we compare RFDNet-SSD with all above methods to discuss the performance of RFDNet and depthwise separable Conv.-based networks (\textit{e.g.} MobileNetV2) on fault detection. Specially, the related parameters in each detector are tuned to the best performance. %CoupleNet\footnote{https://github.com/tshizys/CoupleNet}~\cite{zhu2017couplenet},

\textbf{Accuracy and model size.}
As shown in Table~\ref{meanaccuray}, LR FTI-FDet achieves 98.60\% mCDR which outperforms RFDNet-SSD, all traditional methods, one-stage, light-weight and most two-stage detectors. The accuracy of both FTI-FDet and Light FTI-FDet are slightly higher than our method, but their model size is too large. Although the model size of each traditional methods is the smallest, but their accuracy is the lowest. The model sizes of our RFDNet-SSD and LR FTI-FDet are 11.4MB and 19.6 MB respectively, which is comparable to light-weight detectors and far less than all one- and two-stage detectors. After replacing backbone (VGG16) with RFDNet in Light FTI-FDet, our method achieves 0.43\% higher mCDR with 1.4$\times$ smaller than the Light FTI-FDet. Especially, the model size of LR FTI-FDet is 28.4/4.6$\times$ smaller than FTI-FDet/Light FTI-FDet with VGG16.
However, our method is unsatisfactory for the robustness of noise and the disturbance from other similar structures without faults. The comparisons between ground-truths and failure examples obtained by our method are shown in Fig.~\ref{QFresults}. We will solve it by expanding the datasets through adding more samples and performing data augmentation in the future. These operations will also improve the generalization ability of our method.

\textbf{Computational cost and speed.}
In Table~\ref{meanaccuray}, both training and testing speeds of our method are faster than traditional methods, one- and two-stage detectors. The testing speed ($>$38 fps) of our LR FTI-FDet is the same as YOLOv3 and 2.7/2.2$\times$ faster than FTI-FDet/Light FTI-FDet with 2.6/2.2$\times$ less memory usage. Our RFDNet-SSD has a comparable performance with MobileNetV2-SSD on fault detection while our method has smaller model size and memory usage. The speed of MobileNetV2-SSDLite is slight faster than our method, but its memory usage is higher. The main reason is that our LR FTI-FDet is a two-stage detector containing RPN and position-sensitive RoI pooling, which needs more computations than a one-stage detector MobileNetV2-SSDLite. But there are efficient DSF modules in RFDNet and many shared layers among RFDNet, multi-RPN, and MLPS score maps, so that the memory usage of our LR FTI-FDet is smaller, which merely needs 713 MB.

The experimental results confirm that our method achieves a much better trade-off between resources and accuracy than the state-of-the-art methods. The experiments on six typical fault datasets also indicate that our method is robust to the illumination variation with high versatility. Therefore, our method is the most suitable for real-time fault detection of freight train images, even though under strict memory and computational budget constraints.

%\begin{figure*}[!t]
%  \centering
%  \includegraphics[width=6.9in]{Figures/Figure4.jpg}
%  \caption{Qualitative results of our method.}
%  \label{QCresults}
%\end{figure*}

\begin{figure}[!t]
  \centering
  \includegraphics[width=3.3in]{Figures/Figure6.jpg}
  \caption{Visualization of ground-truths (top) and failure examples (bottom) obtained by our method. Green bounding boxes mean normal parts, and red bounding boxes are fault areas. Our method is unsatisfactory for the robustness of noise and the disturbance from other similar structures without faults.}
  \label{QFresults}
\end{figure}

\section{Conclusion and future work}
\label{conclusion}
In this paper, we present a light-weight framework LR FTI-FDet in an end-to-end manner for real-time fault detection of freight train images in the wild. The proposed framework consists of a multi-RPN over RFDNet for fault proposal generation and MLPS score maps for fault proposal detection. Experiments show that the Top-1 accuracy of our RFDNet is 6.9\% higher than SqueezeNet with 1.4$\times$ less computation on ImageNet. Our RFDNet achieves 5.8\% mAP higher than SqueezeNet on VOC 2007, and our [0.5:0.95] result is 3.3\% higher than SqueezeNet on MS COCO. The detection results on six fault datasets indicate that our method is much faster during both training and testing as the light-weight detectors. Our method achieves competitive accuracy, 28.4/4.6$\times$ smaller model size and 2.6/2.2$\times$ less memory usage than FTI-FDet/Light FTI-FDet. The proposed LR FTI-FDet has lower resource requirements with the same testing speed as YOLOv3 up to 38 fps, 2.7/2.2$\times$ faster than FTI-FDet/Light FTI-FDet.

In the future, we plan to apply our method on embedded platforms (Raspberry Pi and Jetson Nano) to achieve real-time multi-fault detection in the wild, and further enhance accuracy and detection speed. 

\bibliographystyle{IEEEtran}
\bibliography{References}

% use section* for acknowledgment
%\section*{Acknowledgment}

% Can use something like this to put references on a page
% by themselves when using endfloat and the captionsoff option.
\ifCLASSOPTIONcaptionsoff
  \newpage
\fi

%
% that's all folks
\end{document}


