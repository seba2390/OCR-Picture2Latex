\documentclass[10pt,conference,letterpaper]{IEEEtran}

\usepackage{booktabs} % For formal tables
\usepackage{amsmath}
\usepackage{subfigure}
\usepackage{times}
\let\labelindent\relax
\usepackage[shortlabels]{enumitem}
%\usepackage{algorithm}
\usepackage[ruled, vlined, linesnumbered]{algorithm2e}
\usepackage{algorithmicx}
\usepackage{algpseudocode}
\usepackage{multirow}
%\usepackage{flushend}
\usepackage[pdftex]{graphicx}
\usepackage{url}
\usepackage{color}

\newtheorem{definition}{Definition}
%\newtheorem{\algorithm}{Procedure}
\newtheorem{assumption}{Assumption}
\newtheorem{theorem}{Theorem}
%\renewcommand{\algorithmicrequire}{\textbf{Input:}}
%\renewcommand{\algorithmicensure}{\textbf{Output:}}

\newcommand{\figcaptionspace}{0pt}

\title{Enabling Quality Control for Entity Resolution: A Human and Machine Cooperation Framework \vspace{-0.4cm}}

\author{
{
\resizebox{\linewidth}{!}{
\mbox{Zhaoqiang Chen{\small $~^{\#1}$}, Qun Chen{\small $~^{\#2}$}, Fengfeng Fan{\small $~^{\#3}$}, Yanyan Wang{\small $~^{\#4}$}, Zhuo Wang{\small $~^{\#5}$}, Youcef Nafa{\small $~^{\#6}$}, Zhanhuai Li{\small $~^{\#7}$}, Hailong Liu{\small $~^{\#8}$}, Wei Pan{\small $~^{\#9}$}
}}
}%
\vspace{1.6mm}\\
\fontsize{10}{10}\selectfont\itshape
$^{\#}$\,School of Computer Science, Northwestern Polytechnical University\\
$^{\#}$\,Key Laboratory of Big Data Storage and Management, Northwestern Polytechnical University, Ministry \\ of Industry and Information Technology\\
1 Dongxiang Road, Xi'an Shaanxi, China\\
\fontsize{9}{9}\selectfont\ttfamily\upshape
\{$^{2}$\,chenbenben, $^{7}$\,lizhh, $^{8}$\,liuhailong, $^{9}$\,panwei1002 \}@nwpu.edu.cn\\
\{$^{1}$\,chenzhaoqiang, $^{3}$\,fanfengfeng, $^{4}$\,wangyanyan, $^{5}$\,wzhuo918, $^{6}$\,youcef.nafa \}@mail.nwpu.edu.cn\\
}
\begin{document}

\maketitle
% The default list of authors is too long for headers}
%\renewcommand{\shortauthors}{B. Trovato et al.}
\begin{abstract}
Even though many machine algorithms have been proposed for entity resolution, it remains very challenging to find a solution with quality guarantees. In this paper, we propose a novel HUman and Machine cOoperation (HUMO) framework for entity resolution (ER), which divides an ER workload between the machine and the human. HUMO enables a mechanism for quality control that can flexibly enforce both precision and recall levels. We introduce the optimization problem of HUMO, minimizing human cost given a quality requirement, and then present three optimization approaches: a conservative baseline one purely based on the monotonicity assumption of precision, a more aggressive one based on sampling and a hybrid one that can take advantage of the strengths of both previous approaches. Finally, we demonstrate by extensive experiments on real and synthetic datasets that HUMO can achieve high-quality results with reasonable return on investment (ROI) in terms of human cost, and it performs considerably better than the state-of-the-art alternatives in quality control.
\end{abstract}

%
% The code below should be generated by the tool at
% http://dl.acm.org/ccs.cfm
% Please copy and paste the code instead of the example below.
%

%\begin{CCSXML}
%<ccs2012>
% <concept>
%  <concept_id>10002951.10002952.10003219.10003223</concept_id>
%  <concept_desc>Information systems~Entity resolution</concept_desc>
%  <concept_significance>500</concept_significance>
% </concept>
%</ccs2012>
%\end{CCSXML}

%\ccsdesc[500]{Information systems~Entity resolution}

% We no longer use \terms command
%\terms{Theory}

%\keywords{Entity Resolution, Quality Guarantees, Human-and-Machine Cooperation}


%\maketitle

Neural networks are powerful models that excel at a wide range of tasks.
However, they are notoriously difficult to interpret and extracting explanations 
    for their predictions is an open research problem. Linear models, in contrast, are generally considered interpretable, because
    the \emph{contribution} 
    (`the weighted input') of every dimension to the output is explicitly given.
Interestingly, many modern neural networks implicitly model the output as a linear transformation of the input;
    a ReLU-based~\cite{nair2010rectified} neural network, e.g.,
    is piece-wise linear and the output thus a linear transformation of the input, cf.~\cite{montufar2014number}.
    However, due to the highly non-linear manner in which these linear transformations are `chosen', the corresponding contributions per input dimension do not seem to represent the learnt model parameters well, cf.~\cite{adebayo2018sanity}, and a lot of research is being conducted to find better explanations for the decisions of such neural networks, cf.~\cite{simonyan2013deep,springenberg2014striving,zhou2016CAM,selvaraju2017grad,shrikumar2017deeplift,sundararajan2017axiomatic,srinivas2019full,bach2015pixel}.
    
In this work, we introduce a novel network architecture, the \textbf{Convolutional Dynamic Alignment Networks (CoDA-Nets)}, {for which the model-inherent contribution maps are faithful projections of the internal computations and thus good `explanations' of the model prediction.} 
There are two main components to the interpretability of the CoDA-Nets. 
    First, the CoDA-Nets are \textbf{dynamic linear}, i.e., they compute their outputs through a series of input-dependent linear transforms, which are based on our novel \mbox{\textbf{Dynamic Alignment Units (DAUs)}}. 
        As in linear models, the output can thus be decomposed into individual input contributions, see Fig.~\ref{fig:teaser}.
    Second, the DAUs are structurally biased to compute weight vectors that \textbf{align with \mbox{relevant} patterns} in their inputs. 
In combination, the CoDA-Nets thus inherently  
produce contribution maps that are `optimised for interpretability': 
since each linear transformation matrix and thus their combination is optimised to align with discriminative features, the contribution maps reflect the most discriminative features \emph{as used by the model}.

With this work, we present a new direction for building inherently more interpretable neural network architectures with high modelling capacity.
In detail, we would like to highlight the following contributions:
\begin{enumerate}[wide, label={\textbf{(\arabic*)}}, itemsep=-.5em, topsep=0em, labelwidth=0em, labelindent=0pt]
    \item We introduce the Dynamic Alignment Units (DAUs), which 
    improve the interpretability of neural networks and have two key properties:
    they are 
    \emph{dynamic linear} 
    and align their weights with discriminative input patterns.
    \item Further, we show that networks of DAUs \emph{inherit} these two properties. In particular, we introduce Convolutional Dynamic Alignment Networks (CoDA-Nets), which are built out of multiple layers of DAUs. As a result, the \emph{model-inherent contribution maps} of CoDA-Nets highlight discriminative patterns in the input.
    \item We further show that the alignment of the DAUs can be promoted 
    by applying a `temperature scaling' to the final output of the CoDA-Nets. 
    \item We show that the resulting contribution maps 
    perform well under commonly employed \emph{quantitative} criteria for attribution methods. Moreover, under \emph{qualitative} inspection, we note that they exhibit a high degree of detail.
    \item Beyond interpretability, 
    CoDA-Nets are performant classifiers and yield competitive classification accuracies on the CIFAR-10 and TinyImagenet datasets.
\end{enumerate}
\section{related work} \label{sec:related}

  As a classical problem in the area of data quality, entity resolution has been extensively studied in the literature~\cite{christen2012data, elmagarmid2007duplicate, christophides2015entity}. The proposed techniques include those based on rules~\cite{fan2009reasoning, li2015rule, singh2017generating}, probabilistic theory~\cite{fellegi1969theory, singla2006entity} and machine learning~\cite{sarawagi2002interactive, kouki2017collective, arasu2010active, bellare2012active}. However, these traditional techniques lack effective mechanisms for quality control; ergo, they fail in ensuring high-quality guarantees.

  Active learning-based approaches~\cite{arasu2010active, bellare2012active} have been proposed in order to satisfy the precision requirement for ER. The authors of \cite{arasu2010active} proposed a technique that can optimize the recall while ensuring a pre-specified precision goal. The authors in~\cite{bellare2012active} proposed an improved algorithm that approximately maximizes the recall under a precision constraint. Considering that these techniques share the same classification paradigm with traditional machine learning-based ones; the former cannot enforce comprehensive quality guarantees specified by both precision and recall metrics as HUMO does.

   The progressive paradigm for ER~\cite{whang2013pay, altowim2014progressive} has also been proposed for the application scenarios in which ER should be processed efficiently, but it does not necessarily guarantee high-quality results. Taking a pay-as-you-go approach, it studied how to maximize the result's quality given a pre-specified resolution budget, which was defined based on the machine computation cost. A similar iterative algorithm, SiGMa, was proposed in \cite{lacoste2013sigma}. It can leverage both the structure information and string similarity measures to resolve entity alignment across different knowledge bases. Note that built on machine computation, these techniques could not be applied to enforce quality guarantees either.

   It has been well recognized that pure machine algorithms may not be able to produce satisfactory results in many practical scenarios~\cite{li2016crowdsourced}. Many researchers~\cite{wang2012crowder, whang2013question, vesdapunt2014crowdsourcing, gokhale2014corleone, mozafari2014scaling, wang2015crowd, chai2016cost, verroios2017waldo} have studied how to crowdsource an ER workload. For instance, recently, the authors of~\cite{chai2016cost} proposed a cost-effective framework that employs the partial order relationship on instance pairs to reduce the number of asked pairs. Similarly, the authors in~\cite{verroios2017waldo} provided solutions to take advantage of both pairwise and multi-item interfaces in a crowdsourcing setting. While, these works addressed the challenges specific to crowdsourcing; we instead investigate a different problem: how to divide a workload between the human and the machine such that the user-specified quality guarantees can be met. In this paper, we assume that human workload can be performed with high quality; yet we do not investigate the problems targeted by existing interactive and crowdsourcing solutions. Note that the workload assigned to the human by HUMO can be naturally processed in a crowdsourcing manner. Our work can thus be considered orthogonal to existing works on crowdsourcing. It is interesting to investigate how to seamlessly integrate a crowdsourcing platform into HUMO in future work.


\section{Problem Setting} \label{sec:setting}

  Entity resolution's main purpose is to determine whether two records are equivalent. Two records are deemed equivalent if and only if they correspond to the same real-world entity. We denote an ER workload by $D$, $D=\{d_1, d_2, \cdots, d_n\}$, in which $d_i$ represents an instance pair.  An ER solution corresponds to a label assignment $L$ for $D$, $L=\{l_1, l_2, \cdots, l_n\}$, in which $l_i=1$ if $d_i$ is labeled as {\em match} and $l_i=0$ if it is labeled as {\em unmatch}. In this paper, $d_i$ is called a matching pair if its two records are equivalent; otherwise, it is called an unmatching pair.

  As usual, we measure the quality of an ER solution by the metrics of precision and recall. Precision is the fraction of matching pairs among the pairs labeled as {\em match}, while recall is the fraction of correctly labeled matching pairs among all the matching pairs. Formally, we denote the ground-truth labeling solution for $D$ by $\hat{L}$, $\hat{L} = \{\hat{l}_1, \hat{l}_2, \cdots, \hat{l}_n\}$, in which $\hat{l}_i=1$ if $d_i$ is a matching pair and $\hat{l}_i=0$ otherwise. Given a labeling solution $L$, we use $D_{tp}$ to denote its set of true positive pairs, $D_{tp}=\{d_i|\hat{l}_i=1 \wedge l_i=1 \}$, $D_{fp}$ to denote its set of false positive pairs, $D_{fp}=\{d_i|\hat{l}_i=0 \wedge l_i=1 \}$, and $D_{fn}$ to denote its set of false negative pairs, $D_{fn}=\{d_i|\hat{l}_i=1 \wedge l_i=0\}$. Accordingly, the achieved precision level of $L$ can be represented by
\begin{equation} \label{eq:precision}
    \emph{precision(D,L)} = \frac{|D_{tp}|}{|D_{tp}| + |D_{fp}|}.
\end{equation}
Similarly, the achieved recall level of $L$ can be represented by
\begin{equation} \label{eq:recall}
    \emph{recall(D,L)} = \frac{|D_{tp}|}{|D_{tp}| + |D_{fn}|}.
\end{equation}

   Formally, the problem of entity resolution with quality guarantees specified at both precision and recall fronts is defined as follows:
\begin{definition}
\label{problemsetting}
{\bf [Entity Resolution with Quality Guarantees]}  Given a set of instance pairs, $D=\{d_1, d_2, \cdots, d_n\}$, the problem of entity resolution with quality guarantees is to give a labeling solution $L$ for $D$ provided with a confidence level $\theta$, $precision(D,L)\geq\alpha$ and $recall(D,L)\geq\beta$, in which $\alpha$ and $\beta$ denote the user-specified precision and recall values respectively.
\end{definition}


\section{HUMO Framework} \label{sec:framework}

  In this section, we first give an overview on HUMO, then introduce its optimization problem.


\subsection{Framework Overview}

    The primary idea behind HUMO is to enforce quality guarantees by dividing an ER workload between the human and the machine. It assigns easy instances, which can be automatically labeled with high accuracy, to the machine, while leaving more challenging instances for human-operated manual verification.

		
		Suppose that each instance pair in $D$ can be evaluated by a machine metric. This metric can be pair similarity or other classification metrics (e.g. match probability \cite{fellegi1969theory} and Support Vector Machine distance \cite{kopcke2010evaluation}). Note that entity resolution by classification usually categorizes pairs into $match$ and $unmatch$ based on a selected metric. Given a machine metric, HUMO assumes that $D$ statistically satisfies monotonicity of precision. Given a set of instance pairs, its precision refers to the proportion of matching pairs among all pairs. Intuitively, the monotonicity assumption of precision states that the higher (resp. lower) metric values a set of pairs have, the more probably they are matching pairs (resp. unmaching pairs). It can be observed that given a machine metric, the monotonicity assumption of precision underlies its effectiveness as a classification metric. {\em For simplicity of presentation, we use pair similarity as a machine metric example in this paper. However, HUMO is similarly effective with other machine metrics}. For instance, with the metric of SVM, each pair can be measured by its distance to a classification plane; with the metric of match probability, each pair can be measured by its estimated probability.
			
		
 Formally, we define the monotonicity assumption of precision, which was first proposed in \cite{arasu2010active}, as follows:
\begin{assumption}[Monotonicity of Precision]
  A value interval $I_i$ is dominated by another interval $I_j$, denoted by $I_i\preceq I_j$, if every value in $I_i$ is less than every value in $I_j$. We say that precision is monotonic with respect to a pair metric if for any two value intervals $I_i\preceq I_j$ in [0,1], we have $\mathsf{R}(I_i)\leq\mathsf{R}(I_j)$, in which $\mathsf{R}(I_i)$ denotes the precision of the set of instance pairs whose metric values are located in $I_i$.
\label{monotonicity}
\end{assumption}


\begin{figure}[h]
\setlength{\abovecaptionskip}{\figcaptionspace}
\centering
\includegraphics[width=\linewidth]{figures/framework.pdf}
\caption{The HUMO framework.}
\label{fig_basic_idea}
\end{figure}


  With the metric of pair similarity, the underlying intuition of Assumption \ref{monotonicity} is that the more similar two records are, the more likely they refer to the same real-world entity. According to the monotonicity assumption, a pair with high similarity has a correspondingly high probability of being a matching pair. A pair with low similarity instead has a correspondingly low probability of being a matching pair. These two groups of instance pairs can be supposed to be easy in that they can be automatically labeled by the machine with high accuracy. In comparison, the instance pairs having medium similarities are more challenging because labeling them either way by machine would introduce more considerable errors.


  The HUMO framework is shown in Figure~\ref{fig_basic_idea}. It divides the similarity interval [0,1] into three disjoint intervals, $I_-$, $I_H$ and $I_+$, in which $I_-$=[0,$v^-$), $I_H$=[$v^-$,$v^+$] and $I_+$=($v^+$,1], and correspondingly $D$ into three disjoint subsets, $D_-$, $D_H$ and $D_+$. It automatically labels the pairs in $D_-$ as {\em unmatch}, the pairs in $D_+$ as {\em match}, and assigns the pairs in $D_H$ to the human for manual verification. It can be observed that HUMO can flexibly enforce quality guarantees by adjusting the range of $D_H$. In the extreme case of $D_H=\emptyset$, HUMO boils down to a straightforward machine-based classification technique. With the assumption that the human performs better than the machine on a quality basis, enlarging the range of $D_H$ would result in improved quality. In the opposite extreme case of $D_H=D$, HUMO achieves the best performance, which is the same as the human's.

  Generally, given a HUMO solution $S$ consisting of $D_-$, $D_H$ and $D_+$, the lower bound of its achieved precision level can be represented by
\begin{equation}
   precision_l(S)=\frac{N^+_l(D_+)+N^+_l(D_H)}{N(D_+)+N(D_H)},
\label{eq:precision-bound}
\end{equation}
in which $N(\cdot)$ denotes the total number of pairs in a set and $N^+_l(\cdot)$ denotes the lower bound of the total number of matching pairs in a set. Similarly, the lower bound of its achieved recall level can be represented by
\begin{equation}
  recall_l(S)=\frac{N^+_l(D_+)+N^+_l(D_H)}{N^+_l(D_+)+N^+_l(D_H)+N^+_u(D_-)},
\label{eq:recall-bound}
\end{equation}
in which $N^+_u(\cdot)$ denotes the upper bound of the total number of matching pairs in a set. In this paper, for the sake of presentation simplicity, we assume that the pairs in $D_H$ can be manually labeled accurately (100\% accuracy with 100\% confidence). With that being said, we emphasize that HUMO's effectiveness does not depend on said assumption, since it can work properly provided that quality guarantees can be enforced on $D_H$. In the case that human errors are introduced in $D_H$, the lower bounds of the achieved precision and recall can be similarly estimated based on Eq.~\ref{eq:precision-bound} and Eq.~\ref{eq:recall-bound}. Nonetheless, it is worthy to point out that under the assumption that the human yields higher quality matches than the machine, the best quality guarantees HUMO can achieve are no better than human attained ones on $D_H$.


\subsection{Optimization Problem}

  Since human labor is in practice much more expensive than machine computation, HUMO aims to minimize human cost provided that user-specified quality requirements can be satisfied. By quantifying human cost by the number of manually inspected instance pairs in $D_H$, we formally define HUMO's optimization problem as follows:

\begin{definition}
\label{optimization}
{\bf [Minimizing Human Cost in HUMO].} Given a set of instance pairs, $D$, a confidence level $\theta$, a precision level $\alpha$ and a recall level $\beta$, HUMO's optimization problem is represented by
\begin{equation}
\begin{split}
& \quad \underset{S_i}{argmin} (|D_H(S_i)|)\\
& subject \quad to \quad P(precision(D,S_i)\geq\alpha)\geq\theta , \\
& \hspace{0.7in} P(recall(D,S_i)\geq\beta)\geq\theta ,
\end{split}
\label{eq:minimization}
\end{equation}
in which $S_i$ denotes a HUMO solution, $D_H(S_i)$ denotes the set of instance pairs assigned to the human by $S_i$, $precision(D,S_i)$ denotes the achieved precision by $S_i$, $recall(D,S_i)$ denotes the achieved recall by $S_i$, and $P(\cdot)$ denotes the probability of a required quality level being met.
\end{definition}

  Note that in Definition~\ref{optimization}, $P(\cdot)$, or the probability of satisfying a certain required quality level, is statistically equivalent to the confidence level $\theta$ defined in Definition.~\ref{problemsetting}. It can be observed that HUMO achieves a 100\% precision and recall levels in the extreme case when all the instance pairs are assigned to the human (i.e. $D_H$=$D$). In general, its achieved precision and recall levels tend to decrease as $D_H$ becomes smaller. However, the problem of searching for the minimum size $D_H$ is challenging due to the fact that the ground-truth match proportions of $D_-$ and $D_+$ are unknown. In the following sections, we propose three search approaches: a conservative baseline one purely based on the monotonicity assumption of precision (Section~\ref{sec:conservative}), a more aggressive sampling-based one (Section~\ref{sec:aggressive}), and a hybrid one that benefits from the strengths of both previous approaches (Section~\ref{sec:hybrid}). They estimate the match proportions of $D_-$ and $D_+$ based on different assumptions.


\iffalse	
	According to Eq.~\ref{eq:precision} and ~\ref{eq:recall}, $precision(D,S_i)$ can be represented by
\begin{equation}
  \frac{|D_H|\cdot \mathsf{R}(D_H)+|D_+|\cdot \mathsf{R}(D_+)}{|D_H|\cdot \mathsf{R}(D_H)+|D_+|},
\end{equation}
in which $\mathsf{R}(D_*)$ denotes the ground-truth match proportion of the pair instances in $D_*$. Similarly, $recall(D,S_i)$ can be represented by
\begin{equation}
  \frac{|D_H|\cdot \mathsf{R}(D_H)+|D_+|\cdot \mathsf{R}(D_+)}{|D_-|\cdot \mathsf{R}(D_-)+|D_H|\cdot \mathsf{R}(D_H)+|D_+|\cdot \mathsf{R}(D_+)}.
\end{equation}
\fi













\section{Baseline Approach}\label{sec:conservative}

  The baseline approach assumes that the instance pairs in the workload of $D$ statistically satisfy monotonicity of precision. It begins with an initial medium similarity value (e.g. the boundary value of a classifier or simply a median value), and then incrementally identifies the upper and lower bounds of the similarity interval of $D_H$, $v^-$ and $v^+$.

\begin{figure}[!htb]
\setlength{\abovecaptionskip}{\figcaptionspace}
\centering
\subfigure[Incrementally moving the upper bound of $D_H$ right.]
{\includegraphics[width=\linewidth]{figures/baseline_precision.pdf}
\label{basic_idea_precision}}
\subfigure[Incrementally moving the lower bound of $D_H$ left.]
{\includegraphics[width=\linewidth]{figures/baseline_recall.pdf}
\label{basic_idea_recall}}
\caption{The demonstration of the baseline solution.}
\end{figure}

   Initially, it sets $v^-$ and $v^+$ to a common value $v_0$, $v^-=v^+=v_0$. Then, it iteratively enlarges the similarity interval of $D_H$ until the desired precision and recall requirements are satisfied. Since both lower and upper bounds affect the precision and recall estimates, the search process alternately moves $v^-$ left and $v^+$ right.

  Suppose that $v^+$ is moved from $v_{i-1}^+$ to a higher value $v_i^+$, as shown in Figure \ref{basic_idea_precision}. It is clear that as the mark of $v^+$ is moved right, the number of true positives would remain constant while the number of false positives would decrease. As a result, the achieved precision would in turn increase. We denote the interval $(v_{i-1}^+,v_i^+]$ by $I_i^+$. According to the monotonicity assumption of precision, the match proportion of the pairs in the interval $(v_i^+,1]$ is no less than $\mathsf{R}(I_i^+)$, in which $\mathsf{R}(I_i^+)$ denotes the observed match proportion of the pairs in $I_i^+$. Therefore, with $v^-$ and $v^+=v_i^+$, the lower bound of the achieved precision level can be represented by
\begin{equation}
  \frac{|D_H|\cdot \mathsf{R}(D_H)+|D_+|\cdot \mathsf{R}(I_i^+)}{|D_H|\cdot \mathsf{R}(D_H)+|D_+|},
\end{equation}
in which $|D_H|$ and $|D_+|$ denote the total numbers of pairs in $D_H$ and $D_+$ respectively. Accordingly, given the precision requirement $\alpha$, the match proportion of the interval $I_i^+$ should satisfy
\begin{equation}
\mathsf{R}(I_i^+)\geq\frac{\alpha\cdot |D_+|-(1-\alpha)\cdot \mathsf{R}(D_H)\cdot |D_H|}{|D_+|}.
\label{eq:baseline-precision-condition}
\end{equation}
In other words, the precision requirement $\alpha$ would be satisfied once the observed match proportion of the interval $I_i^+$ reaches the threshold presented in Eq.~\ref{eq:baseline-precision-condition}.


  Similarly, suppose that the lower bound $v^-$ is moved from $v_{j-1}^-$ to a lower value $v_j^-$, as shown in Figure~\ref{basic_idea_recall}. We denote the interval $[v_j^-,v_{j-1}^-)$ by $I_j^-$. According to the monotonicity assumption of precision, the match proportion of the pairs in the interval $[0,v_j^-)$ is no larger than $\mathsf{R}(I_j^-)$. Therefore, with $v^+=v_i^+$ and $v^-=v_j^-$, the lower bound of the achieved recall level can be represented by
\begin{equation} \frac{|D_H|\cdot\mathsf{R}(D_H)+|D_+|\cdot\mathsf{R}(I_i^+)}{|D_-|\cdot\mathsf{R}(I_j^-)+|D_H|\cdot\mathsf{R}(D_H)+|D_+|\cdot\mathsf{R}(I_i^+)}.
\end{equation}
Accordingly, given the recall requirement $\beta$, the match proportion of the interval $I_j^-$ should satisfy
\begin{equation}
  \mathsf{R}(I_j^-)\leq\frac{(1-\beta)(|D_H|\cdot\mathsf{R}(D_H)+|D_+|\cdot\mathsf{R}(I_i^+))}{\beta\cdot |D_-|}.
\label{eq:baseline-recall-condition}
\end{equation}
In other words, the recall requirement $\beta$ would be satisfied once the observed match proportion of $I_j^-$ is below or equal to the threshold presented in Eq.~\ref{eq:baseline-recall-condition}.

	The search process alternately moves $v^+$ right and $v^-$ left to enforce precision and recall requirements. Once $\mathsf{R}(I_i^+)$ reaches the threshold specified in Eq.~\ref{eq:baseline-precision-condition}, the upper bound of $D_H$ would be finally fixed at $v_i^+$. It can be observed that with the upper bound fixed at $v_i^+$, moving $v^-$ to a lower value would only increase the estimated precision level. Similarly, once $\mathsf{R}(I_j^-)$ falls below the threshold specified in Eq.~\ref{eq:baseline-recall-condition}, the lower bound of $D_H$ would be finally fixed at $v_j^-$. Due to the monotonicity assumption, with the lower bound fixed at $v_j^-$, moving $v^+$ to a higher value would only increase the estimated recall level. In practical implementation, we can set the unit movement of $v^-$ and $v^+$ by the number of instance pairs: the intervals $(v_{i-1}^+,v_i^+]$ and $[v_j^-,v_{j-1}^-)$ always contain the same number of instance pairs. Further details on the search process are omitted here due to space limits, and a more thorough explanation can be referred to in our technical report \cite{chen2017humoreport}.

   By following the above reasoning, the baseline search process can return a solution satisfying the user-specified precision and recall levels with a 100\% confidence, provided that the monotonicity assumption holds. Its computational complexity is only linear with the number of instance pairs in $D$ in the worst case. Finally, we conclude this section with Theorem~\ref{theorem-baseline}, whose proof follows naturally from our above analysis.

\begin{theorem} \label{theorem-baseline}
  Given an ER workload of $D$, the baseline search process returns a HUMO solution that can ensure the precision and recall levels of $\alpha$ and $\beta$ respectively with the confidence of 100\% provided that the monotonicity assumption holds on $D$.
\end{theorem}


\section{Sampling-based Approach} \label{sec:aggressive}

  The baseline approach estimates the match proportions of $D_-$ and $D_+$ by the observed match proportions of the intervals in $D_H$. However, it can be noticeable that the match proportion of $D_-$ is usually significantly smaller than that of $D_H$, while the match proportion of $D_+$ is usually considerably larger than that of $D_H$. Strictly speaking, the baseline approach may overestimate the match proportion of $D_-$, and may also underestimate the match proportion of $D_+$. As a result, it would require considerably more than necessary human cost to enforce quality guarantees. To alleviate this limitation, we propose a more aggressive sampling-based approach in this section. Compared with the baseline approach, it is more aggressive in that it estimates the match proportions of $D_-$ and $D_+$ by directly sampling them.

  The sampling-based approach divides $D$ into multiple disjoint unit subsets and estimates their match proportions by sampling. We first present an all-sampling solution that samples all the subsets. To reduce human cost, we also present an improved partial-sampling solution that only requires sampling a portion of the subsets.

\begin{figure}[!htb]
\setlength{\abovecaptionskip}{\figcaptionspace}
\centering
\includegraphics[width=\linewidth]{figures/sampling.pdf}
\caption{The demonstration of sampling-based solution.}
\label{sampling_based_demonstration}
\end{figure}


\subsection{All-Sampling Solution} \label{sec:all-sampling}

   Suppose that $D$ is divided into $m$ disjoint subsets, $D=D_1\cup\cdots\cup D_m$, and the subsets are ordered by their similarity values. If $i<j$, then $\forall d\in D_i$ and $\forall d'\in D_j$, we have $sim(d)\leq sim(d')$, in which $sim(d)$ denotes the similarity value of $d$. With the notation of $D_i$, we can represent $D_H$ by a union of subsets, $D_H=D_i\cup D_{i+1}\cdots\cup D_j$, in which $D_i$ is the lower bound subset of $D_H$ while $D_j$ is its upper bound subset. We also denote the sampled match proportion of $D_i$ by $\mathsf{R}_i$. We first consider the hypothetical case that the estimate of $\mathsf{R}_i$ is accurate, and then integrate sampling errors into bound computation.

  When the estimate of $\mathsf{R}_i$ is hypothetically accurate, the achieved recall level of a HUMO solution solely depends on $D_H$'s lower bound. Therefore, the all-sampling solution first identifies $D_H$'s lower bound subset to meet the constraint on recall, then identifies its upper bound subset to meet the precision constraint. With the lower bound of $D_H$ set at $D_i$, the achieved recall level can be estimated by
\begin{equation}
\label{eq:all-recall}
  recall(D,S)=\frac{\sum_{i\leq k\leq m}{|D_k|\cdot \mathsf{R}_k}}{\sum_{1\leq k\leq m}{|D_k|\cdot \mathsf{R}_k}}.
\end{equation}
Therefore, to minimize the size of $D_H$ while ensuring the recall level $\beta$, the search process initially sets the lower bound to $D_1$, and then iteratively moves it right from $D_k$ to $D_{k+1}$ until the estimated recall level specified in Eq.~\ref{eq:all-recall} falls below $\beta$.

  The search process then deals with the precision constraint in a similar way by incrementally identifying the upper bound of $D_H$. Suppose that the lower bound of $D_H$ has been identified to be $D_i$. With its upper bound set at $D_j$, the achieved precision level can be estimated by
\begin{equation}
\label{eq:all-precision}
  precision(D,S)=\frac{\sum_{i\leq k\leq m}{|D_k|\cdot \mathsf{R}_k}}{\sum_{i\leq k\leq j}{|D_k|\cdot \mathsf{R}_k}+\sum_{j+1\leq k\leq m}{|D_k|}}
\end{equation}
Therefore, to minimize the size of $D_H$ while ensuring the precision level $\alpha$, the search process initially sets the upper bound to $D_m$, and then iteratively moves it left from $D_{k}$ to $D_{k-1}$ until the estimated precision level specified in Eq.~\ref{eq:all-precision} falls below $\alpha$.



  Now we describe how to integrate sampling errors into bound computation. For fulfilling the confidence level, we resort to the theory of stratified random sampling \cite{cochran1977sampling} to estimate sampling error margins. We denote the total number of pairs in $D$ by $n$ and the number of pairs in the subset $D_i$ by $n_i$. Based on the sampled match proportion estimate of $D_i$, we can compute the mean of the match proportion of $D$ and its estimated standard deviation, which are denoted by ${\bar{\mathsf{R}}_D}$ and $\sigma_D$ respectively. The details on how to compute ${\bar{\mathsf{R}}_D}$ and $\sigma_D$ can be found in~\cite{chen2017humoreport}. Given the confidence level $\theta$, the total number of matching pairs in $D$ falls reasonably within the interval
\begin{equation}
  [n\cdot(\bar{\mathsf{R}}_D-t_{(1-\theta, d.f.)}\cdot\sigma_{D}), n\cdot(\bar{\mathsf{R}}_D+t_{(1-\theta, d.f.)}\cdot\sigma_{D})],
\label{eq:confidenceintervals}
\end{equation}
in which $t_{(1-\theta, d.f.)}$ is {\em Student's t value} for {\em d.f.} degrees of freedom and the confidence level $\theta$ for two-sided critical regions. In Eq.~\ref{eq:confidenceintervals}, as typical in stratified sampling \cite{cochran1977sampling}, we use Student's t value to take account of the sampling error due to limited sample size. Suppose that a random variable $T$ has a Student's t-distribution, a Student's t value for confidence level $\theta$ for two-sided critical regions is the value, let's say $\tilde{t}$, that satisfies $P(-\tilde{t} < T < \tilde{t})=\theta$, where $P(\cdot)$ represents the probability.

  Next, we apply the analysis results of confidence error margins in the recall and precision estimates as presented in Eq.~\ref{eq:all-recall} and ~\ref{eq:all-precision}. According to Eq.~\ref{eq:all-recall}, the lower bound of the recall estimate can be guaranteed by setting a lower bound on $n_{[i,m]}^{+}$ and an upper bound on $n_{[1,i-1]}^{+}$, in which $n_{[i,j]}^{+}$ denotes the total number of matching pairs in the subset union, $D_i\cup D_{i+1}\cdots\cup D_j$. Suppose that the lower bound of $D_H$ is set at $D_i$. Given the confidence level $\theta$ and the recall level $\beta$, the HUMO solution meets the recall requirement if
\begin{equation}
\label{eq:recall-condition}
  \beta\leq \frac{lb(n_{[i,m]}^{+}, \sqrt{\theta})}{ub(n_{[1,i-1]}^{+}, \sqrt{\theta}) + lb(n_{[i,m]}^{+}, \sqrt{\theta})},
\end{equation}
in which $lb(n_{[i,m]}^{+}, \sqrt{\theta})$ denotes the lower bound of $n_{[i,m]}^{+}$ with the confidence $\sqrt{\theta}$, and $ub(n_{[1,i-1]}^{+}, \sqrt{\theta})$ denotes the upper bound of $n_{[1,i-1]}^{+}$ with the confidence $\sqrt{\theta}$. Since the bound estimations on $n_{[i,m]}^{+}$ and $n_{[1,i-1]}^{+}$ are independent, the lower bound of the recall level specified in Eq.~\ref{eq:recall-condition} has the desired confidence $\theta$.

  Similarly, suppose that the lower and upper bounds of $D_H$ are set at $D_i$ and $D_j$ respectively. Given the confidence level $\theta$ and the precision level $\alpha$, the HUMO solution meets the precision requirement if
\begin{equation}
\label{eq:precision-condition}
  \alpha\leq \frac{lb(n_{[i,j]}^{+}, \sqrt{\theta}) + lb(n_{[j+1,m]}^{+}, \sqrt{\theta})}{lb(n_{[i,j]}^{+}, \sqrt{\theta}) + n_{[j+1,m]}}.
\end{equation}
Since the bound estimations on $n_{[i,j]}^{+}$ and $n_{[j+1,m]}^{+}$ are independent, the lower bound of the precision level specified in Eq.~\ref{eq:precision-condition} has the desired confidence $\theta$.

  The all-sampling search process iteratively searches for the lower and upper bounds of $D_H$. It first identifies the maximal value of $i$ such that the condition specified in Eq.~\ref{eq:recall-condition} is satisfied. It begins with $i=1$ and then iteratively moves the lower bound right from $D_i$ to $D_{i+1}$. Similarly, with the lower bound of $D_H$ set at $D_i$, it identifies the minimal value of $j$ such that the condition specified in Eq.~\ref{eq:precision-condition} is satisfied. It begins with $j=m$ and then iteratively moves the upper bound left from $D_j$ to $D_{j-1}$. More details on the search process are however omitted here due to space constraints. They can be found in our technical report \cite{chen2017humoreport}.

  The worst-case computational complexity of the all-sampling search process can be represented by ${\bf O}(n+m^2)$, in which $n$ denotes the total number of pairs in $D$ and $m$ denotes the total number of subsets. Finally, we conclude this subsection with the following theorem, whose proof follows naturally from our above analysis:
\begin{theorem}
  Given an ER workload of $D$, a confidence level $\theta$, a precision level $\alpha$ and a recall level $\beta$, the all-sampling search process returns a HUMO solution that can ensure the precision and recall levels of $\alpha$ and $\beta$ respectively with the confidence $\theta$.
\end{theorem}


\subsection{Partial-Sampling Solution} \label{sec:partial-sampling}

 Note that samples should be labeled by the human and the optimization objective of HUMO is to minimize human cost. The all-sampling solution has to sample every subset; therefore its human cost consumed on labeling samples is usually prohibitive. In this subsection, we propose an improved solution that only needs to sample a portion of the subsets. It achieves the purpose by approximating the match proportions of unsampled subsets based on those observed on sampled ones. We use the Gaussian process (GP) \cite{rasmussen2006gaussian}, which is a classical technique for non-parametric regression. GP assumes that the match proportions of subsets have a joint Gaussian distribution. It can smoothly integrate sampling error margins into the approximation process.

   Given $k$ sampled subsets, we denote their observed match proportions by $\mathsf{R} = [\mathsf{R}_1, \mathsf{R}_2, \ldots, \mathsf{R}_k]^T$, and their corresponding average similarity values by $V = [v_1, v_2, \ldots, v_k]^T$. The Gaussian process estimates the match proportion, $\mathsf{R}_*$, of a new similarity value, $v_*$, based on $\mathsf{R}$, the observed match proportions of $V$. According to the assumption of GP, the random variables $[V^T, v_*]^T$ satisfy a joint Gaussian distribution, which can be represented by
  \begin{equation}
  \begin{bmatrix} V \\ v_* \end{bmatrix} \sim
  \mathcal{N}\left(0, \begin{bmatrix} \mathbf{K}(V, V) & \mathbf{K}(V, v_*) \\ \mathbf{K}(v_*, V) & \mathbf{K}(v_*, v_*) \end{bmatrix}\right),
  \label{eq:jointdistribution}
  \end{equation}
in which $\mathbf{K}(\cdot , \cdot)$ represents the covariance matrix. The details of how to compute the covariance matrix $\mathbf{K}(\cdot , \cdot)$ can be found in~\cite{chen2017humoreport}. Based on Eq.~\ref{eq:jointdistribution}, the mean of the match proportion of $v_*$, $\mathsf{R}_*$, can be represented by
  \begin{equation}
  \bar{\mathsf{R}}_* = \mathbf{K}(v_*, V) \cdot \mathbf{K}^{-1}(V, V) \cdot \mathsf{R}.
  \label{eq:gpr:mean}
  \end{equation}
The variance of $\mathsf{R}_*$ can be also represented by
\begin{equation}
  \sigma_{\mathsf{R}_*}^2 = \mathbf{K}(v_*, v_*) - \mathbf{K}(v_*, V) \cdot \mathbf{K}^{-1}(V, V) \cdot \mathbf{K}(V, v_*).
\label{eq:gpr:variance}
\end{equation}
Accordingly, the distribution of $\mathsf{R}_*$, the match proportion of $v_*$, can be represented by the following Gaussian function
\begin{equation}
  \mathsf{R}_* \sim \mathcal{N}\left(\bar{\mathsf{R}}_*, \sigma_{\mathsf{R}_*}^2\right).
\end{equation}

  Now we are ready to describe how to aggregate the estimations of multiple subsets. Note that the distribution of each subset's match proportion satisfies a Gaussian function. Given the $t$ subsets of $D_*$, $D_*$ = $\{D_*^1,D_*^2,\ldots,D_*^t\}$, we denote their corresponding numbers of pairs by $\{n_*^1, n_*^2,\ldots, n_*^t\}$, and their similarity values by $V_*=[v_*^1, v_*^2, \ldots, v_*^t]^T$. Then, the total number of match pairs in $D_*$, denoted by $n_*$, satisfies a Gaussian distribution. Its mean can be represented by
\begin{equation}
  \bar{n}_* = \sum_{i=1}^{t}n_*^i\cdot\bar{\mathsf{R}}_*^i,
\end{equation}
in which $\bar{\mathsf{R}}_*^i$ represents the mean of the match proportion of $D_*^i$. Its standard deviation can also be represented by
\begin{equation}
  \sigma_{D_*} = \sqrt{\sum_{1\leq i\leq t,1\leq j\leq t}n_*^i\cdot n_*^j\cdot cov(v_*^i, v_*^j)},
\end{equation}
in which $cov(v_*^i, v_*^j)$ is the covariance between two estimates and its value is the {\em (i,j)-th} element in the covariance matrix $\mathbf{K}(V_*,V_*)-\mathbf{K}(V_*,V)\cdot\mathbf{K^{-1}}(V,V)\cdot\mathbf{K}(V,V_*)$. Therefore, given the confidence level $\theta$, the corresponding confidence interval of the number of match pairs in $D_*$ can be represented by
\begin{equation}
  [\bar{n}_* - \mathcal{Z}_{(1-\theta)} \cdot \sigma_{D_*}, \bar{n}_* + \mathcal{Z}_{(1-\theta)} \cdot \sigma_{D_*}],
\label{eq:gpr:confidenceintervals}
\end{equation}
in which $\mathcal{Z}_{(1-\theta)}$ is the $(1-\frac{1-\theta}{2})$ point of {\em standard normal distribution}.


\begin{algorithm}
\setlength{\textfloatsep}{0pt}
\caption{Gaussian Regression of Match Proportion Function}
\label{alg:fit-gp}
\KwIn{Sorted disjoint subsets $\{D_1, D_2, ..., D_m\}$; Sampling cost range $[p^l, p^u]$; Error threshold $\varepsilon$.}
\KwOut{The function of match proportion, $F_k$.}
$j \gets m\cdot p^l$\;
$TrainSet \gets$ select j equidistance subsets $\{D_{i_1},$ $D_{i_2},$ $...,$ $D_{i_j}\}$\;
$\mathsf{V}, \mathsf{R} \gets$ sample every subset in $TrainSet$ to get their match proportion estimates\;
$F_k \gets$ use $\mathsf{V}, \mathsf{R}$ to train Gaussian process model\;
$IndexQueue \gets [(i_1, i_2), ..., (i_k, i_{k+1}), ..., (i_{j-1}, i_j)]$\;
\While{$IndexQueue$ is not empty \\ \qquad and $|TrainSet| < m\cdot p^u$}
{
    $(i_k, i_{k+1}) \gets IndexQueue.pop()$\;
    $D_x \gets$ the middle subset between $D_{i_k}$ and $D_{i_{k+1}}$\;
    $\mathsf{R}_x \gets$ match proportion of $D_x$ estimated by sampling\;
    \If{$|F_k(v_x) - \mathsf{R}_x| \geq \varepsilon$}
    {
        $IndexQueue.append([(i_k, x),(x, i_{k+1})])$\;
    }
    Add $D_x, v_x, \mathsf{R}_x$ to $TrainSet, \mathsf{V}, \mathsf{R}$ respectively\;
    $F_k \gets$ use $\mathsf{V}, \mathsf{R}$ to train Gaussian process model\;
}
return $F_k$.
\end{algorithm}

  The partial-sampling search process consists of two phases. It trains the function of match proportion by Gaussian regression in the first phase, it then searches for the lower and upper bounds of $D_H$ based on the trained function in the second phase. The function training's procedure is sketched in Algorithm\ref{alg:fit-gp}. Note that $D$ is divided into $m$ disjoint subsets \{$D_1$, $D_2$, $\ldots$, $D_m$\}. To balance approximation accuracy and sampling cost, it presets a range, $[p^l, p^u]$ (e.g. $[1\%, 5\%]$), for the proportion of sampled subsets among all subsets. Initially, the training set consists of $j$ sampled subsets, \{$D_{i_1}$, $D_{i_2}$, $\ldots$, $D_{i_j}$\}, in which $j=m\times p^l$ and $\forall 1\leq k\leq j-2$, $i_{k+1}-i_k=i_{k+2}-i_{k+1}$. In each iteration, the algorithm first trains an approximation function, denoted by $F_k$, by Gaussian regression based on the sampled subsets. It then uses $F_k$ to estimate the match proportion of a subset that is located in the middle point between two neighbouring sampled subsets. Suppose that $D_x$ denotes the subset between the sampled subsets $D_{i_k}$ and $D_{i_{k+1}}$. If the difference between the estimated value based on $F_k$ and the observed match proportion based on sampling exceeds a small threshold $\epsilon$, the algorithm would add $D_x$ into the training set; otherwise, it would not sample any other subset between $D_{i_k}$ and $D_{i_{k+1}}$ (except $D_x$) in the following iterations. Finally, the algorithm trains the function with the updated training set. This cycle of sampling and training is iteratively invoked until the trained function achieves a good approximation or the sampling cost reaches the upper bound of the pre-specified range (i.e. $p^u$).

  Similar to the procedure for all-sampling solution, the partial-sampling search process first identifies the maximal lower bound of $D_H$ to meet the recall requirement, and then identifies the minimal upper bound of $D_H$ to meet the precision requirement. The only difference is that the lower bounds of the achieved recall and precision levels of a HUMO solution should be estimated by the confidence intervals specified in Eq.~\ref{eq:gpr:confidenceintervals}.

  The worst-case computational complexity of Alg.~\ref{alg:fit-gp} is in the order of ${\bf O}(k^4)$, in which $k$ denotes the number of sampled unit subsets. The worst-case computational complexity of the search process can be represented by ${\bf O}(m\cdot k^2+m^3)$. Therefore, the worst-case computational complexity of the partial-sampling solution can be represented by ${\bf O}(n+m^3+m\cdot k^2+k^4)$. It can be observed that the effectiveness of the partial-sampling solution in ensuring quality guarantees depends on the accuracy of the Gaussian approximation. As shown by our empirical evaluation in Section~\ref{sec:experiment}, the partial-sampling solution is highly effective due to the powerfulness and robustness of the Gaussian process.

\section{Hybrid Approach}\label{sec:hybrid}

  The baseline approach usually overestimates the match proportion of $D_-$ while underestimating that of $D_+$. The sampling-based approach can alleviate both drawbacks to a large extent by directly sampling $D_-$ and $D_+$. However, it still has to consider confidence margins in the estimations of $D_-$ and $D_+$. Furthermore, it usually cannot afford to sample all the subsets in $D_-$ and $D_+$ due to prohibitive sampling cost. Generally, less samples would result in larger error margins. Therefore, there is no guarantee that a sampling-based estimation would always be better than the corresponding baseline one. As we show in Section~\ref{sec:experiment}, their relative performance actually depends on the characteristics of the given ER workload. This observation motivates us to propose a hybrid approach, which can take advantage of both estimations and use the better of both worlds in the process of bound computation.

  The hybrid approach begins with a HUMO solution of the partial-sampling approach. We denote the initial solution by $S_0$ and its lower and upper bounds of $D_H$ by $D_i$ and $D_j$ respectively. It searches for a better solution than $S_0$ by incrementally redefining $D_H$'s bounds using the better between the baseline and sampling-based estimates. Initially, it sets $D_H$ to be the single median subset of $D_i$ and $D_j$, $D_{\frac{i+j}{2}}$. Similar to the baseline approach, it alternately extends $D_H$'s upper and lower bounds until both precision and recall requirements are met. However, on reasoning about the match proportions of $D_-$ and $D_+$, instead of being purely based on the monotonicity of precision, it uses the better of both estimates. It alternately moves the upper bound from $D_u$ to $D_{u+1}$ and the lower bound from $D_l$ to $D_{l-1}$. After each movement of the upper bound, it checks whether the current solution satisfies the precision requirement. Similarly, after each movement of the lower bound, it checks whether the current solution satisfies the recall requirement. Note that the new range of $D_H$ can not exceed the range of $[D_i, D_j]$ in the initial solution $S_0$. Therefore, the resulting HUMO solution of the hybrid approach is at least as good as $S_0$. The details of the hybrid search process are omitted here due to space limits, but can be found in our technical report \cite{chen2017humoreport}.

  The worst-case computational complexity of the hybrid solution is the same as that of the partial-sampling solution, bounded by ${\bf O}(n+m^3+m\cdot k^2+k^4)$. Its effectiveness in ensuring quality guarantees depends on both the monotonicity assumption of precision and the accuracy of Gaussian approximation. As shown by our empirical evaluation in Section~\ref{sec:experiment}, the hybrid solution is highly effective in ensuring quality guarantees for HUMO.


\section{Experiments}\label{sec:experiments}
Here, we present BARNEY finetuning performance on several different tasks. \nikos{we need to be more specific here. what are the questions we would like to answer specifically or our hypothesis. make sure the points we make correspond to the claims that we make in the intro/abstract and are clearly stated.}
For fair comparison, we pretrain \textbf{BARNEY} on the same Wikipedia + BooksCorpus dataset from BERT. To avoid training from scratch we use the pretrained BERT model as a starting point for our encoder function. We consider several different methods of pretraining. \nikos{these sound like experimental details, I would mention the basic ones in a subsection here and defer the rest for the supplementary (make sure we include everything there).}


% \subsection{BARNEY Pretraining for Classification}
% We outline several methods to pretrain BARNEY, and perform experiments on the sentence similarity task (SST2) with RoBERTa-base BARNEY, to observe the relative performance on single-sentence classification.

% \input{tables/barney_training_sst}

\noindent \textbf{Datasets} \nikos{Mention some basic details about the datasets used here.}
 
\noindent \textbf{Model configuration} \nikos{Only the basic ones and defer to the supplementary for the details such as number of epochs, learning rate etc. (e.g. we follow configuration from ...)}

\noindent \textbf{Baselines} \nikos{Describe our baselines and the versions of our model that we examined.}

\subsection{GLUE}
The General Language Understanding Evaluation (GLUE) benchmark \citep{wang2018glue} is a collection of diverse natural language understanding text classification tasks. We evaluate BARNEY pretrained with the Denoising, Fixed methodology report its performance on the dev set of each task in Table 1. \\
We find that on multiple tasks BARNEY on the base models end up performing on par with the large models with a mere fraction of additional parameters and compute. 

% \ivan{Since these are classification tasks, I feel like BARNEY would have a good chance at beating BERT as a baseline. These classification task require backpropagation through BARNEY, in addition to the linear classification layer, in the finetuning process.}
% \nikos{We should definitely try this. I am curious to see how the autoencoder pretraining objective impacts the results. } 



% \subsection{SentEval}
% SentEval [cite] is an evaluation framework for fixed sentence embeddings on 17 downstream tasks. We follow a setup similar to \cite{Reimers2019SentenceBERT} by finetuning on a combination of Natural Language Inference and Semantic Textual Similarity training sets, then finetuning a linear regression head on the fixed representations for each task. We examine the perfomrance of both BARNEY and BERT + CAB.
% % \ivan{These are 17 downstream tasks that \textit{don't} backpropagate through the sentence encoder, but rather, evaluate the fixed-length sentence embeddings themselves. \url{https://github.com/facebookresearch/SentEval}. If we include BERT's [CLS] token in this framework, we definitely have a good shot here}
% \nikos{I don't believe that this evaluation will measure the full potential of the method since it's best on fixed embeddings. By the way some of these tasks overlap with GLUE where we plan to test anyway with finetuning (e.g.SST/MRPC). So, I'd rather suggest to look for some controlled generation task such as sentiment style transfer (e.g. like the one used by Shen et al 2020) to demonstrate the benefits for a real  generation task (other than the reconstruction tests below).}
% % \ivan{Gotcha. In SentenceBERT they test the fixed embeddings after finetuning to the NLI dataset, which I feel like a good table to include would be an extra row for our method on table 5: \url{https://arxiv.org/pdf/1908.10084.pdf}}


\subsection{Sentence Representations}
\citet{Reimers2019SentenceBERT} perform the methodology of \citet{conneau2017supervised} of finetuning on NLI by using simple pooling methods of BERT representations, such as mean, max, and cls, and evaluate their performance on sentence similarity task. We train and evaluate BARNEY with a similar setup and its unique context attention bottleneck to compare against their results.
 
% \citet{conneau2017supervised} show that one can obtain universal sentence representations by finetuning on natural langauge inference data. \citet{Reimers2019SentenceBERT} show how this methodology can be applied to pooled BERT representations to obtain such sentence representations. We compaire

\begin{table}[t]
	\centering 
	\footnotesize
	\begin{tabular}{l|c}
		\toprule
		\textbf{Pooling} & \textbf{Spearman} \\ \midrule
% 		\multicolumn{3}{|l|}{\textit{Pooling Strategy}} \\ \hline
		\texttt{MEAN} & 80.78 \\
		\texttt{MAX} & 78.76 \\
		\texttt{CLS} &  79.67 \\
		$\beta$ \text{ (ours)} & \textbf{81.88} \\
		\bottomrule
% 	    \textbf{Pooling} & \textbf{NLI} & \textbf{STSb} \\ \midrule
% % 		\multicolumn{3}{|l|}{\textit{Pooling Strategy}} \\ \hline
% 		\texttt{MEAN} & 80.78   & 87.44 \\
% 		\texttt{MAX} & 79.07 & 69.92 \\
% 		\texttt{CLS} & 79.80 & 86.62  \\
% 		\texttt{CAB}, Fixed (ours) & \textbf{81.88} & \\
% 		\bottomrule
% 		\hline
% 		\multicolumn{3}{|l|}{\textit{Concatenation}} \\ \hline
% 		$(u, v)$ & 66.04 & -\\
% 		$(|u-v|)$ & 69.78 & - \\
% 		$(u*v)$ & 70.54 & -\\
% 		$(|u-v|, u*v)$ & 78.37  & -\\
% 		$(u, v, u*v)$ & 77.44 & -  \\
% 		$(u, v, |u-v|)$ &  \textbf{80.78} & - \\
% 		$(u, v, |u-v|, u*v)$ & 80.44 & - \\ 
% 		\hline	
	\end{tabular}
	\caption{Performance of sentence representations from RoBERTa trained with different pooling methods on NLI data and then evaluated on STS benchmark's development set %(STSb)
	in terms of Spearman's rank correlation.}
	\label{tab:pooling}
\end{table}

% DEV RESULTS TO ADD
% MEAN  0.8092
% MAX   0.7876
% CLS   0.7967  0.7999
% SB    
% 




% 		\texttt{MEAN} & 80.78 \\
% 		\texttt{MAX} & 79.07 \\
% 		\texttt{CLS} & 79.80 \\
% 		\texttt{SB} (ours) & \textbf{81.88} \\

We find that using the context attention bottleneck provides significant gains over using the other simple pooling methods. We suspect is due to the bottleneck acting as "weighted pooling" by attending over all the final tokens, to compute the final representation rather than mean/max equally considering all tokens or cls considering the representations before the final layer.



\begin{table}[t]
	\centering 
	\footnotesize
	\renewcommand{\arraystretch}{1.3}
	\begin{tabular}{l | c | c}
		\toprule
		\textbf{Model} & \textbf{Spearman} & \textbf{Parameters} \\ \midrule
		\multicolumn{3}{l}{\textit{Unsupervised}} \\\midrule
% 		\multicolumn{2}{l}{\textit{Trained on NLI (not STS Benchmark)}} \\ \hline
		Avg.\ GloVe embeddings & 58.02 & - \\
		Avg.\ BERT embeddings &  46.35 & - \\
		\textsc{Autobot}-base unsup. & \textbf{58.49} & - \\\midrule
		\multicolumn{3}{l}{\textit{Supervised}} \\\midrule
		InferSent - GloVe &  68.03 & - \\
		Universal Sentence Encoder &  74.92 & - \\
% 		SBERT-base & 77.03 & \\
% 		SBERT-large & 79.23 & \\\hline

        % BERT-base & 74.81 & 110M \\
        RoBERTa-base & 75.37 & 125M\\
% 		SBERT-base & 76.81 & 110M \\
		SRoBERTa-base & 76.89 & 125M \\
% 		AUTOBOT BERT-base & 77.03 & 111M \\
		\textsc{Autobot}-base (ours) & \textbf{78.59} & 127M \\\hline
        % BERT-large & 78.67 & 336M \\
        RoBERTa-large & 80.16 & 355M \\
% 		SBERT-large & 79.23 & 336M \\
% 		SRoBERTa-large &  \textbf{80.32}  & 355M \\
% 		AUTOBOT BERT-large & 77.01 & 338M \\
% 		AUTOBOT RoBERTa-large & 79.93 & 360M \\
		 
% 		AUTOBOT RoBERTa-base ft1 & 77.24 & \\
% 		ft 2 & 76.17 & \\
% 		ft 3 & 76.20 & \\
% 		ft 1 10k & 78.26 & \\
% 		ft 2 10k & 77.03 & \\
% 		ft 3 10k & 77.37 & \\
% 		SBERT-base  &  85.35 &  \\
% 		SRoBERTa-base  & 84.79  & \\
% 		SBERT-large & 86.10 &  \\
% 		SRoBERTa-large & 86.15  &\\\hline 
% 		BARNEY BERT-base &  84.25 &  \\  % 84.31
% 		BARNEY RoBERTa-base &  & \\
% 		BARNEY BERT-large &  &\\
% 		BARNEY RoBERTa-large &  &\\
% 		\multicolumn{2}{l}{\textit{Trained on STS Benchmark}} \\ \hline
% 		BERT-base & 84.30 $\pm$ 0.76  \\
% 		SBERT-base & 84.67 $\pm$ 0.19 \\ 
% 		SRoBERTa-base & 84.92 $\pm$ 0.34 \\
% 		BARNEY RoBERTa-base &  $\pm$  \\
% 		BARNEY BERT-base &  $\pm$  \\ \hline 
		
% 		BERT-large  & 85.64 $\pm$ 0.81 \\ 
% 		SBERT-large & 84.45 $\pm$ 0.43 \\ 
% 		SRoBERTa-large & 85.02 $\pm$ 0.76 \\ 
% 		BARNEY RoBERTa-base &  $\pm$  \\
% 		BARNEY BERT-base &  $\pm$  \\ \midrule
		
% 		\multicolumn{2}{l}{\textit{Trained on NLI + STS benchmark}} \\ \hline
		
% 		BERT-base & 88.33 $\pm$ 0.19 \\ 
% 		SBERT-base & 85.35 $\pm$ 0.17 \\ 
% 		SRoBERTa-base & 84.79 $\pm$ 0.38 \\ 
% 		BARNEY RoBERTa-base &  $\pm$  \\
% 		BARNEY BERT-base &  $\pm$  \\ \hline 
		
% 		BERT-large & 88.77 $\pm$ 0.46 \\ 
% 		BARNEY RoBERTa-base &  $\pm$  \\
% 		BARNEY BERT-base &  $\pm$  \\
    \bottomrule
	\end{tabular}
	\caption{ \label{tab:nli_sts}On semantic textual similarity (STS), \textsc{Autobot} outperforms previous sentence representation methods and reaches a score similar to RoBERTa-large while having fewer parameters.   %The transformer models were finetuned on the natural language inference training set, and 
	We report Spearman's rank correlation on the test set and the model sizes are reported in terms of trained parameter size.}
% 	The test performance of different models finetuned on the NLI training set then evacuated on the STS test set. The model sizes are reported in parameter size for comparison. 

% 	\ivan{SBERT, whose framework we evaluate in using their hyperparameters, doesn't even have a significant improvement in the large model. I suspect this is due to not enough hyperparameter search. Should we keep just RoBERTa-large for the large models to keep our claim?} \ivan{They also actually don't report RoBERTa-large results}
% 	Trained only on NLI, eval on STS 
	
% 	\ivan{Only show this, and rerun these experiments. Might just show RoBERTa results for simplicity} \nikos{fix acronyms here and in other places in the text. btw are these results up-to-date?}
	 % Evaluation on the STS benchmark test set. BERT systems were trained with 10 random seeds and 4 epochs. SBERT was fine-tuned on the STSb dataset, SBERT-NLI was pretrained on the NLI datasets, then fine-tuned on the STSb dataset.
	
\end{table}



% 	\begin{tabular}{l|c}
% 		\toprule
% 		\textbf{Model} & \textbf{Spearman} \\ \midrule
% 		\multicolumn{2}{l}{\textit{Trained on NLI (not STS Benchmark)}} \\ \hline
% 		Avg.\ GloVe embeddings & 58.02\\
% 		Avg.\ BERT embeddings &  46.35\\
% 		InferSent - GloVe &  68.03 \\
% 		Universal Sentence Encoder &  74.92\\
% 		SBERT-base  &  77.03\\
% 		SBERT-large & 79.23 \\
% 		BARNEY BERT-base & \\
% 		BARNEY BERT-large & \\ \midrule
% 		\multicolumn{2}{l}{\textit{Trained on STS Benchmark}} \\ \hline
% 		BERT-base & 84.30 $\pm$ 0.76  \\
% 		SBERT-base & 84.67 $\pm$ 0.19 \\ 
% 		SRoBERTa-base & \textbf{84.92} $\pm$ 0.34 \\
% 		BARNEY RoBERTa-base &  $\pm$  \\
% 		BARNEY BERT-base &  $\pm$  \\ \hline 
		
% 		BERT-large  & \textbf{85.64} $\pm$ 0.81 \\ 
% 		SBERT-large & 84.45 $\pm$ 0.43 \\ 
% 		SRoBERTa-large & 85.02 $\pm$ 0.76 \\ 
% 		BARNEY RoBERTa-base &  $\pm$  \\
% 		BARNEY BERT-base &  $\pm$  \\ \midrule
		
% 		\multicolumn{2}{l}{\textit{Trained on NLI + STS benchmark}} \\ \hline
		
% 		BERT-base & \textbf{88.33} $\pm$ 0.19 \\ 
% 		SBERT-base & 85.35 $\pm$ 0.17 \\ 
% 		SRoBERTa-base & 84.79 $\pm$ 0.38 \\ 
% 		BARNEY RoBERTa-base &  $\pm$  \\
% 		BARNEY BERT-base &  $\pm$  \\ \hline 
		
% 		BERT-large & \textbf{88.77} $\pm$ 0.46 \\ 
% 		BARNEY RoBERTa-base &  $\pm$  \\
% 		BARNEY BERT-base &  $\pm$  \\ \bottomrule
% 	\end{tabular}






% % 		\multicolumn{2}{l}{\textit{Trained on NLI (not STS Benchmark)}} \\ \hline
% 		Avg.\ GloVe embeddings & 58.02 & - \\
% 		Avg.\ BERT embeddings &  46.35 & - \\
% 		InferSent - GloVe &  68.03 & - \\
% 		Universal Sentence Encoder &  74.92 & - \\\hline
% % 		SBERT-base & 77.03 & \\
% % 		SBERT-large & 79.23 & \\\hline

%         BERT-base & 74.81 & 110M \\
%         RoBERTa-base & 75.37 & 125M\\
%         BERT-large & & 336M \\
%         RoBERTa-large & & 355M \\\hline
		
		
% 		SBERT-base & 76.81 & 110M \\
% 		SRoBERTa-base & 76.89 & 125M \\
% 		SBERT-large & 79.23 & 336M \\
% 		SRoBERTa-large &   & 355M \\\hline
		
% 		AUTOBOT BERT-base & 77.03 & \\
% 		AUTOBOT RoBERTa-base & 78.59 & \\
% % 		AUTOBOT RoBERTa-base ft1 & 77.24 & \\
% % 		ft 2 & 76.17 & \\
% % 		ft 3 & 76.20 & \\
% % 		ft 1 10k & 78.26 & \\
% % 		ft 2 10k & 77.03 & \\
% % 		ft 3 10k & 77.37 & \\
% 		AUTOBOT BERT-large & \\
% 		AUTOBOT RoBERTa-large & \\
% % 		SBERT-base  &  85.35 &  \\
% % 		SRoBERTa-base  & 84.79  & \\
% % 		SBERT-large & 86.10 &  \\
% % 		SRoBERTa-large & 86.15  &\\\hline 
% % 		BARNEY BERT-base &  84.25 &  \\  % 84.31
% % 		BARNEY RoBERTa-base &  & \\
% % 		BARNEY BERT-large &  &\\
% % 		BARNEY RoBERTa-large &  &\\
% % 		\multicolumn{2}{l}{\textit{Trained on STS Benchmark}} \\ \hline
% % 		BERT-base & 84.30 $\pm$ 0.76  \\
% % 		SBERT-base & 84.67 $\pm$ 0.19 \\ 
% % 		SRoBERTa-base & 84.92 $\pm$ 0.34 \\
% % 		BARNEY RoBERTa-base &  $\pm$  \\
% % 		BARNEY BERT-base &  $\pm$  \\ \hline 
		
% % 		BERT-large  & 85.64 $\pm$ 0.81 \\ 
% % 		SBERT-large & 84.45 $\pm$ 0.43 \\ 
% % 		SRoBERTa-large & 85.02 $\pm$ 0.76 \\ 
% % 		BARNEY RoBERTa-base &  $\pm$  \\
% % 		BARNEY BERT-base &  $\pm$  \\ \midrule
		
% % 		\multicolumn{2}{l}{\textit{Trained on NLI + STS benchmark}} \\ \hline
		
% % 		BERT-base & 88.33 $\pm$ 0.19 \\ 
% % 		SBERT-base & 85.35 $\pm$ 0.17 \\ 
% % 		SRoBERTa-base & 84.79 $\pm$ 0.38 \\ 
% % 		BARNEY RoBERTa-base &  $\pm$  \\
% % 		BARNEY BERT-base &  $\pm$  \\ \hline 
		
% % 		BERT-large & 88.77 $\pm$ 0.46 \\ 
% % 		BARNEY RoBERTa-base &  $\pm$  \\
% % 		BARNEY BERT-base &  $\pm$  \\

Using this setup, we compare directly the performance of BARNEY to other models on the sentence similarity task, ones which have not been trained on STS data. We find that BARNEY ends up performing singificantly better, and ends up achieving SBERT-large level performance with significantly less parameters.


% \subsection{BARNEY Pretraining for Generation}
% We outline several methods to pretrain BARNEY, and perform experiments on the sentence similarity task (SST2) with RoBERTa-base BARNEY, to observe the relative performance on single-sentence classification.


\subsection{Unsupervised Style Transfer}
% sentiment style transfer (e.g. like the one used by Shen et al 2020)
To evaluate properties of the latent space of BARNEY, we perform the experiment of \citet{shen2019educating} were we compute a “sentiment vector” $v$ from 100 negative and positive sentences, and change the sentiment of a sentence by encoding it, adding a multiple of the sentiment vector to the sentence representation, then decoding the resulting representation. We observe how the accuracy, BLEU, and perplexity change as we add a larger multiple of the sentiment vector to the representation in Table 4.

% and use it to change the sentiment of the test sentences.

%% !TEX root=econ_dispatch.tex
In this section, we first propose a reliable static renewable power scenario generation method in each time interval $1,\dots,T$. Then we present an efficient dynamic renewable power scenario generation method for the entire time horizon.

\subsection {Static Scenario Generation}

By the joint distribution of multiple RPPs in \eqref{cjdistribution}, scenarios can be generated to represent the uncertainties and spatial correlation of all RPPs in the system. However, with the increase of the number of RPPs, classical random sampling methods such as inverse transform sampling and Latin hypercube sampling \cite{L_sampling} become hard to be employed due to matrix size and computational limitations. Other classical sampling methods such as rejection sampling tend to have very large rejection rate for a high number of dimensions.

To this end, a reliable static renewable power scenario generation method based on Gibbs sampling \cite{Gibbs} is proposed to sample for the conditional joint distribution function of actual available power of RPPs in \eqref{cjdistribution}. Compared with directly sampling by the conditional joint distribution \cite{copula_Zhang}, Gibbs sampling converts the sampling process of joint distribution in \eqref{cjdistribution} to $J+K$ sampling processes of conditional distribution in \eqref{ccdistribution}. Namely, let $U$ be a random variable generated uniformly within $[0,1]$, then each RPP can be sampled via the inverse transform:
\begin{equation} \label{inversesampling}
w_{a,j}=F_{a,j}^{-1}(U),\quad s_{a,k}=F_{a,k}^{-1}(U)
\end{equation}
where $F_{a,j}^{-1}$ and $F_{a,k}^{-1}$ is the inverse function of $F_{a,j}$ and $F_{a,k}$, respectively.

Gibbs sampling needs a burn-in process \cite{burn_in} before it converges to the true distribution in \eqref{cjdistribution}. So we throw out $N_{b}$ (e.g. 1000) samples in the beginning the process. The detailed procedure of static scenarios generation is:
\begin{enumerate}%[noitemsep,nolistsep]
	\item Setting the number of renewable power scenarios: $N_{sc}$ (e.g. 5000), the total number of samples is $N_{sc}+N_{b}$.
	\item Setting the initial sampling values to be the forecasted power for each RPP.
	% $w_{a,{1}}^{i}$,...,$w_{a,j}^{i}$,..., $w_{a,J}^{i}$, $s_{a,{\it 1}}^{i}$,...,$s_{a,k}^{i}$,...,$s_{a,K}^{i}$, {\it i}=0...$N_{sc}+N_{b}$, ({\it i}=0 at this step). To  speed up the burn-in process, the forecast power of each RPP (i.e. $F_{re}$) are regarded as the initial sampling value.
	\item Employing inverse transform sampling in \eqref{inversesampling} in a round robin fashion for each scenario generation step (indexed by $i$):

\begin{itemize}
	\item $f(w_{a,{1}}^{i}|w_{a,2}^{i}...w_{a,J}^{i},s_{a,{1}}^{i}...s_{a,K}^{i},\mathbf{f})$
	\item $f(w_{a,{\it j}}^{i}|w_{a,{1}}^{i+1}...w_{a,{{\it j}-1}}^{i+1},w_{a,{{\it j}+1}}^{i}...w_{a,J}^{i},s_{a,{1}}^{i}...s_{a,K}^{i},\mathbf{f})$
	\item $...$
	\item $f(s_{a,{\it k}}^{i}|w_{a,{1}}^{i+1}...w_{a,J}^{i+1},s_{a,{1}}^{i+1}...s_{a,{{\it k}-1}}^{i+1},s_{a,{{\it k}+1}}^{i}...s_{a,K}^{i},\mathbf{f})$
	\item $f(s_{a,{\it K}}^{i}|w_{a,{1}}^{i+1}...w_{a,J}^{i+1},s_{a,{1}}^{i+1}...s_{a,{{\it K}-1}}^{i+1},\mathbf{f})$
\end{itemize}

	\item Repeating 3 from {\it i}=1...$N_{sc}+N_{b}$. Disregard the first $N_{b}$ scenarios and we get $N_{sc}$ renewable power scenarios.

\end{enumerate}

{An important feature of the proposed static scenario generation method is that with the increase of the number of RPPs, the computational space complexity remains same and the computational time complexity increases linearly, effectively mitigating the curse of dimensionality.}

\subsection {Dynamic Scenario Generation}
%\todo{Why is this dynamic? Also, does variability just mean correlation?}
{A dynamic scenario is a scenario that considers the variability (i.e., temporal correlation) of the output of a RPP.} The method presented in the last section can generate renewable power scenarios of conditional joint distribution (c.f. \eqref{cjdistribution}) which captures the marginal uncertainties and spatial correlation. In this section we extend it to capture the temporal correlation among the time points in a scenario, which is also of vital importance in power system operations~\cite{sce_generation_Ma,PCA,sce_generation_Pinson}.
 % which represent the uncertainties and correlations in each time interval \todo{(i.e., spatial correlation)}. However, for renewable power scenarios, variability is as same importance as uncertainties \cite{sce_generation_Ma}\cite{PCA}\cite{sce_generation_Pinson}.

To capture the variability, some new variables are introduced. Take a WPP for instance, a new random variable $Z_{a,j}^{t}$ is introduced which follows
the standard Gaussian distribution with zero mean and unit standard deviation. Since the value of CDF of $Z_{a,j}^{t}$ is uniformly distributed over [0,1], the uniform distribution $U$ in \eqref{inversesampling} can be replaced by a CDF $\Phi(Z_{a,j}^{t})$.  Given the realization of random variable $Z_{a,j}^{t}$, $w_{a,j}^{t}$ can be sampled as follows:



\begin{equation} \label{transform}
\begin{aligned}
w_{a,j}^t=F_{a,j}^{-1}(\Phi(Z_{a,j}^{t}))
\end{aligned}
\end{equation}

To consider the variability of each RPP, it is assumed that the joint distribution of $Z_{a,j}^{t}$ follows a multivariate Gaussian distribution $Z_{a,j}^{t} \sim N(\mu_{j},\Sigma_{j})$. The expectation of $\mu_{j}$ is a vector of zeros and the covariance matrix $\Sigma_{j}$ satisfies


\begin{equation} \label{matrix}
\Sigma_j=\left[
\begin{matrix}
\sigma_{1,1}^{j}&\sigma_{1,2}^{j}&\dots&\sigma_{1,{\it T}}^{j}&\\
\sigma_{2,1}^{j}&\sigma_{2,2}^{j}&\dots&\sigma_{2,{\it T}}^{j}&\\
\vdots&\vdots&\ddots&\vdots&\\
\sigma_{{\it T},1}^{j}&\sigma_{{\it T},2}^{j}&\dots&\sigma_{{\it T},{\it T}}^{j}&\\
\end{matrix}
\right]
\end{equation}

\noindent where $\sigma_{m,n}^{j}=cov(Z_{a,j}^{m},Z_{a,j}^{n})$, {\it m}, {\it n}=1,2...{\it T}, $\sigma_{{\it m}, {\it n}}^{j}$ is the covariance of $Z_{a,j}^{m}$ and $Z_{a,j}^{n}$.

The covariance structure of $\Sigma_j$ can be identified by covariance $\sigma_{m,n}^{j}$. As is done in \cite{sce_generation_Ma}\cite{sce_generation_Pinson}, an exponential covariance function is employed to model $\sigma_{m,n}^{j}$ in \eqref{matrix},

\begin{equation} \label{exponential}
\begin{aligned}
\sigma_{m,n}^{j}=\rm exp(-\frac{|{\it m}-{\it n}|}{\epsilon_{\it j}}) \quad 0 \le {\it m},  {\it n} \le {\it T}
\end{aligned}
\end{equation}

\noindent where $\epsilon_{\it j}$ is the range parameter controlling the strength of the
correlation of random variables $Z_{a,j}^{t}$ among the set of lead-time. Similar to \cite{sce_generation_Ma}, $\epsilon_{\it j}$ can be determined by comparing the distribution of renewable power variability of the generated scenarios by the indicator in \cite{sce_generation_Ma}. Here, assuming that the  range parameter $\epsilon_{\it j}$ of each RPP have been obtained, the flowchart of dynamic renewable power scenario generation method is as shown in Fig.~\ref{flowchart}.

\begin{figure}[!htb]
	\begin{center}
		\includegraphics[trim = 10 250 60 200, clip, width=1.0\columnwidth]{flowchart.eps}\\
		\caption{Flowchart of dynamic renewable power scenario generation method}\label{flowchart}
	\end{center}
\end{figure}

Before generating $N_{sc}$ scenarios, small amount of scenarios are generated to obtain the range parameter of each RPP. After all the range parameters in \eqref{matrix} are obtained, we can start the dynamic wind power scenarios generation in Fig.~\ref{flowchart}. At each time interval, they follow the conditional joint distribution in \eqref{cjdistribution} and among the time horizon, the variability is considered.

One thing that need to be noticed is that each static scenario generation process in Fig. 1 does not affect each other after the random data set is determined. Parallel computing can be employed to increase the computation efficiency to meet the real-time requirement.

In scenario-based method, the above generated scenarios should be reduced to certain number of scenarios that deemed as the most probability occur. A scenario reduction method in \cite{YishenWang} is employed in this paper for the reason that it has great efficiency compared with other methods to meet the real-time requirement.


% \subsection{Reconstruction Quality}
% \ivan{Perhaps we should evaluate the reconstruction ability compared to other autoencoders? We could focus on just BooksCorpus, create our own test set, and test reconstruction quality. If TAE/BARNEY is really good, we could introduce EM as a metric}
% \nikos{ What do you mean by EM? I worry that we may not have the space for introducing a new evaluation too (we have new architecture + pretraining framework already). How about using the Yelp dataset to compare directly with Shen et al 2020 in their own setup? }


% \subsection{Latent Properties}
% \ivan{We could also show an example of encoding two sentences, then showing the decoding of the linear interpolation of between the two? Perhaps a 2D dim-red of the embeddings?} \nikos{Sure, that'd be great. E.g. like the ones here \url{https://arxiv.org/pdf/1511.06349.pdf}}



% \subsection{Multilingual}

% \subsection{Pretraining}
% \begin{itemize}
%     \item Train two models on the exact same data for the exact same amount of training steps. To simulate the same amount of parameters, use one extra layer for the MLM approach
%     \item Model 1: 7 layer, 512 hidden size transformer encoder trained on just the MLM objective
%     \item Model 2: 6 layer, 512 hidden size BARNEY trained on MLM objective in conjunction with reconstruction objective.
%     \item Show the down-stream MNLI performance difference after certain amounts of steps (could be a plot)
%     \item Show the downstream SQuAD (MNLI?) performance difference at the end of BARNEY wihtout the conetext attention bottleneck (see if the reconstruction objective helps with better token-level representations, since all tokens are updated each step, rather than only 15\% of them in MLM)
% \end{itemize}
\section{Conclusions and Future Work}\label{section-conclusion}
In this work, we have systematically studied different key notions and results concerning anti-unification of unordered goals, i.e. sets of atoms. We have defined different anti-unification operators and we have studied several desirable characteristics for a common generalization, namely optimal cardinality (lcg), highest $\tau$-value (msg) and variable dataflow optimizations. For each case we have provided detailed worst-case time complexity results and proofs. An interesting case arises when one wants to minimize the number of generalization variables or constrain the generalization relations so as they are built on injective substitutions. In both cases, computing a relevant generalization becomes an NP-complete problem, results that we have formally established.
In addition, we have proven that an interesting abstraction -- namely $k$-swap stability which was introduced in earlier work -- can be computed in polynomially bounded time, a result that was only conjectured in  earlier work. 

Our discussion of dataflow optimization in Section~\ref{section-relation-2} essentially corresponds to a reframing of what authors of related work sometimes call the \textit{merging} operation in rule-based anti-unification approaches as in~\cite{Baumgartner2017}. Indeed, if the "store" manipulated by these approaches contains two anti-unification problems with variables generalizing the same terms, then one can "merge" the two variables to produce their most specific generalization. If the merging is exhaustive, this technique results in a generalization with as few different variables as possible. In this work we isolated dataflow optimization from that specific use case and discussed it as an anti-unification problem in its own right.

While anti-unification of goals in logic programming is not in itself a new subject, to the best of our knowledge our work is the first systematic treatment of the problem in the case where the goals are not sequences but unordered sets. Our work is motivated by the need for a practical (i.e. tractable) generalization algorithm in this context. The current work provides the theoretical basis behind these abstractions, and our concept of $k$-swap stability is a first attempt that is worth exploring in work on clone detection such as~\cite{clones}. 

Other topics for further work include adapting the $k$-swap stable abstraction from the $\preceq^\iota$ relation to dealing with the $\sqsubseteq^\iota$ relation. 
A different yet related topic in need of further research is the question about what anti-unification relation is best suited for what applications. For example, in our own work centered around clone detection in Constraint Logic Programming, anti-unification is seen as a way to measure the distance amongst predicates in order to guide successive syntactic transformations. Which generalization relation is best suited to be applied at a given moment and whether this depends on the underlying constraint context remain open questions that we plan to investigate in the future. 

%The main results of this paper are the polynomial algorithms solving specific anti-unification problems, along with several worst-case time complexity results and proofs. 

% have made efforts to extend the classical anti-unification concepts to the case where the artefacts to generalize are unordered goals. We have done this by considering different levels of atomic abstraction through different generalization relations. W





%Throughout the paper, we have introduced four generalization relations. Figure~\ref{fig-interconnexion} shows how the four relations are linked on a conceptual level. $\sqsubseteq$ is the most general relation as generalization is defined with any substitution. Restricting the definition to injective substitutions or to renamings yields more specific relations, the intersection of which is relation $\preceq^\iota$ where variables are generalized through injective renamings. 

%\begin{figure}[htbp]
%	\begin{center}
%		\begin{tikzpicture}[x=0.75pt,y=0.75pt,yscale=-1,xscale=1]
%		%uncomment if require: \path (0,300); %set diagram left start at 0, and has height of 300
%		
%		%Shape: Ellipse [id:dp330479544492589] 
%		\draw   (150.05,144.64) .. controls (150.05,72.29) and (186.09,13.64) .. (230.55,13.64) .. controls (275,13.64) and (311.05,72.29) .. (311.05,144.64) .. controls (311.05,216.99) and (275,275.64) .. (230.55,275.64) .. controls (186.09,275.64) and (150.05,216.99) .. (150.05,144.64) -- cycle ;
%		%Shape: Ellipse [id:dp3140532351606715] 
%		\draw   (165,87.62) .. controls (165,64.99) and (193.75,46.64) .. (229.22,46.64) .. controls (264.69,46.64) and (293.45,64.99) .. (293.45,87.62) .. controls (293.45,110.26) and (264.69,128.61) .. (229.22,128.61) .. controls (193.75,128.61) and (165,110.26) .. (165,87.62) -- cycle ;
%		%Shape: Ellipse [id:dp9580468020324391] 
%		\draw   (261.25,53.46) .. controls (285.77,54.29) and (304.38,90.37) .. (302.81,134.06) .. controls (301.24,177.74) and (280.08,212.49) .. (255.56,211.67) .. controls (231.03,210.85) and (212.43,174.76) .. (214,131.08) .. controls (215.57,87.39) and (236.72,52.64) .. (261.25,53.46) -- cycle ;
%		
%		% Text Node
%		\draw (172,266) node   {$\sqsubseteq $};
%		% Text Node
%		\draw (235,216) node   {$\preceq $};
%		% Text Node
%		\draw (170,125) node   {$\sqsubseteq^\iota $};
%		% Text Node
%		\draw (254,92) node   {$\preceq^\iota $};
%		\end{tikzpicture}
%	\end{center}
%	\caption{The interconnexions of four generalization relations}
%	\label{fig-interconnexion}
%\end{figure}

%Figure~\ref{fig-interconnexion} shows how the four relations are linked on a conceptual level. When needed in concrete applications, the right generalization operator (or an abstraction) should be used; this of course depends on whether or not the atomic structure should be generalized and the variable dataflow preserved. 

%Future work will focus on the use of such generalization operators in the purpose of applying synctatic transformations on predicates in such a way that the structural distance between them decreases; such a synctatic distance can be evaluated over the most specific generalization of the predicates under scrutiny.
\section*{Acknowledgment}
\vspace{-1pt}
This work was supported by the Ministry of Science and Technology of China, National Key Research and Development Program (2016YFB1000703), NSF of China (61732014, 61332006, 61472321, 61502390 and 61672432).


\bibliographystyle{IEEEtran}
\bibliography{mybibfile}

\end{document}
