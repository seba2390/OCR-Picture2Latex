\section{introduction}
%\vspace{-0.15cm}
  Entity resolution (ER) usually refers to identifying the relational records that correspond to the same real-world entity in a dataset. Extensively studied in the literature \cite{christen2012data}, ER can be performed based on rules \cite{fan2009reasoning, li2015rule, singh2017generating}, probabilistic theory \cite{fellegi1969theory} or machine learning \cite{sarawagi2002interactive, kouki2017collective, arasu2010active, bellare2012active}. Unfortunately, most of the existing techniques fall short of effective mechanisms for quality control. As a result, they cannot enforce quality guarantees. Even though active learning based approaches \cite{arasu2010active, bellare2012active} can optimize recall while ensuring a user-specified precision level, it is usually desirable in practice that an ER result can have more comprehensive quality guarantees specified at both precision and recall fronts.

  To flexibly impose quality guarantees, we propose a novel human and machine cooperation framework, HUMO, for ER. Its primary idea is to divide instance pairs in an ER workload into easy ones, which can be automatically labeled by a machine with high accuracy; and more challenging ones, which require manual verification. HUMO is, to some extent, inspired by the success of human intervention in problem solving as demonstrated by numerous crowdsourcing applications \cite{li2016crowdsourced}. However, existing crowdsourcing solutions for ER \cite{wang2012crowder, whang2013question, vesdapunt2014crowdsourcing, gokhale2014corleone, mozafari2014scaling, wang2015crowd, chai2016cost} mainly focused on how to make humans more effective and efficient on a given workload. Targeting the challenge of quality control, HUMO instead investigates the problem of how to divide an ER workload between the human and the machine such that a given quality requirement can be met.

  HUMO is motivated by the observation that pure machine-based solutions usually struggle in ensuring desired quality guarantees for tasks as challenging as entity resolution. Even though humans usually perform better than machines in terms of quality on such tasks, human labor is much more expensive than machine computation. Therefore, HUMO has been designed with the purpose of minimizing human cost given a particular quality requirement. Note that a prototype system of HUMO has been demonstrated in~\cite{chen2017humo}. The major contributions of this technical paper can be summarized as follows:

\begin{enumerate}
\item We propose a human and machine cooperation framework, HUMO, for entity resolution. The attractive property of HUMO is that it enables an effective mechanism for comprehensive quality control at both precision and recall fronts;
\item We introduce the optimization problem of HUMO, i.e. minimizing human cost given a quality requirement, and present three optimization approaches: a conservative baseline one purely based on the monotonicity assumption of precision, a more aggressive one based on sampling, and a hybrid one that can take advantage of the strengths of both approaches;
\item We validate the efficacy of HUMO by extensive experiments on both real and synthetic datasets. Our empirical evaluation shows that HUMO can achieve high-quality results with reasonable ROI in terms of human cost, and it performs considerably better than the state-of-the-art alternatives in quality control. On minimizing human cost, the hybrid approach performs better than both the baseline and sampling-based approaches.
\end{enumerate}

  The rest of this paper is organized as follows: Section~\ref{sec:related} reviews more related work. Section~\ref{sec:setting} defines the task.  Section~\ref{sec:framework} presents the framework. Section~\ref{sec:conservative} describes the baseline approach based on the monotonicity assumption of precision. Section~\ref{sec:aggressive} describes the sampling-based approach. Section~\ref{sec:hybrid} describes the hybrid approach. Section~\ref{sec:experiment} presents our empirical evaluation results.  Finally, Section~\ref{sec:conclusion} concludes this paper with some thoughts on future works.
