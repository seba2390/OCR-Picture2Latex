\section{Conclusion and Future Work} \label{sec:conclusion}
   In this paper, we have proposed a human and machine cooperation framework, HUMO, for entity resolution. It represents a new paradigm that enables a flexible mechanism for comprehensive quality control at both precision and recall levels. Our extensive experiments on real and synthetic datasets have also validated its efficacy.

   Future work can be pursued in two directions. Firstly, for large datasets, crowdsourcing may be the only feasible solution for human workload. It is interesting to integrate HUMO into existing crowdsourcing platforms. On crowdsourcing platforms, monetary cost may be a more appropriate metric of human cost than the number of manually inspected pairs used in this paper. Secondly, as a general paradigm, HUMO can be potentially applied to other challenging classification tasks requiring high quality guarantees (e.g. financial fraud detection \cite{ngai2011application} and malware detection \cite{ye2017survey}). It is interesting to investigate its efficacy on them in future work.
