\section{Hybrid Approach}\label{sec:hybrid}

  The baseline approach usually overestimates the match proportion of $D_-$ while underestimating that of $D_+$. The sampling-based approach can alleviate both drawbacks to a large extent by directly sampling $D_-$ and $D_+$. However, it still has to consider confidence margins in the estimations of $D_-$ and $D_+$. Furthermore, it usually cannot afford to sample all the subsets in $D_-$ and $D_+$ due to prohibitive sampling cost. Generally, less samples would result in larger error margins. Therefore, there is no guarantee that a sampling-based estimation would always be better than the corresponding baseline one. As we show in Section~\ref{sec:experiment}, their relative performance actually depends on the characteristics of the given ER workload. This observation motivates us to propose a hybrid approach, which can take advantage of both estimations and use the better of both worlds in the process of bound computation.

  The hybrid approach begins with a HUMO solution of the partial-sampling approach. We denote the initial solution by $S_0$ and its lower and upper bounds of $D_H$ by $D_i$ and $D_j$ respectively. It searches for a better solution than $S_0$ by incrementally redefining $D_H$'s bounds using the better between the baseline and sampling-based estimates. Initially, it sets $D_H$ to be the single median subset of $D_i$ and $D_j$, $D_{\frac{i+j}{2}}$. Similar to the baseline approach, it alternately extends $D_H$'s upper and lower bounds until both precision and recall requirements are met. However, on reasoning about the match proportions of $D_-$ and $D_+$, instead of being purely based on the monotonicity of precision, it uses the better of both estimates. It alternately moves the upper bound from $D_u$ to $D_{u+1}$ and the lower bound from $D_l$ to $D_{l-1}$. After each movement of the upper bound, it checks whether the current solution satisfies the precision requirement. Similarly, after each movement of the lower bound, it checks whether the current solution satisfies the recall requirement. Note that the new range of $D_H$ can not exceed the range of $[D_i, D_j]$ in the initial solution $S_0$. Therefore, the resulting HUMO solution of the hybrid approach is at least as good as $S_0$. The details of the hybrid search process are omitted here due to space limits, but can be found in our technical report \cite{chen2017humoreport}.

  The worst-case computational complexity of the hybrid solution is the same as that of the partial-sampling solution, bounded by ${\bf O}(n+m^3+m\cdot k^2+k^4)$. Its effectiveness in ensuring quality guarantees depends on both the monotonicity assumption of precision and the accuracy of Gaussian approximation. As shown by our empirical evaluation in Section~\ref{sec:experiment}, the hybrid solution is highly effective in ensuring quality guarantees for HUMO.
