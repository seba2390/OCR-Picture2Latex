\begin{figure*}[bht]
\begin{center}
\begin{tabular}{c}
\includegraphics[width=5in]{aging1.png} \\
(a) \\
\includegraphics[width=5in]{aging2.png} \\
(b) 
\end{tabular}
\end{center}
\caption{\label{fig:xstream-aging} \XSCC aging mode. (a) A token notifies processors
that they must apply an aging predicate.  All surviving edges become unresolved (gray).
(b) Normal operation continues uninterrupted for new edges, while
unresolved edges circulate back to be incorporated into a new data structure. Each processor
in turn becomes the loading processor ($p_L$) and recycles its unresolved edges.}
\end{figure*}

\begin{figure*}[thb]
\begin{center}
\includegraphics[width=5in]{aging_properties.png} \\
\end{center}
\caption{\label{fig:xstream-aging-props} \XSCC aging nomenclature. both
primary
and payload edges are called \emph{resolved} when they have been
classified as tree or non-tree. Duplicate detection leaves empty
slots, and processors \emph{ingest} and \emph{emit} bundles of edges.}
\Jon{$p_H$ label, fix last ingest/emit example (my Office is currently broken;
will fix when Office fixed.}
\end{figure*}

\XSCC handles infinite streams via a bulk deletion operation we call
an \emph{aging} event.
Our model is thus unlike most previous work, in that we do not expect or
support individual
edge deletions embedded within the stream.  Rather, we expect the 
system administrator
to schedule bulk deletions to ensure that the oldest and/or 
least useful data are deleted in a timely manner.

To begin aging, the system administrator introduces an aging 
predicate (for example, a timestamp threshold) into the input stream.
The predicate
propagates through the system, and each processor suspends query 
processing upon receipt.  However, a new stream edge might arrive in the 
\XStream tick immediately after the aging predicate arrives from the 
I/O processor.
This and all other new edges must be ingested and processed without exception.  
Thus, the connectivity data structures must be rebuilt concurrently with
normal stream edge processing. When this rebuild is complete, queries
are accepted once again.

We now describe how \XSCC processes the aging predicate and prove correctness.
In Section~\ref{sec:aging-conditions} we provide theoretical guarantees relating 
the fraction of
system capacity used after the deletion predicate has been applied, the 
bandwidth expansion
factor, the proportion of query downtime that is tolerable, and the 
expected stream edge duplication rate.

\subsection{Aging process}
\label{sec:aging-process}
Figure~\ref{fig:xstream-aging} illustrates the aging process. An aging token arrives
with an edge-deletion predicate.  As the token propagates downstream, all
edges are reclassified to be \emph{untested}. If an edge later passes the
aging predicate it becomes \emph{unresolved} since the old connectivity
structure is no longer valid.
Immediately after the aging token is received
by the head processor, new stream edges may continue to arrive. These are
processed as normal, starting from empty data structures, so we maintain 
Invariants~\ref{inv:normal-1} and \ref{inv:normal-2} even during aging.

%\Jon{ATTN Cindy:  this paragraph is new text attempting to explain a subtle 
%point; please vet.}
Conceptually, upon receipt of aging notification the deletion of all
edges that fail the aging predicate and
reclassification of all surviving edges to \emph{unresolved} is instantaneous.
However, in
practice each processor takes $\frac{s}{k-1}$ \XStream ticks to execute
a ``testing phase'' that applies the aging predicate to each stored
edge.  Without careful attention to detail,
implementers could allow a case in which there is no space yet for a new
stream edge.  In Section~\ref{sec:pseudocode} we give exact
specifications for a correct procedure that ensures no stream edge is
dropped, even in the \XStream tick immediately after aging notification.
If the testing phase has yet not identified empty space for a new stream
edge, then one of the unresolved edges can be sent downstream
in a primary slot.  This is an example of the jeopardy condition described
later in this section, corresponding to Line 21 in Algorithm~\ref{algo:pseudocode-driver}.

In addition to normal processing of new stream edges, \XSCC recycles all
unresolved edges that survive the aging predicate.
As depicted in Figure~\ref{fig:xstream-aging}, we introduce a new designation $p_L$
for a \emph{loading processor} or ``loader.''  Upon each activation to process
a stream edge, the loader packs unresolved edges into any available payload
slots in the output \bundlens.  Such bundles propagate around the ring. After
a bundle
reaches the head processor $p_H$, its payload edges are processed as if they
were new edges.  When the loader has emitted all of its unresolved edges, it
passes the loader token downstream to its successor.  Aging is complete when
the last processor with any unresolved edges has completed its loader duties.

\begin{figure*}[thb]
\begin{center}
\includegraphics[width=5in]{jeopardy.png} \\
\end{center}
\caption{\label{fig:jeopardy} The \XSCC aging ``jeopardy
condition.'' Processor $p_B$ currently bears both building and loading
responsibilities, is completely full of edges, and must ingest a \bundle
with no empty slots. It ingests $k$ slots, finds no duplicates,
and must emit $k$ slots. Therefore an unresolved ``jeopardy edge''
must be emitted in the primary slot.  If it doesn't settle in a 
processor before leaving the tail, the system is completely full and raises a
FAIL condition. Note that in this illustration, $p_S$ will be able to store
the jeopardy edge, so the jeopardy condition will soon be mitigated.}
\end{figure*}

The complete \XSCC protocols defined in Section~\ref{sec:pseudocode} 
enforce the previous invariants at all times, as well as the following
invariant during aging.  
% CAP: Not sure this is adding much at this point.  Put it back later (probably changed) if it makes sense
%It formalizes the idea that the loader never leaves behind any unresolved edges.
% CAP: here is the old invariant: During aging, let $p_L$ be the loading processor. Then $p_i$ is completely full of resolved edges for $0 \le i < L$ at all times, and there are no resolved edges for $i > L$.
% \Cindy{I don't think the first part is true.  Processors are only packed up to $p_S$, and the loader can be downstream of $p_S$.  Fix this.  Maybe no unresolved edges upstream}
\begin{invariant}
During aging, let $p_L$ be the loading processor and $p_B$ be the building processor. 
Then $B \le L$.  Also, processor $p_i$ has no unresolved edges for $i < L$
% Then $p_i$ is completely full of resolved edges for $0 \le i < L$ at all times,
and $p_i$ has no resolved edges for $i > L$. \label{inv:aging-1}
\end{invariant}

The combination of all invariants ensures that all processors from the head to the builder are running \XSCC in normal mode on all incoming (and recycled) edges. All resolved edges are packed to the front (upstream).  When all edges have been recycled and aging ends, the layout of edges returns to normal mode.

%\Jon{(updated by CAP) This begs a statement that the processors between 0 and B are still doing XS normal mode (on the incoming edges union the payload edges.}

%\Jon{This next invariant is wrong by our current pseudocode.  Don't think
%it was used in the first place.}
%Our protocols also enforce the following relationship between
%$p_B$, $p_L$, and $p_S$.  We must never build connectivity structure
%downstream of the loader.
%\begin{invariant}
%During aging, let $p_B$ be the building processor, $p_L$ be the loading
%processor, and $p_S$ be the first processor with empty space.  Then
%%$B \le L \le S$. \label{inv:aging-2}
%\end{invariant}


Figure~\ref{fig:xstream-aging-props} puts the
nomenclature of our arguments into context. An edge becomes \emph{resolved}
when an XS-CC processor determines that it is a tree or non-tree edge,
regardless of whether it is a new stream edge in a primary slot or an
unresolved edge being recycled as payload.  Processors \emph{ingest} and
\emph{emit} bundles of edges.  With one exception we will discuss presently,
the complexity of processing input bundles and packing edges into
output bundles prior to emission is relegated to Section~\ref{sec:pseudocode}.

Aging is generally a straightforward process in which the loader token
steadily advances from $p_H$ to $p_T$, unresolved edges are recycled and
resolved, and the XS-CC connectivity structure is rebuilt.  
When builder and loader designations coincide in the same processor, 
that processor packs unresolved edges for emission first, then non-tree 
edges.  Edge bundles containing transit edges have one primary slot and $k-1$ 
payload slots, where
$k$ is the bandwidth expansion factor.  New stream edges reside in primary
slots, and unresolved edges circulate in payload slots until they are
resolved.  Payload edges continue in their assigned slots until allowed to 
settle, per the invariants.

There is a single exception to this last point, illustrated in
Figure~\ref{fig:jeopardy}. We call this the \emph{jeopardy condition}
and use it to specify exactly when the system fills
to capacity during aging (indicating that the aging command was too late
or did not remove enough edges).
In the jeopardy condition, processor $p_B$ is also the loader,
is already storing edges to its capacity $s$, and must ingest an edge \bundle
with no empty slots. It ingests $k$ slots, finds no duplicates,
and by conservation of space, must emit $k$ slots. 
Therefore an unresolved edge $e_j$ must reside in the primary slot.
If $e_j$ cannot be offloaded before exiting the tail,
the system is completely full and raises a FAIL condition.

The above discussion and the more detailed discussion in Section~\ref{sec:pseudocode} show the following property necessary for proving aging correctness holds:
\begin{property}
\label{prop:all-recycled}
During aging, every surviving edge is incorporated into the new 
connected components data structure either by $p_H$ directly or 
by traveling back to $p_H$ as a payload edge.
\end{property}

\subsection{Aging correctness}
We now argue correctness of the aging process.
We say that any implementation of \XSCC aging that maintains
Invariants~\ref{inv:normal-1}, \ref{inv:normal-2} and \ref{inv:aging-1} and property~\ref{prop:all-recycled} is \emph{compliant}.
A compliant aging process ensures that during aging there is a monotonic ordering of
edges in the system, with tree (red) edges never allowed downstream of
non-tree (blue) edges, and unresolved (gray) edges never allowed upstream
of non-tree edges. In the argument below, we slightly abuse notation by
using the graph $G_t$ in place of its edge set $E(G_t)$.

\begin{theorem}
Suppose a compliant XS-CC implementation receives an aging command at tick $t$ and reauthorizes queries at tick $t'$.  Let $F$ be the set of edges in $G_t$ that fail the aging predicate and let $E_{t\rightarrow t'}$ be the set of edges that arrive between time $t$ and $t'$. Then at tick $t'$, the x-stream system stores graph $G_{t'} = G_t - F \cup E_{t\rightarrow t'}$, can properly answer queries and stores each edge in $G_{t'}$ exactly once.
\label{thm:aging-correctness}
\end{theorem}
\begin{proof}
\Cindy{consistency in graph vs. edge set}\Jon{mitigated by text above. Cindy?}
As the aging command that arrived at time $t$ propagates through the
processors, they reclassify all current edges to ``untested'' as
described in Section~\ref{sec:aging-process}, forgetting the current
union-find structure. Thus the system starts processing a new graph
from an empty state at time $t+1$. As described in
Section~\ref{sec:aging-process}, processors delete all edges in $F$,
those who fail the predicate.  Each remaining edge in $G_t - F$ is
eventually loaded into payload slot by
Property~\ref{prop:all-recycled}, and processed at the head as
arriving edges. Invariants~\ref{inv:normal-1} and~\ref{inv:normal-2}
hold with the newly-created data structures thoughout aging.
Invariant~\ref{inv:aging-1} ensures that all unresolved edges are in
the builder processor or downstream. Those in the builder do not
affect the connectivity computation and are eventually moved
downstream. Thus, all edges arriving from outside the system are
processed as in normal mode and all edges arriving in the payload
slots are processed as in normal mode (other than traveling in a
payload slot).  Thus at time $t'$ when the tail processor passes the
loading token out of the system and enables queries, the \XStream
system stores exactly the edges in $G_t - F \cup E_{t\rightarrow t'}$,
with duplicates appropriately removed.  This is the graph the system
is required to hold by definition of aging and the requirement that it
drop no incoming edges during aging. The edges are processed into the
data structures with arbitrary mixing of new edges and recycled
(surviving) edges.  By Observation~\ref{obs:wstreamreqs}, and the equivalence of
\DFR and \XSCC in normal mode, the ordering of
the input edges does not matter for future query correctness.
By Theorem~\ref{thm:query-correctness}, the \XStream system will now correctly
answer querries on the graph starting at time $t'$.

During aging, some edges may be stored up to twice.  If a duplicate of a suriving edge $e$
enters the system before edge $e$ circulates back to the head processor, then edge $e$ is
stored both in the new data structure as a tree or non-tree edge and as an unresolved edge.
However, when edge $e$ is eventually recycled, it will be recognized as a duplicate and not
stored again.  By Theorem~\ref{thm:non-dup}, any edge that enters from outside the system during aging will be stored at most once in the new data structure.
\qed
\end{proof}

\iffalse
\Cindy{OLD:}
Since Invariants~\ref{inv:normal-1} and \ref{inv:normal-2} still hold
during aging, we can define a corollary to Lemma~\ref{lemma:B-stream}
to argue about the partially rebuilt
connectivity structure.

Suppose that a compliant aging process begins at \XStream tick $t$, 
and consider tick $t'>t$ during
the aging process. Let $E'$ be the set of stream edges that have arrived during
$\{t+1,\ldots,t'\}$ and $U$ be the set of unresolved edges that remain. We call
\[G^r_{t'} = G_t - U \cup E'\] the \emph{resolved graph at time $t'$}.  This
contains all active edges that either entered the system since time $t$ or
become resolved after circulating as payload.
\Jon{account for edges that got deleted by the predicate.}
\Jon{define this differently: the key is the set of payload edges that have
entered the head}

\begin{corollary}
\label{corollary:partial}
(corollary to Lemma~\ref{lemma:wx-correspondence}) The equivalence
between \DFR and \XSCC expressed in Lemma~\ref{lemma:wx-correspondence}
holds for $G^r_{t'}$, the resolved graph at aging time $t'$.
\end{corollary}
\begin{proof}
The argument is exactly the same as that of Lemma~\ref{lemma:wx-correspondence},
replacing $G_t$ with $G^r_{t'}$.
\end{proof}

Thus, as aging proceeds, the connected component information
stored by \XSCC regarding the resolved graph
aligns with that W-Stream would have computed had it been given $G^r_{t'}$
as a stream.  All that remains for the correctness argument
is to define the conditions under which
a compliant aging process completes. We address this in Section~\ref{sec:aging-conditions}. If aging does complete at some time $t''$,
then $G_{t''} = G^r_{t''}$ and Corollary~\ref{corollary:partial} implies
that the system can leave its aging mode and resume query processing.
\fi

\section{Conditions for successful aging}
\label{sec:aging-conditions}
In this section, we define the conditions under which
a compliant aging process completes before the system fails for lack of space.
We consider properties of the system, properties of the input stream, and user
preferences.


\begin{tcolorbox}
\begin{definition}
\ \ \\
We define the following as tradeoff parameters associated with infinite runs of \XSCCns.
\label{def:infinite-run-params}
\begin{description}
\item[{\bf c:}] fraction of the total system storage occupied by
edges that survive the aging predicate
\item[{\bf d:}] percentage of \XStream ticks that the system is unavailable for queries due to aging
\item[{\bf u:}] estimate of the percentage of incoming stream edges that will be unique
\item[{\bf k:}] the bandwidth expansion factor: the size of an \XStream bundle (a set of edge-sized slots that circulates in the ring)
\item[{\bf p:}] number of \XStream processors
\item[{\bf S:}] aggregate storage available in the system
\item[{\bf s:}] storage per processor in a homogeneous system ($s = S/p$)
\end{description}
\end{definition}
\end{tcolorbox}

%\Jon{New lemma: must age before you're full.  Precondition: must be at least
%$P+1$ empty slots in the whole \XStream system at the time you invoke aging.}
Aging must be initiated before the system becomes too full,
or else jeopardy edges will lead to a FAIL condition.  We quantify this
 decision point as follows.

\begin{lemma} \label{lemma:aging-lead-time}
In the worst case, there must be at least 
$$\frac{cS}{p(k-1)} + \frac{3}{2}p$$
open space in the system when an aging command is issued to be guaranteed sufficient space for aging, where
$c$, $S$, $p$ and $k$ are given in Definition~\ref{def:infinite-run-params}.  
\end{lemma}
\begin{proof}
When the aging command arrives, there could be $p$ edges in transit
that all must be stored. Because iteration over the untested list
doesn't imply
any specific ordering, in the worst case, when processors test the
edges against the predicate, all $cS$ surviving edges are tested
before an edge fails the predicate. This gives the latest time when
space becomes free for new edges. When a processor receives the aging
command, it processes $(k-1)$ untested edges each tick until it has
tested all its edges. In the $p$ ticks required for the aging command
to reach the tail, the head tests $p(k-1)$ edges, the second processor
tests $(p-1)(k-1)$ edges and so on, while the tail tests $(k-1)$
edges.  Thus in the first $p$ ticks after aging starts, the system tests
$\frac{p(1+p)(k-1)}{2}$ edges.  After that, the system tests $p(k-1)$
edges per tick. If the system tested $p(k-1)$, every tick, it would
require  $\frac{cS}{p(k-1)}$ ticks. But the first $p$ ticks are only half
as efficient, so we require an extra $p/2$ ticks. Thus the total
number of ticks before the system is guaranteed to remove an edge that
fails the predicate is at most $\frac{cS}{p(k-1)} + \frac{3}{2}p$.
\qed
\end{proof}

If the system is homogeneous, the empty space expression in Lemma~\ref{lemma:aging-lead-time} becomes $cs/(k-1) + \frac{3}{2}p$. For example, for a homogeneous system, assuming that $s \gg p$, if $c = 1/2$ and $k = 5$, then one should start aging while $1/8$ of the last processor is still empty. The last processor can issue a warning when it starts to fill and again closer to the deadline given $c$ and $k$.

\begin{theorem} \label{thm:infinite-runs}
In any \XSCC aging process initiated in accordance with Lemma~\ref{lemma:aging-lead-time}, if
$c$, $d$, $u$, $k$, $p$, $S$, and $s$ from Definition~\ref{def:infinite-run-params}
are set such that 
             \[k \ge   1 + \frac{(cp + 1) u}{dp (1-c)},\]
then the aging process will finish before the system storage fills completely.
\end{theorem}

\begin{proof}
After the aging token arrives, the head processor must apply the aging predicate
to its $s$ edges.  It processes $k-1$ per tick, as described in Section~\ref{sec:aging-process}.
Thus, after $\frac{s}{k-1}$ ticks, the head processor passes the loader token to the
second processor. By that time, all other processors have applied the predicate to
all of their edges and have a list of surviving edges.
Once unresolved edges begin circulating from
the loader (ignoring additive latencies such as the time until the first
payload reaches the head processor since $P \ll s$), $k-1$ edges re-enter the system to be
resolved at each
tick.  Since $cS$ unresolved edges survived the aging predicate, in the worst case
(when they are all in the second processor or later) it will take
$\frac{cS}{k-1}$ ticks to complete aging. During this time, every $\frac{1}{u}$
ticks yields a new, non-duplicate stream edge. Thus, the system will fill
to capacity in $\frac{1}{u} (1-c)S$ ticks.
The proportion $d$ constrains these two tick counts as follows:
\[ d \frac{1}{u} (1-c) S \ge \frac{cS + s}{k-1} = (cS / (k-1)) * (p+1)/p.\]
Simplifying this inequality and solving for $k$ (with Wolfram Alpha~\cite{wa},
for example)
yields the result. \qed
\end{proof}

The parameters $c$ and $d$ are user preferences, but $k$ is dictated by 
computer architecture.  Reasonable values of $k$ for current architectures are
$5-10$, but emerging data flow architectures may provide upward flexibility.
The parameter $u$ must be estimated by the user based on knowledge of the
input streams that s/he will feed to \XSCCns.


We can now state the central result of this paper. 
%\Cindy{TODO: wrangle this into usable shape}
%We assume that the graph-edge stream is {\em effectively infinite}. This means that as long as the algorithm is running, it must always be prepared for the arrival of another edge. At any time, the system has seen only a finite set of edges and need store only a finite graph representation.  However this graph can be arbitrarily large and may eventually exceed any particular finite storage. 
%\Jon{I think less is more in this case.}

\begin{theorem}
\XSCC can process an effectively infinite stream of graph edges without failing,
answering
connectivity queries correctly when in normal mode, as long as
the system is configured in accordance with Theorem~\ref{thm:infinite-runs}
and aging is started with sufficient space available obeying
Lemma~\ref{lemma:aging-lead-time}.
\end{theorem}
\begin{proof}
Assuming that the proportion of \XStream ticks that yield a new,
non-duplicate stream edge is $u$,
an empty system will fill and fail in $\frac{1}{u}S$ ticks. Compliant
aging in accordance with Lemma~\ref{lemma:aging-lead-time} ensures that
aging will always complete before the system fills.  During normal
mode operation, Lemma~\ref{lemma:B-stream} and 
Theorem~\ref{thm:query-correctness} ensure, respectively, that 
accurate connected
component information is stored, and that connectivity queries are
answered correctly.  As long as the
system adminstrator adheres to such a schedule, \XSCC operation
can continue through an arbitrary number of aging events. \qed
\end{proof}
We note that queries yielding system capacity usage are TODO constant-size. In
the case of a simple aging predicate such as a timestamp threshold,
given a target proportion $c$ of edges that survive an aging
event, the \XStream system administrator could use an automated process to
trigger the aging process.
