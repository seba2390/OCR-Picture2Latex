% !TEX root = XStreamFull.tex

During \XSCC normal mode, at any \XStream tick exactly one
processor is in a state of \emph{building}.
This building processor, $p_B$, accepts new edges, maintains connected
components with a \uf data structure, and stores spanning tree edges.
\XSCC normal mode maintains two key invariants, stated below and
illustrated in Figure~\ref{fig:xstream-normal}.
\begin{invariant}
Let $p_B$ be the building processor.  Then $p_i$ is completely full of
spanning tree edges for $0 \le i < B$ at all times, and $p_i$ has no
spanning tree edges for $i > B$. \label{inv:normal-1}
\end{invariant}

When $p_B$ fills with tree edges, a building ``token'' is passed downstream
to $p_B$'s successor, which assumes building responsibilities.  Thus, \XSCC
maintains a spanning forest of the input graph, packed into prefix processors
$\{p_0, \ldots, p_B\}$.
The \XSCC normal mode protocols maintain Invariant~\ref{inv:normal-1} and one other:

\begin{invariant}
Let $p_S$ be the first processor with any empty space.  Then $p_i$ is
completely full of edges for $0 \le i < S$ at all times, and $p_i$ has no
tree or non-tree edges for $i > S$. \label{inv:normal-2}
\end{invariant}

%\Jon{Blue invariant is used for arguing no duplicate storage.  We don't have
%a corresponding lemma yet, but should.}

\setcounter{algorithm}{0}  % -1
\begin{algorithm*}
\caption{This diagnostic routine is helpful for understanding correctness;
it would never be called in practice.  \label{algo:dump} 
This assumes the \WStream convention of choosing a vertex representative to name each 
supernode.
}
\XSCC diagnostic: dump connected components \\
\makealgtitle
\begin{algorithmic}[1]
\LeftComment{Precondition: the input stream has stopped. This never happens during normal operation.}
\LeftComment{Description: Processor $p_T$ emits correct finite stream connected components output.}
\Procedure{DumpComponentLabels}{$p_i$}
\LeftComment{Relabel all \Call{DumpComponentLabels}{} output from upstream}
          \While {\Call{Receive}{$u$, $R_{i-1}(u)$}}\label{lab:startRelabel}
\State        \Call{Emit}{$u$, $R_{i}(u)$}\label{lab:endRelabel}
          \EndWhile
\LeftComment{Now emit each union-find relationship}
          \For {$b_x : \nodeloc(b_x) = p_i$}   \Comment{$\nodeloc$ is the supernode relationship; see Definition~\ref{def:alpha}}\label{lab:startUFdump}
\State        \Call{Emit}{$b_x$, $R_{i}(b_x)$} \label{lab:endUFdump}
          \EndFor
\EndProcedure
\end{algorithmic}
\end{algorithm*}

Invariants ~\ref{inv:normal-1} and \ref{inv:normal-2} are illustrated in
Figure~\ref{fig:xstream-normal}, with sets of spanning tree edges represented
in red, and sets of non-tree edges represented in blue. In normal mode 
operation, single edges arrive at each X-Stream tick and propagate downstream
to the builder, being relabeled along the way. They settle into $p_B$ if 
they are
found to be tree edges, and into $p_S$ otherwise.  The figure
shows the system at \XStream tick $t+4$.  In this notional example, the
edge that arrived in the previous tick has passed through the head processor
$p_H$, but has not yet been resolved as ``tree'' or ``non-tree.''  Edge
$e_{t+2}$ has passed through two processors, and relabeling of the endpoints
has identified it as a non-tree edge. The basic protocol is thus quite simple;
the subtleties of \XSCC normal operation arise in maintaining the invariants. For example, the
builder may need to jettison non-tree edges downstream to make room for new tree edges.
We provide full detail in Section~\ref{sec:pseudocode}.

\subsection{\XSCC normal mode correctness}
We now show that Invariant~\ref{inv:normal-1} and
\XSCC relabeling implies an exact correspondence between
the connectivity structures computed by \XSCC and \DFRns.  
%\Cindy{Maybe delete this next sentence? We don't talk about maintaining the invariants until the pseudocode sections and the discussion of the bulk delection is in the next section.  This is the lead in to what's in this subsection.} 
%After that
%we will detail the mainentance of the invariants and our bulk deletion
%operation.

Algorithm~\ref{algo:dump} is a diagnostic routine intended to test
implementations of \XSCCns. At
\XStream time $t$, a call to this routine streams out the connected
components of the active graph $G_t$ as a stream of (vertex, label)
pairs.  Although we could correctly stream these $|V_t|$ vertex pairs
out even as new edges change the connected components (see
Section~\ref{sec:non-constant} for additional algorithm steps)\Jon{TODO: there
is nothing yet about this in Section~\ref{sec:non-constant}}, this version is more illustrative.
%is strictly for debugging. It's not, really - it's for correctness args too. 
For simplicity we assume that the input stream pauses at time $t$.

% JWB: using ``dump components'' rather than dump-components since LaTeX
% is struggling to handle line breaks with the latter
When head processor $p_0$ receives the ``dump components'' command in a primary slot, it copies the command to the primary slot of the \bundle it will emit. 
Then, in Lines~\ref{lab:startUFdump}-\ref{lab:endUFdump} of Algorithm~\ref{algo:dump}, processor $p_0$ fills the remaining $k-1$ payload slots with relationships from its union-find structure, the way $\DFR$ outputs union-find information into the $B$ streams. Specifically, if $b_x$ is the name of a building block encapsulated by local component $b_y$ in $p_0$ (i.e. $\beta(b_x) = p_0$), then processor $p_0$ outputs a pair $(b_x, b_y = R_i(b_x))$, where $R_i(b_x)$ is the encapsulating supernode, the result of processor $p_0$ relabeling block $b_x$.  Processor $p_0$ then fills the payload slots of subsequent bundles until it has output all of its union-find relationships.

For downstream processor $p_i$ ($i > 0$), the \bundle that has the ``dump components'' command has $(u,R_{i-1}(u))$ pairs in the payload slots. In Lines~\ref{lab:startRelabel}-\ref{lab:endRelabel} of Algorithm~\ref{algo:dump}, processor $p_i$ relabels any block names $u$ that have been encapsulated by a new supernode in $p_i$. Otherwise $R_i(u) = R_{i-1}(u)$. After all relationships from upstream have arrived, there is empty payload for $p_i$ to output its union-find information as described above.

We now argue the correctness of \XSCC based on the correctness of
W-Stream. Let $D_i$ denote the output of processor $p_i$ from this diagnostic.  For the formal arguments, we require the following definitions:

\begin{definition}
During a union operation joining sets with representatives $b_x$ and $b_y$,
the \emph{supernode naming function} is $\eta: {\cal B} \times {\cal B} \rightarrow {\cal B}$ such
that $\eta(b_x,b_y)$ decides
whether $b_x$ or $b_y$ becomes the new set representative.
\end{definition}
For example, we
might choose a supernode naming function $\eta(b_x,b_y) = \min(b_x,b_y)$. This
is the function used in Figure~\ref{fig:wstream-example}.

\begin{definition}
\label{def:alg-with-params}
$\DFRns(s, \eta, A)$ is an implementation of \DFR with
processor union-find capacity $s$ and supernode naming function
$\eta$ run on input stream $A$. 
% can't seem to use macros here; it interferes with line wrapping
%$\XSCCns(s,\eta,A)$ is
XS-CC$(s,\eta,A)$ is
defined similarly for XS-CC, with each processor's union-find
capacity set to $s$.
\end{definition}

A \emph{resolved} edge is one that has been classified as ``tree''
or ``non-tree.''\ Stream edges arrive
in an \emph{unresolved} state. 
In \DFRns, the stream $A_i$ written from pass $i$ contains only those edges that resolve to ``tree'' edges
%\Jon{In previous text we had said ``unresolved'' here, but DFR emits only resolved edges (and drops non-tree edges (i.e., the resolution decision has been made)).}
(they connect supernodes in the current version of the contracted graph).  \DFR deletes as ``non-tree'' any edge that it determines to be contained inside a supernode.  
In contrast, \XSCC must retain all non-duplicate edges,
even after resolution.  In particular, non-tree edges
must be retained in case they are needed to reconnect pieces of the 
graph after bulk deletion. 
\XSCC removes duplicate edges from the stream after updating their 
timestamps.

By Invariant~\ref{inv:normal-1}, all unique non-tree edges (those contracted inside a supernode) are stored at the end of the \XSCC data structure in spare space in the builder, or in processors downstream of the builder. These downstream
processors have no
union-find structure. The following lemma ignores known non-tree edges.

\begin{lemma}
The stream of unresolved edges sent from processor $p_i$ to $p_{i+1}$ in 
$\XSCCns(s,\eta,A_0)$ is exactly $A_i$ from
$\DFRns(s,\eta,A_0)$.
\label{lem:A-stream}
\end{lemma}

\begin{proof}
We prove this lemma by induction.  For the base case, the first pass of \DFR and the first processor of \XSCC receive the same same finite stream of unresolved edges from the outside (logical processor $p_{-1}$), namely the input stream of edges $A_0$. Suppose that the stream of unresolved edges sent from processor $p_{i-1}$ to processor $p_i$ is the same as stream $A_{i-1}$ from \DFR.  We show that the stream of unresolved edges processor $p_i$ sends to $p_{i+1}$ is exactly \DFR stream $A_i$.

Processor $p_i$ of \XSCC and pass $i$ of \DFR begin by computing connected components via union-find.  Every edge that changes connectivity (starts a new component or joins two components) uses one of the $s$ possible union operations for this processor/pass. When they have both done $s$ union operations (their capacity), they have computed identical union-find data structures since they have done the same computations on the same input stream.  At this point, \DFR has not yet emitted any edges and \XSCC has emitted only resolved non-tree edges. Now \DFR processes the remaining edges of $A_{i-1}$, relabeling the endpoints, deleting edges where both endpoints are contained in the same supernode, and emitting the others to stream $A_i$. \XSCC relabels these remaining edges the same way, and emits the same stream of unresolved edges (among resolved non-tree edges).\qed
\end{proof}

Since \XSCC runs on unending streams, there is no ``end of stream'' mark to trigger creation of and processing of an \DFR $B_i$ stream.  However, the ``dump components'' diagnostic creates these $B$ streams.

\begin{lemma}
For $\XSCCns(s,\eta,A_0)$ followed by a call
to $\Call{DumpComponentLabels}{}$, stream $D_{i-1}$ is identical to
stream $B_i$ from $\DFRns(s,\eta,A_0)$.
\label{lemma:B-stream}
\end{lemma}
\begin{proof}
The ``dump components'' command after ingestion of a finite stream serves as an end-of-stream marker for \XSCC. We prove the lemma by induction. For the base case, streams $B_0$ and $D_{-1}$, the component information input to pass $0$ of \DFR and processor $p_0$ of \XSCC respectively are both empty. Suppose that stream $B_i$ from $\DFRns(s,\eta,A_0)$ is the same as stream $D_{i-1}$ from 
%$\XSCCns(s,\eta,A_0)$ followed   macro messes LaTeX up.
XS-CC$(s,\eta,A_0)$ followed
by a call to\\ $\Call{DumpComponentLabels}{}$. We show that stream $B_{i+1}$ is the same stream $D_i$. From the proof of Lemma~\ref{lem:A-stream}, the runs of \XSCC and \DFR compute the same connected components in processor $p_i$ and pass $i+1$ respectively. Because \XSCC and \DFR are using the same supernode naming function and have the same capacity, the union-find data structures (names of representatives and names of set elements) are identical. As described above, processor $p_{i}$ relabels and emits all elements of $D_{i-1}$ the same way that \DFR pass $i+1$ relabels elements stream $B_i$ to stream $B_{i+1}$. Then processor $p_i$ and \DFR pass $i+1$ output the information in their identical union-find structures in identical ways, completing streams $D_i$ and $B_{i+1}$ respectively. \qed
\end{proof}

\iffalse
\begin{lemma}
Suppose the capacity of \WStream and \XStream processors are configured equivalently: they
both can perform the same number of union operations.
The connectivity structure output by W-Stream pass $i$ ($G_{B_i}$) is
exactly the same as that stored in \XStream processors
$\{p_0,\ldots,p_{i-1}\}$ after processing edge stream $G_t$.
\label{lemma:wx-correspondence}
\Jon{change statement to accomplish: relating \XSCC relabeling to \DFR $G_B$ graphs.  Also, use induction from $i$ to $i+1$ to get rid of $i-2$ 
usage.}
\Jon{either here or in a subsequent lemma, argue about any permutation of $G_t$}
\end{lemma}
\begin{proof}
(induction on $i$) \emph{Base case}: When $i=1$, the same union
operations occur in both \DFR and \XSCCns.  W-Stream emits connectivity
structure $G_{B_1}$, but \XSCC retains a spanning forest of that
structure in processor $p_0$.
\emph{Induction}:  Suppose that the
\DFR connectivity structure $G_{B_{i-1}}$ is the same as that stored in
\XStream processors $\{p_0,\ldots,p_{i-2}\}$.  Note that
$R_{i-2}(u) = R_{i-2}(v)$
iff vertices $u$ and $v$ are in the same connected component of $G_{B_{i-1}}$.
During pass $i$ of W-Stream, $s$ union operations occur and \XStream
processor $p_{i-1}$ is the builder.  Each
stream edge $(u,v)$ connects components of $G_{B_{i-1}}$ iff
$R_{i-1}(u) \neq R_{i-1}(v)$. \XStream processor $i-1$ has space to
store $(u,v)$ and by Invariant~\ref{inv:normal-1}, any previously-stored
non-tree edges can be sent downstream to ensure that the processor can
use its entire capacity for connectivity information. Therefore, \XStream 
processors $0,\ldots,i-1$ store the same connectivity structure as $G_{B_i}$. $\qed$
\end{proof}
\fi

\subsection{\XSCC queries}
The most basic \XSCC query is a connectivity query: are nodes $u$ and $v$ in
the same connected component? A query that arrives at \XStream tick $t$ will
be answered with respect to the graph $G_t$.  The query $(u,v)$ enters
the system from the I/O processor and propagates through the processors
just as new stream edges do.  Each processor relabels the endpoints,
and the tail processor returns ``$(u,v)$:yes'' if the labels are the same and ``$(u,v)$:no''
otherwise.  This holds even if  one or both of the endpoints have
never been seen before.
%Theorem~\ref{thm:query-correctness}
% leverages our previous definitions and Lemma~\ref{lemma:wx-correspondence} to
%shows query correctness
% at single-edge granularity.
% Recall the difference between transit edges and settled
% edges.\Jon{if this is the only place the concept is used, move the definition here.
% Otherwise...}
The following theorem shows that connectivity queries are correct at single-edge granularity, and therefore
that \XSCC in normal mode correctly computes the connected components of an
edge stream.

\begin{theorem}
Suppose that the connectivity query $(u,v)$ arrives at the head processor of an X-Stream
system with $P$ processors at X-Stream tick $t$. Then the I/O processor will
receive the boolean query answer at time $t+P$.  The answer will be \emph{True}
iff $u$ was connected to $v$ in $G_t$, the logical graph that existed at time $t$.
\label{thm:query-correctness}
\end{theorem}
\begin{proof}
Recall that $p_B$ is the building processor.
The query answer will be determined by
X-Stream tick $t+B$ at the latest, since, by Invariant~\ref{inv:normal-1},  
$p_B$ is the last
one to store any tree edges and hence, any union-find information.  Thus it is the 
last processor that can change a label. The $(u,v)$ query travels processor-to-processor in a primary basket, just as
the dump-components command does. If there are any transit edges in $G_t$ when the query arrives, they travel in slots of bundles strictly ahead of the query. Thus transit edges will settle into a processor before the query arrives.  Similarly, any edges that arrive after the query travel in bundles strictly behind the query and cannot affect the query relabeling. Thus when the bundle with the query arrives at processor $p_i$, the union-find data structure, and the processor's status as the builder or not, are set exactly according to the graph $G_t$ in the system when the query arrived.

Processing query $(u,v)$ is closely related to processing $\Call{DumpComponentLabels}{}$. Instead of dumping information for every vertex, starting at the point where a vertex is first encapsulated in a supernode, simple queries only consider two vertices. The label for $u$ will only change from $u$ to a supernode label $b_y$ at the processor $p_i$ that first incorporates $u$ into a local component ($p_i = \beta(u)$).  In $\Call{DumpComponentLabels}{}$, processor $p_i$ is the first that outputs any pair $(u, b_y)$, with first component $u$ into the stream $D_i$.  Thus, after the query has passed the building processor, the labels for vertices $u$ and $v$ are identical to their output values, which exit the system at time $t+P$. By Lemma~\ref{lemma:B-stream}, these are the same labels they would have if \DFR is run on graph $G_t$.  Because \DFR is a correct connected components algorithm, vertices $u$ and $v$ will have the same label if an only if they are in the same connected component. \qed
\end{proof}

We call queries that \XSCC answers with latency $p$ \emph{constant} queries.
See section~\ref{sec:non-constant} for examples of non-constant queries.

%\Jon{Need non-duplication theorem:  O(1) space per edge}
The next theorem shows that \XSCC is space-efficient, storing the current graph in asymptotically optimal space.
\begin{theorem}
\label{thm:non-dup}
In normal operation of \XSCCns, each edge is stored in exactly one processor,
requiring $O(1)$ space.
\end{theorem}
\begin{proof}
%The only duplication scenario is when a copy of an edge exists downstream
%a processor that has empty space.
In normal operation, when a new edge $e$ arrives at a processor $p_i$ that
already stores a copy of $e$, processor $p_i$ removes $e$ from the stream and
updates the timestamp of $e$.
Invariant~\ref{inv:normal-1} ensures that incoming tree edge $e$ encounters any
previously-stored copies of itself before it reaches $p_B$, the building
processor, which recognizes it as a tree edge.
Invariant~\ref{inv:normal-2} ensures that incoming non-tree edge $e$
encounters any previously-stored copies of itself before it reaches $p_S$,
the first processor with any empty space. Furthermore, this invariant also
ensures that there are no edges stored downstream of $p_S$.  
\qed
\end{proof}

% CAP: save previous proof for now.
\iffalse
Case (I): there are no transit edges in the system. Then neither
the building processor designation $p_B$ nor the logical graph $G(t)$ change between query
entry time $t$ and time $t+B$ (when the query arrives at the builder).
By Lemma~\ref{lemma:wx-correspondence}, $u$ and $v$ are connected iff
$R_{p_B}(u) = R_{p_B}(v)$.  Further relabeling does not change this result, and the correct
answer will propagate to the I/O processor at tick $t+P$.
Case (II): there are transit edges in the system at X-Stream tick $t$.  We must argue that
any transit edges that affect connectivity have time to settle (a prerequisite for
invoking Lemma~\ref{lemma:wx-correspondence}) before the query arrives at the building processor.
Noting that the building processor might advance as the query propagates, let $B'$ be
the index of the current builder at time the query arrives there.
Since the
storage capacity $s$ always exceeds $P$,  the building processor will be either $p_{B'-1}$
(if the builder designation advanced) or
$p_{B'}$ when the query reaches it.  In either case, Invariant~\ref{inv:normal-1} ensures
that no tree edge is downstream of the current builder. Therefore, any transit edge will arrive at
the builder at time $t' = t + B' - 1 \le t+B-1$.  This is at least one tick before the query
arrives.  At that point, Case (I) applies. $\qed$
\fi



Theorem~\ref{thm:query-correctness} shows that basic connectivity queries are
answered correctly by \XSCCns.  In Section~\ref{sec:non-constant}
we informally discuss three additional types of feasible queries: complex queries such as
finding all vertices not in the giant component of a social network \Cindy{Can we ask for the maximum size of any connected component in a constant query? I think so, since every union-find component has a vertex count.  This gives us the size of the giant component.}\Jon{Last call for adding this.}, vertex-based
queries like finding the degree or neighborhood, and diagnostic
queries regarding system capacity used.  Also, by Invariant~\ref{inv:normal-1},
X-Stream always knows a spanning tree of the streaming graph by construction. This
tree could be checkpointed, for example, if processors share a filesystem. \Cindy{The information in the spanning tree is about the size of the output of dump components.}\Jon{Wasn't sure what to do about this comment.}
\iffalse
\begin{algorithm*}
\caption{Process Connectivity Query on Processor $p_a$}\label{algo:connectq}
Process Connectivity Query on Processor $p_a$\\\makealgtitle
\begin{algorithmic}[1]
\Procedure{ConnectedQuery}{$([v,R_{a-1}(v)],[u,R_{a-1}(u)],answer)$}
\If{answer = false}
	\State $R_a(v) \gets \Call{Find\_Set}{R_{a-1}(v)}$
	\State $R_a(u) \gets \Call{Find\_Set}{R_{a-1}(u)}$
\If{$R_a(v) = R_a(u)$}
\State answer = true
\EndIf
\EndIf
\State \Call{SendDownstream}{QUERY $([v,R_a(v)],[u,R_a(u)], answer)$}\Comment{we don't care about updating label if answer is already true}
\State \Return
\EndProcedure
\end{algorithmic}
\end{algorithm*}






\subsection{Full X-Stream Normal Mode Algorithm}\label{sec:algo:normal}
We next describe how to expand the implementation of W-Stream UCONN in X-Stream so that we will later be able to age correctly and as a result, handle infinite streams. Algorithm~\ref{algo:normaledge1} describes the method of processing an edge on each processor in normal mode, using Algorithm~\ref{algo:unrollwstream} (which simulates W-Stream) as a subroutine. The main difference is that we now keep non-tree edges, which may effect connectivity after aging. As a result, we also have to verify that an edge does not already exist somewhere in the system before it can be stored. This is done with lap counter metadata that is passed around the ring along with the edge.
\begin{algorithm*}
\caption{Process Edge in Normal Mode on Processor $p_a$}\label{algo:normaledge1}
Process Edge in Normal Mode on Processor $p_a$\\\makealgtitle
\begin{algorithmic}[1]
\LeftComment{\Call{Forward}{} and \Call{Store\_Or\_Forward}{} include relabeling \Alex{but not if already non-tree edge?}}
\LeftComment  \Call{Find\_Set}{$v$} returns the representative node in the union find structure for the set containing $v$. The structure is initialized such that \Call{Find\_Set}{$v$} returns NULL for all $v$.

\Procedure{ProcessEdge}{$e=(([u,R_{a-1}(u)],[v,R_{a-1}(v)],t'),(c,p))$}
\If{c=0}
	\If {\Call{ProcessTreeEdge}{$e$,$S_u$,$S_v$}}
	\State  \Return
	\Else
	  	\State $(c,p) \gets (1,a)$ 
		  \State \Call{Forward}{e}
 		 \State \Return
	\EndIf
\EndIf
\If {$(c,p)=(2,a)$}
 \State \Call{Full\_Stop}{} \Comment{system is completely full}
\EndIf

\If{$c = 1$}  
  \If{$p=a$} $(c,p) \gets (2,a)$ 
  \Else
    \If {Timestamps.lookup(e)}  Timestamps.update(e,t')
    \Else \ \Call{Forward}{e}  \Comment{first lap not complete, continue checking for duplicates}
    \EndIf
    \State  \Return
  \EndIf
\EndIf
\LeftComment{process non-tree edge with $c=2$}
\State \Call{Store\_Or\_Forward}{$e$,$t'$}

\EndProcedure

\Procedure{Store\_Or\_Forward}{$e$,$t'$}
\If{Timestamps.lookup(e)} 
  \State Timestamps.update(e)
  \State \Return
\EndIf
\If {HAVE\_SPACE}{\mbox{}}  \Comment{number of stored edges is less than capacity}
  \State Timestamps.insert($e$,$t'$)  \Comment{and remove from basket (we haven't been including that}
  \State \Return
\Else 
  \State \Call{FORWARD}{$e$}
\EndIf
\EndProcedure

\end{algorithmic}
\end{algorithm*}














\subsection{X-Stream Implementing W-Stream Correctness}
We first show that executing Algorithm~\ref{algo:unrollwstream} computes the same intermediate pass streams computed by the algorithm in~\cite{} for solving undirected connectivity (UCON) in the W-Stream model. We will show that the correctness of query answers using Algorithm~\ref{algo:connectq}) follows. Suppose we have an instance of X-Stream with processors with storage space $s$, and the input graph has $n$ vertices. Solving UCON in W-Stream requires $\Omega(n/s)$ passes (\cite{AMP:aggarwal2004streaming} Theorem 2.1). We assume that we have $p$ processors such that UCON can be solved in $p$ passes. Recall from Section~\ref{sec:intro:wstream} that $G_{A_i}$ is the first part of the stream outputted by pass $i$ and $G_{B_i}$ is the second part of the stream.
\begin{observation}\label{obs:ga}
The set of a relabeled edges emitted by processor $i$ in an X-Stream system induce the graph $G_{A_i}$. However, the endpoints use the set labels constructed in the X-Stream algorithm rather than reusing vertex labels as set labels.
\end{observation}


\begin{observation}\label{obs:gb}
The \uf structures upstream from processor $i$ contain the same information as $G_{B_i}$. Thus $G_{B_i}$ could be reconstructed by letting all primitive vertices and local component labels found on $p_0,...,p_{i-1}$ be $V(G_{B_i})$ and then letting the label pairs $(x,R_i(x))$ be the edges of $G_{B_i}$.
\end{observation}

Invariant~\ref{inv:wstreamreqs} holds for the values of $G_{A_i}$ and $G_{B_i}$ since it holds for the algorithm given by~\cite{AMP:aggarwal2004streaming} to solve UCON in W-Stream.

We also observe the property that 
every processor sees input stream elements in order. Thus when a query arrives at a processor, it has seen all edges. We can now show that:
 \begin{lemma}
 Given an input graph $G$, consider an X-Stream system consists of processors with storage capacity $s$ and a sufficient number of processors, $p$, such that the W-Stream UCON algorithm requires at most $p$ passes on $G$. Suppose the X-Stream system executes Algorithm~\ref{algo:unrollwstream} to process $G$ as an input stream. Then
connectivity queries processed by the X-Stream system using Algorithm~\ref{algo:connectq} will be answered correctly. \end{lemma}
\begin{proof}
\Alex{main todo is to show query ordering is ok so correctness follows from invariant like in W-Stream}
Consider processor $p$, the final processor in the ring. $G_{A_p}$ will be null because processor $p$ does not emit any edges during the execution of Algorithm~\ref{algo:unrollwstream}. 
$G_{B_p}$ is a set of stars which fully describes the vertices contained in each component on processor $p$. Thus correctness follows from the correctness of W-Stream.
\end{proof}


\subsection{Normal Mode Correctness}
We first describe the correctness during normal mode. We must argue that a connectivity query ``Are vertices $u$ and $v$ connected?" will always be answered correctly. During normal mode, the following properties hold:

\begin{enumerate}
\item If an edge has been added to a data structure in some processor it is only stored once. All edges in the system are either = \emph{in transit} or \emph{settled} in exactly one processor.   \label{prop:edgeonce}
\item The elements of the \uf structure in each processor are unique, i.e. do not appear in the \uf on any other processor. \Alex{note that the UF structure doesn't contain nodes/elements that are LC's, just BB's}  \label{prop:onelc}
\item The set of all  tree edges over all processors is a spanning forest of the logical graph. \Alex{Need a def of stored graph: settled tree edges are spanning forest of the subgraph of the logical graph $G(t)$ comprised of edges that are settled  at $t$ - define $G_s(t)$?}\label{prop:nontree}
\end{enumerate}

\begin{lemma}\label{lemma:normalprops}
Properties $1-3$ hold up until normal mode ends for the first time. \Alex{i.e., when the first aging starts}
\end{lemma}
\Alex{

\begin{proof}
Property 1:\\
We induct on the sequence of edges received. Initially, no edge is stored so the property holds. Now suppose the property holds at time $t$ and a new edge $e = ([u,b_u],[v,b_v],t)$
arrives. If $e$ is received by a ring processor which holds an edge $e'$ with the same endpoints as $e$, i.e. $e' = ([v,b_v],[u,b_u],t')$, the processor replaces $t'$ with $t$ and removes $e$ from the stream, so it will not be stored as a duplicate. Thus we show if processor $p_s$ accepts edge $e$ as a new edge at time $t$, no such $e'$ with the same endpoints previously existed in the ring.

Case 1: Suppose processor $p_s$ accepts $e$ as a storage edge. Then $e$ must have marker pair $(c,p_i)$ such that $c=2$. If $c=2$, then $e$ was processed by every processor in the ring at least once: processors $p_1,...,p_i$ saw it when $c=0$, processors $p_{i+1},...,p_i$ (mod ring size) saw it  when $c= 1$, and now processors $p_{i+1},...,p_s$ (mod ring size) saw it  when $c=2$. If the same $(u,v)$ had previously been stored on any processor, $e$ would have been consumed by that processor and the timestamp updated to $t$.

Case 2: Suppose processor $p_s$ accepts $e$ as a tree edge. Then since no $p_i$ with $i<s$ removed $e$ from the stream, no such $p_i$ has an edge $e'$ with the same endpoints. Consider if some processor $p_i$ with $i>s$ had such an $e'$. Since $p_s$ is filling, $e'$ would have to be a storage edge on $p_i$ for some $p_j$, $j\leq s$. But then $e$ would be relabeled as storage on $p_j$ as well, and not reach $p_s$ as a tree edge. Thus no $p_i$, $i\neq s$ has and $e'$ with the same ends, and $p_s$ can accept $e$ if $p_s$ itself does not have an edge $e'$.
\\\\Property 2: We induct on the processor index, showing processor $p_i$'s \uf sets contain no elements used in the \uf sets on $p_0,...,p_{i-1}$. Clearly this holds for $p_0$. Now suppose  the union-find structures in $p_0,...p_i$ contain disjoint element sets and we show $p_{i+1}$ will not reuse any of these elements. When an edge is stored by a processor, its relabeled ends are uses as elements in the union-find structure. Consider the labels on any edge $e=([u,R_i(u)],[v,R_i(v)],t)$ sent from $p_i$ to $p_{i+1}$. If  $R_i(u) = R_i(v)$ it is a storage edge and nothing is added to the union find structure. If $R_i(u) =u$ then no processor $p_0,...,p_i$ has any edges containing $u$, or the primitive BB corresponding to $u$ in their union-find structures. In any other case, $R_i(u)$ is set name on some processor $j \in \{0,...,i\}$.

\end{proof}
}
\begin{proof}
Property 1: We induct on the sequence of edges received. Initially, no edge is stored so the property holds. Now suppose the property holds at time $t$ and a new edge $e = ([v,b_v],[u,b_u],t)$
arrives. If $e$ is received by a ring processor which holds an edge $e'$ with the same endpoints as $e$, i.e. $e' = ([v,b_v],[u,b_u],t')$, the processor replaces $t'$ with $t$ and removes $e$ from the stream, so it will not be stored as a duplicate. Thus we show if processor $p_s$ accepts edge $e$ as a new edge at time $t$, no such $e'$ with the same endpoints previously existed in the ring.

Case 1: Suppose processor $p_s$ accepts $e$ as a storage edge for processor $p_j$. Then when $e$ reached $p_j$ on its first lap around the ring, it would have been marked as storage and had $first\_lap$ set to true. Then $p_s$ would not be accepting $e$ until after $e$ had completed a lap from the head to the tail and had $first\_lap$ set to false. Thus no $e'$ with the same endpoints existed, or a $e$ would not have completed that whole lap.

Case 2: Suppose processor $p_s$ accepts $e$ as a tree edge. Then since no $p_i$ with $i<s$ removed $e$ from the stream, no such $p_i$ has an edge $e'$ with the same endpoints. Consider if some processor $p_i$ with $i>s$ had such an $e'$. Since $p_s$ is filling, $e'$ would have to be a storage edge on $p_i$ for some $p_j$, $j\leq s$. But then $e$ would be relabeled as storage on $p_j$ as well, and not reach $p_s$ as a tree edge. Thus no $p_i$, $i\neq s$ has and $e'$ with the same ends, and $p_s$ can accept $e$ if $p_s$ itself does not have an edge $e'$.

Case 3: Suppose processor $p_s$ accepts $e$ as mortgage.
\end{proof}


%\begin{proof}
%Property 1: We induct on the sequence of edges received. Initially, no edge is stored so the property holds. Now suppose the property holds at time $t$ and a new edge $e = ([v,b_v],[u,b_u],t)$ arrives.

%Case 1: Suppose no other edge $e' = ([v,b_v],[u,b_u],t')$ exists in the system, so $e$ is the first instance of an edge with ends $u$ and $v$.

%Case 2: Next suppose $e$ already exists in the system. When a processor receives an edge it already holds, it removes it from the stream and updates the time stamp, so the edge cannot be stored a second time on different processor. Thus we show the new instance of $e$ always reaches the processor already holding $e$:

%Case 2a: Suppose $e$ is a tree or mortgage edge on some porcessor $p_s$. Let $p_f$ be the filling processor, so $f\geq s$. Then each processor $p_i$, for $i<s$, is sealed and will not take $e$ as a tree or mortgage edge. Also, relabeling ends of $e$ will give $R_i(v) \neq R_i(u)$, or the previous instance of $e$ would have also been relabeled with $R_i(v) =R_i(u)$, making $e$ a storage edge and not a tree or mortgage edge for $p_s$. Thus any new copy of edge $e$ will reach $p_s$.

%Case 2b: Suppose $e$ is a storage edge for some processor $p_s$, held by some processor $p_j$, $j\neq s$. As before, if $p_f$ is filling, $f\geq s$. Since each $p_i$ with $i<s$ is sealed it cannot accept $e$ as a tree or mortgage edge. Then if $j<s$, $e$ reaches $p_j$. Otherwise, $e$ reaches $p_s$, at which point it is labeled as storage with $R_s(u) = R_s(v)$ and on its first lap. Then $e$ cannot be taken by a processor other than $p_j$ before it reaches the tail to finish the lap, so $e$ will reach $p_j$.

%Properties $2-4$: omitted for space.
%\end{proof}

\begin{theorem}\label{theorem:propstoquery}
If properties $1-4$ hold for $G_t$, then there exists a processor $p$ that gives vertices $u$ and $v$ the same label at time $t$ if and only if the vertices are in the same connected component in $G_t$. Equivalently, $R_m(v) = R_m(u)$ if and only if there exists a $u,v$-path in $G_t$.
\end{theorem}
\begin{proof}
Let $E_t^i$ be the set of tree edges on processors $p_1,...,p_i$ at time $t$. Let $G_t^i = (V_t,E_t^i)$. Then $G_t^i$ is subgraph of $G_t^{i+1}$ for $i = 0,...m-1$. By Property~\ref{prop:nontree}, $G_t^m$ is a spanning forest of $G_t$. By Property~\ref{prop:oneprimitive}, the primitive building block $b_u \in B_v$ for vertex $u$ is unique so only one possible relabeling $R_i(u)$ corresponds to vertex $u$. We show by induction that there exists a $u,v$-path in $G_t^i$ if and only if $R_i(u) = R_i(v)$. At $p_1$, all building blocks are primtive, so $R_1(u) = R_1(v)$ iff there is a $u,v$-path in $G_t^1$.
Now suppose at $p_i$, $R_i(u) = R_i(v)$ iff there exists a $u,v$-path in $G_t^i$. We show this implies the same for $p_{i+1}$:

Case 1: Suppose $R_i(u) =R_i(v)$. Then $R_{i+1}(u)= R_{i+1}(v)$ and since $G_t^i$ is a subgraph of $G_t^{i+1}$, the $u,v$-path in $G_t^i$ is also in $G_t^{i+1}$.

Case 2: Suppose $R_i(u)\neq R_i(v)$. Then $R_{i+1}(u) = R_{i+1}(v)$ iff $R_i(u)$ and $R_i(v)$ are BBs in the same LC on $p_{i+1}$, by Property~\ref{prop:onelc}. By definition of LC, $R_i(u)$ and $R_i(v)$ are in the same LC iff there exists a sequence of edges $(x_1,y_1),(x_2,y_2),...(x_l,y_l)$ on $p_{i+1}$ with $x_j,y_{j-1}$ in the same BB for $j=2,...,l$ and $x_1 \in R_i(u)$ and $y_l \in R_i(v)$. By induction there exists a $x_j,y_{j-1}$-path in $G_t^{i+1}$ for $j=2,...,l$, so there exists a $x_1,y_l$-path in $G_t^{i+1}$. By induction we also have a $u,x_1$-path and a $y_l,v$-path in $G_t^{i+1}$ so all together, there is a $u,v$-path. Thus $R_i(u)$ and $R_i(v)$ are in the same LC on $p_{i+1}$ ($R_{i+1}(u) = R_{i+1}(v)$) iff there exists a $u,v$-path in $G_t^{i+1}$.

Since there exists a $u,v$-path in $G_t^i$ if and only if $R_i(u) = R_i(v)$ for $i = 1,...,m$ we have shown $R_m(u)=R_m(v)$ iff there is a $u,v$-path in $G_t$.
\end{proof}
\begin{theorem}\label{theorem:queriesnormal}
During normal mode, there exists a processor $p$ that gives vertices $u$ and $v$ the same label at time $t$ if and only if the vertices are in the same connected component in $G_t$.
\end{theorem}
\begin{proof}
Properties $1-4$ hold during normal mode, so by Theorem~\ref{theorem:propstoquery} queries are answered correctly.
\end{proof}
 An edge will remain in a outside slot for at most two laps around the ring 

From this we, we can also guarantee that the input stream will not be starved, as long as there are at least two outside slots per \bundlens. At the head processor, begin putting an edge into the first outside slot of each \bundle that is passed. If an edge is in the first outside slot of a bundle received from the tail, but that edge in the second outside slot of the \bundlens. On its second lap, an edge will be accepted by a processor, so no edges will be passed from the tail to head in the second outside slot.
\fi

\iffalse  OLD

If a new edge arriving at the filling processor connects two separate components, it is a \emph{spanning tree edge}, since the set of all such edges a processor has received forms a spanning forest of the identified components. For example, in Figure~\ref{fig:BBLCexample} edge $(v_0,v_1)$ on $p_0$ and $(v_0,v_4)$ on $p_1$ are tree edges. When storing a tree edge $e$, processor $p_i$ updates the disjoint-set structure to connect the building blocks named $R_{i-1}(v)$ and $R_{i-1}(u)$ in some local component $b_x$. Processor $p_i$ also stores the fact that $v$ is encapsulated by $R_{i-1}(v)$ and $u$ is encapsulated by $R_{i-1}(u)$; $u$ and $v$ are then \emph{known vertices} on $p_i$. 

If a node $p_i$ receives an edge $([u,R_{i-1}(u)],[v,R_{i-1}(v)])$ and relabels it such that $R_i(u) = R_i(v)$, then $(u,v)$ is a \emph{non-tree edge}. While these edges are not initially necessary for maintaining connected components, they may be necessary for determining correct components after aging, and thus must be kept some where in the system, not necessarily on $p_i$. Non-tree edges held on the same processor as the component containing them are called \emph{mortgage edges}. Non-tree edges stored on a different processor are called \emph{storage edges}.

If $p_i$ is the filling processor, it either has available space or contains some storage edges for other processors. If $p_i$ receives a non-tree edge contained in one of its own local components, it prefers that mortgage edge over any storage edges it may have. Thus if it does not have room for the mortgage edge, it will \emph{jettison}, or pass downstream, a storage edge to make room for the mortgage edge. The filling processor also prefers new tree edges over storage edges, and will jettison a storage edge if a new tree edge is received and there is not available space. When the processor is full of only tree and mortgage edges, it becomes sealed and $p_{i+1}$ becomes the filling processor. 

If a sealed processor $p_i$ receives an non-tree edge $e$ contained in one of its local components, it will take $e$ as a  mortgage edge if it has space from a previous aging. A sealed processor $p_i$ will also jettison a storage edge to make room for a mortgage edge. If $p_i$ does not have space or a storage edge to jettison, $e$ becomes a storage edge for $p_i$, and $e$ is passed downstream.
For example, in Figure~\ref{fig:BBLCexample}, edge $(v_1,v_2)$ on $p_2$ is a storage edge for $p_0$. The only edges arriving to non-filling processors downstream of the filling processor, which are not sealed, are storage edges. These storage edges will be accepted by an unsealed non-filling processor if it has room; otherwise they will be passed downstream to other non-filling processors. Storage edges may also be passed around the ring past the head, and accepted by a sealed processor with available space that opened up during aging.

\fi
