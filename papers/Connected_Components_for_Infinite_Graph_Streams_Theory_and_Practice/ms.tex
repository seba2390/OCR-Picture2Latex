%%%%%%%%%%%%%%%%%%%%%%% file template.tex %%%%%%%%%%%%%%%%%%%%%%%%%
%
% This is a general template file for the LaTeX package SVJour3
% for Springer journals.          Springer Heidelberg 2010/09/16
%
% Copy it to a new file with a new name and use it as the basis
% for your article. Delete % signs as needed.
%
% This template includes a few options for different layouts and
% content for various journals. Please consult a previous issue of
% your journal as needed.
%
%%%%%%%%%%%%%%%%%%%%%%%%%%%%%%%%%%%%%%%%%%%%%%%%%%%%%%%%%%%%%%%%%%%
%
% First comes an example EPS file -- just ignore it and
% proceed on the \documentclass line
% your LaTeX will extract the file if required
\begin{filecontents*}{example.eps}
%!PS-Adobe-3.0 EPSF-3.0
%%BoundingBox: 19 19 221 221
%%CreationDate: Mon Sep 29 1997
%%Creator: programmed by hand (JK)
%%EndComments
gsave
newpath
  20 20 moveto
  20 220 lineto
  220 220 lineto
  220 20 lineto
closepath
2 setlinewidth
gsave
  .4 setgray fillt
grestore
stroke
grestore
\end{filecontents*}
%
\RequirePackage{fix-cm}
%
%\documentclass{svjour3}                     % onecolumn (standard format)
%\documentclass[smallcondensed]{svjour3}     % onecolumn (ditto)
%\documentclass[smallextended]{svjour3}       % onecolumn (second format)
\documentclass[twocolumn]{svjour3}          % twocolumn
%
\smartqed  % flush right qed marks, e.g. at end of proof
%
\usepackage{graphicx}
\usepackage{algorithm}
\usepackage[noend]{algpseudocode}
\usepackage{color}
\usepackage{xcolor}
\usepackage{algpseudocode,caption}
\usepackage{tcolorbox}

\usepackage{amsfonts,amsmath,graphicx,subfigure}
\algnewcommand\algorithmicswitch{\textbf{switch}}
\algnewcommand\algorithmiccase{\textbf{case}}
\algnewcommand\algorithmicassert{\texttt{assert}}
\algnewcommand\Assert[1]{\State \algorithmicassert(#1)}%
% New "environments"
\algdef{SE}[SWITCH]{Switch}{EndSwitch}[1]{\algorithmicswitch\ #1\ \algorithmicdo}   {\algorithmicend\ \algorithmicswitch}%
\algdef{SE}[CASE]{Case}{EndCase}[1]{\algorithmiccase\ #1}{\algorithmicend\     \algorithmiccase}%
\algtext*{EndSwitch}%
\algtext*{EndCase}%
\algnewcommand\algorithmicleftcomment{$\triangleright $}
\algnewcommand\LeftComment{\State\algorithmicleftcomment\ }%for no line # do \item[\algorithmicleftcomment]


\makeatletter
% The 'plainruled' float style
\newcommand\fs@plainruled{\def\@fs@cfont{\rmfamily}\let\@fs@capt\floatc@plainruled
  \def\@fs@pre{\hrule height.8pt depth0pt \kern2pt}%
  \def\@fs@post{}%
  \def\@fs@mid{\kern4pt\hrule height.8pt depth0pt \kern5pt}%
  \let\@fs@iftopcapt\iffalse}
\makeatother

\floatstyle{plainruled}
\restylefloat{algorithm}



\newtheorem{defn}{Definition}
\newtheorem{invariant}{Invariant}
\newtheorem{observation}{Observation}

%\newcommand{\Alex}[1]{{\footnotesize \sf {\color{black!50!green}{\bf Alex:} #1}}}
%\newcommand{\Jon}[1]{{\footnotesize \sf {\color{blue}{\bf Jon:} #1}}}
%\newcommand{\Cindy}[1]{{\footnotesize \sf {\color{red}{\bf Cindy:} #1}}}
\newcommand{\Alex}[1]{}
\newcommand{\Jon}[1]{}
\newcommand{\Cindy}[1]{}


\newcommand{\makealgtitle}{ {\vspace{-0.2cm}  \hrule height.8pt depth0pt \kern2pt}}
\newcommand{\nodeloc}{\beta}
\newcommand{\lcloc}{\alpha}
\newcommand{\nodeusr}{\gamma}

\newcommand{\DFRns}{\mbox{DFR}}
\newcommand{\DFR}{\mbox{DFR\ }}

\newcommand{\WStreamns}{\mbox{W-Stream}}
\newcommand{\WStream}{\mbox{W-Stream\ }}

\newcommand{\XSns}{\mbox{XS}}
\newcommand{\XS}{\mbox{XS\ }}

\newcommand{\XSCCns}{\mbox{XS-CC}}
\newcommand{\XSCC}{\mbox{XS-CC\ }}

\newcommand{\XStreamns}{\mbox{X-Stream}}
\newcommand{\XStream}{\mbox{X-Stream\ }}

\newcommand{\ufns}{\mbox{union-find}}
\newcommand{\uf}{\mbox{union-find\ }}

% bundle-> a group of slots passed between pe's
% as opposed to block-> building block or local component
% we were using "block" 
\newcommand{\bundlens}{\mbox{bundle}}
\newcommand{\bundle}{\mbox{bundle\ }}

\MakeRobust{\Call}

%
% \usepackage{mathptmx}      % use Times fonts if available on your TeX system
%
% insert here the call for the packages your document requires
%\usepackage{latexsym}
% etc.
%
% please place your own definitions here and don't use \def but
% \newcommand{}{}
%
% Insert the name of "your journal" with
% \journalname{myjournal}
%
\begin{document}

\title{Connected Components for Infinite Graph Streams: Theory and Practice}

%\titlerunning{Short form of title}        % if too long for running head

\author{Jonathan Berry \and Cynthia Phillips \and Alexandra\ Porter}


%\authorrunning{Short form of author list} % if too long for running head

\institute{
           J. Berry \at
           Sandia National Laboratories\\
	\email{jberry@sandia.gov} 
	\and
	C. Phillips \at
           Sandia National Laboratories\\
	   \email{caphill@sandia.gov
        \and
       A. Porter \at
              Stanford University \\
              \email{amporter@cs.stanford.edu}           %  \\
%             \emph{Present address:} of F. Author  %  if needed
          } 
}

%\date{Received: date / Accepted: date}
\date{\today}
% The correct dates will be entered by the editor


\maketitle

\begin{abstract}
  In this paper, we explore the connection between secret key agreement and secure omniscience within the setting of the multiterminal source model with a wiretapper who has side information. While the secret key agreement problem considers the generation of a maximum-rate secret key through public discussion, the secure omniscience problem is concerned with communication protocols for omniscience that minimize the rate of information leakage to the wiretapper. The starting point of our work is a lower bound on the minimum leakage rate for omniscience, $\rl$, in terms of the wiretap secret key capacity, $\wskc$. Our interest is in identifying broad classes of sources for which this lower bound is met with equality, in which case we say that there is a duality between secure omniscience and secret key agreement. We show that this duality holds in the case of certain finite linear source (FLS) models, such as two-terminal FLS models and pairwise independent network models on trees with a linear wiretapper. Duality also holds for any FLS model in which $\wskc$ is achieved by a perfect linear secret key agreement scheme. We conjecture that the duality in fact holds unconditionally for any FLS model. On the negative side, we give an example of a (non-FLS) source model for which duality does not hold if we limit ourselves to communication-for-omniscience protocols with at most two (interactive) communications.  We also address the secure function computation problem and explore the connection between the minimum leakage rate for computing a function and the wiretap secret key capacity.
  
%   Finally, we demonstrate the usefulness of our lower bound on $\rl$ by using it to derive equivalent conditions for the positivity of $\wskc$ in the multiterminal model. This extends a recent result of Gohari, G\"{u}nl\"{u} and Kramer (2020) obtained for the two-user setting.
  
   
%   In this paper, we study the problem of secret key generation through an omniscience achieving communication that minimizes the 
%   leakage rate to a wiretapper who has side information in the setting of multiterminal source model.  We explore this problem by deriving a lower bound on the wiretap secret key capacity $\wskc$ in terms of the minimum leakage rate for omniscience, $\rl$. 
%   %The former quantity is defined to be the maximum secret key rate achievable, and the latter one is defined as the minimum possible leakage rate about the source through an omniscience scheme to a wiretapper. 
%   The main focus of our work is the characterization of the sources for which the lower bound holds with equality \textemdash it is referred to as a duality between secure omniscience and wiretap secret key agreement. For general source models, we show that duality need not hold if we limit to the communication protocols with at most two (interactive) communications. In the case when there is no restriction on the number of communications, whether the duality holds or not is still unknown. However, we resolve this question affirmatively for two-user finite linear sources (FLS) and pairwise independent networks (PIN) defined on trees, a subclass of FLS. Moreover, for these sources, we give a single-letter expression for $\wskc$. Furthermore, in the direction of proving the conjecture that duality holds for all FLS, we show that if $\wskc$ is achieved by a \emph{perfect} secret key agreement scheme for FLS then the duality must hold. All these results mount up the evidence in favor of the conjecture on FLS. Moreover, we demonstrate the usefulness of our lower bound on $\wskc$ in terms of $\rl$ by deriving some equivalent conditions on the positivity of secret key capacity for multiterminal source model. Our result indeed extends the work of Gohari, G\"{u}nl\"{u} and Kramer in two-user case.
\keywords{streaming\and graph algorithms \and dynamic graphs \and connected 
components}
% \PACS{PACS code1 \and PACS code2 \and more}
% \subclass{MSC code1 \and MSC code2 \and more}
\end{abstract}

\section{Introduction} \label{sec:intro}
% !TEX root = ../arxiv.tex

Unsupervised domain adaptation (UDA) is a variant of semi-supervised learning \cite{blum1998combining}, where the available unlabelled data comes from a different distribution than the annotated dataset \cite{Ben-DavidBCP06}.
A case in point is to exploit synthetic data, where annotation is more accessible compared to the costly labelling of real-world images \cite{RichterVRK16,RosSMVL16}.
Along with some success in addressing UDA for semantic segmentation \cite{TsaiHSS0C18,VuJBCP19,0001S20,ZouYKW18}, the developed methods are growing increasingly sophisticated and often combine style transfer networks, adversarial training or network ensembles \cite{KimB20a,LiYV19,TsaiSSC19,Yang_2020_ECCV}.
This increase in model complexity impedes reproducibility, potentially slowing further progress.

In this work, we propose a UDA framework reaching state-of-the-art segmentation accuracy (measured by the Intersection-over-Union, IoU) without incurring substantial training efforts.
Toward this goal, we adopt a simple semi-supervised approach, \emph{self-training} \cite{ChenWB11,lee2013pseudo,ZouYKW18}, used in recent works only in conjunction with adversarial training or network ensembles \cite{ChoiKK19,KimB20a,Mei_2020_ECCV,Wang_2020_ECCV,0001S20,Zheng_2020_IJCV,ZhengY20}.
By contrast, we use self-training \emph{standalone}.
Compared to previous self-training methods \cite{ChenLCCCZAS20,Li_2020_ECCV,subhani2020learning,ZouYKW18,ZouYLKW19}, our approach also sidesteps the inconvenience of multiple training rounds, as they often require expert intervention between consecutive rounds.
We train our model using co-evolving pseudo labels end-to-end without such need.

\begin{figure}[t]%
    \centering
    \def\svgwidth{\linewidth}
    \input{figures/preview/bars.pdf_tex}
    \caption{\textbf{Results preview.} Unlike much recent work that combines multiple training paradigms, such as adversarial training and style transfer, our approach retains the modest single-round training complexity of self-training, yet improves the state of the art for adapting semantic segmentation by a significant margin.}
    \label{fig:preview}
\end{figure}

Our method leverages the ubiquitous \emph{data augmentation} techniques from fully supervised learning \cite{deeplabv3plus2018,ZhaoSQWJ17}: photometric jitter, flipping and multi-scale cropping.
We enforce \emph{consistency} of the semantic maps produced by the model across these image perturbations.
The following assumption formalises the key premise:

\myparagraph{Assumption 1.}
Let $f: \mathcal{I} \rightarrow \mathcal{M}$ represent a pixelwise mapping from images $\mathcal{I}$ to semantic output $\mathcal{M}$.
Denote $\rho_{\bm{\epsilon}}: \mathcal{I} \rightarrow \mathcal{I}$ a photometric image transform and, similarly, $\tau_{\bm{\epsilon}'}: \mathcal{I} \rightarrow \mathcal{I}$ a spatial similarity transformation, where $\bm{\epsilon},\bm{\epsilon}'\sim p(\cdot)$ are control variables following some pre-defined density (\eg, $p \equiv \mathcal{N}(0, 1)$).
Then, for any image $I \in \mathcal{I}$, $f$ is \emph{invariant} under $\rho_{\bm{\epsilon}}$ and \emph{equivariant} under $\tau_{\bm{\epsilon}'}$, \ie~$f(\rho_{\bm{\epsilon}}(I)) = f(I)$ and $f(\tau_{\bm{\epsilon}'}(I)) = \tau_{\bm{\epsilon}'}(f(I))$.

\smallskip
\noindent Next, we introduce a training framework using a \emph{momentum network} -- a slowly advancing copy of the original model.
The momentum network provides stable, yet recent targets for model updates, as opposed to the fixed supervision in model distillation \cite{Chen0G18,Zheng_2020_IJCV,ZhengY20}.
We also re-visit the problem of long-tail recognition in the context of generating pseudo labels for self-supervision.
In particular, we maintain an \emph{exponentially moving class prior} used to discount the confidence thresholds for those classes with few samples and increase their relative contribution to the training loss.
Our framework is simple to train, adds moderate computational overhead compared to a fully supervised setup, yet sets a new state of the art on established benchmarks (\cf \cref{fig:preview}).


\section{X-Stream model}\label{sec:model}
Online convex optimization with memory has emerged as an important and challenging area with a wide array of applications, see \citep{lin2012online,anava2015online,chen2018smoothed,goel2019beyond,agarwal2019online,bubeck2019competitively} and the references therein.  Many results in this area have focused on the case of online optimization with switching costs (movement costs), a form of one-step memory, e.g., \citep{chen2018smoothed,goel2019beyond,bubeck2019competitively}, though some papers have focused on more general forms of memory, e.g., \citep{anava2015online,agarwal2019online}. In this paper we, for the first time, study the impact of feedback delay and nonlinear switching cost in online optimization with switching costs. 

An instance consists of a convex action set $\mathcal{K}\subset\mathbb{R}^d$, an initial point $y_0\in\mathcal{K}$, a sequence of non-negative convex cost functions $f_1,\cdots,f_T:\mathbb{R}^d\to\mathbb{R}_{\ge0}$, and a switching cost $c:\mathbb{R}^{d\times(p+1)}\to\mathbb{R}_{\ge0}$. To incorporate feedback delay, we consider a situation where the online learner only knows the geometry of the hitting cost function at each round, i.e., $f_t$, but that the minimizer of $f_t$ is revealed only after a delay of $k$ steps, i.e., at time $t+k$.  This captures practical scenarios where the form of the loss function or tracking function is known by the online learner, but the target moves over time and measurement lag means that the position of the target is not known until some time after an action must be taken. 
To incorporate nonlinear (and potentially nonconvex) switching costs, we consider the addition of a known nonlinear function $\delta$ from $\mathbb{R}^{d\times p}$ to $\mathbb{R}^d$ to the structured memory model introduced previously.  Specifically, we have
\begin{align}
c(y_{t:t-p}) = \frac{1}{2}\|y_t-\delta(y_{t-1:t-p})\|^2,    \label{e.newswitching}
\end{align}
where we use $y_{i:j}$ to denote either $\{y_i, y_{i+1}, \cdots, y_j\}$ if $i\leq j$, or  $\{y_i, y_{i-1}, \cdots, y_j\}$ if $i > j$ throughout the paper. Additionally, we use $\|\cdot\|$ to denote the 2-norm of a vector or the spectral norm of a matrix.

In summary, we consider an online agent that interacts with the environment as follows:
% \begin{inparaenum}[(i)] 
\begin{enumerate}%[leftmargin=*]
    \item The adversary reveals a function $h_t$, which is the geometry of the $t^\mathrm{th}$ hitting cost, and a point $v_{t-k}$, which is the minimizer of the $(t-k)^\mathrm{th}$ hitting cost. Assume that $h_t$ is $m$-strongly convex and $l$-strongly smooth, and that $\arg\min_y h_t(y)=0$.
    \item The online learner picks $y_t$ as its decision point at time step $t$ after observing $h_t,$  $v_{t-k}$.
    \item The adversary picks the minimizer of the hitting cost at time step $t$: $v_t$. 
    \item The learner pays hitting cost $f_t(y_t)=h_t(y_t-v_t)$ and switching cost $c(y_{t:t-p})$ of the form \eqref{e.newswitching}.
\end{enumerate}

The goal of the online learner is to minimize the total cost incurred over $T$ time steps, $cost(ALG)=\sum_{t=1}^Tf_t(y_t)+c(y_{t:t-p})$, with the goal of (nearly) matching the performance of the offline optimal algorithm with the optimal cost $cost(OPT)$. The performance metric used to evaluate an algorithm is typically the \textit{competitive ratio} because the goal is to learn in an environment that is changing dynamically and is potentially adversarial. Formally, the competitive ratio (CR) of the online algorithm is defined as the worst-case ratio between the total cost incurred by the online learner and the offline optimal cost: $CR(ALG)=\sup_{f_{1:T}}\frac{cost(ALG)}{cost(OPT)}$.

It is important to emphasize that the online learner decides $y_t$ based on the knowledge of the previous decisions $y_1\cdots y_{t-1}$, the geometry of cost functions $h_1\cdots h_t$, and the delayed feedback on the minimizer $v_1\cdots v_{t-k}$. Thus, the learner has perfect knowledge of cost functions $f_1\cdots f_{t-k}$, but incomplete knowledge of $f_{t-k+1}\cdots f_t$ (recall that $f_t(y)=h_t(y-v_t)$).

Both feedback delay and nonlinear switching cost add considerable difficulty for the online learner compared to versions of online optimization studied previously. Delay hides crucial information from the online learner and so makes adaptation to changes in the environment more challenging. As the learner makes decisions it is unaware of the true cost it is experiencing, and thus it is difficult to track the optimal solution. This is magnified by the fact that nonlinear switching costs increase the dependency of the variables on each other. It further stresses the influence of the delay, because an inaccurate estimation on the unknown data, potentially magnifying the mistakes of the learner. 

The impact of feedback delay has been studied previously in online learning settings without switching costs, with a focus on regret, e.g., \citep{joulani2013online,shamir2017online}.  However, in settings with switching costs the impact of delay is magnified since delay may lead to not only more hitting cost in individual rounds, but significantly larger switching costs since the arrival of delayed information may trigger a very large chance in action.  To the best of our knowledge, we give the first competitive ratio for delayed feedback in online optimization with switching costs. 

We illustrate a concrete example application of our setting in the following.

\begin{example}[Drone tracking problem]
\label{example:drone} \emph{
Consider a drone with vertical speed $y_t\in\mathbb{R}$. The goal of the drone is to track a sequence of desired speeds $y^d_1,\cdots,y^d_T$ with the following tracking cost:}
\begin{equation}
    \sum_{t=1}^T \frac{1}{2}(y_t-y^d_t)^2 + \frac{1}{2}(y_t-y_{t-1}+g(y_{t-1}))^2,
\end{equation}
\emph{where $g(y_{t-1})$ accounts for the gravity and the aerodynamic drag. One example is $g(y)=C_1+C_2\cdot|y|\cdot y$ where $C_1,C_2>0$ are two constants~\cite{shi2019neural}. Note that the desired speed $y_t^d$ is typically sent from a remote computer/server. Due to the communication delay, at time step $t$ the drone only knows $y_1^d,\cdots,y_{t-k}^d$.}

\emph{This example is beyond the scope of existing results in online optimization, e.g.,~\cite{shi2020online,goel2019beyond,goel2019online}, because of (i) the $k$-step delay in the hitting cost $\frac{1}{2}(y_t-y_t^d)$ and (ii) the nonlinearity in the switching cost $\frac{1}{2}(y_t-y_{t-1}+g(y_{t-1}))^2$ with respective to $y_{t-1}$. However, in this paper, because we directly incorporate the effect of delay and nonlinearity in the algorithm design, our algorithms immediately provide constant-competitive policies for this setting.}
\end{example}


\subsection{Related Work}
This paper contributes to the growing literature on online convex optimization with memory.  
Initial results in this area focused on developing constant-competitive algorithms for the special case of 1-step memory, a.k.a., the Smoothed Online Convex Optimization (SOCO) problem, e.g., \citep{chen2018smoothed,goel2019beyond}. In that setting, \citep{chen2018smoothed} was the first to develop a constant, dimension-free competitive algorithm for high-dimensional problems.  The proposed algorithm, Online Balanced Descent (OBD), achieves a competitive ratio of $3+O(1/\beta)$ when cost functions are $\beta$-locally polyhedral.  This result was improved by \citep{goel2019beyond}, which proposed two new algorithms, Greedy OBD and Regularized OBD (ROBD), that both achieve $1+O(m^{-1/2})$ competitive ratios for $m$-strongly convex cost functions.  Recently, \citep{shi2020online} gave the first competitive analysis that holds beyond one step of memory.  It holds for a form of structured memory where the switching cost is linear:
$
    c(y_{t:t-p})=\frac{1}{2}\|y_t-\sum_{i=1}^pC_iy_{t-i}\|^2,
$
with known $C_i\in\mathbb{R}^{d\times d}$, $i=1,\cdots,p$. If the memory length $p = 1$ and $C_1$ is an identity matrix, this is equivalent to SOCO. In this setting, \citep{shi2020online} shows that ROBD has a competitive ratio of 
\begin{align}
    \frac{1}{2}\left( 1 + \frac{\alpha^2 - 1}{m} + \sqrt{\Big( 1 + \frac{\alpha^2 - 1}{m}\Big)^2 + \frac{4}{m}} \right),
\end{align}
when hitting costs are $m$-strongly convex and $\alpha=\sum_{i=1}^p\|C_i\|$. 


Prior to this paper, competitive algorithms for online optimization have nearly always assumed that the online learner acts \emph{after} observing the cost function in the current round, i.e., have zero delay.  The only exception is \citep{shi2020online}, which considered the case where the learner must act before observing the cost function, i.e., a one-step delay.  Even that small addition of delay requires a significant modification to the algorithm (from ROBD to Optimistic ROBD) and analysis compared to previous work. 

As the above highlights, there is no previous work that addresses either the setting of nonlinear switching costs nor the setting of multi-step delay. However, the prior work highlights that ROBD is a promising algorithmic framework and our work in this paper extends the ROBD framework in order to address the challenges of delay and non-linear switching costs. Given its importance to our work, we describe the workings of ROBD in detail in Algorithm~\ref{robd}. 

\begin{algorithm}[t!]
  \caption{ROBD \citep{goel2019beyond}}
  \label{robd}
\begin{algorithmic}[1]
  \STATE {\bfseries Parameter:} $\lambda_1\ge0,\lambda_2\ge0$
  \FOR{$t=1$ {\bfseries to} $T$}
  \STATE {\bfseries Input:} Hitting cost function $f_t$, previous decision points $y_{t-p:t-1}$
  \STATE $v_t\leftarrow\arg\min_yf_t(y)$
  \STATE $y_t\leftarrow\arg\min_yf_t(y)+\lambda_1c(y,y_{t-1:t-p})+\frac{\lambda_2}{2}\|y-v_t\|^2_2$
  \STATE {\bfseries Output:} $y_t$
  \ENDFOR
   
\end{algorithmic}
\end{algorithm}

Another line of literature that this paper contributes to is the growing understanding of the connection between online optimization and adaptive control. The reduction from adaptive control to online optimization with memory was first studied in \citep{agarwal2019online} to obtain a sublinear static regret guarantee against the best linear state-feedback controller, where the approach is to consider a disturbance-action policy class with some fixed horizon.  Many follow-up works adopt similar reduction techniques \citep{agarwal2019logarithmic, brukhim2020online, gradu2020adaptive}. A different reduction approach using control canonical form is proposed by \citep{li2019online} and further exploited by \citep{shi2020online}. Our work falls into this category.  The most general results so far focus on Input-Disturbed Squared Regulators, which can be reduced to online convex optimization with structured memory (without delay or nonlinear switching costs).  As we show in \Cref{Control}, the addition of delay and nonlinear switching costs leads to a significant extension of the generality of control models that can be reduced to online optimization. 

\section{Normal mode}\label{sec:normal}
% !TEX root = XStreamFull.tex

During \XSCC normal mode, at any \XStream tick exactly one
processor is in a state of \emph{building}.
This building processor, $p_B$, accepts new edges, maintains connected
components with a \uf data structure, and stores spanning tree edges.
\XSCC normal mode maintains two key invariants, stated below and
illustrated in Figure~\ref{fig:xstream-normal}.
\begin{invariant}
Let $p_B$ be the building processor.  Then $p_i$ is completely full of
spanning tree edges for $0 \le i < B$ at all times, and $p_i$ has no
spanning tree edges for $i > B$. \label{inv:normal-1}
\end{invariant}

When $p_B$ fills with tree edges, a building ``token'' is passed downstream
to $p_B$'s successor, which assumes building responsibilities.  Thus, \XSCC
maintains a spanning forest of the input graph, packed into prefix processors
$\{p_0, \ldots, p_B\}$.
The \XSCC normal mode protocols maintain Invariant~\ref{inv:normal-1} and one other:

\begin{invariant}
Let $p_S$ be the first processor with any empty space.  Then $p_i$ is
completely full of edges for $0 \le i < S$ at all times, and $p_i$ has no
tree or non-tree edges for $i > S$. \label{inv:normal-2}
\end{invariant}

%\Jon{Blue invariant is used for arguing no duplicate storage.  We don't have
%a corresponding lemma yet, but should.}

\setcounter{algorithm}{0}  % -1
\begin{algorithm*}
\caption{This diagnostic routine is helpful for understanding correctness;
it would never be called in practice.  \label{algo:dump} 
This assumes the \WStream convention of choosing a vertex representative to name each 
supernode.
}
\XSCC diagnostic: dump connected components \\
\makealgtitle
\begin{algorithmic}[1]
\LeftComment{Precondition: the input stream has stopped. This never happens during normal operation.}
\LeftComment{Description: Processor $p_T$ emits correct finite stream connected components output.}
\Procedure{DumpComponentLabels}{$p_i$}
\LeftComment{Relabel all \Call{DumpComponentLabels}{} output from upstream}
          \While {\Call{Receive}{$u$, $R_{i-1}(u)$}}\label{lab:startRelabel}
\State        \Call{Emit}{$u$, $R_{i}(u)$}\label{lab:endRelabel}
          \EndWhile
\LeftComment{Now emit each union-find relationship}
          \For {$b_x : \nodeloc(b_x) = p_i$}   \Comment{$\nodeloc$ is the supernode relationship; see Definition~\ref{def:alpha}}\label{lab:startUFdump}
\State        \Call{Emit}{$b_x$, $R_{i}(b_x)$} \label{lab:endUFdump}
          \EndFor
\EndProcedure
\end{algorithmic}
\end{algorithm*}

Invariants ~\ref{inv:normal-1} and \ref{inv:normal-2} are illustrated in
Figure~\ref{fig:xstream-normal}, with sets of spanning tree edges represented
in red, and sets of non-tree edges represented in blue. In normal mode 
operation, single edges arrive at each X-Stream tick and propagate downstream
to the builder, being relabeled along the way. They settle into $p_B$ if 
they are
found to be tree edges, and into $p_S$ otherwise.  The figure
shows the system at \XStream tick $t+4$.  In this notional example, the
edge that arrived in the previous tick has passed through the head processor
$p_H$, but has not yet been resolved as ``tree'' or ``non-tree.''  Edge
$e_{t+2}$ has passed through two processors, and relabeling of the endpoints
has identified it as a non-tree edge. The basic protocol is thus quite simple;
the subtleties of \XSCC normal operation arise in maintaining the invariants. For example, the
builder may need to jettison non-tree edges downstream to make room for new tree edges.
We provide full detail in Section~\ref{sec:pseudocode}.

\subsection{\XSCC normal mode correctness}
We now show that Invariant~\ref{inv:normal-1} and
\XSCC relabeling implies an exact correspondence between
the connectivity structures computed by \XSCC and \DFRns.  
%\Cindy{Maybe delete this next sentence? We don't talk about maintaining the invariants until the pseudocode sections and the discussion of the bulk delection is in the next section.  This is the lead in to what's in this subsection.} 
%After that
%we will detail the mainentance of the invariants and our bulk deletion
%operation.

Algorithm~\ref{algo:dump} is a diagnostic routine intended to test
implementations of \XSCCns. At
\XStream time $t$, a call to this routine streams out the connected
components of the active graph $G_t$ as a stream of (vertex, label)
pairs.  Although we could correctly stream these $|V_t|$ vertex pairs
out even as new edges change the connected components (see
Section~\ref{sec:non-constant} for additional algorithm steps)\Jon{TODO: there
is nothing yet about this in Section~\ref{sec:non-constant}}, this version is more illustrative.
%is strictly for debugging. It's not, really - it's for correctness args too. 
For simplicity we assume that the input stream pauses at time $t$.

% JWB: using ``dump components'' rather than dump-components since LaTeX
% is struggling to handle line breaks with the latter
When head processor $p_0$ receives the ``dump components'' command in a primary slot, it copies the command to the primary slot of the \bundle it will emit. 
Then, in Lines~\ref{lab:startUFdump}-\ref{lab:endUFdump} of Algorithm~\ref{algo:dump}, processor $p_0$ fills the remaining $k-1$ payload slots with relationships from its union-find structure, the way $\DFR$ outputs union-find information into the $B$ streams. Specifically, if $b_x$ is the name of a building block encapsulated by local component $b_y$ in $p_0$ (i.e. $\beta(b_x) = p_0$), then processor $p_0$ outputs a pair $(b_x, b_y = R_i(b_x))$, where $R_i(b_x)$ is the encapsulating supernode, the result of processor $p_0$ relabeling block $b_x$.  Processor $p_0$ then fills the payload slots of subsequent bundles until it has output all of its union-find relationships.

For downstream processor $p_i$ ($i > 0$), the \bundle that has the ``dump components'' command has $(u,R_{i-1}(u))$ pairs in the payload slots. In Lines~\ref{lab:startRelabel}-\ref{lab:endRelabel} of Algorithm~\ref{algo:dump}, processor $p_i$ relabels any block names $u$ that have been encapsulated by a new supernode in $p_i$. Otherwise $R_i(u) = R_{i-1}(u)$. After all relationships from upstream have arrived, there is empty payload for $p_i$ to output its union-find information as described above.

We now argue the correctness of \XSCC based on the correctness of
W-Stream. Let $D_i$ denote the output of processor $p_i$ from this diagnostic.  For the formal arguments, we require the following definitions:

\begin{definition}
During a union operation joining sets with representatives $b_x$ and $b_y$,
the \emph{supernode naming function} is $\eta: {\cal B} \times {\cal B} \rightarrow {\cal B}$ such
that $\eta(b_x,b_y)$ decides
whether $b_x$ or $b_y$ becomes the new set representative.
\end{definition}
For example, we
might choose a supernode naming function $\eta(b_x,b_y) = \min(b_x,b_y)$. This
is the function used in Figure~\ref{fig:wstream-example}.

\begin{definition}
\label{def:alg-with-params}
$\DFRns(s, \eta, A)$ is an implementation of \DFR with
processor union-find capacity $s$ and supernode naming function
$\eta$ run on input stream $A$. 
% can't seem to use macros here; it interferes with line wrapping
%$\XSCCns(s,\eta,A)$ is
XS-CC$(s,\eta,A)$ is
defined similarly for XS-CC, with each processor's union-find
capacity set to $s$.
\end{definition}

A \emph{resolved} edge is one that has been classified as ``tree''
or ``non-tree.''\ Stream edges arrive
in an \emph{unresolved} state. 
In \DFRns, the stream $A_i$ written from pass $i$ contains only those edges that resolve to ``tree'' edges
%\Jon{In previous text we had said ``unresolved'' here, but DFR emits only resolved edges (and drops non-tree edges (i.e., the resolution decision has been made)).}
(they connect supernodes in the current version of the contracted graph).  \DFR deletes as ``non-tree'' any edge that it determines to be contained inside a supernode.  
In contrast, \XSCC must retain all non-duplicate edges,
even after resolution.  In particular, non-tree edges
must be retained in case they are needed to reconnect pieces of the 
graph after bulk deletion. 
\XSCC removes duplicate edges from the stream after updating their 
timestamps.

By Invariant~\ref{inv:normal-1}, all unique non-tree edges (those contracted inside a supernode) are stored at the end of the \XSCC data structure in spare space in the builder, or in processors downstream of the builder. These downstream
processors have no
union-find structure. The following lemma ignores known non-tree edges.

\begin{lemma}
The stream of unresolved edges sent from processor $p_i$ to $p_{i+1}$ in 
$\XSCCns(s,\eta,A_0)$ is exactly $A_i$ from
$\DFRns(s,\eta,A_0)$.
\label{lem:A-stream}
\end{lemma}

\begin{proof}
We prove this lemma by induction.  For the base case, the first pass of \DFR and the first processor of \XSCC receive the same same finite stream of unresolved edges from the outside (logical processor $p_{-1}$), namely the input stream of edges $A_0$. Suppose that the stream of unresolved edges sent from processor $p_{i-1}$ to processor $p_i$ is the same as stream $A_{i-1}$ from \DFR.  We show that the stream of unresolved edges processor $p_i$ sends to $p_{i+1}$ is exactly \DFR stream $A_i$.

Processor $p_i$ of \XSCC and pass $i$ of \DFR begin by computing connected components via union-find.  Every edge that changes connectivity (starts a new component or joins two components) uses one of the $s$ possible union operations for this processor/pass. When they have both done $s$ union operations (their capacity), they have computed identical union-find data structures since they have done the same computations on the same input stream.  At this point, \DFR has not yet emitted any edges and \XSCC has emitted only resolved non-tree edges. Now \DFR processes the remaining edges of $A_{i-1}$, relabeling the endpoints, deleting edges where both endpoints are contained in the same supernode, and emitting the others to stream $A_i$. \XSCC relabels these remaining edges the same way, and emits the same stream of unresolved edges (among resolved non-tree edges).\qed
\end{proof}

Since \XSCC runs on unending streams, there is no ``end of stream'' mark to trigger creation of and processing of an \DFR $B_i$ stream.  However, the ``dump components'' diagnostic creates these $B$ streams.

\begin{lemma}
For $\XSCCns(s,\eta,A_0)$ followed by a call
to $\Call{DumpComponentLabels}{}$, stream $D_{i-1}$ is identical to
stream $B_i$ from $\DFRns(s,\eta,A_0)$.
\label{lemma:B-stream}
\end{lemma}
\begin{proof}
The ``dump components'' command after ingestion of a finite stream serves as an end-of-stream marker for \XSCC. We prove the lemma by induction. For the base case, streams $B_0$ and $D_{-1}$, the component information input to pass $0$ of \DFR and processor $p_0$ of \XSCC respectively are both empty. Suppose that stream $B_i$ from $\DFRns(s,\eta,A_0)$ is the same as stream $D_{i-1}$ from 
%$\XSCCns(s,\eta,A_0)$ followed   macro messes LaTeX up.
XS-CC$(s,\eta,A_0)$ followed
by a call to\\ $\Call{DumpComponentLabels}{}$. We show that stream $B_{i+1}$ is the same stream $D_i$. From the proof of Lemma~\ref{lem:A-stream}, the runs of \XSCC and \DFR compute the same connected components in processor $p_i$ and pass $i+1$ respectively. Because \XSCC and \DFR are using the same supernode naming function and have the same capacity, the union-find data structures (names of representatives and names of set elements) are identical. As described above, processor $p_{i}$ relabels and emits all elements of $D_{i-1}$ the same way that \DFR pass $i+1$ relabels elements stream $B_i$ to stream $B_{i+1}$. Then processor $p_i$ and \DFR pass $i+1$ output the information in their identical union-find structures in identical ways, completing streams $D_i$ and $B_{i+1}$ respectively. \qed
\end{proof}

\iffalse
\begin{lemma}
Suppose the capacity of \WStream and \XStream processors are configured equivalently: they
both can perform the same number of union operations.
The connectivity structure output by W-Stream pass $i$ ($G_{B_i}$) is
exactly the same as that stored in \XStream processors
$\{p_0,\ldots,p_{i-1}\}$ after processing edge stream $G_t$.
\label{lemma:wx-correspondence}
\Jon{change statement to accomplish: relating \XSCC relabeling to \DFR $G_B$ graphs.  Also, use induction from $i$ to $i+1$ to get rid of $i-2$ 
usage.}
\Jon{either here or in a subsequent lemma, argue about any permutation of $G_t$}
\end{lemma}
\begin{proof}
(induction on $i$) \emph{Base case}: When $i=1$, the same union
operations occur in both \DFR and \XSCCns.  W-Stream emits connectivity
structure $G_{B_1}$, but \XSCC retains a spanning forest of that
structure in processor $p_0$.
\emph{Induction}:  Suppose that the
\DFR connectivity structure $G_{B_{i-1}}$ is the same as that stored in
\XStream processors $\{p_0,\ldots,p_{i-2}\}$.  Note that
$R_{i-2}(u) = R_{i-2}(v)$
iff vertices $u$ and $v$ are in the same connected component of $G_{B_{i-1}}$.
During pass $i$ of W-Stream, $s$ union operations occur and \XStream
processor $p_{i-1}$ is the builder.  Each
stream edge $(u,v)$ connects components of $G_{B_{i-1}}$ iff
$R_{i-1}(u) \neq R_{i-1}(v)$. \XStream processor $i-1$ has space to
store $(u,v)$ and by Invariant~\ref{inv:normal-1}, any previously-stored
non-tree edges can be sent downstream to ensure that the processor can
use its entire capacity for connectivity information. Therefore, \XStream 
processors $0,\ldots,i-1$ store the same connectivity structure as $G_{B_i}$. $\qed$
\end{proof}
\fi

\subsection{\XSCC queries}
The most basic \XSCC query is a connectivity query: are nodes $u$ and $v$ in
the same connected component? A query that arrives at \XStream tick $t$ will
be answered with respect to the graph $G_t$.  The query $(u,v)$ enters
the system from the I/O processor and propagates through the processors
just as new stream edges do.  Each processor relabels the endpoints,
and the tail processor returns ``$(u,v)$:yes'' if the labels are the same and ``$(u,v)$:no''
otherwise.  This holds even if  one or both of the endpoints have
never been seen before.
%Theorem~\ref{thm:query-correctness}
% leverages our previous definitions and Lemma~\ref{lemma:wx-correspondence} to
%shows query correctness
% at single-edge granularity.
% Recall the difference between transit edges and settled
% edges.\Jon{if this is the only place the concept is used, move the definition here.
% Otherwise...}
The following theorem shows that connectivity queries are correct at single-edge granularity, and therefore
that \XSCC in normal mode correctly computes the connected components of an
edge stream.

\begin{theorem}
Suppose that the connectivity query $(u,v)$ arrives at the head processor of an X-Stream
system with $P$ processors at X-Stream tick $t$. Then the I/O processor will
receive the boolean query answer at time $t+P$.  The answer will be \emph{True}
iff $u$ was connected to $v$ in $G_t$, the logical graph that existed at time $t$.
\label{thm:query-correctness}
\end{theorem}
\begin{proof}
Recall that $p_B$ is the building processor.
The query answer will be determined by
X-Stream tick $t+B$ at the latest, since, by Invariant~\ref{inv:normal-1},  
$p_B$ is the last
one to store any tree edges and hence, any union-find information.  Thus it is the 
last processor that can change a label. The $(u,v)$ query travels processor-to-processor in a primary basket, just as
the dump-components command does. If there are any transit edges in $G_t$ when the query arrives, they travel in slots of bundles strictly ahead of the query. Thus transit edges will settle into a processor before the query arrives.  Similarly, any edges that arrive after the query travel in bundles strictly behind the query and cannot affect the query relabeling. Thus when the bundle with the query arrives at processor $p_i$, the union-find data structure, and the processor's status as the builder or not, are set exactly according to the graph $G_t$ in the system when the query arrived.

Processing query $(u,v)$ is closely related to processing $\Call{DumpComponentLabels}{}$. Instead of dumping information for every vertex, starting at the point where a vertex is first encapsulated in a supernode, simple queries only consider two vertices. The label for $u$ will only change from $u$ to a supernode label $b_y$ at the processor $p_i$ that first incorporates $u$ into a local component ($p_i = \beta(u)$).  In $\Call{DumpComponentLabels}{}$, processor $p_i$ is the first that outputs any pair $(u, b_y)$, with first component $u$ into the stream $D_i$.  Thus, after the query has passed the building processor, the labels for vertices $u$ and $v$ are identical to their output values, which exit the system at time $t+P$. By Lemma~\ref{lemma:B-stream}, these are the same labels they would have if \DFR is run on graph $G_t$.  Because \DFR is a correct connected components algorithm, vertices $u$ and $v$ will have the same label if an only if they are in the same connected component. \qed
\end{proof}

We call queries that \XSCC answers with latency $p$ \emph{constant} queries.
See section~\ref{sec:non-constant} for examples of non-constant queries.

%\Jon{Need non-duplication theorem:  O(1) space per edge}
The next theorem shows that \XSCC is space-efficient, storing the current graph in asymptotically optimal space.
\begin{theorem}
\label{thm:non-dup}
In normal operation of \XSCCns, each edge is stored in exactly one processor,
requiring $O(1)$ space.
\end{theorem}
\begin{proof}
%The only duplication scenario is when a copy of an edge exists downstream
%a processor that has empty space.
In normal operation, when a new edge $e$ arrives at a processor $p_i$ that
already stores a copy of $e$, processor $p_i$ removes $e$ from the stream and
updates the timestamp of $e$.
Invariant~\ref{inv:normal-1} ensures that incoming tree edge $e$ encounters any
previously-stored copies of itself before it reaches $p_B$, the building
processor, which recognizes it as a tree edge.
Invariant~\ref{inv:normal-2} ensures that incoming non-tree edge $e$
encounters any previously-stored copies of itself before it reaches $p_S$,
the first processor with any empty space. Furthermore, this invariant also
ensures that there are no edges stored downstream of $p_S$.  
\qed
\end{proof}

% CAP: save previous proof for now.
\iffalse
Case (I): there are no transit edges in the system. Then neither
the building processor designation $p_B$ nor the logical graph $G(t)$ change between query
entry time $t$ and time $t+B$ (when the query arrives at the builder).
By Lemma~\ref{lemma:wx-correspondence}, $u$ and $v$ are connected iff
$R_{p_B}(u) = R_{p_B}(v)$.  Further relabeling does not change this result, and the correct
answer will propagate to the I/O processor at tick $t+P$.
Case (II): there are transit edges in the system at X-Stream tick $t$.  We must argue that
any transit edges that affect connectivity have time to settle (a prerequisite for
invoking Lemma~\ref{lemma:wx-correspondence}) before the query arrives at the building processor.
Noting that the building processor might advance as the query propagates, let $B'$ be
the index of the current builder at time the query arrives there.
Since the
storage capacity $s$ always exceeds $P$,  the building processor will be either $p_{B'-1}$
(if the builder designation advanced) or
$p_{B'}$ when the query reaches it.  In either case, Invariant~\ref{inv:normal-1} ensures
that no tree edge is downstream of the current builder. Therefore, any transit edge will arrive at
the builder at time $t' = t + B' - 1 \le t+B-1$.  This is at least one tick before the query
arrives.  At that point, Case (I) applies. $\qed$
\fi



Theorem~\ref{thm:query-correctness} shows that basic connectivity queries are
answered correctly by \XSCCns.  In Section~\ref{sec:non-constant}
we informally discuss three additional types of feasible queries: complex queries such as
finding all vertices not in the giant component of a social network \Cindy{Can we ask for the maximum size of any connected component in a constant query? I think so, since every union-find component has a vertex count.  This gives us the size of the giant component.}\Jon{Last call for adding this.}, vertex-based
queries like finding the degree or neighborhood, and diagnostic
queries regarding system capacity used.  Also, by Invariant~\ref{inv:normal-1},
X-Stream always knows a spanning tree of the streaming graph by construction. This
tree could be checkpointed, for example, if processors share a filesystem. \Cindy{The information in the spanning tree is about the size of the output of dump components.}\Jon{Wasn't sure what to do about this comment.}
\iffalse
\begin{algorithm*}
\caption{Process Connectivity Query on Processor $p_a$}\label{algo:connectq}
Process Connectivity Query on Processor $p_a$\\\makealgtitle
\begin{algorithmic}[1]
\Procedure{ConnectedQuery}{$([v,R_{a-1}(v)],[u,R_{a-1}(u)],answer)$}
\If{answer = false}
	\State $R_a(v) \gets \Call{Find\_Set}{R_{a-1}(v)}$
	\State $R_a(u) \gets \Call{Find\_Set}{R_{a-1}(u)}$
\If{$R_a(v) = R_a(u)$}
\State answer = true
\EndIf
\EndIf
\State \Call{SendDownstream}{QUERY $([v,R_a(v)],[u,R_a(u)], answer)$}\Comment{we don't care about updating label if answer is already true}
\State \Return
\EndProcedure
\end{algorithmic}
\end{algorithm*}






\subsection{Full X-Stream Normal Mode Algorithm}\label{sec:algo:normal}
We next describe how to expand the implementation of W-Stream UCONN in X-Stream so that we will later be able to age correctly and as a result, handle infinite streams. Algorithm~\ref{algo:normaledge1} describes the method of processing an edge on each processor in normal mode, using Algorithm~\ref{algo:unrollwstream} (which simulates W-Stream) as a subroutine. The main difference is that we now keep non-tree edges, which may effect connectivity after aging. As a result, we also have to verify that an edge does not already exist somewhere in the system before it can be stored. This is done with lap counter metadata that is passed around the ring along with the edge.
\begin{algorithm*}
\caption{Process Edge in Normal Mode on Processor $p_a$}\label{algo:normaledge1}
Process Edge in Normal Mode on Processor $p_a$\\\makealgtitle
\begin{algorithmic}[1]
\LeftComment{\Call{Forward}{} and \Call{Store\_Or\_Forward}{} include relabeling \Alex{but not if already non-tree edge?}}
\LeftComment  \Call{Find\_Set}{$v$} returns the representative node in the union find structure for the set containing $v$. The structure is initialized such that \Call{Find\_Set}{$v$} returns NULL for all $v$.

\Procedure{ProcessEdge}{$e=(([u,R_{a-1}(u)],[v,R_{a-1}(v)],t'),(c,p))$}
\If{c=0}
	\If {\Call{ProcessTreeEdge}{$e$,$S_u$,$S_v$}}
	\State  \Return
	\Else
	  	\State $(c,p) \gets (1,a)$ 
		  \State \Call{Forward}{e}
 		 \State \Return
	\EndIf
\EndIf
\If {$(c,p)=(2,a)$}
 \State \Call{Full\_Stop}{} \Comment{system is completely full}
\EndIf

\If{$c = 1$}  
  \If{$p=a$} $(c,p) \gets (2,a)$ 
  \Else
    \If {Timestamps.lookup(e)}  Timestamps.update(e,t')
    \Else \ \Call{Forward}{e}  \Comment{first lap not complete, continue checking for duplicates}
    \EndIf
    \State  \Return
  \EndIf
\EndIf
\LeftComment{process non-tree edge with $c=2$}
\State \Call{Store\_Or\_Forward}{$e$,$t'$}

\EndProcedure

\Procedure{Store\_Or\_Forward}{$e$,$t'$}
\If{Timestamps.lookup(e)} 
  \State Timestamps.update(e)
  \State \Return
\EndIf
\If {HAVE\_SPACE}{\mbox{}}  \Comment{number of stored edges is less than capacity}
  \State Timestamps.insert($e$,$t'$)  \Comment{and remove from basket (we haven't been including that}
  \State \Return
\Else 
  \State \Call{FORWARD}{$e$}
\EndIf
\EndProcedure

\end{algorithmic}
\end{algorithm*}














\subsection{X-Stream Implementing W-Stream Correctness}
We first show that executing Algorithm~\ref{algo:unrollwstream} computes the same intermediate pass streams computed by the algorithm in~\cite{} for solving undirected connectivity (UCON) in the W-Stream model. We will show that the correctness of query answers using Algorithm~\ref{algo:connectq}) follows. Suppose we have an instance of X-Stream with processors with storage space $s$, and the input graph has $n$ vertices. Solving UCON in W-Stream requires $\Omega(n/s)$ passes (\cite{AMP:aggarwal2004streaming} Theorem 2.1). We assume that we have $p$ processors such that UCON can be solved in $p$ passes. Recall from Section~\ref{sec:intro:wstream} that $G_{A_i}$ is the first part of the stream outputted by pass $i$ and $G_{B_i}$ is the second part of the stream.
\begin{observation}\label{obs:ga}
The set of a relabeled edges emitted by processor $i$ in an X-Stream system induce the graph $G_{A_i}$. However, the endpoints use the set labels constructed in the X-Stream algorithm rather than reusing vertex labels as set labels.
\end{observation}


\begin{observation}\label{obs:gb}
The \uf structures upstream from processor $i$ contain the same information as $G_{B_i}$. Thus $G_{B_i}$ could be reconstructed by letting all primitive vertices and local component labels found on $p_0,...,p_{i-1}$ be $V(G_{B_i})$ and then letting the label pairs $(x,R_i(x))$ be the edges of $G_{B_i}$.
\end{observation}

Invariant~\ref{inv:wstreamreqs} holds for the values of $G_{A_i}$ and $G_{B_i}$ since it holds for the algorithm given by~\cite{AMP:aggarwal2004streaming} to solve UCON in W-Stream.

We also observe the property that 
every processor sees input stream elements in order. Thus when a query arrives at a processor, it has seen all edges. We can now show that:
 \begin{lemma}
 Given an input graph $G$, consider an X-Stream system consists of processors with storage capacity $s$ and a sufficient number of processors, $p$, such that the W-Stream UCON algorithm requires at most $p$ passes on $G$. Suppose the X-Stream system executes Algorithm~\ref{algo:unrollwstream} to process $G$ as an input stream. Then
connectivity queries processed by the X-Stream system using Algorithm~\ref{algo:connectq} will be answered correctly. \end{lemma}
\begin{proof}
\Alex{main todo is to show query ordering is ok so correctness follows from invariant like in W-Stream}
Consider processor $p$, the final processor in the ring. $G_{A_p}$ will be null because processor $p$ does not emit any edges during the execution of Algorithm~\ref{algo:unrollwstream}. 
$G_{B_p}$ is a set of stars which fully describes the vertices contained in each component on processor $p$. Thus correctness follows from the correctness of W-Stream.
\end{proof}


\subsection{Normal Mode Correctness}
We first describe the correctness during normal mode. We must argue that a connectivity query ``Are vertices $u$ and $v$ connected?" will always be answered correctly. During normal mode, the following properties hold:

\begin{enumerate}
\item If an edge has been added to a data structure in some processor it is only stored once. All edges in the system are either = \emph{in transit} or \emph{settled} in exactly one processor.   \label{prop:edgeonce}
\item The elements of the \uf structure in each processor are unique, i.e. do not appear in the \uf on any other processor. \Alex{note that the UF structure doesn't contain nodes/elements that are LC's, just BB's}  \label{prop:onelc}
\item The set of all  tree edges over all processors is a spanning forest of the logical graph. \Alex{Need a def of stored graph: settled tree edges are spanning forest of the subgraph of the logical graph $G(t)$ comprised of edges that are settled  at $t$ - define $G_s(t)$?}\label{prop:nontree}
\end{enumerate}

\begin{lemma}\label{lemma:normalprops}
Properties $1-3$ hold up until normal mode ends for the first time. \Alex{i.e., when the first aging starts}
\end{lemma}
\Alex{

\begin{proof}
Property 1:\\
We induct on the sequence of edges received. Initially, no edge is stored so the property holds. Now suppose the property holds at time $t$ and a new edge $e = ([u,b_u],[v,b_v],t)$
arrives. If $e$ is received by a ring processor which holds an edge $e'$ with the same endpoints as $e$, i.e. $e' = ([v,b_v],[u,b_u],t')$, the processor replaces $t'$ with $t$ and removes $e$ from the stream, so it will not be stored as a duplicate. Thus we show if processor $p_s$ accepts edge $e$ as a new edge at time $t$, no such $e'$ with the same endpoints previously existed in the ring.

Case 1: Suppose processor $p_s$ accepts $e$ as a storage edge. Then $e$ must have marker pair $(c,p_i)$ such that $c=2$. If $c=2$, then $e$ was processed by every processor in the ring at least once: processors $p_1,...,p_i$ saw it when $c=0$, processors $p_{i+1},...,p_i$ (mod ring size) saw it  when $c= 1$, and now processors $p_{i+1},...,p_s$ (mod ring size) saw it  when $c=2$. If the same $(u,v)$ had previously been stored on any processor, $e$ would have been consumed by that processor and the timestamp updated to $t$.

Case 2: Suppose processor $p_s$ accepts $e$ as a tree edge. Then since no $p_i$ with $i<s$ removed $e$ from the stream, no such $p_i$ has an edge $e'$ with the same endpoints. Consider if some processor $p_i$ with $i>s$ had such an $e'$. Since $p_s$ is filling, $e'$ would have to be a storage edge on $p_i$ for some $p_j$, $j\leq s$. But then $e$ would be relabeled as storage on $p_j$ as well, and not reach $p_s$ as a tree edge. Thus no $p_i$, $i\neq s$ has and $e'$ with the same ends, and $p_s$ can accept $e$ if $p_s$ itself does not have an edge $e'$.
\\\\Property 2: We induct on the processor index, showing processor $p_i$'s \uf sets contain no elements used in the \uf sets on $p_0,...,p_{i-1}$. Clearly this holds for $p_0$. Now suppose  the union-find structures in $p_0,...p_i$ contain disjoint element sets and we show $p_{i+1}$ will not reuse any of these elements. When an edge is stored by a processor, its relabeled ends are uses as elements in the union-find structure. Consider the labels on any edge $e=([u,R_i(u)],[v,R_i(v)],t)$ sent from $p_i$ to $p_{i+1}$. If  $R_i(u) = R_i(v)$ it is a storage edge and nothing is added to the union find structure. If $R_i(u) =u$ then no processor $p_0,...,p_i$ has any edges containing $u$, or the primitive BB corresponding to $u$ in their union-find structures. In any other case, $R_i(u)$ is set name on some processor $j \in \{0,...,i\}$.

\end{proof}
}
\begin{proof}
Property 1: We induct on the sequence of edges received. Initially, no edge is stored so the property holds. Now suppose the property holds at time $t$ and a new edge $e = ([v,b_v],[u,b_u],t)$
arrives. If $e$ is received by a ring processor which holds an edge $e'$ with the same endpoints as $e$, i.e. $e' = ([v,b_v],[u,b_u],t')$, the processor replaces $t'$ with $t$ and removes $e$ from the stream, so it will not be stored as a duplicate. Thus we show if processor $p_s$ accepts edge $e$ as a new edge at time $t$, no such $e'$ with the same endpoints previously existed in the ring.

Case 1: Suppose processor $p_s$ accepts $e$ as a storage edge for processor $p_j$. Then when $e$ reached $p_j$ on its first lap around the ring, it would have been marked as storage and had $first\_lap$ set to true. Then $p_s$ would not be accepting $e$ until after $e$ had completed a lap from the head to the tail and had $first\_lap$ set to false. Thus no $e'$ with the same endpoints existed, or a $e$ would not have completed that whole lap.

Case 2: Suppose processor $p_s$ accepts $e$ as a tree edge. Then since no $p_i$ with $i<s$ removed $e$ from the stream, no such $p_i$ has an edge $e'$ with the same endpoints. Consider if some processor $p_i$ with $i>s$ had such an $e'$. Since $p_s$ is filling, $e'$ would have to be a storage edge on $p_i$ for some $p_j$, $j\leq s$. But then $e$ would be relabeled as storage on $p_j$ as well, and not reach $p_s$ as a tree edge. Thus no $p_i$, $i\neq s$ has and $e'$ with the same ends, and $p_s$ can accept $e$ if $p_s$ itself does not have an edge $e'$.

Case 3: Suppose processor $p_s$ accepts $e$ as mortgage.
\end{proof}


%\begin{proof}
%Property 1: We induct on the sequence of edges received. Initially, no edge is stored so the property holds. Now suppose the property holds at time $t$ and a new edge $e = ([v,b_v],[u,b_u],t)$ arrives.

%Case 1: Suppose no other edge $e' = ([v,b_v],[u,b_u],t')$ exists in the system, so $e$ is the first instance of an edge with ends $u$ and $v$.

%Case 2: Next suppose $e$ already exists in the system. When a processor receives an edge it already holds, it removes it from the stream and updates the time stamp, so the edge cannot be stored a second time on different processor. Thus we show the new instance of $e$ always reaches the processor already holding $e$:

%Case 2a: Suppose $e$ is a tree or mortgage edge on some porcessor $p_s$. Let $p_f$ be the filling processor, so $f\geq s$. Then each processor $p_i$, for $i<s$, is sealed and will not take $e$ as a tree or mortgage edge. Also, relabeling ends of $e$ will give $R_i(v) \neq R_i(u)$, or the previous instance of $e$ would have also been relabeled with $R_i(v) =R_i(u)$, making $e$ a storage edge and not a tree or mortgage edge for $p_s$. Thus any new copy of edge $e$ will reach $p_s$.

%Case 2b: Suppose $e$ is a storage edge for some processor $p_s$, held by some processor $p_j$, $j\neq s$. As before, if $p_f$ is filling, $f\geq s$. Since each $p_i$ with $i<s$ is sealed it cannot accept $e$ as a tree or mortgage edge. Then if $j<s$, $e$ reaches $p_j$. Otherwise, $e$ reaches $p_s$, at which point it is labeled as storage with $R_s(u) = R_s(v)$ and on its first lap. Then $e$ cannot be taken by a processor other than $p_j$ before it reaches the tail to finish the lap, so $e$ will reach $p_j$.

%Properties $2-4$: omitted for space.
%\end{proof}

\begin{theorem}\label{theorem:propstoquery}
If properties $1-4$ hold for $G_t$, then there exists a processor $p$ that gives vertices $u$ and $v$ the same label at time $t$ if and only if the vertices are in the same connected component in $G_t$. Equivalently, $R_m(v) = R_m(u)$ if and only if there exists a $u,v$-path in $G_t$.
\end{theorem}
\begin{proof}
Let $E_t^i$ be the set of tree edges on processors $p_1,...,p_i$ at time $t$. Let $G_t^i = (V_t,E_t^i)$. Then $G_t^i$ is subgraph of $G_t^{i+1}$ for $i = 0,...m-1$. By Property~\ref{prop:nontree}, $G_t^m$ is a spanning forest of $G_t$. By Property~\ref{prop:oneprimitive}, the primitive building block $b_u \in B_v$ for vertex $u$ is unique so only one possible relabeling $R_i(u)$ corresponds to vertex $u$. We show by induction that there exists a $u,v$-path in $G_t^i$ if and only if $R_i(u) = R_i(v)$. At $p_1$, all building blocks are primtive, so $R_1(u) = R_1(v)$ iff there is a $u,v$-path in $G_t^1$.
Now suppose at $p_i$, $R_i(u) = R_i(v)$ iff there exists a $u,v$-path in $G_t^i$. We show this implies the same for $p_{i+1}$:

Case 1: Suppose $R_i(u) =R_i(v)$. Then $R_{i+1}(u)= R_{i+1}(v)$ and since $G_t^i$ is a subgraph of $G_t^{i+1}$, the $u,v$-path in $G_t^i$ is also in $G_t^{i+1}$.

Case 2: Suppose $R_i(u)\neq R_i(v)$. Then $R_{i+1}(u) = R_{i+1}(v)$ iff $R_i(u)$ and $R_i(v)$ are BBs in the same LC on $p_{i+1}$, by Property~\ref{prop:onelc}. By definition of LC, $R_i(u)$ and $R_i(v)$ are in the same LC iff there exists a sequence of edges $(x_1,y_1),(x_2,y_2),...(x_l,y_l)$ on $p_{i+1}$ with $x_j,y_{j-1}$ in the same BB for $j=2,...,l$ and $x_1 \in R_i(u)$ and $y_l \in R_i(v)$. By induction there exists a $x_j,y_{j-1}$-path in $G_t^{i+1}$ for $j=2,...,l$, so there exists a $x_1,y_l$-path in $G_t^{i+1}$. By induction we also have a $u,x_1$-path and a $y_l,v$-path in $G_t^{i+1}$ so all together, there is a $u,v$-path. Thus $R_i(u)$ and $R_i(v)$ are in the same LC on $p_{i+1}$ ($R_{i+1}(u) = R_{i+1}(v)$) iff there exists a $u,v$-path in $G_t^{i+1}$.

Since there exists a $u,v$-path in $G_t^i$ if and only if $R_i(u) = R_i(v)$ for $i = 1,...,m$ we have shown $R_m(u)=R_m(v)$ iff there is a $u,v$-path in $G_t$.
\end{proof}
\begin{theorem}\label{theorem:queriesnormal}
During normal mode, there exists a processor $p$ that gives vertices $u$ and $v$ the same label at time $t$ if and only if the vertices are in the same connected component in $G_t$.
\end{theorem}
\begin{proof}
Properties $1-4$ hold during normal mode, so by Theorem~\ref{theorem:propstoquery} queries are answered correctly.
\end{proof}
 An edge will remain in a outside slot for at most two laps around the ring 

From this we, we can also guarantee that the input stream will not be starved, as long as there are at least two outside slots per \bundlens. At the head processor, begin putting an edge into the first outside slot of each \bundle that is passed. If an edge is in the first outside slot of a bundle received from the tail, but that edge in the second outside slot of the \bundlens. On its second lap, an edge will be accepted by a processor, so no edges will be passed from the tail to head in the second outside slot.
\fi

\iffalse  OLD

If a new edge arriving at the filling processor connects two separate components, it is a \emph{spanning tree edge}, since the set of all such edges a processor has received forms a spanning forest of the identified components. For example, in Figure~\ref{fig:BBLCexample} edge $(v_0,v_1)$ on $p_0$ and $(v_0,v_4)$ on $p_1$ are tree edges. When storing a tree edge $e$, processor $p_i$ updates the disjoint-set structure to connect the building blocks named $R_{i-1}(v)$ and $R_{i-1}(u)$ in some local component $b_x$. Processor $p_i$ also stores the fact that $v$ is encapsulated by $R_{i-1}(v)$ and $u$ is encapsulated by $R_{i-1}(u)$; $u$ and $v$ are then \emph{known vertices} on $p_i$. 

If a node $p_i$ receives an edge $([u,R_{i-1}(u)],[v,R_{i-1}(v)])$ and relabels it such that $R_i(u) = R_i(v)$, then $(u,v)$ is a \emph{non-tree edge}. While these edges are not initially necessary for maintaining connected components, they may be necessary for determining correct components after aging, and thus must be kept some where in the system, not necessarily on $p_i$. Non-tree edges held on the same processor as the component containing them are called \emph{mortgage edges}. Non-tree edges stored on a different processor are called \emph{storage edges}.

If $p_i$ is the filling processor, it either has available space or contains some storage edges for other processors. If $p_i$ receives a non-tree edge contained in one of its own local components, it prefers that mortgage edge over any storage edges it may have. Thus if it does not have room for the mortgage edge, it will \emph{jettison}, or pass downstream, a storage edge to make room for the mortgage edge. The filling processor also prefers new tree edges over storage edges, and will jettison a storage edge if a new tree edge is received and there is not available space. When the processor is full of only tree and mortgage edges, it becomes sealed and $p_{i+1}$ becomes the filling processor. 

If a sealed processor $p_i$ receives an non-tree edge $e$ contained in one of its local components, it will take $e$ as a  mortgage edge if it has space from a previous aging. A sealed processor $p_i$ will also jettison a storage edge to make room for a mortgage edge. If $p_i$ does not have space or a storage edge to jettison, $e$ becomes a storage edge for $p_i$, and $e$ is passed downstream.
For example, in Figure~\ref{fig:BBLCexample}, edge $(v_1,v_2)$ on $p_2$ is a storage edge for $p_0$. The only edges arriving to non-filling processors downstream of the filling processor, which are not sealed, are storage edges. These storage edges will be accepted by an unsealed non-filling processor if it has room; otherwise they will be passed downstream to other non-filling processors. Storage edges may also be passed around the ring past the head, and accepted by a sealed processor with available space that opened up during aging.

\fi


\section{Aging mode}\label{sec:aging}
\begin{figure*}[bht]
\begin{center}
\begin{tabular}{c}
\includegraphics[width=5in]{aging1.png} \\
(a) \\
\includegraphics[width=5in]{aging2.png} \\
(b) 
\end{tabular}
\end{center}
\caption{\label{fig:xstream-aging} \XSCC aging mode. (a) A token notifies processors
that they must apply an aging predicate.  All surviving edges become unresolved (gray).
(b) Normal operation continues uninterrupted for new edges, while
unresolved edges circulate back to be incorporated into a new data structure. Each processor
in turn becomes the loading processor ($p_L$) and recycles its unresolved edges.}
\end{figure*}

\begin{figure*}[thb]
\begin{center}
\includegraphics[width=5in]{aging_properties.png} \\
\end{center}
\caption{\label{fig:xstream-aging-props} \XSCC aging nomenclature. both
primary
and payload edges are called \emph{resolved} when they have been
classified as tree or non-tree. Duplicate detection leaves empty
slots, and processors \emph{ingest} and \emph{emit} bundles of edges.}
\Jon{$p_H$ label, fix last ingest/emit example (my Office is currently broken;
will fix when Office fixed.}
\end{figure*}

\XSCC handles infinite streams via a bulk deletion operation we call
an \emph{aging} event.
Our model is thus unlike most previous work, in that we do not expect or
support individual
edge deletions embedded within the stream.  Rather, we expect the 
system administrator
to schedule bulk deletions to ensure that the oldest and/or 
least useful data are deleted in a timely manner.

To begin aging, the system administrator introduces an aging 
predicate (for example, a timestamp threshold) into the input stream.
The predicate
propagates through the system, and each processor suspends query 
processing upon receipt.  However, a new stream edge might arrive in the 
\XStream tick immediately after the aging predicate arrives from the 
I/O processor.
This and all other new edges must be ingested and processed without exception.  
Thus, the connectivity data structures must be rebuilt concurrently with
normal stream edge processing. When this rebuild is complete, queries
are accepted once again.

We now describe how \XSCC processes the aging predicate and prove correctness.
In Section~\ref{sec:aging-conditions} we provide theoretical guarantees relating 
the fraction of
system capacity used after the deletion predicate has been applied, the 
bandwidth expansion
factor, the proportion of query downtime that is tolerable, and the 
expected stream edge duplication rate.

\subsection{Aging process}
\label{sec:aging-process}
Figure~\ref{fig:xstream-aging} illustrates the aging process. An aging token arrives
with an edge-deletion predicate.  As the token propagates downstream, all
edges are reclassified to be \emph{untested}. If an edge later passes the
aging predicate it becomes \emph{unresolved} since the old connectivity
structure is no longer valid.
Immediately after the aging token is received
by the head processor, new stream edges may continue to arrive. These are
processed as normal, starting from empty data structures, so we maintain 
Invariants~\ref{inv:normal-1} and \ref{inv:normal-2} even during aging.

%\Jon{ATTN Cindy:  this paragraph is new text attempting to explain a subtle 
%point; please vet.}
Conceptually, upon receipt of aging notification the deletion of all
edges that fail the aging predicate and
reclassification of all surviving edges to \emph{unresolved} is instantaneous.
However, in
practice each processor takes $\frac{s}{k-1}$ \XStream ticks to execute
a ``testing phase'' that applies the aging predicate to each stored
edge.  Without careful attention to detail,
implementers could allow a case in which there is no space yet for a new
stream edge.  In Section~\ref{sec:pseudocode} we give exact
specifications for a correct procedure that ensures no stream edge is
dropped, even in the \XStream tick immediately after aging notification.
If the testing phase has yet not identified empty space for a new stream
edge, then one of the unresolved edges can be sent downstream
in a primary slot.  This is an example of the jeopardy condition described
later in this section, corresponding to Line 21 in Algorithm~\ref{algo:pseudocode-driver}.

In addition to normal processing of new stream edges, \XSCC recycles all
unresolved edges that survive the aging predicate.
As depicted in Figure~\ref{fig:xstream-aging}, we introduce a new designation $p_L$
for a \emph{loading processor} or ``loader.''  Upon each activation to process
a stream edge, the loader packs unresolved edges into any available payload
slots in the output \bundlens.  Such bundles propagate around the ring. After
a bundle
reaches the head processor $p_H$, its payload edges are processed as if they
were new edges.  When the loader has emitted all of its unresolved edges, it
passes the loader token downstream to its successor.  Aging is complete when
the last processor with any unresolved edges has completed its loader duties.

\begin{figure*}[thb]
\begin{center}
\includegraphics[width=5in]{jeopardy.png} \\
\end{center}
\caption{\label{fig:jeopardy} The \XSCC aging ``jeopardy
condition.'' Processor $p_B$ currently bears both building and loading
responsibilities, is completely full of edges, and must ingest a \bundle
with no empty slots. It ingests $k$ slots, finds no duplicates,
and must emit $k$ slots. Therefore an unresolved ``jeopardy edge''
must be emitted in the primary slot.  If it doesn't settle in a 
processor before leaving the tail, the system is completely full and raises a
FAIL condition. Note that in this illustration, $p_S$ will be able to store
the jeopardy edge, so the jeopardy condition will soon be mitigated.}
\end{figure*}

The complete \XSCC protocols defined in Section~\ref{sec:pseudocode} 
enforce the previous invariants at all times, as well as the following
invariant during aging.  
% CAP: Not sure this is adding much at this point.  Put it back later (probably changed) if it makes sense
%It formalizes the idea that the loader never leaves behind any unresolved edges.
% CAP: here is the old invariant: During aging, let $p_L$ be the loading processor. Then $p_i$ is completely full of resolved edges for $0 \le i < L$ at all times, and there are no resolved edges for $i > L$.
% \Cindy{I don't think the first part is true.  Processors are only packed up to $p_S$, and the loader can be downstream of $p_S$.  Fix this.  Maybe no unresolved edges upstream}
\begin{invariant}
During aging, let $p_L$ be the loading processor and $p_B$ be the building processor. 
Then $B \le L$.  Also, processor $p_i$ has no unresolved edges for $i < L$
% Then $p_i$ is completely full of resolved edges for $0 \le i < L$ at all times,
and $p_i$ has no resolved edges for $i > L$. \label{inv:aging-1}
\end{invariant}

The combination of all invariants ensures that all processors from the head to the builder are running \XSCC in normal mode on all incoming (and recycled) edges. All resolved edges are packed to the front (upstream).  When all edges have been recycled and aging ends, the layout of edges returns to normal mode.

%\Jon{(updated by CAP) This begs a statement that the processors between 0 and B are still doing XS normal mode (on the incoming edges union the payload edges.}

%\Jon{This next invariant is wrong by our current pseudocode.  Don't think
%it was used in the first place.}
%Our protocols also enforce the following relationship between
%$p_B$, $p_L$, and $p_S$.  We must never build connectivity structure
%downstream of the loader.
%\begin{invariant}
%During aging, let $p_B$ be the building processor, $p_L$ be the loading
%processor, and $p_S$ be the first processor with empty space.  Then
%%$B \le L \le S$. \label{inv:aging-2}
%\end{invariant}


Figure~\ref{fig:xstream-aging-props} puts the
nomenclature of our arguments into context. An edge becomes \emph{resolved}
when an XS-CC processor determines that it is a tree or non-tree edge,
regardless of whether it is a new stream edge in a primary slot or an
unresolved edge being recycled as payload.  Processors \emph{ingest} and
\emph{emit} bundles of edges.  With one exception we will discuss presently,
the complexity of processing input bundles and packing edges into
output bundles prior to emission is relegated to Section~\ref{sec:pseudocode}.

Aging is generally a straightforward process in which the loader token
steadily advances from $p_H$ to $p_T$, unresolved edges are recycled and
resolved, and the XS-CC connectivity structure is rebuilt.  
When builder and loader designations coincide in the same processor, 
that processor packs unresolved edges for emission first, then non-tree 
edges.  Edge bundles containing transit edges have one primary slot and $k-1$ 
payload slots, where
$k$ is the bandwidth expansion factor.  New stream edges reside in primary
slots, and unresolved edges circulate in payload slots until they are
resolved.  Payload edges continue in their assigned slots until allowed to 
settle, per the invariants.

There is a single exception to this last point, illustrated in
Figure~\ref{fig:jeopardy}. We call this the \emph{jeopardy condition}
and use it to specify exactly when the system fills
to capacity during aging (indicating that the aging command was too late
or did not remove enough edges).
In the jeopardy condition, processor $p_B$ is also the loader,
is already storing edges to its capacity $s$, and must ingest an edge \bundle
with no empty slots. It ingests $k$ slots, finds no duplicates,
and by conservation of space, must emit $k$ slots. 
Therefore an unresolved edge $e_j$ must reside in the primary slot.
If $e_j$ cannot be offloaded before exiting the tail,
the system is completely full and raises a FAIL condition.

The above discussion and the more detailed discussion in Section~\ref{sec:pseudocode} show the following property necessary for proving aging correctness holds:
\begin{property}
\label{prop:all-recycled}
During aging, every surviving edge is incorporated into the new 
connected components data structure either by $p_H$ directly or 
by traveling back to $p_H$ as a payload edge.
\end{property}

\subsection{Aging correctness}
We now argue correctness of the aging process.
We say that any implementation of \XSCC aging that maintains
Invariants~\ref{inv:normal-1}, \ref{inv:normal-2} and \ref{inv:aging-1} and property~\ref{prop:all-recycled} is \emph{compliant}.
A compliant aging process ensures that during aging there is a monotonic ordering of
edges in the system, with tree (red) edges never allowed downstream of
non-tree (blue) edges, and unresolved (gray) edges never allowed upstream
of non-tree edges. In the argument below, we slightly abuse notation by
using the graph $G_t$ in place of its edge set $E(G_t)$.

\begin{theorem}
Suppose a compliant XS-CC implementation receives an aging command at tick $t$ and reauthorizes queries at tick $t'$.  Let $F$ be the set of edges in $G_t$ that fail the aging predicate and let $E_{t\rightarrow t'}$ be the set of edges that arrive between time $t$ and $t'$. Then at tick $t'$, the x-stream system stores graph $G_{t'} = G_t - F \cup E_{t\rightarrow t'}$, can properly answer queries and stores each edge in $G_{t'}$ exactly once.
\label{thm:aging-correctness}
\end{theorem}
\begin{proof}
\Cindy{consistency in graph vs. edge set}\Jon{mitigated by text above. Cindy?}
As the aging command that arrived at time $t$ propagates through the
processors, they reclassify all current edges to ``untested'' as
described in Section~\ref{sec:aging-process}, forgetting the current
union-find structure. Thus the system starts processing a new graph
from an empty state at time $t+1$. As described in
Section~\ref{sec:aging-process}, processors delete all edges in $F$,
those who fail the predicate.  Each remaining edge in $G_t - F$ is
eventually loaded into payload slot by
Property~\ref{prop:all-recycled}, and processed at the head as
arriving edges. Invariants~\ref{inv:normal-1} and~\ref{inv:normal-2}
hold with the newly-created data structures thoughout aging.
Invariant~\ref{inv:aging-1} ensures that all unresolved edges are in
the builder processor or downstream. Those in the builder do not
affect the connectivity computation and are eventually moved
downstream. Thus, all edges arriving from outside the system are
processed as in normal mode and all edges arriving in the payload
slots are processed as in normal mode (other than traveling in a
payload slot).  Thus at time $t'$ when the tail processor passes the
loading token out of the system and enables queries, the \XStream
system stores exactly the edges in $G_t - F \cup E_{t\rightarrow t'}$,
with duplicates appropriately removed.  This is the graph the system
is required to hold by definition of aging and the requirement that it
drop no incoming edges during aging. The edges are processed into the
data structures with arbitrary mixing of new edges and recycled
(surviving) edges.  By Observation~\ref{obs:wstreamreqs}, and the equivalence of
\DFR and \XSCC in normal mode, the ordering of
the input edges does not matter for future query correctness.
By Theorem~\ref{thm:query-correctness}, the \XStream system will now correctly
answer querries on the graph starting at time $t'$.

During aging, some edges may be stored up to twice.  If a duplicate of a suriving edge $e$
enters the system before edge $e$ circulates back to the head processor, then edge $e$ is
stored both in the new data structure as a tree or non-tree edge and as an unresolved edge.
However, when edge $e$ is eventually recycled, it will be recognized as a duplicate and not
stored again.  By Theorem~\ref{thm:non-dup}, any edge that enters from outside the system during aging will be stored at most once in the new data structure.
\qed
\end{proof}

\iffalse
\Cindy{OLD:}
Since Invariants~\ref{inv:normal-1} and \ref{inv:normal-2} still hold
during aging, we can define a corollary to Lemma~\ref{lemma:B-stream}
to argue about the partially rebuilt
connectivity structure.

Suppose that a compliant aging process begins at \XStream tick $t$, 
and consider tick $t'>t$ during
the aging process. Let $E'$ be the set of stream edges that have arrived during
$\{t+1,\ldots,t'\}$ and $U$ be the set of unresolved edges that remain. We call
\[G^r_{t'} = G_t - U \cup E'\] the \emph{resolved graph at time $t'$}.  This
contains all active edges that either entered the system since time $t$ or
become resolved after circulating as payload.
\Jon{account for edges that got deleted by the predicate.}
\Jon{define this differently: the key is the set of payload edges that have
entered the head}

\begin{corollary}
\label{corollary:partial}
(corollary to Lemma~\ref{lemma:wx-correspondence}) The equivalence
between \DFR and \XSCC expressed in Lemma~\ref{lemma:wx-correspondence}
holds for $G^r_{t'}$, the resolved graph at aging time $t'$.
\end{corollary}
\begin{proof}
The argument is exactly the same as that of Lemma~\ref{lemma:wx-correspondence},
replacing $G_t$ with $G^r_{t'}$.
\end{proof}

Thus, as aging proceeds, the connected component information
stored by \XSCC regarding the resolved graph
aligns with that W-Stream would have computed had it been given $G^r_{t'}$
as a stream.  All that remains for the correctness argument
is to define the conditions under which
a compliant aging process completes. We address this in Section~\ref{sec:aging-conditions}. If aging does complete at some time $t''$,
then $G_{t''} = G^r_{t''}$ and Corollary~\ref{corollary:partial} implies
that the system can leave its aging mode and resume query processing.
\fi

\section{Conditions for successful aging}
\label{sec:aging-conditions}
In this section, we define the conditions under which
a compliant aging process completes before the system fails for lack of space.
We consider properties of the system, properties of the input stream, and user
preferences.


\begin{tcolorbox}
\begin{definition}
\ \ \\
We define the following as tradeoff parameters associated with infinite runs of \XSCCns.
\label{def:infinite-run-params}
\begin{description}
\item[{\bf c:}] fraction of the total system storage occupied by
edges that survive the aging predicate
\item[{\bf d:}] percentage of \XStream ticks that the system is unavailable for queries due to aging
\item[{\bf u:}] estimate of the percentage of incoming stream edges that will be unique
\item[{\bf k:}] the bandwidth expansion factor: the size of an \XStream bundle (a set of edge-sized slots that circulates in the ring)
\item[{\bf p:}] number of \XStream processors
\item[{\bf S:}] aggregate storage available in the system
\item[{\bf s:}] storage per processor in a homogeneous system ($s = S/p$)
\end{description}
\end{definition}
\end{tcolorbox}

%\Jon{New lemma: must age before you're full.  Precondition: must be at least
%$P+1$ empty slots in the whole \XStream system at the time you invoke aging.}
Aging must be initiated before the system becomes too full,
or else jeopardy edges will lead to a FAIL condition.  We quantify this
 decision point as follows.

\begin{lemma} \label{lemma:aging-lead-time}
In the worst case, there must be at least 
$$\frac{cS}{p(k-1)} + \frac{3}{2}p$$
open space in the system when an aging command is issued to be guaranteed sufficient space for aging, where
$c$, $S$, $p$ and $k$ are given in Definition~\ref{def:infinite-run-params}.  
\end{lemma}
\begin{proof}
When the aging command arrives, there could be $p$ edges in transit
that all must be stored. Because iteration over the untested list
doesn't imply
any specific ordering, in the worst case, when processors test the
edges against the predicate, all $cS$ surviving edges are tested
before an edge fails the predicate. This gives the latest time when
space becomes free for new edges. When a processor receives the aging
command, it processes $(k-1)$ untested edges each tick until it has
tested all its edges. In the $p$ ticks required for the aging command
to reach the tail, the head tests $p(k-1)$ edges, the second processor
tests $(p-1)(k-1)$ edges and so on, while the tail tests $(k-1)$
edges.  Thus in the first $p$ ticks after aging starts, the system tests
$\frac{p(1+p)(k-1)}{2}$ edges.  After that, the system tests $p(k-1)$
edges per tick. If the system tested $p(k-1)$, every tick, it would
require  $\frac{cS}{p(k-1)}$ ticks. But the first $p$ ticks are only half
as efficient, so we require an extra $p/2$ ticks. Thus the total
number of ticks before the system is guaranteed to remove an edge that
fails the predicate is at most $\frac{cS}{p(k-1)} + \frac{3}{2}p$.
\qed
\end{proof}

If the system is homogeneous, the empty space expression in Lemma~\ref{lemma:aging-lead-time} becomes $cs/(k-1) + \frac{3}{2}p$. For example, for a homogeneous system, assuming that $s \gg p$, if $c = 1/2$ and $k = 5$, then one should start aging while $1/8$ of the last processor is still empty. The last processor can issue a warning when it starts to fill and again closer to the deadline given $c$ and $k$.

\begin{theorem} \label{thm:infinite-runs}
In any \XSCC aging process initiated in accordance with Lemma~\ref{lemma:aging-lead-time}, if
$c$, $d$, $u$, $k$, $p$, $S$, and $s$ from Definition~\ref{def:infinite-run-params}
are set such that 
             \[k \ge   1 + \frac{(cp + 1) u}{dp (1-c)},\]
then the aging process will finish before the system storage fills completely.
\end{theorem}

\begin{proof}
After the aging token arrives, the head processor must apply the aging predicate
to its $s$ edges.  It processes $k-1$ per tick, as described in Section~\ref{sec:aging-process}.
Thus, after $\frac{s}{k-1}$ ticks, the head processor passes the loader token to the
second processor. By that time, all other processors have applied the predicate to
all of their edges and have a list of surviving edges.
Once unresolved edges begin circulating from
the loader (ignoring additive latencies such as the time until the first
payload reaches the head processor since $P \ll s$), $k-1$ edges re-enter the system to be
resolved at each
tick.  Since $cS$ unresolved edges survived the aging predicate, in the worst case
(when they are all in the second processor or later) it will take
$\frac{cS}{k-1}$ ticks to complete aging. During this time, every $\frac{1}{u}$
ticks yields a new, non-duplicate stream edge. Thus, the system will fill
to capacity in $\frac{1}{u} (1-c)S$ ticks.
The proportion $d$ constrains these two tick counts as follows:
\[ d \frac{1}{u} (1-c) S \ge \frac{cS + s}{k-1} = (cS / (k-1)) * (p+1)/p.\]
Simplifying this inequality and solving for $k$ (with Wolfram Alpha~\cite{wa},
for example)
yields the result. \qed
\end{proof}

The parameters $c$ and $d$ are user preferences, but $k$ is dictated by 
computer architecture.  Reasonable values of $k$ for current architectures are
$5-10$, but emerging data flow architectures may provide upward flexibility.
The parameter $u$ must be estimated by the user based on knowledge of the
input streams that s/he will feed to \XSCCns.


We can now state the central result of this paper. 
%\Cindy{TODO: wrangle this into usable shape}
%We assume that the graph-edge stream is {\em effectively infinite}. This means that as long as the algorithm is running, it must always be prepared for the arrival of another edge. At any time, the system has seen only a finite set of edges and need store only a finite graph representation.  However this graph can be arbitrarily large and may eventually exceed any particular finite storage. 
%\Jon{I think less is more in this case.}

\begin{theorem}
\XSCC can process an effectively infinite stream of graph edges without failing,
answering
connectivity queries correctly when in normal mode, as long as
the system is configured in accordance with Theorem~\ref{thm:infinite-runs}
and aging is started with sufficient space available obeying
Lemma~\ref{lemma:aging-lead-time}.
\end{theorem}
\begin{proof}
Assuming that the proportion of \XStream ticks that yield a new,
non-duplicate stream edge is $u$,
an empty system will fill and fail in $\frac{1}{u}S$ ticks. Compliant
aging in accordance with Lemma~\ref{lemma:aging-lead-time} ensures that
aging will always complete before the system fills.  During normal
mode operation, Lemma~\ref{lemma:B-stream} and 
Theorem~\ref{thm:query-correctness} ensure, respectively, that 
accurate connected
component information is stored, and that connectivity queries are
answered correctly.  As long as the
system adminstrator adheres to such a schedule, \XSCC operation
can continue through an arbitrary number of aging events. \qed
\end{proof}
We note that queries yielding system capacity usage are TODO constant-size. In
the case of a simple aging predicate such as a timestamp threshold,
given a target proportion $c$ of edges that survive an aging
event, the \XStream system administrator could use an automated process to
trigger the aging process.


\section{X-Stream edge processing specification}\label{sec:pseudocode}
\setcounter{algorithm}{0}
\begin{algorithm*}
\caption{This is the driver function for an X-Stream implementation that is
compliant with Invariants~\ref{inv:normal-1}, \ref{inv:normal-2}, and
\ref{inv:aging-1}, and Property~\ref{prop:all-recycled}.
         \label{algo:pseudocode-driver}}
         X-Stream connected components driver \\
\makealgtitle
\begin{algorithmic}[1]

\Procedure{ProcessBundle}{\Call{PrimaryEdge}{e}, \Call{PayloadEdges}{$e_i$}}
%\LeftComment{Precondition: input ``raft'' containing $\Call{Primary}{e}, \Call{Payload}{e_i}$, where payload exists only during aging}
\State  \Call{PackingSpaceAvailable}{} $= k$  \Comment{ the output buffer has size $k$ and is initially empty}
    \If {\Call{EmptyEdge}{$e$}}
\State  \Call{Pack}{EmptyEdge}  \Comment{any call to \Call{Pack}{} decrements \Call{PackingSpaceAvailable}{}}
    \Else \ \ \ \Call{ProcessEdge}{$e$}
    \EndIf
    \For {$e_i \in \Call{PayloadEdges}{}$} \Comment{there are payload edges only during aging or non-constant query processing}
\State  \Call{ProcessEdge}{$e_i$}
    \EndFor

\If {\Call{Not}{AGING}}
\State \Call{Emit}{PackedBundle}
\State \Return
\EndIf
\LeftComment{Aging-related logic}
    \If {\Call{Loader}{}}
        \If {\Call{Head}{}}  \Comment{the Head's testing \& resolution phase}
            \For {$i = 1, k-1$} 
\State          $e'$ = {\Call{PopEdge}{UNTESTED}} 
                \If {\Call{EmptyEdge}{$e'$}}
\State              \Call{Pack}{LoaderToken}
\State              \Call{Break}{}
                \EndIf
                \If {\Call{AgingPredicatePassed}{$e'$}}
                    \If {$i = k-1$ \mbox{and} \Call{Full}{}} 
\State                  \Call{Pack}{$e', \Call{Primary}{}$} \Comment{jeopardy edge}
                    \Else
\State                  \Call{ProcessEdge}{$e'$} \Comment{\Call{Head}{} immediately resolves surviving edge}
                    \EndIf
                \Else
\State              \Call{Delete}{$e'$}
                \EndIf
            \EndFor
        \Else   \Comment{ Downstream resolution phase (all testing has finished)}
            \For {$i \in 0, \ldots, \Call{PackingSpaceAvailable}{} - 1$}
\State          $e'$ = \Call{PopEdge}{UNRESOLVED}
                \If {\Call{NULL}{$e'$}}
\State              \Call{Pack}{LoaderToken}
\State              \Call{Break}{}
                \Else
\State              \Call{Pack}{e'}
                \EndIf
            \EndFor
        \EndIf
    \Else  \Comment{ Downstream testing phase}
        \If {\Call{Not}{HEAD} \Call{And}{} \Call{NotEmpty}{UNTESTED}}
            \For {$i = 1, k-1$} 
\State          $e'$ = {\Call{PopEdge}{UNTESTED}}
                \If {\Call{AgingPredicatePassed}{$e'$}}
\State              \Call{PushEdge}{$e'$, UNRESOLVED} 
                \Else
\State              \Call{Delete}{$e'$}
                \EndIf
            \EndFor
        \EndIf
    \EndIf
\State \Call{Emit}{PackedBundle}
\EndProcedure
\end{algorithmic}
\end{algorithm*}


Algorithms~\ref{algo:pseudocode-driver} and \ref{algo:pseudocode-functions}
show the \XSCC driver and constituent functions, respectively, for 
processing edges. We do not show full detail for token passes, commands,
and queries.
These functions maintain the invariants and produce a compliant XS-CC
% macro foils LaTeX line breaking
implementation.  We used this pseudocode as guidance for the code
that produces our experimental results.

Each \XStream processor executes \Call{ProcessBundle}{} whenever it receives
the next bundle of edge slots, regardless of its current execution mode
(normal or aging).  It will process each slot in turn, and the 
constituent functions \Call{ProcessEdge}{}, \Call{ProcessPotentialTreeEdge}{},
and \Call{StoreOrForward}{} determine what to pack into an
output bundle destined to flow downstream.

Note that the top-level logic of processing the primary and payload
edges of a bundle is the same in Algorithm~\ref{algo:pseudocode-driver}, 
regardless of execution mode.  When a new
edge arrives from the stream, processors upstream of (and including) the 
building processor will classify it as tree or non-tree using the
relabeling logic of Section~\ref{sec:relabeling} (Lines 15-18 of 
\Call{ProcessEdge}{} and Lines 2-4 of \Call{ProcessPotentialTreeEdge}{}).
The builder stores any new tree edge. We ensure that this is
possible via logic to jettison an unresolved edge if one exists 
(only during aging; Lines 9 and 16 of \Call{StoreOrForward}{}), or else 
to jettison a non-tree edge (Line 15
of \Call{StoreOrForward}{}).  This progression of jettison logic maintains
Invariants~\ref{inv:normal-1} and \ref{inv:normal-2}.

Suppose that the head processor $p_H$ receives notification of an aging event
at \XStream tick $t$.  \XStream ticks $t$ and $t+1$ are especially interesting. 
If a new edge arrives in the input stream at $t+1$, it must be stored 
in $p_H$ (which
is now acting as both the builder $p_B$ and the loader $p_L$) in
order to maintain Invariant~\ref{inv:aging-1}.  However, $p_H$ has had only
one tick to initiate the process of testing its edges against the aging
predicate.  That means that it tested $k-1$ edges in tick $t$.  Suppose all
of these edges survived the predicate and therefore couldn't be deleted.
This is a jeopardy condition, and it was handled during tick $t$ by 
Lines 20-21 of 
\Call{ProcessBundle}{}.  Favoring the new edge, $p_H$ jettisoned in the
primary slot of its output bundle the last of the $k-1$ unresolved edges
it created in that tick.  Therefore, at tick $t+1$ we are assured that $p_H$ 
can store a new stream edge.

During aging, the loader $p_L$ packs unresolved edges into the empty payload
slots in incoming bundles to be sent around the ring.  When these edges
arrive at $p_H$, they are processed as if they were new stream edges,
classified as tree or non-tree, and incorporated into the data structures
in $\{p_H,\ldots,p_B\}$ by the same invariant-maintaining constituent 
functions that handle new edges.  One optimization we include is that
$p_H$ need not actually pack and send its unresolved edges around the
ring.  Rather, in Lines 13-23 of \Call{PackBundle}{}, $p_H$ simply tests
against the aging predicate and immediately processes its tested edges
rather than calling them unresolved.  

As aging proceeds, the \Call{Loader}{} token is passed downstream whenever
a processor exhausts its list of unresolved edges (Lines 28-31 of 
\Call{ProcessBundle}{}). Once the \Call{Loader}{} token exits the tail
processor, Property~\ref{prop:all-recycled} is established.


\begin{algorithm*}
\caption{These three constituent functions comprise the 
X-Stream algorithm for maintaining connected components.\label{algo:pseudocode-functions}}
         X-Stream constituent functions \\
\makealgtitle
\begin{algorithmic}[1]
\LeftComment{Processor $p_a$ receives an edge}
\Procedure{ProcessEdge}{$e=(u,v,R_{a-1}(u),R_{a-1}(v))$}
          \If {\Call{Duplicate}{e}} \Comment{regardless of $p_a$'s position in the chain, duplicate edges don't propagate downstream}
\State        \Call{SetNewestTimestamp}{e}  \Comment{During aging, either $e$ or its stored duplicate could be the newest}
              \If {\Call{Primary}{e}}
\State            \Call{Pack}{EmptyEdge} \Comment{bundles drive \XS ticks, \Call{ProcessBundle}{} requires a primary edge}
              \EndIf
\State        \Call{Return}{}
          \EndIf
          \If {\Call{DownstreamOfBuilder}{}} \Comment{$p_B \prec p_a$ : $p_a$ stores only non-tree and/or unresolved edges}
              \If {\Call{Primary}{e}} \Comment{need to store this edge if we can in order to ensure Invariant~\ref{inv:normal-2}}
\State            \Call{StoreOrForward}{e}        \Comment{\Call{StoreOrForward}{} accepts $e$ or packs it for output}
              \Else \Comment{\Call{Payload}{e}, i.e., aging}
\State            \Call{Pack}{e}         \Comment{processors downstream of the Builder simply propagate payload edges}
              \EndIf
\State        \Call{Return}{}
          \EndIf
\LeftComment{Processor $p_a$ contains connected component information, i.e., $p_a \prec = p_B$} 
          \If {$R_{a-1}(u) = R_{a-1}(v)$}   \Comment{previously-discovered non-tree edge}
\State        \Call{StoreOrForward}{e} 
          \ElsIf {\Call{Not}{\Call{ProcessPotentialTreeEdge}{e}}}\Comment{newly-discovered non-tree edge}
\State        \Call{StoreOrForward}{e}
          \EndIf
\EndProcedure
\end{algorithmic}

\begin{algorithmic}[1]
\Procedure{ProcessPotentialTreeEdge}{e=(u,v,\ldots)}
\State   $(R_a(u),R_a(v)) = $ \Call{Relabel}{e}
         \If {$R_a(u) = R_a(v)$}  \Comment{newly-discovered non-tree edge}
\State        \Call{Return}{FALSE}
         \EndIf
         \If {\Call{Builder}{}}      \Comment{builder $p_B$ must ingest tree edge $e$}
\State        \Call{Assert}{\Call{StoreOrForward}{e} = STORE} \Comment{ensure Invariant~\ref{inv:normal-1}}
              \If {\Call{FullOfTreeEdges}{}}
\State             \Call{Pack}{BuilderToken} \Comment{can be encoded with a bit; doesn't take a whole slot}
              \EndIf
         \Else \ \ \ \Call{Pack}{e}  \Comment{$p_a$ has previously sealed, so it is already full of tree edges}
         \EndIf
\State   \Call{Return}{TRUE}\Comment{still a potential tree edge; downstream processors will determine that}
\EndProcedure
\end{algorithmic}

\begin{algorithmic}[1]
\Procedure{StoreOrForward}{e=(u,v,\ldots)}
\LeftComment{Precondition: if $e$ is a tree edge, this processor is not full of TREE edges}
         \If {\Call{Full}{}}
             \If {\Call{Tail}{}}
\State            \Call{Fail}{} \Comment{the system is totally full}
             \EndIf
             \If {\Call{Unresolved}{$e$}}
\State                 \Call{Pack}{$e$}
\State                 \Call{Return}{FORWARD}
             \EndIf
\State       $e_p = $ \Call{PopEdge}{UNRESOLVED} \Comment{jettison an unresolved edge to keep a resolved one, if possible}
             \If {\Call{EmptyEdge}{$e_p$}} \Comment{no more edges to resolve}
                  \If {\Call{NonTree}{$e$}}
\State                 \Call{Pack}{$e$}  \Comment{no need to jettison a non-tree edge if $e$ is non-tree}
\State                 \Call{Return}{FORWARD}
                  \Else \Comment{by precondition, there must be a non-tree edge to jettison}
\State                 \Call{Pack}{\Call{PopEdge}{NONTREE}} \Comment{jettison a non-tree edge to keep a tree edge}
                  \EndIf
             \Else \ \ \ \Call{Pack}{$e_p$}
             \EndIf
         \EndIf
         \If{\Call{Primary}{e}} \Call{Pack}{\Call{EmptyEdge}{}}  \Comment{every output raft needs a primary edge}
         \EndIf
\State   \Call{Accept}{e}  \Comment{perform UNION/FIND if $\Call{Tree}{e}$}
\State   \Call{Return}{STORE}
\EndProcedure
\end{algorithmic}
\end{algorithm*}


\section{Related work}\label{sec:related-work}
\section{Related Work}
%\mz{We lack a comparison to this paper: https://arxiv.org/abs/2305.14877}
%\anirudh{refine to be more on-topic?}
\iffalse
\paragraph{In-Context Learning} As language models have scaled, the ability to learn in-context, without any weight updates, has emerged. \cite{brown2020language}. While other families of large language models have emerged, in-context learning remains ubiquitous \cite{llama, bloom, gptneo, opt}. Although such as HELM \cite{helm} have arisen for systematic evaluation of \emph{models}, there is no systematic framework to our knowledge for evaluating \emph{prompting methods}, and validating prompt engineering heuristics. The test-suite we propose will ensure that progress in the field of prompt-engineering is structured and objectively evaluated. 

\paragraph{Prompt Engineering Methods} Researchers are interested in the automatic design of high performing instructions for downstream tasks. Some focus on simple heuristics, such as selecting instructions that have the lowest perplexity \cite{lowperplexityprompts}. Other methods try to use large language models to induce an instruction when provided with a few input-output pairs \cite{ape}. Researchers have also used RL objectives to create discrete token sequences that can serve as instructions \cite{rlprompt}. Since the datasets and models used in these works have very little intersection, it is impossible to compare these methods objectively and glean insights. In our work, we evaluate these three methods on a diverse set of tasks and models, and analyze their relative performance. Additionally, we recognize that there are many other interesting angles of prompting that are not covered by instruction engineering \cite{weichain, react, selfconsistency}, but we leave these to future work.

\paragraph{Analysis of Prompting Methods} While most prompt engineering methods focus on accuracy, there are many other interesting dimensions of performance as well. For instance, researchers have found that for most tasks, the selection of demonstrations plays a large role in few-shot accuracy \cite{whatmakesgoodicexamples, selectionmachinetranslation, knnprompting}. Additionally, many researchers have found that even permuting the ordering of a fixed set of demonstrations has a significant effect on downstream accuracy \cite{fantasticallyorderedprompts}. Prompts that are sensitive to the permutation of demonstrations have been shown to also have lower accuracies \cite{relationsensitivityaccuracy}. Especially in low-resource domains, which includes the large public usage of in-context learning, these large swings in accuracy make prompting less dependable. In our test-suite we include sensitivity metrics that go beyond accuracy and allow us to find methods that are not only performant but reliable.

\paragraph{Existing Benchmarks} We recognize that other holistic in-context learning benchmarks exist. BigBench is a large benchmark of 204 tasks that are beyond the capabilities of current LLMs. BigBench seeks to evaluate the few-shot abilities of state of the art large language models, focusing on performance metrics such as accuracy \cite{bigbench}. Similarly, HELM is another benchmark for language model in-context learning ability. Rather than only focusing on performance, HELM branches out and considers many other metrics such as robustness and bias \cite{helm}. Both BigBench and HELM focus on ranking different language model, while fix a generic instruction and prompt format. We instead choose to evaluate instruction induction / selection methods over a fixed set of models. We are the first ever evaluation script that compares different prompt-engineering methods head to head. 
\fi

\paragraph{In-Context Learning and Existing Benchmarks} As language models have scaled, in-context learning has emerged as a popular paradigm and remains ubiquitous among several autoregressive LLM families \cite{brown2020language, llama, bloom, gptneo, opt}. Benchmarks like BigBench \cite{bigbench} and HELM \cite{helm} have been created for the holistic evaluation of these models. BigBench focuses on few-shot abilities of state-of-the-art large language models, while HELM extends to consider metrics like robustness and bias. However, these benchmarks focus on evaluating and ranking \emph{language models}, and do not address the systematic evaluation of \emph{prompting methods}. Although contemporary work by \citet{yang2023improving} also aims to perform a similar systematic analysis of prompting methods, they focus on simple probability-based prompt selection while we evaluate a broader range of methods including trivial instruction baselines, curated manually selected instructions, and sophisticated automated instruction selection.

\paragraph{Automated Prompt Engineering Methods} There has been interest in performing automated prompt-engineering for target downstream tasks within ICL. This has led to the exploration of various prompting methods, ranging from simple heuristics such as selecting instructions with the lowest perplexity \cite{lowperplexityprompts}, inducing instructions from large language models using a few annotated input-output pairs \cite{ape}, to utilizing RL objectives to create discrete token sequences as prompts \cite{rlprompt}. However, these works restrict their evaluation to small sets of models and tasks with little intersection, hindering their objective comparison. %\mz{For paragraphs that only have one work in the last line, try to shorten the paragraph to squeeze in context.}

\paragraph{Understanding in-context learning} There has been much recent work attempting to understand the mechanisms that drive in-context learning. Studies have found that the selection of demonstrations included in prompts significantly impacts few-shot accuracy across most tasks \cite{whatmakesgoodicexamples, selectionmachinetranslation, knnprompting}. Works like \cite{fantasticallyorderedprompts} also show that altering the ordering of a fixed set of demonstrations can affect downstream accuracy. Prompts sensitive to demonstration permutation often exhibit lower accuracies \cite{relationsensitivityaccuracy}, making them less reliable, particularly in low-resource domains.

Our work aims to bridge these gaps by systematically evaluating the efficacy of popular instruction selection approaches over a diverse set of tasks and models, facilitating objective comparison. We evaluate these methods not only on accuracy metrics, but also on sensitivity metrics to glean additional insights. We recognize that other facets of prompting not covered by instruction engineering exist \cite{weichain, react, selfconsistency}, and defer these explorations to future work. 

\section{Experiments} \label{sec:experiments}
In this section we conduct comprehensive experiments to emphasise the effectiveness of DIAL, including evaluations under white-box and black-box settings, robustness to unforeseen adversaries, robustness to unforeseen corruptions, transfer learning, and ablation studies. Finally, we present a new measurement to test the balance between robustness and natural accuracy, which we named $F_1$-robust score. 


\subsection{A case study on SVHN and CIFAR-100}
In the first part of our analysis, we conduct a case study experiment on two benchmark datasets: SVHN \citep{netzer2011reading} and CIFAR-100 \cite{krizhevsky2009learning}. We follow common experiment settings as in \cite{rice2020overfitting, wu2020adversarial}. We used the PreAct ResNet-18 \citep{he2016identity} architecture on which we integrate a domain classification layer. The adversarial training is done using 10-step PGD adversary with perturbation size of 0.031 and a step size of 0.003 for SVHN and 0.007 for CIFAR-100. The batch size is 128, weight decay is $7e^{-4}$ and the model is trained for 100 epochs. For SVHN, the initial learinnig rate is set to 0.01 and decays by a factor of 10 after 55, 75 and 90 iteration. For CIFAR-100, the initial learning rate is set to 0.1 and decays by a factor of 10 after 75 and 90 iterations. 
%We compared DIAL to \cite{madry2017towards} and TRADES \citep{zhang2019theoretically}. 
%The evaluation is done using Auto-Attack~\citep{croce2020reliable}, which is an ensemble of three white-box and one black-box parameter-free attacks, and various $\ell_{\infty}$ adversaries: PGD$^{20}$, PGD$^{100}$, PGD$^{1000}$ and CW$_{\infty}$ with step size of 0.003. 
Results are averaged over 3 restarts while omitting one standard deviation (which is smaller than 0.2\% in all experiments). As can be seen by the results in Tables~\ref{black-and_white-svhn} and \ref{black-and_white-cifar100}, DIAL presents consistent improvement in robustness (e.g., 5.75\% improved robustness on SVHN against AA) compared to the standard AT 
%under variety of attacks 
while also improving the natural accuracy. More results are presented in Appendix \ref{cifar100-svhn-appendix}.


\begin{table}[!ht]
  \caption{Robustness against white-box, black-box attacks and Auto-Attack (AA) on SVHN. Black-box attacks are generated using naturally trained surrogate model. Natural represents the naturally trained (non-adversarial) model.
  %and applied to the best performing robust models.
  }
  \vskip 0.1in
  \label{black-and_white-svhn}
  \centering
  \small
  \begin{tabular}{l@{\hspace{1\tabcolsep}}c@{\hspace{1\tabcolsep}}c@{\hspace{1\tabcolsep}}c@{\hspace{1\tabcolsep}}c@{\hspace{1\tabcolsep}}c@{\hspace{1\tabcolsep}}c@{\hspace{1\tabcolsep}}c@{\hspace{1\tabcolsep}}c@{\hspace{1\tabcolsep}}c@{\hspace{1\tabcolsep}}c}
    \toprule
    & & \multicolumn{4}{c}{White-box} & \multicolumn{4}{c}{Black-Box}  \\
    \cmidrule(r){3-6} 
    \cmidrule(r){7-10}
    Defense Model & Natural & PGD$^{20}$ & PGD$^{100}$  & PGD$^{1000}$  & CW$^{\infty}$ & PGD$^{20}$ & PGD$^{100}$ & PGD$^{1000}$  & CW$^{\infty}$ & AA \\
    \midrule
    NATURAL & 96.85 & 0 & 0 & 0 & 0 & 0 & 0 & 0 & 0 & 0 \\
    \midrule
    AT & 89.90 & 53.23 & 49.45 & 49.23 & 48.25 & 86.44 & 86.28 & 86.18 & 86.42 & 45.25 \\
    % TRADES & 90.35 & 57.10 & 54.13 & 54.08 & 52.19 & 86.89 & 86.73 & 86.57 & 86.70 &  49.50 \\
    $\DIAL_{\kl}$ (Ours) & 90.66 & \textbf{58.91} & \textbf{55.30} & \textbf{55.11} & \textbf{53.67} & 87.62 & 87.52 & 87.41 & 87.63 & \textbf{51.00} \\
    $\DIAL_{\ce}$ (Ours) & \textbf{92.88} & 55.26  & 50.82 & 50.54 & 49.66 & \textbf{89.12} & \textbf{89.01} & \textbf{88.74} & \textbf{89.10} &  46.52  \\
    \bottomrule
  \end{tabular}
\end{table}


\begin{table}[!ht]
  \caption{Robustness against white-box, black-box attacks and Auto-Attack (AA) on CIFAR100. Black-box attacks are generated using naturally trained surrogate model. Natural represents the naturally trained (non-adversarial) model.
  %and applied to the best performing robust models.
  }
  \vskip 0.1in
  \label{black-and_white-cifar100}
  \centering
  \small
  \begin{tabular}{l@{\hspace{1\tabcolsep}}c@{\hspace{1\tabcolsep}}c@{\hspace{1\tabcolsep}}c@{\hspace{1\tabcolsep}}c@{\hspace{1\tabcolsep}}c@{\hspace{1\tabcolsep}}c@{\hspace{1\tabcolsep}}c@{\hspace{1\tabcolsep}}c@{\hspace{1\tabcolsep}}c@{\hspace{1\tabcolsep}}c}
    \toprule
    & & \multicolumn{4}{c}{White-box} & \multicolumn{4}{c}{Black-Box}  \\
    \cmidrule(r){3-6} 
    \cmidrule(r){7-10}
    Defense Model & Natural & PGD$^{20}$ & PGD$^{100}$  & PGD$^{1000}$  & CW$^{\infty}$ & PGD$^{20}$ & PGD$^{100}$ & PGD$^{1000}$  & CW$^{\infty}$ & AA \\
    \midrule
    NATURAL & 79.30 & 0 & 0 & 0 & 0 & 0 & 0 & 0 & 0 & 0 \\
    \midrule
    AT & 56.73 & 29.57 & 28.45 & 28.39 & 26.6 & 55.52 & 55.29 & 55.26 & 55.40 & 24.12 \\
    % TRADES & 58.24 & 30.10 & 29.66 & 29.64 & 25.97 & 57.05 & 56.71 & 56.67 & 56.77 & 24.92 \\
    $\DIAL_{\kl}$ (Ours) & 58.47 & \textbf{31.19} & \textbf{30.50} & \textbf{30.42} & \textbf{26.91} & 57.16 & 56.81 & 56.80 & 57.00 & \textbf{25.87} \\
    $\DIAL_{\ce}$ (Ours) & \textbf{60.77} & 27.87 & 26.66 & 26.61 & 25.98 & \textbf{59.48} & \textbf{59.06} & \textbf{58.96} & \textbf{59.20} & 23.51  \\
    \bottomrule
  \end{tabular}
\end{table}


% \begin{table}[!ht]
%   \caption{Robustness comparison of DIAL to Madry et al. and TRADES defense models on the SVHN dataset under different PGD white-box attacks and the ensemble Auto-Attack (AA).}
%   \label{svhn}
%   \centering
%   \begin{tabular}{llllll|l}
%     \toprule
%     \cmidrule(r){1-5}
%     Defense Model & Natural & PGD$^{20}$ & PGD$^{100}$ & PGD$^{1000}$ & CW$_{\infty}$ & AA\\
%     \midrule
%     $\DIAL_{\kl}$ (Ours) & $\mathbf{90.66}$ & $\mathbf{58.91}$ & $\mathbf{55.30}$ & $\mathbf{55.12}$ & $\mathbf{53.67}$  & $\mathbf{51.00}$  \\
%     Madry et al. & 89.90 & 53.23 & 49.45 & 49.23 & 48.25 & 45.25  \\
%     TRADES & 90.35 & 57.10 & 54.13 & 54.08 & 52.19 & 49.50 \\
%     \bottomrule
%   \end{tabular}
% \end{table}


\subsection{Performance comparison on CIFAR-10} \label{defence-settings}
In this part, we evaluate the performance of DIAL compared to other well-known methods on CIFAR-10. 
%To be consistent with other methods, 
We follow the same experiment setups as in~\cite{madry2017towards, wang2019improving, zhang2019theoretically}. When experiment settings are not identical between tested methods, we choose the most commonly used settings, and apply it to all experiments. This way, we keep the comparison as fair as possible and avoid reporting changes in results which are caused by inconsistent experiment settings \citep{pang2020bag}. To show that our results are not caused because of what is referred to as \textit{obfuscated gradients}~\citep{athalye2018obfuscated}, we evaluate our method with same setup as in our defense model, under strong attacks (e.g., PGD$^{1000}$) in both white-box, black-box settings, Auto-Attack ~\citep{croce2020reliable}, unforeseen "natural" corruptions~\citep{hendrycks2018benchmarking}, and unforeseen adversaries. To make sure that the reported improvements are not caused by \textit{adversarial overfitting}~\citep{rice2020overfitting}, we report best robust results for each method on average of 3 restarts, while omitting one standard deviation (which is smaller than 0.2\% in all experiments). Additional results for CIFAR-10 as well as comprehensive evaluation on MNIST can be found in Appendix \ref{mnist-results} and \ref{additional_res}.
%To further keep the comparison consistent, we followed the same attack settings for all methods.


\begin{table}[ht]
  \caption{Robustness against white-box, black-box attacks and Auto-Attack (AA) on CIFAR-10. Black-box attacks are generated using naturally trained surrogate model. Natural represents the naturally trained (non-adversarial) model.
  %and applied to the best performing robust models.
  }
  \vskip 0.1in
  \label{black-and_white-cifar}
  \centering
  \small
  \begin{tabular}{cccccccc@{\hspace{1\tabcolsep}}c}
    \toprule
    & & \multicolumn{3}{c}{White-box} & \multicolumn{3}{c}{Black-Box} \\
    \cmidrule(r){3-5} 
    \cmidrule(r){6-8}
    Defense Model & Natural & PGD$^{20}$ & PGD$^{100}$ & CW$^{\infty}$ & PGD$^{20}$ & PGD$^{100}$ & CW$^{\infty}$ & AA \\
    \midrule
    NATURAL & 95.43 & 0 & 0 & 0 & 0 & 0 & 0 &  0 \\
    \midrule
    TRADES & 84.92 & 56.60 & 55.56 & 54.20 & 84.08 & 83.89 & 83.91 &  53.08 \\
    MART & 83.62 & 58.12 & 56.48 & 53.09 & 82.82 & 82.52 & 82.80 & 51.10 \\
    AT & 85.10 & 56.28 & 54.46 & 53.99 & 84.22 & 84.14 & 83.92 & 51.52 \\
    ATDA & 76.91 & 43.27 & 41.13 & 41.01 & 75.59 & 75.37 & 75.35 & 40.08\\
    $\DIAL_{\kl}$ (Ours) & 85.25 & $\mathbf{58.43}$ & $\mathbf{56.80}$ & $\mathbf{55.00}$ & 84.30 & 84.18 & 84.05 & \textbf{53.75} \\
    $\DIAL_{\ce}$ (Ours)  & $\mathbf{89.59}$ & 54.31 & 51.67 & 52.04 &$ \mathbf{88.60}$ & $\mathbf{88.39}$ & $\mathbf{88.44}$ & 49.85 \\
    \midrule
    $\DIAL_{\awp}$ (Ours) & $\mathbf{85.91}$ & $\mathbf{61.10}$ & $\mathbf{59.86}$ & $\mathbf{57.67}$ & $\mathbf{85.13}$ & $\mathbf{84.93}$ & $\mathbf{85.03}$  & \textbf{56.78} \\
    $\TRADES_{\awp}$ & 85.36 & 59.27 & 59.12 & 57.07 & 84.58 & 84.58 & 84.59 & 56.17 \\
    \bottomrule
  \end{tabular}
\end{table}



\paragraph{CIFAR-10 setup.} We use the wide residual network (WRN-34-10)~\citep{zagoruyko2016wide} architecture. %used in the experiments of~\cite{madry2017towards, wang2019improving, zhang2019theoretically}. 
Sidelong this architecture, we integrate a domain classification layer. To generate the adversarial domain dataset, we use a perturbation size of $\epsilon=0.031$. We apply 10 of inner maximization iterations with perturbation step size of 0.007. Batch size is set to 128, weight decay is set to $7e^{-4}$, and the model is trained for 100 epochs. Similar to the other methods, the initial learning rate was set to 0.1, and decays by a factor of 10 at iterations 75 and 90. 
%For being consistent with other methods, the natural images are padded with 4-pixel padding with 32-random crop and random horizontal flip. Furthermore, all methods are trained using SGD with momentum 0.9. For $\DIAL_{\kl}$, we balance the robust loss with $\lambda=6$ and the domains loss with $r=4$. For $\DIAL_{\ce}$, we balance the robust loss with $\lambda=1$ and the domains loss with $r=2$. 
%We also introduce a version of our method that incorporates the AWP double-perturbation mechanism, named DIAL-AWP.
%which is trained using the same learning rate schedule used in ~\cite{wu2020adversarial}, where the initial 0.1 learning rate decays by a factor of 10 after 100 and 150 iterations. 
See Appendix \ref{cifar10-additional-setup} for additional details.

\begin{table}[ht]
  \caption{Black-box attack using the adversarially trained surrogate models on CIFAR-10.}
  \vskip 0.1in
  \label{black-box-cifar-adv}
  \centering
  \small
  \begin{tabular}{ll|c}
    \toprule
    \cmidrule(r){1-2}
    Surrogate (source) model & Target model & robustness \% \\
    % \midrule
    \midrule
    TRADES & $\DIAL_{\ce}$ & $\mathbf{67.77}$ \\
    $\DIAL_{\ce}$ & TRADES & 65.75 \\
    \midrule
    MART & $\DIAL_{\ce}$ & $\mathbf{70.30}$ \\
    $\DIAL_{\ce}$ & MART & 64.91 \\
    \midrule
    AT & $\DIAL_{\ce}$ & $\mathbf{65.32}$ \\
    $\DIAL_{\ce}$ & AT  & 63.54 \\
    \midrule
    ATDA & $\DIAL_{\ce}$ & $\mathbf{66.77}$ \\
    $\DIAL_{\ce}$ & ATDA & 52.56 \\
    \bottomrule
  \end{tabular}
\end{table}

\paragraph{White-box/Black-box robustness.} 
%We evaluate all defense models using Auto-Attack, PGD$^{20}$, PGD$^{100}$, PGD$^{1000}$ and CW$_{\infty}$ with step size 0.003. We constrain all attacks by the same perturbation $\epsilon=0.031$. 
As reported in Table~\ref{black-and_white-cifar} and Appendix~\ref{additional_res}, our method achieves better robustness compared to the other methods. Specifically, in the white-box settings, our method improves robustness over~\citet{madry2017towards} and TRADES by 2\% 
%using the common PGD$^{20}$ attack 
while keeping higher natural accuracy. We also observe better natural accuracy of 1.65\% over MART while also achieving better robustness over all attacks. Moreover, our method presents significant improvement of up to 15\% compared to the the domain invariant method suggested by~\citet{song2018improving} (ATDA).
%in both natural and robust accuracy. 
When incorporating 
%the double-perturbation mechanism of 
AWP, our method improves the results of $\TRADES_{\awp}$ by almost 2\%.
%and reaches state-of-the-art results for robust models with no additional data. 
% Additional results are available in Appendix~\ref{additional_res}.
When tested on black-box settings, $\DIAL_{\ce}$ presents a significant improvement of more than 4.4\% over the second-best performing method, and up to 13\%. In Table~\ref{black-box-cifar-adv}, we also present the black-box results when the source model is taken from one of the adversarially trained models. %Then, we compare our model to each one of them both as the source model and target model. 
In addition to the improvement in black-box robustness, $\DIAL_{\ce}$ also manages to achieve better clean accuracy of more than 4.5\% over the second-best performing method.
% Moreover, based on the auto-attack leader-board \footnote{\url{https://github.com/fra31/auto-attack}}, our method achieves the 1st place among models without additional data using the WRN-34-10 architecture.

% \begin{table}
%   \caption{White-box robustness on CIFAR-10 using WRN-34-10}
%   \label{white-box-cifar-10}
%   \centering
%   \begin{tabular}{lllll}
%     \toprule
%     \cmidrule(r){1-2}
%     Defense Model & Natural & PGD$^{20}$ & PGD$^{100}$ & PGD$^{1000}$ \\
%     \midrule
%     TRADES ~\cite{zhang2019theoretically} & 84.92  & 56.6 & 55.56 & 56.43  \\
%     MART ~\cite{wang2019improving} & 83.62  & 58.12 & 56.48 & 56.55  \\
%     Madry et al. ~\cite{madry2017towards} & 85.1  & 56.28 & 54.46 & 54.4  \\
%     Song et al. ~\cite{song2018improving} & 76.91 & 43.27 & 41.13 & 41.02  \\
%     $\DIAL_{\ce}$ (Ours) & $ \mathbf{90}$  & 52.12 & 48.88 & 48.78  \\
%     $\DIAL_{\kl}$ (Ours) & 85.25 & $\mathbf{58.43}$ & $\mathbf{56.8}$ & $\mathbf{56.73}$ \\
%     \midrule
%     $\DIAL_{\kl}$+AWP (Ours) & $\mathbf{85.91}$ & $\mathbf{61.1}$ & - & -  \\
%     TRADES+AWP \cite{wu2020adversarial} & 85.36 & 59.27 & 59.12 & -  \\
%     % MART + AWP & 84.43 & 60.68 & 59.32 & -  \\
%     \bottomrule
%   \end{tabular}
% \end{table}


% \begin{table}
%   \caption{White-box robustness on MNIST}
%   \label{white-box-mnist}
%   \centering
%   \begin{tabular}{llllll}
%     \toprule
%     \cmidrule(r){1-2}
%     Defense Model & Natural & PGD$^{40}$ & PGD$^{100}$ & PGD$^{1000}$ \\
%     \midrule
%     TRADES ~\cite{zhang2019theoretically} & 99.48 & 96.07 & 95.52 & 95.22 \\
%     MART ~\cite{wang2019improving} & 99.38  & 96.99 & 96.11 & 95.74  \\
%     Madry et al. ~\cite{madry2017towards} & 99.41  & 96.01 & 95.49 & 95.36 \\
%     Song et al. ~\cite{song2018improving}  & 98.72 & 96.82 & 96.26 & 96.2  \\
%     $\DIAL_{\kl}$ (Ours) & 99.46 & 97.05 & 96.06 & 95.99  \\
%     $\DIAL_{\ce}$ (Ours) & $\mathbf{99.49}$  & $\mathbf{97.38}$ & $\mathbf{96.45}$ & $\mathbf{96.33}$ \\
%     \bottomrule
%   \end{tabular}
% \end{table}


% \paragraph{Attacking MNIST.} For consistency, we use the same perturbation and step sizes. For MNIST, we use $\epsilon=0.3$ and step size of $0.01$. The natural accuracy of our surrogate (source) model is 99.51\%. Attacks results are reported in Table~\ref{black-and_white-mnist}. It is worth noting that the improvement margin is not conclusive on MNIST as it is on CIFAR-10, which is a more complex task.

% \begin{table}
%   \caption{Black-box robustness on MNIST and CIFAR-10 using naturally trained surrogate model and best performing robust models}
%   \label{black-box-mnist-cifar}
%   \centering
%   \begin{tabular}{lllllll}
%     \toprule
%     & \multicolumn{3}{c}{MNIST} & \multicolumn{3}{c}{CIFAR-10} \\
%     \cmidrule(r){2-4} 
%     \cmidrule(r){5-7}  
%     Defense Model & PGD$^{40}$ & PGD$^{100}$ & PGD$^{1000}$ & PGD$^{20}$ & PGD$^{100}$ & PGD$^{1000}$ \\
%     \midrule
%     TRADES ~\cite{zhang2019theoretically} & 98.12 & 97.86 & 97.81 & 84.08 & 83.89 & 83.8 \\
%     MART ~\cite{wang2019improving} & 98.16 & 97.96 & 97.89  & 82.82 & 82.52 & 82.47 \\
%     Madry et al. ~\cite{madry2017towards}  & 98.05 & 97.73 & 97.78 & 84.22 & 84.14 & 83.96 \\
%     Song et al. ~\cite{song2018improving} & 97.74 & 97.28 & 97.34 & 75.59 & 75.37 & 75.11 \\
%     $\DIAL_{\kl}$ (Ours) & 98.14 & 97.83 & 97.87  & 84.3 & 84.18 & 84.0 \\
%     $\DIAL_{\ce}$ (Ours)  & $\mathbf{98.37}$ & $\mathbf{98.12}$ & $\mathbf{98.05}$  & $\mathbf{89.13}$ & $\mathbf{88.89}$ & $\mathbf{88.78}$ \\
%     \bottomrule
%   \end{tabular}
% \end{table}



% \subsubsection{Ensemble attack} In addition to the white-box and black-box settings, we evaluate our method on the Auto-Attack ~\citep{croce2020reliable} using $\ell_{\infty}$ threat model with perturbation $\epsilon=0.031$. Auto-Attack is an ensemble of parameter-free attacks. It consists of three white-box attacks: APGD-CE which is a step size-free version of PGD on the cross-entropy ~\citep{croce2020reliable}. APGD-DLR which is a step size-free version of PGD on the DLR loss ~\citep{croce2020reliable} and FAB which  minimizes the norm of the adversarial perturbations, and one black-box attack: square attack which is a query-efficient black-box attack ~\citep{andriushchenko2020square}. Results are presented in Table~\ref{auto-attack}. Based on the auto-attack leader-board \footnote{\url{https://github.com/fra31/auto-attack}}, our method achieves the 1st place among models without additional data using the WRN-34-10 architecture.

%Additional results can be found in Appendix ~\ref{additional_res}.

% \begin{table}
%   \caption{Auto-Attack (AA) on CIFAR-10 with perturbation size $\epsilon=0.031$ with $\ell_{\infty}$ threat model}
%   \label{auto-attack}
%   \centering
%   \begin{tabular}{lll}
%     \toprule
%     \cmidrule(r){1-2}
%     Defense Model & AA \\
%     \midrule
%     TRADES ~\cite{zhang2019theoretically} & 53.08  \\
%     MART ~\cite{wang2019improving} & 51.1  \\
%     Madry et al. ~\cite{madry2017towards} & 51.52    \\
%     Song et al.   ~\cite{song2018improving} & 40.18 \\
%     $\DIAL_{\ce}$ (Ours) & 47.33  \\
%     $\DIAL_{\kl}$ (Ours) & $\mathbf{53.75}$ \\
%     \midrule
%     DIAL-AWP (Ours) & $\mathbf{56.78}$ \\
%     TRADES-AWP \cite{wu2020adversarial} & 56.17 \\
%     \bottomrule
%   \end{tabular}
% \end{table}


% \begin{table}[!ht]
%   \caption{Auto-Attack (AA) Robustness (\%) on CIFAR-10 with $\epsilon=0.031$ using an $\ell_{\infty}$ threat model}
%   \label{auto-attack}
%   \centering
%   \begin{tabular}{cccccc|cc}
%     \toprule
%     % \multicolumn{8}{c}{Defence Model}  \\
%     % \cmidrule(r){1-8} 
%     TRADES & MART & Madry & Song & $\DIAL_{\ce}$ & $\DIAL_{\kl}$ & DIAL-AWP  & TRADES-AWP\\
%     \midrule
%     53.08 & 51.10 & 51.52 &  40.08 & 47.33  & $\mathbf{53.75}$ & $\mathbf{56.78}$ & 56.17 \\

%     \bottomrule
%   \end{tabular}
% \end{table}

% \begin{table}[!ht]
% \caption{$F_1$-robust measurement using PGD$^{20}$ attack in white-box and black-box settings on CIFAR-10}
%   \label{f1-robust}
%   \centering
%   \begin{tabular}{ccccccc|cc}
%     \toprule
%     % \multicolumn{8}{c}{Defence Model}  \\
%     % \cmidrule(r){1-8} 
%     Defense Model & TRADES & MART & Madry & Song & $\DIAL_{\kl}$ & $\DIAL_{\ce}$ & DIAL-AWP  & TRADES-AWP\\
%     \midrule
%     White-box & 0.659 & 0.666 & 0.657 & 0.518 & $\mathbf{0.675}$  & 0.643 & $\mathbf{0.698}$ & 0.682 \\
%     Black-box & 0.844 & 0.831 & 0.846 & 0.761 & 0.847 & $\mathbf{0.895}$ & $\mathbf{0.854}$ &  0.849 \\
%     \bottomrule
%   \end{tabular}
% \end{table}

\subsubsection{Robustness to Unforeseen Attacks and Corruptions}
\paragraph{Unforeseen Adversaries.} To further demonstrate the effectiveness of our approach, we test our method against various adversaries that were not used during the training process. We attack the model under the white-box settings with $\ell_{2}$-PGD, $\ell_{1}$-PGD, $\ell_{\infty}$-DeepFool and $\ell_{2}$-DeepFool \citep{moosavi2016deepfool} adversaries using Foolbox \citep{rauber2017foolbox}. We applied commonly used attack budget 
%(perturbation for PGD adversaries and overshot for DeepFool adversaries) 
with 20 and 50 iterations for PGD and DeepFool, respectively.
Results are presented in Table \ref{unseen-attacks}. As can be seen, our approach  gains an improvement of up to 4.73\% over the second best method under the various attack types and an average improvement of 3.7\% over all threat models.


\begin{table}[ht]
  \caption{Robustness on CIFAR-10 against unseen adversaries under white-box settings.}
  \vskip 0.1in
  \label{unseen-attacks}
  \centering
%   \small
  \begin{tabular}{c@{\hspace{1.5\tabcolsep}}c@{\hspace{1.5\tabcolsep}}c@{\hspace{1.5\tabcolsep}}c@{\hspace{1.5\tabcolsep}}c@{\hspace{1.5\tabcolsep}}c@{\hspace{1.5\tabcolsep}}c@{\hspace{1.5\tabcolsep}}c}
    \toprule
    Threat Model & Attack Constraints & $\DIAL_{\kl}$ & $\DIAL_{\ce}$ & AT & TRADES & MART & ATDA \\
    \midrule
    \multirow{2}{*}{$\ell_{2}$-PGD} & $\epsilon=0.5$ & 76.05 & \textbf{80.51} & 76.82 & 76.57 & 75.07 & 66.25 \\
    & $\epsilon=0.25$ & 80.98 & \textbf{85.38} & 81.41 & 81.10 & 80.04 & 71.87 \\\midrule
    \multirow{2}{*}{$\ell_{1}$-PGD} & $\epsilon=12$ & 74.84 & \textbf{80.00} & 76.17 & 75.52 & 75.95 & 65.76 \\
    & $\epsilon=7.84$ & 78.69 & \textbf{83.62} & 79.86 & 79.16 & 78.55 & 69.97 \\
    \midrule
    $\ell_{2}$-DeepFool & overshoot=0.02 & 84.53 & \textbf{88.88} & 84.15 & 84.23 & 82.96 & 76.08 \\\midrule
    $\ell_{\infty}$-DeepFool & overshoot=0.02 & 68.43 & \textbf{69.50} & 67.29 & 67.60 & 66.40 & 57.35 \\
    \bottomrule
  \end{tabular}
\end{table}


%%%%%%%%%%%%%%%%%%%%%%%%% conference version %%%%%%%%%%%%%%%%%%%%%%%%%%%%%%%%%%%%%
\paragraph{Unforeseen Corruptions.}
We further demonstrate that our method consistently holds against unforeseen ``natural'' corruptions, consists of 18 unforeseen diverse corruption types proposed by \citet{hendrycks2018benchmarking} on CIFAR-10, which we refer to as CIFAR10-C. The CIFAR10-C benchmark covers noise, blur, weather, and digital categories. As can be shown in Figure \ref{corruption}, our method gains a significant and consistent improvement over all the other methods. Our method leads to an average improvement of 4.7\% with minimum improvement of 3.5\% and maximum improvement of 5.9\% compared to the second best method over all unforeseen attacks. See Appendix \ref{corruptions-apendix} for the full experiment results.


\begin{figure}[h]
 \centering
  \includegraphics[width=0.4\textwidth]{figures/spider_full.png}
%   \caption{Summary of accuracy over all unforeseen corruptions compared to the second and third best performing methods.}
  \caption{Accuracy comparison over all unforeseen corruptions.}
  \label{corruption}
\end{figure}


%%%%%%%%%%%%%%%%%%%%%%%%% conference version %%%%%%%%%%%%%%%%%%%%%%%%%%%%%%%%%%%%%

%%%%%%%%%%%%%%%%%%%%%%%%% Arxiv version %%%%%%%%%%%%%%%%%%%%%%%%%%%%%%%%%%%%%
% \newpage
% \paragraph{Unforeseen Corruptions.}
% We further demonstrate that our method consistently holds against unforeseen "natural" corruptions, consists of 18 unforeseen diverse corruption types proposed by \cite{hendrycks2018benchmarking} on CIFAR-10, which we refer to as CIFAR10-C. The CIFAR10-C benchmark covers noise, blur, weather, and digital categories. As can be shown in Figure  \ref{spider-full-graph}, our method gains a significant and consistent improvement over all the other methods. Our approach leads to an average improvement of 4.7\% with minimum improvement of 3.5\% and maximum improvement of 5.9\% compared to the second best method over all unforeseen attacks. Full accuracy results against unforeseen corruptions are presented in Tables \ref{corruption-table1} and \ref{corruption-table2}. 

% \begin{table}[!ht]
%   \caption{Accuracy (\%) against unforeseen corruptions.}
%   \label{corruption-table1}
%   \centering
%   \tiny
%   \begin{tabular}{lcccccccccccccccccc}
%     \toprule
%     Defense Model & brightness & defocus blur & fog & glass blur & jpeg compression & motion blur & saturate & snow & speckle noise  \\
%     \midrule
%     TRADES & 82.63 & 80.04 & 60.19 & 78.00 & 82.81 & 76.49 & 81.53 & 80.68 & 80.14 \\
%     MART & 80.76 & 78.62 & 56.78 & 76.60 & 81.26 & 74.58 & 80.74 & 78.22 & 79.42 \\
%     AT &  83.30 & 80.42 & 60.22 & 77.90 & 82.73 & 76.64 & 82.31 & 80.37 & 80.74 \\
%     ATDA & 72.67 & 69.36 & 45.52 & 64.88 & 73.22 & 63.47 & 72.07 & 68.76 & 72.27 \\
%     DIAL (Ours)  & \textbf{87.14} & \textbf{84.84} & \textbf{66.08} & \textbf{81.82} & \textbf{87.07} & \textbf{81.20} & \textbf{86.45} & \textbf{84.18} & \textbf{84.94} \\
%     \bottomrule
%   \end{tabular}
% \end{table}


% \begin{table}[!ht]
%   \caption{Accuracy (\%) against unforeseen corruptions.}
%   \label{corruption-table2}
%   \centering
%   \tiny
%   \begin{tabular}{lcccccccccccccccccc}
%     \toprule
%     Defense Model & contrast & elastic transform & frost & gaussian noise & impulse noise & pixelate & shot noise & spatter & zoom blur \\
%     \midrule
%     TRADES & 43.11 & 79.11 & 76.45 & 79.21 & 73.72 & 82.73 & 80.42 & 80.72 & 78.97 \\
%     MART & 41.22 & 77.77 & 73.07 & 78.30 & 74.97 & 81.31 & 79.53 & 79.28 & 77.8 \\
%     AT & 43.30 & 79.58 & 77.53 & 79.47 & 73.76 & 82.78 & 80.86 & 80.49 & 79.58 \\
%     ATDA & 36.06 & 67.06 & 62.56 & 70.33 & 64.63 & 73.46 & 72.28 & 70.50 & 67.31 \\
%     DIAL (Ours) & \textbf{48.84} & \textbf{84.13} & \textbf{81.76} & \textbf{83.76} & \textbf{78.26} & \textbf{87.24} & \textbf{85.13} & \textbf{84.84} & \textbf{83.93}  \\
%     \bottomrule
%   \end{tabular}
% \end{table}


% \begin{figure}[!ht]
%   \centering
%   \includegraphics[width=9cm]{figures/spider_full.png}
%   \caption{Accuracy comparison with all tested methods over unforeseen corruptions.}
%   \label{spider-full-graph}
% \end{figure}
% %%%%%%%%%%%%%%%%%%%%%%%%% Arxiv version %%%%%%%%%%%%%%%%%%%%%%%%%%%%%%%%%%%%%
%%%%%%%%%%%%%%%%%%%%%%%%% Arxiv version %%%%%%%%%%%%%%%%%%%%%%%%%%%%%%%%%%%%%

\subsubsection{Transfer Learning}
Recent works \citep{salman2020adversarially,utrera2020adversarially} suggested that robust models transfer better on standard downstream classification tasks. In Table \ref{transfer-res} we demonstrate the advantage of our method when applied for transfer learning across CIFAR10 and CIFAR100 using the common linear evaluation protocol. see Appendix \ref{transfer-learning-settings} for detailed settings.

\begin{table}[H]
  \caption{Transfer learning results comparison.}
  \vskip 0.1in
  \label{transfer-res}
  \centering
  \small
\begin{tabular}{c|c|c|c}
\toprule

\multicolumn{2}{l}{} & \multicolumn{2}{c}{Target} \\
\cmidrule(r){3-4}
Source & Defence Model & CIFAR10 & CIFAR100 \\
\midrule
\multirow{3}{*}{CIFAR10} & DIAL & \multirow{3}{*}{-} & \textbf{28.57} \\
 & AT &  & 26.95  \\
 & TRADES &  & 25.40  \\
 \midrule
\multirow{3}{*}{CIFAR100} & DIAL & \textbf{73.68} & \multirow{3}{*}{-} \\
 & AT & 71.41 & \\
 & TRADES & 71.42 &  \\
%  \midrule
% \multirow{3}{}{SVHN} & DIAL &  &  & \multirow{3}{}{-} \\
%  & Madry et al. &  &  &  \\
%  & TRADES &  &  &  \\ 
\bottomrule
\end{tabular}
\end{table}


\subsubsection{Modularity and Ablation Studies}

We note that the domain classifier is a modular component that can be integrated into existing models for further improvements. Removing the domain head and related loss components from the different DIAL formulations results in some common adversarial training techniques. For $\DIAL_{\kl}$, removing the domain and related loss components results in the formulation of TRADES. For $\DIAL_{\ce}$, removing the domain and related loss components results in the original formulation of the standard adversarial training, and for $\DIAL_{\awp}$ the removal results in $\TRADES_{\awp}$. Therefore, the ablation studies will demonstrate the effectiveness of combining DIAL on top of different adversarial training methods. 

We investigate the contribution of the additional domain head component introduced in our method. Experiment configuration are as in \ref{defence-settings}, and robust accuracy is based on white-box PGD$^{20}$ on CIFAR-10 dataset. We remove the domain head from both $\DIAL_{\kl}$, $\DIAL_{\awp}$, and $\DIAL_{\ce}$ (equivalent to $r=0$) and report the natural and robust accuracy. We perform 3 random restarts and omit one standard deviation from the results. Results are presented in Figure \ref{ablation}. All DIAL variants exhibits stable improvements on both natural accuracy and robust accuracy. $\DIAL_{\ce}$, $\DIAL_{\kl}$, and $\DIAL_{\awp}$ present an improvement of 1.82\%, 0.33\%, and 0.55\% on natural accuracy and an improvement of 2.5\%, 1.87\%, and 0.83\% on robust accuracy, respectively. This evaluation empirically demonstrates the benefits of incorporating DIAL on top of different adversarial training techniques.
% \paragraph{semi-supervised extensions.} Since the domain classifier does not require the class labels, we argue that additional unlabeled data can be leveraged in future work.
%for improved results. 

\begin{figure}[ht]
  \centering
  \includegraphics[width=0.35\textwidth]{figures/ablation_graphs3.png}
  \caption{Ablation studies for $\DIAL_{\kl}$, $\DIAL_{\ce}$, and $\DIAL_{\awp}$ on CIFAR-10. Circle represent the robust-natural accuracy without using DIAL, and square represent the robust-natural accuracy when incorporating DIAL.
  %to further investigate the impact of the domain head and loss on natural and robust accuracy.
  }
  \label{ablation}
\end{figure}

\subsubsection{Visualizing DIAL}
To further illustrate the superiority of our method, we visualize the model outputs from the different methods on both natural and adversarial test data.
% adversarial test data generated using PGD$^{20}$ white-box attack with step size 0.003 and $\epsilon=0.031$ on CIFAR-10. 
Figure~\ref{tsne1} shows the embedding received after applying t-SNE ~\citep{van2008visualizing} with two components on the model output for our method and for TRADES. DIAL seems to preserve strong separation between classes on both natural test data and adversarial test data. Additional illustrations for the other methods are attached in Appendix~\ref{additional_viz}. 

\begin{figure}[h]
\centering
  \subfigure[\textbf{DIAL} on natural logits]{\includegraphics[width=0.21\textwidth]{figures/domain_ce_test.png}}
  \hspace{0.03\textwidth}
  \subfigure[\textbf{DIAL} on adversarial logits]{\includegraphics[width=0.21\textwidth]{figures/domain_ce_adversarial.png}}
  \hspace{0.03\textwidth}
    \subfigure[\textbf{TRADES} on natural logits]{\includegraphics[width=0.21\textwidth]{figures/trades_test.png}}
    \hspace{0.03\textwidth}
    \subfigure[\textbf{TRADES} on adversarial logits]{\includegraphics[width=0.21\textwidth]{figures/trades_adversarial.png}}
  \caption{t-SNE embedding of model output (logits) into two-dimensional space for DIAL and TRADES using the CIFAR-10 natural test data and the corresponding PGD$^{20}$ generated adversarial examples.}
  \label{tsne1}
\end{figure}


% \begin{figure}[ht]
% \centering
%   \begin{subfigure}{4cm}
%     \centering\includegraphics[width=3.3cm]{figures/domain_ce_test.png}
%     \caption{\textbf{DIAL} on nat. examples}
%   \end{subfigure}
%   \begin{subfigure}{4cm}
%     \centering\includegraphics[width=3.3cm]{figures/domain_ce_adversarial.png}
%     \caption{\textbf{DIAL} on adv. examples}
%   \end{subfigure}
  
%   \begin{subfigure}{4cm}
%     \centering\includegraphics[width=3.3cm]{figures/trades_test.png}
%     \caption{\textbf{TRADES} on nat. examples}
%   \end{subfigure}
%   \begin{subfigure}{4cm}
%     \centering\includegraphics[width=3.3cm]{figures/trades_adversarial.png}
%     \caption{\textbf{TRADES} on adv. examples}
%   \end{subfigure}
%   \caption{t-SNE embedding of model output (logits) into two-dimensional space for DIAL and TRADES using the CIFAR-10 natural test data and the corresponding adversarial examples.}
%   \label{tsne1}
% \end{figure}



% \begin{figure}[ht]
% \centering
%   \begin{subfigure}{6cm}
%     \centering\includegraphics[width=5cm]{figures/domain_ce_test.png}
%     \caption{\textbf{DIAL} on nat. examples}
%   \end{subfigure}
%   \begin{subfigure}{6cm}
%     \centering\includegraphics[width=5cm]{figures/domain_ce_adversarial.png}
%     \caption{\textbf{DIAL} on adv. examples}
%   \end{subfigure}
  
%   \begin{subfigure}{6cm}
%     \centering\includegraphics[width=5cm]{figures/trades_test.png}
%     \caption{\textbf{TRADES} on nat. examples}
%   \end{subfigure}
%   \begin{subfigure}{6cm}
%     \centering\includegraphics[width=5cm]{figures/trades_adversarial.png}
%     \caption{\textbf{TRADES} on adv. examples}
%   \end{subfigure}
%   \caption{t-SNE embedding of model output (logits) into two-dimensional space for DIAL and TRADES using the CIFAR-10 natural test data and the corresponding adversarial examples.}
%   \label{tsne1}
% \end{figure}



\subsection{Balanced measurement for robust-natural accuracy}
One of the goals of our method is to better balance between robust and natural accuracy under a given model. For a balanced metric, we adopt the idea of $F_1$-score, which is the harmonic mean between the precision and recall. However, rather than using precision and recall, we measure the $F_1$-score between robustness and natural accuracy,
using a measure we call
%We named it
the
\textbf{$\mathbf{F_1}$-robust} score.
\begin{equation}
F_1\text{-robust} = \dfrac{\text{true\_robust}}
{\text{true\_robust}+\frac{1}{2}
%\cdot
(\text{false\_{robust}}+
\text{false\_natural})},
\end{equation}
where $\text{true\_robust}$ are the adversarial examples that were correctly classified, $\text{false\_{robust}}$ are the adversarial examples that were miss-classified, and $\text{false\_natural}$ are the natural examples that were miss-classified.
%We tested the proposed $F_1$-robust score using PGD$^{20}$ on CIFAR-10 dataset in white-box and black-box settings. 
Results are presented in Table~\ref{f1-robust} and demonstrate that our method achieves the best $F_1$-robust score in both settings, which supports our findings from previous sections.

% \begin{table}[!ht]
%   \caption{$F_1$-robust measurement using PGD$^{20}$ attack in white and black box settings on CIFAR-10}
%   \label{f1-robust}
%   \centering
%   \begin{tabular}{lll}
%     \toprule
%     \cmidrule(r){1-2}
%     Defense Model & White-box & Black-box \\
%     \midrule
%     TRADES & 0.65937  & 0.84435 \\
%     MART & 0.66613  & 0.83153  \\
%     Madry et al. & 0.65755 & 0.84574   \\
%     Song et al. & 0.51823 & 0.76092  \\
%     $\DIAL_{\ce}$ (Ours) & 0.65318   & $\mathbf{0.88806}$  \\
%     $\DIAL_{\kl}$ (Ours) & $\mathbf{0.67479}$ & 0.84702 \\
%     \midrule
%     \midrule
%     DIAL-AWP (Ours) & $\mathbf{0.69753}$  & $\mathbf{0.85406}$  \\
%     TRADES-AWP & 0.68162 & 0.84917 \\
%     \bottomrule
%   \end{tabular}
% \end{table}

\begin{table}[ht]
\small
  \caption{$F_1$-robust measurement using PGD$^{20}$ attack in white and black box settings on CIFAR-10.}
  \vskip 0.1in
  \label{f1-robust}
  \centering
%   \small
  \begin{tabular}{c
  @{\hspace{1.5\tabcolsep}}c @{\hspace{1.5\tabcolsep}}c @{\hspace{1.5\tabcolsep}}c @{\hspace{1.5\tabcolsep}}c
  @{\hspace{1.5\tabcolsep}}c @{\hspace{1.5\tabcolsep}}c @{\hspace{1.5\tabcolsep}}|
  @{\hspace{1.5\tabcolsep}}c
  @{\hspace{1.5\tabcolsep}}c}
    \toprule
    % \cmidrule(r){8-9}
     & TRADES & MART & AT & ATDA & $\DIAL_{\ce}$ & $\DIAL_{\kl}$ & $\DIAL_{\awp}$ & $\TRADES_{\awp}$ \\
    \midrule
    White-box & 0.659 & 0.666 & 0.657 & 0.518 & 0.660 & \textbf{0.675} & \textbf{0.698} & 0.682 \\
    Black-box & 0.844 & 0.831 & 0.845 & 0.761 & \textbf{0.890} & 0.847 & \textbf{0.854} & 0.849 \\ 
    \bottomrule
  \end{tabular}
\end{table}

\section{Non-constant queries and commands} \label{sec:non-constant}
As we have shown, connectivity queries propagate through the \XStream ring
processings in $p$ \XStream ticks, and the query answer is sent from the
tail processor to the head, then back to the I/O processor. Another potentially
useful query that finishes in $p$ \XStream ticks is ``How many edges are in the system?''
\Cindy{Add some more here.  It can just be a list. Many that one might think should be constant probably require some data structure support, in whcih case, add them at the end.} \Cindy{Revisit notation.}

\XStream also supports queries with non-constant-sized output. At most
one such query can be active at a time. The answer to the query is
output in constant-sized pieces using the payload slots. The canonical
non-constant query is a request to output all vertices in small
connected components. Specially, the answer is the names of all
components with at most $\lambda$ vertices and the list of vertices
within them.  This query makes practical sense only in graphs that
have a giant connected component, but most real graphs have one. We
describe how \XStream executes this specific query.

For a local component with name $\eta$, let $s_{\eta}$ be the number
of vertices in $\eta$.  A processor can compute the size of a local
component as the sum of the number of vertices in each building block.
This is $1$ for a primitive building block.  For this discussion, we
assume processors keep track of the number of primitive building
blocks for each local component while building these components. This
adds only constant work per union-find operation.  However, it's also
possible to inialize local-component sizes to zero and compute them
on-the-fly for this query.  But then, the processor does at most k-1
work counting primitive building blocks or outputing the messages
below, which will further delay the query response. Processors receive
the size of non-primitive building blocks from upstream processors.

When the head processor receives the query ``Output the vertices in
components that have at most $\lambda$ vertices'' in the primary slot
of a bundle, it passes the query downstream in the primary slot. This
allows all processors to learn the type of query and the parameter
$\lambda$. The head then uses the $k-1$ payload slots to start
answering the query.  The query is answered in two phases.  In the
first phase, processors compute component sizes.  For each local
component with name $\eta$, such that $s_{\eta} \le \lambda$, the head
processor (eventually) sends a message ``($\eta, s_{\eta})$'' in a
payload basket. The head outputs $k-1$ of these messages per bundle if
it already knows its component sizes. After the last message, it
outputs a ``query phase done'' token.

Each downstream processor passes the initial query downstream.  Then
for each message $(\eta, s_{\eta})$, the processor checks to see if
$\eta$ is a building block for one of its local components $\eta'$.
If it is, then it increments the size of $\eta'$.  If $\eta$ is not a
local building block, the processor sends the message downstream.
When the processor receives the ``query phase done'' token, it knows
the size of all its non-primitive building blocks, and hence knows
the size of all of its local components.  It sends its own ``($\eta,
s_{\eta})$'' messages for each local component $\eta$ such that
$s_{\eta} \le \lambda$. When it has sent all its messages, it passes
the ``query phase done'' token downstream. If the current graph has a
connected component $\eta_G$ that has size at most $\lambda$, the
message with its final size is passed through the tail and out to the
analyst.  The tail also passes the ``query phase done'' token to the
head.

Sealed processors (full of tree edges) can set a flag indicating they
have computed their component sizes. If there is another such query
before an aging, then it removes messages associated with its local
building blocks without incrementing any size counters.

In the second phase, the head processor (eventually) sends a message
$(\eta, v_i)$, for each primitive vertex $v_i$ in each local component
$\eta$ reported in the first phase.  For the head, all building blocks
are primitive vertices. It's possible to put more than one vertex in
the latter kind of message (e.g. $(\eta, v_1, v_2, v_3)$, depending
upon the size of a slot.  After the last such message, the head passes
a ``query done'' token downstream.

When a downstream processor receives a message $(\eta, v_i)$ from
upstream in the second phase, it checks to see if $\eta$ is a building
block for one of its local components $\eta'$. If not, then it passes
the message downstream.  If so, then $s_{\eta'} \le \lambda$ (i.e. the processor
reported local component $\eta'$ in the first phase, it relabels the message,
sending $(\eta', v_i)$ downstream.  If $\eta'$ is too large, it just removes
the message from the system.

When a downstream processor receives the ``query done'' token, it outputs
messages $(\eta, v_i)$, where $\eta$ is a local componet with $s_{\eta} \le \lambda$
and $v_i$ is a primitive building block (vertex) in local component $\eta$.

A somewhat easier non-constant query is spanning tree. Starting with the head, each
processor outputs its tree edges.

Some queries can be either constant-size (latency $p$) or non-constant
depending upon what additional data structures the processors
maintain.  One example is ``What is the degree of node $v$?'' Suppose
each processor maintains adjacency lists for the subgraph it
holds. Then the processor can find the number of edges adjacent to a
vertex $v$ in constant time, given a hash table to access the
adjacency list for each vertex.  In this case, the vertex-degree query
has latency $p$.  The query makes one pass around the ring with the answer progressing
one processor per tick.  Otherwise, without this data structure, each processor will need time to
compute the number of edges it holds that are adjacent to vertex $v$.
In this case, it is a non-constant query.  The message still touches
each processor once, but the processor may require multiple ticks to
compute the number to add to the accumulating degree value.

Linear algebraic computations typically involve a matrix-vector product,
which would be unweildy to compute directly in the \XStream model.
However, the emerging field of randomized linear
algebra~\cite{drineas2018lectures}
offers a path forward. If we devote some space in the tail processor to
accommodate a sample of edges (adjusting Lemma~\ref{lemma:aging-lead-time}
accordingly), payload slots can be used to accumulate a random sample
of the graph.  Techniques such as randomized 
PageRank~\cite{gasnikov2015efficient} or others might then be applied in a 
separate
thread in the tail processor, still with minimal interruption to the input 
stream.

\section{Conclusion and Future work}
\section{Conclusions}
\label{sec:conclusions}

In this paper, we apply shared-workload techniques at the \sql level for
improving the throughput of \qaasl systems without incurring in additional
query execution costs. Our approach is based on query rewriting for grouping
multiple queries together into a single query to be executed in one go. This
results in a significant reduction of the aggregated data access done by the
shared execution compared to executing queries independently.

%execution times and costs of the shared scan operator when
%varying query selectivity and predicate evaluation. We observed that for
%\athena, although the cost only depends on the amount of data read, it is
%conditioned to its ability to use its statistics about the data. In some cases
%a wrong query execution plan leads to higher query execution costs, which the
%end-user has to pay. 

%\bigquery's minimum query execution cost is determined by
%the input size of a query.  However, the query cost can increase depending not
%just in the amount of computation it requires, but in the mix of resources the
%query requires.  

We presented a cost and runtime evaluation of the shared operator driving data access costs. 
Our experimental study using the TPC-H benchmark confirmed the benefits of our
query rewrite approach. Using a shared execution approach reduces significantly
the execution costs. For \athena, we are able to make it 107x cheaper and for
\bigquery, 16x cheaper taking into account Query 10 which we cannot execute,
but 128x if it is not taken into account. Moreover, when having queries that do
not share their entire execution plan, i.e., using a single global plan, we
demonstrated that it is possible to improve throughput and obtain a 10x cost
reduction in \bigquery.

%We followed the TPC systems pricing guideline for
%computing how expensive is to have a TPC-H workload working on the evaluated
%\qaasl systems. The result is that even though we are able to reduce overall
%costs a TPC-H workload in 15x for \bigquery (128x excluding query 10 which we
%could not optimize) and in 107x for \athena, the overall price is at least 10x
%more expensive than the cheapest system price published by the TPC.

There are multiple ways to extend our work. The first is
to implement a full \sql-to-\sql translation layer to encapsulate the proposed
per-operator rewrites.  Another one is to incorporate the initial work on
building a cost-based optimizer for shared execution
\cite{Giannikis:2014:SWO:2732279.2732280} as an external component for \qaasl
systems.  Moreover, incorporating different lines of work (e.g., adding
provenance computation \cite{GA09} capabilities) also based on query
rewriting is part of our future work to enhance our system.



\begin{acknowledgements}
Sandia National Laboratories is a multimission laboratory managed and operated by National Technology \& Engineering Solutions of Sandia, LLC, a wholly owned subsidiary of Honeywell International Inc., for the U.S. Department of Energy’s National Nuclear Security Administration under contract DE-NA0003525.
This research was funded through the Laboratory Directed Research and Development (LDRD) program at Sandia.
This paper describes objective technical results and analysis. Any subjective views or opinions that might be expressed in the paper do not necessarily represent the views of the U.S. Department of Energy or the United States Government.
We thank Siva Rajamanickam, Cannada Lewis, and Si Hammond for useful
discussions and baseline code for the TBB benchmark.
\end{acknowledgements}


% Authors must disclose all relationships or interests that 
% could have direct or potential influence or impart bias on 
% the work: 
%
% \section*{Conflict of interest}
%
% The authors declare that they have no conflict of interest.


% BibTeX users please use one of
\bibliographystyle{plain}      % basic style, author-year citations
%\bibliographystyle{spmpsci}      % mathematics and physical sciences
%\bibliographystyle{spphys}       % APS-like style for physics
%\bibliography{}   % name your BibTeX data base
\bibliography{ms}


\end{document}
% end of file template.tex

