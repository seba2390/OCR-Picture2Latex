%%%%%%%%%%%%%%%%%%%%%%% file template.tex %%%%%%%%%%%%%%%%%%%%%%%%%
%
% This is a general template file for the LaTeX package SVJour3
% for Springer journals.          Springer Heidelberg 2010/09/16
%
% Copy it to a new file with a new name and use it as the basis
% for your article. Delete % signs as needed.
%
% This template includes a few options for different layouts and
% content for various journals. Please consult a previous issue of
% your journal as needed.
%
%%%%%%%%%%%%%%%%%%%%%%%%%%%%%%%%%%%%%%%%%%%%%%%%%%%%%%%%%%%%%%%%%%%
%
% First comes an example EPS file -- just ignore it and
% proceed on the \documentclass line
% your LaTeX will extract the file if required
\begin{filecontents*}{example.eps}
%!PS-Adobe-3.0 EPSF-3.0
%%BoundingBox: 19 19 221 221
%%CreationDate: Mon Sep 29 1997
%%Creator: programmed by hand (JK)
%%EndComments
gsave
newpath
  20 20 moveto
  20 220 lineto
  220 220 lineto
  220 20 lineto
closepath
2 setlinewidth
gsave
  .4 setgray fillt
grestore
stroke
grestore
\end{filecontents*}
%
\RequirePackage{fix-cm}
%
%\documentclass{svjour3}                     % onecolumn (standard format)
%\documentclass[smallcondensed]{svjour3}     % onecolumn (ditto)
%\documentclass[smallextended]{svjour3}       % onecolumn (second format)
\documentclass[twocolumn]{svjour3}          % twocolumn
%
\smartqed  % flush right qed marks, e.g. at end of proof
%
\usepackage{graphicx}
\usepackage{algorithm}
\usepackage[noend]{algpseudocode}
\usepackage{color}
\usepackage{xcolor}
\usepackage{algpseudocode,caption}
\usepackage{tcolorbox}

\usepackage{amsfonts,amsmath,graphicx,subfigure}
\algnewcommand\algorithmicswitch{\textbf{switch}}
\algnewcommand\algorithmiccase{\textbf{case}}
\algnewcommand\algorithmicassert{\texttt{assert}}
\algnewcommand\Assert[1]{\State \algorithmicassert(#1)}%
% New "environments"
\algdef{SE}[SWITCH]{Switch}{EndSwitch}[1]{\algorithmicswitch\ #1\ \algorithmicdo}   {\algorithmicend\ \algorithmicswitch}%
\algdef{SE}[CASE]{Case}{EndCase}[1]{\algorithmiccase\ #1}{\algorithmicend\     \algorithmiccase}%
\algtext*{EndSwitch}%
\algtext*{EndCase}%
\algnewcommand\algorithmicleftcomment{$\triangleright $}
\algnewcommand\LeftComment{\State\algorithmicleftcomment\ }%for no line # do \item[\algorithmicleftcomment]


\makeatletter
% The 'plainruled' float style
\newcommand\fs@plainruled{\def\@fs@cfont{\rmfamily}\let\@fs@capt\floatc@plainruled
  \def\@fs@pre{\hrule height.8pt depth0pt \kern2pt}%
  \def\@fs@post{}%
  \def\@fs@mid{\kern4pt\hrule height.8pt depth0pt \kern5pt}%
  \let\@fs@iftopcapt\iffalse}
\makeatother

\floatstyle{plainruled}
\restylefloat{algorithm}



\newtheorem{defn}{Definition}
\newtheorem{invariant}{Invariant}
\newtheorem{observation}{Observation}

%\newcommand{\Alex}[1]{{\footnotesize \sf {\color{black!50!green}{\bf Alex:} #1}}}
%\newcommand{\Jon}[1]{{\footnotesize \sf {\color{blue}{\bf Jon:} #1}}}
%\newcommand{\Cindy}[1]{{\footnotesize \sf {\color{red}{\bf Cindy:} #1}}}
\newcommand{\Alex}[1]{}
\newcommand{\Jon}[1]{}
\newcommand{\Cindy}[1]{}


\newcommand{\makealgtitle}{ {\vspace{-0.2cm}  \hrule height.8pt depth0pt \kern2pt}}
\newcommand{\nodeloc}{\beta}
\newcommand{\lcloc}{\alpha}
\newcommand{\nodeusr}{\gamma}

\newcommand{\DFRns}{\mbox{DFR}}
\newcommand{\DFR}{\mbox{DFR\ }}

\newcommand{\WStreamns}{\mbox{W-Stream}}
\newcommand{\WStream}{\mbox{W-Stream\ }}

\newcommand{\XSns}{\mbox{XS}}
\newcommand{\XS}{\mbox{XS\ }}

\newcommand{\XSCCns}{\mbox{XS-CC}}
\newcommand{\XSCC}{\mbox{XS-CC\ }}

\newcommand{\XStreamns}{\mbox{X-Stream}}
\newcommand{\XStream}{\mbox{X-Stream\ }}

\newcommand{\ufns}{\mbox{union-find}}
\newcommand{\uf}{\mbox{union-find\ }}

% bundle-> a group of slots passed between pe's
% as opposed to block-> building block or local component
% we were using "block" 
\newcommand{\bundlens}{\mbox{bundle}}
\newcommand{\bundle}{\mbox{bundle\ }}

\MakeRobust{\Call}

%
% \usepackage{mathptmx}      % use Times fonts if available on your TeX system
%
% insert here the call for the packages your document requires
%\usepackage{latexsym}
% etc.
%
% please place your own definitions here and don't use \def but
% \newcommand{}{}
%
% Insert the name of "your journal" with
% \journalname{myjournal}
%
\begin{document}

\title{Connected Components for Infinite Graph Streams: Theory and Practice}

%\titlerunning{Short form of title}        % if too long for running head

\author{Jonathan Berry \and Cynthia Phillips \and Alexandra\ Porter}


%\authorrunning{Short form of author list} % if too long for running head

\institute{
           J. Berry \at
           Sandia National Laboratories\\
	\email{jberry@sandia.gov} 
	\and
	C. Phillips \at
           Sandia National Laboratories\\
	   \email{caphill@sandia.gov
        \and
       A. Porter \at
              Stanford University \\
              \email{amporter@cs.stanford.edu}           %  \\
%             \emph{Present address:} of F. Author  %  if needed
          } 
}

%\date{Received: date / Accepted: date}
\date{\today}
% The correct dates will be entered by the editor


\maketitle

\begin{abstract}
\begin{abstract}
\label{sec:abstract}

%% 1. what is the problem 
Scientific applications that run on leadership computing facilities often face the challenge 
of being unable to fit leading science cases onto accelerator devices due to memory constraints 
(memory-bound applications).
%
% 2. what is your solution 
In this work, the authors studied one such US Department of Energy mission-critical condensed matter 
physics application, Dynamical Cluster Approximation (DCA++), and this paper discusses how device memory-bound challenges were successfully reduced  by proposing an effective 
``all-to-all'' communication method---a ring communication algorithm. 
%
This implementation takes advantage of acceleration on GPUs and remote direct memory access (RDMA) for fast data exchange between GPUs. 
%
\\Additionally, the ring algorithm was optimized with sub-ring communicators
and multi-threaded support to further reduce communication overhead and 
expose more concurrency, respectively.
%
% 3. What's the cherry-picked evaluation result you want to mention
The computation and communication were also analyzed 
by using the Autonomic Performance Environment for Exascale 
(APEX) profiling tool,  and this paper further discusses the 
performance trade-off for the ring algorithm implementation. 
%
The memory analysis on the ring algorithm shows that the allocation size for the authors' most 
memory-intensive data structure per GPU is now reduced to $1/p$ of the original size, where $p$ is the number of GPUs in the ring communicator.
%
The communication analysis suggests that 
the distributed Quantum Monte Carlo execution time grows linearly as sub-ring size increases, and the cost of messages passing through the network interface connector could be a limiting factor.


%
% \todoRed{Ronnie: Next sentence needs rewrite, too much information about Green's function that no one knows in the abstract; recommend generalizing.} \emph {However, DCA++ is currently facing memory-bound challenge as 
% a larger device array $G_t$ is limited by device memory size, where
% $G_t$ is a two-particle Green's function that allows condensed matter
% scientists to explore larger and more complex (higher fidelity)
% physics cases.}

\end{abstract}

\keywords{DCA++, Quantum Monte Carlo, GPU Remote Direct Memory Access, memory-bound issue, exascale machines}

\keywords{streaming\and graph algorithms \and dynamic graphs \and connected 
components}
% \PACS{PACS code1 \and PACS code2 \and more}
% \subclass{MSC code1 \and MSC code2 \and more}
\end{abstract}

\section{Introduction} \label{sec:intro}
Reinforcement learning has achieved great success in areas such as Game-playing \citep{silver2018general,vinyals2019grandmaster}, robotics \cite{kober2013reinforcement}, large language models \citep{ouyang2022training}, etc.
However, due to safety concerns or physical limitations, in some real-world reinforcement learning problems, we must consider additional constraints that may influence the optimal policy and the learning process \citep{garcia2015comprehensive}.
% For example, a robotic arm must not take actions that may cause harm to itself or the environments.
A standard framework to handle such cases is the constrained Markov Decision Process (CMDP) \citep{altman1999constrained}.
Within the CMDP framework, the agent has to maximize
the expected cumulative reward while
obeying a finite number of constraints, which are usually in the form of expected cumulative cost criteria.

However, we are sometimes concerned with the problem with a continuum of constraints.
For example,
the constraints we meet might be time-evolving or subject to uncertain parameters, which
cannot be formulated as an ordinary CMDP
(see Examples \ref{Example_Time_Evolving} and  \ref{Example_Uncertain}).
In this paper we would study a generalized CMDP  
to address the above problem.  Because the constraints are not only infinite-number but also lie
in a continuous set,
the generalization is not trivial. Fortunately, we find that we can borrow the idea behind semi-infinite programming (SIP) \citep{remez1934determination, hettich1993semi} to deal with the semi-infinite constraints.
Accordingly, we propose \emph{semi-infinitely constrained Markov decision processes} (SICMDPs)
as a novel complement to the ordinary CMDP framework.
%More specifically,  an SICMDP model %, we consider 
%contains a continuum of constraints whereas an ordinary CMDP contains a finite number of constraints. 

%This generalization is natural but not trivial. However, we can brows the idea  
%The idea is quite natural and can be backtracked
%to the practice of extending linear programming to linear semi-infinite programming (LSIP) %\cite{remez1934determination, GobernaLSIO1998}.
%In addition, 
%As a complementary approach to the ordinary CMDP framework, 
%SICMDP can be used to model these problems  which cannot be described by a finite number of constraints
%that are not covered by .
%For example,
%the restrictions we consider can be time-evolving or subject to uncertain parameters
%, thus
%cannot be described by a finite number of constraints but a continuum of constraints 
%(see Examples \ref{Example_Time_Evolving} and  \ref{Example_Uncertain}).

We also present two reinforcement learning algorithms to solve SICMDPs called SI-CRL and SI-CPO, respectively.
SI-CRL is a model-based reinforcement learning algorithm designed for tabular cases, and SI-CPO is a policy optimization algorithm for non-tabular cases.
% and analyze its performance both theoretically and empirically.
The main challenge is that we need to deal with a continuum of constraints, thus reinforcement learning algorithms for ordinary CMDPs do not work anymore.
In SI-CRL, we tackle this difficulty by first transforming the reinforcement learning problem to an equivalent LSIP problem, which can then be solved using methods in the LSIP literature like the dual exchange methods \citep{Hu1990,reemtsen1998numerical}.
In SI-CPO, we resort to the idea of cooperative stochastic approximation developed in \cite{lan2020algorithms, wei2020comirror}.
As far as we know, we are the first to introduce tools from semi-infinitely programming (SIP) into the reinforcement learning community for solving constrained reinforcement learning problems.

% To the best of our knowledge, we are the first to apply tools from semi-infinitely programming (SIP) to solve reinforcement learning problems.
Furthermore, we give theoretical analysis for both SI-CRL and SI-CPO.
We decompose the error of SI-CRL into two parts: the statistical error from approximating the true SICMDP with an offline dataset and the optimization error due to the fact that the solution of the LSIP problem obtained by the dual exchange method is inexact.
On the optimization side, we show that the iteration complexity of SI-CRL is $O\left(\left\{\mathrm{diam}(Y)L\sqrt{|\gS|^2|\gA|m}/\left[(1-\gamma)\epsilon\right]\right\}^m\right)$.
On the statistical side, we show that the sample complexity of SI-CRL is $\widetilde O\left(\frac{|S|^2|A|^2}{\epsilon^2(1-\gamma)^3}\right)$ if the offline dataset is generated by a generative model, and $\widetilde O\left(\frac{|S||A|}{\nu_{\min} \epsilon^2(1-\gamma)^3}\right)$ if the dataset is generated by a probability measure $\nu$ as considered in \cite{chen2019information}.
Here $\widetilde O$ means that all logarithm terms are discarded.
For SI-CPO, things become a little more complicated because other than the statistical error and the optimization error, we also need to consider the function approximation error, which comes from imperfect policy parametrizations.
It is shown if the function approximation error can be controlled to $O(\epsilon)$ order, the iteration complexity of SI-CPO is $\widetilde{O}\left(\frac{1}{\epsilon^2(1-\gamma)^6}\right)$ and the sample complexity of SI-CPO is $\widetilde{O}(\frac{1}{\epsilon^4(1-\gamma)^{10}})$.
Here our iteration complexity bound is equivalent to a typical $\widetilde O(1/\sqrt{T})$ global convergence rate.

We perform a set of numerical experiments to illustrate the SICMDP model and validate our proposed algorithms.
Specifically, we examine two numerical examples, namely the discharge of sewage and ship route planning.
Through the discharge of sewage example, we show the advantage of the SICMDP framework over the CMDP baseline obtained by naive discretization in modeling realistic sequential decision-making problems.
Moreover, we demonstrate the effectiveness of the SI-CRL and SI-CPO algorithms in such tabular environments. 
In the ship route planning example, we illustrate the benefits of the SICMDP framework and the ability of the SI-CPO algorithm to address complex continuous control tasks involving continuous state spaces with modern deep reinforcement learning techniques.

% In summary, our contributions are listed as follows.
% First, we present the SICMDP model, which can be viewed as a generalization of the ordinary CMDP model.
% Second, we propose an algorithm to perform reinforcement learning for SICMDPs, which is called SI-CRL, and we believe that we are the first to apply tools from SIP
% to solve reinforcement learning problems.
% Third, we give a theoretical analysis of SI-CRL and identify both its sample complexity and iteration complexity.
% In addition, we perform numerical experiments to illustrate the SICMDP model and validate the SI-CRL algorithm.
% \{This paragraph can be removed!!! \}






\section{X-Stream model}\label{sec:model}
\section{The \MakeLowercase{i}W\MakeLowercase{inr}NFL model}
\label{sec:model}

In this section we are going to present the data we used to develop our in-game probability model as well as the design details of {\method}. 

{\bf Data: }In order to perform our analysis we utilize a dataset collected from NFL's Game Center for all the regular season games between the seasons 2009 and 2016. 
We access the data using the Python {\tt nflgame} API \cite{nflgame}. 
The dataset includes detailed play-by-play information for every game that took place during these seasons. 
This information is used to obtain the state of the game that will drive the design of {\method}. 
In total, we collected information for 2,048 regular season games and a total of 338,294 snaps/plays. 

{\bf Model: }
{\method} is based on a logistic regression model that calculates the probability of the home team winning given the current status of the game as: 

\begin{equation}
\Pr(H=1| \mathbf{x})= \frac{\exp(\mathbf{\weight}^T\cdot\mathbf{x})}{1+\exp(\mathbf{\weight}^T\cdot\mathbf{x})}
\label{eq:reg}
\end{equation}
where $H$ is the dependent random variable of our model representing whether the home team wins or not, $\mathbf{x}$ is the vector with the independent variables, while the coefficient vector $\mathbf{\weight}$ includes the weights for each independent variable and is estimated using the corresponding data.  
For a game of infinite duration a linear model could be a very good approximation.  
However, the boundary effects from the finite duration of a game create several non-linearities \cite{winston2012mathletics}.  
For this reason, we enhance our model - using the same set of features - with a Support Vector Machine classifier with radial kernel for the last three minutes of regulation.  
In order to obtain a probability output from the SVM classifier, we further use Platt's scaling \cite{platt1999probabilistic}: 

\begin{equation}
\Pr(H=1| \mathbf{x})= \frac{1}{1+\exp{(Af(x)+B)}}
\label{eq:platt}
\end{equation}
where $f(x)$ is the uncalibrated value produced by the SVM classifier: 

\begin{equation}
f(x) = \sum_{i} (\alpha_i y_i k(\mathbf{x}_i\cdot\mathbf{x}))+ b
\label{eq:svm}
\end{equation}
where $k(\mathbf{x},\mathbf{x}')$ is the kernel used for the SVM.   
Figure \ref{fig:iwinrNFL} depicts the simple flow chart of {\method}. 


\begin{figure}[t]
\begin{center}
\includegraphics[scale=0.35]{plots/iwinrNFL.pdf}%\vspacecap
 \caption{{\method} includes a linear and a non-linear component.}
 \label{fig:iwinrNFL}
\end{center}
\end{figure}

In order to describe the status of the game we use the following variables:

\begin{enumerate}
\item {\bf Ball Possession Team:} This binary feature captures whether the home or the visiting team has the ball possession
\item {\bf Score Differential:} This feature captures the current score differential (home - visiting)
\item {\bf Timeouts Remaining:} This feature is represented by two independent variables - one for the home and one for the away team - and they capture the number of timeouts remaining for each of the teams
%\item {\bf Quarter:} This feature captures the current quarter of the game
%\item {\bf Time Remaining:} This feature captures the time (in seconds) remaining for the current quarter to end
\item {\bf Time Elapsed: } This feature captures the time elapsed since the beginning of the game
\item {\bf Down:} This feature represents the down of the team in possession
\item {\bf Field Position:} This feature captures the distance covered by the team in possession from their own yard line
\item {\bf Yards-to-go:} This variables represents the number of yards needed for a first down
\item {\bf Ball Possession Time: } This variable captures the time that the offensive unit of the home team is on the field 
\item {\bf Ranking Differential: } This variable represents the difference of the win percentage for the two team (home - visiting)
\end{enumerate}

The last independent variable is representative of the power ranking difference between the two teams. 
Most of the existing models that include such a variable are using the Vegas line spread for each game.  
We choose not to do so for the following reason.  
The objective of the Vegas line is not to predict game outcomes but rather distribute money across the different bets.  
Exactly because of this objective the line is changing during the week before the game.  
While this line can change due to new information for the competing teams (e.g., injury updates), the line is mainly changing when a particular team has accumulated the majority of the bets. 
In this case it will also be hard to choose which line to use (e.g., the opening, the closing or some average of them).  
Therefore, we choose to use the win percentage differential of the two teams as an indicator of their strength (even though this has its own issues given the uneven schedule in NFL).  
However, note that if one would like to use the point spread as a variable this can be easily incorporated in the model. 
Table \ref{tab:iwinrnfl} presents the coefficients of the logistic regression model of {\method} with standardized independent variables for better comparisons. 


\begin{table}[ht]
\begin{center}
\def\sym#1{\ifmmode^{#1}\else\(^{#1}\)\fi}
\begin{tabular}{l*{1}{c}}
\toprule
                    &\multicolumn{1}{c}{(1)}\\
                    &\multicolumn{1}{c}{Winner}\\
\midrule
Possession Team (H)         &      0.41\sym{***}\\
                    &     (49.19)         \\
\addlinespace
Score Differential           &      3.59\sym{***}\\
                    &    (247.34)         \\
\addlinespace
Home Timeouts           &     0.12\sym{***}\\
                    &      (8.74)         \\
\addlinespace
Away Timeouts           &     -0.11\sym{***}\\
                    &    (-12.47)         \\
\addlinespace
Ball Possession Time  &     -0.05.\\
                    &    (-1.66)         \\
\addlinespace
Time Lapsed       &   -0.05.\\
                    &      (-1.66)         \\
\addlinespace
Down                &   -0.01         \\
                    &      (0.04)         \\
\addlinespace
Field Position            &   0.02\sym{**} \\
                    &      (2.71)         \\
\addlinespace
Yards-to-go                &  -0.01         \\
                    &      (0.23)         \\
\addlinespace
Rating differential         &       0.75\sym{***}\\
                    &     (80.47)         \\
\addlinespace
Intercept            &       0.57\sym{*}\\
                    &    (2.09)         \\
\midrule
Observations        &      338,294         \\
\bottomrule
\multicolumn{2}{l}{\footnotesize \textit{t} statistics in parentheses}\\
\multicolumn{2}{l}{\footnotesize \sym{$_.$} \(p<0.1\), \sym{*} \(p<0.05\), \sym{**} \(p<0.01\), \sym{***} \(p<0.001\)}\\
\end{tabular}
\end{center}
\caption{Standardized logisitic regression coefficients for {\method}.}
\label{tab:iwinrnfl}
\end{table}


As we can see, as one might have expected the current scoring differential exhibits the strongest correlation with the in-game win probability.  
The only factors that do not appear to be statistically significant predictors of the dependent variable are the down and the yards-to-go. 
Even though the corresponding coefficients are negative as one might have expected (e.g., being at an earlier down gives you more chances to advance the ball), they are not significant in estimating the win probability. 
On the contrary, all else being equal timeouts appear to be quiet important since they can help a team stop the clock, while teams with better win percentage appear to have an advantage as well, since this can be a sign of a better team. 
In the following section we provide a detailed evaluation of {\method}.

\section{Normal mode}\label{sec:normal}
% !TEX root = XStreamFull.tex

During \XSCC normal mode, at any \XStream tick exactly one
processor is in a state of \emph{building}.
This building processor, $p_B$, accepts new edges, maintains connected
components with a \uf data structure, and stores spanning tree edges.
\XSCC normal mode maintains two key invariants, stated below and
illustrated in Figure~\ref{fig:xstream-normal}.
\begin{invariant}
Let $p_B$ be the building processor.  Then $p_i$ is completely full of
spanning tree edges for $0 \le i < B$ at all times, and $p_i$ has no
spanning tree edges for $i > B$. \label{inv:normal-1}
\end{invariant}

When $p_B$ fills with tree edges, a building ``token'' is passed downstream
to $p_B$'s successor, which assumes building responsibilities.  Thus, \XSCC
maintains a spanning forest of the input graph, packed into prefix processors
$\{p_0, \ldots, p_B\}$.
The \XSCC normal mode protocols maintain Invariant~\ref{inv:normal-1} and one other:

\begin{invariant}
Let $p_S$ be the first processor with any empty space.  Then $p_i$ is
completely full of edges for $0 \le i < S$ at all times, and $p_i$ has no
tree or non-tree edges for $i > S$. \label{inv:normal-2}
\end{invariant}

%\Jon{Blue invariant is used for arguing no duplicate storage.  We don't have
%a corresponding lemma yet, but should.}

\setcounter{algorithm}{0}  % -1
\begin{algorithm*}
\caption{This diagnostic routine is helpful for understanding correctness;
it would never be called in practice.  \label{algo:dump} 
This assumes the \WStream convention of choosing a vertex representative to name each 
supernode.
}
\XSCC diagnostic: dump connected components \\
\makealgtitle
\begin{algorithmic}[1]
\LeftComment{Precondition: the input stream has stopped. This never happens during normal operation.}
\LeftComment{Description: Processor $p_T$ emits correct finite stream connected components output.}
\Procedure{DumpComponentLabels}{$p_i$}
\LeftComment{Relabel all \Call{DumpComponentLabels}{} output from upstream}
          \While {\Call{Receive}{$u$, $R_{i-1}(u)$}}\label{lab:startRelabel}
\State        \Call{Emit}{$u$, $R_{i}(u)$}\label{lab:endRelabel}
          \EndWhile
\LeftComment{Now emit each union-find relationship}
          \For {$b_x : \nodeloc(b_x) = p_i$}   \Comment{$\nodeloc$ is the supernode relationship; see Definition~\ref{def:alpha}}\label{lab:startUFdump}
\State        \Call{Emit}{$b_x$, $R_{i}(b_x)$} \label{lab:endUFdump}
          \EndFor
\EndProcedure
\end{algorithmic}
\end{algorithm*}

Invariants ~\ref{inv:normal-1} and \ref{inv:normal-2} are illustrated in
Figure~\ref{fig:xstream-normal}, with sets of spanning tree edges represented
in red, and sets of non-tree edges represented in blue. In normal mode 
operation, single edges arrive at each X-Stream tick and propagate downstream
to the builder, being relabeled along the way. They settle into $p_B$ if 
they are
found to be tree edges, and into $p_S$ otherwise.  The figure
shows the system at \XStream tick $t+4$.  In this notional example, the
edge that arrived in the previous tick has passed through the head processor
$p_H$, but has not yet been resolved as ``tree'' or ``non-tree.''  Edge
$e_{t+2}$ has passed through two processors, and relabeling of the endpoints
has identified it as a non-tree edge. The basic protocol is thus quite simple;
the subtleties of \XSCC normal operation arise in maintaining the invariants. For example, the
builder may need to jettison non-tree edges downstream to make room for new tree edges.
We provide full detail in Section~\ref{sec:pseudocode}.

\subsection{\XSCC normal mode correctness}
We now show that Invariant~\ref{inv:normal-1} and
\XSCC relabeling implies an exact correspondence between
the connectivity structures computed by \XSCC and \DFRns.  
%\Cindy{Maybe delete this next sentence? We don't talk about maintaining the invariants until the pseudocode sections and the discussion of the bulk delection is in the next section.  This is the lead in to what's in this subsection.} 
%After that
%we will detail the mainentance of the invariants and our bulk deletion
%operation.

Algorithm~\ref{algo:dump} is a diagnostic routine intended to test
implementations of \XSCCns. At
\XStream time $t$, a call to this routine streams out the connected
components of the active graph $G_t$ as a stream of (vertex, label)
pairs.  Although we could correctly stream these $|V_t|$ vertex pairs
out even as new edges change the connected components (see
Section~\ref{sec:non-constant} for additional algorithm steps)\Jon{TODO: there
is nothing yet about this in Section~\ref{sec:non-constant}}, this version is more illustrative.
%is strictly for debugging. It's not, really - it's for correctness args too. 
For simplicity we assume that the input stream pauses at time $t$.

% JWB: using ``dump components'' rather than dump-components since LaTeX
% is struggling to handle line breaks with the latter
When head processor $p_0$ receives the ``dump components'' command in a primary slot, it copies the command to the primary slot of the \bundle it will emit. 
Then, in Lines~\ref{lab:startUFdump}-\ref{lab:endUFdump} of Algorithm~\ref{algo:dump}, processor $p_0$ fills the remaining $k-1$ payload slots with relationships from its union-find structure, the way $\DFR$ outputs union-find information into the $B$ streams. Specifically, if $b_x$ is the name of a building block encapsulated by local component $b_y$ in $p_0$ (i.e. $\beta(b_x) = p_0$), then processor $p_0$ outputs a pair $(b_x, b_y = R_i(b_x))$, where $R_i(b_x)$ is the encapsulating supernode, the result of processor $p_0$ relabeling block $b_x$.  Processor $p_0$ then fills the payload slots of subsequent bundles until it has output all of its union-find relationships.

For downstream processor $p_i$ ($i > 0$), the \bundle that has the ``dump components'' command has $(u,R_{i-1}(u))$ pairs in the payload slots. In Lines~\ref{lab:startRelabel}-\ref{lab:endRelabel} of Algorithm~\ref{algo:dump}, processor $p_i$ relabels any block names $u$ that have been encapsulated by a new supernode in $p_i$. Otherwise $R_i(u) = R_{i-1}(u)$. After all relationships from upstream have arrived, there is empty payload for $p_i$ to output its union-find information as described above.

We now argue the correctness of \XSCC based on the correctness of
W-Stream. Let $D_i$ denote the output of processor $p_i$ from this diagnostic.  For the formal arguments, we require the following definitions:

\begin{definition}
During a union operation joining sets with representatives $b_x$ and $b_y$,
the \emph{supernode naming function} is $\eta: {\cal B} \times {\cal B} \rightarrow {\cal B}$ such
that $\eta(b_x,b_y)$ decides
whether $b_x$ or $b_y$ becomes the new set representative.
\end{definition}
For example, we
might choose a supernode naming function $\eta(b_x,b_y) = \min(b_x,b_y)$. This
is the function used in Figure~\ref{fig:wstream-example}.

\begin{definition}
\label{def:alg-with-params}
$\DFRns(s, \eta, A)$ is an implementation of \DFR with
processor union-find capacity $s$ and supernode naming function
$\eta$ run on input stream $A$. 
% can't seem to use macros here; it interferes with line wrapping
%$\XSCCns(s,\eta,A)$ is
XS-CC$(s,\eta,A)$ is
defined similarly for XS-CC, with each processor's union-find
capacity set to $s$.
\end{definition}

A \emph{resolved} edge is one that has been classified as ``tree''
or ``non-tree.''\ Stream edges arrive
in an \emph{unresolved} state. 
In \DFRns, the stream $A_i$ written from pass $i$ contains only those edges that resolve to ``tree'' edges
%\Jon{In previous text we had said ``unresolved'' here, but DFR emits only resolved edges (and drops non-tree edges (i.e., the resolution decision has been made)).}
(they connect supernodes in the current version of the contracted graph).  \DFR deletes as ``non-tree'' any edge that it determines to be contained inside a supernode.  
In contrast, \XSCC must retain all non-duplicate edges,
even after resolution.  In particular, non-tree edges
must be retained in case they are needed to reconnect pieces of the 
graph after bulk deletion. 
\XSCC removes duplicate edges from the stream after updating their 
timestamps.

By Invariant~\ref{inv:normal-1}, all unique non-tree edges (those contracted inside a supernode) are stored at the end of the \XSCC data structure in spare space in the builder, or in processors downstream of the builder. These downstream
processors have no
union-find structure. The following lemma ignores known non-tree edges.

\begin{lemma}
The stream of unresolved edges sent from processor $p_i$ to $p_{i+1}$ in 
$\XSCCns(s,\eta,A_0)$ is exactly $A_i$ from
$\DFRns(s,\eta,A_0)$.
\label{lem:A-stream}
\end{lemma}

\begin{proof}
We prove this lemma by induction.  For the base case, the first pass of \DFR and the first processor of \XSCC receive the same same finite stream of unresolved edges from the outside (logical processor $p_{-1}$), namely the input stream of edges $A_0$. Suppose that the stream of unresolved edges sent from processor $p_{i-1}$ to processor $p_i$ is the same as stream $A_{i-1}$ from \DFR.  We show that the stream of unresolved edges processor $p_i$ sends to $p_{i+1}$ is exactly \DFR stream $A_i$.

Processor $p_i$ of \XSCC and pass $i$ of \DFR begin by computing connected components via union-find.  Every edge that changes connectivity (starts a new component or joins two components) uses one of the $s$ possible union operations for this processor/pass. When they have both done $s$ union operations (their capacity), they have computed identical union-find data structures since they have done the same computations on the same input stream.  At this point, \DFR has not yet emitted any edges and \XSCC has emitted only resolved non-tree edges. Now \DFR processes the remaining edges of $A_{i-1}$, relabeling the endpoints, deleting edges where both endpoints are contained in the same supernode, and emitting the others to stream $A_i$. \XSCC relabels these remaining edges the same way, and emits the same stream of unresolved edges (among resolved non-tree edges).\qed
\end{proof}

Since \XSCC runs on unending streams, there is no ``end of stream'' mark to trigger creation of and processing of an \DFR $B_i$ stream.  However, the ``dump components'' diagnostic creates these $B$ streams.

\begin{lemma}
For $\XSCCns(s,\eta,A_0)$ followed by a call
to $\Call{DumpComponentLabels}{}$, stream $D_{i-1}$ is identical to
stream $B_i$ from $\DFRns(s,\eta,A_0)$.
\label{lemma:B-stream}
\end{lemma}
\begin{proof}
The ``dump components'' command after ingestion of a finite stream serves as an end-of-stream marker for \XSCC. We prove the lemma by induction. For the base case, streams $B_0$ and $D_{-1}$, the component information input to pass $0$ of \DFR and processor $p_0$ of \XSCC respectively are both empty. Suppose that stream $B_i$ from $\DFRns(s,\eta,A_0)$ is the same as stream $D_{i-1}$ from 
%$\XSCCns(s,\eta,A_0)$ followed   macro messes LaTeX up.
XS-CC$(s,\eta,A_0)$ followed
by a call to\\ $\Call{DumpComponentLabels}{}$. We show that stream $B_{i+1}$ is the same stream $D_i$. From the proof of Lemma~\ref{lem:A-stream}, the runs of \XSCC and \DFR compute the same connected components in processor $p_i$ and pass $i+1$ respectively. Because \XSCC and \DFR are using the same supernode naming function and have the same capacity, the union-find data structures (names of representatives and names of set elements) are identical. As described above, processor $p_{i}$ relabels and emits all elements of $D_{i-1}$ the same way that \DFR pass $i+1$ relabels elements stream $B_i$ to stream $B_{i+1}$. Then processor $p_i$ and \DFR pass $i+1$ output the information in their identical union-find structures in identical ways, completing streams $D_i$ and $B_{i+1}$ respectively. \qed
\end{proof}

\iffalse
\begin{lemma}
Suppose the capacity of \WStream and \XStream processors are configured equivalently: they
both can perform the same number of union operations.
The connectivity structure output by W-Stream pass $i$ ($G_{B_i}$) is
exactly the same as that stored in \XStream processors
$\{p_0,\ldots,p_{i-1}\}$ after processing edge stream $G_t$.
\label{lemma:wx-correspondence}
\Jon{change statement to accomplish: relating \XSCC relabeling to \DFR $G_B$ graphs.  Also, use induction from $i$ to $i+1$ to get rid of $i-2$ 
usage.}
\Jon{either here or in a subsequent lemma, argue about any permutation of $G_t$}
\end{lemma}
\begin{proof}
(induction on $i$) \emph{Base case}: When $i=1$, the same union
operations occur in both \DFR and \XSCCns.  W-Stream emits connectivity
structure $G_{B_1}$, but \XSCC retains a spanning forest of that
structure in processor $p_0$.
\emph{Induction}:  Suppose that the
\DFR connectivity structure $G_{B_{i-1}}$ is the same as that stored in
\XStream processors $\{p_0,\ldots,p_{i-2}\}$.  Note that
$R_{i-2}(u) = R_{i-2}(v)$
iff vertices $u$ and $v$ are in the same connected component of $G_{B_{i-1}}$.
During pass $i$ of W-Stream, $s$ union operations occur and \XStream
processor $p_{i-1}$ is the builder.  Each
stream edge $(u,v)$ connects components of $G_{B_{i-1}}$ iff
$R_{i-1}(u) \neq R_{i-1}(v)$. \XStream processor $i-1$ has space to
store $(u,v)$ and by Invariant~\ref{inv:normal-1}, any previously-stored
non-tree edges can be sent downstream to ensure that the processor can
use its entire capacity for connectivity information. Therefore, \XStream 
processors $0,\ldots,i-1$ store the same connectivity structure as $G_{B_i}$. $\qed$
\end{proof}
\fi

\subsection{\XSCC queries}
The most basic \XSCC query is a connectivity query: are nodes $u$ and $v$ in
the same connected component? A query that arrives at \XStream tick $t$ will
be answered with respect to the graph $G_t$.  The query $(u,v)$ enters
the system from the I/O processor and propagates through the processors
just as new stream edges do.  Each processor relabels the endpoints,
and the tail processor returns ``$(u,v)$:yes'' if the labels are the same and ``$(u,v)$:no''
otherwise.  This holds even if  one or both of the endpoints have
never been seen before.
%Theorem~\ref{thm:query-correctness}
% leverages our previous definitions and Lemma~\ref{lemma:wx-correspondence} to
%shows query correctness
% at single-edge granularity.
% Recall the difference between transit edges and settled
% edges.\Jon{if this is the only place the concept is used, move the definition here.
% Otherwise...}
The following theorem shows that connectivity queries are correct at single-edge granularity, and therefore
that \XSCC in normal mode correctly computes the connected components of an
edge stream.

\begin{theorem}
Suppose that the connectivity query $(u,v)$ arrives at the head processor of an X-Stream
system with $P$ processors at X-Stream tick $t$. Then the I/O processor will
receive the boolean query answer at time $t+P$.  The answer will be \emph{True}
iff $u$ was connected to $v$ in $G_t$, the logical graph that existed at time $t$.
\label{thm:query-correctness}
\end{theorem}
\begin{proof}
Recall that $p_B$ is the building processor.
The query answer will be determined by
X-Stream tick $t+B$ at the latest, since, by Invariant~\ref{inv:normal-1},  
$p_B$ is the last
one to store any tree edges and hence, any union-find information.  Thus it is the 
last processor that can change a label. The $(u,v)$ query travels processor-to-processor in a primary basket, just as
the dump-components command does. If there are any transit edges in $G_t$ when the query arrives, they travel in slots of bundles strictly ahead of the query. Thus transit edges will settle into a processor before the query arrives.  Similarly, any edges that arrive after the query travel in bundles strictly behind the query and cannot affect the query relabeling. Thus when the bundle with the query arrives at processor $p_i$, the union-find data structure, and the processor's status as the builder or not, are set exactly according to the graph $G_t$ in the system when the query arrived.

Processing query $(u,v)$ is closely related to processing $\Call{DumpComponentLabels}{}$. Instead of dumping information for every vertex, starting at the point where a vertex is first encapsulated in a supernode, simple queries only consider two vertices. The label for $u$ will only change from $u$ to a supernode label $b_y$ at the processor $p_i$ that first incorporates $u$ into a local component ($p_i = \beta(u)$).  In $\Call{DumpComponentLabels}{}$, processor $p_i$ is the first that outputs any pair $(u, b_y)$, with first component $u$ into the stream $D_i$.  Thus, after the query has passed the building processor, the labels for vertices $u$ and $v$ are identical to their output values, which exit the system at time $t+P$. By Lemma~\ref{lemma:B-stream}, these are the same labels they would have if \DFR is run on graph $G_t$.  Because \DFR is a correct connected components algorithm, vertices $u$ and $v$ will have the same label if an only if they are in the same connected component. \qed
\end{proof}

We call queries that \XSCC answers with latency $p$ \emph{constant} queries.
See section~\ref{sec:non-constant} for examples of non-constant queries.

%\Jon{Need non-duplication theorem:  O(1) space per edge}
The next theorem shows that \XSCC is space-efficient, storing the current graph in asymptotically optimal space.
\begin{theorem}
\label{thm:non-dup}
In normal operation of \XSCCns, each edge is stored in exactly one processor,
requiring $O(1)$ space.
\end{theorem}
\begin{proof}
%The only duplication scenario is when a copy of an edge exists downstream
%a processor that has empty space.
In normal operation, when a new edge $e$ arrives at a processor $p_i$ that
already stores a copy of $e$, processor $p_i$ removes $e$ from the stream and
updates the timestamp of $e$.
Invariant~\ref{inv:normal-1} ensures that incoming tree edge $e$ encounters any
previously-stored copies of itself before it reaches $p_B$, the building
processor, which recognizes it as a tree edge.
Invariant~\ref{inv:normal-2} ensures that incoming non-tree edge $e$
encounters any previously-stored copies of itself before it reaches $p_S$,
the first processor with any empty space. Furthermore, this invariant also
ensures that there are no edges stored downstream of $p_S$.  
\qed
\end{proof}

% CAP: save previous proof for now.
\iffalse
Case (I): there are no transit edges in the system. Then neither
the building processor designation $p_B$ nor the logical graph $G(t)$ change between query
entry time $t$ and time $t+B$ (when the query arrives at the builder).
By Lemma~\ref{lemma:wx-correspondence}, $u$ and $v$ are connected iff
$R_{p_B}(u) = R_{p_B}(v)$.  Further relabeling does not change this result, and the correct
answer will propagate to the I/O processor at tick $t+P$.
Case (II): there are transit edges in the system at X-Stream tick $t$.  We must argue that
any transit edges that affect connectivity have time to settle (a prerequisite for
invoking Lemma~\ref{lemma:wx-correspondence}) before the query arrives at the building processor.
Noting that the building processor might advance as the query propagates, let $B'$ be
the index of the current builder at time the query arrives there.
Since the
storage capacity $s$ always exceeds $P$,  the building processor will be either $p_{B'-1}$
(if the builder designation advanced) or
$p_{B'}$ when the query reaches it.  In either case, Invariant~\ref{inv:normal-1} ensures
that no tree edge is downstream of the current builder. Therefore, any transit edge will arrive at
the builder at time $t' = t + B' - 1 \le t+B-1$.  This is at least one tick before the query
arrives.  At that point, Case (I) applies. $\qed$
\fi



Theorem~\ref{thm:query-correctness} shows that basic connectivity queries are
answered correctly by \XSCCns.  In Section~\ref{sec:non-constant}
we informally discuss three additional types of feasible queries: complex queries such as
finding all vertices not in the giant component of a social network \Cindy{Can we ask for the maximum size of any connected component in a constant query? I think so, since every union-find component has a vertex count.  This gives us the size of the giant component.}\Jon{Last call for adding this.}, vertex-based
queries like finding the degree or neighborhood, and diagnostic
queries regarding system capacity used.  Also, by Invariant~\ref{inv:normal-1},
X-Stream always knows a spanning tree of the streaming graph by construction. This
tree could be checkpointed, for example, if processors share a filesystem. \Cindy{The information in the spanning tree is about the size of the output of dump components.}\Jon{Wasn't sure what to do about this comment.}
\iffalse
\begin{algorithm*}
\caption{Process Connectivity Query on Processor $p_a$}\label{algo:connectq}
Process Connectivity Query on Processor $p_a$\\\makealgtitle
\begin{algorithmic}[1]
\Procedure{ConnectedQuery}{$([v,R_{a-1}(v)],[u,R_{a-1}(u)],answer)$}
\If{answer = false}
	\State $R_a(v) \gets \Call{Find\_Set}{R_{a-1}(v)}$
	\State $R_a(u) \gets \Call{Find\_Set}{R_{a-1}(u)}$
\If{$R_a(v) = R_a(u)$}
\State answer = true
\EndIf
\EndIf
\State \Call{SendDownstream}{QUERY $([v,R_a(v)],[u,R_a(u)], answer)$}\Comment{we don't care about updating label if answer is already true}
\State \Return
\EndProcedure
\end{algorithmic}
\end{algorithm*}






\subsection{Full X-Stream Normal Mode Algorithm}\label{sec:algo:normal}
We next describe how to expand the implementation of W-Stream UCONN in X-Stream so that we will later be able to age correctly and as a result, handle infinite streams. Algorithm~\ref{algo:normaledge1} describes the method of processing an edge on each processor in normal mode, using Algorithm~\ref{algo:unrollwstream} (which simulates W-Stream) as a subroutine. The main difference is that we now keep non-tree edges, which may effect connectivity after aging. As a result, we also have to verify that an edge does not already exist somewhere in the system before it can be stored. This is done with lap counter metadata that is passed around the ring along with the edge.
\begin{algorithm*}
\caption{Process Edge in Normal Mode on Processor $p_a$}\label{algo:normaledge1}
Process Edge in Normal Mode on Processor $p_a$\\\makealgtitle
\begin{algorithmic}[1]
\LeftComment{\Call{Forward}{} and \Call{Store\_Or\_Forward}{} include relabeling \Alex{but not if already non-tree edge?}}
\LeftComment  \Call{Find\_Set}{$v$} returns the representative node in the union find structure for the set containing $v$. The structure is initialized such that \Call{Find\_Set}{$v$} returns NULL for all $v$.

\Procedure{ProcessEdge}{$e=(([u,R_{a-1}(u)],[v,R_{a-1}(v)],t'),(c,p))$}
\If{c=0}
	\If {\Call{ProcessTreeEdge}{$e$,$S_u$,$S_v$}}
	\State  \Return
	\Else
	  	\State $(c,p) \gets (1,a)$ 
		  \State \Call{Forward}{e}
 		 \State \Return
	\EndIf
\EndIf
\If {$(c,p)=(2,a)$}
 \State \Call{Full\_Stop}{} \Comment{system is completely full}
\EndIf

\If{$c = 1$}  
  \If{$p=a$} $(c,p) \gets (2,a)$ 
  \Else
    \If {Timestamps.lookup(e)}  Timestamps.update(e,t')
    \Else \ \Call{Forward}{e}  \Comment{first lap not complete, continue checking for duplicates}
    \EndIf
    \State  \Return
  \EndIf
\EndIf
\LeftComment{process non-tree edge with $c=2$}
\State \Call{Store\_Or\_Forward}{$e$,$t'$}

\EndProcedure

\Procedure{Store\_Or\_Forward}{$e$,$t'$}
\If{Timestamps.lookup(e)} 
  \State Timestamps.update(e)
  \State \Return
\EndIf
\If {HAVE\_SPACE}{\mbox{}}  \Comment{number of stored edges is less than capacity}
  \State Timestamps.insert($e$,$t'$)  \Comment{and remove from basket (we haven't been including that}
  \State \Return
\Else 
  \State \Call{FORWARD}{$e$}
\EndIf
\EndProcedure

\end{algorithmic}
\end{algorithm*}














\subsection{X-Stream Implementing W-Stream Correctness}
We first show that executing Algorithm~\ref{algo:unrollwstream} computes the same intermediate pass streams computed by the algorithm in~\cite{} for solving undirected connectivity (UCON) in the W-Stream model. We will show that the correctness of query answers using Algorithm~\ref{algo:connectq}) follows. Suppose we have an instance of X-Stream with processors with storage space $s$, and the input graph has $n$ vertices. Solving UCON in W-Stream requires $\Omega(n/s)$ passes (\cite{AMP:aggarwal2004streaming} Theorem 2.1). We assume that we have $p$ processors such that UCON can be solved in $p$ passes. Recall from Section~\ref{sec:intro:wstream} that $G_{A_i}$ is the first part of the stream outputted by pass $i$ and $G_{B_i}$ is the second part of the stream.
\begin{observation}\label{obs:ga}
The set of a relabeled edges emitted by processor $i$ in an X-Stream system induce the graph $G_{A_i}$. However, the endpoints use the set labels constructed in the X-Stream algorithm rather than reusing vertex labels as set labels.
\end{observation}


\begin{observation}\label{obs:gb}
The \uf structures upstream from processor $i$ contain the same information as $G_{B_i}$. Thus $G_{B_i}$ could be reconstructed by letting all primitive vertices and local component labels found on $p_0,...,p_{i-1}$ be $V(G_{B_i})$ and then letting the label pairs $(x,R_i(x))$ be the edges of $G_{B_i}$.
\end{observation}

Invariant~\ref{inv:wstreamreqs} holds for the values of $G_{A_i}$ and $G_{B_i}$ since it holds for the algorithm given by~\cite{AMP:aggarwal2004streaming} to solve UCON in W-Stream.

We also observe the property that 
every processor sees input stream elements in order. Thus when a query arrives at a processor, it has seen all edges. We can now show that:
 \begin{lemma}
 Given an input graph $G$, consider an X-Stream system consists of processors with storage capacity $s$ and a sufficient number of processors, $p$, such that the W-Stream UCON algorithm requires at most $p$ passes on $G$. Suppose the X-Stream system executes Algorithm~\ref{algo:unrollwstream} to process $G$ as an input stream. Then
connectivity queries processed by the X-Stream system using Algorithm~\ref{algo:connectq} will be answered correctly. \end{lemma}
\begin{proof}
\Alex{main todo is to show query ordering is ok so correctness follows from invariant like in W-Stream}
Consider processor $p$, the final processor in the ring. $G_{A_p}$ will be null because processor $p$ does not emit any edges during the execution of Algorithm~\ref{algo:unrollwstream}. 
$G_{B_p}$ is a set of stars which fully describes the vertices contained in each component on processor $p$. Thus correctness follows from the correctness of W-Stream.
\end{proof}


\subsection{Normal Mode Correctness}
We first describe the correctness during normal mode. We must argue that a connectivity query ``Are vertices $u$ and $v$ connected?" will always be answered correctly. During normal mode, the following properties hold:

\begin{enumerate}
\item If an edge has been added to a data structure in some processor it is only stored once. All edges in the system are either = \emph{in transit} or \emph{settled} in exactly one processor.   \label{prop:edgeonce}
\item The elements of the \uf structure in each processor are unique, i.e. do not appear in the \uf on any other processor. \Alex{note that the UF structure doesn't contain nodes/elements that are LC's, just BB's}  \label{prop:onelc}
\item The set of all  tree edges over all processors is a spanning forest of the logical graph. \Alex{Need a def of stored graph: settled tree edges are spanning forest of the subgraph of the logical graph $G(t)$ comprised of edges that are settled  at $t$ - define $G_s(t)$?}\label{prop:nontree}
\end{enumerate}

\begin{lemma}\label{lemma:normalprops}
Properties $1-3$ hold up until normal mode ends for the first time. \Alex{i.e., when the first aging starts}
\end{lemma}
\Alex{

\begin{proof}
Property 1:\\
We induct on the sequence of edges received. Initially, no edge is stored so the property holds. Now suppose the property holds at time $t$ and a new edge $e = ([u,b_u],[v,b_v],t)$
arrives. If $e$ is received by a ring processor which holds an edge $e'$ with the same endpoints as $e$, i.e. $e' = ([v,b_v],[u,b_u],t')$, the processor replaces $t'$ with $t$ and removes $e$ from the stream, so it will not be stored as a duplicate. Thus we show if processor $p_s$ accepts edge $e$ as a new edge at time $t$, no such $e'$ with the same endpoints previously existed in the ring.

Case 1: Suppose processor $p_s$ accepts $e$ as a storage edge. Then $e$ must have marker pair $(c,p_i)$ such that $c=2$. If $c=2$, then $e$ was processed by every processor in the ring at least once: processors $p_1,...,p_i$ saw it when $c=0$, processors $p_{i+1},...,p_i$ (mod ring size) saw it  when $c= 1$, and now processors $p_{i+1},...,p_s$ (mod ring size) saw it  when $c=2$. If the same $(u,v)$ had previously been stored on any processor, $e$ would have been consumed by that processor and the timestamp updated to $t$.

Case 2: Suppose processor $p_s$ accepts $e$ as a tree edge. Then since no $p_i$ with $i<s$ removed $e$ from the stream, no such $p_i$ has an edge $e'$ with the same endpoints. Consider if some processor $p_i$ with $i>s$ had such an $e'$. Since $p_s$ is filling, $e'$ would have to be a storage edge on $p_i$ for some $p_j$, $j\leq s$. But then $e$ would be relabeled as storage on $p_j$ as well, and not reach $p_s$ as a tree edge. Thus no $p_i$, $i\neq s$ has and $e'$ with the same ends, and $p_s$ can accept $e$ if $p_s$ itself does not have an edge $e'$.
\\\\Property 2: We induct on the processor index, showing processor $p_i$'s \uf sets contain no elements used in the \uf sets on $p_0,...,p_{i-1}$. Clearly this holds for $p_0$. Now suppose  the union-find structures in $p_0,...p_i$ contain disjoint element sets and we show $p_{i+1}$ will not reuse any of these elements. When an edge is stored by a processor, its relabeled ends are uses as elements in the union-find structure. Consider the labels on any edge $e=([u,R_i(u)],[v,R_i(v)],t)$ sent from $p_i$ to $p_{i+1}$. If  $R_i(u) = R_i(v)$ it is a storage edge and nothing is added to the union find structure. If $R_i(u) =u$ then no processor $p_0,...,p_i$ has any edges containing $u$, or the primitive BB corresponding to $u$ in their union-find structures. In any other case, $R_i(u)$ is set name on some processor $j \in \{0,...,i\}$.

\end{proof}
}
\begin{proof}
Property 1: We induct on the sequence of edges received. Initially, no edge is stored so the property holds. Now suppose the property holds at time $t$ and a new edge $e = ([v,b_v],[u,b_u],t)$
arrives. If $e$ is received by a ring processor which holds an edge $e'$ with the same endpoints as $e$, i.e. $e' = ([v,b_v],[u,b_u],t')$, the processor replaces $t'$ with $t$ and removes $e$ from the stream, so it will not be stored as a duplicate. Thus we show if processor $p_s$ accepts edge $e$ as a new edge at time $t$, no such $e'$ with the same endpoints previously existed in the ring.

Case 1: Suppose processor $p_s$ accepts $e$ as a storage edge for processor $p_j$. Then when $e$ reached $p_j$ on its first lap around the ring, it would have been marked as storage and had $first\_lap$ set to true. Then $p_s$ would not be accepting $e$ until after $e$ had completed a lap from the head to the tail and had $first\_lap$ set to false. Thus no $e'$ with the same endpoints existed, or a $e$ would not have completed that whole lap.

Case 2: Suppose processor $p_s$ accepts $e$ as a tree edge. Then since no $p_i$ with $i<s$ removed $e$ from the stream, no such $p_i$ has an edge $e'$ with the same endpoints. Consider if some processor $p_i$ with $i>s$ had such an $e'$. Since $p_s$ is filling, $e'$ would have to be a storage edge on $p_i$ for some $p_j$, $j\leq s$. But then $e$ would be relabeled as storage on $p_j$ as well, and not reach $p_s$ as a tree edge. Thus no $p_i$, $i\neq s$ has and $e'$ with the same ends, and $p_s$ can accept $e$ if $p_s$ itself does not have an edge $e'$.

Case 3: Suppose processor $p_s$ accepts $e$ as mortgage.
\end{proof}


%\begin{proof}
%Property 1: We induct on the sequence of edges received. Initially, no edge is stored so the property holds. Now suppose the property holds at time $t$ and a new edge $e = ([v,b_v],[u,b_u],t)$ arrives.

%Case 1: Suppose no other edge $e' = ([v,b_v],[u,b_u],t')$ exists in the system, so $e$ is the first instance of an edge with ends $u$ and $v$.

%Case 2: Next suppose $e$ already exists in the system. When a processor receives an edge it already holds, it removes it from the stream and updates the time stamp, so the edge cannot be stored a second time on different processor. Thus we show the new instance of $e$ always reaches the processor already holding $e$:

%Case 2a: Suppose $e$ is a tree or mortgage edge on some porcessor $p_s$. Let $p_f$ be the filling processor, so $f\geq s$. Then each processor $p_i$, for $i<s$, is sealed and will not take $e$ as a tree or mortgage edge. Also, relabeling ends of $e$ will give $R_i(v) \neq R_i(u)$, or the previous instance of $e$ would have also been relabeled with $R_i(v) =R_i(u)$, making $e$ a storage edge and not a tree or mortgage edge for $p_s$. Thus any new copy of edge $e$ will reach $p_s$.

%Case 2b: Suppose $e$ is a storage edge for some processor $p_s$, held by some processor $p_j$, $j\neq s$. As before, if $p_f$ is filling, $f\geq s$. Since each $p_i$ with $i<s$ is sealed it cannot accept $e$ as a tree or mortgage edge. Then if $j<s$, $e$ reaches $p_j$. Otherwise, $e$ reaches $p_s$, at which point it is labeled as storage with $R_s(u) = R_s(v)$ and on its first lap. Then $e$ cannot be taken by a processor other than $p_j$ before it reaches the tail to finish the lap, so $e$ will reach $p_j$.

%Properties $2-4$: omitted for space.
%\end{proof}

\begin{theorem}\label{theorem:propstoquery}
If properties $1-4$ hold for $G_t$, then there exists a processor $p$ that gives vertices $u$ and $v$ the same label at time $t$ if and only if the vertices are in the same connected component in $G_t$. Equivalently, $R_m(v) = R_m(u)$ if and only if there exists a $u,v$-path in $G_t$.
\end{theorem}
\begin{proof}
Let $E_t^i$ be the set of tree edges on processors $p_1,...,p_i$ at time $t$. Let $G_t^i = (V_t,E_t^i)$. Then $G_t^i$ is subgraph of $G_t^{i+1}$ for $i = 0,...m-1$. By Property~\ref{prop:nontree}, $G_t^m$ is a spanning forest of $G_t$. By Property~\ref{prop:oneprimitive}, the primitive building block $b_u \in B_v$ for vertex $u$ is unique so only one possible relabeling $R_i(u)$ corresponds to vertex $u$. We show by induction that there exists a $u,v$-path in $G_t^i$ if and only if $R_i(u) = R_i(v)$. At $p_1$, all building blocks are primtive, so $R_1(u) = R_1(v)$ iff there is a $u,v$-path in $G_t^1$.
Now suppose at $p_i$, $R_i(u) = R_i(v)$ iff there exists a $u,v$-path in $G_t^i$. We show this implies the same for $p_{i+1}$:

Case 1: Suppose $R_i(u) =R_i(v)$. Then $R_{i+1}(u)= R_{i+1}(v)$ and since $G_t^i$ is a subgraph of $G_t^{i+1}$, the $u,v$-path in $G_t^i$ is also in $G_t^{i+1}$.

Case 2: Suppose $R_i(u)\neq R_i(v)$. Then $R_{i+1}(u) = R_{i+1}(v)$ iff $R_i(u)$ and $R_i(v)$ are BBs in the same LC on $p_{i+1}$, by Property~\ref{prop:onelc}. By definition of LC, $R_i(u)$ and $R_i(v)$ are in the same LC iff there exists a sequence of edges $(x_1,y_1),(x_2,y_2),...(x_l,y_l)$ on $p_{i+1}$ with $x_j,y_{j-1}$ in the same BB for $j=2,...,l$ and $x_1 \in R_i(u)$ and $y_l \in R_i(v)$. By induction there exists a $x_j,y_{j-1}$-path in $G_t^{i+1}$ for $j=2,...,l$, so there exists a $x_1,y_l$-path in $G_t^{i+1}$. By induction we also have a $u,x_1$-path and a $y_l,v$-path in $G_t^{i+1}$ so all together, there is a $u,v$-path. Thus $R_i(u)$ and $R_i(v)$ are in the same LC on $p_{i+1}$ ($R_{i+1}(u) = R_{i+1}(v)$) iff there exists a $u,v$-path in $G_t^{i+1}$.

Since there exists a $u,v$-path in $G_t^i$ if and only if $R_i(u) = R_i(v)$ for $i = 1,...,m$ we have shown $R_m(u)=R_m(v)$ iff there is a $u,v$-path in $G_t$.
\end{proof}
\begin{theorem}\label{theorem:queriesnormal}
During normal mode, there exists a processor $p$ that gives vertices $u$ and $v$ the same label at time $t$ if and only if the vertices are in the same connected component in $G_t$.
\end{theorem}
\begin{proof}
Properties $1-4$ hold during normal mode, so by Theorem~\ref{theorem:propstoquery} queries are answered correctly.
\end{proof}
 An edge will remain in a outside slot for at most two laps around the ring 

From this we, we can also guarantee that the input stream will not be starved, as long as there are at least two outside slots per \bundlens. At the head processor, begin putting an edge into the first outside slot of each \bundle that is passed. If an edge is in the first outside slot of a bundle received from the tail, but that edge in the second outside slot of the \bundlens. On its second lap, an edge will be accepted by a processor, so no edges will be passed from the tail to head in the second outside slot.
\fi

\iffalse  OLD

If a new edge arriving at the filling processor connects two separate components, it is a \emph{spanning tree edge}, since the set of all such edges a processor has received forms a spanning forest of the identified components. For example, in Figure~\ref{fig:BBLCexample} edge $(v_0,v_1)$ on $p_0$ and $(v_0,v_4)$ on $p_1$ are tree edges. When storing a tree edge $e$, processor $p_i$ updates the disjoint-set structure to connect the building blocks named $R_{i-1}(v)$ and $R_{i-1}(u)$ in some local component $b_x$. Processor $p_i$ also stores the fact that $v$ is encapsulated by $R_{i-1}(v)$ and $u$ is encapsulated by $R_{i-1}(u)$; $u$ and $v$ are then \emph{known vertices} on $p_i$. 

If a node $p_i$ receives an edge $([u,R_{i-1}(u)],[v,R_{i-1}(v)])$ and relabels it such that $R_i(u) = R_i(v)$, then $(u,v)$ is a \emph{non-tree edge}. While these edges are not initially necessary for maintaining connected components, they may be necessary for determining correct components after aging, and thus must be kept some where in the system, not necessarily on $p_i$. Non-tree edges held on the same processor as the component containing them are called \emph{mortgage edges}. Non-tree edges stored on a different processor are called \emph{storage edges}.

If $p_i$ is the filling processor, it either has available space or contains some storage edges for other processors. If $p_i$ receives a non-tree edge contained in one of its own local components, it prefers that mortgage edge over any storage edges it may have. Thus if it does not have room for the mortgage edge, it will \emph{jettison}, or pass downstream, a storage edge to make room for the mortgage edge. The filling processor also prefers new tree edges over storage edges, and will jettison a storage edge if a new tree edge is received and there is not available space. When the processor is full of only tree and mortgage edges, it becomes sealed and $p_{i+1}$ becomes the filling processor. 

If a sealed processor $p_i$ receives an non-tree edge $e$ contained in one of its local components, it will take $e$ as a  mortgage edge if it has space from a previous aging. A sealed processor $p_i$ will also jettison a storage edge to make room for a mortgage edge. If $p_i$ does not have space or a storage edge to jettison, $e$ becomes a storage edge for $p_i$, and $e$ is passed downstream.
For example, in Figure~\ref{fig:BBLCexample}, edge $(v_1,v_2)$ on $p_2$ is a storage edge for $p_0$. The only edges arriving to non-filling processors downstream of the filling processor, which are not sealed, are storage edges. These storage edges will be accepted by an unsealed non-filling processor if it has room; otherwise they will be passed downstream to other non-filling processors. Storage edges may also be passed around the ring past the head, and accepted by a sealed processor with available space that opened up during aging.

\fi


\section{Aging mode}\label{sec:aging}
\begin{figure*}[bht]
\begin{center}
\begin{tabular}{c}
\includegraphics[width=5in]{aging1.png} \\
(a) \\
\includegraphics[width=5in]{aging2.png} \\
(b) 
\end{tabular}
\end{center}
\caption{\label{fig:xstream-aging} \XSCC aging mode. (a) A token notifies processors
that they must apply an aging predicate.  All surviving edges become unresolved (gray).
(b) Normal operation continues uninterrupted for new edges, while
unresolved edges circulate back to be incorporated into a new data structure. Each processor
in turn becomes the loading processor ($p_L$) and recycles its unresolved edges.}
\end{figure*}

\begin{figure*}[thb]
\begin{center}
\includegraphics[width=5in]{aging_properties.png} \\
\end{center}
\caption{\label{fig:xstream-aging-props} \XSCC aging nomenclature. both
primary
and payload edges are called \emph{resolved} when they have been
classified as tree or non-tree. Duplicate detection leaves empty
slots, and processors \emph{ingest} and \emph{emit} bundles of edges.}
\Jon{$p_H$ label, fix last ingest/emit example (my Office is currently broken;
will fix when Office fixed.}
\end{figure*}

\XSCC handles infinite streams via a bulk deletion operation we call
an \emph{aging} event.
Our model is thus unlike most previous work, in that we do not expect or
support individual
edge deletions embedded within the stream.  Rather, we expect the 
system administrator
to schedule bulk deletions to ensure that the oldest and/or 
least useful data are deleted in a timely manner.

To begin aging, the system administrator introduces an aging 
predicate (for example, a timestamp threshold) into the input stream.
The predicate
propagates through the system, and each processor suspends query 
processing upon receipt.  However, a new stream edge might arrive in the 
\XStream tick immediately after the aging predicate arrives from the 
I/O processor.
This and all other new edges must be ingested and processed without exception.  
Thus, the connectivity data structures must be rebuilt concurrently with
normal stream edge processing. When this rebuild is complete, queries
are accepted once again.

We now describe how \XSCC processes the aging predicate and prove correctness.
In Section~\ref{sec:aging-conditions} we provide theoretical guarantees relating 
the fraction of
system capacity used after the deletion predicate has been applied, the 
bandwidth expansion
factor, the proportion of query downtime that is tolerable, and the 
expected stream edge duplication rate.

\subsection{Aging process}
\label{sec:aging-process}
Figure~\ref{fig:xstream-aging} illustrates the aging process. An aging token arrives
with an edge-deletion predicate.  As the token propagates downstream, all
edges are reclassified to be \emph{untested}. If an edge later passes the
aging predicate it becomes \emph{unresolved} since the old connectivity
structure is no longer valid.
Immediately after the aging token is received
by the head processor, new stream edges may continue to arrive. These are
processed as normal, starting from empty data structures, so we maintain 
Invariants~\ref{inv:normal-1} and \ref{inv:normal-2} even during aging.

%\Jon{ATTN Cindy:  this paragraph is new text attempting to explain a subtle 
%point; please vet.}
Conceptually, upon receipt of aging notification the deletion of all
edges that fail the aging predicate and
reclassification of all surviving edges to \emph{unresolved} is instantaneous.
However, in
practice each processor takes $\frac{s}{k-1}$ \XStream ticks to execute
a ``testing phase'' that applies the aging predicate to each stored
edge.  Without careful attention to detail,
implementers could allow a case in which there is no space yet for a new
stream edge.  In Section~\ref{sec:pseudocode} we give exact
specifications for a correct procedure that ensures no stream edge is
dropped, even in the \XStream tick immediately after aging notification.
If the testing phase has yet not identified empty space for a new stream
edge, then one of the unresolved edges can be sent downstream
in a primary slot.  This is an example of the jeopardy condition described
later in this section, corresponding to Line 21 in Algorithm~\ref{algo:pseudocode-driver}.

In addition to normal processing of new stream edges, \XSCC recycles all
unresolved edges that survive the aging predicate.
As depicted in Figure~\ref{fig:xstream-aging}, we introduce a new designation $p_L$
for a \emph{loading processor} or ``loader.''  Upon each activation to process
a stream edge, the loader packs unresolved edges into any available payload
slots in the output \bundlens.  Such bundles propagate around the ring. After
a bundle
reaches the head processor $p_H$, its payload edges are processed as if they
were new edges.  When the loader has emitted all of its unresolved edges, it
passes the loader token downstream to its successor.  Aging is complete when
the last processor with any unresolved edges has completed its loader duties.

\begin{figure*}[thb]
\begin{center}
\includegraphics[width=5in]{jeopardy.png} \\
\end{center}
\caption{\label{fig:jeopardy} The \XSCC aging ``jeopardy
condition.'' Processor $p_B$ currently bears both building and loading
responsibilities, is completely full of edges, and must ingest a \bundle
with no empty slots. It ingests $k$ slots, finds no duplicates,
and must emit $k$ slots. Therefore an unresolved ``jeopardy edge''
must be emitted in the primary slot.  If it doesn't settle in a 
processor before leaving the tail, the system is completely full and raises a
FAIL condition. Note that in this illustration, $p_S$ will be able to store
the jeopardy edge, so the jeopardy condition will soon be mitigated.}
\end{figure*}

The complete \XSCC protocols defined in Section~\ref{sec:pseudocode} 
enforce the previous invariants at all times, as well as the following
invariant during aging.  
% CAP: Not sure this is adding much at this point.  Put it back later (probably changed) if it makes sense
%It formalizes the idea that the loader never leaves behind any unresolved edges.
% CAP: here is the old invariant: During aging, let $p_L$ be the loading processor. Then $p_i$ is completely full of resolved edges for $0 \le i < L$ at all times, and there are no resolved edges for $i > L$.
% \Cindy{I don't think the first part is true.  Processors are only packed up to $p_S$, and the loader can be downstream of $p_S$.  Fix this.  Maybe no unresolved edges upstream}
\begin{invariant}
During aging, let $p_L$ be the loading processor and $p_B$ be the building processor. 
Then $B \le L$.  Also, processor $p_i$ has no unresolved edges for $i < L$
% Then $p_i$ is completely full of resolved edges for $0 \le i < L$ at all times,
and $p_i$ has no resolved edges for $i > L$. \label{inv:aging-1}
\end{invariant}

The combination of all invariants ensures that all processors from the head to the builder are running \XSCC in normal mode on all incoming (and recycled) edges. All resolved edges are packed to the front (upstream).  When all edges have been recycled and aging ends, the layout of edges returns to normal mode.

%\Jon{(updated by CAP) This begs a statement that the processors between 0 and B are still doing XS normal mode (on the incoming edges union the payload edges.}

%\Jon{This next invariant is wrong by our current pseudocode.  Don't think
%it was used in the first place.}
%Our protocols also enforce the following relationship between
%$p_B$, $p_L$, and $p_S$.  We must never build connectivity structure
%downstream of the loader.
%\begin{invariant}
%During aging, let $p_B$ be the building processor, $p_L$ be the loading
%processor, and $p_S$ be the first processor with empty space.  Then
%%$B \le L \le S$. \label{inv:aging-2}
%\end{invariant}


Figure~\ref{fig:xstream-aging-props} puts the
nomenclature of our arguments into context. An edge becomes \emph{resolved}
when an XS-CC processor determines that it is a tree or non-tree edge,
regardless of whether it is a new stream edge in a primary slot or an
unresolved edge being recycled as payload.  Processors \emph{ingest} and
\emph{emit} bundles of edges.  With one exception we will discuss presently,
the complexity of processing input bundles and packing edges into
output bundles prior to emission is relegated to Section~\ref{sec:pseudocode}.

Aging is generally a straightforward process in which the loader token
steadily advances from $p_H$ to $p_T$, unresolved edges are recycled and
resolved, and the XS-CC connectivity structure is rebuilt.  
When builder and loader designations coincide in the same processor, 
that processor packs unresolved edges for emission first, then non-tree 
edges.  Edge bundles containing transit edges have one primary slot and $k-1$ 
payload slots, where
$k$ is the bandwidth expansion factor.  New stream edges reside in primary
slots, and unresolved edges circulate in payload slots until they are
resolved.  Payload edges continue in their assigned slots until allowed to 
settle, per the invariants.

There is a single exception to this last point, illustrated in
Figure~\ref{fig:jeopardy}. We call this the \emph{jeopardy condition}
and use it to specify exactly when the system fills
to capacity during aging (indicating that the aging command was too late
or did not remove enough edges).
In the jeopardy condition, processor $p_B$ is also the loader,
is already storing edges to its capacity $s$, and must ingest an edge \bundle
with no empty slots. It ingests $k$ slots, finds no duplicates,
and by conservation of space, must emit $k$ slots. 
Therefore an unresolved edge $e_j$ must reside in the primary slot.
If $e_j$ cannot be offloaded before exiting the tail,
the system is completely full and raises a FAIL condition.

The above discussion and the more detailed discussion in Section~\ref{sec:pseudocode} show the following property necessary for proving aging correctness holds:
\begin{property}
\label{prop:all-recycled}
During aging, every surviving edge is incorporated into the new 
connected components data structure either by $p_H$ directly or 
by traveling back to $p_H$ as a payload edge.
\end{property}

\subsection{Aging correctness}
We now argue correctness of the aging process.
We say that any implementation of \XSCC aging that maintains
Invariants~\ref{inv:normal-1}, \ref{inv:normal-2} and \ref{inv:aging-1} and property~\ref{prop:all-recycled} is \emph{compliant}.
A compliant aging process ensures that during aging there is a monotonic ordering of
edges in the system, with tree (red) edges never allowed downstream of
non-tree (blue) edges, and unresolved (gray) edges never allowed upstream
of non-tree edges. In the argument below, we slightly abuse notation by
using the graph $G_t$ in place of its edge set $E(G_t)$.

\begin{theorem}
Suppose a compliant XS-CC implementation receives an aging command at tick $t$ and reauthorizes queries at tick $t'$.  Let $F$ be the set of edges in $G_t$ that fail the aging predicate and let $E_{t\rightarrow t'}$ be the set of edges that arrive between time $t$ and $t'$. Then at tick $t'$, the x-stream system stores graph $G_{t'} = G_t - F \cup E_{t\rightarrow t'}$, can properly answer queries and stores each edge in $G_{t'}$ exactly once.
\label{thm:aging-correctness}
\end{theorem}
\begin{proof}
\Cindy{consistency in graph vs. edge set}\Jon{mitigated by text above. Cindy?}
As the aging command that arrived at time $t$ propagates through the
processors, they reclassify all current edges to ``untested'' as
described in Section~\ref{sec:aging-process}, forgetting the current
union-find structure. Thus the system starts processing a new graph
from an empty state at time $t+1$. As described in
Section~\ref{sec:aging-process}, processors delete all edges in $F$,
those who fail the predicate.  Each remaining edge in $G_t - F$ is
eventually loaded into payload slot by
Property~\ref{prop:all-recycled}, and processed at the head as
arriving edges. Invariants~\ref{inv:normal-1} and~\ref{inv:normal-2}
hold with the newly-created data structures thoughout aging.
Invariant~\ref{inv:aging-1} ensures that all unresolved edges are in
the builder processor or downstream. Those in the builder do not
affect the connectivity computation and are eventually moved
downstream. Thus, all edges arriving from outside the system are
processed as in normal mode and all edges arriving in the payload
slots are processed as in normal mode (other than traveling in a
payload slot).  Thus at time $t'$ when the tail processor passes the
loading token out of the system and enables queries, the \XStream
system stores exactly the edges in $G_t - F \cup E_{t\rightarrow t'}$,
with duplicates appropriately removed.  This is the graph the system
is required to hold by definition of aging and the requirement that it
drop no incoming edges during aging. The edges are processed into the
data structures with arbitrary mixing of new edges and recycled
(surviving) edges.  By Observation~\ref{obs:wstreamreqs}, and the equivalence of
\DFR and \XSCC in normal mode, the ordering of
the input edges does not matter for future query correctness.
By Theorem~\ref{thm:query-correctness}, the \XStream system will now correctly
answer querries on the graph starting at time $t'$.

During aging, some edges may be stored up to twice.  If a duplicate of a suriving edge $e$
enters the system before edge $e$ circulates back to the head processor, then edge $e$ is
stored both in the new data structure as a tree or non-tree edge and as an unresolved edge.
However, when edge $e$ is eventually recycled, it will be recognized as a duplicate and not
stored again.  By Theorem~\ref{thm:non-dup}, any edge that enters from outside the system during aging will be stored at most once in the new data structure.
\qed
\end{proof}

\iffalse
\Cindy{OLD:}
Since Invariants~\ref{inv:normal-1} and \ref{inv:normal-2} still hold
during aging, we can define a corollary to Lemma~\ref{lemma:B-stream}
to argue about the partially rebuilt
connectivity structure.

Suppose that a compliant aging process begins at \XStream tick $t$, 
and consider tick $t'>t$ during
the aging process. Let $E'$ be the set of stream edges that have arrived during
$\{t+1,\ldots,t'\}$ and $U$ be the set of unresolved edges that remain. We call
\[G^r_{t'} = G_t - U \cup E'\] the \emph{resolved graph at time $t'$}.  This
contains all active edges that either entered the system since time $t$ or
become resolved after circulating as payload.
\Jon{account for edges that got deleted by the predicate.}
\Jon{define this differently: the key is the set of payload edges that have
entered the head}

\begin{corollary}
\label{corollary:partial}
(corollary to Lemma~\ref{lemma:wx-correspondence}) The equivalence
between \DFR and \XSCC expressed in Lemma~\ref{lemma:wx-correspondence}
holds for $G^r_{t'}$, the resolved graph at aging time $t'$.
\end{corollary}
\begin{proof}
The argument is exactly the same as that of Lemma~\ref{lemma:wx-correspondence},
replacing $G_t$ with $G^r_{t'}$.
\end{proof}

Thus, as aging proceeds, the connected component information
stored by \XSCC regarding the resolved graph
aligns with that W-Stream would have computed had it been given $G^r_{t'}$
as a stream.  All that remains for the correctness argument
is to define the conditions under which
a compliant aging process completes. We address this in Section~\ref{sec:aging-conditions}. If aging does complete at some time $t''$,
then $G_{t''} = G^r_{t''}$ and Corollary~\ref{corollary:partial} implies
that the system can leave its aging mode and resume query processing.
\fi

\section{Conditions for successful aging}
\label{sec:aging-conditions}
In this section, we define the conditions under which
a compliant aging process completes before the system fails for lack of space.
We consider properties of the system, properties of the input stream, and user
preferences.


\begin{tcolorbox}
\begin{definition}
\ \ \\
We define the following as tradeoff parameters associated with infinite runs of \XSCCns.
\label{def:infinite-run-params}
\begin{description}
\item[{\bf c:}] fraction of the total system storage occupied by
edges that survive the aging predicate
\item[{\bf d:}] percentage of \XStream ticks that the system is unavailable for queries due to aging
\item[{\bf u:}] estimate of the percentage of incoming stream edges that will be unique
\item[{\bf k:}] the bandwidth expansion factor: the size of an \XStream bundle (a set of edge-sized slots that circulates in the ring)
\item[{\bf p:}] number of \XStream processors
\item[{\bf S:}] aggregate storage available in the system
\item[{\bf s:}] storage per processor in a homogeneous system ($s = S/p$)
\end{description}
\end{definition}
\end{tcolorbox}

%\Jon{New lemma: must age before you're full.  Precondition: must be at least
%$P+1$ empty slots in the whole \XStream system at the time you invoke aging.}
Aging must be initiated before the system becomes too full,
or else jeopardy edges will lead to a FAIL condition.  We quantify this
 decision point as follows.

\begin{lemma} \label{lemma:aging-lead-time}
In the worst case, there must be at least 
$$\frac{cS}{p(k-1)} + \frac{3}{2}p$$
open space in the system when an aging command is issued to be guaranteed sufficient space for aging, where
$c$, $S$, $p$ and $k$ are given in Definition~\ref{def:infinite-run-params}.  
\end{lemma}
\begin{proof}
When the aging command arrives, there could be $p$ edges in transit
that all must be stored. Because iteration over the untested list
doesn't imply
any specific ordering, in the worst case, when processors test the
edges against the predicate, all $cS$ surviving edges are tested
before an edge fails the predicate. This gives the latest time when
space becomes free for new edges. When a processor receives the aging
command, it processes $(k-1)$ untested edges each tick until it has
tested all its edges. In the $p$ ticks required for the aging command
to reach the tail, the head tests $p(k-1)$ edges, the second processor
tests $(p-1)(k-1)$ edges and so on, while the tail tests $(k-1)$
edges.  Thus in the first $p$ ticks after aging starts, the system tests
$\frac{p(1+p)(k-1)}{2}$ edges.  After that, the system tests $p(k-1)$
edges per tick. If the system tested $p(k-1)$, every tick, it would
require  $\frac{cS}{p(k-1)}$ ticks. But the first $p$ ticks are only half
as efficient, so we require an extra $p/2$ ticks. Thus the total
number of ticks before the system is guaranteed to remove an edge that
fails the predicate is at most $\frac{cS}{p(k-1)} + \frac{3}{2}p$.
\qed
\end{proof}

If the system is homogeneous, the empty space expression in Lemma~\ref{lemma:aging-lead-time} becomes $cs/(k-1) + \frac{3}{2}p$. For example, for a homogeneous system, assuming that $s \gg p$, if $c = 1/2$ and $k = 5$, then one should start aging while $1/8$ of the last processor is still empty. The last processor can issue a warning when it starts to fill and again closer to the deadline given $c$ and $k$.

\begin{theorem} \label{thm:infinite-runs}
In any \XSCC aging process initiated in accordance with Lemma~\ref{lemma:aging-lead-time}, if
$c$, $d$, $u$, $k$, $p$, $S$, and $s$ from Definition~\ref{def:infinite-run-params}
are set such that 
             \[k \ge   1 + \frac{(cp + 1) u}{dp (1-c)},\]
then the aging process will finish before the system storage fills completely.
\end{theorem}

\begin{proof}
After the aging token arrives, the head processor must apply the aging predicate
to its $s$ edges.  It processes $k-1$ per tick, as described in Section~\ref{sec:aging-process}.
Thus, after $\frac{s}{k-1}$ ticks, the head processor passes the loader token to the
second processor. By that time, all other processors have applied the predicate to
all of their edges and have a list of surviving edges.
Once unresolved edges begin circulating from
the loader (ignoring additive latencies such as the time until the first
payload reaches the head processor since $P \ll s$), $k-1$ edges re-enter the system to be
resolved at each
tick.  Since $cS$ unresolved edges survived the aging predicate, in the worst case
(when they are all in the second processor or later) it will take
$\frac{cS}{k-1}$ ticks to complete aging. During this time, every $\frac{1}{u}$
ticks yields a new, non-duplicate stream edge. Thus, the system will fill
to capacity in $\frac{1}{u} (1-c)S$ ticks.
The proportion $d$ constrains these two tick counts as follows:
\[ d \frac{1}{u} (1-c) S \ge \frac{cS + s}{k-1} = (cS / (k-1)) * (p+1)/p.\]
Simplifying this inequality and solving for $k$ (with Wolfram Alpha~\cite{wa},
for example)
yields the result. \qed
\end{proof}

The parameters $c$ and $d$ are user preferences, but $k$ is dictated by 
computer architecture.  Reasonable values of $k$ for current architectures are
$5-10$, but emerging data flow architectures may provide upward flexibility.
The parameter $u$ must be estimated by the user based on knowledge of the
input streams that s/he will feed to \XSCCns.


We can now state the central result of this paper. 
%\Cindy{TODO: wrangle this into usable shape}
%We assume that the graph-edge stream is {\em effectively infinite}. This means that as long as the algorithm is running, it must always be prepared for the arrival of another edge. At any time, the system has seen only a finite set of edges and need store only a finite graph representation.  However this graph can be arbitrarily large and may eventually exceed any particular finite storage. 
%\Jon{I think less is more in this case.}

\begin{theorem}
\XSCC can process an effectively infinite stream of graph edges without failing,
answering
connectivity queries correctly when in normal mode, as long as
the system is configured in accordance with Theorem~\ref{thm:infinite-runs}
and aging is started with sufficient space available obeying
Lemma~\ref{lemma:aging-lead-time}.
\end{theorem}
\begin{proof}
Assuming that the proportion of \XStream ticks that yield a new,
non-duplicate stream edge is $u$,
an empty system will fill and fail in $\frac{1}{u}S$ ticks. Compliant
aging in accordance with Lemma~\ref{lemma:aging-lead-time} ensures that
aging will always complete before the system fills.  During normal
mode operation, Lemma~\ref{lemma:B-stream} and 
Theorem~\ref{thm:query-correctness} ensure, respectively, that 
accurate connected
component information is stored, and that connectivity queries are
answered correctly.  As long as the
system adminstrator adheres to such a schedule, \XSCC operation
can continue through an arbitrary number of aging events. \qed
\end{proof}
We note that queries yielding system capacity usage are TODO constant-size. In
the case of a simple aging predicate such as a timestamp threshold,
given a target proportion $c$ of edges that survive an aging
event, the \XStream system administrator could use an automated process to
trigger the aging process.


\section{X-Stream edge processing specification}\label{sec:pseudocode}
\setcounter{algorithm}{0}
\begin{algorithm*}
\caption{This is the driver function for an X-Stream implementation that is
compliant with Invariants~\ref{inv:normal-1}, \ref{inv:normal-2}, and
\ref{inv:aging-1}, and Property~\ref{prop:all-recycled}.
         \label{algo:pseudocode-driver}}
         X-Stream connected components driver \\
\makealgtitle
\begin{algorithmic}[1]

\Procedure{ProcessBundle}{\Call{PrimaryEdge}{e}, \Call{PayloadEdges}{$e_i$}}
%\LeftComment{Precondition: input ``raft'' containing $\Call{Primary}{e}, \Call{Payload}{e_i}$, where payload exists only during aging}
\State  \Call{PackingSpaceAvailable}{} $= k$  \Comment{ the output buffer has size $k$ and is initially empty}
    \If {\Call{EmptyEdge}{$e$}}
\State  \Call{Pack}{EmptyEdge}  \Comment{any call to \Call{Pack}{} decrements \Call{PackingSpaceAvailable}{}}
    \Else \ \ \ \Call{ProcessEdge}{$e$}
    \EndIf
    \For {$e_i \in \Call{PayloadEdges}{}$} \Comment{there are payload edges only during aging or non-constant query processing}
\State  \Call{ProcessEdge}{$e_i$}
    \EndFor

\If {\Call{Not}{AGING}}
\State \Call{Emit}{PackedBundle}
\State \Return
\EndIf
\LeftComment{Aging-related logic}
    \If {\Call{Loader}{}}
        \If {\Call{Head}{}}  \Comment{the Head's testing \& resolution phase}
            \For {$i = 1, k-1$} 
\State          $e'$ = {\Call{PopEdge}{UNTESTED}} 
                \If {\Call{EmptyEdge}{$e'$}}
\State              \Call{Pack}{LoaderToken}
\State              \Call{Break}{}
                \EndIf
                \If {\Call{AgingPredicatePassed}{$e'$}}
                    \If {$i = k-1$ \mbox{and} \Call{Full}{}} 
\State                  \Call{Pack}{$e', \Call{Primary}{}$} \Comment{jeopardy edge}
                    \Else
\State                  \Call{ProcessEdge}{$e'$} \Comment{\Call{Head}{} immediately resolves surviving edge}
                    \EndIf
                \Else
\State              \Call{Delete}{$e'$}
                \EndIf
            \EndFor
        \Else   \Comment{ Downstream resolution phase (all testing has finished)}
            \For {$i \in 0, \ldots, \Call{PackingSpaceAvailable}{} - 1$}
\State          $e'$ = \Call{PopEdge}{UNRESOLVED}
                \If {\Call{NULL}{$e'$}}
\State              \Call{Pack}{LoaderToken}
\State              \Call{Break}{}
                \Else
\State              \Call{Pack}{e'}
                \EndIf
            \EndFor
        \EndIf
    \Else  \Comment{ Downstream testing phase}
        \If {\Call{Not}{HEAD} \Call{And}{} \Call{NotEmpty}{UNTESTED}}
            \For {$i = 1, k-1$} 
\State          $e'$ = {\Call{PopEdge}{UNTESTED}}
                \If {\Call{AgingPredicatePassed}{$e'$}}
\State              \Call{PushEdge}{$e'$, UNRESOLVED} 
                \Else
\State              \Call{Delete}{$e'$}
                \EndIf
            \EndFor
        \EndIf
    \EndIf
\State \Call{Emit}{PackedBundle}
\EndProcedure
\end{algorithmic}
\end{algorithm*}


Algorithms~\ref{algo:pseudocode-driver} and \ref{algo:pseudocode-functions}
show the \XSCC driver and constituent functions, respectively, for 
processing edges. We do not show full detail for token passes, commands,
and queries.
These functions maintain the invariants and produce a compliant XS-CC
% macro foils LaTeX line breaking
implementation.  We used this pseudocode as guidance for the code
that produces our experimental results.

Each \XStream processor executes \Call{ProcessBundle}{} whenever it receives
the next bundle of edge slots, regardless of its current execution mode
(normal or aging).  It will process each slot in turn, and the 
constituent functions \Call{ProcessEdge}{}, \Call{ProcessPotentialTreeEdge}{},
and \Call{StoreOrForward}{} determine what to pack into an
output bundle destined to flow downstream.

Note that the top-level logic of processing the primary and payload
edges of a bundle is the same in Algorithm~\ref{algo:pseudocode-driver}, 
regardless of execution mode.  When a new
edge arrives from the stream, processors upstream of (and including) the 
building processor will classify it as tree or non-tree using the
relabeling logic of Section~\ref{sec:relabeling} (Lines 15-18 of 
\Call{ProcessEdge}{} and Lines 2-4 of \Call{ProcessPotentialTreeEdge}{}).
The builder stores any new tree edge. We ensure that this is
possible via logic to jettison an unresolved edge if one exists 
(only during aging; Lines 9 and 16 of \Call{StoreOrForward}{}), or else 
to jettison a non-tree edge (Line 15
of \Call{StoreOrForward}{}).  This progression of jettison logic maintains
Invariants~\ref{inv:normal-1} and \ref{inv:normal-2}.

Suppose that the head processor $p_H$ receives notification of an aging event
at \XStream tick $t$.  \XStream ticks $t$ and $t+1$ are especially interesting. 
If a new edge arrives in the input stream at $t+1$, it must be stored 
in $p_H$ (which
is now acting as both the builder $p_B$ and the loader $p_L$) in
order to maintain Invariant~\ref{inv:aging-1}.  However, $p_H$ has had only
one tick to initiate the process of testing its edges against the aging
predicate.  That means that it tested $k-1$ edges in tick $t$.  Suppose all
of these edges survived the predicate and therefore couldn't be deleted.
This is a jeopardy condition, and it was handled during tick $t$ by 
Lines 20-21 of 
\Call{ProcessBundle}{}.  Favoring the new edge, $p_H$ jettisoned in the
primary slot of its output bundle the last of the $k-1$ unresolved edges
it created in that tick.  Therefore, at tick $t+1$ we are assured that $p_H$ 
can store a new stream edge.

During aging, the loader $p_L$ packs unresolved edges into the empty payload
slots in incoming bundles to be sent around the ring.  When these edges
arrive at $p_H$, they are processed as if they were new stream edges,
classified as tree or non-tree, and incorporated into the data structures
in $\{p_H,\ldots,p_B\}$ by the same invariant-maintaining constituent 
functions that handle new edges.  One optimization we include is that
$p_H$ need not actually pack and send its unresolved edges around the
ring.  Rather, in Lines 13-23 of \Call{PackBundle}{}, $p_H$ simply tests
against the aging predicate and immediately processes its tested edges
rather than calling them unresolved.  

As aging proceeds, the \Call{Loader}{} token is passed downstream whenever
a processor exhausts its list of unresolved edges (Lines 28-31 of 
\Call{ProcessBundle}{}). Once the \Call{Loader}{} token exits the tail
processor, Property~\ref{prop:all-recycled} is established.


\begin{algorithm*}
\caption{These three constituent functions comprise the 
X-Stream algorithm for maintaining connected components.\label{algo:pseudocode-functions}}
         X-Stream constituent functions \\
\makealgtitle
\begin{algorithmic}[1]
\LeftComment{Processor $p_a$ receives an edge}
\Procedure{ProcessEdge}{$e=(u,v,R_{a-1}(u),R_{a-1}(v))$}
          \If {\Call{Duplicate}{e}} \Comment{regardless of $p_a$'s position in the chain, duplicate edges don't propagate downstream}
\State        \Call{SetNewestTimestamp}{e}  \Comment{During aging, either $e$ or its stored duplicate could be the newest}
              \If {\Call{Primary}{e}}
\State            \Call{Pack}{EmptyEdge} \Comment{bundles drive \XS ticks, \Call{ProcessBundle}{} requires a primary edge}
              \EndIf
\State        \Call{Return}{}
          \EndIf
          \If {\Call{DownstreamOfBuilder}{}} \Comment{$p_B \prec p_a$ : $p_a$ stores only non-tree and/or unresolved edges}
              \If {\Call{Primary}{e}} \Comment{need to store this edge if we can in order to ensure Invariant~\ref{inv:normal-2}}
\State            \Call{StoreOrForward}{e}        \Comment{\Call{StoreOrForward}{} accepts $e$ or packs it for output}
              \Else \Comment{\Call{Payload}{e}, i.e., aging}
\State            \Call{Pack}{e}         \Comment{processors downstream of the Builder simply propagate payload edges}
              \EndIf
\State        \Call{Return}{}
          \EndIf
\LeftComment{Processor $p_a$ contains connected component information, i.e., $p_a \prec = p_B$} 
          \If {$R_{a-1}(u) = R_{a-1}(v)$}   \Comment{previously-discovered non-tree edge}
\State        \Call{StoreOrForward}{e} 
          \ElsIf {\Call{Not}{\Call{ProcessPotentialTreeEdge}{e}}}\Comment{newly-discovered non-tree edge}
\State        \Call{StoreOrForward}{e}
          \EndIf
\EndProcedure
\end{algorithmic}

\begin{algorithmic}[1]
\Procedure{ProcessPotentialTreeEdge}{e=(u,v,\ldots)}
\State   $(R_a(u),R_a(v)) = $ \Call{Relabel}{e}
         \If {$R_a(u) = R_a(v)$}  \Comment{newly-discovered non-tree edge}
\State        \Call{Return}{FALSE}
         \EndIf
         \If {\Call{Builder}{}}      \Comment{builder $p_B$ must ingest tree edge $e$}
\State        \Call{Assert}{\Call{StoreOrForward}{e} = STORE} \Comment{ensure Invariant~\ref{inv:normal-1}}
              \If {\Call{FullOfTreeEdges}{}}
\State             \Call{Pack}{BuilderToken} \Comment{can be encoded with a bit; doesn't take a whole slot}
              \EndIf
         \Else \ \ \ \Call{Pack}{e}  \Comment{$p_a$ has previously sealed, so it is already full of tree edges}
         \EndIf
\State   \Call{Return}{TRUE}\Comment{still a potential tree edge; downstream processors will determine that}
\EndProcedure
\end{algorithmic}

\begin{algorithmic}[1]
\Procedure{StoreOrForward}{e=(u,v,\ldots)}
\LeftComment{Precondition: if $e$ is a tree edge, this processor is not full of TREE edges}
         \If {\Call{Full}{}}
             \If {\Call{Tail}{}}
\State            \Call{Fail}{} \Comment{the system is totally full}
             \EndIf
             \If {\Call{Unresolved}{$e$}}
\State                 \Call{Pack}{$e$}
\State                 \Call{Return}{FORWARD}
             \EndIf
\State       $e_p = $ \Call{PopEdge}{UNRESOLVED} \Comment{jettison an unresolved edge to keep a resolved one, if possible}
             \If {\Call{EmptyEdge}{$e_p$}} \Comment{no more edges to resolve}
                  \If {\Call{NonTree}{$e$}}
\State                 \Call{Pack}{$e$}  \Comment{no need to jettison a non-tree edge if $e$ is non-tree}
\State                 \Call{Return}{FORWARD}
                  \Else \Comment{by precondition, there must be a non-tree edge to jettison}
\State                 \Call{Pack}{\Call{PopEdge}{NONTREE}} \Comment{jettison a non-tree edge to keep a tree edge}
                  \EndIf
             \Else \ \ \ \Call{Pack}{$e_p$}
             \EndIf
         \EndIf
         \If{\Call{Primary}{e}} \Call{Pack}{\Call{EmptyEdge}{}}  \Comment{every output raft needs a primary edge}
         \EndIf
\State   \Call{Accept}{e}  \Comment{perform UNION/FIND if $\Call{Tree}{e}$}
\State   \Call{Return}{STORE}
\EndProcedure
\end{algorithmic}
\end{algorithm*}


\section{Related work}\label{sec:related-work}
The industry standard for pose edition is to create rigs, a collection of pieces of software designed to manipulate a character's skeleton. The rig describes the skeleton's bones, how they relate to each other, are constrained in their possible motion and are deformed. These rules are loosely specified and creating a good rig requires a detailed understanding of physics and anatomy, as well as technical and artistic skills. Rigging is thus a time consuming task even for experienced animators, and even more so in large scale productions which often require a different in-depth rig for each character in the cast.
Previous work has helped alleviate this difficulty by providing efficient tools to speed up/and or ease the rigging process, relying on inverse kinematics or data-driven methods.
\subsection{Character pose design}
\subsubsection{Inverse Kinematics (IK)}
IK solvers are a family of methods commonly used in robotics, engineering and computer graphics, in which the parameterization of a kinematic chain is determined from the position of its end effector.
They are a staple tool in pose design software, ensuring the respect of elementary constraints during pose edition. Their de-facto role is to guarantee the length of the limbs, and in some cases to enforce the orientation angle range of a joint.
Many IK solutions have been studied over the years \cite{aristidou_inverse_2018}; usually revolving around approximated linearizations or heuristics. 

Numerical methods require a set of iterations to achieve a satisfactory solution formulated by a cost function to be minimized.
IK solutions can generally be divided into three sub-categories: Jacobian \cite{Siciliano_Handbook_Robot_2007}, Newtonians \cite{cohen_ik_1996} and Heuristics. Most software implement heuristic methods such as Cyclic Coordinate Descent (CCD) \cite{wang_ccd_1991} or 
Forward-Backward Reaching IK (FABRIK) \cite{aristidou_fabrik:_2011} due to their simplicity and extensibility. 

The main drawback of 
these solvers is that they manipulate kinematic chains without taking into account many morphological aspects that make a pose more or less plausible. They offer a first level of help to users but are not sufficient to guarantee a realistic pose. Many joints constraints are dependent on each other and require subjective, human-made approximations.

\subsubsection{Data-driven pose edition}
Data-driven methods offer promising opportunities to solve these approximations. Using real-life data can help in modelling the complex inter-dependencies of skeletons and providing users with smarter edition tools.
While it is still an early field of research, some solutions have been studied. Wu \etal \cite{wu_posing_2009} propose a method for natural character posing from a large motion database. It employs adaptive KD-clustering to select a representative frame from a database and sparse approximations to accelerate training and posing. 
Huang \etal in \cite{Huang_IK_MGDM_2017} present a method based on the formulation of multi-variate Gaussian distribution models (MGDMs), which learn the joint constraints of a kinematic skeleton from motion capture data. 

Some work has also been dedicated to finding new editing interfaces. \modify{}{Instead of the usual setup manipulating joints directly, Guay \etal \cite{guay_line_2013} articulate a framework based on the conceptual "line of action" which describes the overall pose dynamics. They provide a mathematical definition of the line of action, and a interface in which the software modifies the pose to follow a user-provided line. In the same line of though} Garcia \etal \cite{garcia_sketching_2019} propose \modify{a method transforming doodle of trajectories (position and orientation over time) }{a virtual reality-based interface where the user's hands motion (position and orientation over time) are transformed} into sequences of actions and then into detailed character animations using a dataset of parametrized motion clips automatically fitted to the trajectory. 

% ==> DL et Latent Space. 
\subsection{Neural modelling of human motion}
Neural networks have received a great amount of attention over the last decade and shown impressive result in modelling complex data. Human motion has not been spared and deep learning methods have proven their capability of generating realistic motion in a number of difficult cases. 

The literature in neural-based animation include example in user-controlled character navigation \cite{Holden2017} and interactions with the environment \cite{starke_neural_2019}. 
Holden \etal \cite{Holden2020} also show that neural networks can be used to replace parts of existing data-driven methods, improving their scalability potential.
More recently, some work has also focused on improving smaller parts of the animation pipeline rather than replacing it completely. Berson et al. \cite{berson_intuitive_2020} leverage neural networks to provide an interactive system to edit facial animation. 

% Wrap up
Data-driven IK and pose editing can relieve animators from time-consuming, back-and-forth pose adjustments by applying constraints extracted from real-world data. Recently, neural-network-based approaches have demonstrated their ability to model the intricacies of human motion while scaling to large amount of data and retaining a fast inference time. In this paper we seek to take advantage of these properties to create an efficient posing tool, intuitively usable even by a inexperienced user.

\section{Experiments} \label{sec:experiments}

\section{Experiments}\label{sec:experiments}
We validate our approach using multiple datasets containing real-life data from the fields of criminal risk assessment, credit, lending, and college admissions. In each of the datasets we select a binary feature and treat it as the protected attribute (e.g., race or gender), which is the feature we require our trained classifier to behave fairly upon. Our proposed method performs well on all of these datasets, succeeding in removing unfairness almost entirely, at a very modest price in terms of accuracy.


\begin{table*}[h]
\centering
\resizebox{\textwidth}{!}{
\def\arraystretch{1.2}

\begin{tabular}{c c c | c | c | c || c | c | c || c | c | c |}

\cline{4-12}
&&&
\multicolumn{9}{ c| }{\textbf{COMPAS Dataset}}
\\ \cline{4-12}
&&&
\multicolumn{3}{ c|| }{\textbf{FPR Considerations}}&
\multicolumn{3}{ c|| }{\textbf{FNR Considerations}}&
\multicolumn{3}{ c| }{\textbf{Both Considerations}}
\\ \cline{4-12}
&&&
 $\mathbf{Acc.}$ &  $\mathbf{D_{FPR}}$ &  $\mathbf{D_{FNR}}$ &  $\mathbf{Acc.}$ &  $\mathbf{D_{FPR}}$ &  $\mathbf{D_{FNR}}$ &  $\mathbf{Acc.}$ &  $\mathbf{D_{FPR}}$ &  $\mathbf{D_{FNR}}$
\\  \cline{4-12}
\vspace*{-0.5ex}
\\ \cline{1-2} \cline{4-12}
\multicolumn{1}{ |c  }{} &
\multicolumn{1}{ c|  }{  \textbf{Our Method (AVD Penalizers)}}  &&
$\mathbf{0.660}$    &  $\mathbf{0.01}$  &  $0.04$ &
$\mathbf{0.653}$    &  $0.02$   &  $\mathbf{0.04}$ &
$\mathbf{0.654}$    &  $\mathbf{0.02}$  &  $\mathbf{0.04}$
\\ \cline{1-2} \cline{4-12}
\multicolumn{1}{ |c  }{} &
\multicolumn{1}{ c|  }{  \textbf{Our Method (SD Penalizers)}}  &&
$\mathbf{0.664}$    &  $\mathbf{0.02}$  &  $0.09$ &
$\mathbf{0.661}$    &  $0.05$   &  $\mathbf{0.03}$ &
$\mathbf{0.661}$    &  $\mathbf{0.02}$  &  $\mathbf{0.03}$
\\ \cline{1-2} \cline{4-12}
\multicolumn{1}{ |c  }{} &
\multicolumn{1}{ c|  }{  Zafar et al.~(\citeyear{disparatemistreatment})}  &&
$0.660$    &   $0.06$    &   $0.14$  &
$0.662$    &   $0.03$    &   $0.10$  &
$0.661$    &   $0.03$    &   $0.11$
\\ \cline{1-2} \cline{4-12}
\multicolumn{1}{ |c  }{} &
\multicolumn{1}{ c|  }{  Zafar et al. Baseline~(\citeyear{disparatemistreatment})}  &&
$0.643$    &   $0.03$    &   $0.11$  &
$0.660$    &   $0.00$    &   $0.07$  &
$0.660$    &   $0.01$    &   $0.09$
\\ \cline{1-2} \cline{4-12}
\multicolumn{1}{ |c  }{} &
\multicolumn{1}{ c|  }{  Hardt et al.~(\citeyear{hardt})}  &&
$0.659$    &  $0.02$    &   $0.08$  &
$0.653$    &  $0.06$   &    $0.01$  &
$0.645$    &  $0.01$   &    $0.01$
\\ \cline{1-2} \cline{4-12}
\multicolumn{1}{ |c  }{} &
\multicolumn{1}{ c|  }{  \textbf{Vanilla Regularized Logistic Regression}}  &&
$\mathbf{0.672}$    &   $\mathbf{0.20}$    &   $\mathbf{0.30}$  &
$\mathbf{0.672}$    &   $\mathbf{0.20}$    &   $\mathbf{0.30}$  &
$\mathbf{0.672}$    &   $\mathbf{0.20}$    &   $\mathbf{0.30}$
\\ \cline{1-2} \cline{4-12}
\end{tabular}
}
\vspace{3mm}
\caption{Performance comparison on the COMPAS dataset. For the approaches in bold -- Accuracy, FPR difference and FNR difference are evaluated on the test set, averaging over five runs and using a 70-30 training/test split. The performance of the remaining three approaches is stated as reported in Zafar et al.~(\citeyear{disparatemistreatment}).} \label{table:comparison_results}
\end{table*}



\begin{figure*}[b]
  \includegraphics[scale=0.6]{compas0-400.png}
  \caption{COMPAS Dataset. Accuracy, FPR difference ($\mathbf{D_{FPR}}$), and FNR difference ($\mathbf{D_{FNR}}$) (all evaluated on the test set) of the learned classifier, as a function of the weight $c=c_1 = c_2 \geq 0$ placed on the fairness penalizer terms. On the left we use the Absolute Value Difference (AVD) penalizer, and the Squared Difference (SD) penalizer on the right, both as presented in Section~\ref{regularization}. ``Relaxed FPR/FNR Diff.'' plots the value of the relevant penalization term.} %In this particular run, parameters chosen for the absolute value relaxation were: $c=80, q_c=60$, and for the squared relaxation: $c=220, q_c=30$.}
  \label{fig:compas}
\end{figure*}


\subsection{Implementation}
\textbf{Our method} 
%We instantiate our method in the following way: Given dataset $Q$, we split it randomly into a training set $S$ (which we will use for learning) and a test set $T$ (which we will only use for reporting performance). 
For the purpose of comparison with  Zafar et al.~(\citeyear{disparatemistreatment}) and Hardt et al.~\cite{hardt} on the COMPAS data, we use a parameter $c$ to induce three possible combinations of weights on the FPR and FNR penalization terms: $c = c_1$ and $c_2 = 0$; $c_1 = 0$ and $c = c_2$; and $c = c_1 = c_2$. For the other three datasets, we consider only $c = c_1 = c_2$.\footnote{The reason for varying the values of $c$ in the training phase is since we shifted to a proxy problem, in which we rely on the distance from the decision boundary rather the actual classifications. 
%Our hope is that there is no need for a worst-case cross validation between all of the combinations of $c_1, c_2, c_3$, and that the training scheme we propose is sufficient. 
It is possible, of course, that even better results are attainable using our scheme with other combinations of $c_1, c_2$, and $q$.} To explore the accuracy/fairness trade-off curve for the relaxed optimization problem~(\ref{eq:2}), we train for different values of $c$, starting at $c=0$ (which is just standard logistic regression), and growing gradually.



Given a dataset $Q$ and fixing a $d_1, d_2 \in \{0, 1\}$ of interest, we use the following training scheme:
\begin{enumerate}
\item Split $Q$ at random into training set $S$ and test set $T$.
\item For each $c$, perform cross-validation on $S$ to select the corresponding best value $q_c$ for the regularization parameter.
\item For each $(c,q_c)$, let $\theta_c = \argmin\limits_{\theta} \text{Proxy}(\theta;S,c,c,q_c)$.
\item Select $\theta^* \in \argmin\limits_{\theta_c} \text{Objective}(\theta_c;S,d_1,d_2)$.
\item Evaluate performance using $\theta^*$ on test set $T$.
\end{enumerate}
We report the average of five such runs, each with a fresh training-test split.




%We instantiate our method by solving the relaxed optimization problem~(\ref{eq:2}), in place of the original, non-convex problem~(\ref{eq:1}).  
%We test our approach with three different combinations of weights on the penalization terms:
%\katrina{What are the $d$, and how are they related to the $c$s?}
%\begin{enumerate}
%\item FPR considerations only: $d_1 = 1, d_2 = 0$.
%\item FNR considerations only: $d_1 = 0, d_2 = 1$.
%\item Both FPR, FNR considerations, assigned similar significance: $d_1 = 1, d_2 = 1$.
%\end{enumerate}
%One could, of course, pick any other combination of the FPR and FNR penalty weights.

%\katrina{I don't understand how the below is distinct from the list above}
%Learning is done by training the parameters of a logistic regressor to solve~\ref{eq:2}, while picking the value of $c_1, %c_2$ as the following:
%\begin{enumerate}
%\item FPR considerations only: $c_1 = c \geq 0$, $c_2 = 0$.
%\item FNR considerations only: $c_1 = 0$, $c_2 = c \geq 0$.
%\item Both FPR, FNR considerations, assigned similar significance: $c_1 = c_2 = c \geq 0$
%\end{enumerate}



% We then cross-validate to pick the best $c_3$ (the weight on the standard $\ell_2$-regularization term) given $c$.\footnote{The reason for varying the values of $c$ in the training phase is since we shifted to a proxy problem, in which we rely on the distance from the decision boundary rather the actual classifications. 
%Our hope is that there is no need for a worst-case cross validation between all of the combinations of $c_1, c_2, c_3$, and that the training scheme we propose is sufficient. 
%It is possible, of course, that even better results are attainable using our scheme with other combinations of $c_1, c_2, c_3$.} For each such combination, we report results as the averages of multiple \katrina{how many?} different runs, each time splitting data randomly into training and test sets.
%\yahav{We need to shorten this description.}

We solve the relaxed convex optimization problem using the CVXPY solver. Due to stability issues with large training sets, we use a train/test split of 30-70 on the larger datasets, rather than 70-30 as on the COMPAS dataset\footnote{The code implementing our method can be found at https://github.com/jjgold012/lab-project-fairness}.

%
%
%We then report the results (as evaluated on the test set) attained by a regressor $\theta \in \mathbb{R}^d$ that minimizes (on the training set $S$) a weighted combination of the $0$-$1$ loss and the differences in FPR and FNR across populations:
%\begin{equation*}
%\begin{aligned}
%&\underset{\theta}{\text{argmin}}
%& & L_{S}^{0\text{-}1}(\theta) \\
%&&& + d_1|FPR_{A=0}(\theta;S)-FPR_{A=1}(\theta;S)| \\
%&&& + d_2|FNR_{A=0}(\theta;S)-FNR_{A=1}(\theta;S)|
%\end{aligned}
%\end{equation*}
%
%\katrina{What is $d_1$ vs. $c_1$ etc.?}



%For classification, we decided use a standard cut-off threshold of $c=0.5$. There are of course, further possible interactions between the FPR, FNR considerations, and picking a certain cut-off level. These are not straightforward, since  these interactions are data-specific. 



%allows for flexibility in picking the values of $c_1, c_2$, which reflect the significance we wish to place on the objectives of achieving accuracy, equal FPR, and equal FNR. As for $c_3$, we will want to find the value of it that achieves the best results, for any combined objective of accuracy and fairness defined by a specific selection of $c_1,c_2$. Therefore, given a specific selection of $c_1, c_2$, we apply cross-validation to select the value of $c_3$. 




We briefly describe the other algorithmic approaches to which we compare:\\
\textbf{Zafar et al.}~(\citeyear{disparatemistreatment}) performs optimization by considering a proxy for the bias: the covariance between the samples' sensitive attributes and the signed distance between the feature vectors of misclassified users and the classifier decision boundary.\\
\textbf{Zafar et al. Baseline}~(\citeyear{disparatemistreatment}) tries to enforce equal FP/FN rates on the different groups by introducing different penalties for misclassified data points with different sensitive attribute values during the training phase.\\
\textbf{Hardt et al.}~(\citeyear{hardt}) performs post-processing on a standard trained (unfair) logistic regressor, picking different decision thresholds for different groups, and possibly adding randomization.


\subsection{Experimental Results}

In what follows, we use the following notation, given a trained classifier $\hat{Y}$:
\begin{align*}
\mathbf{D_{FPR}}&=\left|FPR_{A=0}(\hat{Y})-FPR_{A=1}(\hat{Y})\right| \\ 
\mathbf{D_{FNR}}&=\left|FNR_{A=0}(\hat{Y})-FNR_{A=1}(\hat{Y})\right|
\end{align*}
The values $FPR_{A=0}(\hat{Y})$, $FPR_{A=1}(\hat{Y})$, $FNR_{A=0}(\hat{Y})$, $FNR_{A=1}(\hat{Y})$ are reported as evaluated on the test set.

\paragraph{The COMPAS Dataset\footnote{https://github.com/propublica/compas-analysis}} The Correctional Offender Management Profiling for Alternative Sanctions (COMPAS) records from Broward County, Florida 2013-2014, made available online by ProPublica, are perhaps the best-studied data in the context of fairness.  The goal in this scenario is to successfully predict recidivism within two years, based on features such as age, gender, race, number of prior offenses, and charge degree. The dataset contains 5,278 samples. The protected attribute in this scenario is race, where $A$ indicates black or white. We filtered the dataset using the same features as Zafar et al.~(\citeyear{disparatemistreatment}), to allow for comparison.

%\begin{table}[h]
%\centering
%\begin{tabularx}{\columnwidth}{c|c|c|c}
%\hline
%  &  Recid. ($y = 1$)        & No Recid.  ($y = 0$)       & Total \\ \hline
%Black &  $ 1661   $ & $ 1514 $ &  $ 3175 $ \\ \hline
%White &  $ 822   $  & $1281  $ &  $ 2103 $ \\ \hline
%Total &  $ 2483  $  & $2795 $ &  $ 5278 $ \\\hline
%\end{tabularx}
%\caption{Statistics of the ProPublica COMPAS data.} \label{table:compas-stats}
%\label{tab:stats}
%\end{table}
%\vspace{-1em}

%\begin{table}[h]
%\centering
%\begin{tabularx}{\columnwidth}{c|c}
%\hline
%Feature  &  Description \\ \hline
%Age Category &  $<25$, between $25$ and $45$, $>45$ \\
%Gender &  Male or Female \\
%Race &  White or Black \\
%Priors Count &  0--37 \\
%Charge Degree &  Misconduct or Felony \\
%\hline
%2-year-recid. & Whether or not the  \\
%(target feature)  & defendant recidivated within two years
%\end{tabularx}
%\caption{Description of features used from ProPublica COMPAS data.} \label{table:compas-features}
%\label{tab:features}
%\end{table}




\begin{table*}[t]
\centering
\caption{A description of the datasets used, along with parameters of the training procedure used for each.}
\label{table:datasets_description}
\begin{adjustbox}{max width=\textwidth}
\begin{tabular}{|l|l|l|l|l|l|l|l|}
\hline
\textbf{Dataset} & \textbf{No. Samples} & \textbf{No. Features} & \textbf{Train/Test Split} & \textbf{No. Repetitions} & \textbf{No. Folds in CV} & \textbf{Protected Feature} & \textbf{Target Variable} \\ \hline
COMPAS           & 5,278                     & 5                          & 70-30                     & 5                        & 5                                 & Race                       & 2-Year-Recidivism        \\ \hline
Adult            & 30,162                    & 10                         & 30-70                     & 5                        & 5                                 & Gender                     & Income Over/Under 50K    \\ \hline
Default          & 30,000                    & 23                         & 30-70                     & 5                        & 3                                 & Gender                     & Defaulting On Payments   \\ \hline
Admissions       & 20,839                    & 17                         & 30-70                     & 5                        & 3                                 & Race                       & Passing Bar Exam         \\ \hline
\end{tabular}
\end{adjustbox}
\end{table*}


\begin{table*}[t]
\centering
\resizebox{\textwidth}{!}{
\def\arraystretch{1.2}

\begin{tabular}{c c c | c | c | c || c | c | c || c | c | c |}

\cline{4-12}
&&&
\multicolumn{3}{ c|| }{\textbf{Adult Dataset}}&
\multicolumn{3}{ c|| }{\textbf{Default Dataset}}&
\multicolumn{3}{ c| }{\textbf{Admissions Dataset}}
\\ \cline{4-12}
%&&&
%\multicolumn{3}{ c|| }{\textbf{Both Considerations}}&
%\multicolumn{3}{ c|| }{\textbf{Both Considerations}}&
%\multicolumn{3}{ c| }{\textbf{Both Considerations}}
%\\ \cline{4-12}
&&&
 $\mathbf{Acc.}$ &  $\mathbf{D_{FPR}}$ &  $\mathbf{D_{FNR}}$ &  $\mathbf{Acc.}$ &  $\mathbf{D_{FPR}}$ &  $\mathbf{D_{FNR}}$ &  $\mathbf{Acc.}$ &  $\mathbf{D_{FPR}}$ &  $\mathbf{D_{FNR}}$
\\  \cline{4-12}
\vspace*{-0.5ex}
\\ \cline{1-2} \cline{4-12}
\multicolumn{1}{ |c  }{} &
\multicolumn{1}{ c|  }{  \textbf{Our Method (AVD Penalizers)}}  &&
$\mathbf{0.776}$    &  $\mathbf{0.00}$  &  $\mathbf{0.04}$ &
$\mathbf{0.807}$    &  $\mathbf{0.00}$   &  $\mathbf{0.01}$ &
$\mathbf{0.950}$    &  $\mathbf{0.01}$  &  $\mathbf{0.00}$
\\ \cline{1-2} \cline{4-12}
\multicolumn{1}{ |c  }{} &
\multicolumn{1}{ c|  }{  \textbf{Our Method (SD Penalizers)}}  &&
$\mathbf{0.783}$    &  $\mathbf{0.00}$  &  $\mathbf{0.09}$ &
$\mathbf{0.806}$    &  $\mathbf{0.01}$   &  $\mathbf{0.02}$ &
$\mathbf{0.950}$    &  $\mathbf{0.00}$  &  $\mathbf{0.00}$
\\ \cline{1-2} \cline{4-12}
\multicolumn{1}{ |c  }{} &
\multicolumn{1}{ c|  }{  \textbf{Vanilla Regularized Logistic Regression}}  &&
$\mathbf{0.800}$    &   $\mathbf{0.08}$    &   $\mathbf{0.39}$  &
$\mathbf{0.807}$    &   $\mathbf{0.01}$    &   $\mathbf{0.05}$  &
$\mathbf{0.951}$    &   $\mathbf{0.16}$    &   $\mathbf{0.02}$
\\ \cline{1-2} \cline{4-12}
\end{tabular}
}
\vspace{3mm}
\caption{Performance on the Adult, Loan Default, and Admissions datasets, penalizing for both FPR and FNR difference. Accuracy, FPR difference and FNR difference are evaluated on the test set, averaging over five runs and using a 30-70 training/test split.} \label{table:comparison_results_rest}
\end{table*}


In Table~\ref{table:comparison_results}, we compare the performance of our approach with that of three other techniques from the literature. Each method was trained based on logistic regression.  As a basis for comparison, we also present the performance of vanilla logistic regression, absent fairness considerations, with the regularization parameter selected via cross-validation.\footnote{Zafar et al.~(\citeyear{disparatemistreatment}) do not incorporate regularization in any of the approaches they report.}
%Results are reported as the averages of 5 different runs \katrina{Is that still correct?}, each time splitting data evenly and randomly into training and test sets. 
Results for Zafar et al., Zafar et al. baseline, and Hardt et al. appear here as reported in Zafar et al.~(\citeyear{disparatemistreatment}).\footnote{Our method selects the classifier based on the training set only and reports its performance over the test set. Results for the three other approaches, reported by Zafar et al.~(\citeyear{disparatemistreatment}), are based on tuning parameters after seeing the trade-off curve over the test set, and reporting according to the best selection of these parameters.}
%\katrina{Perhaps here is the right place for a footnote about the discrepancy with the Zafar baseline}

We find that the vanilla logistic regressor (absent fairness considerations) results in significant unfairness, as $\mathbf{D_{FPR}}=0.20$, and $\mathbf{D_{FNR}}=0.30$. The overall accuracy of this classifier measured on the test set was $0.672$.\footnote{Zafar et al.~(\citeyear{disparatemistreatment}) report a slightly different baseline of: Accuracy = 0.668, $\mathbf{D_{FPR}}=0.18$, $\mathbf{D_{FNR}}=0.30$.} Our SD penalization approach empirically achieves approximately the same accuracy as the Zafar et al.~(\citeyear{disparatemistreatment}) approach, with significantly better fairness. It is difficult to compare fairness-accuracy tradeoffs with the Hardt et al.~(\citeyear{hardt}) approach, since their accuracy is significantly lower than ours. A more direct comparison is possible by noting that our learned classifier can be post-processed to improve its fairness at a direct cost to accuracy. Hence, we can achieve accuracy of $0.659$ with $\mathbf{D_{FPR}} = \mathbf{D_{FNR}} = 0.01$, which compares very favorably with the Hardt et al. accuracy rate of 0.645 given the same FPR and FNR rates.\footnote{For completeness, we note that using a 50-50 training-test split (again not using the test set for parameter selection), our method (SD, both considerations) produces a classifier that provides: Accuracy = 0.659, $\mathbf{D_{FPR}} = 0.01, \mathbf{D_{FNR}} = 0.05$. This classifier can be post-processed to achieve rates of: Accuracy = 0.655, $\mathbf{D_{FPR}} = \mathbf{D_{FNR}} = 0.01$.}

Figure \ref{fig:compas} illustrates the accuracy/fairness trade-offs achievable using our scheme. Increasing the weight $c$ on the proxy fairness penalizers results in reducing their magnitude. The figure also illustrates how our relaxed penalizers succeed in tracking the real FPR and FNR differences. 
%
%
%\katrina{Must rewrite the following paragraph}
%We observe that our method succeeds in eliminating unfairness almost completely on the COMPAS dataset, while retaining most of the accuracy, when compared to the vanilla logistic regression. We achieve very low difference rates when penalizing for achieving each of the FPR and FNR criteria individually, and also for both. We achieve preferable results comparing to Zafar et al. and Zafar et al. baseline in all 3 scenarios, and also comparing to Hardt et al. in the settings of false positive/false negative considerations only. In the setting of both considerations - The Hardt et al. method removes a larger portion of the unfairness, however it results in major accuracy loss as it achieves accuracy rate of 0.645 in comparison to our method which results in accuracy of 0.665, retaining most of the original accuracy rate while removing most of the unfairness.




%The Hardt et al.~\cite{hardt} approach as reported removes a smaller portion of the bias in the different scenarios, however for FP/FN constraints alone, it provides higher accuracy rates. The Zafar et al.~(\citeyear{disparatemistreatment}) approach as reported retains significant bias (in most cases), but in some cases  achieves slightly superior accuracy rates to the methods above. 

%These performance comparisons are incomplete in the sense that each of the compared techniques has the potential to trade off between accuracy and fairness, using some degree of parameter tuning; what we report here is only one point on the achievable trade-off frontier for each algorithm. The ``correct'' trade-off, and, in particular, the best manner in which to weigh unfairness in the FPR against unfairness in the FNR, are matters of opinion. We have chosen to report our method's performance under parameters designed to very aggressively mitigate unfairness, at some cost to the accuracy.

%It would certainly be desirable to evaluate these and other approaches to fair learning on other datasets and on different tasks, particularly on larger datasets, which might afford both greater accuracy and better bias-reduction. The present empirical evaluations, however, suggest that our regularization-based approach provides a new tool worthy of consideration---we succeed in almost entirely eliminating bias on the hold-out set, at a modest price in terms of accuracy.

%Due to the fact that our true objective includes the original non-convex penalization terms, our approach does not carry any formal guarantees. However, the ease of implementation, generality, and empirical results are encouraging. Figure~\ref{fig:test1} illustrates the rate of convergence to a fair, accurate classifier on this dataset.
%In terms of computation costs, given that at each iteration we must calculate the gradient according to the FPR and FNR regularizers, we are required to predict the labels for the entire training set at each step. 
%However, this does not pose a computational burden, as it is already required by the (classic) gradient descent algorithm in our logistic regressor fitting scheme. Furthermore, when given a sufficiently large dataset (one or two orders of magnitude larger than the one currently available for the COMPAS scores data), this could be relaxed to sampling only a mini-batch of samples from the training data set at each iteration (much as is done in stochastic gradient descent).






\subsection{Additional Datasets}


Table~\ref{table:datasets_description} provides summary statistics on each of the datasets on which we tested our approach. We also briefly describe the datasets below. 


{\bf The Adult Dataset}\footnote{http://archive.ics.uci.edu/ml/datasets/Adult} is based on 1994 US Census data. The task we consider is to predict whether the income of each individual is over or under 50K dollars per year, based on features such as occupation, marital status, and education. The protected attribute selected in this task is gender. 

{\bf The Loan Default Dataset}\footnote{{\scriptsize https://archive.ics.uci.edu/ml/datasets/default+of+credit+card+clients}}
contains data regrading Taiwanese credit card users. The task we consider is to predict whether an individual will default on payments, based on features such as history of past payments, age, and the amount of given credit. The protected attribute is gender.

{\bf The Admissions Dataset}\footnote{http://www2.law.ucla.edu/sander/Systemic/Data.htm}
contains records of law school students who went on to take the bar exam. The task we consider is to predict whether a student will pass the exam based on features such as LSAT score, undergraduate GPA, and family income. The protected attribute is set to race.

Table~\ref{table:comparison_results_rest} describes the performance of our approach on these datasets, and Figures~\ref{fig:adult},~\ref{fig:default}, and~\ref{fig:lawschool} illustrate the fairness-accuracy trade-offs we achieve in each context. Overall, we see that unfairness is nearly eliminated while accuracy remains quite high. The dataset on which accuracy suffers most under our approach is the Adult dataset, which is also the dataset on which the vanilla regression is the most unfair.


\begin{figure*}[]
  \includegraphics[scale=0.6]{adult0-800.png}
  \caption{Adult Dataset. Fairness-Accuracy tradeoffs, as in Figure~\ref{fig:compas}.}
  \label{fig:adult}  
\end{figure*}



\begin{figure*}[]
  \includegraphics[scale=0.6]{default0-50.png}
  \caption{Loan Default Dataset. Fairness-Accuracy tradeoffs, as in Figure~\ref{fig:compas}.}
  \label{fig:default}
\end{figure*}



\begin{figure*}[]
  \includegraphics[scale=0.6]{admissions0-400.png}
  \caption{Admissions Dataset. Fairness-Accuracy tradeoffs, as in Figure~\ref{fig:compas}.}
  \label{fig:lawschool}
\end{figure*}



\section{Non-constant queries and commands} \label{sec:non-constant}
As we have shown, connectivity queries propagate through the \XStream ring
processings in $p$ \XStream ticks, and the query answer is sent from the
tail processor to the head, then back to the I/O processor. Another potentially
useful query that finishes in $p$ \XStream ticks is ``How many edges are in the system?''
\Cindy{Add some more here.  It can just be a list. Many that one might think should be constant probably require some data structure support, in whcih case, add them at the end.} \Cindy{Revisit notation.}

\XStream also supports queries with non-constant-sized output. At most
one such query can be active at a time. The answer to the query is
output in constant-sized pieces using the payload slots. The canonical
non-constant query is a request to output all vertices in small
connected components. Specially, the answer is the names of all
components with at most $\lambda$ vertices and the list of vertices
within them.  This query makes practical sense only in graphs that
have a giant connected component, but most real graphs have one. We
describe how \XStream executes this specific query.

For a local component with name $\eta$, let $s_{\eta}$ be the number
of vertices in $\eta$.  A processor can compute the size of a local
component as the sum of the number of vertices in each building block.
This is $1$ for a primitive building block.  For this discussion, we
assume processors keep track of the number of primitive building
blocks for each local component while building these components. This
adds only constant work per union-find operation.  However, it's also
possible to inialize local-component sizes to zero and compute them
on-the-fly for this query.  But then, the processor does at most k-1
work counting primitive building blocks or outputing the messages
below, which will further delay the query response. Processors receive
the size of non-primitive building blocks from upstream processors.

When the head processor receives the query ``Output the vertices in
components that have at most $\lambda$ vertices'' in the primary slot
of a bundle, it passes the query downstream in the primary slot. This
allows all processors to learn the type of query and the parameter
$\lambda$. The head then uses the $k-1$ payload slots to start
answering the query.  The query is answered in two phases.  In the
first phase, processors compute component sizes.  For each local
component with name $\eta$, such that $s_{\eta} \le \lambda$, the head
processor (eventually) sends a message ``($\eta, s_{\eta})$'' in a
payload basket. The head outputs $k-1$ of these messages per bundle if
it already knows its component sizes. After the last message, it
outputs a ``query phase done'' token.

Each downstream processor passes the initial query downstream.  Then
for each message $(\eta, s_{\eta})$, the processor checks to see if
$\eta$ is a building block for one of its local components $\eta'$.
If it is, then it increments the size of $\eta'$.  If $\eta$ is not a
local building block, the processor sends the message downstream.
When the processor receives the ``query phase done'' token, it knows
the size of all its non-primitive building blocks, and hence knows
the size of all of its local components.  It sends its own ``($\eta,
s_{\eta})$'' messages for each local component $\eta$ such that
$s_{\eta} \le \lambda$. When it has sent all its messages, it passes
the ``query phase done'' token downstream. If the current graph has a
connected component $\eta_G$ that has size at most $\lambda$, the
message with its final size is passed through the tail and out to the
analyst.  The tail also passes the ``query phase done'' token to the
head.

Sealed processors (full of tree edges) can set a flag indicating they
have computed their component sizes. If there is another such query
before an aging, then it removes messages associated with its local
building blocks without incrementing any size counters.

In the second phase, the head processor (eventually) sends a message
$(\eta, v_i)$, for each primitive vertex $v_i$ in each local component
$\eta$ reported in the first phase.  For the head, all building blocks
are primitive vertices. It's possible to put more than one vertex in
the latter kind of message (e.g. $(\eta, v_1, v_2, v_3)$, depending
upon the size of a slot.  After the last such message, the head passes
a ``query done'' token downstream.

When a downstream processor receives a message $(\eta, v_i)$ from
upstream in the second phase, it checks to see if $\eta$ is a building
block for one of its local components $\eta'$. If not, then it passes
the message downstream.  If so, then $s_{\eta'} \le \lambda$ (i.e. the processor
reported local component $\eta'$ in the first phase, it relabels the message,
sending $(\eta', v_i)$ downstream.  If $\eta'$ is too large, it just removes
the message from the system.

When a downstream processor receives the ``query done'' token, it outputs
messages $(\eta, v_i)$, where $\eta$ is a local componet with $s_{\eta} \le \lambda$
and $v_i$ is a primitive building block (vertex) in local component $\eta$.

A somewhat easier non-constant query is spanning tree. Starting with the head, each
processor outputs its tree edges.

Some queries can be either constant-size (latency $p$) or non-constant
depending upon what additional data structures the processors
maintain.  One example is ``What is the degree of node $v$?'' Suppose
each processor maintains adjacency lists for the subgraph it
holds. Then the processor can find the number of edges adjacent to a
vertex $v$ in constant time, given a hash table to access the
adjacency list for each vertex.  In this case, the vertex-degree query
has latency $p$.  The query makes one pass around the ring with the answer progressing
one processor per tick.  Otherwise, without this data structure, each processor will need time to
compute the number of edges it holds that are adjacent to vertex $v$.
In this case, it is a non-constant query.  The message still touches
each processor once, but the processor may require multiple ticks to
compute the number to add to the accumulating degree value.

Linear algebraic computations typically involve a matrix-vector product,
which would be unweildy to compute directly in the \XStream model.
However, the emerging field of randomized linear
algebra~\cite{drineas2018lectures}
offers a path forward. If we devote some space in the tail processor to
accommodate a sample of edges (adjusting Lemma~\ref{lemma:aging-lead-time}
accordingly), payload slots can be used to accumulate a random sample
of the graph.  Techniques such as randomized 
PageRank~\cite{gasnikov2015efficient} or others might then be applied in a 
separate
thread in the tail processor, still with minimal interruption to the input 
stream.

\section{Conclusion and Future work}

\section{Discussion and Conclusions}



Our method based on stabilizing forward and backward pass, resulted in improved accuracy over the baseline and it was able to predict optimal dampening, sharpness and tail-fatness before training. 
Our findings are coherent with the line of research that has established that stabilizing gradients and representations at initialization results in better performance \cite{glorot2010understanding, orthogonal_initialization, he2015delving, roberts2022principles, defazio2022scaling, bengio1994learning, hochreiter1997long, hochreiter2001gradient, arjovsky2016unitary, pascanu2013difficulty}. Moreover it gives an initial reply to the question raised by
\cite{surrogate2019, zenke2021remarkable}, which asked  for a theoretical justification of initialization and SG choice for Spiking Neural Networks. With a similar intention, \cite{rossbroich2022fluctuation} proposed an approach that guarantees sparsity of activity at initialization to pick the weights distribution at initialization, resulting in improved accuracy. Our method differs from theirs in that it starts from a principle of stability to derive constraints, instead of a principle of sparsity. It differs also in that we use it to define the SG shape at initialization, not only the weights distribution, and we can show mathematically how weights initialization is intertwined to the SG shape choice. Our results suggest that a tedious hyper-parameter grid-search can be often avoided by making use of sound and established principles of learning.

One of the conditions was designed to hit the most sensitive part of an SG, its center, which resulted in a low sparsity requirement at initialization. This is very uncommon in the Neuromorphic literature, since sparsity brings large energy gains \cite{henderson2020towards,blouw2019benchmarking, 9395703,taulsnn, rossbroich2022fluctuation}.
However, the energy gains of SNNs also come from their binary activity. A matrix-vector multiplication, with a $\mathbb{R}^{m\times n}$ matrix, has an energy cost of $mnE_{MAC}$ for a real vector, and of $mn\rho E_{AC}$ for a binary vector, where $\rho$ is the Bernouilli probability of the binary vector, and in our case the neuron firing rate, and $E_{AC}, E_{MAC}$ are the energies of an accumulate and a multiply-accumulate operation \cite{yin2021accurate, hunger2005floating}. Since MAC are more costly than AC, 31 times on a $45$nm complementary metal–oxide–semiconductor \cite{yin2021accurate, horowitz20141}, we have energy savings with any $\rho$, e.g., when all neurons fire ($\rho=1$) and when they fire half of the time steps ($\rho=1/2$). This gain does not depend on the simulation speed, since it compares a spiking and an analogue computation, at the same computation speed.
Typically requiring more sparsity through a sparsity encouraging loss term, leads to a measurable decrease in performance \cite{zenke2021remarkable, rossbroich2022fluctuation}. However we observed that it is actually possible to achieve higher performance with higher sparsity, by starting with a strong firing rate at initialization, since their synergy acts as a regularization mechanism. This was possible also because the sparsity encouraging loss term was introduced gradually, and because its contribution was kept comparable to the task loss towards the end of training.

We observed that the more complex the task is and the more complex the network to train is, the more drastic is the difference in performance of different SG shapes. It is known that learning is possible with a wide variety of SG shapes \cite{zenke2021remarkable} and the community has not yet settled for one shape or one method to reliably choose which SG to use in each case \cite{surrogate2019}. We showed how to apply a well known stability principle to the forward and backward pass of the simplest Spiking Neural Network, the LIF, as a starting point, but we think that the principles of good Neuromorphic initialization can be further elaborated, in order to tackle more complex tasks and networks.





\begin{acknowledgements}
Sandia National Laboratories is a multimission laboratory managed and operated by National Technology \& Engineering Solutions of Sandia, LLC, a wholly owned subsidiary of Honeywell International Inc., for the U.S. Department of Energy’s National Nuclear Security Administration under contract DE-NA0003525.
This research was funded through the Laboratory Directed Research and Development (LDRD) program at Sandia.
This paper describes objective technical results and analysis. Any subjective views or opinions that might be expressed in the paper do not necessarily represent the views of the U.S. Department of Energy or the United States Government.
We thank Siva Rajamanickam, Cannada Lewis, and Si Hammond for useful
discussions and baseline code for the TBB benchmark.
\end{acknowledgements}


% Authors must disclose all relationships or interests that 
% could have direct or potential influence or impart bias on 
% the work: 
%
% \section*{Conflict of interest}
%
% The authors declare that they have no conflict of interest.


% BibTeX users please use one of
\bibliographystyle{plain}      % basic style, author-year citations
%\bibliographystyle{spmpsci}      % mathematics and physical sciences
%\bibliographystyle{spphys}       % APS-like style for physics
%\bibliography{}   % name your BibTeX data base
\bibliography{ms}


\end{document}
% end of file template.tex

