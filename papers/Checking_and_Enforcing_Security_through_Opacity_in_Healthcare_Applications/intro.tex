\addcontentsline{toc}{section}{Introduction}
\section{Introduction}

Real-world usage of IoT in health-care necessitates the dealing with new security challenges. In fact, and since this type of application would handle medical and personal information, their employment carries serious risks for personal privacy. Accordingly, it is paramount to protect any sensitive data against recovery or deduction by third-parties to avoid the compromise of individual privacy. The most common security preservation practice is the use of cryptographic techniques. However, cryptographic protocols do not provide perfect security as the inference of critical information from non-critical ones remains a possibility. Indeed, the discovery of vulnerabilities of simple crypto-systems like that of the Needham-Schroeder public key protocol \cite{lowe1995attack} proved that cryptography is not enough to guarantee the privacy of information. Furthermore, the various techniques available are computationally intensive. This is why they cannot be immediately adopted in IoT where the network nodes are powered by battery. To facilitate the adoption of IoT in health-care, we need formal (preferably automated) verification of security properties. Formal verification entails the use of mathematical techniques to ensure that the system's design conforms to the desired behavior. Information flow properties are the most formal security properties. In fact, various ones have been defined in the literature including non-interference \cite{DBLPM82}, intransitive non-interference \cite{nbha1} and others (e.g. secrecy, and anonymity). Interested in confidentiality properties, we consider opacity, a general information flow property, to analyze IoT privacy in a heart attack detection system. Opacity's main interest is to formulate the need to hide information from a passive observer. It was first introduced in \cite{Mazar} and was later generalized to transition systems~\cite{Bryans2008}. It has since, been studied several times allowing the formal verification of system models (usually given as non deterministic automata or labeled transition systems). Its wide study led to the birth of several definitions (variants) as well as verification and enforcement techniques. If classified according to the security policy, then we are dealing with simple, $K$-step, initial, infinite as well as strong and weak opacity alongside their extensions (e.g, $K$-step weak and $K$-step strong opacity). The efforts of these studies also made possible not only opacity verification, but also its assurance via supervision \cite{Dubrieil2009}, \cite{SabooriH12} or enforcement \cite{Falcone2013}. A key limitation of these studies is that they have been very theoretical in the way they have approached and applied opacity. 

In this paper, we wish to bridge the gap between the theory of opacity and its practical application through the synthesis of an opaque IoT-based heart attack detection system. Building on the SOG-based verification approach developed in \cite{bourouis2015checking}, the purpose is to verify opacity in three of its forms (simple, $K$-step weak opacity and $K$-step strong opacity) to detect security violations in our synthesized system. Then to contribute an algorithmic approach that enforces simple opacity by padding the system with minimal dummy behavior.

This paper is organized as follows: Section 2 establishes all necessary basic notions including the SOG structure and the opacity property. In Section 3, we detail the case study. In Section 4, we illustrate the practical usefulness of the opacity verification approach in the heart attack detection system. Section 5 details our proposed approach to enforce simple opacity. Finally, we conclude in Section 6, and list some potential future works. 
