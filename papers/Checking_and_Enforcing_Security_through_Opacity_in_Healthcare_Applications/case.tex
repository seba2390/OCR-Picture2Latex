\addcontentsline{toc}{section}{Motivating Scenario}
\section{Motivating Scenario}
Heart disease is the first cause of morbidity and mortality in the world, accounting for 28.30\% of total deaths each year in Tunisia alone \cite{Organization2014}. Investment in preventive health care such as the use of IoT monitoring devices and tools may help lower the cost of processing and the development of serious health problems. In fact, integrating clinical decisions with electronic medical records could decrease medical errors, reduce undesirable variations in practice, and improve patient outcomes.

Our case study considers IoT integration with cloud computing. We use a connected bracelet, fog nodes, a private and a public Cloud, and a mobile application, which together form a medical application. This latter provides continuous monitoring of the vital data of a given patient. Regular or routine measurements could help to detect the first symptoms of heart malfunction, and makes it possible to immediately trigger an alert. The vital information collected by the bracelet worn by the patient includes cardiac activity, blood pressure, oxygen levels and, temperature. As mentioned earlier, we consider an IoT application in a hybrid cloud/fog environment. The cloud \cite{zhang2010cloud} is considered as a highly promising approach to deliver services to users, and provide applications with low-cost elastic resources. Given the fact that IoT  suffers limited computational power, storage capacity and bandwidth, cloud computing ease the issues in enabling the access, the storage, and the processing of the large amount of generated data.

Public clouds provide cheap scalable resources. Making it useful for analyzing the patient's vital data which would be costly as it requires extensive computing and storage resources. However, we must take into account that storage of health records on a public environment is a privacy risk. To avoid such security leaks, we could deploy the application on a secure private cloud. But seeing this latter's limited resources, this may degrade the overall performance. To prevent this, the workflow can be partitioned between a private cloud and a public one. Therefore, the confidential medical data will be processed on the private cloud. Other workflow actions can be deployed on the public cloud dealing with anonymized data.

The use of a cloud-based framework poses the problem of delay when sending and receiving data between the objects and geographically far cloud resources thus jeopardizing the patients' well-being given that triggering timely responses is the purpose of this data. To resolve this issue, data gathering can be moved from the cloud domain to that of the fog \cite{BonomiMZA12}. Bringing this action closer to the connected object shortens the transmission time, and reduces the amount of data transferred to the cloud.
The proposed workflow is described as follows:
	\begin{compactitem}
		\item A patient may register via the mobile app by entering his information. This information include personal data and medical history (personal and family medical histories, surgical history, drug prescriptions, and the doctors' notes). 
		\item The patient's medical history is then transmitted to the private cloud. After reception, this latter anonymizes the data by stripping off all that could identify the patient leaving only medical data, which it sends to the public cloud.
		\item The public cloud receives the anonymized data, and proceeds to the classification attaching to each medical file a class.
		\item The patient is equipped with a  measuring bracelet connected to the processing components (Fog nodes). The data sent to the fog domain is a set of vital data recorded over a period of time. 
		\item The fog node collects the data then compares it to its predecessors, searching for any vital signs changes. When the node determines that a change has occurred, it sends the data to the private cloud. 
		\item The private cloud links the gathered data with the patient, transmitting this data and the class ascribed to the patient, to the public cloud.
		\item The public cloud reads the data, analyzes it, and then provides results. When the risk of heart attack is detected, it immediately notifies the patient's app.
	\end{compactitem}


