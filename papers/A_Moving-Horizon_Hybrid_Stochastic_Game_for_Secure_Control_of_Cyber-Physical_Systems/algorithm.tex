\section{A Moving-Horizon Approach for Hybrid Stochastic Game}
\label{sec:algorithm}%Unlike the receding-horizon Stackelberg game with the leader and follower for correlated jamming attacks in~\cite{MartZhu_game}, the equilibrium considered in this section is different and the game state is hybrid.a time window of size $T$ is used, with physical state $x_{[k-T, k]}$ information by looking back $T$ stages and its associated cyber state $\delta_l$
In this section, we propose a moving-horizon algorithm to compute the saddle-point equilibrium strategy at each stage of the hybrid stochastic game. A saddle-point equilibrium strategy is computed at each stage $k$ by predicting anticipated future cost based on the hybrid state of the system $(x_{[k-T, k]},\delta_l )$. We develop Algorithm~\ref{finite_new} based on this concept, provides a scalable and a computationally tractable process, and compare the computational costs with Algorithm~\ref{finite}. The saddle-point equilibrium strategy and the value of the moving-horizon game at each stage involves solving finite zero-sum matrix games. By looking one stage ahead of the game state at $k$, predicting the physical dynamics $\mathbf{x}_{k+1}$ given any action pair, we obtain an objective function that reflects the payoff of the current stage and future expectation for computing the strategies at $k$. %The moving horizon process is illustrated as Figure~\ref{sg}. 

Given any action pair $(a_{ik},u_{jk})$ at stage $k$, we first update the state space form of the system dynamics $\mathbf{x}_{k+1}$ based on $\mathbf{x}_{[k-T,k]}$ as~\eqref{dynamicgame}. We view $\mathbf{x}_{k+1}$ as a function of $(\mathbf{x}_{[k-T, k]}, a_{ik},u_{jk})$, the immediate payoff function ${r}^{ij}(s_{(k+1)h})$ (for stage $k+1$) defined as~\eqref{payoff} is also a function of the current game state and players' actions. We denote this relation as $r_{k+1}(\mathbf{x}_{[k-T, k]}, a_{ik},u_{jk}, \delta_h)$ in the following algorithms to distinguish it between definition~\eqref{payoff}, where the latter is the payoff results from the action of two players' at stage $k+1$. Then, we compute the value of the matrix game at stage $k+1$, by looking one stage ahead and consider stage $k+1$ as the terminal stage of the game, the value of game stage $k+1$ is now directly calculated via for $r^{ij}(\mathbf{x}_{[k-T,k]},a_{ik},u_{jk},\delta_h)$, $h=1,2,3,$ $i\in \{1,\cdots, M\}, j \in \{1,\cdots, N\}$ as~\eqref{prev}:
\begin{align}
v^{ij}_{k+1}(x_{[x-T,k]},\delta_h)= \min\limits_{\mathbf{g}}\max\limits_{\mathbf{f}}(r(\mathbf{x}_{[k-T,k]},a_{ik},u_{jk},\delta_h)),
\label{prev}
\end{align}
where $v_{k+1}(x_{[k-T,k]},\delta_h) \in \mathbb{R}^{M \times N}$ is the value matrix of stage $k+1$ estimated at stage $k$ based on the current game state and all possible action pairs. With the predicted value from the next stage, define the moving-horizon auxiliary matrix for stage $k$ as:
%\footnotesize
\begin{align}
\begin{split}
&Q_{k}(s_{kl})\\=&r(s_{kl})
                         +\sum_{s_h\in S} P_k( s_{(k+1)h} |s_{kl})\cdot v_{k+1}(x_{[k-T,k]},\delta_h),
\end{split}
\label{Q_k}
\end{align}
The dot products of matrices ${P}_{k}( s_{(k+1)h} |s_{kl})$, $v_{k+1}(x_{[k-T,k]}$, $\delta_h)$ is an element-wise product of two elements at the same position of the two matrices. The value and stationary equilibrium strategies that Algorithm~\ref{finite_new} calculates at each stage $k$ is defined as following.
\begin{defn}
Given $s_{kl}$, $v_{k+1}(x_{[k-T,k]},\delta_h)$ as~\eqref{prev}, and auxiliary matrix $Q_{k}(s_{kl})$ as~\eqref{Q_k}, the value and equilibrium strategies at $k$ are defined as the following equation: 
\begin{align}
v(s_{kl})= \min\limits_{\mathbf{g}_k(s_{kl})}\max\limits_{\mathbf{f}_k(s_{kl})} \mathbf{f}_k(s_{kl})^T Q_{k}(s_{kl}) \mathbf{g}_k(s_{kl}),
\label{v_k}
\end{align}
where we treat the auxiliary matrix $Q_{k}(s_{kl})$ as the payoff matrix of a zero-sum game of stage $k$. 
%The value and equilibrium strategies of the matrix game $Q_{k}(s_{kl})$ consist with the meaning of value and equilibrium strategies in Definition~\ref{zero_sum}.
\end{defn}

At each stage $k$, we repeat calculating $Q_{k}(s_{kl})$ and the corresponding value and equilibrium strategies, then update the system dynamics by the strategies for computation of next stage. The complete process is summarized as Algorithm~\ref{finite_new}. 
%\newtheorem{algorithm}{Algorithm}
\begin{alg}
%\label{finite_new}
\textbf{: Moving-Horizon Algorithm for A Hybrid Stochastic Game}\\
%\begin{algorithmic}
\textbf{Input}: System model parameters and game parameters.
\\\textbf{Initialization}: $\hat{\mathbf{x}}_{[0,T]}$.
\\\textbf{Iteration}: For $k=T, \cdots, K+T-1$,                                                                                 
 $s_{kl}=(x_{[k-T,k]},\delta_l),$  $l=1,2,3$:  
get  the auxiliary matrix~\eqref{Q_k};
%\[Q_{k}(s_{kl}) = r_{k}(s_{kl})+\sum_{s_h\in S} \tilde{P}_{k}( s_{(k+1)h} |s_{kl})\cdot v_{k+1}(x_{[k-T,k]},\delta_h),\]
compute the value and equilibrium strategies of every matrix game:\\
$v(s_{kl})= \min\limits_{\mathbf{g}(s_{kl})}\max\limits_{\mathbf{f}(s_{kl})} \mathbf{f}(s_{kl})^T Q_{k}(s_{kl}) \mathbf{g}(s_{kl})$,\\
$\mathbf{f}_k^{*}(s_{kl})=\arg \max\limits_{\mathbf{f}_k(s_{kl})}\mathbf{f}_k(s_{kl})^T Q_{k}(s_{kl} )\mathbf{g}_k^*(s_{kl})$, \\
$\mathbf{g}_k^{*}(s_{kl})=\arg \min\limits_{\mathbf{g}_k(s_{kl})} [\mathbf{f}_k^{*}(s_{kl})]^T Q_{k}(s_{kl}) \mathbf{g}_k(s_{kl})$.\\
Update the system dynamics with strategies $\mathbf{f}_k^{*}(s_{kl}),\mathbf{g}_k^{*}(s_{kl}),$ $l=1,2,3$ as described in~\ref{dynamicgame} for the next stage.
\\\textbf{Return}: the concatenation of strategies for both players $\mathbf{f}=\{f_k^{*}(s_{kl})\},\mathbf{g}=\{\mathbf{g}^*_k(s_{kl})\}$ and the value sequence $v_{k}(s_{kl}),k=T,\cdots, K+T, l=1,2,3$.
\label{finite_new}
\end{alg}
To get the total payoff till stage $k$ by Algorithm~\ref{finite_new}, we plug the strategies $\mathbf{f}, \mathbf{g}$ into the system dynamics and calculate the sum of payoff for all stages. It is worth noting that~Algorithm~\ref{finite_new} reduces the computational overhead for the hybrid stochastic game. The complexity of Algorithm~\ref{finite_new} is equivalent to the complexity of solving $(KMN)$ times of minimax problem with an $M \times N$ payoff matrix, while the complexity of suboptimal Algorithm~\ref{finite} is equivalent to the complexity of solving $((MN)^{K})$ times of minimax problem with an $M\times N$ payoff matrix. %Because it takes the total expected payoff of $K$ stages ahead as an objective function. %Numerical comparisons are shown in Section~\ref{sec:simulation}. %This is because Algorithm~\ref{finite} looks $K$ stages ahead at once and compute a robust game for every iteration. The advantage of suboptimal Algorithm~\ref{finite} is to provide an upper bound of the total finite cost..

\begin{rem}
%Given the sets of models for each component of the subsystems and attacks, 
The system dynamics are defined by a sequence of action pairs $(a_{ik},u_{jk})$ randomly chosen by the attacker and the system, and are equivalent with a system that randomly switches among $N$ subsystems according to the stochastic game strategies $\mathbf{f}_k(s_{kl})$. The strategy sequences $\mathbf{f}_k^*(s_{kl})$, $\mathbf{g}_k^*(s_{kl})$ of the stochastic game converge to $\mathfrak{f}^l,\mathfrak{g}^l $, $l=1,2,3$, i.e.,
\\\centerline{$\mathfrak{f}^l=\lim\limits_{k\to\infty}\mathbf{f}_k^*(s_{kl}), \mathfrak{g}^l=\lim\limits_{k\to\infty}\mathbf{g}_k^*(s_{kl}), l=1,2,3,$}
if updating system dynamics at stage $k+1$ by ($\mathfrak{f}^l,\mathfrak{g}^l$) results in:
\\\centerline{$\lim \limits_{k \to \infty} Q_{k}(s_{kl})=\lim \limits_{k \to \infty} Q_{k}(s_{(k+1)l}), l=1,2,3.$}
%\end{prop}
This is because according to Algorithm~\ref{finite_new}, $\mathbf{f}^*_k(s_{kl})$, $\mathbf{g}^*_k(s_{kl}),l=1,2,3$ are the saddle-point equilibrium strategies for the auxiliary matrices $Q_{k}(s_{kl}),l=1,2,3$. When the strategy sequences of both players converge, the switched system dynamics converge to a discrete-time Markov jump linear system (with delays when the attacker's strategies include replay attacks), and the stability properties of the system that switches among stable and unstable subsystems is analyzed by~\cite{delay_mlj} and~\cite{switch_unstable}.\end{rem} 
%It is possible that some subsystems $u_{jk}, j \in \{1,\cdots, N\}$ are unstable under specific types of attacks, and the system switches among stable and unstable subsystems. Stability properties of continuous time linear switched systems including unstable modes are analyzed by. To guarantee exponential stability, the total activation time of unstable subsystems need to be relatively small compared with that of stable subsystems. More analysis of system stability conditions based on the moving horizon stochastic game framework will be an avenue of future work. With the heuristic moving horizon algorithm, by calculating strategies of the stochastic game for a large enough stage number, we obtain the switched dynamic process of the system under different types of attacks, and check whether stability conditions are violated