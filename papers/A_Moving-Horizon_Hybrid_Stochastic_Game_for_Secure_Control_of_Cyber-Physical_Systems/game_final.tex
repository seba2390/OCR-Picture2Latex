\documentclass[twocolumn]{autart}  % Comment this line out
                                                          % if you need a4paper
\usepackage{color}
\usepackage{graphicx}

\usepackage{hyperref}
\usepackage{amsmath,amsfonts, amssymb}
\usepackage{multirow}
\usepackage{natbib}
%\bibliographysyle{agsm}
\begin{document}

\begin{frontmatter}
\title{A Moving-Horizon Hybrid Stochastic Game for Secure Control of Cyber-Physical Systems\thanksref{footnoteinfo}} % Title, preferably not more 
                                                % than 10 words.

\thanks[footnoteinfo]{This material is based on research sponsored by DARPA under agreement number FA8750-12-2-0247.  The U.S. Government is authorized to reproduce and distribute reprints for Governmental purposes notwithstanding any copyright notation thereon.  The views and conclusions contained herein are those of the authors and should not be interpreted as necessarily representing the official policies or endorsements, either expressed or implied, of DARPA or the U.S. Government. This work was also supported in part by NSF CNS-1505701, CNS-1505799 grants, and the Intel-NSF Partnership for Cyber-Physical Systems Security and Privacy. This paper was not presented at any IFAC meeting. Part of the results in this work appeared at the 52nd Conference of Decision and Control, Florence, Italy, December 2013~\cite{cdc_replay} and the 53rd Conference of Decision and Control, Los Angeles, CA, USA, December 2014~\cite{game_cdc14}. Corresponding author F.~Miao. Tel. 2154216608.}

\author[Uconn]{Fei Miao}\ead{fei.miao@uconn.edu},  % Add the 
\author[NYU]{Quanyan Zhu}\ead{quanyan.zhu@nyu.edu},               % e-mail address 
\author[Duke]{Miroslav Pajic}\ead{miroslav.pajic@duke.edu},  % (ead) as shown
\author[Upenn]{George J. Pappas}\ead{pappasg@seas.upenn.edu}

\address[Uconn]{ University of Connecticut, Storrs, CT, USA}
\address[Upenn]{ University of Pennsylvania, Philadelphia, PA, USA}  % Please supply                                              
\address[NYU]{ New York University, Brooklyn, NY, USA}             % full addresses
\address[Duke]{Duke University, Durham, NC, USA}
%\address[Pappas]{ University of Pennsylvania, Philadelphia, PA, USA}

\begin{keyword}
Stochastic Game, Secure Control, Saddle-Point Equilibrium
\end{keyword}
%\maketitle

\begin{abstract}
\label{abstract}
 In this paper, we establish a zero-sum, hybrid state stochastic game model for designing defense policies for cyber-physical systems against different types of attacks. With the increasingly integrated properties of cyber-physical systems (CPS) today, security is a challenge for critical infrastructures. Though resilient control and detecting techniques for a  specific model of attack have been proposed, to analyze and design detection and defense mechanisms against multiple types of attacks for CPSs requires new system frameworks. Besides security, other requirements such as optimal control cost also need to be considered. The hybrid game model we propose contains physical states that are described by the system dynamics, and a cyber state that represents the detection mode of the system composed by a set of subsystems. A strategy means selecting a subsystem by combining one controller, one estimator and one detector among a finite set of candidate components at each state. Based on the game model, we propose a suboptimal value iteration algorithm for a finite horizon game, and prove that the algorithm results an upper bound for the value of the finite horizon game. A moving-horizon approach is also developed in order to provide a scalable and real-time computation of the switching strategies. Both algorithms aims at obtaining a saddle-point equilibrium policy for balancing the system's security overhead and control cost. 
%This approach leads to a real-time algorithm that yields a sequence of Nash equilibrium strategies which can be shown to converge. 
The paper illustrates these concepts using numerical examples, and we compare the results with previously system designs that only equipped with one type of controller. 
\end{abstract}

\end{frontmatter}

\section{Introduction}  \label{sec:introduction}

\newcommand\inexpIntro[3]{#1?(#2,#3).}
\newcommand\rinexpIntro[3]{*#1?(#2,#3).}
\newcommand\outexpIntro[3]{#1!(#2,#3).}
\newcommand\outatomIntro[3]{#1!(#2,#3)}

We propose a fully automated method for proving termination of \(\pi\)-calculus processes.
Although there have been a lot of studies on termination analysis for the \(\pi\)-calculus
and related calculi~\cite{Deng06IC,Demangeon07,SangiorgiTermination,KobayashiHybrid,Yoshida04IC,DBLP:journals/jlp/DemangeonHS10,Venet98SAS}, most of them have been rather theoretical,
and there have been surprisingly little efforts in developing  fully automated termination
verification methods and tools based on them. To our knowledge,
Kobayashi's \typical{}~\cite{TyPiCal,KobayashiHybrid} is the only exception that
can prove termination of \(\pi\)-calculus processes (extended with natural numbers)
fully automatically, but its termination analysis is quite limited (see Section~\ref{sec:relatedwork}).

Our method is based on a reduction to termination analysis for sequential programs:
we translate a \(\pi\)-calculus process \(P\) to a sequential program \(S_P\), so that
if \(S_P\) is terminating, so is \(P\). The reduction allows us to use
powerful, mature methods and tools
for termination analysis of sequential programs~\cite{heizmann2016ultimate,freqterm,DBLP:conf/lics/PodelskiR04,Kuwahara2014Termination,DBLP:journals/cacm/CookPR11}.

The idea of the translation is to convert a chain of communications on replicated input
channels to a chain of recursive function calls of the target sequential program.
Let us consider the following Fibonacci process:
\begin{align*}
    & \rinexpIntro{\fib}{n}{r}
        \ifexp{n<2}{ \soutatom{r}{1} \\ &\quad}
                   { \nuexp{s_1} \nuexp{s_2} (\outatomIntro{\fib}{n-1}{s_1} \PAR \outatomIntro{\fib}{n-2}{s_2} \PAR \sinexp{s_1}{x}\sinexp{s_2}{y}\soutatom{r}{x+y}) \\}
    & \PAR \outatomIntro{\fib}{m}{r}
\end{align*}
Here, the process
$\rinexpIntro{\fib}{n}{r} \ldots$ is a function server that computes the \(n\)-th Fibonacci number
in parallel and returns the result to \(r\),
and $\outatom{\fib}{m}{r}$ sends a request for computing the \(m\)-th Fibonacci number;
those who are not familiar with the syntax of the \(\pi\)-calculus may wish to consult
Section~\ref{sec:targetlanguage} first.
To prove that the process above is terminating for any integer \(m\),
it suffices to show that there is no infinite chain of communications on $\fib$:
\[
    \fib(m,r) \to \fib(m_1,r_1) \to \fib(m_2,r_2) \to \cdots.
\]
We convert the process above to the following program:\footnote{The actual translation
  given later is a little more complex.}
\begin{verbatim}
 let rec fib(n) = if n<2 then () else (fib(n-1) [] fib(n-2)) in
 fib(m)
\end{verbatim}
Here, \texttt{[]} represents the non-deterministic choice.
Note that, although the calculation of Fibonacci numbers is not preserved,
for each chain of communications on \texttt{fib}, there is a corresponding
sequence of recursive calls:
\[
\mathtt{fib}(m) \to \mathtt{fib}(m_1) \to \mathtt{fib}(m_2) \to \cdots.
\]
Thus, the termination of the sequential program above implies the termination of
the original process.
As shown in the example above, (i) each communication on a replicated input channel
is converted to a function call, (ii) each communication on a non-replicated input
channel is just removed (or, in the actual translation, replaced by a call of
a trivial function defined by \(f(\seq{x})=(\,)\)), and (iii) parallel composition
is replaced by a non-deterministic choice.
We formalize the translation outlined above and prove its correctness.

The basic translation sketched above sometimes loses too much information.
For example, consider the following process:
\begin{align*}
    & \rinexpIntro{\pre}{n}{r} \soutatom{r}{n-1} \\
    & \PAR \rinexpIntro{f}{n}{r} \ifexp{n<0}{ \soutatom{r}{1} }
                                       { \nuexp{s} (\outatomIntro{\pre}{n}{s} \PAR \sinexp{s}{x}\outatomIntro{f}{x}{r}) } \\
    & \PAR \outatomIntro{f}{m}{r}
\end{align*}
The translation sketched above would yield:
\begin{verbatim}
  let pred(n) = n-1 in
  let rec f(n) = if n<0 then () else (pred(n) [] f(*)) in
  f(m)
\end{verbatim}
Here, \texttt{*} represents a non-deterministic integer: since we have removed
the input $\sinatom{s}{x}$, we do not have information about the value of \( x \).
As a result, the sequential program above is non-terminating, although the original
process is terminating.
To remedy this problem, we also refine the basic translation above by using a refinement
type system for the \(\pi\)-calculus. Using the refinement type system,
we can infer that the value of \(x\) in the original process is less than \(n\),
so that we can refine the definition of \texttt{f} to:
\begin{verbatim}
 let rec f(n) = ... else (pred(n) [] let x=* in assume(x<n);f(x))
\end{verbatim}
The target program is now terminating, from which
we can deduce that the original process is also terminating.
We have implemented an automated tool based on the refined translation above.

The contributions of this paper are summarized as follows.
\begin{itemize}
\item The formalization of the basic translation from the \(\pi\)-calculus
  (extended with integers) to sequential programs, and a proof of its correctness.
\item The formalization of a refined translation based on a refinement type system.
\item An implementation of the refined translation, including automated refinement type
  inference based on CHC solving, and experiments to evaluate the effectiveness of
  our method.
\end{itemize}

The rest of this paper is structured as follows.
Section~\ref{sec:targetlanguage} introduces the source and target languages
of our translation.
Section~\ref{sec:approach} 
formalizes the basic translation, and proves its correctness.
Section~\ref{sec:refinement} refines the basic translation by using a refinement type system.
Section~\ref{sec:implementation} reports an implementation and experiments.
Section~\ref{sec:relatedwork} discusses related work,
and Section~\ref{sec:conclusion} concludes the paper.


\section{Switched System and Attack Model}
\label{sec:replay1}
\begin{figure}[b!]
\centering
\includegraphics [width=0.38\textwidth]{system6.pdf}
\vspace{-10pt}
\caption{Switching system diagram, where the system is equipped with $N_1$ controllers, $N_2$ estimators and $N_3$ detectors and switches among $N$ subsystems. A subsystem (controller $N_1$, estimator $N_2$, and detector $N_3$) is chosen here.} %attacks can change sensor measurements in this example.}
\label{system}
%\vspace{-10pt}
\end{figure}
%\FM{add: fault detection filter and detector here?}
We consider the CPS security problem when both the system and attacker have limited knowledge about the opponent. The system is equipped with multiple controllers/estimators/detectors, such that each combination of these components constitute a subsystem. A subsystem has a probability to detect specific types of attacks with different control and detection costs. To balance the security overhead and the control cost under various attacks, we consider switching among subsystems (choose a model for every component) according to the system dynamics and detector information. A switched system model is shown in Figure~\ref{system}, and the model of each component is described with a concrete example in the rest of this section. It is worth noting that the set of subsystems is not restricted and can be further generalized. %We assume that the controller/estimator/detector are not compromised, and attackers can not hack code implemented in these components in this work.

\textbf{LTI plant and sensor attack model}: Consider a class of LTI plants described by: % standard state-space form:
%
%\vspace{-3pt}
\begin{align}
\begin{split}
\mathbf{x}_{k+1}= \mathbf{Ax}_{k}+\mathbf{Bu}_{k}+\mathbf{w}_{k},\quad
\mathbf{y}_{k}= \mathbf{Cx}_{k}+\mathbf{v}_{k},
\end{split}
\label{system}
\end{align}
where $\mathbf{x}_{k} \in \mathbb{R}^{n}, \mathbf{u}_{k} \in \mathbb{R}^{p}$ and $\mathbf{y}_{k} \in \mathbb{R}^{m}$ denote the discrete time state, input and output vectors respectively, and $\mathbf{w}_{k}\sim \mathcal{N}(0,\mathbf{Q})$, $\mathbf{v}_{k}\sim  \mathcal{N}(0,\mathbf{R})$ are independent and identically distributed (IID) Gaussian random noise. The initial state is $\mathbf{x}_{0}\sim  \mathcal{N}(\bar{\mathbf{x}}_{0}, \Sigma)$. Sensors or the communication between sensors and estimators are vulnerable, and attacker can change values $\mathbf{y}_k$ that sent from sensors of system~\eqref{system}, and the compromised sensor measurements are defined as $\mathbf{y}'_k$ according to the types of attacks we consider. For instance, if the attacker can inject arbitrary data $\mathbf{y}^a_k$ to sensors, $\mathbf{y}'_k=\mathbf{y}_k+\mathbf{y}^a_k$; for replay attacks, the attacker can choose the replay window size $T_2$, let $\mathbf{y}'_{k} =\mathbf{y}_{k-T_2}$ and decide whether to send the delayed plant outputs at $k$. 

\textbf{Estimators}: % Kalman filter is widely applied for noisy systems. 
The physical dynamical state of the system is provided by an estimator, for instance, attack resilient estimator~(\cite{res_estimator}), $l_1$ norm state estimator~(\cite{resest_cdc15}), fault detection filter~\cite{fd_filter}, or the widely applied Kalman filter.
When $(\mathbf{A},\mathbf{B})$ is stabilizable, $(\mathbf{A},\mathbf{C})$ is detectable, a steady state Kalman filter exists. 



\textbf{Controllers}: A state feedback control law is described as $\mathbf{u}_k=L(\hat{x}_{k|k})$, where $L(\cdot)$ is a linear function, $\hat{x}_{k|k}$ is the estimated state. \cite{replay} increase the detection rate by adding an IID Gaussian signal $\Delta \mathbf{u}_{k} \sim \mathcal{N}(0, \mathcal{L})$ to $\mathbf{u}_k^*$ to an optimal LQG controller as $\mathbf{u}_{k} = \mathbf{u}^{*}_{k} + \Delta \mathbf{u}_{k} \label{nlqg}$, and increase the control cost. Then always applying the non-optimal controller for detecting a replay attack is not cost optimal, especially when there is no replay at all during a long time.

\textbf{Detectors}: We assume that every detector of the subsystem provides a detection rate for a specific type of attack, and a system is equipped with several detectors in order to deal with multiple types of attacks. Researchers have designed probabilistic detectors with respect to different attacks. For instance, \cite{fd_filter} design a fault detection filter, including a residual estimator and a threshold and a decision logic unit. Hypothesis testing strategies such as maximum likelihood (MLE), maximum a posteriori (MAP), and minimum mean square error (MMSE) account for GPS spoofing attack is presented by~\cite{GPS_spoof}. %By testing the statistical profile of a spoofing attack with methods  the probability and cost of detecting one spoofing attack are evaluated. 

%However, each method of hypothesis testing has the largest detection rate with respect to a different type of spoofing attack with some cost. We need to decide which detection strategy to use in order to secure the system with some probability and not cost too much. 

\textbf{Cyber state -- discrete modes of the system}:
We denote the modes of a vulnerable system as three constants $S=\{\delta_1,\delta_2, \delta_3\}$. State $\delta_{1}=safe$ describes that the system has already successfully detected an attack; $\delta_{2}= no~detection$ specifies that the alarm is not triggered; finally, the system enters state $\delta_{3}= false~alarm~trigger$ when the alarm is triggered while no attack has yet occurred. The mode depends on the probability detection rate. We assume that once the alarm is triggered, the system will stop the execution and decide whether to react to occurred attacks or it is a false alarm. %When the system is hijacked, the estimator, detector and controller are fed with false data, until an alarm is triggered and the system reacts to the attack.

%With different types of controllers or probabilistic detectors, it is necessary to introduce a new framework to balance the security overhead and system's performance. With system dynamics model, the cost of compromised by one type of attack and the payoff of detecting it or keeping resilient are also changing under different system states. A switching policy is then promising as a solution under this scenario. 

\iffalse
For the steady state Kalman filter, a $\chi^2$ detector triggers the alarm when the estimation residual is greater than the threshold with a specific false detection rate.
It is worth noting that even without the knowledge of system dynamics, a replay attacker can compromise values from all sensors. 
\fi

\iffalse
\vspace{-3pt}
\begin{align}
\begin{split}
\hat{\mathbf{x}}_{0|-1}&=\bar{\mathbf{x}}_{0}, \mathbf{P}_{0|-1}=\Sigma, \mathbf{P}_{k+1|k}=\mathbf{AP}_k\mathbf{A}^T+\mathbf{Q},\\
%\mathbf{P}_{k+1|k}&=\mathbf{AP}_{k|k}\mathbf{A}^{T}+\mathbf{Q},\mathbf{K}_{k}=\mathbf{P}_{k|k-1}\mathbf{C}^{T}(\mathbf{CP}_{k|k-1}\mathbf{C}^{T}+\mathbf{R})^{-1},\\
\mathbf{P}&=\lim\limits_{k\to\infty} \mathbf{P}_{k|k-1}, \mathbf{K}=\mathbf{PC}^{T}(\mathbf{CPC}^{T}+\mathbf{R})^{-1},\\
\hat{\mathbf{x}}_{k|k} &= \hat{\mathbf{x}}_{k|k-1}+\mathbf{K}(\mathbf{y}_{k}-\mathbf{C}\hat{\mathbf{x}}_{k|k-1}),\\
\hat{\mathbf{x}}_{k+1|k}&=\mathbf{A}\hat{\mathbf{x}}_{k|k}+\mathbf{Bu}_{k}.
%\mathbf{P}\triangleq &\lim_{k\to\infty}\mathbf{P}_{k|k-1}, \mathbf{K}=\mathbf{PC}^{T}(\mathbf{CPC}^{T}+\mathbf{R})^{-1}.
\end{split}
\label{eq:kf1}
\end{align}
\fi

\iffalse
\begin{align}
\begin{split}
\mathbf{y}'_{k}=\begin{cases} \mathbf{y}_{k},\ \ \text{sensor output is not changed at}\ k \\%\quad \quad \quad 0\le k \le T_{0}+T-1\\
                                  \mathbf{y}_{k-T_2},\ \text{replay attack occurs at}\ k.%\quad T_{0}+T\le k \le T_{0}+2T-1 
                                  \end{cases}
\end{split}
\label{replay_y}
\end{align}
For example, with Kalman Filter $\hat{\mathbf{x}}_{k|k}$ as a state estimator, an optimal LQG controller is described as $\mathbf{u}_k=\mathbf{ L \hat{x}}_{k|k}$. Here $L$ is a time invariant matrix satisfying $L\triangleq -(\mathbf{B}^{T}\mathbf{SB}+\mathbf{U})^{-1}\mathbf{B}^{T}\mathbf{SA}$, and $\mathbf{S}$ is the solution of the Riccati equation 
%\begin{equation*}
$\mathbf{S}=\mathbf{A}^{T}\mathbf{SA}+\mathbf{W}-\mathbf{A}^{T}\mathbf{SB}(\mathbf{B}^{T}\mathbf{SB}+\mathbf{U})^{-1}\mathbf{B}^{T}\mathbf{SA},$
%\end{equation*}
for some $\mathbf{W,U} \succ 0$ of corresponding dimensions.
\fi

%. Replay attack is undetectable for a subsystem with optimal LQG control, Kalman filter and $\chi^2$ detector. The expectation of residual signal will increase under replay attack, the quadratic cost also increase to $J'= J+trace{[(\mathbf{U}+\mathbf{B}^{T}\mathbf{SB})\mathcal{L}]}$,where $J$ denotes the cost for optimal LQG control input $\mathbf{u}^{*}_{k}$~(\cite{replay}). The probability of having a residual that greater than the alarm threshold value for a normal $\chi^2$ detector is also increased under replay attacks.

\section{A Hybrid Stochastic Game Model}
\label{sec:game_form}
To obtain a switching policy that minimizes the expected real-time worst case payoff for the given subsystems, 
we formulate a zero-sum, hybrid stochastic game between the system and the attacker. System dynamics knowledge are combined with the game definition, and the quantitative process for the game parameters will be introduced in this section. We assume that one game stage $k$ is also one time step of the physical system. The total stage number is $K$. The hybrid game state space $(X_{[k-T,k]}\times S)$ contains information about both the system dynamics $\mathbf{x}_k$ and the discrete modes $\delta_l, l=1,2,3$. Here, $T$ is the window size of system dynamics needed to keep the state transition between stages $k$ and $(k+1)$ Markov. The joint state includes information we need to compute the game strategy at the current stage. This is the main difference compared with the previous work~(\cite{cdc_replay}), while the latter is not Markov since it needs to consider all the possible histories of strategies for deciding the physical dynamics and getting a strategy. At each stage $k \in \{T,\cdots, K+T\}$, parameters include the action space for the attacker (system) $A_{t}$ ($A_{s}$), the state transition probability matrix $\mathbb{P}_{k}$, and the immediate payoff matrix $r_{k}$. The solution set of the game is mixed strategies $\mathbf{F}_{k}$ for the attacker, and $\mathbf{G}_{k}$ for the system. Formally, the game is defined as a sequence of tuples:
 $\{(X_{[k-T,k]} \times S),A_{t},A_{s}, \mathbf{F}_{k},\mathbf{G}_{k}, P, r\}$.

\iffalse
\begin{table*}
\centering
%\caption{Parameters of the hybrid stochastic game between the system and the attacker}
\begin{tabular}{|c|c|}
  \hline
   $s_{kl}=(x_{[k-T,k]}, \delta_l)$& Joint game state: sequence of physical dynamics, and piecewise constant mode \\ \hline
   $A_{t}$ & Attacker's action space \\ \hline
   $A_{s}$ & System's action space \\ \hline  
   $\mathbf{f}_k(s_{kl})$ & Strategy of the attacker at stage $k$, 
                                           state $s_{kl}, l=1,2,3$  \\ \hline     
      $\mathbf{g}_k(s_{kl})$ & Strategy of the system at stage $k$, 
                                            state $s_{kl}, l=1,2,3$  \\ \hline     
    ${P}(s_{(k+1)h}|s_{kl})$ & Probability that system transits from state $s_{kl}$ 
                                              at stage $k$ to state $s_{(k+1)l}$ at stage $k+1$\\ \hline
    ${r}(s_{kl})$ & Immediate payoff matrix at stage $k$ \\ 
   \hline
\end{tabular}
\centering
%\captionsetup{justification=centerlast}
\caption{Parameters of the hybrid stochastic game between the system and the attacker}
\label{game_parameter}
\end{table*} 
\fi

\textbf{Game State Space}: The joint state of the system at stage $k$ is described by the pair $s_{kl}=(x_{[k-T,k]}, \delta_l)$, where
\centerline{$
x_{[k-T,k]}=(x_{k-T}, x_{k-T+1}, \cdots, x_k ) \in X_{[k-T,k]}$} is the discrete-time dynamics of the physical process provided to the system--the state estimations $\hat{x}_{k-T},\cdots, \hat{x}_k$, $\delta_l \in S=\{\delta_1,\delta_2, \delta_3\}$ denote the cyber state of the system. We assume that once the game reach $\delta_1$, the system wins and will not enter other modes till next game, i.e., $\delta_1$ is an absorbing state. The moving-horizon transition of the joint states on stage axis is shown as Figure~\ref{sg}. The window size of system dynamics $T$ keeps the state transition between time $k$ and $k+1$ Markov. For instance, if the detector of the system requires system dynamics $\hat{x}_{[k-T_1,k]}$, and we consider sensor data injection attacks and replay attacks with replay windows less than $T_2$ steps, then $T=max\{T_1, T_2\}$. 
%With the system dynamics $\hat{x}_{[k-T, k]}$ denoted as $x_{[k-T,k]}$ for game stage $k$,  information needed to define the following action space of two players, the payoff and state transition probability is included. 

%i.e., the detector needs information for T steps to decide whether the alarm should be triggered. 
%\FM{In replay attack, we can say: once alarm is triggered, the system can stop the execution and check whether attack occurred, is this true for other attacks? Can the system distinguish between successfully detection and false alarm trigger?}
\begin{figure}[t!]
%\vspace{-5pt}
\centering
\includegraphics [width=0.32\textwidth]{xk.pdf}
\vspace{-8pt}
\caption{Joint state transition of the hybrid stochastic game when moving the horizon of game state one step ahead. When the state transits from stage $k$ to $k+1$, we slice the window of the sequence of physical dynamics one step ahead, add $x_{k+1}$ and remove $x_{k-T}$,  thus $x_{[k-T,k]} \to x_{[k-T+1,k+1]}$. The piecewise constant modes $\delta_l$, $\delta_h$ describe the cyber states provided by the detector at stage $k$, respectively.}
\label{sg}
\vspace{-5pt}
\end{figure}

%For simplicity, we omit the subscript k and just write state at every time k as $s_i,  i=1,2,3$.
\textbf{Attacker's Action Space}: We assume that the system is vulnerable to different attack models described by the action space $A_{t}$, where 
\\\centerline{$
A_{t}=\{a_{1}(x_{[k-T,k]}), a_{2}(x_{[k-T,k]}), \cdots, a_{M}(x_{[k-T,k]})\}
$}
is the attacker's action space at stage $k$, and $a_{1}$ means no attack. Here we only consider discretized action space of the attacker for computational efficiency. For the LTI system dynamics considered in this work, the distance of a continuous point to its nearest discrete point in action space is bounded. With bounded error of the dynamics by discretized continuous action space, the quality of game solutions under different conditions is analyzed by work~\cite{disaction}. 

The actions can describe both multiple types of attacks and the same type attack with different values. For instance, when considering only sensor data injection attacks with different norms of injection value, we will denote $a_i (x_{[k-T,k]}), i=2, 3,\dots$ as changing the sensor value from $\mathbf{y}_k=\mathbf{Cx}_k+\mathbf{v}_k$ to $\mathbf{y}'_k= \mathbf{y}_k+\mathbf{y}_{k,i}^a$, where any injection $\mathbf{y}_k^a$ is classified as $a_i (x_{[k-T,k]}), i=inf\{i:\ \|\mathbf{y}_k^a-\mathbf{y}_{k,i}^a\|_2\}$ in attacker's action space.
Similarly, for replay attack only, the action space is discretized as changing sensor values from $\mathbf{y}_k=\mathbf{Cx}_k+\mathbf{v}_k$ to $\mathbf{y}'_k= \mathbf{y}_{k-T_i}$ for action index $a_i (x_{[k-T,k]})$, where any replay time length $T_a$ is classified as $a_i (x_{[k-T,k]}), i=inf\{i:|T_a - T_i|\}$. Considering multiple types of attacks, we assume that the system is valnerable under $m_a$ types of attacks, and attack type $A_i$ is corresponding to $M_{a,i}$ discretized actions in the action space, then there are $\sum_{i=1}^{m_a} M_{a,i}+1$ actions in total within the attacker's action space $A_t$.
 

% with discretized, bounded norm $\|\mathbf{y}_{k,i}^a\|_2 \leqslant b$, since the attacker has limited energy for every data injection. This means any injection data that satisfies $\|\mathbf{y}_{k}^a\|_2 \leqslant b$ is considered as $inf\{i:\ \|\mathbf{y}_k^a-\mathbf{y}_{k,i}^a\|_2\}$ in attacker's action space. 

%For example, 
%when considering replay attacks and false data injection attacks, we take $a_{2k}$ as~\eqref{replay_y} for a given replay window size, and $a_{3k}$ as~\eqref{attackmodel} for a given data injection range. 

\iffalse by any given controller/estimator/detector combination of the system \fi
%; here, $y_{k}$ is the real sensor value and $\mathbf{y}_{k-t_{i}}$ denotes any replay sensor value in the strategy set. %\FM{here we assume during time $k \in \{1,...,K\}$ the replay window size $T$ does not change for simulation.}
%\item
%\FM{rewrite the action space definition}

\textbf{System's Action Space}: The system's action space at stage $k$ is defined as
\\\centerline{$
A_{s}=\{u_{1}(x_{[k-T,k]}), u_{2}(x_{[k-T,k]}),\cdots, u_{N}(x_{[k-T,k]})\},
$}  %= \{\mathbf{u}^{*}_{k}, \mathbf{u}^{*}_{k} + \Delta \mathbf{u}_{k}\}$
where $u_{j}$ is the index for the $j$th subsystem. We assume that the $N$ subsystems (a model for each component in Figure~\ref{system}) are determined priorly. For example, a subsystem can be the plant with a given optimal LQG controller, a Kalman filter and a $\chi^2$ detector. A subsystem can also be the plant with an optimal LQG controller, a resilient state estimator~\cite{res_estimator} and its corresponding estimation residual checking component. We assume that the attacker's action space is defined, with corresponding system's action or a subsystem that the detection rate is greater than $0$. A switched system does not ensure performance under the attack outside the action space of the game.

\textbf{Mixed Strategy}: Let $f^{i}_{k}(s_{kl})$ ($g^{j}_{k}(s_{kl})$) be the probability that the attacker (system) chooses action $a_{i}(x_{[k-T,k}) \in A_{t}$ ($u_{j}(x_{[k-T,k}) \in A_{s}$) at state $s_{kl}\in (X_{[k-T,k]}\times S)$. Define $\mathbf{F}_{k}$ and $\mathbf{G}_{k}$ as the mixed strategy sets of the attacker and the system for stage $k$:
$\mathbf{F}_{k} :=\{\mathbf{f}_{k}= [\mathbf{f}_{k}(s_{k1}), \mathbf{f}_{k}(s_{k2}), \mathbf{f}_{k}(s_{k3})]
|f_{k}^{i}(s_{kl})\geq 0, \mathbf{f}_k \in [0,1]^{M\times 3},
\sum \limits_{a_{ik} \in A_{tk}}f_{k}^{i}(s_{kl}) = 1,\mathbf{f}_{k}(s_{kl})\in \mathbb{R}^{M}, \forall s_{kl} \in(X_{[k-T,k]}\times S)\},$
$\mathbf{G}_{k}:=\{\mathbf{g}_{k}= [\mathbf{g}_{k}(s_{k1}), \mathbf{g}_{k}(s_{k2}), \mathbf{g}_{k}(s_{k3})]|$
$g_{k}^{j}(s_{kl})\geq 0, \mathbf{g}_k \in [0,1]^{N \times 3}, %&\forall u_{jk} \in A_{sk},%k \in \{1,...,K\},\\
\sum \limits_{u_{jk} \in A_{sk}}g_{k}^{j}(s_{kl}) = 1, \mathbf{g}_{k}(s_{kl}) \in \mathbb{R}^{N}, \forall s_{kl} \in (X_{[k-T,k]}\times S)\}. $ Note that $\mathbf{x}_{[k-T,k]}$ provides exogenous information for the strategy $\mathbf{f}_k (\mathbf{g}_k)$, since for every $l$, $\mathbf{f}_{k}(s_{kl}) (\mathbf{g}_{k}(s_{kl}))$ is the strategy at mode $\delta_l$ for the same $\mathbf{x}_{[k-T,k]}$ at stage $k$. Hence, $\mathbf{g}_k$ and $\mathbf{f}_k$ are finite dimensional vectors, that the stationary strategy chosen by each player at stage $k$ depends on the cyber state. %Mixed strategy set $\mathbf{F}_k$ also include the case that the attacker only implement one specific type of attack in the action space at time instance $k$, since we do not have know the strategy of the attacker, we are able to consider all possible combinations of attacks by exploiting mixed strategies. 

%\textbf{}:
%\label{dynamicgame}
%With all the above definition, %for any strategy history $h_{k}$ (a sequence of switching policy), 


\textbf{System and Subsystem Dynamics under game framework}: Given the subsystem and attack models in Section~\ref{sec:replay1} and the game definition,  
%we can transform it to a new system dynamic model decided by both players' actions. 
we show the dynamics at stage $k$ given an action pair $(a_{i}(x_{[k-T,k}),u_{j}(x_{[k-T,k}))$ (assume initial $\mathbf{\hat{x}}_{1|0}=\bar{\mathbf{x}}_{0}$, $\mathbf{x}_{1}=\mathbf{x}_{0}$). Each action pair $(a_{i}(x_{[k-T,k]}),u_{j}(x_{[k-T,k]}))$ defines the corresponding system dynamics at $k$. For instance, when we focus on sensor attacks (like replay or false data injection), let $\mathbf{\gamma}_{k}(a_{i}(x_{[k-T,k]}), u_{j}(x_{[k-T,k]}))$ be the control input with $(a_{i}(x_{[k-T,k]}),u_{j}(x_{[k-T,k]}))$, a subsystem $u_{j}(x_{[k-T,k]})$ with a Kalman filter, an optimal LQG controller has the following dynamics (we denote $(a_{i}(x_{[k-T,k]}),u_{j}(x_{[k-T,k]}))$ as $(a_{ik}, u_{jk})$ for convenience): 
\begin{align}
\begin{split}
&\mathbf{x}_{k}=\mathbf{Ax}_{k-1}+ \mathbf{Bu}_{k-1}+\mathbf{w}_{k-1},\\
& \mathbf{y}_{k}=\begin{cases}a_{1k} = \mathbf{Cx}_k+\mathbf{v}_{k},\ \text{without attack}\\
a_{ik}, i=2,\cdots, M, \ \ \text{with attack,} \end{cases}\\
&\hat{\mathbf{x}}_{k|k-1}= \mathbf{A\hat{x}}_{k-1|k-1}+\mathbf{Bu}_{k-1},\\
%&\mathbf{z}_{k+1}(h_{k},a_{ik},u_{jk})=a_{ik}(h_{k}) - \mathbf{C\hat{x}}_{k+1|k}(h_{k},a_{ik},u_{jk}),\\
&\hat{\mathbf{x}}_{k|k}(a_{ik}) =\hat{\mathbf{x}}_{k|k-1}+ \mathbf{K}(a_{ik} - \mathbf{C\hat{x}}_{k|k-1}),\\
%\end{split}
&\mathbf{\hat{x}}_{k+1|k}(a_{ik},u_{jk})=\mathbf{A\hat{x}}_{k|k}(a_{ik})+\mathbf{B\gamma}_{k}(a_{ik},u_{jk}),\\
& \mathbf{\gamma}_{k}(a_{ik}, u_{jk}) = \mathbf{L\hat{x}}_{k|k} (a_{ik}),\\%+\Delta \mathbf{u}_{k},\\
&\mathbf{z}_{k+1}(a_{ik},u_{jk})=a_{ik} - \mathbf{C\hat{x}}_{k+1|k}(a_{ik},u_{jk}).
\label{dynamicgame}
\end{split}
\end{align}
\textbf{State Transition Probability}: Given a set of subsystem models, define the state transition probability $P$ as a function of the state of the game and both players' actions $P:\ (X_{[k-T,k]}\times S) \times A_{t} \times A_{s}\to [0, 1],$
where
\\\centerline{$
P(s_{(k+1)h}|s_{kl},a_{ik}, u_{jk}), h=1,2,3
$}
%\end{align*}
is the probability that system transits from state $s_{kl}$ to state $s_{(k+1)h}$ at stage $k+1$, given both players' action $(a_{ik},u_{jk})$ at stage $k$. Given the current game state $s_{kl}=(x_{[k-T,k]}, \delta_l)$ and an action pair $(a_{ik},u_{jk})$, the dynamics of the system at stage $k+1$ is described as $x_{[k-T+1,k+1]}$ for all possible cyber modes $\delta_h \in S$, hence the dimension of state transition probability $P(s_{(k+1)h}|s_{kl},a_{ik}, u_{jk})$ is determined by the number of cyber modes of the game. We denote $P(s_{(k+1)h}|s_{kl}, a_{ik}, u_{jk})$ as $P^{ij}(s_{(k+1)h}|s_{kl})$ for short.
%and $\tilde{P}^{ij}(s_{(k+1)h}|s_{kl})$ is the entry at the $i$-th row and $j$-the column  of the state transition matrix $\tilde{P}(s_{(k+1)h}|s_{kl})$ of the game at hybrid state $s_{kl}$.
 As a state transition probability, this function should also satisfy
%\begin{align*}
\\\centerline{$\sum_{\delta_h \in S} {P}^{ij}(s_{(k+1)h}|s_{kl}) = 1,\quad \forall (a_{ik},u_{jk}) \in A_{t} \times A_{s},$}
\\\centerline{$s_{(k+1)h} \in (X_{[k-T+1,k+1]}\times S), s_{kl} \in(X_{[k-T,k]}\times S).$}
%\end{align*}
The transition probability is provided by intrusion detectors of the subsystem. 
%For computational efficiency, we assume that every element of the state transition matrix is a convex function of the system dynamics $x_{[k-T,k]}$ or can be convexified with bounded error. 
%For example, if a $\chi^{2}$ detector is the detector component of subsystem $u_{j}$, we apply~\eqref{alarm} to decide the state transition probability.

\textbf{Immediate Payoff Function}: The immediate payoff matrix at stage $k$ is a $\mathbb{R}^{M\times N}$ matrix for given game state and every action pair $(a_{ik}, u_{jk})$. We define the immediate payoff function as a continuous, convex function of the hybrid game state and the actions of both players
\\\centerline{$r: (X_{[k-T,k]}\times S) \times A_{t} \times A_{s} \to \mathbb{R}^{M \times N},$}
where $r(s_{kl}, a_{ik}, u_{jk}) \geqslant 0$ is the payoff at joint state $s_{kl}$ given action pair $(a_{ik}, u_{jk})$. For definition convenience, we denote ${r}(s_{kl}, a_{ik}, u_{jk})$ as ${r}^{ij}(s_{kl})$ for short, since it is the element on the $i$-th row and $j$-th column of the payoff matrix ${r}(s_{kl})$. It is a zero-sum game between the system and the attacker, and we assume the system is the minimizer and the attacker is the maximizer, hence the payoff function for the attacker and the system is defined as
\centerline{$
{r}^{ij}(s_{kl})={r}_t^{ij}(s_{kl})=-{r}_s^{ij}(s_{kl}).
$}
For instance, when the linear quadratic cost is a metric of system performance, let $\gamma_{k}(a_{ik}, u_{jk})$ be the control input given action pair $(a_{ik}, u_{jk})$, then the payoff function is defined as
\begin{align}
\begin{split}
{r}^{ij} (s_{k1}) =&\mathbb{E}[\mathbf{\hat{x}}^{T}_{k}]\mathbf{W}\mathbb{E}[\mathbf{\hat{x}}_{k}]+\mathbb{E}[\mathbf{\gamma}^{T}_{k}(a_{1k},u_{jk})]\mathbf{U}\mathbb{E}[\mathbf{\gamma}_{k}(a_{1k},u_{jk})],\\
{r}^{ij} (s_{k2}) =&\mathbb{E}[\mathbf{\hat{x}}^{T}_{k}]\mathbf{W}\mathbb{E}[\mathbf{\hat{x}}_{k}]+\mathbb{E}[\mathbf{\gamma}^{T}_{k}(a_{ik},u_{jk})]\mathbf{U}\mathbb{E}[\mathbf{\gamma}_{k}(a_{ik}, u_{jk})],\\
{r}^{ij} (s_{k3}) =& p_f,
\end{split}
\label{payoff}
\end{align}
where $p_f$ is the false alarm trigger penalty, the cost that the system needs to stop execution, check the reason of an alarm, and restart later; $\mathbf{x}_{k}$ is the physical state under the game framework. At mode $\delta_{1}$ the system wins, so the payoff is a normal system payoff with correct sensor data. The larger $p_f$ is, the less probable it is for the system to choose a strategy to transit to state $s_{k3}$.

\textbf{System dynamics update with strategies at stage k}:
 Let $p(s_{kl})$ be the probability system is at state $s_{kl}$ at stage $k$. The initial state distribution $p(s_{1l})$ is given. With  a strategy $\mathbf{f}_{k},\mathbf{g}_{k}$, the attacker and the system randomly sample an action pair $(a_{ik}, u_{jk})$ according to the probability distribution. Then, the control input and sensor value for calculating expectation cost are: 
%\begin{align*}
\centerline{$
\mathbf{u}_{k}=\sum\limits_{j=1}^{N}\sum\limits_{i=1}^{M} \sum\limits_{l=1}^{3}p(s_{kl})f_{k}^{i}(s_{kl})g^{j}_{k}(s_{kl})\mathbf{\gamma}_{k}(a_{ik},u_{jk}),
$}
$\text{ }\quad\quad\mathbf{y}_{k} =\sum\limits_{i=1}^{M}\sum\limits_{l=1}^{3} p(s_{kl})f_{k}^{i}(s_{kl}) a_{ik}.$
%\end{align*}
\\The probability that system is at state $s_{(k+1)h}$ for $k+1$ is:
\\\centerline{$
%\begin{align*}
p(s_{(k+1)h})= \sum\limits_{l=1}^{3}p(s_{kl})[\mathbf{f}_{k}(s_{kl})]^{T}{P}_{k}(s_{(k+1)h}|s_{kl})\mathbf{g}_{k}(s_{kl}). 
%\end{align*}
$}

\iffalse
\begin{align*}
\mathbf{F}_{k} :=\{&\mathbf{f}_{k}= [\mathbf{f}_{k}(s_{k1}), \mathbf{f}_{k}(s_{k2}), \mathbf{f}_{k}(s_{k3})]
|f_{k}^{i}(s_{kl})\geq 0,\\& \mathbf{f}_k \in [0,1]^{M\times 3} %&\forall a_{ik} \in A_{tk},  %k\in \{1,...,K\},\\
\sum \limits_{a_{ik} \in A_{tk}}f_{k}^{i}(s_{kl}) = 1,\mathbf{f}_{k}(s_{kl})\in \mathbb{R}^{M},\\&\forall s_{kl} \in(X_{[k-T,k]}\times S)\},\\
\mathbf{G}_{k}:=\{&\mathbf{g}_{k}= [\mathbf{g}_{k}(s_{k1}), \mathbf{g}_{k}(s_{k2}), \mathbf{g}_{k}(s_{k3})]|
g_{k}^{j}(s_{kl})\geq 0,\\& \mathbf{g}_k \in [0,1]^{N \times 3}, %&\forall u_{jk} \in A_{sk},%k \in \{1,...,K\},\\
\sum \limits_{u_{jk} \in A_{sk}}g_{k}^{j}(s_{kl}) = 1, \mathbf{g}_{k}(s_{kl}) \in \mathbb{R}^{N},\\ &\forall s_{kl} \in (X_{[k-T,k]}\times S)\}. 
\end{align*} 
\fi

\section{Existence of An Optimal Strategy and Suboptimal Algorithm for A Finite Game}
\label{sec:algorithm_finite}

Based on the game formulation, in this section we discuss the existence of an optimal solution for the finite form of the hybrid stochastic game, and present an algorithm to compute a suboptimal system strategy.%\PA{what do you mean by ``approximation computing algorithm''}
 
\subsection{Existence of the System's Optimal Strategy}

%\iffalse 
We define the concatenation of strategies for $K$-stage game of each player ($\mathbf{f}$ for attacker and $\mathbf{g}$ for system) as $\mathbf{f}=\mathbf{f}_1\cdots \mathbf{f}_K,\quad\mathbf{f}_{k} \in \mathbf{F}_{k},\quad\mathbf{f}\in \mathbf{F},$
$\mathbf{g}=\mathbf{g}_{1}\cdots\mathbf{g}_K,\quad\mathbf{g}_{k} \in \mathbf{G}_{k},\quad\mathbf{g}\in\mathbf{G}$,
$k=1, 2,\dots,K$.
\begin{defn}
Let the random variable $\zeta_{k}$ describe the discrete state of the hybrid game at stage $k$, we define the conditional expected total payoff till $\tilde{K}$ for any $\mathbf{f},\mathbf{g}$ as%\begin{align*}
\\$\text{ }\quad R_{\tilde{K}}(s, \mathbf{f}, \mathbf{g})$
\\\centerline{$
=\sum \limits^{\tilde{K}}_{k=1}\sum_{l=1}^{3} p(\zeta_{k}=\delta_{l}|\zeta_{1} = s)[\mathbf{f}_{k}(s_{kl})]^{T}\tilde{r}_{k}(s_{kl})\mathbf{g}_{k}(s_{kl}),
$}
where $p(\zeta_{k}=\delta_{l}|\zeta_{1} = s)$ is the probability that the discrete state of the hybrid game is $\delta_{l}$ at stage $k$ given its initial discrete state $\zeta_{1}=s$.
\label{R_K}
\end{defn}
Since the immediate payoff of each stage satisfies that $0 \leq \tilde{r}_{k}^{ij}(s_{kl})< \infty,\ \text{for all}\ k,i,j,$
we have that $R_{\tilde{K}}(s, \mathbf{f}, \mathbf{g})$ is a nonnegative real-valued, nondecreasing function with $\tilde{K}$. 
%\PA{when you cut an old sentence to save space, make sure that the new one makes sense; for example you should have said - from the immediate payoff definition we have that ...' or ``since...''}  
Furthermore, for finite $K$%a finite stage game
%\PA{shouldn't it be ``for a finite game'' or ``for a finite stage game''}
%\vspace{-4pt}
\begin{align}
%\sup_{\mathbf{f} \in F^{K}}R_{K}(s,\mathbf{f}, \mathbf{g}^{*}) < \infty,s \in S.
%\sup_{\mathbf{f}}
R_{K}(s,\mathbf{f}, \mathbf{g}) < \infty, \forall s \in S, \mathbf{f}\in \mathbf{F}, \mathbf{g} \in \mathbf{G}.
\label{fR}
\end{align}
Similarly as the definition of value and optimal strategy for a zero-sum, finite discrete state, finite stage stochastic game, we define the value and optimal strategy for the hybrid state stochastic game defined in this work as the following.
\begin{defn}
A two-person zero-sum $K$-stage stochastic game is said to have a value vector $v^{*}_{K}$ if $v^{*}_{K,s}=\underbar{v}_{K,s}=\bar{v}_{K,s},$ for any initial cyber state $s\in S$, where
\\\centerline{$
\underbar{v}_{K,s}= \sup_{\mathbf{f}\in\mathbf{F}}\inf_{\mathbf{g}\in\mathbf{G}}R_{K}(s,\mathbf{f}, \mathbf{g}),
$}
\centerline{$
\bar{v}_{K,s}=\inf_{\mathbf{g}\in\mathbf{G}}\sup_{\mathbf{f}\in\mathbf{F}}R_{K}(s,\mathbf{f},\mathbf{g}).
$}
For the finite value $K$-stage stochastic game, strategies $\mathbf{g}^{*}$ and $\mathbf{f}^{*}$ are called optimal at the saddle-point equilibrium for player two (the system) and player one (the attacker), respectively, if for all $s\in S$,
\centerline{$
v^*_{K,s} = \inf\limits_{\mathbf{g}\in \mathbf{G}}R_{K}(s, \mathbf{f}^{*}, \mathbf{g}),\quad v^*_{K,s} = \sup\limits_{\mathbf{f}\in \mathbf{F}}R_{K}(s, \mathbf{f}, \mathbf{g}^{*}).
$}
\end{defn}
%The existence conditions of the value and optimal strategies for a general finite horizon, finite state, zero-sum stochastic game are. 
The game defined in this paper has finite action spaces, finite strategy space, finite discrete cyber modes and satisfies~\eqref{fR} with bounded total payoff in finite horizon. Therefore, there exists the value of the considered  game and an saddle-point equilibrium or optimal strategy for the system shown in~\cite{dgt_Basar}.
\subsection{Suboptimal algorithm for the finite game}
Existing value iterative algorithms or dynamic programming algorithms for finite stochastic games cannot be used to solve the finite hybrid stochastic game defined in this work, since the discrete time dynamics $x_{[k-T,k]}$ of the game at stage $k$ depends on that of the stage $k-1$, which is only available in the future algorithm iterations. Hence, we design a suboptimal algorithm based on the value iteration method for a finite horizon, finite discrete state stochastic game~(\cite{plangame}) and robust game techniques~(\cite{RGT}). The value iteration algorithm for a finite horizon, discrete state stochastic game (with fixed payoff $r$ and state transition probability $P$ at every stage) works in the way that if a player knew how to play in the game optimally from the next stage on, then, at the current stage, he would play with such strategies. The value of $K$-stage game is finally provided by the last step of iteration. 



For a multi-stage game, to calculate the game value, we define the auxiliary matrix at stage $k$ for every cyber state $\delta_l$ with system dynamics $x_{[k-T,k]}$ as $Q(s_{kl}) \in \mathbb{Q}_{k} \subset \mathbb{R}^{M \times N}$, and each element of $Q(s_{kl})$ for action pair $(a_{ik}, u_{jk})$ is defined as
\begin{align}
\begin{split}
&Q^{ij}(s_{kl})\\
=&r^{ij}(s_{kl})
                         +\sum_{\delta_h\in S} {P}^{ij}(s_{(k+1)h} |s_{kl})\cdot {v}_{k+1}(s_{(k+1)h}), 
\end{split}
\label{Q_k}
\end{align} 
where ${v}_{k+1}(s_{(k+1)h})$ is the game value from stage $k+1$, state $s_{(k+1)h}$ (with cyber mode $\delta_h$) to the final stage $K$. For the final stage $K$, we define $Q(s_{Kl})=r(s_{Kl})$. We define a one-shot game at stage $k$ as a finite action space, zero-sum game between the system and the attacker with payoff matrix $Q(s_{kl})$, i.e., $Q^{ij}(s_{kl})$ is the payoff for action pair $(a_{ik},u_{jk})$ of stage $k$. In each one-shot game, the system only consider a strategy $f_k(s_{kl})$ to minimize the worst case payoff caused by the attacker according to matrix $Q(s_{kl})$. Here $Q(s_{kl})$ is defined based on the the system dynamics and the state transition probability provided by the detector. An alternative algorithm with unknown transition matrix or payoffs will be our future work.

Similarly as the value iteration algorithm for a discrete state stochastic game~(\cite{plangame}), Algorithm~\ref{finite} of the finite hybrid state stochastic game starts from the last stage, then gets the optimal one-stage strategy and the upper bound of game value at each stage. By calculating values of all stages until backwards to the first stage, Algorithm~\ref{finite} returns an upper bound for the value of the total payoff in $K$-stages. 

To estimate the values at each step, we consider the immediate payoff $r(s_{kl})$, the state transition probability ${P}(s_{(k+1)h}|s_{kl})$ and the game value estimated at the previous step uncertain parameters for the one shot robust game (\cite{RGT}). Then approximate each iteration value as the value of the robust one shot zero sum game. 
Algorithm~\ref{finite} provides an upper bound for the game value and the corresponding suboptimal strategy for the system. The idea is to solve a robust game at each iteration step -- i.e., minimize the worst-case caused by extreme points of the set of auxiliary matrix $\mathbb{Q}_k$ defined for all possible dynamics $x_{[k-T,k]}$. 


To quantify the boundary of the set of auxiliary matrix $\mathbb{Q}_k$ %($r_{k}$, $\mathbb{P}_{k}$), 
we need the expected values of system dynamics $\mathbf{x}_{k}, \mathbf{u}_{k}$, $\mathbf{y}_{k}, k=1,\cdots,K$ defined in equations~\eqref{dynamicgame}, which is determined by the strategies from stage $1$ till stage $k$. 
We first analyze the uncertain sets of the immediate payoff function at stage $k$, and the extreme points for the uncertain set $\mathbb{Q}_k$ depend on pure strategies . Let $\mathbf{f}^p_{k-1}$, $\mathbf{g}^p_{k-1}$ be the concatenation of previous pure strategies of the attacker and the system till stage $k \geqslant 2$, respectively, where
\\\centerline{
$\mathbf{f}^{p}_{k-1}=\mathbf{f}^{p}_{1}\cdots\mathbf{f}^{p}_{k-1},\quad  \mathbf{g}^{p}_{k-1}=\mathbf{g}^{p}_{1}\cdots\mathbf{g}^{p}_{k-1}$
}
satisfies that all $\mathbf{f}^{p}_{t}(s)$ ($\mathbf{g}^{p}_{t}(s)$) for $t=1,2,\dots, k$ have only one non-zero element, i.e., the player chooses the corresponding action or the \emph{pure} strategy.

Define a pure strategy auxiliary matrix $Q^p(s_{kl}) \in \mathbb{Q}^p_{k}$ as:
\begin{align}
\begin{split}
%\vspace{-5pt}
&Q^{p}(s_{kl}) \\
=&r^{p}(s_{kl})+\sum_{\delta_h\in S} {P}^{p} ( s_{(k+1)h} |s_{kl})\cdot \bar{v}^p_{k+1}(s_{(k+1)h}),
\label{eq:Q}
\end{split}
\end{align}
%\normalsize
for stages $k=1,\dots,K-1$, and for the final stage $k=K$,
\begin{align}
Q^{p}(s_{Kl})=r^{p}(s_{Kl}).
\label{Q_p}
\end{align}
For each stage $k$, $\bar{v}^p_{k}(s_{kl})$ is defined as
% and relates to matrix games defined by $\mathbb{Q}_{(k+1)p}$.
\begin{align}
\bar{v}_{k}^{p}(s_{kl})=\max_{Q^{p}(s_{kl})\in \mathbb{Q}^p_{k}}v^*[Q^{p}(s_{kl})],%v_{k}^{s_{l}}(h_{k}^{p})&=\max_{(r^{p}_{k}(h^{p}_{k},s_{l}), P^{p}_{k}(h^{p}_{k},s_{l}))}
%v^*[r^{p}_{k}(h^{k}_{p},s_{l})+\sum_{s'\in S} P^{p}_{k}( s' |h^{p}_{k},s_{l})v_{k+1}^{s'}(h^{p}_{k+1})],
\label{pickv}
\end{align}
where $v^*$ is the function that yields the value of a zero-sum matrix game. Then the value $\bar{v}^p_{k+1}(s_{(k+1)h}) \geq 0$ to calculate the auxiliary matrix~\ref{eq:Q} is the upper bound of robust game value from stage $k+1$ till stage $K$, resulting from the iteration at stage $k+1$. This value iteration process is the key idea of the following~Algorithm~\ref{finite}. 

\begin{alg}
%\vspace{-5pt}
\textbf{: Suboptimal Algorithm for A Finite Hybrid Stochastic Game}\\
\textbf{Input}: System model parameters and game parameters.
\\\textbf{Initialization}:
             Compute the set of $\mathbb{Q}^p_k$ for every stage $k=T,\dots,T+ K$ given $\hat{\mathbf{x}}_{[0,T]}$;
              %For all $s_{l} \in S, l =1,2,3, h^{p}_{K}\in H^{p}_{K} :$ get %$4^{K-1}$ 
              %backup matrix set $\{Q_{k}^{p}(h_{k}^{p},s_{l})\}$:
             get the robust game value and corresponding strategies at stage $K$: $Q^{p}(s_{(K+T)l})=r^{p}(s_{(K+T)l})$,
                    $f^{*}(s_{(K+T)l}), g^{*}(s_{(K+T)l}), \bar{v}_{K+T}^{p}(s_{(K+T)l}) \leftarrow \pi(Q^{p}(s_{(K+T)l})).$
\\\textbf{Iteration}: For $k=(K+T-1), \cdots, T$, obtain a set of auxiliary matrices $\mathbb{Q}_{k}^{p}$ for all $\mathbf{f}^{p}_{k}$, $\mathbf{g}^p_k$, where each matrix is defined in~\eqref{eq:Q}, then calculate:
%\[Q_{k}^{p}(h^{p}_{k},s_{l}) = r^{p}_{k}(h^{p}_{k},s_{l})+\sum_{s'\in S} P^{p}_{k}( s' |h^{p}_{k},s_{l})v_{k+1}^{s'}(h^{p}_{k+1}),\]

 $f^{*}(s_{kl}), g^{*}(s_{kl}),\bar{v}_{k}^{p}(s_{kl}) \leftarrow \pi(Q^{p}(s_{kl}))$.
 
%$\mathbf{f}^*_k=[\mathbf{f}^{*}(h^{p}_{k}, s_{l}),\mathbf{f}^{*}(h^{p}_{k}, s_{2}), \mathbf{f}^{*}(h^{p}_{k}, s_{3})],$ $ \mathbf{g}^{*}_k=[\mathbf{g}(h^{p}_{k}, s_{1}),\mathbf{g}(h^{p}_{k}, s_{2}),\mathbf{g}(h^{p}_{k}, s_{3})].$ 

$\mathbf{f}^*_k=[\mathbf{f}^{*}(s_{kl}),l=1,2,3],$ $ \mathbf{g}^{*}_k=[\mathbf{g}^*(s_{kl}), l=1,2,3].$ 

\textbf{Return}:strategies $\mathbf{f}_{a}=\mathbf{f}^{*}_T\cdots\mathbf{f}^*_{K+T},$ $\mathbf{g}_{a}=\mathbf{g}^{*}_T\cdots\mathbf{g}^*_{K+T}$ and the value upper bound $\bar{v}_1^p(s_{1l}),l=1,2,3$.
\label{finite}
\end{alg}

Now consider the iteration for calculating $\bar{v}^p_{k}(s_{kl})$ from all matrix games $Q^{p}(s_{kl})\in \mathbb{Q}_{k}^p$ applying Algorithm~\ref{finite}. We define any strategy concatenations to stage $k-1$ with at most one non-pure strategy at stage $(k-1)$ as
\begin{align}
\begin{split}
\mathbf{f}^{np}_{k-1}&=\mathbf{f}^p_{k-2}\mathbf{f}_{k-1},\quad \mathbf{f}_{k-1} \in \mathbf{F}_{k-1},\\ \mathbf{g}^{np}_{k-1}&=\mathbf{g}^p_{k-2}\mathbf{g}_{k-1},\quad \mathbf{g}_{k-1} \in \mathbf{G}_{k-1},
\end{split}
\label{np_strategy}
\end{align}
where $\mathbf{f}^p_{k-2},\mathbf{g}^p_{k-2}$ are concatenations of pure strategies to stage $(k-1)$. We denote the corresponding auxiliary matrix as $\tilde{Q}(s_{kl}) \in \tilde{\mathbb{Q}}_k$ for cyber state $\delta_l$, the one shot game value based on payoff matrix $\tilde{Q}(s_{kl})$ as $\tilde{v}_{k}(s_{kl})$, i.e.,
\begin{align}
\begin{split}
&\tilde{Q}(s_{kl}) \\
=&\tilde{r}(s_{kl})+\sum_{\delta_h\in S} \tilde{P}( s_{(k+1)h} |s_{kl})\cdot \tilde{v}_{k+1}(s_{(k+1)h}).
\end{split}
\label{tilde_Q}
\end{align}
Here each possible hybrid state $s_{kl}$ for time instant $k$ is calculated from a none pure strategy defined as~\eqref{np_strategy}. Similarly, the value is defined as
\begin{align}
\tilde{v}_{k}(s_{kl})=\max_{\tilde{Q}(s_{kl})\in \mathbb{\tilde{Q}}_k}v^*[\tilde{Q}(s_{kl})].
\label{tilde_v}
\end{align}
The following theorem shows that at every stage $k$, $\bar{v}_{k}^{p}(s_{kl})$ is greater than or equal to $\tilde{v}_{k}(s_{kl})$.
\\
\begin{thm}
Consider the value iteration for stage $k$ as a one shot robust game. %given a pure strategy history $h^{p}_{k}$, 
Based on $\bar{v}_{k}^{p}(s_{kl}) \geq 0$ of previous iteration, 
we define the robust game value  obtained at $k$ as~\eqref{pickv}. Then for $k=2,\cdots, K$, $\tilde{v}_{k}(s_{kl})$~\eqref{tilde_v} is upper bounded by $\bar{v}_{k}^{p}(s_{kl})$, i.e., $\tilde{v}_{k}(s_{kl}) \leqslant \bar{v}_{k}^{p}(s_{kl}).$
\label{robust}
\end{thm}
\begin{pf}
Since $\bar{v}_{k+1}^{p}(s_{(k+1)h})$ is a nonnegative scalar value, the extreme points of the set $\mathbb{Q}_{k}$ is a subset of the extreme points  of set $\mathbb{Q}^p_k$. Hence, by considering the value of matrix game $Q^{p} (s_{kl})\in \mathbb{Q}^p_k$ defined in~\eqref{eq:Q}, we will get the upper bound of the maximum game value from extreme points of $\mathbb{Q}_k$. 

Consider the following optimization problem for the system with constraint inequality~\eqref{optr} for any possible attacker's strategy vector $\mathbf{f}$ at each stage $k$
%From the system's perspective, to compute the value of the matrix game is equivalent with:
%\vspace{-5pt}
\begin{align}
%\begin{split}
%\vspace{-5pt}
\min_{\mathbf{g}} \quad & z\label{ob_z}\\
\text{subject to}\quad &z \geq \max_{\tilde{Q}(s_{kl})\in \tilde{\mathbb{Q}}_{k}}\mathbf{f}^{T} [\tilde{Q}(s_{kl})]\mathbf{g}.
%\end{split}
\label{optr}
\end{align}
As proven by~Lemma 5 in \cite{RGT},~\eqref{optr} is equivalent to the following constraint that considers only the extreme points
\begin{align}
\quad z \geq \max_{Q^{p}(s_{kl})\in \mathbb{Q}_{k}^p}\mathbf{f}^{T} [Q^{p}(s_{kl})]\mathbf{g},
\end{align}
For the worst-case $f$, the above is also true. Hence, let 
\begin{align}
\label{eq1}
v^p_{k}(s_{kl})
%=\max_{Q_{k}^{p}(h^{p}_{k},s_{l})\in \mathbb{Q}_{kp}}v^*[Q_{k}^{p}(h^{p}_{k},s_{l})]
=\max_{Q^{p}(s_{kl})\in \mathbb{Q}_k^p} \min\limits_{\mathbf{g}}\max\limits_{\mathbf{f}}\mathbf{f}^T[Q^{p}(s_{kl})]\mathbf{g}.
%\label{pickv}
\end{align}
For optimal policies $\mathbf{f}^{*}(s_{kl})$ and $\mathbf{g}^{*} (s_{kl})$, the above optimization problem~\eqref{eq1} results in a cost
\\\centerline{$
%\begin{align*}
 \max\limits_{Q^{p}(s_{kl})\in \mathbb{Q}_{k}^p}v^*[Q^{p}(s_{kl})].
%\end{align*} 
$}
However, $(\mathbf{f}^{*}(s_{kl}),\mathbf{g}^{*}(s_{kl}))$ can be non-pure strategies, meaning that when we apply  $(\mathbf{f}^{*}(s_{kl}),\mathbf{g}^{*}(s_{kl}))$  to calculate system dynamics such as equations~\eqref{dynamicgame}, they will not result in any extreme point of set $\mathbb{Q}_{k+1}$. 

Now consider the final stage $K$, we have
\\\centerline{$
%\begin{align*}
Q^p(s_{Kl})=r^p(s_{Kl}), \tilde{Q}(s_{Kl})=\tilde{r}(s_{Kl}),
%\end{align*}
$}
and use the $Q^p(s_{Kl})$ and $\tilde{Q}(s_{Kl})$ in the above proof, value $\tilde{v}_k(s_{Kl})$ from $\tilde{Q}(s_{Kl})$ is smaller than $\bar{v}_k^p(s_{Kl})$ from the extreme points auxiliary matrix $Q^p(s_{Kl})$, i.e., for $K$, the following inequality holds
\\\centerline{$
%\begin{align*}
\tilde{v}_k(s_{Kl}) \leqslant \bar{v}_k^p(s_{Kl}).
%\end{align*}
$}
Then, by induction, with the value $\tilde{v}_{k+1}(s_{(k+1)h})$ of iteration for stage $k+1, 2\leqslant k \leqslant K-1$ satisfies %%%%%(from $\mathbb{Q}_{(k+1)p}$):
%\begin{align*}
\\\centerline{$\tilde{v}_{k+1}(s_{(k+1)h})\leqslant \bar{v}_{k+1}^{p}(s_{(k+1)h}),$} 
%\end{align*}
and nonnegative payoff and state transition probability $r_{k}^{ij}\geqslant 0$ and $\tilde{P}_{k}^{ij}\geqslant 0$, replacing $\tilde{v}_{k+1}(s_{(k+1)h})$ by $v_{k+1}^{p}(s_{(k+1)h})$ in~\eqref{eq:Q} will make every entry of matrix $\tilde{Q}(s_{kl})$ smaller than matrix $Q^{p}(s_{kl})$. 
% since $r_{k}^{ij}\geqslant 0$ and $\tilde{P}_{k}^{ij}\geqslant 0$. %%% by definition. 
%The system is a minimizer, and is possible to get a smaller value at time k, with some %optimal strategy history $h^{*}_{k} \neq h^{p}_{k}$, 
With a similar argument in the next iteration for stage $k-1$, we have 
%\begin{align*}
\\\centerline{$
\tilde{v}_{k}(s_{kl}) \leqslant \bar{v}_{k}^{p}(s_{kl}).
$}
%\end{align*} 
%(Note that here $v_{k}^{s_{l}}(h^{p}_{k-1})$ is not the game value at $k$ either, because the strategies from 1 to $k-2$ must be pure to get it).
\end{pf}

Based on the above observation, we arrive at the suboptimal algorithm to compute the equilibrium solutions, illustrated in the Algorithm~\ref{finite}. Note that for keeping the physical state $x_{[k-T,k]}$ of the first stage of the game starts at $\hat{x}_0$, in the above Algorithm~\ref{finite} the $K$-stage game  starts at $k=T$. This does not affect our proofs in this section for considering $k=1,\dots, T$.
According to~Theorem~\ref{robust}, we use Algorithm~\ref{finite} to compute an upper bound of the value and the corresponding suboptimal strategy for every step. The % Nash Equilibrium selection 
function $\pi$ computes the strategy and robust value as defined in~\eqref{pickv}. 

The values of the finite stage game $\tilde{v}_k(s_{kl})$ and $\bar{v}_k^p(s_{kl})$ resulting from two auxiliary matrices $\tilde{Q}(s_{kl})$ $Q^p(s_{kl})$ are based on strategy concatenations that only differ at stage $k-1$ (i.e., the same and pure strategies from stages 1 to $(k-2)$). By value iteration backward to stage $1$, we compare the game value for all possible strategies and the robust game value $\bar{v}_1^p(s_{1l})$ of~Algorithm~\ref{finite} in the following theorem.
\begin{cor}
%Moreover, by following this iteration method,
~Algorithm~\ref{finite} yields an upper bound $v_1(s_{1l})$ for the value of the $K$-stage game, together with suboptimal strategies $\mathbf{f}_{a}$ and $\mathbf{g}_{a}$. 
\end{cor}
%\begin{pf}
The strategies $\mathbf{f}_{a}, \mathbf{g}_{a}$ of~Algorithm~\ref{finite} are possibly not pure. %strategy history $h^{p}_{K} \in H^{p}_{K}$. 
According to~Theorem~\ref{robust}, we obtain $\tilde{v}_{k}(s_{kl})\leqslant \bar{v}^{p}_{k}(s_{kl}),$ and the proof holds for every $k=2,\cdots,K$. Consider the value iteration for $k=1$, with 
$\tilde{v}_{2}(s_{2l})\leqslant \bar{v}^p_k(s_{2l})$, and $\text{\ }\quad Q^{ij}( s_{1l})$
$=r^{ij}(s_{1l})+\sum\limits_{\delta_h\in S}{P}^{ij}(s_{2h}|s_{1l})v^{p}_{2}(s_{2h}) \leqslant Q^{p,ij}(s_{2l}),$
%\end{align*}
%\normalsize
thus the true value of the K-stage game $v^{*}[Q(s_{1l})] \leqslant \bar{v}^{p}_{1}(s_{1l})$. The iterative value based on pure strategy auxiliary matrix sets $\mathbb{Q}_{k}^p, k=1,\cdots, K,$ obtained from~Algorithm~\ref{finite} is an upper bound for the game value.
%\end{pf}
Let $v^*[Q(s_{na})]$ represent the minimum total payoff of the system when the strategy is calculated given that there is no attack at all in $K$ stages, then $\bar{v}^{p}_{1}(s_{1l})-v^{*}[Q(s_{1l})] \leqslant \bar{v}^{p}_{1}(s_{1l})-v^*[Q(s_{na})]$, since  $v^*[Q(s_{na})] \leqslant v^{*}[Q(s_{1l})]$ when the system operates in normal state without sacrificing any control cost to play against attacks. The sub-optimality of value $\bar{v}^{p}_{1}(s_{1l})$ calculated from Algorithm 1 is then bounded though we do not know the true value $v^{*}[Q(s_{1l})]$ of the game.


\begin{algorithm}[!ht]
\begin{algorithmic}[1]
\Require Query workload $Q$, event stream $I$, \app\ graph $G$, hash table of snapshots $S$
\Ensure Hash table of results $R$ 
\State $G \leftarrow \emptyset$, $S, R \leftarrow$ empty hash tables
\ForAll {event $e \in I$ with $e.type=E$} 
    \State $//$ \textbf{\app\ graph construction}
    \ForAll {$q \in Q$ \text{ with event types }T}
        \ForAll {$E' \in T,\ E' \neq E$}
            \State $G_{E'} \leftarrow \mathit{getGraphlet}(G,E')$,
            $G_{E'}.\mathit{active} \leftarrow \mathit{false}$
        \EndFor
    \EndFor
    \If {\textbf{not} $G_E.\mathit{active}$}
        \State $G_E \leftarrow \mathit{createGraphlet()}$, $G_{E}.\mathit{active} \leftarrow \mathit{true}$,
        $G \leftarrow G \cup G_E$
        \If {$G_E.\mathit{shared}$ by $Q_E \subseteq Q$}
            $x \leftarrow \mathit{createSnapshot()}$ 
            \ForAll {$q \in Q_E$}
                \ForAll{$E' \in \mathit{pt}(E,q), E' \neq E$}
                    \State $G_{E'} \leftarrow \mathit{getGraphlet}(G,E')$
                    \State $S(x,q) \leftarrow S(x,q) + sum(G_{E'},q)$ \hspace{0.5cm}$//$ Eq.~5
                \EndFor
            \EndFor
        \EndIf    
    \EndIf
    \State insert $e$ into $G_E$
    \State $//$ \textbf{Trend count computation}
    \If {$G_E.\mathit{shared}$ by $Q_E \subseteq Q$}
        \If {$\forall q \in Q_E\ pe(e,q)$ are identical}
            \State $count(e,Q_E) \leftarrow count(e,q)$ \hspace{2.3cm}$//$ Eq.~2
        \Else\ $y \leftarrow \mathit{createSnapshot()}$, $count(e,Q_E) = y$
            \ForAll {$q \in Q_E$}
                $S(y,q) \leftarrow count(e,q)$ \hspace{0.2cm}$//$ Eq.~2
            \EndFor
          \EndIf
    \Else\ $count(e,q)$ \hspace{5.2cm}$//$ Eq.~2
    \EndIf
    \ForAll{$q \in Q$}
  	    \If {$E \in \mathit{end}(q)$} 
  		    $R(q) \leftarrow R(q) + count(e,q)$ $//$ Eq.~3
        \EndIf
    \EndFor
\EndFor
\State \Return $R$
\end{algorithmic}
\caption{\app\ shared online trend aggregation}
\label{algo:snapshot-propagation}
\end{algorithm}


\section{Simulation}

\subsection{motivation} %
The most efficient way to figure out the answers to the questions we posed in the introduction is to deploy the proposed framework on a real-world platform and analyze how users adopt different and complex privacy policies to optimize their rewards.
However, direct deployment of these strategies and investments is currently impractical due to the following reasons.


Firstly, the most important reason is that such an online experiment may lead to the decline of the recommendation performances as well as the user experience, which harms the platform's revenue.
In the real world, nearly all the companies determine their platform mechanism driven by interest, and the revenues of the platforms are highly correlated with the recommendation performances. 
Therefore, it's nearly impossible to persuade any platform to directly deploy proposed strategies and mechanism online without other benefits.  



Secondly, the experiments are \czq{built} upon several simplifications, mentioned in \cref{assumptions}, which poses challenges towards recommendation model training process.
For example, we assume when a user \czq{adjusts} his data disclosure policy, the recommendation system will forget his un-disclosed data. 
To facilitate such challenges, model unlearning or other privacy-preserving technologies are imposed.
However, in real-world applications, very few the e-commercial platforms have deployed these privacy-preserving technologies during the deep recommendation model training and evaluation processing.
As a result, we may still fail to guarantee the assumptions and simulation methodology becomes a substitution.










In summary, inspired by the success of simulation study on dynamic interactive problems in real-world applications~\cite{Ie:arxiv19:RecSim,krauth2020offline,lucherini2021t,yao2021measuring},  we employ the simulation to study the effects of the proposed framework and the possible game between users and the platform.















\subsection{Simplified Assumptions}
\label{assumptions}

To simplify the simulation process for easier analysis, we make some necessary assumptions to simplify the problem.

\begin{assumption}[Static Assumption] User $i$ optimizes her/his policy on the fixed data $\di{}$ which is not affected by user policy $\pi_i$.
\label{assumption:static}
\end{assumption}

Here static means the user data $\di{}$ is fixed during the simulation, but the disclosed data $\si{}$ produced by different user policies is dynamic. 
It is also the most common setting for recommendation task in research papers~\cite{Rendle:www10:Factorizing,Hidasi:ICLR2016:gru4rec,NCF,kang2018self,Sun:cikm19:BERT4Rec}.
In the simulation, we train the recommendation system $\texttt{M}_{{\scriptscriptstyle \mathcal{S}}}$ on the collected dynamic data $\mathcal{S}$ and validate the recommendation efficiency on a fixed test set. 
In real-world applications, the data $\di{}$, which contains the behavior data from the interaction with the recommender $\texttt{M}_{{\scriptscriptstyle \mathcal{S}}}$, is also dynamically changing with the user's policy $\pi_i$.
It is beyond the scope of this paper and we leave it as the future work. 


\begin{assumption}[Immediate Assumption] The recommendation model $\texttt{M}_{{\scriptscriptstyle \mathcal{S}}}$ can only use the data $\si{}$ currently disclosed by each user $i$.
\label{assumption:forget}
\end{assumption}
The motivation of this assumption is that an untrusted platform can leverage user $i$' all data $\di{}$ if it can use the data disclosed in previous actions.
Without this constraint, the privacy right discussed in this paper is meaningless.
To achieve this, the platform can retrain the model from scratch with new data $\si{'}$ or quick unlearn the data in $\si{}$ then finetune with data $\si{'}$~\cite{cao2015towards,bourtoule2021machine,chen2022recommendation}.


However, the \cref{assumption:forget} also raises a new challenge that the asynchronous changes of user policy bring intractable computation costs for the platform since each time the user changes the disclosed data, the platform needs to update the model.
Here, we make an assumption for simplifying the simulation, assuming all users realize that the platform will cyclically (e.g., once a day) collect their privacy decisions and update recommender systems.
\begin{assumption}
[%
Cyclical Assumption]
Platform cyclically collects user privacy choices, and then the platform updates the model using all newly disclosed data. 
\label{assumption:synchronization}
\end{assumption}



In summary, for easy analysis in simulations, we introduce these assumptions to ignore the time and dynamic effects in this feedback system, just like the traditional recommendation task formulation.
























\subsection{Platform Mechanism Simulation}
\label{sec:plat_mech}
In order to validate the effect of the platform mechanism, we adopt several mechanisms during simulation. 
For easy comparison, we utilize one mechanism at each experiment. %


\subsubsection{\textbf{Data Split Rule}}

\czq{In our simulation, we do not split the profile attribution and the user can determine whether to share all of their attributes.}
For behavior data, we apply ``percentage split'' as $\delta_b$ with different split granularity $p$ (e.g., 1/3) to split the behavior sequence into $1/p$ parts. 
One obvious advantage of ``percentage split'' is that it can normalize the size of user action space and decrease the inconvenience of the interaction between the user and the platform.

\subsubsection{\textbf{Data Disclosure Strategy}}
\label{sec:data_disclose_choice}

As the platforms have certain flexibility to implement different data disclosure strategies, we discuss three representative disclosure strategies used in our study for behavior data in this subsection.
These strategies determine the data disclosure action space $\Pi$ the user can choose.
For profile attributes, we found that all users tend not to disclose them in the experiments since these features are negligible for improving recommendation utility in the presence of behavior data.
Similar result that user profile features contribute very marginal to the recommendation results in the case of strong user behavior modeling on public benchmark datasets has also been reported in other works~\cite{kang2018self,Sun:cikm19:BERT4Rec}.
Thus, in the following study, we mainly focus on modeling only  behavior data.

The ``\textit{separate}'' rule gives the users the control to freely disclose any split personal data.
For this rule, the size of user $i$'s the action space is exponentially expended on the size of the spilt data set $|\delta_b ({\scriptstyle \mathcal{D}_{i,b}})|$, denoted as $2^{|\delta_b ({\scriptstyle \mathcal{D}_{i,b}})|}$. 
However, too many choices might make it difficult for users to make better data disclosure decisions.

Another data disclosure strategy named ``\textit{oldest continuous}'' provides users the choices to disclose continuous behavior data from the beginning time, such as selecting ``the oldest 33\% data''.
In this strategy, to disclose newer behavior data ${\scriptstyle \mathcal{S}_{i,bj}}$, users must also disclose all behavior data before it.
Take an already split data $\delta_b ({\scriptstyle \mathcal{D}_{i,b}}) = \{\scriptstyle \mathcal{S}_{i,b1}, \scriptstyle \mathcal{S}_{i,b2}, \scriptstyle \mathcal{S}_{i,b3}\}$ as an example, the action space provided by oldest continuous strategy is $\Pi = \{[0,0,0], [1,0,0], [1,1,0], [1,1,1]\}$, and its corresponding disclosed data is $\{\varnothing ,
    \{\! {\scriptstyle \mathcal{S}_{i,b1}} \!\},
    \{\! {\scriptstyle \mathcal{S}_{i,b1}} , {\scriptstyle \mathcal{S}_{i,b2}}\! \},$
    $\{ {\scriptstyle \mathcal{S}_{i,b1}}, {\!\scriptstyle \mathcal{S}_{i,b2}},$ ${\scriptstyle \mathcal{S}_{i,b3}}\} \}$.
``\textit{Latest continuous}'' mechanism is similar to ``oldest continuous'', with the only difference in the opposite direction.
The size of these two mechanisms' action spaces is $|\delta_b ({\scriptstyle \mathcal{D}_{i,b}})|$.









\subsection{User Policy Simulation}
\label{sec:user}


In this subsection, we introduce the simulation of user policy in our proposed framework.
As defined in \cref{eq:S_i}, the disclosed data $\si{}$ is result of the platform mechanism $\mathrm{G}$ and user's disclosure policy $\pi_i$. 
Meanwhile, in \cref{eq:updated_rec}, the recommendation utility $\texttt{U}_i( \si{})=\texttt{U}'(\si{},\, \texttt{M}_{{\scriptscriptstyle \mathcal{S}}})$ is also determined by the recommendation model $\texttt{M}_{{\scriptscriptstyle \mathcal{S}}}$, which is \czq{built} upon the all users' disclosed data $\mathcal{S}$. 
The reward of user $i$ may be varied even when $i$ keeps the disclosed data $\si{}$ unchanged since other users might change their disclosed data and the recommender system is changed.
Thus, the expectation rewards are considered rather than immediate value defined in \cref{eq:framework} and we assume all the users are rational and seek for the optimal privacy disclosure action  $\alpha_i^*$ to the optimal expected reward $E[ \texttt{R}_i | \alpha_i ] $ as his policy, i.e., %
\begin{equation}
    \begin{aligned}
    \alpha_i^{*} &= \argmax_{\alpha_i \in \Pi} E[ \texttt{R}_i | \alpha_i ]=  \argmax_{\si{\in [ \Pi \otimes \mathrm{\delta}(\di{}) ] }} E[ \texttt{R}_i(\si{)} ] \\
& =\argmax_{\alpha_i \in \Pi } E\Bigl[ -\lambda_i \texttt{C}_i\bigl( \alpha_i \otimes \mathrm{\delta}(\di{}) \bigr) + \texttt{U}_i\bigl( \alpha_i \otimes \mathrm{\delta}(\di{}) )\bigr) \Bigr].
    \end{aligned}
\label{eq:opt_pi}
\end{equation}

As mentioned before, recommendation utility $\texttt{U}_i$ has been discussed in \cref{sec:platform_obj}.
To study this objective, we need to define the privacy cost function $\texttt{C}_i$ and sensitive weight $\lambda_i$.



\subsubsection{\textbf{Privacy Cost Function}}
\label{sec:privacy_cost}
We simulate every user with the same cost function $\texttt{C}$ and leave the diversity of user privacy sensitivity to the parameter $\lambda_i$. 
Following current experiment specifications in the economics literature~\cite{lin2019valuing,tang2019value}, we model the privacy cost function as a linear summation\footnote{See the Eq. 2 in \cite{lin2019valuing} and the dis-utility from disclosure in the econometric specification session in \cite{tang2019value}.} of disclosed personal data.


\czq{According to the comprehensive survey on privacy value definition \cite{MKT-053}, people will measure the value of their privacy into the intrinsic value of privacy and the instrumental value of privacy.}
\czq{
The intrinsic loss indicates the sake of protecting their intrinsic private data, which measures the valuation on the intrinsic properties such as the education or the income levels. }
\czq{In this work, we denote the intrinsic loss towards the privacy cost on amount of the sharing user profile attributes.}
\czq{The instrumental value of privacy indicates how the transaction efficiency would be affected by sharing user data, especially the data generated in the applications. 
In this work we denote the privacy cost towards the percentage of shard user historical behavior data. 
Therefore, the privacy cost function is described below,}
\begin{equation}
    \texttt{C}_i(\si{})= \beta_i * | {\scriptstyle \mathcal{S}_{i,a}} | + \frac{| {\scriptstyle \mathcal{S}_{i,b}} |}{ | {\scriptstyle \mathcal{D}_{i,b}} |}
    \label{eq:cost_function0}
\end{equation}
\czq{where the first term indicates the intrinsic loss and the second term indicates the instrumental loss.} 
\czq{If user does not tend to disclose profile attribute, such privacy cost function can be simplified to the following format with the instrumental value alone.}
As mentioned in \cref{sec:plat_mech}, user tends not to disclose profile attributes $\scriptstyle \mathcal{D}_{i,a}$ due to no gains in our experiments, so we only consider behavior data here, i.e.,
\begin{equation}
    \texttt{C}_i(\si{})=\texttt{C}(\si{}) =  \frac{| {\scriptstyle \mathcal{S}_{i,b}} |}{ | {\scriptstyle \mathcal{D}_{i,b}} |}
    , %
    \label{eq:cost_function}
\end{equation}
where the $|x|$ is the number of elements in $x$.
Here, the percentage based measurement regards different amount of users' data equally. %


This reduced form specification is not unrealistic as it captures the substitution effect among personal data and incorporates the idea of constant marginal privacy cost. 
One might argue for a higher order functional to capture richer implications. 
However, there is little experimental evidence that the higher order form for privacy cost exists and how the functional form looks like.






\subsubsection{\textbf{Privacy Sensitive Weight}}
\label{sec:user_type}
For user $i$ who disclosed all her/his data (i.e., $\si{} = \di{}$), her/his privacy cost compared to not sharing any data (i.e., $\si{} = \varnothing $) is 
\begin{equation}
    \texttt{C}(\di{}) - \texttt{C}(\varnothing).
\label{eq:privacy_diff}
\end{equation}
Meanwhile, her/his anticipated recommendation utility compared to not sharing any data is:
\begin{equation}
    \texttt{U}(\di{}) - \texttt{U}(\varnothing).
\label{eq:utility_diff}
\end{equation}
We assume all users have accessed to the recommendation utility $\texttt{U}(\di{})=\texttt{U}'(\di{},\texttt{M}_{{\scriptscriptstyle \mathcal{D}}})$ computed on all the data $\di{}$ and the recommendation utility without their data $\texttt{U}(\varnothing)$ before they can take data disclosing actions, which can be regard as a prior knowledge, like the experiences before the platform adopted our framework.
With \cref{eq:privacy_diff} and \cref{eq:utility_diff}, we define the privacy sensitive weight $\lambda_i$ as: 
\begin{equation}
    \lambda_i = w_i  * \frac{\texttt{U}(\di{})  - \texttt{U}(\varnothing) } {  \texttt{C}(\di{}) - \texttt{C}(\varnothing) },
    \label{marginal_define}
\end{equation}
where $w_i$ indicates the diversity of user types towards privacy sensitivity.
The users with $w_i > 1$ is privacy sensitive users, as they will not be willing to disclose the corresponding data $\di{}$ if they only get $\texttt{U}(\di{})$ as before.
While users  with $w_i < 1$ are just the opposite. %
Therefore, the user's privacy sensitive weight is pre-computed, and the $\texttt{U}(\di{})$ can be regarded as the benchmark expectation of the platform.
The formulation of the privacy sensitive weight $\lambda_i$ also meets the idea from \cite{lin2019valuing}, where the heterogeneity from users' social demographic variety should also be explicitly characterized. %



\subsubsection{\textbf{Simulation Algorithm}}

As users behave rationally to find the optimal strategy with a trade-off of exploration and exploitation, it just meets the idea of the reinforcement learning algorithm. 
Therefore, we model each user as a unique agent and apply a multi-agent reinforcement learning method to simulate user possible policy adaptation. 
The recommender system is regarded as the environment to provide feedback, which is built upon the disclosed user data.
All agents' policies are optimized simultaneously by determining their actions, i.e., the disclosed data $\scriptstyle \mathcal{S}^t$ at simulation epoch $t$, which is used to train the recommendation model $\texttt{M}_{{\scriptscriptstyle \mathcal{S}}^{t}}$.
As mentioned before, users tend to find an optimal action over possible action space $\Pi$ to maximize his expected reward, which is determined by all agents in this dynamic MARL environment. 


We assume each user (agent) realizes this situation that the immediate reward is the result of all agents, but no communication or observation among agents is permitted. 
Then, each agent is concerned about her/his own utility and regards the environment as a dynamic system that is partially correlated to herself/himself. 
Now, it is simplified to a Multi-Armed Bandit problem~\cite{katehakis1987multi}.


However, the challenge of the exploration and explication problem also exists in our simulation. 
To address it, we adopt a simple but efficient method, Epsilon Greedy~\cite{sutton2018reinforcement} algorithm, to simulate user's policy $\pi_i$ as following, %
\begin{equation}
     \alpha_i^{t+1} = \left\{
\begin{array}{l l }
\alpha_i \sim \texttt{P}^t_i,
& \text{with possibility } \epsilon \\
\argmax_{\alpha_i} Q_i^t(\alpha_i), & \text{with possibility }  1-\epsilon \\
\end{array} \right. 
    \label{epsilon_greedy}
\end{equation}
where $Q_i^t(\alpha)$ is the user $i$'s estimation value at simulation epoch $t$ on action $\alpha$, and $\texttt{P}^t_i$ denotes a random sample policy.
To conduct an efficient policy exploration, we sample a less explored action with a higher possibility as following,
\begin{equation}
    \texttt{P}^t_i(\alpha) = \frac{ 1/ (N^{t-1}_i(\alpha) +1) } { \sum_{x \in \Pi} 1/(N^{t-1}_i(x) +1) },
    \label{random_rule}
\end{equation}
where $N^{t-1}_i(\alpha)$ represents the total number of action $\alpha$ was taken by user $i$ from start to the last simulation epoch $t{-}1$.
In convenience, we adopt the approximated expected estimation results and 
update it with the residual between the estimation $Q_i^{t-1}(\alpha_i^{t-1})$ and immediate reward  $\texttt{R}_i^{t-1}$ when she/he takes action $\alpha_i^{t-1}$ as following.
\begin{equation*}
       Q_i^t(\alpha) {=} \left\{\!\!\!
\begin{array}{l l}
Q_i^{t-1}(\alpha), &  \text{if } \alpha_i^{t-1} {\neq} \alpha \\
Q_i^{t{-}1}(\alpha) {+} \frac{1}{N^t_i(\alpha)} \bigl(\texttt{R}_i^{t-1}(\! \alpha {\otimes} \mathrm{\delta}(\di{})  ) {-} Q_i^{t{-}1}(\alpha) \bigr), &  \text{if }  \alpha_i^{t{-}1} {=} \alpha \\
\end{array} \right. 
\end{equation*}
where $\texttt{R}_i^{t-1}$ is user $i$-th immediate objective at simulation epoch $t{-}1$, computed by \cref{eq:framework}. 
$Q_i^0(\alpha) $ is the user $i$'s initial expected reward if she/he takes action $\alpha$. 
which is initialized to $0$ as users have no prior about their behaviors on the new dynamic environment.


In our simulation, we set initial $\epsilon=0.5$ for all agents and decay a half during the MARL training processing. The detailed decay epoch is co-related to the size of possible action space $\Pi$.
Here, we define it as $\epsilon = 0.5^{ t /(3 * |\Pi |) }  $, 
where $t$ is the epoch during the reinforcement learning training processing. 

\subsection{\textbf{Discussion}}
To figure out how the platform mechanism affects users' behavior, we turn to the simulation built upon several simplified assumptions. 
One fundamental assumption is the hypothesis of rational man, where users will seek their optimal policies to maximize their objectives. 
However, in the real-world scenarios, human behaviors are also affected by psychological factors, which should also be modeled in future work.
One detailed example is that some users may feel exhausted digging out all the potential privacy choices with the provided platform mechanism.
In our simulation, we assume there remains no mental cost when a user adjusts his policy. 
However, in the reality, some users may refuse to change their policy frequently, especially in complex user interaction applications.
For such situation, a convenient user interface (UI) could be a potential solution to mitigate users' fatigues. 
\czq{
Another important factor is that users may adjust their trust level towards the platform during their exploration. One detailed example is that if the platform or even the recommender system \cite{zhang2022pipattack} is easy to be attacked or the platform will abuse their disclosed data to other applications, they may re-consider their privacy sensitivity. 
Though some works have discussed the utilization of trusted platform or the privacy-preserved recommendation model, the possible effects on user psychological factors might be tackled by a dynamic modeling on the user privacy sensitive weights, which is out of the scope of this work.
}
We simplify the influences of the psychological factors in this work and leave the exploration of psychological effects in mechanism designs and UI designs for future works. 







\begin{comment}
\begin{figure}
\includegraphics[width=\linewidth]{figs/beyond_tss_lesion.pdf}
\caption[]{End-to-End runtime lesion study of the entire MNIST dataset and the FMA featurized music dataset. Each of DROP's contributions provides a runtime improvement.}
\label{fig:beyond_lesion}
\end{figure}
\end{comment}



\section{Conclusion}
\label{sec:conclusion}

Advanced data analytics techniques must scale to rising data volumes. 
DR techniques offer a powerful toolkit when processing these datasets, with PCA frequently outperforming popular techniques in exchange for high computational cost. 
In response, we propose DROP, a new dimensionality reduction optimizer. 
DROP combines progressive sampling, progress estimation, and online aggregation to identify high quality low dimensional bases via PCA without processing the entire dataset by balancing the runtime of downstream tasks and achieved dimensionality. 
Thus, DROP provides a first step in bridging the gap between quality and efficiency in end-to-end DR for downstream \red{analytics}. 

%We revisit canonical operators for time series dimensionality reduction and the measurement study of~\cite{keogh-study}, and show that PCA is more effective than popular alternatives in the data mining literature often by a margin of over $2\times$ on average on gold-standard time series benchmark data sets with respect to output data dimension. More surprisingly, we empirically demonstrate that a small number of samples are sufficient to accurately characterize directions of maximum variance and obtain a high-quality low-dimensional transformation.




\bibliographystyle{agsm}        
{  \small 
\bibliography{gamejs}
}

%\input{bio_coding}
\end{document}