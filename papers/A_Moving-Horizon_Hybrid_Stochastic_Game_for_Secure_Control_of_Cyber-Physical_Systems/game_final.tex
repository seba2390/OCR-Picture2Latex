\documentclass[twocolumn]{autart}  % Comment this line out
                                                          % if you need a4paper
\usepackage{color}
\usepackage{graphicx}

\usepackage{hyperref}
\usepackage{amsmath,amsfonts, amssymb}
\usepackage{multirow}
\usepackage{natbib}
%\bibliographysyle{agsm}
\begin{document}

\begin{frontmatter}
\title{A Moving-Horizon Hybrid Stochastic Game for Secure Control of Cyber-Physical Systems\thanksref{footnoteinfo}} % Title, preferably not more 
                                                % than 10 words.

\thanks[footnoteinfo]{This material is based on research sponsored by DARPA under agreement number FA8750-12-2-0247.  The U.S. Government is authorized to reproduce and distribute reprints for Governmental purposes notwithstanding any copyright notation thereon.  The views and conclusions contained herein are those of the authors and should not be interpreted as necessarily representing the official policies or endorsements, either expressed or implied, of DARPA or the U.S. Government. This work was also supported in part by NSF CNS-1505701, CNS-1505799 grants, and the Intel-NSF Partnership for Cyber-Physical Systems Security and Privacy. This paper was not presented at any IFAC meeting. Part of the results in this work appeared at the 52nd Conference of Decision and Control, Florence, Italy, December 2013~\cite{cdc_replay} and the 53rd Conference of Decision and Control, Los Angeles, CA, USA, December 2014~\cite{game_cdc14}. Corresponding author F.~Miao. Tel. 2154216608.}

\author[Uconn]{Fei Miao}\ead{fei.miao@uconn.edu},  % Add the 
\author[NYU]{Quanyan Zhu}\ead{quanyan.zhu@nyu.edu},               % e-mail address 
\author[Duke]{Miroslav Pajic}\ead{miroslav.pajic@duke.edu},  % (ead) as shown
\author[Upenn]{George J. Pappas}\ead{pappasg@seas.upenn.edu}

\address[Uconn]{ University of Connecticut, Storrs, CT, USA}
\address[Upenn]{ University of Pennsylvania, Philadelphia, PA, USA}  % Please supply                                              
\address[NYU]{ New York University, Brooklyn, NY, USA}             % full addresses
\address[Duke]{Duke University, Durham, NC, USA}
%\address[Pappas]{ University of Pennsylvania, Philadelphia, PA, USA}

\begin{keyword}
Stochastic Game, Secure Control, Saddle-Point Equilibrium
\end{keyword}
%\maketitle

\begin{abstract}
\label{abstract}
 In this paper, we establish a zero-sum, hybrid state stochastic game model for designing defense policies for cyber-physical systems against different types of attacks. With the increasingly integrated properties of cyber-physical systems (CPS) today, security is a challenge for critical infrastructures. Though resilient control and detecting techniques for a  specific model of attack have been proposed, to analyze and design detection and defense mechanisms against multiple types of attacks for CPSs requires new system frameworks. Besides security, other requirements such as optimal control cost also need to be considered. The hybrid game model we propose contains physical states that are described by the system dynamics, and a cyber state that represents the detection mode of the system composed by a set of subsystems. A strategy means selecting a subsystem by combining one controller, one estimator and one detector among a finite set of candidate components at each state. Based on the game model, we propose a suboptimal value iteration algorithm for a finite horizon game, and prove that the algorithm results an upper bound for the value of the finite horizon game. A moving-horizon approach is also developed in order to provide a scalable and real-time computation of the switching strategies. Both algorithms aims at obtaining a saddle-point equilibrium policy for balancing the system's security overhead and control cost. 
%This approach leads to a real-time algorithm that yields a sequence of Nash equilibrium strategies which can be shown to converge. 
The paper illustrates these concepts using numerical examples, and we compare the results with previously system designs that only equipped with one type of controller. 
\end{abstract}

\end{frontmatter}

% \leavevmode
% \\
% \\
% \\
% \\
% \\
\section{Introduction}
\label{introduction}

AutoML is the process by which machine learning models are built automatically for a new dataset. Given a dataset, AutoML systems perform a search over valid data transformations and learners, along with hyper-parameter optimization for each learner~\cite{VolcanoML}. Choosing the transformations and learners over which to search is our focus.
A significant number of systems mine from prior runs of pipelines over a set of datasets to choose transformers and learners that are effective with different types of datasets (e.g. \cite{NEURIPS2018_b59a51a3}, \cite{10.14778/3415478.3415542}, \cite{autosklearn}). Thus, they build a database by actually running different pipelines with a diverse set of datasets to estimate the accuracy of potential pipelines. Hence, they can be used to effectively reduce the search space. A new dataset, based on a set of features (meta-features) is then matched to this database to find the most plausible candidates for both learner selection and hyper-parameter tuning. This process of choosing starting points in the search space is called meta-learning for the cold start problem.  

Other meta-learning approaches include mining existing data science code and their associated datasets to learn from human expertise. The AL~\cite{al} system mined existing Kaggle notebooks using dynamic analysis, i.e., actually running the scripts, and showed that such a system has promise.  However, this meta-learning approach does not scale because it is onerous to execute a large number of pipeline scripts on datasets, preprocessing datasets is never trivial, and older scripts cease to run at all as software evolves. It is not surprising that AL therefore performed dynamic analysis on just nine datasets.

Our system, {\sysname}, provides a scalable meta-learning approach to leverage human expertise, using static analysis to mine pipelines from large repositories of scripts. Static analysis has the advantage of scaling to thousands or millions of scripts \cite{graph4code} easily, but lacks the performance data gathered by dynamic analysis. The {\sysname} meta-learning approach guides the learning process by a scalable dataset similarity search, based on dataset embeddings, to find the most similar datasets and the semantics of ML pipelines applied on them.  Many existing systems, such as Auto-Sklearn \cite{autosklearn} and AL \cite{al}, compute a set of meta-features for each dataset. We developed a deep neural network model to generate embeddings at the granularity of a dataset, e.g., a table or CSV file, to capture similarity at the level of an entire dataset rather than relying on a set of meta-features.
 
Because we use static analysis to capture the semantics of the meta-learning process, we have no mechanism to choose the \textbf{best} pipeline from many seen pipelines, unlike the dynamic execution case where one can rely on runtime to choose the best performing pipeline.  Observing that pipelines are basically workflow graphs, we use graph generator neural models to succinctly capture the statically-observed pipelines for a single dataset. In {\sysname}, we formulate learner selection as a graph generation problem to predict optimized pipelines based on pipelines seen in actual notebooks.

%. This formulation enables {\sysname} for effective pruning of the AutoML search space to predict optimized pipelines based on pipelines seen in actual notebooks.}
%We note that increasingly, state-of-the-art performance in AutoML systems is being generated by more complex pipelines such as Directed Acyclic Graphs (DAGs) \cite{piper} rather than the linear pipelines used in earlier systems.  
 
{\sysname} does learner and transformation selection, and hence is a component of an AutoML systems. To evaluate this component, we integrated it into two existing AutoML systems, FLAML \cite{flaml} and Auto-Sklearn \cite{autosklearn}.  
% We evaluate each system with and without {\sysname}.  
We chose FLAML because it does not yet have any meta-learning component for the cold start problem and instead allows user selection of learners and transformers. The authors of FLAML explicitly pointed to the fact that FLAML might benefit from a meta-learning component and pointed to it as a possibility for future work. For FLAML, if mining historical pipelines provides an advantage, we should improve its performance. We also picked Auto-Sklearn as it does have a learner selection component based on meta-features, as described earlier~\cite{autosklearn2}. For Auto-Sklearn, we should at least match performance if our static mining of pipelines can match their extensive database. For context, we also compared {\sysname} with the recent VolcanoML~\cite{VolcanoML}, which provides an efficient decomposition and execution strategy for the AutoML search space. In contrast, {\sysname} prunes the search space using our meta-learning model to perform hyperparameter optimization only for the most promising candidates. 

The contributions of this paper are the following:
\begin{itemize}
    \item Section ~\ref{sec:mining} defines a scalable meta-learning approach based on representation learning of mined ML pipeline semantics and datasets for over 100 datasets and ~11K Python scripts.  
    \newline
    \item Sections~\ref{sec:kgpipGen} formulates AutoML pipeline generation as a graph generation problem. {\sysname} predicts efficiently an optimized ML pipeline for an unseen dataset based on our meta-learning model.  To the best of our knowledge, {\sysname} is the first approach to formulate  AutoML pipeline generation in such a way.
    \newline
    \item Section~\ref{sec:eval} presents a comprehensive evaluation using a large collection of 121 datasets from major AutoML benchmarks and Kaggle. Our experimental results show that {\sysname} outperforms all existing AutoML systems and achieves state-of-the-art results on the majority of these datasets. {\sysname} significantly improves the performance of both FLAML and Auto-Sklearn in classification and regression tasks. We also outperformed AL in 75 out of 77 datasets and VolcanoML in 75  out of 121 datasets, including 44 datasets used only by VolcanoML~\cite{VolcanoML}.  On average, {\sysname} achieves scores that are statistically better than the means of all other systems. 
\end{itemize}


%This approach does not need to apply cleaning or transformation methods to handle different variances among datasets. Moreover, we do not need to deal with complex analysis, such as dynamic code analysis. Thus, our approach proved to be scalable, as discussed in Sections~\ref{sec:mining}.

\section{Switched System and Attack Model}
\label{sec:replay1}
\begin{figure}[b!]
\centering
\includegraphics [width=0.38\textwidth]{system6.pdf}
\vspace{-10pt}
\caption{Switching system diagram, where the system is equipped with $N_1$ controllers, $N_2$ estimators and $N_3$ detectors and switches among $N$ subsystems. A subsystem (controller $N_1$, estimator $N_2$, and detector $N_3$) is chosen here.} %attacks can change sensor measurements in this example.}
\label{system}
%\vspace{-10pt}
\end{figure}
%\FM{add: fault detection filter and detector here?}
We consider the CPS security problem when both the system and attacker have limited knowledge about the opponent. The system is equipped with multiple controllers/estimators/detectors, such that each combination of these components constitute a subsystem. A subsystem has a probability to detect specific types of attacks with different control and detection costs. To balance the security overhead and the control cost under various attacks, we consider switching among subsystems (choose a model for every component) according to the system dynamics and detector information. A switched system model is shown in Figure~\ref{system}, and the model of each component is described with a concrete example in the rest of this section. It is worth noting that the set of subsystems is not restricted and can be further generalized. %We assume that the controller/estimator/detector are not compromised, and attackers can not hack code implemented in these components in this work.

\textbf{LTI plant and sensor attack model}: Consider a class of LTI plants described by: % standard state-space form:
%
%\vspace{-3pt}
\begin{align}
\begin{split}
\mathbf{x}_{k+1}= \mathbf{Ax}_{k}+\mathbf{Bu}_{k}+\mathbf{w}_{k},\quad
\mathbf{y}_{k}= \mathbf{Cx}_{k}+\mathbf{v}_{k},
\end{split}
\label{system}
\end{align}
where $\mathbf{x}_{k} \in \mathbb{R}^{n}, \mathbf{u}_{k} \in \mathbb{R}^{p}$ and $\mathbf{y}_{k} \in \mathbb{R}^{m}$ denote the discrete time state, input and output vectors respectively, and $\mathbf{w}_{k}\sim \mathcal{N}(0,\mathbf{Q})$, $\mathbf{v}_{k}\sim  \mathcal{N}(0,\mathbf{R})$ are independent and identically distributed (IID) Gaussian random noise. The initial state is $\mathbf{x}_{0}\sim  \mathcal{N}(\bar{\mathbf{x}}_{0}, \Sigma)$. Sensors or the communication between sensors and estimators are vulnerable, and attacker can change values $\mathbf{y}_k$ that sent from sensors of system~\eqref{system}, and the compromised sensor measurements are defined as $\mathbf{y}'_k$ according to the types of attacks we consider. For instance, if the attacker can inject arbitrary data $\mathbf{y}^a_k$ to sensors, $\mathbf{y}'_k=\mathbf{y}_k+\mathbf{y}^a_k$; for replay attacks, the attacker can choose the replay window size $T_2$, let $\mathbf{y}'_{k} =\mathbf{y}_{k-T_2}$ and decide whether to send the delayed plant outputs at $k$. 

\textbf{Estimators}: % Kalman filter is widely applied for noisy systems. 
The physical dynamical state of the system is provided by an estimator, for instance, attack resilient estimator~(\cite{res_estimator}), $l_1$ norm state estimator~(\cite{resest_cdc15}), fault detection filter~\cite{fd_filter}, or the widely applied Kalman filter.
When $(\mathbf{A},\mathbf{B})$ is stabilizable, $(\mathbf{A},\mathbf{C})$ is detectable, a steady state Kalman filter exists. 



\textbf{Controllers}: A state feedback control law is described as $\mathbf{u}_k=L(\hat{x}_{k|k})$, where $L(\cdot)$ is a linear function, $\hat{x}_{k|k}$ is the estimated state. \cite{replay} increase the detection rate by adding an IID Gaussian signal $\Delta \mathbf{u}_{k} \sim \mathcal{N}(0, \mathcal{L})$ to $\mathbf{u}_k^*$ to an optimal LQG controller as $\mathbf{u}_{k} = \mathbf{u}^{*}_{k} + \Delta \mathbf{u}_{k} \label{nlqg}$, and increase the control cost. Then always applying the non-optimal controller for detecting a replay attack is not cost optimal, especially when there is no replay at all during a long time.

\textbf{Detectors}: We assume that every detector of the subsystem provides a detection rate for a specific type of attack, and a system is equipped with several detectors in order to deal with multiple types of attacks. Researchers have designed probabilistic detectors with respect to different attacks. For instance, \cite{fd_filter} design a fault detection filter, including a residual estimator and a threshold and a decision logic unit. Hypothesis testing strategies such as maximum likelihood (MLE), maximum a posteriori (MAP), and minimum mean square error (MMSE) account for GPS spoofing attack is presented by~\cite{GPS_spoof}. %By testing the statistical profile of a spoofing attack with methods  the probability and cost of detecting one spoofing attack are evaluated. 

%However, each method of hypothesis testing has the largest detection rate with respect to a different type of spoofing attack with some cost. We need to decide which detection strategy to use in order to secure the system with some probability and not cost too much. 

\textbf{Cyber state -- discrete modes of the system}:
We denote the modes of a vulnerable system as three constants $S=\{\delta_1,\delta_2, \delta_3\}$. State $\delta_{1}=safe$ describes that the system has already successfully detected an attack; $\delta_{2}= no~detection$ specifies that the alarm is not triggered; finally, the system enters state $\delta_{3}= false~alarm~trigger$ when the alarm is triggered while no attack has yet occurred. The mode depends on the probability detection rate. We assume that once the alarm is triggered, the system will stop the execution and decide whether to react to occurred attacks or it is a false alarm. %When the system is hijacked, the estimator, detector and controller are fed with false data, until an alarm is triggered and the system reacts to the attack.

%With different types of controllers or probabilistic detectors, it is necessary to introduce a new framework to balance the security overhead and system's performance. With system dynamics model, the cost of compromised by one type of attack and the payoff of detecting it or keeping resilient are also changing under different system states. A switching policy is then promising as a solution under this scenario. 

\iffalse
For the steady state Kalman filter, a $\chi^2$ detector triggers the alarm when the estimation residual is greater than the threshold with a specific false detection rate.
It is worth noting that even without the knowledge of system dynamics, a replay attacker can compromise values from all sensors. 
\fi

\iffalse
\vspace{-3pt}
\begin{align}
\begin{split}
\hat{\mathbf{x}}_{0|-1}&=\bar{\mathbf{x}}_{0}, \mathbf{P}_{0|-1}=\Sigma, \mathbf{P}_{k+1|k}=\mathbf{AP}_k\mathbf{A}^T+\mathbf{Q},\\
%\mathbf{P}_{k+1|k}&=\mathbf{AP}_{k|k}\mathbf{A}^{T}+\mathbf{Q},\mathbf{K}_{k}=\mathbf{P}_{k|k-1}\mathbf{C}^{T}(\mathbf{CP}_{k|k-1}\mathbf{C}^{T}+\mathbf{R})^{-1},\\
\mathbf{P}&=\lim\limits_{k\to\infty} \mathbf{P}_{k|k-1}, \mathbf{K}=\mathbf{PC}^{T}(\mathbf{CPC}^{T}+\mathbf{R})^{-1},\\
\hat{\mathbf{x}}_{k|k} &= \hat{\mathbf{x}}_{k|k-1}+\mathbf{K}(\mathbf{y}_{k}-\mathbf{C}\hat{\mathbf{x}}_{k|k-1}),\\
\hat{\mathbf{x}}_{k+1|k}&=\mathbf{A}\hat{\mathbf{x}}_{k|k}+\mathbf{Bu}_{k}.
%\mathbf{P}\triangleq &\lim_{k\to\infty}\mathbf{P}_{k|k-1}, \mathbf{K}=\mathbf{PC}^{T}(\mathbf{CPC}^{T}+\mathbf{R})^{-1}.
\end{split}
\label{eq:kf1}
\end{align}
\fi

\iffalse
\begin{align}
\begin{split}
\mathbf{y}'_{k}=\begin{cases} \mathbf{y}_{k},\ \ \text{sensor output is not changed at}\ k \\%\quad \quad \quad 0\le k \le T_{0}+T-1\\
                                  \mathbf{y}_{k-T_2},\ \text{replay attack occurs at}\ k.%\quad T_{0}+T\le k \le T_{0}+2T-1 
                                  \end{cases}
\end{split}
\label{replay_y}
\end{align}
For example, with Kalman Filter $\hat{\mathbf{x}}_{k|k}$ as a state estimator, an optimal LQG controller is described as $\mathbf{u}_k=\mathbf{ L \hat{x}}_{k|k}$. Here $L$ is a time invariant matrix satisfying $L\triangleq -(\mathbf{B}^{T}\mathbf{SB}+\mathbf{U})^{-1}\mathbf{B}^{T}\mathbf{SA}$, and $\mathbf{S}$ is the solution of the Riccati equation 
%\begin{equation*}
$\mathbf{S}=\mathbf{A}^{T}\mathbf{SA}+\mathbf{W}-\mathbf{A}^{T}\mathbf{SB}(\mathbf{B}^{T}\mathbf{SB}+\mathbf{U})^{-1}\mathbf{B}^{T}\mathbf{SA},$
%\end{equation*}
for some $\mathbf{W,U} \succ 0$ of corresponding dimensions.
\fi

%. Replay attack is undetectable for a subsystem with optimal LQG control, Kalman filter and $\chi^2$ detector. The expectation of residual signal will increase under replay attack, the quadratic cost also increase to $J'= J+trace{[(\mathbf{U}+\mathbf{B}^{T}\mathbf{SB})\mathcal{L}]}$,where $J$ denotes the cost for optimal LQG control input $\mathbf{u}^{*}_{k}$~(\cite{replay}). The probability of having a residual that greater than the alarm threshold value for a normal $\chi^2$ detector is also increased under replay attacks.

\section{A Hybrid Stochastic Game Model}
\label{sec:game_form}
To obtain a switching policy that minimizes the expected real-time worst case payoff for the given subsystems, 
we formulate a zero-sum, hybrid stochastic game between the system and the attacker. System dynamics knowledge are combined with the game definition, and the quantitative process for the game parameters will be introduced in this section. We assume that one game stage $k$ is also one time step of the physical system. The total stage number is $K$. The hybrid game state space $(X_{[k-T,k]}\times S)$ contains information about both the system dynamics $\mathbf{x}_k$ and the discrete modes $\delta_l, l=1,2,3$. Here, $T$ is the window size of system dynamics needed to keep the state transition between stages $k$ and $(k+1)$ Markov. The joint state includes information we need to compute the game strategy at the current stage. This is the main difference compared with the previous work~(\cite{cdc_replay}), while the latter is not Markov since it needs to consider all the possible histories of strategies for deciding the physical dynamics and getting a strategy. At each stage $k \in \{T,\cdots, K+T\}$, parameters include the action space for the attacker (system) $A_{t}$ ($A_{s}$), the state transition probability matrix $\mathbb{P}_{k}$, and the immediate payoff matrix $r_{k}$. The solution set of the game is mixed strategies $\mathbf{F}_{k}$ for the attacker, and $\mathbf{G}_{k}$ for the system. Formally, the game is defined as a sequence of tuples:
 $\{(X_{[k-T,k]} \times S),A_{t},A_{s}, \mathbf{F}_{k},\mathbf{G}_{k}, P, r\}$.

\iffalse
\begin{table*}
\centering
%\caption{Parameters of the hybrid stochastic game between the system and the attacker}
\begin{tabular}{|c|c|}
  \hline
   $s_{kl}=(x_{[k-T,k]}, \delta_l)$& Joint game state: sequence of physical dynamics, and piecewise constant mode \\ \hline
   $A_{t}$ & Attacker's action space \\ \hline
   $A_{s}$ & System's action space \\ \hline  
   $\mathbf{f}_k(s_{kl})$ & Strategy of the attacker at stage $k$, 
                                           state $s_{kl}, l=1,2,3$  \\ \hline     
      $\mathbf{g}_k(s_{kl})$ & Strategy of the system at stage $k$, 
                                            state $s_{kl}, l=1,2,3$  \\ \hline     
    ${P}(s_{(k+1)h}|s_{kl})$ & Probability that system transits from state $s_{kl}$ 
                                              at stage $k$ to state $s_{(k+1)l}$ at stage $k+1$\\ \hline
    ${r}(s_{kl})$ & Immediate payoff matrix at stage $k$ \\ 
   \hline
\end{tabular}
\centering
%\captionsetup{justification=centerlast}
\caption{Parameters of the hybrid stochastic game between the system and the attacker}
\label{game_parameter}
\end{table*} 
\fi

\textbf{Game State Space}: The joint state of the system at stage $k$ is described by the pair $s_{kl}=(x_{[k-T,k]}, \delta_l)$, where
\centerline{$
x_{[k-T,k]}=(x_{k-T}, x_{k-T+1}, \cdots, x_k ) \in X_{[k-T,k]}$} is the discrete-time dynamics of the physical process provided to the system--the state estimations $\hat{x}_{k-T},\cdots, \hat{x}_k$, $\delta_l \in S=\{\delta_1,\delta_2, \delta_3\}$ denote the cyber state of the system. We assume that once the game reach $\delta_1$, the system wins and will not enter other modes till next game, i.e., $\delta_1$ is an absorbing state. The moving-horizon transition of the joint states on stage axis is shown as Figure~\ref{sg}. The window size of system dynamics $T$ keeps the state transition between time $k$ and $k+1$ Markov. For instance, if the detector of the system requires system dynamics $\hat{x}_{[k-T_1,k]}$, and we consider sensor data injection attacks and replay attacks with replay windows less than $T_2$ steps, then $T=max\{T_1, T_2\}$. 
%With the system dynamics $\hat{x}_{[k-T, k]}$ denoted as $x_{[k-T,k]}$ for game stage $k$,  information needed to define the following action space of two players, the payoff and state transition probability is included. 

%i.e., the detector needs information for T steps to decide whether the alarm should be triggered. 
%\FM{In replay attack, we can say: once alarm is triggered, the system can stop the execution and check whether attack occurred, is this true for other attacks? Can the system distinguish between successfully detection and false alarm trigger?}
\begin{figure}[t!]
%\vspace{-5pt}
\centering
\includegraphics [width=0.32\textwidth]{xk.pdf}
\vspace{-8pt}
\caption{Joint state transition of the hybrid stochastic game when moving the horizon of game state one step ahead. When the state transits from stage $k$ to $k+1$, we slice the window of the sequence of physical dynamics one step ahead, add $x_{k+1}$ and remove $x_{k-T}$,  thus $x_{[k-T,k]} \to x_{[k-T+1,k+1]}$. The piecewise constant modes $\delta_l$, $\delta_h$ describe the cyber states provided by the detector at stage $k$, respectively.}
\label{sg}
\vspace{-5pt}
\end{figure}

%For simplicity, we omit the subscript k and just write state at every time k as $s_i,  i=1,2,3$.
\textbf{Attacker's Action Space}: We assume that the system is vulnerable to different attack models described by the action space $A_{t}$, where 
\\\centerline{$
A_{t}=\{a_{1}(x_{[k-T,k]}), a_{2}(x_{[k-T,k]}), \cdots, a_{M}(x_{[k-T,k]})\}
$}
is the attacker's action space at stage $k$, and $a_{1}$ means no attack. Here we only consider discretized action space of the attacker for computational efficiency. For the LTI system dynamics considered in this work, the distance of a continuous point to its nearest discrete point in action space is bounded. With bounded error of the dynamics by discretized continuous action space, the quality of game solutions under different conditions is analyzed by work~\cite{disaction}. 

The actions can describe both multiple types of attacks and the same type attack with different values. For instance, when considering only sensor data injection attacks with different norms of injection value, we will denote $a_i (x_{[k-T,k]}), i=2, 3,\dots$ as changing the sensor value from $\mathbf{y}_k=\mathbf{Cx}_k+\mathbf{v}_k$ to $\mathbf{y}'_k= \mathbf{y}_k+\mathbf{y}_{k,i}^a$, where any injection $\mathbf{y}_k^a$ is classified as $a_i (x_{[k-T,k]}), i=inf\{i:\ \|\mathbf{y}_k^a-\mathbf{y}_{k,i}^a\|_2\}$ in attacker's action space.
Similarly, for replay attack only, the action space is discretized as changing sensor values from $\mathbf{y}_k=\mathbf{Cx}_k+\mathbf{v}_k$ to $\mathbf{y}'_k= \mathbf{y}_{k-T_i}$ for action index $a_i (x_{[k-T,k]})$, where any replay time length $T_a$ is classified as $a_i (x_{[k-T,k]}), i=inf\{i:|T_a - T_i|\}$. Considering multiple types of attacks, we assume that the system is valnerable under $m_a$ types of attacks, and attack type $A_i$ is corresponding to $M_{a,i}$ discretized actions in the action space, then there are $\sum_{i=1}^{m_a} M_{a,i}+1$ actions in total within the attacker's action space $A_t$.
 

% with discretized, bounded norm $\|\mathbf{y}_{k,i}^a\|_2 \leqslant b$, since the attacker has limited energy for every data injection. This means any injection data that satisfies $\|\mathbf{y}_{k}^a\|_2 \leqslant b$ is considered as $inf\{i:\ \|\mathbf{y}_k^a-\mathbf{y}_{k,i}^a\|_2\}$ in attacker's action space. 

%For example, 
%when considering replay attacks and false data injection attacks, we take $a_{2k}$ as~\eqref{replay_y} for a given replay window size, and $a_{3k}$ as~\eqref{attackmodel} for a given data injection range. 

\iffalse by any given controller/estimator/detector combination of the system \fi
%; here, $y_{k}$ is the real sensor value and $\mathbf{y}_{k-t_{i}}$ denotes any replay sensor value in the strategy set. %\FM{here we assume during time $k \in \{1,...,K\}$ the replay window size $T$ does not change for simulation.}
%\item
%\FM{rewrite the action space definition}

\textbf{System's Action Space}: The system's action space at stage $k$ is defined as
\\\centerline{$
A_{s}=\{u_{1}(x_{[k-T,k]}), u_{2}(x_{[k-T,k]}),\cdots, u_{N}(x_{[k-T,k]})\},
$}  %= \{\mathbf{u}^{*}_{k}, \mathbf{u}^{*}_{k} + \Delta \mathbf{u}_{k}\}$
where $u_{j}$ is the index for the $j$th subsystem. We assume that the $N$ subsystems (a model for each component in Figure~\ref{system}) are determined priorly. For example, a subsystem can be the plant with a given optimal LQG controller, a Kalman filter and a $\chi^2$ detector. A subsystem can also be the plant with an optimal LQG controller, a resilient state estimator~\cite{res_estimator} and its corresponding estimation residual checking component. We assume that the attacker's action space is defined, with corresponding system's action or a subsystem that the detection rate is greater than $0$. A switched system does not ensure performance under the attack outside the action space of the game.

\textbf{Mixed Strategy}: Let $f^{i}_{k}(s_{kl})$ ($g^{j}_{k}(s_{kl})$) be the probability that the attacker (system) chooses action $a_{i}(x_{[k-T,k}) \in A_{t}$ ($u_{j}(x_{[k-T,k}) \in A_{s}$) at state $s_{kl}\in (X_{[k-T,k]}\times S)$. Define $\mathbf{F}_{k}$ and $\mathbf{G}_{k}$ as the mixed strategy sets of the attacker and the system for stage $k$:
$\mathbf{F}_{k} :=\{\mathbf{f}_{k}= [\mathbf{f}_{k}(s_{k1}), \mathbf{f}_{k}(s_{k2}), \mathbf{f}_{k}(s_{k3})]
|f_{k}^{i}(s_{kl})\geq 0, \mathbf{f}_k \in [0,1]^{M\times 3},
\sum \limits_{a_{ik} \in A_{tk}}f_{k}^{i}(s_{kl}) = 1,\mathbf{f}_{k}(s_{kl})\in \mathbb{R}^{M}, \forall s_{kl} \in(X_{[k-T,k]}\times S)\},$
$\mathbf{G}_{k}:=\{\mathbf{g}_{k}= [\mathbf{g}_{k}(s_{k1}), \mathbf{g}_{k}(s_{k2}), \mathbf{g}_{k}(s_{k3})]|$
$g_{k}^{j}(s_{kl})\geq 0, \mathbf{g}_k \in [0,1]^{N \times 3}, %&\forall u_{jk} \in A_{sk},%k \in \{1,...,K\},\\
\sum \limits_{u_{jk} \in A_{sk}}g_{k}^{j}(s_{kl}) = 1, \mathbf{g}_{k}(s_{kl}) \in \mathbb{R}^{N}, \forall s_{kl} \in (X_{[k-T,k]}\times S)\}. $ Note that $\mathbf{x}_{[k-T,k]}$ provides exogenous information for the strategy $\mathbf{f}_k (\mathbf{g}_k)$, since for every $l$, $\mathbf{f}_{k}(s_{kl}) (\mathbf{g}_{k}(s_{kl}))$ is the strategy at mode $\delta_l$ for the same $\mathbf{x}_{[k-T,k]}$ at stage $k$. Hence, $\mathbf{g}_k$ and $\mathbf{f}_k$ are finite dimensional vectors, that the stationary strategy chosen by each player at stage $k$ depends on the cyber state. %Mixed strategy set $\mathbf{F}_k$ also include the case that the attacker only implement one specific type of attack in the action space at time instance $k$, since we do not have know the strategy of the attacker, we are able to consider all possible combinations of attacks by exploiting mixed strategies. 

%\textbf{}:
%\label{dynamicgame}
%With all the above definition, %for any strategy history $h_{k}$ (a sequence of switching policy), 


\textbf{System and Subsystem Dynamics under game framework}: Given the subsystem and attack models in Section~\ref{sec:replay1} and the game definition,  
%we can transform it to a new system dynamic model decided by both players' actions. 
we show the dynamics at stage $k$ given an action pair $(a_{i}(x_{[k-T,k}),u_{j}(x_{[k-T,k}))$ (assume initial $\mathbf{\hat{x}}_{1|0}=\bar{\mathbf{x}}_{0}$, $\mathbf{x}_{1}=\mathbf{x}_{0}$). Each action pair $(a_{i}(x_{[k-T,k]}),u_{j}(x_{[k-T,k]}))$ defines the corresponding system dynamics at $k$. For instance, when we focus on sensor attacks (like replay or false data injection), let $\mathbf{\gamma}_{k}(a_{i}(x_{[k-T,k]}), u_{j}(x_{[k-T,k]}))$ be the control input with $(a_{i}(x_{[k-T,k]}),u_{j}(x_{[k-T,k]}))$, a subsystem $u_{j}(x_{[k-T,k]})$ with a Kalman filter, an optimal LQG controller has the following dynamics (we denote $(a_{i}(x_{[k-T,k]}),u_{j}(x_{[k-T,k]}))$ as $(a_{ik}, u_{jk})$ for convenience): 
\begin{align}
\begin{split}
&\mathbf{x}_{k}=\mathbf{Ax}_{k-1}+ \mathbf{Bu}_{k-1}+\mathbf{w}_{k-1},\\
& \mathbf{y}_{k}=\begin{cases}a_{1k} = \mathbf{Cx}_k+\mathbf{v}_{k},\ \text{without attack}\\
a_{ik}, i=2,\cdots, M, \ \ \text{with attack,} \end{cases}\\
&\hat{\mathbf{x}}_{k|k-1}= \mathbf{A\hat{x}}_{k-1|k-1}+\mathbf{Bu}_{k-1},\\
%&\mathbf{z}_{k+1}(h_{k},a_{ik},u_{jk})=a_{ik}(h_{k}) - \mathbf{C\hat{x}}_{k+1|k}(h_{k},a_{ik},u_{jk}),\\
&\hat{\mathbf{x}}_{k|k}(a_{ik}) =\hat{\mathbf{x}}_{k|k-1}+ \mathbf{K}(a_{ik} - \mathbf{C\hat{x}}_{k|k-1}),\\
%\end{split}
&\mathbf{\hat{x}}_{k+1|k}(a_{ik},u_{jk})=\mathbf{A\hat{x}}_{k|k}(a_{ik})+\mathbf{B\gamma}_{k}(a_{ik},u_{jk}),\\
& \mathbf{\gamma}_{k}(a_{ik}, u_{jk}) = \mathbf{L\hat{x}}_{k|k} (a_{ik}),\\%+\Delta \mathbf{u}_{k},\\
&\mathbf{z}_{k+1}(a_{ik},u_{jk})=a_{ik} - \mathbf{C\hat{x}}_{k+1|k}(a_{ik},u_{jk}).
\label{dynamicgame}
\end{split}
\end{align}
\textbf{State Transition Probability}: Given a set of subsystem models, define the state transition probability $P$ as a function of the state of the game and both players' actions $P:\ (X_{[k-T,k]}\times S) \times A_{t} \times A_{s}\to [0, 1],$
where
\\\centerline{$
P(s_{(k+1)h}|s_{kl},a_{ik}, u_{jk}), h=1,2,3
$}
%\end{align*}
is the probability that system transits from state $s_{kl}$ to state $s_{(k+1)h}$ at stage $k+1$, given both players' action $(a_{ik},u_{jk})$ at stage $k$. Given the current game state $s_{kl}=(x_{[k-T,k]}, \delta_l)$ and an action pair $(a_{ik},u_{jk})$, the dynamics of the system at stage $k+1$ is described as $x_{[k-T+1,k+1]}$ for all possible cyber modes $\delta_h \in S$, hence the dimension of state transition probability $P(s_{(k+1)h}|s_{kl},a_{ik}, u_{jk})$ is determined by the number of cyber modes of the game. We denote $P(s_{(k+1)h}|s_{kl}, a_{ik}, u_{jk})$ as $P^{ij}(s_{(k+1)h}|s_{kl})$ for short.
%and $\tilde{P}^{ij}(s_{(k+1)h}|s_{kl})$ is the entry at the $i$-th row and $j$-the column  of the state transition matrix $\tilde{P}(s_{(k+1)h}|s_{kl})$ of the game at hybrid state $s_{kl}$.
 As a state transition probability, this function should also satisfy
%\begin{align*}
\\\centerline{$\sum_{\delta_h \in S} {P}^{ij}(s_{(k+1)h}|s_{kl}) = 1,\quad \forall (a_{ik},u_{jk}) \in A_{t} \times A_{s},$}
\\\centerline{$s_{(k+1)h} \in (X_{[k-T+1,k+1]}\times S), s_{kl} \in(X_{[k-T,k]}\times S).$}
%\end{align*}
The transition probability is provided by intrusion detectors of the subsystem. 
%For computational efficiency, we assume that every element of the state transition matrix is a convex function of the system dynamics $x_{[k-T,k]}$ or can be convexified with bounded error. 
%For example, if a $\chi^{2}$ detector is the detector component of subsystem $u_{j}$, we apply~\eqref{alarm} to decide the state transition probability.

\textbf{Immediate Payoff Function}: The immediate payoff matrix at stage $k$ is a $\mathbb{R}^{M\times N}$ matrix for given game state and every action pair $(a_{ik}, u_{jk})$. We define the immediate payoff function as a continuous, convex function of the hybrid game state and the actions of both players
\\\centerline{$r: (X_{[k-T,k]}\times S) \times A_{t} \times A_{s} \to \mathbb{R}^{M \times N},$}
where $r(s_{kl}, a_{ik}, u_{jk}) \geqslant 0$ is the payoff at joint state $s_{kl}$ given action pair $(a_{ik}, u_{jk})$. For definition convenience, we denote ${r}(s_{kl}, a_{ik}, u_{jk})$ as ${r}^{ij}(s_{kl})$ for short, since it is the element on the $i$-th row and $j$-th column of the payoff matrix ${r}(s_{kl})$. It is a zero-sum game between the system and the attacker, and we assume the system is the minimizer and the attacker is the maximizer, hence the payoff function for the attacker and the system is defined as
\centerline{$
{r}^{ij}(s_{kl})={r}_t^{ij}(s_{kl})=-{r}_s^{ij}(s_{kl}).
$}
For instance, when the linear quadratic cost is a metric of system performance, let $\gamma_{k}(a_{ik}, u_{jk})$ be the control input given action pair $(a_{ik}, u_{jk})$, then the payoff function is defined as
\begin{align}
\begin{split}
{r}^{ij} (s_{k1}) =&\mathbb{E}[\mathbf{\hat{x}}^{T}_{k}]\mathbf{W}\mathbb{E}[\mathbf{\hat{x}}_{k}]+\mathbb{E}[\mathbf{\gamma}^{T}_{k}(a_{1k},u_{jk})]\mathbf{U}\mathbb{E}[\mathbf{\gamma}_{k}(a_{1k},u_{jk})],\\
{r}^{ij} (s_{k2}) =&\mathbb{E}[\mathbf{\hat{x}}^{T}_{k}]\mathbf{W}\mathbb{E}[\mathbf{\hat{x}}_{k}]+\mathbb{E}[\mathbf{\gamma}^{T}_{k}(a_{ik},u_{jk})]\mathbf{U}\mathbb{E}[\mathbf{\gamma}_{k}(a_{ik}, u_{jk})],\\
{r}^{ij} (s_{k3}) =& p_f,
\end{split}
\label{payoff}
\end{align}
where $p_f$ is the false alarm trigger penalty, the cost that the system needs to stop execution, check the reason of an alarm, and restart later; $\mathbf{x}_{k}$ is the physical state under the game framework. At mode $\delta_{1}$ the system wins, so the payoff is a normal system payoff with correct sensor data. The larger $p_f$ is, the less probable it is for the system to choose a strategy to transit to state $s_{k3}$.

\textbf{System dynamics update with strategies at stage k}:
 Let $p(s_{kl})$ be the probability system is at state $s_{kl}$ at stage $k$. The initial state distribution $p(s_{1l})$ is given. With  a strategy $\mathbf{f}_{k},\mathbf{g}_{k}$, the attacker and the system randomly sample an action pair $(a_{ik}, u_{jk})$ according to the probability distribution. Then, the control input and sensor value for calculating expectation cost are: 
%\begin{align*}
\centerline{$
\mathbf{u}_{k}=\sum\limits_{j=1}^{N}\sum\limits_{i=1}^{M} \sum\limits_{l=1}^{3}p(s_{kl})f_{k}^{i}(s_{kl})g^{j}_{k}(s_{kl})\mathbf{\gamma}_{k}(a_{ik},u_{jk}),
$}
$\text{ }\quad\quad\mathbf{y}_{k} =\sum\limits_{i=1}^{M}\sum\limits_{l=1}^{3} p(s_{kl})f_{k}^{i}(s_{kl}) a_{ik}.$
%\end{align*}
\\The probability that system is at state $s_{(k+1)h}$ for $k+1$ is:
\\\centerline{$
%\begin{align*}
p(s_{(k+1)h})= \sum\limits_{l=1}^{3}p(s_{kl})[\mathbf{f}_{k}(s_{kl})]^{T}{P}_{k}(s_{(k+1)h}|s_{kl})\mathbf{g}_{k}(s_{kl}). 
%\end{align*}
$}

\iffalse
\begin{align*}
\mathbf{F}_{k} :=\{&\mathbf{f}_{k}= [\mathbf{f}_{k}(s_{k1}), \mathbf{f}_{k}(s_{k2}), \mathbf{f}_{k}(s_{k3})]
|f_{k}^{i}(s_{kl})\geq 0,\\& \mathbf{f}_k \in [0,1]^{M\times 3} %&\forall a_{ik} \in A_{tk},  %k\in \{1,...,K\},\\
\sum \limits_{a_{ik} \in A_{tk}}f_{k}^{i}(s_{kl}) = 1,\mathbf{f}_{k}(s_{kl})\in \mathbb{R}^{M},\\&\forall s_{kl} \in(X_{[k-T,k]}\times S)\},\\
\mathbf{G}_{k}:=\{&\mathbf{g}_{k}= [\mathbf{g}_{k}(s_{k1}), \mathbf{g}_{k}(s_{k2}), \mathbf{g}_{k}(s_{k3})]|
g_{k}^{j}(s_{kl})\geq 0,\\& \mathbf{g}_k \in [0,1]^{N \times 3}, %&\forall u_{jk} \in A_{sk},%k \in \{1,...,K\},\\
\sum \limits_{u_{jk} \in A_{sk}}g_{k}^{j}(s_{kl}) = 1, \mathbf{g}_{k}(s_{kl}) \in \mathbb{R}^{N},\\ &\forall s_{kl} \in (X_{[k-T,k]}\times S)\}. 
\end{align*} 
\fi

\section{Existence of An Optimal Strategy and Suboptimal Algorithm for A Finite Game}
\label{sec:algorithm_finite}

Based on the game formulation, in this section we discuss the existence of an optimal solution for the finite form of the hybrid stochastic game, and present an algorithm to compute a suboptimal system strategy.%\PA{what do you mean by ``approximation computing algorithm''}
 
\subsection{Existence of the System's Optimal Strategy}

%\iffalse 
We define the concatenation of strategies for $K$-stage game of each player ($\mathbf{f}$ for attacker and $\mathbf{g}$ for system) as $\mathbf{f}=\mathbf{f}_1\cdots \mathbf{f}_K,\quad\mathbf{f}_{k} \in \mathbf{F}_{k},\quad\mathbf{f}\in \mathbf{F},$
$\mathbf{g}=\mathbf{g}_{1}\cdots\mathbf{g}_K,\quad\mathbf{g}_{k} \in \mathbf{G}_{k},\quad\mathbf{g}\in\mathbf{G}$,
$k=1, 2,\dots,K$.
\begin{defn}
Let the random variable $\zeta_{k}$ describe the discrete state of the hybrid game at stage $k$, we define the conditional expected total payoff till $\tilde{K}$ for any $\mathbf{f},\mathbf{g}$ as%\begin{align*}
\\$\text{ }\quad R_{\tilde{K}}(s, \mathbf{f}, \mathbf{g})$
\\\centerline{$
=\sum \limits^{\tilde{K}}_{k=1}\sum_{l=1}^{3} p(\zeta_{k}=\delta_{l}|\zeta_{1} = s)[\mathbf{f}_{k}(s_{kl})]^{T}\tilde{r}_{k}(s_{kl})\mathbf{g}_{k}(s_{kl}),
$}
where $p(\zeta_{k}=\delta_{l}|\zeta_{1} = s)$ is the probability that the discrete state of the hybrid game is $\delta_{l}$ at stage $k$ given its initial discrete state $\zeta_{1}=s$.
\label{R_K}
\end{defn}
Since the immediate payoff of each stage satisfies that $0 \leq \tilde{r}_{k}^{ij}(s_{kl})< \infty,\ \text{for all}\ k,i,j,$
we have that $R_{\tilde{K}}(s, \mathbf{f}, \mathbf{g})$ is a nonnegative real-valued, nondecreasing function with $\tilde{K}$. 
%\PA{when you cut an old sentence to save space, make sure that the new one makes sense; for example you should have said - from the immediate payoff definition we have that ...' or ``since...''}  
Furthermore, for finite $K$%a finite stage game
%\PA{shouldn't it be ``for a finite game'' or ``for a finite stage game''}
%\vspace{-4pt}
\begin{align}
%\sup_{\mathbf{f} \in F^{K}}R_{K}(s,\mathbf{f}, \mathbf{g}^{*}) < \infty,s \in S.
%\sup_{\mathbf{f}}
R_{K}(s,\mathbf{f}, \mathbf{g}) < \infty, \forall s \in S, \mathbf{f}\in \mathbf{F}, \mathbf{g} \in \mathbf{G}.
\label{fR}
\end{align}
Similarly as the definition of value and optimal strategy for a zero-sum, finite discrete state, finite stage stochastic game, we define the value and optimal strategy for the hybrid state stochastic game defined in this work as the following.
\begin{defn}
A two-person zero-sum $K$-stage stochastic game is said to have a value vector $v^{*}_{K}$ if $v^{*}_{K,s}=\underbar{v}_{K,s}=\bar{v}_{K,s},$ for any initial cyber state $s\in S$, where
\\\centerline{$
\underbar{v}_{K,s}= \sup_{\mathbf{f}\in\mathbf{F}}\inf_{\mathbf{g}\in\mathbf{G}}R_{K}(s,\mathbf{f}, \mathbf{g}),
$}
\centerline{$
\bar{v}_{K,s}=\inf_{\mathbf{g}\in\mathbf{G}}\sup_{\mathbf{f}\in\mathbf{F}}R_{K}(s,\mathbf{f},\mathbf{g}).
$}
For the finite value $K$-stage stochastic game, strategies $\mathbf{g}^{*}$ and $\mathbf{f}^{*}$ are called optimal at the saddle-point equilibrium for player two (the system) and player one (the attacker), respectively, if for all $s\in S$,
\centerline{$
v^*_{K,s} = \inf\limits_{\mathbf{g}\in \mathbf{G}}R_{K}(s, \mathbf{f}^{*}, \mathbf{g}),\quad v^*_{K,s} = \sup\limits_{\mathbf{f}\in \mathbf{F}}R_{K}(s, \mathbf{f}, \mathbf{g}^{*}).
$}
\end{defn}
%The existence conditions of the value and optimal strategies for a general finite horizon, finite state, zero-sum stochastic game are. 
The game defined in this paper has finite action spaces, finite strategy space, finite discrete cyber modes and satisfies~\eqref{fR} with bounded total payoff in finite horizon. Therefore, there exists the value of the considered  game and an saddle-point equilibrium or optimal strategy for the system shown in~\cite{dgt_Basar}.
\subsection{Suboptimal algorithm for the finite game}
Existing value iterative algorithms or dynamic programming algorithms for finite stochastic games cannot be used to solve the finite hybrid stochastic game defined in this work, since the discrete time dynamics $x_{[k-T,k]}$ of the game at stage $k$ depends on that of the stage $k-1$, which is only available in the future algorithm iterations. Hence, we design a suboptimal algorithm based on the value iteration method for a finite horizon, finite discrete state stochastic game~(\cite{plangame}) and robust game techniques~(\cite{RGT}). The value iteration algorithm for a finite horizon, discrete state stochastic game (with fixed payoff $r$ and state transition probability $P$ at every stage) works in the way that if a player knew how to play in the game optimally from the next stage on, then, at the current stage, he would play with such strategies. The value of $K$-stage game is finally provided by the last step of iteration. 



For a multi-stage game, to calculate the game value, we define the auxiliary matrix at stage $k$ for every cyber state $\delta_l$ with system dynamics $x_{[k-T,k]}$ as $Q(s_{kl}) \in \mathbb{Q}_{k} \subset \mathbb{R}^{M \times N}$, and each element of $Q(s_{kl})$ for action pair $(a_{ik}, u_{jk})$ is defined as
\begin{align}
\begin{split}
&Q^{ij}(s_{kl})\\
=&r^{ij}(s_{kl})
                         +\sum_{\delta_h\in S} {P}^{ij}(s_{(k+1)h} |s_{kl})\cdot {v}_{k+1}(s_{(k+1)h}), 
\end{split}
\label{Q_k}
\end{align} 
where ${v}_{k+1}(s_{(k+1)h})$ is the game value from stage $k+1$, state $s_{(k+1)h}$ (with cyber mode $\delta_h$) to the final stage $K$. For the final stage $K$, we define $Q(s_{Kl})=r(s_{Kl})$. We define a one-shot game at stage $k$ as a finite action space, zero-sum game between the system and the attacker with payoff matrix $Q(s_{kl})$, i.e., $Q^{ij}(s_{kl})$ is the payoff for action pair $(a_{ik},u_{jk})$ of stage $k$. In each one-shot game, the system only consider a strategy $f_k(s_{kl})$ to minimize the worst case payoff caused by the attacker according to matrix $Q(s_{kl})$. Here $Q(s_{kl})$ is defined based on the the system dynamics and the state transition probability provided by the detector. An alternative algorithm with unknown transition matrix or payoffs will be our future work.

Similarly as the value iteration algorithm for a discrete state stochastic game~(\cite{plangame}), Algorithm~\ref{finite} of the finite hybrid state stochastic game starts from the last stage, then gets the optimal one-stage strategy and the upper bound of game value at each stage. By calculating values of all stages until backwards to the first stage, Algorithm~\ref{finite} returns an upper bound for the value of the total payoff in $K$-stages. 

To estimate the values at each step, we consider the immediate payoff $r(s_{kl})$, the state transition probability ${P}(s_{(k+1)h}|s_{kl})$ and the game value estimated at the previous step uncertain parameters for the one shot robust game (\cite{RGT}). Then approximate each iteration value as the value of the robust one shot zero sum game. 
Algorithm~\ref{finite} provides an upper bound for the game value and the corresponding suboptimal strategy for the system. The idea is to solve a robust game at each iteration step -- i.e., minimize the worst-case caused by extreme points of the set of auxiliary matrix $\mathbb{Q}_k$ defined for all possible dynamics $x_{[k-T,k]}$. 


To quantify the boundary of the set of auxiliary matrix $\mathbb{Q}_k$ %($r_{k}$, $\mathbb{P}_{k}$), 
we need the expected values of system dynamics $\mathbf{x}_{k}, \mathbf{u}_{k}$, $\mathbf{y}_{k}, k=1,\cdots,K$ defined in equations~\eqref{dynamicgame}, which is determined by the strategies from stage $1$ till stage $k$. 
We first analyze the uncertain sets of the immediate payoff function at stage $k$, and the extreme points for the uncertain set $\mathbb{Q}_k$ depend on pure strategies . Let $\mathbf{f}^p_{k-1}$, $\mathbf{g}^p_{k-1}$ be the concatenation of previous pure strategies of the attacker and the system till stage $k \geqslant 2$, respectively, where
\\\centerline{
$\mathbf{f}^{p}_{k-1}=\mathbf{f}^{p}_{1}\cdots\mathbf{f}^{p}_{k-1},\quad  \mathbf{g}^{p}_{k-1}=\mathbf{g}^{p}_{1}\cdots\mathbf{g}^{p}_{k-1}$
}
satisfies that all $\mathbf{f}^{p}_{t}(s)$ ($\mathbf{g}^{p}_{t}(s)$) for $t=1,2,\dots, k$ have only one non-zero element, i.e., the player chooses the corresponding action or the \emph{pure} strategy.

Define a pure strategy auxiliary matrix $Q^p(s_{kl}) \in \mathbb{Q}^p_{k}$ as:
\begin{align}
\begin{split}
%\vspace{-5pt}
&Q^{p}(s_{kl}) \\
=&r^{p}(s_{kl})+\sum_{\delta_h\in S} {P}^{p} ( s_{(k+1)h} |s_{kl})\cdot \bar{v}^p_{k+1}(s_{(k+1)h}),
\label{eq:Q}
\end{split}
\end{align}
%\normalsize
for stages $k=1,\dots,K-1$, and for the final stage $k=K$,
\begin{align}
Q^{p}(s_{Kl})=r^{p}(s_{Kl}).
\label{Q_p}
\end{align}
For each stage $k$, $\bar{v}^p_{k}(s_{kl})$ is defined as
% and relates to matrix games defined by $\mathbb{Q}_{(k+1)p}$.
\begin{align}
\bar{v}_{k}^{p}(s_{kl})=\max_{Q^{p}(s_{kl})\in \mathbb{Q}^p_{k}}v^*[Q^{p}(s_{kl})],%v_{k}^{s_{l}}(h_{k}^{p})&=\max_{(r^{p}_{k}(h^{p}_{k},s_{l}), P^{p}_{k}(h^{p}_{k},s_{l}))}
%v^*[r^{p}_{k}(h^{k}_{p},s_{l})+\sum_{s'\in S} P^{p}_{k}( s' |h^{p}_{k},s_{l})v_{k+1}^{s'}(h^{p}_{k+1})],
\label{pickv}
\end{align}
where $v^*$ is the function that yields the value of a zero-sum matrix game. Then the value $\bar{v}^p_{k+1}(s_{(k+1)h}) \geq 0$ to calculate the auxiliary matrix~\ref{eq:Q} is the upper bound of robust game value from stage $k+1$ till stage $K$, resulting from the iteration at stage $k+1$. This value iteration process is the key idea of the following~Algorithm~\ref{finite}. 

\begin{alg}
%\vspace{-5pt}
\textbf{: Suboptimal Algorithm for A Finite Hybrid Stochastic Game}\\
\textbf{Input}: System model parameters and game parameters.
\\\textbf{Initialization}:
             Compute the set of $\mathbb{Q}^p_k$ for every stage $k=T,\dots,T+ K$ given $\hat{\mathbf{x}}_{[0,T]}$;
              %For all $s_{l} \in S, l =1,2,3, h^{p}_{K}\in H^{p}_{K} :$ get %$4^{K-1}$ 
              %backup matrix set $\{Q_{k}^{p}(h_{k}^{p},s_{l})\}$:
             get the robust game value and corresponding strategies at stage $K$: $Q^{p}(s_{(K+T)l})=r^{p}(s_{(K+T)l})$,
                    $f^{*}(s_{(K+T)l}), g^{*}(s_{(K+T)l}), \bar{v}_{K+T}^{p}(s_{(K+T)l}) \leftarrow \pi(Q^{p}(s_{(K+T)l})).$
\\\textbf{Iteration}: For $k=(K+T-1), \cdots, T$, obtain a set of auxiliary matrices $\mathbb{Q}_{k}^{p}$ for all $\mathbf{f}^{p}_{k}$, $\mathbf{g}^p_k$, where each matrix is defined in~\eqref{eq:Q}, then calculate:
%\[Q_{k}^{p}(h^{p}_{k},s_{l}) = r^{p}_{k}(h^{p}_{k},s_{l})+\sum_{s'\in S} P^{p}_{k}( s' |h^{p}_{k},s_{l})v_{k+1}^{s'}(h^{p}_{k+1}),\]

 $f^{*}(s_{kl}), g^{*}(s_{kl}),\bar{v}_{k}^{p}(s_{kl}) \leftarrow \pi(Q^{p}(s_{kl}))$.
 
%$\mathbf{f}^*_k=[\mathbf{f}^{*}(h^{p}_{k}, s_{l}),\mathbf{f}^{*}(h^{p}_{k}, s_{2}), \mathbf{f}^{*}(h^{p}_{k}, s_{3})],$ $ \mathbf{g}^{*}_k=[\mathbf{g}(h^{p}_{k}, s_{1}),\mathbf{g}(h^{p}_{k}, s_{2}),\mathbf{g}(h^{p}_{k}, s_{3})].$ 

$\mathbf{f}^*_k=[\mathbf{f}^{*}(s_{kl}),l=1,2,3],$ $ \mathbf{g}^{*}_k=[\mathbf{g}^*(s_{kl}), l=1,2,3].$ 

\textbf{Return}:strategies $\mathbf{f}_{a}=\mathbf{f}^{*}_T\cdots\mathbf{f}^*_{K+T},$ $\mathbf{g}_{a}=\mathbf{g}^{*}_T\cdots\mathbf{g}^*_{K+T}$ and the value upper bound $\bar{v}_1^p(s_{1l}),l=1,2,3$.
\label{finite}
\end{alg}

Now consider the iteration for calculating $\bar{v}^p_{k}(s_{kl})$ from all matrix games $Q^{p}(s_{kl})\in \mathbb{Q}_{k}^p$ applying Algorithm~\ref{finite}. We define any strategy concatenations to stage $k-1$ with at most one non-pure strategy at stage $(k-1)$ as
\begin{align}
\begin{split}
\mathbf{f}^{np}_{k-1}&=\mathbf{f}^p_{k-2}\mathbf{f}_{k-1},\quad \mathbf{f}_{k-1} \in \mathbf{F}_{k-1},\\ \mathbf{g}^{np}_{k-1}&=\mathbf{g}^p_{k-2}\mathbf{g}_{k-1},\quad \mathbf{g}_{k-1} \in \mathbf{G}_{k-1},
\end{split}
\label{np_strategy}
\end{align}
where $\mathbf{f}^p_{k-2},\mathbf{g}^p_{k-2}$ are concatenations of pure strategies to stage $(k-1)$. We denote the corresponding auxiliary matrix as $\tilde{Q}(s_{kl}) \in \tilde{\mathbb{Q}}_k$ for cyber state $\delta_l$, the one shot game value based on payoff matrix $\tilde{Q}(s_{kl})$ as $\tilde{v}_{k}(s_{kl})$, i.e.,
\begin{align}
\begin{split}
&\tilde{Q}(s_{kl}) \\
=&\tilde{r}(s_{kl})+\sum_{\delta_h\in S} \tilde{P}( s_{(k+1)h} |s_{kl})\cdot \tilde{v}_{k+1}(s_{(k+1)h}).
\end{split}
\label{tilde_Q}
\end{align}
Here each possible hybrid state $s_{kl}$ for time instant $k$ is calculated from a none pure strategy defined as~\eqref{np_strategy}. Similarly, the value is defined as
\begin{align}
\tilde{v}_{k}(s_{kl})=\max_{\tilde{Q}(s_{kl})\in \mathbb{\tilde{Q}}_k}v^*[\tilde{Q}(s_{kl})].
\label{tilde_v}
\end{align}
The following theorem shows that at every stage $k$, $\bar{v}_{k}^{p}(s_{kl})$ is greater than or equal to $\tilde{v}_{k}(s_{kl})$.
\\
\begin{thm}
Consider the value iteration for stage $k$ as a one shot robust game. %given a pure strategy history $h^{p}_{k}$, 
Based on $\bar{v}_{k}^{p}(s_{kl}) \geq 0$ of previous iteration, 
we define the robust game value  obtained at $k$ as~\eqref{pickv}. Then for $k=2,\cdots, K$, $\tilde{v}_{k}(s_{kl})$~\eqref{tilde_v} is upper bounded by $\bar{v}_{k}^{p}(s_{kl})$, i.e., $\tilde{v}_{k}(s_{kl}) \leqslant \bar{v}_{k}^{p}(s_{kl}).$
\label{robust}
\end{thm}
\begin{pf}
Since $\bar{v}_{k+1}^{p}(s_{(k+1)h})$ is a nonnegative scalar value, the extreme points of the set $\mathbb{Q}_{k}$ is a subset of the extreme points  of set $\mathbb{Q}^p_k$. Hence, by considering the value of matrix game $Q^{p} (s_{kl})\in \mathbb{Q}^p_k$ defined in~\eqref{eq:Q}, we will get the upper bound of the maximum game value from extreme points of $\mathbb{Q}_k$. 

Consider the following optimization problem for the system with constraint inequality~\eqref{optr} for any possible attacker's strategy vector $\mathbf{f}$ at each stage $k$
%From the system's perspective, to compute the value of the matrix game is equivalent with:
%\vspace{-5pt}
\begin{align}
%\begin{split}
%\vspace{-5pt}
\min_{\mathbf{g}} \quad & z\label{ob_z}\\
\text{subject to}\quad &z \geq \max_{\tilde{Q}(s_{kl})\in \tilde{\mathbb{Q}}_{k}}\mathbf{f}^{T} [\tilde{Q}(s_{kl})]\mathbf{g}.
%\end{split}
\label{optr}
\end{align}
As proven by~Lemma 5 in \cite{RGT},~\eqref{optr} is equivalent to the following constraint that considers only the extreme points
\begin{align}
\quad z \geq \max_{Q^{p}(s_{kl})\in \mathbb{Q}_{k}^p}\mathbf{f}^{T} [Q^{p}(s_{kl})]\mathbf{g},
\end{align}
For the worst-case $f$, the above is also true. Hence, let 
\begin{align}
\label{eq1}
v^p_{k}(s_{kl})
%=\max_{Q_{k}^{p}(h^{p}_{k},s_{l})\in \mathbb{Q}_{kp}}v^*[Q_{k}^{p}(h^{p}_{k},s_{l})]
=\max_{Q^{p}(s_{kl})\in \mathbb{Q}_k^p} \min\limits_{\mathbf{g}}\max\limits_{\mathbf{f}}\mathbf{f}^T[Q^{p}(s_{kl})]\mathbf{g}.
%\label{pickv}
\end{align}
For optimal policies $\mathbf{f}^{*}(s_{kl})$ and $\mathbf{g}^{*} (s_{kl})$, the above optimization problem~\eqref{eq1} results in a cost
\\\centerline{$
%\begin{align*}
 \max\limits_{Q^{p}(s_{kl})\in \mathbb{Q}_{k}^p}v^*[Q^{p}(s_{kl})].
%\end{align*} 
$}
However, $(\mathbf{f}^{*}(s_{kl}),\mathbf{g}^{*}(s_{kl}))$ can be non-pure strategies, meaning that when we apply  $(\mathbf{f}^{*}(s_{kl}),\mathbf{g}^{*}(s_{kl}))$  to calculate system dynamics such as equations~\eqref{dynamicgame}, they will not result in any extreme point of set $\mathbb{Q}_{k+1}$. 

Now consider the final stage $K$, we have
\\\centerline{$
%\begin{align*}
Q^p(s_{Kl})=r^p(s_{Kl}), \tilde{Q}(s_{Kl})=\tilde{r}(s_{Kl}),
%\end{align*}
$}
and use the $Q^p(s_{Kl})$ and $\tilde{Q}(s_{Kl})$ in the above proof, value $\tilde{v}_k(s_{Kl})$ from $\tilde{Q}(s_{Kl})$ is smaller than $\bar{v}_k^p(s_{Kl})$ from the extreme points auxiliary matrix $Q^p(s_{Kl})$, i.e., for $K$, the following inequality holds
\\\centerline{$
%\begin{align*}
\tilde{v}_k(s_{Kl}) \leqslant \bar{v}_k^p(s_{Kl}).
%\end{align*}
$}
Then, by induction, with the value $\tilde{v}_{k+1}(s_{(k+1)h})$ of iteration for stage $k+1, 2\leqslant k \leqslant K-1$ satisfies %%%%%(from $\mathbb{Q}_{(k+1)p}$):
%\begin{align*}
\\\centerline{$\tilde{v}_{k+1}(s_{(k+1)h})\leqslant \bar{v}_{k+1}^{p}(s_{(k+1)h}),$} 
%\end{align*}
and nonnegative payoff and state transition probability $r_{k}^{ij}\geqslant 0$ and $\tilde{P}_{k}^{ij}\geqslant 0$, replacing $\tilde{v}_{k+1}(s_{(k+1)h})$ by $v_{k+1}^{p}(s_{(k+1)h})$ in~\eqref{eq:Q} will make every entry of matrix $\tilde{Q}(s_{kl})$ smaller than matrix $Q^{p}(s_{kl})$. 
% since $r_{k}^{ij}\geqslant 0$ and $\tilde{P}_{k}^{ij}\geqslant 0$. %%% by definition. 
%The system is a minimizer, and is possible to get a smaller value at time k, with some %optimal strategy history $h^{*}_{k} \neq h^{p}_{k}$, 
With a similar argument in the next iteration for stage $k-1$, we have 
%\begin{align*}
\\\centerline{$
\tilde{v}_{k}(s_{kl}) \leqslant \bar{v}_{k}^{p}(s_{kl}).
$}
%\end{align*} 
%(Note that here $v_{k}^{s_{l}}(h^{p}_{k-1})$ is not the game value at $k$ either, because the strategies from 1 to $k-2$ must be pure to get it).
\end{pf}

Based on the above observation, we arrive at the suboptimal algorithm to compute the equilibrium solutions, illustrated in the Algorithm~\ref{finite}. Note that for keeping the physical state $x_{[k-T,k]}$ of the first stage of the game starts at $\hat{x}_0$, in the above Algorithm~\ref{finite} the $K$-stage game  starts at $k=T$. This does not affect our proofs in this section for considering $k=1,\dots, T$.
According to~Theorem~\ref{robust}, we use Algorithm~\ref{finite} to compute an upper bound of the value and the corresponding suboptimal strategy for every step. The % Nash Equilibrium selection 
function $\pi$ computes the strategy and robust value as defined in~\eqref{pickv}. 

The values of the finite stage game $\tilde{v}_k(s_{kl})$ and $\bar{v}_k^p(s_{kl})$ resulting from two auxiliary matrices $\tilde{Q}(s_{kl})$ $Q^p(s_{kl})$ are based on strategy concatenations that only differ at stage $k-1$ (i.e., the same and pure strategies from stages 1 to $(k-2)$). By value iteration backward to stage $1$, we compare the game value for all possible strategies and the robust game value $\bar{v}_1^p(s_{1l})$ of~Algorithm~\ref{finite} in the following theorem.
\begin{cor}
%Moreover, by following this iteration method,
~Algorithm~\ref{finite} yields an upper bound $v_1(s_{1l})$ for the value of the $K$-stage game, together with suboptimal strategies $\mathbf{f}_{a}$ and $\mathbf{g}_{a}$. 
\end{cor}
%\begin{pf}
The strategies $\mathbf{f}_{a}, \mathbf{g}_{a}$ of~Algorithm~\ref{finite} are possibly not pure. %strategy history $h^{p}_{K} \in H^{p}_{K}$. 
According to~Theorem~\ref{robust}, we obtain $\tilde{v}_{k}(s_{kl})\leqslant \bar{v}^{p}_{k}(s_{kl}),$ and the proof holds for every $k=2,\cdots,K$. Consider the value iteration for $k=1$, with 
$\tilde{v}_{2}(s_{2l})\leqslant \bar{v}^p_k(s_{2l})$, and $\text{\ }\quad Q^{ij}( s_{1l})$
$=r^{ij}(s_{1l})+\sum\limits_{\delta_h\in S}{P}^{ij}(s_{2h}|s_{1l})v^{p}_{2}(s_{2h}) \leqslant Q^{p,ij}(s_{2l}),$
%\end{align*}
%\normalsize
thus the true value of the K-stage game $v^{*}[Q(s_{1l})] \leqslant \bar{v}^{p}_{1}(s_{1l})$. The iterative value based on pure strategy auxiliary matrix sets $\mathbb{Q}_{k}^p, k=1,\cdots, K,$ obtained from~Algorithm~\ref{finite} is an upper bound for the game value.
%\end{pf}
Let $v^*[Q(s_{na})]$ represent the minimum total payoff of the system when the strategy is calculated given that there is no attack at all in $K$ stages, then $\bar{v}^{p}_{1}(s_{1l})-v^{*}[Q(s_{1l})] \leqslant \bar{v}^{p}_{1}(s_{1l})-v^*[Q(s_{na})]$, since  $v^*[Q(s_{na})] \leqslant v^{*}[Q(s_{1l})]$ when the system operates in normal state without sacrificing any control cost to play against attacks. The sub-optimality of value $\bar{v}^{p}_{1}(s_{1l})$ calculated from Algorithm 1 is then bounded though we do not know the true value $v^{*}[Q(s_{1l})]$ of the game.


%% This declares a command \Comment
%% The argument will be surrounded by /* ... */
\SetKwComment{Comment}{/* }{ */}

\begin{algorithm}[t]
\caption{Training Scheduler}\label{alg:TS}
% \KwData{$n \geq 0$}
% \KwResult{$y = x^n$}
\LinesNumbered
\KwIn{Training data $\mathcal{D}_{train}=\{(q_i, a_i, p_i^+)\}_{i=1}^m$, \\
\qquad \quad Iteration number $L$.}
\KwOut{A set of optimal model parameters.}

\For{$l=1,\cdots, L$}{
    Sample a batch of questions $Q^{(l)}$\\
    \For{$q_i\in Q^{(l)}$}{
        $\mathcal{P}_{i}^{(l)} \gets \mathrm{arg\,max}_{p_{i,j}}(\mathrm{sim}(q_i^{en},p_{i,j}),K)$\\
        $\mathcal{P}_{Gi}^{(l)} \gets \mathcal{P}_{i}^{(l)}\cup\{p^+_i\}$\\
        Compute $\mathcal{L}^i_{retriever}$, $\mathcal{L}^i_{postranker}$, $\mathcal{L}^i_{reader}$\\ according to Eq.\ref{eq:retriever}, Eq.\ref{eq:rerank}, Eq.\ref{eq:reader}\\
    }
    % $\mathcal{L}^{(l)}_{retriever} \gets \frac{1}{|Q^{(l)}|}\sum_i\mathcal{L}^i_{retriever}$\\
    % $\mathcal{L}^{(l)}_{retriever} \gets \mathrm{Avg}(\mathcal{L}^i_{retriever})$,
    % $\mathcal{L}^{(l)}_{rerank} \gets \mathrm{Avg}(\mathcal{L}^i_{rerank})$,
    % $\mathcal{L}^{(l)}_{reader} \gets \mathrm{Avg}(\mathcal{L}^i_{reader})$\\
    % Compute $\mathcal{L}^{(l)}_{retriever}$, $\mathcal{L}^{(l)}_{rerank}$, and $\mathcal{L}^{(l)}_{reader}$ by averaging over $Q^{(l)}$\\
    $\mathcal{L}^{(l)} \gets \frac{1}{|Q^{(l)}|}\sum_i(\mathcal{L}^{i}_{retriever} + \mathcal{L}^{i}_{postranker}+ \mathcal{L}^{i}_{reader})$\\
    $\mathcal{P}^{(l)}_K\gets\{\mathcal{P}^{(l)}_i|q_i\in Q^{(l)}\}$,\quad $\mathcal{P}^{(l)}_{KG}\gets\{\mathcal{P}^{(l)}_{Gi}|q_i\in Q^{(l)}\}$\\
    Compute the coefficient $v^{(l)}$ according to Eq.~\ref{eq:v}\\
  \eIf{$ v^{(l)}=1$}{
    $\mathcal{L}^{(l)}_{final} \gets \mathcal{L}^{(l)}(\mathcal{P}_{KG}^{(l)})$\\
  }{
      $\mathcal{L}^{(l)}_{final} \gets \mathcal{L}^{(l)}(\mathcal{P}^{(l)}_{K}),$\\
    }
    Optimize $\mathcal{L}^{(l)}_{final}$
}
\end{algorithm}


%  \eIf{$ \mathcal{L}^{(l-1)}_{retriever}<\lambda$}{
%     $\mathcal{L}^{(l)}_{final} \gets \mathcal{L}^{(l)}(\mathcal{P}_K^{(l)})$\\
%   }{
%       $\mathcal{L}^{(l)}_{final} \gets \mathcal{L}^{(l)}(\mathcal{P}^{(l)}_{KG}),$\\
%     }

\section{Simulation Studies}\label{sec:simulation}
In this section, we are mainly interested in the empirical performance of the ABESS algorithm on logistic regression and Poisson regression.
Logistic regression is widely used for classification tasks, and Poisson regression is appropriate when the response is a count.
\if0\informsMOR{In ``Additional Simulation'' of Supplementary Material, we }\else{We }\fi
also consider the performance of ABESS algorithm on multi-response linear regression (a.k.a., multi-task learning).
Before formally analyzing the simulation results,
we illustrate our simulation settings in Section~\ref{subsec:setup}.
% This subsection develops parallel with Section \ref{subsec:logistic}.

%In this section, we study the empirical performance of ABESS for GLM on two generalized linear models,
%logistic regression and gamma regression,
%where logistic regression is widely used for classification and
%gamma regression model is useful for modeling positive continuous response variables.
%Before formally studying logistic regression and gamma regression in Section~\ref{subsec:logistic} and Section~\ref{subsec:gamma}, respectively, we illustrate our simulation setting in Section~\ref{subsec:setup}.

\subsection{Setup}\label{subsec:setup}
To synthesize a dataset, we generate multivariate Gaussian realizations $\boldsymbol{x}_1, \ldots, \boldsymbol{x}_n \overset{i.i.d.}{\sim} \mathcal{MVN}(0,\Sigma)$,
where $\Sigma$ is a $p$-by-$p$ covariance matrix.
%We generate i.i.d error $\epsilon\sim N(0,\sigma^2)$.
%Define the signal to noise ratio (SNR) by $SNR = \frac{\beta^{\top}\Sigma\beta}{\sigma^2}$.
We consider two covariance structures for $\Sigma$: the independent structure ($\Sigma$ is an identity matrix)
and the constant structure ($\Sigma_{ij} = \rho^{I(i\neq j)}$ for some positive constant $\rho$). The value of $\rho$ and $p$ will be specified later.
We set the true regression coefficient $\boldsymbol{\beta}^*$ as a sparse vector with $k$ non-zero entries that have equi-spaced indices in $\{1, \ldots, p\}$.
Finally, given a design matrix $\mathbf{X} = (\boldsymbol{x}_1, \ldots, \boldsymbol{x}_n)^\top$ and $\boldsymbol{\beta}^*$,
we draw response realizations $\{y_i\}_{i=1}^n$ according to the GLMs.

We assess our proposal via the following criteria.
First, to measure the performance of subset selection,
we consider the probabilities of covering true active and inactive sets: $\mathbb{P}(\mathcal{A}^* \subseteq \hat{\mathcal{A}})$ and
$\mathbb{P}(\mathcal{I}^* \subseteq \hat{\mathcal{I}})$ (here, $\mathcal{I}^* = (\mathcal{A}^*)^c$).
We also consider exact support recover probability as $\mathbb{P}(\mathcal{A}^* = \hat{\mathcal{A}})$.
Since the probability is unknown, we empirically examine the proportion of recovery for the active set, inactive set, and exact recovery in 200 replications for instead.
As for parameter estimation performance, we examine relative error (ReErr) on parameter estimations:
$\|\hat{\boldsymbol{\beta}}-\boldsymbol{\beta}^*\|_{2} /\|\boldsymbol{\beta}^*\|_{2}$.
Finally, computational efficiency is directly measured by the runtime.

In addition to our proposed algorithms, we compare classical variable selection methods: LASSO \citep{tibshirani1996regression}, SCAD \citep{fan2001variable}, and MCP \citep{zhang2010nearly}.
%, and a recently proposed coordinate descent (CD) method for $\ell_0$-regularized classification \citep{antoine2021l0learn}.
For all these methods, we apply 10-fold cross-validation (CV) and the GIC to select the tuning parameter, respectively.
% For all these methods, we apply 10-fold cross-validation (CV) to select the tuning parameter.
% ABESS also uses generalized information criterion (GIC) \citep{fan2013tuning} because,
% by combining GIC, ABESS can consistently recover $\mathcal{A}^*$ under linear models \citep{zhu2020polynomial}.
The software for these methods is available at R CRAN (\url{https://cran.r-project.org}).
The software of all methods is summarized in Table~\ref{tab:implementation-details}.
All experiments are carried out on an R environment in a Linux platform with Intel(R) Xeon(R) Gold 6248 CPU @ 2.50GHz. 
% Note that, all experiments result are based on 200 random synthetic datasets.
%Ubuntu platform with Intel(R) Xeon(R) Gold 6248 CPU @ 2.50GHz.

% Model selection methods such as cross-validation and information criteria are widely used.
% Recently, \citet{fan2013tuning} explored generalized information criterion (GIC) in tuning parameter selection for
% penalized likelihood methods under GLM.
% Here, we use a GIC-type information criterion to recovery support size, which is defined as:
% $\mathrm{F}(\hat{\boldsymbol \beta}) = l_n( \hat{\boldsymbol \beta} ) + |\text{supp}(\hat{\boldsymbol \beta})| \log(p) \log\log n.$
% Intuitively speaking, the model complexity penalty term $|\text{supp}(\hat{\boldsymbol \beta})| \log p \log\log n$ is set to prevent over-fitting,
% where the term $\log\log n$ with a slow diverging rate is used to prevent under-fitting.
% Combining the Algorithm~\ref{alg:fbess} with GIC, we select the support size that minimizes the $F(\hat{\boldsymbol{\beta}})$.}

% \begin{table}[htbp]
% \caption{Implementation details for all methods.
% The values in the parentheses indicate the version number of R packages.}\label{tab:implementation-details}
% \centering
% \begin{tabular}{ccc}
% \toprule
% Method & Software & Tuning method \\
% \midrule
% ABESS-GIC & abess (0.4.0) & GIC \\
% LASSO-GIC & glmnet (4.1-3) & GIC \\
% SCAD-GIC & ncvreg (3.13.0) & GIC \\
% MCP-GIC & ncvreg (3.13.0) & GIC \\
% CD-GIC & L0Learn (2.0.3) & GIC \\
% ABESS-CV & abess (0.4.0) & 10-folds CV \\
% LASSO-CV & glmnet (4.1-3) & 10-folds CV \\
% SCAD-CV & ncvreg (3.13.0) & 10-folds CV \\
% MCP-CV & ncvreg (3.13.0) & 10-folds CV \\
% CD-CV & L0Learn (2.0.3) & 10-folds CV \\
% \bottomrule
% \end{tabular}
% \end{table}
\begin{table}[htbp]
\caption{Software for all methods.
The values in the parentheses indicate the version number of R packages.The tuning parameter within the MCP/SCAD penalty is fixed at 3/3.7.}\label{tab:implementation-details}
\centering
\if0\informsMOR{
% \begin{tabular}{ccccccc}
% \toprule
% Method & ABESS & LASSO & SCAD & MCP & CD \\
% \midrule
% Software & \textsf{abess} (0.4.0) & \textsf{glmnet} (4.1-3) & \textsf{ncvreg} (3.13.0) & \textsf{ncvreg} (3.13.0) & \textsf{L0Learn} (2.0.3) \\
% Tuning & sparsity $s$ & $\ell_1$ penalty & $\lambda$ & $\lambda$& $\lambda$ \\
% \bottomrule
% \end{tabular}
\begin{tabular}{cccccc}
    \toprule
    Method & ABESS & LASSO & SCAD & MCP \\
    \midrule
    Software & \textsf{abess} (0.4.0) & \textsf{glmnet} (4.1-3) & \textsf{ncvreg} (3.13.0) & \textsf{ncvreg} (3.13.0) \\
    Tuning & sparsity $s$ & $\ell_1$ penalty & $\lambda$ & $\lambda$ \\
    \bottomrule
    \end{tabular}
}\else{
% \begin{tabular}{ccccccc}
% \hline
% Method & ABESS & LASSO & SCAD & MCP & CD \\
% \hline
% Software & \textsf{abess} (0.4.0) & \textsf{glmnet} (4.1-3) & \textsf{ncvreg} (3.13.0) & \textsf{ncvreg} (3.13.0) & \textsf{L0Learn} (2.0.3) \\
% Tuning & sparsity $s$ & $\ell_1$ penalty & {\color{red}SCAD penalty} & {\color{red}MCP penalty} & $\ell_0$ penalty \\
% \hline
% \end{tabular}
\begin{tabular}{cccccc}
\hline
Method & ABESS & LASSO & SCAD & MCP \\
\hline
Software & \textsf{abess} (0.4.0) & \textsf{glmnet} (4.1-3) & \textsf{ncvreg} (3.13.0) & \textsf{ncvreg} (3.13.0) \\
Tuning & sparsity $s$ & $\ell_1$ penalty & {\color{red}SCAD penalty} & {\color{red}MCP penalty} \\
\hline
\end{tabular}
}\fi
\end{table}
% We implement our proposal in an R package abess \citep{zhu-abess-arxiv}.

\subsection{Logistic Regression}\label{subsec:logistic}
% In this subsection, we illustrate the power of ABESS on logistic regression, which is one of the most popular GLMs widely used for classification tasks.
% In terms of logistic regression, the response $y_i$ is a binary variable following a Bernoulli distribution $B(1, p_i)$,
% where $p_i \coloneqq \mathbb{P}(y_i=1)$ is determined by $\log(\frac{p_i}{1-p_i}) = \boldsymbol x_i^\top \boldsymbol{\beta }$.
% Here, the link function is known as the logit function, defined by $logit(p) = \log(\frac{p}{1-p})$.
% As a result, the negative log-likelihood is given by
% \begin{equation*}
% l_n(\boldsymbol\beta) = -\sum_{i=1}^{N}\left\{y_{i} \boldsymbol {x}_i^\top \boldsymbol \beta-\log \left(1+e^{\boldsymbol {x}_i^\top \boldsymbol \beta}\right)\right\}.
% \end{equation*}
% Empirically, we generate $x_i$ and $\beta$ as described in Section \ref{subsec:setup}.
% Binary response $y_i$ is then drawn from the Bernoulli distribution according to (\ref{eqn:formula_binomial}).
% Let $H_j = \sum\limits_{i=1}^{n} \frac{e^{\boldsymbol {x}_i^\top \hat{\boldsymbol \beta}}}{(1 + e^{\boldsymbol {x}_i^\top \hat{\boldsymbol \beta}})^2} x_{ij}^2$ and
% the gradient of $l_n(\boldsymbol{\beta})$ at $\hat{\boldsymbol{\beta}}$ be $\hat{\boldsymbol d} = -\sum\limits_{i=1}^{n}(y_i - \frac{e^{\boldsymbol {x}_i^\top \hat{\boldsymbol \beta}}}{1 + e^{\boldsymbol {x}_i^\top \hat{\boldsymbol \beta}}}) \boldsymbol {x}_i$,
% \eqref{eqn:approx_sacrifice} can be explicit expressed as:
% $\xi_j = H_j (\hat{\boldsymbol{\beta}}_j)^2$ for $j\in \mathcal{A}$ and
% $\zeta_j = H_j^{-1}( \hat{\boldsymbol d}_j )^2$ for $j\in \mathcal{I}$.
% \begin{equation*}
% \begin{aligned}
% % \hat{\boldsymbol d} &= -\sum_{i=1}^{n}(y_i - \frac{e^{\boldsymbol {x}_i^\top \boldsymbol \beta}}{1 + e^{\boldsymbol {x}_i^\top \boldsymbol \beta}}) \boldsymbol {x}_i. \\
% \xi_j
% & = H_j (\hat{\boldsymbol{\beta}}_j)^2, j\in \mathcal{A},\\
% \zeta_j
% & = H_j^{-1}
% ( \hat{\boldsymbol d}_j )^2, j\in \mathcal{I}.
% \end{aligned}
% \end{equation*}
% Given the explicit expression of \eqref{eqn:approx_sacrifice},
% we can conduct Algorithm~\ref{alg:abess} to estimate $\boldsymbol{\beta}$.

The dimension $p$ is fixed as 500 for the logistic regression model. For the constant correlation case, we set $\rho = 0.4$.
The non-zero coefficients $\boldsymbol{\beta}^*_{\mathcal{A}^*}$ are set to be $(2,2,8,8,8,8,10,10,10,10)^\top$. 
Now we compare methods listed in Table~\ref{tab:implementation-details}.
Figures~\ref{fig:rate_binomial} and \ref{fig:ReErr_binomial} present the results on subset selection and parameter estimation when the sample size increases. Out of clarity, we omit the CV results here and defer these results to the Additional Figures in Supplementary Material.


\begin{figure}[htbp]
\centering
\includegraphics[width=1.0\textwidth]{figure/rate_binomial_gic.pdf}
\if0\informsMOR{
\vspace{-30pt}
}\fi
\caption{Performance on subset selection under logistic regression when covariates have independent correlation structure (Upper) and constant correlation structure (Lower), measured by three kinds probabilities: $\mathbb{P}(\mathcal{A}^* \subseteq \hat{\mathcal{A}})$, $\mathbb{P}(\mathcal{I}^* \subseteq \hat{\mathcal{I}})$, and $\mathbb{P}(\mathcal{A}^* = \hat{\mathcal{A}})$ that are presented in Left, Middle and Right panels, respectively.
}
\label{fig:rate_binomial}
\end{figure}
\begin{figure}[htbp]
\centering
\includegraphics[width=0.8\textwidth]{figure/ReErr_binomial_gic.pdf}
\if0\informsMOR{
\vspace{-10pt}
}\fi
\caption{Performance on parameter estimation under logistic regression models when covariance matrices have independent correlation structure (Left) and exponential correlation structure (Right). The $y$-axis is the median of ReErr in a log scale.}
\label{fig:ReErr_binomial}
\end{figure}

As depicted in the left panel of Figure~\ref{fig:rate_binomial}, the probability $\mathbb{P}(\mathcal{A}^* \subseteq \hat{\mathcal{A}})$ approaches 1 as the sample size increases, indicating that all methods, except LASSO in the high correlation setting, can provide a no-false-exclusion estimator when the sample size is sufficiently large. However, when considering $\mathbb{P}(\mathcal{I}^* \subseteq \hat{\mathcal{I}})$, as observed in the middle panel of Figure~\ref{fig:rate_binomial}, the LASSO estimator consistently exhibits false inclusions, and the SCAD/MCP estimator shows false inclusions when the covariates are highly correlated. In contrast, only ABESS guarantees that $\mathbb{P}(\mathcal{I}^* \subseteq \hat{\mathcal{I}})$ approaches 1 for large sample sizes. 

Furthermore, as evident from the right panel of Figure~\ref{fig:rate_binomial}, ABESS accurately recovers the true subset under both correlation settings. While SCAD and MCP can also achieve exact support recovery given a sufficient sample size, ABESS demonstrates support recovery consistency with the smallest sample size, particularly when variables are correlated. It is important to note that although our theory imposes restrictions on the correlation among a small subset of variables (see Assumption~\ref{con:technical-assumption}), our algorithm still performs effectively in the constant correlation setting. This setting (i.e., $\rho=0.4$) violates Assumption~\ref{con:technical-assumption} as the correlation between any two variables exceeds 0.183, which is the maximum acceptable pairwise correlation satisfying Assumption~\ref{con:technical-assumption}.

Moving on to Figure~\ref{fig:ReErr_binomial}, it illustrates the superiority of ABESS in parameter estimation. ABESS visibly outperforms other methods in the small sample size regime and maintains highly competitive performance as the sample size increases. This superiority in parameter estimation is not surprising, as ABESS yields an oracle estimator when the support set is correctly identified. Although SCAD and MCP do not provide algorithmic guarantees for finding the local minimum, they exhibit competitive parameter estimation performance due to their asymptotic unbiasedness. Conversely, the LASSO estimator is biased and performs the worst among all the methods.

%\begin{figure}
%	\centering
%	\includegraphics[width=\textwidth]{figure/Performance_binomial.pdf}
%	\caption{Performance comparison under two correlation structures: independent and exponential. (A) Performance for subset selection, measured by support recover probability. (B) Performance for parameter estimation, measured by median ReErr. (C) Average runtime, measured in seconds. L0Learn is omitted since its runtime is far longer than others.}
%	\label{fig:Performance_binomial}
%\end{figure}

\subsection{Poisson Regression}\label{seubsec:poisson}
% As regard to Poisson regression, the response $y_i$ is a integer variable following a Poisson distribution $\mathcal{P}(\lambda_i)$ where $\lambda_i = \exp(\boldsymbol x_i^\top \boldsymbol{\beta})$.
% As a result, the negative log-likelihood is given by
% \begin{equation*}
% l_n(\boldsymbol\beta) = -\sum_{i=1}^{N}\left\{y_{i} \boldsymbol {x}_i^\top \boldsymbol\beta - e^{\boldsymbol {x}_i^\top \boldsymbol \beta} -\log(y_i!)\right\}.
% \end{equation*}
% Empirically, we generate $x_i$ and $\beta$ as described in Section \ref{subsec:setup}.
% Binary response $y_i$ is then drawn from the Bernoulli distribution according to (\ref{eqn:formula_binomial}).
% Let $H_j = \sum\limits_{i=1}^{n} \exp(\boldsymbol {x}_i^\top \hat{\boldsymbol \beta}) x_{ij}^2$ and
% the gradient of $l_n(\boldsymbol{\beta})$ at $\hat{\boldsymbol{\beta}}$ be $\hat{\boldsymbol d} = -\sum\limits_{i=1}^{n}(y_i - \exp(\boldsymbol {x}_i^\top \hat{\boldsymbol \beta})) \boldsymbol {x}_i$,
% \eqref{eqn:approx_sacrifice} can be explicit expressed as:
% $\xi_j = H_j (\hat{\boldsymbol{\beta}}_j)^2$ for $j\in \mathcal{A}$ and
% $\zeta_j = H_j^{-1}( \hat{\boldsymbol d}_j )^2$ for $j\in \mathcal{I}$.
% Given the explicit expression of \eqref{eqn:approx_sacrifice},
% we can conduct Algorithm~\ref{alg:abess} to estimate $\boldsymbol{\beta}$.


For the Poisson regression model, we consider a fixed $p$ value of 500, and set $\rho = 0.2$ for the constant correlation case. The non-zero coefficients $\boldsymbol{\beta}^*_{\mathcal{A}^*}$ are specified as $(1, 1, 1)^\top$. Figures~\ref{fig:rate_poisson_gic}-\ref{fig:ReErr_poisson_gic} present the evaluation of subset selection and parameter estimation quality. Examining Figures~\ref{fig:rate_poisson_gic}, we observe that for ABESS/SCAD/MCP, the probabilities $\mathbb{P}(\mathcal{A}^* \subseteq \hat{\mathcal{A}})$, $\mathbb{P}(\mathcal{I}^* \subseteq \hat{\mathcal{I}})$, and $\mathbb{P}(\mathcal{A}^* = \hat{\mathcal{A}})$ gradually approach 1 as the sample size $n$ increases. In contrary, the LASSO, regardless of the highest inclusion probability for $\mathcal{A}^*$, still has a chance of including ineffective variables, especially when variables are correlated. Comparing ABESS, SCAD, and MCP, it is evident that ABESS achieves the highest exact selection probability, followed by SCAD and MCP. Similar to the results in logistic regression, ABESS achieves exact selection of the effective variables with the smallest sample size under the constant correlation structure.
Regarding the quality of parameter estimation, the ReErr of all methods reasonably decreases as the sample size $n$ increases. Again, ABESS exhibits the least estimation error in terms of the $\ell_2$-norm, which coincides with the results on logistic regression. It is worth noting that our method demonstrates consistency and polynomial complexity under Poisson regression, even though it violates the sub-Gaussian assumption. This is because the current framework of proofs allows for the relaxation of Assumption~\ref{con:subgaussian} to a sub-exponential distribution assumption, enabling the establishment of similar theoretical properties.

\begin{figure}[htbp]
\centering
\includegraphics[width=1.0\textwidth]{figure/rate_poisson_gic.pdf}
\if0\informsMOR{
\vspace{-30pt}
}\fi
\caption{Performance on subset selection under Poisson regression when covariates have independent correlation structure (Upper) and constant correlation structure (Lower), measured by three kinds probabilities: $\mathbb{P}(\mathcal{A}^* \subseteq \hat{\mathcal{A}})$, $\mathbb{P}(\mathcal{I}^* \subseteq \hat{\mathcal{I}})$, and $\mathbb{P}(\mathcal{A}^* = \hat{\mathcal{A}})$ that are presented in Left, Middle and Right panels, respectively.}
\label{fig:rate_poisson_gic}
\end{figure}
\begin{figure}[htbp]
\centering
\includegraphics[width=0.8\textwidth]{figure/ReErr_poisson_gic.pdf}
\if0\informsMOR{
\vspace{-5pt}
}\fi
\caption{Performance on parameter estimation under Poisson regression models when covariance matrices have independent correlation structure (Left) and exponential correlation structure (Right). The $y$-axis is the median of ReErr in a log scale.}
\label{fig:ReErr_poisson_gic}
\end{figure}

\subsection{Computational analysis}

We compare the runtime of different methods in Table~\ref{tab:implementation-details} for the logistic regression and Poisson regression models in Sections~\ref{subsec:logistic} to \ref{seubsec:poisson}. The runtime results are summarized in Figure~\ref{fig:simu_runtime}, indicating that ABESS demonstrates superior computational efficiency compared to state-of-the-art variable selection methods. For instance, when $n = 3000$, ABESS is at least four times faster than its competitors in logistic regression under an independent correlation structure. Furthermore, regardless of logistic regression or Poisson regression, ABESS exhibits similar computational performance, while other competitors run much faster when the pairwise correlation is higher. Lastly, it is important to note that the runtime of ABESS scales polynomially with sample sizes, aligning with the complexity presented in Theorem~\ref{thm:complexity}.
%In contrast, the runtime of other methods grows more rapidly as the sample size increases
%and appears like a quadratic function of the sample size in the independent scenario.
%Increasing iteration numbers for convergence may lead to this result.
%Moreover, ABESS-GIC is faster than ABESS-CV, demonstrating the superiority of the proposed adaptive parameter tuning procedure.
% Finally, according to the computational comparison presented in Figure~\ref{fig runtime_poisson_gic}, the ABESS has the least runtime and is much faster than the MCP and SCAD when variables are independent.

\begin{figure}[htbp]
\centering
\includegraphics[width=0.8\textwidth]{figure/runtime_binomial_gic.pdf}
\includegraphics[width=0.8\textwidth]{figure/runtime_poisson_gic.pdf}
\if0\informsMOR{
\vspace{-10pt}
}\fi
\caption{Average runtime (measured in seconds) on logistic regression (Upper panel) and Poisson regression (Lower panel). The results on two types of covariances matrix $\Sigma$, the independent correlation structure and constant correlation structure, are presented in the left and right panels, respectively. The error bars represent two times the standard errors.
}
\label{fig:simu_runtime}
\end{figure}

%% For the MOR template, uncomment this line and comment on the code blocks

\if1\informsMOR
{
\input{../appendix_numerical}
}\fi


% \vspace{-0.5em}
\section{Conclusion}
% \vspace{-0.5em}
Recent advances in multimodal single-cell technology have enabled the simultaneous profiling of the transcriptome alongside other cellular modalities, leading to an increase in the availability of multimodal single-cell data. In this paper, we present \method{}, a multimodal transformer model for single-cell surface protein abundance from gene expression measurements. We combined the data with prior biological interaction knowledge from the STRING database into a richly connected heterogeneous graph and leveraged the transformer architectures to learn an accurate mapping between gene expression and surface protein abundance. Remarkably, \method{} achieves superior and more stable performance than other baselines on both 2021 and 2022 NeurIPS single-cell datasets.

\noindent\textbf{Future Work.}
% Our work is an extension of the model we implemented in the NeurIPS 2022 competition. 
Our framework of multimodal transformers with the cross-modality heterogeneous graph goes far beyond the specific downstream task of modality prediction, and there are lots of potentials to be further explored. Our graph contains three types of nodes. While the cell embeddings are used for predictions, the remaining protein embeddings and gene embeddings may be further interpreted for other tasks. The similarities between proteins may show data-specific protein-protein relationships, while the attention matrix of the gene transformer may help to identify marker genes of each cell type. Additionally, we may achieve gene interaction prediction using the attention mechanism.
% under adequate regulations. 
% We expect \method{} to be capable of much more than just modality prediction. Note that currently, we fuse information from different transformers with message-passing GNNs. 
To extend more on transformers, a potential next step is implementing cross-attention cross-modalities. Ideally, all three types of nodes, namely genes, proteins, and cells, would be jointly modeled using a large transformer that includes specific regulations for each modality. 

% insight of protein and gene embedding (diff task)

% all in one transformer

% \noindent\textbf{Limitations and future work}
% Despite the noticeable performance improvement by utilizing transformers with the cross-modality heterogeneous graph, there are still bottlenecks in the current settings. To begin with, we noticed that the performance variations of all methods are consistently higher in the ``CITE'' dataset compared to the ``GEX2ADT'' dataset. We hypothesized that the increased variability in ``CITE'' was due to both less number of training samples (43k vs. 66k cells) and a significantly more number of testing samples used (28k vs. 1k cells). One straightforward solution to alleviate the high variation issue is to include more training samples, which is not always possible given the training data availability. Nevertheless, publicly available single-cell datasets have been accumulated over the past decades and are still being collected on an ever-increasing scale. Taking advantage of these large-scale atlases is the key to a more stable and well-performing model, as some of the intra-cell variations could be common across different datasets. For example, reference-based methods are commonly used to identify the cell identity of a single cell, or cell-type compositions of a mixture of cells. (other examples for pretrained, e.g., scbert)


%\noindent\textbf{Future work.}
% Our work is an extension of the model we implemented in the NeurIPS 2022 competition. Now our framework of multimodal transformers with the cross-modality heterogeneous graph goes far beyond the specific downstream task of modality prediction, and there are lots of potentials to be further explored. Our graph contains three types of nodes. while the cell embeddings are used for predictions, the remaining protein embeddings and gene embeddings may be further interpreted for other tasks. The similarities between proteins may show data-specific protein-protein relationships, while the attention matrix of the gene transformer may help to identify marker genes of each cell type. Additionally, we may achieve gene interaction prediction using the attention mechanism under adequate regulations. We expect \method{} to be capable of much more than just modality prediction. Note that currently, we fuse information from different transformers with message-passing GNNs. To extend more on transformers, a potential next step is implementing cross-attention cross-modalities. Ideally, all three types of nodes, namely genes, proteins, and cells, would be jointly modeled using a large transformer that includes specific regulations for each modality. The self-attention within each modality would reconstruct the prior interaction network, while the cross-attention between modalities would be supervised by the data observations. Then, The attention matrix will provide insights into all the internal interactions and cross-relationships. With the linearized transformer, this idea would be both practical and versatile.

% \begin{acks}
% This research is supported by the National Science Foundation (NSF) and Johnson \& Johnson.
% \end{acks}

\bibliographystyle{agsm}        
{  \small 
\bibliography{gamejs}
}

%\input{bio_coding}
\end{document}