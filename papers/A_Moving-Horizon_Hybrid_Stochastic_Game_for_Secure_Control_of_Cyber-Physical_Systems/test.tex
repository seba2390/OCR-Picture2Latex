\documentclass[letterpaper, 10pt, conference, english]{ieeeconf}   % Comment this line out
                                                          % if you need a4paper
%\documentclass[a4paper, 10pt, conference]{ieeeconf}      % Use this line for a4
                                                          % paper

%\IEEEoverridecommandlockouts                              % This command is only
                                                          % needed if you want to
                                                          % use the \thanks command
\overrideIEEEmargins
% See the \addtolength command later in the file to balance the column lengths
% on the last page of the document

%\documentclass[conference]{IEEEtran}
%\usepackage[latin9]{inputenc}
%\usepackage[letterpaper]{geometry}
%\geometry{verbose,tmargin=1in,bmargin=1in,lmargin=1in,rmargin=1in}
%\usepackage[cmex10]{amsmath}
%\usepackage{amssymb}
\usepackage{graphicx}
%%\usepackage[dvipdfmx]{graphicx}
%\usepackage{float}
%\usepackage{array}
%\usepackage{tikz}
%\usepackage{amssymb}
\usepackage{epstopdf}
%%\usepackage{ifpdf}
%\usepackage{cite}
%\usepackage{algorithm}
%\usepackage{algorithmic}
%\usepackage{amsthm}
%
%\DeclareGraphicsExtensions{.eps,.pdf,.jpg,.png}
%\DeclareGraphicsRule{.jpg}{eps}{.bb}{}
%%%%%%\newtheorem{lemma}{Lemma}
%%%%%%\newcommand{\ind}[1]{\mathbf{1}\left(#1\right)}
%%%%%%\newcommand{\bx}{\mathbf{x}}
%%%%%%\newcommand{\E}{\mathbf{E}}
%%%%%%\newtheorem{theorem}{Theorem}%[section]
%%%%%%\newenvironment{definition}[1][Definition]{\begin{trivlist}
%%%%%%\item[\hskip \labelsep {\bfseries #1}]}{\end{trivlist}}
%\hyphenation{op-tical net-works semi-conduc-tor}

\usepackage{amsmath,amsfonts, amssymb}
\newtheorem{problem}{Problem}
\newtheorem{theorem}{Theorem}
\newtheorem{corollary}{Corollary}
\newtheorem{lemma}{Lemma}
\newtheorem{remark}{Remark}
\newtheorem{definition}{Definition}
\newtheorem{example}{Example}
\newtheorem{algorithm}{Algorithm}
%\newtheorem{proof}{Proof}

% The following packages can be found on http:\\www.ctan.org
%\usepackage{graphics} % for pdf, bitmapped graphics files
\usepackage{epsfig} % for postscript graphics files
\usepackage{psfrag}
%\usepackage{mathptmx} % assumes new font selection scheme installed
%\usepackage{times} % assumes new font selection scheme installed
%\usepackage{amsmath} % assumes amsmath package installed
%\usepackage{amssymb}  % assumes amsmath package installed

\def\endtheorem{\hspace*{\fill}~\QEDopen\par\endtrivlist\unskip}
\def\endlemma{\hspace*{\fill}~\QEDopen\par\endtrivlist\unskip}
\def\endcorollary{\hspace*{\fill}~\QEDopen\par\endtrivlist\unskip}
\def\endexample{\hspace*{\fill}~\QEDopen\par\endtrivlist\unskip}
\def\endremark{\hspace*{\fill}~\QEDopen\par\endtrivlist\unskip}
\def\enddefinition{\hspace*{\fill}~\QEDopen\par\endtrivlist\unskip}

\usepackage{times}
\usepackage[tight,footnotesize]{subfigure}
\usepackage{bibspacing}
%\setlength{\bibspacing}{\baselineskip}

\newcommand{\PA}[1]{$\spadesuit$\footnote{PAJA: #1}}
\newcommand{\FM}[1]{$\clubsuit$\footnote{FEI: #1}}
\newcommand\fixme[1]{$\star$ \emph{\small #1} $\star$}

\newcommand{\tableref}[1]{Table~\ref{tab:#1}}
\newcommand{\figref}[1]{Figure~\ref{fig:#1}}


%%%%%%%%%%%%%%%%%%%%%%%
\usepackage{verbatim}
\usepackage{amsmath}
\usepackage{babel}
\usepackage{listings}
%%%%%%%%%%%%%%%%%%%%\usepackage[]{graphicx}
\usepackage{comment}
\usepackage{multirow}
\usepackage{algorithmic}
\usepackage{algorithm}

%%%% Math Packages %%%%%%%%%%%%
%%\usepackage{amsmath}
%%\usepackage{amsthm}
%%\usepackage{amsfonts}

\usepackage[tight]{subfigure}
%%%%%%\usepackage{wrapfig}


%% Line Spacing %%%%%%%%%%%%%%%%%%%%%%%%%%%%%%%%%%%%%%%%%%%%%
%\usepackage{setspa ce}
%\singlespacing        %% 1-spacing (default)
%\onehalfspacing       %% 1,5-spacing
%\doublespacing        %% 2-spacing
%\usepackage[belowskip=-15pt,aboveskip=0pt]{caption}
%\setlength{\intextsep}{10pt plus 2pt minus 2pt}
%\allowdisplaybreaks
\begin{document}
\title{\LARGE \bf Stochastic Game Approach for Attack Detection}
\author{Fei Miao \and Quanyan Zhu%\and Rahul Mangharam\and George J. Pappas% <-this % stops a space
%\thanks{This research has been partially supported by the NSF-CPS 0931239 grant.}% <-this % stops a space
%\thanks{F. Miao, M. Pajic, R. Mangharam and G. J. Pappas are with the Department of Electrical and Systems Engineering, University of Pennsylvania, Philadelphia, PA, USA 19014. Email: \{{\tt miaofei,pajic,rahulm,pappasg\}@seas.upenn.edu}.}%
}
\maketitle

\input{model_replay}

\section{A Hybrid Stochastic Game Model}
\label{sec:game_form}
To obtain a switching policy that minimizes the expected real-time worst case payoff for the given subsystems, 
we formulate a zero-sum, hybrid stochastic game between the system and the attacker. System dynamics knowledge are combined with the game definition, and the quantitative process for the game parameters will be introduced in this section. We assume that one game stage $k$ is also one time step of the physical system. The total stage number is $K$. The hybrid game state space $(X_{[k-T,k]}\times S)$ contains information about both the system dynamics $\mathbf{x}_k$ and the discrete modes $\delta_l, l=1,2,3$. Here, $T$ is the window size of system dynamics needed to keep the state transition between stages $k$ and $(k+1)$ Markov. The joint state includes information we need to compute the game strategy at the current stage. This is the main difference compared with the previous work~(\cite{cdc_replay}), while the latter is not Markov since it needs to consider all the possible histories of strategies for deciding the physical dynamics and getting a strategy. At each stage $k \in \{T,\cdots, K+T\}$, parameters include the action space for the attacker (system) $A_{t}$ ($A_{s}$), the state transition probability matrix $\mathbb{P}_{k}$, and the immediate payoff matrix $r_{k}$. The solution set of the game is mixed strategies $\mathbf{F}_{k}$ for the attacker, and $\mathbf{G}_{k}$ for the system. Formally, the game is defined as a sequence of tuples:
 $\{(X_{[k-T,k]} \times S),A_{t},A_{s}, \mathbf{F}_{k},\mathbf{G}_{k}, P, r\}$.

\iffalse
\begin{table*}
\centering
%\caption{Parameters of the hybrid stochastic game between the system and the attacker}
\begin{tabular}{|c|c|}
  \hline
   $s_{kl}=(x_{[k-T,k]}, \delta_l)$& Joint game state: sequence of physical dynamics, and piecewise constant mode \\ \hline
   $A_{t}$ & Attacker's action space \\ \hline
   $A_{s}$ & System's action space \\ \hline  
   $\mathbf{f}_k(s_{kl})$ & Strategy of the attacker at stage $k$, 
                                           state $s_{kl}, l=1,2,3$  \\ \hline     
      $\mathbf{g}_k(s_{kl})$ & Strategy of the system at stage $k$, 
                                            state $s_{kl}, l=1,2,3$  \\ \hline     
    ${P}(s_{(k+1)h}|s_{kl})$ & Probability that system transits from state $s_{kl}$ 
                                              at stage $k$ to state $s_{(k+1)l}$ at stage $k+1$\\ \hline
    ${r}(s_{kl})$ & Immediate payoff matrix at stage $k$ \\ 
   \hline
\end{tabular}
\centering
%\captionsetup{justification=centerlast}
\caption{Parameters of the hybrid stochastic game between the system and the attacker}
\label{game_parameter}
\end{table*} 
\fi

\textbf{Game State Space}: The joint state of the system at stage $k$ is described by the pair $s_{kl}=(x_{[k-T,k]}, \delta_l)$, where
\centerline{$
x_{[k-T,k]}=(x_{k-T}, x_{k-T+1}, \cdots, x_k ) \in X_{[k-T,k]}$} is the discrete-time dynamics of the physical process provided to the system--the state estimations $\hat{x}_{k-T},\cdots, \hat{x}_k$, $\delta_l \in S=\{\delta_1,\delta_2, \delta_3\}$ denote the cyber state of the system. We assume that once the game reach $\delta_1$, the system wins and will not enter other modes till next game, i.e., $\delta_1$ is an absorbing state. The moving-horizon transition of the joint states on stage axis is shown as Figure~\ref{sg}. The window size of system dynamics $T$ keeps the state transition between time $k$ and $k+1$ Markov. For instance, if the detector of the system requires system dynamics $\hat{x}_{[k-T_1,k]}$, and we consider sensor data injection attacks and replay attacks with replay windows less than $T_2$ steps, then $T=max\{T_1, T_2\}$. 
%With the system dynamics $\hat{x}_{[k-T, k]}$ denoted as $x_{[k-T,k]}$ for game stage $k$,  information needed to define the following action space of two players, the payoff and state transition probability is included. 

%i.e., the detector needs information for T steps to decide whether the alarm should be triggered. 
%\FM{In replay attack, we can say: once alarm is triggered, the system can stop the execution and check whether attack occurred, is this true for other attacks? Can the system distinguish between successfully detection and false alarm trigger?}
\begin{figure}[t!]
%\vspace{-5pt}
\centering
\includegraphics [width=0.32\textwidth]{xk.pdf}
\vspace{-8pt}
\caption{Joint state transition of the hybrid stochastic game when moving the horizon of game state one step ahead. When the state transits from stage $k$ to $k+1$, we slice the window of the sequence of physical dynamics one step ahead, add $x_{k+1}$ and remove $x_{k-T}$,  thus $x_{[k-T,k]} \to x_{[k-T+1,k+1]}$. The piecewise constant modes $\delta_l$, $\delta_h$ describe the cyber states provided by the detector at stage $k$, respectively.}
\label{sg}
\vspace{-5pt}
\end{figure}

%For simplicity, we omit the subscript k and just write state at every time k as $s_i,  i=1,2,3$.
\textbf{Attacker's Action Space}: We assume that the system is vulnerable to different attack models described by the action space $A_{t}$, where 
\\\centerline{$
A_{t}=\{a_{1}(x_{[k-T,k]}), a_{2}(x_{[k-T,k]}), \cdots, a_{M}(x_{[k-T,k]})\}
$}
is the attacker's action space at stage $k$, and $a_{1}$ means no attack. Here we only consider discretized action space of the attacker for computational efficiency. For the LTI system dynamics considered in this work, the distance of a continuous point to its nearest discrete point in action space is bounded. With bounded error of the dynamics by discretized continuous action space, the quality of game solutions under different conditions is analyzed by work~\cite{disaction}. 

The actions can describe both multiple types of attacks and the same type attack with different values. For instance, when considering only sensor data injection attacks with different norms of injection value, we will denote $a_i (x_{[k-T,k]}), i=2, 3,\dots$ as changing the sensor value from $\mathbf{y}_k=\mathbf{Cx}_k+\mathbf{v}_k$ to $\mathbf{y}'_k= \mathbf{y}_k+\mathbf{y}_{k,i}^a$, where any injection $\mathbf{y}_k^a$ is classified as $a_i (x_{[k-T,k]}), i=inf\{i:\ \|\mathbf{y}_k^a-\mathbf{y}_{k,i}^a\|_2\}$ in attacker's action space.
Similarly, for replay attack only, the action space is discretized as changing sensor values from $\mathbf{y}_k=\mathbf{Cx}_k+\mathbf{v}_k$ to $\mathbf{y}'_k= \mathbf{y}_{k-T_i}$ for action index $a_i (x_{[k-T,k]})$, where any replay time length $T_a$ is classified as $a_i (x_{[k-T,k]}), i=inf\{i:|T_a - T_i|\}$. Considering multiple types of attacks, we assume that the system is valnerable under $m_a$ types of attacks, and attack type $A_i$ is corresponding to $M_{a,i}$ discretized actions in the action space, then there are $\sum_{i=1}^{m_a} M_{a,i}+1$ actions in total within the attacker's action space $A_t$.
 

% with discretized, bounded norm $\|\mathbf{y}_{k,i}^a\|_2 \leqslant b$, since the attacker has limited energy for every data injection. This means any injection data that satisfies $\|\mathbf{y}_{k}^a\|_2 \leqslant b$ is considered as $inf\{i:\ \|\mathbf{y}_k^a-\mathbf{y}_{k,i}^a\|_2\}$ in attacker's action space. 

%For example, 
%when considering replay attacks and false data injection attacks, we take $a_{2k}$ as~\eqref{replay_y} for a given replay window size, and $a_{3k}$ as~\eqref{attackmodel} for a given data injection range. 

\iffalse by any given controller/estimator/detector combination of the system \fi
%; here, $y_{k}$ is the real sensor value and $\mathbf{y}_{k-t_{i}}$ denotes any replay sensor value in the strategy set. %\FM{here we assume during time $k \in \{1,...,K\}$ the replay window size $T$ does not change for simulation.}
%\item
%\FM{rewrite the action space definition}

\textbf{System's Action Space}: The system's action space at stage $k$ is defined as
\\\centerline{$
A_{s}=\{u_{1}(x_{[k-T,k]}), u_{2}(x_{[k-T,k]}),\cdots, u_{N}(x_{[k-T,k]})\},
$}  %= \{\mathbf{u}^{*}_{k}, \mathbf{u}^{*}_{k} + \Delta \mathbf{u}_{k}\}$
where $u_{j}$ is the index for the $j$th subsystem. We assume that the $N$ subsystems (a model for each component in Figure~\ref{system}) are determined priorly. For example, a subsystem can be the plant with a given optimal LQG controller, a Kalman filter and a $\chi^2$ detector. A subsystem can also be the plant with an optimal LQG controller, a resilient state estimator~\cite{res_estimator} and its corresponding estimation residual checking component. We assume that the attacker's action space is defined, with corresponding system's action or a subsystem that the detection rate is greater than $0$. A switched system does not ensure performance under the attack outside the action space of the game.

\textbf{Mixed Strategy}: Let $f^{i}_{k}(s_{kl})$ ($g^{j}_{k}(s_{kl})$) be the probability that the attacker (system) chooses action $a_{i}(x_{[k-T,k}) \in A_{t}$ ($u_{j}(x_{[k-T,k}) \in A_{s}$) at state $s_{kl}\in (X_{[k-T,k]}\times S)$. Define $\mathbf{F}_{k}$ and $\mathbf{G}_{k}$ as the mixed strategy sets of the attacker and the system for stage $k$:
$\mathbf{F}_{k} :=\{\mathbf{f}_{k}= [\mathbf{f}_{k}(s_{k1}), \mathbf{f}_{k}(s_{k2}), \mathbf{f}_{k}(s_{k3})]
|f_{k}^{i}(s_{kl})\geq 0, \mathbf{f}_k \in [0,1]^{M\times 3},
\sum \limits_{a_{ik} \in A_{tk}}f_{k}^{i}(s_{kl}) = 1,\mathbf{f}_{k}(s_{kl})\in \mathbb{R}^{M}, \forall s_{kl} \in(X_{[k-T,k]}\times S)\},$
$\mathbf{G}_{k}:=\{\mathbf{g}_{k}= [\mathbf{g}_{k}(s_{k1}), \mathbf{g}_{k}(s_{k2}), \mathbf{g}_{k}(s_{k3})]|$
$g_{k}^{j}(s_{kl})\geq 0, \mathbf{g}_k \in [0,1]^{N \times 3}, %&\forall u_{jk} \in A_{sk},%k \in \{1,...,K\},\\
\sum \limits_{u_{jk} \in A_{sk}}g_{k}^{j}(s_{kl}) = 1, \mathbf{g}_{k}(s_{kl}) \in \mathbb{R}^{N}, \forall s_{kl} \in (X_{[k-T,k]}\times S)\}. $ Note that $\mathbf{x}_{[k-T,k]}$ provides exogenous information for the strategy $\mathbf{f}_k (\mathbf{g}_k)$, since for every $l$, $\mathbf{f}_{k}(s_{kl}) (\mathbf{g}_{k}(s_{kl}))$ is the strategy at mode $\delta_l$ for the same $\mathbf{x}_{[k-T,k]}$ at stage $k$. Hence, $\mathbf{g}_k$ and $\mathbf{f}_k$ are finite dimensional vectors, that the stationary strategy chosen by each player at stage $k$ depends on the cyber state. %Mixed strategy set $\mathbf{F}_k$ also include the case that the attacker only implement one specific type of attack in the action space at time instance $k$, since we do not have know the strategy of the attacker, we are able to consider all possible combinations of attacks by exploiting mixed strategies. 

%\textbf{}:
%\label{dynamicgame}
%With all the above definition, %for any strategy history $h_{k}$ (a sequence of switching policy), 


\textbf{System and Subsystem Dynamics under game framework}: Given the subsystem and attack models in Section~\ref{sec:replay1} and the game definition,  
%we can transform it to a new system dynamic model decided by both players' actions. 
we show the dynamics at stage $k$ given an action pair $(a_{i}(x_{[k-T,k}),u_{j}(x_{[k-T,k}))$ (assume initial $\mathbf{\hat{x}}_{1|0}=\bar{\mathbf{x}}_{0}$, $\mathbf{x}_{1}=\mathbf{x}_{0}$). Each action pair $(a_{i}(x_{[k-T,k]}),u_{j}(x_{[k-T,k]}))$ defines the corresponding system dynamics at $k$. For instance, when we focus on sensor attacks (like replay or false data injection), let $\mathbf{\gamma}_{k}(a_{i}(x_{[k-T,k]}), u_{j}(x_{[k-T,k]}))$ be the control input with $(a_{i}(x_{[k-T,k]}),u_{j}(x_{[k-T,k]}))$, a subsystem $u_{j}(x_{[k-T,k]})$ with a Kalman filter, an optimal LQG controller has the following dynamics (we denote $(a_{i}(x_{[k-T,k]}),u_{j}(x_{[k-T,k]}))$ as $(a_{ik}, u_{jk})$ for convenience): 
\begin{align}
\begin{split}
&\mathbf{x}_{k}=\mathbf{Ax}_{k-1}+ \mathbf{Bu}_{k-1}+\mathbf{w}_{k-1},\\
& \mathbf{y}_{k}=\begin{cases}a_{1k} = \mathbf{Cx}_k+\mathbf{v}_{k},\ \text{without attack}\\
a_{ik}, i=2,\cdots, M, \ \ \text{with attack,} \end{cases}\\
&\hat{\mathbf{x}}_{k|k-1}= \mathbf{A\hat{x}}_{k-1|k-1}+\mathbf{Bu}_{k-1},\\
%&\mathbf{z}_{k+1}(h_{k},a_{ik},u_{jk})=a_{ik}(h_{k}) - \mathbf{C\hat{x}}_{k+1|k}(h_{k},a_{ik},u_{jk}),\\
&\hat{\mathbf{x}}_{k|k}(a_{ik}) =\hat{\mathbf{x}}_{k|k-1}+ \mathbf{K}(a_{ik} - \mathbf{C\hat{x}}_{k|k-1}),\\
%\end{split}
&\mathbf{\hat{x}}_{k+1|k}(a_{ik},u_{jk})=\mathbf{A\hat{x}}_{k|k}(a_{ik})+\mathbf{B\gamma}_{k}(a_{ik},u_{jk}),\\
& \mathbf{\gamma}_{k}(a_{ik}, u_{jk}) = \mathbf{L\hat{x}}_{k|k} (a_{ik}),\\%+\Delta \mathbf{u}_{k},\\
&\mathbf{z}_{k+1}(a_{ik},u_{jk})=a_{ik} - \mathbf{C\hat{x}}_{k+1|k}(a_{ik},u_{jk}).
\label{dynamicgame}
\end{split}
\end{align}
\textbf{State Transition Probability}: Given a set of subsystem models, define the state transition probability $P$ as a function of the state of the game and both players' actions $P:\ (X_{[k-T,k]}\times S) \times A_{t} \times A_{s}\to [0, 1],$
where
\\\centerline{$
P(s_{(k+1)h}|s_{kl},a_{ik}, u_{jk}), h=1,2,3
$}
%\end{align*}
is the probability that system transits from state $s_{kl}$ to state $s_{(k+1)h}$ at stage $k+1$, given both players' action $(a_{ik},u_{jk})$ at stage $k$. Given the current game state $s_{kl}=(x_{[k-T,k]}, \delta_l)$ and an action pair $(a_{ik},u_{jk})$, the dynamics of the system at stage $k+1$ is described as $x_{[k-T+1,k+1]}$ for all possible cyber modes $\delta_h \in S$, hence the dimension of state transition probability $P(s_{(k+1)h}|s_{kl},a_{ik}, u_{jk})$ is determined by the number of cyber modes of the game. We denote $P(s_{(k+1)h}|s_{kl}, a_{ik}, u_{jk})$ as $P^{ij}(s_{(k+1)h}|s_{kl})$ for short.
%and $\tilde{P}^{ij}(s_{(k+1)h}|s_{kl})$ is the entry at the $i$-th row and $j$-the column  of the state transition matrix $\tilde{P}(s_{(k+1)h}|s_{kl})$ of the game at hybrid state $s_{kl}$.
 As a state transition probability, this function should also satisfy
%\begin{align*}
\\\centerline{$\sum_{\delta_h \in S} {P}^{ij}(s_{(k+1)h}|s_{kl}) = 1,\quad \forall (a_{ik},u_{jk}) \in A_{t} \times A_{s},$}
\\\centerline{$s_{(k+1)h} \in (X_{[k-T+1,k+1]}\times S), s_{kl} \in(X_{[k-T,k]}\times S).$}
%\end{align*}
The transition probability is provided by intrusion detectors of the subsystem. 
%For computational efficiency, we assume that every element of the state transition matrix is a convex function of the system dynamics $x_{[k-T,k]}$ or can be convexified with bounded error. 
%For example, if a $\chi^{2}$ detector is the detector component of subsystem $u_{j}$, we apply~\eqref{alarm} to decide the state transition probability.

\textbf{Immediate Payoff Function}: The immediate payoff matrix at stage $k$ is a $\mathbb{R}^{M\times N}$ matrix for given game state and every action pair $(a_{ik}, u_{jk})$. We define the immediate payoff function as a continuous, convex function of the hybrid game state and the actions of both players
\\\centerline{$r: (X_{[k-T,k]}\times S) \times A_{t} \times A_{s} \to \mathbb{R}^{M \times N},$}
where $r(s_{kl}, a_{ik}, u_{jk}) \geqslant 0$ is the payoff at joint state $s_{kl}$ given action pair $(a_{ik}, u_{jk})$. For definition convenience, we denote ${r}(s_{kl}, a_{ik}, u_{jk})$ as ${r}^{ij}(s_{kl})$ for short, since it is the element on the $i$-th row and $j$-th column of the payoff matrix ${r}(s_{kl})$. It is a zero-sum game between the system and the attacker, and we assume the system is the minimizer and the attacker is the maximizer, hence the payoff function for the attacker and the system is defined as
\centerline{$
{r}^{ij}(s_{kl})={r}_t^{ij}(s_{kl})=-{r}_s^{ij}(s_{kl}).
$}
For instance, when the linear quadratic cost is a metric of system performance, let $\gamma_{k}(a_{ik}, u_{jk})$ be the control input given action pair $(a_{ik}, u_{jk})$, then the payoff function is defined as
\begin{align}
\begin{split}
{r}^{ij} (s_{k1}) =&\mathbb{E}[\mathbf{\hat{x}}^{T}_{k}]\mathbf{W}\mathbb{E}[\mathbf{\hat{x}}_{k}]+\mathbb{E}[\mathbf{\gamma}^{T}_{k}(a_{1k},u_{jk})]\mathbf{U}\mathbb{E}[\mathbf{\gamma}_{k}(a_{1k},u_{jk})],\\
{r}^{ij} (s_{k2}) =&\mathbb{E}[\mathbf{\hat{x}}^{T}_{k}]\mathbf{W}\mathbb{E}[\mathbf{\hat{x}}_{k}]+\mathbb{E}[\mathbf{\gamma}^{T}_{k}(a_{ik},u_{jk})]\mathbf{U}\mathbb{E}[\mathbf{\gamma}_{k}(a_{ik}, u_{jk})],\\
{r}^{ij} (s_{k3}) =& p_f,
\end{split}
\label{payoff}
\end{align}
where $p_f$ is the false alarm trigger penalty, the cost that the system needs to stop execution, check the reason of an alarm, and restart later; $\mathbf{x}_{k}$ is the physical state under the game framework. At mode $\delta_{1}$ the system wins, so the payoff is a normal system payoff with correct sensor data. The larger $p_f$ is, the less probable it is for the system to choose a strategy to transit to state $s_{k3}$.

\textbf{System dynamics update with strategies at stage k}:
 Let $p(s_{kl})$ be the probability system is at state $s_{kl}$ at stage $k$. The initial state distribution $p(s_{1l})$ is given. With  a strategy $\mathbf{f}_{k},\mathbf{g}_{k}$, the attacker and the system randomly sample an action pair $(a_{ik}, u_{jk})$ according to the probability distribution. Then, the control input and sensor value for calculating expectation cost are: 
%\begin{align*}
\centerline{$
\mathbf{u}_{k}=\sum\limits_{j=1}^{N}\sum\limits_{i=1}^{M} \sum\limits_{l=1}^{3}p(s_{kl})f_{k}^{i}(s_{kl})g^{j}_{k}(s_{kl})\mathbf{\gamma}_{k}(a_{ik},u_{jk}),
$}
$\text{ }\quad\quad\mathbf{y}_{k} =\sum\limits_{i=1}^{M}\sum\limits_{l=1}^{3} p(s_{kl})f_{k}^{i}(s_{kl}) a_{ik}.$
%\end{align*}
\\The probability that system is at state $s_{(k+1)h}$ for $k+1$ is:
\\\centerline{$
%\begin{align*}
p(s_{(k+1)h})= \sum\limits_{l=1}^{3}p(s_{kl})[\mathbf{f}_{k}(s_{kl})]^{T}{P}_{k}(s_{(k+1)h}|s_{kl})\mathbf{g}_{k}(s_{kl}). 
%\end{align*}
$}

\iffalse
\begin{align*}
\mathbf{F}_{k} :=\{&\mathbf{f}_{k}= [\mathbf{f}_{k}(s_{k1}), \mathbf{f}_{k}(s_{k2}), \mathbf{f}_{k}(s_{k3})]
|f_{k}^{i}(s_{kl})\geq 0,\\& \mathbf{f}_k \in [0,1]^{M\times 3} %&\forall a_{ik} \in A_{tk},  %k\in \{1,...,K\},\\
\sum \limits_{a_{ik} \in A_{tk}}f_{k}^{i}(s_{kl}) = 1,\mathbf{f}_{k}(s_{kl})\in \mathbb{R}^{M},\\&\forall s_{kl} \in(X_{[k-T,k]}\times S)\},\\
\mathbf{G}_{k}:=\{&\mathbf{g}_{k}= [\mathbf{g}_{k}(s_{k1}), \mathbf{g}_{k}(s_{k2}), \mathbf{g}_{k}(s_{k3})]|
g_{k}^{j}(s_{kl})\geq 0,\\& \mathbf{g}_k \in [0,1]^{N \times 3}, %&\forall u_{jk} \in A_{sk},%k \in \{1,...,K\},\\
\sum \limits_{u_{jk} \in A_{sk}}g_{k}^{j}(s_{kl}) = 1, \mathbf{g}_{k}(s_{kl}) \in \mathbb{R}^{N},\\ &\forall s_{kl} \in (X_{[k-T,k]}\times S)\}. 
\end{align*} 
\fi

%% This declares a command \Comment
%% The argument will be surrounded by /* ... */
\SetKwComment{Comment}{/* }{ */}

\begin{algorithm}[t]
\caption{Training Scheduler}\label{alg:TS}
% \KwData{$n \geq 0$}
% \KwResult{$y = x^n$}
\LinesNumbered
\KwIn{Training data $\mathcal{D}_{train}=\{(q_i, a_i, p_i^+)\}_{i=1}^m$, \\
\qquad \quad Iteration number $L$.}
\KwOut{A set of optimal model parameters.}

\For{$l=1,\cdots, L$}{
    Sample a batch of questions $Q^{(l)}$\\
    \For{$q_i\in Q^{(l)}$}{
        $\mathcal{P}_{i}^{(l)} \gets \mathrm{arg\,max}_{p_{i,j}}(\mathrm{sim}(q_i^{en},p_{i,j}),K)$\\
        $\mathcal{P}_{Gi}^{(l)} \gets \mathcal{P}_{i}^{(l)}\cup\{p^+_i\}$\\
        Compute $\mathcal{L}^i_{retriever}$, $\mathcal{L}^i_{postranker}$, $\mathcal{L}^i_{reader}$\\ according to Eq.\ref{eq:retriever}, Eq.\ref{eq:rerank}, Eq.\ref{eq:reader}\\
    }
    % $\mathcal{L}^{(l)}_{retriever} \gets \frac{1}{|Q^{(l)}|}\sum_i\mathcal{L}^i_{retriever}$\\
    % $\mathcal{L}^{(l)}_{retriever} \gets \mathrm{Avg}(\mathcal{L}^i_{retriever})$,
    % $\mathcal{L}^{(l)}_{rerank} \gets \mathrm{Avg}(\mathcal{L}^i_{rerank})$,
    % $\mathcal{L}^{(l)}_{reader} \gets \mathrm{Avg}(\mathcal{L}^i_{reader})$\\
    % Compute $\mathcal{L}^{(l)}_{retriever}$, $\mathcal{L}^{(l)}_{rerank}$, and $\mathcal{L}^{(l)}_{reader}$ by averaging over $Q^{(l)}$\\
    $\mathcal{L}^{(l)} \gets \frac{1}{|Q^{(l)}|}\sum_i(\mathcal{L}^{i}_{retriever} + \mathcal{L}^{i}_{postranker}+ \mathcal{L}^{i}_{reader})$\\
    $\mathcal{P}^{(l)}_K\gets\{\mathcal{P}^{(l)}_i|q_i\in Q^{(l)}\}$,\quad $\mathcal{P}^{(l)}_{KG}\gets\{\mathcal{P}^{(l)}_{Gi}|q_i\in Q^{(l)}\}$\\
    Compute the coefficient $v^{(l)}$ according to Eq.~\ref{eq:v}\\
  \eIf{$ v^{(l)}=1$}{
    $\mathcal{L}^{(l)}_{final} \gets \mathcal{L}^{(l)}(\mathcal{P}_{KG}^{(l)})$\\
  }{
      $\mathcal{L}^{(l)}_{final} \gets \mathcal{L}^{(l)}(\mathcal{P}^{(l)}_{K}),$\\
    }
    Optimize $\mathcal{L}^{(l)}_{final}$
}
\end{algorithm}


%  \eIf{$ \mathcal{L}^{(l-1)}_{retriever}<\lambda$}{
%     $\mathcal{L}^{(l)}_{final} \gets \mathcal{L}^{(l)}(\mathcal{P}_K^{(l)})$\\
%   }{
%       $\mathcal{L}^{(l)}_{final} \gets \mathcal{L}^{(l)}(\mathcal{P}^{(l)}_{KG}),$\\
%     }

\section{Simulation Studies}\label{sec:simulation}
In this section, we are mainly interested in the empirical performance of the ABESS algorithm on logistic regression and Poisson regression.
Logistic regression is widely used for classification tasks, and Poisson regression is appropriate when the response is a count.
\if0\informsMOR{In ``Additional Simulation'' of Supplementary Material, we }\else{We }\fi
also consider the performance of ABESS algorithm on multi-response linear regression (a.k.a., multi-task learning).
Before formally analyzing the simulation results,
we illustrate our simulation settings in Section~\ref{subsec:setup}.
% This subsection develops parallel with Section \ref{subsec:logistic}.

%In this section, we study the empirical performance of ABESS for GLM on two generalized linear models,
%logistic regression and gamma regression,
%where logistic regression is widely used for classification and
%gamma regression model is useful for modeling positive continuous response variables.
%Before formally studying logistic regression and gamma regression in Section~\ref{subsec:logistic} and Section~\ref{subsec:gamma}, respectively, we illustrate our simulation setting in Section~\ref{subsec:setup}.

\subsection{Setup}\label{subsec:setup}
To synthesize a dataset, we generate multivariate Gaussian realizations $\boldsymbol{x}_1, \ldots, \boldsymbol{x}_n \overset{i.i.d.}{\sim} \mathcal{MVN}(0,\Sigma)$,
where $\Sigma$ is a $p$-by-$p$ covariance matrix.
%We generate i.i.d error $\epsilon\sim N(0,\sigma^2)$.
%Define the signal to noise ratio (SNR) by $SNR = \frac{\beta^{\top}\Sigma\beta}{\sigma^2}$.
We consider two covariance structures for $\Sigma$: the independent structure ($\Sigma$ is an identity matrix)
and the constant structure ($\Sigma_{ij} = \rho^{I(i\neq j)}$ for some positive constant $\rho$). The value of $\rho$ and $p$ will be specified later.
We set the true regression coefficient $\boldsymbol{\beta}^*$ as a sparse vector with $k$ non-zero entries that have equi-spaced indices in $\{1, \ldots, p\}$.
Finally, given a design matrix $\mathbf{X} = (\boldsymbol{x}_1, \ldots, \boldsymbol{x}_n)^\top$ and $\boldsymbol{\beta}^*$,
we draw response realizations $\{y_i\}_{i=1}^n$ according to the GLMs.

We assess our proposal via the following criteria.
First, to measure the performance of subset selection,
we consider the probabilities of covering true active and inactive sets: $\mathbb{P}(\mathcal{A}^* \subseteq \hat{\mathcal{A}})$ and
$\mathbb{P}(\mathcal{I}^* \subseteq \hat{\mathcal{I}})$ (here, $\mathcal{I}^* = (\mathcal{A}^*)^c$).
We also consider exact support recover probability as $\mathbb{P}(\mathcal{A}^* = \hat{\mathcal{A}})$.
Since the probability is unknown, we empirically examine the proportion of recovery for the active set, inactive set, and exact recovery in 200 replications for instead.
As for parameter estimation performance, we examine relative error (ReErr) on parameter estimations:
$\|\hat{\boldsymbol{\beta}}-\boldsymbol{\beta}^*\|_{2} /\|\boldsymbol{\beta}^*\|_{2}$.
Finally, computational efficiency is directly measured by the runtime.

In addition to our proposed algorithms, we compare classical variable selection methods: LASSO \citep{tibshirani1996regression}, SCAD \citep{fan2001variable}, and MCP \citep{zhang2010nearly}.
%, and a recently proposed coordinate descent (CD) method for $\ell_0$-regularized classification \citep{antoine2021l0learn}.
For all these methods, we apply 10-fold cross-validation (CV) and the GIC to select the tuning parameter, respectively.
% For all these methods, we apply 10-fold cross-validation (CV) to select the tuning parameter.
% ABESS also uses generalized information criterion (GIC) \citep{fan2013tuning} because,
% by combining GIC, ABESS can consistently recover $\mathcal{A}^*$ under linear models \citep{zhu2020polynomial}.
The software for these methods is available at R CRAN (\url{https://cran.r-project.org}).
The software of all methods is summarized in Table~\ref{tab:implementation-details}.
All experiments are carried out on an R environment in a Linux platform with Intel(R) Xeon(R) Gold 6248 CPU @ 2.50GHz. 
% Note that, all experiments result are based on 200 random synthetic datasets.
%Ubuntu platform with Intel(R) Xeon(R) Gold 6248 CPU @ 2.50GHz.

% Model selection methods such as cross-validation and information criteria are widely used.
% Recently, \citet{fan2013tuning} explored generalized information criterion (GIC) in tuning parameter selection for
% penalized likelihood methods under GLM.
% Here, we use a GIC-type information criterion to recovery support size, which is defined as:
% $\mathrm{F}(\hat{\boldsymbol \beta}) = l_n( \hat{\boldsymbol \beta} ) + |\text{supp}(\hat{\boldsymbol \beta})| \log(p) \log\log n.$
% Intuitively speaking, the model complexity penalty term $|\text{supp}(\hat{\boldsymbol \beta})| \log p \log\log n$ is set to prevent over-fitting,
% where the term $\log\log n$ with a slow diverging rate is used to prevent under-fitting.
% Combining the Algorithm~\ref{alg:fbess} with GIC, we select the support size that minimizes the $F(\hat{\boldsymbol{\beta}})$.}

% \begin{table}[htbp]
% \caption{Implementation details for all methods.
% The values in the parentheses indicate the version number of R packages.}\label{tab:implementation-details}
% \centering
% \begin{tabular}{ccc}
% \toprule
% Method & Software & Tuning method \\
% \midrule
% ABESS-GIC & abess (0.4.0) & GIC \\
% LASSO-GIC & glmnet (4.1-3) & GIC \\
% SCAD-GIC & ncvreg (3.13.0) & GIC \\
% MCP-GIC & ncvreg (3.13.0) & GIC \\
% CD-GIC & L0Learn (2.0.3) & GIC \\
% ABESS-CV & abess (0.4.0) & 10-folds CV \\
% LASSO-CV & glmnet (4.1-3) & 10-folds CV \\
% SCAD-CV & ncvreg (3.13.0) & 10-folds CV \\
% MCP-CV & ncvreg (3.13.0) & 10-folds CV \\
% CD-CV & L0Learn (2.0.3) & 10-folds CV \\
% \bottomrule
% \end{tabular}
% \end{table}
\begin{table}[htbp]
\caption{Software for all methods.
The values in the parentheses indicate the version number of R packages.The tuning parameter within the MCP/SCAD penalty is fixed at 3/3.7.}\label{tab:implementation-details}
\centering
\if0\informsMOR{
% \begin{tabular}{ccccccc}
% \toprule
% Method & ABESS & LASSO & SCAD & MCP & CD \\
% \midrule
% Software & \textsf{abess} (0.4.0) & \textsf{glmnet} (4.1-3) & \textsf{ncvreg} (3.13.0) & \textsf{ncvreg} (3.13.0) & \textsf{L0Learn} (2.0.3) \\
% Tuning & sparsity $s$ & $\ell_1$ penalty & $\lambda$ & $\lambda$& $\lambda$ \\
% \bottomrule
% \end{tabular}
\begin{tabular}{cccccc}
    \toprule
    Method & ABESS & LASSO & SCAD & MCP \\
    \midrule
    Software & \textsf{abess} (0.4.0) & \textsf{glmnet} (4.1-3) & \textsf{ncvreg} (3.13.0) & \textsf{ncvreg} (3.13.0) \\
    Tuning & sparsity $s$ & $\ell_1$ penalty & $\lambda$ & $\lambda$ \\
    \bottomrule
    \end{tabular}
}\else{
% \begin{tabular}{ccccccc}
% \hline
% Method & ABESS & LASSO & SCAD & MCP & CD \\
% \hline
% Software & \textsf{abess} (0.4.0) & \textsf{glmnet} (4.1-3) & \textsf{ncvreg} (3.13.0) & \textsf{ncvreg} (3.13.0) & \textsf{L0Learn} (2.0.3) \\
% Tuning & sparsity $s$ & $\ell_1$ penalty & {\color{red}SCAD penalty} & {\color{red}MCP penalty} & $\ell_0$ penalty \\
% \hline
% \end{tabular}
\begin{tabular}{cccccc}
\hline
Method & ABESS & LASSO & SCAD & MCP \\
\hline
Software & \textsf{abess} (0.4.0) & \textsf{glmnet} (4.1-3) & \textsf{ncvreg} (3.13.0) & \textsf{ncvreg} (3.13.0) \\
Tuning & sparsity $s$ & $\ell_1$ penalty & {\color{red}SCAD penalty} & {\color{red}MCP penalty} \\
\hline
\end{tabular}
}\fi
\end{table}
% We implement our proposal in an R package abess \citep{zhu-abess-arxiv}.

\subsection{Logistic Regression}\label{subsec:logistic}
% In this subsection, we illustrate the power of ABESS on logistic regression, which is one of the most popular GLMs widely used for classification tasks.
% In terms of logistic regression, the response $y_i$ is a binary variable following a Bernoulli distribution $B(1, p_i)$,
% where $p_i \coloneqq \mathbb{P}(y_i=1)$ is determined by $\log(\frac{p_i}{1-p_i}) = \boldsymbol x_i^\top \boldsymbol{\beta }$.
% Here, the link function is known as the logit function, defined by $logit(p) = \log(\frac{p}{1-p})$.
% As a result, the negative log-likelihood is given by
% \begin{equation*}
% l_n(\boldsymbol\beta) = -\sum_{i=1}^{N}\left\{y_{i} \boldsymbol {x}_i^\top \boldsymbol \beta-\log \left(1+e^{\boldsymbol {x}_i^\top \boldsymbol \beta}\right)\right\}.
% \end{equation*}
% Empirically, we generate $x_i$ and $\beta$ as described in Section \ref{subsec:setup}.
% Binary response $y_i$ is then drawn from the Bernoulli distribution according to (\ref{eqn:formula_binomial}).
% Let $H_j = \sum\limits_{i=1}^{n} \frac{e^{\boldsymbol {x}_i^\top \hat{\boldsymbol \beta}}}{(1 + e^{\boldsymbol {x}_i^\top \hat{\boldsymbol \beta}})^2} x_{ij}^2$ and
% the gradient of $l_n(\boldsymbol{\beta})$ at $\hat{\boldsymbol{\beta}}$ be $\hat{\boldsymbol d} = -\sum\limits_{i=1}^{n}(y_i - \frac{e^{\boldsymbol {x}_i^\top \hat{\boldsymbol \beta}}}{1 + e^{\boldsymbol {x}_i^\top \hat{\boldsymbol \beta}}}) \boldsymbol {x}_i$,
% \eqref{eqn:approx_sacrifice} can be explicit expressed as:
% $\xi_j = H_j (\hat{\boldsymbol{\beta}}_j)^2$ for $j\in \mathcal{A}$ and
% $\zeta_j = H_j^{-1}( \hat{\boldsymbol d}_j )^2$ for $j\in \mathcal{I}$.
% \begin{equation*}
% \begin{aligned}
% % \hat{\boldsymbol d} &= -\sum_{i=1}^{n}(y_i - \frac{e^{\boldsymbol {x}_i^\top \boldsymbol \beta}}{1 + e^{\boldsymbol {x}_i^\top \boldsymbol \beta}}) \boldsymbol {x}_i. \\
% \xi_j
% & = H_j (\hat{\boldsymbol{\beta}}_j)^2, j\in \mathcal{A},\\
% \zeta_j
% & = H_j^{-1}
% ( \hat{\boldsymbol d}_j )^2, j\in \mathcal{I}.
% \end{aligned}
% \end{equation*}
% Given the explicit expression of \eqref{eqn:approx_sacrifice},
% we can conduct Algorithm~\ref{alg:abess} to estimate $\boldsymbol{\beta}$.

The dimension $p$ is fixed as 500 for the logistic regression model. For the constant correlation case, we set $\rho = 0.4$.
The non-zero coefficients $\boldsymbol{\beta}^*_{\mathcal{A}^*}$ are set to be $(2,2,8,8,8,8,10,10,10,10)^\top$. 
Now we compare methods listed in Table~\ref{tab:implementation-details}.
Figures~\ref{fig:rate_binomial} and \ref{fig:ReErr_binomial} present the results on subset selection and parameter estimation when the sample size increases. Out of clarity, we omit the CV results here and defer these results to the Additional Figures in Supplementary Material.


\begin{figure}[htbp]
\centering
\includegraphics[width=1.0\textwidth]{figure/rate_binomial_gic.pdf}
\if0\informsMOR{
\vspace{-30pt}
}\fi
\caption{Performance on subset selection under logistic regression when covariates have independent correlation structure (Upper) and constant correlation structure (Lower), measured by three kinds probabilities: $\mathbb{P}(\mathcal{A}^* \subseteq \hat{\mathcal{A}})$, $\mathbb{P}(\mathcal{I}^* \subseteq \hat{\mathcal{I}})$, and $\mathbb{P}(\mathcal{A}^* = \hat{\mathcal{A}})$ that are presented in Left, Middle and Right panels, respectively.
}
\label{fig:rate_binomial}
\end{figure}
\begin{figure}[htbp]
\centering
\includegraphics[width=0.8\textwidth]{figure/ReErr_binomial_gic.pdf}
\if0\informsMOR{
\vspace{-10pt}
}\fi
\caption{Performance on parameter estimation under logistic regression models when covariance matrices have independent correlation structure (Left) and exponential correlation structure (Right). The $y$-axis is the median of ReErr in a log scale.}
\label{fig:ReErr_binomial}
\end{figure}

As depicted in the left panel of Figure~\ref{fig:rate_binomial}, the probability $\mathbb{P}(\mathcal{A}^* \subseteq \hat{\mathcal{A}})$ approaches 1 as the sample size increases, indicating that all methods, except LASSO in the high correlation setting, can provide a no-false-exclusion estimator when the sample size is sufficiently large. However, when considering $\mathbb{P}(\mathcal{I}^* \subseteq \hat{\mathcal{I}})$, as observed in the middle panel of Figure~\ref{fig:rate_binomial}, the LASSO estimator consistently exhibits false inclusions, and the SCAD/MCP estimator shows false inclusions when the covariates are highly correlated. In contrast, only ABESS guarantees that $\mathbb{P}(\mathcal{I}^* \subseteq \hat{\mathcal{I}})$ approaches 1 for large sample sizes. 

Furthermore, as evident from the right panel of Figure~\ref{fig:rate_binomial}, ABESS accurately recovers the true subset under both correlation settings. While SCAD and MCP can also achieve exact support recovery given a sufficient sample size, ABESS demonstrates support recovery consistency with the smallest sample size, particularly when variables are correlated. It is important to note that although our theory imposes restrictions on the correlation among a small subset of variables (see Assumption~\ref{con:technical-assumption}), our algorithm still performs effectively in the constant correlation setting. This setting (i.e., $\rho=0.4$) violates Assumption~\ref{con:technical-assumption} as the correlation between any two variables exceeds 0.183, which is the maximum acceptable pairwise correlation satisfying Assumption~\ref{con:technical-assumption}.

Moving on to Figure~\ref{fig:ReErr_binomial}, it illustrates the superiority of ABESS in parameter estimation. ABESS visibly outperforms other methods in the small sample size regime and maintains highly competitive performance as the sample size increases. This superiority in parameter estimation is not surprising, as ABESS yields an oracle estimator when the support set is correctly identified. Although SCAD and MCP do not provide algorithmic guarantees for finding the local minimum, they exhibit competitive parameter estimation performance due to their asymptotic unbiasedness. Conversely, the LASSO estimator is biased and performs the worst among all the methods.

%\begin{figure}
%	\centering
%	\includegraphics[width=\textwidth]{figure/Performance_binomial.pdf}
%	\caption{Performance comparison under two correlation structures: independent and exponential. (A) Performance for subset selection, measured by support recover probability. (B) Performance for parameter estimation, measured by median ReErr. (C) Average runtime, measured in seconds. L0Learn is omitted since its runtime is far longer than others.}
%	\label{fig:Performance_binomial}
%\end{figure}

\subsection{Poisson Regression}\label{seubsec:poisson}
% As regard to Poisson regression, the response $y_i$ is a integer variable following a Poisson distribution $\mathcal{P}(\lambda_i)$ where $\lambda_i = \exp(\boldsymbol x_i^\top \boldsymbol{\beta})$.
% As a result, the negative log-likelihood is given by
% \begin{equation*}
% l_n(\boldsymbol\beta) = -\sum_{i=1}^{N}\left\{y_{i} \boldsymbol {x}_i^\top \boldsymbol\beta - e^{\boldsymbol {x}_i^\top \boldsymbol \beta} -\log(y_i!)\right\}.
% \end{equation*}
% Empirically, we generate $x_i$ and $\beta$ as described in Section \ref{subsec:setup}.
% Binary response $y_i$ is then drawn from the Bernoulli distribution according to (\ref{eqn:formula_binomial}).
% Let $H_j = \sum\limits_{i=1}^{n} \exp(\boldsymbol {x}_i^\top \hat{\boldsymbol \beta}) x_{ij}^2$ and
% the gradient of $l_n(\boldsymbol{\beta})$ at $\hat{\boldsymbol{\beta}}$ be $\hat{\boldsymbol d} = -\sum\limits_{i=1}^{n}(y_i - \exp(\boldsymbol {x}_i^\top \hat{\boldsymbol \beta})) \boldsymbol {x}_i$,
% \eqref{eqn:approx_sacrifice} can be explicit expressed as:
% $\xi_j = H_j (\hat{\boldsymbol{\beta}}_j)^2$ for $j\in \mathcal{A}$ and
% $\zeta_j = H_j^{-1}( \hat{\boldsymbol d}_j )^2$ for $j\in \mathcal{I}$.
% Given the explicit expression of \eqref{eqn:approx_sacrifice},
% we can conduct Algorithm~\ref{alg:abess} to estimate $\boldsymbol{\beta}$.


For the Poisson regression model, we consider a fixed $p$ value of 500, and set $\rho = 0.2$ for the constant correlation case. The non-zero coefficients $\boldsymbol{\beta}^*_{\mathcal{A}^*}$ are specified as $(1, 1, 1)^\top$. Figures~\ref{fig:rate_poisson_gic}-\ref{fig:ReErr_poisson_gic} present the evaluation of subset selection and parameter estimation quality. Examining Figures~\ref{fig:rate_poisson_gic}, we observe that for ABESS/SCAD/MCP, the probabilities $\mathbb{P}(\mathcal{A}^* \subseteq \hat{\mathcal{A}})$, $\mathbb{P}(\mathcal{I}^* \subseteq \hat{\mathcal{I}})$, and $\mathbb{P}(\mathcal{A}^* = \hat{\mathcal{A}})$ gradually approach 1 as the sample size $n$ increases. In contrary, the LASSO, regardless of the highest inclusion probability for $\mathcal{A}^*$, still has a chance of including ineffective variables, especially when variables are correlated. Comparing ABESS, SCAD, and MCP, it is evident that ABESS achieves the highest exact selection probability, followed by SCAD and MCP. Similar to the results in logistic regression, ABESS achieves exact selection of the effective variables with the smallest sample size under the constant correlation structure.
Regarding the quality of parameter estimation, the ReErr of all methods reasonably decreases as the sample size $n$ increases. Again, ABESS exhibits the least estimation error in terms of the $\ell_2$-norm, which coincides with the results on logistic regression. It is worth noting that our method demonstrates consistency and polynomial complexity under Poisson regression, even though it violates the sub-Gaussian assumption. This is because the current framework of proofs allows for the relaxation of Assumption~\ref{con:subgaussian} to a sub-exponential distribution assumption, enabling the establishment of similar theoretical properties.

\begin{figure}[htbp]
\centering
\includegraphics[width=1.0\textwidth]{figure/rate_poisson_gic.pdf}
\if0\informsMOR{
\vspace{-30pt}
}\fi
\caption{Performance on subset selection under Poisson regression when covariates have independent correlation structure (Upper) and constant correlation structure (Lower), measured by three kinds probabilities: $\mathbb{P}(\mathcal{A}^* \subseteq \hat{\mathcal{A}})$, $\mathbb{P}(\mathcal{I}^* \subseteq \hat{\mathcal{I}})$, and $\mathbb{P}(\mathcal{A}^* = \hat{\mathcal{A}})$ that are presented in Left, Middle and Right panels, respectively.}
\label{fig:rate_poisson_gic}
\end{figure}
\begin{figure}[htbp]
\centering
\includegraphics[width=0.8\textwidth]{figure/ReErr_poisson_gic.pdf}
\if0\informsMOR{
\vspace{-5pt}
}\fi
\caption{Performance on parameter estimation under Poisson regression models when covariance matrices have independent correlation structure (Left) and exponential correlation structure (Right). The $y$-axis is the median of ReErr in a log scale.}
\label{fig:ReErr_poisson_gic}
\end{figure}

\subsection{Computational analysis}

We compare the runtime of different methods in Table~\ref{tab:implementation-details} for the logistic regression and Poisson regression models in Sections~\ref{subsec:logistic} to \ref{seubsec:poisson}. The runtime results are summarized in Figure~\ref{fig:simu_runtime}, indicating that ABESS demonstrates superior computational efficiency compared to state-of-the-art variable selection methods. For instance, when $n = 3000$, ABESS is at least four times faster than its competitors in logistic regression under an independent correlation structure. Furthermore, regardless of logistic regression or Poisson regression, ABESS exhibits similar computational performance, while other competitors run much faster when the pairwise correlation is higher. Lastly, it is important to note that the runtime of ABESS scales polynomially with sample sizes, aligning with the complexity presented in Theorem~\ref{thm:complexity}.
%In contrast, the runtime of other methods grows more rapidly as the sample size increases
%and appears like a quadratic function of the sample size in the independent scenario.
%Increasing iteration numbers for convergence may lead to this result.
%Moreover, ABESS-GIC is faster than ABESS-CV, demonstrating the superiority of the proposed adaptive parameter tuning procedure.
% Finally, according to the computational comparison presented in Figure~\ref{fig runtime_poisson_gic}, the ABESS has the least runtime and is much faster than the MCP and SCAD when variables are independent.

\begin{figure}[htbp]
\centering
\includegraphics[width=0.8\textwidth]{figure/runtime_binomial_gic.pdf}
\includegraphics[width=0.8\textwidth]{figure/runtime_poisson_gic.pdf}
\if0\informsMOR{
\vspace{-10pt}
}\fi
\caption{Average runtime (measured in seconds) on logistic regression (Upper panel) and Poisson regression (Lower panel). The results on two types of covariances matrix $\Sigma$, the independent correlation structure and constant correlation structure, are presented in the left and right panels, respectively. The error bars represent two times the standard errors.
}
\label{fig:simu_runtime}
\end{figure}

%% For the MOR template, uncomment this line and comment on the code blocks

\if1\informsMOR
{
\input{../appendix_numerical}
}\fi


\bibliographystyle{IEEEtran}
{  \small
\bibliography{Greplay2}
}
\end{document}\end{document} 