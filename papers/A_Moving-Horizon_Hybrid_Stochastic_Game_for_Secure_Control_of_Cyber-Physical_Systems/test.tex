\documentclass[letterpaper, 10pt, conference, english]{ieeeconf}   % Comment this line out
                                                          % if you need a4paper
%\documentclass[a4paper, 10pt, conference]{ieeeconf}      % Use this line for a4
                                                          % paper

%\IEEEoverridecommandlockouts                              % This command is only
                                                          % needed if you want to
                                                          % use the \thanks command
\overrideIEEEmargins
% See the \addtolength command later in the file to balance the column lengths
% on the last page of the document

%\documentclass[conference]{IEEEtran}
%\usepackage[latin9]{inputenc}
%\usepackage[letterpaper]{geometry}
%\geometry{verbose,tmargin=1in,bmargin=1in,lmargin=1in,rmargin=1in}
%\usepackage[cmex10]{amsmath}
%\usepackage{amssymb}
\usepackage{graphicx}
%%\usepackage[dvipdfmx]{graphicx}
%\usepackage{float}
%\usepackage{array}
%\usepackage{tikz}
%\usepackage{amssymb}
\usepackage{epstopdf}
%%\usepackage{ifpdf}
%\usepackage{cite}
%\usepackage{algorithm}
%\usepackage{algorithmic}
%\usepackage{amsthm}
%
%\DeclareGraphicsExtensions{.eps,.pdf,.jpg,.png}
%\DeclareGraphicsRule{.jpg}{eps}{.bb}{}
%%%%%%\newtheorem{lemma}{Lemma}
%%%%%%\newcommand{\ind}[1]{\mathbf{1}\left(#1\right)}
%%%%%%\newcommand{\bx}{\mathbf{x}}
%%%%%%\newcommand{\E}{\mathbf{E}}
%%%%%%\newtheorem{theorem}{Theorem}%[section]
%%%%%%\newenvironment{definition}[1][Definition]{\begin{trivlist}
%%%%%%\item[\hskip \labelsep {\bfseries #1}]}{\end{trivlist}}
%\hyphenation{op-tical net-works semi-conduc-tor}

\usepackage{amsmath,amsfonts, amssymb}
\newtheorem{problem}{Problem}
\newtheorem{theorem}{Theorem}
\newtheorem{corollary}{Corollary}
\newtheorem{lemma}{Lemma}
\newtheorem{remark}{Remark}
\newtheorem{definition}{Definition}
\newtheorem{example}{Example}
\newtheorem{algorithm}{Algorithm}
%\newtheorem{proof}{Proof}

% The following packages can be found on http:\\www.ctan.org
%\usepackage{graphics} % for pdf, bitmapped graphics files
\usepackage{epsfig} % for postscript graphics files
\usepackage{psfrag}
%\usepackage{mathptmx} % assumes new font selection scheme installed
%\usepackage{times} % assumes new font selection scheme installed
%\usepackage{amsmath} % assumes amsmath package installed
%\usepackage{amssymb}  % assumes amsmath package installed

\def\endtheorem{\hspace*{\fill}~\QEDopen\par\endtrivlist\unskip}
\def\endlemma{\hspace*{\fill}~\QEDopen\par\endtrivlist\unskip}
\def\endcorollary{\hspace*{\fill}~\QEDopen\par\endtrivlist\unskip}
\def\endexample{\hspace*{\fill}~\QEDopen\par\endtrivlist\unskip}
\def\endremark{\hspace*{\fill}~\QEDopen\par\endtrivlist\unskip}
\def\enddefinition{\hspace*{\fill}~\QEDopen\par\endtrivlist\unskip}

\usepackage{times}
\usepackage[tight,footnotesize]{subfigure}
\usepackage{bibspacing}
%\setlength{\bibspacing}{\baselineskip}

\newcommand{\PA}[1]{$\spadesuit$\footnote{PAJA: #1}}
\newcommand{\FM}[1]{$\clubsuit$\footnote{FEI: #1}}
\newcommand\fixme[1]{$\star$ \emph{\small #1} $\star$}

\newcommand{\tableref}[1]{Table~\ref{tab:#1}}
\newcommand{\figref}[1]{Figure~\ref{fig:#1}}


%%%%%%%%%%%%%%%%%%%%%%%
\usepackage{verbatim}
\usepackage{amsmath}
\usepackage{babel}
\usepackage{listings}
%%%%%%%%%%%%%%%%%%%%\usepackage[]{graphicx}
\usepackage{comment}
\usepackage{multirow}
\usepackage{algorithmic}
\usepackage{algorithm}

%%%% Math Packages %%%%%%%%%%%%
%%\usepackage{amsmath}
%%\usepackage{amsthm}
%%\usepackage{amsfonts}

\usepackage[tight]{subfigure}
%%%%%%\usepackage{wrapfig}


%% Line Spacing %%%%%%%%%%%%%%%%%%%%%%%%%%%%%%%%%%%%%%%%%%%%%
%\usepackage{setspa ce}
%\singlespacing        %% 1-spacing (default)
%\onehalfspacing       %% 1,5-spacing
%\doublespacing        %% 2-spacing
%\usepackage[belowskip=-15pt,aboveskip=0pt]{caption}
%\setlength{\intextsep}{10pt plus 2pt minus 2pt}
%\allowdisplaybreaks
\begin{document}
\title{\LARGE \bf Stochastic Game Approach for Attack Detection}
\author{Fei Miao \and Quanyan Zhu%\and Rahul Mangharam\and George J. Pappas% <-this % stops a space
%\thanks{This research has been partially supported by the NSF-CPS 0931239 grant.}% <-this % stops a space
%\thanks{F. Miao, M. Pajic, R. Mangharam and G. J. Pappas are with the Department of Electrical and Systems Engineering, University of Pennsylvania, Philadelphia, PA, USA 19014. Email: \{{\tt miaofei,pajic,rahulm,pappasg\}@seas.upenn.edu}.}%
}
\maketitle

\input{model_replay}

\section{A Hybrid Stochastic Game Model}
\label{sec:game_form}
To obtain a switching policy that minimizes the expected real-time worst case payoff for the given subsystems, 
we formulate a zero-sum, hybrid stochastic game between the system and the attacker. System dynamics knowledge are combined with the game definition, and the quantitative process for the game parameters will be introduced in this section. We assume that one game stage $k$ is also one time step of the physical system. The total stage number is $K$. The hybrid game state space $(X_{[k-T,k]}\times S)$ contains information about both the system dynamics $\mathbf{x}_k$ and the discrete modes $\delta_l, l=1,2,3$. Here, $T$ is the window size of system dynamics needed to keep the state transition between stages $k$ and $(k+1)$ Markov. The joint state includes information we need to compute the game strategy at the current stage. This is the main difference compared with the previous work~(\cite{cdc_replay}), while the latter is not Markov since it needs to consider all the possible histories of strategies for deciding the physical dynamics and getting a strategy. At each stage $k \in \{T,\cdots, K+T\}$, parameters include the action space for the attacker (system) $A_{t}$ ($A_{s}$), the state transition probability matrix $\mathbb{P}_{k}$, and the immediate payoff matrix $r_{k}$. The solution set of the game is mixed strategies $\mathbf{F}_{k}$ for the attacker, and $\mathbf{G}_{k}$ for the system. Formally, the game is defined as a sequence of tuples:
 $\{(X_{[k-T,k]} \times S),A_{t},A_{s}, \mathbf{F}_{k},\mathbf{G}_{k}, P, r\}$.

\iffalse
\begin{table*}
\centering
%\caption{Parameters of the hybrid stochastic game between the system and the attacker}
\begin{tabular}{|c|c|}
  \hline
   $s_{kl}=(x_{[k-T,k]}, \delta_l)$& Joint game state: sequence of physical dynamics, and piecewise constant mode \\ \hline
   $A_{t}$ & Attacker's action space \\ \hline
   $A_{s}$ & System's action space \\ \hline  
   $\mathbf{f}_k(s_{kl})$ & Strategy of the attacker at stage $k$, 
                                           state $s_{kl}, l=1,2,3$  \\ \hline     
      $\mathbf{g}_k(s_{kl})$ & Strategy of the system at stage $k$, 
                                            state $s_{kl}, l=1,2,3$  \\ \hline     
    ${P}(s_{(k+1)h}|s_{kl})$ & Probability that system transits from state $s_{kl}$ 
                                              at stage $k$ to state $s_{(k+1)l}$ at stage $k+1$\\ \hline
    ${r}(s_{kl})$ & Immediate payoff matrix at stage $k$ \\ 
   \hline
\end{tabular}
\centering
%\captionsetup{justification=centerlast}
\caption{Parameters of the hybrid stochastic game between the system and the attacker}
\label{game_parameter}
\end{table*} 
\fi

\textbf{Game State Space}: The joint state of the system at stage $k$ is described by the pair $s_{kl}=(x_{[k-T,k]}, \delta_l)$, where
\centerline{$
x_{[k-T,k]}=(x_{k-T}, x_{k-T+1}, \cdots, x_k ) \in X_{[k-T,k]}$} is the discrete-time dynamics of the physical process provided to the system--the state estimations $\hat{x}_{k-T},\cdots, \hat{x}_k$, $\delta_l \in S=\{\delta_1,\delta_2, \delta_3\}$ denote the cyber state of the system. We assume that once the game reach $\delta_1$, the system wins and will not enter other modes till next game, i.e., $\delta_1$ is an absorbing state. The moving-horizon transition of the joint states on stage axis is shown as Figure~\ref{sg}. The window size of system dynamics $T$ keeps the state transition between time $k$ and $k+1$ Markov. For instance, if the detector of the system requires system dynamics $\hat{x}_{[k-T_1,k]}$, and we consider sensor data injection attacks and replay attacks with replay windows less than $T_2$ steps, then $T=max\{T_1, T_2\}$. 
%With the system dynamics $\hat{x}_{[k-T, k]}$ denoted as $x_{[k-T,k]}$ for game stage $k$,  information needed to define the following action space of two players, the payoff and state transition probability is included. 

%i.e., the detector needs information for T steps to decide whether the alarm should be triggered. 
%\FM{In replay attack, we can say: once alarm is triggered, the system can stop the execution and check whether attack occurred, is this true for other attacks? Can the system distinguish between successfully detection and false alarm trigger?}
\begin{figure}[t!]
%\vspace{-5pt}
\centering
\includegraphics [width=0.32\textwidth]{xk.pdf}
\vspace{-8pt}
\caption{Joint state transition of the hybrid stochastic game when moving the horizon of game state one step ahead. When the state transits from stage $k$ to $k+1$, we slice the window of the sequence of physical dynamics one step ahead, add $x_{k+1}$ and remove $x_{k-T}$,  thus $x_{[k-T,k]} \to x_{[k-T+1,k+1]}$. The piecewise constant modes $\delta_l$, $\delta_h$ describe the cyber states provided by the detector at stage $k$, respectively.}
\label{sg}
\vspace{-5pt}
\end{figure}

%For simplicity, we omit the subscript k and just write state at every time k as $s_i,  i=1,2,3$.
\textbf{Attacker's Action Space}: We assume that the system is vulnerable to different attack models described by the action space $A_{t}$, where 
\\\centerline{$
A_{t}=\{a_{1}(x_{[k-T,k]}), a_{2}(x_{[k-T,k]}), \cdots, a_{M}(x_{[k-T,k]})\}
$}
is the attacker's action space at stage $k$, and $a_{1}$ means no attack. Here we only consider discretized action space of the attacker for computational efficiency. For the LTI system dynamics considered in this work, the distance of a continuous point to its nearest discrete point in action space is bounded. With bounded error of the dynamics by discretized continuous action space, the quality of game solutions under different conditions is analyzed by work~\cite{disaction}. 

The actions can describe both multiple types of attacks and the same type attack with different values. For instance, when considering only sensor data injection attacks with different norms of injection value, we will denote $a_i (x_{[k-T,k]}), i=2, 3,\dots$ as changing the sensor value from $\mathbf{y}_k=\mathbf{Cx}_k+\mathbf{v}_k$ to $\mathbf{y}'_k= \mathbf{y}_k+\mathbf{y}_{k,i}^a$, where any injection $\mathbf{y}_k^a$ is classified as $a_i (x_{[k-T,k]}), i=inf\{i:\ \|\mathbf{y}_k^a-\mathbf{y}_{k,i}^a\|_2\}$ in attacker's action space.
Similarly, for replay attack only, the action space is discretized as changing sensor values from $\mathbf{y}_k=\mathbf{Cx}_k+\mathbf{v}_k$ to $\mathbf{y}'_k= \mathbf{y}_{k-T_i}$ for action index $a_i (x_{[k-T,k]})$, where any replay time length $T_a$ is classified as $a_i (x_{[k-T,k]}), i=inf\{i:|T_a - T_i|\}$. Considering multiple types of attacks, we assume that the system is valnerable under $m_a$ types of attacks, and attack type $A_i$ is corresponding to $M_{a,i}$ discretized actions in the action space, then there are $\sum_{i=1}^{m_a} M_{a,i}+1$ actions in total within the attacker's action space $A_t$.
 

% with discretized, bounded norm $\|\mathbf{y}_{k,i}^a\|_2 \leqslant b$, since the attacker has limited energy for every data injection. This means any injection data that satisfies $\|\mathbf{y}_{k}^a\|_2 \leqslant b$ is considered as $inf\{i:\ \|\mathbf{y}_k^a-\mathbf{y}_{k,i}^a\|_2\}$ in attacker's action space. 

%For example, 
%when considering replay attacks and false data injection attacks, we take $a_{2k}$ as~\eqref{replay_y} for a given replay window size, and $a_{3k}$ as~\eqref{attackmodel} for a given data injection range. 

\iffalse by any given controller/estimator/detector combination of the system \fi
%; here, $y_{k}$ is the real sensor value and $\mathbf{y}_{k-t_{i}}$ denotes any replay sensor value in the strategy set. %\FM{here we assume during time $k \in \{1,...,K\}$ the replay window size $T$ does not change for simulation.}
%\item
%\FM{rewrite the action space definition}

\textbf{System's Action Space}: The system's action space at stage $k$ is defined as
\\\centerline{$
A_{s}=\{u_{1}(x_{[k-T,k]}), u_{2}(x_{[k-T,k]}),\cdots, u_{N}(x_{[k-T,k]})\},
$}  %= \{\mathbf{u}^{*}_{k}, \mathbf{u}^{*}_{k} + \Delta \mathbf{u}_{k}\}$
where $u_{j}$ is the index for the $j$th subsystem. We assume that the $N$ subsystems (a model for each component in Figure~\ref{system}) are determined priorly. For example, a subsystem can be the plant with a given optimal LQG controller, a Kalman filter and a $\chi^2$ detector. A subsystem can also be the plant with an optimal LQG controller, a resilient state estimator~\cite{res_estimator} and its corresponding estimation residual checking component. We assume that the attacker's action space is defined, with corresponding system's action or a subsystem that the detection rate is greater than $0$. A switched system does not ensure performance under the attack outside the action space of the game.

\textbf{Mixed Strategy}: Let $f^{i}_{k}(s_{kl})$ ($g^{j}_{k}(s_{kl})$) be the probability that the attacker (system) chooses action $a_{i}(x_{[k-T,k}) \in A_{t}$ ($u_{j}(x_{[k-T,k}) \in A_{s}$) at state $s_{kl}\in (X_{[k-T,k]}\times S)$. Define $\mathbf{F}_{k}$ and $\mathbf{G}_{k}$ as the mixed strategy sets of the attacker and the system for stage $k$:
$\mathbf{F}_{k} :=\{\mathbf{f}_{k}= [\mathbf{f}_{k}(s_{k1}), \mathbf{f}_{k}(s_{k2}), \mathbf{f}_{k}(s_{k3})]
|f_{k}^{i}(s_{kl})\geq 0, \mathbf{f}_k \in [0,1]^{M\times 3},
\sum \limits_{a_{ik} \in A_{tk}}f_{k}^{i}(s_{kl}) = 1,\mathbf{f}_{k}(s_{kl})\in \mathbb{R}^{M}, \forall s_{kl} \in(X_{[k-T,k]}\times S)\},$
$\mathbf{G}_{k}:=\{\mathbf{g}_{k}= [\mathbf{g}_{k}(s_{k1}), \mathbf{g}_{k}(s_{k2}), \mathbf{g}_{k}(s_{k3})]|$
$g_{k}^{j}(s_{kl})\geq 0, \mathbf{g}_k \in [0,1]^{N \times 3}, %&\forall u_{jk} \in A_{sk},%k \in \{1,...,K\},\\
\sum \limits_{u_{jk} \in A_{sk}}g_{k}^{j}(s_{kl}) = 1, \mathbf{g}_{k}(s_{kl}) \in \mathbb{R}^{N}, \forall s_{kl} \in (X_{[k-T,k]}\times S)\}. $ Note that $\mathbf{x}_{[k-T,k]}$ provides exogenous information for the strategy $\mathbf{f}_k (\mathbf{g}_k)$, since for every $l$, $\mathbf{f}_{k}(s_{kl}) (\mathbf{g}_{k}(s_{kl}))$ is the strategy at mode $\delta_l$ for the same $\mathbf{x}_{[k-T,k]}$ at stage $k$. Hence, $\mathbf{g}_k$ and $\mathbf{f}_k$ are finite dimensional vectors, that the stationary strategy chosen by each player at stage $k$ depends on the cyber state. %Mixed strategy set $\mathbf{F}_k$ also include the case that the attacker only implement one specific type of attack in the action space at time instance $k$, since we do not have know the strategy of the attacker, we are able to consider all possible combinations of attacks by exploiting mixed strategies. 

%\textbf{}:
%\label{dynamicgame}
%With all the above definition, %for any strategy history $h_{k}$ (a sequence of switching policy), 


\textbf{System and Subsystem Dynamics under game framework}: Given the subsystem and attack models in Section~\ref{sec:replay1} and the game definition,  
%we can transform it to a new system dynamic model decided by both players' actions. 
we show the dynamics at stage $k$ given an action pair $(a_{i}(x_{[k-T,k}),u_{j}(x_{[k-T,k}))$ (assume initial $\mathbf{\hat{x}}_{1|0}=\bar{\mathbf{x}}_{0}$, $\mathbf{x}_{1}=\mathbf{x}_{0}$). Each action pair $(a_{i}(x_{[k-T,k]}),u_{j}(x_{[k-T,k]}))$ defines the corresponding system dynamics at $k$. For instance, when we focus on sensor attacks (like replay or false data injection), let $\mathbf{\gamma}_{k}(a_{i}(x_{[k-T,k]}), u_{j}(x_{[k-T,k]}))$ be the control input with $(a_{i}(x_{[k-T,k]}),u_{j}(x_{[k-T,k]}))$, a subsystem $u_{j}(x_{[k-T,k]})$ with a Kalman filter, an optimal LQG controller has the following dynamics (we denote $(a_{i}(x_{[k-T,k]}),u_{j}(x_{[k-T,k]}))$ as $(a_{ik}, u_{jk})$ for convenience): 
\begin{align}
\begin{split}
&\mathbf{x}_{k}=\mathbf{Ax}_{k-1}+ \mathbf{Bu}_{k-1}+\mathbf{w}_{k-1},\\
& \mathbf{y}_{k}=\begin{cases}a_{1k} = \mathbf{Cx}_k+\mathbf{v}_{k},\ \text{without attack}\\
a_{ik}, i=2,\cdots, M, \ \ \text{with attack,} \end{cases}\\
&\hat{\mathbf{x}}_{k|k-1}= \mathbf{A\hat{x}}_{k-1|k-1}+\mathbf{Bu}_{k-1},\\
%&\mathbf{z}_{k+1}(h_{k},a_{ik},u_{jk})=a_{ik}(h_{k}) - \mathbf{C\hat{x}}_{k+1|k}(h_{k},a_{ik},u_{jk}),\\
&\hat{\mathbf{x}}_{k|k}(a_{ik}) =\hat{\mathbf{x}}_{k|k-1}+ \mathbf{K}(a_{ik} - \mathbf{C\hat{x}}_{k|k-1}),\\
%\end{split}
&\mathbf{\hat{x}}_{k+1|k}(a_{ik},u_{jk})=\mathbf{A\hat{x}}_{k|k}(a_{ik})+\mathbf{B\gamma}_{k}(a_{ik},u_{jk}),\\
& \mathbf{\gamma}_{k}(a_{ik}, u_{jk}) = \mathbf{L\hat{x}}_{k|k} (a_{ik}),\\%+\Delta \mathbf{u}_{k},\\
&\mathbf{z}_{k+1}(a_{ik},u_{jk})=a_{ik} - \mathbf{C\hat{x}}_{k+1|k}(a_{ik},u_{jk}).
\label{dynamicgame}
\end{split}
\end{align}
\textbf{State Transition Probability}: Given a set of subsystem models, define the state transition probability $P$ as a function of the state of the game and both players' actions $P:\ (X_{[k-T,k]}\times S) \times A_{t} \times A_{s}\to [0, 1],$
where
\\\centerline{$
P(s_{(k+1)h}|s_{kl},a_{ik}, u_{jk}), h=1,2,3
$}
%\end{align*}
is the probability that system transits from state $s_{kl}$ to state $s_{(k+1)h}$ at stage $k+1$, given both players' action $(a_{ik},u_{jk})$ at stage $k$. Given the current game state $s_{kl}=(x_{[k-T,k]}, \delta_l)$ and an action pair $(a_{ik},u_{jk})$, the dynamics of the system at stage $k+1$ is described as $x_{[k-T+1,k+1]}$ for all possible cyber modes $\delta_h \in S$, hence the dimension of state transition probability $P(s_{(k+1)h}|s_{kl},a_{ik}, u_{jk})$ is determined by the number of cyber modes of the game. We denote $P(s_{(k+1)h}|s_{kl}, a_{ik}, u_{jk})$ as $P^{ij}(s_{(k+1)h}|s_{kl})$ for short.
%and $\tilde{P}^{ij}(s_{(k+1)h}|s_{kl})$ is the entry at the $i$-th row and $j$-the column  of the state transition matrix $\tilde{P}(s_{(k+1)h}|s_{kl})$ of the game at hybrid state $s_{kl}$.
 As a state transition probability, this function should also satisfy
%\begin{align*}
\\\centerline{$\sum_{\delta_h \in S} {P}^{ij}(s_{(k+1)h}|s_{kl}) = 1,\quad \forall (a_{ik},u_{jk}) \in A_{t} \times A_{s},$}
\\\centerline{$s_{(k+1)h} \in (X_{[k-T+1,k+1]}\times S), s_{kl} \in(X_{[k-T,k]}\times S).$}
%\end{align*}
The transition probability is provided by intrusion detectors of the subsystem. 
%For computational efficiency, we assume that every element of the state transition matrix is a convex function of the system dynamics $x_{[k-T,k]}$ or can be convexified with bounded error. 
%For example, if a $\chi^{2}$ detector is the detector component of subsystem $u_{j}$, we apply~\eqref{alarm} to decide the state transition probability.

\textbf{Immediate Payoff Function}: The immediate payoff matrix at stage $k$ is a $\mathbb{R}^{M\times N}$ matrix for given game state and every action pair $(a_{ik}, u_{jk})$. We define the immediate payoff function as a continuous, convex function of the hybrid game state and the actions of both players
\\\centerline{$r: (X_{[k-T,k]}\times S) \times A_{t} \times A_{s} \to \mathbb{R}^{M \times N},$}
where $r(s_{kl}, a_{ik}, u_{jk}) \geqslant 0$ is the payoff at joint state $s_{kl}$ given action pair $(a_{ik}, u_{jk})$. For definition convenience, we denote ${r}(s_{kl}, a_{ik}, u_{jk})$ as ${r}^{ij}(s_{kl})$ for short, since it is the element on the $i$-th row and $j$-th column of the payoff matrix ${r}(s_{kl})$. It is a zero-sum game between the system and the attacker, and we assume the system is the minimizer and the attacker is the maximizer, hence the payoff function for the attacker and the system is defined as
\centerline{$
{r}^{ij}(s_{kl})={r}_t^{ij}(s_{kl})=-{r}_s^{ij}(s_{kl}).
$}
For instance, when the linear quadratic cost is a metric of system performance, let $\gamma_{k}(a_{ik}, u_{jk})$ be the control input given action pair $(a_{ik}, u_{jk})$, then the payoff function is defined as
\begin{align}
\begin{split}
{r}^{ij} (s_{k1}) =&\mathbb{E}[\mathbf{\hat{x}}^{T}_{k}]\mathbf{W}\mathbb{E}[\mathbf{\hat{x}}_{k}]+\mathbb{E}[\mathbf{\gamma}^{T}_{k}(a_{1k},u_{jk})]\mathbf{U}\mathbb{E}[\mathbf{\gamma}_{k}(a_{1k},u_{jk})],\\
{r}^{ij} (s_{k2}) =&\mathbb{E}[\mathbf{\hat{x}}^{T}_{k}]\mathbf{W}\mathbb{E}[\mathbf{\hat{x}}_{k}]+\mathbb{E}[\mathbf{\gamma}^{T}_{k}(a_{ik},u_{jk})]\mathbf{U}\mathbb{E}[\mathbf{\gamma}_{k}(a_{ik}, u_{jk})],\\
{r}^{ij} (s_{k3}) =& p_f,
\end{split}
\label{payoff}
\end{align}
where $p_f$ is the false alarm trigger penalty, the cost that the system needs to stop execution, check the reason of an alarm, and restart later; $\mathbf{x}_{k}$ is the physical state under the game framework. At mode $\delta_{1}$ the system wins, so the payoff is a normal system payoff with correct sensor data. The larger $p_f$ is, the less probable it is for the system to choose a strategy to transit to state $s_{k3}$.

\textbf{System dynamics update with strategies at stage k}:
 Let $p(s_{kl})$ be the probability system is at state $s_{kl}$ at stage $k$. The initial state distribution $p(s_{1l})$ is given. With  a strategy $\mathbf{f}_{k},\mathbf{g}_{k}$, the attacker and the system randomly sample an action pair $(a_{ik}, u_{jk})$ according to the probability distribution. Then, the control input and sensor value for calculating expectation cost are: 
%\begin{align*}
\centerline{$
\mathbf{u}_{k}=\sum\limits_{j=1}^{N}\sum\limits_{i=1}^{M} \sum\limits_{l=1}^{3}p(s_{kl})f_{k}^{i}(s_{kl})g^{j}_{k}(s_{kl})\mathbf{\gamma}_{k}(a_{ik},u_{jk}),
$}
$\text{ }\quad\quad\mathbf{y}_{k} =\sum\limits_{i=1}^{M}\sum\limits_{l=1}^{3} p(s_{kl})f_{k}^{i}(s_{kl}) a_{ik}.$
%\end{align*}
\\The probability that system is at state $s_{(k+1)h}$ for $k+1$ is:
\\\centerline{$
%\begin{align*}
p(s_{(k+1)h})= \sum\limits_{l=1}^{3}p(s_{kl})[\mathbf{f}_{k}(s_{kl})]^{T}{P}_{k}(s_{(k+1)h}|s_{kl})\mathbf{g}_{k}(s_{kl}). 
%\end{align*}
$}

\iffalse
\begin{align*}
\mathbf{F}_{k} :=\{&\mathbf{f}_{k}= [\mathbf{f}_{k}(s_{k1}), \mathbf{f}_{k}(s_{k2}), \mathbf{f}_{k}(s_{k3})]
|f_{k}^{i}(s_{kl})\geq 0,\\& \mathbf{f}_k \in [0,1]^{M\times 3} %&\forall a_{ik} \in A_{tk},  %k\in \{1,...,K\},\\
\sum \limits_{a_{ik} \in A_{tk}}f_{k}^{i}(s_{kl}) = 1,\mathbf{f}_{k}(s_{kl})\in \mathbb{R}^{M},\\&\forall s_{kl} \in(X_{[k-T,k]}\times S)\},\\
\mathbf{G}_{k}:=\{&\mathbf{g}_{k}= [\mathbf{g}_{k}(s_{k1}), \mathbf{g}_{k}(s_{k2}), \mathbf{g}_{k}(s_{k3})]|
g_{k}^{j}(s_{kl})\geq 0,\\& \mathbf{g}_k \in [0,1]^{N \times 3}, %&\forall u_{jk} \in A_{sk},%k \in \{1,...,K\},\\
\sum \limits_{u_{jk} \in A_{sk}}g_{k}^{j}(s_{kl}) = 1, \mathbf{g}_{k}(s_{kl}) \in \mathbb{R}^{N},\\ &\forall s_{kl} \in (X_{[k-T,k]}\times S)\}. 
\end{align*} 
\fi

\begin{algorithm}[!ht]
\begin{algorithmic}[1]
\Require Query workload $Q$, event stream $I$, \app\ graph $G$, hash table of snapshots $S$
\Ensure Hash table of results $R$ 
\State $G \leftarrow \emptyset$, $S, R \leftarrow$ empty hash tables
\ForAll {event $e \in I$ with $e.type=E$} 
    \State $//$ \textbf{\app\ graph construction}
    \ForAll {$q \in Q$ \text{ with event types }T}
        \ForAll {$E' \in T,\ E' \neq E$}
            \State $G_{E'} \leftarrow \mathit{getGraphlet}(G,E')$,
            $G_{E'}.\mathit{active} \leftarrow \mathit{false}$
        \EndFor
    \EndFor
    \If {\textbf{not} $G_E.\mathit{active}$}
        \State $G_E \leftarrow \mathit{createGraphlet()}$, $G_{E}.\mathit{active} \leftarrow \mathit{true}$,
        $G \leftarrow G \cup G_E$
        \If {$G_E.\mathit{shared}$ by $Q_E \subseteq Q$}
            $x \leftarrow \mathit{createSnapshot()}$ 
            \ForAll {$q \in Q_E$}
                \ForAll{$E' \in \mathit{pt}(E,q), E' \neq E$}
                    \State $G_{E'} \leftarrow \mathit{getGraphlet}(G,E')$
                    \State $S(x,q) \leftarrow S(x,q) + sum(G_{E'},q)$ \hspace{0.5cm}$//$ Eq.~5
                \EndFor
            \EndFor
        \EndIf    
    \EndIf
    \State insert $e$ into $G_E$
    \State $//$ \textbf{Trend count computation}
    \If {$G_E.\mathit{shared}$ by $Q_E \subseteq Q$}
        \If {$\forall q \in Q_E\ pe(e,q)$ are identical}
            \State $count(e,Q_E) \leftarrow count(e,q)$ \hspace{2.3cm}$//$ Eq.~2
        \Else\ $y \leftarrow \mathit{createSnapshot()}$, $count(e,Q_E) = y$
            \ForAll {$q \in Q_E$}
                $S(y,q) \leftarrow count(e,q)$ \hspace{0.2cm}$//$ Eq.~2
            \EndFor
          \EndIf
    \Else\ $count(e,q)$ \hspace{5.2cm}$//$ Eq.~2
    \EndIf
    \ForAll{$q \in Q$}
  	    \If {$E \in \mathit{end}(q)$} 
  		    $R(q) \leftarrow R(q) + count(e,q)$ $//$ Eq.~3
        \EndIf
    \EndFor
\EndFor
\State \Return $R$
\end{algorithmic}
\caption{\app\ shared online trend aggregation}
\label{algo:snapshot-propagation}
\end{algorithm}


\section{Simulation}

\subsection{motivation} %
The most efficient way to figure out the answers to the questions we posed in the introduction is to deploy the proposed framework on a real-world platform and analyze how users adopt different and complex privacy policies to optimize their rewards.
However, direct deployment of these strategies and investments is currently impractical due to the following reasons.


Firstly, the most important reason is that such an online experiment may lead to the decline of the recommendation performances as well as the user experience, which harms the platform's revenue.
In the real world, nearly all the companies determine their platform mechanism driven by interest, and the revenues of the platforms are highly correlated with the recommendation performances. 
Therefore, it's nearly impossible to persuade any platform to directly deploy proposed strategies and mechanism online without other benefits.  



Secondly, the experiments are \czq{built} upon several simplifications, mentioned in \cref{assumptions}, which poses challenges towards recommendation model training process.
For example, we assume when a user \czq{adjusts} his data disclosure policy, the recommendation system will forget his un-disclosed data. 
To facilitate such challenges, model unlearning or other privacy-preserving technologies are imposed.
However, in real-world applications, very few the e-commercial platforms have deployed these privacy-preserving technologies during the deep recommendation model training and evaluation processing.
As a result, we may still fail to guarantee the assumptions and simulation methodology becomes a substitution.










In summary, inspired by the success of simulation study on dynamic interactive problems in real-world applications~\cite{Ie:arxiv19:RecSim,krauth2020offline,lucherini2021t,yao2021measuring},  we employ the simulation to study the effects of the proposed framework and the possible game between users and the platform.















\subsection{Simplified Assumptions}
\label{assumptions}

To simplify the simulation process for easier analysis, we make some necessary assumptions to simplify the problem.

\begin{assumption}[Static Assumption] User $i$ optimizes her/his policy on the fixed data $\di{}$ which is not affected by user policy $\pi_i$.
\label{assumption:static}
\end{assumption}

Here static means the user data $\di{}$ is fixed during the simulation, but the disclosed data $\si{}$ produced by different user policies is dynamic. 
It is also the most common setting for recommendation task in research papers~\cite{Rendle:www10:Factorizing,Hidasi:ICLR2016:gru4rec,NCF,kang2018self,Sun:cikm19:BERT4Rec}.
In the simulation, we train the recommendation system $\texttt{M}_{{\scriptscriptstyle \mathcal{S}}}$ on the collected dynamic data $\mathcal{S}$ and validate the recommendation efficiency on a fixed test set. 
In real-world applications, the data $\di{}$, which contains the behavior data from the interaction with the recommender $\texttt{M}_{{\scriptscriptstyle \mathcal{S}}}$, is also dynamically changing with the user's policy $\pi_i$.
It is beyond the scope of this paper and we leave it as the future work. 


\begin{assumption}[Immediate Assumption] The recommendation model $\texttt{M}_{{\scriptscriptstyle \mathcal{S}}}$ can only use the data $\si{}$ currently disclosed by each user $i$.
\label{assumption:forget}
\end{assumption}
The motivation of this assumption is that an untrusted platform can leverage user $i$' all data $\di{}$ if it can use the data disclosed in previous actions.
Without this constraint, the privacy right discussed in this paper is meaningless.
To achieve this, the platform can retrain the model from scratch with new data $\si{'}$ or quick unlearn the data in $\si{}$ then finetune with data $\si{'}$~\cite{cao2015towards,bourtoule2021machine,chen2022recommendation}.


However, the \cref{assumption:forget} also raises a new challenge that the asynchronous changes of user policy bring intractable computation costs for the platform since each time the user changes the disclosed data, the platform needs to update the model.
Here, we make an assumption for simplifying the simulation, assuming all users realize that the platform will cyclically (e.g., once a day) collect their privacy decisions and update recommender systems.
\begin{assumption}
[%
Cyclical Assumption]
Platform cyclically collects user privacy choices, and then the platform updates the model using all newly disclosed data. 
\label{assumption:synchronization}
\end{assumption}



In summary, for easy analysis in simulations, we introduce these assumptions to ignore the time and dynamic effects in this feedback system, just like the traditional recommendation task formulation.
























\subsection{Platform Mechanism Simulation}
\label{sec:plat_mech}
In order to validate the effect of the platform mechanism, we adopt several mechanisms during simulation. 
For easy comparison, we utilize one mechanism at each experiment. %


\subsubsection{\textbf{Data Split Rule}}

\czq{In our simulation, we do not split the profile attribution and the user can determine whether to share all of their attributes.}
For behavior data, we apply ``percentage split'' as $\delta_b$ with different split granularity $p$ (e.g., 1/3) to split the behavior sequence into $1/p$ parts. 
One obvious advantage of ``percentage split'' is that it can normalize the size of user action space and decrease the inconvenience of the interaction between the user and the platform.

\subsubsection{\textbf{Data Disclosure Strategy}}
\label{sec:data_disclose_choice}

As the platforms have certain flexibility to implement different data disclosure strategies, we discuss three representative disclosure strategies used in our study for behavior data in this subsection.
These strategies determine the data disclosure action space $\Pi$ the user can choose.
For profile attributes, we found that all users tend not to disclose them in the experiments since these features are negligible for improving recommendation utility in the presence of behavior data.
Similar result that user profile features contribute very marginal to the recommendation results in the case of strong user behavior modeling on public benchmark datasets has also been reported in other works~\cite{kang2018self,Sun:cikm19:BERT4Rec}.
Thus, in the following study, we mainly focus on modeling only  behavior data.

The ``\textit{separate}'' rule gives the users the control to freely disclose any split personal data.
For this rule, the size of user $i$'s the action space is exponentially expended on the size of the spilt data set $|\delta_b ({\scriptstyle \mathcal{D}_{i,b}})|$, denoted as $2^{|\delta_b ({\scriptstyle \mathcal{D}_{i,b}})|}$. 
However, too many choices might make it difficult for users to make better data disclosure decisions.

Another data disclosure strategy named ``\textit{oldest continuous}'' provides users the choices to disclose continuous behavior data from the beginning time, such as selecting ``the oldest 33\% data''.
In this strategy, to disclose newer behavior data ${\scriptstyle \mathcal{S}_{i,bj}}$, users must also disclose all behavior data before it.
Take an already split data $\delta_b ({\scriptstyle \mathcal{D}_{i,b}}) = \{\scriptstyle \mathcal{S}_{i,b1}, \scriptstyle \mathcal{S}_{i,b2}, \scriptstyle \mathcal{S}_{i,b3}\}$ as an example, the action space provided by oldest continuous strategy is $\Pi = \{[0,0,0], [1,0,0], [1,1,0], [1,1,1]\}$, and its corresponding disclosed data is $\{\varnothing ,
    \{\! {\scriptstyle \mathcal{S}_{i,b1}} \!\},
    \{\! {\scriptstyle \mathcal{S}_{i,b1}} , {\scriptstyle \mathcal{S}_{i,b2}}\! \},$
    $\{ {\scriptstyle \mathcal{S}_{i,b1}}, {\!\scriptstyle \mathcal{S}_{i,b2}},$ ${\scriptstyle \mathcal{S}_{i,b3}}\} \}$.
``\textit{Latest continuous}'' mechanism is similar to ``oldest continuous'', with the only difference in the opposite direction.
The size of these two mechanisms' action spaces is $|\delta_b ({\scriptstyle \mathcal{D}_{i,b}})|$.









\subsection{User Policy Simulation}
\label{sec:user}


In this subsection, we introduce the simulation of user policy in our proposed framework.
As defined in \cref{eq:S_i}, the disclosed data $\si{}$ is result of the platform mechanism $\mathrm{G}$ and user's disclosure policy $\pi_i$. 
Meanwhile, in \cref{eq:updated_rec}, the recommendation utility $\texttt{U}_i( \si{})=\texttt{U}'(\si{},\, \texttt{M}_{{\scriptscriptstyle \mathcal{S}}})$ is also determined by the recommendation model $\texttt{M}_{{\scriptscriptstyle \mathcal{S}}}$, which is \czq{built} upon the all users' disclosed data $\mathcal{S}$. 
The reward of user $i$ may be varied even when $i$ keeps the disclosed data $\si{}$ unchanged since other users might change their disclosed data and the recommender system is changed.
Thus, the expectation rewards are considered rather than immediate value defined in \cref{eq:framework} and we assume all the users are rational and seek for the optimal privacy disclosure action  $\alpha_i^*$ to the optimal expected reward $E[ \texttt{R}_i | \alpha_i ] $ as his policy, i.e., %
\begin{equation}
    \begin{aligned}
    \alpha_i^{*} &= \argmax_{\alpha_i \in \Pi} E[ \texttt{R}_i | \alpha_i ]=  \argmax_{\si{\in [ \Pi \otimes \mathrm{\delta}(\di{}) ] }} E[ \texttt{R}_i(\si{)} ] \\
& =\argmax_{\alpha_i \in \Pi } E\Bigl[ -\lambda_i \texttt{C}_i\bigl( \alpha_i \otimes \mathrm{\delta}(\di{}) \bigr) + \texttt{U}_i\bigl( \alpha_i \otimes \mathrm{\delta}(\di{}) )\bigr) \Bigr].
    \end{aligned}
\label{eq:opt_pi}
\end{equation}

As mentioned before, recommendation utility $\texttt{U}_i$ has been discussed in \cref{sec:platform_obj}.
To study this objective, we need to define the privacy cost function $\texttt{C}_i$ and sensitive weight $\lambda_i$.



\subsubsection{\textbf{Privacy Cost Function}}
\label{sec:privacy_cost}
We simulate every user with the same cost function $\texttt{C}$ and leave the diversity of user privacy sensitivity to the parameter $\lambda_i$. 
Following current experiment specifications in the economics literature~\cite{lin2019valuing,tang2019value}, we model the privacy cost function as a linear summation\footnote{See the Eq. 2 in \cite{lin2019valuing} and the dis-utility from disclosure in the econometric specification session in \cite{tang2019value}.} of disclosed personal data.


\czq{According to the comprehensive survey on privacy value definition \cite{MKT-053}, people will measure the value of their privacy into the intrinsic value of privacy and the instrumental value of privacy.}
\czq{
The intrinsic loss indicates the sake of protecting their intrinsic private data, which measures the valuation on the intrinsic properties such as the education or the income levels. }
\czq{In this work, we denote the intrinsic loss towards the privacy cost on amount of the sharing user profile attributes.}
\czq{The instrumental value of privacy indicates how the transaction efficiency would be affected by sharing user data, especially the data generated in the applications. 
In this work we denote the privacy cost towards the percentage of shard user historical behavior data. 
Therefore, the privacy cost function is described below,}
\begin{equation}
    \texttt{C}_i(\si{})= \beta_i * | {\scriptstyle \mathcal{S}_{i,a}} | + \frac{| {\scriptstyle \mathcal{S}_{i,b}} |}{ | {\scriptstyle \mathcal{D}_{i,b}} |}
    \label{eq:cost_function0}
\end{equation}
\czq{where the first term indicates the intrinsic loss and the second term indicates the instrumental loss.} 
\czq{If user does not tend to disclose profile attribute, such privacy cost function can be simplified to the following format with the instrumental value alone.}
As mentioned in \cref{sec:plat_mech}, user tends not to disclose profile attributes $\scriptstyle \mathcal{D}_{i,a}$ due to no gains in our experiments, so we only consider behavior data here, i.e.,
\begin{equation}
    \texttt{C}_i(\si{})=\texttt{C}(\si{}) =  \frac{| {\scriptstyle \mathcal{S}_{i,b}} |}{ | {\scriptstyle \mathcal{D}_{i,b}} |}
    , %
    \label{eq:cost_function}
\end{equation}
where the $|x|$ is the number of elements in $x$.
Here, the percentage based measurement regards different amount of users' data equally. %


This reduced form specification is not unrealistic as it captures the substitution effect among personal data and incorporates the idea of constant marginal privacy cost. 
One might argue for a higher order functional to capture richer implications. 
However, there is little experimental evidence that the higher order form for privacy cost exists and how the functional form looks like.






\subsubsection{\textbf{Privacy Sensitive Weight}}
\label{sec:user_type}
For user $i$ who disclosed all her/his data (i.e., $\si{} = \di{}$), her/his privacy cost compared to not sharing any data (i.e., $\si{} = \varnothing $) is 
\begin{equation}
    \texttt{C}(\di{}) - \texttt{C}(\varnothing).
\label{eq:privacy_diff}
\end{equation}
Meanwhile, her/his anticipated recommendation utility compared to not sharing any data is:
\begin{equation}
    \texttt{U}(\di{}) - \texttt{U}(\varnothing).
\label{eq:utility_diff}
\end{equation}
We assume all users have accessed to the recommendation utility $\texttt{U}(\di{})=\texttt{U}'(\di{},\texttt{M}_{{\scriptscriptstyle \mathcal{D}}})$ computed on all the data $\di{}$ and the recommendation utility without their data $\texttt{U}(\varnothing)$ before they can take data disclosing actions, which can be regard as a prior knowledge, like the experiences before the platform adopted our framework.
With \cref{eq:privacy_diff} and \cref{eq:utility_diff}, we define the privacy sensitive weight $\lambda_i$ as: 
\begin{equation}
    \lambda_i = w_i  * \frac{\texttt{U}(\di{})  - \texttt{U}(\varnothing) } {  \texttt{C}(\di{}) - \texttt{C}(\varnothing) },
    \label{marginal_define}
\end{equation}
where $w_i$ indicates the diversity of user types towards privacy sensitivity.
The users with $w_i > 1$ is privacy sensitive users, as they will not be willing to disclose the corresponding data $\di{}$ if they only get $\texttt{U}(\di{})$ as before.
While users  with $w_i < 1$ are just the opposite. %
Therefore, the user's privacy sensitive weight is pre-computed, and the $\texttt{U}(\di{})$ can be regarded as the benchmark expectation of the platform.
The formulation of the privacy sensitive weight $\lambda_i$ also meets the idea from \cite{lin2019valuing}, where the heterogeneity from users' social demographic variety should also be explicitly characterized. %



\subsubsection{\textbf{Simulation Algorithm}}

As users behave rationally to find the optimal strategy with a trade-off of exploration and exploitation, it just meets the idea of the reinforcement learning algorithm. 
Therefore, we model each user as a unique agent and apply a multi-agent reinforcement learning method to simulate user possible policy adaptation. 
The recommender system is regarded as the environment to provide feedback, which is built upon the disclosed user data.
All agents' policies are optimized simultaneously by determining their actions, i.e., the disclosed data $\scriptstyle \mathcal{S}^t$ at simulation epoch $t$, which is used to train the recommendation model $\texttt{M}_{{\scriptscriptstyle \mathcal{S}}^{t}}$.
As mentioned before, users tend to find an optimal action over possible action space $\Pi$ to maximize his expected reward, which is determined by all agents in this dynamic MARL environment. 


We assume each user (agent) realizes this situation that the immediate reward is the result of all agents, but no communication or observation among agents is permitted. 
Then, each agent is concerned about her/his own utility and regards the environment as a dynamic system that is partially correlated to herself/himself. 
Now, it is simplified to a Multi-Armed Bandit problem~\cite{katehakis1987multi}.


However, the challenge of the exploration and explication problem also exists in our simulation. 
To address it, we adopt a simple but efficient method, Epsilon Greedy~\cite{sutton2018reinforcement} algorithm, to simulate user's policy $\pi_i$ as following, %
\begin{equation}
     \alpha_i^{t+1} = \left\{
\begin{array}{l l }
\alpha_i \sim \texttt{P}^t_i,
& \text{with possibility } \epsilon \\
\argmax_{\alpha_i} Q_i^t(\alpha_i), & \text{with possibility }  1-\epsilon \\
\end{array} \right. 
    \label{epsilon_greedy}
\end{equation}
where $Q_i^t(\alpha)$ is the user $i$'s estimation value at simulation epoch $t$ on action $\alpha$, and $\texttt{P}^t_i$ denotes a random sample policy.
To conduct an efficient policy exploration, we sample a less explored action with a higher possibility as following,
\begin{equation}
    \texttt{P}^t_i(\alpha) = \frac{ 1/ (N^{t-1}_i(\alpha) +1) } { \sum_{x \in \Pi} 1/(N^{t-1}_i(x) +1) },
    \label{random_rule}
\end{equation}
where $N^{t-1}_i(\alpha)$ represents the total number of action $\alpha$ was taken by user $i$ from start to the last simulation epoch $t{-}1$.
In convenience, we adopt the approximated expected estimation results and 
update it with the residual between the estimation $Q_i^{t-1}(\alpha_i^{t-1})$ and immediate reward  $\texttt{R}_i^{t-1}$ when she/he takes action $\alpha_i^{t-1}$ as following.
\begin{equation*}
       Q_i^t(\alpha) {=} \left\{\!\!\!
\begin{array}{l l}
Q_i^{t-1}(\alpha), &  \text{if } \alpha_i^{t-1} {\neq} \alpha \\
Q_i^{t{-}1}(\alpha) {+} \frac{1}{N^t_i(\alpha)} \bigl(\texttt{R}_i^{t-1}(\! \alpha {\otimes} \mathrm{\delta}(\di{})  ) {-} Q_i^{t{-}1}(\alpha) \bigr), &  \text{if }  \alpha_i^{t{-}1} {=} \alpha \\
\end{array} \right. 
\end{equation*}
where $\texttt{R}_i^{t-1}$ is user $i$-th immediate objective at simulation epoch $t{-}1$, computed by \cref{eq:framework}. 
$Q_i^0(\alpha) $ is the user $i$'s initial expected reward if she/he takes action $\alpha$. 
which is initialized to $0$ as users have no prior about their behaviors on the new dynamic environment.


In our simulation, we set initial $\epsilon=0.5$ for all agents and decay a half during the MARL training processing. The detailed decay epoch is co-related to the size of possible action space $\Pi$.
Here, we define it as $\epsilon = 0.5^{ t /(3 * |\Pi |) }  $, 
where $t$ is the epoch during the reinforcement learning training processing. 

\subsection{\textbf{Discussion}}
To figure out how the platform mechanism affects users' behavior, we turn to the simulation built upon several simplified assumptions. 
One fundamental assumption is the hypothesis of rational man, where users will seek their optimal policies to maximize their objectives. 
However, in the real-world scenarios, human behaviors are also affected by psychological factors, which should also be modeled in future work.
One detailed example is that some users may feel exhausted digging out all the potential privacy choices with the provided platform mechanism.
In our simulation, we assume there remains no mental cost when a user adjusts his policy. 
However, in the reality, some users may refuse to change their policy frequently, especially in complex user interaction applications.
For such situation, a convenient user interface (UI) could be a potential solution to mitigate users' fatigues. 
\czq{
Another important factor is that users may adjust their trust level towards the platform during their exploration. One detailed example is that if the platform or even the recommender system \cite{zhang2022pipattack} is easy to be attacked or the platform will abuse their disclosed data to other applications, they may re-consider their privacy sensitivity. 
Though some works have discussed the utilization of trusted platform or the privacy-preserved recommendation model, the possible effects on user psychological factors might be tackled by a dynamic modeling on the user privacy sensitive weights, which is out of the scope of this work.
}
We simplify the influences of the psychological factors in this work and leave the exploration of psychological effects in mechanism designs and UI designs for future works. 






\bibliographystyle{IEEEtran}
{  \small
\bibliography{Greplay2}
}
\end{document}\end{document} 