\section{Introduction}
\label{sec:intro}
Cyber-Physical Systems (CPS) feature a tight integration of embedded computation, networks, controlled physical processes, and provide the foundation of critical infrastructures such as transportation systems, smart grids, water service systems and so on~(\cite{PRKumar_cps}). However, the integration structures also result in vulnerability under malicious attacks (\cite{challenge}). Recoded incidents caused by attacks show that CPS attacks can disrupt critical infrastructures and lead to undesirable, catastrophic consequences~(\cite{scada_ex}). While cyber security tools have focused on prevention mechanisms, there are still challenges on how to leverage the ability of control systems to keep system resilient under a smart adversary. 

%The vulnerability of CPS to malicious attacks result from their interaction structures among continuous physical dynamics, discrete communications, and computation substrates. 
Detection methods for various types of attacks have been analyzed in the literature. \cite{Bullo_di} propose a framework for attacks and monitors of CPS perspectives. \cite{Mo_grid} analyze security challenges and countermeasures in smart grids. \cite{res_estimator} present resilient state estimators for systems with noise and modeling errors. \cite{GPS_spoof} analyzes spoofing attacks against cryptographically-secured Global Navigation Satellite System (GNSS) signals and detection strategies. \cite{Miao_coding_tcns16} design a coding scheme for sensor outputs to detect stealthy data injection attacks over the communication channel. %Four different jamming attack models and corresponding detection schemes are analyzed by~\cite{jam}. 

In general, attack models are used as parameters to design defense schemes. However, a specific detection approach alone is not sufficient, when the system does not have knowledge which attack will happen among various types of potential attacks. CPS are usually resource constrained systems, which prevents running all available modules at the same time. Besides security, other requirements like optimal cost need to be addressed during control systems design. Consequently, considering control and defense costs with the effects of multiple attacks, strategic methods that balance the system performance and security requirements are necessary. In this work, we consider the case that at each time instant, only one detector is active because of the limits of resources. Our approach can be generalized to more than one detector being active at every time instance.

The application of game theory to security problems has raised a lot of interest in recent years. Selected works that apply game-theoretic approaches in computer networks security and privacy problems are summarized by~\cite{gt_ns}. \cite{MartZhu_game} propose a receding-horizon dynamic Stackelberg game model for systems under correlated jamming attacks. \cite{Basar_2015} propose game-theoretic methods for robust and resilient control of CPSs. However, none of these works have considered switching policies under multiple types of attacks, with payoffs as functions of system dynamics and probabilistic detection rate.%A minimax game in the presence of faults is discussed in~\cite{minimax}. 

Building a framework that captures the hybrid system dynamics and interactions with attacks is pivotal for security analysis and design of CPS. To achieve this goal, our first step is to establish a zero-sum hybrid stochastic game model. The hybrid state of the game model contains a dynamic system state that captures the evolution of the physical processes, and discrete cyber modes that represent different security states of the CPS according to information provided by the detector. Then a suboptimal value iteration algorithm is developed for the finite horizon hybrid stochastic game. Compared with our previous game model~(\cite{cdc_replay}) that only switches between two controllers against replay attacks and needs strategy history to calculate a strategy, in this work the hybrid state stochastic game strategy calculation process does not depend on the strategy history. % In this work, we consider a hybrid state game for a system with multiple types of state estimators, controllers and detectors against various types of attacks, and the suboptimal algorithm for calculating game strategies.

We then propose a moving-horizon computation methodology to reduce the computational complexity of finding a saddle-point equilibrium for the hybrid stochastic game. This is a scalable and computationally efficient algorithm. At each stage, the system selects a window of finite length for the physical state, and computes the stationary saddle-point strategies for the associated finite stochastic game, with the game state reformulated as the joint cyber and physical states. A preliminary result of the moving-horizon algorithm appeared in the conference paper~\cite{game_cdc14}; in this journal version, we have included more detail about different types of attacks and each element of the game model, revised analysis of the moving horizon algorithm compared with the suboptimal algorithm, and added more simulation results. The cost comparison with the suboptimal algorithm shows that the real-time algorithm does not sacrifice system performance much. The contributions of this work are summarized as follows: 
\begin{enumerate}
\item We formulate a zero-sum, hybrid stochastic game framework for designing a switching policy for a system under various types of attacks. 
\item We design a suboptimal algorithm for the finite horizon hybrid stochastic game, and prove that the algorithm provides an upper bound for the optimal cost of the system. 
\item We develop a real-time algorithm to reduce the computation overhead of the game model.   
\end{enumerate}

This paper is organized as follows. We describe the system, attack models, and motivation of game-theoretic techniques for switching policies in Section~\ref{sec:replay1}. In Section~\ref{sec:game_form}, we formulate a zero-sum,  hybrid stochastic game between the system and the attacker. A suboptimal algorithm for the finite horizon game is developed in Section~\ref{sec:algorithm_finite}. The moving horizon algorithm and its computational complexity are analyzed in Section~\ref{sec:algorithm}. Section~\ref{sec:simulation} compares the complexity and system performance of the finite horizon and the receding horizon algorithms. Finally, Section~\ref{sec:concl} provides concluding remarks.

\iffalse
When the sequence of game strategies converges, the state transition probability of the game converges, and we leverage the stability analysis of Markov jump systems~(\cite{delay_mlj}) to check system stability.
 \cite{taxreplay} presents a taxonomy of replay attacks on cryptographic protocols.
that uses a moving window to select a sequence of physical state information, and computes a stationary saddle-point equilibrium strategy with the state being a joint cyber and physical state 
\fi