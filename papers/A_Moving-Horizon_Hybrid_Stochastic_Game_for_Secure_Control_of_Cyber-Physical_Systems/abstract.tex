\begin{abstract}
\label{abstract}
 In this paper, we establish a zero-sum, hybrid state stochastic game model for designing defense policies for cyber-physical systems against different types of attacks. With the increasingly integrated properties of cyber-physical systems (CPS) today, security is a challenge for critical infrastructures. Though resilient control and detecting techniques for a  specific model of attack have been proposed, to analyze and design detection and defense mechanisms against multiple types of attacks for CPSs requires new system frameworks. Besides security, other requirements such as optimal control cost also need to be considered. The hybrid game model we propose in this work contains physical states that are described by the system dynamics, and a cyber state that represents the detection mode of the system composed by a set of subsystems. A strategy means selecting a subsystem by combining one controller, one estimator and one detector among a finite set of candidate components at each state. Based on the game model, we propose a suboptimal value iteration algorithm for a finite horizon game, and prove that the algorithm results an upper bound for the value of the finite horizon game. A moving-horizon approach is also developed in order to provide a scalable and real-time computation of the switching strategies. Both algorithms aims at obtaining a saddle-point equilibrium policy for balancing the system's security overhead and control cost. 
%This approach leads to a real-time algorithm that yields a sequence of Nash equilibrium strategies which can be shown to converge. 
The paper illustrates these concepts using numerical examples, and we compare the results with previously system designs that only equipped with one type of controller. 


\iffalse
In the hybrid model, a system is equipped with a given finite set of controllers/estimators/detectors, and switching mechanisms can be used to select a subsystem--a combination of one controller, one estimator and one detector at each time step. 
\fi

\iffalse
The moving-horizon approach uses a moving window to select a sequence of physical state information, and computes a stationary saddle-point equilibrium strategies with the state being a joint cyber and physical state. 
\fi
%Game parameters are quantified with knowledge of the system dynamics and attack models.
%The game state describes both a sequence of physical dynamics and a piecewise constant detection mode of the system, such that the state transition is Markov and the game moves the horizon of system information at every stage.
%developed, for minimizing the expected current stage payoff when looking ahead into ensuing stages, and reducing the computation overhead of previous proposed suboptimal algorithm. The convergence property of the algorithm is shown.%, followed by the stability condition of the system. 
%Using a linear system with control-cost optimal (but nonsecure) and secure (but cost-suboptimal) controllers in presence of replay attacks as an example, 
 
\end{abstract}