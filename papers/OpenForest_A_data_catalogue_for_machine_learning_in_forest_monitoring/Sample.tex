\documentclass{CUP-JNL-DTM}%



%%%% Packages
\usepackage{graphicx}
\usepackage{multicol,multirow}
\usepackage{amsmath,amssymb,amsfonts}
\usepackage{mathrsfs}
\usepackage{amsthm}
\usepackage{rotating}
\usepackage{appendix}
% For JDM please remove the natbib package:
\usepackage[numbers]{natbib}
% And use biblatex-apa with a .bib file to format your references according to the APA7 style.
% \usepackage[natbib,style=apa]{biblatex}
% \addbibresource{your-refs.bib}
\usepackage{ifpdf}
%\usepackage[french]{babel}
\usepackage[T1]{fontenc}
\usepackage[utf8]{inputenc}

\usepackage{newtxtext}
\usepackage{newtxmath}
\usepackage{textcomp}
\usepackage[table, dvipsnames]{xcolor}
\usepackage{lipsum}
\usepackage{changepage}
%\usepackage[colorlinks,allcolors=blue,backref=page]{hyperref}
% \usepackage{xurl}
\usepackage[pagebackref=true,breaklinks=true,colorlinks,citecolor=blue,bookmarks=false]{hyperref}
\usepackage{arydshln}


\definecolor{lightgray}{gray}{0.9}

% commenting macro
% Etienne Laliberté
\newcommand{\ELc}[1]{\textcolor{orange}{[\textbf{Eti}: #1]}}
\newcommand{\EL}[1]{\textcolor{orange}{#1}}
% David Rolnick
\newcommand{\DRc}[1]{\textcolor{blue}{[\textbf{Dav}: #1]}}
\newcommand{\DR}[1]{\textcolor{blue}{#1}}
% Arthur Ouaknine
\newcommand{\AOc}[1]{\textcolor{cyan}{[\textbf{Art}: #1]}}
\newcommand{\AO}[1]{\textcolor{cyan}{#1}}
% Teja Kattenborn
\newcommand{\TKc}[1]{\textcolor{olive}{[\textbf{Tej}: #1]}}
\newcommand{\TK}[1]{\textcolor{olive}{#1}}
% Oliver Sonnentag
\newcommand{\OSc}[1]{\textcolor{lime}{[\textbf{Oli}: #1]}}
\newcommand{\OS}[1]{\textcolor{lime}{#1}}


\newtheorem{theorem}{Theorem}[section]
\newtheorem{lemma}[theorem]{Lemma}
\theoremstyle{definition}
\newtheorem{remark}[theorem]{Remark}
\newtheorem{example}[theorem]{Example}
\numberwithin{equation}{section}


\jname{Data/Math}
\articletype{ARTICLE TYPE}
%\artid{20}
\jyear{2023}
%\jvol{4}
%\jissue{1}
%\raggedbottom


\begin{document}
\UseRawInputEncoding
\begin{Frontmatter}
%\title[Article Title]{OpenForest: A data hub for machine learning in forest monitoring}
\title[Article Title]{OpenForest: A data catalogue for machine learning in forest monitoring}


% There is no need to include ORCID IDs in your .pdf; this information is captured by the submission portal when a manuscript is submitted. 
\author[1,2]{Arthur Ouaknine}
\author[3,4]{Teja Kattenborn}
\author[5]{Etienne Lalibert\'e}
\author[1,2]{David Rolnick}

\authormark{Ouaknine \textit{et al}.}

\address[1]{\orgdiv{School of Computer Science}, \orgname{McGill University}, \orgaddress{\city{Montr\'eal}, \postcode{H3A 2A7}, \state{Qu\'ebec},  \country{Canada}}}

\address[2]{\orgdiv{Mila}, \orgname{Quebec AI Institute}, \orgaddress{\city{Montr\'eal}, \postcode{H2S 3H1}, \state{Qu\'ebec},  \country{Canada}}. \email{arthur.ouaknine@mila.quebec}}

\address[3]{\orgdiv{Remote Sensing Centre for Earth System Research}, \orgname{Leipzig University}, \orgaddress{\city{Leipzig}, \postcode{04109}, \country{Germany}}}

\address[4]{\orgdiv{German Centre for Integrative Biodiversity Research (iDiv)}, \orgaddress{\city{Halle-Jena-Leipzig}, \postcode{04103}, \country{Germany}}}

\address[5]{\orgdiv{Institut de recherche en biologie v\'eg\'etale, D\'epartement de sciences biologiques}, \orgname{Universit\'e de Montr\'eal}, \orgaddress{\city{Montr\'eal}, \postcode{H1X 2B2}, \state{Qu\'ebec},  \country{Canada}}}

\keywords{datasets, forest monitoring, machine learning, remote sensing}

% \keywords[MSC Codes]{\codes[Primary]{CODE1}; \codes[Secondary]{CODE2, CODE3}}

%250 words max
\abstract{
Forests play a crucial role in Earth's system processes and provide a suite of social and economic ecosystem services, but are significantly impacted by human activities, leading to a pronounced disruption of the equilibrium within ecosystems.
Advancing forest monitoring worldwide offers advantages in mitigating human impacts and enhancing our comprehension of forest composition, alongside the effects of climate change.
While statistical modeling has traditionally found applications in forest biology, recent strides in machine learning and computer vision have reached important milestones using remote sensing data, such as tree species identification, tree crown segmentation and forest biomass assessments.
For this, the significance of open access data remains essential in enhancing such data-driven algorithms and methodologies.
Here, we provide a comprehensive and extensive overview of 86 open access forest datasets across spatial scales, encompassing inventories, ground-based, aerial-based, satellite-based recordings, and country or world maps.
These datasets are grouped in \textbf{OpenForest}, a dynamic catalogue open to contributions that strives to reference all available open access forest datasets.
Moreover, in the context of these datasets, we aim to inspire research in machine learning applied to forest biology by establishing connections between contemporary topics, perspectives and challenges inherent in both domains.
We hope to encourage collaborations among scientists, fostering the sharing and exploration of diverse datasets through the application of machine learning methods for large-scale forest monitoring.
\textbf{OpenForest} is available at the following url: \url{https://github.com/RolnickLab/OpenForest}.
}

\end{Frontmatter}

\section*{Impact Statement}

OpenForest establishes a constantly evolving catalogue of open access forest datasets. This catalogue is open for contributions and aims to provide a single central hub for such datasets within the open-source community.
In addition to introducing the OpenForest catalogue, we provide in this paper a detailed overview of complementary research topics and challenges in forest monitoring and machine learning, so as to better enable the impactful use of these datasets in interdisciplinary research.
We hope this work will ultimately contribute substantially to enhancing our comprehension of global forest composition as well the development of innovative machine learning algorithms.

% \AOc{Definition of "impact statement" from the journal; Beneath the abstract authors must provide a 120-word impact that summarises the significance of their work for a broad audience. This will be published in the article itself.}
% \AOc{Not sure if we should add the potential futures impacts on ML community: people will be open to merge datasets with diversity in modalities, tasks and scale to train large models?}
% \AOc{Maximum length: 120 words}
%Some Data journals (DAP, DCE) require an `Impact Statement' section. Comment out this section if it is not required.

% Some math journals (FLO) require a table of contents. Comment out this line if no ToC is needed.
\localtableofcontents

\section[Introduction]{Introduction}


\begin{figure}[t]%
\FIG{\includegraphics[width=1\textwidth]{figures/challenges.png}}
{\caption{\textbf{Overview of forest monitoring topics and challenges associated to machine learning perspectives and challenges.} Each forest monitoring topics and challenges are detailed with their corresponding section number (in red). They are associated to the three main machine learning perspectives and challenges categories, namely generalization, limited data and domain-specific objectives, alongside with their corresponding section number (in red)}
\label{fig:challenges}}
\end{figure}


%Forests cover X\% of the surface of the Earth. 
Forests cover one third of the Earth's land surface \cite{the_food_and_agriculture_organization_of_the_united_nations_global_2020}.
They provide a range of valuable ecosystem goods and services to humanity, including timber provision, water and climate regulation, and atmospheric carbon sequestration \cite{assessment_millennium_2001, bonan_forests_2008}. 
They also serve as habitat for a myriad of plant, animal, and microbial species. 
However, human activities have had, and continue to have, a major impact on forests worldwide. 

More than 3000 ha of forests disappear every hour from deforestation \cite{the_food_and_agriculture_organization_of_the_united_nations_global_2020, hansen_high-resolution_2013}.
% For example, X ha of forests disappear from logging every X h (... or min). 
Yet forests are also increasingly recognized as natural solutions to the joint climate and biodiversity crises \cite{griscom_natural_2017, griscom_national_2020, drever_natural_2021}.
Forest-based adaptation through avoided forest conversion, improved forest management, and forest restoration could mitigate over 2 Gt CO\textsubscript{2}-eq emissions per year by 2030 \cite{intergovernmental_panel_on_climate_change_ipcc_climate_2023},
with variations observed in different regions worldwide \cite{griscom_natural_2017, bastin_global_2019, busch_potential_2019, griscom_national_2020, drever_natural_2021}.

%\EL{Because of their economical and ecological importance, there is much interest in monitoring forests. }
Due to their significant economic and ecological importance, monitoring forests has attracted considerable attention. 
%In the field of forestry, forest resources are monitored at both individual and global levels. This monitoring includes the assessment of ecosystem functional properties, as well as the evaluation of forest health, vitality, stress, and diseases (see~\ref{sec:forest_topics}).
It includes the assessment of ecosystem functional properties, as well as the evaluation of forest health, vitality, stress, and diseases (see Sec.~\ref{sec:forest_topics}).
However, monitoring forests presents significant challenges, especially using field-based approaches (see Sec.~\ref{sec:forest_challenges}). 
Forests cover huge areas, and can be difficult to access. 
Consequently, remote sensing has and continues to play an important role in forest monitoring worldwide. 
Nowadays, a wide array of remote sensing platforms and sensors to monitor forests are available and being used. 
This includes platforms such as drones (also referred to as unoccupied aerial vehicles or UAVs), airplanes, or satellites, and sensors ranging from passive optical imagery, to active methods such as light detection and ranging (LiDAR) or synthetic aperture RADAR (SAR) \citep{white_remote_2016, verrelst_optical_2015}.



In recent years, the applications of remote sensing data for Earth-related purposes \cite{campsvalls_deep_2021, ma_deep_2019}, such as forest monitoring \cite{fassnacht_review_2016, kattenborn_review_2021, diez_deep_2021, michalowska_review_2021}, have gained momentum due to the adoption of machine learning methods and algorithms. 
It has been inspired by continuous improvements in the performance of deep learning models used in computer vision challenges in the past decade \cite{deng_imagenet_2009,everingham_pascal_2015,lin_microsoft_2014}.
Recent advances in deep learning model architectures have enabled the integration of remote sensing data from various sensors and resolutions - spatial, temporal, or spectral - which presents promising opportunities to enhance forest monitoring practices \cite{rahaman_general_2022, cong_satmae_2022, reed_scale-mae_2022, tseng_lightweight_2023}.

% to tackle climate change \cite{rolnick_tackling_2023}, and particularly

%
%In recent years, forest monitoring has reached important milestones using deep learning algorithms, such as for tree species estimation \citep{ienco_combining_2019, bolyn_mapping_2022}, tree crown segmentation \citep{cloutier_influence_2023, schiefer_mapping_2020, li_ace_2022} and land cover estimation \cite{bastani_satlas_2022, feng_m_arctic-boreal_2022, potapov_global_2022}.
%It has been inspired by continuous improvements in the performance of deep learning models used in computer vision challenges in the past decade \cite{deng_imagenet_2009,everingham_pascal_2015,lin_microsoft_2014}.
% Over the past decade, there has been a continuous improvement in the performance of deep learning models used in computer vision challenges \cite{deng_imagenet_2009,everingham_pascal_2015,lin_microsoft_2014}.
%The impressive capabilities of these algorithms have generated considerable interest across diverse research areas in computer vision, such as autonomous driving, healthcare, or surveillance. 
%The applications of remote sensing data for Earth-related purposes \cite{campsvalls_deep_2021, ma_deep_2019}, such as forest monitoring \cite{fassnacht_review_2016, kattenborn_review_2021, diez_deep_2021, michalowska_review_2021}, have gained momentum due to the adoption of machine learning methods and algorithms. Recent advances in deep learning model architectures have enabled the integration of remote sensing data from various sensors and resolutions - spatial, temporal, or spectral - which presents promising opportunities to enhance forest monitoring practices \cite{rahaman_general_2022, cong_satmae_2022, reed_scale-mae_2022, tseng_lightweight_2023}.
% Machine learning methods and algorithms have also gain momentum on the applications of remote sensing data for Earth applications \cite{campsvalls_deep_2021, ma_deep_2019}, including forest monitoring \cite{fassnacht_review_2016, kattenborn_review_2021, diez_deep_2021, michalowska_review_2021}.
% In particular, recent architectures facilitate the use of remote sensing data from different modalities and with different spatial, temporal, or spectral resolutions, providing exciting new opportunities to improve forest monitoring.
%
%In recent times, there has been a growing exploration of machine learning techniques for forest monitoring purposes. This exploration aims to enhance our understanding of forest composition, with a specific focus on tree classification, detection, and segmentation \cite{fassnacht_review_2016, kattenborn_review_2021, diez_deep_2021, michalowska_review_2021}. These tasks are accomplished using diverse modalities to gather comprehensive and detailed information.

%\EL{At the same time, new and rapid developments in machine learning enables these remote sensing data to be leveraged for forest monitoring. 
%In particular, recent architectures facilitate the use of remote sensing data from different modalities and with different spatial, temporal, or spectral resolutions, providing exciting new opportunities to improve forest monitoring.}

%As many machine learning challenges applied to forest monitoring remain unexplored, benefiting from their astonishing performances necessit to gather relevant forest datasets, which can be very time-consuming.
Numerous machine learning challenges related to forest monitoring have yet to be explored (see Fig.~\ref{fig:challenges}), and addressing them will require diverse and large forest datasets \cite{liang_importance_2020}. %, a process that can be time-consuming.
%While there is a wealth of remote sensing data that can be freely accessed over the internet, forest datasets are sparsely spread and usually restricted to access
%Machine learning researchers wishing to accelerate progress in forest monitoring must first gather relevant forest datasets, which can be very time-consuming. 
While there is a wealth of remote sensing data that is freely available, these data can be difficult to access because they involve a wide variety of sensory modalities, geographies, and tasks and are spread out across many repositories. To our knowledge, no comprehensive, central repository of open access forest datasets currently exists, a gap which we fill here with \textbf{OpenForest}\footnote{\url{https://github.com/RolnickLab/OpenForest}}. The \textbf{OpenForest} catalogue is designed to simplify the process of accessing and highlight forest monitoring datasets for researchers in the field of machine learning and forest biology, thereby accelerating progress in these domains.
%The purpose of this paper is to motivate and provide such a central repository, which we call OpenForest. It aims to accelerate progress in ML applications of forest monitoring by making it easy for ML researchers to access datasets relevant for forest monitoring.}

In this paper, we present the existing biological topics and challenges related to forest monitoring that scientists are currently investigating (Sec.~\ref{sec:forest_topics_and_challenges}) and which could be of interest of machine learning practitioners. 
%The objective of this study is to present the existing biological challenges related to forest monitoring that scientists are currently investigating (Sec. \ref{sec:bio_challenges}). 
Additionally, we briefly introduce several machine learning research topics, exploring their potential applications in addressing biology-related challenges (Sec. \ref{sec:ml_challenges}). These applications hold promise in assisting ecologists and biologists in their work.
%Additionally, This study briefly introduces and discusses several machine learning research topics, exploring their potential applications in addressing the related biology challenges (Sec. \ref{sec:ml_challenges}). 
Moreover, we conduct a thorough review of open access forest datasets across different spatial scales (Fig.~\ref{fig:scale}) to support both machine learning and biology research communities in their work (Sec.~\ref{sec:review}). 
%Finally, we conclude with a discussion of the perspectives on machine learning applications and forest datasets (Sec.~\ref{sec:perspectives}).
Finally, we provide perspectives on the space of machine learning applications with forest datasets (Sec.~\ref{sec:perspectives}).
%This work aims to present the current biology challenges applied to forest monitoring that scientists are exploring (Sec. \ref{sec:bio_challenges}).
%Several machine learning research topics are brievly introduced and discussed on how they could be applied to the related biology challenges (Sec. \ref{sec:ml_challenges}). open access forest datasets are then reviewed (Sec. \ref{sec:review}, at different scales such as illustrated in Figure \ref{fig:scale}. The perspectives in machine learning appications and forest dataset are finally discussed (Sec. \ref{sec:perspectives}).}




%\subsection{Motivations (forest)}
%\label{sec:intro_forest}

%\begin{itemize}
    %\item The Food and Agriculture Organization of the United Nations have estimated in its 2020 report that one third of the Earth lands are covered by forests.
    % \item Global warming and climate change (IPCC reports): forest conservation and expansion.
    % \item Biodiversity conservation.
    %\item Better understanding of the forests of our planet.
    %\item change dynamics in forests as consequence of climate change \cite{fassnacht_remote_2023} + other refs
    %\item High potential in forest protection, management and restoration as a potential natural leverage for emission mitigation \cite{griscom_natural_2017, bastin_global_2019, busch_potential_2019, griscom_national_2020, drever_natural_2021}
    %\item Forests have also other roles as been habitats for biodiversity (ref?), drain water (ref?) and avoid flooding with mangroves (ref? Global mangrove watch?)
    %\item From Etienne: mention the different monitoring tasks and forest properties people monitor, and include a spatial scale aspect (from individual trees to entire landscapes).
%\end{itemize}

%\subsection{Motivations (ML applications)}
%\begin{itemize}
    %\item Wide range of sensors and best practices to record data for forest monitoring: inventories, ground imagery, aerial, satellite.
    %\item Foundation models: multi-modal and multi-task, (to the best of my knowledge) multi-scale has not been explored yet.
    %item ML challenges similar to other research fields
%\end{itemize}
 
%\subsection{Motivations (datasets)}
%\begin{itemize}
    %\item Wide range of open source forest datasets.
    %\item Opacity on really open source datasets.
    %\item Datasets motivate the ML community to apply research in this field. Also successfully ML application in forest context often requires a lot of data; especially for forests and their porperties, where contrasts in predictors are often low (since they are mostly all green).
    %\item Create a dynamic database updated by researchers.
%\end{itemize}




\section{Forest monitoring: current topics and challenges}
\label{sec:forest_topics_and_challenges}

\begin{figure}[t]%
\includegraphics[width=1.0\textwidth]{figures/scale.png}
\caption{
\textbf{Illustration of forest monitoring datasets at different scales.} Inventories are in situ measurements realised at the tree level. Ground-based datasets are recorded within or below the canopy of the trees. Aerial datasets are composed of recordings from sensors mounted on unoccupied (drones) or occupied aircrafts. Satellite datasets are collected from sensors mounted on satellites orbiting the Earth. Map datasets are generated at the country or world level using datasets at the aerial or satellite scales}
\label{fig:scale}
\end{figure}
% Note copyright form: https://www.cambridge.org/core/services/authors/journals/seeking-permission-to-use-copyrighted-material

%Forest monitoring is an active field of research in both biology and machine learning communities. They both have in common the need of open source datasets for statistical analysis and modeling. In particular, deep learning algorithms are known to be performing in many different tasks and applications while requiring large scale datasets.
%
Forest monitoring is an empirical science that is increasingly based on data-driven machine learning methods and, as such, benefits by improved data access through open data \cite{wulder_opening_2012, de_lima_making_2022}.
% A shared requirement between these fields is the availability of open source datasets for statistical analysis and modeling. 
In particular, deep learning algorithms are widely recognized for their strong performance in diverse tasks, but their successful application often relies on large datasets to unleash their performance and enhance their generalization potential.
%
%This sections aims to motivate the need of open source forest datasets. First to better explore current biology challenges related to forest monitoring (Sec. \ref{sec:bio_challenges}). Second to adapt machine learning strategies to the aforementioned biology challenges (Sec. \ref{sec:ml_challenges}).
%
%This section seeks to emphasize the importance of open source forest datasets for two primary purposes. Firstly, to facilitate a more comprehensive exploration of current topics in the context of forest monitoring (Sec.~\ref{sec:forest_topics}) and related challenges (Sec. \ref{sec:forest_challenges}), in particular for machine learning practitioners. 
%Secondly, to enable the adaptation of machine learning strategies to address the aforementioned biology challenges (Sec. \ref{sec:ml_challenges}) that would be helpful for biologists and ecologists.
This section seeks to emphasize the importance of open access forest datasets for two primary purposes. First, to facilitate a more comprehensive exploration of current topics in the context of forest monitoring (Sec.~\ref{sec:forest_topics}). Second, to better assess forest monitoring related challenges (Sec. \ref{sec:forest_challenges}), in particular for machine learning practitioners. 
% Secondly, to enable the adaptation of machine learning strategies to address the aforementioned biology challenges (Sec. \ref{sec:ml_challenges}) that would be helpful for biologists and ecologists.

\subsection{Forest monitoring topics}
\label{sec:forest_topics}

% Given the economic and ecological importance of forests, forest monitoring encapsulates various properties of forests, which are briefly synthesized in the following:
Considering the significance of forests both economically and ecologically, forest monitoring encompasses a range of trackable forest attributes. 
Each of them can be sensed with different sensors, platforms and across different scales.  
The forest attribute itself (such as a regression of a biochemical property or the detection of tree individuals) together with the structure of the remotely sensed signals, % define suitable ML algorithms. , 
collectively determine the appropriate machine learning algorithms to be employed.
These aspects are succinctly discussed in the following sections. 

\subsubsection{Forest extent and forest type mapping} 
\label{sec:topic_forest_map}
Tracking the extent of forests is crucial to understand the spatial distribution of forest resources, ecosystem services and assess the role of forest in land surface dynamics \cite{keenan_dynamics_2015}. Thereby, forest can be classified to different management, functional or ecosystems types (\textit{e.g.} coniferous, deciduous forests) \cite{zhang_glc_fcs30_2021, buchhorn_copernicus_2020}. In this regard, Earth observation data from long-term satellite missions (\textit{e.g.} Landsat or Sentinel described in Sec. \ref{sec:review_satellite}) enable to track forest extent dynamics across past decades \citep{hansen_high-resolution_2013}, enabling to assess conservation efforts and anthropogenic land cover change, such as deforestation for agricultural expansion \cite{curtis_classifying_2018}. 

\subsubsection{Tree species mapping} 
\label{sec:topic_species}
A fine-scaled representation of forest stands in terms of their species composition is relevant for forestry (\textit{e.g.} species-specific timber supply), biogeographical assessments (\textit{e.g.} climate change-induced shifts in species distributions) or biodiversity monitoring \citep{fassnacht_review_2016, wang_remote_2019, cavender-bares_remote_2020}. Recent developments in machine learning greatly advanced the identification of tree species in high resolution data (\textit{e.g.} imagery or LiDAR point clouds from drones and airplanes) using semantic and instance segmentation methods \citep{cloutier_influence_2023, schiefer_mapping_2020, li_ace_2022}. At large spatial scales, Earth observation satellite data, providing coarser spatial but high temporal and spectral resolutions, enable accurate assessments of tree species distributions using spatio-temporal machine learning methods \citep{ienco_combining_2019, bolyn_mapping_2022}.

    
\subsubsection{Biomass quantification} 
\label{sec:topic_biomass}
Forests provide cardinal ecosystem services through their provision of timber and their role as sinks in the terrestrial carbon cycle \citep{regnier_land--ocean_2022}. Tree biomass is primarily a product of the wood volume and density, while both properties are challenging to obtain from remote sensing data. Accurate biomass estimates of individuals trees can be obtained from close-range 3D representations acquired from terrestrial or drone-based LiDAR systems \citep{brede_non-destructive_2022}. More indirectly related information on crown and canopy structure derived from airborne or spaceborne LiDAR and SAR data can be used to estimate biomass at the stand scale \citep{le_toan_biomass_2011, lu_survey_2016}. Some studies have indicated the value of passive optical data from satellites, since forest biomass is partially correlated with foliage density \citep{besnard_mapping_2021, potapov_mapping_2021}. Given that precise large-scale biomass distributions can not be directly revealed through a single remote sensing modality alone, deep learning may play a crucial role to simultaneously exploit the suite and high dimensionality of available data modalities \citep{yang_new_2020}.


\subsubsection{Forest health, disturbance and mortality} 
\label{sec:topic_health}
In many regions, forest ecosystems are under pressure as globalization facilitates the introduction of exotic pests and pathogens, climate change exceeds the resilience and resistance of trees \citep{hartmann_climate_2022}, while nutrient and water cycles are affected by anthropogenic activities \citep{steffen_global_2005, trumbore_forest_2015}. A decline in tree health, \textit{e.g.} due to pathogen infestations or shortages of water and nutrients, can lead to a variety of symptoms, such as changing concentrations of multiple biochemical tissue properties (\textit{e.g.} pigments, carbohydrates, water content), which in turn can be sensed through multispectral or hyperspectral reflectance \citep{zarco-tejada_chlorophyll_2019, zarco-tejada_previsual_2018}. 
In this context, deep learning algorithms are very promising, due to their capability to exploit high dimensional data (\textit{e.g.} hyperspectral) and to translate it into a suite of foliage properties relevant to vegetation health \cite{cherif_spectra_2023}. %\citep[see][for an example on multi-output models]{cherif_spectra_2023}.
%In this context, deep learning is very promising, due to its capability to fully exploit a dimensional (hyperspectral) data and to translate it into a suite of foliage properties relevant to vegetation health \citep[see][for an example on multi-output models]{cherif_spectra_2023}.
%A related and topic is the global phenomenon of accelerated forest mortality rates 
An interconnected topic are globally increased rates of forest mortality \citep{hartmann_climate_2022, allen_global_2010}. In this context, a wealth of approaches was successfully employed at local scales, such as the detection of dead trees via semantic or instance segmentation techniques \citep{cloutier_influence_2023, sani-mohammed_instance_2022}, or at large-scales, such as the regression of annual cover of dead tree crowns in satellite image pixels with deep learning-based time series analysis \citep{schiefer_uav-based_2023}.


\subsubsection{Biophysical traits and functional ecosystem properties} 
\label{sec:topic_properties}
With accelerated biodiversity decline and environmental change, understanding  functional properties, their diversity across stands and landscapes, as well as their phenology (temporal dynamics), are essential to assess the resilience and resistance of ecosystems \citep{sakschewski_resilience_2016, thompson_forest_2009}. Given that trees through evolution developed different strategies to interact with light, their appearance studied with optical remote sensing signals can inform on a variety of functional traits, such as the foliage density, date of green up, or contents of different pigments, and carbohydrates \citep{schneider_mapping_2017, cherif_spectra_2023}. Such functional traits determine the configuration of an ecosystem and thereby modulate functional ecosystem processes \citep{gomarasca_leaf-level_2023, migliavacca_three_2021}, \textit{i.e.} fluxes of energy and matter between the terrestrial biosphere, pedosphere, hydrosphere, and atmosphere, including carbon, evapotranspiration, latent and sensible heat. Due to the cardinal importance of these fluxes in the Earth system, considerable efforts have been made to monitor these using a ground-based sensor network of flux towers (\textit{e.g.} FLUXNET \cite{baldocchi_fluxnet_2001}). Given the complexity of these ecosystem processes, deep learning is assumed to greatly enlarge our capabilities to exploit local flux towers and globally available remote sensing data to spatially and temporally extrapolate and understand forest ecosystem process \citep{jung_fluxcom_2019, elghawi_hybrid_2023, reichstein_deep_2019, campsvalls_deep_2021}.


%\ELc{perhaps also phenological measurements, such as date of key events (greenup, end of season etc, LAI? important as input in climate or earth system models?}

%\TKc{i tried to integrate phenology above. Hope thats fine? I tried not to be too exhaustive, since I feel that these sections are already somehwat long.}

\subsection{Forest monitoring challenges}
\label{sec:forest_challenges}

Forests are complex ecosystems dominated by trees. As living organisms, trees are affected by various abiotic and biotic factors, which influence their remotely sensed signal via their foliage properties and crown architecture \cite{kulawardhana_remote_2011}. Machine learning researchers wishing to develop algorithms to monitor forests using remote sensing data must be aware of these sources of biological variation and their origin. Because some of this biological variation is largely unpredictable but potentially clustered in space and time (\textit{e.g.} insect outbreaks affecting tree health, random genetic variation within tree species populations), it can be seen as a challenge as it might lead to systematic errors for prediction tasks. On the other hand, part of this variation is deterministic (\textit{e.g.} changes in leaf colour and other phenological changes driven by seasonal fluctuations that occur every year) and could be leveraged to improve model performances \cite{cloutier_influence_2023}. Another major, pervasive challenge in forest monitoring are the difficulties associated with the acquisition of ground data to train or validate machine learning models using remote sensing data. Below we summarise these primary challenges.


\subsubsection{Tree species} 
\label{sec:challenge_species}

There are an estimated 73,000 tree species on Earth \cite{cazzolla_gatti_number_2022}, the majority of which are found in the tropics. While tree species show many similarities (\textit{e.g.}, the presence of woody stems and branches), every tree species differs from one another in their chemical and structural make-up and how they will reflect solar radiation \cite{asner_functional_2014}. For example, tree foliage of different species comes in various shades of green that reflect the concentrations of pigments (\textit{e.g.} chlorophylls and carotenoids) \cite{gates_spectral_1965}. Likewise, tree species differ from each other in their leaf form crown structure \cite{verbeeck_time_2019}, which will affect the remotely sensed signal. Such foliar biophysical and crown structural variation among tree species is the result of millions of years of evolution and of adaptations to various environmental conditions \cite{meireles_leaf_2020}.

From a machine learning perspective, the biggest challenge associated with tree species diversity is that models trained on data from a given set of tree species might transfer poorly to other regions that host different species. However, the phylogeny and evolutionary distances of tree species are fairly well known \cite{zanne_three_2014}, and tree species that are closer phylogenetically tend to be more similar in their traits \cite{ackerly_conservatism_2009}. As such, phylogenetic correlations and distances among tree species can potentially be leveraged to improve model transferability.
% An additionnal related challenge discussed in the next sections, is that tree species are not static in terms of their leaf biophysical and crown structural properties. 
Another interconnected challenge, elaborated upon in the following sections, pertains to the dynamic nature of tree species in relation to their leaf biophysical and crown structural characteristics.
Instead, each individual differs according to their abiotic (\textit{e.g.} microclimate, soil) and biotic environment (competition, herbivory) and as such the expression of foliage and crown properties can overlap between species \citep{fassnacht_review_2016}.


\subsubsection{Seasons and phenology}
\label{sec:challenge_phenology}

Trees are sessile organisms but they still respond dynamically to fluctuating seasons. In some cases, phenological properties, such as leaf onset, flowers or seeds, might be  of direct interest to monitor ecological phenomena or species \cite{wagner_flowering_2021}, in which case high-frequency multi-temporal imagery might be required. Indeed, phenological changes among species, for example changes in leaf color during autumn senescence, can help to distinguish tree species based on colour, which can be used to improve species classification models \cite{cloutier_influence_2023}. However, phenological properties may also hinder the transferability of models across time \citep{kattenborn_spatially_2022}. 
For instance, the information learnt by a machine learning model using data acquired in summer may not transfer to the same location in fall as trees may have changed in their leaf biochemical properties or the fraction of flowers and seeds in the canopy  \cite{schiefer_retrieval_2021}. 
%For instance, a model trained on a dataset acquired in summer may not be transferable to the same location in autumn, as trees may have changed in their leaf biochemical properties or fraction of flowers and seeds in the canopy  \cite{schiefer_retrieval_2021}. 
In such cases, the temporal representativeness of data on individual tree species can be key \cite{kattenborn_spatially_2022}.

\subsubsection{Abiotic factors}
\label{sec:challenge_abiotic}
The structure and composition of forests is strongly influenced by \emph{abiotic factors} such as climate, geology and soils, as well as water availability. For example, declining temperatures and/or growing season lengths with increasing latitude and/or elevation can filter out tree species that cannot tolerate low temperatures (\textit{e.g.} low frost resistance) or that do not have enough time to produce mature tissue once the growing season becomes too short \cite{korner_where_2016}. In addition, changes in soil nutrient availability driven by geomorphological processes can influence forest canopy biochemistry \cite{chadwick_landscape_2018}. Water supply is also important: too much water favours trees that can tolerate waterlogging, whereas too little water favours trees that can resist or recover from xylem cavitation \cite{choat_global_2012}. Much of these environmental influences on forest composition express themselves via tree species turnover; that is, changes in tree species composition across these spatial environmental gradients or discontinuities. However, changes in forest composition and structure can also arise through intraspecific variation. 
Applications of machine learning methods to forest monitoring should integrate these sources of variation.
%For the ML researcher wishing to apply ML models for forest monitoring, it is important to be aware of these sources of variation. 
In particular, incorporating environmental drivers of forest composition and structure as model inputs may help to transfer forest monitoring models from one region to the other.


\subsubsection{Biotic factors} 
\label{sec:challenge_biotic}
Tree monitoring can also be affected by \emph{biotic factors} -- that is, by other organisms. First, pests and pathogens can impact tree health, foliage chemistry and/or water content, which in turn can affect the remotely sensed signal of forest canopies \cite{sapes_canopy_2022}. The health status of trees is often directly expressed via their foliage properties and crown architecture and therefore can cause a large variability in remote sensing signals \citep{zarco-tejada_previsual_2018, kattenborn2022anglecam}. In addition, the remotely sensed signals of trees can also be `overshadowed' by other organisms that live in their crowns (epiphytes), particularly in tropical environments \cite{baldeck_operational_2015}. Prominent examples are lianas or mistletoes.


\subsubsection{Data collection}
\label{sec:challenge_data}
% As discussed above, forests can be highly heterogeneous across space and time in terms of their composition and structure for various reasons. 
As previously mentioned, forests can exhibit significant diversity in terms of their composition and structure across different locations and time periods due to a variety of factors.
% This heterogeneity makes it especially challenging to develop forest monitoring machine learning models from one region that will transfer well to other regions that may fall outside of the training data distribution. 
This heterogeneity poses a particular difficulty in creating machine learning models for forest monitoring. Models developed for one region may not easily generalize to other regions that lie beyond the scope of the training data distribution.
% One obvious solution to this problem would be to train such models with large amounts of data that are representative of the entire range of conditions found across all forests. 
One solution for this issue would involve training these models using extensive datasets that encompass the complete spectrum of conditions present in diverse forest environments.
%Remote sensing data of forests across the globe, particularly from spaceborne sensors are plentiful and for the most part widely accessible. By contrast, there is paucity of ground data (labels or annotations).
Forest remote sensing data worldwide, especially acquired from sensors on satellites, are abundant and generally easily obtainable. In sharp contrast, there is a scarcity of ground-based data, including labels or annotations.

In contrast to other disciplines, annotating remote sensing data in the context of vegetation is often time consuming, costly, and complex. 
% The reason is that vegetation of different species or states often look very similar (`green stuff').
This phenomenon arises due to the fact that vegetation of various species or conditions frequently exhibit striking similarities, often referred to as `greenery'.
Moreover, vegetation communities often show smooth transitions across species or growth forms along environmental gradients. %, which further complicates the separation of plant individuals, species or growth forms \cite{kattenborn_review_2021}. 
This aspect adds another layer of complexity to the task of distinguishing between individual plants, species, or growth forms \cite{kattenborn_review_2021}.
% Often, field inventories are needed to validate annotations, such as tree species identifications \cite{cloutier_influence_2023, kattenborn_convolutional_2020}, or because the properties of interest (e.g. stem diameters) are not directly retrievable from the remote sensing data and must be measured directly in the field by human observers. 
Often, field inventories become essential to validate annotations, such as the identification of tree species \cite{cloutier_influence_2023, kattenborn_convolutional_2020}, or when the properties of interest, such as stem diameters, cannot be directly extracted from remote sensing data and require on-site measurements conducted by human observers.
Gathering such field data is typically a time-intensive, expensive, and gradual process, leading to significant constraints on its accessibility.
% Such field data acquisitions are commonly extremely time-consuming, costly and slow, which greatly limits its availability. 
% In addition, spatial coordinates are often the only way to link the field data to the remote sensing data, but GNSS geolocation in forest environments often come with large uncertainties (meters to tens of meters), making it difficult to precisely link field observations to remote sensing data a posteriori \cite{kattenborn_review_2021}.
Moreover, spatial coordinates frequently serve as the sole means of connecting field data to remote sensing data. However, GPS or GNSS geolocation in forest settings often introduces substantial uncertainties (ranging from meters to tens of meters), thereby posing challenges in accurately establishing a posteriori links between field observations and remote sensing data \cite{kattenborn_review_2021}.

%Combining multiple datasets is, hence, a strategy to overcome shortage of annotated data and reduce  annotation costs and the reliance on ground truthing in the field. 
Therefore, integrating various datasets is a strategy aimed at addressing the scarcity of annotated data, cutting down annotation expenses, and lessening the dependency on field-based ground truthing.
% However, this may result heterogenous datasets: For many applications annotations differ (e.g. boxes, polygons, points, quality). 
Nonetheless, this may result heterogenous datasets: In numerous cases, annotations vary (such as boxes, polygons, points), as well as their quality, across different applications.
% Annotations are also often tailored to specific remote sensing data (e.g. spatial resolution). 
Annotations are frequently customized to match particular remote sensing data characteristics, such as spatial resolution.
% Thus, combining labels across datasets often not directly possible or at least workarounds are required, such as weakly supervised learning, where a low quality of labels is compensated by large quantities.
Hence, directly merging labels from different datasets is often not feasible, or at the very least, alternative approaches are necessary. One such approach is weakly supervised learning, where the potential lack of label quality is counteracted by leveraging a substantial quantity of data (see Sec.~\ref{sec:ml_weakly}).


% The take-home message from this section is that there will always be far more unlabeled remote sensing data than labeled ground-truth data when it comes to developing ML models for forest monitoring, because labeled data are difficult to acquire. 
The key takeaway from this section is that the development of machine learning models for forest monitoring will consistently involve a substantial surplus of unlabeled remote sensing data in comparison to labeled ground-truth data. This disparity arises due to the inherent challenges in obtaining labeled data.
% This situation is not unique to forest monitoring but is some common to other geospatial applications using remote sensing data \cite{rahaman_general_2022, mai_opportunities_2023}. 
This scenario is not exclusive to forest monitoring; rather, it is a prevalent aspect shared with other geospatial applications using remote sensing data \cite{rahaman_general_2022, mai_opportunities_2023} (see Sec.~\ref{sec:ml_foundation_models}).
This has two main implications for machine learning research aimed at improving forest monitoring. 
% First, there is a need to develop ML approaches to forest monitoring that make the most with little labelled data, for example making greater use of self-supervised learning (SSL) to derive useful representations. 
Firstly, there is a need to develop machine learning methods to forest monitoring that that can effectively utilize limited labeled data. One approach involves leveraging self-supervised learning techniques to extract valuable representations from the available data (see Sec.~\ref{sec:ml_ssl}).
% Second, there is also a need for new ML approaches, such as active learning or other forms of model-assisted labelling to accelerate the acquisition of labels by human observers and to lower its cost.
Secondly, there exists a necessity for novel machine learning strategies, including active learning or alternative forms of model-assisted labeling. These approaches aim to expedite the process of label collection by human observers and reduce associated costs (see Sec.~\ref{sec:ml_active}).


%\Challenges to be discussed and included in this section:
%\begin{itemize}
    %\item Forest composition (fine grained understanding at a large scale, need high resolution sensors)
    %\item Estimation of the \textbf{above ground biomass and carbon stocks} (including height, species, crown size, wood volume, basal area...)
    %\item Study of the phenology: Plants are not static, but with phenology change through time. Phenology might be a property of direct interest \cite{wagner_flowering_2021}. However, phenology may also not be of interest but hinder the transferability of models across time. For instance, a model trained on a dataset acquired in summer may not be transferable to the same sites in autumn (changes in leaf properties, flowers, inflorescence) \cite{schiefer_retrieval_2021}. In such case, the temporal representativeness of classes in dataset for the corresponding domain is key \cite{kattenborn_spatially_2022}
    %\item \textbf{Annotation quantity and quality}:
    
    % \item Data quality and quantity? Inventories, high resolution recodings, 2D / 3D recordings, multi-spectral / hyper-spectral ...
    %\item Training transferable models requires \textbf{temporal and spatial representativeness of training data}: Often datasets (particularly the labels) cover a local or regional extent,  are only available for a few years or specific vegetation seasons or a restricted to one sensor and its specific characteristics (e.g. optics). Extrapolation to new domains (time and space and sensors) is, hence, often limited \cite{meyer_predicting_2021, f_dormann_methods_2007}. Representativeness of datasets in the target domain is key (both temporally and spatially). %\AOc{This should be in domain adaptation (ML section)}
    %\item Evaluating transferability of models in time and space requires representative validation data: With remote sensing, we usually aim to extrapolate (look into the unknown). However, evaluating models for their transferability to new domains (years, regions) based on a spatial or temporal sample (without having data on the manifold) is challenging, since the new domain may be not from the same distribution as the training data. The transferability of a model can thus only be estimated. This model performance estimation is ideally done with independent test data, while full independence in a geospatial analysis is not possible \cite{f_dormann_methods_2007, roberts_cross-validation_2017}, see also  (Toblers law: everything is related to everything else, but near things are more related than distant things, \cite{tobler_computer_1970}). Thus, careful spatial validation procedures (cf. Meyer papers) and ample data required \cite{meyer_predicting_2021, kattenborn_spatially_2022}. \AOc{Defining "dataset splits" in biology is not straightforward, spatial splits vs random splits, ML links: domain adptation, OOD.}
    %\item Take into account the context (soil, topology, location...)
    %\item Forest management (details? What does this even means?): %\url{https://en.wikipedia.org/wiki/Forest_management}
    %\item How different scale of information are used from inventories to satellite data
    %\item Forest accessibility for biologist (aerial inventories could be the next step)
    %\item Disturbance, sickness, insect infestation ..? Maybe not in the scope
%\end{itemize}




\section{Machine learning perspectives and challenges}
\label{sec:ml_challenges}


Machine learning algorithms in computer vision have gained significant capabilities over the past decade, \textit{e.g.} in image classification \cite{krizhevsky_imagenet_2012, simonyan_very_2015, szegedy_going_2015, szegedy_rethinking_2016, he_deep_2016, szegedy_inception-v4_2017, hu_squeeze-and-excitation_2018, dosovitskiy_image_2021,liu_swin_2021, touvron_training_2021}, object detection \cite{ren_faster_2015, redmon_you_2016, redmon_yolov3_2018, liu_ssd_2016, he_mask_2017, li_mask_2023} and segmentation \cite{long_fully_2015, ronneberger_u-net_2015, lin_focal_2017, he_mask_2017, chen_encoder-decoder_2018, cheng_masked-attention_2022, kirillov_segment_2023}. 
%
%The performances of these algorithms have raised interests in various computer vision research fields as autonomous driving, healthcare or surveillance.
% These applications also use sensors with different physical properties requiring to adapt mainstream architecture.
% The impressive capabilities of these algorithms have generated considerable interest across diverse research areas in computer vision, such as autonomous driving, healthcare, and surveillance. 
% Moreover, these applications involve the utilization of sensors recording data with distinct physical properties, thereby necessitating the adaptation of mainstream model architectures.
While many successful algorithmic paradigms have been established, different applications differ widely across sensory modalities and domain-specific constraints, necessitating the adaptation of algorithms to fit specific needs.
For instance, detecting, localizing and segmenting objects in a scene have been explored for LiDAR point cloud \cite{yang_pixor_2018}, automotive RADAR  \cite{ouaknine_multi-view_2021} and medical echocardiography \cite{leclerc_deep_2019}.

% Deep learning algorithms have been extensively explored for earth monitoring 
Machine learning algorithms for remote sensing have been the subject of extensive innovation and application \cite{campsvalls_deep_2021, ma_deep_2019} for problems involving classification \cite{cheng_remote_2020, maxwell_implementation_2018}, object detection \cite{cheng_survey_2016, li_object_2020} and segmentation \cite{yuan_review_2021, hoeser_object_2020}.
%
% Machine learning applied to forest monitoring has been recently explored to better understand the composition of the forest, in particular for tree classification, detection and segmentation using various modalities 
In recent times, there has been a growing exploration of such techniques for forest monitoring purposes, aiming to enhance our understanding of forest composition, with a specific focus on tree species mapping (see Sec.~\ref{sec:challenge_species} and \ref{sec:topic_species}), \textit{i.e.} tree classification, tree detection, and tree segmentation \cite{fassnacht_review_2016, kattenborn_review_2021, diez_deep_2021, michalowska_review_2021}. These tasks are accomplished using modalities from diverse sensors to gather complementary information.

%
%However, it has not been studied as much as autonomous driving or medical imagery and many machine learning challenges remain unexplored to better monitor forests.
%
%Is adapting the challenges of machine learning for forest monitoring useful for exploring the challenges of biology?
%
%This section will details current machine learning challenges and the different strategies to delve into them.
%This section will provide an overview of the present challenges in machine learning and the various strategies employed to address them w.r.t. forest monitoring.
Nevertheless, the study of machine learning for forest monitoring has not received as much attention as autonomous driving or medical imagery, despite the importance of forest conservation, restoration, and management as natural solutions to the joint climate and biodiversity crises \cite{intergovernmental_panel_on_climate_change_ipcc_climate_2023}. Consequently, there are numerous unexplored machine learning challenges that need to be addressed in order to tackle climate change \cite{rolnick_tackling_2023}, including improving forest monitoring practices.
%
Can the challenges encountered in adapting machine learning for forest monitoring be beneficial in exploring the challenges in the field of biology and ecology?
%
This section will outline the current challenges in machine learning linked to forest monitoring, as described in Figure \ref{fig:challenges}, and discuss the diverse strategies employed to tackle them.


\subsection{Generalization} 
\label{sec:ml_generalization}

\emph{Generalization} in machine learning refers to the ability of an algorithm to continue to perform well when evaluated on data different from that it was trained on \cite{zhang_understanding_2017}. One may speak of both \emph{in-distribution} generalization (performance on data relatively similar to training data) and \emph{out-of-distribution} generalization (performance on highly different data). Out-of-distribution generalization can be especially relevant to forest-monitoring, since as mentioned in Section \ref{sec:review}, forest datasets have a wide range of variation in terms of geographical locations, species composition, sensors and scale (see Figure \ref{fig:scale}). Such variations introduce distinct data distribution shifts that need to be considered and addressed in forest monitoring tasks. For example, the effects of geographic variability of data have been examined in the context of tree species distributions \cite{f_dormann_methods_2007} and biomass estimation \citep{ploton2020spatial}. Simple algorithmic approaches to improve generalization include various forms of regularization \cite{zou_regularization_2005}, data augmentation \cite{shorten_survey_2019}, dropout \cite{srivastava_dropout_2014} and batch normalization \cite{ioffe_batch_2015}, while improving the breadth of training data, where possible, is also almost always beneficial in practice. However, generalization remains a very active field of research in machine learning. We consider two areas of work that may be of especial interest in forest-monitoring.
%
% Transfer learning has been extensively explored within the field of machine learning to fine-tune a pre-trained model on a specific task \cite{weiss_survey_2016}. 

\subsubsection{Domain adaptation}
\label{sec:ml_domain}

\emph{Transfer learning} refers to transferring information learnt by a model on one set of problems to different set of problems \cite{weiss_survey_2016}. For example, one may pre-train a model on a large, commonly used dataset and then fine-tune it on a smaller dataset representing the specific problem in question.
Transfer learning can boost generalization when there is a significant distribution shift between training and inference \cite{csurka_domain_2017}. 
%Transfer learning has been extensively explored in machine learning to fine-tune models improve the generalization through data distribution shift over different tasks or datasets \cite{csurka_domain_2017}.
%
One particularly notable approach to transfer learning is \emph{domain adaptation}, where a model must be applied to target domains that are unknown or lacking labeled data \citep{soltani_transfer_2022}. Some domain adaptation approaches have already been applied in plant identification \cite{ganin_unsupervised_2015}.
%In particular, the domain adaption scheme aims to learn a model from one or several source domains and using it on one or several, unseen or unlabelled, target domains \cite{ganin_unsupervised_2015}.
%
%Unsupervised domain adaptation (UDA), considering unlabelled or unseen source or target domain(s), has been particularly explored for autonomous driving 
%
%Autonomous driving has witnessed significant exploration in the realm of unsupervised domain adaptation (UDA), specifically in the context of unlabelled or unseen source or target domains \cite{wilson_survey_2020} using generative \cite{hoffman_cycada_2018} or adversarial methods \cite{vu_advent_2019}. The UDA framework as also been explored for cross-modal learning considering domains as from different sensor modalities \cite{jaritz_xmuda_2020}.
Autonomous driving has witnessed significant exploration in the realm of \emph{unsupervised domain adaptation} (UDA), where a model is trained on labelled data from a source domain and unlabelled data from the target domain, with the objective of improving its performance specifically on the target domain.
It has been explored in the context of unlabelled or unseen source or target domains \cite{wilson_survey_2020} using generative \cite{hoffman_cycada_2018} or adversarial methods \cite{vu_advent_2019}. The UDA framework has also been explored for cross-modal learning considering domains as from different sensor modalities \cite{jaritz_xmuda_2020}.
Within the context of forest monitoring, this framework could prove particularly valuable for adapting a model from one forest to another, regardless of whether they belong to the same biome or not, to identify similar species across both the source and target domains (see Sec.~\ref{sec:topic_species} and ~\ref{sec:challenge_species}).
%from one forest to another, belonging the same biome or not, to be able to identify similar species from the source to the target domains.
%
% It would also be helpful to adapt the model the the distribution shift which happen between different sensors.
Additionally, this framework would be beneficial for adapting the model to address distribution shifts that occur between tree signature distributions (see Sec.~\ref{sec:topic_health}, ~\ref{sec:challenge_phenology} and ~\ref{sec:challenge_biotic}) as well as between different sensors (see Sec.~\ref{sec:challenge_data}).
% or from one sensor to another, all while accommodating observations of new tree species.
%
Domain adaptation has been investigated in the field of remote sensing mostly in the context of extrapolation across time or geographical region, including approaches for both aerial and satellite data \cite{wang_exploring_2022, shi_unsupervised_2022, xu_universal_2023, ma_unsupervised_2023, arnaudo_hierarchical_2023} .
Such work holds potential for training generalizable algorithms for forest monitoring, such as adapting models from PhenoCams to satellite images \cite{kosmala_integrating_2018}. 

% \subsubsection{Out-of-distribution samples}
% \label{sec:ml_ood}
% \emph{Out-of-distribution} (OOD) samples are samples lying beyond the training data distribution, and can reflect either individual outliers or broader distribution shifts.
% These samples can significantly impact the generalization of machine learning models and call for specific strategies \cite{shalev_out--distribution_2019, hendrycks_many_2021}.
% %The presence of out-of-distribution (OOD) samples poses a significant challenge to both generalization and addressing distribution shifts. These samples, which lie beyond the training data distribution, can significantly impact the performance and robustness of machine learning models, requiring specific strategies%to handle them effectively 
% %\cite{shalev_out--distribution_2019, hendrycks_many_2021}.
% %
% % Out-of-distribution (OOD) samples are also an important challenge for generalization and distribution shift \cite{shalev_out--distribution_2019, hendrycks_many_2021}. 
% % The OOD samples induce issues as anomaly detection, multi-class novelty detection, open-set recognition or outlier detection
% Specific areas of work involving OOD generalization include anomaly detection, multi-class novelty detection, open-set recognition, and outlier detection \cite{yang_generalized_2022, corbiere_robust_2022}.
% %Monitoring forests at a small or large geographic scale includes OOD samples as rare tree species appearing only in the test splits, or species located only on specific geographic locations.
% In the context of forest-monitoring, OOD samples can take the form, for example, of rare tree species that only appear in the test splits or species that are confined to specific geographic locations \cite{oijen_effects_2005} (see Sec.~\ref{sec:challenge_species}). 


% \subsubsection{Training biases}
% \label{sec:ml_biases}
% \emph{Training biases} are unintended correlations or features learnt by a model caused by an unrepresentative training dataset or erroneous assumptions in the machine learning process \cite{mehrabi_survey_2022}.
% The biases learnt during training, such as spurious correlations, constrain the generalization capacities of a model.
% %The generalization capabilities of a model are constrained by the biases it learns during training, such as spurious correlations. 
% These biases may involve the model focusing excessively on the image background, texture, or style instead of accurately addressing the specific task at hand \cite{roburin_take_2022}. 
% In remote sensing, a model is able to learn a correlation between vegetation and soil moisture \cite{zwieback_vegetationsoil_2019}.
% If a model is trained on snow-covered boreal forests, it may develop a correlation between the pine tree family and the presence of snow cover, even though both elements can exist independently in other scenarios (see Sec.~\ref{sec:challenge_abiotic}).
% % Although methods to mitigate biases in models for forest monitoring have the potential to be beneficial, they have not been extensively investigated thus far. 
% Spatial correlation bias has been notably emphasized in the context of tree species distributions \cite{f_dormann_methods_2007} or biomass estimation \citep{ploton2020spatial}. Additionally, comprehensive data curation and statistical analysis would prove beneficial in reducing bias when training models for forest monitoring.

% The generalization capacities of a model are limited by the bias learnt during training, namely spurious correlations, as focusing on an image background, on its texture or on its style to solve a precise task \cite{roburin_take_2022}. For instance, a model trained on snow-covered boreal forests will tend to learn a correlation pine tree family and a snow cover while both could be found in other cases.


\subsubsection{Foundation models}
\label{sec:ml_foundation_models}
\emph{Foundation models} are models that can operate on diverse sets of input modalities, scales, data regimes and downstream tasks.
They refer to large-scale multi-modal and multi-task models, which have opened up research in generalization capacities such as increasing performances in applications unseen during training \cite{bommasani_opportunities_2021}. 
Most of the discussed machine learning strategies can be further explored by training foundation models with diverse datasets, thereby enhancing their generalization capabilities. 
Forest datasets encompass a wide range of scales, ranging from field measurements to estimated world maps (refer to Figure \ref{fig:scale}), as well as varying resolutions and modalities for different tasks (as outlined in Section \ref{sec:review}). The data diversity necessitates the utilization of generalized deep learning architectures.
% Forest datasets includes data at different scale, from field measurements to estimated world maps (Figure \ref{fig:scale}), different resolution and different modalities for different tasks (Section \ref{sec:review}). The diversity of the data implies to use generalist deep learning architectures.
%
Influenced by the success of large language models (LLMs) \cite{radford_language_2019, devlin_bert_2019, brown_language_2020, chowdhery_palm_2022, hoffmann_training_2022, radford_learning_2021, touvron_llama_2023, driess_palm-e_2023}, recent advancements in computer vision have led to the development of models that incorporate multiple modalities and can perform multiple tasks simultaneously. 
%computer vision models have been recently developed to integrate multi modalities while performing multi tasks simultaneously. 
In recent studies, researchers have explored the concept of multi-task vision by utilizing natural images \cite{cheng_per-pixel_2021, cheng_masked-attention_2022, li_mask_2023, kirillov_segment_2023} or by employing text to enhance performance in vision-based tasks \cite{dancette_dynamic_2022, xu_groupvit_2022, jain_oneformer_2023}. 
%Recent works have performed multi-task vision from natural images \cite{cheng_per-pixel_2021, cheng_masked-attention_2022, li_mask_2023, kirillov_segment_2023} or used text to better perform on vision-based tasks \cite{dancette_dynamic_2022, xu_groupvit_2022, jain_oneformer_2023}. 
% In the realm of integrating image and text for performing tasks on both modalities, alternative approaches have been developed \cite{zhu_uni-perceiver_2022, li_uni-perceiver_2023}. These approaches explore the potential benefits of combining textual and visual information to improve performance. 
In the realm of integrating image and text for performing tasks on both modalities, alternative approaches have been developed to improve performances by benefiting from their combination \cite{zhu_uni-perceiver_2022, li_uni-perceiver_2023}.
Additionally, generalist models have been constructed to be agnostic to specific modalities and tasks \cite{jaegle_perceiver_2021, jaegle_perceiver_2022}, enabling them to handle diverse modalities and tasks with a unified approach.
% Other approaches have been developed to integrate image and text while performing tasks on both \cite{zhu_uni-perceiver_2022, li_uni-perceiver_2023}. 
% Generalist models have also been constructed to be modality and tasks agnostic \cite{jaegle_perceiver_2021, jaegle_perceiver_2022}.
%
% The emergence of large-scale multi-modal and multi-task models, referred to as `foundation models' has opened up research in various applications unseen during training \cite{bommasani_opportunities_2021}. 
In the field of computer vision, for example, the Segment Anything Model (SAM) \cite{kirillov_segment_2023} has demonstrated the capability to perform instance segmentation by leveraging prompts in conjunction with input images. 
These architecture frameworks hold significant value for forest monitoring tasks, enabling the detection, segmentation, and estimation of tree properties over large geographical areas, such as their canopy surface or their above-ground biomass \cite{tucker_sub-continental-scale_2023, tolan_sub-meter_2023}.
%These large multi modal and multi task models, so-called `foundation models' are capable of solving numerous downstream applications \cite{bommasani_opportunities_2021}.
%In computer vision for instance, the Segment Anything Model (SAM) \cite{kirillov_segment_2023} is capable of performing instance segmentation using prompts combined with an image as input.
%These architecture are valuable for forest monitoring to detect, segment and estimate tree properties at a large geographical scale.

%The application of foundation models to remote sensing data is at its early stage. 
%Multi-modal \cite{zhang_mmformer_2023} and temporal-based \cite{garnot_panoptic_2021, garnot_multi-modal_2021, tarasiou_vits_2023} architecture have been developed for remote sensing for a precise task.
The utilization of foundation models with remote sensing data is still in its infancy. 
However, promising advances have been made in developing multi-modal architectures \cite{zhang_mmformer_2023} and temporal-based approaches \cite{garnot_panoptic_2021, garnot_multi-modal_2021, tarasiou_vits_2023} specifically tailored for precise tasks in remote sensing applications. 
%
Based on the masked autoencoder (MAE) pretraining method \cite{he_masked_2022}, multi-modal and multi-task architectures have been developed for Earth observation applications, in particular for land use and land cover (LULC) estimation \cite{sun_ringmo_2022, cong_satmae_2022, reed_scale-mae_2022, tseng_lightweight_2023}.
%
%They are dealing with data from sensors recording different physical measurements, \textit{e.g.} multi-spectral or SAR. Even if different resolution are taken into account in remote sensing \cite{reed_scale-mae_2022, yamazaki_aerialformer_2023, pan_multi-scale_2023} or natural images \cite{themyr_full_2023}, high resolution gap between multi-scale datasets from aerial and satellite recordings have not been yet explored.
%Multi-modal, multi-task and multi-scale architectures are expected to improve generalization on forest monitoring tasks across the globe by being able to ingest many type of datasets and being finetuned to perform specific tasks.
%
These architectures address the challenges posed by data collected from sensors that record diverse physical measurements, such as multispectral or SAR data in remote sensing \cite{reed_scale-mae_2022, yamazaki_aerialformer_2023, pan_multi-scale_2023}, as well as in natural images \cite{themyr_full_2023}. While different spectral, spatial and temporal resolutions have been considered in previous works, there remains a lack of exploration regarding the resolution gap between datasets captured by aerial and satellite sensors.
%high-resolution gap between multi-scale datasets captured by aerial and satellite sensors.
%
The integration of multi-modal, multi-task, and multi-scale architectures is expected to significantly enhance the generalization capabilities of models for forest monitoring tasks at global scale (see Sec.~\ref{sec:challenge_data}). 
%By accommodating various types of datasets and fine-tuning them for specific tasks, 
By training these algorithms with various type of datasets and specialising them for forest monitoring tasks, they could effectively
% they can effectively address the challenges associated with monitoring forests and 
deliver improved performance across different geographical regions such as for species cover estimation or aboveground biomass estimation (see Sec.~\ref{sec:topic_biomass}).
%
%The next paragraph will discuss challenges of learning from limited data, including details on training strategies of foundation models.
%In the following paragraph, we will delve into the challenges related to learning from limited data, specifically focusing on training strategies.% employed by foundation models.


\subsection{Learning from limited data}
\label{sec:ml_limited_data}

There are a growing number of massive datasets and algorithms leveraging them, including across remote sensing \cite{bastani_satlas_2022, sumbul_bigearthnet-mm_2021,rahaman_general_2022, mai_opportunities_2023}. 
%The utilization of supervised learning, relying on human annotations for training machine learning algorithms has been prevalent, albeit demanding in terms of cost and time to generate.
However, many of the most powerful machine learning approaches are \emph{supervised}, and therefore require labels, which can be challenging, time-consuming, and costly to obtain.
%The self-supervised learning thus help to learn generalist features from large scale unlabelled datasets to be then fine tuned on small scale datasets with manual annotations. 
There has been considerable attention given to the problem of learning from limited labeled data; we here present several families of approaches and their relevance to forest-monitoring.

\subsubsection{Self-supervised learning}
\label{sec:ml_ssl}
%Since the rise of machine learning algorithms, more and more datasets are released at larger and larger scale offering many opportunities \cite{mai_opportunities_2023}. Supervised learning has been commonly used to train machine learning algorithms using human annotations, costly and time consuming to create.
Situated in-between supervised and unsupervised learning, the \emph{self-supervised learning} paradigm involves training a model to reconstruct certain known relationships between or within the datapoints themselves. The resulting model can then be fine-tuned with actual labeled data or directly applied to solve the downstream task.
%
%In-between supervised and unsupervised learning, the self-supervised learning paradigm consists in learning a model with a pretext task to fine-tune, or directly solve a downstream task.
% Situated in-between supervised and unsupervised learning, the self-supervised learning approach involves training a model with a pretext task, which can then be fine-tuned or directly applied to solve a downstream task.
%
%Computer vision applications have been directed between discriminative approaches separating representation from positive and negative samples 
Self-supervised approaches in computer vision include discriminative approaches that distinguish between positive and negative samples while separating their  representation (\textit{e.g.} contrastive learning) \cite{gidaris_unsupervised_2018, he_momentum_2020, chen_simple_2020, caron_emerging_2021, oquab_dinov2_2023}, and generative approaches learning representation by reconstruction \cite{lehtinen_noise2noise_2018, he_masked_2022}.
%
% This application has gained momentum in remote sensing thanks to the large amount of unlabelled open source data \cite{tao_self-supervised_2023}.
The utilization of self-supervised learning in remote sensing has experienced significant growth due to the abundance of unlabelled open access data \cite{tao_self-supervised_2023}.
For instance, geolocation of satellite images have been exploited with a contrastive approach \cite{mai_csp_2023,ayush_geography-aware_2021}. 
%
Multi-spectral and SAR data have been reconstructed based on the temporal information \cite{cong_satmae_2022, yadav_unsupervised_2022}, for multi-scale reconstruction \cite{reed_scale-mae_2022} and for denoising \cite{dalsasso_sar2sar_2021, dalsasso_as_2022, meraoumia_multitemporal_2023}.
%Cross-modal approaches, both discriminative \cite{jain_multimodal_2022} and generative \cite{jain_multi-modal_2023}, are emerging to exploit the complementarity of aligned samples.
Emerging cross-modal approaches, encompassing both discriminative \cite{jain_multimodal_2022} and generative \cite{jain_multi-modal_2023} techniques, are being developed to harness the complementary nature of aligned samples.
%
Self-supervised learning will greatly unleash the potential of remote sensing data in the area of forests \cite{ge_novel_2023} by learning textural and geometrical structures of forests and trees without labels (see Sec.~\ref{sec:challenge_species} and ~\ref{sec:challenge_data}).
%Self-supervised learning has been barely investigated for forest monitoring \cite{ge_novel_2023} whereas textural and geometrical structures of forests and trees could be learnt without labels.


\subsubsection{Weakly-supervised learning}
\label{sec:ml_weakly}
%Precise and fine-grained annotations are costly and time consuming. 
%Self-supervised learning aims to learn representations from pretext tasks but still requires precise annotations to fine-tune the model on a downstream task.
%Coarse-grained and inaccurate annotations, or even single point location, are also used as weak labels for weakly-supervised learning 
Obtaining precise and detailed annotations, \textit{e.g.} for tree crown instance segmentation, can be both costly and time-consuming. 
Although self-supervised learning aims to learn representations from pretext tasks, it still necessitates precise annotations for fine-tuning the model in a downstream task. 
In cases where precise annotations are not available, coarse-grained and potentially inaccurate annotations, or even single point locations, can be utilized as weak labels in \emph{weakly-supervised learning} approaches \cite{zhou_brief_2018}.
%
% Since weak annotations are cheaper and faster to obtain, computer vision methods have been developed to exploit them while handling their inaccuracy.
Given their cost-effectiveness and efficiency, computer vision methods have been developed to leverage weak annotations while addressing their inherent inaccuracies \cite{zhou_brief_2018}.
%
Weakly-supervised learning has been explored in the realm of object location \cite{oquab_is_2015}, object relationship estimation \cite{peyre_weakly-supervised_2017}, instance segmentation \cite{ahn_weakly_2019} and contrastive learning \cite{zheng_weakly_2021}.
%
%Since high quality annotations of remote sensing data are difficult to obtain, these methods have also been explored for earth monitoring, such as object detection 
Obtaining high-quality annotations for remote sensing data is particularly difficult due to their poor spatial resolution or the physics of the sensors used. Weakly-supervised learning has therefore been investigated for Earth observation tasks 
% Due to the challenges associated with obtaining high-quality annotations for remote sensing data, these methods have also been investigated for Earth observation tasks, 
including object detection \cite{dingwen_zhang_weakly_2015, han_object_2015, yao_automatic_2021}, LULC semantic segmentation \cite{wang_weakly_2020, yao_semantic_2016} and plant traits regression \cite{cherif_spectra_2023, schiller_deep_2021}.
%
% Weakly-supervised methods have recently been explored tree classification 
Recently, weakly-supervised methods have been investigated in the areas of tree classification \cite{illarionova_tree_2021}, tree counting \cite{amirkolaee_treeformer_2023}, tree detection \cite{aygunes_weakly_2021}, and segmentation \cite{gazzea_tree_2022} using multispectral data (see Sec.~\ref{sec:challenge_data}).


\subsubsection{Active learning}
\label{sec:ml_active}
% Even if the annotations are precise and fine-grained, they can be available in small quantities which is a limitation to train or even finetuned deep learning models.
%Active learning strategies have been developed to select optimal samples for training \cite{cohn_active_1996}, and thus reducing the amount of annotated data required.
Even highly precise and fine-grained annotations are generally less useful if present in only small quantities. 
To address this limitation, \emph{active learning} strategies have been developed to identify and select the optimal way to select a small set of training datapoints to label \cite{cohn_active_1996}.
%
%These sample selection strategies are generally based on estimating the uncertainty of a model 
These strategies for sample selection often rely on estimating the uncertainty of a model \cite{gal_deep_2017}, for instance using variational approaches \cite{sinha_variational_2019} or estimated with a loss function \cite{yoo_learning_2019}.
% They have shown to be efficient strategies for reduced regime of labelled data for image classification 
They have demonstrated their effectiveness in scenarios where the amount of labeled data is limited, particularly in the context of image classification \cite{gal_deep_2017, sinha_variational_2019, yoo_learning_2019}, object detection \cite{roy_deep_2019} and semantic segmentation \cite{siddiqui_viewal_2020}.
%
%Active learning  have also been explored for remote sensing applications, such as classification 
Active learning has also been investigated for remote sensing applications, including classification \cite{tuia_survey_2011}, object detection \cite{qu_deep_2020} and LULC semantic segmentation with hyperspectral data \cite{li_semisupervised_2010, li_hyperspectral_2011, zhang_active_2016}.
%
%However it has not been extensively investigated for forest monitoring.
% Nevertheless, its application in the domain of forest monitoring has not been extensively explored.
Its application would be helpful for forest monitoring to optimize and create relevant human annotations (see Sec.~\ref{sec:challenge_data}).



\subsubsection{Few-shot learning}
\label{sec:ml_fewshot}
%Other machine learning approaches have been developed to tackle the lack of data.
%Few-shot learning has been develop to efficiently finetuned a model with a few annotated samples from unseen classes. 
Yet another approach to limited data availability is \emph{few-shot learning}, which refers to efficient fine-tuning of a pretrained model using only a few annotated datapoints.
%
%It has been formulated differently by comparing the small annotated dataset and the data used to pretrain the model, by quantifying their similarities \cite{vinyals_matching_2016}, by creating mixture of the feature embedding \cite{snell_prototypical_2017} or by adapting the optimization scheme with meta learning \cite{finn_model-agnostic_2017}.
Few-shot learning has been approached from different perspectives, considering the comparison between the small annotated dataset and the data used for pretraining the model -- for example, by quantifying the similarities between these datasets \cite{vinyals_matching_2016}, constructing mixtures of feature embeddings \cite{snell_prototypical_2017} or adapting the optimization scheme through meta-learning \cite{finn_model-agnostic_2017}.
%
%Remote sensing applications for few-shot learning have been explored motivated by the low quantity of annotations available.
Motivated by the limited availability of annotations, applications of few-shot learning in remote sensing tasks have been investigated. 
For instance, methods based on feature similarity \cite{zhang_few-shot_2020, alajaji_few-shot_2020, alosaimi_self-supervised_2023} and metric learning \cite{liu_deep_2019}, aiming at separating representations in an embedding space, have been explored for LULC classification with either multispectral or hyperspectral data.
%
Objects have also been detected by learning meta features \cite{deng_few-shot_2022}.
%
%Few-shot learning methods have also been applied for semantic segmentation using metric learning 
Metric learning techniques have also been utilized in the context of few-shot learning for semantic segmentation tasks \cite{jiang_few-shot_2022} or meta learning with multispectral and SAR data \cite{ruswurm_meta-learning_2020}.
%
% These methods have not been widely explored for forest monitoring except for tree species classification using feature similarity \cite{chen_new_2021}.
Few-shot learning has been explored for tree species classification using feature similarity \cite{chen_new_2021} and would be beneficial to recognize a species or estimate the characteristics of a tree with minimal manual annotations (see Sec.~\ref{sec:challenge_phenology} and ~\ref{sec:challenge_data}).


\subsubsection{Zero-shot learning}
\label{sec:ml_zeroshot}
%The machine learning community has also been interested in developing methods to train algorithms to distinguish unseen classes without annotated sample, namely zero-shot learning 
The machine learning community has also shown interest in developing methods for training algorithms to differentiate unseen classes without any explicitly annotated samples at all, which is known as \emph{zero-shot learning} \cite{xian_zero-shot_2018}.
%
%This task has been performed by projecting embeddings of images and words 
In order to categorize unseen classes, the task of zero-shot learning has been accomplished by projecting image and word embeddings \cite{socher_zero-shot_2013} or known semantic attributes \cite{lampert_attribute-based_2014} into a shared space.
%in the same space to categorize unseen classes. 
%
%Zero-shot learning has also been explored by considering mixture of source domain embeddings before computing similarities with the target domain containing the unseen classes 
Zero-shot learning has also been investigated by incorporating a mixture of embeddings from the source domain before computing similarities with the target domain, which includes the unseen classes \cite{zhang_zero-shot_2015}.
%
Generative approaches have been developed to create visual feature embeddings of unseen classes from word embeddings for zero-shot semantic segmentation \cite{bucher_zero-shot_2019}.
%
%Since zero-shot learning does not require labels to classify new classes, it has gain interest in remote sensing applications.
Zero-shot learning has also garnered attention in remote sensing applications, including
%Direct applications from computer vision have been applied combining multispectral data and word embeddings for classification 
 combining multispectral data and word embeddings for classification tasks \cite{li_zero-shot_2017, li_robust_2021}
%including preliminary work in classifying hyperspectral data 
and initial exploration of applying zero-shot learning to classify hyperspectral data \cite{freitas_hyperspectral_2022}.
Generative approaches have also been used with remote sensing data to create visual embeddings from word embeddings  \cite{li_generative_2022}.
%
% Zero-shot learning is a promising strategy for forest monitoring to adapt a model in a region with unseen species.
Zero-shot learning presents a promising approach for forest monitoring, enabling the adaptation of models in regions where previously unseen species are encountered (see Sec.~\ref{sec:challenge_phenology} and ~\ref{sec:challenge_data}).
% Tree taxonomy hierarchy and meta characteristics could be matched with visual embeddings 
By leveraging tree taxonomy hierarchy and meta characteristics to align with visual embeddings \cite{sumbul_fine-grained_2018}, a vast research potential emerges. 
%opening up a large potential for research, in particular by using foundation models have show to be great zero shot learners 
Notably, utilizing foundation models that have demonstrated strong zero-shot learning capabilities \cite{brown_language_2020, radford_learning_2021} further enhances this potential.
%
In the following paragraph %methods for domain relevant metrics will be addressed to discuss physical and biological constraints and applications. 
will explore methods concerning domain-specific objectives, focusing on the consideration of physical and biological constraints and their applications.






\subsection{Domain-specific objectives}
\label{sec:ml_metrics}

Machine learning methods commonly use a fairly limited set of metrics to evaluate success, such as (macro or micro) accuracy of labels and cross-entropy loss for classification tasks, mean squared error or mean average error for regression tasks, etc. However, these uniform metrics do not necessarily reflect the realities of real-world use cases  \cite{birhane2022values}, where criteria for success may be much more nuanced or domain-specific. In this section, we consider two other families of objectives that may frequently be of relevance in forest-monitoring.
%The general theory of machine learning challenges has been adapted with different applications. Specifies of the data structure and their properties are taken into account to better adapt the methods to researcher needs.
%
%Physical properties help to enforce models to learn data structures by enforcing them in the model architecture, in the loss function or in the optimization scheme.
%
%As soft constrains may includes restriction on the loss function or the output of the network, hard constrains directly impact the optimization scheme.

\subsubsection{Constraints on data}
\label{sec:ml_constraints}

Depending on the domain of application, the outputs of a machine learning pipeline may have specific constraints that must be satisfied if the answer is to be useful or even possible. For example, climate variables may need to obey physical laws such as conservation of energy, engineered systems may need to obey the laws of mechanics, etc. Machine learning models to work with such variables have increasingly been designed with soft constraints \cite{harder_physics-informed_2022,ouaknine_multi-view_2021}, which impose penalties for constraint violation, or hard constraints \cite{donti_dc3_2021,geiss_strict_2021,harder_physics-constrained_2023}, where the constraints are strictly enforced by the design of the algorithm.
% %
% %Soft penalties have been used to constrain a loss function for aerosol microphysics emulation 
% Soft penalties have been employed to impose constraints on a loss function for emulating aerosol microphysics \cite{harder_physics-informed_2022} or to %take into account physical properties of a radar sensor in autonomous driving 
% consider the physical properties of a RADAR sensor in the domain of autonomous driving \cite{ouaknine_multi-view_2021}.
% %
% Hard constraints have been enforced in deep learning-based optimization algorithms, \textit{e.g.} for power flows on the electrical grid \cite{donti_dc3_2021} or for climate-related downscaling \cite{geiss_strict_2021}.
% %
% Both soft and hard constraints have been simultaneously enforced for climate downscaling \cite{harder_physics-constrained_2023}.
% %
%These constraints could be defined w.r.t. forest and tree properties (see Sec.~\ref{sec:forest_topics}) to guide machine learning model during training.
Compared to physics- and engineered-based constraints, fewer authors have to date integrated biological constraints into ML algorithms. 
%For instance, neural network architectures have been designed while being inspired from the sensory cortex \cite{yamins_using_2016}. 
Dynamics of biological systems have been included in a deep learning optimization scheme as hard constraints from ordinary differential equations \cite{yazdani_systems_2020}.
% Hard constraints including ordinary differential equations in a deep learning optimization scheme to include the dynamics of biological systems \cite{yazdani_systems_2020}.
There are potential opportunities for incorporating biological constraints in forest monitoring by considering phenological \cite{richardson_intercomparison_2018} or biophysical traits, or ecosystem properties (see Sec.~\ref{sec:topic_properties}), for example by considering the ratio of tree height and canopy size. These constraints could be particularly valuable in tasks such as semantic segmentation or biomass estimation.

Domain-specific constraints on data may also pose opportunities for improving the design of machine learning models. 
The design of deep learning model architectures can incorporate considerations for, or reconstruction of, physical properties. 
For instance, a physics-informed architecture has been developed for super-resolution in turbulent flows, incorporating partial differential equations as a form of regularization \cite{jiang_meshfreeflownet_2020}. 
Similarly, RADAR-based architectures have been created to reconstruct physical properties for scene understanding in the context of autonomous driving \cite{ouaknine_multi-view_2021, rebut_raw_2022}.
% In super-resolution for turbulent flows, MeshfreeFlowNet \AOc{Jiang/MeshfreeFlowNet} employs a physics-informed model which adds PDEs (partial differential equations) as regularization terms to the loss function.
%
%Exploiting multiple sensor properties has also been used to fuse their representation \cite{ouaknine_deep_2022} or to create annotations from one modality to another \cite{ouaknine_carrada_2021}.
Leveraging the properties of multiple sensors has also been employed to fuse their representations \cite{ouaknine_deep_2022} or to generate annotations from one modality to another \cite{ouaknine_carrada_2021, schiefer_uav-based_2023}.
%
In remote sensing, self-supervised learning has benefited from SAR physical properties by considering a pretext denoising task \cite{dalsasso_sar2sar_2021, meraoumia_multitemporal_2023}, or by separating and reconstructing the real from the imaginary part of the signal \cite{dalsasso_as_2022}.
Such methods could also be explored by exploiting various sensors to learn representation of forests and trees (see Sec.~\ref{sec:challenge_data}).

% \subsubsection{Biology-informed methods}
% \label{sec:ml_bio}
%
% Less work have considered biological constraints for machine learning applications.

%There are opportunities in forest monitoring by enforcing biological constraints depending on the phenology of a species or size boundaries depending on its height and canopy size ratio for semantic segmentation or biomass estimation.


%
%Example with radar autonomous driving
%Radar physical properties: taken into account to build the architecture \cite{ouaknine_multi-view_2021, rebut_raw_2022}, 
%loss to introduce physical constraints based on the sensor properties \cite{ouaknine_multi-view_2021}
%
%sensor fusion based on physical constraints \cite{ouaknine_deep_2022}\AOc{+ lidar / radar PC fusion?}, generate annotation automatically based data physical properties \cite{ouaknine_carrada_2021}.
%
%In remote sensing: physic information might be helpful for self-supervised learning to create pretext tasks \cite{dalsasso_sar2sar_2021, dalsasso_as_2022, meraoumia_multitemporal_2023}.


\subsubsection{Uncertainty quantification}
\label{sec:ml_uncertainty}
%Biological phenomenons follow rules that are difficult to estimate and are suggest to uncertainties. Estimation of prediction uncertainty help to better estimate the strengths and weaknesses of a model.
Biological phenomena adhere to intricate rules that are challenging to estimate and often exhibit inherent uncertainties. 
The estimation of prediction uncertainty aids in obtaining a better understanding of the strengths and limitations of a machine learning model.
%
% Aleatoric and epistemic uncertainties can be distinguished by containing the noise of the data and the model uncertainty respectively.
% Aleatoric and epistemic uncertainties can be differentiated based on their sources, where aleatoric uncertainty pertains to the inherent noise present in the data, while epistemic uncertainty is associated to the model itself.
%
%The overall uncertainty is composed of aleatoric and epistemic uncertainties \cite{gal_uncertainty_2016}. The aleatoric uncertainty is related to the noise in the data and label distributions. The epistemic uncertainty is related to the model, including its estimated parameters and its structure.
The overall uncertainty of these models comprises both aleatoric and epistemic uncertainties \cite{gal_uncertainty_2016}.
They both can be distinguished based on their origins. 
Aleatoric uncertainty arises from the inherent noise present in the data and label distributions, while epistemic uncertainty is associated with the model itself, encompassing its estimated parameters and structural characteristics. 
%
%Methods have been developed to estimate the uncertainties of deep neural networks
Approaches have been devised to estimate the uncertainties of deep neural networks, \textit{e.g.} by using a Bayesian approach such as Monte Carlo dropout \cite{gal_dropout_2016}, by using adversarial training combined with model ensembles \cite{lakshminarayanan_simple_2017}, by predicting the uncertainty distribution \cite{malinin_predictive_2018} or by learning an auxiliary confidence score from the data \cite{corbiere_addressing_2019, corbiere_robust_2022}.
%
%In general: monte carlo dropout \cite{gal_dropout_2016} , auxiliary estimation \cite{corbiere_addressing_2019, corbiere_robust_2022}
Similar methods have been applied to estimate uncertainty in remote sensing data for crop yield estimation \cite{ma_corn_2021} or for road segmentation \cite{haas_uncertainty_2021}.
%
%The uncertainty of forest monitoring methods have been quantified to evaluate both aleatoric and epistemic uncertainties. It has been usually done to evaluate uncertainty of predictions at large scale maps using low resolution data as satellite data.
The quantification of uncertainties in forest monitoring methods has been carried out to assess both aleatoric and epistemic uncertainties (see Sec.~\ref{sec:forest_challenges}). This is commonly performed to evaluate the uncertainty of predictions on large-scale maps, utilizing low-resolution satellite data.
%
The uncertainty of plant functional type has been studied for classification in Siberia \cite{ottle_use_2013}.
%
% The above ground biomass uncertainty has also been estimated to define a range of estimated values in carbon stock maps 
Estimating the uncertainty of above ground biomass has also been conducted to establish a range of estimated values in carbon stock maps \cite{patterson_statistical_2019, santoro_global_2021} (see Sec.~\ref{sec:topic_biomass}).
%
%These method are using standard deviations or output probabilities of the model to quantify the uncertainty. Recent work estimated tree carbon stocks in semi-arid sub-Saharan Africa north of the Equator have went further by combining the uncertainty of allometric equations and from the predicted crown segmentation using field measurements 
To quantify uncertainty, these methods utilize standard deviations or output probabilities of the model.
Recent studies have taken a step further in estimating tree carbon stocks in semi-arid sub-Saharan Africa north of the Equator by combining uncertainty from both allometric equations and predicted crown segmentation, utilizing field measurements \cite{tucker_sub-continental-scale_2023}. 
%
%Advanced uncertainty quantification methods, either related to the data or to the predictive model, have not been extensively applied for forest monitoring.
There has been limited application of advanced uncertainty quantification methods, whether associated with the data or the predictive model, in the context of forest monitoring.

% Conclusion
Despite the extensive application of the presented machine learning techniques in remote sensing, their utilization for forest monitoring has been relatively limited. 
This presents numerous opportunities to gain deeper insights into the composition of forests while achieving generalization at a large scale. 
However, it is crucial to have access to high-quality, diverse, and sufficient datasets in order to effectively explore machine learning strategies. 
In the following section, we will review open access forest datasets, providing information on their size, tasks, scale, and modalities.




\section[Review of open access forest datasets]{Review of open access forest datasets}
\label{sec:review}

Open access datasets are essential to drive the scientific community in general to exploring forest biology challenges, in particular by using machine learning strategies (see Sec. \ref{sec:ml_challenges}).
Deep learning algorithms have demonstrated strong performance in various forest monitoring tasks, such as tree classification or segmentation \cite{kattenborn_review_2021}.
The availability of open access datasets has played a significant role in enhancing the algorithm performances and expanding their applications on a larger scale.
% Machine learning algorithms are designed and trained using datasets, they could be helpful to understand and monitor forests at a global scale (Sec. \ref{sec:challenges}).
In this particular field, the use of data, from the tree to the country level (see Fig. \ref{fig:scale}), distributed in the entire globe, must be taken into consideration.
%Algorithms have been trained for forest monitoring using datasets at different scales with various modalities and tasks \cite{guimaraes_forestry_2020, kattenborn_review_2021, michalowska_review_2021}.
Algorithms have been trained for forest monitoring by leveraging datasets that encompass different scales, modalities, and tasks \cite{guimaraes_forestry_2020, kattenborn_review_2021, michalowska_review_2021}.
% However data sources are often not available limit the access to the public and thus, extended research projects.
However, the limited availability of data sources often restricts public access, thereby impeding the progress of extended research projects.
% Although scientific communities encourage reproducible experiments, datasets sometimes do not comply with fair principles\footnote{\url{https://www.go-fair.org/fair-principles/}} such as documentation or findability or 
While the scientific community emphasizes the importance of reproducible experiments, it is worth noting that some datasets do not fully adhere to the fair principles\footnote{\url{https://www.go-fair.org/fair-principles/}}, which encompass aspects like documentation and findability.

%There is still a large quantity of datasets publicly available. However, they have limits reducing their impact in machine learning applications for forest composition analysis, such as their size or type of released data. 
While there is still a considerable quantity of publicly available datasets, it is important to acknowledge that they may have certain limitations that restrict their impact in machine learning applications for forest composition analysis. These limitations can include factors such as the size of the dataset or the specific type of data that is released.
This section aims to review forest monitoring datasets considering the following criteria:
\begin{enumerate}
    \item The dataset should be open access, \textit{i.e.} without any request requirement;
    \item The dataset should be related to at least one published article, exceptions have been made for datasets that are available as preprints, but are considered to be must-see datasets;
    \item The dataset should be focused on the composition of the forest, excluding event-based specific ones (\textit{i.e.} wildfire detection);
    \item A land use and/or land cover (LULC) dataset should contain more than a single plant functional type (\textit{i.e.} conifers or deciduous) since a focus is made on better understanding the composition of the forest;
    \item The dataset should be at the tree level at least, excluding datasets at the organ or cellular level considered as out of the scope of this review (\textit{e.g.} leaf spectra or root scans);
    \item The dataset should contained at least $O(1000)$ trees. %, excluding those with fewer individual observations. % small scale volume ones.
\end{enumerate}
%
% More than 80 datasets have been listed according to these criteria. There are spread all over the world, using data from 1974 to 2022, related to publications from 2005 to 2023 as illustrated in Figure \ref{fig:distributions}.
Based on these criteria, 86 datasets have been identified representing a wide range of geographical locations and spanning from 1974 to 2022. The datasets are associated with publications from 2005 to 2023, as depicted in Figure \ref{fig:distributions}.

The scope of the presented review is broad, it is likely that other datasets meeting these requirements have been missed.
Based on this motivation, the study is supported by \textbf{OpenForest}\footnote{The catalogue contains all urls to access the datasets which are not included in this article to ensure a temporal consistency. \textbf{OpenForest} is available here: \url{https://github.com/RolnickLab/OpenForest}}, a dynamic catalogue integrating the reviewed datasets and open to updates from the community. Updates on \textbf{OpenForest} will be restricted with the criteria detailed above. We hope to motivate researchers by grouping our efforts to create the largest database of open access forest datasets and thus create synergies %through the computer vision and machine learning communities to increase 
on forest monitoring applications.

This section will review open access forest datasets grouped at different scales as presented in Figure \ref{fig:scale}: inventories (Sec. \ref{sec:review_inventories}), ground-based recordings (Sec. \ref{sec:review_ground}), aerial recordings (Sec. \ref{sec:review_aerial}), satellite recordings (Sec. \ref{sec:review_satellite}) and country or world maps (Sec. \ref{sec:review_maps}). Datasets composed of mixed scales are finally presented (Sec. \ref{sec:review_mixed}). 

Each section will detail the overall scope of the presented datasets with the specificity of the sensors used to record the data, the information related to each dataset and their applications.
%
In each section, the reviewed datasets will be categorized in tables respectively to the scale of the released data. 
In these tables, the publication and recording years are differentiated to better understand the temporal scope of the datasets. The recordings years are distinguished with a new line while time series are represented by an upper dash.
Each table will relate the available modalities in the `data' column. This one is separated with the `spatial resolution', or `spatial precision' columns (except for inventories) with a dashed line to associate a resolution to the corresponding modality.
%
Each section will also discuss the limits of current open access datasets to motivate our perspectives presented in Section \ref{sec:perspectives}.
%
The following section will review inventory datasets as the smallest scale of recordings that have been taken into account.


\begin{figure}[t]%
\includegraphics[width=1.0\textwidth]{figures/review_distributions.png}
\caption{
\textbf{Distribution of the reviewed open access forest datasets.} (Left) World map of the location of the reviewed datasets at the country level. Most of the datasets are regional and do not reflect the entire associated country. The datasets categorized with a `Worldwide' location or at the continent level have been excluded for visualization purposes. (Right) Distributions of the publication years and recording years used and / or released in the associated datasets}
\label{fig:distributions}
\end{figure}



\subsection{Inventories}
\label{sec:review_inventories}

% Concept / utility
%Historically, biologists have mostly locally or regionally  inventoried forests based on stratified plot samples acquired in the field  \AOc{ref?}. 
Historically, forests have been mostly locally or regionally inventoried based on stratified plot samples acquired in the field \cite{jucker_tallo_2022}.
Digitized and open access inventories generally cover small areas, consisting of dozens or a few hundred trees, which limits their impact on the machine learning community (Sec. \ref{sec:ml_challenges}). 
As defined in Section \ref{sec:review}, this section is focused on medium to large scale inventories with at least $O(1000)$ trees.
%A significant part of reviewed inventory datasets are mixed with modalities at different scales, \textit{i.e.} aerial recordings or estimated maps. These specific cases will be detailed in Section \ref{sec:review_mixed}.
A significant part of reviewed inventory datasets are mixed with modalities at different scales, which will be detailed in Section \ref{sec:review_mixed}.

% volume, data and resolution
Inventory datasets are summarized in Table \ref{tab:inventories}, the size of the datasets is quantified by the number of trees. 
%The geographic coordinates are provided for each tree and their spatial resolution is defined by the precision of the GPS coordinates. We neither assess the accuracy of the coordinate measurement, nor the size of the plot in which they are contained since this information is not always available.
Inventory datasets are composed of various measurements. They commonly contain tree height, canopy diameter, diameter at breast height (DBH) or diameter at soil height (DSH) \cite{national_ecological_observatory_network_neon_vegetation_2023, gastauer_tree_2015, perez-luque_dataset_2015, perez-luque_land-use_2021, oliveira_structure_2017, jucker_tallo_2022}. In specific cases, wood density, bark density and bark thickness are also measured \cite{schepaschenko_forest_2019, farias_dataset_2020, kindermann_dataset_2022}. These information are particularly useful to estimate the tree density, the above ground biomass (AGB) or the tree carbon stock at large scale \cite{tucker_sub-continental-scale_2023} even if the inventories have not been released with the estimated maps \cite{patterson_statistical_2019, dionizio_carbon_2020}.

% classes
Species, genus and family of the trees are generally provided. 
This hierarchy of labels coming alongside with the tree geo-location make inventories a very accurate datasets for understanding forest composition.
However, they are geographically sparse and centered in a specific location to reduce measurement efforts \cite{laar_forest_2007, motz_sampling_2010}. 
As an exception, Tallo \cite{jucker_tallo_2022} groups inventories from all around the world with an unprecedented number of species reported. The latter could have an impact on estimating tree species distribution at large scale.

% tasks
% specificities and openings 
Considering that inventories contain annotations of trees or tree clusters, they open possibilities to segment tree canopies according to their taxonomic levels, regress continuous metrics (\textit{i.e.} height, biomass) or even locate tree individuals by predicting their coordinates or crown perimeter \cite{tucker_sub-continental-scale_2023}.
% If inventories are considered as annotations, we propose potential tasks to solve using machine learning algorithms. One could locate trees by predicting their coordinates, classify them according to their taxonomies level or regress their characteristics (\textit{i.e.} height).
Another example could be to estimate the wood density of a tree or its carbon stock using allometric equations with information on taxonomy and height measured on the field \cite{zianis_biomass_2005}. 
% Considering the class estimated on the field and the height of the tree, an algorithm could astimate its wood density or its carbon stock automatically.
% Inventories could also be considered as annotations and combined with other modalities used as input features. 
Inventories could also be combined with other modalities and used as annotations for larger scale tasks.
As an example, remote sensing datasets presented in the following sections in the same geographic locations could be associated to inventories to enhance the precision of their annotations. 
While establishing this connection between ground measurements and remote sensing data presents its own set of challenges.
In the following section, we will review datasets of ground-based recordings.
%\ELc{I think it might be worth talking about the challenges of linking these inventories to remote sensing data as "labels" at the tree level. This is because individual trees themselves are rarely geolocated (plot centers typically are), and geolocation errors tend to be large and often unreported, because of difficulty in obtaining accurate GNSS coordinates under a canopy due to multipath errors.}

%%%%%%%%%% THIS TABLE CONTAINS THE SPATIAL PRECISION/RESOLUTION
% \begin{table*}[ht]
% \centering
% %\footnotesize
% \fontsize{6.5pt}{7.5pt}\selectfont % Font size 
% \renewcommand{\arraystretch}{1.5} % Size between lines
% \setlength\tabcolsep{5pt} % intercolumn size
% \caption{Review of open source forest inventories datasets}
% {\begin{fntable}
% \begin{tabular}{p{1.3cm} | p{0.4cm} | p{1cm} | p{1cm} | p{1.4cm} : p{1cm} | p{0.4cm} | p{1cm} | p{1cm} | p{1cm} | p{1.1cm}}
% \toprule
% Dataset  & Publi. year & Recording year & Dataset size & Data  & Spatial resolution & Time series & Potential task(s) & \#Classes & Location  & License  \\
% \midrule

% NEON Vegetation Structure \cite{kampe_neon_2010, national_ecological_observatory_network_neon_vegetation_2023} & 2010 & 2014-2021 & Unknown & Location \newline Height \newline Crown diam. \newline Stem diam. \newline Health & 10cm \newline N/A \newline N/A \newline N/A \newline N/A & Yes& OL \newline Classif. \newline Reg. & 2826 species \newline 949 genus \newline 316 families & USA & CC0-1.0 \\

% Seu Nico Forest	\cite{gastauer_tree_2015} & 2015 & 2001-2010 & 2868 trees & Location \newline DBH \newline Height \newline Soil & 10km \newline N/A \newline N/A \newline 10m & Yes & OL \newline Classif. \newline Reg. & 228 species \newline 139 genera \newline 54 families & Brazil & CC0-1.0	\\

% MIGRAME	\cite{perez-luque_dataset_2015, perez-luque_land-use_2021} & 2015 & 2012-2014 & 3839 trees & Location \newline Diameter \newline Height & 1m \newline N/A \newline N/A & No & OL \newline Classif. \newline Reg.	& 6 species \newline 5 genus \newline 5 families & Spain & CC-BY-NC-4.0 \\

% Northern Brazilian Amazonia	\cite{oliveira_structure_2017} & 2017 & 2014 \newline 2015 & 1026 trees & Location \newline DBH \newline DSH & 10km \newline N/A \newline N/A & No & OL \newline Classif. \newline Reg. & 52 species \newline 28 families & Brazil & CC-BY-4.0 \\

% Forest Observation System (FOS) \cite{schepaschenko_forest_2019} & 2019 & between 1999 and 2018 & 1646 trees & Location \newline Canopy height \newline Tree density \newline Wood density \newline AGB & 1km \newline N/A \newline Unknown \newline Unknown \newline N/A & No & OL \newline Reg. & N/A & Worldwide & CC-BY-4.0 \\ 

% SeasonWatch	\cite{ramaswami_using_2020} & 2020 & 2011-2019 & 352K trees & Location \newline Estimation of leaf, flower and fruit quantity & 10km \newline N/A & Yes & OL \newline Classif. \newline Reg.  & 136 & India & CC-BY-4.0 \\

% Maraca Ecological Station	\cite{farias_dataset_2020} & 2020 & 2018 \newline 2019 & 680 trees & Location \newline Bark thickness \newline Bark density \newline Wood density & 10cm \newline N/A \newline 0.001g \newline 0.001g & No & OL \newline Classif. \newline Reg. & 110 species \newline 40 families & Brazil & CC-BY-4.0 \\

% Tallo \cite{jucker_tallo_2022} & 2022 & Unknown & 499K trees & Location \newline Diameter \newline Height \newline Crown radius & 100m \newline N/A \newline N/A \newline N/A & No & OL \newline Classif. \newline Reg. & 5163 species \newline 1453 genus \newline 187 families & Worldwide & CC-BY-4.0 \\

% African Savanna	\cite{kindermann_dataset_2022} & 2022 & 2018 \newline 2019 & 6179 trees and shrub & Location \newline Height \newline Stem circ. \newline Canopy diam. \newline Wood density & >1cm \newline N/A \newline N/A \newline N/A \newline 0.001g & No & OL \newline Classif. \newline Reg. & 65 species \newline 6 categ. & Namibia & CC-BY-4.0 \\

% \bottomrule

% \end{tabular}
% \footnotetext[]{{Acronyms}: \textbf{N/A}: non applicable; \textbf{Unknown}: non provided by the authors; \textbf{DBH}: diameter at breast height, \textbf{DSH}: diameter at soil height; \textbf{AGB}: above ground biomass; \textbf{OL}: object localization; \textbf{Reg.}: regression; \textbf{Classif.}: classification. Note that the dataset size measured in \textbf{K} are $O(10^3)$.}
% \footnotetext[]{Note that the spatial resolution of the tree location is defined by the precision of the GPS coordinates (see Sec.~\ref{sec:review_inventories} for more details).}
% \end{fntable}}
% \label{tab:inventories}
% \end{table*}

\begin{table*}[ht]
%\centering
%\footnotesize
\fontsize{6.5pt}{7.5pt}\selectfont % Font size 
\renewcommand{\arraystretch}{1.5} % Size between lines
\setlength\tabcolsep{5pt} % intercolumn size
\caption{Review of open access forest inventories datasets}
{\begin{fntable}
\centering
%\rowcolors{1}{}{lightgray}
\begin{tabular}{p{1.3cm} | p{0.4cm} | p{1cm} | p{1cm} | p{1.4cm} | p{0.4cm} | p{1cm} | p{1cm} | p{1cm} | p{1.1cm}}
\toprule
Dataset  & Publi. year & Recording year & Dataset size & Data & Time series & Potential task(s) & \#Classes & Location  & License  \\
\midrule

NEON Vegetation Structure \cite{kampe_neon_2010, national_ecological_observatory_network_neon_vegetation_2023} & 2010 & 2014-2021 & Unknown & Location \newline Height \newline Crown diam. \newline Stem diam. \newline Health & Yes & OL \newline Classif. \newline Reg. & 2826 species \newline 949 genus \newline 316 families & USA & CC0-1.0 \\

\rowcolor{lightgray}
Seu Nico Forest	\cite{gastauer_tree_2015} & 2015 & 2001-2010 & 2868 trees & Location \newline DBH \newline Height \newline Soil & Yes & OL \newline Classif. \newline Reg. & 228 species \newline 139 genera \newline 54 families & Brazil & CC0-1.0	\\

MIGRAME	\cite{perez-luque_dataset_2015, perez-luque_land-use_2021} & 2015 & 2012-2014 & 3839 trees & Location \newline Diameter \newline Height & No & OL \newline Classif. \newline Reg. & 6 species \newline 5 genus \newline 5 families & Spain & CC-BY-NC-4.0 \\

\rowcolor{lightgray}
Northern Brazilian Amazonia	\cite{oliveira_structure_2017} & 2017 & 2014 \newline 2015 & 1026 trees & Location \newline DBH \newline DSH & No & OL \newline Classif. \newline Reg. & 52 species \newline 28 families & Brazil & CC-BY-4.0 \\

Forest Observation System (FOS) \cite{schepaschenko_forest_2019} & 2019 & between 1999 and 2018 & 1646 trees & Location \newline Canopy height \newline Tree density \newline Wood density \newline AGB & No & OL \newline Reg. & N/A & Worldwide & CC-BY-4.0 \\ 

\rowcolor{lightgray}
SeasonWatch	\cite{ramaswami_using_2020} & 2020 & 2011-2019 & 352K trees & Location \newline Estimation of leaf, flower and fruit quantity & Yes & OL \newline Classif. \newline Reg.  & 136 & India & CC-BY-4.0 \\

Maraca Ecological Station	\cite{farias_dataset_2020} & 2020 & 2018 \newline 2019 & 680 trees & Location \newline Bark thickness \newline Bark density \newline Wood density & No & OL \newline Classif. \newline Reg. & 110 species \newline 40 families & Brazil & CC-BY-4.0 \\

\rowcolor{lightgray}
Tallo \cite{jucker_tallo_2022} & 2022 & Unknown & 499K trees & Location \newline Diameter \newline Height \newline Crown radius & No & OL \newline Classif. \newline Reg. & 5163 species \newline 1453 genus \newline 187 families & Worldwide & CC-BY-4.0 \\

African Savanna	\cite{kindermann_dataset_2022} & 2022 & 2018 \newline 2019 & 6179 trees and shrub & Location \newline Height \newline Stem circ. \newline Canopy diam. \newline Wood density & No & OL \newline Classif. \newline Reg. & 65 species \newline 6 categ. & Namibia & CC-BY-4.0 \\

\bottomrule

\end{tabular}
\footnotetext[]{{Acronyms}: \textbf{N/A}: non applicable; \textbf{Unknown}: non provided by the authors; \textbf{DBH}: diameter at breast height, \textbf{DSH}: diameter at soil height; \textbf{AGB}: above ground biomass; \textbf{OL}: object localization; \textbf{Reg.}: regression; \textbf{Classif.}: classification. Note that the dataset size measured in \textbf{K} are $O(10^3)$.}
\end{fntable}}
\label{tab:inventories}
\end{table*}


\subsection{Ground-based recordings}
\label{sec:review_ground}

% Concept / utility

%Datasets relate recordings in forests below the canopy of the trees. Different sensors are used to better understand the structure and the composition inside the forests.

The fine-scaled composition of forests can be understood by visualising the trees within or under their canopy. 
Ground-based datasets are composed of recordings inside the forests, under the tree canopy. 
Trunks and small trees, invisible from a bird's eye view, can be captured with cameras recording red-green-blue (RGB) images per example. These data are sometimes recorded in time series \textit{e.g.} PhenoCams \cite{klosterman_evaluating_2014, brown_using_2016}. 
%The use of sensors helps to have a broader context and more tree information in the recordings compared to inventories.
The use of data recorded by sensors by machine learning algorithms help to have a broader context and more tree information in the samples compared to inventories.

% volume, data and resolution
Ground-based datasets are reviewed in Table \ref{tab:ground}. The dataset size has been measured in hectares (ha) corresponding to the studied surface, in number of trees in the area or in number of samples, which may differ between synthetic or real samples \cite{grondin_tree_2022}.

% data and resolution
Stereo cameras are parameterized to estimate the depth of a scene differentiating trees and objects from the background in the forest \cite{grondin_tree_2022}. 
Thermal cameras have also been used to record trees' signature \cite{still_thermal_2019} and distinguish them from other objects \cite{da_silva_visible_2021, da_silva_unimodal_2021, da_silva_edge_2022, reis_forest_2020}.
In specific cases, camera images have been annotated with bounding boxes around trees to detect them \cite{tremblay_automatic_2020, grondin_tree_2022}.
Only two reviewed datasets located in Canada have been annotated with several species classes to combine detection and classification of trees \cite{tremblay_automatic_2020, grondin_tree_2022}.
%
Since these datasets also provide inertial measurement unit (IMU), a potential task could be to predict the next move of an automated agent in a forest.

Forest geometry is also being intensively studied from the ground by using LiDAR - typically referred to as terrestrial laser scanning (TLS). This active sensor records 3-dimensional scenes with photon reflections and can be applied from tripods or be combined with IMUs to enable mobile laser scanning. % similarly as used in autonomous driving \AOc{ref?}. 
It is not impacted by sun lighting conditions and well suited to understand the structure of forests and trees such as measuring, gap fraction, stand density, tree height, DBH, volume or biomass \cite{hackenberg_simpletree_2015, liang_terrestrial_2016, tremblay_automatic_2020}. The spatial resolution of ground-based LiDAR recordings are either expressed in the averaged number points per meter squared, or in the precision of localisation of each point, based on information provided by the authors.
%
The generated LiDAR point clouds have been used for instance segmentation \cite{burt_extracting_2018, tremblay_automatic_2020, grondin_tree_2022}, \textit{i.e.} segment each tree independently and associate them an identification number, or key-point detection, \textit{i.e.} localizing points of interest for each tree. 
%

% specificities and openings 
Ground-based datasets are useful to understand the composition of forests under the tree canopy and recordings were difficult to automatize until recently \cite{calders_strucnet_2023}.
Literature lacks large-scale annotated datasets although they can provide information at high spatial and temporal resolution and from perspectives that aerial and satellite recordings cannot. 
Providing both ground-based and aerial-based recordings \cite{soltani_transfer_2022} informing both above and below tree canopy would facilitate transfer and bridging machine learning applications between different modality scales (for details see Section \ref{sec:perspectives}).
The next section will review aerial recordings datasets.


\begin{table*}[t!]
%\footnotesize
\fontsize{6.5pt}{7.5pt}\selectfont % Font size 
\renewcommand{\arraystretch}{1.5} % Size between lines
\setlength\tabcolsep{5pt} % intercolumn size
\caption{Review of open access ground-based forest datasets}
{\begin{fntable}
\centering
\begin{tabular}{p{1.5cm} | p{0.4cm} | p{1cm} | p{1.3cm} | p{1.2cm} : p{1cm} | p{0.4cm} | p{0.7cm} | p{0.8cm} | p{1cm} | p{1.1cm}}
\toprule
Dataset  & Publi. year & Recording year & Dataset size & Data  & Spatial precision & Time series & Potential task(s) & \#Classes & Location  & License  \\
\midrule

NOU-11 / KARA-001 \cite{burt_extracting_2018} & 2018 & 2015 & 1 ha and 425 trees / \newline 0.25 ha and 40 trees & LiDAR PC & 400k pts-m2 /\newline 20k pts-m2 & No & IS & 1 & Guyana \newline Australia & Unknown \\

\rowcolor{lightgray}
AgRob V18 \cite{reis_forest_2020} & 2020 & 2019 & Unknown & LiDAR PC \newline Stereo RGB \newline Thermal \newline IMU & err +-3cm \newline N/A \newline N/A \newline N/A & No & Reg. & N/A & Portugal & Unknown \\

Montmorency dataset	\cite{tremblay_automatic_2020} & 2020 & Unknown & 1.4 ha \newline 1453 trees & LiDAR PC \newline RGB \newline DBH \newline IMU & Unknown \newline N/A \newline \newline 3.5cm \newline N/A & No & OD \newline IS \newline Reg. & 18 & Canada & Unknown \\

\rowcolor{lightgray}
ForTrunkDetV2 \cite{da_silva_visible_2021, da_silva_unimodal_2021, da_silva_edge_2022} & 2022 & 2021 \newline 2022 & 5325 images & RGB \newline Thermal & N/A \newline N/A & No & OD & 1 & Portugal & CC-BY-4.0 \\

SynthTree43k and CanaTree100 \cite{grondin_tree_2022} & 2022 & 2020 \newline 2021 & 43K synth. \newline 100 real & RGB-Depth & N/A & No & OD \newline IS \newline KD & 17 & Canada & Apache 2.0 \\

\bottomrule
\end{tabular}
\footnotetext[]{{Acronyms}: \textbf{N/A}: non applicable; \textbf{Unknown}: non provided by the authors; \textbf{ha}: hectares; \textbf{PC}: point cloud; \textbf{RGB}: red-green-blue images; \textbf{DBH}: diameter at breast height;  \textbf{IMU}: inertial measurement unit; \textbf{IS}: instance segmentation; \textbf{Reg.}: regression; \textbf{OD}: object detection; \textbf{KD}: key-point detection. Note that the dataset size measured in \textbf{K} are $O(10^3)$.}
\end{fntable}}
\label{tab:ground}
\end{table*}




\subsection{Aerial recordings}
\label{sec:review_aerial}

% Concept / utility
Aerial datasets consist of recordings of sensors mounted on unoccupied (drones) or occupied aircrafts flying above the tree canopy, offering a broader perspective of the forest without the hindrance of obstacles impeding the automatic recording process.
The diversity in aerial datasets has increased in the past few years since they are used for diverse applications such as vegetation segmentation, disease detection, fire detection and numerous others \cite{guimaraes_forestry_2020}. This is in part also boosted as governmental organizations are increasingly making the imagery of repeated official aerial campaigns openly available (\textit{e.g.} for entire countries). Furthermore, the decreasing costs of UAVs and the miniaturization of high-quality sensors have served as strong incentives for their adoption within the community.

% volume
Aerial-based recordings are reviewed in Table \ref{tab:aerial}. The dataset size is expressed in kilometer squared ($\text{km}^2$), or in hectares (ha) if the studied area is small. It is also quantified by the number of samples or number of trees if applicable.

% data and resolution
Multiple sensors can be carried by UAVs,
%UAVs are able to carry a multitude of sensors, 
including RGB and thermal cameras, multispectral sensors, hyperspectral sensors,
%\ELc{hyperspectral sensors are generally pushbroom scanners, not cameras taking 2D snapshot images, although snapshot hyperspectral cameras do exist. I think I would stick with "sensors" here.}
and LiDAR, which collectively contribute to a captivating array of recorded data, offering diverse perspectives and insights.
Cameras mounted on UAVs facilitate the acquisition of overlapping images with a spatial resolution of a few mm to cm.
Such high-resolution image datasets can be applied in concert with photogrammetric workflows, that enable a triangulation of common features found in overlapping images, enabling to precisely reconstruct camera parameters and orientations in hundreds of images automatically. Such workflows enable to reconstruct digital surface models and reprojections of the imagery to generate geocoded image mosaics with orthographic projection \cite{guimaraes_forestry_2020, diez_deep_2021}. 
%
Most of the recently publicly released aerial datasets contain RGB images generated by photogrammetry since they are relatively simple and cheap to collect \cite{morales_automatic_2018, kattenborn_convolutional_2019, kentsch_computer_2020, schiefer_mapping_2020, nguyen_individual_2021, kattenborn_convolutional_2020, galuszynski_automated_2022, reiersen_reforestree_2022}. But the original RGB point cloud carrying the height information used to generate the DSM is generally not provided with some exceptions \cite{brieger_advances_2019, van_geffen_sidroforest_2022}. This is unfortunate because there would be opportunities for new multi-modal models to leverage both the RGB and point cloud modalities to improve model performance.

%Another approach to study topology of the ground and canopies depending on the forest structure is to use a LiDAR sensor.
An alternative method for studying the topography of both the ground and canopies, depending on the structure of the forest, involves the utilization of airborne LiDAR acquisitions \cite{ferraz_carbon_2018, kalinicheva_multi-layer_2022}. 
In contrast to terrestrial LiDAR, these measurements commonly have lower point densities, but cover large areas. 
Airborne LiDAR sensors are operated with IMU sensors which enables geo-referenced flights transects across large spatial extents. 
These sensors typically can record multiple returns per LiDAR pulse so that acquisitions can resemble the vertical structure of forest stands, including multiple overlapping tree layers, the understory and even the ground topography \cite{kalinicheva_multi-layer_2022}.
The spatial resolution of airborne LiDAR products is estimated by the averaged number of points per square meter.

%Multispectral and hyperspectral cameras are passive sensors recordings photons with wavelength bands including and beyond the visible spectra. 
Multispectral and hyperspectral sensors are passive, capturing reflected or emitted photons \cite{mavrovic_reviews_2023} from the sun across wavelength bands that extend beyond the visible spectrum, allowing for comprehensive recording of electromagnetic radiation throughout the near up to the shortwave infrared region.
%They are particularly useful to evaluate the composition of the trees, and thus distinguish different species, depending on the measure of intensity at each spectral bands. 
They are especially valuable in assessing the composition of forest canopies, enabling the differentiation of species or retrieving biochemical and structural properties based on the spectral characteristics  across spectral bands \cite{fassnacht_review_2016, cherif_spectra_2023}.
A trade-off is usually required between acquiring information with a high spectral and a low spatial resolution \cite{paz-kagan_multiscale_2017}, or with a low spectral and a high spatial resolution \cite{garioud_flair_2022}, given that the radiation reflected by plant canopies does not suffice the acquisition at high spectral and high spatial resolution simultaneously.


%tasks
% The tasks that can be explored using aerial datasets rely on the sensors employed and the annotations provided alongside the data.
Forest monitoring can be explored in different ways using aerial datasets relying on the sensors employed and the annotations provided alongside the data.
% As an example, semantic segmentation is the most common approach to estimate tree species since it is well suited to canopy shapes unlike object detection predicting rectangular bounding boxes for each object. 
For instance, semantic segmentation is a prevalent method employed to classify forest canopies into tree species \cite{morales_automatic_2018, kattenborn_convolutional_2019, kentsch_computer_2020, schiefer_mapping_2020, kattenborn_convolutional_2020, galuszynski_automated_2022}. Depending on the canopy structural complexity and data quality, the classification might combined with a delineation of individual tree crowns using instance segmentation approaches. Thereby, instance segmentation captures the intricate shapes of tree crowns, unlike object detection, which typically predicts rectangular bounding boxes or centroids for individual objects \cite{reiersen_reforestree_2022}.
% In this case, each pixel or point will be classified in a category, it is well suited to canopy shapes unlike object detection predicting rectangular bounding boxes for each object. 
% Part of the reviewed datasets provide a DSM which could be used for canopy height estimation when combined with tree localization, this task has not been extensively explored at the aerial scale for now.
Some of the reviewed datasets include a DSM, which can be utilized with tree localization to estimate canopy height. 
This application using deep learning algorithms and aerial data is an actual active field of research \cite{yue_treeunet_2019, moradi_potential_2022, reiersen_reforestree_2022, wagner_sub-meter_2023}.
%However, this task has not been extensively explored using deep learning algorithms and aerial data thus far \cite{reiersen_reforestree_2022, wagner_sub-meter_2023}.


% specificities and openings 
%Datasets composed of aerial recordings are growing thanks to the high resolution of the data making the forest understandable at the tree level. 
Due to the high spatial resolution, datasets of aerial recordings enable a granular understanding of forests at the individual tree level.
There are still many open challenges which could be explored at the tree level such as segmenting individual tree crowns in dense forests, classify them between a wide range of species or adapt algorithms from a forest to another (see Sec. \ref{sec:forest_topics_and_challenges}). 
%However, battery and recording capacities, in particular for drones, are limited to scale at a national forest level. 
%The following section will review satellite datasets which are well suited to have a broader scope of the forest.
Nevertheless, the scale of aerial datasets, especially for drones, is constrained by limited battery life and recording capacities, making it challenging to regularly assess and thus monitor large forest areas. 
Consequently, the next section will explore satellite datasets, which are better suited for capturing a broader scope of forest landscapes at high frequencies.



\begin{table*}[ht]
%\footnotesize
\fontsize{6.5pt}{7.5pt}\selectfont % Font size 
\renewcommand{\arraystretch}{1.5} % Size between lines
\setlength\tabcolsep{5pt} % intercolumn size
\caption{Review of open access aerial forest datasets}
{\begin{fntable}
\centering
\begin{tabular}{p{1.5cm} | p{0.4cm} | p{1cm} | p{1.3cm} | p{1.2cm} : p{1.1cm} | p{0.4cm} | p{0.7cm} | p{0.8cm} | p{1cm} | p{1.1cm}}
\toprule
Dataset  & Publi. year & Recording year & Dataset size & Data  & Spatial resolution & Time series & Potential task(s) & \#Classes & Location  & License  \\
\midrule

Dorot and Negba land use \cite{paz-kagan_multiscale_2017} & 2017 & Unknown & 22.9 km2 & Hyperspectral & 1m & No & Classif. \newline Seg. & 23 & Israel & Unknown \\

\rowcolor{lightgray}
Kalimantan Lidar \cite{ferraz_carbon_2018} & 2018 & 2014 & 1.1K km2 & Lidar PC \newline CHM \newline DTM \newline DSM & 4-10 pts-m2 \newline N/A \newline N/A \newline N/A & No & Reg. & N/A & Indonesia & \href{https://www.earthdata.nasa.gov/learn/use-data/data-use-policy?}{Specific} \\

MauFlex	\cite{morales_automatic_2018} & 2018 & 2015-2018 & 25K samples & RGB & 1.4-2.5cm & Yes & Seg. & 1 & Peru & Unknown \\

\rowcolor{lightgray}
Mueller Glacier	\cite{kattenborn_convolutional_2019}  & 2019 & 2017 & 15.75 ha & RGB & 5cm & No & Seg. & 2 & New Zealand & CC-BY-4.0 \\

Cactus Aerial Photos \cite{lopez-jimenez_columnar_2019}	 & 2019 & Unknown & 21K samples & RGB & Unknown & No & Classif. & 2 & Mexico & GPL 2 \\

\rowcolor{lightgray}
Woody invasive species \cite{kattenborn_uav_2019} & 2019 & 2016/2017 & 151.7 ha & RGB \newline Hyperspectral & 3cm \newline 10cm & No & Seg. & 3 & Chile & CC-BY-4.0 \\

YURF and Shonai Coastal Forest \cite{kentsch_computer_2020} & 2020 & 2018 \newline 2019 & 2800 samples & RGB & 2.79cm to 4.48cm & Yes & MC \newline Seg. & 9 & Japan & CC-BY-4.0 \\

\rowcolor{lightgray}
TreeSeg	\cite{schiefer_mapping_2020} & 2020 & 2017-2019 & 51 ha & RGB & $\leq 2$cm & No & Seg. & 14 & Germany & CC-BY-4.0 \\


Zao Mountain \cite{nguyen_individual_2021} & 2021 & 2019 & 18 ha \newline 5354 trees & RGB & 1.5 to 2.1cm & No & OL \newline Classif. \newline Seg. & 3 & Japan & CC-BY-4.0 \\

\rowcolor{lightgray}
New Zealand primary forest \cite{kattenborn_convolutional_2020}	 & 2022 & 2017 & 4.3 ha & RGB \newline DBH \newline DSM & 3cm \newline N/A \newline N/A & No & Seg. \newline Reg. & 2 & New Zealand	 & CC-BY-4.0 \\

Portulacaria afra canopies \cite{galuszynski_automated_2022} & 2022 & 2020 \newline 2021 & 75 ha & RGB & 1cm & No & Seg. & 1 & South Africa & CC-BY-4.0 \\

\rowcolor{lightgray}
FLAIR\#1 \cite{garioud_flair_2022} & 2022 & 2019-2021 & 810 km2 \newline 77K samples & Multispectral \newline Panchromatic \newline Elevation \newline spat. DTM \newline vert. DTM & 0.2m \newline 0.2m \newline 0.4m \newline 1m \newline 0.3-7m & Yes & Seg. & 19 & France & Open Licence 2.0 \\

WildForest3D \cite{kalinicheva_multi-layer_2022} & 2022 & Unknown & 4.7 ha \newline 2000 trees & Lidar PC & 60 pts-m2 & No & Seg. & 6 & France & Unknown \\

\bottomrule

\end{tabular}
\footnotetext[]{{Acronyms}: \textbf{N/A}: non applicable; \textbf{Unknown}: non provided by the authors; \textbf{ha}: hectares; \textbf{PC}: point cloud; \textbf{CHM}: canopy height model; \textbf{DTM}: digital terrain model (spatial or vertical); \textbf{DSM}: digital surface model; \textbf{RGB}: red-green-blue images; \textbf{DBH}: diameter at breast height; \textbf{Classif.}: classification; \textbf{Seg.}: semantic segmentation; \textbf{Reg.}: regression; \textbf{MC}: multi-classification; \textbf{OD}: object detection. Note that the dataset size measured in \textbf{K} are $O(10^3)$.}
\end{fntable}}
\label{tab:aerial}
\end{table*}









\subsection{Satellite recordings}
\label{sec:review_satellite}

% Concept / utility
% A few words about the satellite missions

% bSatellite imagery is commonly used for earth observation \AOc{details / refs?}. 
% Satellite imagery are recorded all around the world for decades opening research % on temporal earth observation at different resolutions with various sensors.
Satellite imagery has been consistently recorded across the globe for many years, enabling extensive research in the field of temporal remote sensing.
This abundance of data have opened up research in machine learning applied to earth observation, in particular deep learning approaches \cite{campsvalls_deep_2021}, in the past few years.
The datasets generated by diverse satellite missions encompass a wide range of resolutions and employ various sensors, enabling studies of diverse phenomena over both space and time \cite{swain_spatio-temporal_2023}.

% Historically the Landsat missions have used multispectral camera to monitor land use and land cover at large scale. 
% The Landsat missions\footnote{\url{https://www.usgs.gov/landsat-missions/landsat-satellite-missions}} are joint effort between the U.S. Geological Survey (USGS), U.S. Department of the Interior (DOI), National Aeronautics and Space Administration (NASA), and the U.S. Department of Agriculture (USDA) as the oldest and the first attempt to use multispectral cameras for earth observation.
The Landsat missions\footnote{\url{https://www.usgs.gov/landsat-missions/landsat-satellite-missions}}, a collaborative endeavor started in the seventies involving the U.S. Geological Survey (USGS), U.S. Department of the Interior (DOI), National Aeronautics and Space Administration (NASA), and the U.S. Department of Agriculture (USDA), represent the earliest and pioneering attempt to utilize multispectral cameras for Earth observation \cite{wulder_fifty_2022}.
Landsat missions 4 and 5 capture images with between 4 and 7 spectral bands, offering spatial resolutions ranging from 30 to 120 meters. %The more recent Landsat missions 7 and 8 have 8 and 9 spectral bands respectively, ranging from 15 to 60 meters, and from 15 to 30 meters spatial resolution respectively. 
The more recent Landsat missions, namely Landsat 7 and Landsat 8, are record images with 8 and 9 spectral bands respectively. These missions provide spatial resolutions ranging from 15 to 60 meters for Landsat 7 and 15 to 30 meters for Landsat 8.
All of the Landsat missions have a 16-day repeat cycle.
Most of the reviewed datasets used 30 meters resolution spectral bands to ensure a consistency between the bands used for their final application \cite{robinson_large_2019, potapov_annual_2019, irvin_forestnet_2020, de_almeida_pereira_active_2021, feng_m_arctic-boreal_2022, potapov_global_2022, lee_multiearth_2022}.


The Sentinel missions\footnote{\url{https://sentinel.esa.int/web/sentinel/missions}}, managed by the European Space Agency (ESA), have been designed to comprehensively monitor the Earth's various domains, encompassing air, land, ocean, and atmospheric measurements. These missions employ multiple sensors, enabling a wide range of Earth observation capabilities.
% The Sentinel missions\footnote{\url{https://sentinel.esa.int/web/sentinel/missions}} ran by the European Space Agency (ESA) aimed to monitor earth with multiple sensors, from air, land and ocean monitoring to atmospheric measurements. 
Sentinel-1 includes a SAR generating electromagnetic waves with wavelengths not impacted by clouds. The reviewed datasets provide or use Level-1 Ground Range Detected (GRD) products at a $10 \times 10$ meters resolution \cite{schmitt_sen12ms_2019, sumbul_bigearthnet-mm_2021, lee_multiearth_2022}.
Since two satellites (Sentinel-1A and Sentinel-1B) are recording data on the same orbit, the mission has a 6 days exact repeat cycle with less than a day of revisit frequency at high latitudes.

Sentinel-2 utilizes multispectral sensors to scan photon reflectance across multiple spectral bands.
The spatial resolution of the recorded data depends on the spectral bands: four bands at 10 meters, six bands at 20 meters and three bands at 60 meters. Released datasets either kept 10m resolution bands \cite{schmitt_sen12ms_2019, bastani_satlas_2022, lee_multiearth_2022} or all bands \cite{sumbul_bigearthnet-mm_2021}.
The revisit frequency of the combined constellation of Sentinel-2A and B is 5 days on most of the globe.

Data from the Landsat and Sentinel missions are the most commonly provided in the reviewed datasets, but other interesting satellite sources are also explored. For instance, the Moderate Resolution Imaging Spectroradiometer (MODIS)\footnote{\url{https://modis.gsfc.nasa.gov}} instrument, introduced by NASA and integrated into the Terra and Aqua missions, generates data that are also used for large-scale forest monitoring purposes.
% instrument introduced by NASA and part of the Terra and Acqua missions has also generates data used for forest monitoring. 
The MODIS instrument offers recordings from 36 spectral bands, each defined for diverse observations, including atmospheric gases, ocean components, land boundaries, and properties \cite{schmitt_sen12ms_2019, levin_unveiling_2021}.
% It provides recordings from 36 spectral bands defined for various observations such as atmospheric gaze, ocean components or land boundaries and properties \cite{schmitt_sen12ms_2019, levin_unveiling_2021}.  
%
Another example is the Visible Infrared Imaging Radiometer Suite (VIIRS) instrument\footnote{\url{https://www.earthdata.nasa.gov/learn/find-data/near-real-time/viirs}}, part of the NOAA-20 missions, which also have generated data contained in a forest dataset for land and atmospheric observations \cite{levin_unveiling_2021}.
%
%Note that some of the authors have used recordings from DigitalGlobe missions (GeoEye, WorldView or QuickBird) providing multispectral data with spatial resolution below the meter \cite{brandt_unexpectedly_2020}. Unfortunately, these data are not accessible to the public due to the associated license.
It should be noted that researchers have used recordings from PlanetLabs\footnote{\url{https://www.planet.com/}}, PlanetScope\footnote{\url{https://earth.esa.int/eogateway/missions/planetscope}} or Maxar\footnote{\url{https://www.maxar.com/}} missions (\textit{e.g.} GeoEye, WorldView, or QuickBird), which provide multispectral data with sub-meter spatial resolution \cite{brandt_unexpectedly_2020}. However, these datasets are not publicly accessible due to the associated licensing restrictions.


% Missions explored: DigitalGlobe satellites (GeoEye 1 WorldView 2 World View 3 and QuickBird 2) but not open source, LandsatSPOT, Landsat 4/5/7/8, NASA MODIS C6, Visible Infrared Imaging Radiometer Suite (VIIRS) \url{https://www.earthdata.nasa.gov/learn/find-data/near-real-time/viirs}
%Climate data from PRISM Climate Group \url{https://prism.oregonstate.edu/}
% difference temporal coverate?
% difference resolution
% other missions??

%The reviewed datasets contain satellite data from different missions and products, at different locations with different spatial and temporal resolutions. 
The datasets that have been reviewed encompass satellite data obtained from various missions and products, originating from different locations, and exhibiting diverse spatial and temporal resolutions.
The details of datasets published before 2020 and included, or after 2020, are provided respectively in Table \ref{tab:satellite1} and Table \ref{tab:satellite2}.
% volume
The dataset size is expressed in kilometer squared ($\text{km}^2$), or in hectares (ha) if the studied area is small. It is also quantified by the number of samples, trees or events if applicable.

%tasks
Satellite datasets are frequently used for classification, multi-classification or segmentation of satellite tiles, including LULC and tree species distribution.
Other tasks include regression applications for forest cover estimation \cite{bastani_satlas_2022, feng_m_arctic-boreal_2022}, canopy height \cite{forkuor_above-ground_2020, lang_high-resolution_2022}, or live fuel moisture content estimation \cite{rao_sar-enhanced_2020}.
% An interesting application is to evaluate change detection of forest covers at large scale using satellite time series to estimate deforestation, afforestation and reforestation \cite{potapov_global_2022}.
An intriguing application involves utilizing satellite time series data to evaluate change detection of forest covers at a large scale \cite{wang_extensive_2020}. This approach enables the estimation of deforestation, afforestation, and reforestation activities \cite{potapov_global_2022}.

% specificities and openings 
%Satellite recordings are helpful for Earth monitoring at large scale, they are manually or automatically processed to estimate world maps of forest cover per example.
%World maps of above ground biomass, land use and land cover have also been estimated and publicly released. 
%In the next section, the maps datasets at the country or world level will be reviewed.
Satellite recordings play a crucial role in Earth observation on a large scale, as they are manually or automatically processed to estimate global maps of forest cover, among other applications. Additionally, world maps depicting above-ground biomass, land use, and land cover have been estimated and made publicly available. In the following section, datasets containing maps at the country or global level will be reviewed.



\begin{table*}[ht]
%\footnotesize
\fontsize{6.5pt}{7.5pt}\selectfont % Font size 
\renewcommand{\arraystretch}{1.5} % Size between lines
\setlength\tabcolsep{5pt} % intercolumn size
\caption{Review of open access satellite forest datasets before 2020 (included)}
{\begin{fntable}
\centering
\begin{tabular}{p{1.5cm} | p{0.4cm} | p{1cm} | p{1.3cm} | p{1.2cm} : p{1cm} | p{0.4cm} | p{0.7cm} | p{0.8cm} | p{1cm} | p{1.1cm}}
\toprule
Dataset  & Publi. year & Recording year & Dataset size & Data  & Spatial resolution & Time series & Potential Task(s) & \#Classes & Location  & License  \\
\midrule

GlobCover 2009	\cite{arino_olivier_global_2010} & 2010 & 2009 & 2485 samples \newline ($5^{\circ} \times 5^{\circ}$) & Multispectral & 300m & No & Classif. & 22 & Worldwide & \href{http://due.esrin.esa.int/page_globcover.php}{Specific} \\

\rowcolor{lightgray}
Spatial Database of Planted Trees \cite{harris_spatial_2019} & 2019 & 2015 & 223M ha & Multispectral & $\leq$ 30m & No & Classif. & 4 & Worldwide & CC-BY-4.0 \\

Lower Mekong \cite{potapov_annual_2019} & 2019 & 2000-2017 & 112M ha \newline 56K samples & Multispectral & 30m & Yes & Classif. \newline Reg. & 2 & Myanmar \newline Laos \newline Thailand \newline Cambodia \newline Vietnam & Unknown \\

\rowcolor{lightgray}
Chesapeake Land Cover \cite{robinson_large_2019} & 2019 & 2011-2017 & 160km2 & Multispectral & 30m & Yes & Seg. & 4 and 16 & USA & \href{https://www.usgs.gov/information-policies-and-instructions/copyrights-and-credits}{Specific} \newline \href{https://opendatacommons.org/licenses/odbl/}{ODbL} \\

SEN12MS	\cite{schmitt_sen12ms_2019} & 2019 & 2016-2017 & 542K samples & SAR \newline Multispectral \newline LULC maps & 10m \newline 10m \newline 10m & No & Classif. & 33 & Worldwide & CC-BY-4.0 \\

\rowcolor{lightgray}
Non-forest trees \cite{brandt_unexpectedly_2020}  & 2020 & 2005-2018 & 50K samples \newline 90K trees & Multispectral & 0.5m & No & Seg. & 1 & W. Sahara \newline Mauritania \newline Senegal \newline Gambia \newline Guinea-Bissau \newline Mali & \href{https://www.earthdata.nasa.gov/learn/use-data/data-use-policy?}{Specific} \\

ForestNet \cite{irvin_forestnet_2020} & 2020 & 2012-2020 & 76K samples \newline 2.8K events & Multispectral \newline Topography \newline Climate \newline Soil \newline Accessibility \newline Proximity & 15m/30m \newline 30m \newline 56km \newline N/A \newline N/A \newline N/A & Yes & Seg. & 4 & Indonesia & CC-BY-4.0 \\

\rowcolor{lightgray}
Live Fuel Moisture Content \cite{rao_sar-enhanced_2020} & 2020 & 2015-2019 & 164M km2 & LMFC \newline Variables \newline Location \newline Topology \newline CHM \newline LULC maps & 250m \newline 250m \newline 250m \newline 250m \newline 250m \newline 250m & Yes & Classif. \newline Reg. & 6 & USA & CC-BY-NC-ND-4.0 \\


\bottomrule

\end{tabular}
\footnotetext[]{{Acronyms}: \textbf{N/A}: non applicable; \textbf{ha}: hectares; \textbf{SAR}: synthetic-aperture RADAR; \textbf{LULC}: land use and/or land cover; \textbf{CHM}: canopy height model; \textbf{LMFC}: live fuel moisture content; \textbf{Classif.}: classification; \textbf{Seg.}: semantic segmentation; \textbf{Reg.}: regression.
Note that the dataset size measured in \textbf{K} are $O(10^3)$ and in \textbf{M} are $O(10^6)$.
}
\end{fntable}}
\label{tab:satellite1}
\end{table*}






\begin{table*}[ht]
%\footnotesize
\fontsize{6.5pt}{7.5pt}\selectfont % Font size 
\renewcommand{\arraystretch}{1.5} % Size between lines
\setlength\tabcolsep{5pt} % intercolumn size
\caption{Review of satellite recording datasets after 2021 (included)}
{\begin{fntable}
\centering
\begin{tabular}{p{1.5cm} | p{0.4cm} | p{1cm} | p{1.3cm} | p{1.2cm} : p{1cm} | p{0.4cm} | p{0.7cm} | p{0.8cm} | p{1cm} | p{1.1cm}}
\toprule
Dataset  & Publi. year & Recording year & Dataset size & Data  & Spatial resolution & Time series & Potential Task(s) & \#Classes & Location  & License  \\
\midrule

Australian Black Summer	\cite{levin_unveiling_2021} & 2021 & 2001-2021 & 265K km2 \newline 391 fires & Locations \newline Variables & 500m \newline 500m & No & Seg. & 3 & Australia	 & CC-BY-4.0 \\

\rowcolor{lightgray}
Active fire detection \cite{de_almeida_pereira_active_2021} & 2021 & 2020 & 150K samples & Multispectral & 30m & No & Seg. & 1 & Worldwide & Unknown \\

BigEarthNet-MM	\cite{sumbul_bigearthnet-mm_2021} & 2021 & 2017 \newline 2018 & 590K samples ($\times 2$) & SAR \newline Multispectral	& 10m \newline 10m to 60m & No & MC & 19 & 10 countries (Europe)$\star$& \href{https://bigearth.net/downloads/documents/License.pdf}{Specific} \\

\rowcolor{lightgray}
Satlas \cite{bastani_satlas_2022} & 2022 & 2016-2021 & 85M km2 \newline 5M megapixels & Multispectral & 10m / 1m & Yes & Seg. \newline Classif. \newline Reg. & 137 & Worldwide & Apache License 2.0 \\

GLAD Global Land Cover and Land Use change	\cite{potapov_global_2022} & 2022 & 2000-2020 & 504 samples \newline ($10^{\circ} \times 10^{\circ}$) & Multispectral \newline LULC maps & 30m \newline $1 \times 1^{\circ}$ & Yes & Classif. \newline Reg. \newline CD & 5 & Worldwide & CC-BY-4.0 \\

\rowcolor{lightgray}
ABoVE \cite{feng_m_arctic-boreal_2022} & 2022 & 1984-2020 & 224K samples & Multispectral & 30m & Yes & Reg. & N/A & Worldwide & \href{https://www.earthdata.nasa.gov/learn/use-data/data-use-policy?}{Specific} \\

MultiEarth 2022 Deforestation Challenge	\cite{lee_multiearth_2022} & 2022 & 2014-2021 \newline 2018-2021 \newline 1984-2012 \newline 2013-2021 & 12M samples & SAR \newline Multispectral & 10m \newline 10m / 30m & Yes & Seg. & 1 & Brazil & CC-BY-4.0 \\

\rowcolor{lightgray}
EarthNet2021x \cite{robin_learning_2022} & 2022 & 2016-2021 & Unknown & NDVI map \newline Multispectral \newline DEM \newline LULC map \newline Geomorpho. \newline Meterology \newline Land surface soil moisture & 30m \newline 30m \newline 30m \newline 30m \newline 90m \newline Unknown \newline Unknown & Yes & Reg. & N/A & Africa	 & CC-BY-NC-S-4.0 \\

\bottomrule

\end{tabular}
\footnotetext[]{{$\star$}: The list of countries is detailed in the OpenForest catalogue.}
\footnotetext[]{{Acronyms}: \textbf{N/A}: non applicable; \textbf{Unknown}: non provided by the authors; \textbf{ha}: hectares; \textbf{SAR}: synthetic-aperture RADAR; \textbf{LULC}: land use and/or land cover; \textbf{NDVI}: normalized difference vegetation index; \textbf{Classif.}: classification; \textbf{Seg.}: semantic segmentation; \textbf{Reg.}: regression; \textbf{MC}: multi-classification; \textbf{CD}: change detection.
Note that the dataset size measured in \textbf{K} are $O(10^3)$ and in \textbf{M} are $O(10^6)$.}
\end{fntable}}
\label{tab:satellite2}
\end{table*}







\subsection{Country or world maps}
\label{sec:review_maps}

% Concept / utility
% Mention that they are estimated, sometimes using ML algos, and validated with inventories, which are not provided here, or not validated

% Earth observation using remote sensing has been explored to monitor forests at a global scale.
% Earth observation applications have been summarized in country, continent or world maps estimated using machine learning algorithms with manual expert features from satellite data, from multispectral \cite{friedl_modis_2010, pflugmacher_mapping_2019} to climatic and elevation \cite{chaves_mapping_2020}. 
Earth observation applications have been resumed into maps at the country, continent, or global level. They are estimated using machine learning algorithms that incorporate manual expert features derived from satellite data such as statistics of the data distribution or vegetation indexes. These features encompass various aspects, ranging from multispectral information \cite{friedl_modis_2010, pflugmacher_mapping_2019} to climatic and elevation data \cite{chaves_mapping_2020}.
%
%
%
The majority of the released maps have been estimated using machine learning algorithms trained on satellite data, as these algorithms demonstrate excellent scalability for predicting at a large scale and low resolution.
The results obtained from these algorithms have been validated using field inventories. However, it is important to note that the field inventories themselves have not been included in the reviewed datasets discussed in this section\footnote{Open access datasets releasing both maps and inventories are reviewed in Section \ref{sec:review_mixed}, Table \ref{tab:mixed1}.}. 
Nonetheless, the reviewed map datasets are notable for their global coverage, which adds to their significance.
%, they are scaled at the country, continent or world level. 
Despite containing inherent uncertainties in their estimations, these maps have the potential to offer valuable meta-knowledge to downstream applications in the realm of forest monitoring.


% data and resolution
% Researchers have exploited satellite data to create maps of forest cover, above ground biomass, height canopy maps and so on. 
Large scale maps datasets before 2019 (included) are reviewed in Table \ref{tab:maps1} while maps datasets released after 2020 (included) are reviewed in Table \ref{tab:maps2}.
% volume
The dataset size is expressed in kilometer squared ($\text{km}^2$), or in hectares (ha) if the studied area is small. It is also quantified by the number of samples or points if applicable.





%tasks
% Classif or segmentation??
% The significant part of map datasets provide land use and land cover (see Sec. \ref{sec:review} for the proposed definition) which are used to differentiate type of forests at large scale \cite{bartholome_glc2000_2005, friedl_modis_2010, griffiths_forest_2014, pflugmacher_mapping_2019, thonfeld_long-term_2020, bonannella_forest_2022}.
A significant portion of map datasets focuses on providing information about LULC (see Sec. \ref{sec:review} for the proposed definition), which plays a crucial role in distinguishing different types of forests at a large scale \cite{bartholome_glc2000_2005, friedl_modis_2010, griffiths_forest_2014, pflugmacher_mapping_2019, thonfeld_long-term_2020, bonannella_forest_2022}.
%
Within in LULC maps, the reviewed works have also estimated the extend of forest cover, including time series data.
These time series are particularly valuable for quantifying forest loss, \textit{i.e.} deforestation detection, as well as forest gain, \textit{i.e.} afforestation, reforestation monitoring or deadwood maps \cite{hansen_high-resolution_2013, curtis_classifying_2018, verhegghen_mapping_2022, bunting_global_2022, schiefer_uav-based_2023}.
%
Another category of maps is specifically designed to differentiate between different plant functional types, particularly distinguishing between broad-leaved and needle-leaf forests, as well as identifying summer-green and evergreen forests across tropical, boreal, and temperate regions \cite{ottle_use_2013}.

% Another type of maps is focused on differentiating plant functional types in particular between broad-leaved and needle-leaf, summer-green and evergreen from tropical, boreal or temperate forests \cite{ottle_use_2013}.


As mentioned in Section \ref{sec:forest_challenges}, accurate estimation of above-ground biomass is crucial for a comprehensive quantification of the carbon stocks that forests worldwide hold.
% the above ground biomass estimation is crucial to better quantify the carbon stocks that world forests represent. 
%
World maps of above ground biomass have been estimated at different resolutions \cite{patterson_statistical_2019, tang_high-resolution_2021, ma_high-resolution_2021, santoro_global_2021}.
%
%Uncertainty of above ground biomass maps is also important to quantify since it relies on many factors such as the species, the height of the trees and the size of the canopy \cite{patterson_statistical_2019, santoro_global_2021}.
Quantifying the uncertainty of above-ground biomass maps is also important as it depends on multiple factors, including tree species, tree height, canopy size, and reference data distribution \cite{patterson_statistical_2019, santoro_global_2021, ploton2020spatial}.
%
Canopy height maps have also been quantified in sparse boreal forests
%\ELc{arctic forests? you mean boreal forests? Arctic = tundra, for the most part.}
\cite{bartsch_land_2016, bartsch_feasibility_2020}.
These canopy height maps, both at the country \cite{tolan_sub-meter_2023} or world level \cite{lang_global_2022, lang_high-resolution_2022}, have been estimated using LiDAR sensors as a ground truth.
%
%The canopy height estimation is crucial to evaluate the above-ground biomass, this is why a few works have used the GEDI mission, recoding LiDAR data from the International Space Station, to estimate a DSM as a reference.
Accurate estimation of canopy height is crucial for evaluating above-ground biomass, which is why a few studies have utilized data from the Global Ecosystem Dynamics Investigation (GEDI) mission\footnote{\url{https://gedi.umd.edu/}} \cite{patterson_statistical_2019, tang_high-resolution_2021, ma_high-resolution_2021, lang_global_2022, tolan_sub-meter_2023}. 
GEDI records LiDAR data from the International Space Station, allowing for the estimation of a DSM that serves as a valuable reference for canopy height estimation.
%
The overall biomass estimation of forests also includes below ground biomass \cite{chen_maps_2023} and soil carbon stock \cite{dionizio_carbon_2020} which have been quantified using SAR satellite data penetrating dense canopies.


% specificities and openings 
%Even if the open source map datasets are estimated, and thus bounded by the error and uncertainty of the method used by the authors, they are still an interesting source of data for having a large scale knowledge of forests or to integrate a meta knowledge in future analyses.
Although open access map datasets are subject to the limitations and uncertainties inherent in the estimation methods employed by the authors, they remain a valuable source of data for obtaining a broad-scale understanding of forests or integrating meta-knowledge into future analyses. These datasets could be helpful for conducting further research and expanding our knowledge of forest ecosystems.

%In the previous sections, the reviewed datasets have been presented at different scales independently including maps. In the following section, datasets providing data mixed at different scales will be detailed.
In the preceding sections, the reviewed datasets were presented with a focus on different scales. However, in the forthcoming section, datasets that offer a combination of data at various scales will be discussed in detail.


\begin{table*}[ht]
%\footnotesize
\fontsize{6.5pt}{7.5pt}\selectfont % Font size 
\renewcommand{\arraystretch}{1.5} % Size between lines
\setlength\tabcolsep{5pt} % intercolumn size
\caption{Review of open access map forest datasets before 2019 (included)}
{\begin{fntable}
\centering
\begin{tabular}{p{1.5cm} | p{0.4cm} | p{1cm} | p{1.3cm} | p{1.2cm} : p{1cm} | p{0.4cm} | p{0.7cm} | p{0.8cm} | p{1cm} | p{1.1cm}}
\toprule
Dataset  & Publi. year & Recording year & Dataset size & Data  & Spatial resolution & Time series & Potential task(s) & \#Classes & Location  & License  \\
\midrule

GCL2000	\cite{bartholome_glc2000_2005} & 2005 & 1999-2000 & 148M km2 & LULC maps & $1/112^{\circ}$ & Yes & Classif. \newline Seg. & 22 & Worldwide & CC-BY-4.0 \\

\rowcolor{lightgray}
MODIS MCD12Q1 \cite{friedl_modis_2010} & 2010 & 2001-2023 & Unknown & LULC maps	& 500m & Yes & Classif. \newline Seg. & 12 & Worldwide & \href{https://lpdaac.usgs.gov/data/data-citation-and-policies/}{Specific} \\

Carpathian ecoregion \cite{griffiths_forest_2014} & 2010 & from 1985 to 2010 $\blacklozenge$ & 390K km2	& Disturbance \newline LULC maps & 900m & Yes & Classif. \newline Seg. & 10 & Czechia \newline Austria \newline Poland \newline Hungary \newline Ukraine \newline Romania \newline Slovakia & CC-BY-3.0 \\

\rowcolor{lightgray}
Forest Cover Change	\cite{hansen_high-resolution_2013} & 2013 & 2000-2021 & 504 samples \newline  ($10^{\circ} \times 10^{\circ}$) & LULC maps & 30m & Yes & Reg. & N/A & Worldwide & CC-BY-4.0 \\

Siberian plant functional type \cite{ottle_use_2013} & 2013 & 2000-2010 & 5M km2 & PFT maps & 1km & No & Classif. \newline Seg. & 16 & Russia & CC-BY-3.0 \\

\rowcolor{lightgray}
Intact Forest Landscape	\cite{potapov_last_2017} & 2017 & 2000 \newline 2013 \newline 2016 \newline 2020 & 58M km2 & IFL maps & 10km & Yes & Classif. \newline Seg. & 4 & Worldwide & CC-BY-4.0 \\

Global Forest Loss	\cite{curtis_classifying_2018} & 2018 & 2000-2015 & 65K km2 & LULC maps & 10km & Yes & Classif. \newline Seg. & 5 & Worldwide & Unknown \\

\rowcolor{lightgray}
Hybrid estimators of AGB \cite{patterson_statistical_2019} & 2019 & 2019-2021 & 10 maps & AGB maps & 1km & Yes & Reg. & N/A & Worldwide & \href{https://www.earthdata.nasa.gov/learn/use-data/data-use-policy?}{Specific} \\


Pan-Europe land cover \cite{pflugmacher_mapping_2019} & 2019 & 2014-2016 & Unknown & LULC maps & 30m & No & Classif. \newline Seg. & 12 & 28 countries (Europe)$\star$ & CC-BY-SA-4.0 \\

\bottomrule

\end{tabular}
\footnotetext[]{{$\blacklozenge$}: The list of recording years is detailed in the OpenForest catalogue.}
\footnotetext[]{{$\star$}: The list of countries is detailed in the OpenForest catalogue.}
\footnotetext[]{{Acronyms}: \textbf{N/A}: non applicable; \textbf{Unknown}: non provided by the authors; \textbf{LULC}: land use and/or land cover; \textbf{PFT}: plant functional type; \textbf{IFL}: intact forest landscape; \textbf{AGB}: above ground biomass; \textbf{Classif.}: classification; \textbf{Seg.}: semantic segmentation; \textbf{Reg.}: regression.
Note that the dataset size measured in \textbf{K} are $O(10^3)$ and in \textbf{M} are $O(10^6)$.}
\end{fntable}}
\label{tab:maps1}
\end{table*}


\begin{table*}[ht]
%\footnotesize
\fontsize{6.5pt}{7.5pt}\selectfont % Font size 
\renewcommand{\arraystretch}{1.5} % Size between lines
\setlength\tabcolsep{5pt} % intercolumn size
\caption{Review of open access map forest datasets after 2020 (included)}
{\begin{fntable}
\centering
\begin{tabular}{p{1.5cm} | p{0.4cm} | p{1cm} | p{1.3cm} | p{1.2cm} : p{1cm} | p{0.4cm} | p{0.7cm} | p{0.8cm} | p{1cm} | p{1.1cm}}
\toprule
Dataset  & Publi. year & Recording year & Dataset size & Data  & Spatial resolution & Time series & Potential task(s) & \#Classes & Location  & License  \\
\midrule

Peruvian Amazonia floristic patterns \cite{chaves_mapping_2020} & 2020 & 2013-2018 & 790K km2 & Floristic maps & 450m & No & MC & 10 & Peru & CC-BY-4.0 \\

\rowcolor{lightgray}
Florida mangrove resilience \cite{lagomasino_storm_2020} & 2020 & 2017 & 1.3 km2 & Mangrove resilience maps  & 30m & No & Classif. \newline Seg. & 3 & USA & CC-BY-4.0 \\

Cerrado biome \cite{dionizio_carbon_2020} & 2020 & 1990-2018 & 131K km2 & AGB maps \newline BGB maps \newline SCS maps & 30m \newline 30m \newline 30m & Yes & Reg. & N/A & Brazil & CC-BY-4.0 \\

\rowcolor{lightgray}
Arctic trees height	\cite{bartsch_land_2016, bartsch_feasibility_2020} & 2020 & from 2005 to 2018 $\blacklozenge$ & Unknown & CH maps & 20m & Yes & Reg. & N/A & Russia \newline  USA \newline Canada & CC-BY-4.0 \\

Kilombero Valley land cover	\cite{thonfeld_long-term_2020} & 2020 & 1974 \newline 1994 \newline 2004 \newline 2014 \newline 2015 & 40.2km2 \newline 32.7K points & LULC maps \newline LULC points & 30m \newline 1cm & Yes & Classif. \newline Seg. & 11 & Tanzania & CC-BY-4.0 \\

\rowcolor{lightgray}
Pleroma trees \cite{wagner_flowering_2021} & 2021 & 2016-2020 & 2M km2 & LULC maps & 1.28km2 & Yes & Classif. & 1 & Brazil & CC-BY-4.0 \\

Global Forest AGB 2010 \cite{santoro_global_2021} & 2021 & 2010 & 58 samples \newline ($40^{\circ} \times 40^{\circ}$) & GSV maps \newline AGB maps & 1ha \newline 1ha & No & Reg. & N/A & Worldwide & CC-BY-4.0 \\

\rowcolor{lightgray}
New England AGB	\cite{tang_high-resolution_2021, ma_high-resolution_2021} & 2021 & 2010-2015 & 187K km2 & AGB maps \newline CH maps \newline LULC maps & 30m \newline 30m \newline 30m & No & Reg. & N/A & USA & \href{https://www.earthdata.nasa.gov/learn/use-data/data-use-policy?}{Specific} \\

Dry tropical forest cover \cite{verhegghen_mapping_2022} & 2022 & 2018 & 953.5K km2 & LULC maps & 10m & No & Classif. \newline Seg. & 9 & Tanzania & CC-BY-4.0 \\

\rowcolor{lightgray}
Global Mangrove Watch (v3.0) \cite{bunting_global_2022} & 2022 & from 1996 to 2020 $\blacklozenge$ & 152K km2 \newline 17K points  & LULC maps & 25m & Yes & CD & 4 & Worldwide & CC-BY-4.0 \\

European Trees \cite{bonannella_forest_2022} & 2022 & from 2000 to 2020 $\blacklozenge$  & 4.4M points & LULC maps & 30m & Yes & Classif. \newline Seg. & 16 & Europe & CC-BY-4.0 \\

\rowcolor{lightgray}
Global canopy height \cite{lang_high-resolution_2022, lang_global_2022} & 2022 & 2019/2020 & 600M samples & CH map \newline LULC map & 10m \newline 10m & No & Reg. & N/A & Worldwide & CC-BY-4.0 \\

AGBC and BGBC of China \cite{chen_maps_2023} & 2023 & 2002-2021 & 9.6M km2 & AGB maps \newline BGB maps & 1km \newline 1km & Yes & Reg. & N/A & China & CC-BY-4.0 \\

\rowcolor{lightgray}
High-resolution canopy height map \cite{tolan_sub-meter_2023} & 2023 & 2017-2020 & Unknown & CH maps & 0.5m & No & Reg. & N/A & USA \newline Brazil & CC-BY-4.0 \\

Deadwood cover \cite{schiefer_uav-based_2023} & 2023 & 2018 \newline 2019 \newline 2020 \newline 2021 & 727.33ha & Deadwood map & 10m & No & Classif. & 1 & Germany \newline Finland & CC-BY-NC 4.0 \\

\bottomrule

\end{tabular}
\footnotetext[]{{$\blacklozenge$}: The list of recording years is detailed in the OpenForest catalogue.}
\footnotetext[]{{Acronyms}: \textbf{N/A}: non applicable; \textbf{Unknown}: non provided by the authors; \textbf{AGB}: above ground biomass; \textbf{BGB}: below ground biomass; \textbf{SCS}: soil carbon stock; \textbf{LULC}: land use and/or land cover; \textbf{GSV}: growing stock volume; \textbf{CH}: canopy height; \textbf{Classif.}: classification; \textbf{Seg.}: semantic segmentation; \textbf{Reg.}: regression; \textbf{MC}: multi-classification; \textbf{CD}: change detection.
Note that the dataset size measured in \textbf{K} are $O(10^3)$ and in \textbf{M} are $O(10^6)$.}
\end{fntable}}
\label{tab:maps2}
\end{table*}





\subsection{Datasets mixed at different scales}
\label{sec:review_mixed}

% Concept / utility
% Datasets providing data at different scales are important to create a bridge between modalities recorded by different sensors. 
Datasets that offer data at various scales play an important role in establishing a bridge between different modalities recorded by diverse sensors. 
By integrating information from multiple sources, these datasets facilitate a comprehensive understanding of forests and enable cross-modal analysis.
%
%Inventories, ground-based and aerial-based datasets are available at small scale but usually come alongside precise annotations at all tree levels.
%Satellite and maps datasets are available at large scale, however they are difficult to annotate precisely due to their poor precision. 
%Mixing data at different scales should be helpful to generalize local knowledge at a larger scale.
%
Inventories, ground-based and aerial-based datasets are available at small scale but usually come alongside precise annotations at all tree levels.
Conversely, satellite and maps datasets are available at a larger scale but often lack precise annotations due to their lower resolution. 
Integrating data from different scales can be advantageous in generalizing and extrapolating local knowledge to a larger scale, bridging the gap between detailed annotations and broader coverage \cite{kattenborn_uav_2019, schiefer_uav-based_2023}.


% volume
% As presented in the previous sections, 
The size of each dataset is expressed in kilometer squared ($\text{km}^2$), or in hectares (ha) if the studied area is small. It is also quantified by the number of samples, points or trees if applicable.

% data and resolution
%tasks
Mixed datasets composed of inventories and aerial-based recordings (IA); inventories, aerial-based and satellite-based recordings (IAS); and inventories and maps (IM) are reviewed in Table \ref{tab:mixed1}.
%
Inventories provide an additional value to imagery recordings by providing geo-located annotations, depending on the level of precision they offer.
These inventories enhance the spatial context and accuracy of the annotations.
%, enriching the understanding of the visual data.
% Mixing inventories and aerial-based recordings would be useful to precisely match tree measurements with aerial recordings, in particular with LiDAR \cite{weiser_individual_2022} or RGB \cite{brieger_advances_2019, van_geffen_sidroforest_2022} recordings to better estimate carbon stocks at aerial scale per example.
Combining inventories with aerial-based recordings would be highly beneficial for accurately aligning tree measurements with aerial data, especially with LiDAR \cite{weiser_individual_2022} or RGB \cite{brieger_advances_2019, van_geffen_sidroforest_2022} recordings. This integration enables improved estimation of carbon stocks at the aerial scale, among other applications.
%
Field measurements have also been used to validate country or world maps, they are often released together to enable reproducibility of the results.
This integration of field measurements and map data enhances the accuracy and reliability of the generated maps.
Similarly to datasets presented in Section \ref{sec:review_maps}, maps of 
% vegetation index \cite{perez-luque_land-use_2021}, 
forest age \cite{besnard_mapping_2021}, carbon stocks \cite{tucker_sub-continental-scale_2023}, land use and land cover \cite{koskinen_participatory_2019, bendini_combining_2020, bendini_combining_2020, shevtsova_strong_2020, european_commission_statistical_office_of_the_european_union_lucas_2021} have been estimated by machine learning algorithms while being calibrated and validated with inventories.




Mixed datasets composed of ground-based and aerial-based recordings (GA); aerial-based and satellite-based recordings (AS); aerial-based recordings and maps (AM); and satellite-based recordings and maps (SM) are reviewed in Table \ref{tab:mixed2}.
%
% Aligned recordings of ground-based and aerial-based imagery \cite{soltani_transfer_2022} is helpful to align information above and below the canopies of a forest.
Aligning ground-based and aerial-based imagery recordings is valuable in integrating information from both above and below the canopies of a forest. For instance, models can be trained using ground recordings sourced from citizen science-based photographs, and then effectively transferred to aerial data \cite{soltani_transfer_2022}. 
This alignment could enable a comprehensive understanding of the forest ecosystem by bridging the gap between ground-level and aerial-level observations.
%
% Mapping aerial-based and satellite-based recordings is also of interest to generalize high resolution information at a small scale to a lower resolution at a larger scale. 
% For instance, aerial LiDAR has been used to validate canopy height models using satellite imagery \cite{marconi_data_2019, weinstein_remote_2021}. 

Mapping aerial-based and satellite-based recordings is helpful for extrapolating high-resolution information at a small scale to a lower resolution at a larger scale. 
This process allows for the transfer of detailed information captured through aerial LiDAR, for example, to validate canopy height models derived from satellite imagery \cite{marconi_data_2019, weinstein_remote_2021, lang_high-resolution_2022}. The integration of these datasets facilitates a more comprehensive and accurate representation of forest characteristics across different spatial scales.
%
%Mixing SAR and multispectral satellite data to aerial imagery could also improve detection or classification results depending on the variance of reflection and absorption of different family of trees \cite{ahlswede_treesatai_2022}.
Integrating SAR and multispectral satellite data with aerial imagery can potentially enhance model performances \cite{schmitt_data_2016}, particularly by leveraging the varying reflection and absorption characteristics of different tree species \cite{ahlswede_treesatai_2022}.
%
Aerial LiDAR metrics have also been used as validation points to estimate the above ground biomass at large scale \cite{hudak_carbon_2020}.
At a larger scale, satellite data analyzed with multi-classification algorithms have also been useful to monitor and detect forest loss \cite{turubanova_ongoing_2018}.


% specificities and openings
% open access datasets with modalities at different scales have been released to reproduce results. 
% But it also incorporate diversity in the way of observing forests. 
%Various modalities aligned at different scales are now released to improve generalization of machine learning algorithms at large scale. 
% This also open up research in solving tasks at different scales depending on the modality.
Open access datasets featuring modalities at different scales have been made available to enable result reproducibility and promote heterogeneity in the way of observing forests. 
These datasets incorporate various modalities aligned at different scales, which could aim to enhance the generalization capabilities of machine learning algorithms at a larger scale. 
This not only facilitates research in solving tasks at different scales depending on the modality but also fosters a comprehensive understanding of forests through multi-modal analysis.
%
%Multi-modal, multi-scale and multi-task approaches have not been explored in the publications related to the reviewed datasets. We hope that the machine learning and computer vision communities will explore forest monitoring in this pathway to improve our comprehension of the composition of worldwide forest.
To date, the publications related to the reviewed datasets have not extensively explored multi-modal (\textit{e.g.} point clouds with raster data, point observations with spatially continuous data), multi-scale, and multi-task approaches. 
However, it is our hope that the machine learning and computer vision communities will venture into forest monitoring along this path, as it holds great potential for advancing our understanding of the composition of forests worldwide. 
By embracing these comprehensive approaches, we can enhance our comprehension of forests and contribute to more effective and efficient forest management strategies.
%
% In the next section, perspectives on forest datasets and associated challenges will be presented.
The upcoming section will explore perspectives on forest datasets, shedding light on the potential challenges that researchers could prioritize and address in their work.




\begin{table*}[ht]
%\footnotesize
\fontsize{6.5pt}{7.5pt}\selectfont % Font size 
\renewcommand{\arraystretch}{1.5} % Size between lines
\setlength\tabcolsep{5pt} % intercolumn size
\caption{Review of open access mixed forest datasets, including: inventories and aerial-based (IA); inventories, aerial-based and satellite-based (IAS); inventories and maps (IM)}
{\begin{fntable}
\centering
\begin{adjustwidth}{-0.8cm}{}
\begin{tabular}{p{1.4cm} | p{0.3cm} | p{0.4cm} | p{1cm} | p{1.4cm} | p{1.4cm} : p{1.5cm} | p{0.4cm} | p{0.7cm} | p{1cm} | p{1cm} | p{1.1cm}}
\toprule
Dataset  & Type & Publi. year & Recording year & Dataset size & Data  & Spatial resolution or precision & Time series & Potential task(s) & \#Classes & Location  & License  \\
\midrule

% ReforestTree \cite{reiersen_reforestree_2022} & IA & 2022 & 2020 & 4663 trees \newline 3ha & Location \newline Height \newline DBH \newline AGB \newline BGB \newline Carbon Stock \newline Aerial RGB & $\leq$ 1cm \newline N/A \newline N/A \newline N/A \newline N/A \newline N/A \newline 2cm & No & OL \newline Classif. \newline Reg. \newline OD & 2 & Ecuador & CC-BY-4.0 \\

ReforestTree \cite{reiersen_reforestree_2022} & IA & 2022 & 2020 & 4663 trees \newline 3ha & Location \newline Height \newline DBH \newline Biomass \newline Aerial RGB & <1cm \newline N/A \newline N/A \newline N/A \newline 2cm & No & OL \newline Classif. \newline Reg. \newline OD & 2 & Ecuador & CC-BY-4.0 \\

\rowcolor{lightgray}
Individual Tree Point Clouds \cite{weiser_individual_2022} & IA & 2022 & 2019 & 58.2 ha \newline 1491 trees & Location \newline Lidar PC \newline DBH \newline Height \newline Crown diam. \newline CBH  &  10cm \newline $\bullet$ pts-m2 \newline N/A \newline N/A \newline N/A \newline N/A & No & Seg. \newline IS \newline Reg. & 22 & Germany & CC-BY-SA-4.0 \\

%72.5pts-m2 \newline ULS: 1029.2pts-m2 \newline TLS: unknown

% Canopy traits \cite{cherif_spectra_2023} & IA & 2023 & Unknown & 5573 samples & Hyperspec. & 0.2m to 5m & No & Reg. & N/A & Belgium \newline Germany \newline USA \newline France \newline Canada \newline Portugal \newline Israel	& Unknown \\
Canopy traits \cite{cherif_spectra_2023} & IA & 2023 & Unknown & 5573 samples & Hyperspectral & 0.2m to 5m & No & Reg. & N/A & 7 countries (world) $\star$ & Unknown \\

\rowcolor{lightgray}
SiDroForest (4 datasets) \cite{brieger_advances_2019, van_geffen_sidroforest_2022} & IAS & 2022 & 2018 & 19.3K crowns \newline 872 trees \newline 10K synth. \newline 550 samples & Location \newline Height \newline Crown diam. \newline $\dagger$ Aerial record. \newline Sat. Multispec. & 10cm \newline N/A \newline N/A \newline 3cm \newline 10m & Yes & OL \newline Classif. \newline Seg. \newline Reg. & 1 \newline 11 \newline 2 \newline 11 & Russia & CC-BY-4.0 \\

South Korea land cover \cite{seo_deriving_2014} & IM & 2014 & 2009-2011 & 64.4 km2 \newline 3.4k polygones & Polygones and polylines & 1m & No & MC \newline Seg. & 67 & South Korea	& CC-BY-NC-3.0 \\ %3377

\rowcolor{lightgray}
Tazmanian tree plantation \cite{koskinen_participatory_2019} & IM & 2019 & 2010 \newline 2013-2016 & 7500 points \newline 250k km2 & Location \newline LULC maps & 10cm \newline 30m & No & Classif. & 4 & Tanzania	 & CC-BY-4.0 \\

Brazilian Savanna \cite{bendini_combining_2020} & IM & 2020 & from 2008 to 2019 $\blacklozenge$ & 2828 trees \newline Unknown & Location \newline LULC maps & <1cm \newline 30m & No & Classif. & 3 & Brazil & CC-BY-4.0 \\

\rowcolor{lightgray}
Siberia land cover \cite{shevtsova_strong_2020} & IM & 2020 & from 2000 to 2017 $\blacklozenge$ & 2696 points & Location \newline LULC maps & 1cm \newline 30m & Yes & Classif. & 4 & Russia & CC-BY-4.0 \\

Pyrenean oak forest	\cite{perez-luque_land-use_2021} & IM & 2021 & 2016 & 232 km2 \newline 4347 trees & Location \newline DBH \newline Height \newline EVI maps & 1cm \newline N/A \newline N/A \newline 250m & No & Reg. & Unknown & Spain & CC-BY-4.0 \\

\rowcolor{lightgray}
ForestAgeBGI \cite{besnard_mapping_2021} & IM & 2021 & 2000-2019 & 44K trees & Field measur. \newline Climatic var. \newline LULC maps \newline AGB \newline Disturbance & 1km \newline 1km \newline 1km \newline 1km \newline 1km & No & Regress. & N/A & Worldwide & CC-BY-4.0 \\

LUCAS \cite{european_commission_statistical_office_of_the_european_union_lucas_2021} & IM & 2021 & from 2009 to 2018 $\blacklozenge$ & 1.6M points & Location \newline LULC maps & >1cm \newline 2km2 & No & Classif. & 12 \newline 114 subclasses & 27 countries (Europe)$\star$ & \href{https://esdac.jrc.ec.europa.eu/projects/LUCAS/Documents/LUCAS_SOIL_LIC_AGR_final_for_web.pdf}{Specific} \\

\rowcolor{lightgray}
Sub-Saharan carbon stocks \cite{tucker_sub-continental-scale_2023} & IM & 2023 & 2002-2020 & 9.9B trees \newline 10M km2 & $\ddagger$ Field measur. \newline LULC maps & 50cm \newline 50cm & No & Seg. \newline Reg. & 1	& 26 countries (Africa)$\star$ & \href{https://daac.ornl.gov/about/#citation_policy}{Specific} \\

\bottomrule

\end{tabular}
\end{adjustwidth}
\footnotetext[]{{$\blacklozenge$}: The list of recording years is detailed in the OpenForest catalogue.}
\footnotetext[]{{$\bullet$}: The dataset includes three LiDAR with different resolutions which are ALS: 72.5pts-m2, ULS: 1029.2pts-m2, TLS: Unknown.}
\footnotetext[]{{$\dagger$}: Aerial recordings (3cm resolution) are aerial RGB, SfM PC, RGB PC, RGN images, DEM, CHM, DSM, DTM.}
\footnotetext[]{{$\ddagger$}: Field measurements (50cm resolution) are location, crown area, wood mass, mass, root dry mass, count density, coverage density, area density.}
\footnotetext[]{{$\star$}: The list of countries is detailed in the OpenForest catalogue.}
\footnotetext[]{{Acronyms}: \textbf{IA}: inventories and aerial; \textbf{IAS}: inventories, aerial and satellite; \textbf{IM}: inventories and maps; \textbf{N/A}: non applicable; \textbf{Unknown}: non provided by the authors; \textbf{RGB}: red-green-blue; \textbf{DBH}: diameter at breast height; \textbf{PC}: point cloud; \textbf{CBH}: crown base height; \textbf{LULC}: land use and/or land cover; \textbf{EVI}: enhanced vegetation index; \textbf{AGB}: above ground biomass; \textbf{OL}: object localization; \textbf{Classif.}: classification; \textbf{Seg.}: semantic segmentation; \textbf{IS}: instance segmentation; \textbf{Reg.}: regression; \textbf{MC}: multi-classification.
Note that the dataset size measured in \textbf{K} are $O(10^3)$, in \textbf{M} are $O(10^6)$ and in \textbf{B} are $O(10^9)$.}
\end{fntable}}
\label{tab:mixed1}
\end{table*}



\begin{table*}[ht]
%\footnotesize
\fontsize{6.5pt}{7.5pt}\selectfont % Font size 
\renewcommand{\arraystretch}{1.5} % Size between lines
\setlength\tabcolsep{5pt} % intercolumn size
\caption{Review of open access mixed forest datasets, including: ground-based and aerial-based (GA); aerial-based and satellite-based (AS); aerial-based and maps (AM); satellite-based and maps (SM)}
{\begin{fntable}
\centering
\begin{adjustwidth}{-0.7cm}{}
\begin{tabular}{p{1.2cm} | p{0.3cm} | p{0.4cm} | p{1cm} | p{1.4cm} | p{1.4cm} : p{1.5cm} | p{0.4cm} | p{0.7cm} | p{1cm} | p{1cm} | p{1.1cm}}
\toprule
Dataset  & Type & Publi. year & Recording year & Dataset size & Data  & Spatial resolution or precision & Time series & Potential task(s) & \#Classes & Location  & License  \\
\midrule

Knotweed UAV imagery \cite{soltani_transfer_2022} & GA & 2022 & 2021 & 20 km2 \newline 10K samples & Ground RGB \newline Aerial RGB  & N/A \newline 0.3cm & No & Seg. & 1 & Germany & CC-BY-4.0 \\

\rowcolor{lightgray}
IDTReeS	\cite{marconi_data_2019} & AS & 2021 & 2014 \newline 2015 & 0.344 ha & Lidar PC \newline Aerial hypersp. \newline Lidar CHM \newline Sat. RGB & 5pts-m2 \newline 1m \newline 1m \newline 10cm & No & Classif. \newline Seg. \newline Align. & 9 & USA & CC-BY-4.0 \\

NeonTree Evaluation	\cite{weinstein_benchmark_2021} & AS & 2021 & 2017 \newline 2018 \newline 2019 & 31K trees & Lidar PC \newline Aerial hypersp. \newline Sat. RGB & 5pts-m2 \newline 1m \newline 10cm & No & OD \newline Seg. & 1 & USA & CC-BY-4.0 \\

\rowcolor{lightgray}
NEON Crowns	\cite{weinstein_remote_2021} & AS & 2021 & 2018 \newline 2019 & 104M trees & Lidar PC \newline Height \newline Sat. RGB & 5pts-m2 \newline N/A \newline 10cm & No & OD  & 1 & USA & CC-BY-4.0 \\

TreeSatAI \cite{ahlswede_treesatai_2022} & AS & 2022 & 2011-2020 \newline 2015-2020 & 50K samples & Aerial RGB \newline Sat. SAR \newline Sat. Multispec. & 20cm \newline 10m \newline 10m & Yes & MC & 20 & Germany	& CC-BY-4.0 \\

\rowcolor{lightgray}
CMS AGB USA	\cite{hudak_carbon_2020} & AM & 2020 & 2000-2016 & 130K km2 \newline 97K trees & Lidar PC \newline AGB maps & 5pts-m2 \newline 30m & Yes & Reg. & N/A & USA & \href{https://www.earthdata.nasa.gov/learn/use-data/data-use-policy?}{Specific} \\

Primary Forest Loss	\cite{turubanova_ongoing_2018} & SM & 2018 & 2002-2014 & 49K samples & Sat. Multispec. \newline NDVI maps \newline LULC maps & 30m \newline 30m \newline 30m & Yes & MC & 4 & Brazil \newline Dem. Rep. Congo \newline Indonesia & CC-BY-4.0 \\


\bottomrule

\end{tabular}
\end{adjustwidth}
\footnotetext[]{{Acronyms}: \textbf{GA}: ground and aerial; \textbf{AS}: aerial and satellite; \textbf{AM}: aerial and maps; \textbf{SM}: satellite and maps; \textbf{N/A}: non applicable; \textbf{Unknown}: non provided by the authors; \textbf{RGB}: red-green-blue; \textbf{PC}: point cloud; \textbf{CHM}: canopy height model; \textbf{SAR}: synthetic-aperture RADAR; \textbf{AGB}: above ground biomass; \textbf{NDVI}: normalized difference vegetation index; \textbf{LULC}: land use and/or land cover; \textbf{Classif.}: classification; \textbf{Seg.}: semantic segmentation; \textbf{Align.}: alignment; \textbf{OD}: object detection; \textbf{Reg.}: regression; \textbf{MC}: multi-classification.
Note that the dataset size measured in \textbf{K} are $O(10^3)$ and in \textbf{M} are $O(10^6)$.}
\end{fntable}}
\label{tab:mixed2}
\end{table*}



% Useful url: \url{https://gis.stackexchange.com/questions/8650/measuring-accuracy-of-latitude-and-longitude}


\section[Perspectives]{Perspectives}
\label{sec:perspectives}

%The interest in forest monitoring is growing to protect it and its biodiversity.
%Avoided forest conversion and forest management projects require an important level of montoring.
%As well for reforestation and afforestation projects to improve survival rate \cite{martin_people_2021} and avoid diseases \cite{van_lierop_global_2015}.
%Climate change also impacts dynamics in forests \cite{fassnacht_remote_2023} requiring increased surveillance of the forest. 

% The interest in forest monitoring is increasing as a means to safeguard forests and their ecological and societal values. 
The enthusiasm for forest monitoring is on the rise, serving as a safeguard to protect forests and their ecological and societal significance.
Proper monitoring is essential for avoided forest conversion, supporting forest management initiatives, and ensuring successful reforestation and afforestation projects by enhancing survival rates and preventing diseases \cite{van_lierop_global_2015, martin_people_2021}. Additionally, the effects of climate change on forest dynamics \cite{fassnacht_remote_2023} imply a growing need for heightened surveillance of these ecosystems.
%
%This monitoring would benefit from open access, diverse and large datasets alongside with machine learning research.
%This work aims both at related existing challenge and strategies in research while providing an extensive review of open access forest dataset to motivate the research community to better explore this field.
%
%Forest biology is an active field of research as described in Section \ref{sec:bio_challenges}. A lot of questions are ongoing explored on tree species, phenology, abiotic factors and exogenous impacts. 

As a data-driven and empirical science, respectively, forest monitoring benefits from open access, diverse and large datasets, coupled with advancements in machine learning research \cite{de_lima_making_2022}.
This endeavor seeks to address existing challenges and research strategies while extensively reviewing open access forest datasets, with the ultimate goal of encouraging the research community to further investigate this field.

As evident from Section \ref{sec:forest_challenges}, forest monitoring remains an active area of research. Numerous ongoing inquiries delve into various aspects, such as tree species identification, phenology, abiotic factors, exogenous influences and many more.
Machine learning already greatly advanced our capabilities to monitor forests through novel analytical tools and capacities. 
This involves sensing past, current and dynamic forest states through predictive modelling. 
Such models and information, build to be explainable by design, will greatly advance our understanding of forests, including insights into how diverse environmental and anthropogenic drivers impact forest dynamics, as well as the operational mechanisms of forest ecosystems.
% Models and information crafted with an inherent focus on explainability will significantly propel our comprehension of forests. This encompasses insights into how diverse environmental and human-induced factors impact forest dynamics, as well as the operational mechanisms of forest ecosystems.
%
A related and cardinal interest lies in the projection of future forest dynamics to guide decision makers, to improve management and anticipate consequences \cite{requena-mesa_predicting_2018}. 
In this context, it is important to consider that ongoing and accelerated changes induced by global and climate change reshape the dynamics of the Earth system and respective data (making data through time non-stationary). % Therefore, solutions have to be exploited to facilitate that data-driven ML methods trained with current and past data are transferable and reliable to future conditions.
Therefore, it becomes essential to explore solutions that streamline the adaptability and transferability of data-driven machine learning methods, ensuring their efficiency in extrapolating from existing and historical data to future circumstances.


%
%Machine learning and computer vision have a continuously growing impact on many applications including forest monitoring as detailed in Section \ref{sec:ml_challenges}. The related strategies such as model generalization, learning schemes and forestry-based metrics would be helpful to deeper explore biology challenges.
%open access forest datasets have been reviewed in Section \ref{sec:review} according to a list of criteria and to their identified scale. 
%These datasets have been grouped in \textbf{OpenForest}, a dynamic repository, open to updates from the community, to enhance the communication and motivate new applications of machine learning in forest monitoring.
%Benefits and limits of these datasets have been discussed 
%
Machine learning and computer vision are exerting a progressively increasing influence across various domains, including forest monitoring, as elaborated in Section \ref{sec:ml_challenges}. Strategies related to model generalization, learning schemes, and forestry-based metrics are valuable for delving further into the challenges presented by forest biology.

Enhancing the generalization capabilities of models involves better adaptation to diverse spatial and temporal domains, encompassing different forests, sensors, and resolutions. To achieve this, machine learning strategies will be explored, focusing on leveraging existing datasets through weakly-supervised (see Sec.~\ref{sec:ml_weakly}) or few-shot (see Sec.~\ref{sec:ml_fewshot} and \ref{sec:ml_zeroshot}) learning approaches. Moreover, hybrid-models, which integrate physical knowledge (see Sec.~\ref{sec:ml_metrics}), or space-for-time substitutions, which enable to learn temporal dynamics from spatial dynamics, may greatly advance our capabilities to design robust data-driven machine learning applications for monitoring and forecasting in a non-stationary world.
In line with this perspective, the \textbf{OpenForest} dynamic catalog could serve as a suitable reference to consistently enhance and refine models using the latest data.

Implementing active learning methods (see Sec.~\ref{sec:ml_active}) can significantly optimize the process of generating annotations for future datasets. As datasets continue to grow in size, self-supervised learning methods (see Sec.~\ref{sec:ml_ssl}) offer a valuable perspective to learn meaningful representations in deep learning algorithms for forest monitoring without relying heavily on manual annotations.

Incorporating multi-modal and multi-task computer vision architectures into forest monitoring presents an intriguing opportunity to capitalize on task complementarity. 
For instance, by predicting multiple foliage traits from hyperspectral data, a model can learn the covariance among different traits and, hence, provide more robust estimates for challenging traits based on their relation with more accurately predicted ones \cite{cherif_spectra_2023, schiller_deep_2021}. %Or another potential research field could be to predict both tree species and height can enhance carbon stock estimation.
An additional area of potential research could involve enhancing carbon stock estimation by simultaneously predicting both tree species and height.

Foundation models (see Sec.~\ref{sec:ml_foundation_models}) have demonstrated remarkable capabilities in managing various modalities, such as LiDAR, RADAR, and hyperspectral data, with varying spatial and temporal resolutions. These models remain task-agnostic and can achieve high zero-shot performances, making their pretraining a challenging yet promising endeavor for forest monitoring. Once pretrained, they can be adapted to multiple other tasks in this domain.
While relying on the complementarity of large scale multi-modal and multi-tasks datasets, research on foundation models for forest monitoring worldwide would benefit from the \textbf{OpenForest} catalogue dynamically enriched by the community.
% \TKc{I think foundation models are also partly explained above (at least mentioned). I would thus try to put them in context to OpenForest here. E.g. that for foundation models the great challenge is probably to get as much data as possible across modalities, domains, etc...In this context open data but also findable data (and machine learning readable data) is of cardinal importance. Here, OpenForest may be a great ressource. }


Section \ref{sec:review} provides a comprehensive review of open-source forest datasets, categorized according to specific criteria and identified scales. These datasets are grouped in \textbf{OpenForest}, a dynamic catalogue open for updates from the community. The aim is to foster communication, inspire new applications of machine learning in forest monitoring, and motivate advancements in this field.


% \AOc{What are the perspective in biology?}

% \TKc{Consider to rephrase into ML related perspectives "for" biology, since this is paper is not focused in biology per se but the value of ML/Remote sensing for biology.}

%Models should increase their generalization capacities by better adapting to new domains, including different forests, sensors and resolution.
%Machine learning strategies will be explored to better exploit the existing datasets, with weakly-supervised or few-shot learning.
%Active learning would be a great method to optimize the way of generating annotations in future datasets.
%As larger and larger datasets will be released, self-supervised learning methods will be helpful to pretrained models for forest monitoring without requiring too much annotations.
%Adapting multi-modal and multi-task computer vision architectures for forest monitoring would be interesting to benefit from the complementarity of the tasks, eg. predicting the species and the height of trees would be beneficial for carbon stock estimation
%Foundation models have have shown capacities of managing various modalities (point cloud, radar, hyperspectral) with different spatial and temporal resolution  while being agnostic to the task to solve.  Pretrain these models remain an open challenge, but will be beneficial for forest monitoring since they have reach high zero shot performances and can be adapted to multiple other tasks.

% \TKc{I had the feeling that we are so far a bit focused on predicting things in imagery, while it might be worth to also refer to the overarching motivations of modelling, in terms of predictions (object detection, etc...), understanding (feature attribution, xAI, etc...) and forecasting (simulating the future). Looking into the future is assumed to be challenging in forest and in the context of climage change, because we only have data on past and current climate and now how forests behave in future worlds. This is an essential shortcoming of data-driven methods. I hope the following paragraphs synthesis this in a comprehensive manner:}

%Machine learning already greatly advanced our capabilities to monitor forests through novel analytical tools and capacities. This involves sensing past and current forest states and dynamics through predictive modelling. Such models and information together with explainable AI will greatly advance our understanding of forests, including how different environmental and anthropogenic drivers affect forest dynamics and how forest ecosystems function. A related and cardinal interest lies in the projection of future forest dynamics to guide decision makers, to improve management and anticipate consequences \cite{requena-mesa_predicting_2018}. In this context, it is important to consider that ongoing and accelerated changes induced by global and climate change reshape the dynamics of the Earth system and respective data (making data through time non-stationary). Therefore, solutions have to be exploited to facilitate that data-driven ML methods trained with current and past data are transferable and reliable to future conditions.

%Enhancing the generalization capabilities of models involves better adaptation to diverse spatial and temporal domains, encompassing different forests, sensors, and resolutions. To achieve this, machine learning strategies will be explored, focusing on leveraging existing datasets through weakly-supervised (see Sec.~\ref{sec:ml_weakly}) or few-shot (see Sec.~\ref{sec:ml_fewshot} and \ref{sec:ml_zeroshot}) learning approaches. Moreover, hybrid-models, which integrate physical knowledge, or space-for-time substitutions, which enable to learn temporal dynamics from spatial dynamics, may greatly advance our capabilities to design robust data-driven machine learning applications for monitoring and forecasting in a non-stationary world. In line with this perspective, the \textbf{OpenForest} dynamic catalog could serve as a suitable reference to consistently enhance and refine models using the latest data.

% Implementing active learning methods (see.~\ref{sec:ml_active}) can significantly optimize the process of generating annotations for future datasets. As datasets continue to grow in size, self-supervised learning methods offer a valuable means to pretrain models for forest monitoring without relying heavily on manual annotations.

% Incorporating multi-modal and multi-task computer vision architectures into forest monitoring presents an intriguing opportunity to capitalize on task complementarity. For instance, predicting both tree species and height can enhance carbon stock estimation.
% \TKc{This is from an operational point of view. But from a ML perspective the added value is that one model predicitng multi things can learn covariances and thereby the output might be more robust. For isntance, its unrealistic to have high carbon, when the tree height is small. We showed this for instance here: \cite{cherif_spectra_2023}and here: \cite{schiller_deep_2021}. Maybe you want to cite this.
% Mabye you want to use this example? \textit{For instance, by predicting multiple foliage traits from hyperspectral data, a model can learn the covariance among different traits and, hence,  provide more robust estimates for challenging traits based on their relation with traits that are more accuractely predicted.}}


% Foundation models have demonstrated remarkable capabilities in managing various modalities, such as point cloud, radar, and hyperspectral data, with varying spatial and temporal resolutions. These models remain task-agnostic and can achieve high zero-shot performances, making their pretraining a challenging yet promising endeavor for forest monitoring. Once pretrained, they can be adapted to multiple other tasks in this domain.
% \TKc{I think foundation models are also partly explained above (at least mentioned). I would thus try to put them in context to OpenForest here. E.g. that for foundation models the great challenge is probably to get as much data as possible across modalities, domains, etc...In this context open data but also findable data (and machine learning readable data) is of cardinal importance. Here, OpenForest may be a great ressource. }




%Datasets lack of geographical homogeneity distribution particularly in African and Asia as illustrated in Figure \ref{fig:distributions}. 
%Ideally, new datasets should contain high resolution satellite imagery which is not yet possible due to limited license restriction of the private data providers.
%It would be preferable to align multi-modalities according to their temporal and spatial resolutions, when possible, while providing annotations in the highest resolution possible.
%
%Alongside with a list of open access datasets\footnote{\AOc{Need url link here}}, \textbf{OpenForest} will also list data providers which will be also updated by the community.
%Data providers are important source of data to create structured datasets while targeting the needs. In particular, citizen data such as OpenAerialMap\footnote{\url{https://openaerialmap.org/}} are particularly valuable since they integrate data from all the world which could be used for self-supervised learning. It should be better explored to constructed valuable and structured datasets.
%Evolution of UAVs also provide interesting perspective with cheaper and easier vehicle to pilot while carrying sensors with higher and higher resolution. This would be helpful to perform high resolution analysis of forests, potentially inaccessible by foot, while making the bridge with open access and low resolution satellite imagery.
%
%The Biomass\footnote{\url{https://www.esa.int/Applications/Observing_the_Earth/FutureEO/Biomass}} mission will be a must to integrate in future dataset which will provide P-band SAR that will be beneficial for worldwide forests tomography \AOc{Berenger/deep} and while better understanding the overall forest carbon stock.



Datasets, as prerequisite of machine learning applications, commonly exhibit a lack of geographical representativeness, particularly noticeable in African and Asian regions, as depicted in Figure \ref{fig:distributions}. %Ideally, new datasets should encompass high-resolution satellite imagery, but this remains challenging due to limited license restrictions imposed by private data providers. 
% Whenever feasible, it is preferable to align multi-modal data based on their temporal and spatial resolutions while offering annotations in the highest available resolution.
Whenever feasible, ideal datasets would preferably align multi-modal recordings based on their temporal and spatial resolutions while offering annotations in the highest available resolution.
As demonstrated in remote sensing \cite{lacoste_geo-bench_2023}, aggregating numerous diverse forest datasets could foster research in developing specialized foundation models for effective forest monitoring
%\TKc{Establish a link/perspective in the context of OpenForest, e.g.: OpenForest could provide a easier overview of suitable datasets, so people can integrate as much data as possible but also test their models across domains/geographic regions.}

The \textbf{OpenForest} catalogue, in addition to providing a list of open access datasets, will also curate information about data providers\footnote{See related information at \url{https://github.com/RolnickLab/OpenForest}.}, a crucial resource for generating well-structured datasets that cater to specific needs. Citizen-generated data, such as curated on OpenAerialMap\footnote{\url{https://openaerialmap.org/}} or GBIF\footnote{\url{https://www.gbif.org/}}, holds significant value since it integrates information from across the globe, making it ideal for self-supervised learning. The data provider list will also be frequently updated to integrate most recent initiatives. For instance, incorporating data from the Biomass mission\footnote{\url{https://www.esa.int/Applications/Observing_the_Earth/FutureEO/Biomass}} as soon as possible into future datasets is essential. The mission's provision of P-band SAR data will greatly benefit worldwide forest tomography \cite{berenger_deep_2023}, advancing our comprehension of forest carbon stock and its dynamics.
Exploring its potential can lead to the creation of valuable and structured datasets.


Recordings from aerial data, especially UAVs, gain momentum by offering promising prospects, with more affordable and easier-to-pilot vehicles equipped with higher-resolution sensors. Leveraging UAV technology allows for high-resolution forest analysis, even in remote or inaccessible areas, and can moreover advance large-scale assessments by its integration with Earth observation satellite missions \citep{schiefer_uav-based_2023}.

As more and more datasets are released at various scales, the \textbf{OpenForest} catalogue offers the opportunity to centralize these information with details. It will help to motivate research in bridging the gab between scales, sensors and resolutions while hopefully motivate collaborations between researchers.


%\TKc{The previous two paragraphs are a bit disconnected from the OpenForest idea. Maybe we should provide a perspective that so many people are creating UAV datasets and that curating information on this in one place (OpenForest) is essential to exploit the respective potential}



%\ELc{finishes rather abruptly on a technical note about Biomass - shouldn't we conclude with a general statement about the future value of OpenForest or something like that?}

%\TKc{Agree with Etienne. As I mentioned above, I would rather stress that OpenForests provides a very promising catalogue that is connected to the above mentioned perspectives (model generalization, domain transfer, etc... I would probably not discuss individual satellite missions or products. Maybe rather a general statement that a catalouge such as OpenForests will enable in a timely manner to integrate upcoming datasets, e.g. of future satellite missions, such as BIOMASS or NASA SBG and also provide researchers working with such missions with reference data. }

%\TKc{Maybe provide a perspective that findable, searchable and open data is of cardinal importance and that we hope that OpenForest will provide a tool and attract people to provide data according to such means.}


%\begin{itemize}
    %\item Wide range of data providers: start a list from the overleaf. We may think about a table in the corpus or in the appendix.
    %\item Discussion about citizen databases
    %\item Even if reforestation if controversial \ELc{Controversial in what way exactly?}, we need more monitoring to improve survival rate \cite{martin_people_2021} and avoid diseases \cite{van_lierop_global_2015} (other refs?).
    %\item Which dataset do we want to emerge in the future? Opening with BIOMASS mission \ELc{I would also mention drones, specifically microdrones which are cheap and easy to use and that allow people around the world to acquire high-resolution data over forests, albeit with small extents. Mention open access repositories like OpenAerialMap.org which hosts a lot of UAV imagery over forests. Potentially very useful for self-supervised learning etc.}
    %\item What are the most important challenges to tackle in the near future? Both biology and ML (probably domain adaptation and zero shot)
    %\item What about a multi-modal forest foundation model?
    % \item Argument on change dynamics in forests as consequence of climate change \cite{fassnacht_remote_2023}: we need more monitoring than ever.
%\end{itemize}



% \section{Equations (template)}

% Equations in \LaTeX{} can either be inline or on-a-line by itself. For
% inline equations use the \verb+$...$+ commands. Eg: The equation
% $H\psi = E \psi$ is written via the command $H \psi = E \psi$.

% For on-a-line by itself equations (with auto generated equation numbers)
% one can use the equation or eqnarray environments \textit{D}.
% \begin{equation}
% \mathcal{L} = i {\psi} \gamma^\mu D_\mu \psi
%     - \frac{1}{4} F_{\mu\nu}^a F^{a\mu\nu} - m {\psi} \psi
% \label{eq1}
% \end{equation}
% where,
% \begin{align}
% D_\mu &=  \partial_\mu - ig \frac{\lambda^a}{2} A^a_\mu
% \nonumber \\
% F^a_{\mu\nu} &= \partial_\mu A^a_\nu - \partial_\nu A^a_\mu
%     + g f^{abc} A^b_\mu A^a_\nu
% \label{eq2}
% \end{align}
% Notice the use of \verb+\nonumber+ in the align environment at the end
% of each line, except the last, so as not to produce equation numbers on
% lines where no equation numbers are required. The \verb+\label{}+ command
% should only be used at the last line of an align environment where
% \verb+\nonumber+ is not used.
% \begin{equation}
% Y_\infty = \left( \frac{m}{\textrm{GeV}} \right)^{-3}
%     \left[ 1 + \frac{3 \ln(m/\textrm{GeV})}{15}
%     + \frac{\ln(c_2/5)}{15} \right]
% \end{equation}
% The class file also supports the use of \verb+\mathbb{}+, \verb+\mathscr{}+ and
% \verb+\mathcal{}+ commands. As such \verb+\mathbb{R}+, \verb+\mathscr{R}+
% and \verb+\mathcal{R}+ produces $\mathbb{R}$, $\mathscr{R}$ and $\mathcal{R}$
% respectively.

% \section{Figures (template)}

% As per the \LaTeX\ standards eps images in \verb!latex! and pdf/jpg/png images in
% \verb!pdflatex! should be used. This is one of the major differences between \verb!latex!
% and \verb!pdflatex!. The images should be single page documents. The command for inserting images
% for latex and pdflatex can be generalized. The package that should be used
% is the graphicx package.

% \begin{figure}[t]%
% \FIG{\includegraphics[width=0.9\textwidth]{Fig}}
% {\caption{This is a widefig. This is an example of long caption this is an example of long caption  this is an example of long caption this is an example of long caption}
% \label{fig1}}
% \end{figure}



% \section{Tables (template)}

% Tables can be inserted via the normal table and tabular environment. To put
% footnotes inside tables one has to use the additional ``fntable" environment
% enclosing the tabular environment. The footnote appears just below the table
% itself.

% \begin{table}[t]
% \tabcolsep=0pt%
% \TBL{\caption{Tables which are too long to fit,
% should be written using the ``table*'' environment as~shown~here\label{tab2}}}
% {\begin{fntable}
% \begin{tabular*}{\textwidth}{@{\extracolsep{\fill}}lcccccc@{}}\toprule%
%  & \multicolumn{3}{@{}c@{}}{\TCH{Element 1}}& \multicolumn{3}{@{}c@{}}{\TCH{Element 2\smash{\footnotemark[1]}}}
%  \\\cmidrule{2-4}\cmidrule{5-7}%
% \TCH{Projectile} & \TCH{Energy} & \TCH{$\sigma_{\mathit{calc}}$} & \TCH{$\sigma_{\mathit{expt}}$} &
% \TCH{Energy} & \TCH{$\sigma_{\mathit{calc}}$} & \TCH{$\sigma_{\mathit{expt}}$} \\\midrule
% \TCH{Element 3}&990 A &1168 &$1547\pm12$ &780 A &1166 &$1239\pm100$\\
% {\TCH{Element 4}}&500 A &961 &$\hphantom{0}922\pm10$ &900 A &1268 &$1092\pm40\hphantom{0}$\\
% \botrule
% \end{tabular*}%
% \footnotetext[]{{Note:} This is an example of table footnote this is an example of table footnote this is an example of table footnote this is an example of~table footnote this is an example of table footnote}
% \footnotetext[1]{This is an example of table footnote}%
% \end{fntable}}
% \end{table}



% \section{Cross referencing (template)}

% Environments such as figure, table, equation, align can have a label
% declared via the \verb+\label{#label}+ command. For figures and table
% environments one should use the \verb+\label{}+ command inside or just
% below the \verb+\caption{}+ command.  One can then use the
% \verb+\ref{#label}+ command to cross-reference them. As an example, consider
% the label declared for Figure \ref{fig1} which is
% \verb+\label{fig1}+. To cross-reference it, use the command
% \verb+ Figure \ref{fig1}+, for which it comes up as
% ``Figure \ref{fig1}''.
% %The reference citations should used as per the ``natbib'' packages. Some sample citations:  \cite{bib1,bib2,bib3,bib4,bib5}.

% \section{Lists (template)}

% List in \LaTeX{} can be of three types: enumerate, itemize and description.
% In each environments, new entry is added via the \verb+\item+ command.
% Enumerate creates numbered lists, itemize creates bulleted lists and
% description creates description lists.
% \begin{enumerate}[1.]
% \item First item in the number list.
% \item Second item in the number list.
% \item Third item in the number list.
% \end{enumerate}
% List in \LaTeX{} can be of three types: enumerate, itemize and description.
% In each environments, new entry is added via the \verb+\item+ command.
% \begin{itemize}
% \item First item in the bullet list.
% \item Second item in the bullet list.
% \item Third item in the bullet list.
% \end{itemize}

% \begin{appendix}
% \section{Appendix. Title for Appendix Section (template)}\label{appendixA}
% Appendix text here.
% \end{appendix}


% \section{Conclusion (template)}

% Some Conclusions here.



\begin{Backmatter}

\paragraph{Acknowledgments} The authors are grateful for the valuable feedback of O.~Sonnentag.


\paragraph{Funding Statement}
%This research was supported by grants from the <funder-name><doi>(<award ID>); <funder-name><doi>(<award ID>).
%No specific funding exists.
This work was funded through the IVADO program on ``AI, Biodiversity and Climate Change'' and the Canada CIFAR AI Chairs program. 
It was also funded through the German Research Foundation (DFG) under the project PANOPS (project number 504978936) and BigPlantSens (project number 444524904).


\paragraph{Competing Interests}
%A statement about any financial, professional, contractual or personal relationships or situations that could be perceived to impact the presentation of the work --- or `None' if none exist
None

\paragraph{Data Availability Statement}
The \textbf{OpenForest} catalogue is available and open to contributions in the following repository: \url{https://github.com/RolnickLab/OpenForest}.
%A statement about how to access data, code and other materials allowing users to understand, verify and replicate findings --- e.g. Replication data and code can be found in Harvard Dataverse: \verb+\url{https://doi.org/link}+.

\paragraph{Ethical Standards}
The research meets all ethical guidelines, including adherence to the legal requirements of the study country.

\paragraph{Author Contributions}
%Please provide an author contributions statement using the CRediT taxonomy roles as a guide {\verb+\url{https://www.casrai.org/credit.html}+}. 
Conceptualization: A.O.; D.R. Methodology: A.O.; T.K.; E.L.; D.R. Data curation: A.O. Data visualisation: A.O. Writing original draft: A.O.; T.K.; E.L. Writing - Review \& Editing: A.O.; T.K.; E.L.; D.R.. All authors approved the final submitted draft.



\bibliographystyle{apalike}

\bibliography{openforest_cleaned}


\end{Backmatter}

\end{document}