%%%%%%%%%%%%%%%
\section{Introduction} \label{sec:int}
%%%%%%%%%%%%%%%%%%%%%%%%

The so-called $h$-function, or harmonic-measure distribution function, associated with a particular planar domain $\Omega$ in the extended complex plane $\overline{\CC}=\CC\cup\{\infty\}$ can be viewed as a signature of the principal geometric characteristics of the boundary components of the domain $\Omega$ relative to some basepoint $z_0\in\Omega$. Such functions are also probabilities that a Brownian particle reaches a boundary component of $\Omega$ within a certain distance from its point of release $z_0$. The properties of two-dimensional Brownian motions were investigated extensively by Kakutani~\cite{ka44} who found a deep connection between Brownian motion and harmonic functions (see also~\cite{ka45,SnipWard16,bookBM}).
Stemming from a problem proposed by Stephenson and listed in~\cite{bh89}, the $h$-functions were first introduced as objects of study by Walden \& Ward~\cite{wawa96}. Since then the theory of $h$-functions has been successively developed in several works~\cite{asww,bawa14,beso03,gswc,SnipWard05,SnipWard08,wawa96,wawa01,wawa07}. For a overview of the main properties of $h$-functions, the reader is referred to the survey paper~\cite{SnipWard16}. 

Given a domain $\Omega$ in the extended complex plane $\overline{\CC}$ and a fixed basepoint $z_0\in\Omega$, the $h$-function is the piecewise smooth, non-decreasing function, $h\,:\,[0,\infty)\to[0,1]$, defined by
\begin{equation}\label{eq:hr}
h(r)= \omega(z_0,\partial\Omega\cap\overline{B(z_0,r)},\Omega)
\end{equation}
where $\omega$ denotes harmonic measure and $B(z_0,r)=\{z\in\CC\,:\,|z-z_0|<r\}$, i.e., $h(r)$ is the value of the harmonic measure of $\partial\Omega\cap\overline{B(z_0,r)}$ with respect to $\Omega$ at the point $z_0$. 
Thus $h(r)$ is equal to $u(z_0)$, where $u(z)$ is the solution to the following boundary value problem (BVP) of Dirichlet-type:
\begin{subequations}\label{eq:bdv-u}
	\begin{align}
		\label{eq:u-Lap}
		\nabla^2 u(z) &= 0 \quad \mbox{if }z\in \Omega, \\
		\label{eq:u-1}
		u(z)&= 1 \quad \mbox{if }z\in \partial\Omega\cap\overline{B(z_0,r)}, \\
		\label{eq:u-0}
		u(z)&= 0 \quad \mbox{if }z\in \partial\Omega\backslash \overline{B(z_0,r)}. 
	\end{align}
\end{subequations}
This is illustrated in Figure~\ref{fig:h30} for the domain $\Omega_m$ defined in~\eqref{eq:Omeg-m} below, and we refer to $\partial B(z_0,r)$ as a `capture circle' of radius $r$ and center $z_0$.

The main purpose of this work is to develop a fast and accurate numerical method for approximating the values of $h$-functions which are close to the actual, hypothetical, $h$-function associated with the middle-thirds Cantor set $\mathcal{C}$, which is defined by
\begin{equation}\label{eq:E}
\mathcal{C}=\bigcap_{\ell=0}^{\infty} E_\ell
\end{equation}
where the sets $E_\ell$ are defined recursively by
\begin{equation}\label{eq:Ek}
E_\ell=\frac{1}{3}\left(E_{\ell-1}-\frac{1}{3}\right)\bigcup\frac{1}{3}\left(E_{\ell-1}+\frac{1}{3}\right), \quad \ell\ge 1,
\end{equation}
and $E_0=[-1/2,1/2]$. It is a type of fractal domain which is infinitely connected. We will demonstrate that the $h$-function for the Cantor set $\mathcal{C}$ can be approximated by computing $h$-functions for multiply connected slit domains of finite, but high, connectivity.
 
For a fixed $\ell=0,1,2,\ldots$, the set $E_\ell$ consists of $m = 2^\ell$ slits $I_{j}$, $j = 1, 2, \ldots, m$, numbered from left to right.  The slits have the same length $L = (1/3)^\ell$.
The center of the slit $I_{j}$ is denoted by $w_{j}$.
The domain $\Omega_m$ is the complement of the closed set $E_\ell$ with respect to the extended complex plane $\overline{\CC}$, i.e.,
\begin{equation}\label{eq:Omeg-m}
	\Omega_m = \overline{\CC}\setminus\bigcup_{j=1}^{m} I_j, \quad m=2^\ell.
\end{equation}
As the value of $m$ is increased, the computed $h$-function for the domain $\Omega_m$ provides a successively better approximation of the actual $h$-function associated with the middle-thirds Cantor set $\mathcal{C}$.

The particular location of the basepoint $z_0$ is intrinsic to $h$-function calculations. In the current work, we consider $h$-functions for $\Omega_m$ with two different basepoint locations: one strictly to the left of all slits at $z_0=-3/2$, and another at $z_0=0$, as illustrated in Figure~\ref{fig:h30} for $m=2$. 
It should be highlighted that the method to be proposed in this paper, with modifications, can still be used to calculate $h$-functions for other basepoint locations.


\begin{figure}[ht] %
	\centerline{\hfill
		\scalebox{0.4}{\includegraphics[trim=0 0 0 0,clip]{figh3}}\hfill
		\scalebox{0.4}{\includegraphics[trim=0 0 0 0,clip]{figh0}}\hfill
	}
	\caption{The domain $\Omega_2$ and a typical capture circle $\partial B(z_0,r)$ of radius $r$ used in the definition of the $h$-function with basepoint $z_0=-3/2$ (left) and with basepoint $z_0=0$ (right).}
	\label{fig:h30}
\end{figure}

There have been several calculations and analyses of $h$-functions over simply connected domains \cite{SnipWard16}. Recently, the first explicit formulae of $h$-functions in the multiply connected setting were derived by Green \textit{et al.}~\cite{gswc}. Their formulae were constructed by making judicious use of the calculus of the Schottky-Klein prime function, a special transcendental function which is ideally suited to solving problems in multiply connected domains, and the reader is referred to the monograph by Crowdy \cite{CrowdyBook} for further details.
The formulae in \cite{gswc} have been generalized in~\cite{ArMa} for other classes of multiply connected domains and basepoint locations.

In~\cite{gswc}, the planar domains of interest were chosen to be a particular class of multiply connected slit domain, i.e. unbounded domains in the complex plane whose boundary components consist of a finite number of finite-length linear segments lying on the real axis. Indeed, this choice of planar domain was motivated in part by the middle-thirds Cantor set. 
Whilst it is important to point out that there is now reliable software available to compute the prime function  based on the analysis of Crowdy \textit{et al.}~\cite{ckgn}, the number of computational operations needed to be executed is relatively high for domains with many boundary components such as those which will be considered in this paper. To this end, to compute $h$-functions for \textit{highly} multiply connected domains, we turn to a boundary integral equation (BIE) method which significantly reduces the number of computational operations required while also maintaining a good level of accuracy. 
The proposed method is based on a BIE with the Neumann kernel (see~\cite{Weg-Nas,Nas-ETNA}). 
The integral equation method that will be employed in this paper has also been used in the new numerical scheme for computing the prime function \cite{ckgn}. The integral equation has found multitudinous application in several areas, including fluid stirrers, vortex dynamics, composite materials, and conformal mappings (see~\cite{Nas-ETNA,NG18,NK} and the references cited therein).
In particular, the integral equation has been used in~\cite{LSN17,Nvm} to compute the logarithmic and conformal capacities for the Cantor set and the Cantor dust.

With the middle-thirds Cantor $\mathcal{C}$ set in mind, the goal herein is to numerically calculate $h$-functions in domains which effectively imitate the $h$-function associated with $\mathcal{C}$.
As mentioned above, to imitate $\mathcal{C}$ in a practical sense, we again consider multiply connected slit domains as in \cite{gswc}, but significantly increase the number of boundary components by successively removing the middle-third portion of each slit, and then calculate the corresponding $h$-functions.
We also use the computed values of these $h$-functions to study numerically their asymptotic behavior.
The calculation of such $h$-functions, together with their asymptotic features, was motivated by the list of open problems in Snipes \& Ward \cite{SnipWard16}. 

Computing the $h$-function requires solving a Dirichlet BVP in the domain $\Omega_m$, which is not an easy task since the boundary of the domain consists of the slits $I_1,\ldots,I_m$ and hence the solution of the BVP admits singularities at the slit endpoints. 
Since the Dirichlet BVP is invariant under conformal mapping, one way to overcome such a difficulty is to map the domain $\Omega_m$ onto another domain with a simpler geometry. 
We therefore appeal to the iterative method presented in~\cite{NG18} to numerically compute an unbounded preimage domain $G_m$ bordered by smooth Jordan curves. 
We will summarize the construction of this preimage domain in \S\ref{sec:pre} below. 


The main objective of this paper is to present a fast numerical method towards computing the $h$-function for the middle-thirds Cantor set. 
The method is based on using the BIE to compute the $h$-function in terms of the solution of the BVP~\eqref{eq:bdv-u}.
Our paper mostly has a potential theoretical flavor and has the following structure. 
In Section 2, we introduce the conformal mapping we use, together with an outline of the BIE with the Neumann kernel. 
When the capture circle passes through the gaps between the slits, the values the $h$-function $h(r)$ takes on are equal to the harmonic measures of the slits enclosed by the capture circle with respect to the domain $\Omega_m$. In this case, with the help of a conformal mapping, the BVP~\eqref{eq:bdv-u} will be transformed into an equivalent problem in an unbounded circular domain $G_m$.
The transformed problem is a particular case of the BVP considered in~\cite{Nvm} and hence it can be solved using the BIE method as in~\cite{Nvm}. This will be considered in Section~\ref{sec:step}. 
We calculate the values of $h$-functions associated with slit domains of connectivity $64$ and $1024$ for the basepoint $z_0=-3/2$; and for connectivity $64$ and $2048$ for the basepoint $z_0=0$. However, when the capture circle intersects one of the slits, the right-hand side of the BVP~\eqref{eq:bdv-u} becomes discontinuous. As in  Section~\ref{sec:step}, we use conformal mapping to transform the slit domain $\Omega_m$ onto a circular domain $G_m$, and hence the BVP~\eqref{eq:bdv-u} will be turned into an equivalent problem in $G_m$ whose right-hand side is also discontinuous. With the help of two M{\"o}bius mappings, the BVP in $G_m$ can be modified so that it will have a continuous right-hand side. 
The BIE method is then used to solve this modified BVP and to compute the values of the $h$-function $h(r)$. This case will be considered in Section~\ref{sec:h-fun}.
In Section 5, we analyze numerically the behavior of the $h$-function $h(r)$ when $r$ is close to $1$ for the basepoint $z_0=-3/2$; and when $r$ is close to $1/6$ for the basepoint $z_0=0$. Section 6 provides our concluding remarks.

