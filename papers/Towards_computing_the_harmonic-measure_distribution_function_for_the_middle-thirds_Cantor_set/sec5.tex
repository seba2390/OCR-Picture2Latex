%%%%%%%%%%%%%%%%%%%%%%%%%%%%%%%%%%
\section{Asymptotic behavior of the $h$-function}\label{sec:asy}
%%%%%%%%%%%%%%%%%%%%%%%%%%%%%%%%%%

To complement the preceding computations, in this section we analyze numerically some asymptotic features of the $h$-functions as $r\to1^+$ for the basepoint $z_0=-3/2$ and as $r\to(1/6)^+$ for the basepoint $z_0=0$. 

\subsection{The basepoint $z_0=-3/2$}

For this case, we study numerically the behavior of the $h$-function $h(r)$ when $r$ is near to $1$ for several values of $\ell$. 
For $E_0=[-1/2,1/2]$, i.e., $\ell=0$, we have~\cite{gswc}
\begin{equation}\label{eq:h-ell0}
h(r)=\frac{2}{\pi}\tan^{-1}\left(\sqrt{2}\sqrt{\frac{r-1}{2-r}}\right)
\end{equation}
which implies that
\[
h(r)\sim C_0(r-1)^{\beta_0}, \qquad C_0=\frac{2\sqrt{2}}{\pi}, \quad \beta_0=\frac{1}{2}.
\]
For $\ell\ge1$, up to the best of our knowledge, there are no results about the asymptotic behavior of the $h$-function as $r\to1^+$. Here, we attempt to address this issue numerically.
We choose $20$ values of $r$ in the interval $(1,1+\varepsilon)$ with $\varepsilon=10^{-6}$ (as illustrated in Figure~\ref{fig:3hr}, these $20$ values of $r$ are chosen by choosing $20$ points on the upper-half of the circle $C_1$ such that the images of these points under the conformal mapping $F$ are in  $(1,1+\varepsilon)$). Then, we use the method described in the previous section to compute the values of the $h$-function $h(r)$ for these values of $r$. 
We consider only the values of $\ell=1,\ldots,8$. To consider larger values of $\ell$, it is required to consider smaller values of $\varepsilon$ which is challenging numerically. For $\ell=1,\ldots,8$, the graphs of the functions $h(r)$ on $(1,1+\varepsilon)$ are shown in Figure~\ref{fig:h-asy} (left). The figure shows also the graph of the function $h(r)$ for $\ell=0$ which is given by~\eqref{eq:h-ell0}.
Figure~\ref{fig:h-asy} (right) shows the graphs of the function $\log(h(r))$ as a function of $\log(r-1)$ for $\ell=0,1,\ldots,8$.

\begin{figure}[htb] %
	\centerline{
		\scalebox{0.5}{\includegraphics[trim=0 0 0 0,clip]{fig_h_asy_sq}}
		\hfill
		\scalebox{0.5}{\includegraphics[trim=0 0 0 0,clip]{fig_h_asy_L}}
	}
	\caption{The graphs of the $h$-function (left) and  $\log(h(r))$ (right) for $\ell=0,1,\ldots,8$ and $r\in(1,1+\varepsilon)$ with $\varepsilon=10^{-6}$ for $z_0=-3/2$.}
	\label{fig:h-asy}
\end{figure}

Figure~\ref{fig:h-asy} suggests that the functions $h(r)$ for $\ell=1,\ldots,8$ have the same behavior near $r=1$ as for the case $\ell=0$. 
For each $\ell$, $\ell=1,\ldots,8$, we use these $20$ values of $r$, i.e., $r_1,\ldots,r_{20}$, and the approximate values of the $h$-function $h(r)$ at these values to find approximations of real constants $C_\ell$ and $\beta_\ell$ such that
\begin{equation}\label{eq:hrj-C}
h(r_j)\approx C_\ell (r_j-1)^{\beta_\ell}.
\end{equation}
The approximations of $C_\ell$ and $\beta_\ell$ will be computed using the least square method, i.e., we find the values of $C_\ell$ and $\beta_\ell$ that minimize the least square error 
\begin{equation}\label{eq:LSE}
\mathcal{E}_\ell = \sum_{j=1}^{20}\left(h(r_j)-C_\ell(r_j-1)^{\beta_\ell}\right)^2.
\end{equation}
By taking the logarithm of both sides of~\eqref{eq:hrj-C}, we obtain
\[
\log(h(r_j))\approx \log(C_\ell)+\beta_\ell\log(r_j-1),
\]
which can be written as 
\[
y_j\approx\log(C_\ell)+\beta_\ell x_j, \quad j=1,\ldots,20,
\]
where $y_j= \log(h(r_j))$ and $x_j=\log(r_j-1)$. The approximate values of the constants $\log(C_\ell)$ and $\beta_\ell$ will be computed by using the {\sc Matlab} function \verb|polyfit| to find the best line fitting for the points $(x_j,y_j)$, $j=1,\ldots,20$. The values of these constant as well as the error $\mathcal{E}_\ell$ in~\eqref{eq:LSE} are given in Table~\ref{tab:Error} for $\ell=1,\ldots,8$.

\begin{table}[h]
	\caption{The values of the constants $C_\ell$ and $\beta_\ell$, as well as the error $\mathcal{E}_\ell$ for  $z_0=-3/2$ (left) and  $z_0=0$ (right).}
	\label{tab:Error}
	\centering
	\begin{tabular}{l|lll|lll}  \hline
		& \multicolumn{3}{c|}{$z_0=-3/2$}   & \multicolumn{3}{c}{$z_0=0$}  \\ \cline{2-7}
$\ell$   & $C_\ell$      & $\beta_\ell$    & $\mathcal{E}_\ell$     & $C_\ell$  & $\beta_\ell$  & $\mathcal{E}_\ell$ \\ 
		\hline
$0$      & $0.900316$   & $0.5$         & ---                    & ---         & ---         & ---         \\
$1$      & $0.939343$   & $0.500000$    & $2.71\times10^{-21}$   & $2.351932$  & $0.500000$  & $9.00\times10^{-18}$  \\
$2$      & $0.977556$   & $0.500000$    & $1.31\times10^{-20}$   & $2.395871$  & $0.500000$  & $5.42\times10^{-18}$  \\
$3$      & $1.018398$   & $0.500000$    & $1.89\times10^{-19}$   & $2.466099$  & $0.500000$  & $3.14\times10^{-18}$  \\
$4$      & $1.061124$   & $0.500000$    & $2.67\times10^{-18}$   & $2.555452$  & $0.500000$  & $4.37\times10^{-19}$  \\
$5$      & $1.105679$   & $0.500002$    & $1.08\times10^{-17}$   & $2.655781$  & $0.500001$  & $6.70\times10^{-17}$  \\
$6$      & $1.152042$   & $0.500000$    & $5.67\times10^{-16}$   & $2.763722$  & $0.500004$  & $8.10\times10^{-16}$  \\
$7$      & $1.200444$   & $0.500001$    & $4.72\times10^{-15}$   & $2.878107$  & $0.500011$  & $1.07\times10^{-14}$  \\
$8$      & $1.251569$   & $0.500037$    & $2.03\times10^{-14}$   & $2.998958$  & $0.500032$  & $1.31\times10^{-13}$   \\
		\hline
	\end{tabular}
\end{table}

  
It is clear from Table~\ref{tab:Error} that the values of the constant $C_\ell$ increases as $\ell$ increases. However, it is not clear from the presented values if $C_\ell$ is bounded as $\ell\to\infty$. As a numerical attempt to study the behavior of $C_\ell$ for large values $\ell$, we use extrapolation to find approximate values of $C_\ell$ for $\ell>8$. 
To use the exact value $C_0$ and the computed values $C_1,\ldots,C_8$ to extrapolate the values of $C_\ell$ for $\ell>8$, we note that $\log(C_\ell)$ behave linearly as a function $\ell$ (see Figure~\ref{fig:C08} (left)).   
We again use the {\sc Matlab} function \verb|polyfit| for computing the best line $a\ell+b$ that fits the points $(\ell,\log C_\ell)$ for $\ell=0,1,\ldots,8$. By computing the approximate values of the coefficients, the values of $C_\ell$ can be approximated by
\begin{equation}\label{eq:C-ell}
C_\ell \approx 0.900613 e^{0.041069\ell}\quad {\rm for}\quad \ell\ge 0.
\end{equation}
The least square error in this formula is
\[
\sum_{\ell=0}^{8} \left(C_\ell - 0.900613 e^{0.041069\ell}\right)^2\approx 1.78\times10^{-6}.
\]

The formula~\eqref{eq:C-ell} suggests that $C_\ell$ is unbounded as $\ell\to\infty$. 
The graph of the $C_\ell$ as a function $\ell$ is shown in Figure~\ref{fig:C08} (right) where the (red) circles are the values of $C_\ell$ presented in Table~\ref{tab:Error}, for $\ell=0,1,\ldots,8$, and the (blue) circles are the extrapolated values, for $\ell=9,\ldots,20$. 

\begin{figure}[htb] %
	\centerline{
		\scalebox{0.5}{\includegraphics[trim=0 0 0 0,clip]{fig_C3_L}}
		\hfill
		\scalebox{0.5}{\includegraphics[trim=0 0 0 0,clip]{fig_C3_E}}
	}
	\caption{The values of $\log C_\ell$ (left) and $C_\ell$ (right) for the case $z_0=-3/2$.}
	\label{fig:C08}
\end{figure}



\subsection{The basepoint $z_0=0$}

For this case, it is not possible to consider $\ell=0$ because $z_0=0$ is in the middle of the interval $[-1/2,1/2]$. For $\ell\ge1$, as in the previous case, we choose $20$ values of $r$ in the interval $(1/6,1/6+\varepsilon)$ with $\varepsilon=10^{-6}$ (by choosing $20$ points on the upper-half of the circle $C_1$ such that the images of these points under the conformal mapping $F$ are in  $(1/6,1/6+\varepsilon)$). We then use the method described in Section~\ref{sec:h-fun} to compute the values of the $h$-function $h(r)$ for these values of $r$. 
As in the previous subsection, we consider only the values of $\ell$, $\ell=1,\ldots,8$. The graph of the functions $h(r)$ for these values of $\ell$ on $(1/6,1/6+\varepsilon)$ are shown in Figure~\ref{fig:h0-asy} (left). 
Figure~\ref{fig:h0-asy} (right) shows the graphs of the function $\log(h(r))$ for $\ell=1,\ldots,8$.

\begin{figure}[htb] %
	\centerline{
		\scalebox{0.5}{\includegraphics[trim=0 0 0 0,clip]{fig_h0_asy_sq}}
		\hfill
		\scalebox{0.5}{\includegraphics[trim=0 0 0 0,clip]{fig_h0_asy_L}}
	}
	\caption{The graphs of the $h$-function (left) and  $\log(h(r))$ (right) for $\ell=1,\ldots,8$ and $r\in(1/6,1/6+\varepsilon)$ with $\varepsilon=10^{-6}$ for $z_0=0$.}
	\label{fig:h0-asy}
\end{figure}

Figure~\ref{fig:h0-asy} suggests that the functions $h(r)$ for $\ell=1,\ldots,8$ have the same behavior near $r=1/6$ as for the case $z_0=-3/2$ near $r=1$.
We use the $20$ values of $r$, i.e., $r_1,\ldots,r_{20}\in(1/6,1/6+\varepsilon)$, and the approximate values of the $h$-function $h(r)$ at these values to find approximations of real constants $C_\ell$ and $\beta_\ell$ such that
\begin{equation}\label{eq:hrj-C0}
	h(r_j)\approx C_\ell (r_j-1/6)^{\beta_\ell}
\end{equation}
for each $\ell$, $\ell=1,\ldots,8$.
The approximations of $C_\ell$ and $\beta_\ell$ are computed using the least square method and by using the {\sc Matlab} function \verb|polyfit| as in the case $z_0=-3/2$. The values of these constant as well as the least square error $\mathcal{E}_\ell$ are given in Table~\ref{tab:Error} for $\ell=1,\ldots,8$. By analogy to the case $z_0=-3/2$, it seems from Table~\ref{tab:Error} that the values of $C_\ell$ are also unbounded as $\ell\to\infty$. However, we have been unsuccessful in finding a best fit formula for $C_\ell$ as a function of $\ell$ akin to~\eqref{eq:C-ell} in the case $z_0=-3/2$. 



