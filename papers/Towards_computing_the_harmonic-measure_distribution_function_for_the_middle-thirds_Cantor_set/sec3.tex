%%%%%%%%%%%%%%%%%%%%%%%%%%%%%%%%%%
\section{Step heights of the $h$-function}\label{sec:step}
%%%%%%%%%%%%%%%%%%%%%%%%%%%%%%%%%%

In this section, we determine the step heights of the $h$-functions of interest. We refer to the `step heights' as the constant values of $h(r)$ coinciding with when the capture circle passes through the gaps between the slits, for the two basepoint locations $z_0=-3/2$ and $z_0=0$ (see Figure~\ref{fig:h30}). The restriction of the $h$-function $h(r)$ to these values of $r$ will be denoted by $\omega(r)$. The step heights of the $h$-function for an example with two slits have been computed in~\cite{SnipWard16} and for two, four and eight slits in~\cite{gswc}. Furthermore, it should be pointed out that there are several methods for computing harmonic measures in multiply connected domains, each of which may alternatively be used to compute the step heights of the $h$-function; see e.g.,~\cite{CrowdyBook,Del-harm,Tre-Gre,garmar,Tre-Ser} and the references cited therein.

By the mapping function $\zeta=F^{-1}(z)$ described in \S\ref{sec:pre}, the multiply connected slit domain $\Omega_m$ (exterior to the $m$ slits $I_j$, $j=1,\ldots,m$) is mapped conformally onto a multiply connected circular domain $G_m$ exterior to $m$ disks such that each slit $I_{j}$ is mapped onto the circle $C_j$ with the center $c_j$ and radius $r_j$, $j=1,\ldots,m$. The basepoint $z_0$ is also mapped by $\zeta=F^{-1}(z)$ onto a point $\zeta_0$ in the domain $G_m$.
Owing to the symmetry of $\Omega_m$ with respect to the real axis, the centers of the disks as well as the point $\zeta_0$ are on the real line (see Figure~\ref{fig:map} for $m=4$ and $z_0=-3/2$).  


\begin{figure}[ht] %
\centerline{\hfill
\scalebox{0.4}{\includegraphics[trim=0 0 0 0,clip]{map3s}}\hfill
\scalebox{0.4}{\includegraphics[trim=0 0 0 0,clip]{map3d}}\hfill
}
\caption{The slit domain $\Omega_m$ (left) and the circular domain $G_m$ (right) with basepoint $z_0=-3/2$ in the case when $m=4$.}
\label{fig:map}
\end{figure}






\subsection{Harmonic measures}

For $k=1,\ldots,m$, let $\sigma_k$ be the harmonic measure of $C_k$ with respect to $G_m$, i.e., $\sigma_k(\zeta)$ is the unique solution of the Dirichlet problem:
\begin{subequations}\label{eq:bdv-sig}
	\begin{align}
	\label{eq:sig-Lap}
	\nabla^2 \sigma_k(\zeta) &= 0 \quad\quad \mbox{if }\zeta\in G_m, \\
	\label{eq:sig-j}
	\sigma_k(\zeta)&= \delta_{k,j} \quad \mbox{if }\zeta\in C_j, \quad j=1,\ldots,m, 
	\end{align}
\end{subequations}
where $\delta_{k,j}$ is the Kronecker delta function.
The function $\sigma_k$ is assumed to be bounded at infinity. 
The harmonic function $\sigma_k$ is the real part of an analytic function $g_k$ in $G_m$ which is not necessarily single-valued. The function $g_k$ can be written as~\cite{Gak,garmar}
\begin{equation}\label{eq:F-u}
g_k(\zeta)=b_k+f_k(\zeta)-\sum_{j=1}^{m} a_{kj}\log(\zeta-c_j)
\end{equation}
where  $c_j$ is the center of the circle $C_j$, $f_k$ is a single-valued analytic function in $G_m$ with $f_k(\infty)=0$, $b_k$ and $a_{k,j}$ are undetermined real constants such that $\sum_{j=1}^{m}a_{kj}=0$ for $k=1,\ldots,m$. The condition $\sum_{j=1}^{m}a_{kj}=0$ implies that $g_k(\infty)=b_k$ since $f_k(\infty)=0$. Since we are interested in computing the real function $\sigma_k=\Re[g_k]$, we may assume that $b_k$ is real, and then $b_k=g_k(\infty)=\sigma_k(\infty)$.

The BVP~\eqref{eq:bdv-sig} above is a particular case of the problem considered in~\cite[Eq.~(4)]{Nvm}; hence, it can be solved by the method presented in~\cite{Nvm} which is reviewed in the following subsection. 


\subsection{Computing the harmonic measures}

For $k=1,\ldots,m$, it follows from \eqref{eq:F-u} that computing the harmonic measure $\sigma_k=\Re[g_k]$ requires computing the values of the analytic function $f_k$ and the values of the $m+1$ real constants $a_{k,1},\ldots,a_{k,m},b_k$. The constants $a_{k,1},\ldots,a_{k,m}$ in~\eqref{eq:F-u} can be computed as described in Theorem~3 in~\cite{Nvm} (with $\ell=0$ and $A=1$). 
For each $j=1,2,\ldots,m$, let the function $\gamma_j$ be defined by
\begin{equation}\label{eq:gam-j}
\gamma_j(t)=\log|\eta(t)-c_j|,
\end{equation}
let $\mu_j$ be the unique solution of the boundary integral equation with the Neumann kernel
\begin{equation}\label{eq:ie}
(\bI-\bN)\mu_j=-\bM\gamma_j,
\end{equation}
and let the piecewise constant function $\nu_j=(\nu_{1,j},\nu_{2,j},\ldots,\nu_{m,j})$ be given by
\begin{equation}\label{eq:hj}
\nu_j=\left(\bM\mu_j-(\bI-\bN)\gamma_j\right)/2.
\end{equation}
Then, for each $k=1,2,\ldots,m$, the boundary values of the function $f_k(\zeta)$ in~\eqref{eq:F-u} are given by
\begin{equation}\label{eq:fk}
f_k(\eta(t))=\sum_{j=1}^m a_{kj}\left(\gamma_j(t)+\nu_j(t)+\i\mu_j(t)\right)
\end{equation}
and the $m+1$ unknown real constants $a_{k,1},\ldots,a_{k,m},b_k$ are the unique solution of the linear system
\begin{equation}\label{eq:sys-method}
\left[\begin{array}{ccccc}
\nu_{1,1}    &\nu_{1,2}    &\cdots &\nu_{1,m}      &1       \\
\nu_{2,1}    &\nu_{2,2}    &\cdots &\nu_{2,m}      &1       \\
\vdots     &\vdots     &\ddots &\vdots       &\vdots  \\
\nu_{m,1}    &\nu_{m,2}    &\cdots &\nu_{m,m}      &1       \\
1          &1          &\cdots &1            &0       \\
\end{array}\right]
\left[\begin{array}{c}
a_{k,1}    \\a_{k,2}    \\ \vdots \\ a_{k,m} \\  b_k 
\end{array}\right]
= \left[\begin{array}{c}
\delta_{k,1} \\  \delta_{k,2} \\  \vdots \\ \delta_{k,m} \\ 0  
\end{array}\right].
\end{equation}
The integral operators $\bN$ and $\bM$ in~\eqref{eq:ie} and~\eqref{eq:hj} are the same operators introduced in Section~\ref{sec:NK}.

It is clear that computing the $m+1$ unknown real constants $a_{k,1},\ldots,a_{k,m},b_k$ requires solving $m$ integral equations with the Neumann kernel~\eqref{eq:ie} and computing $m$ piecewise constant functions $\nu_j$ in~\eqref{eq:hj} for $j=1,\ldots,m$. 
This can be done using the {\sc Matlab} function \verb|fbie| as described in Section~\ref{sec:pre}. 
The complexity of solving each of these integral equations is $\mathcal{O}(mn\log n)$ operations and hence solving the $m$ integral equations~\eqref{eq:ie} requires $\mathcal{O}(m^2n\log n)$ operations.



The linear system~\eqref{eq:sys-method} has an $(m+1)\times(m+1)$ constant coefficient matrix and $m$ different right-hand sides. 
These $m$ linear systems are solved in $\mathcal{O}(m^3)$ operations by computing the inverse of the coefficient matrix and then using this inverse matrix to compute the solution for each right-hand side. In doing so, we obtain the values of the real constants $a_{k,1},\ldots,a_{k,m},b_k$. The boundary values of the analytic function $f_k$ are given by~\eqref{eq:fk}, and hence its values in the domain $G_m$ can be computed by the Cauchy integral formula. The harmonic measure $\sigma_k$ can then be computed for $\zeta\in G_m$ by
\begin{equation}\label{eq:sgj-gj}
\sigma_k(\zeta)=\Re g_k(\zeta)
\end{equation}
where $g_k(\zeta)$ is given by~\eqref{eq:F-u}. 




\subsection{Basepoint $z_0=-3/2$}

The capture circle of radius $r$ intersects with the real line at two real numbers where the largest of these numbers is denoted by $z_r$ (see Figure~\ref{fig:h30} (left)).
%Let $z_r \in \mathbb{R}$ be the point of intersection of the capture circle of radius $r$ with a slit $I_j$ for $j=1,\ldots,m$ (see Figure~\ref{fig:h30} (left)). 
By `step heights', we mean the constant values of $h(r)$ when $r$ is such that $z_r$ lies in between a pair of slits $I_{j}$ and $I_{j+1}$, $j=1,\ldots,m-1$. In addition, $h(r) = 0$ when $r$ is such that $z_r$ lies strictly to the left of all $m$ slits (i.e., $0\le r\le1$), and $h(r) = 1$ when $r$ is such that $z_r$ lies strictly to the right of all $m$ slits (i.e., $2\le r<\infty$). 

Here, we use the harmonic measures $\sigma_1,\ldots,\sigma_m$ to compute the $m-1$ step heights of the $h$-functions for the slit domains $\Omega_m$ (the exterior domain of the closed set $E_\ell$) with basepoint location $z_0=-3/2$. That is, we compute the values $\omega(r)$ of these $h$-functions at those values of $r$ corresponding to capture circles that pass through the $m-1$ gaps between the slits $I_1,\ldots,I_m$. 
In the next section, we compute the values of the $h$-functions for $\Omega_m$ for the remaining values of $r$, i.e. those values of $r$ which correspond to capture circles intersecting a slit $I_{j}$, for some $j=1,\ldots,m$. 

More precisely, the height $\omega(r)$ of the $k$th step of the $h$-function associated with $\Omega_m$ can be written in terms of the the harmonic measures $\sigma_1,\ldots,\sigma_m$ as 
\begin{equation}\label{eq:wr-3}
\omega(r)=\sum_{j=1}^{k}\sigma_j(\zeta_0), \quad k=1,2,\ldots,m-1,
\end{equation}
where $\zeta_0=F^{-1}(z_0)$, $\sigma_j$ is the harmonic measure of $C_j$ with respect to $G_m$, and $k$ is the largest integer such that the slit $I_{k}$ is inside the circle $|z-z_0|=r$ and the slit $I_{k+1}$ is outside the circle $|z-z_0|=r$.
The computed numerical values for the step heights of the $h$-function for the domains $\Omega_2$, $\Omega_4$, and $\Omega_8$ are presented in Table~\ref{tab:3}. These values agree with the values computed by the method presented in~\cite{gswc}. The plots of the step height as a function of $r$ for the domains $\Omega_{64}$ and $\Omega_{1024}$ are shown in Figure~\ref{fig:hm-3}.

\begin{table}[h]
  \caption{The $m-1$ step heights of the $h$-function for the domains $\Omega_m$ when $z_0=-3/2$ for $m=2,4,8$, computed using the proposed method and the method presented in~\cite{gswc}.}
	\label{tab:3}
	\centering
	\begin{tabular}{lll|lll}  \hline
		 \multicolumn{3}{c|}{The proposed method}  & \multicolumn{3}{c}{Green \emph{et al.}~\cite{gswc}}  \\ \hline
              &              & $0.23081722$  &              &              & $0.23081722$  \\ 
              & $0.37725094$ & $0.37469279$  &              & $0.37725094$ & $0.37469279$ \\
              &              & $0.48515843$  &              &              & $0.48515843$ \\
 $0.60527819$ & $0.60254652$ & $0.60117033$  & $0.60527819$ & $0.60254652$ & $0.60117033$ \\
              &              & $0.70056784$  &              &              & $0.70056784$ \\
              & $0.78306819$ & $0.78288753$  &              & $0.78306819$ & $0.78288753$ \\
              &              & $0.87276904$  &              &              & $0.87276904$ \\
		\hline
	\end{tabular}
\end{table}

\begin{figure}[ht] %
\centerline{
\scalebox{0.6}{\includegraphics[trim=0 0 0 0,clip]{hmfun64}}
\hfill
\scalebox{0.6}{\includegraphics[trim=0 0 0 0,clip]{hmfun1024}}
}
\caption{The step height of the $h$-function for the domain $\Omega_{64}$ (left) and $\Omega_{1024}$ (right) when $z_0=-3/2$.}
\label{fig:hm-3}
\end{figure}


\subsection{Basepoint $z_0=0$}

We now compute the step heights of the $h$-functions for $\Omega_m$ with basepoint location $z_0=0$. That is, we compute the values of these $h$-functions at those values of $r$ corresponding to capture circles that pass through gaps between the slits $I_{k}$ for $k=1,\ldots,m$.	
In this case, the capture circle of radius $r$ intersects with the real line at two real numbers $\pm z_r$ where $z_r>0$ (see Figure~\ref{fig:h30} (right)).
Here, $h(r) = 0$ when $r$ is such that $z_r$ lies strictly in-between the two slits $I_{m/2}$ and $I_{m/2+1}$ (i.e., $0\le r\le1/6$), and $h(r) = 1$ when $r$ is such that $z_r$ lies strictly to the right of all $m$ slits (i.e., $1/2\le  r<\infty$). 

For this case, the height $\omega(r)$ of the $k$th step of the $h$-function associated with $\Omega_m$ is thus
\begin{equation}\label{eq:wr-0}
\omega(r)=\sum_{j=1}^{k}\left(\sigma_{\frac{m}{2}-j+1}(\zeta_0)+\sigma_{\frac{m}{2}+j}(\zeta_0)\right)
\end{equation}
where $\zeta_0=F^{-1}(z_0)$, $\sigma_j$ is the harmonic measure of $C_j$ with respect to $G_m$, and $k$ is the largest integer such that the two slits $I_{m/2-j+1}$ and $I_{m/2+j}$ are inside the circle $|z-z_0|=r$ and the two slits $I_{m/2-j}$ and $I_{m/2+j+1}$ are outside the circle $|z-z_0|=r$.
The computed numerical values for the step heights of the $h$-function for the domains $\Omega_4$, $\Omega_8$, and $\Omega_{16}$ are presented in Table~\ref{tab:0}. The plots of the step height as a function of $r$ for the domains $\Omega_{64}$ and $\Omega_{2048}$ are shown in Figure~\ref{fig:hm-0}.


\begin{table}[h]
  \caption{The $m/2-1$ step heights of the $h$-function for the domains $\Omega_m$ when $z_0=0$ for $m=4,8,16$.}
  \label{tab:0}
\centering
\begin{tabular}{ccc}
    \hline
  $\Omega_4$       & $\Omega_8$        & $\Omega_{16}$  \\ \hline
                   &                   & $0.32412730$  \\
                   & $0.50657767$      & $0.50171513$  \\
                   &                   & $0.61794802$  \\
 $0.73555154$      & $0.72992958$      & $0.72702487$  \\
                   &                   & $0.80333761$  \\
                   & $0.86334249$      & $0.86206402$  \\
                   &                   & $0.92098300$  \\
\hline
\end{tabular}
\end{table}




\begin{figure}[ht] %
\centerline{
\scalebox{0.6}{\includegraphics[trim=0 0 0 0,clip]{hmfunO64}}
\hfill
\scalebox{0.6}{\includegraphics[trim=0 0 0 0,clip]{hmfunO2048}}}
\caption{The step height of the $h$-function for the domain $\Omega_{64}$ (left) and $\Omega_{2048}$ (right) when $z_0=0$.}
\label{fig:hm-0}
\end{figure}
