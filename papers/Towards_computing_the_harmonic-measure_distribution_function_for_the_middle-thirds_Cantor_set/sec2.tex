%%%%%%%%%%%%%%%%%%%%%%%%%%%%%%%%%%
\section{The conformal mapping}\label{sec:cm}
%%%%%%%%%%%%%%%%%%%%%%%%%%%%%%%%%%

Computing the values of the $h$-function requires solving the BVP~\eqref{eq:bdv-u} in the unbounded multiply connected domain $\Omega_m$ bordered by slits.
In this section, we review an iterative conformal mapping method from~\cite{AST13,NG18} which `opens up' the slits $I_k$ and consequently maps the domain $\Omega_m$ onto an unbounded multiply connected circular domain $G_m$. 
The BVP~\eqref{eq:bdv-u} will then be transformed into an equivalent problem in the circular domain $G_m$ where it can be solved in a straightforward manner. Solving the transformed problem in the new circular domain $G_m$ will be discussed in Sections~\ref{sec:step} and~\ref{sec:h-fun} below. 

\subsection{The Neumann kernel}\label{sec:NK}

Let $G_m$ be an unbounded multiply connected circular domain obtained by removing $m$ non-overlapping disks from the extended complex plane $\overline{\CC}$. The boundaries of these disks are the circles $C_j$ with centers $c_j$ and radii $r_j$, $j=1,\ldots,m$. 
We parametrize each circle $C_j$ by 
\begin{equation}\label{eq:eta-j}
\eta_j(t)=c_j+r_j e^{-\i t}, \quad t\in J_j=[0,2\pi], \quad j=1,\ldots,m.
\end{equation}
We define the total parameter domain $J$ as the disjoint union of the $m$ intervals $J_j=[0,2\pi]$, $j=1,\ldots,m$. Thus, the whole boundary $\Gamma$ is parametrized by
\begin{equation}\label{eq:eta}
\eta(t)= \left\{ \begin{array}{l@{\hspace{0.5cm}}l}
\eta_1(t),&t\in J_1, \\
\quad\vdots & \\
\eta_m(t),&t\in J_m.
\end{array}
\right.
\end{equation}
See~\cite{Nas-ETNA,NG18} for more details.

The Neumann kernel $N(s,t)$ is defined for $(s,t)\in J\times J$ by
\begin{equation}\label{eq:N}
N(s,t) =
\frac{1}{\pi}\Im\left(\frac{\eta'(t)}{\eta(t)-\eta(s)}\right).
\end{equation}
It is a particular case of the generalized Neumann kernel considered in~\cite{Weg-Nas} when $A(t)=1$, $t\in J$.
We also define the following kernel 
\begin{equation}\label{eq:M}
M(s,t) =
\frac{1}{\pi}\Re\left(\frac{\eta'(t)}{\eta(t)-\eta(s)}\right), \quad (s,t)\in J\times J.
\end{equation}
which is a particular case of the kernel $M$ considered in~\cite{Weg-Nas} when $A(t)=1$.
The kernel $N(s,t)$ is continuous and the kernel $M(s,t)$ is singular.
Hence, the integral operator 
\[
\bN\mu(s) = \int_J N(s,t) \mu(t) dt, \quad s\in J,
\]
is compact and the integral operator 
\[
\bM\mu(s) = \int_J  M(s,t) \mu(t) dt, \quad s\in J,
\]
is singular. Further details can be found in~\cite{Weg-Nas}.


%---------------------------------------
\subsection{The preimage domain}\label{sec:pre}

Let $\Omega_m$ be the multiply connected slit domain obtained by removing the $m$ horizontal slits $I_1,\ldots,I_m$, described in Section~\ref{sec:int}, from the extended complex plane $\overline{\CC}$. In this subsection, we will summarize  an iterative method from~\cite{AST13,NG18} for the construction of a preimage unbounded circular domain $G_m$. The method has been tested numerically in several works~\cite{LSN17,NK,Nvm}.

For a fixed $\ell=0,1,2,\ldots$, the closed set $E_\ell$, defined in~\eqref{eq:Ek}, consists of $m=2^\ell$ slits, $I_1,\ldots,I_m$, of equal length $L=(1/3)^\ell$ and with centers $w_j$ for $j=1,\ldots,m$. Our objective now is to find the centers $c_j$ and radii $r_j$ of non-overlapping circles $C_j$ and a conformal mapping $z=F(\zeta)$ from the circular domain $G_m$ exterior to $C=\cup_{j=1}^m C_j$ onto $\Omega_m$. With the normalization
\[
F(\infty)=\infty, \quad \lim_{\zeta\to\infty}(F(\zeta)-\zeta)=0,
\]
such a conformal mapping is unique.

The conformal mapping $z=F(\zeta)$ can be computed using the following boundary integral equation method from~\cite{Nas-Siam1}. Let the function $\gamma$ be defined by
\begin{equation}\label{eq:gam}
\gamma(t)=\Im\eta(t), \quad t\in J.
\end{equation}
Let also $\mu$ be the unique solution of the boundary integral equation with the Neumann kernel
\begin{equation}\label{eq:ie-g}
(\bI-\bN)\mu=-\bM\gamma,
\end{equation}
and let the piecewise constant function $\nu=(\nu_1,\nu_2,\ldots,\nu_\ell)$ be given by
\begin{equation}\label{eq:h-g}
\nu=\left(\bM\mu-(\bI-\bN)\gamma\right)/2,
\end{equation}
i.e., the function $\nu$ is constant on each boundary component $C_j$ and its value on $C_j$ is a real constant $\nu_j$.
Then the function $f$ with the boundary values
\begin{equation}\label{eq:f-rec}
f(\eta(t))=\gamma(t)+\nu(t)+\i\mu(t)
\end{equation}
is analytic in $G_m$ with $f(\infty)=0$, and the conformal mapping $F$ is 
given by
\begin{equation}\label{eqn:omega-app}
F(\zeta)=\zeta-\i f(\zeta), \quad \zeta\in G_m\cup\Gamma.
\end{equation} 



The application of this method requires that the domain $G_m$ is known. However, in our case, the slit domain $\Omega_m$ is known and the domain $G_m$ is unknown and needs to be determined alongside the conformal mapping $z=F(\zeta)$ from $G_m$ onto $\Omega_m$. 
This preimage domain $G_m$ as well as the conformal mapping $z=F(\zeta)$ will be computed using the iterative method presented in~\cite{NG18} (see also~\cite{AST13}). 
For the convenience of the reader and for the completeness of this paper, we now briefly review a slightly modified version of this iterative method. 
In~\cite{NG18}, the boundary components of the preimage domain $G_m$ are assumed to be ellipses. Here, the slits $I_{j}$, $j=1,2,\ldots,m$, are well-separated and hence the boundary components of the preimage domain $G_m$ are assumed to be circles. The method generates a sequence of multiply connected circular domains $G_m^0,G_m^1,G_m^2,\ldots,$ which converge numerically to the required preimage domain $G_m$. 
Recall that the length of each slit $I_{j}$ is $L$ and the center of $I_{j}$ is $w_j$ for $j=1,\ldots,m$. 
In the iteration step $i=0,1,2,\ldots$, we assume that $G_m^{(i)}$ is bordered by the $m$ circles $C^{(i)}_1,\ldots,C^{(i)}_m$ parametrized by
\begin{equation}\label{eq:eta-i}
\eta^{(i)}_j(t)=c^{(i)}_j+r^{(i)}_je^{\i t}, \quad 0\le t\le 2\pi, \quad j=1,\ldots,m.
\end{equation}
The centers $c^{(i)}_j$ and radii $r^{(i)}_j$ are computed using the following iterative method:
\begin{enumerate}
	\item Set
	\[
	c^{(0)}_j=w_j, \quad r^{(0)}_j=\frac{\,L\,}{2}, \quad j=1,\ldots,m.
	\]
	\item For $i=1,2,3,\ldots,$
	\begin{itemize}
		\item Compute the conformal mapping from the preimage domain $G^{(i-1)}$ to a canonical horizontal rectilinear slit domain $\Omega^{(i)}$ which is the entire $\zeta$-plane with $m$ horizontal slits $I^{(i)}_{j}$, $j=1,\ldots,m$ (using the method presented in equations~\eqref{eq:gam}--\eqref{eqn:omega-app} above). Let $L^{(i)}_j$ denote the length of the slit $I^{(i)}_{j}$ and let $w^{(i)}_j$ denote its center.
		\item Define	
\[
c^{(i)}_j = c^{(i-1)}_j-(w^{(i)}_j-w_j), \quad 
r^{(i)}_j = r^{(i-1)}_j-\frac{\,1\,}{4}(L^{(i)}_j -L), \quad j=1,\ldots,m.
\]	
	
	
	\end{itemize}
	\item Stop the iteration if 
	\[
	\frac{1}{2m}\sum_{j=1}^{m}\left(|w^{(i)}_j - w_j|+|L^{(i)}_j -L|\right)<\varepsilon \quad{\rm or}\quad i>{\tt Max}
	\]
	where $\varepsilon$ is a given tolerance and ${\tt Max}$ is the maximum number of iterations allowed.		
\end{enumerate}

The above iterative method generates sequences of parameters $c^{(i)}_j$ and $r^{(i)}_j$ that converge numerically to $c_j$ and $r_j$, respectively, and then the boundary components of the preimage domain $G_m$ are parametrized by~\eqref{eq:eta-j}. In our numerical implementations, we used $\varepsilon=10^{-14}$ and ${\tt Max}=100$.


It is clear that in each iteration of the above method, it is required to solve the integral equation with the Neumann kernel~(\ref{eq:ie-g}) and to compute the function $\nu$ in~(\ref{eq:h-g}) which can be done with the fast method presented in~\cite{Nas-ETNA} by applying the {\sc Matlab} function \verb|fbie|. In \verb|fbie|, the integral equation is discretized by the Nystr\"om method using the trapezoidal rule with $n$ equidistant nodes
\begin{equation}\label{eq:sji}
	s_{j,i}=(i-1)\frac{2\pi}{n}\in J_j=[0,2\pi], \quad i=1,2,\ldots,n, 
\end{equation}
for each sub-interval $J_j$, $j=1,2,\ldots,m$.
This yields an $mn\times mn$ linear system which is solved by the generalized minimal residual (GMRES) method through the {\sc Matlab} function \verb|gmres| where the matrix-vector product in GMRES is computed using the {\sc Matlab} function \verb|zfmm2dpart| from the fast multipole method (FMM) toolbox~\cite{Gre-Gim12}. 

The boundary components of $\Omega_m$, i.e., the $m$ slits $I_1,\ldots,I_m$, are well-separated and hence the circles $C_1,\ldots,C_m$, which are the boundary components of the domain $G_m$, are also well-separated (see Figure~\ref{fig:map} for $m=4$). Thus, accurate results can be obtained even with small values of $n$~\cite{Nvm}. 
In our numerical computations, we used $n=16$.
For the {\sc Matlab} function \verb|zfmm2dpart|, we chose the FMM precision flag \verb|iprec=5| which means that the tolerance of the FMM method is $0.5\times10^{-15}$. 
In the function \verb|gmres|, we chose the tolerance of the method to be $10^{-13}$, the maximum number of iterations to be $100$, and we apply the method without restart. The complexity of solving this integral equation, and hence the complexity of each iteration in the above iterative method, is $\mathcal{O}(mn\log n)$ operations.
For more details, we refer the reader to~\cite{Nas-ETNA}.


