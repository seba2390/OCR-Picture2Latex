%%%%%%%%%%%%%%%%%%%%%%%%%%%%%%%%%%
\section{Conclusions}\label{sec:con}
%%%%%%%%%%%%%%%%%%%%%%%%%%%%%%%%%%

In this paper, we have presented a fast and accurate numerical method for approximating the $h$-function for the middle-thirds Cantor set $\mathcal{C}$. We achieved this by computing $h$-functions for multiply connected slit domains $\Omega_m$ of high connectivity. Computing the $h$-functions for the slit domains $\Omega_m$ requires solving Dirichlet BVPs on these domains. 
We opted to map the slit domains $\Omega_m$ onto conformally equivalent circular domains $G_m$ where the transformed Dirichlet problems are solved and the $h$-functions are then calculated. 
Computing the conformal mapping from $\Omega_m$ onto $G_m$ as well as solving the transformed Dirichlet BVPs in the domain $G_m$ are performed using  the BIE with the Neumann kernel~\eqref{eq:ie-g}. 
This is the same BIE which was also used in~\cite{LSN17} to approximate the logarithmic capacity of $\mathcal{C}$.

We presented results for the step heights of the $h$-function up to connectivity $m=1024$ for the basepoint $z_0=-3/2$ and $m=2048$ for the basepoint $z_0=0$ (see Figures~\ref{fig:hm-3} and~\ref{fig:hm-0}). Our method could certainly be used for larger values of $m$, with still arguably competitive computation times. It is also important to point out that the actual graph of the $h$-function associated with $\mathcal{C}$, regardless of basepoint location, will consist of a union of infinitely many points and constant intervals. However, the main qualitative features of the $h$-functions we have presented for slit domains of high connectivity, such as those in Figures~\ref{fig:hm-3} and~\ref{fig:hm-0}, will differ only very slightly to those exhibited by the actual graph of the $h$-function associated with $\mathcal{C}$. 



%%%%%%%%%%%%%%%%%%%%%%%%%%%%%%%%%%
\section*{Acknowledgements}
%%%%%%%%%%%%%%%%%%%%%%%%%%%%%%%%%%
We thank Lesley Ward and Marie Snipes who suggested to us the topic of this paper. CCG is grateful for the hospitality of the Department of Mathematics, Statistics \& Physics at Qatar University where this work was initiated. CCG also acknowledges support from an Australian Research Council DECRA grant (DE180101098).

