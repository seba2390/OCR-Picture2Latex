%%%%%%%%%%%%%%%%%%%%%%%%%%%%%%%%%%
\section{Numerical computation of the $h$-function}\label{sec:h-fun}
%%%%%%%%%%%%%%%%%%%%%%%%%%%%%%%%%%


The step heights $\omega(r)$ of the $h$-function, i.e., the values of the $h$-function $h(r)$ coinciding with when the capture circle passes through the gaps between the slits $I_k$, $k=1,\ldots,m$, have been computed in the previous section. In this section, we present a method for computing the values of the $h$-function $h(r)$ when the capture circle intercepts with the slits $I_k$ for the two cases of basepoints considered in the previous section.  


\subsection{Basepoint $z_0=-3/2$}

Assume that the capture circle intercepts the slit $I_k$ for some $k=1,...,m$. 
Owing to the up-down symmetry of the domain $\Omega_m$ in the $z$-plane, we can assume that the pair of preimages of the intersection point of the capture circle with the slit $I_k$ are the points $\xi$ and $\overline{\xi}$ on the circle $C_k$.
Let us also assume that $\xi_1$ is the intersection of the circle $C_k$ with the real axis lying on the arc between $\overline{\xi}$ and $\xi$ that is closest to the point $\zeta_0$. This arc will be denoted by $C'_k$. The remaining part of the circle is denoted by $C''_k$ (see Figure~\ref{fig:h3d}). 
Let the function $U_k(\zeta)$ be the unique solution of the Dirichlet problem:
\begin{subequations}\label{eq:bdv-U3}
	\begin{align}
	\label{eq:U3-Lap}
	\nabla^2 U_k(\zeta) &= 0 \quad \mbox{if }\zeta\in G_m, \\
	\label{eq:U3-m}
	U_k(\zeta)&= 1 \quad \mbox{if }\zeta\in C_j, \quad j=1,\ldots,k-1, \\
	\label{eq:U3-k'}
	U_k(\zeta)&= 1 \quad \mbox{if }\zeta\in C'_k,  \\
	\label{eq:U3-k''}
	U_k(\zeta)&= 0 \quad \mbox{if }\zeta\in C''_k,  \\
	\label{eq:U3-p}
	U_k(\zeta)&= 0 \quad \mbox{if }\zeta\in C_j, \quad j=k+1,\ldots,m, 
	\end{align}
\end{subequations}
where the function $U_k$ is assumed to be bounded at infinity. 

\begin{figure}[ht] %
\centerline{
\scalebox{0.4}{\includegraphics[trim=0 0 0 0,clip]{figh3d}}
}
\caption{The arcs $C'_k$ and $C''_k$ for $z_0=-3/2$.}
\label{fig:h3d}
\end{figure}


Note that the boundary conditions on the circle $C_k$ are not continuous. To remove this discontinuity, let us now introduce a function $\Psi_k$ in terms of the two M\"obius maps
\begin{equation}
\psi(\zeta,\xi,\xi_1)=\frac{(\zeta-\xi)(\xi_1-\overline{\xi})+\i(\zeta-\overline{\xi})(\xi_1-\xi)}{(\zeta-\overline{\xi})(\xi_1-\xi)+\i(\zeta-\xi)(\xi_1-\overline{\xi})}, \quad
\phi(w)=\frac{w-\i}{\i w-1}.
\end{equation}
Note that $\psi(\zeta,\xi,\xi_1)$ maps the exterior of the unit disc onto the unit disk such that the three points $\overline{\xi}$, $\xi_1$, and $\xi$ on the unit circle are mapped onto the three points $-\i$, $1$, and $\i$, respectively, and $\phi(w)$ maps the unit disc onto the upper half-plane such that the three points $-\i$, $1$, and $\i$ are mapped onto the three points $\infty$, $-1$, and $0$, respectively.
Then, for $\xi\in C_k$, we define 
\begin{equation}\label{eq:Psi}
\Psi_k(\zeta)=\frac{1}{\pi}\Im \log \phi(\psi(\zeta,\xi,\xi_1)), \quad \zeta\in G_m.
\end{equation}
This function $\Psi_k(\zeta)$ is harmonic everywhere in the domain $G_m$ exterior to the $m$ discs, bounded at infinity, equal to 1 on the arc $C'_k$, and equal to $0$ on the arc $C''_k$.
Hence, for $k=1,\ldots,m$, the function 
\begin{equation}\label{eq:Uk-Vk}
V_k(\zeta)=U_k(\zeta)-\Psi_k(\zeta)
\end{equation}
is bounded at infinity and satisfies the following Dirichlet problem:
\begin{subequations}\label{eq:bdv-V3}
	\begin{align}
	\label{eq:V3-Lap}
	\nabla^2 V_k(\zeta) &= 0  ~~~~~~~~~~~~~~\,    \mbox{if }\zeta\in G_m, \\
	\label{eq:V3-m}
	V_k(\zeta)&= 1-\Psi_k(\zeta) \quad   \mbox{if }\zeta\in C_j, \quad j=1,\ldots,k-1, \\
	\label{eq:V3-k'}
	V_k(\zeta)&= 0         ~~~~~~~~~~~~~~\,        \mbox{if }\zeta\in C_k,  \\
	\label{eq:V3-p}
	V_k(\zeta)&= -\Psi_k(\zeta)  ~~~~~~  \mbox{if }\zeta\in C_j, \quad j=k+1,\ldots,m. 
	\end{align}
\end{subequations}
The boundary conditions of the new problem~\eqref{eq:bdv-V3} are now continuous on all circles. 




Note that $\phi(\psi(\zeta,\xi,\xi_1))$ is itself a M\"obius map which maps the circle $C_k$ onto the real line. For $j=1,\ldots,m$, $j\ne k$, it  maps the circle $C_j$ onto a circle of very small radius in the upper half-plane (see Figure~\ref{fig:small} (left)). Hence in problem~\eqref{eq:bdv-V3}, the values of the function on the right-hand side of the boundary conditions are almost constant (see Figure~\ref{fig:small} (right)). As such, and in view of (\ref{eq:Psi}), we may approximate
\begin{equation}
\Psi_k(\zeta) \approx P_{kj}= \Psi_k(c_j), \quad \zeta \in C_j, \quad j=1,\ldots,m,
\end{equation} 
where $c_j$ is the center of the circle $C_j$. Thus, the Dirichlet problem~\eqref{eq:bdv-V3} becomes
\begin{subequations}\label{eq:bdv-V23}
	\begin{align}
	\label{eq:V23-Lap}
	\nabla^2 V_k(\zeta) &= 0 ~~~~~~~~~~\:\: \mbox{if }\zeta\in G_m, \\
	\label{eq:V23-m}
	V_k(\zeta)&= 1-P_{kj} \quad \mbox{if }\zeta\in C_j, \quad j=1,\ldots,k-1, \\
	\label{eq:V23-k'}
	V_k(\zeta)&= 0 ~~~~~~~~~~\:\: \mbox{if }\zeta\in C_k,  \\
	\label{eq:V23-p}
	V_k(\zeta)&= -P_{kj}  ~~~~\:\:\: \mbox{if }\zeta\in C_j, \quad j=k+1,\ldots,m. 
	\end{align}
\end{subequations}
The function $V_k$ can be then written in terms of the harmonic measures $\sigma_1,\ldots,\sigma_m$ through
\[
V_k(\zeta)=\sum_{j=1}^{k-1} (1-P_{kj})\sigma_j(\zeta)-\sum_{j=k+1}^{m} P_{kj}\sigma_j(\zeta).
\]
Thus, by~\eqref{eq:Uk-Vk}, the unique solution $U_k(\zeta)$ to the Dirichlet problem~\eqref{eq:bdv-U3} is given by
\begin{equation}\label{eq:U_k}
	U_k(\zeta) = \Psi_k(\zeta)+\sum_{j=1}^{k-1} \sigma_j(\zeta) -\sum_{\begin{subarray}{c} j=1\\j\ne k\end{subarray}}^{m} P_{kj}\sigma_j(\zeta).
\end{equation}


\begin{figure}[htb] %
	\centerline{\hfill
		\scalebox{0.4}{\includegraphics[trim=0 0 0 0,clip]{figh3smallcir1}}
		\hfill
		\scalebox{0.4}{\includegraphics[trim=0 0 0 0,clip]{figh3smallcir2}}\hfill
	}
	\caption{The image of the circles $C_j$ (left) and the values of the functions in the right-hand side of the boundary conditions on~\eqref{eq:bdv-V3} (right) for $m=8$, $k=1$, and $\xi=c_k+r_k e^{3\pi\i/4}\in C_k$.}
	\label{fig:small}
\end{figure}

Due to the symmetry of both domains $\Omega_m$ and $G_m$ with respect to the real line and since the image of the circle $C_k$ under the conformal mapping $F$ is the slit $I_k$, we have $F(\xi)=F(\overline{\xi})\in I_k$ and the image of the arc $C_k'$ is the part of $I_k$ lying to the left of $F(\xi)$ (see Figure~\ref{fig:3hr}). 
For a given $r$ such that the capture circle passes through the slit $I_k$, for $k=1,...,m$, then $\xi$ is on the circle $C_k$ and depends on $r$ as (see Figure~\ref{fig:3hr}) 
\begin{equation}\label{eq:r-xi}
r= F(\xi)-z_0.
\end{equation}
For this value of $r$, the value of the $h$-function is then given by
\begin{equation}\label{eq:hr3-xi}
h(r)=U_k(\zeta_0).
\end{equation}


In~\eqref{eq:r-xi}, we may assume that $r$ is given and then solve the non-linear equation~\eqref{eq:r-xi} for $\xi\in C_k$. Alternatively, we may assume that $\xi$ on the circle $C_k$ is given and then compute the values of $r$ through~\eqref{eq:r-xi}, which is simpler. 
For computing the value of $r$ in~\eqref{eq:r-xi}, the value of the mapping function $F(\xi)$ can be approximated numerically for $\xi\in C_k$ as follows.
Since $\xi\in C_k$, then it follows from the parametrization~\eqref{eq:eta-j} of the circle $C_k$ that $\xi=\eta_k(s)$ for some $s\in J_k=[0,2\pi]$. Then, it follows from the method described in Section~\ref{sec:pre} that $F(\xi)=\xi-\i f(\xi)$ where
\[
f(\xi)=f(\eta_k(s))=\gamma_k(s)+h_k(s)+\i\mu_k(s)
\]
where $\gamma_k$, $h_k$, and $\mu_k$ are the restrictions of the functions $\gamma$, $h$, and $\mu$ to the interval $J_k=[0,2\pi]$. Note that, by~\eqref{eq:gam}, $\gamma_k(s)=\Im\eta_k(s)$ is known and $h_k(s)=h_k$ where the constant $h_k$ is also known. However, the value of $\mu_k(s)$ will be known only if $s$ is one of the discretization points $s_{k,i}$, $i=1,\ldots,n$, given by~\eqref{eq:sji}. If $s$ is not one of these points, then the value of $\mu_k(s)$ can be approximated using the Nystr{\"o}m interpolation formula (see~\cite{Atk97}). Here, we will approximate $\mu_k(s)$ by finding a trigonometric interpolation polynomial that interpolates the function $\mu_k$ at its known values $\mu_k(s_{k,i})$, $i=1,\ldots,n$. The polynomial can be then used to approximate $\mu_k(s)$ for any $s\in[0,2\pi]$. Finding this polynomial and computing its values is done using the fast Fourier transform (FFT).


In our numerical computations below, to compute the non-constant components of the $h$-function, i.e., to compute the values of $h(r)$ when the capture circle intercept with a slit $I_k$ for $k=1,\ldots,m$, we choose $31$ equidistant values of $\xi$ on the upper half of the circle $C_k$. For each of these points $\xi$, we compute the value of $r$ through~\eqref{eq:r-xi} and the values of $h(r)$ through~\eqref{eq:hr3-xi}. 
The values of $h(r)$ for $r$ corresponding to the capture circle passing through the gaps between the slits $I_k$, $k=1,\ldots,m$, are computed as described earlier in Section~\ref{sec:step}.
The graphs of the function $h(r)$ for $m=16$ and $m=32$ are given in Figure~\ref{fig:h-3}.


\begin{figure}[htb] %
	\centerline{\hfill
		\scalebox{0.4}{\includegraphics[trim=0 0 0 0,clip]{figh3r}}
		\hfill
		\scalebox{0.4}{\includegraphics[trim=0 0 0 0,clip]{figh3dr}}\hfill
	}
	\caption{Computing the value of $r$ in~\eqref{eq:r-xi}.}
	\label{fig:3hr}
\end{figure}



\begin{figure}[htb] %
\centerline{
\scalebox{0.6}{\includegraphics[trim=0 0 0 0,clip]{hfun16}}
\hfill
\scalebox{0.6}{\includegraphics[trim=3 0 0 0,clip]{hfun32}}
}
\caption{The $h$-function for the domain $\Omega_{16}$ (left) and the domain $\Omega_{32}$ (right) when $z_0=-3/2$.}
\label{fig:h-3}
\end{figure}


\subsection{Basepoint $z_0=0$}

For the basepoint $z_0=0$, we assume that the capture circle intercepts the two slits $I_{\frac{m}{2}-k+1}$ and $I_{\frac{m}{2}+k}$ for some $k=1,\ldots,\frac{m}{2}$. Again, due to the symmetry of $G_m$, let $\xi$ and $\overline{\xi}$ be the pair of preimages of the intersection of the capture circle with the slit $I_{\frac{m}{2}+k}$. Also let $\xi_1$ be the intersection of the circle $C_{\frac{m}{2}+k}$ with the positive real axis lying on the arc between $\overline{\xi}$ and $\xi$ which is closest to the point $\zeta_0$. It follows from the symmetry of $G_m$ and $\Omega_m$ that $-\overline{\xi}$ and $-\xi$ are the corresponding pair of preimages of the intersection of the capture circle with the circle $C_{\frac{m}{2}-k+1}$. See Figure~\ref{fig:h0d}.

\begin{figure}[htb] %
\centerline{
\scalebox{0.4}{\includegraphics[trim=0 0 0 0,clip]{figh0d}}
}
\caption{The arcs $C'_{\frac{m}{2}+k}$, $C''_{\frac{m}{2}+k}$, $C'_{\frac{m}{2}-k+1}$ and $C''_{\frac{m}{2}-k+1}$ for $z_0=0$.}
\label{fig:h0d}
\end{figure}

For $k=1,\ldots,m/2$, the function $U_k(\zeta)$ is the unique solution of the following Dirichlet problem:
\begin{subequations}\label{eq:bdv-U0}
	\begin{align}
	\label{eq:U-Lap}
	\nabla^2 U_k(\zeta) &= 0 \quad \mbox{if }\zeta\in G_m, \\
	\label{eq:U-m}
	U_k(\zeta)&= 1 \quad \mbox{if }\zeta\in C_{\frac{m}{2}-j+1}\cup C_{\frac{m}{2}+j}, \quad j=1,\ldots,k-1, \\
	\label{eq:U-k'}
	U_k(\zeta)&= 1 \quad \mbox{if }\zeta\in C'_{\frac{m}{2}-k+1}\cup C'_{\frac{m}{2}+k},  \\
	\label{eq:U-k''}
	U_k(\zeta)&= 0 \quad \mbox{if }\zeta\in C''_{\frac{m}{2}-k+1}\cup C''_{\frac{m}{2}+k},  \\
	\label{eq:U-p}
	U_k(\zeta)&= 0 \quad \mbox{if }\zeta\in C_{\frac{m}{2}-j+1}\cup C_{\frac{m}{2}+j}, \quad j=k+1,\ldots,m/2. 
	\end{align}
\end{subequations}
Here, $C'_{\frac{m}{2}+k}$ is the arc joining $\overline{\xi}$, $\xi_1$, and $\xi$ on the circle $C_{\frac{m}{2}+k}$ and $C''_{\frac{m}{2}+k}$ is the adjacent arc; $C'_{\frac{m}{2}-k+1}$ is the arc joining $-\overline{\xi}$, $-\xi_1$, and $-\xi$ on the circle $C_{\frac{m}{2}-k+1}$ and $C''_{\frac{m}{2}-k+1}$ is the adjacent arc. The function $U_k(\zeta)$ is assumed to be bounded at infinity.

Note that the boundary conditions on the two circles $C_{\frac{m}{2}-k+1}$ and $C_{\frac{m}{2}+k}$ are again not continuous. In a similar fashion to the problem~\eqref{eq:bdv-U3}, we reduce the current problem~\eqref{eq:bdv-U0} to one with continuous boundary data. 
For $k=1,\ldots,m/2$, the function $\Psi_k(\zeta)$ defined by~\eqref{eq:Psi} is harmonic everywhere in the domain exterior to the $m$ discs, equal to 1 on the arc $C'_{\frac{m}{2}+k}$, and equal to $0$ on the arc $C''_{\frac{m}{2}+k}$.
Similarly, the function 
\begin{equation}\label{eq:Phi_k}
\Phi_k(\zeta)=\frac{1}{\pi}\Im \log \phi(\psi(\zeta,-\xi,-\xi_1)), 
\end{equation}
is harmonic everywhere in the domain exterior to the $m$ discs, equal to 1 on the arc $C'_{\frac{m}{2}-k+1}$, and equal to $0$ on the arc $C''_{\frac{m}{2}-k+1}$.
Thus, the function 
\begin{equation}\label{eq:Vk-Uk-0}
V_k(\zeta)=U_k(\zeta)-\Psi_k(\zeta)-\Phi_k(\zeta)
\end{equation}
is a solution of the following Dirichlet problem:
\begin{subequations}\label{eq:bdv-V0}
	\begin{align}
	\label{eq:V-Lap}
	\nabla^2 V_k(\zeta) &= 0 ~~~~~~~~~~~~~~~~~~~~~~~\:\; \mbox{if }\zeta\in G_m, \\
	\label{eq:V-m}
	V_k(\zeta)&= 1-\Psi_k(\zeta)-\Phi_k(\zeta) ~~ \mbox{if }\zeta\in C_{\frac{m}{2}-j+1}\cup C_{\frac{m}{2}+j}, ~ j=1,\ldots,k-1, \\
	\label{eq:V-k'}
	V_k(\zeta)&= -\Phi_k(\zeta) ~~~~~~~~~~~~~~~\:\: \mbox{if }\zeta\in C_{\frac{m}{2}+k},  \\
	\label{eq:V-k''}
	V_k(\zeta)&= -\Psi_k(\zeta) ~~~~~~~~~~~~~~~\:\: \mbox{if }\zeta\in C_{\frac{m}{2}-k+1},  \\
	\label{eq:V-p}
	V_k(\zeta)&= -\Psi_k(\zeta)-\Phi_k(\zeta)  ~~~~~ \mbox{if }\zeta\in C_{\frac{m}{2}-j+1}\cup C_{\frac{m}{2}+j}, ~ j=k+1,\ldots,m/2. 
	\end{align}
\end{subequations}
The function $V_k$ is bounded at infinity.
The boundary data of the new problem~\eqref{eq:bdv-V0} is now continuous on all circles. 

As before, it can be observed also that the image circles under the mappings $\phi(\psi(\zeta,\xi,\xi_1))$ and $\phi(\psi(\zeta,-\xi,-\xi_1))$ are very small, and hence in problem~\eqref{eq:bdv-V0}, the boundary data is almost constant. Hence, in view of (\ref{eq:Psi}) and (\ref{eq:Phi_k}), we may approximate
\begin{equation}
\Psi_k(\zeta) \approx P_{kj}= \Psi_k(c_j), \quad  \Phi_k(\zeta) \approx Q_{kj}=\Phi_k(c_j), \quad \zeta \in C_j, \quad j=1,\ldots,m,
\end{equation} 
where $c_j$ is the center of the circle $C_j$. Thus, the Dirichlet problem~\eqref{eq:bdv-V0} becomes
\begin{subequations}\label{eq:bdv-V2}
	\begin{align}
	\label{eq:V2-Lap}
	\nabla^2 V_k(\zeta) &= 0 ~~~~~~~~~~~~~~~~~~~~~~~~~~~~~~~~~\:\:\; \mbox{if }\zeta\in G_m, \\
	\label{eq:V2-m}
	V_k(\zeta)&= 1-P_{k,\frac{m}{2}-j+1}-Q_{k,\frac{m}{2}-j+1} ~~ \mbox{if }\zeta\in C_{\frac{m}{2}-j+1}, ~ j=1,\ldots,k-1, \\
	\label{eq:V2+m}
	V_k(\zeta)&= 1-P_{k,\frac{m}{2}+j}-Q_{k,\frac{m}{2}+j} ~~~~~~~~ \mbox{if }\zeta\in C_{\frac{m}{2}+j}, ~ j=1,\ldots,k-1, \\
	\label{eq:V2-k'}
	V_k(\zeta)&= -Q_{k,\frac{m}{2}+k} ~~~~~~~~~~~~~~~~~~~~~~~~ \mbox{if }\zeta\in C_{\frac{m}{2}+k},  \\
	\label{eq:V2-k''}
	V_k(\zeta)&= -P_{k,\frac{m}{2}-k+1} ~~~~~~~~~~~~~~~~~~~~\:\: \mbox{if }\zeta\in C_{\frac{m}{2}-k+1},  \\
	\label{eq:V2-p}
	V_k(\zeta)&= -P_{k,\frac{m}{2}-j+1}-Q_{k,\frac{m}{2}-j+1}  ~~~~\: \mbox{if }\zeta\in C_{\frac{m}{2}-j+1}, ~ j=k+1,\ldots,m/2, \\
	\label{eq:V2+p}
	V_k(\zeta)&= -P_{k,\frac{m}{2}+j}-Q_{k,\frac{m}{2}+j}  ~~~~~~~~~~\; \mbox{if }\zeta\in C_{\frac{m}{2}+j}, ~ j=k+1,\ldots,m/2. 
	\end{align}
\end{subequations}
The function $V_k$ can be then written in terms of the harmonic measures $\{\sigma_j\}_{j=1}^m$ through
\begin{eqnarray}
\nonumber	V_k(\zeta)&=&
	\sum_{j=1}^{k-1}\left(\sigma_{\frac{m}{2}-j+1}(\zeta)+\sigma_{\frac{m}{2}+j}(\zeta)\right)\\
\label{eq:Vk-0}	&-&
	\sum_{\begin{subarray}{c} j=1\\j\ne k\end{subarray}}^{m/2} \left([P_{k,\frac{m}{2}-j+1}+Q_{k,\frac{m}{2}-j+1}]\sigma_{\frac{m}{2}-j+1}(\zeta)
	+[P_{k,\frac{m}{2}+j}+Q_{k,\frac{m}{2}+j}]\sigma_{\frac{m}{2}+j}(\zeta)\right)\\
\nonumber	&-&P_{k,\frac{m}{2}-k+1}\sigma_{\frac{m}{2}-k+1}(\zeta)-Q_{k,\frac{m}{2}+k}\sigma_{\frac{m}{2}+k}(\zeta).
\end{eqnarray}
Then the unique solution $U_k(\zeta)$ to the Dirichlet problem~\eqref{eq:bdv-U0} can be computed through~\eqref{eq:Vk-Uk-0}.

Note that the images of the circles $C_{\frac{m}{2}-k+1}$ and $C_{\frac{m}{2}+k}$ under the conformal mapping $F$ are the slits $I_{\frac{m}{2}-k+1}$ and $I_{\frac{m}{2}+k}$, respectively. Due to the symmetry of $\Omega_m$ and $G_m$ with respect to the real line, we have $F(\xi)=F(\overline{\xi})\in I_{\frac{m}{2}+k}$ and $F(-\xi)=F(-\overline{\xi})\in I_{\frac{m}{2}-k+1}$ (see Figure~\ref{fig:h0d}).
For a given $r$ such that the capture circle intersects the two slits $I_{\frac{m}{2}+k}$ and $I_{\frac{m}{2}-k+1}$ for some $k=1,...,m/2$, there are points $\xi\in C_{\frac{m}{2}+k}$ and $-\xi\in C_{\frac{m}{2}-k+1}$ which depend on $r$ via 
\begin{equation}\label{eq:r0-xi}
	r= F(\xi).
\end{equation}
The values of $F(\xi)$ can be computed as in the case of $z_0=-3/2$. Thus, the value of the $h$-function is given by
\begin{equation}\label{eq:hr0-xi}
	h(r)=U_k(\zeta_0).
\end{equation}

To compute the values of $h(r)$ when the capture circle intersects the two slits $I_{\frac{m}{2}+k}$ and $I_{\frac{m}{2}-k+1}$ for some $k=1,\ldots,m/2$, as before we take $\xi$ on the circle $C_{\frac{m}{2}+k}$ and then compute the values of $r$ through~\eqref{eq:r0-xi}. 
Again we choose $31$ values of $\xi$ on the upper half of the circle $C_{\frac{m}{2}+k}$. For each of these points $\xi$, we compute the values of $r$ through~\eqref{eq:r0-xi} and then the values of $h(r)$ through~\eqref{eq:hr0-xi}. 
For the values of $r$ corresponding to the capture circle passing through the gaps between the slits $I_k$, the values of $h(r)$ are computed as described in Section~\ref{sec:step}.
The graphs of the function $h(r)$ for $m=16$ and $m=64$ are given in Figure~\ref{fig:h-0}.

The CPU time (in seconds) required for computing the step height of the $h$-function and the $h$-function for $z_0=-3/2$ and $z_0=0$ is presented in Table~\ref{tab:time}. All computations in this paper are performed in {\sc Matlab} R2022a on an MSI desktop with AMD Ryzen 7 5700G, 3801 Mhz, 8 Cores, 16 Logical Processors, and 16 GB RAM. 

\begin{figure}[htb] %
\centerline{
\scalebox{0.6}{\includegraphics[trim=0 0 0 0,clip]{hfunO16}}
\hfill
\scalebox{0.6}{\includegraphics[trim=0 0 0 0,clip]{hfunO64}}
}
\caption{The $h$-function for the domain $\Omega_{16}$ (left) and the domain $\Omega_{64}$ (right) when $z_0=0$.}
\label{fig:h-0}
\end{figure}



\begin{table}[h]
  \caption{The CPU time (sec) required for computing the step heights of the $h$-function (left) and the total time required for computing the $h$-function (right).}
  \label{tab:time}
\centering
\begin{tabular}{l|cc|cc}  \hline
                  & \multicolumn{2}{c|}{The step height}  & \multicolumn{2}{c}{The $h$-function}  \\ \cline{2-5}
                  & $z_0=-3/2$  & $z_0=0$    & $z_0=-3/2$  & $z_0=0$  \\ \hline
 $\Omega_4$       & $0.4120$    & $0.4066$   & $0.4345$    & $0.4363$\\
 $\Omega_8$       & $0.5007$    & $0.5117$   & $0.5186$    & $0.5316$  \\
 $\Omega_{16}$    & $0.8826$    & $0.9272$   & $0.9227$    & $0.9619$  \\
%
 $\Omega_{256}$   & $24.622$    & $24.715$   & $29.251$    & $29.212$  \\
 $\Omega_{512}$   & $87.541$    & $85.479$   & $104.76$    & $103.08$  \\
 $\Omega_{1024}$  & $330.53$    & $326.17$   & $397.72$    & $392.40$  \\
\hline
\end{tabular}
\end{table}

       
       
        
       
       

