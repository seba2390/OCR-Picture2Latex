% center 環境元に戻す
\def\center{\trivlist \centering\item\relax}
\def\endcenter{\endtrivlist}

\section{\corelang{} Type Safety}
\label{appendix:lambdavl_safety}
\subsection{Resource Properties}
\begin{definition}[Version resource semiring]
The version resource semiring is given by the structural semiring (semiring with preorder) $(\mathcal{R},\oplus,0,\otimes,1,\sqsubseteq)$, defined as follows.
\begin{gather*}
    0 = \bot
    \quad
    1 = \emptyset
    \quad
    \begin{minipage}{0.10\hsize}
        \infax{\bot \sqsubseteq \texttt{r}}
    \end{minipage}
    \quad
    \begin{minipage}{0.10\hsize}
        \infrule{
            \texttt{r}_1 \subseteq \texttt{r}_2
        }{
            \texttt{r}_1 \sqsubseteq \texttt{r}_2
        }
    \end{minipage}
    \\
    \begin{aligned}
        r_1 \oplus r_2 &=
        \left\{
        \begin{aligned}
            &r_1 & &\mbox{$r_2 = \bot$}\\
            &r_2 & &\mbox{$r_1 = \bot$}\\
            &r_1 \cup r_2 & &\mbox{otherwise}
        \end{aligned}
        \right.
        \quad
        r_1 \otimes r_2 &=
        \left\{
        \begin{aligned}
            &\bot & &\mbox{$r_1 = \bot$}\\
            &\bot & &\mbox{$r_2 = \bot$}\\
            &r_1 \cup r_2 & &\mbox{otherwise}
        \end{aligned}
        \right.
    \end{aligned}
\end{gather*}
where $\bot$ is the smallest element of $\mathcal{R}$, and $r_1 \subseteq r_2$ is the standard subset relation over sets defined only when both $r_1$ and $r_2$ are not $\bot$.\\
\end{definition}

\begin{lemma}[Version resource semiring is a structural semiring]
\label{proof:semiring}
\end{lemma}
\begin{proof}
Version resource semiring $(\mathcal{R},\oplus,\bot,\otimes,\emptyset,\sqsubseteq)$ induces a semilattice with $\oplus$ (join).
\begin{itemize}
\item $(\mathcal{R},\oplus,\bot,\otimes,\emptyset)$ is a semiring, that is:
    \begin{itemize}
        \item $(\mathcal{R},\oplus,\bot)$ is a commutative monoid, i.e., for all $p,q,r\in\mathcal{R}$
        \begin{itemize}
            \item (Associativity) $(p\oplus q) \oplus r = p\oplus (q \oplus r)$ holds since $\oplus$ is defined in associative manner with $\bot$.
            \item (Commutativity) $p\oplus q = q\oplus p$  holds since $\oplus$ is defined in commutative manner with $\bot$.
            \item (Identity element) $\bot \oplus p = p \oplus \bot = p$
        \end{itemize}
        \item $(\mathcal{R},\otimes,\emptyset)$ is a monoid, i.e., for all $p,q,r\in\mathcal{R}$
        \begin{itemize}
            \item (Associativity) $(p\otimes q) \otimes r = p\otimes (q \otimes r)$ holds since $\oplus$ is defined in associative manner with $\bot$.
            \item (Identity element) $\emptyset \otimes p = p \otimes \emptyset = p$
            \begin{itemize}
                \item if $p = \bot$ then $\emptyset \otimes \bot = \bot \otimes \emptyset = \bot$
                \item otherwise if $p \neq \bot$ then $\emptyset \otimes p = \emptyset \cup p = p$ and $p \otimes \emptyset = p \cup \emptyset = p$
            \end{itemize}
        \end{itemize}
        \item multiplication $\otimes$ distributes over addition $\oplus$, i.e., for all $p, q, r \in \mathcal{R},r\otimes(p\oplus q) = (r\otimes p) \oplus (r \otimes q)$ and $(p\oplus q)\otimes r = (p\otimes r) \oplus (q \otimes r)$
        \begin{itemize}
            \item if $r = \bot$ then $r\otimes(p\oplus q) = \bot$ and $(r\otimes p) \oplus (r \otimes q) = \bot \oplus \bot = \bot$.
            \item otherwise if $r\neq \bot$ and $p = \bot$ and $q \neq \bot$ then $r\otimes(p\oplus q) = r\otimes q = r\cup q = (r\cup r) \cup q = r\cup (r \cup q) = (r \oplus p) \cup (r \cup q) = (r\otimes p) \oplus (r \otimes q)$
            \item otherwise if $r\neq \bot$ and $p = \bot$ and $q = \bot$ then $r\otimes(p\oplus q) = r\otimes \bot = \bot$ and $(r\otimes p) \oplus (r \otimes q) = \bot \oplus \bot = \bot$.
            \item otherwise if $r\neq \bot$ and $p \neq \bot$ and $q \neq \bot$ then $r\otimes(p\oplus q) = r\cup(p\cup q) = (r\cup p)\cup (r\cup q) = (r\otimes p) \oplus (r \otimes q)$
        \end{itemize}
        The other cases are symmetrical cases.
        \item $\bot$ is absorbing for multiplication: $p\otimes \bot= \bot \otimes p = \bot$ for all $p\in\mathcal{R}$
    \end{itemize}
\item $(\mathcal{R},\sqsubseteq)$ is a bounded semilattice, that is
    \begin{itemize}
        \item $\sqsubseteq$ is a partial order on $\mathcal{R}$ such that the least upper bound of every two elements $p,q \in \mathcal{R}$ exists and is denoted by $p\oplus q$.
        \item there is a least element; for all $r\in\mathcal{R}$, $\bot \sqsubseteq r$.
    \end{itemize}
\item (Motonicity of $\oplus$) $p\sqsubseteq q$ implies $p\oplus r\sqsubseteq q\oplus r$ for all $p, q, r \in \mathcal{R}$
   \begin{itemize}
        \item if $r = \bot$ then $p\oplus r \sqsubseteq q\oplus r \Leftrightarrow p\subseteq q$, so this case is trivial.
        \item otherwise if $r \neq \bot, p = q = \bot$ then $p\oplus r \sqsubseteq q \oplus r \Leftrightarrow r\subseteq r$, so this case is trivial.
        \item otherwise if $r \neq \bot, p = \bot, q\neq \bot$ then $p\oplus r \sqsubseteq q \oplus r \Leftrightarrow r\subseteq q\cup r$, and $r\subseteq q\cup r$ holds in standard subset relation.
        \item otherwise if $r \neq \bot, p \neq \bot, q \neq \bot$ then $p\oplus r \sqsubseteq q \oplus r \Leftrightarrow p \cup r \subseteq q\cup r$, and  $p \subseteq q$ implies $p \cup r \subseteq q\cup r$.
    \end{itemize}
\item (Motonicity of $\otimes$) $p\sqsubseteq q$ implies $p\otimes r\sqsubseteq q\otimes r$ for all $p, q, r \in \mathcal{R}$
    \begin{itemize}
        \item if $r = \bot$ then $p\otimes r \sqsubseteq q\otimes r \Leftrightarrow \bot\subseteq \bot$, so this case is trivial.
        \item otherwise if $r \neq \bot, p = q = \bot$ then $p\otimes r \sqsubseteq q \otimes r \Leftrightarrow \bot\subseteq \bot$, so this case is trivial.
        \item otherwise if $r \neq \bot, p = \bot, q\neq \bot$ then $p\otimes r \sqsubseteq q \otimes r \Leftrightarrow \bot\subseteq q\cup r$, so this case is trivial.
        \item otherwise if $r \neq \bot, p \neq \bot, q \neq \bot$ then $p\otimes r \sqsubseteq q \otimes r \Leftrightarrow p \cup r \subseteq q\cup r$, and  $p \subseteq q$ implies $p \cup r \subseteq q\cup r$.
    \end{itemize}
\end{itemize}
\end{proof}

\begin{definition}[Version resource summation]
Using the addition $+$ of version resource semiring, summation of version resouce is defined as follows:
\begin{align*}
\sum_i r_i = r_1 \oplus \cdots \oplus r_n
\end{align*}
\end{definition}

% \begin{lemma}[Well-typed linear substitution]
% Let $\Delta \vdash t':A$ and $\Gamma,x:A,\Gamma' \vdash t:B$. Then, $\Gamma + \Delta + \Gamma' \vdash [t'/x]t:B$
% \end{lemma}
% \begin{proof}
% By induction on the derivation of $\Gamma, x : A, \Gamma' \vdash t : B$. 
% \end{proof}

% \begin{lemma}[Well-typed versioned substitution]
% Let $[\Delta] \vdash t':A$ and $\Gamma,x:\verctype{A}{r},\Gamma' \vdash t:B$.
% Then, $\Gamma + \bigcup_i(r_i\cdot[\Delta_i]) + \Gamma' \vdash [t'/x]t:B$ where $\Sigma_ir_i = r$ and  $\bigcup_i[\Delta_i] = \Delta$%
% \end{lemma}
% \begin{proof}
% By induction on the derivation of $\Gamma,x:\verctype{A}{r},\Gamma' \vdash t:B$.
% \end{proof}

% \begin{lemma}[Default version overwriting type safety]
% Let $[\Gamma] \vdash t':A$.
% Then, $\exists t'. t@l \equiv t' \land \{l\}\cdot[\Gamma] \vdash t':A$%
% \end{lemma}
% \begin{proof}
% By induction on the derivation of $[\Gamma] \vdash t':A$.
% \end{proof}

% \begin{theorem}[\mylang{} type safety]
% Let $\Gamma \vdash t:A$. Then, (i) $t$ is a value or (ii) $\exists t',\Gamma'.t \leadsto t' \land \Gamma' \vdash t':A' \land \Gamma' \sqsubseteq \Gamma$
% \end{theorem}
% \begin{proof}
% By induction on the derivation of $\Gamma \vdash t:A$.
% \end{proof}




\subsection{Context Properties}
\begin{definition}[Context concatenation]
\label{def:contextconcat}
Two typing contexts can be concatenated by "$,$" if they contain disjoint assumptions. 
Furthermore, the versioned assumptions appearing in both typing contexts can be combined using the context concatenation $+$ defined with the addition $\oplus$ in the version resource semiring as follows.
\begin{align*}
\emptyset + \Gamma &= \Gamma\\
(\Gamma,x:A)+\Gamma' &= (\Gamma + \Gamma'),x:A \hspace{1em} \text{iff}\hspace{0.5em} x \notin \mathrm{dom}(\Gamma')\\
\Gamma + \emptyset &= \Gamma\\
\Gamma+(\Gamma',x:A) &= (\Gamma + \Gamma'),x:A \hspace{1em} \text{iff}\hspace{0.5em} x \notin \mathrm{dom}(\Gamma)\\
(\Gamma,x:\verctype{A}{r})+(\Gamma',x:\verctype{A}{s}) &= (\Gamma + \Gamma'),x:\verctype{A}{r \,\oplus\, s}
\end{align*}
\end{definition}

\begin{definition}[Context multiplication by version resource]
\label{def:multiply}
Assuming that a context contains only version assumptions, denoted $[\Gamma]$ in typing rules, then $\Gamma$ can be multiplied by a version resource $r \in \mathcal{R}$ by using the product $\otimes$ in the version resource semiring, as follows.
\begin{align*}
r \cdot \emptyset\ =\ \emptyset \hspace{4em}
r \cdot (\Gamma,\, x:\verctype{A}{s})\ =\ (r \cdot \Gamma),\, x:\verctype{A}{r\,\otimes\, s}
\end{align*}    
\end{definition}

\begin{definition}[Context summation]
Using the context concatenation $+$, summation of typing contexts is defined as follows:
\begin{align*}
\displaystyle \bigcup_{i=1}^{n} \ \Gamma_i\ =\ \Gamma_1 + \cdots + \Gamma_n
\end{align*}
\end{definition}
% \subsection{型代入 definition}
% \label{appendix:substitution}


% \begin{dfn}[型代入$\theta$]
%  型代入$\theta = (\alpha \mapsto A)$が与えられたとき、型・コエフェクト・カインド・項変数環境・型変数環境・型代入への各型代入を以下のように定義する。\\
% 
% \begin{multicols}{2}
% \textbf{型への型代入}
% \begin{center}
% \begin{tabular}{rcl}
%     $\theta K$ & $=$ & $K$ \\
%     $\theta\alpha$ & $=$ & $A \hspace{1em} (\theta(\alpha)=A)$\\
%     $\theta\alpha$ & $=$ & $\alpha \hspace{1em} (\mathrm{otherwise})$\\
%     $\theta(\ftype{A}{B})$ & $=$ & $\ftype{\theta A}{\theta B}$\\
%     $\theta(\app{A}{B})$ & $=$ & $\app{(\theta A)}{(\theta B)}$\\
%     $\theta(A \op B)$ & $=$ & $(\theta A) \op (\theta B)$\\
%     $\theta(\vertype{r}{A})$ & $=$ & $\vertype{(\theta r)}{(\theta A)}$
% \end{tabular}
% \end{center}
% \textbf{コエフェクトへの型代入}
% \begin{center}
% \begin{tabular}{rcl}
%     $\theta 0$ & $=$ & $0$ \\
%     $\theta 1$ & $=$ & $1$ \\
%     $\theta\alpha$ & $=$ & $A \hspace{1em} (\theta(\alpha)=A)$ \\
%     $\theta\alpha$ & $=$ & $\alpha \hspace{1em} (\mathrm{otherwise})$\\
%     $\theta(r_1\otimes r_2)$ & $=$ & $(\theta r_1)\otimes (\theta r_2)$\\
%     $\theta(r_1\oplus r_2)$ & $=$ & $(\theta r_1)\oplus (\theta r_2)$
% \end{tabular}
% \end{center}
% \end{multicols}
% \begin{multicols}{2}
% \textbf{カインドへの型代入}
% \begin{center}
% \begin{center}
% \begin{tabular}{rcl}
%     $\theta\liftkind{A}$ & $=$ & $\liftkind{(\theta A)}$ \\
%     $\theta (\app{\kappa_1}{\kappa_2})$ & $=$ & $\app{(\theta \kappa_1)}{(\theta \kappa_2)}$\\
%     $\theta \kappa$ & $=$ & $\kappa$
% \end{tabular}
% \end{center}
% \end{center}
% \textbf{項変数環境への型代入}
% \begin{center}
% \begin{tabular}{rcl}
%     $\theta \emptyset$ & $=$ & $\emptyset$ \\
%     $\theta(\Gamma, x:A)$ & $=$ & $\theta\Gamma, x:\theta A$\\
%     $\theta(\Gamma, x:\verctype{A}{r})$ & $=$ & $\theta\Gamma, x:\verctype{(\theta A)}{(\theta r)}$
% \end{tabular}
% \end{center}
% \end{multicols}
% \begin{multicols}{2}
% \textbf{型変数環境への型代入}
% \begin{center}
% \begin{tabular}{rcl}
%     $\theta \emptyset$ & $=$ & $\emptyset$ \\
%     $\theta(\Sigma, \alpha:\kappa)$ & $=$ & $\theta\Sigma, \alpha:\theta \kappa$
% \end{tabular}
% \end{center}
% \textbf{型代入への型代入}
% \begin{center}
% \begin{tabular}{rcl}
%     $\theta \emptyset$ & $=$ & $\emptyset$ \\
%     $\theta(\theta' \uplus \alpha\mapsto A)$ & $=$ & $(\theta\theta') \uplus \alpha \mapsto (\theta A)$
% \end{tabular}
% \end{center}
% \end{multicols}
% 
% \end{dfn}

\begin{definition}[Context partition]
\label{def:restriction}
For typing contexts $\Gamma_1$ and $\Gamma_2$, we define $\incl{\Gamma_1}{\Gamma_2}$ and $\excl{\Gamma_1}{\Gamma_2}$ as follows.
\begin{align*}
\incl{\Gamma_1}{\Gamma_2} &\triangleq \{ x:A\ |\ x \in \mathrm{dom}(\Gamma_1) \land x \in \mathrm{dom}(\Gamma_2)\}\\
\excl{\Gamma_1}{\Gamma_2} &\triangleq \{ x:A\ |\ x \in \mathrm{dom}(\Gamma_1) \land x \notin \mathrm{dom}(\Gamma_2)\}
\end{align*}
$\incl{\Gamma_1}{\Gamma_2}$ is a subsequence of $\Gamma_1$ that contains all the term variables that are \emph{included} in $\Gamma_2$, and
$\excl{\Gamma_1}{\Gamma_2}$ is a subsequence of $\Gamma_1$ that contains all the term variables that are \emph{not included} in $\Gamma_2$.
\end{definition}

Using $\incl{\Gamma_1}{\Gamma_2}$ and $\excl{\Gamma_1}{\Gamma_2}$, we state some corollaries about typing contexts.
These theorems follow straightforwardly from the definitions of \ref{def:restriction}.

\begin{lemma}[Context collapse]
\label{lemma:restriction}
For typing contexts $\Gamma_1$ and $\Gamma_2$,
\begin{align*}
    \incl{\Gamma_1}{\Gamma_2} + \excl{\Gamma_1}{\Gamma_2} = \Gamma_1
\end{align*}
%これは$\incl{\Gamma_1}{\Gamma_2}$と$\excl{\Gamma_1}{\Gamma_2}$ definitionより明らかである。
\end{lemma}


\begin{lemma}[Context shuffle]
\label{lemma:shuffle}
For typing contexts $\Gamma_1$, $\Gamma_2$, $\Gamma_3$ and $\Gamma_4$, and variable $x$ and type $A$:
\begin{align*}
(\Gamma_1,x:A,\Gamma'_1)+\Gamma_2 &= (\Gamma_1 + \incl{\Gamma_2}{\Gamma_1}),x:A,(\Gamma_1' + \excl{\Gamma_2}{\Gamma_1}) \tag{1}\\
\Gamma_1 + (\Gamma_2,x:A,\Gamma'_2) &= (\excl{\Gamma_1}{\Gamma'_2} + \Gamma_2),x:A,(\incl{\Gamma_1}{\Gamma'_2} + \Gamma'_2) \tag{2}\\
(\Gamma_1,\Gamma_2)+(\Gamma_3,\Gamma_4) &= \left((\Gamma_1+\incl{\Gamma_3}{\Gamma_1}+\incl{\Gamma_4}{\Gamma_1}), (\Gamma_2+\excl{\Gamma_3}{\Gamma_1}+\excl{\Gamma_4}{\Gamma_1})\right) \tag{3}
\end{align*}
\end{lemma}

\begin{lemma}[Composition of context shuffle]
\label{lemma:shufflecomposition}
For typing contexts $\Gamma_i$ and $\Gamma_i'$ for $i\in \mathbb{N}$, there exixts typing contexts $\Gamma$ and $\Gamma'$ such that:
\begin{align*}
\bigcup_{i}(\Gamma_i,\Gamma_i') = \Gamma, \Gamma'\ \land\ \bigcup_i(\Gamma_i+\Gamma_i') = \Gamma + \Gamma'
\end{align*}
\end{lemma}

\begin{lemma}[Distribution of version resouce over context addition]
\label{lemma:distributivelaw}
For a typing context $\Gamma$ and resources $r_i \in R$:
\begin{align*}
(r_1 \cdot \Gamma) + (r_2\cdot \Gamma) &= (r_1 \oplus r_2)\cdot\Gamma\\
\bigcup_i(r_i \cdot \Gamma) &= (\sum_i r_i)\cdot\Gamma
\end{align*}
\end{lemma}

\begin{lemma}[Disjoint context collapse]
\label{lemma:collapse}
Given typing contexts $\Gamma_1$, $\Delta$, and $\Gamma_2$ such that $\Gamma_1$ and $\Gamma_2$ are disjoint, then we can conclude the following.
\begin{align*}
(\Gamma_1+\Delta+\Gamma_2) = (\Gamma_1+\incl{\Delta}{\Gamma_1}),\excl{\Delta}{(\Gamma_1,\Gamma_2)},(\Gamma_2+\incl{\Delta}{\Gamma_2})
\end{align*}
\end{lemma}



















%%%%%%%%%%%%%%%%%%%%%%%%%%%%%%%%%%%%%%%%%%%%%%%%%%%%%%%%%%%%%%%%%%%%%%%%%%%%%
%%%%%%%%%%%%%%%%%%%%%%%%%%%%%%%%%%%%%%%%%%%%%%%%%%%%%%%%%%%%%%%%%%%%%%%%%%%%%
%%%%%%%%%%%%%%%%%%%%%%%%%%%%%%%%%%%%%%%%%%%%%%%%%%%%%%%%%%%%%%%%%%%%%%%%%%%%%
%%%%%%%%%%%%%%%%%%%%%%%%%%%%%%%%%%%%%%%%%%%%%%%%%%%%%%%%%%%%%%%%%%%%%%%%%%%%%
%%%%%%%%%%%%%%%%%%%%%%%%%%%%%%%%%%%%%%%%%%%%%%%%%%%%%%%%%%%%%%%%%%%%%%%%%%%%%
%%%%%%%%%%%%%%%%%%%%%%%%%%%%%%%%%%%%%%%%%%%%%%%%%%%%%%%%%%%%%%%%%%%%%%%%%%%%%



\subsection{Substituions Properties}
\label{appendix:lemsubstitution}

\begin{lemma}[Well-typed linear substitution]
\label{lemma:substitution1}
\begin{align*}
    \left.
    \begin{aligned}
          \Delta \vdash t': A\\
          \Gamma,x:A,\Gamma' \vdash t:B
    \end{aligned}
    \right\}
    \hspace{1em}\Longrightarrow\hspace{1em}
    \Gamma + \Delta + \Gamma' \vdash [t'/x]t:B
\end{align*}
\end{lemma}

\begin{proof}
This proof is given by induction on the structure of $\Gamma,x:A,\Gamma' \vdash t:B$ (assumption 2).
Consider the cases for the last rule used in the typing derivation of assumption 2.

\begin{itemize}
\item Case (\textsc{int})
\begin{center}
    \begin{minipage}{.27\linewidth}
        \infrule[int]{
             \\% \\
        }{
            \emptyset \vdash n : \textsf{Int}
        }
    \end{minipage}
    % \hspace{1em}\& \hspace{1em}Case 
    % \begin{minipage}{.50\linewidth} % 0.32だった
    %     \infrule[C]{
    %         (x:\forall\{\overrightarrow{\alpha:\kappa}\}.A) \in D
    %         \andalso
    %         %\theta,\Sigma',\theta_{\kappa'} = \textsf{instantiate}(\overrightarrow{\alpha:\kappa},\theta_{\kappa})
    %         \theta,\Sigma' = \textsf{inst}(\overrightarrow{\alpha:\kappa})
    %     }{
    %         D;\Sigma,\Sigma';\emptyset \vdash x : \theta A
    %         %D;\Sigma,\Sigma';\emptyset \vdash C : (\theta_{\kappa}' \uplus \theta)A
    %     }
    % \end{minipage}
\end{center}
In this case, the above typing context is empty ($= \emptyset$), so this case holds trivially.\\


\item Case (\textsc{var})
\begin{center}
    \begin{minipage}{.50\linewidth}
        \infrule[var]{
            \vdash B
        }{
            y:B \vdash y:B
        }
    \end{minipage}
\end{center}
We are given
\begin{gather*}
\Gamma=\Gamma'=\emptyset,\quad
x=t=y,\quad
A = B
~.
\end{gather*}
Now the conclusion of the lemma is
\begin{align*}
\Delta \vdash [t'/y]y:B
~.
\end{align*}
Since $[t'/y]y=t'$ from the definition of substitution, the conclusion of the lemma is assumption 1 itself.\\


\item Case (\textsc{abs})
\begin{center}
    \begin{minipage}{.75\linewidth}
        \infrule[abs]{
            - \vdash p : B_1 \rhd \Delta'
            \andalso
            \Gamma, x:A, \Gamma', \Delta' \vdash t : B_2%\theta B
        }{
            \Gamma, x:A, \Gamma' \vdash \lam{p}{t} : \ftype{B_1}{B_2}
        }
    \end{minipage}
\end{center}
In this case, by applying the induction hypothesis to the second premise, we know the following:
\begin{align*}
    \Gamma + \Delta + (\Gamma', \Delta') \vdash [t'/x]t : B_2
\end{align*}
where $y:B$ is disjoint with $\Gamma$, $\Delta$, and $\Gamma'$.
Thus, $\Gamma + \Delta + (\Gamma', \Delta') = (\Gamma + \Delta + \Gamma'), \Delta'$ from Lemma \ref{lemma:shuffle} (2), the typing derivation above is equal to the following:
\begin{align*}
    (\Gamma + \Delta + \Gamma'), \Delta' \vdash [t'/x]t : B_2
\end{align*}
We then reapply (\textsc{abs}) to obtain the following:
\begin{center}
    \begin{minipage}{.75\linewidth}
        \infrule[abs]{
             - \vdash p : B_1 \rhd \Delta'
            \andalso
            (\Gamma + \Delta + \Gamma'), \Delta' \vdash [t'/x]t : B_2
        }{
            \Gamma + \Delta + \Gamma' \vdash \lam{p}{[t'/x]t} : \ftype{B_1}{B_2}
        }
    \end{minipage}
\end{center}
By the definition of substitution $\lam{p}{[t'/x]t}=[t'/x](\lam{p}{t})$, and we obtain the conclusion of the lemma.
\\


\item Case (\textsc{let})
\begin{center}
    \begin{minipage}{.75\linewidth}
        \infrule[let]{
             \Gamma_1 \vdash t_1 : \vertype{r}{A}
             \andalso
             \Gamma_2, x:\verctype{A}{r} \vdash t_2 : B
        }{
             \Gamma_1 + \Gamma_2 \vdash \clet{x}{t_1}{t_2} : B
        }
    \end{minipage}
\end{center}
This case is similar to the case (\textsc{app}).
\\

\item Case (\textsc{app})
\begin{center}
    \begin{minipage}{.75\linewidth}
        \infrule[app]{
             \Gamma_1 \vdash t_1 : \ftype{B_1}{B_2}
             \andalso
             \Gamma_2 \vdash t_2 : B_1
        }{
             \Gamma_1 + \Gamma_2 \vdash \app{t_1}{t_2} : B_2
        }
    \end{minipage}
\end{center}
We are given
\begin{gather*}
\Gamma, x:A, \Gamma' = \Gamma_1 + \Gamma_2,\quad
t = \app{t_1}{t_2},\quad
B = B_2
~.
\end{gather*}
By the definition of the context addition $+$, the linear assumption $x:A$ is contained in only one of $\Gamma_1$ or $\Gamma_2$.
\begin{itemize}
\item Suppose $(x:A) \in \Gamma_1$ and $(x:A) \notin \Gamma_2$.\\
Let $\Gamma_1'$ and $\Gamma_1''$ be typing contexts such that they satisfy $\Gamma_1 = (\Gamma_1', x:A, \Gamma_1'')$.
The last typing derivation of ($\textsc{app}$) is rewritten as follows.
\begin{center}
    \begin{minipage}{.60\linewidth}
        \infrule[app]{
             \Gamma_1', x:A, \Gamma_1'' \vdash t_1 : \ftype{B_1}{B_2}
            \\
             \Gamma_2 \vdash t_2 : B_1
        }{
             (\Gamma_1', x:A, \Gamma_1'' ) + \Gamma_2 \vdash \app{t_1}{t_2} : B_2
        }
    \end{minipage}
\end{center}
Now, we compare the typing contexts between the lemma and the above conclusion as follows:
\begin{align*}
(\Gamma, x:A, \Gamma')
    &= (\Gamma_1 + \Gamma_2)\\
    &= (\Gamma_1', x:A, \Gamma_1'' ) + \Gamma_2\tag{$\because$ $\Gamma_1 = (\Gamma_1', x:A, \Gamma_1'')$}\\
    &= (\Gamma_1' + \incl{\Gamma_2}{\Gamma_1'}),x:A,(\Gamma_1'' + \excl{\Gamma_2}{\Gamma_1'})\tag{$\because$ Lemma \ref{lemma:shuffle} (1)}
\end{align*}
By the commutativity of "$,$", we can take $\Gamma$ and $\Gamma'$ arbitrarily so that they satisfy the above equation. So here we know $\Gamma = (\Gamma_1' + \incl{\Gamma_2}{\Gamma_1'})$ and $\Gamma' = (\Gamma_1'' + \excl{\Gamma_2}{\Gamma_1'})$.\par
We then apply the induction hypothesis to each of the two premises and reapply (\textsc{app}) as follows:
\begin{center}
    \begin{minipage}{.75\linewidth}
        \infrule[app]{
             \Gamma_1' + \Delta + \Gamma_1'' \vdash [t'/x]t_1 : \ftype{B_1}{B_2}
            \\
             \Gamma_2 \vdash t_2 : B_1
        }{
             (\Gamma_1' + \Delta + \Gamma_1'' ) + \Gamma_2 \vdash \app{([t'/x]t_1)}{t_2} : B_2
        }
    \end{minipage}
\end{center}
Since $\app{([t'/x]t_1)}{t_2}=[t'/x](\app{t_1}{t_2})$ if $x\notin FV(t_2)$, the conclusion of the above derivation is equivalent to the conclusion of the lemma except for the typing contexts.\par
Finally, we must show that $(\Gamma + \Delta + \Gamma') = ((\Gamma_1 + \Delta + \Gamma_1'') + \Gamma_2)$.
This holds from the following reasoning:
\begin{align*}
(\Gamma + \Delta + \Gamma')
    &= (\Gamma_1' + \incl{\Gamma_2}{\Gamma_1'}) + \Delta + (\Gamma_1'' + \excl{\Gamma_2}{\Gamma_1'})\tag{$\because$ $\Gamma = (\Gamma_1' + \incl{\Gamma_2}{\Gamma_1'})$ and $\Gamma' = (\Gamma_1'' + \excl{\Gamma_2}{\Gamma_1'})$}\\
    &= \Gamma_1' + \incl{\Gamma_2}{\Gamma_1'} + \Delta + \Gamma_1'' + \excl{\Gamma_2}{\Gamma_1'}\tag{$\because$ $+$ associativity}\\
    &= \Gamma_1' + \Delta + \Gamma''_1 + \incl{\Gamma_2}{\Gamma_1'} + \excl{\Gamma_2}{\Gamma_1'}\tag{$\because$ $+$ commutativity}\\
    &= (\Gamma_1' + \Delta + \Gamma''_1) + (\incl{\Gamma_2}{\Gamma_1'} + \excl{\Gamma_2}{\Gamma_1'})\tag{$\because$ $+$ associativity}\\
    &= (\Gamma_1' + \Delta + \Gamma''_1) + \Gamma_2\tag{$\because$ Lemma \ref{lemma:restriction}}
\end{align*}
Thus, we obtain the conclusion of the lemma.

\item Suppose $(x:A) \notin \Gamma_1$ and $(x:A) \in \Gamma_2$\\
Let $\Gamma_2'$ and $\Gamma_2''$ be typing contexts such that they satisfy  $\Gamma_2 = (\Gamma_2', x:A, \Gamma_2'')$.
The last typing derivation of ($\textsc{app}$) is rewritten as follows.
\begin{center}
    \begin{minipage}{.75\linewidth}
        \infrule[app]{
             \Gamma_1 \vdash t_1 : \ftype{B_1}{B_2}
             \andalso
             \Gamma_2', x:A, \Gamma_2'' \vdash t_2 : B_1
        }{
             \Gamma_1 + (\Gamma_2', x:A, \Gamma_2'') \vdash \app{t_1}{t_2} : B_2
        }
    \end{minipage}
\end{center}
This case is similar to the case $(x:A)\in \Gamma_1$, but using \ref{lemma:shuffle} (2) instead of \ref{lemma:shuffle} (1).\\
\end{itemize}


\item Case (\textsc{weak})
\begin{center}
    \begin{minipage}{.55\linewidth}
        \infrule[weak]{
            \Gamma_1, x:A, \Gamma_2 \vdash t : B
            \andalso
            \vdash \Delta'
        }{
            (\Gamma_1, x:A, \Gamma_2) + \verctype{\Delta'}{0} \vdash t : B
        }
    \end{minipage}
\end{center}
In this case, the linear assumption $x:A$ is not contained in versioned context $\verctype{\Delta'}{0}$.
We then compare the typing contexts between the conclusion of the lemma and that of (\textsc{weak}) as follows:
\begin{align*}
    (\Gamma, x:A, \Gamma')
    &= (\Gamma_1, x:A, \Gamma_2) + \verctype{\Delta'}{0}\\
    &= (\Gamma_1 + \incl{(\verctype{\Delta'}{0})}{\Gamma_1}),x:A,(\Gamma_2 + \excl{(\verctype{\Delta'}{0})}{\Gamma_1})\tag{$\because$ Lemma \ref{lemma:shuffle} (1)}
\end{align*}
By the commutativity of "$,$", we can take $\Gamma$ and $\Gamma'$ arbitrarily so that they satisfy the above equation. So here we obtain $\Gamma=\Gamma_1 + \incl{(\verctype{\Delta'}{0})}{\Gamma_1}$ and $\Gamma'=\Gamma_2 + \excl{(\verctype{\Delta'}{0})}{\Gamma_1}$.
We then apply the induction hypothesis to each of the premise and reapply (\textsc{weak}) as follows:
\begin{center}
    \begin{minipage}{.65\linewidth}
        \infrule[weak]{
            \Gamma_1 + \Delta + \Gamma_2 \vdash [t'/x]t : B
            \andalso
            \vdash \Delta'
        }{
            (\Gamma_1 + \Delta + \Gamma_2) + \verctype{\Delta'}{0} \vdash [t'/x]t : B
        }
    \end{minipage}
\end{center}
Since $\app{([t'/x]t_1)}{([t'/x]t_2)}=[t'/x](\app{t_1}{t_2})$, the conclusion of the above derivation is equivalent to the conclusion of the lemma except for typing contexts.\par
Finally, we must show that $(\Gamma_1 + \Delta + \Gamma_2) + \verctype{\Delta'}{0} = \Gamma + \Delta + \Gamma'$.
This holds from the following reasoning:
\begin{align*}
    (\Gamma + \Delta + \Gamma')
    &= (\Gamma_1 + \incl{(\verctype{\Delta'}{0})}{\Gamma_1})+\Delta+(\Gamma_2 + \excl{(\verctype{\Delta'}{0})}{\Gamma_1})\tag{$\because$ $\Gamma=\Gamma_1 + \incl{(\verctype{\Delta'}{0})}{\Gamma_1}$ and $\Gamma'=\Gamma_2 + \excl{(\verctype{\Delta'}{0})}{\Gamma_1}$}\\
    &= \Gamma_1 + \incl{(\verctype{\Delta'}{0})}{\Gamma_1} + \Delta + \Gamma_2 + \excl{(\verctype{\Delta'}{0})}{\Gamma_1} \tag{$\because$ $+$ associativity}\\
    &= \Gamma_1 + \Delta + \Gamma_2 + \incl{(\verctype{\Delta'}{0})}{\Gamma_1} + \excl{(\verctype{\Delta'}{0})}{\Gamma_1} \tag{$\because$ $+$ commutativity}\\
    &= (\Gamma_1 + \Delta + \Gamma_2) + (\incl{(\verctype{\Delta'}{0})}{\Gamma_1} + \excl{(\verctype{\Delta'}{0})}{\Gamma_1}) \tag{$\because$ $+$ associativity}\\
    &= (\Gamma_1 + \Delta + \Gamma_2) + \verctype{\Delta'}{0}\tag{$\because$ Lemma \ref{lemma:restriction}}
\end{align*}
Thus, we obtain the conclusion of the lemma.
\\

\item Case (\textsc{der})
\begin{center}
    \begin{minipage}{.55\linewidth}
        \infrule[der]{
            \Gamma, x:A ,\Gamma'', y:B_1 \vdash t : B_2
        }{
            \Gamma, x:A, \Gamma'', y:\verctype{B_1}{1} \vdash t : B_2
        }
    \end{minipage}
\end{center}
In this case, a linear assumption $x:A$ cannot be a versioned assumption $y:\verctype{B_1}{1}$.
Applying the induction hypothesis to the premise, we obtain the following:
\begin{align*}
    \Gamma + \Delta + (\Gamma'', y:B_1) \vdash [t'/x]t : B_2
\end{align*}
Note that $\Gamma + \Delta + (\Gamma'', y:B_1)= (\Gamma + \Delta + \Gamma''), y:B_1$ holds because $y:B_1$ is a linear assumption and is disjoint with $\Gamma$, $\Delta$, and $\Gamma''$.
Thus, the above judgement is equivalent to the following:
\begin{align*}
    (\Gamma + \Delta + \Gamma''), y:B_1 \vdash [t'/x]t : B_2
\end{align*}
We then reapply (\textsc{der}) to obtain the following:
\begin{center}
    \begin{minipage}{.65\linewidth}
        \infrule[der]{
            (\Gamma + \Delta + \Gamma''), y:B_1 \vdash [t'/x]t : B_2
        }{
            (\Gamma + \Delta + \Gamma''), y:\verctype{B_1}{1} \vdash [t'/x]t : B_2
        }
    \end{minipage}
\end{center}
Finally, since $y:\verctype{B_1}{1}$ is disjoint with $\Gamma+\Delta+\Gamma''$, $((\Gamma + \Delta + \Gamma''), y:\verctype{B_1}{1}) = \Gamma + \Delta + (\Gamma'', y:\verctype{B_1}{1})$ holds. Thus, the conclusion of the above derivation is equivalent to the following:
\begin{align*}
    \Gamma + \Delta + (\Gamma'', y:\verctype{B_1}{1}) \vdash [t'/x]t : B_2
\end{align*}
Thus, we obtain the conclusion of the lemma.
\\

\item Case (\textsc{pr})
\begin{center}
    \begin{minipage}{.4\linewidth}
        \infrule[pr]{
            \verctype{\Gamma}{} \vdash t : B
            \andalso
            \vdash r
        }{
            r\cdot\verctype{\Gamma}{} \vdash [t] : \vertype{r}{B} 
        }
    \end{minipage}
\end{center}
This case holds trivially, because the typing context $[\Gamma]$ contains only versioned assumptions and does not contain any linear assumptions.
\\

\item Case (\textsc{ver})
\begin{center}
    \begin{minipage}{.65\linewidth}
        \infrule[ver]{
            \verctype{\Gamma_i}{} \vdash t_i : A
            \andalso
            \vdash \{\overline{l_i}\}
        }{
            \bigcup_i(\{l_i\}\cdot [\Gamma_i]) \vdash \nvval{\overline{l_i=t_i}} : \vertype{\{\overline{l_i}\}}{A}
        }
    \end{minipage}
\end{center}
This case holds trivially, because the typing context of the conclusion contains only versioned assumptions (by $[\Gamma_i]$ in the premise) and does not contain any linear assumptions.
\\

\item Case (\textsc{veri})
\begin{center}
    \begin{minipage}{.65\linewidth}
        \infrule[veri]{
            \verctype{\Gamma_i}{} \vdash t_i : A
            \andalso
            \vdash \{\overline{l_i}\}
            \andalso
            l_k \in \{\overline{l_i}\}
        }{
            \bigcup_i(\{l_i\}\cdot [\Gamma_i]) \vdash \ivval{\overline{l_i=t_i}}{l_k} : A
        }
    \end{minipage}
\end{center}
This case holds trivially, because the typing context of the conclusion contains only versioned assumptions (by $[\Gamma_i]$ in the premise) and does not contain any linear assumptions.
\\

\item Case (\textsc{extr})
\begin{center}
    \begin{minipage}{.45\linewidth}
        \infrule[extr]{
            \Gamma \vdash t : \vertype{r}{A}
            \andalso
            l \in r
        }{
            \Gamma \vdash t.l : A
        }
    \end{minipage}
\end{center}
In this case, we apply the induction hypothesis to the premise and then reapply (\textsc{extr}), we obtain the conclusion of the lemma.\\

\item Case (\textsc{sub})
\begin{center}
    \begin{minipage}{.50\linewidth}
        \infrule[\textsc{sub}]{
            \Gamma,y:\verctype{B'}{r}, \Gamma' \vdash t : B
            \andalso
            r \sqsubseteq s
            \andalso
            \vdash s
        }{
            \Gamma,y:\verctype{B'}{s}, \Gamma' \vdash t : B
        }
    \end{minipage}
\end{center}
In this case, a linear assumption $x:A$ cannot be a versioned assumption $y:\verctype{B_1}{s}$, and only one of $(x:A)\in\Gamma$ or $(x:A)\in\Gamma'$ holds.
In either case, applying the induction hypothesis to the premise and reappling (\textsc{sub}), we obtain the conclusion of the lemma.

\end{itemize}
\end{proof}


%%%%%%%%%%%%%%%%%%%%%%%%%%%%%%%%%%%%%%%%%%%%%%%%%%%%%%%%%%%%%%%%%%%%%%%%%%%%%
%%%%%%%%%%%%%%%%%%%%%%%%%%%%%%%%%%%%%%%%%%%%%%%%%%%%%%%%%%%%%%%%%%%%%%%%%%%%%
%%%%%%%%%%%%%%%%%%%%%%%%%%%%%%%%%%%%%%%%%%%%%%%%%%%%%%%%%%%%%%%%%%%%%%%%%%%%%
%%%%%%%%%%%%%%%%%%%%%%%%%%%%%%%%%%%%%%%%%%%%%%%%%%%%%%%%%%%%%%%%%%%%%%%%%%%%%
%%%%%%%%%%%%%%%%%%%%%%%%%%%%%%%%%%%%%%%%%%%%%%%%%%%%%%%%%%%%%%%%%%%%%%%%%%%%%
%%%%%%%%%%%%%%%%%%%%%%%%%%%%%%%%%%%%%%%%%%%%%%%%%%%%%%%%%%%%%%%%%%%%%%%%%%%%%

\begin{lemma}[Well-typed versioned substitution]
\label{lemma:substitution2}
\begin{align*}
    \left.
    \begin{aligned}\relax
          [\Delta] \vdash t':A\\
          \Gamma,x:\verctype{A}{r},\Gamma' \vdash t:B
    \end{aligned}
    \right\}
    \hspace{1em}\Longrightarrow\hspace{1em}
    \Gamma + r\cdot\Delta + \Gamma' \vdash [t'/x]t:B
\end{align*}
\end{lemma}
% \begin{lemma}[Well-typed versioned substitution]
% \label{lemma:substitution2}
% % Given $[\Delta] \vdash t':A$ (assumption 1) and $\Gamma,x:\verctype{A}{r},\Gamma' \vdash t:B$ (assumption 2), then $\Gamma + r\cdot\Delta + \Gamma' \vdash [t'/x]t:B$ holds.
% \begin{align*}
%     \left.
%     \begin{aligned}
%           [\Delta] \vdash t':A\\
%           \Gamma,x:\verctype{A}{r},\Gamma' \vdash t:B
%     \end{aligned}
%     \right\}
%     \hspace{1em}\Longrightarrow\hspace{1em}
%     \Gamma + r\cdot\Delta + \Gamma' \vdash [t'/x]t:B
% \end{align*}
% \end{lemma}

\begin{proof}
This proof is given by induction on structure of $\Gamma,x:\verctype{A}{r},\Gamma' \vdash t:B$ (assumption 2).
Consider the cases for the last rule used in the typing derivation of assumption 2.

\begin{itemize}
\item Case (\textsc{int})
\begin{center}
    \begin{minipage}{.22\linewidth}
        \infrule[int]{
             \\% \\
        }{
            \emptyset \vdash n : \textsf{Int}
        }
    \end{minipage}
    % \hspace{1em}\& \hspace{1em}Case 
    % \begin{minipage}{.39\linewidth} % 0.32だった
    %     \infrule[C]{
    %         (x:\forall\{\overrightarrow{\alpha:\kappa}\}.A) \in D
    %         \andalso
    %         %\theta,\Sigma',\theta_{\kappa'} = \textsf{instantiate}(\overrightarrow{\alpha:\kappa},\theta_{\kappa})
    %         \theta,\Sigma' = \textsf{inst}(\overrightarrow{\alpha:\kappa})
    %     }{
    %         D;\Sigma,\Sigma';\emptyset \vdash x : \theta A
    %         %D;\Sigma,\Sigma';\emptyset \vdash C : (\theta_{\kappa}' \uplus \theta)A
    %     }
    % \end{minipage}
\end{center}
This case holds trivially because the typing context of (\textsc{int}) is empty ($=\emptyset$).\\


\item Case (\textsc{var})
\begin{center}
    \begin{minipage}{.35\linewidth}
        \infrule[var]{
            \vdash B
        }{
            y:B \vdash y:B
        }
    \end{minipage}
\end{center}
In this case, $x:\verctype{A}{r}$ is a versioned assumption and $y:B$ is a linear assumption, so $x\neq y$ holds, and yet the typing context besides $y:B$ is empty.
Thus, there are no versioned variables to be substituted, so this case holds trivially.\\


\item Case (\textsc{abs})
\begin{center}
    \begin{minipage}{.8\linewidth}
        \infrule[abs]{
            - \vdash p : B_1 \rhd \Delta'
            \andalso
            \Gamma, x:\verctype{A}{r}, \Gamma', \Delta' \vdash t : B_2%\theta B
        }{
            \Gamma, x:\verctype{A}{r}, \Gamma' \vdash \lam{p}{t} : \ftype{B_1}{B_2}
        }
    \end{minipage}
\end{center}
In this case, we know the following by applying induction hypothesis to the partial derivation of (\textsc{abs}):
\begin{align*}
    \Gamma + r\cdot\Delta + (\Gamma', \Delta') \vdash [t'/x]t : B_2
\end{align*}
where $\Delta'$ ($\mathrm{dom}(\Delta') = \{y\}$) is disjoint with $\Gamma$, $\Delta$, and $\Gamma'$.
Thus, $\Gamma + r\cdot\Delta + (\Gamma', \Delta') = (\Gamma + r\cdot\Delta + \Gamma'), \Delta'$ from Lemma \ref{lemma:shuffle} (2), the typing derivation above is equal to the following:
\begin{align*}
    (\Gamma + r\cdot\Delta + \Gamma'), \Delta' \vdash [t'/x]t : B_2
\end{align*}
We then reapply (\textsc{abs}) to obtain the following:
\begin{center}
    \begin{minipage}{.80\linewidth}
        \infrule[abs]{
             - \vdash p : B_1 \rhd \Delta'
            \andalso
            (\Gamma + r\cdot\Delta + \Gamma'), \Delta' \vdash [t'/x]t : B_2
        }{
            \Gamma + r\cdot\Delta + \Gamma' \vdash \lam{p}{[t'/x]t} : \ftype{B_1}{B_2}
        }
    \end{minipage}
\end{center}
Since $\lam{p}{[t'/x]t}=[t'/x](\lam{p}{t})$ from the definition of substitution, we obtain the conclusion of the lemma.\\


\item Case (\textsc{app})
\begin{center}
    \begin{minipage}{.75\linewidth}
        \infrule[app]{
            \Gamma_1 \vdash t_1 : \ftype{B_1}{B_2}
            \andalso
            \Gamma_2 \vdash t_2 : B_1
        }{
             \Gamma_1 + \Gamma_2 \vdash \app{t_1}{t_2} : B_2
        }
    \end{minipage}
\end{center}

We are given
\begin{gather*}
\Gamma, x:\verctype{A}{r}, \Gamma' = \Gamma_1 + \Gamma_2,\quad
t = \app{t_1}{t_2},\quad
B = B_2
~.
\end{gather*}
By the definition of the context addition $+$, the linear assumption $x:A$ is contained in either or both of the typing context $\Gamma_1$ or $\Gamma_2$.
\begin{itemize}
\item Suppose $(x:\verctype{A}{r}) \in \Gamma_1$ and $x \notin \mathrm{dom}(\Gamma_2)$\\
Let $\Gamma_1'$ and $\Gamma_1''$ be typing contexts such that they satisfy $\Gamma_1 = (\Gamma_1', x:\verctype{A}{r}, \Gamma_1'')$.
The last derivation of (\textsc{app}) is rewritten as follows:
\begin{center}
    \begin{minipage}{.9\linewidth}
        \infrule[app]{
            \Gamma_1', x:\verctype{A}{r}, \Gamma_1'' \vdash t_1 \ftype{B_1}{B_2}
            \andalso
            \Gamma_2 \vdash t_2 : B_1
        }{
             (\Gamma_1', x:\verctype{A}{r}, \Gamma_1'' ) + \Gamma_2 \vdash \app{t_1}{t_2} : B_2
        }
    \end{minipage}
\end{center}
We compare the typing contexts between the conclusion of the lemma and that of the above derivation to obtain the following:
\begin{align*}
(\Gamma, x:\verctype{A}{r}, \Gamma')
    &= (\Gamma_1', x:\verctype{A}{r}, \Gamma_1'' ) + \Gamma_2\\
    &= (\Gamma_1' + \incl{\Gamma_2}{\Gamma_1'}),x:\verctype{A}{r},(\Gamma_1'' + \excl{\Gamma_2}{\Gamma_1'})\tag{$\because$ Lemma \ref{lemma:shuffle} (1)}
\end{align*}
By the commutativity of "$,$", we can take $\Gamma$ and $\Gamma'$ arbitrarily so that they satisfy the above equation. So here we know $\Gamma = (\Gamma_1' + \incl{\Gamma_2}{\Gamma_1'})$ and $\Gamma' = (\Gamma_1'' + \excl{\Gamma_2}{\Gamma_1'})$.\par
We then apply the induction hypothesis to each of the two premises of the last derivation and reapply (\textsc{app}) as follows:
\begin{center}
    \begin{minipage}{.85\linewidth}
        \infrule[app]{
             \Gamma_1' + r\cdot\Delta + \Gamma_1'' \vdash [t'/x]t_1 : \ftype{B_1}{B_2}
            \\
             \Gamma_2 \vdash [t'/x]t_2 : B_1
        }{
             (\Gamma_1' + r\cdot\Delta + \Gamma_1'' ) + \Gamma_2 \vdash \app{([t'/x]t_1)}{([t'/x]t_2)} : B_2
        }
    \end{minipage}
\end{center}
Since $\app{([t'/x]t_1)}{([t'/x]t_2)}=[t'/x](\app{t_1}{t_2})$, the conclusion of the above derivation
is equivalent to the conclusion of the lemma except for the typing contexts.
Finally, we must show that $(\Gamma + r\cdot\Delta + \Gamma') = ((\Gamma_1 + r\cdot\Delta + \Gamma_1'') + \Gamma_2)$.
This holds from the following reasoning:
\begin{align*}
(\Gamma + r\cdot\Delta + \Gamma')
    &= (\Gamma_1' + \incl{\Gamma_2}{\Gamma_1'}) + r\cdot\Delta + (\Gamma_1'' + \excl{\Gamma_2}{\Gamma_1'})\tag{$\because$ $\Gamma = (\Gamma_1' + \incl{\Gamma_2}{\Gamma_1'})$ \& $\Gamma' = (\Gamma_1'' + \excl{\Gamma_2}{\Gamma_1'})$}\\
    &= \Gamma_1' + \incl{\Gamma_2}{\Gamma_1'} + r\cdot\Delta + \Gamma_1'' + \excl{\Gamma_2}{\Gamma_1'}\tag{$\because$ $+$ associativity}\\
    &= \Gamma_1' + r\cdot\Delta + \Gamma''_1 + \incl{\Gamma_2}{\Gamma_1'} + \excl{\Gamma_2}{\Gamma_1'}\tag{$\because$ $+$ commutativity}\\
    &= (\Gamma_1' + r\cdot\Delta + \Gamma''_1) + (\incl{\Gamma_2}{\Gamma_1'} + \excl{\Gamma_2}{\Gamma_1'})\tag{$\because$ $+$ associativity}\\
    &= (\Gamma_1' + r\cdot\Delta + \Gamma''_1) + \Gamma_2\tag{$\because$ Lemma \ref{lemma:restriction}}
\end{align*}
Thus, we obtain the conclusion of the lemma.

\item Suppose $x \notin \mathrm{dom}(\Gamma_1)$ and $(x:\verctype{A}{r}) \in \Gamma_2$\\
Let $\Gamma_2'$ and $\Gamma_2''$ be typing contexts such that they satisfy $\Gamma_2 = (\Gamma_2', x:\verctype{A}{r}, \Gamma_2'')$.
The last typing derivation of ($\textsc{app}$) is rewritten as follows:
\begin{center}
    \begin{minipage}{.8\linewidth}
        \infrule[app]{
            \Gamma_1 \vdash t_1 : \ftype{B_1}{B_2}
            \andalso
            \Gamma_2', x:\verctype{A}{r}, \Gamma_2'' \vdash t_2 : B_1
        }{
             \Gamma_1 + (\Gamma_2', x:\verctype{A}{r}, \Gamma_2'') \vdash \app{t_1}{t_2} : B_2
        }
    \end{minipage}
\end{center}
This case is similar to the case $(x:\verctype{A}{r})\in \Gamma_1$ and $x \notin \mathrm{dom}(\Gamma_2)$, but using \ref{lemma:shuffle} (2) instead of \ref{lemma:shuffle} (1).\\

\item Suppose $(x:\verctype{A}{r_1}) \in \Gamma_1$ and $(x:\verctype{A}{r_2}) \in \Gamma_2$ where $r=r_1\oplus r_2$.\\
Let $\Gamma_1'$, $\Gamma_1''$, $\Gamma_2'$, and $\Gamma_2''$ be typing contexts such that they satisfy $\Gamma_1 = (\Gamma_1', x:\verctype{A}{r_1}, \Gamma_1'')$ and $\Gamma_2 = (\Gamma_2', x:\verctype{A}{r_2}, \Gamma_2'')$.
The last derivation of (\textsc{app}) is rewritten as follows:
\begin{center}
    \begin{minipage}{.80\linewidth}
        \infrule[app]{
             \Gamma_1', x:\verctype{A}{r_1}, \Gamma_1'' \vdash t_1 : \ftype{B_1}{B_2}
            \\
             \Gamma_2', x:\verctype{A}{r_2}, \Gamma_2'' \vdash t_2 : B_1
        }{
             (\Gamma_1', x:\verctype{A}{r_1}, \Gamma_1'') + (\Gamma_2', x:\verctype{A}{r_2}, \Gamma_2'') \vdash \app{t_1}{t_2} : B_2
        }
    \end{minipage}
\end{center}
Now, we compare the typing contexts between the lemma and the above conclusion as follows:
\begin{align*}
(\Gamma, x:\verctype{A}{r}, \Gamma')
    &= (\Gamma_1', x:\verctype{A}{r_1}, \Gamma_1'') + (\Gamma_2', x:\verctype{A}{r_2}, \Gamma_2'')\\
    &= (\Gamma_1', \Gamma_1'', x:\verctype{A}{r_1}) + (\Gamma_2', \Gamma_2'', x:\verctype{A}{r_2})\tag{$\because$ $,$ commutativity}\\
    &= ((\Gamma_1', \Gamma_1'') + (\Gamma_2', \Gamma_2'')), x:\verctype{A}{r_1\oplus r_2}\tag{$\because$ $+$ definiton}\\
    &= ((\Gamma_1'+\incl{\Gamma'_2}{\Gamma'_1}+\incl{\Gamma''_2}{\Gamma'_1}), (\Gamma''_1+\excl{\Gamma'_2}{\Gamma'_1}+\excl{\Gamma''_2}{\Gamma'_1})),x:\verctype{A}{r_1\oplus r_2}\tag{$\because$ Lemma \ref{lemma:shuffle} (3)}\\
    &= (\Gamma_1'+\incl{\Gamma'_2}{\Gamma'_1}+\incl{\Gamma''_2}{\Gamma'_1}), (\Gamma''_1+\excl{\Gamma'_2}{\Gamma'_1}+\excl{\Gamma''_2}{\Gamma'_1}), x:\verctype{A}{r_1\oplus r_2}\tag{$\because$ $,$ associativity}\\
    &= (\Gamma_1'+\incl{\Gamma'_2}{\Gamma'_1}+\incl{\Gamma''_2}{\Gamma'_1}),x:\verctype{A}{r_1\oplus r_2}, (\Gamma''_1+\excl{\Gamma'_2}{\Gamma'_1}+\excl{\Gamma''_2}{\Gamma'_1})\tag{$\because$ $,$ commutativity}
\end{align*}
By the commutativity of "$,$", we can take $\Gamma$ and $\Gamma'$ arbitrarily so that they satisfy the above equation. So here we know $\Gamma = (\Gamma_1'+\incl{\Gamma'_2}{\Gamma'_1}+\incl{\Gamma''_2}{\Gamma'_1})$ and $\Gamma' = (\Gamma''_1+\excl{\Gamma'_2}{\Gamma'_1}+\excl{\Gamma''_2}{\Gamma'_1})$.\par
We then apply the induction hypothesis to each of the two premises of the last derivation and reapply (\textsc{app}) as follows:
\begin{center}
    \begin{minipage}{\linewidth}
        \infrule[app]{
             \Gamma_1' + r_1\cdot\Delta + \Gamma_1'' \vdash [t'/x]t_1 : \ftype{B_1}{B_2}
            \\
             \Gamma_2' + r_2\cdot\Delta + \Gamma_2'' \vdash [t'/x]t_2 : B_1
        }{
             (\Gamma_1' + r_1\cdot\Delta + \Gamma_1'') + (\Gamma_2' + r_2\cdot\Delta + \Gamma_2'') \vdash \app{([t'/x]t_1)}{([t'/x]t_2)} : B_2
        }
    \end{minipage}
\end{center}
Since $\app{([t'/x]t_1)}{([t'/x]t_2)}=[t'/x](\app{t_1}{t_2})$, the conclusion of the above derivation is equivalent to the conclusion of the lemma except for the typing contexts.
Finally, we must show that $\Gamma + r\cdot\Delta + \Gamma' = (\Gamma_1' + r_1\cdot\Delta + \Gamma_1'') + (\Gamma_2' + r_2\cdot\Delta + \Gamma_2'')$.
\begin{align*}
(\Gamma + r\cdot\Delta + \Gamma')
    &= (\Gamma_1'+\incl{\Gamma'_2}{\Gamma'_1}+\incl{\Gamma''_2}{\Gamma'_1}) + (r_1\oplus r_2)\cdot\Delta +  (\Gamma''_1+\excl{\Gamma'_2}{\Gamma'_1}+\excl{\Gamma''_2}{\Gamma'_1})\tag{$\because$ $r=r_1\oplus r_2$ \& $\Gamma = (\Gamma_1'+\incl{\Gamma'_2}{\Gamma'_1}+\incl{\Gamma''_2}{\Gamma'_1})$ \& $\Gamma' = (\Gamma''_1+\excl{\Gamma'_2}{\Gamma'_1}+\excl{\Gamma''_2}{\Gamma'_1})$}\\
    &= \Gamma_1' + (r_1\oplus r_2)\cdot\Delta + \Gamma''_1 + (\incl{\Gamma'_2}{\Gamma'_1}+\excl{\Gamma'_2}{\Gamma'_1}) + (\incl{\Gamma''_2}{\Gamma'_1}+\excl{\Gamma''_2}{\Gamma'_1}) \tag{$\because$ $+$ associativity \& commutativity}\\
    &= \Gamma_1' + (r_1\oplus r_2)\cdot\Delta + \Gamma''_1 + \Gamma'_2 + \Gamma''_2\tag{$\because$ Lemma \ref{lemma:restriction}}\\
    &= \Gamma_1' + r_1\cdot\Delta + r_2\cdot\Delta + \Gamma''_1 + \Gamma'_2 + \Gamma''_2\tag{$\because$ Lemma \ref{lemma:distributivelaw}}\\
    &= (\Gamma_1' + r_1\cdot\Delta + \Gamma''_1) + (\Gamma'_2 + r_2\cdot\Delta + \Gamma''_2)\tag{$\because$ $+$ associativity and commutativity}
\end{align*}
Thus, we obtain the conclusion of the lemma.
\\
\end{itemize}


\item Case (\textsc{weak})
\begin{center}
    \begin{minipage}{.4\linewidth}
        \infrule[weak]{
            \Gamma'' \vdash t : B
            \andalso
            \vdash \Delta'
        }{
            \Gamma'' + \verctype{\Delta'}{0} \vdash t : B
        }
    \end{minipage}
\end{center}

In this case, we know $(\Gamma, x:\verctype{A}{r}, \Gamma') = \Gamma'' + \verctype{\Delta'}{0}$.
There are two cases where the versioned assumption $x:\verctype{A}{r}$ is contained in $\verctype{\Delta'}{0}$ and not included.

\begin{itemize}
\item Suppose $(x:\verctype{A}{r})\in\verctype{\Delta'}{0}$\\
We know $r=0$.
Let $\Delta_1$ and $\Delta_2$ be typing context such that $\Delta' = (\Delta_1, x:\verctype{A}{0}, \Delta_2)$.
% このとき$x:\verctype{A}{0} \notin \Gamma''$としてよい。
The last derivation is rewritten as follows:
\begin{center}
    \begin{minipage}{.55\linewidth}
        \infrule[weak]{
            \Gamma'' \vdash t : B
            \andalso
            \vdash \Delta_1+\Delta+\Delta_2
        }{
            \Gamma'' + [\Delta_1, x:\verctype{A}{0}, \Delta_2]_0 \vdash t : B
        }
    \end{minipage}
\end{center}

We compare the typing contexts between the conclusion of the lemma and that of the above derivation to obtain the following:
\begin{align*}
(\Gamma, x:\verctype{A}{0}, \Gamma')
    &= \Gamma'' + [\Delta_1, x:\verctype{A}{0}, \Delta_2]_0\tag{$\because$ $\Delta' = (\Delta_1, x:\verctype{A}{0}, \Delta_2)$}\\
    &= \Gamma'' + (\verctype{\Delta_1}{0}, x:\verctype{A}{0}, \verctype{\Delta_2}{0}) \tag{$\because$ definiton of $[\Gamma]_0$}\\
    &= (\incl{\Gamma''}{\verctype{\Delta_2}{0}} + \verctype{\Delta_1}{0}), x:\verctype{A}{0}, (\excl{\Gamma''}{\verctype{\Delta_2}{0}} + \verctype{\Delta_2}{0})\tag{$\because$ Lemma \ref{lemma:shuffle} (2)}
\end{align*}
By the commutativity of "$,$", we can take $\Gamma$ and $\Gamma'$ arbitrarily so that they satisfy the above equation. So here we know $\Gamma = (\incl{\Gamma''}{\verctype{\Delta_2}{0}} + \verctype{\Delta_1}{0})$ and $\Gamma' = (\excl{\Gamma''}{\verctype{\Delta_2}{0}} + \verctype{\Delta_2}{0})$.\par
We then apply the induction hypothesis to the premise of the last derivation and reapply (\textsc{weak}) as follows:
\begin{center}
    \begin{minipage}{.6\linewidth}
        \infrule[weak]{
            \Gamma'' \vdash [t'/x]t : B
            \andalso
            \vdash \Delta_1+\Delta+\Delta_2
        }{
            \Gamma'' + \verctype{\Delta_1+\Delta+\Delta_2}{0} \vdash [t'/x]t : B
        }
    \end{minipage}
\end{center}
where we choose $\Delta_1+\Delta+\Delta_2$ as the newly added typing context.
Since $x$ is unused by $t$, thus note that $[t'/x]t = t$, the conclusion of the above derivation is equivalent to the conclusion of the lemma except for typing contexts.\par
Finally, we must show that $(\Gamma + r\cdot\Delta + \Gamma') = \Gamma'' + \verctype{\Delta_1 + \Delta + \Delta_2}{0}$.
\begin{align*}
(\Gamma + r\cdot\Delta + \Gamma')
    &= (\incl{\Gamma''}{\verctype{\Delta_2}{0}} + \verctype{\Delta_1}{0}) + \verctype{\Delta}{0} + (\excl{\Gamma''}{\verctype{\Delta_2}{0}} + \verctype{\Delta_2}{0}) \tag{$\because$ $r=0$ \& $\Gamma = (\incl{\Gamma''}{\verctype{\Delta_2}{0}} + \verctype{\Delta_1}{0})$ \& $\Gamma' = (\excl{\Gamma''}{\verctype{\Delta_2}{0}} + \verctype{\Delta_2}{0})$}\\
    &= (\incl{\Gamma''}{\verctype{\Delta_2}{0}} + \excl{\Gamma''}{\verctype{\Delta_2}{0}}) + (\verctype{\Delta_1}{0} + \verctype{\Delta}{0} + \verctype{\Delta_2}{0}) \tag{$\because$ $+$ associativity and commutativity}\\
    &= \Gamma'' + (\verctype{\Delta_1}{0} + \verctype{\Delta}{0} + \verctype{\Delta_2}{0}) \tag{$\because$ Lemma \ref{lemma:restriction}}\\
    &= \Gamma'' + \verctype{\Delta_1 + \Delta + \Delta_2}{0} \tag{$\because$ definition of $[\Gamma]_0$}
\end{align*}
Thus, we obtain the conclusion of the lemma.

\item Suppose $(x:\verctype{A}{r})\notin\verctype{\Delta'}{0}$\\
Let $\Gamma_1$ and $\Gamma_2$ be typing context such that $\Gamma'' = (\Gamma_1, x:\verctype{A}{r}, \Gamma_2)$.
The last typing derivation of $(\textsc{weak})$ is rewritten as follows:
\begin{center}
    \begin{minipage}{.65\linewidth}
        \infrule[weak]{
            (\Gamma_1, x:\verctype{A}{r}, \Gamma_2) \vdash t : B
            \andalso
            \vdash \Delta'
        }{
            (\Gamma_1, x:\verctype{A}{r}, \Gamma_2) + \verctype{\Delta'}{0} \vdash t : B
        }
    \end{minipage}
\end{center}

We then compare the typing context between the conclusion of the lemma and that of the that of above derivation as follows:
\begin{align*}
(\Gamma, x:\verctype{A}{r}, \Gamma')
    &= (\Gamma_1, x:\verctype{A}{r}, \Gamma_2) + \verctype{\Delta'}{0}\\
    &= (\Gamma_1 + \incl{(\verctype{\Delta'}{0})}{\Gamma_1}),x:\verctype{A}{r},(\Gamma_2 + \excl{(\verctype{\Delta'}{0})}{\Gamma_1})\tag{$\because$ Lemma \ref{lemma:shuffle} (1)}
\end{align*}
By the commutativity of "$,$", we can take $\Gamma$ and $\Gamma'$ arbitrarily so that they satisfy the above equation. So here we know $\Gamma = (\Gamma_1 + \incl{(\verctype{\Delta'}{0})}{\Gamma_1})$ and $\Gamma' = (\Gamma_2 + \excl{(\verctype{\Delta'}{0})}{\Gamma_1})$.
We then apply the induction hypothesis to the premise of the last derivation and reapply (\textsc{weak}) as follows:
\begin{center}
    \begin{minipage}{.7\linewidth}
        \infrule[weak]{
            \Gamma_1 + r\cdot\Delta + \Gamma_2 \vdash [t'/x]t : B
            \andalso
            \vdash \Delta'
        }{
            (\Gamma_1 + r\cdot\Delta + \Gamma_2) + \verctype{\Delta'}{0} \vdash [t'/x]t : B
        }
    \end{minipage}
\end{center}
The conclusion of the above derivation is equivalent to the conclusion of the lemma except for the typing contexts.
Finally, we must show that $(\Gamma + r\cdot\Delta + \Gamma') = (\Gamma_1 + r\cdot\Delta + \Gamma_2) + \verctype{\Delta'}{0}$.
\begin{align*}
(\Gamma + r\cdot\Delta + \Gamma')
    &= (\Gamma_1 + \incl{(\verctype{\Delta'}{0})}{\Gamma_1}) + r\cdot\Delta + (\Gamma_2 + \excl{(\verctype{\Delta'}{0})}{\Gamma_1}) \tag{$\because$ $\Gamma = (\Gamma_1 + \incl{(\verctype{\Delta'}{0})}{\Gamma_1})$ \& $\Gamma' = (\Gamma_2 + \excl{(\verctype{\Delta'}{0})}{\Gamma_1})$}\\
    &= (\Gamma_1 + r\cdot\Delta + \Gamma_2) + (\incl{\verctype{\Delta'}{0}}{\Gamma_1} + \excl{\verctype{\Delta'}{0}}{\Gamma_1}) \tag{$\because$ $+$ associativity and commutativity}\\
    &= (\Gamma_1 + r\cdot\Delta + \Gamma_2) + \verctype{\Delta'}{0}\tag{$\because$ Lemma \ref{lemma:restriction}}
\end{align*}
Thus, we obtain the conclusion of the lemma.\\
\end{itemize}



\item Case (\textsc{der})
\begin{center}
    \begin{minipage}{.38\linewidth}
        \infrule[der]{
            \Gamma'', y:B_1 \vdash t : B_2
        }{
            \Gamma'', y:\verctype{B_1}{1} \vdash t : B_2
        }
    \end{minipage}
\end{center}
In this case, we know $(\Gamma'', y:\verctype{B_1}{1}) = (\Gamma, x:\verctype{A}{r}, \Gamma')$.
There are two cases in which the versioned assumption $x:\verctype{A}{r}$ is equivalent to $y:\verctype{B_1}{1}$ and not equivalent to.
\begin{itemize}
\item Suppose $x:\verctype{A}{r} = y:\verctype{B_1}{1}$\\
We know $x=y$, $A=B_1$, $r=1$, $\Gamma = \Gamma''$, and $\Gamma' = \emptyset$.
The last derivation is rewritten as follows:
\begin{center}
    \begin{minipage}{.38\linewidth}
        \infrule[der]{
            \Gamma'', x:A \vdash t : B_2
        }{
            \Gamma'', x:\verctype{A}{1} \vdash t : B_2
        }
    \end{minipage}
\end{center}
We then apply Lemma \ref{lemma:substitution1} to the premise to obtain the following:
\begin{align*}
    \Gamma'' + \Delta \vdash [t'/x]t : B_2
\end{align*}

Note that $\Delta$ is a versioned assumption by the assumption 1 and thus $\Gamma'' + \Delta = \Gamma'' + r\cdot\Delta$ where $r=1$, we obtain the conclusion of the lemma.

\item Suppose $x:\verctype{A}{r} \neq y:\verctype{B_1}{1}$\\
Let $\Gamma_1$ and $\Gamma_2$ be typing contexts such that $\Gamma'' = (\Gamma_1, x:\verctype{A}{r}, \Gamma_1')$.
The last derivation is rewritten as follows:
\begin{center}
    \begin{minipage}{.60\linewidth}
        \infrule[der]{
            (\Gamma_1, x:\verctype{A}{r}, \Gamma_1'), y:B_1 \vdash t : B_2
        }{
            (\Gamma_1, x:\verctype{A}{r}, \Gamma_1'), y:\verctype{B_1}{1} \vdash t : B_2
        }
    \end{minipage}
\end{center}
We then apply the induction hypothesis to the premise of the last derivation and reapply (\textsc{der}) to obtain the following:
\begin{center}
    \begin{minipage}{.70\linewidth}
        \infrule[der]{
            (\Gamma + r\cdot\Delta + \Gamma''), y:B_1 \vdash [t'/x]t : B_2
        }{
            (\Gamma + r\cdot\Delta + \Gamma''), y:\verctype{B_1}{1} \vdash [t'/x]t : B_2
        }
    \end{minipage}
\end{center}
Since $y:\verctype{B_1}{1}$ is desjoint with $\Gamma+r\cdot\Delta+\Gamma''$ and thus $((\Gamma + r\cdot\Delta + \Gamma''), y:\verctype{B_1}{1}) = \Gamma + r\cdot\Delta + (\Gamma'', y:\verctype{B_1}{1})$, we obtain the conclusion of the lemma.\\
\end{itemize}


\item Case (\textsc{pr})
\begin{center}
    \begin{minipage}{.40\linewidth}
        \infrule[pr]{
             [\Gamma_1] \vdash t : B
             \andalso
             \vdash r'
        }{
            r'\cdot[\Gamma_1] \vdash [t] : \vertype{r'}{B}
        }
    \end{minipage}
\end{center}
Let $r''$ be a version resouce and $\Gamma'_1$ and $\Gamma''_1$ be typing contexts such that $r'' \sqsubseteq r'$ and $[\Gamma_1] = [\Gamma'_1, x:\verctype{A}{r''}, \Gamma''_1]$.
The last derivation is rewritten as follows:
\begin{center}
    \begin{minipage}{.55\linewidth}
        \infrule[pr]{
             [\Gamma'_1, x:\verctype{A}{r''}, \Gamma''_1] \vdash t : B
        }{
            r'\cdot[\Gamma'_1, x:\verctype{A}{r''}, \Gamma''_1] \vdash [t] : \vertype{r'}{B} 
        }
    \end{minipage}
\end{center}
We then compare the conclusion of the lemma and the above conclusion.
\begin{align*}
(\Gamma, x:\verctype{A}{r}, \Gamma') &= r' \cdot [\Gamma_1]\\
&= r' \cdot [\Gamma'_1, x:\verctype{A}{r''}, \Gamma''_1]\tag{$\because$ $[\Gamma_1] = [\Gamma'_1, x:\verctype{A}{r''}, \Gamma''_1]$}\\
&= r'\cdot[\Gamma'_1], \ x:\verctype{A}{r''\otimes r'}, \ r'\cdot[\Gamma''_1]\tag{$\because$ $\cdot$ definition}\\
&= r'\cdot[\Gamma'_1], \ x:\verctype{A}{r'}, \ r'\cdot[\Gamma''_1] \tag{$\because$ $r''\sqsubseteq r'$}
\end{align*}
By the commutativity of "$,$", we can take $\Gamma$ and $\Gamma'$ arbitrarily so that they satisfy the above equation. So here we know $\Gamma = (r'\cdot[\Gamma'_1])$ and $\Gamma' = (r'\cdot[\Gamma''_1])$.\par
We then apply the induction hypothesis to the premise of the last derivation and reapply (\textsc{pr}) to obtain the following:
\begin{center}
    \begin{minipage}{.6\linewidth}
        \infrule[pr]{
             [\Gamma'_1 + r''\cdot\Delta + \Gamma''_1] \vdash [t'/x]t : B
             \andalso
             \vdash r'
        }{
            r'\cdot[\Gamma'_1 + r''\cdot\Delta + \Gamma''_1] \vdash [[t'/x]t] : \vertype{r'}{B} 
        }
    \end{minipage}
\end{center}
where we use $[\Gamma'_1, x:\verctype{A}{r''}, \Gamma''_1] = [\Gamma'_1], x:\verctype{A}{r''}, [\Gamma''_1]$ and $[\Gamma'_1 + r''\cdot\Delta + \Gamma''_1] = [\Gamma'_1] + r''\cdot\Delta + [\Gamma''_1]$ before applying (\textsc{pr}).\par
Since $[[t'/x]t] = [t'/x][t]$ by the definiton of substitution, the above conclusion is equivalent to the conclusion of the lemma except for the typing contexts.
Finally, we must show that $(\Gamma + r'\cdot\Delta + \Gamma') = r'\cdot[\Gamma'_1 + r''\cdot\Delta + \Gamma''_1]$ by the following reasoning:
\begin{align*}
(\Gamma + r'\cdot\Delta + \Gamma')
&= r'\cdot[\Gamma'_1] + r'\cdot\Delta + r'\cdot[\Gamma'_1]\tag{$\because$ $\Gamma = (r'\cdot[\Gamma'_1])$ \& $\Gamma' = (r'\cdot[\Gamma''_1])$}\\
&= r'\cdot[\Gamma'_1] + (r'\otimes r'')\cdot\Delta + r'\cdot[\Gamma'_1] \tag{$\because$ $r''\sqsubseteq r'$}\\
&= r'\cdot[\Gamma'_1] + r'\cdot ( r''\cdot\Delta) + r'\cdot[\Gamma'_1] \tag{$\because$ $\otimes$ associativity}\\
&= r'\cdot[\Gamma'_1 + r''\cdot\Delta + \Gamma'_1] \tag{$\because$ $\cdot$ distributive law over $+$}
\end{align*}
Thus, we obtain the conclusion of the lemma.
\\


\item Case (\textsc{ver})
\begin{center}
    \begin{minipage}{.60\linewidth}
        \infrule[ver]{
            [\Gamma_i] \vdash t_i : B
            \andalso
            \vdash \{\overline{l_i}\}
        }{
            \bigcup_i(\{l_i\}\cdot [\Gamma_i]) \vdash \nvval{\overline{l_i=t_i}} : \vertype{\{\overline{l_i}\}}{B}
        }
    \end{minipage}
\end{center}
% By the second premise, $\forall i,j.\mathrm{dom}(\Gamma_i) = \mathrm{dom}(\Gamma_i) = $.
% By using this partition, the last derivation is rewritten as follows:
% \begin{center}
%     \begin{minipage}{.60\linewidth}
%         \infrule[ver]{
%             [\Gamma'_i, x:\verctype{A}{\sigma_i}, \Gamma''_i] \vdash t_i : B
%         }{
%             \bigcup_i(\{l_i\}\cdot [\Gamma'_i, x:\verctype{A}{\sigma_i}, \Gamma''_i]) \vdash \nvval{\overline{l_i=t_i}} : \vertype{\overline{l_i}}{B}
%         }
%     \end{minipage}
% \end{center}
We compare the typing contexts between the lemma and the above conclusion as follows:
\begin{align*}
(\Gamma, x:\verctype{A}{r}, \Gamma') &= \bigcup_{i} \ \left(\{l_i\}\cdot[\Gamma_i]\right)\\
&= \bigcup_{i\in I_x} \ \left(\{l_i\}\cdot[\Gamma'_i, \ x:\verctype{A}{r_i}, \ \Gamma''_i]\right) + \bigcup_{i\in J_x} \ \left(\{l_i\}\cdot[\Gamma'_i, \ \Gamma''_i]\right)  \tag{$\because$ $I_x=\{i\,|\,x\in\mathrm{dom}(\Gamma_i)\}$ and $J_x=\{i\,|\,x\notin\mathrm{dom}(\Gamma_i)\}$}
\end{align*}
We then reorganise the typing context $\bigcup_{i\in I_x} \ \left(\{l_i\}\cdot[\Gamma'_i, \ x:\verctype{A}{r_i}, \ \Gamma''_i]\right)$ as follows:
\begin{align*}
&\bigcup_{i\in I_x} \ \left(\{l_i\}\cdot[\Gamma'_i, \ x:\verctype{A}{r_i}, \ \Gamma''_i]\right)\\
= &\bigcup_{i\in I_x} \ \left(\{l_i\}\cdot[x:\verctype{A}{r_i}, \ \Gamma'_i, \ \Gamma''_i]\right) \tag{$\because$ $,$ associativity}\\
= &\bigcup_{i\in I_x} \ \left(\{l_i\}\cdot(x:\verctype{A}{r_i}), \ \{l_i\}\cdot[\Gamma'_i], \ \{l_i\}\cdot[\Gamma''_i]\right) \tag{$\because$ $\cdot$ distributive law}\\
= &\bigcup_{i\in I_x} \ \left(\{l_i\}\cdot(x:\verctype{A}{r_i})\right), \ \bigcup_{i\in I_x} \ \left(\{l_i\}\cdot[\Gamma'_i], \{l_i\}\cdot[\Gamma''_i]\right) \tag{$\because$ Sum of each disjoint sub context}\\
= &\bigcup_{i\in I_x} \ \left(x:\verctype{A}{\{l_i\}\otimes r_i}\right), \ \bigcup_{i\in I_x} \ \left(\{l_i\}\cdot[\Gamma'_i], \{l_i\}\cdot[\Gamma''_i]\right) \tag{$\because$ $\cdot$ definition}\\
= &\ x:\verctype{A}{\sum_{i\in I_x}\{l_i\}\otimes r_i}, \ \ \bigcup_{i\in I_x} \ \left(\{l_i\}\cdot[\Gamma'_i], \{l_i\}\cdot[\Gamma''_i]\right) \tag{$\because$ $\bigcup$ and $+$ definition}
\end{align*}
Thus, we obtain the following:
\begin{align*}
&(\Gamma, x:\verctype{A}{r}, \Gamma')\\
= &\left(x:\verctype{A}{\sum_{i\in I_x}\{l_i\}\otimes r_i}, \ \ \bigcup_{i\in I_x} \ \left(\{l_i\}\cdot[\Gamma'_i], \{l_i\}\cdot[\Gamma''_i]\right)\right) + \bigcup_{i\in J_x} \ \left(\{l_i\}\cdot[\Gamma'_i, \ \Gamma''_i]\right)\\
= &\ x:\verctype{A}{\sum_{i\in I_x}\{l_i\}\otimes r_i}, \ \ \bigcup_{i} \ \left(\{l_i\}\cdot[\Gamma'_i], \{l_i\}\cdot[\Gamma''_i]\right) \tag{$\because$ $\bigcup_{i\in J_x} \ \left(\{l_i\}\cdot[\Gamma'_i, \ \Gamma''_i]\right)$ are disjoint with $x:\verctype{A}{\sum_{i\in I_x}\{l_i\}\otimes r_i}$}
\end{align*}
Therefore, By Lemma \ref{lemma:shufflecomposition}, there exists typing contexts $\Gamma'_{\overline{i}}$ and $\Gamma''_{\overline{i}}$ such that:
\begin{align*}
\Gamma'_{\overline{i}}, \Gamma''_{\overline{i}} &= \bigcup_{i} \ \left(\{l_i\}\cdot[\Gamma'_i], \{l_i\}\cdot[\Gamma''_i]\right)\\
\Gamma'_{\overline{i}} + \Gamma''_{\overline{i}} &= \bigcup_{i} \ \left(\{l_i\}\cdot[\Gamma'_i] + \{l_i\}\cdot[\Gamma''_i]\right)
\end{align*}
Thus, we obtain the following:
\begin{align*}
(\Gamma, x:\verctype{A}{r}, \Gamma')
&= x:\verctype{A}{\sum_{i\in I_x}\{l_i\}\otimes r_i}, \ \Gamma'_{\overline{i}}, \ \Gamma''_{\overline{i}}\\
&= \Gamma'_{\overline{i}}, \ x:\verctype{A}{\sum_{i\in I_x}\{l_i\}\otimes r_i}, \ \Gamma''_{\overline{i}} \tag{$\because$ $,$ commutativity}
\end{align*}
By the commutativity of "$,$", we can take $\Gamma$ and $\Gamma'$ arbitrarily so that they satisfy the above equation. So here we know $\Gamma = \Gamma'_{\overline{i}}$, $\Gamma' = \Gamma''_{\overline{i}}$, and $r=\sum_{i\in I_x}(\{l_i\}\otimes r_i)$.
We then apply the induction hypothesis to the premise whose typing context contains $x$.
Here, we define a typing context $\Delta_i$ as follows:
\begin{align*}
\Delta_i =
\left\{
\begin{aligned}
\Delta \hspace{1em} (i\in I_x)\\
\emptyset \hspace{1em} (i\in J_x)\\
\end{aligned}
\right.
\end{align*}
By using $\Delta_i$, we reapply (\textsc{ver}) as follows:
\begin{center}
    \begin{minipage}{.8\linewidth}
        \infrule[ver]{
            [\Gamma'_i + r_i\cdot\Delta_i + \Gamma''_i] \vdash [t'/x]t_i : B
            \andalso
            \vdash \{\overline{l_i}\}
        }{
            \bigcup_i(\{l_i\}\cdot [\Gamma'_i + r_i\cdot\Delta_i + \Gamma''_i]) \vdash \nvval{\overline{l_i=[t'/x]t_i}} : \vertype{\{\overline{l_i}\}}{B}
        }
    \end{minipage}
\end{center}
Since $\{\overline{l_i=[t'/x]t_i}\,|\,l_k\} = [t'/x]\{\overline{l_i=t_i}\,|\,l_k\}$ by the definition of substitution, the above conclusion is equivalent to the conclusion of the lemma except for typing contexts.
Finally, we must show that $(\Gamma + r\cdot\Delta + \Gamma') = \bigcup_i(\{l_i\}\cdot [\Gamma'_i + r_i\cdot\Delta_i + \Gamma''_i])$.
\begin{align*}
(\Gamma + r\cdot\Delta + \Gamma')
&= \Gamma'_{\overline{i}} + r\cdot\Delta + \Gamma''_{\overline{i}}\tag{$\because$ $\Gamma = \Gamma'_{\overline{i}}$ \& $\Gamma' = \Gamma''_{\overline{i}}$}\\
&= r\cdot\Delta + (\Gamma'_{\overline{i}} + \Gamma''_{\overline{i}}) \tag{$+$ associativity \& commutativity}\\
&= r\cdot\Delta + \bigcup_{i} \ \left(\{l_i\}\cdot[\Gamma'_i] + \{l_i\}\cdot[\Gamma''_i]\right) \tag{$\because$ $\Gamma'_{\overline{i}} + \Gamma''_{\overline{i}} = \bigcup_{i} \ \left(\{l_i\}\cdot[\Gamma'_i] + \{l_i\}\cdot[\Gamma''_i]\right)$}\\
&= (\sum_{i\in I_x}(\{l_i\}\otimes r_i))\cdot\Delta + \bigcup_{i} \ \left(\{l_i\}\cdot[\Gamma'_i] + \{l_i\}\cdot[\Gamma''_i]\right) \tag{$\because$ $r=\sum_{i\in I_x}(\{l_i\}\otimes r_i)$}\\
&= \bigcup_{i\in I_x}\left(\{l_i\}\cdot\left(r_i\cdot\Delta\right)\right) + \bigcup_{i} \ \left(\{l_i\}\cdot[\Gamma'_i] + \{l_i\}\cdot[\Gamma''_i]\right) \tag{$\because$ $\bigcup$ definition}\\
&= \bigcup_{i}\left(\{l_i\}\cdot\left(r_i\cdot\Delta_i\right)\right) + \bigcup_{i} \ \left(\{l_i\}\cdot[\Gamma'_i] + \{l_i\}\cdot[\Gamma''_i]\right) \tag{$\because$ $\Delta_i$ definition}\\
&= \bigcup_i\left(\{l_i\}\cdot\left(r_i\cdot\Delta_i\right) + \{l_i\}\cdot[\Gamma'_i] + \{l_i\}\cdot[\Gamma''_i]\right) \tag{$\because$ $+$ commutativity \& associativity}\\
&= \bigcup_{i} \left(\{l_i\}\cdot\left(\left(r_i\cdot\Delta_i\right) + [\Gamma'_i] + [\Gamma''_i]\right)\right) \tag{$\because$ districutive law}\\
&= \bigcup_{i} \left(\{l_i\}\cdot\left([\Gamma'_i] + \left(r_i\cdot\Delta_i\right) + [\Gamma''_i]\right)\right) \tag{$\because$ $+$ commutativity}\\
&= \bigcup_{i} \left(\{l_i\}\cdot[\Gamma'_i + r_i\cdot\Delta_i + \Gamma''_i]\right) \tag{$\because$ $[\cdot]$ definition}
\end{align*}
Thus, we obtain the conclusion of the lemma.
\\


\item Case (\textsc{veri})
\begin{center}
    \begin{minipage}{.55\linewidth}
        \infrule[veri]{
            \verctype{\Gamma_i}{} \vdash t_i : B
            \andalso
            \vdash \{\overline{l_i}\}
            \andalso
            l_k \in \{\overline{l_i}\}
        }{
            \bigcup_i(\{l_i\}\cdot [\Gamma_i]) \vdash \ivval{\overline{l_i=t_i}}{l_k} : B
        }
    \end{minipage}
\end{center}
This case is similar to the case of (\textsc{ver}).
\\

\item Case (\textsc{extr})
\begin{center}
    \begin{minipage}{.45\linewidth}
        \infrule[extr]{
            \Gamma \vdash t : \vertype{r}{A}
            \andalso
            l \in r
        }{
            \Gamma \vdash t.l : A
        }
    \end{minipage}
\end{center}
In this case, we apply the induction hypothesis to the premise and then reapply (\textsc{extr}), we obtain the conclusion of the lemma.
\\

\item Case (\textsc{sub})
\begin{center}
    \begin{minipage}{.55\linewidth}
        \infrule[\textsc{sub}]{
            \Gamma_1,y:\verctype{B'}{r_1}, \Gamma_2 \vdash t : B
            \andalso
            r_1 \sqsubseteq r_2
            \andalso
            \vdash r_2
        }{
            \Gamma_1,y:\verctype{B'}{r_2}, \Gamma_2 \vdash t : B
        }
    \end{minipage}
\end{center}
In this case, we know $(\Gamma, x:\verctype{A}{r}, \Gamma') = (\Gamma, y:\verctype{B'}{r_2}, \Gamma')$.
There are three cases where the versioned assumption $x:\verctype{A}{r}$ is included in $\Gamma_1$, included in $\Gamma_2$, or equal to $y:\verctype{B'}{r_2}$.
\begin{itemize}
\item Suppose $(x:\verctype{A}{r})\in\Gamma_1$.\\
Let $\Gamma'_1$ and $\Gamma''_1$ be typing contexts such that $\Gamma_1 = (\Gamma'_1, x:\verctype{A}{r}, \Gamma''_1)$.
The last derivation is rewritten as follows:
\begin{center}
    \begin{minipage}{.75\linewidth}
        \infrule[\textsc{sub}]{
            \Gamma'_1, x:\verctype{A}{r}, \Gamma''_1,y:\verctype{B'}{r_1}, \Gamma_2 \vdash t : B
            \andalso
            r_1 \sqsubseteq r_2
            \andalso
            \vdash r_2
        }{
            \Gamma'_1, x:\verctype{A}{r}, \Gamma''_1,y:\verctype{B'}{r_2}, \Gamma_2 \vdash t : B
        }
    \end{minipage}
\end{center}
We then apply the induction hypothesis to the premise of the last derivation to obtain the following:
\begin{align}
\label{eq:subset1}
\Gamma'_1 + r\cdot\Delta + (\Gamma''_1, y:\verctype{B'}{r_1}, \Gamma_2) \vdash [t'/x]t : B
\end{align}
The typing context of the above conclusion can be transformed as follows:
\begin{align*}
&\Gamma'_1 + r\cdot\Delta + (\Gamma''_1, y:\verctype{B'}{r_1}, \Gamma_2)\\
=\ &(\Gamma'_1+\incl{(r\cdot\Delta)}{\Gamma'_1}),\excl{(r\cdot\Delta)}{(\Gamma'_1,(\Gamma''_1, y:\verctype{B'}{r_1}, \Gamma_2))},\\
&\hspace{5em} \left((\Gamma''_1, y:\verctype{B'}{r_1}, \Gamma_2)+\incl{(r\cdot\Delta)}{(\Gamma''_1, y:\verctype{B'}{r_1}, \Gamma_2)}\right) \tag{$\because$ Lemma \ref{lemma:collapse}}\\
=\ &(\Gamma'_1+\incl{(r\cdot\Delta)}{\Gamma'_1}),\excl{(r\cdot\Delta)}{(\Gamma'_1,(\Gamma''_1, y:\verctype{B'}{r_1}, \Gamma_2))},\\
&\hspace{5em} \left((\Gamma''_1, y:\verctype{B'}{r_1}, \Gamma_2) + \left(\incl{(r\cdot\Delta)}{\Gamma''_1}, \incl{(r\cdot\Delta)}{(y:\verctype{B'}{r_1})}, \incl{(r\cdot\Delta)}{\Gamma_2})\right)\right) \tag{$\because$ \ref{def:restriction}}\\
=\ &(\Gamma'_1+\incl{(r\cdot\Delta)}{\Gamma'_1}),\excl{(r\cdot\Delta)}{(\Gamma'_1,(\Gamma''_1, y:\verctype{B'}{r_1}, \Gamma_2))},\\
&\hspace{5em} \left(\Gamma''_1+\incl{(r\cdot\Delta)}{\Gamma''_1}\right), \left(y:\verctype{B'}{r_1}+ \incl{(r\cdot\Delta)}{(y:\verctype{B'}{r_1})}\right), \left(\Gamma_2 + \incl{(r\cdot\Delta)}{\Gamma_2}\right) \tag{$\because$ Lemma \ref{lemma:shuffle}}\\
=\ &\Gamma_3, y:\verctype{B'}{r_1\oplus r_3}, \Gamma'_3
\end{align*}
The last equational transformation holds by the following equation \ref{eq:subset2}.\par
Let $\Gamma_3$ and $\Gamma'_3$ be typing contexts that satisfy the following:
\begin{align*}
\Gamma_3 &= (\Gamma'_1+\incl{(r\cdot\Delta)}{\Gamma'_1}),\excl{(r\cdot\Delta)}{(\Gamma'_1,(\Gamma''_1, y:\verctype{B'}{r_1}, \Gamma_2))}, \left(\Gamma''_1+\incl{(r\cdot\Delta)}{\Gamma''_1}\right)\\
\Gamma'_3 &= \left(\Gamma_2 + \incl{(r\cdot\Delta)}{\Gamma_2}\right)
\end{align*}

For $\incl{(r\cdot\Delta)}{(y:\verctype{B'}{r_1})}$,  Let $r_3$ and $r_3'$ be typing contexts such that $r_3=r\otimes r'_3$ and stisfy the following:
\begin{align*}
\incl{(r\cdot\Delta)}{(y:\verctype{B'}{r_1})} =
\left\{
\begin{aligned}
    & r\cdot(y:\verctype{B'}{r'_3}) = y:\verctype{B'}{r\otimes r'_3} = y:\verctype{B'}{r_3}  & (y \in \mathrm{dom}(\Delta))\\
    & \emptyset & (y \notin \mathrm{dom}(\Delta))
\end{aligned}
\right.
\end{align*}
Thus, we obtain the following equation.
\begin{align}
\label{eq:subset2}
y:\verctype{B'}{r_1} + \incl{(r\cdot\Delta)}{(y:\verctype{B'}{r_1})} = 
\left\{
\begin{aligned}
    & y:\verctype{B'}{r_1\oplus r_3} & (y \in \mathrm{dom}(\Delta))\\
    & y:\verctype{B'}{r_1\oplus r_3} = y:\verctype{B'}{r_1} & (y \notin \mathrm{dom}(\Delta))
\end{aligned}
\right.
\end{align}
Applying all of the above transformations and reapplying (\textsc{sub}) to the expression \ref{eq:subset1}, we obtain the following:
\begin{center}
    \begin{minipage}{\linewidth}
        \infrule[\textsc{sub}]{
            \Gamma_3, y:\verctype{B'}{r_1\oplus r_3}, \Gamma'_3 \vdash [t'/x]t : B
            \andalso
            (r_1 \oplus r_3) \sqsubseteq (r_2 \oplus r_3)
            \andalso
            \vdash r_2 \oplus r_3
        }{
            \Gamma_3, y:\verctype{B'}{r_2\oplus r_3}, \Gamma'_3 \vdash [t'/x]t : B
        }
    \end{minipage}
\end{center}
The conclusion of the above derivation is equivalent to the conclusion of the lemma except for the typing contexts. \par
Finally, we must show that $\Gamma'_1 + r\cdot\Delta + (\Gamma''_1, y:\verctype{B'}{r_2}, \Gamma_2) = (\Gamma_3, y:\verctype{B'}{r_2\oplus r_3}, \Gamma'_3)$.
\begin{align*}
&\Gamma'_1 + r\cdot\Delta + (\Gamma''_1, y:\verctype{B'}{r_2}, \Gamma_2)\\
=\ &(\Gamma'_1+\incl{(r\cdot\Delta)}{\Gamma'_1}),\excl{(r\cdot\Delta)}{(\Gamma'_1,(\Gamma''_1, y:\verctype{B'}{r_2}, \Gamma_2))},\\
&\hspace{5em} \left((\Gamma''_1, y:\verctype{B'}{r_2}, \Gamma_2)+\incl{(r\cdot\Delta)}{(\Gamma''_1, y:\verctype{B'}{r_2}, \Gamma_2)}\right) \tag{$\because$ Lemma \ref{lemma:collapse}}\\
=\ &(\Gamma'_1+\incl{(r\cdot\Delta)}{\Gamma'_1}),\excl{(r\cdot\Delta)}{(\Gamma'_1,(\Gamma''_1, y:\verctype{B'}{r_2}, \Gamma_2))},\\
&\hspace{5em} \left((\Gamma''_1, y:\verctype{B'}{r_2}, \Gamma_2) + \left(\incl{(r\cdot\Delta)}{\Gamma''_1}, \incl{(r\cdot\Delta)}{(y:\verctype{B'}{r_2})}, \incl{(r\cdot\Delta)}{\Gamma_2})\right)\right) \tag{$\because$ \ref{def:restriction}}\\
=\ &(\Gamma'_1+\incl{(r\cdot\Delta)}{\Gamma'_1}),\excl{(r\cdot\Delta)}{(\Gamma'_1,(\Gamma''_1, y:\verctype{B'}{r_2}, \Gamma_2))},\\
&\hspace{5em} \left(\Gamma''_1+\incl{(r\cdot\Delta)}{\Gamma''_1}\right), \left(y:\verctype{B'}{r_2}+ \incl{(r\cdot\Delta)}{(y:\verctype{B'}{r_2})}\right), \left(\Gamma_2 + \incl{(r\cdot\Delta)}{\Gamma_2}\right) \tag{$\because$ Lemma \ref{lemma:shuffle}}\\
=\ &\Gamma_3, y:\verctype{B'}{r_2\oplus r_3}, \Gamma'_3
\end{align*}
The last transformation is based on the following equation that can be derived from the definition \ref{def:restriction}.
\begin{align*}
\excl{(r\cdot\Delta)}{(\Gamma'_1,(\Gamma''_1, y:\verctype{B'}{r_1}, \Gamma_2))} &= \excl{(r\cdot\Delta)}{(\Gamma'_1,(\Gamma''_1, y:\verctype{B'}{r_2}, \Gamma_2))}\\
\incl{(r\cdot\Delta)}{(y:\verctype{B'}{r_1})} &= \incl{(r\cdot\Delta)}{(y:\verctype{B'}{r_2})}
\end{align*}
Thus, we obtain the conclusion of the lemma.

\item Suppose $(x:\verctype{A}{r})\in\Gamma_2$.\\
This case is similar to the case of $(x:\verctype{A}{r})\in\Gamma_1$.

\item Suppose $(x:\verctype{A}{r}) = y:\verctype{B'}{r_2}$.\\
The last derivation is rewritten as follows:
\begin{center}
    \begin{minipage}{.55\linewidth}
        \infrule[\textsc{sub}]{
            \Gamma_1,x:\verctype{A}{r'}, \Gamma_2 \vdash t : B
            \andalso
            r' \sqsubseteq r
            \andalso
            \vdash r
        }{
            \Gamma_1,x:\verctype{A}{r}, \Gamma_2 \vdash t : B
        }
    \end{minipage}
\end{center}
We apply the induction hypothesis to the premise and then reapply (\textsc{sub}), we obtain the conclusion of the lemma.
\end{itemize}

\end{itemize}
\end{proof}

%%%%%%%%%%%%%%%%%%%%%%%%%%%%%%%%%%%%%%%%%%%%%%%%%%%%%%%%%%%%%%%%%%%%%%%%%%%%%
%%%%%%%%%%%%%%%%%%%%%%%%%%%%%%%%%%%%%%%%%%%%%%%%%%%%%%%%%%%%%%%%%%%%%%%%%%%%%
%%%%%%%%%%%%%%%%%%%%%%%%%%%%%%%%%%%%%%%%%%%%%%%%%%%%%%%%%%%%%%%%%%%%%%%%%%%%%
%%%%%%%%%%%%%%%%%%%%%%%%%%%%%%%%%%%%%%%%%%%%%%%%%%%%%%%%%%%%%%%%%%%%%%%%%%%%%
%%%%%%%%%%%%%%%%%%%%%%%%%%%%%%%%%%%%%%%%%%%%%%%%%%%%%%%%%%%%%%%%%%%%%%%%%%%%%
%%%%%%%%%%%%%%%%%%%%%%%%%%%%%%%%%%%%%%%%%%%%%%%%%%%%%%%%%%%%%%%%%%%%%%%%%%%%%


\subsection{Type Safety}
\label{appendix:typesafety}
\begin{lemma}[Inversion lemma]
\label{lemma:typedvalue}
Let $v$ be a value such that $\Gamma \vdash v:A$. The followings hold for a type $A$.
\begin{itemize}
  \item $A=\inttype \Longrightarrow $ $v = n$ for some integer constant $n$.
  \item $A=\vertype{r}{B} \Longrightarrow $ $v = \pr{t'}$ for some term $t'$, or $v=\nvval{\overline{l_i=t_i}}$ for some terms $t_i$ and some labels $l_i\in r$.
  \item $A=\apptype{B}{B'} \Longrightarrow  $ $v=\lam{p}{t}$ for some pattern $p$ and term $t$.
  % \item \emph{otherwise} $\Longrightarrow$ $v=x$ for some variable $x\in \mathrm{dom}(\Gamma)$. (変数は値ではない。)
\end{itemize}
\end{lemma}

\begin{lemma}[Type safety for default version overwriting]
\label{lemma:overwriting}

For any version label $l$:
\begin{align*}
    \Gamma \vdash t : A
    \hspace{1em}\Longrightarrow\hspace{1em}
    \Gamma \vdash t@l : A
\end{align*}
\end{lemma}
\begin{proof}
The proof is given by induction on the typing derivation of $\Gamma \vdash t : A$.
Consider the cases for the last rule used in the typing derivation of assumption.
\\

\begin{itemize}
\item Case (\textsc{int})
\begin{center}
    \begin{minipage}{.25\linewidth}
        \infrule[int]{
            \\
        }{
            \emptyset \vdash n:\textsf{Int}
        }
    \end{minipage}
\end{center}
This case holds trivially because $n@l \equiv n$ for any labels $l$.
\\

\item Case (\textsc{var})
\begin{center}
    \begin{minipage}{.35\linewidth}
        \infrule[var]{
            \vdash A
        }{
            x:A \vdash x:A
        }
    \end{minipage}
\end{center}
This case holds trivially because $x@l = x$ for any labels $l$.
\\

\item Case (\textsc{abs})
\begin{center}
    \begin{minipage}{.55\linewidth}
        \infrule[abs]{
            - \vdash p : B_1 \rhd \Delta'
            \andalso
            \Gamma, \Delta' \,\vdash\, t_1 : A_2%\theta B
        }{
            \Gamma \,\vdash\, \lam{p}{t_1} : \ftype{A_1}{A_2}
        }
    \end{minipage}
\end{center}
By induction hypothesis, there exists a term $t_1@l$ such that:
\begin{align*}
\Gamma, \Delta' \,\vdash\, t_1@l : A_2
\end{align*}
We then reapply (\textsc{abs}) to obtain the following:
\begin{center}
    \begin{minipage}{.65\linewidth}
        \infrule[abs]{
            - \vdash p : B_1 \rhd \Delta'
            \andalso
            \Gamma, \Delta' \,\vdash\, t_1@l : A_2%\theta B
        }{
            \Gamma \,\vdash\, \lam{p}{(t_1@l)} : \ftype{A_1}{A_2}
        }
    \end{minipage}
\end{center}
Thus, note that $(\lam{p}{t_1})@l \equiv \lam{p}{(t_1@l)}$, we obtain the conclusion of the lemma.
\\

\item Case (\textsc{app})
\begin{center}
    \begin{minipage}{.65\linewidth}
        \infrule[app]{
            \Gamma_1 \vdash t_1 : \ftype{B}{A}
            \andalso
            \Gamma_2 \vdash t_2 : B
        }{
            \Gamma_1 + \Gamma_2 \vdash \app{t_1}{t_2} : A
        }
    \end{minipage}
\end{center}
By induction hypothesis, there exists terms $t_1@l$ and $t_2@l$ such that:
\begin{align*}
\Gamma_1 &\vdash t_1@l : \ftype{B}{A}\\
\Gamma_2 &\vdash t_2@l : B
\end{align*}
We then reapply (\textsc{app}) to obtain the following:
\begin{center}
    \begin{minipage}{.7\linewidth}
        \infrule[app]{
            \Gamma_1 \vdash t_1@l : \ftype{B}{A}
            \andalso
            \Gamma_2 \vdash t_2@l : B
        }{
            \Gamma_1 + \Gamma_2 \vdash \app{(t_1@l)}{(t_2@l)} : A
        }
    \end{minipage}
\end{center}
Thus, note that $(\app{t_1}{t_2})@l \equiv \app{(t_1@l)}{(t_2@l)}$, we obtain the conclusion of the lemma.
\\

\item Case (\textsc{let})
\begin{center}
    \begin{minipage}{.70\linewidth}
        \infrule[let]{
            \Gamma_1 \,\vdash\, t_1 : \vertype{r}{A}
            \andalso
            \Gamma_2, x:\verctype{A}{r} \,\vdash\, t_2 : B
        }{
            \Gamma_1 + \Gamma_2 \,\vdash\, \clet{x}{t_1}{t_2} : B
        }
    \end{minipage}
\end{center}
By induction hypothesis, there exists terms $t_1@l$ and $2@l$ such that:
\begin{align*}
\Gamma_1 \,&\vdash\, t_1@l : \vertype{r}{A}\\
\Gamma_2, x:\verctype{A}{r} \,&\vdash\, t_2@l : B
\end{align*}
We then reapply (\textsc{let}) to obtain the following:
\begin{center}
    \begin{minipage}{.75\linewidth}
        \infrule[let]{
            \Gamma_1 \,\vdash\, t_1@l : \vertype{r}{A}
            \andalso
            \Gamma_2, x:\verctype{A}{r} \,\vdash\, t_2@l : B
        }{
            \Gamma_1 + \Gamma_2 \,\vdash\, \clet{x}{(t_1@l)}{(t_2@l)} : B
        }
    \end{minipage}
\end{center}
Thus, note that $(\clet{x}{t_1}{t_2})@l \equiv \clet{x}{(t_1@l)}{(t_2@l)}$, we obtain the conclusion of the lemma.
\\

\item Case (\textsc{weak})
\begin{center}
    \begin{minipage}{.45\linewidth}
        \infrule[weak]{
            \Gamma_1 \vdash t : A
            \andalso
            \vdash \Delta'
        }{
            \Gamma_1 + \verctype{\Delta'}{0} \vdash t : A
        }
    \end{minipage}
\end{center}
By induction hypothesis, we know the following:
\begin{align*}
\Gamma_1 \vdash t@l : A
\end{align*}
We then reapply (\textsc{weak}) to obtain the following:
\begin{center}
    \begin{minipage}{.45\linewidth}
        \infrule[weak]{
            \Gamma_1 \vdash t@l : A
        }{
            \Gamma_1 + \verctype{\Delta'}{0} \vdash t@l : A
        }
    \end{minipage}
\end{center}
Thus, we obtain the conclusion of the lemma.
\\

\item Case (\textsc{der})
\begin{center}
    \begin{minipage}{.45\linewidth}
        \infrule[der]{
            \Gamma_1, x:B \vdash t : A
        }{
            \Gamma_1, x:\verctype{B}{1} \vdash t : A
        }
    \end{minipage}
\end{center}
By induction hypothesis, there exists terms $t@l$ such that:
\begin{align*}
\Gamma_1, x:B \vdash t@l : A
\end{align*}
We then reapply (\textsc{der}) to obtain the following:
\begin{center}
    \begin{minipage}{.45\linewidth}
        \infrule[der]{
            \Gamma_1, x:B \vdash t@l : A
        }{
            \Gamma_1, x:\verctype{B}{1} \vdash t@l : A
        }
    \end{minipage}
\end{center}
Thus, we obtain the conclusion of the lemma.
\\

\item Case (\textsc{pr})
\begin{center}
    \begin{minipage}{.45\linewidth}
        \infrule[pr]{
            [\Gamma] \vdash t : B
            \andalso
            \vdash r
        }{
            r\cdot\verctype{\Gamma}{} \vdash \pr{t} : \vertype{r}{B} 
        }
    \end{minipage}
\end{center}
This case holds trivially  because $\pr{t}@l \equiv \pr{t}$ for any labels $l$.
\\

\item Case (\textsc{ver})
\begin{center}
    \begin{minipage}{.65\linewidth}
        \infrule[ver]{
            \verctype{\Gamma_i}{} \vdash t_i : A
            \andalso
            \vdash \{\overline{l_i}\}
        }{
            \bigcup_i(\{l_i\}\cdot [\Gamma_i]) \vdash \nvval{\overline{l_i=t_i}} : \vertype{\{\overline{l_i}\}}{A}
        }
    \end{minipage}
\end{center}
This case holds trivially because $\nvval{\overline{l_i=t_i}}@l\equiv \nvval{\overline{l_i=t_i}}$ for any labels $l$.
\\

\item Case (\textsc{veri})
\begin{center}
    \begin{minipage}{.55\linewidth}
        \infrule[veri]{
            \verctype{\Gamma_i}{} \vdash t_i : A
            \andalso
            \vdash \{\overline{l_i}\}
            \andalso
            l_k \in \{\overline{l_i}\}
        }{
            \bigcup_i(\{l_i\}\cdot [\Gamma_i]) \vdash \ivval{\overline{l_i=t_i}}{l_k} : A
        }
    \end{minipage}
\end{center}
In this case, there are two possibilities for the one step evaluation of $t$.
\begin{itemize}
\item Suppose $l \in \{\overline{l_i}\}$.\\
We can apply the default version overwriting as follows:
\begin{center}
    \begin{minipage}{.40\linewidth}
        \infrule[]{
            l \in \{\overline{l_i}\}
        }{
            \ivval{\overline{l_i=t_i}}{l_k}@l \,\equiv\, \ivval{\overline{l_i=t_i}}{l}
        }
    \end{minipage}
\end{center}
In this case, we can derive the type of $\ivval{\overline{l_i=t_i}}{l}$ as follows:
\begin{center}
    \begin{minipage}{.65\linewidth}
        \infrule[veri]{
            \verctype{\Gamma_i}{} \vdash t_i : A
        }{
            \bigcup_i(\{l_i\}\cdot [\Gamma_i]) \vdash \ivval{\overline{l_i=t_i}}{l} : A
        }
    \end{minipage}
\end{center}
Thus, we obtain the conclusion of the lemma.
\item Suppose $l \notin \{\overline{l_i}\}$.\\
We can apply the default version overwriting as follows:
\begin{center}
    \begin{minipage}{.50\linewidth}
        \infrule[]{
            l \notin \{\overline{l_i}\}
        }{
            \ivval{\overline{l_i=t_i}}{l_k}@l \,\equiv\, \ivval{\overline{l_i=t_i}}{l_k}
        }
    \end{minipage}
\end{center}
This case holds trivially because $\ivval{\overline{l_i=t_i}}{l_k}@l = \ivval{\overline{l_i=t_i}}{l_k}$.\\
\end{itemize}

\item Case (\textsc{extr})
\begin{center}
    \begin{minipage}{.50\linewidth}
        \infrule[extr]{
            \Gamma \vdash t_1 : \vertype{r}{A}
            \andalso
            l_k \in r
        }{
            \Gamma \vdash t_1.l_k : A
        }
    \end{minipage}
\end{center}
By induction hypothesis, there exists a term $t_1@l$ such that:
\begin{align*}
\Gamma \vdash t_1@l : \vertype{r}{A}
\end{align*}
We then reapply (\textsc{extr}) to obtain the following:
\begin{center}
    \begin{minipage}{.55\linewidth}
        \infrule[extr]{
            \Gamma \vdash t_1@l : \vertype{r}{A}
            \andalso
            l_k \in r
        }{
            \Gamma \vdash (t_1@l).l_k : A
        }
    \end{minipage}
\end{center}
Thus, note that $(t_1.l_k)@l \equiv (t_1@l).l_k$, we obtain the conclusion of the lemma.
\\

\item Case (\textsc{sub})
\begin{center}
    \begin{minipage}{.55\linewidth}
        \infrule[\textsc{sub}]{
            \Gamma_1, x:\verctype{B}{r}, \Gamma_2 \vdash t : A
            \andalso
            r \sqsubseteq s
            \andalso
            \vdash s
        }{
            \Gamma_1, x:\verctype{B}{s}, \Gamma_2 \vdash t : A
        }
    \end{minipage}
\end{center}
By induction hypothesis, there exists a term $t@l$ such that:
\begin{align*}
\Gamma_1, x:\verctype{B}{r}, \Gamma_2 \vdash t@l : A
\end{align*}
We then reapply (\textsc{sub}) to obtain the following:
\begin{center}
    \begin{minipage}{.55\linewidth}
        \infrule[\textsc{sub}]{
            \Gamma_1, x:\verctype{B}{r}, \Gamma_2 \vdash t@l : A
            \andalso
            r \sqsubseteq s
            \andalso
            \vdash s
        }{
            \Gamma_1, x:\verctype{B}{s}, \Gamma_2 \vdash t@l : A
        }
    \end{minipage}
\end{center}
Thus, we obtain the conclusion of the lemma.

\end{itemize}
\end{proof}








\begin{lemma}[Type-safe extraction for versioned values]
\label{lemma:extraction}
\begin{align*}
    [\Gamma] \vdash u : \vertype{r}{A}
    \hspace{1em}\Longrightarrow\hspace{1em}
    \forall l_k\in r.\ \exists t'.
    \left\{
    \begin{aligned}
        &u.l_k   \longrightarrow     t' & & \ (progress)\\
        &[\Gamma] \vdash t' : A &  & \ (preservation)
    \end{aligned}
    \right.
\end{align*}
\end{lemma}

\begin{proof}
By inversion lemma (\ref{lemma:typedvalue}), $u$ has either a form $\pr{t''}$ or $\nvval{\overline{l_i=t_i}}$.
% weak, der, subの有限回の適用の後に必ずprがある。(|Γ|が有限とすると、リソースが0 or 1になってる変数も有限個。subもリソース集合が有界なので、真に小さなリソースに対してのみ適用すれば必ず有限回の適用ですべてのリソースが0になる。)
% weak, der, subの有限回の適用の後に必ずverがある。

\begin{itemize}
\item Suppose $u=\pr{t''}$.\\
We can apply (\textsc{E-ex1}) as follows:
\begin{center}
    \begin{minipage}{.40\linewidth}
        \infrule[E-ex1]{
            \\
        }{
            \pr{t''}.l_k \leadsto t''@l_k
        }
    \end{minipage}
\end{center}
Also, we get the following derivation for $v$.
\begin{center}
\begin{prooftree}
\AxiomC{$ [\Gamma'] \vdash t'' : A$}
\AxiomC{$ \vdash r$}
\RightLabel{(\textsc{pr})}
\BinaryInfC{$ r\cdot [\Gamma'] \vdash \pr{t''} : \vertype{r}{A}$}
\RightLabel{(\textsc{weak}) or (\textsc{sub})}
\UnaryInfC{$\vdots$}
\RightLabel{(\textsc{weak}) or (\textsc{sub})}
\UnaryInfC{$ [\Gamma] \vdash \pr{t''} : \vertype{r}{A}$}
\end{prooftree}
\end{center}
By Lemma \ref{lemma:overwriting}, we know the following:
\begin{align*}
[\Gamma'] \vdash t''@l_k : A
\end{align*}
Finally, we can rearrange the typing context as follows:
\begin{center}
\begin{prooftree}
\AxiomC{$ [\Gamma'] \vdash t''@l_k : A$}
\RightLabel{(\textsc{weak}) or (\textsc{sub})}
\UnaryInfC{$\vdots$}
\RightLabel{(\textsc{weak}) or (\textsc{sub})}
\UnaryInfC{$ [\Gamma] \vdash t''@l_k : A$}
\end{prooftree}
\end{center}
Here, we follow the same manner as for the derivation of $\pr{t''}$ (which may use (\textsc{weak}) and (\textsc{sub})) to get $[\Gamma]$ from $r\cdot [\Gamma']$.

Thus, we obtain the conclusion of the lemma.

\item Suppose $u=\nvval{\overline{l_i=t_i}}$.\\
We can apply (\textsc{E-ex2}) as follows:
\begin{center}
    \begin{minipage}{.50\linewidth}
        \infrule[E-ex2]{
            \\
        }{
            \nvval{\overline{l_i=t_i}}.l_k \leadsto t_k@l_k
        }
    \end{minipage}
\end{center}
Also, we get the following derivation for $v$.
\begin{center}
\begin{prooftree}
\AxiomC{$ [\Gamma'_i] \vdash t_i : A$}
\AxiomC{$\vdash \{\overline{l_i}\}$}
\RightLabel{(\textsc{ver})}
\BinaryInfC{$ \bigcup_i(\{l_i\}\cdot[\Gamma'_i]) \vdash \nvval{\overline{l_i=t_i}} : \vertype{\{\overline{l_i}\}}{A}$}
\RightLabel{(\textsc{weak}) or (\textsc{sub})}
\UnaryInfC{$\vdots$}
\RightLabel{(\textsc{weak}) or (\textsc{sub})}
\UnaryInfC{$ [\Gamma] \vdash \nvval{\overline{l_i=t_i}} : \vertype{\{\overline{l_i}\}}{A}$}
\end{prooftree}
\end{center}
By Lemma \ref{lemma:overwriting}, we know the following:
\begin{align*}
[\Gamma'_k] \vdash t_k@l_k : A
\end{align*}
Finally, we can rearrange the typing context as follows:
\begin{center}
\begin{prooftree}
\AxiomC{$ [\Gamma'_k] \vdash t_k@l_k : A$}
\AxiomC{$ r_{kj} \sqsubseteq r_{kj}\otimes \{l_k\}$}
\RightLabel{(\textsc{sub}) $*\,|\Gamma_k'|$}
\BinaryInfC{$ \underbrace{\{l_k\}\cdot[\Gamma'_k] \vdash t_k@l_k : A}_{P}$}
\end{prooftree}
\end{center}

\begin{center}
\begin{prooftree}
\AxiomC{$P$}
\AxiomC{$ r_{kj}\otimes \{l_k\} \sqsubseteq \textstyle{\sum_{i}}(r_{ij}\otimes \{l_i\})$}
\RightLabel{(\textsc{sub}) $*\,|\Gamma_k'|$}
\BinaryInfC{$ \textstyle{\bigcup_i}(\{l_i\}\cdot[\Gamma'_i]) \vdash t_k@l_k : A$}
\RightLabel{(\textsc{weak}) or (\textsc{sub})}
\UnaryInfC{$\vdots$}
\RightLabel{(\textsc{weak}) or (\textsc{sub})}
\UnaryInfC{$ [\Gamma] \vdash t_k@l_k : A$}
\end{prooftree}
\end{center}
Here in the multiple application of (\textsc{sub}), the second premise compares the resources of j-th versioned assumption between the first premise and conclusion.
Also, we follow the same manner as for the derivation of $\nvval{\overline{l_i=t_i}}$ (which may use (\textsc{weak}) and (\textsc{sub})) to get $[\Gamma]$ from $\textstyle{\bigcup_i}(\{l_i\}\cdot[\Gamma'_i])$.

Thus, we obtain the conclusion of the lemma.

\end{itemize}
\end{proof}


















%%%%%%%%%%%%%%%%%%%%%%%%%%%%%%%%%%%%%%%%%%%%
%%%%%%%%%%%%%%%%%%%%%%%%%%%%%%%%%%%%%%%%%%%%
%%%%%%%%%%%%%%%%%%%%%%%%%%%%%%%%%%%%%%%%%%%%
%%%%%%%%%%%%%%%%%%%%%%%%%%%%%%%%%%%%%%%%%%%%
%%%%%%%%%%%%%%%%%%%%%%%%%%%%%%%%%%%%%%%%%%%%
%%%%%%%%%%%%%%%%%%%%%%%%%%%%%%%%%%%%%%%%%%%%


\begin{theorem}[Type preservation for reductions]
\label{lemma:preservationreduction}
\begin{align*}
    \left.
    \begin{aligned}
        &\Gamma \vdash t : A\\
        &t \leadsto t'
    \end{aligned}
    \right\}
    \hspace{1em}\Longrightarrow\hspace{1em}
    \Gamma \vdash t' : A
\end{align*}
\end{theorem}

\begin{proof}
The proof is given by induction on the typing derivation of $t$.
Consider the cases for the last rule used in the typing derivation of the first assumption.
\\

\begin{itemize}
\item Case (\textsc{int})
\begin{center}
    \begin{minipage}{.25\linewidth}
        \infrule[int]{
            \\
        }{
            \emptyset \vdash n:\textsf{Int}
        }
    \end{minipage}
\end{center}
This case holds trivially because there are no reduction rules that can be applied to $n$.
\\

\item Case (\textsc{var})
\begin{center}
    \begin{minipage}{.35\linewidth}
        \infrule[var]{
            \vdash A
        }{
            x:A \vdash x:A
        }
    \end{minipage}
\end{center}
This case holds trivially because there are no reduction rules that can be applied to $x$.
\\

\item Case (\textsc{abs})
\begin{center}
    \begin{minipage}{.65\linewidth}
        \infrule[abs]{
            - \vdash p : B_1 \rhd \Delta'
            \andalso
            \Gamma, \Delta' \,\vdash\, t_1 : A_2%\theta B
        }{
            \Gamma \,\vdash\, \lam{p}{t_1} : \ftype{A_1}{A_2}
        }
    \end{minipage}
\end{center}
This case holds trivially because there are no reduction rules that can be applied to $\lam{p}{t_1}$.
\\

\item Case (\textsc{app})
\begin{center}
    \begin{minipage}{.65\linewidth}
        \infrule[app]{
            \Gamma_1 \vdash t_1 : \ftype{B}{A}
            \andalso
            \Gamma_2 \vdash t_2 : B
        }{
            \Gamma_1 + \Gamma_2 \vdash \app{t_1}{t_2} : A
        }
    \end{minipage}
\end{center}

We perform case analysis for the ruduction rule applied last.
\begin{itemize}
\item Case (\textsc{E-abs1})
\begin{center}
        \begin{minipage}{.5\linewidth}
            \infrule[E-abs1]{
                \\
            }{
                \underbrace{\app{(\lam{x}{t'_1})}{t_2}}_{t} \leadsto \app{(t_2 \rhd x)}{t'_1}
            }
        \end{minipage}
\end{center}
where $t_1=\lam{x}{t'_1}$ for a term $t_1'$.
Then we can apply ($\rhd_{\textrm{var}}$) to obtain the following:
\begin{center}
    \begin{minipage}{.45\linewidth}
        \infrule[$\rhd_{\textrm{var}}$]{
            \\
        }{
            \app{(t_2 \rhd x)}{t'_1} = [t_2 / x]t'_1
        }
    \end{minipage}
\end{center}
In this case, we know the typing derivation of $t$ has the following form:
\begin{center}
\begin{prooftree}
\AxiomC{$ \Gamma'_1, x:B \vdash t'_1 : A$}
\RightLabel{(\textsc{abs})}
\UnaryInfC{$ \Gamma'_1 \vdash \lam{x}{t'_1} : \ftype{B}{A} $}
\RightLabel{(\textsc{weak}), (\textsc{der}), or (\textsc{sub})}
\UnaryInfC{$ \vdots $}
\RightLabel{(\textsc{weak}), (\textsc{der}), or (\textsc{sub})}
\UnaryInfC{$\Gamma_1 \vdash \lam{x}{t'_1} : \ftype{B}{A}$}
\AxiomC{$ \Gamma_2 \vdash t_2 : B$}
\RightLabel{(\textsc{app})}
\BinaryInfC{$ \Gamma_1 + \Gamma_2 \vdash \app{(\lam{x}{t'_1})}{t_2} : A$}
\end{prooftree}
\end{center}
By Lemma \ref{lemma:substitution1}, we know the following:
\begin{align*}
    \left.
    \begin{aligned}
          \Gamma_2 & \vdash t_2:B \\
          \Gamma'_1, x:B & \vdash t'_1:A
    \end{aligned}
    \right\}
    \hspace{1em}\Longrightarrow\hspace{1em}
    \Gamma'_1 + \Gamma_2 \vdash [t_2/x]t'_1:A
\end{align*}
Finally, we can rearrange the typing context as follows:
\begin{center}
\begin{prooftree}
\AxiomC{$ \Gamma'_1 + \Gamma_2 \vdash [t_2/x]t'_1:A $}
\RightLabel{(\textsc{weak}), (\textsc{der}), or (\textsc{sub})}
\UnaryInfC{$ \vdots $}
\RightLabel{(\textsc{weak}), (\textsc{der}), or (\textsc{sub})}
\UnaryInfC{$\Gamma_1 + \Gamma_2 \vdash [t_2/x]t'_1:A$}
\end{prooftree}
\end{center}
Here, there exists a derive tree to get $\Gamma_1+\Gamma_2$ from $\Gamma'_1+\Gamma_2$ as for the derivation of $\lam{x}{t'_1}$ which may use (\textsc{weak}), (\textsc{der}) and (\textsc{sub}).

By choosing $t'=[t_2/x]t'_1$, we obtain the conclusion of the theorem.\\


\item Case (\textsc{E-abs2})
\begin{center}
        \begin{minipage}{.75\linewidth}
            \infrule[E-abs2]{
                \\
            }{
                \underbrace{\app{(\lam{\pr{x}}{t'_1})}{t_2}}_{t} \leadsto \underbrace{\clet{x}{t_2}{t'_1}}_{t'}
            }
        \end{minipage}
\end{center}
In this case, we know the typing derivation of $t$ has the following form:
\begin{center}
\begin{prooftree}
\AxiomC{$ $}
\RightLabel{(\mbox{[}\textsc{pVar}\mbox{]})}
\UnaryInfC{$ r \vdash x : B \rhd x:\verctype{B}{r}$}
\RightLabel{(\textsc{p}$_\square$)}
\UnaryInfC{$ - \vdash \pr{x} : \vertype{r}{B} \rhd x:\verctype{B}{r}$}
\AxiomC{$ \Gamma_1, x:\verctype{B}{r} \vdash t_1' : A$}
\RightLabel{(\textsc{abs})}
\BinaryInfC{$ \underbrace{ \Gamma_1 \vdash  \lam{\pr{x}}{t'_1} : \ftype{\vertype{r}{B}}{A} }_{P}$}
\end{prooftree}
\begin{prooftree}
\AxiomC{$P$}
\AxiomC{$ \Gamma_2 \vdash t_2 : \vertype{r}{B}$}
\RightLabel{(\textsc{app})}
\BinaryInfC{$ \Gamma_1+\Gamma_2 \vdash \app{(\lam{\pr{x}}{t'_1})}{t_2} : A$}
\end{prooftree}
\end{center}
Therefore, we can construct the derivation tree for $t'$ as follows.
\begin{center}
\begin{prooftree}
\AxiomC{$ \Gamma_2 \vdash t_2 : \vertype{r}{B}$}
\AxiomC{$ \Gamma_1, x:\verctype{B}{r} \vdash t_1' : A$}
\RightLabel{(\textsc{app})}
\BinaryInfC{$ \Gamma_1+\Gamma_2 \vdash \clet{x}{t_2}{t'_1} : A$}
\end{prooftree}
\end{center}
Hence, we have the conclusion of the theorem.\\
\end{itemize}


\item Case (\textsc{let})
\begin{center}
    \begin{minipage}{.70\linewidth}
        \infrule[let]{
            \Gamma_1 \,\vdash\, t_1 : \vertype{r}{A}
            \andalso
            \Gamma_2, x:\verctype{A}{r} \,\vdash\, t_2 : B
        }{
            \Gamma_1 + \Gamma_2 \,\vdash\, \clet{x}{t_1}{t_2} : B
        }
    \end{minipage}
\end{center}
The only reduction rule we can apply is (\textsc{E-clet}) with two substitution rules, depending on whether $t_1$ has the form $[t'_1]$ or $\nvval{\overline{l_i=t''_i}}$.
\begin{itemize}
\item Suppose $t_1=[t_1']$.\\
We can apply (\textsc{E-clet}) to obtain the following.
\begin{center}
    \begin{minipage}{.65\linewidth}
        \infrule[E-clet]{
            \\
        }{
            \underbrace{\clet{x}{[t_1']}{t_2}}_{t} \leadsto ([t_1'] \rhd \pr{x})t_2
        }
    \end{minipage}
\end{center}
Thus, we can apply (\textsc{$\rhd_\square$}) and (\textsc{$\rhd_{\textnormal{var}}$}) to obtain the following.
\begin{center}
\begin{prooftree}
\AxiomC{$ $}
\RightLabel{($\rhd_{\textnormal{var}}$)}
\UnaryInfC{$ (t_1' \rhd x)t_2 = [t_1' / x] t_2$}
\RightLabel{($\rhd_{\square}$)}
\UnaryInfC{$ ([t_1'] \rhd \pr{x})t_2 = [t_1' / x] t_2$}
\end{prooftree}
\end{center}
In this case, we know the typing derivation of $t$ has the following form:
\begin{center}
\begin{prooftree}
\AxiomC{$ [\Gamma_1'] \vdash t'_1 : A$}
\AxiomC{$ \vdash r$}
\RightLabel{(\textsc{pr})}
\BinaryInfC{$ r\cdot [\Gamma_1'] \,\vdash\, [t_1'] : \vertype{r}{A} $}
\RightLabel{(\textsc{weak}) or (\textsc{sub})}
\UnaryInfC{$ \vdots $}
\RightLabel{(\textsc{weak}) or (\textsc{sub})}
\UnaryInfC{$ \Gamma_1 \,\vdash\, [t_1'] : \vertype{r}{A}$}
\AxiomC{$ \Gamma_2, x:\verctype{A}{r} \,\vdash\, t_2 : B $}
\RightLabel{(\textsc{let})}
\BinaryInfC{$ \Gamma_1+\Gamma_2 \,\vdash\, \clet{x}{[t_1']}{t_2} : B $}
\end{prooftree}
\end{center}
By Lemma \ref{lemma:substitution2}, we know the following:
\begin{align*}
    \left.
    \begin{aligned}\relax
          [\Gamma_1'] &\,\vdash\, t'_1 : A\\
          \Gamma_2, x:\verctype{A}{r} &\,\vdash\, t_2 : B
    \end{aligned}
    \right\}
    \hspace{1em}\Longrightarrow\hspace{1em}
    \Gamma_2 + r\cdot[\Gamma_1'] \,\vdash\, [t'_1/x]t_2 : B
\end{align*}
Finally, we can rearrange the typing context as follows:
\begin{center}
\begin{prooftree}
\AxiomC{$ \Gamma_2 + r\cdot[\Gamma_1'] \,\vdash\, [t'_1/x]t_2 : B $}
\RightLabel{(\textsc{weak}) or (\textsc{sub})}
\UnaryInfC{$ \vdots $}
\RightLabel{(\textsc{weak}) or (\textsc{sub})}
\UnaryInfC{$ \Gamma_2+\Gamma_1 \,\vdash\, [t'_1/x]t_2 : B$}
\end{prooftree}
\end{center}
Here, there exists a derive tree to get $\Gamma_2+\Gamma_1$ from $\Gamma_2+r\cdot [\Gamma_1']$ as for the derivation of $\pr{t'_1}$ which may use (\textsc{weak}) and (\textsc{sub}).

Thus, by choosing $t' = [t'_1/x]t_2$, we obtain the conclusion of the theorem.

\item Suppose $t_1 = \nvval{\overline{l_i=t''_i}}$.\\
We can apply (\textsc{E-clet}) to obtain the following:
\begin{center}
        \begin{minipage}{.95\linewidth}
            \infrule[E-clet]{
                \\
            }{
                \underbrace{\clet{x}{\nvval{\overline{l_i=t''_i}}}{t_2}}_{t} \leadsto (\nvval{\overline{l_i=t''_i}} \rhd \pr{x})t_2
            }
        \end{minipage}
\end{center}
Thus, we can apply (\textsc{$\rhd_\textnormal{ver}$}) and (\textsc{$\rhd_{\textnormal{var}}$}) to obtain the following.
\begin{center}
\begin{prooftree}
\AxiomC{$ $}
\RightLabel{($\rhd_{\textnormal{var}}$)}
\UnaryInfC{$ (\ivval{\overline{l_i=t''_i}}{l_k} \rhd x)t_2 = [\ivval{\overline{l_i=t''_i}}{l_k} / x] t_2$}
\RightLabel{($\rhd_\textnormal{ver}$)}
\UnaryInfC{$ (\nvval{\overline{l_i=t''_i}} \rhd \pr{x})t_2 = [\ivval{\overline{l_i=t''_i}}{l_k} / x] t_2$}
\end{prooftree}
\end{center}
In this case, we know the typing derivation of $t$ has the following form:
\begin{center}
\begin{prooftree}
\AxiomC{$ [\Gamma_i'] \vdash t''_i : A$}
\AxiomC{$ \vdash \{\overline{l_i}\}$}
\RightLabel{(\textsc{ver})}
\BinaryInfC{$ \bigcup_i(\{l_i\}\cdot [\Gamma'_i]) \,\vdash\, \nvval{\overline{l_i=t''_i}} : \vertype{\{\overline{l_i}\}}{A}$}
\RightLabel{(\textsc{weak}) or (\textsc{sub})}
\UnaryInfC{$ \vdots $}
\RightLabel{(\textsc{weak}) or (\textsc{sub})}
\UnaryInfC{$ \underbrace{\Gamma_1 \,\vdash\, \nvval{\overline{l_i=t''_i}} : \vertype{\{\overline{l_i}\}}{A}}_{P}$}
\end{prooftree}
\end{center}
\begin{center}
\begin{prooftree}
\AxiomC{$ P$}
\AxiomC{$ \Gamma_2, x:A \,\vdash\, t_2 : B $}
\RightLabel{(\textsc{der})}
\UnaryInfC{$ \Gamma_2, x:\verctype{A}{1} \,\vdash\, t_2 : B $}
\RightLabel{(\textsc{sub})$*|\{\overline{l_i}\}|$}
\UnaryInfC{$ \Gamma_2, x:\verctype{A}{\{\overline{l_i}\}} \,\vdash\, t_2 : B $}
\RightLabel{(\textsc{let})}
\BinaryInfC{$ \Gamma_1+\Gamma_2 \,\vdash\, \clet{x}{\nvval{\overline{l_i=t''_i}}}{t_2} : B $}
\end{prooftree}
\end{center}
Then we can derive the type of $\ivval{\overline{l_i=t''_i}}{l_k}$ as follows:
\begin{center}
    \begin{minipage}{.6\linewidth}
        \infrule[veri]{
            [\Gamma'_i] \,\vdash\, t''_i : A
        }{
            \bigcup_i(\{l_i\}\cdot [\Gamma'_i]) \,\vdash\, \ivval{\overline{l_i=t''_i}}{l_k} : A
        }
    \end{minipage}
\end{center}
By Lemma \ref{lemma:substitution1}, we know the following:
\begin{align*}
    \left.
    \begin{aligned}
          \textstyle{\bigcup_i(\{l_i\}\cdot [\Gamma'_i])} &\vdash \ivval{\overline{l_i=t''_i}}{l_k} : A\\
          \Gamma_2, x:A &\vdash t_2 : B
    \end{aligned}
    \right\}
    \hspace{.1em}\Longrightarrow\hspace{.1em}
    \left.
    \begin{aligned}
        \Gamma_2 + &\textstyle{\bigcup_i(\{l_i\}\cdot [\Gamma'_i])}\\
        &\,\vdash\, [\ivval{\overline{l_i=t''_i}}{l_k}/x] t_2 : B
    \end{aligned}
    \right.
\end{align*}
Finally, we can rearrange the typing context as follows:
\begin{center}
\begin{prooftree}
\AxiomC{$ \Gamma_2 + \textstyle{\bigcup_i(\{l_i\}\cdot [\Gamma'_i])} \,\vdash\, [\ivval{\overline{l_i=t''_i}}{l_k}/x] t_2 : B $}
\RightLabel{(\textsc{weak}) or (\textsc{sub})}
\UnaryInfC{$ \vdots $}
\RightLabel{(\textsc{weak}) or (\textsc{sub})}
\UnaryInfC{$ \Gamma_2 + \Gamma_1 \,\vdash\, [\ivval{\overline{l_i=t''_i}}{l_k}/x] t_2 : B $}
\end{prooftree}
\end{center}
Here, there exists a derive tree to get $\Gamma_2+\Gamma_1$ from $\Gamma_2 + \textstyle{\bigcup_i(\{l_i\}\cdot [\Gamma'_i])}$ as for the derivation of $\nvval{\overline{l_i=t''_i}}$ which may use (\textsc{weak}) and (\textsc{sub}).

Thus, by choosing $t' = [\ivval{\overline{l_i=t''_i}}{l_k}/x]t_2$, we obtain the conclusion of the theorem.\\
\end{itemize}


\item Case (\textsc{weak})
\begin{center}
    \begin{minipage}{.45\linewidth}
        \infrule[weak]{
            \Gamma_1 \vdash t : A
            \andalso
            \vdash \Delta'
        }{
            \Gamma_1 + \verctype{\Delta'}{0} \vdash t : A
        }
    \end{minipage}
\end{center}
In this case, $t$ does not change between before and after the last derivation.
The induction hypothesis implies that there exists a term $t''$ such that:
\begin{align*}
        t\leadsto t''
        \ \land\ \Gamma_1 \vdash t'' : A \tag{ih}
\end{align*}
We then reapply (\textsc{weak}) to obtain the following:
\begin{center}
    \begin{minipage}{.38\linewidth}
        \infrule[weak]{
            \Gamma_1 \vdash t'' : A
            \andalso
            \vdash \Delta'
        }{
            \Gamma_1 + \verctype{\Delta'}{0} \vdash t'' : A
        }
    \end{minipage}
\end{center}
Thus, by choosing $t'=t''$, we obtain the conclusion of the theorem.
\\

\item Case (\textsc{der})
\begin{center}
    \begin{minipage}{.45\linewidth}
        \infrule[der]{
            \Gamma_1, x:B \vdash t : A
        }{
            \Gamma_1, x:\verctype{B}{1} \vdash t : A
        }
    \end{minipage}
\end{center}
In this case, $t$ does not change between before and after the last derivation.
The induction hypothesis implies that there exists a term $t''$ such that:
\begin{align*}
        t\leadsto t''
        \ \land\ \Gamma_1, x:B \vdash t'' : A \tag{ih}
\end{align*}
We then reapply (\textsc{der}) to obtain the following:
\begin{center}
    \begin{minipage}{.38\linewidth}
        \infrule[der]{
            \Gamma_1, x:B \vdash t'' : A
        }{
            \Gamma_1, x:\verctype{B}{1} \vdash t'' : A
        }
    \end{minipage}
\end{center}
Thus, by choosing $t'=t''$, we obtain the conclusion of the theorem.
\\

\item Case (\textsc{pr})
\begin{center}
    \begin{minipage}{.35\linewidth}
        \infrule[pr]{
            \verctype{\Gamma}{} \vdash t'' : B
            \andalso
            \vdash r
        }{
            r\cdot\verctype{\Gamma}{} \vdash \pr{t''} : \vertype{r}{B} 
        }
    \end{minipage}
\end{center}
This case holds trivially because there are no reduction rules that can be applied to $\pr{t''}$.
\\

\item Case (\textsc{ver})
\begin{center}
    \begin{minipage}{.65\linewidth}
        \infrule[ver]{
            \verctype{\Gamma_i}{} \vdash t_i : A
            \andalso
            \vdash \{\overline{l_i}\}
        }{
            \bigcup_i(\{l_i\}\cdot [\Gamma_i]) \vdash \nvval{\overline{l_i=t_i}} : \vertype{\{\overline{l_i}\}}{A}
        }
    \end{minipage}
\end{center}
This case holds trivially because there are no reduction rules that can be applied to $\nvval{\overline{l_i=t_i}}$.
\\

\item Case (\textsc{veri})
\begin{center}
    \begin{minipage}{.55\linewidth}
        \infrule[veri]{
            \verctype{\Gamma_i}{} \vdash t_i : A
            \andalso
            \vdash \{\overline{l_i}\}
            \andalso
            l_k \in \{\overline{l_i}\}
        }{
            \bigcup_i(\{l_i\}\cdot [\Gamma_i]) \vdash \ivval{\overline{l_i=t_i}}{l_k} : A
        }
    \end{minipage}
\end{center}
In this case, the only reduction rule we can apply is (\textsc{E-veri}).
\begin{center}
        \begin{minipage}{.50\linewidth}
            \infrule[E-veri]{
                \\
            }{
                \underbrace{\ivval{\overline{l_i=t_i}}{l_k}}_{t} \leadsto t_k@l_k
            }
        \end{minipage}
\end{center}
By Lemma \ref{lemma:overwriting}, we obtain the following:
\begin{align*}
    [\Gamma_k] \vdash t_k : A
    \hspace{1em}\Longrightarrow\hspace{1em}
    [\Gamma_k] \vdash t_k@l_k : A
\end{align*}
Finally, we can rearrange the typing context as follows:
\begin{center}
\begin{prooftree}
\AxiomC{$ [\Gamma_k] \vdash t_k@l_k : A $}
\RightLabel{(\textsc{weak}), (\textsc{der}) or (\textsc{sub})}
\UnaryInfC{$ \vdots $}
\RightLabel{(\textsc{weak}), (\textsc{der}) or (\textsc{sub})}
\UnaryInfC{$ \bigcup_i(\{l_i\}\cdot [\Gamma_i]) \vdash t_k@l_k : A $}
\end{prooftree}
\end{center}
Thus, by choosing $t' = t_k@l_k$, we obtain the conclusion of the theorem.
\\

\item Case (\textsc{extr})
\begin{center}
    \begin{minipage}{.50\linewidth}
        \infrule[extr]{
            \Gamma \vdash t_1 : \vertype{r}{A}
            \andalso
            l_k \in r
        }{
            \Gamma \vdash t_1.l_k : A
        }
    \end{minipage}
\end{center}
In this case, there are two reduction rules that we can apply to $t$, dependenig on whether $t_1$ has the form $[t'_1]$ or $\nvval{\overline{l_i=t''_i}}$.
\begin{itemize}
\item Suppose $t_1 = \pr{t'_1}$.\\
We know the typing derivation of $t$ has the following form:
\begin{prooftree}
    \AxiomC{$ [\Gamma'] \vdash t'_1 : A$}
    \AxiomC{$ \vdash r$}
    \RightLabel{(\textsc{pr})}
    \BinaryInfC{$ r\cdot[\Gamma'] \vdash \pr{t'_1} : \vertype{r}{A}$}
    \RightLabel{(\textsc{weak}) or (\textsc{sub})}
    \UnaryInfC{$ \vdots $}
    \RightLabel{(\textsc{weak}) or (\textsc{sub})}
    \UnaryInfC{$ \Gamma \vdash \pr{t'_1} : \vertype{r}{A}$}
    \AxiomC{$l_k \in r$}
    \RightLabel{(\textsc{extr})}
    \BinaryInfC{$ \Gamma \vdash \pr{t'_1}.l_k : A$}
\end{prooftree}
By Lemma \ref{lemma:extraction}, we know the following:
\begin{align*}
    r\cdot[\Gamma'] \vdash \pr{t'_1} : \vertype{r}{A}
    \hspace{1em}\Longrightarrow\hspace{1em}
    \exists t'.
    \left\{
    \begin{aligned}
        &[t'_1].l_k   \longrightarrow      t' \\
        &r\cdot[\Gamma'] \vdash t' : A
    \end{aligned}
    \right.
\end{align*}
Finally, we can rearrange the typing context as follows:
\begin{center}
\begin{prooftree}
\AxiomC{$ r\cdot[\Gamma'] \vdash t' : A$}
\RightLabel{(\textsc{weak}) or (\textsc{sub})}
\UnaryInfC{$ \vdots $}
\RightLabel{(\textsc{weak}) or (\textsc{sub})}
\UnaryInfC{$ \Gamma \vdash t' : A$}
\end{prooftree}
\end{center}
Here, we follow the same manner as for the derivation of $\pr{t'_1}$ (which may use (\textsc{weak}) and (\textsc{sub})) to get $\Gamma$ from $r\cdot[\Gamma']$.

Thus, we obtain the conclusion of the theorem.

\item Suppose $t_1 = \nvval{\overline{l_i=t_i}}$.\\
The last derivation is rewritten as follows:
\begin{prooftree}
    \AxiomC{$[\Gamma'_i] \vdash t_i : A$}
    \AxiomC{$\vdash \{\overline{l_i}\}$}
    \RightLabel{(\textsc{ver})}
    \BinaryInfC{$ \bigcup_i\{l_i\}\cdot [\Gamma'_i] \vdash \nvval{\overline{l_i=t_i}} : \vertype{\{\overline{l_i}\}}{A}$}
    \RightLabel{(\textsc{weak}) or (\textsc{sub})}
    \UnaryInfC{$ \vdots $}
    \RightLabel{(\textsc{weak}) or (\textsc{sub})}
    \UnaryInfC{$ \Gamma \vdash \nvval{\overline{l_i=t_i}} : \vertype{\{\overline{l_i}\}}{A}$}
    \AxiomC{$l_k \in \{\overline{l_i}\}$}
    \RightLabel{(\textsc{extr})}
    \BinaryInfC{$ \Gamma \vdash \nvval{\overline{l_i=t_i}}.l_k : A$}
\end{prooftree}
By Lemma \ref{lemma:extraction}, we know the following:
\begin{align*}
    \textstyle{\bigcup_i}\{l_i\}\cdot [\Gamma'_i] \vdash \nvval{\overline{l_i=t_i}} : \vertype{\{\overline{l_i}\}}{A}
    \hspace{1em}\Longrightarrow\hspace{1em}
    \exists t'.
    \left\{
    \begin{aligned}
        &\nvval{\overline{l_i=t_i}}.l_k   \longrightarrow  t' \\
        &\textstyle{\bigcup_i}\{l_i\}\cdot [\Gamma'_i] \vdash t' : A
    \end{aligned}
    \right.
\end{align*}
Finally, we can rearrange the typing context as follows:
\begin{center}
\begin{prooftree}
\AxiomC{$ \textstyle{\bigcup_i}\{l_i\}\cdot [\Gamma'_i] \vdash t' : A$}
\RightLabel{(\textsc{weak}) or (\textsc{sub})}
\UnaryInfC{$ \vdots $}
\RightLabel{(\textsc{weak}) or (\textsc{sub})}
\UnaryInfC{$ \Gamma \vdash t' : A$}
\end{prooftree}
\end{center}
Here, we follow the same manner as for the derivation of $\nvval{\overline{l_i=t_i}}$ (which may use (\textsc{weak}) and (\textsc{sub})) to get $\Gamma$ from $\bigcup_i\{l_i\}\cdot [\Gamma'_i]$.

Thus, we obtain the conclusion of the theorem.\\
\end{itemize}

\item Case (\textsc{sub})
\begin{center}
    \begin{minipage}{.65\linewidth}
        \infrule[\textsc{sub}]{
            \Gamma_1, x:\verctype{B}{r}, \Gamma_2 \vdash t : A
            \andalso
            r \sqsubseteq s
            \andalso
            \vdash s
        }{
            \Gamma_1, x:\verctype{B}{s}, \Gamma_2 \vdash t : A
        }
    \end{minipage}
\end{center}
In this case, $t$ does not change between before and after the last derivation.
The induction hypothesis implies that there exists a term $t''$ such that:
\begin{align*}
    t \leadsto t''
    \ \land\ 
    \Gamma_1, x:\verctype{B}{r}, \Gamma_2 \vdash t'' : A \tag{ih}
\end{align*}
We then reapply (\textsc{sub}) to obtain the following:
\begin{prooftree}
    \AxiomC{$\Gamma_1, x:\verctype{B}{r}, \Gamma_2 \vdash t'' : A$}
    \AxiomC{$ r \sqsubseteq s$}
    \AxiomC{$ \vdash s$}
    \RightLabel{(\textsc{sub})}
    \TrinaryInfC{$ \Gamma_1, x:\verctype{B}{s}, \Gamma_2 \vdash t'' : A$}
\end{prooftree}
Thus, by choosing $t' = t''$, we obtain the conclusion of the theorem.

\end{itemize}
\end{proof}


















%%%%%%%%%%%%%%%%%%%%%%%%%%%%%%%%%%%%%%%%%%%%
%%%%%%%%%%%%%%%%%%%%%%%%%%%%%%%%%%%%%%%%%%%%
%%%%%%%%%%%%%%%%%%%%%%%%%%%%%%%%%%%%%%%%%%%%
%%%%%%%%%%%%%%%%%%%%%%%%%%%%%%%%%%%%%%%%%%%%
%%%%%%%%%%%%%%%%%%%%%%%%%%%%%%%%%%%%%%%%%%%%
%%%%%%%%%%%%%%%%%%%%%%%%%%%%%%%%%%%%%%%%%%%%

\begin{theorem}[Type preservation for evaluations]
\label{lemma:preservationevaluation}
\begin{align*}
    \left.
    \begin{aligned}
        &\Gamma \vdash t : A\\
        &t \longrightarrow t'
    \end{aligned}
    \right\}
    \hspace{1em}\Longrightarrow\hspace{1em}
    \Gamma \vdash t' : A
\end{align*}
\end{theorem}

\begin{proof}
The proof is given by induction on the typing derivation of $t$.
Consider the cases for the last rule used in the typing derivation of the first assumption.

\begin{itemize}
\item Case (\textsc{int})
\begin{center}
    \begin{minipage}{.25\linewidth}
        \infrule[int]{
            \\
        }{
            \emptyset \vdash n:\textsf{Int}
        }
    \end{minipage}
\end{center}
This case holds trivially because there are no evaluation rules that can be applied to $n$.
\\

\item Case (\textsc{var})
\begin{center}
    \begin{minipage}{.35\linewidth}
        \infrule[var]{
            \vdash A
        }{
            x:A \vdash x:A
        }
    \end{minipage}
\end{center}
This case holds trivially because there are no evaluation rules that can be applied to $x$.
\\

\item Case (\textsc{abs})
\begin{center}
    \begin{minipage}{.6\linewidth}
        \infrule[abs]{
            - \vdash p : B_1 \rhd \Delta'
            \andalso
            \Gamma, \Delta' \,\vdash\, t_1 : A_2%\theta B
        }{
            \Gamma \,\vdash\, \lam{p}{t_1} : \ftype{A_1}{A_2}
        }
    \end{minipage}
\end{center}
This case holds trivially because there are no evaluation rules that can be applied to $\lam{p}{t_1}$.
\\

\item Case (\textsc{app})
\begin{center}
    \begin{minipage}{.65\linewidth}
        \infrule[app]{
            \Gamma_1 \vdash t_1 : \ftype{B}{A}
            \andalso
            \Gamma_2 \vdash t_2 : B
        }{
            \Gamma_1 + \Gamma_2 \vdash \app{t_1}{t_2} : A
        }
    \end{minipage}
\end{center}
In this case, there are two evaluation rules that can be applied to $t$.

\begin{itemize}
\item Suppose the evaluation rule matches to $[\cdot].$\\
We perform the case analysis for the last ruduction rule.

\begin{itemize}
\item Case (\textsc{E-abs1})
We know the evaluation of the assumption has the following form:
\begin{center}
\begin{prooftree}
    \AxiomC{$ $}
    \RightLabel{$\textsc{E-abs1}$}
    \UnaryInfC{ $\app{(\lam{x}{t'_1})}{t_2} \leadsto \app{(t_2 \rhd x)}{t'_1}$}
    \UnaryInfC{$ \underbrace{\app{(\lam{x}{t'_1})}{t_2}}_{t} \longrightarrow \underbrace{\app{(t_2 \rhd x)}{t'_1}}_{t'}$}
\end{prooftree}
\end{center}
By Lemma \ref{lemma:preservationreduction}, we know the following:
\begin{align*}
    \left.
    \begin{aligned}
        &\Gamma_1+\Gamma_2 \vdash \app{(\lam{x}{t'_1})}{t_2} : A\\
        &\app{(\lam{x}{t'_1})}{t_2} \leadsto \app{(t_2 \rhd x)}{t'_1}
    \end{aligned}
    \right\}
    \hspace{1em}\Longrightarrow\hspace{1em}
    \Gamma_1+\Gamma_2 \vdash \app{(t_2 \rhd x)}{t'_1} : A
\end{align*}
Thus, we obtain the conclusion of the theorem.

\item Case (\textsc{E-abs2})
\begin{center}
\begin{prooftree}
    \AxiomC{$ $}
    \RightLabel{$\textsc{E-abs2}$}
    \UnaryInfC{ $\app{(\lam{\pr{x}}{t'_1})}{t_2} \leadsto \clet{x}{t_2}{t_1'}$}
    \UnaryInfC{$ \underbrace{\app{(\lam{\pr{x}}{t'_1})}{t_2}}_{t} \longrightarrow \underbrace{\clet{x}{t_2}{t_1'}}_{t'}$}
\end{prooftree}
\end{center}
In this case, we know the typing derivation of $t$ has the following form:
\begin{center}
\begin{prooftree}
\AxiomC{$ $}
\RightLabel{(\mbox{[}\textsc{pVar}\mbox{]})}
\UnaryInfC{$ r \vdash x : B \rhd x:\verctype{B}{r}$}
\RightLabel{(\textsc{p}$_\square$)}
\UnaryInfC{$ - \vdash \pr{x} : \vertype{r}{B} \rhd x:\verctype{B}{r}$}
\AxiomC{$ \Gamma_1, x:\verctype{B}{r} \vdash t_1' : A$}
\RightLabel{(\textsc{abs})}
\BinaryInfC{$ \underbrace{ \Gamma_1 \vdash  \lam{\pr{x}}{t'_1} : \ftype{\vertype{r}{B}}{A} }_{P}$}
\end{prooftree}
\begin{prooftree}
\AxiomC{$P$}
\AxiomC{$ \Gamma_2 \vdash t_2 : \vertype{r}{B}$}
\RightLabel{(\textsc{app})}
\BinaryInfC{$ \Gamma_1+\Gamma_2 \vdash \app{(\lam{\pr{x}}{t'_1})}{t_2} : A$}
\end{prooftree}
\end{center}
Therefore, we can construct the derivation tree for $t'$ as follows.
\begin{center}
\begin{prooftree}
\AxiomC{$ \Gamma_2 \vdash t_2 : \vertype{r}{B}$}
\AxiomC{$ \Gamma_1, x:\verctype{B}{r} \vdash t_1' : A$}
\RightLabel{(\textsc{app})}
\BinaryInfC{$ \Gamma_1+\Gamma_2 \vdash \clet{x}{t_2}{t'_1} : A$}
\end{prooftree}
\end{center}
Hence, we have the conclusion of the theorem.\\
\end{itemize}


\item Suppose the evaluation rule matches to $\app{E}{t}$.\\
We know the evaluation of the assumption has the following form:
\begin{center}
\begin{prooftree}
    \AxiomC{$ $}
    \UnaryInfC{ $t'_1 \leadsto t''_1$}
    \UnaryInfC{$ \underbrace{\app{E[t'_1]}{t_2}}_{t} \longrightarrow \app{E[t''_1]}{t_2}$}
\end{prooftree}
\end{center}
where $t_1=E[t'_1]$.

By induction hypothesis, we know the following:
\begin{align*}
    \left.
    \begin{aligned}
        &\Gamma_1 \vdash E[t'_1] : \ftype{B}{A}\\
        &E[t'_1] \longrightarrow E[t''_1]
    \end{aligned}
    \right\}
    \hspace{1em}\Longrightarrow\hspace{1em}
    \Gamma_1 \vdash E[t''_1] : \ftype{B}{A}
    \tag{ih}
\end{align*}
We then reapply (\textsc{app}) to obtain the following:
\begin{center}
    \begin{minipage}{.70\linewidth}
        \infrule[app]{
            \Gamma_1 \vdash E[t''_1] : \ftype{B}{A}
            \andalso
            \Gamma_2 \vdash t_2  : B
        }{
            \Gamma_1+\Gamma_2 \vdash \app{E[t''_1]}{t_2} : A
        }
    \end{minipage}
\end{center}
Thus, we obtain the conclusion of the theorem.\\
\end{itemize}


\item Case (\textsc{let})
\begin{center}
    \begin{minipage}{.70\linewidth}
        \infrule[let]{
            \Gamma_1 \,\vdash\, t_1 : \vertype{r}{A}
            \andalso
            \Gamma_2, x:\verctype{A}{r} \,\vdash\, t_2 : B
        }{
            \Gamma_1 + \Gamma_2 \,\vdash\, \clet{x}{t_1}{t_2} : B
        }
    \end{minipage}
\end{center}
In this case, there are two evaluation rules that we can apply to $t$.
\begin{itemize}
\item Suppose the evaluation rule matches to $[\cdot]$.\\
We know the evaluation of the assumption has the following form:
\begin{center}
\begin{prooftree}
    \AxiomC{$ $}
    \RightLabel{(\textsc{E-clet})}
    \UnaryInfC{ $\clet{x}{[t_1']}{t_2} \leadsto ([t_1'] \rhd \pr{x})t_2$}
    \UnaryInfC{$ \underbrace{\clet{x}{[t_1']}{t_2}}_{t} \longrightarrow ([t_1'] \rhd \pr{x})t_2$}
\end{prooftree}
\end{center}
By Lemma \ref{lemma:preservationreduction}, we know the following:
\begin{align*}
    \left.
    \begin{aligned}
        &\Gamma_1+\Gamma_2 \vdash \clet{x}{[t_1']}{t_2} : B\\
        &\clet{x}{[t_1']}{t_2} \leadsto ([t_1'] \rhd \pr{x})t_2
    \end{aligned}
    \right\}
    \hspace{1em}\Longrightarrow\hspace{1em}
    \Gamma_1+\Gamma_2 \vdash ([t_1'] \rhd \pr{x})t_2 : B
\end{align*}
Thus, we obtain the conclusion of the theorem.

\item Suppose the evaluation rule matches to $\clet{x}{E}{t}$.\\
We know the evaluation of the assumption has the following form:
\begin{center}
\begin{prooftree}
    \AxiomC{$ $}
    \UnaryInfC{ $t'_1 \leadsto t''_1$}
    % \UnaryInfC{ $t_1 \longrightarrow t'_1$}
    \UnaryInfC{$ \underbrace{\clet{x}{E[t'_1]}{t_2}}_{t} \longrightarrow \clet{x}{E[t''_1]}{t_2}$}
\end{prooftree}
\end{center}
where $t_1=E[t'_1]$.

By induction hypothesis, we know the following:
\begin{align*}
    \left.
    \begin{aligned}
        &\Gamma_1 \vdash E[t'_1] : \vertype{r}{A}\\
        &E[t'_1] \longrightarrow E[t''_1]
    \end{aligned}
    \right\}
    \hspace{1em}\Longrightarrow\hspace{1em}
    \Gamma_1 \vdash E[t''_1] : \vertype{r}{A}
    \tag{ih}
\end{align*}
We then reapply (\textsc{let}) to obtain the following:
\begin{center}
    \begin{minipage}{.70\linewidth}
        \infrule[let]{
            \Gamma_1 \,\vdash\, E[t''_1] : \vertype{r}{A}
            \andalso
            \Gamma_2, x:\verctype{A}{r} \,\vdash\, t_2 : B
        }{
            \Gamma_1 + \Gamma_2 \,\vdash\, \clet{x}{E[t''_1]}{t_2} : B
        }
    \end{minipage}
\end{center}
Thus, we obtain the conclusion of the theorem.\\
\end{itemize}


\item Case (\textsc{weak})
\begin{center}
    \begin{minipage}{.45\linewidth}
        \infrule[weak]{
            \Gamma_1 \vdash t : A
            \andalso
            \vdash \Delta'
        }{
            \Gamma_1 + \verctype{\Delta'}{0} \vdash t : A
        }
    \end{minipage}
\end{center}
In this case, $t$ does not change between before and after the last derivation.
The induction hypothesis implies that there exists a term $t'$ such that:
\begin{align*}
        t \longrightarrow t'
        \ \land\ \Gamma_1 \vdash t' : A \tag{ih}
\end{align*}
We then reapply (\textsc{weak}) to obtain the following:
\begin{center}
    \begin{minipage}{.38\linewidth}
        \infrule[weak]{
            \Gamma_1 \vdash t' : A
            \andalso
            \vdash \Delta'
        }{
            \Gamma_1 + \verctype{\Delta'}{0} \vdash t' : A
        }
    \end{minipage}
\end{center}
Thus, we obtain the conclusion of the theorem.
\\

\item Case (\textsc{der})
\begin{center}
    \begin{minipage}{.45\linewidth}
        \infrule[der]{
            \Gamma_1, x:B \vdash t : A
        }{
            \Gamma_1, x:\verctype{B}{1} \vdash t : A
        }
    \end{minipage}
\end{center}
In this case, $t$ does not change between before and after the last derivation.
The induction hypothesis implies that there exists a term $t'$ such that:
\begin{align*}
        t \longrightarrow t'
        \ \land\ \Gamma_1, x:B \vdash t' : A \tag{ih}
\end{align*}
We then reapply (\textsc{der}) to obtain the following:
\begin{center}
    \begin{minipage}{.38\linewidth}
        \infrule[der]{
            \Gamma_1, x:B \vdash t' : A
        }{
            \Gamma_1, x:\verctype{B}{1} \vdash t' : A
        }
    \end{minipage}
\end{center}
Thus, we obtain the conclusion of the theorem.
\\

\item Case (\textsc{pr})
\begin{center}
    \begin{minipage}{.40\linewidth}
        \infrule[pr]{
            \verctype{\Gamma}{} \vdash t'' : B
            \andalso
            \vdash r
        }{
            r\cdot\verctype{\Gamma}{} \vdash \pr{t''} : \vertype{r}{B} 
        }
    \end{minipage}
\end{center}
This case holds trivially because there are no evaluation rules that can be applied to $\pr{t''}$.
\\

\item Case (\textsc{ver})
\begin{center}
    \begin{minipage}{.65\linewidth}
        \infrule[ver]{
            \verctype{\Gamma_i}{} \vdash t_i : A
            \andalso
            \vdash \{\overline{l_i}\}
        }{
            \bigcup_i(\{l_i\}\cdot [\Gamma_i]) \vdash \nvval{\overline{l_i=t_i}} : \vertype{\{\overline{l_i}\}}{A}
        }
    \end{minipage}
\end{center}
This case holds trivially because there are no evaluation rules that can be applied to $\nvval{\overline{l_i=t_i}}$.
\\

\item Case (\textsc{veri})
\begin{center}
    \begin{minipage}{.55\linewidth}
        \infrule[veri]{
            \verctype{\Gamma_i}{} \vdash t_i : A
            \andalso
            \vdash \{\overline{l_i}\}
            \andalso
            l_k \in \{\overline{l_i}\}
        }{
            \bigcup_i(\{l_i\}\cdot [\Gamma_i]) \vdash \ivval{\overline{l_i=t_i}}{l_k} : A
        }
    \end{minipage}
\end{center}
In this case, the only evaluation rule we can apply is evaluation for $[\cdot]$.
We know the evaluation of the assumption has the following form:
\begin{center}
\begin{prooftree}
    \AxiomC{$ $}
    \RightLabel{\textsc{E-veri}}
    \UnaryInfC{ $\ivval{\overline{l_i=t_i}}{l_k} \leadsto t_k@l_k$}
    \UnaryInfC{ $\underbrace{\ivval{\overline{l_i=t_i}}{l_k}}_{t} \longrightarrow t_k@l_k$}
\end{prooftree}
\end{center}
By Lemma \ref{lemma:preservationreduction}, we know the following:
\begin{align*}
    \left.
    \begin{aligned}
        &\textstyle{\bigcup_i}(\{l_i\}\cdot [\Gamma_i]) \vdash \ivval{\overline{l_i=t_i}}{l_k} : A\\
        &\ivval{\overline{l_i=t_i}}{l_k} \leadsto t_k@l_k
    \end{aligned}
    \right\}
    \hspace{1em}\Longrightarrow\hspace{1em}
    \textstyle{\bigcup_i}(\{l_i\}\cdot [\Gamma_i]) \vdash t_k@l_k : A
\end{align*}
Thus, we obtain the conclusion of the theorem.
\\

\item Case (\textsc{extr})
\begin{center}
    \begin{minipage}{.50\linewidth}
        \infrule[extr]{
            \Gamma \vdash t_1 : \vertype{r}{A}
            \andalso
            l_k \in r
        }{
            \Gamma \vdash t_1.l_k : A
        }
    \end{minipage}
\end{center}
In this case, there are two evaluation rules that we can apply to $t$.
\begin{itemize}
\item Suppose the evaluation rule matches to $[\cdot]$.\\
We know the evaluation of the assumption has the following form:
\begin{center}
\begin{prooftree}
    \AxiomC{$ $}
    \RightLabel{\textsc{E-ex1} or \textsc{E-ex2}}
    \UnaryInfC{$ t_1.l_k \leadsto t'_1$}
    \UnaryInfC{$ \underbrace{t_1.l_k}_{t} \longrightarrow t'_1$}
\end{prooftree}
\end{center}
By Lemma \ref{lemma:preservationreduction}, we know the following:
\begin{align*}
    \left.
    \begin{aligned}
        &\Gamma \vdash t_1.l_k : A\\
        &t_1.l_k \leadsto t'_1
    \end{aligned}
    \right\}
    \hspace{1em}\Longrightarrow\hspace{1em}
    \Gamma \vdash t'_1 : A
\end{align*}
Thus, we obtain the conclusion of the theorem.

\item Suppose the evaluation rule matches to $E.l$.\\
We know the evaluation of the assumption has the following form:
\begin{center}
\begin{prooftree}
    \AxiomC{$ $}
    \UnaryInfC{$ t'_1 \leadsto t''_1$}
    \UnaryInfC{$ \underbrace{E[t'_1].l_k}_{t} \longrightarrow E[t''_1].l_k$}
\end{prooftree}
\end{center}
where $t_1=E[t'_1]$.

By induction hypothesis, we know the following:
\begin{align*}
    \left.
    \begin{aligned}
        &\Gamma \vdash E[t'_1] : \vertype{r}{A}\\
        &E[t'_1] \longrightarrow E[t''_1]
    \end{aligned}
    \right\}
    \hspace{1em}\Longrightarrow\hspace{1em}
    \Gamma \vdash E[t''_1] : \vertype{r}{A}
    \tag{ih}
\end{align*}
We the reapply (\textsc{extr}) to obtain the following:
\begin{center}
    \begin{minipage}{.55\linewidth}
        \infrule[extr]{
            \Gamma \vdash E[t''_1] : \vertype{r}{A}
            \andalso
            l_k \in r
        }{
            \Gamma \vdash E[t''_1].l_k : A
        }
    \end{minipage}
\end{center}
Thus, we obtain the conclusion of the theorem.\\
\end{itemize}


\item Case (\textsc{sub})
\begin{center}
    \begin{minipage}{.65\linewidth}
        \infrule[\textsc{sub}]{
            \Gamma_1, x:\verctype{B}{r}, \Gamma_2 \vdash t : A
            \andalso
            r \sqsubseteq s
            \andalso
            \vdash s
        }{
            \Gamma_1, x:\verctype{B}{s}, \Gamma_2 \vdash t : A
        }
    \end{minipage}
\end{center}
In this case, $t$ does not change between before and after the last derivation.
The induction hypothesis implies that there exists a term $t'$ such that:
\begin{align*}
    t \longrightarrow t'
    \ \land\ 
    \Gamma_1, x:\verctype{B}{r}, \Gamma_2 \vdash t' : A \tag{ih}
\end{align*}
We then reapply (\textsc{sub}) to obtain the following:
\begin{prooftree}
    \AxiomC{$\Gamma_1, x:\verctype{B}{r}, \Gamma_2 \vdash t' : A$}
    \AxiomC{$ r \sqsubseteq s$}
    \AxiomC{$ \vdash s$}
    \RightLabel{(\textsc{sub})}
    \TrinaryInfC{$ \Gamma_1, x:\verctype{B}{s}, \Gamma_2 \vdash t' : A$}
\end{prooftree}
Thus, we obtain the conclusion of the theorem.

\end{itemize}
\end{proof}























%%%%%%%%%%%%%%%%%%%%%%%%%%%%%%%%%%%%%%%%%%%%
%%%%%%%%%%%%%%%%%%%%%%%%%%%%%%%%%%%%%%%%%%%%
%%%%%%%%%%%%%%%%%%%%%%%%%%%%%%%%%%%%%%%%%%%%
%%%%%%%%%%%%%%%%%%%%%%%%%%%%%%%%%%%%%%%%%%%%
%%%%%%%%%%%%%%%%%%%%%%%%%%%%%%%%%%%%%%%%%%%%
%%%%%%%%%%%%%%%%%%%%%%%%%%%%%%%%%%%%%%%%%%%%

\begin{theorem}[\corelang{} progress]
\label{lemma:progress}
\begin{align*}
    \emptyset \vdash t:A \Longrightarrow
    (\textnormal{\textsf{value}}\ t) \lor (\exists t'. t \longrightarrow t')
\end{align*}
\end{theorem}

\begin{proof}
The proof is given by induction on the typing derivation of $t$.
Consider the cases for the last rule used in the typing derivation of the assumption.
\\

\begin{itemize}
\item Case (\textsc{int})
\begin{center}
    \begin{minipage}{.25\linewidth}
        \infrule[int]{
            \\
        }{
            \emptyset \vdash n:\textsf{Int}
        }
    \end{minipage}
\end{center}
This case holds trivially because \textsf{value} $n$.
\\

\item Case (\textsc{var})
This case holds trivially because $x:A$ cannot be $\emptyset$.
\\

\item Case (\textsc{abs})
\begin{center}
    \begin{minipage}{.55\linewidth}
        \infrule[abs]{
            - \vdash p : A_1 \rhd \Delta'
            \andalso
            \Delta' \,\vdash\, t : A_2%\theta B
        }{
            \emptyset \,\vdash\, \lam{p}{t} : \ftype{A_1}{A_2}
        }
    \end{minipage}
\end{center}
This case holds trivially because \textsf{value} $\lam{p}{t}$.
\\

\item Case (\textsc{app})
\begin{center}
    \begin{minipage}{.65\linewidth}
        \infrule[app]{
            \emptyset \vdash t_1 : \ftype{B}{A}
            \andalso
            \emptyset \vdash t_2 : B
        }{
            \emptyset \vdash \app{t_1}{t_2} : A
        }
    \end{minipage}
\end{center}
There are two cases whether $t_1$ is a value or not.
\begin{itemize}
\item Suppose $t_1$ is a value.\\
By the inversion lemma (\ref{lemma:typedvalue}), we know that there exists a term $t'_1$ and $t_1=\lam{p}{t'_1}$.
Thus, we can apply two rules to $t$ as follows.

\begin{itemize}
\item Case (\textsc{E-abs1})
\begin{center}
\begin{prooftree}
\AxiomC{$ $}
\RightLabel{(\textsc{E-abs1})}
\UnaryInfC{$ \app{(\lam{x}{t'_1})}{t_2} \leadsto \app{(t_2 \rhd x)}{t'_1}$}
\RightLabel{}
\UnaryInfC{$ \underbrace{\app{(\lam{x}{t'_1})}{t_2}}_{t} \longrightarrow \app{(t_2 \rhd x)}{t'_1}$}
\end{prooftree}
\end{center}
Furthermore, we know the following:
\begin{center}
        \begin{minipage}{.50\linewidth}
            \infrule[\ensuremath{\rhd_{\mathrm{var}}}]{
                \\
            }{
                \app{(t_2 \rhd x)}{t'_1} = [t_2/x]t'_1
            }
        \end{minipage}
\end{center}
By choosing $t'=[t_2/x]t'_1$, we obtain the conclusion of the theorem.

\item Case (\textsc{E-abs2})
\begin{center}
\begin{prooftree}
    \AxiomC{$ $}
    \RightLabel{$\textsc{E-abs2}$}
    \UnaryInfC{ $\app{(\lam{\pr{x}}{t'_1})}{t_2} \leadsto \clet{x}{t_2}{t_1'}$}
    \UnaryInfC{$ \underbrace{\app{(\lam{\pr{x}}{t'_1})}{t_2}}_{t} \longrightarrow \underbrace{\clet{x}{t_2}{t_1'}}_{t'}$}
\end{prooftree}
\end{center}
By choosing $t'=\clet{x}{t_2}{t_1'}$, we obtain the conclusion of the theorem.\\
\end{itemize}

\item Suppose $t_1$ is not a value.\\
There exists a term $t'_1$ such that:
\begin{align*}
    \begin{minipage}{.13\linewidth}
        \infrule[]{
            t_1 \leadsto t'_1
        }{
            t_1 \longrightarrow t'_1
        }
    \end{minipage}
\end{align*}
Also, we can apply evaluation for application to $t$.
\begin{center}
    \begin{minipage}{.27\linewidth}
        \infrule[]{
            t_1 \leadsto t'_1
        }{
            \underbrace{\app{t_1}{t_2}}_{t} \longrightarrow \app{t'_1}{t_2}
        }
    \end{minipage}
\end{center}
Thus, by choosing $t'=\app{t'_1}{t_2}$, we obtain the conclusion of the theorem.\\
\end{itemize}


\item Case (\textsc{let})
\begin{center}
    \begin{minipage}{.65\linewidth}
        \infrule[let]{
            \emptyset \,\vdash\, t_1 : \vertype{r}{A}
            \andalso
            x:\verctype{A}{r} \,\vdash\, t_2 : B
        }{
            \emptyset \,\vdash\, \clet{x}{t_1}{t_2} : B
        }
    \end{minipage}
\end{center}
There are two cases whether $t_1$ is a value or not.
\begin{itemize}
\item Suppose $t_1$ is a value.\\
By the inversion lemma (\ref{lemma:typedvalue}), we know that $t_1$ has either a form of $[t'_1]$ or and $\nvval{\overline{l_i=t''_i}}$.
\begin{itemize}
\item Case $t_1=[t_1']$.\\
In this case, we can apply (\textsc{E-clet}) to obtain the following.
\begin{center}
    \begin{minipage}{.8\linewidth}
        \infrule[E-clet]{
            \\
        }{
            \clet{x}{[t_1']}{t_2} \leadsto ([t_1'] \rhd \pr{x})t_2
        }
    \end{minipage}
\end{center}
Thus, we can apply (\textsc{$\rhd_\square$}) and (\textsc{$\rhd_{\textnormal{var}}$}) to obtain the following.
\begin{center}
\begin{prooftree}
\AxiomC{$ $}
\RightLabel{($\rhd_{\textnormal{var}}$)}
\UnaryInfC{$ (t_1' \rhd x)t_2 = [t_1' / x] t_2$}
\RightLabel{($\rhd_{\square}$)}
\UnaryInfC{$ ([t_1'] \rhd \pr{x})t_2 = [t_1' / x] t_2$}
\end{prooftree}
\end{center}
Thus, by choosing $t' = [t'_1/x]t_2$, we obtain the conclusion of the theorem.

\item Case $t_1 = \nvval{\overline{l_i=t''_i}}$.\\
In this case, we can apply (\textsc{E-clet}) to obtain the following:
\begin{center}
\begin{prooftree}
\AxiomC{$ $}
\RightLabel{(\textsc{E-clet})}
\UnaryInfC{$ \clet{x}{\nvval{\overline{l_i=t''_i}}}{t_2} \leadsto (\ivval{\overline{l_i=t''_i}}{l_k} \rhd \pr{x})t_2$}
\RightLabel{}
\UnaryInfC{$ \underbrace{\clet{x}{\nvval{\overline{l_i=t''_i}}}{t_2}}_{t} \longrightarrow (\ivval{\overline{l_i=t''_i}}{l_k} \rhd \pr{x})t_2 $}
\end{prooftree}
\end{center}
Also, we can apply (\textsc{$\rhd_\textnormal{ver}$}) and (\textsc{$\rhd_{\textnormal{var}}$}) to obtain the following.
\begin{center}
\begin{prooftree}
\AxiomC{$ $}
\RightLabel{($\rhd_{\textnormal{var}}$)}
\UnaryInfC{$ (\ivval{\overline{l_i=t''_i}}{l_k} \rhd x)t_2 = [\ivval{\overline{l_i=t''_i}}{l_k} / x] t_2$}
\RightLabel{($\rhd_\textnormal{ver}$)}
\UnaryInfC{$ (\nvval{\overline{l_i=t''_i}} \rhd \pr{x})t_2 = [\ivval{\overline{l_i=t''_i}}{l_k} / x] t_2$}
\end{prooftree}
\end{center}
Thus, by choosing $t' = [\ivval{\overline{l_i=t''_i}}{l_k}/x]t_2$, we obtain the conclusion of the theorem.
\end{itemize}

\item Suppose $t_1$ is not a value.\\
There exists terms $t'_1$ such that:
\begin{align*}
         \begin{minipage}{.13\linewidth}
            \infrule[]{
                 t_1\leadsto t'_1 \\
            }{
                 t_1\longrightarrow t'_1 
            }
        \end{minipage}
\end{align*}
Also, we can apply evaluation for contextual let bindings to $t$.
\begin{center}
    \begin{minipage}{.80\linewidth}
        \infrule[]{
            t_1 \leadsto t'_1
        }{
            \underbrace{\clet{x}{t_1}{t_2}}_{t} \longrightarrow \clet{x}{t'_1}{t_2}
        }
    \end{minipage}
\end{center}
Thus, by choosing $t'=(\clet{x}{t'_1}{t_2})$, we obtain the conclusion of the theorem.\\
\end{itemize}


\item Case (\textsc{weak})
\begin{center}
    \begin{minipage}{.34\linewidth}
        \infrule[weak]{
            \emptyset \vdash t : A
            \andalso
            \vdash \emptyset
        }{
            \emptyset \vdash t : A
        }
    \end{minipage}
\end{center}
In this case, $t$ does not change between before and after the last derivation.
Thus, we can obtain the conclusion of the theorem by induction hypothesis.
\\

\item Case (\textsc{der})\\
This case hold trivially because $\Gamma_1, x:\verctype{B}{1}$ cannot be $\emptyset$.
\\

\item Case (\textsc{pr})
\begin{center}
    \begin{minipage}{.29\linewidth}
        \infrule[pr]{
            \emptyset \vdash t : B
            \andalso
            \vdash r
        }{
            \emptyset \vdash \pr{t} : \vertype{r}{B} 
        }
    \end{minipage}
\end{center}
This case holds trivially because \pr{t} is a value.
\\

\item Case (\textsc{ver})
\begin{center}
    \begin{minipage}{.52\linewidth}
        \infrule[ver]{
            \emptyset \vdash t_i : A
            \andalso
            \vdash \{\overline{l_i}\}
        }{
            \emptyset \vdash \nvval{\overline{l_i=t_i}} : \vertype{\{\overline{l_i}\}}{A}
        }
    \end{minipage}
\end{center}
This case holds trivially because $\nvval{\overline{l_i=t_i}}$ is a value.
\\

\item Case (\textsc{veri})
\begin{center}
    \begin{minipage}{.45\linewidth}
        \infrule[veri]{
            \emptyset \vdash t_i : A
            \andalso
            \vdash \{\overline{l_i}\}
            \andalso
            l_k \in \{\overline{l_i}\}
        }{
            \emptyset \vdash \ivval{\overline{l_i=t_i}}{l_k} : A
        }
    \end{minipage}
\end{center}
In this case, we can apply (\textsc{E-veri}).
\begin{center}
\begin{prooftree}
    \AxiomC{$ $}
    \RightLabel{(\textsc{E-veri})}
    \UnaryInfC{$\ivval{\overline{l_i=t_i}}{l_k} \leadsto t_k@l_k$}
    \RightLabel{}
    \UnaryInfC{$\ivval{\overline{l_i=t_i}}{l_k} \longrightarrow t_k@l_k$}
\end{prooftree}
\end{center}
Thus, by choosing $t'=t_k@l_k$, we obtain the conclusion of the theorem.
\\

\item Case (\textsc{extr})
\begin{center}
    \begin{minipage}{.45\linewidth}
        \infrule[extr]{
            \emptyset \vdash t_1 : \vertype{r}{A}
            \andalso
            l_k \in r
        }{
            \emptyset \vdash t_1.l_k : A
        }
    \end{minipage}
\end{center}
In this case, we have two cases whether $t_1$ is a value or not.
\begin{itemize}
\item Suppose $t_1$ is a value. ($t_1=v_1$)\\
By Lemma \ref{lemma:extraction}, we know the following:
\begin{align*}
    \emptyset \,\vdash\, v_1 : \vertype{r}{A}
    \hspace{1em}\Longrightarrow\hspace{1em}
    \exists t'.
    \left\{
    \begin{aligned}
        &v_1.l_k   \longrightarrow  t' \\
        &\emptyset \vdash t' : A
    \end{aligned}
    \right.
\end{align*}
Thus, we obtain the conclusion of the theorem.

\item Suppose $t_1$ is not a value.\\
There exists a term $t_1$ such that:
\begin{align*}
    \begin{minipage}{.13\linewidth}
        \infrule[]{
            t_1 \leadsto t'_1
        }{
            t_1 \longrightarrow t'_1
        }
    \end{minipage}
\end{align*}
Also, we can apply an exaluation rule for extraction to $t$.
\begin{center}
    \begin{minipage}{.15\linewidth}
        \infrule[]{
            t_1 \leadsto t'_1
        }{
            \underbrace{t_1.l_k}_{t} \longrightarrow t'_1.l_k
        }
    \end{minipage}
\end{center}
Thus, by choosing $t'=t_1'.l_k$, we obtain the conclusion of the theorem.\\
\end{itemize}


\item Case (\textsc{sub})
\begin{center}
    \begin{minipage}{.65\linewidth}
        \infrule[\textsc{sub}]{
            \Gamma_1, x:\verctype{B}{r}, \Gamma_2 \vdash t : A
            \andalso
            r \sqsubseteq s
            \andalso
            \vdash s
        }{
            \Gamma_1, x:\verctype{B}{s}, \Gamma_2 \vdash t : A
        }
    \end{minipage}
\end{center}
In this case, $t$ does not change between before and after the last derivation.
Thus, by induction hypothesis, we obtain the conclusion of the theorem.

\end{itemize}
\end{proof}



















