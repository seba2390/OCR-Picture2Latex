% Section Template

\section{Implementation}
\label{implementation} % Change X to a consecutive number; for referencing this Section elsewhere, use \ref{SectionX}

%----------------------------------------------------------------------------------------
%	section 1
%----------------------------------------------------------------------------------------

% \begin{figure}[t]
\begin{tikzpicture}[node distance=1.5cm, align=center
    ]
    \tikzset{%
      >={Latex[width=2mm,length=2mm]},
      % Specifications for style of nodes:
               label/.style = {fill=white, font=\sffamily},
                base/.style = {rectangle, rounded corners, draw=black,
                               minimum width=3cm, minimum height=1cm,
                               text centered, font=\sffamily},
         constraints/.style = {base, fill=green!30},
           interface/.style = {base, fill=orange!15, font=\ttfamily},
            multimod/.style = {rounded rectangle,  draw,  text centered, line width=2pt, 
                               text width=3cm, minimum height=1cm, font=\ttfamily},
           singlemod/.style = {rounded rectangle,  draw,  text centered, 
                               text width=3cm, minimum height=1cm, font=\ttfamily},
      multimodsample/.style = {rounded rectangle,  draw,  text centered, line width=2pt, 
                               text width=2cm, font=\ttfamily},
     singlemodsample/.style = {rounded rectangle,  draw,  text centered, 
                               text width=2cm, font=\ttfamily},
    }
  % Specification of nodes (position, etc.)
  \node (source) 
        [multimod]
        {VL Source};
  \node (entryAbsyn)
        [singlemod, right of=source, xshift=4cm]
        {Lang.Absyn};
  \node (entryLambdaVL)
        [singlemod, below of=entryAbsyn, yshift=-1cm]
        {Lang.LambdaVL};
  \node (duplicatedLambdaVL)
        [singlemod, below of=entryLambdaVL, yshift=-1cm]
        {Lang.LambdaVL};
  \node (inferredLambdaVL)
        [singlemod, below of=duplicatedLambdaVL, yshift=-1cm]
        {Lang.LambdaVL};
  \node (bundledLambdaVL)
        [multimod, below of=inferredLambdaVL, yshift=-1cm]
        {Lang.LambdaVL};
  \node (specifiedLambdaVL)
        [multimod, below of=bundledLambdaVL, yshift=-1cm] 
        {Lang.LambdaVL};
  \node (endAbsyn)
        [multimod, below of=specifiedLambdaVL, yshift=-1cm] 
        {Lang.Absyn};
  \node (typeinfo)
        [interface, right of=inferredLambdaVL, xshift=2.5cm]
        {Syntax.Type};
  \node (interface)
        [interface, below of=typeinfo, yshift=-1cm]
        {Syntax.Env\\(Interface)};
  \node (Constraints)
        [constraints, left of=duplicatedLambdaVL, xshift=-4cm]
        {Constraints};
  \node (Vectors)
        [constraints, below of=Constraints, yshift=-1cm]
        {Vectors};
  \node (labels)
        [singlemod, left of=bundledLambdaVL, xshift=-4cm]  
        {Labels};
  \node (HsSyn)
        [multimod, left of=endAbsyn, xshift=-4cm]
        {GHC.HsSyn};  

  \node (sampleSingle) 
        [singlemodsample, right of=entryAbsyn, xshift=3cm]
        {Single module};
  \node (sampleMulti) 
        [multimodsample, right of=entryAbsyn, xshift=3cm, yshift=-1.2cm]
        {Multiple modules};

  % Specification of lines between nodes specified above
  % with aditional nodes for description 
  \draw[->]             (source) -- node[font=\sffamily,midway,below]
                                    {parser\\and\\dependency\\analyser} (entryAbsyn);
  \draw[->]         (entryAbsyn) -- node[label]{girardFwd}
                                    (entryLambdaVL);
  \draw[->]      (entryLambdaVL) -- node[label](duplication)
                                    {Duplication} (duplicatedLambdaVL);
  \draw[->] (duplicatedLambdaVL) -- node[label](typeinference)
                                    {Type Inference} (inferredLambdaVL);
  \draw[->]   (inferredLambdaVL) -- node[label](bundling)
                                    {Bundling} (bundledLambdaVL);
  \draw[->]      (typeinference) -| (typeinfo);
  \draw[->]           (typeinfo) -- node[label] {Bundling} (interface);
  \draw[->]      (typeinference) -- ++ (-2.5, 0) |- (Constraints);
  \draw[->]           (bundling) -- ++ (-2.5, 0) |- (Constraints);
  \draw[->]        (Constraints) -- node[label] {Vectorize} (Vectors);
  \draw[->]            (Vectors) -- node[label] {z3} (labels);
  \draw[->]    (bundledLambdaVL) -- (specifiedLambdaVL);
  \draw[->]  (specifiedLambdaVL) -- node[label] {girardBck} (endAbsyn);
  \draw[->]          (interface) -- ++ (2,0) |- node[label] {import} (duplication);
  \draw[->]        (labels) -- ++ (0,-1) -| node [font=\sffamily, below, pos=0.25]
                                                      {Version\\specialization}
                                                 (specifiedLambdaVL);
  \draw[->]           (endAbsyn) -- (HsSyn);
  % \draw[->]    (bundledLambdaVL) -| node(priorityXMemory)
  %                                   {higher priority $\rightarrow$ more memory}
  %                                   (Constraints);
  % \draw[->]      (Constraints)  |- node [yshift=-2cm, text width=3.1cm]
  %                                   {User navigates back to the activity}
  %                                   (absyn);
  % \draw[->] (bundledLambdaVL.east) -- ++(2.6,0) -- ++(0,2) -- ++(0,2) --                
  %    node[xshift=1.2cm,yshift=-1.5cm, text width=2.5cm]
  %    {The activity comes to the foreground}(duplicatedLambdaVL.east);
  \end{tikzpicture}
  \caption{The pipeline of VL Language System}
  \label{fig:pipeline}
\end{figure}
% \subsection{Overview}
We implement the \mylang{} compiler\footnote{\url{https://github.com/yudaitnb/vl}} on GHC (v9.2.4) with haskell-src-exts\footnote{\url{https://hackage.haskell.org/package/haskell-src-exts}} as its parser with an extension of versioned control terms, and z3~\cite{10.1007/978-3-540-78800-3_24} as its constraint solver.
The \mylang{} compiler performs the code generation by compiling \vlmini{} programs back into $\lambda$-calculus via Girard's translation and then translating them into Haskell ASTs using the version in the result version labels.

% \vspace{-1\baselineskip}
% \vspace{-.5\baselineskip}
\paragraph{\textbf{Ad-hoc Version Polymorphism via Duplication}}
\label{sec:adhocversionpolymorphism}
The \mylang{} compiler replicates external variables to assign individual versions to homonymous external variables.
Duplication is performed before type checking of individual versions and renames every external variable along with the type and constraint environments generated from the import declarations.
Such ad hoc conversions are necessary because \vlmini{} is monomorphic, and the type inference of \vlmini{} generates constraints by referring only to the variable's name in the type environment.
Therefore, assigning different versions to homonymous variables requires manual renaming in the preliminary step of the type inference.
A further discussion on version polymorphism can be found in Section \ref{sec:fullversionpolymorphism}.

% \vspace{-.5\baselineskip}
\paragraph{\textbf{Constraints Solving with z3}}
We use sbv\footnote{\url{https://hackage.haskell.org/package/sbv-9.0}} as the binding of z3.
The sbv library internally converts constraints into SMT-LIB2 scripts~\cite{barrett2010smt} and supplies it to z3.
Dependency constraints are represented as vectors of symbolic integers, where the length of the vector equals the number of external modules, and the elements are unique integers signifying each module's version number. Constraint resolution identifies the expected vectors for symbolic variables, corresponding to the label on which external identifiers in \mylang{} should depend. If more than one label satisfies the constraints, the default action is to select a newer one.
