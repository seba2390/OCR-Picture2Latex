% Section Template

\section{Compilation} % Main Section title
\label{compilation} % Change X to a consecutive number; for referencing this Section elsewhere, use \ref{SectionX}

\begin{figure}[tb]
\centering
\begin{tikzpicture}[thick]
    \tikzset{vecArrow/.style={thick, decoration={markings,mark=at position
    1 with {\arrow[semithick]{open triangle 60}}},
    double distance=1.4pt, shorten >= 5.5pt,
    preaction = {decorate},
    postaction = {draw,line width=1.4pt, white,shorten >= 4.5pt}}};
    \tikzset{innerWhite/.style={semithick, white,line width=1.4pt, shorten >= 4.5pt}};
    \tikzset{Package/.style={rectangle, fill=cyan!10, text centered, rounded corners, minimum width=1.9cm, minimum height=0.65cm}};
    \tikzset{App/.style={rectangle, text centered, minimum width=1.2cm, minimum height=0.65cm}};

    \newlength\mywidth
    \newlength\myheight
    \newlength\tempdimen
    
    \newcommand\getdimensions[1]{
        \pgfextractx{\mywidth}{\pgfpointanchor{#1}{east}}
        \pgfextractx{\tempdimen}{\pgfpointanchor{#1}{west}}
        \addtolength{\mywidth}{-\tempdimen}
        \pgfextracty{\myheight}{\pgfpointanchor{#1}{north}}
        \pgfextracty{\tempdimen}{\pgfpointanchor{#1}{south}}
        \addtolength{\myheight}{-\tempdimen}
    }

    \tikzset{
      pics/stacked rectangles/.style n args={4}{
        code={
          \def\rectangleWidth{2cm}  % 長方形の幅を固定
          \def\rectangleHeight{1cm} % 長方形の高さを固定
          \pgfmathsetmacro\offsetY{(#1+1)*#2/2} % Yのオフセットを計算
          
          % 一時的なrect1を描画してサイズを取得
          \node[draw=none, align=center] (tempRect1) at (1*#2,1*#2-\offsetY) { #3 };
          
          % ここでtempRect1の寸法を取得
          \getdimensions{tempRect1}
          
          % 長方形を描画
          \foreach \i in {#1,...,2} {
            \node[draw, fill=white, minimum width=\mywidth, minimum height=\myheight] (rect\i) at (\i*#2,\i*#2-\offsetY) {};
          }

          % 一番前の長方形を描画
          \node[draw, fill=white, minimum width=\mywidth, minimum height=\myheight, align=center] (rect1) at (1*#2,1*#2-\offsetY) { #3 };
    
          % すべての長方形を囲む仮想的なノード
          \node[draw=none, inner sep=0pt, fit={(rect1.south west) (rect#1.north east)}, name=#4] {};
        }
      }
    }

    \tikzset{
      pics/stackedty rectangles/.style n args={4}{
        code={
          \def\rectangleWidth{2cm}  % 長方形の幅を固定
          \def\rectangleHeight{1cm} % 長方形の高さを固定
          \pgfmathsetmacro\offsetY{(#1+1)*#2/2} % Yのオフセットを計算
          
          % 一時的なrect1を描画してサイズを取得
          \node[draw=none, align=center] (tempRect1) at (1*#2,1*#2-\offsetY) { #3 };
          
          % ここでtempRect1の寸法を取得
          \getdimensions{tempRect1}
          
          % 長方形を描画
          \foreach \i in {#1,...,2} {
            \node[draw, fill=white, minimum width=\mywidth, minimum height=\myheight, rounded corners] (rect\i) at (\i*#2,\i*#2-\offsetY) {};
          }

          % 一番前の長方形を描画
          \node[draw, fill=white, minimum width=\mywidth, minimum height=\myheight, align=center, rounded corners] (rect1) at (1*#2,1*#2-\offsetY) { #3 };
    
          % すべての長方形を囲む仮想的なノード
          \node[draw=none, inner sep=0pt, fit={(rect1.south west) (rect#1.north east)}, name=#4] {};
        }
      }
    }


    \tikzset{
      pics/stackedtybundled rectangles/.style n args={4}{
        code={
          \def\rectangleWidth{2cm}  % 長方形の幅を固定
          \def\rectangleHeight{1cm} % 長方形の高さを固定
          \pgfmathsetmacro\offsetY{(#1+1)*#2/2} % Yのオフセットを計算
          
          % 一時的なrect1を描画してサイズを取得
          \node[draw=none, align=center] (tempRect1) at (1*#2,1*#2-\offsetY) { #3 };
          
          % ここでtempRect1の寸法を取得
          \getdimensions{tempRect1}
          
          % 長方形を描画
          \foreach \i in {#1,...,2} {
            \node[draw, fill=white, minimum width=\mywidth, minimum height=\myheight, rounded corners, line width=2pt] (rect\i) at (\i*#2,\i*#2-\offsetY) {};
          }

          % 一番前の長方形を描画
          \node[draw, fill=white, minimum width=\mywidth, minimum height=\myheight, align=center, rounded corners, line width=2pt] (rect1) at (1*#2,1*#2-\offsetY) { #3 };
    
          % すべての長方形を囲む仮想的なノード
          \node[draw=none, inner sep=0pt, fit={(rect1.south west) (rect#1.north east)}, name=#4] {};
        }
      }
    }

    \pic at (0,0) {stacked rectangles={4}{0.1cm}{\mylang{}\\program}{vl1}};
    % \node[draw,App, align=center] at (3.0,0) (vlmini1) {\vlmini{}\\($V_{M_i}$)};
    \pic at (3.3,0) {stacked rectangles={4}{0.1cm}{\vlmini{}\\program}{vlmini1}};
    % \node[draw,App, align=center] at (6.0,0) (vlmini2) {\vlmini{}\\interface};
    \pic at (6.5,0) {stackedty rectangles={4}{0.1cm}{\vlmini{}\\interface}{vlmini2}};
    \pic at (9.7,1.4) {stackedtybundled rectangles={3}{0.12cm}{\vphantom{b}\hspace{3em}}{vlmini4}};
    \node[draw,App, align=center, rounded corners, line width=2pt] at (10,0) (vlmini3) {\vlmini{}\\interface\\(bundled)};
    % \node[draw,App, align=center, rounded corners, line width=2pt, minimum width=1.62cm, minimum height=0.5cm] at (10,1.4) (vlmini4) {};
    \coordinate (topofinference) at ($(vlmini1.east)!0.5!(vlmini2.west)$);
    \coordinate (botofinference) at ($(vlmini1.east)!0.5!(vlmini2.west) + (0, -1.75)$);
    \coordinate (topofbundling) at ($(vlmini2.east)!0.5!(vlmini3.west)$);
    \coordinate (botofbundling) at ($(vlmini2.east)!0.5!(vlmini3.west) + (0, -1.75)$);
    % \node[draw,App,double,align=center] at (botofinference) (vdep) {\footnotesize{Variable}\\\footnotesize{dependency}};
    % \node[draw,App,double,align=center] at (botofbundling) (ldep) {\footnotesize{Label}\\\footnotesize{dependency}};
    \node[draw,App,double] at (10,-1.5) (constraints) {Constraints};
    \fill (topofinference) circle (2.5pt);
    \fill (topofbundling) circle (2.5pt);
    
    \draw[-latex] (vl1.east) to [yshift=-5]node[midway,below,align=center] {\footnotesize{Girard's}\\\footnotesize{translation}\\\footnotesize{(version-wise)}} (vlmini1.west);
    \draw[-latex] (vlmini1.east) to node[midway,below] {} (vlmini2.west);
    \draw[-latex] (vlmini1.east) to [yshift=-5]node[midway,below,align=center] (inference){\footnotesize{Type}\\\footnotesize{inference}} (vlmini2.west);
    \draw[-latex] (vlmini2.east) to [yshift=-5]node[midway,below,align=center] (bundling) {\footnotesize{Bundling}} (vlmini3.west);
    \draw[-latex] (vlmini4.west) -| node[near start,below,align=center] {\footnotesize{Import modules}} ([yshift=2]topofinference);
    \draw[-latex] (inference.south) |- (constraints.west);
    \draw[-latex] (bundling.south) |- (constraints.west);
    % \draw[-latex] (inference.south) -- (vdep.north);
    % \draw[-latex] (bundling.south) -- (ldep.north);
\end{tikzpicture}
\caption{The translation phases for a single module with multiple versions.}
\label{fig:translationoverview}
\end{figure}
The entire translation consists of three parts: (1) \emph{Girard's translation}, (2) an \emph{algorithmic type inference}, and (3) \emph{bundling}.
Figure \ref{fig:translationoverview} shows the translation process of a single module. First, through Girard's translation, each version of the \mylang{} program undergoes a version-wise translation into the \vlmini{} program. 
Second, the type inference synthesizes types and constraints for top-level symbols. Variables imported from external modules reference the bundled interface generated in the subsequent step.
Finally, to make the external variables act as multi-version expressions, bundling consolidates each version's interface into one \vlmini{} interface.
These translations are carried out in order from downstream of the dependency tree.
By resolving all constraints up to the main module, the appropriate version for every external variable is determined.

It is essential to note that the translations focus on generating constraints for dispatching external variables into version-specific code. While implementing versioned records in \corelang{} presents challenges, such as handling many version labels and their code clones, our method is a constraint-based approach in \vlmini{} that enables static inference of version labels without their explicit declaration.

In the context of coeffect languages, constraint generation in \mylang{} can be seen as the automatic generation of type declarations paired with resource constraints.
Granule\cite{Orchard:2019:Granule} can handle various resources as coeffects, but it requires type declarations to indicate resource constraints. \mylang{} restricts its resources solely to the version label set. This specialization enables the automatic collection of version information from external sources outside the codebase.

\subsection{An Intermediate Language, \vlmini{}}
% \vspace{-1\baselineskip}
\subsubsection{Syntax of \vlmini{}}
\begin{figure}[tb]
    \hrule
    \medskip
    \begin{minipage}{.9\textwidth}
        \textbf{\vlmini{} syntax (w/o data constructors and version control terms)}
    \end{minipage}
    \begin{minipage}{\textwidth}
        \vspace{-.8\baselineskip}
        \begin{minipage}{.475\textwidth}
          \begin{align*}
            t & ::= n \mid x \mid \app{t_1}{t_2} \mid \lam{p}{t} \mid \pr{t} \tag{terms}\\
            p & ::= x \mid [x] \tag{patterns}\\
            A, B & ::= \inttype{} \mid \alpha \mid \ftype{A}{B} \mid \vertype{r}{A} \tag{types}\\
            \kappa & ::= \typekind{} \mid \labelskind{} \tag{kinds}
          \end{align*}
        \end{minipage}
        \begin{minipage}{.475\textwidth}
          \begin{align*}
            \Gamma & ::= \emptyset \mid \Gamma, x:A \mid \Gamma, x:\verctype{A}{r} \tag{contexts}\\
            \Sigma & ::= \emptyset \mid \Sigma, \alpha:\kappa \tag{type variable kinds}\\
            R      & ::= - \mid r \tag{resource contexts}\\
          \end{align*}
        \end{minipage}
    \end{minipage}
    \begin{minipage}{.9\textwidth}
        \medskip\textbf{Extended with data constructors}
    \end{minipage}
    \begin{minipage}{\textwidth}
        \vspace{-.5\baselineskip}
        \begin{minipage}{.52\textwidth}
          \begin{align*}
            t & ::= \ldots \mid C\,\overline{t_i} \mid \caseof{t}{\overline{p_i \mapsto t_i}}\tag{terms}\\
            p & ::= \ldots \mid C\,\overline{p_i} \tag{patterns}\\
            C & ::= (,) \mid [,] \tag{constructors}
          \end{align*}
        \end{minipage}
        \begin{minipage}{.44\textwidth}
          \begin{align*}
            A, B & ::= ... \mid K \overline{A_i} \tag{types}\\
            K    & ::= (,) \mid [,] \tag{type constructors}\\
          \end{align*}
        \end{minipage}
    \end{minipage}
    \begin{minipage}{.9\textwidth}
        \medskip\textbf{Extended with version control terms}
    \end{minipage}
    \begin{minipage}{\textwidth}
        \vspace{-.5\baselineskip}
        \begin{minipage}{\textwidth}
          \begin{align*}
            t & ::= \ldots \mid \verof{\{\overline{M_i=V_i}\}}{t} \mid \unver{t}\hspace{6em}\tag{terms}
          \end{align*}
        \end{minipage}
    \end{minipage}
    \begin{minipage}{.9\textwidth}
        \medskip\textbf{\vlmini{} constraints}
    \end{minipage}
    \begin{minipage}{\textwidth}
      \vspace{-.5\baselineskip}
      \begin{align*}
        \mathcal{C} & ::= \underbrace{\top \mid \mathcal{C}_1 \land \mathcal{C}_2 \mid \mathcal{C}_1 \lor \mathcal{C}_2}_{\substack{\text{propositional}\\\text{formulae}}} \mid \underbrace{\vphantom{\mid}\alpha \preceq \alpha'}_{\substack{\text{variable}\\\text{dependencies}}} \mid \underbrace{\vphantom{\mid}\alpha \preceq \mathcal{D}}_{\substack{\text{label}\\\text{dependencies}}}\tag{dependency 
 constraints}\\
        \mathcal{D} & ::= \cs{\overline{M_i = V_i}} \tag{dependent labels}\\
        \Theta      & ::= \top \mid \Theta_1 \land \Theta_2 \mid \{A \sim B\} \tag{type constraints}
      \end{align*}
    \end{minipage}
  % \begin{align*}
  %   t & ::= n \mid x \mid \app{t_1}{t_2} \mid \lam{p}{t} \mid \pr{t} \mid C\,\overline{t_i} \mid \caseof{t}{\overline{p_i \mapsto t_i}}\tag{terms}\\
  %   p & ::= n \mid x \mid \_ \mid [p] \mid C\,\overline{p_i} \tag{patterns}\\
  %   C & ::= (,) \mid [,] \tag{constructors}\\
  %   A, B & ::= \inttype{} \mid K \overline{A_i} \mid \bgr{\alpha} \mid \ftype{A}{B} \mid \vertype{r}{A} \tag{types}\\
  %   K    & ::= (,) \mid [,] \tag{type constructors}\\
  %   \kappa & ::= \typekind{} \mid \labelskind{} \tag{kinds}\\
  %   \Gamma & ::= \emptyset \mid \Gamma, x:A \mid \Gamma, x:\verctype{A}{r} \tag{contexts}\\
  %   \Sigma & ::= \emptyset \mid \Sigma, \alpha:\kappa \tag{type variable kinds}\\
  %   R      & ::= - \mid r \tag{resource contexts}\\
  %   \mathcal{C} & ::= \top \mid \mathcal{C}_1 \land \mathcal{C}_2 \mid \mathcal{C}_1 \lor \mathcal{C}_2 \mid \underbrace{\vphantom{\mid}\alpha \preceq \alpha'}_{\textnormal{variable dependencies}} \mid \underbrace{\vphantom{\mid}\alpha \preceq L}_{\textnormal{label dependencies}}\tag{constraints}\\
  %   L & ::= \cs{\overline{M_i \mapsto V_i}} \tag{dependent labels}
  % \end{align*}
  \medskip
  \hrule
  \caption{The syntax of \vlmini{}.}
  \label{syntax:vlmini}
\end{figure}
Figure \ref{syntax:vlmini} shows the syntax of \vlmini{}.
\vlmini{} encompasses all the terms in \corelang{} except for versioned records $\nvval{l_i=t_i}$, intermediate term $\ivval{\overline{l_i=t_i}}{l_k}$, and extractions $t.l_k$. As a result, its terms are analogous to those in \lrpcf{}\cite{brunel_core_2014} and GrMini\cite{Orchard:2019:Granule}. However, \vlmini{} is specialized to treat version resources as coeffects.
We also introduce data constructors by introduction $C\,t_1,...,t_n$ and elimination $\caseof{t}{\overline{p_i \mapsto t_i}}$ for lists and pairs, and version control terms \unver{t} and \verof{\{\overline{M_i=V_i}\}}{t}. 
Here, contextual-let in \corelang{} is a syntax sugar of lambda abstraction applied to a promoted pattern.
\begin{align*}
\clet{x}{t_1}{t_2} \triangleq \app{(\lam{\pr{x}}{t_2})}{t_1}
\end{align*}

Types, version labels, and version resources are almost the same as \corelang{}.
Type constructors are also added to the type in response to the \vlmini{} term having a data constructor.
The remaining difference from \corelang{} is type variables $\alpha$. Since \vlmini{} is a monomorphic language, type variables act as unification variables; type variables are introduced during the type inference and are expected to be either concrete types or a set of version labels as a result of constraint resolution.
To distinguish those two kinds of type variables, we introduce kinds $\kappa$.
The kind \labelskind{} is given to type variables that can take a set of labels $\{\overline{l_i}\}$ and is used to distinguish them from those of kind \typekind{} during algorithmic type inference.

% \vspace{-1\baselineskip}
\subsubsection{Constraints}
% \subsection{Constraints}

In many trajectory optimization problems, there are constraints on the trajectory. It should be noted that these are different than a simple bound on the system input or state. In walking robots, common constraints are: 

\begin{itemize}
	\item reaction forces in friction cone
	\item robot above the ground
	\item robot moving forward
	\item foot hits ground from above... 
\end{itemize}

The most important thing to notice about constraints is that they must also be consistent and smooth. It isn't enough to say ``The lowest point of the foot during the step must be above the ground'' because this will pass data from a different grid-point on each function call. Rather, you need to say ``Every point of the foot trajectory must be above the ground'' (or use some very fancy smoothing in your maximization function). All rules that apply to objective functions regarding smoothness and consistency, also should be applied to calculations of constraints.

\par It is also important to minimize the number of unused constraints, provided that the problem remains smooth and consistent. One way to do this is to leave all of the constraints active, and then as you refine the solution, remove the constraints that are not being actively used. At the end of each run, check to make sure that the solution still satisfies the constraints.


The lower part of Figure \ref{syntax:vlmini} shows constraints generated through bundling and type inference.
Dependency constraints comprise \emph{variable dependencies} and \emph{label dependencies} in addition to propositional formulae.
Variable dependencies $\alpha \sqsubseteq \alpha'$ require that if a version label for $\alpha'$ expects a specific version for a module, then $\alpha$ also expects the same version.
Similarly, label dependencies $\alpha \preceq \cs{\overline{M_i = V_i}}$ require that a version label expected for $\alpha$ must be $V_i$ for $M_i$. For example, assuming that versions $1.0.0$ and $2.0.0$ exist for both modules \mn{A} and \mn{B}, the minimal upper bound set of version labels satisfying $\alpha \preceq \cs{\mn{A}\mapsto 1.0.0}$ is $\alpha = \{\{\mn{A}=1.0.0,\mn{B}=1.0.0\},\{\mn{A}=1.0.0,\mn{B}=2.0.0\}\}$. If the constraint resolution is successful, $\alpha$ will be specialized with either of two labels.
$\Theta$ is a set of type equations resolved by the type unification.

\subsection{Girard's Translation for \vlmini{}}
\label{sec:VLMini}
We extend Girard's translation between \mylang{} (lambda calculus) to \vlmini{} following Orchard's approach~\cite{Orchard:2019:Granule}.
\begin{align*}
\llbracket n \rrbracket \equiv n \qquad
\llbracket x \rrbracket \equiv x \qquad
\llbracket \lam{x}{t} \rrbracket \equiv \lam{[x]}{\llbracket t \rrbracket} \qquad
\llbracket t\ s\rrbracket \equiv \llbracket t\rrbracket\ [ \llbracket s \rrbracket ]
\end{align*}

The translation replaces lambda abstractions and function applications of \mylang{} by lambda abstraction with promoted pattern and promotion of \vlmini{}, respectively. From the aspect of types, this translation replaces all occurrences of $\ftype{A}{B}$ with $\ftype{\vertype{r}{A}}{B}$ with a version resource $r$.
This translation inserts a syntactic annotation $[*]$ at each location where a version resource needs to be addressed. Subsequent type inference will compute the resource at the specified location and produce constraints to ensure version consistency at that point.

The original Girard's translation~\cite{girardlinear1987} is well-known as a translation between the simply-typed $\lambda $-calculus and an intuitionistic linear calculus. The approach involves replacing every intuitionistic arrow $A \rightarrow B$ with $!A \multimap B$, and subsequently unboxing via let-in abstraction and promoting during application \cite{Orchard:2019:Granule}.













\subsection{Bundling}
\label{sec:bundling}
Bundling produces an interface encompassing types and versions from every module version, allowing top-level symbols to act as multi-version expressions. During this process, bundling reviews interfaces from across module versions, identifies symbols with the same names and types after removing $\square_r$ using Girard's transformation, and treats them as multiple versions of a singular symbol (also discussed in Section \ref{sec:typeincompatibilities}). In a constraint-based approach, bundling integrates label dependencies derived from module versions, ensuring they align with the version information in the typing rule for versioned records of \corelang{}.

For example, assuming that the $\mathit{id}$ that takes an \inttype{} value as an argument is available in version 1.0.0 and 2.0.0 of \mn{M} as follows:
\begin{align*}
\mathit{id} &: \vertype{\alpha_1}{(\ftype{\vertype{\alpha_2}{\inttype}}{\inttype})}\ |\ \mathcal{C}_1 \tag{\textnormal{version 1.0.0}}\\
\mathit{id} &: \vertype{\beta_1}{(\ftype{\vertype{\beta_2}{\inttype}}{\inttype})}\ |\ \mathcal{C}_2 \tag{\textnormal{version 2.0.0}}
\end{align*}
where $\alpha_1$ and $\alpha_2$ are version resource variables given from type inference. They capture the version resources of $\mathit{id}$ and its argument value in version 1.0.0. $\mathcal{C}_1$ is the constraints that resource variables of version 1.0.0 will satisfy. Likewise for $\beta_1$, $\beta_2$, and $\mathcal{C}_2$.
Since the types of $\mathit{id}$ in both versions become $\ftype{\inttype}{\inttype}$ via Girard's translation, they can be bundled as follows:
\begin{align*}
\mathit{id} : \vertype{\gamma_1}{(\ftype{\vertype{\gamma_2}{\textsf{Int}}}{\textsf{Int}})}\ |\
\mathcal{C}_1 \land \mathcal{C}_2 \land \Bigl(\ 
     &(\gamma_1 \preceq \cs{\mn{M} = 1.0.0} \land \gamma_1 \preceq \alpha_1 \land \gamma_2 \preceq \alpha_2)\\
\lor\ &(\gamma_1 \preceq \cs{\mn{M} = 2.0.0} \land \gamma_1 \preceq \beta_1 \land \gamma_2 \preceq \beta_2)\ \Bigr)
\end{align*}
where $\gamma_1$ and $\gamma_2$ are introduced by this conversion for the bundled $id$ interface, with label and variable dependencies that they will satisfy.
$\gamma_1$ captures the version resource of the bundled $\mathit{id}$. The generated label dependencies $\gamma_1 \preceq \cs{\mn{M} = 1.0.0}$ and $\gamma_1 \preceq \cs{\mn{M} = 2.0.0}$ indicate that $\mathit{id}$ is available in either version 1.0.0 or 2.0.0 of \mn{M}.
These label dependencies are exclusively\footnote{In the type checking rules for $\verof{l}{t}$, type inference exceptionally generates label dependencies. Please see Appendix \ref{appendix:vlmini_version_control_terms}} generated during bundling.
The other new variable dependencies indicate that the $\mathit{id}$ bundled interface depends on one of the two version interfaces. The dependency is made apparent by pairing the new resource variables with their respective version resource variable for each version.
These constraints are retained globally, and the type definition of the bundled interface is used for type-checking modules importing $\mathit{id}$. 

