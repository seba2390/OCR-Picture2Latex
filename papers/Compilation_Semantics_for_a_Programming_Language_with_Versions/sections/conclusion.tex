% Section Template

\section{Related Work, Future Work, and Conclusion} % Main Section title
\label{conclusion} % Change X to a consecutive number; for referencing this Section elsewhere, use \ref{SectionX}

% \vspace{-.5\baselineskip}
\subsubsection{Managing Dependency Hell}
Mainstream techniques for addressing dependency hell stand in stark contrast to our approach, which seeks to manage dependencies at a finer granularity.
\emph{Container}~\cite{container:docker} encapsulates each application with all its dependencies in an isolated environment, a container, facilitating multiple library versions to coexist on one physical machine. However, it does not handle internal dependencies within the container.
\emph{Monorepository}~\cite{monorepo:google,monorepo:facebook} versions logically distinct libraries within a single repository, allowing updates across multiple libraries with one commit. It eases testing and bug finding but can lower the system modularity.

% \vspace{-.5\baselineskip}
\subsubsection{Toward a Language Considering Compatibility}
The next step in this research is to embed compatibility tracking within the language system. The current \mylang{} considers different version labels incompatible unless a programmer uses \texttt{\vlkey{unversion}}. Since many updates maintain backward compatibility and change only minor parts of the previous version, the existing type system is overly restrictive.

To illustrate, consider Figure \ref{fig2-3} again with more version history. The module \mn{Hash} uses the MD5 algorithm for \fn{mkHash} and \fn{match} in the 1.x.x series. However, it adopts the SHA-3 algorithm in version 2.0.0, leaving other functions the same. The hash by \fn{mkHash} version 1.0.1 (an MD5 hash) aligns with any MD5 hash from the 1.x.x series. Therefore, we know that comparing the hash using \fn{match} version 1.0.0 is appropriate. However, the current \mylang{} compiler lacks mechanisms to express such compatibility in constraint resolution. The workaround involves using \texttt{\vlkey{unversion}}, risking an MD5 hash's use with \fn{match} version 2.0.0.

One promising approach to convey compatibilities is integrating semantic versioning~\cite{preston2013semantic} into the type system.
If we introduce semantics into version labels, the hash generated in version 1.0.1 is backward compatible with version 1.0.0. Thus, by constructing a type system that respects explicitly defined version compatibilities, we can improve \mylang{} to accept a broader range of programs.

It is important to get reliable versions to achieve this goal.
Lam et al.~\cite{10.1145/3426428.3426922} emphasize the need for tool support to manage package modifications and the importance of analyzing compatibility through program analysis.
\emph{Delta-oriented programming}~\cite{10.1145/1868688.1868696,10.5555/1885639.1885647,10.1145/1960275.1960283} could complement this approach by facilitating the way modularizing addition, overriding, and removal of programming elements and include application conditions for those modifications.
This could result in a sophisticated package system that provides granular compatibility information.

Such a language could be an alternative to existing technologies for automatic update, collectively known as \emph{adoptation}.
These methods generate replacement rules based on structural similarities~\cite{Cossette2014,5970177} and extract API replacement patterns from migrated code bases \cite{10.1145/1368088.1368153}.
Some techniques involve library maintainers recording refactorings~\cite{10.1002/smr.328,1553570} and providing annotations~\cite{565039} to describe how to update client code. However, the reported success rate of these techniques is less than 20\% on average~\cite{10.1145/2393596.2393661}.

% \vspace{-.5\baselineskip}
\subsubsection{Supporting Type Incompatibility}
\label{sec:typeincompatibilities}
One of the apparent problems with the current \mylang{} does not support \emph{type incompatibilities}.
\mylang{} forces terms of different versions to have the same type, both on the theoretical (typing rules in \corelang{}) and implementation (bundling in \vlmini{}) aspects. Supporting type incompatibility is important because type incompatibility is one of the top reasons for error-causing incompatibilities \cite{RAEMAEKERS2017140}.
The current \mylang{} is designed in such a way because it retains the principle that equates the types of promotions and versioned records in \corelang{}, easing the formalization of the semantics.

A promising approach to address this could be to decouple version inference from type inference and develop a version inference system on the polymorphic record calculus \cite{10.1145/218570.218572}.
The idea stems from the fact that versioned types $\vertype{\{l_1,l_2\}}{A}$ are structurally similar to record types $\{ l_1 : A,\, l_2 : A\}$ of $\Lambda^{\forall,\bullet}$. 
Since $\Lambda^{\forall,\bullet}$ allows different record-element types for different labels and has concrete inference algorithms with polymorphism, implementing version inference on top of $\Lambda^{\forall,\bullet}$ would also make \mylang{} more expressive.


% \vspace{-\baselineskip}
\subsubsection{Adequate Version Polymorphism}
\label{sec:fullversionpolymorphism}
In the current \mylang{}, there is an issue that the version label of top-level symbols in imported modules must be specified one, whereas users can select specific versions of external variables using \texttt{\vlkey{unversion}} within the importing module.
Consider using a generic function like \fn{List.concat} in Figure \ref{fig6-5}. If it is used in one part of the program within the context of \mn{Matrix} version 1.0.0, the solution of the resource variable of \fn{List.concat} version 1.0.0 becomes confined to $\{\mn{Matrix}=1.0.0,\mn{List}=\ldots\}$. As a result, it is impossible to utilize \fn{List.concat} version 1.0.0 with \mn{Matrix} version 2.0.0 elsewhere in the program. This problem becomes apparent when we define a generic module like a standard library.

It is necessary to introduce full-version polymorphism in the core calculus instead of duplication to address this problem.
The idea is to generate a type scheme by solving constraints for each module during bundling and instantiate each type and resource variable at each occurrence of an external variable.
Such resource polymorphism is similar to that already implemented in Gr~\cite{Orchard:2019:Granule}. However, unlike Gr, \vlmini{} provides a type inference algorithm that collects constraints on a per-module basis, so we need the well-defined form of the principal type. This extension is future work.

% \vspace{-\baselineskip}
\subsubsection{Conclusion}
This paper proposes a method for dependency analysis and version control at the expression level by incorporating versions into language semantics, which were previously only identifiers of packages.
This enables the simultaneous use of multiple versions and identifies programs violating version consistency at the expression level, which is impossible with conventional languages.

Our next step is to extend the version label, which currently only identifies versions, to \emph{semantic versions} and to treat the notion of compatibility with language semantics.
Like automatic updates by modern build tools based on semantic versioning, it would be possible to achieve incremental updates, which would be done step-by-step at the expression level.
Working with existing package managers to collect compatibility information at the expression level would be more feasible to realize the goal.
