\begin{figure}[tb]
\centering
\begin{tikzpicture}[thick]
    \tikzset{vecArrow/.style={thick, decoration={markings,mark=at position
    1 with {\arrow[semithick]{open triangle 60}}},
    double distance=1.4pt, shorten >= 5.5pt,
    preaction = {decorate},
    postaction = {draw,line width=1.4pt, white,shorten >= 4.5pt}}};
    \tikzset{innerWhite/.style={semithick, white,line width=1.4pt, shorten >= 4.5pt}};
    \tikzset{Package/.style={rectangle, fill=cyan!10, text centered, rounded corners, minimum width=1.9cm, minimum height=0.65cm}};
    \tikzset{App/.style={rectangle, text centered, minimum width=1.2cm, minimum height=0.65cm}};

    \newlength\mywidth
    \newlength\myheight
    \newlength\tempdimen
    
    \newcommand\getdimensions[1]{
        \pgfextractx{\mywidth}{\pgfpointanchor{#1}{east}}
        \pgfextractx{\tempdimen}{\pgfpointanchor{#1}{west}}
        \addtolength{\mywidth}{-\tempdimen}
        \pgfextracty{\myheight}{\pgfpointanchor{#1}{north}}
        \pgfextracty{\tempdimen}{\pgfpointanchor{#1}{south}}
        \addtolength{\myheight}{-\tempdimen}
    }

    \tikzset{
      pics/stacked rectangles/.style n args={4}{
        code={
          \def\rectangleWidth{2cm}  % 長方形の幅を固定
          \def\rectangleHeight{1cm} % 長方形の高さを固定
          \pgfmathsetmacro\offsetY{(#1+1)*#2/2} % Yのオフセットを計算
          
          % 一時的なrect1を描画してサイズを取得
          \node[draw=none, align=center] (tempRect1) at (1*#2,1*#2-\offsetY) { #3 };
          
          % ここでtempRect1の寸法を取得
          \getdimensions{tempRect1}
          
          % 長方形を描画
          \foreach \i in {#1,...,2} {
            \node[draw, fill=white, minimum width=\mywidth, minimum height=\myheight] (rect\i) at (\i*#2,\i*#2-\offsetY) {};
          }

          % 一番前の長方形を描画
          \node[draw, fill=white, minimum width=\mywidth, minimum height=\myheight, align=center] (rect1) at (1*#2,1*#2-\offsetY) { #3 };
    
          % すべての長方形を囲む仮想的なノード
          \node[draw=none, inner sep=0pt, fit={(rect1.south west) (rect#1.north east)}, name=#4] {};
        }
      }
    }

    \tikzset{
      pics/stackedty rectangles/.style n args={4}{
        code={
          \def\rectangleWidth{2cm}  % 長方形の幅を固定
          \def\rectangleHeight{1cm} % 長方形の高さを固定
          \pgfmathsetmacro\offsetY{(#1+1)*#2/2} % Yのオフセットを計算
          
          % 一時的なrect1を描画してサイズを取得
          \node[draw=none, align=center] (tempRect1) at (1*#2,1*#2-\offsetY) { #3 };
          
          % ここでtempRect1の寸法を取得
          \getdimensions{tempRect1}
          
          % 長方形を描画
          \foreach \i in {#1,...,2} {
            \node[draw, fill=white, minimum width=\mywidth, minimum height=\myheight, rounded corners] (rect\i) at (\i*#2,\i*#2-\offsetY) {};
          }

          % 一番前の長方形を描画
          \node[draw, fill=white, minimum width=\mywidth, minimum height=\myheight, align=center, rounded corners] (rect1) at (1*#2,1*#2-\offsetY) { #3 };
    
          % すべての長方形を囲む仮想的なノード
          \node[draw=none, inner sep=0pt, fit={(rect1.south west) (rect#1.north east)}, name=#4] {};
        }
      }
    }


    \tikzset{
      pics/stackedtybundled rectangles/.style n args={4}{
        code={
          \def\rectangleWidth{2cm}  % 長方形の幅を固定
          \def\rectangleHeight{1cm} % 長方形の高さを固定
          \pgfmathsetmacro\offsetY{(#1+1)*#2/2} % Yのオフセットを計算
          
          % 一時的なrect1を描画してサイズを取得
          \node[draw=none, align=center] (tempRect1) at (1*#2,1*#2-\offsetY) { #3 };
          
          % ここでtempRect1の寸法を取得
          \getdimensions{tempRect1}
          
          % 長方形を描画
          \foreach \i in {#1,...,2} {
            \node[draw, fill=white, minimum width=\mywidth, minimum height=\myheight, rounded corners, line width=2pt] (rect\i) at (\i*#2,\i*#2-\offsetY) {};
          }

          % 一番前の長方形を描画
          \node[draw, fill=white, minimum width=\mywidth, minimum height=\myheight, align=center, rounded corners, line width=2pt] (rect1) at (1*#2,1*#2-\offsetY) { #3 };
    
          % すべての長方形を囲む仮想的なノード
          \node[draw=none, inner sep=0pt, fit={(rect1.south west) (rect#1.north east)}, name=#4] {};
        }
      }
    }

    \pic at (0,0) {stacked rectangles={4}{0.1cm}{\mylang{}\\program}{vl1}};
    % \node[draw,App, align=center] at (3.0,0) (vlmini1) {\vlmini{}\\($V_{M_i}$)};
    \pic at (3.3,0) {stacked rectangles={4}{0.1cm}{\vlmini{}\\program}{vlmini1}};
    % \node[draw,App, align=center] at (6.0,0) (vlmini2) {\vlmini{}\\interface};
    \pic at (6.5,0) {stackedty rectangles={4}{0.1cm}{\vlmini{}\\interface}{vlmini2}};
    \pic at (9.7,1.4) {stackedtybundled rectangles={3}{0.12cm}{\vphantom{b}\hspace{3em}}{vlmini4}};
    \node[draw,App, align=center, rounded corners, line width=2pt] at (10,0) (vlmini3) {\vlmini{}\\interface\\(bundled)};
    % \node[draw,App, align=center, rounded corners, line width=2pt, minimum width=1.62cm, minimum height=0.5cm] at (10,1.4) (vlmini4) {};
    \coordinate (topofinference) at ($(vlmini1.east)!0.5!(vlmini2.west)$);
    \coordinate (botofinference) at ($(vlmini1.east)!0.5!(vlmini2.west) + (0, -1.75)$);
    \coordinate (topofbundling) at ($(vlmini2.east)!0.5!(vlmini3.west)$);
    \coordinate (botofbundling) at ($(vlmini2.east)!0.5!(vlmini3.west) + (0, -1.75)$);
    % \node[draw,App,double,align=center] at (botofinference) (vdep) {\footnotesize{Variable}\\\footnotesize{dependency}};
    % \node[draw,App,double,align=center] at (botofbundling) (ldep) {\footnotesize{Label}\\\footnotesize{dependency}};
    \node[draw,App,double] at (10,-1.5) (constraints) {Constraints};
    \fill (topofinference) circle (2.5pt);
    \fill (topofbundling) circle (2.5pt);
    
    \draw[-latex] (vl1.east) to [yshift=-5]node[midway,below,align=center] {\footnotesize{Girard's}\\\footnotesize{translation}\\\footnotesize{(version-wise)}} (vlmini1.west);
    \draw[-latex] (vlmini1.east) to node[midway,below] {} (vlmini2.west);
    \draw[-latex] (vlmini1.east) to [yshift=-5]node[midway,below,align=center] (inference){\footnotesize{Type}\\\footnotesize{inference}} (vlmini2.west);
    \draw[-latex] (vlmini2.east) to [yshift=-5]node[midway,below,align=center] (bundling) {\footnotesize{Bundling}} (vlmini3.west);
    \draw[-latex] (vlmini4.west) -| node[near start,below,align=center] {\footnotesize{Import modules}} ([yshift=2]topofinference);
    \draw[-latex] (inference.south) |- (constraints.west);
    \draw[-latex] (bundling.south) |- (constraints.west);
    % \draw[-latex] (inference.south) -- (vdep.north);
    % \draw[-latex] (bundling.south) -- (ldep.north);
\end{tikzpicture}
\caption{The translation phases for a single module with multiple versions.}
\label{fig:translationoverview}
\end{figure}