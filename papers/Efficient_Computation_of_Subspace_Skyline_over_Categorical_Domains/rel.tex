\section{Related Work}\label{sec:relWork}

%%%%%%%%%%%%%%%%%%%%%%%%%%%%%%%%%%%%%%%%%%%%%%%%%%%%%%%%%%%%%%%%%%%%%%%%%%%%%%%%
%\begin{table}[!tb]
%\centering
%\caption{Taxonomy of Skyline Algorithms}\label{tab:taxonomySkylineAlgorithms}
%\begin{tabular}{p{1.4cm}|p{1.6cm}|p{1.8cm}|p{1.8cm}}
%    
%    \hline 
%    \multicolumn{3}{c|}{Fixed Attribute Skyline} & Subspace Skyline\\
%    \hline
%    
%    Sorting-based & \multicolumn{2}{c|}{Partition-based} & \multirow{3}{1.8cm}{Skycube~\cite{yuan2005efficient}, Skyey~\cite{pei2005catching}, Subsky~\cite{tao2006subsky}, CSC~\cite{xia2012online}, FMC~\cite{maabout2016skycube}} \\ \cline{1-3}
%    
%    \multirow{2}{1.4cm}{BNL~\cite{borzsony2001skyline}, SFS~\cite{chomicki2003skyline}, LESS~\cite{godfrey2005maximal}, SaLSa~\cite{bartolini2008efficient}} & Index & No Index & \\ \cline{2-3}
%    
%     & NN~\cite{kossmann2002shooting}, BBS~\cite{papadias2003optimal}, ZSearch~\cite{lee2007approaching} & D\&C~\cite{borzsony2001skyline}, OSPS~\cite{zhang2009scalable}, BSkyTree~\cite{lee2014scalable}, LS~\cite{morse2007efficient} & \\
%    \hline
%\end{tabular}
%\end{table}
%%%%%%%%%%%%%%%%%%%%%%%%%%%%%%%%%%%%%%%%%%%%%%%%%%%%%%%%%%%%%%%%%%%%%%%%%%%%%%%%

In the database context, the skyline operator was first introduced in \cite{borzsony2001skyline}. Since then much work aims to improve the performance of skyline computation in different scenarios. In this paper, we consider skyline algorithms designed for centralized database systems. 

To the best of our knowledge, LS~\cite{morse2007efficient} and Hexagon~\cite{preisinger2007hexagon} are the only two algorithms designed to compute skylines over categorical attributes. Both algorithms operate by first creating the complete lattice of possible attribute-value combinations. Using the lattice structure, non-skyline tuples are then discarded. Even though LS and Hexagon can discover the skylines in linear time, the requirement to construct the entire lattice for each skyline is strict and not scalable. The size of the lattice is exponential in the number of attributes in a skyline query. Moreover, in order to discover the skylines, the algorithms have to scan the entire dataset twice, which is not ideal for online applications.

Most of the existing work on skyline computation concerns relations with {\em numeric attributes}. Broadly speaking, skyline algorithms for numerical attributes can be categorized as follows. {\em Sorting-based Algorithms} utilize sorting to improve the performance of skyline computation aiming to discard nonskyline objects using a small number of dominance checks~\cite{chomicki2005skyline}~\cite{godfrey2005maximal}. For any subspace skyline query, such approaches will require sorting the dataset. SaLSa~\cite{bartolini2008efficient} is the best in this category and we demonstrated how our adaptation on categorical domains, namely ST-S outperforms SaLSa. 

{\em Partition-based Algorithms} recursively partition the dataset into a set of disjoint regions, compute local skylines for each region and merge the results 
\cite{borzsony2001skyline}~\cite{zhang2009scalable}. Among these, BSkyTree~\cite{lee2014scalable} has been shown to be the best performer. We
demonstrated that our adaptation of this algorithm, namely ST-P, for categorical domains outperforms the vanilla BSkyTree when applied to our application scenario. Other partitioning algorithms, such as  NN~\cite{kossmann2002shooting}, BBS~\cite{papadias2003optimal} 
and ZSearch~\cite{lee2007approaching} utilize indexing structures such as R-tree, ZB-tree for efficient region level dominance tests. However,
adaptations of such algorithms in the subspace skyline problem would incur exponential space overhead which is not in line with the scope
of our work (at most linear to the number of attributes overhead). 

A body of work is also devoted to {\em Subspace Skyline Algorithms}~\cite{yuan2005efficient, pei2005catching} which utilize pre-computation to
compute skylines for each subspace skyline query. These algorithms impose exponential space overhead, however. Further improvements
to reduce the overhead ~\cite{tao2006subsky}~\cite{xia2006refreshing}~\cite{xia2012online}~\cite{maabout2016skycube} are highly data
dependent and offer no guarantees for their storage requirements.