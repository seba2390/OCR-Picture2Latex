\section{Subspace Skyline using Sorted \\ Lists} \label{sec:subsky}

In this section, we consider the availability of sorted lists $L_1, L_2, \ldots L_m$, as per \S\ref{sec:preliminaries} and utilize them to design efficient algorithms for subspace skyline discovery.
We first briefly discuss a baseline approach that is an extension of LS~\cite{morse2007efficient}.
Then in \S\ref{sec:topdown}, we overcome the barriers of the baseline approach proposing an algorithm named {\bf TOP-DOWN}. The algorithm applies a top-down on-the-fly parsing of the subspace lattice and prunes the dominated branches.
However, the expected cost of TOP-DOWN {\em exponentially} depends on the value of $m$  (Appendix~\ref{ap:top-down}).
We then propose {\bf TA-SKY} (Threshold Algorithm for Skyline) in \S\ref{sec:TASky} that does not have such a dependency. In addition to the sorted lists, TA-SKY also utilizes the ST algorithm proposed in \S\ref{sec:3} for computing skylines.

\begin{table}[!t]
\centering
\caption{Example: Input Table}\label{tab:runningExampleSubspaceSkyline}
\begin{tiny}
\begin{tabular}{cccccc}
    \hline 
     & $A_1$ & $A_2$ & $A_3$ & $A_4$ & $A_5$ \\
    \hline 
    $t_1$ & 0 & 1 & 0 & 1 & 1\\
    \hline
    $t_2$ & 0 & 0 & 1 & 1 & 0\\
    \hline
    $t_3$ & 0 & 0 & 1 & 0 & 1\\
    \hline
    $t_4$ & 0 & 0 & 0 & 1 & 1\\
    \hline
    $t_5$ & 1 & 0 & 1 & 1 & 1\\
    \hline
    $t_6$ & 1 & 1 & 1 & 0 & 0\\
    \hline
\end{tabular}
\end{tiny}
\end{table}


\begin{figure}[!ht]
  \begin{minipage}[t]{0.49\linewidth}
    \centering
    \includegraphics[scale=1.2]{figures/sortedLists.pdf}
    \caption{Example: Sorted Lists, Organization 1} 
    \label{fig:sortedLists}
  \end{minipage}
  \hspace{1mm}
  \begin{minipage}[t]{0.49\linewidth}
    \centering
    \includegraphics[scale=1.2]{figures/sortedListsOptimized.pdf}
    \caption{Example: Sorted Lists, Organization 2}
    \label{fig:sortedListsOptimized}
  \end{minipage}
\end{figure}

\begin{exmp}\label{exmp:subspaceSkyline}
Let $\mathcal{Q} \subseteq \mathcal{A}$ denotes the set of attributes in a subspace skyline query and $D_{\mathcal{Q}}$ be the projection of $D$ in $\mathcal{Q}$. We denote the set of sorted lists corresponding to a query (one for each attribute involved in the query) as $\mathcal{L_Q}$, $\mathcal{L_Q} = \{ L_i | A_i \in \mathcal{Q} \}$. Also, let $m' \leq m$ be $|\mathcal{Q}|$. Our running example uses the relation shown in Table~\ref{tab:runningExampleSubspaceSkyline} through out this section. There are a
total of  $n=6$ tuples, each having $m=5$ attributes. Consider a subspace skyline query $\mathcal{Q} = \{A_1, A_2, A_3, A_4\}$, thus, $m' = 4$. Figure~\ref{fig:sortedLists} shows the corresponding sorted lists $\mathcal{L_Q} = \{L_1, L_2, L_3, L_4 \}$.
\end{exmp}

\vspace{1mm}
\noindent{\bf BASELINE:} We use sorted lists in $\mathcal{L_Q}$ to construct the projection of each tuple $t \in D$ in the query space. For this, we shall perform $n$ sequential accesses on sorted list $L_1 \in \mathcal{L_Q}$. For each $(tupleID, value)$ pair returned by sequential access, we create a new tuple $t_{new}$. $t_{new}$ has $tupleID$ as its id and $t_{new}[A_1] = value$. The remaining attribute values of $t_{new}$ are set by performing random access on sorted list $L_j$ ($\forall j \in [2,m']$). After computing the projections of all tuples in query space, we create a lattice over $\mathcal{Q}$ and 
run the LS algorithm to discover the subspace skyline.

\vspace{3mm}
\noindent We identify the following problems with BASELINE:
\begin{itemize}
\itemsep0em
\item It makes two passes over all the tuples in the relation.
\item It requires the construction of the complete lattice of size $|Dom(\mathcal{Q})|$. For example, when $Dom(A_i) = 4$ and $m'=15$, the lattice has more than {\em one billion} nodes; yet the algorithm needs to map the tuples into the lattice.
\end{itemize}

One observation is that for relations with categorical attributes, especially when $m'$ is relatively small, skyline tuples are more likely to be discovered at the upper levels of the lattice. This motivated us to seek alternate approaches.
Unlike BASELINE, TOP-DOWN and the TA-SKY algorithm are designed in a way that they are capable of answering subspace skyline queries by traversing a small portion of the lattice, and more importantly {\em without the need to access the entire relation}.

\subsection{TOP-DOWN}\label{sec:topdown}

\noindent{\bf Key Idea:} Given a subspace skyline query $\mathcal{Q}$, we create a lattice capturing the dominance relationships among the tuples in $D_{\mathcal{Q}}$. Each node in the lattice represents a specific attribute value combination in query space, hence, corresponds to a potential tuple in $D_{\mathcal{Q}}$. For a given lattice node $u$, if there exist tuples in $D_{\mathcal{Q}}$ with attribute value combination same as $u$, then all tuples in $D_{\mathcal{Q}}$ corresponding to nodes dominated by $u$ in the lattice are also dominated. TOP-DOWN utilizes this observation to compute skylines for a given subspace skyline query. Instead of iterating over the tuples, TOP-DOWN traverses the lattice nodes from top to bottom; it utilizes sorted lists $\mathcal{L_Q}$ to search for tuples with specific attribute value combinations. When $|\mathcal{Q}|$ is relatively small, it is likely one will discover all the skyline tuples just by checking few attribute value combinations, without considering the rest of the lattice. However, the expected cost of TOP-DOWN increases exponentially as we increase the query  length.
Please refer to Appendix~\ref{ap:top-down} for the details and the limitations of TOP-DOWN.

\subsection{TA-SKY}\label{sec:TASky}
We now propose our second algorithm, Threshold Algorithm for Skyline (TA-SKY) in order to answer subspace skyline queries. Unlike TOP-DOWN that exponentially depends on $m$, as we shall show in \S\ref{sec:TASKY-performance}, TA-SKY has a worst case time complexity of $O(m'n^2)$; in addition, we shall also study the expected cost of TA-SKY.
The main innovation in TA-SKY is that it follows the style of the well-known Threshold Algorithm (TA)~\cite{fagin2003optimal} for Top-$k$ query processing, except that it is used for solving a skyline problem rather than a Top-$k$ problem. 

TA-SKY iterates over the sorted lists $\mathcal{L_Q}$ until a stopping condition is satisfied. At each iteration, we perform $m'$ parallel sorted access, one for each sorted list in $\mathcal{L_Q}$. Let $cv_{ij}$ denote the current value returned from sorted access on list $L_j \in \mathcal{L_Q}$ $(1 \leq j \leq m')$ at iteration $i$. Consider $\tau_i$ be the set of values returned at iteration $i$, $\tau_i = \{cv_{i1}, cv_{i1}, \ldots, cv_{im'}\}$. We create a synthetic tuple $t_{syn}$ as the \textit{threshold value} to establish a stopping condition for TA-SKY. The attribute values of synthetic tuple $t_{syn}$ are set according to the current values returned by each sorted list. Specifically, at iteration $i$, $t_{syn}[A_j] = cv_{ij}, \forall j \in [1, m']$. In other words, $t_{syn}$ corresponds to a potential tuple with the highest possible attribute values that has not
been seen by TA-SKY yet. 

In addition, TA-SKY also maintains a candidate skyline set. The candidate skyline set materializes the skylines among the tuples seen till the last stopping condition check. We use the tree structure described in \S\ref{sec:ST} to organize the candidate skyline set. Note that instead of checking the stopping condition at each iteration, TA-SKY considers the stopping condition at iteration $i$ only when $\tau_i \neq \tau_{i-1}$ $(2 \leq i \leq n)$.  $\tau_i \neq \tau_{i-1}$ if and only if $cv_{(i-1)j} \neq cv_{ij}$ $(1 \leq j \leq m')$ for at least one of the $m'$ sequential accesses. This is because the stopping condition does not change among iterations that have the same $\tau$ value. Let us assume the value of $\tau$ changes at the current iteration $i$ and the stopping condition was last checked at iteration $i'$ ($i' < i)$. Let $\mathcal{T}$ be the set of tuples that are returned in, at least one of the sequential accesses between iteration $i'$ and $i$. For each tuple $t \in \mathcal{T}$, we perform random access in order to retrieve the values of missing attributes (i.e., attributes of $t_\mathcal{Q}$ for which we do not know the values yet). Once the tuples in $\mathcal{T}$ are fully constructed, TA-SKY compares them against the tuples in the candidate skyline set. For each tuple $t \in \mathcal{T}$  three scenarios can arise:
\begin{enumerate}
    \itemsep0em
    \item $t$ dominates a tuple $t'$ in the tree (i.e., candidate skyline set), $t'$ is deleted from the tree.
    \item $t$ is dominated by a tuple $t'$ in the tree, it is discarded since it cannot be skyline.
    \item $t$ is not dominated by any tuple $t'$ in the tree, it is inserted in the tree.
\end{enumerate}

Once the candidate skyline set is updated with tuples in $\mathcal{T}$, we compare $t_{syn}$ with the tuples in the candidate skyline set. The algorithm stops when $t_{syn}$ is dominated by any tuple in the candidate skyline set.

We shall now explain TA-SKY for the subspace skyline query $\mathcal{Q}$ of Example~\ref{exmp:subspaceSkyline}. Sorted lists $\mathcal{L_Q}$ corresponding 
to query $\mathcal{Q}$ are shown in Figure~\ref{fig:sortedLists}. At iteration 1, TA-SKY retrieves the tuples $t_1, t_2$ and $ t_5$ by sequential access. For $t_1$ we know its value on attributes $A_2$ and $A_4$ whereas for $t_2$ and $t_5$ we know their value on $A_3$ and $A_1$ respectively. At this position we have $\mathcal{T} = \{ t_1, t_2, t_5 \}$ and $\tau_1 = \{1, 1, 1, 1\}$. Note that in addition to storing the tupleIDs that we have seen so far, we also keep track of the attribute values that are known from sequential access. After iteration 2,  $\mathcal{T} = \{ t_1, t_2, t_3, t_5, t_6\}$ and $\tau_2 = \{1, 1, 1, 1\}$. At iteration 3 we retrieve the values of $t_1, t_2, t_5$ and $ t_4$ on attributes $A_1, A_2, A_3,$ and $A_4$ respectively and update the corresponding entries $\mathcal{T}$.  Since $\tau_3 = \{0, 0, 1, 1\}$ is different from $\tau_2$, TA-SKY checks the stopping condition. First, we get the missing attribute values (attribute values which are not known from sequential access) of each tuple $t \in \mathcal{T}$. This is done performing random access on the appropriate sorted list in $\mathcal{L_Q}$. After all the tuples in $\mathcal{T}$ are fully constructed, we update the candidate skyline set using them. The final candidate skyline set is constructed after considering all the tuples in $\mathcal{T}$ is $\{t_1, t_5, t_6 \}$. Since the synthetic tuple $t_{syn} = \langle 0, 0, 1, 1 \rangle$ corresponds to $\tau_3$ is dominated by the candidate skyline set, we stop scanning the sorted lists and output the tuples in the candidate skyline set as the skyline answer set.

The number of tuples inserted into $\mathcal{T}$ (i.e., partially retrieved by sequential accesses) before the stopping condition is satisfied, impacts the performance of TA-SKY. This is because for each tuple $t \in \mathcal{T}$, we have to first perform random accesses in order to get the missing attribute values of $t$ and then compare $t$ with the tuples in the candidate skyline set in order to check if $t$ is skyline. Both the number of random accesses and number of dominance tests increase the execution time of TA-SKY. Hence, it is desirable to have a small number of entries in $\mathcal{T}$.  We noticed that the number of tuples inserted in $\mathcal{T}$ by TA-SKY depends on the organization of \textit{(tupleID, value)} pairs (i.e., ordering of pairs having same $value$) in sorted lists. Figure~\ref{fig:sortedListsOptimized} displays sorted lists $\mathcal{L'_Q}$ for the same relation in Example~\ref{exmp:subspaceSkyline} but with different organization. Both with $\mathcal{L_Q}$ and $\mathcal{L'_Q}$ TA-SKY stops at iteration 3. However, For $\mathcal{L_Q}$ after iteration 3, $\mathcal{T} = \{t_1, t_2, t_3, t_4, t_5, t_6\}$ and we need to make a total of 12 random accesses and 12 dominance tests\footnote{For each tuple $t \in \mathcal{T}$, we need to perform two dominance checks: i) if $t$ is dominating any tuple in the candidate skyline set and ii) if $t$ is dominated by tuples in the candidate skyline set.}. On the other hand, with $\mathcal{L'_Q}$, after iteration 3 we have $\mathcal{T} = \{t_1, t_2, t_5, t_6\}$, requiring only 4 random accesses and 8 dominance tests.



One possible approach to improve the performance of TA-SKY is to re-organize the sorted lists before running the algorithm for a given subspace skyline query. Specifically, $\forall t, t' \in D$ that $t[A_i] = t'[A_i]$, position $t$ before $t'$ in the sorted list $L_i$ $(1 \leq i \leq m')$ if $t$ has better value than $t'$ on the remaining attributes. However, re-arranging the sorted lists for each subspace skyline query will be costly. %Moreover, this also diminishes the goal of TA-SKY - we want to discover the skylines without the need to access all the tuples and with a small number of dominance checks.

We now propose several optimization techniques that enable TA-SKY to compute skylines without considering all the entries in $\mathcal{T}$.

\vspace{1mm}
\noindent{\bf Selecting appropriate entries in $\mathcal{T}$:} Our goal is to only perform random access and dominance checks for tuples in $\mathcal{T}$ that are likely to be skyline for a given subspace skyline query. Consider a scenario where TA-SKY needs to check the stopping condition at iteration $k$, i.e, $\tau_k \neq \tau_{(k-1)}$. Let $\mathcal{Q'}$ be the set of attributes for which the value returned by sequential access at iteration $k$ is different from $(k-1)$-th iteration, $\mathcal{Q'} = \{A_i | A_i \in \mathcal{Q}, cv_{ki} < cv_{(k-1)i} \}$. In order for the tuple $t_{syn}$ to be dominated, there must exist a tuple $t' \in \mathcal{T}$ that has $t'[A_i] \geq t_{syn}[A_i]$, $\forall A_i \in \mathcal{Q}$ and $\exists A_i \in \mathcal{Q}$ $s.t.$ $t'[A_i] > t_{syn}[A_i]$. Note that each tuple $t \in \mathcal{T}$ has $t[A_i] = t_{syn}[A_i], \forall A_i \in \mathcal{Q} \setminus \mathcal{Q'}$. This is because for all $A_i \in \mathcal{Q} \setminus \mathcal{Q'}$ sorted access returns same value on both $(k-1)$-th and $k$-th iteration (i.e., $cv_{(k-1)i} = cv_{ki}$). Hence, the only way a tuple $t' \in \mathcal{T}$ can dominate $t_{syn}$ is to have a larger value on any of the attributes in $\mathcal{Q'}$. Therefore, we only need to consider a subset of tuples $\mathcal{T'} = \{ t | t \in \mathcal{T}, \exists A_i \in \mathcal{Q} \setminus \mathcal{Q} \text{ s.t. } t[A_i] = cv_{(k-1)i} \}$. Note that it is still possible that $\exists t, t' \in \mathcal{T'} \text{ s.t. } t \succ_{\mathcal{Q}} t'$. Thus, we need to only consider the tuples that are skylines among $\mathcal{T'}$ and the candidate skyline set. To summarize, before checking the stopping condition at iteration $k$, we have to perform the following  operations: (i) Select a subset of tuples $\mathcal{T'}$ from $\mathcal{T}$ that are likely to dominate $t_{syn}$, (ii) For each tuple $t \in \mathcal{T}$ get the missing attribute values of $t$ performing random access on appropriate sorted lists, (iii) Update the candidate skyline set using the skylines in $\mathcal{T'}$, and (iv) Check if $t_{syn}$ is dominated by the updated candidate skyline set.

Note that in addition to reducing the number of random access and dominance test, the above optimization technique makes the TA-SKY algorithm {\em progressive}, i.e, tuples that are inserted into the candidate skyline set will always be skyline in the query space $\mathcal{Q}$.
This characteristic of TA-SKY makes it suitable for real-world web applications where instead of waiting for all the results to be returned users want a subset of the results very quickly.

%We now highlight this optimization technique for the subspace skyline query $\mathcal{Q}$ of Example~\ref{exmp:subspaceSkyline} and the sorted lists in Figure~\ref{fig:sortedLists}. After iteration 3 we have $\mathcal{T} = \{t_1, t_2, t_3, t_4, t_5, t_6\}$ and $\tau_3 = {0, 0, 1, 1}$. Since $\tau_3$ is different from $\tau_2$ on attributes $A_1$ and $A_2$, we only need to consider tuples in $\mathcal{T}$ that have value 1 on $A_1$ and/or $A_2$. Therefore, $\mathcal{T'} = \{ t_1, t_5, t_6 \}$. After obtaining the missing attribute values of tuples in  $\mathcal{T'}$, using random access, it turns out that all of them are skyline in $\mathcal{T'}$. Hence, we update the candidate skyline set using tuples in $\mathcal{T'}$. Finally, since the synthetic tuple $t_{syn} = \langle 0, 0, 1, 1 \rangle$, corresponding to iteration 3, is dominated by the candidate skyline set, we stop the algorithm. Compared to the basic TA-SKY algorithm which requires a total of 12 random accesses and 12 dominance tests, this optimization enables TA-SKY to stop after only 5 random accesses and 6 dominance tests.

\vspace{1mm}
\noindent{\bf Utilizing the ST algorithms:} We can utilize the ST algorithms for discovering the skyline tuples from $\mathcal{T'}$. This way we can take advantages of the optimization approaches proposed in \S\ref{sec:3}. For example, we can call ST-S algorithm with parameter: tree $T$ (stores all the tuples discovered so far) and tuple list $\mathcal{T'}$. The output skyline tuples in  $\mathcal{T'}$ that are not dominated by $T$. Moreover, after sorting the tuples in ST-S, if we identify that $score(t_i) = score(t_{i-1})$ $(2 \leq i \leq |\mathcal{T'}|)$ and $t_{i-1}$ is dominated, we can safely mark $t_i$ as dominated. This is because $score(t_i) = score(t_{i-1})$ implies that both $t_i$ and $t_{i-1}$ have same attribute value assignment. When the number of attributes in a subspace skyline query is small, this approach allows us to skip a large number of dominance tests.


The pseudocode of TA-SKY, after applying the optimizations above, is presented in Algorithm~\ref{alg:taSky}.

\begin{algorithm}[!htb]
\caption{{\bf TA-SKY}}
\begin{algorithmic}[1]
\label{alg:taSky}
\STATE {\bf Input:} Query $\mathcal{Q}$, Sorted lists $\mathcal{L_Q}$; \\ {\bf Output:} $\mathcal{S}_\mathcal{Q}$.
\STATE $T = $ {\it New} Tree(); $\mathcal{T} = \emptyset$
\STATE {\bf repeat}
    \STATE \hindent $\tau = \emptyset$
    \STATE \hindent {\bf for} each sorted list $L_i \in \mathcal{L_Q}$
        \STATE \hindent[2] $A_i$ = Attribute corresponds to $L_i$
        \STATE \hindent[2] $(tupleID, value) = SortedAccess(L)$
        \STATE \hindent[2] $\mathcal{T}[tupleID][A_i] = value$
        \STATE \hindent[2] $\tau[A_i] = value$
    \STATE \hindent {\bf if} $\tau$ remains unchanged from prev. iteration:
        \STATE \hindent[2] {\bf continue;}
    \STATE \hindent $\mathcal{Q'} = \{A_i | A_i \in \mathcal{Q}, \tau[A_i] \text{ changed from prev.iteration}\}$
    \STATE \hindent $\mathcal{T'} = \{t | t \in \mathcal{T}, \exists A_i \in \mathcal{Q'}, \mathcal{T}[t][A_i] \text{ is set} \}$
    \STATE \hindent Delete entries from $\mathcal{T}$ that are inserted in $\mathcal{T'}$
    \STATE \hindent {\bf for} each $t \in \mathcal{T'}$
        \STATE \hindent[2] {\bf for} each attribute $A_i \in Q \setminus Q'$
            \STATE \hindent[3] {\bf if} $t[A_i]$ is missing:
                \STATE \hindent[4] $t[A_i]= RandomAccess(L, A_i)$
        
        \STATE \hindent[2] Update $score$ of $t$
    %\STATE \hindent Sort $\mathcal{T'}$ is descending order of tuple's $score$ value
    %\STATE \hindent {\bf for} each $t \in \mathcal{T'}$
    %    \STATE \hindent[2] {\bf if not} IS-DOMINATED($t$, $T.root$, 1, $score(t)$):
    %        \STATE \hindent[3] INSERT($t$, $T.rootNode$, 1)
    %        \STATE \hindent[3] Output $t$ as skyline tuple.
    \STATE \hindent ST-S($\mathcal{T}$, $\mathcal{Q}$, $T$)
    \STATE \hindent $t_{syn}=$ Synthetic tuple with values of $\tau$
\STATE {\bf until} IS-DOMINATED($t_{syn}$, $T.root$, 1, $score(t_{syn})$)
\end{algorithmic}
\end{algorithm}

\subsubsection{Performance Analysis}\label{sec:TASKY-performance}

\vspace{1mm}
\noindent{\bf Worst Case Analysis:} In the worst case, TA-SKY will exhaust all the $m^\prime$ sorted lists. Hence, will perform $O(m^\prime n)$ sorted and $O(m^\prime n)$ random accesses. After all the tuples are fully constructed, for each tuple $t$, we need to check whether any other tuple in $T$ dominates $t$. The cost of each dominance check operation is $O(m^\prime n)$. Hence, cost of $n$ dominance checks is $O(m^\prime n^2)$. Therefore, the worst case time complexity of TA-SKY is $O(m'n^2)$

\vspace{1mm}
\noindent{\bf Expected Cost Analysis:}

\begin{lemma}\label{lemma:expectedDiscovery}
Considering $p_i$ as the probability that a tuple has value 1 on the binary attribute $A_i$, the expected number of tuples discovered by TA-SKY after $i$ iterations is:
\begin{align}\label{eq:expectedDiscovery}
n P_{seen}(t,i)
\end{align}
where $P_{seen}(t,i)$ is computed using Equation~\ref{eq:seen}.
\begin{align}\label{eq:seen}
\nonumber
P_{seen}(t,i) = 1 - \prod_{j=1}^{m^\prime} \Big( & (1 - p_j) \big(\sum_{k=0}^{i-1}P_{L_j}(k)\frac{n-i}{n-k} + \sum_{k=i}^{n}P_{L_j} \big) \\
                                                  & + p_j\sum_{k=i+1}^n P_{L_j}(k) %\frac{k-i}{k} \Big)
\end{align}
\end{lemma}
Refer to Appendix~\ref{sec:appendixProof} for the proof.



\begin{theorem}\label{thm:expectedCostTA-SKY}
Given a subspace skyline query $\mathcal{Q}$, the expected number of sorted accesses performed by TA-SKY on an $n$ tuple boolean relation with probability of having value $1$ on attribute $A_j$ being $p_j$ is,
\begin{align}
m^\prime \sum_{i=1}^n i\times P_{stop}(i)
\end{align}
where $P_{stop}(i)$ is computed using Equations~\ref{eq:stopi-1},~\ref{eq:stopi-2}, and~\ref{eq:stopi-3}.
\begin{align}\label{eq:stopi-1}
P_{stop}(i) &= \sum_{k=1}^m P_0(i, k) \times {m^\prime \choose k} \times (1 - (1 - P_{stop}(t, \mathcal{Q}_k))^{i^\prime})
\end{align}
\begin{align}\label{eq:stopi-2}
P_0(i, k) = {m^\prime \choose k} \prod_{A_j \in \mathcal{Q}_k} (1 - p_j)^{n-i} \prod_{A_j \in \mathcal{Q} \setminus \mathcal{Q}_k} \big(1 - (1 - p_j)^{n-i}\big)
\end{align}
\begin{align}\label{eq:stopi-3}
P_{stop}(t, \mathcal{Q}_k) &= \underset{\forall A_j \in \mathcal{Q}\backslash \mathcal{Q}_k} {\Pi} p_j (1-\underset{\forall A_j \in \mathcal{Q}_k} {\Pi} (1 - p_j))
\end{align}
\end{theorem}
The proof is available in Appendix~\ref{sec:appendixProof}
