\section{Preliminaries}\label{sec:preliminaries}
Consider a relation $D$ with $n$ tuples and $m+1$ attributes. 
One of the attributes is $tupleID$, which has a unique value for each tuple.
Let the remaining $m$ categorical attributes be $\mathcal{A}=\{A_1,\dots ,A_m\}$. 
Let $Dom(\cdot)$ be a function that returns the domain of one or more attributes. For example, $Dom(A_i)$ represents the
domain of $A_i$, while $Dom(\mathcal{A})$ represents the Cartesian product
of the domains of attributes in $\mathcal{A}$. $|Dom(A_i)|$ represents the cardinality of $Dom(A_i)$. We use $t[A_i]$ to denote the value of $t$ on the attribute $A_i$.  
We also assume that for each attribute, the values in the domain have a total ordering by preference
(we shall  use overloaded notation such as $a > b$ to indicate that value $a$ is preferred over value $b$).
%\textcolor{red}{Gautam: Please make sure the rest of the paper, including pseudocode, uses notation such as $t[tupleID]$ and not $t.tupleID$}

\subsection{Skyline}
We now define the notions of \textit{dominance} %, \textit{incomparability} (\textcolor{red}{\em is this a proper term? maybe non-comparable?})
and \textit{skyline}~\cite{borzsony2001skyline} formally. 

\begin{definition}{(Dominance).}
A tuple $t\in D$ dominates a tuple $t^\prime\in D$, denoted by $t \succ t^\prime$, {\it iff} $\forall A\in\mathcal{A},\, t[A] \geq t^\prime[A]$ and $\exists A \in \mathcal{A}, \, t[A] > t^\prime[A]$.
Moreover, a tuple $t\in D$ is not comparable with a tuple $t^\prime\in D$, denoted by $t \sim t^\prime$, {\it iff} $t \nsucc t^\prime$ and $t^\prime \nsucc t$.
\end{definition}

\begin{definition}{(Skyline).}
Skyline, $\mathcal{S}$, is the set of tuples  that are not dominated by any other tuples in $D$, i.e.: $\mathcal{S} = \{t\in D|\nexists t^\prime \in D \mbox{ s.t. } t^\prime \succ t\}$
\end{definition}

For each tuple $t \in D$, we shall also be interested in computing its $score$ value, denoted by $score(t)$, using a monotonic function $F(\cdot)$. A function $F(\cdot)$ satisfies the monotonicity condition if $F(t) \geq F(t') \Rightarrow t' \nsucc t$.

\vspace{1mm}
\noindent{\bf Subspace Skyline:} Let $\mathcal{Q} \subseteq \mathcal{A}$ be a subset of attributes. The attributes in $\mathcal{Q}$ forms a $|\mathcal{Q}|$-dimensional subspace of $\mathcal{A}$. The projection of a tuple $t \in D$ in subspace $\mathcal{Q}$ is denoted by $t_{\mathcal{Q}}$ where $t_{\mathcal{Q}}[A] = t[A], \, \forall A \in \mathcal{Q}$. Let $D_{\mathcal{Q}}$ be the projection of all tuples of $D$ in subspace $\mathcal{Q}$ . A tuple $t_{\mathcal{Q}} \in D_{\mathcal{Q}}$ dominates another tuple $t^\prime_{\mathcal{Q}} \in D_{\mathcal{Q}}$ in subspace $\mathcal{Q}$ (denoted by $t_{\mathcal{Q}} \succ_{\mathcal{Q}} t^\prime_{\mathcal{Q}}$) if $t^\prime_\mathcal{Q}$ is not preferred to $t$ on any attribute in $\mathcal{Q}$ while $t$ is preferred to $t^\prime$ on least one attribute in $\mathcal{Q}$.

\begin{definition}{(Subspace Skyline).}
Given a subspace $\mathcal{Q}$, the Subspace Skyline, $\mathcal{S_\mathcal{Q}}$, is the set of tuples in $D_{\mathcal{Q}}$ that are not dominated by any other tuples, i.e.: $\mathcal{S_\mathcal{Q}} = \{t_{\mathcal{Q}} \in D_{\mathcal{Q}} | \nexists t^\prime_{\mathcal{Q}} \in D_{\mathcal{Q}} \mbox{ s.t. } t^\prime_{\mathcal{Q}} \succ_{\mathcal{Q}} t_{\mathcal{Q}}\}$
\end{definition}

\subsection{Sorted Lists}
{\em Sorted lists} are popular data structures widely used by many access-based techniques in data management~\cite{fagin1996combining,fagin2003optimal}.
Let $\mathcal{L} = \{ L_1, L_2, \ldots, L_m \}$ be $m$ sorted lists, where $L_i$ corresponds to a (descending) sorted list for attribute $A_i$. All these lists have the same length, $n$ (i.e., one entry for each tuple in the relation). Each entry of $L_i$ is a pair of the form $(tupleID, t[A_i])$. % where $tupleId$ is a unique id corresponding to tuple $t \in D$. Entries in list $L_i$ are sorted in descending order according to $t[A_i]$ (ties are broken arbitrarily). 

%\textcolor{red}{Gautam: the above definition of an inverted index is a problem. An inverted index is associated with a unique attribute value. But in your case, you are associating a list with an entire attribute.}

A sorted list supports two modes of access: (i) \textit{sorted (or sequential) access}, and (ii) \textit{random access}. Each call to \textit{sorted access} returns an entry with the next highest attribute value. Performing \textit{sorted access} $k$ times on list $L_i$ will return the first $k$ entries in the list. In \textit{random access} mode, we can retrieve the attribute value of a specific tuple. A \textit{random access} on list $L_i$ assumes $tupleID$ of a tuple $t$ as input and returns the corresponding attribute value $t[A_i]$. 


\subsection{Problem Definition}
In this paper, we address the efficient computation of {\em subspace skyline} queries over a relation with categorical attributes.
Formally:

\medskip\noindent
 \framebox[\columnwidth]{\parbox{0.9\columnwidth}{ \textsc{Subspace Skyline Discovery:}
Given
a relation $D$ with the set of categorical attributes $\mathcal{A}$ 
and a subset of attributes in the form of a subspace skyline query $\mathcal{Q}\subseteq \mathcal{A}$,
find
the skyline over $\mathcal{Q}$, denoted by $\mathcal{S}_{\mathcal{Q}}$.
}}\\

In answering subspace skyline queries we consider two scenarios: (i) no precomputed indices are available, and (ii) existence of precomputed sorted lists.


%\subsection{Table of Notations}
Table~\ref{tab:notations} lists all the notations that are used throughout the paper (we shall introduce some of these later in the paper).
\begin{table}[!t]
\begin{tiny}
\centering
\caption{Table of notations}\label{tab:notations}
\begin{tabular}{|l|p{6cm}|}
    \hline 
    {\bf Notation} & {\bf Semantics}\\
    \hline
    $D$ & Relation\\
    \hline
    $n$ & Number of tuples in the relation\\
    \hline
    $m$ & Number of attributes\\
    \hline
    $t_1, \ldots, t_n$ & Set of tuples in $D$\\
    \hline
    $\mathcal{A}$ & Set of attributes in $D$\\
    \hline
    $Dom(\cdot)$ & Domain of a set of attributes\\
    \hline
    $score(t)$ & Score of the tuple $t$ computed using a monotonic function $F(\cdot)$\\
    \hline
    $t \succ t^\prime$ & $t$ dominates $t^\prime$\\
    \hline
    $\mathcal{L}$ & Set of $m$ sorted lists\\
    \hline
    $\mathcal{Q}$ & Subspace skyline query\\
    \hline
    $m^\prime$ & Number of attributes in $\mathcal{Q}$\\
    \hline
    $D_{\mathcal{Q}}$ & Projection of $D$ in query space $\mathcal{Q}$\\
    \hline
    $\mathcal{S}_\mathcal{Q}$ & Set of skyline tuples in $D_{\mathcal{Q}}$\\
    \hline
    $t_{\mathcal{Q}}$ & Projection of tuple $t$ in $\mathcal{Q}$\\
    \hline
    $t_{\mathcal{Q}} \succ_{\mathcal{Q}} t^\prime_{\mathcal{Q}}$ & $t_{\mathcal{Q}}$ dominates $t^\prime_{\mathcal{Q}}$ on query space $\mathcal{Q}$ \\
    \hline
    $\mathcal{L_Q}$ & Set of sorted lists corresponds to attributes in $\mathcal{Q}$\\
    \hline
    $cv_{ij}$ & Attribute value returned by $i$-th sorted access on list $L_j$\\
    \hline
    $T$ & Tree for storing the candidate skyline tuples\\
    \hline
    $p_i$ & the probability that the binary attribute $A_i$ is $1$\\
    \hline
\end{tabular}
\end{tiny}
\end{table}
