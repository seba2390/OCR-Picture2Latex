\pdfoutput=1
\def\year{2021}\relax
%File: formatting-instructions-latex-2021.tex
%release 2021.1
\documentclass[letterpaper]{article} % DO NOT CHANGE THIS
\usepackage[switch]{lineno}
\usepackage{aaai21}  % DO NOT CHANGE THIS
\usepackage{times}  % DO NOT CHANGE THIS
\usepackage{helvet} % DO NOT CHANGE THIS
\usepackage{courier}  % DO NOT CHANGE THIS
\usepackage[hyphens]{url}  % DO NOT CHANGE THIS
\usepackage{graphicx} % DO NOT CHANGE THIS
\urlstyle{rm} % DO NOT CHANGE THIS
\def\UrlFont{\rm}  % DO NOT CHANGE THIS
\usepackage{graphicx}  % DO NOT CHANGE THIS
\usepackage{natbib}  % DO NOT CHANGE THIS AND DO NOT ADD ANY OPTIONS TO IT
\usepackage{caption} % DO NOT CHANGE THIS AND DO NOT ADD ANY OPTIONS TO IT
\frenchspacing  % DO NOT CHANGE THIS
\setlength{\pdfpagewidth}{8.5in}  % DO NOT CHANGE THIS
\setlength{\pdfpageheight}{11in}  % DO NOT CHANGE THIS


\usepackage[percent]{overpic}

\usepackage{microtype}
\usepackage{tabularx}
\usepackage{graphicx}
\usepackage{subfigure}
\usepackage{booktabs} % for professional tables
\usepackage{times}
\usepackage{epsfig}
\usepackage{graphicx}
\usepackage{amsmath}
\usepackage{amssymb}
\usepackage{multirow}
\usepackage{algorithm}
\usepackage{algorithmic}
\usepackage{subfigure}
\usepackage{amsmath}



\def\cput(#1,#2)#3{\put(#1,#2){\hbox to 0pt{\hss{#3}\hss}}}
\newtheorem{definition}{Definition}
\newtheorem{experiment}{Experiment}
\newtheorem{theorem}{Theorem}
\newtheorem{lemma}{Lemma}
\newtheorem{inference}{Inference}
\newtheorem{conclusion}{Conclusion}
\newtheorem{conjecture}{Conjecture}
\newtheorem{proof}{Proof}[section]
\newtheorem{corollary}{Corollary}

\floatname{algorithm}{Experiment}
\setcounter{secnumdepth}{0}
\title{THE DISTANCE BETWEEN THE WEIGHTS OF THE NEURAL NETWORK IS
MEANINGFUL\\}
\author{
    %Authors
    % All authors must be in the same font size and format.
    Liqun Yang \textsuperscript{\rm 1}
    Yijun Yang \textsuperscript{\rm 2}
    Yao Wang \textsuperscript{\rm 2}
    Zhenyu Yang \textsuperscript{\rm 2}
    Wei Zeng \textsuperscript{\rm 2}\\
}
\affiliations{
    %Afiliations
    \textsuperscript{\rm 1}Florida International University, School of Computing and Information Science\\
    \textsuperscript{\rm 2} Xi'an Jiaotong University, School of Computer Science and Technology\\
}
% \author {
%     % Authors
%     First Author Name,\textsuperscript{\rm 1}
%     Second Author Name, \textsuperscript{\rm 2}
%     Third Author Name \textsuperscript{\rm 1} \\
% }
% \affiliations {
%     % Affiliations
%     \textsuperscript{\rm 1} Affiliation 1 \\
%     \textsuperscript{\rm 2} Affiliation 2 \\
% }

\begin{document}

\maketitle

\begin{abstract}
In the application of neural networks, we need to select a suitable model based on the problem complexity and the dataset scale. To analyze the network's capacity, quantifying the information learned by the network is necessary. This paper proves that the distance between the neural network weights in different training stages can be used to estimate the information accumulated by the network in the training process directly. The experiment results verify the utility of this method. An application of this method related to the label corruption is shown at the end.
\end{abstract}

\section{Introduction}
Since Dr. Hebb opened the door of machine learning in \cite{hebb1949organization}, people have obtained endless wealth from it. At the beginning of this century, neural networks' potential in machine learning tasks was discovered in many fields. With more and more people noticing this delicate model's power, applications based on the neural network develop rapidly and change the world gradually \cite{rumelhart1986learning,hochreiter1997long,fukushima1980neocognitron,lecun2015deep,hinton1994autoencoders,sajjadi2017enhancenet,he2017mask,an2015variational,arjovsky2017wasserstein}, which makes people eager to reveal the essence of the neural network.

As \cite{hornik1991approximation} proves, the network exists that is capable of arbitrarily accurate approximation to a specific function and its derivatives. Therefore, We need to find a suitable network structure for a specific task. To explain it, we denote one network's simulation capability as $I_0$, the information quantity of the relationship between two variables as $I_1$, and the information quantity of the training dataset as $I_2$. To get a trustworthy model, the basic requirement is $I_2 \geq I_0 >> I_1$. If $I_0<I1$, the model cannot simulate the relationship, which will make it hard to train to fine-tuned (underfitting). Else if $I_1 > I_2$, the dataset cannot express the relationship between the variables, which will mislead the model. Else if $I_0 > I_2$, there are too many options to simulate the relationship, which will make the model overfitted in most cases. In practice, $I_1$ is the exploration target, which can be estimated based on the background research, and $I_2$ can be calculated directly. Now, the question is reduced to \textbf{how to estimate the capability of one model.} There is no doubt that we can use the network's scale to estimate its capacity, but it is just a theoretical estimation. The theoretical upper limit of the human brain far exceeds what we can use. As a system with a similar structure, the network's real capacity is far less than its theoretical estimation. That is the reason why some small models can perform better than the bigger ones. What we take care of is the part we can use.

Therefore, the question is reduced to \textbf{how to quantifying the information learned by the neural network in training.} In \cite{tishby2000information}, the author uses the mutual information to analyze the changing of the information quantity in the neural network and put up with the concept called "information bottleneck." The information bottleneck theory describes the neural network's behavior and defines the optimal target, preserving the relevant information about another variable (maximize the bottleneck). One step forward, in \cite{tishby2015deep}, the author develops this method and puts it up with a tool, information plain, to visualize the neural network's behavior. Series of methods related to it are put forward \cite{alemi2016deep,higgins2016beta,yu2020understanding,achille2019information}, and people start to open the black box of the neural network.

\section{Motivation}
The similarity of the methods mentioned above is that they analyze the network's information by analyzing the data's changes. These methods have their rationality, and the weakness is also apparent.

Generally, errors in statistics cannot be eliminated. Comparing with the scale of the domain of the possible networks' input, the scale of the test case is too small, which will further magnify statistical errors. Specifically, let $I(X;\tilde{X})$ be the mutual information between the input data $X$ and the compressed representation (like the output of one layer) $\tilde{X}$.
\begin{equation}
    I(X ; \tilde{X})=\sum_{x \in X} \sum_{\dot{x} \in \tilde{X}} p(x, \tilde{x}) \log \left[\frac{p(\tilde{x} \mid x)}{p(\tilde{x})}\right]
    \label{eq_mutual_information}
\end{equation}
Based on Eq. \ref{eq_mutual_information}, we need to know the joint distribution $P(X,\tilde{X})$ and the distribution $P(\tilde{X})$, which two need to be counted in the experiment. Theoretically, for the neural network, $\tilde{X}$ is a continuous space. Discretization is necessary to count the probability distributions mentioned above. Different discretization functions will impact the result of observation significantly. Moreover, $\tilde{X}$ is tightly related to the layer's activate function based on the definition. If we just discretize $\tilde{X}$ evenly without further discussion, the activate function's feature will impact the observation result and mislead us. For example, in \cite{shwartz2017opening}, the author explains one behavior of the network. As mentioned in that paper, the experiment result indicates that the network's goal is to optimize the Information Bottleneck (IB) trade-off between compression and prediction, successively, for each layer. However, the related conclusion is challenged by other researchers. In \cite{saxe2019information}, the author proves that the two phases are just a special case caused by the non-linear activate function.

In a word, based on analyzing the relationship between input and output, the results will be impacted by the experiment's bias. Whereas directly analyzing the weights of the neural network can avoid the errors mentioned above.

In this paper, we will provide proof to show that the difference between the initial weights and the training's output weight can be used to estimate the quantity of information of the network accumulating in training. Moreover, we apply this method to analyze the impact of the label corruption. The corresponding experiment is shown at the end of this paper.

\section{The uncertainty of weights}
As Shannon mentioned in \cite{shannon1948mathematical}, information can be thought of as the resolution of uncertainty. Generally, we can use $H(X)$ to represent the chaos of the variable $X$ ($H(X)=\mathrm{E}[I(X)]$). The increase in information or energy will lead to a decrease in system entropy in the view of physics. Oppositely, if the entropy decrease is quantified, we can quantify the quantity of information accumulated by the system in this process.
\begin{equation}
    I = H_{P_0}(X) - H_p(X)
    \label{eq_information}
\end{equation}

Letting $I$ be the quantified information, which equal to the chaos reduction, we can use Eq. \ref{eq_information} to calculate it, where $H_{P_0}(X)$ and $H_{P}(X)$ is the system's entropy before and after receiving the information. To measure the information stored in the weights, we need to understand the uncertainty of the weights. Our viewpoint is the weights of the network are a variable in the training process. The appearance of a specific value of weights is uncertain because of the random factors in the training process, like the randomized initialization, optimizer (e.g., SGD), order of training data, etc. A certain training process is that in which there are no random factors. And it can be viewed as a special case of the uncertain ones.

There is another view to understanding the uncertainty of the weights. Generally, the calculation in the layer (including the activate function) is irreversible, which gives the capability of generalization to the network. Otherwise, for a network with specific weights, we can use the output to recover the input, which means the relationship between the input data $X$ and its corresponding compress representation $T$ is bijective, and the network degenerates to a codebook of input $X$. Therefore, for the external observer (only the network's input and output are visible), the network's weights cannot be calculated, which is the same as we cannot make sure the status of the Erwin Schrödinger's Cat \cite{marshall1997s}. The weights' value is hidden in an unknown wave function, like the cat's status is unknown after closing the box. Once we open the black box and observe the weights, the wave function collapses into a constant \cite{von2018mathematical}, which is the same as taking a sample from the current wave function.

In the view of information theory, the happening of an uncertain event (probability less than 100\%) provides the information to the receiver, called the variable's self-information \cite{jones1979elementary}. Respectively, we can receive the information through the appearance of specific weights. For a specific training stage, the appearance of a specific weight has a probability. With the network being trained continuously, the possibility distribution of the weights' appearance is changing respectively. Therefore, the information increase before and after the training can evaluate the information quantity provided by the training process.

\section{Probability space of neural networks weights}
For a specific neural network architecture, the status of one neural network can be identified by its weight uniquely. The set of all its possible weights is denoted as $\Omega$. $\mathbb{P}$ is the corresponding probability mass function. For any $\Sigma \in \mathbb{F}$, $p(\Sigma)$ is the corresponding probability. $(\Omega,\mathbb{F},\mathbb{P})$ is a probability space. The following discussion is based on this space. As we all know, the backpropagation (BP) algorithm, which is the basic method for network optimization, is a method working on Euclidean space. Therefore, essentially, this space is a Euclidean space and has two features.
\begin{enumerate}
    \item The dimension of elements in the space is high, which means there are too many elements in the space to enumerate.
    \item The possibility of one event happening $p(\{\omega\}) (\omega \in \Omega)$ is low, which means a single event occurs is almost impossible to observe.
\end{enumerate}
Moreover, two adjacent elements in this space might have different appearance probabilities in the specific training stage because of the limitation of computers' precision. For example, there are two weights $\omega_1$ and $\omega_2$, $\omega_1$ equals to $\omega_2 + \mathbb{\epsilon}$, where $\mathbb{\epsilon}$ is a vector whose components in each dimension are almost equal to the lowest precision error. For the same input, the output of these two layers might be different. This error will be magnified as the network depth increases, and the loss of these two weights can be different. The one with higher accuracy has a higher probability of appearing at the end of the training process. Therefore, the original space's discretization is hard to calculate directly, which means the prior probability $\mathbb{P}$ cannot be counted by the traditional method.

\subsection{The weights distribution visualization}
As Eq. \ref{eq_information} shows, to quantify the information accumulated by the network in the training process, we need to know the probability measure $\mathbb{P}_0$ and $\mathbb{P}$, where $\mathbb{P}_0$ and $\mathbb{P}$ are the probability mass function of the appearance of the weight before and after training respectively. The initialization of the weights is randomized, and $\mathbb{P}_0$ is an even distribution in a range defined by the initialization function. Now, the question is \textbf{how to estimate $\mathbb{P}$.}

To observe the distribution of $\mathbb{P}$, we use the same training configuration to repeatedly train a specific network and collect the input weight (randomized) and output weights. Then, we use multiple dimensional scaling (MDS) to visualizing the level of similarity of individual weights. MDS is a method used to translate "information about the pairwise 'distances' among a set of n objects or individuals" into a configuration of $n$ points mapped into an abstract Cartesian space \cite{mead1992review}. The most important feature of MDS is that it can keep the Euclidean distance after dimensional reduction. The classic MDS algorithm \cite{wickelmaier2003introduction} is shown in Alg. \ref{MDS_fig}

\begin{algorithm}[htbp]
\caption*{\textbf{Algorithm 1} Multidimensional Scaling}
\begin{algorithmic}[1]
    \STATE Set up the squared proximity matrix $D^{(2)}=\left[d_{i j}^{2}\right]$, where $d_{ij}$ is the Euclidean distance between $i^{th}$ and $j^{th}$ elements.
    \STATE Apply double centering \cite{marden1996analyzing}: \\${\textstyle B=-{\frac {1}{2}}JD^{(2)}J}$ using the centering matrix ${\textstyle J=I-{\frac {1}{n}}11'}$, where ${\textstyle n}{\textstyle n}$ is the number of objects, $1$ being an $N ×1$ column vector of all ones.
    \STATE Determine the ${\textstyle m}$ largest eigenvalues ${\textstyle \lambda _{1},\lambda _{2},...,\lambda _{m}}$ and corresponding eigenvectors ${\textstyle e_{1},e_{2},...,e_{m}}$ of ${\textstyle B}$(where ${\textstyle m}$ is the number of dimensions desired for the output).
    \STATE Now, ${\textstyle X=E_{m}\Lambda _{m}^{1/2}}$, where ${\textstyle E_{m}}$ is the matrix of ${\textstyle m}$ eigenvectors and ${\textstyle \Lambda _{m}}$ is the diagonal matrix of ${\textstyle m}$ eigenvalues of ${\textstyle B}$.
\label{MDS_alg}
\end{algorithmic}
\end{algorithm}

Specifically, in this experiment, we use a TensorFlow CNN Demo \cite{TensorflowDemo} to identify the images in CIFAR-10 \cite{Krizhevsky09learningmultiple}. We fix all the super parameters in the training process and train the network from different scratches repeatedly. The experiment process is shown in Exp. \ref{Randomizedinit_exp} and the training configuration is shown in Tab. \ref{Tab_traininginfo}.



\begin{algorithm}
\caption{Input and output weights distribution by randomized training from scratches \label{Randomizedinit_exp}}
\begin{algorithmic}[1]
    \STATE Fix the training configuration, including all super parameters.
    \STATE Initialize 1000 networks randomly and save their initial value of weights.
    \STATE Train the networks to fine-tuned and save their value of weights at the end of the training.
    \STATE Use the MDS algorithm to visualize the input and output weights.
\end{algorithmic}
\end{algorithm}

As Fig. \ref{MDS_init} ($\Bar{r} = 1.28, std = 0.007$) and Fig \ref{MDS_end} ($\Bar{r} = 1.34, std = 0.12$) shows, all the points are distributed on the spherical surface evenly, which means their source vectors are also distributed evenly in the corresponding high-dimensional space.

\begin{figure}[htb]
    \centering
    \subfigure[] {
    \label{MDS_init}
    \includegraphics[width=0.22\textwidth]{Sphere_begining.pdf}
    }
    \subfigure[] {
    \label{MDS_end}
    \includegraphics[width=0.22\textwidth]{Sphere_distribution.pdf}
    }
    \caption{The left shows the MDS result of the weights at the beginning. The right shows the MDS result of the weights at the end.}
    \label{MDS_fig}
\end{figure}

As the reference, we add a constraint that limits the initial weights into a small range $\{w_0', w_0" \}$. Then train the network repeatedly. The experiment process is shown in Exp. \ref{Twoinit_exp}.
\begin{algorithm}
\caption{Input and output Weights distribution by randomized training from 2 specific scratches\label{Twoinit_exp}}
\begin{algorithmic}[1]
    \STATE Fix the training configuration, including all super parameters.
    \STATE Initialize 200 networks. Half of them use $w_0'$ as the initial weights, and the others use the $w_0''$.
    \STATE Train the networks to fine-tuned and save their value of weights at the end of the training.
    \STATE Use the MDS algorithm \ref{MDS_alg} to visualize the input and output weights.
\end{algorithmic}
\end{algorithm}

\begin{figure}[htb]
    \centering
     \includegraphics[width=0.22\textwidth]{polarized_distribution.pdf}
    \caption{The initial weights are mapped to these two red points, and the output weights are mapped to these blue points.}
    \label{MDS_fig_polarized}
\end{figure}
We use the MDS algorithm to reduce the dimension of the initial weights and the output weights together. The output is shown in Fig. \ref{MDS_fig_polarized}. It shows that the output weights' mapping points (shown as the blue points) are distributed near their initial weights' mapping points (shown as the red points), which means the weights have higher appearance probability if its mapping point is in the region with more points. We can infer that if the mapping points distribute evenly in a region, their source's appearance probability is similar.

Based on the experiments mentioned above, we have Thm. \ref{even_distribution}
\begin{theorem}
    For a randomized training process, all the weights have the same appearance probability if they can appear.
    \label{even_distribution}
\end{theorem}

Letting $supp(p_0)$ and $supp(p)$ be the support of $p_0$ and $p$ respectively ($|supp(p_0)| \geq |supp(p_0)| > 0$), based on Thm. \ref{even_distribution}, we have
\begin{align*}
    H_{p_0}(X) &=  \log(\frac{1}{|supp(p_0)|}) \\
    H_{p}(X)  &=  \log(\frac{1}{|supp(p)|}). \\
\end{align*}
And we have
\begin{align*}
    I &= \log(\frac{1}{|supp(p_0)|}) - \log(\frac{1}{|supp(p)|}) \\
    &= \log (\frac{|supp(p)|}{|supp(p_0)|}).
\end{align*}

\begin{theorem}
\label{support_set_size}
    The information provided by training can be measured by the ratio between the support size before and after training.
\end{theorem}

Generally, if one training process cannot make the network converge into a stable status, the training fails. In this paper, we only discuss the successful training process ($supp(p) \subset Supp(p_0)$ and $I > 0$), and we have Thm. \ref{support_set_size}. Letting $r(p_0,p) = \frac{|supp(p)|}{|supp(p_0)|}$, the question is reduced to \textbf{how to estimate the ratio $r$ between the scale of support before and after the training.}

\section{Quasi-Monte Carlo method to estimate the set scale ratio}
Generally, the Monte Carlo method \cite{kroese2014monte} (MCM) can be used to estimate the scale shrink between one set and its subset. However, when one set's scale is much smaller than the other and elements are in a high dimensional space, the traditional MCM is not helpful. We provide a new quasi-Monte Carlo method (QMCM) to estimate the scale differences between two sets $ X $ and $X'$ when $X' \subset X$. Briefly, \textbf{the QMCM uses the expectation of the shortest distance to estimate the ratio between $X$ and $X'$.} For more details, we show the derivation below.

\subsection{Basic assumption}
For a set $X$ which can be embedded into a measurable space and its non-empty proper subset $X'$, we define $d_{X'}(x)$ as the distance between $x$ and its closest element in $X'$ as Fig. \ref{dm_definition} shows, and we have $d_{X'}(x) = 0$ if $x \in X'$. Letting $D_{X'}(X)$ denote the sum of shortest distance for $x \in X$, we have
\begin{align*}
    D_{X'}(X) &= \sum_{x\in X}d_{X'}(x). \\
    \Bar{d}_{X'}(X) &= \frac{D_{X'}(X)}{|X|}(x \in X)
\end{align*}
Abbreviating $\Bar{d}_{X'}(X) $ as $\Bar{d}_{X'}$, for specific set $X$, $|X|$ is a constant.
If we want to use $\Bar{d}_{X'}$ to estimate $r$, we need to prove Thm. \ref{estimate_distance}.

\begin{theorem}
    Letting $r(X') = \frac{|X|}{|X'|}$, $\Bar{d}_{X'}$ is a monotonically increasing function of $r$ on for any subset $X'$ of $X$.
    \label{estimate_distance}
\end{theorem}

\textbf{Proof.} Letting $X''$ be the union of set $X'$ and the $\{x''\}$ ($x''\in X-X'$), we have
\begin{equation*}
    \frac{|X|}{|X''|}  < \frac{|X|}{|X'|}.
\end{equation*}
$d_{X'}(x'')$ is always positive, we have
\begin{equation*}
    D_{X'}(X) - d_{X'}(x'') < D_{X'}(X),
\end{equation*}
which leads to
\begin{equation*}
    D_{X'}(X-\{x''\}) < D(X,X').
\end{equation*}
Adding a new element $x''$ to $X'$ will update the distance from some elements to $X'$, noted as $\sum \Delta d$, and we have

\begin{equation*}
    D_{X''}(X) = D_{X'}(X-\{x''\})-\sum \Delta d.
\end{equation*}
Therefore, we have
\begin{equation*}
     D_{X'}(X-\{x''\}) - \sum \Delta d < D_{X'}(X-\{x''\}),
\end{equation*}
and we have
\begin{equation*}
    D_{X''}(X) < D_{X'}(X-\{x''\}),
\end{equation*}
which leads to
\begin{equation*}
    D_{X''}(X) < D_{X'}(X).
\end{equation*}
And we have
\begin{equation*}
    \frac{D_{X''}(X)}{|X|} < \frac{D_{X'}(X)}{|X|},
\end{equation*}

\begin{equation*}
    \Bar{d}_{X''} < \Bar{d}_{X'}.
\end{equation*}
And we have
\begin{equation*}
    r(X'') < r(X') \Rightarrow \Bar{d}_{X''} < \Bar{d}_{X'}.
\end{equation*}
The proof of other side is similar, and we have
\begin{equation*}
    r(X'') < r(X') \Leftrightarrow  \Bar{d}_{X''} < \Bar{d}_{X'} \square
\end{equation*}

\begin{figure}[htbp]
    \centering
    \includegraphics[width = 0.45 \textwidth]{sample_distance.pdf}
    \caption{$d_{X'}(x)$ is the distance between $x$ and $X'$}
    \label{dm_definition}
\end{figure}

Denoting the support of function $d_{X'}$ as $supp(d)$ and the expectation of $d_{X'}$ on $supp(d)$ as $\Bar{d}(supp(d))$. Based on the assumption,
\begin{align*}
    |supp(d)| = (1-r)|X|,\\
\end{align*}
we have
\begin{equation*}
    \Bar{d}(supp(d)) = (1-r)\Bar{d}(X)
\end{equation*}
For a specific $X'$, $r$ is a constant. When $1>>r$, we have
\begin{equation*}
    \Bar{d}(supp(d)) \approx \Bar{d}(X).
\end{equation*}
Now, the question is reduced to \textbf{how to estimate $\Bar{d}(supp(d))$.}


\subsection{The distribution of element-wised shortest distance}
Generally, we can use repeated random sampling to get a numerical approximation of $\Bar{d}(supp(d))$. If the cost of one sampling is high, we cannot take enough samples to prove the estimation accuracy. However, for a particular case, \textbf{when the mode of the distribution is equal to its mean, we can estimate the mean of the population with very few samples.}

To prove this, we need to analyze the distribution of $d_{X'}(x) (x \in supp(d))$. We use numerical simulation to analyze this function.
\begin{enumerate}
    \item Generate a set $X$ randomly with 10,000 elements, which are 100-dimensional normalized vectors.
    \item Select $r\%$ elements from $X$ randomly as the subset $X'$.
    \item Calculate the shortest distance from $x (x \in X-X')$ to $X'$ and count its distribution.
\end{enumerate}

As an example, Fig. \ref{Distribution_sample} shows the result of the distribution when $r = 10\%$, and it is similar to the corresponding normal distribution (with the same mean value and standard difference value). Moreover, we change $r$ and calculate the KL divergence \cite{kullback1951information} between the shortest distance probability distribution ($P$) and the corresponding normal distribution ($Q$) as Eq. \ref{eq_KLdiv} shows. The result is shown in Fig. \ref{KL_curve}. Except for the cases when $r \geq 80\% $, the distribution trend is similar to a normal distribution, and we have Thm. \ref{support_distance_theorom}.

\begin{equation}
    D_{\mathrm{KL}}(P \| Q)=-\sum_{i} P(i) \ln \frac{Q(i)}{P(i)}
    \label{eq_KLdiv}
\end{equation}

\begin{figure}[p]
    \centering
    \includegraphics[width = 0.45\textwidth]{sample_distribution.pdf}
    \caption{The shortest distance distribution is the distribution when $r = 10\%$, and the normal distribution is a randomized normal distribution with the same mean and standard difference value.}
    \label{Distribution_sample}
\end{figure}

\begin{figure}[p]
    \centering
    \includegraphics[width = 0.45\textwidth]{KL_div.pdf}
    \caption{The curve of the KL divergence between the shortest distance distribution and the corresponding normal distribution.}
    \label{KL_curve}
\end{figure}

\begin{theorem}
    For the elements in $supp(d)$, the distribution of their function value can be viewed as a normal distribution.
    \label{support_distance_theorom}
\end{theorem}

As a normal distribution, the mode of this distribution is equal to its mean. Letting $Y$ be the subset of $X$, when $c_v(d(y)) \leq t$ where $t$ is a threshold, we have
\begin{equation*}
    \Bar{d}(Y) \approx \Bar{d}(X)
\end{equation*}
And the QMCM method can be described as Alg. 2.

\begin{algorithm}[htbp]
\caption*{\textbf{Algorithm 2} QMCM for Subset scale estimation}
\begin{algorithmic}[1]
    \STATE Set the threshold $t$ and amount of samples $n$.
    \STATE Take $n$ samples from $X$ as $Y$ and calculate $Var(d(y))$.
    \WHILE {$c_v(d(y)) > t$}
    \STATE Remove the sample from $Y$ with the largest deviation from the mean.
    \STATE Resample and add the sample to $Y$.
    \ENDWHILE
    \STATE Output $\Bar{d}(Y)$.
\end{algorithmic}
\end{algorithm}
We can adjust the accuracy of the estimation by adjusting the value of $t$ and $n$. In the following experiment, we set $t = 0.3, n = 200$,

Unlike the traditional MCM, the novelty of this method uses the distance between related two points to estimate the ratio.

Abbreviating $\Bar{d}(Y)$ as $\hat{d}$, if we have two training process $t_1$ and $t_2$, we have
\begin{equation}
    \hat{d}_{t_1} < \hat{d}_{t_2} \Leftrightarrow I_{t_1} < I_{t_2}.
    \label{information_comparing}
\end{equation}
If $ \hat{d}_{t_1} < \hat{d}_{t_2}$, we have $I_{t_1} < I_{t_2}$. Oppositely, if $I_{t_1} < I_{t_2}$, we have $\hat{d}_{t_1} < \hat{d}_{t_2}$, which can be used to verify the correctness of our method.

\subsection{Apply on the neural network}
Correspondingly, for the network's training process, we have its initial weights and the output weights. The condition to implement QMCM to estimate the information is that the output weight is the closest one of the initial weights in $supp(P)$.

We select seven network models from simple to complex to verify this, TensorFlow MNIST classification Demo (classical version) \cite{TensorFlowMNIST}, TensorFlow CNN Demo \cite{TensorflowDemo}, GoogleNet \cite{szegedy2015going}, AlexNet \cite{krizhevsky2017imagenet}, ResNet \cite{he2016deep}, VGG \cite{simonyan2014very} and Yolo v3 \cite{redmon2018yolov3}. To ensure that these networks are used in scenarios that adapt to them, we select 4 dataset with different input scale and complexity, MNIST \cite{lecun1998gradient}, CIFAR-10 \cite{Krizhevsky09learningmultiple}, TensorFlow Flowers \cite{tfflowers} and Pascal VOC \cite{pascal-voc-2007}. We train each kind of model from scratch to fine-tune it with the same configuration 1000 times repeatedly. The basic information of the training is shown in Tab. \ref{Tab_traininginfo}. And then, we calculate the distance of arbitrary pairs of initial and end states. The result shows that all the output weight is the closest one of the initial weights in $supp(P)$. Therefore, we can use Eq. \ref{information_comparing} to compare the influence of the two training processes to the same model.

Moreover, we calculate the mean, standard difference, and coefficient of variation ($c_v$) value of the distance (see Tab. \ref{Tab_distancestatistic}). It shows that although the difference in the mean value among models is big, the coefficient of variation is still at a low level, which reflects the stability of this estimation.

\begin{table*}[]
    \centering
    \begin{tabularx}{\textwidth}{|c|X|X|X|X|X|X|X|X|X|X|}
    \hline
        \multicolumn{2}{|c|}{Model} & TF CNN & TF MNIST & GoogleNet & ResNet & VGG & AlexNet & Yolo v3\\
        \hline
        \multicolumn{2}{|c|}{Dataset} & CIFAR-10 & MNIST & TF Flower & TF Flower & TF Flower & TF Flower & Pascal\\
        \hline
        \multirow{2}{*}{Super} & LR & 0.01 & 0.01 & 0.01 & 0.01(decay) &0.01(decay) & 0.01(decay) & 0.01(decay)\\ \cline{2-9}

        \multirow{2}{*}{params}& Epoch & 10 & 10 & 20 & 40 & 20 & 20 & 20\\ \cline{2-9}
        & Batch Size & 128 &128 & 32 & 32&32 &32 &64 \\ \cline{2-9}
        & Optimizer & \multicolumn{7}{c|}{Stochastic Gradient Descent}\\
        \hline
         & Normlize & \multicolumn{7}{c|}{Yes} \\ \cline{2-9}
        Pre-& Mean Sub & \multicolumn{2}{c|}{No} &Yes &Yes &Yes &Yes  & No \\ \cline{2-9}
        process & Rescale &  \multicolumn{2}{c|}{Yes} &Yes &Yes &Yes &Yes & Yes\\\cline{2-9}
        & Standardize & \multicolumn{2}{c|}{Yes}&No &No &No &No  & Yes\\\cline{2-9}
    \hline
    \end{tabularx}
    \caption{Training configuration of experiments.}
    \label{Tab_traininginfo}
\end{table*}

\begin{table}[]
    \centering
    \begin{tabular}{|c|c|c|c|}
    \hline
    Network &  Mean & STD & $c_v$(\%) \\
    \hline
    TF CNN Demo & 4.026  & 0.053 & 1.316 \\
    TF MNIST Demo & 1.358 & 0.031 & 2.282 \\
    GoogleNet & 28.084 & 0.482 & 1.718\\
    ResNet & 8785.243 & 1172.836 & 13.350 \\
    VGG & 0.556 & 0.012 & 2.158 \\
    AlexNet & 4.849 & 0.589 & 12.163\\
    Yolo v3 & 26.863 & 6.286 & 23.400  \\
    \hline
    \end{tabular}
    \caption{The statistic result (mean value, standard difference and coefficient of variation) about the distance between initial and end states of neural networks.}
    \label{Tab_distancestatistic}
\end{table}

\section{Verification}
As mentioned in Eq. \ref{information_comparing}, we have  $I_{t_1} < I_{t_2} \Rightarrow \hat{d}_{t_1} < \hat{d}_{t_2}$, which can be used to verify the correctness of our method. We can construct two training process $t_1$ and $t_2$ such that $I(t_1) < I(t_2)$. Then calculate  $\hat{d}_{t_1}$ and $\hat{d}_{t_2}$ respectively.

Now, we need to construct these two training process $t_1$ and $t_2$. Based on the theory of information \cite{shannon1948mathematical}, higher entropy means more information. If we control all the other random factors and make $t_1$ contains more kinds of samples than $t_2$, we have $I(t_1)>I(t_2)$. Specifically, we use the same models to verify our method (see Tab. \ref{Tab_traininginfo}).

\begin{algorithm}[ht]
\caption{Training with different amount of labels\label{labels_exp}}
\begin{algorithmic}[1]
    \STATE Fix the training configuration, including all super parameters.
    \STATE Initialize 5 networks $\{m_i\} (i \in [0,4])$.
    \STATE Use data with $(20\times i)\%$ labels to train the network separately and record the weights changing.
    \STATE Calculate $d(w,w')$ of each weight, where $w$ is the current weights, $w'$ is the output weights of the same training process.
\end{algorithmic}
\end{algorithm}

In \cite{wang2020generalizing}, the author proves that few samples can also be used to guide the network to complete a complex task. Because of that, to ensure  $I(t_1)>I(t_2)$, we control the numbers of labels. There are $[20\%,40\%,60\%,80\%,100\%]$ kinds of samples are used in the training. To eliminate the impact of the weights update numbers, we train the same amount of steps in all the model training process. Finally, we calculate the $\hat{d_{t_i}}$ for each training process (see Exp. \ref{labels_exp}).
\begin{figure*}[p]
    \centering
    \subfigure[TF CNN Demo] {
    \includegraphics[width=0.45\textwidth]{TFCNN_classification.pdf}
    }
    \subfigure[TF MNIST Demo] {
    \includegraphics[width=0.45\textwidth]{TFMNIST_classificaton.pdf}
    }
    \subfigure[GoogleNet] {
    \includegraphics[width=0.45\textwidth]{GoogleNet_classification.pdf}
    }
    \subfigure[VGG] {
    \includegraphics[width=0.45\textwidth]{VGG_classificaton.pdf}
    }
    \subfigure[ResNet] {
    \includegraphics[width=0.45\textwidth]{ResNet_classification.pdf}
    }
    \subfigure[AlexNet] {
    \includegraphics[width=0.45\textwidth]{AlexNet_classification.pdf}
    }
    \subfigure[Yolo] {
    \includegraphics[width=0.45\textwidth]{yolo_classification.pdf}
    }
    \caption{The summary of the Exp. \ref{labels_exp} results of the verification.}
    \label{ed_1}
\end{figure*}

As Fig. \ref{ed_1} shows, in all experiments, as the $I$ increases, $\hat{d_{t_i}}$ increases, which verifies the utility of our method.

\section{Application: impact of label corruption}
The most significant application is to evaluate the training process.

In most cases, the trained network's performance is the only measure in the evaluation of the training process. However, as the author mentioned in \cite{geirhos2020shortcut}, some of the information in the dataset can be the "shortcut" to complete tasks, which leads to the training's failure. If the test dataset has similar "shortcuts," this failure will be unnoticeable without complex analysis. Moreover, the Clever Hans effect has been observed in the early version of BERT \cite{devlin2018bert} when it completes the argument reasoning comprehension task. The network makes the correct judgment based on a hidden trick but not the logic we want it to learn, which Niven and Kao noticed in \cite{niven2019probing} . By analyzing the quantity of information, we can reveal the essence of the networks' learning process and avoid being misled by the errors mentioned above.

Here, we use the impact of the label corruption as an example to show our method's application. For the impact of the label corruption, as the author mentions in \cite{zhang2016understanding}, some networks can build the relationship between the label and the data even the label is randomized. Their experiment shows that the time of overfitting of their models increases with the error label rate. In other words, the network needs more time to learn the noisy dataset. Based on this phenomenon, there are two assumptions based on our intuition.

\begin{enumerate}
    \item The label corruption makes the relationship between the data and the label more complex. It makes the cost to describe the relationship increases, which means the network can accumulate more information in training.
    \item The label corruption makes the dataset contains more conflict information, which decreases the quantity of information accumulated by the network.
\end{enumerate}
The performance-based evaluation is not useful to answer these questions, and we use our method to verify these two assumptions. Briefly, if the first assumption is correct, the network accumulates more information in the same length of time. Otherwise, if the second assumption is correct, the network accumulates less information within the same time.

We use TensorFlow CNN Demo as the test model and CIFAR-10 as the test dataset with the same training configuration shown in Tab. \ref{Tab_traininginfo}. The experiment process is shown in Exp. \ref{labelcorruption_exp}.

\begin{algorithm}[ht]
\caption{Impact of label corruption \label{labelcorruption_exp}}
\begin{algorithmic}[1]
    \STATE Create eleven datasets based on CIFAR-10, $\{A_i\} (i\in [0,100])$. In $A_i$, $(i\times 10)\%$ labels are shuffled randomly.
    \STATE Initialize 10 networks by the same initial weights.
    \STATE Use these datasets to train the model separately within the same time.
    \STATE calculate the distance $d(w_i, w)$, where $w_i$ is the current weights, and $w$ is the final weights after training.
\end{algorithmic}
\end{algorithm}

The result is shown in Fig. \ref{Label_corruption}. The intercept of the curve shows that the amount of the information accumulated by the network with the label error rate increasing, which means the label corruption hinders the network's learning.
\begin{figure}[htb]
    \centering
    \includegraphics[width=0.44\textwidth]{Label_corruption.pdf}
    \caption{The label corruption can decrease the quantity of the information provided by the dataset.}
    \label{Label_corruption}
\end{figure}

\section{Discussion}
\textbf{Question about information effectiveness}. This paper provides a tool to analyze the information accumulated by the network in the training process. The precondition of this method is that most of the information learned by the network needs to be useful, which means the training process's output needs to be fine-tuned. Otherwise, we cannot guarantee that the nearest points of the initial weights in $supp(P)$ might not be the output weights.

However, in practice, this precondition is not always satisfied. The best counterexample is the overfitting phenomenon, which shows that the network can perform well on the training dataset but performs badly on the test dataset. One of the accepted explanations is that the network learned too much knowledge from the training dataset, not the commonality between the training dataset and the test dataset. It indicates that not all the information learned by the network is useful and meaningful.

Therefore, to analyze such failed training cases, we still need further research to quantify the information's effectiveness.

\textbf{Question about cross-modal verification}. We use four datasets from simple to complex (MNIST, CIFAR-10, TensorFlow Flowers, Pascal VOC). And we use seven models to accept the information from these four from simple to complex (TF MNIST Demo, TF CNN Demo, AlexNet, VGG, ResNet, GoogleNet, Yolo). Although our method is verified by all of these models when the model's structure is fixed, the size order of the information quantity is not consistent with our expectations when the network structure is different (see Tab. \ref{Tab_distanceorder}).

On the one hand, to train the model to fine-tuned, we use different training configuration to train the model as Tab. \ref{Tab_traininginfo} shows, which might impact the information accumulation. On the other hand, for different models, their capacity of representation is different. It means to store a specific piece of information, the bigger ones' weights need less change than the smaller ones', which means the presented information quantity is less than the others.
% If we view the information as some kinds of energy that can also lead to the entropy reduction, the weights changing is the result of the energy inputting essentially, and there is another attribute of the neural network which plays a vital role in the training, the mass of the network. Theoretically, the mass measures the thing's resistance to acceleration.
% In this paper, we only discuss the impact of the energy
Even though we do not deny that this reflects our research's limitations, it indicates that we need further study to provide a more general model based on the current achievement.


% \begin{table*}
%     \centering
%     \scalebox{0.45}{
%     \begin{tabularx}{\textwidth}{|l|X|}
%         \hline
%         Subset (\%) & Labels\\
%         \hline
%         $A_{20}$ & Aeroplanes, Bicycles, Birds, Boats \\
%         &\\
%         $A_{40}$ & Aeroplanes, Bicycles, Birds, Boats, Bottles, Buses, Cars, Cats\\
%         &\\
%         $A_{60}$ & Aeroplanes, Bicycles, Birds, Boats, Bottles, Buses, Cars, Cats, Chairs, Cows, Dining tables, Dogs\\
%         &\\
%         $A_{80}$ & Aeroplanes, Bicycles, Birds, Boats, Bottles, Buses, Cars, Cats, Chairs, Cows, Dining, Tables, Dogs, Horses, Motorbikes, People, Potted plants\\
%         &\\
%         $A_{100}$ & Aeroplanes, Bicycles, Birds, Boats, Bottles, Buses, Cars, Cats, Chairs, Cows, Dining tables, Dogs, Horses, Motorbikes, People, Potted plants, Sheep, Sofas, Trains, TV/Monitors \\
%         \hline
%     \end{tabularx}%
%     }
%     \caption{Pascal VOC dataset separation in the experiment.}
%     \label{Tab_seperation_pascal}
% \end{table*}%

\begin{table}[]
    \centering
    \scalebox{0.85}{
    \begin{tabular}{|c|c|c|}
    \hline
    Rank & Expected order &  Actual order  \\
    \hline
    1& TF MNIST Demo (MNIST) & VGG (TF Flower) \\
    2& TF CNN Demo (cifar-10) & TF MNIST Demo (MNIST)\\
    3& AlexNet (cifar-10) & TF CNN Demo (cifar-10)\\
    4& VGG (TF Flower)& AlexNet (cifar-10)\\
    5& ResNet (TF Flower)& Yolo (Pascal VOC)\\
    6& GoogleNet (TF Flower) & GoogleNet  (TF Flower)\\
    7& Yolo v3 (Pascal VOC) & ResNet  (TF Flower)\\
    \hline
    \end{tabular}}
    \caption{The information quantity rank in the experiments.}
    \label{Tab_distanceorder}
\end{table}

\section{Future Works}
Even though this work still gives us a new view to analyze the essence of the neural network. Quantify the information is great progress to answer the questions about the network explanation. We will apply it in the following fields to solve the related questions.
\begin{enumerate}
    \item By quantifying the information in a different part of the network, we can target the data's key feature with more confidence.
    \item By quantifying the network's information, we can evaluate the training process and optimize it.
    \item By analyzing the information quantity changing, we can reveal the essence of the double decent phenomenon \cite{nakkiran2019deep}.
\end{enumerate}
Moreover, we will keep working in this field for a more general model.

\def\year{2021}\relax
%File: formatting-instructions-latex-2021.tex
%release 2021.1
\documentclass[letterpaper]{article} % DO NOT CHANGE THIS
\usepackage{aaai21}  % DO NOT CHANGE THIS
\usepackage{times}  % DO NOT CHANGE THIS
\usepackage{helvet} % DO NOT CHANGE THIS
\usepackage{courier}  % DO NOT CHANGE THIS
\usepackage[hyphens]{url}  % DO NOT CHANGE THIS
\usepackage{graphicx} % DO NOT CHANGE THIS
\urlstyle{rm} % DO NOT CHANGE THIS
\def\UrlFont{\rm}  % DO NOT CHANGE THIS
\usepackage{natbib}  % DO NOT CHANGE THIS AND DO NOT ADD ANY OPTIONS TO IT
\usepackage{caption} % DO NOT CHANGE THIS AND DO NOT ADD ANY OPTIONS TO IT
\frenchspacing  % DO NOT CHANGE THIS
\setlength{\pdfpagewidth}{8.5in}  % DO NOT CHANGE THIS
\setlength{\pdfpageheight}{11in}  % DO NOT CHANGE THIS
%\nocopyright
%PDF Info Is REQUIRED.
% For /Author, add all authors within the parentheses, separated by commas. No accents or commands.
% For /Title, add Title in Mixed Case. No accents or commands. Retain the parentheses.
\pdfinfo{
/Title (AAAI Press Formatting Instructions for Authors Using LaTeX -- A Guide)
/Author (AAAI Press Staff, Pater Patel Schneider, Sunil Issar, J. Scott Penberthy, George Ferguson, Hans Guesgen, Francisco Cruz, Marc Pujol-Gonzalez)
/TemplateVersion (2021.1)
} %Leave this
% /Title ()
% Put your actual complete title (no codes, scripts, shortcuts, or LaTeX commands) within the parentheses in mixed case
% Leave the space between \Title and the beginning parenthesis alone
% /Author ()
% Put your actual complete list of authors (no codes, scripts, shortcuts, or LaTeX commands) within the parentheses in mixed case.
% Each author should be only by a comma. If the name contains accents, remove them. If there are any LaTeX commands,
% remove them.

% DISALLOWED PACKAGES
% \usepackage{authblk} -- This package is specifically forbidden
% \usepackage{balance} -- This package is specifically forbidden
% \usepackage{color (if used in text)
% \usepackage{CJK} -- This package is specifically forbidden
% \usepackage{float} -- This package is specifically forbidden
% \usepackage{flushend} -- This package is specifically forbidden
% \usepackage{fontenc} -- This package is specifically forbidden
% \usepackage{fullpage} -- This package is specifically forbidden
% \usepackage{geometry} -- This package is specifically forbidden
% \usepackage{grffile} -- This package is specifically forbidden
% \usepackage{hyperref} -- This package is specifically forbidden
% \usepackage{navigator} -- This package is specifically forbidden
% (or any other package that embeds links such as navigator or hyperref)
% \indentfirst} -- This package is specifically forbidden
% \layout} -- This package is specifically forbidden
% \multicol} -- This package is specifically forbidden
% \nameref} -- This package is specifically forbidden
% \usepackage{savetrees} -- This package is specifically forbidden
% \usepackage{setspace} -- This package is specifically forbidden
% \usepackage{stfloats} -- This package is specifically forbidden
% \usepackage{tabu} -- This package is specifically forbidden
% \usepackage{titlesec} -- This package is specifically forbidden
% \usepackage{tocbibind} -- This package is specifically forbidden
% \usepackage{ulem} -- This package is specifically forbidden
% \usepackage{wrapfig} -- This package is specifically forbidden
% DISALLOWED COMMANDS
% \nocopyright -- Your paper will not be published if you use this command
% \addtolength -- This command may not be used
% \balance -- This command may not be used
% \baselinestretch -- Your paper will not be published if you use this command
% \clearpage -- No page breaks of any kind may be used for the final version of your paper
% \columnsep -- This command may not be used
% \newpage -- No page breaks of any kind may be used for the final version of your paper
% \pagebreak -- No page breaks of any kind may be used for the final version of your paperr
% \pagestyle -- This command may not be used
% \tiny -- This is not an acceptable font size.
% \vspace{- -- No negative value may be used in proximity of a caption, figure, table, section, subsection, subsubsection, or reference
% \vskip{- -- No negative value may be used to alter spacing above or below a caption, figure, table, section, subsection, subsubsection, or reference

\setcounter{secnumdepth}{0} %May be changed to 1 or 2 if section numbers are desired.

% The file aaai21.sty is the style file for AAAI Press
% proceedings, working notes, and technical reports.
%

% Title

% Your title must be in mixed case, not sentence case.
% That means all verbs (including short verbs like be, is, using,and go),
% nouns, adverbs, adjectives should be capitalized, including both words in hyphenated terms, while
% articles, conjunctions, and prepositions are lower case unless they
% directly follow a colon or long dash

\title{AAAI Press Formatting Instructions \\for Authors Using \LaTeX{} --- A Guide }
\author{

    %Authors
    % All authors must be in the same font size and format.
    Written by AAAI Press Staff\textsuperscript{\rm 1}\thanks{With help from the AAAI Publications Committee.}\\
    AAAI Style Contributions by Pater Patel Schneider,
    Sunil Issar,  \\
    J. Scott Penberthy,
    George Ferguson,
    Hans Guesgen,
    Francisco Cruz,
    Marc Pujol-Gonzalez
    \\
}
\affiliations{
    %Afiliations

    \textsuperscript{\rm 1}Association for the Advancement of Artificial Intelligence\\
    %If you have multiple authors and multiple affiliations
    % use superscripts in text and roman font to identify them.
    %For example,

    % Sunil Issar, \textsuperscript{\rm 2}
    % J. Scott Penberthy, \textsuperscript{\rm 3}
    % George Ferguson,\textsuperscript{\rm 4}
    % Hans Guesgen, \textsuperscript{\rm 5}.
    % Note that the comma should be placed BEFORE the superscript for optimum readability

    2275 East Bayshore Road, Suite 160\\
    Palo Alto, California 94303\\
    % email address must be in roman text type, not monospace or sans serif
    publications21@aaai.org

    % See more examples next
}
\iffalse
%Example, Single Author, ->> remove \iffalse,\fi and place them surrounding AAAI title to use it
\title{My Publication Title --- Single Author}
\author {
    % Author
    Author Name \\
}

\affiliations{
    Affiliation \\
    Affiliation Line 2 \\
    name@example.com
}
\fi

\iffalse
%Example, Multiple Authors, ->> remove \iffalse,\fi and place them surrounding AAAI title to use it
\title{My Publication Title --- Multiple Authors}
\author {
    % Authors

        First Author Name,\textsuperscript{\rm 1}
        Second Author Name, \textsuperscript{\rm 2}
        Third Author Name \textsuperscript{\rm 1} \\
}
\affiliations {
    % Affiliations
    \textsuperscript{\rm 1} Affiliation 1 \\
    \textsuperscript{\rm 2} Affiliation 2 \\
    firstAuthor@affiliation1.com, secondAuthor@affilation2.com, thirdAuthor@affiliation1.com
}
\fi
\begin{document}

\maketitle

\begin{abstract}
AAAI creates proceedings, working notes, and technical reports directly from electronic source furnished by the authors. To ensure that all papers in the publication have a uniform appearance, authors must adhere to the following instructions.
\end{abstract}

\noindent Congratulations on having a paper selected for inclusion in an AAAI Press proceedings or technical report! This document details the requirements necessary to get your accepted paper published using PDF\LaTeX{}. If you are using Microsoft Word, instructions are provided in a different document. AAAI Press does not support any other formatting software.

The instructions herein are provided as a general guide for experienced \LaTeX{} users. If you do not know how to use \LaTeX{}, please obtain assistance locally. AAAI cannot provide you with support and the accompanying style files are \textbf{not} guaranteed to work. If the results you obtain are not in accordance with the specifications you received, you must correct your source file to achieve the correct result.

These instructions are generic. Consequently, they do not include specific dates, page charges, and so forth. Please consult your specific written conference instructions for details regarding your submission. Please review the entire document for specific instructions that might apply to your particular situation. All authors must comply with the following:

\begin{itemize}
\item You must use the 2021 AAAI Press \LaTeX{} style file and the aaai21.bst bibliography style files, which are located in the 2021 AAAI Author Kit (aaai21.sty, aaai21.bst).
\item You must complete, sign, and return by the deadline the AAAI copyright form (unless directed by AAAI Press to use the AAAI Distribution License instead).
\item You must read and format your paper source and PDF according to the formatting instructions for authors.
\item You must submit your electronic files and abstract using our electronic submission form \textbf{on time.}
\item You must pay any required page or formatting charges to AAAI Press so that they are received by the deadline.
\item You must check your paper before submitting it, ensuring that it compiles without error, and complies with the guidelines found in the AAAI Author Kit.
\end{itemize}

\section{Copyright}
All papers submitted for publication by AAAI Press must be accompanied by a valid signed copyright form. They must also contain the AAAI copyright notice at the bottom of the first page of the paper. There are no exceptions to these requirements. If you fail to provide us with a signed copyright form or disable the copyright notice, we will be unable to publish your paper. There are \textbf{no exceptions} to this policy. You will find a PDF version of the AAAI copyright form in the AAAI AuthorKit. Please see the specific instructions for your conference for submission details.

\section{Formatting Requirements in Brief}
We need source and PDF files that can be used in a variety of ways and can be output on a variety of devices. The design and appearance of the paper is strictly governed by the aaai style file (aaai21.sty).
\textbf{You must not make any changes to the aaai style file, nor use any commands, packages, style files, or macros within your own paper that alter that design, including, but not limited to spacing, floats, margins, fonts, font size, and appearance.} AAAI imposes requirements on your source and PDF files that must be followed. Most of these requirements are based on our efforts to standardize conference manuscript properties and layout. All papers submitted to AAAI for publication will be recompiled for standardization purposes. Consequently, every paper submission must comply with the following requirements:

\begin{quote}
\begin{itemize}
\item Your .tex file must compile in PDF\LaTeX{} --- (you may not include .ps or .eps figure files.)
\item All fonts must be embedded in the PDF file --- including includes your figures.
\item Modifications to the style file, whether directly or via commands in your document may not ever be made, most especially when made in an effort to avoid extra page charges or make your paper fit in a specific number of pages.
\item No type 3 fonts may be used (even in illustrations).
\item You may not alter the spacing above and below captions, figures, headings, and subheadings.
\item You may not alter the font sizes of text elements, footnotes, heading elements, captions, or title information (for references and mathematics, please see the limited exceptions provided herein).
\item You may not alter the line spacing of text.
\item Your title must follow Title Case capitalization rules (not sentence case).
\item Your .tex file must include completed metadata to pass-through to the PDF (see PDFINFO below).
\item \LaTeX{} documents must use the Times or Nimbus font package (you may not use Computer Modern for the text of your paper).
\item No \LaTeX{} 209 documents may be used or submitted.
\item Your source must not require use of fonts for non-Roman alphabets within the text itself. If your paper includes symbols in other languages (such as, but not limited to, Arabic, Chinese, Hebrew, Japanese, Thai, Russian and other Cyrillic languages), you must restrict their use to bit-mapped figures. Fonts that require non-English language support (CID and Identity-H) must be converted to outlines or 300 dpi bitmap or removed from the document (even if they are in a graphics file embedded in the document).
\item Two-column format in AAAI style is required for all papers.
\item The paper size for final submission must be US letter without exception.
\item The source file must exactly match the PDF.
\item The document margins may not be exceeded (no overfull boxes).
\item The number of pages and the file size must be as specified for your event.
\item No document may be password protected.
\item Neither the PDFs nor the source may contain any embedded links or bookmarks (no hyperref or navigator packages).
\item Your source and PDF must not have any page numbers, footers, or headers (no pagestyle commands).
\item Your PDF must be compatible with Acrobat 5 or higher.
\item Your \LaTeX{} source file (excluding references) must consist of a \textbf{single} file (use of the ``input" command is not allowed.
\item Your graphics must be sized appropriately outside of \LaTeX{} (do not use the ``clip" or ``trim'' command) .
\end{itemize}
\end{quote}

If you do not follow these requirements, your paper will be returned to you to correct the deficiencies.

\section{What Files to Submit}
You must submit the following items to ensure that your paper is published:
\begin{itemize}
\item A fully-compliant PDF file that includes PDF metadata.
\item Your \LaTeX{} source file submitted as a \textbf{single} .tex file (do not use the ``input" command to include sections of your paper --- every section must be in the single source file). (The only allowable exception is .bib file, which should be included separately).
\item The bibliography (.bib) file(s).
\item Your source must compile on our system, which includes only standard \LaTeX{} 2020 TeXLive support files.
\item Only the graphics files used in compiling paper.
\item The \LaTeX{}-generated files (e.g. .aux,  .bbl file, PDF, etc.).
\end{itemize}

Your \LaTeX{} source will be reviewed and recompiled on our system (if it does not compile, your paper will be returned to you. \textbf{Do not submit your source in multiple text files.} Your single \LaTeX{} source file must include all your text, your bibliography (formatted using aaai21.bst), and any custom macros.

Your files should work without any supporting files (other than the program itself) on any computer with a standard \LaTeX{} distribution.

\textbf{Do not send files that are not actually used in the paper.} We don't want you to send us any files not needed for compiling your paper, including, for example, this instructions file, unused graphics files, style files, additional material sent for the purpose of the paper review, and so forth.

\textbf{Do not send supporting files that are not actually used in the paper.} We don't want you to send us any files not needed for compiling your paper, including, for example, this instructions file, unused graphics files, style files, additional material sent for the purpose of the paper review, and so forth.

\textbf{Obsolete style files.} The commands for some common packages (such as some used for algorithms), may have changed. Please be certain that you are not compiling your paper using old or obsolete style files.

\textbf{Final Archive.} Place your PDF and source files in a single archive which should be compressed using .zip. The final file size may not exceed 10 MB.
Name your source file with the last (family) name of the first author, even if that is not you.


\section{Using \LaTeX{} to Format Your Paper}

The latest version of the AAAI style file is available on AAAI's website. Download this file and place it in the \TeX\ search path. Placing it in the same directory as the paper should also work. You must download the latest version of the complete AAAI Author Kit so that you will have the latest instruction set and style file.

\subsection{Document Preamble}

In the \LaTeX{} source for your paper, you \textbf{must} place the following lines as shown in the example in this subsection. This command set-up is for three authors. Add or subtract author and address lines as necessary, and uncomment the portions that apply to you. In most instances, this is all you need to do to format your paper in the Times font. The helvet package will cause Helvetica to be used for sans serif. These files are part of the PSNFSS2e package, which is freely available from many Internet sites (and is often part of a standard installation).

Leave the setcounter for section number depth commented out and set at 0 unless you want to add section numbers to your paper. If you do add section numbers, you must uncomment this line and change the number to 1 (for section numbers), or 2 (for section and subsection numbers). The style file will not work properly with numbering of subsubsections, so do not use a number higher than 2.

\subsubsection{The Following Must Appear in Your Preamble}
\begin{quote}
\begin{scriptsize}\begin{verbatim}
\def\year{2021}\relax
% File: formatting-instruction.tex
\documentclass[letterpaper]{article} % DO NOT CHANGE THIS
\usepackage{aaai21} % DO NOT CHANGE THIS
\usepackage{times} % DO NOT CHANGE THIS
\usepackage{helvet} % DO NOT CHANGE THIS
\usepackage{courier} % DO NOT CHANGE THIS
\usepackage[hyphens]{url} % DO NOT CHANGE THIS
\usepackage{graphicx} % DO NOT CHANGE THIS
\urlstyle{rm} % DO NOT CHANGE THIS
\def\UrlFont{\rm} % DO NOT CHANGE THIS
\usepackage{graphicx}  % DO NOT CHANGE THIS
\usepackage{natbib} % DO NOT CHANGE THIS OR ADD OPTIONS
\usepackage{caption} % DO NOT CHANGE THIS OR ADD OPTIONS
\frenchspacing % DO NOT CHANGE THIS
\setlength{\pdfpagewidth}{8.5in} % DO NOT CHANGE THIS
\setlength{\pdfpageheight}{11in} % DO NOT CHANGE THIS
%
% PDF Info Is REQUIRED.
% For /Author, add all authors within the parentheses,
% separated by commas. No accents or commands.
% For /Title, add Title in Mixed Case.
% No accents or commands. Retain the parentheses.
\pdfinfo{
/Title (AAAI Press Formatting Instructions for Authors
Using LaTeX -- A Guide)
/Author (AAAI Press Staff, Pater Patel Schneider,
Sunil Issar, J. Scott Penberthy, George Ferguson,
Hans Guesgen, Francisco Cruz, Marc Pujol-Gonzalez)
/TemplateVersion (2021.1)
}
\end{verbatim}\end{scriptsize}
\end{quote}

\subsection{Preparing Your Paper}

After the preamble above, you should prepare your paper as follows:
\begin{quote}
\begin{scriptsize}\begin{verbatim}
%
\begin{document}
\maketitle
\begin{abstract}
%...
\end{abstract}\end{verbatim}\end{scriptsize}
\end{quote}

\noindent You should then continue with the body of your paper. Your paper must conclude with the references, which should be inserted as follows:
\begin{quote}
\begin{scriptsize}\begin{verbatim}
% References and End of Paper
% These lines must be placed at the end of your paper
\bibliography{Bibliography-File}
\end{document}
\end{verbatim}\end{scriptsize}
\end{quote}

\subsection{Inserting Document Metadata with \LaTeX{}}
PDF files contain document summary information that enables us to create an Acrobat index (pdx) file, and also allows search engines to locate and present your paper more accurately. \textit{Document metadata for author and title are REQUIRED.} You may not apply any script or macro to implementation of the title, author, and metadata information in your paper.

\textit{Important:} Do not include \textit{any} \LaTeX{} code or nonascii characters (including accented characters) in the metadata. The data in the metadata must be completely plain ascii. It may not include slashes, accents, linebreaks, unicode, or any \LaTeX{} commands. Type the title exactly as it appears on the paper (minus all formatting). Input the author names in the order in which they appear on the paper (minus all accents), separating each author by a comma. You may also include keywords in the optional Keywords field.

\begin{quote}
\begin{scriptsize}\begin{verbatim}
\begin{document}\\
\maketitle\\
...\\
\bibliography{Bibliography-File}\\
\end{document}\\
\end{verbatim}\end{scriptsize}
\end{quote}

\subsection{Commands and Packages That May Not Be Used}
\begin{table*}[t]
\centering

\begin{tabular}{l|l|l|l}
\textbackslash abovecaption &
\textbackslash abovedisplay &
\textbackslash addevensidemargin &
\textbackslash addsidemargin \\
\textbackslash addtolength &
\textbackslash baselinestretch &
\textbackslash belowcaption &
\textbackslash belowdisplay \\
\textbackslash break &
\textbackslash clearpage &
\textbackslash clip &
\textbackslash columnsep \\
\textbackslash float &
\textbackslash input &
\textbackslash input &
\textbackslash linespread \\
\textbackslash newpage &
\textbackslash pagebreak &
\textbackslash renewcommand &
\textbackslash setlength \\
\textbackslash text height &
\textbackslash tiny &
\textbackslash top margin &
\textbackslash trim \\
\textbackslash vskip\{- &
\textbackslash vspace\{- \\
\end{tabular}
%}
\caption{Commands that must not be used}
\label{table1}
\end{table*}

\begin{table}[t]
\centering
%\resizebox{.95\columnwidth}{!}{
\begin{tabular}{l|l|l|l}
    authblk & babel & cjk & dvips \\
    epsf & epsfig & euler & float \\
    fullpage & geometry & graphics & hyperref \\
    layout & linespread & lmodern & maltepaper \\
    navigator & pdfcomment & pgfplots & psfig \\
    pstricks & t1enc & titlesec & tocbind \\
    ulem
\end{tabular}
\caption{LaTeX style packages that must not be used.}
\label{table2}
\end{table}

There are a number of packages, commands, scripts, and macros that are incompatable with aaai21.sty. The common ones are listed in tables \ref{table1} and \ref{table2}. Generally, if a command, package, script, or macro alters floats, margins, fonts, sizing, linespacing, or the presentation of the references and citations, it is unacceptable. Note that negative vskip and vspace may not be used except in certain rare occurances, and may never be used around tables, figures, captions, sections, subsections, subsections, or references.


\subsection{Page Breaks}
For your final camera ready copy, you must not use any page break commands. References must flow directly after the text without breaks. Note that some conferences require references to be on a separate page during the review process. AAAI Press, however, does not require this condition for the final paper.


\subsection{Paper Size, Margins, and Column Width}
Papers must be formatted to print in two-column format on 8.5 x 11 inch US letter-sized paper. The margins must be exactly as follows:
\begin{itemize}
\item Top margin: .75 inches
\item Left margin: .75 inches
\item Right margin: .75 inches
\item Bottom margin: 1.25 inches
\end{itemize}


The default paper size in most installations of \LaTeX{} is A4. However, because we require that your electronic paper be formatted in US letter size, the preamble we have provided includes commands that alter the default to US letter size. Please note that using any other package to alter page size (such as, but not limited to the Geometry package) will result in your final paper being returned to you for correction.


\subsubsection{Column Width and Margins.}
To ensure maximum readability, your paper must include two columns. Each column should be 3.3 inches wide (slightly more than 3.25 inches), with a .375 inch (.952 cm) gutter of white space between the two columns. The aaai21.sty file will automatically create these columns for you.

\subsection{Overlength Papers}
If your paper is too long and you resort to formatting tricks to make it fit, it is quite likely that it will be returned to you. The best way to retain readability if the paper is overlength is to cut text, figures, or tables. There are, a few acceptable ways to reduce paper size that don't affect readability. First, turn on \textbackslash frenchspacing, which will reduce the space after periods. Next, move all your figures and tables to the top of the page. Consider removing less important portions of a figure. If you use \textbackslash centering instead of \textbackslash begin\{center\} in your figure environment, you can also buy some space. For mathematical environments, you may reduce fontsize {\bf but not below 6.5 point}.


Commands that alter page layout are forbidden. These include \textbackslash columnsep,  \textbackslash float, \textbackslash topmargin, \textbackslash topskip, \textbackslash textheight, \textbackslash textwidth, \textbackslash oddsidemargin, and \textbackslash evensizemargin (this list is not exhaustive). If you alter page layout, you will be required to pay the page fee. Other commands that are questionable and may cause your paper to be rejected include \textbackslash parindent, and \textbackslash parskip. Commands that alter the space between sections are forbidden. The title sec package is not allowed. Regardless of the above, if your paper is obviously ``squeezed" it is not going to to be accepted. Options for reducing the length of a paper include reducing the size of your graphics, cutting text, or paying the extra page charge (if it is offered).


\subsection{Type Font and Size}
Your paper must be formatted in Times Roman or Nimbus. We will not accept papers formatted using Computer Modern or Palatino or some other font as the text or heading typeface. Sans serif, when used, should be Courier. Use Symbol or Lucida or Computer Modern for \textit{mathematics only. }

Do not use type 3 fonts for any portion of your paper, including graphics. Type 3 bitmapped fonts are designed for fixed resolution printers. Most print at 300 dpi even if the printer resolution is 1200 dpi or higher. They also often cause high resolution imagesetter devices to crash. Consequently, AAAI will not accept electronic files containing obsolete type 3 fonts. Files containing those fonts (even in graphics) will be rejected. (Authors using blackboard symbols must be avoid those packages that use type 3 fonts.)

Fortunately, there are effective workarounds that will prevent your file from embedding type 3 bitmapped fonts. The easiest workaround is to use the required times, helvet, and courier packages with \LaTeX{}2e. (Note that papers formatted in this way will still use Computer Modern for the mathematics. To make the math look good, you'll either have to use Symbol or Lucida, or you will need to install type 1 Computer Modern fonts --- for more on these fonts, see the section ``Obtaining Type 1 Computer Modern.")

If you are unsure if your paper contains type 3 fonts, view the PDF in Acrobat Reader. The Properties/Fonts window will display the font name, font type, and encoding properties of all the fonts in the document. If you are unsure if your graphics contain type 3 fonts (and they are PostScript or encapsulated PostScript documents), create PDF versions of them, and consult the properties window in Acrobat Reader.

The default size for your type must be ten-point with twelve-point leading (line spacing). Start all pages (except the first) directly under the top margin. (See the next section for instructions on formatting the title page.) Indent ten points when beginning a new paragraph, unless the paragraph begins directly below a heading or subheading.


\subsubsection{Obtaining Type 1 Computer Modern for \LaTeX{}.}

If you use Computer Modern for the mathematics in your paper (you cannot use it for the text) you may need to download type 1 Computer fonts. They are available without charge from the American Mathematical Society:
http://www.ams.org/tex/type1-fonts.html.

\subsubsection{Nonroman Fonts.}
If your paper includes symbols in other languages (such as, but not limited to, Arabic, Chinese, Hebrew, Japanese, Thai, Russian and other Cyrillic languages), you must restrict their use to bit-mapped figures.

\subsection{Title and Authors}
Your title must appear in mixed case (nouns, pronouns, and verbs are capitalized) near the top of the first page, centered over both columns in sixteen-point bold type (twenty-four point leading). This style is called ``mixed case," which means that means all verbs (including short verbs like be, is, using, and go), nouns, adverbs, adjectives, and pronouns should be capitalized, (including both words in hyphenated terms), while articles, conjunctions, and prepositions are lower case unless they directly follow a colon or long dash. Author's names should appear below the title of the paper, centered in twelve-point type (with fifteen point leading), along with affiliation(s) and complete address(es) (including electronic mail address if available) in nine-point roman type (the twelve point leading). (If the title is long, or you have many authors, you may reduce the specified point sizes by up to two points.) You should begin the two-column format when you come to the abstract.

\subsubsection{Formatting Author Information.}
Author information has to be set according the following specification depending if you have one or more than one affiliation.  You may not use a table nor may you employ the \textbackslash authorblk.sty package. For one or several authors from the same institution, please just separate with commas and write the affiliation directly below using the macros \textbackslash author and \textbackslash affiliations:

\begin{quote}\begin{scriptsize}\begin{verbatim}
\author{
    Author 1, ..., Author n \\
}
\affiliations {
    Address line \\
    ... \\
    Address line
}
\end{verbatim}\end{scriptsize}\end{quote}


\noindent For authors from different institutions, use \textbackslash textsuperscript \{\textbackslash rm x \} to match authors and affiliations.

\begin{quote}\begin{scriptsize}\begin{verbatim}
\author{
    AuthorOne,\textsuperscript{\rm 1}
    AuthorTwo,\textsuperscript{\rm 2}
    AuthorThree,\textsuperscript{\rm 3}
    AuthorFour,\textsuperscript{\rm 4}
    AuthorFive \textsuperscript{\rm 5}}\\
}
\affiliations {
    \textsuperscript{\rm 1}AffiliationOne, \\
    \textsuperscript{\rm 2}AffiliationTwo, \\
    \textsuperscript{\rm 3}AffiliationThree, \\
    \textsuperscript{\rm 4}AffiliationFour, \\
    \textsuperscript{\rm 5}AffiliationFive \\
    \{email, email\}@affiliation.com,
    email@affiliation.com,
    email@affiliation.com,
    email@affiliation.com
}
\end{verbatim}\end{scriptsize}\end{quote}

Note that you may want to  break the author list for better visualization. You can achieve this using a simple line break (\textbackslash  \textbackslash).

\subsection{\LaTeX{} Copyright Notice}
The copyright notice automatically appears if you use aaai21.sty. It has been hardcoded and may not be disabled.

\subsection{Credits}
Any credits to a sponsoring agency should appear in the acknowledgments section, unless the agency requires different placement. If it is necessary to include this information on the front page, use
\textbackslash thanks in either the \textbackslash author or \textbackslash title commands.
For example:
\begin{quote}
\begin{small}
\textbackslash title\{Very Important Results in AI\textbackslash thanks\{This work is
 supported by everybody.\}\}
\end{small}
\end{quote}
Multiple \textbackslash thanks commands can be given. Each will result in a separate footnote indication in the author or title with the corresponding text at the botton of the first column of the document. Note that the \textbackslash thanks command is fragile. You will need to use \textbackslash protect.

Please do not include \textbackslash pubnote commands in your document.

\subsection{Abstract}
Follow the example commands in this document for creation of your abstract. The command \textbackslash begin\{abstract\} will automatically indent the text block. Please do not indent it further. {Do not include references in your abstract!}

\subsection{Page Numbers}

Do not \textbf{ever} print any page numbers on your paper. The use of \textbackslash pagestyle is forbidden.

\subsection{Text }
The main body of the paper must be formatted in black, ten-point Times Roman with twelve-point leading (line spacing). You may not reduce font size or the linespacing. Commands that alter font size or line spacing (including, but not limited to baselinestretch, baselineshift, linespread, and others) are expressly forbidden. In addition, you may not use color in the text.

\subsection{Citations}
Citations within the text should include the author's last name and year, for example (Newell 1980). Append lower-case letters to the year in cases of ambiguity. Multiple authors should be treated as follows: (Feigenbaum and Engelmore 1988) or (Ford, Hayes, and Glymour 1992). In the case of four or more authors, list only the first author, followed by et al. (Ford et al. 1997).

\subsection{Extracts}
Long quotations and extracts should be indented ten points from the left and right margins.

\begin{quote}
This is an example of an extract or quotation. Note the indent on both sides. Quotation marks are not necessary if you offset the text in a block like this, and properly identify and cite the quotation in the text.

\end{quote}

\subsection{Footnotes}
Avoid footnotes as much as possible; they interrupt the reading of the text. When essential, they should be consecutively numbered throughout with superscript Arabic numbers. Footnotes should appear at the bottom of the page, separated from the text by a blank line space and a thin, half-point rule.

\subsection{Headings and Sections}
When necessary, headings should be used to separate major sections of your paper. Remember, you are writing a short paper, not a lengthy book! An overabundance of headings will tend to make your paper look more like an outline than a paper. The aaai21.sty package will create headings for you. Do not alter their size nor their spacing above or below.

\subsubsection{Section Numbers.}
The use of section numbers in AAAI Press papers is optional. To use section numbers in \LaTeX{}, uncomment the setcounter line in your document preamble and change the 0 to a 1 or 2. Section numbers should not be used in short poster papers.

\subsubsection{Section Headings.}
Sections should be arranged and headed as follows:

\subsubsection{Acknowledgments.}
The acknowledgments section, if included, appears after the main body of text and is headed ``Acknowledgments." This section includes acknowledgments of help from associates and colleagues, credits to sponsoring agencies, financial support, and permission to publish. Please acknowledge other contributors, grant support, and so forth, in this section. Do not put acknowledgments in a footnote on the first page. If your grant agency requires acknowledgment of the grant on page 1, limit the footnote to the required statement, and put the remaining acknowledgments at the back. Please try to limit acknowledgments to no more than three sentences.

\subsubsection{Appendices.}
Any appendices follow the acknowledgments, if included, or after the main body of text if no acknowledgments appear.

\subsubsection{References.}
The references section should be labeled ``References" and should appear at the very end of the paper (don't end the paper with references, and then put a figure by itself on the last page). A sample list of references is given later on in these instructions. Please use a consistent format for references. Poorly prepared or sloppy references reflect badly on the quality of your paper and your research. Please prepare complete and accurate citations.

\subsection{Illustrations and  Figures}

\begin{figure}[t]
\centering
\includegraphics[width=0.9\columnwidth]{figure1} % Reduce the figure size so that it is slightly narrower than the column. Don't use precise values for figure width.This setup will avoid overfull boxes.
\caption{Using the trim and clip commands produces fragile layers that can result in disasters (like this one from an actual paper) when the color space is corrected or the PDF combined with others for the final proceedings. Crop your figures properly in a graphics program -- not in LaTeX}.
\label{fig1}
\end{figure}

\begin{figure*}[t]
\centering
\includegraphics[width=0.8\textwidth]{figure2} % Reduce the figure size so that it is slightly narrower than the column.
\caption{Adjusting the bounding box instead of actually removing the unwanted data resulted multiple layers in this paper. It also needlessly increased the PDF size. In this case, the size of the unwanted layer doubled the paper's size, and produced the following surprising results in final production. Crop your figures properly in a graphics program. Don't just alter the bounding box.}
\label{fig2}
\end{figure*}

% Using the \centering command instead of \begin{center} ... \end{center} will save space
% Positioning your figure at the top of the page will save space and make the paper more readable
% Using 0.95\columnwidth in conjunction with the


Your paper must compile in PDF\LaTeX{}. Consequently, all your figures must be .jpg, .png, or .pdf. You may not use the .gif (the resolution is too low), .ps, or .eps file format for your figures.

Figures, drawings, tables, and photographs should be placed throughout the paper on the page (or the subsequent page) where they are first discussed. Do not group them together at the end of the paper. If placed at the top of the paper, illustrations may run across both columns. Figures must not invade the top, bottom, or side margin areas. Figures must be inserted using the \textbackslash usepackage\{graphicx\}. Number figures sequentially, for example, figure 1, and so on. Do not use minipage to group figures.

If you normally create your figures using pgfplots, please create the figures first, and then import them as pdfs with proper bounding boxes, as the bounding and trim boxes created by pfgplots are fragile and not valid.

When you include your figures, you must crop them \textbf{outside} of \LaTeX{}. The command \textbackslash includegraphics*[clip=true, viewport 0 0 10 10]{...} might result in a PDF that looks great, but the image is \textbf{not really cropped.} The full image can reappear (and obscure whatever it is overlapping) when page numbers are applied or color space is standardized. Figures \ref{fig1}, and \ref{fig2} display some unwanted results that often occur.

If your paper includes illustrations that are not compatible with PDF\TeX{} (such as .eps or .ps documents), you will need to convert them. The epstopdf package will usually work for eps files. You will need to convert your ps files to PDF however.

\subsubsection {Figure Captions.}The illustration number and caption must appear \textit{under} the illustration. Labels, and other text with the actual illustration must be at least nine-point type. However, the font and size of figure captions must be 10 point roman. Do not make them smaller, bold, or italic. (Individual words may be italicized if the context requires differentiation.)

\subsection{Tables}

Tables should be presented in 10 point roman type. If necessary, they may be altered to 9 point type. You may not use any commands that further reduce point size below nine points. Tables that do not fit in a single column must be placed across double columns. If your table won't fit within the margins even when spanning both columns, you must split it. Do not use minipage to group tables.

\subsubsection {Table Captions.} The number and caption for your table must appear \textit{under} (not above) the table.  Additionally, the font and size of table captions must be 10 point roman and must be placed beneath the figure. Do not make them smaller, bold, or italic. (Individual words may be italicized if the context requires differentiation.)



\subsubsection{Low-Resolution Bitmaps.}
You may not use low-resolution (such as 72 dpi) screen-dumps and GIF files---these files contain so few pixels that they are always blurry, and illegible when printed. If they are color, they will become an indecipherable mess when converted to black and white. This is always the case with gif files, which should never be used. The resolution of screen dumps can be increased by reducing the print size of the original file while retaining the same number of pixels. You can also enlarge files by manipulating them in software such as PhotoShop. Your figures should be 300 dpi when incorporated into your document.

\subsubsection{\LaTeX{} Overflow.}
\LaTeX{} users please beware: \LaTeX{} will sometimes put portions of the figure or table or an equation in the margin. If this happens, you need to make the figure or table span both columns. If absolutely necessary, you may reduce the figure, or reformat the equation, or reconfigure the table.{ \bf Check your log file!} You must fix any overflow into the margin (that means no overfull boxes in \LaTeX{}). \textbf{Nothing is permitted to intrude into the margin or gutter.}


\subsubsection{Using Color.}
Use of color is restricted to figures only. It must be WACG 2.0 compliant. (That is, the contrast ratio must be greater than 4.5:1 no matter the font size.) It must be CMYK, NOT RGB. It may never be used for any portion of the text of your paper. The archival version of your paper will be printed in black and white and grayscale. The web version must be readable by persons with disabilities. Consequently, because conversion to grayscale can cause undesirable effects (red changes to black, yellow can disappear, and so forth), we strongly suggest you avoid placing color figures in your document. If you do include color figures, you must (1) use the CMYK (not RGB) colorspace and (2) be mindful of readers who may happen to have trouble distinguishing colors. Your paper must be decipherable without using color for distinction.

\subsubsection{Drawings.}
We suggest you use computer drawing software (such as Adobe Illustrator or, (if unavoidable), the drawing tools in Microsoft Word) to create your illustrations. Do not use Microsoft Publisher. These illustrations will look best if all line widths are uniform (half- to two-point in size), and you do not create labels over shaded areas. Shading should be 133 lines per inch if possible. Use Times Roman or Helvetica for all figure call-outs. \textbf{Do not use hairline width lines} --- be sure that the stroke width of all lines is at least .5 pt. Zero point lines will print on a laser printer, but will completely disappear on the high-resolution devices used by our printers.

\subsubsection{Photographs and Images.}
Photographs and other images should be in grayscale (color photographs will not reproduce well; for example, red tones will reproduce as black, yellow may turn to white, and so forth) and set to a minimum of 300 dpi. Do not prescreen images.

\subsubsection{Resizing Graphics.}
Resize your graphics \textbf{before} you include them with LaTeX. You may \textbf{not} use trim or clip options as part of your \textbackslash includegraphics command. Resize the media box of your PDF using a graphics program instead.

\subsubsection{Fonts in Your Illustrations.}
You must embed all fonts in your graphics before including them in your LaTeX document.

\subsection{References}
The AAAI style includes a set of definitions for use in formatting references with BibTeX. These definitions make the bibliography style fairly close to the one specified below. To use these definitions, you also need the BibTeX style file ``aaai21.bst," available in the AAAI Author Kit on the AAAI web site. Then, at the end of your paper but before \textbackslash end{document}, you need to put the following lines:

\begin{quote}
\begin{small}
\textbackslash bibliography\{bibfile1,bibfile2,...\}
\end{small}
\end{quote}

Please note that the aaai21.sty class already sets the bibliographystyle for you, so you do not have to place any \textbackslash bibliographystyle command in the document yourselves. The aaai21.sty file is incompatible with the hyperref and navigator packages. If you use either, your references will be garbled and your paper will be returned to you.

References may be the same size as surrounding text. However, in this section (only), you may reduce the size to \textbackslash small if your paper exceeds the allowable number of pages. Making it any smaller than 9 point with 10 point linespacing, however, is not allowed. A more precise and exact method of reducing the size of your references minimally is by means of the following command: \begin{quote}
\textbackslash fontsize\{9.8pt\}\{10.8pt\}
\textbackslash selectfont\end{quote}

\noindent You must reduce the size equally for both font size and line spacing, and may not reduce the size beyond \{9.0pt\}\{10.0pt\}.

The list of files in the \textbackslash bibliography command should be the names of your BibTeX source files (that is, the .bib files referenced in your paper).

The following commands are available for your use in citing references:
\begin{quote}
{\em \textbackslash cite:} Cites the given reference(s) with a full citation. This appears as ``(Author Year)'' for one reference, or ``(Author Year; Author Year)'' for multiple references.\smallskip\\
{\em \textbackslash shortcite:} Cites the given reference(s) with just the year. This appears as ``(Year)'' for one reference, or ``(Year; Year)'' for multiple references.\smallskip\\
{\em \textbackslash citeauthor:} Cites the given reference(s) with just the author name(s) and no parentheses.\smallskip\\
{\em \textbackslash citeyear:} Cites the given reference(s) with just the date(s) and no parentheses.
\end{quote}



Formatted bibliographies should look like the following examples.

\smallskip \noindent \textit{Book with Multiple Authors}\\
Engelmore, R., and Morgan, A. eds. 1986. \textit{Blackboard Systems.} Reading, Mass.: Addison-Wesley.

\smallskip \noindent \textit{Journal Article}\\
Robinson, A. L. 1980a. New Ways to Make Microcircuits Smaller. \textit{Science} 208: 1019--1026.

\smallskip \noindent \textit{Magazine Article}\\
Hasling, D. W.; Clancey, W. J.; and Rennels, G. R. 1983. Strategic Explanations in Consultation. \textit{The International Journal of Man-Machine Studies} 20(1): 3--19.

\smallskip \noindent \textit{Proceedings Paper Published by a Society}\\
Clancey, W. J. 1983. Communication, Simulation, and Intelligent Agents: Implications of Personal Intelligent Machines for Medical Education. In \textit{Proceedings of the Eighth International Joint Conference on Artificial Intelligence,} 556--560. Menlo Park, Calif.: International Joint Conferences on Artificial Intelligence, Inc.

\smallskip \noindent \textit{Proceedings Paper Published by a Press or Publisher}\\
Clancey, W. J. 1984. Classification Problem Solving. In \textit{Proceedings of the Fourth National Conference on Artificial Intelligence,} 49--54. Menlo Park, Calif.: AAAI Press.

\smallskip \noindent \textit{University Technical Report}\\
Rice, J. 1986. Poligon: A System for Parallel Problem Solving, Technical Report, KSL-86-19, Dept. of Computer Science, Stanford Univ.

\smallskip \noindent \textit{Dissertation or Thesis}\\
Clancey, W. J. 1979. Transfer of Rule-Based Expertise through a Tutorial Dialogue. Ph.D. diss., Dept. of Computer Science, Stanford Univ., Stanford, Calif.

\smallskip \noindent \textit{Forthcoming Publication}\\
Clancey, W. J. 2021. The Engineering of Qualitative Models. Forthcoming.

For the most up to date version of the AAAI reference style, please consult the \textit{AI Magazine} Author Guidelines at \url{https://aaai.org/ojs/index.php/aimagazine/about/submissions#authorGuidelines}



\section{Proofreading Your PDF}
Please check all the pages of your PDF file. The most commonly forgotten element is the acknowledgements --- especially the correct grant number. Authors also commonly forget to add the metadata to the source, use the wrong reference style file, or don't follow the capitalization rules or comma placement for their author-title information properly. A final common problem is text (expecially equations) that runs into the margin. You will need to fix these common errors before submitting your file.

\section{Improperly Formatted Files }
In the past, AAAI has corrected improperly formatted files submitted by the authors. Unfortunately, this has become an increasingly burdensome expense that we can no longer absorb). Consequently, if your file is improperly formatted, it will be returned to you for correction.

\subsection{\LaTeX{} 209 Warning}
If you use \LaTeX{} 209 your paper will be returned to you unpublished. Convert your paper to \LaTeX{}2e.

\section{Naming Your Electronic File}
We require that you name your \LaTeX{} source file with the last name (family name) of the first author so that it can easily be differentiated from other submissions. Complete file-naming instructions will be provided to you in the submission instructions.

\section{Submitting Your Electronic Files to AAAI}
Instructions on paper submittal will be provided to you in your acceptance letter.

\section{Inquiries}
If you have any questions about the preparation or submission of your paper as instructed in this document, please contact AAAI Press at the address given below. If you have technical questions about implementation of the aaai style file, please contact an expert at your site. We do not provide technical support for \LaTeX{} or any other software package. To avoid problems, please keep your paper simple, and do not incorporate complicated macros and style files.

\begin{quote}
\noindent AAAI Press\\
2275 East Bayshore Road, Suite 160\\
Palo Alto, California 94303\\
\textit{Telephone:} (650) 328-3123\\
\textit{E-mail:} See the submission instructions for your particular conference or event.
\end{quote}

\section{Additional Resources}
\LaTeX{} is a difficult program to master. If you've used that software, and this document didn't help or some items were not explained clearly, we recommend you read Michael Shell's excellent document (testflow doc.txt V1.0a 2002/08/13) about obtaining correct PS/PDF output on \LaTeX{} systems. (It was written for another purpose, but it has general application as well). It is available at www.ctan.org in the tex-archive.

\section{ Acknowledgments}
AAAI is especially grateful to Peter Patel Schneider for his work in implementing the original aaai.sty file, liberally using the ideas of other style hackers, including Barbara Beeton. We also acknowledge with thanks the work of George Ferguson for his guide to using the style and BibTeX files --- which has been incorporated into this document --- and Hans Guesgen, who provided several timely modifications, as well as the many others who have, from time to time, sent in suggestions on improvements to the AAAI style. We are especially grateful to Francisco Cruz, Marc Pujol-Gonzalez, and Mico Loretan for the improvements to the Bib\TeX{} and \LaTeX{} files made in 2020.

The preparation of the \LaTeX{} and Bib\TeX{} files that implement these instructions was supported by Schlumberger Palo Alto Research, AT\&T Bell Laboratories, Morgan Kaufmann Publishers, The Live Oak Press, LLC, and AAAI Press. Bibliography style changes were added by Sunil Issar. \verb+\+pubnote was added by J. Scott Penberthy. George Ferguson added support for printing the AAAI copyright slug. Additional changes to aaai21.sty and aaai21.bst have been made by Francisco Cruz, Marc Pujol-Gonzalez, and Mico Loretan.

\bigskip
\noindent Thank you for reading these instructions carefully. We look forward to receiving your electronic files!

\end{document}

\bibstyle{aaai21}




\end{document}


