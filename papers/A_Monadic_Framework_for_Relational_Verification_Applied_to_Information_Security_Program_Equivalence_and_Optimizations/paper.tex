% Needed for proper arXiv processing, it seems
\pdfoutput=1

% These flags are for the CPP submission
% !!! PLEASE DON'T CHANGE THESE !!! INSTEAD DEFINE YOUR OWN texdirectives.tex !!!
\newif\ifdraft\draftfalse  % draft = comments
\newif\ifsubmit\submitfalse % anon = CPP submissions are not anonymous
\newif\iffull\fullfalse   % full = includes things that were cut from
                          % conf. proceedings only because of space
                          % might be turned true for a journal submission
\newif\iflongrefs\longrefsfalse % Long references (e.g. for journal)
\newif\ifbackref\backreffalse % backref option for hyperref
                              % useful for shrinking references
\newif\ifsooner\soonerfalse
\newif\iflater\laterfalse
\newif\ifcamera\cameratrue    % Camera-ready version
\newif\ifcheckpagebudget\checkpagebudgetfalse
% !!! PLEASE DON'T CHANGE THESE !!! INSTEAD DEFINE YOUR OWN texdirectives.tex !!!
\newcommand{\xxx}{}
\makeatletter \@input{texdirectives.tex} \makeatother

%; whizzy -pdf xpdf

\ifcamera
%% For final camera-ready submission
\documentclass[sigplan,screen]{acmart}\settopmatter{}

 \begin{CCSXML}
<ccs2012>
<concept>
<concept_id>10003752.10010124.10010138.10010142</concept_id>
<concept_desc>Theory of computation~Program verification</concept_desc>
<concept_significance>500</concept_significance>
</concept>
<concept>
<concept_id>10002978.10003006.10011608</concept_id>
<concept_desc>Security and privacy~Information flow control</concept_desc>
<concept_significance>500</concept_significance>
</concept>
</ccs2012>
\end{CCSXML}

\ccsdesc[500]{Theory of computation~Program verification}
\ccsdesc[500]{Security and privacy~Information flow control}

\else
\ifsubmit
%% For submission
\documentclass[sigplan,10pt,review]{acmart}\settopmatter{printfolios=true,printccs=false,printacmref=false}
\else
%% For technical report
\documentclass[sigplan,10pt]{acmart}\settopmatter{}
\fi
\fi

\keywords{
  Relational Verification,
  Monadic Effects,
  Proof Assistants,
  Program Verification,
  SMT-based Automation,
  Weakest Preconditions,
  Information-Flow Control,
  Program Equivalence and Refinement,
  Certified Optimizations
}


\copyrightyear{2018}
\acmYear{2018}
\ifcamera\setcopyright{acmlicensed}\else\setcopyright{rightsretained}\fi
\acmConference[CPP'18]{7th ACM SIGPLAN International Conference on Certified Programs and Proofs}{January 8--9, 2018}{Los Angeles, CA, USA}
\acmBooktitle{CPP '18: CPP '18: Certified Proofs and Programs , January 8--9, 2018, Los Angeles, CA, USA}
\acmPrice{15.00}
\acmDOI{10.1145/3167090}
\acmISBN{978-1-4503-5586-5/18/01}

% \makeatletter\@ACM@natbibtrue\makeatother

\ifcamera\else
\makeatletter
\def\@copyrightpermission{This work is licensed under a \href{https://creativecommons.org/licenses/by/4.0/}{Creative Commons Attribution 4.0 International License}}
\makeatother
\fi

\ifcamera\else
\makeatletter
\def\@authorsaddresses{}
\fancypagestyle{firstpagestyle}{%
  \fancyhf{}%
  \renewcommand{\headrulewidth}{\z@}%
  \renewcommand{\footrulewidth}{\z@}%
    \fancyhead[L]{\ifsubmit\ACM@linecountL\fi}%
    % \fancyhead[R]{\ACM@linecountR}%
    % \fancyhead[LE]{}%
    % \fancyhead[RO]{}%
    % \fancyhead[RE]{}%
    % \fancyhead[LO]{}%
    % \fancyfoot[RO,LE]{}%
}
\fancypagestyle{standardpagestyle}{%
  \fancyhf{}%
  \renewcommand{\headrulewidth}{\z@}%
  \renewcommand{\footrulewidth}{\z@}%
    \fancyhead[LE]{\ifsubmit\ACM@linecountL\fi\@headfootfont\thepage}%
    \fancyhead[RO]{\@headfootfont\thepage}%
    \fancyhead[RE]{\@headfootfont\@shortauthors}%
    \fancyhead[LO]{\ifsubmit\ACM@linecountL\fi\@headfootfont\shorttitle}%
    \fancyfoot[RO,LE]{}%
}
\pagestyle{standardpagestyle}
\def\@mkbibcitation{}
\makeatother
\fi


%% Bibliography style
%% \bibliographystyle{ACM-Reference-Format}
%% Citation style
%% Note: author/year citations are required for papers published as an
%% issue of PACMPL.
%% \citestyle{acmauthoryear}   %% For author/year citations


% Gaining some space on figure captions
\setlength{\abovecaptionskip}{4pt} % default: 12.0pt
\setlength{\belowcaptionskip}{0pt} % default: 0.0pt
\setlength{\textfloatsep}{10pt plus 8.0pt minus 5.0pt} % default: 15pt

%% ** Packages
\usepackage[justification=centering]{caption}
\usepackage{afterpage}
\usepackage{color}
\usepackage{amssymb,amsmath,amsfonts}
\usepackage{extarrows}
%\usepackage{mathptmx} -- BROKEN! Makes parens in math mode dissapear!
\usepackage{version}
\usepackage{xspace}
\usepackage{graphicx}
\usepackage{listings}
\usepackage{wrapfig}
\usepackage{ifpdf}
\usepackage{semantic}
\usepackage{mathpartir} % this takes over some things from semantic
\usepackage{natbib}
\setcitestyle{numbers}
% \setcitestyle{numbers,sort&compress,open={[},close={]},citesep={,},notesep={,}}
\newcommand\citepos[1]{\citeauthor{#1}'s\ \citeyear{#1}}
% \setcitestyle{square}
\bibpunct();A{},
% \let\cite=\citep
\usepackage{amsthm}
\usepackage{stmaryrd}
\usepackage{subfigure}
\usepackage{multirow}
\usepackage{proof}
% \usepackage{mweights} --  Undefined control sequence. \mdseries@rm
\ifcamera
\usepackage{balance} % to balance columns on the last page for final version
\fi
% \usepackage[curve]{xypic}
\definecolor{darkblue}{rgb}{0.0,0.0,0.3}

% Conflicts if passing options
% \usepackage[
%     pdftex,%
%     pdfpagelabels\ifcamera\else,%
%     colorlinks,%
%     linkcolor=darkblue,%
%     citecolor=darkblue,%
%     filecolor=darkblue,%
%     urlcolor=darkblue\fi%
% ]{hyperref}
\usepackage{hyperref}
\hypersetup{breaklinks=true}

\lstloadlanguages{caml}

% 
%\usepackage[OT1,T1]{fontenc}

\usepackage[numbers,sort&compress]{natbib}
\renewcommand{\bibfont}{\footnotesize}
%\usepackage{cite}
%\usepackage{mystyle}
%%%%%%%%%%%%%%%%%%%%%%%%%%%%%%%%%%%%
\makeatletter

\usepackage{etex}

%%% Review %%%

\usepackage{zref-savepos}

\newcounter{mnote}%[page]
\renewcommand{\themnote}{p.\thepage\;$\langle$\arabic{mnote}$\rangle$}

\def\xmarginnote{%
  \xymarginnote{\hskip -\marginparsep \hskip -\marginparwidth}}

\def\ymarginnote{%
  \xymarginnote{\hskip\columnwidth \hskip\marginparsep}}

\long\def\xymarginnote#1#2{%
\vadjust{#1%
\smash{\hbox{{%
        \hsize\marginparwidth
        \@parboxrestore
        \@marginparreset
\footnotesize #2}}}}}

\def\mnoteson{%
\gdef\mnote##1{\refstepcounter{mnote}\label{##1}%
  \zsavepos{##1}%
  \ifnum20432158>\number\zposx{##1}%
  \xmarginnote{{\color{blue}\bf $\langle$\arabic{mnote}$\rangle$}}% 
  \else
  \ymarginnote{{\color{blue}\bf $\langle$\arabic{mnote}$\rangle$}}%
  \fi%
}
  }
\gdef\mnotesoff{\gdef\mnote##1{}}
\mnoteson
\mnotesoff








%%% Layout %%%

% \usepackage{geometry} % override layout
% \geometry{tmargin=2.5cm,bmargin=m2.5cm,lmargin=3cm,rmargin=3cm}
% \setlength{\pdfpagewidth}{8.5in} % overrides default pdftex paper size
% \setlength{\pdfpageheight}{11in}

\newlength{\mywidth}

%%% Conventions %%%

% References
\newcommand{\figref}[1]{Fig.~\ref{#1}}
\newcommand{\defref}[1]{Definition~\ref{#1}}
\newcommand{\tabref}[1]{Table~\ref{#1}}
% general
%\usepackage{ifthen,nonfloat,subfigure,rotating,array,framed}
\usepackage{framed}
%\usepackage{subfigure}
\usepackage{subcaption}
\usepackage{comment}
%\specialcomment{nb}{\begingroup \noindent \framed\textbf{n.b.\ }}{\endframed\endgroup}
%%\usepackage{xtab,arydshln,multirow}
% topcaption defined in xtab. must load nonfloat before xtab
%\PassOptionsToPackage{svgnames,dvipsnames}{xcolor}
\usepackage[svgnames,dvipsnames]{xcolor}
%\definecolor{myblue}{rgb}{.8,.8,1}
%\definecolor{umbra}{rgb}{.8,.8,.5}
%\newcommand*\mybluebox[1]{%
%  \colorbox{myblue}{\hspace{1em}#1\hspace{1em}}}
\usepackage[all]{xy}
%\usepackage{pstricks,pst-node}
\usepackage{tikz}
\usetikzlibrary{positioning,matrix,through,calc,arrows,fit,shapes,decorations.pathreplacing,decorations.markings,decorations.text}

\tikzstyle{block} = [draw,fill=blue!20,minimum size=2em]

% allow prefix to scope name
\tikzset{%
	prefix node name/.code={%
		\tikzset{%
			name/.code={\edef\tikz@fig@name{#1 ##1}}
		}%
	}%
}


\@ifpackagelater{tikz}{2013/12/01}{
	\newcommand{\convexpath}[2]{
		[create hullcoords/.code={
			\global\edef\namelist{#1}
			\foreach [count=\counter] \nodename in \namelist {
				\global\edef\numberofnodes{\counter}
				\coordinate (hullcoord\counter) at (\nodename);
			}
			\coordinate (hullcoord0) at (hullcoord\numberofnodes);
			\pgfmathtruncatemacro\lastnumber{\numberofnodes+1}
			\coordinate (hullcoord\lastnumber) at (hullcoord1);
		}, create hullcoords ]
		($(hullcoord1)!#2!-90:(hullcoord0)$)
		\foreach [evaluate=\currentnode as \previousnode using \currentnode-1,
		evaluate=\currentnode as \nextnode using \currentnode+1] \currentnode in {1,...,\numberofnodes} {
			let \p1 = ($(hullcoord\currentnode) - (hullcoord\previousnode)$),
			\n1 = {atan2(\y1,\x1) + 90},
			\p2 = ($(hullcoord\nextnode) - (hullcoord\currentnode)$),
			\n2 = {atan2(\y2,\x2) + 90},
			\n{delta} = {Mod(\n2-\n1,360) - 360}
			in 
			{arc [start angle=\n1, delta angle=\n{delta}, radius=#2]}
			-- ($(hullcoord\nextnode)!#2!-90:(hullcoord\currentnode)$) 
		}
	}
}{
	\newcommand{\convexpath}[2]{
		[create hullcoords/.code={
			\global\edef\namelist{#1}
			\foreach [count=\counter] \nodename in \namelist {
				\global\edef\numberofnodes{\counter}
				\coordinate (hullcoord\counter) at (\nodename);
			}
			\coordinate (hullcoord0) at (hullcoord\numberofnodes);
			\pgfmathtruncatemacro\lastnumber{\numberofnodes+1}
			\coordinate (hullcoord\lastnumber) at (hullcoord1);
		}, create hullcoords ]
		($(hullcoord1)!#2!-90:(hullcoord0)$)
		\foreach [evaluate=\currentnode as \previousnode using \currentnode-1,
		evaluate=\currentnode as \nextnode using \currentnode+1] \currentnode in {1,...,\numberofnodes} {
			let \p1 = ($(hullcoord\currentnode) - (hullcoord\previousnode)$),
			\n1 = {atan2(\x1,\y1) + 90},
			\p2 = ($(hullcoord\nextnode) - (hullcoord\currentnode)$),
			\n2 = {atan2(\x2,\y2) + 90},
			\n{delta} = {Mod(\n2-\n1,360) - 360}
			in 
			{arc [start angle=\n1, delta angle=\n{delta}, radius=#2]}
			-- ($(hullcoord\nextnode)!#2!-90:(hullcoord\currentnode)$) 
		}
	}
}

% circle around nodes

% typsetting math
\usepackage{qsymbols,amssymb,mathrsfs}
\usepackage{amsmath}
\usepackage[standard,thmmarks]{ntheorem}
\theoremstyle{plain}
\theoremsymbol{\ensuremath{_\vartriangleleft}}
\theorembodyfont{\itshape}
\theoremheaderfont{\normalfont\bfseries}
\theoremseparator{}
\newtheorem{Claim}{Claim}
\newtheorem{Subclaim}{Subclaim}
\theoremstyle{nonumberplain}
\theoremheaderfont{\scshape}
\theorembodyfont{\normalfont}
\theoremsymbol{\ensuremath{_\blacktriangleleft}}
\newtheorem{Subproof}{Proof}

\theoremnumbering{arabic}
\theoremstyle{plain}
\usepackage{latexsym}
\theoremsymbol{\ensuremath{_\Box}}
\theorembodyfont{\itshape}
\theoremheaderfont{\normalfont\bfseries}
\theoremseparator{}
\newtheorem{Conjecture}{Conjecture}

\theorembodyfont{\upshape}
\theoremprework{\bigskip\hrule}
\theorempostwork{\hrule\bigskip}
\newtheorem{Condition}{Condition}%[section]


%\RequirePckage{amsmath} loaded by empheq
\usepackage[overload]{empheq} % no \intertext and \displaybreak
%\usepackage{breqn}

\let\iftwocolumn\if@twocolumn
\g@addto@macro\@twocolumntrue{\let\iftwocolumn\if@twocolumn}
\g@addto@macro\@twocolumnfalse{\let\iftwocolumn\if@twocolumn}

%\empheqset{box=\mybluebox}
%\usepackage{mathtools}      % to polish math typsetting, loaded
%                                % by empeq
\mathtoolsset{showonlyrefs=false,showmanualtags}
\let\underbrace\LaTeXunderbrace % adapt spacing to font sizes
\let\overbrace\LaTeXoverbrace
\renewcommand{\eqref}[1]{\textup{(\refeq{#1})}} % eqref was not allowed in
                                       % sub/super-scripts
\newtagform{brackets}[]{(}{)}   % new tags for equations
\usetagform{brackets}
% defined commands:
% \shortintertext{}, dcases*, \cramped, \smashoperator[]{}

\usepackage[Smaller]{cancel}
\renewcommand{\CancelColor}{\color{Red}}
%\newcommand\hcancel[2][black]{\setbox0=\hbox{#2}% colored horizontal cross
%  \rlap{\raisebox{.45\ht0}{\color{#1}\rule{\wd0}{1pt}}}#2}



\usepackage{graphicx,psfrag}
\graphicspath{{figure/}{image/}} % Search path of figures

% for tabular
\usepackage{diagbox} % \backslashbox{}{} for slashed entries
%\usepackage{threeparttable} % threeparttable, \tnote{},
                                % tablenotes, and \item[]
%\usepackage{colortab} % \cellcolor[gray]{0.9},
%\rowcolor,\columncolor,
%\usepackage{colortab} % \LCC \gray & ...  \ECC \\

% typesetting codes
%\usepackage{maple2e} % need to use \char29 for ^
\usepackage{algorithm2e}
\usepackage{listings} 
\lstdefinelanguage{Maple}{
  morekeywords={proc,module,end, for,from,to,by,while,in,do,od
    ,if,elif,else,then,fi ,use,try,catch,finally}, sensitive,
  morecomment=[l]\#,
  morestring=[b]",morestring=[b]`}[keywords,comments,strings]
\lstset{
  basicstyle=\scriptsize,
  keywordstyle=\color{ForestGreen}\bfseries,
  commentstyle=\color{DarkBlue},
  stringstyle=\color{DimGray}\ttfamily,
  texcl
}
%%% New fonts %%%
\DeclareMathAlphabet{\mathpzc}{OT1}{pzc}{m}{it}
\usepackage{upgreek} % \upalpha,\upbeta, ...
%\usepackage{bbold}   % blackboard math
\usepackage{dsfont}  % \mathds

%%% Macros for multiple definitions %%%

% example usage:
% \multi{M}{\boldsymbol{#1}}  % defines \multiM
% \multi ABC.                 % defines \MA \MB and \MC as
%                             % \boldsymbol{A}, \boldsymbol{B} and
%                             % \boldsymbol{C} respectively.
% 
%  The last period '.' is necessary to terminate the macro expansion.
%
% \multi*{M}{\boldsymbol{#1}} % defines \multiM and \M
% \M{A}                       % expands to \boldsymbol{A}

\def\multi@nostar#1#2{%
  \expandafter\def\csname multi#1\endcsname##1{%
    \if ##1.\let\next=\relax \else
    \def\next{\csname multi#1\endcsname}     
    %\expandafter\def\csname #1##1\endcsname{#2}
    \expandafter\newcommand\csname #1##1\endcsname{#2}
    \fi\next}}

\def\multi@star#1#2{%
  \expandafter\def\csname #1\endcsname##1{#2}
  \multi@nostar{#1}{#2}
}

\newcommand{\multi}{%
  \@ifstar \multi@star \multi@nostar}

%%% new alphabets %%%

\multi*{rm}{\mathrm{#1}}
\multi*{mc}{\mathcal{#1}}
\multi*{op}{\mathop {\operator@font #1}}
% \multi*{op}{\operatorname{#1}}
\multi*{ds}{\mathds{#1}}
\multi*{set}{\mathcal{#1}}
\multi*{rsfs}{\mathscr{#1}}
\multi*{pz}{\mathpzc{#1}}
\multi*{M}{\boldsymbol{#1}}
\multi*{R}{\mathsf{#1}}
\multi*{RM}{\M{\R{#1}}}
\multi*{bb}{\mathbb{#1}}
\multi*{td}{\tilde{#1}}
\multi*{tR}{\tilde{\mathsf{#1}}}
\multi*{trM}{\tilde{\M{\R{#1}}}}
\multi*{tset}{\tilde{\mathcal{#1}}}
\multi*{tM}{\tilde{\M{#1}}}
\multi*{baM}{\bar{\M{#1}}}
\multi*{ol}{\overline{#1}}

\multirm  ABCDEFGHIJKLMNOPQRSTUVWXYZabcdefghijklmnopqrstuvwxyz.
\multiol  ABCDEFGHIJKLMNOPQRSTUVWXYZabcdefghijklmnopqrstuvwxyz.
\multitR   ABCDEFGHIJKLMNOPQRSTUVWXYZabcdefghijklmnopqrstuvwxyz.
\multitd   ABCDEFGHIJKLMNOPQRSTUVWXYZabcdefghijklmnopqrstuvwxyz.
\multitset ABCDEFGHIJKLMNOPQRSTUVWXYZabcdefghijklmnopqrstuvwxyz.
\multitM   ABCDEFGHIJKLMNOPQRSTUVWXYZabcdefghijklmnopqrstuvwxyz.
\multibaM   ABCDEFGHIJKLMNOPQRSTUVWXYZabcdefghijklmnopqrstuvwxyz.
\multitrM   ABCDEFGHIJKLMNOPQRSTUVWXYZabcdefghijklmnopqrstuvwxyz.
\multimc   ABCDEFGHIJKLMNOPQRSTUVWXYZabcdefghijklmnopqrstuvwxyz.
\multiop   ABCDEFGHIJKLMNOPQRSTUVWXYZabcdefghijklmnopqrstuvwxyz.
\multids   ABCDEFGHIJKLMNOPQRSTUVWXYZabcdefghijklmnopqrstuvwxyz.
\multiset  ABCDEFGHIJKLMNOPQRSTUVWXYZabcdefghijklmnopqrstuvwxyz.
\multirsfs ABCDEFGHIJKLMNOPQRSTUVWXYZabcdefghijklmnopqrstuvwxyz.
\multipz   ABCDEFGHIJKLMNOPQRSTUVWXYZabcdefghijklmnopqrstuvwxyz.
\multiM    ABCDEFGHIJKLMNOPQRSTUVWXYZabcdefghijklmnopqrstuvwxyz.
\multiR    ABCDEFGHIJKL NO QR TUVWXYZabcd fghijklmnopqrstuvwxyz.
\multibb   ABCDEFGHIJKLMNOPQRSTUVWXYZabcdefghijklmnopqrstuvwxyz.
\multiRM   ABCDEFGHIJKLMNOPQRSTUVWXYZabcdefghijklmnopqrstuvwxyz.
\newcommand{\RRM}{\R{M}}
\newcommand{\RRP}{\R{P}}
\newcommand{\RRe}{\R{e}}
\newcommand{\RRS}{\R{S}}
%%% new symbols %%%

%\newcommand{\dotgeq}{\buildrel \textstyle  .\over \geq}
%\newcommand{\dotleq}{\buildrel \textstyle  .\over \leq}
\newcommand{\dotleq}{\buildrel \textstyle  .\over {\smash{\lower
      .2ex\hbox{\ensuremath\leqslant}}\vphantom{=}}}
\newcommand{\dotgeq}{\buildrel \textstyle  .\over {\smash{\lower
      .2ex\hbox{\ensuremath\geqslant}}\vphantom{=}}}

\DeclareMathOperator*{\argmin}{arg\,min}
\DeclareMathOperator*{\argmax}{arg\,max}

%%% abbreviations %%%

% commands
\newcommand{\esm}{\ensuremath}

% environments
\newcommand{\bM}{\begin{bmatrix}}
\newcommand{\eM}{\end{bmatrix}}
\newcommand{\bSM}{\left[\begin{smallmatrix}}
\newcommand{\eSM}{\end{smallmatrix}\right]}
\renewcommand*\env@matrix[1][*\c@MaxMatrixCols c]{%
  \hskip -\arraycolsep
  \let\@ifnextchar\new@ifnextchar
  \array{#1}}



% sets of number
\newqsymbol{`N}{\mathbb{N}}
\newqsymbol{`R}{\mathbb{R}}
\newqsymbol{`P}{\mathbb{P}}
\newqsymbol{`Z}{\mathbb{Z}}

% symbol short cut
\newqsymbol{`|}{\mid}
% use \| for \parallel
\newqsymbol{`8}{\infty}
\newqsymbol{`1}{\left}
\newqsymbol{`2}{\right}
\newqsymbol{`6}{\partial}
\newqsymbol{`0}{\emptyset}
\newqsymbol{`-}{\leftrightarrow}
\newqsymbol{`<}{\langle}
\newqsymbol{`>}{\rangle}

%%% new operators / functions %%%

\newcommand{\sgn}{\operatorname{sgn}}
\newcommand{\Var}{\op{Var}}
\newcommand{\diag}{\operatorname{diag}}
\newcommand{\erf}{\operatorname{erf}}
\newcommand{\erfc}{\operatorname{erfc}}
\newcommand{\erfi}{\operatorname{erfi}}
\newcommand{\adj}{\operatorname{adj}}
\newcommand{\supp}{\operatorname{supp}}
\newcommand{\E}{\opE\nolimits}
\newcommand{\T}{\intercal}
% requires mathtools
% \abs,\abs*,\abs[<size_cmd:\big,\Big,\bigg,\Bigg etc.>]
\DeclarePairedDelimiter\abs{\lvert}{\rvert} 
\DeclarePairedDelimiter\norm{\lVert}{\rVert}
\DeclarePairedDelimiter\ceil{\lceil}{\rceil}
\DeclarePairedDelimiter\floor{\lfloor}{\rfloor}
\DeclarePairedDelimiter\Set{\{}{\}}
\newcommand{\imod}[1]{\allowbreak\mkern10mu({\operator@font mod}\,\,#1)}

%%% new formats %%%
\newcommand{\leftexp}[2]{{\vphantom{#2}}^{#1}{#2}}


% non-floating figures that can be put inside tables
\newenvironment{nffigure}[1][\relax]{\vskip \intextsep
  \noindent\minipage{\linewidth}\def\@captype{figure}}{\endminipage\vskip \intextsep}

\newcommand{\threecols}[3]{
\hbox to \textwidth{%
      \normalfont\rlap{\parbox[b]{\textwidth}{\raggedright#1\strut}}%
        \hss\parbox[b]{\textwidth}{\centering#2\strut}\hss
        \llap{\parbox[b]{\textwidth}{\raggedleft#3\strut}}%
    }% hbox 
}

\newcommand{\reason}[2][\relax]{
  \ifthenelse{\equal{#1}{\relax}}{
    \left(\text{#2}\right)
  }{
    \left(\parbox{#1}{\raggedright #2}\right)
  }
}

\newcommand{\marginlabel}[1]
{\mbox[]\marginpar{\color{ForestGreen} \sffamily \small \raggedright\hspace{0pt}#1}}


% up-tag an equation
\newcommand{\utag}[2]{\mathop{#2}\limits^{\text{(#1)}}}
\newcommand{\uref}[1]{(#1)}


% Notation table

\newcommand{\Hline}{\noalign{\vskip 0.1in \hrule height 0.1pt \vskip
    0.1in}}
  
\def\Malign#1{\tabskip=0in
  \halign to\columnwidth{
    \ensuremath{\displaystyle ##}\hfil
    \tabskip=0in plus 1 fil minus 1 fil
    &
    \parbox[t]{0.8\columnwidth}{##}
    \tabskip=0in
    \cr #1}}


%%%%%%%%%%%%%%%%%%%%%%%%%%%%%%%%%%%%%%%%%%%%%%%%%%%%%%%%%%%%%%%%%%%
% MISCELLANEOUS

% Modification from braket.sty by Donald Arseneau
% Command defined is: \extendvert{ }
% The "small versions" use fixed-size brackets independent of their
% contents, whereas the expand the first vertical line '|' or '\|' to
% envelop the content
\let\SavedDoubleVert\relax
\let\protect\relax
{\catcode`\|=\active
  \xdef\extendvert{\protect\expandafter\noexpand\csname extendvert \endcsname}
  \expandafter\gdef\csname extendvert \endcsname#1{\mskip-5mu \left.%
      \ifx\SavedDoubleVert\relax \let\SavedDoubleVert\|\fi
     \:{\let\|\SetDoubleVert
       \mathcode`\|32768\let|\SetVert
     #1}\:\right.\mskip-5mu}
}
\def\SetVert{\@ifnextchar|{\|\@gobble}% turn || into \|
    {\egroup\;\mid@vertical\;\bgroup}}
\def\SetDoubleVert{\egroup\;\mid@dblvertical\;\bgroup}

% If the user is using e-TeX with its \middle primitive, use that for
% verticals instead of \vrule.
%
\begingroup
 \edef\@tempa{\meaning\middle}
 \edef\@tempb{\string\middle}
\expandafter \endgroup \ifx\@tempa\@tempb
 \def\mid@vertical{\middle|}
 \def\mid@dblvertical{\middle\SavedDoubleVert}
\else
 \def\mid@vertical{\mskip1mu\vrule\mskip1mu}
 \def\mid@dblvertical{\mskip1mu\vrule\mskip2.5mu\vrule\mskip1mu}
\fi

%%%%%%%%%%%%%%%%%%%%%%%%%%%%%%%%%%%%%%%%%%%%%%%%%%%%%%%%%%%%%%%%

\makeatother

%%%%%%%%%%%%%%%%%%%%%%%%%%%%%%%%%%%%

\usepackage{ctable}
\usepackage{fouridx}
%\usepackage{calc}
\usepackage{framed}
\usetikzlibrary{positioning,matrix}

\usepackage{paralist}
%\usepackage{refcheck}
\usepackage{enumerate}

\usepackage[normalem]{ulem}
\newcommand{\Ans}[1]{\uuline{\raisebox{.15em}{#1}}}



\numberwithin{equation}{section}
\makeatletter
\@addtoreset{equation}{section}
\renewcommand{\theequation}{\arabic{section}.\arabic{equation}}
\@addtoreset{Theorem}{section}
\renewcommand{\theTheorem}{\arabic{section}.\arabic{Theorem}}
\@addtoreset{Lemma}{section}
\renewcommand{\theLemma}{\arabic{section}.\arabic{Lemma}}
\@addtoreset{Corollary}{section}
\renewcommand{\theCorollary}{\arabic{section}.\arabic{Corollary}}
\@addtoreset{Example}{section}
\renewcommand{\theExample}{\arabic{section}.\arabic{Example}}
\@addtoreset{Remark}{section}
\renewcommand{\theRemark}{\arabic{section}.\arabic{Remark}}
\@addtoreset{Proposition}{section}
\renewcommand{\theProposition}{\arabic{section}.\arabic{Proposition}}
\@addtoreset{Definition}{section}
\renewcommand{\theDefinition}{\arabic{section}.\arabic{Definition}}
\@addtoreset{Claim}{section}
\renewcommand{\theClaim}{\arabic{section}.\arabic{Claim}}
\@addtoreset{Subclaim}{Theorem}
\renewcommand{\theSubclaim}{\theTheorem\Alph{Subclaim}}
\makeatother

\newcommand{\Null}{\op{Null}}
%\newcommand{\T}{\op{T}\nolimits}
\newcommand{\Bern}{\op{Bern}\nolimits}
\newcommand{\odd}{\op{odd}}
\newcommand{\even}{\op{even}}
\newcommand{\Sym}{\op{Sym}}
\newcommand{\si}{s_{\op{div}}}
\newcommand{\sv}{s_{\op{var}}}
\newcommand{\Wtyp}{W_{\op{typ}}}
\newcommand{\Rco}{R_{\op{CO}}}
\newcommand{\Tm}{\op{T}\nolimits}
\newcommand{\JGK}{J_{\op{GK}}}

\newcommand{\diff}{\mathrm{d}}

\newenvironment{lbox}{
  \setlength{\FrameSep}{1.5mm}
  \setlength{\FrameRule}{0mm}
  \def\FrameCommand{\fboxsep=\FrameSep \fcolorbox{black!20}{white}}%
  \MakeFramed {\FrameRestore}}%
{\endMakeFramed}

\newenvironment{ybox}{
	\setlength{\FrameSep}{1.5mm}
	\setlength{\FrameRule}{0mm}
  \def\FrameCommand{\fboxsep=\FrameSep \fcolorbox{black!10}{yellow!8}}%
  \MakeFramed {\FrameRestore}}%
{\endMakeFramed}

\newenvironment{gbox}{
	\setlength{\FrameSep}{1.5mm}
\setlength{\FrameRule}{0mm}
  \def\FrameCommand{\fboxsep=\FrameSep \fcolorbox{black!10}{green!8}}%
  \MakeFramed {\FrameRestore}}%
{\endMakeFramed}

\newenvironment{bbox}{
	\setlength{\FrameSep}{1.5mm}
\setlength{\FrameRule}{0mm}
  \def\FrameCommand{\fboxsep=\FrameSep \fcolorbox{black!10}{blue!8}}%
  \MakeFramed {\FrameRestore}}%
{\endMakeFramed}

\newenvironment{yleftbar}{%
  \def\FrameCommand{{\color{yellow!20}\vrule width 3pt} \hspace{10pt}}%
  \MakeFramed {\advance\hsize-\width \FrameRestore}}%
 {\endMakeFramed}

\newcommand{\tbox}[2][\relax]{
 \setlength{\FrameSep}{1.5mm}
  \setlength{\FrameRule}{0mm}
  \begin{ybox}
    \noindent\underline{#1:}\newline
    #2
  \end{ybox}
}

\newcommand{\pbox}[2][\relax]{
  \setlength{\FrameSep}{1.5mm}
 \setlength{\FrameRule}{0mm}
  \begin{gbox}
    \noindent\underline{#1:}\newline
    #2
  \end{gbox}
}

\newcommand{\gtag}[1]{\text{\color{green!50!black!60} #1}}
\let\labelindent\relax
\usepackage{enumitem}

%%%%%%%%%%%%%%%%%%%%%%%%%%%%%%%%%%%%
% fix subequations
% http://tex.stackexchange.com/questions/80134/nesting-subequations-within-align
%%%%%%%%%%%%%%%%%%%%%%%%%%%%%%%%%%%%

\usepackage{etoolbox}

% let \theparentequation use the same definition as equation
\let\theparentequation\theequation
% change every occurence of "equation" to "parentequation"
\patchcmd{\theparentequation}{equation}{parentequation}{}{}

\renewenvironment{subequations}[1][]{%              optional argument: label-name for (first) parent equation
	\refstepcounter{equation}%
	%  \def\theparentequation{\arabic{parentequation}}% we patched it already :)
	\setcounter{parentequation}{\value{equation}}%    parentequation = equation
	\setcounter{equation}{0}%                         (sub)equation  = 0
	\def\theequation{\theparentequation\alph{equation}}% 
	\let\parentlabel\label%                           Evade sanitation performed by amsmath
	\ifx\\#1\\\relax\else\label{#1}\fi%               #1 given: \label{#1}, otherwise: nothing
	\ignorespaces
}{%
	\setcounter{equation}{\value{parentequation}}%    equation = subequation
	\ignorespacesafterend
}

\newcommand*{\nextParentEquation}[1][]{%            optional argument: label-name for (first) parent equation
	\refstepcounter{parentequation}%                  parentequation++
	\setcounter{equation}{0}%                         equation = 0
	\ifx\\#1\\\relax\else\parentlabel{#1}\fi%         #1 given: \label{#1}, otherwise: nothing
}

% hyperlink
\PassOptionsToPackage{breaklinks,letterpaper,hyperindex=true,backref=false,bookmarksnumbered,bookmarksopen,linktocpage,colorlinks,linkcolor=BrickRed,citecolor=OliveGreen,urlcolor=Blue,pdfstartview=FitH}{hyperref}
\usepackage{hyperref}

% makeindex style
\newcommand{\indexmain}[1]{\textbf{\hyperpage{#1}}}
% Something in preamble.tex causes a hard failure
%   Undefined control sequence.
%   <argument> \mdseries@rm
% using minimal preamble instead
\newcommand\maybecolor[1]{\color{#1}}

\newcommand\mypara[1]{\vskip 0.03in \noindent{\textbf{\itshape{#1}}}\quad}

\newcommand{\aset}[1]{{\ensuremath{\{#1\}}}}

\def\lstlanguagefiles{lstfstar.tex}
\lstset{language=fstar}
\let\ls\lstinline

\def\Snospace~{\S{}}
\def\sectionautorefname{\Snospace}
\def\subsectionautorefname{\Snospace}
\def\subsubsectionautorefname{\Snospace}


\hyphenation{meta-language}

\newcommand\fstar{F$^\star$\xspace}
\newcommand\microfstar{$\mu$\fstar} % EMF* emf*
\newcommand\picofstar{$p$\fstar}

\newcommand\emf{{\sc emf}$^\star$\xspace}
\newcommand\deflang{{\sc dm}\xspace} %dm, evoking Dijkstra monad, defining monads, deriving monads, etc.
\newcommand\emfST{{\sc emf}$^\star_{\text {\sc st}}$\xspace} % consistent with 2 others

\newcommand{\un}[1]{\ensuremath{\underline{#1}}}
\newcommand{\F}[2]{\mathrm{F}_{#1}~#2}
%\newcommand{\G}[3]{\mathrm{G}^{#1}_{#2}(#3)}
\newcommand{\returnT}[1]{\mathrm{\bf return}_{\teff}~#1}
\newcommand{\bindT}[3]{\mathrm{\bf bind}_{\teff}~#1~\mathrm{\bf to}~#2~\mathrm{\bf in}~#3}

\newcommand{\nat}{\mathbb{N}}
\newcommand{\sub}{\mathrm{s}}
\newcommand{\sG}{\sub_\Gamma}
\newcommand{\Rand}[2]{#1 \stackrel{\raisebox{-.25ex}[.25ex]%
   {\tiny $\mathdollar$}}{\raisebox{-.2ex}[.2ex]{$\leftarrow$}} #2}
\newcommand{\zq}{\mathbb{Z}_q}
\newcommand{\Return}{\mathsf{return}}

\newcommand{\neu}{\ensuremath{n}}
\newcommand{\teff}{\ensuremath{\tau}}
\newcommand{\narr}{\xrightarrow{\neu}}
\newcommand{\tarr}{\xrightarrow{\teff}}
\newcommand{\earr}{\xrightarrow{\epsilon}}
\newcommand{\dneg}[1]{(#1 -> \typez) -> \typez}
\newcommand{\DmG}{\Delta\mid\Gamma}
\newcommand{\fst}[1]{\mathrm{\bf fst}(#1)}
\newcommand{\snd}[1]{\mathrm{\bf snd}(#1)}
\newcommand{\inl}[1]{\mathrm{\bf inl}(#1)}
\newcommand{\inr}[1]{\mathrm{\bf inr}(#1)}
\newcommand{\mycases}[3]{\mathrm{\bf case}~#1~\mathrm{\bf inl}~#2;~\mathrm{\bf inr}~#3}
\newcommand{\eqdef}{=_{\mathrm{\bf def}}}
\newcommand{\bang}{\,!\,}

\newcommand{\cps}[1]{\ensuremath{#1^{\star}}}          % notation for a CPS'd term
\newcommand{\cpst}[0]{$\star$-translation\xspace}  % name of the CPS transform
\newcommand{\cpsts}[0]{$\star$-translations\xspace}  % name of the CPS transform (plural)

% "Stronger than"
\newcommand{\stg}{\mathrel{\lesssim}}

% Colors
\definecolor{dkblue}{rgb}{0,0.1,0.5}
\definecolor{dkgreen}{rgb}{0,0.4,0}
\definecolor{dkred}{rgb}{0.6,0,0}
\definecolor{dkpurple}{rgb}{0.7,0,1.0}
\definecolor{purple}{rgb}{0.9,0,1.0}
\definecolor{olive}{rgb}{0.4, 0.4, 0.0}
\definecolor{teal}{rgb}{0.0,0.4,0.4}
\definecolor{azure}{rgb}{0.0, 0.5, 1.0}

% Comments
\newcommand{\comm}[3]{\ifcheckpagebudget\else\ifdraft{\maybecolor{#1}[#2: #3]}\fi\fi}
\newcommand{\nik}[1]{\comm{dkpurple}{Nik}{#1}}
\newcommand{\ch}[1]{\comm{teal}{CH}{#1}}
\newcommand{\cf}[1]{\comm{teal}{CF}{#1}}
\newcommand{\mm}[1]{\comm{teal}{MM}{#1}}
\newcommand{\chfoot}[1]{\ifdraft\footnote{\comm{teal}{CH}{#1}}\fi}
%\newcommand{\gm}[1]{\comm{dkgreen}{GM}{#1}} % Guido
\newcommand{\nig}[1]{\comm{dkgreen}{Niklas}{#1}}
\newcommand{\gdp}[1]{\comm{blue}{GDP}{#1}} % Gordon
\newcommand{\aseem}[1]{\comm{magenta}{Aseem}{#1}}
\newcommand{\jp}[1]{\comm{olive}{JP}{#1}}
\newcommand{\km}[1]{\comm{purple}{KM}{#1}}

\newcommand*{\EG}{e.g.,\xspace}
\newcommand*{\IE}{i.e.,\xspace}
\newcommand*{\ETAL}{et al.\xspace}
\newcommand*{\ETC}{etc.\xspace}

\newcommand\surl[1]{{\small\url{#1}}}
\newcommand\ssurl[1]{{\scriptsize\url{#1}}}

\begin{document}

\title{A Monadic Framework for Relational Verification}

% \subtitle{Case Studies in Program Equivalence, Refinement,
%   Information-flow Control, and Cryptography}

\subtitle{Applied to Information Security, Program Equivalence, and Optimizations}

% Breainstorming:
% R3: Relational Reasoning via Reification
%    -- CH: verification is closer than what we do than reasoning;
%           also "relational reasoning" already has a meaning outside CS
%    -- CH: reification is not even the most interesting ingredient
% Relational Verification via Monadic Reification and SMT Embedding
% Relational Reasoning via Monadic Reification and SMT-based Verification
% Relational Verification via Monadic Reification, Non-Relational Weakest
%   Preconditions, and Computational SMT Embedding
%    -- CH: accurate but a mouthful
% Verifying Relational Properties in F*
%    -- CH: Cedric will disagree, but I wouldn't find it bad at all
%           It's both accurate and short. Raises the question about
%           relational F* more than the other alternatives, but that's
%           anyway not a question we can avoid.
% Relational Verification Demystified [in F*] [Functional Pearl]
% Relational Verification from Generic Ingredients
% A New Recipe for Relational Verification
% A New Recipe for Verifying Relational Properties
% A New Recipe for Verifying Relational Properties Using Generic Ingredients
% A New Recipe for Verifying Relational Properties of Effectful Programs
% A New Method for Verifying Relational Properties of Effectful Programs
% A Simple Method for Verifying Relational Properties of Effectful Programs
% Effectful Program Verification by Monadic Reification
%    NS: I like this one
% Relating Reified Effectful Programs
% Relating Effectful Programs
% Effective Reasoning via Monadic Reification
% Reasoning about Effects via Monadic Reification
% Specialized Relational Reasoning is Dead? :-)
% Towards A Unified Framework For Relational Verification
% Towards a Unified Monadic Framework for Relational Verification


% \subtitle{(Functional Pearl)}
% Verification Pearl
% Proof Pearl
% Relational Pearl
% Reified Pearl
% Many Small Pearls
% Relational Pearls
% N-ary Pearl
% Pearls

\xxx{}

\ifcamera
\author{Niklas Grimm} % niklas.grimm@tuwien.ac.at
\affiliation{\institution{Vienna University of Technology}\country{Austria}}
\author{Kenji Maillard} % kenji.maillard@inria.fr
\affiliation{\institution{Inria Paris and ENS Paris}\country{France}}
\author{C\'edric Fournet} % fournet@microsoft.com
\affiliation{\institution{Microsoft Research}\country{United Kingdom}}
\author{C\u{a}t\u{a}lin Hri\c{t}cu} % catalin.hritcu@gmail.com
\affiliation{\institution{Inria Paris}\country{France}}
\author{Matteo Maffei} % matteo.maffei@tuwien.ac.at
\affiliation{\institution{Vienna University of Technology}\country{Austria}}
\author{Jonathan Protzenko} % protz@microsoft.com
\affiliation{\institution{Microsoft Research}\country{United States}}
\author{Tahina Ramananandro} % taramana@microsoft.com
\affiliation{\institution{Microsoft Research}\country{United States}}
\author{Aseem Rastogi} % aseemr@microsoft.com
\affiliation{\institution{Microsoft Research}\country{India}}
\author{Nikhil Swamy} % nswamy@microsoft.com
\affiliation{\institution{Microsoft Research}\country{United States}}
\author{Santiago Zanella-B\'eguelin} % santiago@microsoft.com
\affiliation{\institution{Microsoft Research}\country{United Kingdom}}
\else
\author{\normalsize
      Niklas Grimm$^1$ \quad
      Kenji Maillard$^{2,3}$ \quad
      C\'edric Fournet$^4$ \quad
      C\u{a}t\u{a}lin Hri\c{t}cu$^2$ \quad
      Matteo Maffei$^1$ \quad
      Jonathan Protzenko$^4$ \quad
      Tahina Ramananandro$^4$ \quad
      Aseem Rastogi$^4$ \quad
      Nikhil Swamy$^4$ \quad
      Santiago Zanella-B\'eguelin$^4$\vspace{0.5em}}
\affiliation{$^1$Vienna University of Technology \qquad
            $^2$Inria Paris \qquad
            $^3$ENS Paris \qquad
            $^4$Microsoft Research\vspace{0.5em}}
\fi
\makeatletter
\renewcommand{\@shortauthors}{Grimm~\ETAL}
\makeatother

\begin{abstract}
%
Relational properties describe multiple runs of one or more programs.
%
They characterize many useful notions of security, program refinement,
and equivalence for programs with diverse computational effects, and
they have received much attention in the recent literature.
%
Rather than developing separate tools for special classes of effects and
relational properties, we advocate using a general purpose proof
assistant as a unifying framework for the relational verification of
effectful programs.
%
The essence of our approach is to model effectful computations using
monads and to prove relational properties on their monadic
representations, making the most of existing support for reasoning
about pure programs.

We apply this method in \fstar and evaluate it by encoding a variety of
relational program analyses, including information
flow control\iffull, semantic declassification\fi, program equivalence and
refinement at higher order, correctness of program
optimizations and game-based cryptographic security. 
%
By relying on SMT-based automation, unary weakest preconditions,
user-defined effects, and monadic reification, we show that,
compared to unary properties, verifying relational properties requires
little additional effort from the \fstar{} programmer.
\end{abstract}

\maketitle


\section{Introduction}
\label{sec:intro}

Generalizing unary properties (which describe single runs of programs),
\emph{relational} properties describe multiple runs of one or
more programs.
%
Relational properties are useful when reasoning about program
refinement, approximation, equivalence, provenance,
as well as many notions of security.
%
A great many relational program analyses have been
proposed in the recent literature, including works by \citet{Yang07,
ZaksP08, BentonKBH09, KunduTL09, GodlinS10, BartheGB12, BartheKOB13,
BartheFGSSB14, BartheGAHRS15, HedinS12, KustersTBBKM15, BanerjeeNN16,
AsadaSK16, CiobacaLRR16, AntonopoulosGHKTW17,BeckertKU15%
\iffull,BeckertBGHLU17\fi};
\citet{MurrayMBGBSLGK13};
\citet{FehrenbachC16};
\citet{BauereissG0R16,BauereissG0R17};
and \citet{CicekBG0H17}.
%
 While some systems have been designed for the efficient verification
 of specialized relational properties of programs (notably
 information-flow type systems, e.g., \citet{SabelfeldM03}), others
 support larger classes of properties.
%
These include tools based on product program constructions for
automatically proving relations between first-order imperative
programs (e.g., SymDiff \citep{LahiriHKR12} and Descartes
\citep{SousaD16}), as well as %approaches based on
relational
program logics \cite{benton04relational} that support interactive
verification of relational properties within proof assistants~(e.g.,
EasyCrypt \citep{BartheGB12} and RHTT \citep{NanevskiBG13}).

% slow down the proliferation of -- CF and CH find this too aggressive
% \ch{This still seems both too aggressive and not really true, since
%   most of our examples (sections 3, 4, and 5) don't really save the
%   need for an intermediary set of specialized ``relational lemmas'',
%   which are very much like a relational program logic or IFC type
%   system. We just use \fstar as a generic framework for these
%   special-purpose tools, as is commonly done in proof assistants.
%   How about explaining this in terms of {\em unifying} these strands of research?}%
% %
% To save the need for bespoke relational logics and other
% special-purpose tools, we propose a simple and flexible method for
% relational reasoning within existing dependently typed proof assistants.

We provide a framework in which relational logics and other
special-purpose tools can be recast on top of a general method for
relational reasoning.
%
% The essence of this generic
The method is simple: we use monads to model and program effectful
computations; and we reveal the pure monadic representation of an
effect in support of specification and proof.
%
% use \citepos{Filinski94} monadic reification to carefully control
% monadic abstraction\ch{abstraction seems a detour here, the main point in the
%   following one about reveling the pure monadic representation of an effect;
%   and bringing up Filinski is dangerous} and reveal the pure representations of a
% computation in support of specification and proof.
%
% Whereas Filinski's use was to uniformly \emph{implement} monads using
% continuations,\ch{detour? leave for later} we apply monadic
% reification to reveal the pure representations of a computation, in
% support of specification and proof only.
%
Hence, we reduce the problem of relating effectful computations to
relating their pure representations, and then apply the tools
available for reasoning about pure programs.
While this method should be usable for a variety of proof assistants,
we choose to work in \fstar \cite{mumon}, a dependently typed
programming language and proof assistant.
%
By relying on its support for SMT-based automation, unary
weakest preconditions, and user-defined effects \cite{dm4free}, we
demonstrate, through a diverse set of examples, that our approach enables
the effective verification of relational properties
with an effort comparable to proofs
of unary properties in \fstar and to proofs in relational logics with
SMT-based automation.

Being based on an expressive semantic foundation, our approach can be
directly used to verify relational properties of programs.
%
Additionally, we can still benefit from more specialized automated
proof procedures, such as syntax-directed relational type systems, by
encoding them within our framework.
%
Hence, our approach facilitates comparing and composing
special-purpose relational analyses with more general-purpose
semi-interactive proofs; and it encourages prototyping and
experimenting with special-purpose analyses with a path towards their
certified implementations.

%% can both assure their correctness, as well as
%% encourage initial prototyping


%% Finally, the versatile generic approach still usefully applies for
%% initial prototyping and can often be itself combined with the more
%% automatic certified analyses built on top of it.


%% can be proven sound
%% By encoding specialized relational analyses within a general-purpose
%% dependently typed language,  makes it easier to certify their soundness.


%


%% This versatile generic approach to relational verification can both be
%% used to directly verify programs and as a base for encoding a wide
%% range of specialized relational analyses targetting specific
%% properties or classes of programs and potentially achieving better
%% automation.

%% %
%% %
%% Furthermore,

%

% While it would be premature to claim that our work is an outright
% substitute for the growing body of work on relational verification,
% it at least provides a .
% %
% \ch{I no longer believe in this last claim, and again it's aggressive
%   and dangerous even when watered down. Providing a unified framework
%   for relational analysis won't provide a substitute for all the
%   research in relational verification. A lot of what we do here (sec
%   3,4,5) is just bringing previous relational research into our
%   framework, not substituting it! We also show some generic proofs
%   based solely on reification (sec 2,6?), but then do we really see
%   that as a substitute for specialized tools? If so why do we build
%   specialized tools in sec 3,4,5?}

% \ch{Could instead phrase this in terms of still having to find the
%   optimal balance in practice between doing generic proofs directly
%   and using these generic mechanisms to build up more specialized
%   relational analyses}
% %
% \ch{Is this a reasonable way to look at this: the generic ingredients
%   from \autoref{sec:ingredients} are a foundation for relational
%   verification. One can either use these ingredients directly to
%   verify programs or to build higher-level abstractions and
%   more specialized tools for relational verification.}

% our results encourage us to propose it as a step in a
% direction\ch{what direction?}
% that merits further investigation.\ch{The 2nd part of this last sentence
%   not clear to me; no flow between the 2 parts}

\subsection{\iffull Relational reasoning via monadic reification:\fi{}
  A first example}
\label{sec:first-example}

We sketch the main \iffull elements of our method \else ideas \fi
on a proof of equivalence for
the two stateful, recursive functions below, a task not easily
accomplished using specialized relational program logics:

\begin{lstlisting}
let rec sum_up r lo hi = 
  if lo$\neq$hi then (r := !r+lo; sum_up r (lo+1) hi)
let rec sum_dn r lo hi = 
  if lo$\neq$hi then (r := !r+hi$-$1; sum_dn r lo (hi$-$1))
\end{lstlisting}
%
Both functions sum all numbers between \ls$lo$ and \ls$hi$ into some
accumulator reference \ls$r$, the former function by counting up and the latter
function by counting down.

\mypara{Unary reasoning about monadic computations}
As a first step, we embed these computations within a
dependently typed language. There are many proposals for how to
do this---one straightforward approach is to encapsulate effectful
computations within a parameterized monad \cite{atkey09parameterised}.
In \fstar, as in the original Hoare Type Theory \citep{nmb08htt}, these monads are
indexed by a computation's pre- and postconditions and proofs are
conducted using a unary program logic (\IE not relational), adapted for use with
higher-order, dependently typed programs.
%
Beyond state, \fstar supports reasoning about unary properties of a
wide class of user-defined monadic effects, where the monad can be
chosen to best suit the intended style of unary proof.

\mypara{Relating reified effectful terms} Our goal is
to conveniently state and prove properties that relate
effectful terms, e.g., prove \ls$sum_up$ and \ls$sum_dn$
equivalent.
%
We do so by revealing the monadic representation of these two
computations as pure state-passing functions.
%
However, since doing this na\"ively would preclude the efficient
implementation of primitive effects, such as state in terms of a
primitive heap, our general method relies on an explicit {\em monadic
reification} coercion for exposing the pure monadic representation of
an effectful computation in support of relational reasoning.%
%
\footnote{While this coercion is inspired by \citepos{Filinski94}
  \ls$reify$ operator, we only use it to reveal the pure
  representation of an effectful computation in support of
  specification and proof, whereas Filinski's main use of reification
  was to uniformly implement monads using continuations.}
%% %
%% We adapt \ls$reify$
%% %for use with
%% to indexed monads, using it, for
%% instance, to reveal an \ls$ST a pre post$ computation as an explicitly
%% state-passing total function of type:
%% %
%% \ls$h:heap{pre h} -> Tot (r:(a * heap){post h (fst r) (snd r)})$.
%% %
Thus, in order to relate effectful terms, one simply reasons about
their pure reifications.
%
Turning to our example, we prove the following lemma,
% expressed as a total function in \fstar, -- CH: seems irrelevant to me
stating that running \ls$sum_up$
and \ls$sum_dn$ in the same initial states produces equivalent final
states. (A proof is given in \S\ref{sec:lemmas}.)

% \ch{missing unit return type, is that really on purpose?  Also this is
%   a Pure or Ghost not a Tot! Why not just be fully honest and use Lemma?} DONE
\begin{lstlisting}
r:ref int -> lo:int -> hi:int{hi >= lo} -> h:heap{r $\in$ h} -> 
  reify (sum_up r lo hi) h $\sim$ reify (sum_dn r lo hi) h
\end{lstlisting}


%% Each monadic effect in \fstar can be equipped with two
%% operators: Using
%% \ls$reify$, we state and prove the following equivalence:
%% \nik{Loose ends: restricting these operators to ghost code only}
%% \ch{missing unit return type, and Ghost as good as Tot here, no? (below too)}
%% \ch{What is v? (not explained, not standard at all). Maybe just sweep
%%   under the rug???}


\mypara{Flexible specification and proving style with SMT-\linebreak backed automation}
Although seemingly simple, proving \ls$sum_up$ and \ls$sum_dn$
equivalent is cumbersome, if at all possible, in most prior
relational program logics. Prior relational logics rely on common
syntactic structure and control flow between multiple programs to
facilitate the analysis. To reason about transformations like loop
reversal, rules exploiting syntactic similarity are not very useful
and instead a typical proof in prior systems may involve several
indirections, e.g., first proving the full functional correctness of
each loop with respect to a purely functional specification and then
showing that the two specifications are equivalent.
Through monadic reification, effectful terms are
\emph{self-specifying}, removing the need to rewrite the same
code in purely-functional style just to enable specification and reasoning.

Further, whereas many prior systems are specialized to proving binary
relations, it can be convenient to structure proofs using relations
of a higher arity, a style naturally supported by our method. For
example, a key lemma in the proof of the equivalence above is an
inductive proof of a ternary relation, which states
that \ls$sum_up$ is related to \ls$sum_up$ on a prefix combined
with \ls$sum_dn$ on a suffix of the interval \ls$[lo, hi)$.

Last but not least, using the combination of typechecking, weakest
precondition calculation, and SMT solving provided by \fstar, many
relational proofs go through with a degree of automation comparable to
existing proofs of unary properties,
%at least -- CH: seemed too weak to me
as highlighted by the examples in this paper.

\subsection{Contributions and outline}

% Structure of the paper

% \ch{Unimpressive start. Tempted to focus on what we do, and move the
%   relation to \citet{dm4free} to related work.}
% %
% This paper is inspired by a one-line example given
% by \citet{dm4free}. They introduce monadic reification in \fstar and
% sketch how it might be useful to prove relational properties such as
% information-flow control. Elaborating on their idea.

We propose a methodology for relational verification (\autoref{sec:ingredients}),
covering both broadly applicable ingredients such as representing
effects using monads and exposing their representation using monadic
reification, as well as our use of specific \fstar{} features that enable
proof flexibility and automation. All these ingredients are generic, \IE
none of them is specific to the verification of relational properties.

The rest of the paper is structured as a series of case studies
illustrating our methodology at work. Through these examples we aim to show
that our methodology enables comparing and composing various styles of
relational program verification in the same system, thus taking a step
towards unifying many prior strands of research.
%
Also these examples cover a wide range of applications that, when
taken together, exceed the ability of all previous tools for
relational verification of which we are aware.
%
Our examples are divided into three sections that can be read in any
order, each being an independent case study:

\mypara{Transformations of effectful programs (\autoref{sec:transformations})} 
We \linebreak develop an extensional, semantic characterization of a stateful program's
read and write effects, based on the relational approach
of \citet{benton06aplas}. Based on these semantic read and write effects, we derive
% standard rules\ch{standard rules?
%   tried to rephrase confused statement, make sure it's still correct}
lemmas that we use to prove the correctness of common program
transformations, such as swapping the order of two commands and eliminating
redundant writes. Going further, we encode~\citepos{benton04relational}
relational Hoare logic in our system, providing a syntax-directed proof system
for relational properties as a special-purpose complement
to directly reasoning about a program's effects.
 
\mypara{Cryptographic security proofs (\autoref{sec:crypto})}
We show how to model basic game steps of code-based cryptographic
proofs of security \citep{Bellare2006} by proving equivalences between
probabilistic programs. We prove perfect secrecy of one-time pad
encryption \iffull, and a crucial lemma in the proof of semantic security of
ElGamal encryption\fi, an elementary use of \citepos{BartheGB09}
probabilistic relational Hoare logic.

\mypara{Information-flow control (\autoref{sec:ifc})} We encode
several styles of static information-flow control
analyses \iffull, while accounting for declassification\fi. Highlighting the
ability to compose various proof styles in a single framework, we
combine automated, type-based security analysis
% of imperative programs
with SMT-backed, semantic proofs of noninterference.

\mypara{Proofs of algorithmic optimizations (\autoref{sec:refinement})}
With a few exceptions, prior relational program logics apply to
first-order programs and provide incomplete proof rules that
exploit syntactic similarities between the related programs.
Not being bound by syntax, we prove relations of higher
arities (e.g., $4$-ary and $6$-ary relations) between higher-order,
effectful programs with differing control flow by reasoning directly
about their reifications. We present two larger
examples: First, we show how to memoize a recursive function
using \citepos{McBride15} partiality monad
and we prove it equivalent to the original non-memoized version.
Second, we implement an imperative union-find data structure, adding the
classic union-by-rank and path compression optimizations in several
steps and proving stepwise refinement.

\smallskip

From these case studies, we conclude that our method for relational
reasoning about reified monadic computations is both effective and
versatile.
%
We are encouraged to continue research in this direction, aiming to
place proofs of relational properties of effectful programs on an
equal footing with proofs of pure programs in \fstar as well as other
proof assistants and verification tools.

% \ifsubmit
% The real code for the examples in this paper is available as non-anonymous
% supplementary material.
% \else
The code for the examples in this paper is available at\\
{\small \href{https://github.com/FStarLang/FStar/tree/master/examples/rel}{https://github.com/FStarLang/FStar/tree/master/examples/rel}}\\
% \fi
Compared to this code, the listings in the paper are edited for
clarity and sometimes omit uninteresting details.
\ifsooner
\ch{Should try to switch to real literate programming style in next revision}
\fi
% \ch{This might bite us at CPP}
%
\ifcamera
The extended version \cite{relational} describes some
additional case studies that we omit here because of space.
\fi

\section{Methodology for relational verification}
\label{sec:ingredients}

% Generic \fstar{} features and
% how we use them for relational verification

% \ch{Might change section title to something more interesting:
%   Relational Verification Methodology}

In this section we review in more detail the key \fstar{}
features we use and how each of them contributes to our
verification method for relational properties.
%
Two of these features are general and broadly applicable:
%
(\autoref{sec:monads})~modeling effects using monads and keeping the
effect representation abstract to support efficient implementation of
primitive effects and
%
(\autoref{sec:reification})~using monadic reification to expose the
effect representation.
% , but only in computationally irrelevant code.\ch{It might
%   be more than this Ghost reification}
%
The remaining features are more specific to \fstar and
enable proof flexibility and automation:
%
(\autoref{sec:wp})~using a unary weakest precondition calculus to
  produce verification conditions in an expressive dependently typed
  logic;
%
(\autoref{sec:lemmas})~using dependent types together with pre- and
postconditions to express arbitrary relational properties of reified
computations;
%
(\autoref{sec:smt})~embedding the dependently typed logic into SMT
   logic to enable the SMT solver to reason by
   computation.

None of these generic ingredients is tailored to the verification
of relational properties, and while \fstar{} is currently the only
verification system to provide all these ingredients in a unified package,
each of them also appears in other systems.
%
This makes us hopeful that this relational verification method can
also be applied with other proof assistants (\EG Coq, Lean, Agda,
Idris, etc.), for which the automation would likely come in quite
different styles.

%% , and to other semi-automatic verification systems (\EG Dafny),
%% which may use our technique to gain in expressiveness while retaining
%% their SMT-based automation.

% \ch{Maybe the best way to explain this is take a simple running
%   example that is presented at a high-level in the previous section
%   and here is dissected to show how our methodology handles this
%   example in full detail.  Something in the same order of complexity
%   as factorial \cite{Aguirre16, AlejandroHKS16}.}
% %
% \ch{Thinking of this, it's unlikely that one single example will be
%   enough for illustrating all points we need to make here. Still, we
%   could try to find a main running example and some pointwise ones.}
% %
% \ch{\bf Now there is the example from the intro that I should use to
%   show as many things as possible}

\subsection{Modeling effects using monads}
\label{sec:monads}

At the core of \fstar is a language of dependently typed, total
functions. Function types are written \ls$x:t -> Tot t'$ where
the co-domain \ls$t'$ may depend on the argument \ls$x:t$. Since it
is the default in \fstar{}, we often
drop the \ls$Tot$ annotation (except where needed for emphasis) and
also the name of the formal argument when it is unnecessary, e.g., we
write \ls$int -> bool$ for \ls$_:int -> Tot bool$. We also
write \ls$#x:t -> t'$ to indicate that the argument \ls$x$ is
implicitly instantiated.

Our first step is to describe effects using monads built
from total functions \cite{Moggi89}.
%
For instance, here is the standard monadic representation of state
in \fstar{} syntax.

% \nik{I renamed \ls$M$ to \ls$Tot$, even though it is inaccurate. It's
%   too confusing otherwise and impossible to explain here, despite my
%   attempted footnote: \ls$M$ is just the identity monad, which is
%   syntactically necessary for technical reasons and can be ignored,
%   safely \cite{dm4free}.}
% CH: OK, was a bit hesitating on this change myself, but it's good

%\ch{naming heavily improved below, which we should also do in the real files}
%
% \chfoot{use Kenji's monad, maybe even stick with the state only variant (in comments)?
%   For now listing only state and exceptions.}
%
% \begin{lstlisting}
% type stexn (heap:Type) (a:Type) = heap -> M ((option a) * heap)
% let return_stexn (heap:Type) (a:Type) (x:a) : stexn a = fun h -> Some x, h
% let bind_stexn (heap:Type) (a b : Type) (x:stexn a) (f: a -> stexn b) : stexn b =
%   fun h -> let (z, h') = x h in match z with | None -> None, h'
%                                     | Some xa -> f xa h'
% let get (heap:Type) () : stexn heap = fun h -> Some h, h
% let put (heap:Type) (h:heap) : stexn unit = fun _ -> Some (), h
% \end{lstlisting}
%
% \ch{Could these be inlined in the new\_effect definition?} -- DONE
% let return_st (heap:Type) (a:Type) (x:a) : st a = fun h -> x, h
% let bind_st (heap:Type) (a b : Type) (f:st a) (g : a -> st b) : st b = fun h -> let (z, h') = f h in g z h'
% let get (heap:Type) () : st heap = fun h -> h, h
% let put (heap:Type) h : st unit = fun _ -> (), h
%
\begin{lstlisting}
type st (mem:Type) (a:Type) = mem -> Tot (a * mem)
\end{lstlisting}
%
This defines a type \ls$st$ \ifsubmit indexed \else parameterized \fi
by types for the memory (\ls$mem$) and the result (\ls$a$).
%
We use \ls$st$  as the representation type of a new \ls$STATE_m$ effect we
add to \fstar, with the \ls$total$ qualifier enabling the termination
checker for \ls$STATE_m$ computations.
%
\begin{lstlisting}
total new_effect {$\label{state}$
 STATE_m (mem:Type) : a:Type -> Effect
 with repr = st mem;
  return = fun (a:Type) (x:a) (m:mem) -> x, m;
  bind = fun (a b:Type) (f:st mem a) (g:a -> st mem b) (m:mem) -> 
   let z, m' = f m in g z m';
  get = fun () (m:mem) -> m, m;  put = fun (m:mem) _ -> (), m  }
\end{lstlisting}
%
This defines the \ls$return$ and \ls$bind$ of this monad,
and two actions: \ls$get$ for obtaining the current memory,
and \ls$put$ for updating it.
%
The new effect \ls$STATE_m$ is still parameterized by the type of memories,
which allows us to choose a memory model best suited to the programming
and verification task at hand. We often instantiate \ls$mem$
to \ls$heap$ (a map from references to their values, as in ML), obtaining
the \ls$STATE$ effect shown below---we use other memory types
in \autoref{sec:ifc} and~\autoref{sec:equiv}.
%
\begin{lstlisting}
total new_effect STATE = STATE_m heap
\end{lstlisting}

% \ch{Might also want to explain the total qualifiers: is
%   useful for explaining our limitation to reasoning extrinsically only
%   about terminating things. How does Kenji deal with that for memo?
%   Might be anyway wiser to leave the discussion for \autoref{sec:reification}?
%   Alternatively, just swap the qualifiers under the carpet;
%   the discussion does need to happen though.}

While such monad definitions could in principle be used to directly
extend the implementation of any functional language with the state effect, a
practical language needs to allow keeping the representation of some
effects abstract so that they are efficiently implemented
primitively \cite{PeytonJones01}.
% \chfoot{Idris doesn't exactly do monads, but it does seem
%   to do effects primitively}
%
\fstar{} uses its simple module system to keep the monadic representation
of the \ls$STATE$ effect abstract and implements it under the hood
using the ML heap, rather than state passing (and similarly for
other primitive ML effects such as exceptions).
%
Whether implemented primitively or not, the monadic definition of each
effect is always the {\em model} used by \fstar{} to reason
about effectful code, both intrinsically using a (non-relational)
weakest precondition calculus (\autoref{sec:wp}) and extrinsically
using monadic reification (\autoref{sec:reification}).

\iffalse
In the case of state, the exposed operations in the module interface
are of course not \ls$get$ and \ls$put$, but the familiar and
efficiently implementable \ls$alloc$,
\ls$(!)$, and \ls$(:=)$ operations on individual reference cells
(where we postpone the specifications of these operations until \autoref{sec:wp}).
\begin{lstlisting}
val ref : Type -> Type
val alloc : #a:Type -> init:a -> STATE (ref a) ...
val (!) : #a:Type -> r:ref a -> STATE a ...
val (:=) : #a:Type -> r:ref a -> v:a -> STATE unit ...
\end{lstlisting}
%
All code with \ls$STATE$ effect is written against this
interface, as was the case with the \ls$sum_up$ and \ls$sum_dn$
functions from \autoref{sec:first-example}.
%
\fi

For the purpose of verification, monads provide great flexibility in
the modeling of effects, which enables us to express relational
properties and to conduct proofs at the right level of abstraction.
%
For instance, \iffull in \autoref{sec:declassification} we extend a state
monad with extra ghost state to track declassification, \fi  in
\autoref{sec:crypto} we define a monad for random sampling from a
uniform distribution, and  in \autoref{sec:memo} we define a partiality
monad for memoizing recursive functions.
%
Moreover, since the difficulty of reasoning about effectful code is
proportional to the complexity of the effect, we do not use a single
full-featured monad for all code; instead we define custom monads for
sub-effects and relate them using monadic lifts.
%
For instance, we define a \ls$READER$ monad for computations that only
read the store, lifting \ls$READER$ to \ls$STATE$ only where
necessary (\autoref{sec:ifc-while} provides a detailed example).
%
While \fstar{} code is always written in an ML-like direct style, the
\fstar{} typechecker automatically inserts binds, returns and lifts under
the hood \cite{swamy11coco}.

\subsection{Unary weakest preconditions for user-defined effects and intrinsic proof}
\label{sec:wp}

For each user-defined effect, \fstar derives a weakest precondition
calculus for specifying unary properties and computing verification
conditions for programs using that effect \cite{dm4free}.
%
Each effect definition induces a computation type indexed by a
predicate transformer describing that computation's effectful
semantics.

For state, we obtain a computation type
%
`\ls$STATE a wp$' indexed by a result type \ls$a$ and by~\ls$wp$,
%
a predicate transformer of type
%
\ls$(a -> heap -> Type) -> heap -> Type$,
mapping postconditions (relating the result and final state of the
computation) to preconditions (predicates on the initial state).
%
\iffull For example, t\else{T}\fi{he} types of the \ls$get$ and \ls$put$ actions
of \ls$STATE$ are specified as:

\begin{lstlisting}
val get : unit -> STATE heap (fun post (h:heap) -> post h h)
val put : h':heap -> STATE unit (fun post (h:heap) -> post () h')
\end{lstlisting}

\noindent The type of \ls$get$ states that, in order to prove any
postcondition \ls$post$ of `\ls$get ()$' evaluated in state \ls$h$, it
suffices to prove \ls$post h h$, whereas for \ls$put h'$ it suffices
to prove \ls$post () h'$.
%
%Rather than indexing computations types with predicate transformers,
%
\fstar users find it more convenient to index computations with pre- and
postconditions as in HTT \cite{nmb08htt}, or sometimes
not at all, using the following abbreviations:

\begin{lstlisting}
ST a (requires p) (ensures q) = STATE a (fun post h$_0$ -> 
             p h$_0$ /\ (forall (x:a) (h$_1$:heap). q h$_0$ x h$_1$ ==> post x h$_1$))
St a = ST a (requires (fun _ -> True)) (ensures (fun _ _ _ -> True))
\end{lstlisting}

\fstar{} computes weakest preconditions
generically for any effect.
%
Intuitively, this works by putting the code into an explicit monadic
form and then translating the binds, returns, actions, and lifts from the
expression level to the weakest precondition level.
%
This enables a convenient form of \emph{intrinsic} proof in \fstar,
i.e., one annotates a term with a type capturing properties of
interest; \fstar computes a weakest precondition for the term and
compares it to the annotated type using a built-in subsumption rule,
checked by an SMT solver.

For example, \iffull in the code below, \fstar{} checks that the inferred
computation type is sufficient to prove that a \ls$noop$ function leaves the
memory unchanged.

%% \begin{lstlisting}
%% let noop () : ST unit (requires (fun m -> True)) 
%%   (ensures (fun m _ m' -> m == m')) = put (get ())
%% \end{lstlisting}

For a more interesting example,\fi the \ls$sum_up$ function
from \S\ref{sec:first-example} can be given the following type:

\begin{lstlisting}
r:ref int -> lo:nat -> hi:nat{hi >= lo} -> 
          ST unit (requires fun h -> r $\in$ h) (ensures fun _ _ h -> r $\in$ h)
\end{lstlisting}

This is a dependent function type, for a function with three
arguments \ls$r$, \ls$lo$, and \ls$hi$ returning a terminating, stateful
computation.
%
The \emph{refinement} type \ls$hi:nat{hi >= lo}$ restricts
\ls$hi$ to only those natural numbers greater than or equal to \ls$lo$.
%
The computation type of `\ls$sum_up r lo hi$' simply requires and ensures
that its reference argument \ls$r$ is present in the memory.
%
\fstar computes a weakest precondition from the implementation of
\ls$sum_up$ (using the types of \ls$(!)$ and \ls$(:=)$ %etc.
provided by
the
\ls$heap$ memory model used by \ls$STATE$) and proves that its
inferred specification is subsumed by the user-provided annotation.
The same type can also be given to \ls$sum_dn$.

\ifsooner
\ch{Do we want to show this last example in a bit more detail?}
\fi

%% Below we will show how to exploit this unary program logic to encode
%% proofs of relational properties.
%% %
%% But first, let's look at a basic unary, safety property.
%% %
%% % Our goal is to exploit this underlying unary program logic to encode
%% % proofs of relational properties.
%% % \ch{logically wrong transition, the goal sentence repeated in next para:
%% %   As we will see below, we will exploit this unary program logic }%
%% % Specifically, in \fstar,
%% %
%% Both \ls$sum_up$ and \ls$sum_dn$ have the following type in \fstar:
%% %
%% %
%% %
%% Beyond state, \fstar supports reasoning about unary properties of a
%% wide class of user-defined monadic effects, where the monad used to
%% model an effect can be chosen to best suit the intended style of unary
%% proof.
%% %
%% \nik{Loose ends: compiling ST to primitive state}\ch{Right, that's
%%   important for motivating (controlled) abstraction and reify.}




%% For instance, for computing the weakest precondition of \ls$incr$,
%% we look at its body:
%% %
%% \begin{lstlisting}
%%     let incr () = put (get () + 1)
%% \end{lstlisting}
%% we first turn this into the following explicitly monadic computation
%% \cite{swamy11coco}:
%% \begin{lstlisting}
%%     let incr () = bind$_{ST}$ (get ()) (fun n -> put (n + 1))
%% \end{lstlisting}
%% %
%% and then turn the resulting code to the following weakest precondition for \ls$incr$:
%% %
%% \begin{lstlisting}
%%     val incr : r:ref int -> STATE unit (bind_wp$_{ST}$ get_wp (fun n -> put_wp (n + 1)))
%% \end{lstlisting}
%% %
%% Intuitively, to obtain this the typing rules for \ls|bind$_{ST}$|, \ls$get$, and
%% \ls$put$ relate them to weakest-precondition-level combinators with the same name:
%% %
%% \begin{lstlisting}
%%      val get : unit -> ST int get_wp
%%      val put : n$_1$:int -> ST unit (put_wp n$_1$)
%%      val bind$_{ST}$ : forall wa wb. ST a wa -> (x:a -> ST b (wb x)) -> ST b (bind_wp$_{ST}$ wa wb)
%% \end{lstlisting}
%% %
%% The weakest-precondition-level combinators for return, bind, actions,
%% and lifts are obtained automatically by \fstar{} by performing a
%% selective CPS translation of the monadic effect definition \cite{dm4free}.

%% \ch{TODO: it would be useful to have an example that uses both
%%   non-relational and relational reasoning}\ch{These come later on,
%%   for instance Aseem had some. Could make the point abstractly
%%   and forward reference?}





%% So far, using the \ls$STATE$ monad to program

%% Our methodology relies on a {\em non-relational} weakest precondition
%% calculus to check the non-relational specifications of the verified
%% code and the proof sketches of the relational lemmas. This process
%% produces verification conditions in an expressive dependently typed logic.

%% \ch{Used Guido's example from POPL, but that's for the single integer
%%   reference cell state monad. Changed to use a input reference, for a
%%   bit of extra complexity but more realism.}
%% \ch{How about changing this to the running example? For instance
%%   the very boring spec of \ls$sum_up$ and maybe then the proof sketch
%%   of \ls$sum_up_dn_aux$.}

%% Let's step back and look into how \fstar{} computes weakest preconditions.
%% %
%% In \fstar{} pre- and postconditions are in fact desugared into
%% specifications phrased directly in terms of weakest preconditions.
%% %
%% So the following Hoare-style specification \cite{nmb08htt} for the
%% function incrementing an integer reference:
%% %
%% \begin{lstlisting}
%%     val incr : r:ref int -> ST unit (requires (fun h$_0$ -> True)) (ensures (fun h$_0$ () h$_1$ -> h$_1$ = upd h$_0$ r (sel h$_0$ r) + 1))
%% \end{lstlisting}
%% %
%% is desugared to the following weakest-precondition-style specification
%% \cite{mumon}:
%% \begin{lstlisting}
%%     val incr : r:ref int -> STATE unit (fun post h$_0$ -> post () (upd h$_0$ r (sel h$_0$ r) + 1)))
%% \end{lstlisting}

\subsection{Exposing effect definitions via \iffull monadic \fi reification}
\label{sec:reification}

Intrinsic proofs of effectful programs in \fstar are inherently
restricted to unary properties. Notably, pre- and postconditions
are required to be pure terms, making it impossible for specifications
to refer directly to effectful code,
e.g., \ls$sum_up$ cannot directly use itself or \ls$sum_dn$ in its
specification. To overcome this restriction, we need a way to coerce a terminating
effectful computation to its underlying monadic representation
which is a pure term---\citepos{Filinski94} monadic
reification provides just that facility.%
%
\footnote{Less frequently, we use {\tt reify}'s dual, {\tt reflect},
  which packages a pure function as an effectful computation.
  % Reflect is useful for effect handlers and for hiding unobservable
  % effects.\ch{need to be more careful here, since we can do neither of
  %   these in its most general form at this point! By handlers we
  %   actually mean actions?  Not algebraic effects handlers,
  %   right?}\ch{footnote is misleading and should be fixed; see POPL
  %   reviews discussion (Reviewer D)}
}

% \ch{Where is the right place to explain that we can only reason
%   extrinsically about terminating programs? How about here somewhere? footnote?}
% CH: sneaked in above

Each new effect in \fstar{} induces a \ls$reify$ operator
that exposes the representation of an effectful computation
in terms of its underlying monadic representation \cite{dm4free}.
%
For the \ls$STATE$ effect, \fstar provides the following (derived) rule
for \ls$reify$, to coerce a stateful computation to a total,
explicitly state-passing function of type \ls$heap -> t * heap$.
%
The argument and result types of
\ls$reify$$~e$ are refined to capture the pre- and postconditions
intrinsically proved for $e$.
%
\[\footnotesize
\inferrule{
  S;\Gamma |- e : \text{\ls$ST t (requires pre) (ensures post)$}
}
{
  S;\Gamma |- \text{\ls$reify$}~e:
              \text{\ls$h:heap\{pre h\} -> Tot (r:(t*heap)\{post h (fst r) (snd r)\})$}
}
\]

The semantics of \ls$reify$ is to traverse the term and
to gradually expose the underlying
monadic representation. We illustrate this below for \ls$STATE$,
where the constructs on the right-hand side of the rules
are the pure implementations
of \ls$return$, \ls$bind$, \ls$put$, and \ls$get$ as defined on
page~\pageref{state}, but with type arguments left implicit:
% \ch{Should we mark some arguments implicit in new\_effect?
%   Or should we say we treat all type arguments as implicit?}
\newcommand{\steps}[0]{\ensuremath{\leadsto}}
\newcommand{\stepss}[0]{\ensuremath{\steps^{*}}}
\[\small
\begin{array}{r@{\hspace{0.5em}}c@{\hspace{0.5em}}l}
\text{\ls$reify (return $}~e\text{\ls$)$} &\steps& \text{\ls$STATE.return $}~e\\
\text{\ls$reify (bind x $}\leftarrow e_1~\text{\ls$ in $}~e_2\ls$)$ &\steps&
\begin{array}{l}
  \hspace{-5pt}\text{\ls$STATE.bind (reify $}~e_1\text{\ls$)(fun x->reify $}~e_2\text{\ls$)$}
\end{array} \\
\text{\ls$reify (get $}~e\text{\ls$)$} &\steps& \text{\ls$STATE.get $}~e\\
\text{\ls$reify (put $}~e\text{\ls$)$} &\steps& \text{\ls$STATE.put $}~e
\end{array}
\]

Armed with \ls$reify$, we can write an \emph{extrinsic} proof of a
lemma relating \ls$sum_up$ and \ls$sum_dn$ (discussed in detail
in \S\ref{sec:lemmas}), i.e., an ``after the fact'' proof that
is separate from the definition of \ls$sum_up$ and \ls$sum_dn$
and that relates their reified executions.
%
We further remark that in \fstar{} the standard operational semantics
of effectful computations is modeled in terms of reification, so
proving a property about a reified computation is really the same as
proving the property about the evaluation of the computation itself.
% taken almost verbatim from our rebuttal

The \ls$reify$ operator
clearly breaks the abstraction of the underlying monad and needs to be used
with care. \citet{dm4free} show that programs that do not use \ls$reify$
(or its converse, \ls$reflect$)
can be compiled efficiently. Specifically, if the
computationally relevant part of a program is free of \ls$reify$
%and \ls$reflect$,
then the \ls$STATE$ computations can be compiled
using primitive state with destructive updates.

To retain these benefits of abstraction, we rely on \fstar's module
system to control how the abstraction-breaking \ls$reify$ coercion
can be used in client code.
%
In particular, when abstraction violations cannot be tolerated, we
use \fstar's \ls$Ghost$ effect (explained in
\S\ref{sec:lemmas}) to mark \ls$reify$
%and \ls$reflect$
as being usable only in computationally irrelevant code, limiting the use
of monadic reification to specifications and proofs.
%
This allows one to use reification even though effects like state and
exceptions are implemented primitively in \fstar.


%% For the purpose of extrinsic reasoning,\ch{not explained} the monadic representation of
%% effects can be revealed in specifications and other computationally
%% irrelevant code using monadic reification \cite{dm4free, Filinski94}.%
%% %
%% \chfoot{return to running example, show a lemma that uses
%%   reification, for instance here is one from the dm4free paper. A
%%   better running example would use recursion as well as a precondition.}
%% %
%% %
%% %  stating that regardless of its secret input (\ls$i0$ or \ls$i1$), function
%% % \ls$f$ when run in the some public initial heap \ls$h0$ produces
%% % identical public outputs.
%% % \begin{lstlisting}
%% % let f h = if h then (incr(); let y = get() in decr(); y) else get() + 1
%% % val noninterference : i0:int -> i1:int -> h0:heap -> Lemma (ensures (reify (f i0) h0 == reify (f i1) h0))
%% % \end{lstlisting}
%% %
%% \begin{lstlisting}
%% val eq_sum_up_dn (r:ref int) (lo:int) (hi:int{hi >= lo}) (h:heap{r $\in$ h})
%% : Lemma (requires True) (ensures (reify (sum_up r lo hi) h $\sim$ reify (sum_dn r lo hi) h))
%% \end{lstlisting}
%% %
%% While the effectful computations \ls$sum_up r lo hi$ and \ls$sum_dn r lo hi$ cannot
%% directly appear in specifications, they are first reified to \ls$Pure$
%% state-passing computations by revealing the internal representation of
%% the \ls$STATE$ effect.
%% %
%% In this example, the \ls$reify$ coercion is typed using the following rule:
%% %
%% \ch{This introduces wps too early ... maybe just use ellypsis? Nik uses a
%%   different trick, by stating it in terms of pre-post conditions and refinements:
%%   reify reveals an \ls$ST a pre post$ computation as an explicitly
%%   state-passing total function of type:.}
%% \ls$h:heap{pre h} -> Tot (r:(a * heap){post h (fst r) (snd r)})$
%
% \ch{The following is the type of \ls$ghost_reify$ in S2.7 of dm4free
%   paper. Can we take it as reify here? One thing that's a bit annoying
%   is the extra x argument, can we drop that and just display this as
%   a typing rule?}
% \begin{lstlisting}
%   reify: (x:a -> ST b (wp x))
%         -> Ghost (x:a -> s0:s -> Pure (b * s) (wp x s0))
% \end{lstlisting}
%
%[lab={\hspace{0.2cm}T-Reify}]


%% % The \ls$reify$ coercion reveals the representation of an \ls$ST$
%% % computation as a \ls$Pure$ function.
%% %
%% The $\text{\ls$reify$}~e$ construct itself has \ls$Ghost$ effect,
%% which marks it as computationally irrelevant code and ensures that it
%% does not influence the computationally relevant code.
%% %

%% Stating and proving relational properties as Dafny-style lemmas}
\subsection{Extrinsic specification and proof, eased by SMT-based automation}
\label{sec:lemmas}
\label{sec:smt}

% \nik{I disagree with calling these Dafny-style. We've been using
%   functions with logical effect only since Fable, F7
%   etc.}\ch{Dafny-style was a codename, better names welcome. Don't
%   think that systems with the value restriction can do this properly}
% \ch{SMT-based lemmas? Lemmas without proof terms? Proof-irrelevant lemmas?
%   Please trust me lemmas? Trivial inhabitant lemma? Squashed up
%   lemmas?}
% CH: maybe we don't need a name for them after all?

% Our basic idea of extrinsically proving relational properties via
% monadic reification is simple and should be applicable in a variety of
% settings.
%
We now look at the proof relating \ls$sum_up$ and \ls$sum_dn$ in
detail, explaining along the way several \fstar-specific idioms that
we find essential to making our method work well.

\mypara{Computational irrelevance (\ls$Ghost$ effect)}
%A built-in computation type in \fstar,
The \ls$Ghost$ effect is used to track a
form of computational irrelevance.
%
\ls$Ghost t (requires pre) (ensures post)$ is the type of a
pure computation returning a value of type \ls$t$ satisfying
\ls$post$, provided \ls$pre$ is valid.
%
However, this computation must be erased before running the program,
so it can only be used in specifications and proofs.
% \ch{What program and how is it related to the Ghost computation
%   in the first place?}

\mypara{Adding proof irrelevance (\ls$Lemma$)} 
\fstar
provides two closely related forms of proof irrelevance. First, a pure
term \ls$e:t$ can be given the refinement type \ls|x:t{$\phi$}| when it
validates the formula \ls@$\phi$[e/x]@, although no proof of $\phi$ is
materialized. For example, borrowing the terminology
of \citet{Nogin02}, the value \ls$()$ is a \emph{squashed} proof
of \ls@u:unit{0 <= 1}@. Combining proof and computation irrelevance,
%
\ls$e : Ghost unit pre (fun () -> post)$ is a squashed proof of \ls$pre -> post$.
%
This latter form is so common that we write it as
%
\ls$Lemma (requires pre) (ensures post)$,
further abbreviated as \ls$Lemma post$ when \ls$pre$ is \ls$True$.

\mypara{Proof relating \ls$sum_up$ and \ls$sum_dn$}
Spelling out the main lemma of \S\ref{sec:first-example}, our goal is
a value of the following type:

\begin{lstlisting}
val eq_sum_up_dn (r:ref int)(lo:int)(hi:int{hi >= lo})(h:heap{r $\in$ h})
: Lemma 
  (v r (reify (sum_up r lo hi) h) == v r (reify (sum_dn r lo hi) h))
\end{lstlisting}
where \ls$v r (_, h) = h.[r]$ and \ls$h.[r]$ selects the contents of the
reference \ls$r$ from the heap \ls$h$.

An attempt to give a trivial definition
%
for \ls$eqsum_up_dn$ that simply returns a unit value
\ls$()$ fails, because the SMT solver
cannot automatically prove the strong postcondition above.
%
Instead our proof involves calling an auxiliary
lemma \ls$sum_up_dn_aux$, proving a ternary relation:
%
\begin{lstlisting}
val sum_up_dn_aux (r:ref int) (lo:int) (mid:int{mid >= lo}) 
                   (hi:int{hi >= mid}) (h:heap{r $\in$ h})
: Lemma  (v r (reify (sum_up r lo hi) h)
         == v r (reify (sum_dn r lo mid) h) 
           + v r (reify (sum_up r mid hi) h) $-$ h.[r])
  (decreases (mid $-$ lo))
let eq_sum_up_dn r lo hi h = sum_up_dn_aux r lo hi hi h
\end{lstlisting}
%
While the statement of \ls$eq_sum_up_dn$ is different from the
statement of \ls$sum_up_dn_aux$, the SMT-based automation fills in the
gaps and accepts the proof sketch.
%
In particular, the SMT solver figures out that \ls$sum_up r hi hi$ is
a no-op by looking at its reified definition.
\ifsooner\ch{Normalizer helps too here!}\fi
%
In other cases, the user has to provide more interesting proof sketches
that include not only calls to lemmas that the SMT solver cannot
automatically apply but also the cases of the proof and the recursive
structure.  This is illustrated by the \iffull\else following \fi
proof\iffull of \ls$sum_up_dn_aux$\fi:
%
\begin{lstlisting}
let rec sum_up_dn_aux r lo mid hi h =
  if lo $\neq$ mid then (sum_up_dn_aux r lo (mid $-$ 1) hi h;
                  sum_up_commute r mid hi (mid $-$ 1) h;
                  sum_dn_commute r lo (mid $-$ 1) (mid $-$ 1) h)
\end{lstlisting}
%
This proof is by induction on the difference between \ls$mid$ and
\ls$lo$ (as illustrated by the \ls$decreases$ clause of the lemma,
this is needed because we are working with potentially-negative integers). If
this difference is zero, then the property is trivial since the SMT
solver can figure out that \ls$sum_dn r lo lo$ is a no-op.
% so we can simply prove this case by returning the unit value.
%
Otherwise, we call \ls$sum_up_dn_aux$ recursively for \ls+mid $-$ 1+ as
well as two further commutation lemmas (not shown) about \ls$sum_up$
and \ls$sum_dn$ and the SMT automation can take care of the rest.



% \ch{TODO: Simple recursive example. Could list proof sketch here but
%   explain how type-checking works for this in next subsection. It
%   would be great if this example illustrated calling other lemmas
%   explicitly, recursion, maybe even SMTPats(?)}

%% \fstar{} supports both Dafny-style and Coq-style lemmas, and the two
%% can in fact be mixed to take advantage both of SMT-based automation
%% and the power of explicit proof terms \cite{mumon,dm4free}. Here we
%% focus, however, only on Dafny-style lemmas, since for the examples
%% from this paper explicit proof terms were not needed.\ch{true?}

% \ch{Now with the squash types and stuff we can actually switch between
%   the two style, although it's not particularly convenient, and we
%   don't really use that in our relational examples.}

\mypara{Encoding computations to SMT}
% \label{sec:smt}
%
So how did \fstar{} figure out automatically that \ls$sum_up r hi hi$
and \ls$sum_dn r lo lo$ are no-ops?
%
For a start the \fstar{} normalizer applied the semantics of
\ls$reify$ sketched in \autoref{sec:reification} to partially evaluate the
term and reveal the monadic
representation of the \ls$STATE$ effect by traversing the term and
unfolding the monadic definitions of return, bind, actions and lifts.
%
In the case of \ls$reify (sum_up r hi hi) h$, for instance, reduction
intuitively proceeds as follows:
%
\begin{lstlisting}
reify (sum_up r hi hi) h 
  $\steps$ reify (if hi $\neq$ hi then (r := !r + lo; sum_up r (lo + 1) hi)) h
  $\stepss$ if hi $\neq$ hi then (STATE.bind (reify (Ref.read r) h) (fun x ->
                    STATE.bind (reify (Ref.upd r (x + lo))) 
                      (fun _ -> reified_sum_up r (hi + 1) hi))) h
      else STATE.return () h
  $\stepss$ if hi $\neq$ hi then let x, h' = reify (Ref.read r) h in
                    let _, h'' = reify (Ref.upd r (x + lo)) h' in
                    reified_sum_up r (hi + 1) hi h''
      else ((), h)
\end{lstlisting}

What is left is pure monadic code that \fstar{} then encodes to
the SMT solver in a way that allows it to reason by
computation \cite{AlejandroHKS16}.
%
For \ls$reify (sum_up r hi hi) h$ the SMT solver can
trivially show that \ls|hi $\neq$ hi| is false and thus the
computation returns the pair \ls$((), h)$.

While our work did not require any extension to \fstar{}'s
theory~\cite{dm4free}, we significantly improved \fstar{}'s logical
encoding to perform normalization of open terms based on the semantics
of reify (a kind of symbolic execution) before calling the SMT
solver. This allowed us to scale and validate the theory of
\citet{dm4free} from a single 2-line example to the $\approx$4,300
lines of relationally verified code presented in this paper.

\subsection{Empirical evaluation of our methodology}

For this first example, we reasoned directly about the semantics of two
effectful terms to prove their equivalence.
%
However, we often prefer more structured reasoning principles to prove
or enforce relational properties, e.g., by using program logics,
syntax-directed type systems, or even dynamic analyses.
%
In the rest of this paper, we show through several case studies, that
these approaches can be accommodated, and even composed, within our
framework.

Table~\ref{fig:perf} summarizes the empirical evaluation from these
case studies.
%
Each row describes a specific case study, its size in lines of source code,
and the verification time using \fstar and the Z3-4.5.1 SMT
solver. The verification times were collected on an Intel Xeon E5-2620
at 2.10 GHz and 32GB of RAM.
The ``1st run'' column indicates the time it takes \fstar and Z3
to find a proof. This proof is then used to generate hints (%likely
unsat cores) that can be used as a starting point to verify subsequent
versions of the program.
The ``replay'' column indicates the time it takes to verify the
program given the hints recorded in the first run.
%
Proof replay is usually significantly faster, indicating that although
finding a proof may initially be quite expensive, revising a proof
with hints is fast, which greatly aids interactive proof development.

\begin{table}[t]
  \small
  
\begin{tabular}{|l|r|r|r|r|}
\hline
Subject&Section&1st run (ms)&Replay (ms)&Loc
\\\hline
Loops&{\ref{sec:first-example}}&218192&8943&127\\
Reorderings&{\ref{sec:transformations-spec}}&9239&4749&158\\
Benton (2004)&{\ref{sec:benton2004-rhl}}&832706&22920&1352\\
Cryptography&{\ref{sec:crypto}}&17307&10015&530\\
Static IFC&{\ref{sec:ifc-while}}&68525&15909&730\\
Hybrid IFC&{\ref{sec:ifc-combining}}&55472&1038&34\\
Declassification&\ifcamera{*}\else\ref{sec:declassification}\fi&63763&9811&208\\
IFC Monitor&\ifcamera{*}\else\ref{sec:ifc-monitor}\fi&44589&11480&502\\
Memoization&{\ref{sec:memo}}&12198&12294&427\\
Union-find&{\ref{sec:unionfind}}&89838&33455&295\\
\hline Total&&1411829&130614&4363\\\hline
\end{tabular}

  \caption{Code size (lines of code without comments) and proof-checking time (ms) for our examples. \ifcamera Examples with label * appear in the extended version~\cite{relational}.\fi}
\label{fig:perf}
\ch{Didn't want to touch the generated file (too much), but I would change ref
  with autoref and add subsection 2.4 for Loops. I would also put the
  Loc column before the timings. Finally, I would round and switch to
  seconds, since I don't think our numbers are precise all the way to
  milliseconds; and the numbers look much too big in milliseconds.}
\end{table}

\section{Correctness of program transformations}
\label{sec:transformations}

\nik{About section 3 in the paper (program transformations): a missed opportunity ... I realize now that we should have explicitly made several connections there
1. Some of the equivalences we proved have a very single-sided flavor, the last one in particular. We should have said so, making connections to RHL (edited)
2. We're very close in s3 to have a shallowly embedded while language (we have seq and conditionals, we should have added while)
and we should have pointed out the shallow/deep contrast w.r.t the IFC section
It would be nice to derive all the rules in some canonical RHL for while on this shallowly embedded language ... we're not very far from it}

Several researchers have devised custom program logics for verifying
transformations of imperative
programs \citep{benton04relational, BartheGB09, CarbinKMR12}.
%
We show how to derive similar rules justifying the correctness of
generic program transformations within our monadic framework.
%
We focus on stateful programs with a fixed-domain, finite memory.
%
We leave proving transformations of commands that dynamically allocate
memory to future work.\ch{``in a somewhat different context,
  \S\ref{sec:refinement} contains examples that use dynamic allocation
  and local state'' This last claim seems wrong; paper doesn't show
  anything about this, and not sure if it's true even in the
  implementation}


\let\li\ls
%% We now present two examples of program transformations. Our first example is
%% parameterized over a \emph{command} \li+f+ and relates the program \li+f ()+ to
%% \li+f (); f ()+ (idempotence).
%% \cf{This use of parentheses immediately after syntax is confusing!}
%% Our second example is parameterized over two
%% commands \li+f1+ and \li+f2+ and relates the program \li+f1 (); f2 ()+ to
%% \li+f2 (); f1 ()+ (swap).

%% One cannot relate these programs by reasoning over their syntactic structure.
%% Instead, we need to perform higher-order relational reasoning over
%% one (\li+f+) or many (\li+f1+ and \li+f2+) commands, leveraging the
%% \emph{semantic} properties of the commands.

\subsection{Generic transformations based on read- and write-footprints}
\label{sec:transformations-spec}

Here and in the next subsection,
we represent a command $c$ as a function of type \ls$unit -> St unit$
that may read or write arbitrary references in memory.
%
\begin{lstlisting}
type command = unit -> St unit
\end{lstlisting}
%
In trying to validate transformations of commands, it is traditional to
employ an effect system to delimit the parts of memory that a
command may read or write.
%
Most effect systems are unary, syntactic analyses. For example,
consider the classic frame rule from separation logic:
%
\[
\{ P \} c \{ Q \} \Rightarrow
\{ P \ast R \} c \{ Q \ast R \}
\]
The command $c$ requires ownership of a subset of the heap $P$ in
order to execute, then returns ownership of $Q$ to its
caller. Any distinct heap fragment $R$ remains unaffected by the
function. Reading this rule as an effect analysis, one may conclude
that $c$ may read or write the $P$-fragment of memory---however, this
is just an approximation of $c$'s extensional behavior.
%
\citet{benton06aplas} observe that a more precise, semantic characterization
of effects arises from a relational perspective. Adopting this
perspective, one can define the \ls$footprint$ of a command
extensionally, using two unary properties and one binary property.

Capturing a command's write effect is easy with a
unary property, `\ls$writes c ws$' stating that the initial and final
heaps agree on the contents of their references, except \iffull for \fi those
in \iffull the set\fi \ls$ws$.
%
\begin{lstlisting}
type addrs = S.set addr
let writes (c:command) (ws:addrs) = forall (h:heap).
 let h' = snd  (reify (c ()) h) in
 (forall r. r $\in$ h <==> r $\in$ h') /\ (* no allocation *)
 (forall r. addr_of r $\not\in$ ws ==> h.[r] == h'.[r]) (* no changes except ws*)
\end{lstlisting}

Stating that a command only reads references \ls$rs$ is similar in
spirit to \iffull the statement of \fi noninterference (\iffull
to which we return in \fi\S\ref{sec:ifc-while}).
Interestingly, it is impossible to describe
the set of locations that a command may read without also speaking
about the locations it may write. The relation `\ls$reads c rs ws$'
states that if \ls$c$ writes at most the references in \ls$ws$, then
executing \ls$c$ in heaps that agree on the references in \ls$rs$
produces heaps that agree on \ls$ws$, i.e., \ls$c$ does not depend on
references outside \ls$rs$.

\ch{While the comment above helps intuition a bit, the writes and
  reads names really don't. Guido, Kenji, and me were all confused.
  not sure about the others, but the reason I was confused was that I
  was expecting writes to be the only thing that restricts writes and
  reads to only restrict reads}

\ch{Santiago: I realized this is exactly what we called Observational
  Equivalence in CertiCrypt (section 4.2 in
  \url{https://www.microsoft.com/en-us/research/wp-content/uploads/2016/02/Zanella.2009.POPL_.pdf})
  One funny thing is that if you have \ls$c:cmd rs ws$, then \ls$c:cmd rs (S.union rs ws)$ also holds.}

\begin{lstlisting}
let equiv_on (rs:addr_set) (h$_0$:heap) (h$_1$:heap) =
 forall a (r:ref a). addr_of r $\in$ rs /\ r $\in$ h$_0$ /\ r $\in$ h$_1$ ==> h$_0$.[r] == h$_1$.[r]
let reads (c:command) (rs ws:addrs) = forall (h$_0$ h$_1$: heap).
 let h'$_0$, h'$_1$ = snd (reify (c ()) h$_0$), snd (reify (c ()) h$_1$) in
 (equiv_on rs h$_0$ h$_1$ /\ writes c ws) ==> equiv_on ws h'$_0$ h'$_1$
\end{lstlisting}

Putting the pieces together, we define a read- and
write-footprint-indexed type for commands:

\begin{lstlisting}
type cmd (rs ws:addrs) = c:command{writes c ws /\ reads c rs ws}
\end{lstlisting}

\newcommand\bindseq{\ensuremath{\mbox{\texttt{>>}}}}

One can also define combinators to manipulate footprint-indexed
commands. For example, here is a `\bindseq' combinator for sequential
composition. Its type proves that read and write-footprints compose by
a pointwise union, a higher-order relational property; the proof
requires an (omitted) auxiliary lemma \ls$seq_lem$ (recall that
variables preceded by a \ls$#$ are implicit arguments):

\begin{lstlisting}
let seq (#r1 #w1 #r2 #w2 : addrs) (c1:cmd r1 w1) (c2:cmd r2 w2) : 
  command = c1(); c2()
let ($\bindseq$) #r1 #w1 #r2 #w2 (c1:cmd r1 w1) (c2:cmd r2 w2) :
  cmd (r1 $\cup$ r2) (w1 $\cup$ w2) = seq_lem c1 c2; seq c1 c2
\end{lstlisting}

\iffull
\subsection{Several transformations on commands}
\label{sec:transformations-proofs}
\fi

Making use of relational footprints, we can prove other relations
between commands, e.g., equivalences that justify program
transformations. Command equivalence \ls@c$_0$ $\sim$ c$_1$@ states
that running \ls@c$_0$@ and \ls@c$_1$@ in identical initial heaps
produces (extensionally) equal final heaps.

\begin{lstlisting}
let ($\sim$) (c$_0$:command) (c$_1$:command) = forall h.
 let h$_0$, h$_1$ = snd (reify (c$_0$ ()) h), snd (reify (c$_1$ ()) h) in
 forall (r:ref 'a). (r $\in$ h$_0$ <==> r $\in$ h$_1$) /\ (r $\in$ h$_0$ ==> h$_0$.[r] == h$_1$.[r])
\end{lstlisting}

\iffull
Our first equivalence, listed below, shows that if a command's read and write
footprints are disjoint, then it is idempotent.
%
The proofs of \ls$idem$ and the other lemmas below are perhaps
peculiar to SMT-based proofs. In all cases, the proofs involve simply
mentioning the terms \ls$reify (c ()) h$, which suffice to direct the
SMT solver's quantifier instantiation engine towards finding a
proof. While more explicit proofs are certainly possible, with
experience, concise SMT-based proofs can be easier to write.

\begin{lstlisting}
let idem #rs #ws (c:cmd rs ws): 
  Lemma (requires (disjoint rs ws)) (ensures ((c $\bindseq$ c) $\sim$ c))
  = forall_intro (fun h -> let (), h$_1$ = reify (c ()) h in 
      let _ = reify (c ()) h$_1$ in () 
      <: Lemma (equiv_on_h (c $\bindseq$ c) c h))
\end{lstlisting}

Our next equivalence shows that two commands can be swapped if they
\else
For instance, we can prove that two commands can be swapped if they
\fi
write to disjoint sets, and if the read footprint of one does not
overlap with the write footprint of the other---this lemma is
identical to a rule for swapping commands in a logic presented
by \citet{BartheGB09}.

\begin{lstlisting}
let swap #rs1 #rs2 #ws1 #ws2 (c1:cmd rs1 ws1) (c2:cmd rs2 ws2)
  :Lemma (requires (disjoint ws1 ws2 /\ disjoint rs1 ws2 /\ 
                   disjoint rs2 ws1))
          (ensures ((c1 $\bindseq$ c2) $\sim$ (c2 $\bindseq$ c1)))
  = forall_intro (fun h -> let _ = reify (c1 ()) h, reify (c2 ()) h in 
    () <: Lemma (equiv_on_h (c1 $\bindseq$ c2) (c2 $\bindseq$ c1) h))
\end{lstlisting}

\iffull

Next, we show elimination of redundant writes by proving that \ls{c1 $\bindseq$ c2}
is equivalent to \ls{c2} if \ls{c1}'s write footprint is \iffull (a) \fi a
subset of \ls{c2}'s write footprint, and \iffull (b) \fi disjoint from \ls{c2}'s
read\iffull footprint\else{s}\fi.

\begin{lstlisting}
let redundant$\_$writes #rs1 #rs2 #ws1 #ws2 
  (c1:cmd rs1 ws1) (c2:cmd rs2 ws2)
  : Lemma (requires (disjoint ws1 rs2 $\wedge$ ws1 $\subseteq$ ws2))
          (ensures  ((c1 $\bindseq$ c2) $\sim$ c2))
  = forall_intro (fun h -> let _ = reify (c1 ()) h, reify (c2 ()) h in 
      () <: Lemma (equiv_on_h (c1 $\bindseq$ c2) c2 h))
\end{lstlisting}

\else
The extended version \cite{relational} also verifies command idempotence and
elimination of redundant writes.

\fi

%% TR: with NS, in fact, the following paragraph is not specific to --
%% or even has nothing to do with -- command footprints, and is more
%% akin to program logics, so more relevant to the next section.

%% Finally, we consider conditional commands and show that if two
%% branches of a conditional command are  equivalent, then the
%% conditional command itself is equivalent to both the branches. We
%% first define the conditional combinator, and then show the equivalence.

%% \begin{lstlisting}
%% (* guard does not change the heap *)
%% type guard = f:(unit -> St bool){$\forall$ (h:heap). h == snd (reify (f ()) h)}
%% let ite (c:guard) (c1:command) (c2:command) : command = fun () -> if c() then c1 () else c2 ()
%% let dead$\_$code (c:guard) (c1:command) (c2:command)
%%   : Lemma (requires (c1 $\sim$ c2))
%%           (ensures  ((ite c c1 c2) $\sim$ c1))
%%   = forall_intro (fun h -> let _ = reify (c1 ()) h, reify (c2 ()) h in () <: Lemma (equiv_on_h (ite c c1 c2) c1 h))
%% \end{lstlisting}

\subsection{Relational Hoare Logic}
\label{sec:benton2004-rhl}

Beyond generic footprint-based transformations, one may also prove
program-specific equivalences. Several logics have been devised for
this, including,
e.g., \citepos{benton04relational} Relational Hoare logic (RHL). We
show how to derive RHL within our framework by proving the soundness
of each of its rules as lemmas about a program's reification.

\paragraph{Model}
To support potentially diverging computations, we instrument
shallowly-embedded effectful
computations with a \emph{fuel} argument, where the value of the fuel
is irrelevant for the behavior of a terminating computation.
%
\ch{So what does fuel measure? Is it recursive calls or something
  else? No trivial notion of execution steps with a shallow embedding.}
%
\begin{lstlisting}
type comp = f: (fuel:nat -> St bool)
   { forall h fuel fuel' . fst (reify (f fuel) h) == true /\ fuel' > fuel
     ==> reify (f fuel') h == reify (f fuel) h }
let terminates_on c h = exists fuel . fst (reify (c fuel) h) == true
\end{lstlisting}
We model effectful expressions whose evaluation always terminates and
does not change the memory state, and assignments, conditionals,
sequences of computations, and potentially diverging while
loops.\ch{``... by induction on the fuel'' This phrase makes no sense
  to me and confused POPL reviewer}

\paragraph{Deriving RHL}
%
An RHL judgement `\ls@related c$_1$ c$_2$ pre post@'
(where \ls@c$_1$, c$_2$@ are effectful computations, and
\ls$pre, post$ are relations over memory
states) means that the executions of \ls@c$_1$, c$_2$@ starting
\iffull(respectively)\fi in memories \ls@h$_1$, h$_2$@ related by
\ls$pre$, both diverge or both terminate with memories \ls@h$_1$', h$_2$'@
related by \ls$post$.

\begin{lstlisting}
let related (c1 c2 : comp) (pre post: (heap -> heap -> prop)) =
 (* if precondition holds on initial memory states, then *)
 forall h1 h2 . pre h1 h2 ==> 
 (* c1 and c2 both terminate or both diverge, and *)
 ((c1 `terminates_on` h1 <==> c2 `terminates_on` h2) /\ 
  (forall fuel h1' h2' . (reify (c1 fuel) h1 == (true, h1') /\ 
    reify (c2 fuel) h2 == (true, h2')) ==> (* if both terminate, *)
   post h1' h2')) (* postcondition holds on final memory states *)
\end{lstlisting}

From these reification-based definitions, we prove every rule of RHL. Of
the 20 rules and equations of RHL presented
by \citet{benton04relational}, 16 need at most 5 lines of proof
annotation each, among which 10 need none and are proven
automatically. Rules related to while loops often require some manual
induction on the fuel.
\iffull
Thus, modeling computations, program
logic rules, and their soundness proofs amount about 1500 lines of F*
code overall.\ch{Do we really need this given the table?}
\fi

\iffull
\begin{lstlisting}
(* Example of fully automatic soundness proof: dead while *)
let r_dwhll ($b$: exp bool) ($c$: computation) 
  ($\Phi$: (heap -> heap -> prop)) : Lemma
    (ensures (related (while $b$ $c$) skip $(\Phi \land \lnot b_{\text{left}})$ $(\Phi \land \lnot b_{\text{left}})$)) = ()
\end{lstlisting}
\fi

With RHL in hand, we can prove program equivalences
applying syntax-directed rules, focusing the intellectual effort
on finding and proving inductive invariants to relate loop bodies.
When RHL is not powerful enough, we can escape back to
the reification of commands to complete a direct proof
in terms of the operational semantics.
\ifcamera
In the extended version \cite{relational} we sketch a
program-specific equivalence built using our embedding of RHL
in \fstar.
\fi

\iffull
\paragraph{Example}
Following \citet{benton04relational}, we prove an example hoisting an
assignment out of a loop:
\[
\newcommand\defPhi{\begin{array}{l}I_{\text{left}} = I_{\text{right}} \land \\ N_{\text{left}} = N_{\text{right}} \land \\ Y_{\text{left}} = Y_{\text{right}} \end{array}}
\newcommand\defas[2]{\begin{tabular}{l} \fcolorbox{red}{white}{#1} \\ {\textcolor{red}{#2}} \end{tabular}}
\begin{array}{rcl}
\vdash \defas{\begin{tabular}{l} $\mathsf{while} ~ (I < N)$ \\ \hspace{1em} $X := Y + 1;$ \\ \hspace{1em} $I := I + X$ \end{tabular}}{$L$} &\rightsquigarrow& \defas{\begin{tabular}{l} $X := Y + 1;$ \\ $\mathsf{while} ~ (I < N)$ \\ \hspace{1em} $I := I + X$ \end{tabular}}{$R$} : \\ \defas{$\defPhi$}{$\Phi$} &\Rrightarrow& \defPhi

\end{array}
\]
In other words, the judgement above preserves the invariant $\Phi$ stating that the two programs $L$ and $R$  compute the same values for $I, N, Y$, with $X$ being neglected (which is already useful enough if $X$ is known to be dead in the code following the while loops).

\begin{lstlisting}
let proof () : Lemma (ensures (related $L$ $R$ $\Phi$ $\Phi$)) =
  (* intermediate invariants for the loop bodies *)
  let $\Phi_1$ = $\Phi \land (X_{\text{right}} = Y_{\text{right}} + 1)$ in 
  let $\Phi_2$ = $\Phi_1 \land (X_{\text{left}} = X_{\text{right}})$ in
  assert (related skip (assign $X$ $(Y + 1)$) $\Phi$ $\Phi_1$); (* dead assign *)
  assert (related (assign $X$ $(Y + 1)$) skip $\Phi_1$ $\Phi_2$); (* dead assign *)
  assert (related (assign $I$ $(I + X)$) (assign $I$ $(I + X)$) $\Phi_2$ $\Phi_2$); (* assign *)
  assert (related (seq (assign $X$ $(Y + 1)$) (assign $I$ $(I + X)$)) 
                  (assign i $(I + X)$) $\Phi_1$ $\Phi_2$); (* seq, elim. skip *)
  r_while $(I < N)$ $(I < N)$ (seq (assign $X$ $(Y + 1)$) (assign $I$ $(I + X)$)) 
          (assign $I$ $(I + X)$) $\Phi_1$;
  (* seq, elim. skip *)
  assert (related $L$ (while $(I < N)$ (assign $I$ $(Y + 1)$)) $\Phi_1$ $\Phi$) 
\end{lstlisting}
%
\[
  \inferrule[ $\mathsf{r\_while} ~ B ~ B' ~ C ~ C' ~ \Phi : $]{
  \vdash C \rightsquigarrow C' : \Phi \land B_{\text{left}} \land B'_{\text{right}} \Rrightarrow \Phi \land (B_{\text{left}} = B'_{\text{right}})
}
{
  \vdash \mathsf{while} ~ B ~ \mathsf{do} ~ C \rightsquigarrow \mathsf{while} ~ B' ~ \mathsf{do} ~ C' : \Phi \land (B_{\text{left}} = B'_{\text{right}}) \Rrightarrow \\  \Phi \land \lnot (B_{\text{left}} \lor B'_{\text{right}})
}
\]
%
The proof shows that applications of RHL rules (including dead assignment rules) are
actually syntax-directed, so that the only nontrivial effort needed is
to provide the intermediate verification condition relating the bodies
of the loops.

In more detail, for a given proposition $\phi$, \ls!assert $\phi$!
tries to prove $\phi$ and, if successful, adds $\phi$ to the proof
context as a fact that can be automatically reused by the later parts
of the proof. To prove $\phi$, proof search relies not only on the
current proof context,
% consisting of the facts proven by the earlier
% parts of the proof,
but also on those lemmas in the global context
that are associated with \emph{triggering patterns}: if the shape of
$\phi$ matches the triggering pattern of some lemma $f$ in the global
context, then $f$ is applied (\emph{triggered}) and the proof search
recursively goes on with the preconditions of $f$. This proof search
is actually performed by the Z3 SMT solver
through \emph{e-matching} \cite{MouraB07}.

In our example proof, 
\ls!assert (related skip (assign $X$ $(Y + 1)$) $\Phi$ $\Phi_1$)!
tries to prove that an assignment can be erased; based on the syntax
of both commands of the relation, e-matching successfully selects the
corresponding dead assignment rule of RHL. In fact, this \ls!assert!
also allows specifying the intermediate condition $\Phi_1$ that is to
be used to verify the rest of the bodies of $L$ and $R$, which cannot
always be guessed by proof search. Alternatively, the user can also
explicitly apply an RHL rule by directly calling the corresponding
lemma, which is illustrated by the call to \ls!r_while! to prove that
the two while loops are related. In that case, the postcondition of
the lemma is added to the proof context for the remainder of the
proof. This way, the user can avoid explicitly spelling out the fact
proven by the lemma; moreover, since the lemma to apply is explicitly
given, the SMT solver only has to prove the preconditions of
the lemma, if any.

This example is 33 lines of F* code and takes 25 seconds to
check. This time could be improved substantially. However, perhaps
more interesting, this experiment suggests developing tactics to
automatically use Benton's RHL whenever possible, while still keeping
the possibility to escape back to semantic approaches wherever RHL is
not powerful enough. We leave this as future work.
\fi

%% Equipped with these definitions, we are ready to tackle our first example, and
%% relate \li+f ()+ to \li+f (); f ()+.

% We build upon the \li+footprint f rs ws+
% relational predicate, and build another relational predicate that relates
% \li+h$_0$+, the input heap, \li+h$_1$+, the output heap after executing
% \li+f ()+, and \li+h$_2$+, the output heap after executing \li+f (); f
% ()+.

%% We now wish to relate two programs using the \li+footprint+ predicate. That is,
%% for a given initial heap \li+h$_0$+, and two programs \li+g$_1$+ and
%% \li+g$_2$+, we wish to relate their respective output heaps \li+h$_1$+ and \li+h$_2$+.
%% More precisely, we only focus on those references that were initially in
%% \li+h$_0$+, for the sake of clarity.
%% %
%% For this purpose we introduce the \li+heap_eq+ property stating that
%% \li+h$_1$+ and \li+h$_2$+ coincide on all references that already were in
%% \li+h$_0$+.
%% %
%% \begin{lstlisting}
%% let heap_eq (h$_0$:heap) (h$_1$:heap) (h$_2$:heap) = forall (a:Type) (r:ref a).  h$_0$ `contains` r ==> sel h$_1$ r == sel h$_2$ r
%% \end{lstlisting}


%% We take a function \li+f+ whose read-set is \li+rs+ and write-set is
%% \li+ws+, and turn our attention to the programs \li+let g$_1$ () = f ()+ and
%% \li+let g$_2$ () = f (); f ()+. If \li+ws+ is disjoint from \li+rs+, then it
%% follows that calling \li+f+ a second time results in the same values written in
%% \li+ws+.

%% \begin{lstlisting}
%% let idem (rs ws:addr_set) (f:unit -> STNull unit)  (h$_0$:heap)
%%   : Lemma (requires (S.disjoint rs ws /\ footprint f rs ws))
%%            (ensures (let _, h$_1$ = reify (f ()) h$_0$ in let _, h$_2$ = reify (f ()) h$_1$ in heap_eq h$_0$ h$_1$ h$_2$))
%%   = let _, h1 = reify (f ()) h$_0$ in let _, h2 = reify (f ()) h$_1$ in ()
%% \end{lstlisting}

%% \ch{Don't understand the proof sketch. Why are the lets needed?
%%   (same thing below)}

%% \paragraph{Swap}

%% We take the methodology above a step further and now relate \emph{two} different
%% functions \li+f1+ and \li+f2+ along with their footprints. In a sense, we are
%% now relating two programs that themselves already enjoy relational properties
%% via the \li+footprint+ predicate. We leverage this ``meta-relational'' style to
%% show that one can swap two functions \li+f1+ and \li+f2+ as long as they do not
%% ``trample'' each other. Precisely: as long as one does not write into the
%% other's read-set.

%% % JP: TODO: cite PDF
%% The weaker, perhaps more intuitive property states that if the union of the two
%% write-sets is disjoint from the union of the two read-sets, then \li+f1+ and
%% \li+f2+ may be swapped. That is, if no write touches a reference in one of the
%% read-sets. The property we're showing is more fine grained: the two read-sets
%% may overlap, \li+f1+ may write into \li+rs1 \ rs2+ and \li+f2+ may write into
%% \li+rs2 \ rs1+, but the two functions are still swappable.

%% \begin{lstlisting}
%% val swap (rs1: addr_set) (ws1: addr_set) (f1: unit -> STNull unit)
%%   (rs2: addr_set) (ws2: addr_set) (f2: unit -> STNull unit) (h$_0$: heap) :
%%   Lemma
%%     (requires (S.disjoint ws1 ws2 /\ S.disjoint rs1 ws2 /\ S.disjoint rs2 ws1 /\
%%       footprint f1 rs1 ws1 /\ footprint f2 rs2 ws2))
%%     (ensures (
%%       let (), h$_1$ = reify (f1 ()) h$_0$ in
%%       let (), h'$_1$ = reify (f2 ()) h$_1$ in
%%       let (), h$_2$ = reify (f2 ()) h$_0$ in
%%       let (), h'$_2$ = reify (f1 ()) h$_2$ in
%%       heap_eq h$_0$ h'$_1$ h'$_2$))
%% let swap rs1 ws1 f1 rs2 ws2 f2 h$_0$ =
%%   let (), h$_1$ = reify (f1 ()) h$_0$ in
%%   let (), h$_2$ = reify (f2 ()) h$_0$ in
%%   ()
%% \end{lstlisting}


\section{Cryptographic security proofs}
\label{sec:crypto}

We show how to construct a simple model for reasoning about
probabilistic programs that sample values from discrete
distributions. In this model, we prove the soundness of rules of
probabilistic Relational Hoare Logic (pRHL)~\citep{BartheGB09}
allowing one to derive (in-)equalities on probability quantities from
pRHL judgments. We illustrate our approach by formalizing \iffull two \else a \fi  simple
cryptographic proof\iffull{s}\fi: the perfect secrecy of one-time pad encryption
\iffull and a crucial lemma used by \citet{BartheGB09}
in the proof of semantic security of ElGamal encryption \fi .

The simplicity of our examples pales in comparison with complex proofs
formalized in specialized tools based on pRHL like
EasyCrypt~\citep{BartheGB12} or FCF~\citep{PetcherM15}, yet our
examples hint at a way to prototype and explore proofs in pRHL with a
low entry cost.

\subsection{A monad for random sampling}

We begin by defining a monad for sampling from the uniform
distribution over bitvectors of a fixed length \ls$q$. We implement
the monad as the composition of the state and exception monads where the
state is a finite tape of bitvector values together with a pointer to
a position in the tape.
%
The \ls$RAND$ effect provides a single action, \ls$sample$, which
reads from the tape the value at the current position and advances the
pointer to the next position, or raises an exception if the pointer is
past the end of the tape.
%
\begin{lstlisting}
type value = bv q
type tape = seq value
type id = i:$\nat${i < size}
type store = id * tape
type rand a = store -> M (option a * id)
total new_effect {
  RAND: a:Type -> Effect
  with repr = rand a;
       bind = fun (a b:Type) (c:rand a) (f:a -> rand b) s ->
                let r, next = c s in
                match r with
                | None   -> None, next
                | Some x -> f x (next, snd s);
       return = fun (a:Type) (x:a) (next,_) -> (Some x, next);
       sample = fun () s -> let next, t = s in
                  if next + 1 < size then (Some (t n), n + 1) 
                  else (None, n) }
effect Rand a = RAND a (fun initial_tape post -> forall x. post x)
\end{lstlisting}

Assuming a uniform distribution over initial tapes, we define the
unnormalized measure of a function \ls{$p$:a -> $\nat$} with respect
to the denotation of a reified computation in \ls{$f$:Rand a} as
\ls$let mass f p = sum (fun t -> let r,_ = f (0, t) in p r)$
where \ls'sum: (tape -> $\nat$) -> $\nat$' is the summation operator over
finite tapes.
%
When $p$ only takes values in $\{0,1\}$, it can be regarded as an
\emph{event} whose probability with respect to the distribution generated by
$f$ is
%
$$
\Pr[ f : p ]
  = \frac{1}{|\mathsf{tape}|} \times \sum_{t\ \in\ \mathsf{tape}} p\ (\mathsf{fst}\ (f\ t))
  = \frac{\mathsf{mass}\ f\ p}{|\mathsf{tape}|}
$$
%
We use the shorthand
%
$\Pr[f = v] = {|\mathsf{tape}|}^{-1} \times \mathsf{mass}\ f\ (\mathsf{point}\ v)$
%
for the probability of a successful computation returning a value $v$,
where
%
\ls{let point x = fun y -> if y = Some x then 1 else 0}.


\subsection{Perfect secrecy of one-time pad encryption}

The following effectful program uses a one-time key
\ls$k$ sampled uniformly at random to encrypt a bitvector $m$:
%
\begin{lstlisting}
let otp (m:value) : Rand value = let k = sample () in m $\oplus$ k
\end{lstlisting}
%
We show that this construction, known as \emph{one-time pad}, provides
\emph{perfect secrecy}. That is, a ciphertext does not give away any
information about the encrypted plaintext, provided the encryption key
is used just once. Or equivalently, the distribution of the one-time
pad encryption of a message is independent of the message itself,
$\forall m_0,~m_1,~c.\ \Pr[\mathsf{otp}\ m_0 = c] = \Pr[\mathsf{otp}\ m_1 = c]$.
%
We prove this by applying two rules of pRHL, namely [R-Rand] and
[PrLe].\ch{Important question: have we actually proven these rules?}
%
The former allows us to relate the results of two
probabilistic programs by showing a bijection over initial random
tapes that would make the relation hold (intuitively, permuting
equally probable initial tapes does not change the resulting
distribution over final tapes). The latter allows us to infer a
probability inequality from a proven relation between probabilistic
programs. Together, the two rules allow us to prove the following lemma:
%
\begin{lstlisting}
val mass_leq: #a:Type -> #b:Type -> 
  c1:(store -> M (a * id)) -> c2:(store -> M (b * id)) ->
  p1:(a -> nat) -> p2:(b -> nat) -> bij:bijection -> Lemma
  (requires (forall t. let r1,_ = c1 (to_id 0,t) in 
               let r2,_ = c2 (to_id 0,bij.f t) in p1 r1 <= p2 r2))
  (ensures (mass c1 p1 <= mass c2 p2))
\end{lstlisting}
%
The proof is elementary from rearranging terms in summations according
to the given bijection.
%
The following secrecy proof of one-time pad is immediate from this
lemma using as bijection on initial tapes \iffull the function \fi
%
\ls{fun t -> upd t 0 (t 0 $\oplus$ m0 $\oplus$ m1)}:
%
\begin{lstlisting}
val otp_secure: m0:value -> m1:value -> c:value -> Lemma
  (let f0, f1 = reify (otp m0), reify (otp m1) in 
   mass f0 (point c) == mass f1 (point c))
\end{lstlisting}

\iffull
\subsection{A step in the proof of semantic security of ElGamal encryption}

Another example following a similar principle is a probabilistic
equivalence used in the proof of semantic security of ElGamal
encryption by \citepos{BartheGB09}. This equivalence, named
\texttt{mult\_pad} in that paper, proves the independence of the
adversary's view from the hidden bit $b$ that the adversary has to
guess in the semantic security indistinguishability game, and thus
shows that the adversary cannot do better than a random guess.

ElGamal encryption is parametric on a cyclic group of order $q$, and a
generator $g$.
%
% Given a cyclic group of order $q$, and a generator $g$, the ElGamal
% encryption scheme is defined as follows:
% %
% \newcommand{\Rand}[2]{#1 \stackrel{\raisebox{-.25ex}[.25ex]%
%    {\tiny $\mathdollar$}}{\raisebox{-.2ex}[.2ex]{$\leftarrow$}} #2}
% \newcommand{\zq}{\mathbb{Z}_q}
% \newcommand{\Return}{\mathsf{return}}
% $$
% \begin{array}{lcl}
% \mathsf{KeyGen}(\eta)	    &=& \Rand{x}{\zq}; \Return\ (g^x,x) \\
% \mathsf{Encrypt(\alpha, m)} &=& \Rand{y}{\zq}; \Return\ (g^y, \alpha^y \times m) \\
% \mathsf{Decrypt(x, (\beta,\zeta))} &=& \Return\ (\zeta \times \beta^{-x})
% \end{array}
% $$
%
Roughly stated, the equivalence says that if one applies the group
operation to a uniformly distributed element of the group and some
other element, the result is uniformly distributed, that is
%
$\Rand{z}{\zq}; \zeta \leftarrow g^z \times m_b$
and
$\Rand{z}{\zq}; \zeta \leftarrow g^z$
%
induce the same distribution on $\zeta$ (which is thus independent of
$b$).
%
To prove this, we modify the \ls{RAND} effect
to use random tapes of elements of $\zq$ rather than bitvectors, an define
%
\begin{lstlisting}
let elgamal$_0$ (m:group) : Rand group = let z = sample () in g^z
let elgamal$_1$ (m:group) : Rand group = let z = sample () in (g^z) * m
\end{lstlisting}
%
and prove, again using \ls{mass_leq}, the following lemma
%
\begin{lstlisting}
val elgamal_equiv: m:group -> c:group -> Lemma
  (let f1, f2 = reify (elgamal$_0$ m), reify (elgamal$_1$ m) in 
  mass f1 (point c) == mass f2 (point c))
\end{lstlisting}

\fi


\section{Information-flow control}
\label{sec:ifc}

% \ch{Starting to wonder whether it is best to start the examples with
%   IFC. There is a minority at ICFP that knows / cares about this, so
%   it's still a specialized audience. Let's write it well
%   first, so that we can see if it's simpler than the rest?
%   It seems so now, so maybe it should stay here.}

% \ch{Should make it clearer what the main insights are with all this;
%   beyond ``we can do it''}

% \ch{There should be a progression between the various subsections, they
%   should build on each other, not reexplain everything from scratch.
%   This section is also missing an outline, expressing everything
%   that's in here and how it builds up.}

In this section, we present a case study examining various styles of
information-flow control (IFC), a security paradigm based on
\emph{noninterference} \cite{goguen82security}, a property that
compares two runs of a program differing only in the program's secret
inputs and requires the non-secret outputs to be equal. Many
special-purpose systems, including syntax-directed type systems, have
been devised to enforce noninterference-like security properties
\cite[see, e.g.,][]{SabelfeldMyers06IFC, HedinS12}.

We start our IFC case study by encoding a classic IFC type system 
\citep{volpano1996ifc} for a small deeply-embedded imperative
language and proving its correctness (\autoref{sec:ifc-while}).
In order to augment the permissiveness of our analysis we then show how to
compose our IFC type system with precise semantic proofs
(\autoref{sec:ifc-combining}).
\iffull
As IFC is often too strong for practical use, the final step in our IFC case
study is a semantic treatment of declassification based on delimited release
\citep{SabelfeldMyers03DL} (\autoref{sec:declassification}). An additional case
study on a runtime monitor for IFC is presented in \autoref{sec:ifc-monitor}.
\else
In the extended version~\cite{relational} we additionally treat a runtime monitor or IFC and 
delimited release. 
\fi
%
%\begin{itemize}
%\item We start our IFC case study by showing how a classic IFC type
%system \citep{volpano1996ifc} for a small, interpreted, embedded,
%imperative language can be encoded using our method. We derive the
%type system by proving each rule as a relation between two runs of the
%interpreter, mechanizing a direct correctness proof of the type system,
%rather than resorting to commonly used syntactic proof
%techniques (e.g., \citepos{PottierS03} bracketed
%semantics). (\autoref{sec:ifc-while})
%
%\item Being syntax-directed, IFC type systems provide only an imprecise,
%though automated, analysis. As a first attempt to enhance the
%precision of the analysis, we  consider an IFC dynamic
%  monitor~\cite{SabelfeldR09}, which we embed in our framework,
%  obtaining as a byproduct a  machine-checked
%  proof of soundness for it. (\autoref{sec:ifc-monitor})
%  \nig{Adapt this if removed/moved to appendix}
%
%\item In order to further augment the permissiveness of the analysis, we
%  then show how to compose  our IFC type system with
% precise, semantic proofs for the parts of a program where
%syntactic analysis does not suffice. Inspired by the approach
%of \citet{KustersTBBKM15}, we show how to compose our IFC type system
%with semantic noninterference proofs. (\autoref{sec:ifc-combining})
%
%\item Noninterference is a useful baseline property for IFC. However, on its
%own, it is often too strong for practical use. The final step in our IFC
%case study is a semantic treatment of declassification based on
%delimited release \citep{SabelfeldMyers03DL}. We show how our
%technique is applicable beyond embedded imperative languages, carrying
%out noninterference modulo delimited release proofs for \fstar
%programs in general.  (\autoref{sec:declassification})
%\end{itemize}
%
We conclude that our method for relational
verification is flexible enough to accommodate various IFC
disciplines, allowing comparisons and compositions within the same
framework.

\subsection{Deriving an IFC type system}
\label{sec:ifc-while}

\iffull
\begin{figure}
\def\MathparLineskip{\lineskip=0.1cm}
  \begin{mathpar}
\inferrule[ESub]{
  \Gamma |- e : l_1 \\ l_1 \leq l_2
}
{
  \Gamma |- e : l_2
}
\and
\inferrule[EVar]{}
{
  \Gamma |- r : \Gamma \left( r \right)
}
\and
  \inferrule[EInt]{i:\text{int}
}
{
  \Gamma |- i : \text{L}
}
\and
  \inferrule[EBinOp]{\Gamma |- e_1 : l \\ \Gamma |- e_2 : l
}
{
  \Gamma |- e_1 \oplus  e_2 : l
}
\and
  \inferrule[CSub]{\Gamma,\text{pc}:l_1 |- c \\ l_2 \leq l_1
}
{
  \Gamma, \text{pc}:l_2 |- c
}
\and
  \inferrule[CAssign]{\Gamma |- e : \Gamma \left( r \right)
}
{
  \Gamma, \text{pc}:\Gamma \left( r \right) |- r := e
}
\and
  \inferrule[CSeq]{\Gamma,\text{pc}:l |- c_1 \\ \Gamma,\text{pc}:l |- c_2
}
{
  \Gamma, \text{pc}:l |- c_1; c_2
}
\and
  \inferrule*[left=CCond]{\Gamma |- e : l \\ \Gamma,\text{pc}:l |- c_1 \\ \Gamma,\text{pc}:l |- c_2
}
{
  \Gamma, \text{pc}:l |- \text{if } e = 0 \text{ then } c_1 \text{ else } c_2
}
\and
  \inferrule*[left=CWhile]{\Gamma |- e : l \\ \Gamma,\text{pc}:l |- c  \\ \Gamma |- e' : l'
}
{
  \Gamma, \text{pc}:l |- \text{while } e \neq 0 \text{ do } c \left(\text{decr } e'\right)
}
\and
  \inferrule*[left=CSkip]{}
{
  \Gamma, \text{pc}:\text{H} |- \text{skip}
}
\end{mathpar}
\caption{A classic IFC type system}
\label{fig:ts_ifc}
\end{figure}

\else

\begin{figure}
\def\MathparLineskip{\lineskip=0.1cm}
  \begin{mathpar}
  \inferrule[CSub]{\Gamma,\text{pc}:l_1 |- c \\ l_2 \leq l_1
}
{
  \Gamma, \text{pc}:l_2 |- c
}
\and
  \inferrule[CAssign]{\Gamma |- e : \Gamma \left( r \right)
}
{
  \Gamma, \text{pc}:\Gamma \left( r \right) |- r := e
}
\and
  \inferrule*[left=CCond]{\Gamma |- e : l \\ \Gamma,\text{pc}:l |- c_1 \\ \Gamma,\text{pc}:l |- c_2
}
{
  \Gamma, \text{pc}:l |- \text{if } e = 0 \text{ then } c_1 \text{ else } c_2
}
\end{mathpar}
\caption{A classic IFC type system (selected rules)}
\label{fig:ts_ifc}
\end{figure}
\fi

Consider the following small \emph{while} language consisting of
expressions, which may only read from the heap, but not modify it, and
commands,  which may write to the heap and branch, depending on its contents.
The definition of the language
should be unsurprising, the only subtlety worth noting is
the \ls$decr$ expression in the while command, a metric used to ensure
loop termination. \ch{Why? If this is explained latter need at
  least a forward reference. I don't think it is though; it's a very
  mysterious and non-standard thing though that needs serious
  justification. And it is caused by a limitation in our F*-based
  method that we should at some point explain: we can't reason
  extrinsically about non-terminating programs.}
\ch{I wonder whether Tahina's fuel technique or Kenji's partiality
  monad would be better ways to approach this.}
\ch{Alternatively, for loops instead of while loops?}

\[
  \begin{array}{r c l}
    e & ::= &  i ~|~ r ~|~ e_1 \oplus e_2 \\
    c & ::= &  \text{skip} ~|~ r := e ~|~ c_1 ;c_2 ~|~ \text{if } e = 0 \text{ then } c_1 \text{ else } c_2 \\
    & & ~|~ \text{while } e \neq 0 \text{ do } c \left(\text{decr } e'\right)
  \end{array}
\]
%
\paragraph*{A classic IFC type system}
\citet{volpano1996ifc} devise an IFC type system \iffull for a similar
language \fi to check that programs executing over a memory containing
both secrets (stored in memory locations labeled \ls$High$) and
non-secrets (in locations labeled \ls$Low$) never leak secrets into
non-secret locations. The type system includes two judgments $\Gamma
|- e : l$, which states that the expression \ls$e$ (with free
variables in $\Gamma$) depends only on locations labeled $l$ or
lower; and $\Gamma, \text{pc}:l |- c$, which states that a command $c$
in a context that is \emph{control-dependent} on the contents of
memory locations labeled $l$, does not leak secrets. \iffull The main 
\else Some selected rules \fi of
their system, as adapted to our example language, are shown in
Figure~\ref{fig:ts_ifc}.

\iffull
Our goal in this section is to embed this \emph{while} language
in \fstar, to define an interpreter for it, and to derive Volpano et
al.'s type system by relating multiple runs of the interpreter.
%
In doing so, we highlight several distinctive features of our
approach, including the use of multiple monads to structure our
interpreter and simplify our proofs.
%
\ch{Probably don't have the space for such outlines}
\fi

\paragraph*{Multiple effects to structure the \emph{while} interpreter}
We deeply embed the syntax of \emph{while} in \fstar using
data types \ls$exp$ and \ls$com$, for expressions and commands,
respectively.
%
The expression interpreter \ls$interp_exp$ only requires reading the
value of the variables from the \ls$store$, whereas the command
interpreter, \ls$interp_com$, also requires writes to the
\ls$store$, where \ls$store$ is an integer store mapping
a fixed set of integer references `\ls$ref int$' to
\ls$int$.
%
Additionally, \ls$interp_com$ may also raise an \ls$Out_of_fuel$
exception when it detects that a loop may not terminate (e.g., because
the claimed metric is not actually decreasing).\ch{Is this an
  out of fuel exception (as the current name suggests) or a
  wrong metric exception (as the explanation suggests)?}
%
We could define both interpreters using a single effect, but this
would require us to prove that \ls$interp_exp$ does not change the
store and does not raise exceptions.
%
Avoiding the needless proof overhead, we use a \ls$Reader$ monad
for \ls$interp_exp$ and \ls$StExn$, a combined state and exceptions
monad, for \ls$interp_com$. By defining \ls$Reader$ as a \ls$sub_effect$
of \ls$StExn$, expression interpretation is transparently lifted
by \fstar{} to the larger effect when interpreting commands.
%
\iffull
\begin{lstlisting}
type reader (a:Type) = store -> Tot a
total new_effect { READER : a:Type -> Effect
  with repr = reader;
    return = fun (a:Type) (x:a) (s:store) -> x;
    bind = fun (a b : Type) (f:reader a) (g: a -> reader b) (s:store) -> 
                let z = f s in g z s; get = fun () (s:store) -> s }
type stexn (a:Type) = store -> Tot (either a exn * store)
total new_effect { STEXN $\ldots$ }
sub_effect READER ~> STEXN 
  { lift = fun (a:Type) (f:reader a) (s:store) -> let x = f s in (Inl x, s)  }
\end{lstlisting}
\fi
%
Using these effects, \ls$interp_exp$ and \ls$interp_com$ form a
standard, recursive, definitional interpreter for \emph{while}, with
the following trivial signatures. \iffull Just as we sometimes use \ls$St$,
the unindexed version of \ls$STATE$, here we use \ls$Reader$
and \ls$StExn$, unindexed versions of \ls$READER$ and \ls$STEXN$ with
simple pre- and postconditions.
\fi

\begin{lstlisting}
val interp_exp: exp -> Reader int
val interp_com: com -> StExn unit
\end{lstlisting}



% \ch{could explain the underlying representations (only that) for
% (a) an instrumented state monad with a declassification flag
%   (\ls$IfcDeclassify.fst$);
% (b) a state monad specialized to memoization \autoref{sec:memo}}

\iffull
Similarly, the memoization example from \autoref{sec:memo} uses an
effect that is specialized to the target application: a state monad
where the state is a partial finite map storing all arguments on which
a particular function was called and what answers it returned.

\fi


% \km{Do we prove anything about termination for this language ? I had the
%   impression that all proofs were proving stuff under the assumption
%   that the program correctly terminate. Also I'm not sure it is really
%   clear that the decreasing of the termination clause is checked at
%   runtime }
% \ch{keep reading, it seems accurate; it might need to be better written}
% CH: explained above now


%\begin{lstlisting}
%type exp =
%| AInt : int -> exp
%| AVar : id -> exp
%| AOp  : binop -> exp -> exp -> exp
%
%type metric = exp
%
%type com =
%| Skip   : com
%| Assign : var:id -> term:exp -> com
%| Seq    : first:com -> second:com -> com
%| If     : cond:exp -> then_branch:com -> else_branch:com -> com
%| While  : cond:exp -> body:com -> metric:metric -> com
%\end{lstlisting}

%% We then define the natural interpreting functions on expressions and commands,
%% where the interpretation of expressions uses only a reader monad, while the
%% interpretation function is using full state and exceptions. As we raise an
%% exception if the metric in a while loop is not decreasing, we can prove
%% termination for interpretation of expressions and commands. This allows us to
%% obtain pure variants of the interpretation functions by reification.

\paragraph*{Deriving IFC typing for expressions}
%
For starters, we use a \ls$store_labeling = ref int -> label$,
where \ls@label@ $\in$ \ls@{High, Low}@, to partition the store between
secrets (\ls$High$) and non-secrets (\ls$Low$).
%
An expression is noninterferent at level $l$ when its interpretation
does not depend on locations labeled greater than $l$ in the store.
%
To formalize this, we define a notion of \emph{low-equivalence} on
stores, relating stores that agree on the contents of
all \ls$Low$-labeled references, and noninterferent expressions (at
level \ls$Low$, i.e., \ls$ni_exp env e Low$) as those whose
interpretation is identical in low-equivalent stores.

\begin{lstlisting}
type low_equiv (env:store_labeling) (s0 s1:store) =
    forall (r:ref int). env r=Low ==> s0.[r] == s1.[r]
let ni_exp (env:store_labeling) (e:exp) (l:label) =
    forall (s0 s1:store). (low_equiv env s0 s1 /\ l == Low) ==> 
      reify (interp_exp e) s0 == reify (interp_exp e) s1
\end{lstlisting}

%% This gives us all the tools we need to define noninterference and to prove
%% that it is ensured by the type system. We first introduce the notion of
%% low-equivalence on heaps and then use it to define noninterference for
%% expressions:
%% \km{could the notion of low-equivalence be factored out with the presentation
%%   given in section 3.1 ?}

%% \begin{lstlisting}
%% type label_fun = id -> Tot label

%% type low_equiv (env:label_fun) (h1:rel heap) =
%%   (forall (x:id). env x = Low ==> index (R?.l h1) x = index (R?.r h1) x)

%% let ni_exp (env:label_fun) (e:exp) (l:label) : Tot Type0 =
%%   forall (h: rel heap). (low_equiv env h /\ Low? l) ==>
%%     reify (interpret_exp_st e) (R?.r h) = reify (interpret_exp_st e) (R?.l h)
%% \end{lstlisting}
%% \nig{Do we show the actual code, or the cleaner version that we would like to have?}

With this definition of noninterference for expressions we capture
the semantic interpretation of the typing judgment
$\Gamma |- e : l$: if the expression $e$ can be assigned the label \ls$Low$, 
then the computation of $e$ is only influenced by \ls$Low$ values.
%
\iffull
Using this definition, we can derive the expression rules of
Figure~\ref{fig:ts_ifc}; for instance here is a lemma for the \textsc{EBinOp} rule:

\begin{lstlisting}
let binop_exp (env:store_labeling) (op:binop) (e1 e2:exp) (l:label)
 : Lemma (requires (ni_exp env e1 l /\ ni_exp env e2 l)) 
         (ensures (ni_exp env (AOp op e1 e2) l)) = ()
\end{lstlisting}

We construct a lemma from the inference rule in a straightforward
manner: the premise of the inference rule forms the \ls$requires$
clause, while the conclusion of the rule forms the \ls$ensures$ clause.
The proof for this lemma is simple and can be discharged
purely by SMT, without the need of any further annotations. The other
rules for expressions can be shown in the same way and all of them can
be discharged by SMT.
\fi

\paragraph*{Deriving IFC typing for commands} As explained previously, the
judgment $\Gamma, \text{pc}:l |- c$ deems $c$ noninterferent when run
in context control-dependent only on locations whose label is at most
$l$. More explicitly, the judgment establishes the following two
properties:
(1) locations labeled below $l$ are not modified by $c$---this is
  captured by \ls$no_write_down$, a unary property;
(2) the command $c$ does not leak the contents of a \ls$High$ location to \ls$Low$
  location---this is captured by \ls$ni_com'$, a binary property.

%% \ch{This seems like nonsense, the
%% language is terminating because of the metrics!} \nik{no, the metric
%% need not actually decrease and the interpreter will raise \ls$Out_of_fuel$;
%% but actually, \ls$no_write_down$ will be enforced even for divergent
%% programs. So, why only handle the non-exceptional case here?}
%% \ch{We can probably also handle the exceptional case, still
%%   exceptions and nontermination are not at all the same effect.
%%   IFC people sometimes mix these things up, but I generally call
%%   error-insensitive a NI property that doesn't work for errors/exceptions.}
%% \nig{The problem here is that the diverging loop may be followed by low
%% assignments that are executed only in one of the two runs and may break
%% low-equivalence (\EG \texttt{while h <> 0 do \{skip\}; lo := 0})}


\begin{lstlisting}
let run c s = match reify (interp_com c) s with
  | Inr Out_of_fuel, _ -> Loops  | _, s' -> Returns s'
let no_write_down env c l s = match run c s with
  | Loops -> True    | Returns s' -> forall (i:id). env i < l ==> s'.[i] == s.[i]
let ni_com' env c l s0 s1 = match run c s0, run c s1 with
  | Returns s0', Returns s1' -> low_equiv env s0 s1 ==> 
      low_equiv env s0' s1'
  | Loops, _ | _, Loops -> True
\end{lstlisting}

The type system is termination-insensitive,
meaning that a program may diverge depending on the value of a
secret. Consider, for instance, two runs of the program
\texttt{while hi <> 0 do \{skip\}; lo := 0}, one with \texttt{hi = 0}
and another with \texttt{hi = 1}. The first run terminates and writes
to \texttt{lo}; the second run loops forever. As such, we do not
expect to prove noninterference in case the program loops.
\iffull

\fi
Putting the pieces together, we define
%
$\Gamma, \text{pc}:l |- c$ to be \ls@ni_com $\Gamma$ $c$ $l$@.
%
\begin{lstlisting}
let ni_com (env:store_labeling) (c:com) (l:label) = 
  (forall s0 s1.  ni_com' env c l s0 s1) /\ (forall s.  no_write_down env c l s)
\end{lstlisting}

As in the case of expression typing, we derive each rule of the
command-typing judgment as a lemma about \ls$ni_com$. For example,
here is the statement for the \textsc{CCond} rule:

\begin{lstlisting}
val cond_com (env:store_labeling)(e:exp)(ct:com)(cf:com)(l:label)
: Lemma (requires (ni_exp env e l /\ ni_com env ct l 
                               /\ ni_com env cf l))
        (ensures  (ni_com env (If e ct cf) l))
\end{lstlisting}

\noindent The proofs of many of these rules
are partially automated by SMT---they take about 250 lines of
specification and proof in \fstar. Once proven, we use these rules to
build a certified, syntax-directed typechecker for \emph{while}
programs that repeatedly applies these lemmas to prove that
% an expression satisfies \ls$ni_exp$ and
% val tc_exp : env:store_labeling -> e:exp -> Pure label (requires True) (ensures ni_exp env e)
a program satisfies \ls$ni_com$. This \iffull certified \fi typechecker has the
following type:
%
\begin{lstlisting}
val tc_com : env:store_labeling -> c:com -> 
  Exn label (requires True) (ensures fun Inl l -> ni_com env c l | _ -> True)
\end{lstlisting}

% \ch{If it's easy why not do it?}

% \nig{Not sure which example should be used here, the interesing ones have
%   multiple lemmas and large proofs}


\subsection{Combining syntactic IFC analysis with semantic noninterference proofs}
\label{sec:ifc-combining}

Building on \autoref{sec:ifc-while}, we show how programs that fall
outside the syntactic information-flow typing discipline can be proven
secure using a combination of typechecking and
semantic proofs of noninterference. This example is evocative (though
at a smaller scale) of the work of \citet{KustersTBBKM15}, who combine
automated information-flow analysis in the Joana analyzer
\cite{HammerS09} with semantic proofs
% (of unary properties) -- CH: false claim
in the KeY verifier for Java programs \cite{DarvasHS05, SchebenS11}.
In contrast, we sketch a combination of syntactic and semantic
proofs of relational properties in {\em a single} framework.
Consider the following \emph{while} program, where the label of \ls$c$
and \ls$lo$ is Low and the label of \ls$hi$ is High.
%lo := hi * 0;
% \km{using lstlisting here instead of verbatim to have easily line numbering,
%   language could be changed but don't forget to load it in the preamble}
\begin{lstlisting}[language=caml]
  while c <> 0 do hi := lo + 1; lo := hi + 1; c := c $-$ 1 (decr c)
\end{lstlisting}
The assignment \ls$lo := hi + 1$ is ill-typed in the type system
of \S\ref{sec:ifc-while}, since it directly assigns a \ls$High$
expression to a \ls$Low$ location.
%
However, the previous command overwrites \ls$hi$ so that
\ls$hi$ does not contain a \ls$High$ value anymore at that point.
%
As such, even though the IFC type system cannot prove it, the program
is actually noninterferent.
%
To prove it, one could directly attempt to prove \ls$ni_com$ for the
entire program, which would require a strong enough (relational)
invariant for the loop. A simpler approach is to prove just the
sub-program \ls$hi := lo + 1; lo := hi + 1$ (\ls$c_s$) 
noninterferent, while relying on the
type system for the rest of the program.
The sub-program can be automatically proven secure:

\begin{lstlisting}
let c_s_ni () : Lemma (ni_com env c_s Low) = ()
\end{lstlisting}

\noindent This lemma has exactly the form of the other standard,
typing rules proven previously, except it is specialized to the
command in question. As such, \ls$c_s_ni$ can just be used in place
of the standard sequence-typing rule (\textsc{CSeq}) when proving the
while loop noninterferent.

We can even modify our automatic typechecker from \autoref{sec:ifc-while} 
to take as input a list of commands that are already proved
noninterferent (by whichever means), and simply look up the command it
tries to typecheck in the list before trying to typecheck it syntactically.
%
The type (and omitted implementation) of this typechecker is very
similar to that of \ls$tc_com$, the only difference is the extra list argument:
%
\begin{lstlisting}
val tc_com_hybrid : env:store_labeling -> c:com -> 
  list (cl:(com*label){ni_com env (fst cl) (snd cl)}) ->
  Exn label (ensures fun ol -> Inl? ol ==> ni_com env c (Inl?.v ol))
\end{lstlisting}
%
We can complete the noninterference proof automatically by passing the
\ls$(c_s, Low)$ pair proved in \ls$ni_com$ by lemma
\ls$c_s_ni$ (or directly by SMT) to this hybrid IFC typechecker:
%
\begin{lstlisting}
let c_loop_ni () : Lemma (ensures ni_com env c_loop Low) =
 c_s_ni(); ignore (reify (tc_com_hybrid env c_loop [c_s, Low]) ())
\end{lstlisting}
%
Checking this in \fstar{} works by simply evaluating the invocation
of \ls$tc_com_hybrid$; this reduces fully to \ls$Inl Low$ and
the intrinsic type of \ls$tc_com_hybrid$ ensures the postcondition.

% \iffull
% More explicitly, to show that the command in line 4 satisfies
% noninterference, we could also use the SMT-solver, but we prefer to
% showcase the application of the typing rules.

% \begin{lstlisting}
% let c1_4_ni () : Lemma (ni_com env c1_4 Low) =
%   avar_exp env c; aint_exp env 1;
%   binop_exp env Minus (AVar c) (AInt 1) Low;
%   assign_com env (AOp Minus (AVar c) (AInt 1)) c
% \end{lstlisting}

% One can see, that the applied lemmas perfectly mirror the typing derivation for
% the command: we first TODO
% %

% %
% Using the sequencing rule, we can prove a noninterference lemma \ls$body_ni$
% for the body and conclude the proof for the entire program :

% \begin{lstlisting}
% val p_ni : unit -> Lemma (ni_com env p Low)
% let p_ni () = avar_exp env c; body_ni (); while_com env (AVar c) body (AVar c) Low
% \end{lstlisting}

% \nig{Is this understandable or do we need to define all the variables like c2, body, \ETC ?}
% \km{making the mapping [typechecking-rule -> lemma] explicit may be nice ; Also
%   why not use c\_1\_4 instaead of c since you are using it in the lemma ? }
% \fi

\iffull
\subsection{Semantic declassification}
\label{sec:declassification}

Beyond noninterference, reasoning directly about relational properties
allows us to characterize various forms of \emph{declassification}
where programs intentionally reveal some information about
secrets. For example, \citet{SabelfeldMyers03DL}
propose \emph{delimited release}, a discipline in which programs are
allowed to reveal the value of only certain pure expressions.

In a simple example by Sabelfeld and Myers
some amount of money (\ls$k$) is transferred
from one account (\ls$hi$) to another (\ls$lo$). Simply by
observing whether or not the funds are received, the owner of the
\ls$lo$ account gains some information about the other account, namely
whether or not \ls$hi$ contained at least \ls$k$ units of
currency---this is, however, by design.

\begin{lstlisting}
let transfer (k:int) (hi:ref int) (lo:ref int) = 
  if k < !hi then (hi := !hi $-$ k; lo := !lo + k)
\end{lstlisting}

To characterize this kind of intentional release of information,
delimited release describes two runs of a
program in initial states where the secrets, instead of being
arbitrary, are related in some manner, e.g., the initial states
agree on the value of the term being explicitly declassified. This is
easily captured in our setting. For example, we can prove the
following lemma for \ls$transfer$, which shows that \ls$lo$ gains no
more information than intended.

% \pagebreak[3]

\begin{lstlisting}
let transfer_ok (k:int) (hi lo:ref int{addr_of lo <> addr_of hi}) 
  (s0 s1:heap{lo $\in$ s0 /\ hi $\in$ s0 /\ lo $\in$ s1 /\ hi $\in$ s1}) : Lemma
     (* initial memories agree on lo and on the declassified term *)
     (requires (s0.[lo] == s1.[lo] /\ (k < s0.[hi] <==> k < s1.[hi]))) 
     (ensures ((snd (reify (transfer k hi lo) s0)).[lo] == 
               (snd (reify (transfer k hi lo) s1)).[lo])) = ()
\end{lstlisting}

% \section{When dimension of declassification}
% \label{sec:when-declassification}



Delimited release was about the {\em what} dimension of
declassification \cite{SabelfeldS09}.
%
We also built a very simple model that is targeted at the {\em when}
dimension, illustrating a customization of the monadic model to the
target relational property.
%
For instance, to track when information is declassified, we augment
the state with a bit recording whether the secret component of the
state was declassified and is thus allowed to be leaked.
%
\begin{lstlisting}
type ifc_state = { secret:int; public:int; release:bool }
new_effect STATE_IFC = STATE_h ifc_state
\end{lstlisting}
%
In this case the noninterference property depends on the extra
instrumentation bit we added to the state.
%
\begin{lstlisting}
let ni (f:unit -> St unit) =
 forall s0 s1. let (_, s0'), (_, s1') = reify (f ()) s0, reify (f ()) s1 in
  s0'.release \/ s1'.release \/ (low_equiv s0 s1 ==> low_equiv s0' s1')
\end{lstlisting}

\subsection{Soundness of an IFC monitor}
\label{sec:ifc-monitor}

\begin{figure}
  \begin{mathpar}
\inferrule[EVar]{}
{
  S, \Gamma |- r \rightarrow  \left< S \left( r \right), \Gamma \left( r \right) \right>
}
\and
  \inferrule[EInt]{i:\text{int}
}
{
  S, \Gamma |- i \rightarrow \left<i, \text{L} \right>
}
\and
  \inferrule[EBinOp]{S |- e_1 \rightarrow \left<v_1,l_1\right> \\
    S, \Gamma |- e_2  \rightarrow \left< v_2,l_2 \right>
}
{
  S, \Gamma |- e_1 \oplus  e_2 \rightarrow \left< v_1 \oplus v_2, l_1 \sqcup l_2 \right>
}
%\and
%  \inferrule[CSkip]{}
%{
%  S, \Gamma, \text{pc} |- \text{skip} \rightarrow S
%}
\and
  \inferrule[CAssign]{S, \Gamma |- e \rightarrow \left< v_e, l_e \right> \\
    \Gamma \left( r \right) = l_r \\\\
    l_e \sqcup \text{pc} \leq l_r
}
{
  S, \Gamma, \text{pc} |- r := e \rightarrow S\left[r \mapsto v_e\right]
}
\and
  \inferrule[CCondTrue]{S, \Gamma |- e \rightarrow \left< v_e, l_e \right> \\
    v_e = 0 \\
    S, \Gamma,(\text{pc} \sqcup l_e) |- c_1 \rightarrow S_1
}
{
  S, \Gamma, \text{pc} |- \text{if } e = 0 \text{ then } c_1 \text{ else } c_2 \rightarrow S_1
}
%\and
%  \inferrule[CCondFalse]{S, \Gamma |- e \rightarrow \left< v_e, l_e \right> \\
%    v_e \neq 0 \\
%    S, \Gamma,(\text{pc} \sqcup l_e) |- c_2 \rightarrow S_2
%}
%{
%  S, \Gamma, \text{pc} |- \text{if } e = 0 \text{ then } c_1 \text{ else } c_2 \rightarrow S_2
%}
%\and
%  \inferrule[CSeq]{S, \Gamma,\text{pc} |- c_1 \rightarrow S_1 \\
%  S_1, \Gamma, \text{pc} |- c_2 \rightarrow S_2
%}
%{
%  S, \Gamma, \text{pc} |- c_1; c_2 \rightarrow S_2
%}
%\and
%  \inferrule[CWhileStop]{S, \Gamma |- e \rightarrow \left< v_e, l_e \right> \\
%    v_e = 0
%}
%{
 % S, \Gamma, \text{pc} |- \text{while } e \neq 0 \text{ do } c \left(\text{decr } e'\right) \rightarrow S
%}
%\and
%  \inferrule[CWhileLoop]{S, \Gamma |- e \rightarrow \left< v_e, l_e \right> \\
 %   v_e \neq 0 \\
 %   S, \Gamma,(\text{pc} \sqcup l_e) |- c \rightarrow S_1\\
  %  S, \Gamma |- e' \rightarrow \left< v_{m,0}, l_{m,0} \right> \\
  %  S_1, \Gamma|- e' \rightarrow \left< v_{m,1}, l_{m,1} \right> \\
   % v_{m,1} < v_{m,0} \\
   % S_1, \Gamma,\text{pc} |- \text{while } e \neq 0 \text{ do } c \left(\text{decr } e'\right) \rightarrow S_2
%}
%{
 % S, \Gamma \text{pc} |- \text{while } e \neq 0 \text{ do } c \left(\text{decr } e'\right) \rightarrow S_2
%}
\end{mathpar}
\caption{Semantics of the IFC monitor}
\label{fig:ifc_monitor}
\end{figure}


Another popular technique for the enforcement of IFC are
runtime monitors: the idea is to dynamically track the security labels
of  expressions and to check them at runtime in order to detect  IFC
violations, which cause the execution to halt. Here we implement an interpreter for the while language presented in
\autoref{sec:ifc-while} extended with the security monitor proposed by
\citet{SabelfeldR09}: a selection of the semantic rules is reported in \autoref{fig:ifc_monitor}.
%enhanced with the
%possibility to downgrade labels dynamically, which is
%a step in the direction of flow-sensitive IFC~\cite{HuntS06}.
The store $S$ maps references to integers, while the store labeling $\Gamma$
maps references to security labels, which are then used to derive labels for
expressions. Assignments are subject to the expected security checks at
run-time.
%The rules are self explanatory. We just highlight {\sc CAssign}, which
%performs the expected checks to avoid direct (by comparing the label $l_r$ of
%the reference with the label $l_e$ of the assigned expression)  as well as
%indirect (via the \textit{pc} label) information flows, and at the same time
%sets the reference label to the join of $l_e$ and $pc$, thereby implementing a
%form of flow sensitivity for references.


We embed the monitor in \fstar, obtaining a machine-checked
proof of soundness for it.
% The setting is the same as in \autoref{sec:ifc-while}, with the only difference
% that the store does now hold integers paired with security labels.
The interpretation functions for expressions and commands have the following
signatures:

\begin{lstlisting}
val interp_exp_monitor: store_labeling -> exp -> Reader (int * label)
val interp_com_monitor: store_labeling -> label -> com -> StExn unit
\end{lstlisting}

We prove termination-insensitive non-interference for interpretation with the
monitor and capture this with the following lemma:

\begin{lstlisting}
val dyn_ifc (s0:store) (s1:store) (env:store_labeling) (c:com) (pc:label) :
    Lemma (requires (low_equiv env s0 s1))
      (ensures (match (reify (interp_com_monitor env pc c)) s0, 
                     (reify (interp_com_monitor env pc c)) s1 with
               | (Inl _, s0'), (Inl _, s1')-> low_equiv env s0' s1'
               | _ -> True))
\end{lstlisting}

Intuitively, we show that for any two low-equivalent initial stores,
the two resulting
stores are also low equivalent, if the interpretation with the monitor
terminates without a runtime exception.

While the result looks similar to the one shown for the type system,
there is a subtle difference in the enforced security property. Consider the
following example where the label of \ls$hi$ is High and the label of \ls$lo$
is Low:
\begin{lstlisting}[language=caml]
if (hi=0) skip else lo := 0
\end{lstlisting}
The assignment to a low reference on the else branch is leaking
information about the value of the high reference in the conditional expression.
Nevertheless, if the then-branch of the
conditional is taken, the monitor will not report a violation, as it does not
inspect the else-branch.
%This is a known limitation of purely dynamic IFC monitors and can be solved
%using a hybrid approach~\cite{RussoSabelfeldDunamicvsstatic?}.
This example does however not break our theorem, since our theorem only relates
pairs of programs that terminate normally,
while for all stores in which the else branch
is taken, the execution of the interpreter halts with an error.
%
The monitor is collapsing the implicit-flow channel into an
erroneous termination channel, thereby enforcing error-insensitive
non-interference.
%
% As argued in ~\citet{SabelfeldR09}, the monitor is
% collapsing the implicit-flow channel into a termination channel,
% thereby enforcing termination-insensitive non-interference.\ch{I don't
%   disagree with the main idea, but find this terminology stupid and
%   confusing. Errors are not the same as nontermination, so this monitor
%   is error-insensitive.}
%
For comparison, notice that the (termination-insensitive)
type system from \autoref{sec:ifc-while}  accepts a
variant of the program above, in which the low assignment is replaced by a
non-terminating loop.

%The following example shows that the monitor is more permissive than
%the  type system from \autoref{sec:ifc-while} due to its dynamic
%treatment of security labels:
%\begin{lstlisting}[language=caml,numbers=left,stepnumber=1]
%  hi := lo; $\label{line:dyn2-hi-assign}$
%  lo := hi  $\label{line:dyn2-lo-assign}$
%\end{lstlisting}
%While this example does not typecheck due to the assignment in
%line~\ref{line:dyn2-lo-assign}, the execution with the monitor succeeds. This
%is because we can downgrade the label of the reference \ls$hi$ to Low
%after the assignment in line~\ref{line:dyn2-hi-assign}.

% %
% \ch{How about having a normal ML heap and associating the bit to each
%   of the reference cells? This would be less add hoc, basically the
%   bit is an information-flow control label that can dynamically
%   change during execution. Basically value(?) sensitive.}

\fi

\section{Program optimizations and refinement}
\label{sec:equiv}
\label{sec:refinement}

This section presents two complete examples to prove a few,
classic algorithmic optimizations correct. These properties are very
specific to their application domains and a special-purpose
relational logic would probably not be
suitable. Instead, we make use of the generality of our approach to
prove application-specific relational properties (including $4$- and
$6$-ary relations) of higher-order programs with local state.  In
contrast, most prior relational logics are specialized to proving
binary relations, or, at best, properties of $n$ runs of a single
first-order program \citep{SousaD16}.

\subsection{Effect for memoizing recursive functions}
\label{sec:memo}

First, we look at memoizing total functions,
%\ls$dom -> codom$,
including memoizing a function's recursive calls 
based on a partiality representation
technique due to \citet{McBride15}.
%
We prove that a memoized function is extensionally equal to the original.

%% Our monadic framework for relational verification encourages the use
%% of new monadic effects (\autoref{sec:monads}) to add a tailor-made
%% effect \ls$Memo$ supporting the operations that we need for the
%% modeling task at hand.
%% %
%% Abstracting the concrete details of memory management, the
%% effect \ls$Memo$ is a state monad where the state consists of a
We define a custom effect \ls$Memo$, a monad with a state consisting of a
(partial, finite) mapping from a function's domain type (\ls$dom$)
to its codomain type (\ls$codom$),
%
with two actions:
%
\ls$get : dom -> Memo (option codom)$, which returns a memoized value if it exists; and
\ls$put : dom -> codom -> Memo unit$, which adds a new memoization pair
  to the state.\footnote{
This abstract model could be implemented efficiently,
for instance by an imperative hash-table with a specific
memory-management policy.}

%

\paragraph*{Take 1: Memoizing total functions} Our
goal is to turn a total function \ls$g$ into a memoized
function \ls$f$ computing the same values as \ls$g$.
%
This relation between \ls$f$'s reification and \ls$g$ is captured by
the \ls$computes$ predicate below, depending on an invariant of the
memoization state, \ls$valid_memo$.
%
A memoization state \ls$h$ is valid for memoizing some total
function \ls$g : (dom -> codom)$ when \ls$h$ is a subset of the graph
of \ls$g$:
%
\begin{lstlisting}
let valid_memo (h:memo_st) (g:dom -> codom) = 
  for_all_prop (fun (x,y) -> y == g x) h
let computes (f: dom -> Memo codom) (g:dom -> codom) =
  forall h0. valid_memo h0 g ==> (forall x. (let y, h1 = reify (f x) h0 in 
                                y == g x /\ valid_memo h1 g))
\end{lstlisting}
%
We have \ls$f `computes` g$ when given any state \ls$h0$ containing
a subgraph of \ls$g$, \ls$f x$ returns \ls$g x$ and maintains the
invariant that the result state \ls$h1$ is a subgraph of \ls$g$.
%
It is easy to program and verify a simple memoizing function:
\begin{lstlisting}
let memoize (g : dom -> codom) (x:dom) = 
  match get x with Some y -> y | None -> let y = g x in put x y; y
let memoize_computes g :Lemma ((memoize g) `computes` g) = ...
\end{lstlisting}
%
The proof of this lemma is straightforward: we only need to show that
the value \ls$y$ we get back from the heap in the first branch is
indeed \ls$g x$ which is enforced by the \ls$valid_memo$ in the
precondition of \ls$computes$.

\paragraph*{Take 2: Memoizing recursive calls}
Now, what if we want to memoize a recursive function, for
example, a function computing the Fibonacci sequence?
%
We also want to memoize the intermediate recursive calls, and in order
to achieve it, we need an explicit representation of the recursive
structure of the function.
%
\km{Should the encoding of recursive functions be motivated a little
  more?  explaining why the obvious
%
$(x0:dom -> (x:dom{x << x0} -> codom) -> codom)$ does not work ?}
%
Following \citet{McBride15}, we represent this by a function
\ls$x:dom -> partial_result x$, where a partial result is either a finished
computation of type \ls$codom$ or a request for a recursive call
together with a continuation.
%
\begin{lstlisting}
type partial_result (x0:dom) =
  | Done : codom -> partial_result x0
  | Need : x:dom{x << x0} -> cont:(codom -> partial_result x0) ->
         partial_result x0
\end{lstlisting}
%
As we define the fixed point using \ls$Need x f$, we crucially
require \ls$x << x0$, meaning that the value of the function is requested
at a point \ls$x$ where function's definition already exists.
%
For example encoding Fibonacci
amounts to the following code where the two recursive calls in the
second branch have been replaced by applications of the \ls$Need$
constructor.
We also define the fixpoint of such a function representation \ls$f$:
%
\begin{lstlisting}
let fib_skel (x:dom) : partial_result x =
  if x <= 1 then Done 1 else
    Need (x $-$ 1) (fun y$_1$ -> Need (x $-$ 2) (fun y$_2$ -> Done (y$_1$ + y$_2$)))
let rec fixp (f: x:dom -> partial_result x) (x0:dom) : codom =
  let rec complete_fixp x = function
    | Done y -> y
    | Need x' cont -> let y = fixp f x' in complete_fixp x (cont y)
  in complete_fixp x0 (f x0)
\end{lstlisting}
%
To obtain a memoized fixpoint, we need to memoize functions defined
only on part of the domain, \ls$x:dom{p x}$.
%
\begin{lstlisting}
let partial_memoize (p:dom -> Type) 
  (f : x:dom{p x} -> Memo codom) (x:dom{p x}) =
  match get x with Some y -> y | None -> let y = g x in put x y; y
let rec memoize_rec (f: x:dom -> partial_result x) (x0:dom) =
  let rec complete_memo_rec x :Memo codom = function
    | Done y -> y
    | Need x' cont -> 
      let y = partial_memoize (fun y -> y << x) (memoize_rec f) x' in 
      complete_memo_rec (cont y)
  in complete_memo_rec x0 (f x0)
\end{lstlisting}
%
\iffull Since both functions are syntactically similar i\else{I}\fi{}t
is relatively easy
to prove by structural induction on the code of \ls$memoize_rec$
that, for any skeleton of a recursive function
%
\ls$f$, we have that
% \ls$memoize_rec_lemma f$ asserting
%
\ls$(memoize_rec f) `computes` (fixp f)$.
%
\km{a slight complication is due to the decreasing clauses
but bringing them in will make the explanation pretty messy...}
%
The harder part is proving that \ls$fixp fib_skel$ is extensionally equal to
\ls$fibonacci$, the natural recursive definition of the sequence,
as these two functions are not syntactically similar---however, the 
proof involves reasoning only about pure functions.
%
As we have already proven that
%
\ls$memoize_rec fib_skel$ computes \ls$fixp fib_skel$, we easily gain a proof of the
equivalence of \ls$memoize_rec fib_skel$ to \ls$fibonacci$ by transitivity.
\iffull

Finally, we can encapsulate the \ls$Memo$ effect and provide a pure
state-passing interface:
%
\begin{lstlisting}
type memo_pack (f:dom -> codom) =
  | MemoPack : h0:memo_st{valid_memo h0 f} ->  
    mf:(dom -> Memo codom){mf `computes` f} -> memo_pack f
let apply_memo (#f:dom -> codom) (mp:memo_pack f) (x:dom) : 
    (codom * memo_pack f) =
  let MemoPack h0 mf = mp in let y, h1 = reify (mf x) h0 in 
  y, MemoPack h1 mf
let mk_memo_pack f : memo_pack (fixp f) = memo_lemma f ;
  MemoPack [] (memoize_rec f)
\end{lstlisting}
\fi
%
% \km{Not exactly sure how to bring up observational purity here. Following Nik's
%   advice it is possible to pack up a memoized function with its memoized context
%   so that it can be (nicely) used in a pure setting but this is really similar
%   to state-passing code. Moreover since the memoization context is increasing
%   it is hard to claim that we cannot observe the effect on the returned context
%   (unless the plan was to use module abstraction). It seems a little wiser to
%   leave that claim to a future work using Danel's work on monotonicity.}

\subsection{Stepwise refinement and $n$-ary relations: Union-find with two optimizations}
\label{sec:unionfind}

In this section, we prove several classic optimizations of a
union-find data structure introduced in several stages, each a
refinement.
%
For each refinement step, we employ relational verification to prove
that the refinement preserves the canonical structure of union-find.
%
We specify correctness using, in some cases, $4$- and $6$-ary
relations, which are easily manipulated in our monadic framework.

\mypara{Basic union-find implementation} A union-find data structure
maintains disjoint partitions of a set, such that each
element belongs to exactly one of the partitions. The data structure
supports two operations: \ls$find$, that identifies to which partition an
element belongs, and \ls$union$, that takes as input two elements
and combines their partitions.

An efficient way to implement the union-find data structure is as a
forest of disjoint trees, one tree for each partition, where each node
maintains its parent and the root of each tree is the designated
representative of its partition.
%
The find operation returns the root of a given element's partition (by
traversing the parent links), and the union operation simply points
one of the roots to the other.

We represent a union-find of set
\ls@[0, n $-$ 1]@ as the type `\ls$uf_forest n$' (below),
a sequence of ref cells, where the \ls{i$^\textit{th}$} element in the
sequence is the \ls{i$^\textit{th}$} set element, containing its parent and the
list of all the nodes in the subtree rooted at that node. The list
is computationally irrelevant (\IE \emph{erased})---we only use it to
express the disjointness invariant and the termination metric for
recursive functions (e.g. \ls{find}).

\begin{lstlisting}
type elt (n:$\mathbb{N}$) = i:$\mathbb{N}${i $<$ n} $\times$ erased (list $\mathbb{N}$)
type uf$\_$forest (n:$\mathbb{N}$) = s:seq (ref (elt n)){length s = n}
\end{lstlisting}

\iffull
The liveness and disjointness invariants on a union-find forest are:

\begin{lstlisting}
(* all the refs are distinct and live in the heap *)
let live (#n:$\mathbb{N}$) (uf:uf$\_$forest n) (h:heap) =
($\forall$ i j. i $\not\eq$ j $\Rightarrow$ distinct uf[i] uf[j])  $\wedge$  ($\forall$ i. uf[i] $\in$ h)
let disjoint (#n:$\mathbb{N}$) (uf:uf$\_$forest n) (h:heap) =
  $\forall$ i. i $\in$ (subtree uf i h) $\wedge$  (* i is in its own subtree *)
     (* set$\_$n is the set of all numbers from 0 to n $-$ 1 *)
     subtree uf i h $\subseteq$ set$\_$n n $\wedge$
     (* i's subtree is a subset of its parent's subtree *)
     is$\_$root i $\vee$ subtree uf i h $\subset$ subtree uf (parent uf i h) h $\wedge$
     (* disjointness of subtrees *)
     $\forall$ j. (i $\not\eq$ j $\wedge$ is$\_$root uf i h $\wedge$ is$\_$root uf j h)
       $\Rightarrow$ subtree uf i h $\cap$ subtree uf j h = $\phi$
\end{lstlisting}

\fi

The basic \ls$find$ and \ls$union$ operations are shown below,
where \ls{set} and \ls{get} are stateful functions that read and
write the \ls{i$^\text{th}$} index in the \ls{uf} sequence.
%
Reasoning about mutable pointer structures requires maintaining
invariants regarding the liveness and separation of the memory
referenced by the pointers. While important, these are orthogonal to
the relational refinement proofs---so we elide them here,
but still prove them intrinsically in our code.

\begin{lstlisting}
let rec find #n uf i = let p, $\_$ = 
  get uf i in if p = i then i else find uf p 
let union #n uf i$_1$ i$_2$ = let r$_1$, r$_2$ = find uf i$_1$, find uf i$_2$ in 
    let $\_$, s$_1$ = get uf r$_1$ in let $\_$, s$_2$ = get uf r$_2$ in
    if r$_1$ <> r$_2$ then (set uf r$_1$ (r$_2$, s$_1$); set uf r$_2$ (r$_2$, union s$_1$ s$_2$)) 
\end{lstlisting}

\mypara{Union by rank} The first optimization we consider
is \iffull improving \ls$union$ to \fi \ls$union_by_rank$, which decides whether
to merge \ls@r$_1$@ into \ls@r$_2$@, or vice versa, depending on the
heights of each tree, aiming to keep the trees shallow.
%
We prove this optimization in two steps, first refining the
representation of elements by adding a rank field to \ls$elt n$ and then
proving that \ls$union_by_rank$ maintains the same set partitioning
as \ls$union$.

\begin{lstlisting}
type elt (n:$\mathbb{N}$) = i:$\mathbb{N}${i $<$ n} $\times$ $\mathbb{N}$ $\times$ erased (list nat) (* added rank *)
\end{lstlisting}

We formally reason about the refinement by proving that the outputs of
the \ls{find} and \ls{union} functions do not depend on the newly
added rank field.
%
The \ls$rank_independence$ lemma (a $4$-ary relation) states
that \ls{find} and \ls{union} when run on two heaps that differ only
on the rank field, output equal results and the resulting heaps also
differ only on the rank field.
%
\begin{lstlisting}
let equal_but_rank uf h$_1$ h$_2$ =  $\forall$ i. parent uf i h$_1$ = parent uf i h$_2$
                             $\wedge$ subtree uf i h$_1$ = subtree uf i h$_2$
let rank$\_$independence #n uf i i$_1$ i$_2$ h$_1$ h$_2$ : Lemma
(requires (equal$\_$but$\_$rank uf h$_1$ h$_2$))
(ensures  (let (r$_1$,f$_1$), (r$_2$,f$_2$) = 
  reify (find uf i) h$_1$,reify (find uf i) h$_2$ in
 let ($\_$,u$_1$), ($\_$,u$_2$) = 
  reify (union uf i$_1$ i$_2$) h$_1$,reify (union uf i$_1$ i$_2$) h$_2$ in
 r$_1$ == r$_2$ $\wedge$ equal$\_$but$\_$rank uf f$_1$ f$_2$ /\ equal$\_$but$\_$rank uf u$_1$ u$_2$))
\end{lstlisting}

\iffull

\mypara{Union by rank} The rank based union optimization aims at
minimizing the height of the subtrees, so that the tree traversal is
more efficient. It does so by pointing the root with smaller height to
the other root during the union operation.

\begin{lstlisting}
let union$\_$opt #n uf i$_1$ i$_2$ =
  let r$_1$, r$_2$ = find uf i$_1$, find uf i$_2$ in
  let $\_$, d$_1$, s$_1$ = get uf r$_1$ in let $\_$, d$_2$, s$_2$ = get uf r$_2$ in
  if r$_1$ = r$_2$ then ()
  else begin
    if d$_1$ $<$ d$_2$ then begin (* point r$_1$ to r$_2$ *)
      set uf r$_1$ (r$_2$, d$_1$, s$_1$); set uf r$_2$ (r$_2$, d$_2$, union s$_1$ s$_2$)
    end
    else begin  (* point $r_2$ to r$_1$ and adjust $r_1$'s height *)
      set uf r$_2$ (r$_1$, d$_2$, s$_2$);
      let d$_1$ = if d$_1$ = d$_2$ then d$_1$ + 1 else d$_1$ in
      set uf r$_1$ (r$_1$, d$_1$, union s$_1$ s$_2$)
    end
  end
\end{lstlisting}
\fi

Next, we prove the \ls$union_by_rank$ refinement
sound. Suppose we run \ls{union} and \ls$union_by_rank$ in \ls{h}
on a heap \ls{h} producing \ls{h$_1$} and \ls{h$_2$}.
%
Clearly, we cannot prove that find for a node \ls{j} returns the same
result in \ls{h$_1$} and \ls{h$_2$}. But we prove that the canonical
structure of the forest is the same in \ls{h$_1$} and \ls{h$_2$}, by
showing that two nodes are in the same partition in \ls{h$_1$} if and only if
they are in the same partition in \ls{h$_2$}:

\begin{lstlisting}
val union_by_rank_refinement #n uf i$_1$ i$_2$ h j$_1$ j$_2$ : Lemma
  (let (_, h$_1$), (_, h$_2$) = 
    reify (union uf i$_1$ i$_2$) h, reify (union_by_rank uf i$_1$ i$_2$) h in
   fst (reify (find uf j$_1$) h$_1$) == fst (reify (find uf j$_2$) h$_1$) <==> 
     fst (reify (find uf j$_1$) h$_2$) == fst (reify (find uf j$_2$) h$_2$))
\end{lstlisting}

This property is $6$-ary relation, relating 1 run of \ls$union$ and
1 run of \ls$union_by_rank$ to 4 runs of \ls$find$---its proof is a
relatively straightforward case analysis.

\mypara{Path compression} Finally, we consider \ls$find_compress$,
which, in addition to returning the root for an element, sets the root
as the element's new parent to accelerate subsequent find queries.
\iffull

\begin{lstlisting}
let rec find$\_$opt #n uf i =
  let p, d, s = get uf i in
  if p = i then i
  else
    let r = find$\_$opt uf p in
    set uf i (r, d, s);
    r
\end{lstlisting}

\fi
To prove the refinement of \ls{find} to \ls{find_compress} sound, we
prove a $4$-ary relation showing that if running \ls{find} and
\ls{find_compress} on a heap \ls{h} results in the heaps \ls{h$_1$} and \ls{h$_2$},
then the partition of a node \ls{j} is the same in \ls{h$_1$}
and \ls{h$_2$}. This also implies that \ls{find_compress} retains the
canonical structure of the union-find forest.

\begin{lstlisting}
val find_compress_refinement #n uf i h j
  : Lemma (let (r$_1$, h$_1$), (r$_2$, h$_2$) = 
    reify (find uf i) h, reify (find_compress uf i) h in
    r$_1$ == r$_2$ $\wedge$ fst (reify (find uf j) h$_1$) == fst (reify (find uf j) h$_2$))
\end{lstlisting}

\section{Related work}
\label{sec:related}

\ch{If it becomes public in time also cite: Modular Product
  Programs. Marco Eilers, Peter Müller, and Samuel Hitz. Submitted to ESOP'17.}

% \ch{Some non-conflicted POPL PC people who worked on relational verification:
%      Andrew Myers (IFC),\cite{SabelfeldM03,SabelfeldMyers03DL,SabelfeldMyers06IFC},
     % Lars Birkedal (some relational models, not very related but we cite
     %   4 of his papers in the ``Semantic techniques'' subsection),
     % James Cheney (provenance, no longer mentioned, should add/cite? whatever);
     % Koen Claessen (a tiny bit of IFC, cite? NO);
% {\bf Isil Dillig} \cite{SousaD16};
     % Sorin Lerner (tiny bit of dynamic IFC\cite{ChughMJL09} a while back, cite?)
% {\bf Aleksandar Nanevski} (RHTT\cite{NanevskiBG13});
%      Nadia Polikarpova (recent IFC work with Jean Yang\cite{PolikarpovaYIS16}, cite? DONE);
% {\bf David Sands} (IFC, differential privacy)\cite{DarvasHS05,SabelfeldS09};
     % Zhong Shao (a bit of IFC recently, cite\cite{CostanzoSG16})
     % Éric Tanter (a bit of IFC recently \cite{CruzRST17});
     % Zachary Tatlock (distributed systems verification\cite{WilcoxST17},
     %   not relational, right?);
% {\bf Tachio Terauchi} \cite{YasuokaT14, TerauchiA05, AntonopoulosGHKTW17};
% {\bf Jean Yang} (IFC, cite DONE);
% {\bf Danfeng Zhang} (IFC, differential privacy)\cite{AmtoftDZABHOC12}
%   should also cite \cite{ZhangK17};
% }

%% Relational verification is a large research area, here we only discuss
%% the works that we found to be closest related. Many of the techniques we
%% discuss are specialized to relational reasoning and sometimes even to
%% a specific application domain (\EG IFC, cryptography, differential
%% privacy, program equivalence, etc).

Much of the prior related work focused on checking specific relational
properties of programs, or general relational properties using
special-purpose logics.
%
In contrast, we argue that proof assistants that support reasoning
about pure and effectful programs can, using our methodology, model
and verify relational properties in a generic way.
%
The specific incarnation of our methodology in \fstar exploits its
efficient implementation of effects enabled by abstraction and
controlled reification; a unary weakest precondition calculus as a
base for relational proofs; SMT-based automation; and the convenience
of writing effectful code in direct style with returns, binds, and
lifts automatically inserted.

% \url{https://scholar.google.fr/scholar?q=relational+verification}

\mypara{Static IFC tools}
%%%%%%%%%%%%%%%%%%%%%%%%%
%
\citet{SabelfeldM03} survey a number of IFC type systems and
static analyses for showing noninterference, trading completeness for
automation.
%
More recent verification techniques for IFC aim for better
completeness \cite{BeringerH07, NanevskiBG13, AmtoftB04, AmtoftDZABHOC12,
BanerjeeNN16, SchebenS11, BartheFGSSB14, Rabe16}, while compromising
automation.
%
The two approaches can be combined, as discussed in
in \autoref{sec:ifc-combining}.

\mypara{Relational program logics and type systems}
%%%%%%%%%%%%%%%%%%%%%%%%%%%%%%%%%%
A variety of program logics for reasoning about general relational
properties have been proposed previously \citep{benton04relational,
Yang07, BartheGB09, AguirreBGGS17}, while others apply general
relational logics to specific domains, including access
control \cite{NanevskiBG13}, cryptography \cite{BartheGB12,
BartheGB09, BartheDGKSS13, PetcherM15}, differential
privacy \cite{BartheKOB13, ZhangK17}, mechanism
design \cite{BartheGAHRS15}, cost analysis \cite{CicekBG0H17}, program
approximations \cite{CarbinKMR12}.
%% %
%% Several of these logics are embedded in proof assistants for interactive
%% verification and some of them also provide SMT-based automation based
%% on {\em relational} weakest precondition calculi.

R\fstar, is worth pointing out for its
connection to \fstar. \citepos{BartheFGSSB14} extend a prior, value-dependent
version of \fstar \cite{fstar-pldi13} with a probabilistic semantics
and a type system that combines pRHL with refinement types. Like many
other relational Hoare logics, R\fstar provided an incomplete set of
rules aimed at capturing many relational properties by intrinsic typing only.

In this paper we instead provide a versatile generic method for relational
verification based on modeling effectful computations using monads and
proving relational properties on their monadic representations, making
the most of the support for full dependent types
and SMT-based automation in the latest version of \fstar.
%
This generic method can both be used directly to verify programs or as
a base for encoding specialized relational program logics.

\ch{Can we encode R\fstar?}

% A main difference with these is that they do intrinsic reasoning
% about multiple program runs, while we can freely mix intrinsic and
% extrinsic reasoning. The advantage
% of extrinsic reasoning is that we have full freedom in structuring
% our proofs and are not restricted by the structure of the program.

% Compare to probabilistic relational program logics based tools targeted at
% crypto like CertiCrypt \cite{BartheGB09}, EasyCrypt \cite{BartheDGKSS13}, and
% the Foundational Cryptography Framework (FCF) \cite{PetcherM15}
% \ch{I guess the reason CertiCrypt embedded a little language and
%   defined a relational logic for it is because the kind of crypto
%   things they are doing (probabilities, poly time) are much easier in
%   a little imperative language compared to the whole of Coq.  The FCF
%   work should be a bit closer to what we do, since they use a shallow
%   embedding of a little monadic language. They give a denotational
%   semantics to the little language, but then build up abstractions
%   like a pRHL on top of that. So in this respect very similar to our
%   embedded IFC example.}

\mypara{Product program constructions}
%%%%%%%%%%%%%%%%%%%%%%%%%%%%%%%%%%%%%%
%
Product program constructions and self-composition are techniques
aimed at reducing the verification of k-safety properties
\cite{clarkson10hyp} to the
verification of traditional (unary) safety proprieties of a product
program that emulates the behavior of multiple input programs.
%
Multiple such constructions have been proposed \cite{BartheCK16} targeted for
instance at secure IFC \cite{TerauchiA05, BartheDR11, Naumann06, YasuokaT14},
program equivalence for compiler validation \cite{ZaksP08},
equivalence checking and computing semantic differences
\cite{LahiriHKR12}, program approximation \cite{HeLR16}.
%
\citepos{SousaD16} recent Descartes tool for k-safety properties also
creates k copies of the program, but uses lockstep reasoning to
improve performance by more tightly coupling the key invariants across
the program copies.
%
Recently \citet{AntonopoulosGHKTW17} propose a tool \iffull called Blazer \fi that
obtains better scalability by using a new decomposition of
programs instead of using self-composition for k-safety problems.
\ch{So how do we relate to all this? What we do is quite
  different. Our approaches at this like compose2 didn't work so great
  (see plan.org). We have better support for interaction? Better
  expressiveness? Higher-order programs?
  Nik discussed with the authors of \cite{SousaD16} before ICFP?}
\ch{Should still figure this out since Terauchi and Dillig
  are likely to review this}
\ch{They can't do loop reversal, quite syntactic}

% \ch{another SymDiff paper \cite{HawblitzelKLR11}, relevant or just
%   obsolete given \cite{LahiriHKR12}?  Nik says it's a source of
%   examples to try out}

% verifying noninterference and other k-safety
% hyperproperties \cite{clarkson10hyp} (\IE relational properties
% about multiple runs of the same program).

\mypara{Other program equivalence techniques}
%%%%%%%%%%%%%%%%%%%%%%%%%%%%%%%%%%%%%%%%%%%%%
Beyond the ones already mentioned above, many other techniques targeted at program
equivalence have been proposed; we briefly review several recent works:
%
\citet{BentonKBH09} do manual proofs of correctness of
compiler optimizations using partial equivalence relations.
%
\citet{KunduTL09} do automatic translation validation of compiler
optimizations by checking equivalence of partially specified programs
that can represent multiple concrete programs.
%
\citet{GodlinS10} propose proof rules for proving the
equivalence of recursive procedures.
%
\citet{LucanuR15} and \citet{CiobacaLRR16} generalize this to
a set of co-inductive equivalence proof rules that are language-independent.
%
Automatically checking the equivalence of processes in a process
calculus is an important building block for security protocol
analysis \cite{BlanchetAF08, ChadhaCCK16}.

\ch{Again, how do we relate to all this?}

% Equivalence tools for security protocols, e.g. biprocesses in
% ProVerif. -- based on strong syntactic restriction + message
% multiplicity abstraction, automatic, can do concurrency;
% work for process calculi (but programs can be encoded),

\mypara{Semantic techniques}
%%%%%%%%%%%%%%%%%%%%%%%%%%%%
Many semantic techniques have been proposed for reasoning about
relational properties such as observational equivalence, including
techniques based on binary logical relations \cite{BentonKBH09,
  Mitchell86, AhmedDR09, DreyerNRB10, DreyerAB11, DreyerNB12,
  BentonHN13, Benton0N14}, bisimulations \cite{KoutavasW06,
  SangiorgiKS11, Sumii09} and combinations thereof \cite{HurDNV12, HurNDBV14}.
%
While these very powerful techniques are often not directly automated,
they can be used to provide semantic correctness proofs for relational
program logics \cite{DreyerNRB10, DreyerAB11} and other verification
tools \cite{BentonK0N16}.

% These are really paper proof techniques, no tool support beyond
% sometimes doing this in Coq, right? often targeted at very fancy
% features, like abstraction, which is generally hard to internalize in
% the language itself; need to make it clear that what we do is much
% more mundane, but also much more practically oriented \ch{One reason
%   I'm trying to stress ``verification'' as opposed to just
%   ``reasoning'' is that these guys do reasoning but no verification}

% Denotational semantics for program equivalence might fit in the same bucket.

%%\mypara{Proof assistants}
%%%%%%%%%%%%%%%%%%%%%%%%%
%% We are of course not the first to propose the use of monads in proof assistants,
%% and the idea of using a proof assistant as a powerful unified
%% verification framework for all kinds of properties should not surprise
%% anyone.
%
%% Still we believe the instantiation of these ideas in \fstar{} has some
%% merits over other uses of monads in a proof assistant: the efficient
%% implementation of effects enabled by abstraction and controlled
%% reification; the use of a unary weakest precondition calculus as a
%% base for relational proofs; the powerful
%% SMT-based automation; and the convenience of writing effectful code in
%% direct style with returns, binds, and lifts automatically inserted.

% Should also relate to work using monads in
% Even for proving the monad laws one is already doing relational
% verification. Could cite some representative examples like Greg's
% RockSalt work \cite{MorrisettTTTG12}.\ch{Look for works that are
%   closer to the kind of applications we are targetting. For instance
%   IFC in proof assistants (already looking at crypto above)}

% These guys never call it ``relational verification'' because they are
% already using their favourite proof assistant as an all powerful
% unified verification framework. Some things that might set our work
% apart from other uses of monads in a proof assistant work (1) the
% efficient implementation of effects enabled by abstraction and
% reification; (2) the way we do proofs by relying heavily on SMT-based
% automation; (3) the extra convenience of effectful programming in
% \fstar{}, \EG writing code in direct syntax with automatic lifts.

\iflater
IFC in Isabelle:
% \citet{BeringerH07} relate type-based approaches to ordinary
% (non-relational) program logics for proof certification, -- added
IFC for seL4 \cite{MurrayMBGBSLGK13},
Heiko Mantel's work on automata
(\href{I-MAKS}{https://fg-fomsess.gi.de/fileadmin/Jahrestreffen\_2016/Tasch.pdf})
and concurrency \cite{GreweLMS14, GreweLMS14a, GreweMS14},
% formalizing concurrent and probabilistic noninterference
% \cite{PopescuHN12, PopescuHN13},
% Andrei Popescu's CoCon \cite{KanavL014}
%   and CoSMed \cite{BauereissG0R16} verified systems. -- added
\fi

\iflater

Cost-analysis: How do monads and comonads differ?
Ezgi Cicek, Marco Gaboardi, and Deepak Garg
https://people.mpi-sws.org/~ecicek/DICE16.pdf

A Relational Logic for Higher-Order Programs
Alejandro Aguirre, Gilles Barthe, Marco Gaboardi, Deepak Garg,
Pierre-Yves Strub.  ICFP 2017. \cite{AguirreBGGS17}

They point to this paper for the cost monad:
[13] Shin-ya Katsumata. Parametric effect monads
and semantics of effect systems. In Proc. POPL,
pages 633–646, 2014.
https://dl.acm.org/citation.cfm?id=2535846
\fi

\iffull
\section{Future work}
\label{sec:discussion}

% \km{What about scalability (see below about tactics \& decision procedure) ? and
%   should we say a word about normalization of reified code ?}
% %
% \ch{Don't think we have enough data to talk about scalability other
%   than for future work}

% We now review some limitations of the current tool, sometimes already available
% in other frameworks (lean, Coq), that we wish to work on in some near future.

While we found \fstar to be a versatile tool for relational
verification of effectful programs, we also contemplated about
features that would make it even better suited.

\mypara{Tactics} \fstar's current combination of SMT
solving and dependent typechecking \iffull with higher-order unification and
normalization \fi provides good automation, but the ongoing addition of tactics
will provide more control and the possibility of
user-defined decision procedures.
%
In particular, when using shallow embeddings (like we do in
\autoref{sec:transformations}) tactics will allow us to write
meta-programs that automatically apply derived proof rules
based on the structure of the \fstar{} program we want to verify.

% \ch{WARNING: this is an anonymous submission, we can't
% leak the fact that we're also developing \fstar{}. These are features
% we think would be nice to have.}
% \km{...first time helping with an anonymous submission but you shouldn't
%   claim that you are a developer ? isn't that totally obvious ?}

% %\mypara{Partial correctness}
% \ch{Re-explain the main limitation of this
%   methodology: this only works for code that is proved terminating
%   (so we can only do total correctness, not partial correctness)
%   Any chance for removing this limitation in the future? Any chance we
%   can support partial-correctness reasoning about
%   not-always-terminating code? (e.g. an interpreter for a Turing
%   complete language) Any chance we can support verification
%   of reactive non-terminating code? (e.g. a server)}
% \km{For non-termination the early tentative using partial results in memoization
%   seems rather promising, in particular after reading McBride's Turing
%   completeness totally free, it seems possible to use this approach with
%   bove-cappreta predicates to speak about not-always-terminating code}
% \ch{This seems much more interesting to me that extrinsic termination
%   proofs, which to me seems rather unrelated to what we do here. Isn't it?}
% \km{instead of doing a section on partial correctness I've added a motivated
%   example at the end of next paragraph and renamed the paragraph itself }

\mypara{Extrinsic termination reasoning}
%
Aside from their use in relational reasoning, extrinsic proofs of
reified terms allow programmers to defer proof obligations, rather
than insisting on proofs at the time of
definition (while anticipating all uses).
%
While convenient, extrinsic proofs in \fstar only apply to programs
that are intrinsically proved terminating.
%
Building on our use of \citepos{McBride15} approach
in \S\ref{sec:memo}, we aim to define divergence as a reifiable
effect, placing it on par with other effects in \fstar.
%
We could then reason about the partial correctness of a program
declared in this effect or to prove its termination after its
definition.
%
Going back to the \emph{while} interpreter from
\autoref{sec:ifc-while}, we could forget about the decreasing metric and use
either \citepos{BoveC05} termination witnesses or step-indexing
as in \autoref{sec:benton2004-rhl} \cite{OwensMKT16,AminR17},
proving, for example, noninterference of reachable states
of an interactive non-terminating program.

\iflater
\km{There are also nice ideas in papers by Andreas Abel and others about
  productive coprogramming and verification of such code but It may feel
as totally unrelated to anything F* currently has}
\fi

\mypara{Observational purity}
%%%%%%%%%%%%%%%%%%%%%%%%%%%%%
%
Another desirable feature would be to hide the effect of a term if it
is proven observationally pure, e.g., in \autoref{sec:memo} this would
provide the ability to replace the original pure code by its
equivalent memoized variant.
%
Since we are able to prove that the memoized code has the same
extensional behaviour as the pure code up to some private data that we
could abstract over, we would like to implement a mechanism to
encapsulate observationally pure code.
% , building on techniques already
% available in \fstar to abstract local state \cite{fstar-pldi13}.\ch{Confused
%   about reference here. Is that really the right one?}
%
We hope that this mechanism could also be applied to programs proven
terminating extrinsically.

% \km{Is observational purity also relevant for union-find?}\ch{No
%   clue, but no excuse for vague text. For now focusing on memoization.}

\km{Is there any valid reference to Danel's monotonicity?}
\ch{\cite{preorders}, not clear what the relevance is}

\ch{missing references on forgetting allocation effects: \cite{BentonHN13, Benton0N14}}

\ch{One more citation that could be relevant for the While language
  (although this one is based on coinduction):
  Operational semantics using the partiality monad.
  \url{http://www.cse.chalmers.se/~nad/publications/danielsson-semantics-partiality-monad.pdf}
}

%\mypara{More interesting effects} -- whatever, this is an F* future work!
%%%%%%%%%%%%%%%%%%%%%%%%%%%%%%
% too vague
% Generalizing effect definitions in \fstar{} to encompass a wider
% class of effects would also help relational reasoning.

% harmful:
% allow relational reasoning about programs with
% non-determinism and probabilities.\ch{But we now do probabilities in
%   section 5. This is unqualified claim is hurting us badly!}

% unrelated:
% Another interesting extension
% would be to encode more invariants in the effect type and weakest
% precondition calculus through indexed effects.
%
% One particularly interesting application
% would be to have separation logic-flavroed effects where the
% computation type carries frame information. This development may
% need a theory of effects over indexed types.
% %
% \ch{We're no longer so hopeful about this last part, right?}
% CH: also not so much about relational verification

%% \mypara{User-defined automation}
%% %%%%%%%%%%%%%%%%%%%%%%%%%%%%%%%%
%% Even though relational reasoning can be applied to several arbitrary
%% programs, we are often faced with programs having a common syntactic structure
%% that can be used to simplify the proof of relational properties.
%% %
%% \fstar does not derogate to this rule and we were faced in the presented
%% examples with conceptually simple if tedious proofs just following the syntactic
%% structure of the programs on which we want to prove a relational property.
%% %
%% In order to reduce this tediousness, meta-programming capacities, such
%% as tactics in Lean or Coq could help.
%% %
%% This may also allow to define decision procedures for specific relational logic.
%
% \ch{Tactics in \fstar{} (not a limitation of methodology, but a current
%   limitation of the \fstar{} implementation. the methodology could just say: use
%   tactics to simplify boring proofs.) ... as Jonathan noted, our proofs are very
%   often following the structure of the program. So it would be interesting to
%   use tactics, once we have them, to create such basic proof sketches
%   automatically!}
\fi

\section{Conclusion}

This paper advocates verifying relational properties of effectful
programs using generic tools that are not specific to relational
reasoning: monadic effects, reification, dependent types,
non-relational weakest preconditions, and SMT-based automation.
%
Our experiments in \fstar{} verifying relational properties about a
variety of examples show the wide applicability of this approach.
%
One of the strong points is the great flexibility in modelling effects
and expressing relational properties about code using these effects.
%
The other strong point is the good balance between interactive control,
SMT-based automation, and the ability to encode even more automated
specialized tools where needed.
%
Thanks to this, the effort required from the \fstar{} programmer for
relational verification seems
on par with non-relational reasoning in \fstar{} and with specialized
relational program logics.

% While in this paper we used \fstar{}, we hope that this approach to
% relational reasoning will also be applied with other proof assistants
% (\EG Coq, Lean, Agda, Idris, etc.), for which automation is
% likely to come in quite different styles.

%% NS: this is quite implausible for Dafny. It works as well as it
%%     does because it really hides the representation of state
%%     everywhere. It also bakes in many notions of monotonicity,
%%     which we know is incompatible with reification.

%%     and to other
%%     semi-automatic verification systems (\EG Dafny), which may use
%%     our technique to gain in expressiveness while retaining their
%%     SMT-based automation.


%% Acknowledgments
\begin{acks}                            %% acks environment is optional
                                        %% contents suppressed with 'anonymous'
  %% Commands \grantsponsor{<sponsorID>}{<name>}{<url>} and
  %% \grantnum[<url>]{<sponsorID>}{<number>} should be used to
  %% acknowledge financial support and will be used by metadata
  %% extraction tools.

The work of C\u{a}t\u{a}lin Hri\c{t}cu and Kenji Maillard is
in part supported by the
\grantsponsor{1}{European Research Council}{https://erc.europa.eu/}
under ERC Starting Grant SECOMP (\grantnum{1}{715753}).
\end{acks}

% \ifcamera
% \else
%  \section*{Appendix}
%  \appendix
%  \chapter{Supplementary Material}
\label{appendix}

In this appendix, we present supplementary material for the techniques and
experiments presented in the main text.

\section{Baseline Results and Analysis for Informed Sampler}
\label{appendix:chap3}

Here, we give an in-depth
performance analysis of the various samplers and the effect of their
hyperparameters. We choose hyperparameters with the lowest PSRF value
after $10k$ iterations, for each sampler individually. If the
differences between PSRF are not significantly different among
multiple values, we choose the one that has the highest acceptance
rate.

\subsection{Experiment: Estimating Camera Extrinsics}
\label{appendix:chap3:room}

\subsubsection{Parameter Selection}
\paragraph{Metropolis Hastings (\MH)}

Figure~\ref{fig:exp1_MH} shows the median acceptance rates and PSRF
values corresponding to various proposal standard deviations of plain
\MH~sampling. Mixing gets better and the acceptance rate gets worse as
the standard deviation increases. The value $0.3$ is selected standard
deviation for this sampler.

\paragraph{Metropolis Hastings Within Gibbs (\MHWG)}

As mentioned in Section~\ref{sec:room}, the \MHWG~sampler with one-dimensional
updates did not converge for any value of proposal standard deviation.
This problem has high correlation of the camera parameters and is of
multi-modal nature, which this sampler has problems with.

\paragraph{Parallel Tempering (\PT)}

For \PT~sampling, we took the best performing \MH~sampler and used
different temperature chains to improve the mixing of the
sampler. Figure~\ref{fig:exp1_PT} shows the results corresponding to
different combination of temperature levels. The sampler with
temperature levels of $[1,3,27]$ performed best in terms of both
mixing and acceptance rate.

\paragraph{Effect of Mixture Coefficient in Informed Sampling (\MIXLMH)}

Figure~\ref{fig:exp1_alpha} shows the effect of mixture
coefficient ($\alpha$) on the informed sampling
\MIXLMH. Since there is no significant different in PSRF values for
$0 \le \alpha \le 0.7$, we chose $0.7$ due to its high acceptance
rate.


% \end{multicols}

\begin{figure}[h]
\centering
  \subfigure[MH]{%
    \includegraphics[width=.48\textwidth]{figures/supplementary/camPose_MH.pdf} \label{fig:exp1_MH}
  }
  \subfigure[PT]{%
    \includegraphics[width=.48\textwidth]{figures/supplementary/camPose_PT.pdf} \label{fig:exp1_PT}
  }
\\
  \subfigure[INF-MH]{%
    \includegraphics[width=.48\textwidth]{figures/supplementary/camPose_alpha.pdf} \label{fig:exp1_alpha}
  }
  \mycaption{Results of the `Estimating Camera Extrinsics' experiment}{PRSFs and Acceptance rates corresponding to (a) various standard deviations of \MH, (b) various temperature level combinations of \PT sampling and (c) various mixture coefficients of \MIXLMH sampling.}
\end{figure}



\begin{figure}[!t]
\centering
  \subfigure[\MH]{%
    \includegraphics[width=.48\textwidth]{figures/supplementary/occlusionExp_MH.pdf} \label{fig:exp2_MH}
  }
  \subfigure[\BMHWG]{%
    \includegraphics[width=.48\textwidth]{figures/supplementary/occlusionExp_BMHWG.pdf} \label{fig:exp2_BMHWG}
  }
\\
  \subfigure[\MHWG]{%
    \includegraphics[width=.48\textwidth]{figures/supplementary/occlusionExp_MHWG.pdf} \label{fig:exp2_MHWG}
  }
  \subfigure[\PT]{%
    \includegraphics[width=.48\textwidth]{figures/supplementary/occlusionExp_PT.pdf} \label{fig:exp2_PT}
  }
\\
  \subfigure[\INFBMHWG]{%
    \includegraphics[width=.5\textwidth]{figures/supplementary/occlusionExp_alpha.pdf} \label{fig:exp2_alpha}
  }
  \mycaption{Results of the `Occluding Tiles' experiment}{PRSF and
    Acceptance rates corresponding to various standard deviations of
    (a) \MH, (b) \BMHWG, (c) \MHWG, (d) various temperature level
    combinations of \PT~sampling and; (e) various mixture coefficients
    of our informed \INFBMHWG sampling.}
\end{figure}

%\onecolumn\newpage\twocolumn
\subsection{Experiment: Occluding Tiles}
\label{appendix:chap3:tiles}

\subsubsection{Parameter Selection}

\paragraph{Metropolis Hastings (\MH)}

Figure~\ref{fig:exp2_MH} shows the results of
\MH~sampling. Results show the poor convergence for all proposal
standard deviations and rapid decrease of AR with increasing standard
deviation. This is due to the high-dimensional nature of
the problem. We selected a standard deviation of $1.1$.

\paragraph{Blocked Metropolis Hastings Within Gibbs (\BMHWG)}

The results of \BMHWG are shown in Figure~\ref{fig:exp2_BMHWG}. In
this sampler we update only one block of tile variables (of dimension
four) in each sampling step. Results show much better performance
compared to plain \MH. The optimal proposal standard deviation for
this sampler is $0.7$.

\paragraph{Metropolis Hastings Within Gibbs (\MHWG)}

Figure~\ref{fig:exp2_MHWG} shows the result of \MHWG sampling. This
sampler is better than \BMHWG and converges much more quickly. Here
a standard deviation of $0.9$ is found to be best.

\paragraph{Parallel Tempering (\PT)}

Figure~\ref{fig:exp2_PT} shows the results of \PT sampling with various
temperature combinations. Results show no improvement in AR from plain
\MH sampling and again $[1,3,27]$ temperature levels are found to be optimal.

\paragraph{Effect of Mixture Coefficient in Informed Sampling (\INFBMHWG)}

Figure~\ref{fig:exp2_alpha} shows the effect of mixture
coefficient ($\alpha$) on the blocked informed sampling
\INFBMHWG. Since there is no significant different in PSRF values for
$0 \le \alpha \le 0.8$, we chose $0.8$ due to its high acceptance
rate.



\subsection{Experiment: Estimating Body Shape}
\label{appendix:chap3:body}

\subsubsection{Parameter Selection}
\paragraph{Metropolis Hastings (\MH)}

Figure~\ref{fig:exp3_MH} shows the result of \MH~sampling with various
proposal standard deviations. The value of $0.1$ is found to be
best.

\paragraph{Metropolis Hastings Within Gibbs (\MHWG)}

For \MHWG sampling we select $0.3$ proposal standard
deviation. Results are shown in Fig.~\ref{fig:exp3_MHWG}.


\paragraph{Parallel Tempering (\PT)}

As before, results in Fig.~\ref{fig:exp3_PT}, the temperature levels
were selected to be $[1,3,27]$ due its slightly higher AR.

\paragraph{Effect of Mixture Coefficient in Informed Sampling (\MIXLMH)}

Figure~\ref{fig:exp3_alpha} shows the effect of $\alpha$ on PSRF and
AR. Since there is no significant differences in PSRF values for $0 \le
\alpha \le 0.8$, we choose $0.8$.


\begin{figure}[t]
\centering
  \subfigure[\MH]{%
    \includegraphics[width=.48\textwidth]{figures/supplementary/bodyShape_MH.pdf} \label{fig:exp3_MH}
  }
  \subfigure[\MHWG]{%
    \includegraphics[width=.48\textwidth]{figures/supplementary/bodyShape_MHWG.pdf} \label{fig:exp3_MHWG}
  }
\\
  \subfigure[\PT]{%
    \includegraphics[width=.48\textwidth]{figures/supplementary/bodyShape_PT.pdf} \label{fig:exp3_PT}
  }
  \subfigure[\MIXLMH]{%
    \includegraphics[width=.48\textwidth]{figures/supplementary/bodyShape_alpha.pdf} \label{fig:exp3_alpha}
  }
\\
  \mycaption{Results of the `Body Shape Estimation' experiment}{PRSFs and
    Acceptance rates corresponding to various standard deviations of
    (a) \MH, (b) \MHWG; (c) various temperature level combinations
    of \PT sampling and; (d) various mixture coefficients of the
    informed \MIXLMH sampling.}
\end{figure}


\subsection{Results Overview}
Figure~\ref{fig:exp_summary} shows the summary results of the all the three
experimental studies related to informed sampler.
\begin{figure*}[h!]
\centering
  \subfigure[Results for: Estimating Camera Extrinsics]{%
    \includegraphics[width=0.9\textwidth]{figures/supplementary/camPose_ALL.pdf} \label{fig:exp1_all}
  }
  \subfigure[Results for: Occluding Tiles]{%
    \includegraphics[width=0.9\textwidth]{figures/supplementary/occlusionExp_ALL.pdf} \label{fig:exp2_all}
  }
  \subfigure[Results for: Estimating Body Shape]{%
    \includegraphics[width=0.9\textwidth]{figures/supplementary/bodyShape_ALL.pdf} \label{fig:exp3_all}
  }
  \label{fig:exp_summary}
  \mycaption{Summary of the statistics for the three experiments}{Shown are
    for several baseline methods and the informed samplers the
    acceptance rates (left), PSRFs (middle), and RMSE values
    (right). All results are median results over multiple test
    examples.}
\end{figure*}

\subsection{Additional Qualitative Results}

\subsubsection{Occluding Tiles}
In Figure~\ref{fig:exp2_visual_more} more qualitative results of the
occluding tiles experiment are shown. The informed sampling approach
(\INFBMHWG) is better than the best baseline (\MHWG). This still is a
very challenging problem since the parameters for occluded tiles are
flat over a large region. Some of the posterior variance of the
occluded tiles is already captured by the informed sampler.

\begin{figure*}[h!]
\begin{center}
\centerline{\includegraphics[width=0.95\textwidth]{figures/supplementary/occlusionExp_Visual.pdf}}
\mycaption{Additional qualitative results of the occluding tiles experiment}
  {From left to right: (a)
  Given image, (b) Ground truth tiles, (c) OpenCV heuristic and most probable estimates
  from 5000 samples obtained by (d) MHWG sampler (best baseline) and
  (e) our INF-BMHWG sampler. (f) Posterior expectation of the tiles
  boundaries obtained by INF-BMHWG sampling (First 2000 samples are
  discarded as burn-in).}
\label{fig:exp2_visual_more}
\end{center}
\end{figure*}

\subsubsection{Body Shape}
Figure~\ref{fig:exp3_bodyMeshes} shows some more results of 3D mesh
reconstruction using posterior samples obtained by our informed
sampling \MIXLMH.

\begin{figure*}[t]
\begin{center}
\centerline{\includegraphics[width=0.75\textwidth]{figures/supplementary/bodyMeshResults.pdf}}
\mycaption{Qualitative results for the body shape experiment}
  {Shown is the 3D mesh reconstruction results with first 1000 samples obtained
  using the \MIXLMH informed sampling method. (blue indicates small
  values and red indicates high values)}
\label{fig:exp3_bodyMeshes}
\end{center}
\end{figure*}

\clearpage



\section{Additional Results on the Face Problem with CMP}

Figure~\ref{fig:shading-qualitative-multiple-subjects-supp} shows inference results for reflectance maps, normal maps and lights for randomly chosen test images, and Fig.~\ref{fig:shading-qualitative-same-subject-supp} shows reflectance estimation results on multiple images of the same subject produced under different illumination conditions. CMP is able to produce estimates that are closer to the groundtruth across different subjects and illumination conditions.

\begin{figure*}[h]
  \begin{center}
  \centerline{\includegraphics[width=1.0\columnwidth]{figures/face_cmp_visual_results_supp.pdf}}
  \vspace{-1.2cm}
  \end{center}
	\mycaption{A visual comparison of inference results}{(a)~Observed images. (b)~Inferred reflectance maps. \textit{GT} is the photometric stereo groundtruth, \textit{BU} is the Biswas \etal (2009) reflectance estimate and \textit{Forest} is the consensus prediction. (c)~The variance of the inferred reflectance estimate produced by \MTD (normalized across rows).(d)~Visualization of inferred light directions. (e)~Inferred normal maps.}
	\label{fig:shading-qualitative-multiple-subjects-supp}
\end{figure*}


\begin{figure*}[h]
	\centering
	\setlength\fboxsep{0.2mm}
	\setlength\fboxrule{0pt}
	\begin{tikzpicture}

		\matrix at (0, 0) [matrix of nodes, nodes={anchor=east}, column sep=-0.05cm, row sep=-0.2cm]
		{
			\fbox{\includegraphics[width=1cm]{figures/sample_3_4_X.png}} &
			\fbox{\includegraphics[width=1cm]{figures/sample_3_4_GT.png}} &
			\fbox{\includegraphics[width=1cm]{figures/sample_3_4_BISWAS.png}}  &
			\fbox{\includegraphics[width=1cm]{figures/sample_3_4_VMP.png}}  &
			\fbox{\includegraphics[width=1cm]{figures/sample_3_4_FOREST.png}}  &
			\fbox{\includegraphics[width=1cm]{figures/sample_3_4_CMP.png}}  &
			\fbox{\includegraphics[width=1cm]{figures/sample_3_4_CMPVAR.png}}
			 \\

			\fbox{\includegraphics[width=1cm]{figures/sample_3_5_X.png}} &
			\fbox{\includegraphics[width=1cm]{figures/sample_3_5_GT.png}} &
			\fbox{\includegraphics[width=1cm]{figures/sample_3_5_BISWAS.png}}  &
			\fbox{\includegraphics[width=1cm]{figures/sample_3_5_VMP.png}}  &
			\fbox{\includegraphics[width=1cm]{figures/sample_3_5_FOREST.png}}  &
			\fbox{\includegraphics[width=1cm]{figures/sample_3_5_CMP.png}}  &
			\fbox{\includegraphics[width=1cm]{figures/sample_3_5_CMPVAR.png}}
			 \\

			\fbox{\includegraphics[width=1cm]{figures/sample_3_6_X.png}} &
			\fbox{\includegraphics[width=1cm]{figures/sample_3_6_GT.png}} &
			\fbox{\includegraphics[width=1cm]{figures/sample_3_6_BISWAS.png}}  &
			\fbox{\includegraphics[width=1cm]{figures/sample_3_6_VMP.png}}  &
			\fbox{\includegraphics[width=1cm]{figures/sample_3_6_FOREST.png}}  &
			\fbox{\includegraphics[width=1cm]{figures/sample_3_6_CMP.png}}  &
			\fbox{\includegraphics[width=1cm]{figures/sample_3_6_CMPVAR.png}}
			 \\
	     };

       \node at (-3.85, -2.0) {\small Observed};
       \node at (-2.55, -2.0) {\small `GT'};
       \node at (-1.27, -2.0) {\small BU};
       \node at (0.0, -2.0) {\small MP};
       \node at (1.27, -2.0) {\small Forest};
       \node at (2.55, -2.0) {\small \textbf{CMP}};
       \node at (3.85, -2.0) {\small Variance};

	\end{tikzpicture}
	\mycaption{Robustness to varying illumination}{Reflectance estimation on a subject images with varying illumination. Left to right: observed image, photometric stereo estimate (GT)
  which is used as a proxy for groundtruth, bottom-up estimate of \cite{Biswas2009}, VMP result, consensus forest estimate, CMP mean, and CMP variance.}
	\label{fig:shading-qualitative-same-subject-supp}
\end{figure*}

\clearpage

\section{Additional Material for Learning Sparse High Dimensional Filters}
\label{sec:appendix-bnn}

This part of supplementary material contains a more detailed overview of the permutohedral
lattice convolution in Section~\ref{sec:permconv}, more experiments in
Section~\ref{sec:addexps} and additional results with protocols for
the experiments presented in Chapter~\ref{chap:bnn} in Section~\ref{sec:addresults}.

\vspace{-0.2cm}
\subsection{General Permutohedral Convolutions}
\label{sec:permconv}

A core technical contribution of this work is the generalization of the Gaussian permutohedral lattice
convolution proposed in~\cite{adams2010fast} to the full non-separable case with the
ability to perform back-propagation. Although, conceptually, there are minor
differences between Gaussian and general parameterized filters, there are non-trivial practical
differences in terms of the algorithmic implementation. The Gauss filters belong to
the separable class and can thus be decomposed into multiple
sequential one dimensional convolutions. We are interested in the general filter
convolutions, which can not be decomposed. Thus, performing a general permutohedral
convolution at a lattice point requires the computation of the inner product with the
neighboring elements in all the directions in the high-dimensional space.

Here, we give more details of the implementation differences of separable
and non-separable filters. In the following, we will explain the scalar case first.
Recall, that the forward pass of general permutohedral convolution
involves 3 steps: \textit{splatting}, \textit{convolving} and \textit{slicing}.
We follow the same splatting and slicing strategies as in~\cite{adams2010fast}
since these operations do not depend on the filter kernel. The main difference
between our work and the existing implementation of~\cite{adams2010fast} is
the way that the convolution operation is executed. This proceeds by constructing
a \emph{blur neighbor} matrix $K$ that stores for every lattice point all
values of the lattice neighbors that are needed to compute the filter output.

\begin{figure}[t!]
  \centering
    \includegraphics[width=0.6\columnwidth]{figures/supplementary/lattice_construction}
  \mycaption{Illustration of 1D permutohedral lattice construction}
  {A $4\times 4$ $(x,y)$ grid lattice is projected onto the plane defined by the normal
  vector $(1,1)^{\top}$. This grid has $s+1=4$ and $d=2$ $(s+1)^{d}=4^2=16$ elements.
  In the projection, all points of the same color are projected onto the same points in the plane.
  The number of elements of the projected lattice is $t=(s+1)^d-s^d=4^2-3^2=7$, that is
  the $(4\times 4)$ grid minus the size of lattice that is $1$ smaller at each size, in this
  case a $(3\times 3)$ lattice (the upper right $(3\times 3)$ elements).
  }
\label{fig:latticeconstruction}
\end{figure}

The blur neighbor matrix is constructed by traversing through all the populated
lattice points and their neighboring elements.
% For efficiency, we do this matrix construction recursively with shared computations
% since $n^{th}$ neighbourhood elements are $1^{st}$ neighborhood elements of $n-1^{th}$ neighbourhood elements. \pg{do not understand}
This is done recursively to share computations. For any lattice point, the neighbors that are
$n$ hops away are the direct neighbors of the points that are $n-1$ hops away.
The size of a $d$ dimensional spatial filter with width $s+1$ is $(s+1)^{d}$ (\eg, a
$3\times 3$ filter, $s=2$ in $d=2$ has $3^2=9$ elements) and this size grows
exponentially in the number of dimensions $d$. The permutohedral lattice is constructed by
projecting a regular grid onto the plane spanned by the $d$ dimensional normal vector ${(1,\ldots,1)}^{\top}$. See
Fig.~\ref{fig:latticeconstruction} for an illustration of the 1D lattice construction.
Many corners of a grid filter are projected onto the same point, in total $t = {(s+1)}^{d} -
s^{d}$ elements remain in the permutohedral filter with $s$ neighborhood in $d-1$ dimensions.
If the lattice has $m$ populated elements, the
matrix $K$ has size $t\times m$. Note that, since the input signal is typically
sparse, only a few lattice corners are being populated in the \textit{slicing} step.
We use a hash-table to keep track of these points and traverse only through
the populated lattice points for this neighborhood matrix construction.

Once the blur neighbor matrix $K$ is constructed, we can perform the convolution
by the matrix vector multiplication
\begin{equation}
\ell' = BK,
\label{eq:conv}
\end{equation}
where $B$ is the $1 \times t$ filter kernel (whose values we will learn) and $\ell'\in\mathbb{R}^{1\times m}$
is the result of the filtering at the $m$ lattice points. In practice, we found that the
matrix $K$ is sometimes too large to fit into GPU memory and we divided the matrix $K$
into smaller pieces to compute Eq.~\ref{eq:conv} sequentially.

In the general multi-dimensional case, the signal $\ell$ is of $c$ dimensions. Then
the kernel $B$ is of size $c \times t$ and $K$ stores the $c$ dimensional vectors
accordingly. When the input and output points are different, we slice only the
input points and splat only at the output points.


\subsection{Additional Experiments}
\label{sec:addexps}
In this section, we discuss more use-cases for the learned bilateral filters, one
use-case of BNNs and two single filter applications for image and 3D mesh denoising.

\subsubsection{Recognition of subsampled MNIST}\label{sec:app_mnist}

One of the strengths of the proposed filter convolution is that it does not
require the input to lie on a regular grid. The only requirement is to define a distance
between features of the input signal.
We highlight this feature with the following experiment using the
classical MNIST ten class classification problem~\cite{lecun1998mnist}. We sample a
sparse set of $N$ points $(x,y)\in [0,1]\times [0,1]$
uniformly at random in the input image, use their interpolated values
as signal and the \emph{continuous} $(x,y)$ positions as features. This mimics
sub-sampling of a high-dimensional signal. To compare against a spatial convolution,
we interpolate the sparse set of values at the grid positions.

We take a reference implementation of LeNet~\cite{lecun1998gradient} that
is part of the Caffe project~\cite{jia2014caffe} and compare it
against the same architecture but replacing the first convolutional
layer with a bilateral convolution layer (BCL). The filter size
and numbers are adjusted to get a comparable number of parameters
($5\times 5$ for LeNet, $2$-neighborhood for BCL).

The results are shown in Table~\ref{tab:all-results}. We see that training
on the original MNIST data (column Original, LeNet vs. BNN) leads to a slight
decrease in performance of the BNN (99.03\%) compared to LeNet
(99.19\%). The BNN can be trained and evaluated on sparse
signals, and we resample the image as described above for $N=$ 100\%, 60\% and
20\% of the total number of pixels. The methods are also evaluated
on test images that are subsampled in the same way. Note that we can
train and test with different subsampling rates. We introduce an additional
bilinear interpolation layer for the LeNet architecture to train on the same
data. In essence, both models perform a spatial interpolation and thus we
expect them to yield a similar classification accuracy. Once the data is of
higher dimensions, the permutohedral convolution will be faster due to hashing
the sparse input points, as well as less memory demanding in comparison to
naive application of a spatial convolution with interpolated values.

\begin{table}[t]
  \begin{center}
    \footnotesize
    \centering
    \begin{tabular}[t]{lllll}
      \toprule
              &     & \multicolumn{3}{c}{Test Subsampling} \\
       Method  & Original & 100\% & 60\% & 20\%\\
      \midrule
       LeNet &  \textbf{0.9919} & 0.9660 & 0.9348 & \textbf{0.6434} \\
       BNN &  0.9903 & \textbf{0.9844} & \textbf{0.9534} & 0.5767 \\
      \hline
       LeNet 100\% & 0.9856 & 0.9809 & 0.9678 & \textbf{0.7386} \\
       BNN 100\% & \textbf{0.9900} & \textbf{0.9863} & \textbf{0.9699} & 0.6910 \\
      \hline
       LeNet 60\% & 0.9848 & 0.9821 & 0.9740 & 0.8151 \\
       BNN 60\% & \textbf{0.9885} & \textbf{0.9864} & \textbf{0.9771} & \textbf{0.8214}\\
      \hline
       LeNet 20\% & \textbf{0.9763} & \textbf{0.9754} & 0.9695 & 0.8928 \\
       BNN 20\% & 0.9728 & 0.9735 & \textbf{0.9701} & \textbf{0.9042}\\
      \bottomrule
    \end{tabular}
  \end{center}
\vspace{-.2cm}
\caption{Classification accuracy on MNIST. We compare the
    LeNet~\cite{lecun1998gradient} implementation that is part of
    Caffe~\cite{jia2014caffe} to the network with the first layer
    replaced by a bilateral convolution layer (BCL). Both are trained
    on the original image resolution (first two rows). Three more BNN
    and CNN models are trained with randomly subsampled images (100\%,
    60\% and 20\% of the pixels). An additional bilinear interpolation
    layer samples the input signal on a spatial grid for the CNN model.
  }
  \label{tab:all-results}
\vspace{-.5cm}
\end{table}

\subsubsection{Image Denoising}

The main application that inspired the development of the bilateral
filtering operation is image denoising~\cite{aurich1995non}, there
using a single Gaussian kernel. Our development allows to learn this
kernel function from data and we explore how to improve using a \emph{single}
but more general bilateral filter.

We use the Berkeley segmentation dataset
(BSDS500)~\cite{arbelaezi2011bsds500} as a test bed. The color
images in the dataset are converted to gray-scale,
and corrupted with Gaussian noise with a standard deviation of
$\frac {25} {255}$.

We compare the performance of four different filter models on a
denoising task.
The first baseline model (`Spatial' in Table \ref{tab:denoising}, $25$
weights) uses a single spatial filter with a kernel size of
$5$ and predicts the scalar gray-scale value at the center pixel. The next model
(`Gauss Bilateral') applies a bilateral \emph{Gaussian}
filter to the noisy input, using position and intensity features $\f=(x,y,v)^\top$.
The third setup (`Learned Bilateral', $65$ weights)
takes a Gauss kernel as initialization and
fits all filter weights on the train set to minimize the
mean squared error with respect to the clean images.
We run a combination
of spatial and permutohedral convolutions on spatial and bilateral
features (`Spatial + Bilateral (Learned)') to check for a complementary
performance of the two convolutions.

\label{sec:exp:denoising}
\begin{table}[!h]
\begin{center}
  \footnotesize
  \begin{tabular}[t]{lr}
    \toprule
    Method & PSNR \\
    \midrule
    Noisy Input & $20.17$ \\
    Spatial & $26.27$ \\
    Gauss Bilateral & $26.51$ \\
    Learned Bilateral & $26.58$ \\
    Spatial + Bilateral (Learned) & \textbf{$26.65$} \\
    \bottomrule
  \end{tabular}
\end{center}
\vspace{-0.5em}
\caption{PSNR results of a denoising task using the BSDS500
  dataset~\cite{arbelaezi2011bsds500}}
\vspace{-0.5em}
\label{tab:denoising}
\end{table}
\vspace{-0.2em}

The PSNR scores evaluated on full images of the test set are
shown in Table \ref{tab:denoising}. We find that an untrained bilateral
filter already performs better than a trained spatial convolution
($26.27$ to $26.51$). A learned convolution further improve the
performance slightly. We chose this simple one-kernel setup to
validate an advantage of the generalized bilateral filter. A competitive
denoising system would employ RGB color information and also
needs to be properly adjusted in network size. Multi-layer perceptrons
have obtained state-of-the-art denoising results~\cite{burger12cvpr}
and the permutohedral lattice layer can readily be used in such an
architecture, which is intended future work.

\subsection{Additional results}
\label{sec:addresults}

This section contains more qualitative results for the experiments presented in Chapter~\ref{chap:bnn}.

\begin{figure*}[th!]
  \centering
    \includegraphics[width=\columnwidth,trim={5cm 2.5cm 5cm 4.5cm},clip]{figures/supplementary/lattice_viz.pdf}
    \vspace{-0.7cm}
  \mycaption{Visualization of the Permutohedral Lattice}
  {Sample lattice visualizations for different feature spaces. All pixels falling in the same simplex cell are shown with
  the same color. $(x,y)$ features correspond to image pixel positions, and $(r,g,b) \in [0,255]$ correspond
  to the red, green and blue color values.}
\label{fig:latticeviz}
\end{figure*}

\subsubsection{Lattice Visualization}

Figure~\ref{fig:latticeviz} shows sample lattice visualizations for different feature spaces.

\newcolumntype{L}[1]{>{\raggedright\let\newline\\\arraybackslash\hspace{0pt}}b{#1}}
\newcolumntype{C}[1]{>{\centering\let\newline\\\arraybackslash\hspace{0pt}}b{#1}}
\newcolumntype{R}[1]{>{\raggedleft\let\newline\\\arraybackslash\hspace{0pt}}b{#1}}

\subsubsection{Color Upsampling}\label{sec:color_upsampling}
\label{sec:col_upsample_extra}

Some images of the upsampling for the Pascal
VOC12 dataset are shown in Fig.~\ref{fig:Colour_upsample_visuals}. It is
especially the low level image details that are better preserved with
a learned bilateral filter compared to the Gaussian case.

\begin{figure*}[t!]
  \centering
    \subfigure{%
   \raisebox{2.0em}{
    \includegraphics[width=.06\columnwidth]{figures/supplementary/2007_004969.jpg}
   }
  }
  \subfigure{%
    \includegraphics[width=.17\columnwidth]{figures/supplementary/2007_004969_gray.pdf}
  }
  \subfigure{%
    \includegraphics[width=.17\columnwidth]{figures/supplementary/2007_004969_gt.pdf}
  }
  \subfigure{%
    \includegraphics[width=.17\columnwidth]{figures/supplementary/2007_004969_bicubic.pdf}
  }
  \subfigure{%
    \includegraphics[width=.17\columnwidth]{figures/supplementary/2007_004969_gauss.pdf}
  }
  \subfigure{%
    \includegraphics[width=.17\columnwidth]{figures/supplementary/2007_004969_learnt.pdf}
  }\\
    \subfigure{%
   \raisebox{2.0em}{
    \includegraphics[width=.06\columnwidth]{figures/supplementary/2007_003106.jpg}
   }
  }
  \subfigure{%
    \includegraphics[width=.17\columnwidth]{figures/supplementary/2007_003106_gray.pdf}
  }
  \subfigure{%
    \includegraphics[width=.17\columnwidth]{figures/supplementary/2007_003106_gt.pdf}
  }
  \subfigure{%
    \includegraphics[width=.17\columnwidth]{figures/supplementary/2007_003106_bicubic.pdf}
  }
  \subfigure{%
    \includegraphics[width=.17\columnwidth]{figures/supplementary/2007_003106_gauss.pdf}
  }
  \subfigure{%
    \includegraphics[width=.17\columnwidth]{figures/supplementary/2007_003106_learnt.pdf}
  }\\
  \setcounter{subfigure}{0}
  \small{
  \subfigure[Inp.]{%
  \raisebox{2.0em}{
    \includegraphics[width=.06\columnwidth]{figures/supplementary/2007_006837.jpg}
   }
  }
  \subfigure[Guidance]{%
    \includegraphics[width=.17\columnwidth]{figures/supplementary/2007_006837_gray.pdf}
  }
   \subfigure[GT]{%
    \includegraphics[width=.17\columnwidth]{figures/supplementary/2007_006837_gt.pdf}
  }
  \subfigure[Bicubic]{%
    \includegraphics[width=.17\columnwidth]{figures/supplementary/2007_006837_bicubic.pdf}
  }
  \subfigure[Gauss-BF]{%
    \includegraphics[width=.17\columnwidth]{figures/supplementary/2007_006837_gauss.pdf}
  }
  \subfigure[Learned-BF]{%
    \includegraphics[width=.17\columnwidth]{figures/supplementary/2007_006837_learnt.pdf}
  }
  }
  \vspace{-0.5cm}
  \mycaption{Color Upsampling}{Color $8\times$ upsampling results
  using different methods, from left to right, (a)~Low-resolution input color image (Inp.),
  (b)~Gray scale guidance image, (c)~Ground-truth color image; Upsampled color images with
  (d)~Bicubic interpolation, (e) Gauss bilateral upsampling and, (f)~Learned bilateral
  updampgling (best viewed on screen).}

\label{fig:Colour_upsample_visuals}
\end{figure*}

\subsubsection{Depth Upsampling}
\label{sec:depth_upsample_extra}

Figure~\ref{fig:depth_upsample_visuals} presents some more qualitative results comparing bicubic interpolation, Gauss
bilateral and learned bilateral upsampling on NYU depth dataset image~\cite{silberman2012indoor}.

\subsubsection{Character Recognition}\label{sec:app_character}

 Figure~\ref{fig:nnrecognition} shows the schematic of different layers
 of the network architecture for LeNet-7~\cite{lecun1998mnist}
 and DeepCNet(5, 50)~\cite{ciresan2012multi,graham2014spatially}. For the BNN variants, the first layer filters are replaced
 with learned bilateral filters and are learned end-to-end.

\subsubsection{Semantic Segmentation}\label{sec:app_semantic_segmentation}
\label{sec:semantic_bnn_extra}

Some more visual results for semantic segmentation are shown in Figure~\ref{fig:semantic_visuals}.
These include the underlying DeepLab CNN\cite{chen2014semantic} result (DeepLab),
the 2 step mean-field result with Gaussian edge potentials (+2stepMF-GaussCRF)
and also corresponding results with learned edge potentials (+2stepMF-LearnedCRF).
In general, we observe that mean-field in learned CRF leads to slightly dilated
classification regions in comparison to using Gaussian CRF thereby filling-in the
false negative pixels and also correcting some mis-classified regions.

\begin{figure*}[t!]
  \centering
    \subfigure{%
   \raisebox{2.0em}{
    \includegraphics[width=.06\columnwidth]{figures/supplementary/2bicubic}
   }
  }
  \subfigure{%
    \includegraphics[width=.17\columnwidth]{figures/supplementary/2given_image}
  }
  \subfigure{%
    \includegraphics[width=.17\columnwidth]{figures/supplementary/2ground_truth}
  }
  \subfigure{%
    \includegraphics[width=.17\columnwidth]{figures/supplementary/2bicubic}
  }
  \subfigure{%
    \includegraphics[width=.17\columnwidth]{figures/supplementary/2gauss}
  }
  \subfigure{%
    \includegraphics[width=.17\columnwidth]{figures/supplementary/2learnt}
  }\\
    \subfigure{%
   \raisebox{2.0em}{
    \includegraphics[width=.06\columnwidth]{figures/supplementary/32bicubic}
   }
  }
  \subfigure{%
    \includegraphics[width=.17\columnwidth]{figures/supplementary/32given_image}
  }
  \subfigure{%
    \includegraphics[width=.17\columnwidth]{figures/supplementary/32ground_truth}
  }
  \subfigure{%
    \includegraphics[width=.17\columnwidth]{figures/supplementary/32bicubic}
  }
  \subfigure{%
    \includegraphics[width=.17\columnwidth]{figures/supplementary/32gauss}
  }
  \subfigure{%
    \includegraphics[width=.17\columnwidth]{figures/supplementary/32learnt}
  }\\
  \setcounter{subfigure}{0}
  \small{
  \subfigure[Inp.]{%
  \raisebox{2.0em}{
    \includegraphics[width=.06\columnwidth]{figures/supplementary/41bicubic}
   }
  }
  \subfigure[Guidance]{%
    \includegraphics[width=.17\columnwidth]{figures/supplementary/41given_image}
  }
   \subfigure[GT]{%
    \includegraphics[width=.17\columnwidth]{figures/supplementary/41ground_truth}
  }
  \subfigure[Bicubic]{%
    \includegraphics[width=.17\columnwidth]{figures/supplementary/41bicubic}
  }
  \subfigure[Gauss-BF]{%
    \includegraphics[width=.17\columnwidth]{figures/supplementary/41gauss}
  }
  \subfigure[Learned-BF]{%
    \includegraphics[width=.17\columnwidth]{figures/supplementary/41learnt}
  }
  }
  \mycaption{Depth Upsampling}{Depth $8\times$ upsampling results
  using different upsampling strategies, from left to right,
  (a)~Low-resolution input depth image (Inp.),
  (b)~High-resolution guidance image, (c)~Ground-truth depth; Upsampled depth images with
  (d)~Bicubic interpolation, (e) Gauss bilateral upsampling and, (f)~Learned bilateral
  updampgling (best viewed on screen).}

\label{fig:depth_upsample_visuals}
\end{figure*}

\subsubsection{Material Segmentation}\label{sec:app_material_segmentation}
\label{sec:material_bnn_extra}

In Fig.~\ref{fig:material_visuals-app2}, we present visual results comparing 2 step
mean-field inference with Gaussian and learned pairwise CRF potentials. In
general, we observe that the pixels belonging to dominant classes in the
training data are being more accurately classified with learned CRF. This leads to
a significant improvements in overall pixel accuracy. This also results
in a slight decrease of the accuracy from less frequent class pixels thereby
slightly reducing the average class accuracy with learning. We attribute this
to the type of annotation that is available for this dataset, which is not
for the entire image but for some segments in the image. We have very few
images of the infrequent classes to combat this behaviour during training.

\subsubsection{Experiment Protocols}
\label{sec:protocols}

Table~\ref{tbl:parameters} shows experiment protocols of different experiments.

 \begin{figure*}[t!]
  \centering
  \subfigure[LeNet-7]{
    \includegraphics[width=0.7\columnwidth]{figures/supplementary/lenet_cnn_network}
    }\\
    \subfigure[DeepCNet]{
    \includegraphics[width=\columnwidth]{figures/supplementary/deepcnet_cnn_network}
    }
  \mycaption{CNNs for Character Recognition}
  {Schematic of (top) LeNet-7~\cite{lecun1998mnist} and (bottom) DeepCNet(5,50)~\cite{ciresan2012multi,graham2014spatially} architectures used in Assamese
  character recognition experiments.}
\label{fig:nnrecognition}
\end{figure*}

\definecolor{voc_1}{RGB}{0, 0, 0}
\definecolor{voc_2}{RGB}{128, 0, 0}
\definecolor{voc_3}{RGB}{0, 128, 0}
\definecolor{voc_4}{RGB}{128, 128, 0}
\definecolor{voc_5}{RGB}{0, 0, 128}
\definecolor{voc_6}{RGB}{128, 0, 128}
\definecolor{voc_7}{RGB}{0, 128, 128}
\definecolor{voc_8}{RGB}{128, 128, 128}
\definecolor{voc_9}{RGB}{64, 0, 0}
\definecolor{voc_10}{RGB}{192, 0, 0}
\definecolor{voc_11}{RGB}{64, 128, 0}
\definecolor{voc_12}{RGB}{192, 128, 0}
\definecolor{voc_13}{RGB}{64, 0, 128}
\definecolor{voc_14}{RGB}{192, 0, 128}
\definecolor{voc_15}{RGB}{64, 128, 128}
\definecolor{voc_16}{RGB}{192, 128, 128}
\definecolor{voc_17}{RGB}{0, 64, 0}
\definecolor{voc_18}{RGB}{128, 64, 0}
\definecolor{voc_19}{RGB}{0, 192, 0}
\definecolor{voc_20}{RGB}{128, 192, 0}
\definecolor{voc_21}{RGB}{0, 64, 128}
\definecolor{voc_22}{RGB}{128, 64, 128}

\begin{figure*}[t]
  \centering
  \small{
  \fcolorbox{white}{voc_1}{\rule{0pt}{6pt}\rule{6pt}{0pt}} Background~~
  \fcolorbox{white}{voc_2}{\rule{0pt}{6pt}\rule{6pt}{0pt}} Aeroplane~~
  \fcolorbox{white}{voc_3}{\rule{0pt}{6pt}\rule{6pt}{0pt}} Bicycle~~
  \fcolorbox{white}{voc_4}{\rule{0pt}{6pt}\rule{6pt}{0pt}} Bird~~
  \fcolorbox{white}{voc_5}{\rule{0pt}{6pt}\rule{6pt}{0pt}} Boat~~
  \fcolorbox{white}{voc_6}{\rule{0pt}{6pt}\rule{6pt}{0pt}} Bottle~~
  \fcolorbox{white}{voc_7}{\rule{0pt}{6pt}\rule{6pt}{0pt}} Bus~~
  \fcolorbox{white}{voc_8}{\rule{0pt}{6pt}\rule{6pt}{0pt}} Car~~ \\
  \fcolorbox{white}{voc_9}{\rule{0pt}{6pt}\rule{6pt}{0pt}} Cat~~
  \fcolorbox{white}{voc_10}{\rule{0pt}{6pt}\rule{6pt}{0pt}} Chair~~
  \fcolorbox{white}{voc_11}{\rule{0pt}{6pt}\rule{6pt}{0pt}} Cow~~
  \fcolorbox{white}{voc_12}{\rule{0pt}{6pt}\rule{6pt}{0pt}} Dining Table~~
  \fcolorbox{white}{voc_13}{\rule{0pt}{6pt}\rule{6pt}{0pt}} Dog~~
  \fcolorbox{white}{voc_14}{\rule{0pt}{6pt}\rule{6pt}{0pt}} Horse~~
  \fcolorbox{white}{voc_15}{\rule{0pt}{6pt}\rule{6pt}{0pt}} Motorbike~~
  \fcolorbox{white}{voc_16}{\rule{0pt}{6pt}\rule{6pt}{0pt}} Person~~ \\
  \fcolorbox{white}{voc_17}{\rule{0pt}{6pt}\rule{6pt}{0pt}} Potted Plant~~
  \fcolorbox{white}{voc_18}{\rule{0pt}{6pt}\rule{6pt}{0pt}} Sheep~~
  \fcolorbox{white}{voc_19}{\rule{0pt}{6pt}\rule{6pt}{0pt}} Sofa~~
  \fcolorbox{white}{voc_20}{\rule{0pt}{6pt}\rule{6pt}{0pt}} Train~~
  \fcolorbox{white}{voc_21}{\rule{0pt}{6pt}\rule{6pt}{0pt}} TV monitor~~ \\
  }
  \subfigure{%
    \includegraphics[width=.18\columnwidth]{figures/supplementary/2007_001423_given.jpg}
  }
  \subfigure{%
    \includegraphics[width=.18\columnwidth]{figures/supplementary/2007_001423_gt.png}
  }
  \subfigure{%
    \includegraphics[width=.18\columnwidth]{figures/supplementary/2007_001423_cnn.png}
  }
  \subfigure{%
    \includegraphics[width=.18\columnwidth]{figures/supplementary/2007_001423_gauss.png}
  }
  \subfigure{%
    \includegraphics[width=.18\columnwidth]{figures/supplementary/2007_001423_learnt.png}
  }\\
  \subfigure{%
    \includegraphics[width=.18\columnwidth]{figures/supplementary/2007_001430_given.jpg}
  }
  \subfigure{%
    \includegraphics[width=.18\columnwidth]{figures/supplementary/2007_001430_gt.png}
  }
  \subfigure{%
    \includegraphics[width=.18\columnwidth]{figures/supplementary/2007_001430_cnn.png}
  }
  \subfigure{%
    \includegraphics[width=.18\columnwidth]{figures/supplementary/2007_001430_gauss.png}
  }
  \subfigure{%
    \includegraphics[width=.18\columnwidth]{figures/supplementary/2007_001430_learnt.png}
  }\\
    \subfigure{%
    \includegraphics[width=.18\columnwidth]{figures/supplementary/2007_007996_given.jpg}
  }
  \subfigure{%
    \includegraphics[width=.18\columnwidth]{figures/supplementary/2007_007996_gt.png}
  }
  \subfigure{%
    \includegraphics[width=.18\columnwidth]{figures/supplementary/2007_007996_cnn.png}
  }
  \subfigure{%
    \includegraphics[width=.18\columnwidth]{figures/supplementary/2007_007996_gauss.png}
  }
  \subfigure{%
    \includegraphics[width=.18\columnwidth]{figures/supplementary/2007_007996_learnt.png}
  }\\
   \subfigure{%
    \includegraphics[width=.18\columnwidth]{figures/supplementary/2010_002682_given.jpg}
  }
  \subfigure{%
    \includegraphics[width=.18\columnwidth]{figures/supplementary/2010_002682_gt.png}
  }
  \subfigure{%
    \includegraphics[width=.18\columnwidth]{figures/supplementary/2010_002682_cnn.png}
  }
  \subfigure{%
    \includegraphics[width=.18\columnwidth]{figures/supplementary/2010_002682_gauss.png}
  }
  \subfigure{%
    \includegraphics[width=.18\columnwidth]{figures/supplementary/2010_002682_learnt.png}
  }\\
     \subfigure{%
    \includegraphics[width=.18\columnwidth]{figures/supplementary/2010_004789_given.jpg}
  }
  \subfigure{%
    \includegraphics[width=.18\columnwidth]{figures/supplementary/2010_004789_gt.png}
  }
  \subfigure{%
    \includegraphics[width=.18\columnwidth]{figures/supplementary/2010_004789_cnn.png}
  }
  \subfigure{%
    \includegraphics[width=.18\columnwidth]{figures/supplementary/2010_004789_gauss.png}
  }
  \subfigure{%
    \includegraphics[width=.18\columnwidth]{figures/supplementary/2010_004789_learnt.png}
  }\\
       \subfigure{%
    \includegraphics[width=.18\columnwidth]{figures/supplementary/2007_001311_given.jpg}
  }
  \subfigure{%
    \includegraphics[width=.18\columnwidth]{figures/supplementary/2007_001311_gt.png}
  }
  \subfigure{%
    \includegraphics[width=.18\columnwidth]{figures/supplementary/2007_001311_cnn.png}
  }
  \subfigure{%
    \includegraphics[width=.18\columnwidth]{figures/supplementary/2007_001311_gauss.png}
  }
  \subfigure{%
    \includegraphics[width=.18\columnwidth]{figures/supplementary/2007_001311_learnt.png}
  }\\
  \setcounter{subfigure}{0}
  \subfigure[Input]{%
    \includegraphics[width=.18\columnwidth]{figures/supplementary/2010_003531_given.jpg}
  }
  \subfigure[Ground Truth]{%
    \includegraphics[width=.18\columnwidth]{figures/supplementary/2010_003531_gt.png}
  }
  \subfigure[DeepLab]{%
    \includegraphics[width=.18\columnwidth]{figures/supplementary/2010_003531_cnn.png}
  }
  \subfigure[+GaussCRF]{%
    \includegraphics[width=.18\columnwidth]{figures/supplementary/2010_003531_gauss.png}
  }
  \subfigure[+LearnedCRF]{%
    \includegraphics[width=.18\columnwidth]{figures/supplementary/2010_003531_learnt.png}
  }
  \vspace{-0.3cm}
  \mycaption{Semantic Segmentation}{Example results of semantic segmentation.
  (c)~depicts the unary results before application of MF, (d)~after two steps of MF with Gaussian edge CRF potentials, (e)~after
  two steps of MF with learned edge CRF potentials.}
    \label{fig:semantic_visuals}
\end{figure*}


\definecolor{minc_1}{HTML}{771111}
\definecolor{minc_2}{HTML}{CAC690}
\definecolor{minc_3}{HTML}{EEEEEE}
\definecolor{minc_4}{HTML}{7C8FA6}
\definecolor{minc_5}{HTML}{597D31}
\definecolor{minc_6}{HTML}{104410}
\definecolor{minc_7}{HTML}{BB819C}
\definecolor{minc_8}{HTML}{D0CE48}
\definecolor{minc_9}{HTML}{622745}
\definecolor{minc_10}{HTML}{666666}
\definecolor{minc_11}{HTML}{D54A31}
\definecolor{minc_12}{HTML}{101044}
\definecolor{minc_13}{HTML}{444126}
\definecolor{minc_14}{HTML}{75D646}
\definecolor{minc_15}{HTML}{DD4348}
\definecolor{minc_16}{HTML}{5C8577}
\definecolor{minc_17}{HTML}{C78472}
\definecolor{minc_18}{HTML}{75D6D0}
\definecolor{minc_19}{HTML}{5B4586}
\definecolor{minc_20}{HTML}{C04393}
\definecolor{minc_21}{HTML}{D69948}
\definecolor{minc_22}{HTML}{7370D8}
\definecolor{minc_23}{HTML}{7A3622}
\definecolor{minc_24}{HTML}{000000}

\begin{figure*}[t]
  \centering
  \small{
  \fcolorbox{white}{minc_1}{\rule{0pt}{6pt}\rule{6pt}{0pt}} Brick~~
  \fcolorbox{white}{minc_2}{\rule{0pt}{6pt}\rule{6pt}{0pt}} Carpet~~
  \fcolorbox{white}{minc_3}{\rule{0pt}{6pt}\rule{6pt}{0pt}} Ceramic~~
  \fcolorbox{white}{minc_4}{\rule{0pt}{6pt}\rule{6pt}{0pt}} Fabric~~
  \fcolorbox{white}{minc_5}{\rule{0pt}{6pt}\rule{6pt}{0pt}} Foliage~~
  \fcolorbox{white}{minc_6}{\rule{0pt}{6pt}\rule{6pt}{0pt}} Food~~
  \fcolorbox{white}{minc_7}{\rule{0pt}{6pt}\rule{6pt}{0pt}} Glass~~
  \fcolorbox{white}{minc_8}{\rule{0pt}{6pt}\rule{6pt}{0pt}} Hair~~ \\
  \fcolorbox{white}{minc_9}{\rule{0pt}{6pt}\rule{6pt}{0pt}} Leather~~
  \fcolorbox{white}{minc_10}{\rule{0pt}{6pt}\rule{6pt}{0pt}} Metal~~
  \fcolorbox{white}{minc_11}{\rule{0pt}{6pt}\rule{6pt}{0pt}} Mirror~~
  \fcolorbox{white}{minc_12}{\rule{0pt}{6pt}\rule{6pt}{0pt}} Other~~
  \fcolorbox{white}{minc_13}{\rule{0pt}{6pt}\rule{6pt}{0pt}} Painted~~
  \fcolorbox{white}{minc_14}{\rule{0pt}{6pt}\rule{6pt}{0pt}} Paper~~
  \fcolorbox{white}{minc_15}{\rule{0pt}{6pt}\rule{6pt}{0pt}} Plastic~~\\
  \fcolorbox{white}{minc_16}{\rule{0pt}{6pt}\rule{6pt}{0pt}} Polished Stone~~
  \fcolorbox{white}{minc_17}{\rule{0pt}{6pt}\rule{6pt}{0pt}} Skin~~
  \fcolorbox{white}{minc_18}{\rule{0pt}{6pt}\rule{6pt}{0pt}} Sky~~
  \fcolorbox{white}{minc_19}{\rule{0pt}{6pt}\rule{6pt}{0pt}} Stone~~
  \fcolorbox{white}{minc_20}{\rule{0pt}{6pt}\rule{6pt}{0pt}} Tile~~
  \fcolorbox{white}{minc_21}{\rule{0pt}{6pt}\rule{6pt}{0pt}} Wallpaper~~
  \fcolorbox{white}{minc_22}{\rule{0pt}{6pt}\rule{6pt}{0pt}} Water~~
  \fcolorbox{white}{minc_23}{\rule{0pt}{6pt}\rule{6pt}{0pt}} Wood~~ \\
  }
  \subfigure{%
    \includegraphics[width=.18\columnwidth]{figures/supplementary/000010868_given.jpg}
  }
  \subfigure{%
    \includegraphics[width=.18\columnwidth]{figures/supplementary/000010868_gt.png}
  }
  \subfigure{%
    \includegraphics[width=.18\columnwidth]{figures/supplementary/000010868_cnn.png}
  }
  \subfigure{%
    \includegraphics[width=.18\columnwidth]{figures/supplementary/000010868_gauss.png}
  }
  \subfigure{%
    \includegraphics[width=.18\columnwidth]{figures/supplementary/000010868_learnt.png}
  }\\[-2ex]
  \subfigure{%
    \includegraphics[width=.18\columnwidth]{figures/supplementary/000006011_given.jpg}
  }
  \subfigure{%
    \includegraphics[width=.18\columnwidth]{figures/supplementary/000006011_gt.png}
  }
  \subfigure{%
    \includegraphics[width=.18\columnwidth]{figures/supplementary/000006011_cnn.png}
  }
  \subfigure{%
    \includegraphics[width=.18\columnwidth]{figures/supplementary/000006011_gauss.png}
  }
  \subfigure{%
    \includegraphics[width=.18\columnwidth]{figures/supplementary/000006011_learnt.png}
  }\\[-2ex]
    \subfigure{%
    \includegraphics[width=.18\columnwidth]{figures/supplementary/000008553_given.jpg}
  }
  \subfigure{%
    \includegraphics[width=.18\columnwidth]{figures/supplementary/000008553_gt.png}
  }
  \subfigure{%
    \includegraphics[width=.18\columnwidth]{figures/supplementary/000008553_cnn.png}
  }
  \subfigure{%
    \includegraphics[width=.18\columnwidth]{figures/supplementary/000008553_gauss.png}
  }
  \subfigure{%
    \includegraphics[width=.18\columnwidth]{figures/supplementary/000008553_learnt.png}
  }\\[-2ex]
   \subfigure{%
    \includegraphics[width=.18\columnwidth]{figures/supplementary/000009188_given.jpg}
  }
  \subfigure{%
    \includegraphics[width=.18\columnwidth]{figures/supplementary/000009188_gt.png}
  }
  \subfigure{%
    \includegraphics[width=.18\columnwidth]{figures/supplementary/000009188_cnn.png}
  }
  \subfigure{%
    \includegraphics[width=.18\columnwidth]{figures/supplementary/000009188_gauss.png}
  }
  \subfigure{%
    \includegraphics[width=.18\columnwidth]{figures/supplementary/000009188_learnt.png}
  }\\[-2ex]
  \setcounter{subfigure}{0}
  \subfigure[Input]{%
    \includegraphics[width=.18\columnwidth]{figures/supplementary/000023570_given.jpg}
  }
  \subfigure[Ground Truth]{%
    \includegraphics[width=.18\columnwidth]{figures/supplementary/000023570_gt.png}
  }
  \subfigure[DeepLab]{%
    \includegraphics[width=.18\columnwidth]{figures/supplementary/000023570_cnn.png}
  }
  \subfigure[+GaussCRF]{%
    \includegraphics[width=.18\columnwidth]{figures/supplementary/000023570_gauss.png}
  }
  \subfigure[+LearnedCRF]{%
    \includegraphics[width=.18\columnwidth]{figures/supplementary/000023570_learnt.png}
  }
  \mycaption{Material Segmentation}{Example results of material segmentation.
  (c)~depicts the unary results before application of MF, (d)~after two steps of MF with Gaussian edge CRF potentials, (e)~after two steps of MF with learned edge CRF potentials.}
    \label{fig:material_visuals-app2}
\end{figure*}


\begin{table*}[h]
\tiny
  \centering
    \begin{tabular}{L{2.3cm} L{2.25cm} C{1.5cm} C{0.7cm} C{0.6cm} C{0.7cm} C{0.7cm} C{0.7cm} C{1.6cm} C{0.6cm} C{0.6cm} C{0.6cm}}
      \toprule
& & & & & \multicolumn{3}{c}{\textbf{Data Statistics}} & \multicolumn{4}{c}{\textbf{Training Protocol}} \\

\textbf{Experiment} & \textbf{Feature Types} & \textbf{Feature Scales} & \textbf{Filter Size} & \textbf{Filter Nbr.} & \textbf{Train}  & \textbf{Val.} & \textbf{Test} & \textbf{Loss Type} & \textbf{LR} & \textbf{Batch} & \textbf{Epochs} \\
      \midrule
      \multicolumn{2}{c}{\textbf{Single Bilateral Filter Applications}} & & & & & & & & & \\
      \textbf{2$\times$ Color Upsampling} & Position$_{1}$, Intensity (3D) & 0.13, 0.17 & 65 & 2 & 10581 & 1449 & 1456 & MSE & 1e-06 & 200 & 94.5\\
      \textbf{4$\times$ Color Upsampling} & Position$_{1}$, Intensity (3D) & 0.06, 0.17 & 65 & 2 & 10581 & 1449 & 1456 & MSE & 1e-06 & 200 & 94.5\\
      \textbf{8$\times$ Color Upsampling} & Position$_{1}$, Intensity (3D) & 0.03, 0.17 & 65 & 2 & 10581 & 1449 & 1456 & MSE & 1e-06 & 200 & 94.5\\
      \textbf{16$\times$ Color Upsampling} & Position$_{1}$, Intensity (3D) & 0.02, 0.17 & 65 & 2 & 10581 & 1449 & 1456 & MSE & 1e-06 & 200 & 94.5\\
      \textbf{Depth Upsampling} & Position$_{1}$, Color (5D) & 0.05, 0.02 & 665 & 2 & 795 & 100 & 654 & MSE & 1e-07 & 50 & 251.6\\
      \textbf{Mesh Denoising} & Isomap (4D) & 46.00 & 63 & 2 & 1000 & 200 & 500 & MSE & 100 & 10 & 100.0 \\
      \midrule
      \multicolumn{2}{c}{\textbf{DenseCRF Applications}} & & & & & & & & &\\
      \multicolumn{2}{l}{\textbf{Semantic Segmentation}} & & & & & & & & &\\
      \textbf{- 1step MF} & Position$_{1}$, Color (5D); Position$_{1}$ (2D) & 0.01, 0.34; 0.34  & 665; 19  & 2; 2 & 10581 & 1449 & 1456 & Logistic & 0.1 & 5 & 1.4 \\
      \textbf{- 2step MF} & Position$_{1}$, Color (5D); Position$_{1}$ (2D) & 0.01, 0.34; 0.34 & 665; 19 & 2; 2 & 10581 & 1449 & 1456 & Logistic & 0.1 & 5 & 1.4 \\
      \textbf{- \textit{loose} 2step MF} & Position$_{1}$, Color (5D); Position$_{1}$ (2D) & 0.01, 0.34; 0.34 & 665; 19 & 2; 2 &10581 & 1449 & 1456 & Logistic & 0.1 & 5 & +1.9  \\ \\
      \multicolumn{2}{l}{\textbf{Material Segmentation}} & & & & & & & & &\\
      \textbf{- 1step MF} & Position$_{2}$, Lab-Color (5D) & 5.00, 0.05, 0.30  & 665 & 2 & 928 & 150 & 1798 & Weighted Logistic & 1e-04 & 24 & 2.6 \\
      \textbf{- 2step MF} & Position$_{2}$, Lab-Color (5D) & 5.00, 0.05, 0.30 & 665 & 2 & 928 & 150 & 1798 & Weighted Logistic & 1e-04 & 12 & +0.7 \\
      \textbf{- \textit{loose} 2step MF} & Position$_{2}$, Lab-Color (5D) & 5.00, 0.05, 0.30 & 665 & 2 & 928 & 150 & 1798 & Weighted Logistic & 1e-04 & 12 & +0.2\\
      \midrule
      \multicolumn{2}{c}{\textbf{Neural Network Applications}} & & & & & & & & &\\
      \textbf{Tiles: CNN-9$\times$9} & - & - & 81 & 4 & 10000 & 1000 & 1000 & Logistic & 0.01 & 100 & 500.0 \\
      \textbf{Tiles: CNN-13$\times$13} & - & - & 169 & 6 & 10000 & 1000 & 1000 & Logistic & 0.01 & 100 & 500.0 \\
      \textbf{Tiles: CNN-17$\times$17} & - & - & 289 & 8 & 10000 & 1000 & 1000 & Logistic & 0.01 & 100 & 500.0 \\
      \textbf{Tiles: CNN-21$\times$21} & - & - & 441 & 10 & 10000 & 1000 & 1000 & Logistic & 0.01 & 100 & 500.0 \\
      \textbf{Tiles: BNN} & Position$_{1}$, Color (5D) & 0.05, 0.04 & 63 & 1 & 10000 & 1000 & 1000 & Logistic & 0.01 & 100 & 30.0 \\
      \textbf{LeNet} & - & - & 25 & 2 & 5490 & 1098 & 1647 & Logistic & 0.1 & 100 & 182.2 \\
      \textbf{Crop-LeNet} & - & - & 25 & 2 & 5490 & 1098 & 1647 & Logistic & 0.1 & 100 & 182.2 \\
      \textbf{BNN-LeNet} & Position$_{2}$ (2D) & 20.00 & 7 & 1 & 5490 & 1098 & 1647 & Logistic & 0.1 & 100 & 182.2 \\
      \textbf{DeepCNet} & - & - & 9 & 1 & 5490 & 1098 & 1647 & Logistic & 0.1 & 100 & 182.2 \\
      \textbf{Crop-DeepCNet} & - & - & 9 & 1 & 5490 & 1098 & 1647 & Logistic & 0.1 & 100 & 182.2 \\
      \textbf{BNN-DeepCNet} & Position$_{2}$ (2D) & 40.00  & 7 & 1 & 5490 & 1098 & 1647 & Logistic & 0.1 & 100 & 182.2 \\
      \bottomrule
      \\
    \end{tabular}
    \mycaption{Experiment Protocols} {Experiment protocols for the different experiments presented in this work. \textbf{Feature Types}:
    Feature spaces used for the bilateral convolutions. Position$_1$ corresponds to un-normalized pixel positions whereas Position$_2$ corresponds
    to pixel positions normalized to $[0,1]$ with respect to the given image. \textbf{Feature Scales}: Cross-validated scales for the features used.
     \textbf{Filter Size}: Number of elements in the filter that is being learned. \textbf{Filter Nbr.}: Half-width of the filter. \textbf{Train},
     \textbf{Val.} and \textbf{Test} corresponds to the number of train, validation and test images used in the experiment. \textbf{Loss Type}: Type
     of loss used for back-propagation. ``MSE'' corresponds to Euclidean mean squared error loss and ``Logistic'' corresponds to multinomial logistic
     loss. ``Weighted Logistic'' is the class-weighted multinomial logistic loss. We weighted the loss with inverse class probability for material
     segmentation task due to the small availability of training data with class imbalance. \textbf{LR}: Fixed learning rate used in stochastic gradient
     descent. \textbf{Batch}: Number of images used in one parameter update step. \textbf{Epochs}: Number of training epochs. In all the experiments,
     we used fixed momentum of 0.9 and weight decay of 0.0005 for stochastic gradient descent. ```Color Upsampling'' experiments in this Table corresponds
     to those performed on Pascal VOC12 dataset images. For all experiments using Pascal VOC12 images, we use extended
     training segmentation dataset available from~\cite{hariharan2011moredata}, and used standard validation and test splits
     from the main dataset~\cite{voc2012segmentation}.}
  \label{tbl:parameters}
\end{table*}

\clearpage

\section{Parameters and Additional Results for Video Propagation Networks}

In this Section, we present experiment protocols and additional qualitative results for experiments
on video object segmentation, semantic video segmentation and video color
propagation. Table~\ref{tbl:parameters_supp} shows the feature scales and other parameters used in different experiments.
Figures~\ref{fig:video_seg_pos_supp} show some qualitative results on video object segmentation
with some failure cases in Fig.~\ref{fig:video_seg_neg_supp}.
Figure~\ref{fig:semantic_visuals_supp} shows some qualitative results on semantic video segmentation and
Fig.~\ref{fig:color_visuals_supp} shows results on video color propagation.

\newcolumntype{L}[1]{>{\raggedright\let\newline\\\arraybackslash\hspace{0pt}}b{#1}}
\newcolumntype{C}[1]{>{\centering\let\newline\\\arraybackslash\hspace{0pt}}b{#1}}
\newcolumntype{R}[1]{>{\raggedleft\let\newline\\\arraybackslash\hspace{0pt}}b{#1}}

\begin{table*}[h]
\tiny
  \centering
    \begin{tabular}{L{3.0cm} L{2.4cm} L{2.8cm} L{2.8cm} C{0.5cm} C{1.0cm} L{1.2cm}}
      \toprule
\textbf{Experiment} & \textbf{Feature Type} & \textbf{Feature Scale-1, $\Lambda_a$} & \textbf{Feature Scale-2, $\Lambda_b$} & \textbf{$\alpha$} & \textbf{Input Frames} & \textbf{Loss Type} \\
      \midrule
      \textbf{Video Object Segmentation} & ($x,y,Y,Cb,Cr,t$) & (0.02,0.02,0.07,0.4,0.4,0.01) & (0.03,0.03,0.09,0.5,0.5,0.2) & 0.5 & 9 & Logistic\\
      \midrule
      \textbf{Semantic Video Segmentation} & & & & & \\
      \textbf{with CNN1~\cite{yu2015multi}-NoFlow} & ($x,y,R,G,B,t$) & (0.08,0.08,0.2,0.2,0.2,0.04) & (0.11,0.11,0.2,0.2,0.2,0.04) & 0.5 & 3 & Logistic \\
      \textbf{with CNN1~\cite{yu2015multi}-Flow} & ($x+u_x,y+u_y,R,G,B,t$) & (0.11,0.11,0.14,0.14,0.14,0.03) & (0.08,0.08,0.12,0.12,0.12,0.01) & 0.65 & 3 & Logistic\\
      \textbf{with CNN2~\cite{richter2016playing}-Flow} & ($x+u_x,y+u_y,R,G,B,t$) & (0.08,0.08,0.2,0.2,0.2,0.04) & (0.09,0.09,0.25,0.25,0.25,0.03) & 0.5 & 4 & Logistic\\
      \midrule
      \textbf{Video Color Propagation} & ($x,y,I,t$)  & (0.04,0.04,0.2,0.04) & No second kernel & 1 & 4 & MSE\\
      \bottomrule
      \\
    \end{tabular}
    \mycaption{Experiment Protocols} {Experiment protocols for the different experiments presented in this work. \textbf{Feature Types}:
    Feature spaces used for the bilateral convolutions, with position ($x,y$) and color
    ($R,G,B$ or $Y,Cb,Cr$) features $\in [0,255]$. $u_x$, $u_y$ denotes optical flow with respect
    to the present frame and $I$ denotes grayscale intensity.
    \textbf{Feature Scales ($\Lambda_a, \Lambda_b$)}: Cross-validated scales for the features used.
    \textbf{$\alpha$}: Exponential time decay for the input frames.
    \textbf{Input Frames}: Number of input frames for VPN.
    \textbf{Loss Type}: Type
     of loss used for back-propagation. ``MSE'' corresponds to Euclidean mean squared error loss and ``Logistic'' corresponds to multinomial logistic loss.}
  \label{tbl:parameters_supp}
\end{table*}

% \begin{figure}[th!]
% \begin{center}
%   \centerline{\includegraphics[width=\textwidth]{figures/video_seg_visuals_supp_small.pdf}}
%     \mycaption{Video Object Segmentation}
%     {Shown are the different frames in example videos with the corresponding
%     ground truth (GT) masks, predictions from BVS~\cite{marki2016bilateral},
%     OFL~\cite{tsaivideo}, VPN (VPN-Stage2) and VPN-DLab (VPN-DeepLab) models.}
%     \label{fig:video_seg_small_supp}
% \end{center}
% \vspace{-1.0cm}
% \end{figure}

\begin{figure}[th!]
\begin{center}
  \centerline{\includegraphics[width=0.7\textwidth]{figures/video_seg_visuals_supp_positive.pdf}}
    \mycaption{Video Object Segmentation}
    {Shown are the different frames in example videos with the corresponding
    ground truth (GT) masks, predictions from BVS~\cite{marki2016bilateral},
    OFL~\cite{tsaivideo}, VPN (VPN-Stage2) and VPN-DLab (VPN-DeepLab) models.}
    \label{fig:video_seg_pos_supp}
\end{center}
\vspace{-1.0cm}
\end{figure}

\begin{figure}[th!]
\begin{center}
  \centerline{\includegraphics[width=0.7\textwidth]{figures/video_seg_visuals_supp_negative.pdf}}
    \mycaption{Failure Cases for Video Object Segmentation}
    {Shown are the different frames in example videos with the corresponding
    ground truth (GT) masks, predictions from BVS~\cite{marki2016bilateral},
    OFL~\cite{tsaivideo}, VPN (VPN-Stage2) and VPN-DLab (VPN-DeepLab) models.}
    \label{fig:video_seg_neg_supp}
\end{center}
\vspace{-1.0cm}
\end{figure}

\begin{figure}[th!]
\begin{center}
  \centerline{\includegraphics[width=0.9\textwidth]{figures/supp_semantic_visual.pdf}}
    \mycaption{Semantic Video Segmentation}
    {Input video frames and the corresponding ground truth (GT)
    segmentation together with the predictions of CNN~\cite{yu2015multi} and with
    VPN-Flow.}
    \label{fig:semantic_visuals_supp}
\end{center}
\vspace{-0.7cm}
\end{figure}

\begin{figure}[th!]
\begin{center}
  \centerline{\includegraphics[width=\textwidth]{figures/colorization_visuals_supp.pdf}}
  \mycaption{Video Color Propagation}
  {Input grayscale video frames and corresponding ground-truth (GT) color images
  together with color predictions of Levin et al.~\cite{levin2004colorization} and VPN-Stage1 models.}
  \label{fig:color_visuals_supp}
\end{center}
\vspace{-0.7cm}
\end{figure}

\clearpage

\section{Additional Material for Bilateral Inception Networks}
\label{sec:binception-app}

In this section of the Appendix, we first discuss the use of approximate bilateral
filtering in BI modules (Sec.~\ref{sec:lattice}).
Later, we present some qualitative results using different models for the approach presented in
Chapter~\ref{chap:binception} (Sec.~\ref{sec:qualitative-app}).

\subsection{Approximate Bilateral Filtering}
\label{sec:lattice}

The bilateral inception module presented in Chapter~\ref{chap:binception} computes a matrix-vector
product between a Gaussian filter $K$ and a vector of activations $\bz_c$.
Bilateral filtering is an important operation and many algorithmic techniques have been
proposed to speed-up this operation~\cite{paris2006fast,adams2010fast,gastal2011domain}.
In the main paper we opted to implement what can be considered the
brute-force variant of explicitly constructing $K$ and then using BLAS to compute the
matrix-vector product. This resulted in a few millisecond operation.
The explicit way to compute is possible due to the
reduction to super-pixels, e.g., it would not work for DenseCRF variants
that operate on the full image resolution.

Here, we present experiments where we use the fast approximate bilateral filtering
algorithm of~\cite{adams2010fast}, which is also used in Chapter~\ref{chap:bnn}
for learning sparse high dimensional filters. This
choice allows for larger dimensions of matrix-vector multiplication. The reason for choosing
the explicit multiplication in Chapter~\ref{chap:binception} was that it was computationally faster.
For the small sizes of the involved matrices and vectors, the explicit computation is sufficient and we had no
GPU implementation of an approximate technique that matched this runtime. Also it
is conceptually easier and the gradient to the feature transformations ($\Lambda \mathbf{f}$) is
obtained using standard matrix calculus.

\subsubsection{Experiments}

We modified the existing segmentation architectures analogous to those in Chapter~\ref{chap:binception}.
The main difference is that, here, the inception modules use the lattice
approximation~\cite{adams2010fast} to compute the bilateral filtering.
Using the lattice approximation did not allow us to back-propagate through feature transformations ($\Lambda$)
and thus we used hand-specified feature scales as will be explained later.
Specifically, we take CNN architectures from the works
of~\cite{chen2014semantic,zheng2015conditional,bell2015minc} and insert the BI modules between
the spatial FC layers.
We use superpixels from~\cite{DollarICCV13edges}
for all the experiments with the lattice approximation. Experiments are
performed using Caffe neural network framework~\cite{jia2014caffe}.

\begin{table}
  \small
  \centering
  \begin{tabular}{p{5.5cm}>{\raggedright\arraybackslash}p{1.4cm}>{\centering\arraybackslash}p{2.2cm}}
    \toprule
		\textbf{Model} & \emph{IoU} & \emph{Runtime}(ms) \\
    \midrule

    %%%%%%%%%%%% Scores computed by us)%%%%%%%%%%%%
		\deeplablargefov & 68.9 & 145ms\\
    \midrule
    \bi{7}{2}-\bi{8}{10}& \textbf{73.8} & +600 \\
    \midrule
    \deeplablargefovcrf~\cite{chen2014semantic} & 72.7 & +830\\
    \deeplabmsclargefovcrf~\cite{chen2014semantic} & \textbf{73.6} & +880\\
    DeepLab-EdgeNet~\cite{chen2015semantic} & 71.7 & +30\\
    DeepLab-EdgeNet-CRF~\cite{chen2015semantic} & \textbf{73.6} & +860\\
  \bottomrule \\
  \end{tabular}
  \mycaption{Semantic Segmentation using the DeepLab model}
  {IoU scores on the Pascal VOC12 segmentation test dataset
  with different models and our modified inception model.
  Also shown are the corresponding runtimes in milliseconds. Runtimes
  also include superpixel computations (300 ms with Dollar superpixels~\cite{DollarICCV13edges})}
  \label{tab:largefovresults}
\end{table}

\paragraph{Semantic Segmentation}
The experiments in this section use the Pascal VOC12 segmentation dataset~\cite{voc2012segmentation} with 21 object classes and the images have a maximum resolution of 0.25 megapixels.
For all experiments on VOC12, we train using the extended training set of
10581 images collected by~\cite{hariharan2011moredata}.
We modified the \deeplab~network architecture of~\cite{chen2014semantic} and
the CRFasRNN architecture from~\cite{zheng2015conditional} which uses a CNN with
deconvolution layers followed by DenseCRF trained end-to-end.

\paragraph{DeepLab Model}\label{sec:deeplabmodel}
We experimented with the \bi{7}{2}-\bi{8}{10} inception model.
Results using the~\deeplab~model are summarized in Tab.~\ref{tab:largefovresults}.
Although we get similar improvements with inception modules as with the
explicit kernel computation, using lattice approximation is slower.

\begin{table}
  \small
  \centering
  \begin{tabular}{p{6.4cm}>{\raggedright\arraybackslash}p{1.8cm}>{\raggedright\arraybackslash}p{1.8cm}}
    \toprule
    \textbf{Model} & \emph{IoU (Val)} & \emph{IoU (Test)}\\
    \midrule
    %%%%%%%%%%%% Scores computed by us)%%%%%%%%%%%%
    CNN &  67.5 & - \\
    \deconv (CNN+Deconvolutions) & 69.8 & 72.0 \\
    \midrule
    \bi{3}{6}-\bi{4}{6}-\bi{7}{2}-\bi{8}{6}& 71.9 & - \\
    \bi{3}{6}-\bi{4}{6}-\bi{7}{2}-\bi{8}{6}-\gi{6}& 73.6 &  \href{http://host.robots.ox.ac.uk:8080/anonymous/VOTV5E.html}{\textbf{75.2}}\\
    \midrule
    \deconvcrf (CRF-RNN)~\cite{zheng2015conditional} & 73.0 & 74.7\\
    Context-CRF-RNN~\cite{yu2015multi} & ~~ - ~ & \textbf{75.3} \\
    \bottomrule \\
  \end{tabular}
  \mycaption{Semantic Segmentation using the CRFasRNN model}{IoU score corresponding to different models
  on Pascal VOC12 reduced validation / test segmentation dataset. The reduced validation set consists of 346 images
  as used in~\cite{zheng2015conditional} where we adapted the model from.}
  \label{tab:deconvresults-app}
\end{table}

\paragraph{CRFasRNN Model}\label{sec:deepinception}
We add BI modules after score-pool3, score-pool4, \fc{7} and \fc{8} $1\times1$ convolution layers
resulting in the \bi{3}{6}-\bi{4}{6}-\bi{7}{2}-\bi{8}{6}
model and also experimented with another variant where $BI_8$ is followed by another inception
module, G$(6)$, with 6 Gaussian kernels.
Note that here also we discarded both deconvolution and DenseCRF parts of the original model~\cite{zheng2015conditional}
and inserted the BI modules in the base CNN and found similar improvements compared to the inception modules with explicit
kernel computaion. See Tab.~\ref{tab:deconvresults-app} for results on the CRFasRNN model.

\paragraph{Material Segmentation}
Table~\ref{tab:mincresults-app} shows the results on the MINC dataset~\cite{bell2015minc}
obtained by modifying the AlexNet architecture with our inception modules. We observe
similar improvements as with explicit kernel construction.
For this model, we do not provide any learned setup due to very limited segment training
data. The weights to combine outputs in the bilateral inception layer are
found by validation on the validation set.

\begin{table}[t]
  \small
  \centering
  \begin{tabular}{p{3.5cm}>{\centering\arraybackslash}p{4.0cm}}
    \toprule
    \textbf{Model} & Class / Total accuracy\\
    \midrule

    %%%%%%%%%%%% Scores computed by us)%%%%%%%%%%%%
    AlexNet CNN & 55.3 / 58.9 \\
    \midrule
    \bi{7}{2}-\bi{8}{6}& 68.5 / 71.8 \\
    \bi{7}{2}-\bi{8}{6}-G$(6)$& 67.6 / 73.1 \\
    \midrule
    AlexNet-CRF & 65.5 / 71.0 \\
    \bottomrule \\
  \end{tabular}
  \mycaption{Material Segmentation using AlexNet}{Pixel accuracy of different models on
  the MINC material segmentation test dataset~\cite{bell2015minc}.}
  \label{tab:mincresults-app}
\end{table}

\paragraph{Scales of Bilateral Inception Modules}
\label{sec:scales}

Unlike the explicit kernel technique presented in the main text (Chapter~\ref{chap:binception}),
we didn't back-propagate through feature transformation ($\Lambda$)
using the approximate bilateral filter technique.
So, the feature scales are hand-specified and validated, which are as follows.
The optimal scale values for the \bi{7}{2}-\bi{8}{2} model are found by validation for the best performance which are
$\sigma_{xy}$ = (0.1, 0.1) for the spatial (XY) kernel and $\sigma_{rgbxy}$ = (0.1, 0.1, 0.1, 0.01, 0.01) for color and position (RGBXY)  kernel.
Next, as more kernels are added to \bi{8}{2}, we set scales to be $\alpha$*($\sigma_{xy}$, $\sigma_{rgbxy}$).
The value of $\alpha$ is chosen as  1, 0.5, 0.1, 0.05, 0.1, at uniform interval, for the \bi{8}{10} bilateral inception module.


\subsection{Qualitative Results}
\label{sec:qualitative-app}

In this section, we present more qualitative results obtained using the BI module with explicit
kernel computation technique presented in Chapter~\ref{chap:binception}. Results on the Pascal VOC12
dataset~\cite{voc2012segmentation} using the DeepLab-LargeFOV model are shown in Fig.~\ref{fig:semantic_visuals-app},
followed by the results on MINC dataset~\cite{bell2015minc}
in Fig.~\ref{fig:material_visuals-app} and on
Cityscapes dataset~\cite{Cordts2015Cvprw} in Fig.~\ref{fig:street_visuals-app}.


\definecolor{voc_1}{RGB}{0, 0, 0}
\definecolor{voc_2}{RGB}{128, 0, 0}
\definecolor{voc_3}{RGB}{0, 128, 0}
\definecolor{voc_4}{RGB}{128, 128, 0}
\definecolor{voc_5}{RGB}{0, 0, 128}
\definecolor{voc_6}{RGB}{128, 0, 128}
\definecolor{voc_7}{RGB}{0, 128, 128}
\definecolor{voc_8}{RGB}{128, 128, 128}
\definecolor{voc_9}{RGB}{64, 0, 0}
\definecolor{voc_10}{RGB}{192, 0, 0}
\definecolor{voc_11}{RGB}{64, 128, 0}
\definecolor{voc_12}{RGB}{192, 128, 0}
\definecolor{voc_13}{RGB}{64, 0, 128}
\definecolor{voc_14}{RGB}{192, 0, 128}
\definecolor{voc_15}{RGB}{64, 128, 128}
\definecolor{voc_16}{RGB}{192, 128, 128}
\definecolor{voc_17}{RGB}{0, 64, 0}
\definecolor{voc_18}{RGB}{128, 64, 0}
\definecolor{voc_19}{RGB}{0, 192, 0}
\definecolor{voc_20}{RGB}{128, 192, 0}
\definecolor{voc_21}{RGB}{0, 64, 128}
\definecolor{voc_22}{RGB}{128, 64, 128}

\begin{figure*}[!ht]
  \small
  \centering
  \fcolorbox{white}{voc_1}{\rule{0pt}{4pt}\rule{4pt}{0pt}} Background~~
  \fcolorbox{white}{voc_2}{\rule{0pt}{4pt}\rule{4pt}{0pt}} Aeroplane~~
  \fcolorbox{white}{voc_3}{\rule{0pt}{4pt}\rule{4pt}{0pt}} Bicycle~~
  \fcolorbox{white}{voc_4}{\rule{0pt}{4pt}\rule{4pt}{0pt}} Bird~~
  \fcolorbox{white}{voc_5}{\rule{0pt}{4pt}\rule{4pt}{0pt}} Boat~~
  \fcolorbox{white}{voc_6}{\rule{0pt}{4pt}\rule{4pt}{0pt}} Bottle~~
  \fcolorbox{white}{voc_7}{\rule{0pt}{4pt}\rule{4pt}{0pt}} Bus~~
  \fcolorbox{white}{voc_8}{\rule{0pt}{4pt}\rule{4pt}{0pt}} Car~~\\
  \fcolorbox{white}{voc_9}{\rule{0pt}{4pt}\rule{4pt}{0pt}} Cat~~
  \fcolorbox{white}{voc_10}{\rule{0pt}{4pt}\rule{4pt}{0pt}} Chair~~
  \fcolorbox{white}{voc_11}{\rule{0pt}{4pt}\rule{4pt}{0pt}} Cow~~
  \fcolorbox{white}{voc_12}{\rule{0pt}{4pt}\rule{4pt}{0pt}} Dining Table~~
  \fcolorbox{white}{voc_13}{\rule{0pt}{4pt}\rule{4pt}{0pt}} Dog~~
  \fcolorbox{white}{voc_14}{\rule{0pt}{4pt}\rule{4pt}{0pt}} Horse~~
  \fcolorbox{white}{voc_15}{\rule{0pt}{4pt}\rule{4pt}{0pt}} Motorbike~~
  \fcolorbox{white}{voc_16}{\rule{0pt}{4pt}\rule{4pt}{0pt}} Person~~\\
  \fcolorbox{white}{voc_17}{\rule{0pt}{4pt}\rule{4pt}{0pt}} Potted Plant~~
  \fcolorbox{white}{voc_18}{\rule{0pt}{4pt}\rule{4pt}{0pt}} Sheep~~
  \fcolorbox{white}{voc_19}{\rule{0pt}{4pt}\rule{4pt}{0pt}} Sofa~~
  \fcolorbox{white}{voc_20}{\rule{0pt}{4pt}\rule{4pt}{0pt}} Train~~
  \fcolorbox{white}{voc_21}{\rule{0pt}{4pt}\rule{4pt}{0pt}} TV monitor~~\\


  \subfigure{%
    \includegraphics[width=.15\columnwidth]{figures/supplementary/2008_001308_given.png}
  }
  \subfigure{%
    \includegraphics[width=.15\columnwidth]{figures/supplementary/2008_001308_sp.png}
  }
  \subfigure{%
    \includegraphics[width=.15\columnwidth]{figures/supplementary/2008_001308_gt.png}
  }
  \subfigure{%
    \includegraphics[width=.15\columnwidth]{figures/supplementary/2008_001308_cnn.png}
  }
  \subfigure{%
    \includegraphics[width=.15\columnwidth]{figures/supplementary/2008_001308_crf.png}
  }
  \subfigure{%
    \includegraphics[width=.15\columnwidth]{figures/supplementary/2008_001308_ours.png}
  }\\[-2ex]


  \subfigure{%
    \includegraphics[width=.15\columnwidth]{figures/supplementary/2008_001821_given.png}
  }
  \subfigure{%
    \includegraphics[width=.15\columnwidth]{figures/supplementary/2008_001821_sp.png}
  }
  \subfigure{%
    \includegraphics[width=.15\columnwidth]{figures/supplementary/2008_001821_gt.png}
  }
  \subfigure{%
    \includegraphics[width=.15\columnwidth]{figures/supplementary/2008_001821_cnn.png}
  }
  \subfigure{%
    \includegraphics[width=.15\columnwidth]{figures/supplementary/2008_001821_crf.png}
  }
  \subfigure{%
    \includegraphics[width=.15\columnwidth]{figures/supplementary/2008_001821_ours.png}
  }\\[-2ex]



  \subfigure{%
    \includegraphics[width=.15\columnwidth]{figures/supplementary/2008_004612_given.png}
  }
  \subfigure{%
    \includegraphics[width=.15\columnwidth]{figures/supplementary/2008_004612_sp.png}
  }
  \subfigure{%
    \includegraphics[width=.15\columnwidth]{figures/supplementary/2008_004612_gt.png}
  }
  \subfigure{%
    \includegraphics[width=.15\columnwidth]{figures/supplementary/2008_004612_cnn.png}
  }
  \subfigure{%
    \includegraphics[width=.15\columnwidth]{figures/supplementary/2008_004612_crf.png}
  }
  \subfigure{%
    \includegraphics[width=.15\columnwidth]{figures/supplementary/2008_004612_ours.png}
  }\\[-2ex]


  \subfigure{%
    \includegraphics[width=.15\columnwidth]{figures/supplementary/2009_001008_given.png}
  }
  \subfigure{%
    \includegraphics[width=.15\columnwidth]{figures/supplementary/2009_001008_sp.png}
  }
  \subfigure{%
    \includegraphics[width=.15\columnwidth]{figures/supplementary/2009_001008_gt.png}
  }
  \subfigure{%
    \includegraphics[width=.15\columnwidth]{figures/supplementary/2009_001008_cnn.png}
  }
  \subfigure{%
    \includegraphics[width=.15\columnwidth]{figures/supplementary/2009_001008_crf.png}
  }
  \subfigure{%
    \includegraphics[width=.15\columnwidth]{figures/supplementary/2009_001008_ours.png}
  }\\[-2ex]




  \subfigure{%
    \includegraphics[width=.15\columnwidth]{figures/supplementary/2009_004497_given.png}
  }
  \subfigure{%
    \includegraphics[width=.15\columnwidth]{figures/supplementary/2009_004497_sp.png}
  }
  \subfigure{%
    \includegraphics[width=.15\columnwidth]{figures/supplementary/2009_004497_gt.png}
  }
  \subfigure{%
    \includegraphics[width=.15\columnwidth]{figures/supplementary/2009_004497_cnn.png}
  }
  \subfigure{%
    \includegraphics[width=.15\columnwidth]{figures/supplementary/2009_004497_crf.png}
  }
  \subfigure{%
    \includegraphics[width=.15\columnwidth]{figures/supplementary/2009_004497_ours.png}
  }\\[-2ex]



  \setcounter{subfigure}{0}
  \subfigure[\scriptsize Input]{%
    \includegraphics[width=.15\columnwidth]{figures/supplementary/2010_001327_given.png}
  }
  \subfigure[\scriptsize Superpixels]{%
    \includegraphics[width=.15\columnwidth]{figures/supplementary/2010_001327_sp.png}
  }
  \subfigure[\scriptsize GT]{%
    \includegraphics[width=.15\columnwidth]{figures/supplementary/2010_001327_gt.png}
  }
  \subfigure[\scriptsize Deeplab]{%
    \includegraphics[width=.15\columnwidth]{figures/supplementary/2010_001327_cnn.png}
  }
  \subfigure[\scriptsize +DenseCRF]{%
    \includegraphics[width=.15\columnwidth]{figures/supplementary/2010_001327_crf.png}
  }
  \subfigure[\scriptsize Using BI]{%
    \includegraphics[width=.15\columnwidth]{figures/supplementary/2010_001327_ours.png}
  }
  \mycaption{Semantic Segmentation}{Example results of semantic segmentation
  on the Pascal VOC12 dataset.
  (d)~depicts the DeepLab CNN result, (e)~CNN + 10 steps of mean-field inference,
  (f~result obtained with bilateral inception (BI) modules (\bi{6}{2}+\bi{7}{6}) between \fc~layers.}
  \label{fig:semantic_visuals-app}
\end{figure*}


\definecolor{minc_1}{HTML}{771111}
\definecolor{minc_2}{HTML}{CAC690}
\definecolor{minc_3}{HTML}{EEEEEE}
\definecolor{minc_4}{HTML}{7C8FA6}
\definecolor{minc_5}{HTML}{597D31}
\definecolor{minc_6}{HTML}{104410}
\definecolor{minc_7}{HTML}{BB819C}
\definecolor{minc_8}{HTML}{D0CE48}
\definecolor{minc_9}{HTML}{622745}
\definecolor{minc_10}{HTML}{666666}
\definecolor{minc_11}{HTML}{D54A31}
\definecolor{minc_12}{HTML}{101044}
\definecolor{minc_13}{HTML}{444126}
\definecolor{minc_14}{HTML}{75D646}
\definecolor{minc_15}{HTML}{DD4348}
\definecolor{minc_16}{HTML}{5C8577}
\definecolor{minc_17}{HTML}{C78472}
\definecolor{minc_18}{HTML}{75D6D0}
\definecolor{minc_19}{HTML}{5B4586}
\definecolor{minc_20}{HTML}{C04393}
\definecolor{minc_21}{HTML}{D69948}
\definecolor{minc_22}{HTML}{7370D8}
\definecolor{minc_23}{HTML}{7A3622}
\definecolor{minc_24}{HTML}{000000}

\begin{figure*}[!ht]
  \small % scriptsize
  \centering
  \fcolorbox{white}{minc_1}{\rule{0pt}{4pt}\rule{4pt}{0pt}} Brick~~
  \fcolorbox{white}{minc_2}{\rule{0pt}{4pt}\rule{4pt}{0pt}} Carpet~~
  \fcolorbox{white}{minc_3}{\rule{0pt}{4pt}\rule{4pt}{0pt}} Ceramic~~
  \fcolorbox{white}{minc_4}{\rule{0pt}{4pt}\rule{4pt}{0pt}} Fabric~~
  \fcolorbox{white}{minc_5}{\rule{0pt}{4pt}\rule{4pt}{0pt}} Foliage~~
  \fcolorbox{white}{minc_6}{\rule{0pt}{4pt}\rule{4pt}{0pt}} Food~~
  \fcolorbox{white}{minc_7}{\rule{0pt}{4pt}\rule{4pt}{0pt}} Glass~~
  \fcolorbox{white}{minc_8}{\rule{0pt}{4pt}\rule{4pt}{0pt}} Hair~~\\
  \fcolorbox{white}{minc_9}{\rule{0pt}{4pt}\rule{4pt}{0pt}} Leather~~
  \fcolorbox{white}{minc_10}{\rule{0pt}{4pt}\rule{4pt}{0pt}} Metal~~
  \fcolorbox{white}{minc_11}{\rule{0pt}{4pt}\rule{4pt}{0pt}} Mirror~~
  \fcolorbox{white}{minc_12}{\rule{0pt}{4pt}\rule{4pt}{0pt}} Other~~
  \fcolorbox{white}{minc_13}{\rule{0pt}{4pt}\rule{4pt}{0pt}} Painted~~
  \fcolorbox{white}{minc_14}{\rule{0pt}{4pt}\rule{4pt}{0pt}} Paper~~
  \fcolorbox{white}{minc_15}{\rule{0pt}{4pt}\rule{4pt}{0pt}} Plastic~~\\
  \fcolorbox{white}{minc_16}{\rule{0pt}{4pt}\rule{4pt}{0pt}} Polished Stone~~
  \fcolorbox{white}{minc_17}{\rule{0pt}{4pt}\rule{4pt}{0pt}} Skin~~
  \fcolorbox{white}{minc_18}{\rule{0pt}{4pt}\rule{4pt}{0pt}} Sky~~
  \fcolorbox{white}{minc_19}{\rule{0pt}{4pt}\rule{4pt}{0pt}} Stone~~
  \fcolorbox{white}{minc_20}{\rule{0pt}{4pt}\rule{4pt}{0pt}} Tile~~
  \fcolorbox{white}{minc_21}{\rule{0pt}{4pt}\rule{4pt}{0pt}} Wallpaper~~
  \fcolorbox{white}{minc_22}{\rule{0pt}{4pt}\rule{4pt}{0pt}} Water~~
  \fcolorbox{white}{minc_23}{\rule{0pt}{4pt}\rule{4pt}{0pt}} Wood~~\\
  \subfigure{%
    \includegraphics[width=.15\columnwidth]{figures/supplementary/000008468_given.png}
  }
  \subfigure{%
    \includegraphics[width=.15\columnwidth]{figures/supplementary/000008468_sp.png}
  }
  \subfigure{%
    \includegraphics[width=.15\columnwidth]{figures/supplementary/000008468_gt.png}
  }
  \subfigure{%
    \includegraphics[width=.15\columnwidth]{figures/supplementary/000008468_cnn.png}
  }
  \subfigure{%
    \includegraphics[width=.15\columnwidth]{figures/supplementary/000008468_crf.png}
  }
  \subfigure{%
    \includegraphics[width=.15\columnwidth]{figures/supplementary/000008468_ours.png}
  }\\[-2ex]

  \subfigure{%
    \includegraphics[width=.15\columnwidth]{figures/supplementary/000009053_given.png}
  }
  \subfigure{%
    \includegraphics[width=.15\columnwidth]{figures/supplementary/000009053_sp.png}
  }
  \subfigure{%
    \includegraphics[width=.15\columnwidth]{figures/supplementary/000009053_gt.png}
  }
  \subfigure{%
    \includegraphics[width=.15\columnwidth]{figures/supplementary/000009053_cnn.png}
  }
  \subfigure{%
    \includegraphics[width=.15\columnwidth]{figures/supplementary/000009053_crf.png}
  }
  \subfigure{%
    \includegraphics[width=.15\columnwidth]{figures/supplementary/000009053_ours.png}
  }\\[-2ex]




  \subfigure{%
    \includegraphics[width=.15\columnwidth]{figures/supplementary/000014977_given.png}
  }
  \subfigure{%
    \includegraphics[width=.15\columnwidth]{figures/supplementary/000014977_sp.png}
  }
  \subfigure{%
    \includegraphics[width=.15\columnwidth]{figures/supplementary/000014977_gt.png}
  }
  \subfigure{%
    \includegraphics[width=.15\columnwidth]{figures/supplementary/000014977_cnn.png}
  }
  \subfigure{%
    \includegraphics[width=.15\columnwidth]{figures/supplementary/000014977_crf.png}
  }
  \subfigure{%
    \includegraphics[width=.15\columnwidth]{figures/supplementary/000014977_ours.png}
  }\\[-2ex]


  \subfigure{%
    \includegraphics[width=.15\columnwidth]{figures/supplementary/000022922_given.png}
  }
  \subfigure{%
    \includegraphics[width=.15\columnwidth]{figures/supplementary/000022922_sp.png}
  }
  \subfigure{%
    \includegraphics[width=.15\columnwidth]{figures/supplementary/000022922_gt.png}
  }
  \subfigure{%
    \includegraphics[width=.15\columnwidth]{figures/supplementary/000022922_cnn.png}
  }
  \subfigure{%
    \includegraphics[width=.15\columnwidth]{figures/supplementary/000022922_crf.png}
  }
  \subfigure{%
    \includegraphics[width=.15\columnwidth]{figures/supplementary/000022922_ours.png}
  }\\[-2ex]


  \subfigure{%
    \includegraphics[width=.15\columnwidth]{figures/supplementary/000025711_given.png}
  }
  \subfigure{%
    \includegraphics[width=.15\columnwidth]{figures/supplementary/000025711_sp.png}
  }
  \subfigure{%
    \includegraphics[width=.15\columnwidth]{figures/supplementary/000025711_gt.png}
  }
  \subfigure{%
    \includegraphics[width=.15\columnwidth]{figures/supplementary/000025711_cnn.png}
  }
  \subfigure{%
    \includegraphics[width=.15\columnwidth]{figures/supplementary/000025711_crf.png}
  }
  \subfigure{%
    \includegraphics[width=.15\columnwidth]{figures/supplementary/000025711_ours.png}
  }\\[-2ex]


  \subfigure{%
    \includegraphics[width=.15\columnwidth]{figures/supplementary/000034473_given.png}
  }
  \subfigure{%
    \includegraphics[width=.15\columnwidth]{figures/supplementary/000034473_sp.png}
  }
  \subfigure{%
    \includegraphics[width=.15\columnwidth]{figures/supplementary/000034473_gt.png}
  }
  \subfigure{%
    \includegraphics[width=.15\columnwidth]{figures/supplementary/000034473_cnn.png}
  }
  \subfigure{%
    \includegraphics[width=.15\columnwidth]{figures/supplementary/000034473_crf.png}
  }
  \subfigure{%
    \includegraphics[width=.15\columnwidth]{figures/supplementary/000034473_ours.png}
  }\\[-2ex]


  \subfigure{%
    \includegraphics[width=.15\columnwidth]{figures/supplementary/000035463_given.png}
  }
  \subfigure{%
    \includegraphics[width=.15\columnwidth]{figures/supplementary/000035463_sp.png}
  }
  \subfigure{%
    \includegraphics[width=.15\columnwidth]{figures/supplementary/000035463_gt.png}
  }
  \subfigure{%
    \includegraphics[width=.15\columnwidth]{figures/supplementary/000035463_cnn.png}
  }
  \subfigure{%
    \includegraphics[width=.15\columnwidth]{figures/supplementary/000035463_crf.png}
  }
  \subfigure{%
    \includegraphics[width=.15\columnwidth]{figures/supplementary/000035463_ours.png}
  }\\[-2ex]


  \setcounter{subfigure}{0}
  \subfigure[\scriptsize Input]{%
    \includegraphics[width=.15\columnwidth]{figures/supplementary/000035993_given.png}
  }
  \subfigure[\scriptsize Superpixels]{%
    \includegraphics[width=.15\columnwidth]{figures/supplementary/000035993_sp.png}
  }
  \subfigure[\scriptsize GT]{%
    \includegraphics[width=.15\columnwidth]{figures/supplementary/000035993_gt.png}
  }
  \subfigure[\scriptsize AlexNet]{%
    \includegraphics[width=.15\columnwidth]{figures/supplementary/000035993_cnn.png}
  }
  \subfigure[\scriptsize +DenseCRF]{%
    \includegraphics[width=.15\columnwidth]{figures/supplementary/000035993_crf.png}
  }
  \subfigure[\scriptsize Using BI]{%
    \includegraphics[width=.15\columnwidth]{figures/supplementary/000035993_ours.png}
  }
  \mycaption{Material Segmentation}{Example results of material segmentation.
  (d)~depicts the AlexNet CNN result, (e)~CNN + 10 steps of mean-field inference,
  (f)~result obtained with bilateral inception (BI) modules (\bi{7}{2}+\bi{8}{6}) between
  \fc~layers.}
\label{fig:material_visuals-app}
\end{figure*}


\definecolor{city_1}{RGB}{128, 64, 128}
\definecolor{city_2}{RGB}{244, 35, 232}
\definecolor{city_3}{RGB}{70, 70, 70}
\definecolor{city_4}{RGB}{102, 102, 156}
\definecolor{city_5}{RGB}{190, 153, 153}
\definecolor{city_6}{RGB}{153, 153, 153}
\definecolor{city_7}{RGB}{250, 170, 30}
\definecolor{city_8}{RGB}{220, 220, 0}
\definecolor{city_9}{RGB}{107, 142, 35}
\definecolor{city_10}{RGB}{152, 251, 152}
\definecolor{city_11}{RGB}{70, 130, 180}
\definecolor{city_12}{RGB}{220, 20, 60}
\definecolor{city_13}{RGB}{255, 0, 0}
\definecolor{city_14}{RGB}{0, 0, 142}
\definecolor{city_15}{RGB}{0, 0, 70}
\definecolor{city_16}{RGB}{0, 60, 100}
\definecolor{city_17}{RGB}{0, 80, 100}
\definecolor{city_18}{RGB}{0, 0, 230}
\definecolor{city_19}{RGB}{119, 11, 32}
\begin{figure*}[!ht]
  \small % scriptsize
  \centering


  \subfigure{%
    \includegraphics[width=.18\columnwidth]{figures/supplementary/frankfurt00000_016005_given.png}
  }
  \subfigure{%
    \includegraphics[width=.18\columnwidth]{figures/supplementary/frankfurt00000_016005_sp.png}
  }
  \subfigure{%
    \includegraphics[width=.18\columnwidth]{figures/supplementary/frankfurt00000_016005_gt.png}
  }
  \subfigure{%
    \includegraphics[width=.18\columnwidth]{figures/supplementary/frankfurt00000_016005_cnn.png}
  }
  \subfigure{%
    \includegraphics[width=.18\columnwidth]{figures/supplementary/frankfurt00000_016005_ours.png}
  }\\[-2ex]

  \subfigure{%
    \includegraphics[width=.18\columnwidth]{figures/supplementary/frankfurt00000_004617_given.png}
  }
  \subfigure{%
    \includegraphics[width=.18\columnwidth]{figures/supplementary/frankfurt00000_004617_sp.png}
  }
  \subfigure{%
    \includegraphics[width=.18\columnwidth]{figures/supplementary/frankfurt00000_004617_gt.png}
  }
  \subfigure{%
    \includegraphics[width=.18\columnwidth]{figures/supplementary/frankfurt00000_004617_cnn.png}
  }
  \subfigure{%
    \includegraphics[width=.18\columnwidth]{figures/supplementary/frankfurt00000_004617_ours.png}
  }\\[-2ex]

  \subfigure{%
    \includegraphics[width=.18\columnwidth]{figures/supplementary/frankfurt00000_020880_given.png}
  }
  \subfigure{%
    \includegraphics[width=.18\columnwidth]{figures/supplementary/frankfurt00000_020880_sp.png}
  }
  \subfigure{%
    \includegraphics[width=.18\columnwidth]{figures/supplementary/frankfurt00000_020880_gt.png}
  }
  \subfigure{%
    \includegraphics[width=.18\columnwidth]{figures/supplementary/frankfurt00000_020880_cnn.png}
  }
  \subfigure{%
    \includegraphics[width=.18\columnwidth]{figures/supplementary/frankfurt00000_020880_ours.png}
  }\\[-2ex]



  \subfigure{%
    \includegraphics[width=.18\columnwidth]{figures/supplementary/frankfurt00001_007285_given.png}
  }
  \subfigure{%
    \includegraphics[width=.18\columnwidth]{figures/supplementary/frankfurt00001_007285_sp.png}
  }
  \subfigure{%
    \includegraphics[width=.18\columnwidth]{figures/supplementary/frankfurt00001_007285_gt.png}
  }
  \subfigure{%
    \includegraphics[width=.18\columnwidth]{figures/supplementary/frankfurt00001_007285_cnn.png}
  }
  \subfigure{%
    \includegraphics[width=.18\columnwidth]{figures/supplementary/frankfurt00001_007285_ours.png}
  }\\[-2ex]


  \subfigure{%
    \includegraphics[width=.18\columnwidth]{figures/supplementary/frankfurt00001_059789_given.png}
  }
  \subfigure{%
    \includegraphics[width=.18\columnwidth]{figures/supplementary/frankfurt00001_059789_sp.png}
  }
  \subfigure{%
    \includegraphics[width=.18\columnwidth]{figures/supplementary/frankfurt00001_059789_gt.png}
  }
  \subfigure{%
    \includegraphics[width=.18\columnwidth]{figures/supplementary/frankfurt00001_059789_cnn.png}
  }
  \subfigure{%
    \includegraphics[width=.18\columnwidth]{figures/supplementary/frankfurt00001_059789_ours.png}
  }\\[-2ex]


  \subfigure{%
    \includegraphics[width=.18\columnwidth]{figures/supplementary/frankfurt00001_068208_given.png}
  }
  \subfigure{%
    \includegraphics[width=.18\columnwidth]{figures/supplementary/frankfurt00001_068208_sp.png}
  }
  \subfigure{%
    \includegraphics[width=.18\columnwidth]{figures/supplementary/frankfurt00001_068208_gt.png}
  }
  \subfigure{%
    \includegraphics[width=.18\columnwidth]{figures/supplementary/frankfurt00001_068208_cnn.png}
  }
  \subfigure{%
    \includegraphics[width=.18\columnwidth]{figures/supplementary/frankfurt00001_068208_ours.png}
  }\\[-2ex]

  \subfigure{%
    \includegraphics[width=.18\columnwidth]{figures/supplementary/frankfurt00001_082466_given.png}
  }
  \subfigure{%
    \includegraphics[width=.18\columnwidth]{figures/supplementary/frankfurt00001_082466_sp.png}
  }
  \subfigure{%
    \includegraphics[width=.18\columnwidth]{figures/supplementary/frankfurt00001_082466_gt.png}
  }
  \subfigure{%
    \includegraphics[width=.18\columnwidth]{figures/supplementary/frankfurt00001_082466_cnn.png}
  }
  \subfigure{%
    \includegraphics[width=.18\columnwidth]{figures/supplementary/frankfurt00001_082466_ours.png}
  }\\[-2ex]

  \subfigure{%
    \includegraphics[width=.18\columnwidth]{figures/supplementary/lindau00033_000019_given.png}
  }
  \subfigure{%
    \includegraphics[width=.18\columnwidth]{figures/supplementary/lindau00033_000019_sp.png}
  }
  \subfigure{%
    \includegraphics[width=.18\columnwidth]{figures/supplementary/lindau00033_000019_gt.png}
  }
  \subfigure{%
    \includegraphics[width=.18\columnwidth]{figures/supplementary/lindau00033_000019_cnn.png}
  }
  \subfigure{%
    \includegraphics[width=.18\columnwidth]{figures/supplementary/lindau00033_000019_ours.png}
  }\\[-2ex]

  \subfigure{%
    \includegraphics[width=.18\columnwidth]{figures/supplementary/lindau00052_000019_given.png}
  }
  \subfigure{%
    \includegraphics[width=.18\columnwidth]{figures/supplementary/lindau00052_000019_sp.png}
  }
  \subfigure{%
    \includegraphics[width=.18\columnwidth]{figures/supplementary/lindau00052_000019_gt.png}
  }
  \subfigure{%
    \includegraphics[width=.18\columnwidth]{figures/supplementary/lindau00052_000019_cnn.png}
  }
  \subfigure{%
    \includegraphics[width=.18\columnwidth]{figures/supplementary/lindau00052_000019_ours.png}
  }\\[-2ex]




  \subfigure{%
    \includegraphics[width=.18\columnwidth]{figures/supplementary/lindau00027_000019_given.png}
  }
  \subfigure{%
    \includegraphics[width=.18\columnwidth]{figures/supplementary/lindau00027_000019_sp.png}
  }
  \subfigure{%
    \includegraphics[width=.18\columnwidth]{figures/supplementary/lindau00027_000019_gt.png}
  }
  \subfigure{%
    \includegraphics[width=.18\columnwidth]{figures/supplementary/lindau00027_000019_cnn.png}
  }
  \subfigure{%
    \includegraphics[width=.18\columnwidth]{figures/supplementary/lindau00027_000019_ours.png}
  }\\[-2ex]



  \setcounter{subfigure}{0}
  \subfigure[\scriptsize Input]{%
    \includegraphics[width=.18\columnwidth]{figures/supplementary/lindau00029_000019_given.png}
  }
  \subfigure[\scriptsize Superpixels]{%
    \includegraphics[width=.18\columnwidth]{figures/supplementary/lindau00029_000019_sp.png}
  }
  \subfigure[\scriptsize GT]{%
    \includegraphics[width=.18\columnwidth]{figures/supplementary/lindau00029_000019_gt.png}
  }
  \subfigure[\scriptsize Deeplab]{%
    \includegraphics[width=.18\columnwidth]{figures/supplementary/lindau00029_000019_cnn.png}
  }
  \subfigure[\scriptsize Using BI]{%
    \includegraphics[width=.18\columnwidth]{figures/supplementary/lindau00029_000019_ours.png}
  }%\\[-2ex]

  \mycaption{Street Scene Segmentation}{Example results of street scene segmentation.
  (d)~depicts the DeepLab results, (e)~result obtained by adding bilateral inception (BI) modules (\bi{6}{2}+\bi{7}{6}) between \fc~layers.}
\label{fig:street_visuals-app}
\end{figure*}

% \fi

\ifcamera
\balance
\clearpage
\bibliographystyle{ACM-Reference-Format}
\else
\bibliographystyle{abbrvnaturl}
\fi
\bibliography{fstar}

\end{document}
