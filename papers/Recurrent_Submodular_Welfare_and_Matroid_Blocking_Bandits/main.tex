\documentclass[11pt]{article}

\usepackage[english]{babel}
\usepackage[T1]{fontenc}
\usepackage[margin=1in]{geometry} 
\linespread{1}

\usepackage{tikz}
\usepackage{amsmath}
\usepackage{amsthm}
\usepackage{amssymb}
\usepackage{mathtools}
\usepackage{graphicx}
\usepackage{subcaption}
\usepackage{amssymb}
\usepackage{bm}
\usepackage{xspace}
\usepackage{xcolor}
\usepackage{comment}
\usepackage{float} 
\usepackage{thmtools}
\usepackage{thm-restate}


\newtheorem{theorem}{Theorem}[section]
\newtheorem{lemma}[theorem]{Lemma}
\newtheorem{definition}[theorem]{Definition}
\newtheorem{notation}[theorem]{Notation}
\newtheorem{proposition}[theorem]{Proposition}
\newtheorem{corollary}[theorem]{Corollary}
\newtheorem{conjecture}[theorem]{Conjecture}
\newtheorem{assumption}[theorem]{Assumption}
\newtheorem{observation}[theorem]{Observation}
\newtheorem{fact}[theorem]{Fact}
\newtheorem{algorithm}[theorem]{Algorithm}
\newtheorem{remark}[theorem]{Remark}
\newtheorem{claim}[theorem]{Claim}
\newtheorem{example}[theorem]{Example}
\newtheorem{problem}[theorem]{Problem}
\newtheorem{open}[theorem]{Open Problem}
\newtheorem{hypothesis}[theorem]{Hypothesis}

\DeclareMathOperator{\A}{\mathcal{A}\xspace}
\DeclareMathOperator{\Op}{\mathcal{O}\xspace}
\DeclareMathOperator{\M}{\mathcal{M}}
\DeclareMathOperator{\F}{\mathcal{F}}
\DeclareMathOperator{\B}{\mathcal{B}}
\DeclareMathOperator{\G}{\mathcal{G}}
\DeclareMathOperator{\I}{\mathcal{I}}
\DeclareMathOperator{\T}{\mathcal{T}}
\DeclareMathOperator{\x}{\mathbf{x}\xspace}
\DeclareMathOperator{\z}{\mathbf{z}\xspace}
\DeclareMathOperator{\X}{\mathbf{X}\xspace}
\DeclareMathOperator{\w}{\mathbf{w}\xspace}
\DeclareMathOperator{\mub}{\mathbf{\mu}\xspace}
\input{tikzlibraryipe.code}

\DeclareMathOperator{\rk}{rk\xspace}
\DeclareMathOperator{\off}{\mathbf{r}\xspace}


%\newcommand\event[1]{\mathop{\mathbb{I}\left(#1\right)}}
\newcommand\event[1]{\mathop{\mathcal{X}\left(#1\right)}}

\newcommand\E[1]{\mathop{\mathbb{E}\left[#1\right]}}
\newcommand\Ex[2]{\mathop{\underset{#1}{\mathbb{E}}\left[#2\right]}}

\newcommand\Pro[1]{\mathop{\mathbb{P}\left(#1\right)}}
\newcommand\Prob[2]{\mathop{\underset{#1}{\mathbb{P}}\left(#2\right)}}

\DeclareMathOperator{\extr}{\mathcal{Z}}
\DeclareMathOperator{\extrsub}{\mathcal{Z}^S}
\DeclareMathOperator{\pit}{\tilde{\pi}\xspace}
\DeclareMathOperator{\Reg}{\mathcal{R}eg\xspace}
\DeclareMathOperator{\Rew}{\mathcal{R}\xspace}
\DeclareMathOperator*{\maximize}{\bf{maximize:}}

\newcommand{\rsmfull}{
\textsc{Recurrent Submodular Welfare}%
}
\newcommand{\rsm}{{RSW}\xspace
}
\newcommand{\mbbfull}{
\textsc{Matroid Blocking Bandits}%
}
\newcommand{\mbb}{{MBB}\xspace
}

\usepackage{hyperref}
\hypersetup{
    colorlinks=true,
    linkcolor=black,
    citecolor=blue
}
\newcommand*{\QED}{\hfill\ensuremath{\blacksquare}}%
\usepackage{float}
\usepackage{bookmark}
\raggedbottom

\makeatletter
\newcommand{\printfnsymbol}[1]{%
  \textsuperscript{\@fnsymbol{#1}}%
}
\makeatother

\newcommand{\greedy}{\textsc{greedy}\xspace}
\newcommand{\ig}{\textsc{interleaved-greedy}\xspace}
\newcommand{\IG}{IG\xspace}

\newcommand{\UCB}{IB\xspace}

\newcommand{\is}{\textsc{interleaved-submodular}\xspace}
\newcommand{\IS}{IS\xspace}
\newcommand{\OPT}{\textsc{OPT}\xspace}
\newcommand{\rr}{\textsc{randomized-rounding}\xspace}
\newcommand{\fg}{\textsc{fractional-greedy}\xspace}


\newcommand{\ucb}{\textsc{interleaved-ucb}\xspace}

\title{Recurrent Submodular Welfare and \\
Matroid Blocking Bandits}
\author{
Orestis Papadigenopoulos\\
\texttt{papadig@cs.utexas.edu}\\
Department of Computer Science\\
The University of Texas at Austin, USA.
\and
Constantine Caramanis\\
\texttt{constantine@utexas.edu}\\
Department of Electrical and Computer Engineering\\
The University of Texas at Austin, USA.
}
\date{\today}

\begin{document}

\maketitle


\begin{abstract}
\begin{abstract}
\label{sec:abstract}

%% 1. what is the problem 
Scientific applications that run on leadership computing facilities often face the challenge 
of being unable to fit leading science cases onto accelerator devices due to memory constraints 
(memory-bound applications).
%
% 2. what is your solution 
In this work, the authors studied one such US Department of Energy mission-critical condensed matter 
physics application, Dynamical Cluster Approximation (DCA++), and this paper discusses how device memory-bound challenges were successfully reduced  by proposing an effective 
``all-to-all'' communication method---a ring communication algorithm. 
%
This implementation takes advantage of acceleration on GPUs and remote direct memory access (RDMA) for fast data exchange between GPUs. 
%
\\Additionally, the ring algorithm was optimized with sub-ring communicators
and multi-threaded support to further reduce communication overhead and 
expose more concurrency, respectively.
%
% 3. What's the cherry-picked evaluation result you want to mention
The computation and communication were also analyzed 
by using the Autonomic Performance Environment for Exascale 
(APEX) profiling tool,  and this paper further discusses the 
performance trade-off for the ring algorithm implementation. 
%
The memory analysis on the ring algorithm shows that the allocation size for the authors' most 
memory-intensive data structure per GPU is now reduced to $1/p$ of the original size, where $p$ is the number of GPUs in the ring communicator.
%
The communication analysis suggests that 
the distributed Quantum Monte Carlo execution time grows linearly as sub-ring size increases, and the cost of messages passing through the network interface connector could be a limiting factor.


%
% \todoRed{Ronnie: Next sentence needs rewrite, too much information about Green's function that no one knows in the abstract; recommend generalizing.} \emph {However, DCA++ is currently facing memory-bound challenge as 
% a larger device array $G_t$ is limited by device memory size, where
% $G_t$ is a two-particle Green's function that allows condensed matter
% scientists to explore larger and more complex (higher fidelity)
% physics cases.}

\end{abstract}

\keywords{DCA++, Quantum Monte Carlo, GPU Remote Direct Memory Access, memory-bound issue, exascale machines}

\end{abstract}

\pagenumbering{arabic}
%\linenumbers


\section{Introduction}  \label{sec:introduction}

\newcommand\inexpIntro[3]{#1?(#2,#3).}
\newcommand\rinexpIntro[3]{*#1?(#2,#3).}
\newcommand\outexpIntro[3]{#1!(#2,#3).}
\newcommand\outatomIntro[3]{#1!(#2,#3)}

We propose a fully automated method for proving termination of \(\pi\)-calculus processes.
Although there have been a lot of studies on termination analysis for the \(\pi\)-calculus
and related calculi~\cite{Deng06IC,Demangeon07,SangiorgiTermination,KobayashiHybrid,Yoshida04IC,DBLP:journals/jlp/DemangeonHS10,Venet98SAS}, most of them have been rather theoretical,
and there have been surprisingly little efforts in developing  fully automated termination
verification methods and tools based on them. To our knowledge,
Kobayashi's \typical{}~\cite{TyPiCal,KobayashiHybrid} is the only exception that
can prove termination of \(\pi\)-calculus processes (extended with natural numbers)
fully automatically, but its termination analysis is quite limited (see Section~\ref{sec:relatedwork}).

Our method is based on a reduction to termination analysis for sequential programs:
we translate a \(\pi\)-calculus process \(P\) to a sequential program \(S_P\), so that
if \(S_P\) is terminating, so is \(P\). The reduction allows us to use
powerful, mature methods and tools
for termination analysis of sequential programs~\cite{heizmann2016ultimate,freqterm,DBLP:conf/lics/PodelskiR04,Kuwahara2014Termination,DBLP:journals/cacm/CookPR11}.

The idea of the translation is to convert a chain of communications on replicated input
channels to a chain of recursive function calls of the target sequential program.
Let us consider the following Fibonacci process:
\begin{align*}
    & \rinexpIntro{\fib}{n}{r}
        \ifexp{n<2}{ \soutatom{r}{1} \\ &\quad}
                   { \nuexp{s_1} \nuexp{s_2} (\outatomIntro{\fib}{n-1}{s_1} \PAR \outatomIntro{\fib}{n-2}{s_2} \PAR \sinexp{s_1}{x}\sinexp{s_2}{y}\soutatom{r}{x+y}) \\}
    & \PAR \outatomIntro{\fib}{m}{r}
\end{align*}
Here, the process
$\rinexpIntro{\fib}{n}{r} \ldots$ is a function server that computes the \(n\)-th Fibonacci number
in parallel and returns the result to \(r\),
and $\outatom{\fib}{m}{r}$ sends a request for computing the \(m\)-th Fibonacci number;
those who are not familiar with the syntax of the \(\pi\)-calculus may wish to consult
Section~\ref{sec:targetlanguage} first.
To prove that the process above is terminating for any integer \(m\),
it suffices to show that there is no infinite chain of communications on $\fib$:
\[
    \fib(m,r) \to \fib(m_1,r_1) \to \fib(m_2,r_2) \to \cdots.
\]
We convert the process above to the following program:\footnote{The actual translation
  given later is a little more complex.}
\begin{verbatim}
 let rec fib(n) = if n<2 then () else (fib(n-1) [] fib(n-2)) in
 fib(m)
\end{verbatim}
Here, \texttt{[]} represents the non-deterministic choice.
Note that, although the calculation of Fibonacci numbers is not preserved,
for each chain of communications on \texttt{fib}, there is a corresponding
sequence of recursive calls:
\[
\mathtt{fib}(m) \to \mathtt{fib}(m_1) \to \mathtt{fib}(m_2) \to \cdots.
\]
Thus, the termination of the sequential program above implies the termination of
the original process.
As shown in the example above, (i) each communication on a replicated input channel
is converted to a function call, (ii) each communication on a non-replicated input
channel is just removed (or, in the actual translation, replaced by a call of
a trivial function defined by \(f(\seq{x})=(\,)\)), and (iii) parallel composition
is replaced by a non-deterministic choice.
We formalize the translation outlined above and prove its correctness.

The basic translation sketched above sometimes loses too much information.
For example, consider the following process:
\begin{align*}
    & \rinexpIntro{\pre}{n}{r} \soutatom{r}{n-1} \\
    & \PAR \rinexpIntro{f}{n}{r} \ifexp{n<0}{ \soutatom{r}{1} }
                                       { \nuexp{s} (\outatomIntro{\pre}{n}{s} \PAR \sinexp{s}{x}\outatomIntro{f}{x}{r}) } \\
    & \PAR \outatomIntro{f}{m}{r}
\end{align*}
The translation sketched above would yield:
\begin{verbatim}
  let pred(n) = n-1 in
  let rec f(n) = if n<0 then () else (pred(n) [] f(*)) in
  f(m)
\end{verbatim}
Here, \texttt{*} represents a non-deterministic integer: since we have removed
the input $\sinatom{s}{x}$, we do not have information about the value of \( x \).
As a result, the sequential program above is non-terminating, although the original
process is terminating.
To remedy this problem, we also refine the basic translation above by using a refinement
type system for the \(\pi\)-calculus. Using the refinement type system,
we can infer that the value of \(x\) in the original process is less than \(n\),
so that we can refine the definition of \texttt{f} to:
\begin{verbatim}
 let rec f(n) = ... else (pred(n) [] let x=* in assume(x<n);f(x))
\end{verbatim}
The target program is now terminating, from which
we can deduce that the original process is also terminating.
We have implemented an automated tool based on the refined translation above.

The contributions of this paper are summarized as follows.
\begin{itemize}
\item The formalization of the basic translation from the \(\pi\)-calculus
  (extended with integers) to sequential programs, and a proof of its correctness.
\item The formalization of a refined translation based on a refinement type system.
\item An implementation of the refined translation, including automated refinement type
  inference based on CHC solving, and experiments to evaluate the effectiveness of
  our method.
\end{itemize}

The rest of this paper is structured as follows.
Section~\ref{sec:targetlanguage} introduces the source and target languages
of our translation.
Section~\ref{sec:approach} 
formalizes the basic translation, and proves its correctness.
Section~\ref{sec:refinement} refines the basic translation by using a refinement type system.
Section~\ref{sec:implementation} reports an implementation and experiments.
Section~\ref{sec:relatedwork} discusses related work,
and Section~\ref{sec:conclusion} concludes the paper.

\section{Preliminaries on Matroids and Submodular Functions}\label{sec:definition}

\paragraph{Continuous extensions and the correlation gap of submodular functions.}

Consider any set function $f : 2^{\A} \rightarrow \mathbb{R}_{\geq 0}$ over a ground set $\A$. Recall that $f$ is submodular, if $\forall S,T \subseteq \A$ we have $f(S \cup T) + f(S \cap T) \leq f(S) + f(T)$. For any point $\x \in [0,1]^k$, we denote by $S \sim \x$ the random set $S \subseteq \A$, such that $\Pro{i \in S} = x_i$. We consider two canonical continuous extensions of a set function: 

\begin{definition}[Continuous extensions]
For any set function $f$ the {\em multi-linear extension} is
\begin{align*}
    F(\x) = \Ex{S \sim \x}{f(S)} = \sum_{S \subseteq \A} f(S) \prod_{i \in S} x_i \prod_{i \notin S} (1-x_i).
\end{align*}
Moreover, the {\em concave closure} is defined as
\begin{align*}
    f^+(\x) = \max_{\alpha}\{\sum_{S \subseteq \A} \alpha_S f(S)~|~\sum_{S \subseteq \A} \alpha_S \bm{1}_S = \x, \sum_{S \subseteq \A} \alpha_S = 1, \alpha \succeq 0\}.
\end{align*}
\end{definition}

\begin{lemma}[Correlation gap \cite{CCPV07}] \label{lem:correlationgap}
Let $f: 2^k \rightarrow \mathbb{R}_{\geq 0}$ be a monotone (non-decreasing) submodular function. Then for any point $\x \in [0,1]^k$, we have
$
F(\x) \leq f^+(\x) \leq \left(1 - \frac{1}{e}\right)^{-1} F(\x).
$
\end{lemma}


\paragraph{Matroid polytope and the weighted rank function.} Consider a matroid $\M = (\A, \I)$, where $\A$ is the {\em ground set} and $\I$ is the family of {\em independent sets}
\footnote{Any subset of the ground set $\A$ that is not independent is called {\em dependent}. Any maximal independent set of a matroid, namely, a set $B \in \I$ such that for every $e \in \A \backslash B$, the set $B \cup \{e\}$ is dependent, is called a {\em basis}. Any minimal dependent set, that is, a set $C \notin \I$ such that for each $e \in C$ it holds $C\backslash \{e\} \in \I$ is called a {\em circuit}.}. Recall that in any matroid, the family $\I$ satisfies the following two properties: (i) Every subset of an independent set (including the empty set) is an independent set, namely, if $S' \subset S \subseteq \A$ and $S \in \I$, then $S' \in \I$ ({\em hereditary property}). (ii) Let $S, S' \subseteq \A$ be two independent sets with $|S| < |S'|$, then there exists some $e \in S' \backslash S$ such that $S \cup \{e\} \in \I$ ({\em augmentation property}). See \cite{schrijver03, oxley06} for more details on matroids.

We assume that access to $\M$ is given through an {\em independence oracle} \cite{hausmann81, robinson80}, namely, a black-box routine that, given a set $S \subseteq \A$, answers whether $S$ is an independent set of $\M$. 
For any set $R \subset \A$ we define the {\em restriction} of $\M$ to $R$, denoted by $\M | R$, to be the matroid $\M | R = (R, \{I \in \I~|~I \subseteq R\})$. Every matroid $\M$ is associated with a {\em rank} function\footnote{The rank is {\em monotone} non-decreasing, {\em submodular}, and satisfies $\rk(S) \leq |S|$, $\forall S \subseteq \A$ (see \cite{oxley06}).} $\rk: 2^{\A} \rightarrow \mathbb{N}$, such that for any $S\subseteq \A$, $\rk(S)$ denotes the maximum size of an independent set contained in $S$. Let $\x(S) = \sum_{e \in S} x_e$ for some vector $\x \in \mathbb{R}^k$. For any matroid $\M = (\A,\I)$, the {\em matroid polytope} is defined as

$$\mathcal{P}(\M) \equiv \left\{ \x(S) \leq \rk(S), \forall S \in 2^{\A}, S \neq \emptyset \text{ and } \x \succeq 0 \right\}.
$$
It can be proved \cite{schrijver03} that the above polytope is the convex hull of the indicator vectors of all independent sets. This fact immediately leads to the following lemma: 

\begin{lemma}\label{lem:characteristic}
For any matroid $\M=(\A, \I)$ and point $\x \in \mathcal{P}(\M)$, there exists a collection of $k = |\A|$ independent sets $\I(\x) = \{ I_1, \dots, I_k\} \subseteq \I$ and a probability distribution over $\I(\x)$ such that $\Prob{I \sim \I(\x)}{i \in I} = x_i$, i.e., an element $i$ belongs to a sampled set with marginal probability equal to $x_i$.
\end{lemma}
Given any non-negative linear {\em weight} vector $\w \in \mathbb{R}^k_{\geq 0}$, the problem of computing a maximum weight independent set can be solved optimally by the standard greedy algorithm: Starting from the empty set $S = \emptyset$, add each ground element $e \in \A$ to the set $S$ in a non-increasing order of weights, as long as the set $S \cup \{e\}$ does not contain a circuit. Given a matroid $\M=(\A,\I)$ and a weight vector $\w$, the function $f_{\M,\w}(S) = \max_{I \in \I, I \subseteq S}\{\w(I)\}$ is called the {\em weighted rank function} of $\M$ and returns the weight of the maximum independent set of the restriction $\M|S$.

\begin{lemma}[Weighted rank function \cite{CCPV07}] \label{lem:weightedrank}
For any matroid $\M$ and non-negative weight vector $\w$, the function $f_{\M, \w}(S) = \max_{I \in \I, I \subseteq S}\{\w(I)\}$ is monotone (non-decreasing) submodular.
\end{lemma}




%\begin{lemma}[Correlation Gap] %\label{lem:correlationggap}
%Let $f$ be a monotone (non-decreasing) submodular function. For the sets $S$ and $T_1, \dots, T_k$ constructed as described above, we have:
%\begin{align*}
%    \Ex{S \sim \x}{f(S)} \geq \left(1 - \frac{1}{e}\right) \Ex{T \sim \I(\x)}{f(T)}.
%\end{align*}
%\end{lemma}

 
\section{Recurrent Submodular Welfare}
Let $f(S): 2^{\A} \rightarrow \mathbb{R}_{\geq 0}$ be a monotone submodular function over a universe $\A$ of $k$ elements, such that $f(\emptyset) = 0$. In the {\em blocking} setting, each element $i \in \A$ is associated with a known deterministic {\em delay} $d_i \in \mathbb{N}_{>0}$, such that once the arm is played at some round $t$, it becomes unavailable for the next $d_i-1$ rounds, namely, in the interval $\{t, \dots, t+d_i-1\}$. At each round $t \in [T]$, the player chooses a subset $\A_t$ of available (i.e., non-blocked) elements and collects a reward $f(\A_t)$. The goal is to maximize the total reward collected, i.e., $\sum_{t \in [T]} f(\A_t)$, within an unknown time horizon $T$. 

Before we present our algorithm, we provide ``bad'' instances for two natural approaches to \rsm.

\begin{remark} \label{rem:greedy}
The greedy approach of choosing $\A_t$ to be the set of all available elements at round $t \in [T]$ can be as bad as a $\frac{1}{k}$-approximation. In order to see that, consider the monotone (budget-additive) submodular function $f(S) = \min\{|S|, 1\}$. Let $k$ be the number of elements with delay $d_i = k$ for each $i \in \A$. Assuming an infinite time horizon, the optimal strategy collects an average reward of $1$, simply by choosing one element at a time in a round-robin manner. However, the average reward of the greedy approach in this case is $\frac{1}{k}$.
\end{remark}

\begin{remark}
The independent randomized sampling approach of adding each arm $i$ to $\A_t$ independently with probability $\frac{1}{d_i}$, if available, can be as bad as a $(1 - \frac{1}{\sqrt{e}} )$-approximation. Consider the same setting as in Remark \ref{rem:greedy}, where for $T \to \infty$ the optimal average reward is $1$. However, the average expected reward of the independent randomized sampling strategy is $1 - (1 - p)^k$, where $p = \frac{1}{2k-1}$ is the probability that each element is selected at each round (in stationarity). For $k \to \infty$, we have that $1 - (1 - p)^k \to 1- e^{-\frac{1}{2}} \approx 0.393$.

\end{remark}
We provide an efficient randomized $\left(1-\frac{1}{e}\right)$-approximation algorithm for \rsm. Informally, the algorithm starts by considering, for each element $i \in \A$, a sequence of rational numbers of the form $\{t\cdot \frac{1}{d_i}\}_{t \in [T]}$. Then, these sequences are {\em interleaved} by randomly adding an {\em offset} $r_i$, drawn uniformly at random from $[0,1]$, for each $i \in \A$ to the corresponding sequence. At every round $t \in [T]$, the algorithm chooses a set $\A_t$, consisting only of elements for which the (perturbed) interval $L_{i,t} = [t\cdot \frac{1}{d_i}+ r_i, (t+1)\cdot \frac{1}{d_i}+ r_i )$ contains an integer.

\begin{algorithm}[\is (\IS)]
For each element $i \in \A$, let $r_i \sim U[0,1]$ be a random {\em offset} drawn uniformly from $[0,1]$. 
At every round $t = 1, 2, \dots$,  let $\A_t \subseteq \A$ be the subset of elements such that for any $i \in \A_t$, the interval $L_{i,t} = [t\cdot \frac{1}{d_i} + r_i, (t+1) \cdot \frac{1}{d_i} + r_i)$ contains an integer. Choose the elements $\A_t$ and collect the reward $f(\A_t)$.
\end{algorithm}


\subsection{Correctness and approximation guarantee.} 
We first show the algorithm is correct, namely, that the elements chosen at each round respect the blocking constraints. The correctness is established by the following simple observation:

\begin{restatable}{fact}{restatefactalwaysavailable}\label{inter:fact:alwaysavailable}
At any $t \in [T]$, all the elements in $\A_t$ are available (i.e., not blocked).
\end{restatable}

In order to prove the competitive guarantee of our algorithm, we first construct a convex programming (CP)-based (approximate) upper bound on the optimal reward. Although our algorithm never computes an optimal solution to this CP, this step allows us to prove our guarantee, leveraging results on the correlation gap of submodular functions. For $\bm{d}^{-1} \in \mathbb{R}^k$ such that $(\bm{d}^{-1})_i = \frac{1}{d_i}, \forall i \in [k]$, consider the following formulation based on the concave closure $f^+$ of $f$:
\begin{align}
\maximize_{\z \in \mathbb{R}^k}~~ T \cdot f^+(\z)~~\textbf{s.t.}~~ \bm{0} \preceq \z \preceq \bm{d}^{-1}. \tag{\textbf{CP}} \label{cp:CP}
\end{align}

In \eqref{cp:CP}, each variable $z_{i}$ can be thought of as the fraction of rounds where element $i\in \A$ is chosen. Intuitively, the constraints indicate the fact that, due to the blocking, each element $i \in \A$ can be played at most once every $d_i$ steps. 
In order to derive \eqref{cp:CP}, we start from a non-convex integer program (IP) with 0-1 variables $\{x_{i,t}\}_{i \in \A, t \in [T]}$, each indicating whether element $i \in \A$ is used at round $t \in [T]$. The objective is to maximize $\sum_{t \in [T]} \sum_{S \subseteq \A} f(S) \prod_{i \in S} x_{i,t} \prod_{i \notin S}(1 - x_{i,t})$ subject to natural blocking constraints. For integral solutions, the above objective is equivalent to $\sum_{t \in [T]} f^+(\x_t)$ (where $(\x_t)_i = x_{i,t}$) and, thus, the above relaxation is simply the result of averaging over time the variables and constraints of this IP. By using the concavity of $f^+$, we are able to show that \eqref{cp:CP} yields an (approximate) upper bound on the optimal solution of \rsm, while the approximation becomes exact as $T$ increases.

\begin{restatable}{lemma}{restateStructuralCP}\label{lem:structural:CP}
Let $\Rew^{CP}(T)$ be the optimal solution to \eqref{cp:CP} and $\OPT(T)$ be the optimal solution over $T$ rounds. We have
$
\Rew^{CP}(T) \geq \OPT(T) - \mathcal{O}(d_{\max} f(\A)),
$ where $d_{\max} = \max_{i \in \A}\{d_i\}$.
\end{restatable}

\begin{remark}
By replacing $f^+(\z)$ in \eqref{cp:CP} with the multi-linear extension $F(\z)$, the formulation no longer yields an upper bound on the optimal reward (not even asymptotically). Indeed, consider a function $f$ over a ground set $\A=\{1,2\}$ with $d_1 = d_2 = 2$, such that $f(\emptyset) = 0$, $f(\{1\}) = f(\{2\}) = 2$ and $f(\{1,2\}) = 3$. For $T \to \infty$, the optimal average reward is $2$, simply by choosing the two elements interchangeably. However, the formulation based on $F(\z)$ in that case would be to maximize $2z_1(1-z_2) + 2z_2(1-z_1) + 3 z_1 z_2$ subject to $z_1,z_2 \leq \frac{1}{2}$, which has a global maximum of $\frac{7}{4} < 2$.
\end{remark}


Before we complete the proof of our first main result, we first compute the probability that $i \in \A_t$, i.e., an element $i \in \A$ is sampled at round $t \in [T]$:

\begin{restatable}{fact}{restatefactsampling}\label{inter:fact:sampling}
For any $i \in \A$ and $t \in [T]$, we have
$\Pro{i \in \A_t} = \Pro{L_{i,t} \cap \mathbb{N} \neq \emptyset } = \frac{1}{d_i}.
$
\end{restatable}

\noindent{\em Proof of Theorem \ref{thm:interleavedSubmodular}.} 
Let us denote by $S \sim {\bf p}$ with ${\bf p} \in [0,1]^k$ the random set $S \subseteq \A$, where each element $i$ participates in $S$ independently with probability equal to $p_i$. 
By Fact~\ref{inter:fact:sampling} and due to the randomness of the offsets $\{r_i\}_{i \in \A}$, we have that $\A_t \sim {\bf d}^{-1}$ for each $t \in [T]$. Let $\z^*$ be an optimal solution to \eqref{cp:CP}. By monotonicity of $f$ and the fact that $\z^* \preceq \bm{d}^{-1}$, for the expected value of $f(\A_t)$ at any round $t \in [T]$, we know that $\Ex{\A_t \sim \bm{d}^{-1}}{f(\A_t)} \geq \Ex{\A_t \sim \z^*}{f(\A_t)}$. Moreover, by definition of the multi-linear extension, we have that $\Ex{\A_t \sim \z^*}{f(\A_t)} = F(\z^*)$, while by Lemma~\ref{lem:correlationgap} (the correlation gap of submodular functions), we have that, $F(\z) \geq \left(1 - \frac{1}{e}\right)f^+(\z)$ for any vector $\z \in [0,1]^k$. By combining the above facts, we can see that
\begin{align*}
\Rew^{IS}(T) = \sum_{t \in [T]} \Ex{\A_t \sim \bm{d}^{-1}}{f(\A_t)} \geq 
\sum_{t \in [T]} F(\z^*) \geq \left(1 - \frac{1}{e}\right)T\cdot f^+(\z^*) = \left(1 - \frac{1}{e}\right) \Rew^{CP}(T).
\end{align*}
Therefore, by Lemma~\ref{lem:structural:CP}, we can conclude that $\Rew^{IS}(T) \geq \left(1 - \frac{1}{e}\right)\OPT(T) - \mathcal{O}(d_{\max} f(\A))$.
\qed
\newline

In Appendix \ref{appendix:hardness}, we provide a $\left(1-\frac{1}{e}\right)$-hardness result for \rsm, thus proving that the guarantee of Theorem~\ref{thm:interleavedSubmodular} is asymptotically tight. This result, which holds even for the special case where $d_{\max} = o(T)$ (that is when the delays are significantly smaller than the time horizon), is proved via a reduction from the SWM problem with identical utilities, in a way that the constructed \rsm instance accepts w.l.o.g. solutions of a simple periodic structure.

\begin{restatable}{theorem}{restateSubmodularHardness}\label{thm:submodular:hardness}
For any $\epsilon>0$, there exists no polynomial-time $\left(1-\frac{1}{e} + \epsilon \right)$-approximation algorithm for the \rsm problem, unless ${\bf P}={\bf NP}$, even in the special case where $d_{\max} = o(T)$.
\end{restatable}
\section{Matroid Blocking Bandits}
\label{section:mbb}
Let $\A$ be a set of $k$ arms and $T$ be an unknown time horizon. At any round $t \in [T]$ and for each $i \in \A$ a reward $X_{i,t}$ is drawn independently from an unknown distribution of mean $\mu_{i}$ and bounded support in $[0,1]$. Let $d_i \in \mathbb{N}_{>0}$ be the known determinisitc delay of each arm $i \in \A$, and $d_{\max} = \max_{i \in \A}\{d_i\}$. At any round $t \in [T]$, the player pulls any subset $\A_t$ of the available (i.e., non-blocked) arms, as long as it forms an {independent} set of a given {matroid} $\M = \left(\A,\I\right)$. The player only observes the realized reward of each arm she plays and collects their sum. The goal is to maximize the {\em expected cumulative reward} collected within $T$ rounds, denoted by
$\Rew^{IG}(T) = \Ex{ }{\sum_{t \in [T]} \sum_{i \in \A} X_{i,t} \event{i \in \A_t}}$. 

\subsection{The full-information setting}

The following algorithm is the implementation of \IS in the special case of the full-information \mbb setting, where the mean rewards $\{\mu_{i}\}_{i \in \A}$ are known to the player a priori:


\begin{algorithm}[\ig (\IG)]
For each arm $i \in \A$, let $r_i \sim U[0,1]$ be a random {\em offset} drawn uniformly from $[0,1]$. 
At every round $t = 1, 2, \dots$,  let $\G_t \subseteq \A$ be the subset of arms $i \in \A$, such that the interval $L_{i,t} = [t\cdot \frac{1}{d_i} + r_i, (t+1) \cdot \frac{1}{d_i} + r_i)$ contains an integer. Greedily compute a maximum independent set $\A_t$ of $\M|\G_t$ with respect to $\{\mu_i\}_{i \in \G_t}$ and play these arms.
\end{algorithm}

The correctness of the above algorithm follows directly by Fact~\ref{inter:fact:alwaysavailable} (that is, the sampled arms are never blocked) and by the fact that \IG always plays an independent set of $\M$, i.e., $\A_t \in \I_{\M|\G_t} \subseteq \I$. The approximation guarantee of \IG follows immediately as a special case of Theorem~\ref{thm:interleavedSubmodular}. Indeed, notice that: (i) The reward realizations do not affect the choices of \IG or any optimal algorithm maximizing the total expected reward. Thus, each realization $X_{i,t}$ can be replaced w.l.o.g. by its expected value $\mu_i$. (ii) The value of the greedily computed maximum independent set in $\M|\G_t$ corresponds to the weighted rank function $f_{\M, \mu}(\G_t)$ which, by Lemma~\ref{lem:weightedrank}, is monotone submodular.

\begin{restatable}{theorem}{restateinterleavedGreedy}\label{thm:interleavedGreedy}
The expected reward collected by \ig for $T$ rounds, $\Rew^{IG}(T)$, is at least
$
\left(1 - \frac{1}{e} \right) \OPT(T) - \mathcal{O}(d_{\max} \rk(\M))
$, where $\OPT(T)$ is the optimal expected reward.
\end{restatable}

As a point of interest, in Appendix \ref{appendix:mbb:fullinformation} we provide an alternative proof of the above theorem, which, instead of the concave closure of the weighted rank, now relies on the following (approximate) LP upper bound, based on the matroid polytope. For any set $S \subseteq \A$, let $\z(S) = \sum_{i \in S} z_i$.
\begin{align}
&\maximize_{\z \in \mathbb{R}^k} ~T \cdot \sum_{i \in \A} \mu_{i} z_{i} \tag{\textbf{LP}} \quad 
\textbf{s.t. } \z(S) \leq \rk(S)~ \forall S\subseteq \A
\textbf{, } \bm{0} \preceq \z \preceq \bm{d}^{-1}.
\end{align}

\begin{remark}
The analysis of $\IG$ is tight for rank-1 matroids. Indeed, consider $k$ arms, each of delay $k$ and deterministic reward equal to $1$. For $T \to \infty$, the optimal average reward is equal to $1$, simply by playing the arms in a round-robin manner. However, the probability that at least one arm is sampled at some round $t$ is equal to $\sum^k_{i = 1} {k \choose i} \left(\frac{1}{k}\right)^i \left(1 - \frac{1}{k}\right)^{k - i} = 1 - \left(1 - \frac{1}{k}\right)^k \to 1 - \frac{1}{e}$ as $k \to \infty$.
\end{remark}

\subsection{The bandit setting and regret analysis}

In the setting where the mean rewards are initially unknown, we develop a UCB-based bandit algorithm, \ucb (\UCB). The algorithm is identical to \IG, except for the greedy computation of the maximum independent set over the sampled arms, which is now performed using estimates. Specifically, the algorithm maintains for every $i \in \A$, $t \in [T]$ the following upper estimate of $\mu_i$:
\vspace{-0.5em}
\begin{align*}
\bar{\mu}_{i,t} = \hat{\mu}_{i,T_{i}(t)} + c_{i,t}\text{  with  } c_{i,t} = \sqrt{\frac{2 \ln{(t)}}{T_{i}(t)}},
\end{align*}
where $T_{i}(t)$ denotes the number of times arm $i$ has been played at the beginning of round $t$ and $\hat{\mu}_{i,T_{i}(t)}$ denotes the empirical average of the $T_{i}(t)$ i.i.d. samples from its reward distribution. The term $c_{i,t}$ is the {\em confidence length} around $\hat{\mu}_{i,T_{i}(t)}$ that guarantees $\bar{\mu}_{i,t}$  lies in $[\mu_i, \mu_i + 2c_{i,t}]$ with high probability. Note that all the above quantities are random variables depending on the random offsets and the observed reward realizations.

We are interested in upper bounding the $\alpha$-regret, for $\alpha = 1- \frac{1}{e}$, namely, the difference between $\alpha \OPT(T)$ and the expected reward collected by \UCB. Due to the complex time dynamics, characterizing the optimal expected reward as a function of the instance is hard. However, using Theorem~\ref{thm:interleavedGreedy} we can upper bound $\alpha \OPT(T)$ by the expected reward collected by \IG, thus giving:
\begin{align} \label{eq:regret:twoalgos}
\alpha \OPT(T) - \Rew^{UCB}(T)  \leq \Rew^{IG}(T) - \Rew^{UCB}(T) + \mathcal{O}(d_{\max} \cdot \rk(\M)).
\end{align}
By the above inequality, it becomes clear that in order to upper bound the regret, it suffices to bound the difference between the expected reward collected by \IG and \UCB. This difference not only depends on the reward realizations (through the UCB estimates), but also on the trajectory of sampled arms in each algorithm, which is itself a function of the random offsets. However, by construction of our interleaved scheduling scheme, these offsets are sampled at the initialization phase of each algorithm and are identically distributed. Thus, the trajectories of sampled arms in the two algorithms exhibit a coupled evolution. This allows us to analyse the regret ``pointwise'', under the assumption that the sequences of sampled arms are identical throughout the time horizon. To make this idea precise, let $\off^{\pi} \in [0,1]^k$ be the random offsets used and let $\{\G^{\pi}_t(\off^{\pi})\}_{t \in [T]}$ be the sequence of sampled arms by algorithm $\pi \in \{\IG, \UCB\}$. Using (henceforth) $\mathcal{R}$ to denote the randomness due to the reward realizations of the arms, the next lemma gives our pointwise regret bound.


\begin{restatable}{lemma}{restateRegretCoupling}
\label{lem:regret:coupling} Let $\bar{\mu}_t(S) = \sum_{i \in S} \bar{\mu}_{i,t}$ and $\mu(S) = \sum_{i \in S} \mu_i$. We have
\begin{align*}
    \Rew^{\IG}(T) - \Rew^{\UCB}(T) =  \Ex{\off \sim U[0,1]^k, \mathcal{R}}{\sum_{t \in [T]}\left( \max_{S \subseteq \G_t(\off), S \in \I} \{ \mu\left(S\right)\} - \mu\left(\arg\max_{S \subseteq \G_t(\off), S \in \I}\{\bar\mu_t(S)\}\right)\right)}.
\end{align*}
\end{restatable}

Thus w.l.o.g., we focus on the case where the sequences of sampled arms are identical. Let  $\mathcal{E}_{\off}$ denote the event that both algorithms, \IG and \UCB, sample the same offset vector $\off$, namely, $\off^{\IG} = \off^{\UCB} = \off$. Assuming that $\mathcal{E}_{\off}$ holds for some $\off \in [0,1]^k$, let $\{\G_t\}_{t \in [T]} = \{\G_t(\off)\}_{t \in [T]}$ be the sequence of sampled arms, common in both algorithms. Clearly, \UCB accumulates regret only when it plays independent sets of arms that are suboptimal w.r.t.\ the true means, i.e., when $\mu(\A^{\UCB}_t) < \mu(\A^{\IG}_t)$ for some $t \in [T]$. We assume w.l.o.g.\ that the arms are indexed in decreasing order of mean rewards and that these mean rewards are distinct. We now formally define the gaps related to our analysis:

\begin{definition}[Gaps]\label{def:gaps}
For any subset $S \subseteq \A$ and reward vector $\nu \in \mathbb{R}^k$, we define 
\begin{align*}
\Delta_S(\nu) = \max_{I \in \I, I \subseteq S}\{\mu\left(I\right)\}  - \mu\left(\arg\max_{B \in \I, B \subseteq S}\{\nu\left(B\right)\}\right).
\end{align*}
Moreover, let $\Delta_{i,j} = \mu_i - \mu_{j}$ be the standard suboptimality gap between two arms $i,j \in \A$.
\end{definition}

By Lemma \ref{lem:regret:coupling} and assuming that the event $\mathcal{E}_{\off}$ holds for some $\off$, we are interested in bounding the expectation of $\sum_{t \in [T]} \Delta_{\G_t(\off)}(\bar\mu_t)$ w.r.t.\ the reward realizations. The next step is to decompose the suboptimality of \UCB by noticing that both algorithms play, at each round $t \in [T]$, a basis of $\M|\G_t$ and thus $|\A^{\IG}_t| = |\A^{\UCB}_t|$. We use the following fundamental property of matroids:

\begin{theorem}[Strong Basis Exchange, Corollary 39.12a in \cite{schrijver03}]\label{cor:basesexchange} Let $\M = (\A,\I)$ be a matroid and $I_1, I_2 \in \I$ be two independent sets such that $|I_1| = |I_2|$. Then, there exists a bijection $\sigma : I_1 \rightarrow I_2$, such that for any $i \in I_1$ the set $I_1 - i + \sigma(i)$ is an independent set of $\M$. 
\end{theorem}

Let $\sigma_t: \A^{\UCB}_t \rightarrow \A_t^{\IG}$ for each $t \in [T]$ be the bijection described in Theorem~\ref{cor:basesexchange} with respect to the sets $\A^{\UCB}_t$ and $\A^{\IG}_t$ and let $\sigma_t^{-1}$ be its inverse mapping. 
Note that in any bijection $\sigma_t$ and any $i \in \A^{\UCB}_t \cap \A^{\IG}_t$ we can assume w.l.o.g. that $\sigma_t(i) = i$. Notice, further, that under the event $\mathcal{E}_{\off}$, the bijections $\{\sigma_t\}_{t \in [T]}$ are still random variables that depend on the observed realizations. 

\begin{restatable}{lemma}{restateRegretSub} \label{lem:regret:suboptdecomp} Under the event $\mathcal{E}_{\off}$ and at any time $t \in [T]$, we have 
$\Delta_{\G_t}(\bar{\mu}_t) = \sum_{i \in \A^{\IG}_t} \Delta_{i , \sigma^{-1}_{t}(i)}$.
\end{restatable}
Conditioned on the fact that both algorithms operate on the same sequence $\{\G_t\}_{t \in [T]}$ of sampled arms, Lemma \ref{lem:regret:suboptdecomp} allows us to decompose the suboptimality gap $\Delta_{\G_t}(\bar{\mu}_t)$ of each round $t \in [T]$, into simpler gaps of the form $\Delta_{i,j}$ between any arms $i \in \A^{\IG}_t$ and $j \in \A^{\UCB}_t$ that are perfectly matched according to the bijection $\sigma_t$, namely, $\sigma_t(j) = i$. Assuming that the event $\{\sigma_t(j) = i\}$ directly implies that $i \in A^{\IG}_t$ and $j \in A^{\UCB}_t$, we can further upper bound the regret as
\begin{align*}
\sum_{t \in [T]} \Delta_{\G_t}(\bar{\mu}_t) = \sum_{t \in [T]} \sum_{i \in \A^{\IG}_t} \Delta_{i , \sigma^{-1}_{t}(i)} \leq \sum_{t \in [T]} \sum_{i \in \A^{\IG}_t} \sum_{j \in \A, \Delta_{i,j} > 0} \Delta_{i,j} \event{\sigma_t(j)=i}.
\end{align*}

The above inequality allows us to study the regret attributed to each arm independently, using more standard arguments for UCB-based algorithms in combination with Theorem~\ref{cor:basesexchange}. Specifically, for every pair of arms $i,j \in \A$ with $i < j$ (thus, $\Delta_{i,j} > 0$), we define a threshold $\ell_{i,j}$ with the following key-property: After \UCB ``exchanges'' arm $j$ for arm $i = \sigma_t(j)$ more than $\ell_{i,j}$ times, due to insufficient exploration, then it has collected enough samples to infer that $\mu_j < \mu_i$ with high probability. 
\begin{restatable}{lemma}{restateRegretTechnical} \label{lem:regret:technical}
Let $\ell_{i,j} = \bigg\lfloor \frac{8 \ln(T)}{\Delta^2_{i,j}}\bigg\rfloor$ for any $i<j$. Under event $\mathcal{E}_{\off}$ and for any arm $j>1$, we have
\vspace{-1em}
\begin{align}
&\sum_{t \in [T]} \sum_{i < j} \Delta_{i , j} \event{\sigma_t(j)=i, T_j(t) \leq \ell_{i,j}} \leq \frac{16}{\Delta_{j-1,j}} \ln( T) &\quad\text{(Under-sampled regret)} \label{inq:reg:under}\\
& \Ex{\mathcal{R}}{\sum_{t \in [T]} \sum_{i < j} \Delta_{i , j} \event{\sigma_t(j)=i, T_j(t) > \ell_{i,j}}} \leq \frac{\pi^2}{3}\sum^{j-1}_{i=1} \Delta_{i,j} &\quad\text{(Sufficiently sampled regret)} \label{inq:reg:sufficiently}
\end{align}
\end{restatable}


\begin{proof}[Proof sketch of Theorem \ref{thm:approxregret}] 
By inequality \eqref{eq:regret:twoalgos} and Lemma \ref{lem:regret:coupling}, in order to bound the regret of \UCB, it suffices to upper bound the difference between $\Rew^{\IG}(T)$ and $\Rew^{\UCB}(T)$, conditioned on the fact that both algorithms use exactly the same offset vector $\off$ and, thus, they operate on the exact same sequence of sampled arms, denoted by $\{\G_t\}_{t \in [T]}$. By construction, \IG plays at any round $t \in [T]$ a basis of $\M|\G_t$ of maximum expected reward, while \UCB plays a basis of $\M|\G_t$ that is maximum with respect to the estimates $\{\bar{\mu}_{i,t}\}_{i \in \A}$. By Theorem~\ref{cor:basesexchange}, we can consider a perfect matching between exchangeable arms of $\A^{\IG}_t$ and $\A^{\UCB}_t$ and, thus, to decompose the regret into suboptimality gaps between individual arms. Then, using Lemma~\ref{lem:regret:technical}, we can upper bound on the expected regret due to the fact that \UCB erroneously plays arm $j$ instead of arm $i$, when $\Delta_{i,j} > 0$. The above analysis culminates in a regret bound that is a function of $\{\Delta_{i,j}\}_{i,j\in \A}$. In order to derive a gap-independent regret bound, we partition the gaps into ``small'' and ``large'' and notice that any pair of arms $i,j \in \A$ with $\Delta_{i,j} < \Theta(\sqrt{\frac{\ln(T)}{T}})$ cannot contribute more than $\sqrt{T\ln(T)}$ loss in the regret.
\end{proof}


\begin{comment}
\begin{figure}
\includegraphics[width=\linewidth]{figs/beyond_tss_lesion.pdf}
\caption[]{End-to-End runtime lesion study of the entire MNIST dataset and the FMA featurized music dataset. Each of DROP's contributions provides a runtime improvement.}
\label{fig:beyond_lesion}
\end{figure}
\end{comment}



\section{Conclusion}
\label{sec:conclusion}

Advanced data analytics techniques must scale to rising data volumes. 
DR techniques offer a powerful toolkit when processing these datasets, with PCA frequently outperforming popular techniques in exchange for high computational cost. 
In response, we propose DROP, a new dimensionality reduction optimizer. 
DROP combines progressive sampling, progress estimation, and online aggregation to identify high quality low dimensional bases via PCA without processing the entire dataset by balancing the runtime of downstream tasks and achieved dimensionality. 
Thus, DROP provides a first step in bridging the gap between quality and efficiency in end-to-end DR for downstream \red{analytics}. 

%We revisit canonical operators for time series dimensionality reduction and the measurement study of~\cite{keogh-study}, and show that PCA is more effective than popular alternatives in the data mining literature often by a margin of over $2\times$ on average on gold-standard time series benchmark data sets with respect to output data dimension. More surprisingly, we empirically demonstrate that a small number of samples are sufficient to accurately characterize directions of maximum variance and obtain a high-quality low-dimensional transformation.



\thispagestyle{empty}

\chapter*{Words of Thanks}
%\begin{center}
%{\huge {\bf Words of thanks}}
%\end{center}
\addcontentsline{toc}{chapter}{Word of thanks}


It is always hard to thanks everyone that might helped me to reach this point of my career or
helped me to finish this thesis. Much probably some important person will be forgotten,
and for those (unfairly) forgotten people I promptly apologize.

At first place, I would like to thank my advisor Silvio~P.~Sorella. Thank you, Silvio. You did
inspire me to work in QFT and to always go as deep as possible in the world of non-perturbative
QCD. Thank you for the very long, fruitful and exciting meetings in your office (if it would
have a title, it would be ``The endless river''), and for the coffees after all. 

Dear Professor Dr. David~Dudal, thank you for everything you have made for me (and my wife) once
we have been in Belgium. You spared no efforts to make us feel at home: that was a marvelous
time, professionally and personally. Actually, it is really tough to separate the advisor
from the friend. I wish all the possible happiness to you and your family.

Thank you Prof. Marcio~Capri for have accepted be my co-advisor, for always be
accessible to discuss with me and for trying to teach me the art of Algebraic
Renormalization (yes, it is not a technique, it is an art) --- sorry, I will never reach your
high level of expertise. Thank you Prof. Marcelo Guimar\~aes, Prof. Bruno Mintz and Prof.
Leticia Palhares, which are not my officially advisors, but (lucky me!) were always there to
advise me and to have enlightening discussions with me. Thank you, each one of you, for making
pleasant our long meetings: not everyone is lucky enough to work with friends, as I was.

And talking about friends, thank you Luiz Gustavo, Gustavo Vicente, Ricardo Rodrigues (``o
Tol'') for turning easier, funnier and enjoyable those four years I did spend at UERJ. There
is no doubt that times of joy are fundamental for performing a good work. Equally, I would like
to thank Thomas Mertens. I know it was not easy for him to bear two Brazilian colleagues at
the same time in his office, at UGhent (it is not a fair thing to do with any Belgian). And
finally, to my great friend Diego R. Granado, thank you so much. We did start it together, we
will finish it together. Thank you.

I would like to thank CAPES for financially supporting me during these four years. It may not
seems to be, but physicists also need to eat.

\`A minha fam\'ilia, minha m\~ae, minha irm\~a e meu irm\~ao, agrade\c{c}o do fundo do meu
cora\c{c}\~ao, por voc\^es estarem sempre ao meu lado me dando suporte e muito carinho. Amo
todos voc\^es.

Por final, agrade\c{c}o \`a minha neguinha, Juliana V. Ford. Sem ela nada disso seria
poss\'ivel, ou mesmo faria sentido. Dedico a voc\^e esta tese.

Greetings,\\
\indent Igor F. Justo

\thispagestyle{empty}

\newpage

\thispagestyle{empty}
\ 


\bibliographystyle{plainurl}
\bibliography{ref.bib}

\newpage
\appendix
\section{Technical Notation} \label{appendix:notation}

For any event $\mathcal{E}$, we denote by $\event{\mathcal{E}} \in \{0,1\}$ the indicator variable such that $\event{\mathcal{E}} = 1$, if $\mathcal{E}$ occurs, and $\event{\mathcal{E}} = 0$, otherwise. For any non-negative integer $n \in \mathbb{N}$, we define $[n] = \{1,2, \dots, n\}$. For any vector $\mu \in \mathbb{R}^k$ and set $S \subseteq [k]$, we define $\mu(S) = \sum_{i \in S} \mu_i$. Moreover, we use the notation $t \in [a, b]$ (for $a \leq b$) for some time index $t$, in place of $t \in [T] \cap [a, \ldots , b\}$. Unless otherwise noted, we use the indices $i$, $j$ or $i'$ to refer to arms and $t$, $t'$ or $\tau$ to refer to time. Let $\A^{\pi}_t \in \I$ be the set of arms played by some algorithm $\pi \in \{\IS, \IG, \UCB\}$ at time $t$. Unless otherwise noted, all expectations are taken over the randomness of the offsets $\{r_i\}_{i \in [k]}$ and the reward realizations.

\section{Concentration inequalities}
\begin{theorem}[Hoeffding's Inequality \cite{Hoeffding}]\label{appendix:concentration:hoeffding}
Let $X_1, \dots, X_n$ be independent identically distributed random variables with common support in $[0,1]$ and mean $\mu$. Let $Y = X_1 + \dots + X_n$. Then for any $\delta \geq 0$,
\begin{align*}
    \Pro{Y-n\mu \geq \delta} \leq e^{-2\delta^2/n} \text{   and   }\Pro{Y-n\mu \leq -\delta} \leq e^{-2\delta^2/n}.
\end{align*}
\end{theorem}
\section{Recurrent Submodular Welfare: Omitted Proofs}
\subsection{Correctness and approximation guarantee}

\restatefactalwaysavailable*
\begin{proof}
Recall that at any round $t \in [T]$, the algorithm only chooses a subset $\A_t$ of the elements. Consider any element $i \in \A$ such that $i \in \A_t$ for some $t \in [T]$. By definition of $\A_t$, the interval $L_{i,t} = [t\cdot \frac{1}{d_i} + r_i, (t+1) \cdot \frac{1}{d_i} + r_i)$ contains an integer. It is not hard to see that, in that case, none of the intervals $L_{i,t'}$ for $t' \in [t-d_{i}+1, d_i - 1]$ can contain an integer. Therefore, the last time element $i$ has been chosen must be before $t- d_i$, which implies feasibility with respect to the blocking constraints.
\end{proof}



\restatefactsampling*

\begin{proof}
For any fixed $i \in \A$ and $t \in [T]$, because of the fact that $\frac{1}{d_i} \leq 1$ and $r_i \in [0,1]$, the interval $L_{i,t} = [t\cdot \frac{1}{d_i} + r_i, (t+1) \cdot \frac{1}{d_i} + r_i)$ clearly contains at most one integral point. The event that $\{[t\cdot \frac{1}{d_i} + r_i, (t+1) \cdot \frac{1}{d_i} + r_i)\cap \mathbb{N} \neq \emptyset \}$ is equivalent to the event that a continuous window of size equal to $\frac{1}{d_i}$ starting from the (real) point $t\cdot \frac{1}{d_i} + r_i$ contains an integer. For $r_i$ ranging in $[0,1]$, the starting point of the interval lies between $t\cdot \frac{1}{d_i}$ and $t\cdot \frac{1}{d_i} + 1$. It is not hard to see that fraction of possible realizations of $r_i$ such that the window contains an integer equals its size. The fact follows since for any $i \in \A$, the window has size $\frac{1}{d_i}$ and the offset $r_i$ is sampled uniformly at random from $[0,1]$.
\end{proof}



\restateStructuralCP*
\begin{proof}
In order to prove the lemma, we first construct an (non-convex) IP upper bound on the optimal expected reward over $T$ rounds, based on the multi-linear extension of $f$.

\begin{align}
\textbf{maximize:}& \sum_{i \in [T]} \sum_{S \subseteq \A} f(S) \prod_{i \in S} x_{i,t} \prod_{i \notin S} (1-x_{i,t}) \tag{\textbf{MP}} \label{mp:MP}\\
\textbf{s.t.}& \sum_{t' \in [t,t+d_i-1]} x_{i,t'} \leq 1, \forall i \in \A, \forall t \in [T] \label{mp:window} \\
\qquad &\x_{t} \in \{0,1\}^k, \forall t \in [T] \notag
\end{align}

In the formulation \eqref{mp:MP}, each variable $x_{i,t}$ can be thought of as the 0-1 indicator of playing arm $i\in \A$ at time $t \in [T]$. Intuitively, constraints \eqref{mp:window} of \eqref{mp:MP} indicate the fact that, due to blocking constraints, each arm $i \in \A$ can be played at most once every $d_i$ steps. Clearly, any optimal solution to \rsm can be mapped onto the above formulation and, thus, the optimal solution of \eqref{mp:MP} provides an upper bound on $\OPT(T)$.


Let $\x_t \in \{0,1\}^k$ for each $t \in [T]$ be a vector such that $(\x_t)_i = x_{i,t}$. Notice that for any integral $\x \in \{0,1\}^k$, the multi-linear extension is equal to the concave closure of any set function $f$, that is, $f^+(\x) = F(\x)$. Therefore, \eqref{mp:MP} remains an upper bound, even if we replace its objective function with $g(\x_1, \dots, \x_T) = \sum_{t \in [T]} f^+(\x_t)$.

We now fix any optimal solution $\{x^*_{i,t}\}_{i\in \A, t \in [T]}$ to \eqref{mp:MP} under the objective $g(\x_1, \dots, \x_T) = \sum_{t \in [T]} f^+(\x_t)$. Let us define the variables $\{z'_{i}\}_{i \in \A}$, such that
\begin{align*}
    z'_i = \frac{1}{T} \sum_{t \in [T]} x^*_{i,t} \geq 0, \quad \forall i \in \A.
\end{align*}
In the above definition, each $z'_i$ is the fraction of time an element $i \in \A$ is chosen in an optimal solution. Let $\z' \in [0,1]^k$, such that $(\z')_i = z'_i$ $\forall i \in \A$.

By concavity of $f^+$, we have
\begin{align*}
g(\x^*_1, \dots, \x^*_T) = \sum_{t \in [T]} f^+(\x^*_t) = T \sum_{t \in [T]} \frac{1}{T} f^+(\x^*_t) \leq T f^+(\frac{1}{T}\sum_{t \in [T]}\x^*_t) = T f^+(\z'),
\end{align*}
where the inequality follows by the fact that $\z'$ can be thought of as a convex combination of $\{\x^*_1, \dots, \x^*_T\}$.


Moreover, for each $i \in \A$ and by averaging constraints \eqref{mp:window} of \eqref{mp:MP} over all $t \in [T]$, we can see that 
\begin{align*}
\frac{1}{T}\sum_{t \in [1,d_i-1]} t x^*_{i,t} + \frac{1}{T} \sum_{t \in [d_i,T]} d_i x^*_{i,t} \leq 1 \Leftrightarrow \frac{1}{T} \sum_{t \in [T]} d_i x^*_{i,t} \leq 1 + \frac{1}{T}\sum_{t \in [1,d_i-1]} (d_i - t) x^*_{i,t}. 
\end{align*}
Given the fact that $\sum_{t \in [1, d_i-1]} x^*_{i,j} \leq 1$, the above inequality immediately implies that
\begin{align*}
    z'_i \leq \frac{1}{d_i}\left(1 + \frac{d_i-1}{T} \right)\quad \forall i \in \A.
\end{align*}
Consider now the assignment $z_i = \left(1 + \frac{d_{\max}-1}{T} \right)^{-1} z'_i$, $\forall i \in \A$. For this assignment, we can easily verify that the constraints of \eqref{cp:CP} are trivially satisfied, since $0 \leq z_i \leq \frac{1}{d_i}$, $\forall i \in \A$.

Let $\z \in [0,1]^k$, such that $(\z)_i = z_i$ $\forall i \in \A$. By the above analysis, we can see that
\begin{align*}
    \z = \z' - \frac{d_{\max}-1}{T + d_{\max} -1} \z',
\end{align*}
where we use the fact that $\frac{1}{1+\beta} = 1 - \frac{\beta}{1+ \beta}$ for any $\beta \in \mathbb{R}$.
Finally, by concavity of $f^+$ we have
\begin{align*}
    f^+(\z) &= 
    f^+\left(\left(1- \frac{d_{\max}-1}{T + d_{\max} -1} \right)\z' + \frac{d_{\max}-1}{T + d_{\max} -1} \bm{0} \right)\\
    &\geq \left(1- \frac{d_{\max}-1}{T + d_{\max} -1} \right)f^+(\z') + \frac{d_{\max}-1}{T + d_{\max} -1} f^+(\bm{0})\\
    &\geq f^+(\z') - \frac{d_{\max}-1}{T + d_{\max} -1} f(\A),
\end{align*}
where the last inequality follows by the facts that $f^+(\bm{0}) = f(\bm{0}) = 0$ and $f^+(\z') \leq f^+(\bm{1}) = f(\A)$, since $f$ is monotone.


Therefore, by exhibiting a feasible solution $\z$ of \eqref{cp:CP} such that 
$$
T f^+(\z) \geq T f^+(\z') - \mathcal{O}(d_{\max} f(\A)) \geq g(\x^*_1, \dots, \x^*_T) - \mathcal{O}(d_{\max} f(\A)) \geq \OPT(T) - \mathcal{O}(d_{\max} f(\A)),
$$
the proof is completed.
\end{proof}




\subsection{Hardness of approximation}
\label{appendix:hardness}
The goal of this section is to show that the $\left(1-\frac{1}{e}\right)$-multiplicative factor in the approximation guarantee of Theorem~\ref{thm:interleavedSubmodular} cannot be improved, unless $\textbf{P} = \textbf{NP}$. Specifically, we prove the following result:

\restateSubmodularHardness* 

In order show the above hardness result, we study for simplicity the average version of \rsm, where the objective is to maximize the average reward over $T$ time steps, namely, $\frac{1}{T}\left(\sum_{t \in [T]} f(\A_t)\right)$, where $\A_t$ is the set of elements used at time $t \in [T]$. Notice that in the average case, the additive term in the approximation guarantee of \ig, as presented in Theorem~\ref{thm:interleavedSubmodular}, vanishes as $T \to \infty$. Let $\OPT$ be the average reward collected by any optimal algorithm for \rsm. 

Our proof relies on a reduction from the Submodular Welfare (SW) problem~\cite{Von08}, in the special case where the players have identical utility functions. The problem can be formally defined as follows:

\begin{definition}[Submodular Welfare with Identical Utilities (SWIU)]
We consider a set of $k$ items and $m$ players, each associated with the same monotone submodular utility function $u:2^{[k]} \rightarrow \mathbb{R}_{\geq 0}$ over the items. The goal is to partition the $k$ items into $m$ subsets $S_1,\dots, S_m$, such that to maximize $\sum_{i \in [m]}u(S_i)$.
\end{definition}

As noted in~\cite{Von08}, the hardness result presented in~\cite{KLMM05} for the SW problem also holds for SWIU, namely, the special case of SW where all the players have the same utility function. Note, also that the \rsm problem is defined in the {\em value oracle} model, as we are only allowed to make queries of the function value for any input set.

\begin{theorem}[\cite{KLMM05}]\label{thm:hardness:swiu}
For any $\epsilon > 0$, there exists no polynomial-time $\left(1 - \frac{1}{e} + \epsilon\right)$-approximation algorithm for the SWIU problem in the value oracle model, unless ${\bf P} = {\bf NP}$.
\end{theorem}

We start from a simple construction for the non-average case of \rsm in order to show how our problem is directly associated with SWIU: Consider an instance of SWIU of $k$ items and $m$ players. Let $u:2^{[k]} \rightarrow \mathbb{R}_{\geq 0}$ be the monotone submodular utility function which is commonly used by all players. Given the above instance, we can construct in polynomial time an instance of \rsm as follows: Let $\A$ be the set of $k$ elements, each corresponding to an item, and let $f:2^{\A} \rightarrow \mathbb{R}_{\geq 0}$ be our function, chosen such that $f \equiv u$. We set the delay of each element $i \in \A$ as well as the time horizon to be equal to the number of players, namely, $ d_i = T  = m$ for each $i \in \A$.

Clearly, in the above construction where the delays are all equal to the time horizon, each element can be chosen at most once by any algorithm for \rsm. Therefore, the above constructed instance of \rsm exactly corresponds to SWIU, given that any solution to latter immediately translates into a solution of \rsm of the same total reward, and the opposite. 

The above construction immediately relates the two problems in the case where the delays can be of the same order as the time horizon. However, it does not rule out the possibility that the \rsm problem might become easier in the special case where $d_{\max} = o(T)$. Indeed, one could argue that for small enough delays, exploiting the possible periodicity of the \rsm solutions might lead to improved approximation guarantees. Notice, further, that the approximation guarantee we provide in Theorem~\ref{thm:interleavedSubmodular} for \IS becomes meaningless in the above scenario, since the additive loss for $d_{\max} = T$ becomes $\mathcal{O}(T\cdot f(\A))$.


In order to overcome the above technical issue and show that the multiplicative factor of $\left(1 - \frac{1}{e}\right)$ in Theorem~\ref{thm:interleavedSubmodular} cannot be improved, we map any instance of SWIU onto an instance of \rsm such that $T \gg d_{\max}$. Given any instance of SWIU, we can construct in polynomial time an instance of \rsm as follows: We define $\A$ to be the set of $k$ items, $f \equiv u$ to be the monotone submodular function and $d_i = m$ $\forall i \in \A$ to be the delay of all elements. In this case, we consider a time horizon $T = m \cdot \lceil\text{poly}(k,m) \rceil$, where by $\text{poly}(k,m)$ we denote some polynomial function in $k$ and $m$. 

We first show that, without loss of generality, we can focus our attention on solutions to the average case of \rsm that exhibit a periodic structure of period $m$.

\begin{lemma}\label{lem:hardness:periodic}
Let $\nu: [T] \rightarrow 2^{\A}$ be any feasible assignment to the above instance of \rsm of average reward $R$. We can construct in polynomial time a feasible assignment $\nu': [T] \rightarrow 2^{\A}$ of average reward at least $R' \geq R$, such that $\nu'(t) = \nu(t + m)$ $\forall t \in \mathbb{N}$, namely, $\nu'$ is a periodic assignment of period $m$.
\end{lemma}
\begin{proof}
Given that the average reward of the assignment $\nu$ is $R$, there must exist a continuous subsequence of rounds of length $m$, that is, $\{t, \dots, t + m -1\}$ for some $t \in [T-m]$, such that
\begin{align*}
    \frac{1}{m}\sum^{t+m-1}_{\tau = t} f(\nu(t)) \geq R.
\end{align*}
In the opposite case, we immediately get a contradiction to the fact that the average reward is at least $R$.

Let $L$ with $|L| = m$ be such a sequence. We now construct the periodic assignment $\nu'$ by repeating the assignment of the subinterval $L$, as follows:
\begin{align*}
\nu'(t) = \nu(L(t \mod m)) \in 2^{\A} ~\forall t \in [T].  
\end{align*}
It is not hard to verify that since $d_i = m$ for each $i \in \A$ and since $L$ is a subsequence of a feasible assignment of length $m$, the assignment $\nu'$ never violates the blocking constraints. Moreover, the average reward of $\nu'$ equals the average reward of the interval $L$ which is at least $R$. Finally, notice that the subsequence $L$ can be found in polynomial time, given the fact that the time horizon $T$ is defined to be polynomial in $k$ and $m$.
\end{proof}

We can now complete the proof of our hardness result. 
\newline

\noindent{\it Proof of Theorem~\ref{thm:submodular:hardness}.}
We prove the result via a reduction from the SWIU problem to the average version of the \rsm. Clearly, the average and non-average version of \rsm share the same approximability status, as the two problems are essentially identical up to a scaling of the objective function. 

Given an instance $I$ of SWIU, we can construct in polynomial time an instance $I'$ of the average version of \rsm, as described above. Let $\OPT_{SWIU}(I)$ and $\OPT_{\rsm}(I')$ be the optimal solution of SWIU and \rsm on the corresponding instance, respectively. 

We first show that when $\OPT_{SWIU}(I) \geq R$ for some reward $R$, then we necessarily have that $\OPT_{\rsm}(I') \geq \frac{R}{m}$. Indeed, let $L:[m] \rightarrow 2^{[k]}$ be an allocation that achieves a reward $R' = \OPT_{SWIU}(I) \geq R$ for the instance $I$ of SWIU. As indicated in proof of Lemma~\ref{lem:hardness:periodic}, we can construct in polynomial time a periodic assignment for the \rsm problem of average reward exactly $\frac{R'}{m}$, which implies that $\OPT_{\rsm}(I') \geq \frac{R'}{m} \geq \frac{R}{m}$.

Now, we would like to show that if $\OPT_{SWIU}(I) \leq \alpha R$ for some reward $R$ and $\alpha \in (0,1)$, then it has to be that $\OPT_{\rsm}(I') \leq \alpha \frac{R}{m}$. We prove the statement via its contrapositive, assuming that $\OPT_{\rsm}(I') > \alpha \frac{R}{m}$ for some reward $R$ and $\alpha \in (0,1)$. Let $\frac{R'}{m}> \alpha \frac{R}{m}$ be the optimal average reward of \rsm. By Lemma~\ref{lem:hardness:periodic}, we can assume w.l.o.g. that the assignment $\OPT_{\rsm}(I')$, that achieves an average reward of $\frac{R'}{m}$, is a periodic assignment of period $m$. However, given that all the delays are equal to $m$ in the instance $I'$ of \rsm, it is easy to see that in any period of $m$ consecutive rounds, each element is played at most once. Moreover, the average reward of each period is exactly $\frac{R'}{m}$. Therefore, any continuous subsequence of length $m$ in the solution of the \rsm naturally induces a solution to the instance $I$ of SWIU of total reward exactly $R'$. This, in turn, implies that $\OPT_{SWIU}(I) \geq R' \geq \alpha R$.

By the above discussion, we have completed the proof of a reduction from SWIU to \rsm. Therefore, any polynomial-time $\left(1 - \frac{1}{e} + \epsilon\right)$-approximation algorithm for \rsm, for some $\epsilon>0$, would imply a $\left(1 - \frac{1}{e} + \epsilon\right)$-approximation algorithm for SWIU. However, by Theorem~\ref{thm:hardness:swiu} this is not possible, unless $\textbf{P} = \textbf{NP}$.
\qed
\newline

We believe that, through a similar reduction as above, we can prove information-theoretic hardness of the \rsm problem by leveraging the results in~\cite{MSV08}. We leave this as future work.
\section{Matroid Blocking Bandits: Omitted Proofs}

\subsection{The full-information setting}
\label{appendix:mbb:fullinformation}
We now provide an alternative analysis for \ig (\IG), which, as opposed to Theorem~\ref{thm:interleavedSubmodular} for the \rsm problem, does not rely on the concave closure of submodular functions. We first note that the correctness of \IG follows directly by Fact~\ref{inter:fact:alwaysavailable}, that is, the set of sampled arms $\G_t$ at each round $t \in [T]$ only contains available (i.e., non-blocked) arms, in combination with the fact that the algorithm always plays an independent set $\A_t \in \I_{\M|\G_t} \subseteq \I$. 

For any set $S \subseteq \A$, let $\z(S) = \sum_{i \in S} z_i$. Consider the following LP, based on the matroid polytope associated with $\M$:
\begin{align}
\textbf{maximize: }& T \cdot \sum_{i \in \A} \mu_{i} z_{i} \tag{\textbf{LP}} \label{lp:LP}\\
\textbf{s.t. }& \z(S) \leq \rk(S), \forall S\subseteq \A \label{flp:rank}\\
\qquad & {\bf 0} \preceq \z \preceq \bm{d}^{-1}. \label{flp:window}
\end{align}

In \eqref{lp:LP}, each variable $z_{i}$ can be thought of as the fraction of rounds where arm $i\in \A$ is played. Intuitively, constraints \eqref{flp:window} of \eqref{lp:LP} indicate the fact that, due to blocking constraints, the fraction of time we can play an arm $i \in \A$ is upper bounded by $\frac{1}{d_i}$, while constraints \eqref{flp:rank} impose the rank restrictions in order to guarantee that the set of arms played at any round $t$ corresponds to an independent set of the matroid $\M$. 

As we show in the following lemma, the formulation \eqref{lp:LP} yields an approximate upper bound on $\OPT(T)$, while the approximation becomes exact as $T$ increases.

\begin{restatable}{lemma}{restateStructuralflp}\label{lem:structural:flp}
Let $\Rew^{LP}(T)$ be the optimal solution to \eqref{lp:LP} and $\OPT(T)$ be the optimal expected reward over $T$ rounds. We have
$
\Rew^{LP}(T) \geq \OPT(T) - \mathcal{O}\left(d_{\max} \rk(\M)\right).
$
\end{restatable}
\begin{proof}
In order to prove the Lemma, we first construct an IP upper bound on the optimal expected reward over $T$ rounds, $\OPT(T)$. Then, we construct \eqref{lp:LP} by averaging over time the 0-1 variables of the IP. For any set $S \subseteq \A$, let $\x_t(S) = \sum_{i \in S} x_{i,t}$. 
\begin{align}
\textbf{maximize:}& \sum_{i \in [T]} \sum_{i \in \A} \mu_{i} x_{i,t} \tag{\textbf{IP}} \label{lp:IP}\\
\textbf{s.t.}& \sum_{t' \in [t,t+d_i-1]} x_{i,t'} \leq 1, \forall i \in \A, \forall t \in [T] \label{lp:window} \\
\qquad & \x_t(S) \leq \rk(S), \forall S\subseteq \A, \forall t \in [T] \label{lp:rank} \\
\qquad & \x_{t} \in \{0,1\}^k, \forall t \in [T]. \notag
\end{align}

In \eqref{lp:IP}, each variable $x_{i,t}$ can be thought of as the 0-1 indicator of playing arm $i\in \A$ at time $t \in [T]$. Intuitively, constraints \eqref{lp:window} of \eqref{lp:IP} indicate the fact that, due to the blocking constraints, each arm $i \in \A$ can be played at most once every $d_i$ steps, while constraints \eqref{lp:rank} impose the rank restrictions due to the matroid $\M$ at any round $t$. Let $\Rew^{IP}(T)$ be the optimal solution to \eqref{lp:IP}.

Fix any (optimal) algorithm and let $\A^{*}_t$ be the set of arms played by the algorithm at round $t$. Notice that the sets $\A^{*}_t$ are deterministic, given that the choices of any full-information algorithm that maximizes the expected cumulative reward are independent of the observed reward realizations. By linearity of expectation, the expected reward collected (over the randomness of the reward realizations) by the optimal algorithm can be expressed as 
\begin{align*}
    \Ex{}{\sum_{t \in [T]} \sum_{i \in \A^{*}_t} X_{i,t}} = \sum_{t \in [T]} \sum_{i \in \A^{*}_t} \Ex{}{X_{i,t}} = \sum_{t \in [T]} \sum_{i \in \A^{*}_t} \mu_i.
\end{align*}
Consider a feasible solution of \eqref{lp:IP} such that for each $i \in \A$ and $t \in [T]$, we set $x_{i,t} = 1$, if $i \in \A^{*}_t$, and $x_{i,t}=0$, otherwise. It is not hard to verify that the objective of \eqref{lp:IP} for this assignment coincides with the expected reward collected by the above optimal algorithm. Moreover, constraints \eqref{lp:window} are satisfied, since for any arm $i \in [T]$ and any window of $d_i$ consecutive time steps, the algorithm can play the arm at most once. Finally, constraints \eqref{lp:rank} are satisfied, since for any time $t$, the set of arms played, $\A^{*}_t$, is an independent set of the matroid $\M$, thus satisfying all the rank constraints. Therefore, by exhibiting a feasible solution of \eqref{lp:IP} that has the same objective value as the expected reward of any optimal algorithm, we conclude that $\Rew^{IP}(T) \geq \OPT(T)$.


Consider any optimal solution $\x^*$ of \eqref{lp:IP} for a time horizon $T$. By constraints \eqref{lp:window}, for any $t \in [T]$ and $i \in \A$, we have $\sum_{t' \in [t,t+d_i-1]} x^*_{i,t'} \leq 1$. By working along the lines of the proof of Lemma~\ref{lem:structural:CP} and averaging constraints \eqref{lp:window} over all $t \in [T]$, we get
\begin{align}
    \frac{1}{T}\sum_{t \in [T]} x^*_{i,t} \leq \frac{1}{d_i} \left(1 + \frac{d_i-1}{T}\right), \forall i \in \A \label{eq:flp:window}. 
\end{align}
Similarly, for any set $S \subseteq \A$, by averaging the inequalities of \eqref{lp:rank} over all rounds $t \in [T]$, we get
\begin{align}
    \frac{1}{T}\sum_{t \in [T]} \x^*_t(S) \leq \rk(S), \forall S \subseteq \A. \label{eq:flp:rank}
\end{align}
Now, consider an assignment of \eqref{lp:LP} such that
\begin{align*}
z_i = \left(1 + \frac{d_{\max}-1}{T}\right)^{-1} \frac{1}{T} \sum_{t \in [T]} x^*_{i,t},~~\forall i \in \A.    
\end{align*}

It is not hard to see that by inequality \eqref{eq:flp:window}, we have $z_{i} \leq \frac{1}{d_i}$ for any $i \in \A$. Moreover, given that $\left(1 + \frac{d_{\max}-1}{T}\right)^{-1} \leq 1$, for any set $S \subseteq \A$, we have that $\sum_{i \in S} z_i \leq \frac{1}{T}\sum_{t \in [T]} \x^*_t(S) \leq \rk(S)$. Therefore, the assignment $\z \in \mathbb{R}_{+}$ with $(\z)_i = z_i$ satisfies constraints \eqref{flp:rank} and \eqref{flp:window} of \eqref{lp:LP}. Considering the objective value of \eqref{lp:LP} for the assignment $\z$, we have that
\begin{align*}
    T \sum_{i \in \A} \mu_i z_i &= T  \sum_{i \in \A} \mu_i \left(1 + \frac{d_{\max}-1}{T}\right)^{-1} \frac{1}{T} \sum_{t \in [T]} x^*_{i,t} \\
    &\geq \left(1 + \frac{d_{\max}-1}{T}\right)^{-1} \sum_{t \in [T]} \sum_{i \in \A} \mu_i x^*_{i,t}\\
    &\geq \left(1 - \frac{d_{\max}-1}{d_{\max}-1 + T}\right) \sum_{t \in [T]} \sum_{i \in \A} \mu_i x^*_{i,t},
\end{align*}
where the last inequality follows by the fact that $\frac{1}{1+\beta} = 1 - \frac{\beta}{1+ \beta}$ for any $\beta \in \mathbb{R}$.
By exhibiting a feasible solution of \eqref{lp:LP} of value greater than $\left(1 - \frac{d_{\max}-1}{d_{\max}-1 + T}\right) \Rew^{IP}(T)$, the lemma follows by the fact that $\Rew^{IP}(T) \geq \OPT(T)$ and that $\OPT(T) \leq T \cdot \rk(\M)$, since the rewards of all arms lie in $[0,1]$.
\end{proof}

We are now ready to complete the proof of the following result.

\restateinterleavedGreedy*

\begin{proof}
Before we proceed with the proof, we first emphasize that the algorithm \IG is not aware of the reward realizations of each round before it plays a subset of arms. Therefore, since the objective it to maximize the cumulative expected reward, we can assume that the reward of each arm $i \in \A$ is deterministic and equal to $\mu_i$.

Let $\z^*$ be an optimal solution to \eqref{lp:LP}. Given the fact that the feasible set of \eqref{lp:LP} is essentially the intersection of the matroid polytope $\mathcal{P}(\M)$ and the (downward-closed) blocking constraints $\z^* \leq \bm{d}^{-1}$, it holds that $\z^* \in \mathcal{P}(\M)$. Therefore, the point $\z^*$ can be expressed as a convex combination of characteristic vectors of $k$ independent sets of $\M$, denoted by $T_1, \dots, T_k $, where $T_j \in \I, \forall j \in [k]$. By Lemma \ref{lem:characteristic}, this in turn induces a probability distribution, $\I(\z^*)$, over $T_1, \dots, T_k$, such that the marginal probability of each element $i \in \A$ being in the sampled set is exactly $z^*_i$.

Conditioned on the random offsets $\{r_i\}_{i \in \A}$, the sequence of sampled sets $\{\G_t\}_{t \in [T]}$ is deterministic and independent of the observed rewards. Let $f_{\M,\mu}(\G_t)$ be the weighted rank function over the subset $\G_t$, that is, the expected reward of a maximum independent set of $\M$ contained in $\G_t$. By denoting as $\G_t \sim {\bf p}$ the random set of elements, where each element $i \in \A$ participates with probability equal to $({\bf p})_i = p_i$, we have that $\G_t \sim {\bf d}^{-1}$ for each $t \in [T]$. The expected reward of \IG can be expressed as
\begin{align*}
    \Rew^{IG}(T) = \sum_{t \in [T]} \E{\mu(\A_t)} = \sum_{t \in [T]} \Ex{\G_t \sim {\bf d}^{-1}}{f_{\M,\mu}(\G_t)} \geq  \sum_{t \in [T]} \Ex{\G_t \sim \z^*}{f_{\M,\mu}(\G_t)},
\end{align*}
where the last inequality follows by Lemma \ref{lem:weightedrank}, namely, the fact that the weighted rank function $f_{\M,\mu}(\G_t)$ is a monotone (non-decreasing) and by the fact that $\z^* \preceq \bm{d}^{-1}$.

Let $F_{\M,\mu}(\z)$ and $f_{\M,\mu}^+(\z)$ be the multi-linear extension and the concave closure of function $f_{\M,\mu}$, respectively. By the correlation gap inequality for submodular functions (see Lemma \ref{lem:correlationgap}), for each vector $\z$, we have that $F_{\M,\mu}(\z) \geq \left(1 - \frac{1}{e}\right)f_{\M,\mu}^+(\z)$. Moreover, by definition of the concave closure, it has to be that $f_{\M,\mu}^+(\z) \geq \Ex{I \sim \I(\z)}{f_{\M, \mu}(I)}$, since $f_{\M,\mu}^+(\z)$ is the maximum valued distribution over independent sets, such that the marginal contribution of each element $i \in \A$ is equal to $z_i$, i.e., $\Prob{I \sim \I(\z)}{i \in I} = z_i$. By combining the above facts, we have that
\begin{align*}
    \sum_{t \in [T]} \Ex{\G_t \sim \z^*}{f_{\M,\mu}(\G_t)} = T \cdot F_{\M,\mu}(\z^*) \geq \left(1 - \frac{1}{e}\right) T \cdot f^+_{\M,\mu}(\z^*) \geq \left(1 - \frac{1}{e}\right) T \cdot \Ex{I \sim \I(\z^*)}{f_{\M, \mu}(I)}.
\end{align*}

Using the fact that the greedy algorithm collects every element in $I$ for any independent set $I \in \I$, we have that $\Ex{I \sim \I(\z^*)}{f_{\M, \mu}(I)} = \Ex{I \sim \I(\z^*)}{\mu(I)}$. Finally, since the marginal probability of each element $i \in \A$ being in $I \sim \I(\z^*)$ equals $z^*_i$, we have
\begin{align*}
T \cdot \Ex{I \sim \I(\z^*)}{f_{\M, \mu}(I)} = T \cdot \Ex{I \sim \I(\z^*)}{\mu(I)} = T \cdot \sum_{i \in \A} \mu_i z^*_i = \Rew^{LP}(T).
\end{align*}
By combining the above relations with Lemma \ref{lem:structural:flp}, we get that
\begin{align*}
    \Rew^{IG}(T) \geq \left(1 - \frac{1}{e}\right) \Rew^{LP}(T) \geq \left(1 - \frac{1}{e}\right) \OPT(T) - \mathcal{O}(d_{\max} \rk(\M)),
\end{align*}
thus, the proof is completed.
\end{proof}





\subsection{The bandit setting and regret analysis}

\restateRegretCoupling*
\begin{proof}
Let $\{\G_t(\off)\}_{t \in [T]}$ be the sequence of sampled arms over $T$ rounds as a function of the sampled offsets $\off \in [0,1]^k$. Moreover, let $X_t(S)$ be the realized rewards of a subset $S \subseteq \A$ of arms at round $t \in [T]$. We denote by $\A^{\pi}_t$ the arms played at round $t\in[T]$ and by $H^{\pi}_t = \{\A^{\pi}_1, X_1(\A^{\pi}_1), \dots, \A^{\pi}_t, X_t(\A^{\pi}_t)\}$ the {\em history} of arm playing and observed realizations up to (and including) time $t$ by algorithm $\pi \in \{\IG, \UCB\}$. Recall that we denote by $\mathcal{R}$ the randomness due to the reward realizations of the arms.

Notice that in the case of \UCB and for fixed offsets, the player's actions only depend on the previous realized rewards of the arms. Thus, for any fixed offset vector $\off^{\UCB}$, we have
\begin{align*}
&\Ex{\mathcal{R}}{\sum_{i \in \A} X_{i,t} \event{i \in \arg\max_{S \subseteq \G_t(\off^{\UCB}), S \in \I}\{\bar{\mu}_t(S)\}} }\\
    &= \Ex{\mathcal{R}}{\sum_{i \in \A} \Ex{\mathcal{R}}{ X_{i,t} \event{i \in \arg\max_{S \subseteq \G_t(\off^{\UCB}), S \in \I}\{\bar{\mu}_t(S)\}}~|~H^{\UCB}_{t-1}}}\\
    &= \Ex{\mathcal{R}}{\sum_{i \in \A} \Ex{\mathcal{R}}{ X_{i,t}~|~H^{\UCB}_{t-1}} \event{i \in \arg\max_{S \subseteq \G_t(\off^{\UCB}), S \in \I}\{\bar{\mu}_t(S)\}}} \\
    &= \Ex{\mathcal{R}}{\sum_{i \in \A} \mu_i \event{i \in \arg\max_{S \subseteq \G_t(\off^{\UCB}), S \in \I}\{\bar{\mu}_t(S)\}}}\\
    &=\Ex{\mathcal{R}}{ \mu\left( \arg\max_{S \subseteq \G_t(\off^{\UCB}), S \in \I}\{\bar{\mu}_t(S)\}\right)}.
\end{align*}

Similarly, notice that the algorithm \IG is oblivious to the realized rewards. Therefore, for any fixed offset vector $\off^{\IG}$ and at any time $t \in [T]$, we get
\begin{align*}
    \Ex{\mathcal{R}}{\sum_{i \in \A} X_{i,t} \event{i \in \arg\max_{S \subseteq \G_t(\off^{\IG}), S \in \I}\{{\mu}(S)\}} }
    &= 
    \Ex{\mathcal{R}}{\sum_{i \in \A} \mu_i \event{i \in \arg\max_{S \subseteq \G_t(\off^{\IG}), S \in \I}\{\mu(S)\}}}\\
    &=\Ex{\mathcal{R}}{ \max_{S \subseteq \G_t(\off^{\IG}), S \in \I}\{\mu(S)\}}.
\end{align*}
The lemma follows by observing that the offsets $\off^{\IG}$ and $\off^{\UCB}$ of the two algorithms follow exactly the same distribution. Therefore, we have

\begin{align*}
&\Rew^{\IG}(T) - \Rew^{\UCB}(T) \\
= & \Ex{\off^{\IG} \sim [0,1]^k, \mathcal{R}}{\sum_{t \in [T]}\max_{S \subseteq \G_t(\off^{\IG}), S \in \I}\{\mu(S)\}} - \Ex{\off^{\UCB} \sim [0,1]^k, \mathcal{R}}{\sum_{t \in [T]}\mu\left( \arg\max_{S \subseteq \G_t(\off^{\UCB}), S \in \I}\{\bar{\mu}_t(S)\}\right)} \\
= & \Ex{\off \sim [0,1]^k, \mathcal{R}}{\sum_{t \in [T]}\left(\max_{S \subseteq \G_t(\off), S \in \I}\{\mu(S)\} - \mu\left( \arg\max_{S \subseteq \G_t(\off), S \in \I}\{\bar{\mu}_t(S)\}\right)\right)}.
\end{align*}
\end{proof}

\restateRegretSub*

\begin{proof}
Recall that under the event $\mathcal{E}_{\off}$, both algorithms \IG and \UCB use the same offset vector $\off$ and, thus, they operate on same sequence of sampled arms over time. Let $\G_t = \G_t(\off)$ be the common set of sampled arms and let $\A^{\IG}_t$ and $\A^{\UCB}_t$ be the maximal independent sets computed by \IG and \UCB, respectively, at any round $t \in [T]$. Notice that for any $t \in [T]$ both $\A^{\IG}_t$ and $\A^{\UCB}_t$ are bases of the restricted matroid $\M|\G_t$ and, thus, correspond to independent sets of $\I$ of equal cardinality. Let $\sigma_t$ be the bijection between $\A^{\IG}_t$ and $\A^{\UCB}_t$ described by Theorem~\ref{cor:basesexchange}. For any $t \in [T]$, we have that
$$
\Delta_{\G_t}(\bar\mu) =  \mu(\A^{\IG}_{t}) - \mu(\A^{\UCB}_{t}) = \sum_{i \in \A_t^{\IG}} \mu_i - \sum_{j \in \A_t^{\UCB}} \mu_j = \sum_{i \in \A^{\IG}_t} \left(\mu_i - \mu_{\sigma^{-1}_t(i)}\right) = \sum_{i \in \A^{\IG}_t} \Delta_{i , \sigma^{-1}_{t}(i)}.
$$
\end{proof}

\restateRegretTechnical*

\begin{proof}
We first focus on proving inequality \eqref{inq:reg:under}, that is, the part of the regret attributed to an arm $j >1$ when not enough samples have been collected. Notice that the algorithm $\UCB$ never accumulates regret when it plays the arm $j=1$ of highest mean reward. Recall that for any fixed $j \in \A$, we have $\Delta_{1,j} > \Delta_{2,j} > \dots > \Delta_{j,j} = 0$, since we assume w.l.o.g. that the arms have distinct mean rewards. By construction of our algorithm, if the number of samples from arm $j \in \A$ is increased at some round $t$, it is because there exists exactly one arm $i \in \A$ with $\Delta_{i,j} > 0$, such that $\sigma_t(j) = i$. The above is implied by Theorem~\ref{cor:basesexchange}, given the fact that each bijection $\sigma_t$ for all $t \in [T]$ maps each arm played by \UCB in $\A^{\UCB}_t$ to a single arm played by \IG in $\A_t^{\IG}$. On the other hand, as the number of obtained samples $T_j(t)$ from arm $j \in \A$ by time $t\in [T]$ increases, the maximum suboptimality gap $\Delta_{i,j}$ that can be charged in the under-sampled part of the regret is that of the maximum reward $i \in \A$ that satisfies $T_j(t) \leq \ell_{i,j}$. By the above analysis, for any $j>1$, we get that 
\begin{align}
\sum_{t \in [T]} \sum^{j-1}_{i = 1} \Delta_{i , j} \event{\sigma_t(j)=i, T_j(t) \leq \ell_{i,j}} \notag
&\leq \sum^{j-1}_{i=1} \left(\Delta_{i,j} -  \Delta_{i+1,j}\right)\ell_{i,j} \notag\\
&\leq \sum^{j-1}_{i=1} \left(\Delta_{i,j} -  \Delta_{i+1,j}\right)\frac{8 \ln(T)}{\Delta^2_{i,j}} \label{eq:regret:technical:1},
\end{align}
where the last inequality follows by definition of $\ell_{i,j}$.

The rest of the claim follows by simple algebra. Indeed,
\begin{align*}
    \eqref{eq:regret:technical:1}&\leq \left(\sum^{j-1}_{i=1}\frac{\Delta_{i,j} - \Delta_{i+1,j}}{\Delta^2_{i,j}}\right)8 \ln(T) \notag\\
    &\leq \left(\frac{1}{\Delta_{j-1,j}} + \sum^{j-2}_{i=1}\frac{\Delta_{i,j} - \Delta_{i+1,j}}{\Delta^2_{i,j}}\right)8 \ln(T) \notag\\
    &\leq \left(\frac{1}{\Delta_{j-1,j}} + \sum^{j-2}_{i=1}\frac{\Delta_{i,j} - \Delta_{i+1,j}}{\Delta_{i,j} \Delta_{i+1,j}}\right)8 \ln(T) \notag\\
    &= \left(\frac{1}{\Delta_{j-1,j}} + \sum^{j-2}_{i=1}\left(\frac{1}{\Delta_{i+1,j}} - \frac{1}{\Delta_{i,j}}\right)\right)8 \ln(T) \notag\\
    &= \left(\frac{2}{\Delta_{j-1,j}} - \frac{1}{\Delta_{1,j}}\right)8 \ln(T) \notag\\
    &\leq \frac{16}{\Delta_{j-1,j}} \ln(T) \notag.
\end{align*}




We now focus on proving inequality~\eqref{inq:reg:sufficiently}, that is, the regret accumulated after a sufficient number of samples has been collected from an arm $j > 1$. Notice, that given the event $\mathcal{E}_{\off}$, the expectation in the LHS of inequality~\eqref{inq:reg:sufficiently} is taken only over the randomness of the realized rewards that are observed by \UCB. 


For proving the upper bound, we fix any arm $j > 1$ and focus on each arm $i \in \A$ such that $i < j$ and, thus, $\Delta_{i,j}>0$. Let us fix any such arm $i \in \A$. For any $t \in [T]$, the event $\{\sigma_t(j) = i\}$ implies that $\{\mu_{i} > \mu_j, \bar{\mu}_{i,t} \leq \bar{\mu}_{j,t}\}$, namely, the order of the UCB-indices at time $t \in [T]$ of $i$ and $j$ is inconsistent with the order of their true mean rewards. In the opposite case, the algorithm \UCB would have chosen the set $\A^{\UCB}_t - j + i$, which, as suggested by Theorem~\ref{cor:basesexchange}, is an independent set of $\M$. Therefore, for any arm $i < j$, we have
\begin{align}
    \{\sigma_t(j)=i, T_j(t) > \ell_{i,j}\} \subseteq \{\bar{\mu}_{i,t} \leq \bar{\mu}_{j,t},\mu_i > \mu_j, T_j(t) > \ell_{i,j}\}. \label{eq:lem:ucb:0}
\end{align}
Note that the inclusion in the above expression is because the inconsistency in the order of UCB-indices does not necessarily imply that $\sigma_t(j)=i$ (i.e., that \UCB actually exchanges $j$ for $i$ at time $t \in [T]$).

By definition of the UCB-indices, the event $\bar{\mu}_{i,t} \leq \bar{\mu}_{j,t}$ at time $t \in [T]$ implies that 
\begin{align}
    \hat{\mu}_{i,T_{i}(t)} + \sqrt{\frac{2 \ln{(t)}}{T_{i}(t)}} \leq \hat{\mu}_{j,T_{j}(t)} + \sqrt{\frac{2 \ln{(t)}}{T_{j}(t)}}. \label{eq:lem:ucb:1}
\end{align}

We fix $s_i = T_i(t)$ and $s_j = T_j(t) > \ell_{i,j}$ to be the number of samples obtained from arm $i$ and $j$, respectively, by time $t \in [T]$. Notice that in order for \eqref{eq:lem:ucb:1} to hold, at least one of the following events must be true:
\begin{align*}
     \textbf{(i) }\bigg\{\hat{\mu}_{i,s_{i}} + \sqrt{\frac{2 \ln{(t)}}{s_i}} \leq \mu_i \bigg\}, \textbf{   (ii)  }\bigg\{\hat{\mu}_{j,s_{j}} \geq \mu_j + \sqrt{\frac{2 \ln{(t)}}{s_{j}}}\bigg\},\textbf{   (iii)  } \bigg\{\mu_i < \mu_j + 2 \sqrt{\frac{2 \ln{(t)}}{s_j}}\bigg\}.
\end{align*}
Indeed, it can be easily verified that the simultaneous negation of the above three events contradicts \eqref{eq:lem:ucb:1} for any fixed number of samples $s_i,s_j$. 

By our choice of $\ell_{i,j} = \bigg\lfloor \frac{8 \ln(T)}{\Delta^2_{i,j}}\bigg\rfloor$ and the fact that $s_j \geq \ell_{i,j} + 1 \geq \frac{8 \ln(T)}{\Delta^2_{i,j}}$, we can see that event $\textbf{(iii)}$ cannot be true, since in that case, we have
$$
 \mu_j + 2 \sqrt{\frac{2 \ln{(t)}}{s_j}}  \leq \mu_j + 2 \sqrt{\frac{2 \Delta^2_{i,j}\ln{(t)}}{8 \ln(T)}} \leq \mu_j + \Delta_{i,j} = \mu_i.
$$
Moreover, by Hoeffding's inequality, for the probabilities of the events $\textbf{(i)}$ and $\textbf{(ii)}$, we have that
$$
\Pro{\hat{\mu}_{i,s_{i}} + \sqrt{\frac{2 \ln{(t)}}{s_i}} \leq \mu_i} \leq e^{-4 \ln(t)} = t^{-4}\text{   and   }\Pro{\hat{\mu}_{j,s_{j}} \geq \mu_j + \sqrt{\frac{2 \ln{(t)}}{s_{j}}}} \leq e^{-4 \ln(t)} = t^{-4},
$$
where the probability is taken over the randomness of the reward realizations.

Therefore, for any numbers of samples $s_i = T_i(t)$ and $s_j = T_j(t) > \ell_{i,j}$, we have
\begin{align}
    \Pro{\bar{\mu}_{i,t} \leq \bar{\mu}_{j,t},\mu_i > \mu_j, T_j(t) = s_j, T_i(t)= s_i} 
    &\leq \Pro{\hat{\mu}_{i,s_{i}} + \sqrt{\frac{2 \ln{(t)}}{s_i}} \leq \mu_i} + \Pro{\hat{\mu}_{j,s_{j}} \geq \mu_j + \sqrt{\frac{2 \ln{(t)}}{s_{j}}}} \notag\\
    &\leq 2\cdot t^{-4}  \label{eq:ucb:union}.
\end{align}
Finally, by union bound over the possible number of samples, $s_i$ and $s_j$, and using the aforementioned results, for any $j > 1$ and time $t \in [T]$, we have
\begin{align}
    &\Ex{\mathcal{R}}{\sum_{t \in [T]} \sum^{j-1}_{i = 1} \Delta_{i , j} \event{\sigma_t(j)=i, T_j(t) > \ell_{i,j}}} \notag\\
    &= \Ex{\mathcal{R}}{\sum_{t \in [T]} \sum^{j-1}_{i = 1} \sum^{t-1}_{s_i = 0} \sum^{t-1}_{s_j = \ell_{i,j}+1} \Delta_{i , j} \event{\sigma_t(j)=i, T_j(t) = s_j, T_i(t)= s_i}}\label{lem:ucb:f2}\\
    &\leq \Ex{\mathcal{R}}{\sum_{t \in [T]} \sum^{j-1}_{i = 1} \sum^{t-1}_{s_i = 0} \sum^{t-1}_{s_j = \ell_{i,j}+1} \Delta_{i , j} \event{\bar{\mu}_{i,t} \leq \bar{\mu}_{j,t},\mu_i > \mu_j, T_j(t) = s_j, T_i(t)= s_i}}\label{lem:ucb:f3}\\
    &= \sum_{t \in [T]} \sum^{j-1}_{i = 1} \sum^{t-1}_{s_i = 0} \sum^{t-1}_{s_j = \ell_{i,j}+1} \Delta_{i , j} \Pro{\bar{\mu}_{i,t} \leq \bar{\mu}_{j,t},\mu_i > \mu_j, T_j(t) = s_j, T_i(t)= s_i}\notag\\
    &\leq  \sum_{t \in [T]} \sum^{j-1}_{i = 1} \Delta_{i , j} 2 t(t-1) t^{-4}  \label{lem:ucb:f4},
\end{align}
where in \eqref{lem:ucb:f2} we consider any possible number of samples by time $t$ for each arm. Moreover, inequality \eqref{lem:ucb:f3} follows by \eqref{eq:lem:ucb:0} and \eqref{lem:ucb:f4} follows by \eqref{eq:ucb:union}.
The proof of inequality \eqref{inq:reg:sufficiently} follows by the fact that 
$$\sum_{t \in [T]} t(t-1)t^{-4} \leq \sum_{t \in [T]}t^{-2} \leq \sum^{+\infty}_{t =1}t^{-2} = \frac{\pi^2}{6}.$$
\end{proof}


\subsection{Proof of Theorem \ref{thm:approxregret}}

\restateApproxRegret*

\begin{proof}
By inequality \eqref{eq:regret:twoalgos}, Lemma~\ref{lem:regret:coupling} and Definition \ref{def:gaps}, we can upper bound the $\alpha$-regret, for $\alpha = 1 - \frac{1}{e}$, as
\begin{align}
\alpha \OPT(T) - \Rew^{\UCB}(T) \leq \Ex{\off \sim [0,1]^k, \mathcal{R}}{\sum_{t \in [T]}\Delta_{\G_t(\off)}(\bar{\mu}_t)} + \mathcal{O}(d_{\max}\cdot \rk(\M)), \label{eq:reg:final:1}
\end{align}
where the expectation is taken over the randomness of the offset vector $\off$ and the reward realizations.

Under the event $\mathcal{E}_{\off}$, that is, where both \IG and \UCB use the same offsets $\off$, let $\{\sigma_t\}_{t \in [T]}$ be the sequence of bijections between $\A_t^{\UCB}$ and $\A_t^{\IG}$ over all rounds $t \in [T]$, as described in Theorem~\ref{cor:basesexchange}. Using Lemma~\ref{lem:regret:suboptdecomp}, we have that
\begin{align}
    \Ex{\off \sim [0,1]^k, \mathcal{R}}{\sum_{t \in [T]}\Delta_{\G_t(\off)}(\mu_t)} 
    &= \Ex{\off \sim [0,1]^k, \mathcal{R}}{\sum_{t \in [T]}\sum_{i \in \A^{\IG}_t} \Delta_{i , \sigma^{-1}_{t}(i)}}\notag \\
    &= \Ex{\off \sim [0,1]^k, \mathcal{R}}{\sum_{t \in [T]}\sum_{i \in \A^{\IG}_t}
    \sum_{j \in \A} \Delta_{i,j} \event{\sigma_t(j) = i}}\notag\\
    &\leq \Ex{\off \sim [0,1]^k, \mathcal{R}}{\sum_{t \in [T]}\sum_{j \in \A}\sum_{i < j}
    \Delta_{i,j} \event{\sigma_t(j) = i}}, \label{eq:reg:final:2}
\end{align}
where in the last inequality we restrict ourselves to arms $i < j$, where $\Delta_{i,j}>0$.

Now using the results of Lemma~\ref{lem:regret:technical}, we can further upper bound \eqref{eq:reg:final:2} as 

\begin{align}
&\Ex{\off \sim [0,1]^k, \mathcal{R}}{\sum_{t \in [T]}\sum_{j \in \A}\sum_{i < j} \Delta_{i,j} \event{\sigma_t(j) = i}} \notag \\
&= \Ex{\off \sim [0,1]^k, \mathcal{R}}{\sum_{t \in [T]}\sum_{j \in \A}\sum_{i < j} \Delta_{i,j} \event{\sigma_t(j) = i, T_j(t) \leq \ell_{i,j}}} \notag\\
&\qquad \qquad+ \Ex{\off \sim [0,1]^k}{ \Ex{\mathcal{R}}{\sum_{t \in [T]}\sum_{j \in \A}\sum_{i < j} \Delta_{i,j} \event{\sigma_t(j) = i, T_j(t) > \ell_{i,j}}}} \notag \\
&\leq \sum_{j > 1} \frac{16}{\Delta_{j-1,j}}\ln(T) + \frac{\pi^2}{3} \sum_{j > 1}\sum_{i = 1}^{j-1} \Delta_{i,j}. \label{eq:reg:final:3}
\end{align}

By combining inequalities \eqref{eq:reg:final:1}, \eqref{eq:reg:final:2} and \eqref{eq:reg:final:3}, we can upper bound the regret as a function of the gaps as follows:
\begin{align*}
    \alpha \OPT(T) - \Rew^{\UCB}(T) \leq \sum_{j > 1} \frac{16}{\Delta_{j-1,j}}\ln(T) + \frac{\pi^2}{3} \sum_{j > 1}\sum_{i = 1}^{j-1} \Delta_{i,j} + \mathcal{O}(d_{\max}\cdot \rk(\M))~~ \text{ (gap-dependent regret)}.
\end{align*}

In order to conclude the proof of the theorem, we would like to construct a regret bound that is independent of the gaps. The standard method is to partition the suboptimality gaps into ``small'' and ``large'' and, then, separately study their contribution to the regret. Specifically, for each $j \in \A$ and fixed $\epsilon > 0$, we define:
\begin{align*}
    S_j = \{i < j~|~\Delta_{i,j} \leq \epsilon\}\text{ and }L_j = \{i < j~|~\Delta_{i,j} > \epsilon\}.
\end{align*}
Starting again from \eqref{eq:reg:final:2} and noticing that the total regret due to small gaps can be at most $\epsilon\cdot T$ per arm, we have
\begin{align}
&\Ex{\off \sim [0,1]^k, \mathcal{R}}{\sum_{t \in [T]}\sum_{j \in \A}\sum_{i < j} \Delta_{i,j} \event{\sigma_t(j) = i}} \notag \\
&= \Ex{\off \sim [0,1]^k, \mathcal{R}}{\sum_{t \in [T]}\sum_{j \in \A}\sum_{i \in S_j} \Delta_{i,j} \event{\sigma_t(j) = i}} + \Ex{\off \sim [0,1]^k, \mathcal{R}}{\sum_{t \in [T]}\sum_{j \in \A}\sum_{i \in L_j} \Delta_{i,j} \event{\sigma_t(j) = i}} \notag\\
&\leq 
\epsilon k  T + \Ex{\off \sim [0,1]^k, \mathcal{R}}{\sum_{t \in [T]}\sum_{j \in \A}\sum_{i \in L_j} \Delta_{i,j} \event{\sigma_t(j) = i}}. \label{eq:reg:final:4}
\end{align}

We now focus only on the regret due to the large gaps, namely, the pairs $i,j$ such that $j \in \A$ and $i \in L_j$, which implies that $\Delta_{i,j} > \epsilon$. By exactly the same analysis as in the gap-dependent case, we can reach inequality \eqref{eq:reg:final:3}, in the restricted case where the summations only include pairs of arms such that $\Delta_{i,j} > \epsilon$ (notice that we can apply Lemma~\ref{lem:regret:technical} considering only the set $L_j$ of arms for each $j>1$). In addition, using the fact that $\Delta_{i,j} \leq 1$ for any $i,j \in \A$, we have 
\begin{align}
    \Ex{\off \sim [0,1]^k, \mathcal{R}}{\sum_{t \in [T]}\sum_{j \in \A}\sum_{i \in L_j} \Delta_{i,j} \event{\sigma_t(j) = i}} \leq 
    \sum_{j >1 } \frac{16}{\epsilon}\ln(T) + \frac{\pi^2}{6} k (k-1). \label{eq:reg:final:5}
\end{align}
By combining inequalities \eqref{eq:reg:final:4} and \eqref{eq:reg:final:5} with \eqref{eq:reg:final:1} and \eqref{eq:reg:final:2}, we have
\begin{align*}
    \alpha \OPT(T) - \Rew^{\UCB}(T) \leq \epsilon k T + \frac{16 k }{\epsilon}\ln(T) + \frac{\pi^2}{6} k(k-1) + \mathcal{O}(d_{\max}\cdot \rk(\M)).
\end{align*}
Finally, by setting $\epsilon = 4 \sqrt{\frac{\ln(T)}{T}}$, we get that 
\begin{align*}
   \alpha \OPT(T) - \Rew^{\UCB}(T) \leq 8 k \sqrt{T \ln(T)} + \frac{\pi^2}{6} k(k-1) + \mathcal{O}(d_{\max}\cdot \rk(\M))\quad \text{ (gap-independent regret)}.
\end{align*}
Therefore, we can conclude that the expected reward collected by \UCB in $T$ rounds is at least
\begin{align*}
    \left(1-\frac{1}{e}\right)\OPT(T) - \mathcal{O}\left(k \sqrt{T \ln(T)} + k^2 + d_{\max}\cdot \rk(\M) \right).
\end{align*}
\end{proof}
\section{Additional Results}

\subsection{Tight example for the naive greedy algorithm} \label{appendix:tightexample}

\begin{restatable}{lemma}{restateTightexample}\label{lem:tightexample}
For any $d \geq 2$, there exists an instance of the full-information variant of the \mbb problem (where the mean rewards are known a priori) such that the greedy strategy that plays a maximum mean reward independent set among the available arms collects a $\left(\frac{1}{2} + \frac{1}{2d}\right)$-fraction of the optimal expected reward.
\end{restatable}


\begin{proof}
We consider an infinite time horizon and a graphic matroid based on the graph $G_d = (V_d, E_d)$, which is recursively defined as follows: Let $G_1 = (V_1, E_1)$ with $V_1 = \{u,v\}$, $E_1 = \{\{u,v\}\}$ and assume that the arm associated with edge $\{u,v\}$ has delay $1$ and mean reward $1-\epsilon$, for some $\epsilon > 0$. For the graph $G_d = (V_d, E_d)$, we have $V_d = V_{d-1} \cup \{u_d\}$ and $E_d = E_{d-1} \cup \{\{u,u_d\}, \forall u \in V_{d-1}\}$ (namely, $G_d$ is essentially the result of the join operation between $G_{d-1}$ and a single vertex graph). The arms that are associated with the edges of $E_{d} \setminus E_{d-1}$ all have delay equal to $d$ and mean reward equal to $1- \frac{\epsilon}{d}$. The above recursive construction is illustrated in Figure \ref{fig:graphicmatroid}.

\begin{figure}[h]
\centering
\tikzstyle{ipe stylesheet} = [
  ipe import,
  even odd rule,
  line join=round,
  line cap=butt,
  ipe pen normal/.style={line width=0.4},
  ipe pen heavier/.style={line width=0.8},
  ipe pen fat/.style={line width=1.2},
  ipe pen ultrafat/.style={line width=2},
  ipe pen normal,
  ipe mark normal/.style={ipe mark scale=3},
  ipe mark large/.style={ipe mark scale=5},
  ipe mark small/.style={ipe mark scale=2},
  ipe mark tiny/.style={ipe mark scale=1.1},
  ipe mark normal,
  /pgf/arrow keys/.cd,
  ipe arrow normal/.style={scale=7},
  ipe arrow large/.style={scale=10},
  ipe arrow small/.style={scale=5},
  ipe arrow tiny/.style={scale=3},
  ipe arrow normal,
  /tikz/.cd,
  ipe arrows, % update arrows
  <->/.tip = ipe normal,
  ipe dash normal/.style={dash pattern=},
  ipe dash dashed/.style={dash pattern=on 4bp off 4bp},
  ipe dash dotted/.style={dash pattern=on 1bp off 3bp},
  ipe dash dash dotted/.style={dash pattern=on 4bp off 2bp on 1bp off 2bp},
  ipe dash dash dot dotted/.style={dash pattern=on 4bp off 2bp on 1bp off 2bp on 1bp off 2bp},
  ipe dash normal,
  ipe node/.append style={font=\normalsize},
  ipe stretch normal/.style={ipe node stretch=1},
  ipe stretch normal,
  ipe opacity 10/.style={opacity=0.1},
  ipe opacity 30/.style={opacity=0.3},
  ipe opacity 50/.style={opacity=0.5},
  ipe opacity 75/.style={opacity=0.75},
  ipe opacity opaque/.style={opacity=1},
  ipe opacity opaque,
]
\definecolor{red}{rgb}{1,0,0}
\definecolor{green}{rgb}{0,1,0}
\definecolor{blue}{rgb}{0,0,1}
\definecolor{yellow}{rgb}{1,1,0}
\definecolor{orange}{rgb}{1,0.647,0}
\definecolor{gold}{rgb}{1,0.843,0}
\definecolor{purple}{rgb}{0.627,0.125,0.941}
\definecolor{gray}{rgb}{0.745,0.745,0.745}
\definecolor{brown}{rgb}{0.647,0.165,0.165}
\definecolor{navy}{rgb}{0,0,0.502}
\definecolor{pink}{rgb}{1,0.753,0.796}
\definecolor{seagreen}{rgb}{0.18,0.545,0.341}
\definecolor{turquoise}{rgb}{0.251,0.878,0.816}
\definecolor{violet}{rgb}{0.933,0.51,0.933}
\definecolor{darkblue}{rgb}{0,0,0.545}
\definecolor{darkcyan}{rgb}{0,0.545,0.545}
\definecolor{darkgray}{rgb}{0.663,0.663,0.663}
\definecolor{darkgreen}{rgb}{0,0.392,0}
\definecolor{darkmagenta}{rgb}{0.545,0,0.545}
\definecolor{darkorange}{rgb}{1,0.549,0}
\definecolor{darkred}{rgb}{0.545,0,0}
\definecolor{lightblue}{rgb}{0.678,0.847,0.902}
\definecolor{lightcyan}{rgb}{0.878,1,1}
\definecolor{lightgray}{rgb}{0.827,0.827,0.827}
\definecolor{lightgreen}{rgb}{0.565,0.933,0.565}
\definecolor{lightyellow}{rgb}{1,1,0.878}
\definecolor{black}{rgb}{0,0,0}
\definecolor{white}{rgb}{1,1,1}
\begin{tikzpicture}[ipe stylesheet]
  \draw[red, ipe pen fat]
    (64, 704)
     -- (96, 704)
     -- (96, 704);
  \pic[ipe mark large, fill=darkgray]
     at (64, 704) {ipe fdisk};
  \pic[ipe mark large, fill=darkgray]
     at (96, 704) {ipe fdisk};
  \filldraw[draw=red, ipe pen fat, fill=darkgray, ipe opacity 75]
    (144, 736)
     -- (128, 704)
     -- (128, 704);
  \filldraw[ipe pen fat, fill=white]
    (128, 704)
     -- (160, 704);
  \filldraw[red, ipe pen fat]
    (144, 736)
     -- (160, 704);
  \pic[ipe mark large, fill=darkgray]
     at (144, 736) {ipe fdisk};
  \pic[ipe mark large, fill=white]
     at (128, 704) {ipe fdisk};
  \pic[ipe mark large, fill=white]
     at (160, 704) {ipe fdisk};
  \filldraw[ipe pen fat, fill=white]
    (192, 704)
     -- (224, 704);
  \filldraw[ipe pen fat, fill=white]
    (192, 704)
     -- (208, 736);
  \filldraw[ipe pen fat, fill=white]
    (208, 736)
     -- (224, 704);
  \filldraw[draw=red, ipe pen fat, fill=darkgray]
    (208, 736)
     -- (240, 736);
  \filldraw[draw=red, ipe pen fat, fill=darkgray]
    (192, 704)
     -- (240, 736);
  \filldraw[draw=red, ipe pen fat, fill=black]
    (224, 704)
     -- (240, 736);
  \pic[ipe mark large, fill=white]
     at (192, 704) {ipe fdisk};
  \pic[ipe mark large, fill=white]
     at (208, 736) {ipe fdisk};
  \pic[ipe mark large, fill=white]
     at (224, 704) {ipe fdisk};
  \pic[ipe mark large, fill=darkgray]
     at (240, 736) {ipe fdisk};
  \node[ipe node, font=\Large]
     at (64, 672) {$G_1$};
  \node[ipe node, font=\Large]
     at (128, 672) {$G_2$};
  \node[ipe node, font=\Large]
     at (192, 672) {$G_3$};
  \node[ipe node, font=\Huge]
     at (256, 704) {. . .};
  \node[ipe node, font=\Large]
     at (320, 672) {$G_d$};
  \node[ipe node, font=\Large]
     at (323.895, 717.398) {$G_{d-1}$};
  \draw[ipe pen fat, ipe dash dashed]
    (362.6667, 730.6667)
     .. controls (362.6667, 741.3333) and (341.3333, 746.6667) .. (330.6667, 744)
     .. controls (320, 741.3333) and (320, 730.6667) .. (320, 725.3333)
     .. controls (320, 720) and (320, 720) .. (320, 714.6667)
     .. controls (320, 709.3333) and (320, 698.6667) .. (330.6667, 701.3333)
     .. controls (341.3333, 704) and (362.6667, 720) .. cycle;
  \filldraw[draw=red, ipe pen fat, fill=darkgray]
    (358.3048, 719.1079)
     -- (383.7055, 704.3703);
  \pic[ipe mark large, fill=darkgray]
     at (384, 704) {ipe fdisk};
  \filldraw[draw=red, ipe pen fat, fill=darkgray]
    (348.8779, 710.1407)
     -- (383.7886, 703.0978);
  \filldraw[draw=red, ipe pen fat, fill=darkgray]
    (364.0388, 730.0229)
     -- (385.3779, 704.2612);
  \filldraw[draw=red, ipe pen fat, fill=darkgray]
    (341.3156, 704.9987)
     -- (383.412, 702.607);
  \pic[ipe mark large, fill=darkgray]
     at (384, 704) {ipe fdisk};
\end{tikzpicture}

\caption{Recursive definition of $G_d$.}
\label{fig:graphicmatroid}
\end{figure}

Consider now the arm-pulling schedule constructed by the greedy strategy. Let $T_p = E_p \setminus E_{p-1}$ be the new edges added at each step $p \in [d]$ in the recursive definition of $G_d$ (assuming that $E_0 = \emptyset$). Notice that for any integers $d \geq p_1 > p_2 \geq 1$ the edges of $T_{p_1}$ correspond to arms of higher mean reward than the edges of $T_{p_2}$. Therefore, the algorithm produces a periodic schedule of period $d$ as follows: Initially, the algorithm plays the $d$ arms of group $T_d$, collecting reward $d\left( 1 - \frac{\epsilon}{d}\right) = d - \epsilon$. Notice that, by construction, these edges form a spanning tree in $G_{d}$ and, thus, no additional arm can be played at the same time step. In the second time step of the period, the arms of $T_d$ are blocked and the algorithm plays the arms of $T_{d-1}$ collecting $d-1-\epsilon$ reward. Again, this is the maximum reward independent set of $G_d$ among the available arms. The algorithm proceeds similarly in the following steps and collects an average reward of 
$$
\frac{\sum^d_{p=1}(p-\epsilon)}{d} = \frac{d\cdot(d+1)/2-d\epsilon}{d} = \frac{d+1}{2} - \epsilon.
$$
In the above example, the optimal arm-pulling sequence is to play at each time $t \in [T]$, one arm of each group $T_p$ for $p \in [d]$. Notice that by construction of the delays and at each time step, there always exists at least one arm per group that is available. Moreover, by definition of the graph $G_d$, any such selection of arms never contains a circuit and, thus, it is an independent set of the graphic matroid. The expected reward collected by the optimal algorithm at each step is $d - \epsilon\sum_{p \in [d]}\frac{1}{p} = d - \epsilon H(d)$, were $H(d) = \sum_{p \in [d]}\frac{1}{p}$. 

In the above example, the ratio between the average reward collected by the greedy strategy and the optimal reward for $\epsilon \to 0$ becomes
$$
\lim_{\epsilon \to 0} \frac{\frac{d+1}{2} - \epsilon}{d - \epsilon H(d)} = \frac{1}{2} + \frac{1}{2d}.
$$
Therefore, by choosing large enough $d$, we can bring the approximation ratio of the above example arbitrarily close to $\frac{1}{2}$.
\end{proof}

\end{document}