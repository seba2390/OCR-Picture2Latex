\section{Preliminaries on Matroids and Submodular Functions}\label{sec:definition}

\paragraph{Continuous extensions and the correlation gap of submodular functions.}

Consider any set function $f : 2^{\A} \rightarrow \mathbb{R}_{\geq 0}$ over a ground set $\A$. Recall that $f$ is submodular, if $\forall S,T \subseteq \A$ we have $f(S \cup T) + f(S \cap T) \leq f(S) + f(T)$. For any point $\x \in [0,1]^k$, we denote by $S \sim \x$ the random set $S \subseteq \A$, such that $\Pro{i \in S} = x_i$. We consider two canonical continuous extensions of a set function: 

\begin{definition}[Continuous extensions]
For any set function $f$ the {\em multi-linear extension} is
\begin{align*}
    F(\x) = \Ex{S \sim \x}{f(S)} = \sum_{S \subseteq \A} f(S) \prod_{i \in S} x_i \prod_{i \notin S} (1-x_i).
\end{align*}
Moreover, the {\em concave closure} is defined as
\begin{align*}
    f^+(\x) = \max_{\alpha}\{\sum_{S \subseteq \A} \alpha_S f(S)~|~\sum_{S \subseteq \A} \alpha_S \bm{1}_S = \x, \sum_{S \subseteq \A} \alpha_S = 1, \alpha \succeq 0\}.
\end{align*}
\end{definition}

\begin{lemma}[Correlation gap \cite{CCPV07}] \label{lem:correlationgap}
Let $f: 2^k \rightarrow \mathbb{R}_{\geq 0}$ be a monotone (non-decreasing) submodular function. Then for any point $\x \in [0,1]^k$, we have
$
F(\x) \leq f^+(\x) \leq \left(1 - \frac{1}{e}\right)^{-1} F(\x).
$
\end{lemma}


\paragraph{Matroid polytope and the weighted rank function.} Consider a matroid $\M = (\A, \I)$, where $\A$ is the {\em ground set} and $\I$ is the family of {\em independent sets}
\footnote{Any subset of the ground set $\A$ that is not independent is called {\em dependent}. Any maximal independent set of a matroid, namely, a set $B \in \I$ such that for every $e \in \A \backslash B$, the set $B \cup \{e\}$ is dependent, is called a {\em basis}. Any minimal dependent set, that is, a set $C \notin \I$ such that for each $e \in C$ it holds $C\backslash \{e\} \in \I$ is called a {\em circuit}.}. Recall that in any matroid, the family $\I$ satisfies the following two properties: (i) Every subset of an independent set (including the empty set) is an independent set, namely, if $S' \subset S \subseteq \A$ and $S \in \I$, then $S' \in \I$ ({\em hereditary property}). (ii) Let $S, S' \subseteq \A$ be two independent sets with $|S| < |S'|$, then there exists some $e \in S' \backslash S$ such that $S \cup \{e\} \in \I$ ({\em augmentation property}). See \cite{schrijver03, oxley06} for more details on matroids.

We assume that access to $\M$ is given through an {\em independence oracle} \cite{hausmann81, robinson80}, namely, a black-box routine that, given a set $S \subseteq \A$, answers whether $S$ is an independent set of $\M$. 
For any set $R \subset \A$ we define the {\em restriction} of $\M$ to $R$, denoted by $\M | R$, to be the matroid $\M | R = (R, \{I \in \I~|~I \subseteq R\})$. Every matroid $\M$ is associated with a {\em rank} function\footnote{The rank is {\em monotone} non-decreasing, {\em submodular}, and satisfies $\rk(S) \leq |S|$, $\forall S \subseteq \A$ (see \cite{oxley06}).} $\rk: 2^{\A} \rightarrow \mathbb{N}$, such that for any $S\subseteq \A$, $\rk(S)$ denotes the maximum size of an independent set contained in $S$. Let $\x(S) = \sum_{e \in S} x_e$ for some vector $\x \in \mathbb{R}^k$. For any matroid $\M = (\A,\I)$, the {\em matroid polytope} is defined as

$$\mathcal{P}(\M) \equiv \left\{ \x(S) \leq \rk(S), \forall S \in 2^{\A}, S \neq \emptyset \text{ and } \x \succeq 0 \right\}.
$$
It can be proved \cite{schrijver03} that the above polytope is the convex hull of the indicator vectors of all independent sets. This fact immediately leads to the following lemma: 

\begin{lemma}\label{lem:characteristic}
For any matroid $\M=(\A, \I)$ and point $\x \in \mathcal{P}(\M)$, there exists a collection of $k = |\A|$ independent sets $\I(\x) = \{ I_1, \dots, I_k\} \subseteq \I$ and a probability distribution over $\I(\x)$ such that $\Prob{I \sim \I(\x)}{i \in I} = x_i$, i.e., an element $i$ belongs to a sampled set with marginal probability equal to $x_i$.
\end{lemma}
Given any non-negative linear {\em weight} vector $\w \in \mathbb{R}^k_{\geq 0}$, the problem of computing a maximum weight independent set can be solved optimally by the standard greedy algorithm: Starting from the empty set $S = \emptyset$, add each ground element $e \in \A$ to the set $S$ in a non-increasing order of weights, as long as the set $S \cup \{e\}$ does not contain a circuit. Given a matroid $\M=(\A,\I)$ and a weight vector $\w$, the function $f_{\M,\w}(S) = \max_{I \in \I, I \subseteq S}\{\w(I)\}$ is called the {\em weighted rank function} of $\M$ and returns the weight of the maximum independent set of the restriction $\M|S$.

\begin{lemma}[Weighted rank function \cite{CCPV07}] \label{lem:weightedrank}
For any matroid $\M$ and non-negative weight vector $\w$, the function $f_{\M, \w}(S) = \max_{I \in \I, I \subseteq S}\{\w(I)\}$ is monotone (non-decreasing) submodular.
\end{lemma}




%\begin{lemma}[Correlation Gap] %\label{lem:correlationggap}
%Let $f$ be a monotone (non-decreasing) submodular function. For the sets $S$ and $T_1, \dots, T_k$ constructed as described above, we have:
%\begin{align*}
%    \Ex{S \sim \x}{f(S)} \geq \left(1 - \frac{1}{e}\right) \Ex{T \sim \I(\x)}{f(T)}.
%\end{align*}
%\end{lemma}

 