\section{Additional Results}

\subsection{Tight example for the naive greedy algorithm} \label{appendix:tightexample}

\begin{restatable}{lemma}{restateTightexample}\label{lem:tightexample}
For any $d \geq 2$, there exists an instance of the full-information variant of the \mbb problem (where the mean rewards are known a priori) such that the greedy strategy that plays a maximum mean reward independent set among the available arms collects a $\left(\frac{1}{2} + \frac{1}{2d}\right)$-fraction of the optimal expected reward.
\end{restatable}


\begin{proof}
We consider an infinite time horizon and a graphic matroid based on the graph $G_d = (V_d, E_d)$, which is recursively defined as follows: Let $G_1 = (V_1, E_1)$ with $V_1 = \{u,v\}$, $E_1 = \{\{u,v\}\}$ and assume that the arm associated with edge $\{u,v\}$ has delay $1$ and mean reward $1-\epsilon$, for some $\epsilon > 0$. For the graph $G_d = (V_d, E_d)$, we have $V_d = V_{d-1} \cup \{u_d\}$ and $E_d = E_{d-1} \cup \{\{u,u_d\}, \forall u \in V_{d-1}\}$ (namely, $G_d$ is essentially the result of the join operation between $G_{d-1}$ and a single vertex graph). The arms that are associated with the edges of $E_{d} \setminus E_{d-1}$ all have delay equal to $d$ and mean reward equal to $1- \frac{\epsilon}{d}$. The above recursive construction is illustrated in Figure \ref{fig:graphicmatroid}.

\begin{figure}[h]
\centering
\tikzstyle{ipe stylesheet} = [
  ipe import,
  even odd rule,
  line join=round,
  line cap=butt,
  ipe pen normal/.style={line width=0.4},
  ipe pen heavier/.style={line width=0.8},
  ipe pen fat/.style={line width=1.2},
  ipe pen ultrafat/.style={line width=2},
  ipe pen normal,
  ipe mark normal/.style={ipe mark scale=3},
  ipe mark large/.style={ipe mark scale=5},
  ipe mark small/.style={ipe mark scale=2},
  ipe mark tiny/.style={ipe mark scale=1.1},
  ipe mark normal,
  /pgf/arrow keys/.cd,
  ipe arrow normal/.style={scale=7},
  ipe arrow large/.style={scale=10},
  ipe arrow small/.style={scale=5},
  ipe arrow tiny/.style={scale=3},
  ipe arrow normal,
  /tikz/.cd,
  ipe arrows, % update arrows
  <->/.tip = ipe normal,
  ipe dash normal/.style={dash pattern=},
  ipe dash dashed/.style={dash pattern=on 4bp off 4bp},
  ipe dash dotted/.style={dash pattern=on 1bp off 3bp},
  ipe dash dash dotted/.style={dash pattern=on 4bp off 2bp on 1bp off 2bp},
  ipe dash dash dot dotted/.style={dash pattern=on 4bp off 2bp on 1bp off 2bp on 1bp off 2bp},
  ipe dash normal,
  ipe node/.append style={font=\normalsize},
  ipe stretch normal/.style={ipe node stretch=1},
  ipe stretch normal,
  ipe opacity 10/.style={opacity=0.1},
  ipe opacity 30/.style={opacity=0.3},
  ipe opacity 50/.style={opacity=0.5},
  ipe opacity 75/.style={opacity=0.75},
  ipe opacity opaque/.style={opacity=1},
  ipe opacity opaque,
]
\definecolor{red}{rgb}{1,0,0}
\definecolor{green}{rgb}{0,1,0}
\definecolor{blue}{rgb}{0,0,1}
\definecolor{yellow}{rgb}{1,1,0}
\definecolor{orange}{rgb}{1,0.647,0}
\definecolor{gold}{rgb}{1,0.843,0}
\definecolor{purple}{rgb}{0.627,0.125,0.941}
\definecolor{gray}{rgb}{0.745,0.745,0.745}
\definecolor{brown}{rgb}{0.647,0.165,0.165}
\definecolor{navy}{rgb}{0,0,0.502}
\definecolor{pink}{rgb}{1,0.753,0.796}
\definecolor{seagreen}{rgb}{0.18,0.545,0.341}
\definecolor{turquoise}{rgb}{0.251,0.878,0.816}
\definecolor{violet}{rgb}{0.933,0.51,0.933}
\definecolor{darkblue}{rgb}{0,0,0.545}
\definecolor{darkcyan}{rgb}{0,0.545,0.545}
\definecolor{darkgray}{rgb}{0.663,0.663,0.663}
\definecolor{darkgreen}{rgb}{0,0.392,0}
\definecolor{darkmagenta}{rgb}{0.545,0,0.545}
\definecolor{darkorange}{rgb}{1,0.549,0}
\definecolor{darkred}{rgb}{0.545,0,0}
\definecolor{lightblue}{rgb}{0.678,0.847,0.902}
\definecolor{lightcyan}{rgb}{0.878,1,1}
\definecolor{lightgray}{rgb}{0.827,0.827,0.827}
\definecolor{lightgreen}{rgb}{0.565,0.933,0.565}
\definecolor{lightyellow}{rgb}{1,1,0.878}
\definecolor{black}{rgb}{0,0,0}
\definecolor{white}{rgb}{1,1,1}
\begin{tikzpicture}[ipe stylesheet]
  \draw[red, ipe pen fat]
    (64, 704)
     -- (96, 704)
     -- (96, 704);
  \pic[ipe mark large, fill=darkgray]
     at (64, 704) {ipe fdisk};
  \pic[ipe mark large, fill=darkgray]
     at (96, 704) {ipe fdisk};
  \filldraw[draw=red, ipe pen fat, fill=darkgray, ipe opacity 75]
    (144, 736)
     -- (128, 704)
     -- (128, 704);
  \filldraw[ipe pen fat, fill=white]
    (128, 704)
     -- (160, 704);
  \filldraw[red, ipe pen fat]
    (144, 736)
     -- (160, 704);
  \pic[ipe mark large, fill=darkgray]
     at (144, 736) {ipe fdisk};
  \pic[ipe mark large, fill=white]
     at (128, 704) {ipe fdisk};
  \pic[ipe mark large, fill=white]
     at (160, 704) {ipe fdisk};
  \filldraw[ipe pen fat, fill=white]
    (192, 704)
     -- (224, 704);
  \filldraw[ipe pen fat, fill=white]
    (192, 704)
     -- (208, 736);
  \filldraw[ipe pen fat, fill=white]
    (208, 736)
     -- (224, 704);
  \filldraw[draw=red, ipe pen fat, fill=darkgray]
    (208, 736)
     -- (240, 736);
  \filldraw[draw=red, ipe pen fat, fill=darkgray]
    (192, 704)
     -- (240, 736);
  \filldraw[draw=red, ipe pen fat, fill=black]
    (224, 704)
     -- (240, 736);
  \pic[ipe mark large, fill=white]
     at (192, 704) {ipe fdisk};
  \pic[ipe mark large, fill=white]
     at (208, 736) {ipe fdisk};
  \pic[ipe mark large, fill=white]
     at (224, 704) {ipe fdisk};
  \pic[ipe mark large, fill=darkgray]
     at (240, 736) {ipe fdisk};
  \node[ipe node, font=\Large]
     at (64, 672) {$G_1$};
  \node[ipe node, font=\Large]
     at (128, 672) {$G_2$};
  \node[ipe node, font=\Large]
     at (192, 672) {$G_3$};
  \node[ipe node, font=\Huge]
     at (256, 704) {. . .};
  \node[ipe node, font=\Large]
     at (320, 672) {$G_d$};
  \node[ipe node, font=\Large]
     at (323.895, 717.398) {$G_{d-1}$};
  \draw[ipe pen fat, ipe dash dashed]
    (362.6667, 730.6667)
     .. controls (362.6667, 741.3333) and (341.3333, 746.6667) .. (330.6667, 744)
     .. controls (320, 741.3333) and (320, 730.6667) .. (320, 725.3333)
     .. controls (320, 720) and (320, 720) .. (320, 714.6667)
     .. controls (320, 709.3333) and (320, 698.6667) .. (330.6667, 701.3333)
     .. controls (341.3333, 704) and (362.6667, 720) .. cycle;
  \filldraw[draw=red, ipe pen fat, fill=darkgray]
    (358.3048, 719.1079)
     -- (383.7055, 704.3703);
  \pic[ipe mark large, fill=darkgray]
     at (384, 704) {ipe fdisk};
  \filldraw[draw=red, ipe pen fat, fill=darkgray]
    (348.8779, 710.1407)
     -- (383.7886, 703.0978);
  \filldraw[draw=red, ipe pen fat, fill=darkgray]
    (364.0388, 730.0229)
     -- (385.3779, 704.2612);
  \filldraw[draw=red, ipe pen fat, fill=darkgray]
    (341.3156, 704.9987)
     -- (383.412, 702.607);
  \pic[ipe mark large, fill=darkgray]
     at (384, 704) {ipe fdisk};
\end{tikzpicture}

\caption{Recursive definition of $G_d$.}
\label{fig:graphicmatroid}
\end{figure}

Consider now the arm-pulling schedule constructed by the greedy strategy. Let $T_p = E_p \setminus E_{p-1}$ be the new edges added at each step $p \in [d]$ in the recursive definition of $G_d$ (assuming that $E_0 = \emptyset$). Notice that for any integers $d \geq p_1 > p_2 \geq 1$ the edges of $T_{p_1}$ correspond to arms of higher mean reward than the edges of $T_{p_2}$. Therefore, the algorithm produces a periodic schedule of period $d$ as follows: Initially, the algorithm plays the $d$ arms of group $T_d$, collecting reward $d\left( 1 - \frac{\epsilon}{d}\right) = d - \epsilon$. Notice that, by construction, these edges form a spanning tree in $G_{d}$ and, thus, no additional arm can be played at the same time step. In the second time step of the period, the arms of $T_d$ are blocked and the algorithm plays the arms of $T_{d-1}$ collecting $d-1-\epsilon$ reward. Again, this is the maximum reward independent set of $G_d$ among the available arms. The algorithm proceeds similarly in the following steps and collects an average reward of 
$$
\frac{\sum^d_{p=1}(p-\epsilon)}{d} = \frac{d\cdot(d+1)/2-d\epsilon}{d} = \frac{d+1}{2} - \epsilon.
$$
In the above example, the optimal arm-pulling sequence is to play at each time $t \in [T]$, one arm of each group $T_p$ for $p \in [d]$. Notice that by construction of the delays and at each time step, there always exists at least one arm per group that is available. Moreover, by definition of the graph $G_d$, any such selection of arms never contains a circuit and, thus, it is an independent set of the graphic matroid. The expected reward collected by the optimal algorithm at each step is $d - \epsilon\sum_{p \in [d]}\frac{1}{p} = d - \epsilon H(d)$, were $H(d) = \sum_{p \in [d]}\frac{1}{p}$. 

In the above example, the ratio between the average reward collected by the greedy strategy and the optimal reward for $\epsilon \to 0$ becomes
$$
\lim_{\epsilon \to 0} \frac{\frac{d+1}{2} - \epsilon}{d - \epsilon H(d)} = \frac{1}{2} + \frac{1}{2d}.
$$
Therefore, by choosing large enough $d$, we can bring the approximation ratio of the above example arbitrarily close to $\frac{1}{2}$.
\end{proof}