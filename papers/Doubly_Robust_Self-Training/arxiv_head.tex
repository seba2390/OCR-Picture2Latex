\documentclass{article}
\usepackage[T1]{fontenc}
\usepackage{geometry}
\geometry{verbose,tmargin=1in,bmargin=1in,lmargin=1in,rmargin=1in}
\usepackage{hyperref}       % hyperlinks
\usepackage{url}            % simple URL typesetting
\usepackage{booktabs}       % professional-quality tables
\usepackage{amsfonts}       % blackboard math symbols
\usepackage{nicefrac}       % compact symbols for 1/2, etc.
\usepackage{microtype}      % microtypography
\usepackage{xcolor}         % colors
\usepackage{babel}
\usepackage{verbatim}
\usepackage{mathtools}
\usepackage{bm}
\usepackage{natbib}
\usepackage{amsmath}
\usepackage{amssymb}
\usepackage{multicol}
\usepackage{adjustbox}

\usepackage{bm}
\usepackage{enumitem}
 
\usepackage{babel}
\usepackage{subcaption}
\usepackage{algorithmic}
 
\usepackage{amsthm}
\usepackage{bbm}

\usepackage{tabularx}
\usepackage{multirow}
\usepackage{graphicx}
\usepackage{booktabs}
\usepackage{xspace}
\newlength\savewidth
\newcommand\shline{\noalign{\global\savewidth\arrayrulewidth
  \global\arrayrulewidth 1pt}\hline\noalign{\global\arrayrulewidth\savewidth}}
\newcommand{\my}[1]{\textcolor{red}{[Mingyu: #1]}}
\newcommand{\banghua}[1]{\textcolor{red}{[Banghua: #1]}}

\definecolor{citecolor}{HTML}{0071BC}
\hypersetup{colorlinks,linkcolor={red},citecolor={citecolor}}

\newtheorem{lemma}{\textbf{Lemma}}
\newtheorem{theorem}{\textbf{Theorem}}\setcounter{theorem}{0}
\newtheorem{corollary}{\textbf{Corollary}}
\newtheorem{assumption}{\textbf{Assumption}}
\newtheorem{example}{\textbf{Example}}
\newtheorem{definition}{\textbf{Definition}}
\newtheorem{fact}{\textbf{Fact}}
\newtheorem{proposition}{\textbf{Proposition}}\setcounter{theorem}{0}

\theoremstyle{definition}
\newtheorem{remark}{\textbf{Remark}}
\newtheorem{condition}{Condition}
\newtheorem{claim}{\textbf{Claim}}
\usepackage{natbib} 
\bibliographystyle{abbrvnat}
\setcitestyle{authoryear,open={(},close={)}} %Citation-related commands
\makeatother

\usepackage{todonotes}
\usepackage{url}
\usepackage{graphicx}
\usepackage{xspace}
\usepackage{tikz,pgfplots}

\usepgfplotslibrary{statistics}


\newcommand{\system}{\textsc{Groot}\xspace}

\title{Doubly Robust  Self-Training}


% The \author macro works with any number of authors. There are two commands
% used to separate the names and addresses of multiple authors: \And and \AND.
%
% Using \And between authors leaves it to LaTeX to determine where to break the
% lines. Using \AND forces a line break at that point. So, if LaTeX puts 3 of 4
% authors names on the first line, and the last on the second line, try using
% \AND instead of \And before the third author name.


\author{Banghua Zhu, Mingyu Ding, Philip Jacobson, Ming Wu, \\ 
Wei Zhan, Michael I. Jordan, Jiantao Jiao\thanks{Banghua Zhu, Mingyu Ding, Philip Jacobson, Ming Wu,
Wei Zhan, Michael I. Jordan, Jiantao Jiao are with the Department of Electrical Engineering and Computer Sciences, University of California, Berkeley. Email: \{banghua, myding, philip\_jacobson, mingwu, wzhan, jordan, jiantao\}@berkeley.edu.}}


\begin{document}


\maketitle


\begin{abstract}
Self-training is an important technique for solving semi-supervised learning problems.  It leverages unlabeled data by generating pseudo-labels and combining them with a limited labeled dataset for training. The effectiveness of self-training heavily relies on the accuracy of these pseudo-labels. In this paper, we introduce doubly robust self-training, a novel semi-supervised algorithm that provably balances between two extremes. When the pseudo-labels are entirely incorrect, our method reduces to a training process solely using labeled data. Conversely, when the pseudo-labels are completely accurate, our method transforms into a training process utilizing all pseudo-labeled data and labeled data, thus increasing the effective sample size. Through empirical evaluations on both the ImageNet dataset for image classification and the nuScenes autonomous driving dataset for 3D object detection, we demonstrate the superiority of the doubly robust loss over the standard self-training baseline.
\end{abstract}


\section{Introduction}

Semi-supervised learning considers the problem of learning based on a small labeled dataset together with a large unlabeled dataset. This general framework plays an important role in many problems in machine learning, including model fine-tuning, model distillation, self-training, transfer learning and continual learning~\citep{zhu2005semi,pan2010survey, weiss2016survey, gou2021knowledge, de2021continual}. Many of these problems also involve some form of distribute shift, and accordingly, to best utilize the unlabeled data, an additional assumption is that one has access to a teacher model obtained from prior training.  It is important to study the relationships among the datasets and the teacher model.  In this paper, we ask the following question:
\begin{quote}
    \emph{Given a teacher model, a large unlabeled dataset and a small labeled dataset, how can we design a principled learning process that ensures consistent and sample-efficient learning of the true model?}
\end{quote}

Self-training is one widely adopted and popular approach in computer vision and autonomous driving for leveraging information from all three components~\citep{pseudolabel2013,berthelot2019mixmatch,berthelot2019remixmatch,sohn2020fixmatch, xie2020self, jiang2022improving, qi2021offboard}. This approach involves using a teacher model to generate pseudo-labels for all unlabeled data, and then training a new model on a mixture of both pseudo-labeled and labeled data. However, this method can lead to overreliance on the teacher model and can miss important information provided by the labeled data. As a consequence, the self-training approach becomes highly sensitive to the accuracy of the teacher model. Our study demonstrates that even in the simplest scenario of mean estimation, this method can yield significant failures when the teacher model lacks accuracy.



% In the field of computer vision and driving, self-training is a popular and straightforward solution for the above question.  % proposes a framework that combines active learning with auto-labeling to identify rare examples  for the task of  3D object detection.    During the stage of active learning, they aim at identifying rare examples rather than hard examples, since training on hard examples will not improve the model performance. They propose an estimator  of the rareness of the new example based on the flow model trained from existing dataset.    During the stage of auto-labeling, 
 

To overcome this issue, we propose an alternative method that is \emph{doubly robust}---when the covariate distribution of the unlabeled dataset and the labeled dataset matches, the estimator is always  consistent  no matter whether  the  teacher model is accurate or not. On the other hand, when the teacher model is an accurate predictor, the estimator makes full use of the pseudo-labeled dataset and greatly increases the effective sample size. The idea is inspired by and directly related to missing-data inference and causal inference~\citep{rubin1976inference, kang2007demystifying, birhanu2011doubly, ding2018causal},  to semiparametric mean estimation~\citep{zhang2019semi}, and to recent work on prediction-powered inference~\citep{angelopoulos2023prediction}. 

\subsection{Main results}
The proposed algorithm is based on a simple modification of the standard loss for self-training.   Assume that we are given a set of unlabeled samples, $\mathcal{D}_1 = \{X_1,X_2,\cdots, X_m\}$, drawn from a fixed distribution $\mathbb{P}_X$, a set of labeled samples $\mathcal{D}_2 =\{(X_{m+1}, Y_{m+1}), (X_{m+2}, Y_{m+2}), \cdots, (X_{m+n}, Y_{m+n})\}$, drawn from some joint distribution $\mathbb{P}_X\times \mathbb{P}_{Y|X}$, and a teacher model  $\hat f$. Let $\ell_\theta(x, y)$ be a pre-specified loss function that characterizes the prediction error of the estimator with parameter $\theta$ on the given sample $(X, Y)$. Traditional self-training aims at minimizing the combined loss for both labeled and unlabeled samples, where the pseudo-labels for unlabeled samples are generated using $\hat f$:
\begin{align*}
\mathcal{L}^{\mathsf{SL}}_{\mathcal{D}_1,\mathcal{D}_2}(\theta) 
& = \frac{1}{m+n}  \left(\sum_{i=1}^m \ell_\theta(X_i, \hat f(X_i)) + \sum_{i=m+1}^{m+n} \ell_\theta(X_i, Y_i)\right). %\\
% & = \frac{1}{m+n}  \sum_{i=1}^{m+n} \ell_\theta(X_i, \hat f(X_i)) -  \frac{1}{m+n} \sum_{i=m+1}^{m+n} \ell_\theta(X_i, \hat f(X_i))  + \frac{1}{m+n} \sum_{i=m+1}^{m+n} \ell_\theta(X_i, Y_i). 
\end{align*}
Note that this can also be viewed as first using $\hat f$ to predict all the data, and then replacing the originally labeled points with the known labels: 
\begin{align*}
\mathcal{L}^{\mathsf{SL}}_{\mathcal{D}_1,\mathcal{D}_2}(\theta) 
 & = \frac{1}{m+n}  \sum_{i=1}^{m+n} \ell_\theta(X_i, \hat f(X_i)) -  \frac{1}{m+n} \sum_{i=m+1}^{m+n} \ell_\theta(X_i, \hat f(X_i))  + \frac{1}{m+n} \sum_{i=m+1}^{m+n} \ell_\theta(X_i, Y_i). 
\end{align*}
Our proposed doubly robust loss instead replaces the coefficient $1/(m+n)$ with $1/n$ in the last two terms:
\begin{align*}
\mathcal{L}^{\mathsf{DR}}_{\mathcal{D}_1,\mathcal{D}_2}(\theta) 
& = \frac{1}{m+n}  \sum_{i=1}^{m+n} \ell_\theta(X_i, \hat f(X_i)) -  \frac{1}{n} \sum_{i=m+1}^{m+n} \ell_\theta(X_i, \hat f(X_i))  + \frac{1}{n} \sum_{i=m+1}^{m+n} \ell_\theta(X_i, Y_i). 
\end{align*}
This seemingly minor change has a major beneficial effect---the estimator becomes consistent and doubly robust. 
\begin{theorem}[Informal]
    Let $\theta^\star$ be defined as the minimizer  $\theta^\star = \argmin_{\theta} \mathbb{E}_{(X, Y)\sim \mathbb{P}_{X}\times \mathbb{P}_{Y|X}}[\ell_\theta(X, Y)]$. Under certain regularity conditions, we have 
    \begin{align*}
  \| \nabla_\theta \mathcal{L}^{\mathsf{DR}}_{\mathcal{D}_1,\mathcal{D}_2}(\theta^\star) \|_2 \lesssim  
  \begin{cases}
  \sqrt{\frac{d}{m+n}}, & \text{ when } Y \equiv \hat f(X), \\
  \sqrt{\frac{d}{n}}, & \text{otherwise}. 
  \end{cases}
\end{align*}  
On the other hand, there exists instances such that $  \| \nabla_\theta \mathcal{L}^{\mathsf{SL}}_{\mathcal{D}_1,\mathcal{D}_2}(\theta^\star) \|_2\geq C$ always holds true no matter how large $m, n$ are. 
\end{theorem} 
The result shows that the true parameter $\theta^\star$ is also a local minimum of the doubly robust loss, but not a local minimum of the original self-training loss.  We flesh out this comparison for the special example of mean estimation in Section~\ref{sec:mean}, and present empirical results on image and driving datasets in  Section~\ref{sec:empirical}.


\subsection{Related work}

\textbf{Missing-data inference and causal inference.} 
The general problem of causal inference can be formulated as a missing-data inference problem as follows. For each unit in an experiment, at most one of the potential outcomes---the one corresponding to the treatment to
which the unit is exposed---is observed, and the other
potential outcomes are viewed as missing~\citep{holland1986statistics, ding2018causal}. Two of the standard methods for solving this problem are data imputation~\cite{rubin1979using} and propensity
score weighting~\cite{rosenbaum1983central}. A doubly robust causal inference estimator combines
the virtues of these two methods. The estimator is referred to as ``doubly robust'' due to the following property: if the model for imputation is correctly specified then it is  consistent  no matter whether the propensity score model is correctly specified; on the other hand, if the model propensity score model is correctly specified, then it is consistent no matter whether the model for imputation is correctly specified~\citep{scharfstein1999adjusting,  bang2005doubly, birhanu2011doubly, ding2018causal}. 

We note in passing that double machine learning is another methodology that is inspired by the doubly robust paradigm in causal inference~\citep{semenova2017estimation, chernozhukov2018double, chernozhukov2018biased, foster2019orthogonal}. The  problem in double machine learning is related to the classic semiparametric problem of inference for a low-dimensional parameter in the presence of high-dimensional nuisance parameters, which is different goal than the predictive goal characterizing semi-supervised learning. 

The recent work of prediction-powered inference~\citep{angelopoulos2023prediction} focuses on confidence estimation when there are both unlabeled data, labeled data, along with a teacher model. Their focus is the inferential problem of obtaining a confidence set, while ours is the doubly robust property of a point estimator.  Since they focus on confidence estimation, an important, strong, yet biased baseline point-estimate algorithm that directly combines the ground-truth labels and pseudo-labels is not considered in their case. In our paper, we show with both theory and experiments that the proposed doubly-robust estimator achieves better performance than the naive combination of ground-truth labels and pseudo-labels. 

\vspace{0.1in}

\textbf{Self-training.}
Self-training is a popular semi-supervised learning paradigm in which machine-generated pseudo-labels are used for training with unlabeled data \citep{pseudolabel2013,berthelot2019mixmatch,berthelot2019remixmatch,sohn2020fixmatch, zhao2023towards}. To generate these pseudo-labels, a teacher model is pre-trained on a set of labeled data, and its predictions on the unlabeled data are extracted as pseudo-labels. Previous work seeks to address the noisy quality of pseudo-labels in various ways. MixMatch \citep{berthelot2019mixmatch} ensembles pseudo-labels across several augmented views of the input data. ReMixMatch \citep{berthelot2019remixmatch} extends this by weakly augmenting the teacher inputs and strongly augmenting the student inputs. FixMatch \citep{sohn2020fixmatch} uses confidence thresholding to select only high-quality pseudo-labels for student training.

Self-training has  been applied in  both 2D computer vision problems~\citep{liu2021unbiased,NEURIPS2019_d0f4dae8,Tang2021HumbleTT,sohn2020detection,zhou2022} and 3D problems~\citep{park2022detmatch,wang20213dioumatch,li2023dds3d,liu2023hierarchical} object detection. STAC \citep{sohn2020detection} enforces consistency between strongly augmented versions of confidence-filtered pseudo-labels. Unbiased teacher \citep{liu2021unbiased} updates the teacher during training with an exponential moving average (EMA) of the student network weights. Dense Pseudo-Label \citep{zhou2022} replaces box pseudo-labels with the raw output features of the detector to allow the student to learn richer context. In the 3D domain, 3DIoUMatch \citep{wang20213dioumatch} thresholds pseudo-labels using a model-predicted Intersection-over-Union (IoU). DetMatch \citep{park2022detmatch} performs detection in both the 2D and 3D domains and filters pseudo-labels based on 2D-3D correspondence. HSSDA \citep{liu2023hierarchical} extends strong augmentation during training with a patch-based point cloud shuffling augmentation. Offboard3D \citep{qi2021offboard} utilizes multiple frames of temporal context to improve pseudo-label quality.

There has been a limited amount of theoretical analysis of these methods, focusing on semi-supervised methods for mean estimation and linear regression~\citep{zhang2019semi, azriel2022semi}. Our analysis bridges the gap between these analyses and the doubly robust estimators in the causal inference literature. 

 

\section{Doubly Robust Self-Training}

We begin with the case where the marginal distributions for the covariates of the labeled and unlabeled datasets are the same. 
Assume that we are given a set of unlabeled samples, $\mathcal{D}_1 = \{X_1,X_2,\cdots, X_m\}$, drawn from a fixed distribution $\mathbb{P}_X$ supported on $\mathcal{X}$, a set of labeled samples $\mathcal{D}_2 =\{(X_{m+1}, Y_{m+1}), (X_{m+2}, Y_{m+2}), \cdots, (X_{m+n}, Y_{m+n})\}$, drawn from some joint distribution $\mathbb{P}_X\times \mathbb{P}_{Y|X}$ supported on $\mathcal{X}\times\mathcal{Y}$, and a pre-trained model,  $\hat f:\mathcal{X}\mapsto \mathcal{Y}$. Let $\ell_\theta(\cdot, \cdot):\mathcal{X}\times\mathcal{Y}\mapsto \mathbb{R}$ be a pre-specified loss function that characterizes the prediction error of the estimator with parameter $\theta$ on the given sample $(X, Y)$. 
Our target is to find some $\theta^\star\in\Theta$ that satisfies
\begin{align*}
    \theta^\star \in \argmin_{\theta\in\Theta} \mathbb{E}_{(X, Y)\sim \mathbb{P}_{X}\times \mathbb{P}_{Y|X}}[\ell_\theta(X,Y)].
\end{align*}
For a given loss $\ell_\theta(x, y)$, consider a naive estimator that ignores the predictor $\hat f$ and only trains on the labeled samples:
\begin{align*}
\mathcal{L}^{\mathsf{TL}}_{\mathcal{D}_1,\mathcal{D}_2}(\theta) 
& = \frac{1}{n}    \sum_{i=m+1}^{m+n} \ell_\theta(X_i, Y_i).  
\end{align*}
Although naive, this is a safe choice since it is an empirical risk minimizer. As $n\rightarrow \infty$, the loss converges to the population loss. However, it ignores all the information provided in $\hat f$ and the unlabeled dataset, which makes it inefficient when the predictor $\hat f$ is informative.

On the other hand, traditional self-training aims at minimizing the combined loss for both labeled and unlabeled samples, where the pseudo-labels for unlabeled samples are generated using $\hat f$:\footnote{There are several variants of the traditional self-training loss. For example, \citet{xie2020self} introduce an extra weight $(m+n)/n$ on the labeled samples, and add noise to the student model; \citet{sohn2020fixmatch} use confidence thresholding to filter unreliable pseudo-labels. However, both of these alternatives still suffer from the inconsistency issue. In this paper we focus on the simplest form $\mathcal{L}^{\mathsf{SL}}$.  }
\begin{align*}
\mathcal{L}^{\mathsf{SL}}_{\mathcal{D}_1,\mathcal{D}_2}(\theta) 
& = \frac{1}{m+n}  \left(\sum_{i=1}^m \ell_\theta(X_i, \hat f(X_i)) + \sum_{i=m+1}^{m+n} \ell_\theta(X_i, Y_i)\right) \\
& = \frac{1}{m+n}  \sum_{i=1}^{m+n} \ell_\theta(X_i, \hat f(X_i)) -  \frac{1}{m+n} \sum_{i=m+1}^{m+n} \ell_\theta(X_i, \hat f(X_i))  + \frac{1}{m+n} \sum_{i=m+1}^{m+n} \ell_\theta(X_i, Y_i). 
\end{align*}

As is shown by the last equality, the self-training loss can be viewed as first using $\hat f$ to predict all the samples (including the labeled samples) and computing the average loss, then replacing that part of the loss corresponding to the labeled samples  with the loss on the original labels. Although the loss uses the information arising from the unlabeled samples and $\hat f$, the performance can be poor when the predictor is not accurate.

We propose an alternative loss, which simply replaces the weight $1/(m+n)$ in the last two terms with $1/n$:
\begin{align}
\mathcal{L}^{\mathsf{DR}}_{\mathcal{D}_1,\mathcal{D}_2}(\theta) 
& = \frac{1}{m+n}  \sum_{i=1}^{m+n} \ell_\theta(X_i, \hat f(X_i)) -  \frac{1}{n} \sum_{i=m+1}^{m+n} \ell_\theta(X_i, \hat f(X_i))  + \frac{1}{n} \sum_{i=m+1}^{m+n} \ell_\theta(X_i, Y_i).  \label{eq:dr}
\end{align}

As we will show later,  this is a doubly robust estimator. We provide an intuitive interpretation here:
\begin{itemize}[leftmargin=24pt, itemsep=4pt]
    \vspace{-4pt}
    \item In the case when the given predictor is perfectly accurate, i.e., $\hat f(X) \equiv Y$ always holds (which also means that $Y |X=x$ is a deterministic function of $x$), the last two terms cancel, and the loss minimizes the average loss,  $\frac{1}{m+n}  \sum_{i=1}^{m+n} \ell_\theta(X_i, \hat f(X_i))$, on all of the provided data.  The effective sample size is $m+n$, compared with effective sample size $n$ for training only on a labeled dataset using $\mathcal{L}^{\mathsf{TL}}$. In this case, the loss $\mathcal{L}^{\mathsf{DR}}$ is much better than $\mathcal{L}^{\mathsf{TL}}$, and comparable to $\mathcal{L}^{\mathsf{SL}}$. 
    
    We may as well relax the assumption of $\hat f(X) = Y$  to $\mathbb{E}[\ell_\theta(X, \hat f(X))] = \mathbb{E}[\ell_\theta(X, Y)]$. As $n$ grows larger, the loss is  approximately minimizing the average loss  $\frac{1}{m+n}  \sum_{i=1}^{m+n} \ell_\theta(X_i, \hat f(X_i))$.
    \item On the other hand, no matter how bad  the  given predictor is,  the difference between the first two terms vanishes as either of  $m, n$ goes to infinity since the 
    labeled samples $X_{m+1},\cdots, X_{m+n}$ arise from the same distribution as $X_1,\cdots, X_m$. Thus asymptotically the loss minimizes $ \frac{1}{n} \sum_{i=m+1}^{m+n} \ell_\theta(X_i, Y_i)$, which discards the bad predictor $\hat f$  and focuses only on the labeled dataset. Thus, in this case the loss $\mathcal{L}^{\mathsf{DR}}$ is much better than $\mathcal{L}^{\mathsf{SL}}$, and comparable to $\mathcal{L}^{\mathsf{TL}}$.
\end{itemize}
This loss is appropriate only when the covariate distributions between labeled and unlabeled samples match. In the case where there is a distribution mismatch, we propose an alternative loss; see Section~\ref{sec:mismatch}. 


\subsection{Motivating example: Mean estimation}\label{sec:mean}
 As a concrete example, in the case of one-dimensional mean estimation  we take $\ell_\theta(X, Y) = (\theta-Y)^2$. Our target is to find some $\theta^\star$ that satisfies
\begin{align*}
    \theta^\star = \argmin_{\theta} \mathbb{E}_{(X, Y)\sim \mathbb{P}_{X}\times \mathbb{P}_{Y|X}}[(\theta-Y)^2].
\end{align*}
One can   see that $\theta^\star = \mathbb{E}[Y]$. In this case, the loss for training only on labeled data  becomes
\begin{align*}
\mathcal{L}^{\mathsf{TL}}_{\mathcal{D}_1,\mathcal{D}_2}(\theta) & =   \frac{1}{n}    \sum_{i=m+1}^{m+n} (\theta- Y_i)^2.
\end{align*}
Moreover, the optimal parameter is $\hat \theta_{\mathsf{TL}} =  \frac{1}{n}\sum_{i=m+1}^{m+n} Y_i$, which is a simple empirical average over all observed $Y$'s.

For a given pre-existing predictor $\hat f$, the loss for  self-training  becomes
\begin{align*}
\mathcal{L}^{\mathsf{SL}}_{\mathcal{D}_1,\mathcal{D}_2}(\theta) & =   \frac{1}{m+n}  \left(\sum_{i=1}^m (\theta-\hat f(X_i))^2 + \sum_{i=m+1}^{m+n} (\theta- Y_i)^2\right).
\end{align*}
It is straightforward to see that the minimizer of the loss is the unweighted average between the unlabeled predictors $\hat f(X_i)$'s and the labeled $Y_i$'s:  $$\theta^\star_{\mathsf{SL}} = \frac{1}{m+n}\left(  \sum_{i=1}^m \hat f(X_i)+ \sum_{i=m+1}^{m+n} Y_i\right). $$
In the case of $m\gg n$, the mean estimator is almost the same as the average of all the predicted values on the unlabeled dataset, which can be far from $\theta^\star$ when the predictor $\hat f$ is inaccurate.


On the other hand, for the proposed doubly robust estimator, we have
\begin{align*}
\mathcal{L}^{\mathsf{DR}}_{\mathcal{D}_1,\mathcal{D}_2}(\theta) 
& = \frac{1}{m+n}  \sum_{i=1}^{m+n} (\theta- \hat f(X_i))^2 -  \frac{1}{n} \sum_{i=m+1}^{m+n} (\theta- \hat f(X_i))^2  + \frac{1}{n} \sum_{i=m+1}^{m+n} (\theta-  Y_i)^2  \\
& = \frac{1}{m+n}  \sum_{i=1}^{m+n} (\theta- \hat f(X_i))^2 + \frac{1}{n} \sum_{i=m+1}^{m+n} 2 (\hat f(X_i)-Y_i)\theta + Y_i^2 - \hat f(X_i)^2.
\end{align*}
Note that the loss is still convex, and we have
\begin{align*}  
\theta^\star_{\mathsf{DR}}= \frac{1}{m+n}  \sum_{i=1}^{m+n} \hat f(X_i) - \frac{1}{n}  \sum_{i=m+1}^{m+n} ( \hat f(X_i)-Y_i).
\end{align*}
This recovers the estimator in  prediction-powered inference~\citep{angelopoulos2023prediction}. Assume that $\hat f$ is independent of the labeled data. We can calculate the mean-squared error of the three estimators as follows.


% Need another prop suggesting real meaning of doubly robust.
\begin{proposition}\label{prop:mean_upper}
Let $\Var[{\hat f(X)-Y}] = \mathbb{E}[(\hat f(X)-Y)^2 - \mathbb{E}[(\hat f(X)-Y)]^2]$. We have
   \begin{align*}
    \mathbb{E}[(\theta^\star -  \hat \theta_{\mathsf{TL}})^2] & =  \frac{1}{n} \mathsf{Var}[Y], \\
      \mathbb{E}[(\theta^\star -  \hat \theta_{\mathsf{SL}})^2] &  \leq \frac{2m^2}{(m+n)^2} \mathbb{E}[(\hat f(X)-Y)]^2 + \frac{2m}{(m+n)^2}\Var[{\hat f(X)-Y}]  + \frac{2n}{(m+n)^2} \mathsf{Var}[Y], \\ 
   \mathbb{E}[(\theta^\star -  \hat \theta_{\mathsf{DR}})^2]  & \leq  2\min\Bigg(\frac{1}{n} \mathsf{Var}[Y]+ \frac{m+2n}{(m+n)n } \mathsf{Var}[\hat f(X)], \frac{m+2n}{(m+n)n }\Var[{\hat f(X)-Y}]  + \frac{1}{m+n} \mathsf{Var}[Y]\Bigg).
\end{align*} 
 \end{proposition}
The proof is deferred to Appendix~\ref{proof:mean_upper}. The proposition illustrates the double-robustness of $\hat \theta_{\mathsf{DR}}$---no matter how poor the estimator $\hat f(X)$ is, the rate is always upper bounded by $\frac{4}{n} (\mathsf{Var}[Y]+ \mathsf{Var}[\hat f(X)])$. On the other hand, when $\hat f(X)$ is an accurate estimator of $Y$ (i.e., $\Var[{\hat f(X)-Y}]$ is small), the rate can be improved to $ \frac{2}{m+n} \mathsf{Var}[Y]$. In contrast, the self-training loss always has a non-vanishing term, $\frac{2m^2}{(m+n)^2} \mathbb{E}[(\hat f(X)-Y)]^2$, when $m\gg n$, unless the predictor $\hat f$ is accurate. 

On the other hand, when $\hat f(x) = \hat \beta_{(-1)}^\top x + \hat\beta_{1}$ is a linear predictor trained on the labeled data with $\hat \beta = \argmin_{\beta=[\beta_1, \beta_{(-1)}]} \frac{1}{n}    \sum_{i=m+1}^{m+n} (\beta_{(-1)}^\top X_i +  \beta_{1}- Y_i)^2$, our estimator reduces to the semi-supervised mean estimator in~\citet{zhang2019semi}. Let $\tilde X = [1, X]$. In this case, we also know that the self-training reduces to training only on labeled data, since $\hat\theta_{\mathsf{TL}}$ is also the minimizer of the self-training loss. We have the following result  that reveals the superiority of the doubly robust estimator compared to the other two options. 

 \begin{proposition}[\citep{zhang2019semi}]\label{prop:mean_semi}
We establish the asymptotic behavior of various estimators when   $\hat f$  is a linear predictor trained on the labeled data:
\begin{itemize}
\item Training only on labeled data $\hat \theta_{\mathsf{TL}}$ is equivalent to self-training $\hat\theta_{\mathsf{SL}}$, which gives unbiased estimator but with larger variance:
\begin{align*}
    & \sqrt{n}(\hat \theta_{\mathsf{TL}} - \theta^\star) \rightarrow \mathcal{N}(0, \mathbb{E}[(Y-\beta^\top \tilde X)^2] + \beta_{(-1)}^\top \Sigma\beta_{(-1)}).
\end{align*}

\item Doubly Robust  $\hat \theta_{\mathsf{DR}}$ is unbiased with smaller variance:
\begin{align*}
    &\sqrt{n}(\hat \theta_{\mathsf{DR}} - \theta^\star) \rightarrow \mathcal{N}(0,  \mathbb{E}[(Y-\beta^\top \tilde X)^2] + \frac{n}{m+n} \beta_{(-1)}^\top \Sigma \beta_{(-1)}).
\end{align*}
\end{itemize}
Here $\beta = \argmin_{\beta} \mathbb{E}[(Y-\beta^\top \tilde X)^2]$ and  $\Sigma = \mathbb{E}[(X-\mathbb{E}[X])(X-\mathbb{E}[X])^\top]$.
\end{proposition}

\subsection{Guarantee for general loss} 
In the general case, 
we  show that the doubly robust loss function continues to exhibit desirable properties. In particular,  as $n,m$ goes to infinity, the global minimum of the original loss is also a critical point of the new doubly robust loss, no matter how inaccurate the predictor $\hat f$.  

Let $\theta^\star$ be the minimizer of $\mathbb{E}_{\mathbb{P}_{X, Y}}[\ell_\theta(X, Y)]$. Let $\hat f$ be a pre-existing model that does not depend on the datasets $\mathcal{D}_1, \mathcal{D}_2$. 
We also make the following regularity assumptions.
\begin{assumption}\label{ass:diff}
The loss $\ell_\theta(x, y)$ is  differentiable at $\theta^\star$ for any $x, y$.
\end{assumption}
\begin{assumption}\label{ass:mom}
The random variables $\nabla_\theta \ell_\theta(X, \hat f(X)) $ and  $\nabla_\theta \ell_\theta(X, Y)$ have bounded first and second moments.  
\end{assumption}
Given this assumption, we denote $\Sigma_{\theta}^{Y-\hat f} = \mathsf{Cov}[\nabla_\theta \ell_\theta(X, \hat f(X)) - \nabla_\theta \ell_\theta(X, Y)]$ and let $\Sigma_{\theta}^{\hat f} = \mathsf{Cov}[\nabla_\theta \ell_\theta(X, \hat f(X))]$, $\Sigma_{\theta}^{Y} = \mathsf{Cov}[\nabla_\theta \ell_\theta(X, Y)]$. 
 
% We the  a performance guarantee for the  loss function $\mathcal{L}^{\mathsf{DR1}}_{\mathcal{D}_1,\mathcal{D}_2}(\theta)$ when $\ell_\theta$ is convex. 
\begin{theorem}\label{thm:general} Under Assumptions~\ref{ass:diff} and~\ref{ass:mom}, we have that with probability at least $1-\delta$,
% \my{the width of the Eq exceeds the limit} Banghua: fixed, thx!
\begin{align*}
  \| \nabla_\theta \mathcal{L}^{\mathsf{DR}}_{\mathcal{D}_1,\mathcal{D}_2}(\theta^\star) \|_2 & \leq C   \min\Bigg(\|\Sigma_{\theta^\star}^{\hat f}\|_2\sqrt{\frac{d}{(m+n) \delta}} + \|\Sigma_{\theta^\star}^{Y-\hat f}\|_2\sqrt{\frac{d}{n \delta}},  \\ 
  &  \qquad \qquad \|\Sigma_{\theta^\star}^{\hat f}\|_2\left(\sqrt{\frac{d}{(m+n) \delta}} + \sqrt{\frac{d}{n\delta}} \right) + \|\Sigma_{\theta^\star}^{Y}\|_2\sqrt{\frac{d}{n \delta}}\Bigg),
\end{align*} 
where $C$ is a universal constant, and $\mathcal{L}^{\mathsf{DR}}_{\mathcal{D}_1,\mathcal{D}_2}$ is defined in Equation (\ref{eq:dr}).
\end{theorem}

The proof is deferred to Appendix~\ref{proof:general_guarantee}. From the example of mean estimation we know that one can design instances such that $   \| \nabla_\theta \mathcal{L}^{\mathsf{SL}}_{\mathcal{D}_1,\mathcal{D}_2}(\theta^\star) \|_2  \geq C$ for some positive constant $C$. 

When the loss $\nabla_\theta \mathcal{L}^{\mathsf{DR}}_{\mathcal{D}_1,\mathcal{D}_2}$ is convex, the global minimum of $\nabla_\theta \mathcal{L}^{\mathsf{DR}}_{\mathcal{D}_1,\mathcal{D}_2}$ converges to $\theta^\star$ as both $m, n$ go to infinity. When the loss $\nabla_\theta \mathcal{L}^{\mathsf{DR}}_{\mathcal{D}_1,\mathcal{D}_2}$ is strongly convex, it also implies that  $\|\hat \theta-\theta^\star\|_2$ converges to  zero as both $m, n$ go to infinity, where $\hat \theta$ is the minimizer of $\nabla_\theta \mathcal{L}^{\mathsf{DR}}_{\mathcal{D}_1,\mathcal{D}_2}$.

When $\hat f$ is a perfect predictor with $\hat f(X) \equiv Y$ (and $Y|X=x$ is deterministic), one has  $ 
\mathcal{L}^{\mathsf{DR}}_{\mathcal{D}_1,\mathcal{D}_2}(\theta^\star) = \frac{1}{m+n}  \sum_{i=1}^{m+n} \ell_\theta(X_i, Y_i)$. The effective sample size is $m+n$ instead of $n$ in $\mathcal{L}^{\mathsf{SL}}_{\mathcal{D}_1,\mathcal{D}_2}(\theta)$.

When $\hat f$ is also trained from the labeled data, one may apply data splitting to achieve the same guarantee up to a constant factor. We provide further discussion in Appendix~\ref{app:split}.

\subsection{The case of distribution mismatch}\label{sec:mismatch}


We also consider the case in which the marginal distributions of the covariates for the labeled and unlabeled datasets are different.
Assume in particular that we are given a set of unlabeled samples, $\mathcal{D}_1 = \{X_1,X_2,\cdots, X_m\}$, drawn from a fixed distribution $\mathbb{P}_X$, a set of labeled samples, $\mathcal{D}_2 =\{(X_{m+1}, Y_{m+1}), (X_{m+2}, Y_{m+2}),$ $ \cdots, (X_{m+n}, Y_{m+n})\}$, drawn from some joint distribution $\mathbb{Q}_X\times \mathbb{P}_{Y|X}$, and a pre-trained model  $\hat f$. 
In the case when the labeled samples do not follow the same distribution as the unlabeled samples, we need to introduce an importance weight $\pi(x)$. This yields the following doubly robust estimator:
\begin{align*}
\mathcal{L}^{\mathsf{DR2}}_{\mathcal{D}_1,\mathcal{D}_2}(\theta) 
& = \frac{1}{m}  \sum_{i=1}^{m} \ell_\theta(X_i, \hat f(X_i)) -  \frac{1}{n} \sum_{i=m+1}^{m+n} \frac{1}{\pi(X_i)}\ell_\theta(X_i, \hat f(X_i))  + \frac{1}{n} \sum_{i=m+1}^{m+n} \frac{1}{\pi(X_i)}\ell_\theta(X_i, Y_i). 
\end{align*}
Note that we not only introduce the importance weight $\pi$, but we also change the first term from the average of all the $m+n$ samples to the average of $n$ samples. 

\begin{proposition}\label{prop:dr_mis}
We have $ \mathbb{E}[\mathcal{L}^{\mathsf{DR2}}_{\mathcal{D}_1,\mathcal{D}_2}(\theta) ] = \mathbb{E}_{\mathbb{P}_{X, Y}}[\ell_\theta(X, Y)]$ as long as one of the following two assumptions hold:
        \begin{itemize}
            \item For any $x$, $\pi(x) = \frac{\mathbb{P}_X(x)}{\mathbb{Q}_X(x)}$.
            \item For any $x$, $\ell_\theta(x, \hat f(x))  = \mathbb{E}_{ Y\sim \mathbb{P}_{Y\mid X=x}}[\ell_\theta(x, Y)]$.
        \end{itemize}
\end{proposition}
The proof is deferred to Appendix~\ref{proof:dr_mis}. 
The proposition implies that  as long as either $\pi$ or 
$\hat f$ is accurate, the expectation of the loss is the same as that of the target loss. When the distributions for the unlabeled and labeled samples match each other, this reduces to the case in the previous sections. In this case, taking $\pi(x)=1$ guarantees that the expectation of the doubly robust loss is always the same as that of the target loss. 
% And similarly, we can provide non-asymptotic rates. Notably, we need to make the assumption that $\pi(x)$ is always bounded.
% \begin{assumption}\label{ass:mom2}
% Assume that $\pi(x)<\infty$, and the random variable $\nabla_\theta \ell_\theta(X, \hat f(X)) $ and  $\nabla_\theta \ell_\theta(X, Y)$ have bounded first and second moments under distribution $\mathbb{P}_{X, Y}$.
% \end{assumption}
% With this assumption, we denote $\Sigma_{\theta}^{Y-\hat f} = \mathsf{Cov}[\nabla_\theta \ell_\theta(X, \hat f(X)) - \nabla_\theta \ell_\theta(X, Y)]$, $\Sigma_{\theta}^{\hat f} = \mathsf{Cov}[\nabla_\theta \ell_\theta(X, \hat f(X))]$, $\Sigma_{\theta}^{Y} = \mathsf{Cov}[\nabla_\theta \ell_\theta(X, Y)]$. 
\section{Experiments}\label{sec:empirical}

% \subsection{US Census Dataset}

To employ the new doubly robust loss in practical applications, we need to specify an appropriate optimization procedure, in particular one that is based on (mini-batched) stochastic gradient descent so as to exploit modern scalable machine learning methods.  In preliminary experiments we observed that directly minimizing the doubly robust loss in Equation (\ref{eq:dr}) with stohastic gradient can lead to instability, and thus, we propose instead to minimize the  curriculum-based loss in each epoch:
\begin{align*}
\mathcal{L}^{\mathsf{DR}, t}_{\mathcal{D}_1,\mathcal{D}_2}(\theta) 
& = \frac{1}{m+n}  \sum_{i=1}^{m+n} \ell_\theta(X_i, \hat f(X_i)) - \alpha_t\cdot \left(  \frac{1}{n} \sum_{i=m+1}^{m+n} \ell_\theta(X_i, \hat f(X_i))  - \frac{1}{n} \sum_{i=m+1}^{m+n} \ell_\theta(X_i, Y_i)\right).  %\label{eq:dr_weighted}
\end{align*}
As we show in the experiments below, this choice yields a stable algorithm. We set $\alpha_t = t/T$, where $T$ is the total number of epochs. For the object detection experiments, we introduce the labeled samples only in the final epoch, setting $\alpha_t = 0$ for all epochs before setting $\alpha_t = 1$ in the final epoch. Intuitively, we start from the training with samples only from the pseudo-labels, and gradually introduce the labeled samples in the doubly robust loss for fine-tuning. 

We conduct experiments on both image classification task with ImageNet dataset~\citep{russakovsky2015imagenet} and 3D object detection task with  autonomous driving dataset nuScenes~\citep{nuscenes2019}. The code is available in \url{https://github.com/dingmyu/Doubly-Robust-Self-Training}.

\subsection{Image classification}

\textbf{Datasets and settings.}
We evaluate our doubly robust self-training method on the ImageNet100 dataset, which contains a random subset of 100 classes from ImageNet-1k~\citep{russakovsky2015imagenet}, with  120K training images (approximately 1,200 samples per class) and 5,000 validation images (50 samples per class).
%
To further test the effectiveness of our algorithm in a low-data scenario, we create a dataset that we refer to as mini-ImageNet100 by
randomly sampling 100 images per class from ImageNet100.
%
Two models were evaluated: (1) DaViT-T~\citep{ding2022davit}, a popular vision transformer architecture with state-of-the-art performance on ImageNet, and (2) ResNet50~\citep{he2016deep}, a classic convolutional network to verify the generality of our algorithm. 

 \textbf{Baselines.} 
To provide a comparative evaluation of doubly robust self-training, we establish three baselines: (1) `Labeled Only' for training on labeled data only (partial training set) with a loss $\mathcal{L}^{\mathsf{TL}}$, (2) `Pseudo Only' for training with pseudo labels generated for all training samples, and (3) `Labeled + Pseudo' for a mixture of pseudo-labels and labeled data, with the loss $\mathcal{L}^{\mathsf{SL}}$. 
% More implementation details and ablations are provided in Appendix.
See the Appendix for further implementation details and ablations.
% We also compare model training with different numbers of epochs and different proportions of labeled data. More ablation studies could be found in Appendix.
% We evaluate all the models on the same ImageNet-100 validation set.
% 

% \begin{table}[t]
% \footnotesize
% \centering
% \caption{Comparisons on ImageNet-100-1200, all models trained for 20 epochs.}
% \label{tab:classification_fraction}
% \setlength{\tabcolsep}{10pt}
% \renewcommand\arraystretch{1.2}
% \resizebox{1\linewidth}{!}{
%     \begin{tabular}{c|cc|cc|cc|cc}
%     \shline
%      \multirow{2}{*}{Labeled Data Percent} & \multicolumn{2}{c|}{Labeled Only} & \multicolumn{2}{c|}{Pseudo Only} & \multicolumn{2}{c|}{Labeled + Pseudo} & \multicolumn{2}{c}{Doubly robust Loss} \\
%      & top1 & top5 & top1 & top5 & top1 & top5 & top1 & top5 \\
%     \shline
%     1 & 3.09 & 10.59 & 3.57&12.76	&	3.59	&13.54 & \textbf{7.01}	&\textbf{21.88} \\
%     5 & 9.21 & 25.15	&	11.11&	26.82	&	10.95	&26.75&	\textbf{11.75}&	\textbf{28.09}\\
%     10 &  16.02	& 39.68		&17.02	&38.64		& 19.38	&41.96	&	\textbf{23.26}	&\textbf{50.45}\\
%     20 & 16.86	&39.98	&	18.24	&39.00	&	20.31	&44.52	&	\textbf{25.81}&	\textbf{53.97} \\
%     30 & 23.94&	50.04		&24.40	&48.94	&	28.01	&54.63		&\textbf{32.63}&	\textbf{62.09} \\
%     40 &  29.81	&58.01	&	29.73	&55.03	&	34.03	&62.74	&	\textbf{37.01}&	\textbf{67.90} \\
%     50 & 32.77&	62.58		&31.89&	58.05	&	37.60	&66.72	&	\textbf{39.70}&	\textbf{70.34} \\
%     60 &  36.47	&66.32	&	35.19	&62.43&		41.36	&71.82	&	\textbf{45.52}	&\textbf{74.56} \\
%     70 &  39.56&	69.86	&	37.33	&65.64&		43.76	&73.42&		\textbf{48.44}	&\textbf{77.86} \\
%     80 & 43.12&	72.60	&	39.72	&67.78	&	46.14 & 75.27		& \textbf{51.42}	&\textbf{79.44} \\
%     90 &  45.02&75.25	&	41.16	&69.50	&	47.74	& 77.15		&\textbf{51.52}	& \textbf{79.92} \\
%     100 & 47.84	&77.13&		42.82&	70.32	&	47.84 &	77.13	& \textbf{51.79} & \textbf{80.01} \\
%     \shline
%     \end{tabular}}
%     % \vspace{-8pt}
% \end{table}



\begin{figure}[t]
  % \centering
  \includegraphics[width=0.495\textwidth]{figures/davit-top1.pdf}\hfill
  \includegraphics[width=0.495\textwidth]{figures/davit-top5.pdf}
  
  \vspace{-3pt} \small{\hspace{70pt} (a) Top-1 on DaViT \hspace{125pt} (b) Top-5 on DaViT}
  
  \includegraphics[width=0.495\textwidth]{figures/resnet-top1.pdf}\hfill
  \includegraphics[width=0.495\textwidth]{figures/resnet-top5.pdf}

  \vspace{-3pt} \small{\hspace{63pt} (c) Top-1 on ResNet50 \hspace{115pt} (d) Top-5 on ResNet50}
  \vspace{-0.05in}
  \caption{Comparisons on ImageNet100 using two different network architectures. Both Top-1 and Top-5 accuracies are reported. All models are trained for 20 epochs.
  }
  \label{fig:classification_fraction}
  \vspace{-16pt}
\end{figure}


\textbf{Results on ImageNet100.} 
We first conduct experiments on ImageNet100 by training the model for 20 epochs using different fractions of labeled data from 1\% to 100\%.
From the results shown in Fig.~\ref{fig:classification_fraction}, we observe that: (1) Our model outperforms all baseline methods on both two networks by large margins.
For example, we achieve 5.5\% and 5.3\% gains (Top-1 Acc) on DaViT over the `Labeled + Pseudo' method for 20\% and 80\% labeled data, respectively.
(2) The `Labeled + Pseudo' method consistently beats the `Labeled Only' baseline.
(3) While `Pseudo Only' works for smaller fractions of the labeled data (less than 30\%) on DaViT, it is inferior to `Labeled Only' on ResNet50.

\textbf{Results on mini-ImageNet100.} We also perform comparisons on mini-ImageNet100 to demonstrate the performance when the total data volume is limited.
% where only 100 training samples per class, to show the case when the total amount of data is limited. We train all models 100 epochs.
%
From the results in Table~\ref{tab:classification_100sample}, we see our model generally outperforms all baselines. 
%
As the dataset size decreases and the number of training epochs increases, the gain of our algorithm becomes smaller. This is expected, as (1) the models are not adequately trained and thus have noise issues, and (2) there are an insufficient number of ground truth labels to compute the last term of our loss function. In extreme cases, there is only one labeled sample (1\%) per class.


% \subsubsection{Experiments}
% \noindent \textbf{Results on ImageNet-100-1200.} We conduct experiments on original ImageNet-100, which contains 1200 training samples for each class.
% We train the model for 20 epochs using different fractions of labeled data from 1\% to 100\%.
% From the results shown in Fig.~\ref{fig:classification_fraction}, we observe that: 1) Our model outperforms all baseline methods by large margins.
% For example, we achieve 5.5\% and 5.3\% gains on top1 accuracy over the `Labeled + Pseudo' method for 20\% and 80\% labeled data, respectively.
% 2) The `Labeled + Pseudo' method consistently beats the `Labeled Only' baseline, while `Pseudo Only' works for a smaller fraction of the labeled data, i.e., when it is less than 30\%.


% \noindent \textbf{Results on ImageNet-100-100.} We also perform comparisons on ImageNet-100-100, where only 100 training samples per class, to show the case when the total amount of data is limited. We train all models 100 epochs.
% %
% From the results in Table~\ref{tab:classification_100sample}, we see our model generally outperforms all baselines. 
% %
% As the dataset size gets smaller and the number of training epochs increases, the gain of our algorithm becomes smaller. This is expected, as 1) all models are not well trained so the noise issue exists, and 2) there are not enough ground truths to compute the last term of our loss. In extreme cases, there is only 1 labeled data (1\%) per class.


\begin{table}[t]
\footnotesize
\centering
\caption{Comparisons on mini-ImageNet100, all models trained for 100 epochs.}
\setlength{\tabcolsep}{10pt}
\renewcommand\arraystretch{1.05}
\resizebox{1\linewidth}{!}{
    \begin{tabular}{c|cc|cc|cc|cc}
    \shline
     \multirow{2}{*}{Labeled Data Percent} & \multicolumn{2}{c|}{Labeled Only} & \multicolumn{2}{c|}{Pseudo Only} & \multicolumn{2}{c|}{Labeled + Pseudo} & \multicolumn{2}{c}{Doubly robust Loss} \\
     & top1 & top5 & top1 & top5 & top1 & top5 & top1 & top5 \\
    \shline
    1 & 2.72	&9.18	&	\textbf{2.81}	&9.57	&	2.73&	9.55	&	2.75	&\textbf{9.73} \\
    5 & 3.92	&13.34	&	4.27&	13.66	&	4.27&	14.4	&	\textbf{4.89}	&\textbf{16.38}	\\
    10 & 6.76	&20.84		&7.27&	21.64	&	7.65&	22.48	&	\textbf{8.01}	&\textbf{21.90}  \\
    20 & 12.3&	31.3	&	13.46&	30.79	&	\textbf{13.94}&	\textbf{32.63}	&	13.50	&32.17 \\
    50 & 20.69	&46.86	&	20.92&	45.2	&	24.9	&50.77	&	\textbf{25.31}	&\textbf{51.61}\\
    80 &  27.37	& 55.57		&25.57	&50.85	&	30.63	&58.85	&	\textbf{30.75}	&\textbf{59.41} \\
    100  & 31.07&	60.62	&	28.95	&55.35	&	\textbf{34.33}&	62.78	&	34.01	&\textbf{63.04} \\
    \shline
    \end{tabular}}
    \label{tab:classification_100sample}
    \vspace{-6pt}
\end{table}


%




\subsection{3D object detection}
% \paragraph{
\textbf{Doubly robust object detection.}
Given a visual representation of a scene, 3D object detection aims to generate a set of 3D bounding box predictions $\{b_i\}_{i\in[m+n]}$ and a set of corresponding class predictions $\{c_i\}_{i\in[m+n]}$. Thus, each single ground-truth annotation $Y_i \in Y$ is a set $Y_i = (b_i, c_i)$ containing a box and a class. During training, the object detector is supervised with a sum of the box regression loss $\mathcal{L}_{loc}$ and the classification loss $\mathcal{L}_{cls}$, i.e. $\mathcal{L}_{obj} = \mathcal{L}_{loc} + \mathcal{L}_{cls}$.

In the self-training protocol for object detection, %we have a set of $m$ unlabeled scenes $\{X_i\}_{i=1}^m$, a set of $n$ labeled scenes $\{(X_i,Y_i)\}_{i=m}^{m+n}$, and a labeler $f$ pre-trained on the labeled scenes. 
pseudo-labels for a given scene $X_i$ are selected from the labeler predictions $f(X_i)$ based on some user-defined criteria (typically the model's detection confidence). 
%We denote the selected pseudo-labels as $f(X_i)^{(>\tau)}$, where $\tau$ is the criteria threshold below which detections are discarded. 
Unlike in standard classification or regression, $Y_i$ will contain a differing number of labels depending on the number of objects in the scene. Furthermore, the number of extracted pseudo-labels $f(X_i)$ will generally not be equal to the number of scene ground-truth labels $Y_i$ due to false positive/negative detections. Therefore it makes sense to express the doubly robust loss function in terms of the individual box labels as opposed to the scene-level labels. We define the doubly robust object detection loss as follows:
\begin{align*}
\mathcal{L}^{\mathsf{DR}}_{obj}(\theta) 
& = \frac{1}{M+N_{ps}}  \sum_{i=1}^{M+N_{ps}} \ell_\theta(X_i,  f(X_i)) -  \frac{1}{N_{ps}} \sum_{i=M+1}^{M+N_{ps}} \ell_\theta(X_i',  f(X_i'))  + \frac{1}{N} \sum_{i=M+1}^{M+N} \ell_\theta(X_i, Y_i),
\end{align*}
where $M$ is the total number of pseudo-label boxes from the unlabeled split, $N$ is the total number of labeled boxes, $X'_i$ is the scene with pseudo-label boxes from the \textit{labeled} split, and $N_{ps}$ is the total number of pseudo-label boxes from the \textit{labeled} split. We note that the last two terms now contain summations over a differing number of boxes, a consequence of the discrepancy between the number of manually labeled boxes and pseudo-labeled boxes. Both components of the object detection loss (localization/classification) adopt this form of  doubly robust loss.

% \paragraph{
\textbf{Dataset and setting.} 
To evaluate doubly robust self-training in the autonomous driving setting, we perform experiments on the large-scale 3D detection dataset nuScenes~\citep{nuscenes2019}. The nuScenes dataset is comprised of 1000 scenes (700 training, 150 validation and 150 test) with each frame containing sensor information from RGB camera, LiDAR, and radar scans. Box annotations are comprised of 10 classes, with the class instance distribution following a long-tailed distribution, allowing us to investigate our self-training approach for both common and rare classes. The main 3D detection metrics for nuScenes are mean Average Precision (mAP) and the nuScenes Detection Score (NDS), a dataset-specific metric consisting of a weighted average of mAP and five other true-positive metrics. For the sake of simplicity, we train object detection models using only LiDAR sensor information.


% \begin{figure}
%   \centering
%   \begin{subfigure}{.5\textwidth}
%   \centering
%   %\includegraphics[scale=0.45]{self_labeling_map.png}
%   \caption{}
%   \end{subfigure}%
%   \begin{subfigure}{.5\textwidth}
%   \centering
%   %\includegraphics[scale=0.45]{self_labeling_nds.png}
%   \caption{}
%   \end{subfigure}%
%   \caption{Self-training experiments on nuScenes \textit{val} set using varying fractions of ground-truth labels. In the low label regime, training with the doubly robust loss function significantly improves both mAP (a) and NDS (b) over the baseline training with the naive loss.}
%   \label{nuscenes}
% \end{figure}

\begin{table}[t]
\footnotesize
\setlength{\tabcolsep}{13pt}
\renewcommand\arraystretch{1.05}
\centering
\caption{Performance comparison on nuScenes \textit{val} set.}
\resizebox{1\linewidth}{!}{
    \begin{tabular}{c |c  c | c  c | c  c }
    \shline
    \multirow{2}{*}{Labeled Data Fraction} & \multicolumn{2}{c|}{Labeled Only} & \multicolumn{2}{c|}{Labeled + Pseudo} & \multicolumn{2}{c}{Doubly robust Loss}\\
     & mAP$\uparrow$ & NDS$\uparrow$ & mAP$\uparrow$ & NDS$\uparrow$ & mAP$\uparrow$ & NDS$\uparrow$\\
     \hline
     1/24 & 7.56 & 18.01 & 7.60 & 17.32 & \textbf{8.18} & \textbf{18.33}\\
     1/16 & 11.15 & 20.55 & 11.60 & 21.03 & \textbf{12.30} & \textbf{22.10}\\
     1/4 & 25.66 & 41.41 & \textbf{28.36} & \textbf{43.88} & 27.48 & 43.18 \\
     \shline
    \end{tabular}}
\label{nuscresults}
\vspace{-6pt}
\end{table}

\begin{table*}
\centering
\small
\setlength{\tabcolsep}{9pt}
\renewcommand\arraystretch{1.05}
\caption{Per-class mAP (\%) comparison on nuScenes \textit{val} set using 1/16 of total labels in training.}
\vspace{-0.08in}
\begin{tabular}{c||c | c | c | c | c | c | c}
\shline
 & Car & Ped & Truck & Bus & Trailer & Barrier & Traffic Cone\\
 \shline
 Labeled Only & 48.6 & 30.6 & 8.5 & 6.2 & 4.0 & 6.8 & 4.4\\
 \hline
 Labeled + Pseudo & 48.8 & 30.9 & 8.8 & 7.5 & 5.7 & 6.7 & 4.0\\
 Improvement & +0.2 & +0.3 & +0.3 & +1.3 & \textbf{+1.7} & -0.1 & -0.4\\
 \hline
 Doubly robust Loss & 51.5 & 32.9 &  9.6 & 8.2 & 5.2 & 7.2 & 4.5   \\
 Improvement & \textbf{+2.9} & \textbf{+2.3} & \textbf{+1.1}  & \textbf{+2.0} & +1.2 & \textbf{+0.4} &  \textbf{+0.1}  \\
 \shline
\end{tabular}
\label{nuscclass}
\vspace{-12pt}
\end{table*}

% \paragraph{
\textbf{Results.}
After semi-supervised training, we evaluate our student model performance on the nuScenes \textit{val} set. We compare three settings: training the student model with only the available labeled data (i.e., equivalent to teacher training), training the student model on the combination of labeled/teacher-labeled data using the naive self-training loss, and training the student model on the combination of labeled/teacher-labeled data using our proposed doubly robust loss. We report results for training with 1/24, 1/16, and 1/4 of the total labels in Table \ref{nuscresults}. We find that the doubly robust loss improves both mAP and NDS over using only labeled data and the naive baseline in the lower label regime, whereas performance is slightly degraded when more labels are available. Furthermore, we also show a per-class performance breakdown in Table \ref{nuscclass}. We find that the doubly robust loss consistently improves performance for both common (car, pedestrian) and rare classes. Notably, the doubly robust loss is even able to improve upon the teacher in classes for which pseudo-label training \textit{decreases} performance when using the naive training (e.g., barriers and traffic cones).


\section{Conclusions}

We have proposed a novel doubly robust loss for self-training. Theoretically, we analyzed the double-robustness property of the proposed loss, demonstrating its statistical efficiency when the pseudo-labels are accurate. Empirically, we showed that large improvements can be obtained in both image classification and 3D object detection.

As a direction for future work, it would be interesting to understand how the doubly robust loss might be applied to other domains that have a missing-data aspect, including model distillation, transfer learning, and continual learning. It is also important to find practical and efficient algorithms when the labeled and unlabeled data do not match in distribution. 


\section*{Acknowledgements}
Banghua Zhu and Jiantao 
Jiao are partially supported by NSF IIS-1901252, CIF-1909499 and CIF-2211209. Michael I. Jordan is partially supported by NSF IIS-1901252. Philip Jacobson is supported by the National Defense Science and Engineering Graduate (NDSEG) Fellowship. This work is also partially supported by Berkeley DeepDrive. 

% {\color{red} TODO: 
% Move exp details to Appendix. 

% If still no more space, 
% move Table 4 first, then Table 3.

% Make the table style consistent by changing the style of the 3D object detection table.

% Convert to linechart? 

% Go over the paper again. 

% Upload code. 
% }
\newpage
\bibliography{ref}
\newpage 
\appendix
\chapter{Supplementary Material}
\label{appendix}

In this appendix, we present supplementary material for the techniques and
experiments presented in the main text.

\section{Baseline Results and Analysis for Informed Sampler}
\label{appendix:chap3}

Here, we give an in-depth
performance analysis of the various samplers and the effect of their
hyperparameters. We choose hyperparameters with the lowest PSRF value
after $10k$ iterations, for each sampler individually. If the
differences between PSRF are not significantly different among
multiple values, we choose the one that has the highest acceptance
rate.

\subsection{Experiment: Estimating Camera Extrinsics}
\label{appendix:chap3:room}

\subsubsection{Parameter Selection}
\paragraph{Metropolis Hastings (\MH)}

Figure~\ref{fig:exp1_MH} shows the median acceptance rates and PSRF
values corresponding to various proposal standard deviations of plain
\MH~sampling. Mixing gets better and the acceptance rate gets worse as
the standard deviation increases. The value $0.3$ is selected standard
deviation for this sampler.

\paragraph{Metropolis Hastings Within Gibbs (\MHWG)}

As mentioned in Section~\ref{sec:room}, the \MHWG~sampler with one-dimensional
updates did not converge for any value of proposal standard deviation.
This problem has high correlation of the camera parameters and is of
multi-modal nature, which this sampler has problems with.

\paragraph{Parallel Tempering (\PT)}

For \PT~sampling, we took the best performing \MH~sampler and used
different temperature chains to improve the mixing of the
sampler. Figure~\ref{fig:exp1_PT} shows the results corresponding to
different combination of temperature levels. The sampler with
temperature levels of $[1,3,27]$ performed best in terms of both
mixing and acceptance rate.

\paragraph{Effect of Mixture Coefficient in Informed Sampling (\MIXLMH)}

Figure~\ref{fig:exp1_alpha} shows the effect of mixture
coefficient ($\alpha$) on the informed sampling
\MIXLMH. Since there is no significant different in PSRF values for
$0 \le \alpha \le 0.7$, we chose $0.7$ due to its high acceptance
rate.


% \end{multicols}

\begin{figure}[h]
\centering
  \subfigure[MH]{%
    \includegraphics[width=.48\textwidth]{figures/supplementary/camPose_MH.pdf} \label{fig:exp1_MH}
  }
  \subfigure[PT]{%
    \includegraphics[width=.48\textwidth]{figures/supplementary/camPose_PT.pdf} \label{fig:exp1_PT}
  }
\\
  \subfigure[INF-MH]{%
    \includegraphics[width=.48\textwidth]{figures/supplementary/camPose_alpha.pdf} \label{fig:exp1_alpha}
  }
  \mycaption{Results of the `Estimating Camera Extrinsics' experiment}{PRSFs and Acceptance rates corresponding to (a) various standard deviations of \MH, (b) various temperature level combinations of \PT sampling and (c) various mixture coefficients of \MIXLMH sampling.}
\end{figure}



\begin{figure}[!t]
\centering
  \subfigure[\MH]{%
    \includegraphics[width=.48\textwidth]{figures/supplementary/occlusionExp_MH.pdf} \label{fig:exp2_MH}
  }
  \subfigure[\BMHWG]{%
    \includegraphics[width=.48\textwidth]{figures/supplementary/occlusionExp_BMHWG.pdf} \label{fig:exp2_BMHWG}
  }
\\
  \subfigure[\MHWG]{%
    \includegraphics[width=.48\textwidth]{figures/supplementary/occlusionExp_MHWG.pdf} \label{fig:exp2_MHWG}
  }
  \subfigure[\PT]{%
    \includegraphics[width=.48\textwidth]{figures/supplementary/occlusionExp_PT.pdf} \label{fig:exp2_PT}
  }
\\
  \subfigure[\INFBMHWG]{%
    \includegraphics[width=.5\textwidth]{figures/supplementary/occlusionExp_alpha.pdf} \label{fig:exp2_alpha}
  }
  \mycaption{Results of the `Occluding Tiles' experiment}{PRSF and
    Acceptance rates corresponding to various standard deviations of
    (a) \MH, (b) \BMHWG, (c) \MHWG, (d) various temperature level
    combinations of \PT~sampling and; (e) various mixture coefficients
    of our informed \INFBMHWG sampling.}
\end{figure}

%\onecolumn\newpage\twocolumn
\subsection{Experiment: Occluding Tiles}
\label{appendix:chap3:tiles}

\subsubsection{Parameter Selection}

\paragraph{Metropolis Hastings (\MH)}

Figure~\ref{fig:exp2_MH} shows the results of
\MH~sampling. Results show the poor convergence for all proposal
standard deviations and rapid decrease of AR with increasing standard
deviation. This is due to the high-dimensional nature of
the problem. We selected a standard deviation of $1.1$.

\paragraph{Blocked Metropolis Hastings Within Gibbs (\BMHWG)}

The results of \BMHWG are shown in Figure~\ref{fig:exp2_BMHWG}. In
this sampler we update only one block of tile variables (of dimension
four) in each sampling step. Results show much better performance
compared to plain \MH. The optimal proposal standard deviation for
this sampler is $0.7$.

\paragraph{Metropolis Hastings Within Gibbs (\MHWG)}

Figure~\ref{fig:exp2_MHWG} shows the result of \MHWG sampling. This
sampler is better than \BMHWG and converges much more quickly. Here
a standard deviation of $0.9$ is found to be best.

\paragraph{Parallel Tempering (\PT)}

Figure~\ref{fig:exp2_PT} shows the results of \PT sampling with various
temperature combinations. Results show no improvement in AR from plain
\MH sampling and again $[1,3,27]$ temperature levels are found to be optimal.

\paragraph{Effect of Mixture Coefficient in Informed Sampling (\INFBMHWG)}

Figure~\ref{fig:exp2_alpha} shows the effect of mixture
coefficient ($\alpha$) on the blocked informed sampling
\INFBMHWG. Since there is no significant different in PSRF values for
$0 \le \alpha \le 0.8$, we chose $0.8$ due to its high acceptance
rate.



\subsection{Experiment: Estimating Body Shape}
\label{appendix:chap3:body}

\subsubsection{Parameter Selection}
\paragraph{Metropolis Hastings (\MH)}

Figure~\ref{fig:exp3_MH} shows the result of \MH~sampling with various
proposal standard deviations. The value of $0.1$ is found to be
best.

\paragraph{Metropolis Hastings Within Gibbs (\MHWG)}

For \MHWG sampling we select $0.3$ proposal standard
deviation. Results are shown in Fig.~\ref{fig:exp3_MHWG}.


\paragraph{Parallel Tempering (\PT)}

As before, results in Fig.~\ref{fig:exp3_PT}, the temperature levels
were selected to be $[1,3,27]$ due its slightly higher AR.

\paragraph{Effect of Mixture Coefficient in Informed Sampling (\MIXLMH)}

Figure~\ref{fig:exp3_alpha} shows the effect of $\alpha$ on PSRF and
AR. Since there is no significant differences in PSRF values for $0 \le
\alpha \le 0.8$, we choose $0.8$.


\begin{figure}[t]
\centering
  \subfigure[\MH]{%
    \includegraphics[width=.48\textwidth]{figures/supplementary/bodyShape_MH.pdf} \label{fig:exp3_MH}
  }
  \subfigure[\MHWG]{%
    \includegraphics[width=.48\textwidth]{figures/supplementary/bodyShape_MHWG.pdf} \label{fig:exp3_MHWG}
  }
\\
  \subfigure[\PT]{%
    \includegraphics[width=.48\textwidth]{figures/supplementary/bodyShape_PT.pdf} \label{fig:exp3_PT}
  }
  \subfigure[\MIXLMH]{%
    \includegraphics[width=.48\textwidth]{figures/supplementary/bodyShape_alpha.pdf} \label{fig:exp3_alpha}
  }
\\
  \mycaption{Results of the `Body Shape Estimation' experiment}{PRSFs and
    Acceptance rates corresponding to various standard deviations of
    (a) \MH, (b) \MHWG; (c) various temperature level combinations
    of \PT sampling and; (d) various mixture coefficients of the
    informed \MIXLMH sampling.}
\end{figure}


\subsection{Results Overview}
Figure~\ref{fig:exp_summary} shows the summary results of the all the three
experimental studies related to informed sampler.
\begin{figure*}[h!]
\centering
  \subfigure[Results for: Estimating Camera Extrinsics]{%
    \includegraphics[width=0.9\textwidth]{figures/supplementary/camPose_ALL.pdf} \label{fig:exp1_all}
  }
  \subfigure[Results for: Occluding Tiles]{%
    \includegraphics[width=0.9\textwidth]{figures/supplementary/occlusionExp_ALL.pdf} \label{fig:exp2_all}
  }
  \subfigure[Results for: Estimating Body Shape]{%
    \includegraphics[width=0.9\textwidth]{figures/supplementary/bodyShape_ALL.pdf} \label{fig:exp3_all}
  }
  \label{fig:exp_summary}
  \mycaption{Summary of the statistics for the three experiments}{Shown are
    for several baseline methods and the informed samplers the
    acceptance rates (left), PSRFs (middle), and RMSE values
    (right). All results are median results over multiple test
    examples.}
\end{figure*}

\subsection{Additional Qualitative Results}

\subsubsection{Occluding Tiles}
In Figure~\ref{fig:exp2_visual_more} more qualitative results of the
occluding tiles experiment are shown. The informed sampling approach
(\INFBMHWG) is better than the best baseline (\MHWG). This still is a
very challenging problem since the parameters for occluded tiles are
flat over a large region. Some of the posterior variance of the
occluded tiles is already captured by the informed sampler.

\begin{figure*}[h!]
\begin{center}
\centerline{\includegraphics[width=0.95\textwidth]{figures/supplementary/occlusionExp_Visual.pdf}}
\mycaption{Additional qualitative results of the occluding tiles experiment}
  {From left to right: (a)
  Given image, (b) Ground truth tiles, (c) OpenCV heuristic and most probable estimates
  from 5000 samples obtained by (d) MHWG sampler (best baseline) and
  (e) our INF-BMHWG sampler. (f) Posterior expectation of the tiles
  boundaries obtained by INF-BMHWG sampling (First 2000 samples are
  discarded as burn-in).}
\label{fig:exp2_visual_more}
\end{center}
\end{figure*}

\subsubsection{Body Shape}
Figure~\ref{fig:exp3_bodyMeshes} shows some more results of 3D mesh
reconstruction using posterior samples obtained by our informed
sampling \MIXLMH.

\begin{figure*}[t]
\begin{center}
\centerline{\includegraphics[width=0.75\textwidth]{figures/supplementary/bodyMeshResults.pdf}}
\mycaption{Qualitative results for the body shape experiment}
  {Shown is the 3D mesh reconstruction results with first 1000 samples obtained
  using the \MIXLMH informed sampling method. (blue indicates small
  values and red indicates high values)}
\label{fig:exp3_bodyMeshes}
\end{center}
\end{figure*}

\clearpage



\section{Additional Results on the Face Problem with CMP}

Figure~\ref{fig:shading-qualitative-multiple-subjects-supp} shows inference results for reflectance maps, normal maps and lights for randomly chosen test images, and Fig.~\ref{fig:shading-qualitative-same-subject-supp} shows reflectance estimation results on multiple images of the same subject produced under different illumination conditions. CMP is able to produce estimates that are closer to the groundtruth across different subjects and illumination conditions.

\begin{figure*}[h]
  \begin{center}
  \centerline{\includegraphics[width=1.0\columnwidth]{figures/face_cmp_visual_results_supp.pdf}}
  \vspace{-1.2cm}
  \end{center}
	\mycaption{A visual comparison of inference results}{(a)~Observed images. (b)~Inferred reflectance maps. \textit{GT} is the photometric stereo groundtruth, \textit{BU} is the Biswas \etal (2009) reflectance estimate and \textit{Forest} is the consensus prediction. (c)~The variance of the inferred reflectance estimate produced by \MTD (normalized across rows).(d)~Visualization of inferred light directions. (e)~Inferred normal maps.}
	\label{fig:shading-qualitative-multiple-subjects-supp}
\end{figure*}


\begin{figure*}[h]
	\centering
	\setlength\fboxsep{0.2mm}
	\setlength\fboxrule{0pt}
	\begin{tikzpicture}

		\matrix at (0, 0) [matrix of nodes, nodes={anchor=east}, column sep=-0.05cm, row sep=-0.2cm]
		{
			\fbox{\includegraphics[width=1cm]{figures/sample_3_4_X.png}} &
			\fbox{\includegraphics[width=1cm]{figures/sample_3_4_GT.png}} &
			\fbox{\includegraphics[width=1cm]{figures/sample_3_4_BISWAS.png}}  &
			\fbox{\includegraphics[width=1cm]{figures/sample_3_4_VMP.png}}  &
			\fbox{\includegraphics[width=1cm]{figures/sample_3_4_FOREST.png}}  &
			\fbox{\includegraphics[width=1cm]{figures/sample_3_4_CMP.png}}  &
			\fbox{\includegraphics[width=1cm]{figures/sample_3_4_CMPVAR.png}}
			 \\

			\fbox{\includegraphics[width=1cm]{figures/sample_3_5_X.png}} &
			\fbox{\includegraphics[width=1cm]{figures/sample_3_5_GT.png}} &
			\fbox{\includegraphics[width=1cm]{figures/sample_3_5_BISWAS.png}}  &
			\fbox{\includegraphics[width=1cm]{figures/sample_3_5_VMP.png}}  &
			\fbox{\includegraphics[width=1cm]{figures/sample_3_5_FOREST.png}}  &
			\fbox{\includegraphics[width=1cm]{figures/sample_3_5_CMP.png}}  &
			\fbox{\includegraphics[width=1cm]{figures/sample_3_5_CMPVAR.png}}
			 \\

			\fbox{\includegraphics[width=1cm]{figures/sample_3_6_X.png}} &
			\fbox{\includegraphics[width=1cm]{figures/sample_3_6_GT.png}} &
			\fbox{\includegraphics[width=1cm]{figures/sample_3_6_BISWAS.png}}  &
			\fbox{\includegraphics[width=1cm]{figures/sample_3_6_VMP.png}}  &
			\fbox{\includegraphics[width=1cm]{figures/sample_3_6_FOREST.png}}  &
			\fbox{\includegraphics[width=1cm]{figures/sample_3_6_CMP.png}}  &
			\fbox{\includegraphics[width=1cm]{figures/sample_3_6_CMPVAR.png}}
			 \\
	     };

       \node at (-3.85, -2.0) {\small Observed};
       \node at (-2.55, -2.0) {\small `GT'};
       \node at (-1.27, -2.0) {\small BU};
       \node at (0.0, -2.0) {\small MP};
       \node at (1.27, -2.0) {\small Forest};
       \node at (2.55, -2.0) {\small \textbf{CMP}};
       \node at (3.85, -2.0) {\small Variance};

	\end{tikzpicture}
	\mycaption{Robustness to varying illumination}{Reflectance estimation on a subject images with varying illumination. Left to right: observed image, photometric stereo estimate (GT)
  which is used as a proxy for groundtruth, bottom-up estimate of \cite{Biswas2009}, VMP result, consensus forest estimate, CMP mean, and CMP variance.}
	\label{fig:shading-qualitative-same-subject-supp}
\end{figure*}

\clearpage

\section{Additional Material for Learning Sparse High Dimensional Filters}
\label{sec:appendix-bnn}

This part of supplementary material contains a more detailed overview of the permutohedral
lattice convolution in Section~\ref{sec:permconv}, more experiments in
Section~\ref{sec:addexps} and additional results with protocols for
the experiments presented in Chapter~\ref{chap:bnn} in Section~\ref{sec:addresults}.

\vspace{-0.2cm}
\subsection{General Permutohedral Convolutions}
\label{sec:permconv}

A core technical contribution of this work is the generalization of the Gaussian permutohedral lattice
convolution proposed in~\cite{adams2010fast} to the full non-separable case with the
ability to perform back-propagation. Although, conceptually, there are minor
differences between Gaussian and general parameterized filters, there are non-trivial practical
differences in terms of the algorithmic implementation. The Gauss filters belong to
the separable class and can thus be decomposed into multiple
sequential one dimensional convolutions. We are interested in the general filter
convolutions, which can not be decomposed. Thus, performing a general permutohedral
convolution at a lattice point requires the computation of the inner product with the
neighboring elements in all the directions in the high-dimensional space.

Here, we give more details of the implementation differences of separable
and non-separable filters. In the following, we will explain the scalar case first.
Recall, that the forward pass of general permutohedral convolution
involves 3 steps: \textit{splatting}, \textit{convolving} and \textit{slicing}.
We follow the same splatting and slicing strategies as in~\cite{adams2010fast}
since these operations do not depend on the filter kernel. The main difference
between our work and the existing implementation of~\cite{adams2010fast} is
the way that the convolution operation is executed. This proceeds by constructing
a \emph{blur neighbor} matrix $K$ that stores for every lattice point all
values of the lattice neighbors that are needed to compute the filter output.

\begin{figure}[t!]
  \centering
    \includegraphics[width=0.6\columnwidth]{figures/supplementary/lattice_construction}
  \mycaption{Illustration of 1D permutohedral lattice construction}
  {A $4\times 4$ $(x,y)$ grid lattice is projected onto the plane defined by the normal
  vector $(1,1)^{\top}$. This grid has $s+1=4$ and $d=2$ $(s+1)^{d}=4^2=16$ elements.
  In the projection, all points of the same color are projected onto the same points in the plane.
  The number of elements of the projected lattice is $t=(s+1)^d-s^d=4^2-3^2=7$, that is
  the $(4\times 4)$ grid minus the size of lattice that is $1$ smaller at each size, in this
  case a $(3\times 3)$ lattice (the upper right $(3\times 3)$ elements).
  }
\label{fig:latticeconstruction}
\end{figure}

The blur neighbor matrix is constructed by traversing through all the populated
lattice points and their neighboring elements.
% For efficiency, we do this matrix construction recursively with shared computations
% since $n^{th}$ neighbourhood elements are $1^{st}$ neighborhood elements of $n-1^{th}$ neighbourhood elements. \pg{do not understand}
This is done recursively to share computations. For any lattice point, the neighbors that are
$n$ hops away are the direct neighbors of the points that are $n-1$ hops away.
The size of a $d$ dimensional spatial filter with width $s+1$ is $(s+1)^{d}$ (\eg, a
$3\times 3$ filter, $s=2$ in $d=2$ has $3^2=9$ elements) and this size grows
exponentially in the number of dimensions $d$. The permutohedral lattice is constructed by
projecting a regular grid onto the plane spanned by the $d$ dimensional normal vector ${(1,\ldots,1)}^{\top}$. See
Fig.~\ref{fig:latticeconstruction} for an illustration of the 1D lattice construction.
Many corners of a grid filter are projected onto the same point, in total $t = {(s+1)}^{d} -
s^{d}$ elements remain in the permutohedral filter with $s$ neighborhood in $d-1$ dimensions.
If the lattice has $m$ populated elements, the
matrix $K$ has size $t\times m$. Note that, since the input signal is typically
sparse, only a few lattice corners are being populated in the \textit{slicing} step.
We use a hash-table to keep track of these points and traverse only through
the populated lattice points for this neighborhood matrix construction.

Once the blur neighbor matrix $K$ is constructed, we can perform the convolution
by the matrix vector multiplication
\begin{equation}
\ell' = BK,
\label{eq:conv}
\end{equation}
where $B$ is the $1 \times t$ filter kernel (whose values we will learn) and $\ell'\in\mathbb{R}^{1\times m}$
is the result of the filtering at the $m$ lattice points. In practice, we found that the
matrix $K$ is sometimes too large to fit into GPU memory and we divided the matrix $K$
into smaller pieces to compute Eq.~\ref{eq:conv} sequentially.

In the general multi-dimensional case, the signal $\ell$ is of $c$ dimensions. Then
the kernel $B$ is of size $c \times t$ and $K$ stores the $c$ dimensional vectors
accordingly. When the input and output points are different, we slice only the
input points and splat only at the output points.


\subsection{Additional Experiments}
\label{sec:addexps}
In this section, we discuss more use-cases for the learned bilateral filters, one
use-case of BNNs and two single filter applications for image and 3D mesh denoising.

\subsubsection{Recognition of subsampled MNIST}\label{sec:app_mnist}

One of the strengths of the proposed filter convolution is that it does not
require the input to lie on a regular grid. The only requirement is to define a distance
between features of the input signal.
We highlight this feature with the following experiment using the
classical MNIST ten class classification problem~\cite{lecun1998mnist}. We sample a
sparse set of $N$ points $(x,y)\in [0,1]\times [0,1]$
uniformly at random in the input image, use their interpolated values
as signal and the \emph{continuous} $(x,y)$ positions as features. This mimics
sub-sampling of a high-dimensional signal. To compare against a spatial convolution,
we interpolate the sparse set of values at the grid positions.

We take a reference implementation of LeNet~\cite{lecun1998gradient} that
is part of the Caffe project~\cite{jia2014caffe} and compare it
against the same architecture but replacing the first convolutional
layer with a bilateral convolution layer (BCL). The filter size
and numbers are adjusted to get a comparable number of parameters
($5\times 5$ for LeNet, $2$-neighborhood for BCL).

The results are shown in Table~\ref{tab:all-results}. We see that training
on the original MNIST data (column Original, LeNet vs. BNN) leads to a slight
decrease in performance of the BNN (99.03\%) compared to LeNet
(99.19\%). The BNN can be trained and evaluated on sparse
signals, and we resample the image as described above for $N=$ 100\%, 60\% and
20\% of the total number of pixels. The methods are also evaluated
on test images that are subsampled in the same way. Note that we can
train and test with different subsampling rates. We introduce an additional
bilinear interpolation layer for the LeNet architecture to train on the same
data. In essence, both models perform a spatial interpolation and thus we
expect them to yield a similar classification accuracy. Once the data is of
higher dimensions, the permutohedral convolution will be faster due to hashing
the sparse input points, as well as less memory demanding in comparison to
naive application of a spatial convolution with interpolated values.

\begin{table}[t]
  \begin{center}
    \footnotesize
    \centering
    \begin{tabular}[t]{lllll}
      \toprule
              &     & \multicolumn{3}{c}{Test Subsampling} \\
       Method  & Original & 100\% & 60\% & 20\%\\
      \midrule
       LeNet &  \textbf{0.9919} & 0.9660 & 0.9348 & \textbf{0.6434} \\
       BNN &  0.9903 & \textbf{0.9844} & \textbf{0.9534} & 0.5767 \\
      \hline
       LeNet 100\% & 0.9856 & 0.9809 & 0.9678 & \textbf{0.7386} \\
       BNN 100\% & \textbf{0.9900} & \textbf{0.9863} & \textbf{0.9699} & 0.6910 \\
      \hline
       LeNet 60\% & 0.9848 & 0.9821 & 0.9740 & 0.8151 \\
       BNN 60\% & \textbf{0.9885} & \textbf{0.9864} & \textbf{0.9771} & \textbf{0.8214}\\
      \hline
       LeNet 20\% & \textbf{0.9763} & \textbf{0.9754} & 0.9695 & 0.8928 \\
       BNN 20\% & 0.9728 & 0.9735 & \textbf{0.9701} & \textbf{0.9042}\\
      \bottomrule
    \end{tabular}
  \end{center}
\vspace{-.2cm}
\caption{Classification accuracy on MNIST. We compare the
    LeNet~\cite{lecun1998gradient} implementation that is part of
    Caffe~\cite{jia2014caffe} to the network with the first layer
    replaced by a bilateral convolution layer (BCL). Both are trained
    on the original image resolution (first two rows). Three more BNN
    and CNN models are trained with randomly subsampled images (100\%,
    60\% and 20\% of the pixels). An additional bilinear interpolation
    layer samples the input signal on a spatial grid for the CNN model.
  }
  \label{tab:all-results}
\vspace{-.5cm}
\end{table}

\subsubsection{Image Denoising}

The main application that inspired the development of the bilateral
filtering operation is image denoising~\cite{aurich1995non}, there
using a single Gaussian kernel. Our development allows to learn this
kernel function from data and we explore how to improve using a \emph{single}
but more general bilateral filter.

We use the Berkeley segmentation dataset
(BSDS500)~\cite{arbelaezi2011bsds500} as a test bed. The color
images in the dataset are converted to gray-scale,
and corrupted with Gaussian noise with a standard deviation of
$\frac {25} {255}$.

We compare the performance of four different filter models on a
denoising task.
The first baseline model (`Spatial' in Table \ref{tab:denoising}, $25$
weights) uses a single spatial filter with a kernel size of
$5$ and predicts the scalar gray-scale value at the center pixel. The next model
(`Gauss Bilateral') applies a bilateral \emph{Gaussian}
filter to the noisy input, using position and intensity features $\f=(x,y,v)^\top$.
The third setup (`Learned Bilateral', $65$ weights)
takes a Gauss kernel as initialization and
fits all filter weights on the train set to minimize the
mean squared error with respect to the clean images.
We run a combination
of spatial and permutohedral convolutions on spatial and bilateral
features (`Spatial + Bilateral (Learned)') to check for a complementary
performance of the two convolutions.

\label{sec:exp:denoising}
\begin{table}[!h]
\begin{center}
  \footnotesize
  \begin{tabular}[t]{lr}
    \toprule
    Method & PSNR \\
    \midrule
    Noisy Input & $20.17$ \\
    Spatial & $26.27$ \\
    Gauss Bilateral & $26.51$ \\
    Learned Bilateral & $26.58$ \\
    Spatial + Bilateral (Learned) & \textbf{$26.65$} \\
    \bottomrule
  \end{tabular}
\end{center}
\vspace{-0.5em}
\caption{PSNR results of a denoising task using the BSDS500
  dataset~\cite{arbelaezi2011bsds500}}
\vspace{-0.5em}
\label{tab:denoising}
\end{table}
\vspace{-0.2em}

The PSNR scores evaluated on full images of the test set are
shown in Table \ref{tab:denoising}. We find that an untrained bilateral
filter already performs better than a trained spatial convolution
($26.27$ to $26.51$). A learned convolution further improve the
performance slightly. We chose this simple one-kernel setup to
validate an advantage of the generalized bilateral filter. A competitive
denoising system would employ RGB color information and also
needs to be properly adjusted in network size. Multi-layer perceptrons
have obtained state-of-the-art denoising results~\cite{burger12cvpr}
and the permutohedral lattice layer can readily be used in such an
architecture, which is intended future work.

\subsection{Additional results}
\label{sec:addresults}

This section contains more qualitative results for the experiments presented in Chapter~\ref{chap:bnn}.

\begin{figure*}[th!]
  \centering
    \includegraphics[width=\columnwidth,trim={5cm 2.5cm 5cm 4.5cm},clip]{figures/supplementary/lattice_viz.pdf}
    \vspace{-0.7cm}
  \mycaption{Visualization of the Permutohedral Lattice}
  {Sample lattice visualizations for different feature spaces. All pixels falling in the same simplex cell are shown with
  the same color. $(x,y)$ features correspond to image pixel positions, and $(r,g,b) \in [0,255]$ correspond
  to the red, green and blue color values.}
\label{fig:latticeviz}
\end{figure*}

\subsubsection{Lattice Visualization}

Figure~\ref{fig:latticeviz} shows sample lattice visualizations for different feature spaces.

\newcolumntype{L}[1]{>{\raggedright\let\newline\\\arraybackslash\hspace{0pt}}b{#1}}
\newcolumntype{C}[1]{>{\centering\let\newline\\\arraybackslash\hspace{0pt}}b{#1}}
\newcolumntype{R}[1]{>{\raggedleft\let\newline\\\arraybackslash\hspace{0pt}}b{#1}}

\subsubsection{Color Upsampling}\label{sec:color_upsampling}
\label{sec:col_upsample_extra}

Some images of the upsampling for the Pascal
VOC12 dataset are shown in Fig.~\ref{fig:Colour_upsample_visuals}. It is
especially the low level image details that are better preserved with
a learned bilateral filter compared to the Gaussian case.

\begin{figure*}[t!]
  \centering
    \subfigure{%
   \raisebox{2.0em}{
    \includegraphics[width=.06\columnwidth]{figures/supplementary/2007_004969.jpg}
   }
  }
  \subfigure{%
    \includegraphics[width=.17\columnwidth]{figures/supplementary/2007_004969_gray.pdf}
  }
  \subfigure{%
    \includegraphics[width=.17\columnwidth]{figures/supplementary/2007_004969_gt.pdf}
  }
  \subfigure{%
    \includegraphics[width=.17\columnwidth]{figures/supplementary/2007_004969_bicubic.pdf}
  }
  \subfigure{%
    \includegraphics[width=.17\columnwidth]{figures/supplementary/2007_004969_gauss.pdf}
  }
  \subfigure{%
    \includegraphics[width=.17\columnwidth]{figures/supplementary/2007_004969_learnt.pdf}
  }\\
    \subfigure{%
   \raisebox{2.0em}{
    \includegraphics[width=.06\columnwidth]{figures/supplementary/2007_003106.jpg}
   }
  }
  \subfigure{%
    \includegraphics[width=.17\columnwidth]{figures/supplementary/2007_003106_gray.pdf}
  }
  \subfigure{%
    \includegraphics[width=.17\columnwidth]{figures/supplementary/2007_003106_gt.pdf}
  }
  \subfigure{%
    \includegraphics[width=.17\columnwidth]{figures/supplementary/2007_003106_bicubic.pdf}
  }
  \subfigure{%
    \includegraphics[width=.17\columnwidth]{figures/supplementary/2007_003106_gauss.pdf}
  }
  \subfigure{%
    \includegraphics[width=.17\columnwidth]{figures/supplementary/2007_003106_learnt.pdf}
  }\\
  \setcounter{subfigure}{0}
  \small{
  \subfigure[Inp.]{%
  \raisebox{2.0em}{
    \includegraphics[width=.06\columnwidth]{figures/supplementary/2007_006837.jpg}
   }
  }
  \subfigure[Guidance]{%
    \includegraphics[width=.17\columnwidth]{figures/supplementary/2007_006837_gray.pdf}
  }
   \subfigure[GT]{%
    \includegraphics[width=.17\columnwidth]{figures/supplementary/2007_006837_gt.pdf}
  }
  \subfigure[Bicubic]{%
    \includegraphics[width=.17\columnwidth]{figures/supplementary/2007_006837_bicubic.pdf}
  }
  \subfigure[Gauss-BF]{%
    \includegraphics[width=.17\columnwidth]{figures/supplementary/2007_006837_gauss.pdf}
  }
  \subfigure[Learned-BF]{%
    \includegraphics[width=.17\columnwidth]{figures/supplementary/2007_006837_learnt.pdf}
  }
  }
  \vspace{-0.5cm}
  \mycaption{Color Upsampling}{Color $8\times$ upsampling results
  using different methods, from left to right, (a)~Low-resolution input color image (Inp.),
  (b)~Gray scale guidance image, (c)~Ground-truth color image; Upsampled color images with
  (d)~Bicubic interpolation, (e) Gauss bilateral upsampling and, (f)~Learned bilateral
  updampgling (best viewed on screen).}

\label{fig:Colour_upsample_visuals}
\end{figure*}

\subsubsection{Depth Upsampling}
\label{sec:depth_upsample_extra}

Figure~\ref{fig:depth_upsample_visuals} presents some more qualitative results comparing bicubic interpolation, Gauss
bilateral and learned bilateral upsampling on NYU depth dataset image~\cite{silberman2012indoor}.

\subsubsection{Character Recognition}\label{sec:app_character}

 Figure~\ref{fig:nnrecognition} shows the schematic of different layers
 of the network architecture for LeNet-7~\cite{lecun1998mnist}
 and DeepCNet(5, 50)~\cite{ciresan2012multi,graham2014spatially}. For the BNN variants, the first layer filters are replaced
 with learned bilateral filters and are learned end-to-end.

\subsubsection{Semantic Segmentation}\label{sec:app_semantic_segmentation}
\label{sec:semantic_bnn_extra}

Some more visual results for semantic segmentation are shown in Figure~\ref{fig:semantic_visuals}.
These include the underlying DeepLab CNN\cite{chen2014semantic} result (DeepLab),
the 2 step mean-field result with Gaussian edge potentials (+2stepMF-GaussCRF)
and also corresponding results with learned edge potentials (+2stepMF-LearnedCRF).
In general, we observe that mean-field in learned CRF leads to slightly dilated
classification regions in comparison to using Gaussian CRF thereby filling-in the
false negative pixels and also correcting some mis-classified regions.

\begin{figure*}[t!]
  \centering
    \subfigure{%
   \raisebox{2.0em}{
    \includegraphics[width=.06\columnwidth]{figures/supplementary/2bicubic}
   }
  }
  \subfigure{%
    \includegraphics[width=.17\columnwidth]{figures/supplementary/2given_image}
  }
  \subfigure{%
    \includegraphics[width=.17\columnwidth]{figures/supplementary/2ground_truth}
  }
  \subfigure{%
    \includegraphics[width=.17\columnwidth]{figures/supplementary/2bicubic}
  }
  \subfigure{%
    \includegraphics[width=.17\columnwidth]{figures/supplementary/2gauss}
  }
  \subfigure{%
    \includegraphics[width=.17\columnwidth]{figures/supplementary/2learnt}
  }\\
    \subfigure{%
   \raisebox{2.0em}{
    \includegraphics[width=.06\columnwidth]{figures/supplementary/32bicubic}
   }
  }
  \subfigure{%
    \includegraphics[width=.17\columnwidth]{figures/supplementary/32given_image}
  }
  \subfigure{%
    \includegraphics[width=.17\columnwidth]{figures/supplementary/32ground_truth}
  }
  \subfigure{%
    \includegraphics[width=.17\columnwidth]{figures/supplementary/32bicubic}
  }
  \subfigure{%
    \includegraphics[width=.17\columnwidth]{figures/supplementary/32gauss}
  }
  \subfigure{%
    \includegraphics[width=.17\columnwidth]{figures/supplementary/32learnt}
  }\\
  \setcounter{subfigure}{0}
  \small{
  \subfigure[Inp.]{%
  \raisebox{2.0em}{
    \includegraphics[width=.06\columnwidth]{figures/supplementary/41bicubic}
   }
  }
  \subfigure[Guidance]{%
    \includegraphics[width=.17\columnwidth]{figures/supplementary/41given_image}
  }
   \subfigure[GT]{%
    \includegraphics[width=.17\columnwidth]{figures/supplementary/41ground_truth}
  }
  \subfigure[Bicubic]{%
    \includegraphics[width=.17\columnwidth]{figures/supplementary/41bicubic}
  }
  \subfigure[Gauss-BF]{%
    \includegraphics[width=.17\columnwidth]{figures/supplementary/41gauss}
  }
  \subfigure[Learned-BF]{%
    \includegraphics[width=.17\columnwidth]{figures/supplementary/41learnt}
  }
  }
  \mycaption{Depth Upsampling}{Depth $8\times$ upsampling results
  using different upsampling strategies, from left to right,
  (a)~Low-resolution input depth image (Inp.),
  (b)~High-resolution guidance image, (c)~Ground-truth depth; Upsampled depth images with
  (d)~Bicubic interpolation, (e) Gauss bilateral upsampling and, (f)~Learned bilateral
  updampgling (best viewed on screen).}

\label{fig:depth_upsample_visuals}
\end{figure*}

\subsubsection{Material Segmentation}\label{sec:app_material_segmentation}
\label{sec:material_bnn_extra}

In Fig.~\ref{fig:material_visuals-app2}, we present visual results comparing 2 step
mean-field inference with Gaussian and learned pairwise CRF potentials. In
general, we observe that the pixels belonging to dominant classes in the
training data are being more accurately classified with learned CRF. This leads to
a significant improvements in overall pixel accuracy. This also results
in a slight decrease of the accuracy from less frequent class pixels thereby
slightly reducing the average class accuracy with learning. We attribute this
to the type of annotation that is available for this dataset, which is not
for the entire image but for some segments in the image. We have very few
images of the infrequent classes to combat this behaviour during training.

\subsubsection{Experiment Protocols}
\label{sec:protocols}

Table~\ref{tbl:parameters} shows experiment protocols of different experiments.

 \begin{figure*}[t!]
  \centering
  \subfigure[LeNet-7]{
    \includegraphics[width=0.7\columnwidth]{figures/supplementary/lenet_cnn_network}
    }\\
    \subfigure[DeepCNet]{
    \includegraphics[width=\columnwidth]{figures/supplementary/deepcnet_cnn_network}
    }
  \mycaption{CNNs for Character Recognition}
  {Schematic of (top) LeNet-7~\cite{lecun1998mnist} and (bottom) DeepCNet(5,50)~\cite{ciresan2012multi,graham2014spatially} architectures used in Assamese
  character recognition experiments.}
\label{fig:nnrecognition}
\end{figure*}

\definecolor{voc_1}{RGB}{0, 0, 0}
\definecolor{voc_2}{RGB}{128, 0, 0}
\definecolor{voc_3}{RGB}{0, 128, 0}
\definecolor{voc_4}{RGB}{128, 128, 0}
\definecolor{voc_5}{RGB}{0, 0, 128}
\definecolor{voc_6}{RGB}{128, 0, 128}
\definecolor{voc_7}{RGB}{0, 128, 128}
\definecolor{voc_8}{RGB}{128, 128, 128}
\definecolor{voc_9}{RGB}{64, 0, 0}
\definecolor{voc_10}{RGB}{192, 0, 0}
\definecolor{voc_11}{RGB}{64, 128, 0}
\definecolor{voc_12}{RGB}{192, 128, 0}
\definecolor{voc_13}{RGB}{64, 0, 128}
\definecolor{voc_14}{RGB}{192, 0, 128}
\definecolor{voc_15}{RGB}{64, 128, 128}
\definecolor{voc_16}{RGB}{192, 128, 128}
\definecolor{voc_17}{RGB}{0, 64, 0}
\definecolor{voc_18}{RGB}{128, 64, 0}
\definecolor{voc_19}{RGB}{0, 192, 0}
\definecolor{voc_20}{RGB}{128, 192, 0}
\definecolor{voc_21}{RGB}{0, 64, 128}
\definecolor{voc_22}{RGB}{128, 64, 128}

\begin{figure*}[t]
  \centering
  \small{
  \fcolorbox{white}{voc_1}{\rule{0pt}{6pt}\rule{6pt}{0pt}} Background~~
  \fcolorbox{white}{voc_2}{\rule{0pt}{6pt}\rule{6pt}{0pt}} Aeroplane~~
  \fcolorbox{white}{voc_3}{\rule{0pt}{6pt}\rule{6pt}{0pt}} Bicycle~~
  \fcolorbox{white}{voc_4}{\rule{0pt}{6pt}\rule{6pt}{0pt}} Bird~~
  \fcolorbox{white}{voc_5}{\rule{0pt}{6pt}\rule{6pt}{0pt}} Boat~~
  \fcolorbox{white}{voc_6}{\rule{0pt}{6pt}\rule{6pt}{0pt}} Bottle~~
  \fcolorbox{white}{voc_7}{\rule{0pt}{6pt}\rule{6pt}{0pt}} Bus~~
  \fcolorbox{white}{voc_8}{\rule{0pt}{6pt}\rule{6pt}{0pt}} Car~~ \\
  \fcolorbox{white}{voc_9}{\rule{0pt}{6pt}\rule{6pt}{0pt}} Cat~~
  \fcolorbox{white}{voc_10}{\rule{0pt}{6pt}\rule{6pt}{0pt}} Chair~~
  \fcolorbox{white}{voc_11}{\rule{0pt}{6pt}\rule{6pt}{0pt}} Cow~~
  \fcolorbox{white}{voc_12}{\rule{0pt}{6pt}\rule{6pt}{0pt}} Dining Table~~
  \fcolorbox{white}{voc_13}{\rule{0pt}{6pt}\rule{6pt}{0pt}} Dog~~
  \fcolorbox{white}{voc_14}{\rule{0pt}{6pt}\rule{6pt}{0pt}} Horse~~
  \fcolorbox{white}{voc_15}{\rule{0pt}{6pt}\rule{6pt}{0pt}} Motorbike~~
  \fcolorbox{white}{voc_16}{\rule{0pt}{6pt}\rule{6pt}{0pt}} Person~~ \\
  \fcolorbox{white}{voc_17}{\rule{0pt}{6pt}\rule{6pt}{0pt}} Potted Plant~~
  \fcolorbox{white}{voc_18}{\rule{0pt}{6pt}\rule{6pt}{0pt}} Sheep~~
  \fcolorbox{white}{voc_19}{\rule{0pt}{6pt}\rule{6pt}{0pt}} Sofa~~
  \fcolorbox{white}{voc_20}{\rule{0pt}{6pt}\rule{6pt}{0pt}} Train~~
  \fcolorbox{white}{voc_21}{\rule{0pt}{6pt}\rule{6pt}{0pt}} TV monitor~~ \\
  }
  \subfigure{%
    \includegraphics[width=.18\columnwidth]{figures/supplementary/2007_001423_given.jpg}
  }
  \subfigure{%
    \includegraphics[width=.18\columnwidth]{figures/supplementary/2007_001423_gt.png}
  }
  \subfigure{%
    \includegraphics[width=.18\columnwidth]{figures/supplementary/2007_001423_cnn.png}
  }
  \subfigure{%
    \includegraphics[width=.18\columnwidth]{figures/supplementary/2007_001423_gauss.png}
  }
  \subfigure{%
    \includegraphics[width=.18\columnwidth]{figures/supplementary/2007_001423_learnt.png}
  }\\
  \subfigure{%
    \includegraphics[width=.18\columnwidth]{figures/supplementary/2007_001430_given.jpg}
  }
  \subfigure{%
    \includegraphics[width=.18\columnwidth]{figures/supplementary/2007_001430_gt.png}
  }
  \subfigure{%
    \includegraphics[width=.18\columnwidth]{figures/supplementary/2007_001430_cnn.png}
  }
  \subfigure{%
    \includegraphics[width=.18\columnwidth]{figures/supplementary/2007_001430_gauss.png}
  }
  \subfigure{%
    \includegraphics[width=.18\columnwidth]{figures/supplementary/2007_001430_learnt.png}
  }\\
    \subfigure{%
    \includegraphics[width=.18\columnwidth]{figures/supplementary/2007_007996_given.jpg}
  }
  \subfigure{%
    \includegraphics[width=.18\columnwidth]{figures/supplementary/2007_007996_gt.png}
  }
  \subfigure{%
    \includegraphics[width=.18\columnwidth]{figures/supplementary/2007_007996_cnn.png}
  }
  \subfigure{%
    \includegraphics[width=.18\columnwidth]{figures/supplementary/2007_007996_gauss.png}
  }
  \subfigure{%
    \includegraphics[width=.18\columnwidth]{figures/supplementary/2007_007996_learnt.png}
  }\\
   \subfigure{%
    \includegraphics[width=.18\columnwidth]{figures/supplementary/2010_002682_given.jpg}
  }
  \subfigure{%
    \includegraphics[width=.18\columnwidth]{figures/supplementary/2010_002682_gt.png}
  }
  \subfigure{%
    \includegraphics[width=.18\columnwidth]{figures/supplementary/2010_002682_cnn.png}
  }
  \subfigure{%
    \includegraphics[width=.18\columnwidth]{figures/supplementary/2010_002682_gauss.png}
  }
  \subfigure{%
    \includegraphics[width=.18\columnwidth]{figures/supplementary/2010_002682_learnt.png}
  }\\
     \subfigure{%
    \includegraphics[width=.18\columnwidth]{figures/supplementary/2010_004789_given.jpg}
  }
  \subfigure{%
    \includegraphics[width=.18\columnwidth]{figures/supplementary/2010_004789_gt.png}
  }
  \subfigure{%
    \includegraphics[width=.18\columnwidth]{figures/supplementary/2010_004789_cnn.png}
  }
  \subfigure{%
    \includegraphics[width=.18\columnwidth]{figures/supplementary/2010_004789_gauss.png}
  }
  \subfigure{%
    \includegraphics[width=.18\columnwidth]{figures/supplementary/2010_004789_learnt.png}
  }\\
       \subfigure{%
    \includegraphics[width=.18\columnwidth]{figures/supplementary/2007_001311_given.jpg}
  }
  \subfigure{%
    \includegraphics[width=.18\columnwidth]{figures/supplementary/2007_001311_gt.png}
  }
  \subfigure{%
    \includegraphics[width=.18\columnwidth]{figures/supplementary/2007_001311_cnn.png}
  }
  \subfigure{%
    \includegraphics[width=.18\columnwidth]{figures/supplementary/2007_001311_gauss.png}
  }
  \subfigure{%
    \includegraphics[width=.18\columnwidth]{figures/supplementary/2007_001311_learnt.png}
  }\\
  \setcounter{subfigure}{0}
  \subfigure[Input]{%
    \includegraphics[width=.18\columnwidth]{figures/supplementary/2010_003531_given.jpg}
  }
  \subfigure[Ground Truth]{%
    \includegraphics[width=.18\columnwidth]{figures/supplementary/2010_003531_gt.png}
  }
  \subfigure[DeepLab]{%
    \includegraphics[width=.18\columnwidth]{figures/supplementary/2010_003531_cnn.png}
  }
  \subfigure[+GaussCRF]{%
    \includegraphics[width=.18\columnwidth]{figures/supplementary/2010_003531_gauss.png}
  }
  \subfigure[+LearnedCRF]{%
    \includegraphics[width=.18\columnwidth]{figures/supplementary/2010_003531_learnt.png}
  }
  \vspace{-0.3cm}
  \mycaption{Semantic Segmentation}{Example results of semantic segmentation.
  (c)~depicts the unary results before application of MF, (d)~after two steps of MF with Gaussian edge CRF potentials, (e)~after
  two steps of MF with learned edge CRF potentials.}
    \label{fig:semantic_visuals}
\end{figure*}


\definecolor{minc_1}{HTML}{771111}
\definecolor{minc_2}{HTML}{CAC690}
\definecolor{minc_3}{HTML}{EEEEEE}
\definecolor{minc_4}{HTML}{7C8FA6}
\definecolor{minc_5}{HTML}{597D31}
\definecolor{minc_6}{HTML}{104410}
\definecolor{minc_7}{HTML}{BB819C}
\definecolor{minc_8}{HTML}{D0CE48}
\definecolor{minc_9}{HTML}{622745}
\definecolor{minc_10}{HTML}{666666}
\definecolor{minc_11}{HTML}{D54A31}
\definecolor{minc_12}{HTML}{101044}
\definecolor{minc_13}{HTML}{444126}
\definecolor{minc_14}{HTML}{75D646}
\definecolor{minc_15}{HTML}{DD4348}
\definecolor{minc_16}{HTML}{5C8577}
\definecolor{minc_17}{HTML}{C78472}
\definecolor{minc_18}{HTML}{75D6D0}
\definecolor{minc_19}{HTML}{5B4586}
\definecolor{minc_20}{HTML}{C04393}
\definecolor{minc_21}{HTML}{D69948}
\definecolor{minc_22}{HTML}{7370D8}
\definecolor{minc_23}{HTML}{7A3622}
\definecolor{minc_24}{HTML}{000000}

\begin{figure*}[t]
  \centering
  \small{
  \fcolorbox{white}{minc_1}{\rule{0pt}{6pt}\rule{6pt}{0pt}} Brick~~
  \fcolorbox{white}{minc_2}{\rule{0pt}{6pt}\rule{6pt}{0pt}} Carpet~~
  \fcolorbox{white}{minc_3}{\rule{0pt}{6pt}\rule{6pt}{0pt}} Ceramic~~
  \fcolorbox{white}{minc_4}{\rule{0pt}{6pt}\rule{6pt}{0pt}} Fabric~~
  \fcolorbox{white}{minc_5}{\rule{0pt}{6pt}\rule{6pt}{0pt}} Foliage~~
  \fcolorbox{white}{minc_6}{\rule{0pt}{6pt}\rule{6pt}{0pt}} Food~~
  \fcolorbox{white}{minc_7}{\rule{0pt}{6pt}\rule{6pt}{0pt}} Glass~~
  \fcolorbox{white}{minc_8}{\rule{0pt}{6pt}\rule{6pt}{0pt}} Hair~~ \\
  \fcolorbox{white}{minc_9}{\rule{0pt}{6pt}\rule{6pt}{0pt}} Leather~~
  \fcolorbox{white}{minc_10}{\rule{0pt}{6pt}\rule{6pt}{0pt}} Metal~~
  \fcolorbox{white}{minc_11}{\rule{0pt}{6pt}\rule{6pt}{0pt}} Mirror~~
  \fcolorbox{white}{minc_12}{\rule{0pt}{6pt}\rule{6pt}{0pt}} Other~~
  \fcolorbox{white}{minc_13}{\rule{0pt}{6pt}\rule{6pt}{0pt}} Painted~~
  \fcolorbox{white}{minc_14}{\rule{0pt}{6pt}\rule{6pt}{0pt}} Paper~~
  \fcolorbox{white}{minc_15}{\rule{0pt}{6pt}\rule{6pt}{0pt}} Plastic~~\\
  \fcolorbox{white}{minc_16}{\rule{0pt}{6pt}\rule{6pt}{0pt}} Polished Stone~~
  \fcolorbox{white}{minc_17}{\rule{0pt}{6pt}\rule{6pt}{0pt}} Skin~~
  \fcolorbox{white}{minc_18}{\rule{0pt}{6pt}\rule{6pt}{0pt}} Sky~~
  \fcolorbox{white}{minc_19}{\rule{0pt}{6pt}\rule{6pt}{0pt}} Stone~~
  \fcolorbox{white}{minc_20}{\rule{0pt}{6pt}\rule{6pt}{0pt}} Tile~~
  \fcolorbox{white}{minc_21}{\rule{0pt}{6pt}\rule{6pt}{0pt}} Wallpaper~~
  \fcolorbox{white}{minc_22}{\rule{0pt}{6pt}\rule{6pt}{0pt}} Water~~
  \fcolorbox{white}{minc_23}{\rule{0pt}{6pt}\rule{6pt}{0pt}} Wood~~ \\
  }
  \subfigure{%
    \includegraphics[width=.18\columnwidth]{figures/supplementary/000010868_given.jpg}
  }
  \subfigure{%
    \includegraphics[width=.18\columnwidth]{figures/supplementary/000010868_gt.png}
  }
  \subfigure{%
    \includegraphics[width=.18\columnwidth]{figures/supplementary/000010868_cnn.png}
  }
  \subfigure{%
    \includegraphics[width=.18\columnwidth]{figures/supplementary/000010868_gauss.png}
  }
  \subfigure{%
    \includegraphics[width=.18\columnwidth]{figures/supplementary/000010868_learnt.png}
  }\\[-2ex]
  \subfigure{%
    \includegraphics[width=.18\columnwidth]{figures/supplementary/000006011_given.jpg}
  }
  \subfigure{%
    \includegraphics[width=.18\columnwidth]{figures/supplementary/000006011_gt.png}
  }
  \subfigure{%
    \includegraphics[width=.18\columnwidth]{figures/supplementary/000006011_cnn.png}
  }
  \subfigure{%
    \includegraphics[width=.18\columnwidth]{figures/supplementary/000006011_gauss.png}
  }
  \subfigure{%
    \includegraphics[width=.18\columnwidth]{figures/supplementary/000006011_learnt.png}
  }\\[-2ex]
    \subfigure{%
    \includegraphics[width=.18\columnwidth]{figures/supplementary/000008553_given.jpg}
  }
  \subfigure{%
    \includegraphics[width=.18\columnwidth]{figures/supplementary/000008553_gt.png}
  }
  \subfigure{%
    \includegraphics[width=.18\columnwidth]{figures/supplementary/000008553_cnn.png}
  }
  \subfigure{%
    \includegraphics[width=.18\columnwidth]{figures/supplementary/000008553_gauss.png}
  }
  \subfigure{%
    \includegraphics[width=.18\columnwidth]{figures/supplementary/000008553_learnt.png}
  }\\[-2ex]
   \subfigure{%
    \includegraphics[width=.18\columnwidth]{figures/supplementary/000009188_given.jpg}
  }
  \subfigure{%
    \includegraphics[width=.18\columnwidth]{figures/supplementary/000009188_gt.png}
  }
  \subfigure{%
    \includegraphics[width=.18\columnwidth]{figures/supplementary/000009188_cnn.png}
  }
  \subfigure{%
    \includegraphics[width=.18\columnwidth]{figures/supplementary/000009188_gauss.png}
  }
  \subfigure{%
    \includegraphics[width=.18\columnwidth]{figures/supplementary/000009188_learnt.png}
  }\\[-2ex]
  \setcounter{subfigure}{0}
  \subfigure[Input]{%
    \includegraphics[width=.18\columnwidth]{figures/supplementary/000023570_given.jpg}
  }
  \subfigure[Ground Truth]{%
    \includegraphics[width=.18\columnwidth]{figures/supplementary/000023570_gt.png}
  }
  \subfigure[DeepLab]{%
    \includegraphics[width=.18\columnwidth]{figures/supplementary/000023570_cnn.png}
  }
  \subfigure[+GaussCRF]{%
    \includegraphics[width=.18\columnwidth]{figures/supplementary/000023570_gauss.png}
  }
  \subfigure[+LearnedCRF]{%
    \includegraphics[width=.18\columnwidth]{figures/supplementary/000023570_learnt.png}
  }
  \mycaption{Material Segmentation}{Example results of material segmentation.
  (c)~depicts the unary results before application of MF, (d)~after two steps of MF with Gaussian edge CRF potentials, (e)~after two steps of MF with learned edge CRF potentials.}
    \label{fig:material_visuals-app2}
\end{figure*}


\begin{table*}[h]
\tiny
  \centering
    \begin{tabular}{L{2.3cm} L{2.25cm} C{1.5cm} C{0.7cm} C{0.6cm} C{0.7cm} C{0.7cm} C{0.7cm} C{1.6cm} C{0.6cm} C{0.6cm} C{0.6cm}}
      \toprule
& & & & & \multicolumn{3}{c}{\textbf{Data Statistics}} & \multicolumn{4}{c}{\textbf{Training Protocol}} \\

\textbf{Experiment} & \textbf{Feature Types} & \textbf{Feature Scales} & \textbf{Filter Size} & \textbf{Filter Nbr.} & \textbf{Train}  & \textbf{Val.} & \textbf{Test} & \textbf{Loss Type} & \textbf{LR} & \textbf{Batch} & \textbf{Epochs} \\
      \midrule
      \multicolumn{2}{c}{\textbf{Single Bilateral Filter Applications}} & & & & & & & & & \\
      \textbf{2$\times$ Color Upsampling} & Position$_{1}$, Intensity (3D) & 0.13, 0.17 & 65 & 2 & 10581 & 1449 & 1456 & MSE & 1e-06 & 200 & 94.5\\
      \textbf{4$\times$ Color Upsampling} & Position$_{1}$, Intensity (3D) & 0.06, 0.17 & 65 & 2 & 10581 & 1449 & 1456 & MSE & 1e-06 & 200 & 94.5\\
      \textbf{8$\times$ Color Upsampling} & Position$_{1}$, Intensity (3D) & 0.03, 0.17 & 65 & 2 & 10581 & 1449 & 1456 & MSE & 1e-06 & 200 & 94.5\\
      \textbf{16$\times$ Color Upsampling} & Position$_{1}$, Intensity (3D) & 0.02, 0.17 & 65 & 2 & 10581 & 1449 & 1456 & MSE & 1e-06 & 200 & 94.5\\
      \textbf{Depth Upsampling} & Position$_{1}$, Color (5D) & 0.05, 0.02 & 665 & 2 & 795 & 100 & 654 & MSE & 1e-07 & 50 & 251.6\\
      \textbf{Mesh Denoising} & Isomap (4D) & 46.00 & 63 & 2 & 1000 & 200 & 500 & MSE & 100 & 10 & 100.0 \\
      \midrule
      \multicolumn{2}{c}{\textbf{DenseCRF Applications}} & & & & & & & & &\\
      \multicolumn{2}{l}{\textbf{Semantic Segmentation}} & & & & & & & & &\\
      \textbf{- 1step MF} & Position$_{1}$, Color (5D); Position$_{1}$ (2D) & 0.01, 0.34; 0.34  & 665; 19  & 2; 2 & 10581 & 1449 & 1456 & Logistic & 0.1 & 5 & 1.4 \\
      \textbf{- 2step MF} & Position$_{1}$, Color (5D); Position$_{1}$ (2D) & 0.01, 0.34; 0.34 & 665; 19 & 2; 2 & 10581 & 1449 & 1456 & Logistic & 0.1 & 5 & 1.4 \\
      \textbf{- \textit{loose} 2step MF} & Position$_{1}$, Color (5D); Position$_{1}$ (2D) & 0.01, 0.34; 0.34 & 665; 19 & 2; 2 &10581 & 1449 & 1456 & Logistic & 0.1 & 5 & +1.9  \\ \\
      \multicolumn{2}{l}{\textbf{Material Segmentation}} & & & & & & & & &\\
      \textbf{- 1step MF} & Position$_{2}$, Lab-Color (5D) & 5.00, 0.05, 0.30  & 665 & 2 & 928 & 150 & 1798 & Weighted Logistic & 1e-04 & 24 & 2.6 \\
      \textbf{- 2step MF} & Position$_{2}$, Lab-Color (5D) & 5.00, 0.05, 0.30 & 665 & 2 & 928 & 150 & 1798 & Weighted Logistic & 1e-04 & 12 & +0.7 \\
      \textbf{- \textit{loose} 2step MF} & Position$_{2}$, Lab-Color (5D) & 5.00, 0.05, 0.30 & 665 & 2 & 928 & 150 & 1798 & Weighted Logistic & 1e-04 & 12 & +0.2\\
      \midrule
      \multicolumn{2}{c}{\textbf{Neural Network Applications}} & & & & & & & & &\\
      \textbf{Tiles: CNN-9$\times$9} & - & - & 81 & 4 & 10000 & 1000 & 1000 & Logistic & 0.01 & 100 & 500.0 \\
      \textbf{Tiles: CNN-13$\times$13} & - & - & 169 & 6 & 10000 & 1000 & 1000 & Logistic & 0.01 & 100 & 500.0 \\
      \textbf{Tiles: CNN-17$\times$17} & - & - & 289 & 8 & 10000 & 1000 & 1000 & Logistic & 0.01 & 100 & 500.0 \\
      \textbf{Tiles: CNN-21$\times$21} & - & - & 441 & 10 & 10000 & 1000 & 1000 & Logistic & 0.01 & 100 & 500.0 \\
      \textbf{Tiles: BNN} & Position$_{1}$, Color (5D) & 0.05, 0.04 & 63 & 1 & 10000 & 1000 & 1000 & Logistic & 0.01 & 100 & 30.0 \\
      \textbf{LeNet} & - & - & 25 & 2 & 5490 & 1098 & 1647 & Logistic & 0.1 & 100 & 182.2 \\
      \textbf{Crop-LeNet} & - & - & 25 & 2 & 5490 & 1098 & 1647 & Logistic & 0.1 & 100 & 182.2 \\
      \textbf{BNN-LeNet} & Position$_{2}$ (2D) & 20.00 & 7 & 1 & 5490 & 1098 & 1647 & Logistic & 0.1 & 100 & 182.2 \\
      \textbf{DeepCNet} & - & - & 9 & 1 & 5490 & 1098 & 1647 & Logistic & 0.1 & 100 & 182.2 \\
      \textbf{Crop-DeepCNet} & - & - & 9 & 1 & 5490 & 1098 & 1647 & Logistic & 0.1 & 100 & 182.2 \\
      \textbf{BNN-DeepCNet} & Position$_{2}$ (2D) & 40.00  & 7 & 1 & 5490 & 1098 & 1647 & Logistic & 0.1 & 100 & 182.2 \\
      \bottomrule
      \\
    \end{tabular}
    \mycaption{Experiment Protocols} {Experiment protocols for the different experiments presented in this work. \textbf{Feature Types}:
    Feature spaces used for the bilateral convolutions. Position$_1$ corresponds to un-normalized pixel positions whereas Position$_2$ corresponds
    to pixel positions normalized to $[0,1]$ with respect to the given image. \textbf{Feature Scales}: Cross-validated scales for the features used.
     \textbf{Filter Size}: Number of elements in the filter that is being learned. \textbf{Filter Nbr.}: Half-width of the filter. \textbf{Train},
     \textbf{Val.} and \textbf{Test} corresponds to the number of train, validation and test images used in the experiment. \textbf{Loss Type}: Type
     of loss used for back-propagation. ``MSE'' corresponds to Euclidean mean squared error loss and ``Logistic'' corresponds to multinomial logistic
     loss. ``Weighted Logistic'' is the class-weighted multinomial logistic loss. We weighted the loss with inverse class probability for material
     segmentation task due to the small availability of training data with class imbalance. \textbf{LR}: Fixed learning rate used in stochastic gradient
     descent. \textbf{Batch}: Number of images used in one parameter update step. \textbf{Epochs}: Number of training epochs. In all the experiments,
     we used fixed momentum of 0.9 and weight decay of 0.0005 for stochastic gradient descent. ```Color Upsampling'' experiments in this Table corresponds
     to those performed on Pascal VOC12 dataset images. For all experiments using Pascal VOC12 images, we use extended
     training segmentation dataset available from~\cite{hariharan2011moredata}, and used standard validation and test splits
     from the main dataset~\cite{voc2012segmentation}.}
  \label{tbl:parameters}
\end{table*}

\clearpage

\section{Parameters and Additional Results for Video Propagation Networks}

In this Section, we present experiment protocols and additional qualitative results for experiments
on video object segmentation, semantic video segmentation and video color
propagation. Table~\ref{tbl:parameters_supp} shows the feature scales and other parameters used in different experiments.
Figures~\ref{fig:video_seg_pos_supp} show some qualitative results on video object segmentation
with some failure cases in Fig.~\ref{fig:video_seg_neg_supp}.
Figure~\ref{fig:semantic_visuals_supp} shows some qualitative results on semantic video segmentation and
Fig.~\ref{fig:color_visuals_supp} shows results on video color propagation.

\newcolumntype{L}[1]{>{\raggedright\let\newline\\\arraybackslash\hspace{0pt}}b{#1}}
\newcolumntype{C}[1]{>{\centering\let\newline\\\arraybackslash\hspace{0pt}}b{#1}}
\newcolumntype{R}[1]{>{\raggedleft\let\newline\\\arraybackslash\hspace{0pt}}b{#1}}

\begin{table*}[h]
\tiny
  \centering
    \begin{tabular}{L{3.0cm} L{2.4cm} L{2.8cm} L{2.8cm} C{0.5cm} C{1.0cm} L{1.2cm}}
      \toprule
\textbf{Experiment} & \textbf{Feature Type} & \textbf{Feature Scale-1, $\Lambda_a$} & \textbf{Feature Scale-2, $\Lambda_b$} & \textbf{$\alpha$} & \textbf{Input Frames} & \textbf{Loss Type} \\
      \midrule
      \textbf{Video Object Segmentation} & ($x,y,Y,Cb,Cr,t$) & (0.02,0.02,0.07,0.4,0.4,0.01) & (0.03,0.03,0.09,0.5,0.5,0.2) & 0.5 & 9 & Logistic\\
      \midrule
      \textbf{Semantic Video Segmentation} & & & & & \\
      \textbf{with CNN1~\cite{yu2015multi}-NoFlow} & ($x,y,R,G,B,t$) & (0.08,0.08,0.2,0.2,0.2,0.04) & (0.11,0.11,0.2,0.2,0.2,0.04) & 0.5 & 3 & Logistic \\
      \textbf{with CNN1~\cite{yu2015multi}-Flow} & ($x+u_x,y+u_y,R,G,B,t$) & (0.11,0.11,0.14,0.14,0.14,0.03) & (0.08,0.08,0.12,0.12,0.12,0.01) & 0.65 & 3 & Logistic\\
      \textbf{with CNN2~\cite{richter2016playing}-Flow} & ($x+u_x,y+u_y,R,G,B,t$) & (0.08,0.08,0.2,0.2,0.2,0.04) & (0.09,0.09,0.25,0.25,0.25,0.03) & 0.5 & 4 & Logistic\\
      \midrule
      \textbf{Video Color Propagation} & ($x,y,I,t$)  & (0.04,0.04,0.2,0.04) & No second kernel & 1 & 4 & MSE\\
      \bottomrule
      \\
    \end{tabular}
    \mycaption{Experiment Protocols} {Experiment protocols for the different experiments presented in this work. \textbf{Feature Types}:
    Feature spaces used for the bilateral convolutions, with position ($x,y$) and color
    ($R,G,B$ or $Y,Cb,Cr$) features $\in [0,255]$. $u_x$, $u_y$ denotes optical flow with respect
    to the present frame and $I$ denotes grayscale intensity.
    \textbf{Feature Scales ($\Lambda_a, \Lambda_b$)}: Cross-validated scales for the features used.
    \textbf{$\alpha$}: Exponential time decay for the input frames.
    \textbf{Input Frames}: Number of input frames for VPN.
    \textbf{Loss Type}: Type
     of loss used for back-propagation. ``MSE'' corresponds to Euclidean mean squared error loss and ``Logistic'' corresponds to multinomial logistic loss.}
  \label{tbl:parameters_supp}
\end{table*}

% \begin{figure}[th!]
% \begin{center}
%   \centerline{\includegraphics[width=\textwidth]{figures/video_seg_visuals_supp_small.pdf}}
%     \mycaption{Video Object Segmentation}
%     {Shown are the different frames in example videos with the corresponding
%     ground truth (GT) masks, predictions from BVS~\cite{marki2016bilateral},
%     OFL~\cite{tsaivideo}, VPN (VPN-Stage2) and VPN-DLab (VPN-DeepLab) models.}
%     \label{fig:video_seg_small_supp}
% \end{center}
% \vspace{-1.0cm}
% \end{figure}

\begin{figure}[th!]
\begin{center}
  \centerline{\includegraphics[width=0.7\textwidth]{figures/video_seg_visuals_supp_positive.pdf}}
    \mycaption{Video Object Segmentation}
    {Shown are the different frames in example videos with the corresponding
    ground truth (GT) masks, predictions from BVS~\cite{marki2016bilateral},
    OFL~\cite{tsaivideo}, VPN (VPN-Stage2) and VPN-DLab (VPN-DeepLab) models.}
    \label{fig:video_seg_pos_supp}
\end{center}
\vspace{-1.0cm}
\end{figure}

\begin{figure}[th!]
\begin{center}
  \centerline{\includegraphics[width=0.7\textwidth]{figures/video_seg_visuals_supp_negative.pdf}}
    \mycaption{Failure Cases for Video Object Segmentation}
    {Shown are the different frames in example videos with the corresponding
    ground truth (GT) masks, predictions from BVS~\cite{marki2016bilateral},
    OFL~\cite{tsaivideo}, VPN (VPN-Stage2) and VPN-DLab (VPN-DeepLab) models.}
    \label{fig:video_seg_neg_supp}
\end{center}
\vspace{-1.0cm}
\end{figure}

\begin{figure}[th!]
\begin{center}
  \centerline{\includegraphics[width=0.9\textwidth]{figures/supp_semantic_visual.pdf}}
    \mycaption{Semantic Video Segmentation}
    {Input video frames and the corresponding ground truth (GT)
    segmentation together with the predictions of CNN~\cite{yu2015multi} and with
    VPN-Flow.}
    \label{fig:semantic_visuals_supp}
\end{center}
\vspace{-0.7cm}
\end{figure}

\begin{figure}[th!]
\begin{center}
  \centerline{\includegraphics[width=\textwidth]{figures/colorization_visuals_supp.pdf}}
  \mycaption{Video Color Propagation}
  {Input grayscale video frames and corresponding ground-truth (GT) color images
  together with color predictions of Levin et al.~\cite{levin2004colorization} and VPN-Stage1 models.}
  \label{fig:color_visuals_supp}
\end{center}
\vspace{-0.7cm}
\end{figure}

\clearpage

\section{Additional Material for Bilateral Inception Networks}
\label{sec:binception-app}

In this section of the Appendix, we first discuss the use of approximate bilateral
filtering in BI modules (Sec.~\ref{sec:lattice}).
Later, we present some qualitative results using different models for the approach presented in
Chapter~\ref{chap:binception} (Sec.~\ref{sec:qualitative-app}).

\subsection{Approximate Bilateral Filtering}
\label{sec:lattice}

The bilateral inception module presented in Chapter~\ref{chap:binception} computes a matrix-vector
product between a Gaussian filter $K$ and a vector of activations $\bz_c$.
Bilateral filtering is an important operation and many algorithmic techniques have been
proposed to speed-up this operation~\cite{paris2006fast,adams2010fast,gastal2011domain}.
In the main paper we opted to implement what can be considered the
brute-force variant of explicitly constructing $K$ and then using BLAS to compute the
matrix-vector product. This resulted in a few millisecond operation.
The explicit way to compute is possible due to the
reduction to super-pixels, e.g., it would not work for DenseCRF variants
that operate on the full image resolution.

Here, we present experiments where we use the fast approximate bilateral filtering
algorithm of~\cite{adams2010fast}, which is also used in Chapter~\ref{chap:bnn}
for learning sparse high dimensional filters. This
choice allows for larger dimensions of matrix-vector multiplication. The reason for choosing
the explicit multiplication in Chapter~\ref{chap:binception} was that it was computationally faster.
For the small sizes of the involved matrices and vectors, the explicit computation is sufficient and we had no
GPU implementation of an approximate technique that matched this runtime. Also it
is conceptually easier and the gradient to the feature transformations ($\Lambda \mathbf{f}$) is
obtained using standard matrix calculus.

\subsubsection{Experiments}

We modified the existing segmentation architectures analogous to those in Chapter~\ref{chap:binception}.
The main difference is that, here, the inception modules use the lattice
approximation~\cite{adams2010fast} to compute the bilateral filtering.
Using the lattice approximation did not allow us to back-propagate through feature transformations ($\Lambda$)
and thus we used hand-specified feature scales as will be explained later.
Specifically, we take CNN architectures from the works
of~\cite{chen2014semantic,zheng2015conditional,bell2015minc} and insert the BI modules between
the spatial FC layers.
We use superpixels from~\cite{DollarICCV13edges}
for all the experiments with the lattice approximation. Experiments are
performed using Caffe neural network framework~\cite{jia2014caffe}.

\begin{table}
  \small
  \centering
  \begin{tabular}{p{5.5cm}>{\raggedright\arraybackslash}p{1.4cm}>{\centering\arraybackslash}p{2.2cm}}
    \toprule
		\textbf{Model} & \emph{IoU} & \emph{Runtime}(ms) \\
    \midrule

    %%%%%%%%%%%% Scores computed by us)%%%%%%%%%%%%
		\deeplablargefov & 68.9 & 145ms\\
    \midrule
    \bi{7}{2}-\bi{8}{10}& \textbf{73.8} & +600 \\
    \midrule
    \deeplablargefovcrf~\cite{chen2014semantic} & 72.7 & +830\\
    \deeplabmsclargefovcrf~\cite{chen2014semantic} & \textbf{73.6} & +880\\
    DeepLab-EdgeNet~\cite{chen2015semantic} & 71.7 & +30\\
    DeepLab-EdgeNet-CRF~\cite{chen2015semantic} & \textbf{73.6} & +860\\
  \bottomrule \\
  \end{tabular}
  \mycaption{Semantic Segmentation using the DeepLab model}
  {IoU scores on the Pascal VOC12 segmentation test dataset
  with different models and our modified inception model.
  Also shown are the corresponding runtimes in milliseconds. Runtimes
  also include superpixel computations (300 ms with Dollar superpixels~\cite{DollarICCV13edges})}
  \label{tab:largefovresults}
\end{table}

\paragraph{Semantic Segmentation}
The experiments in this section use the Pascal VOC12 segmentation dataset~\cite{voc2012segmentation} with 21 object classes and the images have a maximum resolution of 0.25 megapixels.
For all experiments on VOC12, we train using the extended training set of
10581 images collected by~\cite{hariharan2011moredata}.
We modified the \deeplab~network architecture of~\cite{chen2014semantic} and
the CRFasRNN architecture from~\cite{zheng2015conditional} which uses a CNN with
deconvolution layers followed by DenseCRF trained end-to-end.

\paragraph{DeepLab Model}\label{sec:deeplabmodel}
We experimented with the \bi{7}{2}-\bi{8}{10} inception model.
Results using the~\deeplab~model are summarized in Tab.~\ref{tab:largefovresults}.
Although we get similar improvements with inception modules as with the
explicit kernel computation, using lattice approximation is slower.

\begin{table}
  \small
  \centering
  \begin{tabular}{p{6.4cm}>{\raggedright\arraybackslash}p{1.8cm}>{\raggedright\arraybackslash}p{1.8cm}}
    \toprule
    \textbf{Model} & \emph{IoU (Val)} & \emph{IoU (Test)}\\
    \midrule
    %%%%%%%%%%%% Scores computed by us)%%%%%%%%%%%%
    CNN &  67.5 & - \\
    \deconv (CNN+Deconvolutions) & 69.8 & 72.0 \\
    \midrule
    \bi{3}{6}-\bi{4}{6}-\bi{7}{2}-\bi{8}{6}& 71.9 & - \\
    \bi{3}{6}-\bi{4}{6}-\bi{7}{2}-\bi{8}{6}-\gi{6}& 73.6 &  \href{http://host.robots.ox.ac.uk:8080/anonymous/VOTV5E.html}{\textbf{75.2}}\\
    \midrule
    \deconvcrf (CRF-RNN)~\cite{zheng2015conditional} & 73.0 & 74.7\\
    Context-CRF-RNN~\cite{yu2015multi} & ~~ - ~ & \textbf{75.3} \\
    \bottomrule \\
  \end{tabular}
  \mycaption{Semantic Segmentation using the CRFasRNN model}{IoU score corresponding to different models
  on Pascal VOC12 reduced validation / test segmentation dataset. The reduced validation set consists of 346 images
  as used in~\cite{zheng2015conditional} where we adapted the model from.}
  \label{tab:deconvresults-app}
\end{table}

\paragraph{CRFasRNN Model}\label{sec:deepinception}
We add BI modules after score-pool3, score-pool4, \fc{7} and \fc{8} $1\times1$ convolution layers
resulting in the \bi{3}{6}-\bi{4}{6}-\bi{7}{2}-\bi{8}{6}
model and also experimented with another variant where $BI_8$ is followed by another inception
module, G$(6)$, with 6 Gaussian kernels.
Note that here also we discarded both deconvolution and DenseCRF parts of the original model~\cite{zheng2015conditional}
and inserted the BI modules in the base CNN and found similar improvements compared to the inception modules with explicit
kernel computaion. See Tab.~\ref{tab:deconvresults-app} for results on the CRFasRNN model.

\paragraph{Material Segmentation}
Table~\ref{tab:mincresults-app} shows the results on the MINC dataset~\cite{bell2015minc}
obtained by modifying the AlexNet architecture with our inception modules. We observe
similar improvements as with explicit kernel construction.
For this model, we do not provide any learned setup due to very limited segment training
data. The weights to combine outputs in the bilateral inception layer are
found by validation on the validation set.

\begin{table}[t]
  \small
  \centering
  \begin{tabular}{p{3.5cm}>{\centering\arraybackslash}p{4.0cm}}
    \toprule
    \textbf{Model} & Class / Total accuracy\\
    \midrule

    %%%%%%%%%%%% Scores computed by us)%%%%%%%%%%%%
    AlexNet CNN & 55.3 / 58.9 \\
    \midrule
    \bi{7}{2}-\bi{8}{6}& 68.5 / 71.8 \\
    \bi{7}{2}-\bi{8}{6}-G$(6)$& 67.6 / 73.1 \\
    \midrule
    AlexNet-CRF & 65.5 / 71.0 \\
    \bottomrule \\
  \end{tabular}
  \mycaption{Material Segmentation using AlexNet}{Pixel accuracy of different models on
  the MINC material segmentation test dataset~\cite{bell2015minc}.}
  \label{tab:mincresults-app}
\end{table}

\paragraph{Scales of Bilateral Inception Modules}
\label{sec:scales}

Unlike the explicit kernel technique presented in the main text (Chapter~\ref{chap:binception}),
we didn't back-propagate through feature transformation ($\Lambda$)
using the approximate bilateral filter technique.
So, the feature scales are hand-specified and validated, which are as follows.
The optimal scale values for the \bi{7}{2}-\bi{8}{2} model are found by validation for the best performance which are
$\sigma_{xy}$ = (0.1, 0.1) for the spatial (XY) kernel and $\sigma_{rgbxy}$ = (0.1, 0.1, 0.1, 0.01, 0.01) for color and position (RGBXY)  kernel.
Next, as more kernels are added to \bi{8}{2}, we set scales to be $\alpha$*($\sigma_{xy}$, $\sigma_{rgbxy}$).
The value of $\alpha$ is chosen as  1, 0.5, 0.1, 0.05, 0.1, at uniform interval, for the \bi{8}{10} bilateral inception module.


\subsection{Qualitative Results}
\label{sec:qualitative-app}

In this section, we present more qualitative results obtained using the BI module with explicit
kernel computation technique presented in Chapter~\ref{chap:binception}. Results on the Pascal VOC12
dataset~\cite{voc2012segmentation} using the DeepLab-LargeFOV model are shown in Fig.~\ref{fig:semantic_visuals-app},
followed by the results on MINC dataset~\cite{bell2015minc}
in Fig.~\ref{fig:material_visuals-app} and on
Cityscapes dataset~\cite{Cordts2015Cvprw} in Fig.~\ref{fig:street_visuals-app}.


\definecolor{voc_1}{RGB}{0, 0, 0}
\definecolor{voc_2}{RGB}{128, 0, 0}
\definecolor{voc_3}{RGB}{0, 128, 0}
\definecolor{voc_4}{RGB}{128, 128, 0}
\definecolor{voc_5}{RGB}{0, 0, 128}
\definecolor{voc_6}{RGB}{128, 0, 128}
\definecolor{voc_7}{RGB}{0, 128, 128}
\definecolor{voc_8}{RGB}{128, 128, 128}
\definecolor{voc_9}{RGB}{64, 0, 0}
\definecolor{voc_10}{RGB}{192, 0, 0}
\definecolor{voc_11}{RGB}{64, 128, 0}
\definecolor{voc_12}{RGB}{192, 128, 0}
\definecolor{voc_13}{RGB}{64, 0, 128}
\definecolor{voc_14}{RGB}{192, 0, 128}
\definecolor{voc_15}{RGB}{64, 128, 128}
\definecolor{voc_16}{RGB}{192, 128, 128}
\definecolor{voc_17}{RGB}{0, 64, 0}
\definecolor{voc_18}{RGB}{128, 64, 0}
\definecolor{voc_19}{RGB}{0, 192, 0}
\definecolor{voc_20}{RGB}{128, 192, 0}
\definecolor{voc_21}{RGB}{0, 64, 128}
\definecolor{voc_22}{RGB}{128, 64, 128}

\begin{figure*}[!ht]
  \small
  \centering
  \fcolorbox{white}{voc_1}{\rule{0pt}{4pt}\rule{4pt}{0pt}} Background~~
  \fcolorbox{white}{voc_2}{\rule{0pt}{4pt}\rule{4pt}{0pt}} Aeroplane~~
  \fcolorbox{white}{voc_3}{\rule{0pt}{4pt}\rule{4pt}{0pt}} Bicycle~~
  \fcolorbox{white}{voc_4}{\rule{0pt}{4pt}\rule{4pt}{0pt}} Bird~~
  \fcolorbox{white}{voc_5}{\rule{0pt}{4pt}\rule{4pt}{0pt}} Boat~~
  \fcolorbox{white}{voc_6}{\rule{0pt}{4pt}\rule{4pt}{0pt}} Bottle~~
  \fcolorbox{white}{voc_7}{\rule{0pt}{4pt}\rule{4pt}{0pt}} Bus~~
  \fcolorbox{white}{voc_8}{\rule{0pt}{4pt}\rule{4pt}{0pt}} Car~~\\
  \fcolorbox{white}{voc_9}{\rule{0pt}{4pt}\rule{4pt}{0pt}} Cat~~
  \fcolorbox{white}{voc_10}{\rule{0pt}{4pt}\rule{4pt}{0pt}} Chair~~
  \fcolorbox{white}{voc_11}{\rule{0pt}{4pt}\rule{4pt}{0pt}} Cow~~
  \fcolorbox{white}{voc_12}{\rule{0pt}{4pt}\rule{4pt}{0pt}} Dining Table~~
  \fcolorbox{white}{voc_13}{\rule{0pt}{4pt}\rule{4pt}{0pt}} Dog~~
  \fcolorbox{white}{voc_14}{\rule{0pt}{4pt}\rule{4pt}{0pt}} Horse~~
  \fcolorbox{white}{voc_15}{\rule{0pt}{4pt}\rule{4pt}{0pt}} Motorbike~~
  \fcolorbox{white}{voc_16}{\rule{0pt}{4pt}\rule{4pt}{0pt}} Person~~\\
  \fcolorbox{white}{voc_17}{\rule{0pt}{4pt}\rule{4pt}{0pt}} Potted Plant~~
  \fcolorbox{white}{voc_18}{\rule{0pt}{4pt}\rule{4pt}{0pt}} Sheep~~
  \fcolorbox{white}{voc_19}{\rule{0pt}{4pt}\rule{4pt}{0pt}} Sofa~~
  \fcolorbox{white}{voc_20}{\rule{0pt}{4pt}\rule{4pt}{0pt}} Train~~
  \fcolorbox{white}{voc_21}{\rule{0pt}{4pt}\rule{4pt}{0pt}} TV monitor~~\\


  \subfigure{%
    \includegraphics[width=.15\columnwidth]{figures/supplementary/2008_001308_given.png}
  }
  \subfigure{%
    \includegraphics[width=.15\columnwidth]{figures/supplementary/2008_001308_sp.png}
  }
  \subfigure{%
    \includegraphics[width=.15\columnwidth]{figures/supplementary/2008_001308_gt.png}
  }
  \subfigure{%
    \includegraphics[width=.15\columnwidth]{figures/supplementary/2008_001308_cnn.png}
  }
  \subfigure{%
    \includegraphics[width=.15\columnwidth]{figures/supplementary/2008_001308_crf.png}
  }
  \subfigure{%
    \includegraphics[width=.15\columnwidth]{figures/supplementary/2008_001308_ours.png}
  }\\[-2ex]


  \subfigure{%
    \includegraphics[width=.15\columnwidth]{figures/supplementary/2008_001821_given.png}
  }
  \subfigure{%
    \includegraphics[width=.15\columnwidth]{figures/supplementary/2008_001821_sp.png}
  }
  \subfigure{%
    \includegraphics[width=.15\columnwidth]{figures/supplementary/2008_001821_gt.png}
  }
  \subfigure{%
    \includegraphics[width=.15\columnwidth]{figures/supplementary/2008_001821_cnn.png}
  }
  \subfigure{%
    \includegraphics[width=.15\columnwidth]{figures/supplementary/2008_001821_crf.png}
  }
  \subfigure{%
    \includegraphics[width=.15\columnwidth]{figures/supplementary/2008_001821_ours.png}
  }\\[-2ex]



  \subfigure{%
    \includegraphics[width=.15\columnwidth]{figures/supplementary/2008_004612_given.png}
  }
  \subfigure{%
    \includegraphics[width=.15\columnwidth]{figures/supplementary/2008_004612_sp.png}
  }
  \subfigure{%
    \includegraphics[width=.15\columnwidth]{figures/supplementary/2008_004612_gt.png}
  }
  \subfigure{%
    \includegraphics[width=.15\columnwidth]{figures/supplementary/2008_004612_cnn.png}
  }
  \subfigure{%
    \includegraphics[width=.15\columnwidth]{figures/supplementary/2008_004612_crf.png}
  }
  \subfigure{%
    \includegraphics[width=.15\columnwidth]{figures/supplementary/2008_004612_ours.png}
  }\\[-2ex]


  \subfigure{%
    \includegraphics[width=.15\columnwidth]{figures/supplementary/2009_001008_given.png}
  }
  \subfigure{%
    \includegraphics[width=.15\columnwidth]{figures/supplementary/2009_001008_sp.png}
  }
  \subfigure{%
    \includegraphics[width=.15\columnwidth]{figures/supplementary/2009_001008_gt.png}
  }
  \subfigure{%
    \includegraphics[width=.15\columnwidth]{figures/supplementary/2009_001008_cnn.png}
  }
  \subfigure{%
    \includegraphics[width=.15\columnwidth]{figures/supplementary/2009_001008_crf.png}
  }
  \subfigure{%
    \includegraphics[width=.15\columnwidth]{figures/supplementary/2009_001008_ours.png}
  }\\[-2ex]




  \subfigure{%
    \includegraphics[width=.15\columnwidth]{figures/supplementary/2009_004497_given.png}
  }
  \subfigure{%
    \includegraphics[width=.15\columnwidth]{figures/supplementary/2009_004497_sp.png}
  }
  \subfigure{%
    \includegraphics[width=.15\columnwidth]{figures/supplementary/2009_004497_gt.png}
  }
  \subfigure{%
    \includegraphics[width=.15\columnwidth]{figures/supplementary/2009_004497_cnn.png}
  }
  \subfigure{%
    \includegraphics[width=.15\columnwidth]{figures/supplementary/2009_004497_crf.png}
  }
  \subfigure{%
    \includegraphics[width=.15\columnwidth]{figures/supplementary/2009_004497_ours.png}
  }\\[-2ex]



  \setcounter{subfigure}{0}
  \subfigure[\scriptsize Input]{%
    \includegraphics[width=.15\columnwidth]{figures/supplementary/2010_001327_given.png}
  }
  \subfigure[\scriptsize Superpixels]{%
    \includegraphics[width=.15\columnwidth]{figures/supplementary/2010_001327_sp.png}
  }
  \subfigure[\scriptsize GT]{%
    \includegraphics[width=.15\columnwidth]{figures/supplementary/2010_001327_gt.png}
  }
  \subfigure[\scriptsize Deeplab]{%
    \includegraphics[width=.15\columnwidth]{figures/supplementary/2010_001327_cnn.png}
  }
  \subfigure[\scriptsize +DenseCRF]{%
    \includegraphics[width=.15\columnwidth]{figures/supplementary/2010_001327_crf.png}
  }
  \subfigure[\scriptsize Using BI]{%
    \includegraphics[width=.15\columnwidth]{figures/supplementary/2010_001327_ours.png}
  }
  \mycaption{Semantic Segmentation}{Example results of semantic segmentation
  on the Pascal VOC12 dataset.
  (d)~depicts the DeepLab CNN result, (e)~CNN + 10 steps of mean-field inference,
  (f~result obtained with bilateral inception (BI) modules (\bi{6}{2}+\bi{7}{6}) between \fc~layers.}
  \label{fig:semantic_visuals-app}
\end{figure*}


\definecolor{minc_1}{HTML}{771111}
\definecolor{minc_2}{HTML}{CAC690}
\definecolor{minc_3}{HTML}{EEEEEE}
\definecolor{minc_4}{HTML}{7C8FA6}
\definecolor{minc_5}{HTML}{597D31}
\definecolor{minc_6}{HTML}{104410}
\definecolor{minc_7}{HTML}{BB819C}
\definecolor{minc_8}{HTML}{D0CE48}
\definecolor{minc_9}{HTML}{622745}
\definecolor{minc_10}{HTML}{666666}
\definecolor{minc_11}{HTML}{D54A31}
\definecolor{minc_12}{HTML}{101044}
\definecolor{minc_13}{HTML}{444126}
\definecolor{minc_14}{HTML}{75D646}
\definecolor{minc_15}{HTML}{DD4348}
\definecolor{minc_16}{HTML}{5C8577}
\definecolor{minc_17}{HTML}{C78472}
\definecolor{minc_18}{HTML}{75D6D0}
\definecolor{minc_19}{HTML}{5B4586}
\definecolor{minc_20}{HTML}{C04393}
\definecolor{minc_21}{HTML}{D69948}
\definecolor{minc_22}{HTML}{7370D8}
\definecolor{minc_23}{HTML}{7A3622}
\definecolor{minc_24}{HTML}{000000}

\begin{figure*}[!ht]
  \small % scriptsize
  \centering
  \fcolorbox{white}{minc_1}{\rule{0pt}{4pt}\rule{4pt}{0pt}} Brick~~
  \fcolorbox{white}{minc_2}{\rule{0pt}{4pt}\rule{4pt}{0pt}} Carpet~~
  \fcolorbox{white}{minc_3}{\rule{0pt}{4pt}\rule{4pt}{0pt}} Ceramic~~
  \fcolorbox{white}{minc_4}{\rule{0pt}{4pt}\rule{4pt}{0pt}} Fabric~~
  \fcolorbox{white}{minc_5}{\rule{0pt}{4pt}\rule{4pt}{0pt}} Foliage~~
  \fcolorbox{white}{minc_6}{\rule{0pt}{4pt}\rule{4pt}{0pt}} Food~~
  \fcolorbox{white}{minc_7}{\rule{0pt}{4pt}\rule{4pt}{0pt}} Glass~~
  \fcolorbox{white}{minc_8}{\rule{0pt}{4pt}\rule{4pt}{0pt}} Hair~~\\
  \fcolorbox{white}{minc_9}{\rule{0pt}{4pt}\rule{4pt}{0pt}} Leather~~
  \fcolorbox{white}{minc_10}{\rule{0pt}{4pt}\rule{4pt}{0pt}} Metal~~
  \fcolorbox{white}{minc_11}{\rule{0pt}{4pt}\rule{4pt}{0pt}} Mirror~~
  \fcolorbox{white}{minc_12}{\rule{0pt}{4pt}\rule{4pt}{0pt}} Other~~
  \fcolorbox{white}{minc_13}{\rule{0pt}{4pt}\rule{4pt}{0pt}} Painted~~
  \fcolorbox{white}{minc_14}{\rule{0pt}{4pt}\rule{4pt}{0pt}} Paper~~
  \fcolorbox{white}{minc_15}{\rule{0pt}{4pt}\rule{4pt}{0pt}} Plastic~~\\
  \fcolorbox{white}{minc_16}{\rule{0pt}{4pt}\rule{4pt}{0pt}} Polished Stone~~
  \fcolorbox{white}{minc_17}{\rule{0pt}{4pt}\rule{4pt}{0pt}} Skin~~
  \fcolorbox{white}{minc_18}{\rule{0pt}{4pt}\rule{4pt}{0pt}} Sky~~
  \fcolorbox{white}{minc_19}{\rule{0pt}{4pt}\rule{4pt}{0pt}} Stone~~
  \fcolorbox{white}{minc_20}{\rule{0pt}{4pt}\rule{4pt}{0pt}} Tile~~
  \fcolorbox{white}{minc_21}{\rule{0pt}{4pt}\rule{4pt}{0pt}} Wallpaper~~
  \fcolorbox{white}{minc_22}{\rule{0pt}{4pt}\rule{4pt}{0pt}} Water~~
  \fcolorbox{white}{minc_23}{\rule{0pt}{4pt}\rule{4pt}{0pt}} Wood~~\\
  \subfigure{%
    \includegraphics[width=.15\columnwidth]{figures/supplementary/000008468_given.png}
  }
  \subfigure{%
    \includegraphics[width=.15\columnwidth]{figures/supplementary/000008468_sp.png}
  }
  \subfigure{%
    \includegraphics[width=.15\columnwidth]{figures/supplementary/000008468_gt.png}
  }
  \subfigure{%
    \includegraphics[width=.15\columnwidth]{figures/supplementary/000008468_cnn.png}
  }
  \subfigure{%
    \includegraphics[width=.15\columnwidth]{figures/supplementary/000008468_crf.png}
  }
  \subfigure{%
    \includegraphics[width=.15\columnwidth]{figures/supplementary/000008468_ours.png}
  }\\[-2ex]

  \subfigure{%
    \includegraphics[width=.15\columnwidth]{figures/supplementary/000009053_given.png}
  }
  \subfigure{%
    \includegraphics[width=.15\columnwidth]{figures/supplementary/000009053_sp.png}
  }
  \subfigure{%
    \includegraphics[width=.15\columnwidth]{figures/supplementary/000009053_gt.png}
  }
  \subfigure{%
    \includegraphics[width=.15\columnwidth]{figures/supplementary/000009053_cnn.png}
  }
  \subfigure{%
    \includegraphics[width=.15\columnwidth]{figures/supplementary/000009053_crf.png}
  }
  \subfigure{%
    \includegraphics[width=.15\columnwidth]{figures/supplementary/000009053_ours.png}
  }\\[-2ex]




  \subfigure{%
    \includegraphics[width=.15\columnwidth]{figures/supplementary/000014977_given.png}
  }
  \subfigure{%
    \includegraphics[width=.15\columnwidth]{figures/supplementary/000014977_sp.png}
  }
  \subfigure{%
    \includegraphics[width=.15\columnwidth]{figures/supplementary/000014977_gt.png}
  }
  \subfigure{%
    \includegraphics[width=.15\columnwidth]{figures/supplementary/000014977_cnn.png}
  }
  \subfigure{%
    \includegraphics[width=.15\columnwidth]{figures/supplementary/000014977_crf.png}
  }
  \subfigure{%
    \includegraphics[width=.15\columnwidth]{figures/supplementary/000014977_ours.png}
  }\\[-2ex]


  \subfigure{%
    \includegraphics[width=.15\columnwidth]{figures/supplementary/000022922_given.png}
  }
  \subfigure{%
    \includegraphics[width=.15\columnwidth]{figures/supplementary/000022922_sp.png}
  }
  \subfigure{%
    \includegraphics[width=.15\columnwidth]{figures/supplementary/000022922_gt.png}
  }
  \subfigure{%
    \includegraphics[width=.15\columnwidth]{figures/supplementary/000022922_cnn.png}
  }
  \subfigure{%
    \includegraphics[width=.15\columnwidth]{figures/supplementary/000022922_crf.png}
  }
  \subfigure{%
    \includegraphics[width=.15\columnwidth]{figures/supplementary/000022922_ours.png}
  }\\[-2ex]


  \subfigure{%
    \includegraphics[width=.15\columnwidth]{figures/supplementary/000025711_given.png}
  }
  \subfigure{%
    \includegraphics[width=.15\columnwidth]{figures/supplementary/000025711_sp.png}
  }
  \subfigure{%
    \includegraphics[width=.15\columnwidth]{figures/supplementary/000025711_gt.png}
  }
  \subfigure{%
    \includegraphics[width=.15\columnwidth]{figures/supplementary/000025711_cnn.png}
  }
  \subfigure{%
    \includegraphics[width=.15\columnwidth]{figures/supplementary/000025711_crf.png}
  }
  \subfigure{%
    \includegraphics[width=.15\columnwidth]{figures/supplementary/000025711_ours.png}
  }\\[-2ex]


  \subfigure{%
    \includegraphics[width=.15\columnwidth]{figures/supplementary/000034473_given.png}
  }
  \subfigure{%
    \includegraphics[width=.15\columnwidth]{figures/supplementary/000034473_sp.png}
  }
  \subfigure{%
    \includegraphics[width=.15\columnwidth]{figures/supplementary/000034473_gt.png}
  }
  \subfigure{%
    \includegraphics[width=.15\columnwidth]{figures/supplementary/000034473_cnn.png}
  }
  \subfigure{%
    \includegraphics[width=.15\columnwidth]{figures/supplementary/000034473_crf.png}
  }
  \subfigure{%
    \includegraphics[width=.15\columnwidth]{figures/supplementary/000034473_ours.png}
  }\\[-2ex]


  \subfigure{%
    \includegraphics[width=.15\columnwidth]{figures/supplementary/000035463_given.png}
  }
  \subfigure{%
    \includegraphics[width=.15\columnwidth]{figures/supplementary/000035463_sp.png}
  }
  \subfigure{%
    \includegraphics[width=.15\columnwidth]{figures/supplementary/000035463_gt.png}
  }
  \subfigure{%
    \includegraphics[width=.15\columnwidth]{figures/supplementary/000035463_cnn.png}
  }
  \subfigure{%
    \includegraphics[width=.15\columnwidth]{figures/supplementary/000035463_crf.png}
  }
  \subfigure{%
    \includegraphics[width=.15\columnwidth]{figures/supplementary/000035463_ours.png}
  }\\[-2ex]


  \setcounter{subfigure}{0}
  \subfigure[\scriptsize Input]{%
    \includegraphics[width=.15\columnwidth]{figures/supplementary/000035993_given.png}
  }
  \subfigure[\scriptsize Superpixels]{%
    \includegraphics[width=.15\columnwidth]{figures/supplementary/000035993_sp.png}
  }
  \subfigure[\scriptsize GT]{%
    \includegraphics[width=.15\columnwidth]{figures/supplementary/000035993_gt.png}
  }
  \subfigure[\scriptsize AlexNet]{%
    \includegraphics[width=.15\columnwidth]{figures/supplementary/000035993_cnn.png}
  }
  \subfigure[\scriptsize +DenseCRF]{%
    \includegraphics[width=.15\columnwidth]{figures/supplementary/000035993_crf.png}
  }
  \subfigure[\scriptsize Using BI]{%
    \includegraphics[width=.15\columnwidth]{figures/supplementary/000035993_ours.png}
  }
  \mycaption{Material Segmentation}{Example results of material segmentation.
  (d)~depicts the AlexNet CNN result, (e)~CNN + 10 steps of mean-field inference,
  (f)~result obtained with bilateral inception (BI) modules (\bi{7}{2}+\bi{8}{6}) between
  \fc~layers.}
\label{fig:material_visuals-app}
\end{figure*}


\definecolor{city_1}{RGB}{128, 64, 128}
\definecolor{city_2}{RGB}{244, 35, 232}
\definecolor{city_3}{RGB}{70, 70, 70}
\definecolor{city_4}{RGB}{102, 102, 156}
\definecolor{city_5}{RGB}{190, 153, 153}
\definecolor{city_6}{RGB}{153, 153, 153}
\definecolor{city_7}{RGB}{250, 170, 30}
\definecolor{city_8}{RGB}{220, 220, 0}
\definecolor{city_9}{RGB}{107, 142, 35}
\definecolor{city_10}{RGB}{152, 251, 152}
\definecolor{city_11}{RGB}{70, 130, 180}
\definecolor{city_12}{RGB}{220, 20, 60}
\definecolor{city_13}{RGB}{255, 0, 0}
\definecolor{city_14}{RGB}{0, 0, 142}
\definecolor{city_15}{RGB}{0, 0, 70}
\definecolor{city_16}{RGB}{0, 60, 100}
\definecolor{city_17}{RGB}{0, 80, 100}
\definecolor{city_18}{RGB}{0, 0, 230}
\definecolor{city_19}{RGB}{119, 11, 32}
\begin{figure*}[!ht]
  \small % scriptsize
  \centering


  \subfigure{%
    \includegraphics[width=.18\columnwidth]{figures/supplementary/frankfurt00000_016005_given.png}
  }
  \subfigure{%
    \includegraphics[width=.18\columnwidth]{figures/supplementary/frankfurt00000_016005_sp.png}
  }
  \subfigure{%
    \includegraphics[width=.18\columnwidth]{figures/supplementary/frankfurt00000_016005_gt.png}
  }
  \subfigure{%
    \includegraphics[width=.18\columnwidth]{figures/supplementary/frankfurt00000_016005_cnn.png}
  }
  \subfigure{%
    \includegraphics[width=.18\columnwidth]{figures/supplementary/frankfurt00000_016005_ours.png}
  }\\[-2ex]

  \subfigure{%
    \includegraphics[width=.18\columnwidth]{figures/supplementary/frankfurt00000_004617_given.png}
  }
  \subfigure{%
    \includegraphics[width=.18\columnwidth]{figures/supplementary/frankfurt00000_004617_sp.png}
  }
  \subfigure{%
    \includegraphics[width=.18\columnwidth]{figures/supplementary/frankfurt00000_004617_gt.png}
  }
  \subfigure{%
    \includegraphics[width=.18\columnwidth]{figures/supplementary/frankfurt00000_004617_cnn.png}
  }
  \subfigure{%
    \includegraphics[width=.18\columnwidth]{figures/supplementary/frankfurt00000_004617_ours.png}
  }\\[-2ex]

  \subfigure{%
    \includegraphics[width=.18\columnwidth]{figures/supplementary/frankfurt00000_020880_given.png}
  }
  \subfigure{%
    \includegraphics[width=.18\columnwidth]{figures/supplementary/frankfurt00000_020880_sp.png}
  }
  \subfigure{%
    \includegraphics[width=.18\columnwidth]{figures/supplementary/frankfurt00000_020880_gt.png}
  }
  \subfigure{%
    \includegraphics[width=.18\columnwidth]{figures/supplementary/frankfurt00000_020880_cnn.png}
  }
  \subfigure{%
    \includegraphics[width=.18\columnwidth]{figures/supplementary/frankfurt00000_020880_ours.png}
  }\\[-2ex]



  \subfigure{%
    \includegraphics[width=.18\columnwidth]{figures/supplementary/frankfurt00001_007285_given.png}
  }
  \subfigure{%
    \includegraphics[width=.18\columnwidth]{figures/supplementary/frankfurt00001_007285_sp.png}
  }
  \subfigure{%
    \includegraphics[width=.18\columnwidth]{figures/supplementary/frankfurt00001_007285_gt.png}
  }
  \subfigure{%
    \includegraphics[width=.18\columnwidth]{figures/supplementary/frankfurt00001_007285_cnn.png}
  }
  \subfigure{%
    \includegraphics[width=.18\columnwidth]{figures/supplementary/frankfurt00001_007285_ours.png}
  }\\[-2ex]


  \subfigure{%
    \includegraphics[width=.18\columnwidth]{figures/supplementary/frankfurt00001_059789_given.png}
  }
  \subfigure{%
    \includegraphics[width=.18\columnwidth]{figures/supplementary/frankfurt00001_059789_sp.png}
  }
  \subfigure{%
    \includegraphics[width=.18\columnwidth]{figures/supplementary/frankfurt00001_059789_gt.png}
  }
  \subfigure{%
    \includegraphics[width=.18\columnwidth]{figures/supplementary/frankfurt00001_059789_cnn.png}
  }
  \subfigure{%
    \includegraphics[width=.18\columnwidth]{figures/supplementary/frankfurt00001_059789_ours.png}
  }\\[-2ex]


  \subfigure{%
    \includegraphics[width=.18\columnwidth]{figures/supplementary/frankfurt00001_068208_given.png}
  }
  \subfigure{%
    \includegraphics[width=.18\columnwidth]{figures/supplementary/frankfurt00001_068208_sp.png}
  }
  \subfigure{%
    \includegraphics[width=.18\columnwidth]{figures/supplementary/frankfurt00001_068208_gt.png}
  }
  \subfigure{%
    \includegraphics[width=.18\columnwidth]{figures/supplementary/frankfurt00001_068208_cnn.png}
  }
  \subfigure{%
    \includegraphics[width=.18\columnwidth]{figures/supplementary/frankfurt00001_068208_ours.png}
  }\\[-2ex]

  \subfigure{%
    \includegraphics[width=.18\columnwidth]{figures/supplementary/frankfurt00001_082466_given.png}
  }
  \subfigure{%
    \includegraphics[width=.18\columnwidth]{figures/supplementary/frankfurt00001_082466_sp.png}
  }
  \subfigure{%
    \includegraphics[width=.18\columnwidth]{figures/supplementary/frankfurt00001_082466_gt.png}
  }
  \subfigure{%
    \includegraphics[width=.18\columnwidth]{figures/supplementary/frankfurt00001_082466_cnn.png}
  }
  \subfigure{%
    \includegraphics[width=.18\columnwidth]{figures/supplementary/frankfurt00001_082466_ours.png}
  }\\[-2ex]

  \subfigure{%
    \includegraphics[width=.18\columnwidth]{figures/supplementary/lindau00033_000019_given.png}
  }
  \subfigure{%
    \includegraphics[width=.18\columnwidth]{figures/supplementary/lindau00033_000019_sp.png}
  }
  \subfigure{%
    \includegraphics[width=.18\columnwidth]{figures/supplementary/lindau00033_000019_gt.png}
  }
  \subfigure{%
    \includegraphics[width=.18\columnwidth]{figures/supplementary/lindau00033_000019_cnn.png}
  }
  \subfigure{%
    \includegraphics[width=.18\columnwidth]{figures/supplementary/lindau00033_000019_ours.png}
  }\\[-2ex]

  \subfigure{%
    \includegraphics[width=.18\columnwidth]{figures/supplementary/lindau00052_000019_given.png}
  }
  \subfigure{%
    \includegraphics[width=.18\columnwidth]{figures/supplementary/lindau00052_000019_sp.png}
  }
  \subfigure{%
    \includegraphics[width=.18\columnwidth]{figures/supplementary/lindau00052_000019_gt.png}
  }
  \subfigure{%
    \includegraphics[width=.18\columnwidth]{figures/supplementary/lindau00052_000019_cnn.png}
  }
  \subfigure{%
    \includegraphics[width=.18\columnwidth]{figures/supplementary/lindau00052_000019_ours.png}
  }\\[-2ex]




  \subfigure{%
    \includegraphics[width=.18\columnwidth]{figures/supplementary/lindau00027_000019_given.png}
  }
  \subfigure{%
    \includegraphics[width=.18\columnwidth]{figures/supplementary/lindau00027_000019_sp.png}
  }
  \subfigure{%
    \includegraphics[width=.18\columnwidth]{figures/supplementary/lindau00027_000019_gt.png}
  }
  \subfigure{%
    \includegraphics[width=.18\columnwidth]{figures/supplementary/lindau00027_000019_cnn.png}
  }
  \subfigure{%
    \includegraphics[width=.18\columnwidth]{figures/supplementary/lindau00027_000019_ours.png}
  }\\[-2ex]



  \setcounter{subfigure}{0}
  \subfigure[\scriptsize Input]{%
    \includegraphics[width=.18\columnwidth]{figures/supplementary/lindau00029_000019_given.png}
  }
  \subfigure[\scriptsize Superpixels]{%
    \includegraphics[width=.18\columnwidth]{figures/supplementary/lindau00029_000019_sp.png}
  }
  \subfigure[\scriptsize GT]{%
    \includegraphics[width=.18\columnwidth]{figures/supplementary/lindau00029_000019_gt.png}
  }
  \subfigure[\scriptsize Deeplab]{%
    \includegraphics[width=.18\columnwidth]{figures/supplementary/lindau00029_000019_cnn.png}
  }
  \subfigure[\scriptsize Using BI]{%
    \includegraphics[width=.18\columnwidth]{figures/supplementary/lindau00029_000019_ours.png}
  }%\\[-2ex]

  \mycaption{Street Scene Segmentation}{Example results of street scene segmentation.
  (d)~depicts the DeepLab results, (e)~result obtained by adding bilateral inception (BI) modules (\bi{6}{2}+\bi{7}{6}) between \fc~layers.}
\label{fig:street_visuals-app}
\end{figure*}

\end{document}