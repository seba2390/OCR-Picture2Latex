
\section{Introduction}
\label{sec:intro}

\begin{figure*}[t]
	\centering
	\includegraphics[width=14cm]{"figures/system".pdf}
	\caption{Diagram illustrating the FPHTC framework.}
	\label{fig:scheme}
\end{figure*}

Traffic classification is critical to the operation of computer networks in many aspects, such as network management, quality-of-service guarantee, and security concerns. As a large segment of network traffic is encrypted in recent years, traditional methods of classification, e.g., the protocol-based approach and content comparison, are no longer appropriate \cite{Finsterbusch14}. In contrast, machine learning algorithms can identify traffic flows with high accuracy using statistical features \cite{Moore05,Auld07,Duffield04,Wang16, Chowdhury19}.

For class-aware routing in a network, the routers need to conduct traffic classification before forwarding the traffic. However, common machine learning algorithms are often too computationally expensive for the routers. For example, even though deep neural networks is known to provide accurate classification outcomes, they are computationally intensive to train, while they need to be updated frequently to adapt to the changing traffic pattern over time. More importantly, the statistical features required for flow-based traffic classification techniques often are not available for real-time classification. In general, statistical features such as the \textit{variance of packet length} are extracted at the end of each flow, after the lengths of all packets are observed. Therefore, they are not useful for a router that must route the packets of a flow as they arrive, with as little delay as possible. Some authors have proposed early recognition of traffic classes by observing a subset of packets from each flow \cite{Bernaille07,Nguyen12, Xie12}. However, this still might cause significant delay as a router generally needs to process millions of packets within a fraction of a second.

A naive alternative is pure packet-based traffic classification, where each packet is observed and classified immediately, i.e., the classifier does not wait for a stream of packets from a flow. The router can look at the simple features embedded in the packet header and make a quick decision per packet. Packet-based traffic classification requires only fast lookup of the packet headers and thus is amenable to practical implementation in high-speed routers. However, a key drawback of this approach is that the classification performance can be poor due to the absence of detailed statistical features that are available in flow-based classification. 

This motivates us to combine the advantages of both flow-based and packet-based traffic classification. We propose a novel Flow-Packet Hybrid Traffic Classification (FPHTC) method, where a low-complexity routing policy with packet classification at the router is designed with the assistance of a flow-based classifier that resides outside the router. To the best of our knowledge, there exists no prior work that considers this hybrid form between flow-based and packet-based methods for network traffic classification in the router.

Our contributions can be summarized as follows:
\begin{itemize}
	\item We propose FPHTC, which generates a low-complexity routing policy to be applied to the incoming packets at a router. The routing policy enables class-aware routing using only simple features, e.g., those that can be directly read from the packet header. We generate the routing policy by exploiting the knowledge learned by a highly accurate flow-based classifier residing outside the router. The routing policy is constructed as a decision tree trained using the packets labeled by the flow-based classifier. In FPHTC, we can employ the flow-based classifier to label any number of packets, and thus, the resulting routing policy can be highly accurate.
	\item We show that FPHTC can be deployed in an online learning setting, where a new routing policy is updated at the router whenever the performance of the current routing policy falls below a certain threshold due to changes in the traffic pattern. This is achieved by adaptively re-training the flow-based classifier and then the routing policy, upon receiving the feedback that routing policy update is required.
	\item We provide theoretical justification for the performance advantage of FPHTC over regular packet-based traffic classification, in terms of the generalization bound. This further enables exploring the trade-off between the cost of labeling data for training the flow-based classifier and the generalization bound of the routing policy.
	\item We conduct extensive experiments using an aggregate dataset of 43590 encrypted traffic flows from \cite{Vpn16} and \cite{Tor17}. We train gradient boosted tree models XGBoost \cite{Xgb16} and LightGBM \cite{Lgbm17} as the flow-based classifier and compare the performance of FPHTC with regular packet-based traffic classification for different training dataset sizes. We observe substantial  performance gain under FPHTC.
\end{itemize}

The rest of this paper is structured as follows. The concept of FPHTC is presented in Section \ref{sec:proposed}, where we describe different components of FPHTC, routing policy design, and update procedure in detail. Section \ref{sec:policy} provides an analytical comparison between FPHTC and regular packet-based traffic classification in terms of the generalization bound. In Section \ref{sec:results}, we present our experimental setup and classification performance of FPHTC. Section \ref{sec:conclusion} concludes the paper.