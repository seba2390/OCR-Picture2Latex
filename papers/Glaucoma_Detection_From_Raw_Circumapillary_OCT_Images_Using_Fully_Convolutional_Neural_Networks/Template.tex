% Template for ICIP-2018 paper; to be used with:
%          spconf.sty  - ICASSP/ICIP LaTeX style file, and
%          IEEEbib.bst - IEEE bibliography style file.
% --------------------------------------------------------------------------
\documentclass{article}
\usepackage{amsmath,graphicx}
\usepackage{cite}
\usepackage{algorithmic}
\usepackage[ruled,vlined,commentsnumbered]{algorithm2e}
\usepackage{balance}
\usepackage{subfigure}
\usepackage[preprint]{spconf}
% Example definitions.
% --------------------
%\def\x{{\mathbf x}}
%\def\L{{\cal L}}


\copyrightnotice{\begin{tabular}[t]{@{}l@{}}© 2020 IEEE.  Personal use of this material is permitted. Permission from IEEE must be obtained for all other uses, in any current or\\future media, including reprinting/republishing this material for advertising or promotional purposes, creating new collective works, for\\resale or redistribution to servers or lists, or reuse of any copyrighted component of this work in other works.\end{tabular}}


% Title.
% ------
\title{GLAUCOMA DETECTION FROM RAW CIRCUMPAPILLARY OCT IMAGES USING FULLY CONVOLUTIONAL NEURAL NETWORKS}
%
% Single address.
% ---------------
%\name{Author(s) Name(s)\thanks{Thanks to XYZ agency for funding.}}
%\address{Author Affiliation(s)}

\name{Gabriel Garc\'{i}a$^{1}$, Roc\'{i}o del Amor$^{1}$, Adri\'{a}n Colomer$^{1}$, Valery Naranjo$^{1}$ \thanks{This work has been funded by GALAHAD project [H2020-ICT-2016-2017, 732613], SICAP project (DPI2016-77869-C2-1-R) and GVA through project PROMETEO/2019/109. The work of Gabriel Garc\'{i}a has been supported by the State Research Spanish Agency PTA2017-14610-I. We thank NVIDIA Corporation for the donation of the Titan V GPU used here.}}
\address{$^{1}$Instituto de Investigaci\'on e Innovaci\'on en Bioingenier\'ia (I3B),\\ Universitat Polit\`ecnica de Val\`encia, Camino de Vera s/n, 46022, Valencia, Spain.}

%
% For example:
% ------------
%\address{School\\
%	Department\\
%	Address}
%
% Two addresses (uncomment and modify for two-address case).
% ----------------------------------------------------------
%\twoauthors
%  {A. Author-one, B. Author-two\sthanks{Thanks to XYZ agency for funding.}}
%	{School A-B\\
%	Department A-B\\
%	Address A-B}
%  {C. Author-three, D. Author-four\sthanks{The fourth author performed the work
%	while at ...}}
%	{School C-D\\
%	Department C-D\\
%	Address C-D}
%
\begin{document}
%\ninept
%
\maketitle
%
\begin{abstract}
Nowadays, glaucoma is the leading cause of blindness worldwide. We propose in this paper two different deep-learning-based approaches to address glaucoma detection just from raw circumpapillary OCT images. The first one is based on the development of convolutional neural networks (CNNs) trained from scratch. The second one lies in fine-tuning some of the most common state-of-the-art CNNs architectures. The experiments were performed on a private database composed of 93 glaucomatous and 156 normal B-scans around the optic nerve head of the retina, which were diagnosed by expert ophthalmologists. The validation results evidence that fine-tuned CNNs outperform the networks trained from scratch when small databases are addressed. Additionally, the VGG family of networks reports the most promising results, with an area under the ROC curve of 0.96 and an accuracy of 0.92, during the prediction of the independent test set.
\end{abstract}
%
\begin{keywords}
Glaucoma detection, deep learning, circumpapillary OCT, fine tuning, class activation maps.

\end{keywords}
%

\section{Introduction}\label{sec:introduction}

%\textit{Expand on the purpose of the paper and contextualise. Most crucially, provide and discuss reference to recent work in terms of motivating the need for the HuCI. Clarify we include arts. Introduce GLAM as a whole sector.}

%\vspace{0.5cm}
%Introduce terminology:
%\begin{itemize}
    %\item HuCI is the Arts\&Humanities citation corpus. HuCI will exist as a virtual corpus, materialized by several co-existing repositories (e.g. OpenCitations, Wikidata) providing access to their citation data via SPARQL and other endpoints. HuCI is, in this sense, fully distributed and existing by means of an infrastructure proposed in what follows.
    %\item Scholar Index is the application layer which will allow to distribute the creation and curation of citation data via a digital library application embedding the necessary machine learning components, and to centrally expose them via a citation index.
%\end{itemize}

Citation  indexes  are  by  now  part  of  the  research  infrastructure  in  use  by  most  scientists:  a  necessary  tool  in  order  to  cope  with  the  increasing  amounts  of  scientific  literature  being  published. However, existing commercial  citation  indexes  are  designed  for  the  sciences  and  have  uneven  coverage  and  unsatisfactory  characteristics  for humanities\footnote{Throughout this paper we use the term \textit{humanities} as a shorthand for Arts \& Humanities (A\&H). To a degree, the Social Sciences are also concerned.} scholars. This situation has both discouraged the usage of citation indexes and hindered bibliometric studies of humanities disciplines.  

The creation of a citation index for the humanities may well appear as a daunting task due to several characteristics of this field, such as its fragmentation into several sub-disciplines, the common practice of publishing research in languages other than English, as well as the amount of scholarship from past centuries that is still waiting to be digitised.

Notwithstanding these challenges, we argue that the creation of such an index can be highly beneficial to humanities scholars for, at least, the following reasons. Firstly, humanities scholars have long been relying on information seeking behaviours that leverage citations and reference lists for the discovery of relevant publications -- a strategy that citation indexes are designed to support and facilitate. Secondly, a comprehensive citation index for the humanities will be a valuable source of data for researchers willing to conduct bibliometric studies of the humanities. Lastly, capturing the wealth of references to primary and secondary sources contained in humanities literature will allow to create links between archives, galleries, libraries and museums where digitized copies of these sources can increasingly be found.

Before continuing with this paper, we introduce key terminology related to citation indexing that will be used throughout this paper, adopting the definitions from \cite{peroni_opencitations_2018}. These are: bibliographic entity, bibliographic resource and bibliographic citation.
%We provide herein some definitions so as to avoid ambiguities when these terms will be mentioned in the rest of the paper. 
A \textbf{bibliographic entity} is any entity which can be part of the bibliographic metadata of a bibliographic artifact: it can be a person, an article, an identifier for a particular entity (e.g., a DOI), a particular role held by a person (e.g., being an author) in the context of defining another entity (e.g., a journal article), and so forth. A \textbf{bibliographic resource} is a kind of bibliographic entity that can cite or be cited by other bibliographic resources (e.g., a journal article), or that contains other  resources (e.g., a journal). A \textbf{bibliographic citation} is another kind of bibliographic entity: a conceptual directional link from a citing bibliographic resource to a cited bibliographic resource.
The citation data defining a particular citation must include the representation of the conceptual directional link of the citation and the basic metadata of the involved bibliographic resources, that is to say sufficient information to create or retrieve textual bibliographic references for each of the bibliographic resources. Following \cite{peroni_open_2018}, we say that a bibliographic citation is an open citation when the citation data needed to define it are compliant with the following principles: structured, separate, open, identifiable, available.

The remaining of this paper is organised as follows. In Section \ref{sec:related-work} we discuss previous work on analysing the behaviour of humanities scholars in relation to information retrieval. We also present the main limitations of existing citation indexes, seen from the perspective of the humanities, and outline the main obstacle that citation indexing has faced in this area. In Section \ref{sec:citation-index-AH-needs} we argue for the need of a Humanities Citation Index (HuCI from now onwards) and in Section \ref{sec:citation-index-characteristics} we present what we believe are the essential characteristics that such an index should have. We then propose a possible implementation of HuCI, based on a federated and distributed research infrastructure (Section \ref{sec:research-infrastructure}). We conclude with some considerations on how HuCI relates to recent efforts to create open infrastructures for research.
\section{Material} \label{sec: Material}

The experiments detailed in this paper were performed on a private database composed of 249 OCT images of dimensions $M\times N=496\times768$ pixels. In particular, 156 normal and 93 glaucomatous circumpapillary samples were analysed from 89 and 59 patients, respectively. Each B-scan was diagnosed by experts ophthalmologists from Oftalvist Ophthalmic Clinic. Note that \textit{Heidelberg Spectrallis} OCT system was employed to acquire the circumpapillary OCT images with an axial resolution of 4-5 $\mu$m. 


\section{Methodology} \label{sec: Methodology}

\subsection{Data Partitioning} \label{subsec: Data_partitioning}
A data partitioning stage was carried out to divide the database into different training and test sets. Specifically, $\frac{4}{5}$ of the circumpapillary images, which corresponds to 73 glaucomatous and 124 normal samples, from 12 and 18 patients respectively, composed the training set, whereas the test set was defined by $\frac{1}{5}$ of the data (20 with glaucoma and 32 normal B-scans from 12 and 18 patients). In addition, for the training set, we also performed an internal cross-validation (ICV) stage to control the overfitting, as well as to select the best neural network hyper-parameters. Finally, the independent test set was used to evaluate the definitive predictive models, which were created using the entire training set. 



\subsection{Learning from scratch} \label{subsec: From_scratch}

Similarly to the methodology exposed in \cite{gomez2019automatic}, we propose in this paper the use of shallow CNNs from scratch to address the glaucoma detection, taking into account the significant differences between our grey-scale circumpapillary OCT images and other large databases containing natural images, which are widely used for transfer-learning techniques.

During the internal cross-validation (ICV) stage, an empirical exploration was carried out to determine the best hyper-parameter combination in terms of minimisation of the binary cross-entropy loss function. Different network architectures composed of diverse learning blocks were developed. In particular, convolutional, pooling, batch normalisation and dropout layers were considered to address the feature extraction stage. The variable components of each layer, such as the convolutional filters, pooling size, dropout coefficients, as well as the number of convolutional layers in each block were optimised during the experimental phase.  
Regarding the top model, the use of flatten, dropout and fully-connected layers with a different number of neurons was studied. Also, global max and global average pooling layers were analysed in order to reduce the number of trainable parameters.
Moreover, we implemented an optimal weighting factor of [1.35, 0.79] during the training of the models to alleviate the unbalanced problem between classes. 

After the ICV stage, the best CNN architecture was found using four convolutional blocks, as it is detailed in Table \ref{CNN_from_scratch}. It is remarkable the use of the global max-pooling (GMP) layer applied in the last block, which allows extracting the maximum activation of each convolutional filter before the classification layer. Also, note that batch normalization and dropout layers were not used because no improvement was reported during the validation phase. Only a dense layer with a \textit{softmax} activation and 2 neurons, corresponding to glaucoma and healthy classes, was defined. 

\begin{table}[htbp]
\caption{Proposed CNN architecture trained from scratch.}
\label{CNN_from_scratch}
\renewcommand{\arraystretch}{1.1} % rows
\setlength\tabcolsep{8 pt} % cols
\small
\begin{center}
\begin{tabular}{ccc}
\hline
 \textbf{Layer name} & \textbf{Output shape}                & \textbf{Filter size}            \\ \hline
Input layer         & 496 x 768 x 1                        & N/A                             \\
Conv1\_1            & 496 x 768 x 32                       & 3 x 3 x 32                      \\
MaxPooling          & 248 x 384 x 32                       & 2 x 2 x 32                      \\ 
Conv2\_1            & 248 x 384 x 64                       & 3 x 3 x 64                      \\ 
MaxPooling          & 124 x 192 x 64                       & 2 x 2 x 64                      \\ 
Conv3\_1            & 124 x 192 x 128                       & 3 x 3 x 128                      \\
MaxPooling          & 62  x 96  x 128                       & 2 x 2 x 128                      \\ 
Conv4\_1            & 62  x 96  x 256                      & 3 x 3 x 256                     \\
MaxGlobalPool       & 256                                  & N/A                             \\ 
Dense (softmax)     & 2                                    & N/A                             \\ \hline
\end{tabular}
\end{center}
\end{table}

The optimal hyper-parameters combination was achieved by training the CNNs during 150 epochs, using Adadelta optimizer with a learning rate of 0.05 and a batch size of 16. It should be noticed that we also proposed the use of data augmentation (DA) techniques \cite{wong2016} to elucidate how important is the creation of artificial samples when addressing small databases. Specifically, a factor ratio of 0.2 was applied here to perform random geometric and dense elastic transformations from the original images.




 


\subsection{Learning by fine tuning} \label{subsec: Deep_learning}

Deeper architectures networks could improve the models' performance, but a large number of images annotated by experts would be necessary for training a deep CNN from scratch. For this reason, we propose in this section the use of fine-tuning techniques \cite{hoo2016}, which allows training CNNs with greater depth using the weights pre-trained on large databases, without the need to train from scratch. In particular, we applied a deep fine-tuning \cite{tajbakhsh2016} strategy to transfer the wide knowledge acquired by several state-of-the-art networks, such as VGG16, VGG19, InceptionV3, Xception and ResNet, when they were trained on the large \textit{ImageNet} data set. Attending to the small database used in this work, only the coefficients of the last convolutional blocks (4 and 5) were retrained with the specific knowledge corresponding to the circumpapillary OCT images. The rest of coefficients were frozen with the values of the weights pre-trained with 14 million of natural images contained in \textit{Imagenet} database.

Additionally, similarly to the proposed learning from scratch strategy, an empirical exploration of different hyper-parameters and top-model architectures was considered for all networks. It is important to notice that InceptionV3, Xception and ResNet architectures reported a poor performance due to their extensive depth (42, 36 and 53 convolutional layers, respectively). However, the family of VGG architectures achieved the best performance, in line with the findings in the literature \cite{gomez2019automatic}. Specifically, VGG16 base model is composed of five convolutional blocks according to Fig. \ref{fig:VGG16}, where blue boxes correspond to convolutional layers activated with \textit{ReLu} functions and red-grey boxes represent max-pooling layers. VGG19 base model is composed of the same architecture, but including an extra convolutional layer in the last three blocks. 

A top model composed of global max pooling and dropout layers with a coefficient of 0.4, followed by a softmax layer with two neurons, provided the best model performance when VGG architectures were fine-tuned (see Fig. \ref{fig:VGG16}). Regarding the selection of hyper-parameters combination, Adadelta optimizer with a learning rate of 0.001 reported the best learning curves when the model was forward, and backward, propagated during 125 epochs with a batch size of 16, trying to minimise the binary cross-entropy loss function.  

Note that an initial down-sampling $\times0.5$ of the original images was necessary to alleviate the GPU memory problems during the training phase. Besides, replicating $\times3$ the channels of the grey-scale was necessary to adapt the input images in order to fine tune the CNNs. Data augmentation (DA) techniques with a factor of 0.2 were also considered.
 

\begin{figure}[h]
\centering
\includegraphics[width=8.5cm]{Figures/VGG16architecture.pdf}
\caption{Network architecture used to discern between glaucomatous and healthy OCT samples by fine-tuning the VGG16 base model. Note that numeric values of the filters are correctly defined in the image, although they do not correspond to the representation size of the boxes due to space problems.}
\label{fig:VGG16}
\end{figure}



\section{Results and discussion} \label{sec: Results}

\subsection{Validation results}

In this stage, we present the results achieved during the ICV stage for each of the proposed CNNs. We expose in Table \ref{valRes_fromScratch} a comparison of the CNNs trained from scratch, in terms of mean $\pm$ standard deviation. Several figures of merit are calculated to evidence the differences between using or not data augmentation (DA) techniques. In particular, sensitivity (SN), specificity (SPC), positive predictive value (PPV), negative predictive value (NPV), F-score (FS), accuracy (ACC) and area under the ROC curve (AUC) are employed. 

\begin{table}[h]
\caption{Classification results reached during the ICV stage from the proposed CNNs trained from scratch.}
\label{valRes_fromScratch}
%\renewcommand{\arraystretch}{1.09}
\setlength\tabcolsep{8 pt}
\small
\begin{center}
\begin{tabular}{cccc}
\hline
\multicolumn{1}{l}{}{} & \textbf{Without DA}       & \textbf{With DA} \\
\hline
\textbf{SN}         & $0.7657 \pm 0.2032$         &                  $\textbf{0.8771}\pm\textbf{0.1281}$\\
\textbf{SPC}        & $\textbf{0.9270} \pm \textbf{0.1302}$  &      $0.8047 \pm 0.1514$ \\
\textbf{PPV}        & $\textbf{0.8721} \pm \textbf{0.0662}$  &      $0.7477\pm 0.14061$  \\
\textbf{NPV}        & $0.8808 \pm  0.0971$                   &      $\textbf{0.9224} \pm \textbf{0.0678}$ \\
\textbf{FS}         & $\textbf{0.8016} \pm \textbf{0.1309}$  &      $0.7980\pm 0.10745$   \\
\textbf{ACC}        & $\textbf{0.8679} \pm \textbf{0.0781}$  &      $0.8315 \pm 0.0985$ \\
\textbf{AUC}        & $0.9152 \pm 0.0490$     &                     $\textbf{0.9319} \pm \textbf{0.0386}$\\
\hline
\end{tabular}
\end{center}
\end{table}

Significant differences between CNNs trained with and without data augmentation techniques can be appreciated in Table \ref{valRes_fromScratch}, especially related to the sensitivity and specificity metrics. 
Worth noting that the learning curves relative to the CNN trained without implementing DA algorithms reported slight overfitting during the validation phase. This fact is evidenced in the high sensitivity standard deviation of the model.

Additionally, we also detail in Table \ref{valtResults_fineTuning} the validation results achieved from the fine-tuned VGG networks, since they provided a considerable outperforming with respect to the rest of state-of-the-art architectures during the ICV stage. Specifically, VGG16 reaches better results for all figures of merit, although both architectures report similar behaviour. In comparison to the CNNs trained from scratch, VGG16 provides the best model performance too. 

\begin{table}[h]
\caption{Results comparison between the best fine-tuned CNNs proposed during the validation phase.}
\label{valtResults_fineTuning}
%\renewcommand{\arraystretch}{1.09}
\setlength\tabcolsep{8 pt}
\small
\begin{center}
\begin{tabular}{cccc}
\hline
\multicolumn{1}{l}{}{} & \textbf{VGG16}       & \textbf{VGG19} \\
\hline
\textbf{SN}         & $\textbf{0.7800}   \pm \textbf{0.1302}$       & $0.7400 \pm 0.1462$ \\
\textbf{SPC}        & $\textbf{0.9677} \pm \textbf{0.0334}$         & $0.9597\pm 0.0283$\\
\textbf{PPV}        & $\textbf{0.9401} \pm  \textbf{0.0643}$      & $0.9180 \pm 0.0602$ \\
\textbf{NPV}        & $\textbf{0.8864} \pm \textbf{0.0662}$        & $0.8670\pm 0.0692$  \\
\textbf{FS}         & $\textbf{0.8466} \pm  \textbf{0.0720}$       & $0.8131\pm 0.0936$   \\
\textbf{ACC}        & $\textbf{0.8984} \pm  \textbf{0.0468}$      & $0.8786 \pm 0.0563$ \\
\textbf{AUC}        & $\textbf{0.9463} \pm  \textbf{0.0339}$     & $0.9416 \pm 0.0501$\\
\hline
\end{tabular}
\end{center}
\end{table}

%After comparing the validation results shown in Table \ref{valRes_fromScratch} and \ref{valtResults_fineTuning}, a significant improvement of the fine-tuned CNNs is demonstrated in relation to the CNNs trained from scratch. This fact evidences that the use of pre-trained architectures is a promising strategy when the amount of images annotated by experts is not enough to train a deep CNN from scratch.

\subsection{Test results}
 
In order to provide reliable results, an independent test set was used to carry out the prediction stage. Table \ref{testResults} shows a comparison between all proposed deep-learning models to evaluate their prediction ability by means of different figures of merit. Additionally, we expose in Fig. \ref{testROCs} the ROC curve relative to each proposed CNN to visualise the differences.

\begin{table}[h]
\caption{Classification results achieved during the prediction stage from the proposed CNNs trained from scratch (FS) and fine-tuning the VGGs network architectures.}
\label{testResults}
%\renewcommand{\arraystretch}{1.1}
%\setlength\tabcolsep{4 pt}
\small
\begin{center}
\begin{tabular}{ccccc}
\hline
\multicolumn{1}{l}{}{} & \textbf{FS without DA}       & \textbf{FS with DA}  & \textbf{VGG16}       & \textbf{VGG19}\\
\hline
\textbf{SN}          & 0.7632           & 0.7895    & \textbf{0.8510}       & \textbf{0.8510}  \\
\textbf{SPC}         & 0.7250           & 0.6750    & 0.9064                & \textbf{0.9688} \\
\textbf{PPV}         & 0.7250           & 0.6977    & 0.8490                & \textbf{0.9444}   \\
\textbf{NPV}         & 0.7632           & 0.7714    & 0.9063                & \textbf{0.9118}   \\
\textbf{FS}          & 0.7436           & 0.7407    & 0.8500                & \textbf{0.8947}    \\
\textbf{ACC}         & 0.7436           & 0.7308     & 0.8846               & \textbf{0.9230}   \\
\textbf{AUC}         & 0.8132           & 0.8230     & 0.9578               & \textbf{0.9594} \\
\hline
\end{tabular}
\end{center}
\end{table}


\begin{figure}[h]
\begin{center}
\includegraphics[height=3cm, width=7.25cm]{Figures/ROCs_test.pdf} \\
\end{center}
\caption{ROC curves corresponding to the prediction results reached from the different proposed CNNs.}
\label{testROCs}
\end{figure}


Test results exposed in Fig. \ref{testResults} are in line with those achieved during the validation phase. However, due to the randomness effect of the data partitioning (which is accentuated in small databases), significant differences may exist in the prediction of each subset. This fact mainly affects to the CNNs trained from scratch because all the weights of the network were trained with the images of a specific subset, whereas the proposed fine-tuned architectures keep most of the weights frozen. Regarding the ROC curves comparison, Fig. \ref{testROCs} shows that fine-tuned CNNs report a significant improvement in relation to the networks trained from scratch.


It is important to remark that an objective comparison with other state-of-the-art studies is difficult because there are no public databases of circumpapillary OCT images. Additionally, each group of researchers addresses glaucoma detection using a different kind of images. Notwithstanding, we detail a subjective comparison with other works based on similar methodologies applied to fundus images. In particular, \cite{gomez2019automatic} fine-tuned the VGG19 architecture and achieved an AUC of 0.94 predicting glaucoma. Also, \cite{christopher2018performance} reached an AUC of 0.91 applying transfer learning techniques to the ResNet architecture. Otherwise, authors in \cite{chen2015glaucoma} proposed a CNN from scratch obtaining AUC values of 0.83 and 0.89 from two independent databases. Basing on this, the proposed learning methodology exceeds the state-of-the-art results, achieving an AUC of 0.96 during the prediction of the test set. 



\BlankLine
\textbf{Class Activation Maps (CAMs)}

We compute the class activation maps to generate heat maps highlighting the interesting regions in which the proposed model is paying attention to determine the class of each specific circumpapillary OCT image. In Fig. \ref{Class_Activation}, we expose the CAMs relative to random specific glaucomatous and normal samples in order to elucidate what is VGG19 taking into account to discern between classes. 


\begin{figure}[h]
\begin{center}
\begin{tabular}{cc}
\includegraphics[height=4.2cm,width=4cm]{Figures/GlaucomaCAMs.pdf} &
\includegraphics[height=4.2cm, width=4cm]{Figures/NormalCAMs.pdf} \\
(a) &
(b) \\
\end{tabular}
\end{center}
\caption{Heat maps extracted from the CAMs computation for (a) glaucomatous and (b) healthy circumpapillary images.}
\label{Class_Activation}
\end{figure}

The findings from the CAMs are directly in line with the reported by expert clinicians, who claim that a thickening of the RNFL is intimately linked with healthy patients, whereas a thinning of the RNFL evidence a glaucomatous case. That is just what heat maps in Fig. \ref{Class_Activation} reveal. Therefore, the results suggest that the proposed circumpapillary OCT-based methodology can provide a great added value for glaucoma diagnosis taking into account that information similar to that of specialists is reported by the model without including any previous clinician knowledge. 


 



\section{Conclusions and Future Works}

\md{In this paper, we show that any graph shuffle can be done by PSSs. 
In particular, we need $2(n+m)$ cards and $|\f{Deg}_G|+1$ PSSs, where $n$ and $m$ are the numbers of vertices and arrows of $G$, respectively. 
We left as open problems (1) to remove the computation of an isomorphism between two isomorphic graphs in a graph shuffle protocol keeping everything efficient and (2) to find another interesting applications for our graph shuffle protocol. 
We hope that this research direction (i.e., constructing a nontrivial shuffle from the standard shuffles such as RCs, RBCs, and PSSs) will attract the interest of researchers on card-based cryptography and new shuffle protocols will be proposed in future work. 
}

\newpage

% References should be produced using the bibtex program from suitable
% BiBTeX files (here: strings, refs, manuals). The IEEEbib.bst bibliography
% style file from IEEE produces unsorted bibliography list.
% -------------------------------------------------------------------------

\balance
\bibliographystyle{IEEEbib}
\bibliography{strings,refs}

\end{document}
