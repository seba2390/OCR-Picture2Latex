\section{Introduction} \label{sec: Introduction}

Glaucoma has become the leading cause of blindness worldwide, according to \cite{jonas2018}. It is characterized by causing progressive structural and functional damage to the retinal optic nerve head (ONH). Recent studies advocate that roughly 50\% of people suffering from glaucoma in the world are undiagnosed and ageing populations suggest that the impact of glaucoma will continue to rise, affecting 111.8 million people in 2040 \cite{reference2_intro}. Therefore, early treatment of this chronic disease could be essential to prevent irreversible vision loss. 

Currently, a complete glaucoma study usually includes medical history, fundus photography, visual field (VF) analysis, tonometry and optic nerve imaging tests such as optical coherence tomography (OCT). Most of the state-of-the-art studies addressed the glaucoma detection via fundus image analysis, making use of visual field tests and relevant parameters like the intraocular pressure (IOP) \cite{kim2017, wang2019}. Specifically, J. Gómez-Valverde et al. \cite{gomez2019automatic} performed a comparison between convolutional neural networks (CNNs) trained from scratch and using fine-tuning techniques. Also, the authors in \cite{shibata2018development,christopher2018performance} considered the use of transfer learning and fine-tuning methods applied to very popular state-of-the-art network architectures to identify glaucoma on fundus images. Other studies such as \cite{muhammad2017hybrid, thakoor2019} carried out a combination between OCT B-scans and fundus images to obtain an RNFL thickness probability map which was used as an input to the CNNs. In this paper, contrary to the studies of the literature, we propose an end-to-end system for glaucoma detection based only on raw circumpapillary OCT images, without using another kind of images or external expensive tests related to the VF and IOP parameters. 
It is important to highlight that circumpapillary OCT images as shown in Fig. \ref{fig:circum} correspond to circular scans located around the ONH, where rich information about different retinal layers structures can be found. Additionally, several studies claimed that circumpapillary retinal nerve fiber layer (RNFL) is essential to detect early glaucomatous damage \cite{reference_objetive1, reference_objetive2, reference_objetive3}. For that reason, one of the main novelties of this paper is focused on demonstrating that a single circumpapillary OCT image may be of great interest when carrying out an accurate glaucoma detection.

\begin{figure}[b]
\centering
\includegraphics[width=8.5cm]{Figures/Circumpapillary.pdf}
\caption{B-scan around the retinal ONH corresponding to a circumpapillary OCT image. RNFL is highlighted in red.}
\label{fig:circum}
\end{figure}

We propose two different data-driven learning strategies to develop computer-aided diagnosis systems capable of discerning between glaucomatous and healthy eyes just from B-scans around the ONH. Several CNNs trained from scratch and different fine-tuned state-of-the-art architectures were considered. Furthermore, we propose, for the first time in this kind of images, the class activation maps computation in order to compare the location information reported by the clinicians with the heat maps generated by the developed models. Heat maps allow highlighting the regions in which the networks pay attention to determine the class of each specific sample.


%The OCT images are widely used in the literature to asses the retinal ONH structure \cite{gomez2019automatic}.  such as the visual field analysis. Most of them are based on retinal fundus images analyses through convolutional neuronal networks (CNNs) \cite{chen2015glaucoma, raghavendra2018deep, shibata2018development, christopher2018performance, gomez2019automatic}.


%Hassan Muhammad et al. \cite{muhammad2017hybrid} use a combination of different thickness and probabilistic maps over the fundus images to glaucoma detection.
%To the best of the authors’ knowledge, all state-of-the-art automatic glaucoma detection methods are focused on fundus images, but of them analyse the raw circumpapillary OCT images. This kind of samples corresponds to circular scans located around the ONH, where information related to different retinal layers structure can be found (see Fig. \ref{fig:circum}). 


%For all of the above, in this paper, two different approaches to  glaucoma detection based on raw circumpapilary OCT analysis are presented. The first one is based on the development and training of a CNN architecture from scratch. The second one consists in the fine-tuning technique on different pre-trained CNN architectures to deal with the overfiting caused by the small number of samples.
