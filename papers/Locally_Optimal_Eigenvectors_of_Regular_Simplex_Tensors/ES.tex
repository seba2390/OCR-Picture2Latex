\section{Skewness  for the regular polygon}\label{polygon}

The  presented  Theorem \ref{Theorem_optimal}  has revealed  the one-to-one  corresponding  relationship between the locally  skewness  directions  and  altitudes of the simplex. 
In  this part, we  further  provide  an  auxiliary  lemma to  illustrate  the  special characteristics  of  the simplex  and  the  triangle. 
 

\begin{lemma}\label{Theorem_ploygon}
	For  the  regular  polygon  with  $p$  vertices  where  $p$ $\ge$ 4,  its  skewness  value   projected  along any  direction  will  always be  equal to  0.  
\end{lemma}


Before  proceeding  our  proof,  the  following  theorem is  very  useful:
\setcounter{theorem}{3}
\begin{theorem}(Ref \cite{upperbound})\label{upperbound}
	The number of  real  Z-eigenpairs for  symmetric tensor  of order $m$ and dimension $ n $ is  at most
	\begin{equation}\label{at_most}
	M(m,n)=
	\frac {(m-1)^{n}-1}  {m-2}.
	\end{equation}
\end{theorem}









Note  that   for  different  data $ \mathbf R$ and  thus  different   symmetric  tensors, 
the number of  these   locally  optimal  solutions (denoted $K$)   may    vary.  Theorem  \ref{upperbound}  provides  a  upper  bound   for  the  number  of  eigenpirs of  tensor, 
and thus also  a  weak   upper  bound  for  $K$.
 For  example,   for  the   symmetric tensors  of order 3 and dimension 3, it  also  always  holds   that 
$M$(3,3) = 7 and $ K$ $\le$ 7.
Equipped  with  such  a  theory, we  begin to proceed  our  proof for  the presented   Lemma \ref{Theorem_ploygon}.
\begin{proof}
	The proof   mainly   includes   two  steps: 
	
	\textbf{Step 1:}  	we    show  that  the skewness value of the   regular  polygon  with  $p$  vertices 
	($p$ $\ge$ 4)  along  any  direction  is  a  constant;
	
	\textbf{Step 2:}   we   demonstrate that  this  constant  is  equal to 0. 
	
	Step  1  is  proved  by contradiction.   
	On  the  one  hand, assume  that   for  the   regular  polygon  with  $p$  vertices 
	($p$ $\ge$ 4)  in  a  2D  plane,  there  is  at  least  one  locally  optimal   direction (denoted $\mathbf u_{1}$),  
	according  to  the  symmetry  property  of  the  regular  polygon, 
	then  the  direction  which    rotates   $\mathbf u_{1}$  with  an  angle  of  $ \frac  {2\pi i} {p}$  ($i=1, 2 , \dots, p$)     will  also be  one locally  optimal   direction.
	Thus,   we  can  conclude  that  $ K \ge p \ge  4$. 
	On the other  hand,  the  third-order  statistical  tensor  of  the   regular  polygon   in  a  2D  plane
	is  always with   a   size  of   2  $\times$ 2  $\times$ 2,  and it  should  hold  that  $ K \le  M(2,3) = 3$    based  on  Theorem  \ref{upperbound}. 
	Therefore,  there  exists  contradiction  and the  assumption do  not  hold,  and  we  can conclude that  the  number of  locally  optimal   directions  is  equal to  0,  which  implies  that  the claim  in  step  1  is  true.  
	
	Then,  for  step 2,   it is  enough    to    show  that  the  skewness  value  in  one  certain  direction  for the   regular  polygon with $p$ $\ge $ 4  is  equal to  0. 
	First,  the  $p$  vertices  for  the  regular  polygon   can   be  denoted  as  matrix  with  the  following  form:
\begin{align}
\notag
\mathbf {R}
= 
\begin{bmatrix}
   sin(0)   &     sin(\frac  {2\pi } {p})  &   sin(\frac  {4\pi } {p})  & \dots   &  sin(\frac  {2\pi (p-1)} {p}) \\  
 cos(0)   &     cos(\frac  {2\pi } {p})  &  cos(\frac  {4\pi } {p})  &  \dots   &  cos(\frac  {2\pi (n-1)} {p})
\end{bmatrix} . 
\end{align}
It  is  easily  checked  that  each  row  of  $ \mathbf {R}$  is  with  zero  mean,   due  to  the  fact  
that  $sin(\frac  {2\pi i } {p}) (i=0,1,2,\dots, p-1)$  is a  periodical  function  whose  period is  $p$.
Next,  we   consider  the  skewness  value  of  $ \mathbf {R}$ along  the    horizontal  direction $ \mathbf  w =[1 ,  0]^{\mathrm T}$,   and  we  only need  to  calculate  the  numerator, which  follows:
\begin{align}
k_{3} (\mathbf  w^{\mathrm T}\mathbf {R})
\notag
&=
\sum\limits_{i=0}^{p-1}  sin^{3}(\frac  {2\pi i } {p})
=
 \sum\limits_{i=0}^{p-1}  sin(\frac  {2\pi i } {p}) (1-cos^{2}(\frac  {2\pi i } {p})) 
\\  \nonumber 
& =
 \sum\limits_{i=0}^{p-1}  sin(\frac  {2\pi i } {p}) -  \sum\limits_{i=0}^{p-1}  sin(\frac  {2\pi i } {p}) cos^{2}(\frac  {2\pi i } {p})
 =
 0. 
\end{align}
And  the  proof  is  complete. $\blacksquare$ 
\end{proof}
