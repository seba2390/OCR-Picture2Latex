\newcommand{\applink}{\url{https://play.google.com/store/apps/details?id=com.mark.myapplication}.}

To demonstrate the utility of our method, we synthesized a music player Android app\footnote{\applink} from a \TSL specification.
A major challenge in developing Android apps is the temporal behavior of an app through the \textit{Android lifecycle}~\cite{Shan16}.
The Android lifecycle describes how an app should handle being paused, when moved to the background, coming back into focus, or being terminated.
In particular, \textit{resume and restart errors} are commonplace and difficult to detect and correct~\cite{Shan16}.
Our music player app demonstrates a situation in which a resume and restart error could be unwittingly introduced when programming by hand, but is avoided by providing a specification.
We only highlight the key parts of this example here to give an intuition of \TSL, leaving a more in-depth exposition to \cref{apx:musicspec}.

Our music player app utilizes the Android music player library~(\name{MP}), as well as its control interface~(\name{Ctrl}). It pauses any playing music when moved to the background (for instance if a call is received), and continues playing the currently selected track~(\name{Tr}) at the last track position when the app is resumed.
In the Android system~(\name{Sys}), the \texttt{leaveApp} method is called whenever the app moves to the background, while the \texttt{resumeApp} method is called when the app is brought back to the foreground. To avoid confusion between pausing music and pausing the app, we use \texttt{leaveApp} and \texttt{resumeApp} in place of the Android methods onPause and onResume.
A programmer might manually write code for this as shown on the left in \cref{fig:smallcode}.

The behavior of this can be directly described in \TSL as shown on the right in \cref{fig:smallcode}.
Even eliding a formal introduction of the notation for now, the specification closely matches the textual specification.
First, when the user leaves the app and the music is playing, the music pauses.
Likewise for the second part, when the user resumes the app, the music starts playing again.

\begin{figure}[t]
\vspace{-1em}
\begin{minipage}{.42\textwidth}
  \vspace{-0.8em}
\begin{lstlisting}
Sys.leaveApp()
  if (MP.musicPlaying())
    Ctrl.pause();
\end{lstlisting}
\vspace{-1em}
\begin{lstlisting}
Sys.resumeApp() {
  pos = MP.trackPos();
  Ctrl.play(Tr,pos);
}
\end{lstlisting}
\vspace{-0.8em}
\end{minipage}%
\vrule{}%
\begin{minipage}{.59\textwidth}
\vspace{-0.8em}
\begin{align*}
& \name{ALWAYS} \; \Big(\name{leaveApp} \ \, \name{Sys} \; \wedge \; \name{musicPlaying} \ \, \name{MP} \\[-0.5em]
& \quad \hspace{2.5em} \impl \upd{\name{Ctrl}}{\const{pause}} \Big) \\[0.8em]
& \name{ALWAYS} \; \Big(\name{resumeApp} \ \, \name{Sys}  \\[-0.5em]
& \quad \hspace{2.5em} \impl  \upd{\name{Ctrl}}{\name{play} \ \, \name{Tr} \ \, (\name{trackPos} \ \, \name{MP})} \Big)
\end{align*}
\vspace{-0.8em}
\end{minipage}
\vspace{-0.5em}
\caption{Sample code and specification for the music player app.}
\label{fig:smallcode}
\end{figure}
%
However, assume we want to change the behavior so that the music only plays on resume when the music had been playing before leaving the app in the first place.
In the manually written program, this new functionality requires an additional variable~\texttt{wasPlaying} to keep track of the music state.
Managing the state requires multiple changes in the code as shown on the left in \cref{fig:bigcode}.
The required code changes include: a conditional in the \texttt{resumeApp} method, setting \texttt{wasPlaying} appropriately in two places in \texttt{leaveApp}, and providing an initial value.
Although a small example, it demonstrates how a minor change in functionality may require wide-reaching code changes.
In addition, this change introduces a globally scoped variable, which then might accidentally be set or read elsewhere.
%
In contrast, it is a simple matter to change the TSL specification to reflect this new functionality.
Here, we only update one part of the specification to say that if the user leaves the app and the music is playing, the music has to play again as soon as the app resumes.

\begin{figure}[t]
\begin{minipage}{.44\textwidth}
\vspace{-0.8em}
\begin{lstlisting}
bool wasPlaying = false;
\end{lstlisting}
\vspace{-0.5em}
\begin{lstlisting}
Sys.leaveApp()
  if (MP.musicPlaying()) {
    wasPlaying = true;
    Ctrl.pause();
  }
  else
    wasPlaying = false;
\end{lstlisting}
\vspace{-0.5em}
\begin{lstlisting}
Sys.resumeApp()
  if (wasPlaying) {
    pos = MP.trackPos();
    Ctrl.play(Tr,pos);
  }
\end{lstlisting}
\vspace{-0.8em}
\end{minipage}%
\vrule{}%
\begin{minipage}{.57\textwidth}
\begin{align*}
& \,\name{ALWAYS} \; \Big( (\name{leaveApp} \ \, \name{Sys} \; \wedge \ \name{musicPlaying} \ \, \name{MP} \\
& \,\quad \hspace{2.5em} \impl \upd{\name{Ctrl}}{\const{pause}} ) \\[0.5em]
& \,\quad \hspace{2.5em} \; \wedge \, (\upd{\name{Ctrl}}{\name{play} \ \, \name{Tr} \ (\name{trackPos} \ \, \name{MP})} \ \\
& \,\quad \hspace{2.5em} \phantom{\impl} \ \ \name{AS\_SOON\_AS} \ \ \name{resumeApp} \ \, \name{Sys} ) \Big)
\end{align*}
\end{minipage}
\vspace{-0.5em}
\caption{The effect of a minor change in functionality on code versus a specification.}
\label{fig:bigcode}
\end{figure}
%
Synthesis allows us to specify a temporal behavior without worrying about the implementation details.
In this example, writing the specification in \TSL has eliminated the need of an additional state variable, similarly to a higher order \texttt{map} eliminating the need for an iteration variable.
However, in more complex examples the benefits compound, as \TSL provides a modular interface to specify behaviors, offloading the management of multiple interconnected temporal behaviors from the user to the synthesis engine.
