We have introduced Temporal Stream Logic, which allows the user to
specify the control flow of a reactive program.  The logic cleanly
separates control from complex data, forming the foundation for our
procedure to synthesize FRP programs. By utilizing the purity of
function transformations our logic scales independently of the
complexity of the data to be handled. While we have shown that
scalablility comes at the cost of undecidability, we addressed this
issue by using a CEGAR loop, which lazily refines the
underapproximation until either a realizing system implementation or
an unrealizability proof is found.

TSL also provides the foundations for further extensions. For example,
a user may want to fix the semantics for a partial set of functions and
predicates to be utilized by the synthesis tool. Such additional
refinements can be implemented as part of a much richer \textit{TSL
  Modulo Theory} framework.

Our experiments indicate that TSL synthesis works well in practice and on
a wide range of programming applications.  In general, we hope the
applications of this new logic and approach to reactive synthesis will
stimulate further research into the scalable, real-world use of
temporal logics for synthesis.
