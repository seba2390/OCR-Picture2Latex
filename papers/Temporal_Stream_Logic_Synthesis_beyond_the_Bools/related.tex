Synthesizing reactive programs has been explored in domains such as imperative programs over finite domains~\cite{madhusudan2011synthesizing} and parallel execution strategies for sequential programs~\cite{conf/fmcad/BloemHKKAS14}.
There have also been many alternatives to LTL for specifying properties of reactive systems, such as $ \mu$-calculus~\cite{Kupferman:2000}, Signal Temporal Logic~\cite{hscc/sanjit15}, Ground Temporal Logic~\cite{cyrluk1994ground}, and coalgebraic logics~\cite{bonsangue2008coalgebraic}.
An example of a synthesis approach that integrates control and data is recent work on strategy synthesis for linear arithmetic games~\cite{Farzan:2017}.
While reactive synthesis has focused on the complex control aspects of reactive systems,
  deductive and inductive synthesis has been concerned with the data transformation aspects in non-reactive and sequential programs~\cite{vechevYY13,kuncak2010complete,osera2015type,solarLezama13,Feser:2015:SDS:2737924.2737977,Isil17}.
Abstraction-based approaches to reactive synthesis~\cite{Beyene:2014:CAS:2535838.2535860,Dimitrova2012,hsu2018multi,mallik2016compositional} can be seen as a link between deductive and reactive synthesis.
In terms of FRP, a Curry-Howard correspondence between LTL and FRP in a dependently typed language was discovered~\cite{plpv/Jeffrey12,jeltsch2012towards} and used to prove properties of FRP programs~\cite{Cave2014Fair,krishnaswami2013higher}.
