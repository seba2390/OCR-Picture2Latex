\subsection{Specifying a Music Player in TSL}
\label[appendix]{apx:musicspec}

To demonstrate the simplicity of TSL, we illustrate the creation of a TSL specification for the music player Android app of \cref{sec:motiv}.
For the concrete implementation, we use the \texttt{MediaPlayer} class (\name{MP}) from the Android API, which provides functions to pause and play, as well as a predicate to check if music is currently playing or not.
Specific to the Android OS, we receive signals when a user leaves or resumes the app.
Specific to our particular app, we also use two buttons in the UI that deliver signals when a user presses play or pause.

In contrast to the FRP model, the Android system uses callback structures and functions have side effects, such as playing the music.
Although the Android code is not using an FRP model, the theoretical foundations provided by FRP make embedding the synthesized control code a straightforward task.

Nevertheless, we need to consider that libraries used by the app and the surrounding android system carry their own state, which needs to be reflected within the model. To do so, we introduce a separate stream for each interface, which we assume to carry all the necessary state. We use the input stream~\name{Sys} to receive system events and button presses, while the input stream~$ \name{MP} $ provides us the interface to the \texttt{MediaPlayer} class. Updates to the \texttt{MediaPlayer} class are provided via its control interface~$ \name{Ctrl} $. This allows us to embed the synthesized program into any larger context to manipulate the music player. We cannot use the same name for input and output here, as we would miss state changes that could be made by a component plugged after this, which is a specific result of the clear modularization utilized within CCA. This is also why we do not need a separate system output here, as we only receive signals from the system. Finally, we also utilize the input stream~\name{Tr}, which provides us with the currently selected music track.

We partition the individual properties of the specification into two
categories: assumptions~$ A $ and guarantees~$ G $. The final
specification then results from their implication:
$ \bigwedge_{\vartheta \in A} \vartheta \rightarrow \bigwedge_{\psi
  \in B} \psi $. Our specification uses the following signals,
function terms and predicates:
%
\begin{align*}
  \inames &= \{\, \name{MP}, \name{Sys},  \name{Tr}\,\} \\
  \onames &= \{\, \name{Ctrl} \,\} \\
  \cells  &= \{\, \name{Ctrl} \,\} \\
  \pterms &= \{\, \name{musicPlaying},\, \name{pauseButton},\name{playButton}, \name{leaveApp}, \name{resumeApp} \,\} \\
  \fterms &= \pterms \cup \{\, \name{pause},\, \name{play},\name{trackPos} \,\} \\
\end{align*}
%
The function~$ \name{pause}~\name{m} $ pauses a played track on the passed music player stream~$ \name{m} $. The function~$\name{play}~\name{t}~\name{p} $ plays the selected track~$ \name{t} $ at the given track position~$ \name{p} $. The track position is carried by the music player stream, and can be extracted using the function~\name{trackPos}. In our model, \name{play} resets any state that is passed by $ \name{MP} $, while \name{pause} does not. The predicate \name{musicPlaying} checks whether music is playing on a music player stream, while the remaining predicates check the respective conditions from the surrounding android system.

\para{Assumptions}
We start with straightforward assumptions about the user interface.
In our model, we assume that the pause and play buttons cannot be pressed at the same time.
Also, from the Android OS behavior, we know the app cannot leave and resume at the same time.
%
\setcounter{equation}{0}
\renewcommand{\theequation}{A\arabic{equation}}
%
\begin{align}
  &\LTLglobally \neg \big( \name{playButton}~\name{Sys} \ \wedge \ \name{pauseButton}~\name{Sys}\big) \\
  &\LTLglobally \neg \big( \name{leaveApp}~\name{Sys} \phantom{\name{o}ri} \wedge \ \name{resumeApp}~\name{Sys}\big)
\end{align}
%
Again from the Android OS, once a user has left the app it is not possible to press the play or pause buttons until the user resumes the app.
%
\begin{equation}
  \begin{array}{l}\LTLglobally \Big( \name{leaveApp}~\name{Sys} \impl \\
    \quad \hspace{2em} \big(\big(\neg \name{playButton}~\name{Sys} \wedge \neg \name{pauseButton}~\name{Sys} \big) \LTLweakuntil \name{resumeApp}~\name{Sys}\big) \Big) \end{array}
\end{equation}
%
We use the $ \name{musicPlaying} $ predicate to monitor changes according to the play and pause actions.
Technically, $ \name{musicPlaying} $ is not necessary to specify correct behavior - we could remember the music playing state on a separate looping stream.
However, the method is provided by the \texttt{MediaPlayer} interface and it helps to improve readability, which is why we also use it for our the specification. To obtain a correct behavior, we do not need to mimic the full behavior of the method's implementation. It suffices to specify the behavior with respect to the pause and play actions.
%
\begin{align}
   \LTLglobally \Big(\hspace{3.4pt}U_{\name{play}} & \to \LTLnext \big(\phantom{\neg\,} \name{musicPlaying}~\name{MP}\, \LTLweakuntil \ U_{\name{pause}} \big) \Big) \\
  \LTLglobally \Big(U_{\name{pause}}                   & \to \LTLnext \big(\neg \, \name{musicPlaying}~\name{MP}\, \LTLweakuntil \ U_{\name{play}} \hspace{3.4pt}\big) \Big)
\end{align}
%
We use $ U_{\name{play}} $ as a shortcut for $ \upd{\name{Ctrl}}{\name{play}~\name{Tr}~(\name{trackPos}~\name{MP})} $ and $ U_{\name{pause}} $ as a shortcut for $ \upd{\name{Ctrl}}{\name{pause}~\name{MP}} $.

\para{Guarantees}
To specify the desired behavior of our app under the given assumptions, we define the following \TSL guarantees.
First, whenever the user presses one of the buttons in the app, the output signal has to take the corresponding action.
This desired behavior is what necessitates the assumption that both buttons cannot be pressed at the same time.
Removing the assumption that both buttons cannot be pressed at the same time would require to relax these guarantees.
%
\setcounter{equation}{0}
\renewcommand{\theequation}{G\arabic{equation}}
%
\begin{align}
  \LTLglobally \big( \phantom{a} \name{playButton}~\name{Sys} & \to U_{\name{play}} \hspace{3.4pt} \big)  \\
  \LTLglobally \big( \name{pauseButton}~\name{Sys} & \to U_{\name{pause}} \big)
\end{align}
%
The only way that the music can be paused is either by the user leaving the app or pressing pause.
In the latter case, the music should not start playing again until either the user resumes the app or presses play.
%
\begin{align}
  \LTLglobally \big(  \hspace{47.02pt}  U_{\name{pause}} &\to \big(\, \name{leaveApp}~\name{Sys} \ \lor \ \name{pauseButton}~\name{Sys} \big)\big) \\
  \LTLglobally \big( \phantom{\name{ton}}\name{leaveApp}~\name{Sys}  & \to \big(\neg \, U_{\name{play}} \, \LTLweakuntil \ \name{resumeApp}~\name{Sys}\big) \hspace{38pt} \big)\\
   \LTLglobally \big( \name{pauseButton}~\name{Sys}                   & \to \big(\neg \, U_{\name{play}} \, \LTLweakuntil \ \name{playButton}~\name{Sys} \big) \hspace{33.6pt} \big)
\end{align}
%
The last two parts of the specification were already introduced in the motivating example.
If the music is playing, the music should pause when leaving the app and start playing again when returning to the app.
\begin{align}
  \begin{split}
    & \LTLglobally \big( \name{musicPlaying}~\name{MP} \ \wedge \ \name{leaveApp}~\name{Sys} \to \, U_{\name{pause}} \big)
  \end{split} \\
  \begin{split} \label{eq:assoonas}
    &\LTLglobally \big( \name{musicPlaying}~\name{MP} \ \wedge \ \name{leaveApp}~\name{Sys} \to  \\
     & \qquad \qquad \big((\name{pauseButton}~\name{Sys} \lor U_{\name{play}})\, \mathop{\mathcal{A}}\, \name{resumeApp}~\name{Sys} \big) \big)
  \end{split}
\end{align}
%
Note that in contrast to \cref{sec:motiv} we also test for the $ \name{event}(\name{pauseButton}) $ in guarantee~\cref{eq:assoonas}. This is necessary, since the specification would be unrealizable otherwise, as revealed by our synthesis tool. Indeed, the user may be smart enough to immediately pause the music when resuming the app, in which case it should not be played again.

\subsection{Proof of \cref{thm:tsl2ltl}}
\label[appendix]{proof:tsl2ltl}

\begin{proof}
  Assume $ \varphi_{\textit{LTL}} $ is realizable. Then there exists a
  winning strategy $ \sigma \from (2^{\pterms})^{+} \to 2^{\utermsp} $
  for the system player in the underlying LTL realizability
  game. Assume for the sake of contradiction that
  $ \varphi_{\textit{TSL}} $ is not realizable. Then there exists in
  input~$ \iota \in \inputs^{\omega} $ and a function
  assignment~$ \assign{\cdot} \from \fterms \to \functions $ such that
  for all $ \kappa \from \inputs^{+} \to \comps $ we have that
  $ \branch{\kappa}{\iota}, \iota, \nsats \varphi_{\textit{TSL}} $. We
  inductively construct the input
  sequence~$ \nu \in (2^{\pterms})^{\omega} $ and
  the computation~$ \comp \in \comps^{\omega} $ over $ t \in \dtime $ as follows:
  %
  \begin{equation*}
    \begin{array}{rcl}
      \nu(t) & = & \set{ \pterm \in \pterms \mid \eval(\comp, \iota, t, \pterm) } \\[0.5em]
      \comp(t)(\name{s}) & = & \fterm \text{, \quad  where } \fterm \text{ is the unique element} \\
             && \mbox{\ } \hspace{4em} \text{such that } \upd{\name{s}}{\fterm} \in \sigma(\nu(0) \nu(1) \ldots \nu(t))
    \end{array}
  \end{equation*}
  %
  Note that $ \fterm $ must be unique, due to the additional constraint
  %
  \begin{equation*}
    \LTLglobally \Big ( \bigwedge_{\name{s} \in \onames} \, \bigvee_{\term \in \utermsp^{\hspace{0.5pt}\name{s}}}
    \big( \term \; \wedge \bigwedge_{\term' \in
      \utermsp^{\hspace{0.5pt}\name{s}} \setminus
      \set{ \term }} \neg \, \term' \big)  \Big)
  \end{equation*}
  %
  introduced in the underapproximation. Furthermore, also note that
  $ \nu $ and $ \comp $ are well-defined, since
  $ \eval(\comp, \iota, t, \pterm) $ only considers values of
  $ \comp $ at previous times~\mbox{$ t' < t $}.
  Since~$ \varphi_{\textit{LTL}} $ is realizable, we have that
  $ \branch{\sigma}{\nu}, \nu \vDash \varphi_{\textit{LTL}} $\footnote{For a
    suitable definition for the semantics of $ \vDash $ consider for
    example~\cite{Schewe:2013}.}, but at the same time
  \mbox{$ \comp, \iota \nsats \varphi_{\textit{TSL}} $} by the
  unrealizability of $ \varphi_{\textit{TSL}} $. We show that this is
  contradictory via a structural induction over the structure
  of~$ \varphi_{\textit{TSL}} $ for all $ t \in \dtime $:
  %
  \begin{itemize}

  \item Case $ \varphi_{\textit{TSL}} = \pterm $:
    %
    \begin{align*}
      & \comp, \iota, t \sats \pterm \\ \Leftrightarrow~ &
      \eval(\comp, \iota, t, \pterm) \\ \Leftrightarrow~ &
      \pterm \in \nu(t) \\ \Leftrightarrow~ &
      \branch{\sigma}{\nu}, \nu, t \vDash \textsc{SyntacticConversion}(\pterm)
    \end{align*}

  \item Case $ \varphi_{\textit{TSL}} = \upd{\name{s}}{\fterm} $:
    %
    \begin{align*}
      & \comp, \iota, t \sats \upd{\name{s}}{\fterm} \\ \Leftrightarrow~&
      \comp(t)(\name{s}) = \fterm \\ \Leftrightarrow~&
      \upd{\name{s}}{\fterm} \in \sigma(\nu(0) \nu(1) \ldots \nu(t)) \\ \Leftrightarrow~&
      \branch{\sigma}{\nu}, \nu, t \vDash \textsc{SyntacticConversion}(\upd{\name{s}}{\fterm})
    \end{align*}

  \item Case $ \varphi_{\textit{TSL}} = \neg \psi $:
    %
    \begin{align*}
      & \comp, \iota, t \sats \neg \psi \\ \Leftrightarrow~ &
      \comp, \iota, t \nsats \psi \\ \overset{IH}{\Leftrightarrow}~ &
      \branch{\sigma}{\nu}, \nu, t \nvDash \textsc{SyntacticConversion}(\psi) \\ \Leftrightarrow~&
      \branch{\sigma}{\nu}, \nu, t \vDash \textsc{SyntacticConversion}(\neg \psi)
    \end{align*}

  \item Case $ \varphi_{\textit{TSL}} = \vartheta \wedge \psi $:
    %
    \begin{align*}
      & \comp, \iota, t \sats \vartheta \wedge \psi \\ \Leftrightarrow~ &
      \comp, \iota, t \sats \vartheta \ \wedge \ \comp, \iota, t \sats \psi \\ \overset{IH}{\Leftrightarrow}~ &
      \branch{\sigma}{\nu}, \nu, t \vDash \textsc{SyntacticConversion}(\vartheta) \ \wedge \branch{\sigma}{\nu}, \nu, t \vDash \textsc{SyntacticConversion}(\psi) \\ \Leftrightarrow~&
      \branch{\sigma}{\nu}, \nu, t \vDash \textsc{SyntacticConversion}(\vartheta \wedge \psi)
    \end{align*}

  \item Case $ \varphi_{\textit{TSL}} = \LTLnext \psi $:
    %
    \begin{align*}
      & \comp, \iota, t \sats \LTLnext \psi \\ \Leftrightarrow~ &
      \comp, \iota, t+1 \sats \psi \\ \overset{IH}{\Leftrightarrow}~ &
      \branch{\sigma}{\nu}, \nu, t+1 \vDash \textsc{SyntacticConversion}(\psi) \\ \Leftrightarrow~&
      \branch{\sigma}{\nu}, \nu, t \vDash \textsc{SyntacticConversion}(\LTLnext \psi)
    \end{align*}


  \item Case $ \varphi_{\textit{TSL}} = \vartheta \LTLuntil \psi $:
    %
    \begin{align*}
      & \comp, \iota, t \sats \vartheta \LTLuntil \psi \\ \Leftrightarrow~ &
\exists t'' \geq t. \ \
         \forall t \leq t' < t''. \ \ \comp, \iota, t' \sats \vartheta \ \,
         \wedge \ \, \comp, \iota, t'' \sats \psi                                                                             \\ \overset{IH}{\Leftrightarrow}~ &
\exists t'' \geq t. \ \
         \forall t \leq t' < t''. \\ & \mbox{\qquad} \branch{\sigma}{\nu}, \nu, t' \vDash \textsc{SyntacticConversion}(\vartheta) \ \,
         \wedge \\ & \mbox{\qquad} \branch{\sigma}{\nu}, \nu, t'' \vDash \textsc{SyntacticConversion}(\psi)                                                                             \\ \Leftrightarrow~ &
      \branch{\sigma}{\nu}, \nu, t \vDash \textsc{SyntacticConversion}(\vartheta \LTLuntil \psi) \hfill \qedhere
    \end{align*}

  \end{itemize}
\end{proof}

\subsection{Extended proof of \cref{thm:decidability}}
\label[appendix]{proof:decidability}

\begin{proof}
  We give a reduction from the Post Correspondence Problem~(PCP).
  Given two finite lists $ w_{0}w_{1}\ldots w_{n} $ and
  $ v_{0}v_{1}\ldots v_{n} $ of equal length, each containing $ n $ finite
  words over some finite alphabet~$ \Sigma $, is there some finite
  sequence $ i_{0}i_{1}\ldots i_{k} \in \nats^{*} $ such that
  $ w_{i_{0}}w_{i_{1}}\ldots w_{i_{k}} = v_{i_{0}} v_{i_{1}} \ldots
  v_{i_{k}} $. The problem is undecidable~\cite{post1946}.
  We reduce PCP to the realizability question of TSL, where we
  translate an arbitrary instance of PCP to a TSL formula~$ \varphi $,
  which is realizable if and only if there is a solution to the PCP
  instance. To this end, let $ n \in \nats $,
  $ w_{0}w_{1}\ldots w_{n} \in \Sigma^{*} $ and
  $ v_{0}v_{1}\ldots v_{n} \in \Sigma^{*} $ be given.  We fix
  $ \pnames = \set{ \name{p} } $ for some unary predicate~$ p $,
  \mbox{$ \fnames = \pnames \cup \Sigma \cup \set{ \name{X} } $}, where every
  $ \name{f} \in \Sigma $ corresponds to a unary function and
  $ \name{X} $ corresponds to a $ 0 $-nary function,
  $ \inames = \emptyset $, $ \onames = \set{ \name{A}, \name{B} } $, and
  $ \cells = \onames $. We define~$ \varphi $ via:
  %
  \begin{equation*}
    \begin{array}{l}
    \varphi =  \Big( \, \upd{\name{A}}{\const{X}} \ \wedge \ \upd{\name{B}}{\const{X}} \, \Big)
    \ \wedge \\[0.2em] \phantom{\varphi \ =} \LTLnext\, \LTLglobally
    \Big( \bigvee_{j=0}^{n} \big( \upd{\name{A}}{\mu(w_{j},\name{A})} \, \wedge \,
    \upd{\name{B}}{\mu(v_{j},\name{B})} \big) \Big) \ \wedge  \\[0.2em]
      \phantom{\varphi \ =} \LTLnext \, \LTLnext \, \LTLfinally \Big( \, \name{p}~\name{A}
    \, \leftrightarrow \, \name{p}~\name{B} \, \Big)
    \end{array}
  \end{equation*}
  %
  where
  $ \mu(x_{0}x_{1}\ldots x_{m},s) = x_{0}~(x_{1}~(\ldots (x_{m}~s) \ldots
  )) $.

  Intuitively, we first assign the signals $ \name{A} $ and
  $ \name{B} $ a constant base value. Then, from the next time step
  on, we have to pick pairs $ (w_{j},v_{j}) $ in every time step. Our
  choice is stored in the signals~$ \name{A} $ and $ \name{B} $,
  respectively. Finally, we check that the constructed sequences of
  function applications are equal at some point in time, where we use the
  universally quantified predicate~$ \name{p} $ to check for equality.

  \medskip

  \noindent The TSL formula~$ \varphi $ is realizable if and only if
  there is an index sequence~$ i_{0}i_{1}\ldots i_{k} $ such that
  $ w_{i_{0}}w_{i_{1}}\ldots w_{i_{k}} = v_{i_{0}} v_{i_{1}} \ldots
  v_{i_{k}} $:

  \smallskip

  ``$ \Rightarrow $'': Assume that $ \varphi $ is realizable, i.e.,
  there is some strategy~$ \sigma \from \inputs^{*} \to \comp $ that
  satisfies $ \varphi $ for
  $ \iota = \emptyset^{\hspace{0.5pt}\omega} $ and all possible
  choices of $ \assign{\cdot} \from \fnames \to \functions $. We fix
  $ \assign{\textit{init}_{\name{A}}} =
  \assign{\textit{init}_{\name{B}}} = \assign{\name{X}} = \varepsilon
  $ and $ \assign{x} \from \Sigma^{*} \to \Sigma^{*} $ with
  $ \assign{x}~w = wx $ for all $ w \in \Sigma^{*} $ and
  $ x \in \Sigma $. We do not fix any assignment to $ \name{p}
  $. Nevertheless, by the realizability of~$ \varphi $, there is a
  position~$ m > 1 $ at which
  $ \name{p}~\name{A} \leftrightarrow \name{p}~\name{A} $ is
  satisfied, independent of the predicate assigned to $ \name{p} $. We
  obtain that
  $ \eval(\branch{\sigma}{\iota},\iota,m,\name{A}) =
  \eval(\branch{\sigma}{\iota},\iota,m,\name{B}) $, \linebreak since
  otherwise there would be a predicate that detects the difference. As
  there is no other influence
  on~$ \branch{\sigma}{\iota} \in \comps^{\omega} $, depending on the
  choice of $ \name{p} $, we obtain that the semantics of
  $ \varrho_{\hspace{-1pt}\langle \hspace{-1pt}\cdot
    \hspace{-1pt}\rangle \hspace{-1pt}, \comp, \iota} \in
  \outputs^{\hspace{0.5pt}\omega} $
  %
  \begin{eqnarray*}
   \varrho_{\hspace{-1pt}\langle \hspace{-1pt}\cdot
    \hspace{-1pt}\rangle \hspace{-1pt}, \comp, \iota} & = & \set{ \varrho_{\hspace{-1pt}\langle \hspace{-1pt}\cdot
    \hspace{-1pt}\rangle \hspace{-1pt}, \comp, \iota}(0)(\name{A}) \mapsto a_{0}, \,
                                                             \varrho_{\hspace{-1pt}\langle \hspace{-1pt}\cdot
    \hspace{-1pt}\rangle \hspace{-1pt}, \comp, \iota}(0)(\name{B}) \mapsto b_{0} } \\
    & & \set{ \varrho_{\hspace{-1pt}\langle \hspace{-1pt}\cdot
    \hspace{-1pt}\rangle \hspace{-1pt}, \comp, \iota}(1)(\name{A})
    \mapsto a_{1}, \, \varrho_{\hspace{-1pt}\langle \hspace{-1pt}\cdot
    \hspace{-1pt}\rangle \hspace{-1pt}, \comp, \iota}(1)(\name{B}) \mapsto b_{1} } \\
    & & \ldots
  \end{eqnarray*}
  %
  are well defined (even without fixing $ \name{p} $) and induce the
  sequences $ a_{0}a_{1}\ldots \in \Sigma^{\omega} $ and
  $ b_{0}b_{1}\ldots \in \Sigma^{\omega} $. First, we observe that
  $ a_{0} = a_{1} = b_{0} = b_{1} = \varepsilon $ by the definition of
  $ \varphi $ and the choice of the initialization functions and
  $ \assign{\cdot} $. From the choice of $ \assign{\cdot} $, we also
  obtain that $ a_{t} = a_{t-1}w_{i_{t}} $ and
  $ b_{t} = b_{t-1}v_{i_{t}} $ for every $ t > 1 $ and some
  $ 0 \leq i_{t} \leq n $. A simple induction shows that every
  $ a_{t} = w_{i_{2}}w_{i_{3}}\ldots w_{i_{t}} $ and
  $ b_{t} = v_{i_{2}}v_{i_{3}} \ldots v_{i_{t}} $ for every $ t > 1
  $. It follows that
  $ w_{i_{2}}w_{i_{3}}\ldots w_{i_{m-1}} = v_{i_{2}}v_{i_{3}} \ldots
  v_{i_{m}} $ from equality of $ a_{m} $ and $ b_{m} $ at
  position~$ m $. This concludes this part of the proof, since we have
  that $ i_{2}i_{3}\ldots i_{m} $ is a solution for the PCP instance.

  \smallskip

  ``$ \Leftarrow $'': Now, assume that there is a
  solution~$ i_{0}i_{1}\ldots i_{k} $ to the PCP instance.
  Since~$ \inames = \emptyset $ it suffices to
  construct the computation~$ \comp = c_{0}c_{1}\ldots $ with
  $ c_{0}(\name{A}) = c_{0}(\name{B}) = \const{X} $ and for all
  $ t > 0 $:
  $ c_{t}(\name{A}) = \mu(w_{(t-1) \!\!\mod (k+1)}, \name{A}) $ and
  $ c_{t}(\name{B}) = \mu(w_{(t-1) \!\!\mod (k+1)}, \name{B}) $. It is
  straightforward to see that $ \comp $ satisfies
  $ \mbox{\ensuremath{\upd{\name{A}}{\name{X}}}} $,
  $ \mbox{\ensuremath{\upd{\name{B}}{\name{X}}}}$ and
  %
  \begin{equation*}
     \LTLnext \, \LTLglobally \, \Big( \bigvee_{j=0}^{n}
    \big( \upd{\name{A}}{\mu(w_{j},\name{A})} \, \wedge \,
    \upd{\name{B}}{\mu(v_{j},\name{B})} \big) \Big).
  \end{equation*}
  %
  Thus, it just remains to argue that $ \comp $ satisfies
  $ \LTLnext \, \LTLnext \, \LTLfinally ( \name{p}~\name{A} \,
  \leftrightarrow \, \name{p}~\name{B} ) $. To this end, let
  $ j_{0}j_{1}\ldots = (i_{0}i_{1}\ldots i_{k})^{\omega} $. Then a
  simple induction shows that
  %
  \begin{equation*}
    \varrho_{\hspace{-1pt}\langle \hspace{-1pt}\cdot
      \hspace{-1pt}\rangle \hspace{-1pt}, \comp, \iota}
    (t)(\name{A}) =
    \eval(\comp,\iota,t,\mu(w_{j_{0}}w_{j_{1}}\ldots w_{j_{t-2}},\name{X}))
  \end{equation*}
  %
  and
  $ \varrho_{\hspace{-1pt}\langle \hspace{-1pt}\cdot
    \hspace{-1pt}\rangle \hspace{-1pt}, \comp, \iota}(t)(\name{B}) =
  \eval(\comp,\iota, t,\mu(v_{j_{0}}v_{j_{1}}\ldots
  v_{j_{t-2}},\name{X})) $ for all $ t > 1 $,
  $ \iota = \emptyset^{\hspace{0.5pt}\omega} $, and all choices of
  $ \assign{\cdot} $. Now, consider that especially for $ t = k+2 $ we
  have that
  %
  \begin{eqnarray*}
    w_{j_{0}}w_{j_{1}}\ldots w_{j_{t-2}} & =
    & w_{i_{0}}w_{i_{1}}\ldots  w_{i_{k}} \\
    & = & v_{i_{0}}v_{i_{1}}\ldots v_{i_{k}} \\
    & = &  v_{j_{0}}v_{j_{1}}\ldots v_{j_{t-2}}
  \end{eqnarray*}
  %
  and, thus, also
  $  \varrho_{\hspace{-1pt}\langle \hspace{-1pt}\cdot
    \hspace{-1pt}\rangle \hspace{-1pt}, \comp, \iota}(k+2)(\name{A}) =  \varrho_{\hspace{-1pt}\langle \hspace{-1pt}\cdot
    \hspace{-1pt}\rangle \hspace{-1pt}, \comp, \iota}(k+2)(\name{B}) $,
  independent of the choice of $ \assign{\cdot} $. As this implies
  that  $ p(\varrho_{\hspace{-1pt}\langle \hspace{-1pt}\cdot
    \hspace{-1pt}\rangle \hspace{-1pt}, \comp, \iota}(k+2)(\name{A})) = p(\varrho_{\hspace{-1pt}\langle \hspace{-1pt}\cdot
    \hspace{-1pt}\rangle \hspace{-1pt}, \comp, \iota}(k+2)(\name{B}))
  $ for any unary predicate~$ p \in \predicates $, it proves that
  $ \comp $ satisfies
  $ \name{p}~\name{A} \, \leftrightarrow \, \name{p}~\name{B} $ at
  position $ k+2 $. Hence, the computation~$ \comp $ also satisfies
  $ \LTLnext \, \LTLnext \, \LTLfinally ( \name{p}~\name{A} \,
  \leftrightarrow \, \name{p}~\name{B} ) $, which concludes the
  proof.
\end{proof}
%
\noindent Note that we do not use any inputs to encode the PCP instance in the proof
above. As a consequence, it immediately follows that the
satisfiability problem of \TSL is undecidable as well.

\subsection{Other Benchmarks}
\label[appendix]{apx:benchmarks}

\subsubsection{Escalator}

As an illustrative example of the logic, we build up the
specification of an escalator controller. To this end, we first commit
to the following physical model.
To interact with the environment the escalator is equipped with a
motor, which moves its steps either up, down, or is turned off. To
observe the environment, it additionally can read from two sensors, at
the bottom and at the top, that reliably detect whenever somebody
enters or exists. To program the controller, we receive the inputs of
the sensors via two input signals: $ \name{bottom} $ and
$ \name{top} $. The steps are controlled via the output signal:
$ \name{steps} $.
We start with a simple, non-reactive version that continuously moves
up. This behavior is described via:
%
\begin{equation*}
  \LTLglobally \; \upd{\name{steps}}{\name{MOVEUP}()}
\end{equation*}
%
Note that we use $ \name{MOVEUP}() $ as a $ 0 $-nary function here, as
the command does not depend on any other given signal. However, for a
concrete system, we instead would replace it with the matching library
call (including possible static parameters), that passes the movement
command to the motor's driver. Nevertheless, reflecting these details here
would go beyond the scope of this illustration.

Next, let us make our escalator reactive, in the sense that it only
moves up, if there is actually somebody on it, using it.
A first specification looks as follows:
%
\begin{equation*}
  \begin{array}{l}
    \LTLglobally \Big( \big( (\name{enterEvent}~\name{bottom} \wedge \neg \name{exitEvent}~\name{top})
    \leftrightarrow \upd{\name{steps}}{\name{MOVEUP}()} \big) \wedge \mbox{\ } \\
    \phantom{\LTLglobally \Big(} \big( (\name{exitEvent}~\name{top} \wedge \neg \name{enterEvent}~\name{bottom})
    \leftrightarrow \upd{\name{steps}}{\name{STOP}()} \big) \Big)
  \end{array}
\end{equation*}
%
While this specification is realizable, it also turns out to be
incomplete. The elevator stops as soon as the first user leaves it at
the top, but there still may be other users behind.

A short inspection reveals that we cannot solve this problem via
purely reacting to the given inputs. Instead, we need to introduce a
counter (used as an internal signal), which keeps track of the users
on the escalator. Hence, we change the previous specification into
%
\begin{equation*}
  \begin{array}{l}
    \LTLglobally \Big( \big( (\name{enterEvent}~\name{bottom} \wedge \neg \name{exitEvent}~\name{top}) \leftrightarrow \upd{\name{users}}{\name{(+)}~\name{users}~\name{1}} \big) \wedge \mbox{\ } \\
    \phantom{\LTLglobally \Big(} \big( (\name{exitEvent}~\name{top} \wedge \neg \name{enterEvent}~\name{bottom})
    \leftrightarrow \upd{\name{users}}{ \name{(+)}~\name{users}~(\name{-1})} \big) \Big)
  \end{array}
\end{equation*}
%
With this change, it only remains to start and stop the escalator
whenever the number of users toggles between zero and non-zero:
%
\begin{equation*}
  \begin{array}{l}
    \LTLglobally \Big( \big( (\name{(==)}~\name{users}~\name{0} \ \wedge \ \LTLnext \neg \, (\name{(==)}~\name{users}~\name{0})) \leftrightarrow \LTLnext \, \upd{\name{steps}}{\name{MOVEUP}()} \big) \\
    \phantom{\LTLglobally \Big( } \big( ( \neg (\name{(==)}~\name{users}~\name{0}) \wedge \LTLnext \, (\name{(==)}~\name{users}~\name{0})) \leftrightarrow \LTLnext \; \upd{\name{steps}}{\name{STOP}()} \ \ \ \big) \Big)
  \end{array}
\end{equation*}
%
In conjunction with the previous part, we obtain a complete
description of a reactive escalator, which we can use to synthesize a
respective FPP program. The result not only satisfies the
specification, but also is immediately executable on the controller
(after compilation).

Note that we can easily extend our specification, by adding even more
properties. This is possible without changing the previous parts at
all.  For example, consider an alarm, that is activated whenever there
are too many users on the escalator.  Another variant would be a smart
version, which moves up and down, triggered by the entrance point of
the next user whenever the system is idle. For the sake of
illustration of these statements, we give a possible realization of
the second variant in the sequel.

We present the specification of a smart escalator, which is able to
move into both directions: up and down. Thereby, the direction is
determined by the entrance point of the first user entering the
escalator when it is empty. If the escalator already moves into a
specific direction, we ignore enter and exit events into the opposite
direction until it stopped again. The specification
$ \varphi = \LTLglobally \psi \rightarrow \bigwedge_{j=0}^{7}
\LTLglobally \vartheta_{j} $ consists of:
%
\begin{align*}
  \psi = \ & \neg \big(\name{(==)}~\name{steps}~\name{MOVEDOWN}() \ \wedge \ \name{(==)}~\name{steps}~\name{MOVEUP}()\big) \wedge \\
           & \neg \big(\name{enterEvent}~\name{top} \wedge  \name{exitEvent}~\name{top} \big) \wedge \mbox{\,}\\
           & \neg \big(\name{enterEvent}~\name{bottom} \wedge  \name{exitEvent}~\name{bottom} \big) \\[0.4em]
  \vartheta_{0} =\ & \big( (\name{enterEvent}~\name{bottom} \wedge \neg \, \name{exitEvent}~\name{top} \wedge \mbox{\ }
                     \neg (\name{(==)}~\name{steps}~\name{MOVEDOWN}()) ) \vee \mbox{ \ }\\
                   & \phantom{\big(} (\name{enterEvent}~\name{top} \wedge \neg \, \name{exitEvent}~\name{bottom} \wedge \mbox{\ }
                     \neg (\name{(==)}~\name{steps}~\name{MOVEUP}())) \big) \\
                   & \leftrightarrow \upd{\name{users}}{\name{(+)}~\name{users}~\name{1}}  \\
  \vartheta_{1} =\ & \big((\name{exitEvent}~\name{top} \wedge \neg \, \name{enterEvent}~\name{bottom} \wedge \mbox{\ }
                     \neg (\name{(==)}~\name{steps}~\name{MOVEDOWN}())) \vee \mbox{\ } \\
                   & \phantom{\big(}(\name{exitEvent}~\name{bottom} \wedge \neg \, \name{enterEvent}~\name{top} \wedge \mbox{\ }
                     \neg (\name{(==)}~\name{steps}~\name{MOVEUP}()))\big) \\
                   & \leftrightarrow \upd{\name{users}}{\name{(+)}~\name{users}~\name{(-1)}}  \\[0.4em]
  \vartheta_{2} =\ & \upd{\name{steps}}{\name{MOVEUP}()} \\
                   & \rightarrow \ \name{(==)}~\name{users}~\name{0} \; \wedge \; \name{enterEvent}~\name{bottom} \\[0.4em]
  \vartheta_{3} =\ & \upd{\name{steps}}{\name{MOVEDOWN}()} \\
                   & \rightarrow \ \name{(==)}~\name{users}~\name{0} \; \wedge \; \name{enterEvent}~\name{top} \\[0.4em]
  \vartheta_{4} =\ & \name{(==)}~\name{users}~\name{0} \; \wedge \; \name{enterEvent}~\name{bottom} \wedge \neg \name{enterEvent}~\name{top} \\
                   & \rightarrow \upd{\name{steps}}{\name{MOVEUP}()}  \\[0.4em]
  \vartheta_{5} =\ & \name{(==)}~\name{users}~\name{0} \; \wedge \; \name{enterEvent}~\name{top} \wedge \mbox{\ } \\
                   & \neg \name{enterEvent}~\name{bottom} \rightarrow \upd{\name{steps}}{\name{MOVEDOWN}()}  \\[0.4em]
  \vartheta_{6} =\ & \name{(==)}~\name{users}~\name{0} \; \wedge \; \name{enterEvent}~\name{top} \wedge \name{enterEvent}~\name{bottom} \\
                   & \rightarrow \big(\upd{\name{steps}}{\name{MOVEUP}()} \vee \upd{\name{steps}}{\name{MOVEDOWN}()} \big)  \\[0.4em]
  \vartheta_{7} =\ & \neg (\name{(==)}~\name{users}~\name{0}) \wedge \LTLnext \! \big(\name{(==)}~\name{users}~\name{0} \wedge
                     \neg \name{enterEvent}~\name{top} \wedge \neg \name{enterEvent}~\name{bottom}\big) \\
                   & \leftrightarrow \upd{\name{steps}}{\name{STOP}} \\
\end{align*}

\subsubsection{FRPZoo}

A special role is played by the
FRPZoo benchmark set, which refers to a standard online benchmark
suite, designed to compare FRP library language designs~\cite{FRPzoo}.
The specification separates between three different behaviors, given
as scenarios 0, 5, and 10. In every scenario, two buttons can be
clicked: a ``click count'' button, which counts the number of clicks,
and a ``toggle'' button, which toggles the enable/disable state of the ``click
count'' button. The value of the counter is displayed via some
output interface. The three scenarios differ with respect to the exact
conditions of when the counter is updated, reset or displayed.

\subsubsection{Haskell-TORCS}

The Haskell-TORCS benchmarks synthesize controllers for an autonomous vehicle.
Our specifications build upon the example of the Haskell-TORCS bindings for building FRP controllers~\cite{SCAV2017} in The Open Race Car Simulator (TORCS)~\cite{torcs}. The bindings are also used to run the synthesized implementations within the simulator.
Autonomous vehicles use limited sensor data about the environment (\eg the distance to nearest obstacle) to control actuators in the car (\eg the steering wheel).
The Haskell-TORCS set of benchmarks synthesize a controller from \TSL specifications where the sensors and actuators are the input and output signals respectively.
The functions used in the \TSL specifications for the Haskell-TORCS benchmarks, for example ``slowDown'' or ``turnLeft'', are implemented after the controller synthesis process.
In this way, we obtain a guarantee on the larger behavior of the system, while still allowing numerically sensitive, data level manipulations, to be optimized as required by the application.

The first ``simple'' Haskell-TORCS controller combines simple functions without states.
The ``advanced'' controllers included more detailed planning behavior when approaching a turn.
The specifications are also modular, in the sense that control of the steering wheel and control of the gears are given separate specifications, and combined into a single FRP program after synthesis.

\subsubsection{Counters}

The benchmarks of \cref{table:results2} consider a user interface that
allows to increment either one or two counters while at the same time
ensuring that each counter stays in range, as restricted by
predicate. The benchmarks are inspired by examples of the Reactive
Banana FRP library~\cite{reactivebanana}, which can be found under
\url{https://wiki.haskell.org/Reactive-banana/Examples}. The
benchmarks work on a small interface, over which the user can
increment or decrement one or two counters.In the two counter case,
he or she can switch between the counters using an additional toggling
input.

In the simple variant, the system only must ensure that each counter
stays in range, where it is allowed to ignore an increment or
decrement request from the user, if this could produce a counter value
out of range. In the more advanced variant, we also assume the
existence of a graphical interface, where the inputs are realized via
pressing different buttons. Here, the synthesizer can only avoid an
increment of the user by disabling a button before pressing it could
lead to an out of range update. For all benchmarks, the synthesizer
must utilize purity of the increment / decrement operation to test
early enough whether the environment could produce an out of range
value to prevent this either by ignoring the input or by disabling the
corresponding buttons.
