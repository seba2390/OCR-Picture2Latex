\tikzstyle{block} = [rectangle, draw, fill=gray!20,
    text width=7em, text centered, rounded corners, minimum height=3em, node distance=2cm]
\tikzstyle{block3} = [rectangle, draw, fill=gray!20,
    text width=7em, text centered, rounded corners, minimum height=3em, node distance=1.5cm]
\tikzstyle{block2} = [rectangle, draw,
    text width=7em, text centered, rounded corners, minimum height=3em, node distance=2cm]
\tikzstyle{input} = [rectangle, draw, fill=green!20,
    text width=7em, text centered, rounded corners, minimum height=2em, node distance=2cm]
\tikzstyle{line} = [draw, -latex']
\tikzstyle{line} = [draw, -latex']
\tikzstyle{line2} = [draw, dotted, -latex']

\tikzstyle{main} = [rectangle, draw, fill=green!30, text width=3.5em, inner sep=0.5em, text centered, rounded corners=4, minimum height=2.5em]

\tikzstyle{main2} = [rectangle, draw, fill=green!40!blue!15!, text width=2.5em, inner sep=0.5em, text centered, minimum height=2em]

\tikzstyle{main3} = [rectangle, draw, fill=blue!10!green!20, minimum width=1.4em, inner sep=0.5em, text centered, rounded corners=4, minimum height=1.4em]

\tikzstyle{conv} = [rectangle, draw, fill=blue!15, inner sep=0.5em, text centered,
 minimum height=2em,,anchor=center, drop shadow]
\tikzstyle{conv2} = [rectangle, draw, fill=white!15, inner sep=0.5em, text centered,
 minimum height=2em,,anchor=center]

\begin{figure}[t]
  \centering
  \begin{tikzpicture}[scale=1.1]
    \clip (-1.1,-4.28) rectangle (10.4,2.56);

    \node[opacity=0] at (10.7,0) {x};

    \node[main,fill=yellow!15] at (0.2,0.55) (tsl) {\large TSL};

    \node[main] at (0.2,-0.55) (tam) {\large CFM};

    \node[conv,anchor=west,minimum height=5.5em,inner sep=0pt,minimum width=13.5em] at (2.1,0) (synt) {
      \begin{tikzpicture}[>=stealth, anchor=center,minimum height=1em,minimum width=1em]
        \node at (0,0) {Synthesis};

        \node[main2] at (-0.8,-1) (ltl) {LTL};
        \node[main2,text width=3em] at (0.8,-1) (mealy) {Circuit};

        \draw[thick] (-2,-0.35) -- (-1,-0.35);
        \draw[thick] (-1,-0.35) edge[dashed] (1,-0.35);
        \draw[thick] (1,-0.35) edge[->] (2,-0.35);
        \draw (-1.7,-0.35) -- (-1.7,-1);
        \draw (-1.7,-1) edge[->] (ltl.west);
        \draw (ltl.east) edge[->] (mealy.west);
        \draw (mealy.east) -- (1.7,-1);
        \draw (1.7,-1) edge[->] (1.7,-0.35);
      \end{tikzpicture}
    };

    \node[draw,circle,fill=yellow!15] at ($ (synt.north) + (-1.9,0.5) $) (bound) {\large $ n $};

    \node[conv,minimum height=2.5em,minimum width=13.5em] at ($ (synt) + (0,-1.6) $) (trans) {
      \begin{tabular}{c}
        FRP Translator
      \end{tabular}
    };

    \node[conv,minimum height=2.5em,minimum width=13.5em] at ($ (trans) + (0,-1.1) $) (inte) {
        Project Context
        \qquad \qquad \
    };

    \node[conv,anchor=west,minimum height=2.5em] at (2.1,-3.8) (compiler) {
      \begin{tabular}{c}
        ~Compiler~ \\[-0.2em]
      \end{tabular}
    };

    \node[conv2,anchor=north,dashed] at ($ (synt.north east) + (2,0) $) (syntt) {
      \begin{tabular}{c}
        LTL \\ Synthesis Tool \\[-0.2em]
      \end{tabular}
    };

    \node at (0,0.55) (syntoutr) {};
    \node at (0,-0.4) (syntoutl) {};
    \node at (0,-2.65) (transout) {};
    \node at (0,-4.8) (compout) {};
    \node at (0,-4.7) (compin) {};

    \node[main,fill=orange!20,text width=4em] at ($ (synt.north) + (0.15,1.1) $) (counter) {
      \begin{tabular}{c}
        Counter \\
        Strategy
      \end{tabular}
    };

    \node[conv,minimum height=2.5em,minimum width=7.5em,anchor=center] at
    (tsl |- counter) (refi) {Refinement};

    \node[main,fill=red!20,text width=6em,xshift=20]  at (counter -| syntt) (unre) {\textbf{unrealizable}};

    \node[conv2,anchor=east,dashed,yshift=21,xshift=-8] at (unre.east |- trans) (pattern) {
      \begin{tabular}{c}
        Design Pattern: \\ Arrow | Applicative \\[-0.2em]
      \end{tabular}
    };

    \node[main] at (tsl |- inte) (frp) {\large FRP};

    \node[main,anchor=east] at (compiler -| inte.east) (exe) {\large EXE};

    \node[conv2,anchor=south east,dashed] at (compiler.south -| unre.east) (terms) {
      \begin{tabular}{c}
        Function \& Predicate\\ Implementations
      \end{tabular}
    };


    \node[conv2,dashed] at ($ (terms -| unre) + (0,1.7) $) (lib) {
      \begin{tabular}{c}
        FRP Library
      \end{tabular}
    };

    \node[main3] at ($ (inte) + (1.3,0) $) (module) {};

    \draw[->,>=stealth, very thick] ($ (tsl) + (-1.4,0) $) -- (tsl);
    \draw[line width=0.5em,color=black!40,->] (tsl) -- (tsl.east -| synt.west);
    \draw[line width=0.5em,color=black!40,->] (bound) -| ($ (synt.north) + (-1.15,0) $);
    \draw[line width=0.5em,color=black!40,->] (tam -| synt.west) to node[above,xshift=5] {\large \color{green!50!black} \scheck} (tam);
    \draw[line width=0.5em,color=black!40,->] (counter) to node[above,xshift=-5] {\color{black} non-spurious} (unre);
    \draw[line width=0.5em,color=black!40,->] (counter) to node[above,xshift=5] {\color{black} spurious} (refi);
    \draw[line width=0.5em,color=black!40,->] (refi.south -| tsl) -- (tsl);
    \draw[line width=0.5em,color=black!40,->] (frp -| inte.west) -- (frp);
    \draw[line width=0.5em,color=black!40,->] (tam) |- (trans);
    \draw[line width=0.5em,color=black!40,->] (frp) |- (compiler.west);

    \draw[line width=0.5em,color=black!40,->] (synt.north -| counter) to node[right,xshift=5,yshift=-1] {\large \color{red!60!black} \serror} (counter);

    \draw[line width=0.5em,color=black!40,->] (compiler.east |- exe) -- (exe);

    \draw[line width=0.5em,color=black!40,->] ($ (trans.south) + (1.3,0) $) -- (module.north);

    \draw[->,>=stealth, very thick] (syntt) -- (syntt -| synt.east);
    \draw[->,>=stealth, very thick] ($ (pattern.south west) + (0.6,0) $) |- ($ (trans.east) + (0,-0.1) $);
    \draw[->,>=stealth, very thick] (lib.south) |- ($ (inte.east) + (0,0.125) $);
    \draw[->,>=stealth, very thick] (terms.north) |- ($ (inte.east) + (0,-0.125) $);

  \end{tikzpicture}
  \caption{The TSL synthesis procedure uses a modular design. Each
    step takes input from the previous step as well as interchangeable
    modules (dashed boxes).}
  \label{fig:system}
\end{figure}
