\documentclass{llncs}

\newcommand{\cL}{{\cal L}}
\let\terms\undefined

%\usepackage[OT1,T1]{fontenc}

\usepackage[numbers,sort&compress]{natbib}
\renewcommand{\bibfont}{\footnotesize}
%\usepackage{cite}
%\usepackage{mystyle}
%%%%%%%%%%%%%%%%%%%%%%%%%%%%%%%%%%%%
\makeatletter

\usepackage{etex}

%%% Review %%%

\usepackage{zref-savepos}

\newcounter{mnote}%[page]
\renewcommand{\themnote}{p.\thepage\;$\langle$\arabic{mnote}$\rangle$}

\def\xmarginnote{%
  \xymarginnote{\hskip -\marginparsep \hskip -\marginparwidth}}

\def\ymarginnote{%
  \xymarginnote{\hskip\columnwidth \hskip\marginparsep}}

\long\def\xymarginnote#1#2{%
\vadjust{#1%
\smash{\hbox{{%
        \hsize\marginparwidth
        \@parboxrestore
        \@marginparreset
\footnotesize #2}}}}}

\def\mnoteson{%
\gdef\mnote##1{\refstepcounter{mnote}\label{##1}%
  \zsavepos{##1}%
  \ifnum20432158>\number\zposx{##1}%
  \xmarginnote{{\color{blue}\bf $\langle$\arabic{mnote}$\rangle$}}% 
  \else
  \ymarginnote{{\color{blue}\bf $\langle$\arabic{mnote}$\rangle$}}%
  \fi%
}
  }
\gdef\mnotesoff{\gdef\mnote##1{}}
\mnoteson
\mnotesoff








%%% Layout %%%

% \usepackage{geometry} % override layout
% \geometry{tmargin=2.5cm,bmargin=m2.5cm,lmargin=3cm,rmargin=3cm}
% \setlength{\pdfpagewidth}{8.5in} % overrides default pdftex paper size
% \setlength{\pdfpageheight}{11in}

\newlength{\mywidth}

%%% Conventions %%%

% References
\newcommand{\figref}[1]{Fig.~\ref{#1}}
\newcommand{\defref}[1]{Definition~\ref{#1}}
\newcommand{\tabref}[1]{Table~\ref{#1}}
% general
%\usepackage{ifthen,nonfloat,subfigure,rotating,array,framed}
\usepackage{framed}
%\usepackage{subfigure}
\usepackage{subcaption}
\usepackage{comment}
%\specialcomment{nb}{\begingroup \noindent \framed\textbf{n.b.\ }}{\endframed\endgroup}
%%\usepackage{xtab,arydshln,multirow}
% topcaption defined in xtab. must load nonfloat before xtab
%\PassOptionsToPackage{svgnames,dvipsnames}{xcolor}
\usepackage[svgnames,dvipsnames]{xcolor}
%\definecolor{myblue}{rgb}{.8,.8,1}
%\definecolor{umbra}{rgb}{.8,.8,.5}
%\newcommand*\mybluebox[1]{%
%  \colorbox{myblue}{\hspace{1em}#1\hspace{1em}}}
\usepackage[all]{xy}
%\usepackage{pstricks,pst-node}
\usepackage{tikz}
\usetikzlibrary{positioning,matrix,through,calc,arrows,fit,shapes,decorations.pathreplacing,decorations.markings,decorations.text}

\tikzstyle{block} = [draw,fill=blue!20,minimum size=2em]

% allow prefix to scope name
\tikzset{%
	prefix node name/.code={%
		\tikzset{%
			name/.code={\edef\tikz@fig@name{#1 ##1}}
		}%
	}%
}


\@ifpackagelater{tikz}{2013/12/01}{
	\newcommand{\convexpath}[2]{
		[create hullcoords/.code={
			\global\edef\namelist{#1}
			\foreach [count=\counter] \nodename in \namelist {
				\global\edef\numberofnodes{\counter}
				\coordinate (hullcoord\counter) at (\nodename);
			}
			\coordinate (hullcoord0) at (hullcoord\numberofnodes);
			\pgfmathtruncatemacro\lastnumber{\numberofnodes+1}
			\coordinate (hullcoord\lastnumber) at (hullcoord1);
		}, create hullcoords ]
		($(hullcoord1)!#2!-90:(hullcoord0)$)
		\foreach [evaluate=\currentnode as \previousnode using \currentnode-1,
		evaluate=\currentnode as \nextnode using \currentnode+1] \currentnode in {1,...,\numberofnodes} {
			let \p1 = ($(hullcoord\currentnode) - (hullcoord\previousnode)$),
			\n1 = {atan2(\y1,\x1) + 90},
			\p2 = ($(hullcoord\nextnode) - (hullcoord\currentnode)$),
			\n2 = {atan2(\y2,\x2) + 90},
			\n{delta} = {Mod(\n2-\n1,360) - 360}
			in 
			{arc [start angle=\n1, delta angle=\n{delta}, radius=#2]}
			-- ($(hullcoord\nextnode)!#2!-90:(hullcoord\currentnode)$) 
		}
	}
}{
	\newcommand{\convexpath}[2]{
		[create hullcoords/.code={
			\global\edef\namelist{#1}
			\foreach [count=\counter] \nodename in \namelist {
				\global\edef\numberofnodes{\counter}
				\coordinate (hullcoord\counter) at (\nodename);
			}
			\coordinate (hullcoord0) at (hullcoord\numberofnodes);
			\pgfmathtruncatemacro\lastnumber{\numberofnodes+1}
			\coordinate (hullcoord\lastnumber) at (hullcoord1);
		}, create hullcoords ]
		($(hullcoord1)!#2!-90:(hullcoord0)$)
		\foreach [evaluate=\currentnode as \previousnode using \currentnode-1,
		evaluate=\currentnode as \nextnode using \currentnode+1] \currentnode in {1,...,\numberofnodes} {
			let \p1 = ($(hullcoord\currentnode) - (hullcoord\previousnode)$),
			\n1 = {atan2(\x1,\y1) + 90},
			\p2 = ($(hullcoord\nextnode) - (hullcoord\currentnode)$),
			\n2 = {atan2(\x2,\y2) + 90},
			\n{delta} = {Mod(\n2-\n1,360) - 360}
			in 
			{arc [start angle=\n1, delta angle=\n{delta}, radius=#2]}
			-- ($(hullcoord\nextnode)!#2!-90:(hullcoord\currentnode)$) 
		}
	}
}

% circle around nodes

% typsetting math
\usepackage{qsymbols,amssymb,mathrsfs}
\usepackage{amsmath}
\usepackage[standard,thmmarks]{ntheorem}
\theoremstyle{plain}
\theoremsymbol{\ensuremath{_\vartriangleleft}}
\theorembodyfont{\itshape}
\theoremheaderfont{\normalfont\bfseries}
\theoremseparator{}
\newtheorem{Claim}{Claim}
\newtheorem{Subclaim}{Subclaim}
\theoremstyle{nonumberplain}
\theoremheaderfont{\scshape}
\theorembodyfont{\normalfont}
\theoremsymbol{\ensuremath{_\blacktriangleleft}}
\newtheorem{Subproof}{Proof}

\theoremnumbering{arabic}
\theoremstyle{plain}
\usepackage{latexsym}
\theoremsymbol{\ensuremath{_\Box}}
\theorembodyfont{\itshape}
\theoremheaderfont{\normalfont\bfseries}
\theoremseparator{}
\newtheorem{Conjecture}{Conjecture}

\theorembodyfont{\upshape}
\theoremprework{\bigskip\hrule}
\theorempostwork{\hrule\bigskip}
\newtheorem{Condition}{Condition}%[section]


%\RequirePckage{amsmath} loaded by empheq
\usepackage[overload]{empheq} % no \intertext and \displaybreak
%\usepackage{breqn}

\let\iftwocolumn\if@twocolumn
\g@addto@macro\@twocolumntrue{\let\iftwocolumn\if@twocolumn}
\g@addto@macro\@twocolumnfalse{\let\iftwocolumn\if@twocolumn}

%\empheqset{box=\mybluebox}
%\usepackage{mathtools}      % to polish math typsetting, loaded
%                                % by empeq
\mathtoolsset{showonlyrefs=false,showmanualtags}
\let\underbrace\LaTeXunderbrace % adapt spacing to font sizes
\let\overbrace\LaTeXoverbrace
\renewcommand{\eqref}[1]{\textup{(\refeq{#1})}} % eqref was not allowed in
                                       % sub/super-scripts
\newtagform{brackets}[]{(}{)}   % new tags for equations
\usetagform{brackets}
% defined commands:
% \shortintertext{}, dcases*, \cramped, \smashoperator[]{}

\usepackage[Smaller]{cancel}
\renewcommand{\CancelColor}{\color{Red}}
%\newcommand\hcancel[2][black]{\setbox0=\hbox{#2}% colored horizontal cross
%  \rlap{\raisebox{.45\ht0}{\color{#1}\rule{\wd0}{1pt}}}#2}



\usepackage{graphicx,psfrag}
\graphicspath{{figure/}{image/}} % Search path of figures

% for tabular
\usepackage{diagbox} % \backslashbox{}{} for slashed entries
%\usepackage{threeparttable} % threeparttable, \tnote{},
                                % tablenotes, and \item[]
%\usepackage{colortab} % \cellcolor[gray]{0.9},
%\rowcolor,\columncolor,
%\usepackage{colortab} % \LCC \gray & ...  \ECC \\

% typesetting codes
%\usepackage{maple2e} % need to use \char29 for ^
\usepackage{algorithm2e}
\usepackage{listings} 
\lstdefinelanguage{Maple}{
  morekeywords={proc,module,end, for,from,to,by,while,in,do,od
    ,if,elif,else,then,fi ,use,try,catch,finally}, sensitive,
  morecomment=[l]\#,
  morestring=[b]",morestring=[b]`}[keywords,comments,strings]
\lstset{
  basicstyle=\scriptsize,
  keywordstyle=\color{ForestGreen}\bfseries,
  commentstyle=\color{DarkBlue},
  stringstyle=\color{DimGray}\ttfamily,
  texcl
}
%%% New fonts %%%
\DeclareMathAlphabet{\mathpzc}{OT1}{pzc}{m}{it}
\usepackage{upgreek} % \upalpha,\upbeta, ...
%\usepackage{bbold}   % blackboard math
\usepackage{dsfont}  % \mathds

%%% Macros for multiple definitions %%%

% example usage:
% \multi{M}{\boldsymbol{#1}}  % defines \multiM
% \multi ABC.                 % defines \MA \MB and \MC as
%                             % \boldsymbol{A}, \boldsymbol{B} and
%                             % \boldsymbol{C} respectively.
% 
%  The last period '.' is necessary to terminate the macro expansion.
%
% \multi*{M}{\boldsymbol{#1}} % defines \multiM and \M
% \M{A}                       % expands to \boldsymbol{A}

\def\multi@nostar#1#2{%
  \expandafter\def\csname multi#1\endcsname##1{%
    \if ##1.\let\next=\relax \else
    \def\next{\csname multi#1\endcsname}     
    %\expandafter\def\csname #1##1\endcsname{#2}
    \expandafter\newcommand\csname #1##1\endcsname{#2}
    \fi\next}}

\def\multi@star#1#2{%
  \expandafter\def\csname #1\endcsname##1{#2}
  \multi@nostar{#1}{#2}
}

\newcommand{\multi}{%
  \@ifstar \multi@star \multi@nostar}

%%% new alphabets %%%

\multi*{rm}{\mathrm{#1}}
\multi*{mc}{\mathcal{#1}}
\multi*{op}{\mathop {\operator@font #1}}
% \multi*{op}{\operatorname{#1}}
\multi*{ds}{\mathds{#1}}
\multi*{set}{\mathcal{#1}}
\multi*{rsfs}{\mathscr{#1}}
\multi*{pz}{\mathpzc{#1}}
\multi*{M}{\boldsymbol{#1}}
\multi*{R}{\mathsf{#1}}
\multi*{RM}{\M{\R{#1}}}
\multi*{bb}{\mathbb{#1}}
\multi*{td}{\tilde{#1}}
\multi*{tR}{\tilde{\mathsf{#1}}}
\multi*{trM}{\tilde{\M{\R{#1}}}}
\multi*{tset}{\tilde{\mathcal{#1}}}
\multi*{tM}{\tilde{\M{#1}}}
\multi*{baM}{\bar{\M{#1}}}
\multi*{ol}{\overline{#1}}

\multirm  ABCDEFGHIJKLMNOPQRSTUVWXYZabcdefghijklmnopqrstuvwxyz.
\multiol  ABCDEFGHIJKLMNOPQRSTUVWXYZabcdefghijklmnopqrstuvwxyz.
\multitR   ABCDEFGHIJKLMNOPQRSTUVWXYZabcdefghijklmnopqrstuvwxyz.
\multitd   ABCDEFGHIJKLMNOPQRSTUVWXYZabcdefghijklmnopqrstuvwxyz.
\multitset ABCDEFGHIJKLMNOPQRSTUVWXYZabcdefghijklmnopqrstuvwxyz.
\multitM   ABCDEFGHIJKLMNOPQRSTUVWXYZabcdefghijklmnopqrstuvwxyz.
\multibaM   ABCDEFGHIJKLMNOPQRSTUVWXYZabcdefghijklmnopqrstuvwxyz.
\multitrM   ABCDEFGHIJKLMNOPQRSTUVWXYZabcdefghijklmnopqrstuvwxyz.
\multimc   ABCDEFGHIJKLMNOPQRSTUVWXYZabcdefghijklmnopqrstuvwxyz.
\multiop   ABCDEFGHIJKLMNOPQRSTUVWXYZabcdefghijklmnopqrstuvwxyz.
\multids   ABCDEFGHIJKLMNOPQRSTUVWXYZabcdefghijklmnopqrstuvwxyz.
\multiset  ABCDEFGHIJKLMNOPQRSTUVWXYZabcdefghijklmnopqrstuvwxyz.
\multirsfs ABCDEFGHIJKLMNOPQRSTUVWXYZabcdefghijklmnopqrstuvwxyz.
\multipz   ABCDEFGHIJKLMNOPQRSTUVWXYZabcdefghijklmnopqrstuvwxyz.
\multiM    ABCDEFGHIJKLMNOPQRSTUVWXYZabcdefghijklmnopqrstuvwxyz.
\multiR    ABCDEFGHIJKL NO QR TUVWXYZabcd fghijklmnopqrstuvwxyz.
\multibb   ABCDEFGHIJKLMNOPQRSTUVWXYZabcdefghijklmnopqrstuvwxyz.
\multiRM   ABCDEFGHIJKLMNOPQRSTUVWXYZabcdefghijklmnopqrstuvwxyz.
\newcommand{\RRM}{\R{M}}
\newcommand{\RRP}{\R{P}}
\newcommand{\RRe}{\R{e}}
\newcommand{\RRS}{\R{S}}
%%% new symbols %%%

%\newcommand{\dotgeq}{\buildrel \textstyle  .\over \geq}
%\newcommand{\dotleq}{\buildrel \textstyle  .\over \leq}
\newcommand{\dotleq}{\buildrel \textstyle  .\over {\smash{\lower
      .2ex\hbox{\ensuremath\leqslant}}\vphantom{=}}}
\newcommand{\dotgeq}{\buildrel \textstyle  .\over {\smash{\lower
      .2ex\hbox{\ensuremath\geqslant}}\vphantom{=}}}

\DeclareMathOperator*{\argmin}{arg\,min}
\DeclareMathOperator*{\argmax}{arg\,max}

%%% abbreviations %%%

% commands
\newcommand{\esm}{\ensuremath}

% environments
\newcommand{\bM}{\begin{bmatrix}}
\newcommand{\eM}{\end{bmatrix}}
\newcommand{\bSM}{\left[\begin{smallmatrix}}
\newcommand{\eSM}{\end{smallmatrix}\right]}
\renewcommand*\env@matrix[1][*\c@MaxMatrixCols c]{%
  \hskip -\arraycolsep
  \let\@ifnextchar\new@ifnextchar
  \array{#1}}



% sets of number
\newqsymbol{`N}{\mathbb{N}}
\newqsymbol{`R}{\mathbb{R}}
\newqsymbol{`P}{\mathbb{P}}
\newqsymbol{`Z}{\mathbb{Z}}

% symbol short cut
\newqsymbol{`|}{\mid}
% use \| for \parallel
\newqsymbol{`8}{\infty}
\newqsymbol{`1}{\left}
\newqsymbol{`2}{\right}
\newqsymbol{`6}{\partial}
\newqsymbol{`0}{\emptyset}
\newqsymbol{`-}{\leftrightarrow}
\newqsymbol{`<}{\langle}
\newqsymbol{`>}{\rangle}

%%% new operators / functions %%%

\newcommand{\sgn}{\operatorname{sgn}}
\newcommand{\Var}{\op{Var}}
\newcommand{\diag}{\operatorname{diag}}
\newcommand{\erf}{\operatorname{erf}}
\newcommand{\erfc}{\operatorname{erfc}}
\newcommand{\erfi}{\operatorname{erfi}}
\newcommand{\adj}{\operatorname{adj}}
\newcommand{\supp}{\operatorname{supp}}
\newcommand{\E}{\opE\nolimits}
\newcommand{\T}{\intercal}
% requires mathtools
% \abs,\abs*,\abs[<size_cmd:\big,\Big,\bigg,\Bigg etc.>]
\DeclarePairedDelimiter\abs{\lvert}{\rvert} 
\DeclarePairedDelimiter\norm{\lVert}{\rVert}
\DeclarePairedDelimiter\ceil{\lceil}{\rceil}
\DeclarePairedDelimiter\floor{\lfloor}{\rfloor}
\DeclarePairedDelimiter\Set{\{}{\}}
\newcommand{\imod}[1]{\allowbreak\mkern10mu({\operator@font mod}\,\,#1)}

%%% new formats %%%
\newcommand{\leftexp}[2]{{\vphantom{#2}}^{#1}{#2}}


% non-floating figures that can be put inside tables
\newenvironment{nffigure}[1][\relax]{\vskip \intextsep
  \noindent\minipage{\linewidth}\def\@captype{figure}}{\endminipage\vskip \intextsep}

\newcommand{\threecols}[3]{
\hbox to \textwidth{%
      \normalfont\rlap{\parbox[b]{\textwidth}{\raggedright#1\strut}}%
        \hss\parbox[b]{\textwidth}{\centering#2\strut}\hss
        \llap{\parbox[b]{\textwidth}{\raggedleft#3\strut}}%
    }% hbox 
}

\newcommand{\reason}[2][\relax]{
  \ifthenelse{\equal{#1}{\relax}}{
    \left(\text{#2}\right)
  }{
    \left(\parbox{#1}{\raggedright #2}\right)
  }
}

\newcommand{\marginlabel}[1]
{\mbox[]\marginpar{\color{ForestGreen} \sffamily \small \raggedright\hspace{0pt}#1}}


% up-tag an equation
\newcommand{\utag}[2]{\mathop{#2}\limits^{\text{(#1)}}}
\newcommand{\uref}[1]{(#1)}


% Notation table

\newcommand{\Hline}{\noalign{\vskip 0.1in \hrule height 0.1pt \vskip
    0.1in}}
  
\def\Malign#1{\tabskip=0in
  \halign to\columnwidth{
    \ensuremath{\displaystyle ##}\hfil
    \tabskip=0in plus 1 fil minus 1 fil
    &
    \parbox[t]{0.8\columnwidth}{##}
    \tabskip=0in
    \cr #1}}


%%%%%%%%%%%%%%%%%%%%%%%%%%%%%%%%%%%%%%%%%%%%%%%%%%%%%%%%%%%%%%%%%%%
% MISCELLANEOUS

% Modification from braket.sty by Donald Arseneau
% Command defined is: \extendvert{ }
% The "small versions" use fixed-size brackets independent of their
% contents, whereas the expand the first vertical line '|' or '\|' to
% envelop the content
\let\SavedDoubleVert\relax
\let\protect\relax
{\catcode`\|=\active
  \xdef\extendvert{\protect\expandafter\noexpand\csname extendvert \endcsname}
  \expandafter\gdef\csname extendvert \endcsname#1{\mskip-5mu \left.%
      \ifx\SavedDoubleVert\relax \let\SavedDoubleVert\|\fi
     \:{\let\|\SetDoubleVert
       \mathcode`\|32768\let|\SetVert
     #1}\:\right.\mskip-5mu}
}
\def\SetVert{\@ifnextchar|{\|\@gobble}% turn || into \|
    {\egroup\;\mid@vertical\;\bgroup}}
\def\SetDoubleVert{\egroup\;\mid@dblvertical\;\bgroup}

% If the user is using e-TeX with its \middle primitive, use that for
% verticals instead of \vrule.
%
\begingroup
 \edef\@tempa{\meaning\middle}
 \edef\@tempb{\string\middle}
\expandafter \endgroup \ifx\@tempa\@tempb
 \def\mid@vertical{\middle|}
 \def\mid@dblvertical{\middle\SavedDoubleVert}
\else
 \def\mid@vertical{\mskip1mu\vrule\mskip1mu}
 \def\mid@dblvertical{\mskip1mu\vrule\mskip2.5mu\vrule\mskip1mu}
\fi

%%%%%%%%%%%%%%%%%%%%%%%%%%%%%%%%%%%%%%%%%%%%%%%%%%%%%%%%%%%%%%%%

\makeatother

%%%%%%%%%%%%%%%%%%%%%%%%%%%%%%%%%%%%

\usepackage{ctable}
\usepackage{fouridx}
%\usepackage{calc}
\usepackage{framed}
\usetikzlibrary{positioning,matrix}

\usepackage{paralist}
%\usepackage{refcheck}
\usepackage{enumerate}

\usepackage[normalem]{ulem}
\newcommand{\Ans}[1]{\uuline{\raisebox{.15em}{#1}}}



\numberwithin{equation}{section}
\makeatletter
\@addtoreset{equation}{section}
\renewcommand{\theequation}{\arabic{section}.\arabic{equation}}
\@addtoreset{Theorem}{section}
\renewcommand{\theTheorem}{\arabic{section}.\arabic{Theorem}}
\@addtoreset{Lemma}{section}
\renewcommand{\theLemma}{\arabic{section}.\arabic{Lemma}}
\@addtoreset{Corollary}{section}
\renewcommand{\theCorollary}{\arabic{section}.\arabic{Corollary}}
\@addtoreset{Example}{section}
\renewcommand{\theExample}{\arabic{section}.\arabic{Example}}
\@addtoreset{Remark}{section}
\renewcommand{\theRemark}{\arabic{section}.\arabic{Remark}}
\@addtoreset{Proposition}{section}
\renewcommand{\theProposition}{\arabic{section}.\arabic{Proposition}}
\@addtoreset{Definition}{section}
\renewcommand{\theDefinition}{\arabic{section}.\arabic{Definition}}
\@addtoreset{Claim}{section}
\renewcommand{\theClaim}{\arabic{section}.\arabic{Claim}}
\@addtoreset{Subclaim}{Theorem}
\renewcommand{\theSubclaim}{\theTheorem\Alph{Subclaim}}
\makeatother

\newcommand{\Null}{\op{Null}}
%\newcommand{\T}{\op{T}\nolimits}
\newcommand{\Bern}{\op{Bern}\nolimits}
\newcommand{\odd}{\op{odd}}
\newcommand{\even}{\op{even}}
\newcommand{\Sym}{\op{Sym}}
\newcommand{\si}{s_{\op{div}}}
\newcommand{\sv}{s_{\op{var}}}
\newcommand{\Wtyp}{W_{\op{typ}}}
\newcommand{\Rco}{R_{\op{CO}}}
\newcommand{\Tm}{\op{T}\nolimits}
\newcommand{\JGK}{J_{\op{GK}}}

\newcommand{\diff}{\mathrm{d}}

\newenvironment{lbox}{
  \setlength{\FrameSep}{1.5mm}
  \setlength{\FrameRule}{0mm}
  \def\FrameCommand{\fboxsep=\FrameSep \fcolorbox{black!20}{white}}%
  \MakeFramed {\FrameRestore}}%
{\endMakeFramed}

\newenvironment{ybox}{
	\setlength{\FrameSep}{1.5mm}
	\setlength{\FrameRule}{0mm}
  \def\FrameCommand{\fboxsep=\FrameSep \fcolorbox{black!10}{yellow!8}}%
  \MakeFramed {\FrameRestore}}%
{\endMakeFramed}

\newenvironment{gbox}{
	\setlength{\FrameSep}{1.5mm}
\setlength{\FrameRule}{0mm}
  \def\FrameCommand{\fboxsep=\FrameSep \fcolorbox{black!10}{green!8}}%
  \MakeFramed {\FrameRestore}}%
{\endMakeFramed}

\newenvironment{bbox}{
	\setlength{\FrameSep}{1.5mm}
\setlength{\FrameRule}{0mm}
  \def\FrameCommand{\fboxsep=\FrameSep \fcolorbox{black!10}{blue!8}}%
  \MakeFramed {\FrameRestore}}%
{\endMakeFramed}

\newenvironment{yleftbar}{%
  \def\FrameCommand{{\color{yellow!20}\vrule width 3pt} \hspace{10pt}}%
  \MakeFramed {\advance\hsize-\width \FrameRestore}}%
 {\endMakeFramed}

\newcommand{\tbox}[2][\relax]{
 \setlength{\FrameSep}{1.5mm}
  \setlength{\FrameRule}{0mm}
  \begin{ybox}
    \noindent\underline{#1:}\newline
    #2
  \end{ybox}
}

\newcommand{\pbox}[2][\relax]{
  \setlength{\FrameSep}{1.5mm}
 \setlength{\FrameRule}{0mm}
  \begin{gbox}
    \noindent\underline{#1:}\newline
    #2
  \end{gbox}
}

\newcommand{\gtag}[1]{\text{\color{green!50!black!60} #1}}
\let\labelindent\relax
\usepackage{enumitem}

%%%%%%%%%%%%%%%%%%%%%%%%%%%%%%%%%%%%
% fix subequations
% http://tex.stackexchange.com/questions/80134/nesting-subequations-within-align
%%%%%%%%%%%%%%%%%%%%%%%%%%%%%%%%%%%%

\usepackage{etoolbox}

% let \theparentequation use the same definition as equation
\let\theparentequation\theequation
% change every occurence of "equation" to "parentequation"
\patchcmd{\theparentequation}{equation}{parentequation}{}{}

\renewenvironment{subequations}[1][]{%              optional argument: label-name for (first) parent equation
	\refstepcounter{equation}%
	%  \def\theparentequation{\arabic{parentequation}}% we patched it already :)
	\setcounter{parentequation}{\value{equation}}%    parentequation = equation
	\setcounter{equation}{0}%                         (sub)equation  = 0
	\def\theequation{\theparentequation\alph{equation}}% 
	\let\parentlabel\label%                           Evade sanitation performed by amsmath
	\ifx\\#1\\\relax\else\label{#1}\fi%               #1 given: \label{#1}, otherwise: nothing
	\ignorespaces
}{%
	\setcounter{equation}{\value{parentequation}}%    equation = subequation
	\ignorespacesafterend
}

\newcommand*{\nextParentEquation}[1][]{%            optional argument: label-name for (first) parent equation
	\refstepcounter{parentequation}%                  parentequation++
	\setcounter{equation}{0}%                         equation = 0
	\ifx\\#1\\\relax\else\parentlabel{#1}\fi%         #1 given: \label{#1}, otherwise: nothing
}

% hyperlink
\PassOptionsToPackage{breaklinks,letterpaper,hyperindex=true,backref=false,bookmarksnumbered,bookmarksopen,linktocpage,colorlinks,linkcolor=BrickRed,citecolor=OliveGreen,urlcolor=Blue,pdfstartview=FitH}{hyperref}
\usepackage{hyperref}

% makeindex style
\newcommand{\indexmain}[1]{\textbf{\hyperpage{#1}}}

\renewcommand\UrlFont{\color{blue}\rmfamily}

\begin{document}

\title{Temporal Stream Logic: \\[0.2em] Synthesis beyond the
  Bools\thanks{Supported by the European Research Council (ERC) Grant
    OSARES (No.\ 683300), the Collaborative Research Center (TRR 248,
    389792660), and the National Science Foundation (NSF) Grant
    CCF-1302327.}}

\author{
       Bernd Finkbeiner\inst{1}
  \and Felix Klein\inst{1}
  \and Ruzica Piskac\inst{2}
  \and Mark Santolucito\inst{2}
}

\institute{Saarland University, Saarbrücken, Germany
  \and
Yale University, New Haven, USA}

\maketitle

\begin{abstract}
  Reactive systems that operate in environments with complex data,
  such as mobile apps or embedded controllers with many sensors, are
  difficult to synthesize.  Synthesis tools usually fail for such
  systems because the state space resulting from the discretization
  of the data is too large.  We introduce TSL, a new temporal logic
  that separates control and data. We provide a CEGAR-based synthesis
  approach for the construction of implementations that are guaranteed
  to satisfy a TSL specification for all possible instantiations of
  the data processing functions.  TSL provides an attractive trade-off
  for synthesis. On the one hand, synthesis from TSL, unlike synthesis
  from standard temporal logics, is undecidable in general. On the
  other hand, however, synthesis from TSL is scalable, because it is
  independent of the complexity of the \mbox{handled} data.  Among
  other benchmarks, we have successfully synthesized a music player
  Android app and a controller for an autonomous vehicle in the Open
  Race Car Simulator (TORCS).
\end{abstract}

\section{Introduction}
\label{sec:intro}
% !TEX root = ../arxiv.tex

Unsupervised domain adaptation (UDA) is a variant of semi-supervised learning \cite{blum1998combining}, where the available unlabelled data comes from a different distribution than the annotated dataset \cite{Ben-DavidBCP06}.
A case in point is to exploit synthetic data, where annotation is more accessible compared to the costly labelling of real-world images \cite{RichterVRK16,RosSMVL16}.
Along with some success in addressing UDA for semantic segmentation \cite{TsaiHSS0C18,VuJBCP19,0001S20,ZouYKW18}, the developed methods are growing increasingly sophisticated and often combine style transfer networks, adversarial training or network ensembles \cite{KimB20a,LiYV19,TsaiSSC19,Yang_2020_ECCV}.
This increase in model complexity impedes reproducibility, potentially slowing further progress.

In this work, we propose a UDA framework reaching state-of-the-art segmentation accuracy (measured by the Intersection-over-Union, IoU) without incurring substantial training efforts.
Toward this goal, we adopt a simple semi-supervised approach, \emph{self-training} \cite{ChenWB11,lee2013pseudo,ZouYKW18}, used in recent works only in conjunction with adversarial training or network ensembles \cite{ChoiKK19,KimB20a,Mei_2020_ECCV,Wang_2020_ECCV,0001S20,Zheng_2020_IJCV,ZhengY20}.
By contrast, we use self-training \emph{standalone}.
Compared to previous self-training methods \cite{ChenLCCCZAS20,Li_2020_ECCV,subhani2020learning,ZouYKW18,ZouYLKW19}, our approach also sidesteps the inconvenience of multiple training rounds, as they often require expert intervention between consecutive rounds.
We train our model using co-evolving pseudo labels end-to-end without such need.

\begin{figure}[t]%
    \centering
    \def\svgwidth{\linewidth}
    \input{figures/preview/bars.pdf_tex}
    \caption{\textbf{Results preview.} Unlike much recent work that combines multiple training paradigms, such as adversarial training and style transfer, our approach retains the modest single-round training complexity of self-training, yet improves the state of the art for adapting semantic segmentation by a significant margin.}
    \label{fig:preview}
\end{figure}

Our method leverages the ubiquitous \emph{data augmentation} techniques from fully supervised learning \cite{deeplabv3plus2018,ZhaoSQWJ17}: photometric jitter, flipping and multi-scale cropping.
We enforce \emph{consistency} of the semantic maps produced by the model across these image perturbations.
The following assumption formalises the key premise:

\myparagraph{Assumption 1.}
Let $f: \mathcal{I} \rightarrow \mathcal{M}$ represent a pixelwise mapping from images $\mathcal{I}$ to semantic output $\mathcal{M}$.
Denote $\rho_{\bm{\epsilon}}: \mathcal{I} \rightarrow \mathcal{I}$ a photometric image transform and, similarly, $\tau_{\bm{\epsilon}'}: \mathcal{I} \rightarrow \mathcal{I}$ a spatial similarity transformation, where $\bm{\epsilon},\bm{\epsilon}'\sim p(\cdot)$ are control variables following some pre-defined density (\eg, $p \equiv \mathcal{N}(0, 1)$).
Then, for any image $I \in \mathcal{I}$, $f$ is \emph{invariant} under $\rho_{\bm{\epsilon}}$ and \emph{equivariant} under $\tau_{\bm{\epsilon}'}$, \ie~$f(\rho_{\bm{\epsilon}}(I)) = f(I)$ and $f(\tau_{\bm{\epsilon}'}(I)) = \tau_{\bm{\epsilon}'}(f(I))$.

\smallskip
\noindent Next, we introduce a training framework using a \emph{momentum network} -- a slowly advancing copy of the original model.
The momentum network provides stable, yet recent targets for model updates, as opposed to the fixed supervision in model distillation \cite{Chen0G18,Zheng_2020_IJCV,ZhengY20}.
We also re-visit the problem of long-tail recognition in the context of generating pseudo labels for self-supervision.
In particular, we maintain an \emph{exponentially moving class prior} used to discount the confidence thresholds for those classes with few samples and increase their relative contribution to the training loss.
Our framework is simple to train, adds moderate computational overhead compared to a fully supervised setup, yet sets a new state of the art on established benchmarks (\cf \cref{fig:preview}).


\section{Motivating Example}
\label{sec:motiv}
\newcommand{\applink}{\url{https://play.google.com/store/apps/details?id=com.mark.myapplication}.}

To demonstrate the utility of our method, we synthesized a music player Android app\footnote{\applink} from a \TSL specification.
A major challenge in developing Android apps is the temporal behavior of an app through the \textit{Android lifecycle}~\cite{Shan16}.
The Android lifecycle describes how an app should handle being paused, when moved to the background, coming back into focus, or being terminated.
In particular, \textit{resume and restart errors} are commonplace and difficult to detect and correct~\cite{Shan16}.
Our music player app demonstrates a situation in which a resume and restart error could be unwittingly introduced when programming by hand, but is avoided by providing a specification.
We only highlight the key parts of this example here to give an intuition of \TSL, leaving a more in-depth exposition to \cref{apx:musicspec}.

Our music player app utilizes the Android music player library~(\name{MP}), as well as its control interface~(\name{Ctrl}). It pauses any playing music when moved to the background (for instance if a call is received), and continues playing the currently selected track~(\name{Tr}) at the last track position when the app is resumed.
In the Android system~(\name{Sys}), the \texttt{leaveApp} method is called whenever the app moves to the background, while the \texttt{resumeApp} method is called when the app is brought back to the foreground. To avoid confusion between pausing music and pausing the app, we use \texttt{leaveApp} and \texttt{resumeApp} in place of the Android methods onPause and onResume.
A programmer might manually write code for this as shown on the left in \cref{fig:smallcode}.

The behavior of this can be directly described in \TSL as shown on the right in \cref{fig:smallcode}.
Even eliding a formal introduction of the notation for now, the specification closely matches the textual specification.
First, when the user leaves the app and the music is playing, the music pauses.
Likewise for the second part, when the user resumes the app, the music starts playing again.

\begin{figure}[t]
\vspace{-1em}
\begin{minipage}{.42\textwidth}
  \vspace{-0.8em}
\begin{lstlisting}
Sys.leaveApp()
  if (MP.musicPlaying())
    Ctrl.pause();
\end{lstlisting}
\vspace{-1em}
\begin{lstlisting}
Sys.resumeApp() {
  pos = MP.trackPos();
  Ctrl.play(Tr,pos);
}
\end{lstlisting}
\vspace{-0.8em}
\end{minipage}%
\vrule{}%
\begin{minipage}{.59\textwidth}
\vspace{-0.8em}
\begin{align*}
& \name{ALWAYS} \; \Big(\name{leaveApp} \ \, \name{Sys} \; \wedge \; \name{musicPlaying} \ \, \name{MP} \\[-0.5em]
& \quad \hspace{2.5em} \impl \upd{\name{Ctrl}}{\const{pause}} \Big) \\[0.8em]
& \name{ALWAYS} \; \Big(\name{resumeApp} \ \, \name{Sys}  \\[-0.5em]
& \quad \hspace{2.5em} \impl  \upd{\name{Ctrl}}{\name{play} \ \, \name{Tr} \ \, (\name{trackPos} \ \, \name{MP})} \Big)
\end{align*}
\vspace{-0.8em}
\end{minipage}
\vspace{-0.5em}
\caption{Sample code and specification for the music player app.}
\label{fig:smallcode}
\end{figure}
%
However, assume we want to change the behavior so that the music only plays on resume when the music had been playing before leaving the app in the first place.
In the manually written program, this new functionality requires an additional variable~\texttt{wasPlaying} to keep track of the music state.
Managing the state requires multiple changes in the code as shown on the left in \cref{fig:bigcode}.
The required code changes include: a conditional in the \texttt{resumeApp} method, setting \texttt{wasPlaying} appropriately in two places in \texttt{leaveApp}, and providing an initial value.
Although a small example, it demonstrates how a minor change in functionality may require wide-reaching code changes.
In addition, this change introduces a globally scoped variable, which then might accidentally be set or read elsewhere.
%
In contrast, it is a simple matter to change the TSL specification to reflect this new functionality.
Here, we only update one part of the specification to say that if the user leaves the app and the music is playing, the music has to play again as soon as the app resumes.

\begin{figure}[t]
\begin{minipage}{.44\textwidth}
\vspace{-0.8em}
\begin{lstlisting}
bool wasPlaying = false;
\end{lstlisting}
\vspace{-0.5em}
\begin{lstlisting}
Sys.leaveApp()
  if (MP.musicPlaying()) {
    wasPlaying = true;
    Ctrl.pause();
  }
  else
    wasPlaying = false;
\end{lstlisting}
\vspace{-0.5em}
\begin{lstlisting}
Sys.resumeApp()
  if (wasPlaying) {
    pos = MP.trackPos();
    Ctrl.play(Tr,pos);
  }
\end{lstlisting}
\vspace{-0.8em}
\end{minipage}%
\vrule{}%
\begin{minipage}{.57\textwidth}
\begin{align*}
& \,\name{ALWAYS} \; \Big( (\name{leaveApp} \ \, \name{Sys} \; \wedge \ \name{musicPlaying} \ \, \name{MP} \\
& \,\quad \hspace{2.5em} \impl \upd{\name{Ctrl}}{\const{pause}} ) \\[0.5em]
& \,\quad \hspace{2.5em} \; \wedge \, (\upd{\name{Ctrl}}{\name{play} \ \, \name{Tr} \ (\name{trackPos} \ \, \name{MP})} \ \\
& \,\quad \hspace{2.5em} \phantom{\impl} \ \ \name{AS\_SOON\_AS} \ \ \name{resumeApp} \ \, \name{Sys} ) \Big)
\end{align*}
\end{minipage}
\vspace{-0.5em}
\caption{The effect of a minor change in functionality on code versus a specification.}
\label{fig:bigcode}
\end{figure}
%
Synthesis allows us to specify a temporal behavior without worrying about the implementation details.
In this example, writing the specification in \TSL has eliminated the need of an additional state variable, similarly to a higher order \texttt{map} eliminating the need for an iteration variable.
However, in more complex examples the benefits compound, as \TSL provides a modular interface to specify behaviors, offloading the management of multiple interconnected temporal behaviors from the user to the synthesis engine.


\section{Preliminaries}
\label{sec:prelim}
%\documentclass[main]{subfiles}

\begin{document}

\section{Preliminaries}
\label{sec:preliminaries}
%\paragraph{Notation} 
\noindent
For $n \in \N$, we denote $[n] := \{1,\ldots,n\}$ and the vector with all ones as $\1_n \in \R^n$.
%\todo{Any vector $x \in \R^n$ is a column vector and its transpose is denoted by $x^T$. The entries of $x \in \R^n$ will be $x_1, \ldots, x_n$. Inequalities like $x \geq 0$ abbreviate the statement $\forall i \in [n] \, : \, x_i \geq 0$. For $i \in [n]$, we will use $e_i \in \R^n$ to denote the $i$-th standard basis vector (with a $1$ in its $i$-th entry and $0$'s anywhere else).} 
 
%\\
%\todo{When working with sets $\{S^i\}_{i=1}^N$, we denote $\displaystyle \bigtimes_{i=1}^N S^i := S^1 \times \ldots \times S^N$.} \\


\subsection{Multiplayer Games} 
A multiplayer game $G$ specifies (a) the number of players $N \in \N, N \geq 2,$ (b) a set of pure strategies $S^i = [m_i]$ for each player~$i$ where $m_i \in \N, m_i \geq 2,$ and (c) the utility payoffs for each player~$i$ given as a function $u_i: S = S^1 \times \ldots \times S^N \longrightarrow \R$. Throughout this paper, all multiplayer games considered shall have the same number of players $N$ and the same strategy sets $S^1, \ldots, S^N$. Hence, any game $G$ will be determined by its utility functions $\{u_i\}_{i \in [N]}$. The players choose their strategies simultaneously and they cannot communicate with each other. A utility function $u_i$ can be summarized by its pure strategy outcomes for player~$i$, captured as an $N$-dimensional array $\big\{ u_i(\ks) \big\}_{\ks \in S}$.

\begin{ex}
$2$-player games are better known as bimatrix games because their $2$-dimensional payoff arrays in become matrices $A,B \in \R^{m \times n}$.
\end{ex}

As usual, we allow the players to randomize over their pure strategies. Then, player~$i$'s strategy space extends to the set of probability distributions over $S^i$. We identify this set with $\Delta(S^i) := \, \Big\{ s^i = (s_k^i)_k \, \in \R^{m_i} \, \Big| \, s_k^i \geq 0 \, \, \forall k \in [m_i] \, \, \text{and} \, \sum_{k \in [m_i]} s_k^i = 1 \Big\}$ and refer to the probability distributions as mixed strategies. A tuple $\strats = (s^1, \ldots, s^N) \in \Delta(S^1) \times \ldots \times \Delta(S^N) =: \Delta(S)$ of mixed strategies is called a strategy profile in $G$\footnote{Note that in our notation, $\Delta(S)$ is not a simplex of higher dimensions itself but only the product of $N$ simplices.}. The utility payoff of player~$i$ for the strategy profile $\strats$ is defined as the player's utility payoff in expectation 
\[ u_i(\strats) := \sum_{\ks \in S} s_{k_1}^1 \cdot \ldots \cdot s_{k_N}^N \cdot u_i(\ks) \, .\]
The goal of each player is to maximize her utility.


We will abbreviate with $S^{-i}$ the set that consists of all possible pure strategy choices $\ks_{-i} = (k_1, \ldots, k_{i-1},k_{i+1}, \ldots, k_N)$ of the opponent players (resp. $\Delta(S^{-i})$ for the set of mixed strategy choices $\strats^{-i} = (s^1, \ldots, s^{i-1},s^{i+1}, \ldots, s^N)$). We will also often use $u_i(k_i,\ks_{-i})$ instead of $u_i(\ks)$ to stress how player~$i$ can only influence her own strategy when it comes to her payoff (resp. $u_i(s^i,\strats^{-i})$ instead of $u_i(\strats)$).
\begin{defn}
The best response set of player~$i$ to the opponents' strategy choices $\strats^{-i}$ is defined as $\BR_{u_i}(\strats^{-i}) :=  \argmax_{t^i \in \Delta(S^i)} \big\{ \, u_i(t^i,\strats^{-i}) \, \big\}$. 
\end{defn} 
Best response strategies capture the idea of optimal play against the other player's strategy choices. The most popular equilibrium concept in non-cooperative games is based on best responses.
\begin{defn}
A strategy profile $\strats \in \Delta(S)$ to a game $G = \{u_i\}_{i \in [N]}$ is called a \NE{} if for all player~$i \in [N]$ we have $s^i \in \BR_{u_i}(\strats^{-i})$.
\end{defn}
\noindent
By \cite{Nash48}, any multiplayer game $G$ admits at least one \NE{}.

\subsection{Positive Affine Transformations} 

The following lemma is a well-known result for $2$-player games\footnote{ See \cite[Lemma 2.1]{heyman}, \cite[Theorem 5.35]{maschler_solan_zamir_2013}, \cite[Chapter 3]{harsanyi1988general} or \cite[Proposition 3.1]{DynGT}.}:
\begin{lemma}
\label{PAT preserves lemma}
Let $(A,B)$ be a $m \times n$ bimatrix game and take arbitrary $\alpha, \beta >0$ and $c \in \R^n, d \in \R^m$. Define $A' = \alpha A + \1_m c^T$ and $B' = \beta B + d \1_n^T$.

Then the game $(A', B')$ has the same best response sets as the game $(A,B)$. Consequently, both games have the same \NE{} set.
\end{lemma}
The intuition behind this lemma is as follows:
player~$1$ wants to maximize her utility given the strategy that player~$2$ chose. A positive rescaling of $u_1$ will change the utility payoffs but will not change the sets of best response strategies. The same holds true if we add utility payoffs to $u_1$ that are only dependent on the strategy choice $s^2$ of her opponent. In the notation of bimatrix games, this intuition yields that the transformation $A \mapsto \alpha A + \1_m c^T$ does not affect the best response sets of player~$1$. The analogous result holds for player~$2$ and the transformation $B \mapsto \beta B + d \1_n^T$.

Let us generalize PATs to multiplayer games.
\begin{defn}
\label{multiplayer PAT defn}
A positive affine transformation (PAT) specifies for each player~$i$ a scaling parameter $\alpha^i \in \R, \alpha^i >0,$ and translation constants $C^i := ( c_{\ks_{-i}})_{\ks_{-i} \in S^{-i}}$ for each choice of pure strategies from the opponents. 
The PAT $H_{\textnormal{PAT}} = \big\{ \alpha^i, C^i \big\}_{i \in [N]}$ applied to an input game $G = \{u_i\}_{i \in [N]}$ returns the transformed game $H_{\textnormal{PAT}}(G) = \{u_i'\}_{i \in [N]}$ in which (only) the utility functions changed to
\begin{align}
\label{PAT transformed utilities}
\begin{aligned}
u_i' : S &\longrightarrow \R \\
\ks &\longmapsto \alpha_i \cdot u_i(\ks) + c_{\ks_{-i}}^i \, .
\end{aligned}
\end{align}
\end{defn}
We could not find multiplayer PATs defined in the literature, so we came up with the natural generalization above. As shown in Section \ref{sec:bimatrix games}, they indeed generalize the 2-player PATs from Lemma~\ref{PAT preserves lemma} to multiplayer settings. Moreover, multiplayer PATs also preserve the best response sets and \NE{} set.
\begin{lemma}
\label{multiplayer PAT preserves}
Take a PAT $H_{\textnormal{PAT}} = \big\{ \alpha^i, C^i \big\}_{i \in [N]}$ and any game $G = \{u_i\}_{i \in [N]}$. Then, the transformed game $H_{\textnormal{PAT}}(G) = \{u_i'\}_{i \in [N]}$ has the same best response sets as the input game $G$. Consequently, $H_{\textnormal{PAT}}(G)$ also has the same \NE{} set as $G$.
\end{lemma}
\begin{proof}
See \ref{sec:helpinglemmas}.
\end{proof}
PATs have found much success as a tool for simplifying an input game precisely because of this property. We want to investigate which other game transformations also preserve the best response sets or the \NE{} set. If we found more of these transformations, we could use them to, e.g., further increase the class of efficiently solvable games.

\subsection{Game Transformations}

There are various ways in which we could define the concept of a game transformation. Section~\ref{literature review} gives an overview of some definitions from the literature that are useful for different purposes. A key component of PATs are that they operate player-wise and strategy-wise, that is, they do not change the player set nor the players' strategy sets. This allows for a direct comparison of the strategic structure between a game and its PAT-transform. We argue that this is a natural desideratum for a definition of more general game transformation.

\begin{defn}
\label{def game trafo}
A game transformation $H = \{H^i\}_{i \in [N]}$ specifies for each player~$i$ a collection of functions $H^i := \Big\{ h_{\ks}^i : \R \longrightarrow \R \Big\}_{\ks \in S}$, indexed by the different pure strategy profiles $\ks$. \\
The transformation $H$ can then be applied to any $N$-player game $G = \{u_i\}_{i \in [N]}$ to construct the transformed game $H(G) = \{H^i(u_i)\}_{i \in [N]}$ where 
\begin{align}
\label{transformed pure utilities evaluation}
    H^i(u_i) : S \to \R, \quad \ks \mapsto h_{\ks}^i \big( u_i(\ks) \big) \, .
\end{align}
\end{defn}
Observe that the utility payoff of player~$i$ in the transformed game $H(G)$ from the pure strategy outcome $\ks$ is only a function of the utility payoff from \textit{that same} player in \textit{that same} pure strategy outcome of the input game~$G$.

We extend the utility functions $H^i(u_i)$ to mixed strategy profiles $\strats \in \Delta(S)$ as usual through $H^i(u_i)(\strats) := \sum_{\ks \in S} s_{k_1}^1 \cdot \ldots \cdot s_{k_N}^N \cdot h_{\ks}^i \big( u_i(\ks) \big)$. To simplify future notation, we will often use $h_{k_i,\ks_{-i}}^i$ to refer to $h_{\ks}^i$.

\begin{rem}
A multiplayer positive affine transformation $H_{\textnormal{PAT}} = \big\{ \alpha^i, C^i \big\}_{i \in [N]}$ makes a game transformation $H = \{H^i\}_{i \in [N]}$ in the above sense by setting $h_{\ks}^i : \, \, \R \to \R$, $z \mapsto \alpha^i  \cdot z + c_{\ks_{-i}}^i$.
\end{rem}
\begin{defn}
\label{defn NE preserving}
Let $H = \{H^i\}_{i \in [N]}$ be a game transformation. Then we say that $H$ universally preserves \NE{} sets if for all input games $G = \{u_i\}_{i \in [N]}$, the transformed game $H(G) = \{H^i(u_i)\}_{i \in [N]}$ has the same \NE{} set as the input game $G$.
\end{defn}

\begin{defn}
\label{defn BR preserving}
Let map $H^i$ come from a game transformation $H$. Then we say that $H^i$ universally preserves best responses if for all utility functions $u_i : S \longrightarrow \R$ and for all opponents' strategy choices $\strats^{-i} \in \Delta(S^{-i})$:
\begin{equation*}
\BR_{H^i(u_i)}(\strats^{-i}) = \argmax_{t^i \in \Delta(S^i)} \big\{ H^i(u_i)(t^i,\strats^{-i}) \big\} = \argmax_{t^i \in \Delta(S^i)} \big\{ u_i(t^i,\strats^{-i}) \big\} = \BR_{u_i}(\strats^{-i}) \, .
\end{equation*}
\end{defn}

\begin{defn}
\label{defn opponent dependence}
Let map $H^i$ come from a game transformation $H$. Then we say that $H^i$ only depends on the strategy choice of the opponents if for all pure strategy choices $\ks_{-i} \in S^{-i}$ of the opponents, we have the map identities
    \[h_{1, \ks_{-i}}^i = \ldots = h_{m_i, \ks_{-i}}^i\,: \R \to \R \, .\]
\end{defn}

\end{document}

\section{Temporal Stream Logic}
\label{sec:TSL}
We present a new logic: Temporal Stream Logic (\TSL), which is
especially designed for synthesis and allows for the manipulatation of infinite
streams of arbitrary (even non-enumerative, or higher order) type. It
provides a straightforward notation to specify how outputs are
computed from inputs, while using an intuitive interface to access
time. The main focus of \TSL is to describe temporal control
flow, while abstracting away concrete implementation details. This not
only keeps the logic intuitive and simple, but also allows a user to identify
problems in the control flow even without a concrete implementation at
hand. In this way, the use of \TSL scales up to any required abstraction, such as API
calls or complex algorithmic transformations.

\medskip

\noindent \textit{Architecture} A TSL formula~$ \varphi $ specifies a
reactive system that in every time step processes a finite number of inputs~$ \inames $
and produces a finite number of outputs~$ \onames $. Furthermore, it
uses cells~$ \cells $ to store a value computed at time~$ t $,
which can then be reused in the next time step~$ t + 1 $. An overview
of the architecture of such a system is given in \cref{fig:tslarchitecture}. In terms
of behavior, the environment produces infinite streams of input data,
while the system uses pure (side-effect free) functions to transform
the values of these input streams in every time step. After their
transformation, the data values are either passed to an output stream
or are passed to a cell, which pipes the output value from one time step back to the corresponding input value of the next.
The behaviour of the system is captured by its infinite execution over time.

\begin{figure}[t]
  \centering

  \begin{subfigure}[b]{0.58\textwidth}
  \begin{tikzpicture}[scale=0.8]

    \node[anchor=east,inner sep=0pt] at (-2.8,-1.5) {
      \small
      \begin{tabular}{c}
        inputs: \\[0.1em] $ \inames $
      \end{tabular}
    };

    \node at (0,1.38) {
      \small
      cells: $ \cells $
    };

    \node[anchor=west,inner sep=0pt] at (2.8,-1.5) {
      \small
      \begin{tabular}{c}
        outputs: \\[0.1em]
        $ \onames $
      \end{tabular}
    };

    \node at (0,0) {
      \begin{tikzpicture}[xscale=0.8,yscale=0.56]
        \node[fill, fill=blue!30,minimum height=5.5em, minimum width=10.5em] (C) {};

        \node at (C) {
          \small
          \begin{tabular}{c}
            \textit{reactive system} \\[0.4em]
            \textit{implementing a} \\[0.4em]
            \textit{TSL specification~$ \varphi $}
            \end{tabular}
        };

        \node[minimum size=0.9em] (H0) at (0,1.95) {};
        \node[minimum size=0.9em] (H1) at (0,2.6) {};
        \node[minimum size=0.9em] (H2) at (0,3.8) {};

        % signals
        \path[->,>=stealth,line width=0.7pt]
        ($ (C.west) + (-0.6,-1.3) $) edge ($ (C.west) + (0,-1.3) $)
        ($ (C.west) + (-0.6,-0.6) $) edge ($ (C.west) + (0,-0.6) $)
        ($ (C.west) + (-0.6,-0.3) $) edge ($ (C.west) + (0,-0.3) $)
        ($ (C.east) + (0,-1.3) $) edge ($ (C.east) + (0.6,-1.3) $)
        ($ (C.east) + (0,-0.6) $) edge ($ (C.east) + (0.6,-0.6) $)
        ($ (C.east) + (0,-0.3) $) edge ($ (C.east) + (0.6,-0.3) $)
        ;

        \node at ($ (C.west) + (-0.35,-0.85) $) {\scalebox{0.6}{$ \vdots $}};
        \node at ($ (C.east) + (0.25,-0.85) $) {\scalebox{0.6}{$ \vdots $}};

        % cells
        \draw[line width=0.7pt,-,>=stealth,gray]
        ($ (C.east) + (0,1.3) $) -- (2.7,1.3) |- (H0);
        \draw[line width=0.7pt,->,>=stealth,gray]
        (H0) -| (-2.7,1.3) -- ($ (C.west) + (0,1.3) $);

        \draw[line width=0.7pt,-,>=stealth,gray]
        ($ (C.east) + (0,1) $) -- (3,1) |- (H1);
        \draw[line width=0.7pt,->,>=stealth,gray]
        (H1) -| (-3,1) |- ($ (C.west) + (0,1) $);

        \draw[line width=0.7pt,-,>=stealth,gray]
        ($ (C.east) + (0,0.3) $) -- (3.6,0.3) |- (H2);
        \draw[line width=0.7pt,->,>=stealth,gray]
        (H2) -| (-3.6,0.3) |- ($ (C.west) + (0,0.3) $);

        \node at ($ (C.west) + (-0.35,0.75) $) {\scalebox{0.6}{$ \vdots $}};
        \node at ($ (C.east) + (0.25,0.75) $) {\scalebox{0.6}{$ \vdots $}};

        \fill[fill=orange!60]
        ($ (H0.north west) + (0,-0.1) $) --
        ($ (H0.south west) + (0,0.1) $) --
        ($ (H0.south west) + (0.1,0) $) --
        ($ (H0.south east) + (-0.1,0) $) --
        ($ (H0.south east) + (0,0.1) $) --
        ($ (H0.north east) + (0,-0.1) $) --
        ($ (H0.north east) + (-0.1,0) $) --
        ($ (H0.north west) + (0.1,0) $) --
        cycle;

        \fill[fill=orange!60]
        ($ (H1.north west) + (0,-0.1) $) --
        ($ (H1.south west) + (0,0.1) $) --
        ($ (H1.south west) + (0.1,0) $) --
        ($ (H1.south east) + (-0.1,0) $) --
        ($ (H1.south east) + (0,0.1) $) --
        ($ (H1.north east) + (0,-0.1) $) --
        ($ (H1.north east) + (-0.1,0) $) --
        ($ (H1.north west) + (0.1,0) $) --
        cycle;

        \fill[fill=orange!60]
        ($ (H2.north west) + (0,-0.1) $) --
        ($ (H2.south west) + (0,0.1) $) --
        ($ (H2.south west) + (0.1,0) $) --
        ($ (H2.south east) + (-0.1,0) $) --
        ($ (H2.south east) + (0,0.1) $) --
        ($ (H2.north east) + (0,-0.1) $) --
        ($ (H2.north east) + (-0.1,0) $) --
        ($ (H2.north west) + (0.1,0) $) --
        cycle;

      \end{tikzpicture}
    };
  \end{tikzpicture}
\caption{Architecture \mbox{\ }}
\label{fig:tslarchitecture}
\end{subfigure}
\begin{subfigure}[b]{0.41\textwidth}
  \begin{tikzpicture}

    \fill[gray!20,rounded corners=3] (-2.5,-2.5) rectangle (2.5,0.7);

    \node[anchor=center] at (0,0.45) {
      \textbf{Function Term:}
    };

    \node[anchor=center] at (0,0) {
      $ \fterm \ := \ \; \name{s}_{\name{i}} \! \! \sep \! \name{f} \ \, \fterm^{0} \
      \, \fterm^{1} \ \, \cdots \ \, \fterm^{n-1} $
    };

    \draw (-2.5,-0.35) -- (2.5,-0.35);

    \node[anchor=center] at (0,-0.65) {
      \textbf{Predicate Term:}
    };

    \node[anchor=center] at (0,-1.1) {
      $ \pterm \ := \ \;
      \name{p} \ \, \fterm^{0} \ \, \fterm^{1} \ \, \cdots \ \, \fterm^{n-1}  $
    };

    \draw (-2.5,-1.45) -- (2.5,-1.45);

    \node[anchor=center] at (0,-1.8) {
      \textbf{Update:}
    };

    \node[anchor=center] at (0,-2.25) {
      $ \upd{\name{s}_{\name{o}}}{\fterm} $
    };
  \end{tikzpicture}
  \caption{Term Definitions \mbox{\ }}
  \label{fig:termdefinitions}
\end{subfigure}

\caption{General architecture of reactive systems that are specified in TSL
  on the left, and the structure of function, predicate and updates
  on the right.}
\end{figure}

\medskip

\fussy

\noindent \textit{Function Terms, Predicate Terms, and Updates} In \TSL we
differentiate between two elements: we use purely functional
transformations, reflected by \mbox{functions~$ f \in \functions $}
and their compositions, and \mbox{predicates~$ p \in \predicates $},
used to control how data flows inside the system.
To argue about both
elements we use a term based notation, where we distinguish between
function terms~$ \fterm $ and predicate terms~$ \pterm $,
respectively. Function terms are either constructed from inputs or
cells \mbox{($ \name{s}_{\name{i}} \in \inames \cup \cells $)}, or from
functions, recursively applied to a set of function terms. Predicate
terms are constructed similarly, by applying a predicate to a set of
function terms.
%
\sloppy
%
Finally, an update takes the result of a function computation and passes it either to an output
or a cell ($ \name{s}_{\name{o}} \in \onames \cup \cells $). An overview of the syntax
of the different term notations is given in \cref{fig:termdefinitions}.
Note that we use curried argument notation similar to functional
programming languages.

We denote sets of function and predicate terms, and updates by
$ \fterms $, $ \pterms $ and $ \uterms $, respectively, where
$ \pterms \subseteq \fterms $.  We use $ \fnames $ to denote the set
of function literals and $ \pnames \subseteq \fnames $ to denote the
set of predicate literals, where the literals $ \name{s}_{\name{i}} $,
$ \name{s}_{\name{o}} $, $ \name{f} $ \linebreak and~$ \name{p} $ are
symbolic representations of inputs and cells, outputs and cells, and
functions and predicates, respectively.  Literals are used to
construct terms as shown in \cref{fig:termdefinitions}. Since we use a
symbolic representation, functions and predicates are not tied to a
specific implementation. However, we still classify them according to
their arity, i.e., the number of function terms they are applied to,
as well as by their type: input, output, cell, function or
predicate. Furthermore, terms can be compared syntactically using the
equivalence relation~$ \equiv $.  To assign a semantic interpretation
to functions, we use an assignment function
\mbox{$ \assign{\cdot} \from \fnames \to \functions $}.

\medskip

\noindent \textit{Inputs, Outputs, and Computations} We consider momentary
inputs \mbox{$ i \in \fspace{\inames}{\values} $}, which are
assignments of inputs~$ \name{i} \in \inames $ to values
$ v \in \values $. For the sake of readability let
$ \inputs = \fspace{\inames}{\values} $. Input streams are infinite
sequences~\mbox{$ \iota \in \inputs^{\hspace{0.2pt}\omega} $}
consisting of infinitely many momentary inputs.

Similarly, a momentary output~$ o \in \fspace{\onames}{\values} $ is
an assignment of outputs~$ \name{o} \in \onames $ to values
$ v \in \values $, where we also use
$ \outputs = \fspace{\onames}{\values} $. Output streams are infinite
sequences~$ \varrho \in \outputs^{\hspace{0.5pt}\omega} $. To capture
the behavior of a cell, we introduce the notion of a
computation~$ \comp $.
A computation fixes the function terms that are used to compute outputs and cell updates,
without fixing semantics of function literals.
 Intuitively, a computation only determines which
function terms are used to compute an output, but abstracts from actually
computing it.

The basic element of a computation is a computation
step~$ \cstep \in \fspace{\onames \cup \cells}{\fterms} $, which is an
assignment of outputs and
cells~$ \name{s}_{\name{o}} \in \onames \cup \cells $ to function
terms~$ \fterm \in \fterms $. For the sake of readability let
$ \comps = \fspace{\onames \cup \cells}{\fterms} $. A computation step
fixes the control flow behaviour at a single point in time. A
computation~\mbox{$ \comp \in \comps^{\omega} $} is an infinite sequence of
computation steps.

As soon as input streams, and function and predicate implementations
are known, computations can be turned into output streams. To this
end, let $ \assign{\cdot} \from \fnames \to \functions $ be some
function assignment.  Furthermore, assume that there are predefined
constants~$ \inits \in \functions \cap \values $ for every
cell~$ \name{c} \in \cells $, which provide an initial
value for each stream at the initial point in time. To receive an
output stream from a computation~$ \comp \in \comps^{\omega} $ under
the input stream $ \iota $, we use an evaluation
function~$ \eval \from \comps^{\omega} \times
\inputs^{\hspace{0.2pt}\omega} \times \dtime \times \fterms \to
\values $:
%
\begin{eqnarray*}
  \eval(\comp, \iota, t, \name{s}_{\name{i}}) & = &
    \begin{cases}
      \iota(t)(\name{s}_{\name{i}}) & \text{if } \name{s}_{\name{i}} \in \inames \\
      \initsE &
      \text{if } \name{s}_{\name{i}} \in \cells \ \wedge \ t = 0 \\
      \eval(\comp, \iota, t-1, \comp(t-1)(\name{s}_{\name{i}})) &
      \text{if } \name{s}_{\name{i}} \in \cells \ \wedge \ t > 0
    \end{cases}
  \\[0.5em]
  \eval(\comp, \iota, t, \name{f} \ \term_{0} \ \cdots \ \term_{m-1}) & = &
  \assign{\name{f}} \ \eval(\comp,\iota, t,\term_{0}) \
  \cdots \ \eval(\comp,\iota, t,\term_{m-1})
\end{eqnarray*}
%
Then
$ \varrho_{\hspace{-1pt}\langle \hspace{-1pt}\cdot
  \hspace{-1pt}\rangle \hspace{-1pt}, \comp, \iota} \in
\outputs^{\hspace{0.5pt}\omega} $ is defined via
$ \varrho_{\hspace{-1pt}\langle \hspace{-1pt}\cdot
  \hspace{-1pt}\rangle \hspace{-1pt}, \comp, \iota}(t)(\name{o}) =
\eval(\comp, \iota, t, \name{o}) $ for all $ t \in \dtime $,
$ \name{o} \in \onames $.

\medskip
\smallskip

\noindent \textit{Syntax} Every TSL formula~$ \varphi $ is built
according to the following grammar:
%
\begin{equation*}
  \varphi \ \ := \ \
  \term \in \pterms\cup \uterms
  \!\sep\! \neg \varphi
  \!\sep\! \varphi \wedge \varphi
  \!\sep\! \LTLnext \varphi
  \!\sep\! \varphi \LTLuntil \varphi
\end{equation*}
%
An atomic proposition $\tau$ consists either of a predicate term, serving as a
Boolean interface to the inputs, or of an update, enforcing a
respective flow at the current point in time. Next, we have the
Boolean operations via negation and conjunction, that allow us to express
arbitrary Boolean combinations of predicate evaluations and
updates. Finally, we have the temporal operator next:
$ \LTLnext \psi $, to specify the behavior at the next point in time
and the temporal operator until:~$ \vartheta \LTLuntil \psi $, which
enforces a property~$ \vartheta $ to hold until the property~$ \psi $
holds, where $ \psi $ must hold at some point in the future
eventually.
%
\medskip
\smallskip

\noindent \textit{Semantics} Formally, this leads to the following
semantics.  Let $ \assign{\cdot} \from \fnames \to \functions $,
\mbox{$ \iota \in \inputs^{\hspace{0.2pt}\omega} $}, and
$ \comp \in \comps^{\omega} $ be given, then the validity of a $ \TSL$
formula~$ \varphi $ with respect to $ \comp $ and $ \iota $ is defined inductively
over $ t \in \dtime $ via:
%
\begin{equation*}
  \begin{array}{lcl}
    \\[-1.8em]
    \comp, \iota, t \sats \name{p} \ \term_{0} \ \cdots \ \term_{m-1} & \ :\Leftrightarrow \ \
    & \eval(\comp,\iota,t,\name{p} \ \term_{0} \ \cdots \ \term_{m-1}) \\[0.2em]
    \comp, \iota, t \sats \upd{\name{s}}{\!\term} & :\Leftrightarrow
    & \comp(t)(\name{s}) \equiv \term \\[0.2em]
    \comp, \iota, t \sats \neg \psi & :\Leftrightarrow
    & \comp, \iota, t \nsats \psi \\[0.2em]
    \comp, \iota, t \sats \vartheta \wedge \psi & :\Leftrightarrow
    & \comp, \iota, t \sats \vartheta \ \wedge \ \comp, \iota, t \sats \psi \\[0.2em]
    \comp, \iota, t \sats \LTLnext \psi & :\Leftrightarrow
    & \comp, \iota, t+1 \sats \psi \\[0.2em]
    \comp, \iota, t \sats \vartheta \LTLuntil \psi & :\Leftrightarrow
    & \exists t'' \geq t. \ \
         \forall t \leq t' < t''. \ \ \comp, \iota, t' \sats \vartheta \ \,
         \wedge \ \, \comp, \iota, t'' \sats \psi
  \end{array}
\end{equation*}
%
Consider that the satisfaction of a predicate depends on the current
computation step and the steps of the past, while for updates it only
depend on the current computation step. Furthermore, updates are only
checked syntactically, while the satisfaction of predicates depends on
the given assignment~$ \assign{\cdot} $ and the input stream
$ \iota $.
%
We say that $ \comp $ and $ \iota $ satisfy $ \varphi $, denoted by
$ \comp, \iota \sats \varphi$, if $ \comp, \iota, 0 \sats \varphi
$.

Beside the basic operators we have the standard derived Boolean
operators, as well as the derived temporal operators:
\textit{release}~$ \varphi \LTLrelease \psi \equiv \neg ((\neg \psi)
\LTLuntil (\neg \varphi)) $,
\textit{finally}~$ \LTLfinally \varphi \equiv \emph{true} \LTLuntil
\varphi $,
\textit{always}~$ \LTLglobally \varphi \equiv \emph{false} \LTLrelease
\varphi $, the \textit{weak} version of \textit{until}
$ \varphi \LTLweakuntil \psi \equiv (\varphi \LTLuntil \psi) \vee
(\LTLglobally \varphi) $, and \textit{as soon
  as}~$ \varphi \mathop{\mathcal{A}}\hspace{0.5pt} \psi \equiv \neg
\psi \LTLweakuntil (\psi \wedge \varphi) $.

\medskip

\noindent \textit{Realizability} We are interested in the following
realizability problem: given a $ \TSL $ formula~$ \varphi $, is there
a strategy~$ \sigma \in \fspace{\inputs^{+}}{\comps} $ such that for every
input $ \iota \in \inputs^{\omega} $ and function implementation
$ \assign{\cdot} \from \fnames \to \functions $, the branch
$ \branch{\sigma}{\iota} $ satisfies $ \varphi $, i.e.,
%
\begin{equation*}
  \exists \sigma \in \fspace{\inputs^{+}}{\comps}. \ \, \forall \iota \in \inputs^{\hspace{0.2pt}\omega}. \ \, \forall \assign{\cdot} \from
  \fnames \to \functions. \ \, \branch{\sigma}{\iota}, \iota \sats \varphi
\end{equation*}
%
If such a strategy~$ \sigma $ exists, we say $ \sigma $ realizes
$ \varphi $. If we additionally ask for a concrete instantiation of
$ \sigma $, we consider the synthesis problem of TSL.


\section{TSL Properties}
\label{sec:props}
In order to synthesize programs from TSL specifications, we give an overview of the first part of our synthesis process, as shown in \cref{fig:system}.
First we show how to approximate the semantics of TSL through a reduction to LTL.
However, due to the approximation, finding a realizable strategy immediately may fail.
Our solution is a CEGAR loop that improves the approximation.
This CEGAR loop is necessary, because the realizability problem of TSL is undecidable in general.

\medskip

\noindent \textit{Approximating TSL with LTL} We approximate TSL
formulas with weaker LTL formulas.  The approximation reinterprets the
syntactic elements, $\pterms$ and $\uterms$, as atomic propositions
for LTL. This strips away the semantic meaning of the function
application and assignment in TSL, which we reconstruct by later
adding assumptions lazily to the LTL formula.

Formally, let $ \pterms $ and $ \uterms $ be the finite sets of
predicate terms and updates, which appear in
$ \varphi_{\textit{TSL}} $, respectively. For every assigned signal, we
partition $ \uterms $ into
$ \biguplus_{\name{s}_{\name{o}} \in \onames \cup \cells}
\uterms^{\hspace{0.5pt}\name{s}_{\name{o}}} $. For every
$ \name{c} \in \cells $ let \mbox{$\utermsp^{\hspace{0.5pt}\name{c}} =
  \uterms^{\hspace{0.5pt}\name{c}} \cup \set{ \upd{\name{c}}{\name{c}}
  } $}, for $ \name{o} \in \onames $ let
$ \utermsp^{\hspace{0.5pt}\name{o}} = \uterms^{\hspace{0.5pt}\name{o}}
$, and let
$ \utermsp = \bigcup_{\name{s}_{\name{o}} \in \onames \cup \cells}
\utermsp^{\hspace{0.5pt}\name{s}_{\name{o}}} $.  We construct the LTL
formula~$ \varphi_{\textit{LTL}} $ over the input
propositions~$ \pterms $ and output propositions $ \utermsp $ as
follows:
%
\begin{equation*}
  \varphi_{\textit{LTL}} \, = \;
  \LTLglobally \Big ( \bigwedge_{\name{s}_{\name{o}} \in \onames \cup \cells} \,
  \bigvee_{\term \in \utermsp^{\hspace{0.5pt}\name{s}_{\name{o}}}}
  \big( \term \; \wedge \bigwedge_{\term' \in
    \utermsp^{\hspace{0.5pt}\name{s}_{\name{o}}} \setminus
    \set{ \term }} \neg \, \term' \big)  \Big) \ \wedge \
  \textsc{SyntacticConversion}\big(\varphi_{\textit{TSL}}\big)
\end{equation*}
%
Intuitively, the first part of the equation partially reconstructs the semantic meaning of updates by ensuring that a signal is not updated with multiple values at a time.
The second part extracts the reactive constraints of the TSL formula without the semantic meaning of functions and updates.
%
\begin{theorem}
  \label{thm:tsl2ltl} If $ \varphi_{\textit{LTL}} $ is realizable, then $ \varphi_{\textit{TSL}} $ is realizable.
\end{theorem}
%
\noindent The proof of \cref{thm:tsl2ltl} is given in \cref{proof:tsl2ltl}.
Note that unrealizability of $\varphi_{\textit{LTL}} $ does not imply that $ \varphi_{\textit{TSL}}$ is unrealizable.
It may be that we have not added sufficiently many environment assumptions to the approximation in order for the system to produce a realizing strategy.

\medskip

\label{ex:asLTL}

\begin{figure*}[t]
    \centering
    \begin{subfigure}[t]{0.28\textwidth}
      \centering
      $ \begin{array}{c}
          \\[-0.5em]
          \LTLglobally \; (\upd{\name{y}}{\name{y}} \, \vee \, \upd{\name{y}}{\name{x}}) \\[0.2em]
          \wedge \ \LTLeventually \, \name{p} \ \name{x} \, \impl \,
          \LTLeventually \, \name{p}\ \name{y} \\[-0.5em]
          \
        \end{array} $
        \caption{TSL specification}
\label{eq:tslSimple}
    \end{subfigure}%
    ~ ~
    \begin{subfigure}[t]{0.30\textwidth}
      \centering
      $ \begin{array}{c}
          \LTLglobally \; \neg ( \name{y\_to\_y} \, \wedge \, \name{x\_to\_y}) \\[0.2em]
          \wedge \ \LTLglobally \; (\name{y\_to\_y} \, \vee \, \name{x\_to\_y}) \\[0.2em]
          \wedge \ \LTLeventually \, \name{p\_x} \, \impl \
          \LTLeventually \, \name{p\_y}
        \end{array} $
\caption{initial approximation}
\label{eq:ltlSimple}
    \end{subfigure}%
    ~\;
    \begin{subfigure}[t]{0.35\textwidth}
      \centering
      \vspace{-1em}
  \begin{tikzpicture}[->,>=stealth',shorten >=1pt,auto,node distance=2.8cm,initial text=]
    \tikzstyle{every state}=[fill=blue!20,draw,text=white,minimum size=1.5em]
    \node[initial,state] (A)                    {};
    \path (A) edge [loop right] node {$\name{p\_x} \; \wedge \; \neg \, \name{p\_y}$} (A);
  \end{tikzpicture}
  \vspace{1.1em}
\caption{spurious counter-strategy}
\label{eq:tslSimpleSoln}
    \end{subfigure}
    \caption{
      A TSL specification~(a) with input~\name{x} and cell~\name{y} that is realizable. A winning strategy is to save~\name{x} to \name{y} as soon as $ \name{p}(\name{x}) $ is satisfied. However, the initial approximation~(b), that is passed to an LTL synthesis solver, is unrealizable, as proven through the counter-strategy~(c) returned by the LTL solver.}
    \label{fig:approx}
\end{figure*}

\noindent \textit{Example} As an example, we present a simple TSL specification in \cref{eq:tslSimple}.
The specification asserts that the environment provides an input~\name{x} for which the predicate~$ \name{p}~\name{x} $ will be satisfied eventually. The system must guarantee that eventually $ \name{p}~\name{y} $ holds.
According to the semantics of TSL the formula is realizable. The system can take the value of $ \name{x} $ when $ \name{p}~\name{x} $ is true and save it to $ \name{y} $, thus guaranteeing that $ \name{p}~\name{y} $ is satisfied eventually.
This is in contrast to LTL, which has no semantics for pure functions - taking the evaluation of $ \name{p}~\name{y} $ as an environmentally controlled value that does not need to obey the consistency of a pure function.

\medskip

\noindent \textit{Refining the LTL Approximation} It is possible that the LTL solver returns a counter-strategy for the environment although the original TSL specification is realizable.
We call such a counter-strategy \textit{spurious} as it exploits the additional freedom of LTL to violate the purity of predicates as made possible by the underapproximation.
Formally, a counter-strategy is an infinite tree $ \pi \from \comps^{*} \to 2^{\pterms} $, which provides predicate evaluations in response to possible update assignments of function terms~$ \fterm \in \fterms $ to outputs~$ \name{o} \in \onames $.
W.l.o.g.\ we can assume that $ \onames $, $ \fterms $ and $ \pterms $ are finite, as they can always be restricted to the outputs and terms that appear in the formula.
A counter-strategy is spurious, iff there is a branch~$ \branch{\pi}{\comp} $ for some computation~$ \comp \in \comps^{\omega} $, for which the strategy chooses an inconsistent evaluation of two equal predicate terms at different points in time, i.e.,
%
\begin{equation*}
  \begin{array}{l}
    \exists \comp \in \comps^{\omega}. \ \exists t, t' \in \dtime. \ \exists \pterm \in \pterms. \\[0.2em]
    \qquad \pterm \in \pi(\comp(0)\comp(1)\ldots\comp(t-1)) \, \wedge \, \pterm \notin \pi(\comp(0)\comp(1)\ldots\comp(t'-1)) \ \wedge \\[0.2em]
    \qquad \forall \assign{\cdot} \from \fnames \to \functions. \ \eval(\comp, \branch{\pi}{\comp}, t, \pterm) \, = \, \eval(\comp, \branch{\pi}{\comp}, t', \pterm).
  \end{array}
\end{equation*}
%
Note that a non-spurious strategy can be inconsistent along multiple
branches. Due to the definition of realizability
the environment can choose function and
predicate assignments differently against every system strategy
accordingly.

By purity of predicates in TSL the environment is forced to
always return the same value for predicate evaluations on equal
values. However, this semantic property cannot be enforced implicitly
in LTL.  To resolve this issue we use the returned counter-strategy to
identify spurious behavior in order to strengthen the LTL
underapproximation with additional environment assumptions.
After adding the derived assumptions, we re-execute the LTL synthesizer to check whether the
added assumptions are sufficient in order to obtain a winning strategy
for the system.  If the solver still returns a spurious strategy, we
continue the loop in a CEGAR fashion until the set of added
assumptions is sufficiently complete.  However, if a non-spurious strategy is
returned, we have found a proof that the given
TSL specification is indeed unrealizable and terminate.

\goodbreak

\begin{algorithm}[t]
  \small
  \caption{Check-Spuriousness} \label{alg:spurious}
  \begin{algorithmic}[1]
    \Require{bound~$ b $, counter-strategy~$ \pi \from \comps^{*}\!\! \to \! 2^{\pterms} $ (finitely represented using $ m $ states)}

    \vspace{0.3em}

    \ForAll{$ v \in \comps^{m \cdot b},\, \pterm \in \pterms, \, t,t' \in \set{ 0,1,\ldots,m\cdot b - 1} $}
      \If{$ \evalid(v,\iota_{\name{id}},t,\pterm) \equiv \evalid(v,\iota_{\name{id}},t',\pterm) \wedge \mbox{\qquad} \qquad \qquad \qquad \qquad \qquad \qquad \qquad $ $ \mbox{\ }\hspace{2.5em} \pterm \in \pi(v_{0}\ldots v_{t-1}) \wedge \pterm \notin \pi(v_{0}\ldots v_{t'-1}) $}
      \State \quad $ w \gets \texttt{reduce}\,(v,\pterm,t,t') $
      \State \quad {\textbf{return} \ $ \LTLglobally \big(\! \bigwedge_{i=0}^{t-1} \LTLnext^{i}\! w_{i} \, \wedge \, \bigwedge_{i = 0}^{t'-1} \LTLnext^{i}\! w_{i} \,\rightarrow\, (\LTLnext^{t}\! \pterm \leftrightarrow \LTLnext^{t'} \!\! \pterm) \big) $}
      \EndIf
    \EndFor
    \State {\textbf{return} \ \texttt{``non-spurious''}}
  \end{algorithmic}
  \vspace{-0.2em}
\end{algorithm}

\cref{alg:spurious} shows how a returned counter-strategy~$ \pi $ is
checked for being spurious. To this end, it is sufficient to
check~$ \pi $ against system strategies bounded by the given
bound~$ b $, as we use bounded
synthesis~\cite{Schewe:2013}. Furthermore, we can assume w.l.o.g.\
that~$ \pi $ is given by a finite state representation, which is
always possible due to the finite model guarantees of LTL. Also note
that~$ \pi $, as it is returned by the LTL synthesizer, responses to
sequences of sets of updates~$ (2^{\utermsp})^{*} $. However, in our
case $ (2^{\utermsp})^{*} $ is an alternative
representation~of~$ \comps^{*} $, due to the additional constraints
added during the construction~of~$ \varphi_{\textit{LTL}} $.

The algorithm iterates over all possible
responses~$ v \in \comps^{m \cdot b} $ of the system up to depth
$ m \cdot b $. This is sufficient, since any deeper exploration would
result in a state repetition of the cross-product of the finite state
representation of~$ \pi $ and any system strategy bounded by~$ b
$. Hence, the same behaviour could also be generated by a smaller
sequence. At the same time, the algorithm iterates over
predicates~$ \pterm \in \pterms $ appearing in
$ \varphi_{\textit{TSL}} $ and times $ t $ and $ t' $ smaller
than~$ m \cdot b $. For each of these elements, spuriousness is
checked by comparing the output of~$ \pi $ for the evaluation of
$ \pterm $ at times~$ t $ and $ t' $, which should only
differ, if the inputs to the predicates are different as well. This
can only happen, if the passed input terms have been constructed
differently over the past. We check it by using the evaluation
function~$ \eta $ equipped with the identity assignment
$ \assign{\cdot}_{\texttt{id}} \from \fnames \to \fnames $, with
$ \assign{\name{f}}_{\texttt{id}} = \name{f} $ for all
$ \name{f} \in \fnames $, and the input sequence
$ \iota_{\texttt{id}} $, with
$ \iota_{\texttt{id}}(t)(\name{i}) = (t,\name{i}) $ for all
$ t \in \dtime$ and $ \name{i} \in \inames $, that always generates a
fresh input. Syntactic inequality of
$ \evalid(v,\iota_{\name{id}},t,\pterm) $ \linebreak and
$ \evalid(v,\iota_{\name{id}},t',\pterm) $ then is a sufficient
condition for the existence of an assignment
$ \assign{\cdot} \from \fterms \to \functions $, for which $ \pterm $
evaluates differently at times $ t $ and~$ t' $.

If spurious behaviour of~$ \pi $ could be found, then the revealing
response~$ v \in \comps^{*} $ is first simplified using
$ \texttt{reduce} $, which turns $ v $ back to a sequence of sets of
updates~$ w \in (2^{\utermsp})^{*} $ and removes updates that do not
affect the behavior of $ \pterm $ at the times $ t $ and $ t' $ to
accelerate the termination of the CEGAR loop. Afterwards, the
sequence~$ w $ is turned into a new assumption that prohibits the found
spurious behavior and, thus, further refines the LTL
underapproximation.

As an example of this process, reconsider the spurious
counter-strategy of \cref{eq:tslSimpleSoln}. Already after the first
system response~$ \upd{\name{y}}{\name{x}} $, the environment produces
an inconsistency by evaluating~$ \name{p} \ \name{x} $ and
$ \name{p} \ \name{y} $ differently. This is inconsistent, as the
cell~$ \name{y} $ holds the same value at time~$ t = 1 $ as the
input~$ \name{x} $ at time~$ t = 0 $. Using \cref{alg:spurious} we generate
the new
assumption~$ \LTLglobally (\upd{\name{y}}{\name{x}} \impl (\name{p} \
\name{x} \leftrightarrow \LTLnext \name{p} \ \name{y})) $. After adding this
strengthening the LTL synthesizer returns a realizability result.

\medskip

\goodbreak

\noindent \textit{Undecidability}
Although we can approximate the semantics of TSL with LTL, there are
TSL formulas that cannot be expressed as LTL formulas of finite
size.
%
\begin{theorem}\label{thm:decidability}
  The realizability problem of $ \TSL $ is undecidable.
\end{theorem}
%
\begin{proof}
  We reduce an instance of the Post Correspondence
  Problem~(PCP)~\cite{post1946}, consisting of an alphabet~$ \Sigma $
  and sequences
  $ w_{0}w_{1}\ldots w_{n}, v_{0}v_{1}\ldots v_{n} \in \Sigma^{*} $,
  to the realizability of a $ \TSL $ formula~$ \varphi $. To this end,
  we fix some unary predicate~$ \name{p} \in \pnames $, a~unary
  function $ \name{f} \in \fnames $ for every alphabet symbol
  $ f \in \Sigma $, and some $ 0 $-nary
  function~$ \name{X} \in \fnames $. The system has no
  inputs~$ \inames $, but two outputs $ \name{A} \in \onames $ and
  $ \name{B} \in \onames $.

  Initially, we assign the signals $ \name{A} $ and $ \name{B} $ the
  constant value~$ \name{X} $. From then on, we non-deterministically
  pick pairs $ (w_{j},v_{j}) $ in every time step, as provided by the
  PCP instance, where every $ w_{j} $ and $ v_{j} $ is represented as
  a stacked composition of the corresponding alphabet functions. Our
  choice is stored in the signals~$ \name{A} $ and $ \name{B} $ for
  $ w_{j} $ and $ v_{j} $, respectively. Finally, we check that the
  sequences of function applications, constructed over time, are equal
  at some point, using the eventually operator~$ \LTLfinally $ and the
  universally quantified predicate~$ \name{p} $ to check for equality.
\end{proof}
%
\noindent A more detailed version of the proof can be found in
\cref{proof:decidability}. Also note that no inputs are used by
the proof, which additionally shows that the \mbox{``satisfiability''} problem of
\TSL is undecidable as well.


\section{TSL Synthesis}
\label{sec:synth}
Our synthesis framework provides a modular refinement process
to synthesize executables from $ \TSL $ specifications, as depicted
in \cref{fig:system}. The user initially provides a
$ \TSL $ specification over predicate and function terms.  At the end
of the procedure, the user receives an executable to control a
reactive system.

The first step of our method answers the synthesis question of TSL: if
the specification is realizable, then a control flow model is
returned.  To this end, an intermediate translation to LTL is used,
utilizing an LTL synthesis solver that produces circuits in the AIGER
format. If the specification is realizable, the resulting control flow
model is turned into Haskell code, which is implemented as an
independent Haskell module. The user has the choice between two
different targets: a module built on Arrows, which is compatible with
any Arrowized FRP library, or a module built on Applicative, which
supports Applicative FRP \mbox{libraries}. Our procedure generates a single
Haskell module per TSL specification. This makes naturally decomposing
a project according to individual tasks possible. Each module provides
a single component, which is parameterized by their initial state and
the pure function and predicate transformations. As soon as these are
provided as part of the surrounding project context, a final
executable can be generated by compiling the Haskell code.

An important feature of our synthesis approach is that implementations
for the terms used in the specification are only required after
synthesis.  This allows the user to explore several possible
specifications before deciding on any term implementations.

\paragraph{Control Flow Model} The first step of our approach is the
synthesis of a \textit{Control Flow Model}~$ \cfm $ (CFM) from the
given $ \TSL $ specification~$ \varphi $, which provides us with a
uniform representation of the control flow structure of our final
program.

\noindent Formally, a CFM~$ \cfm $ is a tuple
$ \cfm = (\inames, \onames, \cells, \vertices, \labeling,
\dependencies), $ where $ \inames $ is a finite set of inputs,
$ \onames $ is a finite set of outputs, $ \cells $ is a finite set of
cells, $ \vertices $ is a finite set of vertices,
$ \labeling \from \vertices \to \fnames $ assigns a
vertex a function~$ \name{f} \in \fnames $ or a
predicate~$ \name{p} \in \pnames $, and
%
\begin{equation*}
  \dependencies \from (\onames \cup \cells \cup \vertices) \times
  \nats \to (\inames \cup \cells \cup \vertices \cup \set{ \bot })
\end{equation*}
%
is a dependency relation that relates every output, cell, and
vertex of the CFM with $ n \in \nats $ arguments, which are either
inputs, cells, or vertices. Outputs and
cells~$ \name{s} \in \onames \cup \cells $ always have only a single
argument, i.e., $ \delta(s, 0) \not\equiv \bot $ and
\mbox{$ \forall m > 0 .\ \delta(\name{s}, m) \equiv \bot $}, while for
vertices~$ x \in \vertices $ the number of arguments $ n \in \nats $
align with the arity of the assigned function or predicate
$ \labeling(x) $, i.e.,
$ \forall m \in \nats .\ \delta(s, m) \equiv \bot \leftrightarrow m >
n $. A CFM is valid if it does not contain circular dependencies,
i.e., on every cycle induced by $ \delta $ there must lie at least a
single cell. We only consider valid CFMs.

\newcommand{\arrow}[3]{
  \node[anchor=west,inner sep=0pt] at #1 (#2) {
    \begin{tikzpicture}[scale=0.8,inner sep=2pt]
      \node[arrow] (A) at (-3,0) {#3};      
      \fill[green!10] (A.south east) -- (A.south west) -- 
      (A.north west) -- (A.north east) 
      -- ($ (A.east) + (0.2,0) $) -- cycle;
      \node[arrow] (A) at (-3,0) {\phantom{#3}};      
      \draw (A.south east) -- (A.south west) -- 
      (A.north west) -- (A.north east) -- 
      ($ (A.east) + (0.2,0) $) -- cycle;
      \node[text depth=0pt] at (A) at (-3,0) {#3};            
    \end{tikzpicture}
  };
}
\begin{figure}[t]
  \centering

\begin{tikzpicture}[circuit logic US]
  \tikzset{ 
    arrow/.style={fill=green!10},      
    and/.style={and gate, inputs={nn}, point right,blue!80!black!50!white,fill},
    or/.style={or gate, inputs={nn}, point right,blue!70,fill},
    not/.style={not gate, point right, scale=0.5,black!50!blue, fill},
    sec/.style={fill, circle,inner sep=0.7pt}, 
  }
  
  \draw[drop shadow,fill=gray!10] (-5.5,-4.6) rectangle (4.25,2);

  \begin{scope}[circuit logic US,line width=0.4,scale=0.45,xshift=-81,yshift=55]
    \tikzset{
      and/.style={and gate, inputs={nn}, point right,blue!80!black!50!white,fill},
      or/.style={or gate, inputs={nn}, point right,blue!70,fill},
      not/.style={not gate, point right, scale=0.5,black!50!blue, fill},
      sec/.style={fill, circle,inner sep=0.7pt},
    }
    \draw[fill=yellow!10,rounded corners=3,fill,drop shadow,thin] (-2,-5.9) rectangle (7.8,2);

    \node at (-2,-2.6) (pause) {}; 
    \node at (-2,1.4) (cfg) {}; 
    \node at (-2,0.7) (play) {}; 
    \node at (-2,-1.3) (resume) {};
    \node at (-2,-5.1) (leave) {};  
    \node at (-2,-5.3) (music) {}; 

    \node at (5.7,-5.9) (out1) {};
    \node at (6.5,-5.9) (out2) {};
    \node at (7.3,-5.9) (out3) {};

    \node at (7.7,0.9) (out4) {};
    \node at (7.7,1.4) (out5) {};

    \node[and] at (0.05,0.6) (a1) {};
    \node[or] at (0,-1.4) (o1) {};
    \node[or] at (0,-2.2) (o2) {};
    \node[or] at (0,-3) (o3) {};
    \node[not] at (1.1,-3) (n1) {};
    \node[not] at (1.1,-2.2) (n2) {};
    \node[not] at (1.1,0.2) (n3) {};
    \node[and] at (2.3,-1) (a2) {};
    \node[or] at (3.6,-0.5) (o4) {};
    \node[and] at (2.3,-3.7) (a3) {};
    \node[and] at (0.05,-5.2) (a4) {};
    \node[and] at (2.3,-4.5) (a5) {};
    \node[or] at (3.6,-4.1) (o5) {};
    \node[not] at (1.1,-5.2) (n4) {};
    \node[and] at (3.7,-3.1) (a6) {};
    \node[and] at (5.3,-1.8) (a7) {};
    \node[or] at (6.6,0.1) (o6) {};
    \node[or] at (5.2,0.9) (o7) {};
    \node[not] at (6.6,0.9) (n5) {};

    \draw (o1.output) ++ (right:0.2) node[sec] {} |- (n3.input);
    \draw (o1.output) -- ++ (right:0.2) |- (a3.input 1);
    \draw (o2.output) -- (n2.input);
    \draw (o3.output) -- (n1.input);
    \draw (n2.output) -- ++ (right:0.2) |- (a2.input 2);
    \draw (a2.output) -- ++ (right:0.2) |- (o4.input 2);
    \draw (a1.output) -| ($ (a2.output) + (0.2,1) $) |- (o4.input 1);
    \draw (a4.output)  ++ (right:0.25) node[sec] {} |- (a5.input 2);
    \draw (a4.output) -- (n4.input);
    \draw (a3.output) -- ++ (right:0.2) |- (o5.input 1);
    \draw (a5.output) ++ (right:0.2) |- (o5.input 2);
    \draw (a5.output) -- ++ (right:0.2) node[sec] {} |- ($ (a6.output) + (0.2,-1.7) $) |- (o7.input 2);
    \draw (n1.output) -- (a6.input 1);
    \draw (n3.output) ++ (right:0.8) node[sec] {} |- (o7.input 1);
    \draw (n3.output) -- (o6.input 1);
    \draw (n4.output) -- ++ (right:0.2) |- (a6.input 2);
    \draw (a6.output) -- ++ (right:0.5) node (A) {} |- (a7.input 2);
    \draw (a7.output) -- ++ (right:0.2) |- (o6.input 2);
    \draw (o7.output) -- (n5.input);

    \draw (o5.output) -| (out1.center);
    \draw (o4.output) -- ++ (right:2.4) |- (out2.center);
    \draw (o6.output) -- ++ (right:0.2) |- (out3.center);
    \draw (n5.output) -- (out4.center);
    \draw (o7.output) ++ (right:0.2) node[sec] {} |- (out5.center);

    \draw (pause.center) -- ++ (right:0.5) |- (o2.input 1);
    \draw (pause.center) ++ (right:0.5) node[sec] (T) {} |- node[sec] {} (o3.input 2);
    \draw (T.center |- o3.input 2) |- (a3.input 2);

    \draw (cfg.center) -- ++ (right:1.3) node[sec] (U) {} |- node[sec] {} (a1.input 2);
    \draw (U.center |- a1.input 2) |- node[sec] {} (o1.input 2);
    \draw (U.center |- o1.input 2) |- node[sec] {} (o2.input 2);
    \draw (U.center |- o2.input 2) |- (a5.input 1);
    \draw (cfg.center) -| ($ (A.center) + (0,4) $)  |- (a7.input 1);

    \draw (play.center) -- (a1.input 1);
    \draw (play.center) ++ (right:0.9) node[sec] {} |- (o3.input 1);

    \draw (resume.center) ++ (right:0.5) node[sec] {} |- (a2.input 1);
    \draw (resume.center) -- (o1.input 1);

    \draw (leave.center) -- ++ (right:0.5) |- (a4.input 1);

    \draw (music.center) -- (a4.input 2);
  \end{scope}
  
  \node at (-5,0) (Y) {};
  \node at (1,-2.1) (Z) {};

  \node at (-5.5,-2.6) (mpin) {};
  \node at (-5.5,0.05) (sys) {};
  \node at (-5.5,-4.2) (tr) {};
  \node at (tr.center |- cfg.center) (cin) {};
  \node at (4.25,0.45) (cout) {};
  \node at (4.25,-3.3) (mpout) {};

  \arrow{(Y.center |- cfg.center)}{nA}{$ \name{(== m}_{\name{0}} \!\name{)} $};
  \arrow{(-5,0.9)}{nB}{$ \name{playButton} $};
  \arrow{(Y.center |- resume.center)}{nC}{$ \name{resumeApp} $};
  \arrow{(Y.center |- pause.center)}{nD}{$ \name{pauseButton} $};
  \arrow{(-5,-0.8)}{nE}{$ \name{leaveApp} $};
  \arrow{(Y.center |- music.center)}{nF}{$ \name{musicPlaying} $};

  \arrow{(2.5,0.7)}{nG}{$ \name{m}_{\name{0}} $};
  \arrow{(2.5,0.2)}{nH}{$ \name{m}_{\name{1}} $};
  \arrow{(-2.75,-2.6)}{nI}{\tikz{\node[inner sep=3pt,minimum height=1em]{$ \name{pause} $};}};
  \arrow{(-4.5,-3.8)}{nJ}{$ \name{trackPos} $};

  \arrow{(-1.3,-4)}{nK}{\tikz{\node[inner sep=5pt,minimum height=2em]{$ \name{play} $};}};

  \draw[fill=blue!10] (3.3,0) rectangle (4,1.1);
  \node[circle,draw,fill=white] at (3.75,0.45) (S1) {};

  \draw[fill=blue!10] (1,-4.3) rectangle (2.3,-2.1);
  \node[circle,draw,fill=white] at (2,-3.3) (S2) {};

  \draw (nA.east) ++(left:0.01) -- (cfg.center);
  \draw (nB.east) ++(left:0.01) -- ++(right:0.4) |- (play.center);
  \draw (nC.east) ++(left:0.01) -- (resume.center);
  \draw (nD.east) ++(left:0.01) -- (pause.center);
  \draw (nE.east) ++(left:0.01) -- ++(right:0.9) |- (leave.center);
  \draw (nF.east) ++(left:0.01) -- (music.center);

  \draw (nG.east) ++(left:0.01) -- ++(right:0.4) node[xshift=-4,yshift=3] {\tiny 1} -- (S1);
  \draw (nH.east) ++(left:0.01) -- ++(right:0.4)  node[xshift=-4,yshift=3] {\tiny 2} -- (S1);

  \draw (out4.center) -| (3.5,1.1) node[below,yshift=2] {\tiny 1};
  \draw (out5.center) -| (3.8,1.1) node[below,yshift=2] {\tiny 2};

  \draw (nJ.east) ++(left:0.01) -- (nJ.east -| nK.west) -- ++(right:0.01);

  \draw (out1.center) -- (Z.center -| out1.center) node[yshift=-3] {\tiny 1};
  \draw (out2.center) -- (Z.center -| out2.center) node[yshift=-3] {\tiny 2};
  \draw (out3.center) -- (Z.center -| out3.center) node[yshift=-3] {\tiny 3};

  \draw (nI.east) ++(left:0.01) -- ++ (right:2.7) node (W) {} -- (S2);
  \draw (nK.east) ++(left:0.01) -- (nK.east -| W.center) -- (S2);

  \draw (mpin.center) -- ++ (right:0.25) node[sec] (Q) {} |- (S2);
  \draw (Q |- S2) node[sec] {} |- (nJ.west) -- ++(right:0.01);
  \draw (Q) -- (nI.west) -- ++(right:0.01);
  \draw (Q |- nI.west) |- (nF.west) -- ++(right:0.01);

  \draw (tr.center) -- (tr -| nK.west) -- ++(right:0.01);
  \draw (Q -| Z) node[xshift=3,yshift=3] {\tiny 1};
  \draw (S2 -| Z) node[xshift=3,yshift=3] {\tiny 3};
  \draw (nK.east  -| Z) node[xshift=3,yshift=3] {\tiny 2};

  \draw (sys.center) -- ++ (right:0.25) node[sec] (L) {} |- (nD.west) -- ++(right:0.01);
  \draw (L.center) |- (nC.west) -- ++(right:0.01);
  \draw (L.center |- nD) node[sec] {} |- (nE.west) -- ++(right:0.01);
  \draw (L.center |- nC) node[sec] {} |- (nB.west) -- ++(right:0.01);

  \draw (cin.center) -- (nA.west) -- ++(right:0.01);

  \draw (S1) -- (cout.center);
  \draw (S2) -- (mpout.center);

  \draw[->,>=stealth,thick] (cout.center) -- ++(right:0.9) -- ++(up:2) -- ++(left:11.55) |- (cin.center);
  \draw[->,>=stealth,thick] (mpout.center) -- ++(right:0.9);

  \draw[<-,>=stealth,thick] (mpin.center) -- ++(left:0.8);
  \draw[<-,>=stealth,thick] (tr.center) -- ++(left:0.8);
  \draw[<-,>=stealth,thick] (sys.center) -- ++(left:0.8);
  
  \draw (cout.north east) node[xshift=-2,yshift=1,anchor=west] {\scalebox{0.8}{$ \name{Cell} $}};
  \draw (mpout.north east) node[xshift=-3,yshift=1,anchor=west] {\scalebox{0.8}{$ \name{Ctrl} $}};

  \draw (mpin.north west) node[xshift=2,yshift=1,anchor=east] {\scalebox{0.8}{$ \name{MP} $}};
  \draw (cin.north west) node[xshift=2,yshift=1,anchor=east] {\scalebox{0.8}{$ \name{Cell} $}};
  \draw (tr.north west) node[xshift=2,yshift=1,anchor=east] {\scalebox{0.8}{$ \name{Tr} $}};
  \draw (sys.north west) node[xshift=2,yshift=1,anchor=east] {\scalebox{0.8}{$ \name{Sys} $}};

  \draw[gray!80,fill=white,rounded corners=3] (2.5,-2.55) rectangle (4.1,-0.55);

  \node[and,scale=0.45,anchor=west] at (2.7,-0.8) {};
  \node at (3.25,-0.8) {$ \equiv $};
  \node[anchor=west] at (3.35,-0.8) {\scalebox{0.8}{\name{and}}};

  \node[or,scale=0.45,anchor=west] at (2.7,-1.3) {};
  \node at (3.25,-1.3) {$ \equiv $};
  \node[anchor=west] at (3.35,-1.3) {\scalebox{0.8}{\name{or}}};

  \node[not,scale=0.45,anchor=west] at (2.7,-1.8) {};
  \node at (3.1,-1.8) {$ \equiv $};
  \node[anchor=west] at (3.2,-1.8) {\scalebox{0.8}{\name{not}}};

  \node[circle,draw,fill=white,anchor=west,scale=0.9] at (2.6,-2.3) {};
  \node at (3.1,-2.3) {$ \equiv $};
  \node[anchor=west] at (3.2,-2.3) {\scalebox{0.8}{\name{mutex}}};
\end{tikzpicture}

\caption{Example CFM of the music player generated from a TSL
  specification.}
\label{fig:cfmexample}
\vspace{-1em}
\end{figure}


An example CFM for our music player of \cref{sec:motiv} is depicted in
\cref{fig:cfmexample}. Inputs~$ \inames $ come from the left
and outputs~$ \onames $ leave on the right. The example
contains a single cell~$ \name{c} \in \cells $, which holds the
stateful memory~\name{Cell}, introduced during synthesis for the
module. The green, arrow shaped boxes depict vertices~$ \vertices $,
which are labeled with functions and predicates names, according
to~$ \labeling $. For the Boolean decisions that define $\delta$, we use circuit symbols for
conjunction, disjunction, and negation. Boolean decisions are piped to
a multiplexer gate that selects the respective
update streams. This allows each update stream to be passed to an output stream if and only if the
respective Boolean trigger evaluates positively, while
our construction ensures mutual exclusion on the Boolean triggers. For
code generation, the logic gates are implemented using the corresponding dedicated Boolean functions.
After building a control structure, we assign semantics to functions
and predicates by providing implementations.  To this end, we use
Functional Reactive Programming (FRP).  Prior work has established
Causal Commutative Arrows (CCA) as an FRP language pattern equivalent
of a CFM~\cite{jfp/LiuCH11,liu2007plugging,yallop2016causal}.  CCAs
are an abstraction subsumed by other functional reactive programming
abstractions, such as Monads, Applicative and
Arrows~\cite{jfp/LiuCH11,lindley2011idioms}.
There are many FRP libraries using
Monads~\cite{elm,hudakFRAN,ploeg2015frpnow},
Applicative~\cite{reactivebanana,clash2015,helbling2016juniper,Reflex},
or Arrows~\cite{courtney2003yampa,murphy2016livefrp,perez2016yampa,UISF},
and since every Monad is also an Applicative and Applicative/Arrows both are universal design patterns, we can give uniform
translations to all of these libraries using translations to just Applicative
and Arrows. Both translations are possible due to the flexible notion of a CFM.

In the last step, the synthesized FRP program is compiled into an
executable, using the provided function and predicate
implementations. This step is not fixed to a single
compiler implementation, but in fact can use any FRP compiler (or
library) that supports a language abstraction at least as expressive as CCA.
For example, instead of creating an Android music player app, we could
target an FRP web interface~\cite{Reflex} to create an online music
player, or an embedded FRP library~\cite{helbling2016juniper} that
allows us to directly instantiate the player on a computationally more
restricted device. By using the strong core of CCA, we even can go
down the whole chain and directly implement the player in hardware,
which is for example possible with the C$ \lambda $aSH  compiler~\cite{clash2015}.
Note that we still need to give separate implementations for the functions and
predicates for each target. However, our specification and the
synthesized CFM always stay the same.


\section{Experimental Results}
\label{sec:eval}
\subsection{Benchmark and Evaluation Setup}
\label{subsec:bench}

\myparagraph{Benchmark}
We have conducted a preliminary evaluation of \tool. The benchmarks used in the 
evaluation are shown in Table~\ref{tab:benchmark}. We used 8 examples 
from different application domains with \neval web assembly files in total. 
The size of benchmark files varies from 2KB to 840KB. We also counted the number of 
\wasm instructions in all files, which manifested a relatively large difference across 
different test cases. The largest one (Zxing) has over 381K instructions, while the smallest 
(Snake) only includes 528 instructions. The benchmarks and results are publicly available at \dataset.

%!TEX ROOT = ../../centralized_vs_distributed.tex

\section{{\titlecap{the centralized-distributed trade-off}}}\label{sec:numerical-results}

\revision{In the previous sections we formulated the optimal control problem for a given controller architecture
(\ie the number of links) parametrized by $ n $
and showed how to compute minimum-variance objective function and the corresponding constraints.
In this section, we present our main result:
%\red{for a ring topology with multiple options for the parameter $ n $},
we solve the optimal control problem for each $ n $ and compare the best achievable closed-loop performance with different control architectures.\footnote{
\revision{Recall that small (large) values of $ n $ mean sparse (dense) architectures.}}
For delays that increase linearly with $n$,
\ie $ f(n) \propto n $, 
we demonstrate that distributed controllers with} {few communication links outperform controllers with larger number of communication links.}

\textcolor{subsectioncolor}{Figure~\ref{fig:cont-time-single-int-opt-var}} shows the steady-state variances
obtained with single-integrator dynamics~\eqref{eq:cont-time-single-int-variance-minimization}
%where we compare the standard multi-parameter design 
%with a simplified version \tcb{that utilizes spatially-constant feedback gains
and the quadratic approximation~\eqref{eq:quadratic-approximation} for \revision{ring topology}
with $ N = 50 $ nodes. % and $ n\in\{1,\dots,10\} $.
%with $ N = 50 $, $ f(n) = n $ and $ \tau_{\textit{min}} = 0.1 $.
%\autoref{fig:cont-time-single-int-err} shows the relative error, defined as
%\begin{equation}\label{eq:relative-error}
%	e \doteq \dfrac{\optvarx-\optvar}{\optvar}
%\end{equation}
%where $ \optvar $ and $ \optvarx $ denote the the optimal and sub-optimal scalar variances, respectively.
%The performance gap is small
%and becomes negligible for large $ n $.
{The best performance is achieved for a sparse architecture with  $ n = 2 $ 
in which each agent communicates with the two closest pairs of neighboring nodes. 
This should be compared and contrasted to nearest-neighbor and all-to-all 
communication topologies which induce higher closed-loop variances. 
Thus, 
the advantage of introducing additional communication links diminishes 
beyond}
{a certain threshold because of communication delays.}

%For a linear increase in the delay,
\textcolor{subsectioncolor}{Figure~\ref{fig:cont-time-double-int-opt-var}} shows that the use of approximation~\eqref{eq:cont-time-double-int-min-var-simplified} with $ \tilde{\gvel}^* = 70 $
identifies nearest-neighbor information exchange as the {near-optimal} architecture for a double-integrator model
with ring topology. 
This can be explained by noting that the variance of the process noise $ n(t) $
in the reduced model~\eqref{eq:x-dynamics-1st-order-approximation}
is proportional to $ \nicefrac{1}{\gvel} $ and thereby to $ \taun $,
according to~\eqref{eq:substitutions-4-normalization},
making the variance scale with the delay.

%\mjmargin{i feel that we need to comment about different results that we obtained for CT and DT double-intergrator dynamics (monotonic deterioration of performance for the former and oscillations for the latter)}
\revision{\textcolor{subsectioncolor}{Figures~\ref{fig:disc-time-single-int-opt-var}--\ref{fig:disc-time-double-int-opt-var}}
show the results obtained by solving the optimal control problem for discrete-time dynamics.
%which exhibit similar trade-offs.
The oscillations about the minimum in~\autoref{fig:disc-time-double-int-opt-var}
are compatible with the investigated \tradeoff~\eqref{eq:trade-off}:
in general, 
the sum of two monotone functions does not have a unique local minimum.
Details about discrete-time systems are deferred to~\autoref{sec:disc-time}.
Interestingly,
double integrators with continuous- (\autoref{fig:cont-time-double-int-opt-var}) ad discrete-time (\autoref{fig:disc-time-double-int-opt-var}) dynamics
exhibits very different trade-off curves,
whereby performance monotonically deteriorates for the former and oscillates for the latter.
While a clear interpretation is difficult because there is no explicit expression of the variance as a function of $ n $,
one possible explanation might be the first-order approximation used to compute gains in the continuous-time case.
%which reinforce our thesis exposed in~\autoref{sec:contribution}.

%\begin{figure}
%	\centering
%	\includegraphics[width=.6\linewidth]{cont-time-double-int-opt-var-n}
%	\caption{Steady-state scalar variance for continuous-time double integrators with $ \taun = 0.1n $.
%		Here, the \tradeoff is optimized by nearest-neighbor interaction.
%	}
%	\label{fig:cont-time-double-int-opt-var-lin}
%\end{figure}
}

\begin{figure}
	\centering
	\begin{minipage}[l]{.5\linewidth}
		\centering
		\includegraphics[width=\linewidth]{random-graph}
	\end{minipage}%
	\begin{minipage}[r]{.5\linewidth}
		\centering
		\includegraphics[width=\linewidth]{disc-time-single-int-random-graph-opt-var}
	\end{minipage}
	\caption{Network topology and its optimal {closed-loop} variance.}
	\label{fig:general-graph}
\end{figure}

Finally,
\autoref{fig:general-graph} shows the optimization results for a random graph topology with discrete-time single integrator agents. % with a linear increase in the delay, $ \taun = n $.
Here, $ n $ denotes the number of communication hops in the ``original" network, shown in~\autoref{fig:general-graph}:
as $ n $ increases, each agent can first communicate with its nearest neighbors,
then with its neighbors' neighbors, and so on. For a control architecture that utilizes different feedback gains for each communication link
	(\ie we only require $ K = K^\top $) we demonstrate that, in this case, two communication hops provide optimal closed-loop performance. % of the system.}

Additional computational experiments performed with different rates $ f(\cdot) $ show that the optimal number of links increases for slower rates: 
for example, 
the optimal number of links is larger for $ f(n) = \sqrt{n} $ than for $ f(n) = n $. 
\revision{These results are not reported because of space limitations.}

\myparagraph{Evaluation Setup}
All the experiments were performed on two environments, \ie, with and without \tee, respectively. 
The \tee environment was set up on Google Cloud confidential computing 
platform\footnote{https://cloud.google.com/confidential-computing}, which uses 
the AMD Epyc processor and SEV-ES\footnote{https://developer.amd.com/sev/} as the underlying TEE setting. 
The cloud machine was configured with dual 2.25GHz cores, 8GB memory and a Ubuntu 18.04 
operating system. Moreover, the non-\tee environment carried Intel i9 processor with dual 2.3GHz 
cores, 8GB memory and a Ubuntu 18.04 operating system. All the computation was executed by only 
one core of the processor in our evaluation.

\subsection{Evaluation Results}
\label{subsec:results}
In the evaluation, we ran \tool with and without the support of TEE (\ie, AMD SEV 
in our case) on all benchmarks. For a test file $f$, \tool performs a systematic 
program analysis based on symbolic execution~\cite{king1976symbolic} to explore the 
state space of $f$ and create a semantic abstraction as well. That said, the execution 
did not include detection of specific types of bugs or errors, as in many existing program 
analyzers. The goal of this evaluation was to understand the performance trade-off with the 
design of \tcpa, rather than assessing the effectiveness of a certain bug-detection algorithm.
With a framework such as \tool, the implementation of a detector is a straightforward task 
even in the context of \tcpa.




\begin{table}[htbp]
\centering
\caption{Time cost. s: second.}\label{tab:time}
\begin{tabular}{c r r r}
\toprule
%\multicolumn{1}{c}{\bf DeFi} & \multicolumn{1}{c}{\bf Codefi Inspect} & \multicolumn{1}{c}{\bf \tool}\\
\textbf{File} & \textbf{TEE (s)} & \textbf{Non-TEE (s)} & \textbf{Overhead} \\
\midrule

module1 & 67.7 & 36.3 & 85.4\% \\

avif\_dec & 1,823.6 & 559.6 & 225.9\% \\

imagequant & 493.0 & 105.4 & 367.5\% \\

mozjpeg\_enc & 659.7 & 338.2 & 95.1\% \\

rotate & 358.4 & 173.1 & 107.0\% \\

zxing & 1,427.5 & 530.4 & 169.3\% \\

tfjs-backend & 712.9 & 661.1 & 7.7\% \\

stdio & 53.5 & 20.4 & 162.3\% \\

string & 26.5 & 10.5 & 152.4\% \\

memory & 117.9 & 46.0 & 156.3\% \\

asm-dom & 382.5 & 117.5 & 225.5\% \\

binjgb & 155.5 & 136.7 & 13.8\% \\

maze & 2.2 & 1.0 & 120.0\%\\

snake & 0.5 & 0.3 & 61.3\%\\

\midrule
average & $\ast$ & $\ast$ & 139.3\% \\

%module1 & 164.5 & 111.7 & 40.2\% \\
%
%avif\_dec & 3,775.7 & 2,067.5 & 82.6\% \\
%
%imagequant & 278.7 & 195.8 & 42.3\% \\
%
%mozjpeg\_enc & 323.4 & 246.1 & 31.3\% \\
%
%rotate & 181.8 & 142.4 & 27.5\% \\
%
%Zxing & 1,687.4 & 903.3 & 86,8\% \\
%
%tfjs-backend & 19,573.3 & 11,079.7 & 76.7\% \\
%
%stdio & 171.2 & 123.8 & 38.3\% \\
%
%asm-dom & 2,791.5 & 2,189.1 & 27.4\% \\
%
%binjgb & 274.0 & 214.2 & 27.9\% \\
%
%maze & 2.1 & 1.8 & 16.7\%\\
%
%snake & 6.8 & 5.9 & 15.3\%\\

\bottomrule
\end{tabular}
\end{table}

\myparagraph{Time Overhead}
The time cost with and without a TEE is described in Table~\ref{tab:time}. 
In the evaluation, we observed the smallest 7.7\% overhead in the case of \texttt{tfjs-backend}, 
while the largest case was 367.5\% for \texttt{imagequant}. The average time overhead was 
139.3\% on all benchmark files. For a subset of the test cases, files with larger sizes introduced 
bigger time overhead as expected. For instance, \texttt{binjgb}, \texttt{module1}, \texttt{avif\_dec} 
led to an increasing level of overhead with a growing size and number of instructions. However, 
there were exceptions in the evaluation where big files manifested small overheads. For example, 
in the case of \texttt{mozjpeg\_enc} (217 KB), running \tool is 95.1\% slower than the non-TEE version. 
For the case of \texttt{rotate} (14 KB) which is only 6.5\% as large as \texttt{mozjpeg\_enc}, the 
overhead was 107.0\% that amounts to a relative 12.5\% growth. Further discussions on root causes of 
the overhead can be found below.

%Specifically, 
%as the size of the benchmark file increased, the overhead introduced by \tool grows 
%accordingly. In our evaluation, the smallest overhead was 15.3\%, while the largest 
%was 86.8\%. The average overhead of \tool was 42.8\%, which is acceptable in practical 
%applications.

\begin{table}[htbp]
\centering
\caption{Memory cost. MB: megabyte.}\label{tab:memory}
\begin{tabular}{c r r r}
\toprule
%\multicolumn{1}{c}{\bf DeFi} & \multicolumn{1}{c}{\bf Codefi Inspect} & \multicolumn{1}{c}{\bf \tool}\\
\textbf{File} & \textbf{TEE (MB)} & \textbf{Non-TEE (MB)} & \textbf{Overhead} \\
\midrule

module1 & 6.4 & 5.7 & 12.3\% \\

avif\_dec & 80.2 & 59.5 & 34.8\% \\

imagequant & 51.2 & 10.8 & 374.1\% \\

mozjpeg\_enc & 78.5 & 75.9 & 3.4\% \\

rotate & 15.7 & 14.8 & 6.1\% \\

zxing & 106.0 & 36.4 & 191.2\% \\

tfjs-backend & 65.4 & 65.6 & -0.3\% \\

stdio & 11.7 & 10.9 & 7.3\% \\

string & 6.1 & 5.7 & 7.0\% \\

memory & 15.2 & 14.2 & 7.0\% \\

asm-dom & 52.3 & 49.4 & 5.9\% \\

binjgb & 20.6 & 14.2 & 45.1 \% \\

maze & 2.8 & 2.0 & 40.0\% \\

snake & 0.8 & 0.6 & 33.3\% \\

\midrule
average & $\ast$ & $\ast$ & 54.8\% \\

\bottomrule
\end{tabular}
\end{table}


\myparagraph{Memory Overhead}
In addition to time cost, we analyzed the memory overhead with \tool in our evaluation. 
Similarly, the analysis was conducted with and without the support of TEE, 
as shown in Table~\ref{tab:memory}. In general, majority of the overheads were below 45\%. 
More specifically, the overheads for half of the test cases were even less than 8\%, which 
we believe is highly acceptable in practical scenarios. On the other hand, there were two cases 
that manifested a 2X and 4X overheads, although the actual memory used were not big, \ie, 10.8MB 
and 36.4MB respectively. The \texttt{tfjs-backend} file was particularly interesting due to the 
fact that \tool consumed less memory with a TEE than the non-TEE version. Detailed explanations 
are given in the following section.


\subsection{Discussions}
\label{subsec:discuss}
We now describe a further discussion on the empirical results with \tool to help understand its 
performance manifested in the evaluation. 

\myparagraph{General Explanation}
First of all, the runtime overheads in general introduced by \tool in our evaluation are easy to understand. 
In the case of time overhead, the execution of \tcpa protocol was encapsulated in a \tee environment, 
which encrypts and decrypts memory accessed by the running program, \ie, in our case the \tool implementation, 
therefore should last longer than running \tcpa without a \tee (139.3\% as described in Table~\ref{tab:time}), 
depending on how efficient the \tee is realized. 
On the other hand, \tool did not manifest a higher level of memory consumption than a non-\tee implementation 
for the majority of test cases used in the evaluation as shown in Table~\ref{tab:memory}, due to the fact that 
encryption and decryption of memory in a \tee are not memory-intensive procedures thus commonly require little 
extra memory in running \tcpa.

\myparagraph{Special Cases}
Despite the general analysis of evaluation results, we did observe that there were exceptions that seemed not to
be consistent with other cases. As shown in Table~\ref{tab:time}, it was much slower for \tool to process 
\texttt{avif\_dec}, \texttt{imagequant} and \texttt{asm-dom} than other test files. The average time overhead for 
the three is 273.0\% which almost doubles the number of total average. As explained above, the time overhead is 
mainly resulted from encryption and decryption of memory used by \tool. More specifically, the overhead is closely 
correlated to the memory complexity of analysis (\eg, the amount of memory used and the frequency to access it) 
adopted in the \tcpa realization, \ie, a symbolic-execution based analysis. Like many other well-designed symbolic 
engines, \tool uses a variety of specific data structures to store intermediate information of program analysis, 
\eg, states of analysis, symbolic contexts, path conditions, \etc Particularly, \tool introduced a graph-based structure to 
separate the modeling of a given program and its symbolic execution process. While the advantage of such design is 
to have better composability via integration with different symbolic execution engines and backend analyzers, it 
inevitably increases the level of memory consumption and access frequency. Moreover, abnormal time overheads were 
partially attributed to the evaluation setting as well. We use an illustrative example in Figure~\ref{fig:setting} 
to explain the cause. 

\begin{figure}[h]
\centering
\includegraphics[width=.85\linewidth]{coverage.pdf}
\caption{\label{fig:setting}An evaluation setting of covering two paths with a specified timeout on each path.}	
\end{figure}

Figure~\ref{fig:setting} describes an evaluation setting to cover two paths in the given program and the exploration 
of each path is bounded within a specified timeout, \eg, one second. Although settings may vary across different 
program analyzers to deal with specific use cases, they commonly share similar fundamental parameters, \eg, level of 
coverage and timeout for SMT solving. In particular, Figure~\ref{fig:setting} demonstrates a scenario where setting 
overhead is introduced. Specifically, a program analyzer without \tee (left) manages to cover the first and second 
paths of a given program and then finishes the analysis without exploring the remaining paths. However, in the case 
on the right, the program analyzer with \tee (right) manages to cover the first path of a given program, fail at 
the second and third due to timeouts of SMT solving, and then cover the last. In such cases, although overheads 
on two visited paths are relatively small, the total overhead becomes much bigger because of unfinished explorations 
on the other two paths.




\begin{table*}[htbp]
\centering
\caption{The preliminary performance validation with simple programs.}\label{tab:validate}
\begin{tabular}{c r r r r r r}
\toprule
%\multicolumn{1}{c}{\bf DeFi} & \multicolumn{1}{c}{\bf Codefi Inspect} & \multicolumn{1}{c}{\bf \tool}\\
\multirow{2}{*}{\textbf{File}} & \multicolumn{3}{c}{\textbf{Time (second)}} & \multicolumn{3}{c}{\textbf{Memory (MB)}} \\

 & \textbf{TEE} & \textbf{Non-TEE} & \textbf{Overhead} & \textbf{TEE} & \textbf{Non-TEE} & \textbf{Overhead}\\
\midrule

self\_addition & 14.2 & 12.5 & 13.6\% & 0.7 & 0.5 & 40.0\%\\

array\_addition & 3.6 & 0.8 & 350.0\% & 389.3 & 389.2 & 0.03\% \\

quick\_sort & 134.4 & 32.4 & 314.8\% & 387.4 & 392.5 & -1.3\%\\

constraint\_addition & 0.4 & 0.3 & 33.3\% & 128.6 & 128.2 & 0.3\%\\

constraint\_division & 5.0 & 2.6 & 92.3\% & 300.1 & 302.8 & -0.9\%\\

\bottomrule
\end{tabular}
\end{table*}

In terms of memory overhead, the evaluation manifested abnormal results as well. Specifically, \texttt{imagequant} 
and \texttt{zxing} introduced a large overhead while \texttt{tfjs-backend} even showed a negative overhead, \ie, 
\tool was faster than the non-\tee version. Commonly, \tcpa does not introduce a high level of memory overhead because 
the encryption process, \eg, AES as used in our case with AMD SEV, often generates ciphertexts with similar sizes as 
plaintexts. However, there might be cases as well where ciphertexts are bigger with specific padding strategies. Another 
factor to potentially affect the measurement of memory overhead is garbage collection in virtual machines. For cases where 
memory consumption is measured right after a garbage collection process, we might have a much smaller number than expected. 
Further investigation on such cases is left as future work.

\myparagraph{Preliminary Validation}
Since the implementation of \tool is non-trivial, we conducted a preliminary validation with a small group of 
simple test cases to justify the root cause analysis as described above. The validation is shown in Table~\ref{tab:validate}.



Specifically, the test cases used in the validation included the following programs:
\begin{itemize}[leftmargin=*]
\item \texttt{self\_addition}: increment a variable $10^7$ times with a specific value

\item \texttt{array\_addition}: add to $10^7$ elements in a given array

\item \texttt{quick\_sort}: quick sort a given list

\item \texttt{constraint\_addition}: a program with an addition constraint for SMT solving

\item \texttt{constraint\_division}: a program with an division constraint for SMT solving
\end{itemize}

As shown in the first two rows of Table~\ref{tab:validate}, \texttt{self\_addition} manifested 
a relatively lower level of time overhead (13.6\%) compared to \texttt{array\_addition} (350.0\%). The gap 
is resulted from different structures of memory accessed by both programs. While 
\texttt{self\_addition} only manipulated a single unit of memory, \texttt{array\_addition} 
is allocated with a consecutive memory space therefore each access to it requires addressing 
with the starting point and the offset. As a result, running \texttt{array\_addition} with \tee 
was much slower than without \tee due to encryption of a more complicated memory. In the case 
of \texttt{self\_addition}, \tee did not slow down too much of the execution. Furthermore, a similar 
explanation can apply to \texttt{quick\_sort}, \ie, the third row of Table~\ref{tab:validate}. Since 
the memory used by a quick sorting algorithm commonly includes a pivot and sub-lists of a given list, 
it introduced a high level of time overhead (314.8\%) as \texttt{array\_addition}. Moreover, the last 
two rows of Table~\ref{tab:validate} demonstrated two cases with simple and complicated path 
constraints for SMT solving, respectively. While \texttt{constraint\_addition} generated a 
constraint with addition, \texttt{constraint\_division} was a division constraint. Therefore, it took 
longer for \tool to solve \texttt{constraint\_division} than \texttt{constraint\_addition}. As 
explained in Figure~\ref{fig:setting}, \tool managed to solve the division constraint without \tee 
but failed with \tee due to timeout (which could be verified based on runtime logs). 
Therefore, the time overhead in the forth row is larger, \ie, 92.3\%. On the 
other hand, the addition constraint can be solved with and without \tee thus did not manifest a large 
overhead in the last row. In terms of memory, all cases introduced slight overheads, which can be 
explained by the fact that the memory encryption process enforced by \tee (\ie, AMD SEV) 
did not require much extra memory space.

\section{Related Work}
\label{sec:related}
\section{Related Work}\label{sec:related}
 
The authors in \cite{humphreys2007noncontact} showed that it is possible to extract the PPG signal from the video using a complementary metal-oxide semiconductor camera by illuminating a region of tissue using through external light-emitting diodes at dual-wavelength (760nm and 880nm).  Further, the authors of  \cite{verkruysse2008remote} demonstrated that the PPG signal can be estimated by just using ambient light as a source of illumination along with a simple digital camera.  Further in \cite{poh2011advancements}, the PPG waveform was estimated from the videos recorded using a low-cost webcam. The red, green, and blue channels of the images were decomposed into independent sources using independent component analysis. One of the independent sources was selected to estimate PPG and further calculate HR, and HRV. All these works showed the possibility of extracting PPG signals from the videos and proved the similarity of this signal with the one obtained using a contact device. Further, the authors in \cite{10.1109/CVPR.2013.440} showed that heart rate can be extracted from features from the head as well by capturing the subtle head movements that happen due to blood flow.

%
The authors of \cite{kumar2015distanceppg} proposed a methodology that overcomes a challenge in extracting PPG for people with darker skin tones. The challenge due to slight movement and low lighting conditions during recording a video was also addressed. They implemented the method where PPG signal is extracted from different regions of the face and signal from each region is combined using their weighted average making weights different for different people depending on their skin color. 
%

There are other attempts where authors of \cite{6523142,6909939, 7410772, 7412627} have introduced different methodologies to make algorithms for estimating pulse rate robust to illumination variation and motion of the subjects. The paper \cite{6523142} introduces a chrominance-based method to reduce the effect of motion in estimating pulse rate. The authors of \cite{6909939} used a technique in which face tracking and normalized least square adaptive filtering is used to counter the effects of variations due to illumination and subject movement. 
The paper \cite{7410772} resolves the issue of subject movement by choosing the rectangular ROI's on the face relative to the facial landmarks and facial landmarks are tracked in the video using pose-free facial landmark fitting tracker discussed in \cite{yu2016face} followed by the removal of noise due to illumination to extract noise-free PPG signal for estimating pulse rate. 

Recently, the use of machine learning in the prediction of health parameters have gained attention. The paper \cite{osman2015supervised} used a supervised learning methodology to predict the pulse rate from the videos taken from any off-the-shelf camera. Their model showed the possibility of using machine learning methods to estimate the pulse rate. However, our method outperforms their results when the root mean squared error of the predicted pulse rate is compared. The authors in \cite{hsu2017deep} proposed a deep learning methodology to predict the pulse rate from the facial videos. The researchers trained a convolutional neural network (CNN) on the images generated using Short-Time Fourier Transform (STFT) applied on the R, G, \& B channels from the facial region of interests.
The authors of \cite{osman2015supervised, hsu2017deep} only predicted pulse rate, and we extended our work in predicting variance in the pulse rate measurements as well.

All the related work discussed above utilizes filtering and digital signal processing to extract PPG signals from the video which is further used to estimate the PR and PRV.  %
The method proposed in \cite{kumar2015distanceppg} is person dependent since the weights will be different for people with different skin tone. In contrast, we propose a deep learning model to predict the PR which is independent of the person who is being trained. Thus, the model would work even if there is no prior training model built for that individual and hence, making our model robust. 

%

\section{Conclusions}
\label{sec:conclusions}
\section{Conclusions}
\label{sec:conclusions}

In this paper, we apply shared-workload techniques at the \sql level for
improving the throughput of \qaasl systems without incurring in additional
query execution costs. Our approach is based on query rewriting for grouping
multiple queries together into a single query to be executed in one go. This
results in a significant reduction of the aggregated data access done by the
shared execution compared to executing queries independently.

%execution times and costs of the shared scan operator when
%varying query selectivity and predicate evaluation. We observed that for
%\athena, although the cost only depends on the amount of data read, it is
%conditioned to its ability to use its statistics about the data. In some cases
%a wrong query execution plan leads to higher query execution costs, which the
%end-user has to pay. 

%\bigquery's minimum query execution cost is determined by
%the input size of a query.  However, the query cost can increase depending not
%just in the amount of computation it requires, but in the mix of resources the
%query requires.  

We presented a cost and runtime evaluation of the shared operator driving data access costs. 
Our experimental study using the TPC-H benchmark confirmed the benefits of our
query rewrite approach. Using a shared execution approach reduces significantly
the execution costs. For \athena, we are able to make it 107x cheaper and for
\bigquery, 16x cheaper taking into account Query 10 which we cannot execute,
but 128x if it is not taken into account. Moreover, when having queries that do
not share their entire execution plan, i.e., using a single global plan, we
demonstrated that it is possible to improve throughput and obtain a 10x cost
reduction in \bigquery.

%We followed the TPC systems pricing guideline for
%computing how expensive is to have a TPC-H workload working on the evaluated
%\qaasl systems. The result is that even though we are able to reduce overall
%costs a TPC-H workload in 15x for \bigquery (128x excluding query 10 which we
%could not optimize) and in 107x for \athena, the overall price is at least 10x
%more expensive than the cheapest system price published by the TPC.

There are multiple ways to extend our work. The first is
to implement a full \sql-to-\sql translation layer to encapsulate the proposed
per-operator rewrites.  Another one is to incorporate the initial work on
building a cost-based optimizer for shared execution
\cite{Giannikis:2014:SWO:2732279.2732280} as an external component for \qaasl
systems.  Moreover, incorporating different lines of work (e.g., adding
provenance computation \cite{GA09} capabilities) also based on query
rewriting is part of our future work to enhance our system.


%% Bibliography
\bibliographystyle{splncs}
\bibliography{biblio}

\newpage

\appendix
\section{Appendix}
\chapter{Supplementary Material}
\label{appendix}

In this appendix, we present supplementary material for the techniques and
experiments presented in the main text.

\section{Baseline Results and Analysis for Informed Sampler}
\label{appendix:chap3}

Here, we give an in-depth
performance analysis of the various samplers and the effect of their
hyperparameters. We choose hyperparameters with the lowest PSRF value
after $10k$ iterations, for each sampler individually. If the
differences between PSRF are not significantly different among
multiple values, we choose the one that has the highest acceptance
rate.

\subsection{Experiment: Estimating Camera Extrinsics}
\label{appendix:chap3:room}

\subsubsection{Parameter Selection}
\paragraph{Metropolis Hastings (\MH)}

Figure~\ref{fig:exp1_MH} shows the median acceptance rates and PSRF
values corresponding to various proposal standard deviations of plain
\MH~sampling. Mixing gets better and the acceptance rate gets worse as
the standard deviation increases. The value $0.3$ is selected standard
deviation for this sampler.

\paragraph{Metropolis Hastings Within Gibbs (\MHWG)}

As mentioned in Section~\ref{sec:room}, the \MHWG~sampler with one-dimensional
updates did not converge for any value of proposal standard deviation.
This problem has high correlation of the camera parameters and is of
multi-modal nature, which this sampler has problems with.

\paragraph{Parallel Tempering (\PT)}

For \PT~sampling, we took the best performing \MH~sampler and used
different temperature chains to improve the mixing of the
sampler. Figure~\ref{fig:exp1_PT} shows the results corresponding to
different combination of temperature levels. The sampler with
temperature levels of $[1,3,27]$ performed best in terms of both
mixing and acceptance rate.

\paragraph{Effect of Mixture Coefficient in Informed Sampling (\MIXLMH)}

Figure~\ref{fig:exp1_alpha} shows the effect of mixture
coefficient ($\alpha$) on the informed sampling
\MIXLMH. Since there is no significant different in PSRF values for
$0 \le \alpha \le 0.7$, we chose $0.7$ due to its high acceptance
rate.


% \end{multicols}

\begin{figure}[h]
\centering
  \subfigure[MH]{%
    \includegraphics[width=.48\textwidth]{figures/supplementary/camPose_MH.pdf} \label{fig:exp1_MH}
  }
  \subfigure[PT]{%
    \includegraphics[width=.48\textwidth]{figures/supplementary/camPose_PT.pdf} \label{fig:exp1_PT}
  }
\\
  \subfigure[INF-MH]{%
    \includegraphics[width=.48\textwidth]{figures/supplementary/camPose_alpha.pdf} \label{fig:exp1_alpha}
  }
  \mycaption{Results of the `Estimating Camera Extrinsics' experiment}{PRSFs and Acceptance rates corresponding to (a) various standard deviations of \MH, (b) various temperature level combinations of \PT sampling and (c) various mixture coefficients of \MIXLMH sampling.}
\end{figure}



\begin{figure}[!t]
\centering
  \subfigure[\MH]{%
    \includegraphics[width=.48\textwidth]{figures/supplementary/occlusionExp_MH.pdf} \label{fig:exp2_MH}
  }
  \subfigure[\BMHWG]{%
    \includegraphics[width=.48\textwidth]{figures/supplementary/occlusionExp_BMHWG.pdf} \label{fig:exp2_BMHWG}
  }
\\
  \subfigure[\MHWG]{%
    \includegraphics[width=.48\textwidth]{figures/supplementary/occlusionExp_MHWG.pdf} \label{fig:exp2_MHWG}
  }
  \subfigure[\PT]{%
    \includegraphics[width=.48\textwidth]{figures/supplementary/occlusionExp_PT.pdf} \label{fig:exp2_PT}
  }
\\
  \subfigure[\INFBMHWG]{%
    \includegraphics[width=.5\textwidth]{figures/supplementary/occlusionExp_alpha.pdf} \label{fig:exp2_alpha}
  }
  \mycaption{Results of the `Occluding Tiles' experiment}{PRSF and
    Acceptance rates corresponding to various standard deviations of
    (a) \MH, (b) \BMHWG, (c) \MHWG, (d) various temperature level
    combinations of \PT~sampling and; (e) various mixture coefficients
    of our informed \INFBMHWG sampling.}
\end{figure}

%\onecolumn\newpage\twocolumn
\subsection{Experiment: Occluding Tiles}
\label{appendix:chap3:tiles}

\subsubsection{Parameter Selection}

\paragraph{Metropolis Hastings (\MH)}

Figure~\ref{fig:exp2_MH} shows the results of
\MH~sampling. Results show the poor convergence for all proposal
standard deviations and rapid decrease of AR with increasing standard
deviation. This is due to the high-dimensional nature of
the problem. We selected a standard deviation of $1.1$.

\paragraph{Blocked Metropolis Hastings Within Gibbs (\BMHWG)}

The results of \BMHWG are shown in Figure~\ref{fig:exp2_BMHWG}. In
this sampler we update only one block of tile variables (of dimension
four) in each sampling step. Results show much better performance
compared to plain \MH. The optimal proposal standard deviation for
this sampler is $0.7$.

\paragraph{Metropolis Hastings Within Gibbs (\MHWG)}

Figure~\ref{fig:exp2_MHWG} shows the result of \MHWG sampling. This
sampler is better than \BMHWG and converges much more quickly. Here
a standard deviation of $0.9$ is found to be best.

\paragraph{Parallel Tempering (\PT)}

Figure~\ref{fig:exp2_PT} shows the results of \PT sampling with various
temperature combinations. Results show no improvement in AR from plain
\MH sampling and again $[1,3,27]$ temperature levels are found to be optimal.

\paragraph{Effect of Mixture Coefficient in Informed Sampling (\INFBMHWG)}

Figure~\ref{fig:exp2_alpha} shows the effect of mixture
coefficient ($\alpha$) on the blocked informed sampling
\INFBMHWG. Since there is no significant different in PSRF values for
$0 \le \alpha \le 0.8$, we chose $0.8$ due to its high acceptance
rate.



\subsection{Experiment: Estimating Body Shape}
\label{appendix:chap3:body}

\subsubsection{Parameter Selection}
\paragraph{Metropolis Hastings (\MH)}

Figure~\ref{fig:exp3_MH} shows the result of \MH~sampling with various
proposal standard deviations. The value of $0.1$ is found to be
best.

\paragraph{Metropolis Hastings Within Gibbs (\MHWG)}

For \MHWG sampling we select $0.3$ proposal standard
deviation. Results are shown in Fig.~\ref{fig:exp3_MHWG}.


\paragraph{Parallel Tempering (\PT)}

As before, results in Fig.~\ref{fig:exp3_PT}, the temperature levels
were selected to be $[1,3,27]$ due its slightly higher AR.

\paragraph{Effect of Mixture Coefficient in Informed Sampling (\MIXLMH)}

Figure~\ref{fig:exp3_alpha} shows the effect of $\alpha$ on PSRF and
AR. Since there is no significant differences in PSRF values for $0 \le
\alpha \le 0.8$, we choose $0.8$.


\begin{figure}[t]
\centering
  \subfigure[\MH]{%
    \includegraphics[width=.48\textwidth]{figures/supplementary/bodyShape_MH.pdf} \label{fig:exp3_MH}
  }
  \subfigure[\MHWG]{%
    \includegraphics[width=.48\textwidth]{figures/supplementary/bodyShape_MHWG.pdf} \label{fig:exp3_MHWG}
  }
\\
  \subfigure[\PT]{%
    \includegraphics[width=.48\textwidth]{figures/supplementary/bodyShape_PT.pdf} \label{fig:exp3_PT}
  }
  \subfigure[\MIXLMH]{%
    \includegraphics[width=.48\textwidth]{figures/supplementary/bodyShape_alpha.pdf} \label{fig:exp3_alpha}
  }
\\
  \mycaption{Results of the `Body Shape Estimation' experiment}{PRSFs and
    Acceptance rates corresponding to various standard deviations of
    (a) \MH, (b) \MHWG; (c) various temperature level combinations
    of \PT sampling and; (d) various mixture coefficients of the
    informed \MIXLMH sampling.}
\end{figure}


\subsection{Results Overview}
Figure~\ref{fig:exp_summary} shows the summary results of the all the three
experimental studies related to informed sampler.
\begin{figure*}[h!]
\centering
  \subfigure[Results for: Estimating Camera Extrinsics]{%
    \includegraphics[width=0.9\textwidth]{figures/supplementary/camPose_ALL.pdf} \label{fig:exp1_all}
  }
  \subfigure[Results for: Occluding Tiles]{%
    \includegraphics[width=0.9\textwidth]{figures/supplementary/occlusionExp_ALL.pdf} \label{fig:exp2_all}
  }
  \subfigure[Results for: Estimating Body Shape]{%
    \includegraphics[width=0.9\textwidth]{figures/supplementary/bodyShape_ALL.pdf} \label{fig:exp3_all}
  }
  \label{fig:exp_summary}
  \mycaption{Summary of the statistics for the three experiments}{Shown are
    for several baseline methods and the informed samplers the
    acceptance rates (left), PSRFs (middle), and RMSE values
    (right). All results are median results over multiple test
    examples.}
\end{figure*}

\subsection{Additional Qualitative Results}

\subsubsection{Occluding Tiles}
In Figure~\ref{fig:exp2_visual_more} more qualitative results of the
occluding tiles experiment are shown. The informed sampling approach
(\INFBMHWG) is better than the best baseline (\MHWG). This still is a
very challenging problem since the parameters for occluded tiles are
flat over a large region. Some of the posterior variance of the
occluded tiles is already captured by the informed sampler.

\begin{figure*}[h!]
\begin{center}
\centerline{\includegraphics[width=0.95\textwidth]{figures/supplementary/occlusionExp_Visual.pdf}}
\mycaption{Additional qualitative results of the occluding tiles experiment}
  {From left to right: (a)
  Given image, (b) Ground truth tiles, (c) OpenCV heuristic and most probable estimates
  from 5000 samples obtained by (d) MHWG sampler (best baseline) and
  (e) our INF-BMHWG sampler. (f) Posterior expectation of the tiles
  boundaries obtained by INF-BMHWG sampling (First 2000 samples are
  discarded as burn-in).}
\label{fig:exp2_visual_more}
\end{center}
\end{figure*}

\subsubsection{Body Shape}
Figure~\ref{fig:exp3_bodyMeshes} shows some more results of 3D mesh
reconstruction using posterior samples obtained by our informed
sampling \MIXLMH.

\begin{figure*}[t]
\begin{center}
\centerline{\includegraphics[width=0.75\textwidth]{figures/supplementary/bodyMeshResults.pdf}}
\mycaption{Qualitative results for the body shape experiment}
  {Shown is the 3D mesh reconstruction results with first 1000 samples obtained
  using the \MIXLMH informed sampling method. (blue indicates small
  values and red indicates high values)}
\label{fig:exp3_bodyMeshes}
\end{center}
\end{figure*}

\clearpage



\section{Additional Results on the Face Problem with CMP}

Figure~\ref{fig:shading-qualitative-multiple-subjects-supp} shows inference results for reflectance maps, normal maps and lights for randomly chosen test images, and Fig.~\ref{fig:shading-qualitative-same-subject-supp} shows reflectance estimation results on multiple images of the same subject produced under different illumination conditions. CMP is able to produce estimates that are closer to the groundtruth across different subjects and illumination conditions.

\begin{figure*}[h]
  \begin{center}
  \centerline{\includegraphics[width=1.0\columnwidth]{figures/face_cmp_visual_results_supp.pdf}}
  \vspace{-1.2cm}
  \end{center}
	\mycaption{A visual comparison of inference results}{(a)~Observed images. (b)~Inferred reflectance maps. \textit{GT} is the photometric stereo groundtruth, \textit{BU} is the Biswas \etal (2009) reflectance estimate and \textit{Forest} is the consensus prediction. (c)~The variance of the inferred reflectance estimate produced by \MTD (normalized across rows).(d)~Visualization of inferred light directions. (e)~Inferred normal maps.}
	\label{fig:shading-qualitative-multiple-subjects-supp}
\end{figure*}


\begin{figure*}[h]
	\centering
	\setlength\fboxsep{0.2mm}
	\setlength\fboxrule{0pt}
	\begin{tikzpicture}

		\matrix at (0, 0) [matrix of nodes, nodes={anchor=east}, column sep=-0.05cm, row sep=-0.2cm]
		{
			\fbox{\includegraphics[width=1cm]{figures/sample_3_4_X.png}} &
			\fbox{\includegraphics[width=1cm]{figures/sample_3_4_GT.png}} &
			\fbox{\includegraphics[width=1cm]{figures/sample_3_4_BISWAS.png}}  &
			\fbox{\includegraphics[width=1cm]{figures/sample_3_4_VMP.png}}  &
			\fbox{\includegraphics[width=1cm]{figures/sample_3_4_FOREST.png}}  &
			\fbox{\includegraphics[width=1cm]{figures/sample_3_4_CMP.png}}  &
			\fbox{\includegraphics[width=1cm]{figures/sample_3_4_CMPVAR.png}}
			 \\

			\fbox{\includegraphics[width=1cm]{figures/sample_3_5_X.png}} &
			\fbox{\includegraphics[width=1cm]{figures/sample_3_5_GT.png}} &
			\fbox{\includegraphics[width=1cm]{figures/sample_3_5_BISWAS.png}}  &
			\fbox{\includegraphics[width=1cm]{figures/sample_3_5_VMP.png}}  &
			\fbox{\includegraphics[width=1cm]{figures/sample_3_5_FOREST.png}}  &
			\fbox{\includegraphics[width=1cm]{figures/sample_3_5_CMP.png}}  &
			\fbox{\includegraphics[width=1cm]{figures/sample_3_5_CMPVAR.png}}
			 \\

			\fbox{\includegraphics[width=1cm]{figures/sample_3_6_X.png}} &
			\fbox{\includegraphics[width=1cm]{figures/sample_3_6_GT.png}} &
			\fbox{\includegraphics[width=1cm]{figures/sample_3_6_BISWAS.png}}  &
			\fbox{\includegraphics[width=1cm]{figures/sample_3_6_VMP.png}}  &
			\fbox{\includegraphics[width=1cm]{figures/sample_3_6_FOREST.png}}  &
			\fbox{\includegraphics[width=1cm]{figures/sample_3_6_CMP.png}}  &
			\fbox{\includegraphics[width=1cm]{figures/sample_3_6_CMPVAR.png}}
			 \\
	     };

       \node at (-3.85, -2.0) {\small Observed};
       \node at (-2.55, -2.0) {\small `GT'};
       \node at (-1.27, -2.0) {\small BU};
       \node at (0.0, -2.0) {\small MP};
       \node at (1.27, -2.0) {\small Forest};
       \node at (2.55, -2.0) {\small \textbf{CMP}};
       \node at (3.85, -2.0) {\small Variance};

	\end{tikzpicture}
	\mycaption{Robustness to varying illumination}{Reflectance estimation on a subject images with varying illumination. Left to right: observed image, photometric stereo estimate (GT)
  which is used as a proxy for groundtruth, bottom-up estimate of \cite{Biswas2009}, VMP result, consensus forest estimate, CMP mean, and CMP variance.}
	\label{fig:shading-qualitative-same-subject-supp}
\end{figure*}

\clearpage

\section{Additional Material for Learning Sparse High Dimensional Filters}
\label{sec:appendix-bnn}

This part of supplementary material contains a more detailed overview of the permutohedral
lattice convolution in Section~\ref{sec:permconv}, more experiments in
Section~\ref{sec:addexps} and additional results with protocols for
the experiments presented in Chapter~\ref{chap:bnn} in Section~\ref{sec:addresults}.

\vspace{-0.2cm}
\subsection{General Permutohedral Convolutions}
\label{sec:permconv}

A core technical contribution of this work is the generalization of the Gaussian permutohedral lattice
convolution proposed in~\cite{adams2010fast} to the full non-separable case with the
ability to perform back-propagation. Although, conceptually, there are minor
differences between Gaussian and general parameterized filters, there are non-trivial practical
differences in terms of the algorithmic implementation. The Gauss filters belong to
the separable class and can thus be decomposed into multiple
sequential one dimensional convolutions. We are interested in the general filter
convolutions, which can not be decomposed. Thus, performing a general permutohedral
convolution at a lattice point requires the computation of the inner product with the
neighboring elements in all the directions in the high-dimensional space.

Here, we give more details of the implementation differences of separable
and non-separable filters. In the following, we will explain the scalar case first.
Recall, that the forward pass of general permutohedral convolution
involves 3 steps: \textit{splatting}, \textit{convolving} and \textit{slicing}.
We follow the same splatting and slicing strategies as in~\cite{adams2010fast}
since these operations do not depend on the filter kernel. The main difference
between our work and the existing implementation of~\cite{adams2010fast} is
the way that the convolution operation is executed. This proceeds by constructing
a \emph{blur neighbor} matrix $K$ that stores for every lattice point all
values of the lattice neighbors that are needed to compute the filter output.

\begin{figure}[t!]
  \centering
    \includegraphics[width=0.6\columnwidth]{figures/supplementary/lattice_construction}
  \mycaption{Illustration of 1D permutohedral lattice construction}
  {A $4\times 4$ $(x,y)$ grid lattice is projected onto the plane defined by the normal
  vector $(1,1)^{\top}$. This grid has $s+1=4$ and $d=2$ $(s+1)^{d}=4^2=16$ elements.
  In the projection, all points of the same color are projected onto the same points in the plane.
  The number of elements of the projected lattice is $t=(s+1)^d-s^d=4^2-3^2=7$, that is
  the $(4\times 4)$ grid minus the size of lattice that is $1$ smaller at each size, in this
  case a $(3\times 3)$ lattice (the upper right $(3\times 3)$ elements).
  }
\label{fig:latticeconstruction}
\end{figure}

The blur neighbor matrix is constructed by traversing through all the populated
lattice points and their neighboring elements.
% For efficiency, we do this matrix construction recursively with shared computations
% since $n^{th}$ neighbourhood elements are $1^{st}$ neighborhood elements of $n-1^{th}$ neighbourhood elements. \pg{do not understand}
This is done recursively to share computations. For any lattice point, the neighbors that are
$n$ hops away are the direct neighbors of the points that are $n-1$ hops away.
The size of a $d$ dimensional spatial filter with width $s+1$ is $(s+1)^{d}$ (\eg, a
$3\times 3$ filter, $s=2$ in $d=2$ has $3^2=9$ elements) and this size grows
exponentially in the number of dimensions $d$. The permutohedral lattice is constructed by
projecting a regular grid onto the plane spanned by the $d$ dimensional normal vector ${(1,\ldots,1)}^{\top}$. See
Fig.~\ref{fig:latticeconstruction} for an illustration of the 1D lattice construction.
Many corners of a grid filter are projected onto the same point, in total $t = {(s+1)}^{d} -
s^{d}$ elements remain in the permutohedral filter with $s$ neighborhood in $d-1$ dimensions.
If the lattice has $m$ populated elements, the
matrix $K$ has size $t\times m$. Note that, since the input signal is typically
sparse, only a few lattice corners are being populated in the \textit{slicing} step.
We use a hash-table to keep track of these points and traverse only through
the populated lattice points for this neighborhood matrix construction.

Once the blur neighbor matrix $K$ is constructed, we can perform the convolution
by the matrix vector multiplication
\begin{equation}
\ell' = BK,
\label{eq:conv}
\end{equation}
where $B$ is the $1 \times t$ filter kernel (whose values we will learn) and $\ell'\in\mathbb{R}^{1\times m}$
is the result of the filtering at the $m$ lattice points. In practice, we found that the
matrix $K$ is sometimes too large to fit into GPU memory and we divided the matrix $K$
into smaller pieces to compute Eq.~\ref{eq:conv} sequentially.

In the general multi-dimensional case, the signal $\ell$ is of $c$ dimensions. Then
the kernel $B$ is of size $c \times t$ and $K$ stores the $c$ dimensional vectors
accordingly. When the input and output points are different, we slice only the
input points and splat only at the output points.


\subsection{Additional Experiments}
\label{sec:addexps}
In this section, we discuss more use-cases for the learned bilateral filters, one
use-case of BNNs and two single filter applications for image and 3D mesh denoising.

\subsubsection{Recognition of subsampled MNIST}\label{sec:app_mnist}

One of the strengths of the proposed filter convolution is that it does not
require the input to lie on a regular grid. The only requirement is to define a distance
between features of the input signal.
We highlight this feature with the following experiment using the
classical MNIST ten class classification problem~\cite{lecun1998mnist}. We sample a
sparse set of $N$ points $(x,y)\in [0,1]\times [0,1]$
uniformly at random in the input image, use their interpolated values
as signal and the \emph{continuous} $(x,y)$ positions as features. This mimics
sub-sampling of a high-dimensional signal. To compare against a spatial convolution,
we interpolate the sparse set of values at the grid positions.

We take a reference implementation of LeNet~\cite{lecun1998gradient} that
is part of the Caffe project~\cite{jia2014caffe} and compare it
against the same architecture but replacing the first convolutional
layer with a bilateral convolution layer (BCL). The filter size
and numbers are adjusted to get a comparable number of parameters
($5\times 5$ for LeNet, $2$-neighborhood for BCL).

The results are shown in Table~\ref{tab:all-results}. We see that training
on the original MNIST data (column Original, LeNet vs. BNN) leads to a slight
decrease in performance of the BNN (99.03\%) compared to LeNet
(99.19\%). The BNN can be trained and evaluated on sparse
signals, and we resample the image as described above for $N=$ 100\%, 60\% and
20\% of the total number of pixels. The methods are also evaluated
on test images that are subsampled in the same way. Note that we can
train and test with different subsampling rates. We introduce an additional
bilinear interpolation layer for the LeNet architecture to train on the same
data. In essence, both models perform a spatial interpolation and thus we
expect them to yield a similar classification accuracy. Once the data is of
higher dimensions, the permutohedral convolution will be faster due to hashing
the sparse input points, as well as less memory demanding in comparison to
naive application of a spatial convolution with interpolated values.

\begin{table}[t]
  \begin{center}
    \footnotesize
    \centering
    \begin{tabular}[t]{lllll}
      \toprule
              &     & \multicolumn{3}{c}{Test Subsampling} \\
       Method  & Original & 100\% & 60\% & 20\%\\
      \midrule
       LeNet &  \textbf{0.9919} & 0.9660 & 0.9348 & \textbf{0.6434} \\
       BNN &  0.9903 & \textbf{0.9844} & \textbf{0.9534} & 0.5767 \\
      \hline
       LeNet 100\% & 0.9856 & 0.9809 & 0.9678 & \textbf{0.7386} \\
       BNN 100\% & \textbf{0.9900} & \textbf{0.9863} & \textbf{0.9699} & 0.6910 \\
      \hline
       LeNet 60\% & 0.9848 & 0.9821 & 0.9740 & 0.8151 \\
       BNN 60\% & \textbf{0.9885} & \textbf{0.9864} & \textbf{0.9771} & \textbf{0.8214}\\
      \hline
       LeNet 20\% & \textbf{0.9763} & \textbf{0.9754} & 0.9695 & 0.8928 \\
       BNN 20\% & 0.9728 & 0.9735 & \textbf{0.9701} & \textbf{0.9042}\\
      \bottomrule
    \end{tabular}
  \end{center}
\vspace{-.2cm}
\caption{Classification accuracy on MNIST. We compare the
    LeNet~\cite{lecun1998gradient} implementation that is part of
    Caffe~\cite{jia2014caffe} to the network with the first layer
    replaced by a bilateral convolution layer (BCL). Both are trained
    on the original image resolution (first two rows). Three more BNN
    and CNN models are trained with randomly subsampled images (100\%,
    60\% and 20\% of the pixels). An additional bilinear interpolation
    layer samples the input signal on a spatial grid for the CNN model.
  }
  \label{tab:all-results}
\vspace{-.5cm}
\end{table}

\subsubsection{Image Denoising}

The main application that inspired the development of the bilateral
filtering operation is image denoising~\cite{aurich1995non}, there
using a single Gaussian kernel. Our development allows to learn this
kernel function from data and we explore how to improve using a \emph{single}
but more general bilateral filter.

We use the Berkeley segmentation dataset
(BSDS500)~\cite{arbelaezi2011bsds500} as a test bed. The color
images in the dataset are converted to gray-scale,
and corrupted with Gaussian noise with a standard deviation of
$\frac {25} {255}$.

We compare the performance of four different filter models on a
denoising task.
The first baseline model (`Spatial' in Table \ref{tab:denoising}, $25$
weights) uses a single spatial filter with a kernel size of
$5$ and predicts the scalar gray-scale value at the center pixel. The next model
(`Gauss Bilateral') applies a bilateral \emph{Gaussian}
filter to the noisy input, using position and intensity features $\f=(x,y,v)^\top$.
The third setup (`Learned Bilateral', $65$ weights)
takes a Gauss kernel as initialization and
fits all filter weights on the train set to minimize the
mean squared error with respect to the clean images.
We run a combination
of spatial and permutohedral convolutions on spatial and bilateral
features (`Spatial + Bilateral (Learned)') to check for a complementary
performance of the two convolutions.

\label{sec:exp:denoising}
\begin{table}[!h]
\begin{center}
  \footnotesize
  \begin{tabular}[t]{lr}
    \toprule
    Method & PSNR \\
    \midrule
    Noisy Input & $20.17$ \\
    Spatial & $26.27$ \\
    Gauss Bilateral & $26.51$ \\
    Learned Bilateral & $26.58$ \\
    Spatial + Bilateral (Learned) & \textbf{$26.65$} \\
    \bottomrule
  \end{tabular}
\end{center}
\vspace{-0.5em}
\caption{PSNR results of a denoising task using the BSDS500
  dataset~\cite{arbelaezi2011bsds500}}
\vspace{-0.5em}
\label{tab:denoising}
\end{table}
\vspace{-0.2em}

The PSNR scores evaluated on full images of the test set are
shown in Table \ref{tab:denoising}. We find that an untrained bilateral
filter already performs better than a trained spatial convolution
($26.27$ to $26.51$). A learned convolution further improve the
performance slightly. We chose this simple one-kernel setup to
validate an advantage of the generalized bilateral filter. A competitive
denoising system would employ RGB color information and also
needs to be properly adjusted in network size. Multi-layer perceptrons
have obtained state-of-the-art denoising results~\cite{burger12cvpr}
and the permutohedral lattice layer can readily be used in such an
architecture, which is intended future work.

\subsection{Additional results}
\label{sec:addresults}

This section contains more qualitative results for the experiments presented in Chapter~\ref{chap:bnn}.

\begin{figure*}[th!]
  \centering
    \includegraphics[width=\columnwidth,trim={5cm 2.5cm 5cm 4.5cm},clip]{figures/supplementary/lattice_viz.pdf}
    \vspace{-0.7cm}
  \mycaption{Visualization of the Permutohedral Lattice}
  {Sample lattice visualizations for different feature spaces. All pixels falling in the same simplex cell are shown with
  the same color. $(x,y)$ features correspond to image pixel positions, and $(r,g,b) \in [0,255]$ correspond
  to the red, green and blue color values.}
\label{fig:latticeviz}
\end{figure*}

\subsubsection{Lattice Visualization}

Figure~\ref{fig:latticeviz} shows sample lattice visualizations for different feature spaces.

\newcolumntype{L}[1]{>{\raggedright\let\newline\\\arraybackslash\hspace{0pt}}b{#1}}
\newcolumntype{C}[1]{>{\centering\let\newline\\\arraybackslash\hspace{0pt}}b{#1}}
\newcolumntype{R}[1]{>{\raggedleft\let\newline\\\arraybackslash\hspace{0pt}}b{#1}}

\subsubsection{Color Upsampling}\label{sec:color_upsampling}
\label{sec:col_upsample_extra}

Some images of the upsampling for the Pascal
VOC12 dataset are shown in Fig.~\ref{fig:Colour_upsample_visuals}. It is
especially the low level image details that are better preserved with
a learned bilateral filter compared to the Gaussian case.

\begin{figure*}[t!]
  \centering
    \subfigure{%
   \raisebox{2.0em}{
    \includegraphics[width=.06\columnwidth]{figures/supplementary/2007_004969.jpg}
   }
  }
  \subfigure{%
    \includegraphics[width=.17\columnwidth]{figures/supplementary/2007_004969_gray.pdf}
  }
  \subfigure{%
    \includegraphics[width=.17\columnwidth]{figures/supplementary/2007_004969_gt.pdf}
  }
  \subfigure{%
    \includegraphics[width=.17\columnwidth]{figures/supplementary/2007_004969_bicubic.pdf}
  }
  \subfigure{%
    \includegraphics[width=.17\columnwidth]{figures/supplementary/2007_004969_gauss.pdf}
  }
  \subfigure{%
    \includegraphics[width=.17\columnwidth]{figures/supplementary/2007_004969_learnt.pdf}
  }\\
    \subfigure{%
   \raisebox{2.0em}{
    \includegraphics[width=.06\columnwidth]{figures/supplementary/2007_003106.jpg}
   }
  }
  \subfigure{%
    \includegraphics[width=.17\columnwidth]{figures/supplementary/2007_003106_gray.pdf}
  }
  \subfigure{%
    \includegraphics[width=.17\columnwidth]{figures/supplementary/2007_003106_gt.pdf}
  }
  \subfigure{%
    \includegraphics[width=.17\columnwidth]{figures/supplementary/2007_003106_bicubic.pdf}
  }
  \subfigure{%
    \includegraphics[width=.17\columnwidth]{figures/supplementary/2007_003106_gauss.pdf}
  }
  \subfigure{%
    \includegraphics[width=.17\columnwidth]{figures/supplementary/2007_003106_learnt.pdf}
  }\\
  \setcounter{subfigure}{0}
  \small{
  \subfigure[Inp.]{%
  \raisebox{2.0em}{
    \includegraphics[width=.06\columnwidth]{figures/supplementary/2007_006837.jpg}
   }
  }
  \subfigure[Guidance]{%
    \includegraphics[width=.17\columnwidth]{figures/supplementary/2007_006837_gray.pdf}
  }
   \subfigure[GT]{%
    \includegraphics[width=.17\columnwidth]{figures/supplementary/2007_006837_gt.pdf}
  }
  \subfigure[Bicubic]{%
    \includegraphics[width=.17\columnwidth]{figures/supplementary/2007_006837_bicubic.pdf}
  }
  \subfigure[Gauss-BF]{%
    \includegraphics[width=.17\columnwidth]{figures/supplementary/2007_006837_gauss.pdf}
  }
  \subfigure[Learned-BF]{%
    \includegraphics[width=.17\columnwidth]{figures/supplementary/2007_006837_learnt.pdf}
  }
  }
  \vspace{-0.5cm}
  \mycaption{Color Upsampling}{Color $8\times$ upsampling results
  using different methods, from left to right, (a)~Low-resolution input color image (Inp.),
  (b)~Gray scale guidance image, (c)~Ground-truth color image; Upsampled color images with
  (d)~Bicubic interpolation, (e) Gauss bilateral upsampling and, (f)~Learned bilateral
  updampgling (best viewed on screen).}

\label{fig:Colour_upsample_visuals}
\end{figure*}

\subsubsection{Depth Upsampling}
\label{sec:depth_upsample_extra}

Figure~\ref{fig:depth_upsample_visuals} presents some more qualitative results comparing bicubic interpolation, Gauss
bilateral and learned bilateral upsampling on NYU depth dataset image~\cite{silberman2012indoor}.

\subsubsection{Character Recognition}\label{sec:app_character}

 Figure~\ref{fig:nnrecognition} shows the schematic of different layers
 of the network architecture for LeNet-7~\cite{lecun1998mnist}
 and DeepCNet(5, 50)~\cite{ciresan2012multi,graham2014spatially}. For the BNN variants, the first layer filters are replaced
 with learned bilateral filters and are learned end-to-end.

\subsubsection{Semantic Segmentation}\label{sec:app_semantic_segmentation}
\label{sec:semantic_bnn_extra}

Some more visual results for semantic segmentation are shown in Figure~\ref{fig:semantic_visuals}.
These include the underlying DeepLab CNN\cite{chen2014semantic} result (DeepLab),
the 2 step mean-field result with Gaussian edge potentials (+2stepMF-GaussCRF)
and also corresponding results with learned edge potentials (+2stepMF-LearnedCRF).
In general, we observe that mean-field in learned CRF leads to slightly dilated
classification regions in comparison to using Gaussian CRF thereby filling-in the
false negative pixels and also correcting some mis-classified regions.

\begin{figure*}[t!]
  \centering
    \subfigure{%
   \raisebox{2.0em}{
    \includegraphics[width=.06\columnwidth]{figures/supplementary/2bicubic}
   }
  }
  \subfigure{%
    \includegraphics[width=.17\columnwidth]{figures/supplementary/2given_image}
  }
  \subfigure{%
    \includegraphics[width=.17\columnwidth]{figures/supplementary/2ground_truth}
  }
  \subfigure{%
    \includegraphics[width=.17\columnwidth]{figures/supplementary/2bicubic}
  }
  \subfigure{%
    \includegraphics[width=.17\columnwidth]{figures/supplementary/2gauss}
  }
  \subfigure{%
    \includegraphics[width=.17\columnwidth]{figures/supplementary/2learnt}
  }\\
    \subfigure{%
   \raisebox{2.0em}{
    \includegraphics[width=.06\columnwidth]{figures/supplementary/32bicubic}
   }
  }
  \subfigure{%
    \includegraphics[width=.17\columnwidth]{figures/supplementary/32given_image}
  }
  \subfigure{%
    \includegraphics[width=.17\columnwidth]{figures/supplementary/32ground_truth}
  }
  \subfigure{%
    \includegraphics[width=.17\columnwidth]{figures/supplementary/32bicubic}
  }
  \subfigure{%
    \includegraphics[width=.17\columnwidth]{figures/supplementary/32gauss}
  }
  \subfigure{%
    \includegraphics[width=.17\columnwidth]{figures/supplementary/32learnt}
  }\\
  \setcounter{subfigure}{0}
  \small{
  \subfigure[Inp.]{%
  \raisebox{2.0em}{
    \includegraphics[width=.06\columnwidth]{figures/supplementary/41bicubic}
   }
  }
  \subfigure[Guidance]{%
    \includegraphics[width=.17\columnwidth]{figures/supplementary/41given_image}
  }
   \subfigure[GT]{%
    \includegraphics[width=.17\columnwidth]{figures/supplementary/41ground_truth}
  }
  \subfigure[Bicubic]{%
    \includegraphics[width=.17\columnwidth]{figures/supplementary/41bicubic}
  }
  \subfigure[Gauss-BF]{%
    \includegraphics[width=.17\columnwidth]{figures/supplementary/41gauss}
  }
  \subfigure[Learned-BF]{%
    \includegraphics[width=.17\columnwidth]{figures/supplementary/41learnt}
  }
  }
  \mycaption{Depth Upsampling}{Depth $8\times$ upsampling results
  using different upsampling strategies, from left to right,
  (a)~Low-resolution input depth image (Inp.),
  (b)~High-resolution guidance image, (c)~Ground-truth depth; Upsampled depth images with
  (d)~Bicubic interpolation, (e) Gauss bilateral upsampling and, (f)~Learned bilateral
  updampgling (best viewed on screen).}

\label{fig:depth_upsample_visuals}
\end{figure*}

\subsubsection{Material Segmentation}\label{sec:app_material_segmentation}
\label{sec:material_bnn_extra}

In Fig.~\ref{fig:material_visuals-app2}, we present visual results comparing 2 step
mean-field inference with Gaussian and learned pairwise CRF potentials. In
general, we observe that the pixels belonging to dominant classes in the
training data are being more accurately classified with learned CRF. This leads to
a significant improvements in overall pixel accuracy. This also results
in a slight decrease of the accuracy from less frequent class pixels thereby
slightly reducing the average class accuracy with learning. We attribute this
to the type of annotation that is available for this dataset, which is not
for the entire image but for some segments in the image. We have very few
images of the infrequent classes to combat this behaviour during training.

\subsubsection{Experiment Protocols}
\label{sec:protocols}

Table~\ref{tbl:parameters} shows experiment protocols of different experiments.

 \begin{figure*}[t!]
  \centering
  \subfigure[LeNet-7]{
    \includegraphics[width=0.7\columnwidth]{figures/supplementary/lenet_cnn_network}
    }\\
    \subfigure[DeepCNet]{
    \includegraphics[width=\columnwidth]{figures/supplementary/deepcnet_cnn_network}
    }
  \mycaption{CNNs for Character Recognition}
  {Schematic of (top) LeNet-7~\cite{lecun1998mnist} and (bottom) DeepCNet(5,50)~\cite{ciresan2012multi,graham2014spatially} architectures used in Assamese
  character recognition experiments.}
\label{fig:nnrecognition}
\end{figure*}

\definecolor{voc_1}{RGB}{0, 0, 0}
\definecolor{voc_2}{RGB}{128, 0, 0}
\definecolor{voc_3}{RGB}{0, 128, 0}
\definecolor{voc_4}{RGB}{128, 128, 0}
\definecolor{voc_5}{RGB}{0, 0, 128}
\definecolor{voc_6}{RGB}{128, 0, 128}
\definecolor{voc_7}{RGB}{0, 128, 128}
\definecolor{voc_8}{RGB}{128, 128, 128}
\definecolor{voc_9}{RGB}{64, 0, 0}
\definecolor{voc_10}{RGB}{192, 0, 0}
\definecolor{voc_11}{RGB}{64, 128, 0}
\definecolor{voc_12}{RGB}{192, 128, 0}
\definecolor{voc_13}{RGB}{64, 0, 128}
\definecolor{voc_14}{RGB}{192, 0, 128}
\definecolor{voc_15}{RGB}{64, 128, 128}
\definecolor{voc_16}{RGB}{192, 128, 128}
\definecolor{voc_17}{RGB}{0, 64, 0}
\definecolor{voc_18}{RGB}{128, 64, 0}
\definecolor{voc_19}{RGB}{0, 192, 0}
\definecolor{voc_20}{RGB}{128, 192, 0}
\definecolor{voc_21}{RGB}{0, 64, 128}
\definecolor{voc_22}{RGB}{128, 64, 128}

\begin{figure*}[t]
  \centering
  \small{
  \fcolorbox{white}{voc_1}{\rule{0pt}{6pt}\rule{6pt}{0pt}} Background~~
  \fcolorbox{white}{voc_2}{\rule{0pt}{6pt}\rule{6pt}{0pt}} Aeroplane~~
  \fcolorbox{white}{voc_3}{\rule{0pt}{6pt}\rule{6pt}{0pt}} Bicycle~~
  \fcolorbox{white}{voc_4}{\rule{0pt}{6pt}\rule{6pt}{0pt}} Bird~~
  \fcolorbox{white}{voc_5}{\rule{0pt}{6pt}\rule{6pt}{0pt}} Boat~~
  \fcolorbox{white}{voc_6}{\rule{0pt}{6pt}\rule{6pt}{0pt}} Bottle~~
  \fcolorbox{white}{voc_7}{\rule{0pt}{6pt}\rule{6pt}{0pt}} Bus~~
  \fcolorbox{white}{voc_8}{\rule{0pt}{6pt}\rule{6pt}{0pt}} Car~~ \\
  \fcolorbox{white}{voc_9}{\rule{0pt}{6pt}\rule{6pt}{0pt}} Cat~~
  \fcolorbox{white}{voc_10}{\rule{0pt}{6pt}\rule{6pt}{0pt}} Chair~~
  \fcolorbox{white}{voc_11}{\rule{0pt}{6pt}\rule{6pt}{0pt}} Cow~~
  \fcolorbox{white}{voc_12}{\rule{0pt}{6pt}\rule{6pt}{0pt}} Dining Table~~
  \fcolorbox{white}{voc_13}{\rule{0pt}{6pt}\rule{6pt}{0pt}} Dog~~
  \fcolorbox{white}{voc_14}{\rule{0pt}{6pt}\rule{6pt}{0pt}} Horse~~
  \fcolorbox{white}{voc_15}{\rule{0pt}{6pt}\rule{6pt}{0pt}} Motorbike~~
  \fcolorbox{white}{voc_16}{\rule{0pt}{6pt}\rule{6pt}{0pt}} Person~~ \\
  \fcolorbox{white}{voc_17}{\rule{0pt}{6pt}\rule{6pt}{0pt}} Potted Plant~~
  \fcolorbox{white}{voc_18}{\rule{0pt}{6pt}\rule{6pt}{0pt}} Sheep~~
  \fcolorbox{white}{voc_19}{\rule{0pt}{6pt}\rule{6pt}{0pt}} Sofa~~
  \fcolorbox{white}{voc_20}{\rule{0pt}{6pt}\rule{6pt}{0pt}} Train~~
  \fcolorbox{white}{voc_21}{\rule{0pt}{6pt}\rule{6pt}{0pt}} TV monitor~~ \\
  }
  \subfigure{%
    \includegraphics[width=.18\columnwidth]{figures/supplementary/2007_001423_given.jpg}
  }
  \subfigure{%
    \includegraphics[width=.18\columnwidth]{figures/supplementary/2007_001423_gt.png}
  }
  \subfigure{%
    \includegraphics[width=.18\columnwidth]{figures/supplementary/2007_001423_cnn.png}
  }
  \subfigure{%
    \includegraphics[width=.18\columnwidth]{figures/supplementary/2007_001423_gauss.png}
  }
  \subfigure{%
    \includegraphics[width=.18\columnwidth]{figures/supplementary/2007_001423_learnt.png}
  }\\
  \subfigure{%
    \includegraphics[width=.18\columnwidth]{figures/supplementary/2007_001430_given.jpg}
  }
  \subfigure{%
    \includegraphics[width=.18\columnwidth]{figures/supplementary/2007_001430_gt.png}
  }
  \subfigure{%
    \includegraphics[width=.18\columnwidth]{figures/supplementary/2007_001430_cnn.png}
  }
  \subfigure{%
    \includegraphics[width=.18\columnwidth]{figures/supplementary/2007_001430_gauss.png}
  }
  \subfigure{%
    \includegraphics[width=.18\columnwidth]{figures/supplementary/2007_001430_learnt.png}
  }\\
    \subfigure{%
    \includegraphics[width=.18\columnwidth]{figures/supplementary/2007_007996_given.jpg}
  }
  \subfigure{%
    \includegraphics[width=.18\columnwidth]{figures/supplementary/2007_007996_gt.png}
  }
  \subfigure{%
    \includegraphics[width=.18\columnwidth]{figures/supplementary/2007_007996_cnn.png}
  }
  \subfigure{%
    \includegraphics[width=.18\columnwidth]{figures/supplementary/2007_007996_gauss.png}
  }
  \subfigure{%
    \includegraphics[width=.18\columnwidth]{figures/supplementary/2007_007996_learnt.png}
  }\\
   \subfigure{%
    \includegraphics[width=.18\columnwidth]{figures/supplementary/2010_002682_given.jpg}
  }
  \subfigure{%
    \includegraphics[width=.18\columnwidth]{figures/supplementary/2010_002682_gt.png}
  }
  \subfigure{%
    \includegraphics[width=.18\columnwidth]{figures/supplementary/2010_002682_cnn.png}
  }
  \subfigure{%
    \includegraphics[width=.18\columnwidth]{figures/supplementary/2010_002682_gauss.png}
  }
  \subfigure{%
    \includegraphics[width=.18\columnwidth]{figures/supplementary/2010_002682_learnt.png}
  }\\
     \subfigure{%
    \includegraphics[width=.18\columnwidth]{figures/supplementary/2010_004789_given.jpg}
  }
  \subfigure{%
    \includegraphics[width=.18\columnwidth]{figures/supplementary/2010_004789_gt.png}
  }
  \subfigure{%
    \includegraphics[width=.18\columnwidth]{figures/supplementary/2010_004789_cnn.png}
  }
  \subfigure{%
    \includegraphics[width=.18\columnwidth]{figures/supplementary/2010_004789_gauss.png}
  }
  \subfigure{%
    \includegraphics[width=.18\columnwidth]{figures/supplementary/2010_004789_learnt.png}
  }\\
       \subfigure{%
    \includegraphics[width=.18\columnwidth]{figures/supplementary/2007_001311_given.jpg}
  }
  \subfigure{%
    \includegraphics[width=.18\columnwidth]{figures/supplementary/2007_001311_gt.png}
  }
  \subfigure{%
    \includegraphics[width=.18\columnwidth]{figures/supplementary/2007_001311_cnn.png}
  }
  \subfigure{%
    \includegraphics[width=.18\columnwidth]{figures/supplementary/2007_001311_gauss.png}
  }
  \subfigure{%
    \includegraphics[width=.18\columnwidth]{figures/supplementary/2007_001311_learnt.png}
  }\\
  \setcounter{subfigure}{0}
  \subfigure[Input]{%
    \includegraphics[width=.18\columnwidth]{figures/supplementary/2010_003531_given.jpg}
  }
  \subfigure[Ground Truth]{%
    \includegraphics[width=.18\columnwidth]{figures/supplementary/2010_003531_gt.png}
  }
  \subfigure[DeepLab]{%
    \includegraphics[width=.18\columnwidth]{figures/supplementary/2010_003531_cnn.png}
  }
  \subfigure[+GaussCRF]{%
    \includegraphics[width=.18\columnwidth]{figures/supplementary/2010_003531_gauss.png}
  }
  \subfigure[+LearnedCRF]{%
    \includegraphics[width=.18\columnwidth]{figures/supplementary/2010_003531_learnt.png}
  }
  \vspace{-0.3cm}
  \mycaption{Semantic Segmentation}{Example results of semantic segmentation.
  (c)~depicts the unary results before application of MF, (d)~after two steps of MF with Gaussian edge CRF potentials, (e)~after
  two steps of MF with learned edge CRF potentials.}
    \label{fig:semantic_visuals}
\end{figure*}


\definecolor{minc_1}{HTML}{771111}
\definecolor{minc_2}{HTML}{CAC690}
\definecolor{minc_3}{HTML}{EEEEEE}
\definecolor{minc_4}{HTML}{7C8FA6}
\definecolor{minc_5}{HTML}{597D31}
\definecolor{minc_6}{HTML}{104410}
\definecolor{minc_7}{HTML}{BB819C}
\definecolor{minc_8}{HTML}{D0CE48}
\definecolor{minc_9}{HTML}{622745}
\definecolor{minc_10}{HTML}{666666}
\definecolor{minc_11}{HTML}{D54A31}
\definecolor{minc_12}{HTML}{101044}
\definecolor{minc_13}{HTML}{444126}
\definecolor{minc_14}{HTML}{75D646}
\definecolor{minc_15}{HTML}{DD4348}
\definecolor{minc_16}{HTML}{5C8577}
\definecolor{minc_17}{HTML}{C78472}
\definecolor{minc_18}{HTML}{75D6D0}
\definecolor{minc_19}{HTML}{5B4586}
\definecolor{minc_20}{HTML}{C04393}
\definecolor{minc_21}{HTML}{D69948}
\definecolor{minc_22}{HTML}{7370D8}
\definecolor{minc_23}{HTML}{7A3622}
\definecolor{minc_24}{HTML}{000000}

\begin{figure*}[t]
  \centering
  \small{
  \fcolorbox{white}{minc_1}{\rule{0pt}{6pt}\rule{6pt}{0pt}} Brick~~
  \fcolorbox{white}{minc_2}{\rule{0pt}{6pt}\rule{6pt}{0pt}} Carpet~~
  \fcolorbox{white}{minc_3}{\rule{0pt}{6pt}\rule{6pt}{0pt}} Ceramic~~
  \fcolorbox{white}{minc_4}{\rule{0pt}{6pt}\rule{6pt}{0pt}} Fabric~~
  \fcolorbox{white}{minc_5}{\rule{0pt}{6pt}\rule{6pt}{0pt}} Foliage~~
  \fcolorbox{white}{minc_6}{\rule{0pt}{6pt}\rule{6pt}{0pt}} Food~~
  \fcolorbox{white}{minc_7}{\rule{0pt}{6pt}\rule{6pt}{0pt}} Glass~~
  \fcolorbox{white}{minc_8}{\rule{0pt}{6pt}\rule{6pt}{0pt}} Hair~~ \\
  \fcolorbox{white}{minc_9}{\rule{0pt}{6pt}\rule{6pt}{0pt}} Leather~~
  \fcolorbox{white}{minc_10}{\rule{0pt}{6pt}\rule{6pt}{0pt}} Metal~~
  \fcolorbox{white}{minc_11}{\rule{0pt}{6pt}\rule{6pt}{0pt}} Mirror~~
  \fcolorbox{white}{minc_12}{\rule{0pt}{6pt}\rule{6pt}{0pt}} Other~~
  \fcolorbox{white}{minc_13}{\rule{0pt}{6pt}\rule{6pt}{0pt}} Painted~~
  \fcolorbox{white}{minc_14}{\rule{0pt}{6pt}\rule{6pt}{0pt}} Paper~~
  \fcolorbox{white}{minc_15}{\rule{0pt}{6pt}\rule{6pt}{0pt}} Plastic~~\\
  \fcolorbox{white}{minc_16}{\rule{0pt}{6pt}\rule{6pt}{0pt}} Polished Stone~~
  \fcolorbox{white}{minc_17}{\rule{0pt}{6pt}\rule{6pt}{0pt}} Skin~~
  \fcolorbox{white}{minc_18}{\rule{0pt}{6pt}\rule{6pt}{0pt}} Sky~~
  \fcolorbox{white}{minc_19}{\rule{0pt}{6pt}\rule{6pt}{0pt}} Stone~~
  \fcolorbox{white}{minc_20}{\rule{0pt}{6pt}\rule{6pt}{0pt}} Tile~~
  \fcolorbox{white}{minc_21}{\rule{0pt}{6pt}\rule{6pt}{0pt}} Wallpaper~~
  \fcolorbox{white}{minc_22}{\rule{0pt}{6pt}\rule{6pt}{0pt}} Water~~
  \fcolorbox{white}{minc_23}{\rule{0pt}{6pt}\rule{6pt}{0pt}} Wood~~ \\
  }
  \subfigure{%
    \includegraphics[width=.18\columnwidth]{figures/supplementary/000010868_given.jpg}
  }
  \subfigure{%
    \includegraphics[width=.18\columnwidth]{figures/supplementary/000010868_gt.png}
  }
  \subfigure{%
    \includegraphics[width=.18\columnwidth]{figures/supplementary/000010868_cnn.png}
  }
  \subfigure{%
    \includegraphics[width=.18\columnwidth]{figures/supplementary/000010868_gauss.png}
  }
  \subfigure{%
    \includegraphics[width=.18\columnwidth]{figures/supplementary/000010868_learnt.png}
  }\\[-2ex]
  \subfigure{%
    \includegraphics[width=.18\columnwidth]{figures/supplementary/000006011_given.jpg}
  }
  \subfigure{%
    \includegraphics[width=.18\columnwidth]{figures/supplementary/000006011_gt.png}
  }
  \subfigure{%
    \includegraphics[width=.18\columnwidth]{figures/supplementary/000006011_cnn.png}
  }
  \subfigure{%
    \includegraphics[width=.18\columnwidth]{figures/supplementary/000006011_gauss.png}
  }
  \subfigure{%
    \includegraphics[width=.18\columnwidth]{figures/supplementary/000006011_learnt.png}
  }\\[-2ex]
    \subfigure{%
    \includegraphics[width=.18\columnwidth]{figures/supplementary/000008553_given.jpg}
  }
  \subfigure{%
    \includegraphics[width=.18\columnwidth]{figures/supplementary/000008553_gt.png}
  }
  \subfigure{%
    \includegraphics[width=.18\columnwidth]{figures/supplementary/000008553_cnn.png}
  }
  \subfigure{%
    \includegraphics[width=.18\columnwidth]{figures/supplementary/000008553_gauss.png}
  }
  \subfigure{%
    \includegraphics[width=.18\columnwidth]{figures/supplementary/000008553_learnt.png}
  }\\[-2ex]
   \subfigure{%
    \includegraphics[width=.18\columnwidth]{figures/supplementary/000009188_given.jpg}
  }
  \subfigure{%
    \includegraphics[width=.18\columnwidth]{figures/supplementary/000009188_gt.png}
  }
  \subfigure{%
    \includegraphics[width=.18\columnwidth]{figures/supplementary/000009188_cnn.png}
  }
  \subfigure{%
    \includegraphics[width=.18\columnwidth]{figures/supplementary/000009188_gauss.png}
  }
  \subfigure{%
    \includegraphics[width=.18\columnwidth]{figures/supplementary/000009188_learnt.png}
  }\\[-2ex]
  \setcounter{subfigure}{0}
  \subfigure[Input]{%
    \includegraphics[width=.18\columnwidth]{figures/supplementary/000023570_given.jpg}
  }
  \subfigure[Ground Truth]{%
    \includegraphics[width=.18\columnwidth]{figures/supplementary/000023570_gt.png}
  }
  \subfigure[DeepLab]{%
    \includegraphics[width=.18\columnwidth]{figures/supplementary/000023570_cnn.png}
  }
  \subfigure[+GaussCRF]{%
    \includegraphics[width=.18\columnwidth]{figures/supplementary/000023570_gauss.png}
  }
  \subfigure[+LearnedCRF]{%
    \includegraphics[width=.18\columnwidth]{figures/supplementary/000023570_learnt.png}
  }
  \mycaption{Material Segmentation}{Example results of material segmentation.
  (c)~depicts the unary results before application of MF, (d)~after two steps of MF with Gaussian edge CRF potentials, (e)~after two steps of MF with learned edge CRF potentials.}
    \label{fig:material_visuals-app2}
\end{figure*}


\begin{table*}[h]
\tiny
  \centering
    \begin{tabular}{L{2.3cm} L{2.25cm} C{1.5cm} C{0.7cm} C{0.6cm} C{0.7cm} C{0.7cm} C{0.7cm} C{1.6cm} C{0.6cm} C{0.6cm} C{0.6cm}}
      \toprule
& & & & & \multicolumn{3}{c}{\textbf{Data Statistics}} & \multicolumn{4}{c}{\textbf{Training Protocol}} \\

\textbf{Experiment} & \textbf{Feature Types} & \textbf{Feature Scales} & \textbf{Filter Size} & \textbf{Filter Nbr.} & \textbf{Train}  & \textbf{Val.} & \textbf{Test} & \textbf{Loss Type} & \textbf{LR} & \textbf{Batch} & \textbf{Epochs} \\
      \midrule
      \multicolumn{2}{c}{\textbf{Single Bilateral Filter Applications}} & & & & & & & & & \\
      \textbf{2$\times$ Color Upsampling} & Position$_{1}$, Intensity (3D) & 0.13, 0.17 & 65 & 2 & 10581 & 1449 & 1456 & MSE & 1e-06 & 200 & 94.5\\
      \textbf{4$\times$ Color Upsampling} & Position$_{1}$, Intensity (3D) & 0.06, 0.17 & 65 & 2 & 10581 & 1449 & 1456 & MSE & 1e-06 & 200 & 94.5\\
      \textbf{8$\times$ Color Upsampling} & Position$_{1}$, Intensity (3D) & 0.03, 0.17 & 65 & 2 & 10581 & 1449 & 1456 & MSE & 1e-06 & 200 & 94.5\\
      \textbf{16$\times$ Color Upsampling} & Position$_{1}$, Intensity (3D) & 0.02, 0.17 & 65 & 2 & 10581 & 1449 & 1456 & MSE & 1e-06 & 200 & 94.5\\
      \textbf{Depth Upsampling} & Position$_{1}$, Color (5D) & 0.05, 0.02 & 665 & 2 & 795 & 100 & 654 & MSE & 1e-07 & 50 & 251.6\\
      \textbf{Mesh Denoising} & Isomap (4D) & 46.00 & 63 & 2 & 1000 & 200 & 500 & MSE & 100 & 10 & 100.0 \\
      \midrule
      \multicolumn{2}{c}{\textbf{DenseCRF Applications}} & & & & & & & & &\\
      \multicolumn{2}{l}{\textbf{Semantic Segmentation}} & & & & & & & & &\\
      \textbf{- 1step MF} & Position$_{1}$, Color (5D); Position$_{1}$ (2D) & 0.01, 0.34; 0.34  & 665; 19  & 2; 2 & 10581 & 1449 & 1456 & Logistic & 0.1 & 5 & 1.4 \\
      \textbf{- 2step MF} & Position$_{1}$, Color (5D); Position$_{1}$ (2D) & 0.01, 0.34; 0.34 & 665; 19 & 2; 2 & 10581 & 1449 & 1456 & Logistic & 0.1 & 5 & 1.4 \\
      \textbf{- \textit{loose} 2step MF} & Position$_{1}$, Color (5D); Position$_{1}$ (2D) & 0.01, 0.34; 0.34 & 665; 19 & 2; 2 &10581 & 1449 & 1456 & Logistic & 0.1 & 5 & +1.9  \\ \\
      \multicolumn{2}{l}{\textbf{Material Segmentation}} & & & & & & & & &\\
      \textbf{- 1step MF} & Position$_{2}$, Lab-Color (5D) & 5.00, 0.05, 0.30  & 665 & 2 & 928 & 150 & 1798 & Weighted Logistic & 1e-04 & 24 & 2.6 \\
      \textbf{- 2step MF} & Position$_{2}$, Lab-Color (5D) & 5.00, 0.05, 0.30 & 665 & 2 & 928 & 150 & 1798 & Weighted Logistic & 1e-04 & 12 & +0.7 \\
      \textbf{- \textit{loose} 2step MF} & Position$_{2}$, Lab-Color (5D) & 5.00, 0.05, 0.30 & 665 & 2 & 928 & 150 & 1798 & Weighted Logistic & 1e-04 & 12 & +0.2\\
      \midrule
      \multicolumn{2}{c}{\textbf{Neural Network Applications}} & & & & & & & & &\\
      \textbf{Tiles: CNN-9$\times$9} & - & - & 81 & 4 & 10000 & 1000 & 1000 & Logistic & 0.01 & 100 & 500.0 \\
      \textbf{Tiles: CNN-13$\times$13} & - & - & 169 & 6 & 10000 & 1000 & 1000 & Logistic & 0.01 & 100 & 500.0 \\
      \textbf{Tiles: CNN-17$\times$17} & - & - & 289 & 8 & 10000 & 1000 & 1000 & Logistic & 0.01 & 100 & 500.0 \\
      \textbf{Tiles: CNN-21$\times$21} & - & - & 441 & 10 & 10000 & 1000 & 1000 & Logistic & 0.01 & 100 & 500.0 \\
      \textbf{Tiles: BNN} & Position$_{1}$, Color (5D) & 0.05, 0.04 & 63 & 1 & 10000 & 1000 & 1000 & Logistic & 0.01 & 100 & 30.0 \\
      \textbf{LeNet} & - & - & 25 & 2 & 5490 & 1098 & 1647 & Logistic & 0.1 & 100 & 182.2 \\
      \textbf{Crop-LeNet} & - & - & 25 & 2 & 5490 & 1098 & 1647 & Logistic & 0.1 & 100 & 182.2 \\
      \textbf{BNN-LeNet} & Position$_{2}$ (2D) & 20.00 & 7 & 1 & 5490 & 1098 & 1647 & Logistic & 0.1 & 100 & 182.2 \\
      \textbf{DeepCNet} & - & - & 9 & 1 & 5490 & 1098 & 1647 & Logistic & 0.1 & 100 & 182.2 \\
      \textbf{Crop-DeepCNet} & - & - & 9 & 1 & 5490 & 1098 & 1647 & Logistic & 0.1 & 100 & 182.2 \\
      \textbf{BNN-DeepCNet} & Position$_{2}$ (2D) & 40.00  & 7 & 1 & 5490 & 1098 & 1647 & Logistic & 0.1 & 100 & 182.2 \\
      \bottomrule
      \\
    \end{tabular}
    \mycaption{Experiment Protocols} {Experiment protocols for the different experiments presented in this work. \textbf{Feature Types}:
    Feature spaces used for the bilateral convolutions. Position$_1$ corresponds to un-normalized pixel positions whereas Position$_2$ corresponds
    to pixel positions normalized to $[0,1]$ with respect to the given image. \textbf{Feature Scales}: Cross-validated scales for the features used.
     \textbf{Filter Size}: Number of elements in the filter that is being learned. \textbf{Filter Nbr.}: Half-width of the filter. \textbf{Train},
     \textbf{Val.} and \textbf{Test} corresponds to the number of train, validation and test images used in the experiment. \textbf{Loss Type}: Type
     of loss used for back-propagation. ``MSE'' corresponds to Euclidean mean squared error loss and ``Logistic'' corresponds to multinomial logistic
     loss. ``Weighted Logistic'' is the class-weighted multinomial logistic loss. We weighted the loss with inverse class probability for material
     segmentation task due to the small availability of training data with class imbalance. \textbf{LR}: Fixed learning rate used in stochastic gradient
     descent. \textbf{Batch}: Number of images used in one parameter update step. \textbf{Epochs}: Number of training epochs. In all the experiments,
     we used fixed momentum of 0.9 and weight decay of 0.0005 for stochastic gradient descent. ```Color Upsampling'' experiments in this Table corresponds
     to those performed on Pascal VOC12 dataset images. For all experiments using Pascal VOC12 images, we use extended
     training segmentation dataset available from~\cite{hariharan2011moredata}, and used standard validation and test splits
     from the main dataset~\cite{voc2012segmentation}.}
  \label{tbl:parameters}
\end{table*}

\clearpage

\section{Parameters and Additional Results for Video Propagation Networks}

In this Section, we present experiment protocols and additional qualitative results for experiments
on video object segmentation, semantic video segmentation and video color
propagation. Table~\ref{tbl:parameters_supp} shows the feature scales and other parameters used in different experiments.
Figures~\ref{fig:video_seg_pos_supp} show some qualitative results on video object segmentation
with some failure cases in Fig.~\ref{fig:video_seg_neg_supp}.
Figure~\ref{fig:semantic_visuals_supp} shows some qualitative results on semantic video segmentation and
Fig.~\ref{fig:color_visuals_supp} shows results on video color propagation.

\newcolumntype{L}[1]{>{\raggedright\let\newline\\\arraybackslash\hspace{0pt}}b{#1}}
\newcolumntype{C}[1]{>{\centering\let\newline\\\arraybackslash\hspace{0pt}}b{#1}}
\newcolumntype{R}[1]{>{\raggedleft\let\newline\\\arraybackslash\hspace{0pt}}b{#1}}

\begin{table*}[h]
\tiny
  \centering
    \begin{tabular}{L{3.0cm} L{2.4cm} L{2.8cm} L{2.8cm} C{0.5cm} C{1.0cm} L{1.2cm}}
      \toprule
\textbf{Experiment} & \textbf{Feature Type} & \textbf{Feature Scale-1, $\Lambda_a$} & \textbf{Feature Scale-2, $\Lambda_b$} & \textbf{$\alpha$} & \textbf{Input Frames} & \textbf{Loss Type} \\
      \midrule
      \textbf{Video Object Segmentation} & ($x,y,Y,Cb,Cr,t$) & (0.02,0.02,0.07,0.4,0.4,0.01) & (0.03,0.03,0.09,0.5,0.5,0.2) & 0.5 & 9 & Logistic\\
      \midrule
      \textbf{Semantic Video Segmentation} & & & & & \\
      \textbf{with CNN1~\cite{yu2015multi}-NoFlow} & ($x,y,R,G,B,t$) & (0.08,0.08,0.2,0.2,0.2,0.04) & (0.11,0.11,0.2,0.2,0.2,0.04) & 0.5 & 3 & Logistic \\
      \textbf{with CNN1~\cite{yu2015multi}-Flow} & ($x+u_x,y+u_y,R,G,B,t$) & (0.11,0.11,0.14,0.14,0.14,0.03) & (0.08,0.08,0.12,0.12,0.12,0.01) & 0.65 & 3 & Logistic\\
      \textbf{with CNN2~\cite{richter2016playing}-Flow} & ($x+u_x,y+u_y,R,G,B,t$) & (0.08,0.08,0.2,0.2,0.2,0.04) & (0.09,0.09,0.25,0.25,0.25,0.03) & 0.5 & 4 & Logistic\\
      \midrule
      \textbf{Video Color Propagation} & ($x,y,I,t$)  & (0.04,0.04,0.2,0.04) & No second kernel & 1 & 4 & MSE\\
      \bottomrule
      \\
    \end{tabular}
    \mycaption{Experiment Protocols} {Experiment protocols for the different experiments presented in this work. \textbf{Feature Types}:
    Feature spaces used for the bilateral convolutions, with position ($x,y$) and color
    ($R,G,B$ or $Y,Cb,Cr$) features $\in [0,255]$. $u_x$, $u_y$ denotes optical flow with respect
    to the present frame and $I$ denotes grayscale intensity.
    \textbf{Feature Scales ($\Lambda_a, \Lambda_b$)}: Cross-validated scales for the features used.
    \textbf{$\alpha$}: Exponential time decay for the input frames.
    \textbf{Input Frames}: Number of input frames for VPN.
    \textbf{Loss Type}: Type
     of loss used for back-propagation. ``MSE'' corresponds to Euclidean mean squared error loss and ``Logistic'' corresponds to multinomial logistic loss.}
  \label{tbl:parameters_supp}
\end{table*}

% \begin{figure}[th!]
% \begin{center}
%   \centerline{\includegraphics[width=\textwidth]{figures/video_seg_visuals_supp_small.pdf}}
%     \mycaption{Video Object Segmentation}
%     {Shown are the different frames in example videos with the corresponding
%     ground truth (GT) masks, predictions from BVS~\cite{marki2016bilateral},
%     OFL~\cite{tsaivideo}, VPN (VPN-Stage2) and VPN-DLab (VPN-DeepLab) models.}
%     \label{fig:video_seg_small_supp}
% \end{center}
% \vspace{-1.0cm}
% \end{figure}

\begin{figure}[th!]
\begin{center}
  \centerline{\includegraphics[width=0.7\textwidth]{figures/video_seg_visuals_supp_positive.pdf}}
    \mycaption{Video Object Segmentation}
    {Shown are the different frames in example videos with the corresponding
    ground truth (GT) masks, predictions from BVS~\cite{marki2016bilateral},
    OFL~\cite{tsaivideo}, VPN (VPN-Stage2) and VPN-DLab (VPN-DeepLab) models.}
    \label{fig:video_seg_pos_supp}
\end{center}
\vspace{-1.0cm}
\end{figure}

\begin{figure}[th!]
\begin{center}
  \centerline{\includegraphics[width=0.7\textwidth]{figures/video_seg_visuals_supp_negative.pdf}}
    \mycaption{Failure Cases for Video Object Segmentation}
    {Shown are the different frames in example videos with the corresponding
    ground truth (GT) masks, predictions from BVS~\cite{marki2016bilateral},
    OFL~\cite{tsaivideo}, VPN (VPN-Stage2) and VPN-DLab (VPN-DeepLab) models.}
    \label{fig:video_seg_neg_supp}
\end{center}
\vspace{-1.0cm}
\end{figure}

\begin{figure}[th!]
\begin{center}
  \centerline{\includegraphics[width=0.9\textwidth]{figures/supp_semantic_visual.pdf}}
    \mycaption{Semantic Video Segmentation}
    {Input video frames and the corresponding ground truth (GT)
    segmentation together with the predictions of CNN~\cite{yu2015multi} and with
    VPN-Flow.}
    \label{fig:semantic_visuals_supp}
\end{center}
\vspace{-0.7cm}
\end{figure}

\begin{figure}[th!]
\begin{center}
  \centerline{\includegraphics[width=\textwidth]{figures/colorization_visuals_supp.pdf}}
  \mycaption{Video Color Propagation}
  {Input grayscale video frames and corresponding ground-truth (GT) color images
  together with color predictions of Levin et al.~\cite{levin2004colorization} and VPN-Stage1 models.}
  \label{fig:color_visuals_supp}
\end{center}
\vspace{-0.7cm}
\end{figure}

\clearpage

\section{Additional Material for Bilateral Inception Networks}
\label{sec:binception-app}

In this section of the Appendix, we first discuss the use of approximate bilateral
filtering in BI modules (Sec.~\ref{sec:lattice}).
Later, we present some qualitative results using different models for the approach presented in
Chapter~\ref{chap:binception} (Sec.~\ref{sec:qualitative-app}).

\subsection{Approximate Bilateral Filtering}
\label{sec:lattice}

The bilateral inception module presented in Chapter~\ref{chap:binception} computes a matrix-vector
product between a Gaussian filter $K$ and a vector of activations $\bz_c$.
Bilateral filtering is an important operation and many algorithmic techniques have been
proposed to speed-up this operation~\cite{paris2006fast,adams2010fast,gastal2011domain}.
In the main paper we opted to implement what can be considered the
brute-force variant of explicitly constructing $K$ and then using BLAS to compute the
matrix-vector product. This resulted in a few millisecond operation.
The explicit way to compute is possible due to the
reduction to super-pixels, e.g., it would not work for DenseCRF variants
that operate on the full image resolution.

Here, we present experiments where we use the fast approximate bilateral filtering
algorithm of~\cite{adams2010fast}, which is also used in Chapter~\ref{chap:bnn}
for learning sparse high dimensional filters. This
choice allows for larger dimensions of matrix-vector multiplication. The reason for choosing
the explicit multiplication in Chapter~\ref{chap:binception} was that it was computationally faster.
For the small sizes of the involved matrices and vectors, the explicit computation is sufficient and we had no
GPU implementation of an approximate technique that matched this runtime. Also it
is conceptually easier and the gradient to the feature transformations ($\Lambda \mathbf{f}$) is
obtained using standard matrix calculus.

\subsubsection{Experiments}

We modified the existing segmentation architectures analogous to those in Chapter~\ref{chap:binception}.
The main difference is that, here, the inception modules use the lattice
approximation~\cite{adams2010fast} to compute the bilateral filtering.
Using the lattice approximation did not allow us to back-propagate through feature transformations ($\Lambda$)
and thus we used hand-specified feature scales as will be explained later.
Specifically, we take CNN architectures from the works
of~\cite{chen2014semantic,zheng2015conditional,bell2015minc} and insert the BI modules between
the spatial FC layers.
We use superpixels from~\cite{DollarICCV13edges}
for all the experiments with the lattice approximation. Experiments are
performed using Caffe neural network framework~\cite{jia2014caffe}.

\begin{table}
  \small
  \centering
  \begin{tabular}{p{5.5cm}>{\raggedright\arraybackslash}p{1.4cm}>{\centering\arraybackslash}p{2.2cm}}
    \toprule
		\textbf{Model} & \emph{IoU} & \emph{Runtime}(ms) \\
    \midrule

    %%%%%%%%%%%% Scores computed by us)%%%%%%%%%%%%
		\deeplablargefov & 68.9 & 145ms\\
    \midrule
    \bi{7}{2}-\bi{8}{10}& \textbf{73.8} & +600 \\
    \midrule
    \deeplablargefovcrf~\cite{chen2014semantic} & 72.7 & +830\\
    \deeplabmsclargefovcrf~\cite{chen2014semantic} & \textbf{73.6} & +880\\
    DeepLab-EdgeNet~\cite{chen2015semantic} & 71.7 & +30\\
    DeepLab-EdgeNet-CRF~\cite{chen2015semantic} & \textbf{73.6} & +860\\
  \bottomrule \\
  \end{tabular}
  \mycaption{Semantic Segmentation using the DeepLab model}
  {IoU scores on the Pascal VOC12 segmentation test dataset
  with different models and our modified inception model.
  Also shown are the corresponding runtimes in milliseconds. Runtimes
  also include superpixel computations (300 ms with Dollar superpixels~\cite{DollarICCV13edges})}
  \label{tab:largefovresults}
\end{table}

\paragraph{Semantic Segmentation}
The experiments in this section use the Pascal VOC12 segmentation dataset~\cite{voc2012segmentation} with 21 object classes and the images have a maximum resolution of 0.25 megapixels.
For all experiments on VOC12, we train using the extended training set of
10581 images collected by~\cite{hariharan2011moredata}.
We modified the \deeplab~network architecture of~\cite{chen2014semantic} and
the CRFasRNN architecture from~\cite{zheng2015conditional} which uses a CNN with
deconvolution layers followed by DenseCRF trained end-to-end.

\paragraph{DeepLab Model}\label{sec:deeplabmodel}
We experimented with the \bi{7}{2}-\bi{8}{10} inception model.
Results using the~\deeplab~model are summarized in Tab.~\ref{tab:largefovresults}.
Although we get similar improvements with inception modules as with the
explicit kernel computation, using lattice approximation is slower.

\begin{table}
  \small
  \centering
  \begin{tabular}{p{6.4cm}>{\raggedright\arraybackslash}p{1.8cm}>{\raggedright\arraybackslash}p{1.8cm}}
    \toprule
    \textbf{Model} & \emph{IoU (Val)} & \emph{IoU (Test)}\\
    \midrule
    %%%%%%%%%%%% Scores computed by us)%%%%%%%%%%%%
    CNN &  67.5 & - \\
    \deconv (CNN+Deconvolutions) & 69.8 & 72.0 \\
    \midrule
    \bi{3}{6}-\bi{4}{6}-\bi{7}{2}-\bi{8}{6}& 71.9 & - \\
    \bi{3}{6}-\bi{4}{6}-\bi{7}{2}-\bi{8}{6}-\gi{6}& 73.6 &  \href{http://host.robots.ox.ac.uk:8080/anonymous/VOTV5E.html}{\textbf{75.2}}\\
    \midrule
    \deconvcrf (CRF-RNN)~\cite{zheng2015conditional} & 73.0 & 74.7\\
    Context-CRF-RNN~\cite{yu2015multi} & ~~ - ~ & \textbf{75.3} \\
    \bottomrule \\
  \end{tabular}
  \mycaption{Semantic Segmentation using the CRFasRNN model}{IoU score corresponding to different models
  on Pascal VOC12 reduced validation / test segmentation dataset. The reduced validation set consists of 346 images
  as used in~\cite{zheng2015conditional} where we adapted the model from.}
  \label{tab:deconvresults-app}
\end{table}

\paragraph{CRFasRNN Model}\label{sec:deepinception}
We add BI modules after score-pool3, score-pool4, \fc{7} and \fc{8} $1\times1$ convolution layers
resulting in the \bi{3}{6}-\bi{4}{6}-\bi{7}{2}-\bi{8}{6}
model and also experimented with another variant where $BI_8$ is followed by another inception
module, G$(6)$, with 6 Gaussian kernels.
Note that here also we discarded both deconvolution and DenseCRF parts of the original model~\cite{zheng2015conditional}
and inserted the BI modules in the base CNN and found similar improvements compared to the inception modules with explicit
kernel computaion. See Tab.~\ref{tab:deconvresults-app} for results on the CRFasRNN model.

\paragraph{Material Segmentation}
Table~\ref{tab:mincresults-app} shows the results on the MINC dataset~\cite{bell2015minc}
obtained by modifying the AlexNet architecture with our inception modules. We observe
similar improvements as with explicit kernel construction.
For this model, we do not provide any learned setup due to very limited segment training
data. The weights to combine outputs in the bilateral inception layer are
found by validation on the validation set.

\begin{table}[t]
  \small
  \centering
  \begin{tabular}{p{3.5cm}>{\centering\arraybackslash}p{4.0cm}}
    \toprule
    \textbf{Model} & Class / Total accuracy\\
    \midrule

    %%%%%%%%%%%% Scores computed by us)%%%%%%%%%%%%
    AlexNet CNN & 55.3 / 58.9 \\
    \midrule
    \bi{7}{2}-\bi{8}{6}& 68.5 / 71.8 \\
    \bi{7}{2}-\bi{8}{6}-G$(6)$& 67.6 / 73.1 \\
    \midrule
    AlexNet-CRF & 65.5 / 71.0 \\
    \bottomrule \\
  \end{tabular}
  \mycaption{Material Segmentation using AlexNet}{Pixel accuracy of different models on
  the MINC material segmentation test dataset~\cite{bell2015minc}.}
  \label{tab:mincresults-app}
\end{table}

\paragraph{Scales of Bilateral Inception Modules}
\label{sec:scales}

Unlike the explicit kernel technique presented in the main text (Chapter~\ref{chap:binception}),
we didn't back-propagate through feature transformation ($\Lambda$)
using the approximate bilateral filter technique.
So, the feature scales are hand-specified and validated, which are as follows.
The optimal scale values for the \bi{7}{2}-\bi{8}{2} model are found by validation for the best performance which are
$\sigma_{xy}$ = (0.1, 0.1) for the spatial (XY) kernel and $\sigma_{rgbxy}$ = (0.1, 0.1, 0.1, 0.01, 0.01) for color and position (RGBXY)  kernel.
Next, as more kernels are added to \bi{8}{2}, we set scales to be $\alpha$*($\sigma_{xy}$, $\sigma_{rgbxy}$).
The value of $\alpha$ is chosen as  1, 0.5, 0.1, 0.05, 0.1, at uniform interval, for the \bi{8}{10} bilateral inception module.


\subsection{Qualitative Results}
\label{sec:qualitative-app}

In this section, we present more qualitative results obtained using the BI module with explicit
kernel computation technique presented in Chapter~\ref{chap:binception}. Results on the Pascal VOC12
dataset~\cite{voc2012segmentation} using the DeepLab-LargeFOV model are shown in Fig.~\ref{fig:semantic_visuals-app},
followed by the results on MINC dataset~\cite{bell2015minc}
in Fig.~\ref{fig:material_visuals-app} and on
Cityscapes dataset~\cite{Cordts2015Cvprw} in Fig.~\ref{fig:street_visuals-app}.


\definecolor{voc_1}{RGB}{0, 0, 0}
\definecolor{voc_2}{RGB}{128, 0, 0}
\definecolor{voc_3}{RGB}{0, 128, 0}
\definecolor{voc_4}{RGB}{128, 128, 0}
\definecolor{voc_5}{RGB}{0, 0, 128}
\definecolor{voc_6}{RGB}{128, 0, 128}
\definecolor{voc_7}{RGB}{0, 128, 128}
\definecolor{voc_8}{RGB}{128, 128, 128}
\definecolor{voc_9}{RGB}{64, 0, 0}
\definecolor{voc_10}{RGB}{192, 0, 0}
\definecolor{voc_11}{RGB}{64, 128, 0}
\definecolor{voc_12}{RGB}{192, 128, 0}
\definecolor{voc_13}{RGB}{64, 0, 128}
\definecolor{voc_14}{RGB}{192, 0, 128}
\definecolor{voc_15}{RGB}{64, 128, 128}
\definecolor{voc_16}{RGB}{192, 128, 128}
\definecolor{voc_17}{RGB}{0, 64, 0}
\definecolor{voc_18}{RGB}{128, 64, 0}
\definecolor{voc_19}{RGB}{0, 192, 0}
\definecolor{voc_20}{RGB}{128, 192, 0}
\definecolor{voc_21}{RGB}{0, 64, 128}
\definecolor{voc_22}{RGB}{128, 64, 128}

\begin{figure*}[!ht]
  \small
  \centering
  \fcolorbox{white}{voc_1}{\rule{0pt}{4pt}\rule{4pt}{0pt}} Background~~
  \fcolorbox{white}{voc_2}{\rule{0pt}{4pt}\rule{4pt}{0pt}} Aeroplane~~
  \fcolorbox{white}{voc_3}{\rule{0pt}{4pt}\rule{4pt}{0pt}} Bicycle~~
  \fcolorbox{white}{voc_4}{\rule{0pt}{4pt}\rule{4pt}{0pt}} Bird~~
  \fcolorbox{white}{voc_5}{\rule{0pt}{4pt}\rule{4pt}{0pt}} Boat~~
  \fcolorbox{white}{voc_6}{\rule{0pt}{4pt}\rule{4pt}{0pt}} Bottle~~
  \fcolorbox{white}{voc_7}{\rule{0pt}{4pt}\rule{4pt}{0pt}} Bus~~
  \fcolorbox{white}{voc_8}{\rule{0pt}{4pt}\rule{4pt}{0pt}} Car~~\\
  \fcolorbox{white}{voc_9}{\rule{0pt}{4pt}\rule{4pt}{0pt}} Cat~~
  \fcolorbox{white}{voc_10}{\rule{0pt}{4pt}\rule{4pt}{0pt}} Chair~~
  \fcolorbox{white}{voc_11}{\rule{0pt}{4pt}\rule{4pt}{0pt}} Cow~~
  \fcolorbox{white}{voc_12}{\rule{0pt}{4pt}\rule{4pt}{0pt}} Dining Table~~
  \fcolorbox{white}{voc_13}{\rule{0pt}{4pt}\rule{4pt}{0pt}} Dog~~
  \fcolorbox{white}{voc_14}{\rule{0pt}{4pt}\rule{4pt}{0pt}} Horse~~
  \fcolorbox{white}{voc_15}{\rule{0pt}{4pt}\rule{4pt}{0pt}} Motorbike~~
  \fcolorbox{white}{voc_16}{\rule{0pt}{4pt}\rule{4pt}{0pt}} Person~~\\
  \fcolorbox{white}{voc_17}{\rule{0pt}{4pt}\rule{4pt}{0pt}} Potted Plant~~
  \fcolorbox{white}{voc_18}{\rule{0pt}{4pt}\rule{4pt}{0pt}} Sheep~~
  \fcolorbox{white}{voc_19}{\rule{0pt}{4pt}\rule{4pt}{0pt}} Sofa~~
  \fcolorbox{white}{voc_20}{\rule{0pt}{4pt}\rule{4pt}{0pt}} Train~~
  \fcolorbox{white}{voc_21}{\rule{0pt}{4pt}\rule{4pt}{0pt}} TV monitor~~\\


  \subfigure{%
    \includegraphics[width=.15\columnwidth]{figures/supplementary/2008_001308_given.png}
  }
  \subfigure{%
    \includegraphics[width=.15\columnwidth]{figures/supplementary/2008_001308_sp.png}
  }
  \subfigure{%
    \includegraphics[width=.15\columnwidth]{figures/supplementary/2008_001308_gt.png}
  }
  \subfigure{%
    \includegraphics[width=.15\columnwidth]{figures/supplementary/2008_001308_cnn.png}
  }
  \subfigure{%
    \includegraphics[width=.15\columnwidth]{figures/supplementary/2008_001308_crf.png}
  }
  \subfigure{%
    \includegraphics[width=.15\columnwidth]{figures/supplementary/2008_001308_ours.png}
  }\\[-2ex]


  \subfigure{%
    \includegraphics[width=.15\columnwidth]{figures/supplementary/2008_001821_given.png}
  }
  \subfigure{%
    \includegraphics[width=.15\columnwidth]{figures/supplementary/2008_001821_sp.png}
  }
  \subfigure{%
    \includegraphics[width=.15\columnwidth]{figures/supplementary/2008_001821_gt.png}
  }
  \subfigure{%
    \includegraphics[width=.15\columnwidth]{figures/supplementary/2008_001821_cnn.png}
  }
  \subfigure{%
    \includegraphics[width=.15\columnwidth]{figures/supplementary/2008_001821_crf.png}
  }
  \subfigure{%
    \includegraphics[width=.15\columnwidth]{figures/supplementary/2008_001821_ours.png}
  }\\[-2ex]



  \subfigure{%
    \includegraphics[width=.15\columnwidth]{figures/supplementary/2008_004612_given.png}
  }
  \subfigure{%
    \includegraphics[width=.15\columnwidth]{figures/supplementary/2008_004612_sp.png}
  }
  \subfigure{%
    \includegraphics[width=.15\columnwidth]{figures/supplementary/2008_004612_gt.png}
  }
  \subfigure{%
    \includegraphics[width=.15\columnwidth]{figures/supplementary/2008_004612_cnn.png}
  }
  \subfigure{%
    \includegraphics[width=.15\columnwidth]{figures/supplementary/2008_004612_crf.png}
  }
  \subfigure{%
    \includegraphics[width=.15\columnwidth]{figures/supplementary/2008_004612_ours.png}
  }\\[-2ex]


  \subfigure{%
    \includegraphics[width=.15\columnwidth]{figures/supplementary/2009_001008_given.png}
  }
  \subfigure{%
    \includegraphics[width=.15\columnwidth]{figures/supplementary/2009_001008_sp.png}
  }
  \subfigure{%
    \includegraphics[width=.15\columnwidth]{figures/supplementary/2009_001008_gt.png}
  }
  \subfigure{%
    \includegraphics[width=.15\columnwidth]{figures/supplementary/2009_001008_cnn.png}
  }
  \subfigure{%
    \includegraphics[width=.15\columnwidth]{figures/supplementary/2009_001008_crf.png}
  }
  \subfigure{%
    \includegraphics[width=.15\columnwidth]{figures/supplementary/2009_001008_ours.png}
  }\\[-2ex]




  \subfigure{%
    \includegraphics[width=.15\columnwidth]{figures/supplementary/2009_004497_given.png}
  }
  \subfigure{%
    \includegraphics[width=.15\columnwidth]{figures/supplementary/2009_004497_sp.png}
  }
  \subfigure{%
    \includegraphics[width=.15\columnwidth]{figures/supplementary/2009_004497_gt.png}
  }
  \subfigure{%
    \includegraphics[width=.15\columnwidth]{figures/supplementary/2009_004497_cnn.png}
  }
  \subfigure{%
    \includegraphics[width=.15\columnwidth]{figures/supplementary/2009_004497_crf.png}
  }
  \subfigure{%
    \includegraphics[width=.15\columnwidth]{figures/supplementary/2009_004497_ours.png}
  }\\[-2ex]



  \setcounter{subfigure}{0}
  \subfigure[\scriptsize Input]{%
    \includegraphics[width=.15\columnwidth]{figures/supplementary/2010_001327_given.png}
  }
  \subfigure[\scriptsize Superpixels]{%
    \includegraphics[width=.15\columnwidth]{figures/supplementary/2010_001327_sp.png}
  }
  \subfigure[\scriptsize GT]{%
    \includegraphics[width=.15\columnwidth]{figures/supplementary/2010_001327_gt.png}
  }
  \subfigure[\scriptsize Deeplab]{%
    \includegraphics[width=.15\columnwidth]{figures/supplementary/2010_001327_cnn.png}
  }
  \subfigure[\scriptsize +DenseCRF]{%
    \includegraphics[width=.15\columnwidth]{figures/supplementary/2010_001327_crf.png}
  }
  \subfigure[\scriptsize Using BI]{%
    \includegraphics[width=.15\columnwidth]{figures/supplementary/2010_001327_ours.png}
  }
  \mycaption{Semantic Segmentation}{Example results of semantic segmentation
  on the Pascal VOC12 dataset.
  (d)~depicts the DeepLab CNN result, (e)~CNN + 10 steps of mean-field inference,
  (f~result obtained with bilateral inception (BI) modules (\bi{6}{2}+\bi{7}{6}) between \fc~layers.}
  \label{fig:semantic_visuals-app}
\end{figure*}


\definecolor{minc_1}{HTML}{771111}
\definecolor{minc_2}{HTML}{CAC690}
\definecolor{minc_3}{HTML}{EEEEEE}
\definecolor{minc_4}{HTML}{7C8FA6}
\definecolor{minc_5}{HTML}{597D31}
\definecolor{minc_6}{HTML}{104410}
\definecolor{minc_7}{HTML}{BB819C}
\definecolor{minc_8}{HTML}{D0CE48}
\definecolor{minc_9}{HTML}{622745}
\definecolor{minc_10}{HTML}{666666}
\definecolor{minc_11}{HTML}{D54A31}
\definecolor{minc_12}{HTML}{101044}
\definecolor{minc_13}{HTML}{444126}
\definecolor{minc_14}{HTML}{75D646}
\definecolor{minc_15}{HTML}{DD4348}
\definecolor{minc_16}{HTML}{5C8577}
\definecolor{minc_17}{HTML}{C78472}
\definecolor{minc_18}{HTML}{75D6D0}
\definecolor{minc_19}{HTML}{5B4586}
\definecolor{minc_20}{HTML}{C04393}
\definecolor{minc_21}{HTML}{D69948}
\definecolor{minc_22}{HTML}{7370D8}
\definecolor{minc_23}{HTML}{7A3622}
\definecolor{minc_24}{HTML}{000000}

\begin{figure*}[!ht]
  \small % scriptsize
  \centering
  \fcolorbox{white}{minc_1}{\rule{0pt}{4pt}\rule{4pt}{0pt}} Brick~~
  \fcolorbox{white}{minc_2}{\rule{0pt}{4pt}\rule{4pt}{0pt}} Carpet~~
  \fcolorbox{white}{minc_3}{\rule{0pt}{4pt}\rule{4pt}{0pt}} Ceramic~~
  \fcolorbox{white}{minc_4}{\rule{0pt}{4pt}\rule{4pt}{0pt}} Fabric~~
  \fcolorbox{white}{minc_5}{\rule{0pt}{4pt}\rule{4pt}{0pt}} Foliage~~
  \fcolorbox{white}{minc_6}{\rule{0pt}{4pt}\rule{4pt}{0pt}} Food~~
  \fcolorbox{white}{minc_7}{\rule{0pt}{4pt}\rule{4pt}{0pt}} Glass~~
  \fcolorbox{white}{minc_8}{\rule{0pt}{4pt}\rule{4pt}{0pt}} Hair~~\\
  \fcolorbox{white}{minc_9}{\rule{0pt}{4pt}\rule{4pt}{0pt}} Leather~~
  \fcolorbox{white}{minc_10}{\rule{0pt}{4pt}\rule{4pt}{0pt}} Metal~~
  \fcolorbox{white}{minc_11}{\rule{0pt}{4pt}\rule{4pt}{0pt}} Mirror~~
  \fcolorbox{white}{minc_12}{\rule{0pt}{4pt}\rule{4pt}{0pt}} Other~~
  \fcolorbox{white}{minc_13}{\rule{0pt}{4pt}\rule{4pt}{0pt}} Painted~~
  \fcolorbox{white}{minc_14}{\rule{0pt}{4pt}\rule{4pt}{0pt}} Paper~~
  \fcolorbox{white}{minc_15}{\rule{0pt}{4pt}\rule{4pt}{0pt}} Plastic~~\\
  \fcolorbox{white}{minc_16}{\rule{0pt}{4pt}\rule{4pt}{0pt}} Polished Stone~~
  \fcolorbox{white}{minc_17}{\rule{0pt}{4pt}\rule{4pt}{0pt}} Skin~~
  \fcolorbox{white}{minc_18}{\rule{0pt}{4pt}\rule{4pt}{0pt}} Sky~~
  \fcolorbox{white}{minc_19}{\rule{0pt}{4pt}\rule{4pt}{0pt}} Stone~~
  \fcolorbox{white}{minc_20}{\rule{0pt}{4pt}\rule{4pt}{0pt}} Tile~~
  \fcolorbox{white}{minc_21}{\rule{0pt}{4pt}\rule{4pt}{0pt}} Wallpaper~~
  \fcolorbox{white}{minc_22}{\rule{0pt}{4pt}\rule{4pt}{0pt}} Water~~
  \fcolorbox{white}{minc_23}{\rule{0pt}{4pt}\rule{4pt}{0pt}} Wood~~\\
  \subfigure{%
    \includegraphics[width=.15\columnwidth]{figures/supplementary/000008468_given.png}
  }
  \subfigure{%
    \includegraphics[width=.15\columnwidth]{figures/supplementary/000008468_sp.png}
  }
  \subfigure{%
    \includegraphics[width=.15\columnwidth]{figures/supplementary/000008468_gt.png}
  }
  \subfigure{%
    \includegraphics[width=.15\columnwidth]{figures/supplementary/000008468_cnn.png}
  }
  \subfigure{%
    \includegraphics[width=.15\columnwidth]{figures/supplementary/000008468_crf.png}
  }
  \subfigure{%
    \includegraphics[width=.15\columnwidth]{figures/supplementary/000008468_ours.png}
  }\\[-2ex]

  \subfigure{%
    \includegraphics[width=.15\columnwidth]{figures/supplementary/000009053_given.png}
  }
  \subfigure{%
    \includegraphics[width=.15\columnwidth]{figures/supplementary/000009053_sp.png}
  }
  \subfigure{%
    \includegraphics[width=.15\columnwidth]{figures/supplementary/000009053_gt.png}
  }
  \subfigure{%
    \includegraphics[width=.15\columnwidth]{figures/supplementary/000009053_cnn.png}
  }
  \subfigure{%
    \includegraphics[width=.15\columnwidth]{figures/supplementary/000009053_crf.png}
  }
  \subfigure{%
    \includegraphics[width=.15\columnwidth]{figures/supplementary/000009053_ours.png}
  }\\[-2ex]




  \subfigure{%
    \includegraphics[width=.15\columnwidth]{figures/supplementary/000014977_given.png}
  }
  \subfigure{%
    \includegraphics[width=.15\columnwidth]{figures/supplementary/000014977_sp.png}
  }
  \subfigure{%
    \includegraphics[width=.15\columnwidth]{figures/supplementary/000014977_gt.png}
  }
  \subfigure{%
    \includegraphics[width=.15\columnwidth]{figures/supplementary/000014977_cnn.png}
  }
  \subfigure{%
    \includegraphics[width=.15\columnwidth]{figures/supplementary/000014977_crf.png}
  }
  \subfigure{%
    \includegraphics[width=.15\columnwidth]{figures/supplementary/000014977_ours.png}
  }\\[-2ex]


  \subfigure{%
    \includegraphics[width=.15\columnwidth]{figures/supplementary/000022922_given.png}
  }
  \subfigure{%
    \includegraphics[width=.15\columnwidth]{figures/supplementary/000022922_sp.png}
  }
  \subfigure{%
    \includegraphics[width=.15\columnwidth]{figures/supplementary/000022922_gt.png}
  }
  \subfigure{%
    \includegraphics[width=.15\columnwidth]{figures/supplementary/000022922_cnn.png}
  }
  \subfigure{%
    \includegraphics[width=.15\columnwidth]{figures/supplementary/000022922_crf.png}
  }
  \subfigure{%
    \includegraphics[width=.15\columnwidth]{figures/supplementary/000022922_ours.png}
  }\\[-2ex]


  \subfigure{%
    \includegraphics[width=.15\columnwidth]{figures/supplementary/000025711_given.png}
  }
  \subfigure{%
    \includegraphics[width=.15\columnwidth]{figures/supplementary/000025711_sp.png}
  }
  \subfigure{%
    \includegraphics[width=.15\columnwidth]{figures/supplementary/000025711_gt.png}
  }
  \subfigure{%
    \includegraphics[width=.15\columnwidth]{figures/supplementary/000025711_cnn.png}
  }
  \subfigure{%
    \includegraphics[width=.15\columnwidth]{figures/supplementary/000025711_crf.png}
  }
  \subfigure{%
    \includegraphics[width=.15\columnwidth]{figures/supplementary/000025711_ours.png}
  }\\[-2ex]


  \subfigure{%
    \includegraphics[width=.15\columnwidth]{figures/supplementary/000034473_given.png}
  }
  \subfigure{%
    \includegraphics[width=.15\columnwidth]{figures/supplementary/000034473_sp.png}
  }
  \subfigure{%
    \includegraphics[width=.15\columnwidth]{figures/supplementary/000034473_gt.png}
  }
  \subfigure{%
    \includegraphics[width=.15\columnwidth]{figures/supplementary/000034473_cnn.png}
  }
  \subfigure{%
    \includegraphics[width=.15\columnwidth]{figures/supplementary/000034473_crf.png}
  }
  \subfigure{%
    \includegraphics[width=.15\columnwidth]{figures/supplementary/000034473_ours.png}
  }\\[-2ex]


  \subfigure{%
    \includegraphics[width=.15\columnwidth]{figures/supplementary/000035463_given.png}
  }
  \subfigure{%
    \includegraphics[width=.15\columnwidth]{figures/supplementary/000035463_sp.png}
  }
  \subfigure{%
    \includegraphics[width=.15\columnwidth]{figures/supplementary/000035463_gt.png}
  }
  \subfigure{%
    \includegraphics[width=.15\columnwidth]{figures/supplementary/000035463_cnn.png}
  }
  \subfigure{%
    \includegraphics[width=.15\columnwidth]{figures/supplementary/000035463_crf.png}
  }
  \subfigure{%
    \includegraphics[width=.15\columnwidth]{figures/supplementary/000035463_ours.png}
  }\\[-2ex]


  \setcounter{subfigure}{0}
  \subfigure[\scriptsize Input]{%
    \includegraphics[width=.15\columnwidth]{figures/supplementary/000035993_given.png}
  }
  \subfigure[\scriptsize Superpixels]{%
    \includegraphics[width=.15\columnwidth]{figures/supplementary/000035993_sp.png}
  }
  \subfigure[\scriptsize GT]{%
    \includegraphics[width=.15\columnwidth]{figures/supplementary/000035993_gt.png}
  }
  \subfigure[\scriptsize AlexNet]{%
    \includegraphics[width=.15\columnwidth]{figures/supplementary/000035993_cnn.png}
  }
  \subfigure[\scriptsize +DenseCRF]{%
    \includegraphics[width=.15\columnwidth]{figures/supplementary/000035993_crf.png}
  }
  \subfigure[\scriptsize Using BI]{%
    \includegraphics[width=.15\columnwidth]{figures/supplementary/000035993_ours.png}
  }
  \mycaption{Material Segmentation}{Example results of material segmentation.
  (d)~depicts the AlexNet CNN result, (e)~CNN + 10 steps of mean-field inference,
  (f)~result obtained with bilateral inception (BI) modules (\bi{7}{2}+\bi{8}{6}) between
  \fc~layers.}
\label{fig:material_visuals-app}
\end{figure*}


\definecolor{city_1}{RGB}{128, 64, 128}
\definecolor{city_2}{RGB}{244, 35, 232}
\definecolor{city_3}{RGB}{70, 70, 70}
\definecolor{city_4}{RGB}{102, 102, 156}
\definecolor{city_5}{RGB}{190, 153, 153}
\definecolor{city_6}{RGB}{153, 153, 153}
\definecolor{city_7}{RGB}{250, 170, 30}
\definecolor{city_8}{RGB}{220, 220, 0}
\definecolor{city_9}{RGB}{107, 142, 35}
\definecolor{city_10}{RGB}{152, 251, 152}
\definecolor{city_11}{RGB}{70, 130, 180}
\definecolor{city_12}{RGB}{220, 20, 60}
\definecolor{city_13}{RGB}{255, 0, 0}
\definecolor{city_14}{RGB}{0, 0, 142}
\definecolor{city_15}{RGB}{0, 0, 70}
\definecolor{city_16}{RGB}{0, 60, 100}
\definecolor{city_17}{RGB}{0, 80, 100}
\definecolor{city_18}{RGB}{0, 0, 230}
\definecolor{city_19}{RGB}{119, 11, 32}
\begin{figure*}[!ht]
  \small % scriptsize
  \centering


  \subfigure{%
    \includegraphics[width=.18\columnwidth]{figures/supplementary/frankfurt00000_016005_given.png}
  }
  \subfigure{%
    \includegraphics[width=.18\columnwidth]{figures/supplementary/frankfurt00000_016005_sp.png}
  }
  \subfigure{%
    \includegraphics[width=.18\columnwidth]{figures/supplementary/frankfurt00000_016005_gt.png}
  }
  \subfigure{%
    \includegraphics[width=.18\columnwidth]{figures/supplementary/frankfurt00000_016005_cnn.png}
  }
  \subfigure{%
    \includegraphics[width=.18\columnwidth]{figures/supplementary/frankfurt00000_016005_ours.png}
  }\\[-2ex]

  \subfigure{%
    \includegraphics[width=.18\columnwidth]{figures/supplementary/frankfurt00000_004617_given.png}
  }
  \subfigure{%
    \includegraphics[width=.18\columnwidth]{figures/supplementary/frankfurt00000_004617_sp.png}
  }
  \subfigure{%
    \includegraphics[width=.18\columnwidth]{figures/supplementary/frankfurt00000_004617_gt.png}
  }
  \subfigure{%
    \includegraphics[width=.18\columnwidth]{figures/supplementary/frankfurt00000_004617_cnn.png}
  }
  \subfigure{%
    \includegraphics[width=.18\columnwidth]{figures/supplementary/frankfurt00000_004617_ours.png}
  }\\[-2ex]

  \subfigure{%
    \includegraphics[width=.18\columnwidth]{figures/supplementary/frankfurt00000_020880_given.png}
  }
  \subfigure{%
    \includegraphics[width=.18\columnwidth]{figures/supplementary/frankfurt00000_020880_sp.png}
  }
  \subfigure{%
    \includegraphics[width=.18\columnwidth]{figures/supplementary/frankfurt00000_020880_gt.png}
  }
  \subfigure{%
    \includegraphics[width=.18\columnwidth]{figures/supplementary/frankfurt00000_020880_cnn.png}
  }
  \subfigure{%
    \includegraphics[width=.18\columnwidth]{figures/supplementary/frankfurt00000_020880_ours.png}
  }\\[-2ex]



  \subfigure{%
    \includegraphics[width=.18\columnwidth]{figures/supplementary/frankfurt00001_007285_given.png}
  }
  \subfigure{%
    \includegraphics[width=.18\columnwidth]{figures/supplementary/frankfurt00001_007285_sp.png}
  }
  \subfigure{%
    \includegraphics[width=.18\columnwidth]{figures/supplementary/frankfurt00001_007285_gt.png}
  }
  \subfigure{%
    \includegraphics[width=.18\columnwidth]{figures/supplementary/frankfurt00001_007285_cnn.png}
  }
  \subfigure{%
    \includegraphics[width=.18\columnwidth]{figures/supplementary/frankfurt00001_007285_ours.png}
  }\\[-2ex]


  \subfigure{%
    \includegraphics[width=.18\columnwidth]{figures/supplementary/frankfurt00001_059789_given.png}
  }
  \subfigure{%
    \includegraphics[width=.18\columnwidth]{figures/supplementary/frankfurt00001_059789_sp.png}
  }
  \subfigure{%
    \includegraphics[width=.18\columnwidth]{figures/supplementary/frankfurt00001_059789_gt.png}
  }
  \subfigure{%
    \includegraphics[width=.18\columnwidth]{figures/supplementary/frankfurt00001_059789_cnn.png}
  }
  \subfigure{%
    \includegraphics[width=.18\columnwidth]{figures/supplementary/frankfurt00001_059789_ours.png}
  }\\[-2ex]


  \subfigure{%
    \includegraphics[width=.18\columnwidth]{figures/supplementary/frankfurt00001_068208_given.png}
  }
  \subfigure{%
    \includegraphics[width=.18\columnwidth]{figures/supplementary/frankfurt00001_068208_sp.png}
  }
  \subfigure{%
    \includegraphics[width=.18\columnwidth]{figures/supplementary/frankfurt00001_068208_gt.png}
  }
  \subfigure{%
    \includegraphics[width=.18\columnwidth]{figures/supplementary/frankfurt00001_068208_cnn.png}
  }
  \subfigure{%
    \includegraphics[width=.18\columnwidth]{figures/supplementary/frankfurt00001_068208_ours.png}
  }\\[-2ex]

  \subfigure{%
    \includegraphics[width=.18\columnwidth]{figures/supplementary/frankfurt00001_082466_given.png}
  }
  \subfigure{%
    \includegraphics[width=.18\columnwidth]{figures/supplementary/frankfurt00001_082466_sp.png}
  }
  \subfigure{%
    \includegraphics[width=.18\columnwidth]{figures/supplementary/frankfurt00001_082466_gt.png}
  }
  \subfigure{%
    \includegraphics[width=.18\columnwidth]{figures/supplementary/frankfurt00001_082466_cnn.png}
  }
  \subfigure{%
    \includegraphics[width=.18\columnwidth]{figures/supplementary/frankfurt00001_082466_ours.png}
  }\\[-2ex]

  \subfigure{%
    \includegraphics[width=.18\columnwidth]{figures/supplementary/lindau00033_000019_given.png}
  }
  \subfigure{%
    \includegraphics[width=.18\columnwidth]{figures/supplementary/lindau00033_000019_sp.png}
  }
  \subfigure{%
    \includegraphics[width=.18\columnwidth]{figures/supplementary/lindau00033_000019_gt.png}
  }
  \subfigure{%
    \includegraphics[width=.18\columnwidth]{figures/supplementary/lindau00033_000019_cnn.png}
  }
  \subfigure{%
    \includegraphics[width=.18\columnwidth]{figures/supplementary/lindau00033_000019_ours.png}
  }\\[-2ex]

  \subfigure{%
    \includegraphics[width=.18\columnwidth]{figures/supplementary/lindau00052_000019_given.png}
  }
  \subfigure{%
    \includegraphics[width=.18\columnwidth]{figures/supplementary/lindau00052_000019_sp.png}
  }
  \subfigure{%
    \includegraphics[width=.18\columnwidth]{figures/supplementary/lindau00052_000019_gt.png}
  }
  \subfigure{%
    \includegraphics[width=.18\columnwidth]{figures/supplementary/lindau00052_000019_cnn.png}
  }
  \subfigure{%
    \includegraphics[width=.18\columnwidth]{figures/supplementary/lindau00052_000019_ours.png}
  }\\[-2ex]




  \subfigure{%
    \includegraphics[width=.18\columnwidth]{figures/supplementary/lindau00027_000019_given.png}
  }
  \subfigure{%
    \includegraphics[width=.18\columnwidth]{figures/supplementary/lindau00027_000019_sp.png}
  }
  \subfigure{%
    \includegraphics[width=.18\columnwidth]{figures/supplementary/lindau00027_000019_gt.png}
  }
  \subfigure{%
    \includegraphics[width=.18\columnwidth]{figures/supplementary/lindau00027_000019_cnn.png}
  }
  \subfigure{%
    \includegraphics[width=.18\columnwidth]{figures/supplementary/lindau00027_000019_ours.png}
  }\\[-2ex]



  \setcounter{subfigure}{0}
  \subfigure[\scriptsize Input]{%
    \includegraphics[width=.18\columnwidth]{figures/supplementary/lindau00029_000019_given.png}
  }
  \subfigure[\scriptsize Superpixels]{%
    \includegraphics[width=.18\columnwidth]{figures/supplementary/lindau00029_000019_sp.png}
  }
  \subfigure[\scriptsize GT]{%
    \includegraphics[width=.18\columnwidth]{figures/supplementary/lindau00029_000019_gt.png}
  }
  \subfigure[\scriptsize Deeplab]{%
    \includegraphics[width=.18\columnwidth]{figures/supplementary/lindau00029_000019_cnn.png}
  }
  \subfigure[\scriptsize Using BI]{%
    \includegraphics[width=.18\columnwidth]{figures/supplementary/lindau00029_000019_ours.png}
  }%\\[-2ex]

  \mycaption{Street Scene Segmentation}{Example results of street scene segmentation.
  (d)~depicts the DeepLab results, (e)~result obtained by adding bilateral inception (BI) modules (\bi{6}{2}+\bi{7}{6}) between \fc~layers.}
\label{fig:street_visuals-app}
\end{figure*}


\end{document}
