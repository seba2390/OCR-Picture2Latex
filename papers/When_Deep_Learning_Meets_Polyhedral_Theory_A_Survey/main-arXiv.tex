\documentclass{article}
\usepackage[utf8]{inputenc}

% Another title idea:
% - Seeing Deep Learning through Discrete Optimization and Polyhedral Theory: A Tutorial

% The title below fits EJOR's template in 2 lines
% - Discrete Optimization and Polyhedral Theory in Deep Learning: Connections, Insights, and Opportunities
\title{When Deep Learning Meets Polyhedral Theory: A Survey}
\author{
Joey Huchette\\{\footnotesize Google Research, USA} \and
Gonzalo Mu\~{n}oz\\{\footnotesize Universidad de O'Higgins, Chile} \and 
Thiago Serra\\{\footnotesize Bucknell University, USA} \and
Calvin Tsay\\{\footnotesize Imperial College London, UK}
}
\date{September 2023}

\usepackage[comma]{natbib}
\bibliographystyle{abbrvnat}

%%%%% NEW MATH DEFINITIONS %%%%%

\usepackage{amsmath,amsfonts,bm}
\usepackage{xifthen}

% Highlight a newly defined term
\newcommand{\newterm}[1]{{\bf #1}}

\def\eps{{\epsilon}}


% Utility for ticks 
\newcommand{\cmark}{\ding{51}}%
\newcommand{\xmark}{\ding{55}}%

% Theorem styles 
\theoremstyle{definition}
\newtheorem{theorem}{Theorem}[section]
\newtheorem{definition}{Definition}[section]
% \newtheorem{remark}{Remark}[theorem] %numbered remark
\newtheorem*{remark}{Remark} %unnumbered remark
\newtheorem{lemma}{Lemma}[section]
\newtheorem{prop}{Proposition}[section]
\newtheorem{corollary}{Corollary}[theorem]
\newtheorem{conjecture}{Conjecture}[section]
\newtheorem{assumption}{Assumption}[section]

\newtheorem{manualtheoreminner}{Theorem}
\newenvironment{manualtheorem}[1]{%
  \renewcommand\themanualtheoreminner{#1}%
  \manualtheoreminner
}{\endmanualtheoreminner}


% Math helper - standard function
\DeclareMathOperator*{\argmax}{arg\,max}
\DeclareMathOperator*{\argmin}{arg\,min}
\DeclareMathOperator{\support}{support}
\DeclareMathOperator{\MAX}{MAX}
\DeclareMathOperator{\term}{\texttt{term}}
\DeclareMathOperator*{\logsumexp}{log-sum-exp}
\DeclareMathOperator*{\TV}{TV}
\newcommand{\norm}[1]{\left\lVert#1\right\rVert}
\DeclarePairedDelimiter\set\{\}
\DeclarePairedDelimiter\abs{\lvert}{\rvert}%
\newcommand*{\mytop}{\mathrel{\scalebox{0.5}{$\top$}}}
\newcommand*{\mybot}{\mathrel{\scalebox{0.5}{$\bot$}}}
\newcommand*{\mydiese}{\mathrel{\scalebox{0.5}{$\#$}}}
\newcommand*{\myplus}{\mathrel{\scalebox{0.5}{$+$}}}
\newcommand*{\myminus}{\mathrel{\scalebox{0.5}{$-$}}}
\newcommand*{\bmg}{\bm{\gamma}}
\newcommand*{\bml}{\bm{\lambda}}

% MDP notation
\renewcommand{\S}{\mathcal{S}}
\newcommand{\X}{\mathcal{X}}
\newcommand{\A}{\mathcal{A}}
\newcommand{\T}{\mathcal{T}}
\newcommand{\M}{\mathcal{M}}
\newcommand{\B}{\mathcal{B}}
\newcommand{\Bset}{\mathfrak{B}}
\newcommand{\Dist}{\mathscr{P}}
\newcommand{\D}{\mathcal{D}}
\newcommand{\Real}{\mathbb{R}}
\renewcommand{\P}{\mathcal{P}}
\newcommand{\E}{\mathop{\mathbb{E}}}
\renewcommand{\H}{\mathcal{H}}
% \newcommand{\R}{\mathcal{R}}
% \newcommand{\C}{\mathcal{C}}

% Extended MDP notation
\newcommand{\Pstar}{p^{\star}}
\newcommand{\Rstar}{\bm{r}^{\star}}
\newcommand{\Cstar}{C^{\star}}
% \newcommand{\rmax}{\textsc{Rmax}}
\newcommand{\rmax}{r_{\mytop}}
\newcommand{\cmax}{\textsc{Cmax}}

\newcommand{\mstar}{m^{\star}}
\newcommand{\mhat}{\hat{m}}
\newcommand{\mopt}{m^{\star}}

\newcommand{\Phat}{\hat{p}}
\newcommand{\Rhat}{\hat{\bm{r}}}
\newcommand{\Chat}{\hat{C}}

% Math helper - custom function
\newcommand{\expwrtpi}[1]{\E_{\pi} [\sum_{t=0}^{\infty} \gamma^t #1(s_t, a_t)]}
\newcommand{\expangle}[1]{\langle #1  \rangle}

% helper function for return and constraints

% for value function, takes arguments:
% #1: policy 
% #2: the function of interest, R or C_i
% #3 (optional): the MDP for which this is estimated
\newcommand{\V}[3]{ %
    \ifthenelse{\isempty{#3}}%
    {V^{#1}(#2)}% #3 is empty 
    {V^{#1}_{#3}(#2)}%
}

\newcommand{\Q}[3]{
    \ifthenelse{\isempty{#3}}
    {Q^{#1}(#2)}% #3 is empty 
    {Q^{#1}_{#3}(#2)}%
}


\newcommand{\Adv}[3]{
    \ifthenelse{\isempty{#3}}
    {A^{#1}(#2)}% #3 is empty 
    {A^{#1}_{#3}(#2)}%
}

% careful diff notation
% 1: pi
% 2: R/C
% 3: M
\newcommand{\J}[3]{
    \ifthenelse{\isempty{#3}}
    {\mathcal{J}^{#1}_{#2}}% #3 is empty -> eg V^{\pi}(x ; R)
    {\mathcal{J}^{#1}_{#3,#2}}% -? eg V^{\pi}_{M}(x ; C)
    % {J_{#2}(#1)}% #3 is empty 
    % {J_{#2}(#1, #3)} %
}



\newcommand{\MRkern}{%
  \mkern-6.5mu
  \mathchoice{}{}{\mkern0.2mu}{\mkern0.5mu}%
}

% for value function, takes arguments:
% #1: policy 
% #2: the function of interest, R or C_i
% #3 (optional): the MDP for which this is estimated
% #4: variables to be given input (x) or (x,a)
\newcommand{\val}[4]{ %
    \ifthenelse{\isempty{#3}}%
    {v^{#1}_{#2}(#4)}% #3 is empty -> eg V^{\pi}(x ; R)
    {v^{#1}_{#3,#2}(#4)}% -? eg V^{\pi}_{M}(x ; C)
    % {V^{#1}_{#3}(#4 ;#2)}% -? eg V^{\pi}_{M}(x ; C)
    % {V_{#2}(#4 ; #1)}% #3 is empty -> eg V_R(x ; \pi)
    % {V_{#2}(#4 ;#1, #3)}% -? eg V_C(x ; \pi, M)
    % {#2 \MRkern V^{#1}_{#3}(#4)}% -? eg V^{\pi}_{M}(x ; C) # combines the letter V and R together
}

\newcommand{\qval}[4]{
    \ifthenelse{\isempty{#3}}
    {q^{#1}_{#2}(#4)}% #3 is empty -> eg V^{\pi}(x ; R)
    {q^{#1}_{#3,#2}(#4)}% -? eg V^{\pi}_{M}(x ; C)
    % {Q^{#1}(#4 ; #2)}% #3 is empty -> eg Q^{\pi}(x,a ; R)
    % {Q^{#1}_{#3}(#4 ;#2)}% -? eg Q^{\pi}_{M}(x,a ; C)
    % {Q_{#2}(#4 ; #1)}% #3 is empty -> eg Q_R(x,a ; \pi)
    % {Q_{#2}(#4 ;#1, #3)}% -? eg Q_C(x,a ; \pi, M)
}
\DeclareMathOperator*{\advantage}{Adv}

\newcommand{\adv}[4]{
    \ifthenelse{\isempty{#3}}
    {\advantage^{#1}_{#2}(#4)}% #3 is empty -> eg V^{\pi}(x ; R)
    {\advantage^{#1}_{#3,#2}(#4)}% -? eg V^{\pi}_{M}(x ; C)
    % {A^{#1}(#4 ; #2)}% #3 is empty -> eg Q^{\pi}(x,a ; R)
    % {A^{#1}_{#3}(#4 ;#2)}% -? eg Q^{\pi}_{M}(x,a ; C)
    % {A_{#2}(#4 ; #1)}% #3 is empty -> eg A_R(x,a ; \pi)
    % {A_{#2}(#4 ;#1, #3)}% -? eg A_C(x,a ; \pi, M)
}




\newcommand{\ci}{C}

\newcommand{\pib}{\pi_{b}}
\newcommand{\piopt}{\pi^{*}}
\newcommand{\pie}{\pi_{t}}

\newcommand{\lR}{\lambda_{R}}
\newcommand{\lC}{\lambda_{C}}
\newcommand{\ephi}{e_{\phi}}

\newcommand{\pr}{\text{Pr}}
\newcommand{\IS}{\text{IS}}
\newcommand{\CI}{\text{CI}}


% SPIBB symbols 
\newcommand{\EpsPib}{(\pi_b, e, \epsilon)}

\usepackage{array}
\usepackage{multirow}

\usepackage{todonotes}
\newtheorem{theorem}{Theorem}
\newtheorem{definition}{Definition}
\newtheorem{example}{Example}

\usepackage{lscape}

%\usepackage{pdflscape}
\usepackage{adjustbox}
\usepackage{url}

\usepackage{tikz-network}
\begin{document}

\maketitle

\begin{abstract}
\noindent 
% Motivation
In the past decade, deep learning became the prevalent methodology for predictive modeling thanks to the remarkable accuracy of deep neural networks in tasks such as computer vision and natural language processing. 
% Problem 
Meanwhile, the structure of neural networks converged back to simpler representations based on piecewise constant and piecewise linear functions such as the Rectified Linear Unit~(ReLU), 
which became the most commonly used type of activation function in neural networks. 
That made certain types of network structure ---such as the typical fully-connected feedforward neural network--- amenable to analysis 
through polyhedral theory and to the application of methodologies such as Linear Programming~(LP) and Mixed-Integer Linear Programming~(MILP) for a variety of purposes. 
% Approach 
In this paper, 
we survey the main topics emerging from this fast-paced area of work, 
% 
which brings a fresh perspective to understanding neural networks in more detail as well as to applying linear optimization techniques to train, verify, and reduce the size of such networks.  
\end{abstract}

\section{Introduction}  \label{sec:introduction}

\newcommand\inexpIntro[3]{#1?(#2,#3).}
\newcommand\rinexpIntro[3]{*#1?(#2,#3).}
\newcommand\outexpIntro[3]{#1!(#2,#3).}
\newcommand\outatomIntro[3]{#1!(#2,#3)}

We propose a fully automated method for proving termination of \(\pi\)-calculus processes.
Although there have been a lot of studies on termination analysis for the \(\pi\)-calculus
and related calculi~\cite{Deng06IC,Demangeon07,SangiorgiTermination,KobayashiHybrid,Yoshida04IC,DBLP:journals/jlp/DemangeonHS10,Venet98SAS}, most of them have been rather theoretical,
and there have been surprisingly little efforts in developing  fully automated termination
verification methods and tools based on them. To our knowledge,
Kobayashi's \typical{}~\cite{TyPiCal,KobayashiHybrid} is the only exception that
can prove termination of \(\pi\)-calculus processes (extended with natural numbers)
fully automatically, but its termination analysis is quite limited (see Section~\ref{sec:relatedwork}).

Our method is based on a reduction to termination analysis for sequential programs:
we translate a \(\pi\)-calculus process \(P\) to a sequential program \(S_P\), so that
if \(S_P\) is terminating, so is \(P\). The reduction allows us to use
powerful, mature methods and tools
for termination analysis of sequential programs~\cite{heizmann2016ultimate,freqterm,DBLP:conf/lics/PodelskiR04,Kuwahara2014Termination,DBLP:journals/cacm/CookPR11}.

The idea of the translation is to convert a chain of communications on replicated input
channels to a chain of recursive function calls of the target sequential program.
Let us consider the following Fibonacci process:
\begin{align*}
    & \rinexpIntro{\fib}{n}{r}
        \ifexp{n<2}{ \soutatom{r}{1} \\ &\quad}
                   { \nuexp{s_1} \nuexp{s_2} (\outatomIntro{\fib}{n-1}{s_1} \PAR \outatomIntro{\fib}{n-2}{s_2} \PAR \sinexp{s_1}{x}\sinexp{s_2}{y}\soutatom{r}{x+y}) \\}
    & \PAR \outatomIntro{\fib}{m}{r}
\end{align*}
Here, the process
$\rinexpIntro{\fib}{n}{r} \ldots$ is a function server that computes the \(n\)-th Fibonacci number
in parallel and returns the result to \(r\),
and $\outatom{\fib}{m}{r}$ sends a request for computing the \(m\)-th Fibonacci number;
those who are not familiar with the syntax of the \(\pi\)-calculus may wish to consult
Section~\ref{sec:targetlanguage} first.
To prove that the process above is terminating for any integer \(m\),
it suffices to show that there is no infinite chain of communications on $\fib$:
\[
    \fib(m,r) \to \fib(m_1,r_1) \to \fib(m_2,r_2) \to \cdots.
\]
We convert the process above to the following program:\footnote{The actual translation
  given later is a little more complex.}
\begin{verbatim}
 let rec fib(n) = if n<2 then () else (fib(n-1) [] fib(n-2)) in
 fib(m)
\end{verbatim}
Here, \texttt{[]} represents the non-deterministic choice.
Note that, although the calculation of Fibonacci numbers is not preserved,
for each chain of communications on \texttt{fib}, there is a corresponding
sequence of recursive calls:
\[
\mathtt{fib}(m) \to \mathtt{fib}(m_1) \to \mathtt{fib}(m_2) \to \cdots.
\]
Thus, the termination of the sequential program above implies the termination of
the original process.
As shown in the example above, (i) each communication on a replicated input channel
is converted to a function call, (ii) each communication on a non-replicated input
channel is just removed (or, in the actual translation, replaced by a call of
a trivial function defined by \(f(\seq{x})=(\,)\)), and (iii) parallel composition
is replaced by a non-deterministic choice.
We formalize the translation outlined above and prove its correctness.

The basic translation sketched above sometimes loses too much information.
For example, consider the following process:
\begin{align*}
    & \rinexpIntro{\pre}{n}{r} \soutatom{r}{n-1} \\
    & \PAR \rinexpIntro{f}{n}{r} \ifexp{n<0}{ \soutatom{r}{1} }
                                       { \nuexp{s} (\outatomIntro{\pre}{n}{s} \PAR \sinexp{s}{x}\outatomIntro{f}{x}{r}) } \\
    & \PAR \outatomIntro{f}{m}{r}
\end{align*}
The translation sketched above would yield:
\begin{verbatim}
  let pred(n) = n-1 in
  let rec f(n) = if n<0 then () else (pred(n) [] f(*)) in
  f(m)
\end{verbatim}
Here, \texttt{*} represents a non-deterministic integer: since we have removed
the input $\sinatom{s}{x}$, we do not have information about the value of \( x \).
As a result, the sequential program above is non-terminating, although the original
process is terminating.
To remedy this problem, we also refine the basic translation above by using a refinement
type system for the \(\pi\)-calculus. Using the refinement type system,
we can infer that the value of \(x\) in the original process is less than \(n\),
so that we can refine the definition of \texttt{f} to:
\begin{verbatim}
 let rec f(n) = ... else (pred(n) [] let x=* in assume(x<n);f(x))
\end{verbatim}
The target program is now terminating, from which
we can deduce that the original process is also terminating.
We have implemented an automated tool based on the refined translation above.

The contributions of this paper are summarized as follows.
\begin{itemize}
\item The formalization of the basic translation from the \(\pi\)-calculus
  (extended with integers) to sequential programs, and a proof of its correctness.
\item The formalization of a refined translation based on a refinement type system.
\item An implementation of the refined translation, including automated refinement type
  inference based on CHC solving, and experiments to evaluate the effectiveness of
  our method.
\end{itemize}

The rest of this paper is structured as follows.
Section~\ref{sec:targetlanguage} introduces the source and target languages
of our translation.
Section~\ref{sec:approach} 
formalizes the basic translation, and proves its correctness.
Section~\ref{sec:refinement} refines the basic translation by using a refinement type system.
Section~\ref{sec:implementation} reports an implementation and experiments.
Section~\ref{sec:relatedwork} discusses related work,
and Section~\ref{sec:conclusion} concludes the paper.




\section{The Polyhedral Perspective}\label{sec:poly}

A feedforward rectifier network models a piecewise linear function \citep{arora2018understanding} in which every such piece is a polyhedron \citep{raghu2017expressive}, 
and represents a special case among neural networks modeling piecewise polynomials \citep{balestriero2018spline}. 
Therefore, training a rectifier network is equivalent to performing a piecewise linear regression,  
and we can potentially interpret such neural networks in terms of what happens in each piece of the function that they model. 
However, we are only beginning to answer some of the questions entailed by such a remark. In this survey, we discuss how insights on this subject may help us answer the following questions.
%\todo[inline]{Q1: Section 3; Q2-Q4: Section 4; Q5-Q6: Section 5}
\begin{enumerate}
\item Which piecewise linear functions can or cannot be obtained from training a neural network given its architecture?
\item Which neural networks are more susceptible to adversarial exploitation?
\item Can we integrate the model learned by a neural network into a broader decision-making problem for which we want to find an optimal solution?
\item Is it possible to obtain a smaller neural network that models exactly the same function as another trained  neural network?
\item Can we exploit the polyhedral geometry present in neural networks in the training phase? 
\item Can we efficiently incorporate extra structure when training neural network, such as linear constraints over the weights?
\end{enumerate}

The first question complements the universal approximation results for neural networks. Namely, there is a limit to what functions can be well approximated when limited computational resources are translated into constraints on the depth and width of the layers of neural networks that can be used in practice. 
The functions that can be modeled depend on the particular choice of hyperparameters subject to the computational resources available, and in the long run that may also lead to a more principled approach for the choice of hyperparameters than the current approaches of neural architecture search. 
%
In Section~\ref{sec:LR}, 
we analyze how a rectifier network partitions the input space into pieces in which it behaves linearly, which we denote as \emph{linear regions}. 
We discuss the geometry of linear regions, the effect of parameters and hyperparameters on the number of linear regions of a neural network, and the extent to which such number of linear regions relates to the accuracy of the network. 
%\begin{itemize}
%    \item What is the effect of parameters and hyperparameters on the number of linear regions?
%    \item Does the number of linear regions defined by a neural network relate to its accuracy? 
%    \item Are some linear regions more important than others?
%\end{itemize}

The second question relies on formal verification methods to evaluate the robustness of neural networks, which can be approached with mathematical optimization formulations that are also relevant for the third and fourth questions. Such formulations are convenient since a direct inspection of every piece of the function modeled by a neural network is prohibitive given how quickly their number scale with the size of the network. 
The linear behavior of the network for every choice of active and inactive units implies that we can use linear formulations with binary variables corresponding to the activation of units to model trained neural networks using MILP.  Therefore, we are able to solve a variety of optimization problems over a trained neural network, such as the neural network verification problem, 
identifying the range of outputs for each ReLU of the network, and modeling a trained neural network as part of a larger decision-making problem. 
%In all of those cases, however, the sizes of the neural networks for which this is possible is still limited. 
%In fact, we can approach the third question through one such family of optimization problems. 
%
In Section~\ref{sec:optimizing}, we discuss how to formulate optimization problems over a trained neural network, the applications of such formulations, and the progress toward obtaining stronger formulations that scale more easily with the network size.

The fifth and sixth questions involve the training procedure of a DNN, where linear programming tools have been applied to partially answer them. In Section~\ref{sec:training}, we overview these developments. In terms of the fifth question ---exploiting polyhedrality in training neural networks--- we describe algorithms that use the polyhedral geometry induced by activation sets to solve training problems. We also cover a recently proposed polyhedral construction that can approximately encode multiple training problems at once, showing a strong relationship across training problems that arise from different datasets, for a fixed architecture. Additionally, we review some recent uses of mixed-integer linear programming in the training phase as an alternative to SGD when the weights are required to be integer. Regarding the sixth question ---the incorporation of extra structure when training--- we review multiple approaches that have included techniques related to linear programming within SGD to impose a desirable structure when training, or to find better step-lengths in the execution of SGD.

% Finally, we could potentially address the fourth question through approximation by  discretizing the weight space and enumerating all the possibilities. However, such a prohibitive approach is not necessary. 
%  we show that LP formulations can be used for that purpose in neural networks. In fact, there has been considerable work on LP-based training of neural networks with a single hidden layer in the past. Similar approaches have also been used in related learning problems, such as obtaining optimal classification trees with MILP~\cite{ClassificationTrees}. In addition, we can use polyhedra to characterize all the models that can be learned from datasets of a given size.



\section{The Linear Regions of a Neural Network}\label{sec:LR}

Every piece of the piecewise linear function modeled by a neural network is a linear region, 
and ---without loss of generality--- we can think of it as a polyhedron. 
In this section, we define a linear region, exemplify how they can be so numerous, and what may affect their count in a neural network. 
We also discuss the practical implications of such insights, as well as other related forms of analyzing the ability of a neural network to represent expressive models. 


\begin{definition}
A linear region corresponds to the set of points from the input space that activates the same units along the neural network, 
and hence can be characterized by the set $\sS^l$ of units that are active in each layer $l \in \sL$.
\end{definition}

% Reasons to care
Since a neural network behaves uniformly over a linear region, the latter is the smallest finite scale in which we can analyze its behavior. 
%Within each linear region 
If we restrict the domain of a neural network to a linear region $\sI \subseteq \mathbb{R}^{n_0}$, then the neural network behaves as an affine transformation $\vy_\sI : \sI \rightarrow \mathbb{R}^{n_{L}}$ of the form $\vy_\sI(\vx) = \mT \vx + \vt$ with a matrix $\mT \in \mathbb{R}^{n_{L} \times n_0}$ and a vector $\vt \in \mathbb{R}^{n_{L}}$ that are directly defined by the network parameters and the set of neurons that are activated by any input $\vx \in \sI$. 
For a small perturbation~$\varepsilon$ to some input $\overline{\vx} \in \sI$ such that $\overline{\vx}+\varepsilon \in \sI$, 
the network output for $\bar{x}+\varepsilon$ is given by $\vy_\sI(\overline{\vx}+\varepsilon)$. 
%
While it is possible that two adjacent regions defined in such way correspond to the same affine transformation, 
thinking of each linear region as having a distinct signature of active units makes it easier to analyze them.

The number of linear regions defined by a neural network is one form with which we can measure the complexity of the models that it can represent \citep{DeepArchitectures}. 
Hence, if a more complex model is desired, we may want to design a neural network that can potentially define more linear regions. %, and also 
%validate whether 
%understand whether the actual number relates to the accuracy of the trained network. 
On the one hand, the number of linear regions may grow exponentially on the depth of a neural network. 
On the other hand, such a number depends on the interplay between network parameters and hyperparameters. 
As we consider how the inputs from adjacent linear regions are evaluated, 
the change to the affine transformation can be characterized in algebraic and geometric terms. 
Understanding such changes may help us grasp how a neural network is capable of telling its inputs apart, including what are the sources of the complexity of the model. 

For neural networks in which the activation function is not piecewise linear, 
\cite{Bianchini2014} have used more elaborate topological measures to compare the expressiveness of shallow and deep neural networks. 
\cite{hu2020curve} followed a closer approach by producing a linear approximation neural network in which the number of linear regions can be counted.



\subsection{The combinatorial aspect of linear regions}

One of the most striking aspects about analyzing a neural network in terms of its linear regions is how quickly such number grows. 
Early work on this topic by~\cite{pascanu2013on} and \cite{montufar2014on} have drawn two important observations.
First, that it is possible to construct simple deep neural networks with a number of linear regions that grows exponentially in the depth.
Second, that the number of linear regions can be exponential in the number of neurons alone. 

The first observation comes from analyzing the role of ReLUs in a very simple setting. 
Namely, that of a neural network in which we regard every layer as having a single input in the $[0,1]$ domain, which is produced by combining the outputs of the units from the preceding layer, 
as illustrated by Example~\ref{ex:zigzag}. 
%
\begin{example}\label{ex:zigzag}
Consider a neural network with input $x$ from the domain $[0,1]$ and layers having 4 neurons with ReLU activation. 
For the first layer, assume that the output of the neurons are given by the following functions: 
$f_1(x)=\max\{4x,0\}$, $f_2(x)=\max\{8x-2,0\}$, $f_3(x)=\max\{6.5x-3.25,0\}$, and $f_4(x)=\max\{12.5x-11.25,0\}$. 
In other words, $\vh^1_i = f_i(x) ~\forall i \in \{1,2,3,4\}$. 
For the subsequent layers, assume that the outputs coming from the previous layer are combined through the function $F(x)=f_1(x)-f_2(x)+f_3(x)-f_4(x)$, 
which substitutes $x$ as the input to the next layer; upon which the same set of functions $\{ f_i(x) \}_{i=1}^4$ defines the output of the next layer. 
In other words, $\vh^l_i = f_i(F(\vh^{l-1})) = f_i(\evh_1^{l-1} - \evh_2^{l-1} + \evh_3^{l-1} - \evh_4^{l-1})~\forall i \in \{1,2,3,4\}, l \in \sL \setminus \{ 1 \}$.

When the output of the units in the first layer is combined as $F(x)$, we obtain a zigzagging function with 4 slopes in the $[0,1]$ domain, 
each of which defining a bijection between segments of the input ---namely, $[0,0.25]$, $[0.25, 0.5]$, $[0.5, 0.9]$, and $[0.9, 1.0]$--- and the image $[0,1]$. 
The effect of repeating such structure in the second layer is that of composing $F(x)$ with itself, 
with 4 slopes being produced within each of those 4 initial segments. 
Hence, the number of slopes ---and therefore of linear regions--- in the output of such a neural network with $L$ layers of activation functions is $4^L$, 
which implies an exponential growth on depth. 

The network structure and the parameters of the neurons in the first two layers are illustrated in Figure~\ref{fig:architecture_zigzag}; the set of functions $\{ f_i(x) \}_{i=1}^4$ and the combined outputs of the first two layers ---$F(x)$ and $F(F(x))$--- are illustrated in Figure~\ref{fig:zigzag}.
\end{example}
%
In Example~\ref{ex:zigzag}, every neuron changes the slope of the resulting function once it becomes active, 
in which we purposely alternate between positive and negative slopes once the function reaches either 0 or 1, respectively. 
By selecting the network parameters accordingly, 
\cite{montufar2014on} were the first to show that a layer with $n$ ReLUs can be used to create a zigzagging function with $n$ slopes on the $[0,1]$ domain, 
with the image along every slope also corresponding to the interval $[0,1]$. 
Consequently, stacking $L$ of such layers results in a neural network with $n^L$ linear regions.  

\begin{figure}[h!]
    \centering
    \includegraphics[width=\textwidth]{Fig_02a_Architecture.pdf}
    \caption{Mapping from the input $x \in [0,1]$ to the intermediary output $\vh^2 \in [0,1]^{4}$ through the first two layers of a neural network in which the number of linear regions growths exponentially on the depth, as described in Example~\ref{ex:zigzag}. 
    The parameters of subsequent layers are the same as those in the second layer.}
    \label{fig:architecture_zigzag}
\end{figure}

\begin{figure}[h!]
    \centering
    \includegraphics[width=\textwidth]{Fig_02_Zigzag.pdf}
    \caption{Set of activation functions $\{ f_i(x) \}_{i=1}^4$ of the units in the first layer and combined outputs of the first two layers ---$F(x) = f_1(x) - f_2(x) + f_3(x) - f_4(x)$ for the first and $F(F(x))$ for the second--- of a neural network in which the number of linear regions grows exponentially on the depth, as described in Example~\ref{ex:zigzag}.}
    \label{fig:zigzag}
\end{figure}

The second observation ---that the number of linear regions can grow exponentially in the number of neurons alone--- comes from the interplay between the parts of the input space in which each the units are active, 
especially in higher-dimensional spaces. 
This is based on some geometric observations that we discuss in Section~\ref{sec:geometry}. 
Even for a \emph{shallow} network ---i.e., the number of layers being $L=1$--- such a number of linear regions may approach $2^n$, 
which corresponds to every possible activation set $\sS \subseteq \{1, \ldots, n\}$ defining a nonempty linear region. 
However, as we discuss later, that is not always the case due to architectural choices such as the number of layers and their width.

 

%Following the characterization of linear regions by activation sets, that number can be as large as $2^{\sum_{i=1}^{L+1} n_l}$ \cite{montufar2014on}. In fact, that number can grow exponentially on the depth of the network.

%Early work  by~~\cite{pascanu2013on} and ~\cite{montufar2014on} focused on showing the benefit of depth by constructing neural networks with a large number of linear regions. 
%They have shown that layer $l$ can define $n_l$ linear regions corresponding to the slopes of a zigzagging function between 0 and 1 for an input of size $n_0=1$ in the range $[0,1]$. 
%If that same construction is used for the other layers, then the resulting neural network defines $\prod_{l=1}^{L+1} n_l$ linear regions. 
%Hence, a deep neural network with uniform layer width $n$ may define as many as $n^{L+1}$ linear regions.

\subsection{The algebra of linear regions}
\label{sec:algebraoflinearegions}

Given the activation sets $\{ \sS^l \}_{l \in \sL}$ denoting which neurons are active for each layer of the neural network, 
we can explicitly describe the affine transformation $\vy_\sI(\vx) = \mT \vx + \vt$ associated with the corresponding linear region $\sI$. 
For every activation set $\sS^l$, layer $l$ defines an affine transformation of the form $\Omega^{\sS^l}(\mW^l \vh^{l-1} + \vb^l)$, where $\Omega^{\sS^l}$ is a diagonal $n_l \times n_l$ matrix in which $\Omega^{\sS^l}_{ii} = 1$ if $i \in \sS^l$ and $\Omega^{\sS^l}_{ii} = 0$ otherwise. 
%Let $\mW^{l, \sS_l} \in \mathbb{R}^{n_l \times n_{l-1}}$ be a matrix in which $\mW^{l, \sS_l}_{i,j} = \mW^l_{i,j} ~ \forall i \in \sS_l, j \in \{1, \ldots, n_{l-1}\}$ and  $\mW^{l, \sS_l}_{i,j} = 0 ~ \forall i \in \{1, \ldots, n_l\} \setminus \sS_l, j \in \{1, \ldots, n_{l-1}\}$, i.e., we preserve the rows of $\mW^l$ associated with active units and replace the other rows with zeros. 
%Likewise, 
%let $\vb^{l, \sS_l} \in \mathbb{R}^{n_l}$ be a vector in which $\vb^{l, \sS_l} = \vb^l_i ~ \forall i \in \sS_l$ and $\vb^{l, \sS_l} = 0 ~ \forall i \in \{1, \ldots, n_l\} \setminus \sS_l$, 
%i.e., we preserve the elements of $\vb^l$ associated with active units and replace the other elements with zeros. 
Hence, the matrix $\mT$ and vector $\vt$ are as follows:
\[
%\mT = \prod_{l=1}^L \mW^{l, \sS_l} + \sum_{l_1=1}^{L-1} \left( \prod_{l_2=l_1+1}^{L} \mW^{l_2, \sS_{l_2}} \right) \vb^{l_1, \sS_{l_1}}
\mT = \prod_{l=1}^L \Omega^{\sS^l} \mW^{l},
\]
\[
\vt = \sum_{l_1=1}^{L} \left( \prod_{l_2=l_1+1}^{L} \Omega^{\sS^{l_2}} \mW^{l_2} \right) \Omega^{\sS^{l_1}} \vb^{l_1}.
\]
On a side note, 
\cite{takai2021functions} proposed a related metric for networks modeling piecewise linear functions by counting the number of distinct functions among linear regions upon equivalence through isometric affine transformation.

Each linear region is associated with a polyhedron, 
and we can describe the union of polyhedra $\mathcal{D}$ on the space $(\vx, \vh^1, \ldots, \vh^L)$ that covers the entire input space $x$ of the neural network as follows:
%We can also describe the inputs associated with those sets of active units across all layers as the following union of polyhedra 
\[
\mathcal{D} = 
\bigvee_{(\sS^1, \ldots, \sS^{L}) \subseteq \{1, \ldots, n_1\} \times \ldots \times \{1, \ldots, n_{L}\} }
\left(
\begin{array}{cc}
\vw_i^l \cdot \vh^{l-1} + b_i^l \geq 0 & \forall l \in \sL, i \in \sS^l \\
h_i^l = \vw_i^l \cdot \vh^{l-1} + b_i^l & \forall l \in \sL, i \in \sS^l \\
\vw_i^l \cdot h^{l-1} + b_i^l \leq 0 & \forall l \in \sL, i \notin \sS^l \\
h_i^l = 0 & \forall l \in \sL, i \notin \sS^l 
\end{array}
\right).
\]
Such partitioning entails an overlap between adjacent linear regions when $\vw_i^l \vh^{l-1} + b_i^l = 0$, i.e., at the boundary in which unit $i$ in layer $l$ is active in one region and inactive in another. 
Nevertheless, for any input $\overline{\vx}$ associated with a point at such a boundary between two linear regions $\sI_1$ and $\sI_2$, it holds that $\vy_{\sI_1}(\overline{\vx}) = \vy_{\sI_2}(\overline{\vx})$ even if those affine transformations are not entirely identical since the output of the neural network is continuous. More importantly, such overlap implies that each term of $\mathcal{D}$ is defined using only equalities and nonstrict inequalities, and therefore that each linear region corresponds to polyhedra in the extended space $(\vx, \vh^1, \ldots, \vh^{L})$. 
Consequently, those linear regions also define polyhedra if projected to the input space $\vx$, 
since by using Fourier-Motzkin elimination \citep{Fourier,Motzkin} 
we can obtain a polyhedral description of the linear region in $\vx$.  
Moreover, the interior of those polyhedra are disjoint. 
If one of those polyhedra does not have an interior, 
which means that it is not full-dimensional, 
then that linear region lies entirely on the boundary of other linear regions.
In such a case, we do not regard it as a proper linear region. 
By looking at the geometry of those linear regions from a different perspective in Section~\ref{sec:geometry} and understanding its impact on the number of linear regions in Section~\ref{sec:number}, we will see that many terms of $\mathcal{D}$ may actually be empty.

The optimization over the union of polyhedra is the subject of disjunctive programming, 
which has contributed to the development of stronger formulations and better algorithms to solve discrete optimization problems. These are formulated as MILPs as well as more general types of problems in recent years \citep{DPBook}, including generalized disjunctive programming for Mixed-Integer Non-Linear Programming~(MINLP) \citep{gdp1994first,gdp2012survey}. 
One of such contributions is the generation of valid inequalities to strengthen MILP formulations, which are also denoted as cutting planes, through the lift-and-project technique \citep{CGLP1,CGLP2}. 
In fact, we can develop stronger formulations for optimization problems involving neural networks through the lenses of disjunctive programming, as we discuss later in Section~\ref{sec:MIPmodels}.

%Another way to relate the study of linear regions with MILP formulations on neural networks is through disjunctive programming, which is the study of optimization over the union of polyhedra. Disjunctive programming has contributed to the development of stronger formulations and better algorithms to solve optimization problems that are formulated as MILP as well as more general types of problems in recent years [54]. One of such contributions is the generation of valid inequalities to strengthen MILP formulations, which are also denoted as cuts, through the lift-and-project technique [55]. The PI has contributed to this area by revisiting the optimization problem associated with generating lift-and-project cuts [56] and by assessing the extent to which lift-and-project cuts differ from cuts obtained with other techniques [57].

\subsection{The geometry of linear regions}\label{sec:geometry}

Another form of looking at the geometry of linear regions is through their transformation along the layers of the neural network. Namely, we can think of the input space as initially being partitioned by the units of the first layer, and then each resulting linear region being further partitioned by the subsequent layers. 
In that sense, we can think of every layer as a particular form of ``slicing'' the input. 
In fact, a layer may slice each linear region that is defined by the preceding layer in a different way due to which neurons are active or not in previous layers.  

% Geometry per layer

Let us begin by illustrating how a given layer $l \in \sL$ partitions its input space $\vh^{l-1}$. Every neuron $i$ in layer $l$ is associated with an \emph{activation hyperplane} of the form $\vw_i^l \cdot \vh^{l-1} + b_i^l = 0$, which divides the  possible inputs of its layer into an open half-space in which the unit is active ($\vw_i^l \cdot \vh^{l-1} + b_i^l > 0$) and a closed half-space in which the unit is inactive ($\vw_i^l \cdot \vh^{l-1} + b_i^l \leq 0$). These hyperplanes define the boundary between adjacent linear regions, and the arrangement of such hyperplanes for a given layer $l \in \sL$ determines how that layer partitions the $\vh^{l-1}$ space. 
In other words, every input in $\vh^{l-1}$ can be located with respect to each of those hyperplanes, which corresponds to the activation set of the linear region to which it belongs. 
However, not every activation set out of the $2^{n_l}$ possible ones maps to a nonempty region of the input space. In the case of Example~\ref{ex:hyperplane_arrangement}, there is no linear region in which the activation set is empty.

\begin{example}\label{ex:hyperplane_arrangement}
Consider a neural network with domain $\vx \in \mathbb{R}^2$ and a single layer having 3 neurons $\alpha$, $\beta$, and $\gamma$ with outputs given as follows: 
$h^1_{\alpha} = \max\{ 2.3 x_1 - 1.9 x_2 +0.6, 0\}$, $h^1_{\beta} = \max\{ -0.9 x_1 - 0.7 x_2 +4.8, 0\}$, and $h^1_{\gamma} = \max\{ 0 x_1 + 3 x_2 -5, 0\}$. 
These neurons define the activation hyperplanes ($\alpha$) $2.3 x_1 - 1.9 x_2 +0.6 = 0$, ($\beta$) $-0.9 x_1 - 0.7 x_2 +4.8 = 0$, and ($\gamma$) $0 x_1 + 3 x_2 -5 = 0$ in the space $\vx$, 
which are illustrated along with the activation sets of the linear regions in Figure~\ref{fig:hyperplane_arrangement}.

Instead of $2^3$ linear regions corresponding to each possible activation set defined by a subset of the neurons in $\{\alpha, \beta, \gamma\}$, the arrangement of such hyperplanes produces 7 linear regions, 
which is in fact the maximum number of 2-dimensional regions that can be defined by drawing 3 lines on a plane. 
\end{example}

\begin{figure}
    \centering
    \includegraphics[width=0.4\textwidth]{Fig_04_HA.pdf}
    \caption{Linear regions defined by the shallow neural network described in Example~\ref{ex:hyperplane_arrangement}. 
    Every line corresponds to the activation hyperplane of a different neuron, which is given by $\alpha$, $\beta$, and $\gamma$ in parentheses. 
    The arrow next to each line points to the half space in which the inputs activate that neuron.
    Every linear region has a subset of $\{\alpha,\beta,\gamma\}$ as its corresponding activation set.}
    \label{fig:hyperplane_arrangement}
\end{figure}

The maximum number of full-dimensional regions resulting from a partitioning defined by $n$ hyperplanes depends on the dimension $d$ of the space in which those hyperplanes are defined \citep{Zaslavsky1975}. 
That number never exceeds $\sum\limits_{i=1}^{\min\{d,n\}} \binom{n}{i}$. Such bound only coincides with $2^n$ if $d \geq n$; otherwise, as illustrated in Example~\ref{ex:hyperplane_arrangement}, that number can be smaller. 
As observed by~\cite{hanin2019deep}, 
that bound is $O\left( \frac{n^d}{d!} \right)$ when $n \gg d$.

In fact,
the above bound is all that we need to determine the maximum number of linear regions in shallow networks. 
While not every shallow network may define as many linear regions, it is always possible to put the hyperplanes in what is called a \emph{general position} in order to reach that bound. 
Thus, the maximum number of linear regions defined by a shallow network is $\sum\limits_{i=0}^{\min\{n_0,n_1\}} \binom{n_1}{n_0}$.

For the polyhedron associated with each linear region, 
being in general position implies that each vertex lies on exactly $d$ activation hyperplanes. 
For context, 
the converse situation in linear programming ---having more hyperplanes active on a vertex than the space dimension--- characterizes degeneracy. 

% Transformation across layers

In the case of deep networks, the partitioning of each linear region by the subsequent layers is based on the output of that linear region. This affects the shape and the number of the linear regions defined by the following layers, which may vary between adjacent linear regions due to which units are active or inactive from one linear region to another, as illustrated in Example~\ref{ex:linear_regions}. 

\begin{example}\label{ex:linear_regions}
Consider a neural network with domain $\vx \in \mathbb{R}^2$ and 2 layers having 2 neurons each ---say neurons $\alpha$ and $\beta$ in layer 1, and neurons $\gamma$ and $\delta$ in layer 2--- with outputs given as follows: 
$h^1_{\alpha} = \max\{ 2.3 x_1 - 1.9 x_2 +1.5, 0\}$, $h^1_{\beta} = \max\{ -0.9 x_1 - 0.7 x_2 +5, 0\}$, 
$h^2_{\gamma} = \max\{ 0.4 h^1_1 - 3.1 h^1_2 +4, 0\}$, $h^2_{\delta} = \max\{ -0.6 h^1_1 - 1.6 h^1_2 +5, 0\}$.
These neurons define the activation hyperplanes ($\alpha$) $2.3 x_1 - 1.9 x_2 +1.5 = 0$ and ($\beta$) $-0.9 x_1 - 0.7 x_2 +5 = 0$ in the $\vx$ space 
and the activation hyperplanes ($\gamma$) $0.4 h^1_1 - 3.1 h^1_2 +4 = 0$ and ($\delta$) $-0.6 h^1_1 - 1.6 h^1_2 +5 = 0$ in the $\vh^1$ space, 
which are illustrated along with the activation sets of the linear regions in the first two plots of Figure~\ref{fig:linear_regions}. 
The third plot illustrates the linear regions jointly defined by the two layers in terms of the input space $\vx$.  

The third plot is repeated in Figure~\ref{fig:dimensions}, in which the shape of each linear region $\sI$ is filled in accordance to the dimension of the image of $\bar{\vy}_{\sI}(\vx)$ ---the output of the neural network for each linear region $\sI$.
\end{example}

\begin{figure}
    \centering
    \includegraphics[width=0.32\textwidth]{Fig_03-A.pdf}
    \includegraphics[width=0.32\textwidth]{Fig_03-B.pdf} 
    \includegraphics[width=0.32\textwidth]{Fig_03-C.pdf}
    \caption{Linear regions defined by the 2 layers of the neural network described in Example~\ref{ex:linear_regions}, 
    following the same notation as in Figure~\ref{fig:hyperplane_arrangement}. The first and second plots show the linear regions and corresponding activation sets defined by the first and the second layers in terms of their input spaces ($\vx$ and $\vh^1$). The third plot shows the linear regions defined by the combination of the 2 layers and the union of their activation sets in terms of the input space of the first layer ($\vx$).}
    \label{fig:linear_regions}
\end{figure}

Example~\ref{ex:linear_regions} highlights two important aspects about the structure of linear regions in deep neural networks. 
First, the linear regions defined by a neural network with multiple layers are different because activation hyperplanes after the first layer may look ``bent'' from the input space $x$, 
such as with the inflections of hyperplanes $(\gamma)$ and $(\delta)$ in the third plot of Figure~\ref{fig:linear_regions} from one linear region defined by the first layer to another. 
This partitioning of the input space would not be possible with a single layer. 

By comparing side by side the first and the third plots of Figure~\ref{fig:linear_regions}, 
we can see  how every linear region of a given layer of a neural network may be partitioned differently by the following layer. 
When defined in terms of the input space $\vx$, 
the hyperplanes associated with the second layer differ across the linear regions defined by the first layer 
because each of those linear regions is associated with a different affine transformation from $\vx$ to $\vh^1$. 
Hence, the activation hyperplanes of layer $l$ may break each linear region from layer $l-1$ differently. 
To every linear region defined by the hyperplane arrangement in the $\vh^{l-1}$ space there is a linear transformation $\vh^{l-1} = \Omega^{\sS^{l-1}}(\mW^{l-1} \vh^{l-2} + \vb^{l-1})$ to the points of that linear region based on the set of active neurons $\sS^{l-1}$. 
%
Consequently, inputs in the $\vh^{l-1}$ space that are associated with different linear regions are transformed differently to the $\vh^l$ space, and therefore the form in which those linear regions are further partitioned by layer $l$ is not the same when seen from the $\vh^{l-1}$ space. 

Second, some combinations of activation sets of multiple layers do not correspond to linear regions even if the activation hyperplanes are in general position with respect to each layer. 
For each layer, the first two plots of Figure~\ref{fig:linear_regions} show that every activation set corresponds to a nonempty region of the layer input. 
However, not every pair of such activation sets would define a nonempty linear region for the neural network. 
For example, the linear region of the first layer associated with the activation set $\sS^1 = \{\}$ defines a linear region in $\vx$ which is always mapped to $\vh^1=0$, 
and thus only corresponds to activation set $\sS^2 = \{ \gamma, \delta \}$ in the second layer because both units are active for such input. 
Thus, no linear region in $\vx$ is associated with only the units in sets $\{\}, \{\gamma\}$, and $\{ \delta \}$ being active ---i.e., there is no linear region such that $\sS^1 \cup \sS^2 = \{\}, \{\gamma\}, \text{or} \{\delta\}$.

% Impact on dimensions
% Take the rank, explain loss of dimension in the image of the function

More generally, the number of units that is active on each linear region defined by the first layer also imposes a geometric limit to how that linear region can be further partitioned. 
If only one unit is active at a layer, that means that the output of the layer within that linear region has dimension 1, 
and, consequently, the subsequent hyperplane arrangements within that linear region are limited to a 1-dimensional space.
For the network in the example, we thus expect no more than $\sum_{i=0}^1 \binom{2}{i} = 3$ linear regions being defined instead of $2^2 = 4$ when only one unit is active. 
In fact, 
that is precisely the number of subdivisions by the second layer of the linear region defined by activation set $\sS^1 = \{\beta\}$ from the first layer. 
%hence implying that we should not expect all possible activation sets in such case. 
%
%We may even observe a smaller number, 
%as in the case of the linear region defined by activation set $\{\alpha\}$ in the first layer only being subdivided in two regions by the second layer. 

%The loss of dimension in the output of the linear regions is actually very common, 
%since the minimum number of active units across all previous layer is an upper bound on the dimension of the output of each linear region. 
%As shown in Figure~\ref{fig:dimensions}, that may be caused by either layer. 
%There are two other linear regions with 0-dimensional output due to the absence of activations in the second layer ($\{\beta\}, \{\alpha, \beta\}$) 
%and another 6 linear regions with 1-dimensional output due to the first ($\{\alpha,\gamma,\delta\}, \{\beta,\gamma,\delta\}$), the second ($\{\alpha,\beta,\gamma\},\{\alpha,\beta,\delta\}$), or both layers ($\{\alpha,\gamma\}, \{\beta,\delta\}$).

\begin{figure}
    \centering
    \includegraphics[width=0.66\textwidth]{Fig_05.pdf}
    \caption{Dimension of the image of the affine function $\vy_{\sI}(\vx)$ associated with each linear region $\sI$ defined by the neural network described in Example~\ref{ex:linear_regions}. The linear regions are the same illustrated in the third plot of Figure~\ref{fig:linear_regions}.}
    \label{fig:dimensions}
\end{figure}


%One of the reasons for such a difference in how each linear region is further partitioned by the following layers is the loss of dimensionality in the output. 
For every linear region defined by layer $l$ with an activation set $\sS^l$, the dimension of the output of the corresponding transformation $\Omega_{\sS^l}(\mW^l \vh^{l-1} + \vb^l)$ is at most $|\sS^l|$ since $\text{rank}(\Omega_{\sS^l}) = |\sS^l|$. Hence, the dimension of the output of every linear region defined by a rectifier network is upper bounded by its smallest activation set across all layers. This phenomenon was first identified by~\cite{serra2018bounding} as the \emph{bottleneck effect}. In neural networks with uniform width, this phenomenon leads to a surprising consequence: the number of linear regions with full-dimensional output is at most one. There are also consequences to the maximum number of linear regions that can be defined, as we discuss later.
% 


\subsubsection{The geometry of decision regions}

It is also common to study what inputs are associated with each class by a neural network.  
The set of inputs associated with the same class define a \emph{decision region}. 
Difficulties in modeling functions such as the Boolean XOR in shallow networks are related to limitations on the form of the decision regions, 
which may be limited by the depth of the neural network. 
For example, \cite{makhoul1989twolayer} showed that two layers suffice to obtain disconnected decision regions. 

The softmax layer is typically used for the output of neural networks trained on classification problems,  
in which the largest output corresponds to the class to which the input is associated. 
%
In rectifier networks coupled with a softmax layer, 
the decision regions can also be defined by polyhedra. 
Although the output of the softmax layer is not piecewise linear, 
its largest output corresponds to its largest input. 
Hence, every linear region $\sI$ defined by layers 1 to $L-1$ is partitioned by the softmax layer into decision regions 
where $\vh^{L-1}_i \geq \vh^{L-1}_j ~ \forall j \neq i$ for each class $i$ associated with the input $\vh_i^{L-1}$ to the softmax layer. 
Therefore, 
each decision region of a rectifier networks consist of a union of polyhedra. 

In fact, we may say further in the typical setting where no hidden layer is wider than the input ---i.e., $n_0 \geq n_l ~ \forall l \in \sL$: 
\cite{pmlr-v80-nguyen18b} showed that at least one layer $l \in \sL$ must be such that $n_l > n_0$ for the network to present disconnected decision regions; and 
\cite{grigsby2022topology} showed that, for an input size $n_0 \geq 2$, the decision regions are either empty or unbounded. 


\subsection{The number of linear regions}\label{sec:number}


We have seen conditions that affect the number of linear regions both positively and negatively. 
We discuss these and other analytical results in Section~\ref{sec:number_analytical}, and then discuss work on counting linear regions in practice in Section~\ref{sec:number_counting}.

\subsubsection{Analytical results}\label{sec:number_analytical}

At least three lines of work on analytical results have brought important insights. 
The first line is based on constructing networks with a large number of linear regions, 
which leads to lower bounds on the maximum number of linear regions. 
The second line is based on showing how the network architecture ---in particular its hyperparameters--- may impact the hyperplane arrangements defined by the layers, 
which leads to upper bounds on the maximum number of linear regions. 
The third line is based on characterizing the parameters of neural networks according to how they are initialized and updated along training, 
which leads to results on the expected number of linear regions for such networks. 

\paragraph{Lower bounds}

The lower bounds on the maximum number of linear regions are obtained through a careful choice of network parameters aimed at increasing the number of linear regions. 
In some cases, they also depend on particular choices of hyperparameters. 
We present them by order of refinement in Table~\ref{tab:lower_bounds}.

The first lower bound was introduced by~\cite{pascanu2013on} and then improved by those authors with a new construction technique in~\cite{montufar2014on}. 
In fact, Example~\ref{ex:zigzag} shows the case in which $n_0=1$ for the technique in~\cite{montufar2014on}. 
While a different construction is proposed by~\cite{Telgarsky2015}, 
subsequent developments in the literature have been based on~\cite{montufar2014on}. 

The lower bound by~\cite{arora2018understanding} is based on a different technique to construct a first wide layer based on zonotopes, 
which is then followed by the same layers as in~\cite{montufar2014on}.  
The first lower bound by~\cite{serra2018bounding} reflects a slight change to the technique used by~\cite{montufar2014on}, 
which in terms of Example~\ref{ex:zigzag} corresponds to using $n$ neurons to define $n+1$ instead of $n$ slopes on $[0,1]$. 
The second lower bound by~\cite{serra2018bounding} extends that of~\cite{arora2018understanding} by changing in the same way the construction of the subsequent layers of the network.

\begin{landscape}
\begin{table}
\caption{Lower bounds on the maximum number of linear regions defined by a neural network.}
\label{tab:lower_bounds}
\centering
\vspace{2ex}
\begin{tabular}{@{\extracolsep{4pt}}cc}
\textbf{Reference} & \textbf{Bound and conditions}  \\
\cline{1-1}
\cline{2-2}
\noalign{\vskip2.5pt}
~\cite{pascanu2013on} & $\left(\prod\limits_{l=1}^{L-1} \left\lfloor \frac{n_l}{n_0} \right\rfloor \right) \sum\limits_{i=0}^{n_0} \binom{n_L}{i}$ \\
\noalign{\vskip4pt}
~\cite{montufar2014on} & $\left(\prod\limits_{l=1}^{L-1} \left\lfloor \frac{n_l}{n_0} \right\rfloor^{n_0}\right) \sum\limits_{i=0}^{n_0} \binom{n_L}{i}$, where $n_l \geq n_0 ~\forall l \in \sL$ \\
\noalign{\vskip4pt}
~\cite{Telgarsky2015} & $2^{\frac{L-3}{2}}$, where $n_i = 1$ for $i$ odd, $n_i = 2$ for $i$ even, and $L-3$ divides by 2 \\
\noalign{\vskip4pt}
~\cite{arora2018understanding} &  $2 \sum\limits_{j=0}^{n_0-1} \binom{m-1}{j} w^{L-1}$, where $2m=n_1$ and $w=n_l ~\forall l \in \sL \setminus \{1\}$ \\
\noalign{\vskip4pt}
~\cite{serra2018bounding} & $\left(\prod\limits_{l=1}^{L-1} \left( \left\lfloor \frac{n_l}{n_0} \right\rfloor + 1 \right)^{n_0} \right) \sum\limits_{i=0}^{n_0} \binom{n_L}{i}$, where $n_l \geq 3 n_0 ~\forall l \in \sL$ \\
\noalign{\vskip4pt}
~\cite{serra2018bounding} & $2 \sum\limits_{j=0}^{n_0-1} \binom{m-1}{j} (w+1)^{L-1}$, where $2m=n_1$ and $w=n_l \geq 3 n_0 ~\forall l \in \sL \setminus \{1\}$
\end{tabular}
\end{table}
\end{landscape}


\paragraph{Upper bounds}

The upper bounds on the maximum number of linear regions are obtained by primarily considering changes to the geometry of the linear regions from one layer to another, as previously outlined and revisited below. 
We present those with a close form by order of refinement in Table~\ref{tab:upper_bounds}. 

\cite{pascanu2013on} established the connection between linear regions and hyperplane arrangements, 
which lead to the tight bound for shallow networks based on~\cite{Zaslavsky1975} for activation hyperplanes in general position. 
\cite{montufar2014on} defined the first bound for deep networks based on enumerating all activation sets. 
The subsequent upper bounds extended the result by~\cite{pascanu2013on} to deep networks by considering its successive application through the sequence of layers of the network. 

In the case of \emph{deep} networks, where $L > 1$, 
we need to consider how the linear regions defined up to a given layer of the network can be further partitioned by the next layers. 
%
We %can 
start by assuming that every linear region defined by the first $l-1$ layers 
is then subdivided into the maximum possible number of linear regions defined by the activation hyperplanes of layer $l$. 
That leads to the bound %of $\prod_{l=1}^{L+1} \sum_{j=0}^{n_{l-1}} \binom{n_l}{j}$ by 
in~\cite{raghu2017expressive}, 
which is implicit in their proof of an asymptotic bound of $O(n^{n_0 L})$, where $n$ is used as the width of every layer. % of the network.  
However, there are many ways in which this bound can be refined upon careful examination. 
First, the dimension of the input of layer $l$ ---i.e., the output of layer $l-1$--- within each linear region is never larger than the smallest dimension among layers $1$ to $l$, since for every linear region we have an affine transformation between inputs and outputs of each layer \citep{montufar2017notes}.  
%That leads to the bound by~~\cite{montufar2017notes}. % of  $\prod_{l=1}^{L+1} \sum_{j=0}^{d_l} \binom{n_l}{j}$ linear regions, where $d_l = \min\{n_0, n_1, \ldots, n_l\}$. 
Second, the dimension of the input coming through each linear region is in fact bounded by the smallest number of active units in each of the previous layers \citep{serra2018bounding}.  
%That leads to the bound by~~\cite{serra2018bounding}. % of  $\sum_{(j_1,\ldots,j_L) \in J} \prod_{l=1}^L \binom{n_l}{j_l}$ linear regions, where $J = \{(j_1, \ldots, j_L) \in \mathbb{Z}^L: 0 \leq j_l \leq \min\{n_0, n_1 - j_1, \ldots, n_{l-1} - j_{l-1}, n_l\}\ \forall l \in \sL \}$. 
This leads to a tight upper bound for $n_0=1$, since it matches the lower bound in ~\cite{serra2018bounding}. 
Finally, 
the activation hyperplane of some units may not partition the linear regions because all possible inputs to the unit are in the same half-space, and in some of those cases the unit may never produce a positive output. 
For the number $k$ of active units in a given layer $l$, we can use the network parameters to calculate the maximum number of units that can be active in the next layer, $\mathcal{A}_l(k)$, as well as the number of units that can be active or inactive for different inputs, $\mathcal{I}_{l}(k)$ \citep{serra2020empirical}. 

\cite{hinz2019framework} observed that the upper bound by \cite{serra2018bounding} can be tightened by explicitly computing a recursive histogram of linear regions on the layers of the neural network according to the dimension of their image subspace. However, the resulting bound is not explicitly defined in terms of the network hyperparameters, and hence cannot be included on the table. This work is further extended in~\cite{hinz2021histograms} by also allowing a composition of bounds on subnetworks instead of only on the sequence of layers. 
Another extension of the framework from~\cite{hinz2019framework} by \cite{yutong2020framework} highlights that residual connections prevent the bottleneck effect in ResNets, 
by which reason such networks tend to have more linear regions. 

\cite{cai2023pruning} proposed a separate recursive bound based on \cite{serra2018bounding} to account for the sparsity of the weight matrices, 
which illustrates how pruning connections may affect the maximum number of linear regions.

The results above have also been extended to other architectures. 
In some cases, results on other types of activations are also part of the papers previously mentioned: 
\cite{montufar2014on} and \cite{serra2018bounding} present upper bounds for \emph{maxout} networks; 
\cite{raghu2017expressive} present an upper bound for networks using \emph{hard tanh} activation. 
In other cases, the ideas discussed above have been adapted for sparser networks with parameter sharing: 
\cite{xiong2020cnn} present upper and lower bounds for convolutional networks, 
which are shown to asymptotically define more linear regions per parameter than rectifier networks with the same input size and number of layers. 
\cite{chen2022gcn} present upper and lower bounds for graph convolutional networks. 
\cite{matoba2022maxpooling} discuss the expresiveness of the maxpooling layers typically used in convolutional neural networks through their equivalence to a sequence of rectifier layers. 
Moreover, \cite{goujon2022role} present results for recently proposed activation functions, 
such as DeepSpline~\citep{agostineli2015spline,unser2019representer,bohra2020learning} and GroupSort~\citep{anil2019groupsort}. 

Some of the results above were also revisited through the lenses of tropical algebra, 
in which every linear region corresponds to a tropical hypersurface \citep{zhang2018tropical,charisopoulos2018tropical,maragos2021tropical}. 
Notably, \cite{montufar2022maxout} presented considerably tighter upper bounds for the number of linear regions in maxout networks with rank $k=3$ or greater. 

% DEPTH GOES HERE
Recently, a converse line of work started exploring the minimum dimensions of a neural network capable of representing a given piecewise linear function, 
starting with considerations about the minimum depth necessary \citep{arora2018understanding} and further refinements of bounds on the network dimensions \citep{he2020finite,hertrich2021depth,chen2022bounds}, 
with \cite{chen2022bounds} proposing an algorithm that can construct such a neural network. 
On a related note, 
\cite{MPC} show that linear time-invariant systems in model predictive control can be exactly expressed by rectifier networks and provide bounds on the width and number of layers necessary for a given system, whereas \cite{ferlez2020aren} describe an algorithm for producing architectures that can be parameterized as an optimal model predictive control strategy.

\begin{landscape}
\begin{table}
\caption{Upper bounds on the maximum number of linear regions defined by a neural network.}
\label{tab:upper_bounds}
\centering
\vspace{2ex}
\begin{tabular}{@{\extracolsep{4pt}}cc}
\textbf{Reference} & \textbf{Bound and conditions}  \\
\cline{1-1}
\cline{2-2}
\noalign{\vskip2.5pt}
~\cite{pascanu2013on} & $\sum\limits_{i=0}^{n_0} \binom{n_1}{n_0}$ for shallow networks, $n_1 \geq n_0$ \\
\noalign{\vskip4pt}
~\cite{montufar2014on} & $2^{\sum\limits_{l=1}^{L} n_l}$ \\
\noalign{\vskip4pt}
~\cite{raghu2017expressive} & $\prod\limits_{l=1}^{L} \sum\limits_{j=0}^{n_{l-1}} \binom{n_l}{j}$ \\
\noalign{\vskip4pt}
~\cite{montufar2017notes} & $\prod\limits_{l=1}^{L} \sum\limits_{j=0}^{d_l} \binom{n_l}{j}$, $d_l = \min\{n_0, n_1, \ldots, n_{l-1}\}$ \\
\noalign{\vskip4pt}
~\cite{serra2018bounding} & $\begin{array}{r}\sum\limits_{(j_1,\ldots,j_L) \in J} \prod\limits_{l=1}^L \binom{n_l}{j_l}, J = \{(j_1, \ldots, j_L) \in \mathbb{Z}^L: 0 \leq j_l \leq d_l ~\forall l \in \sL \},\\ d_l = \min\{n_0, n_1 - j_1, \ldots, n_{l-1} - j_{l-1}, n_l\}\ \end{array}$ \\
\noalign{\vskip4pt}
~\cite{serra2020empirical}  & $\begin{array}{r}\sum\limits_{(j_1,\ldots,j_L) \in J} \prod\limits_{l=1}^L \binom{\mathcal{I}_l(k_{l-1})}{j_l}, J = \{(j_1, \ldots, j_L) \in \mathbb{Z}^L: 0 \leq j_l \leq d_l, \\ d_l = \min\{n_0, k_1, \ldots, k_{l-1}, \mathcal{I}_l(k_{l-1})\}, k_0 = n_0, k_l = \mathcal{A}_{l}(k_{l-1}) - j_{l-1} ~\forall l \in \sL \}\end{array}$
\end{tabular}
\end{table}
\end{landscape}

\paragraph{Expected number}

The third analytical approach has been the evaluation of the expected number of linear regions. 
In a pair of papers, Hanin and Rolnick studied the number of linear regions based on how the network parameters are typically initialized.  
In the first paper \citep{hanin2019complexity}, 
they show that the average number of linear regions along 1-dimensional subspaces of the input grows linearly with respect to the number of neurons, irrespective of the network depth. 
In the second paper \citep{hanin2019deep}, 
they show that the average number of linear regions in higher-dimensional subspaces of the input also grows similarly in deep and shallow networks. 
For $N = \sum_{i=1}^L n_i$ as the total number of linear regions, 
the expected number of linear regions is  $O(2^N)$ if $N \leq n_0$ and $O\left(\frac{(T N)^{n_0}}{n_0!}\right)$ otherwise, 
where $T > 0$ is a constant based on the network parameters. 
Moreover, some of their experiments suggest that the number of linear regions in shallow networks is slightly greater. 
According to the authors, 
these bounds reflect the fact that the family of functions that can be represented by neural networks in the way that they are typically initialized is considerably smaller. 
They further argue that training as currently performed is unlikely to expand the family of functions much further, as illustrated by their experiments. 
% 
Similar results on the expected number of linear regions for maxout networks are presented by~\cite{tseran2021expected}, 
and an application of the results above results to data manifolds is explored by~\cite{tiwari2022manifolds}. 
Additional results for specific architectures of rectifier networks are conjectured by~\cite{wang2022estimation}, although without proof.


%Under reasonable conditions on how the network parameters are initialized, 
%~\cite{hanin2019complexity} have shown that the expected number of linear regions at initialization is linear on the number of units of the network and that it does not depend on its depth. 
%Those authors argue that this is not in conflict with the exponential lower bounds proven in the literature, 
%but instead that such constructions with a large number of linear regions can collapse to relatively smaller numbers through small adjustments to the parameters. 


\subsubsection{Counting linear regions}\label{sec:number_counting}

Counting the actual number of linear regions of a given network has been a more challenging topic to explore. 
\cite{serra2018bounding} have shown that the linear regions of a trained network can be enumerated as the solutions of an MILP formulation, 
which has been slightly corrected in~\cite{cai2023pruning}\footnote{The MILP formulation of neural networks is discussed in Section~\ref{sec:optimizing}.}. 
However, MILP solutions are generally counted one by one \citep{danna2007multiple}, with exception of special cases \citep{serra2020nearoptimal} and small subproblems \citep{serra2020enumerative}, which makes this approach impractical for large neural networks. 
\cite{serra2020empirical} have shown that approximate model counting methods, which are commonly used to count the number of feasible assignments in propositional satisfiability, can be easily adapted to solution counting in MILP, which leads to an order-of-magnitude speedup in comparison with exact counting. 
This type of approach is particularly suitable for obtaining probabilistic lower bounds, which can complement the analytical upper bounds for the maximum number of linear regions.  
In \cite{craighero2020compositional} and \cite{craighero2020understanding}, a directed acyclic graph is used to model the sets of active neurons on each layer and how they connected with those in subsequent layers. 
\cite{yang2020reachability} describe a method for decomposing the input space of rectifier networks into their linear regions by representing each linear region in terms of its face lattice, upon which the splitting operations corresponding to the transformations performed by each layer can be implemented. As the number of linear regions grow, these splitting operations can be processed in parallel. \cite{yang2021reachability} extend that method to convolutional neural networks. 
Moreover, \cite{wang2022estimation} describes an algorithm for enumerating linear regions that counts adjacent linear regions with same corresponding affine function as a single linear region. 

Another approach is to enumerate the linear regions in subspaces, which limits their number and reduces the complexity of the task. 
This idea was first explored by \cite{novak2018sensitivity} for measuring the complexity of a neural network in terms of the number of transitions along a single line. 
\cite{hanin2019complexity,hanin2019deep} also use this method with a bounded line segment or rectangle as a single set representing the input and then sequentially partitioning it. 
If this first set is intersected by the activation hyperplane of a neuron in the first layer, 
then we replace this set by two sets corresponding to the parts of the input space in which that neuron is active and not. 
Once those sets are further subdivided by all activation hyperplanes associated with the neurons in the first layer, 
the process can be continued with the neurons in the following layers. 
This method is used to count the number of linear regions along subspaces of the input with dimension 1 in \cite{hanin2019complexity} and dimension 2 in \cite{hanin2019deep}. 
A generalized version for counting the number of linear regions in affine subspaces spanned by a set of samples using an MILP formulation is presented in \cite{cai2023pruning}. 
An approximate approach for counting the number of linear regions along a line by computing the closest activation hyperplane in each layer is presented in \cite{gamba2022equal}. 


Other approaches have obtained lower bounds on the number of linear regions of a trained network by limiting the enumeration or considering exclusively the inputs from the dataset. 
In \cite{xiong2020cnn}, the number of linear regions is estimated by sampling points from the input space and enumerating all activation patterns identified through this process. 
In \cite{cohan2022evolution}, the counting is restricted to the linear regions found between consecutive states of a neural network modeling a reinforcement learning policy. 



\subsection{Applications and insights}

Thinking about neural networks in terms of linear regions led to a variety of applications. 
In turn, that inspired further studies on the structure and properties of linear regions under different settings. 
We organize the literature about applications and insights around some central themes in the subsections below. 


\subsubsection{The number of linear regions}

From our discussion, the number of linear regions emerges as a potential proxy for the complexity of neural networks, 
which has been studied by some authors and exploited empirically by others. 
\cite{novak2018sensitivity} observed that the number of transitions between linear regions in 1-dimensional subspaces correlates with generalization. 
\cite{hu2020distillation} used bounds on the number of linear regions as proxy to model the capacity of a neural network used for learning through distillation, in which a smaller network is trained based on the outputs of another network. 
\cite{chen2021nas} and \cite{chen2021nas2} present one of the first approaches to training-free neural architectural search through the analysis of network properties. One of the two metrics that they have shown to be effective for that purpose is the number of linear regions associated with a sample of inputs from the training set on randomly initialized networks. 
\cite{biau2021wgans} observed that obtaining a discriminator network for Wasserstein GANs~\citep{arjovsky2017wgan} that correctly approximates the Wasserstein distance entails that such a discriminator network has a growing number of linear regions as the complexity of the data distribution increases. 
\cite{park2021unsupervised} maximized the number of linear regions in unsupervised learning in order to produce more expressive encodings for downstream tasks using simpler classifiers. 
In neural networks modeling reinforcement learning policies, 
\cite{cohan2022evolution} observed that the number of transitions between linear regions in inputs corresponding to consecutive states increases by 50\% with training while the number of repeated linear regions decreases. 
\cite{cai2023pruning} proposed a method for pruning different proportions of parameters from each layer by maximizes the bound on the number of linear regions, 
which lead to better accuracy than uniform pruning across layers. 
On a related note, 
\cite{liang2021brelu} proposed a new variant of the ReLU activation function for dividing the input space into a greater number of linear regions. 

The number of linear regions also inspired further theoretical work. 
\cite{amrami2021benefit} presented an argument for the benefit of depth in neural networks based on the number of linear regions for correctly separating samples associated with different classes. 
\cite{liu2021approximation} studied upper and lower bounds on the optimal approximation error of a convex univariate function based on the number of linear regions of a rectifier network. 
\cite{henriksen2022repairing} used the maximum number of linear regions as a metric for capacity that may limit repairing incorrect classifications in a neural network. 


\subsubsection{The shape of linear regions}

Some studies aimed at understanding what affects the shape of linear regions in practice, including how to train neural networks in such a way to induce certain shapes in the linear regions. 
%
\cite{empirical2020iclr} observed that multiple training techniques may lead to similar accuracy, but very different shape for the linear regions. 
For example, batch normalization~\citep{ioffe2015batchnorm} and dropout~\citep{srivastava2014dropout} lead to more linear regions. 
While batch normalization breaks the space in regions with uniform size, more orthogonal norms, and more gradient variability across adjacent regions; dropout produces more linear regions around decision boundaries, norms are more parallel, and data points less likely to be in the region containing the decision boundary. 
\cite{croce2019max} and \cite{LocallyLinear} applied regularization to the loss function to push the boundary of each linear region away from points in the training set that it contains, as long as those points are correctly classified. They show that this form of regularization improves the robustness of the neural network while making the linear regions larger.
In fact, \cite{zhu2020local} observed that the boundaries of the linear regions move away from the training data; 
and \cite{HashEncoders} conjectured that the linear regions near training samples becomes smaller through training, or that conversely the activation patterns are denser around the training samples. 
\cite{gamba2020biased} presented an empirical study on the angles between activation hyperplanes defined by convolutional layers, and observed that their cosines tend to be similar and more negative with depth after training.

The geometry of linear regions also led to other theoretical and algorithmic advances. 
Theoretically, \cite{phuong2020equivalence} proved that architectures with nonincreasing layer widths have unique parameters ---upon permutation and scaling--- for representing certain functions. 
In other words, some pairs of neural networks are only equivalent if their parameters only differ by permutation and multiplication. 
\cite{grigsby2023symmetries} showed that equivalences other than by permutation are less likely to occur with greater input size and width, but more likely with greater depth.
Algorithmically, 
\cite{ReverseEngineering} proposed a procedure to reconstruct a neural network by evaluating several inputs in order to determine regions of the input space 
for which the output of the neural network can be defined by an affine function ---and thus consist of a single linear region. 
Depending on how the shape changes between adjacent linear regions, 
the boundaries of the linear regions are replicated with neurons in the first hidden layer or in subsequent layers of the reconstructed neural network. 
\cite{masden2022combinatorial} provided theoretical results and an algorithm for characterizing the face lattice of the polyhedron associated with each linear region. 


\subsubsection{Activation patterns and the discrimination of inputs}

Another common theme is understanding how inputs from the training and test sets are distributed among the linear regions, and what can be inferred the encoding of the activation patterns associated with the linear regions.
%
\cite{gopinath2019property} noted that many properties of neural networks, including the classes of different inputs, are associated with activation patterns ---and thus with their linear regions. 
Several works~\citep{HashEncoders,sattelberg2020locally,tropex2021iclr} observed that each training sample is typically located in a different linear region when the neural network is sufficiently expressive; whereas
\cite{HashEncoders} noted that simple machine learning algorithms can be applied using the activation patterns as features, and 
\cite{sattelberg2020locally} noted that there is some similarity between activation patterns of different neural networks under affine mapping, meaning that the training of these neural networks lead to similar models.
\cite{chaudhry2020continual} exploited the idea of continual learning with different tasks being encoded in disjoint subspaces, 
which thus corresponds to a disjoint set of activation sets on each layer being associated with classifications for each of those tasks. 
Based on their approach for enumerating linear regions, 
\cite{craighero2020compositional} and \cite{craighero2020understanding} have found that inputs from larger linear regions are often correctly classified by the neural network, that inputs from smaller linear regions are often incorrectly classified, and that the number of distinct activations sets reduces along the layers of the neural network. 
\cite{gamba2022equal} also discussed the issue of some linear regions being smaller and thus less likely to occur in practice. 
Moreover, they propose a measurement for the similarity of the affine functions associated with linear regions along a line 
and observed that the linear regions tend to be less similar to one another when the network is trained with incorrectly classified labels. 

 
\subsubsection{Function approximation}

Because of the linear behavior of the output within each linear region, 
we can approximate the output of the neural network based on the output of its linear regions. 
%
\cite{chu2018pwnn} and \cite{sudjianto2020unwrapping} produced linear models based on this local behavior; whereas 
\cite{glass2021relumot} observed that we can interpret neural networks as equivalent to local linear model trees~\citep{nelles2000lolimot}, in which a distinct linear model is used at each leaf of a decision tree, and provided a method to produce such models from neural networks.
\cite{tropex2021iclr} described how to extract the linear regions associated with the inputs from the training set as means to approximate the output of the inputs from the test set. 
\cite{robinson2019dissecting} presented another approach for explicitly representing the function modeled by a neural network through the enumeration of its linear regions. 
On a related note, 
\cite{chaudhry2020continual} used the assumption of training samples remaining within the same linear region during gradient descent to simplify the analysis of backpropagation.

This topic also relates to the broad literature on neural networks as universal function approximators, to which the concept of linear regions helps articulating ideal conditions. 
As observed by \cite{mhaskar2020approximation}, the optimal number of linear regions in a neural network ---or, correspondingly, of pieces of the piecewise linear function modeled by it--- depends on the function being approximated.
In addition, linear regions were also used explicitly to build function approximations. 
\cite{kumar2019equivalent} have shown that rectifier networks can we approximated to arbitrary precision with two hidden layers, the largest of which having a neuron corresponding to each different activation pattern of the original network; an exact counterpart of this result was later presented by~\cite{villani2023shallow}.
\cite{fan2020quasiequivalence} described the transformation between sufficiently wide and deep networks while arguing that the fundamental measure of complexity should be counting simplices within linear regions. 
In subsequent work, \cite{fan2023simple} empirically observed that linear regions tend to have a small number of higher dimensional faces, or facets.

More recent studies aimed at understanding the expressiveness and approximability of neural networks in terms of their number of parameters, 
in particular when the number of linear regions is greater than the number of parameters \citep{fractals2019neurips,fractals2020ieee,daubechies2022approximation}. 
They all discuss how the composition the modeled functions tend to present the self-similarity property of fractal distributions, 
which is one reason why they have so many linear regions. 
\cite{keup2022origami} interpreted the connection between linear regions in different parts of the input space in terms of how paper origamis are constructed: by ``folding'' the data for separability. 

Another related topic is computing the Lipschitz constant $\rho$ of the function $f(x)$ modeled by the neural network, 
the smallest $\rho$ for which $\| f(x') - f(x) \| \leq \rho \| x' - x \|$ for any two inputs $x$ and $x'$. 
Note that the first derivative of the output of a linear region is constant, 
which is leveraged by~\cite{hwang2020unrectifying} to evaluate the stability of the network by computing the constant across linear regions by changing the activation pattern. 
Interestingly, \cite{zhou2019lipschitz} showed that the constant grows similarly to the number of linear regions: polynomial in width and exponential in depth. 
A smaller constant limits the susceptibility of the network to adversarial examples~\citep{huster2018limitations}, which are discussed in Section~\ref{sec:optimizing}, 
and also lead to smaller bias variance \citep{loukas2021training}. 
While calculating the exact Lipschitz constant is NP-hard and encourages approximations \citep{scaman2018lipschitz,combettes2019certificates}, 
the exact constant can be computed using MILP \citep{jordan2020exact}. 
Notably, 
many studies have focused on relaxations such as linear programming \citep{zou2019cnnlp}, semidefinite programming \citep{fazlyab2019sdp,chen2020sdp}, and polynomial optimization \citep{latorre2020polynomial}. 
An alternative approach is to use more sophisticated activation functions for limiting the value of the constant \citep{anil2019groupsort,aziznejad2020controledlipschitz}. 



\subsubsection{Optimizing over linear regions}

As an alternative to optimizing over neural networks as described next in Section~\ref{sec:optimizing}, 
a number of approaches have resorted to techniques that are equivalent to systematically enumerating or traversing linear regions and optimizing over them \citep{croce2018gcpr,croce2020ijcv,khedr2020verification,vincent2021icra,xu2022advml}.
Notably, \cite{vincent2021icra} and \cite{xu2022advml} are mindful of the facet-defining inequalities associated with a linear region, 
which are the ones to change when moving toward an adjacent linear region. 
On a related note, 
\cite{seck2021lp} alternates between gradient steps and solving a linear programming model within the current linear region.





\section{Optimizing Over a Trained Neural Network}\label{sec:optimizing}
In Section \ref{sec:training} we will see how polyhedral-based methods can be used to \emph{train} a neural network. In this section, we will focus on how polyhedral-based methods can be used to do something with a neural network \emph{after it has been trained.}
Specifically, after the network architecture and all parameters have been fixed, a neural network $f$ is merely a function. If each activation function $\sigma$ used to describe the network is piecewise linear (as is the case with those presented in Table~\ref{tab:activations}), $f$ is also a piecewise linear function. Therefore, any optimization problem containing $f$ in some way will be a piecewise linear optimization problem. For example, in the simple case where the output of $f$ is univariate, the optimization problem 
\[
    \min_{x \in \mathcal{X}} f(x)
\]
is a piecewise linear optimization problem. 
As discussed in Section \ref{sec:LR}, this problem can have an enormous number of ``pieces'' (linear regions) when $f$ is a neural network; solving this problem thus heavily depends on the size and structure of the neural network $f$. For example, the training procedure by which $f$ is obtained can greatly influence the performance of optimization strategies \citep{tjeng2017evaluating,xiao2018training}. 

In this section, we first explore situations in which you might want to optimize over a trained neural network in this manner. We will then survey available methods for solving this method (either exactly or on the dual side) using polyhedral-based methods. 
We conclude with a brief view of future directions. 

\subsection{Applications of optimization over trained networks}
Applications where you might want to optimize over a trained neural network $f$ broadly fall into two categories: those where $f$ is the ``true'' object of interest, and those where $f$ is a convenient proxy for some unknown, underlying behavior.

\subsubsection{Neural network verification} \label{sec:verification}
Neural network verification is a burgeoning field of study in deep learning. 
Starting in the early 2000s, researchers began to recognize the importance of rigorously verifying the behavior of neural networks, mainly in aviation-related applications \citep{schumann2003verification,zakrzewski2001verification}. 
More recently, the seminal works of \cite{szegedy2014intriguing} and \cite{goodfellow2015explaining} observed that neural networks are unusually susceptible to \emph{adversarial attacks}. These are small, targeted perturbations that can drastically affect the output of the network; as shown in Figure \ref{fig:adversarial}, even powerful models such as MobileNetV2 \citep{sandler2018mobilenetv2} are susceptible. The existence and prevalence of adversarial attacks in deep neural networks has raised justifiable concerns about the deployment of these models in mission-critical systems such as autonomous vehicles \citep{deng2020analysis}, aviation \citep{kouvaros2021formal}, or medical systems \citep{finlayson2019adversarial}. One fascinating empirical work by  \cite{eykholt2018robust} showed the susceptibility of standard image classification networks that might be used in self-driving vehicles to a very analogue form of attacks: black/white stickers, placed in a careful way, could confuse these models enough that they would mis-classify road signs (e.g., mistaking stop signs for ``speed limit 80'' signs).

\begin{figure}
    \centering
    \begin{tikzpicture}
        \node[] at (0,0){\includegraphics[width=2.5cm]{adversarial/before.png}};
        \node[] at (1.9,-0.30){\Large \bf{+}};
        \node[] at (3.8,-0.26){\includegraphics[width=2.5cm]{adversarial/noise.png}};
        \node[] at (5.7,-0.30){\Large \bf{=}};
        \node[] at (7.6,0){\includegraphics[width=2.5cm]{adversarial/after.png}};
        \node[] at (3.8,1.2){$\times (\epsilon = 0.15)$};
    \end{tikzpicture}
    \caption{Example of adversarial attack on MobileNetV2 \citep{sandler2018mobilenetv2}. The original image taken by one of the survey authors is classified as `siberian\_husky,' but is re-classified as `wallaby' with a small (in an $\ell_\infty$-norm sense) targeted attack.}
    \label{fig:adversarial}
\end{figure}

Neural network verification seeks to prove (or disprove) a given input-output relationship, i.e., $x \in \mathcal{X} \Rightarrow y \in \mathcal{Y}$, that gives some indication of model robustness. 
Methods for verifying this relationship are classified as being sound and/or complete. 
A method that is \textit{sound} will only certify the relationship if it is indeed true (no false positives), while a method that is \textit{complete} will (i) always return an answer and (ii) only disprove the relationship if it is false (no false negatives). 
An early set of papers \citep{FischettiMIP,LomuscioMIP,tjeng2017evaluating} recognized that MILP provides an avenue for verification that is both sound and complete, given that $\mathcal{X}$ and $f(x)$ are both linear, or piecewise linear. 
We refer the readers to recent reviews \citep{huang2020survey,leofante2018automated,li2022sok,liu2021algorithms} for a more comprehensive treatment of the landscape of verification methods, including MILP- and LP-based technologies.


\begin{example}
Consider a classification network $f : [0,1]^{n_0} \to \mathbb{R}^d$ where the $j$-th output, $f(x)_j$, corresponds to the probability that input $x$ is of class $j$.\footnote{In actuality, we will instead typically work with the outputs corresponding to ``logits'', or unnormalized probabilities. These are typically fed into a softmax layer that then normalize these values to correspond to a probability distribution over the classes. However, this nonlinear softmax transformation is not piecewise linear. Thankfully, it can be omitted in the context of the verification task without loss of generality.} Then consider a labeled image $\hat{x}$ known to be of class $i$, and a ``target'' adversarial class $k \neq i$. 
Then verifying local robustness of the prediction corresponds to checking $x \in \{ x: ||x-\hat{x}|| \leq \epsilon \} \Rightarrow y=f(x) \in \{ y: y_i \geq y_k \}$, where $\epsilon > 0$ is a constant which prescribes the radius around which $\hat{x}$ we will search for an adversarial example. 

This verification task can be formulated as an optimization problem of the form:
\begin{equation} \label{eq:verification}
\begin{aligned} 
    \max_{x \in [0,1]^{n_0}} \quad& f(x)_k - f(x)_i \\
    \text{s.t.}& ||x - \hat{x}|| \leq \epsilon.
\end{aligned}
\end{equation}Any feasible solution $x$ to this problem with positive cost is an adversarial example: it is very ``close'' to $\hat{x}$ which has true label $i$, yet the network believes it is more likely to be of class $k$.\footnote{Alternative objectives are sometimes used which would allow us to strengthen this statement to say that the network \emph{will} classify $x$ to be of class $k$. However, this will require a more complex reformulation to model this problem via MILP, so we omit it for simplicity.} If, on the other hand, it is proven that the optimal objective value is negative, this proves that $f$ is robust (at least in the neighborhood around $\tilde{x}$). 
Note that the verification problem can terminate once the sign of the optimal objective value is determined, but solving the problem returns an optimal adversarial example. 
\end{example}

The objective function of \eqref{eq:verification} models the desired input-output relationship, $x \in \mathcal{X} \Rightarrow y \in \mathcal{Y}$, while the constraints model the domain $\mathcal{X}$. 
The domain $\mathcal{X}$ is typically a box or hyperrectangular domain. Extensions to this are described in Section \ref{sec:domains}.
Some explanation-focused verification applications define the input-output relationship in a derivative sense, e.g., $x \in \mathcal{X} \Rightarrow \partial y / \partial x \in \mathcal{Y}'$ \citep{wicker2022robust}. 
As the derivative of the ReLU function is also piecewise linear, this class of problems can also be modeled in MILP. For example, in the context of fairness and explainability, \cite{liu2020monotonic} and \cite{jordan2020exact} used MILP to certify monotonicity and to compute local Lipschitz constants, respectively. 


Although in this survey we focus on optimization over trained neural networks, it is important to note that polyhedral theory underlies numerous strategies for neural network verification. 
For example, SAT and SMT (Satisfiability Modulo Theories) solvers designed for Boolean satisfiability problems (and more general problems for the case of SMT) can also be used to search through activation patterns for a neural network \citep{pulina2010abstraction}, resulting in tools that are sound and complete, such as Planet \citep{ehlers2017formal} and Reluplex \citep{katz2017reluplex}. 
\cite{bunel2018unified} presented a unified view to compare MILP and SMT formulations, as well as the relaxations that result from these formulations (we will revisit this in Section~\ref{sec:relaxations}).  
On the other hand, strategies such as ExactReach~\citep{xiang2017reachable} exploit polyhedral theory to compute reachable sets: given an input set to a ReLU function defined as a union of polytopes, the output reachable set is also a union of polytopes. 
Other methods over-approximate the reachable set to improve scalability, e.g., for vision models \citep{yang2021reachability}, often resulting in methods that are sound, but not complete. 


\subsubsection{Neural network as proxy}
Another situation in which you may want to solve an optimization problem containing trained neural networks is when you would like to optimize some other, unknown function for which you have historical input/output data. 
A similar situation arises when you want to solve an optimization problem where (some of) the constraints are overly complicated, but you can query samples from the constraints on which to train a simpler \textit{surrogate} model. 
In these cases, you might imagine training a neural network in a standard supervised learning setting to approximate this underlying, unknown or complicated function. Then, since the neural network is known, you are left with a deterministic piecewise linear optimization problem. 
Note that we focus here on using a neural network as a surrogate; neural networks can additionally learn other components of an optimization problem, e.g., uncertainty sets for robust optimization \citep{goerigk2023data}. 

Several software tools have been developed for this class of problems. For the case of constraint learning, \texttt{JANOS} \citep{bergman2022janos} and \texttt{OptiCL} \citep{maragno2021mixed} both provide functionality for learning a ReLU neural network to approximate a constraint based on data and embedding the learned neural network in MILP. 
The \texttt{reluMIP} package \citep{reluMIP} has also been introduced to handle the latter embedding step. 
More generally, \texttt{OMLT} \citep{ceccon2022omlt} translates neural networks to \texttt{pyomo} optimization blocks, including various MILP formulations and activation functions. 
Finally, recent developments in \texttt{gurobipy}\footnote{\url{https://github.com/Gurobi/gurobi-machinelearning}} enable directly parsing in trained neural networks.


Applications of this paradigm can be envisioned in a number of domain areas. This approach is common in deep reinforcement learning, where neural networks are used to approximate an unknown ``$Q$-function'' which models the long-term cost of taking a particular action in a particular state of the world. In $Q$-learning, this $Q$-function is optimized iteratively to produce new candidate policies, which are then evaluated (typically via simulation) to produce new training data for future iterations. 
Optimization over the learned $Q$-function must be relatively fast in control applications, and several practical methods have been proposed. 
When the action space is discrete, the $Q$-function neural network is trained with one output value for each possible action, simplifying optimization to evaluating the model and selecting the largest output. 
Continuous action spaces require the $Q$ network be optimized over \citep{burtea2023safe,delarue2020reinforcement,ryu2020caql}, or an ``actor'' network can be trained to learn the optimal actions \citep{lillicrap2015continuous}. 
In a related vein, ReLU neural networks can be used as a process model for optimal scheduling or control \citep{wu2020scalable}.

Chemical engineering also presents applications where surrogate models have proven beneficial for optimization, as  is the subject of recent reviews \citep{bhosekar2018advances,mcbride2019overview,tsay2019110th}.  
In particular, ReLU neural networks can be seamlessly embedded in larger MILP problems such as flow networks and reservoir control where the other constraints are also mixed-integer linear \citep{grimstad2019surrogate,Planning,yang2022modeling}. 
Focusing on control applications where the neural network is embedded in a MILP that must be solved repeatedly, \cite{katz2020integrating} showed how multiparametric programming can be used to learn the solution map of the resulting MILP itself, which is also piecewise affine. 
An emerging area of research uses verification tools to reason about neural networks used as controllers, e.g., see \citet{ARCH_COMP_20}. These applications involve optimization formulations combining the neural network with constraints defining the controlled system. 
For example, verification can be used to bound the reachable set \citep{sidrane2022overt} (alongside piecewise linear bounds on the dynamical system) or the maximum error against a baseline controller \citep{schwan2022stability}.


Finally, applications for optimization over neural networks arise in machine learning applications. 
MILP formulations can be used to compress neural networks \citep{serra2020lossless,serra2021compression,elaraby2020importance}, which consequently result in more tractable surrogate models \citep{kody2022modeling}. The main idea is to use MILP to identify \textit{stable} nodes, i.e., nodes that are always on or off over an input domain, which can then be algebraically eliminated. 
Optimization has also been employed in techniques for feature selection, based on identifying strongest input nodes \citep{sildir2022mixed,zhao2023model}. 
In the context of Bayesian optimization, \cite{volpp2020meta} use reinforcement learning to meta-learn acquisition functions parameterized as neural networks; selecting ensuing query points then requires optimization over the trained acquisition function. 
Later work modeled both the acquisition function and feasible region in black-box optimization as neural networks \citep{papalexopoulos22constrained}. 
In that work, exploration and exploitation are balanced via Thompson sampling and training multiple neural networks from a random parameter initialization.

%[Sequence design for DNA?]


\paragraph{A word of caution}
Standard supervised learning algorithms aim to learn a function which fits the underlying function according to some distribution under which the data is generated. However, optimizing a function corresponds to evaluating it at a single point. This means that you may end up with a model that well-approximates the underlying function in distribution, but for which the pointwise minimizer is a poor approximation of the true function. This phenomena is referred to as the ``Optimizer's curse'' \citep{smith2006}.

% \paragraph{Outline}

% We split this section into four main parts. We begin by analyzing the most common case in the simplest setting: A single ReLU neuron. We first explore exact representations for such a neuron using mixed-integer programming. Next, we study relaxed models for a single neuron that can be solved using linear programming (albeit sometimes in a disguised manner). Next, we study extensions to the single neuron model with different types of nonlinearities or domains. Finally, we study how single neuron models can be used to model entire neural networks, and briefy step beyond the single neuron model.

\subsubsection{Single neuron relaxations}

For the following subsections, consider the $i$-th neuron in the $l$-th layer of a neural network, endowed with a ReLU activation function, whose behavior is governed by \eqref{eq:single-neuron}. Presume that a input domain of interest $\mathcal{D}^{l-1} \subset \mathbb{R}^{n_l}$ is a bounded region. Further, since $\mathcal{D}^{l-1}$ is bounded, presume that finite bounds are known on each input component, i.e. that vectors $L^{l-1},U^{l-1} \in \mathbb{R}^{n_l}$ are known such that $\mathcal{D}^{l-1} \subseteq [L^{l-1},U^{l-1}] \subset \mathbb{R}^{n_l}$. We can then write the \emph{graph} of the neuron, which couples together the input and the output of the nonlinear ReLU activation function:
\begin{align*}
    \gr = &\Set{ (\vh^{l-1},h^l_i) \in \mathcal{D}^{l-1} \times \mathbb{R} | h^l_i = 0 \geq \vw^l_i \vh^{l-1} + b^l_i} \\
    \cup &\Set{ (\vh^{l-1},h^l_i) \in \mathcal{D}^{l-1} \times \mathbb{R} | h^l_i = \vw^l_i \vh^{l-1} + b^l_i \geq 0}.
\end{align*}
This is a disjunctive representation for $\gr$ in terms of two polyhedral alternatives.
We assume that every included neuron exhibits this disjunction, i.e., every neuron can be on or off depending on the model input. 
This assumption of \emph{strict activity} implies that $L^{l-1} < 0$ and $U^{l-1} > 0$, noting that neurons not satisfying this property can be exactly pruned from the model \citep{serra2020lossless}. 
%[TODO: Add assumption of strict activity, otherwise prune.]

We observe that, given this (or any) formulation for each individual unit, it is straightforward to construct a formulation for the entire network. For example, if we take $X^l_i = \Set{(\vh^{l-1},h^l_i,z^l_i) | \eqref{eqn:relu-big-m} }$ for each layer $l$ and each unit $i$, we can construct a MILP formulation for the graph of the entire network, $\Set{(x,f(x)) : x \in \mathcal{D}^0}$ as
\[
    (\vh^{l-1},h^l_i,z^l_i) \in X^l_i \quad \forall l \in \sL, i \in \llbracket n_l \rrbracket.
 \]  

This also generalizes in a straightforward manner to more complex feedforward network architectures (e.g. convolutions, or sparse or skip connections), though we omit the explicit description for notational simplicity.

\subsubsection{Beyond the scope of this survey}

The effectiveness of the single-neuron formulations described above is bounded by the tightness of the optimal univariate formulation; this property is known as the ``single-neuron barrier'' \citep{salman2019convex}. 
This has motivated research in convex relaxations that jointly account for multiple neurons within a layer \citep{singh2019beyond}. 
Nevertheless, the analysis of polyhedral formulations for multiple neurons simultaneously quickly becomes intractable, and is beyond the scope of this survey. Instead, we point the interested reader to the recent survey by \cite{roth2021primer}, and highlight a few approaches taken in the literature. Multi-neuron analysis has been used to: improve bounds tightening schemes \citep{rossig2021advances}, prune linearizable neurons \citep{botoeva2020efficient}, design dual decompositions \citep{ferrari2022complete}, and generate strengthening inequalities \citep{serra2020empirical}. %Additional work has investigated mixed-integer programming \cite{Partition-Based Formulations for Mixed-Integer Optimization of Trained ReLU Neural Networks, Between Steps: Intermediate Relaxations between big-M and Convex Hull Formulations} and linear programming \cite{k-ReLU: Beyond the Single Neuron Convex Barrier for Neural Network Certification, https://arxiv.org/pdf/2103.03638.pdf} formulations for multiple neurons in a single layer. 
Similarly, we do not review formulations for ensembles of ReLU networks, though MILP formulations have been proposed \citep{wang2021ensemble,wang2023optimizing}. 

Additionally, recent works have exploited polyhedral structure to develop sampling based strategies, which can be used to warm-start MILP or accelerate local search in verification \citep{perakis2022optimizing,wu2022efficient}. 
\cite{lombardi2017empirical} computationally compare MILP against local search and constraint programming approaches. 
In a related vein, \cite{cheon2022outer} examines local solutions and proposes an outer approximation method to improve gradient-based optimization. 
Finally, following \cite{raghunathan2018semidefinite}, a large body of work has presented optimization-based methods for verification that use semidefinite programming concepts \citep{dathathri2020enabling,fazlyab2020safety,newton2021exploiting}.
Notably, \cite{batten2021efficient} showed how combining semidefinite and MILP formulations can produce a new formulation that is tighter than both. This was later extended with reformulation-linearization technique, or RLT, cuts \citep{lan2022tight}. 
While related to linear programming and other methods based on convex relaxations, this stream of work is beyond the scope of this survey. 
We refer the reader to \cite{zhang2020tightness} for a discussion of the tightness of these formulations. 

\subsection{Exact models using mixed-integer programming}
\label{sec:MIPmodels}

Mixed-integer programming offers a powerful algorithmic framework for \emph{exactly} modeling nonconvex piecewise linear functions. The Operations Research community has studied has a long and storied history of developing MILP-based methods for piecewise linear optimization, with research spanning decades \citep{croxton2003comparison,dantzig1960significance,geissler2012using,huchette2022nonconvex,lee2001polyhedral,misener2012global,padberg2000approximating,vielma2010mixed}.
%\cite{A mixed integer approach for time-dependent gas network optimization,Mixed integer models for the stationary case of gas network optimization}. 
However, many of these techniques are specialized for low-dimensional or separable piecewise linear functions. While a reasonable assumption in many OR problems, this is not the case when modeling neurons in a neural network. Therefore, the standard approach in the literature is to apply general-purpose MILP formulation techniques to model neural networks.

%[TODO: Note that methods like Reluplex and Planet (cite!) and \cite{https://arxiv.org/pdf/2107.12855.pdf} essentially roll their own MIP solvers. 
\paragraph{Connection to Boolean satisfiability}
Some SMT-based methods such as Reluplex \citep{katz2017reluplex} and Planet \citep{ehlers2017formal} effectively construct branching technologies similar to MILP solvers. 
Indeed, \texttt{Marabou} \citep{katz2019marabou} builds on Reluplex, and a recent extension \texttt{MarabouOpt} can optimize over trained neural networks \citep{strong2021global}. 
The authors also outline general procedures to extend verification solvers to optimization. 
%Then, say we focus on methods that can be incorported into off-the-shelf MIP solvers. See also for a discussion of the two: A Unified View of Piecewise Linear Neural Network Verification, Branch and Bound for Piecewise Linear Neural Network Verification]
Our focus in this review is on more general MILP formulations, or those that can be incorporated into off-the-shelf MILP solvers with relative ease. 
\cite{bunel2020branch,bunel2018unified} provide a more comprehensive discussion of similarities and differences to SMT. 

\subsubsection{The big-$M$ formulation} 

The big-$M$ method is a standard technique used to formulate logic and disjunctive constraints using mixed-integer programming \citep{bonami2015mathematical,vielma2015mixed}. Big-$M$ formulations are typically very simple to reason about and implement, and are quite compact, though their convex relaxations can often be quite poor, leading to weak dual bounds and (often) slow convergence when passed to a mixed-integer programming solver. Since $\gr$ is a disjunctive set, the big-$M$ technique can be applied to produce the following formulation:
\begin{subequations} \label{eqn:relu-big-m}
\begin{align}
    h^l_i &\geq \vw^l_i \vh^{l-1} + b^l_i \label{eqn:relu-big-m-1} \\
    h^l_i &\leq \left(\vw^l_i \vh^{l-1} + b^l_i\right) - M^{l}_{i,-}(1-z) \\
    h^l_i &\leq M^{l}_{i,+}z \\
    (\vh^{l-1},h^l_i) &\in [L^{l-1},U^{l-1}] \times \mathbb{R}_{\geq 0} \label{eqn:relu-big-m-4} \\
    z^l_i &\in \{0,1\}. \label{eqn:relu-big-m-5}
\end{align}
\end{subequations}
Here, $M^l_{i,-}$ and $M^l_{i,+}$ are data which must satisfy the inequalities
\begin{align*}
    M^l_{i,-} &\leq \min_{\vh^{l-1} \in \mathcal{D}^{l-1}} \vw^l_i \vh^{l-1} + b^l_i \\
    M^l_{i,+} &\geq \max_{\vh^{l-1} \in \mathcal{D}^{l-1}} \vw^l_i \vh^{l-1} + b^l_i.
\end{align*}
This big-$M$ formulation for ReLU-based networks has been used extensively in the literature \citep{bunel2018unified,Cheng2017,DuttaMIP,FischettiMIP,kumar2019equivalent,LomuscioMIP,serra2020empirical,serra2018bounding,tjeng2017evaluating,xiao2018training}.

The big-$M$ formulation is compact, with one binary variable and $\mathcal{O}(1)$ general inequality constraints for each neuron. 
Applied for each unit in the network, this leads to a MILP formulation with $\mathcal{O}(\sum_{l \in \sL} n_l) = \mathcal{O}(Ln_{\max} )$ binary variables and general inequality constraints, where $n_{\max} = \max_{l \in \sL} n_L$. However, it has been observed \citep{anderson2019strong,anderson2020strong} that this big-$M$ formulation is not strong in the sense that its LP relaxation does not, in general, capture the convex hull of the graph of a given unit; see Figure~\ref{fig:relu-neuron} for an illustration. In fact, this LP relaxation can be arbitrarily bad \citep[Example 2]{anderson2019strong}, even in fixed input dimension.
As MILP solvers often bound the objective function between the best feasible point and its tightest optimal continuous relaxation, a weak formulation can negatively impact performance, often substantially. 

It is worth dwelling on where this lack of strength comes from. If the input $\vh^{l-1}$ is one dimensional, the big-$M$ formulation is \emph{locally} ideal \citep{vielma2015mixed}: the extreme points of the linear programming relaxation (\ref{eqn:relu-big-m-1}-\ref{eqn:relu-big-m-4}) naturally satisfy the integrality constraints \eqref{eqn:relu-big-m-5}. However, this fails to hold in the general multivariate input case. To see why, observe that the bounds on the input variables $\vh^{l-1}$ are only coupled with the logic involving the binary variable $z$ only in an aggregated sense, through the coefficients $M^l_{i,-}$ and $M^l_{i,+}$. In other words, the ``shape'' of the pre-activation domain is not incorporated directly into the big-$M$ formulation. 
Furthermore, the strength of this formulation highly depends on the big-$M$ coefficients. 
These coefficients can be obtained using techniques ranging from basic interval arithmetic to optimization-based bounds tightening. 
\cite{grimstad2019surrogate} show how constraints external to the neural network can yield tighter bounds via optimization- or feasibility-based bounds tightening. 
\cite{rossig2021advances} compare several methods for deriving bounds and further develop optimization-based bounds tightening based on pairwise dependencies between variables. 

\tikzstyle{yzx} = [
  x={(1.2*.9625cm, 1.2*.9625cm)},
  y={(1.2*2.5cm, 0cm)},
  z={(0cm, 1.2*3cm)},
]

\begin{figure}[t]
    \centering
    \begin{tikzpicture}[yzx]
        \draw [->, dashed, line width=1] (0,0,0) -- (1.2,0,0);
        \draw [->, dashed, line width=1] (0,0,0) -- (0,1.2,0);
        \draw [->, dashed, line width=1] (0,0,0) -- (0,0,0.7);
        \node[above right] at (1.2,-0.075,0) {$h^1_2$};
        \node[right] at (0,1.2,0) {$h^1_1$};
        \node[above] at (0,0,.7) {$h^2_1$};
        \coordinate (LL) at (0,0,0);
        \coordinate (UL) at (1,0,0);
        \coordinate (LU) at (0,1,0);
        \coordinate (UU) at (1,1,0.5);
        \coordinate (UM) at (1,0.5,0);
        \coordinate (MU) at (0.5,1,0);
        
        \draw [fill=gray!80] (LL) -- (UL) -- (UM) -- (MU) -- (LU) -- cycle;
        \draw [fill=gray!80] (UU) -- (UM) -- (MU) -- cycle;
        \draw (LL) -- (UL) -- (UU) -- cycle;
        \draw (LL) -- (LU) -- (UU) -- cycle;
    \end{tikzpicture}  \hspace{2em}
    \begin{tikzpicture}[yzx]
        \draw [->, dashed, line width=1] (0,0,0) -- (1.2,0,0);
        \draw [->, dashed, line width=1] (0,0,0) -- (0,1.2,0);
        \draw [->, dashed, line width=1] (0,0,0) -- (0,0,0.7);
        \node[above right] at (1.2,-0.075,0) {$h^1_2$};
        \node[right] at (0,1.2,0) {$h^1_1$};
        \node[above] at (0,0,.7) {$h^2_1$};
        \coordinate (LL) at (0,0,0);
        \coordinate (UL) at (1,0,0);
        \coordinate (LU) at (0,1,0);
        \coordinate (UU) at (1,1,0.5);
        \coordinate (UM) at (1,0.5,0);
        \coordinate (MU) at (0.5,1,0);
        \coordinate (E1) at (1,0,1/4);
        \coordinate (E2) at (0,1,1/4);

        \draw [fill=gray!80] (LL) -- (UL) -- (UM) -- (MU) -- (LU) -- cycle;
        \draw [fill=gray!80] (UU) -- (UM) -- (MU) -- cycle;
        \draw (LL) -- (E2) -- (UU) -- (E1) -- cycle;
        \draw (LL) -- (UL) -- (E1) -- cycle;
        \draw (LL) -- (LU) -- (E2) -- cycle;
        
        % \begin{scope}[canvas is yz plane at x=0]
        %     \draw [fill] (1,1/4) circle [radius=.025];
        % \end{scope}
    \end{tikzpicture}
    \caption{\textbf{Left:} The convex hull of a ReLU neuron \eqref{eqn:balas-relu}, and \textbf{Right:} the convex relaxation offered by the big-$M$ formulation \eqref{eqn:relu-big-m} Adapted from Anderson et al. \cite{anderson2020strong,anderson2019strong}}
    \label{fig:relu-neuron}
\end{figure}

\subsubsection{A stronger extended formulation} 
A much stronger MILP formulation can be constructed through a classical method, the extended formulation for disjunctions \citep{balas1998disjunctive,jeroslow1984modelling}. This formulation for a given ReLU neuron takes the following form~\citep[Section 2.2]{anderson2019strong}:
\begin{subequations} \label{eqn:balas-relu}
\begin{align}
    (\vh^{l-1},h^l_i) &= (x^+,y^+) + (x^-,y^-) \label{eqn:balas-relu-1} \\
    y^- &= 0 \geq \vw^{l}_i x^- + b^l_i(1-z) \\
    y^+ &= \vw^{l}_i x^+ + b^l_iz \geq 0 \\
    L^{l-1}(1-z) &\leq x^- \leq U^{l-1}(1-z) \\
    L^{l-1}z &\leq x^+ \leq U^{l-1}z \label{eqn:balas-relu-5} \\
    z &\in \{0,1\}. \label{eqn:balas-relu-6}
\end{align}
\end{subequations}
This formulation requires one binary variable and $\mathcal{O}(n_{l-1})$ general linear constraints and auxiliary continuous variables. It is also locally ideal, i.e., as strong as possible. While the number of variables and constraints for an individual unit seems quite tame, applying this formulation for unit in a network leads to a formulation with $\mathcal{O}(n_0 + \sum_{l \in \sL} n_ln_{l-1}) = \mathcal{O}(|\sL|n_{\max}^2)$ continuous variables and linear constraints. Moreover, while the formulation for \emph{an individual unit} is locally ideal, the composition of many locally ideal formulations will, in general, fail to be ideal itself. 
Consider that, while each node can be modeled as a two-part disjunction, the full network requires exponentially many disjuncts, each corresponding to one activation pattern. 

Despite its strength and relatively modest increase in size relative to the big-$M$ formulation \eqref{eqn:relu-big-m}, it has been empirically observed that this formulation often performs worse than expected \citep{anderson2019strong,vielma2019small}, both in the verification setting and more broadly.

\subsubsection{A class of intermediate formulations}
The previous sections observed that the big-$M$ formulation \eqref{eqn:relu-big-m} is compact, but may offer a weak convex relaxation, while the extended formulation \eqref{eqn:balas-relu} offers the tightest possible convex relaxation for an individual unit, at the expense of a much larger formulation.
\cite{kronqvist2022psplit,kronqvist2021steps} present a strategy for obtaining formulations intermediate to \eqref{eqn:relu-big-m} and \eqref{eqn:balas-relu} in terms of both size and strength. 
The key idea is to partition $\vw_i^l \vh^{l-1}$ into a number of aggregated variables, $\vw_i^l \vh^{l-1} = \sum_{p=1}^P \hat{x}_p$. 
Each auxiliary variable $\hat{x}_p$ is defined as a sum of a subset of the $j$-th weighted inputs $\hat{x}_p = \sum_{j \in \mathbb{S}_p} w_{i,j}^l h_j^{l-1}$, with $\mathbb{S}_1, ..., \mathbb{S}_P$ partitioning $\{1,...,n_{l-1}\}$.  
This technique can be applied to the ReLU function, giving the convex hull over the directions defined by $\hat{x}_p$ \citep{tsay2021partition}: 
\begin{subequations} \label{eq:relu-partition}
\begin{align}
\left( \sum_{j \in \mathbb{S}_p} w_{i,j}^l h_j^{l-1},h^l_i \right) &= (\hat{x}_p^+,y^+) + (\hat{x}_p^-,y^-) \label{eq:Pextended1} \\
    y^- &= 0 \geq \sum_p \hat{x}_p^- + b^l_i(1-z) \\
    y^+ &= \sum_p \hat{x}_p^+ + b^l_iz \geq 0 \\
    \hat{\boldsymbol{M}}_{i,-}^l(1-z) &\leq \hat{x}^- \leq \hat{\boldsymbol{M}}_{i,+}^l(1-z) \\
    \hat{\boldsymbol{M}}_{i,-}^l z &\leq \hat{x}^+ \leq \hat{\boldsymbol{M}}_{i,+}^l z \\
    z &\in \{0,1\}. \label{eq:Pextended_1end}
\end{align}
\end{subequations}

Here, the $p$-th elements of $\hat{\boldsymbol{M}}_{i,-}^l$ and $\hat{\boldsymbol{M}}_{i,+}^l$ must satisfy the inequalities
\begin{align*}
    \hat{M}^l_{i,-,p} &\leq \min_{\vh^{l-1} \in \mathcal{D}^{l-1}} \sum_{j \in \mathbb{S}_p} w_{i,j}^l h_j^{l-1} \\
    \hat{M}^l_{i,+,p} &\geq \max_{\vh^{l-1} \in \mathcal{D}^{l-1}} \sum_{j \in \mathbb{S}_p} w_{i,j}^l h_j^{l-1}.
\end{align*}

These coefficients can be derived using techniques analagous to those for the big-$M$ formulation (note that tighter bounds may be derived by considering $\hat{x}^-$ and $\hat{x}^+$ separately). 
Observe that when $P=1$, we recover the same tightness as the big-$M$ formulation \eqref{eqn:relu-big-m}, as, intuitively, the formulation is built over a single ``direction'' corresponding to $\vw_i^l \vh^{l-1}$. Conversely, when $P=n_{l-1}$, we recover the tightness of the extended formulation \eqref{eqn:balas-relu}, as each direction corresponds to a single element of $\vh^{l-1}$.
\cite{tsay2021partition} study partitioning strategies and show that intermediate values of $P$ result in formulations that can outperform the two extremes, by balancing formulation size and strength.


\subsubsection{Cutting plane methods: Trading variables for inequalities} \label{sec:strengthening-inequalities}
% To quickly summarize: When formulating a trained neural network, the standard approach is to formulate each unit individually, and then compose a number of such formulations together to represent the entire network. The big-$M$ formulation \eqref{eqn:relu-big-m} is compact, but may offer a weak convex relaxation, while the extended formulation \eqref{eqn:balas-relu} offers the tightest possible convex relaxation for an individual unit, at the expense of a much larger formulation. We now turn our attention to strengthening inequalities that can be added to these formulations that either: i) tighten the big-$M$ formulation to achieve the strength of the extended formulation, or ii) consider more global network structure. These strengthening inequalities are intended to be used as cutting planes: there may be a large family of them, and their intended use is that a small number of them are generated dynamically as-needed to strengthen the relaxation only in the regions of greatest interest.
The extended formulation \eqref{eqn:balas-relu} achieves its strength through the introduction of auxiliary continuous variables. However, it is possible to produce a formulation of equal strength by projecting out these auxiliary variables, leaving an ideal formulation in the ``original'' $(\vh^{l-1},h^l_i,z)$ variable space. While in general this projection may be difficult computationally, for the simple structure of a single ReLU neuron it is possible to characterize in closed form. The formulation is given by \cite{anderson2020strong,anderson2019strong} as
\begin{subequations} \label{eqn:relu-ideal}
\begin{align}
    h^l_i &\geq \vw_i^l \vh^{l-1} + b^l_i \label{eqn:relu-ideal-1} \\
    h^l_i &\leq \sum_{j \in J} w^{l}_{i,j}(h^{l-1}_i - \breve{L}^{l}_j(1-z)) + \left(b + \sum_{j \not\in J} w^{l}_{i,j}\breve{U}_j \right)z \quad \forall J \subseteq \llbracket n_{l-1} \rrbracket \label{eqn:relu-ideal-2} \\
    (\vh^{l-1},h^l_i) &\in \mathcal{D}^{l-1} \times \mathbb{R}_{\geq 0} \label{eqn:relu-ideal-3} \\
    z^l_i &\in \{0,1\},
\end{align}
\end{subequations}
where notationally, for each $j \in \llbracket n_{l-1} \rrbracket$, we take 
\begin{equation*}
\breve{L}^{l-1}_j = \begin{cases} L^{l-1}_j & w^{l}_{i,j} \geq 0 \\ U^{l-1}_j & w^{l}_{i,j} < 0 \end{cases} \quad \mathrm{and} \quad \breve{U}^{l-1}_j = \begin{cases} U^{l-1}_j & w^{l}_{i,j} \geq 0 \\ L^{l-1}_j & w^{l}_{i,j} < 0 \end{cases} 
\end{equation*}

We note a few points of interest about this formulation. First, it is ideal, and so recovers the convex hull of a ReLU activation function, coupled with its preactivation affine function and bounds on each of the inputs to that affine function. Second, it can be shown that, under very mild conditions, each of the exponentially many constraints in \eqref{eqn:relu-ideal-2} are necessary to ensure this property; none are redundant and can be removed without affecting the relaxation quality. Third, note that by selecting only those constraints in \eqref{eqn:relu-ideal-2} corresponding to $J = \emptyset$ and $J = \llbracket n_{l-1} \rrbracket$, we recover the big-$M$ formulation \eqref{eqn:relu-big-m} in the case where $\mathcal{D}^{l-1} = [L^{l-1},U^{l-1}]$. This suggests a practical approach for using this large family of inequalities: Start with the big-$M$ formulation, and then dynamically generate violated inequalities from \eqref{eqn:relu-ideal-2} as-needed in a cutting plane procedure. As shown by \cite{anderson2020strong}, this separation problem is separable in the input variables, and hence can be completed in $\mathcal{O}(n_{l-1})$ time. 

The cutting plane strategy is in general compatible with weaker formulations, such as relaxation-based verification \citep{zhang2022general} and  formulations from the class \eqref{eq:relu-partition}. 
In fact, \cite{tsay2021partition} show that the intermediate formulations in \eqref{eq:relu-partition} effectively pre-select a number of inequalities from \eqref{eqn:relu-ideal-2}, in terms of their continuous relaxations. 
While adding these constraints results in a tighter continuous relaxation, the added constraints can eventually significantly increase the model size. Practical implementations may therefore only perform cut generation at a limited number of branch-and-bound search nodes \citep{depalma2021scaling,tsay2021partition}. 

\paragraph{A subtlety when using \eqref{eqn:relu-ideal}}
This third point above raises a subtlety discussed in the literature~\citep[Appendix F]{depalma2021scaling}. Often, additional structural information is known about $\mathcal{D}^{l-1}$ beyond bounds on the variables. In this case, it is typically possible to derive tighter values for the big-$M$ coefficients. In this case, when using a separation-based approach it is preferable to initialize the formulation with these tightened big-$M$ constraints, and then proceed with the cutting plane approach as normal from there.

\subsection{Scaling further: Convex relaxations and linear programming}
\label{sec:relaxations}

The above demonstrate MILP as a powerful framework for exactly modeling complex, nonconvex trained neural networks, but standard solvers are often not sufficiently scalable to adequately handle large-scale networks. A natural approach to increase the scalability, then, is to \emph{relax} the network in some manner, and then apply convex optimization methods. For the verification problem discussed in Section~\ref{sec:verification}, this yields what is known as an \emph{incomplete verifier}: any certification of robustness provided can be trusted (no false positives), but there may be robust instances that the method cannot prove are (some false negatives). 
In other words, over-approximation produces a verifier that is sound, but not complete. 

While a variety of methods exist for accomplishing this, in this section we briefly outline techniques relevant to polyhedral theory. 
In particular, we focus on some techniques for building convex polyhedral relaxations. 
The most natural convex relaxation for a MILP formulation is its linear programming (LP) relaxation, constructed by dropping any integrality constraints. For example, the LP relaxation of \eqref{eqn:relu-big-m} is given by the system (\ref{eqn:relu-big-m-1}-\ref{eqn:relu-big-m-4}). This is a compact linear programming relaxation for a ReLU-based network, and is the basis for methods due to \cite{bunel2020lagrangian} and \cite{ehlers2017formal}.

\subsubsection{Projecting the big-$M$ and ideal MILP formulations}
This section examines projections of the linear relaxations of formulations \eqref{eqn:relu-big-m} and \eqref{eqn:relu-ideal}. \\

\textbf{(Projecting the big-$M$).}
Note that the LP relaxation given by (\ref{eqn:relu-big-m-1}--\ref{eqn:relu-big-m-4}) maintains the variables $z^l_i$ in the formulation, though they are no longer required to satisfy integrality. Since these variables are ``auxiliary'' and are no longer necessary to encode the nonconvexity of the problem, they can be projected out without altering the quality of the convex relaxation. Doing this yields what is commonly known as the ``triangle'' or ``$\Delta$'' relaxation \citep{salman2019convex}:
\begin{subequations} \label{eqn:triangle-relaxation}
\begin{align}
    h^l_i &\geq \vw^l_i \vh^{l-1} + b^l_i \\
    h^l_i &\leq \frac{M^l_{i,+}}{M^l_{i,+} - M^l_{i,-}}(\vw^l_i \vh^{l-1} + b^l_i) \\
    (\vh^{l-1},h^l_i) &\in [L^{l-1},U^{l-1}] \times \mathbb{R}_{\geq 0}.
\end{align}
\end{subequations}

While the LP relaxation \eqref{eqn:triangle-relaxation} for an individual unit is compact, modern neural network architectures regularly comprise millions of units. The resulting LP relaxation for the entire network may then require millions of variables and constraints. Additionally, unless special precautions are taken, many of these constraints will be relatively dense. All this quickly leads to LP that are beyond the scope of modern off-the-shelf LP solvers. As a result, researchers have explored alternative schemes for scaling LP-based methods to these larger networks.
\cite{salman2019convex} present a framework for LP-based methods (LP solvers, propagation, dual methods), which we review in the following subsections. However, they do not account for the ideal formulation developed in later works \citep{anderson2020strong,depalma2021scaling}. 


\textbf{(Projecting the ideal).}
Figure~\ref{fig:relu-neuron} shows that the triangle (big-$M$) relaxation fails to recover the convex hull of the ReLU activation function and the multivariate input to the affine pre-activation function.  
We can similarly project the LP relaxation of the ideal formulation \eqref{eqn:relu-ideal} into the space of input/output variables \citep{anderson2020strong}, yielding a description for the convex hull of $\{ (\vh^{l-1},h^l_i) | L^{l-1} \leq \vh^{l-1} \leq U^{l-1}, \: h^l_i = \sigma(\vw^{l}_i \vh^{l-1} + b^l_i) \}$:%\cite{anderson2020strong} derive this convex hull description as:
\begin{subequations}
\begin{align}
    h^l_i &\geq \vw^l_i \vh^{l-1} + b^l_i \\
    h^l_i &\leq \sum_{k \in I} w_{i,k}^l (x_k - \breve{L}_k) + \frac{\ell(I)}{\breve{U}_h - \breve{L}_h}(x_h - \breve{L}_h) \quad \forall (I,h) \in \mathcal{J} \label{eqn:projected-ideal-relu-2} \\
    (\vh^{l-1},h^l_i) &\in [L^{l-1},U^{l-1}] \times \mathbb{R}_{\geq 0},
\end{align}
\end{subequations}
where $l(I) \coloneqq \sum_{k \in I} w^l_{i,k}\breve{L}_k + \sum_{k \not\in I} w^l_{i,k} \breve{U}_k + b^l_i$ and
\[
    \mathcal{J} \coloneqq \Set{ (I,h) \in 2^{\llbracket n_{l-1} \rrbracket} \times \llbracket n_{l-1} \rrbracket | l(I) \geq 0, \: l(I \cup \{h\} < 0, \: w^l_{i,k} \neq 0 \forall k \in I }.
\]
\cite{anderson2020strong} also show that the inequalities \eqref{eqn:projected-ideal-relu-2} can be separated over in $\mathcal{O}(n_{l-1})$ time. 
Interestingly, in contrast to \eqref{eqn:relu-ideal}, the number of facet-defining inequalities depends heavily on the affine function. While in the worst case the number of inequalities will grow exponentially in the input dimension, there exist instances where the convex hull can be fully described with only $\mathcal{O}(n_{l-1})$ inequalities.

%\subsubsection{Projecting the ideal formulation}

\subsubsection{Dual decomposition methods}

A first approach for greater scalability for LP-based methods is decomposition, a standard technique in the large-scale optimization community. Indeed, the cutting plane approach of Section~\ref{sec:strengthening-inequalities} can be viewed as a decomposition method operating in the original variable space. However, the method is initialized with the big-$M$ formulation for each neuron, and hence this initial model will be of size roughly equal to the size of the network. Therefore, it should be understood to use decomposition to provide a tighter verification bound, rather than for providing greater scalability to larger networks.

In contrast, dual decomposition can be used to scale inexact verification methods to larger networks. Such methods maintain dual feasible solutions throughout the algorithm, meaning that upon termination they yield valid dual bounds on the verification instance, and hence serve as incomplete verifiers.

\cite{wong2018provable,wong2018scaling} use as their starting point the triangle relaxation \eqref{eqn:triangle-relaxation} for each neuron, and then take the standard LP dual of the (relaxed) verification problem. Alternatively, \cite{dvijotham2018dual} propose a Lagrangian-based approach for decomposing the original nonlinear formulation of the problem \eqref{eq:verification}. Crucially, since the complicating constraints coupling the layers in the network are imposed as objective penalties instead of ``hard'' constraints, the optimization problem (given fixed dual variables) decomposes along each layer and the subproblems induced by the separability can be solved in closed form. This approach dualizes separately the equations characterizing the pre-activation and post-activation functions:
\begin{align*}
    \max_{\mu,\lambda}\quad \min_{\vh,\hat{\vh}} \quad& \left( \mW^L {\vh}^{L-1} + \vb^L \right) + \sum_{k=1}^{L-1} \left( \mu_k^T(\hat{\vh}^k - \mW^k \vh^{k-1} - \vb^k) + \lambda_k^T(\vh^k - \sigma(\hat{\vh}^k) \right) \\
    \text{s.t.} \quad& L^k \leq \hat{\vh}^k \leq U^k \quad \forall k \in \llbracket n - 1 \rrbracket \\
    & \sigma(L^k) \leq \vh^k \leq \sigma(U^k) \quad \forall k \in \llbracket n - 1 \rrbracket.
\end{align*}

Here, the $\hat{\vh}$ variables track the pre-activation values for the neurons in the network. 
The dual variables $\mu_k^T$ correspond to the equality constraints defining the pre-activation values, $\hat{\vh}^k = \mW^k \vh^{k-1} + \vb^k$. 
Likewise, the dual variables $\lambda_k^T$ correspond to enforcing the ReLU activation function, $\vh^k = \sigma(\hat{\vh}^k) = \mathrm{max} (0, \hat{\vh}^k)$. 
Any feasible solution for the neural network is feasible for this dualized problem, making the multiplier terms for $\mu_k^T$ and $\lambda_k^T$ zero. 
Thus, the inner problem gives a lower bound for the original problem---a property known as \emph{weak duality}. 
The outer (dual) problem optimizing over the Lagrangian multipliers then seeks to maximize this lower bound, i.e., to give the tightest possible lower bound. This can be solved using a subgradient-based method, or learned along with the model parameters in a ``predictor-verifier'' approach \citep{dvijotham2018training}. 

%Another work \cite{Lagrangian Decomposition for Neural Network Verification} alternatively explores Lagrangian decomposition (i.e. variable splitting) applied to the verification problem, 
On the other hand, this approach can be combined with other relaxation-based methods. 
The Lagrangian decomposition can be applied to dualize only the coupling constraints between layers, and a convex relaxation used for the activation function \citep{bunel2020lagrangian}: 
\begin{align*}
    \max_\lambda \quad \min_{\vh,\hat{\vh}} \quad& \left( \mW^L {\vh}^{L-1} + \vb^L \right) + \sum_{k=1}^{L-1} \left( \lambda_k^T(\vh^k - \sigma(\hat{\vh}^k) \right) \\
    \text{s.t.} \quad
    %& \vh_0 \in D^0 \\
    %& \hat{\vh}_1 = W_1\vh_0 + b^1 \\
    & L^k \leq \hat{\vh}^k \leq U^k \quad \forall k \in \llbracket n - 1 \rrbracket \\
    & \hat{\vh}^{k} = \mW^{k} \vh^{k-1} + b^k \quad \forall k \in \llbracket n - 1 \rrbracket \\
    & \vh^k \geq 0 \quad \forall k \in \llbracket n - 1 \rrbracket \\
    & \vh_i^k \geq \hat{\vh}_i^k \quad \forall k \in \llbracket n - 1 \rrbracket, \forall i \in \llbracket n_k \rrbracket \\
    & \vh^k_i \leq \frac{U^k_i(\hat{\vh}^k_i - L^k_i)}{U^k_i - L^k_i} \quad \forall k \in \llbracket n - 1 \rrbracket, \forall i \in \llbracket n_k \rrbracket.
\end{align*}
Note that the final three constraints apply the big-$M$/triangle relaxation \eqref{eqn:triangle-relaxation} to each ReLU activation function. 
The dual problem can then be solved via subgradient-based methods, proximal algorithms, or, more recently, a projected gradient descent method applied to a nonconvex reformulation of the problem \citep{bunel2020efficient}.

More recently, \cite{depalma2021scaling} presented a dual decomposition approach based on \eqref{eqn:relu-ideal}. 
However, creating a dual formulation from the exponential number of constraints produces an exponential number of dual variables. The authors therefore propose to maintain an ``active set'' of dual variables to keep the problem sparse. A selection algorithm (e.g., selecting entries that maximize an estimated super-gradient) can then be used to append the active set. 
Similar to the above discussion on cut generation, the frequency of appending the active set should be chosen strategically. 



\subsubsection{Fourier-Motzkin elimination and propagation algorithms}
\newcommand{\defeq}{\vcentcolon=}
Alternatively, one can project out \emph{all} of the decision variables. For example, in order to solve the linear programming problem $\min_{x \in \mathcal{X}} c \cdot x$, we can augment the problem with a new decision variable to $\min_{(x,y) \in \Gamma} y$ for $\Gamma \defeq \Set{(x,y) \in \mathcal{X} \times \mathbb{R} : y = c \cdot x}$, and project out the $x$ variables. The transformed problem is the a trivial univariate optimization problem: $\min_{y \in \operatorname{Proj}_y(\Gamma)} y$.

Of course, the complexity of the approach described is hidden in the projection step, or building $\operatorname{Proj}_y(\Gamma)$. The most well-known algorithm for computing projections of linear inequality systems is Fourier-Motzkin elimination, described by \cite{dantzig1973fourier}, which is notorious for its practical inefficiency. 
The process effectively comprises replacing variables from a set of inequalities with all possible implied inequalities, which can produce many unnecessary constraints.  
However, it turns out that neural network verification problems are well-structured in such a way that Fourier-Motzkin elimination can be performed very efficiently: for instance, by imposing one inequality upper bounding and one inequality lower bounding each ReLU function. 
Note that while Section \ref{sec:algebraoflinearegions} describes the use of Fourier-Motzkin elimination to obtain \emph{exact} input-output relationships in linear regions of neural networks, here we are interested in obtaining linear \emph{bounds} for a nonlinear function.

\begin{figure}
    \centering
    \includegraphics[width=\textwidth]{relaxations.pdf}
    \caption{Convex approximations for the ReLU function commonly used by propagation algorithms, given as a function of the preactivation function $\hat{h_i^l}$. The ReLU applies $h_i^l = \max(0,\hat{h_i^l} )$.}
    \label{fig:convexapproximations}
\end{figure}

In fact, this general approach was independently developed in the verification community.
While MILP research has focused on formulations tighter than the big-M, such as \eqref{eqn:balas-relu} and \eqref{eq:relu-partition}, the verification community often prefers greater scalability at the price of weaker convex relaxations.
The continuous relaxation of the big-M is equivalent to the triangle relaxation \eqref{eqn:triangle-relaxation}: the optimal convex relaxation for a single input, or in terms of the aggregated pre-activation function, as shown in Figure \ref{fig:convexapproximations}.  
However, the lower bound involves two linear constraints, which is not used in several propagation-based verification tools owing to scalability or compatibility. 

Such tools use methods such as abstract transformers to propagate polyhedral bounds, i.e., \textit{zonotopes}, through the layers of a  neural network. 
DeepZ \citep{singh2018fast}, Fast-Lin \citep{weng2018towards}, and Neurify \citep{wang2018efficient} employ a parallel approximation, with the latter also implementing a branch-and-bound procedure towards completeness. 
Subsequently, DeepPoly \citep{singh2019abstract}  and CROWN \citep{zhang2018efficient} select between the zero and identity approximations by minimizing over-approximation area. 
OSIP \citep{hashemi2021osip} selects between the three approximations using optimization: approximations for a layer are select jointly to minimise bounds for the following layer. 
These technologies are also compatible with interval bounds, propagating box domains \citep{mirman2018differentiable}. 
Interestingly, bounds on neural network weights can also be propagated using similar methods, allowing reachability analysis of Bayesian neural networks \citep{wicker2020probabilistic}. 

\cite{tjandraatmadja2020convex} provide an interpretation of these propagation techniques through the lens of Fourier-Motzkin elimination. 
Consider the problem of propagating bounds through a ReLU neural network: for a node $h_i^l = \mathrm{max} \{0, \hat{h_i^l} \}$, convex bounds for $h_i^l$ can be obtained given bounds for $\hat{h}_i^l$ (Figure \ref{fig:convexapproximations}). Assuming the inputs are outputs of ReLU activations in the previous layer, $\hat{h}_i^l = \vw_i^l \vh^{l-1} + b_i^l$. 
Computing an upper bound can then be expressed as:
\begin{equation*}
    \begin{aligned}
    \max_{\vh^{l-1}} \quad& \vw_i^l \vh^{l-1} + b_i^l \\
    \text{s.t.} \quad& \mathcal{L}_k(\vh^{l-2}) \leq h^{l-1}_k \leq \mathcal{G}_k(\vh^{l-2}) \forall k \in \{ 1,...,n_{l-1} \}.
    \end{aligned}
\end{equation*}

As the objective function is linear, the solution of this problem can be computed by propagation without explicit optimization. 
For each element in $\vh^{l-1}$, we only need to consider the associated objective coefficient in $\vw_i^l$ to determine whether $\mathcal{L}_k(\vh^{l-2}) \leq h^{l-1}_k$ or $h^{l-1}_k \leq \mathcal{G}_k(\vh^{l-2})$ will be the active inequality at the optimal solution. 
We can thus replace $h^{l-1}_k$ with $\mathcal{L}_k(\vh^{l-2})$ or $\mathcal{G}_k(\vh^{l-2})$ accordingly. 
This projection is mathematically equivalent to applying Fourier-Motzkin elimination, while avoiding redundant inequalities resulting from the `non-selected' bounding function. 
Repeating this procedure for each layer results in a convex relaxation for the outputs that only involves the input variables.
We naturally observe the desirability of simple lower bounds $\mathcal{L}_k(\vh^{l-2})$: imposing two-part lower bounds in each layer would increase the number of propagated constraints in an exponential manner, similar to Fourier-Motzkin elimination. 

%Overview of DeepPoly (An Abstract Domain for Certifying Neural Networks) and FastLin, draw connection with Fourier-Motzkin elimination
%Fast and Effective Robustness Certification: Modification of DeepPoly (DeepZ) by ``tilting'' zonotopic upper/lower bounding pairs to minimize volume.

\paragraph{A path towards completeness}
Given an input-output bound, the reachable set can be refined by splitting the input space \citep{henriksen2021deepsplit,rubies2019fast}---a strategy similar to spatial branch and bound. In other words, completeness is achieved by branching in the input space, rather than activation patterns: this strategy is especially effective when the input space is low dimensional \citep{strong2022zope}. For example, ReluVal \citep{wang2018formal} propagates symbolic intervals and implements splitting procedures on the input domain. As the interval extension of ReLU is Lipschitz continuous, the method converges to arbitrary accuracy in a finite number of splits.

\subsection{Generalizing the single neuron model}

\subsubsection{Extending to other domains} \label{sec:domains}
In general, we will expect that the effective input domain $\mathcal{D}^{l-1}$ for a given unit may be quite complex. For the first layer ($l=1$) this may derive from explicitly stated constraints on the inputs of the networks, while for later layers this will typically derive from the complex nonlinear transformations applied by the preceding layers. 
For example, in the context of surrogate models \cite{yang2022modeling} propose bounding the input to the convex hull of the training data set, while other works \citep{schweidtmann2022obey,shi2022careful} propose machine learning-inspired techniques for learning the trust region implied by the training data. 
In effect, these methods assume a trained model is locally accurate around training data, which is a property similar to that which verification seeks to prove. 

Nevertheless, most research focuses on hyperrectangular input domains, largely motivated by practical considerations: i) there are efficient, well-studied methods for computing valid (though not necessarily optimally tight) variable bounds, ii) characterizing the exact effective domain may be computationally impractical, and iii) and the hyperrectangular structure makes analysis simpler for complex formulations like those presented in Section~\ref{sec:strengthening-inequalities}. 
We note that \cite{jordan2019neurips} use polyhedral analyses to perform verification over arbitrary (including non-polyhedral) norms, by fitting a $p$-norm ball in the decision region and checking adjacent linear regions. 
On the other hand, robust optimization can be employed to find $p$-norm adversarial regions (rather than verifying robustness), as opposed to a single point adversary \citep{maragno2023finding}. 

\cite{anderson2020strong} present two closely related frameworks for constructing ideal and hereditarily sharp formulations for ReLU units with arbitrary polyhedral input domains. This characterization is derived from Lagrangian duality, and requires an infinite number of constraints (intuitively, one for each choice of dual multipliers). Nonetheless, separation can still be done over this infinite family of inequalities via a subgradient-based algorithm; this approach will be tractable if optimization over $\mathcal{D}^{l-1}$ is tractable. 
Many propagation algorithms are also fully compatible with arbitrary polyhedral input domains, as the projected problem (i.e., a linear input-output relaxation) remains an LP. 
\cite{singh2021overcoming} show that simplex input domains can actually be beneficial, creating tighter formulations by propagating constraints on the inputs through the network layers.
Similarly, optimization-based bound tightening problems based on solving LPs can embed constraints defining polyhedral input domains. 


In certain cases, additional structural information about the input domain can be used to reduce this semi-infinite description to a finite one. For example, this can be done when $\mathcal{D}^{l-1}$ is a Cartesian product of unit simplices \citep{anderson2020strong} (note that this generalizes the box domain case, wherein each simplex is one-dimensional). This particular structure is particularly useful for modeling input domains with combinatorial constraints. For example, a network trained to predict binding propensity of a given length-$n$ DNA sequence is naturally modeled via an input domain that is the product of $n$ 4-dimensional simplices--one simplex for each letter in the sequence, each of which is selected from an alphabet of length 4.

\subsubsection{Extending to other activation functions}
The big-$M$ formulation technique can be any piecewise linear activation function. While much of the literature focuses on the ReLU due to its widespread popularity, models for other activation functions have been explored in the literature. For example, multiple papers \citep[Appendix K]{serra2018bounding} \citep[Appendix A.2]{tjeng2017evaluating} present a big-$M$ formulation for the maxout activation function. 
Adapting a formulation from \cite{anderson2020strong} \citep[Proposition 10]{anderson2020strong}, a formulation for the maxout unit is
\begin{alignat*}{2}
    y^l_i &\leq u_j(\vh^{l-1}) + M^l_{i,j}(1-z_j) \quad ^\forall j \in \llbracket k \rrbracket \\
    y^l_i &\geq u_j(\vh^{l-1}) \quad ^\forall j \in \llbracket k \rrbracket \\
    \sum_{j=1}^k z_j &= 1 \\
    (\vh^{l-1},v^l_i,z) &\in \mathcal{D}^{l-1} \times \mathbb{R} \times \{0,1\}^k,
\end{alignat*}
where each $M^l_{i,j}$ is selected such that
\[
    M^l_{i,j} \geq \max_{\tilde{\vh} \in \mathcal{D}^{l-1}} u_j(\tilde{\vh}).
\]

We can observe that the big-$M$ formulation can also handle other discontinuous activation functions, such as a binary/sign activations \citep{han2021single} or more general quantized activations \citep{nguyen2022}. 
Nevertheless, the binary activation function naturally lends itself towards Boolean satisfiability, and most work therefore focuses on alternative methods such as SAT \citep{cheng2018verification,jia2020efficient,narodytska2018verifying}. 

While this survey focuses on neural networks with piecewise linear activation functions, we note that recent research has also studied smooth activation functions with a similar aim. 
For example, optimization over smooth activation functions can be handled by piecewise linear approximation and conversion to MILP \citep{sildir2022mixed}. 
Researchers have also studied convex/concave bounds for nonlinear activation functions, which can then be embedded in spatial branch-and-bound procedures \citep{schweidtmann2019deterministic,wilhelm2022convex}.
In contrast to MILP formulations for ReLU neural networks, these problems are typically nonlinear programs that must be solved via spatial branch and bound.

Propagation methods \citep{singh2018fast,zhang2018efficient} can also naturally handle general activation functions: given convex polytopic bounds for an activation function, these tools can propagate them through network layers using the same techniques. 
For example, Fastened CROWN \citep{lyu2020fastened} employs a set of search heuristics to quickly select linear upper and lower bounds on ReLU, sigmoid, and hyperbolic tangent activation functions. 
Tighter polyhedral bounds can be employed, such as piecewise linear upper and lower bounds \citep{benussi2022individual}. %Note that these bounds are created in a single dimension, the aggregated preactivation, but more sophisticated methods for multivariate piecewise linear relaxations may also be applicable (e.g., \cite{misener2012global}). 

\subsubsection{Extending to adversarial training} \label{sec:adversarialtraining}
As described in Section \ref{sec:intro}, the \emph{training} of neural networks seeks to minimise a measure of distance between the output $y$ and the correct output $\hat{y}$.
For instance, if this distance is prescribed as loss function $\ERMfunction(y,\hat{y})$, this corresponds to solving the \emph{training} optimization problem: 

\begin{equation} \label{eq:training}
\underset{\{ \mW^l \}_{l \in \sL}, \{ \vb^l \}_{l \in \sL}}{\mathrm{min}} \ERMfunction(y,\hat{y}).
\end{equation}

Further details about the training problem and solution methods are described in the following section. 
Here, we briefly outline how verification techniques can be embedded in training. 
Specifically, solutions or bounds to the verification problem (Section \ref{sec:verification}) provide a metric of how robust a trained neural network is to perturbations. These metrics can be embedded in the training problem to obtain a more robust network during training, often resulting in a bilevel training problem. For instance, the verification problem \eqref{eq:verification} can be embedded as a lower-level problem, giving the robust optimization problem:
\begin{align*}
\underset{\{ \mW^l \}_{l \in \sL}, \{ \vb^l \}_{l \in \sL}}{\mathrm{min}} \quad \underset{||x - \hat{x}|| \leq \epsilon}{\mathrm{max}} \quad \ERMfunction(y=f(x),\hat{y}).
\end{align*}

Solving these problems generally involves either bilevel optimization, or computing an adversarial solution/bound at each training step, conceptually similar to the robust cutting plane approach. 
\cite{madry2018towards} proposed this formulation and solved the nonconvex inner problem using gradient descent, thereby losing a formal certification of robustness. 
These approaches may also benefit from reformulation strategies, such as by taking the dual of the inner problem and using any feasible solution as a bound \citep{wong2018provable}. 
The resulting models are not only more robust, but several works have also found it to be empirically easier to verify robustness in them  \citep{mirman2018differentiable,wong2018provable}. 

Alternatively, robustness can be induced by designing an additional penalty term for the training loss function, in a similar vein to regularization. For example: 
\begin{equation*} 
\underset{\{ \mW^l \}_{l \in \sL}, \{ \vb^l \}_{l \in \sL}}{\mathrm{min}} \kappa \ERMfunction(y,\hat{y}) + (1-\kappa) \ERMfunction_\mathrm{robust}(\cdot).
\end{equation*}

Additionally, if these robustness penalties are differentiable, they can be embedded into standard gradient descent based optimization algorithms \citep{dvijotham2018dual,mirman2018differentiable}. 
In the above formulation, the parameter $\kappa$ controls the relative weighting between fitting the training data and satisfying some robustness criterion, and its value can be scheduled during training, e.g., to first focus on model accuracy \citep{gowal2018effectiveness}.
In these cases, over-approximation of the reachable set is less problematic, as it merely produces a model \emph{more} robust than required. 
Nevertheless, \cite{balunovic2020adversarial} improve relaxation tightness by searching for adversarial examples in the ``latent'' space between hidden layers, reducing the number of propagation steps. \cite{zhang2020towards} provide an implementation that that tightens relaxations by also propagating bounds backwards through the network. 





\section{Training}
In this section we describe the overall training algorithm and the unsupervised criterion we used to select hyper-parameters.

\begin{figure}[!t]
\begin{center}
\includegraphics[width=.75\linewidth]{images/autocrossencoder5.png}
\end{center}
\caption{Illustration of the proposed architecture and training objectives. The architecture is a sequence to sequence model, with both encoder and decoder operating on two languages depending on an input language identifier that swaps lookup tables.
Top (auto-encoding): the model learns to denoise sentences in each domain. Bottom (translation): like before, except that we encode from another language, using as input the translation produced by the model at the previous iteration (light blue box). The green ellipses indicate terms in the loss function.}
\label{fig:model_outline1}
\end{figure}
\subsection{Iterative Training }


The final learning algorithm is described in Algorithm \ref{algo:main_algo} and the general architecture of the model is shown in Figure \ref{fig:model_outline1}. As explained previously, our model relies on an iterative algorithm which starts from an initial translation model $M^{(1)}$ (line 3). This is used to translate the available monolingual data, as needed by the cross-domain loss function of Equation~\ref{eq:xdomain}.
At each iteration, a new encoder and decoder are trained by minimizing the loss of Equation~\ref{eq:loss} -- line 7 of the algorithm. Then, a new translation model $M^{(t+1)}$ is created by composing the resulting encoder and decoder, and the process repeats.

To jump start the process, $M^{(1)}$ simply makes a word-by-word translation of each sentence using a parallel dictionary learned using the unsupervised method proposed by~\citet{wordalign17}, which only leverages monolingual data.

\begin{algorithm}[t] 
\caption{Unsupervised Training for Machine Translation}
\begin{algorithmic}[1]
\Procedure{Training}{$\mathcal{D}_{src}$, $\mathcal{D}_{tgt}$, $T$}
\State Infer bilingual dictionary using monolingual data~\citep{wordalign17}
\State $M^{(1)} \gets $ unsupervised word-by-word translation model using the inferred dictionary
\For{$t=1,T$}
\State using $M^{(t)}$, translate each monolingual dataset 
\State // discriminator training \& model training as in eq.~\ref{eq:loss}
\State $\theta_\mathrm{discr} \gets  \arg \min  \mathcal{L}_{D}$, \hspace{.2cm} $\theta_\mathrm{enc},\theta_\mathrm{dec},\mathcal{Z} \gets \arg \min  \mathcal{L}$
\State $M^{(t+1)} \gets e^{(t)} \circ d^{(t)}$  // update MT model
\EndFor
\State return $M^{(T+1)}$
\EndProcedure
\end{algorithmic}
\label{algo:main_algo}
\end{algorithm}

The intuition behind our algorithm is that as long as the initial translation model $M^{(1)}$ retains at least some information of the input sentence, the encoder will map such translation into a representation in feature space that also corresponds to a cleaner version of the input, since the encoder is  trained to denoise. At the same time, the decoder is trained to predict noiseless outputs, conditioned on noisy features. Putting these two pieces together will produce less noisy translations, which will enable better back-translations at the next iteration, and so on so forth. 

\subsection{Unsupervised Model Selection Criterion}
\label{sec:unsupervised_criterion}

In order to select hyper-parameters, we wish to have a criterion correlated with the translation quality. However, we do not have access to parallel sentences to judge how well our model translates, not even at validation time. Therefore, we propose the surrogate criterion which we show correlates well with BLEU~\citep{bleu}, the metric we care about at test time.

For all sentences $x$ in a domain $\ell_1$, we translate these sentences to the other domain $\ell_2$, and then translate the resulting sentences back to $\ell_1$. The quality of the model is then evaluated by computing the BLEU score over the original inputs and their reconstructions via this two-step translation process. The performance is then averaged over the two directions, and the selected model is the one with the highest average score. 

Given an encoder $e$, a decoder $d$ and two non-parallel datasets $\mathcal{D}_{src}$ and $\mathcal{D}_{tgt}$, we denote $M_{src \rightarrow tgt}(x) = d(e(x,src),tgt)$ the translation model from \textit{src} to \textit{tgt}, and $M_{tgt \rightarrow src}$ the model in the opposite direction. Our model selection criterion  $MS(e,d,\mathcal{D}_{src},\mathcal{D}_{tgt})$ is:
\begin{eqnarray}
    MS(e,d,\mathcal{D}_{src},\mathcal{D}_{tgt}) &=& \frac{1}{2} 
    \mathbb{E}_{x \sim \mathcal{D}_{src}}\left[ \mathrm{BLEU}(x,M_{src \rightarrow tgt} \circ M_{tgt \rightarrow src}(x) ) \right] + \nonumber \\ 
    & & \frac{1}{2} 
    \mathbb{E}_{x \sim \mathcal{D}_{tgt}}\left[ \mathrm{BLEU}(x,M_{tgt \rightarrow src} \circ M_{src \rightarrow tgt}(x) ) \right] \label{eq:ms}
\end{eqnarray}
Figure~\ref{fig:unsupervised_criterion} shows a typical example of the correlation between this measure and the final translation model performance (evaluated here using a parallel dataset).

The unsupervised model selection criterion is used both to a) determine when to stop training and b) to select the best hyper-parameter setting across different experiments. In the former case, the Spearman correlation coefficient between the proposed criterion and BLEU on the test set is 0.95 in average. In the latter case, the coefficient is in average 0.75, which is fine but not nearly as good. For instance, the BLEU score on the test set of models selected with the unsupervised criterion are sometimes up to 1 or 2 BLEU points below the score of models selected using a small validation set of 500 parallel sentences.


\sidecaptionvpos{figure}{c}
\begin{SCfigure}%[tb]
    \centering
   \includegraphics[scale=0.25]{images/unsupervised_criterion}
   \caption{\textbf{Unsupervised model selection.} BLEU score of the source to target and target to source models on the Multi30k-Task1 English-French dataset as a function of the number of passes through the dataset at iteration ${(t)}=1$ of the algorithm (training $M(2)$ given $M(1)$). BLEU correlates very well with the proposed model selection criterion, see Equation~\ref{eq:ms}.\label{fig:unsupervised_criterion}}
\end{SCfigure}




\section{Discussion and Conclusions}



Our method based on stabilizing forward and backward pass, resulted in improved accuracy over the baseline and it was able to predict optimal dampening, sharpness and tail-fatness before training. 
Our findings are coherent with the line of research that has established that stabilizing gradients and representations at initialization results in better performance \cite{glorot2010understanding, orthogonal_initialization, he2015delving, roberts2022principles, defazio2022scaling, bengio1994learning, hochreiter1997long, hochreiter2001gradient, arjovsky2016unitary, pascanu2013difficulty}. Moreover it gives an initial reply to the question raised by
\cite{surrogate2019, zenke2021remarkable}, which asked  for a theoretical justification of initialization and SG choice for Spiking Neural Networks. With a similar intention, \cite{rossbroich2022fluctuation} proposed an approach that guarantees sparsity of activity at initialization to pick the weights distribution at initialization, resulting in improved accuracy. Our method differs from theirs in that it starts from a principle of stability to derive constraints, instead of a principle of sparsity. It differs also in that we use it to define the SG shape at initialization, not only the weights distribution, and we can show mathematically how weights initialization is intertwined to the SG shape choice. Our results suggest that a tedious hyper-parameter grid-search can be often avoided by making use of sound and established principles of learning.

One of the conditions was designed to hit the most sensitive part of an SG, its center, which resulted in a low sparsity requirement at initialization. This is very uncommon in the Neuromorphic literature, since sparsity brings large energy gains \cite{henderson2020towards,blouw2019benchmarking, 9395703,taulsnn, rossbroich2022fluctuation}.
However, the energy gains of SNNs also come from their binary activity. A matrix-vector multiplication, with a $\mathbb{R}^{m\times n}$ matrix, has an energy cost of $mnE_{MAC}$ for a real vector, and of $mn\rho E_{AC}$ for a binary vector, where $\rho$ is the Bernouilli probability of the binary vector, and in our case the neuron firing rate, and $E_{AC}, E_{MAC}$ are the energies of an accumulate and a multiply-accumulate operation \cite{yin2021accurate, hunger2005floating}. Since MAC are more costly than AC, 31 times on a $45$nm complementary metal–oxide–semiconductor \cite{yin2021accurate, horowitz20141}, we have energy savings with any $\rho$, e.g., when all neurons fire ($\rho=1$) and when they fire half of the time steps ($\rho=1/2$). This gain does not depend on the simulation speed, since it compares a spiking and an analogue computation, at the same computation speed.
Typically requiring more sparsity through a sparsity encouraging loss term, leads to a measurable decrease in performance \cite{zenke2021remarkable, rossbroich2022fluctuation}. However we observed that it is actually possible to achieve higher performance with higher sparsity, by starting with a strong firing rate at initialization, since their synergy acts as a regularization mechanism. This was possible also because the sparsity encouraging loss term was introduced gradually, and because its contribution was kept comparable to the task loss towards the end of training.

We observed that the more complex the task is and the more complex the network to train is, the more drastic is the difference in performance of different SG shapes. It is known that learning is possible with a wide variety of SG shapes \cite{zenke2021remarkable} and the community has not yet settled for one shape or one method to reliably choose which SG to use in each case \cite{surrogate2019}. We showed how to apply a well known stability principle to the forward and backward pass of the simplest Spiking Neural Network, the LIF, as a starting point, but we think that the principles of good Neuromorphic initialization can be further elaborated, in order to tackle more complex tasks and networks.




\paragraph{Acknowledgments} We thank Christian Tjandraatmadja and Toon Tran for early feedback on the manuscript and asking questions that helped shaping it. 

Thiago Serra was  supported by
the National Science Foundation (NSF) award IIS 2104583. 
Calvin Tsay was supported by the Engineering \& Physical Sciences Research Council (EPSRC) grant EP/T001577/1. 

%\bibliographystyle{siamplain}
\bibliography{references}

\end{document}

