\section{Introduction}

%In this study, a photon detection system specialized for muonic X-ray spectroscopy was developed. 
A muonic atom is a bound system consisting of a negative muon and a nucleus surrounded by electrons, formed %with 100\% formation probability 
when a negative muon stops in the matter. 
As the lifetime of the muon is sufficiently longer than the time scale of the atomic transition, X-rays originating from muonic transitions are emitted. 
As the muon mass is approximately 207 times heavier than the electron mass, muonic X-rays have higher energy than electric X-rays. 
%Muonic X-ray is a transition X-ray emitted by a muonic atom.%, a bound system of negative muon and a nucleus.
Measurements of muonic X-rays are utilized in several fields of the natural and social sciences, including nuclear, atomic, and particle physics, chemistry, earth and planetary sciences, archaeology, and industrial applications.
%non-destructive analysis, measurement of the nuclear charge radius and distributions, and so on.


Since muonic X-rays have characteristic energies for each element, they are used for nondestructive elemental analysis, the so-called ``muon induced X-ray emission (MIXE).'' 
MIXE is a nondestructive, three-dimensionally position selective, and simultaneous multi-element analysis, which can analyze several elements from lithium to uranium in bulk samples. % because muonic X ray has high transparency owing to its high energy. 
MIXE has been utilized for several applications, such as the analysis of meteorites~\cite{Terada2014-cw, Terada2017-kb, Hofmann2023-kz, Chiu2023-mg}, asteroid samples~\cite{Nakamura2023-ge, Ninomiya2023}, lithium batteries~\cite{Umegaki2020-xi}, and archaeological samples~\cite{Ninomiya2015-jh, Hampshire2019-tc, Shimada-Takaura2021-ae, Biswas2023-xz}. 

%non-destructive, balk analysis, no dependence on Z (wide range)


Muonic X-rays are also used as a method to measure the nuclear charge radius and distributions~\cite{Fitch1953-et, Fricke1995-cm, Angeli2013-to, Saito2022-wl, Antognini2020-zf}. 
Because the muon has a large mass and the atomic radius of the muonic atom is considerably smaller than that of an ordinary electron, the binding energy of the muonic atom is highly sensitive to the charge distribution of the nuclei. 
To measure the charge radius using muonic X-ray spectroscopy, the accuracy of the measured energy is essential.
The energies of the Lyman series of the muonic transition range from a few tens keV to 6 MeV depending on the atomic number.
To conduct charge radius measurements using muonic X-ray spectroscopy, a highly accurate X-ray measurement system that can apply to a wide energy range is required. 

%The muonic X-ray measurements are usually conducted using Ge detectors with traditional analog DAQ system and the ~~ of ???.
Therefore, in this study, we developed a wide-energy-range photon detection system for MIXE and nuclear charge radius measurements. 
%based on Ge detectors
%specialized for muonic X-ray spectroscopy at continuous muon beam facilities
%can be performed efficiently
The detection system is designed for performing muonic X-ray spectroscopy at continuous muon beam facilities such as the Research Center for Nuclear Physics (RCNP), Paul Scherrer Institute (PSI), and Canada's Particle Accelerator Centre (TRIUMF).
High-purity germanium (Ge) detectors are used for high energy resolution X-ray spectroscopy. % as photon detectors. 
%The detection of muonic X-rays from light elements is limited by the signal-to-noise (S/N) ratio in low-energy regions.
%To reduce the background noise, Compton suppressors using bismuth germanium oxide (BGO) scintillators were adopted.
Compton suppressors using bismuth germanium oxide (BGO) scintillators were adopted to reduce the background noise because detecting muonic X-rays from light elements is limited by the signal-to-noise (S/N) ratio in low-energy regions.
The data acquisition was constructed using a waveform digitizer in the system to accommodate for high count-rate measurements at continuous muon beam facilities. 
A Monte Carlo simulation was performed to design the experimental setup and optimize the detector geometry. 
The $^{27}\mathrm{Al}(p,\gamma){}^{28}\mathrm{Si}$ resonance reaction was used to evaluate the response of the photon detector for high-energy muonic X-rays emitted from heavy elements~\cite{Mizuno2023-px}. 
As the calibration method using the resonance reaction is not applicable at muon facilities, a new calibration method using muonic X-rays from $^{197}$Au and $^{209}$Bi as energy and intensity references was proposed. 
%This study makes a novel contribution to the literature by developing a confirmed calibration method of wide-energy-range photon detection with digitizer.


The remainder of this paper is organized as follows: 
The design concept of the detection system is described in Sect.~\ref{sec:Methods_1}. 
The methods used for evaluating the detector system are presented in Sect.~\ref{sec:Methods_2} and Sect.~\ref{sec:Methods_3}. 
The performance evaluation of the detector in terms of linearity, energy resolution, timing resolution, photopeak efficiency, and Compton suppression is presented in Sect.~\ref{sec:linearity}--\ref{sec:compton}.
The measurement results of muonic X-rays from a meteorite, $^{197}$Au, $^{208}$Pb, $^{209}$Bi are discussed in Sect.~\ref{sec:MIXE}. 
Finally, the conclusions of the present study are presented in Sect.~\ref{sec:summary}.
