\subsection{Timing resolution} \label{sec:timing}
In the muonic X-ray and $\gamma$-ray spectroscopy, two subsequent events occur: muon atomic transition and nuclear muon capture~\cite{Measday2001-mw}.
These events can be distinguished by their different time scales. The muonic X rays are emitted within a few ps after muon injection, whereas the de-excitation $\gamma$-rays from the nuclear muon capture reaction are emitted in a longer time scale, typically 2 $\mu$s in the light nuclei and approximately 70 ns in the heavy nuclei.
The typical timing spectrum of the photon (X rays and $\gamma$ rays) detection is shown, for example, in Fig.~3 of Ref.~\cite{Saito2022-wl}.
%The timing resolution above about 10--20 ns is needed to distinguish these events, especially for heavy elements.
The Ge detector usually has a timing resolution of approximately a few tens of ns. The timing resolution obtained using the digitizer was studied to achieve the intrinsic timing resolution of the Ge detectors.

The coincidence measurement was performed using a BaF$_2$ detector and Ge detectors using a standard $\gamma$-ray source, $^{152}$Eu.
The BaF$_2$ detector was adopted because it has a better timing resolution than the Ge detectors, and the timing resolution of the BaF$_2$ is negligible.
The timing resolution of BaF$_2$ was 4.92$\pm$0.20 ns, as deduced from a coincidence measurement using two BaF$_2$ detectors by assuming that the timing resolution of two detectors is the same.
This value is relatively worse compared with the same detector's timing resolution obtained using $^{60}$Co, 0.72$\pm$0.03 ns, 
because $\gamma$-rays from $^{152}$Sm (daughter nucleus of the $\beta^+$ decay from $^{152}$Eu) have an isomeric state with a 1.4-ns half-life. The timing resolution of 4.92$\pm$0.20 ns is sufficiently smaller enough than the Ge detector's timing resolution.
%The timing resolution was deduced in each $\gamma$-ray peak of the Ge detectors and the sigma of the Gaussian fitting of time gap spectra.
The timing resolution was evaluated for the measured $\gamma$-ray peaks by the Gaussian fitting of the timing-difference spectra between the Ge detectors and the BaF$_2$ detector.

Five timing pick-off methods were compared: RC-CR$^2$, constant fraction discriminator (CFD), amplitude and risetime compensated timing (ARC), leading edge timing (LET), and waveform analysis.
RC-CR$^2$ is a widely used timing pick-off method in digital processing because
it can obtain the timing for almost all input signals above the threshold and is easy to be implemented in FPGA~\cite{Guzik2013-zv}.
CFD, ARC, and LET are typical timing pick-off methods usually used in the analog DAQ systems of the Ge detectors.
CFD and ARC can obtain the timing independently with the pulse height, whereas LET shows the energy dependence, called ``time walk''. 
%LET is the most simple method. 
The difference between the CFD and the ARC is the delay time in their algorisms; delay times of 150 ns and 50 ns for the CFD and ARC, respectively, were used in this study. 
RC-CR$^2$ was implemented in the DPP-PHA firmware, and the CFD, ARC, and LET were obtained using the DPP-PSD firmware. 
%The waveform analysis is defined as the same concept as CFD, 
%the time when the pulse height becomes the small fraction (GC3018: 1/100, GX5019: 1/300, BE2820: 1/500) of the full pulse height is picked off with offline waveform analysis.
The waveform obtained using the flash ADC was analyzed offline and the pulse arrival timing was evaluated, which is hereafter referred to as the waveform analysis method.
%The concept of the offline waveform analysis is similar to the CFD and
In the offline waveform analysis method, 
the time when the pulse height becomes a small fraction (GC3018: 1/100, GX5019: 1/300, BE2820: 1/500) of the full pulse height is picked off.
The waveform analysis result is used as the reference value over the results obtained using other FPGA-based methods because the waveform analysis can obtain a timing resolution close to the intrinsic timing resolution of the Ge detectors. 
The waveform analysis is only applicable in low-count-rate measurements with the present system because this method requires large data transfers.

The timing resolutions of GX5019 obtained using the five timing pick-off methods are shown in Fig.~\ref{fig:Tresolution}.
RC-CR$^2$ and CFD showed worse timing resolutions in wide energy regions, and ARC and LET showed a timing resolution of approximately 10--20 ns in energy regions above 400 keV.
LET showed a timing resolution almost similar to that obtained using waveform analysis; however, the timing resolution of LET was worse in energy region below 100 keV.
%Note that the result of ARC is not shown in Fig.~\ref{fig:Tresolution} in the low-energy region
%because the efficiency taken with ARC becomes lower as shown in Fig.~\ref{fig:ARC_efficiency}. 

The ratios of the lost pulses in each timing pick-off method were also compared. 
The numbers of appropriately processed pulses obtained using the CFD, ARC, LET, and waveform analysis are shown in Fig.~\ref{fig:ARC_efficiency}, which are normalized with those obtained using RC-CR$^2$. 
CFD, LET, and waveform analysis results were almost the same as those obtained using RC-CR$^{2}$, whereas ARC lost some parts of the signals, especially in the low-energy region.
%The lost pulses in the ARC method become significant especially in the low-energy region.
Thus, the timing resolution of ARC is not shown in Fig.~\ref{fig:Tresolution} for low-energy regions. 
%The timing resolution of ARC is not shown in Fig.~\ref{fig:Tresolution} in the low-energy region because the efficinecy taken with ARC becomes lower.
ARC needs sufficient pulse height before the delay time to be distinguished from noise. 

\begin{figure}
  \centering
  %\includegraphics[scale=0.26]{Chap3_fig/Timing_resolution_GX5019_subtractBaF2.pdf}
  \includegraphics[scale=0.26]{Timing_resolution_GX5019_subtractBaF2.pdf}
  \caption{Timing resolution of GX5019 obtained using CR-RC$^2$, CFD, ARC, LET, and waveform analysis.
  The fraction and delay time of CFD and ARC are 50\%-150 ns and 50\%-50 ns, respectively.}
  \label{fig:Tresolution}
\end{figure}

\begin{figure}
  \centering
  %\includegraphics[scale=0.28]{Chap3_fig/Timing_efficiency_GX5019.pdf}
  \includegraphics[scale=0.28]{Timing_efficiency_GX5019.pdf}
  \caption{Number of processed pulses obtained using the following timing pick-off methods: CFD, ARC, LET, and waveform analysis, which were normalized with those obtained using RC-CR$^2$.}
  \label{fig:ARC_efficiency}
\end{figure}

The differences in timing resolutions and the numbers of processed pulses among the timing pick-off methods primarily originated from 
the differences in the rise time and shape of the waveform, as shown in Fig.~\ref{fig:rise_shape}, which are the typical waveforms of BE2820.
The difference in the interaction position of the photon inside the Ge crystal leads to a difference in the rise time and shape of the waveform. 
The performance of the preamplifier characterizes the effect of position dispersive on the difference in the rising shape. 
%How the position dispersion affects the shape of the waveforms is characterized by the performance of the pre-amplifier. 
As RC-CR$^2$ and CFD obtain the timing of the constant ratio of the pulse height, their performances are strongly affected by the difference in rise times and rising waveforms, which result in a worse timing resolution.

\begin{figure}
  \centering
  %\includegraphics[scale=0.4]{Chap3_fig/rise_shape_BE2820.pdf}
  \includegraphics[scale=0.4]{rise_shape_BE2820.pdf}
  \caption{Typical waveforms with the same pulse height and 
  aligned with the time calculated using the waveform analysis of BE2820.}
  \label{fig:rise_shape}
\end{figure}

Based on these results, we showed that LET can be used to determine the best timing resolution, which is almost similar to the timing resolution obtained using the waveform analysis in energy regions of approximately more than 200 keV. 
%the intrinsic timing resolution of Ge detectors above around 200 keV.
When a higher timing resolution is required in the low-energy region below 200 keV, the offline waveform analysis should be used for the timing pick-off method. 
Note that, the correction of the time walk is necessary when timing information obtained with LET is compared with that of other detectors, such as coincidence analyses.
