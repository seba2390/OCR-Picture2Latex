\subsection{Compton suppressor}\label{sec:compton}

Figures~\ref{fig:Compton_suppressed_lowE} and \ref{fig:Compton_suppressed_highE} show the energy spectra of the Ge detector with and without BGO Compton suppression analysis obtained using BE2820.
The blue (above) line represents the original spectrum, and the red (below) line represents the spectrum obtained by taking anti-coincidence with BGO detectors.
The Compton component is suppressed by approximately 50--60\% below the Compton edge 
by maintaining the full energy peak counts of more than 95\% in all the energy range below 11 MeV. 
The peak count of SE and DE were also reduced as shown in Fig.~\ref{fig:Compton_suppressed_highE}. 
The spectrum around the Compton edge is not reduced by the Compton suppressors 
because the BGO crystals are placed to suppress the forward scattering to reduce the Compton component in the relatively low-energy regions.

%\begin{figure}
%  \centering
%  \includegraphics[scale=0.6]{Chap3_fig/Compton_suppress.pdf}
%  \caption{Energy spectra with and without Compton suppression analysis of 
%  $^{60}$Co (a) and the high energy $\gamma$-ray measurements (b).
%  The top of the peaks are omitted in (b).}
%  \label{fig:Compton_suppressed}
%\end{figure}

\begin{figure}
  \centering
  %\includegraphics[scale=0.43]{Chap3_fig/Compton_suppressed_lowE.pdf}
  \includegraphics[scale=0.43]{Compton_suppressed_lowE.pdf}
  \caption{Energy spectra with (the blue line) and without (the red line) Compton suppression analysis of $^{60}$Co. Black lines represent the spectra calculated using GEANT4.}
  \label{fig:Compton_suppressed_lowE}
\end{figure}
\begin{figure}
  \centering
  %\includegraphics[scale=0.43]{Chap3_fig/Compton_suppress_highE.pdf}
  \includegraphics[scale=0.43]{Compton_suppress_highE.pdf}
  \caption{Energy spectra with (the blue line) and without (the red line) Compton suppression analysis of the high energy $\gamma$-ray measurements using $^{27}\mathrm{Al}(p,\gamma){}^{28}\mathrm{Si}$ reaction. 
  The top of the peaks are omitted.}
  \label{fig:Compton_suppressed_highE}
\end{figure}

The performance of the Compton suppressor is also simulated in GEANT4.
The black line in Fig.~\ref{fig:Compton_suppressed_lowE} shows the result of the GEANT4 simulation 
normalized by the peak count with the measurements, which reproduced the measured spectrum 
of the standard source well.
Thus, GEANT4 simulation can be used for designing the geometries of the detection system 
to determine the performance of Compton suppression around the objective peak of each measurement.
