\subsection{The photopeak efficiency}
\label{sec:efficiency}

The photopeak efficiency of the Ge detectors was measured with standard $\gamma$-ray sources and in-beam $\gamma$-ray measurements using the ${}^{27}\mathrm{Al}(p,\gamma){}^{28}\mathrm{Si}$ reaction.
Figure~\ref{fig:Efficiency_measure} shows the photopeak efficiencies of each Ge detector placed 10 cm apart from the sources/target.
The efficiencies of GC3018 and BE2820 were measured using $^{60}$Co, $^{133}$Ba, $^{137}$Cs, and $^{152}$Eu.
The efficiency of GX5019 is the result of standard $\gamma$-ray measurements and in-beam $\gamma$-ray measurements. % and smoothed with low energy region below 2 MeV.
Two functions,
\begin{align}
  \label{eq:lowE_eff}
  \epsilon =& a_0E^{-a_1}-a_2\mathrm{exp}(-a_3E) \\
  \label{eq:highE_eff}
  \epsilon =& a_0E^{-a_1}-a_2\mathrm{exp}(-a_3E)-a_4\frac{(E-a_5)}{\sqrt{a_6+(E-a_5)^2}}
  %\epsilon =& (a_0+a_1 E)\left[\frac{1}{2}\mbox{Erfc}\left(\frac{E-a_4}{a_5}\right)\right] ~~~~\\
  %&+(a_2+a_3 E)\left[\frac{1}{2}\mbox{Erf}\left(\frac{E-a_4}{a_5}\right)+\frac{1}{2}\right],
  %\epsilon =& a_0(E^{-a_1}-a_3\textrm{exp}(-a_4 E))\left[\frac{1}{2}\mbox{Erfc}\left(\frac{E-a_5}{a_6}\right)\right] \\
  %&+a_7 E^{-a_8}\left[\frac{1}{2}\mbox{Erf}\left(\frac{E-a_5}{a_6}\right)+\frac{1}{2}\right],
\end{align}
were used as fitting functions in the figure. Eq.~(\ref{eq:lowE_eff}) represents the efficiency curve for GC3018 and BE2820 and
Eq.~(\ref{eq:highE_eff}) represents the efficiency curve for GX5019, where $\epsilon$ is the efficiency and $a_0$--$a_6$ are the fitting parameters. 
%%$a_2$ is fixed at $a_2=a_0+(a_1-a_3)a_4$ to connect the function at $E=a_4$.
%; $a_7$ is fixed at $a_7=a_0(a_5^{-a_1}-a_3\textrm{exp}(-a_4 a_5))/a_5^{-a_8}$ to connect the function at $E=a_5$
As shown in Fig.~\ref{fig:Efficiency_measure}, the efficiency curve of GX5019 shows a sharp turn at approximately 3 MeV.
%The efficiency in the high-energy region was deduced from the measurement without using the extrapolation from below 1.5 MeV.
Therefore, the measured efficiency value differed from the value deduced from extrapolation using Eq.~(\ref{eq:lowE_eff}) in energy regions below 1.5 MeV, as shown in the dotted line in Fig.~\ref{fig:Efficiency_measure}. 
%can be incorrect for the high-energy region.
%reference::Maccallum(1975)
Similar turns at approximately 2--3 MeV of the photo-peak efficiency are previously reported~\cite{Singh1971-as,McCallum1975-fy,Molnar2002-le,Elekes2003-db} and explained using a model, including pair production and multiple processes~\cite{Hajnal1974-ry}.

\begin{figure}
  \centering
  %\includegraphics[scale=0.35]{Chap3_fig/Efficiency_all.pdf}
  \includegraphics[scale=0.35]{Efficiency_all.pdf}
  \caption{Photopeak efficiencies of GC3018, GX5019 and BE2820. 
  The results of GC3018 and BE2820 are measured using $^{60}$Co, $^{133}$Ba, and $^{152}$Eu.
  The result of GX5019 is the result of high-energy $\gamma$-ray measurement with the p+Al reaction and smoothed in the low-energy region. The distance of the Ge detectors from the target was 10 cm. The dotted line shows the extrapolation line from the energy region below 1.5 MeV for GX5019, obtained using Eq.~(\ref{eq:lowE_eff}).}
  \label{fig:Efficiency_measure}
\end{figure}

The GEANT4 simulation was performed for the reproduction of the efficiency curve.
Figure~\ref{fig:Efficiency_g4} shows the result of measurements and the GEANT4 simulation of GX5019 for the full energy, single escape (SE), and double escape (DE) peaks. 
The GEANT4 simulation overestimated the efficiencies because of the inaccuracy of the insensitive volume in the Ge crystal and the uncertainty of the charge collection~\cite{Hurtado2004-lz,CebastienJoel2018-tr}. 
This overestimation is independent of the energy and the distance between the target and the detector and can be corrected with a constant factor.
%, so the efficiency taken by the GEANT4 simulation requires the correction. 
The measured efficiencies were reproduced by the results obtained using GEANT4 divided by the correction factors: 1.172(35) for GC3018, 1.071(32) for GX5019, and 1.086(33) for BE2820.
The corrected efficiency with constant values obtained using GEANT4 reproduced the measured values well, including the SE and DE peaks, as shown in Fig.~\ref{fig:Efficiency_g4}.
The efficiency kink observed in this study can be understood based on the escape of the energy accompanied by the pair creation and multiple processes.
%, and it well reproduced the measurement value. 


\begin{figure}
  \centering
  %\includegraphics[scale=0.35]{Chap3_fig/efficiency_g4_all_rere.pdf}
  \includegraphics[scale=0.35]{efficiency_g4_all_rere.pdf}
  \caption{Efficiency of GX5019 obtained using the measurement and corrected GEANT4 simulation.
  The green and yellow points represent the efficiencies of the SE and DE peaks (circle: measurement; line: GEANT4).}
  \label{fig:Efficiency_g4}
\end{figure}

%The photopeak efficiency curves for all Ge detectors were obtained and reproduced well by the GEANT4 simulation.
The measured efficiency curve suggests that the extrapolation from the measured values using standard sources below 2 MeV is insufficient. 
Therefore, we proposed the use of muonic X-rays from Au and Bi as efficiency standards in high-energy regions at the muon facility, as explained in Sect.~\ref{sec:MIXE}.
%The GEANT4 simulation with the corrected with a constant factor reproduces the efficiency curve well for each detector.