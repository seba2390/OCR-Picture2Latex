%\section{Performance of the detectors}
%\label{sec:Performance}
\subsection{Energy accuracy}\label{sec:linearity}
% In the measurement of nuclear charge radius, energy accuracy is important.
The energy accuracy of the detector system is critical for the measurement of nuclear charge radii as the accuracy of the muonic X-ray energy measurement is directly reflected in that of the nuclear charge radii.
For example, 0.2-keV accuracy on the X-ray energy corresponds to 0.66\% accuracy of the charge radius in $^{108}$Pd~\cite{Saito2022-wl} and 0.04\% in $^{208}$Pb~\cite{Bergem1988-nf}, respectively.
%Therefore, the linearity of the systems and t
The energy accuracy is investigated in this section.
% without using any muonic X-rays.

The statistical uncertainty is usually very small in the muonic X-ray spectroscopy and the energy accuracy is primarily limited by the systematic uncertainty in the energy calibration, namely, the nonlinearity of the detector system.
The linearities of the waveform digitizer, preamplifier, and Ge crystals were independently evaluated using a pulse generator (Ortec 448) and $\gamma$-ray measurements in the range of 50 keV to 10 MeV.
First, the nonlinearity of the digitizer was examined using the pulse generator.
The pulse signals were inputted into the digitizer with the DPP-PHA firmware, and the heights of the pulse generator were changed in 10 steps.
The nonlinearity of the digitizer was found in the residuals from the linear calibration function, as shown in Fig.~\ref{fig:linearity}(a).
The nonlinearity of the digitizer was corrected by a third-order polynomial function.
The constant term of the calibration function was fixed to zero because offset cancellation is included in digital pulse processing.
The residuals from the calibration curve became sufficiently smaller than the channel width of the digitizer.
Second, the nonlinearities of the pre-amplifiers of each Ge detector were evaluated by introducing a pulse generator signal into the test input of each preamplifier built in the Ge detectors. 
Figure~\ref{fig:linearity}(b) shows the residuals from the linear calibration function for the preamplifier of GX5019 after excluding the nonlinearity of the digitizer, which shows a higher-order trend.
Other detectors of GC3018 and BE2820 showed the same trends in the nonlinearity, and a third-order polynomial for all three detectors was used for the correction function to eliminate the nonlinearity.
Finally, the nonlinearity of the Ge crystals was measured with standard $\gamma$-ray sources.
For GX5019, the in-beam measurement of the $^{27}\mathrm{Al}(p,\gamma){}^{28}\mathrm{Si}$ reaction was also used to evaluate the nonlinearity of high-energy $\gamma$-rays.
Figure~\ref{fig:linearity}(c) shows the residuals from the linear calibration function for the Ge crystal of GX5019 after excluding the nonlinearity of the digitizer and preamplifier~\cite{Mizuno2023-px}.
The figure shows that the Ge crystal has a different trend below and above approximately 2000 channels, which corresponds to approximately 3 MeV. 
The calibration function with the combination of a linear function for the low-energy region and a quadratic function for the high-energy region was used.
GC3018 shows nonlinearity below 1.4 MeV and needs a second-order polynomial function for nonlinearity correction.
The Ge crystal of BE2820 showed negligible non-linearity below 1.4 MeV.
Figure~\ref{fig:linearity}(d) shows the residuals from the calibration using all correction functions for GX5019. %Eq.~(\ref{eq:calib_dig})--(\ref{eq:calib_linear-linear}).
The nonlinearity was eliminated in all energy regions below 10 MeV, and the residuals remained within one ADC channel. %corresponding to 0.5 keV.


\begin{figure*}
  \centering
  %\includegraphics[width=1.1\textwidth,scale=0.4]{Chap3_fig/linearity_beforeandafter_correction_GX5019_bigger_re.pdf}
  \includegraphics[width=1.1\textwidth,scale=0.4]{linearity_beforeandafter_correction_GX5019_bigger_re.pdf}
  %\includegraphics[scale=0.5]{Chap3_fig/linearity_before_correction.pdf}
  \caption{Residuals from the linear calibration function for (a) the digitizer (V1730B with a 2.0-V range), (b) the preamplifier of GX5019, (c) the Ge crystal of GX5019, and (d) after nonlinearity correction~\cite{Mizuno2023-px}.}
  \label{fig:linearity}
\end{figure*}



% ================ Wide table =========================
\begin{table*}[width=.75\textwidth, cols=4,pos=h]
\centering
\caption{Energy accuracy after nonlinearity correction for the three Ge detectors.
        The input dynamic range of the waveform digitizer can be selected at 0.5 and 2.0 V and the corresponding energy ranges for each detector are listed in the table.
        The result of BE2820 is only for the 0.5-V range because BE2820 is only used in low-energy regions below 2 MeV.
        The gain of the detectors is set to the maximum ($\times 10$) and the DC offset is fixed at 10\% (see Sect.~\ref{sec:Eresolution}).}
\label{tab:energy_accuracy}
\begin{tabular*}{\tblwidth}{@{}c cc cc cc@{}}
% \begin{tabular}{ccccccc}
    \toprule
    \multirow{2}{*}{Input range} & \multicolumn{2}{c}{GC3018} & \multicolumn{2}{c}{GX5019} & \multicolumn{2}{c}{BE2820} \\
                & Range & Accuracy & Range & Accuracy & Range & Accuracy \\
    \midrule
    0.5 V & 2.35 MeV & 0.05 keV & 2.43 MeV & 0.06 keV & 1.55 MeV & 0.03 keV\\
    2.0 V & 9.42 MeV & 0.2  keV & 9.70 MeV & 0.3  keV & 6.22 MeV & -    \\
    \bottomrule
\end{tabular*}
\end{table*}
% ================ END of wide table ==================


The selection of the dynamic range limits the energy accuracy after the correction of nonlinearity.
The input dynamic range of the digitizer can be changed to 0.5 V and 2.0 V.
The accuracy of the energy is deduced from the residuals from the calibration functions for each input dynamic range.
Table~\ref{tab:energy_accuracy} shows the accuracy of the three Ge detectors and the corresponding dynamic energy range. 
When the dynamic range becomes wider, the energy accuracy becomes worse. 
%An energy accuracy smaller than 0.1 keV can be obtained. %below 1.5 MeV.


The correction of the gain drift and the pile-up effect under the high-count-rate condition is also essential for the high accuracy throughout the spectroscopy experiment when the measurement time is longer, typically more than 1 hour.
These long-term effects were phenomenologically corrected using one or two $\gamma$-ray peaks with a linear function as follows:
\begin{equation}
  \label{eq:gain_drift}
  E = A E_\mathrm{original} + N_p,
\end{equation}
where $A$ is the gain drift parameter, $E_\mathrm{original}$ is the energy before correction, and $N_p$ is the pile-up correction term.
The gain drift can be expressed by the first term %considered not depending on the energy,
because digital pulse processing with baseline correction does not need to consider offset. 
$N_p$ is the pile-up correction term and is 0 in low-count-rate measurements below 1~kHz, 
%The pile-up effect is also essential when the counting rate becomes high,
explained in Sect.~\ref{sec:highcountrate}. 
%$N_p$ is 0 in low count rate measurements like 1~kHz. 
%In this performance study, the gain drift is corrected during the $^{27}Al(p,\gamma){}^{28}Si$ measurement. %using known energy peaks about every hour.

% summary
In summary, the correction of nonlinearity of the system and gain-drift are necessary to achieve high energy accuracy.
After the correction of nonlinearity and the long-term effect, appropriate selection of the dynamic range must be performed depending on the purpose of each measurement.
% The correction of non-linearity is necessary to obtain high energy accuracy.
An energy accuracy smaller than 0.1 keV can be obtained in energy regions below 2 MeV and 0.3 keV for high-energy photon spectroscopy.
