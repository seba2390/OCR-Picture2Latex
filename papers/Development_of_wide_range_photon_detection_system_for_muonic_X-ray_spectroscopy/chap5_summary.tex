
\section{Summary} \label{sec:summary}
In this study, we have developed a wide-energy-range photon detection system for muonic X-ray spectroscopy.
The detection system includes photon detectors consisting of high-purity Ge detectors and BGO Compton suppressors, a data acquisition system based on the waveform digitizer. 
%A Monte-Carlo simulation and 
A calibration method for the photon detector are also provided.

Three types of Ge detectors were used to cover a wide energy range.
The detector performance of the proposed system in the entire energy range below 10 MeV was evaluated through offline source measurements and in-beam experiments using the ${}^{27}\mathrm{Al}(p,\gamma)^{28}\mathrm{Si}$ resonance reaction.
Optimization of the parameter used in the digital waveform processing was essential to achieve the best performance of the system.
With a sufficient number of anchor points for energy calibration, an energy accuracy of 0.3 keV was achieved.
By selecting the appropriate timing pick-off method, a timing resolution ($\sigma$) of 10--20 ns was obtained.
The absolute efficiency was also determined with 3\% accuracy.
%Owing to the fast signal processing of the digitizer, 
The detector performances under high count-rate conditions of up to 10~kHz were investigated.
The Monte-Carlo based simulation using the GEANT4 was performed and the obtained results were in agreement with the measurement results. 
This indicates that configurations of each spectroscopy setup can be optimized using GEANT4.

The performance of the system was demonstrated at PSI.
The performance of BGO Compton suppressors was evaluated through meteorite measurements conducted at PSI by improving the S/N ratio by a factor of 1.9 and 3.9, depending on the energy region. 
The carbon component of the Allende meteorite with 0.3~wt\% could be detected. 
We proposed a calibration method for the photon detector at the muon facility using muonic X-rays emitted from Au and Bi.
%X-ray energy and intensity are obtained.
The energy and intensity references proposed in this study provide a method for performing general calibrations at the muon facility.


% The energy accuracy , timing resolution, detection efficiency and high count rate capability are investigated to be 0.3 keV, 10--20 ns, 3\% and 10kHz, respectively.
%, and BGO scintillators were used as the Compton suppressor. 
% Digitizer
% Calibration method


\section*{Acknowledgements}
This work was supported by the Japan Society for the Promotion of Science (JSPS) KAKENHI Grant Number 18H03739, 19H01357, 22H02107, 22K18735 and Swiss National Science Foundation Synergia project "Deep$\mu$" Grant Number 193691. R. M. is supported by the Fore-front Physics and Mathematics Program to Drive Transformation (FoPM), a World-leading Innovative Graduate Study (WINGS) Program, and the JSR Fellowship from the University of Tokyo. 
This work was partly supported by the RCNP Collaboration Research Network program with project number COREnet-41.
We thank the SUNFLOWER collaboration for providing the Au and Bi targets. 
A part of the experiment was performed at the Pelletron facility (joint-use equipment) at the Wako Campus, RIKEN.
