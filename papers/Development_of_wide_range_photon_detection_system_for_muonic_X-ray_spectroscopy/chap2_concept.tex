%\section{Methods} \label{sec:Methods}
\section{Design of the detection system} \label{sec:Methods_1}
%%%% Ge Detectors %%%%

For muonic X-ray spectroscopy, a large dynamic range and high energy resolution are required.
%Furthermore, 
Ge detector arrays~\cite{Gerchow2023-pp, Tampo2023-qf, Hillier2016-xa} are generally used in the muonic X-ray detection; however, in recent years, cryogenic quantum sensors ~\cite{Ullom2015-yb} and CdTe detectors~\cite{Chiu2022-me} are also used. % in these days. % for light elements lower than nitrogen. 
Ge detectors show a typical energy dynamic range from 1 keV to 10 MeV with an energy resolution of $\Delta E/E\sim0.1\%$. In contrast, cryogenic quantum sensors show approximately 1--10 keV dynamic range and 0.03\% energy resolution, and CdTe detectors show approximately 10--150 keV dynamic range and 1.2\% energy resolution.
%These days, Ge detector array~\cite{Gerchow2023-pp, Tampo2023, Hiller}, TES detector~\cite{} and CdTe detector~\cite{Chiu2022-me} are used for MIXE measurements. % for light elements lower than nitrogen. 
The energies of a muonic $K_\alpha$ X-ray of low atomic number ($Z$) elements, such as carbon, nitrogen and oxygen, are below 150 keV, whereas that of large $Z$ elements, such as the actinides, are above 6 MeV. %tens of keV
To detect the $K_\alpha$ X-rays for high $Z$ elements for nuclear charge radii measurements, high energy resolution in a wide energy region is required; thus, we adopted the Ge detector as a photon detector.

%Since it is difficult to achieve both a large dynamic range and high energy resolution in a low-energy photon with a single type of Ge detector, several kinds types of Ge detectors are combined into a detector array.
As achieving a wide dynamic range and high energy resolution in a low-energy photon with a single type of Ge detector is difficult, various types of Ge detectors are combined to form a detector array.
A planar-type Ge detector with a thin entrance window is used for the low-energy region, and a large-volume coaxial Ge detector with sufficient efficiency is used for the high-energy region. %at the energy of a few MeV.
For this reason, three types of Ge detectors, namely, Canberra BE2820, GX5019, and GC3018, were selected in this study.
Table~\ref{tab:Ge-character} shows the characteristics of each detector.
BE2820 is a broad-energy Ge detector consisting of a planar-shaped crystal with point contact electrode having the best energy resolution. %, especially in the low-energy photon.
BE2820 covers the energy range from 3 keV to 3 MeV owing to its thin carbon composite window and thin front end of the crystal. 
GX5019 is a p-type extended-range coaxial Ge detector with a large crystal volume and a thin-window contact on the front surface and shows relatively large efficiency ranging from 3 keV to more than 10 MeV.
GC3018 is a standard p-type coaxial detector %having 30\% relative efficiency 
and is used as a reference detector to BE2820 and GX5019 because GC3018 shows Ge detector's typical performances.
The relative efficiencies of BE2820, GX5019, and GC3018 are 13, 50, and 30\%, respectively. 
All the preamplifiers built-in the Ge detectors are the resistive feedback type.
The preamplifiers of BE2820 and GX5019 are intelligent preamplifier (iPA)-SL10, which can change the output gain in times 1, 2, 5, and 10 through a control system via a USB interface. 
The preamplifier of GC3018 can also change the output gain in times 1 and 5.
Notably %the end-cap diameters of all the Ge detectors are uniform to fit the existing BGO Compton suppressors explained below. 
the end-cap diameters of BE2820 and GX5019 are modified to 76 mm, which is smaller than their original catalog sizes and is the same diameter as that of GC3018 such that all detectors accommodate the same geometry of the BGO Compton suppressors explained below. 
%BE2820 and GX5019 are adopted to get maximum crystal size in this condition; the end cap diameter is smaller than 76 mm. 
%These Ge detectors will consist spherical Ge array. 
In the present study, two BE2820, one GX5019, and two GC3018 were investigated. 


\begin{table}%[width=.75\textwidth, cols=4,pos=h]
%  \centering
  \caption{The characteristics of Ge detectors. All the detectors are manufactured by Canberra Inc.}
  \label{tab:Ge-character}
 \begin{tabular*}{\tblwidth}{@{} llll@{} }
  %\begin{tabular}{cccccc}\hline
  \toprule
  Ge model         & GC3018      & GX5019     & BE2820 \\
  \midrule
  Crystal          & coaxial     & coaxial    & point contact planar \\
  Type             & P-type      & P-type     & N-type \\
  PreAmp           & 2002CSI     & iPA-SL10   & iPA-SL10 \\
  Rel. efficiency  & 30\%        & 50\%       & 13\% \\ 
  Window           & 1.6 mm (Al) & 0.6 mm (C) & 0.6 mm (C) \\
  Serial number    & 9681, 9682  & 5536       & 13385, 13386\\
  \bottomrule
  \end{tabular*}
\end{table}

%%% Compton suppressor %%%
In the MIXE experiments, especially to obtain a low composite from low $Z$ materials, the signal-to-noise (S/N) ratio needs to be small. 
In some MIXE experiments for earth and planetary materials, the low $Z$ element is measured in the presence of higher Z materials contamination; 
%with the contamination of higher $Z$ materials
for example, carbon, nitrogen, and oxygen are measured with the background of silicon and iron. 
In such cases, the lowest detectable composition of lower $Z$ materials is limited to 1 wt\% because the K$_\alpha$ peaks of lower $Z$ elements are on the large Compton backgrounds of muonic X-rays from higher $Z$ elements~\cite{Terada2014-cw}. 
%This restriction comes from S/N ratio of low energy peaks. 
Therefore, a system to reduce the background component and improve the S/N ratio in low energy regions is required to identify the low composite from low $Z$ elements. %, for example, 0.5 wt\% composition of carbon in Allende meteorite~\cite{Neuland2021}.
The Compton suppression technique helps to reduce the background in the low-energy region.
%An anti-coincidence measurement with BGO scintillators surrounding the Ge detector can reduce the background in the low energy region.
%BGO crystals surround a Ge detector and detect photons scattered in the Ge crystal to reduce the main background in the low-energy regions with anti-coincidence measurement. 
Furthermore, the Compton suppressor helps to avoid cross-talk events. 
Note that thinner Ge detectors are also effective in reducing the Compton backgrounds in low-energy regions owing to low efficiency for high-energy photons. However, the detection efficiency is also reduced with thinner Ge detectors, which have smaller detection areas. The Compton suppressors with BE2820, GX5019, and GC3018 are adopted in this study because they provide both the reasonable detection efficiency and the better S/N ratio simultaneously.


BGO crystals surrounding the Ge detector were used for the Compton suppressor, as shown in Fig.~\ref{fig:Ge_BGO}. 
%BGO crystals surround a Ge detector and detect photons scattered in the Ge crystal. 
This Compton suppressor was originally developed for nuclear structure studies at the University of Tsukuba~\cite{BGO_tsukuba1, BGO_tsukuba2, BGO_tsukuba3}, and the detector geometry was modified to fit our Ge detectors in this study.
Five separated BGO crystals making a decagonal shape with a center hole of 80-mm in diameter were used, and %and divided into five optically separated.
the thickness of each crystal was approximately 22 mm and 44 mm for the front and back sides, respectively. 
%Front BGO crystals originally put in front of the crystal are removed. 
The BGO crystals were placed %relatively forward from the Ge crystal
such that they suppress forward scattered photons at the Ge crystal, which corresponds to the low-energy Compton region of the energy spectrum. 
A lead ring with 5-cm in length was placed in front of the BGO crystals %, where the originally front BGO crystals are, 
to avoid the direct incidence of photons and particles from the target on the BGO crystals.
%Copper and tin shields are surrounded 
Cylindrical copper and tin plates were inserted on the inner surface of the lead ring to absorb electric X-rays originating from lead and tin with energies at 73, 75 keV, and 25, 28 keV, respectively.
The optical readout of the BGO scintillator was performed using photomultiplier tubes (PMT).
%PMTs with the model number 
R6231 manufactured by Hamamatsu Photonics K. K. was used to achieve high sensitivity at 480 nm, which is the wavelength of the scintillation photon.
%are required to have high sensitivity at 480 nm to be used as the readout of BGO scintillators. 
% to fit the shape of the Ge detectors.
%Figure~\ref{fig:Ge_BGO} shows the cross-sectional view and the picture of a Ge detector with Compton suppressors. 
The Ge position ($d$) is defined as the distance from the target/source to the front side of the window of the Ge detector.
%The photon detection system can include 10 Ge detectors with Compton suppressors. 


\begin{figure*}[width=0.85\textwidth, cols=4,pos=h]
  \centering
  %\includegraphics[bb=400 180 500 322,scale=0.6]{Chap2_fig/GeandBGO_schematics.pdf}
  %\includegraphics[scale=0.45]{Chap2_fig/GeandBGO_schematics.pdf}
  \includegraphics[scale=0.45]{GeandBGO_schematics.pdf}
  \caption{Cross-sectional view (a) and schematics (b) of the Ge detector with Compton suppressors. The point in figure~(a) shows the target or source position, and the Ge position ($d$) is defined as the distance from the target/source to the front side of the window of the Ge detector.}
  \label{fig:Ge_BGO}
\end{figure*}


%%%%%%%%%%%%%%%%%%%%%%%%%%%%%%%%%%%%%%%%%%%%%%%%%%%%%%%%%%%%%%%%%%
%\subsection{The Data Acquisition System}
A data acquisition (DAQ) system using waveform digitizers was adapted because a high counting rate is expected in the muon irradiation experiment at Paul Scherrer Institute (PSI) owing to its high-intensity continuous beam.
The advantages of the DAQ system using digitizers are: fewer limitations in parameter tuning for data acquisition,
reduction in the number of circuit modules, and lower dead time because of fast processing.
The reduction in the number of modules enables lower cost and higher stability due to temperature changes.

All the signals from the detectors were acquired using 500-MS/s 14-bit waveform digitizers, CAEN V1730B~\cite{V1730B}.
V1730B consists of a 16-channel flash ADC and FPGA and can acquire digital waveforms, energy, and timing from input signals using FPGA firmware.
The dynamic input range of V1730B is selectable, 0.5 V and 2.0 V.
%Two dynamic input range, 0.5 V and 2.0 V, are selectable with V1730B. 
Two firmware on V1730B are used in the DAQ systems: digital pulse processing for pulse height analysis (DPP-PHA) 
and DPP for charge integration and pulse shape discriminator (DPP-PSD)~\cite{DPP-PHAandPSD}.
DPP-PHA deduces the pulse height of the input signals using the trapezoidal filter and the timing using the RC-CR$^2$ method.
DPP-PSD obtains the charge integration and the timing using the constant fraction discriminator (CFD) or the leading-edge timing (LET) method.
This DAQ system was controlled using CAEN Multi-Parameter Spectroscopy Software (CoMPASS), a DAQ software developed by CAEN S.p.A.~\cite{CoMPASS}.

Figure~\ref{fig:DAQ} shows the schematics of the DAQ system using two V1730B boards.
The firmwares of the digitizer boards were DPP-PHA and DPP-PSD, respectively.
Two pre-amplifier signals from one Ge detector were put in the two digitizer boards:
one for energy information obtained using DPP-PHA and the other for timing information obtained using DPP-PSD.
This complexity is because the timing pick-off method of DPP-PHA firmware does not obtain sufficient timing resolution of the Ge detectors (see Sect.~\ref{sec:timing}). 
PMT signals from BGO scintillators were processed using DPP-PSD firmware to obtain energy and timing.
All signals from the Ge detectors were obtained and saved in the self-trigger mode, 
whereas BGO signals were obtained %with the self-trigger and saved 
only in coincidence with the Ge detector. 
The coincidence time window between the Ge and BGO detectors was set to -1.5 to 0.5 $\mu$s corresponding to sufficient time to obtain all the coincident signals. % because the BGO signal comes faster than Ge signals.

\begin{figure}
  \centering
  %\includegraphics[bb=200 540 250 650, scale=0.7]{Chap2_fig/DAQ_GeandBGO.pdf}
  %\includegraphics[scale=0.5]{Chap2_fig/DAQ_GeandBGO.pdf}
  \includegraphics[scale=0.5]{DAQ_GeandBGO.pdf}
   \caption{Schematics of the data acquisition system.
   The Ge energy was obtained using the DPP-PHA firmware, and the BGO energy and Ge timing were obtained using the DPP-PSD firmware.}
   \label{fig:DAQ}
\end{figure}