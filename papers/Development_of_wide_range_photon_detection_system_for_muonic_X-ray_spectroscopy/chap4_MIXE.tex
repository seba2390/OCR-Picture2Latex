\section{Muonic X-ray spectroscopy} \label{sec:MIXE}
%% Introduction: relative intensity of metal targets, BGO's performance

The performance of the detector system in the MIXE experiment was demonstrated at PSI.

The muonic X-rays of a typical carbonaceous meteorite, an Allende meteorite, were measured to evaluate the performance of BGO Compton suppressors in the MIXE experiments.
The Allende meteorite contains a low composition of carbon (0.3~wt\%) and a high composition of magnesium (15~wt\%), silicon (16~wt\%), and iron (24~wt\%)~\cite{Mason1975-ko, Stracke2012-vw}.
The energy of muonic K$_\alpha$ X-rays of carbon is 75 keV and those of magnesium, silicon, and iron are 297, 400, and 1255 keV, respectively. 
Therefore, the energy peaks of K$_\alpha$ X-rays emitted from carbon in the spectrum are on the top of the Compton component of heavier elements and are difficult to identify when the carbon composition is small~\cite{Terada2014-cw}.
However, the use of Compton suppressors improves the S/N ratio under such conditions.
Hence, the S/N ratio of muonic X-rays emitted from carbon %and sodium (0.34~wt\%) 
of the Allende meteorite was measured with and without Compton suppression.% to evaluate the effect.


Furthermore, the muonic X-ray spectroscopy for $^{197}$Au, $^{208}$Pb, and $^{209}$Bi was conducted to obtain the calibration reference data for high-energy photons. % at the muon facility. 
The energy and efficiency of the prototype detector of GX5019 were evaluated with standard gamma-ray sources and an in-beam measurement using the $^{27}\mathrm{Al}(p,\gamma){}^{28}\mathrm{Si}$ reaction, as explained in Sect.~\ref{sec:Performance}.
As the in-beam measurement for calibrating high-energy photons cannot be conducted at the muon beam facility, general references %photon sources 
for high-energy photon calibration are required for performing in-situ calibrations.
The muonic X-rays of the high $Z$ element can be used as the reference, as they have a high energy of approximately 6 MeV. 
$^{197}$Au and $^{209}$Bi are the best candidates for the reference target because they have the highest $Z$ among the stable nuclei and are enriched in natural composition. $^{208}$Pb is also a candidate for the reference target; however, the preparation of an enriched and flat-thickness target is expensive.


The measurements of muonic X-ray energies and intensities of $^{197}$Au, $^{208}$Pb, and $^{209}$Bi have been conducted in several studies.
The muonic X-ray energies of $^{208}$Pb have been intensively measured with high precision in the context of improving the model for deducing the nuclear charge radius~\cite{Anderson1966-ir, Anderson1969-fr, Powers1968-rq, Backenstoss1970-wj, Kessler1975-sw, Bergem1988-nf, Hoehn1984-op}, and the energy peaks of muonic X rays from $^{208}$Pb are often used as the energy reference at muon facilities. 
The relative intensities of each X-ray have also been measured~\cite{Anderson1969-fr}, which are used for efficiency calibration.
%abs. intensiryは206Pbのみ
For $^{197}$Au, measurements of the muonic X-ray energies~\cite{Acker1966-kh, Powers1974-aa, Measday2007-zh} and intensities~\cite{Hartmann1982-wi, Measday2007-zh} have been reported.
The measurement of X-ray energies from $^{209}$Bi are reported in Refs.~\cite{Acker1966-kh, Powers1968-rq, Bardin1967-py, Engfer1974-km, Schneuwly1972-cv, Measday2007-zh} and corresponding intensities are reported in \cite{Measday2007-zh}.
However, in these measurements, the efficiencies of the photon detectors were deduced by the extrapolation from the standard $\gamma$-ray sources with energies below approximately 1.5 MeV, and this method may overestimate the efficiencies in high-energy regions, as explained in Sect.~\ref{sec:efficiency}.
% and the energy calibration depended on one or two known background peak energies, such as $^{208}$Pb (2614.5 keV) and $^{16}$N (6129.1 keV).
In addition, the relative intensities of muonic X-rays from $^{197}$Au and $^{209}$Bi were reported with an assumption on the probability of nonradiative transitions.
Therefore, the energies and relative intensities of muonic X-rays (K, L, M, and N series) from $^{197}$Au, $^{208}$Pb, and $^{209}$Bi were measured to obtain accurate and useful data for photon detector calibration at muon facilities.

%% Experiments: PSI $\pi$E1 beam line, target
The experiment was performed at the $\pi$E1 beamline of the high-intensity proton accelerator (HIPA) at PSI~\cite{Kiselev2021-se} in Switzerland.
The typical muon rate was approximately 20~kHz at a momentum of 30 MeV/c for the meteorite target, and approximately 39~kHz at a momentum of 40 MeV/c for the metal targets.
The momentum bite of the beam ($\Delta p/p$) was 2\%. %, with about -- \% electrons contamination.
The beam spot was collimated to 18 mm in diameter.
Two Ge detectors, GX5019 and BE2820 with Compton suppressors, were set at an angle of 60 and $-$120 degrees with respect to the beamline and at a distance of 10 cm and 20.3 cm from the targets, respectively.
%These detectors are part of the Ge detector array, GIANT~\cite{Gerchow2023-pp}.
The DAQ system was the same as that explained in Sect.~3.

The Allende meteorite sample was shaped such that resembled approximately 4-cm square with 5-mm thickness and irradiated with a muon beam for 2.5 h. The sample was held using an aluminum target holder to avoid the background of carbon from the target holder. Note that, the contaminations of nitrogen and oxygen from the air were not avoided with this setup.

The $^{197}$Au and $^{209}$Bi metal targets were square-shaped and 50 mm in length on each side with a thickness of 0.482 mm and 1.07 mm, respectively.
%0.742 mm $^{197}$Au disk 
$^{197}$Au target with the thickness of 0.742~mm 
(additional $\phi$48 mm, 0.26-mm thick disk was placed on the square $^{197}$Au target) was also used to evaluate the effect of self-absorption of the X-rays.
The measurement times were 1~h for 0.482-mm and 0.742-mm thick $^{197}$Au each and 1.5~h for 1.07-mm thick $^{209}$Bi.
The $^{208}$Pb was approximately 0.8-mm thick and used to evaluate the energy calibration method, as explained in Sect.~\ref{sec:linearity}.
The standard $^{60}$Co source was placed around the target during the measurement to evaluate the gain drift.

The energy and efficiency of the detector system were calibrated, as discussed in Sect.~3.
%The non-linearity of the digitizer and the pre-amplifier was checked with the pulse generator, and corrected with forth-order polynomial function. 
%Each peak in the energy spectrums was fitted with the Gaussian function having the exponential tail in the lower side of the energy. 
The effect of the gain drift and pile-up during the measurements was corrected using Eq.~(\ref{eq:gain_drift}). 
In addition to a 511-keV peak and $\gamma$-ray peaks from $^{60}$Co, the peaks of 1173 keV and 1332 keV, 266 keV (deexcitation of $^{205}$Tl created by the nuclear muon capture reaction of $^{208}$Pb), and 1809 keV (background $\gamma$ rays from $^{27}$Al) were used as references for the long-term correction of $^{208}$Pb measurements. 
For the long-term correction of the $^{209}$Bi target, peaks of 277 and 2614 keV from $^{208}$Pb created by the muon capture reaction of $^{209}$Bi were used as additional reference peaks. 
Only four peaks at 511, 1173, 1332, and 356 keV(deexcitation of $^{196}$Pt) were used for $^{197}$Au.
For $^{197}$Au, the uncertainty of the pile-up effect correction became large in the energy region above 1.3 MeV because of a lack of known energy peaks in high-energy regions. 

The photo peak efficiency was evaluated using the GEANT4 simulation.  
The effects of self-absorption of the X-rays in the targets were included in the simulation.
The stopping position of the muon was simulated under the same condition as the measurement, and the peak efficiencies of X-rays at each stopping position were calculated.
The peak efficiency curves of the Ge detector were evaluated for each target. 
The accuracy of self-absorption correction was evaluated by comparing the intensity results of 0.48-mm and 0.74-mm thick $^{197}$Au targets. 