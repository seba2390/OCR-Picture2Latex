\subsection{Performance under high count-rate conditions}
\label{sec:highcountrate}

In muonic X-ray spectroscopy, several X-rays and $\gamma$-rays are emitted and the count rate of the detector is estimated to be a few kHz at PSI.
Therefore, the detector's performance, namely, a drift of the peak channel and deterioration of the energy resolution, was studied under high-count-rate conditions. % around a few kHz. 
A high count-rate condition was created 
%This effect is studied
by changing the distance ($d$) between the standard $\gamma$-ray sources and the Ge detectors from 10 cm to 42 cm.

In the DAQ system, the pile-up rejection time was set to 8 $\mu$s and signals detected in 8 $\mu$s from the previous signals were rejected.
As the time constant of the preamplifier is approximately 50 $\mu$s, 
output signals from the preamplifier require approximately 120 $\mu$s to reach the baseline. %, the pole-zero correction is introduced in the pulse height analysis.
The pile-up effect becomes considerable under the high count-rate condition 
because the baseline calculation uses the region in which the pole-zero correction is insufficient and affected by the previous signal.
Therefore, the drift of the peak channel was observed under the high count-rate conditions.
This effect can be corrected by adding the pile-up term $N_p$ introduced in Eq.~(\ref{eq:gain_drift}).
%\begin{equation}
%  \label{eq:gain_drift}
%  E = A\times E_{original} + N_p
%\end{equation}
%where A is the gain drift parameter and $E_{original}$ is the energy before correction.
Some calibration sources or known peaks are necessary during the measurement for conducting this correction.

The pile-up effect also changes the energy resolution depending on the count rate.
The energy resolutions at 344 keV from the $^{152}$Eu $\gamma$-ray source measured using three $N_\mathrm{sb}$ values at 10-, 12-, and 14-bit for GX5019 are shown in Fig.~\ref{fig:Highcount_Eresolution}. 
The same results were obtained for GC3018 and BE2820. 
The energy resolution becomes worse as the count rate increases.
$N_\mathrm{sb}$ should be optimized based on the count rate of the measurement, namely, 14-bit below 2~kHz and 12-bit from 2--10 kHz. 
%as shown in Fig.~\ref{fig:Highcount_Eresolution}.
%In the high count rate condition, the gain drift has an additional parameter, and Nsb should be adjusted to optimize the energy resolution. 
The other parameters listed in Table~\ref{tab:Eresolution_parameters} were independent of the count-rate condition.
%not necessary to change.

The peaks in the energy spectra have a tail on the peak under high count-rate conditions because the baseline calculation is conducted for each signal in the digital pulse height analysis. 
Under such conditions, the response function using the Gaussian function with an exponential tail on the lower side is required. %the muonic X-ray measurement.


\begin{figure}
  \centering
  %\includegraphics[scale=0.35]{Chap3_fig/Highcount_Eresolution_344keV_GX5019.pdf}
  \includegraphics[scale=0.35]{Highcount_Eresolution_344keV_GX5019.pdf}
  \caption{Energy resolution at 344 keV with GX5019 obtained by changing the count-rate condition. The sampling number of the baseline calculation ($N_\mathrm{sb}$) is set to 
  14-bit (blue circle), 12-bit (red box), and 10-bit (green cross).}
  \label{fig:Highcount_Eresolution}
\end{figure}
