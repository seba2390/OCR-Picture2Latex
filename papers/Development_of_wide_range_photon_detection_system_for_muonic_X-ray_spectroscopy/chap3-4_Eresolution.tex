\subsection{Energy resolution}\label{sec:Eresolution}
Optimization of the parameters in digital pulse processing was performed to obtain the best energy resolution of the Ge detectors.
The energy resolution was deduced from the Gaussian fitting of each $\gamma$-ray peak in the energy spectrum.
A trapezoidal filter was used as an algorithm to deduce the energy in the DPP-PHA firmware.
The parameters of the trapezoidal filter are 
the flat top, pole-zero, trapezoidal rise time, peaking time, and number of samples for energy mean calculation ($N_\mathrm{sp}$)~\cite{DPP-PHAandPSD}.
In addition to these parameters, the sampling number of the baseline calculation ($N_\mathrm{sb}$), 
the input dynamic range, and the gain of the preamplifiers were examined. % the energy resolution.
The list of the parameters and their ranges are summarized in Table~\ref{tab:Eresolution_parameters}.
% The parameters that are studied are listed in Table~\ref{tab:Eresolution_parameters}, and the ranges are also shown in the second column. % of the Table~\ref{tab:Eresolution_parameters}.

$N_\mathrm{sb}$, the input dynamic range, and the gain of the preamplifiers depend on the experimental conditions.
%To optimize those parameters, the following procedure is conducted.
$N_\mathrm{sb}$ should be maximized if the count rate of the measurement is lower than 2~kHz as a large $N_\mathrm{sb}$ helps reduce the effect of high-frequency noise in the pulse height calculation.
When the count rate becomes higher, $N_\mathrm{sb}$ should be changed to a lower value to prevent the pile-up effect, as explained in Sect.~\ref{sec:highcountrate}.
The input dynamic range of the digitizer and the gain of preamplifiers should be selected based on the requirement of each experiment as 
%Better energy resolution can be achieved with 
a wider energy dynamic range results in worse energy resolution.
%The energy corresponding to 1 bin in the ADC channel should be minimized to maximize the digitization resolution of 14 bits.
%to reduce the effect of the digitization resolution of 14bit, which makes worse the energy resolution because of the reduction of the pulse height information. 
The energy dynamic ranges with GC3018, GX5019, and BE2820 are presented in Table~\ref{tab:energy_accuracy}.


The parameters of the trapezoidal filter were optimized by comparing the energy resolutions.
Prior to the optimization, the pole-zero parameters 
%, which corresponds to the pre-amplifier's time constant, 
were fixed by evaluating the output waveforms. %(about 50 $\mu$s for the present detectors).
The optimized parameters of the proposed system are summarized in Table~\ref{tab:Eresolution_parameters}. 
The best value of the trapezoidal rise time ranged from 5--8 $\mu$s, and the best values of the flat top ranged from 1.0--1.5 $\mu$s as usual parameters for HPGe detectors.
The trapezoidal rise time lower than 4 $\mu$s does not provide sufficient sampling numbers to reduce high-frequency noise.
The peaking time and $N_\mathrm{sp}$ correspond to the pick-up area in the flat top region of the trapezoid.
Figure~\ref{fig:trapezoid_flattop} shows the trapezoidal-filtered waveforms of BE2820. % taken with parameters shown in Table~\ref{tab:Eresolution_parameters}.
Each waveform had different shapes near the edges of the flat-top region, as indicated by the vertical lines.
This difference is attributed to the fluctuations in the %rise time and rising 
shapes of the rising part of the pulses.
% waveform of the input signals.
Taking samples of the trapezoidal height from the actual flat-top region shown in the figure is needed to obtain a better energy resolution.
Therefore, the best values of the peaking time and the $N_\mathrm{sp}$ were set to 20\% and 64 samples for BE2820, 20\% and 16 samples for GX5019, and 50\% and 16 samples for GC3018, respectively, by comparing the energy resolutions with changing these values iteratively.

\begin{figure}
  \centering
  %\includegraphics[scale=0.3]{Chap3_fig/Trapezoid_flattop.pdf}
  \includegraphics[scale=0.3]{Trapezoid_flattop.pdf}
  \caption{Trapezoidal-filtered waveforms of BE2820 zoomed in the flat-top area.
  The edges of the flat-top region show different shapes 
  with fluctuations in the shapes of the signals.}
  \label{fig:trapezoid_flattop}
\end{figure}


\begin{table*}[width=.75\textwidth, cols=4,pos=h]
  \caption{Parameters of the proposed system. The range of the parameters is shown in the second column and the optimized parameters for obtaining the best energy resolution are listed for each detector. The parameters corresponding to the dynamic range (input dynamic range and Ge pre-amplifier gain) must be selected based on the requirement of each experiment.}
  \label{tab:Eresolution_parameters}
  \begin{tabular*}{\tblwidth}{@{}lcccc@{}}
\toprule
Parameter       &   Parameter range  & GC3018 & GX5019 & BE2820 \\
\midrule
Flat top ($\mu$s)              &  0.5 -- 2.0      & 1 & 1 & 1 \\
Pole-zero ($\mu$s)             &    40-60         & 47 & 49 & 49 \\
Trapezoidal rise time ($\mu$s) &    4-10          & 7 & 7 & 6 \\
Peaking time (\%)              &   0-100          & 50 & 20 & 20 \\
$N_\mathrm{sp}$                & 4, 16, 64        & 16 & 64 & 64 \\
$N_\mathrm{sb}$                & 1024, 4096, 16384& 16384 & 16384 & 16384 \\
Input dynamic range (V)        &   2.0, 0.5       & - & - & -\\
Ge preAmp Gain                 & $\times$1, 2, 5, 10& - & - & - \\
\bottomrule
  \end{tabular*}
\end{table*}

%With these steps, the best parameters for the energy resolution are deduced and summarized in Table~\ref{tab:Eresolution_parameters}. 
%The difference in each parameter between Ge detectors comes from the difference in the fluctuation of the rise time and rising waveform of each detector. 
The energy resolutions of Ge detectors obtained using the best parameters are shown in Fig.~\ref{fig:Eresolution}.
The fitting functions in the figure are expressed as,
\begin{equation}
  \label{eq:Eres_fitting}
  \sigma_\mathrm{FWHM}(E) = \sqrt{aE^2+bE+c},
\end{equation}
where $a, b$, and $c$ are the fitting parameters; $E$ is the energy; and $\sigma_{\mathrm{FMHM}}$ is the energy resolution in full-width at half maximum (FWHM). %~\cite{KNOLL}.
Note that, the energy resolutions of GX5019 in high-energy regions were obtained from the in-beam measurement of the $^{27}\mathrm{Al}(p,\gamma){}^{28}\mathrm{Si}$ reaction, and they showed worse energy resolution than that of low-energy regions obtained from the offline measurement because the noise condition of beam facility was worse.

The obtained energy resolutions were the same or better than those obtained using the analog DAQ systems (using a shaping amplifier and multi-channel analyzer) and the guaranteed values in the specification sheets.
The difference in energy resolutions among the detectors originated from the crystal size and the dynamic range of each detector.

\begin{figure}
  \centering
  %\includegraphics[bb=0 0 100 242, scale=0.5]{Chap3_fig/Eresolution_all.pdf}
  %\includegraphics[scale=0.35]{Chap3_fig/Eresolution_all.pdf}
  \includegraphics[scale=0.35]{Eresolution_all.pdf}
  \caption{Energy resolutions of Ge detectors with the best parameters presented in Table~\ref{tab:Eresolution_parameters}.
  The points represent the measured values and the lines are the fitting curves obtained using Eq.~(\ref{eq:Eres_fitting}).}
  \label{fig:Eresolution}
\end{figure}

