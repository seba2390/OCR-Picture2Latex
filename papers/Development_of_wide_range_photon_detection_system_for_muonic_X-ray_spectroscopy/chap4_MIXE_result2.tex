%% Analysis & Results: Cに対するsupのspectrum & Au, Bi, PbのX-rayの比
\subsection{Results and Discussion: muonic X-rays from $^{197}$Au, $^{208}$Pb, and $^{209}$Bi}
The energy spectra of muonic X-rays from $^{197}$Au, $^{208}$Pb, and $^{209}$Bi are shown in Fig.~\ref{fig:muX_energyspectrum}. 
Each peak in the energy spectra was fitted with the response function.
%the Gaussian function having the exponential tail in the lower side of the energy. 
The measured energies and intensities of muonic X-rays of $^{208}$Pb, $^{197}$Au, and $^{209}$Bi are presented in Table~\ref{tab:muX_Pb}, \ref{tab:muX_Au}, and \ref{tab:muX_Bi}, respectively.
%The previous values
The literature values from the previous measurement are listed in these tables. 
Assignments of transitions are based on the energies compared with the calculated values using MuDirac~\cite{Sturniolo2021-kw}. %and the calculated energies using MuDirac~\cite{} are also shown in the tables. 
$^{197}$Au and $^{209}$Bi showed hyperfine splitting owing to their nonzero nuclear spin.
The energies are listed in the tables for each hyperfine transition if they are clearly separated with the energy resolution of the Ge detector (approximately 7.4 keV in FWHM at 6.0 MeV evaluated from $^{208}$Pb measurements).
If the peaks are overlapped, the listed energies are the mean values of each peak containing several levels of hyperfine splitting.
The relative intensity of $^{208}$Pb is not shown because the thickness of the $^{208}$Pb target was not uniform, and the estimation of self-absorption using the GEANT4 simulation was insufficient for the efficiency calibration. 
The results of relative intensities of 0.48-mm and 0.74-mm thick $^{197}$Au targets were consistent in all the energy ranges, and the accuracy of self-absorption correction was confirmed. The intensities shown in Table~\ref{tab:muX_Au} are the result of 0.48-mm thick $^{197}$Au target. 
The relative intensities shown in Table~\ref{tab:muX_Au} and \ref{tab:muX_Bi} were normalized such that the largest intensity of $4f_{7/2}-3d_{5/2}$ transitions\footnote{The $4f_{7/2}-3d_{5/2}$ transitions contain several unresolved hyperfine splittings.} is 100.
%to 100 at the intensities of $4f_{7/2}-3d_{5/2}$ transitions which show the largest intensity. 
The previous research reported intensities of 47.9 (12)\% for $^{197}$Au from $4f_{7/2}-3d_{5/2}$ transition~\cite{Measday2007-zh} and 75.6 (15)\% from the sum of $4f_{5/2}-3d_{3/2}$ and $4f_{7/2}-3d_{5/2}$ transitions~\cite{Hartmann1982-wi}.
For $^{209}$Bi, an intensity of 42.7 (29)\% from $4f_{7/2}-3d_{5/2}$ transition~\cite{Measday2007-zh} and 
72 (3.6)\% from the sum of $4f_{5/2}-3d_{3/2}$ and $4f_{7/2}-3d_{5/2}$ transitions~\cite{Backenstoss1970-wj} were reported, and the latter values are more reliable according to Ref.~\cite{Measday2007-zh}. 
These values are based on assumptions or model calculations, as discussed below. 
To convert to absolute intensities from the values presented in Table.~\ref{tab:muX_Au} and \ref{tab:muX_Bi}, one must multiply the values with 0.44(4) and 0.42(4), respectively.


% Energy
The muonic X-ray energies were obtained independently from the previous measurements of muonic X-rays. 
The measured energies are consistent with the previous results of $^{208}$Pb.
% guaranteed? (niikura)
As the muonic X-rays from $^{208}$Pb were accurately measured previously~\cite{Bergem1988-nf}, the calibration method conducted in this study confirmed the energy range to be below 6 MeV.
The K, L, M, and N lines were measured from $^{197}$Au target and the muonic K, L, M, N, and O lines were measured from $^{209}$Bi target.
All the energies of muonic X-rays obtained in this study are also consistent with those reported in the previous studies.
%When compared to the energies of measurement and MuDirac, M, N and O lines are all consistent. For $^{208}$Pb, L lines are also compatible, but K lines show about 1 keV discrepancy. 


%Intensity
For the intensities, when comparing the relative values reported in this study and those reported by Measday et al.~\cite{Measday2007-zh}, the L, M, and N lines are consistent with the previous values, except for $6f_{7/2}$-$3d_{5/2}$ and $4d_{3/2}$-$2p_{1/2}$ transitions; however, a large discrepancy was found in the intensities of K rays in $^{197}$Au. 
This discrepancy is attributed to the overestimated efficiency curve in the previous measurement in high-energy regions. 
The intensity results of the present study are consistent with those reported by Hartmann et al.~\cite{Hartmann1982-wi}, except for $2p_{1/2}$-$1s_{1/2}$ and $5g$-$4f$ transitions. The energy of $2p_{1/2}$-$1s_{1/2}$ was 5.6 MeV, and the discrepancy may be attributed to the uncertainty of the efficiency of the detector. The energy of $5g$-$4f$ was approximately 400 keV, and the effect of self-absorption was significant. 
%For $5f$-$3d$ transition, the present result is different from either previous research, so the present result can be overlapped with backgrounds.

For $^{209}$Bi, the discrepancies in the intensities of the K and L lines, and the transitions in 449 keV and 823 keV were found between the present and previous studies. 
Measday et. al.~\cite{Measday2007-zh} reported the normalized intensities with some assumptions of the probability of a nonradiative transition and a nuclear-exciting transition~\cite{Measday2007-zh} for the total probability of the K and L transition series. The discrepancy may be attributed to the normalization assumptions in the previous research. 
For the $4d_{3/2}$-$2p_{1/2}$ transition, the previous research may have included the intensity of another peak at approximately 3696 keV. 
Therefore, the values reported in the present study are more reliable in terms of relative intensities because no assumptions were made.

Hence, we infer that the results of energies and relative intensities of muonic X-rays from $^{197}$Au and $^{209}$Bi provide new calibration reference values for photon detectors in high-energy regions below 8 MeV. 
%Note that these peaks cannot be used as the width reference for the energy resolution calibration.
The uncertainty of the result of this work provided in Table 4,5,6 includes systematic uncertainty in the energy column and excludes systematic uncertainty in the intensity column to be used practically for the calibration. The breakdowns of the uncertainties are written in the footnote of each Table.

\begin{figure*}[width=0.85\textwidth, cols=4,pos=h]
    \centering
    \includegraphics[scale=0.8]{muX_energyspectrum.pdf}
    \caption{Energy spectrum of the muonic X-rays of $^{208}$Pb (a), $^{197}$Au (b), and $^{208}$Bi (c). X-rays from the K, L, M, and N series are identified. The SE and DE peaks are shown in the spectra.}
    \label{fig:muX_energyspectrum}
\end{figure*}

\begin{table}[width=0.5\textwidth, cols=4,pos=h]
  %\centering
  \caption{Muonic X-ray energies for $^{208}$Pb.}
  \label{tab:muX_Pb}
  \begin{tabular*}{\tblwidth}{@{} LLLL@{} }
    \toprule
    & & Energy (keV) & \\
    transition & this work$^*$ & Kessler\cite{Kessler1975-sw} & Bergem\cite{Bergem1988-nf}\\% & MuDirac\\
    \midrule
    $2p_{1/2}$-$1s_{1/2}$ & 5777.44(51) & 5777.91(40)   & 5778.058(100) \\%& 5779.06 \\
    $2p_{3/2}$-$1s_{1/2}$ & 5962.68(41) & 5962.77(42)   & 5962.854(90)  \\%& 5963.81 \\
    %$3p_{3/2}$-$1s_{1/2}$ & -           & 8501.53(97)   &       -       \\%& 8501.91 \\
    %$3d_{3/2}$-$1s_{1/2}$ & -           & 8419.51(75)   &       -       \\%& 8420.84 \\
    $3d_{3/2}$-$2p_{3/2}$ & 2457.72(46) & 2457.20(20)   & 2457.569(70)  \\%& 2457.00 \\
    $3d_{5/2}$-$2p_{3/2}$ & 2500.49(34)  & 2500.33(6)    & 2500.590(30)  \\%& 2499.74 \\
    $3d_{3/2}$-$2p_{1/2}$ & 2642.13(35)  & 2642.11(6)    & 2642.332(30)  \\%& 2641.75 \\
    $3p_{1/2}$-$2s_{1/2}$ & 1460.84(44) &      -        & 1460.558(32)  \\%& 1460.88 \\
    $3p_{3/2}$-$2s_{1/2}$ & 1507.44(37)  & 1507.48(26)   & 1507.754(50)  \\%& 1508.34 \\
    %$2s_{1/2}$-$2p_{3/2}$ & -           & 1030.44(17) & 1030.543(27)  \\%&  \\
    %$2s_{1/2}$-$2p_{1/2}$ & -           & 1215.43(26) & 1215.330(30)  \\%&  \\
    $4f_{5/2}$-$3d_{5/2}$ & -$^{a}$     & -             & 928.883(14)   \\%& 929.024\\%ダメそう930.175(20)
    $4f_{7/2}$-$3d_{5/2}$ & 938.04(32)   & 937.98(6)     & 938.096(18)   \\%& 938.198 \\
    $4f_{5/2}$-$3d_{3/2}$ & 972.00(32) & 971.85(6)     & 971.974(17)   \\%& 971.759 \\
    %$5f_{5/2}$-$3d_{5/2}$ & -           &      -        & 1361.748(250) \\%& 1361.78 \\
    $5f_{7/2}$-$3d_{5/2}$ & 1366.55(38)$^{b}$  & 1366.52(8)    & 1366.347(19)  \\%& 1366.48 \\
    $5f_{5/2}$-$3d_{3/2}$ & 1404.94(46) & 1404.74(8)    & 1404.658(20)  \\%& 1404.51 \\
    %$6f_{7/2}$-$3d_{5/2}$ & -           & 1599.67(15)   &       -       \\%&  \\
    %$6f_{5/2}$-$3d_{3/2}$ & -           & 1639.98(19)   &       -       \\%& 1639.71 \\
    $4d_{3/2}$-$3p_{3/2}$ & 874.09(61)&      -        & 873.761(63)   \\%& 873.365 \\
    $4d_{5/2}$-$3p_{3/2}$ & -           &      -        & 891.383(22)   \\%&  \\
    $4d_{3/2}$-$3p_{1/2}$ & -           &      -        & 920.959(28)   \\%&  \\
    \bottomrule
  \end{tabular*}
  \leftline{*the 0.3-keV systematic uncertainty is included.}
  \leftline{$^{a}$The energy peak at 928 keV is masked by the tail of 938 keV.}
  \leftline{$^{b}$The energy peak at 1366 keV is deduced from the spectrum}
  \leftline{~obtained at 100 ns from the muon irradiation time because it is}
  \leftline{~overlapped with a 1368-keV delayed $\gamma$-ray from $^{24}$Na created}
  \leftline{~by muon capture reaction in the target holder ($^{27}$Al).}
\end{table}


\begin{table*}[width=0.9\textwidth, cols=4,pos=h]
  %\centering
  \caption{Muonic K, L, M, N series energies and intensities for $^{197}$Au.}
  \label{tab:muX_Au}
  \begin{tabular*}{\tblwidth}{@{} LLLLLLL@{} }
    \toprule
    & \multicolumn{3}{c}{Energy (keV)} & \multicolumn{3}{c}{Relative Intensity **}\\
    transition & this work* & Acker\cite{Acker1966-kh} and others & Measday\cite{Measday2007-zh} & 
    this work & Measday\cite{Measday2007-zh} & Hartmann~\cite{Hartmann1982-wi}\\
    %transition & Energy (keV) & Energy (keV) & Intensity (\%) & Energy (keV)* & Relative intensity** (\%)\\
    \midrule
    $2p_{1/2}$-$1s_{1/2}$ & 5591.0(11)     & 5592.8(50)      & (5591.0)$^e$ & 86.2(10)  & 71.4(63)  & 76.1(42)\\
    $2p_{3/2}$-$1s_{1/2}$ & 5764.7(11)     & \multirow{2}{*}{5762.5(50)} & (5763.1)$^e$ & 89.6(15) & \multirow{2}{*}{115.2(84)} & \multirow{2}{*}{127.6(74)}\\
                          & 5746.3(11)     &                             &              & 38.68(94) & & \\
    $3p_{1/2}$-$1s_{1/2}$ & -              &      -          & \{8085\}$^f$ & -         & 3.3(33)   & \multirow{2}{*}{4.39(55)}\\
    $3p_{3/2}$-$1s_{1/2}$ & 8130.4(37)     &      -          & \{8128\}$^f$ & 1.78(59)  & 8.1(42)   & \\
    $3d_{3/2}$-$1s_{1/2}$ & 8065.3(34)     & -               & -            & 2.63(67)  & -         & \multirow{2}{*}{5.92(66)}\\
    $3d_{5/2}$-$1s_{1/2}$ & 8103.2(31)     & -               & -            & 3.91(72)  & -         & \\
    $3d_{3/2}$-$2p_{3/2}$ & 2304.25(68)    &    -            & 2302(2)      & 6.20(39)  & 8.6(35)   & \\
                          & 2319.86(66)    &                 &              & 7.36(51)  & & \\
    $3d_{5/2}$-$2p_{3/2}$ & 2341.52(57)    & \multirow{2}{*}{2343.1(25)} & 2341.2(2)  & 63.3(13) & \multirow{2}{*}{95.6(73)}& \\
                          & 2357.69(58)    &                             &            & 30.60(59) & & \\
    $3d_{3/2}$-$2p_{1/2}$ & 2477.27(59)    & 2474.4(20)      & (2477.8)$^e$ & 66.39(57)   & 63.3(75)  & $^{sum}$179.8(60)\\
    $4d_{5/2}$-$2p_{3/2}$ & 3203.61(90)    &    -            & 3202(5)      & 3.15(86)  & \multirow{2}{*}{6.9(21)} & \\%\\gaussianじゃないかもしれない
                          & 3221.4(12)     &                 &              & 1.28(64)  & & \\
    $4d_{3/2}$-$2p_{1/2}$ & 3361.10(97)    &   -             & 3356(5)      & 3.6(14)   & 7.7(25)   & $^{sum}$9.0(36)\\
    $5d_{5/2}$-$2p_{3/2}$ & -              &   -             & \{3601\}$^f$ & -         & 2.7(27)   & -\\
    $5d_{3/2}$-$2p_{1/2}$ & -              &   -             & \{3762\}$^f$ & -         & 2.1(21)   & -\\
    $3p_{1/2}$-$2s_{1/2}$ & 1391.31(72)    & 1392.31(48)$^c$ &     -        & 0.87(21)  & -         & \multirow{2}{*}{3.9(11)}\\
    $3p_{3/2}$-$2s_{1/2}$ & 1435.72(82)    & 1436.95(37)$^c$ &     -        & 3.7(16)   & -         & \\
    %$2s_{1/2}$-$2p_{1/2}$ & 1105.98(57)   & 1103.49(39)     &     -        & - & - \\
    $4f_{5/2}$-$3d_{5/2}$ & 862.60(44)     & -               & -            & 6.28(40)  & -         & -\\
    $4f_{7/2}$-$3d_{5/2}$ & 870.04(31)     & 869.1(16)       & 870.11(10)   & 100.00(41)  & 100.0(23) & \multirow{2}{*}{172.6(35)} \\
    $4f_{5/2}$-$3d_{3/2}$ & 899.24(31)     & 899.6(14)       & 899.27(10)   & 72.65(41)   & 72.4(25)  &  \\
    $5f_{7/2}$-$3d_{5/2}$ & 1267.77(36)    &    -            & 1267(1)      & 12.38(40)   & 9.2(27)   & \multirow{2}{*}{18.21(78)}\\
    $5f_{5/2}$-$3d_{3/2}$ & 1300.70(40)$^a$&    -            & 1299(1)      & -$^a$     & 3.8(17)   &  \\
    $6f_{7/2}$-$3d_{5/2}$ & 1483.77(64)    &    -            & \{1482\}$^f$ & 2.84(60)  & 0.84(84)  &  \multirow{2}{*}{7.49(69)}\\
    $6f_{5/2}$-$3d_{3/2}$ & -              &    -            & \{1516\}$^f$ & -         & 0.84(84)  & \\
    $7f_{7/2}$-$3d_{5/2}$ & -              &    -            & \{1612\}$^f$ & -         & 1.7(10)   &  \multirow{2}{*}{3.02(73)}\\
    $7f_{5/2}$-$3d_{3/2}$ & -              &    -            & \{1647\}$^f$ & -         & 0.84(84)  & \\
    $5g_{9/2}$-$4f_{7/2}$ & 400.03(30)     &400.14(5)$^d$    & 400.15(15)   & 76.13(52)   & 81.0(44)  &  \multirow{2}{*}{154.4(57)}\\
    $5g_{7/2}$-$4f_{5/2}$ & 405.52(31)     &405.65(5)$^d$    & 405.58(15)   & 63.53(27)   & 63.9(44)  & \\
    $6g_{9/2}$-$4f_{7/2}$ & 615.31(33)     &     -           & 615.5(4)     & 11.30(66)   & 14.4(42)  &  \multirow{2}{*}{20.90(96)}\\
    $6g_{7/2}$-$4f_{5/2}$ & 621.80(33)     &     -           & 621.7(4)     & 10.86(42)   & 12.1(42)  & \\
    $7g_{9/2}$-$4f_{7/2}$ & 745.12(36)     &     -           & 744.9(5)     & 2.56(25)  & 6.1(61)   &  \multirow{2}{*}{5.85(23)}\\
    $7g_{7/2}$-$4f_{5/2}$ & 752.19(37)     &     -           & 752.1(5)     & 2.44(34)  & 4.8(23)   & \\
    $8g_{9/2}$-$4f_{7/2}$ & 828.38(58)$^b$ &     -           & \{829\}$^f$  & -$^b$     & 1.25(63)  &  \multirow{2}{*}{2.42(85)}\\
    $8g_{7/2}$-$4f_{5/2}$ & 836.09(47)$^b$ &     -           & \{836\}$^f$  & -$^b$     & 2.7(13)   & \\
    \bottomrule
  \end{tabular*}
  %\leftline{~*~~First and second parenthesis represent the statistic and systematic errors, respectively.}
  \leftline{~* the 0.3-keV systematic uncertainties are included in the energy region below 1.3 MeV and the energy depend uncertainties are included} 
  \leftline{~~~in the energy region above 1.3 MeV, for example, 1.0 keV systematic uncertainty is included in the $2p_{1/2}$-$1s_{1/2}$ transition.}
  \leftline{~* the listed energies are a weighted average of hyperfine splitting. These energy peaks show a larger FWHM}
  \leftline{~~~than the intrinsic energy resolution of the Ge detector.}
  %\leftline{~**the intensity is normalized at 75.58\% at the sum of the transitions $4f_{5/2}-3d_{3/2}$ and $4f_{7/2}-3d_{5/2}$.}
  %\leftline{~**Another 3.6\% systematic error is added for absolute intensity (See text).}
  \leftline{~**the intensity is normalized at 100\% at the transition of $4f_{7/2}-3d_{5/2}$ which shows the largest intensity.}
  \leftline{~**the intensities reported by Hartmann et al. are renormalized at the total intensity of $4f_{5/2}-3d_{3/2}$ and $4f_{7/2}-3d_{5/2}$}
  \leftline{~~~to 172.6\%, which is the relative intensity value reported in this study for the ease of comparison.}
  \leftline{~**multiply with 0.44 (4) to obtain the absolute intensity (See text).}
  \leftline{~blank energies were not detected because of low statistics.}
  %\text{1435 keV seems overlapped with other peaks}
  \leftline{~$^a$1300 keV was overlapped with a DE peak of $3d_{5/2}$-$2p_{3/2}$ transition (1297.9 keV)}
  \leftline{~$^b$828 and 836 keV peaks are difficult to deduce intensities because of low statistics.}
  \leftline{~$^c$Powers(1974)~\cite{Powers1974-aa}}
  \leftline{~$^d$Engfer(1974)~\cite{Engfer1974-km}}
  \leftline{~$^e$ $K_\alpha$ lines and 2477.8 keV are used as energy calibration references, reported by Measday et al~\cite{Measday2007-zh}.}
  \leftline{~$^f$ curly brackets are the estimated values obtained from calculations.}
\end{table*}


\begin{table*}[width=0.8\textwidth, cols=4,pos=h]
  %\centering
  \caption{Muonic K, L, M, N, O series energies and intensities for $^{209}$Bi.}
  \label{tab:muX_Bi}
  \begin{tabular*}{\tblwidth}{@{} LLLLLL@{} }
    \toprule
    & \multicolumn{3}{c}{Energy (keV)} & \multicolumn{2}{c}{Relative Intensity **} \\
    transition & this work* & Engfer\cite{Engfer1974-km} and others & Measday\cite{Measday2007-zh} & this work & Measday\cite{Measday2007-zh} \\
    %transition & Energy (keV) & Energy (keV) & Intensity (\%) & Energy (keV) & Relative intensity (\%)\\
    \midrule
    $2p_{1/2}$-$1s_{1/2}$ & 5841.48(35)  & 5839.7(55)$^c$ & 5841.5(30) & 98.51(97) & 84.5(35) \\%
    $2p_{3/2}$-$1s_{1/2}$ & 6028.25(37)  & \multirow{2}{*}{6032.2(50)$^c$} & \multirow{2}{*}{6032.4(30)} & 69.6(12) & \multirow{2}{*}{113.8(35)} \\
                          & 6038.83(36)  &                                 &                             & 70.88(89) & \\
    $3p_{3/2}$-$1s_{1/2}$ & 8629.2(27)   &                & 8628(10)   & 1.48(58)  & 1.64(94) \\
    $3d_{3/2}$-$1s_{1/2}$ & 8542.3(18)   &                & 8539(10)   & 2.68(62)  & 5.39(70) \\
    $3d_{5/2}$-$1s_{1/2}$ & 8587.2(17)   &                & 8584(10)   & 4.92(68)  & 6.09(70) \\
    $3d_{3/2}$-$2p_{3/2}$ & 2504.22(49)  & 2501.81(59)    & 2504(2)    & 7.3(11) & 3.5(19) \\
                          & 2513.57(50)  &                &            & 7.6(14) & \\
    $3d_{5/2}$-$2p_{3/2}$ & 2550.15(33)  & \multirow{2}{*}{2554.8(20)$^c$} & 2549.6(3)  & 50.12(72) & 52.7(26) \\
                          & 2559.50(34)  &                                 & 2558.9(3)  & 38.74(90) & 43.1(26) \\
    $3d_{3/2}$-$2p_{1/2}$ & 2700.30(32)  & 2700.50(17)    & 2700.3(2)  & 66.88(44) & 62.5(26) \\
    $4d_{5/2}$-$2p_{3/2}$ & -$^a$        &      -         & 3510(1)    & -         & 9.4(33) \\
    $4d_{3/2}$-$2p_{1/2}$ & 3678.78(61)  &       -        & 3679(1)    & 2.64(31)  & 5.2(14) \\
    $5d_{5/2}$-$2p_{3/2}$ &  -           &        -       & \{3951\}   & -         & 0.7(14) \\
    $5d_{3/2}$-$2p_{1/2}$ &  -           &         -      & \{4132\}   & -         & 1.2(12) \\
    $4f_{7/2}$-$3d_{5/2}$ & 961.28(31)   &  961.18(25)    & 961.3(2)   & 100.00(35)& 100.0(68) \\
    $4f_{5/2}$-$3d_{3/2}$ & 996.76(31)   &  996.67(25)    & 996.8(2)   & 79.31(35) & 80.8(56) \\
    $5f_{7/2}$-$3d_{5/2}$ & 1400.49(35)  &        -       & 1400.4(5)  & 9.34(29)  & 9.1(14) \\
    $5f_{5/2}$-$3d_{3/2}$ & 1440.33(41)  &        -       & 1440.2(5)  & -$^d$     & 6.8(19) \\
    $6f_{7/2}$-$3d_{5/2}$ & 1639.97(39)  &        -       & 1640.1(5)  & 4.07(20)  & 5.2(14) \\
    $6f_{5/2}$-$3d_{3/2}$ &  -$^b$       &        -       & \{1681.0\} & -         & 2.8(23) \\
    $7f_{7/2}$-$3d_{5/2}$ &  -           &       -        & 1783(2)    & -         & 0.94(47) \\
    $7f_{5/2}$-$3d_{3/2}$ &  -           &       -        & 1826(2)    & -         & 1.9(14) \\
    $8f_{7/2}$-$3d_{5/2}$ &  1876.46(50) &       -        & 1876(2)    & 1.34(16)  & 2.3(14) \\
    $8f_{5/2}$-$3d_{3/2}$ &  -           &      -         & 1920(2)    & -         & 1.6(16) \\
    $5g_{9/2}$-$4f_{7/2}$ & 442.09(30)   &  442.107(50)   & 442.0(1)   & 95.18(86) & 93.2(44) \\
    $5g_{7/2}$-$4f_{5/2}$ & 448.76(30)   &  448.828(50)   & 448.7(1)   & 80.5(10)& 74.9(44) \\
    $6g_{9/2}$-$4f_{7/2}$ & 679.75(32)   &         -      & 679.7(1)   & 10.75(22) & 11.2(23) \\
    $6g_{7/2}$-$4f_{5/2}$ & 687.55(33)   &         -      & 687.6(2)   & 7.61(19)  & 8.7(23) \\
    $7g_{9/2}$-$4f_{7/2}$ & 823.38(37)   &         -      & 823.4(5)   & 5.52(40)  & 3.3(12) \\
    $7g_{7/2}$-$4f_{5/3}$ & 831.77(39)   &        -       & 832.0(5)   & -$^d$     & 2.8(14) \\
    $6h$-$5g$             & 239.60(30)   &  239.1(16)$^c$ &    -       & 179.08(52)& - \\
    \bottomrule
  \end{tabular*}
  \leftline{~* the 0.3-keV systematic uncertainty is included the energy region lower than 8 MeV, and the 1.0-keV uncertainty is included}
  \leftline{~~~in the energy region above 8 MeV.}
  \leftline{~* the listed energies are a weighted average of hyperfine splitting. These energy peaks show a larger FWHM}
  \leftline{~~than the intrinsic energy resolution of the Ge detector.}
  %\leftline{~**relative intensity is normalized at 72\% at the sum of the transitions $4f_{5/2}-3d_{3/2}$ and $4f_{7/2}-3d_{5/2}$}
  %\leftline{~**Another 5.8\% systematic error is added for absolute intensity (See text).}
  \leftline{~**the intensity is normalized at 100\% at the $4f_{7/2}-3d_{5/2}$ transition, which shows the largest intensity at 1 MeV.}
  \leftline{~**multiply with 0.42 (4) to obtain the absolute intensity (See text).}
  %\leftline{~$^b$Schneuwly(1972)~\cite{Schneuwly1972-cv}}8624.5,8538.5,8583.3
  \leftline{~$^a$Peaks around 3510 keV were difficult to distinguish from other peaks.}
  \leftline{~$^b$$6f_{5/2}$-$3d_{3/2}$ transition energy peak was overlapped by the double escape peak of $3d_{3/2}$-$1p_{1/2}$ transition. }
  \leftline{~Other blank peaks were not detected because of the low intensity.}
  \leftline{~$^c$Acker(1966)~\cite{Acker1966-kh}}
  \leftline{~$^d$The intensities of 1440 and 832 keV were not shown because other peaks were contained in the tail, and uncertainties are large.}
\end{table*}
