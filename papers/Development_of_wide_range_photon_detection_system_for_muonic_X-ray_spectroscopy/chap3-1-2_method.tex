\section{Performance of the detectors}
\label{sec:Performance}
The performance of the detectors was evaluated through offline measurements using standard $\gamma$-ray sources and an in-beam measurement using the $^{27}$Al(p,$\gamma$)$^{28}$Si resonance reaction prior to the demonstration of the detector system with muonic X-ray measurements at PSI.
The measurement setups for the offline and in-beam experiments are explained in Sect.~\ref{sec:Methods_2}.
A Monte Carlo simulation was performed to optimize the detector geometry and design muonic X-ray spectroscopy setups, as explained in Sect.~\ref{sec:Methods_3}. 
The evaluation of the performances of the Ge detectors as well as the energy accuracy, energy resolution, durability under the high count rate condition, timing resolution, photopeak efficiency, and Compton suppression results are presented in Sect.~\ref{sec:linearity}--\ref{sec:compton}, respectively. 


\subsection{Measurements} \label{sec:Methods_2}
%Some Evaluation experiments of the system are performed to understand the performance in a wide dynamic range.
Measurements using standard $\gamma$-ray sources were performed 
to investigate the essential feature of Ge detectors and Compton suppressors used in the photon detection system using the waveform digitizer. 
The linearity, energy and timing resolutions, photo-peak detection efficiency, high count rate durability of the Ge detectors, and the performance of Compton suppression were evaluated using standard $\gamma$-ray sources of 
$^{152}$Eu, $^{60}$Co, $^{133}$Ba, and $^{137}$Cs in low-energy regions below 1.4 MeV. 


An in-beam measurement using a proton-induced resonance reaction of $^{27}$Al at 992 keV, $^{27}\mathrm{Al}(p,\gamma){}^{28}\mathrm{Si}$, was conducted at the RIKEN tandem accelerator for evaluating the energy resolution and efficiency of the high-energy $\gamma$ rays for GX5019. 
The details of the experiment are presented in Ref.~\cite{Mizuno2023-px}.
The Ge detectors' photo-peak efficiencies and energy resolutions were deduced in a wide energy range from 1.5--10.8 MeV. 


%%%%%%%%%%%%%%%%%%%%%%%%%%%%%%%%%%%%%%%%%%
\subsection{GEANT4 simulation}\label{sec:Methods_3}
A Monte Carlo simulation using the GEANT4 toolkit~\cite{Agostinelli2003-eh} was performed to optimize the detector geometry and design the muonic X-ray spectroscopy setup.
In the present study, \texttt{GEANT4 v10.06.p03} was used, and \texttt{EmLivermorePhysics} was used as the physics model of electromagnetic interaction.
The reproducibility of the simulation was evaluated in terms of detection efficiency and Compton suppression, as presented in Sect.~\ref{sec:efficiency} and \ref{sec:compton}, respectively.

All the possible geometries were included in the simulation.
The geometries of the Ge crystals were taken from the specification sheets provided by the manufacturer.
The measured shape of the BGO crystals was implemented.
Other materials of the detector, such as the case of the detector, cryostat, entrance window, cold finger, liquid nitrogen dewar, BGO crystal holder, and lead ring were included in the simulation.
Furthermore, a target holder and other materials around the target/source were included in each situation. %, standard sources or $^{27}\mathrm{Al}(p,$\gamma$)$^{28}$\mathrm{Si} experiment.
%high energy $\gamma$-ray experiment.
All these materials are important for reproducing the shape of the $\gamma$-ray spectra.