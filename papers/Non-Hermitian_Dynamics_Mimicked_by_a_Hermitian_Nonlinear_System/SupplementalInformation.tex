\documentclass[%
 reprint,
%superscriptaddress,
%groupedaddress,
%unsortedaddress,
%runinaddress,
%frontmatterverbose, 
%preprint,
%preprintnumbers,
%nofootinbib,
%nobibnotes,
%bibnotes,
 amsmath,amssymb,
 aps,
%pra,
%prb,
%rmp,
%prstab,
%prstper,
floatfix,
]{revtex4-2}

\usepackage{graphicx}% Include figure files
\usepackage{dcolumn}% Align table columns on decimal point
\usepackage{bm}% bold math
\usepackage{cleveref}

\bibliographystyle{apsrev4-2}

\begin{document}

\title{Non-Hermitian Dynamics Mimicked by a Hermitian Nonlinear System: Supplementary Material}

\author{Noah Flemens}
\author{Nicolas Swenson}
\author{Jeffrey Moses}
\affiliation{School of Applied and Engineering Physics, Cornell University, Ithaca, New York, 14853, USA}
\affiliation{Corresponding authors: nrf33@cornell.edu, moses@cornell.edu}

% To be edited by editor
% \dates{Compiled \today}

% To be edited by editor
% \doi{\url{http://dx.doi.org/10.1364/optica.99.099999.s001} [supplementary document doi]}

\begin{abstract}
This document provides supplementary information to “Non-Hermitian Dynamics Mimicked by a Hermitian Nonlinear System,” 
\end{abstract}

\maketitle
\subsection{Derivation of Manley-Rowe equations}

The Manley-Rowe equations express conservation of fractional photon number and can be derived by taking the derivatives of the fractional photon number of each field with respect to $\zeta$ and substituting the coupled wave equation for each respective field (we drop the explicit zeta dependence of the fields for clarity):

\begin{align*}
    d_{\zeta}n_p & = d_{\zeta}|u_p|^2\\
                 & = u_p^* d_{\zeta}u_p + u_p d_{\zeta}u_p^*\\
                 & = i u_p^* u_s u_i e^{-i\Delta_{OPA}\zeta} - i u_p u_s^* u_i^* e^{i\Delta_{OPA}\zeta}\\
    d_{\zeta}n_s & = d_{\zeta}|u_s|^2\\
                 & = u_s^* d_{\zeta}u_s + u_s d_{\zeta}u_s^*\\
                 & = i u_p u_s^* u_i^* e^{i\Delta_{OPA}\zeta} - i u_p^* u_s u_i e^{-i\Delta_{OPA}\zeta}\\
    d_{\zeta}n_i & = d_{\zeta}|u_i|^2\\
                 & = u_i^* d_{\zeta}u_i + u_i d_{\zeta}u_i^*\\
                 & = i u_p u_s^* u_i^* e^{i\Delta_{OPA}\zeta} - i u_p^* u_s u_i e^{-i\Delta_{OPA}\zeta}\\ 
                 & + i 2\gamma_0 u_{2i} (u_i^*)^2 e^{i\Delta_{SHG}\zeta} - i 2\gamma_0 u_{2i}^* u_i^2 e^{-i\Delta_{SHG}\zeta}\\
    d_{\zeta}n_{2i} & = d_{\zeta}|u_{2i}|^2\\
                 & = u_{2i}^* d_{\zeta}u_{2i} + u_{2i} d_{\zeta}u_{2i}^*\\
                 & = i \gamma_0 u_{2i}^* u_i^2 e^{-i\Delta_{SHG}\zeta} - i \gamma_0 u_{2i} (u_i^*)^2 e^{i\Delta_{SHG}\zeta}
\end{align*}

\noindent The Manley-Rowe equations are then found by noting:
\begin{align*}
    && d_{\zeta}n_p &= -d_{\zeta}n_s\\
    \implies && d_{\zeta}n_p + d_{\zeta}n_s &= 0\\
    \implies && n_p + n_s &= C_1
\end{align*}

\noindent and

\begin{align*}
    && d_{\zeta}n_i &= -d_{\zeta}n_p - 2d_{\zeta}n_{2i}\\
    \implies && d_{\zeta}n_i + d_{\zeta}n_p + 2d_{\zeta}n_{2i} &= 0\\
    \implies && n_p + n_i + 2n_{2i} &= C_2
\end{align*}

\noindent where $C_1$ and $C_2$ are constants. Under the initial conditions of OPA, $n_{p,0}>n_{s,0}>0$ and $n_{i,0}=n_{2i,0}=0$. Hence,

\begin{align*}
    1 &= n_p + n_s\\
    n_{p,0} &= n_p + n_i + 2n_{2i}
\end{align*}

\noindent where we have used the fact that $\Sigma n_{j,0} = 1$ for $j \in \{p, s, i, 2i \}$ for the first equation.

For the case of QPA, the idler equation becomes:
\begin{align*}
    d_{\zeta}n_i & = d_{\zeta}|u_i|^2\\
                 & = u_i^* d_{\zeta}u_i + u_i d_{\zeta}u_i^*\\
                 & = i u_p u_s^* u_i^* e^{i\Delta_{OPA}\zeta} - i u_p^* u_s u_i e^{-i\Delta_{OPA}\zeta}\\ 
                 & -\frac{2\alpha}{\Gamma_{OPA}} |u_i|^2\\
\end{align*}

\noindent While the Manley-Rowe equation relating the pump and signal remain unchanged, the equation for the pump and idler is found from noting:
\begin{align*}
    && d_{\zeta}n_p &= -d_{\zeta}n_i - \frac{2\alpha}{\Gamma_{OPA}} n_i\\
    \implies && d_{\zeta}n_p + d_{\zeta}n_i + \frac{2\alpha}{\Gamma_{OPA}} n_i &= 0\\
    \implies && n_p + n_i + \frac{2\alpha}{\Gamma_{OPA}}\int^\zeta_0 n_{i} d\zeta' &= C_3
\end{align*}

\noindent where $C_3$ is a constant. Again, from the initial conditions of OPA, we find:
\begin{align*}
    n_{p,0} &= n_p + n_i + \frac{2\alpha}{\Gamma_{OPA}}\int^\zeta_0n_{i} d\zeta'
\end{align*}

\subsection{Proof: $\gamma(\infty)<1 \rightarrow Im \{ \lambda(\zeta) \}=0$ $\forall$ $\zeta$}

From the main text, the eigenvalues for the pump-idler subsystem of SHA are given by $\lambda(\zeta)=\pm \sqrt{n_s(\zeta) - \gamma_0^2 n_{2i}(\zeta)}$. Clearly, $Im \{ \lambda(\zeta) \}=0$ when $n_s(\zeta) > \gamma_0^2 n_{2i}(\zeta)$ for all values of $\zeta$. Starting with the assumption that $\gamma(\infty)<1$,  then:
\begin{align*}
    && \gamma(\infty) &< 1\\
    \implies && \gamma_0^2 n_{2i}(\infty)& < 1\\
    \implies && \gamma_0^2 &< \frac{2}{n_{p,0}}\\
    \implies && \gamma_0^2 n_{2i}(\zeta) &< \frac{2}{n_{p,0}} n_{2i}(\zeta) 
\end{align*}
\noindent where we have used the fact that $n_{2i}(\zeta)$ monotonically increases to $n_{p,0}/2$ as $\zeta \rightarrow \infty$. 

Now we wish to show $n_s(\zeta) \geq \frac{2}{n_{p,0}} n_{2i}(\zeta) > \gamma_0^2 n_{2i}(\zeta)$. By way of contradiction, assume that $n_{p,0} n_s(\zeta) < 2 n_{2i}(\zeta)$, then the Manley-Rowe relations state:
\begin{align*}
    && n_p(\zeta) + n_i(\zeta) + 2n_{2i}(\zeta) &= n_{p,0}\\
    \implies && n_p(\zeta) + n_i(\zeta) + n_{p,0}n_s(\zeta) &< n_{p,0}\\
    \implies && n_p(\zeta) + n_i(\zeta) &< n_{p,0} (1-n_s(\zeta))\\
    \implies && n_i(\zeta) &< n_{p,0} n_p(\zeta) - n_p(\zeta)\\
    \implies && n_i(\zeta) &< -n_{s,0} n_p(\zeta)\\
    \implies && n_i(\zeta) &< 0
\end{align*}
\noindent where we have used $n_p(\zeta)+n_s(\zeta)=1$ and $n_{s,0} n_p(\zeta) > 0$. This is a contradiction since $n_i$ must be greater than $0$, thus: 
\begin{align*}
&& n_{p,0} n_s(\zeta) &\geq 2 n_{2i}(\zeta)\\
\implies && n_s(\zeta) &\geq \frac{2}{n_{p,0}} n_{2i}(\zeta)\\
\implies && n_s(\zeta) &> \gamma_0^2 n_{2i}(\zeta)\\
\implies && Im\{ \lambda(\zeta) \} &= 0 \hspace{0.5cm} \forall \zeta
\end{align*}

\noindent Therefore, the eigenvalues are purely real for all $\zeta$ when $\gamma(\infty)<1$ and the pump and idler modes will oscillate forever.

\subsection{Evaluation of diffraction and beam walk-off in SHA}

We performed an initial analysis of beam propagation effects in order to find the smallest beam size where diffraction and Poynting vector walk-off are negligible for the SHA device considered in our study. Monochromatic fields at frequencies $\omega_{j,0}$ for signal, pump, idler, and idler second harmonic (SH) waves with 1D spatial Gaussian beam profiles were propagated using the four coupled pulse propagation equations for OPA and idler SHG (shown in the spatial Fourier domain) with diffraction and Poynting vector walk-off:
\begin{align}
    d_{z}E_s(k_x) = i& \frac{\omega_{s,0} d_{\text{eff}}}{n_s c} \mathcal{F} \{E_p(x) E_i^*(x)\} - i k_s(\omega_{s,0})E_s(k_x) \nonumber \\
    +& i \frac{k_x^2}{2k_s(\omega_{s,0})} E_s \label{eq:space-s} \\
    d_{z}E_p(k_x) = i& \frac{\omega_{p,0} d_{\text{eff}}}{n_p c} \mathcal{F} \{E_s(x) E_i(x)\} - i k_p(\omega_{p,0})E_p(k_x) \nonumber \\
    +& i \rho_{p} k_x E_{p}(k_x) + i \frac{k_x^2}{2k_p(\omega_{p,0})} E_p(k_x) \label{eq:space-p} \\
    d_{z}E_i(k_x) = i& \frac{\omega_{i,0} d_{\text{eff}}}{n_i c}\mathcal{F} \left\{E_p(x) E_s^*(x) + E_{2i}(x) E_i^*(x)\right\} \nonumber \\ 
    -& i k_i(\omega_{i,0})E_i(k_x) + i \frac{k_x^2}{2k_i(\omega_{i,0})} E_i(k_x) \label{eq:space-i} \\
    d_{z}E_{2i}(k_x) = i& \frac{\omega_{2i,0} d_{\text{eff}}}{2n_{2i} c} \mathcal{F} \{E_i^2(x)\} - i k_{2i}(\omega_{2i,0})E_{2i}(k_x) \nonumber \\
    +& i \rho_{2i} k_x E_{2i}(k_x) + i \frac{k_x^2}{2k_{2i}(\omega_{2i,0})} E_{2i}(k_x) \label{eq:space-2i}
\end{align}
\noindent where  $E_j(k_x)=A_j (k_x)e^{i k_j(\omega_{j,0})z}$, and $A_j$, $k_j$, and $n_j$ are the signal, pump, idler, and idler SH electric field amplitudes, wave vectors in the nonlinear media, and indices of refraction, respectively, and where $k_x$ and $z$ are the transverse spatial and propagation coordinates. The effective quadratic nonlinear coefficient is given by $d_{eff}$ and $c$ is the speed of light. Terms quadratic in $k_x$ represent diffraction in the paraxial regime. Terms linear in $k_x$ represent Poynting vector walk-off, which in CSP is present for pump and idler SH but not signal and idler, as both OPA and SHG processes are Type-I ($o + o = e$). The walk-off angles for the pump and idler SH are given by $\rho_p$ and $\rho_{2i}$, respectively.

Fig. \ref{fig:spatialresults} shows the results for the CSP device, where both diffraction and spatial walk-off are of concern. $\rho_p$ and $\rho_{2i}$ for the 1.03 $\mu$m pump and 3.25 $\mu$m idler SH in this case are 16.64 $\mu$rad and 16.82 $\mu$rad, respectively. For the 2.55 mm optimal crystal length used in the simulation, this corresponds to ~40 $\mu$m of walk-off for the pump. Fig. 1 shows the spatial simulation results of solving Eqs. (S1)-(S4) for three beam sizes, $1/e^2$ beam radii 1, 0.5, and 0.2 mm. For a 1 mm beam radius (the pump beam radius used in the CSP device example in the main text), we find negligible contribution from diffraction and walk-off, resulting in dynamics identical to the simulations where these effects are not included (top two panels). Decreasing to a 0.5 mm beam radius, a very slight deviation appears in the last 10\% of the crystal length, with minor back-conversion setting in just as the crystal terminates. Decreasing even further to a 0.2 mm beam radius, there is no longer a region within the crystal where conversion cycles are fully damped out. These results show that ignoring spatial effects is an excellent approximation for a 1.0 mm pump beam radius and marginal at a 0.5 mm beam radius. At 0.2 mm radius, diffraction and spatial walk-off have a strong effect on the evolution.

This analysis illustrates that spatial effects can disturb the SHA dynamics, preventing uniform spatiotemporal amplification. However, the use of suitably large beams circumvents the problem.

\subsection{Numerical model used for full spatiotemporal analysis of SHA propagation}

For the CSP device simulated in the main text, where diffraction and spatial walk-off were found to have negligible effect as a result of the large beam size, full spatiotemporal evolution (two transverse spatial dimensions plus one temporal dimension) along the propagation axis could be solved independently for each transverse spatial coordinate ($x,y$), with a temporal grid to capture temporal propagation effects. Using a Fourier split-step method, we solved the following four coupled pulse propagation equations for OPA and idler SHG (shown in the frequency domain), accounting for the exact frequency-dependent dispersion, $k_j(\omega)$, given by published Sellmeier equations:
\begin{align}
    d_{z}E_s(\omega) = i& \frac{\omega_{s,0} d_{\text{eff}}}{n_s c} \mathcal{F} \{E_p(t) E_i^*(t)\} - i k_s(\omega)E_s(\omega) \label{eq:time-s} \\
    d_{z}E_p(\omega) = i& \frac{\omega_{p,0} d_{\text{eff}}}{n_p c} \mathcal{F} \{E_s(t) E_i(t)\} - i k_p(\omega)E_p(\omega) \label{eq:time-p} \\
    d_{z}E_i(\omega) = i& \frac{\omega_{i,0} d_{\text{eff}}}{n_i c} \mathcal{F} \left\{E_p(t) E_s^*(t) + E_{2i}(t) E_i^*(t)\right\} \nonumber \\
    -& i k_i(\omega)E_i(\omega) \label{eq:time-i} \\
    d_{z}E_{2i}(\omega) = i& \frac{\omega_{2i,0} d_{\text{eff}}}{2n_{2i} c} \mathcal{F} \{E_i^2(t)\} - i k_{2i}(\omega)E_{2i}(\omega) \label{eq:time-2i}
\end{align}
\noindent where  $E_j (\omega)=A_j (\omega)e^{i k_j (\omega_{j,0})z}$. This model allowed us to capture all non-negligible propagation effects consistent with a collinear geometry. Nonlinear polarization terms beyond quadratic order were not included.


\begin{figure}[htbp]
    \centering
        \includegraphics[width=3.46in]{SHASpatialPropagationResults.pdf}
        \caption{Spatial beam propagation dynamics for Type-I ($ooe$) phase-matched OPA and Idler SHG in CSP at time = 0 for (a) pump beams of radius 1, 0.5, and 0.2 mm (1/$e^2$) and (b) the corresponding signal. The first panels of (a) and (b) show the 1 mm beam radius case where all spatial propagation effects are neglected from the simulation. Simulation parameters correspond to the device parameters in the main text.}
    \label{fig:spatialresults}
\end{figure}

\end{document}