\section{Asymptotic Giant Branch Stars}
\label{sec: Sec 3}
\begin{figure}[t]
    \centering
    \includegraphics{Section3/Figures/Labelled_Kipp2.png}
    \caption{Kippenhahn diagram showing the evolution of the He-B and H-B shells (red), CO core (blue), convective zone (grey) of a $4M_{\odot}$ star throughout the E-AGB. The extent of these regions at time $0.5$~Myr are highlighted by the curly brackets, defined with respect to the radial mass coordinate $M_r$, i.e. the mass enclosed in a spherical shell of radius $r$. Also shown is the H/He discontinuity (green dotted line).}
    \label{fig: 4M Kippenhahn}
\end{figure}


Asymptotic giants are a class of cool, luminous star which have evolved beyond the phase of central helium burning. The asymptotic giant branch (AGB) is an evolutionary stage experienced only by stars in the approximate mass range of $0.8-8M_{\odot}$ \cite{2012sse..book.....K}, which are massive enough for helium burning to occur, but insufficiently massive to support non-degenerate carbon fusion. A comprehensive review can be found at \cite{AGBStarsBook}.





An AGB star has at its centre a core composed of carbon and oxygen, the products of helium fusion. Surrounding the core is a helium-rich layer at the base of which is a shell supporting helium-burning. This shell is a remnant of the previous phase of central He-B. The outer stellar envelope is composed primarily of hydrogen and hosts a convective layer which penetrates from the surface deep within the star and efficiently mixes its contents. A hydrogen-burning shell exists at the bottom of this, which has persisted from the end of the main sequence throughout central He-B.


The evolution of the inner $2M_{\odot}$ of a $4M_{\odot}$ star from the point of exhaustion of central helium is shown in the Kippenhahn diagram Figure \ref{fig: 4M Kippenhahn}. A Kippenhahn diagram depicts changes in stellar structure across evolutionary periods. At a given moment in time, the extent of stellar regions (e.g. the core, convective and burning regions) can be read along the vertical axis and are defined in terms of the radial mass coordinate $M_r$\footnote{Stellar structure equations are typically defined in terms of the radial mass coordinate, i.e. the mass interior to a spherical shell of radius $r$, rather than the radius itself. Spherical symmetry has been assumed.}. For example, when the star has been on the AGB for 0.5 Myr, the innermost $0.25M_{\odot}$, shown in blue, is occupied by the CO core. This is surrounded by the helium-rich zone which extends from $0.25M_{\odot}$ to $\approx0.85M_{\odot}$, with the He-B shell (red) occupying the bottom $0.25M_{\odot}$ of this. The entire region external to this is occupied by the outer hydrogen envelope. The convective layer extends as far down as $M_r\approx1.25M_{\odot}$, and a thin hydrogen-burning shell remains at $M_r\approx0.85M_{\odot}$. The dotted green line indicates the mass coordinate of the boundary between hydrogen- and helium-rich zones.


\begin{figure}[t]
    \centering
    \includegraphics{Section3/Figures/TP_example.pdf}
    \caption{The evolution of the luminosities $L_{\mathrm{H}}$ (blue) and $L_{\mathrm{He}}$ (yellow) throughout a typical pulse cycle for a $4M_{\odot}$ star. Time has been set to zero when quiescent hydrogen burning begins.}
    \label{fig: 4M TP}
\end{figure}

\subsubsection*{The early-AGB:}
The onset of the AGB coincides with considerable structural change within the star. Once the He-B shell has been established surrounding the CO core, its substantial energy output prompts the expansion and cooling of the entire He-rich layer. Consequently, nuclear activity within the superior H-B shell is suppressed and, if $M_{\mathrm{init}}\gtrsim4M_{\odot}$, extinguished. What follows is a period of stable helium shell burning and CO core growth, known as the early-AGB (E-AGB).


For helium fusion to be sustained throughout the E-AGB, the He-B shell must gradually progress outward through the He-rich region. Throughout this process the He-B shell begins to thin, causing its temperature to increase and nuclear activity to intensify. Much of the associated energy-flux drives further expansion and cooling of the outer layers, enabling the convective zone to penetrate more deeply into the stellar envelope.





A critical point is reached when the convective zone reaches the H/He discontinuity ($\sim1.87$ Myr in Figure \ref{fig: 4M Kippenhahn}). In stars with $M_{\mathrm{init}}\lesssim4M_{\odot}$, the still functional H-B shell prevents any deeper incursion of the convective zone and the H/He discontinuity is left unaltered. However in more massive stars, where the H-B shell is dormant, the convective zone breaches the H/He discontinuity and delves into the helium-rich region below, dispersing its contents (namely helium and nitrogen) throughout the outer-envelope. This event is termed the second dredge-up\footnote{The first dredge-up occurs at the end of the main-sequence when a star approaches the red-giant branch.}. Notably the second dredge-up disperses a substantial amount of fuel for the He-B shell, which restricts CO core growth throughout the E-AGB.



\subsubsection*{The thermal pulsating-AGB:}
The E-AGB is brought to a close when the He-B layer approaches the H/He discontinuity and its supply of nuclear fuel dwindles. The associated decline in helium burning activity allows the outer-envelope to contract, reigniting the dormant hydrogen shell. Interestingly, the geometrically thin He-B is thermally unstable, which facilitates the development of pulsations within the star's outer layers. These thermal pulsations characterise the second phase of the AGB, the thermal pulsating-AGB (TP-AGB).


\begin{figure}[t]
\centering
\includegraphics{Section3/Figures/Light_ALP_comp.pdf}
\caption{Comparison between the E-AGB evolution of stellar structure for $4M_{\odot}$ stars with (bottom) and without (top) the inclusion of 10~keV ALPs with $g_{a\gamma\gamma}=0.63\times10^{-10}$~GeV$^{-1}$. All elements of the plot are defined in Figure \ref{fig: 4M Kippenhahn}.}
\label{fig: Low Mass AGB Comp}
\end{figure}
\begin{figure}[t]
    \centering
    \includegraphics{Section3/Figures/TP_comparison.pdf}
    \caption{Comparison between the thermal pulses of the TP-AGB for $4M_{\odot}$ stars with $g_{10}=g_{a\gamma\gamma}/(10^{-10}\ GeV^{-1})=0.0$ (top) and 0.63 (bottom). The hydrogen and helium luminosities are indicated by solid and dotted lines respectively.}
    \label{fig: TP Comparison}
\end{figure}

A typical pulse cycle, illustrated in terms of the hydrogen (blue) and helium (yellow) luminosities, is shown in Figure \ref{fig: 4M TP} for the same $4M_{\odot}$ star whose evolution is illustrated in Figure \ref{fig: 4M Kippenhahn}. Sparse fuel supply in the He-B shell causes nuclear activity therein to dwindle, giving way to a long period of \textit{quiescent} hydrogen shell burning (the inter-pulse period). The helium produced during this time settles onto the helium-rich region below, increasing its mass and causing the pressure and temperature at its base to rise. 

Once the mass of this inter-shell region reaches a certain threshold, helium is re-ignited in an unstable event known as the \textit{helium shell flash}. Such flashes are brief, occurring on scales of $\mathcal{O}(1\ \mathrm{yr})$, and suppress hydrogen-shell burning. They are followed by a period of stable He-B which is sustained for a few hundred years. Once its fuel has been exhausted, nuclear activity within the He-B shell again diminishes, giving way to quiescent H-burning. The duration of the inter-pulse period varies with the core mass, with more massive cores supporting more rapid pulsations \cite{Paczynski-IP-MC-reln, Boothroyd-core-IP-reln}.



Many thermal pulsations occur during the TP-AGB, each of which produces a non-trivial amount of helium, carbon and oxygen which increase the masses of the CO and hydrogen-depleted cores (with boundary defined by the H/He discontinuity) outwards. These can be accompanied by further dredge-up events, in which the convective layer again penetrates into a region containing the ashes of helium-burning (the third dredge-up), which reduces the growth of the cores \cite{2004ApJ...605..425H}. The TP-AGB, and indeed the AGB itself, is ultimately halted by strong stellar wind, which progressively strips the outer envelope and leaves only the remnant white dwarf. In low-mass stars ($\lesssim3M_{\odot}$), the TP-AGB is sufficiently long for thermal pulses to contribute to the final stellar mass $M_{\mathrm{f}}$ by as much as 30\% \cite{Cummings_2019}. More massive stars, however, shed their envelopes much more rapidly and experience only marginal core growth during the TP-AGB \cite{10.1093/mnras/stt1034}. For the $4M_{\odot}$ star shown in Figure \ref{fig: 4M Kippenhahn}, the hydrogen-depleted core grows only from $0.81M_{\odot}$ to $0.825M_{\odot}$, an increase of approximately 2\%. 


%%%%%%%%%%%%%%%%%%%%%%%%%%%%%%%%%%%%%%%%%%%%%%%%%%%%%%%%%%%%%%%
\subsection{Axion-like particles and the AGB}
%%%%%%%%%%%%%%%%%%%%%%%%%%%%%%%%%%%%%%%%%%%%%%%%%%%%%%%%%%%%%5

% AGB stars and ALPs
To probe the impact of ALPs on the AGB, evolutionary simulations of $4M_{\odot}$ stars were computed for two choices of $m_a$ and $g_{a\gamma\gamma}$ values. We distinguish between the cases of low and high mass ALPs below. The first case corroborates the results presented in \cite{Dominguez}, though we include only the ALP-photon interaction. We then show that this behaviour persists to heavier ALPs.



\subsubsection*{Light ALPS:}


The E-AGB phase from the simulation with $g_{a\gamma\gamma}=0.63\times10^{-10}$ GeV$^{-1}$ and $m_a=10\ \mathrm{keV}$ is shown in the lower panel of Figure \ref{fig: Low Mass AGB Comp}. The upper panel contains the same phase given standard astrophysics alone. The inclusion of these ALPs within the simulation expedites the E-AGB. 


The primary culprit for this is ALP-production and escape in the He-B shell. Energy loss within this region prompts it to contract and heat, which accelerates nuclear fuel consumption and causes the entire evolutionary phase to occur more rapidly. More intense nuclear burning also sparks a premature and deeper second dredge-up event, which displaces a greater mass of helium-rich material throughout the convective region. This naturally increases the surface abundances of the remnants of nuclear burning (He and $^{14}$N). A deeper dredge-up event also results in a smaller H-depleted core mass $M_c$, which reduces possible CO core growth during the E-AGB. This is precisely what is seen in our models, where the terminal E-AGB value of $M_c$ and $M_{CO}$ decreases from $0.81M_{\odot}$ and $0.79M_{\odot}$ to $0.72M_{\odot}$ and $0.69M_{\odot}$ respectively.

\begin{figure}[t]
    \centering
    \includegraphics[width=8.5cm]{Section3/Figures/Heavy_ALP_comp.pdf}
    \caption{Comparison between the E-AGB evolution of stellar structure for $4M_{\odot}$ stars with (bottom) and without (top) the inclusion of 316keV ALPs with $g_{a\gamma\gamma}=10^{-9}$ GeV$^{-1}$. This choice of ALP parameters is well outside the HB star bound in Figure \ref{fig: ALP_param_space}.}
    \label{fig: 4M comparison}
\end{figure}

The addition of ALPs also disrupts the evolution of the TP-AGB. As shown in Figure \ref{fig: TP Comparison}, when ALPs are included in the model, the length of the inter-pulse period increases substantially. This is primarily caused by the decreased core mass at the onset of pulsation \cite{Dominguez}. Furthermore, energy-loss within the He-B shell during these pulses produces more extreme third dredge-up events. This produces a potentially observable signature, as the surface abundances of the products of helium-burning increase \cite{Dominguez}. Potential ramifications of lengthening the inter-pulse period as well as deeper third dredge-up events are discussed in Section \ref{sec: Section 5}. 



\subsubsection*{Heavy ALPs:}

In order to investigate the impact of heavier ALPs we recompute the evolution of the $4M_{\odot}$ star with $m_a=316$ keV and $g_{a\gamma\gamma}=10^{-9}$ GeV$^{-1}$, well outside the region constrained by HB stars. The E-AGB evolution of this star is shown in Figure \ref{fig: 4M comparison}.



The production of heavier ALPs in the He-B shell is Boltzmann suppressed and consequently there is minimal reduction in the duration of the early asymptotic giant branch phase. Like the light ALP case, however, the second dredge-up penetrates more deeply into the helium-rich layer and, consequently, an observed reduction in $M_{c}$ and $M_{\mathrm{CO}}$ is retained ($0.81M_{\odot}$ and $0.79M_{\odot}$ to $0.77M_{\odot}$ $0.75M_{\odot}$), though this effect is of moderate strength only.

\begin{figure}[t]
\centering
\begin{minipage}[t]{.5\textwidth}
    \centering
    \captionsetup{width=0.9\textwidth}
    \includegraphics[width = 7.5cm]{Section3/Figures/Temp_Comp.pdf}
    \caption{The evolution of temperature at the location of maximum helium burning in models of $4M_{\odot}$ and $6M_{\odot}$ AGB stars as well as a $0.8M_{\odot}$ horizontal branch star in the absence of ALPs. Evolution is depicted as a fraction of the total duration of the phase.}
    \label{fig: Temperature Evolution}
\end{minipage}%
\begin{minipage}[t]{.5\textwidth}
    \centering
    \captionsetup{width=0.9\textwidth}
    \includegraphics[width=7.5cm]{Section3/Figures/Core_growth_comp.pdf}
    \caption{The mass coordinate of the H/He discontinuity (solid) and $M_{\mathrm{CO}}$ (dashed) in $4M_{\odot}$ stars for 316 keV ALPs with $g_{a\gamma\gamma}=3.16\times10^{-9}$ GeV$^{-1}$ (pink), $g_{a\gamma\gamma}=3.16\times10^{-8}$ GeV$^{-1}$ (yellow) and with the ALP-photon interaction switched off (dark blue).}
    \label{fig: Other tracks}
\end{minipage}
\end{figure}

The persistence of ALP effects on the second dredge-up can be understood by examining the evolution of temperature in the He-B layer during the E-AGB, indicated by the pink line in Figure \ref{fig: Temperature Evolution}. For the majority of the E-AGB the temperature of the shell is below $T=10^{8.2}\ \mathrm{K}\approx 1.6\times10^{8}\ \mathrm{K}$ and little energy-loss to ALP-production occurs (cf. Figure \ref{fig: Eps_masscomp}). As the layer thins, its temperature steadily increases, alleviating the Boltzmann suppression of ALP production. This triggers the positive-feedback loop, which facilitates a more intense spike in helium luminosity and the rapid establishment of a deep dredge-up event.


Figure \ref{fig: Other tracks} depicts the evolution of the H/He discontinuity (solid line) and $M_{\mathrm{CO}}$ (dashed) for 316 keV ALPs with couplings strengths of $g_{a\gamma\gamma}=3.16\times10^{-9}$ GeV$^{-1}$ (purple), $g_{a\gamma\gamma}=3.16\times10^{-8}$ GeV$^{-1}$ (yellow) and with ALP-production switched off (dark blue). As expected, increasing the value of  $g_{a\gamma\gamma}$ causes $M_{\mathrm{CO}}$ and $M_c$ to decrease. Notably, approximately the same reduction in CO core mass is obtained when both 10 keV ALPs with $g_{a\gamma\gamma}=0.63\times10^{-10}$ GeV$^{-1}$ and 316 keV ALPs with $g_{a\gamma\gamma}=3.16\times10^{-9}$ GeV$^{-1}$ are included in the model. However, the latter only shortens the E-AGB duration by 13\%, rather than 41\% in the case of the former. This confirms that heavy ALPs, which only become relevant towards the end of the E-AGB, nevertheless impact stellar structure significantly.


The He-B shells of more massive stars, which comprise the upper-end of the IFMR (e.g. the $6M_{\odot}$ star in Figure \ref{fig: Temperature Evolution}), reach higher temperatures still during the E-AGB. Such objects should therefore show even greater sensitivity to heavy ALPs. This is precisely what we see in our simulations, which predict a 16\% reduction $M_{\mathrm{CO}}$ and a 46\% reduction in E-AGB duration for a $6M_{\odot}$ star experiencing energy-loss to 316 keV ALPs with $g_{a\gamma\gamma}=10^{-9}$ GeV$^{-1}$, compared with 5\% and 1\% respectively for the $4M_{\odot}$ stars in Figure \ref{fig: 4M comparison}.


Compared with the He-B shell of AGB stars, the cores of the $0.8M_{\odot}$ HB stars simulated in \cite{Carenza:2020zil} are cooler across all stages of their respective evolutionary phases (see the yellow line in Figure \ref{fig: Temperature Evolution}). It is precisely this which gives AGB stars great potential to further constrain the cosmological triangle. Although the HB star bound has been consistently refined over time, any constraint we derive based on the established effects on $M_{\mathrm{f}}$ will simply be sensitive to heavier ALPs.


It should be noted that ALPs retain the capacity to elongate the inter-pulse period during the TP-AGB, as this is principally a function of $M_{c}$. However, this does not significantly influence the constraint derived in Section \ref{sec: Section 4} and as such discussion of its potential constraining power is deferred to Section \ref{sec: Section 5}. 
