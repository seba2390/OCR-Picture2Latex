\section{Details of Simulations}

\subsection{Treatment of energy-loss in \texttt{MESA}}
\label{Sec: AppA}
Our treatment of energy-loss to ALP-production within \texttt{MESA} followed the example set by \cite{Friedland:2012hj}, which augmented the module responsible for thermal neutrino rates (\texttt{neu}) and added an additional term corresponding to $\epsilon_a$. The rationale behind this was simply that power loss to neutrinos and freely escaping ALPs yield virtually identical phenomenological effects on stellar evolution, in that they both contribute a negative term to the total energy production rate $\epsilon$.


The functions $F(\xi^2, \mu^2)$ and $G(\mu^2)$, defined in Equations \ref{eq: F function} and \ref{eq: G function} respectively, contain the entire chemical composition and ALP mass dependence of $\epsilon_a$. Recall that $\xi=k_s/(2T)$ and $\mu=m_a/T$. We took the sum of these functions
\begin{equation}
    H(\xi^2, \mu^2)=F(\xi^2, \mu^2)+G(\mu^2) 
\end{equation}
and evaluated it at fixed points, logarithmically spread in $\xi$ and $\mu$, to define a grid which is loaded the first time the \texttt{neu} routine is called. Within each cell of the \texttt{MESA} model, our augmented \texttt{neu} routine determines the appropriate values of $\xi$ and $\mu$ (cell average properties) and interpolates between our grid points to find the corresponding value of $H(\xi^2, \mu^2)$ and hence $\epsilon_a$. This is added to the energy-loss rate to thermal neutrinos, as determined by the standard \texttt{neu} routine.

The aforementioned interpolation is carried out using the PSPLINE\footnote{\url{https://w3.pppl.gov/\~pshare/help/pspline.htm}} bicubic spline algorithm in \texttt{MESA}'s own \texttt{interp\_2d} routine (see \cite{MESA1} for more details). We compared the $\epsilon_a$ values in several of our models to its numerically evaluated counterpart and found it matched to within 0.2\%.
 
As per the \texttt{MESA} manifesto, we have made our \texttt{run\_star\_extras} file, which contains the modified \texttt{neu} routine, as well as our grid of pre-calculated $H(\xi^2, \mu^2)$ values available for download\footnote{\url{https://github.com/fhiskens/MESA\_ALPs}}.

\subsection{Adopted input physics}
\label{sec: AppB}

In addition to making our extension to MESA publicly available, we also include the \texttt{inlist} containing our input physics, which has been taken from the MIST project. The details of these choices as well as their motivation can be found in \cite{MIST0,MIST1}. As all simulations have initial masses below $10M_{\odot}$, we choose the prescription intended for low and intermediate mass stars. Aspects of our \texttt{run\_star\_extras} file also originated with MIST, particularly those which assist with conversion of the stellar model and the switching of boundary conditions after 100 steps of the simulation.


\texttt{MESA} was compiled using the \texttt{MESA} SDK\footnote{\url{http://www.astro.wisc.edu/~townsend/static.php?ref=mesasdk}} \cite{richard_townsend_2019_2669541}. All results were analysed using the \texttt{mesa\_reader} Python package\footnote{\url{https://github.com/wmwolf/py\_mesa\_reader}}.