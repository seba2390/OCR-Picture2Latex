\section{Beyond the Initial-Final Mass Relation}
\label{sec: Section 5}


In Section \ref{sec: Section 4} we elected to use the white dwarf initial-final mass relation as the basis of our constraint. However, the behaviour discussed in Section \ref{sec: Sec 3} has other potential observable effects. Two of these, which were proposed in \cite{Dominguez:2017mia}, reference the impact ALPs can have on the ultimate fates of stars.




The final state into which a star evolves depends sensitively on the nature of carbon fusion, which is principally governed by the CO core mass. Stars with $M_{\mathrm{CO}}<1.06M_{\odot}$ never reach the requisite conditions for carbon ignition and end their lives as CO white dwarfs. Alternatively, if $1.06M_{\odot}<M_{\mathrm{CO}}<1.38M_{\odot}$, the CO core becomes partially degenerate and carbon ignition occurs in an off-centre flash. The inner boundary of this burning region then advances to the centre of the star, paving the way for stable carbon burning and the development of an oxygen-neon (O-Ne) core. Such stars are termed \textit{Super-AGB} stars and are the progenitors for O-Ne white dwarfs. If the CO core is still more massive after the exhaustion of central helium ($M_{\mathrm{CO}}>1.38M_{\odot}$), carbon is ignited in a stable, convective core. Such stars experience all further episodes of nuclear burning and are the progenitors of core collapse supernovae (CCSN)\footnote{Not all stars in this mass range do experience a supernova (see e.g. \cite{2009ARA&A..47...63S}), however this is recognised as the minimum CO core mass for such an event.}.



A great deal of work has been conducted in astrophysics to identify the masses, $M_{\mathrm{up}}$ and $M_{\mathrm{up}}'$, which correspond to the minimum mass of O-Ne white dwarf and CCSN progenitors respectively (for a review see \cite{2009ARA&A..47...63S}). These values naturally vary between models, though it is believed that they lie between $7$-$8M_{\odot}$ and $10$-$12M_{\odot}$ respectively \cite{2012sse..book.....K}.



As the predicted value of $M_{\mathrm{CO}}$ varies significantly when ALPs are included in the stellar model (see Section \ref{sec: Sec 3}), so too do the values of $M_{\mathrm{up}}$ and $M_{\mathrm{up}}'$. Specifically, given that ALP-production reduces the mass of $M_{\mathrm{CO}}$, much larger initial masses are required for the formation of O-Ne WD and CCSN progenitors. This impact, in the context of ALPs below the keV-MeV scale, is the subject of \cite{Dominguez:2017mia}, wherein it is suggested that observational upper limits on $M_{\mathrm{up}}'$, or the rates of Type Ia SN could be used to constrain ALPs. Here we shall briefly comment on the prospects and potential pitfalls of constructing a constraint on ALPs based on these and two other pieces of observational evidence.



\subsubsection*{Core Collapse Supernova Progenitors:}
Theoretical values of $M_{\mathrm{up}}'$, the minimum mass of CCSN progenitors, vary between $10$-$12M_{\odot}$. As such stars reach temperatures substantially higher than in the He-B shells of AGB stars, constructing a constraint based on heavy ALP production in their interiors is an encouraging prospect. However, whether or not such observational constraints can be used depends on the method of their construction. There appear to be two sources of such constraints used in the literature.


The first of these is via the investigation of pre-explosion images. If a CCSN is detected in a region which has previously been photographed, the progenitor candidate can be analysed and an initial mass estimated through a theoretical initial mass-final luminosity relation. In \cite{Smartt2009} this led to a constraint of $8^{+1.0}_{-1.5}M_{\odot}$. The initial mass-final luminosity relation, however, relies on the predictions of stellar evolution models during the AGB and have already been shown to vary when ALP-production is included in the simulations \cite{Straniero:2019dtm}. To prevent self-consistency issues, the analysis in \cite{Smartt2009} would have to be re-derived using an initial mass-final luminosity relation which takes the ALP-dependence into account.




Another source of constraints on $M_{\mathrm{up}}'$ comes from the analysis of supernova remnants (SNRs), e.g. \cite{Diaz-Rodriguez2018-SNRs} which finds $M_{\mathrm{up}}'=7.33^{+0.02}_{-0.16}M_{\odot}$. Key components used in the derivation of this limit are star formation histories (SFHs), which require the use of stellar isochrones. As these isochrones are themselves derived from stellar modelling, it would be necessary to ascertain the degree to which these depend on ALP properties before this constraint could be used. As a minimal requirement, consistency demands that the SFHs be determined using isochrones derived from the same stellar evolution code used to analyse the impact of ALPs (in our case the MIST isochrones). This is further complicated by the dependence of $M_{\mathrm{up}}$ and $M_{\mathrm{up}}'$ on the $^{12}$C+$^{12}$C reaction rate, which is presently a source of great uncertainty in stellar simulations \cite{inbook}.

\begin{figure}[t]
    \centering
    \includegraphics{Section6/Param_concl.pdf}
    \caption{The state of the ALP plane in the keV-MeV region including our constraint (Section \ref{sec: Section 4}), the modified luminosity SN1987A constraint (shaded purple) \cite{Lucente:2020whw}. The projected sensitivity of Belle II \cite{Dolan:2017osp}, and DUNE liquid (LAr) and gaseous argon (GAr) detectors \cite{Brdar:2020dpr} are indicated by the region above the dashed black, blue and red lines respectively.}
    \label{fig: ALP-Decay Constraint}
\end{figure}

\subsubsection*{Type IA Supernova Rates:}
Type Ia SN are believed to occur when the mass of a CO white dwarf in a binary system exceeds the Chandrasekhar limit due to accretion from a main sequence or giant partner (single degenerate pathway) or a merger with a second white dwarf (double degenerate pathway). For a detailed review of these pathways see \cite{WANG2012122}. 

Given the inclusion of ALPs in stellar models leads to larger $M_{\mathrm{up}}$ values, their existence would favour greater populations of CO white dwarfs and, consequently, a higher incidence of Type Ia SN. It is therefore conceivable that bounds on the latter could enable the properties of ALPs to be constrained or, as suggested in \cite{Dominguez:2017mia}, might even hint to their existence.


Constraints on Type Ia SN are typically presented in terms of their delay time distribution, i.e. their rate of incidence as a function of time following a normalised burst of star formation (see \cite{doi:10.1146/annurev-astro-082812-141031} for a more detailed description). It would be interesting to explore the manner in which ALP-production affects the predicted shape of the delay time distribution. Such an analysis would, naturally, require ALPs to be included in binary stellar modelling. A complete analysis on the topic would naturally have to investigate both production pathways.


\subsubsection*{Mira Variable Drifting:}
Mira variable stars are a sub-class of asymptotic giants which are unstable to radial pulsations with periods that are $\mathcal{O}(100\ \mathrm{days})$. Certain Mira variables have descriptive data which go back over a century. As a result, it has been possible to detect period-drifting in these stars, the magnitude of which can be stark. For instance, the period of R Hya has changed from approximately 500 days, as measured in 1700, to 387 at the year 2000 \cite{AGBStarsBook}. Similarly, R Aql's period has decreased from approximately 320 to 267 days in recordings between 1915 and 2000.


It was shown in \cite{WoodMiras1981} that this drifting is consistent with helium shell flashes occurring during the larger-scale thermal pulsations of the TP-AGB, though this is still an open area of debate in the literature \cite{NeilsonMiraPeriodChange}. It has previously been shown \cite{Dominguez}, and we confirmed in Section \ref{sec: Sec 3}, that the thermal pulses of stars with a given initial mass vary when ALPs are included in the model. It would be interesting to investigate whether these changes introduce tension between the agreement of period-drifting and long-term evolution. This likely requires the inclusion of ALP-production within dedicated TP-AGB models.



\subsubsection*{Abundance Ratios:}
A further consequence of the inclusion of ALP-production in stellar modelling is the impact it has on elemental abundance ratios within the star. By accelerating periods of helium burning (both central and shell), the relative abundance of carbon and oxygen in the core is likely to change. A deeper dredge-up event is also likely to increase the presence of fusion products in the surface composition. In fact, the presence of light ALPs in the late-evolutionary phases of $16M_{\odot}$ has already been found to dramatically increase the abundance of oxygen, magnesium and neon for values of $g_{a\gamma\gamma}$ as low as $10^{-11}~\mathrm{GeV}^{-1}$ \cite{Aoyama:2015asa} in the case of the latter. This is likely exacerbated by ALP-enhanced second and third dredge-up events which increase the presence of fusion products at the stellar photosphere.



Observational measurements of these ratios could therefore constitute a potent source of constraints on ALP parameter space. Unfortunately information about AGB stars themselves is somewhat scarce, with examples limited to individual post-AGB stars, e.g. \cite{PostAGBSpec}. This could be mitigated through comparison of theory with abundance constraints from spectroscopic analysis of white dwarfs such as \cite{Coutu_2019} or via the investigation of the composition of planetary nebula (see for example \cite{PlanNebAGB, karakas_lattanzio_2003}).


It is likely, owing to the sensitivity of dredge-up events to the adopted prescription of core overshoot (see Appendix \ref{sec: AppC}) that this would introduce significant systematic uncertainty. Despite this, the use of spectroscopic analysis is a particularly compelling prospect, as it does not rely on any of the results of stellar evolution theory, circumventing any issues of self-consistency.