\section{Introduction}
\thispagestyle{plain}
\begin{figure}[b]
    \centering
    \includegraphics{Introduction/Figures/ALP_Param.pdf}
    \caption{Constraints on ALP mass $m_a$ and coupling strength to photons $g_{a\gamma\gamma}$ in the keV-MeV mass range. Individual bounds are referenced in the text. These are shown at 95\% confidence level. The constraint derived in this work is labelled 'WD-IFMR'.}
    \label{fig: ALP_param_space}  
\end{figure}

Axion-like particles (ALPs) are light, weakly interacting pseudoscalars which feature in many extensions of the Standard Model (SM) of particle physics. They arise as pseudo-Nambu Goldstone bosons (pNGBs) of spontaneously broken symmetries in, for example, the Peccei-Quinn solution of the strong CP problem \cite{PQ1, Peccei:1977ur, Weinberg-40.223, Wilczek:1977pj}, compactification scenarios in string theory \cite{Svrcek:2006yi, Arvanitaki:2009fg, Cicoli:2012sz} and in models of electroweak relaxation \cite{Graham:2015cka}. 




The properties of specific ALPs, such as their masses and coupling strengths to SM particles, are model-dependent, which has sparked investigations of their influence in a wide phenomenological range. Light ALPs with masses below the MeV scale impact astrophysical and cosmological phenomena \cite{Cadamuro:2011fd}, such as Big Bang Nucleosynthesis (BBN) \cite{Updated_BBN}, the Cosmic Microwave Background (CMB) and stellar evolution \cite{Raffelt-Bounds-on-light, RAFFELT1982323, Raffelt:1996wa, Ayala:2014pea, Aoyama:2015asa, Carenza:2020zil, Friedland:2012hj, Dominguez, Dominguez:2017mia}. As pNGBs, they can be naturally light and weakly interacting, which makes them ideal candidates for cold dark matter (DM) \cite{Preskill:1982cy, Abbott:1982af, Dine:1982ah, Arias:2012az}. 





ALPs in the MeV to GeV range, however, are generally too massive to significantly influence cosmology and astrophysics, yet are relevant in aspects of particle physics. It has been suggested that ALPs may contribute to the anomalous muon magnetic moment \cite{Chang:2000ii, Bauer:2017nlg, Marciano:2016yhf}, or act as a portal between the dark sector and SM particles \cite{Nomura:2008ru}. Theoretical and experimental interest in ALPs has risen significantly with the suggestion that DM ALPs can explain the recent $3.5\sigma$ excess in electron-recoil measured at the XENON1T experiment \cite{Athron:2020maw, Takahashi:2020uio, Takahashi:2020bpq}.



In this work our attention is limited to ALPs which couple exclusively to photons via the interaction
\begin{equation}
    \label{eq: ALP-Photon interaction}
    \mathcal{L}_a=-\frac{g_{a\gamma\gamma}}{4}F_{\mu\nu}\Tilde{F}^{\mu\nu}a,
\end{equation}
where $g_{a\gamma\gamma}$ is the ALP-photon coupling strength, $a$ is the ALP-field, $F_{\mu\nu}$ is the electromagnetic field-strength tensor and $\tilde{F}^{\mu\nu}$ its dual. Specifically, we consider such ALPs with masses $m_a$ in the keV-MeV range, the $m_a$-$g_{a\gamma\gamma}$ parameter space of which is shown in Figure \ref{fig: ALP_param_space}. Constraints for this region arise from stellar evolution \cite{Carenza:2020zil}, results at electron beam dumps, in particular the SLAC E137 experiment \cite{Bjorken87, Dolan:2017osp} and the neutrino signal associated with the cooling of SN1987A \cite{Lucente:2020whw} as well as the visible signal associated with a subsequent decay of ALPs into photons \cite{Jaeckel:2017tud}.




These bounds fail to exclude a small triangular region at $g_{a\gamma\gamma}\sim10^{-5}\ \mathrm{GeV}^{-1}$ and $m_a\sim1\ \mathrm{MeV}$, referred to as the \textit{cosmological triangle}. While constraints derived from BBN exclude the cosmological triangle \cite{Cadamuro:2011fd, Updated_BBN}, these relax significantly in certain scenarios of non-standard cosmology \cite{Dolan:2017osp, Depta:2020wmr}. Furthermore, as several approaching experiments will have the capacity to directly probe the cosmological triangle \cite{Dolan:2017osp, Brdar:2020dpr}, it is timely to investigate  phenomenological implications of ALPs within this region. In this paper we revisit the effects ALPs can exert on stellar evolution.







Over nearly four decades, stellar evolution has been frequently deployed to constrain ALPs via the so-called \textit{energy-loss argument}. The general mechanism by which this operates is as follows, though a more detailed account of its effects on stellar structure can be found in \cite{Raffelt:1996wa} and specific production process will be discussed in Section \ref{sec: Section 2}. If sufficiently light and weakly interacting, ALPs produced in stellar interiors can freely escape the star and act as a local energy-sink in that region. This merely results in cooling of the stellar plasma if the zone in which production occurred hosts no nuclear activity. If, on the other hand, this region is undergoing nuclear burning, the star accounts for the energy deficit by contracting and heating, which drives the intensity of fusion upwards. As a result both the rate of consumption of nuclear fuel and the energy-loss rate associated with ALP-production increase and a positive feedback mechanism is defined, which ultimately accelerates the progression of the entire evolutionary phase. For sufficiently strongly interacting ALPs, this effect introduces a contradiction between theory and observation, which leads to a constraint.



The most stringent stellar energy-loss constraints on ALPs interacting with photons alone have been derived from observation of horizontal branch (HB) stars \cite{Raffelt-Bounds-on-light, Raffelt:1996wa, Cadamuro:2011fd, Ayala:2014pea, Carenza:2020zil}. Specifically, population studies in globular clusters (large gravitationally bound collections of old, metal-poor stars) place limits on the helium-burning lifetime $\tau_{\mathrm{He}}$ of $0.8M_{\odot}$ HB stars. If, for a given choice of $m_a$ and $g_{a\gamma\gamma}$, ALP energy-loss causes $\tau_{\mathrm{He}}$ to fall below the observed lower bound, it can be excluded. This is exactly the basis of the HB star constraint in Figure \ref{fig: ALP_param_space}. Notably this relaxes as $m_a$ increases beyond the HB star core temperature of 10 keV, owing to the Boltzmann suppression of ALP-production.


However, ALPs are known to influence many more aspects of stellar evolution than just HB stars. ALPs as massive as 100 MeV can drain energy from the cores of supernovae (SN). The magnitude of novel energy-loss, however, is constrained by the SN1987A neutrino signal, the standard astrophysical source of supernova cooling. The SN1987A ALP constraint shown in Figure \ref{fig: ALP_param_space} was recently computed using a state-of-the-art SN model in \cite{Lucente:2020whw}. Note that it does not extend to arbitrarily high values of $g_{a\gamma\gamma}$. Instead, ALPs become trapped within the supernova core and contribute towards energy transfer. The constraint relaxes when this ALP energy transfer falls below that of neutrinos. This defines the lower $g_{a\gamma\gamma}$ boundary of the cosmological triangle and motivates the search for complementary constraints.



Beyond these, it is also known that ALP production can alter chemical abundances during nucleosynthesis in core collapse supernova progenitors \cite{Aoyama:2015asa}, prevent the blue loop evolutionary phase from occurring in intermediate mass helium-burning stars \cite{Friedland:2012hj}, and cause significant structural changes in late-life intermediate mass stars on the asymptotic giant branch (AGB) \cite{Dominguez, Dominguez:2017mia}. These investigations, however, have been restricted to low-mass ALPs ($m_a\lesssim10$ keV) and their potential for constraining MeV-scale ALPs has never been assessed. 


We rectify this by exploring the influence of keV-MeV mass ALPs on stars on the asymptotic giant branch, the late-life evolutionary phase of stars with masses $\lesssim8M_{\odot}$. It has long been recognised that the production of low-mass ALPs within the helium-burning (He-B) shell of such stars can greatly affect their final mass $M_\mathrm{f}$ \cite{Dominguez}. Though this tendency is likely to have observable consequences for white dwarfs (WDs) and core collapse supernova (CCSN) progenitors, it has not led to the construction of a robust ALP constraint. The He-B shells of intermediate mass AGB stars, however, are typically hotter than the cores of HB stars in globular clusters, making them an enticing prospect for further constraining the cosmological triangle.




We simulate stellar evolution with and without energy-loss to massive ALPs to establish the sensitivity of $M_{\mathrm{f}}$ in a region of the ALP-plane unrestricted by the HB star bound, using an edited version of the open source, 1-D stellar evolution code \textit{Modules for Experiments in Stellar Astrophysics} (\texttt{MESA})  \cite{MESA1, MESA2, MESA3, MESA4, MESA5}. \texttt{MESA} is a suite of modules containing up-to-date astrophysics such as opacity tables, nuclear reaction rates and equations-of-state. Its stellar evolution module \texttt{MESAstar} has shown remarkable versatility at modelling stars over a wide range of initial masses. It has been widely tested and compared to astrophysical observation and other stellar evolution codes.



A constraint on ALPs based on reducing $M_{\mathrm{f}}$ is then established by using the white dwarf initial-final mass relation (IFMR). The IFMR maps the initial mass with which a star forms $M_{\mathrm{init}}$ to the final mass of the white dwarf into which it ultimately evolves. It is used in age and distance determination in globular clusters and informs our understanding of supernovae rates \cite{Greggio-TypeIa}, galactic chemical evolution and the field white dwarf population. Numerous constraints on the IFMR exist \cite{Weidemann2000, Kalirai_2008, Williams2009, Andrews, cummings_2016, El-Badry, Cummings_2018} and have previously been used to restrict stellar physics on the AGB \cite{Cummings_2019}. We redeploy one of these constraints, derived from wide double white dwarf binaries, to produce a robust bound on ALPs derived from AGB stars.


This paper has the following structure. In Section \ref{sec: Section 2} we describe the mechanisms of ALP photo-production in a stellar plasma. We then use the results of our simulations in \texttt{MESA} to explore the impact of MeV scale ALPs on the AGB in Section \ref{sec: Sec 3}. In Section \ref{sec: Section 4} we construct our constraint, account for the possibility of ALP-decay and detail some of the systematic uncertainties affecting our analysis. Other possible observational constraints are mentioned in Section \ref{sec: Section 5} before we summarise and conclude in Section \ref{sec: Section 6}. We describe our treatment of ALP energy-loss in MESA, as well as our adopted input physics in Appendices \ref{Sec: AppA} and \ref{sec: AppB} respectively. Further discussion of the systematic uncertainties relevant to this work is included in Appendix \ref{sec: AppC} while Appendix \ref{sec: AppD} includes an alternative probabilistic approach to deriving a constraint.