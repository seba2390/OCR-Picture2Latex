\section{The White Dwarf Initial-Final Mass Relation}
\label{sec: Section 4}




The IFMR relates the initial mass with which a star forms to the mass of the white dwarf into which it ultimately evolves. IFMRs calibrated to observation therefore provide a constraint on total mass loss throughout a stellar lifetime, as well as free parameters of stellar modelling \cite{KaliraiMAssLoss, Cummings_2019}. The IFMR is instrumental in age and distance determination in globular clusters, our understanding of supernovae rates \cite{Greggio-TypeIa}, galactic chemical evolution and the field white dwarf population \cite{Kalirai:2007tq}. 


%%%%%%%%%%%%%%%%%%%%%%%%%%%%%%%%%%%%%%%%%%%%%%%%%%%
\subsection{Constraints on the IFMR}
%%%%%%%%%%%%%%%%%%%%%%%%%%%%%%%%%%%%%%%%%%%%%%%%%%%

Numerous constraints on the IFMR exist, the majority of which are derived using WDs in star clusters (e.g. \cite{Catalan2008, Kalirai_2008, Salaris_2009, Cummings_2018}). We provide a general description of the construction of these star cluster IFMRs, though a more detailed account can be found in \cite{Cummings_2018}.
\begin{enumerate}
    \item Spectroscopic analysis of the WDs enables their effective temperatures $T_{\mathrm{eff}}$ and surface gravity $\log(g)$ to be derived. These can be converted to white dwarf masses $M_{\mathrm{f}}$ and cooling age $\tau_C$ via application of theoretical white dwarf cooling models.
    \item If the age of the cluster $\tau_{\mathrm{SC}}$ which hosts the WD  is known, the progenitor lifetime can be determined as $\tau_{P}=\tau_{\mathrm{SC}}-\tau_{C}$. Cluster age determination is typically achieved through use of isochrone\footnote{ Isochrones are a complementary tool to the evolutionary models discussed thus far. While stellar tracks report information pertaining to the evolution of a single star with given initial mass \(M_{\textrm{init}}\) and metallicity $Z$, isochrones detail the properties of a cluster of items at a fixed age, as a function of their mass. Implicit in their use is the assumption that all objects described by a single isochrone have formed out of the same homogeneous gas cloud, and consequently have an identical composition.} fitting.
    \item Finally stellar evolution models of appropriate metallicity are used to determine the initial stellar mass $M_{\mathrm{init}}$ associated with the progenitor lifetime $\tau_P$. Repeating this process over an entire sample of WDs yields an IFMR calibrated to observation.
\end{enumerate}
The central values of two star-cluster IFMRs derived in \cite{Cummings_2018} are shown in Figure \ref{fig: IFMR Example}. In this analysis two different sets of isochrones and stellar evolution models were used, which produces the observed differences at large values of $M_{\mathrm{init}}$. These are the PARSEC isochrones \cite{ParsecIsochrones}, computed from the Padova stellar evolution models, as well as those of \texttt{MESA} Isochrones and Stellar Tracks (MIST) \cite{MIST0, MIST1} which are based on \texttt{MESA} simulations.



A large source of uncertainty in these semi-empirical IFMRs arises from the determination of cluster ages, which can vary significantly between stellar models with different treatments of rotation and core overshoot (see Appendix \ref{sec: AppC}) \cite{Andrews}. As our principal aim is to constrain physics beyond the Standard Model, the use of an IFMR constraint which removes much astrophysical uncertainty is desirable.


Recently there has been interest in finding complementary constraints on the IFMR. In \cite{El-Badry} an empirical measurement of the IFMR was sought via analysis of a sample of 1100 WDs from the \textit{Gaia} Data Release 2 \cite{Gaia1, Gaia2}. This is shown in blue in Figure \ref{fig: IFMR Example}. However it was argued in \cite{Cummings_2018} that, by restricting their data to white dwarfs which have previously been spectroscopically identified, non-trivial selection biases are introduced. Furthermore, their derived IFMR is sensitive to the choice of initial-mass function (IMF) for large initial masses, precisely where ALP effects are most significant.


\begin{figure}[t]
    \centering
    \captionsetup{width=0.9\textwidth}
    \includegraphics{Section4/Figures/IFMR_example.pdf}
    \caption{A sample of existing constraints on the IFMR, with the full range of the double white dwarf binary constraint with flexible breakpoints \cite{Andrews} shown in light green. Also included are the 95\% confidence interval of the constraint from 1100 Gaia hydrogen-rich white dwarfs within 100pc \cite{El-Badry} in blue and the star cluster semi-empirical IFMRs derived using stellar models and isochrones from MIST (yellow) and PARSEC (purple) \cite{Cummings_2018}}
    \label{fig: IFMR Example}
\end{figure}


An appropriate IFMR constraint for our purposes was derived in \cite{Andrews} from a sample of 14 wide double white dwarf binary systems of solar metallicity. Like their counterparts in open clusters, the masses and cooling times of these WDs can be determined by application of theoretical WD cooling models. Instead of relying on the computation of absolute progenitor lifetime, however, only the relative lifetime $\Delta\tau_P=\tau_{P}^{(1)}-\tau_{P}^{(2)}$ is necessary. Consequently, the constraint derived from this analysis is independent of cluster ages. As these binaries are wide, we can assume that they have evolved independently of one another as single stars. This detail is crucial, as binary stellar evolution modelling is beyond the scope of this work.



The constraint \cite{Andrews} was determined in the following manner.
\begin{enumerate}
    \item Spectroscopic analysis of the WD atmospheres when combined with synthetic WD cooling tracks enabled the determination of their final masses $M_{\mathrm{f}}^{(1)}$, $M_{\mathrm{f}}^{(2)}$ and cooling times $\tau_{C}^{(1)}$, $\tau_{C}^{(2)}$.
    \item The relative cooling time $\Delta\tau_{C}=\tau_{C}^{(1)}-\tau_{C}^{(2)}$ can then be determined. As binary companions, each pair of stars can be assumed to be the same age. Consequently the difference in progenitor lifetime is given by $\Delta\tau_P'=-\Delta\tau_C$.
    \item An initial parametric model for the IFMR is then assumed, modelled as a three-piece linear relation, which allows estimates for initial masses $M_{\mathrm{init}}^{(1)}$, $M_{\mathrm{init}}^{(2)}$ to be determined for each binary WD.
    \item Stellar evolution models can then be used to convert each initial mass to a theoretical progenitor lifetime $\tau_{P}^{(1)}$, $\tau_{P}^{(2)}$. How well the difference between these $\Delta\tau_P$ matches the observed $\Delta\tau_P'$ is used to define the likelihood that the parametric model of the IFMR matches the observational data.
    \item When this is iterated upon, the best-fit parametric model can be determined.
\end{enumerate}
In \cite{Andrews} this was repeated many times to derive a posterior sample of semi-empirical IFMRs.

For the majority of \cite{Andrews} breakpoints at $2M_{\odot}$ and $4M_{\odot}$ are assumed in the three-piece fit. The motivation for this choice is physical. Stars with $M_{\mathrm{init}}\lesssim2M_{\odot}$ experience a degenerate helium-flash, while in the range $2M_{\odot}\lesssim M_{\mathrm{init}}\lesssim4M_{\odot}$ helium burning proceeds in a stable, non-degenerate convective core \cite{1976ApJS...32..367S}. If $M_{\mathrm{init}}\gtrsim4M_{\odot}$, a second dredge-up event can occur, which flattens the IFMR. The posterior sample for this three-piece fit  was made available by the authors of \cite{Andrews}.

When the values of the breakpoints are allowed to vary, however, a wider spread of IFMRs is obtained, particularly for high initial stellar masses. Given that the value of these breakpoints vary in the literature (e.g. in the MIST IFMR of \cite{Cummings_2018}, upper and lower breakpoints of $2.85M_{\odot}$ and $3.6M_{\odot}$ respectively produce the best fit), we conservatively favour this less restrictive constraint. The posterior sample in this case was not made available, which prevents a probabilistic interpretation of the results. Consequently, we show the entire range of IFMRs allowed in Figure \ref{fig: IFMR Example}.


A degree of tension exists between constraints for low initial masses. When $M_{\mathrm{init}}\lesssim2.6M_{\odot}$, the central fit for the MIST and PARSEC IFMRs is approximately $0.02M_{\odot}$ higher than the upper boundary of the Gaia constraint. This discrepancy is worse for DWD binaries, the upper limit of which falls approximately $0.04M_{\odot}$ below the central values of the cluster IFMRs when $M_{\mathrm{init}}\lesssim2.4M_{\odot}$.


Multiple sources of this tension have been suggested, including errors in star cluster ages or the presence of unresolved binaries in the WD samples \cite{Andrews}. This first possibility in particular motivates our choice of the DWD constraint, which is independent of star cluster ages and consequently more general. Clearly this constraint is far less restrictive than both the cluster IFMRs and the Gaia bound, particularly for the large initial masses which are sensitive to the influence of ALPs. As we shall see, however, even when a conservative approach is adopted, a substantial region of the ALP cosmological triangle can be ruled out.




%%%%%%%%%%%%%%%%%%%%%%%%%%%%%%%%%%%%%%%%%%%%%%%%%%%%%%%%%%%%%%%%
\subsection{Comparing theory and observation}
\label{subsec: comparing theory and observation}
%%%%%%%%%%%%%%%%%%%%%%%%%%%%%%%%%%%%%%%%%%%%%%%%%%%%%%%%%%%%%%%%
\begin{figure}[t]
    \centering
    \includegraphics[width = 15cm]{Section4/Figures/Updated_Diagram.pdf}
    \caption{Diagram illustrating the role of progenitor lifetimes in the derivation of the wide DWD binary constraint of \cite{Andrews}. The top row indicates how the inverse IFMRs $I^{-1}$ supplied in the posterior distribution of \cite{Andrews} were optimised using progenitor lifetime function $\tau(M)$. The second row depicts how the use of modified progenitor lifetimes, given by $\tau_{\mathrm{new}}$, affect this analysis. The final row illustrates how these changes can be overcome by the identification of a updated inverse IFMR $I^{-1}_{\mathrm{new}}$. The dotted red lines indicate how $M_{\mathrm{new}}=I^{-1}_{\mathrm{new}}(M_{\mathrm{f}})$ may be related to $M_{\mathrm{init}}=I^{-1}(M_{\mathrm{f}})$.}
    \label{fig: Progenitor diagram}
\end{figure}
The constraint \cite{Andrews} has been selected to ease the comparison of our simulations with observation. However, some dependence on theoretical calculations is retained in \cite{Andrews} and we now discuss their impact.


\subsubsection*{Progenitor Lifetimes}
%\update{
Given ALP-production drains energy from nuclear burning regions, thereby accelerating evolutionary timescales, it is important to discuss the role which stellar evolution models play in the semi-empirical constraint \cite{Andrews}. Note that a comprehensive discussion of the effects of these models would demand the re-evaluation of this constraint. This is beyond the scope of this work and consequently we present a simpler discussion below.

The role of these progenitor lifetimes is illustrated in the top row of the diagram in Figure \ref{fig: Progenitor diagram}. The constraint begins with a set of binary white dwarf masses $\{M_{\mathrm{f}}^{(1)},\ M_{\mathrm{f}}^{(2)}\}$ and associated difference in cooling times $\{\Delta\tau_C\}$ (shown in grey). An inverse IFMR $I^{-1}$ relates these white dwarf masses to a set of initial masses $\{M_{\mathrm{init}}^{(1)},\ M_{\mathrm{init}}^{(2)}\}$. These initial masses are then converted into their progenitor lifetimes $\{\tau_P^{(1)},\ \tau_P^{(2)}\}$ via the application of a function $\tau(M)$, derived by interpolating over the theoretical predictions of stellar simulations. The set of progenitor lifetime differences $\{\Delta\tau_{P}\}$ can then be calculated and compared with the observed set cooling time differences $\{\Delta\tau_C\}$. The inverse IFMR $I^{-1}$ (or equivalently IFMR) which gives the best fit between the magnitudes of these differences is said to be optimal.

Suppose, however, that the stellar simulations are systematically incorrect, due either to important input physics which has been neglected or the presence of ALPs in the model, and associate a new progenitor lifetime function $\tau_{\mathrm{new}}$ with these updated models. The effects of the utilisation of $\tau_{\mathrm{new}}$ are illustrated in the second line of Figure \ref{fig: Progenitor diagram}. Again the optimised inverse IFMR $I^{-1}$ is applied to the set of white dwarf masses, generating the same set of initial masses as in the first case. However, when new progenitor lifetimes are calculated using $\tau_{\mathrm{new}}$ they result in differences $\{\Delta\tau_{P,\ \mathrm{new}}\}$ which no longer provide the best fit for the set $\{\Delta\tau_C\}$. Consequently a new inverse IFMR $I_{\mathrm{new}}^{-1}$ must be found which maps the white dwarf masses to a set of initial masses $\{M_{\mathrm{new}}^{(1)},\ M_{\mathrm{new}}^{(2)}\}$, which return the original progenitor lifetimes  $\{\tau_P^{(1)},\ \tau_P^{(2)}\}$ (and consequently the optimal set of progenitor life differences $\{\Delta\tau_{P}\}$) upon application of $\tau_{\mathrm{new}}$. This scenario is shown in the final line of Figure \ref{fig: Progenitor diagram}.


We can estimate the impact of changing progenitor lifetimes by considering the relative size of the variable $M_{\mathrm{new}}=I_{\mathrm{new}}^{-1}(M_{\mathrm{f}})$ in terms of the initial mass given by the original optimised inverse IFMR $M_{\mathrm{init}}=I^{-1}(M_{\mathrm{f}})$. From the dashed lines in Figure \ref{fig: Progenitor diagram} it is evident that $M_{\mathrm{new}}$ is given by
\begin{equation}
    M_{\mathrm{new}}=\tau_{\mathrm{new}}^{-1}(\tau(M_{\mathrm{init}})).
    \label{eq: Mnew}
\end{equation}
If the neglected input physics increases progenitor lifetimes (i.e. $\tau_{\mathrm{new}}/\tau>1$ for all initial masses), $M_{\mathrm{new}}$ will be larger than $M_{\mathrm{init}}$\footnote{Progenitor lifetimes decrease with increasing initial mass.}, and the new inverse IFMR will be lower and flatter than the original. Conversely, if the progenitor lifetimes are shorter (e.g. due to the presence of energy loss to ALP production), the new initial masses will be smaller than the original and the updated posterior distribution of IFMRs will be higher and steeper than their counterparts in the set provided in \cite{Andrews}. The opposite will occur if $\tau_{\mathrm{new}}/\tau<1$ for all initial masses.


The constraint derived in the following sections primarily concerns ALPs with masses greater than 100 keV. The production of such ALPs is Boltzmann suppressed during central hydrogen- and helium-burning and, as such, their inclusion in stellar models shortens progenitor lifetimes only marginally ($\tau_{\mathrm{new}}/\tau$ decreases approximately linearly from 0.99 for $2M_{\odot}$ stars to 0.97 for $8M_{\odot}$). Using equation \ref{eq: Mnew} we find that the value of $M_{\mathrm{new}}/M_{\mathrm{init}}$ decreases approximately linearly from 0.994 (a 0.6\% reduction) for $2M_{\odot}$ to 0.985 for $8M_{\odot}$ (a 1.5\% reduction). The impact this reduction has on any two members of the posterior distribution of \cite{Andrews} will differ. Consequently we estimate the effect of neglecting ALP effects on stellar lifetime by applying this shift in $M_{\mathrm{init}}$ to all IFMRs with fixed breakpoints in the posterior distribution and identifying the average shift between these new IFMRs and the originals (the distribution of IFMRs with unfixed breakpoints was not made publicly available). When this approach is taken, we find that the posterior distribution is shifted upward by $0.002M_{\odot}$ and $0.007M_{\odot}$ in the $2$-$4M_{\odot}$ and $4$-$8M_{\odot}$ initial mass ranges respectively. 

These values clearly indicate that effects which reduce progenitor lifetimes, such as the inclusion of energy loss to ALPs, shift the IFMRs in the constraint of \cite{Andrews} upwards. For 100 keV ALPs and above, this magnitude of this effect is small and it will not significantly influence our results. ALPs with masses below a few keV, the production of which is not Boltzmann suppressed, can substantially reduce the magnitude of progenitor lifetimes. For example, 1 keV ALPs with a coupling strength of $g_{a\gamma\gamma}=1.58\times10^{-10}$ GeV$^{-1}$ reduce lifetimes by between 6-8\% for stars in the $4$-$8M_{\odot}$ mass range and as much as 12\% for $2M_{\odot}$ stars. If reductions of this scale were taken into account in the derivation of the constraint \cite{Andrews}, it would improve the constraining power of our bound. This, however, is beyond the scope of the present work.


It should be noted that, in the analysis presented in \cite{Andrews}, altering elements of input physics was found to yield comparable changes in progenitor lifetimes to those identified for 100 keV ALPs. These were deemed unimportant by the authors owing to the large uncertainty in stellar lifetimes generated by, for example, error in white dwarf mass measurement. The above discussion seems to support this conclusion.
%}


\subsubsection*{White dwarf evolutionary models}
\begin{figure}[t]
    \centering
    \includegraphics{Section4/Figures/IFMR_fit.pdf}
    \caption{The $M_{\mathrm{f}}$ values from our simulations without ALPs for initial masses in $0.2M_{\odot}$ intervals between $2-8M_{\odot}$ (black points).  A three-piece linear fit with flexible breakpoints has been applied (solid black line). The double white dwarf binary constraint \cite{Andrews} is provided for comparison (green).}
    \label{fig: IFMR fit}
\end{figure}


The synthetic white dwarf evolutionary models of \cite{2001PASP..113..409F} are also used in the derivation of \cite{Andrews} to generate the necessary cooling times. White dwarf evolution is a fairly well-understood cooling process dominated in phases by neutrino emission, gravothermal settling and crystallisation (for a detailed explanation see \cite{2001PASP..113..409F, Tremblay2019}). As such, cooling models are often used to determine the ages of stellar populations (see e.g. \cite{Bedin}) and place limits on neutrinos \cite{Wignet2004ApJ, Hansen_2015}. 


However,  anomalies in white dwarf cooling may exist. For example, tension exists between the observed and theoretical white dwarf luminosity functions (WDLFs) in certain stellar populations, which probe WD cooling  \cite{Isern1992, Isern:2008nt, Isern:2018uce, Bertolami:2014wua}. Interestingly, this tension can be explained by additional energy-loss to a DFSZ \cite{DINE1981199, Zhitnitsky:1980tq} type axion with $m_a\sim5$ meV. The WDLF, however, is sensitive to other astronomical properties including the initial mass function, star formation histories and the initial-final mass relation itself. Furthermore some WDLFs do not favour the existence of any additional cooling mechanism (e.g. that of \cite{Harris_2006} studied in \cite{Bertolami:2014wua}). Therefore until theoretical and observational uncertainties surrounding the WDLF improve, this remains only a hint towards the existence of such an axion.

Naturally, the existence of enhanced white dwarf cooling would require synthetic models to be updated. Though this could affect constraints such as \cite{Andrews}, it is worth noting that errors pertaining to cooling times are taken into account, with their most significant contribution coming from white dwarf mass measurement uncertainties.

%%%%%%%%%%%%%%%%%%%%%%%%%%%%

With these factors considered, an initial set of stellar simulations was computed in $0.2M_{\odot}$ increments over the $2-8M_{\odot}$ range without energy-loss to ALPs. These were allowed to run from the pre-Main Sequence until the termination of the AGB, where mass loss has reduced the outer envelope to 1\% of the total stellar mass. The adopted input physics for these simulations mimics that used to calculate the MIST isochrones \cite{MIST0, MIST1}.




The initial and final stellar masses for these simulations are shown in as the black points in Figure \ref{fig: IFMR fit}, alongside a theoretical IFMR derived by applying a three-piece linear fit, where the breakpoints remained unfixed. While this fits well within the DWD binary bound, it is considerably lower than the Gaia, MIST and PARSEC IFMR constraints. This discrepancy is well documented (see e.g. \cite{Cummings_2019}) and can be somewhat mitigated by the inclusion of rotation in the stellar models (Section \ref{subsec: rotn}) and core overshoot (Appendix \ref{sec: AppC}).





%%%%%%%%%%%%%%%%%%%%%%%%%%%%%%%%%%%%%%%%%%%%%%%%%%%%%%%5
\subsection{Axion-like particles and the IFMR}
\label{sec: ALPs and the IFMR}
%%%%%%%%%%%%%%%%%%%%%%%%%%%%%%%%%%%%%%%%%%%%%%%%%%%%%%%%%


%%%%%%%%%%%%%%%%%%%%%%%%%%%%%%%%%%%%%%%%%%%%%%%%%%%%%%%%
% Figure
%%%%%%%%%%%%%%%%%%%%%%%%%%%%%%%%%%%%%%%%%%%%%%%%%%%%%%%
\begin{figure}[t]
\centering
\begin{minipage}[t]{.5\textwidth}
    \centering
    \captionsetup{width=0.9\textwidth}
    \includegraphics[width=7.5cm]{Section4/Figures/IFMR_4.0.pdf}
    \caption{The three-piece IFMRs generated from our \texttt{MESA} simulations given the inclusions of energy-loss to 10 keV ALPs with $g_{a\gamma\gamma}=0.316\times10^{-10}$ GeV$^{-1}$ (purple), $0.631\times10^{-10}$ GeV$^{-1}$ (pink) and $10^{-10}$ GeV$^{-1}$ (yellow). The full range of the constraint \cite{Andrews} is included (green), along with our three-piece IFMR given standard astrophysics alone (black).}
    \label{fig: IFMR Run1}
\end{minipage}%
\begin{minipage}[t]{.5\textwidth}
    \centering
    \captionsetup{width=0.9\textwidth}
    \includegraphics[width=7.5cm]{Section4/Figures/IFMR_5.5.pdf}
    \caption{The three-piece IFMRs generated from our \texttt{MESA} simulations given the inclusions of energy-loss to 316 keV ALPs with $g_{a\gamma\gamma}=3.16\times10^{-10}$ GeV$^{-1}$ (purple), $10^{-9}$ GeV$^{-1}$ (pink) and $3.16\times10^{-9}$ GeV$^{-1}$ (yellow).  The full range of the constraint \cite{Andrews} is included (green), along with our three-piece IFMR given standard astrophysics alone (black). The dotted blue line indicates the IFMR given 630 keV ALPs with $g_{a\gamma\gamma}=10^{-5}$ GeV$^{-1}$. This choice of parameters is within the cosmological triangle.}
    \label{fig: IFMR Run2}
\end{minipage}
\end{figure}

To quantify the effects of axion-like particles on the IFMR, the series of simulations detailed in Section \ref{subsec: comparing theory and observation} were repeated including energy-loss to ALP-production. Initially 10~keV ALPs with $g_{a\gamma\gamma}=0.316\times10^{-10}$ GeV$^{-1}$ were considered and a new set of $M_{\mathrm{f}}$ values generated. The same three-piece fit specified in Section \ref{subsec: comparing theory and observation} was applied, the results of which are shown in Figure \ref{fig: IFMR Run1}. The departure from standard astrophysics due to deeper dredge-up events becomes evident for $M_{\mathrm{init}}\gtrsim3M_{\odot}$. This effect becomes more stark when larger values of $g_{a\gamma\gamma}$ are adopted. Ultimately, for $g_{a\gamma\gamma}=0.794\times10^{-10}$ GeV$^{-1}$, the theoretical IFMR falls outside the DWD constraint and this choice of ALP parameters can be excluded.


 Figure \ref{fig: IFMR Run2} depicts the results of repeating this analysis for 316~keV ALPs. The addition of ALP energy-loss again flattens the IFMR, however, this effect is considerably stronger in the $4$-$8M_{\odot}$ range, owing to the increased temperatures of their He-B shells (see Figure \ref{fig: Temperature Evolution}). Note that even the choice of parameters $g_{a\gamma\gamma}=3.16\times10^{-9}$~GeV$^{-1}$ associated with lowest IFMR in Figure \ref{fig: IFMR Run2} is unconstrained by the HB star bound. Also included is the three-piece IFMR generated if the ALP parameters $m_a=10^{5.8}\approx630$ keV and $g_{a\gamma\gamma}=10^{-5}$ GeV$^{-1}$ are chosen. Such ALPs lie within the cosmological triangle. Clearly, the IFMR is sensitive to ALPs from well within this region.
 
 
 
In order to construct a constraint, theoretical IFMRs were generated at logarithmically spaced intervals in $m_a$ and $g_{a\gamma\gamma}$. For each ALP mass, the smallest value of $g_{a\gamma\gamma}$ associated with an excluded theoretical IFMR was recorded. For some of these apparently excluded points the ALP will decay inside the nuclear burning region, so that energy loss does not result. While our simulations do not explicitly incorporate these effects, we take this into account below to derive a bound on the ALP parameter space.

%%%%%%%%%%%%%%%%%%%%%%%%%%%%%%%%%%%%%%%%%%%%%%%%%%%%%%%
\subsection{ALP-decay in AGB stars}
%%%%%%%%%%%%%%%%%%%%%%%%%%%%%%%%%%%%%%%%%%%%%%%%%%%%%%%
Any constraint derived from ALP influence on the He-B shell of AGB stars would not extend to arbitrarily high values of $g_{a\gamma\gamma}$, as ALP-decay becomes an important factor in that region. In order to quantify this, we follow the example of \cite{Carenza:2020zil}, who recently addressed ALP-decay in their HB star bound. In their treatment it was argued that the energy-loss constraint should relax when the decay-length
\begin{equation}
    \lambda_a=5.7\times10^{-5}g_{a\gamma\gamma}^{-2}m_{\mathrm{keV}}^{-3}\frac{\omega}{m_a}\sqrt{1-\bigg(\frac{m_a}{\omega}\bigg)^2} R_{\odot}
\end{equation}
falls below the HB star core radius $R_c\approx0.03R_{\odot}$, where $m_{\mathrm{keV}}=(m_a/1\ \mathrm{keV})$. Though ALPs will continue to contribute towards energy transfer after this (Section \ref{sec: Section 2}), this is sub-leading to convection which is dominant in the cores of such stars.


We similarly argue that the fundamental criterion for the energy loss argument, that energy be removed from a region undergoing nuclear burning, is no longer met in the He-B shell of AGB stars when $\lambda_a$ falls below the shell thickness $R_{\mathrm{He}}\approx0.007R_{\odot}$. Taking this into account yields the solid dark blue line in Fig.~\ref{fig: ALP-IFMR Constraint}.




Unlike HB cores, however, the He-B shell of AGB stars is radiative rather than convective. Consequently, ALPs can be expected to influence AGB stellar structure beyond the boundary of this constraint, though a thorough treatment of this requires dedicated simulations which include ALP contributions to energy transfer, which is beyond the scope of this work.


However, we are able to identify a region of parameter space in which ALPs still influence AGB structure. This is achieved by following the example of \cite{Raffelt_Energy_Transfer} and insisting that photons contribute dominantly towards radiative energy transfer in the He-B shell ($\kappa_a>\kappa_{\gamma}$). The upper-boundary of this region is indicated by the dashed blue line in Figure \ref{fig: ALP-IFMR Constraint}, using values for $T$, $\rho$, $k_s$ and $\kappa_{\gamma}$ taken from our models. Though this does not include any part of the cosmological triangle, it would be interesting to examine how this could be improved upon through detailed stellar simulations.
\begin{figure}[t]
    \centering
    \includegraphics{Section4/Figures/WDIFMR_original_constraint.pdf}
    \caption{A suite of constraints in the keV-MeV region of the ALP plane. Our preliminary constraint including the impact of ALP-decay, is given by the solid dark blue line. The upper boundary of the nominal region in which ALPs are relevant for energy transfer in AGB stars is indicated by the dashed dark blue line. See text below for discussion.}
    \label{fig: ALP-IFMR Constraint}
\end{figure}






%%%%%%%%%%%%%%%%%%%%%%%%%%%%%%%%%%%%%%%%%%%%%%%%%%%%%
\subsection{Stellar Rotation}
\label{subsec: rotn}
%%%%%%%%%%%%%%%%%%%%%%%%%%%%%%%%%%%%%%%%%%%%%%%%%%%%%

We have carefully selected the constraint \cite{Andrews} to minimise dependence on astrophysical uncertainties such as star cluster age determination and the initial mass function. Nevertheless, as our investigative tool is stellar modelling, there are systematic uncertainties which we now discuss.


The AGB phase is notoriously difficult to model accurately. The TP-AGB particularly presents significant challenges, owing to its dependence on a combination of intricate processes such as mass loss and enhanced mixing from core overshoot and rotation. An extensive discussion of these can be found in \cite{kerschbaum2007galaxies}. 


Typically the uncertainty surrounding free parameters in models of the AGB phase is mitigated by comparison with observation or a solar calibration. Examples of model input physics which fall into this category include the efficiency of mass loss rates and core overshoot. Although their influence on the IFMR can be considerable, observation limits these free parameters to a thin range. Consequentially a discussion of their relevance is deferred to Appendix \ref{sec: AppC}.


Unlike these, however, rotation - and the enhanced mixing it elicits - is a reality of stellar physics. Rather than taking one simple value, the angular velocity of stars will vary according to a probability distribution (e.g. that of \cite{Huang2010}). We therefore present a discussion of the importance of rotational mixing to the IFMR. 


\subsubsection*{Effect of rotation on theoretical models}
For simplicity, it has been assumed that all stars simulated in this work are 1-D and do not rotate. The impact of rotation can affect both theoretical predictions for the IFMR and the derivation of its constraints \cite{Cummings_2019}. Rotational mixing increases the supply of hydrogen available when the WD progenitor is evolving on the Main-Sequence, which produces larger stellar cores. This, along with an associated increase in the duration of the E-AGB, means that progenitor stars experiencing rapid rotation will have a higher WD masses than their more slowly rotating counterparts \cite{Dominguez_1996}. Consequently, theoretical predictions of the IFMR including rotation lie above those where it is neglected.

\begin{figure}[t]
    \centering
    \includegraphics{Section4/Figures/IFMR_rotn.pdf}
    \caption{Two-piece fits from the statistical IFMRs of \cite{Cummings_2019} generated using the MIST (yellow), SYCLIST (pink) and MIST/ATON (dark blue) rotating stellar evolution models. The full range of the double white dwarf IFMR constraint \cite{Andrews} and our standard theoretical IFMR (generated from MIST input physics) are also included in green and black respectively. Individual stellar models are referenced in the text.}
    \label{fig: Rotn IFMRs}
\end{figure}

The effects of rotation on theoretical IFMRs was recently investigated in \cite{Cummings_2019}. 40,000 synthetic stars were generated spaced uniformly in initial mass, with rotation drawn from the distribution of \cite{Huang2010}. Values of $M_{\mathrm{f}}$ were then determined through application of three different rotating stellar models - MIST, SYCLIST \cite{SYCLIST1, SYCLIST2} and a combination of ATON/MIST. The two-piece fits of the resulting statistical IFMRs are shown in Figure \ref{fig: Rotn IFMRs}.


The MIST tracks have relatively inefficient rotational mixing, selected to reproduce surface nitrogen abundances in Main Sequence stars \cite{MIST1}. Consequently, the statistical IFMR determined in \cite{Cummings_2019} is weighted strongly towards lower final masses and the two-piece fit in Figure \ref{fig: Rotn IFMRs} is offset from the non-rotating MIST model by an average of $0.04M_{\odot}$ for initial masses between $3.6M_{\odot}$ and $6M_{\odot}$.


The SYCLIST rotating models, however, employ more efficient rotational mixing than those of MIST. Consequently the resulting statistical IFMR is weighted heavily towards larger values of $M_{\mathrm{f}}$. This corresponds to a mean upward shift of $0.06M_{\odot}$ in the same initial mass range. The most significant shift of $0.08M_{\odot}$ is achieved when MIST rotation is applied to ATON models \cite{ATON} and has contributions from both rotational mixing and enhanced convective core overshoot, which we describe in detail in Appendix \ref{sec: AppC}. 



In an analysis which accounts for the effects of rotation, the standard astrophysical IFMR is shifted upward from its non-rotating counterpart. Consequently, for theoretical IFMRs to fall outside the region of the DWD constraint, larger values of $g_{a\gamma\gamma}$ are needed than those derived in Section \ref{sec: ALPs and the IFMR}. This could be achieved by repeating our simulations for different stellar rotations and performing an analysis in the spirit of \cite{Cummings_2019}. This approach would be computationally cumbersome and would risk underestimating the influence of rotation owing to the inefficiency of rotational mixing in the MIST input physics. Consequently, we adopt a much more simplistic approach and simply apply an upward shift to our theoretical IFMRs.



We select the magnitude of this upward shift to be $0.08M_{\odot}$, equal to that of the ATON/MIST models and record the new value of $g_{a\gamma\gamma}$ for specific ALP masses which cause the theoretical IFMR to fall outside the DWD constraint. Strictly speaking, the adopted value of $0.08M_{\odot}$ has a contribution of approximately $0.035M_{\odot}$ from the enhanced convective core overshoot of the ATON models. As shall be discussed in Appendix \ref{sec: AppC}, however, the treatment of overshoot in the MIST models has been calibrated to observation during evolutionary phases which are unaffected by ALPs. We are therefore confident that this choice is a conservative one.


The updated constraint, which takes these effects into account, is indicated by the light blue line in Figure \ref{fig: Rotn constraint}. While less restrictive than its non-rotating counterpart (indicated by the dark blue line), especially for low ALP masses, we have nevertheless ruled out a significant region of the cosmological triangle. 


\subsubsection*{Effect of rotation on the semi-empirical bound}
%\update{
Stars which rotate have extended progenitor lifetimes due to the enhanced mixing and gravitational lifting effects they experience during their Main Sequence evolution. The faster the rate of rotation, the more extreme this effect. The semi-empirical constraint \cite{Andrews} does not take rotation into account in the \texttt{MESA} models used in their analysis. Given stars in this initial mass range have been observed to rotate with angular velocities of between $\Omega/\Omega_{\mathrm{crit}}=0.0$ and $\Omega/\Omega_{\mathrm{crit}}=0.95$, where $\Omega_{\mathrm{crit}}$ is the critical or break-up angular velocity, it is possible that progenitor lifetimes have been systematically underestimated. As discussed in Section \ref{subsec: comparing theory and observation}, effects which increase progenitor lifetimes shift members of the posterior distribution of \cite{Andrews} downwards, potentially making our bound less restrictive. Consequently we must discuss the expected effects of rotation on the semi-empirical constraint we have employed.



We can estimate the impact of neglecting rotation using the method outlined in Section \ref{subsec: comparing theory and observation}. Stellar models were simulated incorporating rotation of $\Omega/\Omega_{\mathrm{crit}}\in\{0.0,\ 0.2,\ 0.4,\ 0.6,\ 0.8,\\ 0.95\}$. The expected increase in progenitor lifetime was then computed by averaging these over the rotational distributions of \cite{Huang2010}. We found that $\tau_{\mathrm{new}}/\tau$ grows linearly from approximately 1.02 for $2M_{\odot}$ stars to 1.09 for $8M_{\odot}$. From Equation \ref{eq: Mnew}, we find that the corresponding values of $M_{\mathrm{new}}/M$ vary from 1.01 (a 1\% increase) to 1.04 (a 4\% increase) for stars with initial masses between $2$-$8M_{\odot}$. By applying these shifts in initial mass to the posterior sample of IFMRs with fixed breakpoints, we find average downward IFMR shifts of $0.004M_{\odot}$, and $0.017M_{\odot}$ for stars in the $2$-$4M_{\odot}$, $4$-$8M_{\odot}$ initial mass ranges respectively. Of these, it is the latter that is the most important for our constraint.


Given ALPs shift the IFMR downwards, it is more pertinent to discuss the impact of rotation on the lower boundary of the constraint \cite{Andrews}. We achieve this by estimating the effect of this increase in progenitor lifetimes on the lower 95\% confidence interval boundary of the posterior distribution (see Figure \ref{fig: IFMR prob} in Appendix \ref{sec: AppD}). In the $2$-$4M_{\odot}$ and $4$-$8M_{\odot}$ initial mass ranges, we find these values reduce by an average of $0.003M_{\odot}$ and $0.013M_{\odot}$ respectively. For the most massive stars, this boundary has a smaller downward shift than that of the average over the posterior distribution. This is because IFMRs towards the lower end of the distribution are generally flatter, and therefore must shift less vertically to accommodate the increase in $M_{\mathrm{init}}$ described above.


Stellar rotation is clearly a significant source of systematic uncertainty in our analysis. The results above, however, suggest that it has a greater influence on the predictions of our simulations rather than the derivation of semi-empirical constraints. In principle, a comprehensive analysis should account for the latter directly. However, given the magnitude of these effects is relatively small, our original analysis excluded the entire range of IFMRs with unfixed breakpoints (rather than a confidence interval) and we have already opted for an upward shift of $0.08M_{\odot}$ in this section (which accounts for the combined effects of rotation and enhanced core overshoot), we ignore this contribution in our final result.

%}

We stress that a conservative approach has been adopted throughout this analysis. We chose to use the DWD binary constraint \cite{Andrews} over others as it is insensitive to star cluster ages and has unfixed breakpoints which leads to a less restrictive range of IFMRs. We have also allowed for the possibility of efficient rotational mixing affecting our predictions. Naturally, there is scope for this constraint to improve dramatically when theoretical uncertainties surrounding AGB physics and rotational mixing are better understood. Furthermore, millions of binary systems have been resolved in the Gaia data release 3 \cite{2021arXiv210105282E}. Given this analysis was performed with only 14 binary systems, it would be of great interest for the method of \cite{Andrews} to be applied to the wide subset of the 1400 double white dwarf binaries identified.


\begin{figure}[t]
    \centering
    \includegraphics{Section4/Figures/WDIFMR_rotn.pdf}
    \caption{A comparison between our derived constraints including (light blue) and excluding (dark blue) a conservative estimate of the effects of efficient rotational mixing on the IFMR in the keV-MeV region of the ALP-plane.}
    \label{fig: Rotn constraint}
\end{figure}