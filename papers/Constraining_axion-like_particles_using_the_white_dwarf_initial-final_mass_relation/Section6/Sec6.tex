\section{Conclusion}
\label{sec: Section 6}



Stellar evolution has a well-established pedigree in constraining physics beyond the Standard Model, most notably for axions and axion-like particles. In this work we provide a detailed investigation of the effects of keV-MeV scale ALPs on stellar evolution simulations. 




The photo-production of such axion-like particles in the keV-MeV mass range significantly impacts the evolution of asymptotic giant branch stars, the late evolutionary phase of stars with initial masses $\lesssim8M_{\odot}$. Specifically, the free streaming of ALPs produced in the helium-burning shells of these stars facilitates more rapid and deeper dredge-up events, which significantly reduce their final masses.



This behaviour has been constrained by appealing to semi-empirical measurements of the white dwarf initial-final mass relation. In particular, analysis of 14 wide double white dwarf binary systems conducted in \cite{Andrews} enabled us to construct a new bound on the ALP-plane which proves more restrictive for large $m_a$ than that derived from horizontal branch stars, most notably in the unconstrained cosmological triangle, even when the effects of stellar rotation have been considered. We expect these results to improve in the near future if the method of \cite{Andrews} were applied to a subset of the 1400 double white dwarf binaries identified in the Gaia early Data Release 3 \cite{2021arXiv210105282E}.





For sufficiently large values of $g_{a\gamma\gamma}$ and $m_a$, the axion-like particle decay-lengths fall below the width of the helium-burning layer and the foundational criterion of the energy-loss argument is no longer met. This reduces our initial constraint to the green shaded region in Figure \ref{fig: ALP_param_space}. As energy transfer within the helium-burning layer is radiative, more strongly interacting axion-like particles still influence the structural evolution of asymptotic giant branch stars. We can estimate the region of the ALP-plane in which this is relevant by insisting that the Rooseland mean opacity of axion-like particles be larger than that of photons. A conclusive statement about these effects would require the addition of axion-like particle energy transfer to stellar models.




Within the last year there has been a resurgent interest in the cosmological triangle. In addition to the recent HB star bound \cite{Carenza:2020zil}, which our work complements, the constraints which define its other boundaries have been revisited. For instance, the constraint derived from the neutrino signal and observed cooling of SN1987A was recomputed recently with a state-of-the-art supernova model \cite{Lucente:2020whw}, which we show in Figures \ref{fig: ALP_param_space}, \ref{fig: ALP-IFMR Constraint} and \ref{fig: ALP-Decay Constraint}. The work in question included a second calculation, based on the condition that only the part of the axion-like particle luminosity that can be readily converted to neutrino energy is relevant for the SN1987A bound \cite{Chang:2016ntp, Ertas:2020xcc}. When this so-called \textit{modified luminosity} criterion is applied a new region above the pre-existing constraint is excluded, and the cosmological triangle shrinks further (see Figure \ref{fig: ALP-Decay Constraint}). 




The definition of new boundaries for the cosmological triangle is timely, as future experiments will be able to directly probe this region. The Belle II experiment, for example, has estimated sensitivity within the relevant mass range to couplings as low as $g_{a\gamma\gamma}\sim10^{-5}$ GeV$^{-1}$ at a luminosity of 50 ab$^{-1}$ \cite{Dolan:2017osp}. Future neutrino experiments, such as DUNE, will also be able to probe the cosmological triangle with both liquid argon (LAr) and gaseous argon (GAr) detectors \cite{Brdar:2020dpr}.




This work, like many before it, employs robust observational astrophysics and ever-more accessible stellar modelling to investigate the impact of axion-like particles on stellar evolution. Though we have directed these tools towards this class of particle, the impact of other weakly interacting particles can be probed in this manner. Stars, as ubiquitous objects in the universe, have a vital role to play in constraining physics beyond the Standard Model.