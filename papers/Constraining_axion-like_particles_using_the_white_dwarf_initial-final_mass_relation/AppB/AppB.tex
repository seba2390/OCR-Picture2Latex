\section{Systematic Uncertainties}
\label{sec: AppC}
In Section \ref{subsec: rotn} we accounted for the influence of rotation on our constraint. However, there are multiple free parameters in our models which can influence the IFMR. Here we discuss two of these - mass loss and convective overshoot - and estimate their influence on our constraint.

\subsubsection*{Mass loss:}
The MIST models \cite{MIST0, MIST1} on which we base our input physics adopt the Reimers \cite{1975MSRSL...8..369R} and Bl\"{o}cker \cite{1995A&A...297..727B} prescriptions for mass loss for the RGB and AGB respectively. These are given by
\begin{equation}
    \Dot{M}_{\mathrm{R}}=4\times10^{-13}\eta_{\mathrm{R}}\frac{(L/L_{\odot})(R/R_{\odot})}{(M/M_{\odot})}\ \mathrm{M}_{\odot}\ \mathrm{yr}^{-1}
    \label{eq: Reimers}
\end{equation}
and
\begin{equation}
    \Dot{M}_{\mathrm{B}}=4.83\times10^{-9}\eta_{\mathrm{B}}\frac{(L/L_{\odot})^{2.7}}{(M/M_{\odot})^{2.1}}\frac{\Dot{M}_{\mathrm{R}}}{\eta_{\mathrm{R}}} \mathrm{M}_{\odot}\ \mathrm{yr}^{-1}
    \label{eq: Blocker}
\end{equation}
where $\eta_{\mathrm{R}}$ and $\eta_{\mathrm{B}}$ are $\mathcal{O}(1)$ parameters. In the MIST models, values of $\eta_{\mathrm{R}}=0.1$ and $\eta_{\mathrm{B}}=0.2$ have been chosen to reproduce the IFMR and AGB luminosities in the Magellanic Clouds.

Mass loss rates are relevant to the IFMR, as they govern how quickly the outer envelope is shed during the TP-AGB and consequently how many thermal pulses occur during this phase. Indeed, separating the effects of mass-loss and ALPs was viewed as the major challenge in establishing an IFMR-derived constraint on ALP parameters when discussed in \cite{Dominguez} for low mass ALPs.


Varying the magnitude of  $\eta_{\mathrm{B}}$ does impact the lower-IFMR ($M_{\mathrm{init}}\lesssim3M_{\odot}$), with more efficient mass loss leading to lower WD masses for a given value of $M_{\mathrm{init}}$ \cite{Cummings_2019}. For larger initial masses, however, the IFMR remains relatively insensitive to the adopted mass loss rate as the TP-AGB of such stars is too rapid for significant core growth to occur \cite{1995A&A...297..727B}.



\begin{figure}[t]
    \centering
    \includegraphics{AppC/Figures/IFMR_prob.pdf}
    \caption{A comparison between double white dwarf constraints from \cite{Andrews}. The full range of the bound where breakpoints are allowed to move is shown in green. The 95\% confidence interval for fixed breakpoints at $2M_{\odot}$ and $4M_{\odot}$ is shown in red. Its average is indicated by the red line. The black solid line is a two-piece linear fit derived from our simulations (the black points in Figure~\ref{fig: IFMR fit}), constrained with a breakpoint fixed at $4M_{\odot}$. }
    \label{fig: IFMR prob}
\end{figure}


\subsubsection*{Convective overshoot:}
Fluid parcels in a convective region are accelerated as they approach its boundary and do not begin to decelerate until they enter the radiative zone. Consequently, if braking is insufficient, they can penetrate a non-negligible distance beyond the convective boundary and increase the efficiency of mixing in this region. This phenomenon is known as convective overshoot \cite{2012sse..book.....K}.


The presence of convective overshoot has two main results on the IFMR. Firstly, the enhanced mixing caused by overshoot facilitates the formation of more massive CO core at the onset of the TP-AGB, which directly shifts the IFMR upwards \cite{2000A&A...360..952H}. On the other hand, increased overshoot during the TP-AGB causes deeper third dredge-up events, which reduces core growth during this phase \cite{2000A&A...360..952H, Cummings_2019}.




The MIST models treat overshoot as a time-dependent, diffusive process, the strength of which is governed by a parameter $f_{\mathrm{ov}}$. For the core, envelope and shell, values of $f_{\mathrm{ov, core}}=0.016$ and $f_{\mathrm{ov, env}}=f_{\mathrm{ov, shell}}=0.0174$ respectively. These are adopted in order to reproduce the Main Sequence turn-off of open cluster Messier 67 ($f_{\mathrm{ov, core}}$) and a solar calibration ($f_{\mathrm{ov, env}}$) \cite{MIST1}. 


The ATON models of \cite{ATON} similarly model overshoot as a diffusive process. However, these have $f_{\mathrm{ov}}=0.02$ for hydrogen and helium burning and $0.002$ during the AGB. As a result, they predict an IFMR which is higher than that of the MIST models by an average of $0.035M_{\odot}$ for initial masses between $3.6$-$6.5M_{\odot}$. This contributes to the mean upward shift of $0.08M_{\odot}$ corresponding the combined MIST/ATON models, which was chosen for our constraint in Section \ref{subsec: rotn}.

%\update{
Enhanced convective core overshoot can also affect progenitor lifetimes relevant to the derivation of semi-empirical IFMR constraints. By essentially enhancing the size of the convective core, a larger supply of hydrogen fuel is available for Main Sequence stars, which increases the duration of this evolutionary phase. To estimate the magnitude of this effect, we recomputed our evolutionary models adopting the values of $f_{\mathrm{ov}}$ specified in the ATON models. The resulting stellar lifetimes were found to be between 1-2\% larger for stars in the $2$-$8M_{\odot}$ initial mass range. Following the argument presented in Section \ref{subsec: comparing theory and observation}, this corresponds to $M_{\mathrm{new}}/M_{\mathrm{init}}$ values increasing linearly from 1.01 to 1.03 for the same range of initial masses. Again, we estimate the influence this has on the constraint of \cite{Andrews} by averaging the resulting downward shift in $M_{\mathrm{WD}}$ over the entire posterior sample of IFMRs with fixed breakpoints. For stars between $2$-$4M_{\odot}$ and $4$-$8M_{\odot}$ we find average downward shifts of $0.001M_{\odot}$ and $0.004M_{\odot}$ respectively. These are an order of magnitude lower than the upward shift adopted in our constraint, and are less than the upward shift obtained when the effects of ALPs on progenitor lifetimes are considered. 

Importantly, both mechanisms for tuning the parameter $f_{\mathrm{ov}}$ pertain to the Main Sequence. They are therefore insensitive to ALPs in the keV-MeV mass range, the production of which is Boltzmann suppressed during evolutionary phases before the AGB. Even ALPs lighter than this, which are readily produced during central helium-burning, will minimally affect the Main Sequence and therefore will not invalidate these calibrations. Because of this, we are confident that the adopted upward shift of $0.08M_{\odot}$ used in Section \ref{subsec: rotn} is conservative.
%}