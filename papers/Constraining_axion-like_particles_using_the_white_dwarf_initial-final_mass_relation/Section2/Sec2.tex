\section{Astrophysical ALP Production}
\label{sec: Section 2}



The impact of ALPs on stellar structure varies with their lifetime $\tau_a=\Gamma_a^{-1}$ and production cross-section. The interaction in Equation \ref{eq: ALP-Photon interaction} leads to the decay width
\begin{equation}
    \label{eq: ALP-decay width}
    \Gamma_a=\frac{g_{a\gamma\gamma}^2m_a^3}{64\pi}.
\end{equation}
For sufficiently small values of $m_a$ and $g_{a\gamma\gamma}$, this lifetime is large and ALPs freely escape the stellar interior and contribute to energy-loss within the star. When the ALP mass and photon-coupling grow large, however, ALPs decay within the star and contribute towards radiative energy-transfer.

%%%%%%%%%%%%%%%%%%%%%%%%%%%%%%%%%%%%%%%%%%%%%%%%%%%%%%%5
\subsection{Energy-loss to ALP-production}
%%%%%%%%%%%%%%%%%%%%%%%%%%%%%%%%%%%%%%%%%%%%%%%%%%%%%%%5

The production of freely-escaping ALPs affects stellar structure by reducing the local energy-gain rate per unit mass $\epsilon$. The magnitude of the ALP energy-loss rate $\epsilon_a$ is given as a sum over the relevant ALP-production processes. For ALPs in the keV-MeV range which couple only to photons two such mechanisms of significance exist, \textit{Primakoff production} and \textit{photon coalescence} (or fusion). 



ALP-Primakoff production refers to the conversion of a photon into an ALP in the presence of an external electromagnetic field \cite{Dicus-PhysRevD.18.1829, Raffelt:1996wa}. In a stellar interior this is facilitated by the Coulomb field of the constituent charged particles in the plasma. The transition rate of a photon with momentum $\Vec{k}$ and energy $\omega$ into an ALP of momentum $\Vec{p}$ is \cite{DiLella:2000dn} 
\begin{equation}
\begin{split}
    \Gamma_{\gamma\rightarrow a}^P=\frac{g_{a\gamma\gamma}^2Tk_s^2}{32\pi}\frac{k}{\omega}
    \Bigg(\frac{((k+p)^2+k_s^2)((k-p)^2+k_s^2)}{4kpk_s^2}&\ln\bigg(\frac{(k+p)^2+k_s^2}{(k-p)^2+k_s^2}\bigg)\\-\frac{(k^2-p^2)^2}{4kpk_s^2}\ln\bigg(\frac{(k+p)^2}{(k-p)^2}\bigg)-1
    \Bigg),
\end{split}
\label{eq: Primakoff Transition Rate}
\end{equation}
where $k=\lvert\Vec{k}\rvert$, $p=\lvert\Vec{p}\rvert$ and $T$ is the temperature of the stellar plasma. Importantly, the Primakoff transition rate is subject to plasma screening effects, the scale of which is set by the Debye-H\"uckel wave number
\begin{equation}
    k_s^2=\frac{4\pi\alpha}{T}\frac{\rho}{m_u}\bigg(Y_e + \sum_jZ_j^2Y_j\bigg),
\end{equation}
where $\rho$ is the local mass density, $m_u$ is the atomic mass unit, $Y_e$ is the number of electrons per baryon and $Y_j$ is the number per baryon of nuclear species with charge $Z_j$.




In the stellar plasma, the effective photon mass is given by the plasma frequency $\omega_{\mathrm{pl}}\approx4\pi\alpha n_e/m_e$. In all scenarios we consider, however, this is small compared with the average photon energy and is therefore neglected. Furthermore, we assume that the mass of the produced ALP is small compared with that of the charged particle. Consequently, its recoil can be ignored and Equation \ref{eq: Primakoff Transition Rate} simplifies to
\begin{equation}
\begin{split}
    \Gamma_{\gamma\rightarrow a}^P=\frac{g_{a\gamma\gamma}^2Tk_s^2}{32\pi}     \Bigg(\frac{(m_a^2-k_s^2)^2+4\omega^2k_s^2}{4\omega pk_s^2}&\ln\bigg(\frac{(\omega+p)^2+k_s^2}{(\omega-p)^2+k_s^2}\bigg)\\-\frac{m_a^4}{4\omega pk_s^2}\ln\bigg(\frac{(\omega+p)^2}{(\omega-p)^2}\bigg)-1
    \Bigg)
\end{split}
\label{eq: Simplified Primakoff Transition Rate}
\end{equation}
where $p=\sqrt{\omega^2-m_a^2}$.

%%% Figure%%%%%%%%%%%%%%%%%%%%%%%%%%%%%%%%%
\begin{figure}[t]
    \centering
    \begin{subfigure}[t]{0.5\textwidth}
        \centering
        \includegraphics{Section2/Figures/Eps_mass.pdf}
        \caption{}
        \label{fig: Eps_processcomp}
    \end{subfigure}%
    \begin{subfigure}[t]{0.5\textwidth}
         \centering
         \includegraphics{Section2/Figures/Eps_temp.pdf}
         \caption{}
         \label{fig: Eps_masscomp}
     \end{subfigure}
    \caption{The magnitude of energy-loss rate per unit mass associated with ALP-production for the Primakoff and Coalescence production mechanisms as a function of ALP mass $m_a$ given conditions within the He-B layer of a $4M_{\odot}$ AGB star (a). A comparison between the total energy-loss rates per unit mass is also shown as a function of temperature given two different ALP masses, 10 keV and 316 keV (b).}
    \label{fig: Fig2}
\end{figure}
%%%%%%%%%%%%%%%%%%%%%%%%%%%%%%%%%%%%%%%%%%%%

The contribution of this process to $\epsilon_a$ can then be computed as \cite{Raffelt-Bounds-on-light, Raffelt:1996wa}
\begin{equation}
    \epsilon_a^P=\frac{2}{\rho}\int\frac{dp\ p^2}{2\pi^2}\Gamma_{\gamma\rightarrow a}\omega f(\omega),
\end{equation}
where the factor of 2 accounts for the photon polarisations and $f(\omega)$ is the thermal photon energy distribution, here given by the Bose-Einstein distribution \cite{DiLella:2000dn}
\begin{equation}
    f(\omega)=\frac{1}{e^{\omega/T}-1}.
\end{equation}
Substituting Equation \ref{eq: Simplified Primakoff Transition Rate} into the expression for $\epsilon_a^P$ then gives
\begin{equation}
    \epsilon_a^P=\frac{g_{a\gamma\gamma}^2T^7}{4\pi\rho}F(\xi^2, \mu^2)
    \label{eq: F function}
\end{equation}
where dimensionless parameters $\mu=m_a/T$ and $\xi=k_s/(2T)$ have been defined. The function $F$ involves an integral over photon phase-space and encompasses the entire $m_a$- and $k_s$-dependence of $\epsilon_a$. 







So long as the kinetic threshold $m_a\geq2\omega_{\mathrm{pl}}$ is met, photon coalescence $\gamma \gamma \to a$ can contribute to ALP production in stellar interiors. The production rate due to this process is \cite{DiLella:2000dn}
\begin{equation}
    \frac{dN_a}{d\omega}=\frac{g_{a\gamma\gamma}^2m_a^4}{128\pi^3}\sqrt{\omega^2-m_a^2}e^{-\omega/T},
\end{equation}
which corresponds to an energy-loss rate
\begin{equation}
    \epsilon_{a}^C=\frac{1}{\rho}\int\omega\frac{dN_a}{d\omega}d\omega=\frac{g_{a\gamma\gamma}^2T^7}{4\pi\rho}G(\mu^2).
    \label{eq: G function}
\end{equation}
Here $G(\mu^2)$, like $F(\xi^2, \mu^2)$ above, contains the entire $m_a$-dependence of $\epsilon_a^C$. The total energy-loss rate per unit mass to ALP production is then given by
\begin{equation}
    \epsilon_a=\frac{g_{a\gamma\gamma}^2T^7}{4\pi\rho}(F(\xi^2, \mu^2)+G(\mu^2)).
\end{equation}
Both functions $F(\xi^2, \mu^2)$ and $G(\mu^2)$ contain integrals over photon phase-space which must be evaluated numerically.




The relative importance of these two production mechanisms was comprehensively discussed in \cite{Carenza:2020zil}. Primakoff production was found to dominate energy-loss in HB star cores for low ALP masses ($\approx30$~keV). When ALPs with masses of $80$~keV were considered, however, photon coalescence contributed most significantly towards $\epsilon_a$. As can be seen in Figure \ref{fig: Eps_processcomp}, the same is true when the conditions of the He-B shell of a $4M_{\odot}$ AGB star are adopted ($T\approx16$~keV, $\rho\approx1.4\times10^3$ g cm$^{-3}$, $k_s\approx35$~keV) when normalised by $g_{10}^2\equiv (g_{a\gamma\gamma}/[10^{-10}\ \mathrm{GeV}^{-1}])^2$. Note that these values have been taken from our models.




Both production mechanisms are Boltzmann suppressed for heavy ALPs. Consequently the temperature sensitivity of $\epsilon_a$ is enhanced significantly as $m_a$ increases. This is depicted for 10~keV and 316~keV ALPs in Figure \ref{fig: Eps_masscomp} given the conditions of the HeB shell. Production of the latter, which is dominated by photon fusion, rapidly increases between $\log\ T=8.2$ and $8.6$ as Boltzmann suppression is alleviated. Energy-loss to 10 keV ALPs, however, increases far more gradually over this temperature interval and falls below that of the 316 keV ALPs at high temperatures. It is precisely this temperature dependence which motivates why moderate increases in temperature can elicit strong improvements in existing constraints. 


It should be noted, however, that this calculation has assumed a constant Debye-H\"uckel wave number, which is not true in a stellar environment. Consequently, the temperature normalised Debye-H\"uckel wave number $\xi=k_s/(2T)$ decreases, which somewhat inhibits Primakoff ALP production, resulting in total energy-loss for the 10~keV ALP being underestimated. The value of 316~keV has been chosen because $316=10^{2.5}$ and we shall be investigating the impact of ALPs logarithmically spaced in mass. 


%%%%%%%%%%%%%%%%%%%%%%%%%%%%%%%%%%%%%%%%%%%%%%%%%%%%%%%%%%
\subsection{ALP contributions to energy-transport}
%%%%%%%%%%%%%%%%%%%%%%%%%%%%%%%%%%%%%%%%%%%%%%%%%%%%%%%%%%%%%%%
Strongly interacting ALPs, which decay before departing the local stellar region, modify stellar structure by contributing an additional term to the radiative opacity of the medium. The magnitude of this term is given by summing over the Rooseland mean opacities of the inverse Primakoff process and direct decay to photons. For the former this is given by \cite{Cadamuro:2011fd}
\begin{equation}
    \kappa_a^{P}=\frac{\int_{m_a}^{\infty}d\omega\ \omega^3\beta\frac{\partial}{\partial T} \frac{1}{\exp(\omega/T)-1}}{\rho \int_{m_a}^{\infty}d\omega\ \omega^3\lambda_a\beta\frac{\partial}{\partial T} \frac{1}{\exp(\omega/T)-1}}
\end{equation}
where \(\omega\) now refers to the ALP energy and \(\beta\) is the ALP velocity. The ALP mean-free path \(\lambda_a\) is given by \cite{Cadamuro:2011fd}
\begin{align}
    \lambda_{a}^{-1}=\sum_{Z}n_Z\sigma_{Z}^{bc}(\omega).
\end{align}
Here \(\sigma_{Z}^{bc}(\omega)\) is the cross-section for the inverse Primakoff process, or \textit{back-conversion} (bc), for a target of charge \(Ze\) and \(n_Z\) refers to its number density. This cross-section is given by \cite{Dolan:2017osp}
\begin{equation}
    \sigma_{Z}^{bc}=\frac{2}{\beta^2}\sigma_{Z}^P(\omega) \, ,
\end{equation}
where $\sigma_{Z}^P$ is the production cross-section.
The contribution towards ALP Rooseland mean opacity due to decay can be calculated in a similar fashion, although only the high-mass limit ($m_a/T\gg1)$ is relevant to us, given by \cite{Raffelt_Energy_Transfer}
\begin{equation}
    \label{eq: ALP opacity}
    \kappa_{a}^D=\frac{(2\pi)^{7/2}}{45\rho} \bigg(\frac{T}{m_a}\bigg)^{5/2}\exp(m_a/T)\ \Gamma_{a}.
\end{equation}
The total ALP opacity is then \(\kappa_a=\kappa_a^P+\kappa_a^D\) \cite{Dolan:2017osp}. For MeV scale ALPs direct decay is the dominant contribution towards their energy transfer. The total radiative opacity $\kappa_{\mathrm{Rad}}$ is then given as
\begin{equation}
    \kappa_{\mathrm{Rad}}^{-1}=\kappa_{\gamma}^{-1}+\kappa_{a}^{-1},
\end{equation}
where $\kappa_{\gamma}$ is the photon opacity. Note that the ALP contribution to energy transport would only impact the structural evolution of a star if the affected stellar region is radiative rather than convective.