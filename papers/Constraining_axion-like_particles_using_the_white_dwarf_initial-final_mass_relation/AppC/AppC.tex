\section{A Probabilistic Approach}
\label{sec: AppD}

\begin{figure}[t]
    \centering
    \includegraphics{AppC/Figures/WDIFMR_appendixC.pdf}
    \caption{A comparison between our derived constraint from \ref{subsec: rotn} (light blue) and that when a probabilistic approach is adopted (dark blue) in the keV-MeV region of the ALP-plane. Both constraints conservatively account for rotational mixing.}
    \label{fig: Prob constraint}
\end{figure}

In Section \ref{sec: Section 4} we adopted the less restrictive IFMR constraint from \cite{Andrews} in which the breakpoints are allowed to vary. Given the posterior sample pertaining to this constraint was not available, instead we conservatively insisted that theoretical IFMRs must fall within its entire range. The posterior sample with fixed breakpoints at $2M_{\odot}$ and $4M_{\odot}$, however, was made available, which enables us to establish what a probabilistic approach would entail.



We first define a 95\% confidence interval about the mean for uniformly spaced values of $M_{\mathrm{init}}$. This is shown by the red region in Figure \ref{fig: IFMR prob}, with mean indicated by the solid red line. The region corresponding to unfixed breakpoints is also included in green. The black solid line is a two-piece fit derived from our simulations (the black points in Figure \ref{fig: IFMR fit}), constrained with a breakpoint fixed at $4M_{\odot}$.


We can then repeat the analysis detailed in Section \ref{sec: ALPs and the IFMR} for a two-piece fit with breakpoint at $4M_{\odot}$ in order to generate a new result. The corresponding constraint, adjusted for ALP-decay and the influence of efficient rotation, is shown in Figure \ref{fig: Prob constraint} in dark blue. Also included is the constraint from Section \ref{subsec: rotn} in light blue.



By adopting this probabilistic approach, our constraint becomes considerably more restrictive. While the HB star bound is still more restrictive at low mass, it is worth noting that we have accounted for efficient rotational mixing in this calculation. If the treatment of rotation is relaxed entirely our constraint moves down to $g_{a\gamma\gamma}=0.316\times10^{-10}$ GeV$^{-1}$, below that of the HB star bound $g_{a\gamma\gamma}=0.66\times10^{-10}$ GeV$^{-1}$. A definitive statement on the matter, however, requires a deeper understanding of rotational mixing in stars.




Although we do not adopt this probabilistic approach for our main constraint, it demonstrates clearly the possible constraining power of the IFMR. It should be noted that the constraint \cite{Andrews} could be improved in the near future if it is applied to the larger double white dwarf binary dataset recently identified in the Gaia early Data Release 3 \cite{2021arXiv210105282E}. Such an analysis would be of great interest to particle physicists hoping to constrain axion-like particles.