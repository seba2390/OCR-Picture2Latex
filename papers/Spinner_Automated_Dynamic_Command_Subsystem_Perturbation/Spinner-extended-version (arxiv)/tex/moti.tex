%\vspace{-1em}
\section{Motivating Example}
\label{sec:motivation}


\autoref{fig:moti_example} shows an example of a \updated{}{shell} command injection attack to a vulnerable server-side program written in PHP. 
%The attack begins from the website that sends a \code{POST} request to the server-side program. %
%
From a website shown in \autoref{fig:moti_example}-(a), an attacker sends a malicious input (with a malicious command) through the textbox on the webpage. The server-side program, shown in \autoref{fig:moti_example}-(b), is vulnerable because it directly passes the input to {\tt system()}, executing the injected command (line 4). It tries to sanitize inputs via {\tt filter\_var()} at line 2 (commonly recommended~\cite{php_sanitize1, php_sanitize2, php_sanitize3}), but it fails. % to filter the malicious input. %: \redcode{http://...;\{rm,./*\}}. % successfully makes its way to the exploitation. %, showing that input sanitizations in practice often fail to work properly.

\noindent
{\bf Command Specification Randomization.} 
%We randomize the command targets (e.g., OS shell in this example) to prevent injected commands that are not aware of our randomization from being executed. 
In this example, \sysname randomizes commands in the shell process. % to prevent an injected OS (or shell) command from being executed.
%\autoref{fig:moti_example}-(d) and (e) show an example of how \sysname works. 
%\noindent
%{\it -- Shell Command Types:}
There are two types of shell commands~\cite{linux_commands_external_internal}: (1) internal commands that are implemented inside of OS/shells such as `\code{cd}' and (2) external commands that are implemented by separate binaries such as `\code{grep}'. 
% linux_commands_external_internal: https://linuxconfig.org/internal-vs-external-linux-shell-commands

For internal commands, we randomize the command names by hooking and overriding APIs in the shell process (Details in Section~\ref{subsubsec:shellcommand_rand}).
For external commands, since the shell process will look up a binary file for the external command to execute (i.e., check whether a binary for the command exists), we randomize the binary file names and paths (e.g., `\code{rm}' $\mapsto$ `\code{\color{red}os}' as shown in \autoref{fig:moti_example}-(e)) in the file I/O APIs. 
%For both types, we randomize the name of commands (e.g., name of internal commands and binary file paths of the external commands) in the OS/shell subsystem.
This will prevent injected malicious (\updated{}{and} {\it not randomized}) commands from being executed. 
%
%Details of how we mitigate the concerns of sophisticated attackers injecting randomized commands are discussed in Section~\ref{xxx}.
%To execute an external command (i.e., the second type), a program will look up a binary file for the external command first (i.e., check whether a binary for the command exists) and then execute the program. 
%\sysname also randomizes the file path of the supported commands, as shown in the fourth column of \autoref{fig:moti_example}-(e).
For the randomization, we use a one-time substitution cipher. % (i.e., 1:1 mapping between the characters that can be used to specify commands). 
%Specifically, 
%We conservatively assume that any printable ASCII code can be used. 
%Hence, 
Specifically, as shown in \autoref{fig:moti_example}-(d), we create a mapping between the original input and its randomized character. %: \optmap{/}{.}, \optmap{b}{l}, \optmap{i}{y}, \optmap{n}{e}, and so on.
%
%Specifically, characters `{\tt a}', `{\tt b}', and `{\tt c}' will be mapped to  `{\tt G}', `{\tt s}', and `{\tt ;}' respectively.  
To execute a command ``\code{wget}'' under this randomization scheme, one should execute ``\code{qjpc}'' as shown in  \autoref{fig:moti_example}-(e). 
To prevent brute-force attacks against the randomized commands, \sysname provides two mitigations. 
First, \sysname creates a new randomization scheme on \emph{every new command} to mitigate attacks leveraging previously used randomized commands.
Second, to further make the brute-force attacks difficult, \sysname supports one to multiple bytes translation, enlarging the searching space.
Details can be found in Appendix~\ref{appendix:bruteforce_attack}.
% shows how commands and binary paths are randomized according to \autoref{fig:moti_example}-(d).




\noindent
{\bf Instrumentation by \sysname.}
Once the commands are randomized in the shell process, the system cannot understand commands that are not randomized. In other words, it affects every command in the program including intended and benign commands, breaking benign functionalities.
To ensure the correct execution of intended commands, we statically analyze the program to identify intended (hence benign and trusted) commands that are originated from trusted sources (e.g., defined as a constant string or loaded from a trusted configuration file). We describe our bidirectional command composition analysis for identifying intended commands in Section~\ref{subsubsec:composition}. 
Then, we instrument the target program to randomize intended commands.
%MA: are you really instrumenting the rand() function? Don't you instrument the binary to detect "wget" as a benign function to randomize it later?
\autoref{fig:moti_example}-(c) shows the instrumented program. At line 9, as ``\code{wget}'' is the intended command (because it is a constant string), it is  instrumented with ``\code{rand()}''. Note that \code{\$url} that includes an injected command ``\code{\{rm,./*\}}'' is not instrumented because it is originated from an untrusted source (\code{\$\_POST[`url']}). %, hence it will not be executed.


\begin{figure}[h]
    \centering
    \vspace{1em}
    \includegraphics[width=0.85\columnwidth]{fig/trusted_specifications.pdf}
    \vspace{-1em}
    \caption{\revised{Trusted Command Specification Examples}}
    \vspace{-1em}
    \label{fig:spec_example}
\end{figure}



\begin{figure*}[ht]
    \centering
    \includegraphics[width=0.9\textwidth]{fig/overview.pdf}
    \vspace{-0.5em}
     \caption{Overview and Workflow of \sysname (Design details are presented in the annotated sections)}
     \vspace{-1em}
     \label{fig:overview}
\end{figure*}