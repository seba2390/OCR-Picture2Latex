\vspace{-0.5em}
\section{Discussion}
\label{sec:discussion}


\noindent
{\bf Prepared Statements.}
Prepared statements~\cite{prepared_statements} aim to prevent SQL injections by separating input data from a SQL query during the query construction.
While effective, they have limitations. First, some SQL keywords are not supported in prepared statements such as \code{PASSWORD} and \code{DESC} (5 more in Appendix~\ref{appendix:enc_sql_statements}). Second, changing existing SQL queries to prepared statements requires manual effort. 
\revised{Note that we manually check 866 SQL queries from all our target programs, and none of them is a prepared statement, showing the needs of \sysname in practice  (Appendix~\ref{appendix:enc_sql_statements}). %Third, SQL injections with prepared statements are still possible~\cite{phpdelusions_net}. 
}
%Third, they may incur higher overhead. 

\revised{
\noindent
{\bf Memory Disclosure on Randomization Records.}
\updated{After \sysname randomizes subsystems, it stores the used keys so that a randomized command name can be derandomized later.}{\sysname maintains randomization records that contain previously used randomization schemes.}
\updated{While the randomization record is deleted soon after a command execution API is called, there is still a short time window that the records exist on the memory.}{}
Attackers who can leak the memory pages containing the records may obtain \sysname's previously used keys. 
However, \sysname chooses a new randomization key on every new input. Hence, knowing previous randomization keys does not help in launching subsequent attacks.  
%\updated{xx}{}
%\CJ{2) [2.1] (\#E) How easy to evade detection (e.g., learning attacker)?}
%Note that in command injection attacks, it is difficult to overwrite already injected commands.
%
Also, existing memory protection techniques~\cite{intel-sgx} can be used to protect the records. 

\noindent 
{\bf Limitations.}
    When a target application is updated, one needs to run \sysname to analyze and instrument the updated target application. 
    Typically, this simply requires re-running \sysname on the updated application. 
    However, if an update significantly changes program code relevant to the trusted commands, it requires manual efforts to redefine the trusted command specifications. Further, we analyze updates of 42 applications, including popular programs, to check whether updates in practice lead to changes in trusted command specifications. The results show that they do not change trusted command specifications. Details are in Appendix~\ref{appendix:software_update_impact}. 
    If an application runs a completely dynamic command (e.g., \code{system(``\$\_GET[`cmd']'')}, \sysname blocks it and notifies users to fix the program. % (e.g., using constant commands selected by inputs can be a fix).
    %\CJ{@MW, fix XXX here}
    %\mw{Once the protection is applied, if a new module or plugin is added to the program, the protection of \sysname will be broken. We need to re-apply \sysname to the target application.}}\CJ{14) limitation}
}
%\noindent
\updated{{\bf Additional Discussions.}
In Appendix~\ref{appendix:additional_discussion}, we present more discussions on an alternative approach to \sysname and preventing the command injections with multiple commands.
}{}