%\vspace{-1em}
\section{Related Work}
\label{sec:related}

\noindent
{\bf Runtime Protection of Web Application.}
There have been many researchers that have proposed runtime protection systems against command and SQL injection attacks~\cite{nguyen-tuong-sql,  Haldar_2005, chin_2009, csse, sqlcheck, sqlrand, halfond06fse, halfond_wasp, sekar_ndss, AMNESIA, CANDID}. 
Taint tracking techniques track untrusted user inputs in server-side applications at runtime~\cite{nguyen-tuong-sql,  Haldar_2005, chin_2009, csse, sqlcheck}.
\cite{sekar_ndss,  AMNESIA} leverage static analysis to infer possible benign commands and use them to detect injection attacks. 
CANDID~\cite{CANDID}  employs dynamic analysis to extract and model an accurate structure of SQL queries. % so that they can be used for malicious queries.
%
\cite{halfond06fse} proposes positive tainting that dynamically tracks trusted inputs.  Unlike them, \sysname focuses on randomizing trusted commands, which is more lightweight than existing approaches (e.g., up to 19\% overhead in \cite{halfond06fse}).
Diglossia~\cite{diglossia} proposes a dual parsing technique that uses different languages during the parsing to detect injected SQL queries. However, it relies on parsers which can be exploited as shown in Sections~\ref{subsubsec:advanced_sql_injection} and \ref{subsec:comparison_existing}. 

Among the existing approaches, SQLRand~\cite{sqlrand, autorand} is the closest work to our approach. It randomizes SQL keywords and uses a proxy that can parse and derandomize the randomized SQL statements. %It then passes the derandomized SQL statements to the existing database engines.
%
%{\color{red}
Compared to SQLRand, \sysname does not rely on parsers which can be attacked and exploited as presented in Section~\ref{subsubsec:advanced_sql_injection}. 
%that require more generic system design, while SQLRand only prevents SQL injection. 
%Specifically, SQLRand does not consider commands defined by configurations files from users, which are commonly observed in command injection attacks. 
%Note that we also analyzed a version of SQLRand, sqlrand-llvm~\cite{sqlrand-llvm}, to show \sysname outperforms it (Section~\ref{subsec:comparison_existing}). 
%
There are also randomization based techniques such as Instruction Set Randomization~\cite{isr, isr2, isr3}. 
While sharing the randomization idea, \sysname's design provides solutions for preventing advanced attacks exploiting ambiguous grammars~\cite{sql_dialect}, as shown in Section~\ref{subsubsec:advanced_sql_injection}.
\cite{isr2} randomizes a programming language, leveraging a similar method to SQLRand. However, it is vulnerable to attacks exploiting language specification changes across different versions. %, which can be mitigated by our design. We leave this as future work.

% because it leverages a scanner (which is grammar agnostic) and the dual randomization scheme.
%\sysname supports dynamic randomization meaning that our randomization scheme is changing at runtime while  SQLRand's randomization is static. 
%AutoRand: https://dl.acm.org/doi/10.1007/978-3-319-40667-1_3 
% Java Bytecode, only for java
% https://link.springer.com/chapter/10.1007%2F978-3-319-40667-1_3
%}




\noindent
{\bf Security Analysis of Web Applications\updated{}{/Randomization}.}
Researchers have proposed various techniques to analyze vulnerabilities in web applications~\cite{webssari, xie_aiken, pixy, wassermann_2007,wassermann_2008,minamide_2005, noxes, saner_2008}. 
\cite{webssari} uses static analysis to identify vulnerabilities in PHP applications. 
Xie et al.~\cite{xie_aiken} propose a symbolic execution based program analysis technique to find SQL injection vulnerabilities. %Pixy~\cite{pixy} is an open source vulnerability analysis tool for PHP applications. 
String-taint analysis~\cite{wassermann_2007,wassermann_2008,minamide_2005} tracks untrusted substrings from user inputs to prevent information leak attacks.
%make sure no trusted scripts can be included in SQL queries and web pages generated by PHP programs.
\cite{saner_2008} combines dynamic and static analysis to find vulnerabilities in input sanitizers.
%
\sysname also uses static taint analysis and data flow analysis. % for randomizing commands and SQL queries. 
\updated{}{\cite{ahmed2020} studies the impact of timing of rerandomization. \sysname rerandomizes subsystems per input event, following the paper's recommendation.}

\noindent
{\bf Security Testing for Web Applications.}
Security testing aims to identify inputs that can expose input validation vulnerabilities in web applications~\cite{survey_bau, huang_2003, mcallister_2008, doupe_2012, martin_2008, kiezun_2009, saxena_2010a, saxena_2010b, saner_2008}. 
\cite{huang_2003} is a pioneer of web application testing by injecting XSS and SQL attacks. Mcallister et al.~\cite{mcallister_2008} propose a guided and stateful fuzzing technique to improve the performance. % to speed up the vulnerability searching.
%
%There are many techniques that enhance the effectiveness and efficiency of security testing.
Doup\'{e} et al.~\cite{doupe_2012} propose incrementally building a state machine during crawling to understand the internal structure of the web applications for better web application fuzzing. % the web application better.
To enhance input generation efficiency, Martin et al.~\cite{martin_2008} leverage model checking and static analysis, ARDILLA~\cite{kiezun_2009} applies symbolic execution, and Saxena et al.~\cite{saxena_2010a, saxena_2010b} use both dynamic taint analysis and symbolic execution for input mutation space pruning.
\updated{}{\cite{10.1145/3319535.3363195} systematically measures security issues in the payment card industry's webservices.}
%
\sysname aims to provide runtime protection. They are orthogonal to \sysname and are complementary. 
%can be used to improve \sysname's program analysis and instrumentation accuracy. 

%\updated{}{\noindent \mw{{\bf Security Measurement on Randomization}}}
%\updated{}{\mw{\cite{ahmed2020} measures the upper bound of re-randomization intervals for a secure randomization system and they pointed out that event-based re-randomization schemes can be effective since it does not rely on time interval which is affected by hardware or program itself but it may trigger unnecessary re-randomization if events are frequent. \sysname's randomization is event-based and the randomization only happens when the sink functions are invoked. }}\CJ{13) reviewer B: missing related papers}

%Moreover, techniques that automatically fix security bugs have been proposed~\cite{xxx}.

% 2. https://www.cc.gatech.edu/home/orso/papers/halfond.orso.ASE05.pdf (paper that describes the building of automata that represent sql statements and their monitoring at runtime -- note: this is from 2005...)
% 2. https://www.cc.gatech.edu/home/orso/papers/halfond.orso.ICSEDEMO06.presentation.pdf (presentation of the above)
% 3. https://patents.google.com/patent/US9882930B2/en (taint analysis technique patented in 2018)
% 4. https://patents.google.com/patent/US10382448B2/en (taint tracking technique that assumes the application's instructions can be differentiated through a "genetic material" that is different from the outside world -- from 2013 UVA!)
% 5. https://ieeexplore.ieee.org/abstract/document/8554270 (machine learning techique to detect command injection attacks -- I don't really like this paper)
% 6. https://sites.cs.ucsb.edu/~vigna/publications/2011_CCS_EAR.pdf (EAR - execution after return -- 2011 CCS)
% 7 https://people.eecs.berkeley.edu/~dawnsong/papers/2011%20Context-sensitive%20auto-sanitization%20in%20web%20templating%20languages%20using%20type%20qualifiers.pdf (type-qualifier based mechanism for sanitization)