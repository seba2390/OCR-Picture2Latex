\section{Simulation Results}
\label{sec:simulationResults}

In this section, we present numerical validation of our proposed approach for different dynamic locomotion. 
% The simulation video for this paper is given in Fig. \ref{fig:roughTerrainSim}. 
The reader is encouraged to watch the supplemental video\footnote{\url{https://youtu.be/Z2s4iuYkuvg}} 
for the visualization of our results.
For our simulation, the bipedal robot model and ground contact model are set up in MATLAB with Spatial v2 software. The MPC sampling frequency is set to $0.03\:\unit{s}$ while the simulation is run at $1\: \unit{kHz}$. One gait cycle that contains 10 horizons is predicted at each time step in MPC, in which each gait cycle is fixed at $0.30\:\unit{s}$. \update{This prediction length has been also used in \cite{di2018dynamic}.}

\update{The weighting factors $\bm Q$ in \eqref{eq:MPCform} are tuned to balance the performance between different control actions. In our simulation, we use $\bm Q_x = \bm Q_y = 50$, $\bm Q_z = 100$, $\bm Q_{\phi} = \bm Q_{\theta} = 100$, and $\bm Q_{\psi} = 20$. The rest weighting factors in $\bm Q$ remains at 1. }

\begin{figure}%[!h]
\vspace{0.5cm}
     \centering
     \begin{subfigure}[b]{0.13\textwidth}
         \centering
         \includegraphics[width=\textwidth]{Figures/pitchModel1.png}
         \caption{Snapshot of Model 1 in Double-leg Stance}
         \label{fig:pitchModel1}
     \end{subfigure}
     %\hfill
     \quad \quad
     \begin{subfigure}[b]{0.13\textwidth}
         \centering
         \includegraphics[width=\textwidth]{Figures/pitchModel3.png}
         \caption{Snapshot of Model 3 in Double-leg Stance}
         \label{fig:pitchModel3}
     \end{subfigure}
     \hfill
     \begin{subfigure}[b]{0.5\textwidth}
         \centering
         \includegraphics[width=0.9\textwidth]{Figures/PitchComp2_final.pdf}
         \caption{Pitch Motion Comparison}
         \label{fig:pitchPlot}
     \end{subfigure}
        \caption{{\bfseries Comparison of Model 1 and Model 3 in Pitch Motion Simulation}  a) Snapshot at the end of simulation with model 1  b) Snapshot at the end of simulation with model 3   c) Pitch motion response comparison with a $10^{\circ}$ desired pitch input.}
        \label{fig:pitchComparison}
\end{figure}

	
\begin{figure}%[!h]
	\hspace{0.2cm}
     \center
     \begin{subfigure}[b]{0.13\textwidth}
         \centering
         \includegraphics[width=\textwidth]{Figures/rolModel2.png}
         \caption{Snapshot of Model 2 in Double-leg Stance}
         \label{fig:rollModel2}
     \end{subfigure}
     %\hfill
     \quad \quad
     \begin{subfigure}[b]{0.14\textwidth}
         \centering
         \includegraphics[width=\textwidth]{Figures/rollModel3.png}
         \caption{Snapshot of Model 3 in Double-leg Stance}
         \label{fig:rollModel3}
     \end{subfigure}
     %\hfill
     \begin{subfigure}[b]{0.5\textwidth}
         \centering
         \includegraphics[width=0.9\textwidth]{Figures/RollComp2_final.pdf}
         \caption{Roll Motion Comparison}
         \label{fig:rollPlot}
     \end{subfigure}
        \caption{{\bfseries Comparison of Model 2 and Model 3 in Roll Motion Simulation}  a) Snapshot at the end of simulation with model 2  b) Snapshot at the end of simulation with model 3   c) Roll motion response comparison with a $10^{\circ}$ desired roll input.}
        \label{fig:rollComparison}
\end{figure}

\subsection{Validation of Simplified Dynamics}
First, we present the simulation results of simple rotation motions during standing with both legs on the ground to validate the claim in Section \ref{sec:dynamicsAndControl} that for the simplified dynamics used for control design, model 3 is a superior choice over model 1 and model 2. 

As mentioned in Section \ref{sec:dynamicsAndControl}, the simplified dynamics model 1 is unable to perform pitch motion. It is shown in Fig. \ref{fig:pitchComparison}, a pitch motion comparison between using simplified dynamics model 1 and model 3. The latter one is what we ultimately chose to use in MPC formulation. It is observed that the simulation result with model 1 does not respond to desired pitch input, whereas model 3 can perform pitch motion. 

We then further simplified model 1 and added 3-D moment inputs to each contact point to form simplified dynamics model 2. However, in the roll motion test, the response with model 2 is incorrect to desired roll input and it also shows a deviation in yaw angle as shown in Fig. \ref{fig:rollComparison}. With model 3, the robot simulation succeed in the roll motion test. 
Therefore, we decide to use model 3 for our proposed approach. Following are simulation results for walking and hopping motion using MPC control for model 3.

\subsection{Velocity Tracking}

In this simulation, we test the MPC performance in forward walking motion(positive $x$-direction) with time-varying desired speed and the desired CoM height of $0.5\:\unit{m}$. 
%The joint torques during this simulation are presented in Fig. \ref{fig:velSimTorque}. Note that during the simulation, all joint torque data are within the maximum torque threshold of our motor choice. There are no extreme torque values found during this entire simulation. 
The velocity tracking plot is shown in Fig. \ref{fig:velTracking}, the actual response curve with MPC shows a good tracking performance. The velocity response has a maximum deviation of $0.076\: \unit{m/s} $ compared to the desired input. Besides walking forward, we also have successful simulation results and demonstrations in walking sideways and diagonally. This result validates the effectiveness of our proposed control framework in realizing 3D dynamic locomotion for bipedal robots. 
% The simulation results can be found in the video (URL is under Fig. \ref{fig:roughTerrainSim}) associated with this paper. 
%\begin{figure}[h]
%	\center
%	\includegraphics[width=1 \columnwidth]{Figures/MPC_tau.png}
%	\caption{{\bfseries Plots of Joint Torques with MPC.} Joint torques of stance leg under the control of MPC in time-varying velocity simulation. }
%	\label{fig:velSimTorque}
%\end{figure}
\begin{figure}[h]
	\hspace{0.2cm}
	\center
	\includegraphics[width=0.75 \columnwidth]{Figures/velTracking_resized.pdf}
	\caption{{\bfseries Velocity Tracking.} Comparison of desired velocity input and actual velocity response in x-direction. 
% 	The desired velocity curve keeps constant from $t=0\:\unit{s}$ to $t=0.6\:\unit{s}$ and from $t=1.5\:\unit{s}$ to $t=3\:\unit{s}$ at $\bm v_{x_d}=0\:\unit{m/s}$ and $\bm v_{x_d}=0.6\:\unit{m/s}$, respectively. From $t=0.6\:\unit{s}$ to $t=1.5\:\unit{s}$ and from $t=3\:\unit{s}$ to $t=3.9\:\unit{s}$, $\bm v_{x_d}$ increases linearly.
	}
	\label{fig:velTracking}
\end{figure}
	
\subsection{High-velocity Walking in Rough Terrain}
We also validated the controller performance in rough terrain locomotion at high speed. Specifically, the robot is commanded to walk through a $2.4$-meter-long rough terrain formed by stairs with various heights and lengths. The stair heights range from $0.020\:\unit{m}$ to $0.075\:\unit{m}$ with a maximum height difference of $0.055\:\unit{m}$ between two consecutive stairs. To validate the feasibility and potential of MPC locomotion through rough terrain, the robot is commanded to follow a high desired velocity $\bm v_{x_d}=1.6 \:\unit{m/s}$. A snapshot of this simulation is provided in Fig. \ref{fig:roughTerrainSim}. 
%\begin{figure}[h]
%	\center
%	\includegraphics[width=0.85 \columnwidth]{Figures/Rough_terrain_spatial.png}
%	\caption{{\bfseries Snapshot from Rough Terrain Simulation.} Terrain model with stairs at various heights, animated with Spatial v2.  }
%	\label{fig:roughTerrain}
%\end{figure}

Plots of CoM location, velocity, and body orientation are shown in Fig. \ref{fig:CoMposVel} and Fig. \ref{fig:CoMeulAng}. It can be observed that the CoM location and orientation during this simulation maintain small tracking errors. 
%The position curves of the joints show that the joint position responses are smooth under the control of MPC and PD Cartesian control. The MPC controller input is presented in Fig. \ref{fig:MPCforce}. 
%The force in the $y$-direction $\bm F_y$ and moment $\bm M_z$ has the largest variation between the two legs. Hence it yields a slight $y$-direction displacement at the end of the simulation. As can be seen in CoM $y$-direction location in Fig. \ref{fig:CoMposVel}, the final $y$-direction location of body CoM is $0.0087 \:\unit m$ at $t = 1.8\: \unit{s}$. 
The joint torques (shown in Fig. \ref{fig:MPCtauRT}) during this entire simulation are in reasonable ranges and satisfy the torque saturation shown in Table \ref{tab:motor}. 
% todo{You should cite the table II about the torque limit here and also mention about the satisfaction of joint speed limit in Fig 11.} 
% (this is not really accurate so I commented it out)It is expected that the ankle joints will exert higher magnitudes of torques. Shown in the corresponding plots, the magnitudes of torques are still under the maximum torque threshold in most occasions. 

%With above simulation results and observations. It is inferred that the framework with the new MPC model presented in this paper can be a feasible option for a 10 DoF bipedal robot in dynamic locomotion. Our future work include extending this control framework to more dynamic motions such as bipedal bounding and running \todo{We have new results so please check sth like this throughout the paper to make sure that it is consistent with the current result}. Eventually, this control framework is expected to be migrated to a physical bipedal robot platform that is under development currently.
\begin{figure}[h]
	\hspace{0.2cm}
	\center
	\includegraphics[width=0.95 \columnwidth]{Figures/RTCoM2_final.pdf}
	\caption{{\bfseries Plots of Body CoM Position and Velocity in Rough Terrain Simulation.}  }
	\label{fig:CoMposVel}
\end{figure}
\begin{figure}[h]
	\center
	\includegraphics[width=0.88 \columnwidth]{Figures/RTCoMRot4_final.pdf}
	\caption{{\bfseries Plots of Robot Orientation in Rough Terrain Simulation. }  }
	\label{fig:CoMeulAng}
\end{figure}
%\begin{figure}
%	\hspace{0.2cm}
%	\center
%	\includegraphics[width=1 \columnwidth]{Figures/RTJoint_final.pdf}
%	\caption{{\bfseries Plots of Joint Position and Velocity  in Rough Terrain Simulation. }  \todo{Remove this}}
%	\label{fig:jointPos}
%\end{figure}
\begin{figure}[!h]
	\center
	\includegraphics[width=0.92 \columnwidth]{Figures/MPCForce3_final.pdf}
	\caption{{\bfseries Plots of MPC Force and Moment in Rough Terrain Simulation.  }}
	\label{fig:MPCforce}
\end{figure}
\begin{figure}%[!h]
	\hspace{0.2cm}
	\center
	\includegraphics[width=0.92 \columnwidth]{Figures/MPCTorque3_final.pdf}
	\caption{{\bfseries Plots of Joint Torques in Rough Terrain Simulation.  } }
	\label{fig:MPCtauRT}
\end{figure}

\subsection{Bipedal Hopping}
On top of the rotation and walking simulations presented earlier in this section, we have also implemented other gaits such as hopping. The hopping gait consists of a double support phase and a flight phase during the last quarter of each gait. 
% (we do not have flight phase in the previous one with trotting, so we should not mention this)With current MPC formulation, we decrease the flight phase duration in each gait cycle to mitigate the effects on performance during flight phase. 
A hopping gait illustration is shown in Fig. \ref{fig:boundGait}. It can be observed that during hopping motion, the robot is in a clear flight phase. 
%To validate the feasibility of hopping motion with the current MPC formulation, we test hopping forward motion with velocity $\bm v_{x_d}=0.5\:\unit{m/s}$. The CoM z-direction position and velocity plots are shown in Fig. \ref{fig:bounding_CoMposvel}. It is observed that the hopping motion can be performed with the current MPC formulation, with the trade-off of a certain degree of error in CoM velocity and height tracking. 
\update{This result validated that our proposed approach can work effectively for different dynamic locomotion on bipedal robots. We plan to optimize the MPC formulation in future work to enable faster and more aggressive motions. }

\begin{figure}%[!h]
	\center
	\includegraphics[width=1 \columnwidth]{Figures/boundGait.png}
	\caption{{\bfseries Illustration of Bipedal Hopping in Simulation  }  }
	\label{fig:boundGait}
\end{figure}
%\begin{figure}[!h]
%	\hspace{0.2cm}
%	\center
%	\includegraphics[width=1 \columnwidth]{Figures/BoundingCoM.pdf}
%	\caption{{\bfseries Plots of Z-direction CoM Location and Velocity in Hopping Simulation }  }
%	\label{fig:bounding_CoMposvel}
%\end{figure}
