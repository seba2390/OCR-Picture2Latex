%!TEX root = ../Main.tex

\section{Bipedal Robot Model and Simulation}
\label{sec:robotModel}

\subsection{Robot Model}
\label{subsec:robotModel}

In this section, we present the robot model that will be used throughout the paper. A 10 DoF bipedal robot consists of 5 joint actuation each leg (see Fig. \ref{fig:robotModel}). Commonly, a 10 DoF bipedal robot has abduction (ab) and hip joints that allow 3-D rotation and ankle joints that allow double-leg-support standing (e.g., \cite{levineblackbird,gong2019feedback}). 

The design of our bipedal consists of the robot body, ab link, hip link, thigh link, calf link, and foot link (see Fig. \ref{fig:legConfig} for the definition of the leg configuration and Table \ref{tab:PRP} for physical parameters). The robot body is adopted from the Unitree A1 robot, in a vertical configuration. The joint actuator modeled in this robot design is the Unitree A1 modular actuator, which is a lightweight and powerful torque-controlled motor that is suitable for mini legged robots (see Table \ref{tab:motor} for the parameters of the actuator). 
\begin{figure}[!h]
	\hspace{0.2cm}
     \center
     \begin{subfigure}[b]{0.227\textwidth}
         \centering
         \includegraphics[width=\textwidth]{Figures/Model_in_Blender.png}
         \caption{ }
         \label{fig:robotModel}
     \end{subfigure}
     %\hfill
     \begin{subfigure}[b]{0.21\textwidth}
         \centering
         \includegraphics[width=\textwidth]{Figures/Leg_config.png}
         \caption{ }
         \label{fig:legConfig}
     \end{subfigure}
        \caption{{\bfseries Bipedal Robot Configuration}  a) Robot CAD Model  b) The link and joint configuration of the bipedal robot left leg.}
        \label{fig:robotConfig}
\end{figure}

\begin{table}[h]
	\vspace{0.2cm}
	\centering
	\caption{Robot Physical Parameters}
	\label{tab:PRP}
	\begin{tabular}{cccc}
		\hline
		Parameter & Symbol & Value & Units\\
		\hline
		Mass & $m$    & 11.84 & $\unit{kg}$  \\[.5ex]
		Body Inertia  & $I_{xx}$  & 0.0443 & $\unit{kg}\cdot \unit{m}^2$ \\[.5ex]
		& $I_{yy}$ & 0.0535  & $\unit{kg}\cdot \unit{m}^2$ \\[.5ex]
		& $I_{zz}$ & 0.0214  & $\unit{kg}\cdot \unit{m} ^2$ \\[.5ex]
		Body Length & $l_{b}$ & 0.114 & $\unit{m}$ \\[.5ex]
		Body Width & $w_{b}$ & 0.194 & $\unit{m}$ \\[.5ex]
		Body Height & $h_{b}$ & 0.247 & $\unit{m}$ \\[.5ex]
		Thigh and Calf Lengths & $l_{1}, l_{2}$ & 0.2 & $\unit{m}$ \\[.5ex]
		Foot Length & $l_{toe}$ & 0.09 & $\unit{m}$ \\[.5ex]
		& $l_{heel}$ & 0.05 & $\unit{m}$ \\[.5ex]
		\hline 
		\label{tab:robot}
	\end{tabular}
\end{table}	

	\begin{table}[h]
%		\hspace{0.2cm}
		\centering
		\caption{Joint Actuator Parameters}
		\begin{tabular}{cccc}
			\hline
			Parameter & Value & Units\\
			\hline
			Max Torque   &  33.5 & $\unit{Nm}$  \\[.5ex]
			Max Joint Speed    & 21  & $\unit{Rad}/\unit{s}$  \\[.5ex]
			\hline 
			\label{tab:motor}
			\end{tabular}
			\end{table}
\subsection{Simulation}
\label{subsec:simulation}

The bipedal robot simulation is built primarily in MATLAB Simulink using Spatial v2 package \cite{featherstone2014rigid}. Fig. \ref{fig:controlArchi} shows the diagram for our control architecture and also the representation of our simulation software.
% is an implementation of spatial vector arithmetic and dynamics algorithms that is available in MATLAB scripts. The software employs Roy Featherstone’s book \emph{Rigid Body Dynamics Algorithms} and provides a series of accessible MATLAB functions for robotic dynamics simulations \cite{featherstone2014rigid}. 
% The scope of the simulation construction is to build easy-to-use and reliable MATLAB and Simulink models that can be used as a test bench for many forthcoming controller designs and optimization implementations (see Fig. \ref{fig:controlArchi}: simulation block diagram).

The simulation software requires the user to input desired states at the beginning of the simulation. The desired states form a column vector that consist desired body center of mass (CoM) position $\bm p_d$, desired body CoM velocity $\bm {\dot p}_d$, desired rotation matrix $\bm R_d$ (resized to 9×1 vector), and desired angular velocity $\bm {\omega}_d$ of robot body. 

\begin{figure}[h]
	\hspace{0.2cm}
		\center
		\includegraphics[width=1 \columnwidth]{Figures/Control_archi.png}
		\caption{{\bfseries Control Diagram.} The simplified block diagram and control architecture of our proposed approach.}
		\label{fig:controlArchi}
	\end{figure}

% The simulation model shown in Fig. \ref{fig:controlArchi} provides a platform for controller implementation with the advantages of fast simulation time, simplicity in modification and debugging, and reliability.