\section{Dynamics and Control}
\label{sec:dynamicsAndControl}
\subsection{Simplified Dynamics}
\label{subsec:simplified_dynamics}
In this Section, we investigate different dynamic models that can be used in our MPC control framework. 
While the whole-body dynamics of the bipedal robot are highly non-linear, we are interested in using simplified rigid-body dynamics to guarantee that our MPC controller can be solved effectively in real-time. 
In addition, in order to enable the capability of absorbing frequent and hard impacts from dynamic locomotion, the design of bipedal robots also requires lightweight limbs and connections. This has allowed the weight and rotation inertia of each link part to be very small compared to the body weight and rotation inertia. Hence, the effect of leg links in the robot dynamics may be neglected, forming simplified rigid-body dynamics \cite{nguyen2019optimized,bledt2018cheetah}. This is also a common assumption in many legged robots’ controller designs \cite{focchi2017high,stephens2010push}.

There are three simplified dynamics models that we have considered and tested, shown in Fig. \ref{fig:simplifiedDynamics}. The main difference between these three options is the number of contact points on each foot, contact location, and contact force and moment formation at each contact point. Model 1 resembles the simplified dynamics choice for quadruped robots mentioned in \cite{bledt2018cheetah}, with 4 contact points exerting 3-D contact forces. However, with this dynamics model applied to our 10-DoF bipedal robot, under rotation motions testings in simulation, the robot is unable to perform pitch motion properly. Model 2 is improved and further simplifies the contact points. However, with this simplified dynamics model, in simulation validation process, the robot is unable to perform roll motion correctly. Hence, we proposed model 3 that excludes external moment around $x$-axis, and contains only 3-D contact forces and 2-D moments around $y$ and $z$-axis. This model has allowed the robot to perform all 3-D rotations effectively. Hence, model 3 is chosen to be the final simplified dynamics design in this framework. More details about the validation of this decision process is shown in Section \ref{sec:simulationResults}.
The detailed derivation of the model 3 dynamics is presented as follows.

The bipedal model in this paper has two legs that both consists of 5 DoF. Commonly, the external contact forces applied to the robot are only limited to 3-D forces in many legged-robot dynamics (e.g., \cite{levineblackbird,nguyen2019optimized}). However, thanks to the additional hip and ankle joint actuation, the external moments can also be included in the robot dynamics, forming a linear relationship between robot body’s acceleration $ {\bm {\ddot p}_c}$, rate of change in angular momentum $\bm {\dot H}$ about CoM \cite{stephens2010push}, and contact force and moments %$ \bm F=[ \bm F_1,\:\bm F_2]^T$ , where $\bm F_i= [\bm F_{ix},\:\bm F_{iy},\:\bm F_{iz},\:\bm M_{ix},\:\bm M_{iy},\:\bm M_{iz} ]^T$, $i=1,2$ , as follows:
\update{$ \bm u=[\bm F_1,\:\bm F_2,\:\bm M_1,\:\bm M_2]^T,  \bm F_i = [\bm F_{ix},\:\bm F_{iy},\:\bm F_{iz}]^T, \bm M_i = [\bm M_{iy},\:\bm M_{iz}]^T, i = 1, 2, $ shown as follows: 
\setlength{\belowdisplayskip}{5pt} \setlength{\belowdisplayshortskip}{5pt}
\setlength{\abovedisplayskip}{5pt} \setlength{\abovedisplayshortskip}{5pt}
\begin{align}
\label{eq:simplifiedDynamics}
\left[\begin{array}{ccc} \bm {{D}_1}   \\ 
 \hspace{0cm} \bm {{D}_2} \end{array}  \right] \bm u= \left[\begin{array}{c} m (\ddot{\pos}_{c} +\bm{g}) \\ \bm {\dot H} \end{array} \right],
\end{align}
where
\begin{align}
\label{eq:D1}
 \bm {D}_1  = \left[\begin{array}{cccc} \mathbf {I}_{3\times3}  & \mathbf {I}_{3\times3} & 
  \mathbf {0}_{3\times2} &  \mathbf {0}_{3\times2} \end{array} \right],
\end{align}
\begin{align}
\label{eq:D2}
 \bm {D}_2  = \left[\begin{array}{cccc} \bm {(p_1-p_c)}\times  & \bm {(p_2-p_c)}\times & 
  \mathbf L &  \mathbf L \end{array} \right],
\end{align}
\begin{align}
\label{eq:L}
 \mathbf L  = \left[\begin{array}{ccc}  \mathbf 0 & \mathbf 0 \\  \mathbf 1 &\mathbf 0 \\  \mathbf 0 &\mathbf 1
 \end{array} \right].
\end{align} }

The term ${(\bm p_i- \bm p_c)}$ denotes the distance vector from the robot body CoM location to the foot $i$ position in the world coordinate; and $\bm {(p_i-p_c)}\times $ represents the skew-symmetric matrix representing the cross product of ${ {(\bm p_i-\bm p_c)} \times \bm F_i }$. Here, $\bm {\dot H}$ can be approximated as $\bm{ \dot H=I_G \dot {{\omega}}}$  (as discussed in \cite{nguyen2019optimized}), where $\bm {I_G}$ stands for the centroid rotation inertia of robot body in the world frame and  $\bm {\dot {\omega}}$ represents the angular velocity of robot body in the world frame \cite{bledt2018cheetah,stephens2010push}.
	


\begin{figure}[!h]
\vspace{0.2cm}
     \centering
     \begin{subfigure}[b]{0.15\textwidth}
         \centering
         \includegraphics[width=\textwidth]{Figures/simplifiedDynamics1.png}
         \caption{Model 1}
         \label{fig:model1}
     \end{subfigure}
     \hfill
     \begin{subfigure}[b]{0.13\textwidth}
         \centering
         \includegraphics[width=\textwidth]{Figures/simplifiedDynamics2.png}
         \caption{Model 2}
         \label{fig:model2}
     \end{subfigure}
     \hfill
     \begin{subfigure}[b]{0.145\textwidth}
         \centering
         \includegraphics[width=\textwidth]{Figures/simplifiedDynamics3.png}
         \caption{Model 3}
         \label{fig:model3}
     \end{subfigure}
        \caption{{\bfseries Three Simplified Dynamics Models.} Yellow dots on the figures represent the contact point locations in simplified dynamics a) 2 contact points at the toe and heel of each robot foot, 3-D forces applied  b) 1 contact point at the middle of each foot, 3-D force and 3-D moments applied  c) 1 contact point at the middle of each foot, 3-D force and 2-D moments applied. Model 3 is the final choice to be used in our proposed approach.}
        \label{fig:simplifiedDynamics}
\end{figure}
%\todo{We should have a figure to illustrate the simplified rigid body model including 1 rigid body, 2 contact point on the ground, force vectors, My, Mz,...}
%\todo{We should also consider make comparisons between different models: the current one (1 contact point each leg: 3 forces + 2 moments for each point), 2 contact point each leg (3 forces + 3 moments for each points), 2 contact point each leg (3 forces + 0 moments for each points)... like you used to explore. This will make the paper stronger.}
\update{Equation (\ref{eq:simplifiedDynamics}) to (\ref{eq:L}) describes the simplified dynamics model 3. This model only used 5 force and moment inputs $\bm U$ which are directly mapped to 5 joint torques in each leg.}

We use rotation matrix $\bm R$ as a state variable to represent the orientation of the robot body, which can be also directly converted to Euler Angles. 
%Reference \cite{di2018dynamic} presents the 
We linearize the rotation matrix by approximating the angular velocity in terms of Euler angle $ {\Theta = [\phi,\:\theta,\:\psi]}^T$, where $\phi$ is roll angle,  $ \theta$ is pitch angle, and $ \psi$ is yaw angle. With the assumption of small roll and pitch angles  \cite{di2018dynamic}, the relation of the rate of change of $\Theta$ and angular velocity $\bm \omega$ in the world coordinate can be approximated as: 
%\begin{align}
%\label{eq:omega}
%%\bm \omega  = \left[\begin{array}{ccc} {\cos{\theta} \cos{\psi}}& -{\sin{\psi}} & 0 \\{\cos{\theta} \sin{\psi}} & {\cos{\psi}} & 0 \\ -{\sin{\theta}} & 0 & 1\end{array} 
%\right ] 
%\begin{bmatrix} {\dot{\phi}} \\ {\dot{\theta}} \\ {\dot{\psi}}\end{bmatrix}
%\end{align} 

%\begin{align}
%\label{eq:omega}
%\bm \omega  = \left[\begin{array}{ccc} %\dot{\phi}\sin{\theta}\sin{\psi}+\dot{\theta}\cos{\psi} \\ 
%\dot{\phi}\sin{\theta}\cos{\psi}-\dot{\theta}\sin{\psi} \\ 
%\dot{\phi}\cos{\theta}+\dot{\psi}
%\end{array} 
%\right ] 
%\end{align} 

%With small pitch and roll angles, equation (\ref{eq:omega}) can be approximated and rewritten as
\begin{align}
\label{eq:omega2}
\begin{bmatrix} {\dot{\phi}} \\ {\dot{\theta}} \\ {\dot{\psi}}\end{bmatrix} \approx \left[\begin{array}{ccc} { \cos{\psi}}& {\sin{\psi}} & 0 \\{ -\sin{\psi}} & {\cos{\psi}} & 0 \\ 0 & 0 & 1\end{array} 
\right ] 
\begin{array}{cc}
     {\bm \omega}
\end{array}.
\end{align} 
% \todo{You should double-check eq 5,6 because they may not be correct in [9].}

Hence the kinematic constraint of the Euler angles is obtained as follow:
\begin{align}
\label{eq:omega3}
\begin{bmatrix} {\dot{\phi}} \\ {\dot{\theta}} \\ {\dot{\psi}}\end{bmatrix} \approx \bm R_z(\psi) \bm \omega.
\end{align} 

Combing the approximated orientation dynamics and the translation dynamics, the simplified dynamics of the robot can be written as: 
\update{
\begin{align}
\label{eq:stateSpace}
\frac{d}{dt}
\begin{bmatrix} 
{\bm \Theta} \\
{\bm p}_c \\ 
{\bm \omega} \\ 
\dot {{\bm p}}_c
\end{bmatrix} = 
\bm A
\begin{bmatrix} 
{\bm \Theta} \\
{\bm p}_c \\ 
{\bm \omega} \\ 
\dot {{\bm p}}_c
\end{bmatrix} 
+ \bm B \bm u +
\begin{bmatrix} 
\mathbf 0 \\ \mathbf 0 \\\mathbf  0 \\\mathbf  g
\end{bmatrix},
\end{align} 
where
\begin{align}
\label{eq:A}
\bm A = 
\begin{bmatrix} 
\mathbf {0}_{3\times3} & \mathbf {0}_{3\times3} & \bm R_z(\psi) & \mathbf {0}_{3\times3}  \\
\mathbf {0}_{3\times3} & \mathbf {0}_{3\times3} & \mathbf {0}_{3\times3} & \mathbf  {I}_{3\times3}  \\ 
\mathbf {0}_{3\times3} & \mathbf {0}_{3\times3} & \mathbf {0}_{3\times3} & \mathbf {0}_{3\times3} \\ 
\mathbf {0}_{3\times3} & \mathbf {0}_{3\times3} & \mathbf {0}_{3\times3} & \mathbf {0}_{3\times3}, 
\end{bmatrix},
\end{align} 
\begin{align}
\setlength\arraycolsep{2.5pt}
\label{eq:B}
\bm B = 
\begin{bmatrix} 
\mathbf {0}_{3\times3} & \mathbf {0}_{3\times3} & \mathbf {0}_{3\times2} & \mathbf {0}_{3\times2}  \\
\mathbf {0}_{3\times3} & \mathbf {0}_{3\times3} & \mathbf {0}_{3\times2} & \mathbf {0}_{3\times2}  \\ 
\bm {{\Tilde{I}^{-1}_G} (p_1-p_c)}\times & \bm {{\Tilde{I}^{-1}_G} (p_2-p_c)}\times & \bm{{\Tilde{I}^{-1}_G} \mathbf L} & \bm{{\Tilde{I}^{-1}_G} \mathbf L} \\ 
\mathbf { {I}_{3\times3}}/m & \mathbf {{I}_{3\times3}}/m & \mathbf {0}_{3\times2} & \mathbf {0}_{3\times2} 
\end{bmatrix}
\end{align} }

\update{Here, $\bm{\Tilde{I}^{-1}_G}$ is approximated by rotation inertia of the robot body in its body frame $\bm I_b$ and $\bm R_z(\psi)$ from \eqref{eq:omega3}:
\begin{equation}
\label{eq:rotationI}
\bm{\Tilde{I}_G} = \bm R_z(\psi) \bm I_b {\bm R_z(\psi)}^T.
\end{equation}
}
By assigning gravity as additional state variable, now state ${{\bm x}} = [{{\bm \Theta}},\: {\bm p}_c,\:{{\bm \omega}},\:\dot { {\bm p}}_c,\:\bm g]^T$ will allow the dynamics in (\ref{eq:stateSpace}) to be rewritten into a linear state-space form with continuous time matrices $\bm {\hat{A_c}}$ and $\bm {\hat{B_c}}$: 
\begin{equation}
\label{eq:linearSS}
\dot{{\bm { x}}}(t) = \bm {\hat{A_c}}(\psi) {{\bm {x}}}(t) + \bm {\hat{B_c}}([p_1-p_c],[p_2-p_c],\psi) \bm u(t).
\end{equation}

The linearized dynamics in (\ref{eq:linearSS}) is now suitable for the convex MPC formulation presented in \cite{di2018dynamic}.

\subsection{MPC Formulation}
\label{subsec:MPC}
Having discussed the dynamics model, we now present details about the formulation of our MPC controller.

The linearized dynamics in (\ref{eq:linearSS}) can be represented in a discrete-time form at each time step $i$
\begin{align}
\label{eq:discreteDynamics}
{\bm {x}}[i+1] = \bm {\hat{A}}[i] \bm x[i] + \bm {\hat{B}}[i]\bm u[i],
\end{align}
where discrete time matrix $\bm {\hat{A}}$ is a constant matrix computed from $\bm {\hat{A_c}}(\psi)$ using a average yaw value during entire reference trajectory; and $\bm {\hat{B}}$ matrix is computed from $ \bm{\hat{B_c}}([p_1-p_c],[p_2-p_c],\psi)$, using the desired values of average yaw and foot location. The only exception is that at the first time step,  $\bm {\hat{B}}[1]$ is computed from current states of the robot instead of reference trajectory.

An MPC problem with a finite horizon length $k$ can be written in the following standard form:
\begin{align}
\label{eq:MPCform}
%\nonumber
\underset{\bm{x,u}}{\operatorname{min}}   \:\:  & \sum_{i = 0}^{k-1}(\bm x_{i+1}-  {\bm x_{i+1}}_{ref})^T\bm Q_i(\bm x_{i+1}- {\bm x_{i+1}}_{ref}) + \| \bm{u}_i \|\bm{R}_i
\end{align}
\begin{align}
\label{eq:dynamicCons}
\:\:\mbox{s.t. }& \quad  {\bm {x}}[i+1] = \bm {\hat{A}}[i]\bm x[i] + \bm {\hat{B}}[i]\bm u[i], \quad i = 0 \dots k-1
\end{align}
\begin{align}
\label{eq:MPCineqCons}
\quad  \bm {c^-}_i \leq \bm C_i\bm u_i \leq \bm {c^+}_i, \quad i = 0 \dots k-1
\end{align}
\begin{align}
\label{eq:MPCeqCons}
& \quad \bm D_i \bm u_i = 0 , \quad i = 0 \dots k-1
\end{align}

%\todo{We should use the format $x^TQx$ instead of $||x||Q$ in the cost function.}

In (\ref{eq:MPCform}), $\bm x_i$ and $\bm u_i$ are system states and control inputs at time step $i$. \update{Note that the MPC prediction is computed based on the measured states of current step (i.e. $i = 0$).} $\bm Q_i$ and $\bm R_i$ are matrices defining the weights of \update{each state and control input variable.} $\bm {\hat{A}}$ and $\bm {\hat{B}}$ in (\ref{eq:dynamicCons}) are the discrete-time system dynamic constraints from (\ref{eq:discreteDynamics}). $\bm {c^-}_i$,$\bm {c^+}_i$, and $\bm C_i$ in (\ref{eq:MPCineqCons}) represents the inequality constraints of the MPC problem. $\bm D_i$ in (\ref{eq:MPCeqCons}) represents the equality constraints. In this problem, the equality constraint governs the optimal control input from MPC controller is a zero vector for swing foot. 

The MPC controller solves the optimal ground contact force and moment with respect to dynamic constraints \eqref{eq:dynamicCons} and the following inequality constraints:
\begin{align}
\label{eq:frictionCons}
\nonumber 
-\mu \bm {F}_{iz} \leq \bm F_{ix} \leq \mu \bm {F}_{iz} \\
-\mu \bm {F}_{iz} \leq \bm F_{iy} \leq \mu \bm {F}_{iz} \\
% \end{align}
% \begin{align}
\label{eq:forceCons}
0< \bm {F}_{min} \leq \bm  F_{iz} \leq \bm {F}_{max} \\
% \end{align}
% \begin{align}
\label{eq:torqueCons}
\bm |{\tau}_{i}| \leq \bm {\tau}_{max}. 
\end{align}
Here, \eqref{eq:frictionCons} governs the contact forces in $x$ and $y$ direction are within the friction pyramid, with $\mu$ being the friction coefficient. The contact forces in $z$-direction should also fall within the upper and lower bounds of force (\ref{eq:forceCons}), where the lower bound is positive to maintain contact with the ground. It is also important to restrict the joint torques to be within the saturation of the physical motor (\ref{eq:torqueCons}). 


\subsection{QP Formulation}
% It is stated in Section \ref{sec:robotModel} of this paper that the scope of developing the robot simulation in MATLAB and Simulink is to have a fast and reliable simulation software to test controller designs. MPC problem can be heavy to solve so it is important to reduce the computation and problem size of MPC by formulating the problem into a quadratic program (QP).
With the linear dynamics in Section \ref{subsec:simplified_dynamics} and the MPC formulation in Section \ref{subsec:MPC}, our controller can be formulated as a quadratic program (QP) that can be solved effectively in real-time. 
% With given equality and inequality constraints, we can form a QP problem based on the dynamics from the condensed formulation in (\ref{eq:QPdynamics}). The Optimization Toolbox in MATLAB provides powerful and fast QP solver which is suitable for this problem. 

Firstly, the dynamic constraints \eqref{eq:linearSS} for the entire MPC prediction horizon can be written as:
\begin{align}
\label{eq:QPdynamics}
\bm X = \bm{A}_{qp} \bm x_0 + \bm{B}_{qp} \bm U,
\end{align}
where $\bm X$ is a column vector containing system states for the next $k$ horizons, $\bm x[i+1],\bm x[i+2] \dots {\bm x}[i+k]$ and $\bm U$ is a column vector containing optimal control inputs of current state $\bm u[i]$ and next $k-1$ horizons, $\bm u[i+1], \bm u[i+2] \dots {\bm u}[i+k-1]$ at time step $i$. 
The MPC controller now can be written as the following QP form:     
\begin{align}
\label{eq:QPform}
%\nonumber
\underset{\bm{U}}{\operatorname{min}}   \:\:  & \frac{1}{2}\bm U^T\bm h \bm U + \bm U^T\bm f \\
% \end{align}
% \begin{align}
\label{eq:QPineqCons}
\:\:\mbox{s.t. }& \quad  \bm C\bm U \leq \bm d \\
% \end{align}
% \begin{align}
\label{eq:QPeqCons}
& \bm A_{eq}\bm U = \bm b_{eq}
\end{align}
where $\bm C$ and $\bm d$ are inequality constraint matrices, $\bm A_{eq}$ and $\bm b_{eq}$ are equality constraint matrices, and 
\begin{align}
\label{eq:h}
\bm h = 2( {\bm B_{qp}}^T\mathbf M {\bm B_{qp}}+\mathbf K), \\
% \end{align}
% \begin{align}
\label{eq:f}
\bm f = 2 {\bm B_{qp}}^T\mathbf M ({\bm A_{qp}}\bm x_0-\bm y).
\end{align}
Diagonal matrices $\mathbf K$ and $\mathbf M$ are the weights for the rate of change of state variables and force/moment magnitude.

\update{The resulting controller input of each leg from QP problem $\bm u_i=[\bm F_i,\: \bm M_i]^T$ is mapped to its joint torques by
\begin{align}
\label{eq:forceTorquemap}
\bm {\tau}_i =  \bm J_i^T \bm u_{i},
\end{align} 
where $\bm J_i$ is the Jacobian matrix of $i$th leg 
\begin{align}
\label{eq:Jacobian}
\bm {J}_i^T = \begin{bmatrix} 
\bm {J}_v^T & \bm {J}_\omega^T \bm L
\end{bmatrix},
\end{align} 
}
with $\bm {J}_v $ and $\bm {J}_{\omega} $ being the linear velocity and angular velocity components of $\bm {J}_i $.

\subsection{Swing Leg Control}
As discussed earlier in this section, due to equality constraints, the robot leg that is under the swing phase does not exert ground contact forces and therefore is not under the control of force-and-moment-based MPC. In order to control the leg and foot position in each gait cycle, the desired foot trajectory is under control in the Cartesian space with a PD position controller. \update{The gait sequence is purely based on timing and the gait cycle length is currently set at $0.3 \unit{s}$.}
We obtain the current foot location using forward kinematics. Foot velocity is computed by:
\begin{align}
\label{eq:footVel}
{\dot {\bm p}}_{{foot}_i} =  \bm J_i^T \dot{\bm q_i},
\end{align}
where $\dot{\bm q_i}$ is the joint velocity state-feedback of each leg at time step $i$.

The desired foot location $\bm p_{{foot}_d}$ in the world frame is determined by the foot placement policy employed in \cite{di2018dynamic}:
\begin{align}
\label{eq:footPlacement}
\bm p_{{foot}_d} =  \bm p_{hip} + \dot{\bm p}_c \Delta t/2,
\end{align}
where $\bm p_{hip}$ is the hip joint location in the world frame and $\Delta t$ is the time that stance foot spends on the ground during one gait cycle. 

The swing leg force can be computed by treating the foot attached to a virtual spring-damper system \cite{chen2020virtual}. The foot weight is reasonable to be neglected since it is very small compared to the robot body \cite{nguyen2019optimized}. Following the PD control law, the foot force can be written as: 
\begin{align}
\label{eq:PDswing}
\bm F_{swing_i}=\bm K_P(\bm p_{{foot}_d}-\bm p_{{foot}_i})+\bm K_D(\dot{\bm p}_{{foot}_d}-\dot{\bm p}_{{foot}_i})
\end{align}
where $\bm K_P$ and $\bm K_D$ are PD control gains, or spring stiffness and damping coefficient of the virtual spring-damper system. %$\bm p_{{foot}_d}$ and $\dot{\bm p}_{{foot}_d}$ are desired foot placement and velocity, respectively.

Similar to (\ref{eq:forceTorquemap}), the joint torque can be computed by:
\begin{align}
\label{eq:forceTorqueMapSwing}
\bm {\tau}_{swing_i} =  \bm J_v^T \bm F_{swing_i}.
\end{align}

With Cartesian PD control, the swing leg can move and be controlled to follow desired foot placement trajectory. The gait generator decides either the robot leg is in the stance phase or swing phase in a fixed gait cycle and assigns the appropriate controller to the corresponding leg. Now the robot has both swing and stance leg control, it is ready to test the MPC in simulation.