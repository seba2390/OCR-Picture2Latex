%!TEX root = ../Main.tex


\section{Introduction}
\label{sec:Introduction}

The motivation of studying bipedal robots is widely promoted by commercial and sociological interests \cite{westervelt2018feedback}. The desired outcomes of bipedal robot applications range from replacing humans in hazardous operations \cite{chen2020a}, in which it requires highly dynamic robots in unknown complex terrains, to the development of highly functional bipedal robot applications in the medical field and rehabilitation processes such as recent development in research of powered lower-limb prostheses for the disabled \cite{zhao2017first}. 

%Building effective force-based controller remains of the most important cornerstones of achieving dynamic motions while keeping balance for legged robots, because bipedal robot has inherently less stable dynamical system, compared to other sub-classes of legged robots such as quadruped robots 

% \update{Building an effective force-based controller remains one of the most important cornerstones of achieving dynamic motions while keeping balance for legged robots, because bipedal robots are inherently less stable dynamical systems, compared to other sub-classes of legged robots such as quadruped robots} \cite{westervelt2018feedback}. 

%There are many controller design and choices that are employed in bipedal balance control, including controllers that utilize 
\update{There are many control strategies that can be used for control of bipedal robots, such as} Zero-Moment-Point (ZMP) or using spring-loaded inverted pendulum (SLIP) model \cite{ames2012dynamically,kajita2006biped,holmes2006dynamics}. Both methods have had success in maintaining stable locomotion of bipedal robots (e.g. \cite{ames2012dynamically,holmes2006dynamics}). Hybrid Zero Dynamics method is another control framework utilizes input-output linearization, a non-linear feedback controller, with virtual constraints that allows dynamic walking on under-actuated bipedal robots \cite{westervelt2018feedback, nguyen2016dynamic, nguyen2017dynamic}. 
% (This is not published nor open source. So, we should not mention it here)Others use forced-based control such as Model Predictive Control (MPC) to maintain stability during dynamically motions \cite{levineblackbird}. 
Recently, force-based MPC control was introduced for dynamic quadruped robots \cite{di2018dynamic}, allowing the robot to perform a wide range of dynamic gaits with robustness to rough terrains. One advantage of the MPC framework in dynamic locomotion is that the controller can predict future motions that may cause instability due to under-actuation and stabilize the system by solving for optimal inputs based on the prediction. 
% There are many successful controller designs that employs MPC control during dynamical motions in the world of legged robots (e.g. \cite{levineblackbird,di2018dynamic}). 

\begin{figure}[t]
		\center
		\includegraphics[width=1 \columnwidth]{Figures/RoughTerrain_sim.png}
		\caption{{\bfseries 10-DoF Bipedal Robot.} The bipedal robot walking through rough terrain with velocity $ {{\bm v}_x}_d=1.6\:\unit{m/s}$. Simulation video: \protect\url{https://youtu.be/Z2s4iuYkuvg}. }
		\label{fig:roughTerrainSim}
	\end{figure}
	
MPC has been also utilized in the control of bipedal robots through various approaches. The control framework proposed in  \cite{powell2015model} applies MPC to minimize the values of rapidly exponentially stabilizing control Lyapunov function (RES-CLF) in a Human-Inspired Control (HIC) approach. A revisited ZMP Preview Control scheme presented in \cite{wieber2006trajectory} attempts to solve an optimal control problem by MPC that finds an optimal sequence of jerks of robot center of mass (CoM). 
However, these approaches are either based on position control to track a joint trajectory resulting from optimization; or tend to address the step-by-step planning problem. In this work, we focus on a real-time feedback control approach that can handle a wide range of walking gaits, without relying on offline trajectory optimization.
%\todo{Add 1 paragraph to talk about other approaches for control of bipedal robots such as Hybrid Zero Dynamics. You can check my stepping stones paper for reference.}

Inspired by the successful force-based MPC approach for quadruped robots presented in \cite{di2018dynamic}, in this paper, we propose a new control framework that utilizes MPC to solve for optimal ground reaction forces and moments to achieve dynamic motion on bipedal robots. We investigate different models that can be used for the MPC framework and introduce the formulation that works most effectively on our bipedal robot model. Our proposed approach allows a 10-DOF bipedal robot to perform high-speed and robust locomotion on rough terrain. We implement and validate our controller design in a high-fidelity physical simulation that is constructed in MATLAB and Simulink with the software dependency of Spatial v2.     


	
%\todo{Add 1 paragraph to emphasize the main contribution of our approach here}
The main contributions of the paper are as follows:
\begin{itemize}
    \item We proposed a new framework of force-and-moment-based MPC for 10-DoF bipedal robot locomotion.
    
    \item We investigate different models of rigid body dynamics that can be used for the MPC framework. The most effective model is then used in our proposed approach.
    
    \item The proposed MPC framework allows 3-D dynamic locomotion with accurate velocity tracking.
    \item Our control framework can enable a wide range of dynamic locomotion such as fast walking, hopping, and running using the same set of control parameters. 
    \item Thanks to using the force-and-moment based control inputs, our approach is also robust to rough terrain. We have successfully demonstrated the problem of fast walking with the velocity of $1.6\:\unit{m/s}$ on rough terrain. The rough terrain consists of stairs with a maximum height of $0.075\:\unit{m}$ and a maximum difference of $0.055\:\unit{m}$ between two consecutive stairs.
    % \item The controller design is planned be extended to physical robot testing in the future work.
\end{itemize}

The rest of the paper is organized as follows. Section \ref{sec:robotModel} introduces the model design and physical parameters of the bipedal robot. Simulation methods and control architecture are also provided in this section. Section \ref{sec:dynamicsAndControl} presents the dynamics and controller choices, design, and formulation of the proposed force-and-moment-based MPC controller. Some result highlights are presented in Section \ref{sec:simulationResults} along with an analysis on controller performance in various dynamic motions. 
