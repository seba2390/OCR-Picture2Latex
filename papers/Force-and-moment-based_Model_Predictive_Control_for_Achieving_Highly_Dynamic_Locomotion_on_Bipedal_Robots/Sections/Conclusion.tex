%!TEX root = ../Main.tex


\section{Conclusions}
\label{sec:Conclusion}
In conclusion, we introduced an effective approach of force-and-moment-based Model Predictive Control to achieve highly dynamic locomotion on rough terrains for 10 degrees of freedom bipedal robots. Our framework also allows the robot to achieve a wide range of 3-D motions using the same control framework with the same set of control parameters. 
% This control framework and robot model is designed to perform in highly dynamic 3-D motion. 
The convex MPC formulation can be translated into a Quadratic Program problem and solved effectively in real-time of less than $1 \unit{ms}$. 
% Working in conjunction with Cartesian PD control for swing leg, we validate the MPC performance in a simulation. 
We explore and find the most suitable dynamics model for the control framework and we have presented successful walking simulations with time-varying velocity input, rough-terrain locomotion with high velocity and results in different dynamic gaits. Simulation results have indicated that the control performance in the velocity tracking test has a maximum deviation of $0.076\: \unit{m/s} $ compared to the desired input. In the rough terrain test, the robot is able to walk through rough terrain with various heights while maintaining a high forward walking velocity at $1.6\:\unit{m/s}$. 
% The control framework can be extended to many other dynamic motions such as bipedal running and hopping. We are working on improving performance and modifying MPC formulation for each type of motion, to achieve better results. 
Future work will include extending the approach for more aggressive motion and experimental validation of the framework on the robot hardware.
% This control framework and MPC design are also expected to be extended to a physical bipedal robot for testing and validation in the near future. 

