
% Sets up the class for the document as (?)
\documentclass[letterpaper, 10 pt, conference]{ieeeconf}
% If the IEEEtran.cls has not been installed into the LaTeX system files,
% manually specify the path to it:
% \documentclass[conference]{../sty/IEEEtran}
%\IEEEoverridecommandlockouts		
%\overrideIEEEmargins

%\pdfminorversion=5
%\pdfcompresslevel=9
%\pdfobjcompresslevel=2

% Imports the needed LaTeX packages
\usepackage{amsmath}
\usepackage{amssymb}
%\usepackage{dblfloatfix}
\let\proof\relax 
\let\endproof\relax

\usepackage{tabularx}
%\usepackage{arydshln,graphicx,xcolor,array}
%\usepackage{amsthm}
\usepackage{graphicx}
\usepackage{bm}
\usepackage{color}
%\usepackage{needspace}
%\usepackage{empheq}
\usepackage{float}
\usepackage{units}
%\usepackage{footmisc}
\usepackage{url}
\usepackage{hyperref}
\usepackage{balance}
\usepackage{amsmath}

\makeatletter 
\def\endfigure{\end@float} 
\def\endtable{\end@float}
\makeatother 

\newcommand{\quat}{\bm{q}}
\newcommand{\pos}{\bm{p}}
\newcommand{\angVel}{\bm{\omega}}
\newcommand{\vel}{\bm{v}}
\newcommand{\posDot}{\bm{\dot{\pos}}}
\newcommand{\posDotDot}{\bm{\ddot{p}}}
\newcommand{\quatDot}{\bm{\dot{\quat}}}
\newcommand{\angVelDot}{\bm{\dot{\angVel}}}
\newcommand{\bOm}    {\mbox{\boldmath $\Omega$}}
\newcommand{\bom}    {\mbox{\boldmath $\omega$}}

%\usepackage{silence} 
%\WarningFilter{caption}{}
%\usepackage{subfig}
\usepackage{subcaption}
\captionsetup{compatibility=false}
\usepackage{caption}
% \usepackage[belowskip=0pt,aboveskip=1pt]{caption}
% \setlength{\abovedisplayskip}{5pt}
% \setlength{\belowdisplayskip}{5pt}

\captionsetup[figure]{font=footnotesize}

\renewcommand{\unit}[1]{{\rm #1} }

\newcommand{\qtn}[1]{{\color{red}QTN: #1}}
\newcommand{\todo}[1]{\textcolor{red}{#1}}
\newcommand{\update}[1]{\textcolor{black}{#1}}

\newtheorem{remark}{Remark}
	
% Imports the custom Commands file
% %!TEX root = ../Main.tex
% Defines commands to easily input variables and equations into the paper.

% Matrix commands for simple matrices
\newcommand{\pMat}[1]{\begin{pmatrix}#1\end{pmatrix}}
\newcommand{\bMat}[1]{\begin{bmatrix}#1\end{bmatrix}}
\newcommand{\BMat}[1]{\begin{Bmatrix}#1\end{Bmatrix}}
\newcommand{\aMat}[1]{\begin{Vmatrix}#1\end{Vmatrix}}
\newcommand{\mat}[1]{\begin{matrix}#1\end{matrix}}

% Pat commands
\newcommand{\cross}[1]{ {[#1]}_{\times} }

% Reference Frames
\newcommand{\IFrame}{\mathcal{I}}
\newcommand{\BFrame}{\mathcal{B}}
\newcommand{\COM}{CoM}

% State definitions
\newcommand{\roll}{\theta}
\newcommand{\pitch}{\phi}
\newcommand{\yaw}{\psi}

\newcommand{\rollhat}{\hat{\theta}}
\newcommand{\pitchhat}{\hat{\phi}}
\newcommand{\yawhat}{\hat{\psi}}



\newcommand{\dRoll}{\dot{\roll}}
\newcommand{\dPitch}{\dot{\pitch}}
\newcommand{\dYaw}{\dot{\yaw}}

\newcommand{\omegaX}{\omega_x}
\newcommand{\omegaY}{\omega_y}
\newcommand{\omegaZ}{\omega_z}

\newcommand{\dX}{\dot{x}}
\newcommand{\dY}{\dot{y}}
\newcommand{\dZ}{\dot{z}}

\newcommand{\quat}{\bm{q}}
\newcommand{\pos}{\bm{p}}
\newcommand{\angVel}{\bm{\omega}}
\newcommand{\vel}{\bm{v}}
\newcommand{\posDot}{\bm{\dot{\pos}}}
\newcommand{\posDotDot}{\bm{\ddot{p}}}
\newcommand{\quatDot}{\bm{\dot{\quat}}}
\newcommand{\angVelDot}{\bm{\dot{\angVel}}}

% Desired Variables
\newcommand{\xd}{\bm{x}_d}
\newcommand{\uref}{\bm{u}_{ref}}

% State component vectors
\newcommand{\quatV}{q_0 \\ q_x \\ q_y \\ q_z}
\newcommand{\quatVector}{\BMat{\quatV}}
\newcommand{\posV}{x \\ y \\ z}
\newcommand{\posVector}{\BMat{\posV}}
\newcommand{\angVelV}{\omegaX \\ \omegaY \\ \omegaZ}
\newcommand{\angVelVector}{\BMat{\angVelV}}
\newcommand{\velV}{\dX \\ \dY \\ \dZ}
\newcommand{\velVector}{\BMat{\velV}}

\newcommand{\rpyV}{\roll \\ \pitch \\ \yaw}
\newcommand{\rpyVector}{\BMat{\rpyV}}
\newcommand{\rpyDotV}{\dRoll \\ \dPitch \\ \dYaw}
\newcommand{\rpyDotVector}{\BMat{\rpyDotV}}

% Parameter Estimates
\newcommand{\state}{\bm{x}}
\newcommand{\stateDot}{\bm{\dot{\state}}}
\newcommand{\stateEstimate}{\bm{\hat{x}}}
\newcommand{\rFootEstimate}{\bm{\hat{r}}}
\newcommand{\forceEstimate}{\bm{\hat{F}}}
\newcommand{\rFootEstimatef}{\bm{\hat{r}}_f}
\newcommand{\forceEstimatef}{\bm{\hat{F}}_f}
\newcommand{\inputU}{\bm{u}}
\newcommand{\inputEstimate}{\bm{\hat{u}}}
\newcommand{\outputEstimate}{\bm{\hat{y}}}

% State and Input Vectors
\newcommand{\stateVector}{\BMat{\posV \\ \quatV \\ \velV \\ \angVelV}}
\newcommand{\stateVectorRed}{\BMat{\posV \\ \rpyV \\ \velV \\ \rpyDotV}}
\newcommand{\inputVector}{\BMat{\rFootEstimate_1 \\ \forceEstimate_1 \\ \rFootEstimate_2 \\ \forceEstimate_2 \\ \rFootEstimate_3 \\ \forceEstimate_3 \\ \rFootEstimate_4 \\ \forceEstimate_4}}

% Variable Transformations
\newcommand{\quatToRPY}{\bMat{\tan^{-1}\pMat{\dfrac{2\pMat{q_0q_1+q_2q_3}}{1-2\pMat{q_1^2+q_2^2}}}\\ \sin^{-1}\pMat{2\pMat{q_0q_2+q_3q_1}} \\ \tan^{-1}\pMat{\dfrac{2\pMat{q_0q_3+q_1q_2}}{1-2\pMat{q_2^2+q_3^2}}}}}
\newcommand{\omegaToRPYDot}{\bMat{\dfrac{\cos\pMat{\yaw}}{\cos\pMat{\pitch}} & \dfrac{\sin\pMat{\yaw}}{\cos\pMat{\pitch}} & 0 \\ -\sin\pMat{\yaw} & \cos\pMat{\yaw} & 0 \\ \dfrac{\cos\pMat{\yaw}\sin\pMat{\pitch}}{\cos\pMat{\pitch}} & \dfrac{\sin\pMat{\yaw}\sin\pMat{\pitch}}{\cos\pMat{\pitch}} & 1}}

% Time step definitions
\newcommand{\n}[1]{#1_n}
\renewcommand{\k}[1]{#1_k}
\newcommand{\nk}[1]{#1_{i}}

% Decision Variables Full
\newcommand{\inputVarsV}{\rFootEstimate \\ \forceEstimate}
\newcommand{\inputVarsVectorf}[1]{\BMat{\inputVarsV}_#1}
\newcommand{\decisionVarsVFull}{\inputVarsVectorf{1} \\ \vdots \\ \inputVarsVectorf{4} \\ \stateEstimate}
\newcommand{\decisionVarsVectorFull}[2]{\BMat{\decisionVarsVFull}_{#1,#2}}
\newcommand{\decisionArrayVFull}{\decisionVarsVectorFull{1}{1} & \hdots & \decisionVarsVectorFull{1}{K_1} & \hdots & \decisionVarsVectorFull{N}{1} & \hdots & \decisionVarsVectorFull{N}{K_N}}
\newcommand{\decisionArrayVectorFull}{\bMat{\decisionArrayVFull}}

% Decision Variables 
\newcommand{\decisionVarsV}{\stateEstimate \\ \inputEstimate}
\newcommand{\decisionVarsVector}{\BMat{\decisionVarsV}}
\newcommand{\decisionArrayV}{\decisionVarsVector_{1,1} & \hdots & \decisionVarsVector_{N,K_N}}
\newcommand{\decisionArrayVector}{\bMat{\decisionArrayV}}

% Simple Dynamics
\newcommand{\forces}{\bm{f}}
\newcommand{\torques}{\bm{\tau}}
\newcommand{\forceIEstimate}{\phantom{{}^\IFrame}\bm{\hat{f}}}
\newcommand{\torquesBEstimate}{^\BFrame\bm{\hat{\tau}}}
\newcommand{\RMatItoB}{^\BFrame\mathbf{R}_\IFrame}
\newcommand{\RMatItoBEstimate}{^\BFrame\mathbf{\hat{R}}_\IFrame}
\newcommand{\RMatBtoI}{^\IFrame\mathbf{R}_\BFrame}
\newcommand{\inputForceTorqueFunction}{ \hat{h}\!\pMat{\nk{\stateEstimate}, \nk{\inputEstimate}}}
\newcommand{\inputForceTorque}{\BMat{\phantom{{}^\BFrame}\forces \\ {}^\BFrame\torques}}
\newcommand{\inputForceTorqueMPC}{\BMat{\forceIEstimate \\ \torquesBEstimate}}
\newcommand{\forceTorqueTransform}{\bMat{\mathbf{I} & 0 \\ 0 & {\bm{R}_z^T(\yawhat)}}}
\newcommand{\uToForceTorqueMat}{\bMat{\mat{\mathbf{I} \\ \bMat{\bm{r}_1}_\times} & \hdots & \mat{\mathbf{I} \\ \bMat{\bm{r}_f}_\times}}}
\newcommand{\uToForceTorqueMatEstimate}{\bMat{\mat{\mathbf{I} \\ \bMat{\rFootEstimate_1}_\times} & \hdots & \mat{\mathbf{I} \\ \bMat{\rFootEstimate_f}_\times}}}
% Equation
\newcommand{\AMatnk}{{\bm{A}}_i}
\newcommand{\BMatnk}{{\bm{B}}_i}
\newcommand{\simpleDynamicsEq}{\AMatnk\nk{\stateEstimate}+\BMatnk\inputForceTorqueFunction+\bm{d}_i}
\newcommand{\simpleDynamics}{\stateEstimate_{i+1}=\simpleDynamicsEq}
% Matrix Linear
\newcommand{\AMatFull}{\bMat{\mathbf{I}_6 & \Delta t_i\mathbf{I}_6 \\ 0 & \mathbf{I}_6}}
\newcommand{\BMatFull}{\bMat{\frac{\Delta t_i^2}{2} \,{}^{\BFrame}\!\bm{I}^{-1} \\[.75ex] \Delta t_i\, {}^{\BFrame}\!\bm{I}^{-1} }}

\newcommand{\BMatFcull}{\bMat{\bMat{\frac{dt_n^2}{2}\mathbf{I}_6 \\ dt_n\mathbf{I}_6}\bMat{\frac{1}{m}\mathbf{I}_3 & 0 \\ 0 & ^\BFrame\bm{I}^{-1}}}}
	
	
	
% Full Dynamics
\newcommand{\sumForces}{\sum^\IFrame\bm{F}}
\newcommand{\sumTorques}{\sum^\IFrame\bm{\tau}}
\newcommand{\stateReducedVector}{\BMat{\pos \\ {}^\IFrame\!\bm{R}_{\BFrame} \\ \posDot \\ \angVel}}

\newcommand{\dynamicsReducedVector}{\BMat{
\posDot_c \\[.5ex]
{}^\IFrame\!\bm{R}_{\BFrame} \left[ {}^\BFrame \angVel\right]_\times \\[.5ex] 
\frac{1}{m}\forces - \bm{g} \\[.5ex]  
{}^\BFrame\,\overline{\!\bm{I}}{}^{-1}\pMat{{}^\BFrame\torques - {}^\BFrame \angVel\times {{}^\BFrame\,\overline{\!\bm{I}}}\,\,{\vphantom{\bm{I}}}^\BFrame\angVel}
}}


\newcommand{\stateDotReducedVector}{\BMat{\posDot_c \\ {}^\IFrame\!\dot{\bm{R}}_{\BFrame} \\ \posDotDot_c \\ {}^\BFrame\angVelDot}}

% Reference Input
\newcommand{\KPercent}{K_\%}
\newcommand{\pFootRef}{\bm{p}_{ref,f,xy,k} = \bm{\hat{p}}_{h,xy} + \posDot_{d,xy}T_{s}\pMat{\frac{1}{2}-\KPercent}+\\\pMat{\bm{\dot{\hat{p}}}_{xy} -\posDot_{d,xy}}\sqrt{\frac{\hat{p}_z}{\aMat{\bm{g}}}}\pMat{1 - \KPercent}}
			
% Error definitions
\newcommand{\stateError}{\bm{\tilde{x}}}
\newcommand{\inputError}{\bm{\tilde{u}}}
\newcommand{\stateErrorFull}{\xd-\outputEstimate}
\newcommand{\inputErrorFull}{\uref-\inputEstimate}
			
% Basic Cost Function
\newcommand{\basicArgs}{\stateEstimate, \hat{\bm{u}}}
\newcommand{\basicMin}{\underset{\basicArgs}{\text{min}}}
\newcommand{\basicCost}{J\pMat{\basicArgs}}
\newcommand{\basicCostFunction}{\basicMin\basicCost}
			
% MPC Function
\newcommand{\MPCSum}{\sum\limits_{n=1}^{N}\sum\limits_{k=1}^{\n{K}}}
%\newcommand{\MPCMin}{\underset{\stateEstimate, \inputEstimate}{\text{min}}}
\newcommand{\MPCMin}{\underset{\chi}{\text{min}}}
\newcommand{\stateCost}{\stateError^T\bm{Q}\stateError}
\newcommand{\inputCost}{\inputError^T\bm{R}\inputError}
\newcommand{\MPCCost}{\stateCost+\inputCost}
\newcommand{\MPCCostFunction}{\MPCMin\MPCSum\pMat{\MPCCost}_{n,k}}
			
% MPC Function Full
\newcommand{\MPCMinFull}{\underset{\stateEstimate, \inputEstimate}{\text{min}}}
\newcommand{\stateCostFull}{\pMat{\stateErrorFull}^T\bm{Q}\pMat{\stateErrorFull}}
\newcommand{\inputCostFull}{\pMat{\inputErrorFull}^T\bm{R}\pMat{\inputErrorFull}}
\newcommand{\MPCCostFull}{\stateCostFull+\inputCostFull}
\newcommand{\MPCCostFunctionFull}{\MPCMinFull\MPCSum\pMat{\MPCCostFull}_{n,k}}
			
% Constraints
\newcommand{\dynamicsConstraint}{\stateEstimate_{nk+1}=\simpleDynamicsEq}
\newcommand{\groundConstraint}{p_{z,f}=\hat{z}_g \pMat{p_{x,f}, p_{y,f}}}
\newcommand{\kinematicLegConstraint}{\aMat{\bm{r}_{h \to f}} \leq l_{f,max}}
\newcommand{\slipConstraint}{\bm{p}_{f,k+1} = \bm{p}_{f,k}}
\newcommand{\frictionConstraint}[1]{{-}\mu\bm{F}_{z,f}\leq\bm{F}_{#1,f}\leq\mu\bm{F}_{z,f}}
\newcommand{\frictionConeConstraint}{\frictionConstraint{x} \\ \frictionConstraint{y}}




%%%%%%%%%%%%%%%%%%%%%%%%%%%%%%%%%%%%%%%%%%%%%%%%%%%%%%%
%%% CONTACT
%%%%%%%%%%%%%%%%%%%%%%%%%%%%%%%%%%%%%%%%%%%%%%%%%%%%%%%


\newcommand{\legState}{s}
\newcommand{\legStateScheduled}{\legState_{\phi}}
\newcommand{\legStateNotScheduled}{\bar{\legState}_{\phi}}
\newcommand{\legStateEstimated}{\hat{\legState}}
\newcommand{\contactState}{c}
\newcommand{\swingState}{\bar{\contactState}}
\newcommand{\contactStateScheduled}{\contactState_{\phi}}
\newcommand{\swingStateScheduled}{\bar{\contactState}_{\phi}}


%%%%%%%%%%%%%%%%%%%%%%%%%%%%%%%%%%%%%%%%%%%%%%%%%%%%%%%
%%% Support Polygon
%%%%%%%%%%%%%%%%%%%%%%%%%%%%%%%%%%%%%%%%%%%%%%%%%%%%%%%
\newcommand{\phaseGain}{\Phi}
\newcommand{\adjLegP}{^{+}}
\newcommand{\adjLegM}{^{-}}
% \input{Utils/pmw_commands.tex}

\overrideIEEEmargins
\IEEEoverridecommandlockouts

% Beginning of actual document
\begin{document} 

% Paper title (needs to change)
\title{\Large \bf 			
Force-and-moment-based Model Predictive Control for Achieving Highly Dynamic Locomotion on Bipedal Robots
}
	
%\newtheorem{assumption}{Assumption}

\author{Junheng Li and Quan Nguyen\thanks{Junheng Li and Quan Nguyen are with the Department of Aerospace and Mechanical Engineering, University of Southern California, Los Angeles, CA 90089.
email:{\tt\small junhengl@usc.edu, quann@usc.edu}}%
}%													
	
% make the title area
\maketitle
%Pins
%Junheng Li 150116

\begin{abstract}

In this paper, we propose a novel framework on force-and-moment-based Model Predictive Control (MPC) for dynamic legged robots. Specifically, we present a formulation of MPC designed for 10 degree-of-freedom (DoF) bipedal robots using simplified rigid body dynamics with input forces and moments. This MPC controller will calculate the optimal inputs applied to the robot, including 3-D forces and 2-D moments at each foot. These desired inputs will then be generated by mapping these forces and moments to motor torques of 5 actuators on each leg. We evaluate our proposed control design on physical simulation of a 10 degree-of-freedom (DoF) bipedal robot. The robot can achieve fast walking speed up to $ 1.6\:\unit{m/s}$ on rough terrain, with accurate velocity tracking. With the same control framework, our proposed approach can achieve a wide range of dynamic motions including walking, hopping, and running using the same set of control parameters.

\end{abstract}


% Introduction and literature review
\section{Introduction}
\label{sec:Introduction}


The goal in top-$\size$ recommendation is to recommend to each
consumer a small set of $\size$ items from a large collection of
items~\cite{cremonesi2010performance}.  For example, Netflix may want
to recommend $\size$ appealing movies to each consumer.  Collaborative
Filtering (CF)~\cite{herlocker2002empirical,lee2012comparative} is a
common top-$\size$ recommendation method.  CF infers user interests by
analyzing partially observed user-item interaction data, such as user
ratings on movies or historical purchase
logs~\cite{kanagal2012supercharging}. The main assumption in CF is that
users with similar interaction patterns have similar interests.


Standard CF methods for top-$\size$ recommendation focus on making  suggestions  that accurately reflect the user's preference history. However, as  observed in previous work,  CF recommendations are generally biased toward  popular items, leading to a rich get richer effect~\cite{vargas2014improving,steck2011item}.  The major reasons for this are \textit{popularity bias} and \textit{sparsity} of CF interaction data (detailed in Section~\ref{sec:related-work}). In a nutshell, to maintain  accuracy, recommendations are generated from the dense regions of the data,  where the popular items lie.  

However,  accurately suggesting popular items, may not be satisfactory for the consumers. For example, in Netflix, an accuracy-focused movie recommender may recommend ``Star Wars: The Force Awakens'' to users who have seen ``Star Wars: Rogue One''.  But, those users are probably already aware of ``The Force Awakens''. Considering additional factors, such as novelty of recommendations,  can lead to more effective suggestions~\cite{cremonesi2010performance,Castells2015,zhang2008avoiding,ziegler2005improving,zhang2012auralist}. 
%Second, accuracy-focused models typically achieve a   overall item-space coverage across their recommendations,  whereas high item-space coverage helps providers of the items increase revenue
%, users satisfaction since they are  likely already aware of or can find these items on their own.  

Focusing on popular items also adversely affects the satisfaction of  the providers of the items. This is because  accuracy-focused models typically achieve a  low overall item space coverage across their recommendations, whereas   high item space coverage helps providers of the items increase their revenue~\cite{vargas2014improving,Castells2015,adomavicius2011maximizing,anderson2006thelongtail, yin2012challenging,adomavicius2012improving}.
%accuracy-focused models typically achieve a

In contrast to the relatively small number of popular items, there are copious  {\it long-tail\/} items that have fewer observations (e.g., ratings) available. More precisely,  using the Pareto  principle (i.e.,~the $80/20$ rule),  long-tail items can be defined as items that generate the lower $20\%$ of observations~\cite{yin2012challenging}. Experimentally we found that these items correspond to almost $85\%$ of the items in several datasets (Sections~\ref{sec:Notation} and \ref{sec:Experiments}). %Table~\ref{tab:DatasetStatsticsSmall})


As previously shown, one way to improve the novelty of top-$\size$ sets is to recommend interesting long-tail items~\cite{cremonesi2010performance,ge2010beyond}.  The intuition  is that since they have fewer observations available,  they are more likely to be unseen~\cite{Kaminskas:2016:DSN:3028254.2926720}.  
 %For example, in online commerce,  newly added items are long-tail items that are yet to be discovered.  
Moreover, long-tail item promotion also results in higher overall coverage of the item space%, which increases profits for providers of the items
~\cite{vargas2014improving,Castells2015,zhang2008avoiding,zhang2012auralist,adomavicius2011maximizing,anderson2006thelongtail,yin2012challenging,jambor2010optimizing}. Because long-tail promotion reduces accuracy~\cite{steck2011item}, there are trade-offs to be explored.


%original submitted to ICDE
%This work studies three aspects of top-$\size$ recommendation: accuracy, novelty, and item-space coverage, and examines their trade-offs. In most previous work, predictions of a base recommendation system are re-ranked to handle their trade-offs~\cite{adomavicius2012improving,jambor2010optimizing,zhang2013personalize,wang2009portfolio}. Due to performance considerations, however, these techniques are not customized per user. For example,  parameters that balance the trade-off between novelty and accuracy are cross-validated at a global level.  This can be detrimental since users have varying preferences for  objectives such as long-tail novelty. We explore how to  automatically infer  user  preference for long-tail novelty, and how to leverage  it to correct  the popularity bias in standard recommender models. Our work does not rely on any additional contextual data, although such data, if available, can help promote newly-added long-tail items~\cite{agarwal2009regression,Saveski:2014:ICR:2645710.2645751}.

This work studies three aspects of top-$\size$ recommendation: accuracy, novelty, and item space coverage, and examines their trade-offs. In most previous work, predictions of a base recommendation algorithm are \textit{re-ranked} to handle these trade-offs~\cite{adomavicius2012improving,jambor2010optimizing,zhang2013personalize,wang2009portfolio}. The re-ranking models are computationally efficient but suffer from two drawbacks. First, due to performance considerations,  parameters that balance the trade-off between novelty and accuracy  are not customized per user. Instead they are cross-validated at a global level.  This can be detrimental since users have varying preferences for  objectives such as long-tail novelty. Second,  the re-ranking methods are often limited to a specific base recommender  that may be sensitive to dataset density. 
As a result, the datasets are pruned and the problem is studied in dense settings~\cite{adomavicius2012improving,ho2014likes}; but real world  scenarios are often sparse~\cite{kanagal2012supercharging,liu2017experimental}.   
% Because  dataset density can impact the performance of most base recommenders (like R-SVD), which in turn affects the performance of the re-ranking model, 

\iffalse
We address these limitations by directly inferring  user  preference for long-tail novelty  from interaction data.  This  allows us to customize the re-ranking  per user, and design a \textit{generic} framework, which resolves the second problem. In particular, since the long-tail novelty preferences are estimated independently of any base  recommender model, we can  plug-in an appropriate base recommender w.r.t. the dataset sparsity.% including ones that are more suitable for sparse settings.  

Modelling  user  preference for  long-tail novelty using only item popularity statistics, e.g., the average popularity of rated items as in~\cite{jugovac2017efficient}, disregards additional information like whether the user found the item interesting and the long-tail preferences of other users  of the items. \iffalse To incorporate them, we introduce the notion of  \emph{item long-tail importance}. Both  user long-tail preferences and item long-tail importance are dependent:  a user has high preference for discovering long-tail items if she is interested in important long-tail items, and an item that is associated with many of these kinds of users is likely to be more important.  We propose a joint optimization framework to directly learn,  from interaction data, both the users' long-tail preferences and the  items' long-tail importance. \fi
We propose an optimization approach that  incorporates  this information and  directly learns,  from interaction data, the users' long-tail novelty preferences.

Next, we use these learned preferences  to design a  top-$\size$ recommendation framework thats is generic, and provides customized balance between accuracy, novelty, and coverage. We refer to it as framework as GANC.  Using GANC, we design a novel algorithm, {\it Ordered Sampling-based Locally Greedy (OSLG)\/}, that relies on the learned long-tail novelty preferences  to scalably correct for popularity bias. Our work does not rely on any additional contextual data, although such data, if available, can help promote newly-added long-tail items~\cite{agarwal2009regression,Saveski:2014:ICR:2645710.2645751}. In summary:
\fi

We address the first limitation by directly inferring  user  preference for long-tail novelty  from interaction data.   Estimating these  preferences  using only item popularity statistics, e.g., the average popularity of rated items as in~\cite{jugovac2017efficient}, disregards additional information, like whether the user found the item interesting or the long-tail preferences of other users  of the items. We propose an approach that  incorporates  this information and  learns the users' long-tail novelty preferences from interaction data.

This approach allows us to customize the re-ranking  per user, and  design a \textit{generic} re-ranking framework, which resolves the second limitation of prior work. In particular, since the long-tail novelty preferences are estimated independently of any base recommender, we can  plug-in an appropriate one w.r.t. different factors, such as the dataset sparsity.

Our top-$\size$ recommendation framework, \textbf{GANC}, is \textbf{G}eneric, and provides customized balance between \textbf{A}ccuracy, \textbf{N}ovelty, and \textbf{C}overage. % Moreover, based on the learned long-tail novelty preferences, we also design a novel algorithm, {\it Ordered Sampling-based Locally Greedy (OSLG)\/}, that relies on the learned long-tail novelty preferences  to scalably correct for popularity bias. 
Our work does not rely on any additional contextual data, although such data, if available, can help promote newly-added long-tail items~\cite{agarwal2009regression,Saveski:2014:ICR:2645710.2645751}. In summary:

%Consider  the following toy example:
\vspace{-0.2cm}
\begin{table}[htb]
\centering
\scriptsize
%\small
\begin{tabular}{ccccccc} 
%\toprule
%&\multirow{2}{*}{}&\multicolumn{7}{c}{Ratings}\\
& & \cellcolor{blue!35}$w_1$ &\cellcolor{blue!18} $w_2$ & $\dots$ &\cellcolor{blue!8} $w_{89}$  &\cellcolor{blue!8} $w_{99}$   
\\
&   &$i_1$&$i_2$&$\dots$&$i_{89}$&$i_{90}$\\ 
\cmidrule(r){3-7} 	 
%\midrule
\cellcolor{red!35}$\theta_1$  &$u_1 $   &5 &   & $\dots$ &  &   \\
\cellcolor{red!28}$\theta_2$  &$u_2$     &5 &    & $\dots$ &  &  \\
 $\theta_3=?$  &$\bf u_3$  &5 &  &   $\dots$ &  &  \\
\cellcolor{red!10}$\theta_4$ & $u_4$  &  &5   & $\dots$ & &\\ 
\cellcolor{red!10}$\theta_5$ & $u_5$  &  & 5  & $\dots$ & &\\ 
$\theta_6=?$  & $\bf u_6$ & &5  &      $\dots$& &  \\ 
 & & $\hdots$  &$\hdots$   &$\hdots$   &$\hdots$   &$\hdots$  \\
%\midrule 
\cmidrule(r){3-7} 	 
\multicolumn{2}{c}{item pop.}  & 3  & 3  & $\dots$ &50&60\\  
%\bottomrule
%$ f_i$    &3  &3  &1  &3  &1  &2  \\  \hline
\end{tabular}
%#.
\caption{Simplified user-item interaction data. The user long-tail novelty preference ($\theta_u$), item long-tail importance weight ($w_i$) are highlighted. Darker colors indicate larger values. } \label{tab:example}
\end{table} 
\vspace{-0.2cm}
\begin{example}  
In Table~\ref{tab:example}, we are interested in estimating $\theta_3$ and $\theta_6$,  the long-tail preference of users $u_3$ and $u_6$ who have each rated a single movie. Additional ratings for other users  are not included here.  Considering only rating information, we observe $i_1$ and $i_2$ are  equally popular $|\mathcal{U}_{i_1}^{\trainset}| = |\mathcal{U}_{i_2}^{\trainset}|=3$, and $r_{31}=5$ and $r_{62}=5$. Using Eq.~\ref{eq:tfidf-risk}  we have $\theta_3 = \theta_6$. However, if we were given the long-tail preferences of the each item's user set, specifically that $u_1$ and $u_2$ have high long-tail preference (darker red), while $u_4$ and $u_5$ have lower long-tail preference (lighter red), we could conclude $i_1$ is a more important long-tail item compared to $i_2$ (indicated by a darker blue shade for $w_1$), and we expect  $\theta_3 \geq \theta_6$.

% On the other hand, if we knew that $u_4$ and $u_5$ have lower long-tail preference, we could conclude $i_2$ is a  less significant long-tail item. Therefore, However, if we  consider the long-tail preferences of other users, we may reason differently.    We need another variable $w_i$ which captures this information. 
%we would conclude that $u_3$ has higher long-tail preference compared to $u_6$, since the users $i_1$ is a more prominent long-tail item. 

% Relying only  on item popularity information, we would  conclude   $u_3$ and $u_6$ have equal long-tail preference, since $i_1$ and $i_2$ are  equally popular. However, considering  the second column,  long-tail preference of users,  long-tail importance for each item,  which captures the long-tail preference of its users. Since  that  both users of $i_1$ have high long-tail preference while  the users of $i_2$ have lower preference,  we may conclude $i_1$ is a more important long-tail item compared to $i_2$. Therefore, $u_3$'s long-tail preference should be at least as large as $u_6$'s preference. Specifically, consider two  items $i_1$ and $i_2$, with the following rating data: $i_1=\{u_1:5, u_2:5, u_3:5 \}$, $i_2=\{u_4:5, u_5:5, u_6:5\}$.  

%Table~\ref{tab:example} shows  simplified rating data. We want an estimate of the long-tail preference of $u_3$ and $u_6$, who have each  rated a single movie.  Relying only  on movie popularity information, we would  conclude   $u_3$ and $u_6$ have similar long-tail preference, since $m_1$ and $m_2$ are  equally popular. However, considering the long-tail preferences of other users of those movies, we may reason differently: since $u_1$ and $u_2$ have high long-tail preference, and $u_4$ and $u_5$ have low long-tail preference, $m_1$ is a more prominent long-tail item compared to $m_2$. Therefore, it is likely that $u_3$ has higher long-tail preference compared to $u_6$.considering the long-tail preferences of other users of those movies, we may reason differently.  For example, 
\label{ex:running}
\end{example}



%------------------------------

\iffalse
\begin{example}
Table~\ref{tab:example} shows rating data for a simplified system. %Note the user-item interaction matrix is sparse.
For this example, we define popular movies as those that have received  three or more ratings; $\{m_1, m_2, m_4\}$ are popular and  $\{m_3, m_5, m_6\}$ are niche movies. We observe $u_1$ and $u_3$  have rated relatively popular movies (risk-averse) while $u_2$ and $u_4$ have rated niche movies (risk-loving). 
\label{ex:running}
\end{example}

\begin{table}[htb]
\centering
\scriptsize
\begin{tabular}{ccccccc} 
\toprule
			&$m_1$ &$m_2$   &$m_3$    &$m_4$   &$m_5$ &$m_6$  \\ \hline 
$u_1 $ &5  &4  & - &-  &-  &-   \\
$u_2$  &-  &-  &-  &-  &5  &5   \\
$u_3$  &-  &4  &-  &5  &-  &-   \\
$u_4$  &-  &-  &3  &-  &-  &4   \\ 
$u_5$  &5  &-  &-  &3  &-  &-   \\ 
$u_6$  &4  &2  &-  &4  &-  &-   \\ 
\bottomrule
%$ f_i$    &3  &3  &1  &3  &1  &2  \\  \hline
\end{tabular}
\caption{User-Movie rating data} \label{tab:example}
\end{table}

It is essential to consider consumer characteristics in designing recommender systems so that they promote long-tail items to the right group of users and spread demand evenly between hit and niche items.  

\fi





%------------------------------
\iffalse
\begin{table}[htb]
\centering
\scriptsize
\begin{tabular}{ccccccc} 
\toprule
			&$m_1$ &$m_2$   &$m_3$    &$m_4$   &$m_5$ &$m_6$  \\ \hline 
$u_1 $ &\textbf{5}  & \textbf{4}  &\textcolor{gray}{ 1.2} &-  &-  &-   \\
$u_2$  &-  &-  &-  &-  & \textbf{5}  &\textbf{5}   \\
$u_3$  &-  &\textbf{4}  &-  &\textbf{5}  &-  &-   \\
$u_4$  &-  &-  &\textbf{3}  &-  &-  &\textbf{4}   \\ 
$u_5$  &\textbf{5}  &-  &-  &\textbf{3}  &-  &-   \\ 
$u_6$  &\textbf{4}  &\textbf{2}  &-  &\textbf{4}  &-  &-   \\ 
\bottomrule
%$ f_i$    &3  &3  &1  &3  &1  &2  \\  \hline
\end{tabular}
\caption{User-Movie rating data} \label{tab:example}
\end{table}
% $\mathcal{P}^1= \{ \mathcal{P}_1^1 \{i_1,i_2,i_3\}, \mathcal{P}_2^1:\{i_2,i_3,i_5\}  \}$
 %$\mathcal{P}^2= \{ \mathcal{P}_1^2: \{i_1,i_2,i_3\}, \mathcal{P}_2^2:\{i_2,i_5,i_6\}  \}$
 %$\mathcal{P}^3= \{ \mathcal{P}_1^3: \{i_7,i_8,i_9\}, \mathcal{P}_2^3:\{i_{10},i_{11},i_{12}\}  \}$
\begin{table}[htb]
\centering
\tiny
\begin{tabular}{ccc} 
\toprule
		&$u_1$&$u_2$  \\ \hline 
$\mathcal{P}^1 $ & $\{i_1,i_2,i_3\}$ & $\{i_2,i_3,i_5\} $ \\
$\mathcal{P}^2$ & $\{i_1,i_2,i_3\}$ & $\{i_2,i_5,i_6\} $ \\
$\mathcal{P}^3$ & $\{i_7,i_8,i_9\}$ & $\{i_{10},i_{11},i_{12} \}$ \\
\bottomrule
%$ f_i$    &3  &3  &1  &3  &1  &2  \\  \hline
\end{tabular}
\caption{Top-$\size$ allocations to users.} \label{tab:paretoExamples}
\end{table}
\fi


\iffalse
When considering long-tail items, it is important to consider consumers' willingness  to explore niche or unpopular items and their propensity towards similar items. In particular, they can be characterized by their  {\it risk degree\/} and {\it focusing degree\/}, respectively.  We compute these estimates  based on historical rating information. The following example further describes these notions in the context of movie rating data. 

\begin{example}  
Table~\ref{tab:example} shows rating data for a simplified system with $6$ users, $6$ movies, and $3$ genres. $m_i^{j}$ implies that movie $m_i$ belongs to genre $j$. Note the user-item interaction matrix is sparse. 
  For this setting, we define popular movies as those that have received  three or more ratings; $\{m_1, m_2, m_4\}$ are popular and  $\{m_3, m_5, m_6\}$ are niche movies. We now profile the users according to their risk and focusing degree. E.g., $u_1$ has rated relatively popular movies belonging to the same genre (risk-averse, high focusing degree); $u_2$ has rated niches movies in the same genre (risk-loving, high focusing degree); $u_3$ has rated popular movies in two different genres (risk-averse, low focusing degree), and $u_4$ has rated niches movies in two different genres (risk-loving, low focusing degree). 
\label{ex:running}
\end{example}
\begin{table}[htb]
\centering
\tiny
\begin{tabular}{ccccccc} 
\toprule
			&$m_1^{1}$ &$m_2^{1}$   &$m_3^{2}$    &$m_4^{3}$   &$m_5^{3}$ &$m_6^{3}$  \\ \hline 
$u_1 $ &5  &4  &-  &-  &-  &-   \\
$u_2$  &-  &-  &-  &-  &5  &5   \\
$u_3$  &-  &4  &-  &5  &-  &-   \\
$u_4$  &-  &-  &3  &-  &-  &4   \\ 
$u_5$  &5  &-  &-  &3  &-  &-   \\ 
$u_6$  &4  &2  &-  &4  &-  &-   \\ 
\bottomrule
%$ f_i$    &3  &3  &1  &3  &1  &2  \\  \hline
\end{tabular}
\caption{User-Movie rating data} \label{tab:example}
\end{table}
It is essential to consider these consumer characteristics in designing recommender systems so that they promote long-tail items to the right group of users and spread demand evenly between the hit and niche items.  
\fi
\iffalse
\begin{center}
\begin{figure*}[tp]
%\scalebox{0.5}{%
\resizebox{1\textwidth}{!}{%
%\small%\addtolength{\tabcolsep}{5pt}% below sums to 8
\begin{tabularx}{1.5\textwidth}{>{\hsize=2.5\hsize}X>{\hsize=2.5\hsize}X>{\hsize=0.5\hsize}X>{\hsize=0.5\hsize}X>{\hsize=0.5\hsize}X>{\hsize=0.5\hsize}X>{\hsize=0.5\hsize}X>{\hsize=0.5\hsize}X}
    \multirow{12}{*}{\includegraphics[scale=0.3]{codeForExample/popularity-movie.png}} & \multirow{12}{*}{\includegraphics[scale=0.3]{codeForExample/scatterplot.png}} & & & & & & \\
%   & &               &       &       &       &       &       \\
    & &\multicolumn{1}{l|}{}               &$m_1^{g1}$   	&$m_2^{g1}$    	&$m_3^{g2}$    &$m_4^{g2}$      &$m_5^{g3}$    \\ \cline{3-8}%\hline
    & &\multicolumn{1}{l|}{u1}          &5  &5  &-  &-   &-  \\
    & &\multicolumn{1}{l|}{u2}    		&-  &-  &4  &4  &5  \\
    & &\multicolumn{1}{l|}{u3}   			&1  &2  &1  &-  &-   \\
    & &\multicolumn{1}{l|}{u4}     		&1  &-  &-  &-  &-  \\
    & &               &       &       &       &       &       \\
    & &               &       &       &       &       &       \\
    & &               &       &       &       &       &       \\
    & &               &       &       &       &       &	\\
    \\
\end{tabularx}}
\caption{User-Movie interaction data a) Popularity-Movie histogram b)Movie genres/clusters c) User-Movie rating data} \label{fig:example}
\end{figure*}
\end{center}
\fi



%We propose a novel approach that allows us to  promote long-tail items in a targeted manner, thereby improving the novelty of top-$\size$ sets, the overall item-space coverage across recommendations, while maintaining reasonable levels of accuracy.

%Next, we integrate these learned preferences  in a generic  top-$\size$ recommendation framework to provide customized balance between accuracy and coverage.

%sequentially make recommendations, while adjusting its parameters with regard to the set of top-$\size$ recommendations made so far. However, since  sequential parameter updates  cause  scalability issues, we propose a sampling based algorithm. This variant of our framework, called {\it Ordered Sampling-based Locally Greedy (OSLG)\/},  allows us to  correct for the popularity bias in recommendations with regard to individual user long-tail preferences. 

%ICDE submission
%Our framework differs with  prior work in the following aspects:  unlike~\cite{adomavicius2011maximizing,adomavicius2012improving,zhang2013personalize,ho2014likes},  the long-tail preference personalization in our framework is learned rather than optimized using cross-validation or parameter tuning. In other words, our personalization method is independent of the underlying base  recommendation models.  Moreover, our framework is  generic. This enables us to  plug-in several base recommenders, and evaluate their  effectiveness without requiring  extensive tuning for the accuracy and coverage trade-off. 


%\vspace{-2.8pt}
\begin{itemize}

\item  We examine various measures for estimating user long-tail novelty preference in Section~\ref{sec:lt-pref} and formulate an optimization problem  to directly learn users' preferences for long-tail  items from interaction data in Section~\ref{sec:learning-lt-pref}. %In addition, we introduce several heuristics for measuring the user preference for less common items from historical rating data.% 

\item  We integrate the user preference estimates into GANC %, a generic re-ranking framework that provides customized balance between accuracy, novelty, and coverage 
(Section~\ref{sec:RiskbasedReranking}), and  introduce {\it Ordered Sampling-based Locally Greedy (OSLG)\/}, a scalable algorithm that relies  on user long-tail preferences to correct the popularity bias (Section~\ref{sec:optimizationAlgorithm}).
%We introduce OSLG, a scalable algorithm that relies  on user long-tail preferences to  maximize item space coverage \textcolor{red}{while maintaining acceptable levels of accuracy} (Section~\ref{sec:optimizationAlgorithm}).

\item   We conduct an extensive empirical study and evaluate performance from  accuracy, novelty, and coverage perspectives (Section~\ref{sec:Experiments}).  We use five  datasets with varying density and difficulty levels. %:  Netflix, MovieTweetings, and MovieLens (100K, 1M, 10M). 
  In contrast to most related work,  our evaluation considers realistic settings that include a large number of infrequent  items and users. %This enables us to study the impact of  data density on the performance trade-offs of several  state of the art top-$\size$ recommendation algorithms. %   %,  and use the all-items ranking protocol~\cite{steck2013evaluation,vargas2014improving}, where performance is measured using all items with train data. to evaluate the performance of several  state of the art top-$\size$ recommendation algorithms 
 
\item Our empirical results confirm that the performance of re-ranking models is impacted by the underlying   base recommender and the dataset density. Our generic approach enables us to easily incorporate a suitable base recommender to devise an effective solution for both dense and sparse settings. In dense settings, we use the same base recommender as existing re-ranking approaches, and we outperform them in accuracy and coverage metrics. For sparse settings, we plug-in a more suitable base recommender, and devise an effective solution that is competitive with existing top-$\size$ recommendation methods in accuracy and novelty. 

%Directly estimating the long-tail novelty preferences allows us to customize re-ranking per user, and  devise a generic framework.   
 
\end{itemize}

Section~\ref{sec:related-work} describes related work. Section~\ref{sec:conclusion} concludes.

% Robot Model and Simulation
%!TEX root = ../Main.tex

\section{Bipedal Robot Model and Simulation}
\label{sec:robotModel}

\subsection{Robot Model}
\label{subsec:robotModel}

In this section, we present the robot model that will be used throughout the paper. A 10 DoF bipedal robot consists of 5 joint actuation each leg (see Fig. \ref{fig:robotModel}). Commonly, a 10 DoF bipedal robot has abduction (ab) and hip joints that allow 3-D rotation and ankle joints that allow double-leg-support standing (e.g., \cite{levineblackbird,gong2019feedback}). 

The design of our bipedal consists of the robot body, ab link, hip link, thigh link, calf link, and foot link (see Fig. \ref{fig:legConfig} for the definition of the leg configuration and Table \ref{tab:PRP} for physical parameters). The robot body is adopted from the Unitree A1 robot, in a vertical configuration. The joint actuator modeled in this robot design is the Unitree A1 modular actuator, which is a lightweight and powerful torque-controlled motor that is suitable for mini legged robots (see Table \ref{tab:motor} for the parameters of the actuator). 
\begin{figure}[!h]
	\hspace{0.2cm}
     \center
     \begin{subfigure}[b]{0.227\textwidth}
         \centering
         \includegraphics[width=\textwidth]{Figures/Model_in_Blender.png}
         \caption{ }
         \label{fig:robotModel}
     \end{subfigure}
     %\hfill
     \begin{subfigure}[b]{0.21\textwidth}
         \centering
         \includegraphics[width=\textwidth]{Figures/Leg_config.png}
         \caption{ }
         \label{fig:legConfig}
     \end{subfigure}
        \caption{{\bfseries Bipedal Robot Configuration}  a) Robot CAD Model  b) The link and joint configuration of the bipedal robot left leg.}
        \label{fig:robotConfig}
\end{figure}

\begin{table}[h]
	\vspace{0.2cm}
	\centering
	\caption{Robot Physical Parameters}
	\label{tab:PRP}
	\begin{tabular}{cccc}
		\hline
		Parameter & Symbol & Value & Units\\
		\hline
		Mass & $m$    & 11.84 & $\unit{kg}$  \\[.5ex]
		Body Inertia  & $I_{xx}$  & 0.0443 & $\unit{kg}\cdot \unit{m}^2$ \\[.5ex]
		& $I_{yy}$ & 0.0535  & $\unit{kg}\cdot \unit{m}^2$ \\[.5ex]
		& $I_{zz}$ & 0.0214  & $\unit{kg}\cdot \unit{m} ^2$ \\[.5ex]
		Body Length & $l_{b}$ & 0.114 & $\unit{m}$ \\[.5ex]
		Body Width & $w_{b}$ & 0.194 & $\unit{m}$ \\[.5ex]
		Body Height & $h_{b}$ & 0.247 & $\unit{m}$ \\[.5ex]
		Thigh and Calf Lengths & $l_{1}, l_{2}$ & 0.2 & $\unit{m}$ \\[.5ex]
		Foot Length & $l_{toe}$ & 0.09 & $\unit{m}$ \\[.5ex]
		& $l_{heel}$ & 0.05 & $\unit{m}$ \\[.5ex]
		\hline 
		\label{tab:robot}
	\end{tabular}
\end{table}	

	\begin{table}[h]
%		\hspace{0.2cm}
		\centering
		\caption{Joint Actuator Parameters}
		\begin{tabular}{cccc}
			\hline
			Parameter & Value & Units\\
			\hline
			Max Torque   &  33.5 & $\unit{Nm}$  \\[.5ex]
			Max Joint Speed    & 21  & $\unit{Rad}/\unit{s}$  \\[.5ex]
			\hline 
			\label{tab:motor}
			\end{tabular}
			\end{table}
\subsection{Simulation}
\label{subsec:simulation}

The bipedal robot simulation is built primarily in MATLAB Simulink using Spatial v2 package \cite{featherstone2014rigid}. Fig. \ref{fig:controlArchi} shows the diagram for our control architecture and also the representation of our simulation software.
% is an implementation of spatial vector arithmetic and dynamics algorithms that is available in MATLAB scripts. The software employs Roy Featherstone’s book \emph{Rigid Body Dynamics Algorithms} and provides a series of accessible MATLAB functions for robotic dynamics simulations \cite{featherstone2014rigid}. 
% The scope of the simulation construction is to build easy-to-use and reliable MATLAB and Simulink models that can be used as a test bench for many forthcoming controller designs and optimization implementations (see Fig. \ref{fig:controlArchi}: simulation block diagram).

The simulation software requires the user to input desired states at the beginning of the simulation. The desired states form a column vector that consist desired body center of mass (CoM) position $\bm p_d$, desired body CoM velocity $\bm {\dot p}_d$, desired rotation matrix $\bm R_d$ (resized to 9×1 vector), and desired angular velocity $\bm {\omega}_d$ of robot body. 

\begin{figure}[h]
	\hspace{0.2cm}
		\center
		\includegraphics[width=1 \columnwidth]{Figures/Control_archi.png}
		\caption{{\bfseries Control Diagram.} The simplified block diagram and control architecture of our proposed approach.}
		\label{fig:controlArchi}
	\end{figure}

% The simulation model shown in Fig. \ref{fig:controlArchi} provides a platform for controller implementation with the advantages of fast simulation time, simplicity in modification and debugging, and reliability.
% Main theory 
\section{Dynamics and Control}
\label{sec:dynamicsAndControl}
\subsection{Simplified Dynamics}
\label{subsec:simplified_dynamics}
In this Section, we investigate different dynamic models that can be used in our MPC control framework. 
While the whole-body dynamics of the bipedal robot are highly non-linear, we are interested in using simplified rigid-body dynamics to guarantee that our MPC controller can be solved effectively in real-time. 
In addition, in order to enable the capability of absorbing frequent and hard impacts from dynamic locomotion, the design of bipedal robots also requires lightweight limbs and connections. This has allowed the weight and rotation inertia of each link part to be very small compared to the body weight and rotation inertia. Hence, the effect of leg links in the robot dynamics may be neglected, forming simplified rigid-body dynamics \cite{nguyen2019optimized,bledt2018cheetah}. This is also a common assumption in many legged robots’ controller designs \cite{focchi2017high,stephens2010push}.

There are three simplified dynamics models that we have considered and tested, shown in Fig. \ref{fig:simplifiedDynamics}. The main difference between these three options is the number of contact points on each foot, contact location, and contact force and moment formation at each contact point. Model 1 resembles the simplified dynamics choice for quadruped robots mentioned in \cite{bledt2018cheetah}, with 4 contact points exerting 3-D contact forces. However, with this dynamics model applied to our 10-DoF bipedal robot, under rotation motions testings in simulation, the robot is unable to perform pitch motion properly. Model 2 is improved and further simplifies the contact points. However, with this simplified dynamics model, in simulation validation process, the robot is unable to perform roll motion correctly. Hence, we proposed model 3 that excludes external moment around $x$-axis, and contains only 3-D contact forces and 2-D moments around $y$ and $z$-axis. This model has allowed the robot to perform all 3-D rotations effectively. Hence, model 3 is chosen to be the final simplified dynamics design in this framework. More details about the validation of this decision process is shown in Section \ref{sec:simulationResults}.
The detailed derivation of the model 3 dynamics is presented as follows.

The bipedal model in this paper has two legs that both consists of 5 DoF. Commonly, the external contact forces applied to the robot are only limited to 3-D forces in many legged-robot dynamics (e.g., \cite{levineblackbird,nguyen2019optimized}). However, thanks to the additional hip and ankle joint actuation, the external moments can also be included in the robot dynamics, forming a linear relationship between robot body’s acceleration $ {\bm {\ddot p}_c}$, rate of change in angular momentum $\bm {\dot H}$ about CoM \cite{stephens2010push}, and contact force and moments %$ \bm F=[ \bm F_1,\:\bm F_2]^T$ , where $\bm F_i= [\bm F_{ix},\:\bm F_{iy},\:\bm F_{iz},\:\bm M_{ix},\:\bm M_{iy},\:\bm M_{iz} ]^T$, $i=1,2$ , as follows:
\update{$ \bm u=[\bm F_1,\:\bm F_2,\:\bm M_1,\:\bm M_2]^T,  \bm F_i = [\bm F_{ix},\:\bm F_{iy},\:\bm F_{iz}]^T, \bm M_i = [\bm M_{iy},\:\bm M_{iz}]^T, i = 1, 2, $ shown as follows: 
\setlength{\belowdisplayskip}{5pt} \setlength{\belowdisplayshortskip}{5pt}
\setlength{\abovedisplayskip}{5pt} \setlength{\abovedisplayshortskip}{5pt}
\begin{align}
\label{eq:simplifiedDynamics}
\left[\begin{array}{ccc} \bm {{D}_1}   \\ 
 \hspace{0cm} \bm {{D}_2} \end{array}  \right] \bm u= \left[\begin{array}{c} m (\ddot{\pos}_{c} +\bm{g}) \\ \bm {\dot H} \end{array} \right],
\end{align}
where
\begin{align}
\label{eq:D1}
 \bm {D}_1  = \left[\begin{array}{cccc} \mathbf {I}_{3\times3}  & \mathbf {I}_{3\times3} & 
  \mathbf {0}_{3\times2} &  \mathbf {0}_{3\times2} \end{array} \right],
\end{align}
\begin{align}
\label{eq:D2}
 \bm {D}_2  = \left[\begin{array}{cccc} \bm {(p_1-p_c)}\times  & \bm {(p_2-p_c)}\times & 
  \mathbf L &  \mathbf L \end{array} \right],
\end{align}
\begin{align}
\label{eq:L}
 \mathbf L  = \left[\begin{array}{ccc}  \mathbf 0 & \mathbf 0 \\  \mathbf 1 &\mathbf 0 \\  \mathbf 0 &\mathbf 1
 \end{array} \right].
\end{align} }

The term ${(\bm p_i- \bm p_c)}$ denotes the distance vector from the robot body CoM location to the foot $i$ position in the world coordinate; and $\bm {(p_i-p_c)}\times $ represents the skew-symmetric matrix representing the cross product of ${ {(\bm p_i-\bm p_c)} \times \bm F_i }$. Here, $\bm {\dot H}$ can be approximated as $\bm{ \dot H=I_G \dot {{\omega}}}$  (as discussed in \cite{nguyen2019optimized}), where $\bm {I_G}$ stands for the centroid rotation inertia of robot body in the world frame and  $\bm {\dot {\omega}}$ represents the angular velocity of robot body in the world frame \cite{bledt2018cheetah,stephens2010push}.
	


\begin{figure}[!h]
\vspace{0.2cm}
     \centering
     \begin{subfigure}[b]{0.15\textwidth}
         \centering
         \includegraphics[width=\textwidth]{Figures/simplifiedDynamics1.png}
         \caption{Model 1}
         \label{fig:model1}
     \end{subfigure}
     \hfill
     \begin{subfigure}[b]{0.13\textwidth}
         \centering
         \includegraphics[width=\textwidth]{Figures/simplifiedDynamics2.png}
         \caption{Model 2}
         \label{fig:model2}
     \end{subfigure}
     \hfill
     \begin{subfigure}[b]{0.145\textwidth}
         \centering
         \includegraphics[width=\textwidth]{Figures/simplifiedDynamics3.png}
         \caption{Model 3}
         \label{fig:model3}
     \end{subfigure}
        \caption{{\bfseries Three Simplified Dynamics Models.} Yellow dots on the figures represent the contact point locations in simplified dynamics a) 2 contact points at the toe and heel of each robot foot, 3-D forces applied  b) 1 contact point at the middle of each foot, 3-D force and 3-D moments applied  c) 1 contact point at the middle of each foot, 3-D force and 2-D moments applied. Model 3 is the final choice to be used in our proposed approach.}
        \label{fig:simplifiedDynamics}
\end{figure}
%\todo{We should have a figure to illustrate the simplified rigid body model including 1 rigid body, 2 contact point on the ground, force vectors, My, Mz,...}
%\todo{We should also consider make comparisons between different models: the current one (1 contact point each leg: 3 forces + 2 moments for each point), 2 contact point each leg (3 forces + 3 moments for each points), 2 contact point each leg (3 forces + 0 moments for each points)... like you used to explore. This will make the paper stronger.}
\update{Equation (\ref{eq:simplifiedDynamics}) to (\ref{eq:L}) describes the simplified dynamics model 3. This model only used 5 force and moment inputs $\bm U$ which are directly mapped to 5 joint torques in each leg.}

We use rotation matrix $\bm R$ as a state variable to represent the orientation of the robot body, which can be also directly converted to Euler Angles. 
%Reference \cite{di2018dynamic} presents the 
We linearize the rotation matrix by approximating the angular velocity in terms of Euler angle $ {\Theta = [\phi,\:\theta,\:\psi]}^T$, where $\phi$ is roll angle,  $ \theta$ is pitch angle, and $ \psi$ is yaw angle. With the assumption of small roll and pitch angles  \cite{di2018dynamic}, the relation of the rate of change of $\Theta$ and angular velocity $\bm \omega$ in the world coordinate can be approximated as: 
%\begin{align}
%\label{eq:omega}
%%\bm \omega  = \left[\begin{array}{ccc} {\cos{\theta} \cos{\psi}}& -{\sin{\psi}} & 0 \\{\cos{\theta} \sin{\psi}} & {\cos{\psi}} & 0 \\ -{\sin{\theta}} & 0 & 1\end{array} 
%\right ] 
%\begin{bmatrix} {\dot{\phi}} \\ {\dot{\theta}} \\ {\dot{\psi}}\end{bmatrix}
%\end{align} 

%\begin{align}
%\label{eq:omega}
%\bm \omega  = \left[\begin{array}{ccc} %\dot{\phi}\sin{\theta}\sin{\psi}+\dot{\theta}\cos{\psi} \\ 
%\dot{\phi}\sin{\theta}\cos{\psi}-\dot{\theta}\sin{\psi} \\ 
%\dot{\phi}\cos{\theta}+\dot{\psi}
%\end{array} 
%\right ] 
%\end{align} 

%With small pitch and roll angles, equation (\ref{eq:omega}) can be approximated and rewritten as
\begin{align}
\label{eq:omega2}
\begin{bmatrix} {\dot{\phi}} \\ {\dot{\theta}} \\ {\dot{\psi}}\end{bmatrix} \approx \left[\begin{array}{ccc} { \cos{\psi}}& {\sin{\psi}} & 0 \\{ -\sin{\psi}} & {\cos{\psi}} & 0 \\ 0 & 0 & 1\end{array} 
\right ] 
\begin{array}{cc}
     {\bm \omega}
\end{array}.
\end{align} 
% \todo{You should double-check eq 5,6 because they may not be correct in [9].}

Hence the kinematic constraint of the Euler angles is obtained as follow:
\begin{align}
\label{eq:omega3}
\begin{bmatrix} {\dot{\phi}} \\ {\dot{\theta}} \\ {\dot{\psi}}\end{bmatrix} \approx \bm R_z(\psi) \bm \omega.
\end{align} 

Combing the approximated orientation dynamics and the translation dynamics, the simplified dynamics of the robot can be written as: 
\update{
\begin{align}
\label{eq:stateSpace}
\frac{d}{dt}
\begin{bmatrix} 
{\bm \Theta} \\
{\bm p}_c \\ 
{\bm \omega} \\ 
\dot {{\bm p}}_c
\end{bmatrix} = 
\bm A
\begin{bmatrix} 
{\bm \Theta} \\
{\bm p}_c \\ 
{\bm \omega} \\ 
\dot {{\bm p}}_c
\end{bmatrix} 
+ \bm B \bm u +
\begin{bmatrix} 
\mathbf 0 \\ \mathbf 0 \\\mathbf  0 \\\mathbf  g
\end{bmatrix},
\end{align} 
where
\begin{align}
\label{eq:A}
\bm A = 
\begin{bmatrix} 
\mathbf {0}_{3\times3} & \mathbf {0}_{3\times3} & \bm R_z(\psi) & \mathbf {0}_{3\times3}  \\
\mathbf {0}_{3\times3} & \mathbf {0}_{3\times3} & \mathbf {0}_{3\times3} & \mathbf  {I}_{3\times3}  \\ 
\mathbf {0}_{3\times3} & \mathbf {0}_{3\times3} & \mathbf {0}_{3\times3} & \mathbf {0}_{3\times3} \\ 
\mathbf {0}_{3\times3} & \mathbf {0}_{3\times3} & \mathbf {0}_{3\times3} & \mathbf {0}_{3\times3}, 
\end{bmatrix},
\end{align} 
\begin{align}
\setlength\arraycolsep{2.5pt}
\label{eq:B}
\bm B = 
\begin{bmatrix} 
\mathbf {0}_{3\times3} & \mathbf {0}_{3\times3} & \mathbf {0}_{3\times2} & \mathbf {0}_{3\times2}  \\
\mathbf {0}_{3\times3} & \mathbf {0}_{3\times3} & \mathbf {0}_{3\times2} & \mathbf {0}_{3\times2}  \\ 
\bm {{\Tilde{I}^{-1}_G} (p_1-p_c)}\times & \bm {{\Tilde{I}^{-1}_G} (p_2-p_c)}\times & \bm{{\Tilde{I}^{-1}_G} \mathbf L} & \bm{{\Tilde{I}^{-1}_G} \mathbf L} \\ 
\mathbf { {I}_{3\times3}}/m & \mathbf {{I}_{3\times3}}/m & \mathbf {0}_{3\times2} & \mathbf {0}_{3\times2} 
\end{bmatrix}
\end{align} }

\update{Here, $\bm{\Tilde{I}^{-1}_G}$ is approximated by rotation inertia of the robot body in its body frame $\bm I_b$ and $\bm R_z(\psi)$ from \eqref{eq:omega3}:
\begin{equation}
\label{eq:rotationI}
\bm{\Tilde{I}_G} = \bm R_z(\psi) \bm I_b {\bm R_z(\psi)}^T.
\end{equation}
}
By assigning gravity as additional state variable, now state ${{\bm x}} = [{{\bm \Theta}},\: {\bm p}_c,\:{{\bm \omega}},\:\dot { {\bm p}}_c,\:\bm g]^T$ will allow the dynamics in (\ref{eq:stateSpace}) to be rewritten into a linear state-space form with continuous time matrices $\bm {\hat{A_c}}$ and $\bm {\hat{B_c}}$: 
\begin{equation}
\label{eq:linearSS}
\dot{{\bm { x}}}(t) = \bm {\hat{A_c}}(\psi) {{\bm {x}}}(t) + \bm {\hat{B_c}}([p_1-p_c],[p_2-p_c],\psi) \bm u(t).
\end{equation}

The linearized dynamics in (\ref{eq:linearSS}) is now suitable for the convex MPC formulation presented in \cite{di2018dynamic}.

\subsection{MPC Formulation}
\label{subsec:MPC}
Having discussed the dynamics model, we now present details about the formulation of our MPC controller.

The linearized dynamics in (\ref{eq:linearSS}) can be represented in a discrete-time form at each time step $i$
\begin{align}
\label{eq:discreteDynamics}
{\bm {x}}[i+1] = \bm {\hat{A}}[i] \bm x[i] + \bm {\hat{B}}[i]\bm u[i],
\end{align}
where discrete time matrix $\bm {\hat{A}}$ is a constant matrix computed from $\bm {\hat{A_c}}(\psi)$ using a average yaw value during entire reference trajectory; and $\bm {\hat{B}}$ matrix is computed from $ \bm{\hat{B_c}}([p_1-p_c],[p_2-p_c],\psi)$, using the desired values of average yaw and foot location. The only exception is that at the first time step,  $\bm {\hat{B}}[1]$ is computed from current states of the robot instead of reference trajectory.

An MPC problem with a finite horizon length $k$ can be written in the following standard form:
\begin{align}
\label{eq:MPCform}
%\nonumber
\underset{\bm{x,u}}{\operatorname{min}}   \:\:  & \sum_{i = 0}^{k-1}(\bm x_{i+1}-  {\bm x_{i+1}}_{ref})^T\bm Q_i(\bm x_{i+1}- {\bm x_{i+1}}_{ref}) + \| \bm{u}_i \|\bm{R}_i
\end{align}
\begin{align}
\label{eq:dynamicCons}
\:\:\mbox{s.t. }& \quad  {\bm {x}}[i+1] = \bm {\hat{A}}[i]\bm x[i] + \bm {\hat{B}}[i]\bm u[i], \quad i = 0 \dots k-1
\end{align}
\begin{align}
\label{eq:MPCineqCons}
\quad  \bm {c^-}_i \leq \bm C_i\bm u_i \leq \bm {c^+}_i, \quad i = 0 \dots k-1
\end{align}
\begin{align}
\label{eq:MPCeqCons}
& \quad \bm D_i \bm u_i = 0 , \quad i = 0 \dots k-1
\end{align}

%\todo{We should use the format $x^TQx$ instead of $||x||Q$ in the cost function.}

In (\ref{eq:MPCform}), $\bm x_i$ and $\bm u_i$ are system states and control inputs at time step $i$. \update{Note that the MPC prediction is computed based on the measured states of current step (i.e. $i = 0$).} $\bm Q_i$ and $\bm R_i$ are matrices defining the weights of \update{each state and control input variable.} $\bm {\hat{A}}$ and $\bm {\hat{B}}$ in (\ref{eq:dynamicCons}) are the discrete-time system dynamic constraints from (\ref{eq:discreteDynamics}). $\bm {c^-}_i$,$\bm {c^+}_i$, and $\bm C_i$ in (\ref{eq:MPCineqCons}) represents the inequality constraints of the MPC problem. $\bm D_i$ in (\ref{eq:MPCeqCons}) represents the equality constraints. In this problem, the equality constraint governs the optimal control input from MPC controller is a zero vector for swing foot. 

The MPC controller solves the optimal ground contact force and moment with respect to dynamic constraints \eqref{eq:dynamicCons} and the following inequality constraints:
\begin{align}
\label{eq:frictionCons}
\nonumber 
-\mu \bm {F}_{iz} \leq \bm F_{ix} \leq \mu \bm {F}_{iz} \\
-\mu \bm {F}_{iz} \leq \bm F_{iy} \leq \mu \bm {F}_{iz} \\
% \end{align}
% \begin{align}
\label{eq:forceCons}
0< \bm {F}_{min} \leq \bm  F_{iz} \leq \bm {F}_{max} \\
% \end{align}
% \begin{align}
\label{eq:torqueCons}
\bm |{\tau}_{i}| \leq \bm {\tau}_{max}. 
\end{align}
Here, \eqref{eq:frictionCons} governs the contact forces in $x$ and $y$ direction are within the friction pyramid, with $\mu$ being the friction coefficient. The contact forces in $z$-direction should also fall within the upper and lower bounds of force (\ref{eq:forceCons}), where the lower bound is positive to maintain contact with the ground. It is also important to restrict the joint torques to be within the saturation of the physical motor (\ref{eq:torqueCons}). 


\subsection{QP Formulation}
% It is stated in Section \ref{sec:robotModel} of this paper that the scope of developing the robot simulation in MATLAB and Simulink is to have a fast and reliable simulation software to test controller designs. MPC problem can be heavy to solve so it is important to reduce the computation and problem size of MPC by formulating the problem into a quadratic program (QP).
With the linear dynamics in Section \ref{subsec:simplified_dynamics} and the MPC formulation in Section \ref{subsec:MPC}, our controller can be formulated as a quadratic program (QP) that can be solved effectively in real-time. 
% With given equality and inequality constraints, we can form a QP problem based on the dynamics from the condensed formulation in (\ref{eq:QPdynamics}). The Optimization Toolbox in MATLAB provides powerful and fast QP solver which is suitable for this problem. 

Firstly, the dynamic constraints \eqref{eq:linearSS} for the entire MPC prediction horizon can be written as:
\begin{align}
\label{eq:QPdynamics}
\bm X = \bm{A}_{qp} \bm x_0 + \bm{B}_{qp} \bm U,
\end{align}
where $\bm X$ is a column vector containing system states for the next $k$ horizons, $\bm x[i+1],\bm x[i+2] \dots {\bm x}[i+k]$ and $\bm U$ is a column vector containing optimal control inputs of current state $\bm u[i]$ and next $k-1$ horizons, $\bm u[i+1], \bm u[i+2] \dots {\bm u}[i+k-1]$ at time step $i$. 
The MPC controller now can be written as the following QP form:     
\begin{align}
\label{eq:QPform}
%\nonumber
\underset{\bm{U}}{\operatorname{min}}   \:\:  & \frac{1}{2}\bm U^T\bm h \bm U + \bm U^T\bm f \\
% \end{align}
% \begin{align}
\label{eq:QPineqCons}
\:\:\mbox{s.t. }& \quad  \bm C\bm U \leq \bm d \\
% \end{align}
% \begin{align}
\label{eq:QPeqCons}
& \bm A_{eq}\bm U = \bm b_{eq}
\end{align}
where $\bm C$ and $\bm d$ are inequality constraint matrices, $\bm A_{eq}$ and $\bm b_{eq}$ are equality constraint matrices, and 
\begin{align}
\label{eq:h}
\bm h = 2( {\bm B_{qp}}^T\mathbf M {\bm B_{qp}}+\mathbf K), \\
% \end{align}
% \begin{align}
\label{eq:f}
\bm f = 2 {\bm B_{qp}}^T\mathbf M ({\bm A_{qp}}\bm x_0-\bm y).
\end{align}
Diagonal matrices $\mathbf K$ and $\mathbf M$ are the weights for the rate of change of state variables and force/moment magnitude.

\update{The resulting controller input of each leg from QP problem $\bm u_i=[\bm F_i,\: \bm M_i]^T$ is mapped to its joint torques by
\begin{align}
\label{eq:forceTorquemap}
\bm {\tau}_i =  \bm J_i^T \bm u_{i},
\end{align} 
where $\bm J_i$ is the Jacobian matrix of $i$th leg 
\begin{align}
\label{eq:Jacobian}
\bm {J}_i^T = \begin{bmatrix} 
\bm {J}_v^T & \bm {J}_\omega^T \bm L
\end{bmatrix},
\end{align} 
}
with $\bm {J}_v $ and $\bm {J}_{\omega} $ being the linear velocity and angular velocity components of $\bm {J}_i $.

\subsection{Swing Leg Control}
As discussed earlier in this section, due to equality constraints, the robot leg that is under the swing phase does not exert ground contact forces and therefore is not under the control of force-and-moment-based MPC. In order to control the leg and foot position in each gait cycle, the desired foot trajectory is under control in the Cartesian space with a PD position controller. \update{The gait sequence is purely based on timing and the gait cycle length is currently set at $0.3 \unit{s}$.}
We obtain the current foot location using forward kinematics. Foot velocity is computed by:
\begin{align}
\label{eq:footVel}
{\dot {\bm p}}_{{foot}_i} =  \bm J_i^T \dot{\bm q_i},
\end{align}
where $\dot{\bm q_i}$ is the joint velocity state-feedback of each leg at time step $i$.

The desired foot location $\bm p_{{foot}_d}$ in the world frame is determined by the foot placement policy employed in \cite{di2018dynamic}:
\begin{align}
\label{eq:footPlacement}
\bm p_{{foot}_d} =  \bm p_{hip} + \dot{\bm p}_c \Delta t/2,
\end{align}
where $\bm p_{hip}$ is the hip joint location in the world frame and $\Delta t$ is the time that stance foot spends on the ground during one gait cycle. 

The swing leg force can be computed by treating the foot attached to a virtual spring-damper system \cite{chen2020virtual}. The foot weight is reasonable to be neglected since it is very small compared to the robot body \cite{nguyen2019optimized}. Following the PD control law, the foot force can be written as: 
\begin{align}
\label{eq:PDswing}
\bm F_{swing_i}=\bm K_P(\bm p_{{foot}_d}-\bm p_{{foot}_i})+\bm K_D(\dot{\bm p}_{{foot}_d}-\dot{\bm p}_{{foot}_i})
\end{align}
where $\bm K_P$ and $\bm K_D$ are PD control gains, or spring stiffness and damping coefficient of the virtual spring-damper system. %$\bm p_{{foot}_d}$ and $\dot{\bm p}_{{foot}_d}$ are desired foot placement and velocity, respectively.

Similar to (\ref{eq:forceTorquemap}), the joint torque can be computed by:
\begin{align}
\label{eq:forceTorqueMapSwing}
\bm {\tau}_{swing_i} =  \bm J_v^T \bm F_{swing_i}.
\end{align}

With Cartesian PD control, the swing leg can move and be controlled to follow desired foot placement trajectory. The gait generator decides either the robot leg is in the stance phase or swing phase in a fixed gait cycle and assigns the appropriate controller to the corresponding leg. Now the robot has both swing and stance leg control, it is ready to test the MPC in simulation.
% results
\section{Simulation Results}
\label{sec:simulationResults}

In this section, we present numerical validation of our proposed approach for different dynamic locomotion. 
% The simulation video for this paper is given in Fig. \ref{fig:roughTerrainSim}. 
The reader is encouraged to watch the supplemental video\footnote{\url{https://youtu.be/Z2s4iuYkuvg}} 
for the visualization of our results.
For our simulation, the bipedal robot model and ground contact model are set up in MATLAB with Spatial v2 software. The MPC sampling frequency is set to $0.03\:\unit{s}$ while the simulation is run at $1\: \unit{kHz}$. One gait cycle that contains 10 horizons is predicted at each time step in MPC, in which each gait cycle is fixed at $0.30\:\unit{s}$. \update{This prediction length has been also used in \cite{di2018dynamic}.}

\update{The weighting factors $\bm Q$ in \eqref{eq:MPCform} are tuned to balance the performance between different control actions. In our simulation, we use $\bm Q_x = \bm Q_y = 50$, $\bm Q_z = 100$, $\bm Q_{\phi} = \bm Q_{\theta} = 100$, and $\bm Q_{\psi} = 20$. The rest weighting factors in $\bm Q$ remains at 1. }

\begin{figure}%[!h]
\vspace{0.5cm}
     \centering
     \begin{subfigure}[b]{0.13\textwidth}
         \centering
         \includegraphics[width=\textwidth]{Figures/pitchModel1.png}
         \caption{Snapshot of Model 1 in Double-leg Stance}
         \label{fig:pitchModel1}
     \end{subfigure}
     %\hfill
     \quad \quad
     \begin{subfigure}[b]{0.13\textwidth}
         \centering
         \includegraphics[width=\textwidth]{Figures/pitchModel3.png}
         \caption{Snapshot of Model 3 in Double-leg Stance}
         \label{fig:pitchModel3}
     \end{subfigure}
     \hfill
     \begin{subfigure}[b]{0.5\textwidth}
         \centering
         \includegraphics[width=0.9\textwidth]{Figures/PitchComp2_final.pdf}
         \caption{Pitch Motion Comparison}
         \label{fig:pitchPlot}
     \end{subfigure}
        \caption{{\bfseries Comparison of Model 1 and Model 3 in Pitch Motion Simulation}  a) Snapshot at the end of simulation with model 1  b) Snapshot at the end of simulation with model 3   c) Pitch motion response comparison with a $10^{\circ}$ desired pitch input.}
        \label{fig:pitchComparison}
\end{figure}

	
\begin{figure}%[!h]
	\hspace{0.2cm}
     \center
     \begin{subfigure}[b]{0.13\textwidth}
         \centering
         \includegraphics[width=\textwidth]{Figures/rolModel2.png}
         \caption{Snapshot of Model 2 in Double-leg Stance}
         \label{fig:rollModel2}
     \end{subfigure}
     %\hfill
     \quad \quad
     \begin{subfigure}[b]{0.14\textwidth}
         \centering
         \includegraphics[width=\textwidth]{Figures/rollModel3.png}
         \caption{Snapshot of Model 3 in Double-leg Stance}
         \label{fig:rollModel3}
     \end{subfigure}
     %\hfill
     \begin{subfigure}[b]{0.5\textwidth}
         \centering
         \includegraphics[width=0.9\textwidth]{Figures/RollComp2_final.pdf}
         \caption{Roll Motion Comparison}
         \label{fig:rollPlot}
     \end{subfigure}
        \caption{{\bfseries Comparison of Model 2 and Model 3 in Roll Motion Simulation}  a) Snapshot at the end of simulation with model 2  b) Snapshot at the end of simulation with model 3   c) Roll motion response comparison with a $10^{\circ}$ desired roll input.}
        \label{fig:rollComparison}
\end{figure}

\subsection{Validation of Simplified Dynamics}
First, we present the simulation results of simple rotation motions during standing with both legs on the ground to validate the claim in Section \ref{sec:dynamicsAndControl} that for the simplified dynamics used for control design, model 3 is a superior choice over model 1 and model 2. 

As mentioned in Section \ref{sec:dynamicsAndControl}, the simplified dynamics model 1 is unable to perform pitch motion. It is shown in Fig. \ref{fig:pitchComparison}, a pitch motion comparison between using simplified dynamics model 1 and model 3. The latter one is what we ultimately chose to use in MPC formulation. It is observed that the simulation result with model 1 does not respond to desired pitch input, whereas model 3 can perform pitch motion. 

We then further simplified model 1 and added 3-D moment inputs to each contact point to form simplified dynamics model 2. However, in the roll motion test, the response with model 2 is incorrect to desired roll input and it also shows a deviation in yaw angle as shown in Fig. \ref{fig:rollComparison}. With model 3, the robot simulation succeed in the roll motion test. 
Therefore, we decide to use model 3 for our proposed approach. Following are simulation results for walking and hopping motion using MPC control for model 3.

\subsection{Velocity Tracking}

In this simulation, we test the MPC performance in forward walking motion(positive $x$-direction) with time-varying desired speed and the desired CoM height of $0.5\:\unit{m}$. 
%The joint torques during this simulation are presented in Fig. \ref{fig:velSimTorque}. Note that during the simulation, all joint torque data are within the maximum torque threshold of our motor choice. There are no extreme torque values found during this entire simulation. 
The velocity tracking plot is shown in Fig. \ref{fig:velTracking}, the actual response curve with MPC shows a good tracking performance. The velocity response has a maximum deviation of $0.076\: \unit{m/s} $ compared to the desired input. Besides walking forward, we also have successful simulation results and demonstrations in walking sideways and diagonally. This result validates the effectiveness of our proposed control framework in realizing 3D dynamic locomotion for bipedal robots. 
% The simulation results can be found in the video (URL is under Fig. \ref{fig:roughTerrainSim}) associated with this paper. 
%\begin{figure}[h]
%	\center
%	\includegraphics[width=1 \columnwidth]{Figures/MPC_tau.png}
%	\caption{{\bfseries Plots of Joint Torques with MPC.} Joint torques of stance leg under the control of MPC in time-varying velocity simulation. }
%	\label{fig:velSimTorque}
%\end{figure}
\begin{figure}[h]
	\hspace{0.2cm}
	\center
	\includegraphics[width=0.75 \columnwidth]{Figures/velTracking_resized.pdf}
	\caption{{\bfseries Velocity Tracking.} Comparison of desired velocity input and actual velocity response in x-direction. 
% 	The desired velocity curve keeps constant from $t=0\:\unit{s}$ to $t=0.6\:\unit{s}$ and from $t=1.5\:\unit{s}$ to $t=3\:\unit{s}$ at $\bm v_{x_d}=0\:\unit{m/s}$ and $\bm v_{x_d}=0.6\:\unit{m/s}$, respectively. From $t=0.6\:\unit{s}$ to $t=1.5\:\unit{s}$ and from $t=3\:\unit{s}$ to $t=3.9\:\unit{s}$, $\bm v_{x_d}$ increases linearly.
	}
	\label{fig:velTracking}
\end{figure}
	
\subsection{High-velocity Walking in Rough Terrain}
We also validated the controller performance in rough terrain locomotion at high speed. Specifically, the robot is commanded to walk through a $2.4$-meter-long rough terrain formed by stairs with various heights and lengths. The stair heights range from $0.020\:\unit{m}$ to $0.075\:\unit{m}$ with a maximum height difference of $0.055\:\unit{m}$ between two consecutive stairs. To validate the feasibility and potential of MPC locomotion through rough terrain, the robot is commanded to follow a high desired velocity $\bm v_{x_d}=1.6 \:\unit{m/s}$. A snapshot of this simulation is provided in Fig. \ref{fig:roughTerrainSim}. 
%\begin{figure}[h]
%	\center
%	\includegraphics[width=0.85 \columnwidth]{Figures/Rough_terrain_spatial.png}
%	\caption{{\bfseries Snapshot from Rough Terrain Simulation.} Terrain model with stairs at various heights, animated with Spatial v2.  }
%	\label{fig:roughTerrain}
%\end{figure}

Plots of CoM location, velocity, and body orientation are shown in Fig. \ref{fig:CoMposVel} and Fig. \ref{fig:CoMeulAng}. It can be observed that the CoM location and orientation during this simulation maintain small tracking errors. 
%The position curves of the joints show that the joint position responses are smooth under the control of MPC and PD Cartesian control. The MPC controller input is presented in Fig. \ref{fig:MPCforce}. 
%The force in the $y$-direction $\bm F_y$ and moment $\bm M_z$ has the largest variation between the two legs. Hence it yields a slight $y$-direction displacement at the end of the simulation. As can be seen in CoM $y$-direction location in Fig. \ref{fig:CoMposVel}, the final $y$-direction location of body CoM is $0.0087 \:\unit m$ at $t = 1.8\: \unit{s}$. 
The joint torques (shown in Fig. \ref{fig:MPCtauRT}) during this entire simulation are in reasonable ranges and satisfy the torque saturation shown in Table \ref{tab:motor}. 
% todo{You should cite the table II about the torque limit here and also mention about the satisfaction of joint speed limit in Fig 11.} 
% (this is not really accurate so I commented it out)It is expected that the ankle joints will exert higher magnitudes of torques. Shown in the corresponding plots, the magnitudes of torques are still under the maximum torque threshold in most occasions. 

%With above simulation results and observations. It is inferred that the framework with the new MPC model presented in this paper can be a feasible option for a 10 DoF bipedal robot in dynamic locomotion. Our future work include extending this control framework to more dynamic motions such as bipedal bounding and running \todo{We have new results so please check sth like this throughout the paper to make sure that it is consistent with the current result}. Eventually, this control framework is expected to be migrated to a physical bipedal robot platform that is under development currently.
\begin{figure}[h]
	\hspace{0.2cm}
	\center
	\includegraphics[width=0.95 \columnwidth]{Figures/RTCoM2_final.pdf}
	\caption{{\bfseries Plots of Body CoM Position and Velocity in Rough Terrain Simulation.}  }
	\label{fig:CoMposVel}
\end{figure}
\begin{figure}[h]
	\center
	\includegraphics[width=0.88 \columnwidth]{Figures/RTCoMRot4_final.pdf}
	\caption{{\bfseries Plots of Robot Orientation in Rough Terrain Simulation. }  }
	\label{fig:CoMeulAng}
\end{figure}
%\begin{figure}
%	\hspace{0.2cm}
%	\center
%	\includegraphics[width=1 \columnwidth]{Figures/RTJoint_final.pdf}
%	\caption{{\bfseries Plots of Joint Position and Velocity  in Rough Terrain Simulation. }  \todo{Remove this}}
%	\label{fig:jointPos}
%\end{figure}
\begin{figure}[!h]
	\center
	\includegraphics[width=0.92 \columnwidth]{Figures/MPCForce3_final.pdf}
	\caption{{\bfseries Plots of MPC Force and Moment in Rough Terrain Simulation.  }}
	\label{fig:MPCforce}
\end{figure}
\begin{figure}%[!h]
	\hspace{0.2cm}
	\center
	\includegraphics[width=0.92 \columnwidth]{Figures/MPCTorque3_final.pdf}
	\caption{{\bfseries Plots of Joint Torques in Rough Terrain Simulation.  } }
	\label{fig:MPCtauRT}
\end{figure}

\subsection{Bipedal Hopping}
On top of the rotation and walking simulations presented earlier in this section, we have also implemented other gaits such as hopping. The hopping gait consists of a double support phase and a flight phase during the last quarter of each gait. 
% (we do not have flight phase in the previous one with trotting, so we should not mention this)With current MPC formulation, we decrease the flight phase duration in each gait cycle to mitigate the effects on performance during flight phase. 
A hopping gait illustration is shown in Fig. \ref{fig:boundGait}. It can be observed that during hopping motion, the robot is in a clear flight phase. 
%To validate the feasibility of hopping motion with the current MPC formulation, we test hopping forward motion with velocity $\bm v_{x_d}=0.5\:\unit{m/s}$. The CoM z-direction position and velocity plots are shown in Fig. \ref{fig:bounding_CoMposvel}. It is observed that the hopping motion can be performed with the current MPC formulation, with the trade-off of a certain degree of error in CoM velocity and height tracking. 
\update{This result validated that our proposed approach can work effectively for different dynamic locomotion on bipedal robots. We plan to optimize the MPC formulation in future work to enable faster and more aggressive motions. }

\begin{figure}%[!h]
	\center
	\includegraphics[width=1 \columnwidth]{Figures/boundGait.png}
	\caption{{\bfseries Illustration of Bipedal Hopping in Simulation  }  }
	\label{fig:boundGait}
\end{figure}
%\begin{figure}[!h]
%	\hspace{0.2cm}
%	\center
%	\includegraphics[width=1 \columnwidth]{Figures/BoundingCoM.pdf}
%	\caption{{\bfseries Plots of Z-direction CoM Location and Velocity in Hopping Simulation }  }
%	\label{fig:bounding_CoMposvel}
%\end{figure}

% Conclusion
\section{Conclusion}
\label{sec:conclusion}
This paper presents a generic top-$\size$ recommendation framework for  trading-off accuracy, novelty, and coverage. To achieve this, we profile the users according to their preference for long-tail novelty. We examine various measures, and formulate an optimization problem to learn these user preferences from interaction data.  We then integrate the user preference estimates in our generic framework, GANC.  Extensive experiments on several datasets confirm that there are trade-offs between accuracy, coverage, and novelty. Almost all re-ranking models increase coverage and novelty at the cost of accuracy. However, existing re-ranking models typically rely on rating prediction models, and are hence more effective in dense settings. Using a generic approach, we can easily incorporate a suitable base accuracy recommender to devise an effective solution for both sparse and dense settings.  %Our results  also indicate there is no single method that outperforms other methods in all metrics. However our techniques obtain a significant improvement in coverage, while  . 
Although we integrated the  long-tail novelty preference estimates into a re-ranking framework, their use-case is not limited to these frameworks. In  the future, we intend to explore the temporal and topical dynamics of long-tail novelty preference, particularly in settings where contextual information is  available.  
%We achieve these objectives without using any additional contextual information.


\iffalse
While we focused on promoting long-tail items across users, we did not consider diversity of individual top-$\size$ recommendations, a factor that has been shown to positively affect consumer satisfaction. This is one direction for future work. Moreover, the sequential setting  in our work, creates a dependency between different batches, where,  the items recommended to a batch of users, depends on those recommended to previous batches. This dependency is created through the parameter $\mathbf{f}$, that is updated every time a top-$\size$ set  is allocated to a batch of users. A future direction for our work is to estimate a distribution over $\mathbf{f}$ that allows us to independently solve the problem for each user, leading to improvements across all performance metrics, including recommendation time. 

We design algorithms that take advantage of the structure in the value functions to obtain both efficient and scalable solutions. 
We design an algorithm that takes advantage of the structure in the value functions to obtain both efficient and scalable solutions. 

\textcolor{red}{Our  sequential  algorithms can be applied for batch recommendation contexts,~e.g., personalized email marketing, where based on prior interaction data between users and items,  a new round of recommendations must be sent to all users in the system.  However, the independent coverage algorithms lift the sequential setting restrictions and allow it be applied for re-ranking the output of base recommender in any setting. }A future direction for our work is to incorporate explicit diversity metrics in the framework. 
\fi


%We have a presented a submodular maximization framework to systematically trade-off relevance and diversity in recommendations to individual users and coverage across the item-space. This ensures both consumer and producer satisfaction. We model users according to their risk and focusing degrees and promote long-tail items to the right group of consumers. Consequently, we obtain a significant improvement in coverage while maintaining reasonable levels of user satisfaction. Furthermore, our methods are able to achieve a more balanced distribution across the set of recommended items. In the future, we plan to investigate the effect of using alternative base recommender systems. 

%Future Work
%However most of these methods assume that the ratings are missing at random (MAR). Since our method of generating recommendations is based on the completed matrix, assuming MAR might introduce additional bias, we will use methods which assume that the ratings at missing not at random (MNAR),explored in~\cite{steck2010training, icml2014c2_hernandez-lobatob14}. 	 
%Long Tail %Recently, authors in~\cite{cremonesi2010performance} conducted extensive experiments to evaluate the performances of various matrix factorization-based algorithms and neighborhood models on the task of recommending long tail items. Their experimental results show that long tail recommendation leads to a decrease in accuracy for all algorithms. They also showed that for this task, SVD outperforms other state-of-the-art algorithms. 


\balance
\bibliographystyle{ieeetr}
\bibliography{reference}

% Document end
\end{document}
