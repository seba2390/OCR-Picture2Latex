
% Sets up the class for the document as (?)
\documentclass[letterpaper, 10 pt, conference]{ieeeconf}
% If the IEEEtran.cls has not been installed into the LaTeX system files,
% manually specify the path to it:
% \documentclass[conference]{../sty/IEEEtran}
%\IEEEoverridecommandlockouts		
%\overrideIEEEmargins

%\pdfminorversion=5
%\pdfcompresslevel=9
%\pdfobjcompresslevel=2

% Imports the needed LaTeX packages
\usepackage{amsmath}
\usepackage{amssymb}
%\usepackage{dblfloatfix}
\let\proof\relax 
\let\endproof\relax

\usepackage{tabularx}
%\usepackage{arydshln,graphicx,xcolor,array}
%\usepackage{amsthm}
\usepackage{graphicx}
\usepackage{bm}
\usepackage{color}
%\usepackage{needspace}
%\usepackage{empheq}
\usepackage{float}
\usepackage{units}
%\usepackage{footmisc}
\usepackage{url}
\usepackage{hyperref}
\usepackage{balance}
\usepackage{amsmath}

\makeatletter 
\def\endfigure{\end@float} 
\def\endtable{\end@float}
\makeatother 

\newcommand{\quat}{\bm{q}}
\newcommand{\pos}{\bm{p}}
\newcommand{\angVel}{\bm{\omega}}
\newcommand{\vel}{\bm{v}}
\newcommand{\posDot}{\bm{\dot{\pos}}}
\newcommand{\posDotDot}{\bm{\ddot{p}}}
\newcommand{\quatDot}{\bm{\dot{\quat}}}
\newcommand{\angVelDot}{\bm{\dot{\angVel}}}
\newcommand{\bOm}    {\mbox{\boldmath $\Omega$}}
\newcommand{\bom}    {\mbox{\boldmath $\omega$}}

%\usepackage{silence} 
%\WarningFilter{caption}{}
%\usepackage{subfig}
\usepackage{subcaption}
\captionsetup{compatibility=false}
\usepackage{caption}
% \usepackage[belowskip=0pt,aboveskip=1pt]{caption}
% \setlength{\abovedisplayskip}{5pt}
% \setlength{\belowdisplayskip}{5pt}

\captionsetup[figure]{font=footnotesize}

\renewcommand{\unit}[1]{{\rm #1} }

\newcommand{\qtn}[1]{{\color{red}QTN: #1}}
\newcommand{\todo}[1]{\textcolor{red}{#1}}
\newcommand{\update}[1]{\textcolor{black}{#1}}

\newtheorem{remark}{Remark}
	
% Imports the custom Commands file
% \newcommand*\rotsem{\multicolumn{1}{R{55}{1em}}}% no optional argument here, please!

\definecolor{unlabeled}{rgb}{0.0, 0.0, 0.0}
\definecolor{car}{rgb}{0.39215686274509803, 0.5882352941176471, 0.9607843137254902}
\definecolor{bicycle}{rgb}{0.39215686274509803, 0.9019607843137255, 0.9607843137254902}
\definecolor{motorcycle}{rgb}{0.11764705882352941, 0.23529411764705882, 0.5882352941176471}
\definecolor{truck}{rgb}{0.3137254901960784, 0.11764705882352941, 0.7058823529411765}
\definecolor{othervehicle}{rgb}{0.0, 0.0, 1.0}
\definecolor{person}{rgb}{1.0, 0.11764705882352941, 0.11764705882352941}
\definecolor{bicyclist}{rgb}{1.0, 0.1568627450980392, 0.7843137254901961}
\definecolor{motorcyclist}{rgb}{0.5882352941176471, 0.11764705882352941, 0.35294117647058826}
\definecolor{road}{rgb}{1.0, 0.0, 1.0}
\definecolor{parking}{rgb}{1.0, 0.5882352941176471, 1.0}
\definecolor{sidewalk}{rgb}{0.29411764705882354, 0.0, 0.29411764705882354}
\definecolor{otherground}{rgb}{0.6862745098039216, 0.0, 0.29411764705882354}
\definecolor{building}{rgb}{1.0, 0.7843137254901961, 0.0}
\definecolor{fence}{rgb}{1.0, 0.47058823529411764, 0.19607843137254902}
\definecolor{vegetation}{rgb}{0.0, 0.6862745098039216, 0.0}
\definecolor{trunk}{rgb}{0.5294117647058824, 0.23529411764705882, 0.0}
\definecolor{terrain}{rgb}{0.5882352941176471, 0.9411764705882353, 0.3137254901960784}
\definecolor{pole}{rgb}{1.0, 0.9411764705882353, 0.5882352941176471}
\definecolor{trafficsign}{rgb}{1.0, 0.0, 0.0}


\definecolor{unlabeled_n}{rgb}{0.0, 0.0, 0.0}
\definecolor{barrier_n}{rgb}{0.4392156863, 0.5019607843, 0.5647058823529412}
\definecolor{bicycle_n}{rgb}{0.8627, 0.0784, 0.2353}
\definecolor{bus_n}{rgb}{1.0000, 0.4980, 0.3137}
\definecolor{car_n}{rgb}{1.0000, 0.6196, 0.0000}
\definecolor{c_v_n}{rgb}{0.9137, 0.5882, 0.2745}
\definecolor{motorcycle_n}{rgb}{1.0000, 0.2392, 0.3882}
\definecolor{pedestrian_n}{rgb}{0.0000, 0.0000, 0.9020}
\definecolor{cone_n}{rgb}{0.1843, 0.3098, 0.3098}
\definecolor{trailer_n}{rgb}{1.0000, 0.5490, 0.0000}
\definecolor{truck_n}{rgb}{1.0000, 0.3882, 0.2784}
\definecolor{drive_n}{rgb}{0.0000, 0.8118, 0.7490}
\definecolor{other_flat_n}{rgb}{0.6863, 0.0000, 0.2941}
\definecolor{sidewalk_n}{rgb}{0.2941, 0.0000, 0.2941}
\definecolor{terrain_n}{rgb}{0.4392, 0.7059, 0.2353}
\definecolor{manmade_n}{rgb}{0.8706, 0.7216, 0.5294}
\definecolor{vegetation_n}{rgb}{0.0000, 0.6863, 0.0000}
% IV21 colormap
%vehicle othervehicle
%person person
%twowheel motorcycle
%rider motorcyclist
%road road
%sidewalk sidewalk
%otherground parking
%building building
%object fence
%vegetation vegetation
%trunk trunk 
%terrain terrain

\newcommand\semcolor[1][black]{\textcolor{#1}{\rule{2.2mm}{2.2mm}}}

% \input{Utils/pmw_commands.tex}

\overrideIEEEmargins
\IEEEoverridecommandlockouts

% Beginning of actual document
\begin{document} 

% Paper title (needs to change)
\title{\Large \bf 			
Force-and-moment-based Model Predictive Control for Achieving Highly Dynamic Locomotion on Bipedal Robots
}
	
%\newtheorem{assumption}{Assumption}

\author{Junheng Li and Quan Nguyen\thanks{Junheng Li and Quan Nguyen are with the Department of Aerospace and Mechanical Engineering, University of Southern California, Los Angeles, CA 90089.
email:{\tt\small junhengl@usc.edu, quann@usc.edu}}%
}%													
	
% make the title area
\maketitle
%Pins
%Junheng Li 150116

\begin{abstract}

In this paper, we propose a novel framework on force-and-moment-based Model Predictive Control (MPC) for dynamic legged robots. Specifically, we present a formulation of MPC designed for 10 degree-of-freedom (DoF) bipedal robots using simplified rigid body dynamics with input forces and moments. This MPC controller will calculate the optimal inputs applied to the robot, including 3-D forces and 2-D moments at each foot. These desired inputs will then be generated by mapping these forces and moments to motor torques of 5 actuators on each leg. We evaluate our proposed control design on physical simulation of a 10 degree-of-freedom (DoF) bipedal robot. The robot can achieve fast walking speed up to $ 1.6\:\unit{m/s}$ on rough terrain, with accurate velocity tracking. With the same control framework, our proposed approach can achieve a wide range of dynamic motions including walking, hopping, and running using the same set of control parameters.

\end{abstract}


% Introduction and literature review
\IEEEraisesectionheading{\section{Introduction}}

\IEEEPARstart{V}{ision} system is studied in orthogonal disciplines spanning from neurophysiology and psychophysics to computer science all with uniform objective: understand the vision system and develop it into an integrated theory of vision. In general, vision or visual perception is the ability of information acquisition from environment, and it's interpretation. According to Gestalt theory, visual elements are perceived as patterns of wholes rather than the sum of constituent parts~\cite{koffka2013principles}. The Gestalt theory through \textit{emergence}, \textit{invariance}, \textit{multistability}, and \textit{reification} properties (aka Gestalt principles), describes how vision recognizes an object as a \textit{whole} from constituent parts. There is an increasing interested to model the cognitive aptitude of visual perception; however, the process is challenging. In the following, a challenge (as an example) per object and motion perception is discussed. 



\subsection{Why do things look as they do?}
In addition to Gestalt principles, an object is characterized with its spatial parameters and material properties. Despite of the novel approaches proposed for material recognition (e.g.,~\cite{sharan2013recognizing}), objects tend to get the attention. Leveraging on an object's spatial properties, material, illumination, and background; the mapping from real world 3D patterns (distal stimulus) to 2D patterns onto retina (proximal stimulus) is many-to-one non-uniquely-invertible mapping~\cite{dicarlo2007untangling,horn1986robot}. There have been novel biology-driven studies for constructing computational models to emulate anatomy and physiology of the brain for real world object recognition (e.g.,~\cite{lowe2004distinctive,serre2007robust,zhang2006svm}), and some studies lead to impressive accuracy. For instance, testing such computational models on gold standard controlled shape sets such as Caltech101 and Caltech256, some methods resulted $<$60\% true-positives~\cite{zhang2006svm,lazebnik2006beyond,mutch2006multiclass,wang2006using}. However, Pinto et al.~\cite{pinto2008real} raised a caution against the pervasiveness of such shape sets by highlighting the unsystematic variations in objects features such as spatial aspects, both between and within object categories. For instance, using a V1-like model (a neuroscientist's null model) with two categories of systematically variant objects, a rapid derogate of performance to 50\% (chance level) is observed~\cite{zhang2006svm}. This observation accentuates the challenges that the infinite number of 2D shapes casted on retina from 3D objects introduces to object recognition. 

Material recognition of an object requires in-depth features to be determined. A mineralogist may describe the luster (i.e., optical quality of the surface) with a vocabulary like greasy, pearly, vitreous, resinous or submetallic; he may describe rocks and minerals with their typical forms such as acicular, dendritic, porous, nodular, or oolitic. We perceive materials from early age even though many of us lack such a rich visual vocabulary as formalized as the mineralogists~\cite{adelson2001seeing}. However, methodizing material perception can be far from trivial. For instance, consider a chrome sphere with every pixel having a correspondence in the environment; hence, the material of the sphere is hidden and shall be inferred implicitly~\cite{shafer2000color,adelson2001seeing}. Therefore, considering object material, object recognition requires surface reflectance, various light sources, and observer's point-of-view to be taken into consideration.


\subsection{What went where?}
Motion is an important aspect in interpreting the interaction with subjects, making the visual perception of movement a critical cognitive ability that helps us with complex tasks such as discriminating moving objects from background, or depth perception by motion parallax. Cognitive susceptibility enables the inference of 2D/3D motion from a sequence of 2D shapes (e.g., movies~\cite{niyogi1994analyzing,little1998recognizing,hayfron2003automatic}), or from a single image frame (e.g., the pose of an athlete runner~\cite{wang2013learning,ramanan2006learning}). However, its challenging to model the susceptibility because of many-to-one relation between distal and proximal stimulus, which makes the local measurements of proximal stimulus inadequate to reason the proper global interpretation. One of the various challenges is called \textit{motion correspondence problem}~\cite{attneave1974apparent,ullman1979interpretation,ramachandran1986perception,dawson1991and}, which refers to recognition of any individual component of proximal stimulus in frame-1 and another component in frame-2 as constituting different glimpses of the same moving component. If one-to-one mapping is intended, $n!$ correspondence matches between $n$ components of two frames exist, which is increased to $2^n$  for one-to-any mappings. To address the challenge, Ullman~\cite{ullman1979interpretation} proposed a method based on nearest neighbor principle, and Dawson~\cite{dawson1991and} introduced an auto associative network model. Dawson's network model~\cite{dawson1991and} iteratively modifies the activation pattern of local measurements to achieve a stable global interpretation. In general, his model applies three constraints as it follows
\begin{inlinelist}
	\item \textit{nearest neighbor principle} (shorter motion correspondence matches are assigned lower costs)
	\item \textit{relative velocity principle} (differences between two motion correspondence matches)
	\item \textit{element integrity principle} (physical coherence of surfaces)
\end{inlinelist}.
According to experimental evaluations (e.g.,~\cite{ullman1979interpretation,ramachandran1986perception,cutting1982minimum}), these three constraints are the aspects of how human visual system solves the motion correspondence problem. Eom et al.~\cite{eom2012heuristic} tackled the motion correspondence problem by considering the relative velocity and the element integrity principles. They studied one-to-any mapping between elements of corresponding fuzzy clusters of two consecutive frames. They have obtained a ranked list of all possible mappings by performing a state-space search. 



\subsection{How a stimuli is recognized in the environment?}

Human subjects are often able to recognize a 3D object from its 2D projections in different orientations~\cite{bartoshuk1960mental}. A common hypothesis for this \textit{spatial ability} is that, an object is represented in memory in its canonical orientation, and a \textit{mental rotation} transformation is applied on the input image, and the transformed image is compared with the object in its canonical orientation~\cite{bartoshuk1960mental}. The time to determine whether two projections portray the same 3D object
\begin{inlinelist}
	\item increase linearly with respect to the angular disparity~\cite{bartoshuk1960mental,cooperau1973time,cooper1976demonstration}
	\item is independent from the complexity of the 3D object~\cite{cooper1973chronometric}
\end{inlinelist}.
Shepard and Metzler~\cite{shepard1971mental} interpreted this finding as it follows: \textit{human subjects mentally rotate one portray at a constant speed until it is aligned with the other portray.}



\subsection{State of the Art}

The linear mapping transformation determination between two objects is generalized as determining optimal linear transformation matrix for a set of observed vectors, which is first proposed by Grace Wahba in 1965~\cite{wahba1965least} as it follows. 
\textit{Given two sets of $n$ points $\{v_1, v_2, \dots v_n\}$, and $\{v_1^*, v_2^* \dots v_n^*\}$, where $n \geq 2$, find the rotation matrix $M$ (i.e., the orthogonal matrix with determinant +1) which brings the first set into the best least squares coincidence with the second. That is, find $M$ matrix which minimizes}
\begin{equation}
	\sum_{j=1}^{n} \vert v_j^* - Mv_j \vert^2
\end{equation}

Multiple solutions for the \textit{Wahba's problem} have been published, such as Paul Davenport's q-method. Some notable algorithms after Davenport's q-method were published; of that QUaternion ESTimator (QU\-EST)~\cite{shuster2012three}, Fast Optimal Attitude Matrix \-(FOAM)~\cite{markley1993attitude} and Slower Optimal Matrix Algorithm (SOMA)~\cite{markley1993attitude}, and singular value decomposition (SVD) based algorithms, such as Markley’s SVD-based method~\cite{markley1988attitude}. 

In statistical shape analysis, the linear mapping transformation determination challenge is studied as Procrustes problem. Procrustes analysis finds a transformation matrix that maps two input shapes closest possible on each other. Solutions for Procrustes problem are reviewed in~\cite{gower2004procrustes,viklands2006algorithms}. For orthogonal Procrustes problem, Wolfgang Kabsch proposed a SVD-based method~\cite{kabsch1976solution} by minimizing the root mean squared deviation of two input sets when the determinant of rotation matrix is $1$. In addition to Kabsch’s partial Procrustes superimposition (covers translation and rotation), other full Procrustes superimpositions (covers translation, uniform scaling, rotation/reflection) have been proposed~\cite{gower2004procrustes,viklands2006algorithms}. The determination of optimal linear mapping transformation matrix using different approaches of Procrustes analysis has wide range of applications, spanning from forging human hand mimics in anthropomorphic robotic hand~\cite{xu2012design}, to the assessment of two-dimensional perimeter spread models such as fire~\cite{duff2012procrustes}, and the analysis of MRI scans in brain morphology studies~\cite{martin2013correlation}.

\subsection{Our Contribution}

The present study methodizes the aforementioned mentioned cognitive susceptibilities into a cognitive-driven linear mapping transformation determination algorithm. The method leverages on mental rotation cognitive stages~\cite{johnson1990speed} which are defined as it follows
\begin{inlinelist}
	\item a mental image of the object is created
	\item object is mentally rotated until a comparison is made
	\item objects are assessed whether they are the same
	\item the decision is reported
\end{inlinelist}.
Accordingly, the proposed method creates hierarchical abstractions of shapes~\cite{greene2009briefest} with increasing level of details~\cite{konkle2010scene}. The abstractions are presented in a vector space. A graph of linear transformations is created by circular-shift permutations (i.e., rotation superimposition) of vectors. The graph is then hierarchically traversed for closest mapping linear transformation determination. 

Despite of numerous novel algorithms to calculate linear mapping transformation, such as those proposed for Procrustes analysis, the novelty of the presented method is being a cognitive-driven approach. This method augments promising discoveries on motion/object perception into a linear mapping transformation determination algorithm.



% Robot Model and Simulation
%!TEX root = ../Main.tex

\section{Bipedal Robot Model and Simulation}
\label{sec:robotModel}

\subsection{Robot Model}
\label{subsec:robotModel}

In this section, we present the robot model that will be used throughout the paper. A 10 DoF bipedal robot consists of 5 joint actuation each leg (see Fig. \ref{fig:robotModel}). Commonly, a 10 DoF bipedal robot has abduction (ab) and hip joints that allow 3-D rotation and ankle joints that allow double-leg-support standing (e.g., \cite{levineblackbird,gong2019feedback}). 

The design of our bipedal consists of the robot body, ab link, hip link, thigh link, calf link, and foot link (see Fig. \ref{fig:legConfig} for the definition of the leg configuration and Table \ref{tab:PRP} for physical parameters). The robot body is adopted from the Unitree A1 robot, in a vertical configuration. The joint actuator modeled in this robot design is the Unitree A1 modular actuator, which is a lightweight and powerful torque-controlled motor that is suitable for mini legged robots (see Table \ref{tab:motor} for the parameters of the actuator). 
\begin{figure}[!h]
	\hspace{0.2cm}
     \center
     \begin{subfigure}[b]{0.227\textwidth}
         \centering
         \includegraphics[width=\textwidth]{Figures/Model_in_Blender.png}
         \caption{ }
         \label{fig:robotModel}
     \end{subfigure}
     %\hfill
     \begin{subfigure}[b]{0.21\textwidth}
         \centering
         \includegraphics[width=\textwidth]{Figures/Leg_config.png}
         \caption{ }
         \label{fig:legConfig}
     \end{subfigure}
        \caption{{\bfseries Bipedal Robot Configuration}  a) Robot CAD Model  b) The link and joint configuration of the bipedal robot left leg.}
        \label{fig:robotConfig}
\end{figure}

\begin{table}[h]
	\vspace{0.2cm}
	\centering
	\caption{Robot Physical Parameters}
	\label{tab:PRP}
	\begin{tabular}{cccc}
		\hline
		Parameter & Symbol & Value & Units\\
		\hline
		Mass & $m$    & 11.84 & $\unit{kg}$  \\[.5ex]
		Body Inertia  & $I_{xx}$  & 0.0443 & $\unit{kg}\cdot \unit{m}^2$ \\[.5ex]
		& $I_{yy}$ & 0.0535  & $\unit{kg}\cdot \unit{m}^2$ \\[.5ex]
		& $I_{zz}$ & 0.0214  & $\unit{kg}\cdot \unit{m} ^2$ \\[.5ex]
		Body Length & $l_{b}$ & 0.114 & $\unit{m}$ \\[.5ex]
		Body Width & $w_{b}$ & 0.194 & $\unit{m}$ \\[.5ex]
		Body Height & $h_{b}$ & 0.247 & $\unit{m}$ \\[.5ex]
		Thigh and Calf Lengths & $l_{1}, l_{2}$ & 0.2 & $\unit{m}$ \\[.5ex]
		Foot Length & $l_{toe}$ & 0.09 & $\unit{m}$ \\[.5ex]
		& $l_{heel}$ & 0.05 & $\unit{m}$ \\[.5ex]
		\hline 
		\label{tab:robot}
	\end{tabular}
\end{table}	

	\begin{table}[h]
%		\hspace{0.2cm}
		\centering
		\caption{Joint Actuator Parameters}
		\begin{tabular}{cccc}
			\hline
			Parameter & Value & Units\\
			\hline
			Max Torque   &  33.5 & $\unit{Nm}$  \\[.5ex]
			Max Joint Speed    & 21  & $\unit{Rad}/\unit{s}$  \\[.5ex]
			\hline 
			\label{tab:motor}
			\end{tabular}
			\end{table}
\subsection{Simulation}
\label{subsec:simulation}

The bipedal robot simulation is built primarily in MATLAB Simulink using Spatial v2 package \cite{featherstone2014rigid}. Fig. \ref{fig:controlArchi} shows the diagram for our control architecture and also the representation of our simulation software.
% is an implementation of spatial vector arithmetic and dynamics algorithms that is available in MATLAB scripts. The software employs Roy Featherstone’s book \emph{Rigid Body Dynamics Algorithms} and provides a series of accessible MATLAB functions for robotic dynamics simulations \cite{featherstone2014rigid}. 
% The scope of the simulation construction is to build easy-to-use and reliable MATLAB and Simulink models that can be used as a test bench for many forthcoming controller designs and optimization implementations (see Fig. \ref{fig:controlArchi}: simulation block diagram).

The simulation software requires the user to input desired states at the beginning of the simulation. The desired states form a column vector that consist desired body center of mass (CoM) position $\bm p_d$, desired body CoM velocity $\bm {\dot p}_d$, desired rotation matrix $\bm R_d$ (resized to 9×1 vector), and desired angular velocity $\bm {\omega}_d$ of robot body. 

\begin{figure}[h]
	\hspace{0.2cm}
		\center
		\includegraphics[width=1 \columnwidth]{Figures/Control_archi.png}
		\caption{{\bfseries Control Diagram.} The simplified block diagram and control architecture of our proposed approach.}
		\label{fig:controlArchi}
	\end{figure}

% The simulation model shown in Fig. \ref{fig:controlArchi} provides a platform for controller implementation with the advantages of fast simulation time, simplicity in modification and debugging, and reliability.
% Main theory 
\section{Dynamics and Control}
\label{sec:dynamicsAndControl}
\subsection{Simplified Dynamics}
\label{subsec:simplified_dynamics}
In this Section, we investigate different dynamic models that can be used in our MPC control framework. 
While the whole-body dynamics of the bipedal robot are highly non-linear, we are interested in using simplified rigid-body dynamics to guarantee that our MPC controller can be solved effectively in real-time. 
In addition, in order to enable the capability of absorbing frequent and hard impacts from dynamic locomotion, the design of bipedal robots also requires lightweight limbs and connections. This has allowed the weight and rotation inertia of each link part to be very small compared to the body weight and rotation inertia. Hence, the effect of leg links in the robot dynamics may be neglected, forming simplified rigid-body dynamics \cite{nguyen2019optimized,bledt2018cheetah}. This is also a common assumption in many legged robots’ controller designs \cite{focchi2017high,stephens2010push}.

There are three simplified dynamics models that we have considered and tested, shown in Fig. \ref{fig:simplifiedDynamics}. The main difference between these three options is the number of contact points on each foot, contact location, and contact force and moment formation at each contact point. Model 1 resembles the simplified dynamics choice for quadruped robots mentioned in \cite{bledt2018cheetah}, with 4 contact points exerting 3-D contact forces. However, with this dynamics model applied to our 10-DoF bipedal robot, under rotation motions testings in simulation, the robot is unable to perform pitch motion properly. Model 2 is improved and further simplifies the contact points. However, with this simplified dynamics model, in simulation validation process, the robot is unable to perform roll motion correctly. Hence, we proposed model 3 that excludes external moment around $x$-axis, and contains only 3-D contact forces and 2-D moments around $y$ and $z$-axis. This model has allowed the robot to perform all 3-D rotations effectively. Hence, model 3 is chosen to be the final simplified dynamics design in this framework. More details about the validation of this decision process is shown in Section \ref{sec:simulationResults}.
The detailed derivation of the model 3 dynamics is presented as follows.

The bipedal model in this paper has two legs that both consists of 5 DoF. Commonly, the external contact forces applied to the robot are only limited to 3-D forces in many legged-robot dynamics (e.g., \cite{levineblackbird,nguyen2019optimized}). However, thanks to the additional hip and ankle joint actuation, the external moments can also be included in the robot dynamics, forming a linear relationship between robot body’s acceleration $ {\bm {\ddot p}_c}$, rate of change in angular momentum $\bm {\dot H}$ about CoM \cite{stephens2010push}, and contact force and moments %$ \bm F=[ \bm F_1,\:\bm F_2]^T$ , where $\bm F_i= [\bm F_{ix},\:\bm F_{iy},\:\bm F_{iz},\:\bm M_{ix},\:\bm M_{iy},\:\bm M_{iz} ]^T$, $i=1,2$ , as follows:
\update{$ \bm u=[\bm F_1,\:\bm F_2,\:\bm M_1,\:\bm M_2]^T,  \bm F_i = [\bm F_{ix},\:\bm F_{iy},\:\bm F_{iz}]^T, \bm M_i = [\bm M_{iy},\:\bm M_{iz}]^T, i = 1, 2, $ shown as follows: 
\setlength{\belowdisplayskip}{5pt} \setlength{\belowdisplayshortskip}{5pt}
\setlength{\abovedisplayskip}{5pt} \setlength{\abovedisplayshortskip}{5pt}
\begin{align}
\label{eq:simplifiedDynamics}
\left[\begin{array}{ccc} \bm {{D}_1}   \\ 
 \hspace{0cm} \bm {{D}_2} \end{array}  \right] \bm u= \left[\begin{array}{c} m (\ddot{\pos}_{c} +\bm{g}) \\ \bm {\dot H} \end{array} \right],
\end{align}
where
\begin{align}
\label{eq:D1}
 \bm {D}_1  = \left[\begin{array}{cccc} \mathbf {I}_{3\times3}  & \mathbf {I}_{3\times3} & 
  \mathbf {0}_{3\times2} &  \mathbf {0}_{3\times2} \end{array} \right],
\end{align}
\begin{align}
\label{eq:D2}
 \bm {D}_2  = \left[\begin{array}{cccc} \bm {(p_1-p_c)}\times  & \bm {(p_2-p_c)}\times & 
  \mathbf L &  \mathbf L \end{array} \right],
\end{align}
\begin{align}
\label{eq:L}
 \mathbf L  = \left[\begin{array}{ccc}  \mathbf 0 & \mathbf 0 \\  \mathbf 1 &\mathbf 0 \\  \mathbf 0 &\mathbf 1
 \end{array} \right].
\end{align} }

The term ${(\bm p_i- \bm p_c)}$ denotes the distance vector from the robot body CoM location to the foot $i$ position in the world coordinate; and $\bm {(p_i-p_c)}\times $ represents the skew-symmetric matrix representing the cross product of ${ {(\bm p_i-\bm p_c)} \times \bm F_i }$. Here, $\bm {\dot H}$ can be approximated as $\bm{ \dot H=I_G \dot {{\omega}}}$  (as discussed in \cite{nguyen2019optimized}), where $\bm {I_G}$ stands for the centroid rotation inertia of robot body in the world frame and  $\bm {\dot {\omega}}$ represents the angular velocity of robot body in the world frame \cite{bledt2018cheetah,stephens2010push}.
	


\begin{figure}[!h]
\vspace{0.2cm}
     \centering
     \begin{subfigure}[b]{0.15\textwidth}
         \centering
         \includegraphics[width=\textwidth]{Figures/simplifiedDynamics1.png}
         \caption{Model 1}
         \label{fig:model1}
     \end{subfigure}
     \hfill
     \begin{subfigure}[b]{0.13\textwidth}
         \centering
         \includegraphics[width=\textwidth]{Figures/simplifiedDynamics2.png}
         \caption{Model 2}
         \label{fig:model2}
     \end{subfigure}
     \hfill
     \begin{subfigure}[b]{0.145\textwidth}
         \centering
         \includegraphics[width=\textwidth]{Figures/simplifiedDynamics3.png}
         \caption{Model 3}
         \label{fig:model3}
     \end{subfigure}
        \caption{{\bfseries Three Simplified Dynamics Models.} Yellow dots on the figures represent the contact point locations in simplified dynamics a) 2 contact points at the toe and heel of each robot foot, 3-D forces applied  b) 1 contact point at the middle of each foot, 3-D force and 3-D moments applied  c) 1 contact point at the middle of each foot, 3-D force and 2-D moments applied. Model 3 is the final choice to be used in our proposed approach.}
        \label{fig:simplifiedDynamics}
\end{figure}
%\todo{We should have a figure to illustrate the simplified rigid body model including 1 rigid body, 2 contact point on the ground, force vectors, My, Mz,...}
%\todo{We should also consider make comparisons between different models: the current one (1 contact point each leg: 3 forces + 2 moments for each point), 2 contact point each leg (3 forces + 3 moments for each points), 2 contact point each leg (3 forces + 0 moments for each points)... like you used to explore. This will make the paper stronger.}
\update{Equation (\ref{eq:simplifiedDynamics}) to (\ref{eq:L}) describes the simplified dynamics model 3. This model only used 5 force and moment inputs $\bm U$ which are directly mapped to 5 joint torques in each leg.}

We use rotation matrix $\bm R$ as a state variable to represent the orientation of the robot body, which can be also directly converted to Euler Angles. 
%Reference \cite{di2018dynamic} presents the 
We linearize the rotation matrix by approximating the angular velocity in terms of Euler angle $ {\Theta = [\phi,\:\theta,\:\psi]}^T$, where $\phi$ is roll angle,  $ \theta$ is pitch angle, and $ \psi$ is yaw angle. With the assumption of small roll and pitch angles  \cite{di2018dynamic}, the relation of the rate of change of $\Theta$ and angular velocity $\bm \omega$ in the world coordinate can be approximated as: 
%\begin{align}
%\label{eq:omega}
%%\bm \omega  = \left[\begin{array}{ccc} {\cos{\theta} \cos{\psi}}& -{\sin{\psi}} & 0 \\{\cos{\theta} \sin{\psi}} & {\cos{\psi}} & 0 \\ -{\sin{\theta}} & 0 & 1\end{array} 
%\right ] 
%\begin{bmatrix} {\dot{\phi}} \\ {\dot{\theta}} \\ {\dot{\psi}}\end{bmatrix}
%\end{align} 

%\begin{align}
%\label{eq:omega}
%\bm \omega  = \left[\begin{array}{ccc} %\dot{\phi}\sin{\theta}\sin{\psi}+\dot{\theta}\cos{\psi} \\ 
%\dot{\phi}\sin{\theta}\cos{\psi}-\dot{\theta}\sin{\psi} \\ 
%\dot{\phi}\cos{\theta}+\dot{\psi}
%\end{array} 
%\right ] 
%\end{align} 

%With small pitch and roll angles, equation (\ref{eq:omega}) can be approximated and rewritten as
\begin{align}
\label{eq:omega2}
\begin{bmatrix} {\dot{\phi}} \\ {\dot{\theta}} \\ {\dot{\psi}}\end{bmatrix} \approx \left[\begin{array}{ccc} { \cos{\psi}}& {\sin{\psi}} & 0 \\{ -\sin{\psi}} & {\cos{\psi}} & 0 \\ 0 & 0 & 1\end{array} 
\right ] 
\begin{array}{cc}
     {\bm \omega}
\end{array}.
\end{align} 
% \todo{You should double-check eq 5,6 because they may not be correct in [9].}

Hence the kinematic constraint of the Euler angles is obtained as follow:
\begin{align}
\label{eq:omega3}
\begin{bmatrix} {\dot{\phi}} \\ {\dot{\theta}} \\ {\dot{\psi}}\end{bmatrix} \approx \bm R_z(\psi) \bm \omega.
\end{align} 

Combing the approximated orientation dynamics and the translation dynamics, the simplified dynamics of the robot can be written as: 
\update{
\begin{align}
\label{eq:stateSpace}
\frac{d}{dt}
\begin{bmatrix} 
{\bm \Theta} \\
{\bm p}_c \\ 
{\bm \omega} \\ 
\dot {{\bm p}}_c
\end{bmatrix} = 
\bm A
\begin{bmatrix} 
{\bm \Theta} \\
{\bm p}_c \\ 
{\bm \omega} \\ 
\dot {{\bm p}}_c
\end{bmatrix} 
+ \bm B \bm u +
\begin{bmatrix} 
\mathbf 0 \\ \mathbf 0 \\\mathbf  0 \\\mathbf  g
\end{bmatrix},
\end{align} 
where
\begin{align}
\label{eq:A}
\bm A = 
\begin{bmatrix} 
\mathbf {0}_{3\times3} & \mathbf {0}_{3\times3} & \bm R_z(\psi) & \mathbf {0}_{3\times3}  \\
\mathbf {0}_{3\times3} & \mathbf {0}_{3\times3} & \mathbf {0}_{3\times3} & \mathbf  {I}_{3\times3}  \\ 
\mathbf {0}_{3\times3} & \mathbf {0}_{3\times3} & \mathbf {0}_{3\times3} & \mathbf {0}_{3\times3} \\ 
\mathbf {0}_{3\times3} & \mathbf {0}_{3\times3} & \mathbf {0}_{3\times3} & \mathbf {0}_{3\times3}, 
\end{bmatrix},
\end{align} 
\begin{align}
\setlength\arraycolsep{2.5pt}
\label{eq:B}
\bm B = 
\begin{bmatrix} 
\mathbf {0}_{3\times3} & \mathbf {0}_{3\times3} & \mathbf {0}_{3\times2} & \mathbf {0}_{3\times2}  \\
\mathbf {0}_{3\times3} & \mathbf {0}_{3\times3} & \mathbf {0}_{3\times2} & \mathbf {0}_{3\times2}  \\ 
\bm {{\Tilde{I}^{-1}_G} (p_1-p_c)}\times & \bm {{\Tilde{I}^{-1}_G} (p_2-p_c)}\times & \bm{{\Tilde{I}^{-1}_G} \mathbf L} & \bm{{\Tilde{I}^{-1}_G} \mathbf L} \\ 
\mathbf { {I}_{3\times3}}/m & \mathbf {{I}_{3\times3}}/m & \mathbf {0}_{3\times2} & \mathbf {0}_{3\times2} 
\end{bmatrix}
\end{align} }

\update{Here, $\bm{\Tilde{I}^{-1}_G}$ is approximated by rotation inertia of the robot body in its body frame $\bm I_b$ and $\bm R_z(\psi)$ from \eqref{eq:omega3}:
\begin{equation}
\label{eq:rotationI}
\bm{\Tilde{I}_G} = \bm R_z(\psi) \bm I_b {\bm R_z(\psi)}^T.
\end{equation}
}
By assigning gravity as additional state variable, now state ${{\bm x}} = [{{\bm \Theta}},\: {\bm p}_c,\:{{\bm \omega}},\:\dot { {\bm p}}_c,\:\bm g]^T$ will allow the dynamics in (\ref{eq:stateSpace}) to be rewritten into a linear state-space form with continuous time matrices $\bm {\hat{A_c}}$ and $\bm {\hat{B_c}}$: 
\begin{equation}
\label{eq:linearSS}
\dot{{\bm { x}}}(t) = \bm {\hat{A_c}}(\psi) {{\bm {x}}}(t) + \bm {\hat{B_c}}([p_1-p_c],[p_2-p_c],\psi) \bm u(t).
\end{equation}

The linearized dynamics in (\ref{eq:linearSS}) is now suitable for the convex MPC formulation presented in \cite{di2018dynamic}.

\subsection{MPC Formulation}
\label{subsec:MPC}
Having discussed the dynamics model, we now present details about the formulation of our MPC controller.

The linearized dynamics in (\ref{eq:linearSS}) can be represented in a discrete-time form at each time step $i$
\begin{align}
\label{eq:discreteDynamics}
{\bm {x}}[i+1] = \bm {\hat{A}}[i] \bm x[i] + \bm {\hat{B}}[i]\bm u[i],
\end{align}
where discrete time matrix $\bm {\hat{A}}$ is a constant matrix computed from $\bm {\hat{A_c}}(\psi)$ using a average yaw value during entire reference trajectory; and $\bm {\hat{B}}$ matrix is computed from $ \bm{\hat{B_c}}([p_1-p_c],[p_2-p_c],\psi)$, using the desired values of average yaw and foot location. The only exception is that at the first time step,  $\bm {\hat{B}}[1]$ is computed from current states of the robot instead of reference trajectory.

An MPC problem with a finite horizon length $k$ can be written in the following standard form:
\begin{align}
\label{eq:MPCform}
%\nonumber
\underset{\bm{x,u}}{\operatorname{min}}   \:\:  & \sum_{i = 0}^{k-1}(\bm x_{i+1}-  {\bm x_{i+1}}_{ref})^T\bm Q_i(\bm x_{i+1}- {\bm x_{i+1}}_{ref}) + \| \bm{u}_i \|\bm{R}_i
\end{align}
\begin{align}
\label{eq:dynamicCons}
\:\:\mbox{s.t. }& \quad  {\bm {x}}[i+1] = \bm {\hat{A}}[i]\bm x[i] + \bm {\hat{B}}[i]\bm u[i], \quad i = 0 \dots k-1
\end{align}
\begin{align}
\label{eq:MPCineqCons}
\quad  \bm {c^-}_i \leq \bm C_i\bm u_i \leq \bm {c^+}_i, \quad i = 0 \dots k-1
\end{align}
\begin{align}
\label{eq:MPCeqCons}
& \quad \bm D_i \bm u_i = 0 , \quad i = 0 \dots k-1
\end{align}

%\todo{We should use the format $x^TQx$ instead of $||x||Q$ in the cost function.}

In (\ref{eq:MPCform}), $\bm x_i$ and $\bm u_i$ are system states and control inputs at time step $i$. \update{Note that the MPC prediction is computed based on the measured states of current step (i.e. $i = 0$).} $\bm Q_i$ and $\bm R_i$ are matrices defining the weights of \update{each state and control input variable.} $\bm {\hat{A}}$ and $\bm {\hat{B}}$ in (\ref{eq:dynamicCons}) are the discrete-time system dynamic constraints from (\ref{eq:discreteDynamics}). $\bm {c^-}_i$,$\bm {c^+}_i$, and $\bm C_i$ in (\ref{eq:MPCineqCons}) represents the inequality constraints of the MPC problem. $\bm D_i$ in (\ref{eq:MPCeqCons}) represents the equality constraints. In this problem, the equality constraint governs the optimal control input from MPC controller is a zero vector for swing foot. 

The MPC controller solves the optimal ground contact force and moment with respect to dynamic constraints \eqref{eq:dynamicCons} and the following inequality constraints:
\begin{align}
\label{eq:frictionCons}
\nonumber 
-\mu \bm {F}_{iz} \leq \bm F_{ix} \leq \mu \bm {F}_{iz} \\
-\mu \bm {F}_{iz} \leq \bm F_{iy} \leq \mu \bm {F}_{iz} \\
% \end{align}
% \begin{align}
\label{eq:forceCons}
0< \bm {F}_{min} \leq \bm  F_{iz} \leq \bm {F}_{max} \\
% \end{align}
% \begin{align}
\label{eq:torqueCons}
\bm |{\tau}_{i}| \leq \bm {\tau}_{max}. 
\end{align}
Here, \eqref{eq:frictionCons} governs the contact forces in $x$ and $y$ direction are within the friction pyramid, with $\mu$ being the friction coefficient. The contact forces in $z$-direction should also fall within the upper and lower bounds of force (\ref{eq:forceCons}), where the lower bound is positive to maintain contact with the ground. It is also important to restrict the joint torques to be within the saturation of the physical motor (\ref{eq:torqueCons}). 


\subsection{QP Formulation}
% It is stated in Section \ref{sec:robotModel} of this paper that the scope of developing the robot simulation in MATLAB and Simulink is to have a fast and reliable simulation software to test controller designs. MPC problem can be heavy to solve so it is important to reduce the computation and problem size of MPC by formulating the problem into a quadratic program (QP).
With the linear dynamics in Section \ref{subsec:simplified_dynamics} and the MPC formulation in Section \ref{subsec:MPC}, our controller can be formulated as a quadratic program (QP) that can be solved effectively in real-time. 
% With given equality and inequality constraints, we can form a QP problem based on the dynamics from the condensed formulation in (\ref{eq:QPdynamics}). The Optimization Toolbox in MATLAB provides powerful and fast QP solver which is suitable for this problem. 

Firstly, the dynamic constraints \eqref{eq:linearSS} for the entire MPC prediction horizon can be written as:
\begin{align}
\label{eq:QPdynamics}
\bm X = \bm{A}_{qp} \bm x_0 + \bm{B}_{qp} \bm U,
\end{align}
where $\bm X$ is a column vector containing system states for the next $k$ horizons, $\bm x[i+1],\bm x[i+2] \dots {\bm x}[i+k]$ and $\bm U$ is a column vector containing optimal control inputs of current state $\bm u[i]$ and next $k-1$ horizons, $\bm u[i+1], \bm u[i+2] \dots {\bm u}[i+k-1]$ at time step $i$. 
The MPC controller now can be written as the following QP form:     
\begin{align}
\label{eq:QPform}
%\nonumber
\underset{\bm{U}}{\operatorname{min}}   \:\:  & \frac{1}{2}\bm U^T\bm h \bm U + \bm U^T\bm f \\
% \end{align}
% \begin{align}
\label{eq:QPineqCons}
\:\:\mbox{s.t. }& \quad  \bm C\bm U \leq \bm d \\
% \end{align}
% \begin{align}
\label{eq:QPeqCons}
& \bm A_{eq}\bm U = \bm b_{eq}
\end{align}
where $\bm C$ and $\bm d$ are inequality constraint matrices, $\bm A_{eq}$ and $\bm b_{eq}$ are equality constraint matrices, and 
\begin{align}
\label{eq:h}
\bm h = 2( {\bm B_{qp}}^T\mathbf M {\bm B_{qp}}+\mathbf K), \\
% \end{align}
% \begin{align}
\label{eq:f}
\bm f = 2 {\bm B_{qp}}^T\mathbf M ({\bm A_{qp}}\bm x_0-\bm y).
\end{align}
Diagonal matrices $\mathbf K$ and $\mathbf M$ are the weights for the rate of change of state variables and force/moment magnitude.

\update{The resulting controller input of each leg from QP problem $\bm u_i=[\bm F_i,\: \bm M_i]^T$ is mapped to its joint torques by
\begin{align}
\label{eq:forceTorquemap}
\bm {\tau}_i =  \bm J_i^T \bm u_{i},
\end{align} 
where $\bm J_i$ is the Jacobian matrix of $i$th leg 
\begin{align}
\label{eq:Jacobian}
\bm {J}_i^T = \begin{bmatrix} 
\bm {J}_v^T & \bm {J}_\omega^T \bm L
\end{bmatrix},
\end{align} 
}
with $\bm {J}_v $ and $\bm {J}_{\omega} $ being the linear velocity and angular velocity components of $\bm {J}_i $.

\subsection{Swing Leg Control}
As discussed earlier in this section, due to equality constraints, the robot leg that is under the swing phase does not exert ground contact forces and therefore is not under the control of force-and-moment-based MPC. In order to control the leg and foot position in each gait cycle, the desired foot trajectory is under control in the Cartesian space with a PD position controller. \update{The gait sequence is purely based on timing and the gait cycle length is currently set at $0.3 \unit{s}$.}
We obtain the current foot location using forward kinematics. Foot velocity is computed by:
\begin{align}
\label{eq:footVel}
{\dot {\bm p}}_{{foot}_i} =  \bm J_i^T \dot{\bm q_i},
\end{align}
where $\dot{\bm q_i}$ is the joint velocity state-feedback of each leg at time step $i$.

The desired foot location $\bm p_{{foot}_d}$ in the world frame is determined by the foot placement policy employed in \cite{di2018dynamic}:
\begin{align}
\label{eq:footPlacement}
\bm p_{{foot}_d} =  \bm p_{hip} + \dot{\bm p}_c \Delta t/2,
\end{align}
where $\bm p_{hip}$ is the hip joint location in the world frame and $\Delta t$ is the time that stance foot spends on the ground during one gait cycle. 

The swing leg force can be computed by treating the foot attached to a virtual spring-damper system \cite{chen2020virtual}. The foot weight is reasonable to be neglected since it is very small compared to the robot body \cite{nguyen2019optimized}. Following the PD control law, the foot force can be written as: 
\begin{align}
\label{eq:PDswing}
\bm F_{swing_i}=\bm K_P(\bm p_{{foot}_d}-\bm p_{{foot}_i})+\bm K_D(\dot{\bm p}_{{foot}_d}-\dot{\bm p}_{{foot}_i})
\end{align}
where $\bm K_P$ and $\bm K_D$ are PD control gains, or spring stiffness and damping coefficient of the virtual spring-damper system. %$\bm p_{{foot}_d}$ and $\dot{\bm p}_{{foot}_d}$ are desired foot placement and velocity, respectively.

Similar to (\ref{eq:forceTorquemap}), the joint torque can be computed by:
\begin{align}
\label{eq:forceTorqueMapSwing}
\bm {\tau}_{swing_i} =  \bm J_v^T \bm F_{swing_i}.
\end{align}

With Cartesian PD control, the swing leg can move and be controlled to follow desired foot placement trajectory. The gait generator decides either the robot leg is in the stance phase or swing phase in a fixed gait cycle and assigns the appropriate controller to the corresponding leg. Now the robot has both swing and stance leg control, it is ready to test the MPC in simulation.
% results
\section{Simulation Results}
\label{sec:simulationResults}

In this section, we present numerical validation of our proposed approach for different dynamic locomotion. 
% The simulation video for this paper is given in Fig. \ref{fig:roughTerrainSim}. 
The reader is encouraged to watch the supplemental video\footnote{\url{https://youtu.be/Z2s4iuYkuvg}} 
for the visualization of our results.
For our simulation, the bipedal robot model and ground contact model are set up in MATLAB with Spatial v2 software. The MPC sampling frequency is set to $0.03\:\unit{s}$ while the simulation is run at $1\: \unit{kHz}$. One gait cycle that contains 10 horizons is predicted at each time step in MPC, in which each gait cycle is fixed at $0.30\:\unit{s}$. \update{This prediction length has been also used in \cite{di2018dynamic}.}

\update{The weighting factors $\bm Q$ in \eqref{eq:MPCform} are tuned to balance the performance between different control actions. In our simulation, we use $\bm Q_x = \bm Q_y = 50$, $\bm Q_z = 100$, $\bm Q_{\phi} = \bm Q_{\theta} = 100$, and $\bm Q_{\psi} = 20$. The rest weighting factors in $\bm Q$ remains at 1. }

\begin{figure}%[!h]
\vspace{0.5cm}
     \centering
     \begin{subfigure}[b]{0.13\textwidth}
         \centering
         \includegraphics[width=\textwidth]{Figures/pitchModel1.png}
         \caption{Snapshot of Model 1 in Double-leg Stance}
         \label{fig:pitchModel1}
     \end{subfigure}
     %\hfill
     \quad \quad
     \begin{subfigure}[b]{0.13\textwidth}
         \centering
         \includegraphics[width=\textwidth]{Figures/pitchModel3.png}
         \caption{Snapshot of Model 3 in Double-leg Stance}
         \label{fig:pitchModel3}
     \end{subfigure}
     \hfill
     \begin{subfigure}[b]{0.5\textwidth}
         \centering
         \includegraphics[width=0.9\textwidth]{Figures/PitchComp2_final.pdf}
         \caption{Pitch Motion Comparison}
         \label{fig:pitchPlot}
     \end{subfigure}
        \caption{{\bfseries Comparison of Model 1 and Model 3 in Pitch Motion Simulation}  a) Snapshot at the end of simulation with model 1  b) Snapshot at the end of simulation with model 3   c) Pitch motion response comparison with a $10^{\circ}$ desired pitch input.}
        \label{fig:pitchComparison}
\end{figure}

	
\begin{figure}%[!h]
	\hspace{0.2cm}
     \center
     \begin{subfigure}[b]{0.13\textwidth}
         \centering
         \includegraphics[width=\textwidth]{Figures/rolModel2.png}
         \caption{Snapshot of Model 2 in Double-leg Stance}
         \label{fig:rollModel2}
     \end{subfigure}
     %\hfill
     \quad \quad
     \begin{subfigure}[b]{0.14\textwidth}
         \centering
         \includegraphics[width=\textwidth]{Figures/rollModel3.png}
         \caption{Snapshot of Model 3 in Double-leg Stance}
         \label{fig:rollModel3}
     \end{subfigure}
     %\hfill
     \begin{subfigure}[b]{0.5\textwidth}
         \centering
         \includegraphics[width=0.9\textwidth]{Figures/RollComp2_final.pdf}
         \caption{Roll Motion Comparison}
         \label{fig:rollPlot}
     \end{subfigure}
        \caption{{\bfseries Comparison of Model 2 and Model 3 in Roll Motion Simulation}  a) Snapshot at the end of simulation with model 2  b) Snapshot at the end of simulation with model 3   c) Roll motion response comparison with a $10^{\circ}$ desired roll input.}
        \label{fig:rollComparison}
\end{figure}

\subsection{Validation of Simplified Dynamics}
First, we present the simulation results of simple rotation motions during standing with both legs on the ground to validate the claim in Section \ref{sec:dynamicsAndControl} that for the simplified dynamics used for control design, model 3 is a superior choice over model 1 and model 2. 

As mentioned in Section \ref{sec:dynamicsAndControl}, the simplified dynamics model 1 is unable to perform pitch motion. It is shown in Fig. \ref{fig:pitchComparison}, a pitch motion comparison between using simplified dynamics model 1 and model 3. The latter one is what we ultimately chose to use in MPC formulation. It is observed that the simulation result with model 1 does not respond to desired pitch input, whereas model 3 can perform pitch motion. 

We then further simplified model 1 and added 3-D moment inputs to each contact point to form simplified dynamics model 2. However, in the roll motion test, the response with model 2 is incorrect to desired roll input and it also shows a deviation in yaw angle as shown in Fig. \ref{fig:rollComparison}. With model 3, the robot simulation succeed in the roll motion test. 
Therefore, we decide to use model 3 for our proposed approach. Following are simulation results for walking and hopping motion using MPC control for model 3.

\subsection{Velocity Tracking}

In this simulation, we test the MPC performance in forward walking motion(positive $x$-direction) with time-varying desired speed and the desired CoM height of $0.5\:\unit{m}$. 
%The joint torques during this simulation are presented in Fig. \ref{fig:velSimTorque}. Note that during the simulation, all joint torque data are within the maximum torque threshold of our motor choice. There are no extreme torque values found during this entire simulation. 
The velocity tracking plot is shown in Fig. \ref{fig:velTracking}, the actual response curve with MPC shows a good tracking performance. The velocity response has a maximum deviation of $0.076\: \unit{m/s} $ compared to the desired input. Besides walking forward, we also have successful simulation results and demonstrations in walking sideways and diagonally. This result validates the effectiveness of our proposed control framework in realizing 3D dynamic locomotion for bipedal robots. 
% The simulation results can be found in the video (URL is under Fig. \ref{fig:roughTerrainSim}) associated with this paper. 
%\begin{figure}[h]
%	\center
%	\includegraphics[width=1 \columnwidth]{Figures/MPC_tau.png}
%	\caption{{\bfseries Plots of Joint Torques with MPC.} Joint torques of stance leg under the control of MPC in time-varying velocity simulation. }
%	\label{fig:velSimTorque}
%\end{figure}
\begin{figure}[h]
	\hspace{0.2cm}
	\center
	\includegraphics[width=0.75 \columnwidth]{Figures/velTracking_resized.pdf}
	\caption{{\bfseries Velocity Tracking.} Comparison of desired velocity input and actual velocity response in x-direction. 
% 	The desired velocity curve keeps constant from $t=0\:\unit{s}$ to $t=0.6\:\unit{s}$ and from $t=1.5\:\unit{s}$ to $t=3\:\unit{s}$ at $\bm v_{x_d}=0\:\unit{m/s}$ and $\bm v_{x_d}=0.6\:\unit{m/s}$, respectively. From $t=0.6\:\unit{s}$ to $t=1.5\:\unit{s}$ and from $t=3\:\unit{s}$ to $t=3.9\:\unit{s}$, $\bm v_{x_d}$ increases linearly.
	}
	\label{fig:velTracking}
\end{figure}
	
\subsection{High-velocity Walking in Rough Terrain}
We also validated the controller performance in rough terrain locomotion at high speed. Specifically, the robot is commanded to walk through a $2.4$-meter-long rough terrain formed by stairs with various heights and lengths. The stair heights range from $0.020\:\unit{m}$ to $0.075\:\unit{m}$ with a maximum height difference of $0.055\:\unit{m}$ between two consecutive stairs. To validate the feasibility and potential of MPC locomotion through rough terrain, the robot is commanded to follow a high desired velocity $\bm v_{x_d}=1.6 \:\unit{m/s}$. A snapshot of this simulation is provided in Fig. \ref{fig:roughTerrainSim}. 
%\begin{figure}[h]
%	\center
%	\includegraphics[width=0.85 \columnwidth]{Figures/Rough_terrain_spatial.png}
%	\caption{{\bfseries Snapshot from Rough Terrain Simulation.} Terrain model with stairs at various heights, animated with Spatial v2.  }
%	\label{fig:roughTerrain}
%\end{figure}

Plots of CoM location, velocity, and body orientation are shown in Fig. \ref{fig:CoMposVel} and Fig. \ref{fig:CoMeulAng}. It can be observed that the CoM location and orientation during this simulation maintain small tracking errors. 
%The position curves of the joints show that the joint position responses are smooth under the control of MPC and PD Cartesian control. The MPC controller input is presented in Fig. \ref{fig:MPCforce}. 
%The force in the $y$-direction $\bm F_y$ and moment $\bm M_z$ has the largest variation between the two legs. Hence it yields a slight $y$-direction displacement at the end of the simulation. As can be seen in CoM $y$-direction location in Fig. \ref{fig:CoMposVel}, the final $y$-direction location of body CoM is $0.0087 \:\unit m$ at $t = 1.8\: \unit{s}$. 
The joint torques (shown in Fig. \ref{fig:MPCtauRT}) during this entire simulation are in reasonable ranges and satisfy the torque saturation shown in Table \ref{tab:motor}. 
% todo{You should cite the table II about the torque limit here and also mention about the satisfaction of joint speed limit in Fig 11.} 
% (this is not really accurate so I commented it out)It is expected that the ankle joints will exert higher magnitudes of torques. Shown in the corresponding plots, the magnitudes of torques are still under the maximum torque threshold in most occasions. 

%With above simulation results and observations. It is inferred that the framework with the new MPC model presented in this paper can be a feasible option for a 10 DoF bipedal robot in dynamic locomotion. Our future work include extending this control framework to more dynamic motions such as bipedal bounding and running \todo{We have new results so please check sth like this throughout the paper to make sure that it is consistent with the current result}. Eventually, this control framework is expected to be migrated to a physical bipedal robot platform that is under development currently.
\begin{figure}[h]
	\hspace{0.2cm}
	\center
	\includegraphics[width=0.95 \columnwidth]{Figures/RTCoM2_final.pdf}
	\caption{{\bfseries Plots of Body CoM Position and Velocity in Rough Terrain Simulation.}  }
	\label{fig:CoMposVel}
\end{figure}
\begin{figure}[h]
	\center
	\includegraphics[width=0.88 \columnwidth]{Figures/RTCoMRot4_final.pdf}
	\caption{{\bfseries Plots of Robot Orientation in Rough Terrain Simulation. }  }
	\label{fig:CoMeulAng}
\end{figure}
%\begin{figure}
%	\hspace{0.2cm}
%	\center
%	\includegraphics[width=1 \columnwidth]{Figures/RTJoint_final.pdf}
%	\caption{{\bfseries Plots of Joint Position and Velocity  in Rough Terrain Simulation. }  \todo{Remove this}}
%	\label{fig:jointPos}
%\end{figure}
\begin{figure}[!h]
	\center
	\includegraphics[width=0.92 \columnwidth]{Figures/MPCForce3_final.pdf}
	\caption{{\bfseries Plots of MPC Force and Moment in Rough Terrain Simulation.  }}
	\label{fig:MPCforce}
\end{figure}
\begin{figure}%[!h]
	\hspace{0.2cm}
	\center
	\includegraphics[width=0.92 \columnwidth]{Figures/MPCTorque3_final.pdf}
	\caption{{\bfseries Plots of Joint Torques in Rough Terrain Simulation.  } }
	\label{fig:MPCtauRT}
\end{figure}

\subsection{Bipedal Hopping}
On top of the rotation and walking simulations presented earlier in this section, we have also implemented other gaits such as hopping. The hopping gait consists of a double support phase and a flight phase during the last quarter of each gait. 
% (we do not have flight phase in the previous one with trotting, so we should not mention this)With current MPC formulation, we decrease the flight phase duration in each gait cycle to mitigate the effects on performance during flight phase. 
A hopping gait illustration is shown in Fig. \ref{fig:boundGait}. It can be observed that during hopping motion, the robot is in a clear flight phase. 
%To validate the feasibility of hopping motion with the current MPC formulation, we test hopping forward motion with velocity $\bm v_{x_d}=0.5\:\unit{m/s}$. The CoM z-direction position and velocity plots are shown in Fig. \ref{fig:bounding_CoMposvel}. It is observed that the hopping motion can be performed with the current MPC formulation, with the trade-off of a certain degree of error in CoM velocity and height tracking. 
\update{This result validated that our proposed approach can work effectively for different dynamic locomotion on bipedal robots. We plan to optimize the MPC formulation in future work to enable faster and more aggressive motions. }

\begin{figure}%[!h]
	\center
	\includegraphics[width=1 \columnwidth]{Figures/boundGait.png}
	\caption{{\bfseries Illustration of Bipedal Hopping in Simulation  }  }
	\label{fig:boundGait}
\end{figure}
%\begin{figure}[!h]
%	\hspace{0.2cm}
%	\center
%	\includegraphics[width=1 \columnwidth]{Figures/BoundingCoM.pdf}
%	\caption{{\bfseries Plots of Z-direction CoM Location and Velocity in Hopping Simulation }  }
%	\label{fig:bounding_CoMposvel}
%\end{figure}

% Conclusion
\section{Conclusion}
We have presented a neural performance rendering system to generate high-quality geometry and photo-realistic textures of human-object interaction activities in novel views using sparse RGB cameras only. 
%
Our layer-wise scene decoupling strategy enables explicit disentanglement of human and object for robust reconstruction and photo-realistic rendering under challenging occlusion caused by interactions. 
%
Specifically, the proposed implicit human-object capture scheme with occlusion-aware human implicit regression and human-aware object tracking enables consistent 4D human-object dynamic geometry reconstruction.
%
Additionally, our layer-wise human-object rendering scheme encodes the occlusion information and human motion priors to provide high-resolution and photo-realistic texture results of interaction activities in the novel views.
%
Extensive experimental results demonstrate the effectiveness of our approach for compelling performance capture and rendering in various challenging scenarios with human-object interactions under the sparse setting.
%
We believe that it is a critical step for dynamic reconstruction under human-object interactions and neural human performance analysis, with many potential applications in VR/AR, entertainment,  human behavior analysis and immersive telepresence.





\balance
\bibliographystyle{ieeetr}
\bibliography{reference}

% Document end
\end{document}
