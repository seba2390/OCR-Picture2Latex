\subsection{The  affine annular algebra over the weight set $ \Lambda $ indexed by $ G $}{$  $}

Let $ \mcal A $ denote the affine annular algebra of $ \mcal C_{NN} $ with respect to $ G $ which indexes the weight set $ \Lambda $.
In our set up, the indexing set $ G $ is more important rather than the set $ \Lambda $; for instance, $ X_h $ and $ X_e $ are identical in $ \Lambda \subset \t{ob} (\mcal C_{NN}) $ when $ h\in H $.

We will recall the definition of $ \mcal A $ here.
For $ g_1,g_2 \in G $, we have a vector space $ \mcal A_{g_1,g_2}$ which is the quotient of the vector space $\us {W\in \t {ob} (\mcal C_{NN})} \bigoplus  \mcal C_{NN} \left(X_{g_1} \us N \otimes W \; , \; W \us N \otimes X_{g_2} \right)$ over the subspace generated by elements of the form $ \left[a \circ (\t {id}_{X_{g_1}} \us N \otimes f ) -  (f \us N \otimes \t {id}_{X_{g_2}}  ) \circ a \right]$ for $ a \in \mcal C_{NN} \left(X_{g_1} \us N \otimes Z \; , \; W \us N \otimes X_{g_2} \right)$ and $ f \in \mcal C_{NN} (W,Z) $.
We denote the quotient map by $ \psi_{g_1 , g_2} $. We will also use the notation $ \psi^W_{g_1,g_2} $ (resp., $ \psi^s_{g_1,g_2} $) for the restriction map $ \left. \us {}{\psi_{g_1,g_2}} \right|_{\mcal C_{NN} \left(X_{g_1} \us N \otimes W \; , \; W \us N \otimes X_{g_2} \right)} $ (resp., $ \left. \us {} {\psi_{g_1,g_2}} \right|_{\mcal C_{NN} \left(X_{g_1} \us N \otimes X_s \; , \; X_s \us N \otimes X_{g_2} \right)} $ for $ s\in G $).
Further, $ \mcal A^W_{g_1,g_2} $ and $ \mcal A^s_{g_1,g_2} $ will denote the range of the maps $ \psi^W_{g_1,g_2} $ and $ \psi^s_{g_1,g_2} $ respectively.

\noindent\textbf{Notation.} For any two vectors $ v_1 $ and $ v_2 $ in any vector space, we will write $ v_1 \sim v_2 $ when $ \t{span } v_1 = \t{span } v_2 $.% $ v_1 $ and $ v_2 $ are `mutually' linearly dependent on each other.
\begin{prop}
$ \mcal A_{g_1,g_2} $ is linearly spanned by the set
$
\left\{
\left. \psi^s_{g_1,g_2} \left( \; \raisebox{-4.4em}{\hexagon{$ g_1 $}{$ s $}{$e $}{$ s \:$}{$ g_2 $}{$ h_2 $}{$ h_1 $}{$e  $}{\psfrag{9}{$ g_1 s $}\psfrag{0}{$ s h_2 g_2 $}}} \hspace{.5em} \right) \right|
h_1,h_2 \in H,s\in G \right\}
$.
\end{prop}
\noindent We denote the above element by $ a(h_1 , g_1 , s , h_2 , g_2) $.
Note that $ h_1 g_1  = s \; h_2 g_2 \; s^{-1} $.
\begin{proof}
Since the weight set $\Lambda = \{X_s : s \in G\} $ is full, we may use the relation satisfied by the quotient map $ \psi_{g_1,g_2} $ to say  $ \mcal A_{g_1,g_2} = \t{span} \left(\us {s \in G} \cup \mcal A^s_{g_1,g_2}\right)$.
So, by Remark \ref{hexagon}, $ \mcal A_{g_1 , g_2} $ can be linearly generated by elements of the form $ \psi^s_{g_1,g_2} \left( \; \raisebox{-4.4em}{\hexagon{$ g_1 $}{$ s $}{$h_3 $}{$ s \:$}{$ g_2 $}{$ h_4 $}{$ h_1 $}{$h_2  $}{\psfrag{9}{$ g_1 h_3 s $}\psfrag{0}{$ s h_4 g_2 $}}} \hspace{.5em} \right) $ for $ h_1,h_2,h_3,h_4 \in H, s\in G $.

Using Proposition \ref{smbox*alg}, we may write \hspace{1em} \raisebox{-2.7 em}
{\psfrag{1}{\reflectbox{$h^{-1}_1 $}}
	\psfrag{2}{$ h_1 g h_2 $}
	\psfrag{3}{$ g $}
	\psfrag{4}{$ h_2 $}
	\psfrag{5}{\reflectbox{$h_1 $}}
	\psfrag{6}{$ g $}
	\psfrag{7}{$ h^{-1}_2 $}
	\includegraphics[scale=0.2]{figures/cnn/2disc.eps}} \hspace{1.5em}
%	 $\raisebox{-1.5em} {$ \os {\displaystyle \smbox{h^{-1}_1}{h_1 g h_2}{g}{h_2}{4}{2}{-0.3}}  {\smbox{\; \; \;\;h_1}{g}{}{h^{-1}_2}{4}{2}{-0.3}}  $} 
$\sim$ \raisebox{-1.5 em}{\1disc{$e $}{$ g $}{$ g $}{$ e $}} 
%\; \smbox{e}{g}{g}{e}{2}{2}{-0.3}
$ = \t{id}_{X_g}$.
Again, using the relation satisfied by the quotient map and setting $ t:= h_3 s h^{-1}_2 $, we get
\[
\psi^s_{g_1,g_2} \left( \; \raisebox{-4.4em}{\hexagon{$ g_1 $}{$ s $}{$h_3 $}{$ s \:$}{$ g_2 $}{$ h_4 $}{$ h_1 $}{$h_2  $}{\psfrag{9}{$ g_1 h_3 s $}\psfrag{0}{$ s h_4 g_2 $}}} \hspace{.5em} \right) 
%\psi^s_{g_1,g_2} \left( \ous {\displaystyle \triup{s}{g_2}{ s h_4 g_2  }{3}{3}{1}} {\displaystyle \smbox{h_1}{}{}{h_2}{2}{2}{-0.3}} {\displaystyle \tridown{g_1}{s}{g_1h_3s}{3}{3}{-0.5}}\right)
\; \sim \;
\psi^t_{g_1,g_2} \left( \left( \raisebox{-1.5 em}{\1disc{$h_3 $}{$ s $}{$ t $}{$ h_2 $}}
%  \smbox{h_3}{s}{t}{h_2}{2}{2}{-0.3}
\hspace{1em} \us N \otimes \t{id}_{X_{g_2}} \right) \circ \;
\raisebox{-4.4em}{\hexagon{$ g_1 $}{$ s $}{$h_3 $}{$ s\: $}{$ g_2 $}{$ h_4 $}{$ h_1 $}{$h_2  $}{\psfrag{9}{$ g_1 h_3 s $}\psfrag{0}{$ s h_4 g_2 $}}} \hspace{1em}
%\left(\ous{\displaystyle \triup{s}{g_2}{ s h_4 g_2  }{3}{3}{1}}{\displaystyle \smbox{h_1}{}{}{h_2}{2}{2}{-0.3}}{\displaystyle \tridown{g_1}{s}{g_1h_3s}{3}{3}{-0.5}}\right)
\circ \left(\t{id}_{X_{g_1}} \us N \otimes \hspace{1em} \raisebox{-1.5 em}{\1disc{$h^{-1}_3 $}{$ t $}{$ s $}{$ h^{-1}_2 $}} %\smbox{h^{-1}_3}{t}{s}{h^{-1}_2}{2}{2}{-0.3}
\right) \right).
\]
We then apply Lemma \ref{tribox} (i) and (ii) to get
\[
\psi^s_{g_1,g_2} \left( \; \raisebox{-4.4em}{\hexagon{$ g_1 $}{$ s $}{$h_3 $}{$ s \:$}{$ g_2 $}{$ h_4 $}{$ h_1 $}{$h_2  $}{\psfrag{9}{$ g_1 h_3 s $}\psfrag{0}{$ s h_4 g_2 $}}} \hspace{.5em} \right)
%\psi^s_{g_1,g_2} \left( \ous{\displaystyle \triup{s}{g_2}{ s h_4 g_2  }{3}{3}{1}}{\displaystyle \smbox{h_1}{}{}{h_2}{2}{2}{-0.3}}{\displaystyle \tridown{g_1}{s}{g_1h_3s}{3}{3}{-0.5}}\right)
\; \sim \; \psi^t_{g_1,g_2} \left(
\raisebox{-1.5em}{\wine{$ t $}{$ g_2 $}{$ th_2 h_4g_2 $}{\!\!$ h_2h_4 $}}
%\triup{t}{g_2}{ t h_2h_4 g_2 \; \;\;  }{3}{3}{1}
\; \circ  \hspace{4em}
\raisebox{-3.85 em}
{\psfrag{1}{\reflectbox{$h_3 $}}
	\psfrag{2}{$ h^{-1}_3 t h_2 h_4 g_2$}
	\psfrag{3}{$ t h_2 h_4 g_2 $}
	\psfrag{4}{$ e $}
	\psfrag{5}{\reflectbox{$h_1 $}}
	\psfrag{6}{$ g_1 t h_2 $}
	\psfrag{7}{$ h_2 $}
	\psfrag{8}{\reflectbox{$ e\: $}}
	\psfrag{9}{$ h^{-1}_2 $}
	\psfrag{0}{$ g_1 t$}
	\includegraphics[scale=0.2]{figures/abh/3disc.eps}} \hspace{4em}
%\left(\ous{\displaystyle \smbox{h_3}{h^{-1}_3 t h_2h_4 g_2}{t h_2h_4 g_2}{e}{6}{2}{-0.3}}{\displaystyle \smbox{\; \; \;h_1}{}{}{h_2}{6}{2}{-0.3}}{\displaystyle \smbox{\; \; \; \; \; \;e}{g_1t}{g_1 t h^{-1}_2}{h^{-1}_2}{6}{2}{-0.3}}\right)
\circ \; 
\raisebox{-1.5em}{\tower{$ g_1 $}{$ t $}{$ g_1t $}{$ e $}}
%\tridown{g_1}{t}{g_1t}{3}{3}{-0.5}
\right).
\]
where the three vertically stacked discs correspond to their composition.
Once we apply the multiplication of these discs as stated in Proposition \ref{smbox*alg} (i), it becomes clear that the resultant (up to a unit scalar) is indeed of the form mentioned in the statement of this proposition.
\end{proof}

We will next unravel the multiplication in $ \mcal A $.
\begin{rem}\label{multrem}
Multiplication of affine annular morphisms is given by
\begin{align*}
\psi^t_{g_2 ,g_3} (c) \circ \psi^s_{g_1 ,g_2} (d)\;  = \; \psi^{X_{s} \us N \otimes X_{t}}_{g_1 , g_3} \left( \left(\t{id}_{X_s} \us N \otimes c\right) \; \circ \; \left(d \us N  \otimes \t{id}_{X_t}\right)  \right)
\end{align*}
for $ c \in \mcal C_{NN} (X_{g_2} \otimes X_t \; , \; X_t \otimes X_{g_3}) $ and $ d \in \mcal C_{NN} (X_{g_1} \otimes X_s \; , \; X_s \otimes X_{g_2}) $.
Using Proposition \ref{resid}, we can rewrite the above as
\begin{align*}
&\;\psi^t_{g_2 ,g_3} (c) \circ \psi^s_{g_1 ,g_2} (d)\\
=& \; \us {h \in H} \sum \psi^{sht}_{g_1 , g_3}
\left( \raisebox{-1.5em}{\tower{$ s $}{$ t $}{$ sht $}{$ h $}}
%\tridown{s}{t}{sht}{3}{3}{-0.5} \; 
\us N \otimes  \t{id}_{X_{g_3}}  \; \circ \; \left(\t{id}_{X_s} \us N \otimes c\right) \; \circ \; \left(d \us N  \otimes \t{id}_{X_t}\right) \; \circ \; \t{id}_{X_{g_1}} \us N \otimes \; 
\raisebox{-1.5em}{\wine{$ s $}{$ t $}{$ sht $}{$ h $}}
%\triup{s}{t}{sht\; \; \;}{3}{3}{1}
\right).
\end{align*}
\end{rem}
\begin{prop}\label{mult}
\begin{align*}
& \left[\omega (h''_2, g_2, t)  \omega (t,h_3,g_3) \;  a(h''_2 , g_2 , t , h_3 , g_3)\right]
\; \circ \;
\left[ \omega (h_1 , g_1 , s) \omega (s, h'_2, g_2) \; a(h_1 , g_1 , s , h'_2 , g_2)\right]\\
= & \; \delta_{h'_2 = h''_2} \; \left[ \omega (s,t,h_3g_3) \ol \omega (s, h'_2 g_2, t)   \omega (h_1g_1 , s, t)    \right] \; \left[\omega (h_1 , g_1 , s t)\; \omega (st,h_3,g_3) \; a(h_1 , g_1 , st , h_3 , g_3)\right]
\end{align*}
\end{prop}
\begin{proof}
The above remark lets us express the element $ \left[a(h''_2 , g_2 , t , h_3 , g_3) \; \circ \; a(h_1 , g_1 , s , h'_2 , g_2)\right] $ as a sum over $ h\in H $ of
\begin{equation*}
\psi^{sht}_{g_1 , g_3}
\left( \us {\displaystyle =: b_h \t{ say}} {\underbracket{
\raisebox{-1.5em}{\tower{$ s $}{$ t $}{$ sht $}{$ h $}}
%\tridown{s}{t}{sht}{3}{3}{-0.5} \;
\us N \otimes  \t{id}_{X_{g_3}}  \; \circ
\; \left(  \t{id}_{X_s}  \us N \otimes \hspace{1em} 
\raisebox{-4.4em}{\hexagon{$ g_2 $}{$ t $}{$e $}{$ t\: $}{$ g_3 $}{$ h_3 $}{$ h''_2 $}{$e  $}{\psfrag{9}{$ g_2 t$}\psfrag{0}{$ t h_3 g_3 $}}}
%\ous{\displaystyle \triup{t}{g_3}{ t h_3 g_3  }{3}{3}{1}}{\displaystyle \smbox{\; h''_2}{}{}{e \;\;\;}{2}{2}{-0.3}}{\displaystyle \tridown{g_2}{t}{g_2 t}{3}{3}{-0.5}}
\; \right) \; \circ \; \left( \;
\raisebox{-4.4em}{\hexagon{$ g_1 $}{$ s $}{$e $}{$ s \:$}{$ g_2 $}{$ h'_2 $}{$ h_1 $}{$ e $}{\psfrag{9}{$ g_1 s $}\psfrag{0}{$ s h'_2 g_2 $}}}
%\ous{\displaystyle \triup{s}{g_2}{ s h'_2 g_2  }{3}{3}{1}}{\displaystyle \smbox{h_1}{}{}{e \;\;\;}{2}{2}{-0.3}}{\displaystyle \tridown{g_1}{s}{g_1s}{3}{3}{-0.5}}
\hspace{1em} \us N \otimes \t{id}_{X_t} \right) \; \circ \; \t{id}_{X_{g_1}} \us N \otimes
\raisebox{-1.5em}{\wine{$ s $}{$ t $}{$ sht $}{$ h $}}
%\; \triup{s}{t}{sht\; \; \;}{3}{3}{1}
}}
\right).
\end{equation*}
%Let us rename the expression inside $ \psi^{sht}_{g_1,g_3} $ as $ b_h $.
We could use Lemma \ref{2tri} at three instances in the above expression of $ b_h $, and thereby we may rewrite $ b_h $ up to a unit scalar as
\begin{align}\label{step1}
	\hspace{1em}
\raisebox{-3em}{
	\psfrag{1}{\reflectbox{$ sht $}}
	\psfrag{2}{$ g_3 $}
	\psfrag{3}{\hspace{-3.5em}$shth_3 g_3 $}
	\psfrag{4}{$ h_3 $}
	\psfrag{5}{\reflectbox{$ s $}}
	\psfrag{6}{$ t h_3 g_3 $}
	\psfrag{7}{$ h $}
	\includegraphics[scale=0.2]{figures/cnn/winetower.eps}}
\hspace{2em} \circ \left(\t{id}_{X_s} \us N \otimes \;
\raisebox{-1.5 em}{\1disc{$h''_2 $}{$ g_2 t$}{$ t h_3 g_3 $}{$ e $}}\;\;
\right) \circ \hspace{1.5em}
\raisebox{-3em}{
	\psfrag{1}{\reflectbox{$ s $}}
	\psfrag{2}{$ g_2 t $}
	\psfrag{3}{\hspace{-3em}$s h'_2 g_2 t $}
	\psfrag{4}{$ h'_2 $}
	\psfrag{5}{\reflectbox{$ s h'_2 g_2 $}}
	\psfrag{6}{$ t  $}
	\psfrag{7}{$ e $}
	\includegraphics[scale=0.2]{figures/cnn/winetower.eps}}\hspace{1em}
\circ \left(
\raisebox{-1.5 em}{\1disc{$h_1 $}{$ g_1 s$}{$s h'_2 g_2 $}{$ e $}} \;
\us N \otimes \t{id}_{X_t} \right)\circ
\raisebox{-3em}{
	\psfrag{1}{\reflectbox{$ g_1s $}}
	\psfrag{2}{$ t $}
	\psfrag{3}{$g_1 sht $}
	\psfrag{4}{$ h $}
	\psfrag{5}{\reflectbox{$ g_1 $}}
	\psfrag{6}{$ sht $}
	\psfrag{7}{$ e $}
	\includegraphics[scale=0.2]{figures/cnn/winetower.eps}}
%%%%%%%%%%%%%%%%%%%%%%%%%%%
\comments{
\\
&\triup{sht}{g_3}{shth_3 g_3}{3}{3}{1} \; \circ \;
\left(  
 \ous
{\! \! \! \!   \displaystyle \tridown{s}{t h_3 g_3}{\! \! \! \! \! \! shth_3 g_3}{3}{3}{-0.5} }
{\t{id}_{X_s}  \us N \otimes \displaystyle \smbox{h''_2}{g_2 t}{t h_3 g_3}{e}{4}{2}{-0.3}}
{\! \! \! \! \! \! \! \! \! \! \! \!  \displaystyle \triup{s}{g_2 t}{\! \! \! sh'_2 g_2 t}{3}{3}{1}} 
\right) \; \circ \; \left( \ous
{\; \; \; \; \; \; \; \; \;  \displaystyle \tridown{s h'_2 g_2}{t}{\; \; \; \; \; \; \; sh'_2 g_2 t}{3}{3}{-0.5}}
{\displaystyle \smbox{h_1}{g_1 s}{sh'_2 g_2}{e \;\;\;}{4}{2}{-0.3} \us N \otimes \t{id}_{X_t}}
{\; \; \; \; \; \; \; \; \; \; \; \; \;  \displaystyle \triup{g_1 s}{t}{g_1 s h t}{3}{3}{1}} \right) \; \circ \; \tridown{g_1}{sht}{g_1 s h t}{3}{3}{-0.5}
}
%%%%%%%%%%%%%%%%%%%%%%%%%%%
\hspace{1.5em}.
\end{align}

In the above expression \ref{step1}, using Lemma \ref{tribox} (ii), we could make the disc in the fourth term pass through the bottom box in the third term to its top.
As a result, expression \ref{step1} turns out to be a scalar multiple of
\begin{align}\label{step2}
	\hspace{1em}
	\raisebox{-3em}{
		\psfrag{1}{\reflectbox{$ sht $}}
		\psfrag{2}{$ g_3 $}
		\psfrag{3}{\hspace{-3.5em}$shth_3 g_3 $}
		\psfrag{4}{$ h_3 $}
		\psfrag{5}{\reflectbox{$ s $}}
		\psfrag{6}{$ t h_3 g_3 $}
		\psfrag{7}{$ h $}
		\includegraphics[scale=0.2]{figures/cnn/winetower.eps}}
	\hspace{2em} \circ \left(\t{id}_{X_s} \us N \otimes \;
	\raisebox{-1.5 em}{\1disc{$h''_2 $}{$ g_2 t$}{$ t h_3 g_3 $}{$ e $}}\;\;
	\right) \circ\hspace{2em}
\raisebox{-4.4em}{\hexagon{$ g_1s $}{$ t $}{$e $}{$ s $}{$ g_2t $}{$ h'_2 $}{$ h_1 $}{$e  $}{\psfrag{9}{\hspace{-2em}$ g_1 st $}\psfrag{0}{\hspace{-3em}$ sh'_2g_2t $}}}
\hspace{1em}	\circ \;
	\raisebox{-3em}{
		\psfrag{1}{\reflectbox{$ g_1s $}}
		\psfrag{2}{$ t $}
		\psfrag{3}{$g_1 sht $}
		\psfrag{4}{$ h $}
		\psfrag{5}{\reflectbox{$ g_1 $}}
		\psfrag{6}{$ sht $}
		\psfrag{7}{$ e $}
		\includegraphics[scale=0.2]{figures/cnn/winetower.eps}}
	\hspace{1.5em}.
\end{align}
Observe that the composition of the bottom box of the third term and the top box in the fourth term (in expression \ref{step2}) is $ \t{id}_{X_{g_1 st}} $ if $ h=e $ and zero otherwise; this follows from Proposition \ref{resid}.
Similarly, Lemma \ref{tribox} (i) allows us to move the disc in the second term up through the bottom box of the first term, and thereby the expression \ref{step2} becomes a scalar multiple of
\begin{align}\label{step3}
	\hspace{1em}
	\raisebox{-3em}{
		\psfrag{1}{\reflectbox{$ sht $}}
		\psfrag{2}{$ g_3 $}
		\psfrag{3}{\hspace{-8.75em}($s h h''_2g_2 t = $)$shth_3 g_3 $}
		\psfrag{4}{$ h_3 $}
		\psfrag{5}{\reflectbox{$ s $}}
		\psfrag{6}{$ g_2 t $}
		\psfrag{7}{\!$ h h''_2 $}
		\includegraphics[scale=0.2]{figures/cnn/winetower.eps}}
	\hspace{1em} \circ\hspace{2em}
	\raisebox{-4.4em}{\hexagon{$ g_1s $}{$ t $}{$e $}{$ s $}{$ g_2t $}{$ h'_2 $}{$ h_1 $}{$e  $}{\psfrag{9}{\hspace{-2em}$ g_1 st $}\psfrag{0}{\hspace{-3em}$ sh'_2g_2t $}}}
	\hspace{1em}	\circ \;
	\raisebox{-3em}{
		\psfrag{1}{\reflectbox{$ g_1s $}}
		\psfrag{2}{$ t $}
		\psfrag{3}{$g_1 sht $}
		\psfrag{4}{$ h $}
		\psfrag{5}{\reflectbox{$ g_1 $}}
		\psfrag{6}{$ sht $}
		\psfrag{7}{$ e $}
		\includegraphics[scale=0.2]{figures/cnn/winetower.eps}}
	\hspace{1.5em}.
\end{align}
Again, by Proposition \ref{resid}, the composition of the bottom box of the first term and the top of the second term in expression \ref{step3} is $ \t {id}_{X_{shth_3 g_3}} $ if $ h h''_2 = h'_2 $ and zero otherwise.
%%%%%%%%%%%%%%%%%%%%%%%%%%%
\comments{
the third term in the above expression turns out to be a unit scalar times 
$ \ous
{\displaystyle \smbox{h_1}{g_1st}{sh'_2 g_2 t}{e}{4}{2}{-0.3}}
{\displaystyle \tridown{g_1s}{t}{}{3}{3}{-0.5}}
{\displaystyle \triup{g_1s}{t}{g_1s ht}{3}{3}{1}}
\t{ which is zero by Proposition \ref{resid} unless } h = e $.
So, we will assume $ h=e $.
Treating the second term in the same way, we will get that it is zero unless $ h'_2 = h''_2 $.
}%%%%%%%%%%%%%%%%%%%%%%%%%%%

We now consider the case $ h=e $ and $ h'_2 = h''_2$ ($ = h_2$ say).
The above discussion implies that in this case, $ \left[a(h_2 , g_2 , t , h_3 , g_3) \; \circ \; a(h_1 , g_1 , s , h_2 , g_2)\right] $ is indeed a scalar multiple of $ a(h_1 , g_1 , st , h_3 , g_3) $.
So, we need to gather the $ 3 $-cocycle arising at various steps.
To obtain step \ref{step1}, Lemma \ref{2tri} will give the following six scalars
\[
\left[ \omega (s,t,h_3) \omega (s,th_3 , g_3)\right]\;
\left[\cancel{\ol \omega (sh_2,g_2,e)} \ol \omega (sh_2,g_2,t) \right]\;
\left[ \cancel{\omega (g_1 , s ,e)} \omega (g_1 , s ,t) \right].
\]
Application of Lemma \ref{tribox} (ii) (resp., (i)) while obtaining step \ref{step2} (resp., \ref{step3}) from step \ref{step1} (resp., \ref{step2}), yield
\[
\left[ \omega(h_1 , g_1 s, t) \cancel{\ol \omega (sh_2 g_2 ,e ,e)} \cancel{\ol \omega (h_1, g_1 s ,e)} \right] \;
%  \; \smbox{h_1}{g_1 s t}{sh_2g_2t}{e}{4}{2}{-0.3}
%\]
%\[
\t{ (resp., } \left[ \cancel{\omega (s,th_3g_3,e)} \ol \omega (s, h_2, g_2 t) \cancel{\omega (s, e, h_2)} \right]
% \; \smbox{e}{sh_2g_2t}{sth_3g_3}{e}{4}{2}{-0.3}
\t{ ).}
\]
%%%%%%%%%%%%%%%%%%%%%%%%%%%%%%%%%%%%%%%
\comments{
Applying Lemma \ref{tribox} (i) (resp., (ii)) and isometry property of upward pointing triangles, the second (resp., third) term of step \ref{step1} becomes 
\[
\left[ \cancel{\omega (s,th_3g_3,e)} \ol \omega (s, h_2, g_2 t) \cancel{\omega (s, e, h_2)} \right] \; \smbox{e}{sh_2g_2t}{sth_3g_3}{e}{4}{2}{-0.3}
\]
\[
\t{(resp., } \left[ \omega(h_1 , g_1 s, t) \cancel{\ol \omega (sh_2 g_2 ,e ,e)} \cancel{\ol \omega (h_1, g_1 s ,e)} \right]  \; \smbox{h_1}{g_1 s t}{sh_2g_2t}{e}{4}{2}{-0.3} \t{ ).}
\]
The product of the above two boxes is $\; \smbox{h_1}{g_1 s t}{sth_3g_3}{e}{4}{2}{-0.3} $.
}%%%%%%%%%%%%%%%%%%%%%%%%%%%%%%%%%%%%%%%
Thus, we obtained the equation
\begin{align*}
& \; a(h_2 , g_2 , t , h_3 , g_3) \; \circ \; a(h_1 , g_1 , s , h_2 , g_2)\\
= & \; \left[ \omega (s,t,h_3) \omega (s,th_3 , g_3) \ol \omega (sh_2,g_2,t) \omega (g_1 , s ,t) \omega(h_1 , g_1 s, t)  \ol \omega (s, h_2, g_2 t)   \right]
a(h_1 , g_1 , st , h_3 , g_3).
\end{align*}
We will be done with the proof once we match the scalars.
Applying the $ 3 $-cocycle relation \ref{3coc} on first and second, third and sixth, fourth and fifth terms separately, we get
\begin{align*}
&\left[ \omega (st,h_3,g_3) \omega (s,t,h_3g_3) \ol \omega (t,h_3,g_3) \right] \; 
\left[ \ol \omega (s, h_2, g_2) \ol \omega (s, h_2 g_2, t) \ol \omega (h_2, g_2, t) \right]\\
&\left[ \omega (h_1g_1 , s, t) \omega (h_1 , g_1 , s t) \ol  \omega (h_1 , g_1 , s)  \right]
\end{align*}
\end{proof}
\noindent \textbf{Notation.} We see that $ \left[ \omega (h_1 , g_1 , s) \omega (s, h_2, g_2) \; a(h_1 , g_1 , s , h_2 , g_2)\right] $ is better behaved with respect to multiplication than $ a(h_1 , g_1 , s , h_2 , g_2)$.
So, we set\\
$ A(h_1 , g_1 , s , h_2 , g_2) := \left[ \omega (h_1 , g_1 , s) \omega (s, h_2, g_2) \; a(h_1 , g_1 , s , h_2 , g_2)\right] $ and the above proposition translates as:\\
$A(h''_2 , g_2 , t , h_3 , g_3) \; \circ \; A(h_1 , g_1 , s , h'_2 , g_2) \; =  \; \delta_{h'_2 = h''_2} \; \left[ \omega (s,t,h_3g_3) \ol \omega (s, h'_2 g_2, t)   \omega (h_1g_1 , s, t)    \right] \; A(h_1 , g_1 , st , h_3 , g_3)$.
\vskip 2 em
Next we will compute the canonical trace $ \Omega $ on $ \mcal A_{g,g} $ for $ g\in G $.
For this, we need orthonormal basis of $ \mcal C_{NN}  ( {}_N L^2 (N) {}_N , X_s) $ for $ s \in G $ with respect to the inner product given by
\[
 \mcal C_{NN}  ( {}_N L^2 (N) {}_N , X_s)  \times  \mcal C_{NN}  ( {}_N L^2 (N) {}_N , X_s)  \ni (c,d) \longmapsto d^* \circ c \in \C.
\]
By Proposition \ref{smbox}, $ \mcal C_{NN}  ( {}_N L^2 (N) {}_N , X_s) $ is zero unless $ s \in H $.
Now, $ X_e = X_h $ for all $ h\in H $.
Since $ N \subset Q $ is irreducible, the space $  \mcal C_{NN}  ( {}_N L^2 (N) {}_N , X_h)  $  is one-dimensional and spanned by the element the inclusion map $ \hat 1 \os {\displaystyle {\iota_h}} \longmapsto [1]_h $.
$ \iota^*_h $ is simply the conditional expectation $ E_N $.

The definition of $ \Omega $ then turns out to be (following \cite{GJ})
\[
\mcal A_{g,g} \ni \psi^s_{g,g} (c ) \os {\displaystyle \Omega} \longmapsto \us {s\in H} \sum \; R^*_g \circ \left(\t{id}_{X_{g^{-1}}} \us N \otimes \; \iota^*_s \; \us N \otimes \t{id}_{X_g}\right) \circ \left(\t{id}_{X_{g^{-1}}} \us N \otimes c\right) \circ \left(\t{id}_{X_{g^{-1}} \us N \otimes X_g} \; \us N \otimes  \iota_s\right) \circ R_g \; \in \C.
\]
\begin{prop}
$ \Omega \left(A(h_1, g,  s , h_2 , g)\right)  = \delta_{h_1 = h_2} \; \delta_{s=e} $.
\end{prop}
\begin{proof}
For $ h\in H $, we need to compute the scalar
\begin{align*}
& \; R^*_g \circ \left(\t{id}_{X_{g^{-1}}} \us N \otimes \; \iota^*_h \; \us N \otimes \t{id}_{X_g}\right) \circ \left(\t{id}_{X_{g^{-1}}} \us N \otimes \hspace{1em}
\raisebox{-4.4em}{\hexagon{$ g $}{$ h $}{$e $}{$ h $}{$ g $}{$ h_2 $}{$ h_1 $}{$ e $}{\psfrag{9}{$ g h $}\psfrag{0}{$ hh_2g $}}}
%\left( \ous{\displaystyle \triup{h}{g}{\!\!\!\!\! h h_2 g  }{3}{3}{1}}{\displaystyle \smbox{h_1}{}{}{e \;\;\;}{2}{2}{-0.3}}{\!\!\displaystyle \tridown{g_1}{h}{g_1h}{3}{3}{-0.5}}\right)
\; \right) \circ \left(\t{id}_{X_{g^{-1}} \us N \otimes X_g} \; \us N \otimes  \iota_h\right) \circ R_g \; (\hat 1)
\end{align*}
\begin{align*}
= & \; \us i \sum R^*_g \circ \left(\t{id}_{X_{g^{-1}}} \us N \otimes \; \iota^*_h \; \us N \otimes \t{id}_{X_g}\right) \circ \left(\t{id}_{X_{g^{-1}}} \us N \otimes
 \hspace{1em}
 \raisebox{-4.4em}{\hexagon{$ g $}{$ h $}{$e $}{$ h $}{$ g $}{$ h_2 $}{$ h_1 $}{$ e $}{\psfrag{9}{$ g h $}\psfrag{0}{$ hh_2g $}}}
%\left( \ous{\displaystyle \triup{h}{g}{\!\!\!\!\! h h_2 g  }{3}{3}{1}}{\displaystyle \smbox{h_1}{}{}{e \;\;\;}{2}{2}{-0.3}}{\!\!\displaystyle \tridown{g_1}{h}{g_1h}{3}{3}{-0.5}}\right)
\;\right) \left[ [u^* (g^{-1} , g) \alpha_{g^{-1}} (b_i)]_{g^{-1}} \us N \otimes [b^*_i]_g \us N \otimes [1]_h \right]\\
= & \; \abs {H}^{-1} \us {i,j} \sum R^*_g \circ \left(\t{id}_{X_{g^{-1}}} \us N \otimes \; \iota^*_h \; \us N \otimes \t{id}_{X_g}\right)\\
& \;\;\;\;\;\;\;\;\;\;\;\;\;\;\;\;  \left[ [u^* (g^{-1} , g) \alpha_{g^{-1}} (b_i)]_{g^{-1}} \us N \otimes [\alpha_{h_1} \left(b^*_i u(g,h) \right) \; u(h_1 , gh) \; u^*(h h_2 ,g) \; \alpha_h(b_j) ]_h \us N \otimes [\alpha_{h^{-1}_2} (b^*_j)]_g \right]\\
= & \; \abs {H}^{-1} \us {j} \sum R^*_g \left[ \left[u^* (g^{-1} , g) \alpha_{g^{-1}} \left( u(g,h) \; \alpha_{h^{-1}_1} \left(  u(h_1 , gh) \; u^*(h h_2 ,g) \alpha_{h} (b_j) \right) \right) \right]_{g^{-1}} \us N \otimes [\alpha_{h^{-1}_2} (b^*_j)]_g \right]\\
= & \; \abs {H}^{-1} \us {j} \sum u^* (g^{-1} , g) \alpha_{g^{-1}} \left( u(g,h) \; \alpha_{h^{-1}_1} \left(  u(h_1 , gh) \; u^*(h h_2 ,g) \alpha_{h} (b_j) \right) \alpha_{h^{-1}_2} (b^*_j) \right) u (g^{-1} , g)\\
= & \; \abs {H}^{-1} \us {j} \sum u^* (g^{-1} , g) \alpha_{g^{-1}} \left( u(g,h) \; \alpha_{h^{-1}_1} \left(  u(h_1 , gh) \; u^*(h h_2 ,g)  \right)  \right)\; \alpha_{g^{-1}} \left( \alpha_{h^{-1}_1 h } (b_j) \alpha_{h^{-1}_2} (b^*_j) \right) u (g^{-1} , g).
\end{align*}
Pulling the sum over the last term, we get $ \alpha_{g^{-1}} \left( \alpha_{h^{-1}_2} \left( \us {j} \sum \alpha_{h_2 h^{-1}_1 h} (b_j) b^*_j \right) \right) = \delta_{h=h_1 h^{-1}_2} \abs H$ (which is a standard fact in fixed-point subfactor of an outer action of finite group).
Let us assume $ h=h_1 h^{-1}_2 $.
But then, $ h_1 g h = h h_2 g$ will imply $ h_1 = h_2 $ and thereby $ h=e $.

Under the assumption $ h=e $ and $ h_1 = h_2 $, in the above expression, the term in between $ u^* (g^{-1} , g) $ and $ u (g^{-1} , g) $, becomes $ 1 $.
This gives the required result.
\end{proof}
\begin{cor}
The set $ \left\{ A(h_1, g_1, s, h_2,g_2) :h_1,h_2 \in H, s\in G \t{ such that } h_1g_1s =sh_2g_2 \right\} $ is a basis for $ \mcal A_{g_1,g_2} $.
\end{cor}
\begin{proof}
This easily follows from that $ \Omega  $ is non-degenerate on $ \mcal A $ (which is a consequence of $ \Omega $ being positive (see \cite{GJ})).
\end{proof}
\vskip 2em
We will now describe the $ * $-structure on $ \mcal A $ which we denote by $ \# $.
From \cite{GJ}, the definition of $ (\psi^s_{g_1 ,g_2} (c))^\# $ is the following:
\[
\psi^{s^{-1}}_{g_2 ,g_1} \left( \left( \t{id}_{X_{s^{-1}}} \us N \otimes \t{id}_{X_{g_1}} \us N \otimes \ol R^*_s \right) \circ \left(\t{id}_{X_{s^{-1}}} \us N \otimes c^*  \us N \otimes   \t{id}_{X_{s^{-1}}}\right)  \circ \left( R_s  \us N \otimes \t{id}_{X_{g_2}} \us N \otimes \t{id}_{X_{s^{-1}}} \right)\right) \in \mcal A_{g_2,g_1}.
\]
\begin{prop}\label{hash}
$ \left(A(h_1,g_1,s,h_2,g_2) \right)^\#$\\
\[
= \; \ol \omega (h_1 g_1 , s , s^{-1}) \; \omega (s , h_2 g_2 , s^{-1}) \; \ol \omega (s , s^{-1} ,h_1 g_1) \; A(h_2, g_2, s^{-1}, h_1,g_1).
\]
\end{prop}
\begin{proof}
Set $ A'(h_1,g_1,s,h_2,g_2) :=  \ol \omega (h_1 g_1 , s , s^{-1}) \; \omega (s , h_2 g_2 , s^{-1}) \; \ol \omega (s , s^{-1} ,h_1 g_1) \; A(h_2, g_2, s^{-1}, h_1,g_1)$.
Now, we get an inner product $ \lab \cdot , \cdot \rab' $ defined as
\[
\left \lab \; A(h_1,g_1,s,h_2,g_2) \; , \; A(h_3,g_3,t,h_4,g_4)\;  \right \rab' := \; \Omega \left( \; A'(h_3,g_3,t,h_4,g_4) \; A(h_1,g_1,s,h_2,g_2) \right)
\]
and extended linearly in the first and conjugate-linearly in the second variable.
In fact, the basis elements are orthonormal with respect to $ \lab \cdot , \cdot \rab' $.
Since $ \Omega \circ \# = \ol \Omega $ (by positivity of $ \Omega  $ (\cite{GJ})), it will be enough to prove $ \left(A(h_1,g_1,s,h_2,g_2) \right)^\# \sim  A'(h_1,g_1,s,h_2,g_2) $.
This is equivalent to proving $ \left(a(h_1,g_1,s,h_2,g_2) \right)^\# \sim a(h_2, g_2, s^{-1}, h_1,g_1) $.
This will follow from
\[
\us {\displaystyle A} {\underbracket
{ \t{id}_{X_{s^{-1}}} \us N \otimes \t{id}_{X_{g_1}} \us N \otimes \ol R^*_s }}
\;\; \circ \;\;
\us {\displaystyle B} {\underbracket
{\t{id}_{X_{s^{-1}}} \us N \otimes \hspace{1em}
\raisebox{-4.4em}{\hexagon{$ s $}{$ g_2 $}{$h_2 $}{$ g_1 $}{$ s $}{$ e $}{$ h^{-1}_1 $}{$ e $}{\psfrag{9}{$ sh_2 g_2 $}\psfrag{0}{$ g_1s $}}}
%\left[\ous{\; \; \; \displaystyle \triup{g_1}{s}{\! \! \! \! \! g_1s  }{3}{3}{1}}{\displaystyle \smbox{h^{-1}_1}{}{}{e}{2}{2}{-0.3}}{\; \; \; \; \; \; \; \displaystyle \tridown{s}{g_2}{\! \! \! s h_2g_2}{3}{3}{-0.5}}\right]
\hspace{1em}\us N \otimes   \t{id}_{X_{s^{-1}}}}}
\;\; \circ \;\;
\us {\displaystyle C} {\underbracket
{R_s  \us N \otimes \t{id}_{X_{g_2}} \us N \otimes \t{id}_{X_{s^{-1}}} }}
\hspace{1em} \sim \hspace{1em}
\raisebox{-4.4em}{\hexagon{$ g_2 $}{$ s^{-1} $}{$ e $}{$ s^{-1} $}{$ g_1 $}{$ h_1 $}{$ h_2 $}{$ e $}{\psfrag{9}{$ g_2 s^{-1} $}\psfrag{0}{$ s^{-1} h_1 g_1 $}}}
%\left[\ous{\displaystyle \triup{s^{-1}}{g_1}{\! \! \! \! \! \! \! s^{-1} h_1 g_1  }{3}{3}{1}}{\displaystyle \smbox{h_2}{}{}{e}{2}{2}{-0.3}}{\; \; \; \; \; \displaystyle \tridown{g_2}{s^{-1}}{g_2s^{-1}}{3}{3}{-0.5}}\right]
\]
The right side acting on $ [x]_{g_2} \us N \otimes [y]_{s^{-1}} $ gives (up to a nonzero scalar)
\begin{equation}\label{hash1}
\us i \sum
\left[
\alpha_{h_2} \left( x \alpha_{g_2} (y) \; u (g_2 ,s^{-1}) \right)\;
u(h_2,g_2 s^{-1}) u^* (s^{-1} h_1 ,g_1) u^* (s^{-1},  h_1) \alpha_{s^{-1}} (b_i)
\right]_{s^{-1}} \us N \otimes [\alpha_{h^{-1}_1} (b^*_i)]_{g_1}
\end{equation}
Next we compute the left side acting on $ [x]_{g_2} \us N \otimes [y]_{s^{-1}} $ (up to a nonzero scalar) in the following way
\begin{align*}
\os {\displaystyle C} \longmapsto & \us i \sum [u^* (s^{-1} , s) \alpha_{s^{-1}} (b_i) ]_{s^{-1}} \us N \otimes [ b^*_i]_s \us N \otimes [x]_{g_2} \us N \otimes [y]_{s^{-1}}
\\
\os {\displaystyle B} \longmapsto & \us {i,j} \sum [u^* (s^{-1} , s) \alpha_{s^{-1}} (b_i) ]_{s^{-1}}\\
& \us N \otimes \left[\alpha_{h^{-1}_1} \left( b^*_i \alpha_s (\alpha_{h_2} (x)) \; u(s,h_2) u(sh_2 , g_2) \; \right) u(h^{-1}_1 , sh_2 g_2) u^* (g_1 ,s ) \alpha_{g_1} (b_j) \right]_{g_1} \us N \otimes [b^*_j]_s \us N \otimes [y]_{s^{-1}}
\\
\os {\displaystyle A} \longmapsto & \us {i,j} \sum [u^* (s^{-1} , s) \alpha_{s^{-1}} (b_i) ]_{s^{-1}} \us N \otimes\\
& \left[\alpha_{h^{-1}_1} \left( b^*_i \alpha_s (\alpha_{h_2} (x)) \; u(s,h_2) u(sh_2 , g_2) \; \right) u(h^{-1}_1 , s h_2 g_2) u^* (g_1 ,s ) \alpha_{g_1} (b_j) \right]_{g_1}
E_N \left( b^*_j \alpha_s (y) u(s,s^{-1}) \right)
\\
\sim \;\; &  \us {i} \sum [u^* (s^{-1} , s) \alpha_{s^{-1}} (b_i) ]_{s^{-1}}\\
& \us N \otimes \left[\alpha_{h^{-1}_1} \left( b^*_i \alpha_s (\alpha_{h_2} (x)) \; u(s,h_2) u(sh_2 , g_2) \; \right) u(h^{-1}_1, sh_2 g_2) u^* (g_1 ,s ) \alpha_{g_1} \left(\alpha_s (y) u(s,s^{-1}) \right) \right]_{g_1}\\
= \;\; &  \us {i,k} \sum [u^* (s^{-1} , s) \alpha_{s^{-1}} (b_i) ]_{s^{-1}} \us N \otimes \\
& \! \! \! \! \! \! \! \left[ E_N \left(b^*_i \alpha_s (\alpha_{h_2} (x)) \; u(s,h_2) u(sh_2 , g_2) \alpha_{h_1} \left(   u(h^{-1}_1, sh_2 g_2) u^* (g_1 ,s ) \alpha_{g_1} \left(\alpha_s (y) u(s,s^{-1}) \right) \right) b_k \right) \alpha_{h^{-1}_1} (b^*_k) \right]_{g_1}\\
=  \;\; &  \us {k} \sum \left[u^* (s^{-1} , s) \alpha_{s^{-1}} \left(\alpha_s (\alpha_{h_2} (x)) \; u(s,h_2) u(sh_2 , g_2) \alpha_{h_1} \left(   u(h^{-1}_1, sh_2 g_2) u^* (g_1 ,s ) \alpha_{g_1} \left(\alpha_s (y) u(s,s^{-1}) \right) \right) b_k \right) \right]_{s^{-1}}\\
& \; \us N \otimes \left[ \alpha_{h^{-1}_1} (b^*_k) \right]_{g_1}\\
\end{align*}
Since the second tensor component matches with that of the expression in \ref{hash1}, we will now work with the first term.
\begin{align*}
& \; u^* (s^{-1} , s) \alpha_{s^{-1}} \left(\alpha_s (\alpha_{h_2} (x)) \; u(s,h_2) u(sh_2 , g_2) \alpha_{h_1} \left(   u(h^{-1}_1, sh_2 g_2) u^* (g_1 ,s ) \alpha_{g_1} \left(\alpha_s (y) u(s,s^{-1}) \right) \right) b_k \right)\\
= & \; \alpha_{h_2} (x) \; u^* (s^{-1} , s) \; \alpha_{s^{-1}} \left(  u(s,h_2) u(sh_2 , g_2) \alpha_{h_1} \left(   u(h^{-1}_1, sh_2 g_2) u^* (g_1 ,s ) \alpha_{g_1} \left(\alpha_s (y) u(s,s^{-1}) \right) \right)  \right) \alpha_{s^{-1}} (b_k)\\
\end{align*}
In the last expression, we pick $ y $ and using the intertwining relation between $ u $ and $ \alpha $, we push it leftwards all the way to the right side of the term $ \alpha_{h_2} (x) $ and it becomes $ \alpha_{h_2} (\alpha_{g_2} (y)) $.
This matches the first two and the last terms with that of the first tensor component of the expression \ref{hash1}.
We are left with showing the $ u $-terms in the middle, namely
\begin{equation}\label{hash2}
u^* (s^{-1} , s) \; \alpha_{s^{-1}} \left(  u(s,h_2) u(sh_2 , g_2) \alpha_{h_1} \left(   u(h^{-1}_1, sh_2 g_2) u^* (g_1 ,s ) \alpha_{g_1} \left( u(s,s^{-1}) \right) \right)  \right)
\end{equation}
is a nonzero multiple of the $ u $-terms in \ref{hash1}, that is,
\begin{equation}\label{hash3}
\alpha_{h_2} \left(u (g_2 ,s^{-1}) \right)\;
u(h_2,g_2 s^{-1}) u^* (s^{-1} h_1 ,g_1) u^* (s^{-1},  h_1)
\end{equation}
Taking the adjoint of \ref{hash2} and \ref{hash3} separately, we get the same automorphism $ \alpha_{h_2} \alpha_{g_2} \alpha_{s^{-1}} \alpha^{-1}_{g_1} \alpha^{-1}_{h_1} \alpha^{-1}_{s^{-1}}$.
Hence, we are done.
\end{proof}
\vskip 2em
In order to describe the representations of $ \mcal A $, we need a few more notations.
As in Section \ref{diag}, $ \mscr C $ will denote the set of conjugacy classes, $ g_C $ will be a representative of $ C \in \mscr C $  and for $ g\in C $, we pick $ w_g $ such that $ g=w_g g_C w^{-1}_g $.
Also, $ \vphi_C $ will be the $ 2 $-cocycle $ \ol \vphi_{g_C} $ of $ G_C $.
%Consider the equivalence relation $ \rho $ on $ H\times G $ defined by $ (h_1 , g_1) \; \rho \; (h_2,g_2) $ if and only if $ h_1 g_1 $ and $ h_2 g_2 $ are in the same conjugacy class.
%Let $\mscr S$ be the set of equivalence classes of $ H\times G $.
For $ C \in \mscr C $, set $ S_C:= \{ (h,g) \in H\times G : hg \in C\} $.
\begin{thm}\label{bhtubalg}
(i) The affine annular algebra $ \mcal A = (( \mcal A_{g_1 , g_2} ))_{\t {fin. supp.}}$% viewed as matrices with rows and columns indexed by $ G $ and entries coming from the spaces $\mcal A_{g_1,g_2}$ such that only finitely many entries are nonzero, 
is isomorphic as a $ * $-algebra to $\us {C \in \mscr C} \bigoplus M_{S_C} \otimes [\C G_C]_{\vphi_C}$ where $ M_{S_C} $ denotes the $ * $-algebra of finitely supported matrices with rows and columns indexed by elements of $ S_C $.\\
(ii) Every Hilbert space representation $ \pi : \mcal A \ra \mcal L (V) $ decomposes uniquely (up to isomorphism) as an orthogonal direct sum of submodules $ V^C := \left\lab \t{Range } \pi \left(a(e,g_C, e , e, g_C)\right) \right\rab$ for $ C \in \mscr C $.
(We will call a representation of $ \mcal A $ `supported on $ C \in \mscr C$' if it is generated by the range of the action of the projection $ a(e,g_C, e , e, g_C) $.)
The category of $ C $-supported representations of $ \mcal C $ is additively equivalent to representation category of $ [\C G_C]_{\vphi_C} $.


\end{thm}
\begin{proof}
(i) Define the map $ \Phi : \mcal A \lra  \us {C \in \mscr C} \bigoplus M_{S_C} \otimes [\C G_C]_{\vphi_C}  $ by
\[
a(h_1,g_1,s,h_2,g_2) \os {\displaystyle \Phi} \longmapsto \ol \gamma_{g_C , w_{h_1 g_1} , w_{h_2 g_2}} E_{(h_2,g_2) , (h_1,g_1)} \otimes [w^{-1}_{h_2 g_2} s^{-1} \; w_{h_1g_1} ]
\]
extended linearly where $ h_1g_1 , h_2 g_2 \in C $.
Using the formula for multiplication and $ \# $ in Propositions \ref{mult} and \ref{hash} and the cocyle relation in Proposition \ref{2coc}, one can imitate the proof of Proposition \ref{diagtubalg} to show that the map $ \Phi $ serves as the required isomorphism.

(ii) Let $\pi: \mcal A \ra \mcal L (V) $ be a Hilbert space representation.
For $ C_1,C_2 \in \mscr C $ such that $ C_1 \neq C_2 $, we need to show $ V^{C_1} $ and $ V^{C_2} $ are orthogonal.
Taking inner product of the generating vectors, we get $ \left\lab \pi (a(e,g_{C_1},s_1,h_1, g)) \xi\; ,\; \pi ( a(e,g_{C_2}, s_2,h_2,g)) \eta \right\rab = \left\lab \pi (a(e,g_{C_2}, s_2,h_2,g)) ^\# \cdot a(e,g_{C_1},s_1,h_1, g)) \xi\; , \; \eta \right\rab$ which is zero unless $ h_1 = h_2 $ but in that case $ C_1 $ and $ C_2 $ have to be the same; so, the inner product is zero.

For the decomposition, it remains to show that $ V \subset \us {C\in \mscr C} \bigoplus V^C $.
Let $ \xi \in V_g $.
Note that the identity $ \us {h \in H} \sum a(h,g,e,h,g) $ of $ \mcal A_{g,g} $ is a sum of orthogonal projections.
So, $ \xi = \us {h \in H} \sum \pi (a(h,g,e,h,g)) \xi  $.
For $ h\in H $, we have $ \pi (a(h,g,e,h,g)) \xi = \pi (a(e,g_{C},e,h,g) \zeta$ where $hg \in C \in \mscr C $ and $\zeta = \pi (a(h,g,e,e,g_C)) \xi \in V^C$.

The proof of equivalence of $ C $-supported representations with representations of $ [\C G]_{\vphi_C} $ is exactly the same as the proof of Theorem \ref{diagtubalg}.
\end{proof}
\begin{rem}
To find the tube algebra $ \mcal T $ of $ \mcal C_{NN} $, we need to first choose a set of representatives in the isomorphism classes of simple objects in $ \mcal C_{NN} $.
By Propositions \ref{smbox} and \ref{smbox*alg}, $ X_{g_1} $ and $ X_{g_2} $ are isomorphic if and only if $ g_1 $ and $ g_2 $ are in the same $ H $-$ H $ double coset where the isomorphism is implemented by \raisebox{-1.5 em}{\1disc{$h_1 $}{$ g_1 $}{$ g_2 $}{$ h_2 $}} for any $ h_1, h_2 \in H $ satisfying $h_1 g_1 = g_2 h_2$.
Now for $ g\in G $, the endomorphism space $ \t {End} (X_g) $ is isomorphic to the group algebra $ H^g := H \cap g^{-1} H g $ twisted by the scalar $ 2 $-cocycle $ H^g \times H^g  \ni (h_1 , h_2) \mapsto \overline{\omega} (g h_1 g^{-1}, g h_2 g^{-1}, g) \; \omega (g h_1 g^{-1} , g , h_2) \; \overline{\omega} (g , h_1, h_2) \in S^1$ via
\[
\C H^g \supset H^g \ni h \longmapsto \hspace{1.5em} \raisebox{-1.5 em}{\1disc{$ g h g^{-1} $}{$ g $}{$ g $}{$ h $}} \in \t {End} (X_g).
\]
For $ g \in G $, fix a maximal set $ \Pi_g $ of mutually orthogonal minimal projections in $ \t {End} (X_g) $.
Let $ H \backslash G / H $ be a set of representatives from all the $ H $-$ H $ double cosets in $ G $.
Then, it follows that% the set
$ \us {g \in  H \backslash G / H }  \bigcup  \left\{  \t{Range} (p) : p \in  \Pi_g \right\}$ is a set of representatives in the isomorphism classes of simple objects in $ \mcal C_{NN} $.
Hence, the tube algebra $ \mcal T $ is isomorphic (as a $ * $-algebra) to
\[
\us {g_1 , g_2 \in H \backslash G / H}  \bigoplus \;\us {p_1 \in \Pi_{g_1}, p_2 \in \Pi_{g_2}}  \bigoplus \left[\psi^{\mathbbm 1}_{g_2,g_2} (p_2) \circ
{\mcal A}_{g_1,g_2} \circ \psi^{\mathbbm 1}_{g_1,g_1} (p_1)\right].
\]
\end{rem}
\begin{rem}
By \cite[Theorem 4.2]{GJ}, we know that the representation categories of the  affine algebra $ \mcal A $ and tube algebra $ \mcal T $ are equivalent although as $ * $-algebras they are non-isomorphic.
There is one thing to notice that this representation category (appearing in Theorem \ref{bhtubalg}) is also equivalent to the category of tube representations of the diagonal subfactor (as in Theorem \ref{diagtubalg}) corresponding to the automorphisms $\alpha_g$, $g\in H\cup K$ of the $ II_1 $-factor $ Q $.
This equivalence can be seen in an alternative way:\\
\indent Let $_A \mcal H_B $ be an extremal bifinite bimodule and $ P $ be its corresponding subfactor planar algebra (namely the `unimodular bimodule planar algebra', in the sense of \cite{pertpa}).
By Theorem 4.2 of \cite{GJ}, the category of (Jones) affine $ P $-modules is equivalent to the representation category of the tube algebra of $ \mcal C_{AA}:= $ the category of bifinite $ A $-$ A $-bimodules generated by $_A \mcal H_B $.
By \cite[Remark 2.16]{DGG1}, the affine module categories corresponding to $ P $ and its dual $ \ol P $ are equivalent.
On the other hand, the dual planar algebra is isomorphic to the subfactor planar algebra associated to the contragradient bimodule $_B \ol {\mcal H}_A $.
Thus the representation category of the tube algebra of $ \mcal C_{AA} $ is equivalent to that of $ \mcal C_{BB}:= $ the category of bifinite $ B $-$ B $-bimodules generated by $_A \mcal H_B $.\\
\indent Next, consider an intermediate extremal finite index subfactor $ N \subset Q \subset M $.
Let $ \Gamma $ denote the bifinite bimodule $_N L^2 (M)_Q  $.
It is easy to check that the category $ \mcal C_{NN} $ of bifinite $ N $-$ N $-bimodules generated by $ \Gamma $ is the same as those which come from the subfactor $ N \subset M $.
Let $ \mcal C_{QQ} $ denote the smallest $ C^* $-tensor category of bifinite $ Q $-$Q $-bimodules coming from the subfactor $ N \subset Q $ as well as $ Q \subset M $.
One can verify that $ \mcal C_{QQ} $ is the same as the category of bifinite $ Q $-$Q $-bimodules generated by $ \Gamma $.
Hence, from the previous paragraph, the category of tube representations of $ \mcal C_{NN} $ is equivalent to that of $ \mcal C_{QQ} $.\\
\indent Coming back to our context of Bisch-Haagerup subfactor $ N = Q^H \subset Q \rtimes K $ as set up in the beginning of this section, it remains to show that $ \mcal C_{QQ} $ is the $ C^* $-tensor category generated by the bimodules $_Q L^2 (Q_{\alpha_g})_Q $ for $ g \in H \cup K $; this is an easy computation.
\end{rem}