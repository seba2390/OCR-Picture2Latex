\section{Bisch-Haagerup Subfactors}
In this section, we intend to find the tube algebra of the Bisch-Haagerup subfactor $N:= Q^{H} \subset Q \rtimes K = : M$ where $ H $ and $ K $ act outerly  on the $II_1$-factor $Q$.
It is well known that the planar algebra of $ N \subset M $ depends on the group $ G $ generated by $ H $ and $ K $ in $ \t {Out} (Q) $ and the scalar $ 3 $-cocycle obstruction (up to $ 2 $-coboundary) (see \cite{BH, BDG1}).

We first lay down the strategy to achieve our goal.
Instead of computing the tube algebra of $ \mcal C_{NN} $ directly (unlike the case of diagonal subfactors because the irreducible bimodules of $ \mcal C_{NN} $ for Bisch-Haagerup subfactors, are not so easy to work with), we will consider the affine annular algebra with respect to a particular full weight set (in the sense of  \cite[Definition 3.4]{GJ}) in $\t {ob} (\mcal C_{NN}) $, and then cut it down by the $ \t{Irr} \; \mcal C_{NN} $.

We need to set up some notations for this.
Pick a representative map $ \t {Out} (Q) \supset G \ni g \mapsto \alpha_g \in \t {Aut} (Q) $ such that $ g = \alpha_g \t {Inn} (Q) $, and $ \left.\alpha\right|_H : H \ra \t {Aut} (Q)$, $ \left.\alpha\right|_K : K \ra \t {Aut} (Q) $ are homomorphisms.
Now, if $ X = _N L^2 (M)_M $, $ Y = _N L^2 (Q)_Q $ and  $ Z = _Q L^2 (M)_M $, then
\[
\left( \; X \; \ol X \; \right)^{\us N \otimes m} = Y \us Q \otimes ( \; Z \; \ol Z \; ) \us Q \otimes ( \; \ol Y \; Y \; ) \us Q \otimes ( \; Z \; \ol Z \; ) \us Q \otimes \cdots \us Q \otimes ( \; Z \; \ol Z \; ) \us Q \otimes \ol Y.
\]
We know that $ ( \; \ol Y \; Y \; ) \us {Q\t {-} Q }\cong \; \us {h\in H} \bigoplus \; _Q  L^2 (Q_{\alpha_h})_Q$ and $ ( \; Z \; \ol Z \; ) \us {Q\t {-} Q } \cong \; \us {k\in K} \bigoplus \; _Q  L^2 (Q_{\alpha_k})_Q$.
So,
\[
\left( \; X \; \ol X \; \right)^{\us N \otimes k} \us {N\t {-} N }\cong \; \us {h_1,h_2,\ldots \in H}{\us {k_1,  k_2, \ldots \in K} \bigoplus} \; _N  L^2 (Q_{\alpha_{k_1} \alpha_{h_1} \alpha_{k_2} \alpha_{h_2} \cdots \alpha_{k_m}}) {} _N  \us {N\t {-} N }\cong \; \us {h_1,h_2,\ldots \in H}{\us {k_1,  k_2, \ldots \in K} \bigoplus} \; _N  L^2 (Q_{\alpha_{k_1 h_1 k_2 h_2 \cdots k_m}}) {} _N
\]
Since the subgroups $ H $ and $ K $ generate $ G $, therefore the set $\Lambda:= \left\{\left. X_g := {}_N L^2 (Q_{\alpha_g} ) {}_N \right| g \in G\right\} $ forms a full weight set in $ \mcal C_{NN} $.
It is possible to reduce the indexing set $ G $ of the weight set $ \Lambda $ further, since $ X_g \cong X_{gh} $ for all $ g \in G, h\in H $.
However, we will not do that since by reducing the weight set, one needs to work with coset representative which makes the calculations more cumbersome.

\subsection{Morphism spaces in $ \mcal C_{NN} $}$ \; $

For the affine annular algebra over $ G $ (indexing the above set), we do not need all morphism spaces of $ \mcal C_{NN} $.
We will instead concentrate on morphisms between elements of $\Lambda  $ and their tensor products.
Before that, we need more notations.
Choose a map $ u:G\times G \ra \mcal U (Q) $ such that $ \alpha_{g_1} \alpha_{g_2} = \t {Ad}_{u(g_1, g_2)} \alpha_{g_1 g_2} $ and 
\begin{equation}\label{ucond}
u \left( H\times H \; \cup \; K\times K \; \cup \; G \times \{e\} \; \cup \; \{e\} \times G \right) = \{1_Q\}.
\end{equation}
Again, associativity of multiplication in $ G $ and condition \ref{ucond} will give us a $ 3 $-cocycle $ \omega $ satisfying Equation \ref{uomega} and 
\begin{equation}\label{omegaHK1}
\left. \omega \right|_{H\times H \times H} \equiv 1 \equiv \left. \omega \right|_{K \times K \times K}.
\end{equation}
This along with Equation \ref{3coc}, implies $\omega (g, l ,l^{-1}) = \omega (gl, l^{-1}, l)$ and $\omega (l^{-1}, l , g) = \omega (l, l^{-1}, lg)$ for all $g \in G$, $l \in H \cup K$.
We will now prove a lemma on scalar cocycles which lets us choose the map $ u $ in such a way that the $ 3 $-cocycle $ \omega $ gets simplified making our calculations easy.
\begin{lem}\label{gllemma}
Any scalar $ 3 $-cocycle $\omega$ of a group $ G $ generated by subgroups $ H $ and $ K $, is coboundarily equivalent to $\omega^{\prime}$ which satisfies the relation \ref{omegaHK1} as well as
\begin{align}\label{gl}
\omega^{\prime} (g,l,l^{-1}) = 1 = \omega^{\prime} (l^{-1}, l , g) \text{ for all $g \in G$, $l \in H \cup K$.}
\end{align}
\end{lem}
\begin{proof}
Consider the subsets
$A_H = (H^\complement \times H^{\times})$,
$A_K = (K^\complement \times K^{\times})$,
$V_H = (H^{\times} \times H^\complement)$,
and
$V_K =(K^{\times} \times K^\complement)$ of
$G\times G$, and
the order $2$ bijections
$G\times G \ni (g_1,g_2) \hat{\mapsto}
(g_1,g_2)\hat{} = (g_1g_2,g^{-1}_2) \in G\times G$
and
$G\times G \ni (g_1,g_2) \check{\mapsto}
(g_1,g_2)\check{} = (g^{-1}_1,g_1g_2) \in
G\times G$
(where $H^\times = H \setminus \{e\}$ and
$K^\times = K \setminus \{e\}$). Note
that $A_H$ and $A_K$ (resp. $V_H$ and $V_K$)
are separately closed under $\hat{}$
(resp. $\check{}$ ) and have no fixed
points. Now,
$A_H \cap V_K = (K \setminus H) \times (H
\setminus K)$ (resp.
$A_K \cap V_H = (H \setminus K) \times (K
\setminus H)$) is mapped into
$A_H \setminus V_K$ (resp.
$A_K \setminus V_H$) under $\hat{}$ and into
$V_K \setminus A_H$ (resp.
$V_H \setminus A_K$) under $\check{}$ .
We choose

(i) a representative in each orbit of
$\; \hat{}$ inside ${A_H \cup A_K}$ such
that
the representative of the orbit containing
$(k,h) \in A_H \cap V_K$ is chosen as
$(kh,h^{-1})$ and the representative of the
one containing $(h,k) \in A_K \cap V_H$ is
chosen as $(hk,k^{-1})$,

(ii) a representative in each orbit of
$\; \check{}$ inside ${V_H \cup V_K}$ such
that
the representative of the orbit containing
$(k,h) \in A_H \cap V_K$ is chosen as
$(k^{-1},kh)$ and the representative of the
one containing $(h,k) \in A_K \cap V_H$ is
chosen as $(h^{-1},hk)$.

Let $A$ (resp. $V$) be the set of
representatives in $A_H \cup A_K$ (resp.
$V_H \cup V_K$). From our choice, it can be
verified that $A \cap V = \emptyset$. Define
$\vphi : G\times G \rightarrow \mathbb T$ by:

(a) $\vphi |_{G\times G \setminus (A \cup V)} = 1$,

(b)
$\vphi (g,l) = \omega (g,l,l^{-1}) = \omega
(gl, l^{-1},l)$ for $(g,l) \in A$,

(c)
$\vphi (l,g) = \omega (l^{-1},l,g) = \omega
(l, l^{-1},lg)$ for $(l,g) \in V$.

It follows that $\partial^2 ( \vphi ) $ is
normalized since $\omega$ is also so, and
$\partial^2 ( \vphi ) $ satisfies the
relation \ref{omegaHK1} since
$(H\times H \cup K \times K) \cap (A \cup V) = \emptyset$
(where $\partial^2$ denotes the $2$-cochain
map). Thus, the $3$-cocycle
$\omega^\prime = \partial^2 (\vphi) \cdot
\omega$ is normalized and satisfies relation \ref{omegaHK1}.

For relation \ref{gl}, we consider $g \in G$
and $l \in H \cup K$. Without loss of generality,
we assume $g \neq e \neq l$. So,
$\left\{ (g,l) , (gl,l^{-1}) \right\}$ (
resp. $\left\{ (l,g) , (l^{-1},lg)
\right\}$) is an orbit of $\; \hat{}$ (resp.
$\check{} \;$) in $A_H \cup A_K$ (resp.
$V_H \cup V_K$), and $\phi$ takes the value
$\omega (g , l , l^{-1}) = \omega (gl ,
l^{-1} , l)$ (resp.
$\omega (l^{-1} , l , g) = \omega (l ,
l^{-1} , lg)$) on the representative of the
orbit and $1$ on the other. This implies
\[
\omega^\prime (g, l, l^{-1}) = \ol{\vphi} (
gl,l^{-1}) \; \ol{\vphi} (g,l) \; \omega (g,
l, l^{-1}) = 1
\]
since $\vphi(g,e) = 1 = \vphi(l,l^{-1})$, and
similarly
$\omega^\prime (l^{-1} , l , g) = 1$.
\end{proof}
By the above lemma, without loss of generality, we may assume $ \omega $ satisfies:
\begin{align}\label{GL}
	\begin{tabular}{rrcl}
\text{(i)} & $\omega (g_1 ,  l , l^{-1}) $ & = \; 1 \; = &
$ \ol{\omega} (g_1 , l , l^{-1})$\\
\text{(ii)} & $\omega (g_1 , g_2 , l)$ & = &
$\ol{\omega} (g_1 , g_2 l , l^{-1})$\\
\text{(iii)} & $\omega (g_1, l, g_2)$ & = &
$\ol{\omega} (g_1 l,l^{-1}, l g_2)$\\
\text{(iv)} & $\omega (l, g_1,g_2)$ & = &
$\ol{\omega} (l^{-1}, l g_1,g_2)$\\
\text{(v)} & $u(g_1, l)$ & = &
$u^*(g_1 l , l^{-1})$\\
\text{(vi)} & $u(l ,g_2)$ & = &
$\alpha_l \left( u^* (l^{-1} , l g_2) \right)$
	\end{tabular}
\end{align}
for all $g_1, g_2 \in G$, $l \in H \cup K$ (where (ii), (iii) and (iv) are immediate implication of (i) and \ref{2coc}).
We will need the relation \ref{GL} only when $ l \in H $; however, we gave the general version, in case any reader is interested to see the actual $ 2 $-category of $ N\subset M $ instead of just $ \mcal C_{NN} $.
\begin{prop}\label{smbox}
The morphism space $ \mcal C_{NN} (X_{g_1} , X_{g_2}) $ is zero unless $ g_1 $ and $ g_2 $ give the same $ H $-$ H $ double coset, and if they do, the space has a basis given by
\[
B_{g_1,g_2} := \left\{ \left.
\begin{tabular}{l}
\\
$ X_{g_1} \ni [x]_{g_1} \os {  } \longmapsto [ \alpha_{h_1} (x) u(h_1 , g_1) u^* (g_2 , h_2)]_{g_2} \in X_{g_2} $\\
denoted by the symbol \raisebox{-1.4 em}{\1disc{$h_1 $}{$ g_1 $}{$ g_2 $}{$ h_2 $}}
\end{tabular}
\; \right| \; 
\begin{tabular}{l}
$ h_1, h_2 \in H $\\
such that \\
$ h_1 g_1 = g_2 h_2 $
\end{tabular} \right\}.
\]
%We will denote the element corresponding to $B_{g_1,g_2}$ by $\; \smbox{h_1}{g_1}{g_2}{h_2}{2}{2}{-0.3}$ $ \in \mcal C_{NN} (g_1,g_2) $.
\end{prop}
\begin{proof}
By Frobenius reciprocity, $ \t {dim}_\C \left( \mcal C_{NN} (X_{g_1} , X_{g_2}) \right) = \t {dim}_\C \left( \mcal C_{NN} \left( _N L^2 (N) _N \; , \; \ol X_{g_1} \us N \otimes X_{g_2}\right) \right) $.
Again $  \ol X_{g_1} \us N \otimes X_{g_2} \us {Q\t {-} Q} \cong \; {}_{\alpha_{g_1}} \left[ Q \rtimes H \right] {}_{\alpha_{g_2}}$ where the left and right actions of $ Q $ on $ Q\rtimes H $ is twisted by $ \alpha_{g_1} $ and $ \alpha_{g_2} $ respectively.
Any element of $ \mcal C_{NN} \left( _N L^2 (N) _N \; , \; {}_{\alpha_{g_1}} \left[ Q \rtimes H \right] {}_{\alpha_{g_2}}\right) $ corresponds to an element of $ Q\rtimes H $ (the image of $ \hat 1 $), say $y = \us {h\in H} \sum y_h \; h$.
By $ N $-$ N $ linearity, we will have $ \alpha_{g_1} (n) \; y_h = y_h \; \alpha_h (\alpha_{g_2} (n)) $ for all $  n\in N, h\in H $, equivalently
\[
n \; \alpha^{-1}_{g_1} (y_h) = \alpha^{-1}_{g_1} (y_h) \; \alpha^{-1}_{g_1}(\alpha_h (\alpha_{g_2} (n))) \t{ for all }  n\in N, h\in H.
\]

The following is a well-known fact for the fixed-point subfactor $ N\subset Q $ of an outer action of $ H $. For $ y \in Q $ and $ \theta \in \t {Aut} (Q) $, the following are equivalent:

(i) $ y \neq 0 $ and $ n y = y \theta (n) $ for all $ n \in N = Q^H $,

(ii) $ y_0 := \displaystyle \frac {y}{\norm{y}} \in \mcal U (Q)$ and $ \t{Ad}_{y_0} \circ \theta \in \{\alpha_h : h \in H\} $.

By the above fact, $ y \neq 0 $ only when there exists $ h_1 , h_2  \in H$ such that $ \alpha^{-1}_{g_1} \alpha_{h_1} \alpha_{g_2} \alpha_{h_2} \in \t {Inn} (Q)$, equivalently $ g_1 $ and $ g_2 $ generate the same $ H $-$ H $ double coset.
In particular, $ y_h = 0$ unless $ h $ belongs to $ H \cap g_1 H g^{-1}_2 $.
And for $ h \in  H \cap g_1 H g^{-1}_2$, for $ y_h \neq 0 $, we have $ \t {Ad}_{\alpha^{-1}_{g_1 } (y_0) } \; \alpha^{-1}_{g_1} \; \alpha_h \; \alpha_{g_2} = \alpha_{g^{-1}_1 h g_2} $ where $ y_0 = \displaystyle \frac{y_h}{\norm {y_h}} $.
This implies $  \t {Ad}_{y_0} \; \alpha_h \; \alpha_{g_2} = \alpha_{g_1} \; \alpha_{g^{-1}_1 h g_2} $, equivalently, $  \t {Ad}_{y_0 u(h ,g_2) } = {Ad}_{u(g_1 , g^{-1}_1 h g_2) } $.
Hence, $ y_h \in \C \{u(g_1 , g^{-1}_1 h g_2) u^*(h ,g_2)\}$.
%Therefore, we obtain $ \t {dim}_\C \left( \mcal C_{NN} (X_{g_1} , X_{g_2}) \right) = \abs {H g_2 \cap g_1 H} = \abs {H g_1 \cap g_2 H}$.
Thus, the set
\[
\left\{ \left(u (g_1 , h^{-1}_2 ) u^* (h^{-1}_1 ,g_2)\right) \; h^{-1}_1 : h_1, h_2 \in H \t { such that } h_1 g_1 = g_2 h_2 \right\} 
\]
forms a basis of the vector space $ V:= \left\{ y \in Q \rtimes H : \alpha_{g_1} (n) y = y \alpha_{g_2} (n) \t { for all } n \in N \right\} $.
To show that the set $ B_{g_1,g_2} $ forms a basis for $ Hg_1 H = H g_2 H $, we need the following explicit isomorphism:
\[
V \ni y \os {\pi} \longmapsto \pi(y) := J y^* J \in  \mcal C_{NN} (X_{g_1} , X_{g_2})
\]
where $ J $ is the canonical anti-unitary of $ L^2 (Q) $.
Set $ y_{h_1,h_2} : = \left(u (g_1 , h^{-1}_2 ) u^* (h^{-1}_1 ,g_2)\right) \; h^{-1}_1 $ for $ h_1 g_1 = g_2 h_2 $.
Then,
\[
\pi (y_{h_1,h_2}) [x]_{g_1} = \left[ \alpha_{h_1} \left( x u (g_1 , h^{-1}_2 ) u^* (h^{-1}_1 ,g_2 ) \right) \right]_{g_2} = \left[ \alpha_{h_1} \left( x \right) \alpha_{h_1} \left(u (g_1 , h^{-1}_2 ) u^* (h^{-1}_1 ,g_2 ) \right) \right]_{g_2}.
\]
We simplify $ \alpha_{h_1} \left(u (g_1 , h^{-1}_2 ) u^* (h^{-1}_1 ,g_2 ) \right) $ using Equations \ref{uomega}, \ref{ucond} and \ref{GL} to get
\begin{align*}
& \; \left\{ \ol \omega (h_1,g_1 , h^{-1}_2) u(h_1,g_1) u (h_1g_1 ,h^{-1}_2) u^*(h_1,g_1 h^{-1}_2)\right\} \; u(h_1 , h^{-1}_1 g_2) \\
= & \; \ol \omega (h_1,g_1 , h^{-1}_2) \; u(h_1,g_1) u (g_2 h_2 ,h^{-1}_2) = \; \ol \omega (h_1,g_1 , h^{-1}_2) \; \left\{u(h_1,g_1) u^* (g_2, h_2)\right\}.
\end{align*}
Hence, $ \pi (y_{h_1,h_2}) $ is a unit scalar multiple of \raisebox{-1.5 em}{\1disc{$h_1 $}{$ g_1 $}{$ g_2 $}{$ h_2 $}}
corresponding to $ (h_1,h_2) $.
\end{proof}
\begin{rem}\label{duality}
The maps
\[
{}_N L^2(N)_N \ni \hat 1 \os {\displaystyle R_g} \longmapsto \us i \sum [u^*(g^{-1} , g) \alpha_{g^{-1}} (b_i) ]_{g^{-1}} \us N \otimes [b^*_i]_g \in X_{g^{-1}} \us N \otimes X_g
\]
\[
{}_N L^2(N)_N \ni \hat 1 \os {\displaystyle \ol R_g} \longmapsto \omega (g,g^{-1},g) \us i \sum [u^*(g, g^{-1}) \alpha_{g} (b_i) ]_{g} \us N \otimes [b^*_i]_{g^{-1}} \in X_{g} \us N \otimes X_{g^{-1}}
\]
are standard solutions to conjugate equations for duality of $ X_g $ where $ \{b_i\}_i $ is a basis for the subfactor $ N \subset Q $.
We will also need the $ * $ of these maps, namely
\[
X_{g^{-1}} \us N \otimes X_g \ni [x]_{g^{-1}} \us N \otimes [y]_g \os {\displaystyle R^*_g} \longmapsto E_N \left(x \alpha_{g^{-1}} (y) u(g^{-1} , g )\right) \in {}_N L^2(N)_N
\]
\[
 X_{g} \us N \otimes X_{g^{-1}} \ni [x]_g \us N \otimes [y]_{g^{-1}} \os {\displaystyle {\ol R}^*_g} \longmapsto \ol \omega (g, g^{-1}, g) \; E_N \left(x \alpha_{g} (y) u(g , g^{-1} )\right) \in {}_N L^2(N)_N
\]
\end{rem}
\begin{prop}\label{smbox*alg}$ { } $

(i) \hspace{1em} \raisebox{-2.7 em}
{\psfrag{1}{\reflectbox{$h_3 $}}
\psfrag{2}{$ g_2 $}
\psfrag{3}{$ g_3 $}
\psfrag{4}{$ h_4 $}
\psfrag{5}{\reflectbox{$h_1 $}}
\psfrag{6}{$ g_1 $}
\psfrag{7}{$ h_2 $}
\includegraphics[scale=0.2]{figures/cnn/2disc.eps}} \; $ := \; \raisebox{-1.5 em}{\1disc{$h_3 $}{$ g_2 $}{$ g_3 $}{$ h_4 $}} \; \circ \; \raisebox{-1.5 em}{\1disc{$h_1 $}{$ g_1 $}{$ g_2 $}{$ h_2 $}} \; = \; \ol \omega (h_3, h_1, g_1) \; \omega (h_3, g_2, h_2) \; \ol \omega (g_3 , h_4 , h_2)$ \;\;\;  \raisebox{-1.5 em}{\1disc{$ h_3 h_1 $}{$ g_1 $}{$ g_3 $}{$ h_4 h_2 $}}

(ii) $  \left[ \raisebox{-1.5 em}{\1disc{$\!\!\! h_1 $}{$ g_1 $}{$ g_2 $}{$ h_2 $}} \right]^* \; = \; \ol \omega (h_1, g_1, h^{-1}_2)  $
\; \raisebox{-1.5 em}{\1disc{$\!\!\!\!\!\! h^{-1}_1 $}{$ g_2 $}{$ g_1 $}{$ h^{-1}_2 $}}
\end{prop}
\begin{proof}
(i) The left side is given by $ [x]_{g_1} \mapsto \left[\alpha_{h_3 h_1} (x) \; \alpha_{h_3} (u(h_1,g_1) u^*(g_2,h_2)) \;  u(h_3,g_2)u^*(g_3,h_4)\right]_{g_3}$.
Observe that
\begin{align*}
& \; \alpha_{h_3} (u(h_1,g_1) u^*(g_2,h_2)) \;  u(h_3,g_2)u^*(g_3,h_4)\\
= & \; \ol \omega (h_3, h_1,g_1) \; u(h_3 h_1,g_1) \; u^*(h_3 , \us {=g_2 h_2} {h_1 g_1}) \;  \alpha_{h_3} (u^*(g_2,h_2)) \;  u(h_3,g_2) \; u^*(g_3,h_4)\\
& \; \t {(applying \ref{uomega} and \ref{ucond} on the first term)}\\
= & \; \ol \omega (h_3, h_1,g_1) \; \omega (h_3,g_2,h_2) \; u(h_3 h_1,g_1) \; u^*(\us {=g_3h_4}{h_3 g_2} , h_2) \; u^*(g_3,h_4)\\
& \; \t {(applying \ref{uomega} on the second, third and fourth terms)}\\
= & \; \ol \omega (h_3, h_1,g_1) \; \omega (h_3,g_2,h_2) \; \ol \omega (g_3, h_4,h_2) \; u(h_3 h_1 , g_1) \; u^*(g_3,h_4h_2)\\
& \; \t {(applying \ref{uomega} and \ref{ucond} on the last two terms)}
\end{align*}
which gives the required result.

(ii) Note that \raisebox{-1.4 em}{\1disc{$\!\!\! h_1 $}{$ g_1 $}{$ g_2 $}{$ h_2 $}} is a unitary which follows right from its definition.
Using part (i), one can easily show that \; \raisebox{-1.4 em}{\1disc{$\!\!\!\!\!\! h^{-1}_1 $}{$ g_2 $}{$ g_1 $}{$ h^{-1}_2 $}} \; is indeed the inverse of \raisebox{-1.4 em}{\1disc{$\!\!\! h_1 $}{$ g_1 $}{$ g_2 $}{$ h_2 $}} where one uses the relations in \ref{GL}.
\end{proof}
Next, we will prove some facts about tensor product of two elements from $ \Lambda $.
For $ g_1,g_2 \in G$ and $ h\in H $, we define \raisebox{-1.5em}{\tower{$ g_1 $}{$ g_2 $}{$ g_1hg_2 $}{$ h $}}  $ : X_{g_1} \us N \otimes X_{g_2} \ra X_{g_1hg_2}$ in the following way
\[
X_{g_1} \us N \otimes X_{g_2} \ni [x]_{g_1} \us N \otimes [y]_{g_2} \longmapsto \abs {H}^{-\frac 1 2} \left[ x \; \alpha_{g_1} (\alpha_h (y)) \; u(g_1 ,h) \; u(g_1 h, g_2) \right]_{g_1h g_2} \in X_{g_1hg_2}.
\]
\begin{rem}
With standard inner product computation, one can show that
\begin{align*}
\left( \raisebox{-1.5em}{\tower{$ g_1 $}{$ g_2 $}{$ g_1hg_2 $}{$ h $}}  \right)^* : [z]_{g_1hg_2} \longmapsto & \; \abs{H}^{- \frac 1 2} \us {i} \sum [z u^*(g_1h ,g_2) u^* (g_1 , h) \alpha_{g_1} (b_i)]_{g_1} \us N \otimes [\alpha_{h^{-1}} (b^*_i)]_{g_2}\\
= & \; \abs{H}^{- \frac 1 2} \us {i} \sum [\alpha_{g_1} (b_i)]_{g_1} \us N \otimes [\alpha_{h^{-1}} \left(b^*_i \alpha^{-1}_{g_1} (z u^*(g_1h ,g_2) u^* (g_1 , h) ) \right) ]_{g_2}
\end{align*}
where $ \{b_i\}_i $ is \textbf{any} basis of $ Q $ over $ N $.
\end{rem}
\noindent To see this, consider $ \left\lab \raisebox{-1.5em}{\tower{$ g_1 $}{$ g_2 $}{$ g_1hg_2 $}{$ h $}}  \left([x]_{g_1} \us N \otimes [y]_{g_2} \right) \; , \; [z]_{g_1hg_2} \right \rab $
\begin{align*}
&= \abs {H}^{- \frac 1 2} \t {tr} \left( x \; \alpha_{g_1} (\alpha_h (y)) \; u(g_1 ,h) \; u(g_1 h, g_2) \; z^* \right)\\
&= \abs {H}^{- \frac 1 2} \us i \sum \t {tr} \left( x \; \alpha_{g_1} \left( E_N (y \alpha_{h^{-1}} ( b_i) ) b^*_i \right) \; u(g_1 ,h) \; u(g_1 h, g_2) \; z^* \right)\\
& = \abs {H}^{- \frac 1 2} \us i \sum \t {tr} \left( x \; \alpha_{g_1} \left( E_N (y \alpha_{h^{-1}} ( b_i) ) \right) \; \left(z \; u^*(g_1 h, g_2) \; u^*(g_1 ,h) \; \alpha_{g_1} (b_i) \right)^* \right)\\
& = \abs {H}^{- \frac 1 2} \us i \sum  \left\lab [x]_{g_1} \;  _N \left\lab [y]_{g_2} , [\alpha_{h^{-1}} ( b^*_i)]_{g_2} \right\rab  \; , \; \left[z \; u^*(g_1 h, g_2) \; u^*(g_1 ,h) \; \alpha_{g_1} (b_i) \right]_{g_1} \right\rab
\end{align*}
$= \left\lab [x]_{g_1} \us N \otimes [y]_{g_2} \; , \; \left(\raisebox{-1.5em}{\tower{$ g_1 $}{$ g_2 $}{$ g_1hg_2 $}{$ h $}}  \right)^* [z]_{g_1hg_2} \right \rab $.
We will denote $\left( \raisebox{-1.5em}{\tower{$ g_1 $}{$ g_2 $}{$ g_1hg_2 $}{$ h $}} \right)^*$ by \raisebox{-1.5em}{\wine{$ g_1 $}{$ g_2 $}{$ g_1hg_2 $}{$ h $}}.
It is straightforward to check \raisebox{-1.5em}{\wine{$ g_1 $}{$ g_2 $}{$ g_1hg_2 $}{$ h $}} preserves inner product and thereby is an isometry.
So, the element
\;\;\raisebox{-3em}{
	\psfrag{1}{\reflectbox{$ g_1 $}}
	\psfrag{2}{$ g_2 $}
	\psfrag{3}{$g_1 h g_2 $}
	\psfrag{4}{$ h $}
	\psfrag{5}{\reflectbox{$ g_1 $}}
	\psfrag{6}{$ g_2 $}
	\psfrag{7}{$ h $}
	\includegraphics[scale=0.2]{figures/cnn/winetower.eps}
	}\hspace{1em}
$ := \left( \raisebox{-1.5em}{\wine{$ g_1 $}{$ g_2 $}{$ g_1hg_2 $}{$ h $}} \; \circ \; \raisebox{-1.5em}{\tower{$ g_1 $}{$ g_2 $}{$ g_1hg_2 $}{$ h $}} \; \right) $ is a projection in $ \t {End } (X_{g_1} \us N \otimes X_{g_2}) $ for every $ h\in H $.
\begin{prop}\label{resid}
The set $ \left\{ \raisebox{-1.5em}{\tower{$ g_1 $}{$ g_2 $}{$ g_1hg_2 $}{$ h $}} : h \in H \right\} $ gives a resolution of the identity in $ \t {End } (X_{g_1} \us N \otimes X_{g_2}) $.
\end{prop}
\begin{proof}
It is enough to check  $\displaystyle \us {h\in H} \sum  \raisebox{-3em}{
	\psfrag{1}{\reflectbox{$ g_1 $}}
	\psfrag{2}{$ g_2 $}
	\psfrag{3}{$g_1 h g_2 $}
	\psfrag{4}{$ h $}
	\psfrag{5}{\reflectbox{$ g_1 $}}
	\psfrag{6}{$ g_2 $}
	\psfrag{7}{$ h $}
	\includegraphics[scale=0.2]{figures/cnn/winetower.eps}
} \hspace{1em} = \; \t {id}_{X_{g_1} \us N \otimes X_{g_2}}$.
The left side acting on $ [x]_{g_1} \us N \otimes [y]_{g_2} $ gives
\begin{align*}
= & \; \abs{H}^{- 1} \us {i,h} \sum [\alpha_{g_1} (b_i)]_{g_1} \us N \otimes [\alpha_{h^{-1}} \left(b^*_i \alpha^{-1}_{g_1} ( x \; \alpha_{g_1} (\alpha_h (y)) ) \right) ]_{g_2}\\
= & \; \abs{H}^{- 1} \us {i,h} \sum [\alpha_{g_1} (b_i)]_{g_1} \us N \otimes [\alpha_{h^{-1}} \left(b^*_i \alpha^{-1}_{g_1} ( x  )  \right) \; y]_{g_2} =  \us {i} \sum [\alpha_{g_1} (b_i)]_{g_1} \us N \otimes [E_N \left(b^*_i \alpha^{-1}_{g_1} ( x  )  \right) \; y]_{g_2} =  [x]_{g_1} \us N \otimes [y]_{g_2}.
\end{align*}
\end{proof}
\begin{rem}\label{hexagon}
From Propositions \ref{smbox} and \ref{resid}, we may conclude that  $ \mcal C_{NN} (X_{g_1} \us N \otimes X_{g_2} , X_{g_3} \us N \otimes X_{g_4}) $ is linearly spanned by the (linearly independent) set
\[
\left\{ \left. \raisebox{-4.4em}{\hexagon{$ g_1 $}{$ g_2 $}{$h_3 $}{$ g_3 $}{$ g_4 $}{$ h_4 $}{$ h_1 $}{$h_2  $}{\psfrag{9}{}\psfrag{0}{}}}
\; := \; \raisebox{-1.5em}{\wine{$ g_3 $}{$ g_4 $}{$ g_3 h_4 g_4 $}{$ h_4 $}} \; \circ \; \raisebox{-1.5 em}{\1disc{$h_1 $}{$ g_1 h_3 g_2 $}{$ g_3 h_4 g_4 $}{$ h_2 $}} \hspace{1em} \circ \raisebox{-1.5em}{\tower{$ g_1 $}{$ g_2 $}{$ g_1h_3g_2 $}{$ h_3 $}} \hspace{1em} \right| h_1,h_2,h_3 ,h_4 \in H \right\}.
\]
\end{rem}
\noindent We will now prove two lemmas which will be very useful in finding the structure the annular algebra.
As for notations, we will use the standard graphical representations of morphism where composition will be represented by stacking the morphisms vertically with the left most being in the top.
\begin{lem}\label{tribox}
\begin{align*}
	\t{(i)} \hspace{2em}& \raisebox{-3em}{\psfrag{1}{\reflectbox{$ g_1 $}}
		\psfrag{2}{$ s $}
		\psfrag{3}{$ g_1 h  s $}
		\psfrag{4}{$ h $}
		\psfrag{5}{\reflectbox{$ h_1 $}}
		\psfrag{6}{$ h_2 $}
		\psfrag{7}{$ t $}
		\includegraphics[scale=0.2]{figures/cnn/triboxilhs.eps}}
%	\raisebox{-1.5em}{\tower{$ g_1 $}{$ s $}{$ g_1hs $}{$ h $}} \circ \left(\t{id}_{X_{g_1}} \us N \otimes \raisebox{-1.5 em}{\1disc{$h_1 $}{$ t $}{$ s $}{$ h_2 $}} \right)
\hspace{1em} =  \left[ \omega (g_1 h , s , h_2) \; \ol \omega (g_1 h , h_1 ,t) \; \omega (g_1 ,  h,h_1) \right] \; \raisebox{-3em}{
	\psfrag{1}{\reflectbox{$ g_1 $}}
	\psfrag{2}{$ t $}
	\psfrag{3}{$ g_1 hh_1 t $}
	\psfrag{4}{$ h h_1 $}
	\psfrag{5}{\reflectbox{$ e \;$}}
	\psfrag{6}{$ h_2 $}
	\psfrag{7}{$ g_1 h s $}
	\includegraphics[scale=0.2]{figures/cnn/triboxirhs.eps}}
%	\hspace{2em} \raisebox{-1.5 em}{\1disc{$e \;$}{$ g_1h h_1 t $}{$ g_1 h s $}{$ h_2 $}} \hspace{1em} \circ \raisebox{-1.5em}{\tower{$ g_1 $}{$ t $}{$ g_1h h_1 t$}{$ h h_1 $}}
\\
\t{(ii)} \hspace{2em} &\raisebox{-3em}{
	\psfrag{1}{\reflectbox{$ s $}}
	\psfrag{2}{$ g_2 $}
	\psfrag{3}{$ s h g_2 $}
	\psfrag{4}{$ h $}
	\psfrag{5}{\reflectbox{$ h_1 $}}
	\psfrag{6}{$ h_2 $}
	\psfrag{7}{$ t $}
	\includegraphics[scale=0.2]{figures/cnn/triboxiilhs.eps}}
%\raisebox{-1.5em}{\tower{$ s $}{$ g_2 $}{$ shg_2 $}{$ h $}} \circ \left( \raisebox{-1.5 em}{\1disc{$h_1 $}{$ t $}{$ s $}{$ h_2 $}} \us N \otimes \t{id}_{X_{g_2}} \right)
\hspace{1em} = \left[ \omega (h_1 , t h^{-1}_2 h , g_2)\; \ol \omega (s, h_2 , h^{-1}_2 h) \; \omega (h_1 , t , h^{-1}_2 h) \right] \; \raisebox{-3em}{
	\psfrag{1}{\reflectbox{$ t $}}
	\psfrag{2}{$ g_2 $}
	\psfrag{3}{$ t h^{-1}_2 h  g_2 $}
	\psfrag{4}{\!\!\!$ h^{-1}_2 h $}
	\psfrag{5}{\reflectbox{$ h_1$}}
	\psfrag{6}{$ e $}
	\psfrag{7}{$s h g_2 $}
	\includegraphics[scale=0.2]{figures/cnn/triboxirhs.eps}}
%\hspace{2em}\raisebox{-1.5 em}{\1disc{$h_1 $}{$ th^{-1}_2 h g_2 $}{$ sh g_2 $}{$ e $}} \hspace{1em} \circ \raisebox{-1.5em}{\tower{$ t $}{$ g_2 $}{$ t h^{-1}_2 h g_2 $}{\!\!\!$ h^{-1}_2h $}}
\end{align*}
\end{lem}
\begin{proof}
	(i) The left side acts on $ [x]_{g_1} \us N \otimes [y]_t $, gives
	\begin{align*}
		\abs{H}^{- \frac 1 2} \left[x \alpha_{g_1} \left(\alpha_h \left(\alpha_{h_1} (y) u(h_1 ,t) u^*(s,h_2)\right)\right) u(g_1,h) u(g_1 h , s) \right]_{g_1h s}
	\end{align*}
	whereas the right side yields
	\[
	\abs{H}^{- \frac 1 2} \left[x \alpha_{g_1} \left(\alpha_{h h_1} (y) \right) u(g_1 , hh_1) u(g_1 h h_1, t)  u^*(g_1 h s , h_2) \right]_{g_1h s}.
	\]
	After striking out the similar terms, we will be left with
	\begin{align*}
		& \; \alpha_{g_1} \left(\alpha_h \left( u(h_1 ,t) u^*(s,h_2)\right)\right) u(g_1,h) u(g_1 h , s)\\
		= & \; u(g_1,h) \alpha_{g_1 h} \left( u(h_1 ,t) u^*(s,h_2)\right)  u(g_1 h , s)\\
		= & \;  u(g_1,h) \alpha_{g_1 h} \left( u(h_1 ,t) \right) \omega (g_1 h , s , h_2) u(g_1h, \us {=h_1 t} {s h_2}) u^* (g_1 h s ,h_2) \\
		= & \; \omega (g_1 h , s , h_2) u(g_1,h) \ol \omega (g_1 h , h_1 ,t) u(g_1 h, h_1) u(g_1h h_1 , t) u^* (g_1 h s ,h_2) \\
		= & \; \left[\omega (g_1 h , s , h_2) \ol \omega (g_1 h , h_1 ,t) \omega (g_1 ,  h,h_1)\right] \; \left( u(g_1 , hh_1) u(g_1 h h_1, t)  u^*(g_1 h s , h_2) \right)
	\end{align*}
	
	(ii) The action of left side on $ [x]_t \us N \otimes [y]_{g_2} $ is
	\begin{align*}
		& \; \abs H^{- \frac 1 2} \left[\alpha_{h_1} (x) u(h_1 , t) u^* (s ,h_2) \; \alpha_s \left( \alpha_h (y)  \right) \; u(s,h) u(sh,g_2)\right]_{shg_2}\\
		= & \; \abs H^{- \frac 1 2} \left[\alpha_{h_1} \left(\; x \; \alpha_t (\alpha_{h^{-1}_2 h} (y) \;  \right) u(h_1 , t) u^* (s ,h_2) \;  u(s,h) u(sh,g_2) \right]_{shg_2}
		%= & \; \left[\alpha_{h_1} \left(\; x \; \alpha_t (\alpha_{h^{-1}_2 h} (y) \; \right) u(h_1 , t) u (\us {= h_1 t} {s h_2} , h^{-1}_2) \; \omega (s,h,g_2) \alpha_s (u(h,g_2)) u(s,hg_2) \right]_{shg_2}\\
		%= & \; \omega (s,h,g_2) \; \left[\alpha_{h_1} \left(\; x \; \alpha_t (\alpha_{h^{-1}_2 h} (y) \; \right) \omega (h_1 ,t ,h^{-1}_2) \alpha_{h_1} (u(t, h^{-1}_2)) u (h_1 , t h^{-1}_2) \;  \alpha_s (u(h,g_2)) u(s,hg_2) \right]_{shg_2}\\
	\end{align*}
	and the right side on the same is
	\begin{align*}
		& \; \abs H^{- \frac 1 2}  \left[\alpha_{h_1} \left(\; x \; \alpha_t (\alpha_{h^{-1}_2 h} (y) \; u(t, h^{-1}_2 h) u(t h^{-1}_2 h, g_2)  \right) u(h_1 , t h^{-1}_2 h g_2) \right]_{shg_2}\\
		= & \; \abs H^{- \frac 1 2}  \left[\alpha_{h_1} \left(\; x \; \alpha_t (\alpha_{h^{-1}_2 h} (y) \right) \; \alpha_{h_1} (u(t, h^{-1}_2 h) ) \; \ol \omega (h_1 , t h^{-1}_2 h , g_2) u (h_1 , t h^{-1}_2 h ) u (h_1  t h^{-1}_2 h , g_2) \right]_{shg_2}\\
		= & \; \ol \omega (h_1 , t h^{-1}_2 h , g_2) \abs H^{- \frac 1 2}  \left[\alpha_{h_1} \left(\; x \; \alpha_t (\alpha_{h^{-1}_2 h} (y) \right) \ol \omega (h_1 , t , h^{-1}_2 h) u (h_1 , t ) u (\us {=s h_2}{h_1  t} , h^{-1}_2 h) \;  u (s h , g_2) \right]_{shg_2}\\
		= & \; \ol \omega (h_1 , t h^{-1}_2 h , g_2)  \ol \omega (h_1 , t , h^{-1}_2 h) \abs H^{- \frac 1 2}\\
		& \;\left[\alpha_{h_1} \left(\; x \; \alpha_t (\alpha_{h^{-1}_2 h} (y) \right) u (h_1 , t ) \; \omega (s, h_2 , h^{-1}_2 h) u^* (s ,h_2) \;  u(s,h) \;  u (s h , g_2) \right]_{shg_2}.
	\end{align*}
\end{proof}
\begin{lem}\label{2tri}$ $\\
(i) \; \raisebox{-3em}{
	\psfrag{1}{\reflectbox{$ g_1 $}}
	\psfrag{2}{$ g_2 $}
	\psfrag{3}{$ g_3 $}
	\psfrag{4}{$ h_1 $}
	\psfrag{5}{$ h_2$}
	\psfrag{6}{\reflectbox{$ g_1 h_1 g_2 $}}
	\psfrag{7}{$g_2 h_2 g_3 $}
	\includegraphics[scale=0.2]{figures/cnn/2triilhs.eps}}
%	\[\left(\t{id}_{X_{g_1}} \us N \otimes \raisebox{-1.5em}{\tower{$ g_2 $}{$ g_3 $}{$ g_2h_2g_3 $}{$ h_2 $}} \;\;\right) \circ \left( \raisebox{-1.5em}{\wine{$ g_1 $}{$ g_2 $}{$ g_1h_1g_2 $}{$ h_1 $}} \;\; \us N \otimes \t{id}_{X_{g_3}} \right)
\; $ = \left[\ol \omega(g_1h_1, g_2 , h_2) \; \ol \omega (g_1h_1 ,g_2 h_2, g_3)\right] \hspace{1em} \raisebox{-3em}{
	\psfrag{1}{\reflectbox{$ g_1 $}}
	\psfrag{2}{$ g_2 h_2 g_3 $}
	\psfrag{3}{$g_1 h_1 g_2 h_2 g_3 $}
	\psfrag{4}{$ h_1 $}
	\psfrag{5}{\reflectbox{$ g_1 h_1g_2 $}}
	\psfrag{6}{$ g_3 $}
	\psfrag{7}{$ h_2 $}
	\includegraphics[scale=0.2]{figures/cnn/winetower.eps}
} $\\
	
\noindent(ii) \; \raisebox{-3em}{
	\psfrag{1}{\reflectbox{$ g_1 $}}
	\psfrag{2}{$ g_2 $}
	\psfrag{3}{$ g_3 $}
	\psfrag{4}{$ h_1 $}
	\psfrag{5}{$ h_2$}
	\psfrag{6}{\reflectbox{$ g_1 h_1 g_2 $}}
	\psfrag{7}{$g_2 h_2 g_3 $}
	\includegraphics[scale=0.2]{figures/cnn/2triiilhs.eps}}
%\[\left( \tridown{g_1}{g_2}{g_1h_1g_2}{3}{3}{-0.5} \; \us N \otimes  \t{id}_{X_{g_3}} \right) \circ \left( \t{id}_{X_{g_1}} \us N \otimes \; \triup{g_2}{g_3}{ g_2h_2g_3  }{3}{3}{1}  \right)
$ = \left[\omega(g_1h_1, g_2 , h_2) \;  \omega (g_1h_1 ,g_2 h_2, g_3)\right] \hspace{1em} \raisebox{-3em}{
	\psfrag{1}{\reflectbox{$ g_1 h_1g_2 $}}
	\psfrag{2}{$ g_3 $}
	\psfrag{3}{$g_1 h_1 g_2 h_2 g_3 $}
	\psfrag{4}{$ h_2 $}
	\psfrag{5}{\reflectbox{$ g_1 $}}
	\psfrag{6}{$ g_2 h_2 g_3 $}
	\psfrag{7}{$ h_1 $}
	\includegraphics[scale=0.2]{figures/cnn/winetower.eps}
}$
\end{lem}
\begin{proof}
The left side acting on $ [x]_{g_1 h_1 g_2} \us N \otimes [y]_{g_3}$ gives
\begin{align*}
& \; \abs{H}^{-1} \us i \sum [x u^*(g_1 h_1 ,g_2) u^* (g_1, h_1) \alpha_{g_1} (b_i)]_{g_1} \us N \otimes [\alpha^{-1}_{h_1} (b^*_i) \alpha_{g_2} (\alpha_{h_2} (y))\;  u(g_2 , h_2) u(g_2 h_2 , g_3) ]_{g_2h_2 g_3}\\
=& \; \abs{H}^{-1} \us {i,j} \sum [x u^*(g_1 h_1 ,g_2) u^* (g_1, h_1) \alpha_{g_1} (b_i)]_{g_1}\\
& \; \us N \otimes [E_N \left(b^*_i \alpha_{h_1}  \left(\alpha_{g_2} (\alpha_{h_2} (y))\;  u(g_2 , h_2) u(g_2 h_2 , g_3)\right) b_j \right) \alpha^{-1}_{h_1} (b^*_j)]_{g_2h_2 g_3}\\
=& \; \abs{H}^{-1} \us {j} \sum [x u^*(g_1 h_1 ,g_2) u^* (g_1, h_1) \alpha_{g_1} \left(\alpha_{h_1}  \left(\alpha_{g_2} (\alpha_{h_2} (y))\;  u(g_2 , h_2) u(g_2 h_2 , g_3)\right) b_j \right)]_{g_1} \us N \otimes [\alpha^{-1}_{h_1} (b^*_j)]_{g_2h_2 g_3}\\
=& \; \abs{H}^{-1} \us {j} \sum [x \; \alpha_{g_1h_1g_2} (\alpha_{h_2} (y)) \; \ul{u^*(g_1 h_1 ,g_2) u^* (g_1, h_1) \; \alpha_{g_1} \left( \alpha_{h_1} \left(  u(g_2 , h_2) u(g_2 h_2 , g_3) \right)  \right)} \; \alpha_{g_1} (b_j)]_{g_1}\\
& \; \us N \otimes [\alpha^{-1}_{h_1} (b^*_j)]_{g_2h_2 g_3}.
\end{align*}
Simplifying the underlined expression, we get
\begin{align*}
& \; u^*(g_1 h_1 ,g_2)  \; \alpha_{g_1 h_1}  \left(  u(g_2 , h_2) u(g_2 h_2 , g_3) \right) \; u^* (g_1, h_1)\\
= & \; \ol \omega(g_1h_1, g_2 , h_2)  \; u(g_1 h_1 g_2 , h_2) u^* (g_1 h_1,g_2h_2) \; \alpha_{g_1 h_1}  \left( u(g_2 h_2 , g_3) \right) \; u^* (g_1, h_1)\\
= & \; \ol \omega(g_1h_1, g_2 , h_2) \; u(g_1 h_1 g_2 , h_2) \; \ol \omega (g_1h_1 ,g_2 h_2, g_3) u(g_1h_1g_2h_2 , g_3) u^*(g_1h_1, g_2h_2g_3) \; u^* (g_1, h_1).
\end{align*}
This is exactly what we wanted from the right side acting on $ [x]_{g_1 h_1 g_2} \us N \otimes [y]_{g_3}$.

(ii) This follows from taking $ * $ on both sides.
\end{proof}