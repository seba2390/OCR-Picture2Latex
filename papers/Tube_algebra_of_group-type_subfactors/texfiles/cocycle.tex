\section{Some basics on group cocycles}

Let $G$ be a group with identity element $e$ and $\omega \in Z^3(G, S^1)$ be a 3-cocycle of $G$, that is, $\omega$ satisfies the following:
\begin{equation}\label{3coc}
\omega(g_1,g_2,g_3) \omega(g_1,g_2g_3,g_4)\omega(g_2,g_3,g_4) = \omega(g_1g_2,g_3,g_4)\omega(g_1,g_2,g_3g_4) \text{ for all } g_1,g_2,g_3,g_4 \in G
\end{equation}
We will use Equation \ref{3coc} at various instances in the article by denoting the particular elements of $G$ which will correspond to $g_1,g_2,g_3,g_4$ simply by $1,2,3,4$ respectively.
% {\bf Pending: Do we need this?}
Up to $3$-coboundary equivalence, we may consider  $\omega$ to be a normalized cocycle, i.e.,  $\omega(g_1, g_2, g_3) = 1$ whenever either $g_1, g_2$ or $g_3$ is $e$; namely, if $\varphi(g_1,g_2) = \omega(g_1, e, e) \overline{\omega}(e, e, g_2)$ for all $g_1, g_1 \in G$, then $(\partial \varphi) \omega$ is normalized.

For $a \in G$, let $G_a$  denote the centralizer subgroup of $a$.
The following result may be well-known to specialists but we include the statement for the sake of completeness.  
\begin{lem}\label{2coc}
	$G_a \times G_a \ni (g,h) \stackrel{\varphi_a}{\longmapsto} \overline{\omega}(a,g,h)\omega(g,a,h)\overline{\omega}(g,h,a) $ is a $2$-cocycle of $G_a$.
\end{lem}
\begin{proof}
Note that $\varphi_a (h_2,h_3)\ol{ \varphi}_a (h_1h_2,h_3) \varphi_a (h_1,h_2h_3)\ol{ \varphi}_a (h_1,h_2)$ contain twelve terms involving $\omega$.

The product of the four $\omega$-terms with $a$ in the first place is
\[
{\omega} (\us{1}{a},\us{2}{h_1},\us{3}{h_2}) {\omega} (\us{1}{a},\us{2~3}{h_1h_2},\us{4}{h_3}) \ol{ \omega} (\us{1}{a},\us{2}{h_1},\us{3~4}{h_2h_3}) \ol{\omega} (a,h_2,h_3) = {\omega} (ah_1,h_2,h_3) \ol{\omega} (h_1,h_2,h_3) \ol{\omega} (a,h_2,h_3),
\]
the product of the four $\omega$-terms with $a$ in the third place is
\[
\ol{\omega} (\us{2}{h_2},\us{3}{h_3},\us{4}{a}) \omega(\us{1~2}{h_1h_2}, \us{3}{h_3}, \us{4}{a}) \ol{\omega}(\us{1}{h_1},\us{2~3}{h_2h_3},\us{4}{a}) \omega(h_1,h_2,a) = \ol{\omega}(h_1,h_2,h_3a) \omega(h_1,h_2,h_3)  \omega(h_1,h_2,a),
\]
and the remaining product is
\begin{align*}
& \left[\omega(\us{2}{h_2},\us{3}{a},\us{4}{h_3}) \ol{\omega}(\us{1~2}{h_1h_2},\us{3}{a},\us{4}{h_3})\right] \left[\omega(\us{1}{h_1},\us{2}{a},\us{3~4}{h_2h_3}) \ol{\omega}(\us{1}{h_1},\us{2}{a},\us{3}{h_2})\right]\\
= ~ & \ol{\omega}(h_1,h_2,a)\ol{\omega}(h_1,h_2a,h_3)\omega(h_1,h_2,ah_3) \omega(h_1,ah_2,h_3) \omega(a,h_2,h_3) \ol{\omega}(h_1a,h_2,h_3).
\end{align*}
Now, since $h_1,h_2,h_3$ commute with $a$, all terms in the grand product cancel amongst each other.
%For $h_1, h_2, h_3 \in G_a$, we have
%\begin{align*}
%& \varphi_a (h_1,h_2)\varphi_a (h_1h_2,h_3)\\
%= & \left[\ol{\omega} (\us{1}{a},\us{2}{h_1},\us{3}{h_2}) \ol{\omega} (\us{1}{a},\us{2~3}{h_1h_2},\us 4 {h_3})\right] \left[\omega(\us 1 {h_1},a,h_2) \omega(h_1h_2,a,h_3)\right] \left[\ol{\omega}(h_1,h_2,a) \ol{\omega}(h_1h_2,h_3,a)\right]
%\end{align*}
\end{proof}
Instead of $a$, if we take $xax^{-1}$ for any $x \in G$, then it is natural to ask whether ${\varphi}_{xax^{-1}} \circ ({\mbox{Ad}}_x \times {\mbox{Ad}}_x) : G_a \times G_a \ra S^1$ is coboundarily equivalent to $\varphi_a$. The
answer is yes; however, we will prove not just this, but a slightly general formula which will be useful later.
\begin{prop}\label{go}
	For all $a,x,y\in G$, there exists ${\gamma}_{a,x,y}: G_a \ra S^1$ such that
	 \begin{align*}
	 & \ol{\omega}(xax^{-1}, xgy^{-1}, yhz^{-1}) \omega(xgy^{-1}, yay^{-1}, yhz^{-1}) \ol{\omega}(xgy^{-1}, yhz^{-1}, zaz^{-1})\\
	  =~ & \left[\gamma_{a,x,y}(g) \gamma_{a,y,z}(h) \ol{\gamma}_{a,x,z}(gh)\right] \varphi_a(g,h)
	 \end{align*}
for all $g,h \in G_a$.
Thus, $\gamma_{a,x,x}$ is a scalar $1$-cochain of $G_a$ which implements the coboundary equivalence between $\varphi_{xax^{-1}}\circ (\t{Ad}_x \times \t{Ad}_x)$ and $\varphi_a$.
\end{prop}
\begin{proof}
We write out the three terms in the L.H.S. of the equation in the statement one by one and expand them using Equation \ref{3coc}. 
In the successive steps, we just repeat the process, each time expanding the last term coming from the previous step. In the final step, some of the terms are decorated with numbers and strike-throughs, or underlines and alphabets, the explanation for which is given below. These are just elementary cocycle calculations which has been exhibited, down to the last detail. 
The first term is:\\
$\t{ }\t{ }\t{ }\ol{\omega}(xax^{-1}, xgy^{-1}, yhz^{-1})$\\
$=\omega(x, ax^{-1}, xghz^{-1}) \ol{\omega}(x, agy^{-1}, yhz^{-1}) \ol{\omega}(x, ax^{-1}, xgy^{-1}) \ol{\omega}(ax^{-1}, xgy^{-1}, yhz^{-1})$\\
$=\omega(x, ax^{-1}, xghz^{-1}) \ol{\omega}(x, agy^{-1}, yhz^{-1}) \ol{\omega}(x, ax^{-1}, xgy^{-1}) \omega(a, x^{-1}, xghz^{-1}) \ol{\omega}(a, x^{-1}, xgy^{-1})$\\
$\t{ }\t{ }\t{ }\ol{\omega}(x^{-1}, xgy^{-1}, yhz^{-1}) \ol{\omega}(a, gy^{-1}, yhz^{-1})$\\
$=\omega(x, ax^{-1}, xghz^{-1}) \ol{\omega}(x, agy^{-1}, yhz^{-1}) \ol{\omega}(x, ax^{-1}, xgy^{-1}) \omega(a, x^{-1}, xghz^{-1}) \ol{\omega}(a, x^{-1}, xgy^{-1})$\\
$\t{ }\t{ }\t{ }\ol{\omega}(x^{-1}, xgy^{-1}, yhz^{-1}) \ol{\omega}(ag, y^{-1}, yhz^{-1}) \omega(a,g,y^{-1}) \omega(g, y^{-1},yhz^{-1}) \ol{\omega}(a,g,hz^{-1})$\\
$=\us{A}{\underbracket{\omega(x, ax^{-1}, xghz^{-1})}} \cancelto{1}{\ol{\omega}(x, agy^{-1}, yhz^{-1})} \us{A}{\underbracket{\ol{\omega}(x, ax^{-1}, xgy^{-1})}} \us{B}{\underbracket{\omega(a, x^{-1}, xghz^{-1})}} \us{B}{\underbracket{\ol{\omega}(a, x^{-1}, xgy^{-1})}}$\\
$\t{ }\t{ }\t{ }\ol{\omega}(x^{-1}, xgy^{-1}, yhz^{-1}) \cancelto{2}{\ol{\omega}(ag, y^{-1}, yhz^{-1})} \us{C}{\underbracket{\omega(a,g,y^{-1})}} \cancelto{7}{\omega(g, y^{-1},yhz^{-1})} \cancelto{3}{\omega(ag, h, z^{-1})}$\\
$\t{ }\t{ }\t{ } \cancelto{6}{\ol{\omega}(g,h,z^{-1})} \us{C}{\underbracket{\ol{\omega}(a, gh,z^{-1})}} \boxed{\ol{\omega}(a,g,h)}$\\
The second term is: \\
$\t{ }\t{ }\t{ }\omega(xgy^{-1}, yay^{-1}, yhz^{-1})$\\
$= \omega(x, gy^{-1}, yay^{-1}) \omega(x, gay^{-1}, yhz^{-1}) \ol{\omega}(x, gy^{-1}, yahz^{-1}) \omega(gy^{-1}, yay^{-1}, yhz^{-1})$\\
$=\omega(x, gy^{-1}, yay^{-1}) \omega(x, gay^{-1}, yhz^{-1}) \ol{\omega}(x, gy^{-1}, yahz^{-1}) \omega(g, y^{-1}, yay^{-1}) \omega(y^{-1}, yay^{-1}, yhz^{-1})$\\
$\t{ }\t{ }\t{ }\ol{\omega}(g, y^{-1}, yahz^{-1}) \omega(g, ay^{-1}, yhz^{-1})$\\
$=\omega(x, gy^{-1}, yay^{-1}) \omega(x, gay^{-1}, yhz^{-1}) \ol{\omega}(x, gy^{-1}, yahz^{-1}) \omega(g, y^{-1}, yay^{-1}) \omega(y^{-1}, yay^{-1}, yhz^{-1})$\\
$\t{ }\t{ }\t{ }\ol{\omega}(g, y^{-1}, yahz^{-1}) \ol{\omega}(g, a, y^{-1}) \ol{\omega}(a, y^{-1}, yhz^{-1}) \omega(ga, y^{-1}, yhz^{-1}) \omega(g, a, hz^{-1})$\\
$=\us{F}{\underbracket{\omega(x, gy^{-1}, yay^{-1})}} \cancelto{1}{\omega(x, gay^{-1}, yhz^{-1})} \cancelto{8}{\ol{\omega}(x, gy^{-1}, yahz^{-1})} \us{E}{\underbracket{\omega(g, y^{-1}, yay^{-1})}} \omega(y^{-1}, yay^{-1}, yhz^{-1})$\\
$\t{ }\t{ }\t{ } \cancelto{5}{\ol{\omega}(g, y^{-1}, yahz^{-1})} \us{D}{\underbracket{\ol{\omega}(g, a, y^{-1})}} \us{B}{\underbracket{\ol{\omega}(a, y^{-1}, yhz^{-1})}} \cancelto{2}{\omega(ga, y^{-1}, yhz^{-1})} \us{C}{\underbracket{\omega(a, h, z^{-1})}} \cancelto{4}{\omega(g, ah, z^{-1})}$\\
$\t{ }\t{ }\t{ } \cancelto{3}{\ol{\omega}(ga, h, z^{-1})} \boxed{\omega(g, a, h)}$\\
The third term is: \\
$\t{ }\t{ }\t{ }\ol{\omega}(xgy^{-1}, yhz^{-1}, zaz^{-1})$\\
$=\omega(x, gy^{-1}, yhaz^{-1}) \ol{\omega}(x, gy^{-1}, yhz^{-1})
\ol{\omega}(x, ghz^{-1}, zaz^{-1}) \ol{\omega}(gy^{-1}, yhz^{-1}, zaz^{-1})$\\
$=\omega(x, gy^{-1}, yhaz^{-1}) \ol{\omega}(x, gy^{-1}, yhz^{-1})
\ol{\omega}(x, ghz^{-1}, zaz^{-1}) \omega(g, y^{-1}, yhaz^{-1})  \ol{\omega}(g, y^{-1}, yhz^{-1})$\\
$\t{ }\t{ }\t{ }\ol{\omega}(y^{-1}, yhz^{-1}, zaz^{-1}) \ol{\omega}(g, hz^{-1}, zaz^{-1})$\\
$=\omega(x, gy^{-1}, yhaz^{-1}) \ol{\omega}(x, gy^{-1}, yhz^{-1})
\ol{\omega}(x, ghz^{-1}, zaz^{-1}) \omega(g, y^{-1}, yhaz^{-1})  \ol{\omega}(g, y^{-1}, yhz^{-1})$\\
$\t{ }\t{ }\t{ }\ol{\omega}(y^{-1}, yhz^{-1}, zaz^{-1}) \omega(g, h, z^{-1}) \omega( h, z^{-1}, zaz^{-1}) 
\ol{\omega}(gh, z^{-1}, zaz^{-1}) \ol{\omega}(g, h, az^{-1})$\\
$=\cancelto{8}{\omega(x, gy^{-1}, yhaz^{-1})} \ol{\omega}(x, gy^{-1}, yhz^{-1})
\us{F}{\underbracket{\ol{\omega}(x, ghz^{-1}, zaz^{-1})}} \cancelto{5}{\omega(g, y^{-1}, yhaz^{-1})}  \cancelto{7}{\ol{\omega}(g, y^{-1}, yhz^{-1})}$\\
$\t{ }\t{ }\t{ }\ol{\omega}(y^{-1}, yhz^{-1}, zaz^{-1}) \cancelto{6}{\omega(g, h, z^{-1})} \us{E}{\underbracket{\omega( h, z^{-1}, zaz^{-1})}} 
\us{E}{\underbracket{\ol{\omega}(gh, z^{-1}, zaz^{-1})}} \us{D}{\underbracket{\omega(gh, a, z^{-1})}}$\\
$\t{ }\t{ }\t{ }\us{D}{\underbracket{\ol{\omega}(h, a, z^{-1})}} \cancelto{4}{\ol{\omega}(g, ha, z^{-1})} \boxed{\ol{\omega}(g, h, a)}$\\
Thus each $\omega$-term has been expressed as a product of $13$
$\omega$-terms. After combining these $39$ terms and noting that \\
(i) $8$ pairs of terms cancel since $g, h \in G_a$ (the cancellations have been marked with numbers for the reader's convenience),\\ 
(ii) the last (boxed) terms on the R.H.S of the three expressions above can be combined to yield $\varphi_a(g, h)$,\\ 
we are left with the following $20$ $\omega$-terms which have been grouped under A,B, C,D, E, F for reasons that will become apparent as we go along (namely, contribution towards defining the function $\gamma$ in the statement of the Proposition.):\\
\noindent
A terms: $= \ol{\omega}(x, ax^{-1}, xgy^{-1}) \omega(x, ax^{-1}, xghz^{-1})$\\
B terms: $\ol{\omega}(a, x^{-1}, xgy^{-1}) \ol{\omega}(a, y^{-1}, yhz^{-1}) \omega(a, x^{-1}, xghz^{-1})$\\
C terms:  $\omega(a,g,y^{-1})  \omega(a, h, z^{-1})  \ol{\omega}(a, gh,z^{-1})$\\
D terms: $\ol{\omega}(g, a, y^{-1}) \ol{\omega}(h, a, z^{-1}) \omega(gh, a, z^{-1}) $\\ 
E terms: $\omega(g, y^{-1}, yay^{-1}) \omega( h, z^{-1}, zaz^{-1}) \ol{\omega}(gh, z^{-1}, zaz^{-1})$\\
F terms: $\omega(x, gy^{-1}, yay^{-1})  \ol{\omega}(x, ghz^{-1}, zaz^{-1})$\\

\noindent The remaining 4 terms are:
\begin{equation}\label{remain}
\ol{\omega}(x^{-1}, xgy^{-1}, yhz^{-1}) \omega(y^{-1}, yay^{-1}, yhz^{-1}) \ol{\omega}(x, gy^{-1}, yhz^{-1}) \ol{\omega}(y^{-1}, yhz^{-1}, zaz^{-1})
\end{equation}

\noindent The second and fourth terms in expression \ref{remain} are again broken up using Equation \ref{3coc} as follows:\\
$\omega(y^{-1}, yay^{-1}, yhz^{-1}) = \ol{\omega}(y^{-1}, y, ay^{-1}) \us{A}{\underbracket{\ol{\omega}(y, ay^{-1}, yhz^{-1})}} \cancelto{9}{\omega(y^{-1}, y, ahz^{-1})}$\\
$\ol{\omega}(y^{-1}, yhz^{-1}, zaz^{-1}) = \us{G}{\underbracket{\omega(y^{-1}, y, hz^{-1})}} \us{F}{\underbracket{\omega(y, hz^{-1}, zaz^{-1})}} \cancelto{9}{\ol{\omega}(y^{-1}, y, haz^{-1})} $\\
and the first and the third terms in \ref{remain} taken together, is:\\
$\ol{\omega}(x^{-1}, xgy^{-1}, yhz^{-1}) \ol{\omega}(x, gy^{-1}, yhz^{-1}) = \us{G}{\underbracket{\omega(x^{-1}, x, gy^{-1})}} \us{G}{\underbracket{\ol{\omega}(x^{-1}, x, ghz^{-1})}}$

We now expand each of the terms in E, using Equation \ref{3coc} again:\\
\begin{tabular}{llll}
& $\omega(g, y^{-1}, yay^{-1})$ & $\omega(h, z^{-1}, zaz^{-1})$ & $\ol{\omega}(gh, z^{-1}, zaz^{-1})$\\
= & $\ol{\omega}(gy^{-1}, y, ay^{-1})$ &$\omega(g, y^{-1}, y)$ & $\omega(y^{-1}, y, ay^{-1})$\\
&$\ol{\omega}(hz^{-1}, z, az^{-1})$ & $\omega(h, z^{-1}, z)$ & $\cancelto{10}{\omega(z^{-1}, z, az^{-1})}$ \\
&  $\omega(ghz^{-1}, z, az^{-1})$ & $\ol{\omega}(gh, z^{-1}, z)$ & $\cancelto{10}{\ol{\omega}(z^{-1}, z, az^{-1})}$ \\
\end{tabular}\\
Call the terms in the first column as $E_1$ and the terms in the second column as $E_2$. The new A and F terms that popped up from breaking down \ref{remain}, and the $E_1$ and $E_2$ terms are added to the existing list:\\
\noindent
A terms: $= \ol{\omega}(x, ax^{-1}, xgy^{-1}) \ol{\omega}(y, ay^{-1}, yhz^{-1}) \omega(x, ax^{-1}, xghz^{-1})$\\
B terms: $\ol{\omega}(a, x^{-1}, xgy^{-1}) \ol{\omega}(a, y^{-1}, yhz^{-1}) \omega(a, x^{-1}, xghz^{-1})$\\
C terms:  $\omega(a,g,y^{-1})  \omega(a, h, z^{-1})  \ol{\omega}(a, gh,z^{-1})$\\
D terms: $\ol{\omega}(g, a, y^{-1}) \ol{\omega}(h, a, z^{-1}) \omega(gh, a, z^{-1}) $\\ 
$E_1$ terms: $\ol{\omega}(gy^{-1}, y, ay^{-1}) \ol{\omega}(hz^{-1}, z, az^{-1}) \omega(ghz^{-1}, z, az^{-1})$\\
$E_2$ terms: $\omega(g, y^{-1}, y) \omega(h, z^{-1}, z) \ol{\omega}(gh, z^{-1}, z)$\\
F terms: $\omega(x, gy^{-1}, yay^{-1})  \omega(y, hz^{-1}, zaz^{-1}) \ol{\omega}(x, ghz^{-1}, zaz^{-1})$\\
G terms: $\omega(x^{-1}, x, gy^{-1}) \omega(y^{-1}, y, hz^{-1}) \ol{\omega}(x^{-1}, x, ghz^{-1})$\\

 Thus we define ${\gamma}_{a,x,y}(g) = \ol{\omega}(x, ax^{-1}, xgy^{-1}) \ol{\omega}(a, x^{-1}, xgy^{-1}) \omega(a,g,y^{-1}) \ol{\omega}(g, a, y^{-1})$\\
 $\ol{\omega}(gy^{-1}, y, ay^{-1}) \omega(g, y^{-1}, y) \omega(x, gy^{-1}, yay^{-1}) \omega(x^{-1}, x, gy^{-1})$. \\
 This fits the bill.
\end{proof}