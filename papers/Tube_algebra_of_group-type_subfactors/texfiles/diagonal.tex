\section{Diagonal Subfactors}\label{diag}

In this section, we will describe the affine module category of the planar algebra of a diagonal subfactor associated associated to a `$ G $-kernel' where $ G $ is a finitely generated discrete group.
Recall that $ G $\textit{-kernel} is simply an injective homomorphism $ \chi : G \ra \t {Out} (N) $ where $ N $ is a $ II_1 $ factor.
If $ \alpha : G \ra \t {Aut} (N) $ is a lift of $ \chi $ (that is, $ \chi (g) = \alpha_g \; \t {Inn} (N)$ for all $ g \in G $) and $ I $ is a set of generators of $ G $, then the associated diagonal subfactor is given by
\[
N \ni x \hookrightarrow \t{diag} (\alpha_i (x))_{i\in I} \in M_I (N) =: M.
\]
Further for $ g_1,g_2 \in G$, we may choose $ u(g_1,g_2) \in \mcal U (N) $ such that $ \alpha_{g_1} \alpha_{g_2} = \t {Ad}_{u(g_1,g_2)} \alpha_{g_1g_2} $ for all $ g_1 , g_2 \in G $.
Associativity of multiplication in $ G $ gives us a $ 3 $-cocycle $ \omega : G^{ \times 3} \ra S^1 $ such that
\begin{equation}\label{uomega}
u(g_1, g_2) \; u(g_1 g_2, g_3) = \omega (g_1, g_2, g_3) \;  \alpha_{g_1} (u(g_2, g_3)) \;  u(g_1, g_2 g_3)
\end{equation}
for $ g_1, g_2, g_3 \in G $.
One may easily  check that the coboundary class of the $ 3 $-cocyle $\omega  $ in $ H^3 (G, S^1) $ does not depend on the choice of the lift $ \alpha $ and the unitaries $ u(\cdot , \cdot) $; this class is referred as the obstruction of the $ G $-kernel $ \chi $.
It is well-known (see \cite{frenchpop}) that the standard invariant of the above subfactor $ N \subset M $ depends only on the group $G$, its generators and the $3$-cocycle obstruction.

We will find the tube algebra of the category $ \mcal C_{NN} $ of $ N $-$ N $ bifinite bimodules coming from this subfactor and then find the tube representations.
Note that this will suffice since, by \cite{GJ}, the representation category of the tube algebra of $ \mcal C_{NN} $ is (tensor) equivalent to the category of annular representations with respect to any full weight set in ob$ (\mcal C_{NN}) $ (in particular, $ \left\{\left(_N L^2(M)_N \right)^{\us N \otimes k} : k\in \N\right\} $ which gives the affine modules of Jones).
In fact, if $\omega \equiv 1$, the affine modules were obtained in \cite{GJ}.

All simple objects in $ \mcal C_{N,N} $ are invertible.
This is clear because $ L^2(M) \us {N\t{-}N } \cong  \us {i\in I} \bigoplus L^2( N_{\alpha_i} ) $. Here the notation $_N L^2 (N_\theta)_N $ (for $ \theta \in \t{Aut} (N) $) denotes the bimodule obtained from the Hilbert space $ L^2 (N) $ with left $ N $-action being the usual left multiplication whereas the right one is twisted by $ \theta $. This bimodule depends only on the class defined by $ \theta $ in $ \t{Out} (N)$ up to isomorphism, and the tensor $ \us N \otimes $ and the contragradient of such bimodules correspond to multiplication and inverse in $ \t{Out} (N)$.
Since $ \t{id}_N $ is in the set $\{\alpha_i : i\in I\}  $, we get all such index one bimodules corresponding to any $ g \in G $, appearing as sub-bimodules of $ \left(_N L^2(M)_N \right)^{\us N \otimes k}  $ as we vary $ k $.
Moreover, up to isomorphism these are the only irreducible bimodules of $\mcal C _{N N} $.
Thus, the fusion algebra of $ \mcal C_{NN} $ is just given by $ G $.
It is then easy to verify that $ \mcal C_{NN} $ is tensor equivalent to the category $ \t {Vec} (G, \omega) $ of $ G $-graded vector spaces with associativity constraint given by the $ 3 $-cocycle obstruction $ \omega $.
So, our job boils down to finding out the tube representations of $ \t {Vec} (G, \omega) $.
However, we will work with bimodules in $ \mcal C_{NN} $ instead, as the framework will be useful in the next section.

Since the standard invariant (and thereby the category $ \mcal C_{NN} $) is independent of the lift $ \alpha $, without loss of generality we assume
%Consider a representative map $ G \ni g \os \alpha \mapsto \alpha_g \in \t {Aut} (N) $ such that $ \alpha_g $ goes to $ g  $ under the quotient map over inner automorphisms and 
$ \alpha_e = \t {id}_N $.
Further, we may set $ u(g_1,e) = 1_N = u(e,g_2) $ for all $ g_1 , g_2 \in G $.
These assumptions make the $ 3 $-cocycle $ \omega : G^{ \times 3} \ra S^1 $ normalized.
%%%%%%%%%%%%%%%%%%%%%%%%
\comments{
\hrule
In this section, we will describe the affine module category of the planar algebra associated to the diagonal subfactor associated to the finite set $I$ indexing a subset $\{\alpha_i\}_{i\in I}$ (containing the identity) of the automorphism group of a $II_1$ factor $N$, that is,
\[
N \ni x \hookrightarrow \t{diag} (\alpha_i (x))_{i\in I} \in M_I (N) =: M.
\]
It is well known (see \cite{frenchpop}) that the standard invariant of this subfactor depends only on the group $G$ generated by $\{g_i := [\alpha_i] / {i\in I}\}$ in $\t{Out}(N)$ under the quotient map and a scalar $3$-cocycle $\omega$ of $G$.
\comments{
In this section, we will describe the tube algebra and its representations of the diagonal subfactor associated to the finite set $I$ indexing a subset $\{\alpha_i\}_{i\in I}$ (containing the identity) of the automorphism group of a $II_1$ factor $N$, that is,
\[
N \ni x \hookrightarrow \t{diag} (\alpha_i (x))_{i\in I} \in M_I (N) =: M.
\]
It is well known (see \cite{frenchpop}) that the standard invariant of this subfactor depends only on the group $G$ generated by $\{g_i := [\alpha_i] / {i\in I}\}$ in $\t{Out}(N)$ under the quotient map and a scalar $3$-cocycle $\omega$ of $G$.
%If $\omega \equiv 1$, the affine modules were obtained in the context of unshaded planar algebras in ??.
}
%In order to find the affine module category of the planar algebra of $ N \subset M $, 
We will find the tube algebra of the category $ \mcal C_{NN} $ of $ N $-$ N $ bifinite bimodules coming from this subfactor and then find the tube representations.
Note that this will suffice since, by \cite{GJ}, the representation category of the tube algebra of $ \mcal C_{N,N} $ is (tensor) equivalent to the representation category of the annular algebroid with respect to any full weight set in ob$ (\mcal C_{NN}) $ (in particular, $ \left\{\left(_N L^2(M)_N \right)^{\us N \otimes k} : k\in \N\right\} $ which gives the affine modules of Jones).
In fact, if $\omega \equiv 1$, the affine modules were obtained in \cite{GJ}.

All simple objects in $ \mcal C_{N,N} $ are invertible.
This is clear because $ L^2(M) \us {N\t{-}N } \cong  \us {i\in I} \bigoplus L^2( N_{\alpha_i} ) $. Here the notation $_N L^2 (N_\theta)_N $ (for $ \theta \in \t{Aut} (N) $) denotes the bimodule obtained from the Hilbert space $ L^2 (N) $ with left $ N $-action being the usual left multiplication whereas the right one is twisted by $ \theta $; up to isomorphism, this bimodule depends only on the class defined by $ \theta $ in $ \t{Out} (N)$, and the tensor $ \us N \otimes $ and the contragradient of such bimodules correspond to multiplication and inverse in $ \t{Out} (N)$.
Since $ \t{id}_N $ is in the set $\{\alpha_i : i\in I\}  $, we get all such index one bimodules corresponding to any $ g \in G $, appearing as sub-bimodules of $ \left(_N L^2(M)_N \right)^{\us N \otimes k}  $ as we vary $ k $.
Moreover, up to isomorphism these are the only irreducible bimodules of $\mcal C _{N N} $.
Thus, the fusion algebra of $ \mcal C_{NN} $ is just given by $ G $.
It is then easy to verify that $ \mcal C_{NN} $ is tensor equivalent to the category $ \t {Vec} (G, \omega) $ of $ G $-graded vector spaces with associativity constraint given by the $ 3 $-cocycle obstruction $ \omega $.
So, our job boils down to finding out the tube representations of $ \t {Vec} (G, \omega) $.
However, we will work with bimodules in $ \mcal C_{NN} $ instead since the framework will be useful in the next section.

Consider a representative map $ G \ni g \os \alpha \mapsto \alpha_g \in \t {Aut} (N) $ such that $ \alpha_g $ goes to $ g  $ under the quotient map over inner automorphisms and $ \alpha_e = \t {id}_N $.
For $ g_1,g_2 \in G$, choose $ u(g_1,g_2) \in \mcal U (N) $ such that $ \alpha_{g_1} \alpha_{g_2} = \t {Ad}_{u(g_1,g_2)} \alpha_{g_1g_2} $ and $ u(g_1,e) = 1_N = u(e,g_2) $ for all $ g_1 , g_2 \in G $.
Associativity of multiplication in $ G $ gives us a normalized $ 3 $-cocycle $ \omega : G^{ \times 3} \ra S^1 $ such that

\begin{equation}\label{uomega}
u(g_1, g_2) \; u(g_1 g_2, g_3) = \omega (g_1, g_2, g_3) \;  \alpha_{g_1} (u(g_2, g_3)) \;  u(g_1, g_2 g_3).
\end{equation}
\hrule}%%%%%%%%%%%%%%%%%%%%%%%%





For $ g \in G $, let $ X_g := _N L^2 (N_{\alpha_g})_N$.
The morphism space in $ \mcal C_{NN} $ from object $ U $ to object $ V $, will be denoted by $ \mcal C_{NN} (U,V) $.
The tube morphism from $ X_{g_1} $ to $ X_{g_2} $ is then given by $ \mcal T_{g_1, g_2} := \us {s\in G} \bigoplus \mcal T^s_{g_1,g_2}$ where $\mcal T^s_{g_1,g_2} = \mcal C_{NN} (X_{g_1} \otimes X_s , X_s \otimes X_{g_2}  )$.
Clearly, $ \mcal T_{g_1,g_2}  \neq \{0\}$ if and only if $ g_1 $ and $ g_2 $ are conjugates of each other.
Further, if $ g_1 = s g_2 s^{-1} $, then $ \mcal T^s_{g_1, g_2} $ is one-dimensional; we will fix a distinguished element in this space, namely, $ a(g_1 , s , g_2) $ defined by
\[
X_{g_1} \us N \otimes X_s \; \ni \; [1]_{g_1} \us N \otimes [1]_s \; \os {a(g_1,s,g_2)} \longmapsto \; [u(g_1 , s) u^*(s, g_2)]_s \us N \otimes [1]_{g_2} \; \in \; X_s \us N \otimes X_{g_2}.
\]
It is an easy exercise to check that the above map is indeed an $ N $-$ N $-linear unitary.

Before we multiply two nonzero tube morphisms $ a(g_1 , s , g_2) $ and $ a(g_2, t , g_3) $, we need to know the one dimensional spaces $ \mcal C_{NN} (X_s \us N \otimes X_t , X_{st})  = \C \left\{[1]_s \us N \otimes [1]_t \os {\beta_{s,t}} \longmapsto [u(s,t) ]_{st}\right\}$ and $ \mcal C_{NN} (X_{st} , X_s \us N \otimes X_t) = \C \left\{ [1]_{st} \os {\beta^*_{s,t}} \longmapsto  [u^* (s,t)]_s \us N \otimes [1]_t \}\right\}$.
Following the multiplication defined in \cite[Section 3]{GJ}, we have
\[
a(g_2,t,g_3) \cdot a(g_1,s,g_2) = \beta_{s,t} \us N \otimes \t {id}_{X_{g_3}} \; \circ \; \t {id}_{X_s} \us N \otimes a(g_2 ,t , g_3) \; \circ \; a(g_1, s, g_2) \us N \otimes \t {id}_{X_t} \; \circ \; \t {id}_{X_{g_1}} \us N \otimes \beta^*_{s,t} .
\]
Right from the definitions, one can easily see that $ a(g_2,t,g_3) \cdot a(g_1,s,g_2) $ sends $ [1]_{g_1} \us N \otimes [1]_{st} $ to
\[
\left[ \alpha_{g_1} (u^* (s,t)) \; u(g_1,s) \; u^* (s,g_2) \; \alpha_s \left( u(g_2, t) u^* (t,g_3) \right) \; u(s,t) \right]_{st} \us N \otimes [1]_{g_3}
\]
Now,
\begin{align*}
&\; \alpha_{g_1} (u^* (s,t)) \; u(g_1,s) \; u^* (s,g_2) \; \alpha_s \left( u(g_2, t) u^* (t,g_3) \right) \; u(s,t)\\
= &\;  \omega (g_1,s,t) \; u(g_1,st) \; u^* (\us {=s g_2} {g_1 s} ,t) \; u^* (s,g_2) \; \alpha_s ( u(g_2, t) )  \; \alpha_s ( u^* (t,g_3) ) \; u(s,t)\\
&\; \t {(using Equation \ref{uomega} on the first two terms)}\\
= &\;  \omega (g_1,s,t) \; u(g_1,st) \; \ol \omega (s , g_2 , t) \; u^* (s, \us {= t g_3} {g_2 t})  \; \alpha_s ( u^* (t,g_3) ) \; u(s,t)\\
&\; \t {(using Equation \ref{uomega} on the third, fourth and fifth terms)}\\
= &\;  \omega (g_1,s,t) \; u(g_1,st) \; \ol \omega (s , g_2 , t)  \; \omega (s,t,g_3) \; u^*(st,g_3)\\
&\; \t {(using Equation \ref{uomega} on the last three terms)}\\
= &\; \left[\omega (g_1,s,t) \; \ol \omega (s , g_2 , t)  \; \omega (s,t,g_3)\right] \; u(g_1,st) \; u^*(st,g_3)
\end{align*}
Thus, multiplication is given by
\[
a(g_2,t,g_3) \cdot a(g_1,s,g_2) =  \left[\omega (g_1,s,t) \; \ol \omega (s , g_2 , t)  \; \omega (s,t,g_3)\right] a(g_1 , st , g_3).
\]

Next we will obtain the $ * $-structure on the tube algebra which we denote by $ \# $ following the notation in \cite{GJ}.
For this, we need a standard solution to the conjugate equations for the pair $ (X_s , X_{s^{-1}}) $.
We set $ R_s := \beta^*_{s^{-1},s} : X_e \ra X_{s^{-1}} \us N \otimes X_s$ and $ \ol R_s := \left[ \omega (s, s^{-1} , s) \beta^*_{s, s^{-1}} \right] : X_e \ra X_s \us N \otimes X_{s^{-1}} $.
It is completely routine to check that $ (R_s , \ol R_s) $ satisfies the conjugate equation and is standard.
Now by \cite{GJ},
\[
\left( a(g_1, s , g_2)\right)^\# = \left[\t {id}_{X_{s^{-1}} } \us N \otimes \t {id}_{X_{g_1}} \us N \otimes (\ol R_s)^*\right] \circ \left[\t {id}_{X_{s^{-1}} } \us N \otimes (a(g_1,s,g_2))^* \us N \otimes \t {id}_{X_{s^{-1}} }\right] \circ \left[R_s \us N \otimes \t {id}_{X_{g_2} } \us N \otimes \t {id}_{X_{s^{-1}} }\right].
\]
The map $ (a(g_1 , s , g_2))^* $ sends $ [1]_s \us N \otimes [1]_{g_2} $ to $ [u(s,g_2) u^* (g_1,s)]_{g_1} \us N \otimes [1]_s $.
Using all the three maps $ (a(g_1 , s , g_2))^* $, $ R_s $ and $ \ol R_s $, we can express the image of $ [1]_{g_2} \us N \otimes [1]_{s^{-1}} $ under $ \left( a(g_1, s , g_2)\right)^\# $ as
\[
\left[ \us {= \omega (s^{-1} , s , s^{-1})} {\ol \omega (s , s^{-1} , s)} \; u^*(s^{-1} , s) \; \alpha_{s^{-1}} \left( u(s,g_2) u^* (g_1 , s) \right) \; \alpha_{s^{-1}} (\alpha_{g_1} (u(s,s^{-1}))) \right]_{s^{-1}} \us N \otimes [1]_{g_1}.
\]
We will simplify the first tensor component in the following way:
\begin{align*}
& \; \omega (s^{-1} , s , s^{-1}) \; u^*(s^{-1} , s) \; \alpha_{s^{-1}} ( u(s,g_2) ) \; \alpha_{s^{-1}} \left( u^* (g_1 , s) \alpha_{g_1} (u(s,s^{-1}))\right)\\
= & \; \omega (s^{-1} , s , s^{-1}) \; \ol \omega (s^{-1},s,g_2) \; u^* (s^{-1}, sg_2) \; \ol \omega (g_1 , s , s^{-1}) \; \alpha_{s^{-1}} (u(\us {=sg_2} {g_1s} , s^{-1}))\\
& \; \t{(using Equation \ref{uomega} on the second and third, and fourth and fifth terms separately)}\\
= & \; \omega (s^{-1} , s , s^{-1}) \; \ol \omega (s^{-1},s,g_2) \; \ol \omega (g_1 , s , s^{-1}) \; \ol \omega (s^{-1} , s g_2 , s^{-1}) \; u(g_2 , s^{-1}) \; u^* (s^{-1} , g_1)\\
%= & \; \omega (s^{-1} , s , s^{-1}) \; \ol \omega (s^{-1},s,g_2) \; \ol \omega (g_1 , s , s^{-1}) \; \ol \omega (s^{-1} , s g_2 , s^{-1}) \; u(g_2 , s^{-1}) \; u^* (s^{-1} , g_1)\\
& \; \t{(using Equation \ref{uomega} on the third and fifth terms)}\\
= & \; \omega (s^{-1} , s , s^{-1}) \; \ol \omega (g_1 , s , s^{-1}) \; \omega (s, g_2, s^{-1}) \; \ol \omega (s^{-1},s, \us {=s^{-1} g_1} {g_2 s^{-1}}) \; u(g_2 , s^{-1}) \; u^* (s^{-1} , g_1)\\
& \; \t{(using Equation \ref{3coc} on the second and fourth terms)}\\
= & \; \left(\ol \omega (g_1 , s , s^{-1}) \; \omega (s, g_2, s^{-1}) \; \ol \omega (s , s^{-1}, g_1)\right) \; \left(u(g_2 , s^{-1}) \; u^* (s^{-1} , g_1)\right)\\
& \; \t{(using Equation \ref{3coc} on the first and fourth terms)}
\end{align*}
Hence, $ \# $ is given by the formula:
\[
\left(a(g_1 , s , g_2)\right)^\# = \left[\ol \omega (g_1 , s , s^{-1}) \; \omega (s, g_2, s^{-1}) \; \ol \omega (s , s^{-1}, g_1)\right] \; a(g_2, s^{-1} , g_1).
\]

The canonical (faithful) trace $ \Omega $ on the tube algebra (as defined in \cite{GJ}) is given by $ \Omega (a(g_1,s,g_2)) = \delta_{g_1 = g_2} \delta_{s=e}$.
Thus, the set $ \{a(g_1,s,g_2) : g_1,g_2,s \in G \t{ satisfying } g_1 s = s g_2 \}$ becomes an orthonormal basis with respect to the inner product arising from $ \Omega $ and $ \# $.

To have a better understanding of the $ * $-algebra structure of the tube algebra, we will now set up some notations.
Let $ \mscr C $ denote the set of conjugacy classes of $ G $.
For each $ C \in \mscr C $, we pick a representative $ g_C \in C $ and for each $ g \in C $, we fix $ w_g \in G $ such that $ g = w_g \; g_C \; w^{-1}_g $ and $ w_{g_C} = e$.
Also for $ C \in \mscr C $, we will denote the centralizer subgroup of $ g_C $ by $ G_C := \{ s \in G : g_C = s \; g_C \; s^{-1}\} $, and $ \vphi_C$ will denote the $ 2 $-cocycle on $ G_C $ given by $\vphi_C (s,t) := \ol \vphi_{g_C} (t^{-1} , s^{-1} ) $ (recall the definition of $ \vphi_{g_C} $ in Lemma \ref{2coc}).

With the above notation, we give an alternate description of $ * $-algebra structure of the tube algebra in the following proposition which will be handy in classifying the representations.
\begin{thm}\label{diagtubalg}
(i) The tube algebra $ \mcal T  = (( \mcal T_{g_1 , g_2} ))_{\t {fin. supp.}}$ is isomorphic to $ \us {C \in \mscr C} \bigoplus \; M_C \otimes \left[\C G_C\right]_{\vphi_C} $ as a $ * $-algebra where $ \left[\C G_C\right]_{\vphi_C}  $ is the $ 2 $-cocycle twisted group algebra and $ M_C $ denotes the $ * $-algebra of finitely supported matrices whose rows and columns are indexed by elements of $ C $.\\
(ii) Every Hilbert space representation $\Pi : \mcal T \ra \mcal L (V)$ decomposes over $ C \in \mscr C $ uniquely (up to isomorphism) as an orthogonal direct sum of submodules generated by the range of the projection $\Pi (a(g_C, e, g_C)) $ (which is the $ {g_C}^{\t {th}} $-space of $ V $).
(We will call a representation of $ \mcal T $ `supported on $ C \in \mscr C$' if it is generated by its vectors in the $ {g_C}^{\t {th}} $-space.)
The category of $ C $-supported representations of $ \mcal T $ is additively equivalent to representation category of $ [\C G_C]_{\vphi_C} $.
\end{thm}
\begin{proof}
(i) We will send the orthonormal basis of $ \mcal T $ (discussed above) via a map $ \Phi $ to a canonical basis of $ \us {C \in \mscr C} \bigoplus \; M_C \otimes \left[\C G_C\right]_{\vphi_C}  $ in the follwoing way:\\
for $ g_1 , g_2 \in C $  and $ s\in G $ such that $ g_1 s = s g_2 $ (implying  $ w^{-1}_{g_1} s w_{g_2} \in G_C $)
\[
a(g_1 , s , g_2) \; \os \Phi \longmapsto \; \ol \gamma_{g_C , w_{g_1} , w_{g_2}} (w^{-1}_{g_1} s w_{g_2}) \; E_{g_2,g_1} \otimes [w^{-1}_{g_2} s^{-1} w_{g_1}] 
\]
where we use the family of functions $\left\{ \gamma_{a,x,y}: G_a \ra S^1  \right\}_{a,x,y\in G}$ appearing in Proposition \ref{go}.

To show $ \Phi $ preserves multiplication, consider
\begin{align*}
& \; \Phi (a(g_2,t,g_3)) \; \Phi (a(g_1,t,g_2))\\
= & \; \left[\ol \gamma_{g_C , w_{g_2} , w_{g_3}} (w^{-1}_{g_2} t w_{g_3}) \; E_{g_3,g_2} \otimes [w^{-1}_{g_3} t^{-1} w_{g_2}]\right] \; \left[\ol \gamma_{g_C , w_{g_1} , w_{g_2}} (w^{-1}_{g_1} s w_{g_2}) \; E_{g_2,g_1} \otimes [w^{-1}_{g_2} s^{-1} w_{g_1}]\right]\\
= & \; \ol \gamma_{g_C , w_{g_1} , w_{g_2}} (w^{-1}_{g_1} s w_{g_2}) \; \ol \gamma_{g_C , w_{g_2} , w_{g_3}} (w^{-1}_{g_2} t w_{g_3}) \; \vphi_C (w^{-1}_{g_3} t^{-1} w_{g_2} \; , \; w^{-1}_{g_2} s^{-1} w_{g_1} ) \; E_{g_3,g_1} \otimes [w^{-1}_{g_3} (st)^{-1} w_{g_1}]\\
= &  \; \ol \gamma_{g_C , w_{g_1} , w_{g_2}} (w^{-1}_{g_1} s w_{g_2}) \; \ol \gamma_{g_C , w_{g_2} , w_{g_3}} (w^{-1}_{g_2} t w_{g_3}) \;  \gamma_{g_C , w_{g_1} , w_{g_3}} (w^{-1}_{g_1} st w_{g_3}) \; \ol \vphi_{g_C} (w^{-1}_{g_1} s w_{g_2} \; , \; w^{-1}_{g_2} t w_{g_3} )\\
& \; \Phi (a(g_1 , st, g_3))\\
= & \; \omega (g_1 ,s , t) \ol \omega (s,g_2 , t) \omega (s , t, g_3) \; \Phi (a(g_1 , st, g_3)) \t { (using Proposition \ref{remain})}\\
= & \; \Phi \left(a(g_2 , t, g_3) \; a(g_1 , s, g_2)\right).
\end{align*}
The map $ \Phi $ is $ * $ preserving because $ \left[ \Phi ( a(g_1,s,g_2) ) \right]^*
 $
\begin{align*}
= & \; \gamma_{g_C, w_{g_1} , w_{g_2} } (w^{-1}_{g_1} s w_{g_2}) \; \ol \vphi_C ( w^{-1}_{g_1} s w_{g_2} , w^{-1}_{g_2} s^{-1} w_{g_1} ) \; E_{g_1, g_2} \otimes [w^{-1}_{g_1} s w_{g_2}]\\
= & \; \gamma_{g_C, w_{g_1} , w_{g_2} } (w^{-1}_{g_1} s w_{g_2}) \; \vphi_{g_C} ( w^{-1}_{g_1} s w_{g_2} , w^{-1}_{g_2} s^{-1} w_{g_1} ) \; \gamma_{g_C , w_{g_2} , w_{g_1} } ( w^{-1}_{g_2} s^{-1} w_{g_1}) \; \Phi (a(g_2,s^{-1},g_1))\\
= & \; \gamma_{g_C , w_{g_1} , w_{g_1} } (e) \; \ol \omega (g_1 , s , s^{-1}) \; \omega (s , g_2 , s^{-1}) \; \ol \omega (s , s^{-1} , g_1) \; \Phi (a(g_2 , s^{-1} , g_1)) \; \t {(using Proposition \ref{remain})}\\
=& \; \Phi \left( \left[  a(g_1,s,g_2)  \right]^\# \right)
\end{align*}
where we use $ \gamma_{g_C , w_{g_1} , w_{g_1} } (e) =1$ at the very last step which follows directly from the definition of $ \gamma_{a,x,y} $ in the proof of Proposition \ref{remain}.

(ii) The decomposition follows easily from the $ * $-algebra structure described in part (i).

Fix $ C\in \mscr C $. If $ \Pi: \mcal T \ra \mcal L (W) $ is $ C $-supported, then we can define the representation $ \pi : [\C G_C]_{\vphi_C} \ra \mcal L (W_{g_C}) $ defined by $ \pi (s) = \Pi\left(\Phi^{-1} (E_{g_C , g_C} \otimes [s])\right) $.
Conversely, if $ \pi : [\C G_C]_{\vphi_C} \ra \mcal L (U) $ is a representation, then one can consider the unique extension 
\[
\Pi : \mcal T \ra \mcal L (l^2 (C) \otimes U) \; \t { defined by } \; \Pi \left( \Phi^{-1} (E_{g_1 , g_2} \otimes [s] ) \right) := \delta_{g_1 \in C} \; E_{g_1,g_2} \otimes \pi (s) .
\]
\end{proof}
\begin{rem}
Note that the canonical trace $ \Omega $ on $ \mcal T $ corresponds to the direct sum of the canonical traces on $ M_C \otimes [ \C G_C]_{\vphi_C} $.
Also, the $ * $-algebra $ \mcal T_{e,e} $ (by definition) is isomorphic to the fusion algebra which is basically the group algebra $\C G $ without any nontrivial $ 2 $-cocycle twist (since $ \vphi_{e}$ is the constant function $1$ which follows from its definition in Lemma \ref{2coc}).
Thus, the analytic properties (such as, amenability, Haagerup, property (T)) of the bimodule category corresponding to the subfactor $ N \subset M $ corresponds exactly to that of the group $ G $; this fact was obtained by Sorin Popa long time back in \cite{frenchpop} and \cite{Pop94}.
However, the analytic properties in the higher weight spaces (as defined in \cite{GJ}) depend on the corresponding centralizer subgroup.
%\textbf{PENDING}
\end{rem}
%%%%%%%%%%%%%%%%%%%%%%%%%%%%%%%%%%%%%%%%%%%%%%%%%%%%%%%%
\comments{\begin{thm}\label{diagthm}
Every Hilbert representation $ V $ of $ \mcal T $ decomposes uniquely (up to isomorphism) as an orthogonal direct sum of submodules $ V^C := \lab V_{g_C} \rab$ for $ C \in \mscr C $.
(We will call a representation of $ \mcal T $ `supported on $ C \in \mscr C$' if it is generated by its vectors in the $ {g_C}^{\t {th}} $-space.)
The category of $ C $-supported representations of $ \mcal T $ is additively equivalent to representation category of $ [\C G_C]_{\vphi_C} $.
\end{thm}
\begin{proof}
The decomposition follows easily from the $ * $-algebra structure described in Proposition \ref{diagtubalg}.

For the second part, fix $ C\in \mscr C $. If $ \Pi: \mcal T \ra \mcal L (W) $ is $ C $-supported, then we can define the representation $ \pi : [\C G_C]_{\vphi_C} \ra \mcal L (W_{g_C}) $ defined by $ \pi (s) = \Pi\left(\Phi^{-1} (E_{g_C , g_C} \otimes [s])\right) $.
Conversely, if $ \pi : [\C G_C]_{\vphi_C} \ra \mcal L (U) $ is a representation, then one can consider the unique extension 
\[
\Pi : \mcal T \ra \mcal L (l^2 (C) \otimes U) \; \t { defined by } \; \Pi \left( \Phi^{-1} (E_{g_1 , g_2} \otimes [s] ) \right) := \delta_{g_1 \in C} \; E_{g_1,g_2} \otimes \pi (s) .
\]
\end{proof}}