\documentclass[letterpaper,11pt]{article}
\title{Polyhedral Clinching Auctions for Indivisible Goods} %TODO Please add
\author{Hiroshi Hirai\thanks{Graduate School of Mathematics, Nagoya University, Nagoya, 464-8602, Japan. \\hirai.hiroshi@math.nagoya-u.ac.jp} \and Ryosuke Sato \thanks{Graduate School of Information Science and Technology, The University of Tokyo, Tokyo, 113-8656, Japan. ryosuke-sato-517@g.ecc.u-tokyo.ac.jp}}

\usepackage[top=1truein,bottom=1truein,left=1truein,right=1truein]{geometry}
\usepackage{amsmath}
\usepackage{amsfonts}
\usepackage{algorithm}
\usepackage{algorithmic}
\usepackage{amsthm}
\bibliographystyle{plainurl}% the mandatory bibstyle
\usepackage[dvipdfmx]{graphicx}
\usepackage[dvipdfmx,hidelinks]{hyperref}

\usepackage{amssymb}
\usepackage{comment}


\begin{document}
\theoremstyle{definition}
\newtheorem{definition}{Definition}[section]
\newtheorem{proposition}[definition]{Proposition}
\newtheorem{lemma}[definition]{Lemma}
\newtheorem*{main}{Main Theorem}
\newtheorem{theorem}[definition]{Theorem}
\newtheorem{corollary}[definition]{Corollary}
\newtheorem{remark}[definition]{Remark}
\newtheorem{fact}[definition]{Fact}
\newtheorem{claim}[definition]{Claim}
\newtheorem{observation}[definition]{Observation}
\newtheorem{assmp}[definition]{Assumption}
\newtheorem{example}[definition]{Example}


%
\maketitle              % typeset the header of the contribution
%
\begin{abstract}
In this study, we propose the polyhedral clinching auction for indivisible goods, 
which has so far been studied for divisible goods. As in the divisible setting by Goel et al. (2015), 
our mechanism enjoys incentive compatibility, individual rationality, and Pareto optimality, 
and works with polymatroidal environments. A notable feature for the indivisible setting 
is that the whole procedure can be conducted in time polynomial of the number of buyers and goods. 
Moreover, we show additional efficiency guarantees, recently established by Sato for the divisible setting: 
The liquid welfare (LW) of our mechanism achieves more than 1/2 of the optimal LW, 
and that the social welfare is more than the optimal LW. 
%\keywords{Auctions \and Budget Constraints \and Polymatroids  \and Liquid Welfare.}
\end{abstract}
%
%
%

\section{Introduction}
The theoretical foundation for budget-constrained auctions is an unavoidable step toward further social implementation of auction theory. A representative example of such auctions is ad auctions (e.g., Edelman et al. \cite{EOS2007}), where advertisers naturally have budgets for their advertising costs. However, it is well known  \cite{DLN2012,DL2014}  that designing auctions with budget constraints 
is theoretically difficult: Any budget-feasible mechanism cannot achieve the desirable goal of satisfying all of incentive compatibility (IC), individual rationality (IR), and constant approximation to the optimal social welfare (SW).

When budgets are public, 
Dobzinski et al. \cite{DLN2012} proposed a budget-feasible mechanism that 
builds on the clinching framework of Ausubel \cite{A2004}. 
Their mechanism satisfies IC, IR, and Pareto Optimality (PO), 
a weaker notion of efficiency than SW. 
They also showed that their mechanism is the only budget-feasible mechanism satisfying IC, IR, and PO. 
These results have inspired further research \cite{BHLS2015, DHS2015, FLSS2011, GMP2014, GMP2015, GMP2020, HS2022} for extending their mechanism to various settings.

{\it Polyhedral clinching auction} by Goel et al. \cite{GMP2015} is the most outstanding of these, 
and can even be applied to complex environments expressed by polymatroids. 
Their mechanism is a clever fusion of auction theory and polymatroid theory. 
This has brought  further extensions, such as concave budget constraints \cite{GMP2014} and two-sided markets \cite{HS2022, S2023}. 
Particularly, a recent result by Sato \cite{S2023} established a new type of efficiency guarantees in the mechanism. 
Thus, the polyhedral clinching auction is a standard framework for the theory of budget-constrained auctions. 

These results of the polyhedral clinching auctions are all restricted to auctions of divisible goods, though many auctions deal with indivisible goods. In this study, we address the polyhedral clinching auction for indivisible goods to enlarge its power of applicability.

\subsubsection*{Our Contributions}
We propose the polyhedral clinching auction for indivisible goods, 
based on the framework of the one for divisible goods in Goel et al. \cite{GMP2015}. 
Our mechanism exhibits the desirable properties expected from theirs.
That is, it satisfies IC, IR, and PO and works with polymatroidal environments. 
This means that it is applicable to a wide range of auctions, such as 
multi-unit auctions in Dobzinski et al. \cite{DLN2012}, matching markets in Fiat et al. \cite{FLSS2011}, 
ad slot auctions in Colini-Baldeschi et al. \cite{BHLS2015}, and video-on-demand in Bikhchandani et al. \cite{BSV2011}.
In addition, a promising future research is two-sided extensions of our results, 
as already proceeded for the divisible settings in Hirai and Sato \cite{HS2022} and Sato \cite{S2023}.
Particularly, such an extension includes the reservation exchange markets in Goel et al. \cite{GLMNP2016}\ ---\ a setting of two-sided markets for display advertising. 

As in the divisible setting in \cite{GMP2015}, each iteration of our mechanism can be done in polynomial time. 
A notable feature specific to the indivisible setting is iteration complexity. The total number of the iterations is also polynomially bounded in the number of buyers and the goods. 
Thus, the whole procedure can be implemented in polynomial time. 

In addition to the above PO, we establish two types of efficiency guarantees.
The first one is that 
our mechanism achieves {\it liquid welfare} (LW) more than 1/2 of the optimal LW.
This is the first LW guarantee for clinching auctions with indivisible goods. Here LW \cite{DL2014, ST2013} is a payment-based efficiency measure for budget-constrained auctions, and is defined as the sum of the total admissibility-to-pay, which is the minimum of valuation of the allocated goods (willingness-to-pay) and budget (ability-to-pay). 
Our LW guarantee is understood as an indivisible and one-sided version of the one recently established by Sato \cite{S2023} for the divisible setting. The notable point is that the LW guarantee 
holds for such general auctions, even in indivisible setting, while other existing work \cite{DL2014} on 
LW guarantees for clinching auctions is only limited to simple settings of multi-unit auctions. 

The second one is that our mechanism achieves SW more than the optimal LW.
This type of efficiency guarantee, which compares the SW of mechanisms with the optimal LW, was initiated by Syrgkanis and Tardos \cite{ST2013} in the Bayesian setting, and was recently obtained by Sato \cite{S2023} for clinching auctions with divisible goods in the prior-free setting.
In budget-constrained auctions, modifying the valuations to make 
the market non-budgeted is often considered; see, e.g., Lehmann et al. \cite{LLN2006}. 
If each buyer's valuation is modified to the budget-additive valuation, 
then LW is interpreted as the SW. In this approach, the optimal LW is used as the target value of SW 
and can be thought as a reasonable benchmark of SW. 
Thus, this guarantee provides another evidence for high efficiency of our mechanism.


\subsubsection*{Our Techniques}
\begin{comment}
{\it Tight sets lemma} is a powerful tool for efficiency guarantees of polyhedral clinching auctions, 
which characterizes the dropping of buyers and the final allocation.  
We establish the first tight sets lemma for the indivisible settings. In Goel et al. \cite{GMP2015}, which deals with our divisible version, this lemma is obtained by exploiting the simple structure of their mechanism. However, the structure of our mechanism is more complicated due to the indivisibility.
To overcome the difficulties, we use the {\it dropping prices} 
%to illustrate the state of buyers at the end of the auction, and use 
and {\it unsaturation} developed by Goel et al. \cite{GMP2014}, who extend the work of \cite{GMP2015} to a different direction from ours. 
\end{comment}

{\it Tight sets lemma} \cite{GMP2014,GMP2015} characterizes the dropping of buyers and the 
final allocation, and is a powerful tool for efficiency guarantees of  
polyhedral clinching auctions. 
For showing efficiency guarantees (PO, LW, SW) mentioned above, 
we establish a new and the first tight sets lemma for indivisible setting. 
Although our (hard-budget) setting is a natural indivisible version of 
the one in Goel et al. \cite{GMP2015},  
the indivisibility causes complications in various places 
and prevents straightforward generalization in both its formulation and proof. 
We utilize the notions of {\it dropping prices} and {\it unsaturation}
by Goel et al. \cite{GMP2014} invented for a more complex divisible setting 
(concave-budget setting), and formalize and prove our indivisible tight sets lemma.

Even with our new tight sets lemma,
the indivisibility still prevents a straightforward adaptation
of previous techniques showing efficiency guarantees,
especially the LW guarantee.
The proof of the above 2-approximation LW guarantee
is based on the idea of Sato \cite{S2023} for divisible setting, 
and is obtained by establishing the inequality
\[
{\rm LW}^{\rm M} \geq p^{\rm f}(N) \geq {\rm LW}^{\rm OPT} - {\rm LW}^{\rm M},
\]
where ${\rm LW}^{\rm M}$ and $p^{\rm f}(N)$ 
are the LW value and the total payments, respectively, in our mechanism, 
and ${\rm LW}^{\rm OPT}$ is the optimal LW value.
In the divisible setting of \cite{S2023}, the second inequality is obtained
by using the LW optimal allocation to provide a lower bound on future payments.
However, this approach does not fit in our setting due to indivisibility.
Instead, we introduce a new technique of lower bounding the future payments
via virtual buyers and the associated virtual optimal LW allocation
to overcome the difficulty.
This new technique is interesting in its own right,
and expected to be applied to LW guarantees of other auctions.

%${\rm LW}^{\rm OPT}$ admits a simple explicit formula 
%from which the second inequality can be deduced.
%In our indivisible setting, however, the formula changes 
%to a complicated one, and is unable to deduce the inequality.

%to link the dropping prices of buyers. %These tools were 
\begin{comment}
{\it Tight sets lemma} is a powerful tool for efficiency guarantees of polyhedral clinching auctions, 
which characterizes the dropping of buyers and the final allocation. 
We establish the first tight sets lemma for the indivisible settings. 
Although we deal with hard budget constraints as in Goel et al. \cite{GMP2015}, 
due to the indivisibility, the tight sets lemma in \cite{GMP2015}, 
who deal with our divisible version, cannot be formulated. 
Therefore, we use the analysis of Goel et al. \cite{GMP2014}, who deal with concave budget constraints in the divisible settings, and obtain the sharp tight sets lemma as theirs. 

%However, the structure of our mechanism is more complicated due to indivisibility.
%To overcome the difficulties, we use the {\it dropping prices} 
%to illustrate the state of buyers at the end of the auction, and use 
%and the {\it unsaturation} 
%to link the dropping prices of buyers. 
%These tools were 
%developed by Goel et al. \cite{GMP2014} to capture concave budget constraints, 
%a very different situation from ours.

Even with our new tight sets lemma, another technical difficulty of indivisibility prevents a straightforward adaptation of the above efficiency guarantees, especially the LW guarantee. In the divisible setting, Sato \cite{S2023} provided a lower bound on the total payment via the LW optimal allocation since the total payment is a lower bound on LW. However, the LW optimal allocation does not fit in our indivisible setting. Instead, we divide each buyer into two {\it virtual buyers} and use the LW optimal allocation for virtual buyers to establish a sharp lower bound on the total payment.
\end{comment}

\subsubsection*{Other Related Works}

Auctions of indivisible goods are ubiquitous in the real-world. For unbudgeted settings, its theory already has a wealth of knowledge; see, e.g., Krishna \cite{K2010} and Nisan et al. \cite{NRTV2007}. Auctions with (poly)matroid constraints were initiated by Bikhchandani et al. \cite{BSV2011}. They considered buyers who have concave valuations and no budget limits. Our framework captures a budgeted extension of their framework if all buyers have additive valuations within their budgets. For budgeted auctions, Dobzinski et al. \cite{DLN2012} proposed the adaptive clinching auction and showed that it has IC, IR, and PO. Later, Fiat et al. \cite{FLSS2011} and Colini-Baldeschi et al. \cite{BHLS2015} extended their mechanism to a market represented by a bipartite graph. Our framework is also viewed as a generalization of theirs to polymatroidal settings.

LW was introduced independently and simultaneously by Dobzinski and Leme \cite{DL2014} and Syrgkanis and Tardos \cite{ST2013}. The existing LW guarantees for auctions (with public budgets) are as follows: For clinching auctions, Dobzinski and Leme \cite{DL2014} showed that the adaptive clinching auction in Dobzinski et al. \cite{DLN2012} achieves 2-approximation to the optimal LW. Recently, Sato \cite{S2023} showed that the polyhedral clinching auction in Hirai and Sato \cite{HS2022} achieves 2-approximation to the optimal LW even under polymatroidal constraints. Our results are viewed as an indivisible and one-sided version of his results. For unit price auctions, Dobzinski and Leme \cite{DL2014} also showed that their unit price auction achieves 2-approximation to the optimal LW. Later, Lu and Xiao \cite{LX2015} proposed another unit price auction and improved the guarantee to $(1+\sqrt{5})/2$. 
It is an interesting research direction to incorporate polymatroidal constraints with their mechanisms, for which our results may help.

\subsubsection*{Organization of this paper}
The rest of the paper is organized as follows.
In Section 2, we introduce our model. In Section 3, we propose our mechanism and provide some basic properties.
In Section 4, we analyze the structure of our mechanism and obtain the tight sets lemma.
In Section 5, we provide the efficiency guarantees for our mechanism with respect to PO, LW, and SW.

\subsubsection*{Notation}
Let $\mathbb R_+$ (resp. $\mathbb Z_+$) denote the set of nonnegative real numbers (resp. integers), and 
let $\mathbb R_{++}$ (resp. $\mathbb Z_{++}$) denote the set of positive real numbers (resp. integers).
For a set $N$, let $\mathbb R^N_+$ (resp. $\mathbb Z^N_+$) denote the set of 
all functions from $N$ to $\mathbb R_+$ (resp. $\mathbb Z_+$). 
For $x\in \mathbb R^N_+$, we often denote $x(i)$ by $x_i$, and write it as $x= (x_i)_{i\in N}$. 
For $S\subseteq N$, let $x(S)$ denote the sum of $x(i)$ over $i\in S$, i.e., $x(S) := \sum_{i\in S} x(i)$.
In addition, we often denote a singleton $\{i\}$ by $i$. 



\section{Our Model}
Consider a market with multiple buyers and one seller who plays the role of the auctioneer. 
The seller auctions multiple units of a single indivisible good. 
Let $N:=\{1,2,\ldots,n\}$ be the set of buyers. 
Each buyer $i\in N$ has three real numbers $v_i, v'_i, B_i $, where   
$v_i, v'_i \in \mathbb R_+$ are a valuation and a bid of buyer $i$, respectively, for a unit of the good, 
and $B_i\in \mathbb R_{++}$ is a budget of buyer $i$, i.e., the maximum total payment that $i$ can pay in the auction.
The valuation of each buyer is private information unknown to other buyers and the seller, and we assume that the budget is public information available to the seller.
The seller determines the allocation based on a predetermined mechanism. 


The allocation $\mathcal A:=(x, p)$ is a pair of $x\in \mathbb Z^N_+$ and $p\in \mathbb R^N$, where $x_i$ is the number of indivisible goods allocated to buyer $i$, and $p_i$ is the payment of buyer $i$ for their goods. 
Then, the budget constraints are described as $p_i\leq B_i\ (i\in N)$.
We are given an integer-valued monotone submodular function $f: 2^N\to \mathbb Z_+$ 
that represents the feasible allocation of goods.
Note that an integer-valued function 
$f: 2^N\to \mathbb Z_+$ is monotone submodular if it satisfies 
(i) $f(\emptyset)=0$, (ii) $f(S)\leq f(T)$ for each $S\subseteq T$, and (iii) 
$f(S\cup e)-f(S)\geq f(T\cup e)-f(T)$ for each $S\subseteq T$ and $e\in N\setminus T$.
For any set of buyers $S\subseteq N$, the buyers in $S$ can transact at most $f(S)$ amounts of goods through the auction. Note that $f(N)$ means the total goods sold in the auction.
This condition is equivalent to $x\in P(f)$ using the polymatroid
\[
P(f):=\{x\in \mathbb R^N_{+}\mid x(S)\leq f(S)\ (S\subseteq N)\}.
\]
We often denote $P(f)$ by $P$.
We also assume $f(N)=f(N\setminus i)$ for each $i\in N$, 
which implies that competition exists among buyers for each good at the beginning.

The utilities of the buyers are defined by 
	\[
	u_i(\mathcal A):=
	\begin{cases}
	\displaystyle 
	v_i x_i-p_i\quad\,  {\rm if}\ p_i\leq B_i, \\
	-\infty \qquad\quad  {\rm otherwise}.
	\end{cases}
	\]
Thus, the utilities of the buyers are quasi-linear if the budget constraints are satisfied, 
and otherwise, the utilities go to $-\infty$.
The utility of the seller is defined as the revenue of the seller, i.e., $u_s(\mathcal A):=\sum_{i\in N}p_i$.
The mechanism is a map $\mathcal M: \mathcal I\to \mathcal A$ from information $\mathcal I$ to allocation $\mathcal A$. 
Note that $\mathcal I$ includes all information that the seller can access, 
and thus $\mathcal  I=(N, \{v'_i\}_{i\in N},\{B_i\}_{i\in N},f)$.
We call a mechanism $\mathcal M$ budget feasible if, for any $\mathcal I$, mechanism $\mathcal M$ 
outputs an allocation that satisfies the budget constraints.


We consider to design an efficient budget feasible 
mechanism $\mathcal M$ that satisfies incentive compatibility (IC) and individual rationality (IR).
A mechanism satisfies IC if for any $(\mathcal I, \{v_i\}_{i\in N})$, it holds 
$u_i(\mathcal M(\mathcal I_i))\geq  u_i(\mathcal M(\mathcal I))$ for each $i\in N$, where $\mathcal I_i$ denotes the information obtained from $\mathcal I$ by replacing $v'_i$ with $v_i$.
Intuitively, IC guarantees that the best strategy for each buyer is to report their true valuation.
When the mechanism satisfies IC, it satisfies IR if, for any $(\mathcal I, \{v_i\}_{i\in N})$, it holds $u_i(\mathcal M(\mathcal I_i))\geq 0$ for each $i\in N$.
Intuitively, IR guarantees that each buyer obtains nonnegative utility when the buyer reports the true valuation.


The efficiency of the mechanism can be evaluated by the following:
Social welfare(SW) is defined as the sum of the valuations of the allocated goods for all buyers, and 
it can be interpreted as the sum of the utilities of all participants.
In other words, ${\rm SW}(\mathcal A):=\sum_{i\in N}v_i x_i=\sum_{i\in N}u_i (\mathcal A)+u_s(\mathcal A)$.
This is the standard efficiency measure used for the auctions. 
As mentioned, however, it is known (e.g., Dobzinski and Leme \cite{DL2014}, Lemma 2) that for any $\alpha<n$, there is no budget feasible mechanism that achieves $\alpha$-approximation to the optimal SW with IC and IR.

The alternative measure for budget-constrained auctions is LW, which is
defined by ${\rm LW}(\mathcal A):=\sum_{i\in N}\min(v_i x_i, B_i)$ for allocation $\mathcal A$.
LW represents the sum of the possible payments that buyers can pay for their allocated goods.
Another type of efficiency guarantee suitable for budget constraints is Pareto optimality (PO).
A mechanism satisfies PO if for any $(\mathcal I, \{v_i\}_{i\in N})$, there is no other allocation $\mathcal A'$ with (i) $u_i(\mathcal A')\geq u_i(\mathcal M(\mathcal I))$ for each $i\in N$, 
(ii) $u_s(\mathcal A')\geq u_s(\mathcal M(\mathcal I))$, and (iii) at least one inequality holds without equality.

\section{Polyhedral Clinching Auctions for Indivisible Goods}
In this section, we describe our mechanism. Our mechanism  
incorporates the polyhedral approach of Goel et al. \cite{GMP2015} to the (budgeted) clinching auctions in previous indivisible settings  
 (e.g., \cite{BHLS2015, DLN2012, FLSS2011}).
A full description of our mechanism is presented in Algorithms 1 and 2.

	\begin{algorithm}[htb]
	\caption{Polyhedral\ Clinching\ Auction for Indivisible Goods} 
	\begin{algorithmic}[1]
	  \STATE $x_i=0,\ p_i:=0,\ d_i:=f(i)+1 \ 
	  (i\in N)\ {\rm and}\ c:=0$.\\
	  \WHILE{Active buyers exist}
	  \STATE Increase $c$ until there appears an active buyer 
	  $j$ such that $v'_{j}=c$ or $d_{j}=\frac{B_{j}-p_{j}}{c}$.
	  \WHILE{$\exists$ active buyer $j$ with $v'_{j}=c$}
	  \STATE Pick such a buyer $j$ and let $d_{j}:=0$
	  \STATE Clinching$(f,x,p,d,c)$
	  \ENDWHILE
	  \WHILE{$\exists$ active buyer $j$ with $d_i=\frac{B_i-p_i}{c}$}
	  \STATE Pick such a buyer $j$ and let $d_{j}:=d_{j}-1$
	  \STATE Clinching$(f,x,p,d,c)$
	  \ENDWHILE
	 \ENDWHILE
	 \STATE Output $(x^{\rm f}, p^{\rm f}):=(x,p)$.
	\end{algorithmic}
	\end{algorithm}
	
	\begin{comment}
	\begin{algorithm}[htb]
	\caption{Clinching $(f,x,p,c,d)$} 
	\begin{algorithmic}[1]
	  \FOR{$i=1,2,\ldots,n$} 
	 \STATE $P_{x,d}\rightarrow\{y\in \mathbb R^N_{+}\mid x+y\in P,\ y_i\leq d_i \ (i\in N)\}$,\ $P^i_{x,d}(w_i)\rightarrow\{u\vert_{N\setminus i}\mid \exists u\in P_{x,d}\ {\rm s.t.}\ u_i=w_i\}$
	  \STATE $\delta_i\rightarrow\sup\{w_i\geq 0\mid P^i_{x,d}(w_i)=P^i_{x,d}(0)\}$
	  \STATE 
	  $\displaystyle x_i:=x_i+\delta_i,\ p_i:=p_i+c \delta_i,\ d_i:=d_i-\delta_i$
	  \ENDFOR
	\end{algorithmic}
	\end{algorithm}
	\end{comment}
	
	\begin{algorithm}[htb]
	\caption{Clinching $(f,x,p,c,d)$} 
	\begin{algorithmic}[1]
	  \FOR{$i=1,2,\ldots,n$} 
	  \STATE Clinch a maximal increase 
	  $\delta_i$ satisfying $P^i_{x,d}(\delta_{i})=P^i_{x,d}(0)$; see equations (\ref{feasible_transaction}) 
	  and (\ref{clinching_definition})
	  \STATE 
	  $\displaystyle x_i:=x_i+\delta_i,\ p_i:=p_i+c \delta_i,\ d_i:=d_i-\delta_i$
	  \ENDFOR
	\end{algorithmic}
	\end{algorithm}

	
Now we outline the mechanism.
The price clock $c\in \mathbb R_{+}$ represents a transaction price for one unit of the good. 
Our mechanism is an ascending auction, where $c$ gradually increases.
For the current price $c$, the demand vector $d:=(d_i)_{i\in N}\in \mathbb Z^{N}_{+}$ represents 
the maximum possible amounts of transaction for buyers.
We call buyer $i$ {\it active} if $d_i>0$, and {\it dropping} if $d_i$ just reaches zero.	
In our mechanism, buyers are dropping by either of (i) demand update in line 5 or 9, 
or (ii) clinching the goods in Algorithm 2. 

Initially, the allocation $(x,p)$ is all zero, the price clock $c$ is set to zero, and the demand $d_i$ is set to $f(i)+1$ for each buyer $i\in N$. Then, the following procedure is repeated as long as there are active buyers: At the beginning of an iteration, the price clock $c$ is updated to $\min_{i\in N; d_i>0}\min\{v'_i, (B_i-p_i)/d_i\}$ in line 3. When the price clock $c$ is updated, there exists a set of active buyers $i$ such that $c=v'_i$ or $d_i=(B_i-p_i)/c$. Then, the demands of such buyers are updated: The demand $d_i$ of a buyer $i$ with $c=v'_i$ decreases to zero in line 5, and that of a buyer $i$ with $c<v'_i$ and $d_i=(B_i-p_i)/c$ decreases by one in line 9. After each case of the demand update, if the clinching condition (described below) is satisfied, the buyers clinch some amount of goods. Then, the allocation $(x,p)$ and the demands $d$ are updated. After processing both cases, if there exists an active buyer, the next iteration is performed. Otherwise, Algorithm 1 terminates and outputs the final allocation $(x^{\rm f},p^{\rm f})$.

The clinching steps in lines 6 and 10 are described in Algorithm 2.
Let $x$ and $d$ be the allocation of goods and the demand vector, 
respectively, just before the execution of Algorithm 2 in an iteration.
We consider two polytopes that represent the feasible transactions 
of buyers and describe the clinching condition.
For a polymatroid $P$, and vectors $x\in P$ and $d\in \mathbb R^N_{+}$, 
we define the {\it remnant supply polytope} $P_{x,d}$ by
\begin{equation*}
\label{remnant_supply_polytope}
P_{x,d}:=\{y\in \mathbb R^N_{+}\mid x+y\in P,\ y_i\leq d_i \ (i\in N)\}, 
\end{equation*} 
which indicates the feasible transaction of buyers from the current iteration.
%Then, $P_{x,d}$ is a polymatroid again because it is obtained 
%through contraction by $x$ and reduction by $x+d$ for polymatroid $P$.
In addition, when buyer $i$ clinches $w_i\in \mathbb Z_+$ units, 
\begin{equation}
\label{feasible_transaction}
P^i_{x,d}(w_i):=\{u\vert_{N\setminus i}\mid  u\in P_{x,d}\ {\rm and}\ u_i=w_i\}
\end{equation}
represents a feasible transaction of buyers $N\setminus i$.
The clinching amount $\delta_i$ of buyer $i$ is then described by the following:
\begin{equation}
\label{clinching_definition}
\delta_i=\sup\{w_i\geq 0\mid P^i_{x,d}(w_i)=P^i_{x,d}(0)\}.
\end{equation}
Thus, each buyer $i$ clinches the maximal possible amount $\delta_i$, 
not affecting the feasible transactions of other buyers $N\setminus i$. 
This intuition of clinching is consistent with the ones in previous works
\cite{A2004, BCMX2010, BHLS2015,DHS2015, DLN2012, FLSS2011,GMP2014,GMP2015, GMP2020} 
on clinching auctions.

The polytopes $P_{x,d}$ and $P^{i}_{x,d}(w_i)$ are known to be polymatroids. 
For any vectors $x\in P$ and $d\in \mathbb R^{N}_{+}$, 
the monotone submodular function $f_{x,d}$ for the polymatroid  
$P_{x,d}$ is defined by 
\begin{equation}
\label{naive}
f_{x,d}(S):=\min_{S'\subseteq S}\{\min_{S''\supseteq S'}\{f(S')-x(S')\}+d(S\setminus S')\},
\end{equation} 
see Section 3.1 of Fujishige \cite{F2005}. 
Sato \cite{S2023} pointed out that $f_{x,d}$ can be described by a simple formula 
in the polyhedral clinching auction of his divisible setting.
We show that this result also holds in our indivisible setting. 
\begin{theorem}
\label{invariant}
Let $x$ and $d$ be the allocation of goods and the demand vector, respectively, in Algorithm~1. 
Then, it holds 
$\displaystyle f_{x,d}(S)=\min_{S'\subseteq S}\{f(S')-x(S')+d(S\setminus S')\}$ for any $S\subseteq N$.

\end{theorem}
Theorem \ref{invariant} is useful for analyzing our mechanism. 
For instance, $\delta$ can be computed as follows: 

\begin{proposition}
\label{clinch_amount}
In the execution of Algorithm 2, it holds $\delta_i=f_{x,d}(N)-f_{x,d}(N\setminus i)\leq d_i$ for each $i\in N$,  
where $x$ and $d$ are the allocation of goods and the demand vector, respectively, just before the execution of Algorithm 2.
\end{proposition}

The proofs are provided in Appendix A. 
Proposition \ref{clinch_amount} implies that, provided the value oracle of $f$,
the value $\delta_i$ can be computed in polynomial time 
by a submodular minimization algorithm, such as in Lee et al. \cite{LSW2015}.
In addition, Proposition \ref{clinch_amount} implies that the amount of 
goods allocated to each buyer in Algorithm 2 is independent of the order of the buyers. 


We investigate the properties of our mechanism in the following. 
To this end, we fix an input $(\mathcal I, v)$.
We first consider the properties specific to our indivisible setting: 
There are two major differences between our mechanism and that in Goel et al.  \cite{GMP2015}.
The first is on buyer's demand: 
The demands must be an integer vector because our model deals with indivisible goods.
Thus, the demand for fewer than one unit is rounded down, 
which makes the function $f_{x,d}$ integer-valued.
Therefore, $\delta$ is an integer vector in each iteration.

\begin{proposition}
\label{integer-clinching}
In Algorithm 2, it holds $\delta_i\in \mathbb Z_+$ for each $i\in N$.
\end{proposition}
\begin{proof}
We prove that $f_{x,d}$ is integer-valued throughout the auction.
Then, by Proposition \ref{clinch_amount}, it holds $\delta_i\in \mathbb Z_+$ for each $i\in N$, as required.
At the beginning of the auction, the initial values of $x$ and $d$ are integers.
Since $f$ is integer-valued, so is $f_{x,d}$.

After the demand update in line 5 or 9, suppose that both $x$ and $d$ remain integer vectors. 
Then, $f_{x,d}$ is again integer-valued.
During the execution of Algorithm 2, by Proposition \ref{clinch_amount}, 
$\delta_i$ represents an integer for each $i\in N$.
Then, $x$ is updated to $x_i=x_i+\delta_i$ and $d$ is updated to $d_i=d_i-\delta_i$.
Thus, $x$ and $d$ remain integer vectors. 
Therefore, $f_{x,d}$ is integer-valued since $f$ is~integer-valued.
\end{proof}



The second is on price update: 
Our mechanism sets a common price $c$ for all buyers and does not use a fixed step size for price increases.
This is based on the idea of Bikhchandani et al. \cite{BSV2011} and Fiat et al. \cite{FLSS2011}, 
and it plays an essential role 
in the iteration bounds; therefore, in the computational complexity.
In our mechanism, the total sum of initial demands is $\sum_{i\in N} f(i)+n$, 
and in each iteration, it is guaranteed that the total sum of the demands is decreased by at least one.
Therefore, we have the following: 

\begin{observation}
Our mechanism terminates after at most $\sum_{i\in N} f(i)+n$ iterations.
\end{observation}

Note that it holds $\sum_{i\in N} f(i)+n\leq n(f(N)+1)$
based on the monotonicity of $f$.
Further, as stated above, 
each iteration can be computed using the submodular minimization algorithm \cite{LSW2015},
which has runtime polynomial in the number of $n$, provided the value oracle of $f$ is given.
Therefore, our mechanism can be computed in polynomial time.
\footnote{If the number of goods $f(N)$ is given in the binary representation, it is pseudo-polynomial. Such a model is applied for the case where $f(N)$ is assumed to be large; e.g., Dobzinski and Nisan \cite{DN2010}. On the other hand, it is also natural to regard a large amount of goods as being divisible. Therefore, we do not make such an assumption, and consider the number $f(N)$ as a part of the input size.} 

	Subsequently, we show the budget feasibility of our mechanism: 
	\begin{theorem}
	\label{BF}
	Our mechanism is budget feasible.
	\end{theorem}
	To prove Theorem \ref{BF}, we use the following lemma that is also used in other proofs.
	\begin{lemma}
	\label{d}
	%For an active buyer $i$ whose demand is never updated their demand in line 9, 
	%it holds $d_i=f(i)+1-x_i\leq \frac{B_i-p_i}{c}$.
	%For an active buyer $i$ whose demand was updated  
	%in line 9 at least once, it holds $d_i=\frac{B_i-p_i}{c}-1$ if $i$'s demand has 
	%just been updated in line 9 in an iteration, and it holds 
	%$d_i=\left\lfloor \frac{B_i-p_i}{c}\right\rfloor$ otherwise.
Let $i$ be an active buyer.
If the demand $d_i$ has never been updated in line 9, then $d_i=f(i)+1-x_i\leq \frac{B_i-p_i}{c}$. If the demand $d_i$ has just been updated in line 9 of an iteration, then it holds $d_i=\frac{B_i-p_i}{c}-1$ for the rest of that iteration. In other cases, it holds $d_i=\left\lfloor \frac{B_i-p_i}{c}\right\rfloor$.
	\end{lemma}
	In the proof of Lemma \ref{d}, we use the fact that if buyer $i$ clinches $\delta_i$ unit of the goods in Algorithm 2, then all of $d_i$, $\frac{B_{i}-p_{i}}{c}$, $\left\lfloor\frac{B_{i}-p_{i}}{c}\right\rfloor$ decrease by $\delta_i$ since $\delta_i\in \mathbb Z_{+}$ by Proposition \ref{integer-clinching}.	 
	\begin{proof}[Proof of Lemma \ref{d}]
	At the beginning of the auction, it holds $d_j\leq \frac{B_{j}-p_{j}}{c}$ for each $j\in N$ by $c=0$.
	Suppose that the demand $d_i$ has never been updated in line 9. 
	Then, the demand $d_i$ is only changed by clinching in Algorithm~2, 
	where $d_i$ and $\frac{B_{i}-p_{i}}{c}$ are changed by the same amounts. 
	Therefore, we have $d_i=f(i)+1-x_i\leq \frac{B_i-p_i}{c}$. 
	
	If the demand $d_i$ has just been updated in line 9 of an iteration, 
	it holds $d_i=\frac{B_{i}-p_{i}}{c}-1$ after the update in that iteration. 
	Moreover, once the demand $d_i$ has been updated in line 9, after that, 
	the price $c$ is increased with keeping the equality $d_i=\left\lfloor\frac{B_{i}-p_{i}}{c}\right\rfloor$ 
	by the price update in line 3. 
	These equalities are maintained by the execution of Algorithm 2 because 
	$d_i$, $\frac{B_{j}-p_{j}}{c}$, $\left\lfloor\frac{B_{i}-p_{i}}{c}\right\rfloor$ 
	are changed by the same amounts in Algorithm 2.
	
	\end{proof}
	\begin{proof}[Proof of Theorem \ref{BF}]
	We consider the case where an active buyer $i$ is dropping in an iteration.
	Let $c$ be the price clock in the iteration, and  
	$p$ and $d$ be the payment vector and the demand vector, respectively, 
	just before the dropping of buyer $i$.
	Then, Lemma \ref{d} implies that $B_i-p_i\geq c d_i>0$. 
	If buyer $i$ drops out of the auction by the demand update in line 5 or 9, 
	then we have $B_i-p^{\rm f}_i=B_i-p_i\geq c d_i>0$.
	Suppose that buyer $i$ drops out of the auction by clinching $\delta_i$ amount of goods. 
	Then, since $\delta_i\leq d_i$ by Proposition \ref{clinch_amount}, 
	we have $B_i-p^{\rm f}_i=B_i-p_i-c\delta_i\geq B_i-p_i-c d_i\geq 0$.
	Therefore, our mechanism is budget feasible. 
	\end{proof}


Moreover, our mechanism inherits several desirable properties from 
the polyhedral clinching auction by Goel et al. \cite{GMP2015}: 
\begin{theorem}
	\label{IC_IR}
	Our mechanism satisfies IC and IR.
\end{theorem}	
\begin{proof}
	For each buyer $i$, the bid $v'_i$ is used to only determine when the buyer $i$ 
	drops. If $v'_i<v_i$, buyer $i$ may miss some goods at a price below their valuation.
	In addition, if $v'_i>v_i$, buyer $i$ may clinch some goods at a price greater than their valuation. Then, truthful bidding is 
	the best strategy for buyers, and therefore our mechanism satisfies IC. 
	The mechanism satisfies IR because each buyer clinches goods at a price lower than the buyer's valuation if $v'_i=v_i$. 
\end{proof}
\begin{proposition}
\label{all_goods}
At the end of the auction, it holds $x^{\rm f}(N)=f(N)$.
\end{proposition}
	To prove this, we use the following lemmas. 
	

	\begin{lemma}
	\label{beginning}
	At the beginning of the auction, $f_{x,d}(S)=f(S)$ for each $S\subseteq N$.
	\end{lemma}
	\begin{proof}
	At the beginning of the auction, by $x_i=0$ for each $i\in N$ and Theorem \ref{invariant},
	we have $f_{x,d}(S)=\min_{S'\subseteq S}\{f(S')+d(S\setminus S')\}.$
	In addition, by submodularity and $d_i=f(i)+1$ for each $i\in N$, it holds $f(S'\cup i)-f(S')\leq f(i)<d_i$ for each $i\in N$ and $S'\subseteq N\setminus i$. 
	From this, we have $\min_{S'\subseteq S}\{f(S')+d(S\setminus S')\}=f(S)$ for each $S\subseteq N$. 
	%where the first inequality holds by the submodularity of $f$.
	Therefore, we have $f_{x,d}(S)=f(S)$. 
	%Similarly, $\min_{S'\subseteq S}\{f(S')-x(S')+d(S\setminus S')\}=\min_{S'\subseteq S}\{f(S')+d(S\setminus S')\}=f(S)$.
	%Therefore, we have $f_{x,d}(S)=\min_{S'\subseteq S}\{f(S')-x(S')+d(S\setminus S')\}=f(S)$.
	\end{proof}

\begin{lemma}
\label{unchanged}
For each $S\subseteq N$, the value
$x(S)+f_{x,d}(S)$ is unchanged by Algorithm~2.
\end{lemma}
\begin{proof}
By Theorem \ref{invariant}, we have 
\begin{equation}
\label{x+f_xd}
x(S)+f_{x,d}(S)=\min_{S'\subseteq S}\{f(S')+x(S\setminus S')+d(S\setminus S')\}
\end{equation}
for each $S\subseteq N$ just before the execution of Algorithm 2. 
Suppose that buyers in $S$ clinch $\delta(S)$ amounts of goods 
in the execution of Algorithm 2.
Then, $x_i$ increases by $\delta_i$ and $d_i$ decreases by $\delta_i$, 
and thus $x_i+d_i$ is unchanged for each $i\in S$.
Since equation (\ref{x+f_xd}) holds for $x$ and $d$ after the execution of Algorithm 2,
the value $x(S)+f_{x,d}(S)$ is unchanged by the execution of Algorithm~2. 
\end{proof}

	\begin{lemma}
	\label{equation_of_all_goods}
	Just after the execution of Algorithm 2, 
	it holds $f_{x,d}(N)=f_{x,d}(N\setminus i)$ for each $i\in N$.
	Moreover, $f_{x,d}(N)$ is unchanged by the demand update in line 5 or 9.
	\end{lemma}
	\begin{proof}
	At the beginning of the auction, by Lemma \ref{beginning} and $x_j=0\ (j\in N)$, we have 
	$x(N)+f_{x,d}(N)=f(N)$ and $x(N\setminus j)+f_{x,d}(N\setminus j)=f(N\setminus j)$ for each $j\in N$.
	Then, by the assumption of $f(N)=f(N\setminus j)$, we have 
	$f_{x,d}(N\setminus j)=f_{x,d}(N)$ for each $j\in N$.

	In the execution of Algorithm 2 in an iteration, 
	it holds $\delta_i=f_{x,d}(N)-f_{x,d}(N\setminus i)$ for each $i\in N$ by Proposition \ref{clinch_amount}.
	Then, we have 
	$f_{x,d}(N)-\delta(N)=f_{x,d}(N)-\delta_i-
	\delta(N\setminus i)=f_{x,d}(N\setminus i)-\delta(N\setminus i)$.
	By (the proof of) Lemma \ref{unchanged}, 
	$f_{x,d}(N)-\delta(N)$ and $f_{x,d}(N\setminus i)-\delta(N\setminus i)$ 
	correspond to $f_{x,d}(N)$ and $f_{x,d}(N\setminus i)$
	just after the execution of Algorithm 2, respectively. 
	Therefore, we have $f_{x,d}(N)=f_{x,d}(N\setminus i)$ for each $i\in N$  
	just after the execution of Algorithm 2.
	
	By the demand update of buyer $i$, the value $f_{x,d}(N\setminus i)$ is unchanged 
	since it is independent of $d_i$. 
	This implies that $f_{x,d}(N)$ is unchanged 
	by $f_{x,d}(N)=f_{x,d}(N\setminus i)$, 
	the monotonicity of $f_{x,d}$, and the fact that $f_{x,d}(N)$ is non-increasing by the decrease of $d_i$.
	\end{proof}

\begin{proof}[Proof of Proposition \ref{all_goods}]
By Lemma \ref{equation_of_all_goods}, 
the value $x(N)+f_{x,d}(N)$ is unchanged by the demand update of buyers.
Moreover, by Lemma \ref{unchanged}, $x(N)+f_{x,d}(N)$ is also unchanged by the execution of Algorithm 2.
Combining these with Lemma \ref{beginning}, 
we have $x(N)+f_{x,d}(N)=f(N)$ throughout the execution of Algorithm 1.
At the end of the auction, by Theorem \ref{invariant} and $d_i=0\ (i\in N)$, we have $f_{x,d}(N)=0$,
which means $x^{\rm f}(N)=f(N)$.
\end{proof}


From Theorem \ref{IC_IR}, our mechanism satisfies IC, and therefore, 
we assume in the rest of the paper that all buyers bid truthfully, that is, $v'_i=v_i$ for every $i\in N$. 

%The following proposition indicates that our mechanism sells all goods to the buyers. 


%provided the value oracle of $f$ is given.
%has runtime polynomial in the number $n$ of buyers and 
%the number $f(N)$ of goods,

\section{Structural Properties of the Mechanism}
In this section, we establish the {\it tight sets lemma} for our mechanism,
which is necessary for our efficiency guarantees in Section 5.  
We call a set $T\subseteq N$ {\it tight} if $x^{\rm f}(T)=f(T)$.


\subsection{Tight Sets Lemma}	 
	We provide the characterization of the dropping of buyers.
	%In our mechanism, %the buyer who drops by clinching   
	%does not necessarily exhaust their budget in our mechanism 
	In our mechanism, the dropping prices are needed to describe the state of such buyers.
	%because the demand for less than one unit is rounded down. 
	
	%, which were first introduced in Goel et al. \cite{GMP2014} 
	%to deal with concave budget constraints.

	\begin{definition}[Goel et al. \cite{GMP2014}]
	{\rm In an execution of Algorithm 1, the} dropping price 
	{\rm $\phi_i$ of buyer  $i$ is defined as the first price for which $i$ had zero demand.}
	\end{definition}

	Obviously, it holds that $v_i\geq \phi_i$ for each $i\in N$. 
	The goal of this section is as follows:

	\begin{theorem}[Tight sets lemma]
	\label{tightsets}
	Let $i_1, i_2,\ldots, i_t$ be the buyers dropping by demand update in line 5 or 9, 
	where they are sorted in the reverse order of their dropping, 
	that is, $\phi_{i_1}\geq \cdots\geq \phi_{i_t}$. 
	For each $k = 1, 2,\ldots, t$,
	let $X_k$ denote the set of active buyers just before the drop of $i_k$. Then we obtain:
	\begin{itemize}
	\item[{\rm (i)}] It holds $i_k\in X_k \setminus X_{k-1}$ for each $k$. 
	Moreover, it holds $\phi_{i_k}=v_{i_k}$ or $B_{i_k}-p^{\rm f}_{i_k}= \phi_{i_k}$.
	\item[{\rm (ii)}] For buyer $i \in X_k\setminus (X_{k-1} \cup i_k)$, it holds that $\phi_i =\phi_{i_k}$ 
	and  $B_i-p^{\rm f}_i\leq \phi_i$. 
	Moreover, if there exists a buyer $\ell \in X_k\setminus (X_{k-1}\cup i_k)$ with $B_{\ell}-p^{\rm f}_{\ell}	=\phi_{\ell}$;  then it holds $v_{i}>\phi_{i}$ for each $i \in X_k\setminus X_{k-1}$, and $B_{i_k}-p^{\rm f}_{i_k}=\phi_{i_k}$.
	\item[{\rm (iii)}] $\emptyset = X_0 \subset X_1 \subset X_2 \subset \cdots \subset X_t = N$ is a chain of tight sets.
	\end{itemize}
	\end{theorem}

	
The tight sets lemma by Goel et al. \cite{GMP2015} utilizes
a simple structure of their mechanism that
buyers dropping by clinching exhaust their budget,
and other buyers have valuations equal to their dropping prices.
However, these are {\it not} preserved in our setting.
Therefore, we need the notions of dropping prices
and unsaturation in \cite{GMP2014} 
to illustrate sharper information for dropping of buyers and their dropping prices.


	
	
	
	%This can be seen from the fact that Property (i) and (ii) of Theorem \ref{tightsets} are different.
	%For example, for each $k$ and $i \in X_k\setminus (X_{k-1} \cup i_k)$, 
	%it holds $B_i-p^{\rm f}_i\leq \phi_i$. 
	%in Theorem \ref{tightsets}, 
	%whereas, in their tight sets lemma, it holds $p^{\rm f}_i=B_i$. 


\subsection{Unsaturation}

%To analyze the structure behind our mechanism, we use the notion of {\it unsaturation}. 
%In our mechanism, many properties regarding unsaturation are inherited from those in \cite{GMP2014}.
%Since a similar technical analysis is performed, the proofs of this subsection are provided in Appendix D. 
We begin by the definition of unsaturation. 
Throughout this section, 
let $x$ and $d$ denote the allocation of goods and demand vector in an iteration.
Also, for demand vector $d\in \mathbb Z_+^N$, we define $d^{-k}$ by $d^{-k}:=(0, d_{-k})$,
where $d_{-k}\in \mathbb Z^{N\setminus k}_{+}$ denotes the demands of buyers $N\setminus k$.

\begin{definition}[Goel et al. \cite{GMP2014}]
\label{unsaturation}
{\rm For buyers $i,k\in N$, buyer $i$ is} $k$-unsaturated {\rm if for any maximal vector 
$z\in P_{x,d^{-k}}$, it holds $z_i =d^{-k}_i$. 
Buyer $i$ is} $k$-saturated {\rm if buyer $i$ is not $k$-unsaturated.}
\end{definition}

The binary relation ``$i$ is $k$-unsaturated'' is denoted by ``$i \lesssim k$''.
Goel et al. \cite{GMP2014} showed the reflexivity, 
while we show the transitivity.
Although the transitivity is not used in other parts of this paper, it is interesting in its own right.
These results can be summarized as follows: 
\begin{lemma}[Goel et al. \cite{GMP2014}]
\label{reflexive}
The binary relation ``$\lesssim$'' is reflexive.
\end{lemma}
\begin{lemma}
\label{transitive}
The binary relation ``$\lesssim$'' is transitive.
\end{lemma}
\begin{corollary}
\label{quasi_order}
The binary relation ``$\lesssim$'' is quasi-order.
\end{corollary}

The proof of Lemma \ref{transitive} is provided in Appendix A.
Unsaturation is closely related to the demands of buyers and the clinching amount.
These relationships are illustrated by the following.

\begin{lemma}
\label{unsaturation_demand}
	If $i\lesssim k$ just before line 5 or 9 in an iteration, then it holds $d_i\leq d_k$.
\end{lemma}
\begin{proposition}
	\label{unsaturation_clinch}
	Suppose that the demand of buyer $i$ decreases from $d_i$ to $d'_i$
	in line 5 or 9 in an iteration.
	For each buyer $k\neq i$, 
	if $i\lesssim k$ just before the demand update, 
	then in the subsequent execution of Algorithm 2, 
	it holds $\delta_k =d_i-d'_i$.
	Otherwise, $\delta_k<d_i-d'_i$.
\end{proposition}


Now we prove these properties. 
We use the following property, which immediately holds because $f_{x,d}(S\setminus i)$ 
and $f_{x,d^{-i}}(S\setminus i)$ is independent of $d_i$:
\begin{observation}
\label{fact}
It holds $f_{x,d}(S\setminus i)=f_{x,d^{-i}}(S\setminus i)$ for any $S\ni i$.
\end{observation}

In the following, we use the characterization in Lemma \ref{characterization_unsaturation} 
instead of Definition \ref{unsaturation}.
The following lemma is an extension of Lemmas 4.13 and 4.15 
of Goel et al. \cite{GMP2014} to our mechanism.

\begin{lemma}
\label{characterization_unsaturation}
It holds 
$i\lesssim k\  {\rm if\ and\ only\ if}\  f_{x,d^{-k}}(N)-f_{x,d^{-k}}(N\setminus i)=d^{-k}_i$.
Moreover, for any $S\ni i$,  if $i\lesssim k$, then $f_{x,d^{-k}}(S)-f_{x,d^{-k}}(S\setminus i)=d^{-k}_i$.
\end{lemma}
\begin{proof}
Let $\tilde{z}$ be a maximal vector in  $P_{x,d^{-k}}$ that minimizes $z_i$.
Since $P_{x,d^{-k}}$ is a polymatroid, $\tilde{z}$ can be obtained by the greedy algorithm,
where each coordinate is increased up to the feasibility constraint 
and $i$ is set to be the last coordinate.
Thus, we have $\tilde{z}_i=f_{x,d^{-k}}(N)-f_{x,d^{-k}}(N\setminus i)$.
Therefore, $i\lesssim k$ if and only if $f_{x,d^{-k}}(N)-f_{x,d^{-k}}(N\setminus i)=d^{-k}_i$.

If $i\lesssim k$, by the submodularity of $f_{x,d^{-k}}$, 
we have $f_{x,d^{-k}}(S)-f_{x,d^{-k}}(S\setminus i)\geq f_{x,d^{-k}}(N)-f_{x,d^{-k}}(N\setminus i)=d_i$.
Moreover, by equation (\ref{naive}), we have $f_{x,d^{-k}}(S)\leq f_{x,d^{-k}}(S\setminus i)+d_i$.
Therefore, we have $f_{x,d^{-k}}(S)-f_{x,d^{-k}}(S\setminus i)= d_i$ for any $S\ni i$.
\end{proof}

Moreover, we use the following lemma: 

\begin{lemma}
\label{beginning_saturation}
Let $i$ and $k$ be buyers with $i\neq k$.
At the beginning of the auction, $i$ is $k$-saturated.
\end{lemma}
	\begin{proof}
	At the beginning of the auction, 
	we have 
	\[
	f_{x,d^{-k}}(N)-f_{x,d^{-k}}(N\setminus i)\leq f_{x,d^{-k}}(i)=\min\{\min_{S\ni i}\{f(i)\}, d_i\}=f(i)<d_i,
	\]
	where the first inequality holds by the submodularity of $f_{x,d^{-k}}$, 
	the first equality holds by equation (\ref{naive}), 
	and the second inequality holds by $d_i=f(i)+1$ at the beginning of the auction and the monotonicity of $f$.
	Then, $i$ is $k$-saturated.
	\end{proof}


Now we prove Lemma \ref{unsaturation_demand} and Proposition \ref{unsaturation_clinch}.

\begin{proof}[Proof of Lemma \ref{unsaturation_demand}]
	Since the demands are unchanged in line 3, 
	It is sufficient to consider the situation just after the execution of Algorithm 2. 
	Suppose that $i\lesssim k$ just after the execution of Algorithm 2. 
	Then, by $i\lesssim k$ and $k\lesssim k$, 
	$f_{x,d^{-k}}(N)=f_{x,d^{-k}}(N\setminus i)+d_i=f_{x,d^{-k}}(N\setminus \{i,k\})+d_i$.
	By equation (\ref{naive}) and Observation \ref{fact}, it holds 
	$f_{x,d}(N\setminus i)\leq f_{x,d}(N\setminus \{i,k\})+d_k=f_{x,d^{-k}}(N\setminus \{i,k\})+d_k$.
	
	Now we show that $f_{x,d}(N\setminus i)=f_{x,d}(N)=f_{x,d^{-k}}(N)$ 
	just after the execution of Algorithm 2.
	By Lemma \ref{equation_of_all_goods}, it suffices to show the second equality.
	If $d_k=0$, it trivially holds $f_{x,d}(N)=f_{x,d^{-k}}(N)$.
	If $d_k>0$, it holds $f_{x,d}(N)=f_{x,d}(N\setminus k)<f_{x,d}(N\setminus k)+d_k$, 
	which implies $k\in S'$ in equation (\ref{naive}) for $f_{x,d}(N)$. Therefore, we have
	\begin{equation}
	\label{f_xd_N}
	f_{x,d}(N)=\min_{S'\subseteq N\setminus k }\{\min_{S''\supseteq S'}\{f(S''\cup k)-x(S''\cup k)\}+d(N\setminus (S''\cup k))\}.
	\end{equation}
	Then, using equation (\ref{naive}), it holds 
	\begin{align*}
	\displaystyle f_{x,d^{-k}}(N)&=\min\{ \min_{S'\subseteq N\setminus k }\{\min_{S''\supseteq S'}\{f(S''\cup k)-x(S''\cup k)\}+d(N\setminus (S''\cup k))\}, f_{x,d}(N\setminus k)\}\\
	&=\min\{f_{x,d}(N),f_{x,d}(N\setminus k)\}=f_{x,d}(N),
	\end{align*}
	where the second equality holds by (\ref{f_xd_N}), and the last equality holds by 
	$f_{x,d}(N)=f_{x,d}(N\setminus k)$ just after the execution of Algorithm 2 
	due to Lemma \ref{equation_of_all_goods}.
	Thus, we have $f_{x,d}(N\setminus i)=f_{x,d}(N)=f_{x,d^{-k}}(N)$.
	
	By $f_{x,d^{-k}}(N)=f_{x,d^{-k}}(N\setminus \{i,k\})+d_i$ and 
	$f_{x,d}(N\setminus i)\leq f_{x,d^{-k}}(N\setminus \{i,k\})+d_k$, 
	if $i\lesssim k$ just before the demand update, then it holds $d_i\leq d_k$.
\end{proof}	




\begin{proof}[Proof of Proposition \ref{unsaturation_clinch}]
	By Lemma \ref{beginning_saturation}, 
	$i$ is $k$-saturated at the beginning of the auction. 
	Again, it suffices to consider the case of $i\lesssim k$ just after the execution of Algorithm 2.
	Let $d':=(d'_i, d_{-i})$ be the demand vector after the demand update. 
	Then, by Lemma \ref{equation_of_all_goods}, it holds 
	$f_{x,d'}(N)=f_{x,d}(N)=f_{x,d}(N\setminus k)$.
	By $i\lesssim k$, it holds $f_{x,d^{-k}}(N)-f_{x,d^{-k}}(N\setminus i)= d_i$.
	Also by $k\lesssim k$ and Observation \ref{fact}, 
	it holds $f_{x,d^{-k}}(N)=f_{x,d^{-k}}(N\setminus k)=f_{x,d}(N\setminus k)$.
	Therefore, we have $f_{x,d}(N\setminus k)=f_{x,d^{-k}}(N\setminus i)+d_i$.
	%Let $d'$ be the demand vector just after $i$'s demand update. 
	Then, 
	by equation (\ref{naive}), we have 
	\[
	f_{x,d'}(N\setminus k)=\min(f_{x,d}(N\setminus k),f_{x,d}(N\setminus \{i,k\})+d'_i).
	\]
	By Observation \ref{fact} and Lemma \ref{characterization_unsaturation}, we have 
	\[
	f_{x,d}(N\setminus \{i,k\})=f_{x,d^{-k}}(N\setminus \{i,k\})=f_{x,d^{-k}}(N\setminus k)-d_i
	=f_{x,d}(N\setminus k)-d_i.
	\]
	Using the aboves, we have $f_{x,d'}(N\setminus k)=f_{x,d}(N\setminus k)-(d_i-d'_i)$ by $d_i>d'_i$.
	Then, by Proposition \ref{clinch_amount}, we have 
	$\delta_k=f_{x,d'}(N)-f_{x,d'}(N\setminus k)=f_{x,d}(N\setminus k)-(f_{x,d}(N\setminus k)-(d_i-d'_i))=d_i-d'_i$.
	If $i$ is $k$-saturated, 
	it holds $f_{x,d}(N\setminus k)<f_{x,d^{-k}}(N\setminus i)+d_i$, 
	which implies $\delta_k<d_i-d'_i$. 
\end{proof}




\subsection{Proof of Theorem \ref{tightsets}}
	
	%We prove our tight sets lemma via structural analysis using unsaturation. 
	Our main focus in the proof is the relationship between the dropping prices of buyers in each tight set, 
	where the following two propositions will help. 
	
	\begin{proposition}
	\label{dropping_of_buyers}
	If a buyer drops by clinching some goods in Algorithm 2, then there is a buyer 
	who drops by the demand update in line 5 or 9 
	just before the execution of Algorithm~2.
	\end{proposition}
	
	Proposition \ref{dropping_of_buyers} 
	implies that buyers who drop by clinching have the same dropping prices 
	as the last buyer who drops by the demand update. 

	\begin{proof}[Proof of Proposition \ref{dropping_of_buyers}] 
	%We prove that if a buyer drops by clinching in an iteration, 
	%at least one of $c=v'_i$ and $c\geq B_i-p_i$ holds for some $i$. 
	We show that if the demand $d_i$ of buyer $i$ decreases 
	by one in line 9 and still has a positive demand $d'_i$,
	then no buyer drops in the subsequent execution of Algorithm~2.
	After the demand update of buyer $i$, it still holds  
	$f_{x,d'}(N)=f_{x,d}(N)=f_{x,d}(N\setminus i)=f_{x,d'}(N\setminus i)$ 
	by Lemma \ref{equation_of_all_goods}.
	By Proposition \ref{clinch_amount}, 
	we have $\delta_i=f_{x,d'}(N)-f_{x,d'}(N\setminus i)=0$, which means $i$ is still active 
	just after the execution of Algorithm 2.
	
	Now we consider the clinching amount $\delta_k$ of buyer $k\neq i$ in Algorithm 2.
	If $i\lesssim k$ just before the demand update, it follows from Lemma \ref{unsaturation_demand} that $d_i\leq d_k$.
	Thus, it follows from Proposition \ref{unsaturation_clinch} that 
	$\delta_k=d_i-d'_i\leq d_k-d'_i<d_k$, where the last inequality follows by $d'_i>0$.
	Otherwise, it follows from Proposition \ref{unsaturation_clinch} that $\delta_k<d_i-d'_i=1$,
	and thus, we have $\delta_k=0$ since $\delta_k$ is an integer.
	Therefore, no buyer drops out of the auction by clinching just after $i$'s demand update.
	This implies that if a buyer drops by clinching, 
	then there exists a buyer who drops by the demand update 
	just before the execution of Algorithm 2.
	\end{proof}
	\begin{proposition}
	\label{active_buyers_tight}
	After the execution of Algorithm 2, it holds $x^{\rm f}(S) = f(S)$, where $S$ is the set of active buyers.
	\end{proposition}
	
	\begin{proof}
	We show that if the set of active buyers 
	after an execution of Algorithm 2
	is tight, then the set of active buyers after the next iteration is tight.
	At the beginning of the auction, all buyers are active, and 
	we have $x^{\rm f}(N)=f(N)$ by Proposition \ref{all_goods}.
	
	Suppose that $x^{\rm f} (T)=f(T)$, where $T$ represents the set of active buyers in an iteration.
	The claim holds trivially if the number of active buyers remains unchanged.
	From Proposition \ref{dropping_of_buyers}, it suffices to consider the case 
	in which there exists a buyer $i\in T$ who drops by the demand update in line 5 or 9.
	Let $x$ and $d$ denote the allocation and the demand vector, 
	respectively, immediately before the demand update of $i$.
	For buyer $k$ ($k\neq i$), if $k\lesssim i$ just after the demand update, 
	then by Proposition \ref{clinch_amount}, we have 
	$\delta_k=f_{x,d^{-i}}(N)-f_{x,d^{-i}}(N\setminus k)=d_k$.
	Let $S$ denote the set of $i$-saturated buyers just before the demand update of $i$. 
	Obviously, it holds $i\notin S$ by Lemma \ref{reflexive}. 
	By Proposition \ref{unsaturation_clinch}, it holds $\delta_j<d_j$ for each $j\in S$; thus, 
	only buyers in $S$ are still active after the execution of Algorithm 2.
	It remains to show that $x^{\rm f} (S)=f(S)$, 
	which is obtained by 
	$f_{x,d}(S)=f(S)-x(S)$ and $f_{x,d}(S)=x^{\rm f}(S)-x(S)$.


	Firstly, we show that $f_{x,d}(S)=f(S)-x(S)$.
	Suppose that there exists a set $S'\subset S$ with 
	$f_{x,d}(S)=f(S')-x(S')+d(S\setminus S')$ by Theorem \ref{invariant}.
	Then, for any $j\in S\setminus S'$, it holds  
	\[
	f_{x,d}(S\setminus j)+d_j\leq f(S')-x(S')+d(S\setminus (S'\cup j))+d_j= f_{x,d}(S)\leq f_{x,d}(S\setminus j)+d_j,
	\]
	where the first (resp. second) inequality holds by the minimality of $f_{x,d}(S\setminus j)$ 
	(resp. $f_{x,d}(S)$) due to Theorem \ref{invariant}.
	Since the above inequalities hold by equality, we have  
	$f_{x,d^{-i}}(S)-f_{x,d^{-i}}(S\setminus j)=f_{x,d}(S)-f_{x,d}(S\setminus j)=d_j$, 
	where the first equality holds by  $i\notin S$ and Observation \ref{fact}.
	Subsequently, let $l$ be a buyer in $N\setminus S$.
	Then, we have 
	\[
	f_{x,d^{-l}}(N\setminus i)+d_i\leq f_{x,d^{-l}}(N)\leq f_{x,d^{-l}}(N\setminus i)+d_i,
	\]
	where the first inequality holds by the monotonicity of $f_{x,d^{-l}}$ and $d_i=0$, and 
	the second inequality holds by equation (\ref{naive}).
	Therefore, the inequality holds by equality, which means $i\lesssim l$.
	By iteratively applying Lemma \ref{characterization_unsaturation}, 
	it holds that $f_{x,d^{-i}}(N)=f_{x,d^{-i}}(S)+d(N\setminus S)$ and 
	 $f_{x,d^{-i}}(N\setminus j)=f_{x,d^{-i}}(S\setminus j)+d(N\setminus S)$.
	Using this, we have 
	$f_{x,d^{-i}}(N)-f_{x,d^{-i}}(N\setminus j)=f_{x,d^{-i}}(S)-f_{x,d^{-i}}(S\setminus j)=d_j$, 
	which means $j\lesssim i$, contradicting $j$ is $i$-saturated.
	Therefore, we have $f_{x,d}(S)=f(S)-x(S)$. 
	
	Secondly, we show that $f_{x,d}(S)=x^{\rm f}(S)-x(S)$.
	By Lemma \ref{beginning}, it holds $x(T)+f_{x,d}(T)=f(T)$
	at the beginning of the auction.
	Moreover, the value $x(T)+f_{x,d}(T)$
	is unchanged due to the execution of Algorithm 2 by Lemma \ref{unchanged}, 
	and is non-increasing by the demand update of buyers in $T$ in lines 5 and 9 due to Theorem \ref{invariant}.
	Then, by the assumption at the end of the auction, $x(T)+f_{x,d}(T)=x^{\rm f}(T)=f(T)$, 
	which implies $f_{x,d}(T)=f(T)-x(T)$ holds throughout the auction.
	Let $x'$ and $d'$ denote the allocation and the demand vector, 
	respectively, just after the execution of Algorithm 2.
	Then, since $d'_i=0$ for each $i\in T\setminus S$, 
	we have $f_{x',d'}(T)=f_{x',d'}(S)$ by Theorem \ref{invariant}.
	Since buyers in $T\setminus S$ are dropping just after the execution of Algorithm 2, we have
	\[
	x'(S)+f_{x',d'}(S)=x'(S)+f_{x',d'}(T)=f(T)-x'(T\setminus S)=f(T)-x^{\rm f}(T\setminus S)=x^{\rm f}(S), 
	\]
	where the second equality holds by $f_{x',d'}(T)=f(T)-x'(T)$,
	and the last equality holds by the assumption that $x^{\rm f} (T)=f(T)$.
	Moreover, $x(S)+f_{x,d}(S)$ is unchanged 
	by the demand update of $i$ due to Theorem \ref{invariant} 
	and the execution of Algorithm 2 due to Lemma \ref{unchanged}.
	Using this, we have 
	\[
	x(S)+f_{x,d}(S)=x'(S)+f_{x',d'}(S)=x^{\rm f}(S).
	\]
	Therefore, we have $f_{x,d}(S)=x^{\rm f}(S)-x(S)$.
	\end{proof}

	
	
	
	
	This proposition immediately proves Property (iii) in Theorem \ref{tightsets}.
	Combining this with Proposition \ref{dropping_of_buyers}, we have 
	$\phi_i=\phi_j$ for each $k=1,2,\ldots,t$ and $i, j\in X_{k}\setminus X_{k-1}$.
	Now we prove Theorem~\ref{tightsets}. 
	
\begin{proof}[Proof of Theorem \ref{tightsets}]
Property (iii) holds from Proposition \ref{active_buyers_tight}.
For Property (i), let $i_k\ (k\in N)$ be a buyer who drops in line 5 or 9 of Algorithm 1.
Buyer $i_k$ drops by either of the following:
\begin{itemize}
\item The price is equal to their valuation (line 5).
\item The price is equal to the remaining budget (line 9). 
\end{itemize}
These two cases correspond to the first and the second cases in Property (i), respectively.

For Property (ii), let $i\in X_{k}\setminus (X_{k-1}\cup i_k)$.
Suppose that buyer $i$'s demand has never been updated in line 5 or 9 before the dropping.
Then, by Lemma \ref{d} and the polymatroid constraint $x^{\rm f}_i\leq f(i)$, 
we have $d_i=f(i)+1-x^{\rm f}_i>0$, 
which means $i$ must be an active buyer.
Therefore, the demand $d_i$ has been updated in line 9 at least once. 
After the demand update, $d_i$ changes with keeping the inequality 
$d_i\geq \frac{B_i-p_i}{c}-1$ by Lemma \ref{d}.
Just when the demand $d_i$ decreases to zero by clinching, 
$\frac{B_i-p_i}{\phi_i}$ decreases by the same amount in Algorithm 2.
This implies $B_i-p^{\rm f}_i\leq\phi_i$.
Moreover, $\phi_i=\phi_{i_k}$ holds by Propositions \ref{dropping_of_buyers} and \ref{active_buyers_tight}.

Suppose that there exists a buyer $\ell \in X_k\setminus (X_{k-1}\cup i_k)$ 
with $B_{\ell}-p^{\rm f}_{\ell}=\phi_{\ell}$.
This implies just before the execution of Algorithm 2 that $\ell$ drops, 
it holds $d_{\ell}=\frac{B_{\ell}-p_{\ell}}{\phi_{\ell}}-1$ because 
$d_{\ell}$ and $\frac{B_{\ell}-p_{\ell}}{\phi_{\ell}}$ decrease by the same amount in Algorithm 2.
Then, by Lemma \ref{d}, $\ell$'s demand has been updated in line 9 in the same iteration as their dropping. 
This means $i_{k}$ drops by the demand update in line 9 (after that of $\ell$), 
and thus $B_{i_{k}}-p^{\rm f}_{i_{k}}=\phi_{i_{k}}$.
Moreover, just before the demand update of $i_k$, the buyers whose valuations are equal to $\phi_{i_k}$ 
have already dropped in the demand update in line 5 in the same iteration.
Therefore, it holds $v_i>\phi_i$ for $i\in X_k\setminus X_{k-1}$. 
\end{proof}

%before the update of $i_k$'s demand when $c=\phi_{i_k}$.
%After the update, $\ell$'s demand is changed keeping the equality 
%$d_{\ell}=\frac{B_{\ell}-p_{\ell}}{\phi_{\ell}}-1$.

\subsection{Relation to No Trading Path Property}
Our tight sets lemma provides a new and stronger evidence
for the efficiency of clinching auctions for indivisible goods.
Indeed, it explains {\it no trading path property}, that was used to prove the efficiency
for the special case of our setting in Fiat et al. \cite{FLSS2011}  and Colini-Baldeschi et al. \cite{BHLS2015}.
We reveal the relationship between our tight sets lemma and their
no trading path property. 
Although this result is unrelated to efficiency guarantees in Section 5, 
it is interesting in its own right.
	
	First, we introduce a {\it trading pair}, which is a generalization of 
	the trading path in \cite{BHLS2015, FLSS2011}.
	We define the saturation and dependence functions to characterize the trading pair. 
	A saturation function ${\rm sat}:P \to 2^N$ is defined by 
	${\rm sat}(x):=\{i\mid i\in N, \forall \alpha>0, x+\alpha \chi_i\notin P\}$,
	where $\chi_{A}$ represents an indicator vector for $A\subseteq N$.
	Moreover, for $x\in P$ and $i \in {\rm sat}(x)$, 
	a dependence function ${\rm dep} : P \times N \to 2^N$ is defined by
	${\rm dep}(x,i):=\{i'\mid i'\in N, \exists \alpha>0, x+\alpha (\chi_i-\chi_{i'})\in P\}$.
	Then, the trading pair is defined by the following: 
		
\begin{definition}
\label{trading_pair}
{\rm A pair of buyers $(i, j)$ is a} trading pair {\rm with respect to the allocation
$(x, p)$ if the following hold: (i)\, $j\in {\rm dep}(x, i)$, (ii)\, $v_i$ is strictly greater than $v_j$, and 
(iii) the remaining budget $B_i-p_i$ is not less than $v_j$.}
\end{definition}

	Using our tight sets lemma, 
	we can easily show that our mechanism satisfies the no trading pair property.
	Proposition \ref{no_trading} implies that we obtain a stronger evidence 
	for the efficiency of clinching auctions for indivisible goods.

\begin{proposition}
\label{no_trading}
There is no trading pair with respect to $(x^{\rm f}, p^{\rm f})$.
\end{proposition}
\begin{proof}
Suppose that there exists a trading pair $(i, j)$, and $i\in X_{k'}\setminus X_{k'-1}$ for some $k'\in \{1,2,\ldots,t\}$, where $\{X_k\}_{k\in \{0,1\ldots,t\}}$ is the chain of tight sets in Theorem \ref{tightsets}.
By Property (iii) of Theorem \ref{tightsets}, it holds that $x^{\rm f}(X_{k'})=f(X_{k'})$.
Thus, we have $j\in X_{k'}$ by $j\in {\rm dep}(x, i)$ 
and $i\in X_{k'}\setminus X_{k'-1}$.
Thus, we have $v_j\geq \phi_j\geq \phi_i$ by Theorem \ref{tightsets}.
If $v_{i}=\phi_{i}$, then it holds $v_j\geq \phi_j\geq \phi_i=v_i$, 
contradicting the condition (ii) of Definition \ref{trading_pair}. 
It suffices to consider the case where $v_i>\phi_i$.
By Property (i) and (ii) of Theorem \ref{tightsets}, it holds $B_i-p^{\rm f}_i\leq \phi_i$.
Thus, if $v_j>\phi_j$, it holds that $B_i-p^{\rm f}_i\leq \phi_i\leq \phi_j<v_j$, which contradicts the condition (iii) of Definition \ref{trading_pair}. 
It remains the case where $v_i>\phi_i$ and $v_j=\phi_j$.
If $\phi_j>\phi_i$, 
we have $B_i-p^{\rm f}_i\leq\phi_i< \phi_j=v_j$, which contradicts the condition (iii) of Definition \ref{trading_pair}.
If $\phi_i=\phi_j$, then by $j\in X_{k'}$, $i\in X_{k'}\setminus X_{k'-1}$, $v_i>\phi_i$, and $v_j=\phi_j$, 
buyer $i$ drops out in line 6 of Algorithm 1 
because $i$ drops out of the auction before or in the same iteration as $j$. 
By (the proof of) Property (ii) of Theorem \ref{tightsets}, we have $(B_i-p^{\rm f}_i)/\phi_i<1$. 
Then, again, we have $B_i-p^{\rm f}_i<\phi_i=\phi_j=v_j$, 
which contradicts the condition (iii) of Definition \ref{trading_pair}.
\end{proof}





\section{Efficiency}
In this section, we provide three types of efficiency guarantees for our mechanism.
Our tight sets lemma (Theorem \ref{tightsets}) plays critical roles in the proofs.

\subsection{Pareto Optimality}
We first show that our mechanism satisfies Pareto optimality, which has been the 
efficiency goal in many previous studies for clinching auctions with budgets. 

	\begin{theorem}
	\label{PO}
	Our mechanism satisfies PO.
	\end{theorem}

The proof is an inductive argument with respect to $\{X_k\}_{k\in \{0,1,\ldots,t\}}$ 
in the tight set lemma (Theorem 4), as in the proof of Goel et al. \cite{GMP2014} for their divisible setting.
Instead of dropping prices (as they used),  
we use a new non-increasing sequence $\{\theta_k\}_{k\in \{1,2,\ldots,t\}}$ defined by  
\begin{equation}
\label{theta}
\theta_k=\displaystyle \min_{i\in X_{k}}v_{i}\quad (k\in \{1,2,\ldots,t\})  
\end{equation}
due to the difference of the tight sets lemma.
%it might hold $B_i-p^{\rm f}_i=\phi_{i_k}$ for some $k$ and $i\in X_k\setminus X_{k-1}$ with $v_i>\phi_{i_k}$. 
By construction, the following properties hold.  



%	In the proof, we apply mathematical induction for $\{X_k\}_{k\in \{0,1,\ldots,t\}}$ 
%	in Theorem \ref{tightsets} as in Sato \cite{S2023}.
%	On the other hand, we need to use $\{\phi_{i_k}\}_{k\in \{1,2\ldots,t\}}$ instead of 
%	$\{v_{i_k}\}_{k\in \{1,2\ldots,t\}}$ because 
%	$\{v_{i_k}\}_{k\in \{1,2\ldots,t\}}$ is not necessarily monotone in our setting.
%	Then, by equation (\ref{LW_OPT}), we have 



\begin{lemma}
\label{prepare_PO}
The sequence $\{\theta_k\}_{k\in \{1,\ldots,t\}}$ constructed by (\ref{theta}) satisfies the following: 
\begin{itemize}
\item[\ \ (i)] It holds $v_i\geq \theta_{k}$ for each $k\in\{1,2,\ldots,t\}$ and $i\in X_k$.
\item[(ii)] $\{\theta_{k}\}_{k\in \{1,\ldots,t\}}$ is non-increasing on $k$.
\item[\ \ (iii)] For each $k\in\{1,2,\ldots,t\}$, if $\phi_{i_k}=v_{i_k}$, then $\phi_{i_k}=\theta_{k}=v_{i_k}$.
Otherwise, $\phi_{i_k}<\theta_{k}\leq v_{i_k}$.
\item[\ \ \ (iv)] It holds $B_i-p^{\rm f}_i<\theta_{k}$ for each $k\in\{1,2,\ldots,t\}$ and $i\in X_k\setminus X_{k-1}$ with $v_i>\phi_{i_k}$.
\end{itemize}
\end{lemma}

\begin{proof}
Property (i) holds by $\displaystyle \theta_k=\min_{j\in X_k}v_j\leq v_i$ for each $i\in X_k$.
Property (ii) holds by $\displaystyle \theta_k=\min_{i\in X_k}v_i\leq \min_{i\in X_{k-1}}v_i= \theta_{k-1}$ for each $k$, 
where the inequality holds by $X_{k-1}\subseteq X_{k}$ due to Theorem \ref{tightsets}.

Now we show Property (iii). Since $\{\phi_{i_k}\}_{k\in \{1,2,\ldots,t\}}$ is non-increasing on $k$, 
we have $v_i\geq \phi_i=\phi_{i_{k'}}\geq \phi_{i_k}$ for each $i\in X_k$, 
where $k'$ is some positive integer satisfying $k'\leq k$ and $i\in X_{k'}\setminus X_{k'-1}$.
If $v_{i_k}=\phi_{i_k}$, then we have $\displaystyle \theta_k=\min_{i\in X_k}v_i=v_{i_k}$ 
by $v_i\geq \phi_{i_k}$ for each $i\in X_k$.
If $v_{i_k}>\phi_{i_k}$, then buyer $i_k$ is dropped in line 9 of Algorithm 1.
This means that $X_k$ comprises buyers with the valuations higher than $\phi_{i_k}$ 
since the set of buyers $i$ with $v_{i}=\phi_{i_k}$ has already dropped in line 5 in the same iteration 
if they exist. Therefore, we have $\displaystyle \phi_{i_k}<\min_{i\in X_k}v_i\ (=\theta_k) \leq v_{i_k}$.

Finally, we show Property (iv).
If $v_{i_k}>\phi_{i_k}$, then we have $B_i-p_i\leq \phi_{i_k}<\theta_k$ 
for each $k\in \{1,2\ldots,t\}$ and $i\in X_k\setminus X_{k-1}$ by Theorem \ref{tightsets} and Property (iii). 
If $v_{i_k}=\phi_{i_k}$, then by Property (iii), it holds $v_{i_k}=\theta_k=\phi_{i_k}$.
This means that buyers in $i\in X_{k}\setminus X_{k-1}$ drop in line 5 or 6 of Algorithm 1. 
Then, it holds $B_i-p_i< \phi_{i_k}=\theta_k$ for each $k\in \{1,2\ldots,t\}$ and $i\in X_k\setminus (X_{k-1}\cup i_k)$.
Therefore, it holds $B_i-p^{\rm f}_i<\theta_{k}$ for each 
$k\in\{1,2,\ldots,t\}$ and $i\in X_k\setminus X_{k-1}$ with $v_i>\phi_{i_k}$.
\end{proof}




\begin{proof}[Proof of Theorem \ref{PO}]
Suppose that there exists an allocation $\mathcal A:=(x',p')$ satisfying (i)
$v_i x^{\rm f}_i-p^{\rm f}_i\leq v_i x'_i - p'_i$ for each $i\in N$, (ii) $p^{\rm f}(N)\leq p'(N)$, and 
(iii) at least one inequality holds without equality. 
Combining these inequalities, 
we have $\sum_{i\in N}v_i x^{\rm f}<\sum_{i\in N}v_i x'_i$.

Let $\{\theta_{k}\}_{k\in \{1,2,\ldots,t\}}$ be a non-increasing sequence in Lemma \ref{prepare_PO}.
Then, we show that $\theta_{k} (x^{\rm f}_{i}-x'_{i}) \leq p^{\rm f}_{i}- p'_{i}$ for each $k\in \{1,2,\ldots,t\}$ and $i\in X_{k}\setminus X_{k-1}$ by the following case-by-case analysis.
%We use Lemma \ref{prepare_PO} in .
%$B_i-p^{\rm f}_i<\theta_k$ and $v_i\geq \theta_k$ 
%for each $k$ and $i\in X_{k}\setminus (X_{k-1}\cup i_k)$, which hold by 

\begin{enumerate}
\item Suppose that $x^{\rm f}_i\geq x'_i$.
% and $\phi_{i_k}=\theta_{k}=v_{i_k}$. 
By Property (i) of Lemma \ref{prepare_PO}, it holds $\theta_{k} (x^{\rm f}_i-x'_i)\leq v_i (x^{\rm f}_i-x'_i) \leq p^{\rm f}_i- p'_i$.
%\item Suppose that $x^{\rm f}_i\geq x'_i$ and $\phi_{i_k}<\theta_{k}\leq v_{i_k}$.
%Then, it holds $\theta_{k} (x^{\rm f}_i-x'_i)< v_i (x^{\rm f}_i-x'_i) \leq p^{\rm f}_i- p'_i$.
\item Suppose that $x^{\rm f}_i< x'_i$ and $v_i> \phi_{i_k}$. By the indivisibility of the good and $p'_i\leq B_i<p^{\rm f}_i+\theta_{k}$ from Property (iv) of Lemma \ref{prepare_PO}, 
it holds $\theta_{k} (x^{\rm f}_i-x'_i)\leq -\theta_{k} < p^{\rm f}_i- p'_i$.
\item Suppose that $x^{\rm f}_i< x'_i$ and $v_i=\phi_{i_k}$. 
This means that buyer $i_k$ drops out of the auction in line 5 of Algorithm 1.
Thus, we have  $v_{i_k}=\theta_{k}=\phi_{i_k}=v_i$ 
by Property (iii) of Lemma \ref{prepare_PO}.
Then, it holds $\theta_{k} (x^{\rm f}_i-x'_i)=v_{i} (x^{\rm f}_i-x'_i) \leq p^{\rm f}_i- p'_i$.
\end{enumerate}
From the above, we can also see that if all the inequalities hold in equality, 
we have $v_i=\theta_{k}$ or $x^{\rm f}_i= x'_i$ 
for each $k\in \{1,2,\ldots,t\}$ and $i\in X_k\setminus X_{k-1}$.


Now we prove
$p^{\rm f}(X_k)-p'(X_k)\geq \theta_{k}(x^{\rm f}(X_k)-x'(X_{k}))\geq 0$
for each $k\in \{0,1,\ldots,t\}$, where $\theta_0:=\theta_1$.
For $k=0$, the inequality trivially holds by $X_0=\emptyset$.
Suppose that the inequality holds for $k-1$. We prove it holds for $k$. 
Since $x'(X_{k-1})\leq f(X_{k-1})=x^{\rm f}(X_{k-1})$ and $x'(X_{k})\leq f(X_{k})=x^{\rm f}(X_{k})$ 
by Property (iii) of Theorem \ref{tightsets}, we have  
\begin{eqnarray*}
0&\leq&\theta_{k}(x^{\rm f}(X_{k})-x'(X_{k}))\leq\theta_{k-1}(x^{\rm f}(X_{k-1})-x'(X_{k-1}))+\theta_{k}(x^{\rm f}(X_{k}\setminus X_{k-1})-x'(X_{k}\setminus X_{k-1}))\\
&\leq& p^{\rm f}(X_{k-1})-p'(X_{k-1})+\theta_{k}(x^{\rm f}(X_{k}\setminus X_{k-1})-x'(X_{k}\setminus X_{k-1}))\\
&\leq& p^{\rm f}(X_{k-1})-p'(X_{k-1})+p^{\rm f}(X_{k}\setminus X_{k-1})-p'(X_{k}\setminus X_{k-1}))=p^{\rm f}(X_{k})-p'(X_{k}),
\end{eqnarray*}
where the second inequality follows by Property (ii) of Lemma \ref{prepare_PO}, the third inequality follows by the assumption, and the fourth inequality follows by 
$\theta_{k} (x^{\rm f}_{i}-x'_{i}) \leq p^{\rm f}_{i}- p'_{i}$ for each 
$k\in \{1,2,\ldots,t\}$ and $i\in X_{k}\setminus X_{k-1}$.
By substituting $k$ with $t$, we have $p^{\rm f}(N)-p'(N)\geq \theta_{t}(x^{\rm f}(N)-x'(N))\geq 0.$
Since we assume $p^{\rm f}(N)\leq p'(N)$, all the inequalities hold in equality.
Therefore, 
we have $v_i=\theta_{k}$ or $x^{\rm f}_i= x'_i$ 
for each $k\in \{1,2,\ldots,t\}$ and $i\in X_k\setminus X_{k-1}$.

Using this, we prove $\sum_{i\in X_{k}}v_i (x^{\rm f}_i-x'_i)\geq \theta_{k}(x^{\rm f}(X_{k})-x'(X_{k}))\geq 0$  for each $k\in \{0,1,\ldots,t\}$.
For $k=0$, the inequality trivially holds by $X_0=\emptyset$. 
Suppose that $\sum_{i\in X_{k-1}}v_i (x^{\rm f}_i-x'_i)\geq \theta_{k-1}(x^{\rm f}(X_{k-1})-x'(X_{k-1}))$ holds for $k-1$. We prove it holds for $k$. 
Then, we have
\begin{eqnarray*}
\sum_{i\in X_{k}}v_i (x^{\rm f}_i-x'_i)&\geq& \theta_{k-1} (x^{\rm f}(X_{k-1})-x'(X_{k-1}))+\sum_{i\in X_{k}\setminus X_{k-1}}\theta_{k} (x^{\rm f}_i-x'_i)\\
&\geq& \theta_{k} (x^{\rm f}(X_{k-1})-x'(X_{k-1}))+\sum_{i\in X_{k}\setminus X_{k-1}}\theta_{k} (x^{\rm f}_i-x'_i)\\
&=& \theta_{k} (x^{\rm f}(X_{k})-x'(X_{k}))\geq 0,
\end{eqnarray*}
where the first inequality holds by the assumption, and $v_i=\theta_{k}$ or $x^{\rm f}_i= x'_i$ for each $i\in X_{k}\setminus X_{k-1}$, and 
the second inequality holds by Property (ii) of Lemma \ref{prepare_PO}, 
and the third inequality holds by $x^{\rm f}(X_{k})=f(X_{k})\geq x'(X_{k})$ 
by Property (iii) of Theorem \ref{tightsets}.
Thus, it holds $\sum_{i\in X_{k}}v_i (x^{\rm f}_i-x'_i)\geq \theta_{k} (x^{\rm f}(X_{k})-x'(X_{k}))\geq 0$ for any $k\in \{0,1,\ldots,t\}$.
Substitute $k$ with $t$, we have $\sum_{i\in N}v_i x^{\rm f}_i\geq \sum_{i\in N}v_i x'_i$, which contradicts the hypothesis. 
\end{proof}


\subsection{Liquid Welfare}
	Let ${\rm LW}^{\rm M}$ and ${\rm LW}^{\rm OPT}$ denote the LW of 
	our mechanism and the optimal LW, respectively.
	We provide the first LW guarantee for clinching auctions with indivisible goods, 
	which holds even in polymatroidal environments.
	The main result in this section is: 
	
	\begin{theorem}
	\label{LW}
	It holds ${\rm LW}^{\rm M}\geq \frac{1}{2}{\rm LW}^{\rm OPT}$.
	\end{theorem}
	\begin{remark}
	In Section 5.4.2, we provide an example to see that the inequality in Theorem \ref{LW} is tight. 	
	Moreover, by extending Theorem 5.1 of Dobzinski and Leme \cite{DL2014}, 
	we also provide a lower bound of 4/3 for the LW approximation in our indivisible setting.
	\end{remark}
	To prove Theorem \ref{LW}, we establish 
	\begin{equation}	
	\label{ProofLW}
	{\rm LW}^{\rm M}\geq p^{\rm f}(N)\geq {\rm LW}^{\rm OPT}-{\rm LW}^{\rm M}, 
	\end{equation}
	implying ${\rm LW}^{\rm M}\geq {\rm LW}^{\rm OPT}$ as in Sato \cite{S2023}. 
	The first inequality is easy by  
	\begin{equation}
	\label{lowerbound}
	{\rm LW}^{\rm M}=\sum_{i\in N}\min(v_i x^{\rm f}_i, B_i)\geq p^{\rm f}(N), 
	\end{equation}
	where 
	$v_i x^{\rm f}_i \geq p^{\rm f}_i$ holds by IR of buyers (Theorem \ref{IC_IR}) and 
	$B_i\geq p^{\rm f}_i$ holds by budget feasibility (Theorem \ref{BF}) for each $i\in N$.
	Therefore, in the following, we prove the second inequality of (\ref{ProofLW}).
	
	We follow the outline of the proof in Sato \cite{S2023}: 
We first provide the formula for an LW optimal allocation, and then show a lower bound on the number of goods remaining at any point in our mechanism. Since the remaining goods are sold at the price equal to or higher than the current one, we can provide a lower bound on future payments. Then, considering the initial step of the auction, we obtain a lower bound on the total payments sufficient for the LW guarantee.
	
	However, several differences due to indivisibility prevent us from making a straightforward extension. 
	One difference is the change in the formula of an LW optimal allocation. 
	The formula for the divisible setting of Sato \cite{S2023} 
	does not necessarily yield an LW optimal allocation in our indivisible setting.
	Another difference is the number of remaining goods. 
	Since the demands for fewer than one unit are rounded down due to the indivisibility, 
	the number of remaining goods might be fewer than the lower bound in \cite{S2023}. 
	These differences are illustrated in the examples of~Section 5.4.1. 

	To provide the formula for an LW optimal allocation in our setting, 
	we divide each buyer into two virtual buyers in the following way: 
	For each buyer $i\in N$, consider two copies $i_a, i_b$, 
	where $i_{a}$ represents a buyer whose valuation is $v_i$ and their budget is $\left\lfloor \frac{B_i}{v_i}\right\rfloor v_i$,
	and $i_{b}$ represents a buyer whose valuation and budget are $B_i-\left\lfloor \frac{B_i}{v_i}\right\rfloor v_i$\footnote{If $B_i-\left\lfloor \frac{B_i}{v_i}\right\rfloor v_i=0$, then we define 
	$0<v_{i_b}<v_i$ and $B_{i_b}=0$.}. 
	Let $N':= \cup_{i\in N}\{i_a, i_b\}$.
	We define a map $\Gamma:2^{N'}\to 2^N$ that outputs 
	the union of buyers $i$ with $\{i_a, i_b\}\cap S'\neq\emptyset$ for each $S'\subseteq N'$. 
	In addition, we define a new monotone submodular function $f': 2^{N'}\to \mathbb Z_{+}$ by 
	$f'(S'):=f(\Gamma(S'))\ (S'\subseteq N')$.  
	Note that $f'$ is again a monotone submodular function; see, e.g., Section 44.6g of Schrijver \cite{S2003}.
	Then, an allocation for the original market can be constructed 
	simply by summing the allocated goods of the corresponding two buyers from an allocation for $N'$.
	
	Then, an LW optimal allocation in our setting can be obtained by the greedy procedure described below.	
	Suppose that buyers in $N'$ are ordered in descending order according to their valuations.
	For each buyer $i'\in N'$, let $H_{i'}:=\{1,2,\ldots,i'-1\}$ denote the set of buyer $k\in N'$ 
	who has a higher valuation than $i'$, or who has the same valuation 
	as $i'$ and $\Gamma (k)$ is numbered before $\Gamma(i')$ in $N$. 
	Then, the following holds:
	\begin{proposition}
	\label{optimal}
	An LW optimal allocation $\tilde{x}^*:=(\tilde{x}^*_i)_{i\in N}$ is given by 
	$\tilde{x}^*_i=x^*_{i_{a}}+x^*_{i_{b}}$ for each $i\in N$, where 
	$x^{*}_{i'}=\min(\frac{B_{i'}}{v_{i'}},{\min_{H\subseteq H_{{i'}}}\{f'(H\cup i')-x^*_{i'}(H)}\})$ 
	for each $i'\in N'$.
	\end{proposition}
	The proof of Proposition \ref{optimal} is provided in Appendix A.
	Using this allocation, we have 
	\begin{equation}
	\label{LW_OPT}
	{\rm LW}^{\rm OPT}=\sum_{i\in N}\min(v_i \tilde{x}^{*}_i, B_i)=\sum_{i\in N} (v_{i_a} x^*_{i_a}+v_{i_b} x^*_{i_b}),
	\end{equation}
	where the second equality holds by the definition of $v_{i_b}$.
	Proposition \ref{optimal} indicates that for each $i\in N$, 
	the buyer $i_{a}$ (resp. $i_{b}$) can obtain at most 
	$\left\lfloor \frac{B_i}{v_i}\right\rfloor$ (resp. $1$) units in this optimal allocation.
	
	Using the LW optimal allocation, we then provide a lower bound 
	on the remaining goods in Algorithm 1.
	Consider an iteration of Algorithm 1, where $x$ and $d$ are the allocation 
	of goods and the demand vector, respectively, and $c$ is the price.
	Define $X:=\{i\in N\mid d_i>0\}$, $Y:=\{i\in X\mid x^{\rm f}_i< \tilde{x}^*_i\}$ as in Sato \cite{S2023}.
	In addition, we define $Y_c:=\{i\in Y\mid c< v_{i_{b}} \}$, and 
	$x^*_a(S):= \sum_{i\in S} x^*_{i_a}$ and $x^*_b(S) := \sum_{i\in S} x^*_{i_b}$ for any $S\subseteq N$.
	
	\begin{proposition}
	\label{opt_algo_relation}
	It holds $f_{x,d}(Y)-x_a^*(Y)+x(Y)-x_b^*(Y_c)\geq 0$.
	\end{proposition}
	
	If $Y_c$ of Proposition \ref{opt_algo_relation} is replaced by $Y$,
	the left-hand side is changed to $f_{x,d}(Y)-\tilde{x}^*(Y)+x(Y)$, 
	which is the same as the one in Theorem 3.9 of Sato \cite{S2023}. 
	However, due to the indivisibility, the number of remaining goods might be fewer than this bound. 
	Thus, to handle with the difference, 
	we use the optimal allocation of virtual buyers $x^*_a$ and $x^*_b$ instead of $\tilde{x}^*$, 
	and define a new set $Y_c$ to obtain the sharp lower bound. 
	
	In the proof of Proposition \ref{opt_algo_relation}, we use the following property, which is obtained by the fact that buyers clinch their allocated goods at a price less than or equal to the current price in Algorithm~1.
	\begin{observation}
	\label{x}
	It holds $x_i\geq \left\lceil \frac{p_i}{c}\right\rceil$ for each $i\in N$.
	\end{observation}
	\begin{proof}[Proof of Proposition \ref{opt_algo_relation}]
	Let $Y'\subseteq Y$ be a minimizer of $f_{x,d}(Y)$, i.e. 
	$f_{x,d}(Y)=f(Y')-x(Y')+d(Y\setminus Y')$
	by Theorem \ref{invariant}. 
	We define $\Delta:=(\Delta_i)_{i\in N}$ by 
	\[
	\Delta_i=
	\begin{cases}
	1& {\rm if}\ c< v_{i_{b}}\\
	0& {\rm otherwise}
	\end{cases}
	\]
	for each $i\in N$. Then, it holds 
	\begin{align*}
	&f_{x,d}(Y)-x_a^*(Y)+x(Y)-x_b^*(Y_c)\\
	&\quad=f(Y')-x(Y')+d(Y\setminus Y')-x_a^*(Y)+x(Y)-x_b^*(Y_c)\\
	&\quad=f(Y')-x_a^*(Y')-x_b^*(Y'\cap Y_c)+x(Y\setminus Y')+d(Y\setminus Y')-x_a^*(Y\setminus Y')-x_b^*(Y_c\setminus Y')\\
	&\quad\geq x(Y\setminus Y')+d(Y\setminus Y')-x_a^*(Y\setminus Y')-x_b^*(Y_c\setminus Y')
	\geq\sum_{i\in Y\setminus Y'}(x_i+d_i-x^*_{i_a}-\Delta_i), 
	\end{align*}
	where the first inequality holds by polymatroid constraint 
	$f(Y')-x_a^*(Y')-x_b^*(Y'\cap Y_c)\geq f(Y')-\tilde{x}^*(Y')\geq 0$, 
	and the second inequality holds by 
	$x^*_{b}(Y_c\setminus Y')\leq |Y_c\setminus Y'|=\Delta(Y\setminus Y')$ 
	due to the definition of $\Delta$ and $x^*_{i_b}\in\{0,1\}$ for each $i\in N$.
	In the following, we prove $x_i+d_i-x^*_{i_{a}}-\Delta_i\geq 0$ for each $i\in Y\setminus Y'$.
	If the demand $d_i$ has never been updated in line 9, 
	by Lemma \ref{d} and the polymatroid constraint, 
	we have $x_i+d_i-x^*_{i_{a}}-\Delta_i= f(i)+1-x^*_{i_{a}}-\Delta_i\geq 0$.
	Suppose that the demand $d_i$ has been updated in line 9 at least once.
	We have 
	$x_i\geq \left\lceil \frac{p_i}{c}\right\rceil$ 
	by Observation \ref{x} and $x^*_{i_{a}}\leq \left\lfloor\frac{B_i}{v_i}\right\rfloor$ by Proposition \ref{optimal}.
	If $c< v_{i_{b}}<v_i$, then $\Delta_i=1$ and 
	$x_i\geq \left\lceil \frac{p_i}{c}\right\rceil\geq\left\lceil \frac{p_i}{v_{i_{b}}}\right\rceil$. We have 
	\[
	d_i\geq \left\lfloor \frac{B_i-p_i}{v_{i_{b}}}\right\rfloor\geq 
	\left\lfloor \frac{\left\lfloor \frac{B_i}{v_i}\right\rfloor v_{i_b}+v_{i_{b}}-p_i}{v_{i_{b}}}\right\rfloor
	=\left\lfloor \frac{B_i}{v_i}\right\rfloor +1-\left\lceil\frac{p_i}{v_{i_{b}}}\right\rceil, 
	\]
	where the first inequality holds by Lemma \ref{d} and $c< v_{i_{b}}$, 
	and the second inequality holds by $B_i=\left\lfloor \frac{B_i}{v_i}\right\rfloor v_i+v_{i_{b}}\geq 
	\left\lfloor \frac{B_i}{v_i}\right\rfloor v_{i_b}+v_{i_{b}}$ due to the definition of $v_{i_b}$.
	Therefore, we have
	\begin{eqnarray*}
	x_i+d_i-x^*_{i_{a}}-\Delta_i&\geq& %\left\lceil \frac{p_i}{v_{i_{b}}}\right\rceil+\left\lfloor \frac{\left\lfloor \frac{B_i}{v_i}\right\rfloor 
	%v_{i_b}+v_{i_{b}}-p_i}{v_{i_{b}}}\right\rfloor- \left\lfloor \frac{B_i}{v_i}\right\rfloor-1
	%&=& 
	\left\lceil\frac{p_i}{v_{i_{b}}}\right\rceil+ \left\lfloor \frac{B_i}{v_i}\right\rfloor+1-\left\lceil\frac{p_i}{v_{i_{b}}}\right\rceil- \left\lfloor \frac{B_i}{v_i}\right\rfloor-1=0.
	\end{eqnarray*}
	If $v_{i_{b}}\leq c$, then $\Delta_i=0$. By Lemma \ref{d} and $c\leq v_i$, we have 
	$d_i\geq\left\lfloor \frac{B_i-p_i}{v_i}\right\rfloor$ because 
	it holds $d_i=\frac{B_i-p_i}{c}-1$ only if $c<v_i$ and $\frac{B_i-p_i}{c}\in \mathbb Z_{+}$. 
	Therefore, we have 
	\[
	x_i+d_i-x^*_{i_{a}}-\Delta_i\geq \left\lceil \frac{p_i}{v_i}\right\rceil+\left\lfloor \frac{B_i-p_i}{v_i}\right\rfloor- \left\lfloor \frac{B_i}{v_i}\right\rfloor\geq \left\lceil \frac{p_i}{v_i}\right\rceil+\left\lfloor \frac{B_i}{v_i}\right\rfloor-\left\lceil \frac{p_i}{v_i}\right\rceil- \left\lfloor \frac{B_i}{v_i}\right\rfloor=0.
	\]
	\end{proof}
	
	Using Proposition \ref{opt_algo_relation}, 
	we provide a lower bound of payments in our mechanism for the set of buyer $i$ 
	with $x^{\rm f}_i\geq \tilde{x}^*_i$.
	As in Proposition \ref{opt_algo_relation}, 
	we use the optimal allocation of virtual buyers instead of 
	that of original buyers.
	
	
	\begin{theorem}
	\label{payment}
	It holds
	$\displaystyle \sum_{i\in N; x^{\rm f}_i\geq \tilde{x}^*_i}p^{\rm f}_i\geq \sum_{i\in N; x^{\rm f}_i< \tilde{x}^*_i}
	(\phi_{i}x_{i_{a}}^*+v_{i_{b}}x_{i_{b}}^*-\phi_{i}x^{\rm f}_{i}).$
	\end{theorem}

	In the proof, we use the following proposition derived from our tight sets lemma. 
	This proposition illustrates the relationship between the allocation of goods in our mechanism and 
	the optimal allocation for some buyers.

	\begin{proposition}
	\label{algo<opt}
	If $x^{\rm f}_i<\tilde{x}^*_i$ and $v_i> \phi_i$ for some $i\in N$, then it holds $x^{\rm f}_i=x^*_{i_{a}}=\left\lfloor \frac{B_i}{v_i}\right\rfloor$,
	$x^*_{i_{b}}=1$, and $v_{i_{b}}\leq \phi_i$.
	\end{proposition}
	To prove this, we use the following lemma.
	\begin{lemma}
	\label{prepare_OPT}
	It holds $x^*_{i_{b}}=\frac{B_{i_b}}{v_{i_b}}=1$ only if $x^*_{i_{a}}
	=\frac{B_{i_a}}{v_{i_a}}=\left\lfloor \frac{B_i}{v_i}\right\rfloor$ for each $i\in N$. 
	\end{lemma} 
	\begin{proof}
	Suppose that $x^*_{i_{a}}<\frac{B_{i_a}}{v_{i_a}}=\left\lfloor \frac{B_i}{v_i}\right\rfloor$ 
	for some $i\in N$. 
	By Proposition \ref{optimal}, there exists a set $H^*\subseteq H_{i_{a}}$ such that 
	$x^*(H^*\cup i_{a})=f'(H^*\cup i_{a})$.
	By the definitions of $f'$ and $\Gamma$, 
	it holds $f'(H^*\cup i_{a}\cup i_{b})=f(\Gamma(H^*)\cup i)=f'(H^*\cup i_{a})$. 
	By Proposition \ref{optimal} and 
	$H_{i_{b}}\supseteq H_{i_{a}}\cup i_{a}$ due to $v_{i_{a}}>v_{i_{b}}$, 
	we have 
	\[
	x^*_{i_{b}}\leq\min_{H\subseteq H_{{i_{b}}}}\{f'(H\cup i_{b})-x^*(H)\}
	\leq f'(H^*\cup i_{a} \cup i_{b})-x^*(H\cup i_{a})=f'(H^*\cup i_{a})-x^*(H\cup i_{a})=0.
	\]
	\end{proof}


	
	\begin{proof}[Proof of Proposition \ref{algo<opt}]
	By Property (ii) of Theorem \ref{tightsets} and $v_i>\phi_i$, 
	it holds $\left\lfloor \frac{B_i-p^{\rm f}_i}{v_i} \right\rfloor\leq 
	\frac{B_i-p^{\rm f}_i}{v_i}<\frac{B_i-p^{\rm f}_i}{\phi_i}\leq 1$, 
	which means $\left\lfloor \frac{B_i-p^{\rm f}_i}{v_i}\right\rfloor =0$.
	Combining this with Observation \ref{x} and Proposition \ref{optimal}, we have 
	$x^{\rm f}_i\geq \left\lceil \frac{p^{\rm f}_i}{v_i}\right\rceil \geq \left\lfloor \frac{B_i}{v_i}\right\rfloor\geq x^*_{i_{a}}$.
	If $x^{\rm f}_i<x^*_i$, this implies 
	$x^{\rm f}_i=x^*_{i_{a}}=\left\lfloor \frac{B_i}{v_i}\right\rfloor$ and $x^*_{i_{b}}=1$
	by Lemma~\ref{prepare_OPT}.

	Suppose that $v_{i_{b}}>\phi_i$. 
	Since each buyer must clinch the goods at the price equal to or less than their dropping price,  
	we have $p^{\rm f}_i\leq \phi_i x^{\rm f}_i= \left\lfloor \frac{B_i}{v_i}\right\rfloor\phi_i$.
	By $v_i\geq\phi_i$ and the definition of $v_{i_{b}}$, we have
	\begin{eqnarray*}
	B_i-p^{\rm f}_i \geq  \left\lfloor\frac{B_i}{v_i}\right\rfloor v_i+v_{i_{b}}-
	\left\lfloor \frac{B_i}{v_i}\right\rfloor \phi_i
	\geq v_{i_{b}}>\phi_i,
	\end{eqnarray*}
	contradicting Property (i) or (ii) of Theorem \ref{tightsets}. Thus, we have $v_{i_{b}}\leq \phi_i$.
	\end{proof}
	
	In the proof of Theorem \ref{payment}, we use {\it backward} mathematical induction as in the proof of Theorem 5.9 in Sato \cite{S2023}. 
	We show that, throughout the execution of Algorithm 1, it holds
	\begin{equation}
	\label{payment2}
	\sum_{i\in X\setminus Y}(p^{\rm f}_i-p_i)\geq\sum_{i\in Y}\phi_{i}(x_{i_{a}}^*-x^{\rm f}_{i})+\sum_{i\in Y_c}v_{i_{b}}x_{i_{b}}^*+c(f_{x,d}(X)-x_a^*(Y)-x_b^*(Y_c)+x(Y)).
	\end{equation} 
	Note that inequality (\ref{payment2}) provides a lower bound on future payments of buyers. 
	We show that both sides of (\ref{payment2}) are zero at the end of the auction, and that the left-hand side gradually becomes larger than the 
	right-hand side as we go back to the beginning. 
	We use Proposition \ref{algo<opt} to show that if $i$ drops by clinching, then it must be $i\notin Y_c$.
	This plays a critical role in showing that the above relation on both sides of (\ref{payment2}) holds in every case of Algorithm 1.
	
	\begin{proof}[Proof of Theorem \ref{payment}]
	At the end of Algorithm 1, inequality (\ref{payment2})
	holds because both sides are equal to $0$ by $X=Y=Y_c=\emptyset$.
	Using the following case-by-case analysis, 
	we prove that if it holds at the end of an iteration, it holds at the beginning of the iteration.
	In the following, we use 
	\begin{align*}
	f_{x,d}(X)+x(Y)=f_{x,d}(N)+x(Y)=f(N)-x(N)+x(Y)=f(N)-x(N\setminus Y),
	\end{align*}
	where the first equality follows from $f_{x,d}(X)\leq f_{x,d}(N)$ by the monotonicity of $f_{x,d}$
	and $f_{x,d}(N)\leq f_{x,d}(X)+d(N\setminus X)=f_{x,d}(X)$ due to the definition of $f_{x,d}$ and $X$, 
	and the second equality follows from Proposition \ref{all_goods}.
	
	\noindent(i) Execution of Algorithm 2:\\
	For buyer $i$ who belongs to $X\setminus Y$ just before clinching $\delta_i>0$ amount of goods, 
	the left-hand side of inequality (\ref{payment2}) is decreased by $c\delta_i$.
	The first and the second terms on the right-hand side are unchanged because $i\notin Y$, and
	the third term on the right side is decreased by $c\delta_i$ 
	by $f_{x,d}(X)+x(Y)=f(N)-x(N\setminus Y)$.
	Thus, both sides of inequality (\ref{payment2}) are decreased by $c\delta_i$.
	For buyer $i\in Y$ who is still active after clinching, 
	the left-hand side and the first and the second terms on the right-hand side are unchanged.
	Moreover, the third term on the right side is also unchanged by $f_{x,d}(X)+x(Y)=f(N)-x(N\setminus Y)$.
	Thus, both sides of inequality (\ref{payment2}) are unchanged.
	For buyer $i\in Y$ whose demand is positive before the execution of Algorithm 2 and becomes zero by clinching, 
	it holds from $d_i>0$ that $c\leq v_i$.
	If $c=v_i$, then $v_{i_b}< v_i=c=\phi_i$, and thus $i\notin Y_c$.
	Otherwise, from Proposition \ref{algo<opt}, we have $v_{i_b}\leq \phi_i=c$, 
	which means $i\notin Y_c$.
	Therefore, the left-hand side is unchanged by $i\in Y\setminus Y_c$.
	On the right-hand side, the first term is decreased by $\phi_i (x^*_{i_a}-x^{\rm f}_i)=c(x^*_{i_a}-x^{\rm f}_i)$, 
	and the second term is unchanged, 
	and the third term is increased by $c(x^*_{i_a}-x^{\rm f}_i)$ by $f_{x,d}(X)+x(Y)=f(N)-x(N\setminus Y)$.
	Then, both sides of inequality (\ref{payment2}) are unchanged. 
	Therefore, if (\ref{payment2}) holds after the execution of Algorithm 2, 
	it holds before that.
	
	\noindent(ii) The demand update:\\
	Suppose that the demand of buyer $i$ is updated in line 5 or 9.
	If the demand $d_i$ is still positive after the update, 
	both sides of inequality (\ref{payment2}) are obviously unchanged 
	because $f_{x,d}(X)+x(Y)=f(N)-x(N\setminus Y)$ is unchanged.
	In the following, we consider the case where the demand $d_i$ becomes zero.
	The left-hand side of (\ref{payment2}) is unchanged by $p_i=p^{\rm f}_i$ before the update.
	Suppose that $i\in X\setminus Y$. Then, the right-hand side is obviously unchanged.
	Suppose that $i\in Y_c$ before the update.
	Then, it holds $c=\phi_i<v_{i_b}<v_i$, which contradicts with Proposition \ref{algo<opt}.
	Suppose that $i\in Y\setminus Y_c$ before the update.
	The first term on the right-hand side is reduced by $\phi_i(x^*_{i_a}-x^{\rm f}_i)=c(x^*_{i_a}-x^{\rm f}_i)$, 
	the second term remains unchanged, 
	and the third term is increased by $c(x^*_{i_a}-x^{\rm f}_i)$.
	Thus, the right-hand side remains unchanged.
	Therefore, if (\ref{payment2}) holds after the price update, 
	it holds before that.
	
	\noindent(iii) The price update:\\
	It suffices to consider the change of $Y_{c}$ in the second and third terms on the right-hand side of (\ref{payment2}) 
	because $x$ and $d$ are unchanged by the price update.
	Let $\tilde{c}$ be the price before the update.
	By Proposition \ref{opt_algo_relation}, it holds that 
	$f_{x,d}(Y)-x_a^*(Y)+x(Y)\geq x_b^*(Y_{\tilde{c}})\geq x_b^*(Y_{c})$ by $Y_{\tilde{c}}\supseteq Y_{c}$. 
	Thus, 
	\begin{align*}
	\sum_{i\in Y_c}&v_{i_{b}}x_{i_{b}}^*+c(f_{x,d}(X)-x_a^*(Y)-x_b^*(Y_c)+x(Y))\\
	&\geq\sum_{i\in Y_c}v_{i_{b}}x_{i_{b}}^*+c(f_{x,d}(X)-x_a^*(Y)-x_b^*(Y_{\tilde{c}})+x(Y))+c x_b^*(Y_{\tilde{c}}\setminus Y_c) \\
	&\geq\sum_{i\in Y_{\tilde{c}}}v_{i_{b}}x_{i_{b}}^*+\tilde{c}(f_{x,d}(X)-x_a^*(Y)-x_b^*(Y_{\tilde{c}})+x(Y)),
	\end{align*}
	where the last inequality holds for $\tilde{c}<v_{i_b}\leq c$ for each $i\in Y_{\tilde{c}}\setminus Y_{c}$.
	Thus, the left-hand side of inequality (\ref{payment2}) is unchanged, and the 
	right-hand side is increased by the price update. Therefore, if inequality (\ref{payment2}) holds after the price update, 
	it holds before the price update.
	
	We conclude that inequality (\ref{payment2}) holds throughout Algorithm 2.
	Thus, at the beginning of the auction, we have 
	$\sum_{i\in N; x^{\rm f}_i\geq \tilde{x}^*_i}p^{\rm f}_i\geq \sum_{i\in N; x^{\rm f}_i<\tilde{x}^*_i}
	(\phi_{i}x_{i_{a}}^*+v_{i_{b}}x_{i_{b}}^*-\phi_{i}x^{\rm f}_{i}).$
	\end{proof}

	The following lemma shows that Theorem \ref{payment} is sufficient to obtain Theorem \ref{LW}.
	In the proof, we again use Proposition \ref{algo<opt}.
	
	\begin{lemma}
	\label{prepare_LW}
	For each $i\in N$ with $x^{\rm f}_i< \tilde{x}^*_i$, it holds 
	\[
	\phi_{i}x_{i_{a}}^*+v_{i_{b}}x_{i_{b}}^*-\phi_{i}x^{\rm f}_{i}= \min(v_i \tilde{x}^{*}_i, B_i)-\min(v_i x^{\rm f}_i, B_i).
	\]
	\end{lemma}
	\begin{proof}
	Suppose that $v_i>\phi_i$ for buyer $i$ with $x^{\rm f}_i< \tilde{x}^*_i$.
	By Proposition \ref{algo<opt}, it holds 
	$x^{\rm f}_i=x^*_{i_{a}}=\left\lfloor \frac{B_i}{v_i}\right\rfloor$, $x^*_{i_{b}}=1$, and $v_{i_{b}}\leq \phi_i$. 
	Thus, we have 
	$\min(v_i \tilde{x}^{*}_i, B_i)-\min(v_i x^{\rm f}_i, B_i)=v_{i_{b}}=
	\phi_{i}x_{i_{a}}^*+v_{i_{b}}x_{i_{b}}^*-\phi_{i}x^{\rm f}_{i}.$
	
	Suppose that $v_i=\phi_i=v_{i_a}$ for buyer $i$ with $x^{\rm f}_i< \tilde{x}^*_i$.
	If $v_i x^{\rm f}_i <v_i \tilde{x}^{*}_i< B_i$, then it holds $\tilde{x}^{*}_i=x^{*}_{i_a}$ and $x^{*}_{i_b}=0$ 
	by Proposition \ref{optimal} and Lemma \ref{prepare_OPT}. Then, we have 
	\[
	\min(v_i \tilde{x}^{*}_i, B_i)-\min(v_i x^{\rm f}_i, B_i)=v_i (x^{*}_{i_a}-x^{\rm f}_i)
	= \phi_{i}x_{i_{a}}^*+v_{i_{b}}x_{i_{b}}^*-\phi_{i}x^{\rm f}_{i}
	\]
	 by $v_i=\phi_i$.
	If $v_i x^{\rm f}_i\leq B_i \leq v_i \tilde{x}^{*}_i$, we have $B_i=v_{i_a}x_{i_{a}}^*+v_{i_{b}}x_{i_{b}}^*$.
	Then, it holds 
	\[
	\min(v_i \tilde{x}^{*}_i, B_i)-\min(v_i x^{\rm f}_i, B_i)=B_i-v_i x^{\rm f}_i
	=v_{i_a}x_{i_{a}}^*+v_{i_{b}}x_{i_{b}}^*-v_i x^{\rm f}_{i}
	=\phi_{i}x_{i_{a}}^*+v_{i_{b}}x_{i_{b}}^*-\phi_{i}x^{\rm f}_{i}
	\]
	by Proposition \ref{optimal} and $v_i=v_{i_a}=\phi_i$.
	Note that $B_i<v_i x^{\rm f}_i <v_i \tilde{x}^{*}_i$ implies 
	$\tilde{x}^{*}_i\geq x^{\rm f}_i+1>\frac{B_i}{v_i}+1\geq x^{*}_{i_a}+x^{*}_{i_b}$, 
	which never happens by Proposition \ref{optimal}.
	Therefore, it holds 
	$\phi_{i}x_{i_{a}}^*+v_{i_{b}}x_{i_{b}}^*-\phi_{i}x^{\rm f}_{i}
	=\min(v_i \tilde{x}^{*}_i, B_i)-\min(v_i x^{\rm f}_i, B_i)$ 
	for each buyer $i\in N$ with $x^{\rm f}_i< \tilde{x}^*_i$.
	\end{proof}
	
	
	Now we are ready for the proof of Theorem \ref{LW}.

\begin{proof}[Proof of Theorem \ref{LW}]
	For buyer $i$ with $x^{\rm f}_i< \tilde{x}^*_i$, by Lemma \ref{prepare_LW}, 
	it holds $\phi_{i}x_{i_{a}}^*+v_{i_{b}}x_{i_{b}}^*-\phi_{i}x^{\rm f}_{i}= \min(v_i \tilde{x}^{*}_i, B_i)-\min(v_i x^{\rm f}_i, B_i)$.
	Then, by $p^{\rm f}_i\geq 0\ (i\in N)$ and Theorem \ref{payment}, we have
	\begin{align*}
	p^{\rm f}(N) &\geq \sum_{i\in N; x^{\rm f}_i\geq \tilde{x}^*_i}p^{\rm f}_i\geq \sum_{i\in N; x^{\rm f}_i< \tilde{x}^*_i}\bigl(\phi_{i}x_{i_{a}}^*+v_{i_{b}}x_{i_{b}}^*-\phi_{i}x^{\rm f}_{i}\bigr)= \sum_{i\in N; x^{\rm f}_i< \tilde{x}^*_i}\bigl(\min(v_i \tilde{x}^{*}_i, B_i)-\min(v_i x^{\rm f}_i, B_i)\bigr)\\
	&\geq  \sum_{i\in N}\bigl(\min(v_i \tilde{x}^{*}_i, B_i)-\min(v_i x^{\rm f}_i, B_i)\bigr)={\rm LW}^{\rm OPT}-{\rm LW}^{\rm M}, 
	\end{align*}
	By inequality (\ref{lowerbound}), we have 
	${\rm LW}^{\rm M}\geq p^{\rm f}(N)\geq {\rm LW}^{\rm OPT}-{\rm LW}^{\rm M}$, 
	which means ${\rm LW}^{\rm M}\geq \frac{1}{2}{\rm LW}^{\rm OPT}$.
	
%	By ${\rm LW}^{\rm M}\geq \sum_{i\in N; x^{\rm f}_i\geq \tilde{x}^*_i}\min(v_i x^{\rm f}_i, B_i)\geq  \sum_{i\in N;x^{\rm f}_i\geq \tilde{x}^*_i}\min(v_i \tilde{x}^{*}_i, B_i)$, we have 
%	\begin{equation*}
%	2{\rm LW}^{\rm M}\geq \sum_{i\in N; x^{\rm f}_i< \tilde{x}^*_i}\min(v_i \tilde{x}^*_i, B_i)+\sum_{i\in N; x^{\rm f}_i\geq \tilde{x}^*_i} \min(v_i \tilde{x}^*_i, B_i)
%	=\sum_{i\in N} \min(v_i \tilde{x}^*_i, B_i)={\rm LW}^{\rm OPT}.
%	\end{equation*}
\end{proof}



\subsection{Social Welfare}
	Let ${\rm SW}^{\rm M}$ denote the SW of our mechanism.
	Now we obtain a lower bound on ${\rm SW}^{\rm M}$ as~follows:
	
	\begin{theorem}
	\label{SW}
	It holds ${\rm SW}^{\rm M}=\sum_{i\in N}v_i x^{\rm f}_i \geq {\rm LW}^{\rm OPT}$.
	\end{theorem}
	
	This inequality is tight because the optimal LW is equal to the optimal SW 
	if the budgets of all buyers are sufficiently large. 
	In this case, Theorem \ref{SW} implies that our mechanism outputs an allocation that maximizes the SW.
	
	
	The proof is an inductive argument with respect to $\{X_k\}_{k\in \{0,1,\ldots,t\}}$ 
	in the tight set lemma (Theorem 4) as in Sato \cite{S2023}.
	On the other hand, we need to use $\{\phi_{i_k}\}_{k\in \{1,2\ldots,t\}}$ instead of 
	$\{v_{i_k}\}_{k\in \{1,2\ldots,t\}}$ because 
	$\{v_{i_k}\}_{k\in \{1,2\ldots,t\}}$ is not necessarily monotone in our setting.
	%where $i_1,i_2,\ldots,i_t$ are the buyers dropping by demand update in Theorem \ref{tightsets}.
	Then, by equation (\ref{LW_OPT}), we have 
	\[
	{\rm SW}^{\rm M}- {\rm LW}^{\rm OPT}
	=\sum_{i\in N}\bigl(v_i x^{\rm f}_i-(v_{i_{a}} x^*_{i_{a}}+v_{i_{b}} x^*_{i_{b}})\bigr).
	\]
	Moreover, to link the right-hand side to dropping prices, we use the following lemma: 
	\begin{lemma}
	\label{prepare_SW}
	For each $i\in N$, it holds 
	$v_i x^{\rm f}_i-(v_{i_{a}} x^*_{i_{a}}+v_{i_{b}} x^*_{i_{b}})\geq \phi_i(x^{\rm f}_i-\tilde{x}^*_{i})$.
	\end{lemma}
	\begin{proof}
	If $x^{\rm f}_i\geq \tilde{x}^*_i$, then it holds 
	$v_i x^{\rm f}_i-(v_{i_{a}} x^*_{i_{a}}+v_{i_{b}} x^*_{i_{b}})\geq v_i (x^{\rm f}_i-\tilde{x}^*_{i})\geq \phi_i(x^{\rm f}_i-\tilde{x}^*_{i})$ by $v_i=v_{i_{a}}\geq v_{i_{b}}$.
	If $x^{\rm f}_i< \tilde{x}^*_i$ and $v_i=\phi_i$, it holds 
	$v_i x^{\rm f}_i-(v_{i_{a}} x^*_{i_{a}}+v_{i_{b}} x^*_{i_{b}})\geq v_i(x^{\rm f}_i-\tilde{x}^*_i)= \phi_i(x^{\rm f}_i-\tilde{x}^*_i)$.
	If $x^{\rm f}_i< \tilde{x}^*_i$ and $v_i>\phi_i$, by Proposition \ref{algo<opt}, 
	it holds $x^{\rm f}_i=x^*_{i_{a}}=\left\lfloor \frac{B_i}{v_i}\right\rfloor$,
	$x^*_{i_{b}}=1$, and $v_{i_{b}}\leq \phi_i$.
	Thus,  
	it holds $v_i x^{\rm f}_i-(v_{i_{a}} x^*_{i_{a}}+v_{i_{b}} x^*_{i_{b}})=-v_{i_{b}}x^*_{i_{b}}\geq -\phi_{i}x^*_{i_{b}}
	=\phi_i(x^{\rm f}_i-\tilde{x}^*_i)$.
	Therefore, it holds $v_i x^{\rm f}_i-(v_{i_{a}} x^*_{i_{a}}+v_{i_{b}} x^*_{i_{b}})\geq \phi_i(x^{\rm f}_i-\tilde{x}^*_i)$ for each $i\in N$.
\end{proof}	



	\begin{proof}[Proof of Theorem \ref{SW}]
	Using Theorem \ref{tightsets}, we prove 
	\[
	\sum_{i\in X_{k}} \bigl(v_i x^{\rm f}_i-
	(v_{i_{a}} x^*_{i_{a}}+v_{i_{b}} x^*_{i_{b}})\bigr)
	\geq \phi_{i_k} (x^{\rm f}(X_k)-\tilde{x}^*(X_k))
	\]
	for each $k\in \{0,1,\ldots,t\}$ by mathematical induction (where we set $\phi_{i_0}:=\phi_{i_1}$). 
	For $k=0$, both sides are equal to $0$ by $X_0=\emptyset$.
	Suppose that the inequality holds for $k-1$.
	By Property (iii) of Theorem \ref{tightsets}, it holds $x^{\rm f}(X_{k-1})=f(X_{k-1})\geq \tilde{x}^*(X_{k-1})$.
	By Lemma \ref{prepare_SW} and $\phi_i=\phi_{i_k}$ 
	due to Theorem \ref{tightsets}, we have 
	$v_i x^{\rm f}_i-(v_{i_{a}} x^*_{i_{a}}+v_{i_{b}} x^*_{i_{b}})\geq \phi_{i_k}(x^{\rm f}_i-\tilde{x}^*_{i})$ 
	for $i\in X_{k}\setminus X_{k-1}$.
	Therefore, we have
	\begin{align*}
	\sum_{i\in X_{k}} (v_i x^{\rm f}_i- (v_{i_{a}} x^*_{i_{a}}+v_{i_{b}} x^*_{i_{b}}))&\geq \phi_{i_{k-1}} (x^{\rm f}(X_{k-1})-\tilde{x}^*(X_{k-1}))+
	 \phi_{i_k}\sum_{i\in X_{k}\setminus X_{k-1}}(x^{\rm f}_i- \tilde{x}^*_{i})\\ 
	&\geq \phi_{i_{k}} (x^{\rm f}(X_{k})-\tilde{x}^*(X_{k})), 
	\end{align*}
	where the second inequality holds by $\phi_{i_{k-1}}\geq \phi_{i_{k}}$ and $x^{\rm f}(X_{k-1})\geq \tilde{x}^*(X_{k-1})$.
	Substituting $k$ with $t$, it holds $\sum_{i\in X_t}x^{\rm f}_i=f(N)\geq \sum_{i\in X_t} \tilde{x}^*_{i}$
	by Proposition \ref{all_goods} and $X_t=N$. Thus, we have 
	\[
	{\rm SW}^{\rm M}=\sum_{i\in N}v_i x^{\rm f}_i \geq \sum_{i\in N} (v_{i_{a}} x^*_{i_{a}}+v_{i_{b}} x^*_{i_{b}})={\rm LW}^{\rm OPT},
	\] 
	where the last equality holds by equation (\ref{LW_OPT}).
	
	\end{proof}


\subsection{Discussions}
\subsubsection{Divisible Case and Indivisible Case}

In Section 5.2, we provide the LW guarantee for our indivisible setting, 
where several differences due to indivisibility prevent us from making a straightforward extension.
Now we provide some examples to show these differences.
Firstly, the following example shows that the formula of an LW optimal allocation for the divisible setting of Sato \cite{S2023} 
does not necessarily yield an LW optimal allocation in our setting.

\begin{example}
\label{difference_optimal}
Consider a market where the seller owns 3 units of a single indivisible good.
\footnote{Throughout Section 4.4, the markets are represented by a complete bipartite graph with no constraints on the allocation of goods.}
Buyer 1 has valuation $10$ and budget $11$, and buyer 2 has valuation $3.1$ and budget $6$. If the goods are divisible, a simple greedy procedure that allocates as many goods as possible to buyers in descending order of their valuation until LW remains unchanged leads to the optimal solution. 
In this case, both buyers get $1.1$ units and buyer 2 gets the rest. 
However, in the indivisible case, the LW increase of buyer 1 is $10$ for the first unit, $1$ for the second unit, and $0$ for the third unit, 
and the LW increase of buyer 2 is $3.1$ for the first unit, $2.9$ for the second unit, and $0$ for the third unit.
Therefore, the optimal allocation is to allocate one unit to buyer 1 and two units to buyer 2.
\end{example}

Secondly, the following example shows that 
the number of remaining goods might be fewer than 
the lower bound in Sato \cite{S2023} for their divisible setting: 
\begin{example}
	Consider the setting in Example \ref{difference_optimal}.
	By Proposition \ref{optimal}, the LW optimal allocation for virtual buyers 
	is $x^{*}_{1_a}=x^{*}_{2_a}=x^{*}_{2_b}=1$ and $x^{*}_{1_b}=0$.
	In Algorithm 1, at $c=2$, the demand of buyer 2 decreases to $2$, and 
	buyer 1 clinches the first unit. At $c=3$, the demand of buyer 2 decreases to $1$, and 
	buyer 1 clinches the second unit.
	Also, at $c=3.1$, the demand of buyer 2 decreases to $0$, and 
	buyer 1 clinches the third unit. 
	Thus, $x^{\rm f}_{1}=3$, $x^{\rm f}_{2}=0$.
	
	Just after the execution of Algorithm 2 at $c=3$, 
	the remaining goods is one unit.
	By $Y=\{2\}$, since it holds $f_{x,d}(Y)=x_a^*(Y)=x_b^*(Y)=1$ and $x(Y)=0$, 
	we have  $f_{x,d}(Y)-\tilde{x}^*(Y)+x(Y)= 1-2+0=-1<0$. 
	Thus, the number of remaining goods is fewer than the lower bound in Sato \cite{S2023}.
	Since $c=3>2=v_{2_b}$, it holds $Y_c=\emptyset$. 
	Therefore, $f_{x,d}(Y)-x_a^*(Y)+x(Y)-x_b^*(Y_c)= 1-1+0-0=0$.
	This shows that Proposition \ref{opt_algo_relation} holds 
	even for the case of $f_{x,d}(Y)-\tilde{x}^*(Y)+x(Y)<0$.
\end{example}

These examples illustrate the need for our approach of splitting each buyer into two virtual buyers, 
and using the allocation for virtual buyers.


\subsubsection{Tightness of Our LW Guarantee and a Lower Bound}
	In Theorem \ref{LW}, we show that our mechanism achieves 2-approximation to the optimal LW value.
	The following example shows that this LW guarantee is tight.
	\begin{example}
	\label{tight_example}
	Consider a market where 
	the seller owns $k$\ $(k\gg 1)$ units of the good and 
	two buyers participate in the auction.
	Buyer 1 has valuation $1$, and buyer 2 has valuation $k$ 
	and the budgets of both buyers are $k$.
	Then, LW is maximized when buyer 1 gets $k-1$ units and buyer 2 gets one unit.
	Thus, the optimal LW value is $2k-1$.
	In Algorithm 1, the price goes up to 1 in line 3 of the first iteration, 
	and the demand of buyer 1 decreases to zero.
	Then, all the goods are allocated to buyer 2, and 
	thus the LW of our mechanism is $\min(k^2, k)=k$.
	Therefore, 
	${\rm LW}^{\rm M}=k\leq \frac{k}{2k-1}{\rm LW}^{\rm OPT}$.
	Taking $k\to\infty$, we have $\frac{k}{2k-1}\to \frac{1}{2}$, 
	and thus the LW guarantee in Theorem \ref{LW} is tight.
	\end{example}

	Subsequently, we provide a lower bound on the approximation ratio of LW in our indivisible setting. 
	The following is a natural extension 
	of Theorem 5.1 of Dobzinski and Leme \cite{DL2014} to indivisible settings. 
	As in their proof, we use the famous Myerson's Lemma \cite{M1981} to 
	show the monotonicity of the allocations to valuations in mechanisms satisfying IC. 
	\begin{proposition}
	\label{lower_bound}
	There is no mechanism satisfying IC and IR that 
	approximates the optimal LW by a factor better than $\frac{4}{3}$.
	\end{proposition}
	\begin{proof}
	Consider the setting in Example \ref{tight_example}, where the valuations of buyers can be changed.
	Suppose that there exists a mechanism satisfying IC and IR that achieves 
	$\gamma$-approximation to the optimal LW value. 
	Then, let $\bar{x}(v_1,v_2)$ denote the allocation by the mechanism in the above market when 
	the valuations of buyer 1 and buyer 2 are $v_1$ and $v_2$, respectively.
	
	By Myerson's lemma \cite{M1981}, $\bar{x}_i(v_1,v_2)$ is non-decreasing in $v_i$ for each buyer $i$.
	Then, we have
	 $\bar{x}_1(1, k)\leq \bar{x}_1(k, k)$ and  $\bar{x}_2(k, 1)\leq \bar{x}_2(k, k)$.
	 Thus, since $\bar{x}_1(k, k)+\bar{x}_2(k, k)\leq k$, we can assume that 
	 $\bar{x}_1(1, k)\leq \bar{x}_1(k, k)\leq \lfloor \frac{k}{2}\rfloor$ without loss of generality.
	In the case of $v_1=1$ and $v_2=k$, the optimal LW value is $2k-1$ as in Example \ref{tight_example}.
	 Then, the LW of the allocation $\bar{x}(1, k)$ is at most 
	 \[
	\min( \bar{x}_1(1, k), k)+\min( k \bar{x}_2(1, k), k)\leq \Bigl\lfloor \frac{k}{2}\Bigr\rfloor + k \leq 
	\frac{3k}{2}=\frac{3}{4-\frac{2}{k}}(2k-1),
	 \]
	 which implies $\gamma\leq \frac{4}{3}$ by taking $k$ to $\infty$.
	\end{proof}


\subsubsection{SW Guarantees within Envy-free Allocations}
	Another SW guarantee of clinching auctions for divisible goods was provided by Devanur et al.~\cite{DHH2013}. They showed that in their symmetric setting, clinching auctions yield an envy-free allocation and achieve a two-approximation of SW to the maximum SW among all envy-free allocations. 
	\begin{definition}[e.g. Devanur et al. \cite{DHH2013}]
	An allocation $\mathcal A$ is envy-free if it holds 
	$u_i(\mathcal A)\geq u_i(\mathcal A_{ij})$ for each $i,j\in N$, 
	where $\mathcal A_{ij}$ represents the allocation obtained from $\mathcal A$ by swapping the allocation of buyers $i$ and $j$.
	\end{definition}
	However, SW guarantees of our mechanism among all (possibly approximate) envy-free allocations 
	seems unrealistic due to indivisibility by the following example:
	
\begin{example}
	\label{envy_free}
	The seller owns only $1$ unit of a single indivisible good. 
	Buyer 1 has valuation $k\ (k\gg 2)$ and Buyer 2 has valuation $2$, 
	and they have common budgets of $1$. 
	In Algorithm 1, at $c=1$, the demand $d_1$ decreases to $0$, and 
	buyer 2 clinches one unit of the good at price $1$. 
	Thus, we have $(x^{\rm f}, p^{\rm f})=((0,1), (0,1))$.
	Then, buyer 1 has envy about the allocation of buyer 2.
	
	If we change (the valuations of) buyer 1 and buyer 2, the allocation is unchanged.
	This implies that even if we consider using some approximate notion of envy-freeness, 
	we must use the one that admits $(x^{\rm f}, p^{\rm f})=((0,1), (0,1))$ and $(x^{\rm f}, p^{\rm f})=((1,0), (1,0))$.
	Then, the SW for $(x^{\rm f}, p^{\rm f})=((0,1), (0,1))$ is 2, and that for $(x^{\rm f}, p^{\rm f})=((1,0), (1,0))$ is $k$.
	Therefore, the approximation ratio of SW is at most $2/k$, which reaches zero as we take $k\to\infty$.
\end{example}


\subsubsection{Concluding Remarks}
	In this study, we propose the polyhedral clinching auction for indivisible goods, 
	which satisfies IC and IR, and works with polymatroidal environments. 
	Moreover, we provided the tight sets lemma and three types of efficiency guarantees. 
	Since many of these were not revealed even in the special case of our setting, 
	we believe that our results significantly advance the efficiency guarantee 
	of clinching auctions for indivisible goods and that they might be helpful on extending our mechanism to two-sided markets. 
	
	We note some possible future directions of our study.
	First, there still remains a gap between the approximation ratio and the lower bound for LW guarantees, 
	which can be seen by Example \ref{tight_example} and Proposition \ref{lower_bound}.
	In the divisible setting, Lu and Xiao \cite{LX2015} proposed a mechanism 
	that satisfies IC and IR and achieves a better approximation ratio than clinching auctions. 
	Seeking for such mechanisms in the indivisible settings, 
	especially for polymatroidal environments, could be an interesting future work.
	Second, we extend the SW guarantee of Sato \cite{S2023} to our indivisible setting,
	whereas SW guarantees among envy-free allocations in Devanur et al. \cite{DHH2013} seem unrealistic in our mechanism.
	It may be of particular interest to find another type of SW guarantee suitable for indivisible and budgeted settings.

	\section*{Acknowledgements}
	We thank anonymous reviewers for helpful feedback and suggestions.
	This work was supported by Grant-in-Aid for JSPS Research Fellow Grant Number JP22KJ1137, Japan 
	and Grant-in-Aid for Challenging Research (Exploratory) Grant Number JP21K19759, Japan.

	
	\bibliography{indivisible}
	\appendix

\section{Omitted Proofs}
\subsection{Proof of Theorem \ref{invariant}}
	Now we prove Theorem \ref{invariant}. We begin by the following lemma:
	\begin{lemma}
	\label{prepare_inv}
	In the execution of Algorithm 2, it holds
	$\displaystyle \delta(T\setminus S) \leq f_{x,d}(T)-f_{x,d}(S)$ 
	for each $S,T\subseteq N$ such that $T\supseteq S$, 
	where $x$ and $d$ are the allocation of goods and the demand vector, respectively, just before the execution of Algorithm 2.
	\end{lemma}
	\begin{proof}
	Suppose that $T\setminus S=\{\ell_1,\ell_2,\ldots, \ell_q\}$ for some positive integer $q$.
	We consider the transaction $\delta_i$ of buyer $i\in T\setminus S$.
	By Proposition \ref{clinch_amount} and the submodularity of $f_{x,d}$, 
	we have $\delta_i = f_{x,d}(N)-f_{x,d}(N\setminus i)\leq  f_{x,d}(U)-f_{x,d}(U\setminus i)$ 
	for each $U\ni i$, and thus 
	\begin{align*}
	\delta(T\setminus S) &=\sum_{i\in T\setminus S}(f_{x,d}(N)-f_{x,d}(N\setminus i)) \\
	&\leq f_{x,d}(T)-f_{x,d}(T\setminus \ell_1)+f_{x,d}(T\setminus \ell_1)-
	\cdots +f_{x,d}(S\cup \ell_q)-f_{x,d}(S)=f_{x,d}(T)-f_{x,d}(S).
	\end{align*}
	\end{proof}
	
	\begin{proof}[Proof of Theorem \ref{invariant}]
	At the beginning of the auction, the equation in Theorem \ref{invariant} holds by Lemma \ref{beginning}.
	By the following case-by-case analysis, we prove that 
	when the equation holds for each $S\subseteq N$ 
	at the beginning of an iteration,
	then it also holds at the end of the iteration.
	
	\noindent (i) The price update: \\
	Since $x$ and $d$ are unchanged by the price update, 
	the equation trivially holds even after the update.
	
	\noindent (ii) The demand update:  \\
	Suppose that the demand of buyer $i$ changes from $d_i$ to $d'_i$.
	Let $d':=(d'_i, d_{-i})$ be the demand vector after the update.
	Then, in equation (\ref{naive}) for $f_{x,d'}(S)$, if $i\in S''$, then it holds $f_{x,d'}(S)=f_{x,d}(S)$.
	If $i\notin S''$, then it holds $f_{x,d'}(S)=f_{x,d}(S\setminus i)+d'_i$.
	Therefore, after the demand update, it holds 
	\[
	f_{x,d'}(S)=\min\{f_{x,d}(S),f_{x,d}(S\setminus i)+d'_i\}=
	\min_{S'\subseteq S}\{f(S')-x(S')+d'(S\setminus S')\} 
	\]
	by the hypothesis and 
	$f_{x,d}(S\setminus i)+d'_i<f_{x,d}(S\setminus i)+d_i$.
	
	
	\noindent (iii) Clinching Step: \\
	Let $x$ and $d$ (resp. $\tilde{x}$ and $\tilde{d}$) 
	be the allocation of goods and the demand vector, respectively, 
	just after (resp. before) the execution of Algorithm 2. 
	By the hypothesis, 
	it holds $f_{\tilde{x},\tilde{d}}(S)=\min_{S'\subseteq S}\{f(S')-\tilde{x}(S')+\tilde{d}(S\setminus S')\}$.
	When buyer $i\in S$ clinches $\delta_i$ goods in Algorithm 2,
	this increases $\tilde{x}_i$ by $\delta_i$ and decreases $\tilde{d}_i$ by $\delta_i$.
	Then, $\min_{S'\subseteq S}\{f(S')-\tilde{x}(S')+\tilde{d}(S\setminus S')\}$ 
	was decreased by $\delta(S)$ whether the buyers 
	belong to $S'$ or $S\setminus S'$. 
	%where $x$ and $d$ are the allocation of goods and the demands, 
	%respectively, just after the execution of Algorithm 2.
	This means that
	\[
	\min_{S'\subseteq S}\{f(S')-x(S')+d(S\setminus S')\}=\min_{S'\subseteq S}
	\{f(S')-\tilde{x}(S')+\tilde{d}(S\setminus S')\}-\delta(S)=f_{\tilde{x},\tilde{d}}(S)-\delta(S).
	\] 
	
	Suppose that the equation in Theorem \ref{invariant} 
	does not hold just after the execution of Algorithm~2,
	then there exists $S^*\subseteq S$ and $S^{\dag} \supset S^*$ such that
	\begin{eqnarray}
	\label{X_aft}
	f_{x,d}(S)=f(S^\dag)-x(S^\dag)+d(S\setminus S^*)
	<\min_{S'\subseteq S}\{f(S')-x(S')+d(S\setminus S')\}=f_{\tilde{x},\tilde{d}}(S)-\delta(S).
	\end{eqnarray}
	Without loss of generality, we can take $S^*\subseteq S$ and 
	$S^{\dag} \supset S^*$ such that $(S^\dag\setminus S^*)\cap S=\emptyset$.
	If that is not the case, define $\tilde{S}:=(S^\dag\setminus S^*)\cap S$ and we have
	$f(S^\dag)-x(S^\dag)+d(S\setminus (S^*\cup \tilde{S}))\leq f(S^\dag)-x(S^\dag)+d(S\setminus S^*)$.
	Thus, we can replace $S^{*}$ by $S^*\cup \tilde{S}$. 
	From this, we assume $(S^\dag\setminus S^*)\cap S=\emptyset$.
	
	Since it holds $S^{\dag}\setminus S^*=S^{\dag}\setminus S$ 
	and $f_{x,d}(S)=f(S^\dag)-x(S^\dag)+d(S\setminus S^*)$, we have 
	\begin{align}
	\label{X_bef}
	f_{x,d}(S)&=f(S^\dag)-x(S^\dag)+d(S\setminus S^*) 
	\geq \min_{S'\subseteq S^\dag\cup S}\{f(S')-x(S')+d((S^\dag\cup S)\setminus S')\} \nonumber\\ 
	&= \min_{S'\subseteq S^\dag\cup S}\{f(S')-\tilde{x}(S')+\tilde{d}((S^\dag\cup S)\setminus S')\}-\delta(S^\dag\cup S) 
	=f_{\tilde{x},\tilde{d}}(S^\dag\cup S)-\delta(S^\dag\cup S),
	\end{align}
	where the first inequality holds by $S^{\dag} \supset S^*$, 
	the second equality follows from that when buyer $i\in S^\dag\cup S$ clinches $\delta_i$ goods 
	in line 3 of Algorithm 2, 
	this increases $\tilde{x}_i$ by $\delta_i$ and decreases $\tilde{d}_i$ by $\delta_i$.
	Also, the third equality is from the hypothesis.
	Combining the inequalities (\ref{X_aft}) and (\ref{X_bef}), we have
	\begin{eqnarray*}
	f_{\tilde{x},\tilde{d}}(S)-\delta(S)>f_{x,d}(S)\geq f_{\tilde{x},\tilde{d}}(S^\dag\cup S)-\delta(S^\dag\cup S).
	\end{eqnarray*}
	This means $\delta(S^\dag\setminus S)>f_{\tilde{x},\tilde{d}}(S^\dag\cup S)-f_{\tilde{x},\tilde{d}}(S)$, 
	which contradicts with Lemma \ref{prepare_inv}.
	Therefore, the equation in Theorem \ref{invariant} also holds just after the execution of Algorithm 2.
	
	\end{proof}


\subsection{Proof of Proposition \ref{clinch_amount}}
	Now we prove Proposition \ref{clinch_amount}. 
	Since Theorem \ref{invariant} is proved using Proposition \ref{clinch_amount}, 
	we prove this without using Theorem \ref{invariant}.
	

\begin{lemma}[See, e.g., Section 44.4 of Schrijver \cite{S2003}]
\label{polymatroid_equal}
Two polymatroids are equal if and only if the corresponding monotone submodular functions corresponding to the polymatroids are equal.
\end{lemma}
\begin{proof}[Proof of Proposition \ref{clinch_amount}]
The monotone submodular function $f^{i, w_i}_{x,d}: 2^{N\setminus i}\to \mathbb Z_{+}$ that defines $P^i_{x,d}(w_i)$ is given by 
$f^{i, w_i}_{x,d}(S):=f_{\hat{x},d}(S)$ for each $S\subseteq N\setminus i$, 
where $\hat{x}_i=x_i+w_i$ and $\hat{x}_j=x_j$ 
for each $j\in N\setminus i$. Note that we only consider $w_i$ with $w_i\leq d_i$.
If $i\in S''$ in equation (\ref{naive}) for $f_{\hat{x},d}(S)$, 
then this implies $f_{x,d}(S\cup i)=f_{\hat{x},d}(S)+w_i$ by equation (\ref{naive}) for $f_{x,d}(S\cup i)$.
If $i\notin S''$, then $f_{x,d}(S)=f_{\hat{x},d}(S)$ by equation (\ref{naive}) for $f_{x,d}(S)$.
Thus, by $f_{x,d}(S)+d_i-w_i\geq f_{x,d}(S)$, we have 
\[
f^{i, w_i}_{x,d}(S)=\min\{f_{x,d}(S\cup i)-w_i, f_{x,d}(S)\}
\] 
for each $S\subseteq N\setminus i$ as in Goel et al. \cite{GMP2015}.
Thus, by Lemma \ref{polymatroid_equal}, 
the condition $P^i_{x,d}(w_i)=P^i_{x,d}(0)$ is equivalent to  
$f_{x,d}(S)\leq f_{x,d}(S\cup i)-w_i$ for any $S\subseteq N\setminus i$.
Therefore, by the definition of $\delta_i$ and the submodularity of $f_{x,d}$, we have
$\delta_i=\min_{S\subseteq N\setminus i} \{f_{x,d}(S\cup i)-f_{x,d}(S)\}=f_{x,d}(N)-f_{x,d}(N\setminus i)$. 
Since $f_{x,d}(N)\leq f_{x,d}(N\setminus i)+d_i$ by equation (\ref{naive}), 
we have $\delta_i=f_{x,d}(N)-f_{x,d}(N\setminus i)\leq d_i$.

Subsequently, we show that the amount of clinching is independent from the order of buyers.
In equation (\ref{naive}), for each $S\subseteq N$, 
we can take $S'\subseteq S$ and $S''\supseteq S'$ 
such that $S''\cap (S\setminus S')=\emptyset$ and 
$f_{x,d}(S)=f(S'')-x(S'')+d(S\setminus S')$ 
by $d_i\geq 0$ for each $i\in N$.
Then, we have
\begin{align*}
f_{x,d}(N)&=\min_{S'\subseteq N}\{\min_{S''\supseteq S'}\{f(S'')-x(S'')\}+d(N\setminus S')\}=\min_{S'\subseteq N}\{f(S')-x(S')+d(N\setminus S')\}, \\
f_{x,d}(N\setminus i)&=\min_{S'\subseteq N\setminus i}
\{\min\Bigl(f(S'\cup i)-x(S'\cup i)\Bigr), f(S')-x(S')\}+d(N\setminus S') \\
&= \min\Bigl(\min_{S'\subseteq N\setminus i}
\{f(S'\cup i)-x(S'\cup i)+d(N\setminus S')\}, 
\min_{S'\subseteq N\setminus i} \{f(S')-x(S')+d(N\setminus S')\}\Bigr)\\
&\geq \min\Bigl(f_{x,d}(N), \min_{S'\subseteq N\setminus i} 
\{f(S')-x(S')+d(N\setminus S')\}\Bigr).
\end{align*}
In fact, the last inequality holds by equality because we have $f_{x,d}(N)\geq f_{x,d}(N\setminus i)$ due to  
the monotonicity of $f_{x,d}$. 
Using the above, we have the following:
\begin{align*}
&f_{x,d}(N)-f_{x,d}(N\setminus i)\\
&=\max(0, \min_{S'\subseteq N}\{f(S')-x(S')+d(N\setminus S')\}-\min_{S''\subseteq N\setminus i}\{f(S'')-x(S'')+d((N\setminus i)\setminus S'')\}).
\end{align*}
Consider two buyers $i$ and $k$ where $i$ is ordered just before $k$.
When buyer $i$ clinches $\delta_i$ amount of goods before $k$'s clinching, 
then both $f(S')-x(S')+d(N\setminus S')$ and 
$f(S'')-x(S'')+d((N\setminus k)\setminus S'')$ decrease by $\delta_i$ for each 
$S'\subseteq N$ and $S''\subseteq N\setminus k$ 
since $x_i$ is updated to $x_i=x_i+\delta_i$ and $d_i$ is updated to $d_i=d_i-\delta_i$.
This implies that $f_{x,d}(N)-f_{x,d}(N\setminus k)$ is unchanged by the clinching of buyers numbered before $k$.
\end{proof}




%From Lemma \ref{characterization_unsaturation}, we can see that just after buyer $k$ drops 
%in line 5 or 9 of Algorithm 1, it holds $\delta_i=d_i$ for buyer $i$ with $i\lesssim k$ in the subsequent execution of Algorithm 2. 
%This implies that buyer $i$ also drops in the same iteration.
\subsection{Proof of Lemma \ref{transitive}}
\begin{proof}[Proof of Lemma \ref{transitive}]
 It suffices to consider the case that $i, j, k$ are different from each other.
	Suppose that it holds $i\lesssim j$ and $j\lesssim k$.
	%It remains to prove the opposite side.
	By the submodularity of $f_{x,d^{-j}}$,
	it holds 
	\[f_{x,d^{-j}}(N\setminus j)-f_{x,d^{-j}}(N\setminus \{i,j\})
	\geq f_{x,d^{-j}}(N)-f_{x,d^{-j}}(N\setminus i)=d_i,
	\] 
	where the equality holds by $i\lesssim j$ (and Lemma \ref{characterization_unsaturation}).
	Similarly, by $j\lesssim k$, it holds 
	\[
	f_{x,d^{-k}}(N)-f_{x,d^{-k}}(N\setminus j)-d_j=0.
	\]
	By the submodularity of $f_{x,d^{-k}}$,
	we have 
	\[
	f_{x,d^{-k}}(N\setminus j)-f_{x,d^{-k}}(N\setminus \{j,k\})\geq f_{x,d^{-k}}(N)-f_{x,d^{-k}}(N\setminus k)=0,
	\]
	where the equality holds by Lemmas \ref{reflexive}.
	Summing the above, we have 
	\begin{equation*}
	f_{x,d^{-j}}(N\setminus j)-f_{x,d^{-j}}(N\setminus \{i,j\})+f_{x,d^{-k}}(N)-f_{x,d^{-k}}(N\setminus \{j,k\})-d_j
	\geq d_i
	\end{equation*}
	In the following, we prove $f_{x,d^{-k}}(N)- f_{x,d^{-k}}(N\setminus i)$ is more than the left-hand side.
	We have 
	\begin{align*}
	&f_{x,d^{-j}}(N\setminus j)-f_{x,d^{-j}}(N\setminus \{i,j\})+f_{x,d^{-k}}(N)-f_{x,d^{-k}}(N\setminus \{j,k\})-d_j\\
	&=f_{x,d}(N\setminus j)-f_{x,d}(N\setminus \{i,j\})+f_{x,d^{-k}}(N)-f_{x,d}(N\setminus \{j,k\})-d_j \\
	&\leq f_{x,d}(N\setminus \{i,j\})-f_{x,d}(N\setminus \{i,j\})+f_{x,d^{-k}}(N)-f_{x,d}(N\setminus \{i,j,k\})-d_j \\
	&=f_{x,d^{-k}}(N)-f_{x,d}(N\setminus \{i,j,k\})-d_j,
	\end{align*}
	where the first equality holds by Observation \ref{fact},
	and the second inequality holds by the submodularity of $f_{x,d}$.
	Then, by equation (\ref{naive}), it holds
	\[
	f_{x,d}(N\setminus \{i,j,k\})+d_j \geq f_{x,d}(N\setminus \{i,k\})
	=f_{x,d^{-k}}(N\setminus \{i,k\})= f_{x,d^{-k}}(N\setminus i),
	\]
	where the first equality holds by Observation \ref{fact} and the second equality 
	holds by Lemmas \ref{reflexive} and \ref{characterization_unsaturation}.
	Therefore, we have 
	$d_i\leq f_{x,d^{-k}}(N)-f_{x,d^{-k}}(N\setminus \{i,j,k\})-d_j
	\leq f_{x,d^{-k}}(N)- f_{x,d^{-k}}(N\setminus i)$.
	Using this and $d_i\geq f_{x,d^{-k}}(N)- f_{x,d^{-k}}(N\setminus i)$ by equation (\ref{naive}), 
	the above inequality holds by equality. 
	Therefore, we have $i\lesssim k$. 
\end{proof}
\subsection{Proof of Proposition \ref{optimal}}
	For the proof of Proposition \ref{optimal}, we first consider the following problem:
	\begin{align}
	\label{opt2}
	&{\text{maximize}}\ \ \ \ \sum_{i'\in N'}v_{i'} \min(x_{i'}, \frac{B_{i'}}{v_{i'}})\\
	&\text{subject to}\ \ \ \, x\in P(f')\cap \mathbb Z^{N'}\nonumber.
	\end{align}
	Indeed, we can find a optimal solution for problem (\ref{opt2}) by solving 
	\begin{align}
	\label{opt1}
	&{\text{maximize}}\ \ \ \ \sum_{i'\in N'}v_{i'} x_{i'}\\
	&\text{subject to}\ \ \ \, x\in P_{d'}(f')\cap \mathbb Z^{N'}\nonumber,
	\end{align}
	where $d':=\{\frac{B_{i'}}{v_{i'}}\mid i'\in N'\}$ and 
	$P_{d'}(f'):=\{y\in P(f')\mid y_{i'}\leq d'_{i'}\ (i'\in N')\}$.
	Note that $P_{d'}(f')$ is a reduction of an integer polymatroid by an integer vector 
	$d'$ and is again an integer polymatroid; see Section 3.1 of Fujishige \cite{F2005}.
	Then, the problem (\ref{opt1}) is a linear optimization on polymatroids, 
	known to be solved efficiently by a greedy procedure; see Section 3.2 of Fujishige \cite{F2005}. 
	Thus, we obtain the optimal allocation $x^*\in \mathbb Z^{N'}_+$ by allocating as many goods as possible 
	(according to the polymatroid constraints) to buyers in descending order of their valuations.
	Therefore, we have $x^*_{i'}=f'_{d'}(H_{i'}\cup i')-f'_{d'}(H_{i'})$, where 
	$f'_{d'}:2^{N'}\to \mathbb Z_{+}$ is a monotone submodular function that defines $P_{d'}$, and is defined by  
	$f'_{d'}(S)=\min_{S'\subseteq S}\{f'(S\setminus S')+d'(S')\}$ for each $S\subseteq N'$.
	
	Then, an optimal soluation $x^{*}$ of (\ref{opt1}) is recursively obtained by the following:
	\begin{equation}
	\label{greedy}
	 x^{*}_{i'}=\min\bigl(\frac{B_{i'}}{v_{i'}},{\min_{H\subseteq H_{i'}}\{f(H\cup i')-x^{*}(H)}\}\bigr)\ (i'\in N').
	\end{equation}
	In the following, we prove this using the idea of Sato \cite{S2023} in the proof of their Proposition 2.1.
	
	Suppose that $f'_{d'}(H_{i'}\cup i')=f'(H)+d'(H_{i'}\setminus H)+d'_{i'}$ for some $H\subseteq H_{i'}$.
	This implies $f'_{d'}(H_{i'})=f'(H)+d'(H_{i'}\setminus H)$.
	Therefore, we have $x^*_{i'}=f'_{d'}(H_{i'}\cup i')-f_{d'}(H_{i'})=d'_{i'}=\frac{B_{i'}}{v_{i'}}$. 
	
	Suppose that $f_{d'}(H_i'\cup i')=f'(H\cup i')+d'(H_{i'}\setminus H)$ for some $H\subseteq H_{i'}$.
	Then, we have 
	\[
	d'(H_{i'}\setminus H)=f'_{d'}(H_{i'}\cup i')-f'(H\cup i')\leq f'_{d'}(H_{i'}\cup i')-f'_{d'}(H\cup i')\leq f'_{d'}(H_{i'})-f'_{d'}(H),
	\]
	where the first (resp. second) inequality holds by the definition (resp. submodularity) of $f'_{d'}$.
	Then, 
	\begin{align*}
	x^{*}(H_{i'})&=x^{*}(H)+x^{*}(H_{i'}\setminus H)\leq f'_{d'}(H)+d'(H_{i'}\setminus H)\\
	&\leq f'_{d'}(H)+\bigl(f'_{d'}(H_{i'})-f'_{d'}(H)\bigr)=f'_{d^{'}}(H_{i'})=x^{*}(H_{i'}), 
	\end{align*}
	where the first inequality holds by the polymatroid constraint $x^{*}(H)\leq f'_{d'}(H)$ and 
	$x^{*}_{i'}\leq d^{'}_{i'}$ for each $i\in N$ due to $x^{*}\in P_{d'}(f')$, 
	and the last equality holds by $x^*_{i'}=f'_{d'}(H_{i'}\cup i')-f'_{d'}(H_{i'})$ for each $i'\in N'$.
	Thus, all the inequalities in the above hold in equality.
	From this, we have $x^{*}(H)=f'_{d'}(H)$ and $x^{*}(H_{i'}\setminus H)=d^{'}(H_{i'}\setminus H)$.
	Using this, we have 
	\[
	x^*_{i'}=f'_{d'}(H_{i'}\cup i')-f'_{d'}(H_{i'})=f'(H\cup {i'})+d'(H_{i'}\setminus H)-x'(H_{i'})
	=f'(H\cup i')-x^{*}(H).
	\]
	In this case, $x^*_{i'}$ is calculated as the minimum of $f'(H\cup i)-x^{*}(H)$ with respect to $H\subseteq H_{i'}$.
	
	Moreover, using the same idea as the proof of Lemma \ref{prepare_OPT}, we have the following:
	\begin{observation}
	\label{greedy_property}
	In formula (\ref{greedy}), $x^*_{i_{b}}=\frac{B_{i_b}}{v_{i_b}}=1$ only if $x^*_{i_{a}}
	=\frac{B_{i_a}}{v_{i_a}}=\left\lfloor \frac{B_i}{v_i}\right\rfloor$ for each $i\in N$. 
	\end{observation}

	


\begin{proof}[Proof of Proposition \ref{optimal}]
	The LW maximizing problem can be written as:
	\begin{align}
	\label{optimization}
	&{\text{maximize}}\ \ \ \ \sum_{i\in N}\min(v_i x_i,B_i) \\
	&\text{subject to}\ \ \ \, x\in P(f)\cap \mathbb Z^N. \nonumber
	\end{align}
	Since $x\in\mathbb Z^N$, for the objective function, we have the following: 
	\begin{align*}
	\sum_{i\in N}\min(v_i x_i,B_i)&=\sum_{i\in N}\bigl(v_{i}\min(x_{i},\left\lfloor \frac{B_i}{v_i}\right\rfloor)+(B_i-v_i\left\lfloor \frac{B_i}{v_i}\right\rfloor)\min(\max(x_{i}-\left\lfloor \frac{B_i}{v_i}\right\rfloor,0),1)\bigr)\\
	&=\sum_{i\in N}\bigl(v_{i_a}\min\bigl(x_{i},\left\lfloor \frac{B_i}{v_i}\right\rfloor\bigr)+v_{i_b}\min(\max(x_{i}-\left\lfloor \frac{B_i}{v_i}\right\rfloor,0),1)\bigr).
	\end{align*}
	For a vector $\tilde{x}'\in \mathbb Z^N_{+}$, construct an allocation of virtual buyers by 
	$x'_{i_a}=\min(\tilde{x}'_i, \left\lfloor \frac{B_i}{v_i}\right\rfloor)$ and $x'_{i_b}=\tilde{x}'_i-x'_{i_a}$ for each $i\in N$.
	Since $x'_{i_a}+x'_{i_b}=\tilde{x}'_i$ for each $i\in N$, 
	it holds that $\tilde{x}'\in P(f)$ if and only if $x'\in P(f')$.
	Then, the allocation of virtual buyers obtained from an optimal solution of (\ref{optimization}) is a feasible solution of 
	the problem (\ref{opt2}), where the objective value is unchanged between (\ref{opt2}) 
	and (\ref{optimization}) by the above.
	Moreover, for an optimal allocation $x^{*}\in \mathbb Z^{N'}_{+}$ of (\ref{opt2}) obtained by (\ref{greedy}), 
	define $\tilde{x}^*_i=x^*_{i_a}+x^*_{i_b}$ for each $i\in N$.
	Then, $\tilde{x}^*$ is a feasible solution of (\ref{optimization}), 
	where the objective value is unchanged between (\ref{opt2}) 
	and (\ref{optimization}) by Observation \ref{greedy_property}.
	Therefore, the formula (\ref{greedy}) and defining $\tilde{x}^*_i=x^*_{i_a}+x^*_{i_b}$ for each $i\in N$ 
	provides an optimal solution for the problem (\ref{optimization}).
\end{proof}


\end{document}