\documentclass[twoside]{article}      % Comments after  % are ignored
\usepackage{amsmath,amssymb,amsfonts} % Typical maths resource packages
\usepackage{graphics}                 % Packages to allow inclusion of graphics
\usepackage{color}                    % For creating coloured text and background
\usepackage{hyperref}                 % For creating hyperlinks in cross references
\usepackage{comment} 			% comment
\usepackage{enumitem}

\oddsidemargin 0cm
\evensidemargin 0cm

\pagestyle{myheadings}         % Option to put page headers

                               % Needed \documentclass[a4paper,twoside]{article}
\markboth{{\small\it Description of Mechanical Learning}}
{{\small\it Chuyu Xiong} }

\textwidth 15.5cm
\topmargin -1cm
\parindent 0cm
\textheight 24cm
\parskip 1.8mm
\newtheorem{theorem}{Theorem}[section]
\newtheorem{proposition}[theorem]{Proposition}
%\newtheorem{corollary}[theorem]{Corollary}
\newtheorem{corollary}{Corollary}[theorem]
\newtheorem{lemma}[theorem]{Lemma}
\newtheorem{remark}[theorem]{Remark}
\newtheorem{definition}[theorem]{Definition}


\def\R{\mathbb{ R}}
\def\S{\mathbb{ S}}

\date{\small\it \today}
\title{Descriptions of Objectives and Processes of \\ Mechanical Learning
\footnote{Great thanks for whole heart support of my wife. Thanks for Internet and research contents contributers to Internet.}}

\author{ Chuyu Xiong \\
{\small Independent researcher, New York, USA} \\
{\small Email: chuyux99@gmail.com}
}

\begin{document}
\maketitle
\begin{abstract}
\input abs
\end{abstract}

{\sc Keywords: Mechanical learning, learning machine, objective and subjective patterns, X-form, universal learning, learning by teaching, internal representation space, data sufficiency, learning strategy, squeez to higher, embed to parameter space} \\ 
 \\

If you want to know the taste of a pear, you must change the pear by eating it yourself. ...... \\
All genuine knowledge originates in direct experience. \\
\indent \hspace{20pt} ----Mao Zedong \\

But, though all our knowledge begins with experience, it by no means follows that \\
all arises out of experience. \\
\indent \hspace{20pt} ----Immanuel Kant \\

Our problem, ...... is to explain how the transition is made from a lower level of knowledge \\
to a level that is judged to be higher. \\
\indent \hspace{20pt} ----Jean Piaget


\section{Introduction}
\input intro
\section{Learning Machine}\label{table}
\input lmachine
\section{Pattern, Examples, Objective and Subjective}\label{table}
\input patterns
\section{Learning by Teaching}\label{table}
\input learnbyteach
\section{Learning without Teaching Sequence}\label{table}
\input learnwithoutteach
\section{More Discussions about Learning Machine}\label{table}
\input morediscuss
\section{How to Design Effective Learning Machine}\label{table}
\input design

\begin{thebibliography}{99}
\input bib
\end{thebibliography}

\section*{Appendix}\label{table}
\input appendix


\end{document}
