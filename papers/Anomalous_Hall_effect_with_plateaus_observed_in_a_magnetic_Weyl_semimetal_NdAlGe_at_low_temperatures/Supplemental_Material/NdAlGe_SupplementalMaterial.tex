% ****** Start of file apssamp.tex ******
%
%   This file is part of the APS files in the REVTeX 4.2 distribution.
%   Version 4.2a of REVTeX, December 2014
%
%   Copyright (c) 2014 The American Physical Society.
%
%   See the REVTeX 4 README file for restrictions and more information.
%
% TeX'ing this file requires that you have AMS-LaTeX 2.0 installed
% as well as the rest of the prerequisites for REVTeX 4.2
%
% See the REVTeX 4 README file
% It also requires running BibTeX. The commands are as follows:
%
%  1)  latex apssamp.tex
%  2)  bibtex apssamp
%  3)  latex apssamp.tex
%  4)  latex apssamp.tex
%
\documentclass[%
%reprint,
%superscriptaddress,
%groupedaddress,
%unsortedaddress,
%runinaddress,
%frontmatterverbose, 
preprint,
%preprintnumbers,
%nofootinbib,
%nobibnotes,
%bibnotes,
 amsmath,amssymb,
 aps,
%pra,
%prb,
%rmp,
%prstab,
%prstper,
%floatfix,
]{revtex4-2}

\usepackage{graphicx}% Include figure files
\usepackage{dcolumn}% Align table columns on decimal point
\usepackage{bm}% bold math
%\usepackage{hyperref}% add hypertext capabilities
%\usepackage[mathlines]{lineno}% Enable numbering of text and display math
%\linenumbers\relax % Commence numbering lines

%\usepackage[showframe,%Uncomment any one of the following lines to test 
%%scale=0.7, marginratio={1:1, 2:3}, ignoreall,% default settings
%%text={7in,10in},centering,
%%margin=1.5in,
%%total={6.5in,8.75in}, top=1.2in, left=0.9in, includefoot,
%%height=10in,a5paper,hmargin={3cm,0.8in},
%]{geometry}

\begin{document}

%\preprint{APS/123-QED}

\title{Supplemental Material for\\ ``Anomalous Hall effect with plateaus observed in a magnetic Weyl semimetal NdAlGe at low temperatures''}% Force line breaks with \\
%\thanks{A footnote to the article title}%

\author{
Naoki Kikugawa$^{1}$, 
Shinya Uji$^{2}$, 
Taichi Terashima$^{2}$
}
 \affiliation{
$^{1}$Center for Basic Research on Materials (CBRM), National Institute for Materials Science, 3-13 Sakura, Tsukuba, Ibaraki 305-0003, Japan\\
$^{2}$Research Center for Materials Nanoarchitectonics (MANA), National Institute for Materials Science, 3-13 Sakura, Tsukuba, Ibaraki 305-0003, Japan
}

%\affiliation{%
 %Authors' institution and/or address\\
 %This line break forced with \textbackslash\textbackslash
%}%

%\collaboration{CLEO Collaboration}%\noaffiliation

%\date{\today}% It is always \today, today,
             %  but any date may be explicitly specified

%\begin{abstract}
%\end{abstract}

%\keywords{magnetic Weyl semimetal, anomalous Hall effect, NdAlGe}%Use showkeys class option if keyword
                              %display desired
\maketitle

%\tableofcontents


\setcounter{equation}{0}
\setcounter{figure}{0}
\setcounter{table}{0}
\setcounter{page}{1}
\makeatletter
\renewcommand{\theequation}{S\arabic{equation}}
\renewcommand{\thefigure}{S\arabic{figure}}
\renewcommand{\bibnumfmt}[1]{[S#1]}
\renewcommand{\citenumfont}[1]{S#1}


The Supplemental Material contains: 

\begin{description}
%\setcounter{enumi}{0}
%
%
%\renewcommand{\theenumii}{\arabic{enumi}}
%
%
%\vspace{-6mm}
%
\item[S1] Specific heat ($C_{P}$) of NdAlGe without a magnetic field down to 0.4\,K. 

\item[S2] Hall resistivity ($\rho_{yx}$) of NdAlGe to show sample dependence. 

\item[S3] Isothermal magnetization ($M$) of NdAlGe at 2\,K between $-$16 and $+$16\,T 
under a magnetic field applied along $\lbrack$001$\rbrack$ direction. 

\end{description}

\clearpage

\hspace{-7 mm}
\textbf{S1: Specific heat ($C_{P}$) of NdAlGe without a magnetic field down to 0.4 K}

%
As described in the Results section, we measured the specific heat ($C_{P}$) of NdAlGe down to 0.4\,K. 
%
A clear jump of $C_{P}$ was observed at the magnetic ordered temperature at $T_{\rm M}$\,=\,13.5\,K, 
suggesting that the transition occurs in bulk. 
%
Also, no other transition was seen in our floating-zone crystals. 
%
These results are in contrast that two successive transitions were observed at lower temperatures 
($\sim$5 K and $\sim$6\,$-$\,7\,K) \cite{Yang_PhysRevMater_2023_SM,Dhital_PhyrevB_2023_SM}, 
or a single transition was observed at 5\,$-$\,6 K, 
lower temperatures than our crystals, in flux-grown NdAlGe crystals 
\cite{Zhao_NewJPhys_2022_SM,Cho_SSRN_2022_SM}. 
We also note that the transition width at $T_{\rm M}$ is as sharp as 0.4\,K in our floating-zone crystals. 
%

%
\begin{figure}[h]
%\vspace*{-8mm}
%\hspace*{20mm}
\includegraphics[width=65mm]{FigS1}% Here is how to import EPS art
\caption{
Temperature dependence of specific heat of NdAlGe under zero magnetic field down to 0.4\,K. 
}
\end{figure}
%

\clearpage


\hspace{-7 mm}
\textbf{S2: Hall resistivity ($\rho_{yx}$) of NdAlGe to show sample dependence}

%
To check for sample dependence of the Hall resistivity as described in the Results section, 
Figure S2(a) shows the Hall resistivity $\rho_{yx}$ of NdAlGe at 40\,mK under a magnetic field 
between $-$17.5 and $+$17.5\,T measured on another sample. 
%
From the slope of the $\rho_{yx}$ above 1\,T, 
the ordinally Hall coefficient was evaluated $R_{0}$\,=\,$+$1.25\,$\times$\,10$^{-3}$\,cm$^{3}$/C, 
which agrees well with the value of $+$1.28\,$\times$\,10$^{-3}$\,cm$^{3}$/C obtained from Fig. 1(b) 
(in the Main manuscript) within experimental error. 
%
Figure S2(b) shows the Hall resistivity $\rho_{yx}$ of the same sample below 1.1\,K 
to show that the plateaus are reproducible.  
Also, the $\rho_{yx}$ reaches 2.7\,$\mu\Omega$cm in the field-induced polarized state above 0.5\,T.
%

%
\begin{figure}[h]
%\vspace*{-8mm}
%\hspace*{20mm}
\includegraphics[width=65mm]{FigS2}% Here is how to import EPS art
\caption{
(a) Hall resistivity $\rho_{yx}$ of NdAlGe at 40\,mK on another sample to check for sample dependence. 
(b) The Hall resistivity $\rho_{yx}$ of the same sample below 1.1\,K.  
}
\end{figure}
%

\clearpage


\hspace{-7 mm}
\textbf{S3: Isothermal magnetization ($M$) of NdAlGe at 2\,K between $-$16 and $+$16\,T under a magnetic field applied along $\lbrack$001$\rbrack$ direction. }

%
We investigated how the Hall resistivity in NdAlGe correlates the magnetic property. 
%
In Figure S3(a), we show the isothermal magnetization ($M$) of NdAlGe under a magnetic field 
along $\lbrack$001$\rbrack$ direction at 2\,K. 
%
The extrapolated zero-field magnetization $M_{0}$ is defined. 
%
The hysteresis between up and down field-sweep is obvious and is closely linked 
to the Hall resistivity (Fig. 2(a) in the Main Manuscript). 
%
To check the anomaly in the magnetization, we expand the measurement between $-$16 and $+$16 T 
at 2\,K, as shown in Fig. S3(b). 
%
No anomaly was observed above 1\,T, 
which is consistent with the Hall resistivities in our floating-zone crystals (Figs. 1(b) and S2(a)) 
but is in sharp contrast to the results observed in the flux-grown crystals 
where a metamagnetic-like behavior was seen around 3\,T 
\cite{Zhao_NewJPhys_2022_SM,Yang_PhysRevMater_2023_SM,Cho_SSRN_2022_SM,Dhital_PhyrevB_2023_SM}. 
%

%
\begin{figure}[h]
%\vspace*{-8mm}
%\hspace*{20mm}
\includegraphics[width=65mm]{FigS3}% Here is how to import EPS art
\caption{
(a) Isothermal magnetization ($M$) of NdAlGe under $H$\,$\parallel$\,$\lbrack$001$\rbrack$ 
direction at 2\,K. 
(b) The isothermal magnetization at 2\,K between $-$16 and $+$16\,T.
}
\end{figure}

\clearpage


\bibliography{NdAlGe_SupplementalMaterial_Reference}% Produces the bibliography via BibTeX.










\end{document}
%
% ****** End of file apssamp.tex ******
