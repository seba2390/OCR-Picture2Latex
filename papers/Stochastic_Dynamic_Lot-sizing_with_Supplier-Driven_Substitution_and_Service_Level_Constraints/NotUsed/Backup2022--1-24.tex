%%    TEMPLATE for articles submitted to the IWLS 2011
%%
%%
%%     Please do not remove lines commented out with %+
%%           these are for the editors' use.
%%

\documentclass[10pt]{article}
\usepackage{epsfig}
\usepackage{lscape}
\usepackage{adjustbox}
\usepackage{mathtools}
\usepackage{amsmath}
\usepackage{comment}
\usepackage{booktabs}
\usepackage{array}
\usepackage{xcolor}
\usepackage[margin=1 in]{geometry}
\usepackage[ruled,vlined]{algorithm2e}
\usepackage{amssymb}





\usepackage{textcomp}
\usepackage{caption}
\usepackage{multirow}
\usepackage{here}
\usepackage{amsmath}
\usepackage{dsfont}
\newcommand{\R}{\mathds{R}}
\newcommand{\E}{\mathds{E}}
\usepackage{graphicx}
\usepackage{subfig}
\allowdisplaybreaks

% For the mathematical formulation

\newcommand{\ti}{t} %For the time period index
\newcommand{\TI}{\mathcal{T}}
\newcommand{\Ti}{T}
\newcommand{\ka}{k} % For product index
\newcommand{\KA}{\mathcal{K}}
\newcommand{\Ka}{K}
\newcommand{\jey}{j} % For product index
\newcommand{\Graf}{\mathcal{A}} %For the y variable
\newcommand{\Bi}{B} %For the backlog
\newcommand{\Vi}{v} %For the order upto level
\newcommand{\Es}{S} %For the Substitution
\newcommand{\Zed}{z} %For the z variable
\newcommand{\x}{x} %For the x variable
\newcommand{\y}{y} %For the y variable
\newcommand{\InvPos}{inventory level after production }
\newcommand{\cn}{\mathcal{C}(n) }

\newcommand{\Csub}{\mathcal{K}^+_k}
\newcommand{\Psub}{\mathcal{K}^-_k}

\newcommand{\tAct}{\hat{\ti}} % actual time period in infinite-horizon





%%%%%%%%%%%%%%%%%%%%%%%%%%%%%%%%%%%%%%%%%%%%%%%%%%%%%%%%%%%%%%%%%%%%%%%%%%%%
%%  Do not change these:
%\textwidth=6.0in  \textheight=8.25in

%%  Adjust these for your printer:
%\leftmargin=-0.3in   \topmargin=-0.20in


%%%%%%%%%%%%%%%%%%%%%%%%%%%%%%%%%%%%%%%%%%%%%%%%%%%%%%%%%%%%%%%%%%%%%%%%%%%%
%  personal abbreviations and macros
%    the following package contains macros used in this document:
%%%%%%%%%%%%%%%%%%%%%%%%%%%%%%%%%%%%%%%%%%%%%%%%%%%%%%%%%%%%%%%%%%%%%%%%%%%
%
%  To include an item in the INDEX of the conference volume,
%           flag it with    \index{<item name>}
%  The use of this macro is illustrated in the text.
%
%%%%%%%%%%%%%%%%%%%%%%%%%%%%%%%%%%%%%%%%%%%%%%%%%%%%%%%%%%%%%%%%%%%%%%%%%%%%%

\newcommand{\cred}{\color{red!65!black}}

\def\Title#1{\begin{center} {\Large {\bf #1} } \end{center}}

\begin{document}


\Title{Multi-stage stochastic lot sizing with substitution and joint $\alpha$ Service Level }

\bigskip\bigskip

%+\addtocontents{toc}{{\it First Author}}
%+\label{AuthorNameStart}

%%    TEMPLATE for articles submitted to the IWLS 2011
%%
%%
%%     Please do not remove lines commented out with %+
%%           these are for the editors' use.
%%







 \textbf{\large Key words:} Stochastic lot sizing, $\alpha$ service level, Product substitution, Joint chance-constraint, Dual-based decomposition methods



\section{Abstract}

We consider multi-stage lot sizing problem with stochastic demand and the possibility of product substitution. Considering different production costs, the use of substitution can increase the revenue and customer satisfaction specially when the demand is uncertain. The goal is to minimize the total expected cost while satisfying a predetermined service level. In our model, we consider the $\alpha$ service level which limits the probability of stock outs, defined as a chance-constraint jointly over multiple products. The stochsticity is represented as a scenario tree, and we propose different solution policies for this multi-stage stochastic using mixed integer mathematical models and dual-based decomposition methods modified for this problem. The policies are applied and evaluated in rolling-horizon framework. Through extensive numerical experiments we compare different policies, investigate the value of substitution and propose some managerial insights. 

\section{Introduction}
Contents of this section:
\begin{itemize}
    \item Lot sizing problem 
    \item Stochastic lot sizing problem with service levels
    \item Motivation on substitution
    \item Motivation of stochastic lot sizing + random demand +joint service level + substitution 
    \item industrial relevance
    \item Contribution of the work
\end{itemize}

The basic lot sizing problem is a multi-period production planning problem which considers the trade-off between setup costs and inventory holding cost, and defines the optimal timing and quantity of production to minimize the total cost. 
Uncertainty in demand is a challenge that many suppliers are dealing with in real world applications. 
Common approaches to deal with this challenge is whether to consider the backorder cost of tangible or intangible effects which is difficult to estimate or put a service level criteria. 
In this research we study the stochastic lot sizing problem is trying to minimize cost while satisfying a service level criteria. The $\alpha$ service level is an event-oriented service level which puts limits on the probability of stock outs. This service level is usually defined as a chance-constraint. 

In the market, when an item is out of stock, the firm has the option to substitute it with another product. This substitution is initiated by the firm. Having a limited capacity, using the substitution option can increase the revenue and customer satisfaction especially when dealing with demand uncertainty. Considering this option in production decisions may result in cost savings, and it is worthwhile to investigate. In this research, we consider the  lot sizing problem with stochastic demand and the possibility of product substitution. This problem has a practical relevance in electronics and steel industries where it is possible to substitute a lower-grade product with a higher-grade one. Semiconductors or microchips are good examples of these types of products~\cite{lang2010efficient}. 

%The basic lot sizing problem is a multi-period production planning problem which considers the trade-off between the setup costs and inventory holding cost and its goal is to define the optimal timing and quantity of production to minimize the total cost. 
The lot sizing problem has been extensively studied and applied in real world situations. 
In this research, we consider the stochastic lot sizing problem with joint $\alpha$ service level over different products and when there is the possibility of product substitution. Substitution option and joint service level over all product are in line with each other as in both cases, we consider all products together to reach the required service level and minimizing the total cost. 

%The basic assumption to approximate and solve  these models is that the demand for each product are independent from each other and there is no autocorrelation~\cite{tempelmeier2011column}. 
%In this research, we model the stochastic lot sizing problem with substitution and a joint $\alpha$ service level using scenario tree.
%for which there is no need to have these assumptions and it is possible to use any demand scenarios. 
We consider infinite-horizon problem in which at each point of time we need to make decision on setup and production decision based on the current state of the problem. We follow the ``dynamic" strategy \cite{bookbinder1988strategies} for which the setups and production decisions are defined through the planning horizon and can be dynamically updated when the demands are realized. To solve this problem we consider a finite-horizon problem and apply it in a rolling-horizon environment. The stochastic demand is represented as a scenario tree for this problem.
%Having multi-stage problem and following the ``dynamic" strategy \cite{bookbinder1988strategies} the setups and production decisions are defined through the planning horizon and updated with regard to the demand realization. The stochasticity in demand in reflected in scenario tree.

The challenge of these stochastic models is that with increasing the number of scenarios the solution time will increase extensively and makes it difficult to reach a reasonable solution in a reasonable amount of time. In this research we will use dual based decomposition method to deal with this challenge. 
The contribution of this research can be summarized as follows:

    
\begin{itemize}
\item Considering the multi-stage lot sizing problem with joint $\alpha$ service level constraints and firm-driven substitution, which is new to the literature.

%\item Proposing mathematical model for the stochastic dynamic lot sizing problem with joint $\alpha$ service level constraints and firm-driven substitution possibility.
\item Proposing a dynamic programming formulation for the finite-horizon problem which can be applied in a rolling-horizon fashion.
\item Approximating the models as a two-stage stochastic programming using scenario tree. (in which the first stage decisions are setups and production amounts and the substitution decisions and amount of inventory and backlog are the recourse actions and second stage decisions.)
\item Proposing efficient policies including deterministic and chance-constraint policies to solve the resulting model and comparing these policies to solve this problem in a rolling-horizon using simulation.
    \item Providing managerial insight based on numerical experiments under different setting.% (as in ~\cite{jiang2017service} and  ~\cite{jiang2017production}) modified for the stochastic lot sizing problem with substitution.
    


\end{itemize}

\section{Literature review}
The related literature of this work can be categorized in two streams. The first part is dedicated to the lot sizing inventory models with substitution and the second part is dedicated to the stochastic lot sizing problem and joint service level. To the best of our knowledge, no research has investigated stochastic lot sizing problem with substitution and joint service levels.
%and the current research addresses this gap in the literature.  

\subsection{Lot sizing and inventory problems with substitution}
\subsubsection{Deterministic models}
In the literature, there are two types of substitution, the demand-driven substitution and firm-driven substitution. In the demand-driven substitution the customer decides which product to substitute~\cite{zeppetella2017optimal}, while in the firm-driven case, it is the firm which makes the substitution decisions~\cite{rao2004multi}. Hsu et al.~\cite{hsu2005dynamic} studied two different versions of the dynamic uncapacitated lot sizing problem with one-way substitution, when there is a need for physical conversion before substitution, and when it does not require any conversion. The authors used dynamic programming and proposed a heuristic algorithm to solve the problem.  Lang and Domschke~\cite{lang2010efficient} considered the uncapacitated lot sizing problem with general substitution in which a specific class of demand can be satisfied by different products based on a substitution graph. They model the problem as an Mixed-integer linear program (MILP) and proposed a plant location reformulation and some valid inequalities for it.
\subsubsection{Stochastic models}
Many studies in the field of the stochastic inventory planning have considered the possibility of substitution. While the majority of them investigated the demand-driven substitution, some research considered the firm-driven substitutions. In this research we consider the firm-driven substitutions. In the same vein, Bassok et al.~\cite{bassok1999single} investigated the single-period inventory problem with random demand and downward substitution in which a lower-grade item can be substituted with the ones with higher-grade. This model is an extension of the newsvendor problem and there is no setup cost in case of ordering. The sequence of decisions is as follows: first, the order quantity for each of the items is defined by the ordering decision, and in the next step when the demand is realized, the allocation decisions are defined. Rao et al.~\cite{rao2004multi} considered a single-period problem with stochastic demand and downward substitution, and model it as a two-stage stochastic program. In their model they consider the initial inventory and the ordering cost as well. 

Most of the research about the  product substitution with stochastic demand investigated the demand-driven substitution. In these problems the customer may choose another product, if the original item cannot be found. This is known as ``stock out substitution". Akçay et al.~\cite{akccaycategory} investigated a single-period inventory planning problem with substitutable products. Considering the stochastic customer-driven and stock out based substitution, their optimization based method jointly defines the stocking of each product, while satisfying a service level. They adapt the Type II service level or ``fill rate" for each individual product and overall within a category of products.  

Another research stream considers the possibility of having multiple graded output items from a single input item, which is known as ``co-production"~\cite{ng2012robust}. In these problems, there is a hierarchy in the grade of output items and it is possible to substitute a lower grade item with the ones with higher grade~\cite{bitran1992ordering}. Hsu and Bassok~\cite{hsu1999random} considered the single-period production system with random demand  and random yields. Although they didn't mention ``co-production" in their research, their model defines the production amount of a single item and the allocation of its different output items to different demand classes~\cite{hsu1999random}.
Birtan and Dasu~\cite{bitran1992ordering} studied the multi-item, multi-period co-production problem with deterministic demand and random yield, and proposed two approximation algorithms to solve it. The first approximation is based on a rolling-horizon implementation of the finite-horizon stochastic model. For the second approximation, they considered the two-period, two-item problem and then applied a simple heuristic based on the optimal allocation policy for that, in a multi-period setting. Bitran and Leong~\cite{bitran1992deterministic} considered the same problem as Bitran and Dasu~\cite{bitran1992ordering} with the service level constraint. They proposed deterministic near-optimal approximations within a fixed planning horizon. To adapt their model to the revealed information, they applied the proposed model using simple heuristics in a rolling planning horizon~\cite{bitran1992deterministic}.  Bitran and Gilbert~\cite{bitran1994co} considered the co-production and random yield in a semiconductor industry and propose heuristic methods to solve it. 
 
%\subsection{chance-constraints optimization}
\subsection{Stochastic lot sizing problem and $\alpha$ service level constraints}
In the basic lot sizing problem, all the parameters are deterministic. Stochastic lot sizing problems address this restrictive assumption by considering uncertainty in different parameters such as demand and cost parameters. In lot sizing problems, we are dealing with the sequence of decisions over the planning horizon, and multi-stage stochastic programming is a method to incorporate uncertainty. In multi-stage stochastic problems, the uncertainty is typically represented as a scenario tree. 

 Haugen et al.~\cite{haugen2001progressive} considered the multi-stage uncapacitated lot sizing problem and proposed a progressive hedging algorithm to solve it. Guan and Miller~\cite{guan2008polynomial} proposed a dynamic programming algorithm for a similar version. Using the same algorithm, Guan~\cite{guan2011stochastic} studied the capacitated version of the problem with the possibility of backlogging. Lulli and Sen~\cite{lulli2004branch} proposed a branch and price algorithm for multi-stage stochastic integer programming and applied the method to the stochastic batch-sizing problem. In this problem, they consider that the demand, production, inventory and set up costs are uncertain. The difference between this problem and the lot sizing problem is that the production quantities are in batches and the production decision variables are discrete values which define the number of batches that will be produced. This problem is a more general case of lot sizing problem. In another research, Lulli and Sen~\cite{lulli2004branch}  proposed a scenario updating method for the stochastic batch-sizing problem. As backlogging is allowed in this problem, it is considered as a stochastic model with complete fixed recourse and all the scenarios are feasible.  

Another common approach to deal with stochastic demand is using service levels. In this context the planners put  a demand fulfillment criterion to mitigate the risk of demand uncertainty. Stochastic lot sizing problems with service level constraints have been studied extensively~\cite{tempelmeier2007stochastic}. One of the main service levels is the $\alpha$ service level which is an event-oriented service level, and imposes limits on the probability of a stock out. The $\alpha$ service level is presented as a chance-constraint and is usually defined for each period and product separately. Bookbinder and Tan~\cite{bookbinder1988strategies} investigate stochastic lot sizing problems with an $\alpha$ service level and propose three different strategies based on the timing of the setup and production decisions, for this problems. These strategies are the \textit{static}, \textit{dynamic}, and \textit{static-dynamic} strategy. In the \textit{static} strategy, both the setup and production decisions are determined at the beginning of the planning horizon and they remain fixed when the demand is realized. In the \textit{dynamic} strategy, both the setup and production decisions are dynamically changed with the demand realization throughout the planning horizon. The \textit{static-dynamic} strategy is between these two strategies in which the setups are fixed at the beginning of the planning horizon and the production decisions are updated when the demands are realized. The $dynamic$ strategy can be modeled as a multi-stage stochastic lot sizing problem.  

There are some studies which define the $\alpha$ service level jointly over different planning periods. Liu and K{\"u}{\c{c}}{\"u}kyavuz~\cite{liu2018polyhedral} considered the uncapacitated lot sizing problem with a joint service level constraint and studied the polyhedral structure of the problem and proposed different valid inequalities and a reformulation for this problem. Jiang et al.~\cite{jiang2017production} considered the same problem with and without pricing decisions. Gicquel and Cheng~\cite{gicquel2018joint} investigated the capacitated version of the same problem, and following the same methodology as Jiang et al.~\cite{jiang2017production} they used a sample approximation method to solve it. This method is a variation of the sample average approximation method which is proposed by Luedtke and Ahmed~\cite{luedtke2008sample} to solve models with chance-constraints using scenario sets. All the mentioned studies consider single item models in which the joint service level is defined over all periods. In this research we consider substitution and the joint service level is defined over all products.


\section{Problem definition and formulation}
We consider a stochastic lot sizing problem with the possibility of substitution in an infinite time horizon for which the discrete planning periods are indexed by $\tAct$. There are multiple types of products, whose index set is $\KA= \{1,...,\Ka\}$, with random demand, denoted by vector $D_{\tAct} = ({D}_{\tAct 1}, {D}_{\tAct 2},..., {D}_{\tAct \Ka})$. At each stage, we need to make decisions about the production setups, production and substitution amounts, and accordingly define the potential inventory and backlog levels. There is production lead time of one, i.e., what is produced in the current stage is available at the next stages. 
These decisions are made sequentially at each stage, based on the available inventory and backlog in the system, random future demand, and the history of realized demand, such that a joint service level over all products is to be satisfied in the following stage. This is in line with the  ``dynamic'' strategy that is defined for the stochastic lot sizing problem~\cite{bookbinder1988strategies}. 

%This is in line with the  ``dynamic'' strategy that is defined for the stochastic lot sizing problem in which both setup and production decisions are updated when the demand are realized ~\cite{bookbinder1988strategies}. 

Figure~\ref{MultistageDynamics} illustrates the dynamics of decisions for this problem at each stage. In this problem, decision-making stages are the same as time periods, and for the rest of this paper, we will use them interchangeably. 
%There are $\Ka$ different products with random demand and the notations are shorthand for the vectors which corresponding elements are defined for each product.
At each point of time, $\tAct$, the demand realization vector $\hat{D}_{\tAct} =(\hat{D}_{\tAct 1}, \hat{D}_{\tAct 2},..., \hat{D}_{\tAct \Ka})$ is observed, and also given the initial state of the system, described by the vector of current on-hand inventory, $\hat{\Vi}_{\tAct} =(\hat{\Vi}_{\tAct 1}, \hat{\Vi}_{\tAct 2},..., \hat{\Vi}_{\tAct \Ka})$, and the backlog vector, $\hat{\Bi}_{\tAct}=(\hat{\Bi}_{\tAct 1}, \hat{\Bi}_{\tAct 2},..., \hat{\Bi}_{\tAct \Ka})$, two sets of decisions are made. The first set includes substitution, inventory, and backlog decisions denoted by $\Es_{\tAct}= ({\Es}_{\tAct 11}, {\Es}_{\tAct 12},..., {\Es}_{\tAct \Ka \Ka}), I_{\tAct} = ({I}_{\tAct 1}, {I}_{\tAct 2},..., {\I}_{\tAct \Ka}), \Bi_{\tAct} =({\Bi}_{\tAct 1}, {\Bi}_{\tAct 2},..., {\Bi}_{\tAct \Ka})$ vectors, respectively. 
The rest of the decisions in the current period are production, setup, and \InvPos at the end of current period which are denoted by $x_{\tAct} = (x_{\tAct 1}, x_{\tAct 2},..., x_{\tAct \Ka}), y_{\tAct} = ({y}_{\tAct 1}, {y}_{\tAct 2},..., {y}_{\tAct \Ka}), \Vi_{\tAct} = ({\Vi}_{\tAct 1}, {\Vi}_{\tAct 2},..., {\Vi}_{\tAct \Ka})$ vectors, respectively. 
It should be noted that all these decisions are made simultaneously, but having lead time of one period and assuming that demand in period $\tAct$ is observed at the beginning of the period, the production quantities made during period $\tAct$ can be used only in the next periods, i.e., they are not available to satisfy the same period demand or backlogged demand. Therefore, we defined two different inventory level vectors, namely, $I_{\tAct}$ as the inventory level immediately after demand satisfaction, but before production, and $\Vi_{\tAct}$ as the inventory level at the end of the period, also taking into account the production in period $\tAct$. 
%These two parameters shows the state of the system at beginning of the current period. The rest of the decisions in the current period are production, setup, and inventory position at the end of current period which are denoted by $x, y, v$, respectively.  It should be noted that the last three decisions are used to satisfy the future periods demand. In other words, the production in the current period will not be used for the same period demand.
The values of $\Vi_{\tAct}$ and $\Bi_{\tAct}$ will be the inputs for the next period, describing the next state of the system.
%In each period, we should define the decisions such that the joint service level in the next stage is satisfied, considering different demand scenarios. 
\begin{figure}[!h]
\begin{center}
\includegraphics[scale=0.6]{Diagram.png}
\caption{Dynamics of decisions at each stage} 
\label{MultistageDynamics}
\end{center}
\end{figure}


The inventory of a product can be used to satisfy its own demand or another product demand based on the substitution graph $G$ with vertex set $\KA$ and arc set $\Graf$. If $(\ka, \jey) \in \Graf$ then product $\ka$  can fulfill demand of product $\jey$ but a substitution cost of $c^{\text{sub}}_{\tAct \ka \jey }$ per unit is incurred at period $\tAct$. Note that $(\ka, \ka) \in \Graf  \text{ for all } \ka \in \KA$, and $S_{\tAct \ka \ka}$ corresponds to the amount of product $\ka$ which is used to satisfy its own demand.  Demand for a specific product is met either from the inventory of that product or from the inventory of another product through substitution, or else the demand is backlogged. In each period $\tAct$, while insufficient inventory will lead to backlog denoted by $\Bi_{\tAct \ka}$, unnecessary stocks will increase the holding cost. An inventory holding cost of $c^{\text{hold}}_{\tAct \ka}$ per unit is charged for the quantity being stored after the demand satisfaction in each period, denoted by $I_{\tAct \ka}$. Furthermore, in each period where production occurs, a setup has to be performed which incurs a fixed setup cost of $c^{\text{setup}}_{\tAct \ka}$. We consider the trade-off between these costs while making decisions at each period, also ensuring that the random demand in the next period can be satisfied with high probability based on a predefined service level. 

To deal with this infinite-horizon problem we propose to solve the finite-horizon variant of the problem, and in turn apply it in a rolling-horizon framework, as illustrated in Figure~\ref{fig:FiniteVSInfinite}.
%We will use  $\tAct$ as the time period index for actual period in infinite time horizon. 
For the finite-horizon problem with $\Ti$ planning periods indexed by $\ti \in \TI=\{1 , ..., \Ti\}$, we propose a multi-stage stochastic programming model with joint chance-constraints. 
%with the possibility of substitution
%In this finite-horizon model, we need to decide about the setup timing, and the production and substitution amounts for each product through the planning periods.
Being at an actual decision-making period $\tAct$ (which is equivalent to $t=1$ in the finite-horizon model), given the state of the system at $\tAct$ the model considers decisions for the $\Ti$ stages to guide the implementable first-stage ($\ti =1$) decisions that would satisfy the joint service level in the next period, $\ti=2$.



\begin{figure}[!h]
\begin{center}
\includegraphics[scale=0.6]{FiniteVSInfinite.png}
\caption{Rolling-horizon framework} 
\label{fig:FiniteVSInfinite}
\end{center}
\end{figure}


%\begin{figure}[!h]
%\begin{center}
%\includegraphics[scale=0.5]{Diagram.png}
%\caption{Dynamics of the multi period model} 
%\label{Diagram}
%\end{center}
%\end{figure}




%which is adapted from  Lang and Domschke~\cite{lang2010efficient} for deterministic uncapacitated lot sizing problem with substitution. 
%This model considers a general substitution graph in which a demand class can be fulfilled with multiple products based on the substitution graph and at a substitution cost.


\begin{table}[H]
\centering
\caption{Notation for the mathematical model}
\begin{adjustbox}{width=1\textwidth,center=\textwidth}
\begin{tabular}{ll}
\toprule
{\textbf {Sets}} & {\textbf {Definition}} \\ \midrule
%$\hat{\TI}$  & Set of actual planning periods of infinite-horizon, indexed by $1, ..., \hat{\ti}, ... ,\hat{\Ti}$ \\ 
$\TI$  & Set of planning periods, indexed by $1, ... ,\Ti$ \\ 
$\KA$  & Set of products, indexed by $1, ... ,\Ka$ \\
%$N$  & Set of demand classes\\ 
%$V = K \cup N$  & Vertex set of substitution graph\\
$ G = (\KA,\mathcal{A})$  & Substitution graph \\
$ \Graf \subseteq \KA \times \KA$  & Directed arcs of substitution graph denoting feasible substitutions, which include self loops
%: $(k, j) \in \Graf$ if product $\ka$  can fulfill demand of product $\jey$
\\
%$ G = (V,\Graf)$  & Substitution graph \\
$  \Csub = \{\ka' \mid (\ka,\ka') \in \Graf\}$  & Set of products whose demand can be fulfilled by product $\ka$  \\
$ \Psub = \{\ka \mid (\ka',\ka) \in \Graf\}$  & Set of products that can fulfill the demand of product $\jey$  \\
$ \mathcal{N} $  & Set of nodes in the scenario tree \\
$ \cn $  & Set of child nodes for node $n$ in the scenario tree \\
\midrule 
{\textbf {Parameters}} & {\textbf {Definition}} \\ \midrule
$c^{\text{setup}}_{\ti \ka}$ & Setup cost for product $\ka$ in period $\ti$ \\ 
$c^{\text{hold}}_{\ti \ka}$  & Inventory holding cost for product $k$ in period $\ti$  \\ 
$c^{\text{sub}}_{\ti \ka \jey }$  & Substitution cost if product $\ka$  is used to fulfill the demand of product $\jey$  in period $\ti$  \\ 
$c^{\text{prod}}_{\ti \ka}$  & Production cost for product $\ka$ in period $\ti$  \\
$c^{\text{back}}_{\ti \ka}$  & Backlog cost for product $\ka$  in period $\ti$ \\
$\alpha$  & Minimum required joint service level \\ 
$M_{\ti \ka}$  & A sufficiently large number \\ 
%$I_{k0}$ & The initial inventory for product $\ka$  \\ 
%${d}_{jt}$  & Demand for class {\it j} in period $\ti$ (model input)\\ 
${D}_{\ti \ka}$ & Random demand variable for product $\ka $ in period $\ti$  \\ 
${D}^\text{Hist}_{\ti \ka}$ & Random demand history from period 1 to period $\ti$ for product $\ka $  \\ 
$\ti(n)$ & Time period at node $n$  \\ 
%$\hat{\Vi}_{p(n), \ka} $&  The amount of initial \InvPos for product $\ka$ at parent of node  $n$\\
%$\hat{\Bi}_{p(n), \ka} $&  The amount of initial backlog for product $\ka$ at the parent of node $n$  \\
$\hat{\Vi}_{\ka} $&  The amount of initial inventory level for product $\ka$ \\
$\hat{\Bi}_{\ka} $&  The amount of initial backlog for product $\ka$   \\
$q_{nm} $&  The probability of child node $m$ from node $n$  \\
$\mathbb{P} $&  The probability distribution of the demand process (It should be demand or demand process??)\\ 
\midrule
{\textbf {Decision variables}} & {\textbf {Definition}} \\ \midrule
$\y_{n \ka}$ & Binary variable which is equal to 1 if there is a setup for product $k$ at node $n$, 0 otherwise \\ 
$\x_{n \ka}$ & Amount of production for product $\ka$  at node $n$  \\ 
$\Es_{n \ka \jey}$ & Amount of product $\ka$  used to fulfill the demand of product $\jey$  at  node $n$   \\
${I}_{n \ka}$ & Amount of physical inventory for product $k$ after the demand satisfaction for node $n$  \\
${\Bi}_{n \ka}$ & Amount of backlog for product $k$ at the end of period for node $n$  \\
${\Vi}_{n \ka}$ & The \InvPos for product $\ka$ at the end of period for node $n$  \\
 \bottomrule
% & $^*{ ( \Csub \text{ and }  \Psub  \text{ sets include } k) }$
\end{tabular}
\end{adjustbox}
 \label{tab:Sub_parameters}
\end{table}





 
 





%\begin{flalign}
%&pr((\sum_{t'=1}^{t} \sum_{k\in P_{j}} \overline{S}_{\ka \jey \ti'} -\sum_{t'=1}^{t} \overline{D}_{jt'}) \geq 0~~~~~\forall \jey \in N ) \geq \alpha_{c} &   \forall \ti  \in \TI,  & \label{eq:Sub_ST_Service}
%\end{flalign}


%Constraints (\ref{eq:Sub_ST_Service}) guarantee the minimum service level.
%These chance-constraints ensure that the probability of a stock out is less than (1-$\alpha_{c}$). \\


%\subsection{Dynamic programming model}


%In this section, we present the dynamic programming model for the stochastic lot sizing problem with a firm-driven substitution and joint service level constraint. Different sets, parameters and decision variables are presented in Table~\ref{tab:Sub_parameters}. Considering the underlying scenario tree, the mathematical model \ref{DynamicProgramming} is proposed for the node {\it n} in the tree. This model is based on the inventory position at the beginning and ending of the planning period correspond to current node {\it n}. At each stage $\hat{\Vi}_{p(n),{\ka}}$ and $\hat{\Bi}_{p(n),{\ka}}$ illustrate the state of the system for the current node. Considering $D_{\ka\ti}$ as the random variable for product $\ka$ in period $\ti$, $\hat{D}_{n \ka}$ is its realization at node $n$. Considering the scenario tree, $D^{\ti \ka}$ is the history of  random demand from period 1 to period $\ti$ and $\hat{D}^{nk}$ is its realization until node $n$.  

%In this section, we present the dynamic programming formulation for the finite-horizon stochastic lot sizing problem with a firm-driven substitution and joint service level constraint over all products. 
We model the underlying stochastic process of the demand as a scenario tree, for the finite model with $\Ti$ stages. There are $N$ nodes in the tree, represented by the set $\mathcal{N} = \{1, ..., N\}$, where node 1 is the root node. 
 Each node in the tree has a parent node, $p(n)$, a set of child nodes, $\cn$, and $q_{nm}$ is the transition probability from node $n$ to its child node $m$. $\ti(n)$ is the time period/stage corresponding to node $n$. 
$D_{\ti \ka}$ is the random demand variable for product $\ka$ in period $\ti$, whereas $\hat{D}_{n \ka}$ denotes its realization at node $n$.
$D^\text{Hist}_{\ti \ka}$ represents the random demand path from period 1 to period $\ti$ for product $\ka$, and $\hat{D}^\text{Hist}_{n\ka}$ denotes its realization (the history) until node $n$.


We next present our proposed multi-stage stochastic programming model with chance-constraint for the finite-horizon variant of the considered lot sizing problem with substitution.
Notation for different sets, parameters and node-based versions of the decision variables are presented in Table~\ref{tab:Sub_parameters}. 
We present a node-based formulation in a dynamic programming fashion where $F_n(.)$ and $\overline{F}_n(.)$ denote the cost-to-go and expected cost-to-go functions at node $n$, respectively. 
%the mathematical model (\ref{DynamicProgrammingNoden}) is proposed for each node $n \in \mathcal{N} \setminus \{1\}$ in the tree.
%This model is based on the \InvPos and backlog at the beginning and end of the planning period corresponding to the current node $n$.
 For each node $n \in \mathcal{N}$, the cost-to-go is obtained via ?????:
%We formulate the multi-stage stochastic lot sizing problem with substitution as the following dynamic program model:


\begin{subequations}
\label{DynamicProgrammingNoden}
\begin{flalign}
F_{n}(\Vi_{p(n)}, \Bi_{p(n)}) =&  \min && \sum_{\ka \in \KA} \left( c^{\text{setup}}_{\ti(n) \ka}\y_{n \ka} + c^{\text{prod}}_{\ti(n) \ka}\x_{n \ka} + c^{\text{hold}}_{\ti(n) \ka} {I}_{n \ka}+  \sum_{\jey \in \Csub}c^{\text{sub}}_{\ti(n) \ka \jey} \Es_{n  \ka \jey}\right) + \overline{F}_{n}(\Vi_{n}, \Bi_{n}) &&
\label{eq:Dyn_g2_Sub_ST_Production_Flow}& \\
&\text{s.t.} && \x_{n \ka} \leq M_{n \ka} \y_{n \ka} \quad\quad  \forall \ka  \in \KA & & \label{eq:Dyn_Sub_Setup}&\\
& && \sum_{\jey \in  \Psub} {S}_{n \jey \ka} + \Bi_{n  \ka}  = \hat{D}_{n \ka} + {\Bi}_{p(n) \ka} \quad\quad \forall \ka  \in \KA&& \label{eq:Bhat}& \\
& &&\sum_{\jey \in  \Csub} {S}_{n \ka \jey} + I_{n \ka} = {\Vi}_{p(n)\ka}  \quad \forall \ka  \in \KA& \label{eq:vhat}&\\
& &&\Vi_{n \ka} = I_{n  \ka} + \x_{n  \ka}  \quad\quad \forall \ka  \in \KA  &\label{eq:vdef}&
%& v^-_{n \ka} = \Bi_{n  \ka}  \quad \forall \ka  \in \KA \\
\\
& &&\mathbb{P}_{D_{\ti(n)+1}}\{ ({\Vi}_{n}, {\Bi}_{n} ) \in Q(D_{\ti(n)+1} )| D^\text{Hist}_{\ti(n)} = \hat{D}^\text{Hist}_{n} \} \geq \alpha &&\label{eq:SL}&
\\*[0.2cm]
%%%%%%%%%%%%%%%%%%%%%%%
& &&{x}_{ n },  {v}_{ n },  {I}_{ n } , {\Bi}_{ n } \in \mathbb{R}_{+}^{\Ka} , {S}_{n} \in \mathbb{R}_{+}^{|\Graf|} ,{y}_{ n } \in \{0,1\}^{\Ka}&& \label{eq:Dyn_F_Sub_ST_bound1}&
%\\
%&{S}_{n \ka \jey} \geq 0 & \forall(k,j) \in \Graf & \label{eq:Dyn_F_Sub_ST_bound3}&
%\\
%& (I_{\cdot n},\Bi_{\cdot n},\Es_{\cdot n}) \in %\mathcal{X}^{\text{rest}}_\ell &&\label{eq:Dyn_F_Sub_ST_bound4}&
\end{flalign}
\end{subequations}


%Here, $\Vi_{n{\ka}} =$ current inventory position of product $\ka$. \\
%Note: $I_{n{\ka}}$: the ending inventory of node $n$ for product $\ka$\\

whereas, the expected cost-to-go function is defined as follows:
%equation (\ref{eq:cost-to-go-exp}).
\begin{align}
&\overline{F}_{n}(\Vi_{n}, \Bi_{n}) =  \left\{ 
  \begin{array}{ c l }
    0 & \quad \textrm{if } \mathcal{C}(n) = \emptyset \\
    \sum\limits_{m \in \cn}  q_{nm}F_{m}(\Vi_{n}, {\Bi_{n}}  )                 & \quad \textrm{otherwise}
  \end{array}
\right.&  
\label{eq:cost-to-go-exp} 
\end{align}


The objective function of the model at node $n$ shown as (\ref{eq:Dyn_g2_Sub_ST_Production_Flow}), in which $F_{n}({\Vi}_p(n),{\Bi}_p(n)) $ represents the expected optimal cost of decisions from node $n$ in the scenario tree to the end of the horizon given the initial inventory level vector and backlog vector. More specifically it minimizes the current stage total cost plus the expected cost-to-go function, This which includes the total setup cost, production cost, holding cost, and substitution cost.  $\Vi_n$ and $\Bi_n$ are shorthand for the vectors $\{ v_{nk}, \ka \in \KA\}$ and $\{ B_{nk}, \ka \in \KA\}$, respectively.

Constraints (\ref{eq:Dyn_Sub_Setup}) are the set up constraints which guarantee that when there is production, the setup variable is forced to take the value 1. 
Constraints (\ref{eq:Bhat}) show that the demand of each product is satisfied by its own production and the substitution by other products or it will be backlogged to the next period.
%In this constraint, $\Es_{\ka \ka n}$ is equal to the amount which is allocated to satisfy the demand of product $k$ from its own production. 
Constraints (\ref{eq:vhat}) show that the inventory of product $\ka$ at the beginning of the current period may be used to satisfy its own demand or other products demand through substitution or it will be stored as inventory for future periods. 
Constraints (\ref{eq:vdef}) define the \InvPos at the end of the current period which is equal to the amount of inventory (immediately after demand satisfaction) plus the amount of production during the current period.
%Constraints (\ref{eq:vdef}) are to define the inventory position at the end of current period which is equal to the amount inventory plus the amount of production for the future periods.

Constraint~(\ref{eq:SL}) is to ensure the joint service level for period $\ti(n)+1$ which is modeled as a chance-constraint. In this constraint, $Q(D_ {\ti+1})$ is the set of \InvPos and backlog quantities in which customer demands given by $D_{\ti+1}$ can all be met and there is no stock out for any of the products.
\begin{align} Q(D_ {\ti +1}) := \{ (v_n,B_n) \in \mathbb{R}_{+}^{2\ka} \in  :  \exists \overline{\Es} \in \mathbb{R}_{+}^{|\Graf|} \in \geq 0 \ \text{s.t.} \ 
 & \sum_{\jey \in  \Psub} \overline{\Es}_{\jey \ka } = D_{\ti+1,\ka} + \Bi_{n\ka} \ \forall \ka  \in \KA  \quad \textit{and }\notag \\
 & \sum_{\jey \in  \Csub} \overline{\Es}_{\ka \jey} \leq \Vi_{n\ka} \ \forall \ka  \in \KA \} & \label{eq:SetQ}
 \end{align}
  The service level constraint guarantees that the probability of having a feasible solution with respect to the service level is greater than or equal to $\alpha$. 
 This probability is defined over the demand distribution until period $\ti(n)+1$, having that part of the demand history until node $n$ is realized and known. Lastly, constraints~(\ref{eq:Dyn_F_Sub_ST_bound1}) define the domains of different variables in the model.
In addition to these constraints it is possible to add different types of constraints such as capacity constraints to the model.
%For example, if the company decide to satisfy a percentage of each product expected demand  by its own production, the following constraint can be added to the model.

Considering the infinite-horizon, at each period $\tAct$, our main goal is to compute $F_{1}(.)$, which is the cost-to-go function for the root node $1 \in \mathcal{N}$. In this case, $\Vi_{p(n)}, \Bi_{p(n)}$ are equal to $\hat{\Vi},\hat{\Bi}$ which indicate the vectors for the initial state of the system, and the service level constraint is defined for the second period in finite-horizon model (eq (\ref{eq:SL_Root})).



%\begin{subequations}
%\label{DynamicProgramming}
%\begin{flalign}
%&F_{1}(\hat{\Vi},\hat{\Bi}) =  \min  \sum_{\ka \in \KA} \left( c^{\text{setup}}_{\ti(1) \ka}\y_{1 \ka} + c^{\text{prod}}_{\ti(1) \ka}\x_{1 \ka} + c^{\text{hold}}_{\ti(1) \ka} {I}_{1 \ka}+  \sum_{\jey \in \Csub}c^{\text{sub}}_{\ti(1) \ka \jey} \Es_{1  \ka \jey}\right) + \overline{F}_{n}(\Vi_{1}, \Bi_{1}) & \label{eq:Dyn_Root} 
%\end{flalign}

\begin{flalign}
%\text{s.t.} \ & \x_{1 \ka} \leq M_{1 \ka} \y_{1 \ka} \quad  &\forall \ka  \in \KA  & \label{eq:Dyn_Sub_Setup_Root}\\
%& \sum_{\jey \in  \Psub} {S}_{1 \jey \ka} + \Bi_{1  \ka}  = \hat{D}_{1 \ka} + \hat{\Bi}_{ \ka} \quad &\forall \ka  \in \KA \label{eq:Bhat_Root} \\
%& \sum_{\jey \in  \Csub} {S}_{1 \ka \jey} + I_{1 \ka} = \hat{\Vi}_{\ka}  \quad &\forall \ka  \in \KA \label{eq:vhat_Root}\\
%& \Vi_{1 \ka} = I_{1  \ka} + \x_{1  \ka}  \quad &\forall \ka  \in \KA  \label{eq:vdef_Root}
%& v^-_{n \ka} = \Bi_{n  \ka}  \quad \forall \ka  \in \KA \\
& \mathbb{P}_{D_2}\{ ({\Vi}_{1}, {\Bi}_{1} ) \in Q(D_2 )| D^\text{Hist}_{1} = \hat{D}^\text{Hist}_{1} \} \geq \alpha& \label{eq:SL_Root}
&
%\\*[0.5cm]
%%%%%%%%%%%%%%%%%%%%%%%
%& {x}_{ 1 },  {v}_{ 1 },  {I}_{ 1 } , {\Bi}_{ 1 } \in \mathbb{R}_{+}^{\Ka} , {S}_{1} \in \mathbb{R}_{+}^{|\Graf|} ,{y}_{ 1 } \in \{0,1\}^{\Ka} &  & \label{eq:Dyn_F_Sub_ST_bound1_Root}
%\\
%&{S}_{n \ka \jey} \geq 0 & \forall(k,j) \in \Graf & \label{eq:Dyn_F_Sub_ST_bound3}&
%\\
%& (I_{\cdot n},\Bi_{\cdot n},\Es_{\cdot n}) \in %\mathcal{X}^{\text{rest}}_\ell &&\label{eq:Dyn_F_Sub_ST_bound4}&
\end{flalign}
%\end{subequations}


It should be noted that in this model, the feasibility of the next stage service level should be guaranteed. This can be satisfied if we have at least one uncapacitated production option for each product whether by its own production or substitution. As there is no capacity constraint in our model, this feasibility is guaranteed.
%If at each stage, there exist an unlimited source for production of each of the products or some of the products that can substitute others this feasibility condition is satisfied.


{\cred Note: I just keep this red part in case we need to have mathematical proof

In this model, it is not explicit that the recourse stage should be feasible. It is using the convention that if it is infeasible, the value function return infinite cost.}


{\cred * Given stage t and the history, $\xi^t, \exists x^t(\xi^t) \in \chi^t(x^t) \rightarrow v(\xi^t)$\\
$s.t.$ \\
$P_{\xi^{t+1}|\xi^t}\{v^{t}(\xi^t)-\sum_{\jey \in  \Csub}\Es_{kj}(\xi^{t+1})+\sum_{\jey \in  \Psub}\Es_{jk}(\xi^{t+1}) \geq D^{t+1}(\xi^{t+1}) $ for some $\Es_{kj}(\xi^{t+1}) , \Es_{jk}(\xi^{t+1}) \in  \chi ^{t+1}(\xi^{t+1}) \} \geq \alpha$ \\
This can be satisfied for instance if we have at least one uncapacitated product option (whether by its own production or substitution) for each product.

}


\section{Decision policies}

In this section, we will explain how to make different decisions at each stage using different policies. These policies map the state of the system to different decisions and can be used in a rolling-horizon framework. We propose policies guided by our proposed multi-stage stochastic programming model. 
%In this section we propose different policies to approximate the dynamic programming model proposed in the previous section. %These policies which are approximations for the proposed dynamic programming model (not sure), are MIP based. 
More specifically, we define two policies to be applied in sequence at each period, inspired by the dynamics of the decision-making process, where two groups of decisions are made as shown in Figure~\ref{MultistageDynamics}. The first policy, which we call the ``current stage modification policy", is focused on the first group of decisions, namely substitution, inventory and backlog, only for the current period, to satisfy the current-stage demand. {\cred(or to guarantee the feasibility of the current-stage demand satisfaction.)} 
Using this policy we will define which product observed demand has the potential to be fully satisfied without any backlog. Based on the result of this policy, and considering the stochasticity in future periods, the second policy, which call the ``production policy" decides on the setup, production, and substitution decisions such that the service level in the next period can be satisfied. The mathematical model in this policy is our "main" model which consider the whole $\TI$ planning periods, and the result of ``current stage modification policy" will be added to this model as extra constraints for the first planning period.
The decisions of the current period, suggested by the ``production policy" are implemented, the state of the system is updated based on the observed demand, and this process is repeated as we move forward in the rolling-horizon framework. These policies are explained in detail in the following sections. Being at period $\tAct$ as the actual time, for the $\Ti$ planning periods for the finite-horizon models, the actual time indices $(\tAct, ..., \tAct +\Ti -1) $ are mapped into $(1, ..., \Ti)$ for convenience. Additional notation used in the policy models, are presented in Table~\ref{tab:Sub_Policy_parameters}. 

\begin{table}[H]
\centering
\caption{Notation for new parameters and decision variables for decision policies at stage $\tAct$}
\begin{adjustbox}{width=1\textwidth,center=\textwidth}
\begin{tabular}{ll}
\toprule
{\bf Parameters} & {\bf Definition} \\ \midrule
$\hat{B}_{\tAct-1, \ka}$  & The amount of initial backlog for product $\ka$ (current state)\\
$\hat{v}_{\tAct-1, \ka}$  & The amount of initial inventory level for product $\ka$ (current state) \\
$\mathbb{Q}_{i}({D}_{\ti \ka})$& The $i^{th}$ percentile of the demand over all scenarios for product $\ka$ in period $\ti$\\
\midrule
{\bf Decision variables} & {\bf Definition} \\ \midrule
$\y_{\ti \ka}$ & Binary variable which is equal to 1 if there is a setup for product $\ka$ at period $\ti$, 0 otherwise \\ 
$\x_{\ti \ka}$ & Amount of production for product $\ka$  at period $\ti$  \\ 
$\Es_{\ti \ka \jey}$ & Amount of product $\ka$  used to fulfill the demand of product $\jey$  at period $\ti$   \\
${I}_{\ti \ka}$ & Amount of physical inventory for product $k$ after the demand satisfaction at period $\ti$   \\
${\Bi}_{\ti \ka}$ & Amount of backlog for product $k$ at the end of period at period $\ti$   \\
${\Vi}_{\ti \ka}$ & The \InvPos for product $\ka$ at the end of period at period $\ti$ \\ \bottomrule
\end{tabular}
\end{adjustbox}
 \label{tab:Sub_Policy_parameters}
\end{table}


 \subsection{Current stage modification policy}

%In each period, the production amounts consider the future periods, where there exists stochasticity in demand. 
This policy is a prerequisite for the ``production policy" in which, the decisions in the current period are made based on different cost parameters, and they guarantee the service level in the next stage. The``production policy" does not impose any constraints to minimize the amount of backorder and satisfy the service level in the current period as much as possible. 
To overcome this challenge, in each period, first we apply the ``current stage modification policy". In this policy, we solve the linear programming (LP) model~(\ref{Currentstage}) which minimizes the backlog in the current period.
{\cred This model is a feasibility problem and based on the result of this model, if it is possible to fully satisfy a product demand without any backlog, we force the backlog in the current period equal to zero in ``production policy" model.} It should be noted that the ``current stage modification policy" does not define how the demand should be satisfied, and this will be later defined in the ``production policy".
(question: The $\ti$ index should be 1 or $\tAct$)


\begin{subequations}
\label{Currentstage}

\begin{flalign}
&\min  \sum_{\ka \in \KA}  {B}_{ 1 \ka} & \label{eq:Current_obj} 
\end{flalign}
 subject to:
\begin{flalign}
  &  \sum_{\jey \in  \Psub} {S}_{1  \jey \ka} + B_{1 \ka}  = \hat{D}_{1 \ka} + \hat{B}_{\tAct-1, \ka} &\forall \ka \in \KA  &     \label{eq:Current_inventory_tn}&\\
&  \sum_{\jey \in  \Csub} {S}_{ \ka \jey} + I_{ 1 \ka} = \hat{v}_{ \tAct-1, \ka} &\forall \ka \in \KA  &     \label{eq:Current_Orderup}&\\
& {I}_{ 1 } \in \mathbb{R}_{+}^{\Ka} , {B}_{ 1 } \in \mathbb{R}_{+}^{\Ka} , {S}_{1} \in \mathbb{R}_{+}^{|\Graf|} &    & \label{eq:Current_bound2}
\end{flalign}
\end{subequations}
The objective function (\ref{eq:Current_obj}) is to minimize the backlog. Constraints (\ref{eq:Current_inventory_tn}) guarantee that the demand and backlog will be satisfied in the current period or it will be backlogged. Constraints (\ref{eq:Current_Orderup}) show that the available inventory is either used to satisfy the demand of different products, or will be stored as an inventory.
If the optimal value of  $B_{ 1 \ka}$  is equal to zero, the constraint $B_{ 1 \ka} = 0$ will be added to the MIP main model of the production policy. How the demand of this product is satisfied is defined in the main model based on different cost parameters.

It should be noted that it is also possible to use other policy models in the ``current stage modification policy". The advantage of the current policy model is that its result can be easily applied to all the ``production policies" which will be explained in the next section.

\subsection{The production policy}
At each period $\tAct$, based on the current state of the system we apply a ``production policy", which take $\hat{v}_{\tAct-1}, \hat{B}_{\tAct-1 }$, and $\hat{D}_1$ as inputs. This policy, approximates the dynamic programming model (\ref{DynamicProgrammingNoden}) at the root node, which is challenging to solve {\cred due to its complexity and recursive nature.}
%(\ref{DynamicProgrammingNoden}).
%Appr$(\tAct,\hat{v}_{\tAct-1}, \hat{B}_{\tAct-1 })$, which is an approximation for the dynamic programming model at the root node.
%(\ref{DynamicProgramming}).
In this section, we will explain two deterministic policies (average and quantile policies), and a policy base on the chance-constraint (chance-constraint policy). We will later show that while the deterministic policies have the advantage of faster execution time, the chance-constraint policy result in more accurate solutions.
%This policy can be a deterministic policy or any other approximation policies. 
 
\subsubsection{Deterministic policies}

%To evaluate the value of the  stochastic  model, we can compare it against the deterministic models applied in a rolling-horizon environment.
In these policies, we represent the future demand by a single scenario, and we propose two different deterministic policies based on that. 
These policies will approximate the dynamic programming model at the root node
%model~(\ref{DynamicProgramming}) 
by eliminating the chance-constraint and substituting the stochastic demand  with the deterministic value, and solve the resulting lot sizing model. In the first policy, namely the ``average policy", the stochastic demand for all the products and in all periods is substituted by its expected value. In the second policy which is called the "quantile policy" the stochastic demand for the next immediate period is substituted by the quantile of the future demand distribution, which is defined based on the service level. We also assume that there is no backlog for $\ti > 1$, which means that we should at least satisfy the expected demand in those periods.
%This model can be solved in a receding horizon environment, in which the first period/stage decision is fixed and the information such as initial inventory and backlog will be updated based on the realized demand. This procedure repeated until the last period. 
%We will next present two deterministic policies.
Model (\ref{mod:Det}) represents the ``average policy".


 
\begin{subequations}
\label{mod:Det}

\begin{flalign}
&\min \sum_{\ti \in \TI} \sum_{\ka \in \KA} \left( c^{\text{setup}}_{\ti \ka}y_{\ti \ka} + c^{\text{prod}}_{\ti \ka}x_{\ti \ka}+ c^{\text{hold}}_{\ti \ka}I_{\ti \ka} +\sum_{\jey \in  \Csub}c^{\text{sub}}_{\ti \ka \jey} S_{\ti \ka \jey}  \right) 
%+ bc_{\ti \ka} {B}_{\ti \ka} )
& \label{eq:Sub_Det_obj} 
\end{flalign}
 subject to:
\begin{flalign}
&x_{\ti \ka} \leq M_{\ti \ka} y_{\ti \ka} &  \forall \ti  \in \TI, \forall \ka \in \KA & \label{eq:Sub_Det_Setup}\\
  &  \sum_{\jey \in  \Psub} {S}_{1  \jey \ka} + B_{1 \ka}  = \hat{D}_{1 \ka} + \hat{B}_{\tAct-1, \ka} &\forall \ka\in \KA  &     \label{eq:Det_inventory_1}&\\
   &  \sum_{\jey \in  \Psub} {S}_{\ti \jey \ka} + B_{\ti \ka} = \mathbb{E}_{|n}[{D}_{\ti \ka}] + {B}_{\ti-1 , \ka} &\forall \ti \in \TI \setminus\{1\},\forall \ka\in \KA  &     \label{eq:Det_inventory_2}& \\
&  \sum_{\jey \in  \Csub} {S}_{1 \ka \jey} + I_{ 1 \ka} = \hat{\Vi}_{\tAct-1 , \ka} &\forall \ka \in \KA  &     \label{eq:Det_inventory_3}&\\
&  \sum_{\jey \in  \Csub} {S}_{\ti \ka \jey} + I_{\ti \ka} = \Vi_{\ti-1 , \ka} &\forall \ti  \in \TI \setminus\{1\},\forall \ka \in \KA  &     \label{eq:Det_inventory_4}&\\
%&  I_{ \ti(n) \ka} - B_{ \ti(n) \ka} = \hat{v}_{\ti(n)-1 , \ka} - \sum_{\jey \in  \Csub} {S}_{\ti(n) \ka \jey} + \sum_{\jey \in  \Psub} {S}_{\ti(n)  \jey \ka}  - \hat{D}_{n \ka} \quad &\forall \ka \in \KA  &     \label{eq:Det_inventory_tn}&\\
%& v_{ \ti(n) \ka} = I_{ \ti(n) \ka} + x_{ \ti(n) \ka}  \quad &\forall \ka \in \KA  &     \label{eq:Det_inventoryPos_tn}& \\
%&  I_{\ti \ka} - B_{\ti \ka} = v_{\ti-1 , \ka} - \sum_{\jey \in  \Csub} {S}_{\ti \ka \jey} + \sum_{\jey \in  \Psub} {S}_{\ti \jey \ka}  - E_{|n}[\overline{D}_{\ti \ka}] \quad &\forall t \ \in T, t > t_{n},\forall \ka \in \KA  &     \label{eq:Dyn_F_Sub_ST_Service}&\\
& v_{\ti \ka} = I_{\ti \ka} + x_{\ti \ka}  \quad &\forall t \in T,\forall \ka \in \KA & \label{eq:Det_inventory_5} \\
%&\sum_{\ka \in \KA} (st_{\ti \ka}y_{\ti \ka} + pt_{\ti \ka}x_{\ti \ka}) \leq Cap_{t} & \forall t  \in T ,  t\geq \ti(n)   & \label{eq:Sub_FD_Capacity}  \\
&\Bi_{\ti \ka} =0 \text{we can eliminate this constraint by totally eliminate B in 6}& \forall \ti \in \TI \setminus\{1\}, \forall \ka \in \KA &\label{eq:Det_Back}\\
&  {x}_{ \ti },  {v}_{ \ti },  {I}_{ \ti} , {\Bi}_{ \ti } \in \mathbb{R}_{+}^{\Ka} , {S}_{\ti} \in \mathbb{R}_{+}^{|\Graf|} ,{y}_{ n } \in \{0,1\}^{\Ka} &    & \label{eq:Sub_FD_bound2}
\end{flalign}
where
{\cred{
\begin{flalign}
  &  M_{\ti \ka} = \hat{B}_{\tAct-1, \ka}+ \sum_{\jey \in  \Csub}( \sum_{t = 1}{D}_{kt} + \sum_{\ti >1}E_{|n}[\overline{D}_{\ti \ka}])  &\ti> 1 ,\forall \ka \in \KA  &  
  %\label{eq:BigM_Deterministic_Appr}
  \notag
  \end{flalign}}}
\end{subequations}

The objective function (\ref{eq:Sub_Det_obj}) is to minimizes the total cost of setup, production, holding and substitution cost. Constraints (\ref{eq:Sub_Det_Setup}) guarantee that in each planning period, when there is a production, there will be a setup. Constraints (\ref{eq:Det_inventory_1}) to (\ref{eq:Det_inventory_4}) are the inventory, backlog, and substitution balance constraints which are defined for $\ti=1$, and $\ti > 1$, separately. Constraints (\ref{eq:Det_inventory_5}), define the available inventory after production. Constraints (\ref{eq:Det_Back}) show that the backlog is equal to zero for $\ti > 1 $ which guarantee the average demand satisfaction in those periods.

The quantile policy model is the same as model (\ref{mod:Det}), except for one set of constraints. In this policy, for $\ti =2$ constraints (\ref{eq:Det_inventory_2}) are replaced by constraints (\ref{eq:Quan_inventory_2}). The difference between the average policy and the quantile policy is that in the second period, where we need to consider the service level, the average demand will be replaced by the $i^{th}$ percentile of the demand. This percentile is defined by the service level $\alpha$. {\cred(Can I write  $\alpha$ percentile ??)}

\begin{flalign}
     \sum_{\jey \in  \Psub} {S}_{\ti \jey \ka} + B_{\ti \ka} = \mathbb{Q}_{i}({D}_{\ti \ka}) + {B}_{\ti-1 , \ka} & \ti = 2,\forall \ka \in \KA & \label{eq:Quan_inventory_2} 
\end{flalign}

\subsubsection{Chance-constraint (Two-stage) policy}
The deterministic policies that we explained in the previous section did not consider the service level constraint explicitly in their models. In the chance-constraint policy, being at period $\tAct$ in the infinite time horizon or ($\ti =1$) in the finite horizon model, we consider all the child nodes in the scenario tree for period $\ti =2$ and replace the stochastic demand with its average in the following periods. 

\begin{subequations}
\label{mod:ExtnSL}

\begin{flalign}
\min &
\sum_{\ka \in \KA} ( c^{\text{setup}}_{ \ti(n) \ka}y_{ \ti(n) \ka} + c^{\text{prod}}_{ \ti(n) \ka}x_{ \ti(n) \ka}+ \sum_{\jey \in  \Csub}c^{\text{sub}}_{\ti(n) \ka \jey} S_{\ti(n) \ka \jey}  + c^{\text{hold}}_{ \ti(n) \ka}I_{ \ti(n) \ka} + bc_{ \ti(n) \ka} {B}_{ \ti(n) \ka} ) + \notag \\
&\sum_{\ka \in \KA} ( c^{\text{setup}}_{ \ti(n) \ka}y_{\ti(n)+1 , \ka} + c^{\text{prod}}_{ \ti(n) \ka}x_{\ti(n)+1 , \ka}+  c^{\text{hold}}_{\ti(n)+1 , \ka}I_{\ti(n)+1 , \ka} + bc_{\ti(n)+1 , \ka} {B}_{\ti(n)+1 , \ka} + \frac{1}{|M|} \sum_{m \in M} \sum_{\jey \in  \Csub} c^{\text{sub}}_{\ti(n)+1,\ka,\jey} S_{\ti(n)+1,\ka,\jey,m} ) + \notag \\
& \sum_{t =\ti(n)+2 }^{T} \sum_{\ka \in \KA} ( c^{\text{setup}}_{\ti \ka}y_{\ti \ka} + c^{\text{prod}}_{\ti \ka}x_{\ti \ka}+ \sum_{\jey \in  \Csub}c^{\text{sub}}_{\ti \ka \jey} S_{\ti \ka \jey}  + c^{\text{hold}}_{\ti \ka}I_{\ti \ka} + bc_{\ti \ka} {B}_{\ti \ka} ) & \label{eq:Sub_Roll_obj_ext} 
\end{flalign}
 subject to:
\begin{flalign}
&x_{\ti \ka} \leq M_{\ti \ka} y_{\ti \ka} &  \forall t  \in T , t \geq \ti(n)  ,\forall \ka \in \KA & \label{eq:Sub_FD_Setup}\\
  &  \sum_{\jey \in  \Psub} {S}_{\ti(n)  \jey \ka} + B_{\ti(n) \ka}  = \hat{D}_{n \ka} + \hat{B}_{\ti(n)-1, \ka} &\forall \ka \in \KA  &     \label{eq:Det_backorder_tn}&\\
  &  \sum_{\jey \in  \Csub} {S}_{\ti(n) \ka \jey} + I_{ \ti(n) \ka} = \hat{v}_{\ti(n)-1 , \ka} &\forall \ka \in \KA  &     \label{eq:Det_inventory_tn}&\\
  &  \sum_{\jey \in  \Psub} {S}_{\ti(n)+1,\jey,\ka,m} + B'_{\ti(n)+1 , \ka,m}  = d_{km} + {B}_{\ti(n) , \ka} &\forall \ka \in \KA, \forall m \in M &     \label{eq:Det_backorder_tnp}& \\
&  \sum_{\jey \in  \Csub} {S}_{\ti(n)+1,\ka,\jey,m} + I'_{\ti(n)+1 , \ka,m} = v_{\ti(n) , \ka} &\forall \ka \in \KA, \forall m \in M  &     \label{eq:Det_inventory_tnp}&\\
   &  \sum_{\jey \in  \Psub} {S}_{\ti \jey \ka} + B_{k t}  = \mathbb{E}_{|n}[{D}_{\ti \ka}] + {B}_{\ti -1 ,\ka} &\forall t \ \in T, t > t_{n}+1,\forall \ka \in \KA  &   \label{eq:Det_backorder_ext}& \\
&  \sum_{\jey \in  \Csub} {S}_{\ti \ka \jey} + I_{\ti \ka} = v_{\ti-1 , \ka} &\forall t \ \in T, t > t_{n}+1,\forall \ka \in \KA  &     \label{eq:Det_inventory_ext}&\\
& \frac{1}{|M|} \sum_{m \in M} I'_{\ti(n)+1,k,m} = I_{\ti(n)+1, \ka} &\forall \ka \in \KA & \label{eq:Average_Inventory} \\
& \frac{1}{|M|} \sum_{m \in M} B'_{\ti(n)+1,k,m} = B_{\ti(n)+1, \ka} &\forall \ka \in \KA & \label{eq:Average_Backlog}\\
%&  I_{ \ti(n) \ka} - B_{ \ti(n) \ka} = \hat{v}_{\ti(n)-1 , \ka} - \sum_{\jey \in  \Csub} {S}_{\ti(n) \ka \jey} + \sum_{\jey \in  \Psub} {S}_{\ti(n)  \jey \ka}  - \hat{D}_{n \ka} \quad &\forall \ka \in \KA  &     \label{eq:Det_inventory_tn}&\\
%& v_{ \ti(n) \ka} = I_{ \ti(n) \ka} + x_{ \ti(n) \ka}  \quad &\forall \ka \in \KA  &     \label{eq:Det_inventoryPos_tn}& \\
%&  I_{\ti \ka} - B_{\ti \ka} = v_{\ti-1 , \ka} - \sum_{\jey \in  \Csub} {S}_{\ti \ka \jey} + \sum_{\jey \in  \Psub} {S}_{\ti \jey \ka}  - E_{|n}[\overline{D}_{\ti \ka}] \quad &\forall t \ \in T, t > t_{n},\forall \ka \in \KA  &     \label{eq:Dyn_F_Sub_ST_Service}&\\
& v_{\ti \ka} = I_{\ti \ka} + x_{\ti \ka}  \quad &\forall t \in T, t \geq \ti(n),\forall \ka \in \KA \\
%&\sum_{\ka \in \KA} (st_{\ti \ka}y_{\ti \ka} + pt_{\ti \ka}x_{\ti \ka}) \leq Cap_{t} & \forall t  \in T ,  t\geq \ti(n)   & \label{eq:Sub_FD_Capacity}  \\
%&y_{\ti \ka} \in \{0, 1\} & \forall t  \in T,  t \geq \ti(n) ,\forall\ka \in \KA &\label{eq:Sub_FD_base_bin}\\
%&x_{\ti \ka}  \geq 0 &  \forall t  \in T, t\geq \ti(n),\forall \ka \in \KA  & \label{eq:Sub_FD_bound1}\\
%& I_{\ti \ka} , B_{\ti \ka} \geq 0 &  \forall t  \in T,  t \geq \ti(n) ,\forall \ka \in \KA  & \label{eq:Sub_FD_bound2}\\
%&S_{\ti \ka \jey} \geq 0 &  \forall t  \in T,  t \geq \ti(n) , \forall(k,j) \in \Graf & \label{eq:Sub_FD_bound3}
&{x}_{ \ti },  {v}_{ \ti },  {I}_{ \ti} , {\Bi}_{ \ti } \in \mathbb{R}_{+}^{\Ka} , {S}_{\ti} \in \mathbb{R}_{+}^{|\Graf|} ,{y}_{ n } \in \{0,1\}^{\Ka} & \label{eq:Sub_FD_bound3}
\end{flalign}

  \end{subequations}
  
 To have a more clear description, the objective function of the extensive form (\ref{eq:Sub_Roll_obj_ext}) has broken into three parts, the cost of current period $\ti(n)$, the cost of period $\ti(n)+1$, and the cost of periods $\ti(n)+2$ to $\TI$. Constraints (\ref{eq:Det_backorder_tn}) to (\ref{eq:Det_inventory_ext}) are the inventory, backlog, and substitution balance constraints. Constraints (\ref{eq:Det_backorder_tn}) and (\ref{eq:Det_inventory_tn}) are for $\ti(n)$ period, constraints (\ref{eq:Det_backorder_tnp}) and (\ref{eq:Det_inventory_tnp}) are for $\ti(n)+1$ period, and constraint (\ref{eq:Det_backorder_ext}) and (\ref{eq:Det_inventory_ext}) are for the periods $\ti(n)+2$ to $\Ti$. Constraints (\ref{eq:Average_Inventory}) and (\ref{eq:Average_Backlog}), define the average inventory and backlog for period $\ti(n)+1$, which is equal to the average inventory and backlog over all the child nodes at this period. These averages will be used in the inventory and backlog balance constraints in period $\ti(n)+2$.\\

 {\cred
\large{The next section will be eliminated, I just keep it for now in case there was a discussion for the appendix}

  \subsubsection{Benders cuts to project out the second stage substitution variables}
  
  
 To reduce the size of the master problem, we use Benders cuts to eliminate the substitution variables $S_{\ti(n)+1,\jey,\ka,m}$ in the next immediate stage. For each $m \in M$, introduce a variable $\theta_m$ to represent the substitution cost in child node $m \in M$. Then, the term 
 \[ \sum_{\ka \in \KA} \frac{1}{|M|} \sum_{m \in M} \sum_{\jey \in  \Csub} S_{k,j,\ti(n)+1,m} \]
 in the objective is replaced by the term:
 \[ \frac{1}{|M|} \sum_{m \in M} \theta_m. \]

The constraints \eqref{eq:Det_backorder_tnp} and \eqref{eq:Det_inventory_tnp} are eliminated from the master model. 

Given a master problem solution, say with values $\overline{B}'_{\ti(n)+1 , \ka,m}$,$\bar{I}'_{\ti(n)+1 , \ka,m}$,$\bar{B}_{ \ti(n) \ka}$, $\bar{v}_{\ti(n) , \ka}$, $\bar{\theta}_m$ we solve the following subproblem for each $m \in M$:

}

    
  \subsubsection{chance-constraint policy}
  Another approach to approximate this problem is to enforce the service level directly in the next planning period. The MIP model for this problem is as follows in which the constraint (\ref{eq:SL_Root}) is represented as constraints (\ref{eq:Backorder_Child} - \ref{eq:Child_Service}).
  %, in which $M'_{km}$ is defined by (\ref{eq:BigM_Child}).
  The $Bc_{km}$ represents the backlog at the child node $m$.
  \begin{flalign}
  & Bc_{km} +\sum_{\jey \in  \Psub} c^{\text{setup}}_{\jey \ka m} = D_{km}  +\Bi_{n \ka}  & \forall m \in M, \forall \ka  \in \KA& \label{eq:Backorder_Child}\\
  & \sum_{\jey \in  \Csub} c^{\text{setup}}_{\ka \jey m} \leq \Vi_{n \ka}   & \forall m \in M, \forall \ka  \in \KA& \label{eq:OrderUptoLevel_Child}\\
&  Bc_{km} \leq M'_{km}\Zed_{ m}  & \forall m \in M , \forall \ka  \in \KA &     \label{eq:Child_Service}
 \end{flalign}
 where
 \begin{flalign}
 &  M'_{km}=  \hat{\Bi}_{\ti(n)-1 , \ka} +d_{n \ka} + D_{km} & \forall m \in M , \forall \ka  \in \KA &     \label{eq:BigM_Child} \notag
 \end{flalign}
 In addition to mathematical model, we will propose a branch and cut algorithm to solve this problem, which is explained in the next section.


  
  \section{Branch and cut algorithm }

In this section, we propose a branch and cut algorithm to solve the extensive form of the chance-constraint policy. This method is based on the algorithm proposed by Luedtke~\cite{luedtke2014branch} for joint chance-constraints which is modified for our problem. In this model, the binary variables are $\Zed_m$, where $\Zed_m=0$ indicates the current inventory position is adequate to meet demands in child node $m$ without backordering, and $\Zed_m=1$ otherwise, for $m \in \cn$. When $\Zed_m=0$, we should enforce that $\Vi_{n}, \Bi_{n}$ lie within the polyhedron:
\begin{align*} Q_m := \{ (v,B) :  \exists \Es_m \geq 0 \ \text{s.t.} \ 
 & \sum_{\jey \in  \Psub} \Es_{\jey \ka m} = D_{km} + \Bi_k \ \forall \ka  \in \KA \\
 & \sum_{\jey \in  \Csub} \Es_{\ka \jey m} \leq \Vi_k \ \forall \ka  \in \KA \}
 \end{align*}
 
\newcommand{\vsol}{\hat{\Vi}}
\newcommand{\zsol}{\hat{\Zed}}
\newcommand{\bsol}{\hat{\Bi}}
\newcommand{\pisol}{\hat{\pi}}
\newcommand{\betasol}{\hat{\beta}}

Assume we have solved a master problem at node $n$, and have obtained a master problem solution $(\zsol,\vsol,\bsol)$. Note that this solution may or may not satisfy the integrality constraints (e.g., if we have solved an LP relaxation of the master problem). Given a child node $m \in \cn$ with $\zsol_m < 1$, our task is to assess if $(\vsol,\bsol) \in Q_m$, and if not, attempt to generate a cut to remove this solution. In the case of an integer feasible solution, we will always be able to do so when $(\vsol,\bsol) \notin Q_m$.

We can test if given $(\vsol,\bsol) \in Q_m$ by solving the following LP:
\begin{align*}
\Vi_m(\vsol,\bsol) :=  \min_{w,\Es_m} \ & \sum_{\ka  \in \KA} w_k \\
    \text{subject to: } & \sum_{\jey \in  \Psub} \Es_{\jey \ka m} + w_k = D_{km} + \bsol_k \ \forall \ka  \in \KA & (\pi_k) \\
    &-\sum_{\jey \in  \Csub} \Es_{\ka \jey m} \geq -\vsol_k \ \forall \ka  \in \KA& (\beta_k) \\
    & w \in \mathbb{R}_+^{\Ka}, S \geq 0
\end{align*}
By construction, $(\vsol,\bsol) \in Q_m$ if and only if $\Vi_m(\vsol,\bsol) \leq 0$. Furthermore, if $(\pisol,\betasol)$ is an optimal dual solution, then by weak duality, the cut:
 \[ \sum_{\ka  \in \KA} \pisol_k (D_{km} + \Bi_k) - \sum_{\ka  \in \KA} \betasol_k \Vi_k \leq 0 \]
is a valid inequality for $Q_m$. Rearranging this, it takes the form:
\[ \sum_{\ka  \in \KA} \betasol_k \Vi_k - \sum_{\ka  \in \KA} \pisol_k \Bi_k  \geq  \sum_{\ka  \in \KA} \pisol_k D_{km}. \]
If $\Vi_m(\vsol,\bsol) > 0$ then the corresponding cut will be violated by $(\vsol,\bsol)$. 

The inequality derived above is only valid when $\Zed_m = 0$. We thus need to modify it to make it valid for the master problem.
To derive strong cuts based on this base inequality, we solve an additional set of subproblems once we have coefficients $(\pisol,\betasol)$. In particular, for every child node $m'$, we solve:
\begin{align*}
h_{m'}(\pisol,\betasol) := \min_{v,B,\Es_{m'},\Es_n} \ & \sum_{\ka  \in \KA} \betasol_k \Vi_{n \ka} - \sum_{\ka  \in \KA} \pisol_k \Bi_{n  \ka} \\
\text{subject to: } &  \sum_{\jey \in  \Psub} \Es_{\jey \ka m'} = D_{km'} + \Bi_{n  \ka} \ &\forall \ka  \in \KA  \\
    &\sum_{\jey \in  \Csub} \Es_{\ka \jey m'} \leq \Vi_{n \ka} \ &\forall \ka  \in \KA \\
    & \sum_{\jey \in  \Psub} {S}_{n \jey \ka} + \Bi_{n  \ka}  = D_{n \ka} + \Bi_{p(n), \ka} \quad &\forall \ka  \in \KA \\
    & \sum_{\jey \in  \Csub} {S}_{n \ka \jey} \leq \Vi_{p(n), \ka}  \quad &\forall \ka  \in \KA\\
    & \Es_m \geq 0, \Es_n \geq 0
\end{align*}
Note that in this problem we consider substitution variables both for the current node $n$ and for the child node under consideration, $m'$. The substitution variable for the child node $m'$ are to enforce that $(v,B) \in Q_{m'}$. The substitution variables for the current node $n$ are to enforce that $B$ satisfies the current node constraints. Note that we could also require $v$ to satisfy the current node constraints, by introducing the $I_{n \ka}$ and $\x_{n \ka}$ variables and including the constraints \eqref{eq:vdef}, and any constraints on the $x$ variables, such as capacity constraints. However, if our test instances do not have capacity restrictions, this would not likely to be helpful, so I have left this out for simplicity.

After evaluating $h_{m'}(\pisol,\betasol)$ for each $m' \in \cn$, we then sort the values to obtain a permutation $\sigma$ of $\cn$ which satisfies:
\[ h_{\sigma_1}(\pisol,\betasol) \geq h_{\sigma_1}(\pisol,\betasol)  \geq \cdots \geq h_{\sigma_{|\cn|}(\pisol,\betasol)}
 \]
Then, letting $p = \lfloor (1-\alpha) N \rfloor$, the following inequalities are valid for the master problem:
\[ \sum_{\ka  \in \KA} \betasol_k \Vi_k - \sum_{\ka  \in \KA} \pisol_k \Bi_k + 
(h_{\sigma_1}(\pisol,\betasol) - h_{\sigma_i}(\pisol,\betasol))\Zed_{\sigma_1} + 
(h_{\sigma_i}(\pisol,\betasol) - h_{\sigma_{p+1}}(\pisol,\betasol))\Zed_{\sigma_i} \geq h_{\sigma_1}(\pisol,\betasol), \quad i=1,\ldots, p \]
Any of these inequalities could be added, if violated by the current solution $(\zsol,\vsol,\bsol)$.

The final step for obtaining strong valid inequalities is to search for {\it mixing inequalities}, which have the following form. Given a subset $T = \{t_1,t_2,\ldots,t_{\ell}\} \subseteq \{\sigma_1,\sigma_2,\ldots,\sigma_p\}$, the inequality:
\[  \sum_{\ka  \in \KA} \betasol_k \Vi_k - \sum_{\ka  \in \KA} \pisol_k \Bi_k + 
\sum_{i=1}^{\ell} (h_{t_i}(\pisol,\betasol) - h_{t_{i+1}}(\pisol,\betasol))\Zed_{t_i}
\geq  h_{t_1}(\pisol,\betasol) \]
is valid for the master problem. Although the number of such inequalities grows exponentially with $p$, there is an efficient algorithm for finding a most violated inequality for given $(\zsol,\bsol,\vsol)$. 

\begin{algorithm}[H]
\SetAlgoLined
{OUTPUT: A most violated mixing inequality defined by ordered index set $\ti$ } \;
INPUT: $\zsol_{\sigma_i}, \sigma_i, h_{\sigma_i}(\pisol,\betasol)$, $i=1,\ldots,p+1$  {\cred is it p or p+1}\;
Sort the $\zsol_{\sigma_i}$ values to obtain permutation $\rho$ of the indices satisfying:
$\zsol_{\rho_1} \leq \zsol_{\rho_2} \leq \cdots \leq \zsol_{\rho_{p+1}} $ \;
$v \gets h_{\sigma_{p+1}}(\pisol,\betasol)$\;
$T \gets \{ \}$\;
$ i \gets 1$\;
\While{$v < h_{\sigma_1}(\pisol,\betasol)$}{
  \If{$h_{\rho_i}(\pisol,\betasol) > v$}{
  $T \gets T \cup \{\rho_i\}$\;
  $v \gets h_{\rho_i}(\pisol,\betasol)$\;  }
  $i \gets i+1$\;
}

\end{algorithm}

{\bf Potential faster cut generation.} Given the structure of this problem,  we can obtain potentially weaker cuts, but saving significant work, by using $h_{m'}(\pisol,\betasol) = \sum_{\ka  \in \KA} \pisol_k D_{km'}$ for each $m' \in \cn$, instead of solving the above defined LP. This should be valid because the dual feasible region of the set $Q_{m'}$ is independent of $m'$, so a dual solution from one $m$ can be used to define an inequality valid for any other $m'$.



  

\section{Computational experiments}
\subsection{Rolling-horizon framework}
\label{Sec:Rolling}

Considering the scenario tree,  at each stage, we solve two-stage approximation for the multistage problem over
subtree with initial values of $(n,\hat{\Vi}_{\ti(n)-1 , \ka},\hat{\Bi}_{\ti(n)-1 , \ka})$.

Algorithm (\ref{Al:PolicyEvaluation}) illustrates the steps for the a policy evaluation in which for each scenario the problem is solved using specific policy and then the evaluations measures over all the scenarios are calculated. 
The policies are evaluated  using simulation. The demand for each product at each period is generated based on the autoregressive process, and the models are solved repeatedly in a rolling-horizon fashion. To compare the policies we calculate different cost based on the realized demand and solution values. 
Another criteria to compare the policies is the joint service level, which is calculated over all products for each planning period in the simulation. 
In the simulation process, we ignore few initial periods as the warm up periods. The remaining periods are divided into number of batches, and for each batch the average objective and joint service level are calculated. Based on the number of batches and the calculated averages, the confidence intervals are calculated. The experiments are run in a rolling-horizon for 4000 periods 




\begin{algorithm}[]
\SetAlgoLined
{OUTPUT: The confidence interval of the total cost and the service level} \;
INPUT: \
 A demand simulation over $\TI_{Sim}$ periods, A production policy  \;
 $n =1,\hat{\Vi}_{\ti_0} =0,\hat{\Bi}_{\ti_0} = 0$ \;
 \While{$n \leq \TI_{Sim}$ }{
 Solve the Current stage modification model, and fix the the selected $\Bi_{\ti(n)}$ variables equal to 0\;
  Solve the  model Appr$(n,\hat{v}_{\ti(n)-1 , \ka}, \hat{B}_{\ti(n)-1 , \ka})$ and the realized demand $d_{\ti(n)}$\;
  Let $v^*_{ \ti(n)}, x^{*}_{\ti(n)}, y^{*}_{\ti(n)}, S^{*}_{\ti(n)},I^{*}_{\ti(n)},B^{*}_{\ti(n)}$ be the resulting solution for period $\ti(n)$, and the $Obj_{\ti(n)}$ be the total cost of period $\ti(n)$ based on the optimal solution \;
  \If{$B^{*}_{\ti(n)} \geq 0$}{  
  $Z_{\ti(n)} = 1$ }
  $n \gets n+1$ \;}
  Average Cost $=\sum_{n= n_{warm}}^{\TI_{Sim}}Obj_{\ti(n)}/ |T_{Sim}-n_{warm}|$ \;
  Average Service Level$=\sum_{n= n_{warm}}^{\TI_{Sim}}Z_{\ti(n)}/ |T_{Sim}-n_{warm}|$
  
  \caption{Rolling-horizon implementation}
  \label{Al:PolicyEvaluation}
\end{algorithm}

\subsection{Data generation}

To test the policies, and the algorithm we generate many instances based on Rao et al.~\cite{rao2004multi} and Helber et al.~\cite{helber2013dynamic} with some justification for the current problem. Table~\ref{tab:Sub_FD_parameters} illustrates different parameters in the model and how to define them based on data generation parameters. In the base case one way substitution is available for 4 consequent products ordered based on their values. It should be noted that the backlog cost for the next immediate period is calculated based on equation??? considering both substitution cost and the backlog cost. Table \ref{tab:BaseSensitivity} summarize the data generation parameters, their base value and their variation for sensitivity analysis. 




\begin{table}[H]
\centering
\caption{Data generation }
%\footnotesize
\begin{tabular}{ll}
\toprule
%{\bf Sets} &  \\ \midrule
%$\ti$   & Set of planning periods \\ 
%$\ka$   & Set of products  \\ 
%$G$  & Substitution graph (one way substitution) \\
{\bf Parameters} &  \\ \midrule
$c^{\text{prod}}_{\ti \ka}$  & $1+\eta \times(\Ka-\ka)$   , $\eta = 0.1, 0.2 , 0.5$ \\
$c^{\text{sub}}_{\ti \ka \jey }$  & $\max(0,(1+\tau) \times (c^{\text{prod}}_{\jey \ti} - c^{\text{prod}}_{\ti \ka}))$ , $\tau = 0 , 0.5 , 1$  \\ 
$c^{\text{hold}}_{\ti \ka}$  & $\rho \times c^{\text{prod}}_{\ti \ka} $ , $\rho = 0.005, 0.01, 0.02, 0.05 ,0.8$   \\ 
$TBO$  &  $TBO = 1, 2, 4$   \\ 
$c^{\text{setup}}_{\ti \ka}$ & $E[\overline{D_{\ti \ka}}] \times TBO^2 \times c^{\text{hold}}_{\ti \ka} /2$ \\ 
$bc_{\ti \ka}$  &  $\pi \times c^{\text{hold}}_{\ti \ka}$ \\
$SL$  &  $ 75\%, 80\%, 90\%, 95\%, 99\%$ \\
${d}_{\ti \ka}$  & Generated based on AR procedure
 \\ \bottomrule
\end{tabular}
 \label{tab:Sub_FD_parameters}
\end{table}





\begin{table}[H]
\centering
%\footnotesize
\caption{ Parameters for the base case and the sensitivity analysis} \label{tab:BaseSensitivity}
%\begin{adjustbox}{width=1\textwidth,center=\textwidth}
\begin{tabular}{lll}
\toprule
{\bf Parameters} & Base Case & Variation \\ \midrule
$\Ti$   & 6 & 4, 6 , 8 , 10\\ 
$\Ka$   & 10 & 5, 10, 15, 20\\ 
$\eta$  &   0.2 & 0.1, 0.2 , 0.5   \\ 
$\tau$  &   1.5 & 1, 1.25 , 1.5, 1.75, 2, 2.5   \\ 
$\rho $  &   0.05 & 0.02, 0.05, 0.1 , 0.2 , 0.5   \\ 
$ TBO $  &   1 & 1, 1.25, 1.5, 1.75, 2   \\ 
$ SL $  &95\% & 80\%, 90\%, 95\%, 99\%  \\ 
%$ \pi $  &   2 & 0, 2, 10  \\ 
\bottomrule 
\end{tabular}
%\end{adjustbox}
\end{table}

\subsubsection{AR procedure for demand generation}

To generate the random demand we used autoregressive process model which consider the correlation in different stage demand as follows~\cite{jiang2017production} as (\ref{eq:AR1}) where $C, AR_1,$ and $AR_2$ are parameters of the model, and $\epsilon_{\ka \ti+1}$ is a random noise with normal distribution with the mean of 0 and 1 standard deviation. 

 
{\cred{
\begin{flalign}
  &  d_{\ka \ti+1} = C + AR_1 \times d_{\ti \ka} + AR_2 \times \epsilon_{\ka \ti+1}   &\forall \ka  \in \KA , \forall \ti \in \TI  &     \label{eq:AR1}&
  \end{flalign}}}
  
  In our data sets, $C = 20$, $AR_1 = 0.8$, and $AR2 = 0.1 \times 100$. With these data the expected demand for each product in each period is equal to 100.  
  
  


{\cred{As we have no production in the first period without lose of generality we assume that the demand in the first period is equal to zero, other wise the if there is no initial inventory, the service level constraint wont be satisfied. In the AR data generation procedure we start with the defined average in the first period and then follow the procedure for the rest of the periods. Then to make the first period demand equal to 0 we subtract this average from the first period demand.}}\\
Demand generation options:












%\subsection{Numerical experiments}
\subsection{Methodology evaluation}
In this section, we analyse the efficiency of different methodologies used to solve the chance-constraint against the extensive form formulation for example based on the number of scenarios. To this end, we solve only one stage of the problem with three different methods: extensive form (Big-M), strong branch and cut (labeled as Strong $B\& C$), and fast branch and cut (labeled as Fast $B\&C$). As the first period demand is equal to 0, we decided to use the second period model. To have a fair comparison, we solved the first period model Fast $B\&C$, and used the resulting ${\Bi}$ and ${\Vi}$ as the $\hat{\Bi}$ and $\hat{\Vi}$ for the second period. The measure are the average CPU time in second (Time), the average integrality gap (GAP), and the average number of optimal solution over all the instances in one group. Each instance is solved five times.

The algorithms are implemented in python and MIP models are solve using IBM ILOG CPLEX 12.8. We performed the experiments on a 2.4 GHz Intel Gold processor with only one thread on the Compute Canada computing grid.

Table (\ref{tab:MethodologyCompare}) illustrates that the faster version of the branch cut algorithm has obviously better performance compared to the other methods, and hence it will be used for the rest of the experiments.

In the next set of experiments we perform a sensitivity analyse to investigate the performance of the faster branch and cut algorithm based on the change in different parameters. To this end a base case instance is selected and, the solution time is reported based on the changes in different parameters as shown in table ???. 
Each of these instances are solved with 100 branches and 4000 simulations iterations. 

\begin{comment}

\begin{table}[]
\caption{Comparison of methodology to solve the model with the service level}
\label{tab:MethodologyCompare}
\begin{tabular}{lllllllllllll}
Method      & \multicolumn{4}{c}{Extensive form} & \multicolumn{4}{c}{Strong Cut}    & \multicolumn{4}{c}{Fast Cut}     \\ \hline
\# Branches & Time    & Gap   & \# Opt & BestObj & Time   & Gap   & \# Opt & BestObj & Time  & Gap   & \# Opt & BestObj \\ \hline
100         & 800.2   & 0.2\% & 4.8    & 10612.1 & 9.3    & 0.0\% & 5.0    & 10611.7 & 3.2   & 0.0\% & 5.0    & 10611.7 \\
200         & 2110.8  & 1.4\% & 4.1    & 10513.3 & 45.4   & 0.0\% & 5.0    & 10509.0 & 7.4   & 0.0\% & 5.0    & 10509.0 \\
300         & 3264.9  & 2.5\% & 3.2    & 10419.6 & 131.2  & 0.0\% & 5.0    & 10407.5 & 10.3  & 0.0\% & 5.0    & 10407.5 \\
500         & 4061.1  & 3.8\% & 2.5    & 10752.6 & 582.7  & 0.0\% & 5.0    & 10712.6 & 25.9  & 0.0\% & 5.0    & 10712.6 \\
1000        & 5321.6  & 7.5\% & 1.6    & 12369.2 & 3265.4 & 1.1\% & 4.0    & 10722.1 & 220.3 & 0.0\% & 4.9    & 10700.5 \\ \hline
Average     & 3111.7  & 3.1\% & 3.2    & 10933.4 & 806.8  & 0.2\% & 4.8    & 10592.6 & 53.4  & 0.0\% & 5.0    & 10588.3
\end{tabular}


\end{table}
\end{comment}

\begin{table}[]
\caption{Comparison of methodologies to solve the model with the service level}
\label{tab:MethodologyCompare}
\begin{tabular}{lrccrccrcc}
Method      & \multicolumn{3}{c}{Extensive form} & \multicolumn{3}{c}{Strong B\&C} & \multicolumn{3}{c}{Fast B\&C} \\ \hline
\# Branches & Time       & Gap       & \# Opt    & Time      & Gap     & \# Opt   & Time    & Gap     & \# Opt   \\ \hline
100         & 800.2      & 0.2\%     & 4.8       & 9.3       & 0.0\%   & 5.0      & 3.2     & 0.0\%   & 5.0      \\
200         & 2110.8     & 1.4\%     & 4.1       & 45.4      & 0.0\%   & 5.0      & 7.4     & 0.0\%   & 5.0      \\
300         & 3264.9     & 2.5\%     & 3.2       & 131.2     & 0.0\%   & 5.0      & 10.3    & 0.0\%   & 5.0      \\
500         & 4061.1     & 3.8\%     & 2.5       & 582.7     & 0.0\%   & 5.0      & 25.9    & 0.0\%   & 5.0      \\
1000        & 5321.6     & 7.5\%     & 1.6       & 3265.4    & 1.1\%   & 4.0      & 220.3   & 0.0\%   & 4.9      \\ \hline
Average     & 3111.7     & 3.1\%     & 3.2       & 806.8     & 0.2\%   & 4.8      & 53.4    & 0.0\%   & 5.0     
\end{tabular}

\end{table}


\subsection{Policy evaluation}

Three different policies are compared against each other. The first two are the deterministic policies, and the third one is the policy which consider the service level explicitly in the model. The first deterministic policy is the average policy which does not consider the service level, and for the demand we substitute it with the average demand which is calculated based on the realized demand and the autoregressive process.
The second deterministic policy is the quantile policy in which the demand for the next immediate period is substituted based on the quantile percentage in the demand scenario branches of next immediate period, and for the rest of the planning periods we calculate it with the same procedure as the average policy.
The third policy considers the service level in the next stage. The demand in each planning period in this policy is substitute with the average demand same as the average policy.

In this section, the policies are compared based on objective function and the respect for the joint service level, using the procedure explained in section~\ref{Sec:Rolling}. Table \ref{PolicyComp} compares the three policies using these measures at two different TBOs and four different service levels. Among the three policies the chance-constraint policy is the only policy which respects the service level in all the instances. In all the instances with acceptable service level the chance-constraint policy has the lower cost. Among the three policies the average policy is not sensitive to the service level and it has poor performance in this measure. When TBO is equal to 1 the joint service level is about 21\%, and when the TBO is equal to 2 the service level is slightly more than 78\%. This result show that this policy is not a reliable policy, and it will not be used in the rest of experiments. The quantile policy has an acceptable performance in both measures. 


\newcolumntype{L}{>{$}l<{$}}
\newcolumntype{C}{>{$}c<{$}}
\newcolumntype{R}{>{$}r<{$}}
\newcommand{\nm}[1]{\textnormal{#1}}


\begin{table} [h]
\centering
\small
\begin{tabular}{RRRRRRRR}
\toprule
\multicolumn{1}{R}{$TBO$} &
\multicolumn{1}{R}{$SL \%$} &
\multicolumn{3}{c}{Total cost}    &
\multicolumn{3}{c}{Service level(\%)}    \\ 
\cmidrule(lr){3-5}
\cmidrule(lr){6-8}

&&
\multicolumn{1}{r}{Average policy} &
\multicolumn{1}{r}{Quantile Policy}     &
\multicolumn{1}{r}{SL policy} &
\multicolumn{1}{r}{Average policy} &
\multicolumn{1}{r}{Quantile Policy}     &
\multicolumn{1}{r}{SL policy}  \\
\midrule

	&	80	&	74.4	\pm	0.4	&	66.4	\pm	0.2	&	66.7	\pm	0.2	&	21.4	\pm	1.4	&	76.5	\pm	1.4	&	84.9	\pm	1.3	\\
	&	90	&	74.4	\pm	0.4	&	68.2	\pm	0.1	&	67.1	\pm	0.2	&	21.4	\pm	1.4	&	90.6	\pm	1.1	&	90.3	\pm	1.1	\\
1	&	95	&	74.4	\pm	0.4	&	71.0	\pm	0.1	&	67.6	\pm	0.2	&	21.4	\pm	1.4	&	94.1	\pm	0.9	&	95.3	\pm	0.7	\\
	&	99	&	74.4	\pm	0.4	&	76.1	\pm	0.1	&	69.1	\pm	0.2	&	21.4	\pm	1.4	&	99.1	\pm	0.3	&	99.1	\pm	0.3	\\
	&	80	&	204.3	\pm	0.6	&	204.3	\pm	0.5	&	191.2	\pm	0.6	&	78.1	\pm	2.7	&	93.2	\pm	1.2	&	98.7	\pm	0.4	\\
2	&	90	&	204.3	\pm	0.6	&	207.2	\pm	0.4	&	192.4	\pm	0.6	&	78.1	\pm	2.7	&	97.4	\pm	0.6	&	99.5	\pm	0.3	\\
	&	95	&	204.3	\pm	0.6	&	210.0	\pm	0.4	&	193.5	\pm	0.6	&	78.1	\pm	2.7	&	98.5	\pm	0.5	&	99.8	\pm	0.2	\\
	&	99	&	204.3	\pm	0.6	&	215.3	\pm	0.4	&	195.4	\pm	0.6	&	78.1 \pm	2.7	&	99.7	\pm	0.2	&	100.0	\pm	0.0	\\

\midrule[\heavyrulewidth]
\multicolumn{7}{l}{\footnotesize 95\% of confidence interval} \\
\bottomrule
\end{tabular}
\caption{Three policies comparison}\label{PolicyComp}
\end{table}


The rest of this section is dedicated to compare the chance-constraint and quantile policies using more instances. To this end, the relative cost change $\Delta Cost$ (\ref{eq:ِDeltaCost}), and the joint service level is used to compare these two policies under different settings. Figure~\ref{fig:TBOComp} shows the comparison of the the quantile policy and the chance-constraint policy under different values of TBO. In all cases, the chance-constraint policy has better performance in terms of service level. The chance-constraint policy has lower cost in all cases in which both policies have acceptable service level. When TBO is more than 1 the service level is over satisfied. This is mostly because of the current stage modification policy. We will discuss later the necessity of this policy and if it imposes any extra costs.  

\begin{flalign}
  & \Delta Cost (\%) = \dfrac{Total Cost_{Quantile} - Total cost _{chance-constraint}}{Total Cost_{Quantile}} \times 100& \label{eq:ِDeltaCost}
 \end{flalign}

\begin{figure}%
    \centering
    \subfloat[\centering Service level]{{\includegraphics[width=9cm]{TBOSL.pdf} }}%
   % \quad
    \subfloat[\centering Total cost ]{{\includegraphics[width=9cm]{TBOOBJ.pdf} }}%
    \caption{Comparison based on TBO}%
    \label{fig:TBOComp}%
\end{figure}

% The performance of the policies can be also compared base on different cost structure. For example when the backlog cost is equal to 0 we expect that the deterministic model with the average demand does not show a good performance in terms of service level. The service level for the next stage and for the whole planning period.
%As the results show, the deterministic models are sensitive to the cost of backlog and has different pattern in different settings. A low backlog may lead to very low service level, and high one may impose unnecessary cost to the model. Using these policies one should do a sensitivity analysis to come up with a proper backlog cost to come up with an acceptable service level. It should be noted that even if the company come up with the exact cost of backlog, it may not be enough to reach the desired service level. However, in the policy with the service level, it is not necessary to consider the backlog cost as the quality of the solution is not sensitive to it. 



\subsubsection{Chance-constraint policy vs quantile policy}



Figure~\ref{fig:ETAComp} shows the comparison based on different values of $\eta$ under two different values of TBO, 1 and 2. In all cases, the chance-constraint policy has better performance in terms of joint service level, and the total cost. 

\begin{figure}%
    \centering
    \subfloat[\centering Service level]{{\includegraphics[width=9cm]{ETASL.pdf} }}%
  %  \quad
    \subfloat[\centering Total cost ]{{\includegraphics[width=9cm]{ETAOBJ.pdf} }}%
    \caption{Comparison based on $\eta$}%
    \label{fig:ETAComp}%
\end{figure}

Figure~\ref{fig:SLComp} shows the comparison based on different service level values under two different values of TBO, 1 and 2. In all cases, the chance-constraint policy respect the service level and in cases where both policies have acceptable service level, the chance-constraint policy has better performance in terms of joint service level, and the total cost. It should be noted that when the service level increase, the performance of chance-constraint policy against the quantile policy will improve. Figure~\ref{fig:SLTotalCost} is a complementary figure to Figure Figure~\ref{fig:SLComp} and illustrates the trend of total cost for different values service level. As can be seen in this figure, the total cost of quantile policy increase at higher speed with increase in the service level, which is not the case in chance-constraint policy.

Figure~\ref{fig:TIComp} shows similar comparison based on $\tau$ values. In all cases, the chance-constraint policy has better performance compared to the quantile policy.

\begin{figure}%
    \centering
    \subfloat[\centering Service level]{{\includegraphics[width=9cm]{SLSL.pdf} }}%
    %\quad
    \subfloat[\centering Total cost ]{{\includegraphics[width=9cm]{SLOBJ.pdf} }}%
    \caption{Comparison based on service level}%
    \label{fig:SLComp}%
\end{figure}

\begin{figure}[!h]
\begin{center}
\includegraphics[scale=0.6]{SLTotalCost.pdf}
\caption{Comparison based on total cost trend} 
\label{fig:SLTotalCost}
\end{center}
\end{figure}


\begin{figure}%
    \centering
    \subfloat[\centering Service level]{{\includegraphics[width=9cm]{TISL.pdf} }}%
    %\quad
    \subfloat[\centering Total cost ]{{\includegraphics[width=9cm]{TIOBJ.pdf} }}%
    \caption{Comparison based on $\tau$}%
    \label{fig:TIComp}%
\end{figure}

\subsubsection{Sensitivity analysis}
%An interesting option for the sensitivity analysis is to show that having both service level and approximate backlog cost is the better than version when we have both options together. This can be done with a sensitivity analysis for the backlog cost with and without the service level and check how the model is robust in the former case, and may be show underestimating the backlog is better than overestimating that.
In this section, we perform some sensitivity analysis based on different elements of cost function, the setup cost, inventory holding cost and the substitution cost. 
Figure~\ref{fig:TBOSen} show the cost change base on different values for TBO. It is intuitive that in lot sizing problem, by increase in TBO, there will be an increase in setup cost plus the inventory holding cost. In addition to this increase we can see a constant increase in the substitution cost, which means increase in the amount of substitution. 

\begin{figure}[!h]
\begin{center}
\includegraphics[scale=0.3]{TBOSen.pdf}
\caption{Cost analysis based on TBO} 
\label{fig:TBOSen}
\end{center}
\end{figure}

Figure~\ref{fig:TISen} illustrates different cost changes based on changes in parameter $\tau$ under two different TBO. When TBo is equal to 1, by increasing the the substitution cost, the inventory cost slightly increase, and the substitution cost will not increase. We can conclude that by increasing the substitution cost, the amount of inventory will increase and the amount of substitution will decrease. This is more obvious when TBO is equal to 2. In this case the, increase in the substitution cost per unit result in total substitution cost decrease, total holding cost increase, and a slight setup cost increase.  

\begin{figure}[H]
\begin{center}
\includegraphics[scale=0.5]{TISen.pdf}
\caption{Cost analysis based on $\tau$} 
\label{fig:TISen}
\end{center}
\end{figure}


\subsubsection{The necessity of current stage modification policy}
    When TBO is greater than 1 the service level is oversatisfied. This is due to the fact that to save on the setup cost the production amount will be higher than the average demand of one period. By this higher production many of the demands can be satisfied in the current period, when we apply the current stage modification policy. In this section, we will show that this oversatisfaction of the service level will not impose a huge cost to the model. In these experiments, we cancel the current stage modification. We will see that, even with very low service levels, there is very low cost difference compared to the case in which the model tries to minimise the backlog in the current stage as much as possible. This experiments also show the necessity of current stage modification policy, without which it is not possible to satisfy the required service level.
    
    
\begin{figure} [H]
    \centering
    \subfloat[\centering Service level]{{\includegraphics[width=9cm]{CurrentStageSL.pdf} }}%
   % \quad
    \subfloat[\centering Total cost ]{{\includegraphics[width=9cm]{CurrentStageOBJ.pdf} }}%
    \caption{The necessity of current stage modification}%
    \label{fig:TBOComp}%
\end{figure}




\subsection{Effect of substitution}

In this section we want to investigate the effect of substitution. To this end we run some experiments and we eliminate the possibility of substitution for different service levels. Table ??? provide the result on these experiments. In this table the average cost, and average service level their interval and different cost in detail are provided. These result are based on $\pi =0$. As expected, eliminating the possibility of substitution will increase the costs. 

It should be mentioned that the confidence interval of the objective function is about ??? and for the service level is about ??? which is reasonable.

\begin{table}[]
\begin{tabular}{llllllllllllll}
\multicolumn{14}{c}{With Substitution} \\
SL & TBO & Pi & Time & OBJ & \multicolumn{2}{c}{OBJ (CI)} & SL & \multicolumn{2}{c}{SL (CI)} & STC & HC & SUBC & BLC \\
80\% & 1 & 0 & 9740.82 & 66.86 & 66.66 & 67.06 & 80.80\% & 79.39\% & 82.21\% & 52.95 & 10.09 & 3.82 & 0.00 \\
90\% & 1 & 0 & 7988.94 & 67.56 & 67.37 & 67.75 & 90.60\% & 89.53\% & 91.67\% & 52.95 & 11.08 & 3.54 & 0.00 \\
95\% & 1 & 0 & 3945.63 & 68.03 & 67.83 & 68.23 & 95.30\% & 94.55\% & 96.05\% & 52.95 & 11.15 & 3.93 & 0.00 \\
99\% & 1 & 0 & 3913.23 & 69.39 & 69.20 & 69.57 & 99.10\% & 98.78\% & 99.42\% & 52.95 & 13.15 & 3.29 & 0.00 \\
80\% & 1 & 1 & 8944.49 & 67.13 & 67.09 & 66.69 & 80.80\% & 79.39\% & 82.21\% & 52.89 & 10.17 & 3.83 & 0.25 \\
90\% & 1 & 1 & 8536.70 & 67.71 & 67.79 & 67.41 & 90.60\% & 89.53\% & 91.67\% & 52.88 & 11.16 & 3.56 & 0.11 \\
95\% & 1 & 1 & 9261.46 & 68.17 & 68.30 & 67.91 & 95.3\% & 94.55\% & 96.05\% & 52.88 & 11.29 & 3.94 & 0.07 \\
99\% & 1 & 1 & 7672.03 & 69.85 & 70.05 & 69.63 & 99.20\% & 98.91\% & 99.49\% & 52.85 & 13.63 & 3.36 & 0.01 \\
\multicolumn{14}{c}{Without substitution} \\
80\% & 1 & 0 & 21990.41 & 72.68 & 72.58 & 72.77 & 80.00\% & 78.44\% & 81.56\% & 53.00 & 19.68 & 0.00 & 0.00 \\
90\% & 1 & 0 & 3328.39 & 74.69 & 74.59 & 74.78 & 89.40\% & 88.27\% & 90.53\% & 52.99 & 21.69 & 0.00 & 0.00 \\
95\% & 1 & 0 & 2048.29 & 76.39 & 76.30 & 76.48 & 94.60\% & 93.85\% & 95.35\% & 53.00 & 23.40 & 0.00 & 0.00 \\
99\% & 1 & 0 & 1386.27 & 78.89 & 78.79 & 78.98 & 98.90\% & 98.54\% & 99.26\% & 53.00 & 25.89 & 0.00 & 0.00 \\
80\% & 1 & 1 & 14967.64 & 73.14 & 73.10 & 72.89 & 80.00\% & 78.46\% & 81.54\% & 52.85 & 20.15 & 0.00 & 0.14 \\
90\% & 1 & 1 & 6709.10 & 75.30 & 75.33 & 75.11 & 89.40\% & 88.27\% & 90.53\% & 52.81 & 22.42 & 0.00 & 0.08 \\
95\% & 1 & 1 & 3705.33 & 77.18 & 77.25 & 77.01 & 94.70\% & 93.95\% & 95.45\% & 52.81 & 24.32 & 0.00 & 0.05 \\
99\% & 1 & 1 & 1656.95 & 80.08 & 80.20 & 79.92 & 98.90\% & 98.54\% & 99.26\% & 52.88 & 27.19 & 0.00 & 0.01
\end{tabular}
\end{table}


\begin{table}[]
\begin{tabular}{llllllllllllll}
\multicolumn{14}{c}{With Substitution} \\
SL & TBO & Pi & Time & OBJ & \multicolumn{2}{c}{OBJ (CI)} & SL & \multicolumn{2}{c}{SL (CI)} & STC & HC & SUBC & BLC \\
95\% & 1 & 0 & 3920.83 & 68.03 & 68.23 & 67.83 & 95.3\% & 94.55\% & 96.05\% & 52.95 & 11.15 & 3.93 & 0.00 \\
95\% & 1.5 & 0 & 5507.96 & 133.69 & 133.94 & 133.44 & 95.20\% & 94.45\% & 95.95\% & 117.52 & 11.17 & 5.00 & 0.00 \\
95\% & 2 & 0 & 10003.63 & 218.93 & 219.50 & 218.36 & 95.10\% & 94.31\% & 95.89\% & 178.06 & 11.28 & 29.59 & 0.00 \\
95\% & 1 & 1 & 9283.92 & 68.17 & 68.30 & 67.91 & 95.30\% & 94.55\% & 96.05\% & 52.88 & 11.29 & 3.94 & 0.07 \\
95\% & 1.5 & 1 & 8881.62 & 140.45 & 140.76 & 140.14 & 99.80\% & 99.62\% & 99.98\% & 69.48 & 61.13 & 9.84 & 0.00 \\
95\% & 2 & 1 & 13011.88 & 192.59 & 193.25 & 191.92 & 99.80\% & 99.66\% & 99.94\% & 114.85 & 57.47 & 20.26 & 0.00 \\
\multicolumn{14}{c}{Without Substitution} \\
95\% & 1 & 0 & 1923.74 & 76.39 & 76.48 & 76.30 & 94.60\% & 93.85\% & 95.35\% & 53.00 & 23.40 & 0.00 & 0.00 \\
95\% & 1.5 & 0 & 1958.76 & 142.39 & 142.48 & 142.29 & 94.60\% & 93.85\% & 95.35\% & 118.99 & 23.40 & 0.00 & 0.00 \\
95\% & 2 & 0 & 2277.90 & 233.38 & 233.48 & 233.27 & 94.60\% & 93.85\% & 95.35\% & 209.98 & 23.40 & 0.00 & 0.00 \\
95\% & 1 & 1 & 4027.09 & 77.18 & 77.25 & 77.01 & 94.70\% & 93.95\% & 95.45\% & 52.81 & 24.32 & 0.00 & 0.05 \\
95\% & 1.5 & 1 & 3559.08 & 193.81 & 194.28 & 193.32 & 97.80\% & 97.30\% & 98.30\% & 102.03 & 91.77 & 0.00 & 0.01 \\
95\% & 2 & 1 & 5116.17 & 273.15 & 273.91 & 272.37 & 98.00\% & 97.50\% & 98.50\% & 179.02 & 94.13 & 0.00 & 0.01
\end{tabular}
\end{table}

\section{Conclusion}

An extension to this problem is to approximate this the problem as a two stage stochastic problem in addition to have the service level in the next stage. This model will computationally expensive, but it will provide a better approximation for the problem.

Acknowledgements:
Calcule quebec
FRQNT


\begin{thebibliography}{99}



%%

%\bibitem{RefJ}
% Format for Journal Reference
%Author, Article title, Journal, Volume, page numbers (year)
% Format for books
%\bibitem{RefB}
%Author, Book title, page numbers. Publisher, place (year)

%a,b

%a,b

\bibitem{akccaycategory}
Akçay Yalçın. and Yunke Li and Harihara Prasad Natarajan, Category Inventory Planning With
Service Level Requirements and Dynamic Substitutions.
Production and Operations Management (2020), https://doi.org/doi:10.1111/poms.13240

\bibitem{bassok1999single}
Bassok, Yehuda and Anupindi, Ravi and Akella, Ram,
Single-period multiproduct inventory models with substitution, Operations Research, 47, 4, 632--642, (1999)
 

\bibitem{bookbinder1988strategies}
 Bookbinder, James H and Tan, Jin-Yan, Strategies for the probabilistic lot-sizing problem with service-level constraints, Management Science, 34, 9, 1096--1108,
 (1988)


\bibitem{bitran1992deterministic}
Bitran, Gabriel R and Leong, Thin-Yin, Deterministic approximations to co-production problems with service constraints and random yields, Management science, 38, 5, 724--742, (1992)
  
  
  \bibitem{bitran1992ordering}
Bitran, Gabriel R and Dasu, Sriram, Ordering policies in an environment of stochastic yields and substitutable demands,
Operations Research,
40, 5,
999--1017, (1992)

\bibitem{bitran1994co}
Bitran, Gabriel R and Gilbert, Stephen M,
  Co-production processes with random yields in the semiconductor industry, Operations Research, 42, 3, 476--491, (1994)
%c,d
\bibitem{chen2020dynamic}
Chen, Boxiao and Chao, Xiuli, Dynamic inventory control with stockout substitution and demand learning, Management Science, (2020) , https://doi.org/10.1287/mnsc.2019.3474
%e,f

%g,h

\bibitem{guan2011stochastic} Guan, Yongpei, Stochastic lot-sizing with backlogging: computational complexity analysis,  Journal of Global Optimization, 49, 4, 651--678, (2011)

\bibitem{guan2008polynomial}
  Guan, Yongpei and Miller, Andrew J, Polynomial-time algorithms for stochastic uncapacitated lot-sizing problems,  Operations Research, 56, 5, 1172--1183,
(2008)


\bibitem{gicquel2018joint}
Gicquel, C{\'e}line and Cheng, Jianqiang, A joint chance-constrained programming approach for the single-item capacitated lot-sizing problem with stochastic demand, Annals of Operations Research, 264, 1-2, 123--155, (2018)
\bibitem{helber2013dynamic}
 Helber, Stefan and Sahling, Florian and Schimmelpfeng, Katja, Dynamic capacitated lot sizing with random demand and dynamic safety stocks, OR Spectrum, 35, 1, 75--105, (2013)
 

\bibitem{hsu1999random}
Hsu, Arthur and Bassok, Yehuda, Random yield and random demand in a production system with downward substitution, Operations Research, 47, 2, 277--290,
  (1999)
  
\bibitem{hsu2005dynamic}
Hsu, Vernon Ning and Li, Chung-Lun and Xiao, Wen-Qiang, Dynamic lot size problems with one-way product substitution, IIE transactions, 37, 3, 201--215, (2005)

\bibitem{haugen2001progressive}Haugen, Kjetil K and L{\o}kketangen, Arne and Woodruff, David L, Progressive hedging as a meta-heuristic applied to stochastic lot-sizing, European Journal of Operational Research, 132, 1, 116--122, (2001)

%i,j

%\bibitem{jiang2017service}
 %Jiang, Yuchen and Shi, Cong and Shen, Siqian, Service Level Constrained Inventory Systems, Production and Operations Management, 28, 9, 2365–-2389,
 %(2017)
 
 \bibitem{jiang2017production}
 Jiang, Yuchen and Xu, Juan and Shen, Siqian and Shi, Cong, Production planning problems with joint service-level guarantee: a computational study, International Journal of Production Research, 55, 1, 38--58,
 (2017)
 %l
 
\bibitem{lang2010efficient}
 Lang, Jan Christian and Domschke, Wolfgang, Efficient reformulations for dynamic lot-sizing problems with product substitution, OR spectrum,
32, 2, 263--291, (2010)

\bibitem{lulli2006heuristic} Lulli, Guglielmo and Sen, Suvrajeet, A heuristic procedure for stochastic integer programs with complete recourse, European Journal of Operational Research,
171, 3, 879--890, (2006)

\bibitem{lulli2004branch}Lulli, Guglielmo and Sen, Suvrajeet, A branch-and-price algorithm for multistage stochastic integer programming with application to stochastic batch-sizing problems, Management Science, 50, 6, 786--796, (2004)
\bibitem{liu2018polyhedral}
Liu, Xiao and K{\"u}{\c{c}}{\"u}kyavuz, Simge, A polyhedral study of the static probabilistic lot-sizing problem, Annals of Operations Research, 261, 1-2,
233--254, (2018)

\bibitem{liu2016decomposition}
Liu, Xiao and K{\"u}{\c{c}}{\"u}kyavuz, Simge and Luedtke, James, 
Decomposition algorithms for two-stage chance-constrained programs, Mathematical Programming, 157,
  1, 219--243, (2016)
 

\bibitem{luedtke2008sample}
Luedtke, James and Ahmed, Shabbir, A sample approximation approach for optimization with probabilistic constraints, 
SIAM Journal on Optimization, 19,
2, 674--699, (2008)


\bibitem{luedtke2014branch}
  Luedtke, James, A branch-and-cut decomposition algorithm for solving chance-constrained mathematical programs with finite support,
  Mathematical Programming, 146, 1, 219--244,
 (2014)
%m, n 

\bibitem{ng2012robust}
Ng, Tsan Sheng and Fowler, John and Mok, Ivy, Robust demand service achievement for the co-production newsvendor, IIE Transactions,
  44,
  5,
 327--341,
  (2012)
  
%o, p , q
%R
 \bibitem{rao2004multi}
 Rao, Uday S and Swaminathan, Jayashankar M and Zhang, Jun,
 Multi-product inventory planning with downward substitution, stochastic demand and setup costs,
 IIE Transactions, 36, 1, 59--71, (2004)
 %t 
 
 \bibitem{tempelmeier2007stochastic}
Tempelmeier, Horst, On the stochastic uncapacitated dynamic single-item lotsizing problem with service level constraints, European Journal of Operational Research, 181, 1, 184--194, (2007)
 

 \bibitem{tempelmeier2011column}
Tempelmeier, Horst, A column generation heuristic for dynamic capacitated lot sizing with random demand under a fill rate constraint, Omega, 39, 6, 627--633, (2011)
 % v
 

\bibitem{zeppetella2017optimal}
Zeppetella, Luca and Gebennini, Elisa and Grassi, Andrea and Rimini, Bianca, Optimal production scheduling with customer-driven demand substitution, International Journal of Production Research, 55, 6, 1692--1706, (2017)
 



\end{thebibliography}




\end{document}