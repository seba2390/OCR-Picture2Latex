%%    TEMPLATE for articles submitted to the IWLS 2011
%%
%%
%%     Please do not remove lines commented out with %+
%%           these are for the editors' use.
%%

\documentclass[10pt]{article}
\usepackage{epsfig}
\usepackage{lscape}
\usepackage{adjustbox}
\usepackage{mathtools}
\usepackage{amsmath}
\usepackage{comment}
\usepackage{booktabs}
\usepackage{array}
\usepackage{xcolor}
\usepackage[margin=1 in]{geometry}
\usepackage[ruled,vlined]{algorithm2e}
\usepackage{amssymb}



\usepackage{textcomp}
\usepackage{caption}
\usepackage{multirow}
\usepackage{here}
\usepackage{amsmath}
\usepackage{dsfont}
\newcommand{\R}{\mathds{R}}
\newcommand{\E}{\mathds{E}}
\usepackage{graphicx}
\usepackage{subfig}
\allowdisplaybreaks

% For the mathematical formulation

\newcommand{\ti}{t} %For the time period index
\newcommand{\TI}{T}
\newcommand{\ka}{k} % For product index
\newcommand{\KA}{K}
\newcommand{\jey}{j} % For product index
\newcommand{\Bi}{B} %For the backlog
\newcommand{\Vi}{v} %For the order upto level
\newcommand{\Es}{S} %For the Substitution
\newcommand{\Zed}{z} %For the z variable
\newcommand{\x}{x} %For the x variable
\newcommand{\y}{y} %For the y variable



%%%%%%%%%%%%%%%%%%%%%%%%%%%%%%%%%%%%%%%%%%%%%%%%%%%%%%%%%%%%%%%%%%%%%%%%%%%%
%%  Do not change these:
%\textwidth=6.0in  \textheight=8.25in

%%  Adjust these for your printer:
%\leftmargin=-0.3in   \topmargin=-0.20in


%%%%%%%%%%%%%%%%%%%%%%%%%%%%%%%%%%%%%%%%%%%%%%%%%%%%%%%%%%%%%%%%%%%%%%%%%%%%
%  personal abbreviations and macros
%    the following package contains macros used in this document:
%%%%%%%%%%%%%%%%%%%%%%%%%%%%%%%%%%%%%%%%%%%%%%%%%%%%%%%%%%%%%%%%%%%%%%%%%%%
%
%  To include an item in the INDEX of the conference volume,
%           flag it with    \index{<item name>}
%  The use of this macro is illustrated in the text.
%
%%%%%%%%%%%%%%%%%%%%%%%%%%%%%%%%%%%%%%%%%%%%%%%%%%%%%%%%%%%%%%%%%%%%%%%%%%%%%

\newcommand{\cred}{\color{red!65!black}}

\def\Title#1{\begin{center} {\Large {\bf #1} } \end{center}}

\begin{document}


\Title{Multi-stage stochastic lot sizing with substitution and joint $\alpha$ Service Level }

\bigskip\bigskip

%+\addtocontents{toc}{{\it First Author}}
%+\label{AuthorNameStart}

%%    TEMPLATE for articles submitted to the IWLS 2011
%%
%%
%%     Please do not remove lines commented out with %+
%%           these are for the editors' use.
%%












\section{Abstract}

We consider the lot sizing problem with stochastic demand and the possibility of product substitution. Considering different production costs, the use of substitution can increase the revenue and customer satisfaction specially when the demand is uncertain. The goal is to minimize the total expected cost while satisfying a predetermined service level. In our model, we consider the $\alpha$ service level which limits the probability of stock outs, defined as a chance constraint. The stochsticity is represented as a scenario tree, and we investigate different solution policies for this chance-constrained multi-stage stochastic using mixed integer mathematical models. To solve the chance constraint models we use dual-based decomposition methods modified for this problem. Through extensive numerical experiments we conclude with some managerial insights. 

\section{Introduction}


In the market, when the customers cannot find a specific product, an option is to substitute it with another product whether by the customer or by the firm. Having a limited capacity and considering different production costs, using the substitution option can increase the revenue and customer satisfaction specially when dealing with demand uncertainty. Considering this option in production decisions may result in cost saving and it is worthwhile to investigate.
The basic lot sizing problem is a multi-period production planning problem which considers the trade-off between the setup costs and inventory holding cost and its goal is to define the optimal timing and quantity of production to minimize the total cost. The lot sizing problem has been extensively studied and applied in real world situations. In this research, we consider the  lot sizing problem with stochastic demand in which there is the possibility of product substitution. \\
The common approach to deal with the stochastic lot sizing problem is trying to minimize cost while satisfying a service level criteria. The $\alpha$ service level is an event-oriented service level which puts limits on the probability of stock outs. This service level is usually defined as a chance constraint. 
%The basic assumption to approximate and solve  these models is that the demand for each product are independent from each other and there is no autocorrelation~\cite{tempelmeier2011column}. 
In this research, we model the stochastic lot sizing problem with substitution and a joint $\alpha$ service level using scenario tree.
%for which there is no need to have these assumptions and it is possible to use any demand scenarios. 
Having multi-stage problem and following the ``dynamic" strategy \cite{bookbinder1988strategies} the setups and production decisions are defined through the planning horizon and updated with regard to the demand realization. 

The challenge of these stochastic models is that with increasing the number of scenarios the solution time will increase extensively and makes it difficult to reach a reasonable solution in a reasonable amount of time. In this research we will use dual based decomposition method to deal with this challenge. 
The contribution of this research can be summarized as follows:

    
\begin{itemize}
\item   
\item Proposing mathematical model for the stochastic dynamic lot sizing problem with joint $\alpha$ service level constraints and firm-driven substitution possibility.
\item Approximating the models as a two-stage stochastic programming using scenario tree in which the first stage decisions are setups and production amounts and the substitution decisions and amount of inventory and backlog are the recourse actions and second stage decisions.
    \item Proposing an efficient policies to solve the resulting model and comparing different policies to solve this problem in a rolling horizon using simulation.
    \item Providing managerial insight based on numerical experiments and scenario generation with different setting.% (as in ~\cite{jiang2017service} and  ~\cite{jiang2017production}) modified for the stochastic lot sizing problem with substitution.
    


\end{itemize}

\section{Literature review}
The related literature of this work can be categorized in two streams. The first part is dedicated to the lot sizing problem with substitution and the second part is dedicated to the chance constraint optimization and its application in lot sizing.

\subsection{Lot sizing problem and substitution}
\subsubsection{Deterministic models}
In the literature, there are two types of substitution, the demand-driven substitution and firm-driven substitution. In the demand driven substitution the customer decides which product to substitute~\cite{zeppetella2017optimal}, while in the firm-driven case, it is the firm which makes the substitution decisions~\cite{rao2004multi}. Hsu et al.~\cite{hsu2005dynamic} studied two different versions of dynamic uncapacitated lot sizing problem with one way substitution, when there is a need for physical conversion before substitution, and when it doesn't require any conversion. The Authors used dynamic programming and proposed heuristic algorithm to solve the problem.  Lang and Domschke~\cite{lang2010efficient} considered the uncapacitated lot sizing problem with general substitution in which a specific class of demand can be satisfied by different products based on a substitution graph. They model the problem as an MILP and proposed a plant location reformulation and some valid inequality for it.

\subsubsection{Stochastic models}
Many research in the field of the stochastic inventory planning have considered the possibility of substitution. While the majority of them investigated the demand driven substitution, a few research considered the firm driven substitutions. In the same vein, Bassok et al.~\cite{bassok1999single} investigated the single period inventory problem with downward substitution and random demand. This model is an extension to newsvendor problem and there is no setup cost in case of ordering. The sequence of decisions is as follows: first, the order quantity for each of the items is defined by the ordering decision, and in the next step when the demand is realized, the allocation decisions are defined. Rao et al.~\cite{rao2004multi} considered a single period problem with stochastic demand and downward substitution, model it as a two stage stochastic programming. Having an initial inventory, their model considers the ordering cost as well.  

Many of the research about the  product substitution with stochastic demand, have investigated the demand driven substitution. In these problems the customer may choose another product, if the original item cannot be found. This is known as ``stockout substitution". Akçay et al.~\cite{akccaycategory} investigated a single period inventory planning problem with substitutable products. Considering the stochastic customer driven and stockout based substitution, their optimization based method, jointly defines the stocking of each product, while satisfying a service level. They adapt the Type II service level or ``fill rate" for each individual product and overall within a category of products.  

Another stream of these researches, consider the possibility of having multiple graded output items from a single input item, which is known as ``co-production"~\cite{ng2012robust}. In these problems, there is a hierarchy in the grade of output items and it is possible to substitute a lower grade item with the ones with higher grade~\cite{bitran1992ordering}. Hsu and Bassok~\cite{hsu1999random} considered the single period production system with random demand  and random yields. Although they didn't mention ``co-production" in their research, but their model define the production amount of a single item and the allocation of its different output items to different demand classes~\cite{hsu1999random}.
Birtan and Dasu~\cite{bitran1992ordering} studied the multi-item, multi-period co-production problem with deterministic demand and random yield, and proposed two approximation algorithm to solve it. The first approximation is based on rolling horizon implementation of the finite horizon stochastic model. For the second approximation, they considered the two periods, two items problem and then applied simple heuristic based on the optimal allocation policy for that, in multi-period setting. Bitran and Leong~\cite{bitran1992deterministic} consider the same problem as Bitran and Dasu~\cite{bitran1992ordering} with the service level constraint. They proposed deterministic near optimal approximations within a fixed planning horizon. To adapt their model to the revealed information, they applied the proposed model using simple heuristics in a rolling planning horizon~\cite{bitran1992deterministic}.  Bitran and Gilbert~\cite{bitran1994co} consider the co-production and random yield in a semiconductor industry and propose heuristic methods to solve it. 
 
\subsection{Chance constraints optimization}

\subsubsection{Stochastic lot sizing problem and joint service level constraints}
Stochastic lot sizing problem with service level constraints has been studied extensively~\cite{tempelmeier2007stochastic}. One of the most famous service levels is the $\alpha$ service level which is an event-oriented service level, and puts limits on the probability of a stock-out. The $\alpha$ service level is presented as a chance constraint and is usually defined for each period and product separately. There are some research which define it jointly over different planning periods. Liu and K{\"u}{\c{c}}{\"u}kyavuz~\cite{liu2018polyhedral} considered the uncapacitated lot sizing problem with joint service level constraint and studied the polyhedral structure of the problem and proposed different valid inequalities and a reformulation for this problem. Jiang et al.~\cite{jiang2017production} considered the same problem with and without pricing decisions. Gicquel and Cheng~\cite{gicquel2018joint} investigated the capacitated version of the same problem, and following the same methodology as Jiang et al.~\cite{jiang2017production} they used a sample approximation method to solve it. This method is a variation of sample average approximation method which is proposed by Luedtke and Ahmed~\cite{luedtke2008sample} to solve model with chance constraints using scenario sets.


\section{Problem definition}
We consider multi-stage stochastic lot sizing problem with the possibility of substitution, in which different decisions are defined sequentially based on the state of the system and the realized demand. This is in line with the  ``dynamic'' strategy that is defined for the stochastic lot sizing problem in which both setup and production decisions are updated when the demand are realized ~\cite{bookbinder1988strategies}. Figure~(\ref{MultistageDynamics}) illustrates the dynamics of decisions  in this model. At each each stage/period when the demand is realized and based on the state of the systems, two sets of decisions are made. The first set includes substitution, inventory, and backlog denoted by $\Es, I, \Bi$, respectively. These variables are defined based on the inventory position, $\hat{\Vi}$, and the amount of backlog $\hat{\Bi}$ coming from the previous period. These two parameters shows the state of the system at beginning of the current period. The rest of the decisions in the current period are production, setup, and inventory position at the end of current period which are denoted by $x, y, v$, respectively.  It should be noted that the last three decisions are used to satisfy the future periods demand. In other words, the production in the current period will not be used for the same period demand. After solving the model, the value of $\Vi$, and $\Bi$ will be the state variables, for the next period. In each period, we should define the decisions such that the joint service level in the next stage is satisfied, considering different demand scenarios. 


%\begin{figure}[!h]
%\begin{center}
%\includegraphics[scale=0.5]{Diagram.png}
%\caption{Dynamics of the multi period model} 
%\label{Diagram}
%\end{center}
%\end{figure}

\begin{figure}[!h]
\begin{center}
\includegraphics[scale=0.6]{MultiSatge.png}
\caption{Dynamics of decisions in multi period model} 
\label{MultistageDynamics}
\end{center}
\end{figure}



%which is adapted from  Lang and Domschke~\cite{lang2010efficient} for deterministic uncapacitated lot sizing problem with substitution. 
%This model considers a general substitution graph in which a demand class can be fulfilled with multiple products based on the substitution graph and at a substitution cost.


\begin{table}[H]
\centering
\caption{Parameters and decision variables for stochastic lot sizing model with substitution}
\begin{adjustbox}{width=1\textwidth,center=\textwidth}
\begin{tabular}{ll}
\toprule
{\bf Sets} & {\bf Definition} \\ \midrule
$\TI$  & Set of planning periods \\ 
$\KA$  & Set of products \\
%$N$  & Set of demand classes\\ 
%$V = K \cup N$  & Vertex set of substitution graph\\
$E \subseteq \KA \times \KA$  & Directed edges of substitution graph denoting feasible substitutions: $(k, j) \in E$ if product $\ka$  can fulfil demand of product $\jey$\\
%$ G = (V,E)$  & Substitution graph \\
$ C_{k} = \{j \mid (k,j) \in E\}$  & Set of products whose demand can be fulfilled by product $\ka$  \\
$ P_{j} = \{k \mid (k,j) \in E\}$  & Set of products that can fulfil the demand of product $j$ \\
$ c_(n) $  & Set of child nodes for node $n$. \\

{\bf Parameters} & {\bf Definition} \\ \midrule
$sc_{\ti \ka}$ & Setup cost for product {\it k} in period {\it t} \\ 
$hc_{\ti \ka}$  & Inventory holding cost for product {\it k} in period {\it t}  \\ 
$stc_{\ti \ka \jey }$  & Substitution cost if product $\ka$  is used to fulfill the demand calss $j$ in period {\it t}  \\ 
$pc_{\ti \ka}$  & Production cost for product {\it k} in period {\it t}  \\
$\alpha$  & Minimum required joint service level \\ 
$bc_{\ti \ka}$  & Backorder cost for product $\ka$  in period $\ti$ \\
$M_{\ti \ka}$  & A sufficiently large number \\ 
$I_{k0}$ & The initial inventory for product $\ka$  \\ 
%${d}_{jt}$  & Demand for class {\it j} in period {\it t} (model input)\\ 
${D}_{\ti \ka}$ & Random demand variable for product {\ka j} in period {\it t}  \\ 
$\hat{\Vi}_{p(n), \ka} $&  The amount of inventory position for product $\ka$ at parent of node $n$\\
$\hat{\Bi}_{p(n), \ka} $&  The amount of backlog for product $\ka$ at the parent of node $n$  \\
{\bf Decision variables} & {\bf Definition} \\ \midrule
$\y_{n \ka}$ & Binary variable which is equal to 1 if there is a setup for product {\it k} at node {\it n}, 0 otherwise \\ 
$\x_{n \ka}$ & Amount of production for product $\ka$  at node{\it n}  \\ 
$\Es_{n \ka \jey}$ & Amount of product $\ka$  used to fulfill the demand of product $\jey$  at  node {\it n}   \\
${I}_{n \ka}$ & Amount of physical inventory for product {\it k} at the end of period for node {\it n}  \\
${\Bi}_{n \ka}$ & Amount of Backlog for product {\it k} at the end of period for node {\it n}  \\
${\Vi}_{n \ka}$ & The inventory position for product {\it \ka} at the end of period for node {\it n}  \\
 \bottomrule
 & ${\cred (C_k \text{ and } P_k  \text{ sets include } k) }$
\end{tabular}
\end{adjustbox}
 \label{tab:Sub_parameters}
\end{table}



 
 





%\begin{flalign}
%&pr((\sum_{t'=1}^{t} \sum_{k\in P_{j}} \overline{S}_{\ka \jey \ti'} -\sum_{t'=1}^{t} \overline{D}_{jt'}) \geq 0~~~~~\forall j \in N ) \geq \alpha_{c} &   \forall \ti  \in \TI,  & \label{eq:Sub_ST_Service}
%\end{flalign}


%Constraints (\ref{eq:Sub_ST_Service}) guarantee the minimum service level.
%These chance constraints ensure that the probability of a stock out is less than (1-$\alpha_{c}$). \\


\subsection{Dynamic programming model}


In this section, we present the dynamic programming model for the stochastic lot sizing problem with a firm-driven substitution and joint service level constraint. Different sets, parameters and decision variables are presented in Table~\ref{tab:Sub_parameters}. Considering the underlying scenario tree, the mathematical model \ref{DynamicProgramming} is proposed for the node {\it n} in the tree. This model is based on the inventory position at the beginning and ending of the planning period correspond to current node {\it n}. At each stage $\hat{\Vi}_{p(n),{\ka}}$ and $\hat{\Bi}_{p(n),{\ka}}$ illustrate the state of the system for the current node. Considering $D_{\ka\ti}$ as the random variable for product $\ka$ in period $\ti$, $\hat{D}_{n \ka}$ is its realization at node $n$. Considering the scenario tree, $D^{\ti \ka}$ is the history of  random demand from period 1 to period $\ti$ and $\hat{D}^{nk}$ is its realization until node $n$.  

%Here, $\Vi_{n{\ka}} =$ current inventory position of product $\ka$. \\
%Note: $I_{n{\ka}}$: the ending inventory of node $n$ for product $\ka$\\
\begin{subequations}
\label{DynamicProgramming}
\begin{flalign}
&F_{n}(\Vi_{p(n)},\Bi_{p(n)}) =  \min  \sum_{\ka \in \KA} ( sc_{\ti_n \ka}\y_{n \ka} +  hc_{\ti_n \ka} {I}_{n \ka}+  \sum_{j\in C_{\ka }}stc_{\ti_n \ka \jey} \Es_{n  \ka \jey}  + bc_{\ti_n \ka} {\Bi}_{n \ka} )& \notag \\ 
& + \sum_{m \in c(n)}  q_{nm} \underbrace{ F_{m}(\Vi_{n}, {\Bi_{n}} ) }_{\text{future}} & \label{eq:Dyn_g2_Sub_ST_Production_Flow} 
\end{flalign}

\begin{flalign}
\text{s.t.} \ & \x_{n \ka} \leq M_{n \ka} \y_{n \ka} \quad  &\forall \ka  \in \KA  & \label{eq:Dyn_Sub_Setup}\\
%& \Zed_{m} = 0 \ \Rightarrow \ m \ \text{subproblem is feasible with } \Vi_{\cdot n} & \label{eq:Dyn_Sub_ST_Service1}\\
%& \sum_{m \in c(n)} q_{nm} \Zed_{m} \leq 1-\alpha, & &     \label{eq:Dyn_Sub_ST_Service}\\*[0.5cm]
%& v^+_{n \ka} - v^-_{n \ka} = I_{n  \ka} + \x_{n  \ka} - \Bi_{n  \ka}  \quad \forall \ka  \in \KA \\
%& \sum_{j\in C_{k}} {S}_{n \ka \jey} \leq v^+_{p(n), \ka} \quad \forall \ka  \in \KA \\
%&  I_{n \ka} - \Bi_{n \ka} = v^+_{p(n), \ka} - v^-_{p(n), \ka} - \sum_{j\in C_{k}} {S}_{n \ka \jey} + \sum_{j\in P_{k}} {S}_{n \jey \ka}  - D_{n \ka} \quad \forall \ka  \in \KA  &     \label{eq:Dyn_F_Sub_ST_Service}&\\
%& I_{n \ka} \leq v^+_{p(n), \ka} - \sum_{j\in C_{k}} {S}_{n \ka \jey} \quad \forall \ka  \in \KA \\*[0.5cm]
%%%% Like Rao's %%%%%
& \sum_{j\in P_{k}} {S}_{n \jey \ka} + \Bi_{n  \ka}  = \hat{D}_{n \ka} + \hat{\Bi}_{p(n), \ka} \quad &\forall \ka  \in \KA \label{eq:Bhat} \\
& \sum_{j\in C_{k}} {S}_{n \ka \jey} + I_{n \ka} = \hat{\Vi}_{p(n), \ka}  \quad &\forall \ka  \in \KA \label{eq:vhat}\\
& \Vi_{n \ka} = I_{n  \ka} + \x_{n  \ka}  \quad &\forall \ka  \in \KA  \label{eq:vdef}
%& v^-_{n \ka} = \Bi_{n  \ka}  \quad \forall \ka  \in \KA \\
\\
& \mathbb{P}_{D^{t_n+1}}\{ (\hat{\Vi}_{n}, \hat{\Bi}_{n}) \in Q(D^{\ti_n+1} )| D^{\ti_n} = \hat{D}^{n} \} \geq \alpha& \label{eq:SL}
& \\*[0.5cm]
%%%%%%%%%%%%%%%%%%%%%%%
& {x}_{ n } \in \mathbb{R}_{+}^{|\KA|} , {v}_{ n } \in \mathbb{R}_{+}^{|\KA|} ,  {I}_{ n } \in \mathbb{R}_{+}^{|\KA|} , {\Bi}_{ n } \in \mathbb{R}_{+}^{|\KA|} , {S}_{n} \in \mathbb{R}_{+}^{|E|} ,{y}_{ n } \in \{0,1\}^{|\KA|} &  & \label{eq:Dyn_F_Sub_ST_bound1}
%\\
%&{S}_{n \ka \jey} \geq 0 & \forall(k,j) \in E  & \label{eq:Dyn_F_Sub_ST_bound3}&
%\\
%& (I_{\cdot n},\Bi_{\cdot n},\Es_{\cdot n}) \in %\mathcal{X}^{\text{rest}}_\ell &&\label{eq:Dyn_F_Sub_ST_bound4}&
\end{flalign}
\end{subequations}
The objective function of the model (\ref{eq:Dyn_g2_Sub_ST_Production_Flow}) is to minimize the current stage total cost plus the expected cost to go function. This cost includes the total setup costs, holding cost, substitution cost, and backlog costs.
Constraints (\ref{eq:Dyn_Sub_Setup}) are the production, set up constraints which guarantee that when there is production, the setup variable is forced to take the value 1, and 0 otherwise. 

Constraints (\ref{eq:Bhat}) show that the demand of each product is satisfied by its own production and the substitution by other product or it will be backlogged to the next stage. In this constraint, $\Es_{\ka \ka n}$ is equal to production of product $k$ which is used to satisfy its own demand. Constraints (\ref{eq:vhat}) show that the inventory of product $\ka$ at the beginning of current period may be used to satisfy its own demand or other products demand through substitution or it will be stored as an inventory at the end of current period. Constraints (\ref{eq:vdef}) are to define the inventory position at the end of current period which is equal to the amount inventory plus the amount of production for the future periods. Constraint~(\ref{eq:SL}) is the joint service level for period $\ti_n+1$. This constraint guarantees that the probability of having a feasible solution with respect to the service level is greater than or equal to $\alpha$.  Constraint~(\ref{eq:Dyn_F_Sub_ST_bound1}) define the domains of different variables in the model. In addition to these constraints it is possible to add different types of constraint such as capacity constraints to the model.
%For example, if the company decide to satisfy a percentage of each product expected demand  by its own production, the following constraint can be added to the model.

{\cred Note:} In the stochastic case with service level, when we want to implement it in a rolling horizon environment, the feasibility of the next stage service level should be guaranteed. If at each stage, there exist an unlimited source for production of each of the products or some of the products that can substitute others this feasibility condition is satisfied.


{\cred Note: In this model, it is not explicit that the recourse stage should be feasible. It is using the convention that if it is infeasible, the value function return infinite cost.}


{\cred * Given stage t and the history, $\xi^t, \exists x^t(\xi^t) \in \chi^t(x^t) \rightarrow v(\xi^t)$\\
$s.t.$ \\
$P_{\xi^{t+1}|\xi^t}\{v^{t}(\xi^t)-\sum_{j\in c_k}\Es_{kj}(\xi^{t+1})+\sum_{j\in p_k}\Es_{jk}(\xi^{t+1}) \geq D^{t+1}(\xi^{t+1}) $ for some $\Es_{kj}(\xi^{t+1}) , \Es_{jk}(\xi^{t+1}) \in  \chi ^{t+1}(\xi^{t+1}) \} \geq \alpha$ \\
This can be satisfied for instance if we have at least one uncapacitated product option (whether by its own production or substitution) for each product.

}


\section{Decision policies at each stage}

We will explain how we are going to define different decisions at each stage. In Figure~\ref{MultistageDynamics} we categorize the decisions into 2 groups. The first group are the decisions to satisfy the current stage demand. These decisions are defined using ``current stage modification" policy. The second group of decision is to define the set up and production decision which cannot be use to satisfy the current stage demand. These decisions will be defined based on the ``production" policies. These policies can be used in a rolling horizon environment. 

 \subsection{Current stage modification policy}
All the policies that mentioned in the previous section, consider the future periods, where there exists stochasticity in demand. The decision in the current period is defined based on different cost parameters, and they are not forced to minimize the amount of backorder and satisfy the demand as much as possible. To overcome this challenge, at the beginning of each iteration of the receding horizon, starting from the second period (the first period backorder is equal to zero), we solve the following linear model which minimize the backlog in the current period. This model is a feasibility problem and based on the result of this model if a product is fully satisfied without any backlog, we force the that backlog in the current stage of the main model equal to zero. It should be noted that this model does not define how the demand should be satisfied. 

In this section we propose different policies to approximate the dynamic programming model proposed in the previous section. This policies can be used in a rolling horizon environment.
\begin{subequations}
\label{Currentstage}

\begin{flalign}
&\min  \sum_{k \in K}  {B}_{ t_n \ka} & \label{eq:Current_obj} 
\end{flalign}
 subject to:
\begin{flalign}
  &  \sum_{j\in P_{k}} {S}_{\ti_n  \jey \ka} + B_{t_n \ka}  = \hat{D}_{n \ka} + \hat{B}_{t_n-1, \ka} &\forall k \in K  &     \label{eq:Current_inventory_tn}&\\
&  \sum_{j\in C_{k}} {S}_{\ti_n \ka \jey} + I_{ t_n \ka} = \hat{v}_{t_n-1 , \ka} &\forall k \in K  &     \label{eq:Current_Orderup}&\\
& {I}_{ \ti_n } \in \mathbb{R}_{+}^{|\KA|} , {B}_{ \ti_n } \in \mathbb{R}_{+}^{|\KA|} , {S}_{\ti_n} \in \mathbb{R}_{+}^{|E|} &    & \label{eq:Sub_FD_bound2}\\
 \label{eq:Sub_FD_bound3}
\end{flalign}
\end{subequations}
If the optimal value of  $B_{ t_n \ka}$  is equal to zero the constraint $B_{ t_n \ka} = 0$ will be added to the main model. How the demand of this product is satisfied is defined in the main model based on different cost parameters.\\
It should be noted that it is also possible to use other policies for the current stage decision. The advantage of this method is that it can be used for all the ``production" policies which are explained in the next section.

\subsection{The production policy}
At each stage, on the current state of the system we will apply a ``production" policy, Appr$(n,\hat{v}_{t_n-1 ,k}, \hat{B}_{t_n-1 , \ka})$, which is an approximation for the dynamic programming model (\ref{DynamicProgramming}). In this section we will explain 2 deterministic policy, and policy base on the chance constraint. This policy can be a deterministic policy or any other approximation policies. 
 
\subsubsection{Deterministic policies}

To evaluate the value of the  stochastic  model, we can compare it against the deterministic models applied in a rolling horizon environment.
In the first case, we can substitute the stochastic demand with the average demand, and solve the deterministic lot sizing model. This model can be solved in a receding horizon environment, in which the first period/stage decision is fixed and the information such as initial inventory and backlog will be updated based on the realized demand. 
This procedure repeated until the last period. 
\begin{table}[H]
\centering
\caption{Parameters and decision variables for $(t_n)^{th}$ period}
\begin{adjustbox}{width=1\textwidth,center=\textwidth}
\begin{tabular}{ll}
\toprule
{\bf Parameters} & {\bf Definition} \\ \midrule
$\hat{B}_{\ti_n-1, \ka}$  & The fixed amount of production for product $k$ in period $t$, $t<t_n$\\
$\hat{v}_{\ti_n-1, \ka}$  & The inventory position of product $k$ in period $t$, $t<t_n$ \\
$P_{i}(\overline{D}_{\ti \ka})$& is the $i-th$ percentile based on the demand the over all scenarios\\
{\bf Decision variables} & {\bf Definition} \\ \midrule
$y_{\ti \ka}$ & Binary variable which is equal to 1 if there is a setup for product {\it k} in period {\it t}, 0 otherwise \\ 
$x_{\ti \ka}$ & Amount of production for product $k$ in period $t$ \\ 
$s_{\ti \ka \jey}$ & Amount of product $k$ used to fulfill the demand of class $j$  in period $t$   \\
${I}_{\ti \ka}$ & Amount of physical inventory for product {\it k} at the end of period {\it t}  \\ \bottomrule
\end{tabular}
\end{adjustbox}
 \label{tab:Sub_FD_parameters}
\end{table}


Model Appr$(n,\hat{v}_{t_n-1 , \ka}, \hat{B}_{t_n-1 , \ka})$
\begin{subequations}
\label{mod:Det}

\begin{flalign}
&\min \sum_{t =t_n }^{T} \sum_{k \in K} ( sc_{\ti \ka}y_{\ti \ka} + pc_{\ti \ka}x_{\ti \ka}+ \sum_{j\in C_{k}}stc_{\ti \ka \jey} S_{\ti \ka \jey}  + hc_{\ti \ka}I_{\ti \ka} + bc_{\ti \ka} {B}_{\ti \ka} )& \label{eq:Sub_Roll_obj} 
\end{flalign}
 subject to:
\begin{flalign}
&x_{\ti \ka} \leq M_{\ti \ka} y_{\ti \ka} &  \forall t  \in T , t \geq t_n  , \forall k \in K & \label{eq:Sub_FD_Setup}\\
  &  \sum_{j\in P_{k}} {S}_{\ti_n  \jey \ka} + B_{t_n \ka}  = \hat{D}_{n \ka} + \hat{B}_{t_n-1, \ka} &\forall k \in K  &     \label{eq:Det_inventory_tn}&\\
   &  \sum_{j\in P_{k}} {S}_{\ti \jey \ka} + B_{k t}  = E_{|n}[\overline{D}_{\ti \ka}] + {B}_{k,t -1} &\forall t \ \in T, t > t_{n},\forall k \in K  &     \label{eq:Det_inventory}& \\
&  \sum_{j\in C_{k}} {S}_{\ti_n \ka \jey} + I_{ t_n \ka} = \hat{v}_{t_n-1 , \ka} &\forall k \in K  &     \label{eq:Det_inventory}&\\
&  \sum_{j\in C_{k}} {S}_{\ti \ka \jey} + I_{\ti \ka} = v_{k,t-1} &\forall t \ \in T, t > t_{n},\forall k \in K  &     \label{eq:Det_inventory_tn}&\\
%&  I_{ t_n \ka} - B_{ t_n \ka} = \hat{v}_{t_n-1 , \ka} - \sum_{j\in C_{k}} {S}_{\ti_n \ka \jey} + \sum_{j\in P_{k}} {S}_{\ti_n  \jey \ka}  - \hat{D}_{n \ka} \quad &\forall k \in K  &     \label{eq:Det_inventory_tn}&\\
%& v_{ t_n \ka} = I_{ t_n \ka} + x_{ t_n \ka}  \quad &\forall k \in K  &     \label{eq:Det_inventoryPos_tn}& \\
%&  I_{\ti \ka} - B_{\ti \ka} = v_{k,t-1} - \sum_{j\in C_{k}} {S}_{\ti \ka \jey} + \sum_{j\in P_{k}} {S}_{\ti \jey \ka}  - E_{|n}[\overline{D}_{\ti \ka}] \quad &\forall t \ \in T, t > t_{n},\forall k \in K  &     \label{eq:Dyn_F_Sub_ST_Service}&\\
& v_{\ti \ka} = I_{\ti \ka} + x_{\ti \ka}  \quad &\forall t \in T, t \geq t_n, \forall k \in K \\
%&\sum_{k \in K} (st_{\ti \ka}y_{\ti \ka} + pt_{\ti \ka}x_{\ti \ka}) \leq Cap_{t} & \forall t  \in T ,  t\geq t_n   & \label{eq:Sub_FD_Capacity}  \\
&\Bi_{\ti_n+1 \ka} =0 & \forall k \in K &\label{eq:Sub_FD_base_bin}\\
&y_{\ti \ka} \in \{0, 1\} & \forall t  \in T,  t \geq t_n ,\forall k \in K &\label{eq:Sub_FD_base_bin}\\
&x_{\ti \ka}  \geq 0 &  \forall t  \in T, t\geq t_n, \forall k \in K  & \label{eq:Sub_FD_bound1}\\
& I_{\ti \ka} , B_{\ti \ka} \geq 0 &  \forall t  \in T,  t \geq t_n , \forall k \in K  & \label{eq:Sub_FD_bound2}\\
&S_{\ti \ka \jey} \geq 0 &  \forall t  \in T,  t \geq t_n , \forall(k,j) \in E  & \label{eq:Sub_FD_bound3}
\end{flalign}


  {\cred{
\begin{flalign}
  &  M_{\ti \ka} = \sum_{j\in C_{k}}( \sum_{t \leq t_n}{d}_{kn} + \sum_{t >t_n}E_{|n}[\overline{D}_{\ti \ka}])  &t> t_n ,\forall k \in K  &     \label{eq:BigM_Deterministic_Appr}&
  \end{flalign}}}
\end{subequations}

\subsubsection{Two-stage (chance constraint) policy}

\begin{subequations}
\label{mod:ExtnSL}

\begin{flalign}
\min &
\sum_{k \in K} ( sc_{ t_n \ka}y_{ t_n \ka} + pc_{ t_n \ka}x_{ t_n \ka}+ \sum_{j\in C_{k}}stc_{\ti_n \ka \jey} S_{\ti_n \ka \jey}  + hc_{ t_n \ka}I_{ t_n \ka} + bc_{ t_n \ka} {B}_{ t_n \ka} ) + \notag \\
&\sum_{k \in K} ( sc_{ t_n \ka}y_{k,t_n+1} + pc_{ t_n \ka}x_{k,t_n+1}+  hc_{k,t_n+1}I_{k,t_n+1} + bc_{k,t_n+1} {B}_{k,t_n+1} + \frac{1}{|M|} \sum_{m \in M} \sum_{j \in C_k} stc_{k,j,t_n+1} S_{k,j,t_n+1,m} ) + \notag \\
& \sum_{t =t_n+2 }^{T} \sum_{k \in K} ( sc_{\ti \ka}y_{\ti \ka} + pc_{\ti \ka}x_{\ti \ka}+ \sum_{j\in C_{k}}stc_{\ti \ka \jey} S_{\ti \ka \jey}  + hc_{\ti \ka}I_{\ti \ka} + bc_{\ti \ka} {B}_{\ti \ka} ) & \label{eq:Sub_Roll_obj_ext} 
\end{flalign}
 subject to:
\begin{flalign}
  &  \sum_{j\in P_{k}} {S}_{\ti_n  \jey \ka} + B_{t_n \ka}  = \hat{D}_{n \ka} + \hat{B}_{t_n-1, \ka} &\forall k \in K  &     \label{eq:Det_backorder_tn}&\\
  &  \sum_{j\in C_{k}} {S}_{\ti_n \ka \jey} + I_{ t_n \ka} = \hat{v}_{t_n-1 , \ka} &\forall k \in K  &     \label{eq:Det_inventory_tn}&\\
  &  \sum_{j\in P_{k}} {S}_{j,k,t_n+1,m} + B'_{k,t_n+1,m}  = d_{km} + {B}_{k,t_n} &\forall k \in K, \forall m \in M &     \label{eq:Det_backorder_tnp}& \\
&  \sum_{j\in C_{k}} {S}_{k,j,t_n+1,m} + I'_{k,t_n+1,m} = v_{k,t_n} & \forall k \in K, \forall m \in M  &     \label{eq:Det_inventory_tnp}&\\
   &  \sum_{j\in P_{k}} {S}_{\ti \jey \ka} + B_{k t}  = E_{|n}[\overline{D}_{\ti \ka}] + {B}_{k,t -1} &\forall t \ \in T, t > t_{n}+1,\forall k \in K  &   \label{eq:Det_backorder_ext}& \\
&  \sum_{j\in C_{k}} {S}_{\ti \ka \jey} + I_{\ti \ka} = v_{k,t-1} &\forall t \ \in T, t > t_{n}+1,\forall k \in K  &     \label{eq:Det_inventory_ext}&\\
& \frac{1}{|M|} \sum_{m \in M} I'_{k,t_n+1,m} = I_{k,t_n+1} & \forall k \in K & \label{eq:Average_Inventory} \\
& \frac{1}{|M|} \sum_{m \in M} B'_{k,t_n+1,m} = B_{k,t_n+1} & \forall k \in K & \label{eq:Average_Backlog}\\
%&  I_{ t_n \ka} - B_{ t_n \ka} = \hat{v}_{t_n-1 , \ka} - \sum_{j\in C_{k}} {S}_{\ti_n \ka \jey} + \sum_{j\in P_{k}} {S}_{\ti_n  \jey \ka}  - \hat{D}_{n \ka} \quad &\forall k \in K  &     \label{eq:Det_inventory_tn}&\\
%& v_{ t_n \ka} = I_{ t_n \ka} + x_{ t_n \ka}  \quad &\forall k \in K  &     \label{eq:Det_inventoryPos_tn}& \\
%&  I_{\ti \ka} - B_{\ti \ka} = v_{k,t-1} - \sum_{j\in C_{k}} {S}_{\ti \ka \jey} + \sum_{j\in P_{k}} {S}_{\ti \jey \ka}  - E_{|n}[\overline{D}_{\ti \ka}] \quad &\forall t \ \in T, t > t_{n},\forall k \in K  &     \label{eq:Dyn_F_Sub_ST_Service}&\\
& v_{\ti \ka} = I_{\ti \ka} + x_{\ti \ka}  \quad &\forall t \in T, t \geq t_n, \forall k \in K \\
&x_{\ti \ka} \leq M_{\ti \ka} y_{\ti \ka} &  \forall t  \in T , t \geq t_n  , \forall k \in K & \label{eq:Sub_FD_Setup}\\
%&\sum_{k \in K} (st_{\ti \ka}y_{\ti \ka} + pt_{\ti \ka}x_{\ti \ka}) \leq Cap_{t} & \forall t  \in T ,  t\geq t_n   & \label{eq:Sub_FD_Capacity}  \\
&y_{\ti \ka} \in \{0, 1\} & \forall t  \in T,  t \geq t_n ,\forall k \in K &\label{eq:Sub_FD_base_bin}\\
&x_{\ti \ka}  \geq 0 &  \forall t  \in T, t\geq t_n, \forall k \in K  & \label{eq:Sub_FD_bound1}\\
& I_{\ti \ka} , B_{\ti \ka} \geq 0 &  \forall t  \in T,  t \geq t_n , \forall k \in K  & \label{eq:Sub_FD_bound2}\\
&S_{\ti \ka \jey} \geq 0 &  \forall t  \in T,  t \geq t_n , \forall(k,j) \in E  & \label{eq:Sub_FD_bound3}
\end{flalign}

  \end{subequations}
  
 To have a more clear description, the objective function of the extensive form (\ref{eq:Sub_Roll_obj_ext}) has broken into three parts, the cost of current period $\ti_n$, the cost of period $\ti_n+1$, and the cost of periods $t_n+2$ to $\TI$. Constraints (\ref{eq:Det_backorder_tn}) to (\ref{eq:Det_inventory_ext}) are the inventory, backlog, and substitution balance constraints. Constraints (\ref{eq:Det_backorder_tn}) and (\ref{eq:Det_inventory_tn}) are for $\ti_n$ period, constraints (\ref{eq:Det_backorder_tnp}) and (\ref{eq:Det_inventory_tnp}) are for $\ti_n+1$ period, and constraint (\ref{eq:Det_backorder_ext}) and (\ref{eq:Det_inventory_ext}) are for the periods $\ti_n+2$ to $\TI$. Constraints (\ref{eq:Average_Inventory}) and (\ref{eq:Average_Backlog}), define the average inventory and backlog for period $\ti_n+1$, which is equal to the average inventory and backlog over all the child nodes at this period. These averages will be used in the inventory and backlog balance constraints in period $\ti_n+2$.
 
  \subsubsection{Benders cuts to project out the second stage substitution variables}
  
 To reduce the size of the master problem, we use Benders cuts to eliminate the substitution variables $S_{j,k,t_n+1,m}$ in the next immediate stage. For each $m \in M$, introduce a variable $\theta_m$ to represent the substitution cost in child node $m \in M$. Then, the term 
 \[ \sum_{k \in K} \frac{1}{|M|} \sum_{m \in M} \sum_{j \in C_k} S_{k,j,t_n+1,m} \]
 in the objective is replaced by the term:
 \[ \frac{1}{|M|} \sum_{m \in M} \theta_m. \]

The constraints \eqref{eq:Det_backorder_tnp} and \eqref{eq:Det_inventory_tnp} are eliminated from the master model. 

Given a master problem solution, say with values $\overline{B}'_{k,t_n+1,m}$,$\bar{I}'_{k,t_n+1,m}$,$\bar{B}_{ t_n \ka}$, $\bar{v}_{k,t_n}$, $\bar{\theta}_m$ we solve the following subproblem for each $m \in M$:



    
  \subsubsection{Chance constraint policy}
  Another approach to approximate this problem is to enforce the service level directly in the next planning period. The MIP model for this problem is as follows in which the constraint (\ref{eq:Dyn_Sub_ST_Service1}) is represented as constraints (\ref{eq:Backorder_Child} to \ref{eq:Child_Service}), in which $M'_{km}$ is defined by (\ref{eq:BigM_Child}).
  The $Bc_{km}$ represents the backlog at the child node $m$.
  \begin{flalign}
  & Bc_{km} +\sum_{j \in P_k} Sc_{jkm} = D_{km}  +\Bi_{n \ka}  & \forall m \in M, \forall \ka  \in \KA& \label{eq:Backorder_Child}\\
  & \sum_{j \in C_k} Sc_{kjm} \leq \Vi_{n \ka}   & \forall m \in M, \forall \ka  \in \KA& \label{eq:OrderUptoLevel_Child}\\
&  Bc_{km} \leq M'_{km}\Zed_{ m}  & \forall m \in M , \forall \ka  \in \KA &     \label{eq:Child_Service}
 \end{flalign}
 
 \begin{flalign}
 &  M'_{km}=  \hat{\Bi}_{k,\ti_n -1} +d_{n \ka} + D_{km} & \forall m \in M , \forall \ka  \in \KA &     \label{eq:BigM_Child}
 \end{flalign}
 In addition to mathematical model, we will propose a branch and cut algorithm to solve this problem, which is explained in the next section.


  
  \section{Branch and cut algorithm~\cite{luedtke2014branch} }

In this section, I will work with the simpler formulation that just has a single level,  so the binary variables are $\Zed_m$, where $\Zed_m=0$ indicates the current inventory position is adequate to meet demands in child node $m$ without backordering, and $\Zed_m=1$ otherwise, for $m \in c(n)$. When $\Zed_m=0$, we should enforce that $\Vi_{\cdot n}, \Bi_{\cdot n}$ lie within the polyhedron:
\begin{align*} Q_m := \{ (v,B) :  \exists \Es_m \geq 0 \ \text{s.t.} \ 
 & \sum_{j \in P_k} \Es_{jkm} = D_{km} + \Bi_k \ \forall \ka  \in \KA \\
 & \sum_{j \in C_k} \Es_{kjm} \leq \Vi_k \ \forall \ka  \in \KA \}
 \end{align*}
 
\newcommand{\vsol}{\hat{\Vi}}
\newcommand{\zsol}{\hat{\Zed}}
\newcommand{\bsol}{\hat{\Bi}}
\newcommand{\pisol}{\hat{\pi}}
\newcommand{\betasol}{\hat{\beta}}

Assume we have solved a master problem at node $n$, and have obtained a master problem solution $(\zsol,\vsol,\bsol)$. Note that this solution may or may not satisfy the integrality constraints (e.g., if we have solved an LP relaxation of the master problem). Given a child node $m \in c(n)$ with $\zsol_m < 1$, our task is to assess if $(\vsol,\bsol) \in Q_m$, and if not, attempt to generate a cut to remove this solution. In the case of an integer feasible solution, we will always be able to do so when $(\vsol,\bsol) \notin Q_m$.

We can test if given $(\vsol,\bsol) \in Q_m$ by solving the following linear program:
\begin{align*}
\Vi_m(\vsol,\bsol) :=  \min_{w,\Es_m} \ & \sum_{\ka  \in \KA} w_k \\
    \text{subject to: } & \sum_{j \in P_k} \Es_{jkm} + w_k = D_{km} + \bsol_k \ \forall \ka  \in \KA (\pi_k) \\
    &-\sum_{j \in C_k} \Es_{\ka jm} \geq -\vsol_k \ \forall \ka  \in \KA (\beta_k) \\
    & w \geq 0, S \geq 0
\end{align*}
By construction, $(\vsol,\bsol) \in Q_m$ if and only if $\Vi_m(\vsol,\bsol) \leq 0$. Furthermore, if $(\pisol,\betasol)$ is an optimal dual solution, then by weak duality, the cut:
 \[ \sum_{\ka  \in \KA} \pisol_k (D_{km} + \Bi_k) - \sum_{\ka  \in \KA} \betasol_k \Vi_k \leq 0 \]
is a valid inequality for $Q_m$. Rearranging this, it takes the form:
\[ \sum_{\ka  \in \KA} \betasol_k \Vi_k - \sum_{\ka  \in \KA} \pisol_k \Bi_k  \geq  \sum_{\ka  \in \KA} \pisol_k D_{km}. \]
If $\Vi_m(\vsol,\bsol) > 0$ then the corresponding cut will be violated by $(\vsol,\bsol)$. 

The inequality derived above is only valid when $\Zed_m = 0$. We thus need to modify it to make it valid for the master problem.
To derive strong cuts based on this base inequality, we solve an additional set of subproblems once we have coefficients $(\pisol,\betasol)$. In particular, for every child node $m'$, we solve:
\begin{align*}
h_{m'}(\pisol,\betasol) := \min_{v,B,\Es_{m'},\Es_n} \ & \sum_{\ka  \in \KA} \betasol_k \Vi_{n \ka} - \sum_{\ka  \in \KA} \pisol_k \Bi_{n  \ka} \\
\text{subject to: } &  \sum_{j \in P_k} \Es_{jkm'} = D_{km'} + \Bi_{n  \ka} \ &\forall \ka  \in \KA  \\
    &\sum_{j \in C_k} \Es_{kjm'} \leq \Vi_{n \ka} \ &\forall \ka  \in \KA \\
    & \sum_{j\in P_{k}} {S}_{n \jey \ka} + \Bi_{n  \ka}  = D_{n \ka} + \Bi_{p(n), \ka} \quad &\forall \ka  \in \KA \\
    & \sum_{j\in C_{k}} {S}_{n \ka \jey} \leq \Vi_{p(n), \ka}  \quad &\forall \ka  \in \KA\\
    & \Es_m \geq 0, \Es_n \geq 0
\end{align*}
Note that in this problem we consider substitution variables both for the current node $n$ and for the child node under consideration, $m'$. The substitution variable for the child node $m'$ are to enforce that $(v,B) \in Q_{m'}$. The substitution variables for the current node $n$ are to enforce that $B$ satisfies the current node constraints. Note that we could also require $v$ to satisfy the current node constraints, by introducing the $I_{n \ka}$ and $\x_{n \ka}$ variables and including the constraints \eqref{eq:vdef}, and any constraints on the $x$ variables, such as capacity constraints. However, if our test instances do not have capacity restrictions, this would not likely to be helpful, so I have left this out for simplicity.

After evaluating $h_{m'}(\pisol,\betasol)$ for each $m' \in c(n)$, we then sort the values to obtain a permutation $\sigma$ of $c(n)$ which satisfies:
\[ h_{\sigma_1}(\pisol,\betasol) \geq h_{\sigma_1}(\pisol,\betasol)  \geq \cdots \geq h_{\sigma_{|c(n)|}(\pisol,\betasol)}
 \]
Then, letting $p = \lfloor (1-\alpha) N \rfloor$, the following inequalities are valid for the master problem:
\[ \sum_{\ka  \in \KA} \betasol_k \Vi_k - \sum_{\ka  \in \KA} \pisol_k \Bi_k + 
(h_{\sigma_1}(\pisol,\betasol) - h_{\sigma_i}(\pisol,\betasol))\Zed_{\sigma_1} + 
(h_{\sigma_i}(\pisol,\betasol) - h_{\sigma_{p+1}}(\pisol,\betasol))\Zed_{\sigma_i} \geq h_{\sigma_1}(\pisol,\betasol), \quad i=1,\ldots, p \]
Any of these inequalities could be added, if violated by the current solution $(\zsol,\vsol,\bsol)$.

The final step for obtaining strong valid inequalities is to search for {\it mixing inequalities}, which have the following form. Given a subset $T = \{t_1,t_2,\ldots,t_{\ell}\} \subseteq \{\sigma_1,\sigma_2,\ldots,\sigma_p\}$, the inequality:
\[  \sum_{\ka  \in \KA} \betasol_k \Vi_k - \sum_{\ka  \in \KA} \pisol_k \Bi_k + 
\sum_{i=1}^{\ell} (h_{t_i}(\pisol,\betasol) - h_{t_{i+1}}(\pisol,\betasol))\Zed_{t_i}
\geq  h_{t_1}(\pisol,\betasol) \]
is valid for the master problem. Although the number of such inequalities grows exponentially with $p$, there is an efficient algorithm for finding a most violated inequality for given $(\zsol,\bsol,\vsol)$. 

\begin{algorithm}[H]
\SetAlgoLined
{OUTPUT: A most violated mixing inequality defined by ordered index set $\ti$ } \;
INPUT: $\zsol_{\sigma_i}, \sigma_i, h_{\sigma_i}(\pisol,\betasol)$, $i=1,\ldots,p+1$  {\cred is it p or p+1}\;
Sort the $\zsol_{\sigma_i}$ values to obtain permutation $\rho$ of the indices satisfying:
$\zsol_{\rho_1} \leq \zsol_{\rho_2} \leq \cdots \leq \zsol_{\rho_{p+1}} $ \;
$v \gets h_{\sigma_{p+1}}(\pisol,\betasol)$\;
$T \gets \{ \}$\;
$ i \gets 1$\;
\While{$v < h_{\sigma_1}(\pisol,\betasol)$}{
  \If{$h_{\rho_i}(\pisol,\betasol) > v$}{
  $T \gets T \cup \{\rho_i\}$\;
  $v \gets h_{\rho_i}(\pisol,\betasol)$\;  }
  $i \gets i+1$\;
}

\end{algorithm}

{\bf Potential faster cut generation.} Given the structure of this problem,  we can obtain potentially weaker cuts, but saving significant work, by using $h_{m'}(\pisol,\betasol) = \sum_{\ka  \in \KA} \pisol_k D_{km'}$ for each $m' \in c(n)$, instead of solving the above defined linear program. This should be valid because the dual feasible region of the set $Q_{m'}$ is independent of $m'$, so a dual solution from one $m$ can be used to define an inequality valid for any other $m'$.



  

\section{Computational experiments}
\subsection{Rolling horizon implementation}

Considering the scenario tree,  at each stage, we solve two-stage approximation for the multistage problem over
subtree with initial values of $(n,\hat{\Vi}_{\ti_n-1 , \ka},\hat{\Bi}_{\ti_n-1 , \ka})$.

\begin{algorithm}[]
\SetAlgoLined
{OUTPUT: The confidence interval of the total cost and the service level} \;
INPUT: \
 A large scenario set $S$, a solution including the $\hat{x}, \hat{y}$ \;
 Current Solution$_0 =0$ \;
 \While{$s \leq |S|$ }{
  Solve the  model with $\hat{x}, \hat{y}$ fixed in the model and deterministic demand $D^s$\;
  Let $v^*_{t_n \ka}, \hat{x}_{\ti \ka}, \hat{y}_{\ti \ka}, S^{*}_{\ti \ka},I^{*}_{\ti \ka},B^{*}_{\ti \ka}$ be the resulting solution\;
  Let $TotObj = TotObj + Obj^*_s$ \;
  \If{$B^{*}_{ t_n \ka} \geq 0$}{  
  $Z_t = Z_t+1$ }
  $s \gets s+1$ \;}
  Average Cost $=TotObj/ |S|$ \;
  Average Service Level$_t = Z_t/|S|$
  \caption{Evaluation process}
\end{algorithm}


\begin{algorithm}[]
\SetAlgoLined
{OUTPUT: The confidence interval of the total cost and the service level} \;
INPUT: \
 A demand simulation over $\TI_{Sim}$ periods.  \;
 $n =1,\hat{\Vi}_{\ti_0} =0,\hat{\Bi}_{\ti_0} = 0$ \;
 \While{$n \leq \TI_{Sim}$ }{
 Solve the Current stage modification model, and fix the the selected $\Bi_{\ti_n}$ variables equal to 0\;
  Solve the  model Appr$(n,\hat{v}_{t_n-1 , \ka}, \hat{B}_{t_n-1 , \ka})$ and the realized demand $d_{\ti_n}$\;
  Let $v^*_{ t_n}, x^{*}_{t_n}, y^{*}_{t_n}, S^{*}_{\ti_n},I^{*}_{\ti_n},B^{*}_{\ti_n}$ be the resulting solution for period $\ti_n$, and the $Obj_{\ti_n}$ be the total cost of period $\ti_n$ based on the optimal solution \;
  \If{$B^{*}_{\ti_n} \geq 0$}{  
  $Z_{\ti_n} = 1$ }
  $n \gets n+1$ \;}
  Average Cost $=\sum_{n= n_{warm}}^{\TI_{Sim}}Obj_{\ti_n}/ |T_{Sim}-n_{warm}|$ \;
  Average Service Level$=\sum_{n= n_{warm}}^{\TI_{Sim}}Z_{\ti_n}/ |T_{Sim}-n_{warm}|$
  \caption{Rolling horizon implementation}
\end{algorithm}

\subsection{Data generation}

To test the policies, and the algorithm we generate many instances based on Rao et al.~\cite{rao2004multi} and Helber et al.~\cite{helber2013dynamic} with some justification for the current problem. Table~\ref{tab:Sub_FD_parameters} illustrates different parameters in the model and how to define them based on data generation parameters. In the base case one way substitution is available for 4 consequent products ordered based on their values. It should be noted that the backlog cost for the next immediate period is calculated based on equation??? considering both substitution cost and the backlog cost. Table \ref{tab:BaseSensitivity} summarize the data generation parameters, their base value and their variation for sensitivity analysis. 




\begin{table}[H]
\centering
\caption{Data generation }
%\footnotesize
\begin{tabular}{ll}
\toprule
%{\bf Sets} &  \\ \midrule
%$\ti$   & Set of planning periods \\ 
%$\ka$   & Set of products  \\ 
%$G$  & Substitution graph (one way substitution) \\
{\bf Parameters} &  \\ \midrule
$pc_{\ti \ka}$  & $1+\eta \times(|\KA|-\ka)$   , $\eta = 0.1, 0.2 , 0.5$ \\
$stc_{\ti \ka \jey }$  & $\max(0,(1+\tau) \times (pc_{jt} - pc_{\ti \ka}))$ , $\tau = 0 , 0.5 , 1$  \\ 
$hc_{\ti \ka}$  & $\rho \times pc_{\ti \ka} $ , $\rho = 0.005, 0.01, 0.02, 0.05 ,0.8$   \\ 
$TBO$  &  $TBO = 1, 2, 4$   \\ 
$sc_{\ti \ka}$ & $E[\overline{D_{\ti \ka}}] \times TBO^2 \times hc_{\ti \ka} /2$ \\ 
$bc_{\ti \ka}$  &  $\pi \times hc_{\ti \ka}$ \\
$SL$  &  $ 75\%, 80\%, 90\%, 95\%, 99\%$ \\
${d}_{\ti \ka}$  & Generated based on AR procedure
 \\ \bottomrule
\end{tabular}
 \label{tab:Sub_FD_parameters}
\end{table}





\begin{table}[H]
\centering
%\footnotesize
\caption{ Parameters for the base case and the sensitivity analysis} \label{tab:BaseSensitivity}
%\begin{adjustbox}{width=1\textwidth,center=\textwidth}
\begin{tabular}{lll}
\toprule
{\bf Parameters} & Base Case & Variation \\ \midrule
$|\TI|$   & 6 & 4, 6 , 8 , 10\\ 
$|\KA|$   & 10 & 5, 10, 15, 20\\ 
$\eta$  &   0.2 & 0.1, 0.2 , 0.5   \\ 
$\tau$  &   0.5 & 0.1, 0.5 , 0.0   \\ 
$\rho $  &   0.05 & 0.02, 0.05, 0.1 , o.2 , 0.5   \\ 
$ TBO $  &   1 & 1.5, 1, 2, 4   \\ 
$ SL $  &95\% & 80\%, 90\%, 95\%, 99\%  \\ 
$ \pi $  &   2 & 0, 2, 10  \\ 
\bottomrule 
\end{tabular}
%\end{adjustbox}
\end{table}

\subsubsection{AR procedure for demand generation}

To generate the random demand we used autoregressive process model which consider the correlation in different stage demand as follows~\cite{jiang2017production} as (\ref{eq:AR1}) where $C, AR_1,$ and $AR_2$ are parameters of the model, and $\epsilon_{\ka \ti+1}$ is a random noise with normal distribution with the mean of 0 and 1 standard deviation. 

 
{\cred{
\begin{flalign}
  &  d_{\ka \ti+1} = C + AR_1 \times d_{\ti \ka} + AR_2 \times \epsilon_{\ka \ti+1}   &\forall \ka  \in \KA , \forall \ti \in \TI  &     \label{eq:AR1}&
  \end{flalign}}}
  
  In our data sets, $C = 20$, $AR_1 = 0.8$, and $AR2 = 0.1 \times 100$. With these data the expected demand for each product in each period is equal to 100.  
  
  


{\cred{As we have no production in the first period without lose of generality we assume that the demand in the first period is equal to zero, other wise the if there is no initial inventory, the service level constraint wont be satisfied. In the AR data generation procedure we start with the defined average in the first period and then follow the procedure for the rest of the periods. Then to make the first period demand equal to 0 we subtract this average from the first period demand.}}\\
Demand generation options:


\subsection{Evaluation algorithm}



\begin{algorithm}[]
\SetAlgoLined
{OUTPUT: The confidence interval of the total cost and the service level} \;
INPUT: \
 A large scenario set $S$, a solution including the $\hat{x}, \hat{y}$ \;
 Current Solution$_0 =0$ \;
 \While{$s \leq |S|$ }{
  Solve the  model with $\hat{x}, \hat{y}$ fixed in the model and deterministic demand $D^s$\;
  Let $v^*_{k \ti_n}, \hat{x}_{\ti \ka}, \hat{y}_{\ti \ka}, S^{*}_{\ti \ka},I^{*}_{\ti \ka},B^{*}_{\ti \ka}$ be the resulting solution\;
  Let $TotObj = TotObj + Obj^*_s$ \;
  \If{$B^{*}_{\ti_n \ka} \geq 0$}{  
  $\Zed_t = \Zed_t+1$ }
  $s \gets s+1$ \;}
  Average Cost $=TotObj/ |S|$ \;
  Average Service Level$_t = \Zed_t/|S|$
  \caption{Evaluation process}
\end{algorithm}



Algorithm (\ref{Al:PolicyEvaluation}) illustrates the steps for the a policy evaluation in which for each scenario the problem is solved using specific policy and then the evaluations measures over all the scenarios are calculated.  



\begin{algorithm}[]
\SetAlgoLined
{OUTPUT: The confidence interval of the total cost and the service level} \;
INPUT: \
 A large scenario set $S$, a policy (for example average demand or quantile)\;
%Current Solution$_0 =0$ \;
 \While{$s \leq |S|$ }{
  Solve the  model with Algorithm 1 using scenario $s$ and the selected policy\;
  Let $v^*_{..s}, {x}^*_{..s}, {y}^*_{..s}, S^{*}_{..s},I^{*}_{..s},B^{*}_{..s}$ be the resulting solution and $Obj^{*}_{s}$ is the resulting objective function\;
  $TotObj = TotObj + Obj^*_s$ \;
  \For{t in T}{
  \If{$\max_{k}(B^{*}_{.ts} > 0)$}{  
  $\Zed_t = \Zed_t+1$ }}
  $s \gets s+1$ \;}
  Average Cost $=TotObj/ |S|$ \;
  Standard Deviation = $std(Obj^{*})$\;
  Average Service Level$_t = \Zed_t/|S|$
  \caption{Evaluation process for a policy}
  \label{Al:PolicyEvaluation}
\end{algorithm}


\subsection{Numerical experiments}
\subsubsection{Methodology evaluation}
In this section, we analyse the efficiency of different methodologies used to solve the chance constraint against the extensive form formulation for example based on the number of scenarios. To this end, we solve only one stage of the problem with three different methods: extensive form (Big-M), strong branch and cut (labeled as Strong $B\& C$), and fast branch and cut (labeled as Fast $B\&C$). As the first period demand is equal to 0, we decided to use the second period model. To have a fair comparison, we solved the first period model Fast $B\&C$, and used the resulting ${\Bi}$ and ${\Vi}$ as the $\hat{\Bi}$ and $\hat{\Vi}$ for the second period. The measure are the average CPU time in second (Time), the average integrality gap (GAP), and the average number of optimal solution over all the instances in one group. Each instance is solved five times.

Table (\ref{tab:MethodologyCompare}) illustrates that the faster version of the branch cut algorithm has obviously better performance compared to the other methods, and hence it will be used for the rest of the experiments.

In the next set of experiments we perform a sensitivity analyse to investigate the performance of the faster branch and cut algorithm based on the change in different parameters. To this end a base case instance is selected and, the solution time is reported based on the changes in different parameters as shown in table ???. 
Each of these instances are solved with 500 branches and 1000 scenarios. 

\begin{comment}

\begin{table}[]
\caption{Comparison of methodology to solve the model with the service level}
\label{tab:MethodologyCompare}
\begin{tabular}{lllllllllllll}
Method      & \multicolumn{4}{c}{Extensive form} & \multicolumn{4}{c}{Strong Cut}    & \multicolumn{4}{c}{Fast Cut}     \\ \hline
\# Branches & Time    & Gap   & \# Opt & BestObj & Time   & Gap   & \# Opt & BestObj & Time  & Gap   & \# Opt & BestObj \\ \hline
100         & 800.2   & 0.2\% & 4.8    & 10612.1 & 9.3    & 0.0\% & 5.0    & 10611.7 & 3.2   & 0.0\% & 5.0    & 10611.7 \\
200         & 2110.8  & 1.4\% & 4.1    & 10513.3 & 45.4   & 0.0\% & 5.0    & 10509.0 & 7.4   & 0.0\% & 5.0    & 10509.0 \\
300         & 3264.9  & 2.5\% & 3.2    & 10419.6 & 131.2  & 0.0\% & 5.0    & 10407.5 & 10.3  & 0.0\% & 5.0    & 10407.5 \\
500         & 4061.1  & 3.8\% & 2.5    & 10752.6 & 582.7  & 0.0\% & 5.0    & 10712.6 & 25.9  & 0.0\% & 5.0    & 10712.6 \\
1000        & 5321.6  & 7.5\% & 1.6    & 12369.2 & 3265.4 & 1.1\% & 4.0    & 10722.1 & 220.3 & 0.0\% & 4.9    & 10700.5 \\ \hline
Average     & 3111.7  & 3.1\% & 3.2    & 10933.4 & 806.8  & 0.2\% & 4.8    & 10592.6 & 53.4  & 0.0\% & 5.0    & 10588.3
\end{tabular}


\end{table}
\end{comment}

\begin{table}[]
\caption{Comparison of methodologies to solve the model with the service level}
\label{tab:MethodologyCompare}
\begin{tabular}{lrccrccrcc}
Method      & \multicolumn{3}{c}{Extensive form} & \multicolumn{3}{c}{Strong B\&C} & \multicolumn{3}{c}{Fast B\&C} \\ \hline
\# Branches & Time       & Gap       & \# Opt    & Time      & Gap     & \# Opt   & Time    & Gap     & \# Opt   \\ \hline
100         & 800.2      & 0.2\%     & 4.8       & 9.3       & 0.0\%   & 5.0      & 3.2     & 0.0\%   & 5.0      \\
200         & 2110.8     & 1.4\%     & 4.1       & 45.4      & 0.0\%   & 5.0      & 7.4     & 0.0\%   & 5.0      \\
300         & 3264.9     & 2.5\%     & 3.2       & 131.2     & 0.0\%   & 5.0      & 10.3    & 0.0\%   & 5.0      \\
500         & 4061.1     & 3.8\%     & 2.5       & 582.7     & 0.0\%   & 5.0      & 25.9    & 0.0\%   & 5.0      \\
1000        & 5321.6     & 7.5\%     & 1.6       & 3265.4    & 1.1\%   & 4.0      & 220.3   & 0.0\%   & 4.9      \\ \hline
Average     & 3111.7     & 3.1\%     & 3.2       & 806.8     & 0.2\%   & 4.8      & 53.4    & 0.0\%   & 5.0     
\end{tabular}

\end{table}


\subsubsection{Policy evaluation}

Three different policies are compared against each other. The first two are the deterministic policies, and the third one is the policy which consider the service level explicitly in the model. The first deterministic policy is the average policy which does not consider the service level, and for the demand we substitute it with the average demand which is calculated based on the realized demand and the autoregressive process.
The second deterministic policy is the quantile policy in which the demand for the next immediate period is substituted based on the quantile percentage in the demand scenario branches of next immediate period, and for the rest of the planning periods we calculate it with the same procedure as the average policy.
The third policy considers the service level in the next stage. The demand in each planning period in this policy is substitute with the average demand same as the average policy.

The policies are evaluated  using simulation. The demand for each product at each period is generated based on the autoregressive process, and the models are solved repeatedly in a rolling horizon fashion. To compare the policies we calculate different cost based on the realized demand and solution values. As the backlog cost is a parameter in deterministic models to reach the desired service level, we eliminate it in the calculation of the solution cost. Another criteria to compare the policies is the joint service level, which is calculated over all products for each planning period in the simulation. 
In the simulation process, we ignore few initial periods as the warm up periods. The remaining periods are divided into number of batches, and for each batch the average objective and joint service level are calculated. Based on the number of batches and the calculated averages, the confidence intervals are calculated. The experiments are run in a rolling horizon for 4000 periods 

\begin{itemize}
\item The policies are compared based on objective function and respect for the service level. 
\item The performance of the policies can be also compared base on different cost structure. For example when the backlog cost is equal to 0 we expect that the deterministic model with the average demand does not show a good performance in terms of service level. The service level for the next stage and for the whole planning period.
\item As the results show, the deterministic models are sensitive to the cost of backlog and has different pattern in different settings. A low backlog may lead to very low service level, and high one may impose unnecessary cost to the model. Using these policies one should do a sensitivity analysis to come up with a proper backlog cost to come up with an acceptable service level. It should be noted that even if the company come up with the exact cost of backlog, it may not be enough to reach the desired service level. However, in the policy with the service level, it is not necessary to consider the backlog cost as the quality of the solution is not sensitive to it. 


\end{itemize}

\newcolumntype{L}{>{$}l<{$}}
\newcolumntype{C}{>{$}c<{$}}
\newcolumntype{R}{>{$}r<{$}}
\newcommand{\nm}[1]{\textnormal{#1}}


\begin{table} [h!]
\centering
\begin{tabular}{LLLLLLL}
\toprule
\multicolumn{1}{l}{$\pi$} &
\multicolumn{3}{c}{Total cost}    &
\multicolumn{3}{c}{Service level(\%)}    \\ 
\cmidrule(lr){2-4}
\cmidrule(lr){5-7}

&
\multicolumn{1}{c}{Average policy} &
\multicolumn{1}{c}{Quantile Policy}     &
\multicolumn{1}{c}{SL policy} &
\multicolumn{1}{c}{Average policy} &
\multicolumn{1}{c}{Quantile Policy}     &
\multicolumn{1}{c}{SL policy}  \\
\midrule

0 & 62.83\pm0.55 &	60.63\pm0.08 &	68.03\pm0.18&	00.0\pm0.00	&00.0\pm0.00	&95.3\pm0.67\\
1&78.05\pm0.37	&71.36\pm0.09	&68.17\pm0.18&	21.3\pm1.25	&92.4\pm0.94	&95.3\pm0.67 \\
2&	81.02\pm0.37	&71.50\pm0.09 &	68.24\pm0.18&	21.4\pm1.26&	94.1\pm0.8&	95.4\pm0.67 \\
3&	83.86\pm0.37&	71.53\pm0.09&	68.30\pm0.17&	21.4\pm1.26&	94.1\pm0.8&	95.3\pm0.67\\
4&	86.68\pm0.38&	71.55\pm0.09&	68.37\pm0.17&	21.4\pm1.26&	94.1\pm0.8&	95.3\pm0.67\\
5&	89.46\pm0.38&	71.57\pm0.09&	68.43\pm0.17&	21.4\pm1.26&	94.1\pm0.8&	95.3\pm0.67\\
10&	103.09\pm0.42&	71.69\pm0.09&	68.78\pm0.17&	21.4\pm1.26&	94.1\pm0.8&	95.3\pm0.67\\
20&	130.10\pm0.42&	71.92\pm0.09&	69.46\pm0.18&	21.4\pm1.26&	94.1\pm0.8&	95.4\pm0.67\\

\midrule[\heavyrulewidth]
\multicolumn{7}{l}{\footnotesize 95\% of confidence interval} \\
\bottomrule
\end{tabular}
\caption{Comparing 3 policies for the base case}\label{beta}
\end{table}

\subsubsection{Sensitivity analysis}

An interesting option for the sensitivity analysis is to show that having both service level and approximate backlog cost is the better than version when we have both options together. This can be done with a sensitivity analysis for the backlog cost with and without the service level and check how the model is robust in the former case, and may be show underestimating the backlog is better than overestimating that.

\begin{itemize}
    \item TBO:\\
    When TBO is greater than 1 and the backlog cost is greater than 0 the service level is oversatisfied. This is due to the fact that the production in each period is increased to reduce the backlog in the future period. By this higher production many of the demands can be satisfied in the current period. In the following tables we will show that this oversatisfaction of the service level will not impose a huge cost to the model. In these experiments we cancelled the current stage modification. We will see that, even with very low service levels which are the cases with low backlog, there is very low cost difference with the case in which the model tries to minimise the backlog in the current stage as much as possible.
\end{itemize}

\subsubsection{Effect of substitution}

In this section we want to investigate the effect of substitution. To this end we run some experiments and we eliminate the possibility of substitution for different service levels. Table ??? provide the result on these experiments. In this table the average cost, and average service level their interval and different cost in detail are provided. These result are based on $\pi =0$. As expected, eliminating the possibility of substitution will increase the costs. 

It should be mentioned that the confidence interval of the objective function is about ??? and for the service level is about ??? which is reasonable.

\begin{table}[]
\begin{tabular}{llllllllllllll}
\multicolumn{14}{c}{With Substitution} \\
SL & TBO & Pi & Time & OBJ & \multicolumn{2}{c}{OBJ (CI)} & SL & \multicolumn{2}{c}{SL (CI)} & STC & HC & SUBC & BLC \\
80\% & 1 & 0 & 9740.82 & 66.86 & 66.66 & 67.06 & 80.80\% & 79.39\% & 82.21\% & 52.95 & 10.09 & 3.82 & 0.00 \\
90\% & 1 & 0 & 7988.94 & 67.56 & 67.37 & 67.75 & 90.60\% & 89.53\% & 91.67\% & 52.95 & 11.08 & 3.54 & 0.00 \\
95\% & 1 & 0 & 3945.63 & 68.03 & 67.83 & 68.23 & 95.30\% & 94.55\% & 96.05\% & 52.95 & 11.15 & 3.93 & 0.00 \\
99\% & 1 & 0 & 3913.23 & 69.39 & 69.20 & 69.57 & 99.10\% & 98.78\% & 99.42\% & 52.95 & 13.15 & 3.29 & 0.00 \\
80\% & 1 & 1 & 8944.49 & 67.13 & 67.09 & 66.69 & 80.80\% & 79.39\% & 82.21\% & 52.89 & 10.17 & 3.83 & 0.25 \\
90\% & 1 & 1 & 8536.70 & 67.71 & 67.79 & 67.41 & 90.60\% & 89.53\% & 91.67\% & 52.88 & 11.16 & 3.56 & 0.11 \\
95\% & 1 & 1 & 9261.46 & 68.17 & 68.30 & 67.91 & 95.30\% & 94.55\% & 96.05\% & 52.88 & 11.29 & 3.94 & 0.07 \\
99\% & 1 & 1 & 7672.03 & 69.85 & 70.05 & 69.63 & 99.20\% & 98.91\% & 99.49\% & 52.85 & 13.63 & 3.36 & 0.01 \\
\multicolumn{14}{c}{Without substitution} \\
80\% & 1 & 0 & 21990.41 & 72.68 & 72.58 & 72.77 & 80.00\% & 78.44\% & 81.56\% & 53.00 & 19.68 & 0.00 & 0.00 \\
90\% & 1 & 0 & 3328.39 & 74.69 & 74.59 & 74.78 & 89.40\% & 88.27\% & 90.53\% & 52.99 & 21.69 & 0.00 & 0.00 \\
95\% & 1 & 0 & 2048.29 & 76.39 & 76.30 & 76.48 & 94.60\% & 93.85\% & 95.35\% & 53.00 & 23.40 & 0.00 & 0.00 \\
99\% & 1 & 0 & 1386.27 & 78.89 & 78.79 & 78.98 & 98.90\% & 98.54\% & 99.26\% & 53.00 & 25.89 & 0.00 & 0.00 \\
80\% & 1 & 1 & 14967.64 & 73.14 & 73.10 & 72.89 & 80.00\% & 78.46\% & 81.54\% & 52.85 & 20.15 & 0.00 & 0.14 \\
90\% & 1 & 1 & 6709.10 & 75.30 & 75.33 & 75.11 & 89.40\% & 88.27\% & 90.53\% & 52.81 & 22.42 & 0.00 & 0.08 \\
95\% & 1 & 1 & 3705.33 & 77.18 & 77.25 & 77.01 & 94.70\% & 93.95\% & 95.45\% & 52.81 & 24.32 & 0.00 & 0.05 \\
99\% & 1 & 1 & 1656.95 & 80.08 & 80.20 & 79.92 & 98.90\% & 98.54\% & 99.26\% & 52.88 & 27.19 & 0.00 & 0.01
\end{tabular}
\end{table}


\begin{table}[]
\begin{tabular}{llllllllllllll}
\multicolumn{14}{c}{With Substitution} \\
SL & TBO & Pi & Time & OBJ & \multicolumn{2}{c}{OBJ (CI)} & SL & \multicolumn{2}{c}{SL (CI)} & STC & HC & SUBC & BLC \\
95\% & 1 & 0 & 3920.83 & 68.03 & 68.23 & 67.83 & 95.30\% & 94.55\% & 96.05\% & 52.95 & 11.15 & 3.93 & 0.00 \\
95\% & 1.5 & 0 & 5507.96 & 133.69 & 133.94 & 133.44 & 95.20\% & 94.45\% & 95.95\% & 117.52 & 11.17 & 5.00 & 0.00 \\
95\% & 2 & 0 & 10003.63 & 218.93 & 219.50 & 218.36 & 95.10\% & 94.31\% & 95.89\% & 178.06 & 11.28 & 29.59 & 0.00 \\
95\% & 1 & 1 & 9283.92 & 68.17 & 68.30 & 67.91 & 95.30\% & 94.55\% & 96.05\% & 52.88 & 11.29 & 3.94 & 0.07 \\
95\% & 1.5 & 1 & 8881.62 & 140.45 & 140.76 & 140.14 & 99.80\% & 99.62\% & 99.98\% & 69.48 & 61.13 & 9.84 & 0.00 \\
95\% & 2 & 1 & 13011.88 & 192.59 & 193.25 & 191.92 & 99.80\% & 99.66\% & 99.94\% & 114.85 & 57.47 & 20.26 & 0.00 \\
\multicolumn{14}{c}{Without Substitution} \\
95\% & 1 & 0 & 1923.74 & 76.39 & 76.48 & 76.30 & 94.60\% & 93.85\% & 95.35\% & 53.00 & 23.40 & 0.00 & 0.00 \\
95\% & 1.5 & 0 & 1958.76 & 142.39 & 142.48 & 142.29 & 94.60\% & 93.85\% & 95.35\% & 118.99 & 23.40 & 0.00 & 0.00 \\
95\% & 2 & 0 & 2277.90 & 233.38 & 233.48 & 233.27 & 94.60\% & 93.85\% & 95.35\% & 209.98 & 23.40 & 0.00 & 0.00 \\
95\% & 1 & 1 & 4027.09 & 77.18 & 77.25 & 77.01 & 94.70\% & 93.95\% & 95.45\% & 52.81 & 24.32 & 0.00 & 0.05 \\
95\% & 1.5 & 1 & 3559.08 & 193.81 & 194.28 & 193.32 & 97.80\% & 97.30\% & 98.30\% & 102.03 & 91.77 & 0.00 & 0.01 \\
95\% & 2 & 1 & 5116.17 & 273.15 & 273.91 & 272.37 & 98.00\% & 97.50\% & 98.50\% & 179.02 & 94.13 & 0.00 & 0.01
\end{tabular}
\end{table}

\section{Conclusion}

An extension to this problem is to approximate this the problem as a two stage stochastic problem in addition to have the service level in the next stage. This model will computationally expensive, but it will provide a better approximation for the problem.

\begin{thebibliography}{99}



%%

%\bibitem{RefJ}
% Format for Journal Reference
%Author, Article title, Journal, Volume, page numbers (year)
% Format for books
%\bibitem{RefB}
%Author, Book title, page numbers. Publisher, place (year)

%a,b

\bibitem{akccaycategory}
Akçay Yalçın. and Yunke Li and Harihara Prasad Natarajan, Category Inventory Planning With
Service Level Requirements and Dynamic Substitutions.
Production and Operations Management (2020), https://doi.org/doi:10.1111/poms.13240

\bibitem{bassok1999single}
Bassok, Yehuda and Anupindi, Ravi and Akella, Ram,
Single-period multiproduct inventory models with substitution,Operations Research, 47, 4, 632--642, (1999)
 

\bibitem{bookbinder1988strategies}
 Bookbinder, James H and Tan, Jin-Yan, Strategies for the probabilistic lot-sizing problem with service-level constraints, Management Science, 34, 9, 1096--1108,
 (1988)


\bibitem{bitran1992deterministic}
Bitran, Gabriel R and Leong, Thin-Yin, Deterministic approximations to co-production problems with service constraints and random yields, Management science, 38, 5, 724--742, (1992)
  
  
  \bibitem{bitran1992ordering}
Bitran, Gabriel R and Dasu, Sriram, Ordering policies in an environment of stochastic yields and substitutable demands,
Operations Research,
40, 5,
999--1017, (1992)

\bibitem{bitran1994co}
Bitran, Gabriel R and Gilbert, Stephen M,
  Co-production processes with random yields in the semiconductor industry, Operations Research, 42, 3, 476--491, (1994)
%c,d
\bibitem{chen2020dynamic}
Chen, Boxiao and Chao, Xiuli, Dynamic inventory control with stockout substitution and demand learning, Management Science, (2020) , https://doi.org/10.1287/mnsc.2019.3474
%e,f

%g,h
\bibitem{gicquel2018joint}
Gicquel, C{\'e}line and Cheng, Jianqiang, A joint chance-constrained programming approach for the single-item capacitated lot-sizing problem with stochastic demand, Annals of Operations Research, 264, 1-2, 123--155, (2018)
\bibitem{helber2013dynamic}
 Helber, Stefan and Sahling, Florian and Schimmelpfeng, Katja, Dynamic capacitated lot sizing with random demand and dynamic safety stocks, OR Spectrum, 35, 1, 75--105, (2013)
 

\bibitem{hsu1999random}
Hsu, Arthur and Bassok, Yehuda, Random yield and random demand in a production system with downward substitution, Operations Research, 47, 2, 277--290,
  (1999)
  
\bibitem{hsu2005dynamic}
Hsu, Vernon Ning and Li, Chung-Lun and Xiao, Wen-Qiang, Dynamic lot size problems with one-way product substitution, IIE transactions, 37, 3, 201--215, (2005)

%i,j

\bibitem{jiang2017service}
 Jiang, Yuchen and Shi, Cong and Shen, Siqian, Service Level Constrained Inventory Systems, Production and Operations Management, Wiley Online Library 28, 9, 2365–-2389,
 (2017)
 
 \bibitem{jiang2017production}
 Jiang, Yuchen and Xu, Juan and Shen, Siqian and Shi, Cong, Production planning problems with joint service-level guarantee: a computational study, International Journal of Production Research, Taylor \& Francis, 55, 1, 38--58,
 (2017)
 %l
 
\bibitem{lang2010efficient}
 Lang, Jan Christian and Domschke, Wolfgang, Efficient reformulations for dynamic lot-sizing problems with product substitution, OR spectrum,
32, 2, 263--291, (2010)

\bibitem{liu2018polyhedral}
Liu, Xiao and K{\"u}{\c{c}}{\"u}kyavuz, Simge, A polyhedral study of the static probabilistic lot-sizing problem, Annals of Operations Research, 261, 1-2,
233--254, (2018)

\bibitem{liu2016decomposition}
Liu, Xiao and K{\"u}{\c{c}}{\"u}kyavuz, Simge and Luedtke, James, 
Decomposition algorithms for two-stage chance-constrained programs, Mathematical Programming, 157,
  1, 219--243, (2016)
 

\bibitem{luedtke2008sample}
Luedtke, James and Ahmed, Shabbir, A sample approximation approach for optimization with probabilistic constraints, 
SIAM Journal on Optimization, 19,
2, 674--699, (2008)


\bibitem{luedtke2014branch}
  Luedtke, James, A branch-and-cut decomposition algorithm for solving chance-constrained mathematical programs with finite support,
  Mathematical Programming, 146, 1, 219--244,
 (2014)
%m, n 

\bibitem{ng2012robust}
Ng, Tsan Sheng and Fowler, John and Mok, Ivy, Robust demand service achievement for the co-production newsvendor, IIE Transactions,
  44,
  5,
 327--341,
  (2012)
  
%o, p , q
%R
 \bibitem{rao2004multi}
 Rao, Uday S and Swaminathan, Jayashankar M and Zhang, Jun,
 Multi-product inventory planning with downward substitution, stochastic demand and setup costs,
 IIE Transactions, 36, 1, 59--71, (2004)
 %t 
 
 \bibitem{tempelmeier2007stochastic}
Tempelmeier, Horst, On the stochastic uncapacitated dynamic single-item lotsizing problem with service level constraints, European Journal of Operational Research, 181, 1, 184--194, (2007)
 

 \bibitem{tempelmeier2011column}
Tempelmeier, Horst, A column generation heuristic for dynamic capacitated lot sizing with random demand under a fill rate constraint, Omega, 39, 6, 627--633, (2011)
 % v
 

\bibitem{zeppetella2017optimal}
Zeppetella, Luca and Gebennini, Elisa and Grassi, Andrea and Rimini, Bianca,Optimal production scheduling with customer-driven demand substitution, International Journal of Production Research, 55, 6, 1692--1706, (2017)


 



\end{thebibliography}




\end{document}

