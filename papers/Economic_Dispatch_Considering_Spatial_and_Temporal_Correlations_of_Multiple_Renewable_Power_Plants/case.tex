%% !TEX root=econ_dispatch.tex
%The 1, 2, 3, 4-{\it th} WPP, 1 and 2-{\it th} CSPP is connected on the 11, 15, 27, 62, 78 and 80-{\it th} bus, respectively and each capacity is 200MW. The system load remains 4242MW in the RTED horizon. The forecast outputs of all RPPs from  10:05-11:00 ({\it t}=1, {\it t}=2,...,{\it t}=12) a.m. on October 11{\it th} are used and as shown in Fig.~\ref{forecast}.

The IEEE 118-bus system is employed to validate the stochastic dynamic RTED model with multiple RPPs. There are 10 WPPs and 4 PVPPs each with a capacity of 200MW, {connecting on the 10, 24, 25, 26, 61, 65, 69, 72, 73, 87, 89, 91, 111 and 113 buses, respectively.} The data of RPPs are obtained by synchronous data in Kansas 2006 produced by NREL~\cite{Nrel}.
% All the locations of RPPs are as shown in Fig.~\ref{location}.
Their corresponding forecast power of each RPP is generating by the persistence forecast method. Gaussian Copula is used to model the conditional distribution that needed in this paper. We consider a time period with 12 intervals, modeling the 5 minute dispatch within an hour.
% Assuming that the current time is 0min and the schedule horizon is 5min, 10min, ..., 60min.}
% \begin{figure}[ht]
% 	\begin{center}
% 		\includegraphics[width=3.3in]{renewlocation.png}\\
% 		\caption{The locations of WPPs and  CSPPs in State of Kansas}\label{location}
% 	\end{center}
% \end{figure}

% \begin{figure}[H]
% 	\begin{center}
% 		\includegraphics[width=3.2in]{forecast.png}\\
% 		\caption{Forecast outputs of WPPs and CSPPs}\label{forecast}
% 	\end{center}
% \end{figure}

\subsection {Uncertainties Modeling of Multiple RPPs}

Conditional PDF of actual power of 3'th WPP, 1'th PVPP, sum renewable power and renewable power in 180'th transmission line  are as shown in Fig.~\ref{actual}. We can see that with the increase of forecast power of 3'th WPP from 5min-60min, the location of marginal distribution of actual power also moved to right. Although the conditional PDF of wind power seems to be relatively fat, the forecast error of sum renewable power tends to be thinner due to the independence of WPPs and PVPPs.
\begin{figure}[ht]
	\begin{center}
		\includegraphics[trim = 30 300 40 270, clip, width=1.0\columnwidth]{CPDFofWPP3.eps}\\
		\caption{Conditional PDFs of actual outputs}\label{actual}
	\end{center}
\end{figure}


To further analyze the renewable power independence, conditional joint PDFs of two RPPs of {\it t}=1 are shown in Fig.~\ref{jointpdf} for its high sensitivity. We can see that the power of 1'th WPP and 3'th WPP show the feature of positive correlation since they are geographically close. In contrast, the 1'th WPP and 4'th WPP are further apart and has smaller correlations.
%\todo{Change Fig.~\ref{jointpdf} to only include these two plots.}
% Interestingly, the 2-{\it th} WPP and 1-{\it th} CSPP that has nearly same location, negative correlation can be seen. The above qualitative feature is same with quantitative feature by Gaussian Copula function.
\begin{figure}[ht]
	\begin{center}
		\includegraphics[trim = 30 350 60 370, clip, width=1.0\columnwidth]{jointpdf.eps}\\
		\caption{Conditional joint PDFs of actual output of two RPPs}\label{jointpdf}
	\end{center}
\end{figure}

\vspace{-1em}

\subsection {Renewable Power Scenarios}
To generate renewable power scenarios that captures the spatial correlation, renewable power scenarios should follow the joint distribution in \eqref{cjdistribution}. However, it is usually hard to use \eqref{cjdistribution} directly. For instance, a $100^{14}$ size matrix would be needed to store the joint distribution with 0.01p.u. resolution in this case. In contrast, thanks to the Gibbs theory, renewable power scenarios can be generated by the proposed method with only 100 size matrix to store the conditional distribution in \eqref{ccdistribution} with same resolution.

To show the effect of variability (temporal correlation), we generate renewable power scenarios by our proposed method and the method in \cite{copula_Zhang}, respectively. $N_{sc}$ is 5000, $N_{b}$ is 1000 in this case. Fig.~\ref{scenario1} shows the former 50 scenarios in $N_{sc}$ of 3'rd WPP based on our proposed method~(the left figure) and the method in \cite{copula_Zhang}~(the right figure). The red line and black line in Fig.~\ref{scenario1} is the forecast power and actual power, respectively. We can see that the above two scenarios set have the same distribution in each time interval while the renewable power scenarios of our method are much more similar to the actual renewable power. The economic comparison is discussed in Section~\ref{sec:econ}.
\begin{figure}[ht]
	\begin{center}
		\includegraphics[trim = 135 340 130 360, clip, width=1.0\columnwidth]{scenariosofWPP3.eps}\\
		\caption{The left picture shows the scenarios generated when time correlations are considered (our method) v.s. scenarios that do not consider correlations in time (on the right, standard method). By considering temporal correlations, much more realistic scenarios can be generated.}\label{scenario1}
		%\caption{Scenarios generation influenced by wind power variability }\label{scenarios}
	\end{center}
\end{figure}




% \begin{figure}[ht]
% 	\begin{center}
% 		\includegraphics[width=3.5in]{reducedscenario.png}\\
% 		\caption{Scenarios after reduction to represent renewable power uncertainties at time {\it t}=1}% % \label{reducedscenarios}
% 	\end{center}
% \end{figure}

\vspace{-1em}

\subsection {Economy Comparison of Different RTED Methods} \label{sec:econ}

The generated renewable power scenarios are reduced to 10 scenarios and incorporated in proposed scenario-based ED. To compare the economy of the proposed RTED model, scheduled power of CPPs obtained by different RTED model are tested with other generated 10000 scenarios that consider the variability. {The cost coefficients of upward and downward reserve are all 10\$/MW.} The penalty coefficients of LS and REC is 1000\$/MW and 80\$/MW, respectively. The following five RTED models are compared in this paper.

{\it Case1}: The proposed distribution-based RTED model. {\it Case2}: The proposed distribution-based RTED model while the transmission capacity constraint use the forecast renewable power in \cite{VersatileMixture}. {\it Case3}: The proposed scenario-based RTED model. {\it Case4}: The proposed scenario-based RTED model while does not consider the variability of renewable power as in \cite{copula_Zhang}. {\it Case5}: Scenario-based RTED model that uses the marginal distribution of each RPP by \cite{sce_generation_Ma}, i.e. does not consider the spatial correlation of RPPs. The average costs of the above five cases are shown in Table.~\ref{cost}.
%\todo{Don't use case 1, case 2, ... as the table columns. Use better names.}

\begin{table}[h]
	\caption{{Total cost of different RTED models}}
	\label{cost}
	\begin{center}
		\begin{tabular}{|p{2.6cm}<{\centering}|c|c|c|c|c|}
			\hline
			{Cost/\$} & Fuel & Reserve & LS & REC & Total\\ \hline
			Proposed distribution- based model & \rule{0pt}{0.3cm} 36584 &4590&308&830& 42312\\\hline
			Model in \cite{VersatileMixture} &35877  & 4589& 702& 5705& 46873 \\\hline
			Proposed scenario- based model & \rule{0pt}{0.3cm} 35010 & 2986& 4514& 5298& 47808\\\hline
			Model in \cite{copula_Zhang} & 35032 & 2667& 7018& 5361& 50078\\ \hline
			Model in \cite{sce_generation_Ma} &35002& 2710 & 7741& 5425& 50878  \\ \hline

		\end{tabular}
	\end{center}
\end{table}

\begin{figure}[ht]
	\begin{center}
		\includegraphics[trim = 35 230 35 190, clip, width=1.0\columnwidth]{scewithdiffno.eps}\\
		\caption{{Original scenarios and scenarios after reduction.}}\label{reducedscenarios}
	\end{center}
\end{figure}

{As shown in Table.~\ref{cost}, compared with the proposed distribution-based model, model in \cite{VersatileMixture} has larger LS and REC penalty for the reason that it has not considered the transmission congestion caused by renewable power uncertainties. Compared with the proposed scenario-based model, model in \cite{copula_Zhang} has larger cost for the reason that it could not capture the renewable power variability. Compared with the proposed scenario-based model, model in \cite{sce_generation_Ma} has larger cost for the reason that it has not considered the correlations between different RPPs. Overall, the scenario-based RTED method has much larger LS and REC penalty for the reason that it underestimate the uncertainties of renewable power compared with distribution-based RTED method, as shown in Fig.~\ref{reducedscenarios}. The slim blue lines in the upper left figure are the original renewable scenarios by the proposed scenario generation method and the figure also shows the 10 scenarios after reduction. Renewable power scenarios generated by \cite{copula_Zhang} are also reduced to 10, as shown in the upper right figure.}



{To further analyze the scenario-based RTED, the original renewable scenarios by the proposed scenario generation method are reduced to 50, 500 and 2000 scenarios and embedded in the RTED model. As shown in Table.~\ref{cost2}, with more scenarios embedded in the ED model, system cost decreases. The 50 and 500 scenarios after reduction are shown in the lower left figure and lower right figures, respectively. It can be seen that with 500 scenarios, scenario-based RTED has similar performance with the distribution-based model for the reason that renewable power uncertainties could be well represented. When 2000 scenarios embedded in the RTED model, scenario-based ED shows a better performance compared with of the distrbution-based one since it has a more flexible manner to model the risk of renewable power such as REC caused by certain transmission line congestion.}




%To analyze its discrete feature, the sum renewable power and renewable power in 78'th line at time {\it t}=1 are shown in Fig.~\ref{reducedscenarios}. We can see that although scenario-based ED can consider the cost of LS and REC caused by system reserve deficiency and transmission congestion in theory, the discrete feature of scenario-based ED reduce the effect. As shown in Fig.~\ref{reducedscenarios}, the renewable uncertainties are underestimate compared with distribution-based ED. The reason is that in order to represent the renewable power in scenario reduction method, extreme scenarios are abandoned, especially when the number of RPPs is large in power system. The economy caused by the discrete feature of scenario-based ED is discussed in Section~\ref{sec:econ}.

%\todo{Is lower cost the better in the table? If not, what distinguishes between scenario based and case 2?}


\begin{table}[h]
	\caption{{Total cost of proposed scenario-based model based on different numbers of scenarios after reduction}}
	\label{cost2}
	\begin{center}
		\begin{tabular}{|p{2.5cm}<{\centering}|c|c|c|c|c|}
			\hline
			{Cost/\$} & Fuel & Reserve & LS & REC & Total\\ \hline
			50 scenarios& 35083 & 3124& 3464& 3954& 45625\\\hline
			500 scenarios& 36572  & 4584& 319& 854& 42329\\\hline
			2000 scenarios& 36591 &4528&285&785& 42189\\\hline
		\end{tabular}
	\end{center}
\end{table}

%To show the relationship of system reserve and the potential risk of REC and LS, different penalty coefficients of LS and REC are set to change the potential risk of REC and LS. As shown in Table.~\ref{aci}, the confidence level of enough downward reserve increases if the penalty coefficients of REC increase. This means that when the potential risk of REC increase, more downward reserves are employed for the overall economy. In addition, the confidence level of enough downward reserve of different time interval in Table.~\ref{aci} varies due to the different uncertainties. Compared with other classical stochastic ED methods that use a predefined confidence level, the proposed can seek for the optimal confidence level and obtain the overall economy to cope with the uncertainties and correlations of RPPs.




\vspace{-1em}

\subsection {RTED With Different Penalty Coefficients}

To show the relationships of system reserve and the potential risk of REC and LS, different penalty coefficients of LS and REC are set to change the potential risk of REC and LS. As shown in Table.~\ref{aci}, the confidence level of enough downward reserve increases if the penalty coefficients of REC increase. This means that when the potential risk of REC increase, more downward reserves are employed for the overall economy. In addition, the confidence level of enough downward reserve of different time interval in Table.~\ref{aci} varies due to the different uncertainties. Compared with other classical stochastic ED methods that use a predefined confidence level, the proposed can seek for the optimal confidence level.

\begin{table}[h]
	\caption{Confidence level of enough downward reserve under different REC penalty coefficients}
	\label{aci}
	\begin{center}
		\begin{tabular}{|c|c|c|c|c|c|}
			\hline
			{} & \multicolumn{5}{c|}{Penalty coefficients of REC (\$/MW$\cdot$h)}\\
			\hline
			{{\it time}/min} & {40} & {60} & {80} & {120} & {200}\\
			\hline
			05 & 74.84\% & 80.05\% & 81.98\% & 86.74\% & 93.05\% \\
			10 & 75.04\% & 80.26\% & 82.15\% & 87.08\% & 92.62\% \\
			15 & 74.73\% & 79.93\% & 81.82\% & 87.35\% & 92.16\% \\
			20 & 74.98\% & 81.29\% & 83.25\% & 86.88\% & 92.61\% \\
			25 & 74.99\% & 79.91\% & 82.64\% & 86.89\% & 93.11\% \\
			30 & 75.32\% & 81.92\% & 84.68\% & 88.88\% & 91.74\% \\
			35 & 74.98\% & 81.37\% & 84.12\% & 87.31\% & 92.06\% \\
			40 & 75.03\% & 81.80\% & 83.99\% & 88.67\% & 93.18\% \\
			45 & 75.14\% & 81.89\% & 84.08\% & 87.93\% & 93.23\% \\
			50 & 74.97\% & 81.75\% & 83.95\% & 88.63\% & 93.72\% \\
			55 & 74.71\% & 81.17\% & 83.98\% & 87.71\% & 92.20\% \\
			60 & 74.98\% & 80.86\% & 83.70\% & 88.69\% & 93.10\% \\
			\hline
		\end{tabular}
	\end{center}
\end{table}
