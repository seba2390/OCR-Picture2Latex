%% !TEX root=econ_dispatch.tex
In this section we study the RTED problem where CPPs outputs, system reserve and potential risk of {renewable energy curtailment} and {load shedding} are balanced. We consider an hourly dispatch with $T=12$ intervals where each one is 5 minutes long. The objective function of the ED problem is:
%\todo{spell out REC and LS}
% ED model based on the distribution of sum actual available power of all RPPs in \eqref{sumdistribution} is proposed in this section, in which the
%
% % \subsection {Distribution-Based Cost Modeling}
% RTED with renewable power integration creates scheduled power of CPPs, WPPs and CSPPs of interval {\it t}=1 . . .{\it T} by the newest load, wind power and solar power forecast values of {\it T} intervals (in this paper, {\it T}=12, each time interval is 5min and the time horizon is 60min) in RTED, minimizing the total cost under the constraints. Distribution-based RTED model with multiple RPPs is as follows.\\
%
\begin{equation} \label{DB_1}
\begin{aligned}
\begin{aligned}
min\sum _{{\it t}={\it 1}}^{{\it T}}E[f_t]& =\sum _{{\it t}=1}^{{\it T}}E[f_{c,t}(p_{i,t},r_{u,i,t},r_{d,i,t})]\\
& +\sum _{{\it t}=1}^{{\it T}}E[f_{R,t}(w_{c,j,t},s_{c,k,t},l_{s,b,t})]\\
\end{aligned}
\end{aligned}
\end{equation}
where $f_t$  is the total system cost at time {\it t};
$f_{c,t}$ is the total CPP cost at time {\it t}; $f_{R,t}$ is the total penalty cost caused by renewable power uncertainties (REC and LS); $p_{i,t}$ is the schedule power of $i$'th CPP at time {\it t}; $r_{u,i,t}$ and $r_{d,i,t}$ is the upward and downward reserve of $i$'th CPP at time {\it t}, respectively; $w_{c,j,t}$ and $s_{c,k,t}$ is the power of REC of $j$'th WPP and $k$'th PVPP at time {\it t}, respectively; $l_{s,b,t}$ is the power of LS of $b$'th bus at time {\it t}.

The CPP cost is given by
\begin{equation} \label{DB_2}
\begin{aligned}
& f_{c,t}(p_{i,t},r_{u,i,t},r_{d,i,t})\\
=\sum _{{\it i}=1}^{{\it I}}(b_{f,i}p_{i,t} & +c_{f,i}+c_{ur,i}r_{u,i,t}+c_{dr,i}r_{d,i,t})
\end{aligned}
\end{equation}
where {\it I} is the total number of CPPs; $b_{f,i}$ and $c_{f,i}$ are the fuel cost coefficients of $i$'th CPP, respectively; $c_{ur,i}$ and $c_{dr,i}$ are the cost coefficients of upward and downward reserve of $i$'th CPP, respectively.
Penalties with respect to uncertainties in the renewable powers are given by:
\begin{equation} \label{DB_3}
E[f_{R,t}(w_{c,j,t},s_{c,k,t},l_{s,b,t})]=c_{ls}E_{ls,t}+c_{rec}E_{rec,t}\\
\end{equation}
where $c_{ls}$ and $c_{rec}$ is the penalty coefficients of LS and REC, respectively; $E_{ls,t}$ and $E_{rec,t}$ is the expected values of LS and REC, respectively.

For ease of analysis, the sum scheduled renewable energy $R_{t}^{\Sigma}$ is introduced as an internal variable in the distribution-based ED model for the balance of power system.
%, as shown in Fig.~\ref{uncertainties}.

%\begin{figure}[ht]
%	\begin{center}
%		\includegraphics[width=3in]{uncertainties.png}\\
%		\caption{Cost modeling of the renewable energies uncertainties}\label{uncertainties}
%	\end{center}
%\end{figure}

$R_{a,t}^{\Sigma}$ is the sum actual available power of all RPPs at time {\it t} as shown in \eqref{sumdistribution}. $\underline{R}_{t}$ and $\overline{R}_{t}$ is the lower and upper bound that renewable power can be compensated by system reserves at time {\it t}, respectively. In worse case, if the sum actual renewable power locates in the outside of $[\underline{R}_{t}, \overline{R}_{t}]$, system reserve cannot cover all the uncertainties of renewable power. At this time, LS or REC would be employed for the power balance of the system. Then the total penalty cost of renewable power $f_{R,t}(w_{c,j,t},s_{c,k,t},l_{s,b,t})$ can be converted to $f_{R,t}(\underline{R}_{t},\overline{R}_{t})$ and written as
%The penalty cost caused by renewable power include two parts as shown in Fig.~\ref{uncertainties}.
\begin{equation} \label{DB_4}
\begin{aligned}
\begin{aligned}
& E[f_{R,t}(w_{c,j,t},s_{c,k,t},l_{s,b,t})]=f_{R,t}(\underline{R}_{t},\overline{R}_{t})\\
& =c_{ls}\int_{0}^{\underline{R}_{t}}(\underline{R}_{t}-R_{a,t}^{\Sigma})f(R_{a,t}^{\Sigma})dR_{a,t}^{\Sigma}\\
& +c_{rec}\int_{\overline{R}_{t}}^{R_{r}}(R_{a,t}^{\Sigma}-\overline{R}_{t})f(R_{a,t}^{\Sigma})dR_{a,t}^{\Sigma}\\
\end{aligned}
\end{aligned}
\end{equation}
where $R_{r}$ is the total capacity of renewable power.

% For ED problem with multiple RPPs, the penalty costs of renewable power uncertainties that cause LS and REC are considered in \eqref{DB_4}.
Compared with other classical stochastic ED methods that use a predefined confidence level to convert the reserve chance constraints to be linear ones \cite{Versatile}\cite{chance_constrain}\cite{ST_ED}, we can seek the optimal confidence level to find the balance for CPPs outputs, system reserve and potential risk of REC and LS according to different situations.

% \subsection {Total ED Model in Power System With Multiple RPPs}

All constraints of the proposed distribution-based ED model are as follows:
\begin{equation} \label{DB_5}
\sum_{{\it i}=1}^{{\it I}}p_{i,t}+R_{t}^{\Sigma}=L_{t} \quad \forall t
\end{equation}
\begin{equation} \label{DB_6}
\begin{aligned}
& R_{t}^{\Sigma}-\sum_{{\it i}=1}^{{\it I}}r_{u,i,t}=\underline{R}_{t} \quad \forall t \\
& R_{t}^{\Sigma}+\sum_{{\it i}=1}^{{\it I}}r_{d,i,t}=\overline{R}_{t} \quad \forall t
\end{aligned}
\end{equation}
\begin{equation} \label{DB_7}
\begin{aligned}
p_{i,t} + r_{u,i,t} \leq p_{max,i} \quad \forall i,t \\
p_{i,t} - r_{d,i,t} \geq p_{min,i} \quad \forall i,t
\end{aligned}
\end{equation}
\begin{equation} \label{DB_8}
\begin{aligned}
p_{i,t}-p_{i,t-1} & \leq \Delta p_{u,max,i} \quad \forall i,t \\
p_{i,t-1}-p_{i,t} & \leq \Delta p_{d,max,i} \quad \forall i,t
\end{aligned}
\end{equation}
\begin{equation} \label{DB_9}
\begin{aligned}
0 & \leq r_{u,i,t} \leq r_{u,max,i} \quad \forall i,t \\
0 & \leq r_{d,i,t} \leq r_{d,max,i} \quad \forall i,t
\end{aligned}
\end{equation}
\begin{equation} \label{DB_10}
\begin{aligned}
0 \leq \underline{R}_{t},\;  \overline{R}_{t}\leq R_{r} \quad \forall t
\end{aligned}
\end{equation}

%\begin{equation} \label{DB_11}
%\begin{aligned}
%\underline{RE}_{t} & \le F_{RE_{a,t}}^{-1}(1-conf_{ur}) \quad \forall t \\
%\overline{RE}_{t} & \ge F_{RE_{a,t}}^{-1}(conf_{dr}) \quad \forall t \\
%\end{aligned}
%\end{equation}

\vspace{-1em}

\begin{equation} \label{DB_12}
\begin{aligned}
\begin{aligned}
& \sum_{i=1}^{I}k_{l,i}p_{i,t}+\underline{R}_{a}^{L_{\it l}}-\sum_{{\it b}=1}^{{\it Nb}}k_{l,b}L_{b,t} \ge -Pl_l^{max}  \quad \forall l,t\\
& \sum_{i=1}^{I}k_{l,i}p_{i,t}+\overline{R}_{a}^{L_{\it l}}-\sum_{{\it b}=1}^{{\it Nb}}k_{l,b}L_{b,t} \le  Pl_l^{max}  \quad \forall l,t
\end{aligned}
\end{aligned}
\end{equation}
where
\begin{itemize}

	\item \eqref{DB_5} is the supply-demand balance constraint; $L_{t}$ is the forecast power demand at time {\it t};

	\item \eqref{DB_6} is the system reserve constraint;

	\item \eqref{DB_7} are the CPPs scheduled power plus reserve capacity constraint;  $p_{max,i}$  and   $p_{min,i}$ are the upper and lower generation limit of the $i$'th CPP, respectively;

	\item \eqref{DB_8} are the CPPs ramp-rate constraint; $\Delta p_{u,max,i}$ and $\Delta p_{d,max,i}$ are the maximum amount of upward and downward ramp rate of $i$'th CPP within a specific time period (e.g., 5min), respectively;

	\item \eqref{DB_9} are the reserve capacity constraints; $r_{u,max,i}$ and $r_{d,max,i}$ are the maximum amount of up and down reserves that the $i$'th CPP is capable of providing, respectively;

	\item \eqref{DB_10} are the confidence level bound constraint;

%	\item \eqref{DB_11} are the compulsive system reserve that needed under the desired confident level;

	\item \eqref{DB_12} are the transmission capacity constraint; {\it Nb} is the total number of buses; $Pl_l^{max}$ is the transmission capacity limit on transmission line $l$; based on the the distribution of renewable power in the transmission lines $R_{a}^{L_l}$ in \eqref{linedistribution}, the uncertainties and correlations of multiple power energy can be considered compared with the classical model in \cite{AI} and  \cite{VersatileMixture} that used the forecast or scheduled renewable power. A conservative bound such as 99.9\% can be used in this constraint.
	% In this paper, we use the bound of $RE_{a}^{L_l}$ (e.g. 99.9\%) as shown in \eqref{DB_12} for the conservative aim.
\end{itemize}
