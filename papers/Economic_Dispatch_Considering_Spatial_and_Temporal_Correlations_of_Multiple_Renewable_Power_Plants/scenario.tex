%% !TEX root=econ_dispatch.tex
%\todo{Transition better. Explain why scenario based ED is considered. What's the difference with respect to the distribution based ED?}
Different from the distribution-based ED, scenario-based ED incorporate the renewable power uncertainties by a certain number of possible renewable power series (i.e., scenarios). This means that scenario-based ED is essentially a deterministic optimization. This allows a more flexible way to model the risk of renewable power such as REC caused by certain transmission line congestion. However, the performance of scenario-based ED greatly relies on the number of scenarios that are considered in the ED. RTED model based on the scenario of multiple RPPs is proposed in this section. The potential risk of LS and REC caused by system reserve deficiency and transmission congestion are modeled by the scenario-based ED. The penalty cost caused by renewable power uncertainties (REC and LS) in (14) and (16) can be written using scenarios as:
%\todo{What is the drawback of scenario based ED then? Should we always use scenario-based ED? Or sometimes distribution-based ED?}
\begin{equation} \label{SB_1}
\begin{aligned}
\begin{aligned}
&E[f_{R,t}(w_{c,j,t},s_{c,k,t},L_{s,b,t})]  \\
&=\sum_{{\it sc}=1}^{{\it SC}}[p^{sc}(c_{rec}(\sum_{{\it j}=1}^{{\it J}}w_{c,j,t}^{sc}+\sum_{{\it k}=1}^{{\it K}}s_{c,k,t}^{sc})
+c_{ls} \sum_{{\it b}=1}^{{\it Nb}}L_{s,b,t}^{sc})]
\end{aligned}
\end{aligned}
\end{equation}
where $sc$ is the {\it sc}-th scenario for WPPs and PVPPs, $SC$ is the number of renewable power scenarios in RTED model, $w_{c,j,t}^{sc}$ is the amount of wind power curtailment of $j$'th WPP at time {\it t} of $sc$'th scenario; $s_{c,k,t}^{sc}$ is the amount of solar power curtailment of $k$'th PVPP at time {\it t} of $sc$'th scenario; $L_{s,b,t}^{sc}$ is the amount of LS of $b$'th bus at time {\it t} of $sc$'th scenario.

Then the optimization problem is same as in Section~\ref{sec:dist}, except the constraints and objectives are represented with scenarios. In particular, the constraints are:
\begin{equation} \label{SB_2}
\begin{aligned}
0  \leq w_{c,j,t}^{sc} \leq w_{a,j,t}^{sc} \quad \forall j,t,sc \\
0  \leq s_{c,k,t}^{sc} \leq s_{a,k,t}^{sc} \quad \forall k,t,sc
\end{aligned}
\end{equation}

\begin{equation} \label{SB_3}
0  \leq L_{s,b,t}^{sc} \leq L_{b,t} \quad \forall b,t,sc \\
\end{equation}

\vspace{-1em}

\begin{equation} \label{SB_4}
-r_{d,i,t}  \leq r_{a,i,t}^{sc} \leq r_{u,i,t} \quad \forall i,t,sc \\
\end{equation}

\vspace{-1.5em}

\begin{equation} \label{SB_5}
\begin{aligned}
& \sum_{{\it i}=1}^{{\it I}}(p_{i,t}+r_{a,i,t}^{sc})+\sum_{{\it j}=1}^{{\it J}}(w_{a,j,t}^{sc}-w_{c,j,t}^{sc}) \\
& +\sum_{{\it k}=1}^{{\it K}}(s_{a,k,t}^{sc}-s_{c,k,t}^{sc})=L_{t}-\sum_{{\it b}=1}^{{\it Nb}}L_{s,b,t}^{sc} \quad \forall t,sc
\end{aligned}
\end{equation}

\vspace{-1.5em}

\begin{equation} \label{SB_6}
\begin{aligned}
\begin{aligned}
\begin{aligned}
&|\sum_{i=1}^{I}k_{l,i}(p_{i,t}+r_{a,i,t}^{sc})+\sum_{j=1}^{J}k_{l,j}(w_{a,j,t}^{sc}-w_{c,j,t}^{sc}) \\
&+\sum_{k=1}^{K}k_{l,k}(s_{a,k,t}^{sc}-s_{c,k,t}^{sc})-\sum_{{\it b}=1}^{{\it Nb}}k_{l,b}(L_{b,t}-L_{s,b,t}^{sc})|\\ & \le  Pl_l^{max} \quad \forall l,t,sc
\end{aligned}
\end{aligned}
\end{aligned}
\end{equation}
%\begin{equation} \label{SB_7
%\begin{aligned}
%	\sum_{\it sc=1}^{{\it SC}}(p^{sc}(\sum_{\it j=1}^{{\it J}}w_{c,j,t}^{sc}+\sum_{\it k=1}^{{\it K}}s_{c,k,t}^{sc}))\leq RE_{c,set}  \quad %\forall t\\
%	\sum_{\it sc=1}^{{\it SC}}(p^{sc}\sum_{\it b=1}^{{\it Nbus}}L_{s,b,t}^{sc})\leq L_{s,set}\quad \forall t
%\end{aligned}
%\end{equation}}
where
\begin{itemize}

	\item \eqref{SB_2} is the actual amount of REC constraint; $w_{a,j,t}^{sc}$ is actual wind power of $j$'th WPP at time {\it t} of $sc$'th scenario; $s_{a,k,t}^{sc}$ is actual solar power of $k$'th PVPP at time {\it t} of $sc$'th scenario;

	\item \eqref{SB_3} is the actual amount of LS constraint;

	\item \eqref{SB_4} is the actual amount of reserve constraint; $r_{a,i,t}^{sc}$ is actual amount of reserve of $i$'th CPP at time {\it t} of $sc$'th scenario;

	\item \eqref{SB_5} is the supply-demand balance constraint;

	\item \eqref{SB_6} is the transmission capacity constraint;

%	\item \eqref{SB_7} is the compulsive maximal amount of  REC and LS constraint for the conservative aim.

\end{itemize}

% \subsection {Comparison With the Distribution-Based ED Model}

Compared with the proposed distribution-based ED model, the scenario-based ED model can not only model the cost of LS and REC caused by system reserve deficiency but also can model the cost of LS and REC caused by transmission congestion. However, the number of scenarios after reduction is limit due to the computation ability. This would reduce the representation accuracy of renewable power in the above scenario-based RTED model, which would be discussed in Section VI.

When the reserve deficiency or transmission congestion occur, REC and LS have to be employed for the balance of system power. Optimal REC and LS strategies can be obtained by solving the static optimization problem (ob. \eqref{SB_1}, s.t. {\eqref{SB_2}-\eqref{SB_6})} by the deterministic value of CPPs scheduled power, actual reserve, actual power of WPPs and PVPPs (the only scenario in this optimization problem).

% \subsection {The Feature of Distribution-Based ED Method in Power System With Multiple Renewable Energy Variable}

% It is easy to prove that the above distribution-based ED model is convex and SLP-based algorithm can be employed to solved the above model.

% Compared with other modeling method such as scenario-based method that use several dispersed renewable energy curves, the proposed distribution-based ED method has two advantages, The first is that the distribution-based model is continuous and it has a better understanding of the renewable energy uncertainties. The second is that distribution-based ED method has higher computational efficiency.
