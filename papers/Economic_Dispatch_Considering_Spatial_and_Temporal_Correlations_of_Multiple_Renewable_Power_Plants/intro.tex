The increase in penetration of renewable resources around the world is fundamentally changing how power systems are operated \cite{Risk_Based_UC}. The potential risks of power unbalance, transmission congestion become much more complicated due to the dependence of wind power plants (WPPs) and solar photovoltaic power plants (PVPPs) in the system. In power system economic dispatch (ED), system operators should fully consider the uncertainties, spatial and temporal correlations of renewable power plants (RPPs). In this paper, we focus on the problem of economic dispatch (ED) when there are multiple correlated RPPs in the system.


The first step in integrating RPPs into dispatch is to obtain a tractable model that captures the uncertainties and correlations between them. Conditional distribution models for RPPs have proved to be a reliable mathematical method for ED with renewable power integration \cite{Versatile}\cite{sce_generation_Ma}\cite{copula_Zhang}. Conditional distribution models employ the forecast power of RPPs to increase the representation accuracy of the uncertainties and correlations of renewable power and consist with the current ED mode. Forecast bins are employed in \cite{Versatile} and \cite{sce_generation_Ma} and then actual wind power in each forecast bin can be modeled by mathematical distribution \cite{Versatile} or actual probability density histogram (PDH) \cite{sce_generation_Ma}. However, when the number of RPPs increases, forecast bin method becomes hard to be employed due to the curse of dimensionality caused by the number of bins. Copula theory shows better potential in dealing with multiple renewable power variables as in \cite{copula_Zhang}. Wind power conditional distribution model is built in \cite{copula_Zhang} based on copula theory and wind power scenarios are generated by the conditional joint distribution to solve the unit commitment and ED. However, high-dimension distribution model would greatly increase the computation scale and even hard to be employed directly. In addition, although wind power uncertainties and spatial correlation are novelly considered in each time interval, the temporal correlation (variability) of wind power scenarios among the schedule horizon which is regarded same important  \cite{sce_generation_Ma}\cite{PCA} is not considered in \cite{copula_Zhang}. In this paper, we do not use the high-dimension conditional joint distribution model directly. Instead, we convert the scenarios generation using the conditional joint distribution model at one time interval (static scenarios) to scenarios generation using the one-dimension conditional distribution by Gibbs theory \cite{Gibbs}. Then, a dynamic scenario generation method is proposed and the renewable power variability is considered among the schedule horizon. In addition, in order to study the overall effect of renewable power uncertainties and correlations on the system risk of reserve deficiency and transmission congestion, we also build the conditional marginal distribution of sum actual power of all RPPs and actual renewable power in power system transmission lines.


The next step is to incorporate the stochastic model into ED in a computationally efficient fashion. Stochastic ED \cite{chance_constrain}\cite{ST_ED} and robust ED \cite{Robust_1}\cite{Robust_2} are two main methods to solve the ED problem with renewable power integration. Compared with robust ED, stochastic ED offers better mechanisms to manage the uncertainties explicitly \cite{ST_ED}. One way to account for uncertainties in stochastic ED is to employ chance constraints to maintain a predefined risk level for the whole system \cite{chance_constrain}\cite{ST_ED}. However, in a system with possibly congested transmission lines and multiple RPPs, it becomes difficult to define a single risk level for the entire system. Here we consider risk by explicitly considering load shedding (LS) and renewable energy curtailment (REC) caused by system reserve deficiency and transmission congestions. To deal with the uncertainties, correlations and variability more reasonably, we seek for the optimal level of risk. To represent the uncertainties, spatial and temporal correlations of RPPs, distribution-based ED method and scenario-based ED method are proposed and compared to solve the real-time economic dispatch (RTED) problem with multiple RPPs. To test the ED results, we use the scenarios by proposed dynamic scenario generation method. The main contributions of this paper are summarized as follows.

%The uncertainties and correlations of multiple renewable power variables are represented using conditional distribution of actual power of each RPP, conditional marginal distribution of sum actual power and the conditional distribution of actual power in transmission line by copula theory and actual PDH, to fully consider their effects on power system reserve and transmission congestion in the RTED model.


%The proposed dynamic renewable power scenario generation method can consider the uncertainties, correlations and variability of renewable power. Renewable power scenarios are much more consist with the actual renewable power with high efficiency for real-time requirement.

We propose an efficient dynamic scenario generation method that captures the joint distribution and the variability
of multiple renewable power plants. Based on Gibbs sampling, our proposed method avoids directly computing high dimensional distributions to greatly reduce the complexity of sampling correlated renewable power plants.
% required sampling space and time complexity are greatly reduced. Renewable power scenarios are much more consist with the actual renewable power with high efficiency for real-time requirement.
%\todo{(expand on this more, what is this method based on and how does it advance the results for existing state-of-the-art?)}

We then describe two ED methods: distribution-based and scenario-based and compare their performances on economic dispatch problem with multiple renewable power plants. The potential risk of load shedding and renewable power curtailment caused by uncertainties and correlations of renewable power integration are modeled to balance the conventional power plants (CPPs) outputs, system reserve, potential risk of load shedding and renewable power curtailment.

We find that the best method to use depends on the number of scenarios used. With a small set of scenarios, scenario-based economic dispatch shows much worse compared with of the distribution-based one. With the increase of the number of scenarios embedded in the economic dispatch, scenario-based economic dispatch performs better. We provide detailed discussion of the results and illustrate which one should be chosen in practice based on the computational power and information available to the system operator. 
%\todo{(Have a list of contributions summarizing the results in this paper.)}

The remaining of this paper is organized as follows. In Section II, uncertainties and spatial correlation of multiple RPPs are modeled by Copula theory and actual PDH. In Section III, static scenario generation method of multiple RPPs based on Gibbs theory is proposed. Dynamic renewable power scenarios are generated to represent the uncertainties, spatial and temporal correlations of RPPs. A distribution-based and scenario-based RTED method are proposed in Section IV and Section V, respectively. In Section VI, uncertainties and correlations of multiple RPPs are shown and the dynamic renewable power scenarios are discussed. Numerical experiments of the proposed RTED models are conducted and compared using the IEEE 118-bus system. Section VII provides conclusions.
