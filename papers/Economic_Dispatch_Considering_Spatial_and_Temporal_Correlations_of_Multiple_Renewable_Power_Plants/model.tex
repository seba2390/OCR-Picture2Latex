%% !TEX root=econ_dispatch.tex
The uncertainties and spatial correlation of renewable power plants (RPPs) should be modeled to evaluate the potential risk of the load shedding {(LS)} and renewable energy curtailment {(REC)}. This section describes how uncertainties and spatial correlation of multiple RPPs are modeled based on historical data (forecast and actual power of each RPP) using copula theory.
 % how Based on the full use of historical data (forecast and actual power of each RPP) and copula theory, the uncertainties and correlations of multiple RPPs in power system are modeled in this chapter.
By uncertainty, we mean the marginal distribution of the forecast error of each RPP; and by spatial correlation, we mean the joint relationship between the forecast errors. Note that we often interested in the \emph{conditional distribution} of the errors given the forecasted values.

	  % between the The uncertainty of each RPP can be represented by the forecast errors distribution under certain forecast value. The correlations between different RPPs can be represented by the forecast errors joint distribution under certain group of forecast values. This means that the uncertainties and correlations of multiple RPPs can be represent by the conditional joint distribution of actual power of multiple RPPs.}
%\todo{Write out what LS, RPP, REC are again in this section. Don't use abbreviations for too many things. Try limiting use to WPP, CSPP and RPP in the paper}.\todo{Define what uncertainty and correlation mean here}

\subsection {Conditional Distribution of Renewable Power}
The power productions of the RPPs are described by two random vectors: a vector of \emph{forecast values} and a vector of \emph{actual power production} conditioned on the forecasts. Let $w_{f,j}$ denote the forecast of the $j$'th RPP if it is a wind power plant (WPP) and $s_{f,k}$ denote its forecast if it is solar photovoltaic power plant (PVPP). We assume there are $J$ total WPPs and $K$ total PVPPs. Let {$\mathbf{f}$} denote the vector of forecasts, i.e., $(w_{f,{1}}...w_{f,J},s_{f,{1}}...s_{f,K})$. Let $F(w_{f,i})$ denote the marginal cumulative distribution function (CDF) of the forecast of the $i$'th WPP (similar for PVPPs). We use {$\Omega(\mathbf{f})$} to denote the set of marginal forecast CDFs, i.e., $(F(w_{f,{1}})...F(w_{f,J}),F(s_{f,{1}})...F(w_{s,K}))$.
%Similarly, the set of marginal actual power CDFs is denoted by $\Omega(\mathbf{g})$.
%\todo{this notation is very weird, something like $\mathbf{f}$ would be better}


Copula method is an effective way of modeling the multiple variables dependence where the marginal distributions and the correlations between random variables appear separately~\cite{copula_Zhang,WytockEtAl2013}.
% If we regard $F(w_{a,j})$, $F(s_{a,k})$, $F(w_{f,j})$ and $F(s_{f,k})$ ($j=1...J$, $k=1...K$) as random variables, then all of them follow the uniform distributions as follows:
% \begin{equation} \label{eq:00}
% \begin{aligned}
% F(w_{a,j}), F(s_{a,k}), F(w_{f,j}), F(s_{f,k})\sim u(0,1) \quad \forall j,k
% \end{aligned}
% \end{equation}
In the Copula method, the joint CDF of the forecasted and actual RPP productions, $F(w_{a,{1} }. . .w_{a,J},s_{a,{1}}. . .s_{a,K},\mathbf{f})$, is written as:
\begin{equation} \label{eq:01}
\begin{aligned}
& F(w_{a,{1} }. . .w_{a,J},s_{a,{1}}. . .s_{a,K},\mathbf{f}) \\
= & C(F(w_{a,{1}})...F(w_{a,J}),F(s_{a,{1}})...F(s_{a,K}), \Omega(\mathbf{f}))
\end{aligned}
\end{equation}
where the function $C(\cdot)$ is called the Copula function. Essentially, we transform the joint CDF  $F(w_{a,{1} }. . .w_{a,J},s_{a,{1}}. . .s_{a,K},\mathbf{f})$ to a function of the marginal CDFs  $F(w_{a,j})$, $F(s_{a,k})$, $F(w_{f,j})$, $F(s_{f,k})$ linked by the Copula function $C$. Similarly, the joint probability density function (PDF) is:
\begin{equation} \label{eq:02}
\begin{aligned}
& f(w_{a,{1} }. . .w_{a,J},s_{a,{1}}. . .s_{a,K},\mathbf{f}) \\
= & c(F(w_{a,{1}})...F(w_{a,J}),F(s_{a,{1}})...F(s_{a,K}), \Omega(\mathbf{f}))\\
& \cdot \prod_{j=1}^{J}f(w_{a,{j}}) \cdot \prod_{k=1}^{K}f(s_{a,{k}}) \cdot \prod_{j=1}^{J}f(w_{f,{j}}) \cdot \prod_{k=1}^{K} f(s_{f,{k}}).
\end{aligned}
\end{equation}
If only the forecasted values are considered, then similar to \eqref{eq:02}, their joint PDF can be written as:
\begin{equation} \label{eq:03}
f(\mathbf{f})= c(\Omega(\mathbf{f})) \cdot \prod_{j=1}^{J}f(w_{f,{j}}) \cdot \prod_{k=1}^{K} f(s_{f,{k}}).
\end{equation}
Lastly, since the forecasted values are already know in ED, all the randomness are left in the actual power of the RPPs given the forecasted values. Combining \eqref{eq:02} \eqref{eq:03}, the joint conditional PDF of the actual productions is:
% In ED, the forecast power of each RPP can be obtained, then the conditional joint PDF is needed to model the available actual power of all RPPs. Combined by \eqref{eq:02} and \eqref{eq:03}, we can model the conditional joint PDF as shown in \eqref{cjdistribution}. }
%\todo{what is the function $c$? Don't assume readers will check the references. You have to write a self sufficient paper.}
\begin{equation} \label{cjdistribution}
\begin{aligned}
& f(w_{a,{1} }. . .w_{a,J},s_{a,{1}}. . .s_{a,K}|\mathbf{f}) \\
=& \frac{f(w_{a,{1} }. . .w_{a,J},s_{a,{1}}. . .s_{a,K},\mathbf{f})}{f(\mathbf{f})}\\
=& \frac{c(F(w_{a,{1}})...F(w_{a,J}),F(s_{a,{1}})...F(s_{a,K}),\Omega(\mathbf{f}))}{c(\Omega(\mathbf{f}))} \\
& \cdot \prod_{j=1}^{J}f(w_{a,{j}}) \cdot \prod_{k=1}^{K}f(s_{a,{k}})
\end{aligned}
\end{equation}

%\todo{(Saying something is good here, but don't forward reference to equations that have not appeared yet. So either change the equation numbers, or move this paragraph and rewrite something.)} \todo{(Are we learning $C$ and $c$? If so, say it directly.)(No,we calculate it by the historical data set)}
 There are many suitable copula functions (e.g., Gaussian, t, empirical~\cite{G_copula}) that can be used in \eqref{cjdistribution}. In this paper, we adopt the Gaussian copula and use the actual {probability density histogram} (PDH) \cite{sce_generation_Ma} to get the marginal distribution functions in Copula method.
%\todo{This paragraph should be put close to where you first introdcue copula method. What is PDH?}

\subsection {Sampling the Conditional Distribution}
%\todo{(Explain this better. Are you talking about sampling here? If so, then just say we use Gibbs sampling to do...)(Yes)}
In Gibbs sampling theory, conditional distribution function of actual available power of each RPP is needed and can be modeled in \eqref{ccdistribution} (take $j$'th WPP for instance), shown in the top of the next page. Note that the conditioning is on the forecast and actual renewable powers except for $j$'th WPP. This model will be further discussed in Section III.
%\todo{(Be specific, what will be discussed in III that is not discussed here?} \todo{Also, why reset the counters for the equation below?)(If we do not reset the counters, the number of formula could be wrong)}
\newcounter{TempEqCnt}
\setcounter{TempEqCnt}{\value{equation}}
\setcounter{equation}{4}
\begin{figure*}[t]
	\hrulefill
	\begin{equation} \label{ccdistribution}
	\begin{aligned}
	f(w_{a,{\it j} }|w_{a,{1} }. . .w_{a,j-1},w_{a,j+1}...w_{a,J},s_{a,{1}}. . .s_{a,K},\mathbf{f}) \\
	=\frac{c(F(w_{a,{1}})...F(w_{a,J}),F(s_{a,{\it 1}})...F(s_{a,K}),\Omega(\mathbf{f}))}{c(F(w_{a,{1}})...F(w_{a,j-1}),F(w_{a,j+1})...F(w_{a,J}),F(s_{a,{1}})...F(s_{a,K}),\Omega(\mathbf{f}))} & \cdot f(w_{a,{\it j}})
	\end{aligned}
	\end{equation}
\end{figure*}
\setcounter{equation}{\value{TempEqCnt}}

%\vspace{-0.5em}



\subsection {Sum Power of RPPs}

If congestion is neglected in the ED model, then we only need to model the sum of actual powers of RPPs, denoted by $R_a^{\Sigma}$.
%\todo{This notation is weird. Why not just use $R_a^{\sum}$}
% Based on the synchronous historical forecast and actual power of each RPP, the sum actual power of all RPPs can be obtained, then
Similar to \eqref{eq:01}, the joint CDF can be written using a Copula function as:
\setcounter{equation}{5}
\begin{equation} \label{eq:3}
\begin{split}
& F(R_{a}^{\Sigma},\mathbf{f})=C(F(R_{a}^{\Sigma}),\Omega(\mathbf{f}))\\
& R_{a}^{\Sigma}=\sum _{{\it j}={1}}^{{\it J}}w_{a,j}+\sum _{{\it k}={1}}^{{\it K}}s_{a,k} \\
\end{split}
\end{equation}
The conditional distribution of sum actual available power of all RPPs given the forecasts is:
%\todo{I'm confused by that $C(\cdot)$ and $c(\dot)$ means}
\begin{equation} \label{sumdistribution}
f(R_{a}^{\Sigma}|\mathbf{f})= \frac{c(F(R_{a}^{\Sigma}),\Omega(\mathbf{f}))}{c(\Omega(\mathbf{f}))}\cdot   f(R_{a}^{\Sigma})
\end{equation}

\subsection {Conditional Distribution Under Congestion}
The probability of congestions can be evaluated via {shift factors} by finding the contribution of line flows from each RPP~\cite{li2013dynamic}.
%\todo{(How are these shift factors calculated? Based on the forecasted value?)}

% probabilities of transmission congestion can be evaluated by the distribution of renewable power in the transmission lines. Based on the direct current power flow, the distribution of renewable energy in the transmission lines can be modeled by the weighted sum actual available power of RPPs, i.e.
Let $R_{a}^{L_l}$ denote the renewable power flow in the transmission line $L_l$.
% $L_l$ is the $l$'th transmission line in power system.
Based on the synchronous historical forecast and actual power of each RPP, the renewable power in the transmission lines can again be modeled with a copula:
%\todo{Let $R_{a}^{L_l}$ denote ... explain what this symbol means and what is $L_l$}.
\begin{equation} \label{eq:5}
\begin{aligned}
& F(R_{a}^{L_l},\mathbf{f})= C(F(R_{a}^{L_l}),\Omega(\mathbf{f}))\\
& R_{a}^{L_l}=\sum_{j=1}^{J}k_{l,j}w_{a,j}+\sum_{k=1}^{K}k_{l,k}s_{a,k} \\
\end{aligned}
\end{equation}
{where $k_{l,j}$ and $k_{l,k}$ are the generation distribution shift factors of $j$'th WPP and $k$'th PVPP to transmission line $L_l$, respectively.}
Based on the forecast power of RPPs $w_{f,j}$ and $s_{f,k}$, the conditional distribution of renewable power in the transmission lines $RE_{a}^{L_l}$ can be modeled as follows.
%\todo{Is it necessary to state both the CDF and the PDF?} The CDF is has no close form and can be calculated by the numerical integration of PDF.
\begin{equation} \label{linedistribution}
\begin{aligned}
f(R_{a}^{L_l}|\mathbf{f})
= \frac{c(F(R_{a}^{L_l}),\Omega(\mathbf{f}))}{c(\Omega(\mathbf{f}))} \cdot f(R_{a}^{L_l})
\end{aligned}
\end{equation}

In this paper, (5), (7) and (9) are used in the RTED model. By the suitable Copula function and actual PDH, we can model the renewable power uncertainties and spatial correlation accurately using just one-dimension distributions.

%To model the joint distribution of the actual power production, we use the copula method~\cite{copula_Zhang}. Based on Copula theory, the joint CDF of actual available renewable power $w_{a,j}$ ({\it j}=1...{\it J}), $s_{a,k}$ ({\it k}=1...{\it K}) and the forecast renewable power $w_{f,j}$, $s_{f,k}$ can be modelled as follows \cite{copula_Zhang}.
%\todo{Completely rewrite the rest of this subsection. Explain what the copula method is and why it is used here.}

%For convenient representation, the forecast power set of RPPs, i.e., $w_{f,{1}}...w_{f,{\it j}}...w_{f,J},s_{f,{1}}...s_{f,{\it k}}...s_{f,K}$ ($w_{f,{\it j}}$ and $s_{f,{\it k}}$ is the forecast power of $j$'th WPP and $k$'th CSPP, respectively, {\it J} and {\it K} is the number of WPPs and CSPPs in the system, respectively) is record as $F_{re}$ and the set of marginal cumulative distribution function (CDF) values of each forecast power of RPPs, i.e., $F(w_{f,{1}})...F(w_{f,j})...F(w_{f,J}),F(s_{f,{1}})...F(w_{s,k})...F(w_{s,K})$ is record as $\Omega(F_{re})$ in this paper. The actual power production of the $i$'th WPP is denoted by $w_{a,i}$ and the $j$'th CSPP is denoted by $s_{a,j}$.
