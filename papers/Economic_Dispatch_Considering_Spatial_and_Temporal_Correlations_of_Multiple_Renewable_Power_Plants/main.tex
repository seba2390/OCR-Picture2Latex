%
% IEEE Transactions on Microwave Theory and Techniques example
% Tibault Reveyrand - http://www.microwave.fr
%
% http://www.microwave.fr/LaTeX.html
% ---------------------------------------

% ================================================
% Please HIGHLIGHT the new inputs such like this :
% Text :
%  \hl{comment}
% Aligned Eq.
% \begin{shaded}
% \end{shaded}
% ================================================

\documentclass[journal]{IEEEtran}

%\usepackage[retainorgcmds]{IEEEtrantools}
%\usepackage{bibentry}
% \usepackage{xcolor,soul,framed} %,caption
% \usepackage{stfloats}

% \colorlet{shadecolor}{yellow}
% \usepackage{color,soul}
\usepackage{graphicx}
\usepackage{color}
\graphicspath{{../pdf/}{../jpeg/}}
\DeclareGraphicsExtensions{.pdf,.jpeg,.png}

\usepackage[cmex10]{amsmath}
\usepackage{amssymb}
%Mathabx do not work on ScribTex => Removed
%\usepackage{mathabx}
\usepackage{array}
\usepackage{mdwmath}
%\usepackage{mdwtab}
%\usepackage{eqparbox} notice that I have not find the package
\usepackage{url}
\usepackage{float}
\usepackage{subfig}
\usepackage{multirow}

\newcommand{\todo}[1]{\textcolor{red}{#1}}
\newcommand{\E}{\mathbb{E}}

\hyphenation{op-tical net-works semi-conduc-tor}

%\bstctlcite{IEEE:BSTcontrol}

%=== TITLE & AUTHORS ====================================================================
\begin{document}
\bstctlcite{IEEEexample:BSTcontrol}
    % \title{Economic Dispatch Considering Uncertainties and Correlations of Multiple Renewable Power Variables Integration}
    \title{Economic Dispatch Considering Spatial and Temporal Correlations of Multiple Renewable Power Plants}
    % \title{Economic Dispatch Considering Multiple Correlated Renewable Power Plants}
  \author{Chenghui~Tang\IEEEauthorrefmark{1},~\IEEEmembership{Student Member,~IEEE,}
      Yishen~Wang\IEEEauthorrefmark{2},~\IEEEmembership{Student Member,~IEEE,} \\
      Jian Xu\IEEEauthorrefmark{1},~\IEEEmembership{Member,~IEEE,}
      Yuanzhang~Sun\IEEEauthorrefmark{1},~\IEEEmembership{Senior Member,~IEEE}
      and~Baosen~Zhang\IEEEauthorrefmark{2},~\IEEEmembership{Member,~IEEE,}\\% <-this % stops a space
%\IEEEauthorrefmark{1}: Wuhan University, Wuhan, Hubei, China \IEEEauthorrefmark{2}: University of Washington, Seattle, WA, USA

\thanks{This work was supported in part by the National Key R\&D Program of China (2016YFB0900105), in part by the National Natural Science Foundation of China (51477122)}
\thanks{
	C. Tang, J. Xu and Y. Sun are with the School of Electrical Engineering, Wuhan University, Wuhan, 430072 China.}
\thanks{
	Y. Wang and B. Zhang are with the Department of Electrical Engineering, University of Washington, Seattle, WA, 98195 USA.}}
% The paper headers
% \markboth{IEEE TRANSACTIONS ON MICROWAVE THEORY AND TECHNIQUES, VOL.~60, NO.~12, DECEMBER~2012
% }{Roberg \MakeLowercase{\textit\title{•}{et al.}}:
% Economic Dispatch Considering the Uncertainties and Correlations of Multiple Renewable Energy Variables Integration}

% ====================================================================
\maketitle

% === ABSTRACT ====================================================================
% =================================================================================



% For peer review papers, you can put extra information on the cover
% page as needed:
% \ifCLASSOPTIONpeerreview
% \begin{center} \bfseries EDICS Category: 3-BBND \end{center}
% \fi
%
% For peerreview papers, this IEEEtran command inserts a page break and
% creates the second title. It will be ignored for other modes.
% \IEEEpeerreviewmaketitle

\begin{abstract}
	%\boldmath
	The correlations of multiple renewable power plants (RPPs) should be fully considered in the power system with very high penetration renewable power integration. This paper models the uncertainties, spatial correlation of multiple RPPs based on Copula theory and actual probability historical histograms by one-dimension distributions for economic dispatch (ED) problem. An efficient dynamic renewable power scenario generation method based on Gibbs sampling is proposed to generate renewable power scenarios considering the uncertainties, spatial correlation and variability (temporal correlation) of multiple RPPs, in which the sampling space complexity do not increase with the number of RPPs. Distribution-based and scenario-based methods are proposed and compared to solve the real-time ED problem with multiple RPPs. Results show that the proposed dynamic scenario generation method is much more consist with the actual renewable power. The proposed ED methods show better understanding for the uncertainties, spatial and temporal correlations of renewable power and more economical compared with the traditional ones.
\end{abstract}

% === KEYWORDS ====================================================================
% =================================================================================
\begin{IEEEkeywords}
	Uncertainty, spatial correlation, variability, renewable power, scenario generation, economic dispatch
\end{IEEEkeywords}

% For peer review papers, you can put extra information on the cover
% page as needed:
% \ifCLASSOPTIONpeerreview
% \begin{center} \bfseries EDICS Category: 3-BBND \end{center}
% \fi
%
% For peerreview papers, this IEEEtran command inserts a page break and
% creates the second title. It will be ignored for other modes.
\IEEEpeerreviewmaketitle




\section{Introduction}
% !TEX root = ../arxiv.tex

Unsupervised domain adaptation (UDA) is a variant of semi-supervised learning \cite{blum1998combining}, where the available unlabelled data comes from a different distribution than the annotated dataset \cite{Ben-DavidBCP06}.
A case in point is to exploit synthetic data, where annotation is more accessible compared to the costly labelling of real-world images \cite{RichterVRK16,RosSMVL16}.
Along with some success in addressing UDA for semantic segmentation \cite{TsaiHSS0C18,VuJBCP19,0001S20,ZouYKW18}, the developed methods are growing increasingly sophisticated and often combine style transfer networks, adversarial training or network ensembles \cite{KimB20a,LiYV19,TsaiSSC19,Yang_2020_ECCV}.
This increase in model complexity impedes reproducibility, potentially slowing further progress.

In this work, we propose a UDA framework reaching state-of-the-art segmentation accuracy (measured by the Intersection-over-Union, IoU) without incurring substantial training efforts.
Toward this goal, we adopt a simple semi-supervised approach, \emph{self-training} \cite{ChenWB11,lee2013pseudo,ZouYKW18}, used in recent works only in conjunction with adversarial training or network ensembles \cite{ChoiKK19,KimB20a,Mei_2020_ECCV,Wang_2020_ECCV,0001S20,Zheng_2020_IJCV,ZhengY20}.
By contrast, we use self-training \emph{standalone}.
Compared to previous self-training methods \cite{ChenLCCCZAS20,Li_2020_ECCV,subhani2020learning,ZouYKW18,ZouYLKW19}, our approach also sidesteps the inconvenience of multiple training rounds, as they often require expert intervention between consecutive rounds.
We train our model using co-evolving pseudo labels end-to-end without such need.

\begin{figure}[t]%
    \centering
    \def\svgwidth{\linewidth}
    \input{figures/preview/bars.pdf_tex}
    \caption{\textbf{Results preview.} Unlike much recent work that combines multiple training paradigms, such as adversarial training and style transfer, our approach retains the modest single-round training complexity of self-training, yet improves the state of the art for adapting semantic segmentation by a significant margin.}
    \label{fig:preview}
\end{figure}

Our method leverages the ubiquitous \emph{data augmentation} techniques from fully supervised learning \cite{deeplabv3plus2018,ZhaoSQWJ17}: photometric jitter, flipping and multi-scale cropping.
We enforce \emph{consistency} of the semantic maps produced by the model across these image perturbations.
The following assumption formalises the key premise:

\myparagraph{Assumption 1.}
Let $f: \mathcal{I} \rightarrow \mathcal{M}$ represent a pixelwise mapping from images $\mathcal{I}$ to semantic output $\mathcal{M}$.
Denote $\rho_{\bm{\epsilon}}: \mathcal{I} \rightarrow \mathcal{I}$ a photometric image transform and, similarly, $\tau_{\bm{\epsilon}'}: \mathcal{I} \rightarrow \mathcal{I}$ a spatial similarity transformation, where $\bm{\epsilon},\bm{\epsilon}'\sim p(\cdot)$ are control variables following some pre-defined density (\eg, $p \equiv \mathcal{N}(0, 1)$).
Then, for any image $I \in \mathcal{I}$, $f$ is \emph{invariant} under $\rho_{\bm{\epsilon}}$ and \emph{equivariant} under $\tau_{\bm{\epsilon}'}$, \ie~$f(\rho_{\bm{\epsilon}}(I)) = f(I)$ and $f(\tau_{\bm{\epsilon}'}(I)) = \tau_{\bm{\epsilon}'}(f(I))$.

\smallskip
\noindent Next, we introduce a training framework using a \emph{momentum network} -- a slowly advancing copy of the original model.
The momentum network provides stable, yet recent targets for model updates, as opposed to the fixed supervision in model distillation \cite{Chen0G18,Zheng_2020_IJCV,ZhengY20}.
We also re-visit the problem of long-tail recognition in the context of generating pseudo labels for self-supervision.
In particular, we maintain an \emph{exponentially moving class prior} used to discount the confidence thresholds for those classes with few samples and increase their relative contribution to the training loss.
Our framework is simple to train, adds moderate computational overhead compared to a fully supervised setup, yet sets a new state of the art on established benchmarks (\cf \cref{fig:preview}).



\section {Modeling the Uncertainties and Spatial Correlation of Multiple RPPs in Power System}
Online convex optimization with memory has emerged as an important and challenging area with a wide array of applications, see \citep{lin2012online,anava2015online,chen2018smoothed,goel2019beyond,agarwal2019online,bubeck2019competitively} and the references therein.  Many results in this area have focused on the case of online optimization with switching costs (movement costs), a form of one-step memory, e.g., \citep{chen2018smoothed,goel2019beyond,bubeck2019competitively}, though some papers have focused on more general forms of memory, e.g., \citep{anava2015online,agarwal2019online}. In this paper we, for the first time, study the impact of feedback delay and nonlinear switching cost in online optimization with switching costs. 

An instance consists of a convex action set $\mathcal{K}\subset\mathbb{R}^d$, an initial point $y_0\in\mathcal{K}$, a sequence of non-negative convex cost functions $f_1,\cdots,f_T:\mathbb{R}^d\to\mathbb{R}_{\ge0}$, and a switching cost $c:\mathbb{R}^{d\times(p+1)}\to\mathbb{R}_{\ge0}$. To incorporate feedback delay, we consider a situation where the online learner only knows the geometry of the hitting cost function at each round, i.e., $f_t$, but that the minimizer of $f_t$ is revealed only after a delay of $k$ steps, i.e., at time $t+k$.  This captures practical scenarios where the form of the loss function or tracking function is known by the online learner, but the target moves over time and measurement lag means that the position of the target is not known until some time after an action must be taken. 
To incorporate nonlinear (and potentially nonconvex) switching costs, we consider the addition of a known nonlinear function $\delta$ from $\mathbb{R}^{d\times p}$ to $\mathbb{R}^d$ to the structured memory model introduced previously.  Specifically, we have
\begin{align}
c(y_{t:t-p}) = \frac{1}{2}\|y_t-\delta(y_{t-1:t-p})\|^2,    \label{e.newswitching}
\end{align}
where we use $y_{i:j}$ to denote either $\{y_i, y_{i+1}, \cdots, y_j\}$ if $i\leq j$, or  $\{y_i, y_{i-1}, \cdots, y_j\}$ if $i > j$ throughout the paper. Additionally, we use $\|\cdot\|$ to denote the 2-norm of a vector or the spectral norm of a matrix.

In summary, we consider an online agent that interacts with the environment as follows:
% \begin{inparaenum}[(i)] 
\begin{enumerate}%[leftmargin=*]
    \item The adversary reveals a function $h_t$, which is the geometry of the $t^\mathrm{th}$ hitting cost, and a point $v_{t-k}$, which is the minimizer of the $(t-k)^\mathrm{th}$ hitting cost. Assume that $h_t$ is $m$-strongly convex and $l$-strongly smooth, and that $\arg\min_y h_t(y)=0$.
    \item The online learner picks $y_t$ as its decision point at time step $t$ after observing $h_t,$  $v_{t-k}$.
    \item The adversary picks the minimizer of the hitting cost at time step $t$: $v_t$. 
    \item The learner pays hitting cost $f_t(y_t)=h_t(y_t-v_t)$ and switching cost $c(y_{t:t-p})$ of the form \eqref{e.newswitching}.
\end{enumerate}

The goal of the online learner is to minimize the total cost incurred over $T$ time steps, $cost(ALG)=\sum_{t=1}^Tf_t(y_t)+c(y_{t:t-p})$, with the goal of (nearly) matching the performance of the offline optimal algorithm with the optimal cost $cost(OPT)$. The performance metric used to evaluate an algorithm is typically the \textit{competitive ratio} because the goal is to learn in an environment that is changing dynamically and is potentially adversarial. Formally, the competitive ratio (CR) of the online algorithm is defined as the worst-case ratio between the total cost incurred by the online learner and the offline optimal cost: $CR(ALG)=\sup_{f_{1:T}}\frac{cost(ALG)}{cost(OPT)}$.

It is important to emphasize that the online learner decides $y_t$ based on the knowledge of the previous decisions $y_1\cdots y_{t-1}$, the geometry of cost functions $h_1\cdots h_t$, and the delayed feedback on the minimizer $v_1\cdots v_{t-k}$. Thus, the learner has perfect knowledge of cost functions $f_1\cdots f_{t-k}$, but incomplete knowledge of $f_{t-k+1}\cdots f_t$ (recall that $f_t(y)=h_t(y-v_t)$).

Both feedback delay and nonlinear switching cost add considerable difficulty for the online learner compared to versions of online optimization studied previously. Delay hides crucial information from the online learner and so makes adaptation to changes in the environment more challenging. As the learner makes decisions it is unaware of the true cost it is experiencing, and thus it is difficult to track the optimal solution. This is magnified by the fact that nonlinear switching costs increase the dependency of the variables on each other. It further stresses the influence of the delay, because an inaccurate estimation on the unknown data, potentially magnifying the mistakes of the learner. 

The impact of feedback delay has been studied previously in online learning settings without switching costs, with a focus on regret, e.g., \citep{joulani2013online,shamir2017online}.  However, in settings with switching costs the impact of delay is magnified since delay may lead to not only more hitting cost in individual rounds, but significantly larger switching costs since the arrival of delayed information may trigger a very large chance in action.  To the best of our knowledge, we give the first competitive ratio for delayed feedback in online optimization with switching costs. 

We illustrate a concrete example application of our setting in the following.

\begin{example}[Drone tracking problem]
\label{example:drone} \emph{
Consider a drone with vertical speed $y_t\in\mathbb{R}$. The goal of the drone is to track a sequence of desired speeds $y^d_1,\cdots,y^d_T$ with the following tracking cost:}
\begin{equation}
    \sum_{t=1}^T \frac{1}{2}(y_t-y^d_t)^2 + \frac{1}{2}(y_t-y_{t-1}+g(y_{t-1}))^2,
\end{equation}
\emph{where $g(y_{t-1})$ accounts for the gravity and the aerodynamic drag. One example is $g(y)=C_1+C_2\cdot|y|\cdot y$ where $C_1,C_2>0$ are two constants~\cite{shi2019neural}. Note that the desired speed $y_t^d$ is typically sent from a remote computer/server. Due to the communication delay, at time step $t$ the drone only knows $y_1^d,\cdots,y_{t-k}^d$.}

\emph{This example is beyond the scope of existing results in online optimization, e.g.,~\cite{shi2020online,goel2019beyond,goel2019online}, because of (i) the $k$-step delay in the hitting cost $\frac{1}{2}(y_t-y_t^d)$ and (ii) the nonlinearity in the switching cost $\frac{1}{2}(y_t-y_{t-1}+g(y_{t-1}))^2$ with respective to $y_{t-1}$. However, in this paper, because we directly incorporate the effect of delay and nonlinearity in the algorithm design, our algorithms immediately provide constant-competitive policies for this setting.}
\end{example}


\subsection{Related Work}
This paper contributes to the growing literature on online convex optimization with memory.  
Initial results in this area focused on developing constant-competitive algorithms for the special case of 1-step memory, a.k.a., the Smoothed Online Convex Optimization (SOCO) problem, e.g., \citep{chen2018smoothed,goel2019beyond}. In that setting, \citep{chen2018smoothed} was the first to develop a constant, dimension-free competitive algorithm for high-dimensional problems.  The proposed algorithm, Online Balanced Descent (OBD), achieves a competitive ratio of $3+O(1/\beta)$ when cost functions are $\beta$-locally polyhedral.  This result was improved by \citep{goel2019beyond}, which proposed two new algorithms, Greedy OBD and Regularized OBD (ROBD), that both achieve $1+O(m^{-1/2})$ competitive ratios for $m$-strongly convex cost functions.  Recently, \citep{shi2020online} gave the first competitive analysis that holds beyond one step of memory.  It holds for a form of structured memory where the switching cost is linear:
$
    c(y_{t:t-p})=\frac{1}{2}\|y_t-\sum_{i=1}^pC_iy_{t-i}\|^2,
$
with known $C_i\in\mathbb{R}^{d\times d}$, $i=1,\cdots,p$. If the memory length $p = 1$ and $C_1$ is an identity matrix, this is equivalent to SOCO. In this setting, \citep{shi2020online} shows that ROBD has a competitive ratio of 
\begin{align}
    \frac{1}{2}\left( 1 + \frac{\alpha^2 - 1}{m} + \sqrt{\Big( 1 + \frac{\alpha^2 - 1}{m}\Big)^2 + \frac{4}{m}} \right),
\end{align}
when hitting costs are $m$-strongly convex and $\alpha=\sum_{i=1}^p\|C_i\|$. 


Prior to this paper, competitive algorithms for online optimization have nearly always assumed that the online learner acts \emph{after} observing the cost function in the current round, i.e., have zero delay.  The only exception is \citep{shi2020online}, which considered the case where the learner must act before observing the cost function, i.e., a one-step delay.  Even that small addition of delay requires a significant modification to the algorithm (from ROBD to Optimistic ROBD) and analysis compared to previous work. 

As the above highlights, there is no previous work that addresses either the setting of nonlinear switching costs nor the setting of multi-step delay. However, the prior work highlights that ROBD is a promising algorithmic framework and our work in this paper extends the ROBD framework in order to address the challenges of delay and non-linear switching costs. Given its importance to our work, we describe the workings of ROBD in detail in Algorithm~\ref{robd}. 

\begin{algorithm}[t!]
  \caption{ROBD \citep{goel2019beyond}}
  \label{robd}
\begin{algorithmic}[1]
  \STATE {\bfseries Parameter:} $\lambda_1\ge0,\lambda_2\ge0$
  \FOR{$t=1$ {\bfseries to} $T$}
  \STATE {\bfseries Input:} Hitting cost function $f_t$, previous decision points $y_{t-p:t-1}$
  \STATE $v_t\leftarrow\arg\min_yf_t(y)$
  \STATE $y_t\leftarrow\arg\min_yf_t(y)+\lambda_1c(y,y_{t-1:t-p})+\frac{\lambda_2}{2}\|y-v_t\|^2_2$
  \STATE {\bfseries Output:} $y_t$
  \ENDFOR
   
\end{algorithmic}
\end{algorithm}

Another line of literature that this paper contributes to is the growing understanding of the connection between online optimization and adaptive control. The reduction from adaptive control to online optimization with memory was first studied in \citep{agarwal2019online} to obtain a sublinear static regret guarantee against the best linear state-feedback controller, where the approach is to consider a disturbance-action policy class with some fixed horizon.  Many follow-up works adopt similar reduction techniques \citep{agarwal2019logarithmic, brukhim2020online, gradu2020adaptive}. A different reduction approach using control canonical form is proposed by \citep{li2019online} and further exploited by \citep{shi2020online}. Our work falls into this category.  The most general results so far focus on Input-Disturbed Squared Regulators, which can be reduced to online convex optimization with structured memory (without delay or nonlinear switching costs).  As we show in \Cref{Control}, the addition of delay and nonlinear switching costs leads to a significant extension of the generality of control models that can be reduced to online optimization. 

\section{Scenario Generation}
%% !TEX root=econ_dispatch.tex
In this section, we first propose a reliable static renewable power scenario generation method in each time interval $1,\dots,T$. Then we present an efficient dynamic renewable power scenario generation method for the entire time horizon.

\subsection {Static Scenario Generation}

By the joint distribution of multiple RPPs in \eqref{cjdistribution}, scenarios can be generated to represent the uncertainties and spatial correlation of all RPPs in the system. However, with the increase of the number of RPPs, classical random sampling methods such as inverse transform sampling and Latin hypercube sampling \cite{L_sampling} become hard to be employed due to matrix size and computational limitations. Other classical sampling methods such as rejection sampling tend to have very large rejection rate for a high number of dimensions.

To this end, a reliable static renewable power scenario generation method based on Gibbs sampling \cite{Gibbs} is proposed to sample for the conditional joint distribution function of actual available power of RPPs in \eqref{cjdistribution}. Compared with directly sampling by the conditional joint distribution \cite{copula_Zhang}, Gibbs sampling converts the sampling process of joint distribution in \eqref{cjdistribution} to $J+K$ sampling processes of conditional distribution in \eqref{ccdistribution}. Namely, let $U$ be a random variable generated uniformly within $[0,1]$, then each RPP can be sampled via the inverse transform:
\begin{equation} \label{inversesampling}
w_{a,j}=F_{a,j}^{-1}(U),\quad s_{a,k}=F_{a,k}^{-1}(U)
\end{equation}
where $F_{a,j}^{-1}$ and $F_{a,k}^{-1}$ is the inverse function of $F_{a,j}$ and $F_{a,k}$, respectively.

Gibbs sampling needs a burn-in process \cite{burn_in} before it converges to the true distribution in \eqref{cjdistribution}. So we throw out $N_{b}$ (e.g. 1000) samples in the beginning the process. The detailed procedure of static scenarios generation is:
\begin{enumerate}%[noitemsep,nolistsep]
	\item Setting the number of renewable power scenarios: $N_{sc}$ (e.g. 5000), the total number of samples is $N_{sc}+N_{b}$.
	\item Setting the initial sampling values to be the forecasted power for each RPP.
	% $w_{a,{1}}^{i}$,...,$w_{a,j}^{i}$,..., $w_{a,J}^{i}$, $s_{a,{\it 1}}^{i}$,...,$s_{a,k}^{i}$,...,$s_{a,K}^{i}$, {\it i}=0...$N_{sc}+N_{b}$, ({\it i}=0 at this step). To  speed up the burn-in process, the forecast power of each RPP (i.e. $F_{re}$) are regarded as the initial sampling value.
	\item Employing inverse transform sampling in \eqref{inversesampling} in a round robin fashion for each scenario generation step (indexed by $i$):

\begin{itemize}
	\item $f(w_{a,{1}}^{i}|w_{a,2}^{i}...w_{a,J}^{i},s_{a,{1}}^{i}...s_{a,K}^{i},\mathbf{f})$
	\item $f(w_{a,{\it j}}^{i}|w_{a,{1}}^{i+1}...w_{a,{{\it j}-1}}^{i+1},w_{a,{{\it j}+1}}^{i}...w_{a,J}^{i},s_{a,{1}}^{i}...s_{a,K}^{i},\mathbf{f})$
	\item $...$
	\item $f(s_{a,{\it k}}^{i}|w_{a,{1}}^{i+1}...w_{a,J}^{i+1},s_{a,{1}}^{i+1}...s_{a,{{\it k}-1}}^{i+1},s_{a,{{\it k}+1}}^{i}...s_{a,K}^{i},\mathbf{f})$
	\item $f(s_{a,{\it K}}^{i}|w_{a,{1}}^{i+1}...w_{a,J}^{i+1},s_{a,{1}}^{i+1}...s_{a,{{\it K}-1}}^{i+1},\mathbf{f})$
\end{itemize}

	\item Repeating 3 from {\it i}=1...$N_{sc}+N_{b}$. Disregard the first $N_{b}$ scenarios and we get $N_{sc}$ renewable power scenarios.

\end{enumerate}

{An important feature of the proposed static scenario generation method is that with the increase of the number of RPPs, the computational space complexity remains same and the computational time complexity increases linearly, effectively mitigating the curse of dimensionality.}

\subsection {Dynamic Scenario Generation}
%\todo{Why is this dynamic? Also, does variability just mean correlation?}
{A dynamic scenario is a scenario that considers the variability (i.e., temporal correlation) of the output of a RPP.} The method presented in the last section can generate renewable power scenarios of conditional joint distribution (c.f. \eqref{cjdistribution}) which captures the marginal uncertainties and spatial correlation. In this section we extend it to capture the temporal correlation among the time points in a scenario, which is also of vital importance in power system operations~\cite{sce_generation_Ma,PCA,sce_generation_Pinson}.
 % which represent the uncertainties and correlations in each time interval \todo{(i.e., spatial correlation)}. However, for renewable power scenarios, variability is as same importance as uncertainties \cite{sce_generation_Ma}\cite{PCA}\cite{sce_generation_Pinson}.

To capture the variability, some new variables are introduced. Take a WPP for instance, a new random variable $Z_{a,j}^{t}$ is introduced which follows
the standard Gaussian distribution with zero mean and unit standard deviation. Since the value of CDF of $Z_{a,j}^{t}$ is uniformly distributed over [0,1], the uniform distribution $U$ in \eqref{inversesampling} can be replaced by a CDF $\Phi(Z_{a,j}^{t})$.  Given the realization of random variable $Z_{a,j}^{t}$, $w_{a,j}^{t}$ can be sampled as follows:



\begin{equation} \label{transform}
\begin{aligned}
w_{a,j}^t=F_{a,j}^{-1}(\Phi(Z_{a,j}^{t}))
\end{aligned}
\end{equation}

To consider the variability of each RPP, it is assumed that the joint distribution of $Z_{a,j}^{t}$ follows a multivariate Gaussian distribution $Z_{a,j}^{t} \sim N(\mu_{j},\Sigma_{j})$. The expectation of $\mu_{j}$ is a vector of zeros and the covariance matrix $\Sigma_{j}$ satisfies


\begin{equation} \label{matrix}
\Sigma_j=\left[
\begin{matrix}
\sigma_{1,1}^{j}&\sigma_{1,2}^{j}&\dots&\sigma_{1,{\it T}}^{j}&\\
\sigma_{2,1}^{j}&\sigma_{2,2}^{j}&\dots&\sigma_{2,{\it T}}^{j}&\\
\vdots&\vdots&\ddots&\vdots&\\
\sigma_{{\it T},1}^{j}&\sigma_{{\it T},2}^{j}&\dots&\sigma_{{\it T},{\it T}}^{j}&\\
\end{matrix}
\right]
\end{equation}

\noindent where $\sigma_{m,n}^{j}=cov(Z_{a,j}^{m},Z_{a,j}^{n})$, {\it m}, {\it n}=1,2...{\it T}, $\sigma_{{\it m}, {\it n}}^{j}$ is the covariance of $Z_{a,j}^{m}$ and $Z_{a,j}^{n}$.

The covariance structure of $\Sigma_j$ can be identified by covariance $\sigma_{m,n}^{j}$. As is done in \cite{sce_generation_Ma}\cite{sce_generation_Pinson}, an exponential covariance function is employed to model $\sigma_{m,n}^{j}$ in \eqref{matrix},

\begin{equation} \label{exponential}
\begin{aligned}
\sigma_{m,n}^{j}=\rm exp(-\frac{|{\it m}-{\it n}|}{\epsilon_{\it j}}) \quad 0 \le {\it m},  {\it n} \le {\it T}
\end{aligned}
\end{equation}

\noindent where $\epsilon_{\it j}$ is the range parameter controlling the strength of the
correlation of random variables $Z_{a,j}^{t}$ among the set of lead-time. Similar to \cite{sce_generation_Ma}, $\epsilon_{\it j}$ can be determined by comparing the distribution of renewable power variability of the generated scenarios by the indicator in \cite{sce_generation_Ma}. Here, assuming that the  range parameter $\epsilon_{\it j}$ of each RPP have been obtained, the flowchart of dynamic renewable power scenario generation method is as shown in Fig.~\ref{flowchart}.

\begin{figure}[!htb]
	\begin{center}
		\includegraphics[trim = 10 250 60 200, clip, width=1.0\columnwidth]{flowchart.eps}\\
		\caption{Flowchart of dynamic renewable power scenario generation method}\label{flowchart}
	\end{center}
\end{figure}

Before generating $N_{sc}$ scenarios, small amount of scenarios are generated to obtain the range parameter of each RPP. After all the range parameters in \eqref{matrix} are obtained, we can start the dynamic wind power scenarios generation in Fig.~\ref{flowchart}. At each time interval, they follow the conditional joint distribution in \eqref{cjdistribution} and among the time horizon, the variability is considered.

One thing that need to be noticed is that each static scenario generation process in Fig. 1 does not affect each other after the random data set is determined. Parallel computing can be employed to increase the computation efficiency to meet the real-time requirement.

In scenario-based method, the above generated scenarios should be reduced to certain number of scenarios that deemed as the most probability occur. A scenario reduction method in \cite{YishenWang} is employed in this paper for the reason that it has great efficiency compared with other methods to meet the real-time requirement.


\section{Distribution-Based ED} \label{sec:dist}
%% !TEX root=econ_dispatch.tex
In this section we study the RTED problem where CPPs outputs, system reserve and potential risk of {renewable energy curtailment} and {load shedding} are balanced. We consider an hourly dispatch with $T=12$ intervals where each one is 5 minutes long. The objective function of the ED problem is:
%\todo{spell out REC and LS}
% ED model based on the distribution of sum actual available power of all RPPs in \eqref{sumdistribution} is proposed in this section, in which the
%
% % \subsection {Distribution-Based Cost Modeling}
% RTED with renewable power integration creates scheduled power of CPPs, WPPs and CSPPs of interval {\it t}=1 . . .{\it T} by the newest load, wind power and solar power forecast values of {\it T} intervals (in this paper, {\it T}=12, each time interval is 5min and the time horizon is 60min) in RTED, minimizing the total cost under the constraints. Distribution-based RTED model with multiple RPPs is as follows.\\
%
\begin{equation} \label{DB_1}
\begin{aligned}
\begin{aligned}
min\sum _{{\it t}={\it 1}}^{{\it T}}E[f_t]& =\sum _{{\it t}=1}^{{\it T}}E[f_{c,t}(p_{i,t},r_{u,i,t},r_{d,i,t})]\\
& +\sum _{{\it t}=1}^{{\it T}}E[f_{R,t}(w_{c,j,t},s_{c,k,t},l_{s,b,t})]\\
\end{aligned}
\end{aligned}
\end{equation}
where $f_t$  is the total system cost at time {\it t};
$f_{c,t}$ is the total CPP cost at time {\it t}; $f_{R,t}$ is the total penalty cost caused by renewable power uncertainties (REC and LS); $p_{i,t}$ is the schedule power of $i$'th CPP at time {\it t}; $r_{u,i,t}$ and $r_{d,i,t}$ is the upward and downward reserve of $i$'th CPP at time {\it t}, respectively; $w_{c,j,t}$ and $s_{c,k,t}$ is the power of REC of $j$'th WPP and $k$'th PVPP at time {\it t}, respectively; $l_{s,b,t}$ is the power of LS of $b$'th bus at time {\it t}.

The CPP cost is given by
\begin{equation} \label{DB_2}
\begin{aligned}
& f_{c,t}(p_{i,t},r_{u,i,t},r_{d,i,t})\\
=\sum _{{\it i}=1}^{{\it I}}(b_{f,i}p_{i,t} & +c_{f,i}+c_{ur,i}r_{u,i,t}+c_{dr,i}r_{d,i,t})
\end{aligned}
\end{equation}
where {\it I} is the total number of CPPs; $b_{f,i}$ and $c_{f,i}$ are the fuel cost coefficients of $i$'th CPP, respectively; $c_{ur,i}$ and $c_{dr,i}$ are the cost coefficients of upward and downward reserve of $i$'th CPP, respectively.
Penalties with respect to uncertainties in the renewable powers are given by:
\begin{equation} \label{DB_3}
E[f_{R,t}(w_{c,j,t},s_{c,k,t},l_{s,b,t})]=c_{ls}E_{ls,t}+c_{rec}E_{rec,t}\\
\end{equation}
where $c_{ls}$ and $c_{rec}$ is the penalty coefficients of LS and REC, respectively; $E_{ls,t}$ and $E_{rec,t}$ is the expected values of LS and REC, respectively.

For ease of analysis, the sum scheduled renewable energy $R_{t}^{\Sigma}$ is introduced as an internal variable in the distribution-based ED model for the balance of power system.
%, as shown in Fig.~\ref{uncertainties}.

%\begin{figure}[ht]
%	\begin{center}
%		\includegraphics[width=3in]{uncertainties.png}\\
%		\caption{Cost modeling of the renewable energies uncertainties}\label{uncertainties}
%	\end{center}
%\end{figure}

$R_{a,t}^{\Sigma}$ is the sum actual available power of all RPPs at time {\it t} as shown in \eqref{sumdistribution}. $\underline{R}_{t}$ and $\overline{R}_{t}$ is the lower and upper bound that renewable power can be compensated by system reserves at time {\it t}, respectively. In worse case, if the sum actual renewable power locates in the outside of $[\underline{R}_{t}, \overline{R}_{t}]$, system reserve cannot cover all the uncertainties of renewable power. At this time, LS or REC would be employed for the power balance of the system. Then the total penalty cost of renewable power $f_{R,t}(w_{c,j,t},s_{c,k,t},l_{s,b,t})$ can be converted to $f_{R,t}(\underline{R}_{t},\overline{R}_{t})$ and written as
%The penalty cost caused by renewable power include two parts as shown in Fig.~\ref{uncertainties}.
\begin{equation} \label{DB_4}
\begin{aligned}
\begin{aligned}
& E[f_{R,t}(w_{c,j,t},s_{c,k,t},l_{s,b,t})]=f_{R,t}(\underline{R}_{t},\overline{R}_{t})\\
& =c_{ls}\int_{0}^{\underline{R}_{t}}(\underline{R}_{t}-R_{a,t}^{\Sigma})f(R_{a,t}^{\Sigma})dR_{a,t}^{\Sigma}\\
& +c_{rec}\int_{\overline{R}_{t}}^{R_{r}}(R_{a,t}^{\Sigma}-\overline{R}_{t})f(R_{a,t}^{\Sigma})dR_{a,t}^{\Sigma}\\
\end{aligned}
\end{aligned}
\end{equation}
where $R_{r}$ is the total capacity of renewable power.

% For ED problem with multiple RPPs, the penalty costs of renewable power uncertainties that cause LS and REC are considered in \eqref{DB_4}.
Compared with other classical stochastic ED methods that use a predefined confidence level to convert the reserve chance constraints to be linear ones \cite{Versatile}\cite{chance_constrain}\cite{ST_ED}, we can seek the optimal confidence level to find the balance for CPPs outputs, system reserve and potential risk of REC and LS according to different situations.

% \subsection {Total ED Model in Power System With Multiple RPPs}

All constraints of the proposed distribution-based ED model are as follows:
\begin{equation} \label{DB_5}
\sum_{{\it i}=1}^{{\it I}}p_{i,t}+R_{t}^{\Sigma}=L_{t} \quad \forall t
\end{equation}
\begin{equation} \label{DB_6}
\begin{aligned}
& R_{t}^{\Sigma}-\sum_{{\it i}=1}^{{\it I}}r_{u,i,t}=\underline{R}_{t} \quad \forall t \\
& R_{t}^{\Sigma}+\sum_{{\it i}=1}^{{\it I}}r_{d,i,t}=\overline{R}_{t} \quad \forall t
\end{aligned}
\end{equation}
\begin{equation} \label{DB_7}
\begin{aligned}
p_{i,t} + r_{u,i,t} \leq p_{max,i} \quad \forall i,t \\
p_{i,t} - r_{d,i,t} \geq p_{min,i} \quad \forall i,t
\end{aligned}
\end{equation}
\begin{equation} \label{DB_8}
\begin{aligned}
p_{i,t}-p_{i,t-1} & \leq \Delta p_{u,max,i} \quad \forall i,t \\
p_{i,t-1}-p_{i,t} & \leq \Delta p_{d,max,i} \quad \forall i,t
\end{aligned}
\end{equation}
\begin{equation} \label{DB_9}
\begin{aligned}
0 & \leq r_{u,i,t} \leq r_{u,max,i} \quad \forall i,t \\
0 & \leq r_{d,i,t} \leq r_{d,max,i} \quad \forall i,t
\end{aligned}
\end{equation}
\begin{equation} \label{DB_10}
\begin{aligned}
0 \leq \underline{R}_{t},\;  \overline{R}_{t}\leq R_{r} \quad \forall t
\end{aligned}
\end{equation}

%\begin{equation} \label{DB_11}
%\begin{aligned}
%\underline{RE}_{t} & \le F_{RE_{a,t}}^{-1}(1-conf_{ur}) \quad \forall t \\
%\overline{RE}_{t} & \ge F_{RE_{a,t}}^{-1}(conf_{dr}) \quad \forall t \\
%\end{aligned}
%\end{equation}

\vspace{-1em}

\begin{equation} \label{DB_12}
\begin{aligned}
\begin{aligned}
& \sum_{i=1}^{I}k_{l,i}p_{i,t}+\underline{R}_{a}^{L_{\it l}}-\sum_{{\it b}=1}^{{\it Nb}}k_{l,b}L_{b,t} \ge -Pl_l^{max}  \quad \forall l,t\\
& \sum_{i=1}^{I}k_{l,i}p_{i,t}+\overline{R}_{a}^{L_{\it l}}-\sum_{{\it b}=1}^{{\it Nb}}k_{l,b}L_{b,t} \le  Pl_l^{max}  \quad \forall l,t
\end{aligned}
\end{aligned}
\end{equation}
where
\begin{itemize}

	\item \eqref{DB_5} is the supply-demand balance constraint; $L_{t}$ is the forecast power demand at time {\it t};

	\item \eqref{DB_6} is the system reserve constraint;

	\item \eqref{DB_7} are the CPPs scheduled power plus reserve capacity constraint;  $p_{max,i}$  and   $p_{min,i}$ are the upper and lower generation limit of the $i$'th CPP, respectively;

	\item \eqref{DB_8} are the CPPs ramp-rate constraint; $\Delta p_{u,max,i}$ and $\Delta p_{d,max,i}$ are the maximum amount of upward and downward ramp rate of $i$'th CPP within a specific time period (e.g., 5min), respectively;

	\item \eqref{DB_9} are the reserve capacity constraints; $r_{u,max,i}$ and $r_{d,max,i}$ are the maximum amount of up and down reserves that the $i$'th CPP is capable of providing, respectively;

	\item \eqref{DB_10} are the confidence level bound constraint;

%	\item \eqref{DB_11} are the compulsive system reserve that needed under the desired confident level;

	\item \eqref{DB_12} are the transmission capacity constraint; {\it Nb} is the total number of buses; $Pl_l^{max}$ is the transmission capacity limit on transmission line $l$; based on the the distribution of renewable power in the transmission lines $R_{a}^{L_l}$ in \eqref{linedistribution}, the uncertainties and correlations of multiple power energy can be considered compared with the classical model in \cite{AI} and  \cite{VersatileMixture} that used the forecast or scheduled renewable power. A conservative bound such as 99.9\% can be used in this constraint.
	% In this paper, we use the bound of $RE_{a}^{L_l}$ (e.g. 99.9\%) as shown in \eqref{DB_12} for the conservative aim.
\end{itemize}



%SLP-based algorithm \cite{SLP} is employed to solve the proposed distribution-based RTED model in this paper.

\section{Scenario-Based ED}
%% !TEX root=econ_dispatch.tex
%\todo{Transition better. Explain why scenario based ED is considered. What's the difference with respect to the distribution based ED?}
Different from the distribution-based ED, scenario-based ED incorporate the renewable power uncertainties by a certain number of possible renewable power series (i.e., scenarios). This means that scenario-based ED is essentially a deterministic optimization. This allows a more flexible way to model the risk of renewable power such as REC caused by certain transmission line congestion. However, the performance of scenario-based ED greatly relies on the number of scenarios that are considered in the ED. RTED model based on the scenario of multiple RPPs is proposed in this section. The potential risk of LS and REC caused by system reserve deficiency and transmission congestion are modeled by the scenario-based ED. The penalty cost caused by renewable power uncertainties (REC and LS) in (14) and (16) can be written using scenarios as:
%\todo{What is the drawback of scenario based ED then? Should we always use scenario-based ED? Or sometimes distribution-based ED?}
\begin{equation} \label{SB_1}
\begin{aligned}
\begin{aligned}
&E[f_{R,t}(w_{c,j,t},s_{c,k,t},L_{s,b,t})]  \\
&=\sum_{{\it sc}=1}^{{\it SC}}[p^{sc}(c_{rec}(\sum_{{\it j}=1}^{{\it J}}w_{c,j,t}^{sc}+\sum_{{\it k}=1}^{{\it K}}s_{c,k,t}^{sc})
+c_{ls} \sum_{{\it b}=1}^{{\it Nb}}L_{s,b,t}^{sc})]
\end{aligned}
\end{aligned}
\end{equation}
where $sc$ is the {\it sc}-th scenario for WPPs and PVPPs, $SC$ is the number of renewable power scenarios in RTED model, $w_{c,j,t}^{sc}$ is the amount of wind power curtailment of $j$'th WPP at time {\it t} of $sc$'th scenario; $s_{c,k,t}^{sc}$ is the amount of solar power curtailment of $k$'th PVPP at time {\it t} of $sc$'th scenario; $L_{s,b,t}^{sc}$ is the amount of LS of $b$'th bus at time {\it t} of $sc$'th scenario.

Then the optimization problem is same as in Section~\ref{sec:dist}, except the constraints and objectives are represented with scenarios. In particular, the constraints are:
\begin{equation} \label{SB_2}
\begin{aligned}
0  \leq w_{c,j,t}^{sc} \leq w_{a,j,t}^{sc} \quad \forall j,t,sc \\
0  \leq s_{c,k,t}^{sc} \leq s_{a,k,t}^{sc} \quad \forall k,t,sc
\end{aligned}
\end{equation}

\begin{equation} \label{SB_3}
0  \leq L_{s,b,t}^{sc} \leq L_{b,t} \quad \forall b,t,sc \\
\end{equation}

\vspace{-1em}

\begin{equation} \label{SB_4}
-r_{d,i,t}  \leq r_{a,i,t}^{sc} \leq r_{u,i,t} \quad \forall i,t,sc \\
\end{equation}

\vspace{-1.5em}

\begin{equation} \label{SB_5}
\begin{aligned}
& \sum_{{\it i}=1}^{{\it I}}(p_{i,t}+r_{a,i,t}^{sc})+\sum_{{\it j}=1}^{{\it J}}(w_{a,j,t}^{sc}-w_{c,j,t}^{sc}) \\
& +\sum_{{\it k}=1}^{{\it K}}(s_{a,k,t}^{sc}-s_{c,k,t}^{sc})=L_{t}-\sum_{{\it b}=1}^{{\it Nb}}L_{s,b,t}^{sc} \quad \forall t,sc
\end{aligned}
\end{equation}

\vspace{-1.5em}

\begin{equation} \label{SB_6}
\begin{aligned}
\begin{aligned}
\begin{aligned}
&|\sum_{i=1}^{I}k_{l,i}(p_{i,t}+r_{a,i,t}^{sc})+\sum_{j=1}^{J}k_{l,j}(w_{a,j,t}^{sc}-w_{c,j,t}^{sc}) \\
&+\sum_{k=1}^{K}k_{l,k}(s_{a,k,t}^{sc}-s_{c,k,t}^{sc})-\sum_{{\it b}=1}^{{\it Nb}}k_{l,b}(L_{b,t}-L_{s,b,t}^{sc})|\\ & \le  Pl_l^{max} \quad \forall l,t,sc
\end{aligned}
\end{aligned}
\end{aligned}
\end{equation}
%\begin{equation} \label{SB_7
%\begin{aligned}
%	\sum_{\it sc=1}^{{\it SC}}(p^{sc}(\sum_{\it j=1}^{{\it J}}w_{c,j,t}^{sc}+\sum_{\it k=1}^{{\it K}}s_{c,k,t}^{sc}))\leq RE_{c,set}  \quad %\forall t\\
%	\sum_{\it sc=1}^{{\it SC}}(p^{sc}\sum_{\it b=1}^{{\it Nbus}}L_{s,b,t}^{sc})\leq L_{s,set}\quad \forall t
%\end{aligned}
%\end{equation}}
where
\begin{itemize}

	\item \eqref{SB_2} is the actual amount of REC constraint; $w_{a,j,t}^{sc}$ is actual wind power of $j$'th WPP at time {\it t} of $sc$'th scenario; $s_{a,k,t}^{sc}$ is actual solar power of $k$'th PVPP at time {\it t} of $sc$'th scenario;

	\item \eqref{SB_3} is the actual amount of LS constraint;

	\item \eqref{SB_4} is the actual amount of reserve constraint; $r_{a,i,t}^{sc}$ is actual amount of reserve of $i$'th CPP at time {\it t} of $sc$'th scenario;

	\item \eqref{SB_5} is the supply-demand balance constraint;

	\item \eqref{SB_6} is the transmission capacity constraint;

%	\item \eqref{SB_7} is the compulsive maximal amount of  REC and LS constraint for the conservative aim.

\end{itemize}

% \subsection {Comparison With the Distribution-Based ED Model}

Compared with the proposed distribution-based ED model, the scenario-based ED model can not only model the cost of LS and REC caused by system reserve deficiency but also can model the cost of LS and REC caused by transmission congestion. However, the number of scenarios after reduction is limit due to the computation ability. This would reduce the representation accuracy of renewable power in the above scenario-based RTED model, which would be discussed in Section VI.

When the reserve deficiency or transmission congestion occur, REC and LS have to be employed for the balance of system power. Optimal REC and LS strategies can be obtained by solving the static optimization problem (ob. \eqref{SB_1}, s.t. {\eqref{SB_2}-\eqref{SB_6})} by the deterministic value of CPPs scheduled power, actual reserve, actual power of WPPs and PVPPs (the only scenario in this optimization problem).

% \subsection {The Feature of Distribution-Based ED Method in Power System With Multiple Renewable Energy Variable}

% It is easy to prove that the above distribution-based ED model is convex and SLP-based algorithm can be employed to solved the above model.

% Compared with other modeling method such as scenario-based method that use several dispersed renewable energy curves, the proposed distribution-based ED method has two advantages, The first is that the distribution-based model is continuous and it has a better understanding of the renewable energy uncertainties. The second is that distribution-based ED method has higher computational efficiency.



% \subsection {Scenario-Based ED Model}


\section{Case Study}
%% !TEX root=econ_dispatch.tex
%The 1, 2, 3, 4-{\it th} WPP, 1 and 2-{\it th} CSPP is connected on the 11, 15, 27, 62, 78 and 80-{\it th} bus, respectively and each capacity is 200MW. The system load remains 4242MW in the RTED horizon. The forecast outputs of all RPPs from  10:05-11:00 ({\it t}=1, {\it t}=2,...,{\it t}=12) a.m. on October 11{\it th} are used and as shown in Fig.~\ref{forecast}.

The IEEE 118-bus system is employed to validate the stochastic dynamic RTED model with multiple RPPs. There are 10 WPPs and 4 PVPPs each with a capacity of 200MW, {connecting on the 10, 24, 25, 26, 61, 65, 69, 72, 73, 87, 89, 91, 111 and 113 buses, respectively.} The data of RPPs are obtained by synchronous data in Kansas 2006 produced by NREL~\cite{Nrel}.
% All the locations of RPPs are as shown in Fig.~\ref{location}.
Their corresponding forecast power of each RPP is generating by the persistence forecast method. Gaussian Copula is used to model the conditional distribution that needed in this paper. We consider a time period with 12 intervals, modeling the 5 minute dispatch within an hour.
% Assuming that the current time is 0min and the schedule horizon is 5min, 10min, ..., 60min.}
% \begin{figure}[ht]
% 	\begin{center}
% 		\includegraphics[width=3.3in]{renewlocation.png}\\
% 		\caption{The locations of WPPs and  CSPPs in State of Kansas}\label{location}
% 	\end{center}
% \end{figure}

% \begin{figure}[H]
% 	\begin{center}
% 		\includegraphics[width=3.2in]{forecast.png}\\
% 		\caption{Forecast outputs of WPPs and CSPPs}\label{forecast}
% 	\end{center}
% \end{figure}

\subsection {Uncertainties Modeling of Multiple RPPs}

Conditional PDF of actual power of 3'th WPP, 1'th PVPP, sum renewable power and renewable power in 180'th transmission line  are as shown in Fig.~\ref{actual}. We can see that with the increase of forecast power of 3'th WPP from 5min-60min, the location of marginal distribution of actual power also moved to right. Although the conditional PDF of wind power seems to be relatively fat, the forecast error of sum renewable power tends to be thinner due to the independence of WPPs and PVPPs.
\begin{figure}[ht]
	\begin{center}
		\includegraphics[trim = 30 300 40 270, clip, width=1.0\columnwidth]{CPDFofWPP3.eps}\\
		\caption{Conditional PDFs of actual outputs}\label{actual}
	\end{center}
\end{figure}


To further analyze the renewable power independence, conditional joint PDFs of two RPPs of {\it t}=1 are shown in Fig.~\ref{jointpdf} for its high sensitivity. We can see that the power of 1'th WPP and 3'th WPP show the feature of positive correlation since they are geographically close. In contrast, the 1'th WPP and 4'th WPP are further apart and has smaller correlations.
%\todo{Change Fig.~\ref{jointpdf} to only include these two plots.}
% Interestingly, the 2-{\it th} WPP and 1-{\it th} CSPP that has nearly same location, negative correlation can be seen. The above qualitative feature is same with quantitative feature by Gaussian Copula function.
\begin{figure}[ht]
	\begin{center}
		\includegraphics[trim = 30 350 60 370, clip, width=1.0\columnwidth]{jointpdf.eps}\\
		\caption{Conditional joint PDFs of actual output of two RPPs}\label{jointpdf}
	\end{center}
\end{figure}

\vspace{-1em}

\subsection {Renewable Power Scenarios}
To generate renewable power scenarios that captures the spatial correlation, renewable power scenarios should follow the joint distribution in \eqref{cjdistribution}. However, it is usually hard to use \eqref{cjdistribution} directly. For instance, a $100^{14}$ size matrix would be needed to store the joint distribution with 0.01p.u. resolution in this case. In contrast, thanks to the Gibbs theory, renewable power scenarios can be generated by the proposed method with only 100 size matrix to store the conditional distribution in \eqref{ccdistribution} with same resolution.

To show the effect of variability (temporal correlation), we generate renewable power scenarios by our proposed method and the method in \cite{copula_Zhang}, respectively. $N_{sc}$ is 5000, $N_{b}$ is 1000 in this case. Fig.~\ref{scenario1} shows the former 50 scenarios in $N_{sc}$ of 3'rd WPP based on our proposed method~(the left figure) and the method in \cite{copula_Zhang}~(the right figure). The red line and black line in Fig.~\ref{scenario1} is the forecast power and actual power, respectively. We can see that the above two scenarios set have the same distribution in each time interval while the renewable power scenarios of our method are much more similar to the actual renewable power. The economic comparison is discussed in Section~\ref{sec:econ}.
\begin{figure}[ht]
	\begin{center}
		\includegraphics[trim = 135 340 130 360, clip, width=1.0\columnwidth]{scenariosofWPP3.eps}\\
		\caption{The left picture shows the scenarios generated when time correlations are considered (our method) v.s. scenarios that do not consider correlations in time (on the right, standard method). By considering temporal correlations, much more realistic scenarios can be generated.}\label{scenario1}
		%\caption{Scenarios generation influenced by wind power variability }\label{scenarios}
	\end{center}
\end{figure}




% \begin{figure}[ht]
% 	\begin{center}
% 		\includegraphics[width=3.5in]{reducedscenario.png}\\
% 		\caption{Scenarios after reduction to represent renewable power uncertainties at time {\it t}=1}% % \label{reducedscenarios}
% 	\end{center}
% \end{figure}

\vspace{-1em}

\subsection {Economy Comparison of Different RTED Methods} \label{sec:econ}

The generated renewable power scenarios are reduced to 10 scenarios and incorporated in proposed scenario-based ED. To compare the economy of the proposed RTED model, scheduled power of CPPs obtained by different RTED model are tested with other generated 10000 scenarios that consider the variability. {The cost coefficients of upward and downward reserve are all 10\$/MW.} The penalty coefficients of LS and REC is 1000\$/MW and 80\$/MW, respectively. The following five RTED models are compared in this paper.

{\it Case1}: The proposed distribution-based RTED model. {\it Case2}: The proposed distribution-based RTED model while the transmission capacity constraint use the forecast renewable power in \cite{VersatileMixture}. {\it Case3}: The proposed scenario-based RTED model. {\it Case4}: The proposed scenario-based RTED model while does not consider the variability of renewable power as in \cite{copula_Zhang}. {\it Case5}: Scenario-based RTED model that uses the marginal distribution of each RPP by \cite{sce_generation_Ma}, i.e. does not consider the spatial correlation of RPPs. The average costs of the above five cases are shown in Table.~\ref{cost}.
%\todo{Don't use case 1, case 2, ... as the table columns. Use better names.}

\begin{table}[h]
	\caption{{Total cost of different RTED models}}
	\label{cost}
	\begin{center}
		\begin{tabular}{|p{2.6cm}<{\centering}|c|c|c|c|c|}
			\hline
			{Cost/\$} & Fuel & Reserve & LS & REC & Total\\ \hline
			Proposed distribution- based model & \rule{0pt}{0.3cm} 36584 &4590&308&830& 42312\\\hline
			Model in \cite{VersatileMixture} &35877  & 4589& 702& 5705& 46873 \\\hline
			Proposed scenario- based model & \rule{0pt}{0.3cm} 35010 & 2986& 4514& 5298& 47808\\\hline
			Model in \cite{copula_Zhang} & 35032 & 2667& 7018& 5361& 50078\\ \hline
			Model in \cite{sce_generation_Ma} &35002& 2710 & 7741& 5425& 50878  \\ \hline

		\end{tabular}
	\end{center}
\end{table}

\begin{figure}[ht]
	\begin{center}
		\includegraphics[trim = 35 230 35 190, clip, width=1.0\columnwidth]{scewithdiffno.eps}\\
		\caption{{Original scenarios and scenarios after reduction.}}\label{reducedscenarios}
	\end{center}
\end{figure}

{As shown in Table.~\ref{cost}, compared with the proposed distribution-based model, model in \cite{VersatileMixture} has larger LS and REC penalty for the reason that it has not considered the transmission congestion caused by renewable power uncertainties. Compared with the proposed scenario-based model, model in \cite{copula_Zhang} has larger cost for the reason that it could not capture the renewable power variability. Compared with the proposed scenario-based model, model in \cite{sce_generation_Ma} has larger cost for the reason that it has not considered the correlations between different RPPs. Overall, the scenario-based RTED method has much larger LS and REC penalty for the reason that it underestimate the uncertainties of renewable power compared with distribution-based RTED method, as shown in Fig.~\ref{reducedscenarios}. The slim blue lines in the upper left figure are the original renewable scenarios by the proposed scenario generation method and the figure also shows the 10 scenarios after reduction. Renewable power scenarios generated by \cite{copula_Zhang} are also reduced to 10, as shown in the upper right figure.}



{To further analyze the scenario-based RTED, the original renewable scenarios by the proposed scenario generation method are reduced to 50, 500 and 2000 scenarios and embedded in the RTED model. As shown in Table.~\ref{cost2}, with more scenarios embedded in the ED model, system cost decreases. The 50 and 500 scenarios after reduction are shown in the lower left figure and lower right figures, respectively. It can be seen that with 500 scenarios, scenario-based RTED has similar performance with the distribution-based model for the reason that renewable power uncertainties could be well represented. When 2000 scenarios embedded in the RTED model, scenario-based ED shows a better performance compared with of the distrbution-based one since it has a more flexible manner to model the risk of renewable power such as REC caused by certain transmission line congestion.}




%To analyze its discrete feature, the sum renewable power and renewable power in 78'th line at time {\it t}=1 are shown in Fig.~\ref{reducedscenarios}. We can see that although scenario-based ED can consider the cost of LS and REC caused by system reserve deficiency and transmission congestion in theory, the discrete feature of scenario-based ED reduce the effect. As shown in Fig.~\ref{reducedscenarios}, the renewable uncertainties are underestimate compared with distribution-based ED. The reason is that in order to represent the renewable power in scenario reduction method, extreme scenarios are abandoned, especially when the number of RPPs is large in power system. The economy caused by the discrete feature of scenario-based ED is discussed in Section~\ref{sec:econ}.

%\todo{Is lower cost the better in the table? If not, what distinguishes between scenario based and case 2?}


\begin{table}[h]
	\caption{{Total cost of proposed scenario-based model based on different numbers of scenarios after reduction}}
	\label{cost2}
	\begin{center}
		\begin{tabular}{|p{2.5cm}<{\centering}|c|c|c|c|c|}
			\hline
			{Cost/\$} & Fuel & Reserve & LS & REC & Total\\ \hline
			50 scenarios& 35083 & 3124& 3464& 3954& 45625\\\hline
			500 scenarios& 36572  & 4584& 319& 854& 42329\\\hline
			2000 scenarios& 36591 &4528&285&785& 42189\\\hline
		\end{tabular}
	\end{center}
\end{table}

%To show the relationship of system reserve and the potential risk of REC and LS, different penalty coefficients of LS and REC are set to change the potential risk of REC and LS. As shown in Table.~\ref{aci}, the confidence level of enough downward reserve increases if the penalty coefficients of REC increase. This means that when the potential risk of REC increase, more downward reserves are employed for the overall economy. In addition, the confidence level of enough downward reserve of different time interval in Table.~\ref{aci} varies due to the different uncertainties. Compared with other classical stochastic ED methods that use a predefined confidence level, the proposed can seek for the optimal confidence level and obtain the overall economy to cope with the uncertainties and correlations of RPPs.




\vspace{-1em}

\subsection {RTED With Different Penalty Coefficients}

To show the relationships of system reserve and the potential risk of REC and LS, different penalty coefficients of LS and REC are set to change the potential risk of REC and LS. As shown in Table.~\ref{aci}, the confidence level of enough downward reserve increases if the penalty coefficients of REC increase. This means that when the potential risk of REC increase, more downward reserves are employed for the overall economy. In addition, the confidence level of enough downward reserve of different time interval in Table.~\ref{aci} varies due to the different uncertainties. Compared with other classical stochastic ED methods that use a predefined confidence level, the proposed can seek for the optimal confidence level.

\begin{table}[h]
	\caption{Confidence level of enough downward reserve under different REC penalty coefficients}
	\label{aci}
	\begin{center}
		\begin{tabular}{|c|c|c|c|c|c|}
			\hline
			{} & \multicolumn{5}{c|}{Penalty coefficients of REC (\$/MW$\cdot$h)}\\
			\hline
			{{\it time}/min} & {40} & {60} & {80} & {120} & {200}\\
			\hline
			05 & 74.84\% & 80.05\% & 81.98\% & 86.74\% & 93.05\% \\
			10 & 75.04\% & 80.26\% & 82.15\% & 87.08\% & 92.62\% \\
			15 & 74.73\% & 79.93\% & 81.82\% & 87.35\% & 92.16\% \\
			20 & 74.98\% & 81.29\% & 83.25\% & 86.88\% & 92.61\% \\
			25 & 74.99\% & 79.91\% & 82.64\% & 86.89\% & 93.11\% \\
			30 & 75.32\% & 81.92\% & 84.68\% & 88.88\% & 91.74\% \\
			35 & 74.98\% & 81.37\% & 84.12\% & 87.31\% & 92.06\% \\
			40 & 75.03\% & 81.80\% & 83.99\% & 88.67\% & 93.18\% \\
			45 & 75.14\% & 81.89\% & 84.08\% & 87.93\% & 93.23\% \\
			50 & 74.97\% & 81.75\% & 83.95\% & 88.63\% & 93.72\% \\
			55 & 74.71\% & 81.17\% & 83.98\% & 87.71\% & 92.20\% \\
			60 & 74.98\% & 80.86\% & 83.70\% & 88.69\% & 93.10\% \\
			\hline
		\end{tabular}
	\end{center}
\end{table}




\section{Conclusion}

This paper considers the uncertainties and correlations of multiple RPPs in real-time economic dispatch problems. We propose two methods, distribution-based and scenario-based dispatch models that take into account of system reserve and transmission congestion. We propose a scenario generation method that greatly reduces the required computational complexity and can accurately represent renewable power uncertainties, spatial correlation and variability. Results show that although the scenario-based RTED method has a better consideration in the effect of uncertainties and correlations on the system in theory, the discrete feature of scenarios after reduction greatly reduces the effect. Compared with other RTED models, the proposed methods show better economy by capturing renewable power uncertainties, spatial correlation and variability.
% * <immocy@163.com> 2017-02-03T22:23:44.901Z:
%
% ^.


% if have a single appendix:
%\appendix[Proof of the Zonklar Equations]
% or
%\appendix  % for no appendix heading
% do not use \section anymore after \appendix, only \section*
% is possibly needed

% use appendices with more than one appendix
% then use \section to start each appendix
% you must declare a \section before using any
% \subsection or using \label (\appendices by itself
% starts a section numbered zero.)
%

% ============================================
%\appendices
%\section{Proof of the First Zonklar Equation}
%Appendix one text goes here %\cite{Roberg2010}.

% you can choose not to have a title for an appendix
% if you want by leaving the argument blank
%\section{}
%Appendix two text goes here.


% use section* for acknowledgement
%\section*{Acknowledgment}


%The authors would like to thank D. Root for the loan of the SWAP. The SWAP that can ONLY be usefull in Boulder...


% Can use something like this to put references on a page
% by themselves when using endfloat and the captionsoff option.
% \ifCLASSOPTIONcaptionsoff
%  \newpage
% \fi

% trigger a \newpage just before the given reference
% number - used to balance the columns on the last page
% adjust value as needed - may need to be readjusted if
% the document is modified later
%\IEEEtriggeratref{8}
% The "triggered" command can be changed if desired:
%\IEEEtriggercmd{\enlargethispage{-5in}}

% ====== REFERENCE SECTION

%\begin{thebibliography}{1}

% IEEEabrv,

\bibliographystyle{IEEEtran}
\bibliography{Bibliography}
%\end{thebibliography}
% biography section
%
% If you have an EPS/PDF photo (graphicx package needed) extra braces are
% needed around the contents of the optional argument to biography to prevent
% the LaTeX parser from getting confused when it sees the complicated
% \includegraphics command within an optional argument. (You could create
% your own custom macro containing the \includegraphics command to make things
% simpler here.)
%\begin{biography}[{\includegraphics[width=1in,height=1.25in,clip,keepaspectratio]{mshell}}]{Michael Shell}
% or if you just want to reserve a space for a photo:

% ==== SWITCH OFF the BIO for submission
% ==== SWITCH OFF the BIO for submission


%% if you will not have a photo at all:
%\begin{IEEEbiographynophoto}{Ignacio Ramos}
%(S'12) received the B.S. degree in electrical engineering from the University of Illinois at Chicago in 2009, and is currently working toward the Ph.D. degree at the University of Colorado at Boulder. From 2009 to 2011, he was with the Power and Electronic Systems Department at Raytheon IDS, Sudbury, MA. His research interests include high-efficiency microwave power amplifiers, microwave DC/DC converters, radar systems, and wireless power transmission.
%\end{IEEEbiographynophoto}

%% insert where needed to balance the two columns on the last page with
%% biographies
%%\newpage

%\begin{IEEEbiographynophoto}{Jane Doe}
%Biography text here.
%\end{IEEEbiographynophoto}
% ==== SWITCH OFF the BIO for submission
% ==== SWITCH OFF the BIO for submission



% You can push biographies down or up by placing
% a \vfill before or after them. The appropriate
% use of \vfill depends on what kind of text is
% on the last page and whether or not the columns
% are being equalized.

% \vfill

% Can be used to pull up biographies so that the bottom of the last one
% is flush with the other column.
%\enlargethispage{-5in}



% that's all folks
\end{document}
