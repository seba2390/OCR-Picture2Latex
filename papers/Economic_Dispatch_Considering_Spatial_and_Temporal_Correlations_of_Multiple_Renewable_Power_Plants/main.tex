%
% IEEE Transactions on Microwave Theory and Techniques example
% Tibault Reveyrand - http://www.microwave.fr
%
% http://www.microwave.fr/LaTeX.html
% ---------------------------------------

% ================================================
% Please HIGHLIGHT the new inputs such like this :
% Text :
%  \hl{comment}
% Aligned Eq.
% \begin{shaded}
% \end{shaded}
% ================================================

\documentclass[journal]{IEEEtran}

%\usepackage[retainorgcmds]{IEEEtrantools}
%\usepackage{bibentry}
% \usepackage{xcolor,soul,framed} %,caption
% \usepackage{stfloats}

% \colorlet{shadecolor}{yellow}
% \usepackage{color,soul}
\usepackage{graphicx}
\usepackage{color}
\graphicspath{{../pdf/}{../jpeg/}}
\DeclareGraphicsExtensions{.pdf,.jpeg,.png}

\usepackage[cmex10]{amsmath}
\usepackage{amssymb}
%Mathabx do not work on ScribTex => Removed
%\usepackage{mathabx}
\usepackage{array}
\usepackage{mdwmath}
%\usepackage{mdwtab}
%\usepackage{eqparbox} notice that I have not find the package
\usepackage{url}
\usepackage{float}
\usepackage{subfig}
\usepackage{multirow}

\newcommand{\todo}[1]{\textcolor{red}{#1}}
\newcommand{\E}{\mathbb{E}}

\hyphenation{op-tical net-works semi-conduc-tor}

%\bstctlcite{IEEE:BSTcontrol}

%=== TITLE & AUTHORS ====================================================================
\begin{document}
\bstctlcite{IEEEexample:BSTcontrol}
    % \title{Economic Dispatch Considering Uncertainties and Correlations of Multiple Renewable Power Variables Integration}
    \title{Economic Dispatch Considering Spatial and Temporal Correlations of Multiple Renewable Power Plants}
    % \title{Economic Dispatch Considering Multiple Correlated Renewable Power Plants}
  \author{Chenghui~Tang\IEEEauthorrefmark{1},~\IEEEmembership{Student Member,~IEEE,}
      Yishen~Wang\IEEEauthorrefmark{2},~\IEEEmembership{Student Member,~IEEE,} \\
      Jian Xu\IEEEauthorrefmark{1},~\IEEEmembership{Member,~IEEE,}
      Yuanzhang~Sun\IEEEauthorrefmark{1},~\IEEEmembership{Senior Member,~IEEE}
      and~Baosen~Zhang\IEEEauthorrefmark{2},~\IEEEmembership{Member,~IEEE,}\\% <-this % stops a space
%\IEEEauthorrefmark{1}: Wuhan University, Wuhan, Hubei, China \IEEEauthorrefmark{2}: University of Washington, Seattle, WA, USA

\thanks{This work was supported in part by the National Key R\&D Program of China (2016YFB0900105), in part by the National Natural Science Foundation of China (51477122)}
\thanks{
	C. Tang, J. Xu and Y. Sun are with the School of Electrical Engineering, Wuhan University, Wuhan, 430072 China.}
\thanks{
	Y. Wang and B. Zhang are with the Department of Electrical Engineering, University of Washington, Seattle, WA, 98195 USA.}}
% The paper headers
% \markboth{IEEE TRANSACTIONS ON MICROWAVE THEORY AND TECHNIQUES, VOL.~60, NO.~12, DECEMBER~2012
% }{Roberg \MakeLowercase{\textit\title{•}{et al.}}:
% Economic Dispatch Considering the Uncertainties and Correlations of Multiple Renewable Energy Variables Integration}

% ====================================================================
\maketitle

% === ABSTRACT ====================================================================
% =================================================================================



% For peer review papers, you can put extra information on the cover
% page as needed:
% \ifCLASSOPTIONpeerreview
% \begin{center} \bfseries EDICS Category: 3-BBND \end{center}
% \fi
%
% For peerreview papers, this IEEEtran command inserts a page break and
% creates the second title. It will be ignored for other modes.
% \IEEEpeerreviewmaketitle

\begin{abstract}
	%\boldmath
	The correlations of multiple renewable power plants (RPPs) should be fully considered in the power system with very high penetration renewable power integration. This paper models the uncertainties, spatial correlation of multiple RPPs based on Copula theory and actual probability historical histograms by one-dimension distributions for economic dispatch (ED) problem. An efficient dynamic renewable power scenario generation method based on Gibbs sampling is proposed to generate renewable power scenarios considering the uncertainties, spatial correlation and variability (temporal correlation) of multiple RPPs, in which the sampling space complexity do not increase with the number of RPPs. Distribution-based and scenario-based methods are proposed and compared to solve the real-time ED problem with multiple RPPs. Results show that the proposed dynamic scenario generation method is much more consist with the actual renewable power. The proposed ED methods show better understanding for the uncertainties, spatial and temporal correlations of renewable power and more economical compared with the traditional ones.
\end{abstract}

% === KEYWORDS ====================================================================
% =================================================================================
\begin{IEEEkeywords}
	Uncertainty, spatial correlation, variability, renewable power, scenario generation, economic dispatch
\end{IEEEkeywords}

% For peer review papers, you can put extra information on the cover
% page as needed:
% \ifCLASSOPTIONpeerreview
% \begin{center} \bfseries EDICS Category: 3-BBND \end{center}
% \fi
%
% For peerreview papers, this IEEEtran command inserts a page break and
% creates the second title. It will be ignored for other modes.
\IEEEpeerreviewmaketitle




\section{Introduction}
Reinforcement learning has achieved great success in areas such as Game-playing \citep{silver2018general,vinyals2019grandmaster}, robotics \cite{kober2013reinforcement}, large language models \citep{ouyang2022training}, etc.
However, due to safety concerns or physical limitations, in some real-world reinforcement learning problems, we must consider additional constraints that may influence the optimal policy and the learning process \citep{garcia2015comprehensive}.
% For example, a robotic arm must not take actions that may cause harm to itself or the environments.
A standard framework to handle such cases is the constrained Markov Decision Process (CMDP) \citep{altman1999constrained}.
Within the CMDP framework, the agent has to maximize
the expected cumulative reward while
obeying a finite number of constraints, which are usually in the form of expected cumulative cost criteria.

However, we are sometimes concerned with the problem with a continuum of constraints.
For example,
the constraints we meet might be time-evolving or subject to uncertain parameters, which
cannot be formulated as an ordinary CMDP
(see Examples \ref{Example_Time_Evolving} and  \ref{Example_Uncertain}).
In this paper we would study a generalized CMDP  
to address the above problem.  Because the constraints are not only infinite-number but also lie
in a continuous set,
the generalization is not trivial. Fortunately, we find that we can borrow the idea behind semi-infinite programming (SIP) \citep{remez1934determination, hettich1993semi} to deal with the semi-infinite constraints.
Accordingly, we propose \emph{semi-infinitely constrained Markov decision processes} (SICMDPs)
as a novel complement to the ordinary CMDP framework.
%More specifically,  an SICMDP model %, we consider 
%contains a continuum of constraints whereas an ordinary CMDP contains a finite number of constraints. 

%This generalization is natural but not trivial. However, we can brows the idea  
%The idea is quite natural and can be backtracked
%to the practice of extending linear programming to linear semi-infinite programming (LSIP) %\cite{remez1934determination, GobernaLSIO1998}.
%In addition, 
%As a complementary approach to the ordinary CMDP framework, 
%SICMDP can be used to model these problems  which cannot be described by a finite number of constraints
%that are not covered by .
%For example,
%the restrictions we consider can be time-evolving or subject to uncertain parameters
%, thus
%cannot be described by a finite number of constraints but a continuum of constraints 
%(see Examples \ref{Example_Time_Evolving} and  \ref{Example_Uncertain}).

We also present two reinforcement learning algorithms to solve SICMDPs called SI-CRL and SI-CPO, respectively.
SI-CRL is a model-based reinforcement learning algorithm designed for tabular cases, and SI-CPO is a policy optimization algorithm for non-tabular cases.
% and analyze its performance both theoretically and empirically.
The main challenge is that we need to deal with a continuum of constraints, thus reinforcement learning algorithms for ordinary CMDPs do not work anymore.
In SI-CRL, we tackle this difficulty by first transforming the reinforcement learning problem to an equivalent LSIP problem, which can then be solved using methods in the LSIP literature like the dual exchange methods \citep{Hu1990,reemtsen1998numerical}.
In SI-CPO, we resort to the idea of cooperative stochastic approximation developed in \cite{lan2020algorithms, wei2020comirror}.
As far as we know, we are the first to introduce tools from semi-infinitely programming (SIP) into the reinforcement learning community for solving constrained reinforcement learning problems.

% To the best of our knowledge, we are the first to apply tools from semi-infinitely programming (SIP) to solve reinforcement learning problems.
Furthermore, we give theoretical analysis for both SI-CRL and SI-CPO.
We decompose the error of SI-CRL into two parts: the statistical error from approximating the true SICMDP with an offline dataset and the optimization error due to the fact that the solution of the LSIP problem obtained by the dual exchange method is inexact.
On the optimization side, we show that the iteration complexity of SI-CRL is $O\left(\left\{\mathrm{diam}(Y)L\sqrt{|\gS|^2|\gA|m}/\left[(1-\gamma)\epsilon\right]\right\}^m\right)$.
On the statistical side, we show that the sample complexity of SI-CRL is $\widetilde O\left(\frac{|S|^2|A|^2}{\epsilon^2(1-\gamma)^3}\right)$ if the offline dataset is generated by a generative model, and $\widetilde O\left(\frac{|S||A|}{\nu_{\min} \epsilon^2(1-\gamma)^3}\right)$ if the dataset is generated by a probability measure $\nu$ as considered in \cite{chen2019information}.
Here $\widetilde O$ means that all logarithm terms are discarded.
For SI-CPO, things become a little more complicated because other than the statistical error and the optimization error, we also need to consider the function approximation error, which comes from imperfect policy parametrizations.
It is shown if the function approximation error can be controlled to $O(\epsilon)$ order, the iteration complexity of SI-CPO is $\widetilde{O}\left(\frac{1}{\epsilon^2(1-\gamma)^6}\right)$ and the sample complexity of SI-CPO is $\widetilde{O}(\frac{1}{\epsilon^4(1-\gamma)^{10}})$.
Here our iteration complexity bound is equivalent to a typical $\widetilde O(1/\sqrt{T})$ global convergence rate.

We perform a set of numerical experiments to illustrate the SICMDP model and validate our proposed algorithms.
Specifically, we examine two numerical examples, namely the discharge of sewage and ship route planning.
Through the discharge of sewage example, we show the advantage of the SICMDP framework over the CMDP baseline obtained by naive discretization in modeling realistic sequential decision-making problems.
Moreover, we demonstrate the effectiveness of the SI-CRL and SI-CPO algorithms in such tabular environments. 
In the ship route planning example, we illustrate the benefits of the SICMDP framework and the ability of the SI-CPO algorithm to address complex continuous control tasks involving continuous state spaces with modern deep reinforcement learning techniques.

% In summary, our contributions are listed as follows.
% First, we present the SICMDP model, which can be viewed as a generalization of the ordinary CMDP model.
% Second, we propose an algorithm to perform reinforcement learning for SICMDPs, which is called SI-CRL, and we believe that we are the first to apply tools from SIP
% to solve reinforcement learning problems.
% Third, we give a theoretical analysis of SI-CRL and identify both its sample complexity and iteration complexity.
% In addition, we perform numerical experiments to illustrate the SICMDP model and validate the SI-CRL algorithm.
% \{This paragraph can be removed!!! \}







\section {Modeling the Uncertainties and Spatial Correlation of Multiple RPPs in Power System}
\section{The \MakeLowercase{i}W\MakeLowercase{inr}NFL model}
\label{sec:model}

In this section we are going to present the data we used to develop our in-game probability model as well as the design details of {\method}. 

{\bf Data: }In order to perform our analysis we utilize a dataset collected from NFL's Game Center for all the regular season games between the seasons 2009 and 2016. 
We access the data using the Python {\tt nflgame} API \cite{nflgame}. 
The dataset includes detailed play-by-play information for every game that took place during these seasons. 
This information is used to obtain the state of the game that will drive the design of {\method}. 
In total, we collected information for 2,048 regular season games and a total of 338,294 snaps/plays. 

{\bf Model: }
{\method} is based on a logistic regression model that calculates the probability of the home team winning given the current status of the game as: 

\begin{equation}
\Pr(H=1| \mathbf{x})= \frac{\exp(\mathbf{\weight}^T\cdot\mathbf{x})}{1+\exp(\mathbf{\weight}^T\cdot\mathbf{x})}
\label{eq:reg}
\end{equation}
where $H$ is the dependent random variable of our model representing whether the home team wins or not, $\mathbf{x}$ is the vector with the independent variables, while the coefficient vector $\mathbf{\weight}$ includes the weights for each independent variable and is estimated using the corresponding data.  
For a game of infinite duration a linear model could be a very good approximation.  
However, the boundary effects from the finite duration of a game create several non-linearities \cite{winston2012mathletics}.  
For this reason, we enhance our model - using the same set of features - with a Support Vector Machine classifier with radial kernel for the last three minutes of regulation.  
In order to obtain a probability output from the SVM classifier, we further use Platt's scaling \cite{platt1999probabilistic}: 

\begin{equation}
\Pr(H=1| \mathbf{x})= \frac{1}{1+\exp{(Af(x)+B)}}
\label{eq:platt}
\end{equation}
where $f(x)$ is the uncalibrated value produced by the SVM classifier: 

\begin{equation}
f(x) = \sum_{i} (\alpha_i y_i k(\mathbf{x}_i\cdot\mathbf{x}))+ b
\label{eq:svm}
\end{equation}
where $k(\mathbf{x},\mathbf{x}')$ is the kernel used for the SVM.   
Figure \ref{fig:iwinrNFL} depicts the simple flow chart of {\method}. 


\begin{figure}[t]
\begin{center}
\includegraphics[scale=0.35]{plots/iwinrNFL.pdf}%\vspacecap
 \caption{{\method} includes a linear and a non-linear component.}
 \label{fig:iwinrNFL}
\end{center}
\end{figure}

In order to describe the status of the game we use the following variables:

\begin{enumerate}
\item {\bf Ball Possession Team:} This binary feature captures whether the home or the visiting team has the ball possession
\item {\bf Score Differential:} This feature captures the current score differential (home - visiting)
\item {\bf Timeouts Remaining:} This feature is represented by two independent variables - one for the home and one for the away team - and they capture the number of timeouts remaining for each of the teams
%\item {\bf Quarter:} This feature captures the current quarter of the game
%\item {\bf Time Remaining:} This feature captures the time (in seconds) remaining for the current quarter to end
\item {\bf Time Elapsed: } This feature captures the time elapsed since the beginning of the game
\item {\bf Down:} This feature represents the down of the team in possession
\item {\bf Field Position:} This feature captures the distance covered by the team in possession from their own yard line
\item {\bf Yards-to-go:} This variables represents the number of yards needed for a first down
\item {\bf Ball Possession Time: } This variable captures the time that the offensive unit of the home team is on the field 
\item {\bf Ranking Differential: } This variable represents the difference of the win percentage for the two team (home - visiting)
\end{enumerate}

The last independent variable is representative of the power ranking difference between the two teams. 
Most of the existing models that include such a variable are using the Vegas line spread for each game.  
We choose not to do so for the following reason.  
The objective of the Vegas line is not to predict game outcomes but rather distribute money across the different bets.  
Exactly because of this objective the line is changing during the week before the game.  
While this line can change due to new information for the competing teams (e.g., injury updates), the line is mainly changing when a particular team has accumulated the majority of the bets. 
In this case it will also be hard to choose which line to use (e.g., the opening, the closing or some average of them).  
Therefore, we choose to use the win percentage differential of the two teams as an indicator of their strength (even though this has its own issues given the uneven schedule in NFL).  
However, note that if one would like to use the point spread as a variable this can be easily incorporated in the model. 
Table \ref{tab:iwinrnfl} presents the coefficients of the logistic regression model of {\method} with standardized independent variables for better comparisons. 


\begin{table}[ht]
\begin{center}
\def\sym#1{\ifmmode^{#1}\else\(^{#1}\)\fi}
\begin{tabular}{l*{1}{c}}
\toprule
                    &\multicolumn{1}{c}{(1)}\\
                    &\multicolumn{1}{c}{Winner}\\
\midrule
Possession Team (H)         &      0.41\sym{***}\\
                    &     (49.19)         \\
\addlinespace
Score Differential           &      3.59\sym{***}\\
                    &    (247.34)         \\
\addlinespace
Home Timeouts           &     0.12\sym{***}\\
                    &      (8.74)         \\
\addlinespace
Away Timeouts           &     -0.11\sym{***}\\
                    &    (-12.47)         \\
\addlinespace
Ball Possession Time  &     -0.05.\\
                    &    (-1.66)         \\
\addlinespace
Time Lapsed       &   -0.05.\\
                    &      (-1.66)         \\
\addlinespace
Down                &   -0.01         \\
                    &      (0.04)         \\
\addlinespace
Field Position            &   0.02\sym{**} \\
                    &      (2.71)         \\
\addlinespace
Yards-to-go                &  -0.01         \\
                    &      (0.23)         \\
\addlinespace
Rating differential         &       0.75\sym{***}\\
                    &     (80.47)         \\
\addlinespace
Intercept            &       0.57\sym{*}\\
                    &    (2.09)         \\
\midrule
Observations        &      338,294         \\
\bottomrule
\multicolumn{2}{l}{\footnotesize \textit{t} statistics in parentheses}\\
\multicolumn{2}{l}{\footnotesize \sym{$_.$} \(p<0.1\), \sym{*} \(p<0.05\), \sym{**} \(p<0.01\), \sym{***} \(p<0.001\)}\\
\end{tabular}
\end{center}
\caption{Standardized logisitic regression coefficients for {\method}.}
\label{tab:iwinrnfl}
\end{table}


As we can see, as one might have expected the current scoring differential exhibits the strongest correlation with the in-game win probability.  
The only factors that do not appear to be statistically significant predictors of the dependent variable are the down and the yards-to-go. 
Even though the corresponding coefficients are negative as one might have expected (e.g., being at an earlier down gives you more chances to advance the ball), they are not significant in estimating the win probability. 
On the contrary, all else being equal timeouts appear to be quiet important since they can help a team stop the clock, while teams with better win percentage appear to have an advantage as well, since this can be a sign of a better team. 
In the following section we provide a detailed evaluation of {\method}.

\section{Scenario Generation}
%% !TEX root=econ_dispatch.tex
In this section, we first propose a reliable static renewable power scenario generation method in each time interval $1,\dots,T$. Then we present an efficient dynamic renewable power scenario generation method for the entire time horizon.

\subsection {Static Scenario Generation}

By the joint distribution of multiple RPPs in \eqref{cjdistribution}, scenarios can be generated to represent the uncertainties and spatial correlation of all RPPs in the system. However, with the increase of the number of RPPs, classical random sampling methods such as inverse transform sampling and Latin hypercube sampling \cite{L_sampling} become hard to be employed due to matrix size and computational limitations. Other classical sampling methods such as rejection sampling tend to have very large rejection rate for a high number of dimensions.

To this end, a reliable static renewable power scenario generation method based on Gibbs sampling \cite{Gibbs} is proposed to sample for the conditional joint distribution function of actual available power of RPPs in \eqref{cjdistribution}. Compared with directly sampling by the conditional joint distribution \cite{copula_Zhang}, Gibbs sampling converts the sampling process of joint distribution in \eqref{cjdistribution} to $J+K$ sampling processes of conditional distribution in \eqref{ccdistribution}. Namely, let $U$ be a random variable generated uniformly within $[0,1]$, then each RPP can be sampled via the inverse transform:
\begin{equation} \label{inversesampling}
w_{a,j}=F_{a,j}^{-1}(U),\quad s_{a,k}=F_{a,k}^{-1}(U)
\end{equation}
where $F_{a,j}^{-1}$ and $F_{a,k}^{-1}$ is the inverse function of $F_{a,j}$ and $F_{a,k}$, respectively.

Gibbs sampling needs a burn-in process \cite{burn_in} before it converges to the true distribution in \eqref{cjdistribution}. So we throw out $N_{b}$ (e.g. 1000) samples in the beginning the process. The detailed procedure of static scenarios generation is:
\begin{enumerate}%[noitemsep,nolistsep]
	\item Setting the number of renewable power scenarios: $N_{sc}$ (e.g. 5000), the total number of samples is $N_{sc}+N_{b}$.
	\item Setting the initial sampling values to be the forecasted power for each RPP.
	% $w_{a,{1}}^{i}$,...,$w_{a,j}^{i}$,..., $w_{a,J}^{i}$, $s_{a,{\it 1}}^{i}$,...,$s_{a,k}^{i}$,...,$s_{a,K}^{i}$, {\it i}=0...$N_{sc}+N_{b}$, ({\it i}=0 at this step). To  speed up the burn-in process, the forecast power of each RPP (i.e. $F_{re}$) are regarded as the initial sampling value.
	\item Employing inverse transform sampling in \eqref{inversesampling} in a round robin fashion for each scenario generation step (indexed by $i$):

\begin{itemize}
	\item $f(w_{a,{1}}^{i}|w_{a,2}^{i}...w_{a,J}^{i},s_{a,{1}}^{i}...s_{a,K}^{i},\mathbf{f})$
	\item $f(w_{a,{\it j}}^{i}|w_{a,{1}}^{i+1}...w_{a,{{\it j}-1}}^{i+1},w_{a,{{\it j}+1}}^{i}...w_{a,J}^{i},s_{a,{1}}^{i}...s_{a,K}^{i},\mathbf{f})$
	\item $...$
	\item $f(s_{a,{\it k}}^{i}|w_{a,{1}}^{i+1}...w_{a,J}^{i+1},s_{a,{1}}^{i+1}...s_{a,{{\it k}-1}}^{i+1},s_{a,{{\it k}+1}}^{i}...s_{a,K}^{i},\mathbf{f})$
	\item $f(s_{a,{\it K}}^{i}|w_{a,{1}}^{i+1}...w_{a,J}^{i+1},s_{a,{1}}^{i+1}...s_{a,{{\it K}-1}}^{i+1},\mathbf{f})$
\end{itemize}

	\item Repeating 3 from {\it i}=1...$N_{sc}+N_{b}$. Disregard the first $N_{b}$ scenarios and we get $N_{sc}$ renewable power scenarios.

\end{enumerate}

{An important feature of the proposed static scenario generation method is that with the increase of the number of RPPs, the computational space complexity remains same and the computational time complexity increases linearly, effectively mitigating the curse of dimensionality.}

\subsection {Dynamic Scenario Generation}
%\todo{Why is this dynamic? Also, does variability just mean correlation?}
{A dynamic scenario is a scenario that considers the variability (i.e., temporal correlation) of the output of a RPP.} The method presented in the last section can generate renewable power scenarios of conditional joint distribution (c.f. \eqref{cjdistribution}) which captures the marginal uncertainties and spatial correlation. In this section we extend it to capture the temporal correlation among the time points in a scenario, which is also of vital importance in power system operations~\cite{sce_generation_Ma,PCA,sce_generation_Pinson}.
 % which represent the uncertainties and correlations in each time interval \todo{(i.e., spatial correlation)}. However, for renewable power scenarios, variability is as same importance as uncertainties \cite{sce_generation_Ma}\cite{PCA}\cite{sce_generation_Pinson}.

To capture the variability, some new variables are introduced. Take a WPP for instance, a new random variable $Z_{a,j}^{t}$ is introduced which follows
the standard Gaussian distribution with zero mean and unit standard deviation. Since the value of CDF of $Z_{a,j}^{t}$ is uniformly distributed over [0,1], the uniform distribution $U$ in \eqref{inversesampling} can be replaced by a CDF $\Phi(Z_{a,j}^{t})$.  Given the realization of random variable $Z_{a,j}^{t}$, $w_{a,j}^{t}$ can be sampled as follows:



\begin{equation} \label{transform}
\begin{aligned}
w_{a,j}^t=F_{a,j}^{-1}(\Phi(Z_{a,j}^{t}))
\end{aligned}
\end{equation}

To consider the variability of each RPP, it is assumed that the joint distribution of $Z_{a,j}^{t}$ follows a multivariate Gaussian distribution $Z_{a,j}^{t} \sim N(\mu_{j},\Sigma_{j})$. The expectation of $\mu_{j}$ is a vector of zeros and the covariance matrix $\Sigma_{j}$ satisfies


\begin{equation} \label{matrix}
\Sigma_j=\left[
\begin{matrix}
\sigma_{1,1}^{j}&\sigma_{1,2}^{j}&\dots&\sigma_{1,{\it T}}^{j}&\\
\sigma_{2,1}^{j}&\sigma_{2,2}^{j}&\dots&\sigma_{2,{\it T}}^{j}&\\
\vdots&\vdots&\ddots&\vdots&\\
\sigma_{{\it T},1}^{j}&\sigma_{{\it T},2}^{j}&\dots&\sigma_{{\it T},{\it T}}^{j}&\\
\end{matrix}
\right]
\end{equation}

\noindent where $\sigma_{m,n}^{j}=cov(Z_{a,j}^{m},Z_{a,j}^{n})$, {\it m}, {\it n}=1,2...{\it T}, $\sigma_{{\it m}, {\it n}}^{j}$ is the covariance of $Z_{a,j}^{m}$ and $Z_{a,j}^{n}$.

The covariance structure of $\Sigma_j$ can be identified by covariance $\sigma_{m,n}^{j}$. As is done in \cite{sce_generation_Ma}\cite{sce_generation_Pinson}, an exponential covariance function is employed to model $\sigma_{m,n}^{j}$ in \eqref{matrix},

\begin{equation} \label{exponential}
\begin{aligned}
\sigma_{m,n}^{j}=\rm exp(-\frac{|{\it m}-{\it n}|}{\epsilon_{\it j}}) \quad 0 \le {\it m},  {\it n} \le {\it T}
\end{aligned}
\end{equation}

\noindent where $\epsilon_{\it j}$ is the range parameter controlling the strength of the
correlation of random variables $Z_{a,j}^{t}$ among the set of lead-time. Similar to \cite{sce_generation_Ma}, $\epsilon_{\it j}$ can be determined by comparing the distribution of renewable power variability of the generated scenarios by the indicator in \cite{sce_generation_Ma}. Here, assuming that the  range parameter $\epsilon_{\it j}$ of each RPP have been obtained, the flowchart of dynamic renewable power scenario generation method is as shown in Fig.~\ref{flowchart}.

\begin{figure}[!htb]
	\begin{center}
		\includegraphics[trim = 10 250 60 200, clip, width=1.0\columnwidth]{flowchart.eps}\\
		\caption{Flowchart of dynamic renewable power scenario generation method}\label{flowchart}
	\end{center}
\end{figure}

Before generating $N_{sc}$ scenarios, small amount of scenarios are generated to obtain the range parameter of each RPP. After all the range parameters in \eqref{matrix} are obtained, we can start the dynamic wind power scenarios generation in Fig.~\ref{flowchart}. At each time interval, they follow the conditional joint distribution in \eqref{cjdistribution} and among the time horizon, the variability is considered.

One thing that need to be noticed is that each static scenario generation process in Fig. 1 does not affect each other after the random data set is determined. Parallel computing can be employed to increase the computation efficiency to meet the real-time requirement.

In scenario-based method, the above generated scenarios should be reduced to certain number of scenarios that deemed as the most probability occur. A scenario reduction method in \cite{YishenWang} is employed in this paper for the reason that it has great efficiency compared with other methods to meet the real-time requirement.


\section{Distribution-Based ED} \label{sec:dist}
%eigenvalue_histogram

Figure \ref{lognormal} shows that the distribution of largest generalized eigenvalue over the samples behaves like a log-normal distribution rather than normal distribution.
\begin{figure*}[h!]
\centering
  \includegraphics[width=0.7\linewidth]{supp/eigenvalue_histogram.png}
\caption{\em The histogram plot of eigenvalue of Gatys(blue) and ACG(green) shows a log-normal distribution over the samples.     }
\label{lognormal}
\vspace{-3mm}
\end{figure*}
\FloatBarrier



%SLP-based algorithm \cite{SLP} is employed to solve the proposed distribution-based RTED model in this paper.

\section{Scenario-Based ED}
%% !TEX root=econ_dispatch.tex
%\todo{Transition better. Explain why scenario based ED is considered. What's the difference with respect to the distribution based ED?}
Different from the distribution-based ED, scenario-based ED incorporate the renewable power uncertainties by a certain number of possible renewable power series (i.e., scenarios). This means that scenario-based ED is essentially a deterministic optimization. This allows a more flexible way to model the risk of renewable power such as REC caused by certain transmission line congestion. However, the performance of scenario-based ED greatly relies on the number of scenarios that are considered in the ED. RTED model based on the scenario of multiple RPPs is proposed in this section. The potential risk of LS and REC caused by system reserve deficiency and transmission congestion are modeled by the scenario-based ED. The penalty cost caused by renewable power uncertainties (REC and LS) in (14) and (16) can be written using scenarios as:
%\todo{What is the drawback of scenario based ED then? Should we always use scenario-based ED? Or sometimes distribution-based ED?}
\begin{equation} \label{SB_1}
\begin{aligned}
\begin{aligned}
&E[f_{R,t}(w_{c,j,t},s_{c,k,t},L_{s,b,t})]  \\
&=\sum_{{\it sc}=1}^{{\it SC}}[p^{sc}(c_{rec}(\sum_{{\it j}=1}^{{\it J}}w_{c,j,t}^{sc}+\sum_{{\it k}=1}^{{\it K}}s_{c,k,t}^{sc})
+c_{ls} \sum_{{\it b}=1}^{{\it Nb}}L_{s,b,t}^{sc})]
\end{aligned}
\end{aligned}
\end{equation}
where $sc$ is the {\it sc}-th scenario for WPPs and PVPPs, $SC$ is the number of renewable power scenarios in RTED model, $w_{c,j,t}^{sc}$ is the amount of wind power curtailment of $j$'th WPP at time {\it t} of $sc$'th scenario; $s_{c,k,t}^{sc}$ is the amount of solar power curtailment of $k$'th PVPP at time {\it t} of $sc$'th scenario; $L_{s,b,t}^{sc}$ is the amount of LS of $b$'th bus at time {\it t} of $sc$'th scenario.

Then the optimization problem is same as in Section~\ref{sec:dist}, except the constraints and objectives are represented with scenarios. In particular, the constraints are:
\begin{equation} \label{SB_2}
\begin{aligned}
0  \leq w_{c,j,t}^{sc} \leq w_{a,j,t}^{sc} \quad \forall j,t,sc \\
0  \leq s_{c,k,t}^{sc} \leq s_{a,k,t}^{sc} \quad \forall k,t,sc
\end{aligned}
\end{equation}

\begin{equation} \label{SB_3}
0  \leq L_{s,b,t}^{sc} \leq L_{b,t} \quad \forall b,t,sc \\
\end{equation}

\vspace{-1em}

\begin{equation} \label{SB_4}
-r_{d,i,t}  \leq r_{a,i,t}^{sc} \leq r_{u,i,t} \quad \forall i,t,sc \\
\end{equation}

\vspace{-1.5em}

\begin{equation} \label{SB_5}
\begin{aligned}
& \sum_{{\it i}=1}^{{\it I}}(p_{i,t}+r_{a,i,t}^{sc})+\sum_{{\it j}=1}^{{\it J}}(w_{a,j,t}^{sc}-w_{c,j,t}^{sc}) \\
& +\sum_{{\it k}=1}^{{\it K}}(s_{a,k,t}^{sc}-s_{c,k,t}^{sc})=L_{t}-\sum_{{\it b}=1}^{{\it Nb}}L_{s,b,t}^{sc} \quad \forall t,sc
\end{aligned}
\end{equation}

\vspace{-1.5em}

\begin{equation} \label{SB_6}
\begin{aligned}
\begin{aligned}
\begin{aligned}
&|\sum_{i=1}^{I}k_{l,i}(p_{i,t}+r_{a,i,t}^{sc})+\sum_{j=1}^{J}k_{l,j}(w_{a,j,t}^{sc}-w_{c,j,t}^{sc}) \\
&+\sum_{k=1}^{K}k_{l,k}(s_{a,k,t}^{sc}-s_{c,k,t}^{sc})-\sum_{{\it b}=1}^{{\it Nb}}k_{l,b}(L_{b,t}-L_{s,b,t}^{sc})|\\ & \le  Pl_l^{max} \quad \forall l,t,sc
\end{aligned}
\end{aligned}
\end{aligned}
\end{equation}
%\begin{equation} \label{SB_7
%\begin{aligned}
%	\sum_{\it sc=1}^{{\it SC}}(p^{sc}(\sum_{\it j=1}^{{\it J}}w_{c,j,t}^{sc}+\sum_{\it k=1}^{{\it K}}s_{c,k,t}^{sc}))\leq RE_{c,set}  \quad %\forall t\\
%	\sum_{\it sc=1}^{{\it SC}}(p^{sc}\sum_{\it b=1}^{{\it Nbus}}L_{s,b,t}^{sc})\leq L_{s,set}\quad \forall t
%\end{aligned}
%\end{equation}}
where
\begin{itemize}

	\item \eqref{SB_2} is the actual amount of REC constraint; $w_{a,j,t}^{sc}$ is actual wind power of $j$'th WPP at time {\it t} of $sc$'th scenario; $s_{a,k,t}^{sc}$ is actual solar power of $k$'th PVPP at time {\it t} of $sc$'th scenario;

	\item \eqref{SB_3} is the actual amount of LS constraint;

	\item \eqref{SB_4} is the actual amount of reserve constraint; $r_{a,i,t}^{sc}$ is actual amount of reserve of $i$'th CPP at time {\it t} of $sc$'th scenario;

	\item \eqref{SB_5} is the supply-demand balance constraint;

	\item \eqref{SB_6} is the transmission capacity constraint;

%	\item \eqref{SB_7} is the compulsive maximal amount of  REC and LS constraint for the conservative aim.

\end{itemize}

% \subsection {Comparison With the Distribution-Based ED Model}

Compared with the proposed distribution-based ED model, the scenario-based ED model can not only model the cost of LS and REC caused by system reserve deficiency but also can model the cost of LS and REC caused by transmission congestion. However, the number of scenarios after reduction is limit due to the computation ability. This would reduce the representation accuracy of renewable power in the above scenario-based RTED model, which would be discussed in Section VI.

When the reserve deficiency or transmission congestion occur, REC and LS have to be employed for the balance of system power. Optimal REC and LS strategies can be obtained by solving the static optimization problem (ob. \eqref{SB_1}, s.t. {\eqref{SB_2}-\eqref{SB_6})} by the deterministic value of CPPs scheduled power, actual reserve, actual power of WPPs and PVPPs (the only scenario in this optimization problem).

% \subsection {The Feature of Distribution-Based ED Method in Power System With Multiple Renewable Energy Variable}

% It is easy to prove that the above distribution-based ED model is convex and SLP-based algorithm can be employed to solved the above model.

% Compared with other modeling method such as scenario-based method that use several dispersed renewable energy curves, the proposed distribution-based ED method has two advantages, The first is that the distribution-based model is continuous and it has a better understanding of the renewable energy uncertainties. The second is that distribution-based ED method has higher computational efficiency.



% \subsection {Scenario-Based ED Model}


\section{Case Study}
\addcontentsline{toc}{section}{Motivating Scenario}
\section{Motivating Scenario}
Heart disease is the first cause of morbidity and mortality in the world, accounting for 28.30\% of total deaths each year in Tunisia alone \cite{Organization2014}. Investment in preventive health care such as the use of IoT monitoring devices and tools may help lower the cost of processing and the development of serious health problems. In fact, integrating clinical decisions with electronic medical records could decrease medical errors, reduce undesirable variations in practice, and improve patient outcomes.

Our case study considers IoT integration with cloud computing. We use a connected bracelet, fog nodes, a private and a public Cloud, and a mobile application, which together form a medical application. This latter provides continuous monitoring of the vital data of a given patient. Regular or routine measurements could help to detect the first symptoms of heart malfunction, and makes it possible to immediately trigger an alert. The vital information collected by the bracelet worn by the patient includes cardiac activity, blood pressure, oxygen levels and, temperature. As mentioned earlier, we consider an IoT application in a hybrid cloud/fog environment. The cloud \cite{zhang2010cloud} is considered as a highly promising approach to deliver services to users, and provide applications with low-cost elastic resources. Given the fact that IoT  suffers limited computational power, storage capacity and bandwidth, cloud computing ease the issues in enabling the access, the storage, and the processing of the large amount of generated data.

Public clouds provide cheap scalable resources. Making it useful for analyzing the patient's vital data which would be costly as it requires extensive computing and storage resources. However, we must take into account that storage of health records on a public environment is a privacy risk. To avoid such security leaks, we could deploy the application on a secure private cloud. But seeing this latter's limited resources, this may degrade the overall performance. To prevent this, the workflow can be partitioned between a private cloud and a public one. Therefore, the confidential medical data will be processed on the private cloud. Other workflow actions can be deployed on the public cloud dealing with anonymized data.

The use of a cloud-based framework poses the problem of delay when sending and receiving data between the objects and geographically far cloud resources thus jeopardizing the patients' well-being given that triggering timely responses is the purpose of this data. To resolve this issue, data gathering can be moved from the cloud domain to that of the fog \cite{BonomiMZA12}. Bringing this action closer to the connected object shortens the transmission time, and reduces the amount of data transferred to the cloud.
The proposed workflow is described as follows:
	\begin{compactitem}
		\item A patient may register via the mobile app by entering his information. This information include personal data and medical history (personal and family medical histories, surgical history, drug prescriptions, and the doctors' notes). 
		\item The patient's medical history is then transmitted to the private cloud. After reception, this latter anonymizes the data by stripping off all that could identify the patient leaving only medical data, which it sends to the public cloud.
		\item The public cloud receives the anonymized data, and proceeds to the classification attaching to each medical file a class.
		\item The patient is equipped with a  measuring bracelet connected to the processing components (Fog nodes). The data sent to the fog domain is a set of vital data recorded over a period of time. 
		\item The fog node collects the data then compares it to its predecessors, searching for any vital signs changes. When the node determines that a change has occurred, it sends the data to the private cloud. 
		\item The private cloud links the gathered data with the patient, transmitting this data and the class ascribed to the patient, to the public cloud.
		\item The public cloud reads the data, analyzes it, and then provides results. When the risk of heart attack is detected, it immediately notifies the patient's app.
	\end{compactitem}






\section{Conclusion}

This paper considers the uncertainties and correlations of multiple RPPs in real-time economic dispatch problems. We propose two methods, distribution-based and scenario-based dispatch models that take into account of system reserve and transmission congestion. We propose a scenario generation method that greatly reduces the required computational complexity and can accurately represent renewable power uncertainties, spatial correlation and variability. Results show that although the scenario-based RTED method has a better consideration in the effect of uncertainties and correlations on the system in theory, the discrete feature of scenarios after reduction greatly reduces the effect. Compared with other RTED models, the proposed methods show better economy by capturing renewable power uncertainties, spatial correlation and variability.
% * <immocy@163.com> 2017-02-03T22:23:44.901Z:
%
% ^.


% if have a single appendix:
%\appendix[Proof of the Zonklar Equations]
% or
%\appendix  % for no appendix heading
% do not use \section anymore after \appendix, only \section*
% is possibly needed

% use appendices with more than one appendix
% then use \section to start each appendix
% you must declare a \section before using any
% \subsection or using \label (\appendices by itself
% starts a section numbered zero.)
%

% ============================================
%\appendices
%\section{Proof of the First Zonklar Equation}
%Appendix one text goes here %\cite{Roberg2010}.

% you can choose not to have a title for an appendix
% if you want by leaving the argument blank
%\section{}
%Appendix two text goes here.


% use section* for acknowledgement
%\section*{Acknowledgment}


%The authors would like to thank D. Root for the loan of the SWAP. The SWAP that can ONLY be usefull in Boulder...


% Can use something like this to put references on a page
% by themselves when using endfloat and the captionsoff option.
% \ifCLASSOPTIONcaptionsoff
%  \newpage
% \fi

% trigger a \newpage just before the given reference
% number - used to balance the columns on the last page
% adjust value as needed - may need to be readjusted if
% the document is modified later
%\IEEEtriggeratref{8}
% The "triggered" command can be changed if desired:
%\IEEEtriggercmd{\enlargethispage{-5in}}

% ====== REFERENCE SECTION

%\begin{thebibliography}{1}

% IEEEabrv,

\bibliographystyle{IEEEtran}
\bibliography{Bibliography}
%\end{thebibliography}
% biography section
%
% If you have an EPS/PDF photo (graphicx package needed) extra braces are
% needed around the contents of the optional argument to biography to prevent
% the LaTeX parser from getting confused when it sees the complicated
% \includegraphics command within an optional argument. (You could create
% your own custom macro containing the \includegraphics command to make things
% simpler here.)
%\begin{biography}[{\includegraphics[width=1in,height=1.25in,clip,keepaspectratio]{mshell}}]{Michael Shell}
% or if you just want to reserve a space for a photo:

% ==== SWITCH OFF the BIO for submission
% ==== SWITCH OFF the BIO for submission


%% if you will not have a photo at all:
%\begin{IEEEbiographynophoto}{Ignacio Ramos}
%(S'12) received the B.S. degree in electrical engineering from the University of Illinois at Chicago in 2009, and is currently working toward the Ph.D. degree at the University of Colorado at Boulder. From 2009 to 2011, he was with the Power and Electronic Systems Department at Raytheon IDS, Sudbury, MA. His research interests include high-efficiency microwave power amplifiers, microwave DC/DC converters, radar systems, and wireless power transmission.
%\end{IEEEbiographynophoto}

%% insert where needed to balance the two columns on the last page with
%% biographies
%%\newpage

%\begin{IEEEbiographynophoto}{Jane Doe}
%Biography text here.
%\end{IEEEbiographynophoto}
% ==== SWITCH OFF the BIO for submission
% ==== SWITCH OFF the BIO for submission



% You can push biographies down or up by placing
% a \vfill before or after them. The appropriate
% use of \vfill depends on what kind of text is
% on the last page and whether or not the columns
% are being equalized.

% \vfill

% Can be used to pull up biographies so that the bottom of the last one
% is flush with the other column.
%\enlargethispage{-5in}



% that's all folks
\end{document}
