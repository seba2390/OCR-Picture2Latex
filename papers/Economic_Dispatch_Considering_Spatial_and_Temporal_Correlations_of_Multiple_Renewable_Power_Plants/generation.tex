%% !TEX root=econ_dispatch.tex
In this section, we first propose a reliable static renewable power scenario generation method in each time interval $1,\dots,T$. Then we present an efficient dynamic renewable power scenario generation method for the entire time horizon.

\subsection {Static Scenario Generation}

By the joint distribution of multiple RPPs in \eqref{cjdistribution}, scenarios can be generated to represent the uncertainties and spatial correlation of all RPPs in the system. However, with the increase of the number of RPPs, classical random sampling methods such as inverse transform sampling and Latin hypercube sampling \cite{L_sampling} become hard to be employed due to matrix size and computational limitations. Other classical sampling methods such as rejection sampling tend to have very large rejection rate for a high number of dimensions.

To this end, a reliable static renewable power scenario generation method based on Gibbs sampling \cite{Gibbs} is proposed to sample for the conditional joint distribution function of actual available power of RPPs in \eqref{cjdistribution}. Compared with directly sampling by the conditional joint distribution \cite{copula_Zhang}, Gibbs sampling converts the sampling process of joint distribution in \eqref{cjdistribution} to $J+K$ sampling processes of conditional distribution in \eqref{ccdistribution}. Namely, let $U$ be a random variable generated uniformly within $[0,1]$, then each RPP can be sampled via the inverse transform:
\begin{equation} \label{inversesampling}
w_{a,j}=F_{a,j}^{-1}(U),\quad s_{a,k}=F_{a,k}^{-1}(U)
\end{equation}
where $F_{a,j}^{-1}$ and $F_{a,k}^{-1}$ is the inverse function of $F_{a,j}$ and $F_{a,k}$, respectively.

Gibbs sampling needs a burn-in process \cite{burn_in} before it converges to the true distribution in \eqref{cjdistribution}. So we throw out $N_{b}$ (e.g. 1000) samples in the beginning the process. The detailed procedure of static scenarios generation is:
\begin{enumerate}%[noitemsep,nolistsep]
	\item Setting the number of renewable power scenarios: $N_{sc}$ (e.g. 5000), the total number of samples is $N_{sc}+N_{b}$.
	\item Setting the initial sampling values to be the forecasted power for each RPP.
	% $w_{a,{1}}^{i}$,...,$w_{a,j}^{i}$,..., $w_{a,J}^{i}$, $s_{a,{\it 1}}^{i}$,...,$s_{a,k}^{i}$,...,$s_{a,K}^{i}$, {\it i}=0...$N_{sc}+N_{b}$, ({\it i}=0 at this step). To  speed up the burn-in process, the forecast power of each RPP (i.e. $F_{re}$) are regarded as the initial sampling value.
	\item Employing inverse transform sampling in \eqref{inversesampling} in a round robin fashion for each scenario generation step (indexed by $i$):

\begin{itemize}
	\item $f(w_{a,{1}}^{i}|w_{a,2}^{i}...w_{a,J}^{i},s_{a,{1}}^{i}...s_{a,K}^{i},\mathbf{f})$
	\item $f(w_{a,{\it j}}^{i}|w_{a,{1}}^{i+1}...w_{a,{{\it j}-1}}^{i+1},w_{a,{{\it j}+1}}^{i}...w_{a,J}^{i},s_{a,{1}}^{i}...s_{a,K}^{i},\mathbf{f})$
	\item $...$
	\item $f(s_{a,{\it k}}^{i}|w_{a,{1}}^{i+1}...w_{a,J}^{i+1},s_{a,{1}}^{i+1}...s_{a,{{\it k}-1}}^{i+1},s_{a,{{\it k}+1}}^{i}...s_{a,K}^{i},\mathbf{f})$
	\item $f(s_{a,{\it K}}^{i}|w_{a,{1}}^{i+1}...w_{a,J}^{i+1},s_{a,{1}}^{i+1}...s_{a,{{\it K}-1}}^{i+1},\mathbf{f})$
\end{itemize}

	\item Repeating 3 from {\it i}=1...$N_{sc}+N_{b}$. Disregard the first $N_{b}$ scenarios and we get $N_{sc}$ renewable power scenarios.

\end{enumerate}

{An important feature of the proposed static scenario generation method is that with the increase of the number of RPPs, the computational space complexity remains same and the computational time complexity increases linearly, effectively mitigating the curse of dimensionality.}

\subsection {Dynamic Scenario Generation}
%\todo{Why is this dynamic? Also, does variability just mean correlation?}
{A dynamic scenario is a scenario that considers the variability (i.e., temporal correlation) of the output of a RPP.} The method presented in the last section can generate renewable power scenarios of conditional joint distribution (c.f. \eqref{cjdistribution}) which captures the marginal uncertainties and spatial correlation. In this section we extend it to capture the temporal correlation among the time points in a scenario, which is also of vital importance in power system operations~\cite{sce_generation_Ma,PCA,sce_generation_Pinson}.
 % which represent the uncertainties and correlations in each time interval \todo{(i.e., spatial correlation)}. However, for renewable power scenarios, variability is as same importance as uncertainties \cite{sce_generation_Ma}\cite{PCA}\cite{sce_generation_Pinson}.

To capture the variability, some new variables are introduced. Take a WPP for instance, a new random variable $Z_{a,j}^{t}$ is introduced which follows
the standard Gaussian distribution with zero mean and unit standard deviation. Since the value of CDF of $Z_{a,j}^{t}$ is uniformly distributed over [0,1], the uniform distribution $U$ in \eqref{inversesampling} can be replaced by a CDF $\Phi(Z_{a,j}^{t})$.  Given the realization of random variable $Z_{a,j}^{t}$, $w_{a,j}^{t}$ can be sampled as follows:



\begin{equation} \label{transform}
\begin{aligned}
w_{a,j}^t=F_{a,j}^{-1}(\Phi(Z_{a,j}^{t}))
\end{aligned}
\end{equation}

To consider the variability of each RPP, it is assumed that the joint distribution of $Z_{a,j}^{t}$ follows a multivariate Gaussian distribution $Z_{a,j}^{t} \sim N(\mu_{j},\Sigma_{j})$. The expectation of $\mu_{j}$ is a vector of zeros and the covariance matrix $\Sigma_{j}$ satisfies


\begin{equation} \label{matrix}
\Sigma_j=\left[
\begin{matrix}
\sigma_{1,1}^{j}&\sigma_{1,2}^{j}&\dots&\sigma_{1,{\it T}}^{j}&\\
\sigma_{2,1}^{j}&\sigma_{2,2}^{j}&\dots&\sigma_{2,{\it T}}^{j}&\\
\vdots&\vdots&\ddots&\vdots&\\
\sigma_{{\it T},1}^{j}&\sigma_{{\it T},2}^{j}&\dots&\sigma_{{\it T},{\it T}}^{j}&\\
\end{matrix}
\right]
\end{equation}

\noindent where $\sigma_{m,n}^{j}=cov(Z_{a,j}^{m},Z_{a,j}^{n})$, {\it m}, {\it n}=1,2...{\it T}, $\sigma_{{\it m}, {\it n}}^{j}$ is the covariance of $Z_{a,j}^{m}$ and $Z_{a,j}^{n}$.

The covariance structure of $\Sigma_j$ can be identified by covariance $\sigma_{m,n}^{j}$. As is done in \cite{sce_generation_Ma}\cite{sce_generation_Pinson}, an exponential covariance function is employed to model $\sigma_{m,n}^{j}$ in \eqref{matrix},

\begin{equation} \label{exponential}
\begin{aligned}
\sigma_{m,n}^{j}=\rm exp(-\frac{|{\it m}-{\it n}|}{\epsilon_{\it j}}) \quad 0 \le {\it m},  {\it n} \le {\it T}
\end{aligned}
\end{equation}

\noindent where $\epsilon_{\it j}$ is the range parameter controlling the strength of the
correlation of random variables $Z_{a,j}^{t}$ among the set of lead-time. Similar to \cite{sce_generation_Ma}, $\epsilon_{\it j}$ can be determined by comparing the distribution of renewable power variability of the generated scenarios by the indicator in \cite{sce_generation_Ma}. Here, assuming that the  range parameter $\epsilon_{\it j}$ of each RPP have been obtained, the flowchart of dynamic renewable power scenario generation method is as shown in Fig.~\ref{flowchart}.

\begin{figure}[!htb]
	\begin{center}
		\includegraphics[trim = 10 250 60 200, clip, width=1.0\columnwidth]{flowchart.eps}\\
		\caption{Flowchart of dynamic renewable power scenario generation method}\label{flowchart}
	\end{center}
\end{figure}

Before generating $N_{sc}$ scenarios, small amount of scenarios are generated to obtain the range parameter of each RPP. After all the range parameters in \eqref{matrix} are obtained, we can start the dynamic wind power scenarios generation in Fig.~\ref{flowchart}. At each time interval, they follow the conditional joint distribution in \eqref{cjdistribution} and among the time horizon, the variability is considered.

One thing that need to be noticed is that each static scenario generation process in Fig. 1 does not affect each other after the random data set is determined. Parallel computing can be employed to increase the computation efficiency to meet the real-time requirement.

In scenario-based method, the above generated scenarios should be reduced to certain number of scenarios that deemed as the most probability occur. A scenario reduction method in \cite{YishenWang} is employed in this paper for the reason that it has great efficiency compared with other methods to meet the real-time requirement.
