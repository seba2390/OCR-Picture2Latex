In this paper, we propose a subspace representation learning (SRL) framework to tackle few-shot image classification tasks.
It exploits a subspace in local CNN feature space to represent an image, and measures the similarity between two images according to a weighted subspace distance (WSD).
When $K$ images are available for each class, we develop two types of template subspaces to aggregate $K$-shot information: the prototypical subspace (PS) and the discriminative subspace (DS).
%The derivation for both types of template subspace are formulated as an optimization problem on Stiefel manifold.
Based on the SRL framework, we extend metric learning based techniques from vector to subspace representation.
While most previous works adopted global vector representation, using subspace representation can effectively preserve the spatial structure, and diversity within an image.
We demonstrate the effectiveness of the SRL framework on three public benchmark datasets: MiniImageNet, TieredImageNet and Caltech-UCSD Birds-200-2011 (CUB), and the experimental results illustrate competitive/superior performance of our method compared to the previous state-of-the-art.

%In this paper, we propose a subspace representation learning (SRL) framework to tackle few-shot image classification task.
%While most previous works adopted global vector representation, using subspace representation can effectively preserve the spatial structure, and diversity within an image.