Previous approaches to few-shot image classification can be roughly divided into three categories:
(1) Distance metric based methods \cite{snell2017prototypical,vinyals2016matching,relationnet} construct a proper hidden feature space, whose distance metric is used to determine the image-class or image-image similarity.
The distance metric can be a non-parametric function \cite{snell2017prototypical,vinyals2016matching} or a parametric network module \cite{relationnet}.
(2) Optimization based methods \cite{finn2017model,jamal2019task,munkhdalai2018rapid} aim at learning a good initialization for the model so that it can be quickly fine-tuned to a target task with limited amount of data.
(3) Hallucination based methods \cite{hariharan2017low,wang2018low,lin2019semantics,kimmodel} solve data scarcity by generating more training samples.
The generation process is done in either hidden feature space or raw image space.
While achieving promising performance, these methods adopt a global feature vector to represent an image.

As an extension of distance metric based methods, some recent works \cite{zhang2020deepemd,li2019revisiting} rely on a local feature set representation to preserve the spatial structure of an image, and define their own metric to measure the similarity between two feature sets.
Li et al. \cite{li2019revisiting} calculates cosine distance between local feature pairs, and aggregates them by a k-NN classifier.
Zhang et al. \cite{zhang2020deepemd} adopt earth mover's distance (EMD) to discover an optimal matching between local feature sets.
In comparison, our method extracts a subspace representation for each local CNN feature set, and computes the weighted subspace distance between two sets.

The concept of subspace learning has been utilized to solve few-shot image classification \cite{pmlr-v97-yoon19a,simon2020adaptive}.
%However, these works focus on discovering task-specific \cite{pmlr-v97-yoon19a} and class-specific \cite{simon2020adaptive} subspace in global, image-level feature space.
%Also, they made use of subspace projection operation to compute subspace-point distance and measure class-to-image similarity.
However, these works consider the subspace in a global, image-level feature space.
Also, they make use of a subspace projection operation to compute subspace-point distance and measure class-to-image similarity.
In comparison, our method represents an image as a subspace in local feature space, and calculates subspace-subspace distance.
