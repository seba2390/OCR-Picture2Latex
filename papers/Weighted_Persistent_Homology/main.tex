\documentclass{amsart}
\usepackage[utf8]{inputenc}
\usepackage{amssymb}
\usepackage{graphicx}
\usepackage{subcaption}
\usepackage{enumerate}
\usepackage[colorinlistoftodos]{todonotes}
\usepackage{verbatim}
\usepackage{amsthm, url}
\usepackage{xcolor}
\usepackage{tikz-cd}
\usepackage{tikz,pgf}
\usetikzlibrary{decorations.text}
\usepackage{amsmath}
\usepackage{mathtools}
\usepackage{mdframed}

\newtheorem{theorem}{Theorem}[section]
\newtheorem*{fibering lemma}{Fibering Lemma}
\newtheorem*{decomposition lemma}{Decomposition Lemma}
\newtheorem*{hurewicz theorem}{Hurewicz Theorem}
\newtheorem{lemma}[theorem]{Lemma}
\newtheorem{proposition}[theorem]{Proposition}
\newtheorem{corollary}[theorem]{Corollary}

\theoremstyle{definition}
\newtheorem{definition}[theorem]{Definition}
\newtheorem{comm}[theorem]{Comment}
\newtheorem{example}[theorem]{Example}

\newtheorem{question}[theorem]{Question}

\newcommand{\U}[1]{{\mathcal{U}_{#1}}}
\newcommand{\V}[1]{{\mathcal{V}_{#1}}}
\newcommand{\W}[1]{{\mathcal{W}_{#1}}}
\newcommand{\df}[1]{{{\bf #1}}}
\newcommand{\red}[1]{{{\color{red} #1}}}
\newcommand{\RR}{\mathbb{R}}
\newcommand{\C}{\check{C}}
\newcommand{\Ball}{\bar{B}}
\renewcommand{\epsilon}{\varepsilon}
\newcommand{\austin}[1]{\todo[inline, color=green!40]{Austin todo: #1}}
\newcommand{\james}[1]{\todo[inline, color=blue!40]{James todo: #1}}
\newcommand{\greg}[1]{\todo[inline, color=orange!40]{Greg todo: #1}}
\newcommand{\cliff}[1]{\todo[inline, color=yellow!40]{Cliff todo: #1}}
\newcommand{\what}[1]{\todo[inline, color=red!40]{Question: #1}}
\newcommand{\answ}[1]{\todo[inline, color=purple!40]{Answer: #1}}

\newcommand{\VR}{\mathrm{VR}}


\DeclareMathOperator{\as}{asdim}
\DeclareMathOperator{\con}{con}
\DeclareMathOperator{\mesh}{mesh}
\DeclareMathOperator{\diam}{diam}
\DeclareMathOperator{\dist}{dist}
\DeclareMathOperator{\st}{st}
\DeclareMathOperator{\im}{im}
\DeclareMathOperator{\sgn}{sgn}
\DeclareMathOperator{\ord}{ord}
\DeclareMathOperator{\Cech}{\check{C}ech}
\DeclareMathOperator{\Conv}{Conv}
\DeclareMathOperator{\dgm}{dgm}

\begin{document}

\bibliographystyle{abbrv}

\title[Weighted persistent homology]{Weighted persistent homology}
\author{G.~Bell}
\address{Department of Mathematics and Statistics, The University of North Carolina at Greensboro, Greensboro, NC 27402, USA} 
\email{gcbell@uncg.edu}

\author{A.~Lawson}
\address{Department of Mathematics and Statistics, The University of North Carolina at Greensboro, Greensboro, NC 27402, USA} 
\email{azlawson@uncg.edu}

\author{J.~Martin}
\address{Department of Mathematics and Statistics, The University of North Carolina at Greensboro, Greensboro, NC 27402, USA} 
\email{jmmart27@uncg.edu}

\author{J.~Rudzinski}
\address{Department of Mathematics and Statistics, The University of North Carolina at Greensboro, Greensboro, NC 27402, USA} 
\email{jerudzin@uncg.edu}

\author{C.~Smyth }
\address{Department of Mathematics and Statistics, The University of North Carolina at Greensboro, Greensboro, NC 27402, USA} 
\email{cdsmyth@uncg.edu}
\thanks{Clifford Smyth was supported by NSA MSP Grant H98230-13-1-0222 and by a grant from the Simons Foundation (Grant Number 360486, CS)}
\begin{abstract}
We introduce weighted versions of the classical \v{C}ech and Vietoris-Rips complexes. We show that a version of the Vietoris-Rips Lemma holds for these weighted complexes and that they enjoy appropriate stability properties. We also give some preliminary applications of these weighted complexes.
\end{abstract}
\subjclass[2010]{55N35 (primary), 55U99 68U10 (secondary)}
\keywords{persistent homology}

\maketitle
\section{Introduction}
\begin{comment}
Notation (to be consistent, remove when finished going through the paper).
\begin{itemize}
    \item $\mathbf{r}$ is a map from $X$ to the set of differentiable functions $\{\phi: [0,\infty) \to [0,\infty) \hbox{ $\phi'$ exists, $\phi'(t)>0$ for all $t$ and $\lim_{t\to\infty} \phi(t)=+\infty$}\}$
    \item $\check{C}_{X,\mathbf{r}}(t)$ is the \v{C}ech complex given by the nerve $N(\{\Ball(x_i,r_{x_i}(t))\mid x_i\in X\})$ \textcolor{red}{(should this be $\mathcal{N}$ instead of $N$?)}
    \item The geometric realization of $\check{C}_{\mathbf{r}}(t)$ is 
    $\{\sigma: \sigma=\Conv(\{x_0,x_1,\ldots, x_k\}),\ k\ge 0,\ x_i\in X,\ \bigcap_{j=0}^k \Ball(x_j,r_{x_j}(t))\neq\emptyset \}$.
        \textcolor{red}{(Many of these r's are not bold. Is this intensional or should they be bold? Many r's throughout the paper are not bold.)}

    \item $\mathrm{VR}_{X,\mathbf{r}}(t)=\{\sigma:\sigma=\{x_0,\ldots, x_k\}, k\ge 0, \ x_j\in X, \ \|x_i-x_j\|\le r_{x_i}(t)+r_{x_j}(t) \forall 0\le i<j\le k \}$
    \item $(X,\mathbf{r})$ and $(Y,\mathbf{s})$ are two such pairs of finite sets with radius functions. $\eta:X\to Y$ is a map taking $\eta(x_i)=y_i$. $K$ is a compact set containing the sets $X$ and $Y$. 
   	\item Entry function: $f_{X,r}(x)=\min_{x_i\in X}\{r^{-1}_{x_i}(d(x,x_i))\}$.
    \item For each $X,\mathbf{r}$ and each $\ast$, $H_\ast:\mathbf{R}\to Vect$ is the simplicial homology of $f^{-1}_{X,\mathbf{r}}([0, t])$, which is the simplical homology of the nerve, i.e. of $\check{C}_{X,\mathbf{r}}(t)$.
    \item Whenever $s<t$ there is a natural inclusion $f^{-1}_{X,\mathbf{r}}([0,s])\subset f^{-1}_{Y,\mathbf{s}}([0,t])$, which induces $\iota_\ast:H_\ast(s)\to H_\ast(t)$. 
    \item When $\mathbf{r}$ is linear in each element of $X$, we write $\vec{r}\in [0,\infty)^X$. Here, $\mathbf{r}_{x_i}(t)=\vec{r}(x_i)t$. Similarly, write $\mathbf{s}_{y_i}(t)=\vec{s}(y_i)t$. 
\end{itemize}
\end{comment}

Topological data analysis (TDA) provides a means for the power of algebraic topology to be used to better understand the shape of a dataset. In the traditional approach to TDA, isometric balls of a fixed radius $r>0$ are centered at each data point in some ambient Euclidean space. One then constructs the nerve of the union of these balls and computes the simplicial homology of this nerve. Computationally, this approach is infeasible for large data sets or high-dimensional data, so instead one computes the so-called Vietoris-Rips complex, which is the flag complex over the graph obtained by placing an edge between any pair of vertices that are at distance no more than $2r$ from each other. The key idea of TDA is to allow the radius of these balls to vary and to compute simplicial homology for each value of this radius to create a topological profile of the space. This profile is encoded in either a barcode or a persistence diagram. Topological features such as holes or voids that exist for a relatively large interval of radii are said to persist and are believed to be more important than more transient features that exist for very short intervals of radii. (There are, however, important exceptions to this rule of thumb, see~\cite{Brains}).

In the traditional model, the radius of each ball is the same and can be modeled by the linear function of time $r(t) = r t$.  In this paper, we consider a model of computing persistent homology in which the radius of each ball is allowed to be a different monotonic function $r_x(t)$ at each point $x$.  In this way we can emphasize certain data points by assigning or {\em weighting} them with larger and/or more quickly growing balls and de-emphasize others by weighting them with smaller and/or more slowly growing balls. This is appropriate in the case of a noisy dataset, for instance, as an alternative to throwing away data that fails to meet some threshold of significance. Various other methods of enhancing persistence with weights have been considered (e.g.~\cite{Buchet,Edelsbrunner,Petri,Ren-Further,Ren}).

The weighted model we propose fits into the framework of generalized persistence in the sense of \cite{BdSS}. We show that it enjoys many of the properties familiar from the techniques of traditional persistent homology. We prove a weighted Vietoris-Rips Lemma (Theorem~\ref{thm:multiscale-rips}) that relates our weighted \v{C}ech and Rips complexes in the same way that they are related in the case of isometric balls. We also show that the persistent homology computed over weighted complexes is stable with respect to small perturbations of the rates of growth and/or the points in the dataset (Theorem~\ref{thm:stability}). Moreover, packages for computing persistent homology such as Javaplex~\cite{Javaplex} and Perseus~\cite{Perseus} are capable of handling our weighted persistence with the same complexity as unweighted persistence by merely adjusting inputs to the package functions.

As a proof of concept, we apply our methods to the Modified National Institute of Standards and Technology (MNIST) data set of handwritten digits translated into pixel information. Our method proves more effective than isometric persistence in finding the number $8$ from among these handwritten digits. (We chose $8$ for its unique $1$-dimensional homology among these digits.) We found our methods to be $95.8\%$ accurate as opposed to isometric persistence's $92.07\%$ accuracy. This experiment was chosen to demonstrate the performance of weighted persistence over usual persistence, but it should be noted that neither method approaches the accuracy of state-of-the art computer vision and we make no claim that we are improving on known methods.

In Section 2, we provide the background definitions that are needed for what follows and describe our weighted persistence model. In Section 3 we prove the weighted Vietroris-Rips Lemma and indicate how persistent homology packages can be used to compute weighted persistence. In Section 4 we establish our stability results. Our experiments on MNIST data appear in Section 5. We end with some remarks and questions for further study.

\section{Preliminaries}

We begin by defining some terminology and setting our notation. We will assume some familiarity with simplicial homology and the basic ideas of topological data analysis. For details, we refer to~\cite{Ede:10,Rotman}.

In algebraic topology, simplicial homology is a tool that assigns to any simplicial complex $K$ a collection of $\mathbb{Z}$-modules $H_0(K), H_1(K),\ldots$, called \df{homology groups}, in such a way that the rank of $H_n(K)$ describes the number of ``n-dimensional holes'' in $K$. For our purposes, we replace the standard definition in terms of $\mathbb{Z}$-modules with vector spaces (usually over the field with two elements, for ease of computation). We therefore refer to \df{homology vector spaces} instead of homology groups. We do not attempt to define $H_n(K)$ here, but instead refer to any text in algebraic topology, such as~\cite{Rotman}.

Let $\mathcal{U}$ be a collection of sets. We define the \df{nerve} $\mathcal{N}(\mathcal{U})$ to be the abstract simplicial complex with vertex set $\mathcal{U}$ with the property that the subset $\{U_0,U_1,\ldots, U_n\}$ of $\mathcal{U}$ spans an $n$-simplex in $\mathcal{N}$ whenever $\bigcap_{i=0}^nU_i\neq\emptyset$. 

Let $(X,d)$ be a metric space.  We define $B_r(x) = \{y \in X | d(x,y) <r\}$ and 
$\Ball_r(x)=  \{y \in X | d(x,y) \leq r\}$ to be the open and closed balls of radius $r$ about $x$, respectively. (Note that we're abusing notation since in a general metric space $\Ball_r(x)$ is not necessarily the closure of the open ball, usually denoted $B_r(x)$).  We most often consider examples where $X$ is a subset of $\mathbb{R}^d$ and $d(x,y) = \|x-y\|$ is the Euclidean distance between $x$ and $y$. For a real number $r\ge 0$, we define the \df{\v{C}ech complex of $X$ at scale $r$} by $\Cech(r)=\mathcal{N}\{\Ball_r(x)\mid x\in X\}$.  

We generalize this construction by allowing the radius of the ball around each element $x$ to depend on $x$. Let $\mathbf{r}:X\to[0,\infty)$ be any function. We define the \df{weighted $\mathbf{r}$-$\Cech$ complex} $\Cech(\mathbf{r})$ of $X$ by $\Cech(\mathbf{r})=\mathcal{N}\{\Ball_{\mathbf{r}(x)}(x)\}.$ 

In practice, it is difficult to determine whether an intersection of balls is nonempty. A much simpler construction to use is the Vietoris-Rips complex. For a given parameter $r\ge 0$ the \df{Vietoris-Rips complex} is the flag complex of the $1$-skeleton of the $\Cech$ complex, i.e. a collection of $n+1$ balls forms an $n$-simplex in the Vietoris-Rips complex if and only if the balls are pairwise intersecting.  For the Vietoris-Rips complex we identify each ball with its center, so that the \df{Vietoris-Rips complex at scale $r$} is $\VR(r) =\{\sigma\subset X\mid \diam(\sigma) \leq 2r\}.$ Similarly, if $\mathbf{r}:X\to [0,\infty)$, the \df{weighted $\mathbf{r}$-Vietoris-Rips complex} is $\VR(\mathbf{r}) =\{\sigma\subset X\mid d(x,y) \leq \mathbf{r}(x)+\mathbf{r}(y), \text{for all $x, y \in \sigma$ with $x \neq y$}\}$. 

Fix $\mathbf{r}:X\to[0,\infty)$ and consider the simplicial complex $\Cech(\mathbf{r})$ (or $\VR(\mathbf{r})$). Using simplicial homology with field coefficients, one can associate homology vector spaces $H_*\left(\Cech(\mathbf{r})\right)$ to these simplicial complexes. Whenever $t_0 \le t_1$ there is a natural inclusion map of simplicial complexes given by $\iota:\Cech(t_0\mathbf{r})\to \Cech(t_1\mathbf{r})$ (or the corresponding inclusion of the Vietoris-Rips complexes). By functoriality, there is an induced linear map on homology $\iota_*:H_*\Cech(t_0\mathbf{r}))\to H_*\Cech(t_1\mathbf{r})$.


Let $X\subset\mathbb{R}^d$ be finite. Although we defined the weighted complexes above for any function $\mathbf{r}:X\to[0,\infty)$, we want to study the persistence properties of these weighted complexes. For example, in the case of the weighted \v{C}ech complex, we want to study the evolution of homology as the radii of the balls grow to infinity. One straightforward way to do this would be to simply scale our weighted complexes linearly in the same way that one usually scales the isometric balls in persistent homology. We prefer a more flexible approach, which we describe in terms of radius functions. 

Let $\mathcal{C}^1_+=\mathcal{C}^1_+([0,\infty)]$ denote the collection of differentiable bijective functions $\phi:[0,\infty)\to[0,\infty)$ with positive first derivative. By a \df{radius function} on $X$ we mean a function $\mathbf{r}:X\to \mathcal{C}^1_+$. We denote the image function $\mathbf{r}(x)$ by $\mathbf{r}_x$.

For $t \geq 0$, we define the \df{$\Cech$ and Vietoris-Rips complexes at scale $t$} by \[\Cech_{\mathbf{r}}(t) = \mathcal{N}\{\Ball_{\mathbf{r}_x(t)}(x)\}\] and \[VR_{\mathbf{r}}(t) = \{\sigma \subset X | d(x,y) \leq \mathbf{r}_x(t) + \mathbf{r}_y(t) \text{ for all $x,y \in \sigma$ with $x\neq y$}\},\] respectively.
%\{\phi:[0,\infty)\to [0,\infty)\mid \hbox{ bijective, differentiable with } \phi'(t)>0\, \forall t\}$. 
 We define the \df{entry function}, \begin{equation} \label{eq:entry_function}
      f_{X,\mathbf{r}}(y)=\min_{x\in X}\{\mathbf{r}^{-1}_{x}(d(y,x))\}\hbox{.}\end{equation} 
      
      This function captures the scale $t$ at which the point $y\in\mathbb{R}^d$ is first captured by some ball $\Ball_{\mathbf{r}_x(t)}(x)$; we have $f_{X, \mathbf{r}}(y) = t$ if and only if $y \in \Ball_{\mathbf{r}_x(t)}(x)$ for some $x$ in $X$ and $y \not \in \bigcup_{x \in X} B_{\mathbf{r}_x(t)}(x)$. Thus we have the following proposition.

\begin{proposition}
Let $X$ be a finite subset of some Euclidean space $\mathbb{R}^d$. Suppose that $\mathbf{r}$ and $f_{X,\mathbf{r}}$ are defined as above. Then, \[f^{-1}_{X,\mathbf{r}}\left( [0,t]\right)=\bigcup_{x \in X}\Ball\left(x,\mathbf{r}_{x}(t)\right)\hbox{.}\]
\end{proposition}

It follows from the Nerve Lemma (see for example, \cite[Corollary 4G.3]{Hatcher}), that $\Cech_{\mathbf{r}}(t)$ is homotopy equivalent to $f^{-1}_{X,\mathbf{r}}\left([0,t]\right)$.

\iffalse

We begin by defining terminology and setting our notation. The reader is referred to~\cite{Ede:10,Rotman} for more details.
Let $u_0,u_1,\ldots,u_n$ be an affinely independent subset of some Euclidean space $\mathbb{R}^d$; i.e., the collection $\{u_1-u_0,u_2-u_0,\ldots, u_n-u_0\}$ is a linearly independent set. An \df{$n$-simplex} $\sigma$ is the convex hull of the $u_i$, denoted $\sigma = \Conv(\{u_0, u_1, \ldots, u_n\})$. Let $k\le n$ be a positive integer. A $k$-\df{face} $\tau$ of $\sigma$ is the convex hull of $k+1$ of the $u_i$; we call $\rho$ a \df{face} of $\sigma$ if it is a $k$-face for some $k$; we call the face $\tau$ of $\sigma$ a \df{proper face} of $\sigma$ if $\tau\neq\sigma$. We denote the face relation and proper face relation, respectively, by $\tau \leq \sigma$ and $\tau < \sigma$. We refer to $0$-simplices as \df{vertices}; $1$-simplices are called \df{edges}; $2$-simplices are called \df{triangles}; and $3$-simplices are called \df{tetrahedra}.

A simplicial complex is a finite collection $K$ of simplices such that (1) $\sigma \in K$ and $\tau \leq \sigma$ implies $\tau \in K$; and (2) $\sigma, \tau \in K$ implies $\sigma \cap \tau$ is either empty or is a face of both $\sigma$ and $\tau$. The geometric realization of $K$, denoted as $|K|$, is the topological space on the collection of simplices viewed as being embedded in a sufficiently high dimensional Euclidean space with the subspace topology. \begin{comment} A triangulation of a topological space $\mathbb{X}$ is a simplicial complex $K$ together with a homeomorphism between $\mathbb{X}$ and the underlying space $|K|$. The topological space is triangulable if it has a triangulation. \end{comment}

It is sometimes useful to describe simplicial complexes abstractly. Let $A$ be a finite nonempty set. An \df{abstract simplicial complex} is a family $K$ of nonempty subsets of $A$ called \df{simplices} such that (1) if $a\in A$, then $\{a\}\in A$; and (2) if $\sigma\in K$ and $\tau\subset \sigma$, then $\tau\in K$. A simplicial complex is called a \df{flag complex} (or is said to be \df{flag}) if every set collection of pairwise adjacent vertices spans a simplex.




\fi 

\section{A weighted Vietoris-Rips lemma}

The Vietoris-Rips complex is much easier to compute than the \v{C}ech complex in high dimensions. To determine whether $n+1$ balls form an $n$-simplex in the \v{C}ech complex, we must check whether the balls intersect, a computationally complex problem. To determine whether $n+1$ balls $B_{r_i}(x_i)$ form a simplex in the Vietoris-Rips complex is computationally easy, only $\binom{n+1}{2}$ conditions $d(x_i,x_j) \leq r_i +r_j$ need be checked.  Furthermore, if there are $m$ points in $X$, it may be necessary to check all $2^m$ sub-collections of balls to determine the \v{C}ech complex, whereas determining the Rips complex will only require checking $\binom{m}{2}$ pairs of points.

Our weighted \v{C}ech and Vietoris-Rips complexes are similar in spirit to weighted alpha complexes~\cite[III.4]{Ede:10}. Both constructions seek to permit ``balls'' with different sizes. Our constructions are simpler from a conceptual standpoint since the alpha complexes are built as subcomplexes of the Delaunay complex, which comes from the Voronoi diagram. Moreover, our complexes are computationally simple; indeed our method of finding weighted Vietoris-Rips complexes requires only marginally more computation than the unweighted Vietoris-Rips complex.

In particular, Javaplex and Perseus can compute regular (unweighted) persistent homology given input of a distance matrix $M$ with $M_{i,j}=d(x_i,x_j)$. Inputting $M_{i,j}=d(x_i,x_j)/(r_i+r_j)$ allows these packages to compute the persistent homology with $r_{x_i}(t)=r_it$ in the same time.

In computational problems it is common to use the Vietoris-Rips complex instead of the $\Cech$ complex to simplify the calculational overhead. The following theorem justifies this decision by saying that the Vietoris-Rips complex is ``close'' to the weighted \v{C}ech complex. 

The classical Vietoris-Rips Lemma can be stated as follows:
\begin{theorem}\cite{dSG-2007}
Let $X$ be a set of points in $\mathbb{R}^d$ and let $t > 0$.  Then \[\VR(t') \subseteq \Cech(t) \subseteq \VR(t)\] whenever $0< t' \le t \left(\sqrt{2d/(d+1)}\right)^{-1}$.
\end{theorem}

The main result of this section is an extension of this result to the weighted case.  

\begin{theorem}[Weighted Vietoris-Rips Lemma] \label{thm:multiscale-rips} Let $X$ be a set of points in $\RR^d$.  Let $\mathbf{r}: X \to (0,\infty)$ be the corresponding weight function and let $t >0$. Then \[\VR(t'\mathbf{r}) \subseteq \Cech(t \mathbf{r}) \subseteq \VR(t \mathbf{r})\] whenever $0< t' \le t \left(\sqrt{2d/(d+1)}\right)^{-1}$.
\end{theorem}

\begin{proof}
The second containment $\Cech(t \mathbf{r}) \subseteq \VR(t \mathbf{r})$ follows from the fact that the weighted Vietoris-Rips complex is the flag complex of the weighted \v{C}ech complex. 
        
To show that $\VR(t'\mathbf{r})\subset \Cech(t\mathbf{r})$, we suppose there is some finite collection $\sigma = \{x_k\}_{k=0}^\ell \subseteq \RR^d$ with $\ell >0$ that is a simplex in $\VR(t' \mathbf{r})$ and show that this is also a simplex in $\Cech(t \mathbf{r})$. We have $\|x_i-x_j\|_2 \le t' (\mathbf{r}(x_i) + \mathbf{r}(x_j))$ whenever $i \neq j$.

Define a function $f:\RR^d \to \RR$ by \[f(y) = \max_{0 \leq j \leq \ell} \left\{\frac{\|x_j - y\|_2}{\mathbf{r}(x_j)}\right\}.\] Clearly, $f$ is continuous and $f(y) \to \infty$ as $\|y\|_2 \to \infty$. Thus $f$ attains a minimum (say at $y_0$) on some compact set containing $\Conv(\{x_k\}_{k=0}^{\ell})$. (Here $\Conv(S)$ is the convex hull of the set $S \subseteq {\mathbb R}^d$.) We must have $\|x_i-y_0\|_2/ r(x_i) = f(y_0)$ for at least one of the vertices $x_i$. By reordering the vertices, we may assume that  \[f(y_0) = \frac{1}{\mathbf{r}(x_j)} \|x_j - y_0\|_2\qquad \hbox{if $0\le j\le n$}\] and \[f(y_0) > \frac{1}{\mathbf{r}(x_j)} \|x_j - y_0\|_2\qquad \hbox{if $n<j\le \ell$.}\] Let \[g(y) = \max_{0 \leq j \leq n} \left\{\frac{1}{\mathbf{r}(x_j)}\left\|x_j - y\right\|_2\right\}\] and \[h(y) = \max_{n < j \leq \ell} \left\{\frac{1}{\mathbf{r}(x_j)}\left\|x_j - y\right\|_2\right\}.\]
		
Now we wish to show that $y_0 \in \mathrm{Conv}(\{x_j\}_{j=0}^{n})$. To this end we apply the Separation Theorem \cite{Matousek} to obtain: either $y_0 \in \Conv(\{x_j\}_{j=0}^{n})$ or there is a $v \in \RR^d$ and a $C<0$ such that $v \cdot x_j \ge 0$ for all $0 \le j \le n$ and $v \cdot y_0 < C$. Thus if $y_0 \not \in \Conv(\{x_j\}_{j=0}^{n})$ there is a $v \in \RR^d$ so that $v \cdot (x_j - y_0) > 0$ for $0 \leq j \leq n$. We suppose that there is such a $v$ and derive a contraction.

Since \[\left\|x_j - (y_0 + \lambda v)\right\|_2^2 = \left\|x_j - y_0\right\|_2^2 - 2 \lambda v \cdot (x_j - y_0) + \lambda^2 \left\|v\right\|_2^2\] for each $0 \leq j \leq n$, it follows that $g(y_0 + \lambda v) < f(y_0)$ for all $0 < \lambda < \lambda_1$, where \[\lambda_1 = \min_{0 \leq j \leq n} \left\{\frac{2 v \cdot (x_j - y_0)}{\left\|v\right\|_2^2}\right\}.\] Since $h(y)$ is continuous and $h(y_0) < f(y_0)$, there exists a $\lambda_2$ so that $h(y_0 + \lambda v) < f(y_0)$ for $0 < \lambda < \lambda_2$. Thus, there exists a $\lambda > 0$ such that \[f(y_0 + \lambda v) = \max\left\{g(y_0 + \lambda v), h(y_0 + \lambda v)\right\} < f(y_0)\hbox{,}\] contradicting the minimality of $y_0$.

By Carath\'{e}odory's theorem \cite{Matousek} and reordering of vertices if necessary, $y_0$ is a convex combination of some subcollection of vertices $\{x_j\}_{j=0}^{m}$ where $m \leq \min\{d,n\}$. It is not possible that $m = 0$. If so, then $y_0 = x_0$ and $f(y_0) = \frac{1}{\mathbf{r}(x_0)}\|x_0 - y_0\|_2 = 0$ and f is identically zero. Since $\sigma$ has dimension at least $1$, it contains a vertex $x_1 \neq x_0$. It follows that $f(y_0) = f(x_0) > \frac{1}{\mathbf{r}(x_1)} \|x_1 - x_0\|_2 > 0$, which is a contradiction. 
		
Let $\widehat{x_j} = x_j - y_0$ for all $0 \leq j \leq m$. Note that 
\begin{equation}
\|\widehat{x_j}\|_2^2 = \mathbf{r}(x_j)^2 f(y_0)^2.
\end{equation} 
Since $y_0 \in \Conv(\{x_j\}_{j=0}^m)$, $y_0 = \sum_{j=0}^{m} a_j x_j$ for some set of non-negative real numbers $a_0,\ldots, a_m$ that sum to $1$. Thus $\sum_{j=0}^{m} a_j \widehat{x_j}=0$. By relabeling, we may assume that $a_0 \mathbf{r}(x_0) \ge a_j\mathbf{r}(x_j)$ when $j > 0$. Necessarily $a_0 >0$. (Otherwise $a_j = 0$ for all $0 \leq j \leq m$, a contradiction.) Then, \[\widehat{x_0} = -\sum\limits_{j=0}^{m} \frac{a_j}{a_0} \widehat{x}_j\] and so \[\mathbf{r}(x_0)^2 f(y_0)^2 = \|\widehat{x_0}\|_2^2 = - \sum\limits_{j=0}^{m} \frac{a_i}{a_0} \widehat{x}_0 \cdot \widehat{x_j}.\]
		
Among the indices $1,2,\ldots,m$, there is some $j_0$ such that
\begin{equation}
\frac{1}{d} \mathbf{r}(x_0)^2 f(y_0)^2 \leq \frac{1}{m} \mathbf{r}(x_0)^2 f(y_0)^2 \leq -\frac{a_{j_0}}{a_0} \widehat{x_0} \cdot \widehat{x_{j_0}}.
\end{equation}
We must have $a_{j_0}>0$.  (Otherwise, $f(y_0) = 0$, which, as shown earlier, is a contradiction.)
By reordering, we may assume $j_0 = 1$.
Putting (1) and (2) together, we find 
\begin{align*}
f(y_0)^2 \left( \mathbf{r}(x_0)^2 + \frac{2 a_0 \mathbf{r}(x_0)^2}{a_1 d} + \mathbf{r}(x_{1})^2\right) & = f(y_0)^2\mathbf{r}(x_0)^2+\frac{2a_0f(y_0)^2\mathbf{r}(x_0)^2}{a_1d}+f(y_0)^2\mathbf{r}(x_1)^2 \\
&\leq \left\|\widehat{x_0}\right\|_2^2 - 2 \widehat{x_0} \cdot \widehat{x_1} + \left\|				\widehat{x_1}\right\|_2^2 \\
			&= \left\|\widehat{x_0} - \widehat{x_1}\right\|_2^2 \\
			&= \left\|x_0 - x_1\right\|_2^2 \\
			&\leq \left(t' \left(\mathbf{r}(x_0)+\mathbf{r}(x_1)\right)\right)^2.
\end{align*}

We will now show that \[\frac{f(y_0)^2}{t'} \leq \frac{(\mathbf{r}(x_0)^2 + \mathbf{r}(x_1)^2)^2}{\mathbf{r}(x_0)^2 + \frac{2 a_0 \mathbf{r}(x_0)^2}{a_1 d} + \mathbf{r}(x_1)^2} \leq \frac{2d}{d+1}.\] It suffices to show, after cross-multiplying the right-hand inequality, that \[(d - 1 + 4\frac{a_0}{a_1}) \mathbf{r}(x_0)^2 - 2(d+1) \mathbf{r}(x_0) \mathbf{r}(x_1) + (d-1) \mathbf{r}(x_1)^2 \geq 0.\] Since $\dfrac{a_0}{a_1} \geq \dfrac{\mathbf{r}(x_1)}{\mathbf{r}(x_0)}$ we get
\begin{align*}
\left(d - 1 + 4\frac{a_0}{a_1}\right) \mathbf{r}(x_0)^2 &- 2(d+1) \mathbf{r}(x_0) \mathbf{r}(x_1) + (d-1) \mathbf{r}(x_1)^2 \\
			&\geq \left(d - 1 + 4\frac{\mathbf{r}(x_1)}{\mathbf{r}(x_0)}\right) \mathbf{r}(x_0)^2 - 2(d+1) \mathbf{r}(x_0) \mathbf{r}(x_1) + (d-1) \mathbf{r}(x_1)^2 \\
			&= (d-1)\left(\mathbf{r}(x_0) - \mathbf{r}(x_1)\right)^2 \\
			&\geq 0
\end{align*}
as desired. Our assumption that $t' \le t (\sqrt{2d/(d+1)})^{-1}$ implies $f(y_0) \leq t$ and thus \[y_0 \in \bigcap\limits_{i = 0}^{\ell} \Ball_{t \mathbf{r}(x_i)} (x_i).\] Therefore $\sigma \in \check{C}(t\mathbf{r})$ and we are done.
	\end{proof}

\section{Stability} %of Multi-radial filtrations}


In this section we discuss the stability of our weighted persistence.  Let $X$ and $Y$ be finite subsets of $\mathbb{R}^d$ with corresponding radii functionals ${\mathbf r}: X \to \mathcal{C}^1_+$ and $\mathbf{s}: Y \to \mathcal{C}^1_+$.  Informally, we show that if $(X, \mathbf{r})$ and $(Y, \mathbf{s})$ are ``close'', i.e. are small perturbations of each other, then the corresponding entry functions $f_{X, \mathbf{r}}$ and $f_{Y,\mathbf{s}}$ (see \eqref{eq:entry_function}) are also ``close'' and hence the associated persistence diagrams must also be ``close''.  We'll now make the the definitions of these various types of closeness precise.

Let $\eta \subseteq X \times Y$ be a relation such that for every $x \in X$ there is a $y \in Y$ with $(x,y) \in \eta$ and for every $y \in Y$ there is an $x \in X$ with $(x,y) \in \eta$.  We measure the closeness of $X$ and $Y$ with respect to $\eta$ by \[\|\eta\| := \max_{(x,y) \in \eta} d(x,y).\]  If $L$ is any compact set and $h : L \to \mathbb{R}$ is continuous let \[\|h\|_L := \max_{x \in L} |h(x)|.\] Let $K$ be a compact subset of $\mathbb{R}^d$ that contains $X \cup Y$.  The closeness of $\mathbf{r}$ and $\mathbf{s}$ is measured by \[D(\mathbf{r}, \mathbf{s})_{\eta,K} := \max_{(x,y) \in \eta} \|\mathbf{r}_x^{-1} - \mathbf{s}_y^{-1}\|_{[0,\diam(K)]}.\]  
The closeness of $f_{X, \mathbf{r}}$ and $f_{Y,\mathbf{s}}$ is measured by $\|f_{X, \mathbf{r}} - f_{Y,\mathbf{s}}\|_K.$  We also define $S(\mathbf{r})_K := \max_{x \in X} \|(\mathbf{r}^{-1}_x)'\|_{[0,\diam(K)]}$.

As is common, we measure the closeness of persistence diagrams by the bottle-neck distance.  We'll give the definition of this metric in the remarks leading up to Theorem 4.5.



\begin{theorem} \label{thm:stability}
In the above notation we have the following bound on entry functions (see \eqref{eq:entry_function}): 

\[\|f_{X,\mathbf{r}}-f_{Y,\mathbf{s}}\|_K\le D(\mathbf{r},\mathbf{s})_{\eta,K}+ \|\eta\| \max( S(\mathbf{r})_K, S(\mathbf{s})_K)\]
\end{theorem}

\begin{proof}

%For notational convenience, we will take $X = \{x_1, \ldots, x_n\}$, $Y=\{y_1, \ldots, y_m\}$.

There is some point $z$ in the compact set $K$ and some points $x \in X$ and $y \in Y$ so that
\[\|f_{X,\mathbf{r}}-f_{Y,\mathbf{s}}\|_{K}=|f_{X,\mathbf{r}}(z)-f_{Y,\mathbf{s}}(z)| = |\mathbf{r}^{-1}_{x}(d(z,x))-\mathbf{s}^{-1}_y(d(z,y))|.\]

We first suppose $\mathbf{r}^{-1}_{x}(d(z,x)) \geq \mathbf{s}^{-1}_{y}(d(z,y))$. Let $x' \in X$ such that $(x',y) \in \eta$.  Since $f_{X,\mathbf{r}}$ is a minimum, $\mathbf{r}^{-1}_{x'}(d(z,x')) \geq \mathbf{r}^{-1}_{x}(d(z,x))$ and we have
\begin{equation} \label{eq:ineq} \|f_{X,\mathbf{r}}-f_{Y,\mathbf{s}}\|_{K} \leq |\mathbf{r}^{-1}_{x'}(d(z,x'))-\mathbf{s}^{-1}_{y}(d(z,y))| \end{equation}
\[\leq |\mathbf{r}^{-1}_{x'}(d(z,x'))-\mathbf{s}^{-1}_{y}(d(z,x'))| + |\mathbf{s}^{-1}_{y}(d(z,x'))-\mathbf{s}^{-1}_{y}(d(z,y))|\]
Since $d(z,x') \in [0,\diam(K)]$, 
\[|\mathbf{r}^{-1}_{x'}(d(z,x'))-\mathbf{s}^{-1}_{y}(d(z,x'))| \leq D(\mathbf{r},\mathbf{s})_{\eta,K}.\]
Since $|d(z,x') - d(z,y)| \leq d(x',y) \leq \|\eta\|$ we apply the mean value theorem to obtain the bound
\[|\mathbf{s}^{-1}_{y}(d(z,x'))-\mathbf{s}^{-1}_{y}(d(z,y))|\le \|\eta\| \cdot \|(\mathbf{s}^{-1}_y)'\|_{[0,\diam(K)]} \leq \|\eta\| \max( S(\mathbf{r})_K, S(\mathbf{s})_K).\]
Together, these last two bounds give the bound of the theorem.  A similar argument gives the same bound if $\mathbf{r}^{-1}_{x}(d(z,x)) \leq \mathbf{s}^{-1}_{y}(d(z,y))$.
\end{proof}

If one has free choice of the perturbed set $(Y,\mathbf{s})$ it is clear that $\|f_{X,\mathbf{r}} - f_{Y,\mathbf{s}}\|_K$ can be made arbitrarily large. This could be done, say by adding a point to $Y$ that is arbitrarily far from any point in $X$ or by making one $\mathbf{s}_y$ arbitrarily larger than any $\mathbf{r}_x$.  The upper bound of Theorem 4.1 is also a bound on how extreme such perturbations may be.


We have the following immediate corollary of Theorem~\ref{thm:stability}.

\begin{corollary}
If the radii functions are all linear, i.e. if there are positive constants $r_{x}$ and $s_{y}$ for all $x \in X$ and $y \in Y$ such that $r_{x}(t) = r_{x} t$ and $s_{y}(t) = s_{y} t$, then \[\|f_{X,r}-f_{Y,s}\|_K\le \diam(K) \max_{(x,y) \in \eta} \left|\frac{1}{r_{x}}- \frac{1}{s_{y}}\right|+ \|\eta\| \max\left( \max_{x \in X} \frac{1}{r_{x}} , \max_{y \in Y} \frac{1}{s_{y}}\right)\hbox{.}\]


\end{corollary}

For our next two corollaries, let $X$ and $Y$ have the same cardinality and let $m : X \to Y$ be a bijection.  We now consider each point $x \in X$ as being perturbed to a point $m(x) \in Y$ and hence set $\eta = \{ (x,m(x)) : x \in X\}$. We have the following point stability result in which the points are perturbed but the weight functions stay the same.  

\begin{corollary}(Point-stability)
If only the locations of the points are perturbed and the radius functions stay the same, i.e. $\mathbf{s}_{m(x)}(t) = \mathbf{r}_{x}(t)$ for all $x \in X$, then
\[\|f_{X,\mathbf{r}}-f_{Y,\mathbf{s}}\|_K\le \max_{x \in X} d(x,m(x)) \left\|(\mathbf{r}^{-1}_{x})'\right\|_{[0,\diam(k)]}\hbox{.}\]
\end{corollary}

\begin{proof}
We follow the proof of Theorem~\ref{thm:stability}.  Take $x' \in X$ such that $m(x')=y$.  Then $\mathbf{S}_y = \mathbf{r}_{x'}$ and the first term in the upper bound in inequality \eqref{eq:ineq} is $0$.  Since the second term in that upper bound is bounded above by $d(x',m(x')) \|(\mathbf{r}_{x'}^{-1})'\|_{[0,\diam(K)]}$, the bound of the corollary holds.
\end{proof}

The next corollary is a weight-function stability result concerning a case in which the points stay the same ($Y = X$ and $m(x)=x$) but the weight functions are perturbed.
\begin{corollary}(Weight-function stability)
If only the radii functions are perturbed and the points stay the same then, 
\[\|f_{X,\mathbf{r}}-f_{X,\mathbf{s}}\|_K\le \max_{x \in X} \|\mathbf{r}_{x}^{-1}-\mathbf{s}_{x}^{-1}\|_{[0,\diam(K)]}\hbox{.}\]
\end{corollary}

\begin{proof}
Again following the proof of Theorem~\ref{thm:stability} we take $x'=m(x')=y$.  Now the second term in the upper bound of \eqref{eq:ineq} is $0$ and the first term is $|\mathbf{r}^{-1}_{y}-\mathbf{s}^{-1}_{y}|$ where $t=d(z,y)$. The corollary follows.
\end{proof}

We now show the stability of the persistence diagrams of $f_{X,\mathbf{r}}$ under perturbations of $X$ and $\mathbf{r}$.  Let $f: K \to [0,\infty)$ be a real valued function on a compact set $K \subseteq \mathbb{R}^d$. The \df{persistence diagram} of $f$, $\dgm(f)$, is a multi-set of points in $[0, +\infty]^2$ recording the appearance and disappearance of homological features in $f^{-1}([0,t])$ as $t$ increases.  Each point $(b,d)$ in the diagram tracks a single homological feature, recording the scale $t=b$ at which the feature first appears and the scale $t=d$ at which it disappears \cite{Ede:10}. It should also be noted that if one considers the birth-death pair as an interval we obtain the \df{barcode} as seen in \cite{Zom:05} (see Figures 2 and 3).  Given two functions $f,g : K \to [0, \infty]$ let $P = \dgm(f)$ and $Q = \dgm(g)$ be the corresponding persistence diagrams (where as usual we include all points along the diagonal in $P$ and $Q$). We let $N$ denote the set of all bijections from $P$ to $Q$. We recall that the \df{bottleneck distance} between the diagrams~\cite{Ede:10} is given by
\[
d_B(\dgm(f),\dgm(g)) = \inf_{\gamma \in N} \sup_{x \in P} ||x - \gamma(x)||_\infty\text{.}
\]
We have 
\begin{theorem}\cite[Theorem 6.9]{CSEH} Suppose $\mathcal{X}$ is a triangulable space and that $f:\mathcal{X}\to \mathbb{R}$ and $g:\mathcal{X}\to \mathbb{R}$ are tame, continuous functions. If $|f-g|$ is bounded, then for each $n$,
\[d_B(\dgm_n(f), \dgm_n(g))\le \|f-g\|_\infty\]
where $d_B$ denotes the bottleneck distance and $\dgm_n(f)$ denotes the $n$-th persistence diagram of the filtration of $f$.
\end{theorem}
We refer to \cite{Ede:10} for the technical definitions of tame and triangulable.  Note that as our spaces are nerves of balls around finite collections of points, they are finite simplicial complexes.  Hence they are triangulable and only admit tame functions.  Thus for our setting we get the following corollary.
\begin{corollary}
Let $X$ and $Y$ be finite subsets of $\mathbb{R}^d$ and let  $\mathbf{r}:X \to {\mathcal C}^1_+$ and $\mathbf{s} : Y \to {\mathcal C}^1_+$.  Suppose that $\eta\subseteq X\times Y$ is a relation as above and $K$ is a compact subset of $\mathbb{R}^d$ containing $X$ and $Y$.  Then for each $n$,
\[d_B(\dgm_n(f_{X,\mathbf{r}}),\dgm_n(f_{Y,\mathbf{s}})) \le D(\mathbf{r},\mathbf{s})_{\eta,K}+ \|\eta\| \max( S(\mathbf{r})_K, S(\mathbf{s})_K)\hbox{.}\] 
\end{corollary}


\begin{comment}
\section{Multi-radius persistent homology is generalized persistence.}

In this section we present a categorical approach to multi-radius persistence as a generalized persistence module following~\cite{BdSS}. In particular, we will revisit the stability results from the last section using this categorical framework.

A persistence module is a a collection of vector spaces $V_i$ and linear maps between them $V_i\to V_{i+1}$. When they were introduced by Zomorodian and Carlsson~\cite{Zom:05}, the indexing of the vector spaces was taken to be the natural numbers. Later, Chazal, et al~\cite{Cha:12} extended this definition in order to allow persistence modules to be indexed by the real numbers. Following this, Bubenik, de Silva, and Scott~\cite{BdSS} replaced the indexing set by any preordered category, defined below. We give their development of generalized persistence presently.

A \df{preordered set} is a pair $(P,\le)$, where $P$ is a set and $\le$ is reflexive and transitive. We can view a preordered set $P$ as a category $\mathbf{P}$ by taking the objects to be the elements of $P$ and by insisting that there be a unique morphism $s\to t$ whenever $s\le t$ in $P$. A \df{generalized persistence module} is a functor $\mathbf{P}\to\mathbf{D}$ from a preordered category $\mathbf{P}$ to any category $\mathbf{D}$. Usually we will think of $\mathbf{D}$ as the category $\mathbf{Vect}$ of vector spaces.

When $\mathbf{P}$ is the category $\mathbf{n}$ of the first $n$ natural numbers with its usual order, we recover the persistence modules of Zomorodian and Carlsson. When $\mathbf{P}$ is the category $\mathbf{R}$ of real numbers with its usual order, we recover the generalized persistence of Chazal, et al.

A \df{translation} on a preordered set $(P,\le)$ is a function $\Gamma: P\to P$ that is monotone and satisfies $x\le \Gamma(x)$ for all $x\in P$. A shift map induced by $\Gamma$ is the composition $F(x)\to F\Gamma(x)$ where $F:\mathbf{P}\to \mathbf{D}$ is a functor. Let $F$ and $G$ be persistence modules; i.e. $F$ and $G$ are functors $\mathbf{P}\to\mathbf{D}$. A \df{natural transformation} $\varphi:F\implies G$ is a choice of $\mathbf{D}$-morphism $\varphi_x:F(x)\to G(x)$ for each object $x$ of $\mathbf{P}$ such that the following diagram commutes for every $\mathbf{P}$-morphism $\alpha:x\to y$.

\[
\begin{tikzcd}
F(x) \arrow{r}{F(\alpha)} \arrow{d}{\varphi_x}  & F(y) \arrow{d}{\varphi_y}\\
G(x) \arrow{r}{G(\alpha)} & G(y)
\end{tikzcd}\]

Let $\Gamma$ and $K$ be translations on $\mathbf{P}$ and suppose that $F$ and $G$ are functors $\mathbf{P}\to\mathbf{D}$. Then, a $(\Gamma,K)$-{\em interleaving between $F$ and $G$} consists of natural transformations $\varphi:F\implies G\Gamma$ and $\psi:G\implies FK$
\[
\begin{tikzcd}
\mathbf{P} \arrow[r, "\Gamma" above] \arrow[d, "F" left] \arrow[dr, phantom, "\implies" below, "\varphi" above]
    & \mathbf{P} \arrow[d, "G" left] \arrow[r, "K" above] \arrow[dr, phantom, "\implies" below, "\psi" above] & \mathbf{P} \arrow[d, "F" right] \\
  \mathbf{D} \arrow[r, equal]
&\mathbf{D} \arrow[r, equal] & \mathbf{D}\\
\end{tikzcd}\]
such that $(\psi\Gamma)\phi=F\eta_{K\Gamma}$ and $(\varphi K)\psi=G\eta_{\Gamma K}$.

A \df{Lawvere metric space} is a pair $(X,d)$ where $d \colon X \times X \to [0, \infty]$ satisfies $d(x,x) = 0$ and $d(x,z) \leq d(x,y) + d(y,z)$ for any $x,y,z \in X$. Let $\mathbf{Trans}_P$ be the set of translations of a preordered set $P$. A \df{sublinear projection} is a function ${\varsigma \colon \mathbf{Trans}_{P} \to [0, \infty]}$ satisfying ${\varsigma\left(\mathrm{id}_P\right) = 0}$ and $\varsigma \left(\Gamma_0 \circ \Gamma_1 \right) \leq \varsigma \left(\Gamma_0 \right) + \varsigma \left(\Gamma_1 \right)$ for any $\Gamma_0, \Gamma_1 \in \mathbf{Trans}_P$. It is easy to see that $\varsigma\left(\Gamma\right) \coloneqq \sup_{p \in P}\left\{d(p,\Gamma(p)\right\}$ is a sublinear projection for any preordered Lawvere metric space $P$.

We say $\Gamma \in \mathbf{Trans}_P$ is an \df{$\varepsilon$-translation} provided $0 \leq \varsigma\left(\Gamma\right) \leq \varepsilon < \infty$. Generalized persistence modules $F, G \colon \mathbf{P} \to \mathbf{D}$ are \df{$\varepsilon$-interleaved with respect to $\varsigma$} provided $F$ and $G$ are $\left(\Gamma, K \right)$-interleaved for some $\varepsilon$-translations $\Gamma, K \in \mathbf{Trans}_P$. Now we are ready to define an appropriate metric for generalized persistence modules. With $\mathcal{I}_{F,G}$ being the collection of all $\varepsilon \geq 0$ for which $F$ and $G$ are $\varepsilon$-interleaved, the \df{interleaving distance} $d_\mathcal{I}$ is defined by 
\[ d_\mathcal{I} =
  \begin{cases}
    \inf \mathcal{I}_{F,G}       & \quad \text{if } \mathcal{I}_{F,G} \neq \emptyset\\
    \infty  & \quad \text{otherwise}\\
  \end{cases}.
\]


The interleaving distance is an extended pseudometric on generalized persistence modules and it acts as a terminal object in the category of stable metrics on persistence modules \cite{BdSS, Les:11}. In other words, the interleaving distance bounds all other stable pseudometrics. Repeating from Bubenik, de Silva, and Scott \cite{BdSS}, the interleaving distance is stable on generalized persistence modules.
\begin{theorem}\label{interstability}\cite{BdSS}
	If $F,G \colon \mathbf{P} \to \mathbf{D}$ are generalized persistence modules and $H \colon \mathbf{D} \to \mathbf{E}$ is an arbitrary functor, then $d_\mathcal{I}\left(\left(H \circ F\right),\left(H \circ G\right)\right) \leq d_\mathcal{I}\left(F,G\right).$
\end{theorem}

We will close this section by characterizing the stability of $d_\mathcal{I}$ in terms of filtrations of our weighted \v{C}ech and Rips complexes. In this context, it is helpful to think of a \df{generalized radius function} $\mathbf{r}:X\to  \{\phi \colon \mathbf{P} \to [0, \infty)\}$ satisfying $\mathbf{r}_x(p) \leq \mathbf{r}_x(q)$ if and only if $p \leq q \in \mathbf{P}$. A $\mathbf{P}$-filtration of a weighted complex is a functor of the form $\mathbf{P} \to \mathbf{Simp}$, where $\mathbf{Simp}$ is the category of simplicial complexes. Thus, the homology of a filtered weighted complex is a generalized persistence module of the form $\mathbf{P} \to \mathbf{Simp} \to \mathbf{Vect}$. The stability of generalized persistence of weighted complexes follows immediately from \textbf{Theorem} \ref{interstability}. For convenience, We phrase the corollary in accordance to our experimental results below.
\begin{corollary}
	Suppose $K_*$ and $L_*$ are $\mathbb{R}^N$-filtrations of weighted complexes, where $R^N$ has the standard euclidean metric and is preordered by the product order. Then $d_\mathcal{I}\left(H(K_*),H(L_*)\right) \leq d_\mathcal{I}\left(K_*,L_*\right)$.
\end{corollary}
\end{comment}

\begin{comment}

\section{Experimental results}
\cliff{Follow up on approximability of coverage problem.}

Description of problems: coverage and outlier.

Descriptions of methods.

Proof of concept.
\end{comment}

\begin{comment}
\section{Coverage}

We now describe two variants of an optimization problem for which experimental evidence shows that a multi-radius approach is far superior to a uniform radius approach.

The optimization problem considered is the {\em Min-cost Multi-cover} problem (or MCMC problem) with parameters $k \geq 1$ and $\alpha >0$: given a subset of the plane, $X \subset {\mathbb R}^2$, and another set of points in the plane, $Y = \{y_1, \ldots, y_m\} \subset {\mathbb R}^2$, find an assignment of radii $r = \{r_1, \ldots, r_m\}$ to the points in $Y$ such that the set of balls ${\mathcal B} = \{{\overline B}(x_i,r_i) : i=1, \ldots m\}$ is a {\em $k$-cover of $X$}, i.e. every point $x \in X$ lies in at least $k$ of the balls in ${\mathcal B}$, and such that the {\em cost} of ${\mathcal B}$,  $\sum_{i=1}^m  r_i^\alpha$, is minimized.  When $X$ is restricted to be a discrete point set, we refer to this as the {\em discrete} MCMC problem.  Note that when $\alpha = 2$, it is the sum of the areas of the balls that is to be minimized.

The first variant we consider is the {\em discrete set coverage problem}: $k=1$, $\alpha =2$, and $X$ is restricted to be a discrete point set, $X= \{x_1, \ldots, x_n\}$.   The second variant is the {\em polygon coverage problem}: $k=1$, $\alpha = 2$, and $X$ is restricted to be a polygonal region $X = P$.  It is well-known that the power a sensor (or broadcaster) working in the electromagnetic spectrum needs in order to sense (or broadcast signals to) objects within radius $r$ of its location is proportional to $r^2$ [reference]. 

Thus both of these problems may be viewed as important and practical power allocation problems: allocate power to a network of sensors (or broadcasters) fixed at the locations $Y$ in the plane so that they cover a certain region $X$ in the plane and so that the power used, which is proportional to $\sum_{i=1}^m  r_i^2$, is minimized.

The MCMC problem has been extensively studied [10+ references].  Since exact solutions are generally unknown, algorithms with provable {\em $\kappa$-approximation ratios} are of great interest, i.e. algorithms that are guaranteed to find radius assignments with cost less than $\kappa$ times the optimal cost for some $\kappa > 1$.  Lev-Tov and Peleg found a single polynomial time $4$-approximation algorithm for $\alpha=1$ (that also solves the problem optimally on the line) and gave a polynomial time approximation scheme (PTAS) for this problem [reference].  Bilo et al. showed that the discrete MCMC problem is NP-hard for any $\alpha \geq 2$. Chuzhoy gave a polynomial time $9$-approximation algorithm for what we call the {\em discrete set coverage problem} (the discrete MCMC problem, with $k=1$ and $\alpha=2$) [reference].  Also for the case $\alpha = 2$, Abu-Affash et.al gave a $23.02 + 63.95(k-1)$ approximation algorithm for the discrete $k$-MCMC problem and a $63.94 + 177.64(k-1)$ approximation algorithm for the polygonal $k$-MCMC problem.  Their algorithm reduces the polygonal $k$-MCMC problem to the discrete case: using any $\kappa$-approximation algorithm for the discrete $k$-MCMC problem, their algorithm's approximation ratio for the polygonal case is $(25/9) \kappa$.  Thus, using Chuzhoy's algorithm, they give a $25$-approximation algorithm for what we call the {\em polygon coverage} problem (the MCMC problem when $X$ is a polygon, $k=1$, and $\alpha=2$.)

In persistent homology, the structure of a point set $Y$ is deduced from the dynamic topology of a set of growing balls of equal radii centered at the points in $Y$.  We study the corresponding uniform radius approach to the discrete set and polygon cover problems, i.e. grow equal radius balls about the points in $Y$ until $X$ is covered.  We show that competing multi-radius approaches significantly out-perform this algorithm.  For the discrete set coverage problem we compare the cost performance of this uniform radius algorithm to that of Chuzhoy's multi-radius algorithm and our own multi-radius Voronoi-cell algorithm.  We show that, experimentally, Chuzhoy's multi-radius algorithm significantly beats the uniform radius algorithm and that the Voronoi algorithm significantly beats Chuzhoy's algorithm.  For the polygon coverage problem we show that Abu-Affash et. al.s $25$-approximation algorithm using Chuzhoy's algorithm as a subroutine significantly greatly outperforms the same algorithm using the uniform discrete set algorithm as a subroutine and that our own multiradius algorithm significantly beats both in practice.

++++

Given a set $X$, consider the time $t$ at which the balls centered at $Y$ first cover $X$.  This time $t$ may be considered as a parameter that measures how good $Y$ is as a discrete approximation of $X$. We propose the minimum cost of a $1$-cover 

\end{comment}


\section{MNIST eights recognition}


In this section, we give an application of weighted persistence to a simple computer vision problem. We apply our methods to the Modified National Institute of Standards and Technology (MNIST) data set of handwritten digits. We should emphasize that this application is simply a proof of concept; our methods to detect the handwritten number 8 fall well short of state-of-the-art methods~\cite{Cir:12}.

The MNIST dataset consists of handwritten digits (0 through 9) translated into pixel information. Each data point contains a label and 784 other values ranging from 0 to 255 that correspond to a 28 by 28 grid of pixels. The values 0 through 255 correspond to the intensity of the pixels in gray-scale with 0 meaning completely black and 255 meaning completely white. Considering the digits from zero through nine, unweighted persistence would easily be able to classify these numbers as having zero, one, or two holes, provided they are written precisely; however, real handwritten digits present a challenge.  Consider an eight as in Figure~\ref{Crazy8}. Unweighted persistence would pick up on two holes, but one of those holes might be slightly too small and ultimately considered insignificant, see Figure~\ref{UnweightedBar}. Our methods are able to pick up on both holes and would count them as significant, see Figure~\ref{WeightedBar}. We chose to work with the digit eight due to its unique homology.

\begin{figure}[t]
\begin{center}
\includegraphics[height = 3cm]{Rplot}
\caption{An eight converted to a 28 by 28 grid of pixels}\label{Crazy8}
\end{center}
\end{figure}

To begin, we convert each 28 by 28 to a set of points in the plane. We treat the location of a value in the matrix as a location in the plane. That is, the value in the $i$th row, $j$th column corresponds to the point $(i,j)$. The weight on each point is exactly its corresponding pixel intensity. Using this set of points and corresponding weights we calculate persistent homology via weighted Rips complexes. We test this methods performance against the unweighted case where all nonzero pixel values have the uniform weight of 1, again we calculate persistence in this case via Rips complexes.
%We then use Javaplex to compute the persistent homology for weighted persistence using a weighted distance matrix who  and the analogous matrix for unweighted persistence. 

We compare weighted persistence to unweighted persistence by measuring accuracy of classifying eights. Notice in the barcodes that the deciding factor in determining an eight is the ability to distinguish the length of the second longest bar from the length of the third longest and smaller bars. For this reason, we consider the ratio of the third longest bar to the second longest bar. We will say (arbitrarily) that a barcode represents an eight if this ratio is less than $\frac{1}{2}$. For each of the 42,000 handwritten digits in the MNIST data set, we compute both weighted and unweighted persistence and collect the predictions. We obtain the confusion matrices as in Figure~\ref{conf-mat}.



\begin{figure}
\begin{center}
\includegraphics[height = 5cm]{Unweigteh_Bar}
\caption{Weighted persistence on the image from Figure 1 produces a barcode that clearly has two long bars in dimension 1.}\label{WeightedBar}
\end{center}
\end{figure}



\begin{figure}
\begin{center}
%\includegraphics[height = 5cm]{RegularPersistence}
\includegraphics[height = 2.5cm]{RegularPersistence-1.png}%RegularPersistence}
\caption{Unweighted persistence on the image from Figure 1 produces a barcode that has one long bar (in $1$-homology). The second longest bar is hard to distinguish (in length) from the rest.}\label{UnweightedBar}
\end{center}
\end{figure}

\begin{figure}
\begin{center}%\footnotesize
    \begin{tabular}{|c|c|c||c|c|}\hline
    & \multicolumn{2}{c||}{Weighted Persistence} & \multicolumn{2}{c|}{Unweighted Persistence}\\ \hline
    & Predicted not $8$ & Predicted $8$ & Predicted not $8$ & Predicted $8$\\ \hline
    Not $8$ & 36487 & 1450 & 35869 & 2068 \\ \hline
    Is $8$ & 633 & 3430 & 1261 & 2802\\ \hline
\end{tabular}\caption{The confusion matrices show that weighted persistence outperforms its unweighted counterpart.}\label{conf-mat}
\end{center}

\end{figure}

\begin{comment}
\begin{table}[]
\centering
\caption{The confusion matrix for weighted persistence}
\label{Wt_conf}
\begin{tabular}{lccc}
                               & \multicolumn{1}{l}{}   & \multicolumn{1}{l}{Predicted} & \multicolumn{1}{l}{}      \\ \cline{2-4} 
\multicolumn{1}{l|}{}          & \multicolumn{1}{c|}{}  & \multicolumn{1}{c|}{0}        & \multicolumn{1}{c|}{1}    \\ \cline{2-4} 
\multicolumn{1}{c|}{Reference} & \multicolumn{1}{c|}{0} & \multicolumn{1}{c|}{36487}    & \multicolumn{1}{c|}{1450} \\ \cline{2-4} 
\multicolumn{1}{l|}{}          & \multicolumn{1}{c|}{1} & \multicolumn{1}{c|}{633}      & \multicolumn{1}{c|}{3430} \\ \cline{2-4} 
\end{tabular}
\end{table}


\begin{table}[]
\centering
\caption{The confusion matrix for unweighted persistence}
\label{UnWt_conf}
\begin{tabular}{lccc}
                               & \multicolumn{1}{l}{}   & \multicolumn{1}{l}{Predicted} & \multicolumn{1}{l}{}      \\ \cline{2-4} 
\multicolumn{1}{l|}{}          & \multicolumn{1}{c|}{}  & \multicolumn{1}{c|}{0}        & \multicolumn{1}{c|}{1}    \\ \cline{2-4} 
\multicolumn{1}{c|}{Reference} & \multicolumn{1}{c|}{0} & \multicolumn{1}{c|}{35869}    & \multicolumn{1}{c|}{2068} \\ \cline{2-4} 
\multicolumn{1}{l|}{}          & \multicolumn{1}{c|}{1} & \multicolumn{1}{c|}{1261}     & \multicolumn{1}{c|}{2802} \\ \cline{2-4} 
\end{tabular}
\end{table}
\end{comment}

 % tables \ref{Wt_conf} and \ref{UnWt_conf}. 
Notice that the weighted persistence has an accuracy rate of 95.8\% whereas unweighted persistence had an accuracy of 92.07\%. A full summary can be seen in Figure \ref{summarytable}.  We view this result as promising for potential future applications of weighted persistence. 

\begin{figure}[ht]
\begin{center}
\begin{tabular}{|c|c||c|}\hline
& Weighted Persistence & Unweighted Persistence \\ \hline
Accuracy & 0.9504 & 0.9207\\ \hline
Sensitivity & 0.9618 & 0.9455\\ \hline
Specificity & 0.8442 & 0.6896\\ \hline

Pos. Pred. Value & 0.9829 & 0.966\\ \hline

Neg. Pred. Value & 0.7029 & 0.5754\\ \hline

Prevalence & 0.9033 & 0.9033\\ \hline

Balanced Accuracy & 0.903 & 0.8176 \\ \hline

    
\end{tabular}
\caption{Weighted and unweighted persistence compared.}\label{summarytable}
\end{center}
\end{figure}
\section{Concluding remarks and open questions} 
The method of weighted persistence satisfies the appropriate Vietoris-Rips Lemma, is stable under small perturbations of the points, or the weights, or both, and can be successfully applied to data such as the MNIST data set to improve upon usual persistence. Furthermore, it is just as easy to calculate weighted persistence for balls growing at linear rates, as it is to calculate regular persistence. We conclude the paper with some further observations and questions.

One can imagine weighted persistence as interpolating between two extreme approaches to a data set that is partitioned into data $D$ and noise $N$. More precisely, we consider a noisy data set $X$. Various methods exist to filter $X$ into data $D$ and noise $N$. Traditional persistence can be applied to $D\cup N$ in two ways. We can either assign the same radius to every point of $D\cup N$ or we can throw the points of $N$ out entirely and compute persistence on $D$ alone. Using weighted persistence, we can assign the radius $0$ to each point of $N$ and compute weighted persistence of $D\cup N$. It is easy to see that this will differ from persistence of $D$ itself only in dimension $0$. By gradually increasing the $N$-radii from $0$ to $1$, our stability results can be interpreted as producing a continuum of barcodes/persistence diagrams that interpolate between the usual persistence applied to $D$ and the usual persistence applied to $D\cup N$ (in dimensions above $0$), see~\cite{Austin}.

As mentioned in the introduction, weighted persistence fits into the framework of generalized persistence in the sense of \cite{BdSS}. This direction was explored in detail in \cite{Josh}.

Finally, it would be interesting to apply weighted persistence to the MNIST data set to determine its effectiveness in distinguishing the $1$-homology of the other nine digits. One complication is that the number $4$ presents an interesting challenge since it is appropriate to write it both as a simply connected space and as a space with non-trivial $H_1$. Distinguishing $1$-homology creates 3 clusters of digits from which we could use other machine learning techniques to create an ensemble and make accurate predictions.
\bibliography{references}
\end{document}
