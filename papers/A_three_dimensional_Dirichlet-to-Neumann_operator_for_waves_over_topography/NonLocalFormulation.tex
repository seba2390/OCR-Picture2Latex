 %%%%%%%%%%%%%%%%%%
%%%%%%%%%%%%%%%%%%
\section{Variable-depth non-local formulation}
%%%%%%%%%%%%%%%%%%
%%%%%%%%%%%%%%%%%%

In order to study the elliptic boundary value problem (\ref{S2:Eq02})-(\ref{S2:Eq04}) we revisit an alternative non-local formulation of the problem which captures the effects of the bottom topography through the vertical velocity of the fluid, at the linearized free surface, $(z=0)$. As mentioned in the introduction, this non-local formulation was proposed in the work of \citet{Paul1998}, and more recently by \citet{CSNG} and \citet{AFM}.
%In the work of \citet{OliverasDeconinck}, the AFM formulation was used to study the stability of periodic traveling waves of the full non-linear equations. 
%A similar formulation was derived and used by \citet{AblowitzHaut} for the case of solitary (interfacial) waves.
The work of \citet{VasanDeconinck} was one of the first articles in which the AFM formulation was used with 
bathymetry variations. The authors studied the possibility of reconstructing the sea-bed topography  
based on surface-wave measurements.

Our version of the AFM$^\ast$ formulation is linear and in three dimensions, as opposed to that of \citet{VasanDeconinck}  
which was nonlinear and in two dimensions. Our goal is to take one step forward by 
modeling complex (not necessarily smooth) topographies that involve large and rapid variations and, as mentioned, 
perform simulations with fully three dimensional configurations. 
This is the first time that the non-local method is used under these conditions.

%%%%%%%%%%%%%%%%%%
%\subsection{Derivation of the variable-depth formulation}
%%%%%%%%%%%%%%%%%%

The non-local variable-depth formulation arises from the following representation formula, for the solution $\phi(\bold{x},z)$ to the elliptic boundary value problem (\ref{S2:Eq02})-(\ref{S2:Eq04}):
\begin{equation}\label{S3:Eq01}
\phi(\bold{x},z) = \hat{q}(0) + \sum_{\ka\in\Lambda^*}e^{i\ka\cdot\x}\left[\hat{q}(\ka)\frac{\cosh(\mu k(z+1))}{\cosh(\mu k)} + X_\ka\frac{\sinh(\mu kz))}{k\cosh^2(\mu k)}\right].
\end{equation}
%The right hand side of (\ref{S3:Eq01}) represents a superposition of solutions to the elliptic equation (\ref{S2:Eq02}), where the Dirichlet condition, (\ref{S2:Eq03}) also holds.
 In this representation formula (\ref{S3:Eq01}), the complex coefficients $X_\ka$ have to be determined from 
 the bottom boundary condition (\ref{S2:Eq04}). 
 Notice that if $X_\ka$ vanished for all $\ka$, then we would recover (\ref{S2:Eq06}), the solution of the flat bottom case. 
 Therefore $X_\ka$ contains  information related to the geometry of the topography. We will call it the topographic coefficients.

The impermeability boundary condition (\ref{S2:Eq04}) yields the following linear relation between the 
topographic coefficients and the (given) Dirichlet data $q$:

\begin{equation}\label{S3:Eq02}
\sum_{\ka\in\Lambda^*} \hat{q}(\ka)\nabla \cdot \left[ e^{i\ka\cdot\bold{x}} \frac{\sinh(\mu k H(\bold{x}))}{\cosh(\mu k)}\frac{\ka}{k}\right] = \sum_{\ka\in\Lambda^*} X_\ka\nabla\cdot \left[ e^{i\ka\cdot\bold{x}} \frac{\cosh(\mu k(1+H(\bold{x})))}{\cosh^2(\mu k)}\frac{\ka}{k^2}\right].
\end{equation}
In this equation $\nabla = (\partial_x,\partial_y)$. Equation (\ref{S3:Eq02}) couples the topographic 
coefficients $X_{\ka}$ with the Dirichlet data, where  the topography profile $H(\x)$ appears in the relation.

%%%%%%%%%%%%%%%%%%
%\subsubsection{Characterization of the Dirichlet-to-Neumann operator}
%%%%%%%%%%%%%%%%%%
Assuming that the coefficients  $X_\ka$ have been computed by means of equation (\ref{S3:Eq02}), we can use the representation formula (\ref{S3:Eq01}) to compute the vertical derivative of the velocity potential at $z=0$, 
thus characterizing the Dirichlet-to-Neumann operator acting on $q$:
\begin{equation}\label{S3:Eq03}
G[q] = \phi_z(\x,0) = \sum_{\ka\in\Lambda^*}e^{i\ka\cdot\x}\left[\hat{q}(\ka)\mu k\tanh(\mu k) + X_\ka\mu\sech^2(\mu k)\right].
\end{equation}
Notice that the presence of a non trivial topography $H$ yields an extra term in the Fourier representation of the operator $G$, 
when we compare (\ref{S3:Eq03}) with (\ref{S2:Eq11}).  We have still to describe how to compute the topographic coefficients
 $X_\ka$.

%%%%%%%%%%%%%%%%%%
%\subsection{Analysis of the non-local relation (\ref{S3:Eq02})}
%%%%%%%%%%%%%%%%%%

We define the following linear operators, acting on the space of bounded sequences $\ell^{\infty}(\Lambda^*)$ as follows: for any $f_\ka\in\ell^{\infty}(\Lambda^*)$, 
\begin{eqnarray} 
A[f_\ka] &=& \sum_{\ka\in\Lambda^*} f_\ka\nabla \cdot \left[ e^{i\ka\cdot\bold{x}} \frac{\sinh(\mu k H(\bold{x}))}{\cosh(\mu k)}\frac{\ka}{k}\right], \label{S3:Eq04}\\ 
B[f_\ka] &=& \sum_{\ka\in\Lambda^*} f_\ka\nabla\cdot \left[ e^{i\ka\cdot\bold{x}} \frac{\cosh(\mu k(1+H(\bold{x})))}{\cosh^2(\mu k)}\frac{\ka}{k^2}\right].\label{S3:Eq05}
\end{eqnarray}
By introducing these linear operators, equation (\ref{S3:Eq02}) becomes
\begin{equation}\label{S3:Eq06}
B[X_\ka]=A[\hat{q}(\ka)].
\end{equation}
As a consistency test we compute these operators in the flat bottom case $H = 0$. 
Then  for any bounded sequence of complex numbers $f_\ka$, $A[f_\ka]$ vanishes because of the $\sinh$ function in the numerator, whereas for the operator $B$ we have 
\begin{equation}
B[f_\ka] = \sum_{\ka\in\Lambda^*} f_\ka\nabla\cdot \left[ e^{i\ka\cdot\bold{x}} \frac{\cosh(\mu k)}{\cosh^2(\mu k)}\frac{\ka}{k^2}\right] = \sum_{\ka\in\Lambda^*}e^{i\ka\cdot\bold{x}}\sech(\mu k)f_\ka.
\end{equation}
So in the case $H=0$, equation (\ref{S3:Eq06}) simplifies to
\begin{equation}
B[X_\ka] = \sum_{\ka\in\Lambda^*}e^{i\ka\cdot\bold{x}}\sech(\mu k)X_\ka = 0.
\end{equation}
This is only possible if the coefficients $X_\ka$ vanish because of the orthogonality of the complex exponential functions. Thus the operator given in (\ref{S3:Eq03}) reduces to the classical flat bottom operator $G$ in (\ref{S2:Eq11}).
\begin{comment}
\begin{equation}
G[q] = \phi_z(\x,0) = \sum_{\ka\in\Lambda^*}e^{i\ka\cdot\x}\hat{q}(\ka)\mu k\tanh(\mu k).
\end{equation}
\end{comment}

What happens when the depth is constant ($H=h_0$) but different from the reference depth? 
We show that the operator defined through expressions (\ref{S3:Eq02}) and (\ref{S3:Eq03}) is the DtN 
operator corresponding to the constant depth $1+h_0$. This gives us an idea for the vertical flow structure over
a locally flat region.

Assuming that $H = h_0$ is constant, and $f_\ka$ is a bounded sequence of complex numbers,
\begin{equation}
A[f_\ka] = \sum_{\ka\in\Lambda^*} f_\ka\nabla \cdot \left[ e^{i\ka\cdot\bold{x}} \frac{\sinh(\mu k h_0)}{\cosh(\mu k)}\frac{\ka}{k}\right] = \sum_{\ka\in\Lambda^*} e^{i\ka\cdot\bold{x}} k\frac{\sinh(\mu k h_0)}{\cosh(\mu k)} f_\ka.
\end{equation}
On the other hand we have that
\begin{equation}
B[f_\ka] = \sum_{\ka\in\Lambda^*} f_\ka\nabla \cdot \left[ e^{i\ka\cdot\bold{x}} \frac{\cosh(\mu k (1+h_0))}{\cosh^2(\mu k)}\frac{\ka}{k^2}\right] = \sum_{\ka\in\Lambda^*} e^{i\ka\cdot\bold{x}} \frac{\cosh(\mu k (1+h_0))}{\cosh^2(\mu k)} f_\ka.
\end{equation}
Equation (\ref{S3:Eq06}) now yields the following relation between the known Fourier coefficients $\hat{q}(\ka)$ and the unknown coefficients $X_\ka$:
\begin{equation}
B[X_\ka] = \sum_{\ka\in\Lambda^*} e^{i\ka\cdot\bold{x}} \frac{\cosh(\mu k (1+h_0))}{\cosh^2(\mu k)} X_\ka = \sum_{\ka\in\Lambda^*} e^{i\ka\cdot\bold{x}} k\frac{\sinh(\mu k h_0)}{\cosh(\mu k)} \hat{q}(\ka) = A[\hat{q}(\ka)].
\end{equation}
This equation determines $X_\ka$ uniquely in terms of $\hat{q}(\ka)$ because of the orthogonality of the basis functions. The solution is:
\begin{equation}
X_\ka = \hat{q}(\ka)k\cosh^2(\mu k)\left(\tanh(\mu k (1+h0))-\tanh(\mu k )\right).
\end{equation}
Finally equation (\ref{S3:Eq03}) yields the consistent result that
\begin{equation}
G[q] = \sum_{\ka\in\Lambda^*}\hat{q}(\ka)\mu k\tanh(\mu k (1+h0)).
\end{equation}


%%%%%%%%%%%%%%%%%%%%%%%%%%%%%%%%%
%%%%%%%%%%%%%%%%%%%%%%%%%%%%%%%%%
\begin{comment}
We have the following lemma concerning the range of the operators $A$ and $B$. 

\begin{lemma}\label{lemma1}
Let $H$ be such that there exists positive constants $c$ and $d$, with $c<1$, such that $\sup_{\x}|H(\x)| \leq c$ and $\max\{\sup_{\x}|H_x(\x)|,\sup_{\x}|H_y(\x)|\} \leq d$. Then for any sequence $f_\ka\in\ell^{\infty}(\Lambda^*)$ $A[f_\ka]$ and $B[f_\ka]$,
given by the infinite series  (\ref{S3:Eq04})-(\ref{S3:Eq05}), 
represent continuous and doubly periodic functions of $\x$.
\end{lemma}
\begin{proof}
We prove the result for the operator $B$ first. 
As the sequence $f_\ka$ is bounded, there must be a positive constant $M$ such that
\begin{equation}
|f_\ka| \leq M,
\end{equation}
for every $\ka\in \Lambda$.
We begin by computing the absolute value of the divergence of the expression in brackets:
\begin{equation}
\begin{aligned}\label{S3:Eq15}
&\left|f_\ka\nabla \cdot \left[ e^{i\ka\cdot\bold{x}} \frac{\cosh(\mu k(1+H(\bold{x})))}{\cosh^2(\mu k)}\frac{\ka}{k^2}\right]\right| \\ \leq &M\left|e^{i\ka\x}\left[i\frac{\cosh(\mu k(1+H(\bold{x})))}{\cosh^2(\mu k)}+\mu(k_1H_x(\x)+k_2H_y(\x))\frac{\sinh(\mu k(1+H(\bold{x})))}{k\cosh^2(\mu k)}\right]\right| \\ \leq
& M\frac{\cosh(\mu k(1+H(x)))}{\cosh^2(\mu k)}+2M\mu d\frac{\left|\sinh(\mu k(1+H(\bold{x})))\right|}{\cosh^2(\mu k)}\\ \leq
&M(1+2\mu d)\frac{\cosh(\mu k(1+H(\bold{x})))}{\cosh^2(\mu k)} \leq M(1+2\mu d)\frac{\cosh(\mu k(1+c))}{\cosh^2(\mu k)}.
\end{aligned}
\end{equation}
As $0<c<1$ the right hand side of (\ref{S3:Eq15}) decays exponentially as $k$ goes to infinity. 
Hence the following series of positive real numbers converges:
\begin{equation}
\sum_{\ka\in\Lambda^*}M(1+2\mu d)\frac{\cosh(\mu k(1+c))}{\cosh^2(\mu k)} < \infty.
\end{equation}
Thus by the Weierstrass M-test the series of functions defined by 
\begin{equation}\label{S3:Eq20}
\sum_{\ka\in\Lambda^*} f_\ka\nabla \cdot \left[ e^{i\ka\cdot\bold{x}} \frac{\cosh(\mu k(1+H(\bold{x})))}{\cosh^2(\mu k)}\frac{\ka}{k^2}\right],
\end{equation}
converges uniformly to a continuous function. Because each of the functions inside the sum are periodic, 
the limiting functions are also periodic, as stated. The proof that $A[f_\ka]$ is a periodic continuous function is entirely analogous.
\end{proof}
As a corollary of this lemma we have the following important consequence that the operator $B:\ell^2(\Lambda^*)\longrightarrow L^2(T^2)$ is a well defined compact operator.
\end{comment}
%%%%%%%%%%%%%%%%%%%%%%%%%%%%%%%%%
%%%%%%%%%%%%%%%%%%%%%%%%%%%%%%%%%


From the point of view of functional analysis the linear operator $B$ is a Hilbert-Schmidt operator whose kernel is given by: 
\begin{equation}
\nabla\cdot \left[ e^{i\ka\cdot\bold{x}} \frac{\cosh(\mu k(1+H(\bold{x})))}{\cosh^2(\mu k)}\frac{\ka}{k^2}\right].
\end{equation}
Such operators are limits of finite rank operators (\cite{DunfordSchwartz}) and therefore they are compact operators.
An important result due to \citet{CSNG}, regards the solvability of equation (\ref{S3:Eq06}): 
the operator $B$ is always invertible and thus, in principle, it is always possible to solve for the coefficients $X_\ka$ 
in terms of $\hat{q}(\ka)$.  
On the other hand inverting $B$, whose range in nearly finite dimensional, must be done with care 
in order to avoid ill-conditioned behaviour that may arise from the 
discrete versions of the problem. %since the inverse of an injective compact linear operator is unbounded. 
%Thus numerical  discretizations of the problem are ill-conditioned. 

Some ill-conditioned problems, that arise from this kind of operators, have been a matter of research in recent years, particularly in the work of 
\citet{OliverasDeconinck} and \citet{VasanDeconinck} through the AFM formulation. 
More recently, a very detailed numerical study was carried out by \citet{WilkeningVasan} comparing different DtN formulations (including AFM and AFM$^\ast$).

Under this perspective, the action of the Dirichlet-to-Neumann operator will be examined in the presence of
highly variable topographies. In the next section, we present accurate computational results 
for unexplored regimes of the bottom topography.   
But first, we describe our pseudo-spectral numerical method, based on a (physically motivated) Galerkin approximation
for the topographic components of the DtN operator.

%%%%%%%%%%%%%%%%%%
\subsection{The Galerkin approximation}
%%%%%%%%%%%%%%%%%%
We now introduce a Galerkin approximation for equation (\ref{S3:Eq02}). 
We have a compact linear operator
\begin{equation}
B:\ell^2(\Lambda^*) \longrightarrow L^2(T^2),
\end{equation}
together with a linear problem of the form
\begin{equation}
B[X_\ka] = Y,
\end{equation}
where $Y = A[\hat{q}(\ka)]$ is a continuous function of the space variable $\x$.

%Let $e_\ka\in\ell^2(\Lambda^*)$ be the canonical vector.% $e_\ka(\ele) = \delta_{k_1,l_1}\delta_{k_2,l_2}$, where $\delta_{k,l}$ denotes the Kronecker delta. 
Let $e_\ka\in\ell^2(\Lambda^*)$ be the canonical vector. For $M>0$, let $U_M = span\{\delta_\ka \mid k\leq M\}\subset \ell^2(\Lambda^*)$.  Let $V_\M = span\{e^{i\ka\cdot\x} \mid \ k\leq M\}$, be the corresponding linear subspace of $L^2(T^2)$. Notice that both finite dimensional vector spaces have the same dimension.

For any sequence of complex numbers $X_\ka\in U_M$, the action of the operator $B$ is given by the truncated sum:
\begin{equation}\label{S3:Eq20}
B[X_\ka] = \sum_{k\leq M}X_\ka\nabla\cdot \left[ e^{i\ka\cdot\bold{x}} \frac{\cosh(\mu k(1+H(\bold{x})))}{\cosh^2(\mu k)}\frac{\ka}{k^2}\right].
\end{equation}

Although the right hand side of (\ref{S3:Eq20}) defines a differentiable function, the spectral content of such function is unknown. 
If we consider momentarily the case where $H = 0$, then the right hand side of (\ref{S3:Eq20}) defines a band limited function. However due to variations of the bottom geometry $H=H(\x)$, the spectral content, of the otherwise band limited function, is spread over the Fourier spectrum in $L^2(T^2)$. 

According to the truncated expression  (\ref{S3:Eq20}) define $R_M \equiv B[X_\ka] - Y$ as the residual, for a 
given value of the Galerkin parameter. We will find conditions for the residual to be orthogonal to the subspace $V_M\subset L^2(T^2)$.
We first compute the inner product of the residual  with one of the basic functions $e^{i\ele\cdot\x}\in V_M$.
\begin{equation}\label{S3:Eq21}
\left(R_M,e^{i\ele\cdot\x}\right) = \left(B[X_\ka] - Y,e^{i\ele\cdot\x}\right) = \left(B[X_\ka],e^{i\ele\cdot\x}\right) -\left(Y,e^{i\ele\cdot\x}\right).
\end{equation}

A necessary and sufficient condition for the orthogonality of the reminder function is that for $e^{i\ele\cdot\x}\in V_M$, 
\begin{equation}\label{S3:Eq22}
\left(B[X_\ka],e^{i\ele\cdot\x}\right) = \left(Y,e^{i\ele\cdot\x}\right).
\end{equation}

The right hand side of (\ref{S3:Eq22}) corresponds to the Fourier coefficient of the function $Y$, denoted by $\hat{Y}(\ele)$. Then equation (\ref{S3:Eq22}) yields the following square system of linear equations:
\begin{equation}\label{S3:Eq23}
\begin{aligned}
\left(B[X_\ka],e^{i\ele\cdot\x}\right) &= \int_{T^2}e^{-i\ele\cdot\x}\sum_{k\leq M}X_\ka\nabla\cdot \left[ e^{i\ka\cdot\bold{x}} \frac{\cosh(\mu k(1+H(\bold{x})))}{\cosh^2(\mu k)}\frac{\ka}{k^2}\right]\ d\x\\ 
&= \sum_{k\leq M} i(\ele\cdot\ka) X_\ka\int_{T^2} e^{-i(\ele-\ka)\cdot\x}\left[\frac{\cosh(\mu k(1+H(\x)))}{k^2\cosh^2(\mu k)}\right]\ d\x = \hat{Y}(\ele).
\end{aligned}
\end{equation}

Let $\tilde{X}_\ka\in U_M$ be a solution to the system of linear equations (\ref{S3:Eq23}). 
%As $M$ goes to infinity the residual between $B[\tilde{X}_\ka]$ and $Y$ vanishes, because of the orthogonality of the complex exponential functions, and so we recover the solution of equation (\ref{S3:Eq02}).
%Finally in 
In order to compute the action of the Dirichlet-to-Neumann operator on a given input function $q(\x)$ we proceed as follows:
(i) the first term in equations (\ref{S3:Eq03}) is easily computed with an FFT; 
(ii) for the second term, we fix a value of the parameter $M$ and compute an approximate solution $\tilde{X}_\ka$ 
through system (\ref{S3:Eq23}). This approximate solution is zero-padded and  plugged into the second term of equation (\ref{S3:Eq03}). These steps yield an approximation for the Neumann data $G[q]$. 

As noted, the ``flat part" of the DtN operator (\ref{S3:Eq03}) is straightforward and stable numerically. We use all Fourier modes available on our  mesh. In order to find the topographic coefficients
of the DtN's  Fourier decomposition  (\ref{S3:Eq03}) the linear system (\ref{S3:Eq23})
has to be inverted. 
The dimension of the respective subspace of interest can be kept small. 

Next we present 
a physically motivated  strategy to truncate to topographic component of the DtN operator, namely keeping only the relevant modes
that do interact with the topography. This guides our choice for the Galerkin parameter $M$. 
Higher (short-wave) modes which do not interact with the topography are discarded by the projection onto the lower
dimensional subspace determined by the Galerkin method. The presence of higher modes (in a deep water
regime) would only contribute to the ill-conditioning of system  (\ref{S3:Eq23}).

%%%%%%%%%%%%%%%%%%
\subsubsection{A physically motivated choice for the  Galerkin parameter $M$.}
%%%%%%%%%%%%%%%%%%

As mentioned above, the Galerkin parameter $M$ plays an important role in the simulations.  
It can be chosen independently of the number of Fourier modes available on our mesh and it controls 
the size of the linear system to be inverted.  Therefore it has an impact on the numerical performance of the DtN operator.

The choice of the Galerkin parameter comes from  physical properties related to the water wave problem. 
Namely related to  particle trajectories underneath the surface waves. 

Lets assume momentarily that we are in the flat bottom case. The fluid particle trajectories, beneath a time harmonic wave, 
describe approximately an elliptical orbit. Let:
\begin{equation*}
\begin{aligned}
\eta(x,t) &= Ae^{ikx-\omega t},\\
\phi(x,0,t) &= c + \tfrac{iA}{\omega}e^{ikx-\omega t},
\end{aligned}
\end{equation*}
be the time harmonic free surface elevation and its corresponding velocity potential. 
Then a classical result, found in \cite{Constantin} and \cite{DeanDalrymple}, shows that at depth $z_0$
the semi-axis of the ellipse are given by:
\begin{equation*}
\begin{aligned}
\text{major semi-axis}: &\ \frac{A\cosh(\mu k(z_0+1))}{\sinh(\mu k)},\\
\text{minor semi-axis}: &\ \frac{\mu A\sinh(\mu k(z_0+1))}{\sinh(\mu k)}.
\end{aligned}
\end{equation*}
In order to choose the Galerkin parameter $M$ we set $z_0 = \inf_x H(x)$. For a 
given tolerance $\delta>0$, we let $k = k^\ast$ be the smallest wavenumber such that:
\begin{equation*}
\frac{\cosh(\mu k^\ast (z_0+1))}{\sinh(\mu k^\ast)} \leq \delta.
\end{equation*}
We set $M = k^\ast$. 
%This condition forces both the semi axis of the ellipse to be a small fraction of the original wave at the reference depth $z_0$, and therefore we can decide whether or not the wave interacts with any bottom topography below $z_0$.
Particle orbits due to higher modes ($k > k^\ast$) are very small near the topography.  
Hence these short waves do not contribute to the topographic coefficients.
It is important to notice that our physical choice of the Galerkin parameter $M$ depends only on $\mu$, $H$, and $k^\ast$. 
This procedure selects the wavenumber band whose interaction with the bottom is nontrivial and uses this smaller subspace 
in the computation of the topographical contribution to the Dirichlet-to-Neumann operator. We will exhibit 
accurate simulations in the presence of  highly nontrivial topographies.

In most of our simulations the large amplitude topography reaches half the depth. 
Therefore we have chosen $z_0 = 0.5$ in finding $M$.  
The dispersion parameter is $\mu = 0.1$  and  our tolerance for particle motion was $\delta = 10^{-5}$. 
The corresponding value of the Galerkin parameter is $M = 230.25$. 
For a periodic domain of (dimensionless) length equal to $10$, this implies that only the first $366$ 
%{\bf Andr{\'e}: $366$ significa o numero de Fourier modes, $230.25 = 2\pi 366/10$. Para as simulacoes de Bragg e homogenizacao usei $M = 250$ que corresponderia (dominio de comprimento $10$ e $\mu=0.1$) a pegar ate o modo numero $400$. }
Fourier modes are needed for constructing the topographic component of the DtN operator. 
The typical mesh size considered in our simulations,
per side of the free surface, 
ranges from $2^{9}$ to $2^{11}$. The advantage of choosing an appropriate,
physically motivated $M$ is clear.


%%%%%%%%%%%%%%%%%%
%\subsection{Benchmarking}
%%%%%%%%%%%%%%%%%%

%%%%%%%%%%%%%%%%%%
\subsection{The DtN operator through conformal mapping}
%%%%%%%%%%%%%%%%%%

In the presence of a non trivial bottom geometry, the elliptic boundary value problem (\ref{S3:Eq02}) - (\ref{S3:Eq04}) has no closed form analytical solution. 
In two dimensions it is very efficient to define the DtN operator through the conformal mapping of the variable-depth channel.
But this strategy does not apply in three dimensions. 
Nevertheless in order to benchmark our present method we will first compare two dimensional solutions with those
obtained by a conformal mapping technique  (\citet{Nachbin2003}) which we here briefly summarize. 

The cornerstone of the method is finding a conformal mapping from a uniform strip, in the $(\xi,\zeta)-$plane, onto the 
variable-depth fluid domain in the $(x,z)-$plane. In the physical domain the level-curves 
 $\xi$-constant and $\zeta$-constant generate an orthogonal curvilinear coordinate system. In this new coordinate 
 system the Laplacian remains invariant and the bottom boundary is flatten out, being a level-curve in $\zeta$. 
 In general the determination of the conformal mapping is a difficult task. However in the case of polygonal topographies 
 this is achieved with the Schwarz-Christoffel Toolbox (\citet{Driscoll2002,FokasNachbin}).

Having the conformal mapping at hand, the elliptic boundary value problem is written 
in the curvilinear coordinate system $(\xi,\zeta)$as:
\begin{eqnarray}
\mu^2\phi_{\xi\xi} + \phi_{\zeta\zeta} &=& 0, \ \ \ \text{in $-1<\zeta<0$,}\label{S3:Eq24}\\
\phi(\xi,0) &=& q(\xi),\ \ \ \text{on $\zeta = 0$.}\label{S3:Eq26}\\
\phi_\zeta &=& 0, \ \ \ \text{on $\zeta = -1$,}\label{S3:Eq25}
\end{eqnarray}

In this setting it is very easy to compute $\phi_{\zeta}(\xi,0)$ by means of formula (\ref{S2:Eq11}). 
We then proceed to compute $\phi_z(x,0)$ from $\phi_{\zeta}(\xi,0)$ using the following formula (\cite{Nachbin2003}):
\begin{equation}\label{S3:Eq27}
\phi_z(x(\xi,0),0)  = \frac{\phi_{\zeta}(\xi,0)}{M(\xi)}.
\end{equation}
The variable coefficient $M(\xi)$ is defined as
\begin{equation}\label{S3:Eq28}
M(\xi)  = z_\zeta(\xi,0),
\end{equation}
and is the square root of the Jacobian evaluated at the undisturbed free surface. Hence a smooth coefficient.
This metric coefficient is easily computed using the Schwarz-Christoffel Toolbox  as described in \citet{FokasNachbin}.

\begin{figure}
\centerline{\includegraphics[width = \textwidth]{FiguraDTNConvergenciaPaper.eps}}
\caption{Left Panel: Gaussian potential along the undisturbed free surface  ($z=0$) over a large amplitude sloping bottom
located at $(4.5\leq x\leq5.5; -1\leq z \leq -0.5)$. Central Panel: 
The respective Neumann data $\phi_z(x,0)$ computed through the DtN operator acting on the Gaussian potential ($\mu = 0.05$). The 
solid curve was obtain through equations (\ref{S3:Eq02}) and (\ref{S3:Eq03}), while the red crosses were computed with the conformal mapping technique (\ref{S3:Eq27}). Both methods used $2^{10}$ points. Right Panel (log-plot): Relative error between the Galerkin approximation and the 
conformal mapping solution,  as a function of the Galerkin parameter $M$. }
\label{Fig01}
\end{figure}

In order to compare both methods we start with the time independent example as given by (\ref{S3:Eq24})-(\ref{S3:Eq25}). 
As our Dirichlet data, we consider a Gaussian pulse located above the sloping bottom, as 
depicted at the left of figure \ref{Fig01}. The bottom topography is not smooth.
At the central panel of figure \ref{Fig01} the solution with the present nonlocal method (solid line) is in very 
good agreement with that computed through the conformal mapping method (crosses). 
At the right panel of figure \ref{Fig01} we perform a resolution study for the Galerkin approximations, as the parameter $M$ increases. 
The solid line represents the relative error between the Galerkin approximation and the solution with the conformal mapping technique. 
For example  when $M = 300$ the relative error is $0.89107\times10^{-3}$. 
