%%%%%%%%%%%%%%%%%%
\subsection{The three dimensional  DtN operator: refraction by a submerged mound}
%%%%%%%%%%%%%%%%%%

In this section we investigate the refraction of two dimensional surface waves due to bathymetric variations. 
We explore with  changes in  propagation direction as the wave  passes over a circular submerged mound. 
We will report on a special mound, known as the Luneburg lens,  where the front of an incoming plane wave 
bends in a particular fashion.

%%%%%%%%%%%%%%%%%%
\subsubsection{The Luneburg lens}
%%%%%%%%%%%%%%%%%%

We consider a particular submerged mound which, as we shall see, plays a role analogous  to that of the Luneburg lens in optics. 
The mound focuses the wave energy at a determined point in space. 
Our goal is to show that the Dirichlet-to-Neumann operator, in our two-dimensional surface model,  accurately captures 
the complex refraction pattern that arises due to the non-trivial change in depth.

The underwater mound is given by 
\begin{equation}\label{S5:Eq01}
H(r) = 
\begin{cases}
\frac{\alpha^2}{\alpha^2 + 1 - ({r}/{r_0})^2} - 1, \ \ \ \text{if $r < r_0,$}\\
0, \ \ \ \text{if $r\geq r_0$.}
\end{cases}
\end{equation}
We consider $\alpha = 0.8$, $r = \sqrt{(x-8)^2+(y-10)^2}$ and $r_0 = 4$. The water depth above the center of the mound is about 
$0.60$.
The numerical method used $2^{9}$ Fourier modes in each direction
and the Galerkin method used $M=37.69$ for the topographic component of the DtN operator. The use of a smaller $M$ is 
due to memory restrictions. For a two dimensional surface and $M=37.69$ we have $240^2$ Fourier modes representing
functions in our respective subspace for the topographic component of DtN. In order to find the contribution for 
each of these Fourier modes we need to build and invert a matrix $240^2$ by $240^2$.

%{\bf Andr{\'e}: Antes tinha $M = 120$, na verdade o 120 eh o numero de modos escolihdos, o $M$ que faz aqula escolha eh $M = 2\pi 120/20 = 37.69$. O $20$ e por que o dominio agora eh maior (antes era 10). Acho melhor apresentar assim para que fique com a mesma cara que a estimativa das orbitas e com os valores de $M$ anteriores.}

Four snapshots of the solution are presented in  figure \ref{Fig05} and figure \ref{Fig06}. In the top 
panel of figure \ref{Fig05}, corresponding to time $t = 3.1$, we see the incoming plane wave (from the left)
already displaying some dispersive effect. We have used $\mu = 0.1$.
A short oscillatory tail is apparent behind the wave front. 
The topography is in the middle of the computational domain and  no change of direction of propagation is observed.


%%%%%%%%%%%
\begin{figure}
\centerline{\includegraphics[width = \textwidth,keepaspectratio]{LuneburgWaveConv01JFM.eps}}
\caption{Snapshots  at times $t = 3.1$ and $t=8.6$.  The incoming surface elevation is propagating towards the Luneburg lens 
($\alpha = 0.8$)  in the $\mu = 0.1$ regime. An animation for the corresponding velocity potential can be found in the supplemental material of this article.}
\label{Fig05}
\end{figure}
%%%%%%%%%%%% 

In the lower panel of figure \ref{Fig05}, at time $t = 8.6$, we see a strong deformation along the wavefront  due to its interaction with the 
submerged mound. The depth in the middle is decreasing and therefore the speed at the central part of wavefront is smaller.

%%%%%%%%%%%
\begin{figure}
\centerline{\includegraphics[width = \textwidth]{LuneburgWaveConv02JFM.eps}}
\caption{
Snapshots  at times $t = 13.6$ and $t=17$.  The  surface elevation is propagating over the Luneburg lens 
($\alpha = 0.8$)  in the $\mu = 0.1$ regime. The focusing due to refraction is clearly observed.}
\label{Fig06}
\end{figure}
%%%%%%%%%%%%%%%%%%

At the top of figure \ref{Fig06} we have the solution at time $t = 13.6$. We observe the effect of the Luneburg lens: 
it focuses the wave energy at a point behind the mound. 
At this time the largest wave amplitude is observed, exactly at the focusing point as will be confirmed below. 
The wave amplitude at the focusing point is about twice as high as that of the initial incoming wave.
As the high amplitude peak moves down due to gravity  smaller waves (of the dispersive tail) will focus behind it, 
giving rise to a circular pattern that eventually dominates the wave field, at time $t = 17$ (bottom of figure \ref{Fig06}).
We tested with other shapes of the submerged mound (such as a Gaussian profiles) and of course the focusing
effect does not take place. 


%%%%%%%%%%%%%%%%%%
%\subsubsection{Concave lens}
%%%%%%%%%%%%%%%%%%


In particular if the submerged Luneburg mound becomes concave, namely a circular cavity, the wavefront bending is in the other 
direction. The water becomes deeper and the wave propagates faster over the cavity.
The submerged cavity is given by $-H$ as in (\ref{S5:Eq01}). Snapshots of the  corresponding
numerical solution are given in figure \ref{Fig07} and figure \ref{Fig08}. 
At time $t = 3.1$ we have the usual incoming plane wave as shown at the top of figure \ref{Fig05}. 
At time $t = 8.6$ we start to observe the refraction pattern, now with the wavefront moving forward at its central part. 

%%%%%%%%%%
\begin{figure}
\centerline{\includegraphics[width = \textwidth]{LuneburgWaveDiv01JFM.eps}}
\caption{
Snapshots  at times $t = 3.1$ and $t=8.6$.  The incoming surface elevation is propagating towards a concave cavity 
(with the Luneburg lens profile; $\alpha = 0.8$)  in the $\mu = 0.1$ regime. }
\label{Fig07}
\end{figure}


\begin{figure}
\centerline{\includegraphics[width = \textwidth]{LuneburgWaveDiv02JFM.eps}}
\caption{
Snapshots  at times $t = 13.6$ and $t=17$.  The surface elevation is propagating over a concave cavity 
(with the Luneburg lens profile; $\alpha = 0.8$)  in the $\mu = 0.1$ regime.}
\label{Fig08}
\end{figure}
%%%%%%%%%%%%%%%%%%%%%%%
%At time $t=13.6$ we see that the wave front has bent forward and has sped up as expected, whereas a sequence of smaller peaks appeare near the edge of the cavity following a divergence pattern. It is in this sequence of peaks where we encounter the regions of higher oscillations of the wave field. Notice that this region is not in front of the underwater cavity but it appeared towards the lateral edges.

Both simulations show that our numerical method, based on the construction of a three dimensional Dircihlet-to-Neumann 
operator, accurately captured interesting refraction patterns due to lens-like bottom topographies. In the next subsection an
approximate ray theory, for the dispersive equations, confirms the accuracy in the Luneburg lens case. 

%%%%%%%%%%%%%%%%%%
\subsubsection{Ray theory}
%%%%%%%%%%%%%%%%%%

We aim to compare the refraction pattern of the numerical wavefield agains  predictions from the 
theory of refraction for monochromatic waves. 
This is a classical subject in the geometrical optics theory for water waves.  A detailed treatment can be found in \citet{Whitham,Dingemans,Johnson}. Our model is reduced to three equations 
that provide a first order approximation for the refraction pattern of a nearly monochromatic wavetrain.
The monochromatic approximation is suitable for our purposes since we have a linear wave model. 
We consider the long wave regime, where dispersion is weak ($\mu$ is small).


The approximation begins with the following ansatz for the velocity potential: 
\begin{equation}\label{S5:Eq02}
\phi(x,y,t) = a(x,y)e^{i\left(\theta(x,y)-\omega t\right)}.
\end{equation}
The velocity potential is taken as an oscillatory wavetrain that is modulated with an amplitude function $a(x,y)$ and 
a phase function $\theta(x,y)$ .
Both functions $a$ and $\theta$ are assumed to vary on a scale faster than the depth  variations 
of the Luneburg lens.
The phase $\theta$  is determined  by the eikonal equation
\begin{eqnarray}
&&\theta_x^2 + \theta_y^2 = \sigma^2,\label{S5:Eq04}\\
&\mbox{where}&\omega^2 = \frac{\sigma}{\mu}\tanh(\mu(1+H(\x))\sigma),\label{S5:Eq03}
\end{eqnarray}
We have an $\bold{x}$-dependent dispersion relation that determines the variable wavenumber $\sigma(\bold{x})$. 
It plays the role of a variable index of refraction in the eikonal equation for $\theta$. 
%There is also an equation for the amplitude, see \citet{Dingemans,Johnson,Whitham}, but we will only study (\ref{S5:Eq03}) and (\ref{S5:Eq04})  because we are interested in the geometry of the wave field.
%We solve the Cauchy problem for the Eikonal equation with a varying $H$ that represents the Luneburg lens, namely we have a varying index of refraction. 
We use the method of characteristics to solve equation (\ref{S5:Eq04}). In this context it is customary to call the 
characteristic curves $(x,y)$ as rays. It is also known that the wavefronts (level curves in $\theta$) are orthogonal 
to the rays/characteristics (\cite{Zauderer}).

The Cauchy problem for the eikonal equation consists in determining the phase function $\theta$ from an ``initial" constant data $\theta_0$ along the $y$ axis. 
First we need to determine the function $\sigma$ from the dispersion relation (\ref{S5:Eq03}). Notice that outside the circle $r=r_0$, $H$ vanishes and therefore, in that region, the solution to (\ref{S5:Eq03}) is given by a constant $\sigma_0$. Inside the circle we solve the dispersion equation numerically by means of the Newton's method.

Once the function $\sigma$ and its derivatives are determined, 
we are ready to solve the characteristic equations given by:
\begin{equation}\label{S5:Eq05}
\begin{aligned}
\frac{dx}{d\tau} &=2p,\ \ \ \text{with $x(0,s)=0$.}\\
\frac{dy}{d\tau} &=2q,\ \ \ \text{with $y(0,s)=s$.}\\
\frac{dp}{d\tau} &= \frac{\partial\sigma^2}{\partial x},\ \ \ \text{with $p(0,s)=p_0(s)$.}\\
\frac{dq}{d\tau} &= \frac{\partial\sigma^2}{\partial y},\ \ \ \text{with $q(0,s)=q_0(s)$.}\\
\frac{d\theta}{d\tau} &= 2\sigma^2,\ \ \ \text{with $z(0,s)=\theta_0$.}
\end{aligned}
\end{equation}
In these equations we define  $(p,q) = \nabla \theta$ along the rays, which are parametrized by $\tau$. 
The complete set of  initial data must satisfy the following conditions, known as the strip conditions (\cite{Zauderer}):
\begin{equation}\label{S5:Eq06}
\begin{aligned}
&  p_0^2(s) = \sigma_0^2,\\
& q_0(s) = 0.
\end{aligned}
\end{equation}
The characteristic system (\ref{S5:Eq05}) is solved using the fourth-order Runge-Kutta method.
In figure \ref{Fig09} and figure \ref{Fig10} the black solid lines correspond to the rays $(x(\tau,s),y(\tau,s))$. 
Notice how the rays bend and focus into one point inside the lens, represented by the red circle. 
This focusing property distinguishes the Luneburg lens from other types of submerged mounds.
%%%%%%%%%%%
\begin{figure}
\includegraphics[width = 0.8\textwidth]{RaysJFM.eps}
\caption{Evolution of a Gaussian velocity potential over a Luneburg lens with parameters $\alpha = 0.8$ at time $8.6$. The black lines correspond to the rays. Notice that the wavefronts, the level lines of the velocity potential, follow the rays.
The dispersion parameter is $\mu = 0.1$}
\label{Fig09}
\end{figure} 
%%%%%%%%%%%%%%%
%Let $\theta$ be a solution to the eikonal equation. The level curves of $\theta$ are the wave fronts and they are always orthogonal to the rays. 
In figure \ref{Fig09} we also plot  some level curves of our numerical wave  solution, 
computed with the non-local DtN method. Notice that before the caustics, where the method of characteristics breaks down, 
the rays and wave fronts are nearly orthogonal, as expected.
%%%%%%%%%%%
\begin{figure}
\includegraphics[width = 0.8\textwidth]{RaysFocusJFM.eps}
\caption{Evolution of a Gaussian velocity potential over a Luneburg lens with parameters $\alpha = 0.8$, in the $\mu = 0.1$ regime, at time $13.6$ . The black lines correspond to the rays. Notice that the wave field focuses on top of the point where the rays meet.}
\label{Fig10}
\end{figure}
%%%%%%%%%%%%%%%%
Later in time, in figure \ref{Fig10}, we see the focusing of the wave field at exactly the point where the rays meet. 
The point where the rays focus through the method of characteristics is $(11.21,5)$, whereas 
the wave field, from our non-local method, attains its maximum at $(11.28,5)$. 

Our three dimensional simulations captured very well theoretical predictions. 
In the examples considered the topography had a non-trivial amplitude and our method captured remarkably well the 
predicted dynamical features.


