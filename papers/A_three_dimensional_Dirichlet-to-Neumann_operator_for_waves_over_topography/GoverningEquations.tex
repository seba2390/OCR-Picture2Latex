%%%%%%%%%%%%%%%%%%
%%%%%%%%%%%%%%%%%%
\section{Governing equations}
%%%%%%%%%%%%%%%%%%
%%%%%%%%%%%%%%%%%%


%%%%%%%%%%%%%%%%%%
\subsection{Water wave equations}
%%%%%%%%%%%%%%%%%%
Potential theory is the classical hydrodynamic model that describes the motion of an incompressible, irrotational flow under the action of gravity (\cite{Whitham}):
\begin{eqnarray}
	\phi_{xx}+\phi_{yy}+\phi_{zz}&=& 0,\ \ \text{in $-b(\x)<z<\eta(\x,t)$.}\label{S1:Eq01}\\
	\phi_{z} - \phi_{x}b_x - \phi_{y}b_y &=& 0,\ \ \text{on $z=-b(\x)$.}\label{S1:Eq02}\\
	\eta_t + \phi_x\eta_x + \phi_y\eta_y - \phi_z&=& 0 ,\ \ \text{on $z=\eta(\x,t)$.}\label{S1:Eq03}\\
	\phi_t + g\eta +\tfrac{1}{2}\left(\phi_x^2+\phi_y^2+\phi_z^2\right) &=& 0,\ \ \text{on $z=\eta(\x,t)$.}\label{S1:Eq04}
\end{eqnarray}
In these equations $\x = (x,y)$ denotes the horizontal variables, $g$ is the acceleration due to gravity, $\eta$ denotes the height of the surface wave, $\phi$ is the velocity potential of the fluid flow and finally $b$ denotes the 
impermeable bottom topography. We assume that $\eta$, $\phi$ and its derivatives vanish at infinity.

The bottom of the fluid domain is described as
\begin{equation}\label{S1:Eq05}
b(\x) = h + H(\x),
\end{equation}
where $h$ is a positive constant and the depth-variation function $H$ is bounded: 
\begin{equation}\label{S1:Eq06}
|H(\x)|<c<h, ~~c > 0.
\end{equation} 
This restriction implies that no islands nor beaches are present in the fluid domain. But it allows for high amplitude, non-smooth obstacles, as will be presented herein.

%%%%%%%%%%%%%%%%%%
\subsubsection{Non-dimensional form}
%%%%%%%%%%%%%%%%%%

It is customary to work with a non-dimensional form of the equations. This can be accomplished by replacing all the variables in equations (\ref{S1:Eq01})-(\ref{S1:Eq04}) with primes  and  making the following substitutions:

\begin{equation}\label{S1:Eq07}
\begin{aligned}
x' = lx, \ \ \ & y' = ly, \ \ \ z' = hz \ \ \ t' = \frac{l}{\sqrt{gh}}t,\\
\eta' = a\eta,\ \ \ & \phi' = a\sqrt{gh}\frac{l}{h}\phi, \ \ \ H' = hH.\\
\end{aligned}
\end{equation}
In these new variables $l$ denotes a typical length-scale for both horizontal directions, $h$ denotes a typical depth of the fluid domain, $a$ denotes a typical amplitude of the wave and $\sqrt{gh}$ is the typical speed of long waves propagating over an ocean of constant depth $h$. The following dimensionless parameters arise: $\mu = {h}/{l}$ and $\ep = {a}/{h}$.

The dimensionless equations are 
\begin{eqnarray}
\mu^2\left(\phi_{xx}+\phi_{yy}\right)+\phi_{zz} &=& 0,\ \ \ \ \text{in $-(1+H(\x))<z<\ep\eta(\x,t)$,}\label{S2:Eq08}\\
\phi_z -\mu^2(\phi_{x}H_x+\phi_{y}H_y)&=&0,\ \ \ \ \text{on $z=-(1+H(\x))$,}\label{S2:Eq09}\\
\eta_{t} + \ep(\phi_x\eta_x + \phi_y\eta_y) - \tfrac{1}{\mu^2}\phi_z&=&0,\ \ \ \ \text{on $z = \ep\eta(\x,t)$,}\label{S2:Eq10}\\
\phi_t + \eta +\tfrac{\ep}{2}\left(\phi_x^2+\phi_y^2+\phi_z^2\right) &=&0,\ \ \ \ \text{on $z = \ep\eta(\x,t)$.}\label{S2:Eq11}
\end{eqnarray}


%%%%%%%%%%%%%%%%%%
\subsection{Linear model}
%%%%%%%%%%%%%%%%%%
The potential theory equations have two main sources of difficulties, namely, the effects of the underwater topography and the nonlinear effects. Our objective is to study the effects due to the bottom topography.  Therefore
we keep the nonlinear effects aside and consider 
\begin{eqnarray}
	\mu^2(\phi_{xx}+\phi_{yy})+\phi_{zz}&=& 0,\ \ \text{in $-1-H(\x)<z<0$,}\label{S2:Eq12}\\
	\phi_{z} - \mu^2(\phi_{x}H_x + \phi_{y}H_y) &=& 0,\ \ \text{on $z=-1-H(\x)$,}\label{S2:Eq13}\\
	\eta_t  - \tfrac{1}{\mu^2}\phi_z &=& 0 ,\ \ \text{on $z=0$,}\label{S2:Eq14}\\
	\phi_t + \eta  &=& 0,\ \ \text{on $z=0$.}\label{S2:Eq15}
\end{eqnarray}
In terms of physical scales the linear dispersive system of equations is well suited for the investigation of 
small amplitude surface waves,  as for instance, a tsunami that propagates offshore. 
See \citet{ArcasSegur} for further details on this matter.


%%%%%%%%%%%%%%%%%%
\subsubsection{Time evolution equations}
%%%%%%%%%%%%%%%%%%
The three-dimensional linearized water wave equations can be put into an equivalent two-dimensional form,
restricted to the free surface,  by means of the Dirichlet-to-Neumann operator. We introduce the variable $q(x,y,t) = \phi(x,y,0,t)$ 
and denote by $G[q]$ the compatible Neumann data 
regarding the respective elliptic problem. The linearized water wave equations become:
\begin{eqnarray}
\eta_t &=& \tfrac{1}{\mu^2}G[q],\label{S1:Eq16}\\
q_t &=& -\eta.\label{S1:Eq17}
\end{eqnarray}

In the next section we define the corresponding elliptic problem and show how to compute the Dirichlet-to-Neumann operator $G$,  
which captures features of the vertical structure of the flow including the presence of  a highly
variable topography.

Once the operator $G$ is constructed for a given bottom topography $H$, we proceed to solving the evolution equations (\ref{S1:Eq16}) and (\ref{S1:Eq17}) with a fourth order Runge-Kutta method. We will obtain accurate results with this formulation for 
three dimensional flows  in the presence of 
a highly variable topography, a regime not yet explored with the DtN operator.