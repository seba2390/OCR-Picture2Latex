%%%%%%%%%%%%%%%%%%
%%%%%%%%%%%%%%%%%%
\subsection{The elliptic  boundary value problem for water waves}
%%%%%%%%%%%%%%%%%%
%%%%%%%%%%%%%%%%%%

%%%%%%%%%%%%%%%%%%
%\subsection{Elliptic boundary value problem}
%%%%%%%%%%%%%%%%%%

In this section we define the elliptic boundary value problem associated with the linear equations for gravity waves over topography. We assume that the wave dynamics takes place inside a bounded region of space. 
For the free surface we choose a sufficiently large square of sides $L$, that contains the region of interest. Without loss of generality we impose periodic boundary conditions at the ends of this region.

Under these considerations, let 
\begin{equation}\label{S2:Eq01}
D = \{(x,y,z)\in\mathbb{R}^3 \mid 0 <x< L,\ 0 <y<L ,\ \text{and\ } -1-H(\x)<z<0\},
\end{equation}
be the fluid domain and consider the following elliptic boundary value problem: 
\begin{eqnarray}
\mu^2(\phi_{xx}+\phi_{yy})+\phi_{zz}&=& 0,\ \ \text{in $D$,}\label{S2:Eq02}\\
\phi(\x,z) &=& q(\x),\ \ \text{on $z=0$,}\label{S2:Eq03}\\
\phi_{z} - \mu^2(\phi_{x}H_x + \phi_{y}H_y) &=& 0,\ \ \text{on $z=-1-H(\x)$.}\label{S2:Eq04}
\end{eqnarray}
The solution to this problem determines the dimensionless velocity potential $\phi$ inside the fluid domain $D$. The problem is linear and the domain $D$ is fixed. Therefore the solution only depends on the boundary value $q$ at any given time $t$.
Hence time will be temporarily omitted in the presentation that follows.

Regarding the solvability of the above elliptic boundary value problem, \citet{Lannes2005} proved that in the presence of a smooth bottom and appropriate decaying conditions at infinity, the problem always admits a unique smooth solution $\phi(\x,z)$.

For our evolution equations a quantity of great interest is the vertical speed of the fluid ($\phi_z$) along 
the linearized free surface $z = 0$. This term has the information of an underlying harmonic function $\phi$, satisfying a Neumann 
condition on the variable bottom boundary. 
It  is obtained through the Dirichlet-to Neumann (DtN) operator applied to the function $q$, which mathematically 
reduces  one dimension of the problem.  The respective Fourier-type DtN operator 
collapses the three dimensional dynamics onto  free surface equations. We use the notation
\begin{equation}\label{S2:Eq05}
G[q] = \phi_z,\ \ \ \text{on $z = 0$}.
\end{equation}

%%%%%%%%%%%%%%%%%%
\subsubsection{Flat bottom case}
%%%%%%%%%%%%%%%%%%
In order to motivate our strategy for the constructing the variable-depth Dirichlet-to-Neumann operator, 
we first discuss the ($H=0$) flat bottom case. 
In the flat bottom case, Fourier analysis immediately gives us an explicit formula for the solution $\phi$ of equations (\ref{S2:Eq02})-(\ref{S2:Eq04}):
\begin{equation}\label{S2:Eq06}
\phi(\x,z) = \sum_{\ka\in\Lambda}e^{i\ka\cdot\x}\hat{q}(\ka)\frac{\cosh(\mu k(z+1))}{\cosh(\mu k)},
\end{equation}
where $\ka = (k_1,k_2)$, $k=\sqrt{k_1^2+k_2^2}$. 
The Fourier coefficient of the Dirichlet data $q(\x)$ is
defined by
\begin{equation}\label{S2:Eq07}
\hat{q}(\ka) = \int_{T^2}e^{-i\ka\cdot\x}q(\x)\ d\x,
\end{equation}
where we have denoted the undisturbed free surface ($z=0$) by 
\begin{equation}\label{S2:Eq08}
T^2 = \{\x\in\mathbb{R}^2 \mid 0 <x< L,\ \text{and\ } 0 <y<L \}.
\end{equation}
Doubly periodic functions in $T^2$ have their Fourier spectrum in the set
\begin{equation}\label{S2:Eq09}
\Lambda = \tfrac{2\pi}{L}\mathbb{Z}\times\tfrac{2\pi}{L}\mathbb{Z}.
\end{equation}
We denote the set of non-zero wavenumbers by 
\begin{equation}\label{S2:Eq10}
\Lambda^* = \{ (k_1,k_2)\in\Lambda \mid k_1^2 + k^2_2 > 0\}.
\end{equation}


From (\ref{S2:Eq06}) we  compute the vertical derivative of $\phi$ along the undisturbed free surface:
\begin{equation}\label{S2:Eq11}
G[q] := \phi_z(\x,0) = \sum_{\ka\in\Lambda^*}e^{i\ka\cdot\x}\hat{q}(\ka)\mu k\tanh(\mu k)\ d\ka.
\end{equation}
Formula (\ref{S2:Eq11}) defines the Dirichlet-to-Neumann operator for channels with a flat bottom. 
