


\section{Motivation}
\label{motivation}
In this section, we overview the major application domains and requirements of WLANs, the challenges of wireless communication, and the importance of centralization in addressing the requirements of current and emerging applications.





\subsection{Applications and Requirements}
\label{app_req}
WLANs are being used in a variety of applications with various performance requirements.
WLANs installed in enterprises, campuses and public areas usually serve a small to medium number of stationary and mobile users.
For voice and video streams, in addition to high throughput, providing an uninterrupted service for mobile users is an important performance metric \cite{oliveira_characterizing_2016}.
For home WLANs, achieving these service requirements is more challenging due to the adjacency of multiple networks controlled individually \cite{COAP,ResFi}.


Industrial WLANs usually serve a larger number of nodes deployed for monitoring and control purposes.
In addition, while most of the nodes are static, mobility is characterized by the speed and movement scale of robot arms, rotating parts, and mobile robots \cite{li2017industrial}.
In these applications, the network infrastructure must satisfy a set of unique requirements \cite{luvisotto2017ultra,tramarin2016use,RTwifi}, as follows:
First, many of these applications require deterministic packet delivery delay.
This is particularly important when sensors and actuators communicate with controllers because the delay of the control loop directly affects control performance.
Second, a packet delivery reliability of 99\% or higher must be guaranteed to ensure the robust operation of control and monitoring systems.
In addition to sensors and actuators, industrial wireless networks also include high-rate devices such as cameras \cite{silvestre2011online}.
For example, it is important to ensure that the longer transmission duration and higher priority of video packets do not affect the timeliness of short control packets.
Similar to industrial applications, other mission-critical systems, such as medical monitoring, must satisfy a similar set of requirements \cite{REWIMO}.


Besides user devices such as smartphones, more energy-constraint devices like sensors and actuators are being used nowadays in various types of applications \cite{Tozlu2012,CYW43907,BCM4343,S1-energy-MAC,REWIMO}.
For example, wireless sensors and actuators used in industrial environments, specifically those installed in high temperature and hard-to-access areas, are usually battery-powered. 
Furthermore, a greater number of low-power IoT devices and wearables are being used for applications such as medical monitoring.
Therefore, energy is an important requirement of current and emerging applications of WLANs.


\subsection{Challenges}
\label{motiv_chall}
%
Wireless networks introduce a new set of network control challenges, compared to wired networks.
The broadcast nature of the wireless medium causes interference, resulting in link quality and throughput variations.
For example, an AP may under-utilize its spectrum and hardware capacity due to the high interference level caused by neighboring APs or clients \cite{DenseWLAN1,decIntf,smog}.
%Interference is caused by 802.11 equipment as well as those devices sharing the frequency band of 802.11 \cite{Dezfouli2014c} (e.g., 802.15.4 and microwave ovens). 
Interference level depends on \textit{dynamic} factors such as mobility, traffic rate, number of devices, and environmental properties (e.g., path loss, multipath effect, obstacles, etc.) \cite{Pelechrinis2011,RSSIranging,largeScaleMeas,Dezfouli2014c}.
The high path loss of wireless signals, and the inability of wireless transceivers to detect collisions, bring about new challenges such as hidden and exposed terminal problems \cite{transOpp}.

From the network control point of view, \textit{channel assignment} is a widely used mechanism to cope with interference, enhance throughput and reduce channel access delay.
Channel assignment refers to the process of assigning RF channels to APs based on various factors, such as the interference relationship between APs and clients and the traffic demand of clients.
The dynamics of wireless channel, the variability of user traffic, and variations in transmission power make channel assignment a challenging process.
This process is further complicated by the introduction of channel bonding supported by 802.11n and 802.11ac standards.
Channel bonding allows a device to combine multiple channels in order to transmit at a higher rate over a wider channel.
However, the performance benefit of this technique depends on several factors and requires collaboration among APs \cite{deek2013joint,faridi2017analysis}.
For example, using a larger bandwidth might increase contention with nearby APs and result in a longer channel access delay.

In addition to affecting interference level and the load of APs, \textit{mobility} results in connection interruption because of the delay of re-association process \cite{Odin,CloudMAC3}.
Although most of the large WLAN deployments employ a centralized user authentication scheme, the delay of association process is not completely eliminated.
In particular, when a client associates with a new AP, the AP needs to establish the data structures pertaining to the new client.
However, this process might result in communication interruption.


To increase the throughput of WLANs, extend coverage, and facilitate reliable handoff, \textit{network densification} is employed in various types of environments such as enterprises and campuses.
With this approach, the coverage range of the APs is reduced and the density of APs deployed per square meter is increased \cite{APdensity}.
Despite the potential benefits of this approach, the network dynamics caused by interference and mobility are exacerbated by densification.
For example, densification increases the number of re-associations as a client moves.
In addition, due to the limited number of channels available in the 2.4 and 5GHz bands, channel assignment becomes a more challenging process.



% As a typical enterprise WLAN includes several hundreds of APs, hardware or software updates to APs in order to address the challenges of network control would be very costly and time consuming \cite{Demand-aware-LB-Association-15,BeHop,Ethanol,DenseAP,Cai2009a}.


From the energy point of view, the 802.11 standards provide two mechanisms: power saving mode (PSM), and automatic power saving delivery (APSD).
However, the efficiency of these algorithms reduces as network density increases \cite{ting2010power,zeng2011sofa}.
For example, the concurrent wake-up of multiple clients results in collision and higher energy consumption \cite{liu2014energy}.
Furthermore, energy is wasted when a node wakes up during a bad channel condition.




% While all of the aforementioned challenges and requirements can benefit from centralized network control, it is worth noting that the interaction between these requirements further justifies the need for centralized decision making.



\subsection{Architectures and Control Mechanisms}
\label{motiv_centralization}
Central network control plays an important role to satisfy the requirements of existing and emerging applications.
First, \textit{architectures} are required to provide an infrastructure for centralized network control. 
Second, \textit{centralized control mechanisms} are necessary to make control decisions based on network-wide dynamics and requirements.
For example, utilizing centralized network control facilitates addressing the following challenges:
To offer seamless handoff, we can eliminate the need for re-association by maintaining the association information of clients at a central controller;
to enhance throughput, the controller can steer the association point of clients to balance the load of APs;
the performance of channel assignment can be improved by utilizing centralization because channel assignment is, in essence, a graph coloring problem.
In addition to client steering, seamless handoff, and channel allocation, which significantly affect throughput, delay, and energy consumption of clients, centralization enhances the adoption and performance of other techniques, such as scheduling, to further improve the aforementioned metrics \cite{zeng2011sofa,REWIMO,liu2014energy,MARS}.





% {\color{red!50!black}
% Although studies prove the higher performance of centrally-controlled wireless networks, the architectures employed by cellular networks are not applicable in WLANs.
% Cellular networks employ a dedicated control plane, which is used for control purposes, such as scheduling, QoS provisioning, power control and handoff \cite{Crowd,SDNcellular}.
% This control plane uses a significant amount of bandwidth and requires microsecond level synchronization.
% However, WLANs are meant to be simple and low cost, and such a complex control plane would not be a suitable solution \cite{FlashBack}.

% %The literature propose architectures to enable centralized and flexible control
% %\footnote{While some papers \cite{Primitives} separate network control (e.g., rate adaptation) from management (e.g., scheduling), in this paper we refer to both categories as "network control".} 
% %of WLANs in response to network dynamics and emerging applications, as reviewed in the next section.
% }





\section{Architectures}
\label{Architectures}
In this section we overview the existing SDWLAN architectures.
Before presenting these architectures, we first overview the basic concepts of software-defined networking and the standard protocols used for north and south-bound communication.
% As we will discuss later, existing architectures either rely on these standard protocols or propose proprietary protocols to address the shortcomings of standard protocols.




{\color{black!50!black}
%------------------------------------------------------------------ COLOR
\subsection{Principles of Software-Defined Networking}
\label{sdn-intro}
A software-defined network (SDN) decouples control mechanisms from data forwarding paths.
In fact, a SDN has two planes: a (i) \textit{data plane}, such as switches and APs, and a (ii) \textit{control plane}, which is a controller that runs an operating system (e.g., NOX \cite{NOX}, Floodlight \cite{Floodlight}) and mechanisms that control the network operation (e.g., AsC, ChA).
%such as association and channel assignment, routing and load balancing \cite{Ananta,feamster2014road,jain2013b4}.
%
The controller\footnote{To achieve reliability, a logical controller may represent multiple physical controllers \cite{Levin2012,SDNscalability}.} communicates with data plane equipment through a south-bound protocol (e.g., OpenFlow \cite{OpenFlow}, SNMP \cite{SNMP}), and exposes a set of north-bound APIs (e.g., REST) through which network control mechanisms are developed.
In fact, the global network view of the controller and the reprogrammability of data plane significantly simplifies the design and deployment of network control mechanisms.
Network control mechanisms are also referred to as \textit{network applications}, as they are actually applications running on the network operating system.
%


% REVISION
SDN is an enabler of \textit{network virtualization}, a.k.a., \textit{network slicing}, which refers to the abstraction, slicing and sharing of network resources \cite{wang2013network}. 
For example, an abstraction layer (e.g., FlowVisor \cite{sherwood2009flowvisor}) is used to provide applications with an isolated view of resources.
Network virtualization provides control logic isolation, as each application can see and control only the slice presented to it.
Therefore, when a SDWLAN is shared by multiple operators, network slicing enables the enforcement of multiple access policies to the users associated with various network operators.
In addition, virtual networks corresponding to various services can be established to support differentiated services and achieve higher QoS control over resources.
Virtualization also expedites the design and development of novel wireless technologies.
For example, one or multiple experiments can be run on slices of a wireless network while the production network is in operation.
Three levels of slicing are applicable to a SDWLAN: 
(i) Spectrum (a.k.a., link virtualization): requires frequency, time or space multiplexing. 
(ii) Infrastructure: the slicing of physical devices such as APs and switches. 
(iii) Network: the slicing of the network infrastructure.
%As we will see in Section \ref{Archs}, spectrum slicing is used to slice infrastructure, and infrastructure slicing is used to slice network.


Given the simplified configuration of data forwarding paths, SDN promotes the use of \textit{network function virtualization} (NFV). 
Instead of using dedicated hardware, NFV relies on the implementation of services using general computing platforms.
NFV is the virtualization of network functions such as domain name service (DNS), firewall, intrusion detection, load balancing and network address translation (NAT).
%REVISION
To this end, a network service is broken into a set of functions that are executed on general purpose hardware.
These functions can then be introduced to or moved between network devices whenever and wherever required.
As we will show later, software-defined radio (SDR) and virtual APs (VAPs) are the two most notable types of NFV in SDWLAN to offload the processing of APs into general computing platforms.

%------------------------------------------------------------------ COLOR
}






%------------------------------------------------------------------ COLOR
\subsection{Standard Protocols}
\label{Protocols}
In SDN architectures, the controller communicates with the data plane through a south-bound protocol, and provides north-bound APIs for application developers.
To avoid exposing the complexities of south-bound APIs, the north-bound APIs are usually REST-based \cite{zhou2014rest,richardson2008restful}, and network engineers use language abstractions, such as Frenetic \cite{foster2011frenetic}, Pyretic \cite{reich2013modular}, and NetCore \cite{monsanto2012compiler}, for application development \cite{SDNLang}.
For example, FloodLight \cite{Floodlight} and OpenDaylight \cite{medved2014opendaylight} both support REST north-bound APIs. 
In the rest of this section we review the widely-adopted standard south-bound protocols.


\textbf{Simple Network Management Protocol (SNMP)} \cite{case1990simple,SNMP,harrington2002rfc,stallings1998snmp}. 
This is an application layer protocol used to monitor and configure network elements.
SNMP exposes \textit{management data} in the form of variables on the \textit{managed systems}. 
The nature of the communication is fetch-store: either the manager fetches resources or stores the value of an object on the agent.
SNMP cannot directly implement actions. For example, to restart an agent, the manager cannot send a command to reboot the agent; instead, it sets the value of the reboot counter, which indicates the seconds  until next reboot.
%In addition, network elements may send push notifications to the manager.
SNMP  has  been  the  most widely used management protocol in production networks due to  its  simplistic  nature  and  ease  of  usage.
Although SNMPv3 introduces significant security enhancements, SNMP has been mostly used for monitoring, rather than configuration \cite{clemm2006network}.
Furthermore, due to the sequential execution of commands, SNMP introduces high network overhead and may result in network failure in large-scale deployments \cite{kona2002framework,silva2005defining}.


\textbf{Netconf} \cite{enns2011network}. 
%This protocol supports transaction execution, which avoids leaving a device in an unknown state if part of the configuration fails.
%This is in contrast with SNMP where instructions are executed sequentially.
Netconf uses remote procedure call (RPC) to exchange messages and apply configuration through Create, Retrieve, Update, and Delete (CRUD) operations.
In addition, it enables the manager to define what action (e.g., continue, stop, roll-back) should be taken if a command fails.
Netconf has been mostly used for configuration, and the assumption is that another protocol such as SNMP is used for monitoring purposes \cite{clemm2006network}.
YANG is a tree-structured  modelling  language used by Netconf to  describe  the  management  information of network elements.
Since YANG enables the definition of data models, Netconf has been widely used by industry; however, vendors may not support a consistent set of data models.


\textbf{OpenFlow }\cite{OpenFlow}. This protocol assumes that data plane elements are simple packet forwarding devices and their operation is determined by the forwarding rules received from the controller.
Whenever a new flow enters a switch, the forwarding table of the switch is queried to find a matching forwarding rule.
If no forwarding rule is found, either the packet is dropped or an inquiry is sent to the controller to retrieve the required \textit{action}.
OpenFlow 1.1 replaces actions with \textit{instructions}, which enables packet modification and updates actions.
The possibility of connecting to multiple controllers has been introduced in OpenFlow 1.2. 
OpenFlow 1.3 enables per-flow rate control. %this simplifies handover
%As we will see in the next section, most of the SDWLAN architecture use OpenFlow due to the flexibilities of this protocol.


\textbf{Control And Provisioning of Wireless Access Points (CAPWAP) }\cite{CAPWAP}. This protocol enables a controller to establish DTLS \cite{DTLS} connections with APs to exchange data and control messages.
Data messages encapsulate wireless frames, and control messages are used for monitoring and control purposes.
%End point devices send discover messages in search of a controller. 
%If a controller intercepts those requests, then a response is sent to each end point.
In addition to centralizing authentication and network management, CAPWAP defines two MAC operation modes \cite{shao2015ieee}: \textit{local MAC} (LM) and \textit{split-MAC} (SM). 
Using the LM mode, most of 802.11 MAC operations are implemented in APs and the role of controller is almost negligible. 
Using the SM mode, the real-time operations of 802.11 MAC are run on APs, and the rest of operations, such as the generation of beacons and probe responses, are handled by the controller.
% Unfortunately, CAPWAP does not provide any mechanism for fast handoff, virtual access points and network slicing.


\textbf{CPE WAN Management Protocol (CWMP)} \cite{TR069} (a.k.a., TR-069). Published by Broadband Forum, CWMP is a text-based protocol for communication between Auto Configuration Servers (ACS) and Customer Premise Equipment (CPE) such as modems, APs and VoIP phones.
The primary features of CWMP are secure auto-configuration, dynamic service provisioning, software/firmware image management, status and performance monitoring, and diagnostics.
One of the main capabilities of CWMP is to enable the devices behind NAT to securely discover ACSs and establish connection.
In addition, an ACS can request a session start from a CPE.
The commands (e.g., get, set, download, upload, etc.) run over a “SOAP/HTTPS” application layer protocol.
In fact, a CPE acts as a client and the ACS is the HTTP server.
Configuration and monitoring of CPEs is performed by setting and getting the device parameters, which are defined as a hierarchical structure.
CWMP is considered to be an important management protocol in M2M architectures \cite{swetina2014toward,al2015toward}.
However, this protocol does not address client mobility in WLANs.
%This protocol runs on an Auto Configuration Server (ACS). 


As OpenFlow has been primarily designed to configure the flow table of switches, an additional protocol, such as SNMP, Netconf, or a proprietary protocol, should be used to enable the control of wireless devices, as we will see in Section \ref{Archs}.
Therefore, both SNMP and Netconf are widely supported by SDN controllers (e.g., OpenDaylight, ONOS \cite{berde2014onos}).
We provide further discussion about protocols in Section \ref{arch-program}.

There are other protocols, in addition to those mentioned in this section, used for south-bound communication.
For example, although REST APIs are mostly used for north-bound interactions, REST is also employed for south-bound communication by controllers such as Ryu \cite{ryu2015framework}.
%Other protocols include Netflow/IPFix \cite{pras2009using}, CLI, CMIS/CMIP
%In addition to north-bound and south-bound protocols, west-bound protocols enable the collaboration between controllers, which is particularly important to achieve scalability.


%------------------------------------------------------------------ COLOR



\subsection{SDWLAN Architectures}
\label{Archs}
In this section we review the architectures proposed for SDWLANs.
We categorize these architectures based on the main contributed feature into five categories: \textit{observability and configurability}, \textit{programmability}, \textit{virtualization}, \textit{scalability}, and \textit{traffic shaping}. 
However, it should be noted that some of these architectures present multiple features, as we will show in Figure \ref{evolution} and Table \ref{ArchTable}.
Please note that the AsC and ChA mechanisms proposed to benefit from these architectures will be studied in Section \ref{AMmech} and \ref{CMmech}, respectively.


%% ----------------------------------------------------

%------------------------------------------------------------------ COLOR
\subsubsection{\textbf{Observability and Configurability}}
The architectures of this section provide the means for central monitoring and configuration, however, they cannot be used to develop new network control mechanisms.
% REVISION
%------------------------------------------------------------------ COLOR


\textbf{DenseAP. }
\label{DenseAP_arch}
This architecture \cite{DenseAP} addresses the challenges of densely-deployed WLANs in terms of AsC and ChA.
The two main components of this architecture are \textit{DenseAP APs} and \textit{DenseAP Controller}. 
DenseAP APs are off-the-shelf PCs running Windows operating system.
%In addition to the networking stack provided by the operating system, 
Each AP runs a \textit{DenseAP daemon}, which is a user-level service responsible for managing AP functionality.
This daemon monitors NIC and reports to DenseAP Controller periodically.
A report includes the list of associated clients, sample RSSI values, traffic pattern of clients, channel condition, and a list of new clients requesting to join the network. 
%The other responsibility of DenseAP deamon is to configure the AP based on the commands received from the DenseAP Controller.
The DenseAP Controller enables the user to tune parameters, makes control decisions, and informs the daemon to apply the configuration.
We will review the AsC mechanism of DenseAP in Section \ref{DenseAP-AM}.



\textbf{Cisco Unified Wireless Network (CUWN). }
\label{CUWNarch}
%
%
%\begin{figure}[!t]
%	\centering
%	\includegraphics[width=1\linewidth]{CAPWAP.pdf}
%	\caption{Cisco Unified Wireless Network (CUWN) \cite{Cisco} architecture. This architecture enables network monitoring and tuning of network control parameters.}
%	\label{fig_Cisco}
%\end{figure}
%
%why the control plane is extended to clients?
%
CUWN \cite{Cisco} proposes an architecture and a set of configurable services, including seamless mobility control, security and QoS provisioning.
%Figure \ref{fig_Cisco} shows this architecture \cite{Cisco}.
CUWN uses either Lightweight Access Point Protocol (LWAPP) \cite{LWAPP} or CAPWAP \cite{CAPWAP} as its south-bound protocol. 
As supported by CAPWAP, CUWN provides two MAC modes: \textit{local MAC} (LM) and \textit{split-MAC} (SM).
The Radio Resource Management (RRM) \cite{CiscoRRM} of CUWN provides several network control services.
For example, RRM establishes \textit{RF groups} and determines a leader controller for each group.
To this end, each AP periodically transmits Neighbor Discovery Protocol (NDP) messages on all channels.
NDP messages contain information such as the number of APs and clients.
All APs forward the received NDP messages to the controller, which establishes a network map, groups APs into RF Groups, and chooses a leader controller for each group. 
%Through exchanging NDP messages among controllers, different RF groups are established and the leader controllers are selected. 
We will review the AsC and ChA mechanisms of CUWN in Section \ref{CUWN_AM} and \ref{CUWN_CM}, respectively.

%This information can be used for predicting client demands and if there is a need for network growth.
%The interference and noise level from other 802.11-based networks can be measured and the RSSI and SNR of all clients can be monitored. 
%The leader controller is the controller with highest performance in terms of hardware capabilities and limitation of the supported number of connected APs which runs most of the RRM algorithms and services centrally.	
%In order for RRM to calculate channel plans and power settings, it is essential that RRM be aware of the RF Location of APs and their interference relationship. %?
%Based on the RF locations of APs, RRM establishes \textit{RF groups} and determines a leader controller for each RF group. %leader?
%All APs connected to a controller belong to an RF group. 
%In this way, the controllers are aware of RF location of APs and their relation to one another which eases the control of APs centrally and in a clustered manner. 
%The key protocol of RF grouping is Neighbor Discovery Protocol (NDP). 
%This communication is supported through the control connection between controllers (Figure \ref{fig_Cisco}).


%The first open source implementation of the CAPWAP, named OpenCAPWAP, was presented in \cite{OpenCAPWAP}. The threading model of OpenCAPWAP is depicted in Figure \ref{fig_CAPWAP}. The OpenCAPWAP is run based on two main Linux applications: (1) controller-side and (2) AP-side. The controller-side application is a multi-thread implementation of the CAPWAP state machine introduced in RFC 5415. Two main threads are (1) the receiver thread and (2) the session manager thread. The AP-side application is also implemented through multiple threads: (1) receiver thread, (2) main thread and (3) receiver-from-client thread. 


%% ----------------------------------------------------


%------------------------------------------------------------------ COLOR
\subsubsection{\textbf{Programmability}}
In addition to central monitoring, these architectures expose north-bound APIs that enable the implementation of control mechanisms as applications running on a controller.
We overview these architectures as follows.


\textbf{DIRAC.}
This architecture \cite{DIRAC} proposes a proprietary south-bound protocol using two main components: a \textit{router core} (RC) running on a PC, and \textit{router agents} (RA) running on APs.

RAs receive messages from RC and convey them to the NIC's device driver.
There are three types of interactions between the RC and RAs: \textit{events}, \textit{statistics}, and \textit{actions}.
Sample events are association, re-association, and authentication, which are reported by RAs to the RC.
RAs periodically report statistics, such as packet loss rate and SNR.
The RC enforces its policy by sending actions to RAs.
For example, the RC may use \texttt{set\_retransmissions()} to limit the number of retransmissions.
The control plane of the RC has two main components: (i) \textit{management mechanisms}, which are the network applications, and (ii) a \textit{control engine}, which includes components to simplify network application development.
% (i) \textit{EventProcessor}: accepts messages from RAs and notifies interested parties, 
% (ii) \textit{StatisticsMonitor}: includes channel quality information for each client, 
% (iii) \textit{ActionProcessor}: sends a requested action to an RA, and
% (iv) \textit{RegistrationDB}: stores client-AP associations and assigns a unique ID to each client.
%The data plane forwarding engine also supports the development of various mechanisms, such as scheduling.
%DIRAC proposes a simple south-bound protocol.
%the handoff technique is mostly focused on implementation and tunneling- nothing new about decision making- also this is ols


%2008

\textbf{Trantor. }
This architecture \cite{Trantor} is an improved version of DenseAP \cite{DenseAP}, and its south-bound protocol extends the exchange of control messages with clients (in addition to APs).
These messages include packet loss estimation, the RSSI of packets received from APs, channel utilization, and neighborhood size. 
Trantor also supports "active monitoring", through which clients and APs are directed to exchange packets to measure metrics that are used by network control mechanisms.
In addition, Trantor provides a set of APIs used to control the association, channel, transmission rate and transmission power of devices.
For example, using $\texttt{associate }(AP_{i})$ a client is instructed to associate with $AP_{i}$. 
Unfortunately, no network control mechanism has been proposed to benefit from these primitives.



%%%%%% Dyson 2010
\textbf{Dyson. }
This architecture \cite{Dyson} (which is an extension of Trantor) enables both clients and APs to communicate with the controller.
%Figure \ref{fig:dyson-arch} shows this architecture.
% \begin{figure}
% 	\centering
% 	\includegraphics[width=0.9\linewidth]{Figures/Dyson-arch}
% 	\caption{Dyson \cite{Dyson} extends data plane programmability to Dyson-aware clients.}
% 	\label{fig:dyson-arch}
% \end{figure}
Clients are categorized into two groups: \textit{Dyson-aware clients} and \textit{legacy clients}. 
Only Dyson-aware clients can communicate with the controller.
The measurement APIs can be used to collect information such as: the sum of the RSSI values (total RSSI) of all received packets during a measurement window, the number of transmitted packets per PHY rate, the number of transmission failures, and the channel airtime utilization.

% \begin{itemize}
% 	\renewcommand\labelitemi{--}
% 	\item The number of received packets and total bytes 
% 	\item Sum of the RSSI values (total RSSI) of all received packets during a measurement window 
% 	%\item Three-tuple $<$\textit{source node}, number of \textit{received packets} from the source node, \textit{total RSSI} of received packets from the source node$>$
% 	\item Number of transmitted packets per PHY rate 
% 	%\item Total airtime used by packets, which is calculated as the multiplication of packet size by PHY rate
% 	\item Number of transmission failures
% 	\item Channel airtime utilization.
% \end{itemize}
%For each measurement metric, a counter is incremented for each packet received, and the average values over a measurement window are calculated through dividing the counters by the number of received packets. 

%% CUT CANDIDATE

Dyson establishes a network map based on the information collected from APs and Dyson-aware clients.
The network map provides the following components: 
(i) \textit{node locations}: reflects the location of APs (which are fixed) and clients (using \cite{WLANlocalization});
(ii) \textit{connectivity information}: a directed graph that shows from which APs/clients an AP/client can receive packets;
(iii) \textit{airtime utilization}: reflects channel utilization in the vicinity of each node;
(iv) \textit{historical measurement}: a database to store measurements.

%The controller is responsible for applying a set of policies to the current network map, making configuration decisions, and issuing commands to configure the operation of clients and APs.
% The policies are network applications that are developed by network engineers using Dyson's APIs.
Dyson's APIs include commands to configure the channel, transmission power, rate and CCA of APs and clients. 
The API set also includes commands to manage AP-client associations.
A reduced set of control APIs is used to support legacy clients. 
This set enables the controller to control the association point and operating channel of these clients. 
%In addition, APs take into account the impact of legacy clients when report their traffic level to the controller.

%\begin{itemize}
%	\renewcommand\labelitemi{--}	
%	\item \small{\texttt{SetRate(r)}}: sets the PHY rate of a client/AP to rate $r$
%	\item \small{\texttt{SetChannel(c)}}: sets the operating channel of a client/AP to channel $c$
%	\item \texttt{SetTxLevel(t)}: adjusts the transmission power of a client/AP to level $t$
%	\item \texttt{SetCCAThresh(t)}: tunes the CCA threshold of a client/AP to threshold $t$
%	\item \texttt{SetPriority(p)}: sets 802.11e priority to $p$. The 802.11e priorities are voice, video, background and best effort. %APPLIED: NOT CLEAR
%	\item \texttt{Throttle(r)}: throttles the outgoing traffic of an AP or a client at the specific rate $r$.  The Throttle command limits the rate of outgoing traffic. %APPLIED: WHAT DO YOU MEAN BY THROUTTLE- EXPLAIN BRIEFLY
%	\item \texttt{Handoff(c, ap, chan)}: hand-offs the client $c$ to access point $ap$ on channel $chan$
%	\item \texttt{AcceptClient(c)}: commands an AP to accept the client $c$ for association.
%	\item \texttt{EjectClient(c)}: commands an AP to disassociate the client $c$.	
%\end{itemize}


%In addition to the architecture proposed, \cite{Dyson} shows how the provided APIs can be used to implement mechanisms such as AsC, client-specific airtime reservation and uplink/downlink load balancing.
%For example, the authors show that Dyson is capable of throttling the traffic of clients and APs in order to adjust airtime assignment or balance uplink/downlink channel allocation.
%


%%%%% AethetFlow 2015
\textbf{$\AE$therFlow.}
\label{AEtherFlow}
This architecture \cite{AEtherFlow} simplifies data plane programmability by extending OpenFlow based on its formal specifications.
The control interfaces are categorized according to the OpenFlow specification as follows:
(i) \textit{Capabilities}: interfaces through which a controller inquires the capabilities of an AP's radio interfaces. 
These capabilities include the number of channels, transmission power levels, etc.
(ii) \textit{Configuration}: these interfaces enable the configuration of physical ports (e.g., channel, transmission power) and logical wireless ports (e.g., SSID, BSSID).
(iii) \textit{Events}: interfaces for event reporting (e.g., probe, authentication, association) to a controller. 
(iv) \textit{Statistics}: these interfaces enable a controller to query wireless related statistics.
%
$\AE$therFlow has been implemented in CPqD SoftSwitch \cite{ofsoftswitch}, which is installed on OpenWRT \cite{OpenWRT}.
The performance and capability of this architecture has been evaluated using a predictive AsC mechanism, which we will explain in Section \ref{AMmech}.



%%%%%% COAP 2015
\textbf{COAP. }
Coordination framework for Open APs (COAP) \cite{COAP} is a cloud-based architecture for residential deployments. 
In addition to using OpenFlow, this architecture proposes an open API to provide a cloud-based central control of home APs. 
COAP improves client QoS by running central control algorithms as well as enabling cooperation among APs.
%The proposed API set is an extension of OpenFlow API and uses the open-source Floodlight controller \cite{Floodlight}. 
A COAP AP implements three modules: (i) \textit{APConfigManager}, (ii) \textit{DiagnosticStatsReporter}, and (iii) \textit{BasicStatsReporter}. 
APConfigManager is used for AP configuration, and the other two modules provide diagnostic and basic statistics for the cloud-based COAP controller. 
The COAP controller has three modules: (i) \textit{StatsManager}, (ii) \textit{COAPManager}, and (iii) \textit{ConfigManager}, which are implemented within Floodlight. 
The controller collects the required information using the StatsManager module and performs control configurations through ConfigManager. 
The COAPManager module allows network administrators to implement new strategies and policies. 
%
%From paper OpenSDWN
%Flashback [10] proposes a control channel technique for
%WiFi networks, by allowing stations to send short control messages
%concurrently with data transmissions, without affecting throughput.
%This ensures a low overhead control plane for WiFi networks that
%is decoupled from the data plane.



%% 2016
\textbf{ResFi.}
\cite{ResFi} enables information collection and sharing in residential networks. 
\textit{ResFi Agent} is a user-space program that runs on APs and provides IP connectivity over the wired backhaul network.
This agent instructs the APs to exchange messages through which APs discover each other's IP address.
After the discovery phase, each AP establishes a secure point-to-point connection to the agents of discovered APs using the wired backbone.
Each agent can then share information with other APs or a remote controller.
For example, an AP can subscribe to the publish socket of its neighbors to receive their updates.
Using ResFi Agents on an AP does not require any kernel or driver modification as it simply relies on the interfaces provided by hostapd \cite{hostapd}.
The APIs provided enables both distributed and centralized management of residential WLANs.
%mainly evaluated based on the overhead of its security features


% ---------------------------------------------------------------------

%------------------------------------------------------------------ COLOR

\subsubsection{\textbf{Virtualization}}
\label{AP-virtualization}
In this section we review the architectures that offer virtualization to support seamless mobility, slicing, and NFV.
In particular, we discuss \textit{AP virtualization}, which refers to the mapping of multiple virtual APs to a physical AP (e.g., Odin \cite{Odin}, CloudMAC \cite{CloudMAC3}), or mapping a virtual AP to multiple physical APs (e.g., SplitAP \cite{SplitAP}, BIGAP \cite{BIGAP}).
%TODO: mention that VAP may be running concurrently on an AP, or they may be running in a round robin fashion such as WVT
%mention when each case would be more useful - for example, for mobility, slicing, etc.



%------------------------------------------------------------------ COLOR
\textbf{WVT.} Wireless Virtualization Testbed (WVT) \cite {smith2007wireless} relies on AP virtualization to run isolated experiments on a testbed.
Corresponding to each experiment, each physical AP runs a \textit{Node Agent}, where each Node Agent is controlled by a \textit{Node Handler} running on the controller. 
Node Handlers provide the user with APIs for monitoring and controlling Node Agents.
In addition to Node Agents, the APs run a \textit{Node Overseer}, which is responsible for activating the Node Agents in a round robin fashion using TDMA scheduling.
In fact, at each slot boundary, the Node Overseer records the state (e.g., SSID, channel) of the currently running Node Agent and restores the state of the next Node Agent.
Node Overseers can be configured through the \textit{Master Overseer} running on the controller.
This architecture uses an NTP time synchronization server, which is required for TDMA scheduling.



\textbf{OpenRoads}. This architecture \cite{Blueprints,yap2010openroads} adds OpenFlow to APs.
In addition, due to the limitations of OpenFlow, OpenRoads uses SNMP for configuring the APs' parameters.
Resource slicing is provided through using FlowVisor and SNMPVisor; the former slices data path, and the latter slices configuration commands through sending them to the appropriate data path element.
%Given the support for network slicing, OpenRoads has been touted as an enabler of innovation using campus networks.

%REVISION


%------------------------------------------------------------------ COLOR

\textbf{AP Aggregate.}
%REVISION
\label{ap_aggregate}
\cite{nagai2011framework} presents a flexible architecture that enables the aggregation of multiple virtual APs (VAPs) inside an AP, as well as slicing resources on APs.
VAP aggregation is also used for interference reduction by turning off lightly-loaded APs. 
Figure \ref{fig_ap_aggregate} shows the architecture of a physical AP.
%
\begin{figure}[!t]
	\centering
	\includegraphics[width=0.78\linewidth]{AP_aggregate.pdf}
	\caption{The architecture of an AP used by AP Aggregate \cite{nagai2011framework}. }
	\label{fig_ap_aggregate}
\end{figure}
%
The \textit{VAP Manager} is responsible for creating and destroying VAPs based on the commands received from the controller.
The \textit{Policy Plane} is a set of settings that are applicable to the VAP.
For example, these policies may specify the QoS and firewall settings of the VAP.
A \textit{Virtual Wireless Interface} has a SSID, MAC and IP address.
Multiple Virtual Wireless Interfaces are multiplexed into the Wireless NIC using TDMA.
Based on its collected information, a controller moves VAPs between APs.
This architecture employs layer-2 tunneling techniques when VAPs are moved between subnets.
%NOTE: the authors used VLAN and VPN to design a layer-2 flat network in multi-subnet networks.
%the authors evaluated the overhead of ap migration and the effect of l-2 flat network.



%%%%%%----Odin-2012
\textbf{Odin.}
\label{Odin_arch}
Figure \ref{fig_Odin} shows the architecture of Odin \cite{Odin,Odin2,OdinThor,OdinSource}.
%\footnote{The source code of Odin is available at: \texttt{https://sdn.inet.tu-berlin.de}.}.
%
\begin{figure}[!t]
	\centering
	\includegraphics[width=0.85\linewidth]{Odin.pdf}
	\caption{Odin \cite{Odin2} introduces the concept of lightweight virtual APs (LVAP) to reduce the overhead of handoffs.}
	\label{fig_Odin}
\end{figure}
%
This architecture is composed of the following components: 
(i) \textit{Odin Controller}: maintains a global network view including the status of APs, clients and OpenFlow switches.
(ii) \textit{Odin Agents}: run on APs, and enable communication through a proprietary south-bound protocol. Time-critical operations (e.g., ACK transmission) are performed by APs, and non-time-critical operations (e.g., association) are handled by the controller.
This operation results in a split-MAC protocol, which is inspired by CAPWAP. 
(iii) \textit{Applications}: implemented on the controller. Applications may use the information provided by Odin Agents, OpenFlow, and SNMP.
The APIs  support the implementation of various control mechanisms such as AsC, load balancing and hidden-terminal handling.

%
Odin introduces the concept of \textit{light virtual AP} (LVAP) to separate clients' association states from physical APs; thereby reducing the overhead of handoffs.
An LVAP is characterized by the following four-tuple: 
\begin{itemize}
	\item client IP address,
	\item a unique virtual basic service set ID (BSSID),
	\item one or more service set ID (SSID),
	\item a set of OpenFlow rules.
\end{itemize}

For each client, an LVAP resides in an AP (hosted by Odin Agent) to represent client-AP association. 
When the controller decides to associate a client with a new AP, it can simply transfer the LVAP of that client to the new AP. 
%This process is performed without the intervention of clients, and it does not require any layer-2 or layer-3 message exchange.
From a programmer's point of view, multiple clients are connected to different ports (the LVAPs) of a physical AP. 
%More importantly, using LVAPs does not require any client-side modification.
LVAPs also simplify client authentication through adding a session key to the client's LVAP.
% Due to the limitations of OpenFlow to manage the operation of 802.11 devices \cite{Blueprints}, Odin employs the \textit{Odin protocol} to communicate with Odin Agents, and OpenFlow is used for communication with switches.
% Odin controller maintains a permanent TCP connection to Odin Agents to collect statistics and distribute configuration commands.
We will discuss about Odin's AsC mechanism in Section \ref{AMOdin}.


% Future work on the Odin 
% We are exploring further abstractions in order to support the needs of a more diverse set of network applications. 
% We also plan to explore
%           fault- tolerance and 
%           fail-over capabilities and 
%           policy management for Odin.


%%%%%%%  2010
\textbf{SplitAP.}
\cite{SplitAP} proposes an architecture to support network virtualization and manage clients' share of airtime, especially for uplink traffic.
Each AP runs a \textit{SplitAP controller}, which is responsible for VAP management and computing the uplink traffic of each slice.
By relying on the VAP concept, each physical AP broadcasts the beacons of independent virtual networks.
For example, when three different ISPs utilize an infrastructure, each AP emulates three VAPs.
This architecture requires client modification to enforce controller commands.
Specifically, each client runs a \textit{client controller} that adjusts traffic based on the commands received from the associated AP.
When a SplitAP controller realizes that the usage of a particular slice is higher than its threshold, the AP broadcasts a new maximum uplink airtime utilization that can be consumed by clients in that slice.
All the APs are connected to a shared backhaul, through which they receive channel allocation commands from a controller.

%To this end, clients are grouped as slices, and fairness of uplink airtime among different slices is achived through using LPFC and LPFC+ algorithms.



%REVISION - MOVED HERE
%%% 2013
\textbf{VAN.}
\label{VAN_arch}
Virtualized Access Network (VAN) \cite{VAN} proposes an architecture for central control of residential networks.
The architecture is composed of the following components: (i) \textit{residential APs}, (ii) a \textit{controller}, which is part of the ISP, and (iii) \textit{content provider servers} that are part of the content provider network (e.g., Netflix).
The controller provides a set of APIs through which content providers control home networks.
These APIs are suitable for different traffic classes such as video streaming and file transfer.
For example, when a content provider receives the request of a user, the content provider communicates with the controller and reserves resources for the requested data flow.

In this architecture, IP address assignment and authentication are performed centrally per user device; therefore, it is possible to transfer the association of a user device between APs.
More importantly, this architecture enables the sharing of APs' bandwidth between neighbors.
Given a flow bandwidth requirement and the neighborhood of each client, the controller associates clients to APs by employing a heuristic algorithm.



%REVISIION
%the legend size has been reduced
%%%%%%----CloudMAC-2013
\textbf{CloudMAC. }
Figure \ref{fig_CloudMAC} illustrates the CloudMAC \cite{CloudMAC,CloudMAC3} architecture. 
%
\begin{figure}[!t]
	\centering
	\includegraphics[width=1\linewidth]{CloudMAC.pdf}
	\caption{CloudMAC \cite{CloudMAC} reduces the complexity of APs through running VAPs on a cloud computing infrastructure.}
	\label{fig_CloudMAC}
\end{figure}
%
CloudMAC decomposes AP operations into two modules and employs OpenFlow switching tables to manage the communication between these modules.
The main components of CloudMAC are (i) APs, (ii) VAPs, (iii) an OpenFlow switch, and (iv) an OpenFlow controller. 
APs only forward MAC frames. 
All other MAC functionalities are performed by VAPs that reside in a cloud computing infrastructure. 
VAPs are operating system instances on a hypervisor such as VSphere.
Each VAP can include multiple virtual WLAN cards, where each card is implemented as a driver in a VAP; therefore, standard software tools can utilize these cards.
Virtual machines are connected to physical APs by an OpenFlow-based distribution network. 
The generation and processing of MAC frames are performed by VAPs. 
A physical AP, which is a slim AP, sends and receives raw MAC frames from clients and handles time-sensitive operations (e.g., ACK generation and re-transmission).

APs and VAPs are connected by layer-2 tunnels \cite{capsulator} and the OpenFlow switch.
The forwarding table of switches are used to decide how packets must be routed between physical APs and VAPs.
This mechanism simplifies the implementation of control mechanisms such as AsC.
For example, while Odin requires LVAP migration to handle mobility, CloudMAC simply reconfigures the switching tables to change the data forwarding path.

%From the paper: "By reconfiguring the switch table, one AP can easily be moved from one WTP to another, together with all flows passing through it. As each physical card on a WTP can be bound to multiple virtual WLAN cards (traffic can be distinguished by the BSSID and MAC addresses), CloudMAC inherently supports network virtualiza- tion."

When a control command is generated by a network application, the command is processed by VAP and a configuration packet is forwarded toward the OpenFlow switch.
The OpenFlow switch forwards the packet to the OpenFlow controller to check the legitimacy of the configuration command according to user-defined policies. 
If the command is legitimate, the OpenFlow controller forwards the packet to the AP. 
%The controller application inside the AP executes the command and returns the execution result to the OpenFlow controller and the VAP.

%Since some OpenFlow implementations (e.g., OpenVSwitch \cite{OpenvSwitch}) enable the modification of control headers, the implementation of control mechanisms (e.g., power and rate adaptation) is well supported by CloudMAC.


%An AP broadcasts the beacon messages that reduces the available network capacity for data communication. %?
%CloudMAC can enable a scenario that OpenFlow switch does not forward beacon messages from VAPs to the physical AP. 
%In this way, it is possible to have an application on OpenFlow controller to detect the probe requests of clients, and then enable the beacon messages dynamically. 
%This approach is called on-demand AP which can reduce the number of beacon messages in WLAN. 

To study the performance of CloudMAC, the round trip time (RTT) is measured between a client and a VAP. 
Due to the processing overhead of the OpenFlow switch and the delay overhead added by the tunnels, the average RTT is increased from 1.79 ms to 2.28 ms, compared to a standard WLAN.  
However, time-critical MAC frames (e.g., association response frames) are delivered fast enough to satisfy the timeliness requirements.
For large TCP packets, throughput is decreased by almost 8.5$\%$, compared to a standard WLAN. 
This performance degradation is because of the tunnels implemented in user space, which requires context switching. 
%Using kernel-space tunnels is a future work to improve performance. 
%Since a small and simple WLAN system was used to evaluate the performance of CloudMAC, it would be interesting to measure the overhead of this architecture in large-scale scenarios.


%%% LEAVE IT AFTER CLOUD MAC AS IT RELIES ON VAP CONCEPTS OF THE FIRST VERSIONS OF ODIN AND CLOUD MAD
%%%%%%% EmPower 2013

%------------------------------------------------------------------ COLOR
\textbf{EmPOWER.}
\label{EMPOWERarch}
%EmPOWER \cite{EmPOWER} is an SDN/NFV-based testbed for WLANs. 
EmPOWER \cite{EmPOWER} relies on the concept of LVAP (proposed by Odin) in order to decrease the overhead of client mobility.
The controller runs Floodlight as the operating system and FlowVisor to enable virtualization.
In this architecture, each AP is equipped with an Energino \cite{Energino} add-on, which is an open toolkit for energy monitoring.
This add-on provides REST-based APIs that enable network administrators to turn on and off the APs.

%Furthermore, it uses an add-on, named Energino \cite{Energino}, at each AP which measures the energy consumption of access point. The measurement circuit includes a current sensor that works based on the Hall effect, and a voltage sensor that works based on a voltage divider. 




\textbf{EmPOWER2.}
\label{empower2_arch}
%2015
An improved version of EmPOWER has been proposed in \cite{Primitives,EmPOWER-src}.
We refer to this architecture as EmPOWER2.
This architecture provides a full and open set of Python-based APIs by introducing four key abstractions:
(i) \textit{LVAPs}: a per-client VAP similar to Odin.
(ii) \textit{Resource pool}: each resource block is identified as $( (frequency, bandwidth), time)$. For example, an AP working on channel 36 with bandwidth 40MHz is represented as $( (36, 40), \infty)$. 
Similar notation is used to represent the 802.11 standard supported by LVAPs. This representation provides a mean for LVAP to AP mapping. 
(iii) \textit{Channel quality and interference map}: provides network programmers with a global network view in terms of the channel quality between APs and LVAPs. This view enables the programmers to allocate resources in an efficient manner.
(v) \textit{Port}: specifies the configuration of an AP-client link in terms of power, modulation and MIMO configuration.
We will study its AsC and ChA mechanisms in Section \ref{AM-InvidualOpt} and \ref{ChA_traffic_ag}, respectively. 

% the authors have presented a mechanism for monitoring uplink communications.
% They propose a passive time synchronization approach: when multiple APs receive a packet from a client, they all forward the sequence number and their timing to the controller, which performs time synchronization.
% The controller collects frame delivery statistics every 500ms to provide network applications with real-time link quality.



%------------------------------------------------------------------ COLOR
%%%%%SDWLAN 2014
\textbf{Sd-wlan.}
\label{SDWLANarch}
The main idea of Sd-wlan \cite{SDWLAN,SDWLAN2} is the implementation of most MAC functionalities at a controller, which is essentially similar to Odin \cite{Odin2}. 
%APs perform RTS/CTS exchange and acknowledgment generation.
%Other MAC operations including beaconing, probe response, client association/re-association, and client authentication/re-authentication are performed by the controller. 
To this end, an extended version of OpenFlow is proposed as the south-bound protocol. 
Using this protocol, the controller can instruct the APs whether they should send ACK packets or not.
When a client needs to change its point of association, the controller must install new rules on the new AP.
The controller also updates OpenFlow switches to direct the traffic being exchanged between the controller and APs.


%SDWLAN develops an application called \textit{virtual AP management} to provide \textit{One Big AP} illusion. 
%This application creates multiple virtual APs (VAP) in the controller and stores association-related information of all clients so that the clients will be unaware of AP handoff.  
%The authors of SDWLAN also proposed a fast handoff mechanism, which we will explain in Section \ref{SDWLANdam}. 

%%%%Three main new features of SDWLAN in comparison to Odin and CloudMAC architectures are:
%%%%\begin{itemize}
%%%%	\item Fast AP handoff: In Odin, AP handoff is performed by moving the LVAP between APs that takes some time. However, SDWLAN does the AP handoff based on the extended OpenFlow protocol that eliminates the time overhead of handoff.
%%%%	\item Central and secure key management: In Odin, LVAPs stores the keys of clients so that the mobility of clients can lead to key scattering in multiple APs. This key scattering brings about some security issues. However, SDWLAN manages all keys centrally through an en/decryption appliance.
%%%%	\item Per-client AP handoff: In CloudMAC, handoff is done by switching all the associated clients between APs and does not provide per-client AP handoff.
%%%%\end{itemize}


%%%%% BeHop 2015
\textbf{BeHop.}
%BeHop architecture \cite{BeHop} is depicted in Figure \ref{fig_BeHop}. 
%%
%\begin{figure}[!t]
%	\centering
%	\includegraphics[width=0.7\linewidth]{BeHop.pdf}
%	\caption{BeHop \cite{BeHop} architecture. Each BeHop AP creates a VAP per client.}
%	\label{fig_BeHop}
%\end{figure}
%
The main components of BeHop \cite{BeHop} are: (i) \textit{BeHop APs}, (ii) a \textit{BeHop collector}, and (iii) a \textit{BeHop controller}. 
BeHop collector is responsible for monitoring and collecting information from the network.
The controller includes network control mechanisms, in addition to processing and responding to probe, authentication and association requests.
The collector may be implemented internally (inside the controller) or externally as a separate component.
%However, the latter is proffered as decomposing collector from controller separates information plane and control plane, and eases the collection and processing of large amount of data without affecting responsiveness and stability.


A proprietary south-bound protocol has been implemented by running \textit{monitoring agents} on APs.
This protocol is used to collect and forward statistics (e.g., channel utilization, SNR, RSSI, and PHY rate) to the collector.
The controller communicates with the collector through a remote procedure call (RPC) interface to update its status about BeHop APs. 

BeHop APs create a VAP per client, and each AP maintains a \textit{client table}.
%BeHop APs compete to acquire the clients by responding to their probe requests earlier than non-BeHop APs. 
%After acquiring a client, a VAP is created per client. A VAP operates as a physical AP for clients. In other words, the clients only see the VAPs, not physical APs. 
Each entry of the client table includes client control information such as client-VAP mapping and client data rate. 
VAPs can be added to or removed from APs using the south-bound protocol implemented.
Each BeHop AP operates as an OpenFlow switch and exposes APIs for controlling channels, power, and association.
APs forward control traffic (e.g., probe, association) to the controller for further processing.
%BeHop has been evaluated by a testbed in which dual radio APs (2.4GHz and 5GHz) run OpenWRT \cite{OpenWRT}, OpenvSwitch \cite{OpenvSwitch}, and a proprietary south-bound protocol.


%eHop APs expose a configuration API through which the controller can manage their operation.

%In summary, some important WiFi management strategies that can be implemented and run by the BeHop controller are as follows:
%
%\begin{itemize}
%	\item \textbf{Disable a physical AP:} performed by removing all VAPs from the BeHop AP.
%	\item \textbf{Emulation of a high-density WLAN:} performed by adding more than one VAP per physical AP.
%	\item \textbf{Channel assignment and transmit power control:} enabled through the configuration API provided by the BeHop APs.
%	\item \textbf{Association control:} performed based on the mapping between the clients and the VAPs in different BeHop APs so that a moving client's VAP is changed seamlessly.
%	\item \textbf{Per-client control:} supported based on the clients' information stored in the client table. The controller can run a specific protocol by running it only on a subset of clients.
%\end{itemize}

% BeHop has been evaluated by a testbed in which dual radio APs (2.4GHz and 5GHz) run OpenWRT \cite{OpenWRT}, OpenvSwitch \cite{OpenvSwitch}, and a proprietary south-bound protocol.
% The BeHop network has been installed alongside a Cisco LWAPP \cite{Cisco} production network in a 4-story building to serve more than 90 devices.
% The BeHop network provides the same SSID, and therefore, BeHop is indistinguishable for clients.
% To serve clients by the BeHop network, BeHop APs respond to clients' probe messages before the production network.
% Experiments confirm that when association decisions are made by clients, some clients associate with APs over highly-congested channels; this confirms the importance of central AsC.
% Experiments also reveal the importance of dense AP deployment in the 5GHz band in order to achieve a complete coverage.
% However, AP densification increases the interference level at 2.4GHz band.  

%To address these concerns, central control mechanisms are required to carefully manage channel assignment, transmission power and association operations.


%------------------------------------------------------------------ COLOR
%REVISION
\textbf{RCWLAN}. This architecture \cite{nakauchi2012airtime} addresses the coexistence of multiple virtual networks (slices).
For each virtual network, a VAP is created on the physical APs, where each VAP has its own virtual machine and MAC queue.
The set of resources allocated to each slice is controlled by configuring the MAC parameters of the VAPs.
The extension of RCWLAN is vBS, which we explain next.


%
% 2015
\textbf{vBS.} \cite{vBS} supports service-based wireless resource reservation, which is similar to the concept of flow-based virtualization \cite{sherwood2009flowvisor}.
Specifically, by reserving resources for time-critical services (e.g., VoIP), vBS provides QoS guarantees even in the presence of background traffic.
%Figure \ref{fig:vbs} shows this architecture.
%
% \begin{figure}[t]
% 	\centering
% 	\includegraphics[width=0.95\linewidth]{Figures/vBS}
% 	\caption{vBS \cite{vBS} maps multiple physical APs to a virtual base station. %vBS1 is a common network, and vBS2 is a service-specific network. A client connected to vBS1 is handedoff to vBS2 when it initiates a service supported by vBS2.
% 	}
% 	\label{fig:vbs}
% \end{figure}
%
In contrast with SplitAP, which supports virtualization through broadcasting different ESSIDs, this architecture relies on the concept of \textit{virtual Base Station} (vBS).
A vBS is a virtual multichannel AP that uses the resources of multiple physical APs.
All of the APs share the same BSSID (MAC address) and ESSID; therefore, AP selection and handoff are completely handled by the controller.
When a network administrator registers a new service, the controller creates a vBS for that service.
When a client joins the network for the first time, the controller associates the client with an AP of the common vBS.
The common vBS provides a service for non-prioritized traffic.
When the client initiates a service, the controller performs a client handoff to the vBS of that service.
%Therefore, flows pertaining to a service are bound with the virtual network (vBS) providing that service.
The association status of the client is transferred to the new AP, and the OpenFlow switch is updated.
%When the number of clients using a service-specific vBS exceeds a threshold, then the controller assigns more APs to that vBS.

%NOTE: the handover mechanism is not for mibility. it is for handing off a client from a common AP to a dedicated AP.


%%2016
\textbf{BIGAP.}
The main goal of this architecture \cite{BIGAP} is to support seamless mobility by mapping physical APs to a single VAP.
BIGAP assumes that each AP has two wireless interfaces: one for serving as AP, and the second one for statistics collection which works in promiscuous mode to periodically overhear packets on all channels.
The collected information is used by the controller to provide a fast AsC mechanism, which will be explained in Section \ref{BIGAPhandoff}.
In order to reduce the complexities of the handoff process, BIGAP requires all of the APs to share the same BSSID.
However, to avoid collisions, BIGAP requires each AP to select a channel that is not being used by its two-hop neighborhood.
Association of a client with another AP is achieved through a channel switch command and transferring client's status to the new AP.
%Therefore, this architecture can be used with 5GHz networks where a higher number of orthogonal channels are available (25 channels). 
BIGAP provides a rich south-bound protocol for statistics collection and association control.
The AP-related APIs are implemented by modifying hostapd \cite{hostapd}, and the APIs are available as RPC by relying on ZeroRPC \cite{zerorpc} and ZeroMQ \cite{zeromq}.


%%%%%%% OpenSDWN 2015
\textbf{OpenSDWN.}
%This architecture \cite{OpenSDWN} enables per-flow MAC and PHY transmission settings to enhance data plane programmability.
Figure \ref{fig_OpenSDWN} illustrates this architecture \cite{OpenSDWN}. 
%This virtualization results in seamless user mobility and dynamic resource allocation.
%
\begin{figure}[!t]
	\centering
	\includegraphics[width=1\linewidth]{OpenSDWN.pdf}
	\caption{OpenSDWN \cite{OpenSDWN} enhances data plane programmability by introducing virtual middle boxes (vMBs).}
	\label{fig_OpenSDWN}
\end{figure}
%
%OpenSDWN defines per-client LVAP and vMB in order to facilitate the management of WLANs and support QoS provisioning. 
Similar to Odin, OpenSDWN provides per-client VAP; however, the new concepts of OpenSDWN are \textit{virtual middle boxes} (vMB) and \textit{wireless datapath transmission} (WDTX) rules. 
A vMB encapsulates a client's middle box state. 
Due to the small size of vMBs, their transfer across network does not introduce any significant overhead.
Each vMB includes the following information:
(i) the list of tunable parameters of the client,
(ii) the states of client's active connections,
(iii) the packet-based statistics, and
(iv) an event list that defines the behavior of the client. %(already submitted by the controller). 

The controller implements a MB driver and Radio Driver to communicate with the agents of MBs and APs, respectively. 
Through the Radio Driver, the controller manages LVAPs and collects the status of clients. 
Furthermore, the controller can get/set per-flow transmission rules through its radio interface. 
To this end, OpenFlow rules are combined with WDTX rules within APs.
WDTX rules provide per-flow service differentiation using a newly defined action that is compatible with OpenFlow protocol. 
The controller manages vMBs, gathers the statistics of vMBs, and receives the required events from different MBs. 
The latter, in particular, is supported through a publish/subscribe interface to ensure controller update when specific events occur.

A middlebox in OpenSDWN can have two types of interfaces: (i) a \textit{stateful firewall} (FW) agent and (ii) a \textit{deep packet inspection} (DPI) agent. 
The FW agent keeps track of all traffic passing through the MB in both directions and provides statistics per client and per flow. 
Also, the FW agent runs the WDTX rules defined by the controller for different flows and frames. 
The DPI agent is responsible for detecting events such as denial of service (DoS) attacks. 
The agents of a MB generate and send events that are of the interest to the controller. 
For instance, a significant change in the load of an AP may trigger sending a report to the controller. 
The controller can submit a report list to a MB in order to obtain only requested events. 

The OpenSDWN controller exposes a participatory interface that enables network applications to define flow-based and client-based priorities.
This is achieved by providing RESTful APIs through which LVAPs, WDTXs and vMBs are monitored and controlled.  

%The authors conducted experiments to reveal the effects of new features of OpenSDWN, i.e., vMB and WDTX rules, to implement WLAN control mechanisms.
%The testbed includes 25 APs (running OpenWRT) that serve more than 70 devices. 
%Three servers are used as controller, middlebox, and traffic generator.
Through empirical evaluations, the authors showed the small delay overhead of vMB migration among MBs to support mobility, and they demonstrated the effectiveness of using WDTX rules to perform service differentiation. 
%Furthermore, a case study on video-on-demand optimizer is implemented by defining different per-flow WDTX rules with the aim of giving higher priority to video streaming flows. 



%REVISION 
%------------------------------------------------------------------ COLOR
\subsubsection{\textbf{Scalability}}
\label{arch_scalability}
Scalability depends on topology, as well as factors such as the overhead of south-bound protocol, the virtualization mechanisms employed, and the traffic and mobility pattern of clients.
The scalability of SDWLANs is particularly important due to the dynamic nature of these networks.
Specifically, an architecture with a high control plane delay cannot be used for quick reactions to network dynamics.


%%%%%%% AeroFlux 2014
\textbf{AeroFlux.}
This architecture \cite{AeroFlux,AeroFlux2} highlights that per-flow or per-packet control mechanisms require short and bounded control plane delay.
For example, implementing rate and power control mechanisms in a controller may result in an overloaded and high-latency control plane.
To address this concern, AeroFlux employs a 2-tier control plane: \textit{Global Control }(GC) plane, and \textit{Near-Sighted Control} (NSC) plane. 
GC handles non real-time tasks and operations that require a global network knowledge.
For example, authentication, large-scale mobility, and load balancing are handled by the GC.
NSCs are located close to the APs to perform time-critical operations, such as rate control and power adjustment, per packet.
For example, in the case of video streaming, AeroFlux can configure APs to use lower rates and higher transmission power values for key frames, compared to regular frames.

\textbf{CUWN.} This architecture \cite{Cisco} employs split-MAC as well as a multi-tier controller topology to address scalability.
Based on the RF locations of APs, CUWN establishes \textit{RF groups} and determines a leader controller for each RF group. %leader?
All APs connected to a controller belong to an RF group. 
In this way, the controllers are aware of RF location of APs and their interference relationship.

\textbf{Odin and CloudMAC.}
These architectures \cite{Odin,CloudMAC3} rely on the the split-MAC technique proposed by CAPWAP.
In other words, to mitigate the negative effect of controller-AP communication, time-critical operations are handled by APs, and non time-critical operations are offloaded to the controller.
In addition to utilizing split-MAC, during an association, CloudMAC prevents the overhead of moving VAPs between APs through configuring the switches of the distribution network to forward the client's data to another AP.



%------------------------------------------------------------------ COLOR
%REVISION : NOT SURE IF THIS SHOULD BE TRAFFIC SHAPING OR THE CURRENT TITLE 
\subsubsection{\textbf{Traffic Shaping}}
The architectures of this section offer more than the regular programmability used for configuring APs and switches.
In fact, these architectures extend data plane programmability by enabling traffic shaping, which is used for purposes such as scheduling and scalability.


%2009
\textbf{CENTAUR.} \cite{CENTAUR } can be added to SDWLAN architectures to improve the operation of the data plane in terms of channel access and contention resolution.
When data traffic passes through a controller or programmable middleboxes, data plane programmability can be supported without AP modification.
To this end, CENTAUR proposes central packet scheduling mechanisms to avoid hidden-terminal transmissions and exploit the exposed-terminal condition.
For example, when the transmission of two APs to their associated clients may cause a hidden-terminal collision, the controller carefully adjusts the interval between packet transmissions to avoid the problem.
A salient feature of CENTAUR is that it requires minor changes to APs and no client modification is required.
CENTAUR improves UDP and TCP throughput by around 46\% and 61.5\%, compared to DCF \cite{bianchi2005remarks}.


\textbf{CloudMAC}. This architecture \cite{CloudMAC} is capable of implementing downlink scheduling by configuring OpenFlow switches to use simple rate shaping or time division algorithms.
For instance, during a time slot, the switch may forward only the packets of a specific physical AP while queuing the packets of other APs. 
Switching rules may be changed per time slot to provide a time division protocol.



%%%%%% Ethanol 2015
\textbf{Ethanol.}
\label{EthanolArch}
%This work proposes APIs for handling client mobility, QoS, security and AP virtualization. 
Ethanol \cite{Ethanol} is similar to its predecessors (e.g., \cite{Odin,CloudMAC}) in terms of implementing slim APs and shifting most of the MAC functionalities to a controller.
However, Ethanol argues that the original OpenFlow protocol cannot be used for QoS provisioning in wireless networks.
Accordingly, to complement OpenFlow, the Ethanol architecture proposes a customized protocol, named \textit{Ethanol protocol}, to provide control interface for wireless components and QoS control.
Therefore, an Ethanol AP provides two interfaces: (i) Ethanol, through an \textit{Ethanol Agent,} and (ii) OpenFlow. 
The Ethanol Agent provides APIs for QoS control on the APs.
Ethanol uses the APIs provided by the Ethanol Agent to exploit Hierarchical Token Bucket (HTB) \cite{Pantou} scheduling in order to perform per-flow programmability.
Specifically, different queues are defined on Ethanol APs. 
When a flow arrives, it is assigned to the proper queue based on its class of service (e.g., voice, video).

Ethanol \cite{Ethanol} provides the details of API implementation following an object-oriented approach.
\textit{Entities} are defined as physical or virtual objects that could be configured or queried.
Each entity has its own properties.
For example, an AP is a physical entity that includes properties such as beaconing interval.
A flow is a virtual entity that includes properties such as a packet counter.
The controller communicates with entities through their get/set interfaces. 
Entities may also include events, which trigger the controller to take proper actions.
Ethanol API has been implemented on OpenWRT \cite{OpenWRT} by exploiting Pantou \cite{Pantou}, a software package enabling OpenFlow on OpenWRT.

%The authors conducted an experiment where three clients use an Ethanol AP.
%Three queues for three clients are defined with rates proportional to their traffic type.
%The different rate of queues enables bandwidth allocation to clients proportional to their required services. 







% %% ---------------------------------------------------------------------
% {\color{blue!50!black}
% %------------------------------------------------------------------ COLOR
% \subsubsection{\textbf{Home Networks}}
% \R{arch_home_class}
% %   REVISION : THERE ARE A FEW MORE PAPERS THAT WE COULD ADD HERE - REFER TO THE PAPER CITED IN THE FEW LINES BELOW
% The increase in the number of residential APs as well as the need to support emerging applications (e.g., video surveillance, medical monitoring, IoT) necessitate centralized control of these networks in order to achieve the QoS parameters desired.
% %For example, a recent study \cite{largeScaleMeas} shows that AP neighborhood size is around seventeen.
% %As a comprehensive review of residential networks has been recently published in \cite{wSDN1}, 
% In this section we review the state-of-the-art architectures proposed for home networks.
% }


%
%REVISION

%------------------------------------------------------------------ COLOR
% ** CONSIDER CHANGING THE TITLE TO "COMPARISON. LESSONS LEARNED, AND OPEN PROBLEMS"
\subsection{Architectures: Learned Lessons, Comparison, and Open Problems}
\label{archComp}
% REVISION:
% refer to the table here and add a summary of the objectives of the architectures and how did they achieve it

Figure \ref{evolution} highlights the main features and Table \ref{ArchTable} summarizes the properties of reviewed architectures.
In this section, we summarize the learned lessons, study the proposed features and identify future research directions and potential solutions.
%
%
\begin{figure*}[!t]
	\centering
	\includegraphics[width=0.87\linewidth]{Evolution2.pdf}
	\caption{This figure summarizes the main features of SDWLAN architectures based on the categorization presented in Section \ref{Archs}.}
	\label{evolution}
\end{figure*}
%
%



% PROGRAMMABILITY 
% REVISION minor changes

\subsubsection{\textbf{Programmability}}
\label{arch-program}
A \textit{reconfigurable architecture} only enables the adjustment of parameters pertaining to a set of predefined network control mechanisms. 
For example, CUWN's APs are like regular APs with added CAPWAP support; thereby, it is not possible to implement a new AsC mechanism as CUWN employs a set of proprietary mechanisms for network control.
In fact, we can argue that CUWN and DenseAP introduce a \textit{management plane} instead of a control plane.
On the other hand, a \textit{programmable architecture}, such as OpenSDWN and Odin, provides north-bound APIs through which control mechanisms running on the network operating system are developed.

%DenseAP is a centrally-controlled architecture which is not an open architecture, and Trantor is an open centralized architecture with a limited set of high-level interfaces, compared to Dyson. 



After the standardization of OpenFlow, most SDWLAN architectures have included this standard in their south-bound communication, as Table \ref{ArchTable} shows.
However, this protocol has been mainly designed to configure the flow tables of switches; therefore, it cannot be used for the configuration of wireless data plane equipment.
%For example, in addition to using OpenFlow, BeHop utilizes a proprietary protocol for channel, power and association control.
Furthermore, since OpenFlow is very low-level, too many implementation details are exposed to network programmers \cite{Primitives}.
Therefore, we can observe the introduction of proprietary protocols (e.g., Odin, Ethanol) and the extensions of OpenFlow (e.g., AeroFlux, $\AE$therFlow).
However, this results in interoperability issues.
Another shortcoming of OpenFlow is the lack of supporting transactions, which means the operation of a network device during receiving updates is unpredictable.
This has a particular implication on wireless network control mechanisms.
For example, when an AsC mechanism sends updates to multiple APs, network performance may significantly drop during the update process due to the inconsistency of APs' software.
Despite the significant number of studies on the performance and improvement of OpenFlow for wired networks \cite{Levin2012,jarschel2011modeling,vanbever2013hotswap,lara2014network}, there is a very limited study on the use of OpenFlow and other standard protocols (e.g., CWMP, SNMP, Netconf) for SDWLAN design\cite{Primitives,AEtherFlow,rao2015towards,slabicki2015performance}.
%In addition, the abstraction level provided by OpenFlow is unnecessarily complex.
In addition, it is not clear what programmability features are required to develop architectures that include recent high throughput standards.
For example, supporting the 802.11ad \cite{nitsche2014ieee} standard requires additional interfaces for central coordination and training of antennas.




Our review shows that data plane programmability is supported at four levels: controller (e.g., CENTAUR), switches (e.g., CloudMAC), APs (e.g., Ethanol), and clients (e.g., Dyson).
%For example, CENTAUR implements data plane scheduling at the controller.
While controller-based approaches provide higher flexibility, they pose scalability and processing challenges.
Specifically, the variations of data plane delay may inadvertently affect the accuracy of implemented mechanisms. 
For example, deterministic controller-AP delay is an essential requirement when packet scheduling is implemented at a controller in order to avoid collisions caused by concurrent downlink transmissions.
On the other hand, AP-based approaches require low and bounded control plane delay to ensure prompt enforcement of the decisions made centrally.
Meanwhile, the programmability of switching equipment can be employed to improve scalability and cope with overhead issues.
For example, when a client is associated with a new AP, CloudMAC configures the switches of the distribution network to forward the client's data to another AP, thereby preventing the overhead of moving VAPs between APs.
We believe that investigating the pros and cons of these design approaches is necessary, and mechanisms are required to integrate the benefits of these approaches.


AP programmability at layer-1 and layer-2 requires the use of software-defined radio (SDR) platforms such as OpenRadio \cite{OpenRadio}, Atomix \cite{Atomix} or Sora \cite{Sora}.
When integrated into a SDWLAN architecture, using SDRs enables us to offer a richer set of programmability options to network application developers.
Additionally, integrating SDRs with SDWLANs simplifies the upgrade of communication standards when the SDR hardware is capable of supporting the new technology.
For example, when the digital signal processor (DSP) used on a SDR is fast enough to support new modulation and coding schemes, the controller can update APs to match the capabilities of clients.
Furthermore, when signal processing operations are offloaded to a cloud platform, advanced signal processing techniques may be applied to the incoming signals to cope with interference.
While SDWLAN-SDR integration is an ongoing research trend, it is also important to study the implications of SDR platforms (e.g., FPGA, DSP, general-purpose processor) on architecture in terms of factors such as control plane delay and overhead.


Although most architectures only rely on the information collected from APs, some architectures, such as Dyson and SplitAP, also require client modification to collect information from the clients' point of view.
Extending data plane programmability to clients enables the use of more sophisticated control mechanisms.
For example, this would enable the control plane to adjust clients' MAC parameters based on network dynamics and user demands, which also simplifies resource allocation and network slicing.
However, in addition to client modification, which may not be desirable for public access networks, extending control plane to clients introduces traffic overhead.
Although newly proposed mechanisms such as FlashBack \cite{FlashBack} enable low-overhead exchange of control information with clients, the integration of these mechanisms with SDWLANs is not clear. 
Accordingly, we believe that client programmability has not been well studied in the literature.
Additionally, though the 802.11r and 802.11k amendments (which have been recently integrated into 802.11 standard \cite{802.11_2016}) realize richer interactions between clients and APs, the SDWLAN architectures did not benefit from these features.











%
%
\begin{table*}
	\centering
	\scriptsize
	\caption{Comparison of SDWLAN architectures}
	\label{ArchTable} 
	\def\arraystretch{1}
	\begin{tabular}{|c|c|c|c|c|c|c|c|c|c|}
		\Xhline{3\arrayrulewidth}
		\multirow{2}{*}{\textbf{Architecture}}&
		\multirow{2}{*}{\textbf{\makecell{Programmable \\Architecture}}}&
		\multirow{2}{*}{\textbf{\makecell{Details of \\Provided APIs}}}&
		\multirow{2}{*}{\textbf{\makecell{VAP}}}&
		\multirow{2}{*}{\textbf{\makecell{Network\\Slicing}}}&		
		\multirow{2}{*}{\textbf{\makecell{South-bound  \\Protocol}} }&
		\multirow{2}{*}{\textbf{\makecell{Split-MAC \\Support}}}&
		\multirow{2}{*}{\textbf{\makecell{Client \\Modification }}}&		
		\multicolumn{2}{c|}{Implemented Mechanisms}\\\cline{9-10}								
		&&&&&&&& \quad\quad \textbf{AsC} \quad\quad & \textbf{ChA}\\		\Xhline{3\arrayrulewidth}
		
		DenseAP \cite{DenseAP}&$\times$&$\times$&$\times$&$\times$&Proprietary&$\times$& $\times$ &$\checkmark$&$\times$\\\hline
        % SB X + Y    NB: Z
		CUWN \cite{Cisco}&$\times$&$\times$&$\times$&$\times$&CAPWAP&$\checkmark$&$\times$&$\checkmark$&$\checkmark$\\\hline

		DIRAC \cite{DIRAC} &$\checkmark$&$\checkmark$&$\times$&$\times$& Proprietary &$\times$& $\times$ &$\checkmark$&$\times$\\\hline
		
		Trantor \cite{Trantor}&\textbf{$\checkmark$}&$\checkmark$&$\times$&$\times$&Proprietary&$\times$&$\checkmark$&$\times$&$\times$\\\hline
		%
		%		
		Dyson \cite{Dyson}&$\checkmark$&$\checkmark$&$\times$&$\times$&Proprietary&$\times$&$\checkmark$&$\checkmark$&$\times$\\\hline
		%		
		$\AE$therFlow  \cite{AEtherFlow}&$\checkmark$&$\checkmark$&$\times$&$\times$& Extended OpenFlow  &$\times$&$\times$&$\checkmark$&$\times$\\\hline		
        %it is not clear if they support VAP or not. they say mapping is possible, but the paper is very unclear
        %
		WVT \cite{smith2007wireless}&$\times$&$\times$&$\checkmark$& $\checkmark$ & Proprietary &$\times$&$\times$&$\times$&$\times$\\\hline
        %        
		AP Aggregate \cite{nagai2011framework}&$\times$&$\times$&$\checkmark$& $\checkmark$ & Proprietary &$\times$&$\times$&$\checkmark$&$\times$\\\hline
        %        
		OpenRoads \cite{Blueprints,yap2010openroads} & $\checkmark$&$\checkmark$&$\times$& $\checkmark$ & OpenFlow+SNMP &$\times$&$\times$&$\checkmark$&$\times$\\\hline
        %        
		%
		Odin \cite{Odin,Odin2,OdinThor}&$\checkmark$&$\checkmark$&$\checkmark$&$\checkmark$& OpenFlow+Proprietary &$\checkmark$&$\times$&$\checkmark$&$\times$\\\hline
        %
		SplitAP \cite{SplitAP}&$\times$&$\times$&$\checkmark$&$\checkmark$& Proprietary &$\times$&$\checkmark$&$\times$&$\times$\\\hline		
		%
		CloudMAC \cite{CloudMAC,CloudMAC3}&$\checkmark$&$\times$&$\checkmark$&$\checkmark$& OpenFlow &$\checkmark$&$\times$&$\checkmark$&$\checkmark$\\\hline				
		%
		EmPOWER \cite{EmPOWER}&$\checkmark$&$\times$&$\checkmark$&$\checkmark$& OpenFlow+REST &$\checkmark$&$\times$&$\checkmark$&$\times$\\\hline		
		%
		%
		EmPOWER2 \cite{Primitives}&$\checkmark$&$\checkmark$&$\checkmark$&$\checkmark$& OpenFlow &$\checkmark$&$\times$&$\checkmark$&$\checkmark$\\\hline				
		%
		Sd-wlan \cite{SDWLAN}&$\checkmark$&$\times$&$\checkmark$&$\times$& Extended OpenFlow &$\checkmark$&$\times$&$\checkmark$&$\times$\\\hline
        %		
		BeHop \cite{BeHop}&$\checkmark$&$\times$&$\checkmark$&$\times$& OpenFlow+Proprietary &$\checkmark$&$\times$&$\times$&$\times$\\\hline
		%		% a configuration API for channel and power allocation --> No details about API !!!
		%
		RCWLAN \cite{nakauchi2012airtime}&$\times$&$\times$&$\checkmark$&$\checkmark$& Proprietary &$\times$&$\times$&$\times$&$\times$\\\hline	
        %		
		vBS \cite{vBS}&$\checkmark$&$\times$&$\checkmark$&$\checkmark$& OpenFlow &$\times$&$\times$&$\checkmark$&$\times$\\\hline				
        %        
		BIGAP \cite{BIGAP}&$\checkmark$&$\checkmark$&$\checkmark$&$\times$& Proprietary &$\times$&$\times$&$\checkmark$&$\checkmark$\\\hline		
        %        
		AeroFlux \cite{AeroFlux}&$\checkmark$&$\checkmark$&$\checkmark$&$\times$& Extended OpenFlow &$\checkmark$&$\times$&$\checkmark$&$\times$\\\hline		
        %        
		Ethanol \cite{Ethanol}&$\checkmark$&$\checkmark$&$\checkmark$&$\times$& OpenFlow+Proprietary &$\checkmark$&$\times$&$\checkmark$&$\times$\\\hline
        %        
		OpenSDWN \cite{OpenSDWN}&$\checkmark$&$\checkmark$&$\checkmark$&$\checkmark$& OpenFlow+REST  &$\checkmark$&$\times$&$\checkmark$&$\times$\\\hline
        %        
		VAN \cite{VAN} &$\checkmark$&$\checkmark$&$\times$&$\checkmark$& OpenFlow &$\times$&$\times$&$\times$&$\times$\\\hline	
		%		%		
		%		
		COAP \cite{COAP}&$\checkmark$&$\checkmark$&$\times$&$\times$& OpenFlow+Proprietary  &$\times$&$\times$&$\times$&$\checkmark$\\\hline
		%		
		%		%		%
		%
		%		%
		ResFi \cite{ResFi}&$\checkmark$&$\checkmark$&$\checkmark$&$\times$& Proprietary &$\times$&$\times$&$\checkmark$&$\checkmark$\\\Xhline{3\arrayrulewidth}		
		%\\
	\end{tabular}
\end{table*}

%
%
%




\subsubsection{\textbf{Mobility}}
\label{seam_mobility}
Mapping VAPs to physical APs is the main technique used by architectures to support seamless mobility.
As each VAP is represented by a simple data structure that holds client's status, exchanging VAPs between APs removes the burden of re-association, as offered by Odin.
To avoid the overhead of VAP migration, CloudMAC settles the APs in cloud servers, and programs the switches to route data between physical APs and VAPs.
However, performance evaluation of CloudMAC shows that, when traffic  is  transmitted  through  a  shared network, high-priority packets may be lost or delayed, thereby reducing the chance of connection establishment to 55\%.
%To remedy90this problem, CloudMAC extends OpenVSwitch [82] (running91on switches) to incorporate queue management strategies
Although queue management has been proposed as a remedy, the effectiveness of such approaches for different topologies, mobility, and traffic patterns, is an open research area.
Meanwhile, we believe that the match-action paradigm of OpenFlow must be exploited to improve scalability and handoff performance through designing traffic shaping and prioritization mechanisms that ensure the real-time delivery of time-critical packets.
However, the literature does not present any relevant contribution.
Another solution is to use network slicing, as we will explain in Section \ref{net_vir_dis}.
On the other hand, most of the reviewed architectures (except CUWN and AeroFlux) use a single-tier controller topology, and unfortunately, they do not present an evaluation of handoff performance in a real-world scenario with tens of APs, background traffic and variable mobility patterns.
For example, Odin, 	$\AE$therFlow and BIGAP use testbeds with 10, 2 and 2 APs, respectively.
Therefore, a study of architectural implications on handoff performance is missing.
We present further discussion about the challenges of mobility management after the review of central control mechanisms in Sections \ref{AscProblems} and \ref{ChAProblems}.


% VIRTUALIZATION 
% REVISION

%%%%%
\subsubsection{\textbf{Network Virtualization}}
\label{net_vir_dis}
%REVISION
Our review shows that the most common approach of network slicing is through the use of VAPs.
For example, in Odin, a slice is defined as a set of APs (Odin Agents), LVAPs corresponding to clients, SSIDs, and network applications.
Each SSID may belong to one or multiple slices, and each client joins the slice to which the SSID belongs to.
Since each application can only see the LVAPs belonging to its slice, slice isolation is logically provided through the concept of LVAPs.

Although VAP migration can be used for slicing airtime, bandwidth and processing power, it does not provide a fine granularity for resource allocation.
For example, if the controller moves a client's VAP to another AP in order to increase its available bandwidth, the actual bandwidth offered depends on the activity of other associated clients.
In order to achieve high granularity, layer-1 and layer-2 resource slicing techniques are required to support airtime, frequency and space multiplexing.
For example, a middleware should manage the access of VAPs to the AP's physical resources through TDMA scheduling.
However, compared to wired networks, the highly variable nature of wireless communications makes predictable slicing a challenging task.
Specifically, it is hard to predict and manage the effect of one slice on another. 
For example, user mobility and variations of traffic in one slice affect the available bandwidth experienced by the users in another slice \cite{nakauchi2012airtime}.
Compared to LTE networks, resource slicing is more challenging in 802.11 networks because 802.11 relies on random access mechanisms (i.e., CSMA) and does not utilize a dedicated control plane to communicate with clients.
The problem is exacerbated when multiple networks managed by different entities coexist.
To cope with these challenges, the existing network slicing techniques heavily rely on a localized network view \cite{SplitAP,ViFi}.
For example, while SplitAP supports central network control, each AP decides about its uplink airtime allocation individually and through a distributed algorithm.
Based on this discussion, utilizing algorithms that rely on global network view, and the integration of centralized AsC and virtualization are necessary towards improving the capacity and QoS of virtualized networks.


Network slicing can also be used to tackle the challenges of control and data plane communication.
To this end, an abstraction layer slices the resources of switching elements based on the communication demands of higher layer control mechanisms.
For example, based on the mobility pattern and number of clients, an AsC mechanism may request the abstraction layer for the allocation of switching resources.
This is an open research area.

%Although mechanisms such as ViFi \cite{ViFi} address these challenges in a distributed manner, centralized performance measurement and dynamic resources slicing and allocation mechanisms are open research challenges.
%Layer-1 isolation mechanisms for SDWLANs is an open research challenge.


Network slicing would also enable the coexistence of M2M communication with regular user generated traffic, similar to the 5G vision given in \cite{alliance20155g}.
For example, an IoT device may utilize multiple slices of a network to transmit flows with different QoS requirements.
Developing such architectures and their associated control mechanisms is an open research area.
We present further discussion about this after the review of control mechanisms, in Section \ref{ch_disc_int_virt}.

%In addition, to achieve client-level virtualization, the  





\subsubsection{\textbf{Network Function Virtualization}}
\label{nfv_dis}
%---
The most common usage of NFV in SDWLANs is split-MAC, which is mainly implemented through VAP.
Split-MAC was first introduced by the split-MAC protocol implemented in CAPWAP. 
As an another example, CloudMAC separates MAC functionalities between physical AP and cloud platform.
Using the split-MAC strategy, time critical operations run on APs, while other MAC operations are handled by the controller. 
This strategy enables exploiting the resources of remote computing platforms and simplifies MAC and security updates without requiring to replace APs as either no or minimal changes to APs is required.
Despite the benefits of VAPs, it is important to establish a balance between flexibility and communication overhead because, when more functionalities are shifted to a controller, the overhead and delay of distribution system increase.
For example, even for small networks, the importance of utilizing queuing strategies with OpenFlow switches has been demonstrated \cite{CloudMAC3,Vestin2015b,Ethanol}.
Consequently, studies are required to show the tradeoffs between centralization and scalability.

SDR is an another example of NFV.
SDR platforms transfer radio signal processing operations into a general-purpose processor on the same board or a remote server.
In addition to improving programmability, this enables the use of powerful processors for centralized signal processing, which can be used to improve signal decoding probability and coping with interference challenges.
As mentioned in Section \ref{arch-program}, the integration of SDRs with SDWLANs is in its early stages.


%For example, while Odin \cite{Odin2} requires LVAP migration to handle mobility, CloudMAC \cite{CloudMAC} can simply reconfigure switches' forwarding tables using OpenFlow.

%A sample solution is vBS \cite{vBS}, which uses a centralized controller to control client-AP associations.
%However, there is no solution to integrate virtualization with association control (AsC) and channel assignment (ChA).


%From a a survey paper:
% A programmable network architecture simplifies the deploy-ment  of  middleboxes  [52].  Programmable  network  elements can  be  used  to  steer  the  traffic  to/from  the  right  sequence  of middleboxes [38], [52], allowing to change the location of the middlebox on the network or to change the sequence of middle-box traversal according to the flow’s characteristics.
 

%From paper OpenSDWN
%Picasso [22] enables virtualization across the MAC/PHY and
%uses spectrum slicing. It allows a single radio to receive and trans- mit on different frequencies simultaneously. MAClets [7] allows multipleMAC/PHY protocols to share a single RF frontend. 
%These advances can be used by OPENSDWN (and already Odin) to oper- ate multiple LVAPs with different characteristics on top of the same AP. Alternative


%OpenSDWN enables new services and applications based on the global knowledge about the content of data packets throughout the WLAN and provides per-client, per-flow and per-frame prioritizing services.
%OpenSDWN is the first WLAN architecture enables using MBs and vMBs that eases the implementation of network services such as intrusion detection system, deep packet inspection, and firewalls, in a programmable manner.

%from paper OpenSDWN
%SDN is also an enabler for a second paradigm shift in the Internet: Network Functions Virtualization (NFV). Modern networks include many middleboxes to provide a wide range of network functions to improve performance as well as security. For exam- ple, %middleboxes are used for caching and load-balancing, as well as for intrusion detection. NFV aims to virtualize these network functions, and replace dedicated network function hardware with software applications running on generic compute resources.


%from paper Odin
%Virtualization of the PHY layer: Although we have addressed isolation at the IEEE 802.11 MAC layer, our system does not handle virtualization of the PHY layer, which is a logical next step. The IEEE 802.11 stan- dard defines a Point Coordination Function (PCF), for centrally scheduled channel access. However, the PCF is rarely implemented in today’s WiFi hardware/drivers. Picasso [29] enables virtualization across theMAC/PHY. It proposes a technique to perform spectrum slicing and allows a single radio to receive and transmit on different frequencies simultaneously. MAClets [13] allows multi- ple MAC/PHY protocols to share a single RF frontend. These advances can be used by Odin to operate multiple LVAPs with different characteristics on different chan- nels on top of the same AP. Alternative approaches, such as [32] and [36], are incompatible with today’s WiFi MAC/PHY and thus do not fit our design requirements.

%from Ehanol:
%Network virtualization allows multiple isolated logical net-
%works to share the same physical infrastructure even if they use different addressing and forwarding mechanisms. FlowVisor uses SDN to achieve this segmentation, slicing five dimensions [8] of a switching element: bandwidth, topology, traffic, device CPU and forwarding tables. The same dimensions, with the exception of the forwarding tables, are present in APs, and could be subject to control by a centralized entity. The use of virtualization allows the coexistence of a re-
%search wireless networks and production networks at the same physical location, providing not only isolated applications but also management. Also we can benefit of faster provisioning, enabling elastic capacity to provide system provisioning and deployment at a moment’s notice by the activation of cloned virtual access point. But there are drawbacks too, as a wireless device still has a limited number of physical radios.

%
%All architectures (except CUWN) assume the use of one central controller and did not propose any practical strategy for WLANs with multiple controllers. Using multiple controllers is particularly important to address reliability concerns. However, this brings ups new challenges such as the distribution of MAC functionalities among controllers when the split-MAC strategy is used.




%\subsubsection{Implemented mechanisms}
%The capabilities and performance of WLAN architectures have been evaluated by proposing central control algorithms and mechanisms such as association management (AM) and channel management (CM).
%We will review AM and CM mechanisms in Section \ref{AMmech} and \ref{CMmech}, respectively.

%A FlowDiff mechanism (implemented by  Ethanol \cite{Ethanol}, OpenSDWN \cite{OpenSDWN} and COAP \cite{COAP}) centrally manages the priority of different flows (based on flow type or client's priority) using the provided API commands by the architecture
%OpenSDWN \cite{OpenSDWN} enables frame-level prioritization (FrameDiff), which can be useful in different applications such as live streaming (by assigning higher priority to key frames that are more important for service quality). 
%Since some architectures did not provide the details of proposed APIs, it is difficult to infer the possibility of developing different control mechanisms.


% REVISION - newly added
% \subsubsection{\textbf{Scalability}}
% With the increase in the number of APs and clients, it is important to address scalability issues.
% Specifically, since AP virtualization is a widespread technique, is important to study how it would affect the handling of time-critical packets in various deployments.
% Unfortunately, a very limited study of scalability has been performed by Odin, CluodMAC and OpenSDWN, and AeroFlux is the only architecture that presents a solution to this concern.




\subsubsection{\textbf{Home WLANs}}
\label{arch_discus_home_net}
The importance of utilizing SDN-based mechanisms in home networks is justified given the facts that the number of 802.11 connected devices is being increased (e.g., home appliances, lighting and HVAC, medical monitoring) and some of these devices require timely and reliable data exchange with APs \cite{Qin2014,Tozlu2012,REWIMO}.
In addition, the emergence of 802.11 mesh technologies that rely on the installation of multiple APs per home \cite{jing2016multi} makes central control of home networks even more important.
Compared to enterprise networks, the topology of AP placement in home networks is random and variable.
Topology is even more randomized when users or distributed algorithms modify the transmission power and channel of their APs.
This has serious implications:
While ISP-based control (e.g., VAN) simplifies the installation and operation of home networks, the ISP may not be able to collect enough information from the home network because not all the neighbors choose the same ISP.
On the other hand, collaboration between homes (e.g., ResFi) introduces serious security challenges.
Finally, cloud-based control may not be very effective because neighboring users may not subscribe to the same service.
Therefore, the opportunities and challenges of home SDWLANs requires further research.


\subsubsection{\textbf{Security}} 
\label{arch-security}
In addition to running new network control mechanisms such as AsC and ChA, programmability (see Section \ref{arch-program}) simplifies security provisioning.
For example, when a security mechanism does not require hardware replacement (e.g., 802.11i's WPA), it can be implemented by reprogramming APs and clients.
However, the decomposition of control and data plane exposes security threats due to enabling of remote programming of network equipment.
For example, a DoS attack can be implemented by instructing a large number of clients to associate with an AP.
Therefore, it is important to identify security threats and take them into account during the architecture design process.


As mentioned earlier, virtualization through network slicing is an effective approach to control and isolate clients' access to resources.
However, as users will be sharing an infrastructure, achieving a strict isolation becomes more challenging.
For example, a client may simply switch to the frequency band of another slice to disrupt the capacity offered in that slice.
Unfortunately, secure virtualization has not been investigated by the research community.


SDWLANs enable network applications and administrators to detect abnormal activities and network breaches.
In the architectures that extend their control plane to clients in order to exchange monitoring and control packets (e.g., Trantor, Dyson, SplitAP), the reports received from malicious clients result in control decisions that negatively affect network performance.
To cope with this challenge, a malicious client may be detected using the flow statistics collected from APs and switches \cite{OpenSketch}.
In the next step, client location may be found through analyzing RSSI measurements \cite{SpotFi}.
Rogue (unauthorized) APs can be detected through collaboration among APs. 
For example, an abnormal interference level or sniffing the packets being exchanged with a rogue AP can be used by the detection algorithm.
Depending on the architecture in use, a network application can block the access of such APs through various mechanisms.
For example, using Odin, client access can be blocked through updating OpenFlow switches.
Proposing security mechanisms that benefit from the features of SDWLAN architectures is an important future direction, especially for large-scale and public networks.


%[Trantor]: However, it is a much harder problem to determine if a malicious (or faulty) client is sending spurious reports. 
%One potential way to address this problem is to verify such reports with reports from other APs and clients in the neighborhood, but this remains an open issue.



% ----->>>>
% REVISION
% consider writing a section on using millimeter wave communication

%------------------------------------------------------------------ COLOR
