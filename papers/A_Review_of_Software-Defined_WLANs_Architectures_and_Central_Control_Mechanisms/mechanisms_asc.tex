
% REVISION

\section{Centralized Association Control (AsC)}
\label{AMmech}
In this section we review centralized AsC mechanisms. 
We focus on the metrics employed as well as the problem formulation and solving approaches proposed. 
This section also highlights the impact of architecture on design when details are available.
We employ a consistent notation to ease the understanding and comparison of metrics and formulations.
Our review summarizes the performance improvements achieved to reveal the benefits of these centralized mechanisms compared to distributed approaches.\footnote{Note that in this section and the next section (Section \ref{CMmech}) we do not study all the AsC and ChA mechanisms implemented by the architectures reviewed in Section \ref{Architectures}. 
This is because some of these architectures only show the feasibility of implementing control mechanisms, and they do not propose any \textit{new} mechanism to benefit from the features of SDWLANs.}
Although we mostly focus on state-of-the-art centralized mechanisms, we review the seminal distributed mechanisms as well because of their adoption as the baseline to evaluate the performance of centralized mechanisms.



%------------------------------------------------------------------ COLOR
At a high level, we categorize AsC mechanisms from two perspectives: 
%
\begin{itemize}
	\item \textit{\textbf{Seamless handoff}}: refers to the mechanisms that their objective is to reduce the overhead and delay of client handoff,
	\item \textit{\textbf{Client steering}}: refers to the mechanisms that adjust client-AP associations to optimize parameters such as the load of APs and the airtime allocated to clients.
\end{itemize}
%
Supporting seamless handoff is usually addressed by proposing architectures that reduce the overhead of re-association.
In contrast, client steering is performed through proposing AsC mechanisms that run on the control plane.
For example, an AsC mechanism may propose an optimization problem to balance the load of APs, while handoff delays depend on the architectural properties.
From the client steering point of view, AsC is particularly important in dense topologies because there are usually multiple candidate APs for a client in a given location. 
Therefore, using a simple RSSI metric may result in hot spots, unbalanced load of APs and unfair resource allocation to clients.
Hence, one of the main goals of AsC is to achieve \textit{fairness} among clients and APs. 


%Client mobility is handled through dynamic AsC. 
%To this end, low time-complexity AsC mechanisms must be run periodically to decide about client re-association.
%Metrics such as RSSI, client demand and AP load may be considered by AsC mechanisms to choose a destination AP.




%There are numerous AsC mechanisms proposed in the literature.
%In this paper, however, we only focus on mechanisms that rely on global network knowledge.
%Before the overview of these mechanisms, we first explain two widely-adopted distributed AsC mechanisms that have been used as baselines to evaluate the performance of centralized mechanisms.

The distributed AsC mechanism employed by 802.11 standard is \textbf{strongest signal first (SSF)} \cite{SSF,802.11}. 
Using SSF, each client decides about its association based on the RSSI of probe response and beacon messages.
Each client associates with the AP from which highest RSSI has been received.
%Most of association management techniques use SSF as a baseline for their comparison.
%maybe add about 802.11k and r
\textbf{Least load first (LLF)} \cite{LLF} is an another widely-adopted distributed mechanism where APs broadcast their current load through beacon messages to help the clients include AP load when making association decisions.
The load of an AP is represented through various metrics, such as the number of associated clients\footnote{The traffic indication map (TIM) of a beacon packet represents a bitmap that indicates the clients for which the AP has buffered packets.}.
In the following, we review AsC mechanisms.


%------------------------------------------------------------------ COLOR
\subsection{Seamless Handoff}
\label{seam-handoff}
Some of the architectures reviewed in Section \ref{Archs} propose mechanisms to reduce the overhead of client handoff.
Handoff overhead refers to: (i) the packets exchanged between client and AP to establish a connection, and (ii) the delay incurred by the client during this process \cite{pack2007fast}.
In this section we study the contributions of SDWLAN architectures in terms of supporting seamless handoff.


% 2010
% Individual - Seamless handoff
\textbf{Cisco unified wireless network (CUWN).}% demand-agnostic
\label{CUWN_AM}
\cite{Cisco} enables the central configuration of RSSI threshold with hysteresis to perform seamless handoff of CCX \cite{CiscoCCX} compatible clients.
When the RSSI received from associated AP drops below the \textit{scan threshold}, the client increases its AP scanning rate to ensure fast handoff to another AP when the difference between the RSSI of the associated and new AP is equal or greater than the \textit{hysteresis} value specified.
In addition to hysteresis, specifying the \textit{minimum RSSI} value forces the clients to re-associate when their RSSI drops below a minimum RSSI. 

Three types of client roaming scenarios are handled by CUWN: (i) intra-controller roaming, (ii) inter-controller layer-2 roaming, and (iii) inter-controller layer-3 roaming.
The controller simply handles an intra-controller roaming by updating its client database with the new AP connected to the roaming client. 
Inter-controller layer-2 roaming occurs when a client associates with an AP that is controlled by a different controller belonging to the same subnet. 
In this type of roaming, mobility messages are exchanged between the old controller and the new one. 
%({\color{red}e.g.,} {\color{blue} There is no details in \cite{Cisco} about these messages!}) 
Then, the database entry related to the roaming client is moved to the new controller. 
Inter-controller layer-3 roaming occurs when a client is associated with an AP that is controlled by a controller belonging to a different subnet. 
In layer-3 roaming, the database entry of the roaming client is not moved to the new controller. 
Rather, the old controller marks the client's entry as \textit{anchor entry}. 
The entry is copied to the new controller and marked as \textit{foreign entry}. 
Therefore, the original IP address of the roaming client is maintained by the old controller. 
CUWN enables network administrators to establish mobility groups, where each group may consist of up to 24 controllers. 
A client can roam among all the controllers in a mobility group without IP address change, which makes seamless and fast roaming possible.


%\todo[inline,color=cyan]{any details about how the association algorithm works? \\ \textit{The previous paragraphs describe the association algorithms.} }  


%2012 August, 2014 June
% Individual - Seamless handoff
\textbf{Odin's Mobility Manager (OMM).}
\label{AMOdin}
%In this section we review the mobility and load balancing algorithm proposed by Odin \cite{Odin,Odin2} (see Section \ref{Odin_arch} ).
%As mentioned earlier, Odin introduces the concept of Light Virtual AP (LVAP), which is a small data structure residing on APs to indicate client association.
Odin  \cite{Odin,Odin2} enables seamless handoffs through LVAP migration between APs. 
The Odin controller maintains a persistent TCP connection per Odin Agent running on AP, thereby, switching among agents does not require connection reestablishment. 
In addition, the delay of a re-association equals the delay of sending two messages from the Odin controller to the old and new APs. 
The first message removes an LVAP from the old AP, and the second message adds an LVAP to the new AP. 
Assuming that the messages are sent successfully, the delay equals the longest round trip time (RTT) between the controller and the two APs, which depends on network size. 
%note: the controller may communicate with the to APs concurrently

To show the effectiveness of LVAPs, Odin employs a simple RSSI-based AsC mechanism.
The mobility application (running on the controller) selects the Odin Agent with highest RSSI if client movement is detected by the controller. 
The effect of handoff on TCP throughput has been evaluated using two APs and a client. 
While a period of throughput drop in regular 802.11 layer-2 handoff is observable, Odin does not show any throughput degradation.
%The authors also reported the maximum number of LVAP handoffs that does not cause throughput drop.
Performance evaluations also show that executing 10 handoffs per second results in a negligible reduction in TCP throughput.

%It is worth mentioning that, before Odin, seamless mobility support through VAP migration was proposed in \cite{Grunenberger2010a}.
%We do not discuss about the details of this mechanism due to its similarity to Odin.



%%%%%%%%%%%%%%%%%%%
%2015 Nov
% Individual - Seamless handoff


\label{AEtherflow_AM}
\textbf{$\AE$therFlow.} \cite{AEtherFlow} argues that handoff support through LVAP migration (e.g., Odin \cite{Odin2} and OpenSDWN \cite{OpenSDWN}) imposes high computational and communication overhead, especially in large networks with many mobile clients.
They propose a predictive handoff strategy by relying on the extended OpenFlow protocol proposed in this work (see Section \ref{AEtherFlow}).
The handoff mechanism works in three phases: 
\begin{itemize}
	\item \textit{Prediction}: %The controller collects the RSSI of the links between all clients and APs. 
	The controller predicts an association when the RSSI of a client to its associated AP declines while its RSSI to another AP increases. 
	\item \textit{Multicasting}: By updating OpenFlow tables, the controller multicasts the packets of the client to both the current AP and the predicted AP.
	\item \textit{Redirection}: The controller will redirect the client's traffic to the new AP after the handoff completion. If no handoff occurs, then the multicasting is stopped.
\end{itemize}

Experiments using two APs and a client shows that the handoff delay of $\AE$therFlow is around 5.3 seconds, compared to the 7.1 seconds delay of 802.11 standard.
%A 9Mbps UDP traffic is sent to the client.
%Handoff duration is measured as the time interval during which throughput drops below 8Mbps.
% No comparison with Odin and OpenSDWN?


% 2016 April
% Individual - Seamless handoff
\textbf{BIGAP.}
\label{BIGAPhandoff}
\cite{BIGAP} uses the 25 non-overlapping channels of 5GHz band to form disjoint collision domains for handoff.
As explained in Section \ref{Archs}, a separate NIC is used to periodically overhear packets on all channels, which enables the controller to compute the potential SNR values of client-AP links.
A handoff happens when a higher SNR would be achievable for a client.
However, to avoid the ping-pong effect, an 8dB hysteresis value is used. 

BIGAP performs handoff through client channel switching.
Since all APs share the same BSSID, in order to handoff a client from $AP_{1}$ to $AP_{2}$, the controller instructs $AP_{1}$ to send a channel switching command to the client.
The client then switches to the channel being used by $AP_{2}$.
Since both APs use the same BSSID, the client does not notice handoff.

Performance evaluations (using two APs) show that the BIGAP handoff is about 32 times shorter than the regular 802.11 handoff.
In addition, BIGAP results in lower energy consumption because, it moves the overhead of handoff to APs and there is no need for the clients to scan the channels.
In addition, while 802.11 results in zero throughput for about 4 seconds, BIGAP shows only 5\% throughput reduction during the handoff.


%------------------------------------------------------------------ COLOR
\subsection{Client Steering}
\label{client-steering}
Based on the scope of the optimization problem employed, we categorize client steering mechanisms into two groups: \textit{\textbf{centrally-generated hints}} and \textit{\textbf{centrally-made decisions}}.
In AsC mechanisms using centrally-generated hints, the controller relies on the global network view to generate hints for the association of clients.
In other words, the controller does not make the final decision about associations and instead, enables the clients to make more informed decisions through the hints conveyed.
On the other hand, in AsC mechanisms using centrally-made decisions, the controller makes the association decisions and enforces the clients to apply them.
We will discuss in Sections \ref{AM-InvidualOpt} and \ref{AM-GlobalOpt} the two sub-categories of mechanisms based on centrally-made decisions.

\subsubsection{\textbf{Centrally-Generated Hints}} 
\label{Per-clientAM}
In this section we review AsC mechanisms that employ client steering through hints generated centrally.


% REVISION - I REMOVED THIS IN THE NEW VERSION
% %2010
% % Individual - Client steering - hidden node avoidance
% \label{DysonAM}
% %The AsC mechanism of DenseAP \cite{DenseAP} has been improved by Dyson \cite{Dyson} as follows.
% \textbf{Dyson.} Using the network map constructed by Dyson \cite{Dyson}, the controller detects if two clients or APs cause a hidden-terminal collision.
% To remedy this situation, the controller changes the channel of the AP with fewer associated clients.
% Dyson's set of APIs enables the AP to inform its clients about channel switching and avoid the overhead of rediscovery and re-association.
% %Dyson also introduces a simple VoIP handoff strategy through which VoIP clients are connected to special APs referred to as VoIP APs. 
% %Among the potential points of connection, a VoIP client connects to the AP with highest available bandwidth.




%%%%%%%%%%%%
%\label{SDWLANdam}
% SDWLAN \cite{SDWLAN,SDWLAN2} (see Section \ref{SDWLANarch}) proposes a fast handoff mechanism for a WLAN in which all APs operate on the same channel.
%From the clients' point of view, all the APs are conceived as a \textit{One Big AP}.
%When a client needs to change its AP, the controller must install new rules on the new wireless access switch.
%The controller also updates OpenFlow switches to direct the traffic being exchanged between the controller and APs.
%
%The throughput of TCP and UDP are evaluated to measure the performance of this handoff mechanism.
%They found that AP handoff in regular 802.11 networks incurs nearly one second interruption on TCP/UDP session. 
%This interrupt is due to MAC re-authentication, re-association and MAC address relearning. 
%In addition, the TCP session interruption takes more than one second due to TCP's \textit{slow start} mechanism. 
%On the contrary, TCP and UDP throughput are slightly affected when the proposed handoff mechanism is used.

%\todo[inline, color=cyan]{when? \\ If there is a better AP in terms of defined metrics. It is event-triggered not time-triggered technique.}  

%2014 Feb
% Individual - Client steering - maximizing bandwidth
\textbf{BestAP.}
\label{BEST-AP}
\cite{BEST-AP} proposes an AsC mechanism based on the estimation of \textit{available bandwidth} (ABW) for each client at every AP in its vicinity. 
The available bandwidth depends on channel load, which varies with packet loss and PHY rate. 
The estimated available bandwidth at PHY rate $r$ is computed as,
%
\begin{equation}
E[ABW(r)]=\frac{8S_{data}(1-B)}{\sum_{k=0}^{n}(1-p_s(r))^kp_s(r)T_{tx}(r,k)},
\end{equation}
%
where $S_{data}$ is data size (bytes), $p_s$ is the probability of successful packet transmission, $T_{tx}(r,k)$ is the transmission time of a packet during the \textit{k}th transmission attempt with PHY rate $r$, and $B$ is the fraction of channel busy time. 
$B$ is measured through using the CCA register of NIC.
%If there is no such hardware support, $b$ can be calculated using the method proposed in \cite{ChannelLoad}, which is based on measuring the airtime consumed by each packet. 
$p_s(r)$ is measured using the statistics provided by the rate adaptation algorithm. 
All other parameters are configured statically. 

A scheduler is run on each client to allocate a measurement period (e.g., 50ms every 2s) during which the client sends data to all nearby APs in order to update packet loss and channel busy fraction. 
%This updates the ABW of each client at their reachable APs.
A monitoring service is run on APs to collect the statistics from clients, calculate ABW, and send a report to the controller.
The controller sends the estimated ABWs of the best five APs to each client periodically. 
%The report is also sent when the ABW of a client has been changed more than $x\%$ ($x=10$ is used during the experiments). 
BEST-AP only considers the ABW of downlinks.

%Experimental results show that BEST-AP's ABW estimation is more accurate than WBest \cite{WBest}. 
Testbed evaluations show that the delay overhead of ABW estimation is less than 50ms, which is appropriate for a dynamic AsC mechanism.
%\todo[inline, color=cyan]{how do you claim it is appropriate for a DAC mechanism? \\ \textit{It is not my claim. It is the claim of authors !!} }  
Compared to SSF \cite{SSF}, the proposed AsC mechanism shows 81\% and $176\%$ improvement in throughput for static and mobile clients, respectively.
%\todo[inline, color=cyan]{the above sentence needs work- please correct it}  


%2015
% Global - Client steering - equal number of clients per AP
\textbf{Ethanol.}
This architecture \cite{Ethanol} (see Section \ref{EthanolArch}) has been evaluated through running a load-aware AsC mechanism that aims to balance the number of associated clients among APs.
When a client requests to join an AP with higher load, the controller drops the request to force the client look for another AP.
A simple testbed with two APs and up to 120 clients shows that the maximum difference between the number of clients associated to APs is two, which is due to the concurrent arrival of association requests.



%\textbf{vBS.}
%The vBS \cite{vBS} architecture (please see Section \ref{Archs}) also proposes a fast handoff mechanism to reduce reconnection delay.
%Experimental results show that vBS can perform handoff in less than 65ms without any packet drop.


%------------------------------------------------------------------ COLOR
\subsubsection{\textbf{Individual Optimization through Centrally-Made Decisions}}
\label{AM-InvidualOpt}
In this section we review AsC mechanisms that employ client steering through decisions generated centrally. 
These mechanisms, however, do not define a global optimization problem; thereby they do not take into account the effect of an association on other clients/APs.
Due to the individual nature of association control, these mechanisms only improve the overall network performance, and cannot be used to enforce fairness.



%2008
% Individual - Client steering - ap load balancing
\textbf{DenseAP.}
\label{DenseAP-AM}
The AsC mechanism proposed by DenseAP \cite{DenseAP} (see Section \ref{DenseAP_arch}) works as follows.
The \textit{available capacity} metric is defined to rank all the APs a client could be associated with. 
A client associates to the AP with highest available capacity. 
The available capacity of $AP_{i}$ operating on channel $ch_{i}$ is defined as follows,
\begin{equation}
AC_{AP_{i},c_{j}}^{ch_{i}} = F_{AP_{i}}^{ch_{i}} \times r_{AP_{i},c_{j}}^{ch_{i}},
\end{equation} 
where $F_{AP_{i}}^{ch_{i}} $ is the \textit{free airtime} of $AP_{i}$ on channel $ch_{i}$, and $r_{AP_{i},c_{j}}^{ch_{i}}$ is the \textit{expected transmission rate} of $c_{j}$ when communicating with $AP_{i}$.
Free airtime is estimated by measuring the amount of time that the MAC layer contends for channel access to send a high-priority packet.
The expected transmission rate is estimated using the RSSI of probe request frames received at the AP. 
A mapping table is used for this purpose.
APs hear the probe requests of clients and send reports to a controller. 
APs also measure their free airtime and report it to the controller. 
The controller selects the AP with highest available capacity for each client to associate with. 
To instruct a client associate with the selected AP, only the selected AP responds to the client's probe message. 

DenseAP also proposes a dynamic load balancing algorithm that periodically decides about associations.
In particular, the controller checks the free airtime of APs every minute. 
An AP is \textit{overloaded} if its free airtime is less than 20\% and it has at least one associated client. 
If an overloaded AP exists, the controller considers its clients as the potential candidates for association with APs experiencing lower load.
% REMOVAL CANDIDATE
A candidate AP must satisfy these conditions: 
(i) the expected transmission rate of clients when associated with the new AP must not be lower than the current transmission rate, and (ii) the free airtime of new AP must be at least 25\% more than the current AP. 
During each decision period, at most one client is allowed to be associated with a new AP, and two consecutive associations for a client is prohibited to avoid the ping-pong effect.

%The performance of DenseAP's AsC mechanism is evaluated through empirical experiments using the 5GHz band (802.11a) with 8 channels. 
Empirical evaluations show a 40\% to 70\% increase in per-client throughput, compared to SSF. 
Moreover, the authors conduct a small experiment using three clients and two APs to show that the load balancing algorithm improves the throughput of clients by more than 200\%. 



% 2012
% Individual - Client steering - fair bandwidth allocation 
\textbf{Odin's Load Balancing (OLB).}
This load-balancing mechanism \cite{Odin2} periodically (every minute) inquires APs to collect the RSSI relationship between clients and APs, and then LVAPs are evenly distributed between APs to balance their loads.
%Using 10 APs and 32 clients, the efficiency of this mechanism on TCP throughput has been evaluated. %when a fixed rate 6Mbps is used.
The evaluations on a testbed with 10 APs and 32 clients show that around 50\% and 15\% of clients were able to receive a fair amount of throughput when the proposed mechanism was enabled and disabled, respectively.



%2013 Nov, 2015 June
% Individual - Client steering - energy efficiency
\textbf{EmPOWER.} 
\label{EmPowerAM}
This mechanism \cite{EmPOWER,Primitives} relies on client steering to improve the energy efficiency of APs.
APs are partitioned into clusters, where each cluster has one master and multiple slaves. 
Master APs are always active, and they are manually selected during the network deployment phase to provide a full coverage.
Slave APs are deployed to increase network capacity.
These APs are turned on/off using the finite state machine (FSM) depicted in Figure \ref{fig_FSM_EmPOWER}.  
%
\begin{figure}[!t]
	\centering
	\includegraphics[width=0.6\linewidth]{EmPOWER-energy-manager-FSM.pdf}
	\caption{Finite state machine (FSM) of energy manager for a slave AP in EmPOWER \cite{EmPOWER}.}
	\label{fig_FSM_EmPOWER}
\end{figure}
%
%\todo[inline, color=cyan]{what is the clustering strategy? \\ \textit{There is no specific clustering strategy in the paper. I added more explanation to clarify it a little more.}}  
In the ON mode, all wireless interfaces of the AP are on. 
In OFF mode, only the Energino \cite{Energino} module of the AP is on.  

Two metrics are defined for a slave $AP_i$ belonging to cluster $clu(AP_i)$: $\mathcal{C}_{clu(AP_i)}$ is the number of clients in the cluster, and $\mathcal{C}^{min}_{clu(AP_i)}$  is the minimum required number of clients in cluster $clu(AP_i)$ to keep $AP_i$ active. 
A slave $AP_i$ transitions from ON mode to OFF mode if: (i) the number of its cluster's clients is less than $\mathcal{C}^{min}_{clu(AP_i)}$, and (ii) $AP_i$ has been inactive (i.e., no client associated) for at least $T_{idle}$ seconds. 
Also, $AP_i$ transitions from OFF mode to ON mode if: (i) the number of its cluster's clients is at least $\mathcal{C}^{min}_{clu(AP_i)}$, and (ii) $AP_i$ has been OFF for at least $T_{offline}$ seconds. 

The mobility manager associates a client to a new AP with higher SNR. 
However, to establish a balance between performance and energy consumption, the mobility manager may associate a client to an AP with lower SNR but smaller $\mathcal{C}^{min}_{clu(AP_i)}$.  
In this way, the energy manager is able to turn off the APs with higher $\mathcal{C}^{min}_{clu(AP_i)}$  in order to decrease energy consumption. 
Re-association is performed if there is a better AP in terms of SNR and $\mathcal{C}^{min}_{clu(AP_i)}$. 
By relying on the Odin \cite{Odin2} APIs, the authors showed that this AsC mechanism was implemented as a Java network application with only 120 lines of codes.


%2016 July
% Individual - Client steering - ap load balancing
\textbf{Adaptive mobility control (AMC).} 
\cite{mob-Essex-2016} shows that using a fixed RSSI threshold by clients results in an unbalanced load of APs.
% idea
They propose a heuristic algorithm which uses RSSI and traffic load of APs to provide dynamic hysteresis margins on AP traffic load level. 
The load of $AP_{i}$ is defined as,
%
\begin{equation}
\label{eq:eq18}
L_{AP_{i}}=\left\{\begin{array}{ll}
B\textrm{\ \ \ \ \ \ \ \ \ \quad\quad\quad\quad\quad\quad\quad if  $|\mathcal{C}_{AP_{i}}|=0$}\\
0.8\times B +0.2\times |\mathcal{C}_{AP_{i}}| \textrm{\quad\ \  if $|\mathcal{C}_{AP_{i}}|>0$}
\end{array}\right.
\end{equation}
where $|\mathcal{C}_{AP_{i}}|$ is the number of clients associated with $AP_{i}$, and $B$ is channel busy time.
The algorithm defines three thresholds on AP load and RSSI: low, medium, and high.
%These thresholds are determined through applying experimental tests to measure the effect of AP load and RSSI value on network throughput. 
Using these thresholds, a client is associated with a new AP that satisfies one of these conditions: (i) higher signal strength and lower load, (ii) significantly higher signal strength and slightly higher load, or (iii) significantly lower load and slightly lower signal strength.
% QUESTIONS (Second phase) --> COMPLETED
% ^^- AP load metric  
% ^^ - hysteresis margins
% ^^ - How is the algorithm working?
% ^^ - adaptive margins ==> based on which parameter, it is adaptive?
% QQQQQQQQQQQQQQQQQQQQQQQQQQQQQQQQQQQQQQQQQQ

The proposed adaptive mobility manager has been implemented using Odin.
Empirical results show more than 200\% improvement in TCP throughput, compared to SSF.





%2016 June
% Individual - Client steering - reduce delay
\textbf{Associating to good enterprise APs (AGE).}% demand-agnostic
\label{AGE}
The main objective of AGE \cite{WiFiSeer} is the reduction of clients' packet exchange delay over wireless links through client steering.
AGE has two main phases: learning and AP selection. 
%The learning phase uses a training set obtained during network operation for a week every three months.
During the learning phase, the performance metrics and environmental factors are pulled every minute from all APs using SNMP \cite{SNMP}. 
These metrics include AP-client RTT, RSSI, SNR, number of associated clients, channel number, frequency band, AP location, day of week, and time.
%The wireless latency of each client is calculated every minute through a technique called \textit{ping2}. 
%The authors argue that measuring wireless latency through sending only one ping packet might be inaccurate due to the wake up delay of NIC. 
%Therefore, ping2 uses two consecutive ping packets.
%The first ping is used to wake up the client's NIC, if it is in energy-saving mode. 
%The second ping is sent to calculate latency. 

Using the collected training set, the authors use the \textit{random forest} \cite{random-forest} technique to generate a two-class learning model for classifying APs into \textit{high latency} and \textit{low latency}.
Figure \ref{fig_AGE} shows an overview of AGE and its two main components: (i) the \textit{AGE app} that is installed on each client's mobile phone, and (ii) the \textit{AGE controller} that uses the random forest model to classify the APs in the vicinity of each client.
AGE operates as follows:
\begin{itemize}
	\item The AGE application on a client device sends an AGE request (including the list of achievable APs) to the AGE controller, periodically. 
	%AGE requests are sent every 5 and 20 minutes when the device's screen is on and off, respectively.
	\item The AGE controller pulls the SNMP data of the APs requested by AGE app.
	\item The latency class of each AP is predicted using the learning technique mentioned earlier.
	\item The AGE controller informs the AGE application about the best AP nearby.
	\item The client re-associates with a new AP using AGE app. 
\end{itemize}

The authors deployed AGE at Tsinghua University campus where over 1000 devices used the network for 2.5 months. 
Measurements confirmed that the wireless data exchange delay of more than 72$\%$ of clients has been reduced by over 50$\%$.

\begin{figure}[!t]
	\centering
	\includegraphics[width=0.9\linewidth]{AGE.pdf}
	\caption{Associating to Good Enterprise APs (AGE)\cite{WiFiSeer}. Each client is instructed to connect to the AP that provides minimum delay.}
	\label{fig_AGE}
\end{figure}




%------------------------------------------------------------------ COLOR
\subsubsection{\textbf{Global Optimization through Centrally-Made Decisions}}
\label{AM-GlobalOpt}
In this section we review AsC mechanisms that employ client steering by formulating a problem that aims to optimize performance parameters globally.
These mechanisms propose heuristics to solve the NP-hard problems that usually aim to achieve network-wide fairness.



%2004-2007
% Global - Client steering - fair bandwidth allocation
\textbf{Association Control for fairness and load balancing (ACFL).} 
In \cite{F-LB-AsscCtrl-2004} and \cite{F-LB-AsscCtrl-2007}, the authors address the unbalanced load of APs as a result of using SSF or LLF.
%However, their fairness measure is max-min fairness.Max-min throughput fairness can significantly reduce aggregate throughput in multi-rate WLANs. The max-min time fairness problem they consider is intended for single-rate WLANs.
%from the paper:  The collected infor-mation is reported to a network operation center (NOC) which runs our algorithm to come up with the user-AP association de-cisions.
%They consider the bandwidth constraints of APs in both wireless and wired links.
To address this challenge, an AsC problem is formulated to establish the max-min bandwidth fairness among APs. 
Intuitively, the load of a client on its associated AP is inversely proportional to the effective bit rate of AP-client link. 
The load of $AP_{i}$ (denoted as $L_{AP_{i}}$) is modeled as the maximum of the aggregated loads of its wireless and wired links generated by all clients $c\in \mathcal{C}$, as follows,
\begin{equation}
L_{AP_{i}}=\max\left\{ \sum_{\forall c_{j}\in \mathcal{C}} \frac{\omega_{c_{j}} \times X_{AP_i, c_j}}{r_{AP_i, c_j}},  \sum_{\forall c_{j}\in \mathcal{C}} \frac{\omega_{c_{j}} \times X_{AP_{i},c_j}}{R_{AP_{i}}} \right\}
\end{equation}
where $r_{AP_i, c_j}$ is the transmission rate between $AP_{i}$ and $c_{j}$, $R_{AP_{i}}$ is the transmission rate of wired interfaces of $AP_{i}$, $X_{AP_{i},c_{j}}\in\{0,1\}$ is the association state of $c_{j}$ to $AP_{i}$, and $\omega_{c_{j}}$ is the traffic volume size of client $c_{j}$. 
In other words, the load of $AP_{i}$ is defined as the period of time this AP requires to handle the traffic of its associated clients. 
$AP_{i}$ provides bandwidth $X_{AP_{i},c_{j}} \times \omega_{c_{j}}/L_{AP_{i}}$ to client $c_{j}$. 

Two approximation algorithms are introduced to solve the formulated NP-hard max-min fair bandwidth allocation problem. 
The first algorithm solves the problem for unweighted greedy clients. 
The second algorithm proposes a solution for weighted and limited throughput demand of clients.
The algorithms are run periodically by a controller to update client associations. 
Simulation results show over 20$\%$ improvement in terms of average per-client bandwidth, compared to SSF and LLF.

%\todo[inline,color=cyan]{sentence not clear- please rewrite}  
%\todo[inline,color=cyan]{why did you remove the formulas? can you make the explanation more complete?\\ \textit{this paper is so confusing. I tried to add more explanation.}}  







%2014
% Global - Client steering - fair bandwidth allocation 

\textbf{Association control for proportional fairness (ACPF).}
\cite{Proportional-Fairness-AP-2014} argues that using throughput-based fairness (e.g., \cite{Time-based-basic-1, Time-based-basic-2}) in multirate networks leads to low overall network throughput because clients with a low bit rate can occupy the channel longer than those with higher rate. 
The authors investigate the problem of achieving proportional fairness by introducing the following objective function formulated based on the effective bandwidth of clients:
\begin{equation}
\sum_{\forall c_{j}\in \mathcal{C}} \rho_{c_{j}} \times  \log (\beta_{c_{j}}),
\end{equation}
where $\rho_{c_{j}}$ and $\beta_{c_{j}}$ are the priority and effective bandwidth of client $c_{j}$, respectively. 
$\beta_{c_{j}}$ is defined as 
\begin{equation}
\sum_{\forall AP_{i}\in \mathcal{AP}}X_{AP_{i},c_{j}}\times t_{AP_{i},c_{j}}\times r_{AP_{i},c_{j}}, 
\end{equation}
where $X_{AP_{i},c_{j}}\in\{0,1\}$ is the association index, $t_{AP_{i},c_{j}}\in[0,1]$ is the effective normalized time, and $r_{AP_{i},c_{j}}$ is the PHY rate between $c_{j}$ and $AP_{i}$. 
Achieving fairness through association is formulated as a linear-programming problem to maximize the objective function.
%\todo[inline, color=cyan]{"The client traffic-based proportional fairness"- this is not a correct sentence. not clear.}  
A centralized algorithm, called \textit{non-linear approximation optimization for proportional fairness} (NLAO-PF), is proposed to solve the optimization problem.
%This algorithm computes client-AP association relations and transmission time of clients. 
It has been proven that the approximation ratio of NLAO-PF is 50\%.

The authors used the OMNet++ \cite{OMNET} simulator to evaluate the effectiveness of NLAO-PF compared to cvapPF \cite{cvapPF} in terms of average throughput. 
The average improvement is reported as 18.8$\%$ and 35$\%$ for uniformly distributed clients as well as clients in a hotspot area, respectively.



%2014
% Global - Client steering - fair bandwidth allocation 

\textbf{Association control with heterogeneous clients (ACHC).}  % demand-agnostic?
Experimental analysis of the negative impacts of legacy clients (802.11a/b/g) on 802.11n clients has been reported by \cite{802.11n-AP-Association-2014}.
The authors propose a two-dimensional Markov model to calculate the uplink and downlink throughput of heterogeneous clients. 
MAC efficiency of a client $c_{j}$ is defined as follows,
\begin{equation}
\label{MACeffMetr}
\Upsilon({c_{j}}) = \frac{Th_{c_{j}}^{up}+Th_{c_{j}}^{down}}{\min\{1,\; p_{c_{j}}^{up}+p_{c_{j}}^{down}\}\times r_{AP_{i},c_{j}}},
\end{equation}
where $Th_{c_{j}}^{up}$ and $Th_{c_{j}}^{down}$  are the estimated uplink and downlink throughput of client $c_{j}$, respectively;
$p_{c_{j}}^{up}$ and $p_{c_{j}}^{down}$ are, respectively, the uplink and downlink traffic probabilities of $c_{j}$; 
$r_{AP_{i},c_{j}}$ is the optimal PHY rate between $c_{j}$ and $AP_{i}$. 
AsC is formulated through an optimization problem to provide a bandwidth proportional to each client's achievable data rate, which is obtained by maximizing $\sum_{\forall c_{j}\in \mathcal{C}} \log \Upsilon({c_{j}})$.

Two heuristic dynamic AsC algorithms are proposed to solve the optimization problem: \textit{FAir MAC Efficiency} (FAME) and \textit{Categorized}. 
Instead of maximizing all MAC efficiencies, these algorithms maximize the minimum MAC efficiency.
To run FAME, the data rate and traffic load of all clients in nearby BSSs must be collected from each client's point of view.
With \textit{Categorized}, the type of all clients in nearby BSSs must be collected as well.
However, \textit{Categorized} is less sensitive to network dynamics because it tries to maximize the number of similar-standard 802.11 clients connected to each AP.
Therefore, \textit{Categorized} takes advantage of dense deployment and alleviates the performance degradation of 802.11n clients that is caused by the presence of legacy clients. 

Simulation and testbed experiments confirmed the higher performance of \textit{FAME} and \textit{Categorized} compared to SSF and ACFL in terms of TCP and UDP throughput, MAC efficiency and aggregated throughput of clients supporting different standards.
Although the conducted experiments use distributed execution of \textit{FAME} and \textit{Categorized}, collecting the information required by these algorithms imposes a high burden on clients. 
Therefore, these algorithms are mostly suitable for SDWLANs. 
%Furthermore, it is required to change beacon packet format as data collection is performed through exchanging beacon packets.


% demand-agnostic
%2015
% Global - Client steering - fair bandwidth allocation 

\textbf{Association control and CCA adjustment (ACCA).} 
Two central algorithms for AsC and CCA threshold adjustment are proposed in \cite{Association-CCA-2015} for dense AP deployments.
%\todo[inline, color=cyan]{why did you call it a joint approach while the two algorithms are separate? do they work together? \\ \textit{thanks for your precise comment. It is not joint mechanism.}}  
The basic idea of the AsC mechanism is to utilize the SINR of a client perceived by an AP in order to measure the interference (congestion) level of APs. 
The network is modeled as a weighted bipartite graph (WBG), and the AsC problem is formulated as a maximum WBG matching combinatorial optimization problem. 
The weights of edges in WBG are the uplink SINR values. 
The optimization problem's objective is to maximize the sum of the clients' throughput, where the throughput upper-bound is computed using the Shannon-Hartley formula \cite{rappaport1996wireless}.
Per-BSS CCA level is adjusted by using a constraint of the optimization problem based on the cell-edge SINR of each BSS. 
The optimization problem has been solved using Kuhn-Munkres \cite{Kuhn} assignment algorithm.
%The proposed centralized technique requires global knowledge about all clients and APs, which is collected by the functionalities of IEEE 802.11k \cite{802.11k}. 
% Details of graph modeling
% Details of optimizaiton problem

MATLAB simulations show a 15 to 58\% improvement at the 10th percentile of cumulative distribution of clients throughput, compared to SSF.
%In addition, the improvement level is higher for networks with randomly deployed APs, compared to uniformly-spaced AP placement.
Furthermore, the CCA adjustment algorithm significantly improves the throughput of cell-edge clients, compared to the network with fixed CCA threshold.
%\todo[inline, color=cyan]{please correct the last sentence, not clear at all. any numerical values? \\ \textit{More explanations are added.}}  



%2015
% Global - Client steering - fair time fairness allocation 

\textbf{Demand-aware load balancing (DALB).} % more focus on scheduling rather than association (transmission time adjustment) --> it's not based on CSMA/CA
\cite{Demand-aware-LB-Association-15} formulates joint AP association and bandwidth allocation as a mixed-integer nonlinear programming problem that includes the bandwidth demand of clients. 
The objective is to maximize the aggregated bandwidth of clients, which establishes a trade-off between throughput gain and time-based user fairness. 
The key idea of this work is the inclusion of clients' bandwidth demands in the computation of APs' load. 
For a client $c_{j}$, bandwidth demand is defined as $\beta_{c_{j}}=S_{c_{j}}/{T}$, where $S_{c_{j}}$ is the size of data received for client $c_{j}$ by its associated AP during the allocable transmission time of the AP, i.e., $T$. 
The transmission time demand of $c_{j}$ associated to $AP_{i}$ is defined as $T_{AP_{i},c_{j}} = \beta_{c_{j}}\times T/r_{AP_{i},c_{j}}$.

The optimization problem's objective is defined as follows,
\begin{equation}
\label{demandEqu}
\sum_{\forall c_{j} \in \mathcal{C}} \log\sum_{\forall AP_{i} \in \mathcal{AP}} \frac{X_{AP_{i},c_{j}}\times r_{AP_{i},c_{j}} \times t_{AP_{i},c_{j}}}{T},
\end{equation}
where the transmission time allocated to $c_{j}$ associated with $AP_{i}$ is expressed by $t_{AP_{i},c_{j}}$. 
The authors proved that maximizing Equation \ref{demandEqu} subject to a constraint on the transmission time demand of clients, i.e., $0 \le t_{AP_{i},c_{j}} \le T_{AP_{i},c_{j}}$, leads to proportional time fairness among clients.

A two-step heuristic algorithm is introduced to solve the optimization problem. 
The first step associates a client to an AP with the lowest allocated transmission time at each iteration until all clients are associated. 
At each iteration, the algorithm selects a client with the largest bandwidth demand. 
The second step schedules the transmission time of APs to establish proportional transmission time fairness among clients. 

Simulations show a 23$\%$ improvement in clients' throughput compared to SSF, ACPF and NLB \cite{NLB-13}.
In addition, this improvement has been achieved without sacrificing fairness. 
In particular, while bandwidth fairness is slightly better than that of SSF, ACPF and NLB, time fairness is significantly improved compared to these algorithms.





%2016
% Global - Client steering - maximizing throughput of clients

\textbf{Migration-cost-aware association control  (MCAC).} 
\label{Migration-DAM}
\cite{Migration-DAM} investigates the problem of max-min fairness subject to a migration cost constraint.
An integer linear programming problem is formulated with the aim of maximizing the minimum throughput of clients in order to establish max-min fairness.
The migration cost of clients is the main constraint of the formulation.
The load of $AP_{i}$ is defined as $L_{AP_{i}}=\sum_{\forall c_{j} \in \mathcal{C}} X_{AP_{i},c_{j}}/r_{AP_{i},c_{j}}$.
Assuming the achievable throughput of clients is proportional to the inverse of the load of the associated AP (i.e., $1/L_{AP_{i}}$), maximizing the minimum throughput of clients is equivalent to minimizing $\max_{\forall AP_{i} \in \mathcal{AP}}L_{AP_{i}}$. 
%Consequently, the objective of the optimization problem is defined as $\min \max_{i} H_i$, where $H_i$ is the load of AP $i$ after client re-association. 
The migration cost constraint is defined as, 
\begin{equation}
\sum_{\forall AP_{i} \in \mathcal{AP}}\;\sum_{\forall AP_{l} \in \mathcal{AP}} (Y_{AP_{i},AP_{l}}\times M_{c_j}) \le M,
\end{equation}
where $Y_{AP_{i},AP_{l}}\in\{0,1\}$ indicates whether a client $c_{j}$ currently associated with $AP_{i}$ will be associated with $AP_{l}$, $M_{c_{j}}$ is the migration cost of client $c_{j}$ from $AP_{i}$ to $AP_{l}$, and $M$ is the total permissible migration cost.

The authors showed that the proposed problem is NP-hard, and they proposed an approximation algorithm, named the \textit{cost-constrained association control algorithm} (CACA), to solve the optimization problem. 
In this algorithm, the problem is divided into two sub-problems: client removal and client re-association. 
In the first sub-problem, a subset of clients are removed from their current AP with the aim of minimizing the maximum AP load, considering the migration cost constraint. 
%When migration costs are equal, the migration cost constraint is simplified as a limitation on the number of migrating clients, i.e., no more than $K^{\prime}=K/c$ clients can be removed. 
%These clients impose the highest load on the maximum-loaded AP.
In the second sub-problem, the algorithm performs client re-association to minimize the load of maximum-loaded AP.

NS3 \cite{NS3}  simulations and experimental results show improvements in terms of the number of re-associations, throughput, loss ratio and end-to-end delay, compared to SSF, ACPF and GameBased \cite{GameBased}. 
%As SSF, PFairness and GameBased do not consider the number of associated clients as a limitation, the proposed cost-constrained association control algorithm shows lower number of re-associations. 



%\textbf{Joint optimization of energy consumption and interference (JOEI).}
%A short-term mixed-integer programming (SMIP) optimization problem is proposed in \cite{DCA-2016-1} to jointly improve energy efficiency and interference through client-AP associations, dynamic channel selection and controlling the operation modes of APs. 
%Figure \ref{fig_JITwlan} illustrates the block diagram of the proposed optimization problem. 
%The ON and OFF states of APs can be controlled by a controller. 
%The objective function is defined as the average power consumption of APs during a long period. 
%%\todo[inline,color=cyan]{$\checkmark$ Each client should be associated with an AP, and each AP communicates on a channel with all its associated clients -- not clear? \\
%%\textit{- It is not needed to mention this constraint. Because it is the basic assumption of all association algorithms.}} 
%The channel assignment constraint does not allow assigning same channel to two neighboring APs. 
%Furthermore, the client queue stability constraint prevents queue overflow in clients through considering the demand and departure rate of clients' traffic.  
%%This work focuses on downlink communications (i.e., from AP to clients) in order to decrease the average power consumption of APs. 
%The network is modeled similar to \cite{Wcolor-2005} (see WCCA in Section \ref{WCCA}) through an undirected graph where vertices represent APs and an edge represents a potential interference between two APs.
%Since SMIP is an NP-hard problem, authors propose a polynomial-time linear programming relaxation to obtain the values of control variables.
%%Since SMIP is an NP-hard problem, authors propose a polynomial-time linear programming relaxation, named \textit{Just-In-Time WLAN} (JIT-WLAN), to obtain the values of control variables.
%The controller periodically solves the problem to perform (i) AP-client association control, (ii) AP operation mode control, and (iii) channel assignment.
%%The SMIP problem is solved at each time instance that leads to jointly minimize the long-term power consumption of APs and interference (by assigning non-overlapping channels to neighboring APs) without sacrificing clients demands. 
%%
%\begin{figure}[!t]
%	\centering
%	\includegraphics[width=0.8\linewidth]{InfoCom-JIT-WLAN.pdf}
%	\caption{Block diagram of the optimization problem proposed by JOEI \cite{DCA-2016-1}.}
%	\label{fig_JITwlan}
%\end{figure}
%%They demonstrate the performance and practicability of proposed algorithm through extensive simulations and an SDN-based framework. 
%%Each AP communicates with a central controller.
%%The controller runs the polynomial-time JIT-WLAN algorithm to solve the proposed optimization problem periodically (based on the decision making period). 
%
%For stationary clients, testbed evaluations show $8\%$ improvement in aggregate power consumption without sacrificing the throughput of clients, compared to always-on approach.
%For mobile clients, throughput is improved from 15 to 20 Mbps. 
%%However, using the always-on approach, throughput drops from 15 to 10 Mbps. 
%%Due to practical issues related to the overhead of re-association in channel reassignment, the channel assignment module is omitted in SDWLAN-based testbed experiments.



 
%2017
% Global - Client steering - minimizing the delay of download flows

\textbf{QoS-driven association control (QoSAC).}
\label{QoS-DAM}
\cite{flow-level-DAM} investigates the problem of flow-level association to address the QoS demands of clients.
Specifically, the backhaul capacity of APs is included in the association decision making.
This work proposes a mechanism for concurrent association of a client to multiple APs and supports flow-level routing of traffic. 
This mechanism facilitates dedicated management of each flow. 
For example, a client may use $AP_{1}$ for video streaming while using $AP_{2}$ to upload a file.

The main objective is to minimize the \textit{average inter-packet delay} of individual download flows. 
The inter-packet delay for a client associated to an AP is formulated based on the bi-dimensional unsaturated Markov model proposed in \cite{802.11n-AP-Association-2014}. 
An optimization problem is formulated to minimize the sum of the average inter-packet delay of all APs. 
The backhaul capacity of an AP should not be smaller than the sum of the arrival traffic rates of the AP, i.e., the sum of all download flow rates. 
This is included as a constraint of the optimization problem.
%\todo[inline, color=cyan]{not clear what is has to do with flow level? }
%\todo[inline, color=cyan]{not clear how it benefits from multiple associations \\ I added one sentence to clarify it more.}    
The proposed optimization problem is interpreted into a supermodular set function optimization \cite{supermodular}, which is an NP-hard problem. 
This problem has been solved using two heuristics: (i) greedy association, and (ii) bounded local search association. 
The greedy algorithm associates a client with the AP that minimizes the total inter-packet delay in each iteration.
In this algorithm, the association loop is continued until all clients are associated. 
The bounded local search is a polynomial-time algorithm similar to the algorithm proposed in \cite{boundedAlg}.
%\todo[inline, color=cyan]{add a high level explanation of these algorithms}  

Simulation results confirm the reduced average inter-packet delay achieved with this mechanism, compared to SSF and FAME \cite{802.11n-AP-Association-2014}. 
The delay is considerably reduced when the backhaul capacity is limited and clients form hotspots.

%From their paper:
% With the above features, we further consider client association in the high-density scenario. In this paper, “high-density”
%mainly implies the overlapped basic service set (OBSS) case, in which all the APs and clients are located within a certain coverage area and operate on the same channel.



%TABLE DAM
\begin{table*}
	\centering
	\scriptsize
	\caption{  Comparison of Association Control (AsC) Mechanisms  }
	\label{DAMtable} 
	\def\arraystretch{1}
	\begin{tabular}{|c|c|c|c|c|c|c|c|c|c|c|c|}
		\Xhline{3\arrayrulewidth}
		%	\textbf{Ref.} & \multicolumn{2}{|c|}{OneTwoThree} & \multicolumn{2}{|c|}{OneTwoThree}&\textbf{Channels} & \textbf{Dynamic/Static} & \textbf{\multicolumn{2}{|c|}{OneTwoThree}}\\ \Xhline{3\arrayrulewidth}
		%	
		\multirow{3}{*}{\textbf{Mechanism}}& \multicolumn{2}{c|}{Objective} & \multicolumn{3}{c|}{Optimization Scope}&\multirow{3}{*}{\textbf{Decision Metric}}& \multicolumn{2}{c|}{Traffic Awareness}&   \multicolumn{2}{c|}{ Performance Evaluation }\\ \cline{2-6}\cline{8-9}\cline{10-11}
		
		&\textbf{\makecell{Seamless \\ Handoff}}&\textbf{\makecell{Client \\Steering}}&\textbf{\makecell{Cent. Gen.\\ Hints}}&\textbf{\makecell{Indv. Opt. \\Cent. Made\\ Decisions }}&\textbf{\makecell{Glob. Opt. \\ Cent. Made\\ Decisions }}&& \textbf{Downlink}& \textbf{Uplink}& \textbf{Simulation} & \textbf{Testbed}\\ \Xhline{3\arrayrulewidth}
		%1
		%SSF\cite{SSF}&$\checkmark$&$\times$&RSSI&$\checkmark$&$\times$&$\times$&$\times$&$\checkmark$\\\hline
		%	
		\makecell{CUWN \cite{Cisco}} & $\checkmark $ & $\times $ & $\times $ & & $\times$&RSSI&$\times$&$\times$ &$\times$&$\times$\\\hline
		%		
		\makecell{OMM \cite{Odin2}} & $\checkmark $ & $\times $ &$\times $& & $\times$&RSSI&$\times$&$\times$ &$\times$&$\checkmark$\\\hline
		%2
		$\AE$therFlow \cite{AEtherFlow} & $\checkmark $ & $\times $ &$\times $& & $\times$&RSSI&$\times$&$\times$&$\times$&$\checkmark$\\\hline		
        %		
		BIGAP  \cite{BIGAP} & $\checkmark$ & $\times $ &$\times $ &  & $\times$&RSSI&$\times$&$\times$&$\times$&$\checkmark$ 	  \\\hline  		
        %        %        
        %        
        %			
		BestAP \cite{BEST-AP}& $\times$ &  \makecell{Bandwidth\\ Improvement }  &$\checkmark$& & $\times$&\makecell{Available\\ bandwidth}&$\checkmark$&$\times$&$\times$&$\checkmark$\\\hline
		%        
		%
		Ethanol \cite{Ethanol}& $\times$ & \makecell{AP Load \\ Balancing}  &$\checkmark $ &  & $\times$ &Number of clients &$\times$&$\times$&$\times$&$\checkmark$\\\hline		
		%
		DenseAP \cite{DenseAP}& $\times$ & \makecell{AP Load \\ Balancing} &$\times$ & \makecell{Overall\\ performance} & $\times$  & \makecell{Available\\ bandwidth}&$\checkmark$&$\times$&$\times$&$\checkmark$\\\hline
		%
        \makecell{OLB \cite{Odin2}} & $\times$ & \makecell{AP Load \\ Balancing} &$\times$& \makecell{Overall\\ performance} & $\times$ &\makecell{RSSI,\\Number of clients}&$\times$&$\times$ &$\times$&$\checkmark$\\\hline   
        %     
		%
		EmPOWER \cite{EmPOWER}& $\times $ & \makecell{Energy\\Efficiency} &$\times $ & \makecell{Overall\\ performance}& $\times$ &\makecell{RSSI,\\Number of clients}&$\times$&$\times$&$\times$&$\checkmark$\\\hline
        %		
		AMC \cite{mob-Essex-2016}& $\times$ &  \makecell{AP Load \\ Balancing}   &$\times$ & \makecell{Overall\\ performance} & $\times $ &\makecell{RSSI, Free airtime}&$\checkmark$&$\times$&$\times$&$\checkmark$\\\hline
        %        
		AGE \cite{WiFiSeer}& $\times$ & \makecell{Reducing Packet \\ Exchange Delay} &$\times$ & \makecell{Overall\\ performance}& $\times$&\makecell{Latency}&$\times$&$\times$&$\times$&$\checkmark$\\\hline
		%		
		%		
		ACFL \cite{F-LB-AsscCtrl-2007}& $\times$ & \makecell{AP Load \\ Balancing}  &$\times$&  & \makecell{Max-min\\ fairness} &\makecell{Free airtime, \\Transmission rate}&$\checkmark$&$\times$&$\checkmark$&$\times$\\\hline
		%3
        %		
        %		
        %        
        %4        %
		ACPF \cite{Proportional-Fairness-AP-2014}& $\times$ & \makecell{Fair Bandwidth\\ Allocation}  &$\times$&  & \makecell{Proportional\\ Fairness} &\makecell{Available\\ bandwidth}&$\checkmark$&$\times$&$\checkmark$&$\times$\\\hline
		%        
		ACHC \cite{802.11n-AP-Association-2014}& $\times$ & \makecell{Fair Bandwidth\\ Allocation}  &$\times$&  & \makecell{Proportional\\ Fairness} &MAC efficiency&$\checkmark$&$\checkmark$ & $\checkmark$& $\checkmark$\\\hline   
		 %
		%				        
		ACCA \cite{Association-CCA-2015}& $\times$ & \makecell{Maximize\\ Throughput} &$\times$&  & \makecell{Overall\\ Performance} &Uplink SINR&$\times$&$\checkmark$&$\checkmark$&$\times$\\\hline
		%6
		DALB \cite{Demand-aware-LB-Association-15}& $\times$ & \makecell{Maximize\\ Throughput}  &$\times$&  & \makecell{Proportional\\ Fairness} &Transmission time&$\checkmark$&$\times$&$\checkmark$&$\times$\\\hline
		%
		%8
		MCAC \cite{Migration-DAM}& $\times$ & \makecell{Fair Bandwidth\\ Allocation}  &$\times$&  & \makecell{Max-min\\ Fairness}  &\makecell{Throughput,\\Re-association cost}&$\checkmark$&\tiny$\times$&$\checkmark$&$\checkmark$		\\\hline
		%13
		QoSAC \cite{flow-level-DAM} & $\times$ & \makecell{Reducing Packet \\ Exchange Delay} &$\times$ &  & \makecell{Overall\\ Performance}  &Inter-packet delay&$\checkmark$&$\times$&$\checkmark$&$\times$
		\\\Xhline{3\arrayrulewidth}
	\end{tabular}
\end{table*}







\subsection{Association Control: Learned Lessons, Comparison, and Open Problems}
\label{AscProblems}
Table \ref{DAMtable} presents and compares the features of the AsC mechanisms.
Although most of the AsC mechanisms focus on client steering, we should note that client steering and seamless handoff are interdependent.
For example, when association decisions are made centrally to balance the load of APs, a quick reaction to network dynamics requires seamless handoff; otherwise, the effect of load balancing will be compromised by the overhead and delay of re-associations.
In the following, we study the features of AsC mechanisms and identify research directions.

%<><><><><><><><><><><><><><><><><><><><><><><><><><><><><><><><><><><><><><><><><><><><><><><><>


\subsubsection{\textbf{Optimization Scope of Client Steering}} 
\label{asc_opt_scope_cl_ste}
Based on our review, we have classified the client steering approaches into three categories: (i) centrally-generated hints, (ii) individual optimization through centrally-made decisions, and (iii) global optimization through centrally-made decisions. 
Our review shows that most of the AsC mechanisms fall into the last two categories, where their optimization problems is classified as follows:
%
\begin{itemize}
	\item \textit{Overall network performance}: To improve the overall performance of clients and APs. 
	For example, QoSAC minimizes the sum of inter-packet delay for all APs, and ACCA  maximizes the sum of clients' achievable throughput.
	Note that improving overall performance does not necessarily require defining an optimization problem, as Section \ref{AM-InvidualOpt} shows.
    %
	%
	\item \textit{Max-min fairness}: To maximize the minimum performance. In other words, \textit{max-min} fairness indicates providing a client with more resources would not be possible without sacrificing the resources of other clients \cite{Bonald2006}. 
	For example, the objective function of ACFL and MCAC maximizes the throughput of minimum-throughput client.
	%
	\item \textit{Proportional fairness}: Clients occupy the channel proportional to their transmission rate.
	Our review shows that proposing these approaches is motivated by the challenges of achieving fairness in multi-rate networks. 
	Multi-rate is caused by both client heterogeneity (e.g.,  802.11a/b/n) and the rate adaptation mechanism of 802.11 standards.
	Since the 802.11 DCF protocol aims to provide an equal channel access probability for all contending clients, low-rate clients occupy the medium longer than high-rate clients.
	In this case, the max-min throughput fairness results in sacrificing the performance of high-rate clients, which leads to a problem called \textit{performance anomaly}. 
	Performance anomaly in multi-rate networks may result in total throughput degradation, even in comparison to SSF \cite{DAM-G-5,bellalta2016interactions,802.11n-AP-Association-2014,SSF}. 
	A well-known solution is proportional fairness, which is usually achieved by defining a logarithmic objective function and scheduling the transmission time of clients to be proportional to their transmission rate. 
	ACPF, ACHC and DALB are examples of AsC mechanisms that establish proportional fairness. 
	%
\end{itemize}



Using proportional fairness in multi-rate networks decreases the performance of low-rate legacy clients and may dissatisfy their QoS requirements. 
We suggest two approaches to address this problem: (i) hybrid fairness, and (ii) demand-aware proportional fairness. 
In hybrid fairness, a weighted fairness metric is defined by assigning weights for the max-min and proportional fairness metrics. 
Therefore, it is possible to adjust the weights based on the demand of low-rate legacy clients. 
In the second approach, the demand of clients is merged with their rates when formulating the problem.



\subsubsection{\textbf{Decision Metrics}}
\label{asc_dec_met}
The most important component of an AsC mechanism is its decision metrics, as they affect on both accuracy and overhead.
When using the AsC mechanism of 802.11 standard (i.e., SSF), a client associates with the AP that is providing the highest RSSI. 
The main shortcoming of SSF is that it does not recognize the load of APs and the demand of clients.
Therefore, AsC mechanisms introduce additional metrics in their decision process. 

The "Decision Metric" column of Table \ref{DAMtable} summarizes the decision metrics employed.
Generally, our review shows that the most popular decision metrics are RSSI, AP load, throughput of clients, and packet exchange delay experienced by clients. 
Balancing the load of APs is a widely-adopted optimization metric as it implicitly results in an improved performance of clients.
In the mechanisms relying on this metric, load is modeled through parameters such as the number of associated clients, AP throughput, and channel busy time. 
Although these mechanisms implicitly improve the performance of clients, explicit consideration of clients' traffic results in higher performance.

Based on this, we can classify AsC mechanisms into two groups:
%
\begin{itemize}
	\item \textit{Demand-agnostic.} 
	Refers to the AsC mechanisms that associate each client with the best AP nearby, without taking into account clients' demands.
	For instance, DenseAP, ACPF and BestAP associate each client with the nearby AP that is providing the highest available bandwidth, independent of the current demand of clients. 
	Similarly, AGE and MCAC associate each client with the AP that is providing the minimum delay, without considering the tolerable delay of clients in the decision making process. 	
	As a result, demand-agnostic approaches may perform unnecessary associations, which incur handoff overheads that result in performance degradation.	
	
	\item \textit{Demand-aware.}
	Refers to the AsC mechanisms that explicitly include the demand of clients in their decision making process.
	Unfortunately, the number of demand-aware AsC mechanisms (e.g., ACHC, and DALB) is very limited.
\end{itemize}
%

 %these works take into account the throughput demand

Throughput demand, in particular, is composed of uplink and downlink traffic.
Unfortunately, most of the AsC mechanisms assume downlink traffic is dominant when calculating their association decision metrics, as Table \ref{DAMtable} shows. 
However, the emerging applications of SDWLANs justify the importance of uplink traffic.
For example, medical monitoring, industrial process control, surveillance cameras, and interaction with cloud storage, require timely and reliable uplink communication \cite{MARS,RTwifi,REWIMO,IOT-future1, IOT-future2, singh_survey_2014}.


These discussions reveal that designing AsC mechanisms to support both uplink and downlink requirements is an open problem. 
Additionally, due to the resource-constraint nature of IoT devices, it is important to define, compute and include metrics, such as channel contention intensity and packet loss rate, in the decision making process.

In addition to the mentioned challenges, it is also important to identify and address the effect of hardware on the metrics used.
For example, the RSSI perceived by a tiny IoT device would be different from the value perceived by a multi-antenna smartphone, under the same condition.
Therefore, the controller cannot generate a realistic network map based on the information collected from clients.
Similarly, for environments with highly asymmetric links (for example caused by shadowing), the information collected in the controller based on the RSSI perceived at the APs does not reflect the real connectivity and interference relationship.
Understating the effect of metric evaluation irregularity on performance and the calibration of metrics and inclusion in the AsC mechanisms are open research areas.





\subsubsection{\textbf{Dynamicity and Overhead}}
\label{asc_dyn_overhead}
Dynamic and scalable AsC requires minimizing overhead from the following point of views:
\begin{itemize}
\item  \textit{Measurement delay}: the delay of measuring decision metrics must be short to reflect network dynamics;
%
\item  \textit{Bandwidth requirements}: the amount of bandwidth consumed for exchanging control data should be minimized to enhance scalability. 
Each association command requires communication between the controller and APs.
In addition, in a VAP-based architecture such as Odin, each association requires a VAP transfer, where the number of transfers depends on the number of associations.
Therefore, both the frequency and the scale of reassociation should be taken into account for various types of architectures;
%
\item \textit{Deployment cost}: it is ideal to define decision metrics that do not require AP or client modification.
For example, although BestAP  claims low measurement overhead in terms of delay and bandwidth, it requires the modification of APs and clients. 
\end{itemize}

Unfortunately, the existing AsC mechanisms do not take into account the effects of these three factors, the scalability of the proposed mechanisms has not been studied thoroughly, and their effect on the energy consumption of smartphones and resource constraint devices is unknown.
%For example, although DenseAP and Odin report low measurement overhead, a small testbed is used for their evaluation.


%Therefore, investigations are required to establish a balance between the accuracy and overhead of measuring network status metrics.
Although some AsC mechanisms propose their own measurement techniques, continuous and low-overhead network monitoring should be provided as a SDWLAN architectural service to facilitate the development and interoperability of control mechanisms.
For example, as we will see in Section \ref{CMmech}, dynamic ChA mechanisms share some of the metrics used by AsC mechanisms.
%To this end, a shared repository (managed by the OS) could be inquired by network applications.
We propose the followings to provide network applications with low-overhead and continuous measurement of metrics:
(i) designing metrics that can be used by multiple control mechanisms; 
(ii) efficient encoding of control data conveyed by a south-bound protocol, 
(iii) novel hardware/software techniques deployed on network infrastructure devices for passive (i.e., without introducing extra traffic) measurement of network status, 
and (iv) inference and prediction algorithms that extract new metrics and provide insight into the future status of the network.
Compared to distributed implementations, SDWLANs enable in-depth analysis of clients' traffic type and pattern by looking into packet headers \cite{gibb2013design,akyildiz2014roadmap,ng2015developing,Yoon2017,oliveira_characterizing_2016}.
This information could be employed to improve the accuracy of traffic prediction and interference modeling.
Specifically, while the existing approaches rely on the saturated demands of clients when estimating load and interference levels, the throughput and airtime of clients and APs strongly depend on the protocols used at the transport and application layers \cite{sinky_analysis_2015,IOT-future1,hobfeld_challenges_2012}.
For example, the periodic nature of an IoT device's uplink traffic could be exploited to improve AsC performance.
We believe that learning and prediction mechanisms must be employed by SDWLANs to provide network control applications (such as AsC and ChA) with meaningful, realistic, and predictive knowledge regarding network operation.

%In particular, interference modeling is usually done based on physical and MAC layer features in a saturated condition (nodes always have a packet to send). 
%On the other hand, traffic load modeling of APs is performed based on the measurement of throughput or airtime during a specific period and prediction of future demands. 


% REVISION


\textit{Hybrid design} is another approach towards achieving timely reactions against dynamics.
The overhead of network monitoring and control can be improved through hybrid control mechanisms, as proposed by SoftRAN \cite{gudipati2013softran}.
Due to the high dynamics of wireless networks, it is desirable to design control mechanisms that can make control decisions locally, while improving decisions based on the global network view as well.
This solution could also benefit from hierarchical controller topologies to make decisions at multiple levels, depending on the dynamicity of variations.
For example, AeroFlux (Section \ref{Archs}) proposes an architecture to enable the design of multi-level control mechanisms.
Such mechanisms make control decisions in a distributive manner, at local controllers or by a central controller, based on network dynamics, time constraints, and overhead of exchanging control messages.



% {\color{blue!50!black}
% \subsubsection{\textbf{Effects of Virtualization}}
% %REVISION
% Mobility management becomes more challenging when network virtualization is employed.
% When a client requires a new point of association due to its mobility, in addition to parameters such as fair bandwidth allocation, the available resources of APs should also be taken into account. 
% More specifically, for a client belonging to slice $n$, the AsC mechanism should ensure that after the association of this client with a new AP, the the QoS provided by slice $n$ and other slices is not violated.
% However, as this may require client steering, the cost of re-associations should be minimized.

% % In fact, when a user switches between different slices of a network, the controllers should synchronize the operation of these slices to ensure seamless connectivity.
% % However, achieving a low-overhead and fast collaboration among these controllers is a challenging task.
% % In addition, as mobility results in variations of link quality and capacity, dynamic resource allocation mechanisms become necessary to ensure fair allocation of resources.
% }


\subsubsection{\textbf{Security Considerations}}
%Although such metrics enable network applications to implement centralized control mechanisms, they should not compromise network performance because of security breaches.
As centralized mechanisms rely on global network information and usually aim to achieve a network-wide optimization, receiving malicious reports has a more severe effect on performance, compared to distributed control.
Verifying the legitimacy of network status reports and clients' demands have not yet been addressed by SDWLAN architectures and AsC mechanisms.
%

%


%%%%%%%%%%%%%%%%%%%%%%%%%%%%%%%%%%%%%%%%%%%%%%%%%%%%%%%%
%%%%%%%%%%%%%%%%%%%%%%%%%%%%%%%%%%%%%%%%%%%%%%%%%%%%%%%%
%%%%%%%%%%%%%%%%%%%%%%%%%%%%%%%%%%%%%%%%%%%%%%%%%%%%%%%%
%%%%%%%%%%%%%%%%%%%%%%%%%%%%%%%%%%%%%%%%%%%%%%%%%%%%%%%%
%%%%%%%%%%%%%%%%%%%%%%%%%%%%%%%%%%%%%%%%%%%%%%%%%%%%%%%%
%%%%%%%%%%%%%%%%%%%%%%%%%%%%%%%%%%%%%%%%%%%%%%%%%%%%%%%%
%%%%%%%%%%%%%%%%%%%%%%%%%%%%%%%%%%%%%%%%%%%%%%%%%%%%%%%%
%%%%%%%%%%%%%%%%%%%%%%%%%%%%%%%%%%%%%%%%%%%%%%%%%%%%%%%%