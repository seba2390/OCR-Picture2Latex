
\begin{thebibliography}{1}

\bibitem{NextGen-WiFi-Survey-2016} H. A. Omar, K. Abboud, N. Cheng, K. R. Malekshan, A. T. Gamage, and W. Zhuang, "A Survey on High Efficiency Wireless Local Area Networks: Next Generation WiFi," IEEE Communications Surveys Tutorials, vol. 18, no. 4, pp. 2315-2344, 2016.

\bibitem{WiFi-Devices} IC Insights. n.d. "Global Wi-Fi enabled equipment shipments from 2012 to 2017 (in billion units)". Statista. Available: https://www.statista.com/statistics/282512/global-wi-fi-enabled-equipment-shipments/. 

\bibitem{Cisco-statistics} "Cisco Visual Networking Index: Global Mobile Data Traffic Forecast Update, 2015–-2020 White Paper," Cisco. Available: http://www.cisco.com/c/en/us/solutions/collateral/service-provider/visual-networking-index-vni/mobile-white-paper-c11-520862.html.

\bibitem{Offloading} K. Lee, J. Lee, Y. Yi, I. Rhee, and S. Chong, "Mobile Data Offloading: How Much Can WiFi Deliver?," IEEE/ACM Transactions on Networking, vol. 21, no. 2, pp. 536-550, Apr. 2013.

\bibitem{Offloading-survey-2016} Y. He, M. Chen, B. Ge, and M. Guizani, "On WiFi Offloading in Heterogeneous Networks: Various Incentives and Trade-Off Strategies," IEEE Communications Surveys Tutorials, vol. 18, no. 4, pp. 2345–2385, 2016.

\bibitem{Experimenting802.11} P. Serrano, P. Salvador, V. Mancuso, and Y. Grunenberger, “Experimenting with commodity 802.11 hardware: Overview and future directions,” IEEE Commun. Surv. Tutorials, vol. 17, no. 2, pp. 671–699, 2015.

\bibitem{NGwlans-survey-2016} B. Bellalta, L. Bononi, R. Bruno, and A. Kassler, "Next generation IEEE 802.11 Wireless Local Area Networks: Current status, future directions and open challenges," Computer Communications, vol. 75, pp. 1-25, Feb. 2016.



\bibitem{DenseWLAN1} V. P. Mhatre, K. Papagiannaki, and F. Baccelli, “Interference Mitigation Through Power Control in High Density 802.11 WLANs,” 26th IEEE International Conference on Computer Communications (INFOCOM). pp. 535–543, 2007.


\bibitem{DenseWLAN2} I. Broustis, K. Papagiannaki, S. V. Krishnamurthy, M. Faloutsos, and V. P. Mhatre, “Measurement-driven guidelines for 802.11 WLAN design,” IEEE/ACM Trans. Netw., vol. 18, no. 3, pp. 722–735, 2010.

\bibitem{UltraDense} M. Kamel, W. Hamouda, and A. Youssef, “Ultra-Dense Networks: A Survey,” IEEE Commun. Surv. Tutorials, vol. 18, no. 4, pp. 2522–2545, 2016.

\bibitem{BIGAP} A. Zubow, S. Zehl, and A. Wolisz, “BIGAP — Seamless handover in high performance enterprise IEEE 802.11 networks,” in IEEE/IFIP Network Operations and Management Symposium (NOMS), 2016, pp. 445–453.

\bibitem{AmorFi} R. K Sheshadri, M. Y. Arslan, K. Sundaresan, S. Rangarajan, and D. Koutsonikolas, “AmorFi: Amorphous WiFi Networks for High-density Deployments,” in Proceedings of the 12th International on Conference on emerging Networking EXperiments and Technologies - CoNEXT ’16, 2016, pp. 161–175.


\bibitem{QoS-WLANs} J. Choi and K. G. Shin, "QoS provisioning for large-scale multi-ap WLANs," Ad Hoc Networks, vol. 10, no. 2, pp. 174-185, Mar. 2012.

\bibitem{NOX} N. Gude, T. Koponen, J. Pettit, B. Pfaff, M. Casado, N. McKeown, and S. Shenker, “NOX: towards an operating system for networks,” SIGCOMM Comput. Commun. Rev., vol. 38, no. 3, pp. 105–110, 2008.

\bibitem{OpenFlow} N. McKeown, T. Anderson, H. Balakrishnan, G. Parulkar, L. Peterson, J. Rexford, S. Shenker, and J. Turner, “OpenFlow: Enabling Innovation in Campus Networks,” ACM SIGCOMM Comput. Commun. Rev., vol. 38, no. 2, p. 69, 2008.

\bibitem{ONF} Open Networking Foundation. Available: https://www.opennetworking.org/

\bibitem{SDN1} Y. Jarraya, T. Madi, and M. Debbabi, "A Survey and a Layered Taxonomy of Software-Defined Networking," IEEE Communications Surveys Tutorials, vol. 16, no. 4, pp. 1955-1980, 2014.

\bibitem{SDN2} D. B. Rawat and S. R. Reddy, "Software Defined Networking Architecture, Security and Energy Efficiency: A Survey," IEEE Communications Surveys Tutorials, vol. PP, no. 99, pp. 1–1, 2016.

\bibitem{wSDN1} I. T. Haque and N. Abu-Ghazaleh, "Wireless Software Defined Networking: A Survey and Taxonomy," IEEE Communications Surveys Tutorials, vol. 18, no. 4, pp. 2713–2737, 2016.

\bibitem{mSDN2} T. Chen, M. Matinmikko, X. Chen, X. Zhou, and P. Ahokangas, "Software defined mobile networks: concept, survey, and research directions," IEEE Communications Magazine, vol. 53, no. 11, pp. 126-133, Nov. 2015.

\bibitem{Primitives} R. Riggio, M. K. Marina, J. Schulz-Zander, S. Kuklinski, and T. Rasheed, "Programming Abstractions for Software-Defined Wireless Networks," IEEE Transactions on Network and Service Management, vol. 12, no. 2, pp. 146-162, Jun. 2015.



%\bibitem{[2]} B. A. A. Nunes, M. Mendonca, X. N. Nguyen, K. Obraczka, and T. Turletti, “A Survey of Software-Defined Networking: Past, Present, and Future of Programmable Networks,” IEEE Communications Surveys Tutorials, vol. 16, no. 3, pp. 1617–1634, Third 2014.
%[3]F. Hu, Q. Hao, and K. Bao, “A Survey on Software-Defined Network and OpenFlow: From Concept to Implementation,” IEEE Communications Surveys Tutorials, vol. 16, no. 4, pp. 2181–2206, Fourthquarter 2014.
%[4]W. Xia, Y. Wen, C. H. Foh, D. Niyato, and H. Xie, “A Survey on Software-Defined Networking,” IEEE Communications Surveys Tutorials, vol. 17, no. 1, pp. 27–51, Firstquarter 2015.
%[5]Y. Li and M. Chen, “Software-Defined Network Function Virtualization: A Survey,” IEEE Access, vol. 3, pp. 2542–2553, 2015.




%% Association

%% TPC 
%\bibitem{TPC1} V. P. Mhatre, K. Papagiannaki, and F. Baccelli, "Interference Mitigation Through Power Control in High Density 802.11 WLANs," in IEEE INFOCOM 2007 - 26th IEEE International Conference on Computer Communications, 2007, pp. 535-543.
%
%\bibitem{TPC2} H. Ma, J. Zhu, S. Roy, and S. Y. Shin, "Joint transmit power and physical carrier sensing adaptation based on loss differentiation for high density IEEE 802.11 WLAN," Computer Networks, vol. 52, no. 9, pp. 1703-1720, Jun. 2008.
%
%\bibitem{TPC3}J. Chen, S. Olafsson, Y. Yang, and X. Gu, "Joint Distributed Transmit Power Control and Dynamic Channel Allocation for Scalable WLANs," in 2009 IEEE Wireless Communications and Networking Conference, 2009, pp. 1-6.
%
%\bibitem{TPC4} W. Li, Y. Cui, X. Cheng, M. A. Al-Rodhaan, and A. Al-Dhelaan, "Achieving Proportional Fairness via AP Power Control in Multi-Rate WLANs," IEEE Transactions on Wireless Communications (WCNC), vol. 10, no. 11, pp. 3784-3792, Nov. 2011.
%
%\bibitem{TPC5} Y. Wu, Y. Sun, Y. Ji, J. Mao, and Y. Liu, "A joint channel allocation and power control scheme for interference mitigation in high-density WLANs," in 2013 15th IEEE International Conference on Communication Technology, 2013, pp. 98-103.
%
%\bibitem{TPC6} S. Kamiya, K. Nagashima, K. Yamamoto, T. Nishio, M. Morikura, and T. Sugihara, "Joint range adjustment and channel assignment for overlap mitigation in dense WLANs," in 2015 IEEE 26th Annual International Symposium on Personal, Indoor, and Mobile Radio Communications (PIMRC), 2015, pp. 1974-1979.

\bibitem{APdensity} K. Sui et al., "Understanding the Impact of AP Density on WiFi Performance Through Real-World Deployment," in 2016 IEEE International Symposium on Local and Metropolitan Area Networks (LANMAN), 2016, pp. 1-6.

\bibitem{SDN-IoT-survey} K. Sood, S. Yu, and Y. Xiang, "Software-Defined Wireless Networking Opportunities and Challenges for Internet-of-Things: A Review," IEEE Internet of Things Journal, vol. 3, no. 4, pp. 453-463, Aug. 2016.

%%%% Survey Papers

\bibitem{S9-Channel} A. Baid and D. Raychaudhuri, "Understanding channel selection dynamics in dense Wi-Fi networks," IEEE Communications Magazine, vol. 53, no. 1, pp. 110-117, Jan. 2015.

\bibitem{802.11aa} IEEE Std 802.11aa-2012: Specific requirements Part11: Wireless LAN Medium Access Control (MAC) and Physical Layer (PHY) Specifications Amendment2: MAC Enhancements for Robust Audio Video Streaming, 2012.

\bibitem{802.11ac} IEEE Std P802.11ac-2013: Part11: Wireless LAN Medium Access Control (MAC) and Physical Layer (PHY) specifications: Enhancements for Very High Throughput for Operation in Bands below 6GHz, 2013.

\bibitem{802.11af} IEEE802.11af-2013: Local and metropolitan area networks Part11: Wireless LAN Medium Access Control (MAC) and Physical Layer (PHY) Specifications Amendment5: Television White Spaces (TVWS) Operation, 2013.

\bibitem{802.11ah} IEEE P802.11ah/D10.0, "IEEE Approved Draft Standard for Information Technology-Telecommunications and Information Exchange Between Systems-Local and Metropolitan Area Networks-Specific Requirements-Part 11: Wireless LAN Medium Access Control (MAC) and Physical Layer (PHY) Specifications: Amendment 2: Sub 1 GHz License Exempt Operation," 2016.

\bibitem{S1-energy-MAC} S.-L. Tsao and C.-H. Huang, “A survey of energy efficient MAC protocols for IEEE 802.11 WLAN,” Computer Communications, vol. 34, no. 1, pp. 54–67, Jan. 2011.

\bibitem{S2-QoS} E. Charfi, L. Chaari, and L. Kamoun, "PHY/MAC Enhancements and QoS Mechanisms for Very High Throughput WLANs: A Survey," IEEE Communications Surveys Tutorials, vol. 15, no. 4, pp. 1714-1735, Fourth 2013.

\bibitem{S3-QoS} Q. Ni, L. Romdhani, and T. Turletti, “A survey of QoS enhancements for IEEE 802.11 wireless LAN,” Wirel. Commun. Mob. Comput., vol. 4, no. 5, pp. 547–566, Aug. 2004.

\bibitem{S4-QoS} J. Choi and K. G. Shin, "QoS provisioning for large-scale multi-ap WLANs," Ad Hoc Networks, vol. 10, no. 2, pp. 174-185, Mar. 2012.

\bibitem{S5-LB} L. H. Yen, T. T. Yeh, and K. H. Chi, "Load Balancing in IEEE 802.11 Networks," IEEE Internet Computing, vol. 13, no. 1, pp. 56-64, Jan. 2009.

\bibitem{S6-Intf} M. Hoyhtya et al., "Spectrum Occupancy Measurements: A Survey and Use of Interference Maps," IEEE Communications Surveys Tutorials, vol. 18, no. 4, pp. 2386-2414, 2016.

\bibitem{S7-ax} B. Bellalta, "IEEE 802.11ax: High-efficiency WLANs," IEEE Wireless Communications, vol. 23, no. 1, pp. 38-46, Feb. 2016.


\bibitem{largeScaleMeas} S. Biswas, J. Bicket, E. Wong, R. Musaloiu-E, A. Bhartia, and D. Aguayo, “Large-scale Measurements of Wireless Network Behavior,” in Proceedings of the 2015 ACM Conference on Special Interest Group on Data Communication - SIGCOMM ’15, 2015, pp. 153–165.


\bibitem{SDNscalability} S. H. Yeganeh, A. Tootoonchian, and Y. Ganjali, “On scalability of software-defined networking,” IEEE Commun. Mag., vol. 51, no. 2, pp. 136–141, 2013.

\bibitem{SDNsurvey} I. T. Haque and N. Abu-ghazaleh, “Wireless Software Defined Networking: A Survey and Taxonomy,” IEEE Commun. Surv. Tutorials, vol. 18, no. 4, pp. 1–25, 2016.

\bibitem{SDNLang} C. Trois, M. D. D. Del Fabro, L. C. E. de Bona, and M. Martinello, “A Survey on SDN Programming Languages: Towards a Taxonomy,” IEEE Commun. Surv. Tutorials, vol. PP, no. 99, p. 1, 2016.

\bibitem{RSSIranging} A. Zanella, “Best Practice in RSS Measurements and Ranging,” IEEE Commun. Surv. Tutorials, vol. 18, no. 4, pp. 2662–2686, 2016.

\bibitem{decIntf} A. Kashyap, U. Paul, and S. R. Das, “Deconstructing Interference Relations in WiFi Networks,” in 7th Annual IEEE Communications Society Conference on Sensor, Mesh and Ad Hoc Communications and Networks (SECON), 2010, pp. 1–9.

\bibitem{smog} S. Gollakota, F. Adib, D. Katabi, and S. Seshan, “Clearing the RF Smog: Making 802.11 Robust to Cross-Technology Interference,” in Proceedings of the ACM SIGCOMM 2011 conference, 2011, p. 170.


\bibitem{Crowd} H. Ali-Ahmad, C. Cicconetti, A. de la Oliva, M. Draxler, R. Gupta, V. Mancuso, L. Roullet, and V. Sciancalepore, “CROWD: An SDN Approach for DenseNets,” in econd European Workshop on Software Defined Networks, 2013, pp. 25–31.
\bibitem{SDNcellular} L. E. Li, Z. M. Mao, and J. Rexford, “Toward Software-Defined Cellular Networks,” in European Workshop on Software Defined Networking, 2012, pp. 7–12.

\bibitem{FlashBack} A. Cidon, K. Nagaraj, S. Katti, and P. Viswanath, “Flashback: Decoupled Lightweight Wireless Control,” in Proceedings of the ACM SIGCOMM 2012 conference on Applications, technologies, architectures, and protocols for computer communication - SIGCOMM ’12, 2012, p. 223.

\bibitem{SplitAP} G. Bhanage, D. Vete, I. Seskar, and D. Raychaudhuri, “SplitAP: Leveraging Wireless Network Virtualization for Flexible Sharing of WLANs,” in IEEE Global Telecommunications Conference GLOBECOM, 2010, pp. 1–6.

\bibitem{ViFi} K. Guo, S. Sanadhya, and T. Woo, “ViFi: Virtualizing WLAN using commodity hardware,” in 9th ACM MobiCom Workshop on Mobility in the Evolving Internet Architecture, MobiArch 2014, 2014, vol. 18, no. 3, pp. 25–30.

%\bibitem{Ethane} M. Casado, M. J. Freedman, J. Pettit, J. Luo, N. McKeown, and S. Shenker, "Ethane: taking control of the enterprise," SIGCOMM Comput. Commun. Rev., vol. 37, no. 4, pp. 1-12, Aug. 2007.

\bibitem{DIRAC} P. Zerfos, G. Zhong, J. Cheng, H. Luo, S. Lu, and J. J.-R. Li, “DIRAC: A Software-based Wireless Router System,” in Proceedings of the 9th annual international conference on Mobile computing and networking - MobiCom ’03, 2003, p. 230.

\bibitem{DenseAP} R. Murty, J. Padhye, R. Chandra, A. Wolman, and B. Zill, "Designing high performance enterprise Wi-Fi networks," in Proceedings of the 5th USENIX Symposium on Networked Systems Design and Implementation, 2008.

\bibitem{Trantor} R. Murty, J. Padhye, A. Wolman, and M. Welsh, "An Architecture for Extensible Wireless LANs," in ACM Hotnets, pp. 79 - 84, 2008.

\bibitem{Dyson} R. Murty, J. Padhye, A. Wolman, and M. Welsh, "Dyson: An Architecture for Extensible Wireless LANs," in Proceedings of the 2010 USENIX conference on USENIX annual technical conference, 2010.

%\bibitem{CENTAUR} V. Shrivastava et al., "CENTAUR: Realizing the Full Potential of Centralized Wlans Through a Hybrid Data Path," in Proceedings of the 15th Annual International Conference on Mobile Computing and Networking, New York, NY, USA, 2009, pp. 297-308.

\bibitem{WLANlocalization} R. Chandra, J. Padhye, A. Wolman, and B. Zill, "A Location-Based Management System for Enterprise Wireless LANs". In NSDI, 2007.

\bibitem{LWAPP} R. Suri, N. Cam Winget, M. Williams, S. Hares, B. O'Hara, and S. Kelly, "Lightweight Access Point Protocol." [Online]. Available: https://tools.ietf.org/html/rfc5412.

\bibitem{CAPWAP} D. Stanley, P. Calhoun, and M. Montemurro, "Control And Provisioning of Wireless Access Points (CAPWAP) Protocol Specification." [Online]. Available: https://tools.ietf.org/html/rfc5415.

\bibitem{DTLS} N. Modadugu and E. Rescorla, "Datagram Transport Layer Security." [Online]. Available: https://tools.ietf.org/html/rfc4347.

\bibitem{Cisco} J. Smith, J. Woodhams, and R. Marg, Controller-Based Wireless LAN Fundamentals: An end-to-end reference guide to design, deploy, manage, and secure 802.11 wireless networks, Indianapolis, IN: Cisco Press, 2011.
\bibitem{CiscoRRM} Radio Resource Management under Unified Wireless Networks, Available: http://www.cisco.com/c/en/us/support/docs/wireless-mobility/wireless-lan-wlan/71113-rrm-new.html

\bibitem{Odin} L. Suresh, J. Schulz-Zander, R. Merz, A. Feldmann, and T. Vazao, "Towards Programmable Enterprise WLANS with Odin," in Proceedings of the First Workshop on Hot Topics in Software Defined Networks, 2012, pp. 115-120.

\bibitem{Odin2} J. Schulz-Zander, L. Suresh, N. Sarrar, A. Feldmann, T. Hühn, and R. Merz, “Programmatic Orchestration of WiFi Networks,” in USENIX Annual Technical Conference, 2014, pp. 347–358.

\bibitem{OdinThor} R. Riggio, C. Sengul, L. Suresh, J. Schulz–zander, and A. Feldmann, “Thor: Energy programmable WiFi networks,” in IEEE Conference on Computer Communications Workshops (INFOCOM WKSHPS), 2013, pp. 21–22.

\bibitem{OdinSource} “Odin Source.” [Online]. Available: http://sdn.inet.tu-berlin.de.

\bibitem{Blueprints} K.-K. Yap, R. Sherwood, M. Kobayashi, T.-Y. Huang, M. Chan, N. Handigol, N. McKeown, and G. Parulkar, “Blueprint for introducing innovation into wireless mobile networks,” in Proceedings of the second ACM SIGCOMM workshop on Virtualized infrastructure systems and architectures - VISA ’10, 2010, p. 25.

\bibitem{vBS} K. Nakauchi and Y. Shoji, “WiFi Network Virtualization to Control the Connectivity of a Target Service,” IEEE Trans. Netw. Serv. Manag., vol. 12, no. 2, pp. 308–319, Jun. 2015.

\bibitem{VAN} V. Sivaraman, T. Moors, H. Habibi Gharakheili, D. Ong, J. Matthews, and C. Russell, “Virtualizing the access network via open APIs,” in Proceedings of the ninth ACM conference on Emerging networking experiments and technologies - CoNEXT ’13, 2013, pp. 31–42.

\bibitem{CloudMAC} J. Vestin, P. Dely, A. Kassler, N. Bayer, H. Einsiedler, and C. Peylo, "CloudMAC: Towards Software Defined WLANs," ACM SIGMOBILE Mobile Computing and Communications Review, vol. 16, no. 4, pp. 42-45, Feb. 2013.
\bibitem{CloudMAC2} J. Vestin, P. Dely, A. Kassler, N. Bayer, H. Einsiedler, and C. Peylo, “CloudMAC- Towards Software Defined WLANs,” ACM SIGMOBILE Mob. Comput. Commun. Rev., vol. 16, no. 4, p. 42, Feb. 2013.

\bibitem{CloudMAC3} J. Vestin and A. Kassler, “QoS enabled WiFi MAC layer processing as an example of a NFV service,” in Proceedings of the 1st IEEE Conference on Network Softwarization (NetSoft), 2015, pp. 1–9.

\bibitem{capsulator} “Capsulator.” [Online]. Available: http://archive.openow.org/wk/index.php/Capsulator.
\bibitem{OpenWRT} “OpenWRT.” [Online]. Available: https://openwrt.org.
\bibitem{OpenvSwitch} “OpenvSwitch.” [Online]. Available: http://openvswitch.org.

\bibitem{BeHop} Y. Yiakoumis, M. Bansal, A. Covington, J. van Reijendam, S. Katti, and N. McKeown, "BeHop: A Testbed for Dense WiFi Networks," ACM SIGMOBILE Mobile Computing and Communications Review, vol. 18, no. 3, pp. 71-80, Jan. 2015.

\bibitem{OpenSDWN} J. Schulz-Zander, C. Mayer, B. Ciobotaru, S. Schmid, and A. Feldmann, "OpenSDWN: Programmatic Control over Home and Enterprise WiFi," in Proceedings of the 1st ACM SIGCOMM Symposium on Software Defined Networking Research, New York, NY, USA, 2015, p. 16:1-16:12.

\bibitem{SDWLAN} D. Zhao, M. Zhu, and M. Xu, "SDWLAN: A flexible architecture of enterprise WLAN for client-unaware fast AP handoff," in 2014 International Conference on Computing, Communication and Networking Technologies (ICCCNT), 2014, pp. 1-6.
\bibitem{SDWLAN2} D. Zhao, M. Zhu, and M. Xu, “Supporting ‘One Big AP’ illusion in enterprise WLAN: An SDN-based solution,” in Sixth International Conference on Wireless Communications and Signal Processing (WCSP), 2014, pp. 1–6.


\bibitem{EmPOWER} R. Riggio, T. Rasheed, and F. Granelli, "EmPOWER: A Testbed for Network Function Virtualization Research and Experimentation," in Future Networks and Services (SDN4FNS), 2013 IEEE SDN for, 2013, pp. 1-5.

\bibitem{EmPOWER-src} “EmPOWER.” [Online]. Available: http://empower.create-net.org/.


\bibitem{Ethanol} H. Moura, G. V. C. Bessa, M. A. M. Vieira, and D. F. Macedo, "Ethanol: Software defined networking for 802.11 Wireless Networks," in 2015 IFIP/IEEE International Symposium on Integrated Network Management (IM), 2015, pp. 388-396.

\bibitem{COAP} A. Patro and S. Banerjee, "COAP: A Software-Defined Approach for Home WLAN Management Through an Open API," SIGMOBILE Mob. Comput. Commun. Rev., vol. 18, no. 3, pp. 32-40, Jan. 2015.

\bibitem{Energino} Energino website, Available: http://www.energino-project.org.

\bibitem{AeroFlux} J. Schulz-Zander, N. Sarrar and S. Schmid, "AeroFlux: A Near-Sighted Controller Architecture for Software-Defined Wireless Networks", Open Networking Summit (ONS), 2014.  

\bibitem{AeroFlux2} J. Schulz-Zander, N. Sarrar, and S. Schmid, “Towards a scalable and near-sighted control plane architecture for WiFi SDNs,” in Proceedings of the third workshop on Hot topics in software defined networking - HotSDN ’14, 2014, pp. 217–218.



\bibitem{OpenRadio} M. Bansal, J. Mehlman, S. Katti, and P. Levis, “OpenRadio: A Programmable Wireless Dataplane,” in HotSDN, 2012, pp. 109–114.

\bibitem{Atomix} M. Bansal, A. Schulman, and S. Katti, “Atomix: A Framework for Deploying Signal Processing Applications on Wireless Infrastructure,” in 12th USENIX Symposium on Networked Systems Design and Implementation (NSDI 15), 2015, pp. 173–188.

\bibitem{Sora} K. Tan, H. Liu, J. Zhang, Y. Zhang, J. Fang, and G. M. Voelker, “Sora: High-Performance Software Radio Using General-Purpose Multi-Core Processors,” in Communications of the ACM, 2011, vol. 54, no. 1, p. 99.


\bibitem{hostapd} hostapd: IEEE 802.11 AP, IEEE 802.1X/WPA/WPA2/EAP/RADIUS Authenticator, Available: https://w1.fi/hostapd/

\bibitem{zerorpc} J. Petazzoni, “Build reliable, traceable, distributed systems with zeromq (zerorpc),” Available: http://pycon-2012-notes.readthedocs.org/en/latest/- dotcloud zerorpc.html

\bibitem{zeromq} iMatix Corporation, “Zmq - code connected,” Available: http://zeromq.org/.


\bibitem{Picasso} S. S. Hong, J. Mehlman, and S. Katti, “Picasso: Flexible RF and Spectrum Slicing,” in SIGCOMM Comput. Commun. Rev., 2012, vol. 42, no. 4, pp. 37–48.

\bibitem{OpenSketch} M. Yu, L. Jose, and R. Miao, “Software defined traffic measurement with opensketch,” in 10th USENIX Symposium on Networked Systems, 2013, pp. 29–42.

\bibitem{SpotFi} M. Kotaru, K. Joshi, D. Bharadia, and S. Katti, “SpotFi: Decimeter Level Localization Using WiFi,” in ACM SIGCOMM Computer Communication Review, 2015, vol. 45, no. 5, pp. 269–282.

\bibitem{Pantou} Pantou for OpenWRT, Available: https://www.sdxcentral.com/ projects/ pantou-openwrt/.

\bibitem{AEtherFlow} M. Yan, J. Casey, P. Shome, A. Sprintson, and A. Sutton, “$\AE$therFlow: Principled Wireless Support in SDN,” in IEEE 23rd International Conference on Network Protocols (ICNP), 2015, pp. 432–437.

\bibitem{ofsoftswitch} ofsoftswitch, Available: https://github.com/CPqD/ofsoftswitch13

%\bibitem{CROWD} H. Ali-Ahmad et al., "CROWD: An SDN Approach for DenseNets," in 2013 Second European Workshop on Software Defined Networks, 2013, pp. 25-31.
\bibitem{ResFi} S. Zehl, A. Zubow, M. Doring, and A. Wolisz, “ResFi: A secure framework for self organized Radio Resource Management in residential WiFi networks,” in IEEE 17th International Symposium on A World of Wireless, Mobile and Multimedia Networks (WoWMoM), 2016, pp. 1–11.
%\bibitem{SWAN} T. Lei, Z. Lu, X. Wen, X. Zhao, and L. Wang, "SWAN: An SDN based campus WLAN framework," in 2014 4th International Conference on Wireless Communications, Vehicular Technology, Information Theory and Aerospace Electronic Systems (VITAE), 2014, pp. 1-5.
%\bibitem{WiPCon} W. S. Kim, S. H. Chung, and J. Shi, "WiPCon: A Proxied Control Plane for Wireless Access Points in Software Defined Networks," in 2014 IEEE 17th International Conference on Computational Science and Engineering, 2014, pp. 923-929.

%\bibitem{OpenNet} M. C. Chan, C. Chen, J. X. Huang, T. Kuo, L. H. Yen, and C. C. Tseng, "OpenNet: A simulator for software-defined wireless local area network," in 2014 IEEE Wireless Communications and Networking Conference (WCNC), 2014, pp. 3332-3336.

%\bibitem{OneBigAP}D. Zhao, M. Zhu, and M. Xu, "Supporting One Big AP illusion in enterprise WLAN: An SDN-based solution," in 2014 Sixth International Conference on Wireless Communications and Signal Processing (WCSP), 2014, pp. 1-6.

\bibitem{Floodlight} Project floodlight. Available: http://www.projectfloodlight.org/floodlight.


\bibitem{OVS} Open vSwitch website, Available: http://openvswitch.org.

\bibitem{OpenCAPWAP} M. Bernaschi, F. Cacace, G. Iannello, M. Vellucci, and L. Vollero, "OpenCAPWAP: An open source CAPWAP implementation for the management and configuration of WiFi hot-spots," Computer Networks, vol. 53, no. 2, pp. 217-230, Feb. 2009.

\bibitem{LCCS} M. Achanta, "Method and Apparatus for Least Congested Channel Scan for Wireless Access Points," US Patent No. 20060072602, Apr. 2006.

\bibitem{Channel-assignment-survey-2010} S. Chieochan, E. Hossain, and J. Diamond, "Channel assignment schemes for infrastructure-based 802.11 WLANs: A survey," IEEE Communications Surveys Tutorials, vol. 12, no. 1, pp. 124-136, Firstquarter 2010.

\bibitem{DCA-G-1}M. Drieberg, F. C. Zheng, R. Ahmad, and S. Olafsson, "An Asynchronous Distributed Dynamic Channel Assignment Scheme for Dense WLANs," in 2008 IEEE International Conference on Communications, 2008, pp. 2507-2511.
\bibitem{DCA-G-2}X. Yue, C. F. Wong, and S. H. G. Chan, "CACAO: Distributed Client-Assisted Channel Assignment Optimization for Uncoordinated WLANs," IEEE Transactions on Parallel and Distributed Systems, vol. 22, no. 9, pp. 1433-1440, Sep. 2011.
\bibitem{DCA-G-3} K. Zhou, X. Jia, L. Xie, Y. Chang, and X. Tang, "Channel assignment for WLAN by considering overlapping channels in SINR interference model," in 2012 International Conference on Computing, Networking and Communications (ICNC), 2012, pp. 1005-1009.
\bibitem{DCA-G-4} H. Al-Rizzo, M. Haidar, R. Akl, and Y. Chan, "Enhanced Channel Assignment and Load Distribution in IEEE 802.11 WLANs," in 2007 IEEE International Conference on Signal Processing and Communications, 2007, pp. 768-771.

\bibitem{usagePatterns} M. Afanasyev, T. Chen, G. M. Voelker, and A. C. Snoeren, “Usage patterns in an urban WiFi network,” IEEE/ACM Trans. Netw., vol. 18, no. 5, pp. 1359–1372, 2010.

\bibitem{transOpp} D. Giustiniano, D. Malone, D. J. Leith, and K. Papagiannaki, “Measuring transmission opportunities in 802.11 links,” IEEE/ACM Trans. Netw., vol. 18, no. 5, pp. 1516–1529, 2010.

\bibitem{Wcolor-2005} A. Mishra, S. Banerjee, and W. Arbaugh, "Weighted Coloring Based Channel Assignment for WLANs," ACM SIGMOBILE Mobile Computing and Communications Review, vol. 9, no. 3, pp. 19-31, Jul. 2005.

\bibitem{J-DCA-LB-2006} A. Mishra, V. Brik, S. Banerjee, A. Srinivasan, and W. Arbaugh, "A Client-driven Approach for Channel Management in Wireless LANs," in Proc. 25th IEEE International Conference on Computer Communications (INFOCOM’06), 2006.

\bibitem{J-AP-DCA-2006} X. Ling and K. L. Yeung, "Joint access point placement and channel assignment for 802.11 wireless LANs," IEEE Transactions on Wireless Communications, vol. 5, no. 10, pp. 2705-2711, Oct. 2006.

\bibitem{Traffic-aware-CA-2007} E. Rozner, Y. Mehta, A. Akella, and L. Qiu, "Traffic-Aware Channel Assignment in Enterprise Wireless LANs," in 2007 IEEE International Conference on Network Protocols, 2007, pp. 133-143.

\bibitem{802.11ac-PCA} S. Jang and S. Bahk, "A Channel Allocation Algorithm for Reducing the Channel Sensing/Reserving Asymmetry in 802.11ac Networks," IEEE Transactions on Mobile Computing, vol. 14, no. 3, pp. 458-472, Mar. 2015.

\bibitem{DCA-2016-1} K. Lee, Y. Kim, S. Kim, J. Shin, S. Shin, and S. Chong, "Just-in-time WLANs: On-demand interference-managed WLAN infrastructures," in IEEE INFOCOM 2016 - The 35th Annual IEEE International Conference on Computer Communications, 2016, pp. 1-9.

\bibitem{DCA-2} M. Seyedebrahimi, F. Bouhafs, A. Raschellà, M. Mackay, and Q. Shi, "SDN-based channel assignment algorithm for interference management in dense Wi-Fi networks," in 2016 European Conference on Networks and Communications (EuCNC), 2016, pp. 128-132.

\bibitem{Wi-5} Wi-5Project(What to do With the Wi-Fi Wild West),t. Available: http://www.wi5.eu/

\bibitem{DCA-residential-2015} Y. Zhang, C. Jiang, Y. Wang, J. Yuan, and J. Cao, "United Channel Assignments in Residential Environments," in 2015 IEEE Global Communications Conference (GLOBECOM), 2015, pp. 1-6.

%\bibitem{DCA-802.11ac-2015} S. Jang and S. Bahk, "A Channel Allocation Algorithm for Reducing the Channel Sensing/Reserving Asymmetry in 802.11ac Networks," IEEE Transactions on Mobile Computing, vol. 14, no. 3, pp. 458-472, Mar. 2015.

%\bibitem{Ch-usage-based-2016} J. Nogueira and S. Sargento, "Channel Selection Relying on Probabilistic Adjacent Channel Interference Analysis and Pattern Recognition," Wireless Personal Communications, vol. 86, no. 3, pp. 1333-1357, Feb. 2016.

\bibitem{CA-VTC-14} B. A. H. S. Abeysekera, K. Ishihara, Y. Inoue, and M. Mizoguchi, "Network-Controlled Channel Allocation Scheme for IEEE 802.11 Wireless LANs: Experimental and Simulation Study," in 2014 IEEE 79th Vehicular Technology Conference (VTC Spring), 2014, pp. 1-5.

\bibitem{CA-BW-VTC-14} B. A. H. S. Abeysekera, M. Matsui, Y. Asai, and M. Mizoguchi, "Network controlled frequency channel and bandwidth allocation scheme for IEEE 802.11a/n/ac wireless LANs: RATOP," in 2014 IEEE 25th Annual International Symposium on Personal, Indoor, and Mobile Radio Communication (PIMRC), 2014, pp. 1041-1045.

%NEW papers
\bibitem{CAPWAP-based-CA-11} M. Bernaschi, F. Cacace, A. Davoli, D. Guerri, M. Latini, and L. Vollero, "A CAPWAP-based solution for frequency planning in large scale networks of WiFi Hot-Spots," Computer Communications, vol. 34, no. 11, pp. 1283-1293, Jul. 2011.

\bibitem{Measurement-CA-WCNC-10} Y. Liu, W. Wu, B. Wang, T. He, S. Yi, and Y. Xia, "Measurement-Based Channel Management in WLANs," in 2010 IEEE Wireless Communication and Networking Conference, 2010, pp. 1-6.

\bibitem{SDP} L. Vandenberghe, S. Boyd, Semidefinite programming, SIAM Review, vol. 38, no. 1, pp. 49-95, Mar. 1996.

\bibitem{CA-F-WCNC-11} H. Zhang, H. Ji, and W. Ge, "Channel assignment with fairness for multi-AP WLAN based on distributed coordination function," in 2011 IEEE Wireless Communications and Networking Conference, 2011, pp. 392-397.

\bibitem{Thr-estimation-2005} K.Duffy, D.Malone, and D.J.Leith, "Modeling the 802.1 distributed coordination function in non-saturated conditions", IEEE Communications Letters, Vol.9, No.8, pp.715-717, Aug.2005.

\bibitem {cvapPF} L. E. Li, M. Pal, and Y. R. Yang, "Proportional fairness in multi-rate wireless LANs," in Proc. IEEE INFOCOM,Apr. 2008, pp. 1004-1012.


\bibitem{802.11} IEEE Std 802.11-2007 (Revision of IEEE Std 802.11-1999), Jun. 2007.

\bibitem{SSF} I. Papanikos and M. Logothetis, "A study on dynamic load balance for IEEE 802.11b wireless LAN," In Proceedings of COMCON, Rethymna, Greece, 2001.

\bibitem{LLF} K. Sinha, S. C. Ghosh, B. P. Sinha, “Wireless Networks and Mobile Computing,” CRC Press, 2015.

\bibitem{MPD} J. C. Chen et. al., "Effective AP selection and load balancing in IEEE 802.11 wireless LANs," in Proc. IEEE Globecom, 2006.

\bibitem{Proxim} Proxim Wireless Networks. ORINOCO AP-600 data sheet, 2004.
\bibitem{CiscoADM} Cisco Systems Inc. Data sheet of Cisco Aironet 1200 series, 2004. 

\bibitem{F-LB-AsscCtrl-2004}Y. Bejerano, S.-J. Han, and L. (Erran) Li, "Fairness and Load Balancing in Wireless LANs Using Association Control," in Proceedings of the 10th Annual International Conference on Mobile Computing and Networking, 2004, pp. 315-329.


\bibitem{F-LB-AsscCtrl-2007} Y. Bejerano, S. J. Han, and L. Li, "Fairness and Load Balancing in Wireless LANs Using Association Control," IEEE/ACM Transactions on Networking, vol. 15, no. 3, pp. 560-573, Jun. 2007.


\bibitem{mob-Essex-2016} Y. Han and K. Yang, "An adaptive mobility manager for Software-Defined Enterprise WLANs," in 2016 Eighth International Conference on Ubiquitous and Future Networks (ICUFN), 2016, pp. 888-893.

\bibitem{802.11k} "IEEE Standard, Part 11: Wireless LAN Medium Access Control (MAC)and Physical Layer (PHY) Specifications Amendment 1: Radio
Resource Measurement of Wireless LANs," IEEE Std 802.11k-2008 (Amendment to IEEE Std 802.11-2007), pp. 1-244, June 2008.

\bibitem{Association-DenseWLANs-2014} A. Ozyagci, K. W. Sung, and J. Zander, "Association and Deployment Considerations in Dense Wireless LANs," in 2014 IEEE 79th Vehicular Technology Conference (VTC Spring), 2014, pp. 1-5.

\bibitem{DAM-G-3} D. Sajjadi, M. Tanha, and J. Pan, "Meta-Heuristic Solution for Dynamic Association Control in Virtualized Multi-Rate WLANs," in 2016 IEEE 41st Conference on Local Computer Networks (LCN), 2016, pp. 253-261.

\bibitem{DAM-G-4} G. S. Kasbekar, P. Nuggehalli, and J. Kuri, "Online Client-AP Association in WLANs," in 2006 4th International Symposium on Modeling and Optimization in Mobile, Ad Hoc and Wireless Networks, 2006, pp. 1-8.

\bibitem{DAM-G-5} D. Gong and Y. Yang, "AP association in 802.11n WLANs with heterogeneous clients," in 2012 Proceedings IEEE INFOCOM, 2012, pp. 1440-1448.

\bibitem{Association-CCA-2015} P. B. Oni and S. D. Blostein, "AP Association Optimization and CCA Threshold Adjustment in Dense WLANs," in 2015 IEEE Globecom Workshops (GC Wkshps), 2015, pp. 1-6.

\bibitem{Kuhn} G. A. Mills-Tettey, A. Stentz, and M. B. Dias, "The dynamic Hungarian algorithm for the assignment problem with changing costs," Robotics Institute, School of Computer Science, Carnegie Mellon University, PA, Tech. Rep. CMU-RI-TR-07-27, July 2007.

\bibitem{DAM-G-2} H. Gong and J. Kim, "Dynamic load balancing through association control of mobile users in WiFi networks," IEEE Transactions on Consumer Electronics, vol. 54, no. 2, pp. 342-348, May 2008.


\bibitem{802.11n-AP-Association-2014} D. Gong and Y. Yang, "On-Line AP Association Algorithms for 802.11n WLANs with Heterogeneous Clients," IEEE Transactions on Computers, vol. 63, no. 11, pp. 2772-2786, Nov. 2014.

\bibitem{BEST-AP} P. Dely et al., "BEST-AP: Non-intrusive estimation of available bandwidth and its application for dynamic access point selection," Computer Communications, vol. 39, pp. 78-91, Feb. 2014.

\bibitem{ChannelLoad} D. Gupta, D. Wu, P. Mohapatra, and C. N. Chuah, "Experimental Comparison of Bandwidth Estimation Tools for Wireless Mesh Networks," in IEEE INFOCOM 2009, 2009, pp. 2891-2895.


\bibitem{WBest} M. Li, M. Claypool, and R. Kinicki, "WBest: A bandwidth estimation tool for IEEE 802.11 wireless networks," in 2008 33rd IEEE Conference on Local Computer Networks (LCN), 2008, pp. 374-381.


\bibitem{Proportional-Fairness-AP-2014} W. Li et al., "AP Association for Proportional Fairness in Multirate WLANs," IEEE/ACM Transactions on Networking, vol. 22, no. 1, pp. 191-202, Feb. 2014.

\bibitem{Time-based-basic-1} M. Heusse, F. Rousseau, G. Berger-Sabbatel, and A. Duda, “Performance anomaly of 802.11b,” in IEEE INFOCOM 2003 - Twenty-second Annual Joint Conference of the IEEE Computer and Communications Societies, San Francisco, CA, USA, 2003, pp. 836-843.

\bibitem{Time-based-basic-2} G. Tan and J. Guttag, "Time-based Fairness Improves Performance in Multi-rate WLANs," in Proceedings of the Annual Conference on USENIX Annual Technical Conference, Berkeley, CA, USA, 2004, pp. 23-36.




\bibitem{DAM-G-1} H. Ko, J. Shin, D. Kwak, and C. Kim, "A joint approach to bandwidth allocation and AP-client association for WLANs," in IEEE Local Computer Network Conference, 2010, pp. 576-581.

\bibitem{Demand-aware-LB-Association-15} L. Yang, Y. Cui, H. Tang, and S. Xiao, "Demand-Aware Load Balancing in Wireless LANs Using Association Control," in 2015 IEEE Global Communications Conference (GLOBECOM), 2015, pp. 1-6.

\bibitem{Demand-aware-LB-Association-16-1} H. Tang, L. Yang, J. Dong, Z. Ou, Y. Cui, and J. Wu, "Throughput Optimization via Association Control in Wireless LANs," Mobile Netw Appl, vol. 21, no. 3, pp. 453-466, Jun. 2016.

\bibitem{NLB-13} F. Xu, X. Zhu, C. C. Tan, Q. Li, G. Yan, and J. Wu, "SmartAssoc: Decentralized Access Point Selection Algorithm to Improve Throughput," IEEE Transactions on Parallel and Distributed Systems, vol. 24, no. 12, pp. 2482-2491, Dec. 2013.

\bibitem{WiFiSeer} K. Sui et al., "Characterizing and Improving WiFi Latency in Large-Scale Operational Networks," in Proceedings of the 14th Annual International Conference on Mobile Systems, Applications, and Services, New York, NY, USA, 2016, pp. 347-360.

\bibitem{SNMP} S. Waldbusser, M. Rose, J. Case, and K. McCloghrie, "Protocol Operations for version 2 of the Simple Network Management Protocol (SNMPv2)." Available: https://tools.ietf.org/html/rfc1448.

\bibitem{random-forest} T. K. Ho, "Random decision forests," in Proceedings of 3rd International Conference on Document Analysis and Recognition, Montreal, QC, 1995, pp. 278-282.

\bibitem{flow-level-DAM} J. Chen, B. Liu, H. Zhou, Q. Yu, G. Lin, and X. Shen, "QoS-Driven Efficient Client Association in High-Density Software Defined WLAN," IEEE Transactions on Vehicular Technology, under press, 2017.

\bibitem{supermodular} G. Calinescu, C. Chekuri, M. P´al, and J. Vondr´ak, "Maximizing a submodular set function subject to a matroid constraint," in Integer programming and combinatorial optimization. Springer, 2007, pp. 182-196.

\bibitem{boundedAlg} X. Zhe, F. Zhang, W. Wang, and S. He, "A polynomial time algorithm for minimizing a nondecreasing supermodular set function and its performance guarantee," International Journal of Pure and Applied Mathematics, vol. 52, no. 5, pp. 637-643, 2009


\bibitem{Migration-DAM} W. Wong, A. Thakur, and S. H. G. Chan, "An approximation algorithm for AP association under user migration cost constraint," in IEEE INFOCOM 2016 - The 35th Annual IEEE International Conference on Computer Communications, 2016, pp. 1-9.

\bibitem{GameBased} L. H. Yen, J.-J. Li, and C.-M. Lin, "Stability and fairness of native AP selection games in IEEE 802.11 access networks," in 2010 Seventh International Conference on Wireless and Optical Communications Networks - (WOCN), 2010, pp. 1-5.

\bibitem{Jain's-fairness}  R. Jain,  D.M. Chiu and W. Hawe, "A Quantitative Measure of Fairness and Discrimination for Resource Allocation in Shared Computer Systems" DEC-Research Report TR-301, 1984.

\bibitem{Uplink-ref1}J. Oueis and E. C. Strinati, "Uplink Traffic in Future Mobile Networks: Pulling the Alarm," in Cognitive Radio Oriented Wireless Networks, 2016, pp. 583-593.

\bibitem{IOT-future1} I.-G. Lee and M. Kim, "Interference-aware self-optimizing Wi-Fi for high efficiency internet of things in dense networks," Computer Communications, vol. 89-90, pp. 60-74, Sep. 2016.

\bibitem{IOT-future2} K. Sood, S. Yu, and Y. Xiang, "Software-Defined Wireless Networking Opportunities and Challenges for Internet-of-Things: A Review," IEEE Internet of Things Journal, vol. 3, no. 4, pp. 453-463, Aug. 2016.

\bibitem{MARS} B. Dezfouli, M. Radi, and O. Chipara, “Mobility-aware real-time scheduling for low-power wireless networks,” in The 35th Annual IEEE International Conference on Computer Communications (INFOCOM), 2016, pp. 1–9.

\bibitem{RTwifi} Y.-H. Wei, Q. Leng, S. Han, A. K. Mok, W. Zhang, and M. Tomizuka, “RT-WiFi: Real-Time High-Speed Communication Protocol for Wireless Cyber-Physical Control Applications,” in IEEE 34th Real-Time Systems Symposium (RTSS), 2013, pp. 140–149.

\bibitem{CENTAUR} V. Shrivastava, N. Ahmed, S. Rayanchu, S. Banerjee, S. Keshav, K. Papagiannaki, and A. Mishra, “CENTAUR: Realizing the Full Potential of Centralized WLANs through a Hybrid Data Path,” in Proceedings of the 15th annual international conference on Mobile computing and networking - MobiCom ’09, 2009, p. 297.

%%%%Mobile data offloading: how much can WiFi deliver?
%%%%Per-node throughput enhancement in Wi-Fi DenseNets


%%%%%%% testbed-based performance evaluation papers
%%%%%\bibitem{single-BSS-testbed-2016} T. Zahid, F. Y. Dar, X. Hei, and W. Cheng, "A measurement study of a single-BSS software defined WiFi testbed," in 2016 First IEEE International Conference on Computer Communication and the Internet (ICCCI), 2016, pp. 144-147.
%%%%%\bibitem{multi-BSS-testbed-2016} T. Zahid, X. Hei, and W. Cheng, "Understanding performance bottlenecks of a multi-BSS software defined WiFi network testbed," in 2016 First IEEE International Conference on Computer Communication and the Internet (ICCCI), 2016, pp. 153-156.
%%%%%
%%%%%
%%%%%
%%%%%
%%%%%%2
%%%%%\bibitem{MWO-1} S. Srikanteswara, G. Li, and C. Maciocco, "Cross Layer Interference Mitigation Using Spectrum Sensing," in IEEE GLOBECOM 2007 - IEEE Global Telecommunications Conference, 2007, pp. 3553-3557.
%%%%%%3
%%%%%\bibitem{MWO-2} G. Li, S. Srikanteswara, and C. Maciocco, "Spectrum-sensing based interference mitigation for WLAN devices," in 3rd International Conference on Communication Systems Software and Middleware and Workshops, 2008. COMSWARE 2008, 2008, pp. 402-408.
%%%%%%4
%%%%%\bibitem{MBSA-TDL} A. Z. Al-Banna, J. L. LoCicero, and D. R. Ucci, "Multi-Element Adaptive Arrays with Tapped Delay Lines for Interference Mitigation in IEEE 802.11G OFDM Systems," in MILCOM 2007 - IEEE Military Communications Conference, 2007, pp. 1-8.
%%%%%%5
%%%%%\bibitem{power-ctrl-1} V. P. Mhatre, K. Papagiannaki, and F. Baccelli, "Interference Mitigation Through Power Control in High Density 802.11 WLANs," in IEEE INFOCOM 2007 - 26th IEEE International Conference on Computer Communications, 2007, pp. 535-543.
%%%%%%6
%%%%%\bibitem{CRRM-1} S. J. Bae, B. G. Choi, H. S. Chae, and M. Y. Chung, "Self-configuration scheme to alleviate interference among APs in IEEE 802.11 WLAN," in 2012 IEEE 23rd International Symposium on Personal, Indoor and Mobile Radio Communications - (PIMRC), 2012, pp. 1025-1030.
%%%%%%7
%%%%%\bibitem{DenseNets-survey} K. Shin, I. Park, J. Hong, D. Har, and D. h Cho, "Per-node throughput enhancement in Wi-Fi densenets," IEEE Communications Magazine, vol. 53, no. 1, pp. 118-125, Jan. 2015.
%%%%%%8
%%%%%\bibitem{VOID} K. Cai, M. Blackstock, M. J. Feeley, and C. Krasic, "Non-intrusive, Dynamic Interference Detection for 802.11 Networks," in Proceedings of the 9th ACM SIGCOMM Conference on Internet Measurement Conference, New York, NY, USA, 2009, pp. 377-383.
%%%%%%book 
%%%%%\bibitem{MLR-book} D. A. Freedman, Statistical Models: Theory and Practice, 2 edition. Cambridge ; New York: Cambridge University Press, 2009.
%%%%%%9
%%%%%\bibitem{Mitigation-1} N. Ahmed, V. Shrivastava, A. Mishra, S. Banerjee, S. Keshav, and K. Papagiannaki, "Interference Mitigation in Enterprise WLANs Through Speculative Scheduling," in Proceedings of the 13th Annual ACM International Conference on Mobile Computing and Networking, New York, NY, USA, 2007, pp. 342-345.
%%%%%%10
%%%%%\bibitem{Mitigation-2} M. Abusubaih, B. Rathke, and A. Wolisz, "A framework for interference mitigation in multi-BSS 802.11 wireless LANs," in World of Wireless, Mobile and Multimedia Networks Workshops, IEEE WoWMoM 2009, pp. 1-11.
%%%%%%10
%%%%%\bibitem{optimal-design} V. P. Mhatre and K. Papagiannaki, "Optimal Design of High Density 802.11 WLANs," in Proceedings of the 2006 ACM CoNEXT Conference, New York, NY, USA, 2006, p. 8:1-8:12.
%%%%%% citations of 5
%%%%%\bibitem{symphony} K. Ramachandran, R. Kokku, H. Zhang, and M. Gruteser, "Symphony: Synchronous Two-Phase Rate and Power Control in 802.11 WLANs," IEEE/ACM Transactions on Networking, vol. 18, no. 4, pp. 1289-1302, Aug. 2010.
%%%%%
%%%%%\bibitem{symphony-29} "Multiband Atheros driver for WiFi," MadWifi [Online]. Available: http:$//$www.madwifi.org.
%%%%%
%%%%%\bibitem{symphony-41} D. Raychaudhuri, I. Seskar, M. Ott, S. Ganu, K. Ramachandran, H. Kremo, R. Siracusa, H. Liu, and M. Singh, "Overview of the ORBIT radio grid testbed for evaluation of next-generation wireless network protocols," in Proc. IEEE WCNC, Mar. 2005, vol. 3, pp. 1664-1669.
%%%%%
%%%%%\bibitem{MinPACK} W. Wang, Q. Wang, W. K. Leong, B. Leong, and Y. Li, "Uncovering a Hidden wireless menace: Interference from 802.11x MAC acknowledgment frames," in 2014 Eleventh Annual IEEE International Conference on Sensing, Communication, and Networking (SECON), 2014, pp. 117-125.
%%%%%
%%%%%
%%%%%
%%%%%\begin{itemize}
%%%%%\item \textcolor{blue}{\textbf{Papers on 
 %}}
%%%%%\end{itemize}
%%%%%%%%%%
%%%%%\bibitem{ac-27} B. Bellalta, A. Checco, A. Zocca, and J. Barcelo, "On the Interactions Between Multiple Overlapping WLANs Using Channel Bonding," IEEE Transactions on Vehicular Technology, vol. 65, no. 2, pp. 796-812, Feb. 2016.
%%%%%\bibitem{ac-48} W.-S. Jung, K.-W. Lim, and Y.-B. Ko, "Utilising partially overlapped channels for OFDM-based 802.11 WLANs," Computer Communications, vol. 63, pp. 77-86, Jun. 2015.
%%%%%\bibitem{ac-41} Q. Wang, L. J. Greenstein, L. J. Cimini, D. S. Chan, and A. Hedayat, "Multi-User and Single-User Throughputs for Downlink MIMO Channels with Outdated Channel State Information," IEEE Wireless Communications Letters, vol. 3, no. 3, pp. 321-324, Jun. 2014.
%%%%%\bibitem{ac-34} Y. Zeng, P. H. Pathak, and P. Mohapatra, "A first look at 802.11ac in action: Energy efficiency and interference characterization," in Networking Conference, 2014 IFIP, 2014, pp. 1-9.
%%%%%\bibitem{ac-38} M. Yazid, A. Ksentini, L. Bouallouche-Medjkoune, and D. Aissani, "Performance Analysis of the TXOP Sharing Mechanism in the VHT IEEE 802.11ac WLANs," IEEE Communications Letters, vol. 18, no. 9, pp. 1599-1602, Sep. 2014.
%%%%%\bibitem{ac-37} O. Sharon and Y. Alpert, "MAC level Throughput comparison: 802.11ac vs. 802.11n," Physical Communication, vol. 12, pp. 33-49, Sep. 2014.
%%%%%%\bibitem{ac-40} O. Bejarano, E. Magistretti, O. Gurewitz, and E. W. Knightly, "MUTE: Sounding inhibition for MU-MIMO WLANs," in 2014 Eleventh Annual IEEE International Conference on Sensing, Communication, and Networking (SECON), 2014, pp. 135-143.
%%%%%\bibitem{ac-46}C. Zhu, C. Ngo, A. Bhatt, and Y. Kim, "Enhancing WLAN backoff procedures for downlink MU-MIMO support," in 2013 IEEE Wireless Communications and Networking Conference (WCNC), 2013, pp. 368-373.







\end{thebibliography}
