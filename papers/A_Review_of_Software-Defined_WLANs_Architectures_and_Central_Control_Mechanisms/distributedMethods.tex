%\textcolor{blue}{
%\section{Summary of the studied papers}
%}
%\subsection{\textbf{\cite{VOID}}}
%\subsubsection{\textcolor{blue}{General useful information and challenges section of introduction}}
%The state-of-the-art approach to estimate interference relationships:
%\begin{itemize}
%\item Take the network temporarily offline (profile the network)
%\item Inject synthetic traffic for interference measurements 
%	\begin{itemize}
%	\item Sending unicast or broadcast packets and recording the information such as signal strength (RSSI) and packet delivery ratios (PDR).
%	\end{itemize}
%\item Use the measurements to seed various interference models to predict network performance at run-time\\
%\end{itemize}
%The advantage of the above approaches: produce \textbf{accurate static interference maps}\newline\\
%The drawbacks:
%\begin{itemize}
%\item it can be difficult or impossible for an administrator to gain control
%of all the wireless devices and take the network completely offline for measurement.
%\item the interference map may change over time, requiring these measurements to be taken periodically to capture these changes.
%\item network profiling is quite time consuming, requiring a few hours for a moderate-size network consisting of only 20 nodes.
%\item the drawback of the last proposed methods (using conflict tables, micro-probing or CTS-to-self packets)
%	\begin{itemize}
%		\item driver and protocol modification
%		\item cooperation from both APs and mobile clients
%	\end{itemize}
%\end{itemize}
%\subsubsection{\textcolor{blue}{Survey section}}
%\begin{itemize}
%\item (14,16,17,18,19): offline interference measurements
%\item (14,18,19): interference models (estimate interference relationships)
%\item (14): 802.11 state machine
%\item (18): Markov chain
%\item (19): RSSI/PDR
%\end{itemize}
%
%Interference handling techniques:
%\begin{itemize}
%\item (6,9): power adaptation
%\item (6,9,11): channel hopping
%\item (10): throttling at central routers
%\item (7,13): traffic scheduling
%\item (20]: ...
%\item (21): conflict graphs/maps and exchanging them among senders
%\item (5): micro-probing 
%\end{itemize}
%(22): Emulab wireless testbed\\
%\subsubsection{\textcolor{blue}{Simulation platform}}
%Emulab wireless testbed \\
%
%\textit{Brian White, Jay Lepreau, Leigh Stoller, Robert Ricci, Shashi Guruprasad, Mac Newbold, Mike Hibler, Chad Barb, and Abhijeet Joglekar. An integrated experimental environment for distributed systems and networks. In OSDI ’02}
%
%\subsection{\textbf{\cite{Mitigation-1}}}
%\subsubsection{\textcolor{blue}{General useful information and challenges section of introduction}}
%.\\
%Centralization:
%\begin{itemize}
%\item Control plane: feasible and easy
%\item \textbf{Data plane: complicated in high density WLANs (many APs)}\\
%\end{itemize}
%
%Two main approaches to mitigate the interference:
%\begin{itemize}
%\item Channel access management $\Rightarrow$ is not sufficient in dense AP deployments
%\item MAC contention strategy $\Rightarrow$ more important in dense WLANs
%	\begin{itemize}
%	\item Distributed mechanisms: ex. random access (DCF)
%		\begin{itemize}
%		\item simple and robust in practice
%		\item not require significant coordination between competing transmitters
%		\item \textbf{under-utilization}: it uses \textit{time} to resolve contention
%		\end{itemize}
%	\item Centralized mechanisms: ex. scheduled access
%		\begin{itemize}
%		\item if the traffic pattern at different nodes is \textit{regular} and \textit{predictable} 
%		\item $\Rightarrow$ it is possible to pre-compute an optimal transmission schedule
%		\item $\Rightarrow$ Under such \textit{predictability assumptions}, scheduling based approaches can outperform random access approaches.
%		\end{itemize}
%	\end{itemize}
%\end{itemize}
%
%\subsubsection{\textcolor{blue}{Main contribution of paper}}
%\begin{itemize}
%\item a general framework for designing efficient
%channel access mechanisms for enterprise WLANs $\Rightarrow$ more informed use of the wireless medium.
%\item significant centralized scheduling component
%\item limited amount of randomized contention
%\item named \textbf{speculative scheduling}
%\item no change the behavior of the underlying 802.11 MAC standard
%\end{itemize}
%
%key properties of channel contention framework:
%\begin{itemize}
%\item exact knowledge of a significant fraction ($80\%$) of traffic
%\item central entity $\Rightarrow$ co-located with the edge router of the WLAN
%\item two-stage hierarchy
%	\begin{itemize}
%	\item AP-level decisions $\Rightarrow$ a bounded amount of uncertainty at the central scheduler that is managed through speculation. 
%	\item feedback from the APs to central controller can be used to continually refine the scheduler’s global view and mitigate some of the uncertainty in future
%decisions.
%	\end{itemize}
%\item flexible design of scheduling policy: based on throughput, fairness, application-specific prioritization (Voice over WiFi)
%\item not require 802.11 clients to be aware of
%	\begin{itemize}
%	\item the new scheduling techniques
%	\item the existence of a central scheduler
%	\end{itemize}
%\item multi-AP association\\
%\end{itemize}
%
%\subsubsection{\textcolor{blue}{Details of mechanism}}
%\begin{itemize}
%\item compute the conflict graph periodically
%	\begin{itemize}
%	\item vertices: AP-client links
%	\item edge: if the corresponding links cannot be simultaneously active.
%		\begin{itemize}
%		\item AP-AP and client-client conflicts $\Rightarrow$ richer conflict graph (ref. [2])
%		\end{itemize}	
%	\end{itemize}
%\item compute a directed acyclic graph (DAG): presents the scheduling in order to transmit packets 
%\end{itemize}
%
%\subsubsection{\textcolor{blue}{Implementation}}
%.\\
%Two sub-systems:
%\begin{enumerate}
%\item conflict graph constructor (in the control plane)
%\item centralized scheduler (in the data plane)\\
%\end{enumerate}
%
%Both sub-systems run on a \textit{controller}, which is a standard desktop PC.
%
%\begin{itemize}
%\item The controller is connected to 30 custom-built APs using a 100 Mbps Ethernet switch.
%\item Each AP is built from a Soekris 4826 single-board computer node (ref. [1])
%	\begin{itemize}
%	\item running the 2.6.16.19 Linux kernel
%	\item equipped with an Intel 2915ABG mini-PCI wireless adapter
%	\end{itemize}
%\item The APs are deployed on five floors of an academic office building.
%\end{itemize}
%
%\subsubsection{\textcolor{blue}{Survey section}}
%\begin{itemize}
%\item (5,8,10): scheduling strategies 
%	\begin{itemize}
%	\item infer and learn traffic patterns along arbitrary single/multihop paths.
%	\item not use any centralized entity	
%	\end{itemize}
%\item (5,10): simple extensions to 802.11 standards
%\end{itemize}
%
%\subsection{\textbf{\cite{Mitigation-2}}}
%\subsubsection{\textcolor{blue}{Abstract}}
%\begin{itemize}
%\item a framework for interference mitigation
%in multi-BSS infrastructure 802.11 WLANs
%	\begin{itemize}
%	\item based on Access Point (AP) Coordination	
%	\end{itemize}
%\item negotiation among BSSs and switch from 802.11 CSMA/CA to a time-slotted mechanism if there is a 
%	\begin{itemize}
%	\item QoS degradation due to high interference
%	\end{itemize}	
%\item performance objective:
%	\begin{itemize}
%	\item utilize the bandwidth
%	\item improve the fairness among WLAN users
%	\end{itemize}	 
%\end{itemize}
%
%\subsubsection{\textcolor{blue}{General useful information and challenges section of introduction}}
%\begin{itemize}
%\item Channel assignment policies have a
%limited improvement potential especially under high load (multiple interfering BSSs).
%\item The 802.11e standard (enhanced to
%support QoS with multimedia) coordinates channel access within \textit{a BSS}.
%	\begin{itemize}
%	\item not address the problem of overlapping BSSs/cells that use the same channel.
%	\item no mechanism beyond CSMA/CA to
%coordinate the channel access across BSSs
%	\item BSSs operate asynchronously and independently
%	\end{itemize}
%\end{itemize}
%
%\subsubsection{\textcolor{blue}{Survey section}}
%\begin{itemize}
%\item (4): a managed WiFi system to support QoS in 802.11 WLANs with multiple BSSs
%	\begin{itemize}
%	\item \textit{Inter-AP coordination} to allow overlapping BSSs coordinate their operation during up-link transmissions of the \textit{Point Coordination Function (PCF)} modus so as to improve \textit{fairness among STAs}.
%	\item Weaknesses
%		\begin{enumerate}
%		\item purely attenuation-based  propagation model which is not the case in practice due to fading. 
%		\item the PCF modus is not supported by most IEEE 802.11-compliant products  
%		\item the BSS-based scheduling does not efficiently utilize the wireless bandwidth since it does not exploit exposed nodes within interfering BSSs which can simultaneously send their packets.		
%		\end{enumerate}
%	\end{itemize}
%\end{itemize}
%
%\subsubsection{\textcolor{blue}{Main idea}}
%\begin{itemize}
%\item The 802.11 CSMA/CA channel access scheme
%provides 
%	\begin{itemize}
%	\item \underline{best effort} service $\Rightarrow$ easy to implement, not need synchronization among contending nodes, works well at low traffic load  
%	\item At increased traffic levels
%		\begin{itemize}
%		\item interference $\uparrow$ \ $\Rightarrow$ frequent collisions, contention 	and retransmissions $\uparrow$
%	\end{itemize}
%	
%	\end{itemize}
%
%
%\item On the other hand, a collision-free channel access scheme, such as a \underline{time slotted access scheme}
%\begin{itemize}
%\item performs better than the CSMA/CA
%	\begin{itemize}
%	\item at high traffic loads
%	\end{itemize}
%\end{itemize}
%\end{itemize}
%
%$\Rightarrow$ A combination of above approaches is the main idea of this paper.
%
%\subsubsection{\textcolor{blue}{Simulation platform}}
%The NCTUns simulation package (\url{http://nsl.csie.nctu.edu.tw/nctuns.html})
%
%\subsubsection{\textcolor{blue}{Real implementation platform}}
%\begin{itemize}
%\item Two APs and five stationary STAs
%\item The APs are WLAN adapters from \underline{Atheros} configured in the
%master mode (AP mode) through \underline{the MADWIFI driver}.
%\item APs are connected via an Ethernet Switch.
%\item Over the ethernet connection, a \underline{master program} runs on one AP and  \underline{synchronizes both APs}.
%\end{itemize}
%
%\subsection{\textbf{\cite{optimal-design} valuable paper to read in detail!}}
%\subsubsection{\textcolor{blue}{Abstract}}
%\begin{itemize}
%\item high throughput performance $\Leftarrow$ deployment of a high density of Access Points
%	\begin{itemize}
%	\item $\Rightarrow$ additional interference
%	\end{itemize}
%\item the reaction of CSMA/CA protocol to interference 
%    \begin{itemize}
%    \item unnecessarily \underline{conservative} in high density environments
%	\end{itemize}     
%\item recommendations on \textbf{optimum dimensioning of high density networks}
%\item recommendations on \textbf{optimum tuning of the MAC parameters}
%\item results
%	\begin{enumerate}
%	\item 802.11a $\Rightarrow$ noise-dominated
%	\item 802.11g $\Rightarrow$ interference-dominated
%	\item MAC parameter tuning 
%		\begin{itemize}
%		\item limited benefit in properly planned 802.11a networks
%		\item substantial throughput improvements in 802.11g networks
%		\end{itemize}
%	\end{enumerate}
%\item an algorithm for the optimal tuning of MAC parameters in unstructured high density
%environments
%	\begin{itemize}
%	\item  260$\%$ improvement in network throughput	
%	\end{itemize}
%\end{itemize}
%\subsubsection{\textcolor{blue}{General useful information and challenges section of introduction}}
%\begin{itemize}
%\item small CCA threshold $\Rightarrow$ lower interference, lower throughput (reduced spatial reuse) 
%	\begin{itemize}
%	\item (2,3,4,5) papers focused on this issue
%	\end{itemize}
%\item the problem considered in this paper: optimal design and tuning of high density WiFi networks 
%	\begin{itemize}
%	\item at the minimum number of APs
%	\item best throughput performance to users
%	\end{itemize}
%	
%\item A single 802.11a/g AP
%	\begin{itemize}
%	\item coverage up to 100m at a data rate of 6 Mbps
%	\item in the presence of multiple users and time-varying wireless conditions\newline
%	$\Rightarrow$
%	actual throughput on the cell boundary may be \underline{much lower than 6 Mbps}\newline
%	$\Rightarrow$
%	deploy multiple APs over the region\newline
%	$\Rightarrow$
%	high density WLANs (cell radius of 20m to 50m) and interference plays a key role
%	\end{itemize}
%\item interference mitigation in cellular networks 
%	\begin{itemize}
%	\item per-user power control
%	\end{itemize}
%\item interference mitigation in WLANs
%	\begin{itemize}
%	\item based on the MAC protocol
%	\end{itemize}
%\item considering the following factors (In contrast
%to previous works (2,3,4,5): only MAC tuning)
%	\begin{itemize}
%	\item AP density
%	\item available number of orthogonal channels
%	\item different modulation and coding schemes
%	\item noise power
%	\item $SINR=\frac{S_R}{S_N+S_I}$
%	\item AP density $\uparrow$ $\Rightarrow$ $S_R\uparrow$
%	\item AP density $\uparrow$ $\Rightarrow$ $S_I\uparrow$
%	\item SINR $\uparrow$ $\Rightarrow$ data rate (throughput) $\uparrow$
%	\end{itemize}
%\item Identifying the operating-regime $\Rightarrow$ prime importance	
%	\begin{itemize}
%	\item noise-dominated network
%		\begin{itemize}
%		\item carrier sensing: almost \underline{no role} 
%		\item \textcolor{blue}{802.11a $\Rightarrow$ \textbf{12} orthogonal channels} 
%		\end{itemize}
%	\item interference-dominated network
%		\begin{itemize}
%		\item carrier sensing: \underline{pivotal role}
%		\item \textcolor{blue}{802.11g $\Rightarrow$ \textbf{3} orthogonal channels}
%		\end{itemize}		
%	\end{itemize}
%$\Rightarrow$ in determining the network throughput.	
%\item results:
%	\begin{itemize}
%	\item using the highest data rates $\Rightarrow$ may not be optimum from the overall network capacity perspective (in dense WLANs)		
%	\item based on the analytical results $\Rightarrow$ a \textcolor{blue}{CCA adaptation algorithm} $\Rightarrow$ mitigate the interference
%		\begin{itemize}
%		\item 2D and 3D environments
%		\item irrespective of the physical layer employed
%		\item irrespective of the efficiency of the frequency selection mechanism
%		\item no assumptions about the operating regime of the network
%		\item throughput $\Leftrightarrow$ interference
%		\item throughput improvement
%			\begin{itemize}
%			\item 260$\%$: compared to default MAC
%			\item 30$\%$: compared to ECHOS (ref. 18)
%			\end{itemize}
%		\end{itemize}
%	\end{itemize}
%\end{itemize}
%
%\subsubsection{\textcolor{blue}{Main ideas}}
%. \\
%Network design choices in order to avoid the adverse effect of interference in high density WLANs:
%\begin{enumerate}[i)]
%\item identification of PHY technology (g or a)
%\item choice of AP density
%\item channel selection (minimum contention)
%\item tuning of MAC parameters
%\end{enumerate}

% use section* for acknowledgement
%%%%%%% \section*{Acknowledgment}


%The authors would like to thank...

% trigger a \newpage just before the given reference
% number - used to balance the columns on the last page
% adjust value as needed - may need to be readjusted if
% the document is modified later
%\IEEEtriggeratref{8}
% The "triggered" command can be changed if desired:
%\IEEEtriggercmd{\enlargethispage{-5in}}

% references section

% can use a bibliography generated by BibTeX as a .bbl file
% BibTeX documentation can be easily obtained at:
% http://www.ctan.org/tex-archive/biblio/bibtex/contrib/doc/
% The IEEEtran BibTeX style support page is at:
% http://www.michaelshell.org/tex/ieeetran/bibtex/
%\bibliographystyle{IEEEtran}
% argument is your BibTeX string definitions and bibliography database(s)
%\bibliography{IEEEabrv,../bib/paper}
%
% <OR> manually copy in the resultant .bbl file
% set second argument of \begin to the number of references
% (used to reserve space for the reference number labels box)
