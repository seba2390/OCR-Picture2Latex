
% REVISION


\section{Centralized Channel Assignment (ChA)}
\label{CMmech}
In this section, we review centralized ChA mechanisms.
To be consistent with Section \ref{AMmech}, we focus on the metrics employed as well as the problem formulation and solution proposed by these mechanisms.
In addition, we highlight the performance improvements achieved, compared to distributed mechanisms.

Considering the broadcast nature of wireless communication as well as the dense deployment of APs, ChA is an important network control mechanism to reduce co-channel interference and improve capacity.
Basically, a WLAN can be modeled as a conflict graph where the vertices represent APs, and edges reflect the interference level between APs. 
The color of each vertex is the channel assigned to that AP. 
The objective of a ChA mechanism is to color the graph using a minimum number of colors in order to improve channel reuse.

A typical channel assignment algorithm relies on all or a subset of the following inputs: (i) the set of APs and clients, (ii) AP-client associations, (iii) interference relationship among APs and clients, and (iv) total number of available channels. 
Using the aforementioned information, optimization metrics, such as interference, throughput, spectral efficiency, and fairness, are used to define an optimization problem. 
In this section we study centralized ChA mechanisms.



Figure \ref{fig_LCCS} represents the flowchart of the \textbf{least congested channel search (LCCS}) \cite{LCCS} algorithm. 
%
\begin{figure}[!t]
	\centering
	\includegraphics[width=0.85\linewidth]{LCCS.pdf}
	\caption{Least Congested Channel Search (LCCS) algorithm \cite{LCCS}.}
	\label{fig_LCCS}
\end{figure}
%
Although LCCS is not a central ChA mechanism, we briefly discuss its operation as it has been used by commercial off-the-shelf APs and adopted as the baseline for performance comparison of various ChA mechanisms.
In this algorithm, APs scan all channels passively after startup. 
Each AP collects the beacon frames from its neighboring APs and extracts the number of connected clients and their traffic information. 
Then, the AP selects a non-overlapping channel if it is available. 
Otherwise, the AP selects the channel with lowest traffic sum.
To obtain client and traffic information, LCCS relies on the use of optional fields in beacon packets.
Therefore, the use of this mechanism requires vendor support.
%For example, Cisco APs support these features.

A simpler approach, which has been adopted as the comparison baseline by some ChA mechanisms, is to simply sample the RSSI of nearby APs on all channels and choose the channel with the minimum RSSI level.
The literature refers to this approach as \textbf{minRSSI}.

%\todo[inline,color=cyan]{$\checkmark$ how each ap monitors all packets of neighboring aps?}
%\todo[inline,color=cyan]{$\checkmark$ when an ap switches to another channel, then that cannot be involved in tx/rx}  
%\todo[inline,color=cyan]{$\checkmark$ what do you mean it does not consider traffic demand? -clarify} 


We overview ChA mechanisms, categorized into two groups: (i) \textit{traffic-agnostic}, and (ii) \textit{traffic-aware}, as follows.

%Traffic-agnostic techniques do not consider the traffic load of APs and the traffic demands of clients. 
%In contrast, traffic-aware techniques make channel assignment considering the traffic information of APs/clients. 


\subsection{{Traffic-Agnostic Channel Assignment}} 
\label{ChA_traffic_ag}
These mechanisms do not include the traffic load of APs and the traffic demand of clients in their channel assignment process.
In fact, these mechanisms mainly rely on the interference relationship to formulate their graph coloring problems.


\textbf{Weighted coloring channel assignment (WCCA).} 
\label{WCCA}
This seminal work \cite{Wcolor-2005} formulates channel assignment through the weighted graph coloring problem and improves channel re-use by assigning partially-overlapping channels to APs. 
%The proposed model is aware of client distribution by trying to assign less-overlapping channels to the APs with more associated clients. 
Vertices are APs, and edges represent the potential interference between corresponding APs (see Figure \ref{fig_coloring}). 
The colors of the vertices represent the assigned channel number to APs. 
The number of clients associated with two neighboring APs and the interference level between them are used to calculate the weight of conflicting edges. 
\textit{Site-Report} is the proposed method to construct the overlap graph and capture network dynamics. %and provide required information to accurately estimate $W(AP_{i},AP_{j})$ between all AP pairs. 
This method enables APs to request their clients perform channel scanning. 
A list of active APs is generated for each channel.
This list includes the APs in direct communication range and the APs whose associated clients are in direct communication with the client performing Site-Report scan. 
Based on the results of Site-Report, $W(AP_{i},AP_{j})$ is computed as follows,
\begin{equation}
W(AP_{i},AP_{j})= \frac{N_{AP_{i},AP_{j}} + N_{AP_{j},AP_{i}}}{N_{AP_{i}}+N_{AP_{j}}},
\end{equation}
where $N_{AP_{j}}$ is the number of site reports performed by the clients of $AP_{j}$, and $N_{AP_{i}, AP_{j}}$ is the number of site reports of $AP_i$'s clients that reported interference with $AP_j$ or its associated clients. 

The I-factor, $I(ch_i,ch_j)$, is the normalized interference factor between two channels $ch_i$ and $ch_j$.
%------
The value of $I(ch_i,ch_j)$ is computed as follows: 
(i) for two non-overlapping channels (e.g., channel 1 and 6 in the 2.4GHz band) the value is 0; 
(ii) if $ch_i=ch_j$, the value is 1; 
(iii) for two partially-overlapping channels $ch_i$ and $ch_j$, the value is between 0 and 1, depending on the distance between their central frequency.
%-------

The weight of the conflicting edge between $AP_{i}$ and $AP_{j}$ is defined by $Ivalue(AP_{i},AP_{j}) = W(AP_{i},AP_{j})\times I(ch_i,ch_j)$.  
Having $K$ colors (channels), the problem is determining how to color the graph with minimum number of colors in order to optimize the objective function. 
%
\begin{figure}[!t]
	\centering
	\includegraphics[width=0.6\linewidth]{Ivalue.pdf}
	\caption{Weighted coloring channel assignment (WCCA) \cite{Wcolor-2005} formulates a weighted graph coloring problem. This figure represents the I-value of the conflicting edge between $AP_{i}$ and $AP_{j}$.}
	\label{fig_coloring}
\end{figure}
%
%
Three different objective functions, $O_{max}$, $O_{sum}$ and $O_{num}$, were used. 
$O_{max}$ represents the maximum I-value amongst all the links.
$O_{sum}$ is the sum of I-values of all conflicting edges. 
The total number of conflicting edges is represented by $O_{num}$. 
In order to minimize the objective functions, two distributed heuristic algorithms are proposed: (i) \textit{Hminmax} aims to minimize $O_{max}$ without AP coordination, and (ii) \textit{Hsum} aims to jointly minimize $O_{max}$ and $O_{sum}$, which requires coordination between APs in order to minimize the interference level of all conflicting edges. %The authors rename \textbf{Hsum} and \textbf{Hminmax} algorithms by \textbf{ADJ-sum} and \textbf{ADJ-minmax} if partially overlapping channels 
Although the proposed algorithms were run distributively, it would be easier to provide them with their required information using a SDWLAN architecture.

The proposed algorithms have been evaluated using NS2 \cite{NS2} simulator and testbed. %on a network consisting 20 APs. 
Two different scenarios are used in simulations: (i) three non-overlapping channels and (ii) partially overlapping channels (11 channels in the 2.4GHz band). 
For the first scenario, the algorithms showed a 45.5$\%$ and 56$\%$ improvement in interference reduction for sparse and dense networks, respectively. 
For the second scenario, there is a 40$\%$ reduction in interference, compared to LCCS. 
%\todo[inline, color=cyan]{what is the network topology?-sparse or dense \\$\checkmark$ \textit{20 APs}} 

%%%%%%%

\label{CDCA}
\textbf{Client-driven channel assignment (CDCA).}
\cite{J-DCA-LB-2006} formulates a \textit{conflict set coloring} problem and proposes a heuristic algorithm that increases the number of conflict-free clients at each iteration. 
Here, \textit{conflict} refers to the condition in which two nodes (APs or clients) that belong to different BSSs use the same channel.
To measure the interference level experienced by each client, measurement points are selected, and the signal level received from each AP is measured at those points. 
Based on this information, there are two sets assigned to each client: (i) \textit{range set}: the APs that can communicate with the client directly, and (ii) \textit{interference set}: the APs that cannot communicate with the client, but can cause interference. 
Figure \ref{fig-conflict-set-coloring} shows the range set and interference set for a given client.
%
% REVISION - CAPTION REDUCED
\begin{figure}[!t]
	\centering
	\includegraphics[width=0.65\linewidth]{fig-conflict-set-coloring.pdf}
	\caption{Client-driven channel assignment (CDCA) \cite{J-DCA-LB-2006} assigns two sets per client: \textit{range set} (denoted by $\mathcal{RS}(c_{i})$) and \textit{interference set} (denoted by $\mathcal{IS}(c_{i})$). APs are specified by numbered squares. %Note that the interference set of the client contains all APs that can cause interference on the client directly or via their associated clients.
	}
	\label{fig-conflict-set-coloring}
\end{figure}

This work defines two objective functions: (i) one that maximizes the number of conflict-free clients, and (ii) one that minimizes the total conflict in the network.
To achieve the first objective, if a channel $ch$ is assigned to an AP, then no AP in the range set and interference set of the clients associated with that AP should have channel $ch$ assigned to them.
A client satisfying this condition is referred to as a "conflict-free client".
%Note that due to association, there is only one AP that belongs to the range set of the client and uses channel $ch$. 
Since it may not be possible to satisfy this requirement for all the clients, the algorithm minimizes the overall network conflict.
%\todo[inline,color=cyan]{$\checkmark$ make sure the word "try" is correct here} 

%The authors propose the \textit{conflict-free assignment with randomized compaction} (CFAssign-RaC) algorithm. 
The main step of the channel assignment algorithm randomly selects an AP and assigns a channel that results in the maximum number of conflict-free clients. 
This step is repeated for all APs, and the order of APs is determined by a random permutation. 
%After channel assignment to APs, the number of conflict-free clients is re-calculated. 
This process is repeated as long as it increases the number of conflict-free clients. 
%Since CFAssign-RaC operates based on a randomized order of APs, it is run for several times (with different random permutations) in order to find the best solution among the outputs of different runs.

%After applying CFAssign-RaC, there might be clients left with high level of interference.
After the first step, there might be clients that are left with a high level of interference.
To remedy this condition, the conflict level of all non-conflict-free clients is balanced.
This mechanism implicitly results in traffic load balancing.
%To handle mobility, CFAssign-RaC is either run periodically or it is triggered when the objective function is higher than a threshold.
%The superiority of the proposed algorithms has been demonstrated through simulation and testbed, in terms of the number of MAC collisions, throughput and per-packet delay, compared to LCCS \cite{LCCS}.

%TABLE
%  (1) approach: CA and Load balancing
%  (2) interference model: RSSI-based
%%% (3) problem formulation: Conflict set coloring 
% (4) objective function: number of conflict-free client, Total conflict of all WLAN clients
% (5) channels: non-overlapping (802.11b)
% (6) Client/AP-centric: Client-centric
% (7) Client-aware: Yes.
%%% (8) Perfomance evaluation: simulation, real implementation
%%% (9) Dynamic/Static: Dynamic
%%% (10) Optimality: Sub-optimal
% (11) Measurement: 40 usage-points, compute range and interference sets, based on test clients
%    --> Time and computational complexity is not discussed in paper
%%% (12) Requried collecting information: scan all channels => range set and interference set


\label{FPLN}
\textbf{Frequency planning in large-scale networks (FPLN).}
\cite{CAPWAP-based-CA-11} models the channel assignment problem as a weighted graph coloring problem taking into account the external interference generated by non-controllable APs. 
The interference level caused by $AP_{j}$ on $AP_{i}$, which is the weight of edge $AP_{j} \rightarrow AP_{i}$, is defined as follows,
%
\begin{equation}
W(AP_{i},AP_{j})= \mathcal{C}^{active}_{AP_{i}}  \times I(ch_i,ch_j) \times P(AP_{i},AP_{j}),
\end{equation}
where $\mathcal{C}^{active}_{AP_{i}} $ is the number of active associated clients with $AP_{i}$. 
An associated client is called \textit{active} in a time interval if it is sending/receiving data packets.
%
The interference factor between two channels assigned to $AP_{i}$ and $AP_{j}$ is given by $I(ch_i,ch_j)$. 
The power received at $AP_{j}$ from $AP_{i}$ is represented by $P(AP_{i},AP_{j})$. 
%This model neglects transmission power asymmetry (i.e., it assumes $P(i,j)=P(j,i)$). 
The interference factor of $AP_{i}$ is defined as $\sum_{\forall AP_{j} \in \mathcal{AP}}W(AP_{i},AP_{j})$.
The objective function is defined as the sum of all APs' interference factor, including controllable and non-controllable APs. 
The ChA mechanism aims to minimize the objective function when the channels of non-controllable APs are fixed.

The authors propose a two-phase heuristic algorithm to solve the above optimization problem. 
In the first phase, the controllable APs are categorized into separate clusters based on neighborhood relationships, and the local optimization problem is solved for each cluster by finding an appropriate channel per AP to minimize its interference level while the channels of other APs (in the cluster) are fixed. 
This process is started from the AP with largest interference factor. 
In the second phase, a pruning-based exhaustive search is run on each cluster, starting from the AP with highest interference level. 
The algorithm updates the channels of APs to reduce the total interference level of the cluster. 
The pruning strategy deletes the failed channel assignments (i.e., the solutions that increase the interference level obtained in the first phase) from the search space.
FPLN uses CAPWAP (see Section \ref{CUWNarch}) as its south-bound protocol. 
Simulation and testbed experiments show about a 1.2x lower interference level, compared to LCCS.


\textbf{CloudMAC.}
\label{CloudMAC_CM}
The CloudMAC \cite{CloudMAC,CloudMAC3} architecture allows multiple NICs of a physical AP to be mapped to a VAP, where NICs may operate on different channels. 
These NICs  periodically monitor and report channel utilization to the controller. 
Unfortunately, it is not clear how channel measurement is performed.
% controller is OpenFlow \cite{OpenFlow} controller
If a client is operating on a high-interference channel, the controller sends a 802.11h channel switch announcement message to the client (mandatory for 802.11a/n standards), instructing it to switch to another channel with lower interference. 
Since a client is associated with a VAP, the client does not need to re-associate and it can continue its communication through the same VAP using another NIC of the same physical AP.
Additionally, this process does not require any client modification.
Unfortunately, the performance evaluation of this mechanism has not been reported.



%%%%%%%%%%
\label{Odin-CM}
\textbf{Odin.}
\cite{Odin2} proposes a simple channel assignment strategy. 
For each AP, the channel assignment application (running on Odin controller) samples the RSSI value of all channels during various operational hours.
The heuristic algorithm chooses the channel with the smallest maximum and average RSSI for each AP.
Unfortunately, the performance of this mechanism has not been evaluated.


\textbf{Primary channel allocation in 802.11ac networks (PCA).}
%1
\label{PCA}
The \textit{hidden channel} (HC) problem is addressed in \cite{802.11ac-PCA} for 802.11ac networks. 
The HC problem occurs when an AP interferes with the bandwidth of another AP.
This problem occurs due to the heterogeneity of channel bandwidths, different CCA thresholds for primary and secondary channels, and bandwidth-ignorant fixed transmission power. 

Figure \ref{invading} illustrates an example of the HC problem. 
The bandwidth of $AP_{1}$ is 80MHz, including four 20-MHz channels 1, 2, 3 and 4, where channel 1 is its primary channel. 
The primary channel sensing range is greater than the secondary channel sensing range due to the difference in CCA thresholds for primary and secondary channels in 802.11ac. 
$AP_{2}$ uses a 20MHz bandwidth operating on channel 3. 
Suppose $AP_{2}$ is communicating with its clients on channel 3. 
When $AP_{1}$ performs carrier sensing on its channels, it cannot detect the presence of $AP_{2}$ on channel 3 because $AP_{2}$ is not in the secondary sensing range of $AP_{1}$. 
Therefore, $AP_{1}$ starts to send data on channels 1, 2, 3 and 4, and invades $AP_{2}$. 
%In this scenario, $AP_{1}$ invades the bandwidth of $AP_{2}$.
%
\begin{figure}[!t]
	\centering
	\includegraphics[width=0.7\linewidth]{ac-1.pdf}
	\caption{An example of the hidden-channel (HC) problem in 802.11ac \cite{802.11ac-PCA}.}
	\label{invading}
\end{figure}
%
%
%2

In order to study the impact of the HC problem, the authors proposed a Markov chain-based analysis to show the impact of packet length and MAC contention parameters on the performance of APs. 
Furthermore, the effect of various HC scenarios on packet error rate (PER) has been investigated. 
%4
The authors used the graph coloring problem, where APs are vertices and channels are colors, to model the \textit{primary channel allocation} (PCA) problem. 
The edges represent the invasion relationship between APs. 
It is assumed that a controller collects information about the bandwidth of APs and their interference relationship.
The objective is to color the vertices in order to: (i) minimize the interference/invasion relationship between APs, and (ii) maximize channel utilization. 
The problem is formulated as an integer programming optimization problem, which is NP-hard.
%5
A heuristic primary channel allocation algorithm is introduced to solve the problem.
%6
Simulation results show the higher performance of PCA compared to minRSSI and random channel assignment.
Furthermore, PCA is a close-to-optimal solution when compared to the optimal exhaustive search algorithm.

\textbf{EmPOWER2.}
\label{EMPOWER2ChA}
As mentioned in Section \ref{EMPOWERarch}, EmPOWER2 \cite{Primitives} establishes \textit{channel quality and interference map} abstraction at the controller.
The proposed ChA mechanism uses the APIs provided to traverse the map and detect uplink/downlink conflicts between LVAP pairs.
There is an edge between two nodes in the interference map if they are in the communication range of each other.
In addition, the weight of each link corresponds to the channel quality between nodes.
The graph coloring algorithm proposed in \cite{san2012new} is used to assign channels based on the conflict relationships.



\label{Wi5CM}
\textbf{Wi-5 channel assignment (Wi5CA).} 
\cite{DCA-2} proposes a ChA mechanism as part of the Wi-5 project \cite{Wi-5}.
A binary integer linear programming problem is formulated with objective function 
$\textbf{U} = \textbf{G}\times \textbf{A}^{T} . \textbf{I}$, 
where $'\times'$ and $'.'$ are matrix multiplication and element-wise multiplication operators, respectively. $\textbf{G}\in\{0,1\}^{|\mathcal{AP}| \times |\mathcal{AP}|}$ represents the network topology. 
If the average power strength of $AP_i$ on $AP_j$ exceeds a specific threshold, the value of $G_{AP_{i}, AP_{j}}$ will be 1, otherwise it will be 0. $\textbf{A}\in\{0,1\}^{|\mathcal{CH}|\times |\mathcal{AP}|}$ is the current channel assignment for APs.
If channel $ch_i$ is assigned to $AP_j$, the value of $A_{{ch_{i},AP_{j}}}$ is 1, otherwise it is 0. 
$\textbf{I}\in \mathbb{R} ^{|\mathcal{AP}| \times |\mathcal{CH}|}$ represents the predicted interference matrix, where $I_{AP_{i}, ch_{i}}$ is the interference level predicted for $AP_i$ if channel $ch_j$ is assigned to it.
Unfortunately, the interference prediction method is unclear.  
The channel assignment algorithm minimizes 
\begin{equation}
\sum_{\forall AP_{i} \in \mathcal{AP}}\;\sum_{\forall ch_{j} \in \mathcal{CH}} U_{AP_{i}, ch_{j}}, 
\end{equation}
which is the network-wide interference level.
%The authors did not report details about the calculation/measurement of predicted interference matrix $\textbf{I}$. 
%\todo[inline,color=cyan]{$\checkmark$ what do you mean by the "predicted" level of interference?} 
A controller executes the ChA mechanism when the interference level is higher than a threshold.
%However, the controller updates its global knowledge about all APs in the network using OpenFlow. 
%The proposed algorithm is evaluated through MATLAB in terms of three performance metrics: interference level, SINR and spectral efficiency. 
MATLAB simulations show 2dB and 3dB reduction in the average interference level, compared to LCCS and uncoordinated channel assignment, respectively.


%%%%


%  (1) approach: CA
%  (2) interference model: RSSI-based
%%% (3) problem formulation: weighted graph vertex coloring 
% (4) objective function: Total interference of WLAN: Sum of interference level between AP-AP pairs
% (5) channels: (not mentioned: the modeling is general and both partially overlappig and non-overlap channels can be applied.)
% (6) Client/AP-centric: AP-centric
% (7) Client-aware: Yes, the count of active clients associated to APs
%%% (8) Perfomance evaluation: simulation, real implementation
%%% (9) Dynamic/Static: not mentioned (I think, it is static,  But, it can be run in real-time.) --> because it is assumed that the relationship between APs is known.
%%% (10) Optimality: Sub-optimal
% (11) Measurement: Statistical --> non-realtime (column 9)
%%% (12) Requried collecting information: RSSI of APs 



\subsection{Traffic-Aware Channel Assignment} 
\label{ChA_traffic_aware}
Traffic-aware channel assignment mechanisms include the traffic of APs, clients or both in their decision process.
As measuring the effect of interference on performance is complicated, these mechanisms explicitly include metrics such as throughput, delay, and fairness, in their problem formulations.


\label{APCA}
\textbf{AP placement and channel assignment (APCA).} 
\cite{J-AP-DCA-2006} addresses AP placement and channel assignment in order to improve the total network throughput and the fairness established among clients. 
The throughput of clients is estimated as a function of MAC-layer timing parameters, network topology, and data rate of clients. 
The data rate of each client (11, 5.5, 2 or 1Mbps) depends on the link's RSSI. 
The fairness among clients is defined as the throughput deviations of clients based on the Jain's fairness index \cite{Jain's-fairness} as follows,
%
\begin{equation}
\label{eq-fairness}
J  = \frac{(\sum_{\forall c_{i} \in \mathcal{C}}Th_{c_{i}})^2}{|\mathcal{C}|\sum_{\forall c_{i} \in \mathcal{C}}{(Th_{c_{i}}})^2},
\end{equation}
where $Th_{c_{i}}$ is the throughput of client $c_i$.
Maximizing the objective function, defined as $J\times \sum_{\forall c_{i} \in \mathcal{C}} Th_{c_{i}}$, leads to optimizing both total throughput and fairness among clients. 
%Since exhaustive search is not feasible to find the optimal solution of AP placement and channel assignment, 
The authors propose a heuristic local search method, named the \textit{patching} algorithm. 
In each iteration of this algorithm, an AP is placed on one of the predefined locations, and a non-overlapping channel is assigned to the AP to maximize the objective function.
This process is repeated until a predefined number of APs are placed. 
Simulation and testbed results show that the proposed patching algorithm provides close-to-optimal solutions in terms of total throughput and fairness. 
APCA is used only for the initial network configuration phase.

%\todo[inline,color=cyan]{$\checkmark$ you mostly discuss about the goals- there is not enough discussion about the approach} 
%\todo[inline,color=cyan]{$\checkmark$ it is not clear how ap placement works} 
%  (1) approach: CA, Fairness, AP placement
%  (2) interference model: RSSI-based ==> Client throughput
%%% (3) problem formulation: Throughput per client
% (4) objective function: Throughput and fairness
% (5) channels: non-overlapping (802.11b)
% (6) Client/AP-centric: Client-centric
% (7) Client-aware: Yes. Client throughput and fairness among clients' throughput
%%% (8) Perfomance evaluation: simulation
%%% (9) Dynamic/Static: Static
%%% (10) Optimality: Sub-optimal
% (11) Measurement: passive 
%    --> Time and computational complexity is not discussed in paper
%%% (12) Requried collecting information: RSSI of clients
%\subsubsection{\textbf{Client-agnostic and traffic demand-aware techniques}} In this section, we survey the techniques which consider only the traffic demand of APs to make the channel assignment decisions.


\label{TACA}
\textbf{Traffic-aware channel assignment (TACA).}
% solution 
\cite{Traffic-aware-CA-2007} assigns weights to APs and clients proportional to their traffic demands. 
The \textit{weighted channel separation} (WCS) metric is
\begin{equation}
\label{interference-metric}
WCS = \sum_{\substack{i,j  \in \mathcal{AP} \cup \mathcal{C}, \\ BSS(i)\neq BSS(j) }} W(i,j)\times Separation(i,j),
\end{equation}
where $BSS(k)$ is the BSS including AP or client $k$, $Separation(i,j)$ is the distance between operating channels of node $i$ and node $j$, and $W(i,j)$ is a function of the traffic demands of $i$ and $j$.
$Separation(i,j)=\min{(|ch_i-ch_j|,5)}$, therefore, the maximum separation value is 5, which is the distance between orthogonal channels (e.g., channel 1 and 6).
%The traffic-aware weight between two nodes $i$ and $j$ is defined by $W(i,j)=\beta^{snd}_{i} \beta^{snd}_{j} + \beta^{snd}_{i}\beta^{rcv}_{j} + \beta^{snd}_{j}\beta^{rcv}_{i}$ where $\beta^{snd}_{i}$ and $\beta^{rcv}_{i}$ are sending and receiving demands of node $i$, respectively. 
%In particular, the following parameters are available per $AP_{i}$ every $\Delta t$ second: the number of packets/bytes sent ($Out_{AP_{i}}(t)$), the number of packets/bytes received ($In_{AP_{i}}(t)$), and the number of currently associated clients ($\mathcal{C}_{AP_{i}}$). 
%The sending demand of $AP_{i}$ at a given time $t$ is calculated as $\beta^{snd}_{AP_{i}}(t) = \frac{Out(t)-Out(t-\Delta t)}{\Delta t}$. 
%The sending demand for each client is approximated as $S_{c}(t) = \frac{In(t)-In(t-\Delta t)}{N\Delta t}$. 
Since all nodes are included in the calculation of the WCS metric, all AP-AP, AP-client and client-client interferences are taken into account. 

%
%\todo[inline, color=cyan]{$\checkmark$ how are the sending and receiving demands modeled?} 
%
%
%

A heuristic algorithm has been proposed to maximize WCS. 
Figure \ref{fig_traffic-aware} shows the flowchart of this algorithm.
%
\begin{figure}[!t]
	\centering
	\includegraphics[width=0.8\linewidth]{traffic-aware-block-diagram.pdf}
	\caption{Traffic-aware channel assignment (TACA) \cite{Traffic-aware-CA-2007}.}
	\label{fig_traffic-aware}
\end{figure}
%
The interference graph measurement is performed a few times per day. % when the traffic load is light. 
The traffic demands of APs and clients are collected and predicted using the SNMP statistics collected from APs.
In "Step 1", an initial channel assignment is performed to maximize the WCS metric considering APs' traffic demands only. 
In "Step 2", a simulated annealing (SA) approach is used to update channel assignment, which maximizes WCS by taking into account the traffic demands of AP and clients. 
This step randomly selects an AP and its clients, and assigns a new channel at each iteration in order to maximize the WCS metric. 
%The algorithm is repeated 1000 times to obtain a close-to-optimal channel assignment. 
%\todo[inline,color=cyan]{$\checkmark$ what is the interference metric, and what do you mean by maximizing that?} 
%Performance evaluation is performed using NS2 simulator and a 25-node testbed to measure total network throughput. %for both TCP and UDP flows. 
%Three types of scenarios are considered: (i) client-aware/traffic-agnostic, (ii) client-agnostic/traffic-aware, and (iii) client-aware/traffic-aware. 
%The first scenario ignores traffic demands. The second scenario takes into account the traffic demands of APs, and the third scenario includes the traffic demands of APs and clients. 
Empirical results (using a 25-node testbed) show 2.6 times increase in TCP throughput, compared to a downgraded algorithm in which traffic demands are not included.
%The higher throughput of this traffic-aware channel assignment is presented, especially when the traffic demands are highly variable. 
The authors also showed that one channel switching per 5 minutes does not result in throughput degradation. 

\label{VDTCA}
\textbf{Virtual delayed time channel assignment (VDTCA).}
\cite{Measurement-CA-WCNC-10} introduces a new interference prediction model based on the signal strength and traffic demands of APs and clients. 
The extra transmission delay caused by interference between two APs is represented as $VDT = T_{int} - T_{n}$, where $VDT$ is \textit{virtual delayed time}. 
The term "virtual" refers to the calculation of this metric through interference prediction without actually changing channels.  
$T_{int}$ and $T_{n}$ are the time required to transmit a given amount of traffic in the presence and absence of interference, respectively.

%\todo[inline,color=cyan]{$\checkmark$ what do you mean by virtual? \\ \textit{- there is no directly justification in the paper on why they name it virtual !} \\ \textit{By the way, I added a sentence to justify it.} }  
The transmission time is a function of physical layer data rate, which depends on the signal SINR computed through measuring RSSI at receiver side.
%\todo[inline,color=cyan]{$\checkmark$ I modified the above sentence, is it correct?-- these sen} 
On the other hand, the traffic demands of APs and clients have a great impact on the transmission time of neighboring APs. 
For instance, when there is no traffic on an AP, there will be no transmission delay overhead on the neighboring APs and clients. 
Therefore, $T_{int}$ is calculated based on RSSI values and traffic rates of all neighboring APs and their associated clients.
%\todo[inline,color=cyan]{$\checkmark$ traffic rate of whom?}  
Using the proposed VDT model, it is possible to measure the VDT of each AP pair.
Channel assignment is modeled as a graph vertex coloring problem, which uses $VDT(BSS_i, BSS_j)$ as the edge weight between two interfering BSSs, where $BSS_i$ and $BSS_j$ are two APs and their associated clients.
Non-overlapping channels are the colors of vertices. 
The objective is to color the graph with a minimum number of colors while minimizing the sum of edge weights to achieve interference minimization. 
The semi-definite programming (SDP) \cite{SDP} relaxation technique is used to solve this problem.
The authors show that SDP  can solve the channel assignment problem in the order of seconds for a medium-size network. 
For instance, it takes 27.5 seconds to find the solution for a network with 50 APs and 9 non-overlapping channels. 
Testbed results show $30\%$ throughput improvement, compared to LCCS \cite{LCCS}. 

\textbf{Cisco unified wireless network (CUWN).}
\label{CUWN_CM}
In the CUWN architecture \cite{Cisco} (see Section \ref{CUWNarch}), based on the different RF groups established and the cost metric calculated for each AP, each controller updates the assigned channels periodically\footnote{The period can be adjusted by network administrators.}. 
The controller calculates a cost metric for each AP to represent the interference level of the AP, and prepares a list called \textit{channel plan change initiator} (CPCI), which includes the sorted cost metric values of all APs. 
The leader selects the AP with highest cost metric and assigns a channel to the selected AP and its one-hop neighboring APs in order to decrease their cost metrics. 
These APs are removed from the CPCI list, and this process is repeated for all remaining APs in the list.
A sub-optimal channel assignment is calculated through a heuristic algorithm\footnote{The details of ChA mechanism used by CUWN are not available.} that aims to maximize the frequency distance of selected channels. 
Note that a longer frequency distance results in a lower interference between APs. 
CUWN also enables network administrator to set channels manually.

\label{CAFA}
\textbf{Channel assignment with fairness approach (CAFA).}
\cite{CA-F-WCNC-11} formulates ChA as a weighted graph coloring problem where weights are assigned to the set of edges and vertices. 
The distance of overlapping channels is determined by the normalized interference factor, i.e., I-factor, introduced in \cite{Wcolor-2005} (see WCCA explained in Section \ref{WCCA}). 
The normalized throughput of each client is estimated using the approximation introduced in \cite{Thr-estimation-2005}. 
The weight of $AP_i$ is defined as the total \textit{normalized throughput reduction} of $AP_i$, which is calculated based on clients' normalized throughput and I-factor. 
The weight of the edge between two interfering APs represents the throughput reduction caused by their mutual interference.
Furthermore, the authors formulate the fairness among normalized throughputs of APs using Jain's fairness index \cite{Jain's-fairness}. 
The objective function is defined as the joint minimization of clients' throughput reduction and maximization of client's fairness index. 

%\todo[inline,color=cyan]{$\checkmark$ anything special about the heuristic?}  
An iterative heuristic algorithm is introduced to solve the optimization problem. 
The algorithm sorts the APs based on their weights (i.e., normalized throughput reduction).  
At each iteration, the AP with highest weight is selected for channel assignment. 
The channel of the selected AP is assigned so that the throughput reduction of the neighboring APs is minimized.
This process is repeated until channel assignment is performed for all APs. 
Simulation results show $15\%$ and $6\%$ improvement in total throughput for 4-AP and 8-AP scenarios, respectively, compared to WCCA \cite{Wcolor-2005}. 

%  (1) approach: CA and fairness
%  (2) interference model: Throughput, overlapping channel interference 
%%% (3) problem formulation: weighted graph vertex coloring 
% (4) objective function: normalized throughput reduction and fairness
% (5) channels: overlapping (802.11b)
% (6) Client/AP-centric: AP-centric
% (7) Client-aware: Yes, client throughput 
%%% (8) Perfomance evaluation: simulation
%%% (9) Dynamic/Static: Dynamic (no information about time complexity and information gathering complexity)
%%% (10) Optimality: Sub-optimal
% (11) Measurement: no information
%%% (12) Requried collecting information: clients throughput, [not mentioned in detail]


\label{TFACM}
\textbf{Throughput and fairness-aware channel assignment (TFACA).}
\cite{CA-VTC-14} uses spectrum monitoring information and the traffic demand of APs.
This mechanism defines the utility function of $AP_i$ when operating on channel $ch_{j}$ as,
%
\begin{equation}
\label{eqTFACM}
U(AP_{i}, ch_{j})=\frac{\min\left\{F(AP_{i},ch_{j}), T(AP_{i}) \right\}}{T(AP_{i})} ,
\end{equation}
where $F(AP_{i},ch_{j})$ is the free airtime of channel $ch_{j}$ on $AP_{i}$, and $T(AP_{i})$ is the total time required to send data of $AP_i$ in a time unit. 
Note that Equation \ref {eqTFACM} represents the percentage of $AP_i$'s data that can be sent over channel $ch_{j}$ in a given time unit.
Two objective functions are defined:
\begin{equation}
 \max\sum_{\forall AP_{i}\in\mathcal{AP}, \forall ch_{j} \in \mathcal{CH}}{U(AP_{i}, ch_{j})},
\end{equation}
%
and,
\begin{equation}
\max(\min_{\forall AP_{i}\in\mathcal{AP}, \forall ch_{j} \in \mathcal{CH}}{U(AP_{i}, ch_{j})}).
\end{equation}
A heuristic algorithm has been proposed to solve the above NP-hard problem in the following two steps:
(i) temporary channel assignment, which assigns a channel to each AP independent of other APs, to maximize each AP's utility function, and (ii) channel reassignment of the APs with low utility function (starting with the lowest one) without decreasing the sum of all utility function values. 
The second step is repeated for a given number of iterations or until $U(AP_{i}, ch_{j})=1$ for all APs. 
A fairness index similar to Equation \ref{eq-fairness} has been defined to include the utility function of APs. 

The 802.11g protocol with non-overlapping channels are used for performance evaluation through simulation and testbed. 
An $8$ to $15\%$ improvement in the fairness index and an $8$ to $21\%$ increase in the mean utility function are achieved, compared to minRSSI. 
%Furthermore, the authors report the performance of proposed technique in the presence of non-controllable APs (which are working using the minRSSI technique). 

%It is worth noting that the channel and data traffic information required to calculate utility function in the central controller are not collected in a real-time manner.
%Rather, this information is collected passively and based on statistical values. 


%  (1) approach: CA and fairness
%  (2) interference model: available throughput, percentage of operating on the assigned channel
%%% (3) problem formulation: calculus-based 
% (4) objective function: total throughput 
% (5) channels: non-overlapping (802.11g)
% (6) Client/AP-centric: AP-centric
% (7) Client-aware: Yes, traffic demand and the number of clients associated to each AP
%%% (8) Perfomance evaluation: simulation, real implementation
%%% (9) Dynamic/Static: Static
%%% (10) Optimality: Sub-optimal
% (11) Measurement: Statistical --> non-realtime (column 9)
%%% (12) Requried collecting information: traffic demand in AP, the acheivable throughput in a time unit (on a given channel)  


\label{FBWA}
\textbf{Frequency and bandwidth assignment for 802.11n/ac (FBWA)}.
%The authors of \cite{CA-BW-VTC-14} propose a channel and bandwidth allocation technique for 802.11a/n/ac-based WLANs (the proposed approach is similar to \cite{CA-VTC-14}). 
\cite{CA-BW-VTC-14} addresses channel and bandwidth allocation to benefit from the channel bonding feature of 802.11n/ac. 
The utility function of $AP_i$ that works on primary channel $ch_{j}$ and bandwidth $\beta$ is defined as
%
\begin{equation}
U(AP_{i}, ch_{j}, \beta)=\frac{Th(AP_{i}, ch_{j}, \beta)} {\min \left\{ Th^{\mathrm{max}}(AP_{i}, ch_{j}, \beta),\; L_{AP_{i}}^{\mathrm{mean}}   \right\} } 
\end{equation}
where $Th(AP_{i}, ch_{j}, \beta)$ is the expected throughput of $AP_i$ operating on primary channel $ch_{j}$ and channel bandwidth $\beta$,
 $Th^{\mathrm{max}}(AP_{i}, ch_{j}, \beta)$ is the maximum achievable throughput of $AP_i$, and $L_{AP_{i}}^{\mathrm{mean}}$ is the mean traffic load on $AP_i$. 
Through a heuristic algorithm similar to TFACA \cite{CA-VTC-14}, the appropriate primary channels and bandwidths are determined for all APs to maximize the sum of the APs' utility functions.
%\todo[inline,color=cyan]{$\checkmark$ and a jain based ... ??} 

Testbed results show a $65$ to $89\%$ reduction in the number of overlapping BSSs, and a mean throughput that is 3 times higher, compared to random channel assignment and minRSSI. 
%Similar to \cite{CA-VTC-14}, the process of collecting information from APs is not performed in a real-time manner. 

%
\label{ATCM}
\textbf{Normalized-airtime channel assignment (NATCA).}
%In \cite{DCA-residential-2015}, a WiFi Union (WU) framework is proposed to manage AP channel assignment.
\cite{DCA-residential-2015} requires each AP to measure its busy time (referred to as \textit{normalized airtime}) and report to a controller periodically. 
The busy time of an AP is the percentage of time that a channel assigned to that AP is occupied by its neighbors. 
An AP measures this value when sending data towards its clients. 
%The downlink traffic from an AP to its clients is used as the dominant traffic (compared to uplink) to measure APs' busy times. 
A threshold value is defined to categorize APs into two groups based on their busy time: \textit{heavily congested} and \textit{lightly congested}. 
An optimization problem has been proposed to minimize the sum of the normalized airtime of all APs without modifying the status of lightly congested APs. 
A Tabu search algorithm is proposed to solve the optimization problem. 
NS3 simulations show a 1.5x throughput improvement, compared to LCCS \cite{LCCS}. 
The measurement of normalized airtime for all APs is performed passively using the NS3 tracing system, which provides packet-level trace files for all APs.  

%\subsubsection{\textbf{Client-aware and traffic demand-agnostic techniques}} In this section, we survey the algorithms which consider the interference experienced by clients, but do not consider the traffic demands of clients.
%%%%%


%\subsubsection{\textbf{Client-aware and traffic demand-aware techniques}} In this section, we overview the techniques which use the traffic demand of clients and APs in order to measure the interference level of network and channel assignment decisions.


% new algorithms after 2010 (not mentioned in 2010 survey)

%\begin{table*}
%	\centering
%	\caption{Interference Mitigation Algorithms and Protocols}
%	\label{Interference-Mitigation-Algorithms} 
%	\def\arraystretch{1.5}
%	\begin{tabular}{ |m{0.5cm}|m{3.7cm}|m{2.7cm}|m{3.5cm}|m{2cm}|m{1.3cm}|m{1.3cm}|m{1.5cm}|}
%		\Xhline{3\arrayrulewidth}
%		\textbf{Ref.} & \textbf{Approach} & \textbf{Interference Metrics} & \textbf{Objective Function}& \textbf{Channels\newline  Standard}&\textbf{Client\newline AP-Driven } & \textbf{Dynamic \newline Static} & \textbf{Performance \newline Evaluation}\\ \Xhline{3\arrayrulewidth}
%		
%		\cite{Wcolor-2005} &Channel Assignment&-Channel interference\newline -Clients number& I-value&Overlapping\newline 802.11b&Client&Dynamic&Empirical\\\hline
%		\cite{J-DCA-LB-2006}&-Channel Assignment\newline -Load Balancing&RSSI-based&-Number of conflict-free client \newline -Total conflict of all WLAN clients&Non-overlapping\newline 802.11b&Client&Dynamic&-Empirical\newline -Simulation\\\hline
%		\cite{J-AP-DCA-2006}&-Channel Assignment\newline -AP Placement\newline -Fairness Handling&Clients throughput&Throughput and fairness&Non-overlapping\newline 802.11b&Client&Static&Simulation\\\hline
%		
%		
%		\cite{Traffic-aware-CA-2007}&Channel Assignment&-APs throughput \newline -Clients throughput \newline -Number of clients per AP&-Channel separation \newline -Traffic-aware total interference among all nodes (APs and clients)&Overlapping\newline 802.11b/g&AP\newline Client&Dynamic&-Empirical\newline-Simluation\\\hline
%		\cite{802.11ac-PCA}&Primary Channel Assignment&Channel invading relations among APs&-Total invading relations between APs \newline -Channel utilization&Non-overlapping\newline 802.11ac&AP&Dynamic&Simulation\\\hline
%		\cite{DCA-2016-1}&-Channel Assignment\newline -AP Power Control\newline -AP-client Association Control&Non-overlapping channel distance (binary)&Total power consumption of APs&Non-overlapping\newline 802.11b&AP&Dynamic&-Empirical\newline -Simulation\\\hline
%		\cite{DCA-2}&Channel Assignment&Priori knowledge about interference is assumed&Total interference levels of APs&Non-overlapping&AP&Dynamic&Simulation\\\hline		
%		\cite{DCA-residential-2015}&Channel Assignment&Normalized airtime of interfering APs&Total normalized airtime of APs&Non-overlapping\newline 802.11g&AP&Dynamic&Simulation\\\hline
%%		\cite{Ch-usage-based-2016}&&&&&&&\\\hline		
%		\cite{CA-VTC-14} &-Channel Assignment\newline -Fairness Handling&-AP achievable communication period \newline -AP traffic demand&-Ratio of \newline an AP's achievable throughput \newline to its traffic load\newline -Fairness&Non-overlapping\newline 802.11g&AP&Static&-Empirical\newline -Simulation\\\hline
%		\cite{CA-BW-VTC-14} &-Primary Channel Assignment\newline -Bandwidth Allocation\newline -Fairness Handling&-AP achievable throughput \newline-Number of associated clients of an AP\newline -Traffic demand of clients&-Ratio of \newline an AP's achievable throughput \newline to its traffic load\newline -Fairness&Non-overlapping\newline 802.11n/ac&AP&Static&Simulation\\\hline		
%		\cite{CAPWAP-based-CA-11} &Channel Assignment&-Overlapping channel distance \newline -RSSI between APs \newline -Number of active associated clients of an AP&Interference factors of all APs&Overlapping\newline 802.11b&AP&Static&-Empirical \newline -Simulation\\\hline			
%		\cite{Measurement-CA-WCNC-10}&Channel Assignment&-RSSI \newline -Traffic demand&Extra transmission delay&Non-overlapping\newline 802.11a/g&AP and Client&Dynamic&-Empirical\\\hline
%		\cite{CA-F-WCNC-11}&-Channel Assignment\newline -Fairness Handling&-Overlapping channel distance\newline -Client throughput&-Total normalized throughput reduction of APs\newline -Fairness&Overlapping\newline 802.11b&Client&Static&Simulation
%		\\\Xhline{3\arrayrulewidth}
%	\end{tabular}
%\end{table*}

% on --> off (N_{C(AP_n)}<W_n \ \& \  t_{inactive} \geq  T_{idle})
% off --> on (N_{C(AP_n)}\geq W_n \ \& \  t_{off} \geq  T_{offline})



%The authors explain some practical issues to design a fast dynamic channel assignment technique as follows
%\begin{itemize}
%	\item Time complexity of solution finder: the time needed to solve a channel assignment problem should be short enough so that it can be run frequently for highly dynamic WLANs. 
%	\item Collection time of traffic information
%	\item Execution time of channel reassignment
%\end{itemize}
%The computational complexity of solution finder should be reasonable 

%  (1) approach: CA
%  (2) interference model: RSSI-based, traffic demand
%%% (3) problem formulation: weighted graph vertex coloring 
% (4) objective function: total VDT (delay overhead due to the interferes)
% (5) channels: non-overlapping (802.11a/g)
% (6) Client/AP-centric: AP and Client
% (7) Client-aware: Yes, RSSI of clients
%%% (8) Perfomance evaluation: real implementation
%%% (9) Dynamic/Static: Dynamic (very good explanation)
%%% (10) Optimality: Sub-optimal
% (11) Measurement: realtime (periodic) - monitoring interface
%%% (12) Requried collecting information: RSSI and traffic demand




%\subsubsection{\textbf{\textcolor{blue}{Time-domain channel usage-based approach}}}
%
%The research study in \cite{Ch-usage-based-2016} focuses on the time-domain behavior of channel usage to propose a probabilistic model with the aim of  estimating adjacent channel interference. In other words, the authors consider temporal dynamicity of channel usage through pattern recognition to enable a time-varying dynamic channel assignment. The main idea is the scheduling of channel selection based on the probabilistic model of channel usage in time domain. In this way, the solution selects an appropriate channel for each AP and also determines a channel change scheduling with the aim of minimizing interference. 





%TABLE CA
\begin{table*}
	\centering
	\scriptsize
	\caption{Comparison of Channel Assignment (ChA) Mechanisms}
	\label{CA-table} 
	\def\arraystretch{1}
	\begin{tabular}{|c|c|c|c|c|c|c|c|}
		\Xhline{3\arrayrulewidth}
		%	\textbf{Ref.} & \multicolumn{2}{|c|}{OneTwoThree} & \multicolumn{2}{|c|}{OneTwoThree}&\textbf{Channels} & \textbf{Dynamic/Static} & \textbf{\multicolumn{2}{|c|}{OneTwoThree}}\\ \Xhline{3\arrayrulewidth}
		%		
		\multirow{2}{*}{\quad \quad \textbf{Mechanism} \quad\quad}& \multicolumn{2}{c|}{\quad\quad  Traffic-Aware \quad\quad}& \multicolumn{2}{c|}{Standards $\&$ Channels} & \multirow{2}{*}{\textbf{\quad Dynamic \quad}}&  \multicolumn{2}{c|}{\quad Performance Evaluation \quad}\\ \cline{2-5}\cline{7-8}
		
		&\textbf{\quad Downlink \quad}& \textbf{\quad Uplink \quad}& \textbf{\quad\quad Standard \quad\quad} & \textbf{\quad Partially Overlapping \quad} & & \textbf{Simulation} & \textbf{Testbed}\\ \Xhline{3\arrayrulewidth}
		%1
		WCCA \cite{Wcolor-2005} &$\times$&$\times$&802.11b&$\checkmark$&$\checkmark$&$\checkmark$&$\checkmark$\\\hline
		%2
	    CDCA \cite{J-DCA-LB-2006}&$\times$&$\times$&802.11b&$\times$&$\checkmark$&$\checkmark$&$\checkmark$\\\hline
		%3
		%4
		%
		%		
		FPLN \cite{CAPWAP-based-CA-11}&$\times$&$\times$&802.11b&$\checkmark$&$\times$&$\checkmark$&$\checkmark$\\\hline					
		%		
		%6
		% jCI \cite{DCA-2016-1}&$\checkmark$&$\checkmark$&$\times$&802.11b&$\times$&$\checkmark$&$\checkmark$&$\checkmark$\\\hline
		%
		%cloudmac-2013
		CloudMAC \cite{CloudMAC}&$\times$&$\times$&802.11a/b/g/n&$\checkmark$&$\checkmark$&$\times$&$\checkmark$\\\hline		
		%		
		%Odin 2014
		Odin \cite{Odin2}&$\times$&$\times$&802.11a/b/g/n&$\checkmark$&$\checkmark$&$\times$&$\checkmark$\\\hline				
		%		
		% -PCA 2015
		PCA \cite{802.11ac-PCA}&$\times$&$\times$&802.11ac&$\times$&$\checkmark$&$\checkmark$&$\times$\\\hline		
		%
		EMPOWER2 \cite{Primitives}&$\times$&$\times$&802.11a/b/g/n/ac&$\times$&$\checkmark$&$\times$&$\checkmark$\\\hline		
		%
		% -DCA-2 2016
		Wi5CA \cite{DCA-2}&$\times$&$\times$&802.11b/g&$\times$&$\checkmark$&$\checkmark$&$\times$		 \\\hline
		%
		APCA \cite{J-AP-DCA-2006}&$\times$&$\checkmark$&802.11b&$\times$&$\times$&$\checkmark$&$\times$\\\hline		
		%		
		TACA \cite{Traffic-aware-CA-2007}&$\checkmark$&$\checkmark$&802.11b/g&$\checkmark$&$\checkmark$&$\checkmark$&$\checkmark$\\\hline
		%		
		VDTCA \cite{Measurement-CA-WCNC-10}&$\checkmark$&$\checkmark$&802.11a/g&$\times$&$\checkmark$&$\times$&$\checkmark$\\\hline					
		%
		CUWN \cite{Cisco}&$\checkmark$&$\checkmark$&802.11a/b/g/n/ac&$\checkmark$&$\checkmark$&$\times$&$\times$\\\hline					
		%
		CAFA \cite{CA-F-WCNC-11}&$\checkmark$&$\checkmark$&802.11b&$\checkmark$&$\times$&$\checkmark$&$\times$ \\\hline
		%		
		TFACA \cite{CA-VTC-14}&$\checkmark$&$\times$&802.11g&$\times$&$\times$&$\checkmark$&$\checkmark$\\\hline		
		%		
		FBWA \cite{CA-BW-VTC-14}&$\checkmark$&$\times$&802.11a/n/ac&$\times$&$\times$&$\checkmark$&$\times$\\\hline		
		%		
		NATCA \cite{DCA-residential-2015}&$\checkmark$&$\times$&802.11g&$\times$&$\checkmark$&$\checkmark$&$\times$ 
		%\cite{Ch-usage-based-2016}&$\times$&$\times$&$\times$&$\times$&$\times$&$\times$&\\\hline		
		%
		%
		\\\Xhline{3\arrayrulewidth}
	\end{tabular}
\end{table*}



\subsection{Channel Assignment: Learned Lessons, Comparison, and Open Problems}
\label{ChAProblems}
Table \ref{CA-table} presents and compares the features of ChA mechanisms.
In the following, we study these features and identify research directions.


\subsubsection{\textbf{Dynamicity and Traffic-Awareness}}
\label{asc_dyn_tra_awar}
As Table \ref{CA-table} shows, not all the ChA mechanisms support dynamic channel reallocation.
In addition, most of the ChA mechanisms are either traffic-agnostic, or they do not recognize uplink traffic.
%For example, although EmPOWER2 architecture (see Section \ref{empower2_arch})  provides traffic information for network applications, its employed ChA mechanism is very simple and does not benefit from these features.
Furthermore, many of the ChA mechanisms do not actually discuss the data gathering policy employed, and the rest focus on the mean traffic load of APs and the mean throughput demand of clients. 
These limitations are due to two main reasons: First, the higher the rate of central interference map generation, the higher the overhead of control data exchanged by the controller. 
Second, although heuristic algorithms have been proposed to tackle the NP-hardness of the channel assignment problems, the execution duration of these algorithms might not be short enough to respond to network dynamics.
The two approaches earlier proposed in Section \ref{asc_dyn_overhead} (i.e., prediction and hybrid design) can be employed to cope with these challenges.
For example, depending on the time window and accuracy of predictions, a ChA proactively responds to dynamics, which in turn reduces the burden on real-time network mapping and fast algorithm execution.
%Meanwhile, dynamic adjustment of evaluation period and designing monitoring mechanisms that do not consume network resources excessively are open research challenges.
The second solution, i.e., hybrid design, employs multi-level decision making to support fast and low-overhead reaction to network dynamics. 
For the hybrid designs, however, the topology of controllers may be tailored depending on the control mechanism employed.
For example, while a suitable controller topology for a ChA mechanism depends on the interference relationship between APs, a more suitable topology for an AsC mechanism is to connect all the APs of a hallway to a local controller to improve the QoS of clients walking in that area.
Therefore, it is important to design architectures that support flexible communication between controllers.



\subsubsection{\textbf{Joint ChA and AsC}}
\label{asc_joint_des}
As channel switching may lead to re-association (depending on the architecture used), clients may experience communication interruption and violation of their QoS requirements \cite{SDWLAN2,Lv2013,jin2011fast}.
Therefore, it is necessary to measure and reduce the channel switching overhead of dynamic ChA mechanisms. 
In this regard, joint design of ChA and AsC is essential for uninterrupted performance guarantee.
Such a joint mechanism, for example, monitors and predicts clients' traffic pattern, performs channel assignment to maximize spatial reuse, and moves VAPs between APs to support seamless handoff as the traffic changes.
In addition, mobility prediction \cite{manweiler2013predicting,dong2012evaluation} may be employed to minimize the delay of AsC and ChA by enabling the use of proactive control mechanisms instead of using reactive mechanisms that are triggered by performance drop.
Note that mobility prediction is specifically useful in SDWLANs as these architectures enable the central collection and analysis of metrics such as RSSI to apply localization and mobility prediction algorithms.
Unfortunately, the existing joint ChA-AsC mechanisms (e.g., \cite{xu_channel_2011,Zheng2016}) operate distributively and do not provide any performance guarantee, which is required for  mission-critical applications such as medical monitoring or industrial control \cite{REWIMO}.

%most of those are implemented in a distributed way and do not address delay constraints of applications.



\subsubsection{\textbf{Partially Overlapping Channels and AP Density Management}}
\label{cha_part_overlap}
Most ChA mechanisms designed for 2.4GHz networks utilize only non-overlapping channels to decrease the interference level (as shown in Table \ref{CA-table}).
However, due to the dense deployment of APs, this approach limits the trade-off between interference reduction and channel reuse \cite{zhao_dapa:_2016}.
To address this concern, research studies have improved the performance of channel assignment by dynamically assigning partially-overlapping channels \cite{CA-F-WCNC-11}. 
These mechanisms, for example, rely on the interference relationship between APs to increase the distance between assigned channels as the pairwise interference increases.
However, these mechanisms require the underlying SDWLAN to provide APIs for efficient collection of network statistics.


In addition to utilizing partially-overlapping channels, ChA mechanisms can further improve network capacity through AP topology management, which is achieved by three main strategies:
(i) \textit{AP placement}: the site survey is performed to determine the location of APs during the installation phase \cite{zvanovec2003wireless};
(ii) \textit{AP power control}: a SDWLAN controller dynamically adjusts the transmission power level of APs \cite{li2011achieving};
(iii) \textit{AP mode control}: a SDWLAN controller dynamically turns on/off APs \cite{EmPOWER}.
Note that the second and third strategy require architectural support.
%For example, a ChA mechanism needs APIs to monitor network and make decisions that aim to reduce interference and maximize spatial reuse through both channel assignment and transmission power control.
It is also important to coordinate the aforementioned strategies with AsC to avoid clients' service interruption.
For example, by relying on the global network view established, clients association may be changed before channel assignment or transmission power control cause client disconnection. 
Although these problems have been addressed in isolation or distributively \cite{bejerano2009cell,wang2014coverage,DenseWLAN1,huang2010distributed}, centralized and integrated solutions are missing. 



\subsubsection{\textbf{Integration with Virtualization}}
\label{ch_disc_int_virt}
Our review of ChA and AsC mechanisms shows that these mechanisms are oblivious to virtualization.
Specifically, they do not take into account how the pool of resources is assigned to various network slices.
However, virtualization has serious implications on network control.
For example, mobility management becomes more challenging when network virtualization is employed.
When a client requires a new point of association due to its mobility, in addition to parameters such as fair bandwidth allocation, the available resources of APs should also be taken into account. 
More specifically, for a client belonging to slice $n$, the AsC mechanism should ensure that after the association of this client with a new AP, the QoS provided by slice $n$ and other slices is not violated.
However, as this may require client steering, the re-association cost of other clients should be minimized.
%Specifically, as discussed in Section \ref{archComp}, SDWLANs enable the slicing of network resources such as airtime.
%As another example, an AsC mechanism may dynamically update and maintain a subset of APs that should be used for handoff of the clients that belong to a slice $n$.
As an another example, a ChA mechanism may assign non-overlapping and overlapping channels to a mission-critical slice and a regular slice, respectively.
The problem becomes even more complicated when multiple controllers collaborate to manage network resources.
The integration of network slicing and control mechanisms would pave the way to use SDWLANs in emerging applications, such as mission-critical data transfer from mobile and IoT devices.

In addition to network-based slicing of resources, it is desirable to support personalized mobility services as well.
To this end, AsC mechanisms require architectural support to enable client association based on a variety of factors.
As an example, a client's VAP stores the client's handset features and exposes that information to the AsC mechanism to associate the client with APs that result in minimum energy consumption of the client.
Such interactions between architectural components and control mechanisms are unexplored.

Another important challenge of slicing at low level is the complexity of coordination.
For example, assume that different transmission powers are assigned to flows. 
In this case, simply changing the transmission power of APs working on the same channel would cause significant interference and disrupt the services being offered.
Therefore, offering low-level virtualization requires coordination between the network control mechanisms.



\subsubsection{\textbf{Coexistence of High-Throughput and Legacy Clients}}
Although new WiFi standards (e.g., 802.11n/ac) supporting high throughput are broadly used in various environments, we should not neglect the existence of legacy devices (e.g., 802.11b), especially for applications that require low throughput \cite{Tozlu2012,CYW43907,BCM4343}.
In heterogeneous networks, the throughput of high-rate 802.11ac devices is jeopardized by legacy clients, due to the wider channel widths used by 802.11n/ac through channel bonding \cite{Zeng2014,han_fair_2016}. 
Specifically, the channel bonding feature of 802.11n/ac brings about new challenges such as \textit{hidden-channel problem}, \textit{partial channel blocking}, and \textit{middle-channel starvation} \cite{802.11ac-PCA,han_fair_2016,zhang2011adaptive}. 
The bandwidth of a 802.11n/ac device (80/160MHz) is blocked or interrupted by a legacy client (20MHz) due to the partial channel blocking and hidden-channel problem.
In the middle-channel starvation problem, a wideband client is starved because its bandwidth overlaps with the channels of two legacy clients.
Hence, the wideband client can use its entire bandwidth only if both legacy clients are idle. 
Despite the heterogeneity of WLAN environments, our review shows that addressing the challenges of multi-rate networks to satisfy the diverse set of QoS requirements is still premature. 
In fact, only PCA \cite{802.11ac-PCA} proposes a central solution, and its performance has been compared with distributed mechanisms.

%\subsubsection{Multiple spatial streams}
%The emergence of new standards (802.11n/ac/ad) necessitates the use of new mechanisms. For example, when channel bonding feature is available, a DCA mechanism may address both channel assignment and channel width allocation. 
%Although there are a few techniques available for joint frequency and channel width assignment (e.g., \cite{802.11ac-PCA,CA-BW-VTC-14}), there is no technique addressing the effect of MIMO and beamforming on channel assignment.
%For example, a potential approach would be to rely on user location and coverage pattern to achieve a higher level of spatial reuse with channel assignment. 



\subsubsection{\textbf{Security}}
Wireless communication is susceptible to jamming attacks through which malicious signals generated on a frequency band avoid the reception of user traffic on that channel. 
The vulnerability of 802.11 networks to such denial of service (DoS) attacks have been investigated through various empirical measurements \cite{pelechrinis_denial_2011,bayraktaroglu2008performance,pelechrinis2009ares}.
Although the research community proposes various mechanisms (such as frequency hopping \cite{pelechrinis2010efficacy}) to cope with this problem, the existing approaches do not benefit from a central network view.
In particular, by relying on the capabilities of SDWLANs, clients and APs can periodically report their operating channel information (such as channel access time, noise level and packet success rate) to the controller, and mechanisms are required to exploit this information to perform anomaly detection. 


%Consequently, we have four types of channel assignment techniques as follows
%\begin{enumerate}[(I)]
%	\item Client-agnostic, traffic-agnostic
%	\item Client-aware, traffic-agnostic
%	\item Client-agnostic, traffic-aware
%	\item Client-aware, traffic-aware.
%\end{enumerate}

%Furthermore, the performance of channel assignment techniques are evaluated through either \textit{simulations} or \textit{empirical experiments}. Some works consider real implementation through small-scale testbeds to reveal the superiority of proposed algorithms such as \cite{Wcolor-2005,J-DCA-LB-2006,Traffic-aware-CA-2007,DCA-2016-1,CA-VTC-14,CAPWAP-based-CA-11,Measurement-CA-WCNC-10}. However, other techniques perform the performance evaluation using simulations. 




%Flashback [11] proposes a control channel technique for
%WiFi networks, by allowing stations to send short control
%messages concurrently with data transmissions, without
%affecting throughput. This ensures a low overhead con-trol plane for WiFi networks that is decoupled from the
%data plane.