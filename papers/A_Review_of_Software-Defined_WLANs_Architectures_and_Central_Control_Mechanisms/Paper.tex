%\documentclass[journal,transmag]{IEEEtran}

\documentclass[journal,comsoc]{IEEEtran}

\hyphenation{}
\ifCLASSINFOpdf
\usepackage[pdftex]{graphicx}
\usepackage{graphicx}
\usepackage{latexsym}
\usepackage{amssymb}
\usepackage{amsmath} 
\usepackage{enumerate}
\usepackage{mathtools}
\usepackage{hyperref}
\usepackage{multirow}
\usepackage{makecell}
\usepackage{longtable} % for 'longtable' environment
\usepackage{pdflscape,array,booktabs}
\usepackage[table]{xcolor}
\usepackage{footnote}

\usepackage[colorinlistoftodos]{todonotes}

%\usepackage[bottom]{footmisc}

\def\labelitemi{--}


\hypersetup{
    colorlinks=true,
    linkcolor=black,
    filecolor=black,      
    urlcolor=black,
    citecolor=black,        % color of links to bibliography
}

  
  
% The paper headers
\markboth{IEEE Communications Surveys and Tutorials, Accepted July~2018}%
{Dezfouli \MakeLowercase{\textit{et al.}}: A Review of Software-Defined WLANs: Architectures and Central Control Mechanisms}


% \usepackage{tikz}
% \usepackage{tikz-qtree}

\usepackage[switch,pagewise]{lineno}
%\setlength\columnsep{15pt}
%\linenumbers
           
\newcommand{\R}[1]{\label{#1}\linelabel{#1}}
\newcommand{\lr}[1]{page~\pageref{#1}, line~\lineref{#1}}


%\usepackage{subcaption}

%\usepackage[usenames, dvipsnames]{color}
%\definecolor{BBBlue}{RGB}{0, 0, 81}



% new code:
\usepackage{rotating}
\renewcommand\textfraction{.1}
\newcommand{\CoJPoD}{\textit{CoJPoD}}
\usepackage{dcolumn}
\newcommand{\mc}[1]{\multicolumn{1}{c}{#1}}
\newcommand{\astiii}{^{***}}
\newcommand{\astii}{^{**}}
\newcommand{\asti}{^{*}}
\renewcommand{\tabcolsep}{3.5pt}

\newenvironment{grey-color}{\par\color{gray}}{\par}

\definecolor{brightmaroon}{rgb}{0.76, 0.13, 0.28}
\renewcommand{\linenumberfont}{\rmfamily \footnotesize \color{brightmaroon}}


%\usepackage[titletoc,title]{appendix}
%\usepackage[dotinlabels]{titletoc}
%\usepackage{verbatim}
% \usepackage{subfigure}
% \usepackage{wrapfig}
\else
\fi

\usepackage{color}
%\usepackage[framemethod=default]{mdframed}
\usepackage{showexpl}
% \mdfdefinestyle{exampledefault}{%
% linewidth=0.8pt,linecolor=black,innerleftmargin=4,innerrightmargin=4,
% frametitlerule=true,frametitlerulecolor=black,
% frametitlebackgroundcolor=cyan,
% frametitlerulewidth=1.5pt}

%%+Make Index
\usepackage{makeidx}
\usepackage{array}
\usepackage{bbding}
\usepackage{amssymb}

%\usepackage[sort,noadjust]{cite}
\usepackage{cite}
%\bibliographystyle{ieeetr}
%\usepackage{notoccite}

\graphicspath {{Figures/}}

\usepackage{adjustbox}
\usepackage{array}

\newcolumntype{R}[2]{%
	>{\adjustbox{angle=#1,lap=\width-(#2)}\bgroup}%
	l%
	<{\egroup}%
}
\newcommand*\rot{\multicolumn{1}{R{90}{1em}}}% no optional argument here, please!


%\usepackage{caption}
%\captionsetup{justification=justified}
\makeindex
\begin{document}


\twocolumn

%\title{Software-Defined WLANs: State-of-the-Art\\ and Research Challenges}
%\title{{\huge A Review of Software-Defined WLANs: Architecture, Association Control, and Channel Assignment}}
\title{{A Review of Software-Defined WLANs: Architectures and Central Control Mechanisms}}
\author{Behnam~Dezfouli, \textit{Member, IEEE}, Vahid~Esmaeelzadeh, \textit{Student Member, IEEE},\\ Jaykumar Sheth, \textit{Student Member, IEEE}, and Marjan Radi, \textit{Member, IEEE}
\thanks{ 
% Please, write here acknowledgment for financial support if desired.
%Cognitive work was supported in part by the Spanish Council of Research under Grant  BS123456 (sponsor and financial support acknowledgment goes here). Paper titles should be written in uppercase and lowercase letters, not all uppercase. Avoid writing long formulas with subscripts in the title; short formulas that identify the elements are fine (e.g., "Nd-Fe-B"). Full names of authors are preferred in the author field. 
}
\thanks{Behnam Dezfouli, Vahid Esmaeelzadeh and Jaykumar Sheth are with the Internet of Things Research Lab, Department of Computer Engineering, Santa Clara University, Santa Clara, CA, 95053 USA.
E-mail: \{bdezfouli, vesmaeelzadeh, jsheth\}@scu.edu.
}% <-this % stops a space
\thanks{Marjan Radi is with Western Digital Corporation, Milpitas, CA, 95035 USA. E-mail: {marjan.radi}@wdc.edu.
}% <-this % stops a space
}


% The paper headers
%\markboth{}%
%{B. Dezfouli \MakeLowercase{\textit{et al.}}: A Review of Software-Defined WLANs: Architectures and Central Control Mechanisms}
% The only time the second header will appear is for the odd numbered pages
% after the title page when using the twoside option.
%

% make the title area
\maketitle


% The Abstract
\begin{abstract}
The significant growth in the number of WiFi-enabled devices as well as the increase in the traffic conveyed through wireless local area networks (WLANs) necessitate the adoption of new network control mechanisms. 
Specifically, dense deployment of access points, client mobility, and emerging QoS demands bring about challenges that cannot be effectively addressed by distributed mechanisms.
Recent studies show that software-defined WLANs (SDWLANs) simplify network control, improve QoS provisioning, and lower the deployment cost of new network control mechanisms. 
In this paper, we present an overview of SDWLAN architectures and provide a qualitative comparison in terms of features such as programmability and virtualization.
In addition, we classify and investigate the two important classes of centralized network control mechanisms: (i) association control (AsC) and (ii) channel assignment (ChA).
We study the basic ideas employed by these mechanisms, and in particular, we focus on the metrics utilized and the problem formulation techniques proposed.
We present a comparison of these mechanisms and identify open research problems.
\end{abstract}
% The Keywords 
\begin{IEEEkeywords}
Software-Defined Networking, IEEE 802.11, Architecture, Mobility, Interference, Centralized Algorithms
\end{IEEEkeywords}

\IEEEpeerreviewmaketitle


\def\blue!50!blue{blue!50!black}
\def\RevisionColor{blue!50!black}
\def\RevisionColorTwo{black!50!black}


% \leavevmode
% \\
% \\
% \\
% \\
% \\
\section{Introduction}
\label{introduction}

AutoML is the process by which machine learning models are built automatically for a new dataset. Given a dataset, AutoML systems perform a search over valid data transformations and learners, along with hyper-parameter optimization for each learner~\cite{VolcanoML}. Choosing the transformations and learners over which to search is our focus.
A significant number of systems mine from prior runs of pipelines over a set of datasets to choose transformers and learners that are effective with different types of datasets (e.g. \cite{NEURIPS2018_b59a51a3}, \cite{10.14778/3415478.3415542}, \cite{autosklearn}). Thus, they build a database by actually running different pipelines with a diverse set of datasets to estimate the accuracy of potential pipelines. Hence, they can be used to effectively reduce the search space. A new dataset, based on a set of features (meta-features) is then matched to this database to find the most plausible candidates for both learner selection and hyper-parameter tuning. This process of choosing starting points in the search space is called meta-learning for the cold start problem.  

Other meta-learning approaches include mining existing data science code and their associated datasets to learn from human expertise. The AL~\cite{al} system mined existing Kaggle notebooks using dynamic analysis, i.e., actually running the scripts, and showed that such a system has promise.  However, this meta-learning approach does not scale because it is onerous to execute a large number of pipeline scripts on datasets, preprocessing datasets is never trivial, and older scripts cease to run at all as software evolves. It is not surprising that AL therefore performed dynamic analysis on just nine datasets.

Our system, {\sysname}, provides a scalable meta-learning approach to leverage human expertise, using static analysis to mine pipelines from large repositories of scripts. Static analysis has the advantage of scaling to thousands or millions of scripts \cite{graph4code} easily, but lacks the performance data gathered by dynamic analysis. The {\sysname} meta-learning approach guides the learning process by a scalable dataset similarity search, based on dataset embeddings, to find the most similar datasets and the semantics of ML pipelines applied on them.  Many existing systems, such as Auto-Sklearn \cite{autosklearn} and AL \cite{al}, compute a set of meta-features for each dataset. We developed a deep neural network model to generate embeddings at the granularity of a dataset, e.g., a table or CSV file, to capture similarity at the level of an entire dataset rather than relying on a set of meta-features.
 
Because we use static analysis to capture the semantics of the meta-learning process, we have no mechanism to choose the \textbf{best} pipeline from many seen pipelines, unlike the dynamic execution case where one can rely on runtime to choose the best performing pipeline.  Observing that pipelines are basically workflow graphs, we use graph generator neural models to succinctly capture the statically-observed pipelines for a single dataset. In {\sysname}, we formulate learner selection as a graph generation problem to predict optimized pipelines based on pipelines seen in actual notebooks.

%. This formulation enables {\sysname} for effective pruning of the AutoML search space to predict optimized pipelines based on pipelines seen in actual notebooks.}
%We note that increasingly, state-of-the-art performance in AutoML systems is being generated by more complex pipelines such as Directed Acyclic Graphs (DAGs) \cite{piper} rather than the linear pipelines used in earlier systems.  
 
{\sysname} does learner and transformation selection, and hence is a component of an AutoML systems. To evaluate this component, we integrated it into two existing AutoML systems, FLAML \cite{flaml} and Auto-Sklearn \cite{autosklearn}.  
% We evaluate each system with and without {\sysname}.  
We chose FLAML because it does not yet have any meta-learning component for the cold start problem and instead allows user selection of learners and transformers. The authors of FLAML explicitly pointed to the fact that FLAML might benefit from a meta-learning component and pointed to it as a possibility for future work. For FLAML, if mining historical pipelines provides an advantage, we should improve its performance. We also picked Auto-Sklearn as it does have a learner selection component based on meta-features, as described earlier~\cite{autosklearn2}. For Auto-Sklearn, we should at least match performance if our static mining of pipelines can match their extensive database. For context, we also compared {\sysname} with the recent VolcanoML~\cite{VolcanoML}, which provides an efficient decomposition and execution strategy for the AutoML search space. In contrast, {\sysname} prunes the search space using our meta-learning model to perform hyperparameter optimization only for the most promising candidates. 

The contributions of this paper are the following:
\begin{itemize}
    \item Section ~\ref{sec:mining} defines a scalable meta-learning approach based on representation learning of mined ML pipeline semantics and datasets for over 100 datasets and ~11K Python scripts.  
    \newline
    \item Sections~\ref{sec:kgpipGen} formulates AutoML pipeline generation as a graph generation problem. {\sysname} predicts efficiently an optimized ML pipeline for an unseen dataset based on our meta-learning model.  To the best of our knowledge, {\sysname} is the first approach to formulate  AutoML pipeline generation in such a way.
    \newline
    \item Section~\ref{sec:eval} presents a comprehensive evaluation using a large collection of 121 datasets from major AutoML benchmarks and Kaggle. Our experimental results show that {\sysname} outperforms all existing AutoML systems and achieves state-of-the-art results on the majority of these datasets. {\sysname} significantly improves the performance of both FLAML and Auto-Sklearn in classification and regression tasks. We also outperformed AL in 75 out of 77 datasets and VolcanoML in 75  out of 121 datasets, including 44 datasets used only by VolcanoML~\cite{VolcanoML}.  On average, {\sysname} achieves scores that are statistically better than the means of all other systems. 
\end{itemize}


%This approach does not need to apply cleaning or transformation methods to handle different variances among datasets. Moreover, we do not need to deal with complex analysis, such as dynamic code analysis. Thus, our approach proved to be scalable, as discussed in Sections~\ref{sec:mining}.
\begin{table}[htp]
\centering
\scalebox{0.6}{%
\begin{tabular}{llr}
	Layer                & Output          & No. of          \\ 
	Type                 & Dimensions      & Params          \\ 
\hline
	Input                & 1x5x229         &                 \\ 
	Conv (Id)            & 32x5x229@3x3    & 288             \\ 
	BatchNorm            & 32x5x229        & 128             \\ 
	Relu                 & 32x5x229        &                 \\ 
	Conv (Id)            & 32x3x227@3x3    & 9216            \\ 
	BatchNorm            & 32x3x227        & 128             \\ 
	Relu                 & 32x3x227        &                 \\ 
	MaxPool              & 32x3x113@1x2    &                 \\ 
	Dropout, p=0.25      & 32x3x113        &                 \\ 
	Conv (Id)            & 64x1x111@3x3    & 18432           \\ 
	BatchNorm            & 64x1x111        & 256             \\ 
	Relu                 & 64x1x111        &                 \\ 
	MaxPool              & 64x1x55@1x2     &                 \\ 
	Dropout, p=0.25      & 64x1x55         &                 \\ 
	Dense (Id)           & 512             & 1802240         \\ 
	BatchNorm            & 512             & 2048            \\ 
	Relu                 & 512             &                 \\ 
	Dropout, p=0.5       & 512             &                 \\ 
	Dense (Sigmoid)      & 88              & 45144           \\ 
\hline
	                     &                 & $\sum$ 1877880 
\end{tabular}%
}
\caption{The \ConvNet Architecture}
\label{table:convnet}
\end{table}
\begin{table}[htp]
\centering
\scalebox{0.6}{%
\begin{tabular}{llr}
	Layer                & Output          & No. of          \\ 
	Type                 & Dimensions      & Params          \\ 
\hline
	Input                & 1x5x229         &                 \\ 
	Conv (Id)            & 8x5x229@3x3     & 72              \\ 
	BatchNorm            & 8x5x229         & 32              \\ 
	Relu                 & 8x5x229         &                 \\ 
	Conv (Id)            & 8x3x227@3x3     & 576             \\ 
	BatchNorm            & 8x3x227         & 32              \\ 
	Relu                 & 8x3x227         &                 \\ 
	MaxPool              & 8x3x113@1x2     &                 \\ 
	Dropout, p=0.25      & 8x3x113         &                 \\ 
	Conv (Id)            & 8x1x111@3x3     & 576             \\ 
	BatchNorm            & 8x1x111         & 32              \\ 
	Relu                 & 8x1x111         &                 \\ 
	MaxPool              & 8x1x55@1x2      &                 \\ 
	Dropout, p=0.25      & 8x1x55          &                 \\ 
	Conv (Id)            & 8x1x53@1x3      & 192             \\ 
	BatchNorm            & 8x1x53          & 32              \\ 
	Relu                 & 8x1x53          &                 \\ 
	MaxPool              & 8x1x26@1x2      &                 \\ 
	Dropout, p=0.25      & 8x1x26          &                 \\ 
	Dense (Id)           & 16              & 3328            \\ 
	BatchNorm            & 16              & 64              \\ 
	Relu                 & 16              &                 \\ 
	Dropout, p=0.5       & 16              &                 \\ 
	Dense (Sigmoid)      & 23              & 391             \\ 
\hline
	                     &                 & $\sum$ 5327    
\end{tabular}%
}
\caption{The \SmallConvNet Architecture}
\label{table:small_convnet}
\end{table}
\begin{table}[htp]
\centering
\scalebox{0.6}{%
\begin{tabular}{llr}
	Layer                & Output          & No. of          \\ 
	Type                 & Dimensions      & Params          \\ 
\hline
	Input                & 1x256x256       &                 \\ 
	Conv (Id)            & 32x256x256@3x3  & 288             \\ 
	BatchNorm            & 32x256x256      & 128             \\ 
	Elu                  & 32x256x256      &                 \\ 
	Conv (Id)            & 32x256x256@3x3  & 9216            \\ 
	BatchNorm            & 32x256x256      & 128             \\ 
	Elu                  & 32x256x256      &                 \\ 
	MaxPool              & 32x128x128@2x2  &                 \\ 
	Conv (Id)            & 32x128x128@3x3  & 9216            \\ 
	BatchNorm            & 32x128x128      & 128             \\ 
	Elu                  & 32x128x128      &                 \\ 
	Conv (Id)            & 32x128x128@3x3  & 9216            \\ 
	BatchNorm            & 32x128x128      & 128             \\ 
	Elu                  & 32x128x128      &                 \\ 
	MaxPool              & 32x64x64@2x2    &                 \\ 
	Conv (Id)            & 64x64x64@3x3    & 18432           \\ 
	BatchNorm            & 64x64x64        & 256             \\ 
	Elu                  & 64x64x64        &                 \\ 
	Conv (Id)            & 64x64x64@3x3    & 36864           \\ 
	BatchNorm            & 64x64x64        & 256             \\ 
	Elu                  & 64x64x64        &                 \\ 
	MaxPool              & 64x32x32@2x2    &                 \\ 
	Conv (Id)            & 64x32x32@3x3    & 36864           \\ 
	BatchNorm            & 64x32x32        & 256             \\ 
	Elu                  & 64x32x32        &                 \\ 
	Conv (Id)            & 64x32x32@3x3    & 36864           \\ 
	BatchNorm            & 64x32x32        & 256             \\ 
	Elu                  & 64x32x32        &                 \\ 
	MaxPool              & 64x16x16@2x2    &                 \\ 
	Conv (Id)            & 128x16x16@3x3   & 73728           \\ 
	BatchNorm            & 128x16x16       & 512             \\ 
	Elu                  & 128x16x16       &                 \\ 
	Conv (Id)            & 128x16x16@3x3   & 147456          \\ 
	BatchNorm            & 128x16x16       & 512             \\ 
	Elu                  & 128x16x16       &                 \\
\end{tabular}%
\begin{tabular}{llr}
 
	Upscale              & 128x32x32       &                 \\ 
	Concat               & 192x32x32       &                 \\ 
	Conv (Id)            & 128x32x32@3x3   & 221184          \\ 
	BatchNorm            & 128x32x32       & 512             \\ 
	Elu                  & 128x32x32       &                 \\ 
	Conv (Id)            & 128x32x32@3x3   & 147456          \\ 
	BatchNorm            & 128x32x32       & 512             \\ 
	Elu                  & 128x32x32       &                 \\ 
	Upscale              & 128x64x64       &                 \\ 
	Concat               & 192x64x64       &                 \\ 
	Conv (Id)            & 64x64x64@3x3    & 110592          \\ 
	BatchNorm            & 64x64x64        & 256             \\ 
	Elu                  & 64x64x64        &                 \\ 
	Conv (Id)            & 64x64x64@3x3    & 36864           \\ 
	BatchNorm            & 64x64x64        & 256             \\ 
	Elu                  & 64x64x64        &                 \\ 
	Upscale              & 64x128x128      &                 \\ 
	Concat               & 96x128x128      &                 \\ 
	Conv (Id)            & 32x128x128@3x3  & 27648           \\ 
	BatchNorm            & 32x128x128      & 128             \\ 
	Elu                  & 32x128x128      &                 \\ 
	Conv (Id)            & 32x128x128@3x3  & 9216            \\ 
	BatchNorm            & 32x128x128      & 128             \\ 
	Elu                  & 32x128x128      &                 \\ 
	Upscale              & 32x256x256      &                 \\ 
	Concat               & 64x256x256      &                 \\ 
	Conv (Id)            & 32x256x256@3x3  & 18432           \\ 
	BatchNorm            & 32x256x256      & 128             \\ 
	Elu                  & 32x256x256      &                 \\ 
	Conv (Id)            & 32x256x128@3x3  & 9216            \\ 
	BatchNorm            & 32x256x128      & 128             \\ 
	Elu                  & 32x256x128      &                 \\ 
	Conv (Sigmoid)       & 1x256x88@1x41   & 1313            \\ 
\hline
	                     &                 & $\sum$ 964673  
\end{tabular}%
}
\caption{The \AUNet Architecture}
\label{table:aunet}
\end{table}


% REVISION

\section{Centralized Association Control (AsC)}
\label{AMmech}
In this section we review centralized AsC mechanisms. 
We focus on the metrics employed as well as the problem formulation and solving approaches proposed. 
This section also highlights the impact of architecture on design when details are available.
We employ a consistent notation to ease the understanding and comparison of metrics and formulations.
Our review summarizes the performance improvements achieved to reveal the benefits of these centralized mechanisms compared to distributed approaches.\footnote{Note that in this section and the next section (Section \ref{CMmech}) we do not study all the AsC and ChA mechanisms implemented by the architectures reviewed in Section \ref{Architectures}. 
This is because some of these architectures only show the feasibility of implementing control mechanisms, and they do not propose any \textit{new} mechanism to benefit from the features of SDWLANs.}
Although we mostly focus on state-of-the-art centralized mechanisms, we review the seminal distributed mechanisms as well because of their adoption as the baseline to evaluate the performance of centralized mechanisms.



%------------------------------------------------------------------ COLOR
At a high level, we categorize AsC mechanisms from two perspectives: 
%
\begin{itemize}
	\item \textit{\textbf{Seamless handoff}}: refers to the mechanisms that their objective is to reduce the overhead and delay of client handoff,
	\item \textit{\textbf{Client steering}}: refers to the mechanisms that adjust client-AP associations to optimize parameters such as the load of APs and the airtime allocated to clients.
\end{itemize}
%
Supporting seamless handoff is usually addressed by proposing architectures that reduce the overhead of re-association.
In contrast, client steering is performed through proposing AsC mechanisms that run on the control plane.
For example, an AsC mechanism may propose an optimization problem to balance the load of APs, while handoff delays depend on the architectural properties.
From the client steering point of view, AsC is particularly important in dense topologies because there are usually multiple candidate APs for a client in a given location. 
Therefore, using a simple RSSI metric may result in hot spots, unbalanced load of APs and unfair resource allocation to clients.
Hence, one of the main goals of AsC is to achieve \textit{fairness} among clients and APs. 


%Client mobility is handled through dynamic AsC. 
%To this end, low time-complexity AsC mechanisms must be run periodically to decide about client re-association.
%Metrics such as RSSI, client demand and AP load may be considered by AsC mechanisms to choose a destination AP.




%There are numerous AsC mechanisms proposed in the literature.
%In this paper, however, we only focus on mechanisms that rely on global network knowledge.
%Before the overview of these mechanisms, we first explain two widely-adopted distributed AsC mechanisms that have been used as baselines to evaluate the performance of centralized mechanisms.

The distributed AsC mechanism employed by 802.11 standard is \textbf{strongest signal first (SSF)} \cite{SSF,802.11}. 
Using SSF, each client decides about its association based on the RSSI of probe response and beacon messages.
Each client associates with the AP from which highest RSSI has been received.
%Most of association management techniques use SSF as a baseline for their comparison.
%maybe add about 802.11k and r
\textbf{Least load first (LLF)} \cite{LLF} is an another widely-adopted distributed mechanism where APs broadcast their current load through beacon messages to help the clients include AP load when making association decisions.
The load of an AP is represented through various metrics, such as the number of associated clients\footnote{The traffic indication map (TIM) of a beacon packet represents a bitmap that indicates the clients for which the AP has buffered packets.}.
In the following, we review AsC mechanisms.


%------------------------------------------------------------------ COLOR
\subsection{Seamless Handoff}
\label{seam-handoff}
Some of the architectures reviewed in Section \ref{Archs} propose mechanisms to reduce the overhead of client handoff.
Handoff overhead refers to: (i) the packets exchanged between client and AP to establish a connection, and (ii) the delay incurred by the client during this process \cite{pack2007fast}.
In this section we study the contributions of SDWLAN architectures in terms of supporting seamless handoff.


% 2010
% Individual - Seamless handoff
\textbf{Cisco unified wireless network (CUWN).}% demand-agnostic
\label{CUWN_AM}
\cite{Cisco} enables the central configuration of RSSI threshold with hysteresis to perform seamless handoff of CCX \cite{CiscoCCX} compatible clients.
When the RSSI received from associated AP drops below the \textit{scan threshold}, the client increases its AP scanning rate to ensure fast handoff to another AP when the difference between the RSSI of the associated and new AP is equal or greater than the \textit{hysteresis} value specified.
In addition to hysteresis, specifying the \textit{minimum RSSI} value forces the clients to re-associate when their RSSI drops below a minimum RSSI. 

Three types of client roaming scenarios are handled by CUWN: (i) intra-controller roaming, (ii) inter-controller layer-2 roaming, and (iii) inter-controller layer-3 roaming.
The controller simply handles an intra-controller roaming by updating its client database with the new AP connected to the roaming client. 
Inter-controller layer-2 roaming occurs when a client associates with an AP that is controlled by a different controller belonging to the same subnet. 
In this type of roaming, mobility messages are exchanged between the old controller and the new one. 
%({\color{red}e.g.,} {\color{blue} There is no details in \cite{Cisco} about these messages!}) 
Then, the database entry related to the roaming client is moved to the new controller. 
Inter-controller layer-3 roaming occurs when a client is associated with an AP that is controlled by a controller belonging to a different subnet. 
In layer-3 roaming, the database entry of the roaming client is not moved to the new controller. 
Rather, the old controller marks the client's entry as \textit{anchor entry}. 
The entry is copied to the new controller and marked as \textit{foreign entry}. 
Therefore, the original IP address of the roaming client is maintained by the old controller. 
CUWN enables network administrators to establish mobility groups, where each group may consist of up to 24 controllers. 
A client can roam among all the controllers in a mobility group without IP address change, which makes seamless and fast roaming possible.


%\todo[inline,color=cyan]{any details about how the association algorithm works? \\ \textit{The previous paragraphs describe the association algorithms.} }  


%2012 August, 2014 June
% Individual - Seamless handoff
\textbf{Odin's Mobility Manager (OMM).}
\label{AMOdin}
%In this section we review the mobility and load balancing algorithm proposed by Odin \cite{Odin,Odin2} (see Section \ref{Odin_arch} ).
%As mentioned earlier, Odin introduces the concept of Light Virtual AP (LVAP), which is a small data structure residing on APs to indicate client association.
Odin  \cite{Odin,Odin2} enables seamless handoffs through LVAP migration between APs. 
The Odin controller maintains a persistent TCP connection per Odin Agent running on AP, thereby, switching among agents does not require connection reestablishment. 
In addition, the delay of a re-association equals the delay of sending two messages from the Odin controller to the old and new APs. 
The first message removes an LVAP from the old AP, and the second message adds an LVAP to the new AP. 
Assuming that the messages are sent successfully, the delay equals the longest round trip time (RTT) between the controller and the two APs, which depends on network size. 
%note: the controller may communicate with the to APs concurrently

To show the effectiveness of LVAPs, Odin employs a simple RSSI-based AsC mechanism.
The mobility application (running on the controller) selects the Odin Agent with highest RSSI if client movement is detected by the controller. 
The effect of handoff on TCP throughput has been evaluated using two APs and a client. 
While a period of throughput drop in regular 802.11 layer-2 handoff is observable, Odin does not show any throughput degradation.
%The authors also reported the maximum number of LVAP handoffs that does not cause throughput drop.
Performance evaluations also show that executing 10 handoffs per second results in a negligible reduction in TCP throughput.

%It is worth mentioning that, before Odin, seamless mobility support through VAP migration was proposed in \cite{Grunenberger2010a}.
%We do not discuss about the details of this mechanism due to its similarity to Odin.



%%%%%%%%%%%%%%%%%%%
%2015 Nov
% Individual - Seamless handoff


\label{AEtherflow_AM}
\textbf{$\AE$therFlow.} \cite{AEtherFlow} argues that handoff support through LVAP migration (e.g., Odin \cite{Odin2} and OpenSDWN \cite{OpenSDWN}) imposes high computational and communication overhead, especially in large networks with many mobile clients.
They propose a predictive handoff strategy by relying on the extended OpenFlow protocol proposed in this work (see Section \ref{AEtherFlow}).
The handoff mechanism works in three phases: 
\begin{itemize}
	\item \textit{Prediction}: %The controller collects the RSSI of the links between all clients and APs. 
	The controller predicts an association when the RSSI of a client to its associated AP declines while its RSSI to another AP increases. 
	\item \textit{Multicasting}: By updating OpenFlow tables, the controller multicasts the packets of the client to both the current AP and the predicted AP.
	\item \textit{Redirection}: The controller will redirect the client's traffic to the new AP after the handoff completion. If no handoff occurs, then the multicasting is stopped.
\end{itemize}

Experiments using two APs and a client shows that the handoff delay of $\AE$therFlow is around 5.3 seconds, compared to the 7.1 seconds delay of 802.11 standard.
%A 9Mbps UDP traffic is sent to the client.
%Handoff duration is measured as the time interval during which throughput drops below 8Mbps.
% No comparison with Odin and OpenSDWN?


% 2016 April
% Individual - Seamless handoff
\textbf{BIGAP.}
\label{BIGAPhandoff}
\cite{BIGAP} uses the 25 non-overlapping channels of 5GHz band to form disjoint collision domains for handoff.
As explained in Section \ref{Archs}, a separate NIC is used to periodically overhear packets on all channels, which enables the controller to compute the potential SNR values of client-AP links.
A handoff happens when a higher SNR would be achievable for a client.
However, to avoid the ping-pong effect, an 8dB hysteresis value is used. 

BIGAP performs handoff through client channel switching.
Since all APs share the same BSSID, in order to handoff a client from $AP_{1}$ to $AP_{2}$, the controller instructs $AP_{1}$ to send a channel switching command to the client.
The client then switches to the channel being used by $AP_{2}$.
Since both APs use the same BSSID, the client does not notice handoff.

Performance evaluations (using two APs) show that the BIGAP handoff is about 32 times shorter than the regular 802.11 handoff.
In addition, BIGAP results in lower energy consumption because, it moves the overhead of handoff to APs and there is no need for the clients to scan the channels.
In addition, while 802.11 results in zero throughput for about 4 seconds, BIGAP shows only 5\% throughput reduction during the handoff.


%------------------------------------------------------------------ COLOR
\subsection{Client Steering}
\label{client-steering}
Based on the scope of the optimization problem employed, we categorize client steering mechanisms into two groups: \textit{\textbf{centrally-generated hints}} and \textit{\textbf{centrally-made decisions}}.
In AsC mechanisms using centrally-generated hints, the controller relies on the global network view to generate hints for the association of clients.
In other words, the controller does not make the final decision about associations and instead, enables the clients to make more informed decisions through the hints conveyed.
On the other hand, in AsC mechanisms using centrally-made decisions, the controller makes the association decisions and enforces the clients to apply them.
We will discuss in Sections \ref{AM-InvidualOpt} and \ref{AM-GlobalOpt} the two sub-categories of mechanisms based on centrally-made decisions.

\subsubsection{\textbf{Centrally-Generated Hints}} 
\label{Per-clientAM}
In this section we review AsC mechanisms that employ client steering through hints generated centrally.


% REVISION - I REMOVED THIS IN THE NEW VERSION
% %2010
% % Individual - Client steering - hidden node avoidance
% \label{DysonAM}
% %The AsC mechanism of DenseAP \cite{DenseAP} has been improved by Dyson \cite{Dyson} as follows.
% \textbf{Dyson.} Using the network map constructed by Dyson \cite{Dyson}, the controller detects if two clients or APs cause a hidden-terminal collision.
% To remedy this situation, the controller changes the channel of the AP with fewer associated clients.
% Dyson's set of APIs enables the AP to inform its clients about channel switching and avoid the overhead of rediscovery and re-association.
% %Dyson also introduces a simple VoIP handoff strategy through which VoIP clients are connected to special APs referred to as VoIP APs. 
% %Among the potential points of connection, a VoIP client connects to the AP with highest available bandwidth.




%%%%%%%%%%%%
%\label{SDWLANdam}
% SDWLAN \cite{SDWLAN,SDWLAN2} (see Section \ref{SDWLANarch}) proposes a fast handoff mechanism for a WLAN in which all APs operate on the same channel.
%From the clients' point of view, all the APs are conceived as a \textit{One Big AP}.
%When a client needs to change its AP, the controller must install new rules on the new wireless access switch.
%The controller also updates OpenFlow switches to direct the traffic being exchanged between the controller and APs.
%
%The throughput of TCP and UDP are evaluated to measure the performance of this handoff mechanism.
%They found that AP handoff in regular 802.11 networks incurs nearly one second interruption on TCP/UDP session. 
%This interrupt is due to MAC re-authentication, re-association and MAC address relearning. 
%In addition, the TCP session interruption takes more than one second due to TCP's \textit{slow start} mechanism. 
%On the contrary, TCP and UDP throughput are slightly affected when the proposed handoff mechanism is used.

%\todo[inline, color=cyan]{when? \\ If there is a better AP in terms of defined metrics. It is event-triggered not time-triggered technique.}  

%2014 Feb
% Individual - Client steering - maximizing bandwidth
\textbf{BestAP.}
\label{BEST-AP}
\cite{BEST-AP} proposes an AsC mechanism based on the estimation of \textit{available bandwidth} (ABW) for each client at every AP in its vicinity. 
The available bandwidth depends on channel load, which varies with packet loss and PHY rate. 
The estimated available bandwidth at PHY rate $r$ is computed as,
%
\begin{equation}
E[ABW(r)]=\frac{8S_{data}(1-B)}{\sum_{k=0}^{n}(1-p_s(r))^kp_s(r)T_{tx}(r,k)},
\end{equation}
%
where $S_{data}$ is data size (bytes), $p_s$ is the probability of successful packet transmission, $T_{tx}(r,k)$ is the transmission time of a packet during the \textit{k}th transmission attempt with PHY rate $r$, and $B$ is the fraction of channel busy time. 
$B$ is measured through using the CCA register of NIC.
%If there is no such hardware support, $b$ can be calculated using the method proposed in \cite{ChannelLoad}, which is based on measuring the airtime consumed by each packet. 
$p_s(r)$ is measured using the statistics provided by the rate adaptation algorithm. 
All other parameters are configured statically. 

A scheduler is run on each client to allocate a measurement period (e.g., 50ms every 2s) during which the client sends data to all nearby APs in order to update packet loss and channel busy fraction. 
%This updates the ABW of each client at their reachable APs.
A monitoring service is run on APs to collect the statistics from clients, calculate ABW, and send a report to the controller.
The controller sends the estimated ABWs of the best five APs to each client periodically. 
%The report is also sent when the ABW of a client has been changed more than $x\%$ ($x=10$ is used during the experiments). 
BEST-AP only considers the ABW of downlinks.

%Experimental results show that BEST-AP's ABW estimation is more accurate than WBest \cite{WBest}. 
Testbed evaluations show that the delay overhead of ABW estimation is less than 50ms, which is appropriate for a dynamic AsC mechanism.
%\todo[inline, color=cyan]{how do you claim it is appropriate for a DAC mechanism? \\ \textit{It is not my claim. It is the claim of authors !!} }  
Compared to SSF \cite{SSF}, the proposed AsC mechanism shows 81\% and $176\%$ improvement in throughput for static and mobile clients, respectively.
%\todo[inline, color=cyan]{the above sentence needs work- please correct it}  


%2015
% Global - Client steering - equal number of clients per AP
\textbf{Ethanol.}
This architecture \cite{Ethanol} (see Section \ref{EthanolArch}) has been evaluated through running a load-aware AsC mechanism that aims to balance the number of associated clients among APs.
When a client requests to join an AP with higher load, the controller drops the request to force the client look for another AP.
A simple testbed with two APs and up to 120 clients shows that the maximum difference between the number of clients associated to APs is two, which is due to the concurrent arrival of association requests.



%\textbf{vBS.}
%The vBS \cite{vBS} architecture (please see Section \ref{Archs}) also proposes a fast handoff mechanism to reduce reconnection delay.
%Experimental results show that vBS can perform handoff in less than 65ms without any packet drop.


%------------------------------------------------------------------ COLOR
\subsubsection{\textbf{Individual Optimization through Centrally-Made Decisions}}
\label{AM-InvidualOpt}
In this section we review AsC mechanisms that employ client steering through decisions generated centrally. 
These mechanisms, however, do not define a global optimization problem; thereby they do not take into account the effect of an association on other clients/APs.
Due to the individual nature of association control, these mechanisms only improve the overall network performance, and cannot be used to enforce fairness.



%2008
% Individual - Client steering - ap load balancing
\textbf{DenseAP.}
\label{DenseAP-AM}
The AsC mechanism proposed by DenseAP \cite{DenseAP} (see Section \ref{DenseAP_arch}) works as follows.
The \textit{available capacity} metric is defined to rank all the APs a client could be associated with. 
A client associates to the AP with highest available capacity. 
The available capacity of $AP_{i}$ operating on channel $ch_{i}$ is defined as follows,
\begin{equation}
AC_{AP_{i},c_{j}}^{ch_{i}} = F_{AP_{i}}^{ch_{i}} \times r_{AP_{i},c_{j}}^{ch_{i}},
\end{equation} 
where $F_{AP_{i}}^{ch_{i}} $ is the \textit{free airtime} of $AP_{i}$ on channel $ch_{i}$, and $r_{AP_{i},c_{j}}^{ch_{i}}$ is the \textit{expected transmission rate} of $c_{j}$ when communicating with $AP_{i}$.
Free airtime is estimated by measuring the amount of time that the MAC layer contends for channel access to send a high-priority packet.
The expected transmission rate is estimated using the RSSI of probe request frames received at the AP. 
A mapping table is used for this purpose.
APs hear the probe requests of clients and send reports to a controller. 
APs also measure their free airtime and report it to the controller. 
The controller selects the AP with highest available capacity for each client to associate with. 
To instruct a client associate with the selected AP, only the selected AP responds to the client's probe message. 

DenseAP also proposes a dynamic load balancing algorithm that periodically decides about associations.
In particular, the controller checks the free airtime of APs every minute. 
An AP is \textit{overloaded} if its free airtime is less than 20\% and it has at least one associated client. 
If an overloaded AP exists, the controller considers its clients as the potential candidates for association with APs experiencing lower load.
% REMOVAL CANDIDATE
A candidate AP must satisfy these conditions: 
(i) the expected transmission rate of clients when associated with the new AP must not be lower than the current transmission rate, and (ii) the free airtime of new AP must be at least 25\% more than the current AP. 
During each decision period, at most one client is allowed to be associated with a new AP, and two consecutive associations for a client is prohibited to avoid the ping-pong effect.

%The performance of DenseAP's AsC mechanism is evaluated through empirical experiments using the 5GHz band (802.11a) with 8 channels. 
Empirical evaluations show a 40\% to 70\% increase in per-client throughput, compared to SSF. 
Moreover, the authors conduct a small experiment using three clients and two APs to show that the load balancing algorithm improves the throughput of clients by more than 200\%. 



% 2012
% Individual - Client steering - fair bandwidth allocation 
\textbf{Odin's Load Balancing (OLB).}
This load-balancing mechanism \cite{Odin2} periodically (every minute) inquires APs to collect the RSSI relationship between clients and APs, and then LVAPs are evenly distributed between APs to balance their loads.
%Using 10 APs and 32 clients, the efficiency of this mechanism on TCP throughput has been evaluated. %when a fixed rate 6Mbps is used.
The evaluations on a testbed with 10 APs and 32 clients show that around 50\% and 15\% of clients were able to receive a fair amount of throughput when the proposed mechanism was enabled and disabled, respectively.



%2013 Nov, 2015 June
% Individual - Client steering - energy efficiency
\textbf{EmPOWER.} 
\label{EmPowerAM}
This mechanism \cite{EmPOWER,Primitives} relies on client steering to improve the energy efficiency of APs.
APs are partitioned into clusters, where each cluster has one master and multiple slaves. 
Master APs are always active, and they are manually selected during the network deployment phase to provide a full coverage.
Slave APs are deployed to increase network capacity.
These APs are turned on/off using the finite state machine (FSM) depicted in Figure \ref{fig_FSM_EmPOWER}.  
%
\begin{figure}[!t]
	\centering
	\includegraphics[width=0.6\linewidth]{EmPOWER-energy-manager-FSM.pdf}
	\caption{Finite state machine (FSM) of energy manager for a slave AP in EmPOWER \cite{EmPOWER}.}
	\label{fig_FSM_EmPOWER}
\end{figure}
%
%\todo[inline, color=cyan]{what is the clustering strategy? \\ \textit{There is no specific clustering strategy in the paper. I added more explanation to clarify it a little more.}}  
In the ON mode, all wireless interfaces of the AP are on. 
In OFF mode, only the Energino \cite{Energino} module of the AP is on.  

Two metrics are defined for a slave $AP_i$ belonging to cluster $clu(AP_i)$: $\mathcal{C}_{clu(AP_i)}$ is the number of clients in the cluster, and $\mathcal{C}^{min}_{clu(AP_i)}$  is the minimum required number of clients in cluster $clu(AP_i)$ to keep $AP_i$ active. 
A slave $AP_i$ transitions from ON mode to OFF mode if: (i) the number of its cluster's clients is less than $\mathcal{C}^{min}_{clu(AP_i)}$, and (ii) $AP_i$ has been inactive (i.e., no client associated) for at least $T_{idle}$ seconds. 
Also, $AP_i$ transitions from OFF mode to ON mode if: (i) the number of its cluster's clients is at least $\mathcal{C}^{min}_{clu(AP_i)}$, and (ii) $AP_i$ has been OFF for at least $T_{offline}$ seconds. 

The mobility manager associates a client to a new AP with higher SNR. 
However, to establish a balance between performance and energy consumption, the mobility manager may associate a client to an AP with lower SNR but smaller $\mathcal{C}^{min}_{clu(AP_i)}$.  
In this way, the energy manager is able to turn off the APs with higher $\mathcal{C}^{min}_{clu(AP_i)}$  in order to decrease energy consumption. 
Re-association is performed if there is a better AP in terms of SNR and $\mathcal{C}^{min}_{clu(AP_i)}$. 
By relying on the Odin \cite{Odin2} APIs, the authors showed that this AsC mechanism was implemented as a Java network application with only 120 lines of codes.


%2016 July
% Individual - Client steering - ap load balancing
\textbf{Adaptive mobility control (AMC).} 
\cite{mob-Essex-2016} shows that using a fixed RSSI threshold by clients results in an unbalanced load of APs.
% idea
They propose a heuristic algorithm which uses RSSI and traffic load of APs to provide dynamic hysteresis margins on AP traffic load level. 
The load of $AP_{i}$ is defined as,
%
\begin{equation}
\label{eq:eq18}
L_{AP_{i}}=\left\{\begin{array}{ll}
B\textrm{\ \ \ \ \ \ \ \ \ \quad\quad\quad\quad\quad\quad\quad if  $|\mathcal{C}_{AP_{i}}|=0$}\\
0.8\times B +0.2\times |\mathcal{C}_{AP_{i}}| \textrm{\quad\ \  if $|\mathcal{C}_{AP_{i}}|>0$}
\end{array}\right.
\end{equation}
where $|\mathcal{C}_{AP_{i}}|$ is the number of clients associated with $AP_{i}$, and $B$ is channel busy time.
The algorithm defines three thresholds on AP load and RSSI: low, medium, and high.
%These thresholds are determined through applying experimental tests to measure the effect of AP load and RSSI value on network throughput. 
Using these thresholds, a client is associated with a new AP that satisfies one of these conditions: (i) higher signal strength and lower load, (ii) significantly higher signal strength and slightly higher load, or (iii) significantly lower load and slightly lower signal strength.
% QUESTIONS (Second phase) --> COMPLETED
% ^^- AP load metric  
% ^^ - hysteresis margins
% ^^ - How is the algorithm working?
% ^^ - adaptive margins ==> based on which parameter, it is adaptive?
% QQQQQQQQQQQQQQQQQQQQQQQQQQQQQQQQQQQQQQQQQQ

The proposed adaptive mobility manager has been implemented using Odin.
Empirical results show more than 200\% improvement in TCP throughput, compared to SSF.





%2016 June
% Individual - Client steering - reduce delay
\textbf{Associating to good enterprise APs (AGE).}% demand-agnostic
\label{AGE}
The main objective of AGE \cite{WiFiSeer} is the reduction of clients' packet exchange delay over wireless links through client steering.
AGE has two main phases: learning and AP selection. 
%The learning phase uses a training set obtained during network operation for a week every three months.
During the learning phase, the performance metrics and environmental factors are pulled every minute from all APs using SNMP \cite{SNMP}. 
These metrics include AP-client RTT, RSSI, SNR, number of associated clients, channel number, frequency band, AP location, day of week, and time.
%The wireless latency of each client is calculated every minute through a technique called \textit{ping2}. 
%The authors argue that measuring wireless latency through sending only one ping packet might be inaccurate due to the wake up delay of NIC. 
%Therefore, ping2 uses two consecutive ping packets.
%The first ping is used to wake up the client's NIC, if it is in energy-saving mode. 
%The second ping is sent to calculate latency. 

Using the collected training set, the authors use the \textit{random forest} \cite{random-forest} technique to generate a two-class learning model for classifying APs into \textit{high latency} and \textit{low latency}.
Figure \ref{fig_AGE} shows an overview of AGE and its two main components: (i) the \textit{AGE app} that is installed on each client's mobile phone, and (ii) the \textit{AGE controller} that uses the random forest model to classify the APs in the vicinity of each client.
AGE operates as follows:
\begin{itemize}
	\item The AGE application on a client device sends an AGE request (including the list of achievable APs) to the AGE controller, periodically. 
	%AGE requests are sent every 5 and 20 minutes when the device's screen is on and off, respectively.
	\item The AGE controller pulls the SNMP data of the APs requested by AGE app.
	\item The latency class of each AP is predicted using the learning technique mentioned earlier.
	\item The AGE controller informs the AGE application about the best AP nearby.
	\item The client re-associates with a new AP using AGE app. 
\end{itemize}

The authors deployed AGE at Tsinghua University campus where over 1000 devices used the network for 2.5 months. 
Measurements confirmed that the wireless data exchange delay of more than 72$\%$ of clients has been reduced by over 50$\%$.

\begin{figure}[!t]
	\centering
	\includegraphics[width=0.9\linewidth]{AGE.pdf}
	\caption{Associating to Good Enterprise APs (AGE)\cite{WiFiSeer}. Each client is instructed to connect to the AP that provides minimum delay.}
	\label{fig_AGE}
\end{figure}




%------------------------------------------------------------------ COLOR
\subsubsection{\textbf{Global Optimization through Centrally-Made Decisions}}
\label{AM-GlobalOpt}
In this section we review AsC mechanisms that employ client steering by formulating a problem that aims to optimize performance parameters globally.
These mechanisms propose heuristics to solve the NP-hard problems that usually aim to achieve network-wide fairness.



%2004-2007
% Global - Client steering - fair bandwidth allocation
\textbf{Association Control for fairness and load balancing (ACFL).} 
In \cite{F-LB-AsscCtrl-2004} and \cite{F-LB-AsscCtrl-2007}, the authors address the unbalanced load of APs as a result of using SSF or LLF.
%However, their fairness measure is max-min fairness.Max-min throughput fairness can significantly reduce aggregate throughput in multi-rate WLANs. The max-min time fairness problem they consider is intended for single-rate WLANs.
%from the paper:  The collected infor-mation is reported to a network operation center (NOC) which runs our algorithm to come up with the user-AP association de-cisions.
%They consider the bandwidth constraints of APs in both wireless and wired links.
To address this challenge, an AsC problem is formulated to establish the max-min bandwidth fairness among APs. 
Intuitively, the load of a client on its associated AP is inversely proportional to the effective bit rate of AP-client link. 
The load of $AP_{i}$ (denoted as $L_{AP_{i}}$) is modeled as the maximum of the aggregated loads of its wireless and wired links generated by all clients $c\in \mathcal{C}$, as follows,
\begin{equation}
L_{AP_{i}}=\max\left\{ \sum_{\forall c_{j}\in \mathcal{C}} \frac{\omega_{c_{j}} \times X_{AP_i, c_j}}{r_{AP_i, c_j}},  \sum_{\forall c_{j}\in \mathcal{C}} \frac{\omega_{c_{j}} \times X_{AP_{i},c_j}}{R_{AP_{i}}} \right\}
\end{equation}
where $r_{AP_i, c_j}$ is the transmission rate between $AP_{i}$ and $c_{j}$, $R_{AP_{i}}$ is the transmission rate of wired interfaces of $AP_{i}$, $X_{AP_{i},c_{j}}\in\{0,1\}$ is the association state of $c_{j}$ to $AP_{i}$, and $\omega_{c_{j}}$ is the traffic volume size of client $c_{j}$. 
In other words, the load of $AP_{i}$ is defined as the period of time this AP requires to handle the traffic of its associated clients. 
$AP_{i}$ provides bandwidth $X_{AP_{i},c_{j}} \times \omega_{c_{j}}/L_{AP_{i}}$ to client $c_{j}$. 

Two approximation algorithms are introduced to solve the formulated NP-hard max-min fair bandwidth allocation problem. 
The first algorithm solves the problem for unweighted greedy clients. 
The second algorithm proposes a solution for weighted and limited throughput demand of clients.
The algorithms are run periodically by a controller to update client associations. 
Simulation results show over 20$\%$ improvement in terms of average per-client bandwidth, compared to SSF and LLF.

%\todo[inline,color=cyan]{sentence not clear- please rewrite}  
%\todo[inline,color=cyan]{why did you remove the formulas? can you make the explanation more complete?\\ \textit{this paper is so confusing. I tried to add more explanation.}}  







%2014
% Global - Client steering - fair bandwidth allocation 

\textbf{Association control for proportional fairness (ACPF).}
\cite{Proportional-Fairness-AP-2014} argues that using throughput-based fairness (e.g., \cite{Time-based-basic-1, Time-based-basic-2}) in multirate networks leads to low overall network throughput because clients with a low bit rate can occupy the channel longer than those with higher rate. 
The authors investigate the problem of achieving proportional fairness by introducing the following objective function formulated based on the effective bandwidth of clients:
\begin{equation}
\sum_{\forall c_{j}\in \mathcal{C}} \rho_{c_{j}} \times  \log (\beta_{c_{j}}),
\end{equation}
where $\rho_{c_{j}}$ and $\beta_{c_{j}}$ are the priority and effective bandwidth of client $c_{j}$, respectively. 
$\beta_{c_{j}}$ is defined as 
\begin{equation}
\sum_{\forall AP_{i}\in \mathcal{AP}}X_{AP_{i},c_{j}}\times t_{AP_{i},c_{j}}\times r_{AP_{i},c_{j}}, 
\end{equation}
where $X_{AP_{i},c_{j}}\in\{0,1\}$ is the association index, $t_{AP_{i},c_{j}}\in[0,1]$ is the effective normalized time, and $r_{AP_{i},c_{j}}$ is the PHY rate between $c_{j}$ and $AP_{i}$. 
Achieving fairness through association is formulated as a linear-programming problem to maximize the objective function.
%\todo[inline, color=cyan]{"The client traffic-based proportional fairness"- this is not a correct sentence. not clear.}  
A centralized algorithm, called \textit{non-linear approximation optimization for proportional fairness} (NLAO-PF), is proposed to solve the optimization problem.
%This algorithm computes client-AP association relations and transmission time of clients. 
It has been proven that the approximation ratio of NLAO-PF is 50\%.

The authors used the OMNet++ \cite{OMNET} simulator to evaluate the effectiveness of NLAO-PF compared to cvapPF \cite{cvapPF} in terms of average throughput. 
The average improvement is reported as 18.8$\%$ and 35$\%$ for uniformly distributed clients as well as clients in a hotspot area, respectively.



%2014
% Global - Client steering - fair bandwidth allocation 

\textbf{Association control with heterogeneous clients (ACHC).}  % demand-agnostic?
Experimental analysis of the negative impacts of legacy clients (802.11a/b/g) on 802.11n clients has been reported by \cite{802.11n-AP-Association-2014}.
The authors propose a two-dimensional Markov model to calculate the uplink and downlink throughput of heterogeneous clients. 
MAC efficiency of a client $c_{j}$ is defined as follows,
\begin{equation}
\label{MACeffMetr}
\Upsilon({c_{j}}) = \frac{Th_{c_{j}}^{up}+Th_{c_{j}}^{down}}{\min\{1,\; p_{c_{j}}^{up}+p_{c_{j}}^{down}\}\times r_{AP_{i},c_{j}}},
\end{equation}
where $Th_{c_{j}}^{up}$ and $Th_{c_{j}}^{down}$  are the estimated uplink and downlink throughput of client $c_{j}$, respectively;
$p_{c_{j}}^{up}$ and $p_{c_{j}}^{down}$ are, respectively, the uplink and downlink traffic probabilities of $c_{j}$; 
$r_{AP_{i},c_{j}}$ is the optimal PHY rate between $c_{j}$ and $AP_{i}$. 
AsC is formulated through an optimization problem to provide a bandwidth proportional to each client's achievable data rate, which is obtained by maximizing $\sum_{\forall c_{j}\in \mathcal{C}} \log \Upsilon({c_{j}})$.

Two heuristic dynamic AsC algorithms are proposed to solve the optimization problem: \textit{FAir MAC Efficiency} (FAME) and \textit{Categorized}. 
Instead of maximizing all MAC efficiencies, these algorithms maximize the minimum MAC efficiency.
To run FAME, the data rate and traffic load of all clients in nearby BSSs must be collected from each client's point of view.
With \textit{Categorized}, the type of all clients in nearby BSSs must be collected as well.
However, \textit{Categorized} is less sensitive to network dynamics because it tries to maximize the number of similar-standard 802.11 clients connected to each AP.
Therefore, \textit{Categorized} takes advantage of dense deployment and alleviates the performance degradation of 802.11n clients that is caused by the presence of legacy clients. 

Simulation and testbed experiments confirmed the higher performance of \textit{FAME} and \textit{Categorized} compared to SSF and ACFL in terms of TCP and UDP throughput, MAC efficiency and aggregated throughput of clients supporting different standards.
Although the conducted experiments use distributed execution of \textit{FAME} and \textit{Categorized}, collecting the information required by these algorithms imposes a high burden on clients. 
Therefore, these algorithms are mostly suitable for SDWLANs. 
%Furthermore, it is required to change beacon packet format as data collection is performed through exchanging beacon packets.


% demand-agnostic
%2015
% Global - Client steering - fair bandwidth allocation 

\textbf{Association control and CCA adjustment (ACCA).} 
Two central algorithms for AsC and CCA threshold adjustment are proposed in \cite{Association-CCA-2015} for dense AP deployments.
%\todo[inline, color=cyan]{why did you call it a joint approach while the two algorithms are separate? do they work together? \\ \textit{thanks for your precise comment. It is not joint mechanism.}}  
The basic idea of the AsC mechanism is to utilize the SINR of a client perceived by an AP in order to measure the interference (congestion) level of APs. 
The network is modeled as a weighted bipartite graph (WBG), and the AsC problem is formulated as a maximum WBG matching combinatorial optimization problem. 
The weights of edges in WBG are the uplink SINR values. 
The optimization problem's objective is to maximize the sum of the clients' throughput, where the throughput upper-bound is computed using the Shannon-Hartley formula \cite{rappaport1996wireless}.
Per-BSS CCA level is adjusted by using a constraint of the optimization problem based on the cell-edge SINR of each BSS. 
The optimization problem has been solved using Kuhn-Munkres \cite{Kuhn} assignment algorithm.
%The proposed centralized technique requires global knowledge about all clients and APs, which is collected by the functionalities of IEEE 802.11k \cite{802.11k}. 
% Details of graph modeling
% Details of optimizaiton problem

MATLAB simulations show a 15 to 58\% improvement at the 10th percentile of cumulative distribution of clients throughput, compared to SSF.
%In addition, the improvement level is higher for networks with randomly deployed APs, compared to uniformly-spaced AP placement.
Furthermore, the CCA adjustment algorithm significantly improves the throughput of cell-edge clients, compared to the network with fixed CCA threshold.
%\todo[inline, color=cyan]{please correct the last sentence, not clear at all. any numerical values? \\ \textit{More explanations are added.}}  



%2015
% Global - Client steering - fair time fairness allocation 

\textbf{Demand-aware load balancing (DALB).} % more focus on scheduling rather than association (transmission time adjustment) --> it's not based on CSMA/CA
\cite{Demand-aware-LB-Association-15} formulates joint AP association and bandwidth allocation as a mixed-integer nonlinear programming problem that includes the bandwidth demand of clients. 
The objective is to maximize the aggregated bandwidth of clients, which establishes a trade-off between throughput gain and time-based user fairness. 
The key idea of this work is the inclusion of clients' bandwidth demands in the computation of APs' load. 
For a client $c_{j}$, bandwidth demand is defined as $\beta_{c_{j}}=S_{c_{j}}/{T}$, where $S_{c_{j}}$ is the size of data received for client $c_{j}$ by its associated AP during the allocable transmission time of the AP, i.e., $T$. 
The transmission time demand of $c_{j}$ associated to $AP_{i}$ is defined as $T_{AP_{i},c_{j}} = \beta_{c_{j}}\times T/r_{AP_{i},c_{j}}$.

The optimization problem's objective is defined as follows,
\begin{equation}
\label{demandEqu}
\sum_{\forall c_{j} \in \mathcal{C}} \log\sum_{\forall AP_{i} \in \mathcal{AP}} \frac{X_{AP_{i},c_{j}}\times r_{AP_{i},c_{j}} \times t_{AP_{i},c_{j}}}{T},
\end{equation}
where the transmission time allocated to $c_{j}$ associated with $AP_{i}$ is expressed by $t_{AP_{i},c_{j}}$. 
The authors proved that maximizing Equation \ref{demandEqu} subject to a constraint on the transmission time demand of clients, i.e., $0 \le t_{AP_{i},c_{j}} \le T_{AP_{i},c_{j}}$, leads to proportional time fairness among clients.

A two-step heuristic algorithm is introduced to solve the optimization problem. 
The first step associates a client to an AP with the lowest allocated transmission time at each iteration until all clients are associated. 
At each iteration, the algorithm selects a client with the largest bandwidth demand. 
The second step schedules the transmission time of APs to establish proportional transmission time fairness among clients. 

Simulations show a 23$\%$ improvement in clients' throughput compared to SSF, ACPF and NLB \cite{NLB-13}.
In addition, this improvement has been achieved without sacrificing fairness. 
In particular, while bandwidth fairness is slightly better than that of SSF, ACPF and NLB, time fairness is significantly improved compared to these algorithms.





%2016
% Global - Client steering - maximizing throughput of clients

\textbf{Migration-cost-aware association control  (MCAC).} 
\label{Migration-DAM}
\cite{Migration-DAM} investigates the problem of max-min fairness subject to a migration cost constraint.
An integer linear programming problem is formulated with the aim of maximizing the minimum throughput of clients in order to establish max-min fairness.
The migration cost of clients is the main constraint of the formulation.
The load of $AP_{i}$ is defined as $L_{AP_{i}}=\sum_{\forall c_{j} \in \mathcal{C}} X_{AP_{i},c_{j}}/r_{AP_{i},c_{j}}$.
Assuming the achievable throughput of clients is proportional to the inverse of the load of the associated AP (i.e., $1/L_{AP_{i}}$), maximizing the minimum throughput of clients is equivalent to minimizing $\max_{\forall AP_{i} \in \mathcal{AP}}L_{AP_{i}}$. 
%Consequently, the objective of the optimization problem is defined as $\min \max_{i} H_i$, where $H_i$ is the load of AP $i$ after client re-association. 
The migration cost constraint is defined as, 
\begin{equation}
\sum_{\forall AP_{i} \in \mathcal{AP}}\;\sum_{\forall AP_{l} \in \mathcal{AP}} (Y_{AP_{i},AP_{l}}\times M_{c_j}) \le M,
\end{equation}
where $Y_{AP_{i},AP_{l}}\in\{0,1\}$ indicates whether a client $c_{j}$ currently associated with $AP_{i}$ will be associated with $AP_{l}$, $M_{c_{j}}$ is the migration cost of client $c_{j}$ from $AP_{i}$ to $AP_{l}$, and $M$ is the total permissible migration cost.

The authors showed that the proposed problem is NP-hard, and they proposed an approximation algorithm, named the \textit{cost-constrained association control algorithm} (CACA), to solve the optimization problem. 
In this algorithm, the problem is divided into two sub-problems: client removal and client re-association. 
In the first sub-problem, a subset of clients are removed from their current AP with the aim of minimizing the maximum AP load, considering the migration cost constraint. 
%When migration costs are equal, the migration cost constraint is simplified as a limitation on the number of migrating clients, i.e., no more than $K^{\prime}=K/c$ clients can be removed. 
%These clients impose the highest load on the maximum-loaded AP.
In the second sub-problem, the algorithm performs client re-association to minimize the load of maximum-loaded AP.

NS3 \cite{NS3}  simulations and experimental results show improvements in terms of the number of re-associations, throughput, loss ratio and end-to-end delay, compared to SSF, ACPF and GameBased \cite{GameBased}. 
%As SSF, PFairness and GameBased do not consider the number of associated clients as a limitation, the proposed cost-constrained association control algorithm shows lower number of re-associations. 



%\textbf{Joint optimization of energy consumption and interference (JOEI).}
%A short-term mixed-integer programming (SMIP) optimization problem is proposed in \cite{DCA-2016-1} to jointly improve energy efficiency and interference through client-AP associations, dynamic channel selection and controlling the operation modes of APs. 
%Figure \ref{fig_JITwlan} illustrates the block diagram of the proposed optimization problem. 
%The ON and OFF states of APs can be controlled by a controller. 
%The objective function is defined as the average power consumption of APs during a long period. 
%%\todo[inline,color=cyan]{$\checkmark$ Each client should be associated with an AP, and each AP communicates on a channel with all its associated clients -- not clear? \\
%%\textit{- It is not needed to mention this constraint. Because it is the basic assumption of all association algorithms.}} 
%The channel assignment constraint does not allow assigning same channel to two neighboring APs. 
%Furthermore, the client queue stability constraint prevents queue overflow in clients through considering the demand and departure rate of clients' traffic.  
%%This work focuses on downlink communications (i.e., from AP to clients) in order to decrease the average power consumption of APs. 
%The network is modeled similar to \cite{Wcolor-2005} (see WCCA in Section \ref{WCCA}) through an undirected graph where vertices represent APs and an edge represents a potential interference between two APs.
%Since SMIP is an NP-hard problem, authors propose a polynomial-time linear programming relaxation to obtain the values of control variables.
%%Since SMIP is an NP-hard problem, authors propose a polynomial-time linear programming relaxation, named \textit{Just-In-Time WLAN} (JIT-WLAN), to obtain the values of control variables.
%The controller periodically solves the problem to perform (i) AP-client association control, (ii) AP operation mode control, and (iii) channel assignment.
%%The SMIP problem is solved at each time instance that leads to jointly minimize the long-term power consumption of APs and interference (by assigning non-overlapping channels to neighboring APs) without sacrificing clients demands. 
%%
%\begin{figure}[!t]
%	\centering
%	\includegraphics[width=0.8\linewidth]{InfoCom-JIT-WLAN.pdf}
%	\caption{Block diagram of the optimization problem proposed by JOEI \cite{DCA-2016-1}.}
%	\label{fig_JITwlan}
%\end{figure}
%%They demonstrate the performance and practicability of proposed algorithm through extensive simulations and an SDN-based framework. 
%%Each AP communicates with a central controller.
%%The controller runs the polynomial-time JIT-WLAN algorithm to solve the proposed optimization problem periodically (based on the decision making period). 
%
%For stationary clients, testbed evaluations show $8\%$ improvement in aggregate power consumption without sacrificing the throughput of clients, compared to always-on approach.
%For mobile clients, throughput is improved from 15 to 20 Mbps. 
%%However, using the always-on approach, throughput drops from 15 to 10 Mbps. 
%%Due to practical issues related to the overhead of re-association in channel reassignment, the channel assignment module is omitted in SDWLAN-based testbed experiments.



 
%2017
% Global - Client steering - minimizing the delay of download flows

\textbf{QoS-driven association control (QoSAC).}
\label{QoS-DAM}
\cite{flow-level-DAM} investigates the problem of flow-level association to address the QoS demands of clients.
Specifically, the backhaul capacity of APs is included in the association decision making.
This work proposes a mechanism for concurrent association of a client to multiple APs and supports flow-level routing of traffic. 
This mechanism facilitates dedicated management of each flow. 
For example, a client may use $AP_{1}$ for video streaming while using $AP_{2}$ to upload a file.

The main objective is to minimize the \textit{average inter-packet delay} of individual download flows. 
The inter-packet delay for a client associated to an AP is formulated based on the bi-dimensional unsaturated Markov model proposed in \cite{802.11n-AP-Association-2014}. 
An optimization problem is formulated to minimize the sum of the average inter-packet delay of all APs. 
The backhaul capacity of an AP should not be smaller than the sum of the arrival traffic rates of the AP, i.e., the sum of all download flow rates. 
This is included as a constraint of the optimization problem.
%\todo[inline, color=cyan]{not clear what is has to do with flow level? }
%\todo[inline, color=cyan]{not clear how it benefits from multiple associations \\ I added one sentence to clarify it more.}    
The proposed optimization problem is interpreted into a supermodular set function optimization \cite{supermodular}, which is an NP-hard problem. 
This problem has been solved using two heuristics: (i) greedy association, and (ii) bounded local search association. 
The greedy algorithm associates a client with the AP that minimizes the total inter-packet delay in each iteration.
In this algorithm, the association loop is continued until all clients are associated. 
The bounded local search is a polynomial-time algorithm similar to the algorithm proposed in \cite{boundedAlg}.
%\todo[inline, color=cyan]{add a high level explanation of these algorithms}  

Simulation results confirm the reduced average inter-packet delay achieved with this mechanism, compared to SSF and FAME \cite{802.11n-AP-Association-2014}. 
The delay is considerably reduced when the backhaul capacity is limited and clients form hotspots.

%From their paper:
% With the above features, we further consider client association in the high-density scenario. In this paper, “high-density”
%mainly implies the overlapped basic service set (OBSS) case, in which all the APs and clients are located within a certain coverage area and operate on the same channel.



%TABLE DAM
\begin{table*}
	\centering
	\scriptsize
	\caption{  Comparison of Association Control (AsC) Mechanisms  }
	\label{DAMtable} 
	\def\arraystretch{1}
	\begin{tabular}{|c|c|c|c|c|c|c|c|c|c|c|c|}
		\Xhline{3\arrayrulewidth}
		%	\textbf{Ref.} & \multicolumn{2}{|c|}{OneTwoThree} & \multicolumn{2}{|c|}{OneTwoThree}&\textbf{Channels} & \textbf{Dynamic/Static} & \textbf{\multicolumn{2}{|c|}{OneTwoThree}}\\ \Xhline{3\arrayrulewidth}
		%	
		\multirow{3}{*}{\textbf{Mechanism}}& \multicolumn{2}{c|}{Objective} & \multicolumn{3}{c|}{Optimization Scope}&\multirow{3}{*}{\textbf{Decision Metric}}& \multicolumn{2}{c|}{Traffic Awareness}&   \multicolumn{2}{c|}{ Performance Evaluation }\\ \cline{2-6}\cline{8-9}\cline{10-11}
		
		&\textbf{\makecell{Seamless \\ Handoff}}&\textbf{\makecell{Client \\Steering}}&\textbf{\makecell{Cent. Gen.\\ Hints}}&\textbf{\makecell{Indv. Opt. \\Cent. Made\\ Decisions }}&\textbf{\makecell{Glob. Opt. \\ Cent. Made\\ Decisions }}&& \textbf{Downlink}& \textbf{Uplink}& \textbf{Simulation} & \textbf{Testbed}\\ \Xhline{3\arrayrulewidth}
		%1
		%SSF\cite{SSF}&$\checkmark$&$\times$&RSSI&$\checkmark$&$\times$&$\times$&$\times$&$\checkmark$\\\hline
		%	
		\makecell{CUWN \cite{Cisco}} & $\checkmark $ & $\times $ & $\times $ & & $\times$&RSSI&$\times$&$\times$ &$\times$&$\times$\\\hline
		%		
		\makecell{OMM \cite{Odin2}} & $\checkmark $ & $\times $ &$\times $& & $\times$&RSSI&$\times$&$\times$ &$\times$&$\checkmark$\\\hline
		%2
		$\AE$therFlow \cite{AEtherFlow} & $\checkmark $ & $\times $ &$\times $& & $\times$&RSSI&$\times$&$\times$&$\times$&$\checkmark$\\\hline		
        %		
		BIGAP  \cite{BIGAP} & $\checkmark$ & $\times $ &$\times $ &  & $\times$&RSSI&$\times$&$\times$&$\times$&$\checkmark$ 	  \\\hline  		
        %        %        
        %        
        %			
		BestAP \cite{BEST-AP}& $\times$ &  \makecell{Bandwidth\\ Improvement }  &$\checkmark$& & $\times$&\makecell{Available\\ bandwidth}&$\checkmark$&$\times$&$\times$&$\checkmark$\\\hline
		%        
		%
		Ethanol \cite{Ethanol}& $\times$ & \makecell{AP Load \\ Balancing}  &$\checkmark $ &  & $\times$ &Number of clients &$\times$&$\times$&$\times$&$\checkmark$\\\hline		
		%
		DenseAP \cite{DenseAP}& $\times$ & \makecell{AP Load \\ Balancing} &$\times$ & \makecell{Overall\\ performance} & $\times$  & \makecell{Available\\ bandwidth}&$\checkmark$&$\times$&$\times$&$\checkmark$\\\hline
		%
        \makecell{OLB \cite{Odin2}} & $\times$ & \makecell{AP Load \\ Balancing} &$\times$& \makecell{Overall\\ performance} & $\times$ &\makecell{RSSI,\\Number of clients}&$\times$&$\times$ &$\times$&$\checkmark$\\\hline   
        %     
		%
		EmPOWER \cite{EmPOWER}& $\times $ & \makecell{Energy\\Efficiency} &$\times $ & \makecell{Overall\\ performance}& $\times$ &\makecell{RSSI,\\Number of clients}&$\times$&$\times$&$\times$&$\checkmark$\\\hline
        %		
		AMC \cite{mob-Essex-2016}& $\times$ &  \makecell{AP Load \\ Balancing}   &$\times$ & \makecell{Overall\\ performance} & $\times $ &\makecell{RSSI, Free airtime}&$\checkmark$&$\times$&$\times$&$\checkmark$\\\hline
        %        
		AGE \cite{WiFiSeer}& $\times$ & \makecell{Reducing Packet \\ Exchange Delay} &$\times$ & \makecell{Overall\\ performance}& $\times$&\makecell{Latency}&$\times$&$\times$&$\times$&$\checkmark$\\\hline
		%		
		%		
		ACFL \cite{F-LB-AsscCtrl-2007}& $\times$ & \makecell{AP Load \\ Balancing}  &$\times$&  & \makecell{Max-min\\ fairness} &\makecell{Free airtime, \\Transmission rate}&$\checkmark$&$\times$&$\checkmark$&$\times$\\\hline
		%3
        %		
        %		
        %        
        %4        %
		ACPF \cite{Proportional-Fairness-AP-2014}& $\times$ & \makecell{Fair Bandwidth\\ Allocation}  &$\times$&  & \makecell{Proportional\\ Fairness} &\makecell{Available\\ bandwidth}&$\checkmark$&$\times$&$\checkmark$&$\times$\\\hline
		%        
		ACHC \cite{802.11n-AP-Association-2014}& $\times$ & \makecell{Fair Bandwidth\\ Allocation}  &$\times$&  & \makecell{Proportional\\ Fairness} &MAC efficiency&$\checkmark$&$\checkmark$ & $\checkmark$& $\checkmark$\\\hline   
		 %
		%				        
		ACCA \cite{Association-CCA-2015}& $\times$ & \makecell{Maximize\\ Throughput} &$\times$&  & \makecell{Overall\\ Performance} &Uplink SINR&$\times$&$\checkmark$&$\checkmark$&$\times$\\\hline
		%6
		DALB \cite{Demand-aware-LB-Association-15}& $\times$ & \makecell{Maximize\\ Throughput}  &$\times$&  & \makecell{Proportional\\ Fairness} &Transmission time&$\checkmark$&$\times$&$\checkmark$&$\times$\\\hline
		%
		%8
		MCAC \cite{Migration-DAM}& $\times$ & \makecell{Fair Bandwidth\\ Allocation}  &$\times$&  & \makecell{Max-min\\ Fairness}  &\makecell{Throughput,\\Re-association cost}&$\checkmark$&\tiny$\times$&$\checkmark$&$\checkmark$		\\\hline
		%13
		QoSAC \cite{flow-level-DAM} & $\times$ & \makecell{Reducing Packet \\ Exchange Delay} &$\times$ &  & \makecell{Overall\\ Performance}  &Inter-packet delay&$\checkmark$&$\times$&$\checkmark$&$\times$
		\\\Xhline{3\arrayrulewidth}
	\end{tabular}
\end{table*}







\subsection{Association Control: Learned Lessons, Comparison, and Open Problems}
\label{AscProblems}
Table \ref{DAMtable} presents and compares the features of the AsC mechanisms.
Although most of the AsC mechanisms focus on client steering, we should note that client steering and seamless handoff are interdependent.
For example, when association decisions are made centrally to balance the load of APs, a quick reaction to network dynamics requires seamless handoff; otherwise, the effect of load balancing will be compromised by the overhead and delay of re-associations.
In the following, we study the features of AsC mechanisms and identify research directions.

%<><><><><><><><><><><><><><><><><><><><><><><><><><><><><><><><><><><><><><><><><><><><><><><><>


\subsubsection{\textbf{Optimization Scope of Client Steering}} 
\label{asc_opt_scope_cl_ste}
Based on our review, we have classified the client steering approaches into three categories: (i) centrally-generated hints, (ii) individual optimization through centrally-made decisions, and (iii) global optimization through centrally-made decisions. 
Our review shows that most of the AsC mechanisms fall into the last two categories, where their optimization problems is classified as follows:
%
\begin{itemize}
	\item \textit{Overall network performance}: To improve the overall performance of clients and APs. 
	For example, QoSAC minimizes the sum of inter-packet delay for all APs, and ACCA  maximizes the sum of clients' achievable throughput.
	Note that improving overall performance does not necessarily require defining an optimization problem, as Section \ref{AM-InvidualOpt} shows.
    %
	%
	\item \textit{Max-min fairness}: To maximize the minimum performance. In other words, \textit{max-min} fairness indicates providing a client with more resources would not be possible without sacrificing the resources of other clients \cite{Bonald2006}. 
	For example, the objective function of ACFL and MCAC maximizes the throughput of minimum-throughput client.
	%
	\item \textit{Proportional fairness}: Clients occupy the channel proportional to their transmission rate.
	Our review shows that proposing these approaches is motivated by the challenges of achieving fairness in multi-rate networks. 
	Multi-rate is caused by both client heterogeneity (e.g.,  802.11a/b/n) and the rate adaptation mechanism of 802.11 standards.
	Since the 802.11 DCF protocol aims to provide an equal channel access probability for all contending clients, low-rate clients occupy the medium longer than high-rate clients.
	In this case, the max-min throughput fairness results in sacrificing the performance of high-rate clients, which leads to a problem called \textit{performance anomaly}. 
	Performance anomaly in multi-rate networks may result in total throughput degradation, even in comparison to SSF \cite{DAM-G-5,bellalta2016interactions,802.11n-AP-Association-2014,SSF}. 
	A well-known solution is proportional fairness, which is usually achieved by defining a logarithmic objective function and scheduling the transmission time of clients to be proportional to their transmission rate. 
	ACPF, ACHC and DALB are examples of AsC mechanisms that establish proportional fairness. 
	%
\end{itemize}



Using proportional fairness in multi-rate networks decreases the performance of low-rate legacy clients and may dissatisfy their QoS requirements. 
We suggest two approaches to address this problem: (i) hybrid fairness, and (ii) demand-aware proportional fairness. 
In hybrid fairness, a weighted fairness metric is defined by assigning weights for the max-min and proportional fairness metrics. 
Therefore, it is possible to adjust the weights based on the demand of low-rate legacy clients. 
In the second approach, the demand of clients is merged with their rates when formulating the problem.



\subsubsection{\textbf{Decision Metrics}}
\label{asc_dec_met}
The most important component of an AsC mechanism is its decision metrics, as they affect on both accuracy and overhead.
When using the AsC mechanism of 802.11 standard (i.e., SSF), a client associates with the AP that is providing the highest RSSI. 
The main shortcoming of SSF is that it does not recognize the load of APs and the demand of clients.
Therefore, AsC mechanisms introduce additional metrics in their decision process. 

The "Decision Metric" column of Table \ref{DAMtable} summarizes the decision metrics employed.
Generally, our review shows that the most popular decision metrics are RSSI, AP load, throughput of clients, and packet exchange delay experienced by clients. 
Balancing the load of APs is a widely-adopted optimization metric as it implicitly results in an improved performance of clients.
In the mechanisms relying on this metric, load is modeled through parameters such as the number of associated clients, AP throughput, and channel busy time. 
Although these mechanisms implicitly improve the performance of clients, explicit consideration of clients' traffic results in higher performance.

Based on this, we can classify AsC mechanisms into two groups:
%
\begin{itemize}
	\item \textit{Demand-agnostic.} 
	Refers to the AsC mechanisms that associate each client with the best AP nearby, without taking into account clients' demands.
	For instance, DenseAP, ACPF and BestAP associate each client with the nearby AP that is providing the highest available bandwidth, independent of the current demand of clients. 
	Similarly, AGE and MCAC associate each client with the AP that is providing the minimum delay, without considering the tolerable delay of clients in the decision making process. 	
	As a result, demand-agnostic approaches may perform unnecessary associations, which incur handoff overheads that result in performance degradation.	
	
	\item \textit{Demand-aware.}
	Refers to the AsC mechanisms that explicitly include the demand of clients in their decision making process.
	Unfortunately, the number of demand-aware AsC mechanisms (e.g., ACHC, and DALB) is very limited.
\end{itemize}
%

 %these works take into account the throughput demand

Throughput demand, in particular, is composed of uplink and downlink traffic.
Unfortunately, most of the AsC mechanisms assume downlink traffic is dominant when calculating their association decision metrics, as Table \ref{DAMtable} shows. 
However, the emerging applications of SDWLANs justify the importance of uplink traffic.
For example, medical monitoring, industrial process control, surveillance cameras, and interaction with cloud storage, require timely and reliable uplink communication \cite{MARS,RTwifi,REWIMO,IOT-future1, IOT-future2, singh_survey_2014}.


These discussions reveal that designing AsC mechanisms to support both uplink and downlink requirements is an open problem. 
Additionally, due to the resource-constraint nature of IoT devices, it is important to define, compute and include metrics, such as channel contention intensity and packet loss rate, in the decision making process.

In addition to the mentioned challenges, it is also important to identify and address the effect of hardware on the metrics used.
For example, the RSSI perceived by a tiny IoT device would be different from the value perceived by a multi-antenna smartphone, under the same condition.
Therefore, the controller cannot generate a realistic network map based on the information collected from clients.
Similarly, for environments with highly asymmetric links (for example caused by shadowing), the information collected in the controller based on the RSSI perceived at the APs does not reflect the real connectivity and interference relationship.
Understating the effect of metric evaluation irregularity on performance and the calibration of metrics and inclusion in the AsC mechanisms are open research areas.





\subsubsection{\textbf{Dynamicity and Overhead}}
\label{asc_dyn_overhead}
Dynamic and scalable AsC requires minimizing overhead from the following point of views:
\begin{itemize}
\item  \textit{Measurement delay}: the delay of measuring decision metrics must be short to reflect network dynamics;
%
\item  \textit{Bandwidth requirements}: the amount of bandwidth consumed for exchanging control data should be minimized to enhance scalability. 
Each association command requires communication between the controller and APs.
In addition, in a VAP-based architecture such as Odin, each association requires a VAP transfer, where the number of transfers depends on the number of associations.
Therefore, both the frequency and the scale of reassociation should be taken into account for various types of architectures;
%
\item \textit{Deployment cost}: it is ideal to define decision metrics that do not require AP or client modification.
For example, although BestAP  claims low measurement overhead in terms of delay and bandwidth, it requires the modification of APs and clients. 
\end{itemize}

Unfortunately, the existing AsC mechanisms do not take into account the effects of these three factors, the scalability of the proposed mechanisms has not been studied thoroughly, and their effect on the energy consumption of smartphones and resource constraint devices is unknown.
%For example, although DenseAP and Odin report low measurement overhead, a small testbed is used for their evaluation.


%Therefore, investigations are required to establish a balance between the accuracy and overhead of measuring network status metrics.
Although some AsC mechanisms propose their own measurement techniques, continuous and low-overhead network monitoring should be provided as a SDWLAN architectural service to facilitate the development and interoperability of control mechanisms.
For example, as we will see in Section \ref{CMmech}, dynamic ChA mechanisms share some of the metrics used by AsC mechanisms.
%To this end, a shared repository (managed by the OS) could be inquired by network applications.
We propose the followings to provide network applications with low-overhead and continuous measurement of metrics:
(i) designing metrics that can be used by multiple control mechanisms; 
(ii) efficient encoding of control data conveyed by a south-bound protocol, 
(iii) novel hardware/software techniques deployed on network infrastructure devices for passive (i.e., without introducing extra traffic) measurement of network status, 
and (iv) inference and prediction algorithms that extract new metrics and provide insight into the future status of the network.
Compared to distributed implementations, SDWLANs enable in-depth analysis of clients' traffic type and pattern by looking into packet headers \cite{gibb2013design,akyildiz2014roadmap,ng2015developing,Yoon2017,oliveira_characterizing_2016}.
This information could be employed to improve the accuracy of traffic prediction and interference modeling.
Specifically, while the existing approaches rely on the saturated demands of clients when estimating load and interference levels, the throughput and airtime of clients and APs strongly depend on the protocols used at the transport and application layers \cite{sinky_analysis_2015,IOT-future1,hobfeld_challenges_2012}.
For example, the periodic nature of an IoT device's uplink traffic could be exploited to improve AsC performance.
We believe that learning and prediction mechanisms must be employed by SDWLANs to provide network control applications (such as AsC and ChA) with meaningful, realistic, and predictive knowledge regarding network operation.

%In particular, interference modeling is usually done based on physical and MAC layer features in a saturated condition (nodes always have a packet to send). 
%On the other hand, traffic load modeling of APs is performed based on the measurement of throughput or airtime during a specific period and prediction of future demands. 


% REVISION


\textit{Hybrid design} is another approach towards achieving timely reactions against dynamics.
The overhead of network monitoring and control can be improved through hybrid control mechanisms, as proposed by SoftRAN \cite{gudipati2013softran}.
Due to the high dynamics of wireless networks, it is desirable to design control mechanisms that can make control decisions locally, while improving decisions based on the global network view as well.
This solution could also benefit from hierarchical controller topologies to make decisions at multiple levels, depending on the dynamicity of variations.
For example, AeroFlux (Section \ref{Archs}) proposes an architecture to enable the design of multi-level control mechanisms.
Such mechanisms make control decisions in a distributive manner, at local controllers or by a central controller, based on network dynamics, time constraints, and overhead of exchanging control messages.



% {\color{blue!50!black}
% \subsubsection{\textbf{Effects of Virtualization}}
% %REVISION
% Mobility management becomes more challenging when network virtualization is employed.
% When a client requires a new point of association due to its mobility, in addition to parameters such as fair bandwidth allocation, the available resources of APs should also be taken into account. 
% More specifically, for a client belonging to slice $n$, the AsC mechanism should ensure that after the association of this client with a new AP, the the QoS provided by slice $n$ and other slices is not violated.
% However, as this may require client steering, the cost of re-associations should be minimized.

% % In fact, when a user switches between different slices of a network, the controllers should synchronize the operation of these slices to ensure seamless connectivity.
% % However, achieving a low-overhead and fast collaboration among these controllers is a challenging task.
% % In addition, as mobility results in variations of link quality and capacity, dynamic resource allocation mechanisms become necessary to ensure fair allocation of resources.
% }


\subsubsection{\textbf{Security Considerations}}
%Although such metrics enable network applications to implement centralized control mechanisms, they should not compromise network performance because of security breaches.
As centralized mechanisms rely on global network information and usually aim to achieve a network-wide optimization, receiving malicious reports has a more severe effect on performance, compared to distributed control.
Verifying the legitimacy of network status reports and clients' demands have not yet been addressed by SDWLAN architectures and AsC mechanisms.
%

%


%%%%%%%%%%%%%%%%%%%%%%%%%%%%%%%%%%%%%%%%%%%%%%%%%%%%%%%%
%%%%%%%%%%%%%%%%%%%%%%%%%%%%%%%%%%%%%%%%%%%%%%%%%%%%%%%%
%%%%%%%%%%%%%%%%%%%%%%%%%%%%%%%%%%%%%%%%%%%%%%%%%%%%%%%%
%%%%%%%%%%%%%%%%%%%%%%%%%%%%%%%%%%%%%%%%%%%%%%%%%%%%%%%%
%%%%%%%%%%%%%%%%%%%%%%%%%%%%%%%%%%%%%%%%%%%%%%%%%%%%%%%%
%%%%%%%%%%%%%%%%%%%%%%%%%%%%%%%%%%%%%%%%%%%%%%%%%%%%%%%%
%%%%%%%%%%%%%%%%%%%%%%%%%%%%%%%%%%%%%%%%%%%%%%%%%%%%%%%%
%%%%%%%%%%%%%%%%%%%%%%%%%%%%%%%%%%%%%%%%%%%%%%%%%%%%%%%%

% REVISION


\section{Centralized Channel Assignment (ChA)}
\label{CMmech}
In this section, we review centralized ChA mechanisms.
To be consistent with Section \ref{AMmech}, we focus on the metrics employed as well as the problem formulation and solution proposed by these mechanisms.
In addition, we highlight the performance improvements achieved, compared to distributed mechanisms.

Considering the broadcast nature of wireless communication as well as the dense deployment of APs, ChA is an important network control mechanism to reduce co-channel interference and improve capacity.
Basically, a WLAN can be modeled as a conflict graph where the vertices represent APs, and edges reflect the interference level between APs. 
The color of each vertex is the channel assigned to that AP. 
The objective of a ChA mechanism is to color the graph using a minimum number of colors in order to improve channel reuse.

A typical channel assignment algorithm relies on all or a subset of the following inputs: (i) the set of APs and clients, (ii) AP-client associations, (iii) interference relationship among APs and clients, and (iv) total number of available channels. 
Using the aforementioned information, optimization metrics, such as interference, throughput, spectral efficiency, and fairness, are used to define an optimization problem. 
In this section we study centralized ChA mechanisms.



Figure \ref{fig_LCCS} represents the flowchart of the \textbf{least congested channel search (LCCS}) \cite{LCCS} algorithm. 
%
\begin{figure}[!t]
	\centering
	\includegraphics[width=0.85\linewidth]{LCCS.pdf}
	\caption{Least Congested Channel Search (LCCS) algorithm \cite{LCCS}.}
	\label{fig_LCCS}
\end{figure}
%
Although LCCS is not a central ChA mechanism, we briefly discuss its operation as it has been used by commercial off-the-shelf APs and adopted as the baseline for performance comparison of various ChA mechanisms.
In this algorithm, APs scan all channels passively after startup. 
Each AP collects the beacon frames from its neighboring APs and extracts the number of connected clients and their traffic information. 
Then, the AP selects a non-overlapping channel if it is available. 
Otherwise, the AP selects the channel with lowest traffic sum.
To obtain client and traffic information, LCCS relies on the use of optional fields in beacon packets.
Therefore, the use of this mechanism requires vendor support.
%For example, Cisco APs support these features.

A simpler approach, which has been adopted as the comparison baseline by some ChA mechanisms, is to simply sample the RSSI of nearby APs on all channels and choose the channel with the minimum RSSI level.
The literature refers to this approach as \textbf{minRSSI}.

%\todo[inline,color=cyan]{$\checkmark$ how each ap monitors all packets of neighboring aps?}
%\todo[inline,color=cyan]{$\checkmark$ when an ap switches to another channel, then that cannot be involved in tx/rx}  
%\todo[inline,color=cyan]{$\checkmark$ what do you mean it does not consider traffic demand? -clarify} 


We overview ChA mechanisms, categorized into two groups: (i) \textit{traffic-agnostic}, and (ii) \textit{traffic-aware}, as follows.

%Traffic-agnostic techniques do not consider the traffic load of APs and the traffic demands of clients. 
%In contrast, traffic-aware techniques make channel assignment considering the traffic information of APs/clients. 


\subsection{{Traffic-Agnostic Channel Assignment}} 
\label{ChA_traffic_ag}
These mechanisms do not include the traffic load of APs and the traffic demand of clients in their channel assignment process.
In fact, these mechanisms mainly rely on the interference relationship to formulate their graph coloring problems.


\textbf{Weighted coloring channel assignment (WCCA).} 
\label{WCCA}
This seminal work \cite{Wcolor-2005} formulates channel assignment through the weighted graph coloring problem and improves channel re-use by assigning partially-overlapping channels to APs. 
%The proposed model is aware of client distribution by trying to assign less-overlapping channels to the APs with more associated clients. 
Vertices are APs, and edges represent the potential interference between corresponding APs (see Figure \ref{fig_coloring}). 
The colors of the vertices represent the assigned channel number to APs. 
The number of clients associated with two neighboring APs and the interference level between them are used to calculate the weight of conflicting edges. 
\textit{Site-Report} is the proposed method to construct the overlap graph and capture network dynamics. %and provide required information to accurately estimate $W(AP_{i},AP_{j})$ between all AP pairs. 
This method enables APs to request their clients perform channel scanning. 
A list of active APs is generated for each channel.
This list includes the APs in direct communication range and the APs whose associated clients are in direct communication with the client performing Site-Report scan. 
Based on the results of Site-Report, $W(AP_{i},AP_{j})$ is computed as follows,
\begin{equation}
W(AP_{i},AP_{j})= \frac{N_{AP_{i},AP_{j}} + N_{AP_{j},AP_{i}}}{N_{AP_{i}}+N_{AP_{j}}},
\end{equation}
where $N_{AP_{j}}$ is the number of site reports performed by the clients of $AP_{j}$, and $N_{AP_{i}, AP_{j}}$ is the number of site reports of $AP_i$'s clients that reported interference with $AP_j$ or its associated clients. 

The I-factor, $I(ch_i,ch_j)$, is the normalized interference factor between two channels $ch_i$ and $ch_j$.
%------
The value of $I(ch_i,ch_j)$ is computed as follows: 
(i) for two non-overlapping channels (e.g., channel 1 and 6 in the 2.4GHz band) the value is 0; 
(ii) if $ch_i=ch_j$, the value is 1; 
(iii) for two partially-overlapping channels $ch_i$ and $ch_j$, the value is between 0 and 1, depending on the distance between their central frequency.
%-------

The weight of the conflicting edge between $AP_{i}$ and $AP_{j}$ is defined by $Ivalue(AP_{i},AP_{j}) = W(AP_{i},AP_{j})\times I(ch_i,ch_j)$.  
Having $K$ colors (channels), the problem is determining how to color the graph with minimum number of colors in order to optimize the objective function. 
%
\begin{figure}[!t]
	\centering
	\includegraphics[width=0.6\linewidth]{Ivalue.pdf}
	\caption{Weighted coloring channel assignment (WCCA) \cite{Wcolor-2005} formulates a weighted graph coloring problem. This figure represents the I-value of the conflicting edge between $AP_{i}$ and $AP_{j}$.}
	\label{fig_coloring}
\end{figure}
%
%
Three different objective functions, $O_{max}$, $O_{sum}$ and $O_{num}$, were used. 
$O_{max}$ represents the maximum I-value amongst all the links.
$O_{sum}$ is the sum of I-values of all conflicting edges. 
The total number of conflicting edges is represented by $O_{num}$. 
In order to minimize the objective functions, two distributed heuristic algorithms are proposed: (i) \textit{Hminmax} aims to minimize $O_{max}$ without AP coordination, and (ii) \textit{Hsum} aims to jointly minimize $O_{max}$ and $O_{sum}$, which requires coordination between APs in order to minimize the interference level of all conflicting edges. %The authors rename \textbf{Hsum} and \textbf{Hminmax} algorithms by \textbf{ADJ-sum} and \textbf{ADJ-minmax} if partially overlapping channels 
Although the proposed algorithms were run distributively, it would be easier to provide them with their required information using a SDWLAN architecture.

The proposed algorithms have been evaluated using NS2 \cite{NS2} simulator and testbed. %on a network consisting 20 APs. 
Two different scenarios are used in simulations: (i) three non-overlapping channels and (ii) partially overlapping channels (11 channels in the 2.4GHz band). 
For the first scenario, the algorithms showed a 45.5$\%$ and 56$\%$ improvement in interference reduction for sparse and dense networks, respectively. 
For the second scenario, there is a 40$\%$ reduction in interference, compared to LCCS. 
%\todo[inline, color=cyan]{what is the network topology?-sparse or dense \\$\checkmark$ \textit{20 APs}} 

%%%%%%%

\label{CDCA}
\textbf{Client-driven channel assignment (CDCA).}
\cite{J-DCA-LB-2006} formulates a \textit{conflict set coloring} problem and proposes a heuristic algorithm that increases the number of conflict-free clients at each iteration. 
Here, \textit{conflict} refers to the condition in which two nodes (APs or clients) that belong to different BSSs use the same channel.
To measure the interference level experienced by each client, measurement points are selected, and the signal level received from each AP is measured at those points. 
Based on this information, there are two sets assigned to each client: (i) \textit{range set}: the APs that can communicate with the client directly, and (ii) \textit{interference set}: the APs that cannot communicate with the client, but can cause interference. 
Figure \ref{fig-conflict-set-coloring} shows the range set and interference set for a given client.
%
% REVISION - CAPTION REDUCED
\begin{figure}[!t]
	\centering
	\includegraphics[width=0.65\linewidth]{fig-conflict-set-coloring.pdf}
	\caption{Client-driven channel assignment (CDCA) \cite{J-DCA-LB-2006} assigns two sets per client: \textit{range set} (denoted by $\mathcal{RS}(c_{i})$) and \textit{interference set} (denoted by $\mathcal{IS}(c_{i})$). APs are specified by numbered squares. %Note that the interference set of the client contains all APs that can cause interference on the client directly or via their associated clients.
	}
	\label{fig-conflict-set-coloring}
\end{figure}

This work defines two objective functions: (i) one that maximizes the number of conflict-free clients, and (ii) one that minimizes the total conflict in the network.
To achieve the first objective, if a channel $ch$ is assigned to an AP, then no AP in the range set and interference set of the clients associated with that AP should have channel $ch$ assigned to them.
A client satisfying this condition is referred to as a "conflict-free client".
%Note that due to association, there is only one AP that belongs to the range set of the client and uses channel $ch$. 
Since it may not be possible to satisfy this requirement for all the clients, the algorithm minimizes the overall network conflict.
%\todo[inline,color=cyan]{$\checkmark$ make sure the word "try" is correct here} 

%The authors propose the \textit{conflict-free assignment with randomized compaction} (CFAssign-RaC) algorithm. 
The main step of the channel assignment algorithm randomly selects an AP and assigns a channel that results in the maximum number of conflict-free clients. 
This step is repeated for all APs, and the order of APs is determined by a random permutation. 
%After channel assignment to APs, the number of conflict-free clients is re-calculated. 
This process is repeated as long as it increases the number of conflict-free clients. 
%Since CFAssign-RaC operates based on a randomized order of APs, it is run for several times (with different random permutations) in order to find the best solution among the outputs of different runs.

%After applying CFAssign-RaC, there might be clients left with high level of interference.
After the first step, there might be clients that are left with a high level of interference.
To remedy this condition, the conflict level of all non-conflict-free clients is balanced.
This mechanism implicitly results in traffic load balancing.
%To handle mobility, CFAssign-RaC is either run periodically or it is triggered when the objective function is higher than a threshold.
%The superiority of the proposed algorithms has been demonstrated through simulation and testbed, in terms of the number of MAC collisions, throughput and per-packet delay, compared to LCCS \cite{LCCS}.

%TABLE
%  (1) approach: CA and Load balancing
%  (2) interference model: RSSI-based
%%% (3) problem formulation: Conflict set coloring 
% (4) objective function: number of conflict-free client, Total conflict of all WLAN clients
% (5) channels: non-overlapping (802.11b)
% (6) Client/AP-centric: Client-centric
% (7) Client-aware: Yes.
%%% (8) Perfomance evaluation: simulation, real implementation
%%% (9) Dynamic/Static: Dynamic
%%% (10) Optimality: Sub-optimal
% (11) Measurement: 40 usage-points, compute range and interference sets, based on test clients
%    --> Time and computational complexity is not discussed in paper
%%% (12) Requried collecting information: scan all channels => range set and interference set


\label{FPLN}
\textbf{Frequency planning in large-scale networks (FPLN).}
\cite{CAPWAP-based-CA-11} models the channel assignment problem as a weighted graph coloring problem taking into account the external interference generated by non-controllable APs. 
The interference level caused by $AP_{j}$ on $AP_{i}$, which is the weight of edge $AP_{j} \rightarrow AP_{i}$, is defined as follows,
%
\begin{equation}
W(AP_{i},AP_{j})= \mathcal{C}^{active}_{AP_{i}}  \times I(ch_i,ch_j) \times P(AP_{i},AP_{j}),
\end{equation}
where $\mathcal{C}^{active}_{AP_{i}} $ is the number of active associated clients with $AP_{i}$. 
An associated client is called \textit{active} in a time interval if it is sending/receiving data packets.
%
The interference factor between two channels assigned to $AP_{i}$ and $AP_{j}$ is given by $I(ch_i,ch_j)$. 
The power received at $AP_{j}$ from $AP_{i}$ is represented by $P(AP_{i},AP_{j})$. 
%This model neglects transmission power asymmetry (i.e., it assumes $P(i,j)=P(j,i)$). 
The interference factor of $AP_{i}$ is defined as $\sum_{\forall AP_{j} \in \mathcal{AP}}W(AP_{i},AP_{j})$.
The objective function is defined as the sum of all APs' interference factor, including controllable and non-controllable APs. 
The ChA mechanism aims to minimize the objective function when the channels of non-controllable APs are fixed.

The authors propose a two-phase heuristic algorithm to solve the above optimization problem. 
In the first phase, the controllable APs are categorized into separate clusters based on neighborhood relationships, and the local optimization problem is solved for each cluster by finding an appropriate channel per AP to minimize its interference level while the channels of other APs (in the cluster) are fixed. 
This process is started from the AP with largest interference factor. 
In the second phase, a pruning-based exhaustive search is run on each cluster, starting from the AP with highest interference level. 
The algorithm updates the channels of APs to reduce the total interference level of the cluster. 
The pruning strategy deletes the failed channel assignments (i.e., the solutions that increase the interference level obtained in the first phase) from the search space.
FPLN uses CAPWAP (see Section \ref{CUWNarch}) as its south-bound protocol. 
Simulation and testbed experiments show about a 1.2x lower interference level, compared to LCCS.


\textbf{CloudMAC.}
\label{CloudMAC_CM}
The CloudMAC \cite{CloudMAC,CloudMAC3} architecture allows multiple NICs of a physical AP to be mapped to a VAP, where NICs may operate on different channels. 
These NICs  periodically monitor and report channel utilization to the controller. 
Unfortunately, it is not clear how channel measurement is performed.
% controller is OpenFlow \cite{OpenFlow} controller
If a client is operating on a high-interference channel, the controller sends a 802.11h channel switch announcement message to the client (mandatory for 802.11a/n standards), instructing it to switch to another channel with lower interference. 
Since a client is associated with a VAP, the client does not need to re-associate and it can continue its communication through the same VAP using another NIC of the same physical AP.
Additionally, this process does not require any client modification.
Unfortunately, the performance evaluation of this mechanism has not been reported.



%%%%%%%%%%
\label{Odin-CM}
\textbf{Odin.}
\cite{Odin2} proposes a simple channel assignment strategy. 
For each AP, the channel assignment application (running on Odin controller) samples the RSSI value of all channels during various operational hours.
The heuristic algorithm chooses the channel with the smallest maximum and average RSSI for each AP.
Unfortunately, the performance of this mechanism has not been evaluated.


\textbf{Primary channel allocation in 802.11ac networks (PCA).}
%1
\label{PCA}
The \textit{hidden channel} (HC) problem is addressed in \cite{802.11ac-PCA} for 802.11ac networks. 
The HC problem occurs when an AP interferes with the bandwidth of another AP.
This problem occurs due to the heterogeneity of channel bandwidths, different CCA thresholds for primary and secondary channels, and bandwidth-ignorant fixed transmission power. 

Figure \ref{invading} illustrates an example of the HC problem. 
The bandwidth of $AP_{1}$ is 80MHz, including four 20-MHz channels 1, 2, 3 and 4, where channel 1 is its primary channel. 
The primary channel sensing range is greater than the secondary channel sensing range due to the difference in CCA thresholds for primary and secondary channels in 802.11ac. 
$AP_{2}$ uses a 20MHz bandwidth operating on channel 3. 
Suppose $AP_{2}$ is communicating with its clients on channel 3. 
When $AP_{1}$ performs carrier sensing on its channels, it cannot detect the presence of $AP_{2}$ on channel 3 because $AP_{2}$ is not in the secondary sensing range of $AP_{1}$. 
Therefore, $AP_{1}$ starts to send data on channels 1, 2, 3 and 4, and invades $AP_{2}$. 
%In this scenario, $AP_{1}$ invades the bandwidth of $AP_{2}$.
%
\begin{figure}[!t]
	\centering
	\includegraphics[width=0.7\linewidth]{ac-1.pdf}
	\caption{An example of the hidden-channel (HC) problem in 802.11ac \cite{802.11ac-PCA}.}
	\label{invading}
\end{figure}
%
%
%2

In order to study the impact of the HC problem, the authors proposed a Markov chain-based analysis to show the impact of packet length and MAC contention parameters on the performance of APs. 
Furthermore, the effect of various HC scenarios on packet error rate (PER) has been investigated. 
%4
The authors used the graph coloring problem, where APs are vertices and channels are colors, to model the \textit{primary channel allocation} (PCA) problem. 
The edges represent the invasion relationship between APs. 
It is assumed that a controller collects information about the bandwidth of APs and their interference relationship.
The objective is to color the vertices in order to: (i) minimize the interference/invasion relationship between APs, and (ii) maximize channel utilization. 
The problem is formulated as an integer programming optimization problem, which is NP-hard.
%5
A heuristic primary channel allocation algorithm is introduced to solve the problem.
%6
Simulation results show the higher performance of PCA compared to minRSSI and random channel assignment.
Furthermore, PCA is a close-to-optimal solution when compared to the optimal exhaustive search algorithm.

\textbf{EmPOWER2.}
\label{EMPOWER2ChA}
As mentioned in Section \ref{EMPOWERarch}, EmPOWER2 \cite{Primitives} establishes \textit{channel quality and interference map} abstraction at the controller.
The proposed ChA mechanism uses the APIs provided to traverse the map and detect uplink/downlink conflicts between LVAP pairs.
There is an edge between two nodes in the interference map if they are in the communication range of each other.
In addition, the weight of each link corresponds to the channel quality between nodes.
The graph coloring algorithm proposed in \cite{san2012new} is used to assign channels based on the conflict relationships.



\label{Wi5CM}
\textbf{Wi-5 channel assignment (Wi5CA).} 
\cite{DCA-2} proposes a ChA mechanism as part of the Wi-5 project \cite{Wi-5}.
A binary integer linear programming problem is formulated with objective function 
$\textbf{U} = \textbf{G}\times \textbf{A}^{T} . \textbf{I}$, 
where $'\times'$ and $'.'$ are matrix multiplication and element-wise multiplication operators, respectively. $\textbf{G}\in\{0,1\}^{|\mathcal{AP}| \times |\mathcal{AP}|}$ represents the network topology. 
If the average power strength of $AP_i$ on $AP_j$ exceeds a specific threshold, the value of $G_{AP_{i}, AP_{j}}$ will be 1, otherwise it will be 0. $\textbf{A}\in\{0,1\}^{|\mathcal{CH}|\times |\mathcal{AP}|}$ is the current channel assignment for APs.
If channel $ch_i$ is assigned to $AP_j$, the value of $A_{{ch_{i},AP_{j}}}$ is 1, otherwise it is 0. 
$\textbf{I}\in \mathbb{R} ^{|\mathcal{AP}| \times |\mathcal{CH}|}$ represents the predicted interference matrix, where $I_{AP_{i}, ch_{i}}$ is the interference level predicted for $AP_i$ if channel $ch_j$ is assigned to it.
Unfortunately, the interference prediction method is unclear.  
The channel assignment algorithm minimizes 
\begin{equation}
\sum_{\forall AP_{i} \in \mathcal{AP}}\;\sum_{\forall ch_{j} \in \mathcal{CH}} U_{AP_{i}, ch_{j}}, 
\end{equation}
which is the network-wide interference level.
%The authors did not report details about the calculation/measurement of predicted interference matrix $\textbf{I}$. 
%\todo[inline,color=cyan]{$\checkmark$ what do you mean by the "predicted" level of interference?} 
A controller executes the ChA mechanism when the interference level is higher than a threshold.
%However, the controller updates its global knowledge about all APs in the network using OpenFlow. 
%The proposed algorithm is evaluated through MATLAB in terms of three performance metrics: interference level, SINR and spectral efficiency. 
MATLAB simulations show 2dB and 3dB reduction in the average interference level, compared to LCCS and uncoordinated channel assignment, respectively.


%%%%


%  (1) approach: CA
%  (2) interference model: RSSI-based
%%% (3) problem formulation: weighted graph vertex coloring 
% (4) objective function: Total interference of WLAN: Sum of interference level between AP-AP pairs
% (5) channels: (not mentioned: the modeling is general and both partially overlappig and non-overlap channels can be applied.)
% (6) Client/AP-centric: AP-centric
% (7) Client-aware: Yes, the count of active clients associated to APs
%%% (8) Perfomance evaluation: simulation, real implementation
%%% (9) Dynamic/Static: not mentioned (I think, it is static,  But, it can be run in real-time.) --> because it is assumed that the relationship between APs is known.
%%% (10) Optimality: Sub-optimal
% (11) Measurement: Statistical --> non-realtime (column 9)
%%% (12) Requried collecting information: RSSI of APs 



\subsection{Traffic-Aware Channel Assignment} 
\label{ChA_traffic_aware}
Traffic-aware channel assignment mechanisms include the traffic of APs, clients or both in their decision process.
As measuring the effect of interference on performance is complicated, these mechanisms explicitly include metrics such as throughput, delay, and fairness, in their problem formulations.


\label{APCA}
\textbf{AP placement and channel assignment (APCA).} 
\cite{J-AP-DCA-2006} addresses AP placement and channel assignment in order to improve the total network throughput and the fairness established among clients. 
The throughput of clients is estimated as a function of MAC-layer timing parameters, network topology, and data rate of clients. 
The data rate of each client (11, 5.5, 2 or 1Mbps) depends on the link's RSSI. 
The fairness among clients is defined as the throughput deviations of clients based on the Jain's fairness index \cite{Jain's-fairness} as follows,
%
\begin{equation}
\label{eq-fairness}
J  = \frac{(\sum_{\forall c_{i} \in \mathcal{C}}Th_{c_{i}})^2}{|\mathcal{C}|\sum_{\forall c_{i} \in \mathcal{C}}{(Th_{c_{i}}})^2},
\end{equation}
where $Th_{c_{i}}$ is the throughput of client $c_i$.
Maximizing the objective function, defined as $J\times \sum_{\forall c_{i} \in \mathcal{C}} Th_{c_{i}}$, leads to optimizing both total throughput and fairness among clients. 
%Since exhaustive search is not feasible to find the optimal solution of AP placement and channel assignment, 
The authors propose a heuristic local search method, named the \textit{patching} algorithm. 
In each iteration of this algorithm, an AP is placed on one of the predefined locations, and a non-overlapping channel is assigned to the AP to maximize the objective function.
This process is repeated until a predefined number of APs are placed. 
Simulation and testbed results show that the proposed patching algorithm provides close-to-optimal solutions in terms of total throughput and fairness. 
APCA is used only for the initial network configuration phase.

%\todo[inline,color=cyan]{$\checkmark$ you mostly discuss about the goals- there is not enough discussion about the approach} 
%\todo[inline,color=cyan]{$\checkmark$ it is not clear how ap placement works} 
%  (1) approach: CA, Fairness, AP placement
%  (2) interference model: RSSI-based ==> Client throughput
%%% (3) problem formulation: Throughput per client
% (4) objective function: Throughput and fairness
% (5) channels: non-overlapping (802.11b)
% (6) Client/AP-centric: Client-centric
% (7) Client-aware: Yes. Client throughput and fairness among clients' throughput
%%% (8) Perfomance evaluation: simulation
%%% (9) Dynamic/Static: Static
%%% (10) Optimality: Sub-optimal
% (11) Measurement: passive 
%    --> Time and computational complexity is not discussed in paper
%%% (12) Requried collecting information: RSSI of clients
%\subsubsection{\textbf{Client-agnostic and traffic demand-aware techniques}} In this section, we survey the techniques which consider only the traffic demand of APs to make the channel assignment decisions.


\label{TACA}
\textbf{Traffic-aware channel assignment (TACA).}
% solution 
\cite{Traffic-aware-CA-2007} assigns weights to APs and clients proportional to their traffic demands. 
The \textit{weighted channel separation} (WCS) metric is
\begin{equation}
\label{interference-metric}
WCS = \sum_{\substack{i,j  \in \mathcal{AP} \cup \mathcal{C}, \\ BSS(i)\neq BSS(j) }} W(i,j)\times Separation(i,j),
\end{equation}
where $BSS(k)$ is the BSS including AP or client $k$, $Separation(i,j)$ is the distance between operating channels of node $i$ and node $j$, and $W(i,j)$ is a function of the traffic demands of $i$ and $j$.
$Separation(i,j)=\min{(|ch_i-ch_j|,5)}$, therefore, the maximum separation value is 5, which is the distance between orthogonal channels (e.g., channel 1 and 6).
%The traffic-aware weight between two nodes $i$ and $j$ is defined by $W(i,j)=\beta^{snd}_{i} \beta^{snd}_{j} + \beta^{snd}_{i}\beta^{rcv}_{j} + \beta^{snd}_{j}\beta^{rcv}_{i}$ where $\beta^{snd}_{i}$ and $\beta^{rcv}_{i}$ are sending and receiving demands of node $i$, respectively. 
%In particular, the following parameters are available per $AP_{i}$ every $\Delta t$ second: the number of packets/bytes sent ($Out_{AP_{i}}(t)$), the number of packets/bytes received ($In_{AP_{i}}(t)$), and the number of currently associated clients ($\mathcal{C}_{AP_{i}}$). 
%The sending demand of $AP_{i}$ at a given time $t$ is calculated as $\beta^{snd}_{AP_{i}}(t) = \frac{Out(t)-Out(t-\Delta t)}{\Delta t}$. 
%The sending demand for each client is approximated as $S_{c}(t) = \frac{In(t)-In(t-\Delta t)}{N\Delta t}$. 
Since all nodes are included in the calculation of the WCS metric, all AP-AP, AP-client and client-client interferences are taken into account. 

%
%\todo[inline, color=cyan]{$\checkmark$ how are the sending and receiving demands modeled?} 
%
%
%

A heuristic algorithm has been proposed to maximize WCS. 
Figure \ref{fig_traffic-aware} shows the flowchart of this algorithm.
%
\begin{figure}[!t]
	\centering
	\includegraphics[width=0.8\linewidth]{traffic-aware-block-diagram.pdf}
	\caption{Traffic-aware channel assignment (TACA) \cite{Traffic-aware-CA-2007}.}
	\label{fig_traffic-aware}
\end{figure}
%
The interference graph measurement is performed a few times per day. % when the traffic load is light. 
The traffic demands of APs and clients are collected and predicted using the SNMP statistics collected from APs.
In "Step 1", an initial channel assignment is performed to maximize the WCS metric considering APs' traffic demands only. 
In "Step 2", a simulated annealing (SA) approach is used to update channel assignment, which maximizes WCS by taking into account the traffic demands of AP and clients. 
This step randomly selects an AP and its clients, and assigns a new channel at each iteration in order to maximize the WCS metric. 
%The algorithm is repeated 1000 times to obtain a close-to-optimal channel assignment. 
%\todo[inline,color=cyan]{$\checkmark$ what is the interference metric, and what do you mean by maximizing that?} 
%Performance evaluation is performed using NS2 simulator and a 25-node testbed to measure total network throughput. %for both TCP and UDP flows. 
%Three types of scenarios are considered: (i) client-aware/traffic-agnostic, (ii) client-agnostic/traffic-aware, and (iii) client-aware/traffic-aware. 
%The first scenario ignores traffic demands. The second scenario takes into account the traffic demands of APs, and the third scenario includes the traffic demands of APs and clients. 
Empirical results (using a 25-node testbed) show 2.6 times increase in TCP throughput, compared to a downgraded algorithm in which traffic demands are not included.
%The higher throughput of this traffic-aware channel assignment is presented, especially when the traffic demands are highly variable. 
The authors also showed that one channel switching per 5 minutes does not result in throughput degradation. 

\label{VDTCA}
\textbf{Virtual delayed time channel assignment (VDTCA).}
\cite{Measurement-CA-WCNC-10} introduces a new interference prediction model based on the signal strength and traffic demands of APs and clients. 
The extra transmission delay caused by interference between two APs is represented as $VDT = T_{int} - T_{n}$, where $VDT$ is \textit{virtual delayed time}. 
The term "virtual" refers to the calculation of this metric through interference prediction without actually changing channels.  
$T_{int}$ and $T_{n}$ are the time required to transmit a given amount of traffic in the presence and absence of interference, respectively.

%\todo[inline,color=cyan]{$\checkmark$ what do you mean by virtual? \\ \textit{- there is no directly justification in the paper on why they name it virtual !} \\ \textit{By the way, I added a sentence to justify it.} }  
The transmission time is a function of physical layer data rate, which depends on the signal SINR computed through measuring RSSI at receiver side.
%\todo[inline,color=cyan]{$\checkmark$ I modified the above sentence, is it correct?-- these sen} 
On the other hand, the traffic demands of APs and clients have a great impact on the transmission time of neighboring APs. 
For instance, when there is no traffic on an AP, there will be no transmission delay overhead on the neighboring APs and clients. 
Therefore, $T_{int}$ is calculated based on RSSI values and traffic rates of all neighboring APs and their associated clients.
%\todo[inline,color=cyan]{$\checkmark$ traffic rate of whom?}  
Using the proposed VDT model, it is possible to measure the VDT of each AP pair.
Channel assignment is modeled as a graph vertex coloring problem, which uses $VDT(BSS_i, BSS_j)$ as the edge weight between two interfering BSSs, where $BSS_i$ and $BSS_j$ are two APs and their associated clients.
Non-overlapping channels are the colors of vertices. 
The objective is to color the graph with a minimum number of colors while minimizing the sum of edge weights to achieve interference minimization. 
The semi-definite programming (SDP) \cite{SDP} relaxation technique is used to solve this problem.
The authors show that SDP  can solve the channel assignment problem in the order of seconds for a medium-size network. 
For instance, it takes 27.5 seconds to find the solution for a network with 50 APs and 9 non-overlapping channels. 
Testbed results show $30\%$ throughput improvement, compared to LCCS \cite{LCCS}. 

\textbf{Cisco unified wireless network (CUWN).}
\label{CUWN_CM}
In the CUWN architecture \cite{Cisco} (see Section \ref{CUWNarch}), based on the different RF groups established and the cost metric calculated for each AP, each controller updates the assigned channels periodically\footnote{The period can be adjusted by network administrators.}. 
The controller calculates a cost metric for each AP to represent the interference level of the AP, and prepares a list called \textit{channel plan change initiator} (CPCI), which includes the sorted cost metric values of all APs. 
The leader selects the AP with highest cost metric and assigns a channel to the selected AP and its one-hop neighboring APs in order to decrease their cost metrics. 
These APs are removed from the CPCI list, and this process is repeated for all remaining APs in the list.
A sub-optimal channel assignment is calculated through a heuristic algorithm\footnote{The details of ChA mechanism used by CUWN are not available.} that aims to maximize the frequency distance of selected channels. 
Note that a longer frequency distance results in a lower interference between APs. 
CUWN also enables network administrator to set channels manually.

\label{CAFA}
\textbf{Channel assignment with fairness approach (CAFA).}
\cite{CA-F-WCNC-11} formulates ChA as a weighted graph coloring problem where weights are assigned to the set of edges and vertices. 
The distance of overlapping channels is determined by the normalized interference factor, i.e., I-factor, introduced in \cite{Wcolor-2005} (see WCCA explained in Section \ref{WCCA}). 
The normalized throughput of each client is estimated using the approximation introduced in \cite{Thr-estimation-2005}. 
The weight of $AP_i$ is defined as the total \textit{normalized throughput reduction} of $AP_i$, which is calculated based on clients' normalized throughput and I-factor. 
The weight of the edge between two interfering APs represents the throughput reduction caused by their mutual interference.
Furthermore, the authors formulate the fairness among normalized throughputs of APs using Jain's fairness index \cite{Jain's-fairness}. 
The objective function is defined as the joint minimization of clients' throughput reduction and maximization of client's fairness index. 

%\todo[inline,color=cyan]{$\checkmark$ anything special about the heuristic?}  
An iterative heuristic algorithm is introduced to solve the optimization problem. 
The algorithm sorts the APs based on their weights (i.e., normalized throughput reduction).  
At each iteration, the AP with highest weight is selected for channel assignment. 
The channel of the selected AP is assigned so that the throughput reduction of the neighboring APs is minimized.
This process is repeated until channel assignment is performed for all APs. 
Simulation results show $15\%$ and $6\%$ improvement in total throughput for 4-AP and 8-AP scenarios, respectively, compared to WCCA \cite{Wcolor-2005}. 

%  (1) approach: CA and fairness
%  (2) interference model: Throughput, overlapping channel interference 
%%% (3) problem formulation: weighted graph vertex coloring 
% (4) objective function: normalized throughput reduction and fairness
% (5) channels: overlapping (802.11b)
% (6) Client/AP-centric: AP-centric
% (7) Client-aware: Yes, client throughput 
%%% (8) Perfomance evaluation: simulation
%%% (9) Dynamic/Static: Dynamic (no information about time complexity and information gathering complexity)
%%% (10) Optimality: Sub-optimal
% (11) Measurement: no information
%%% (12) Requried collecting information: clients throughput, [not mentioned in detail]


\label{TFACM}
\textbf{Throughput and fairness-aware channel assignment (TFACA).}
\cite{CA-VTC-14} uses spectrum monitoring information and the traffic demand of APs.
This mechanism defines the utility function of $AP_i$ when operating on channel $ch_{j}$ as,
%
\begin{equation}
\label{eqTFACM}
U(AP_{i}, ch_{j})=\frac{\min\left\{F(AP_{i},ch_{j}), T(AP_{i}) \right\}}{T(AP_{i})} ,
\end{equation}
where $F(AP_{i},ch_{j})$ is the free airtime of channel $ch_{j}$ on $AP_{i}$, and $T(AP_{i})$ is the total time required to send data of $AP_i$ in a time unit. 
Note that Equation \ref {eqTFACM} represents the percentage of $AP_i$'s data that can be sent over channel $ch_{j}$ in a given time unit.
Two objective functions are defined:
\begin{equation}
 \max\sum_{\forall AP_{i}\in\mathcal{AP}, \forall ch_{j} \in \mathcal{CH}}{U(AP_{i}, ch_{j})},
\end{equation}
%
and,
\begin{equation}
\max(\min_{\forall AP_{i}\in\mathcal{AP}, \forall ch_{j} \in \mathcal{CH}}{U(AP_{i}, ch_{j})}).
\end{equation}
A heuristic algorithm has been proposed to solve the above NP-hard problem in the following two steps:
(i) temporary channel assignment, which assigns a channel to each AP independent of other APs, to maximize each AP's utility function, and (ii) channel reassignment of the APs with low utility function (starting with the lowest one) without decreasing the sum of all utility function values. 
The second step is repeated for a given number of iterations or until $U(AP_{i}, ch_{j})=1$ for all APs. 
A fairness index similar to Equation \ref{eq-fairness} has been defined to include the utility function of APs. 

The 802.11g protocol with non-overlapping channels are used for performance evaluation through simulation and testbed. 
An $8$ to $15\%$ improvement in the fairness index and an $8$ to $21\%$ increase in the mean utility function are achieved, compared to minRSSI. 
%Furthermore, the authors report the performance of proposed technique in the presence of non-controllable APs (which are working using the minRSSI technique). 

%It is worth noting that the channel and data traffic information required to calculate utility function in the central controller are not collected in a real-time manner.
%Rather, this information is collected passively and based on statistical values. 


%  (1) approach: CA and fairness
%  (2) interference model: available throughput, percentage of operating on the assigned channel
%%% (3) problem formulation: calculus-based 
% (4) objective function: total throughput 
% (5) channels: non-overlapping (802.11g)
% (6) Client/AP-centric: AP-centric
% (7) Client-aware: Yes, traffic demand and the number of clients associated to each AP
%%% (8) Perfomance evaluation: simulation, real implementation
%%% (9) Dynamic/Static: Static
%%% (10) Optimality: Sub-optimal
% (11) Measurement: Statistical --> non-realtime (column 9)
%%% (12) Requried collecting information: traffic demand in AP, the acheivable throughput in a time unit (on a given channel)  


\label{FBWA}
\textbf{Frequency and bandwidth assignment for 802.11n/ac (FBWA)}.
%The authors of \cite{CA-BW-VTC-14} propose a channel and bandwidth allocation technique for 802.11a/n/ac-based WLANs (the proposed approach is similar to \cite{CA-VTC-14}). 
\cite{CA-BW-VTC-14} addresses channel and bandwidth allocation to benefit from the channel bonding feature of 802.11n/ac. 
The utility function of $AP_i$ that works on primary channel $ch_{j}$ and bandwidth $\beta$ is defined as
%
\begin{equation}
U(AP_{i}, ch_{j}, \beta)=\frac{Th(AP_{i}, ch_{j}, \beta)} {\min \left\{ Th^{\mathrm{max}}(AP_{i}, ch_{j}, \beta),\; L_{AP_{i}}^{\mathrm{mean}}   \right\} } 
\end{equation}
where $Th(AP_{i}, ch_{j}, \beta)$ is the expected throughput of $AP_i$ operating on primary channel $ch_{j}$ and channel bandwidth $\beta$,
 $Th^{\mathrm{max}}(AP_{i}, ch_{j}, \beta)$ is the maximum achievable throughput of $AP_i$, and $L_{AP_{i}}^{\mathrm{mean}}$ is the mean traffic load on $AP_i$. 
Through a heuristic algorithm similar to TFACA \cite{CA-VTC-14}, the appropriate primary channels and bandwidths are determined for all APs to maximize the sum of the APs' utility functions.
%\todo[inline,color=cyan]{$\checkmark$ and a jain based ... ??} 

Testbed results show a $65$ to $89\%$ reduction in the number of overlapping BSSs, and a mean throughput that is 3 times higher, compared to random channel assignment and minRSSI. 
%Similar to \cite{CA-VTC-14}, the process of collecting information from APs is not performed in a real-time manner. 

%
\label{ATCM}
\textbf{Normalized-airtime channel assignment (NATCA).}
%In \cite{DCA-residential-2015}, a WiFi Union (WU) framework is proposed to manage AP channel assignment.
\cite{DCA-residential-2015} requires each AP to measure its busy time (referred to as \textit{normalized airtime}) and report to a controller periodically. 
The busy time of an AP is the percentage of time that a channel assigned to that AP is occupied by its neighbors. 
An AP measures this value when sending data towards its clients. 
%The downlink traffic from an AP to its clients is used as the dominant traffic (compared to uplink) to measure APs' busy times. 
A threshold value is defined to categorize APs into two groups based on their busy time: \textit{heavily congested} and \textit{lightly congested}. 
An optimization problem has been proposed to minimize the sum of the normalized airtime of all APs without modifying the status of lightly congested APs. 
A Tabu search algorithm is proposed to solve the optimization problem. 
NS3 simulations show a 1.5x throughput improvement, compared to LCCS \cite{LCCS}. 
The measurement of normalized airtime for all APs is performed passively using the NS3 tracing system, which provides packet-level trace files for all APs.  

%\subsubsection{\textbf{Client-aware and traffic demand-agnostic techniques}} In this section, we survey the algorithms which consider the interference experienced by clients, but do not consider the traffic demands of clients.
%%%%%


%\subsubsection{\textbf{Client-aware and traffic demand-aware techniques}} In this section, we overview the techniques which use the traffic demand of clients and APs in order to measure the interference level of network and channel assignment decisions.


% new algorithms after 2010 (not mentioned in 2010 survey)

%\begin{table*}
%	\centering
%	\caption{Interference Mitigation Algorithms and Protocols}
%	\label{Interference-Mitigation-Algorithms} 
%	\def\arraystretch{1.5}
%	\begin{tabular}{ |m{0.5cm}|m{3.7cm}|m{2.7cm}|m{3.5cm}|m{2cm}|m{1.3cm}|m{1.3cm}|m{1.5cm}|}
%		\Xhline{3\arrayrulewidth}
%		\textbf{Ref.} & \textbf{Approach} & \textbf{Interference Metrics} & \textbf{Objective Function}& \textbf{Channels\newline  Standard}&\textbf{Client\newline AP-Driven } & \textbf{Dynamic \newline Static} & \textbf{Performance \newline Evaluation}\\ \Xhline{3\arrayrulewidth}
%		
%		\cite{Wcolor-2005} &Channel Assignment&-Channel interference\newline -Clients number& I-value&Overlapping\newline 802.11b&Client&Dynamic&Empirical\\\hline
%		\cite{J-DCA-LB-2006}&-Channel Assignment\newline -Load Balancing&RSSI-based&-Number of conflict-free client \newline -Total conflict of all WLAN clients&Non-overlapping\newline 802.11b&Client&Dynamic&-Empirical\newline -Simulation\\\hline
%		\cite{J-AP-DCA-2006}&-Channel Assignment\newline -AP Placement\newline -Fairness Handling&Clients throughput&Throughput and fairness&Non-overlapping\newline 802.11b&Client&Static&Simulation\\\hline
%		
%		
%		\cite{Traffic-aware-CA-2007}&Channel Assignment&-APs throughput \newline -Clients throughput \newline -Number of clients per AP&-Channel separation \newline -Traffic-aware total interference among all nodes (APs and clients)&Overlapping\newline 802.11b/g&AP\newline Client&Dynamic&-Empirical\newline-Simluation\\\hline
%		\cite{802.11ac-PCA}&Primary Channel Assignment&Channel invading relations among APs&-Total invading relations between APs \newline -Channel utilization&Non-overlapping\newline 802.11ac&AP&Dynamic&Simulation\\\hline
%		\cite{DCA-2016-1}&-Channel Assignment\newline -AP Power Control\newline -AP-client Association Control&Non-overlapping channel distance (binary)&Total power consumption of APs&Non-overlapping\newline 802.11b&AP&Dynamic&-Empirical\newline -Simulation\\\hline
%		\cite{DCA-2}&Channel Assignment&Priori knowledge about interference is assumed&Total interference levels of APs&Non-overlapping&AP&Dynamic&Simulation\\\hline		
%		\cite{DCA-residential-2015}&Channel Assignment&Normalized airtime of interfering APs&Total normalized airtime of APs&Non-overlapping\newline 802.11g&AP&Dynamic&Simulation\\\hline
%%		\cite{Ch-usage-based-2016}&&&&&&&\\\hline		
%		\cite{CA-VTC-14} &-Channel Assignment\newline -Fairness Handling&-AP achievable communication period \newline -AP traffic demand&-Ratio of \newline an AP's achievable throughput \newline to its traffic load\newline -Fairness&Non-overlapping\newline 802.11g&AP&Static&-Empirical\newline -Simulation\\\hline
%		\cite{CA-BW-VTC-14} &-Primary Channel Assignment\newline -Bandwidth Allocation\newline -Fairness Handling&-AP achievable throughput \newline-Number of associated clients of an AP\newline -Traffic demand of clients&-Ratio of \newline an AP's achievable throughput \newline to its traffic load\newline -Fairness&Non-overlapping\newline 802.11n/ac&AP&Static&Simulation\\\hline		
%		\cite{CAPWAP-based-CA-11} &Channel Assignment&-Overlapping channel distance \newline -RSSI between APs \newline -Number of active associated clients of an AP&Interference factors of all APs&Overlapping\newline 802.11b&AP&Static&-Empirical \newline -Simulation\\\hline			
%		\cite{Measurement-CA-WCNC-10}&Channel Assignment&-RSSI \newline -Traffic demand&Extra transmission delay&Non-overlapping\newline 802.11a/g&AP and Client&Dynamic&-Empirical\\\hline
%		\cite{CA-F-WCNC-11}&-Channel Assignment\newline -Fairness Handling&-Overlapping channel distance\newline -Client throughput&-Total normalized throughput reduction of APs\newline -Fairness&Overlapping\newline 802.11b&Client&Static&Simulation
%		\\\Xhline{3\arrayrulewidth}
%	\end{tabular}
%\end{table*}

% on --> off (N_{C(AP_n)}<W_n \ \& \  t_{inactive} \geq  T_{idle})
% off --> on (N_{C(AP_n)}\geq W_n \ \& \  t_{off} \geq  T_{offline})



%The authors explain some practical issues to design a fast dynamic channel assignment technique as follows
%\begin{itemize}
%	\item Time complexity of solution finder: the time needed to solve a channel assignment problem should be short enough so that it can be run frequently for highly dynamic WLANs. 
%	\item Collection time of traffic information
%	\item Execution time of channel reassignment
%\end{itemize}
%The computational complexity of solution finder should be reasonable 

%  (1) approach: CA
%  (2) interference model: RSSI-based, traffic demand
%%% (3) problem formulation: weighted graph vertex coloring 
% (4) objective function: total VDT (delay overhead due to the interferes)
% (5) channels: non-overlapping (802.11a/g)
% (6) Client/AP-centric: AP and Client
% (7) Client-aware: Yes, RSSI of clients
%%% (8) Perfomance evaluation: real implementation
%%% (9) Dynamic/Static: Dynamic (very good explanation)
%%% (10) Optimality: Sub-optimal
% (11) Measurement: realtime (periodic) - monitoring interface
%%% (12) Requried collecting information: RSSI and traffic demand




%\subsubsection{\textbf{\textcolor{blue}{Time-domain channel usage-based approach}}}
%
%The research study in \cite{Ch-usage-based-2016} focuses on the time-domain behavior of channel usage to propose a probabilistic model with the aim of  estimating adjacent channel interference. In other words, the authors consider temporal dynamicity of channel usage through pattern recognition to enable a time-varying dynamic channel assignment. The main idea is the scheduling of channel selection based on the probabilistic model of channel usage in time domain. In this way, the solution selects an appropriate channel for each AP and also determines a channel change scheduling with the aim of minimizing interference. 





%TABLE CA
\begin{table*}
	\centering
	\scriptsize
	\caption{Comparison of Channel Assignment (ChA) Mechanisms}
	\label{CA-table} 
	\def\arraystretch{1}
	\begin{tabular}{|c|c|c|c|c|c|c|c|}
		\Xhline{3\arrayrulewidth}
		%	\textbf{Ref.} & \multicolumn{2}{|c|}{OneTwoThree} & \multicolumn{2}{|c|}{OneTwoThree}&\textbf{Channels} & \textbf{Dynamic/Static} & \textbf{\multicolumn{2}{|c|}{OneTwoThree}}\\ \Xhline{3\arrayrulewidth}
		%		
		\multirow{2}{*}{\quad \quad \textbf{Mechanism} \quad\quad}& \multicolumn{2}{c|}{\quad\quad  Traffic-Aware \quad\quad}& \multicolumn{2}{c|}{Standards $\&$ Channels} & \multirow{2}{*}{\textbf{\quad Dynamic \quad}}&  \multicolumn{2}{c|}{\quad Performance Evaluation \quad}\\ \cline{2-5}\cline{7-8}
		
		&\textbf{\quad Downlink \quad}& \textbf{\quad Uplink \quad}& \textbf{\quad\quad Standard \quad\quad} & \textbf{\quad Partially Overlapping \quad} & & \textbf{Simulation} & \textbf{Testbed}\\ \Xhline{3\arrayrulewidth}
		%1
		WCCA \cite{Wcolor-2005} &$\times$&$\times$&802.11b&$\checkmark$&$\checkmark$&$\checkmark$&$\checkmark$\\\hline
		%2
	    CDCA \cite{J-DCA-LB-2006}&$\times$&$\times$&802.11b&$\times$&$\checkmark$&$\checkmark$&$\checkmark$\\\hline
		%3
		%4
		%
		%		
		FPLN \cite{CAPWAP-based-CA-11}&$\times$&$\times$&802.11b&$\checkmark$&$\times$&$\checkmark$&$\checkmark$\\\hline					
		%		
		%6
		% jCI \cite{DCA-2016-1}&$\checkmark$&$\checkmark$&$\times$&802.11b&$\times$&$\checkmark$&$\checkmark$&$\checkmark$\\\hline
		%
		%cloudmac-2013
		CloudMAC \cite{CloudMAC}&$\times$&$\times$&802.11a/b/g/n&$\checkmark$&$\checkmark$&$\times$&$\checkmark$\\\hline		
		%		
		%Odin 2014
		Odin \cite{Odin2}&$\times$&$\times$&802.11a/b/g/n&$\checkmark$&$\checkmark$&$\times$&$\checkmark$\\\hline				
		%		
		% -PCA 2015
		PCA \cite{802.11ac-PCA}&$\times$&$\times$&802.11ac&$\times$&$\checkmark$&$\checkmark$&$\times$\\\hline		
		%
		EMPOWER2 \cite{Primitives}&$\times$&$\times$&802.11a/b/g/n/ac&$\times$&$\checkmark$&$\times$&$\checkmark$\\\hline		
		%
		% -DCA-2 2016
		Wi5CA \cite{DCA-2}&$\times$&$\times$&802.11b/g&$\times$&$\checkmark$&$\checkmark$&$\times$		 \\\hline
		%
		APCA \cite{J-AP-DCA-2006}&$\times$&$\checkmark$&802.11b&$\times$&$\times$&$\checkmark$&$\times$\\\hline		
		%		
		TACA \cite{Traffic-aware-CA-2007}&$\checkmark$&$\checkmark$&802.11b/g&$\checkmark$&$\checkmark$&$\checkmark$&$\checkmark$\\\hline
		%		
		VDTCA \cite{Measurement-CA-WCNC-10}&$\checkmark$&$\checkmark$&802.11a/g&$\times$&$\checkmark$&$\times$&$\checkmark$\\\hline					
		%
		CUWN \cite{Cisco}&$\checkmark$&$\checkmark$&802.11a/b/g/n/ac&$\checkmark$&$\checkmark$&$\times$&$\times$\\\hline					
		%
		CAFA \cite{CA-F-WCNC-11}&$\checkmark$&$\checkmark$&802.11b&$\checkmark$&$\times$&$\checkmark$&$\times$ \\\hline
		%		
		TFACA \cite{CA-VTC-14}&$\checkmark$&$\times$&802.11g&$\times$&$\times$&$\checkmark$&$\checkmark$\\\hline		
		%		
		FBWA \cite{CA-BW-VTC-14}&$\checkmark$&$\times$&802.11a/n/ac&$\times$&$\times$&$\checkmark$&$\times$\\\hline		
		%		
		NATCA \cite{DCA-residential-2015}&$\checkmark$&$\times$&802.11g&$\times$&$\checkmark$&$\checkmark$&$\times$ 
		%\cite{Ch-usage-based-2016}&$\times$&$\times$&$\times$&$\times$&$\times$&$\times$&\\\hline		
		%
		%
		\\\Xhline{3\arrayrulewidth}
	\end{tabular}
\end{table*}



\subsection{Channel Assignment: Learned Lessons, Comparison, and Open Problems}
\label{ChAProblems}
Table \ref{CA-table} presents and compares the features of ChA mechanisms.
In the following, we study these features and identify research directions.


\subsubsection{\textbf{Dynamicity and Traffic-Awareness}}
\label{asc_dyn_tra_awar}
As Table \ref{CA-table} shows, not all the ChA mechanisms support dynamic channel reallocation.
In addition, most of the ChA mechanisms are either traffic-agnostic, or they do not recognize uplink traffic.
%For example, although EmPOWER2 architecture (see Section \ref{empower2_arch})  provides traffic information for network applications, its employed ChA mechanism is very simple and does not benefit from these features.
Furthermore, many of the ChA mechanisms do not actually discuss the data gathering policy employed, and the rest focus on the mean traffic load of APs and the mean throughput demand of clients. 
These limitations are due to two main reasons: First, the higher the rate of central interference map generation, the higher the overhead of control data exchanged by the controller. 
Second, although heuristic algorithms have been proposed to tackle the NP-hardness of the channel assignment problems, the execution duration of these algorithms might not be short enough to respond to network dynamics.
The two approaches earlier proposed in Section \ref{asc_dyn_overhead} (i.e., prediction and hybrid design) can be employed to cope with these challenges.
For example, depending on the time window and accuracy of predictions, a ChA proactively responds to dynamics, which in turn reduces the burden on real-time network mapping and fast algorithm execution.
%Meanwhile, dynamic adjustment of evaluation period and designing monitoring mechanisms that do not consume network resources excessively are open research challenges.
The second solution, i.e., hybrid design, employs multi-level decision making to support fast and low-overhead reaction to network dynamics. 
For the hybrid designs, however, the topology of controllers may be tailored depending on the control mechanism employed.
For example, while a suitable controller topology for a ChA mechanism depends on the interference relationship between APs, a more suitable topology for an AsC mechanism is to connect all the APs of a hallway to a local controller to improve the QoS of clients walking in that area.
Therefore, it is important to design architectures that support flexible communication between controllers.



\subsubsection{\textbf{Joint ChA and AsC}}
\label{asc_joint_des}
As channel switching may lead to re-association (depending on the architecture used), clients may experience communication interruption and violation of their QoS requirements \cite{SDWLAN2,Lv2013,jin2011fast}.
Therefore, it is necessary to measure and reduce the channel switching overhead of dynamic ChA mechanisms. 
In this regard, joint design of ChA and AsC is essential for uninterrupted performance guarantee.
Such a joint mechanism, for example, monitors and predicts clients' traffic pattern, performs channel assignment to maximize spatial reuse, and moves VAPs between APs to support seamless handoff as the traffic changes.
In addition, mobility prediction \cite{manweiler2013predicting,dong2012evaluation} may be employed to minimize the delay of AsC and ChA by enabling the use of proactive control mechanisms instead of using reactive mechanisms that are triggered by performance drop.
Note that mobility prediction is specifically useful in SDWLANs as these architectures enable the central collection and analysis of metrics such as RSSI to apply localization and mobility prediction algorithms.
Unfortunately, the existing joint ChA-AsC mechanisms (e.g., \cite{xu_channel_2011,Zheng2016}) operate distributively and do not provide any performance guarantee, which is required for  mission-critical applications such as medical monitoring or industrial control \cite{REWIMO}.

%most of those are implemented in a distributed way and do not address delay constraints of applications.



\subsubsection{\textbf{Partially Overlapping Channels and AP Density Management}}
\label{cha_part_overlap}
Most ChA mechanisms designed for 2.4GHz networks utilize only non-overlapping channels to decrease the interference level (as shown in Table \ref{CA-table}).
However, due to the dense deployment of APs, this approach limits the trade-off between interference reduction and channel reuse \cite{zhao_dapa:_2016}.
To address this concern, research studies have improved the performance of channel assignment by dynamically assigning partially-overlapping channels \cite{CA-F-WCNC-11}. 
These mechanisms, for example, rely on the interference relationship between APs to increase the distance between assigned channels as the pairwise interference increases.
However, these mechanisms require the underlying SDWLAN to provide APIs for efficient collection of network statistics.


In addition to utilizing partially-overlapping channels, ChA mechanisms can further improve network capacity through AP topology management, which is achieved by three main strategies:
(i) \textit{AP placement}: the site survey is performed to determine the location of APs during the installation phase \cite{zvanovec2003wireless};
(ii) \textit{AP power control}: a SDWLAN controller dynamically adjusts the transmission power level of APs \cite{li2011achieving};
(iii) \textit{AP mode control}: a SDWLAN controller dynamically turns on/off APs \cite{EmPOWER}.
Note that the second and third strategy require architectural support.
%For example, a ChA mechanism needs APIs to monitor network and make decisions that aim to reduce interference and maximize spatial reuse through both channel assignment and transmission power control.
It is also important to coordinate the aforementioned strategies with AsC to avoid clients' service interruption.
For example, by relying on the global network view established, clients association may be changed before channel assignment or transmission power control cause client disconnection. 
Although these problems have been addressed in isolation or distributively \cite{bejerano2009cell,wang2014coverage,DenseWLAN1,huang2010distributed}, centralized and integrated solutions are missing. 



\subsubsection{\textbf{Integration with Virtualization}}
\label{ch_disc_int_virt}
Our review of ChA and AsC mechanisms shows that these mechanisms are oblivious to virtualization.
Specifically, they do not take into account how the pool of resources is assigned to various network slices.
However, virtualization has serious implications on network control.
For example, mobility management becomes more challenging when network virtualization is employed.
When a client requires a new point of association due to its mobility, in addition to parameters such as fair bandwidth allocation, the available resources of APs should also be taken into account. 
More specifically, for a client belonging to slice $n$, the AsC mechanism should ensure that after the association of this client with a new AP, the QoS provided by slice $n$ and other slices is not violated.
However, as this may require client steering, the re-association cost of other clients should be minimized.
%Specifically, as discussed in Section \ref{archComp}, SDWLANs enable the slicing of network resources such as airtime.
%As another example, an AsC mechanism may dynamically update and maintain a subset of APs that should be used for handoff of the clients that belong to a slice $n$.
As an another example, a ChA mechanism may assign non-overlapping and overlapping channels to a mission-critical slice and a regular slice, respectively.
The problem becomes even more complicated when multiple controllers collaborate to manage network resources.
The integration of network slicing and control mechanisms would pave the way to use SDWLANs in emerging applications, such as mission-critical data transfer from mobile and IoT devices.

In addition to network-based slicing of resources, it is desirable to support personalized mobility services as well.
To this end, AsC mechanisms require architectural support to enable client association based on a variety of factors.
As an example, a client's VAP stores the client's handset features and exposes that information to the AsC mechanism to associate the client with APs that result in minimum energy consumption of the client.
Such interactions between architectural components and control mechanisms are unexplored.

Another important challenge of slicing at low level is the complexity of coordination.
For example, assume that different transmission powers are assigned to flows. 
In this case, simply changing the transmission power of APs working on the same channel would cause significant interference and disrupt the services being offered.
Therefore, offering low-level virtualization requires coordination between the network control mechanisms.



\subsubsection{\textbf{Coexistence of High-Throughput and Legacy Clients}}
Although new WiFi standards (e.g., 802.11n/ac) supporting high throughput are broadly used in various environments, we should not neglect the existence of legacy devices (e.g., 802.11b), especially for applications that require low throughput \cite{Tozlu2012,CYW43907,BCM4343}.
In heterogeneous networks, the throughput of high-rate 802.11ac devices is jeopardized by legacy clients, due to the wider channel widths used by 802.11n/ac through channel bonding \cite{Zeng2014,han_fair_2016}. 
Specifically, the channel bonding feature of 802.11n/ac brings about new challenges such as \textit{hidden-channel problem}, \textit{partial channel blocking}, and \textit{middle-channel starvation} \cite{802.11ac-PCA,han_fair_2016,zhang2011adaptive}. 
The bandwidth of a 802.11n/ac device (80/160MHz) is blocked or interrupted by a legacy client (20MHz) due to the partial channel blocking and hidden-channel problem.
In the middle-channel starvation problem, a wideband client is starved because its bandwidth overlaps with the channels of two legacy clients.
Hence, the wideband client can use its entire bandwidth only if both legacy clients are idle. 
Despite the heterogeneity of WLAN environments, our review shows that addressing the challenges of multi-rate networks to satisfy the diverse set of QoS requirements is still premature. 
In fact, only PCA \cite{802.11ac-PCA} proposes a central solution, and its performance has been compared with distributed mechanisms.

%\subsubsection{Multiple spatial streams}
%The emergence of new standards (802.11n/ac/ad) necessitates the use of new mechanisms. For example, when channel bonding feature is available, a DCA mechanism may address both channel assignment and channel width allocation. 
%Although there are a few techniques available for joint frequency and channel width assignment (e.g., \cite{802.11ac-PCA,CA-BW-VTC-14}), there is no technique addressing the effect of MIMO and beamforming on channel assignment.
%For example, a potential approach would be to rely on user location and coverage pattern to achieve a higher level of spatial reuse with channel assignment. 



\subsubsection{\textbf{Security}}
Wireless communication is susceptible to jamming attacks through which malicious signals generated on a frequency band avoid the reception of user traffic on that channel. 
The vulnerability of 802.11 networks to such denial of service (DoS) attacks have been investigated through various empirical measurements \cite{pelechrinis_denial_2011,bayraktaroglu2008performance,pelechrinis2009ares}.
Although the research community proposes various mechanisms (such as frequency hopping \cite{pelechrinis2010efficacy}) to cope with this problem, the existing approaches do not benefit from a central network view.
In particular, by relying on the capabilities of SDWLANs, clients and APs can periodically report their operating channel information (such as channel access time, noise level and packet success rate) to the controller, and mechanisms are required to exploit this information to perform anomaly detection. 


%Consequently, we have four types of channel assignment techniques as follows
%\begin{enumerate}[(I)]
%	\item Client-agnostic, traffic-agnostic
%	\item Client-aware, traffic-agnostic
%	\item Client-agnostic, traffic-aware
%	\item Client-aware, traffic-aware.
%\end{enumerate}

%Furthermore, the performance of channel assignment techniques are evaluated through either \textit{simulations} or \textit{empirical experiments}. Some works consider real implementation through small-scale testbeds to reveal the superiority of proposed algorithms such as \cite{Wcolor-2005,J-DCA-LB-2006,Traffic-aware-CA-2007,DCA-2016-1,CA-VTC-14,CAPWAP-based-CA-11,Measurement-CA-WCNC-10}. However, other techniques perform the performance evaluation using simulations. 




%Flashback [11] proposes a control channel technique for
%WiFi networks, by allowing stations to send short control
%messages concurrently with data transmissions, without
%affecting throughput. This ensures a low overhead con-trol plane for WiFi networks that is decoupled from the
%data plane.
\section{Related Work}
\label{sec:related_work}
We now provide a brief overview of related work in the areas of language grounding and transfer for reinforcement learning.
%There has been work on learning to make optimal local decisions for structured prediction problems~\cite{daume2006searn}.
%
%\newcite{branavan2010reading} looked at a similar task of building a partial model of the environment while following instructions. The differences with our work are (1) the text in their case is instructions, while we only have text describing the environment, and (2) their environment is deterministic, hence the transition function can be learned more easily. 
%
%TODO - model-based RL, value iteration, predictron.


\subsection{Grounding Language in Interactive Environments}
In recent years, there has been increasing interest in systems that can utilize textual knowledge to learn control policies. Such applications include interpreting help documentation~\fullcite{branavan2010reading}, instruction following~\fullcite{vogel2010learning,kollar2010toward,artzi2013weakly,matuszek2013learning,Andreas15Instructions} and learning to play computer games~\fullcite{branavan2011nonlinear,branavan2012learning,narasimhan2015language,he2016deep}. In all these applications, the models are trained and tested on the same domain.

Our work represents two departures from prior work on grounding. First, rather than optimizing control performance for a single domain,
we are interested in the multi-domain transfer scenario, where language 
descriptions drive generalization. Second, prior work used text in the form of strategy advice to directly learn the policy. Since the policies are typically optimized for a specific task, they may be harder to transfer across domains. Instead, we utilize text to bootstrap the induction of the environment dynamics, moving beyond task-specific strategies. 

%Previous work has explored the use of text manuals in game playing, %ranging from constructing useful features by mining patterns in %text~\cite{eisenstein2009reading}, learning a semantic interpreter %with access to limited gameplay examples~\cite{goldwasser2014learning} %to learning through reinforcement from in-game %rewards~\cite{branavan2011learning}. These efforts have demonstrated %the usefulness of exploiting domain knowledge encoded in text to learn %effective policies. However, these methods use the text to infer %directly the best strategy to perform a task. In contrast, our goal is %to learn mappings from the text to the dynamics of an environment and %separate out the learning of the strategy/motives. 

Another related line of work consists of systems that learn spatial and topographical maps of the environment for robot navigation using natural language descriptions~\fullcite{walter2013learning,hemachandra2014learning}. These approaches use text mainly containing appearance and positional information, and integrate it with other semantic sources (such as appearance models) to obtain more accurate maps. In contrast, our work uses language describing the dynamics of the environment, such as entity movements and interactions, which 
is complementary to static positional information received through state observations. Further, our goal is to help an agent learn policies that generalize over different stochastic domains, while their works consider a single domain.

%karthik: I don't see the direct relevance
%Another line of work explores using textual interactive %environments~\cite{narasimhan2015language,he2016deep} to ground %language understanding into actions taken by the system in the %environment. In these tasks, understanding of language is crucial, %without which a system would not be able to take reasonable actions. %Our motivation is different -- we take an existing set of tasks and %domains which are amenable to learning through reinforcement, and %demonstrate how to utilize textual knowledge to learn faster and more %optimal policies in both multitask and transfer setups.

\subsection{Transfer in Reinforcement Learning}
Transferring policies across domains is a challenging problem in reinforcement learning~\fullcite{konidaris2006framework,taylor2009transfer}. The main hurdle lies in learning a good mapping between the state and action spaces of different domains to enable effective transfer. Most previous approaches have either explored skill transfer~\fullcite{konidaris2007building,konidaris2012transfer} or value function/policy transfer~\fullcite{liu2006value,taylor2007transfer,taylor2007cross}. There have also been attempts at model-based transfer for RL~\fullcite{taylor2008transferring,nguyen2012transferring,gavsic2013pomdp,wang2015learning,joshi2018cross} but these methods either rely on hand-coded inter-task mappings for state and actions spaces or require significant interactions in the target task to learn an effective mapping. Our approach doesn't use any explicit mappings and can learn to predict the dynamics of a target task using its descriptions.

% Work by \newcite{konidaris2006autonomous} look at knowledge transfer by learning a mapping from sensory signals to reward functions.

A closely related line of work concerns transfer methods for deep reinforcement learning. \citeA{parisotto2016actor}  train a deep network to mimic pre-trained experts on source tasks using policy distillation. The learned parameters are then used to initialize a network on a target task to perform transfer. Rusu et al.~\citeyear{rusu2016progressive} facilitate transfer by freezing parameters learned on source tasks and adding a new set of parameters for every new target task, while using both sets to learn the new policy. Work by Rajendran et al.~\citeyear{rajendran20172t} uses attention networks to selectively transfer from a set of expert policies to a new task. \textcolor{black}{Barreto et al.~\citeyear{barreto2017successor} use features based on successor representations~\fullcite{dayan1993improving} for transfer across tasks in the same domain. Kansky~et~al.~\citeyear{kansky2017schema} learn a generative model of causal physics in order to help zero-shot transfer learning.} Our approach is orthogonal to all these directions since we use text to bootstrap transfer, and can potentially be combined with these methods to achieve more effective transfer. 

\textcolor{black}{There has also been prior work on zero-shot policy generalization for tasks with input goal specifications. \fullciteA{schaul2015universal} learn a universal value function approximator that can generalize across both states and goals. \fullcite{andreas2016modular} use policy sketches, which are annotated sequences of subgoals, in order to learn a hierarchical policy that can generalize to new goals. \fullciteA{oh2017zero} investigate zero-shot transfer for instruction following tasks, aiming to generalize to unseen instructions in the same domain. The main departure of our work compared to these is in the use of environment descriptions for generalization across domains rather than generalizing across text instructions.}

Perhaps closest in spirit to our hypothesis is the recent work by~\fullcite{harrison2017guiding}. Their approach makes use of paired instances of text descriptions and state-action information from human gameplay to learn a machine translation model. This model is incorporated into a policy shaping algorithm to better guide agent exploration. Although the motivation of language-guided control policies is similar to ours, their work considers transfer across tasks in a single domain, and requires human demonstrations to learn a policy.

\textcolor{black}{
\subsection{Using Task Features for Transfer}
The idea of using task features/dictionaries for zero-shot generalization has been explored previously in the context of image classification. \fullciteA{kodirov2015unsupervised} learn a joint feature embedding space between domains and also induce the corresponding projections onto this space from different class labels. 
\fullciteA{kolouri2018joint} learn a joint dictionary across visual features and class attributes using sparse coding techniques. \fullciteA{romera2015embarrassingly} model the relationship between input features, task attributes and classes as a linear model to achieve efficient yet simple zero-shot transfer for classification. \fullciteA{socher2013zero} learn a joint semantic representation space for images and associated words to perform zero-shot transfer.}

\textcolor{black}{
Task descriptors have also been explored in zero-shot generalization for control policies. \fullciteA{sinapov2015learning} use task meta-data as features to learn a mapping between pairs of tasks. This mapping is then used to select the most relevant source task to transfer a policy from. \fullciteA{isele2016using} build on the ELLA framework~\fullcite{ruvolo2013ella,ammar2014online}, and their key idea is to maintain two shared linear bases across tasks -- one for the policy ($L$) and the other for task descriptors ($D$). Once these bases are learned on a set of source tasks, they can be used to predict policy parameters for a new task given its corresponding descriptor. 
% The training scheme is similar to Actor-mimic scheme~\cite{parisotto2016actor} -- for each task, an expert policy is trained separately and then distilled into policy parameters dependent on the shared basis $L$. 
In these lines of work, the task features were either manually engineered or directly taken from the underlying system parameters defining the dynamics of the environment. In contrast, our framework only requires access to crowd-sourced textual descriptions, alleviating the need for expert domain knowledge.}





% A major difference in our work is that we utilize natural language descriptions of different environments to bootstrap transfer, requiring less exploration in the new task.

% using a policy distillation~\cite{parisotto2016actor,rusu2016progressive,yin2017knowledge} or selective attention over expert networks learnt in the source tasks~\cite{rajendran20172t}. Though these approaches provide some benefits, they still suffer from the requirement of efficiently exploring the new environment to learn how to transfer their existing policies. In contrast, we utilize natural language descriptions of different environments to bootstrap transfer, leading to more focused exploration in the target task. 


% Describe amn in detail





\section{Conclusions}
\label{sec:conclusions}

In this paper, we apply shared-workload techniques at the \sql level for
improving the throughput of \qaasl systems without incurring in additional
query execution costs. Our approach is based on query rewriting for grouping
multiple queries together into a single query to be executed in one go. This
results in a significant reduction of the aggregated data access done by the
shared execution compared to executing queries independently.

%execution times and costs of the shared scan operator when
%varying query selectivity and predicate evaluation. We observed that for
%\athena, although the cost only depends on the amount of data read, it is
%conditioned to its ability to use its statistics about the data. In some cases
%a wrong query execution plan leads to higher query execution costs, which the
%end-user has to pay. 

%\bigquery's minimum query execution cost is determined by
%the input size of a query.  However, the query cost can increase depending not
%just in the amount of computation it requires, but in the mix of resources the
%query requires.  

We presented a cost and runtime evaluation of the shared operator driving data access costs. 
Our experimental study using the TPC-H benchmark confirmed the benefits of our
query rewrite approach. Using a shared execution approach reduces significantly
the execution costs. For \athena, we are able to make it 107x cheaper and for
\bigquery, 16x cheaper taking into account Query 10 which we cannot execute,
but 128x if it is not taken into account. Moreover, when having queries that do
not share their entire execution plan, i.e., using a single global plan, we
demonstrated that it is possible to improve throughput and obtain a 10x cost
reduction in \bigquery.

%We followed the TPC systems pricing guideline for
%computing how expensive is to have a TPC-H workload working on the evaluated
%\qaasl systems. The result is that even though we are able to reduce overall
%costs a TPC-H workload in 15x for \bigquery (128x excluding query 10 which we
%could not optimize) and in 107x for \athena, the overall price is at least 10x
%more expensive than the cheapest system price published by the TPC.

There are multiple ways to extend our work. The first is
to implement a full \sql-to-\sql translation layer to encapsulate the proposed
per-operator rewrites.  Another one is to incorporate the initial work on
building a cost-based optimizer for shared execution
\cite{Giannikis:2014:SWO:2732279.2732280} as an external component for \qaasl
systems.  Moreover, incorporating different lines of work (e.g., adding
provenance computation \cite{GA09} capabilities) also based on query
rewriting is part of our future work to enhance our system.

\\
\stitle{Acknowledgements:} We thank Yifan Wu for the initial inspiration, Anant Bhardwaj for data collection, Laura Rettig on early formulations of the problem, and the support of NSF 1527765 and 1564049.

\bibliographystyle{IEEEtran}
\bibliography{IEEEabrv,references}

%\input{References111}

% biography section

%\begin{IEEEbiographynophoto}{Bruce Springsteen}
%Bruce Frederick Joseph Springsteen (born September 23, 1949) is an American songwriter, singer and guitarist. He has recorded and toured with the E Street Band. Springsteen is widely known for his brand of heartland rock infused with pop hooks, poetic lyrics, and Americana sentiments centered around his native New Jersey. His eloquence in expressing ordinary, everyday problems has earned him numerous awards, including eighteen Grammy Awards and an Academy Award, along with a notoriously dedicated and devoted global fan base. His most famous albums, Born to Run and Born in the U.S.A., epitomize his penchant for finding grandeur in the struggles of daily life. He has sold over 65 million albums in the U.S.

%Springsteen's lyrics often concern men and women struggling to make ends meet. He has gradually become identified with progressive politics. Springsteen is also noted for his support of various relief and rebuilding efforts in New Jersey and elsewhere, and for his response to the September 11, 2001 attacks, on which his album The Rising reflects.
%\end{IEEEbiographynophoto}

%\begin{IEEEbiographynophoto}{Eric Clapton}
%Eric Patrick Clapton (born 30 March 1945), nicknamed Slowhand, is a 19 time Grammy Award-winning English rock guitarist, singer, songwriter and composer. He is one of the most successful musicians of the 20th and 21st centuries, garnering an unprecedented three inductions into the Rock and Roll Hall of Fame (The Yardbirds, Cream, and solo). Often viewed by critics and fans alike as one of the greatest guitarists of all time, Clapton was ranked fourth in Rolling Stone Magazine's list of the ``100 Greatest Guitarists of All Time'' and \# 53 on their list of the Immortals: 100 Greatest Artists of All Time.

%Although Clapton's musical style has varied throughout his career, it has always remained rooted in the blues. Clapton is credited as an innovator in several phases of his career, which have included blues-rock (with John Mayall \& the Bluesbreakers and The Yardbirds) and psychedelic rock (with Cream). Clapton has also achieved great chart success in genres ranging from Delta blues (Me and Mr. Johnson) to pop (``Change the World'') and reggae (Bob Marley's ``I Shot the Sheriff''). Clapton also achieved fame with Derek and the Dominos through the hit song ``Layla''.
%\end{IEEEbiographynophoto}

% insert where needed to balance the two columns on the last page with
% biographies
%\newpage

%\begin{IEEEbiographynophoto}{John Mayall}
%John Mayall, OBE (born 29 November 1933) is a pioneering English blues singer, songwriter, and multi-instrumentalist. His musical career spans over fifty years but the most notable episode in it occurred during the late `60s. He was the founder of John Mayall \& the Bluesbreakers and has been influential in the careers of many instrumentalists, including Eric Clapton, Jack Bruce, Peter Green, John McVie, Mick Fleetwood, Mick Taylor, Don ``Sugarcane'' Harris, Harvey Mandel, Larry Taylor, Aynsley Dunbar, Jon Hiseman, Dick Heckstall-Smith, Andy Fraser, Johnny Almond, Jon Mark, Walter Trout and Coco Montoya.
%\end{IEEEbiographynophoto}

% that's all folks
\end{document}


