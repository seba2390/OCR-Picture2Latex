

\section{Conclusion}
\label{Conclusion}
%Conventional wisdom prefers to manage WLANs in a distributed way to provide more scalable approaches. However, industry and academia have switched to centrally manage of these networks due to the dense deployment of APs in enterprise WLANs. 
SDWLANs enable the implementation of network control mechanisms as applications running on a network operating system.
Specifically, SDWLAN architectures provide a set of APIs for network monitoring and dissemination of control commands.
Additionally, network applications can benefit from the global network view established in the controller to run network control mechanisms in a more efficient way, compared to distributed mechanisms.
In this paper, we categorized and reviewed the existing SDWLAN architectures based on their main contribution, including \textit{observability and configurability}, \textit{programmability}, \textit{virtualization}, \textit{scalability}, \textit{traffic shaping}, and \textit{home networks}. 
Through comparing the existing architectures as well as identifying the growing and potential applications of SDWLANs, we proposed several research directions, such as establishing trade-off between centralization and scalability, investigating and reducing the overhead of south-bound protocols, exploiting layer-1 and layer-2 programmability for virtualization, and opportunities and challenge of SDWLANs with respect to security provisioning.


After reviewing architectures, we focused on association control (AsC) and channel assignment (ChA), which are the two widely-employed network control mechanisms proposed to benefit from the features of SDWLAN architectures.
We reviewed centralized AsC and ChA mechanisms in terms of their metrics used, problem formulation and solving techniques, and the results achieved compared to distributed mechanisms.
%We discussed that due to the diversity of clients and emerging new applications with different QoS requirements and traffic behavior, it is more effective and reasonable to directly include the application-level demands and behavior of clients in formulating association optimization problems. 
%Our study showed that the AM mechanisms tend to be client-oriented with the aim of satisfying QoS requirements in terms of throughput, delay, packet loss probability and re-association costs. 

At a high level, we categorized AsC mechanisms into two groups: \textit{seamless handoff}, used to reduce the overhead of client handoff, and \textit{client steering}, to improve the performance of clients.
Furthermore, we specified several research directions towards improving AsC mechanisms, such as the design and measurement of metrics to include both uplink and downlink traffic into the decision making process, establishing proportional demand-aware fairness among clients, and including the application-level demand and behavior of clients in formulating AsC problems.



We reviewed central ChA mechanisms, and we focused on the approaches employed by these mechanisms to model the interference relationship among APs and clients.
Based on the inclusion of traffic in the decision making process, we categorized ChA mechanisms into \textit{traffic-aware} and \textit{traffic-agnostic}.
Our comparisons revealed that, given the heterogeneity of clients, the new ChA mechanisms should address the coexistence of low- and high-throughput devices.
%In particular, mitigating interference in densely deployed SDWLANs with heterogeneous clients requires optimal use of partially-overlapping channels in the 2.4GHz band and variable channel widths in the 5GHz band, which is a challenging problem.
We also discussed the challenges and potential solutions for the integration of virtualization with AsC and ChA, as well as the importance and potential benefits of AsC and ChA co-design.


