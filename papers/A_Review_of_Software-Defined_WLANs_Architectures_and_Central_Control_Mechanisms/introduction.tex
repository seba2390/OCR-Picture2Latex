%
%
%
%
%
\section{Introduction}
\label{intro}
\IEEEPARstart{W}{ireless} local area networks (WLANs) have been deployed broadly in various types of environments such as enterprises, universities, airports, shopping malls, and homes. 
The IEEE 802.11 standard, commonly known as WiFi, is becoming more popular and ubiquitous. 
In particular, the traffic demand of WiFi networks is increasing due to the emergence of high-definition video streaming, Internet of Things (IoT), cellular data offloading, online gaming, and virtual reality \cite{Tozlu2012,Baird2017,NextGen-WiFi-Survey-2016,Offloading-survey-2016,WifiAlliance1,Ericsson}.
%removed citations \cite{NGwlans-survey-2016}

Statistics and forecasts show a continuous growth in the number of WiFi-enabled devices shipped as well as the amount of traffic conveyed through WLANs. 
Figure \ref{wifi-statistics}(a) shows the number of WiFi-enabled devices, including access points (APs), network interface cards (NICs), routers, switches, laptops, tablets, smartphones, and TVs that were shipped from 2012 to 2017. 
The shipment of WiFi-enabled devices was 1.58 billion units in 2012 and is estimated to be 4.91 billion units in 2017 \cite{WiFi-Devices,WifiAlliance1,Ericsson}. 
Furthermore, the percentage of traffic conveyed by WiFi networks is growing continuously, as illustrated in Figure \ref{wifi-statistics}(b). 
During 2014, 41$\%$ of mobile data traffic was exchanged through WiFi networks, including the traffic offloaded from cellular networks into WLANs \cite{Cisco-statistics,Offloading,Offloading-survey-2016}. 
This traffic is expected to reach $53\%$ by 2019. 
In addition, the scientific community has paid great attention to 802.11 networks, as demonstrated by the 165K citations that this topic has received until 2014, according to \cite{Experimenting802.11}.

%Based on the trend of WLANs usage, it is crucial to investigate and study on the practical and theoretical issues of these networks.
%
\begin{figure}[!t]
	\centering
{\includegraphics[width=\linewidth]{Wifi-statistics.pdf}}
	\caption{(a) Increase in the number of WiFi-enabled devices shipped during 2012 to 2017. (b) Traffic growth of WiFi, cellular and Ethernet networks during 2014 to 2019 \cite{Cisco-statistics}.}
		\label{wifi-statistics}
\end{figure} 
%
%
%\begin{figure}[!t]
%	\centering
%	{\includegraphics[width=0.8\linewidth]{wifitraffic.pdf}}
%	\caption{Traffic growth of WiFi, cellular and Ethernet (wired) networks during 2014 to 2019 \cite{Cisco-statistics}.}	
%		\label{wifi-traffic}
%\end{figure} 
%


The increase in the number of WiFi-enabled devices and their growing traffic demand requires a dense installation of APs to improve spatial reuse and enhance network capacity. 
However, AP densification intensifies channel contention, increases the interference level among APs and their associated clients, and exacerbates the challenges of client handoff among APs. 
Therefore, effective association and interference control are both necessary to satisfy the QoS demands of clients in terms of throughput, delay, reliability and energy efficiency \cite{largeScaleMeas,DenseAP,DenseWLAN2,APdensity,QoS-WLANs}. 
Satisfying these requirements is even more important and challenging when WLANs are used for mission-critical application such as industrial process control, factory automation, and medical monitoring.
In particular, these applications require bounded packet delivery delay and high link reliability, and in some cases energy efficiency is a critical performance metric as well \cite{tramarin2016use,REWIMO}.


%removed reference \ciet{NGwlans-survey-2016,DenseWLAN1}
%
%
% Some approaches try to improve the performance of WLANs by modifying the 802.11 standards which brings about high amount of overhead in terms of operations costs and modifying WiFi-enabled clients  
%%
%Generally, the handling and management of WLANs interference and mobility issues can be done either in distributed or centralized manner \cite{Channel-assignment-survey-2010}. 
%Traditionally, the distributed management approach was used in which the protocols are run on access points with no central entity. By increasing the size of WLANs and demand 
%However, in centralized management of WLANs, there is at least a central entity to run control and monitoring policies and algorithms with the global view of network in order to manage all components of WLAN, i.e., APs and clients \cite{Cisco}. 
%%
%Distributed management of WLANs was an appropriate approach before the evolution of needs and pervasive deployment of WLANs so that it has turned into the main access technology of almost all organizations and user communities. 
%Due to the quick growth in the size, coverage and traffic of modern WLANs,  

In order to tackle the challenges of controlling complex wired networks (such as ISPs and data centers), \textit{software-defined networking} (SDN) has been proposed to simplify and improve the design and development of network control mechanisms by decoupling the control and data plane.
Instead of running network control mechanisms distributively on switching devices, a controller configures the switches to operate according to the decisions made centrally \cite{SDN1,SDN2}. 
Consequently, in addition to simplifying the deployment of network control mechanisms as applications running on the controller, these mechanisms can benefit from the controller's global network view.
Not only for wired networks, SDN has gained popularity for the design of WLANs and cellular networks as well \cite{SDNcellular,Crowd, mSDN2,SDNsurvey}.



A \textit{software-defined WLAN} (SDWLAN) enables central monitoring and control of network operation.
% To this end, the network architecture should support the interaction between control mechanisms and data plane equipment. 
A SDWLAN can be studied from two point of views, as follows:


\begin{itemize}
    \item  \textbf{Architecture}. A SDWLAN architecture: (i) defines the various network components (e.g., controller, APs, switches, middleboxes) used to build the network as well as the topology employed to connect these components;
    (ii) implements application programming interfaces (APIs) through which the controller communicates with network devices to perform data collection and distribute control commands;
    (iii) specifies the separation level between the control plane and data plane.
    The architectural features highly affect the development flexibility and performance of network control mechanisms \cite{Primitives,DenseAP,OpenSDWN,CloudMAC}.
    For example, when medium access control (MAC) functionalities are transferred from the APs to the controller, network control applications may implement AsC mechanisms with various capabilities and performance levels, depending on the separation level and APIs provided.
    %
    %
    \item \textbf{Control Mechanisms.} To benefit from the features of SDWLAN architectures, and in particular, the global network view provided, various control mechanisms, such as \textit{association control} (AsC), \textit{channel assignment }(ChA), transmission power control, and CCA threshold adjustment have been proposed by the research community.
    In this paper, we particularly focus on centralized AsC and ChA mechanisms because they have been widely proposed and adopted to benefit from the features of SDWLANs.
    AsC mechanisms address client-AP association for both static and dynamic clients to achieve various objectives, such as seamless mobility, fairness among clients, load balancing among APs, and interference mitigation. %\cite{WiFiSeer,802.11n-AP-Association-2014,Migration-DAM,flow-level-DAM}.
    ChA mechanisms address the problem of channel assignment to APs in order to minimize the interference level experienced by APs and clients  \cite{Measurement-CA-WCNC-10,CAPWAP-based-CA-11,802.11ac-PCA}. %
\end{itemize}


%In addition, mechanisms such as scheduling are also proposed in the context of SDWLANs to improve network performance.


% distributed management approaches cannot easily provide and guarantee the new quality of service (QoS) requirements, and limits the ability to handle the scaling, security and reliability \cite{G4}. 
%%
%In particular, using distributed approaches increases the cost of applying new algorithms and protocols because of the need to update the new protocols over all access points in the WLAN \cite{G5}. 
%%
%However, in central management of WLANs, most algorithms can be run on a central entity (controller). In this way, most management and control decisions are performed in the controller with a global view of the whole network that leads to more optimal strategies with lower deployment cost because most updates are performed on the central entity instead of updating all access points across the WLAN \cite{G6}. 
%%
%Figure \ref{fig_ref_arch} shows the general architecture of a CCWLAN.
%%


\subsection{Contributions}
\label{contributions_sec}
At a high level, this paper studies SDWLANs from two perspectives: (i) SDWLAN architectures, and (ii) the two main central network control mechanisms, AsC and ChA, that are employed to control network operation.
More precisely, the contributions of this paper are as follows:
%
%
\begin{itemize}
	\item 
	We first present an overview of software-defined networking and the standard protocols used for network monitoring, management and programming.
	Next, we review SDWLAN architectures and reveal their main objective, components, interconnection protocols, and APIs provided.
	Based on their main contribution, we classify architectures into the following categories: \textit{observability and configurability},  \textit{programmability}, \textit{virtualization}, \textit{scalability}, and \textit{traffic shaping}.
	Then, we compare these architectures and highlight future research directions considering the current and upcoming requirements of SDWLANs. 
	%
	\item We review the AsC mechanisms proposed for SDWLANs.
	We first classify these mechanisms into two groups:  
	(i) \textit{seamless handoff}, mechanisms for reducing the overhead and delay of client handoff, 
	and (ii) \textit{client steering}, mechanisms to optimize parameters such as the airtime allocated to clients.
	Client steering approaches, in turn, are classified into two groups: 
	(i) \textit{centrally-generated hints}, where the controller relies on the global network view to generate hints for the association of clients, and (ii) \textit{centrally-made decisions}, where the controller makes the association decisions and enforces the clients to apply them.
	In addition to discussing the impact of architecture on supporting seamless handoff, we compare the reviewed mechanisms in terms of optimization scope, traffic awareness, and the ability to recognize the QoS requirements of clients. 
	By discussing the capabilities and weaknesses of AsC mechanisms, we present research directions towards designing AsC solutions for heterogeneous and dynamic SDWLANs with diverse QoS requirements. 
	We also identify how the potential features of SDWLAN architectures can be employed to design more efficient AsC mechanisms.	
	%
	\item We review the ChA mechanisms proposed for SDWLANs.
	We classify these mechanisms into two groups, \textit{traffic-aware} and \textit{traffic-agnostic}, based on whether they incorporate APs' and clients' traffic into the decision making process.
	We specifically focus on how ChA mechanisms define and measure their interference metrics as well as the approaches employed for modeling and solving the formulated problems.
	Furthermore, we compare these mechanisms in terms of factors like client-awareness, dynamicity, and the unique features of different 802.11 standards. %(e.g., partially-overlapping channels and channel bonding). 
	We propose research directions towards designing mechanisms that address the demands of emerging applications.
	We also identify how the potentials of SDWLANs can be employed to design more efficient ChA mechanisms.

\end{itemize}
%
%
%
When studying AsC and ChA mechanisms, we reformulate the problems using a consistent set of notations to ease the understanding and comparison of these mechanisms.
Table \ref{acronyme-table} presents the key acronyms and notations used in this paper. 
Figure \ref{fig_paper_org} shows the high-level organization of the paper.


%
\begin{figure}[!t]
	\centering
	\includegraphics[width=0.72\linewidth]{toc.pdf}
	\caption{High-level organization of the paper.}
	\label{fig_paper_org}
\end{figure}
%


% The rest of this paper is organized as follows. 
% We present an overview of SDWLAN architectures in Section \ref{Architectures}. 
% We review centralized AsC and ChA mechanisms in Section \ref{AMmech} and \ref{CMmech}, respectively.
% A review of related surveys is given in Section \ref{exis_surveys}.
% Finally, Section \ref{Conclusion} concludes the paper. 
% {
% \footnotesize
% \setcounter{tocdepth}{2} 
% \tableofcontents
% \addtocontents{toc}{\protect\thispagestyle{empty}}
% \pagenumbering{gobble}
% }



%%%%%%\section{IEEE 802.11 Basics}
%%%%%%\label{Overview}
%%%%%%The basic strategies, algorithms and protocols dealing with interference in the PHY and MAC layers of WiFi networks are explained in this section. 
%%%%%%\subsubsection{Physical Layer}
%%%%%%\subsubsection{Medium Access Control (MAC)}
%%%%%%The basic approaches dealing with interference in the MAC layer of WiFi networks are:
%%%%%%\begin{itemize}
%%%%%%\item Basic CSMA/CA protocol
%%%%%%\item RTS/CTS signaling 
%%%%%%\item Channel assignment strategy
%%%%%%\end{itemize}
%%%%%%
%%%%%%\section{Interference Issues}
%%%%%%\label{IntIssues}
%%%%%%The steps of interference handling:
%%%%%%\begin{itemize}
%%%%%%\item Interference identification (detection)
%%%%%%\item Interference mitigation (minimization/cancellation)
%%%%%%\item ...
%%%%%%\end{itemize}
%%%%%%
%%%%%%The most important factors that can worsen the interference status of network and should be considered in order to alleviate the amount of interference are as follows:
%%%%%%\begin{itemize}
%%%%%%\item Mobility of clients
%%%%%%\item Variation of traffic demands
%%%%%%\item Irregularity in the cell shapes of BSSs
%%%%%%\end{itemize}
%%%%%%Another classification:
%%%%%%\begin{itemize}
%%%%%%\item MAC contention between APs
%%%%%%\item MAC contention between APs and clients
%%%%%%\item \textcolor{red}{MAC contention between clients of p2p links (based on standalone APs)}
%%%%%%\end{itemize} 
%%%%%%
%%%%%%Different types of interference in the WiFi networks are as follows:
%%%%%%\begin{itemize}
%%%%%%\item Co-channel (adjacent-channel) interference
%%%%%%\item External interference by non-WiFi devices/networks
%%%%%%	\begin{itemize}
%%%%%%	\item Bluetooth
%%%%%%	\item Zigbee
%%%%%%	\item Microwave ovens and the devices operating in the 2.4GHz band
%%%%%%	\end{itemize}
%%%%%%\end{itemize}
%%%%%%In the literature, several approaches have been introduced to handle the interference in WiFi networks that can be categorized as follows \cite{Channel-assignment-survey-2010}
%%%%%%
%%%%%%\begin{itemize}
%%%%%%\item Association control (load balancing) (user re-association at APs)
%%%%%%\item Power control (Power adaptation)
%%%%%%\item Channel assignment (channel hopping)
%%%%%%\item Access point (AP) placement
%%%%%%\item Traffic scheduling
%%%%%%\item Throttling at central routers
%%%%%%\item ... (we should update this list)\\
%%%%%%\end{itemize}
%%%%%%
%%%%%%Another categorization:
%%%%%%\begin{itemize}
%%%%%%\item Collaborative techniques
%%%%%%	\begin{itemize}
%%%%%%		\item Distributed
%%%%%%		\item Centralized
%%%%%%	\end{itemize}
%%%%%%\item Non-collaborative techniques
%%%%%%\end{itemize}
%%%%%%\end{grey-color}

\begin{table}
	\centering
	\caption{Key acronyms and notations}
	\label{acronyme-table} 
	\def\arraystretch{1}
	\begin{tabular}{ m{0.32\linewidth}m{0.6\linewidth} }
		\Xhline{1\arrayrulewidth}
		\textbf{Acronym} & \textbf{Definition} \\
		\Xhline{1\arrayrulewidth}
		AsC & Association Control\\		
		AP & Access Point\\
		API & Application Programming Interface \\
		BSSID& Basic Service Set ID (AP's MAC address)\\
		ChA & Channel Assignment\\		
		CCA& Clear Channel Assessment\\
		%CHDC& Coverage Hole Detection and Correction\\
		%CUWN& Cisco Unified Wireless Network\\
		DCF & Distributed Coordination Function \\
		%FW&Firewall\\
		%LM&Local MAC\\
		ISP & Internet Service Provider \\
		LVAP&Light Virtual Access Point\\
		%MB&Middle Box\\
		MAC & Medium Access Control   \\
		NFV&Network Function Virtualization\\
		NIC&Network Interface Card\\
	    %RRM&Radio Resource Management\\
	    PHY & Physical \\
	    RSSI&Received Signal Strength Index\\	
	    SDN&Software-Defined Network\\
	    SDWLAN & Software-Defined WLAN \\
	    SINR & Signal to Interference-and-Noise Ratio \\
	    %SM&Split MAC\\
	    %TPC&Transmission Power Control\\
	    VAP&Virtual Access Point\\
	    WLAN & Wireless Local Area Network \\
	    %vMB&virtual Middle Box\\
	    %WDTX&Wireless Datapath Transmission\\
		\Xhline{1\arrayrulewidth}
		\Xhline{1\arrayrulewidth}		
		\textbf{Notation} & \textbf{Definition} \\
		\Xhline{1\arrayrulewidth}	    
		$\mathcal{C} = \{  c_{1}, ..., c_{i} \}$ & Set of clients \\
		$\mathcal{AP} =\{ AP_{1},...,AP_{i} \}$ & Set of access points \\
		$\mathcal{CH} =\{ ch_{1},...,ch_{i} \}$ & Set of channels \\		
		%$r_{AP_{i},c_{j}}$ & PHY transmission rate between $AP_{i}$ and $c_{j}$ \\			
		%$X_{AP_{i},c_{j}}$ & Association of $AP_{i}$ and $c_{j}$ \\	
				
		\Xhline{1\arrayrulewidth}
	\end{tabular}
\end{table}


