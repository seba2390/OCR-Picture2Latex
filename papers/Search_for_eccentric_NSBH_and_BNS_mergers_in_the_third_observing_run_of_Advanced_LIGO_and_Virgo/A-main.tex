\documentclass[floatfix,lengthcheck,showpacs,amssymb,amsmath,amsfonts,twocolumn,nofootinbib,longbibliography]{revtex4-1}
\usepackage{amsfonts,amsmath,units,wasysym,epsfig,graphicx,verbatim,color,subfigure,graphicx}
\usepackage{amsmath}
\usepackage{latexsym}
\usepackage{amssymb}
\usepackage{amsfonts}
\usepackage{mathtools}
\usepackage{bm}
\usepackage{color}
\usepackage{float}
\usepackage{tikz}
\usepackage{adjustbox}
\usepackage{array}
\usepackage{soul}
\usepackage{appendix}
\usepackage{physics}
\usepackage{braket}
\usepackage{xcolor}
\usepackage[none]{hyphenat}
%\usepackage{academicons}
%\usepackage{orcidlink}
\usepackage{lineno}
%\linenumbers

\newcommand{\henning}{\textcolor{green}}
\newcommand{\nitz}{\textcolor{blue}}
\newcommand{\rahul}{\textcolor{yellow}}
\newcommand{\Real}{\operatorname{Re}}
\newcommand{\approximant}[1]{{\fontfamily{qcr}\selectfont{#1} }}

\def\cL{\mathcal{L}}
\def\be{\begin{equation}}
\def\ee{\end{equation}}
\def\bea{\begin{eqnarray}}
\def\eea{\end{eqnarray}}
\def\tr{\mathrm{tr}\, }
\def\no{\nonumber}
\def \D{{\cal D}}
\def \G{{\cal G}}
\def \SG{{\bf G}}
\def \VG{{\mathcal V}_{\G}}
\def \Vp{{\mathcal V}_\perp}
\def \S{{\cal S}}
\def \P{{\cal P}}
\def \N{{\cal N}}
\def \Msun {M_\odot}
\def \A{{\mathcal A}}
\def \Ab{{\bf A}}
\def \Ub{{\bf U}}
\def \Vb{{\bf V}}
\def \Sgm{{\bf \Sigma}}
\def \a{\alpha}
\def \b{\beta}
\def \M{{\mathcal M}}
\def \fl{f_{\rm lower}}
\def \fu{f_{\rm upper}}
\def \eps{\epsilon}
\def \x{{\bf x}}
\def \y{{\bf y}}
\def \h{{\bf h}}
\def \n{{\bf n}}
\def \g{{\bf g}}
\def \s{{\bf s}}
\def \sp{{\s_\perp}}
\def \ts{{\tilde s}}
\def \prj{\mathfrak{p}}
\def \tsp{\ts_{\perp}}
\def \e{{\bf e}}
\def \v{{\bf v}}
\def \w{{\bf w}}
\def \hf {\frac{1}{2}}

\begin{document}

\author{\href{https://orcid.org/0000-0002-5077-8916}{Rahul Dhurkunde}} 
\affiliation{Max-Planck-Institut fur Gravitationsphysik (Albert-Einstein-Institut), D-30167 Hannover, Germany}
\affiliation{Leibniz Universitat Hannover, D-30167 Hannover, Germany}
\author{\href{https://orcid.org/0000-0002-1850-4587}{Alexander H. Nitz}}
\affiliation{Department of Physics, Syracuse University, Syracuse, NY 13244, USA}


\date{\today}
\title{Search for eccentric NSBH and BNS mergers in the third observing run of Advanced LIGO and Virgo}
 
\begin{abstract}
%Current observations of compact-binary mergers are unable to resolve the tension amongst various proposed formation models, and thus, their formation and evolution mechanism remain elusive. 
The possible formation histories of neutron star binaries remain unresolved by current gravitational-wave catalogs. The detection of an eccentric binary system could be vital in constraining compact binary formation models. We present the first search for aligned spin eccentric neutron star-black hole binaries (NSBH) and the most sensitive search for aligned-spin eccentric binary neutron star (BNS) systems using data from the third observing run of the advanced LIGO and advanced Virgo detectors. No new statistically significant candidates are found; we constrain the local merger rate to be less than 150 $\text{Gpc}^{-3}\text{Yr}^{-1}$ for binary neutron stars in the field, and, 50, 100, and 70 $\text{Gpc}^{-3}\text{Yr}^{-1}$ for neutron star-black hole binaries in globular clusters, hierarchical triples and nuclear clusters, respectively, at the 90$\%$ confidence level if we assume that no sources have been observed from these populations. We predict the capabilities of upcoming and next-generation observatory networks; we investigate the ability of three LIGO ($\text{A}^{\#}$) detectors and Cosmic Explorer CE (20km) + CE (40km) to use eccentric binary observations for determining the formation history of neutron star binaries. We find that 2 -- 100 years of observation with three $\text{A}^{\#}$ observatories are required before we observe clearly eccentric NSBH binaries; this reduces to only 10 days -- 1 year with the CE detector network. CE will 
observe tens to hundreds of measurably eccentric binaries from
each of the formation models we consider.
\end{abstract}
 \maketitle

\section{Introduction}
Gravitational-wave (GW) astronomy is becoming routine; nearly $100$ compact binary mergers have been observed to date \cite{Nitz:2021zwj,LIGOScientific:2021djp,Olsen:2022pin} using the Advanced LIGO  \cite{LIGOScientific:2014pky} and Advanced Virgo \cite{Acernese:2015gua} observatories. These observations have fueled interest in the long standing question in astrophysics: \textit{how do compact binary systems form and evolve?} One class of models suggests these systems evolve as isolated stars in the field via common envelope \cite{Belczynski:2001uc, Mennekens:2013dja, Chu:2021evh}, stable mass transfer \cite{Klencki:2021hxe,vandenHeuvel:2017pwp} or via chemically homogeneous mixing \cite{Mandel:2015qlu, Marchant:2016wow}. Alternatively, they may be a result of a dynamical encounter of two or more separately evolved compact objects in dense environments such as globular clusters \cite{Vesperini:2010zi,Fragione:2018vty, Rodriguez:2017pec,Sedda:2020wzl}, nuclear star clusters \cite{Fragione:2018yrb, Wang:2020jsx}, young star clusters \cite{Banerjee:2020qub, Santoliquido:2020bry}, or active galactic nuclei \cite{Bartos:2016dgn, Tagawa:2020qll}.
Current GW catalogs suggest that multiple formation pathways contribute to the population of binary black hole (BBH) mergers in the Universe rather than a single preferred channel \cite{Zevin:2020gbd,KAGRA:2021duu}. However, there is an insufficient number of neutron star binaries (BNS or NSBH) as to determine if there is a preference for a single dominant channel or several competing channels present~\cite{KAGRA:2021duu,Belczynski:2017mqx}.

%observations using GWs \cite{Nitz:2021zwj,LIGOScientific:2021djp,Olsen:2022pin} or radio pulsars \cite{Tauris:2017omb,Ozel:2015fia} 

Each formation channel makes distinct predictions for the distribution of binary properties, e.g. masses, spins, eccentricity and merger rate~\cite{Mandel:2021smh}. Distinguishing these channels could be done by careful comparison of large number of detected events or by identifying rare events with properties that are unique to a specific channel \cite{Zevin:2020gbd,Zevin:2021rtf}. Orbital eccentricity carries a strong signature of a binary's evolutionary history: although the initial eccentricity of a binary is close to unity due to asymmetrical supernovae kicks  \cite{Hobbs:2005yx, Verbunt:2017zqi}, the evolution of eccentricity is influenced by the binary's surrounding environment.  In field binaries, energy dissipation solely occurs through GW emission, resulting in a swift reduction in eccentricity as the system evolves in frequency -- becoming nearly negligible when GW frequency reaches the sensitive band of current GW observatories (e.g 10 Hz) \cite{Peters:1964zz}. Whereas, in dense environments, angular momentum exchanges with a third compact object via either the Lidov-Kozai (LK) \cite{LIDOV1962719,Kozai:1962zz} or the Zeipel-Lidov-Kozai (ZLK) mechanisms \cite{Antognini:2015loa} can result in sustained non-negligible eccentricities at GW frequencies sensitive to current detectors. 


\begin{figure*}[]
    \centering
    \includegraphics[width=\linewidth, height=6.7cm]{eccentricity_dist.png}
    \caption{Distribution of orbital eccentricities (left column) for different formation models \cite{Belczynski:2017mqx, Sedda:2020wzl,Trani:2021tan,Fragione:2018yrb} at the time of formation of compact binaries (green) and when the dominant-mode of GW frequency reach 10 Hz (pink). All four models predict mergers to born with high natal eccentricities. Considering energy loss via GW only, eccentricity is quickly radiated away as evident by the clear shift in the distributions for the isolated BNS channel \cite{Belczynski:2017mqx} and for NSBH mergers in globular clusters \cite{Sedda:2020wzl}. The nuclear cluster \cite{Fragione:2018yrb} and hierarchical triples \cite{Trani:2021tan} models describe eccentricity enhancing scenarios where NSBH binaries can retain non negligible eccentricities at 10 Hz. We also show the chirp mass distribution predicted by each model in the second column.}
    \label{fig:eccentricity-dist}    
\end{figure*}

We highlight four formation scenarios in Fig. \ref{fig:eccentricity-dist} as a fiducial comparison \cite{Belczynski:2017mqx,Sedda:2020wzl,Trani:2021tan,Fragione:2018yrb}. We have considered two models without eccentricity enhancing mechanisms -- one for isolated BNS binaries \cite{Belczynski:2017mqx} and for NSBH systems in globular clusters \cite{Sedda:2020wzl}. We also study two models with eccentricity inducing mechanisms (LK or ZLK) -- NSBH systems in hierarchical triples \cite{Trani:2021tan} and in nuclear clusters \cite{Fragione:2018yrb}. In models influenced by LK or ZLK, up to $80\%$ of the systems could possess eccentricity $e_{10} \geq 0.01$ (eccentricity at dominant-mode gravitational-wave frequency of 10 Hz ($e_{10}$)) \cite{Hamers:2019oeq, Fragione:2018yrb,Trani:2021tan, Silsbee:2016djf,Rodriguez:2018jqu}. In the absence of such eccentricity-boosting mechanisms, only up to $5\%$ of the sources exceed this eccentricity \cite{Sedda:2020wzl, Belczynski:2017mqx}. Observation of an eccentric system would clearly indicate the presence of a dynamical channel. Even a null detection would allow us to put tighter constraints on the predicted merger rates which are highly sensitive to the unconstrained parameters describing physical process such as common-envelope evolution \cite{Marchant:2021hiv,Santoliquido:2020bry, Baibhav:2019gxm}, natal supernovae kicks \cite{Belczynski:2001uc,Hamers:2019oeq,Trani:2021tan,Santoliquido:2020bry,Silsbee:2016djf, Richards:2022fnq, Baibhav:2019gxm} or dynamics of dense environments \cite{Fragione:2018yrb,Ford:2021kcw,Petrovich:2017otm}. 


Four neutron star binary mergers have been observed till date: two BNS mergers GW170817 \cite{LIGOScientific:2017vwq} and GW190425 \cite{LIGOScientific:2020aai} and two potential NSBH mergers GW200105 and GW200115 \cite{LIGOScientific:2021qlt}. All of these binaries were found using searches that only model quasi-circular binary orbits \cite{Messick:2016aqy,Aubin:2020goo,Allen:2005fk,Usman:2015kfa}. Radio pulsar surveys have observed a dozen of BNS systems in the Milky Way \cite{Tauris:2017omb,Ozel:2015fia}. The two BNS mergers have eccentricities limited to $e_{10} \leq 0.024$ and $e_{10} \leq 0.048$ for GW170817 and GW190425, respectively \cite{Lenon:2020oza}. If neutron star binaries have sufficiently high eccentricities, they would be missed by these searches \cite{Huerta:2013qb,Lenon:2021zac}. The measurement of a nonzero eccentricity of a binary merger would be a smoking gun for a dynamical formation channel. In this Letter, we report the results for the first search for NSBH and the latest BNS aligned-spin eccentric systems. The previous search for BNS mergers in the data from Advanced LIGO and Virgo's second observing run used a narrower range of binary masses (shown in Fig. \ref{fig:search_region}) and did not account component-object spins \cite{Nitz:2019spj}. Search for eccentric subsolar binaries has also been performed \cite{Nitz:2021vqh}. Only unmodeled searches have been performed for eccentric BBH systems ~\cite{LIGOScientific:2019dag, LIGOScientific:2023lpe}. While these searches did not yield any new candidates, they constrained the local merger rate to be less than: $1700$ mergers $\text{Gpc}^{-3} \text{Yr}^{-1}$ for BNS systems with $e_{10} \leq 0.43$ and $0.33$ mergers $\text{Gpc}^{-3} \text{Yr}^{-1}$ for BBH systems with total mass $M \in [70M_{\odot},200M_{\odot}]$ and $e_{15} < 0.3$ at 90\% confidence. 


We do not find any new mergers in the public data from the third observing run (O3) of Advanced LIGO and Advanced Virgo observatories. We use our observations and the capabilities of future observatories to constrain an isolated BNS model and three different models for NSBH mergers in globular clusters, nuclear clusters and hierarchical triples: our observations restrict the rate of mergers for BNS binaries to be less than $\sim 150$ mergers $\text{Gpc}^{-3}\text{Yr}^{-1}$ and less than $\sim $ 50 -- 100 mergers $\text{Gpc}^{-3}\text{Yr}^{-1}$ for NSBH binaries at 90\% confidence. These constraints assume that the prior observed BNS and NSBH mergers are from alternate formation channels. If we assume they are from one or more of the channel we do consider, the measured rate would be consistent given the low number statistics. We predict the capabilities of improved second generation and upcoming third generation GW observatories to use eccentric binary observations to constraint formation models in the future. We find that a network of cosmic explorer (CE) (40 km) + CE (20km) observatories will detect the majority of sources from each of these models and could determine that a subset of the population have non-negligible eccentricities. A network of three $\text{A}^{\#}$ observatories will observe a fraction of neutron star binaries and would require at least $\sim$ two years of observation to detect a non-negligible eccentric NSBH merger from hierarchical triples or nuclear clusters.    

\begin{figure}[]
    \centering
    \includegraphics[width=\linewidth, height=6.8cm]{Search_region.png}
    \caption{ Target regions of the various eccentric searches performed to date \cite{Nitz:2019spj,Nitz:2021vqh,LIGOScientific:2019dag,LIGOScientific:2023lpe} as a function of detector-frame masses ($m_1^{det}-m_2^{det}$). No prior searches have been explored for NSBH systems nor BNS systems with spins. The prior search for BNS systems was restricted to a narrower region of masses and eccentricities ($e_{10} \leq 0.43$) and did not include spins \cite{Nitz:2019spj}. We search for spin-aligned neutron star binaries (BNS + NSBH) with eccentricities $e_{10} \leq 0.46$. The only searches for BBH sytems are unmodeled searches \cite{LIGOScientific:2019dag,LIGOScientific:2023lpe} and show the regions used to report their upper limits. The previous search for non spinning subsolar binaries restricted the eccentricities to $e_{10} \leq 0.3$ \cite{Nitz:2021vqh}.}
    \label{fig:search_region}
\end{figure}

\section{Search description and observational results}
To search for eccentric binaries, we use the PyCBC toolkit to perform a template-based matched filtering analysis to find modeled GW signals in the interferometric data \cite{Usman:2015kfa,Nitz:2017svb}. GW candidates are identified by finding peaks of the signal-to-noise (SNR) time-series, mitigating non-Gaussian noise artefacts, and checking the consistency of the data and astrophysical sources between each detector~\cite{Allen:2004gu, Nitz:2017lco,Davies:2020tsx}. Taking into account these factors and the empirically measure noise distribution, each candidate is assigned a ranking statistic value \cite{Nitz:2017svb, Davies:2020tsx, Was:2009vh}.

We search for neutron star binaries using a bank of modeled waveforms (templates) generated using a stochastic placement method \cite{Harry:2009ea,Babak:2008rb}. Our   
search region is described by five intrinsic binary parameters: detector-frame component masses ($m_1^{det}, m_2^{det}$) ranging from [1.0, 9.0] $M_{\odot}$ with cutoff on total mass $M \leq 10 M_{\odot}$, $z-$ component of the individual spins ($s_{1z}, s_{2z} \in [-0.1, 0.1]$), eccentricity $e_{20}$ at 20 Hz $\in [0, 0.28]$, and an additional source orientation parameter $l$ related to the position of the periapsis. Our eccentric bank contains $\sim$ 6 million templates which is roughly two orders of magnitude larger than an equivalent bank for quasi-circular binaries. To model the GW signals, we use TaylorF2e inspiral only waveform model \cite{Moore:2016qxz} which accounts for eccentricity corrections to the aligned-spin quasi-circular TaylorF2 model \cite{Buonanno:2009zt}. The search is reliable when using just the inspiral segment of a signal, as the merger falls outside the sensitive band of current detectors. 

We search the O3 public Advanced LIGO and Virgo dataset using broadly the same search methods as \cite{Nitz:2021zwj}. O3 was divided into two parts -- O3a and O3b, comprising in total of $\sim 249$ days of coincident time when at least two observatories were in operation. Our search does not find any new significant GW candidates and recovered the previously reported multi-detector NSBH event GW200115 with high significance. The most significant candidate has a FAR of about 1 per year which is consistent with the null hypothesis based on the observation duration. The list of top candidates, the template parameters associated with each candidate, and the configuration files necessary to reproduce the analysis are available in our data release~\cite{github}.  


\section{Constraining population models}

For a given astrophysical model with a merger rate density $\mathcal{R}(\theta, z)$, the expected number of detections within an observation period $T_{obs}$ is 
\begin{align}
    N_{detected} & = T_{obs} \int \int \mathcal{R}(\theta,z) f(\theta,z) \dfrac{dV_c}{dz}\dfrac{1}{1+z} d\theta dz ,
    \label{Eq:expected-detections}
\end{align}
where $p(\theta)$ is the distribution function of the binary parameters ($\theta$) predicted by an astrophysical model, $dV_c/dz$ is the differential co-moving volume and $f(z,\theta)$ is the probability of detecting a merger with $\theta$ parameters at a redshift $z$. We can constrain the local merger rate using lack of observations: if we assume a Poisson distribution of observed mergers then the 90\% confidence limit $R_{90}^{local}$ corresponds to a local merger rate when the expected number of detections is $\sim 2.3$ \cite{Biswas:2007ni}.


Upper limits are obtained by estimating the expected number of detections $N_{detected}$ (Eq. \ref{Eq:expected-detections}) via a Monte Carlo (MC) integration scheme for a synthetic population of mergers with binary parameter distributions predicted by the respective models \cite{Tiwari:2017ndi}. For a search, the detection probabilities can be estimated by recovering injected signals into the data. In our simulations we assume the merger rate density follows the star formation rate \cite{Madau:2016jbv} convolved with the inverse time-delay distribution \cite{Zhu:2020ffa}. The injection results from our search and the codes to estimate the observational limits are available as a part of our data release \cite{github}.  


\begin{figure*}[]
    \centering
    \includegraphics[width=\textwidth, height=6.3cm]{expected_ranges.png}
    \caption{ Predicted and observed constraints on the local merger rate for various population of BNS and NSBH sources \cite{Belczynski:2017mqx, Sedda:2020wzl,Trani:2021tan,Fragione:2018yrb}. The predicted merger rates for the reference models are shown with black markers. The observational constraints assuming a null detection from our search for 249 days of observation is shown as blue (hatched) region. Predicted constraints for an idealized search are shown for O3 (blue), three $\text{A}^{\#}$ (green) and CE (40km) + CE (20km) (pink) for an year of observation. In an idealized search signals from high mass mergers can all be detected if they exceed a network SNR of 10: achieving this, our search constraints would be up to $5\times$ tighter reaching the idealized O3 limits (solid blue line). Next generation detectors will detect tens to hundreds of binaries, but not all of them would have sufficiently high eccentricities to be distinguished from non eccentric binaries; such detections may not be associated to a dynamical channel. If we limit ourselves to systems with $e_{10} \geq 0.01$, we notice the upper limits get worse by inversely proportional to the fraction of such sources predicted by the respective model -- up to 80\% of NSBH systems from hierarchical triples or nuclear clusters. A fix threshold does not necessarily capture our ability to measure eccentricity. We define a system to have a measurable non zero eccentricity if the 90\% support of the marginalized eccentricity posterior distribution is not for a circular binary. The criterion of $e_{10} \geq 0.01$ for identifying BNS mergers with noticeable eccentricity via the CE network appears to be excessively stringent. Using our predictions we can estimate the time required for an eccentric merger observation; hierarchical triples model predict maximum merger rate of 0.34 $\text{Gpc}^{-3}\text{Yr}^{-1}$ and the $\text{A}^{\#}$ limits for systems with measurable eccentricity is 0.74 $\text{Gpc}^{-3}\text{Yr}^{-1}$, their inverse ratio suggests at least 2.2 years of observations with $\text{A}^{\#}$ for an eccentric NSBH merger observation. Three $\text{A}^{\#}$ will require 2 -- 100 years of observation to detect an eccentric NSBH merger. CE will be able to detect mergers from every model and able to determine the eccentricity of tens to hundreds of binaries.
    } 

    \label{fig:time-requirement}    
\end{figure*}

%%%%% Astrophysical constraints
\section{Astrophysical implications} 

We investigate how our observational results and the capability of future detectors can constraint different astrophysical models. We investigate four models: three dynamical pathways for NSBH systems within nuclear clusters \cite{Fragione:2018yrb}, globular clusters \cite{Sedda:2020wzl}, and hierarchical triples \cite{Trani:2021tan}, and a BNS formation model in the field \cite{Belczynski:2017mqx} to contrast the two major channels. The modeled predicted merger rates are shown in Fig. \ref{fig:time-requirement}. We constrain the local merger rate to be less than 150 $\text{Gpc}^{-3}\text{Yr}^{-1}$ (isolated BNS), 50 $\text{Gpc}^{-3}\text{Yr}^{-1}$ (NSBH in globular clusters), 100 $\text{Gpc}^{-3}\text{Yr}^{-1}$ (NSBH in hierarchical triples) and 70 $\text{Gpc}^{-3}\text{Yr}^{-1}$ (NSBH in nuclear clusters) under the assumption of non-detection from these channels. Clearly, current GW observatories cannot constrain the dynamical formation models we have considered. If we were to detect a highly eccentric merger it would necessitate a channel that can produce high eccentricities. 

Improved second generation and upcoming third generation observatories are expected a factor of a few and and more than an order of magnitude more sensitive than the current ones, respectively \cite{Asharp_sensitivity, CE_sensitivity}. Third generation detectors are anticipated to be sensitive to the majority of neutron star binaries in the Universe. So the question arises: \textit{ To what extent are future observatories able to determine the formation history of neutron star binaries?} We predict how well upcoming second and third generation observatories will be able to constrain these models using an idealized search. We investigate constraints on the local merger rate for two networks (shown in Fig. \ref{fig:time-requirement}) -- one consisting of three $\text{A}^{\#}$ observatories and another composed of CE (40km) + CE (20km) using their expected noise curves \cite{Asharp_sensitivity, CE_sensitivity}. Furthermore, we have constraints for networks involving A+ and/or Einstein Telescope (ET) which are not presented here but are available in our data release \cite{github}. In agreement with the \cite{Baibhav:2019gxm}, we find that CE will be able to detect majority of sources from each model. While current detectors may require up to $\mathcal{O}(10^3)$ years of observation to observe mergers from the considered dynamical formation models, three $\text{A}^{\#}$ observatories would begin detecting events from these channels with a few years and CE (40km) + CE (20km) with a few days of observation.  
 
Even when we are able to potentially observe binaries from each formation channel, only a fraction of binaries have sufficiently high eccentricities to distinguish them from non eccentric binaries: observations from future observatories may not be convincingly attributed to dynamical channels unless they have high eccentricity. To elucidate this, we show the population constraints for a fixed eccentricity threshold of $e_{10} \geq 0.01$ in Fig.~\ref{fig:time-requirement}. The limits for a fixed threshold scales inversely to the predicted fraction of systems satisfying this threshold: limits for mergers with $e_{10} \geq 0.01$ for NSBH in hierarchical triples or nuclear cluster models are worse only by a few factor due to large fraction of such sources predicted (see Fig. \ref{fig:eccentricity-dist}). 

The ability to measure eccentricity depends on the properties of a binary and the  \cite{Lower:2018seu} capabilities of a detector network which cannot be captured by a fixed threshold -- we assess the potential of future observatories to measure eccentricity for each binary in our simulated population models. We determine the ability of a particular type of source to have non zero eccentricity using a Markov Chain Monte Carlo (MCMC) sampling scheme. We deem a source to have measurable eccentricity if at 90\% credible level we can rule out the quasi-circular binary hypothesis. We find the threshold $e_{10} \geq 0.01$ falls short for measuring eccentric neutron star binaries with three $\text{A}^{\#}$ or for eccentric NSBH binaries with the CE network. In constrast, the CE network is more proficient in measuring eccentricities of BNS systems: the same threshold is overly optimistic for BNS.  Crucially, we find that third generation observatories are poised to detect eccentric BNS systems even in isolated binary channels.

Fig. \ref{fig:time-requirement} suggests that upcoming detector networks will detect many events from the given formation models, and a fraction of them will be eccentric observations which will enable a detailed study of their predicted eccentricity distributions. Hundreds of highly eccentric NSBH sources are expected from nuclear clusters or hierarchical triples; their nondetection would hint at lower merger rates, prompting tighter constraints on the model parameters. For example, in nuclear clusters, the distribution of stars around supermassive black holes (typically depicted by a power law $n(r) \propto r^{-\alpha}$) influences the eccentricity profile of NSBH systems \cite{Fragione:2018yrb}. An increase in $\alpha$ corresponds to more eccentric systems, so a non detection of eccentric sources would constrain the distribution of stars. CE will measure eccentricities of isolated BNS mergers; with a model of natal orbital separations, one could estimate distribution of natal eccentricities. Natal eccentricities are highly sensitive to the supernovae kick velocity \cite{Hobbs:2005yx, Richards:2022fnq} and their estimation would allow constraints on the kick velocity.

\section{Conclusions}
We have performed a search for eccentric NSBH systems and the most sensitive search for eccentric BNS systems in the data from the third observing run (O3a + O3b) of Advanced LIGO and Advanced Virgo detectors. Our search did not find any new statistically significant merger candidates and as a result we put state-of-art upper limits on the local merger rates for an isolated BNS model and three distinct dynamical NSBH models:  150 $\text{Gpc}^{-3}\text{Yr}^{-1}$ (isolated BNS), 50 $\text{Gpc}^{-3}\text{Yr}^{-1}$ (NSBH in globular clusters), 100 $\text{Gpc}^{-3}\text{Yr}^{-1}$ (NSBH in hierarchical triples) and 70 $\text{Gpc}^{-3}\text{Yr}^{-1}$ (NSBH in nuclear clusters) assuming the prior neutron star binary observations do not belong to these models. While current observations are unable to constrain these models, the observation of a single highly eccentric source would have implications on the possible formation channels and suggest the presence of a dynamical formation mechanism. Our current search is limited by the finite boundaries of our template bank and the lack of accurate high mass highly eccentric waveforms ~\cite{Nagar:2021gss,Ramos-Buades:2021adz}. 
An idealized search might be able to further tighten observational constraints on the total merger rate by $10 \%$ when all eccentricities and up to $\sim 5\times$ when all masses are modeled accurately. This motivates the development of models suitable at very high ($e_{10} \sim 0.8$) eccentricities, large mass ratio, and incorporating full models of their merger and ringdown. 

We make concrete predictions on how well upcoming detectors will be able to constrain formation models through observations of eccentric binary systems. A network of three $\text{A}^{\#}$ observatories and the combined capabilities of CE (40km) + CE (20km) will likely detect sources from these models. Our key results show that the CE detector network will identify eccentric NSBH sources within days of observation, while a network of $\text{A}^{\#}$ detectors might need anywhere between 2 to 100 years depending on the formation model considered.


\acknowledgments 
We would like to acknowledge Xisco J. Forteza, Sumit Kumar and Shichao Wu for the helpful discussions and feedback on the manuscript. We acknowledge the Max Planck Gesellschaft and the Atlas cluster computing team at Albert-Einstein Institute (AEI) Hannover for support. AHN acknowledges support from NSF grant PHY-2309240. This research has made use of data, software and/or web tools obtained from the Gravitational Wave Open Science Center (https://www.gw-openscience.org), a service of LIGO Laboratory, the LIGO Scientific Collaboration and the Virgo Collaboration. LIGO is funded by the U.S. National Science Foundation. Virgo is funded by the French Centre National de Recherche Scientifique (CNRS), the Italian Istituto Nazionale della Fisica Nucleare (INFN) and the Dutch Nikhef, with contributions by Polish and Hungarian institutes.

\clearpage
\bibliography{references}
\end{document}

