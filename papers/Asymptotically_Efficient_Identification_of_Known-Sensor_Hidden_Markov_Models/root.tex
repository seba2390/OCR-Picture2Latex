
%% bare_jrnl.tex
%% V1.4b
%% 2015/08/26
%% by Michael Shell
%% see http://www.michaelshell.org/
%% for current contact information.
%%
%% This is a skeleton file demonstrating the use of IEEEtran.cls
%% (requires IEEEtran.cls version 1.8b or later) with an IEEE
%% journal paper.
%%
%% Support sites:
%% http://www.michaelshell.org/tex/ieeetran/
%% http://www.ctan.org/pkg/ieeetran
%% and
%% http://www.ieee.org/

%%*************************************************************************
%% Legal Notice:
%% This code is offered as-is without any warranty either expressed or
%% implied; without even the implied warranty of MERCHANTABILITY or
%% FITNESS FOR A PARTICULAR PURPOSE! 
%% User assumes all risk.
%% In no event shall the IEEE or any contributor to this code be liable for
%% any damages or losses, including, but not limited to, incidental,
%% consequential, or any other damages, resulting from the use or misuse
%% of any information contained here.
%%
%% All comments are the opinions of their respective authors and are not
%% necessarily endorsed by the IEEE.
%%
%% This work is distributed under the LaTeX Project Public License (LPPL)
%% ( http://www.latex-project.org/ ) version 1.3, and may be freely used,
%% distributed and modified. A copy of the LPPL, version 1.3, is included
%% in the base LaTeX documentation of all distributions of LaTeX released
%% 2003/12/01 or later.
%% Retain all contribution notices and credits.
%% ** Modified files should be clearly indicated as such, including  **
%% ** renaming them and changing author support contact information. **
%%*************************************************************************


% *** Authors should verify (and, if needed, correct) their LaTeX system  ***
% *** with the testflow diagnostic prior to trusting their LaTeX platform ***
% *** with production work. The IEEE's font choices and paper sizes can   ***
% *** trigger bugs that do not appear when using other class files.       ***                          ***
% The testflow support page is at:
% http://www.michaelshell.org/tex/testflow/



\documentclass[journal]{IEEEtran}
%
% If IEEEtran.cls has not been installed into the LaTeX system files,
% manually specify the path to it like:
% \documentclass[journal]{../sty/IEEEtran}

\newif\ifIFAC                                                                  
\IFACfalse
                                                                               
\section{Preliminaries}
In chore division (resp. cake cutting) problem, we are asked to partition a divisible undesirable (resp. desirable) object, usually modeled by the interval $[0,1]$, among $n$ players. Let $N = \{1,2,\ldots,n\}$ be the set of players. Each player $i$ has a valuation function $v_i$ that indicates, given a subinterval $I\subseteq [0,1]$, the cost (resp. profit) of that interval for the player $i$.
%In chore division and cake cutting problems, we are asked to partition a divisible undesirable object among $n$ players. A chore is usually modeled by the interval $[0,1]$. Let $N=\{1,2,\cdots,n\}$ be the set of players, each player $i$ has a valuation function $v_i$ that indicates his cost for each subinterval in $[0,1]$. In the cake cutting problem, this function represents how good an interval is for that player.
For an interval $[x,y]$, we use $v_i(x,y)$ to denote the player $i$'s valuation for this interval. We assume that valuation functions are \textit{non-negative}, \textit{additive}, \textit{divisible} and \textit{normalized}, in other words, for each player $i$, his valuation function $v_i$ satisfies the following properties:
\begin{itemize}
\item \textit{Non-negative}: $v_i(I) \ge 0$ for every subinterval $I$ in $[0,1]$.
\item \textit{Additive}: $v_i(I_1 \cup I_2) = v_i(I_1) + v_i(I_2)$ for all disjoint intervals $I_1$ and $I_2$.
\item \textit{Divisible}: for every interval $I$ and $0 \le \lambda \le 1$, there exists an interval $I' \subseteq I$ such that $v_i(I')= \lambda v_i(I)$.
\item \textit{Normalized}: $v_i(0,1)=1$.
\end{itemize}

For an interval $I=[x,y]$, we denote $Left(I)=x$ and $Right(I)=y$. Also, we use $|I|=y-x$ to denote the width of $I$. We say that an interval $I$ is non-empty if $|I|>0$.

 We say that $P$ is a \textit{piece} of the chore if it is union of finite disjoint intervals, i.e., $P=\cup_{i=1}^{k} I_i$. For a piece $P$, we use $|P|$ to denote its width which is
$$
|P| = \sum_{i=1}^{k} |I_i| = \sum_{i=1}^{k} Right(I_i)-Left(I_i) \,.
$$
Similarly, we use $v(P)$ to denote the value of the function $v$ for $P$. It follows from additivity of valuation functions that
$$
v(P) = \sum_{i=1}^{k} v(I_i) \,.
$$
Also, we use $D_v(P)=v(P)/|P|$ to denote the density of $P$. 

The complexity of a protocol is the number of queries it makes. We use the standard Robertson and Webb query model which allows two types of queries on a valuation function $v$.
\begin{itemize}
\item ${\ev}_{v}(x,y) : $ returns $v(x,y)$.
\item ${\ct}_{v} (x,r) : $ returns $y \in [0,1]$ such that $v(x,y)=r$ or declares that answer does not exist.
\end{itemize}

An \textit{allocation} $X=(X_1,X_2,\cdots,X_n)$ is a partitioning of chore into $n$ parts  $X_1,X_2,\cdots,X_n$ such that each player $i$ receives $X_i$. We say that an allocation $X=(X_1,X_2,\cdots,X_n)$ is proportional if $v_i(X_i) \le 1/n$ for every player $i$.
%This definition can be generalized to the case players have unequal entitlements. Formally, we call an allocation $X$ proportional if $v_i(X_i) \le e_i$ for each player $i$ where $e_i > 0$ is his entitlement. Entitlements always add up to 1, i.e., $\sum_{i=1}^{n} e_i =1$. 

% For Box-plots (not supported on arXiv)
% \usepgfplotslibrary{statistics}

\usepackage{cite} % For [6]-[10] citations
\usepackage{balance} % To balance the references

\newcommand{\lbstat}{\underaccent{\bar}{\pi}}
\newcommand{\polylb}{\underaccent{\bar}{\Pi}}
\newcommand{\pii}{\pi_\infty}
\newcommand{\hMii}{\hat M_\infty}

\newcommand{\thetaNR}{\hat \theta_\text{NR}}
\newcommand{\thetaMM}{\hat \theta_\text{MM}}

\newcommand{\thetainit}{\hat \theta_\text{init}}
\newcommand{\hesslik}{\nabla_\theta^2 l_N}
\newcommand{\gradlik}{\nabla_\theta l_N}
\newcommand{\Fisher}{I_F}

\newcommand{\op}{o_\text{p}}
\newcommand{\Op}{\mathcal{O}_p}

% For dummy text
\usepackage[english]{babel}
\usepackage{blindtext}

% For page-break in equations
\allowdisplaybreaks

% Some very useful LaTeX packages include:
% (uncomment the ones you want to load)


% *** MISC UTILITY PACKAGES ***
%
%\usepackage{ifpdf}
% Heiko Oberdiek's ifpdf.sty is very useful if you need conditional
% compilation based on whether the output is pdf or dvi.
% usage:
% \ifpdf
%   % pdf code
% \else
%   % dvi code
% \fi
% The latest version of ifpdf.sty can be obtained from:
% http://www.ctan.org/pkg/ifpdf
% Also, note that IEEEtran.cls V1.7 and later provides a builtin
% \ifCLASSINFOpdf conditional that works the same way.
% When switching from latex to pdflatex and vice-versa, the compiler may
% have to be run twice to clear warning/error messages.






% *** CITATION PACKAGES ***
%
%\usepackage{cite}
% cite.sty was written by Donald Arseneau
% V1.6 and later of IEEEtran pre-defines the format of the cite.sty package
% \cite{} output to follow that of the IEEE. Loading the cite package will
% result in citation numbers being automatically sorted and properly
% "compressed/ranged". e.g., [1], [9], [2], [7], [5], [6] without using
% cite.sty will become [1], [2], [5]--[7], [9] using cite.sty. cite.sty's
% \cite will automatically add leading space, if needed. Use cite.sty's
% noadjust option (cite.sty V3.8 and later) if you want to turn this off
% such as if a citation ever needs to be enclosed in parenthesis.
% cite.sty is already installed on most LaTeX systems. Be sure and use
% version 5.0 (2009-03-20) and later if using hyperref.sty.
% The latest version can be obtained at:
% http://www.ctan.org/pkg/cite
% The documentation is contained in the cite.sty file itself.






% *** GRAPHICS RELATED PACKAGES ***
%
\ifCLASSINFOpdf
  % \usepackage[pdftex]{graphicx}
  % declare the path(s) where your graphic files are
  % \graphicspath{{../pdf/}{../jpeg/}}
  % and their extensions so you won't have to specify these with
  % every instance of \includegraphics
  % \DeclareGraphicsExtensions{.pdf,.jpeg,.png}
\else
  % or other class option (dvipsone, dvipdf, if not using dvips). graphicx
  % will default to the driver specified in the system graphics.cfg if no
  % driver is specified.
  % \usepackage[dvips]{graphicx}
  % declare the path(s) where your graphic files are
  % \graphicspath{{../eps/}}
  % and their extensions so you won't have to specify these with
  % every instance of \includegraphics
  % \DeclareGraphicsExtensions{.eps}
\fi
% graphicx was written by David Carlisle and Sebastian Rahtz. It is
% required if you want graphics, photos, etc. graphicx.sty is already
% installed on most LaTeX systems. The latest version and documentation
% can be obtained at: 
% http://www.ctan.org/pkg/graphicx
% Another good source of documentation is "Using Imported Graphics in
% LaTeX2e" by Keith Reckdahl which can be found at:
% http://www.ctan.org/pkg/epslatex
%
% latex, and pdflatex in dvi mode, support graphics in encapsulated
% postscript (.eps) format. pdflatex in pdf mode supports graphics
% in .pdf, .jpeg, .png and .mps (metapost) formats. Users should ensure
% that all non-photo figures use a vector format (.eps, .pdf, .mps) and
% not a bitmapped formats (.jpeg, .png). The IEEE frowns on bitmapped formats
% which can result in "jaggedy"/blurry rendering of lines and letters as
% well as large increases in file sizes.
%
% You can find documentation about the pdfTeX application at:
% http://www.tug.org/applications/pdftex





% *** MATH PACKAGES ***
%
%\usepackage{amsmath}
% A popular package from the American Mathematical Society that provides
% many useful and powerful commands for dealing with mathematics.
%
% Note that the amsmath package sets \interdisplaylinepenalty to 10000
% thus preventing page breaks from occurring within multiline equations. Use:
%\interdisplaylinepenalty=2500
% after loading amsmath to restore such page breaks as IEEEtran.cls normally
% does. amsmath.sty is already installed on most LaTeX systems. The latest
% version and documentation can be obtained at:
% http://www.ctan.org/pkg/amsmath





% *** SPECIALIZED LIST PACKAGES ***
%
%\usepackage{algorithmic}
% algorithmic.sty was written by Peter Williams and Rogerio Brito.
% This package provides an algorithmic environment fo describing algorithms.
% You can use the algorithmic environment in-text or within a figure
% environment to provide for a floating algorithm. Do NOT use the algorithm
% floating environment provided by algorithm.sty (by the same authors) or
% algorithm2e.sty (by Christophe Fiorio) as the IEEE does not use dedicated
% algorithm float types and packages that provide these will not provide
% correct IEEE style captions. The latest version and documentation of
% algorithmic.sty can be obtained at:
% http://www.ctan.org/pkg/algorithms
% Also of interest may be the (relatively newer and more customizable)
% algorithmicx.sty package by Szasz Janos:
% http://www.ctan.org/pkg/algorithmicx




% *** ALIGNMENT PACKAGES ***
%
%\usepackage{array}
% Frank Mittelbach's and David Carlisle's array.sty patches and improves
% the standard LaTeX2e array and tabular environments to provide better
% appearance and additional user controls. As the default LaTeX2e table
% generation code is lacking to the point of almost being broken with
% respect to the quality of the end results, all users are strongly
% advised to use an enhanced (at the very least that provided by array.sty)
% set of table tools. array.sty is already installed on most systems. The
% latest version and documentation can be obtained at:
% http://www.ctan.org/pkg/array


% IEEEtran contains the IEEEeqnarray family of commands that can be used to
% generate multiline equations as well as matrices, tables, etc., of high
% quality.




% *** SUBFIGURE PACKAGES ***
%\ifCLASSOPTIONcompsoc
%  \usepackage[caption=false,font=normalsize,labelfont=sf,textfont=sf]{subfig}
%\else
%  \usepackage[caption=false,font=footnotesize]{subfig}
%\fi
% subfig.sty, written by Steven Douglas Cochran, is the modern replacement
% for subfigure.sty, the latter of which is no longer maintained and is
% incompatible with some LaTeX packages including fixltx2e. However,
% subfig.sty requires and automatically loads Axel Sommerfeldt's caption.sty
% which will override IEEEtran.cls' handling of captions and this will result
% in non-IEEE style figure/table captions. To prevent this problem, be sure
% and invoke subfig.sty's "caption=false" package option (available since
% subfig.sty version 1.3, 2005/06/28) as this is will preserve IEEEtran.cls
% handling of captions.
% Note that the Computer Society format requires a larger sans serif font
% than the serif footnote size font used in traditional IEEE formatting
% and thus the need to invoke different subfig.sty package options depending
% on whether compsoc mode has been enabled.
%
% The latest version and documentation of subfig.sty can be obtained at:
% http://www.ctan.org/pkg/subfig




% *** FLOAT PACKAGES ***
%
%\usepackage{fixltx2e}
% fixltx2e, the successor to the earlier fix2col.sty, was written by
% Frank Mittelbach and David Carlisle. This package corrects a few problems
% in the LaTeX2e kernel, the most notable of which is that in current
% LaTeX2e releases, the ordering of single and double column floats is not
% guaranteed to be preserved. Thus, an unpatched LaTeX2e can allow a
% single column figure to be placed prior to an earlier double column
% figure.
% Be aware that LaTeX2e kernels dated 2015 and later have fixltx2e.sty's
% corrections already built into the system in which case a warning will
% be issued if an attempt is made to load fixltx2e.sty as it is no longer
% needed.
% The latest version and documentation can be found at:
% http://www.ctan.org/pkg/fixltx2e


%\usepackage{stfloats}
% stfloats.sty was written by Sigitas Tolusis. This package gives LaTeX2e
% the ability to do double column floats at the bottom of the page as well
% as the top. (e.g., "\begin{figure*}[!b]" is not normally possible in
% LaTeX2e). It also provides a command:
%\fnbelowfloat
% to enable the placement of footnotes below bottom floats (the standard
% LaTeX2e kernel puts them above bottom floats). This is an invasive package
% which rewrites many portions of the LaTeX2e float routines. It may not work
% with other packages that modify the LaTeX2e float routines. The latest
% version and documentation can be obtained at:
% http://www.ctan.org/pkg/stfloats
% Do not use the stfloats baselinefloat ability as the IEEE does not allow
% \baselineskip to stretch. Authors submitting work to the IEEE should note
% that the IEEE rarely uses double column equations and that authors should try
% to avoid such use. Do not be tempted to use the cuted.sty or midfloat.sty
% packages (also by Sigitas Tolusis) as the IEEE does not format its papers in
% such ways.
% Do not attempt to use stfloats with fixltx2e as they are incompatible.
% Instead, use Morten Hogholm'a dblfloatfix which combines the features
% of both fixltx2e and stfloats:
%
% \usepackage{dblfloatfix}
% The latest version can be found at:
% http://www.ctan.org/pkg/dblfloatfix




%\ifCLASSOPTIONcaptionsoff
%  \usepackage[nomarkers]{endfloat}
% \let\MYoriglatexcaption\caption
% \renewcommand{\caption}[2][\relax]{\MYoriglatexcaption[#2]{#2}}
%\fi
% endfloat.sty was written by James Darrell McCauley, Jeff Goldberg and 
% Axel Sommerfeldt. This package may be useful when used in conjunction with 
% IEEEtran.cls'  captionsoff option. Some IEEE journals/societies require that
% submissions have lists of figures/tables at the end of the paper and that
% figures/tables without any captions are placed on a page by themselves at
% the end of the document. If needed, the draftcls IEEEtran class option or
% \CLASSINPUTbaselinestretch interface can be used to increase the line
% spacing as well. Be sure and use the nomarkers option of endfloat to
% prevent endfloat from "marking" where the figures would have been placed
% in the text. The two hack lines of code above are a slight modification of
% that suggested by in the endfloat docs (section 8.4.1) to ensure that
% the full captions always appear in the list of figures/tables - even if
% the user used the short optional argument of \caption[]{}.
% IEEE papers do not typically make use of \caption[]'s optional argument,
% so this should not be an issue. A similar trick can be used to disable
% captions of packages such as subfig.sty that lack options to turn off
% the subcaptions:
% For subfig.sty:
% \let\MYorigsubfloat\subfloat
% \renewcommand{\subfloat}[2][\relax]{\MYorigsubfloat[]{#2}}
% However, the above trick will not work if both optional arguments of
% the \subfloat command are used. Furthermore, there needs to be a
% description of each subfigure *somewhere* and endfloat does not add
% subfigure captions to its list of figures. Thus, the best approach is to
% avoid the use of subfigure captions (many IEEE journals avoid them anyway)
% and instead reference/explain all the subfigures within the main caption.
% The latest version of endfloat.sty and its documentation can obtained at:
% http://www.ctan.org/pkg/endfloat
%
% The IEEEtran \ifCLASSOPTIONcaptionsoff conditional can also be used
% later in the document, say, to conditionally put the References on a 
% page by themselves.




% *** PDF, URL AND HYPERLINK PACKAGES ***
%
%\usepackage{url}
% url.sty was written by Donald Arseneau. It provides better support for
% handling and breaking URLs. url.sty is already installed on most LaTeX
% systems. The latest version and documentation can be obtained at:
% http://www.ctan.org/pkg/url
% Basically, \url{my_url_here}.




% *** Do not adjust lengths that control margins, column widths, etc. ***
% *** Do not use packages that alter fonts (such as pslatex).         ***
% There should be no need to do such things with IEEEtran.cls V1.6 and later.
% (Unless specifically asked to do so by the journal or conference you plan
% to submit to, of course. )


% correct bad hyphenation here
\hyphenation{op-tical net-works semi-conduc-tor}


\begin{document}
%
% paper title
% Titles are generally capitalized except for words such as a, an, and, as,
% at, but, by, for, in, nor, of, on, or, the, to and up, which are usually
% not capitalized unless they are the first or last word of the title.
% Linebreaks \\ can be used within to get better formatting as desired.
% Do not put math or special symbols in the title.
%\title{Fast and Efficient Estimation \\ for Hidden Markov Models}
%\title{A Fast Asymptotically Efficient Hybrid Algorithm for Hidden
%Markov Models with Known Sensor}
\title{Asymptotically Efficient Identification of Known-Sensor Hidden Markov Models}
%
%
% author names and IEEE memberships
% note positions of commas and nonbreaking spaces ( ~ ) LaTeX will not break
% a structure at a ~ so this keeps an author's name from being broken across
% two lines.
% use \thanks{} to gain access to the first footnote area
% a separate \thanks must be used for each paragraph as LaTeX2e's \thanks
% was not built to handle multiple paragraphs
%


\author{Robert~Mattila,
        Cristian~R.~Rojas,~\IEEEmembership{Member,~IEEE},\\
        Vikram~Krishnamurthy,~\IEEEmembership{Fellow,~IEEE}
        and~Bo~Wahlberg,~\IEEEmembership{Fellow,~IEEE}% <-this % stops a space
\thanks{This work was partially supported by the Swedish Research
Council and the Linnaeus Center ACCESS at KTH. Robert Mattila, Cristian R. Rojas and Bo Wahlberg are with the Department of
    Automatic Control and ACCESS, School of Electrical Engineering, KTH Royal
    Institute of Technology. Stockholm, Sweden. (e-mails: 
\{rmattila,crro,bo\}@kth.se).
Vikram Krishnamurthy is with the Department of Electrical and Computer
Engineering, Cornell University. Cornell Tech, NY, USA. (e-mail:
vikramk@cornell.edu).}%
        }

%\author{Michael~Shell,~\IEEEmembership{Member,~IEEE,}
        %John~Doe,~\IEEEmembership{Fellow,~OSA,}
        %and~Jane~Doe,~\IEEEmembership{Life~Fellow,~IEEE}% <-this % stops a space
%\thanks{M. Shell was with the Department
%of Electrical and Computer Engineering, Georgia Institute of Technology, Atlanta,
%GA, 30332 USA e-mail: (see http://www.michaelshell.org/contact.html).}% <-this % stops a space
%\thanks{J. Doe and J. Doe are with Anonymous University.}% <-this % stops a space
%\thanks{Manuscript received April 19, 2005; revised August 26, 2015.}}

% note the % following the last \IEEEmembership and also \thanks - 
% these prevent an unwanted space from occurring between the last author name
% and the end of the author line. i.e., if you had this:
% 
% \author{....lastname \thanks{...} \thanks{...} }
%                     ^------------^------------^----Do not want these spaces!
%
% a space would be appended to the last name and could cause every name on that
% line to be shifted left slightly. This is one of those "LaTeX things". For
% instance, "\textbf{A} \textbf{B}" will typeset as "A B" not "AB". To get
% "AB" then you have to do: "\textbf{A}\textbf{B}"
% \thanks is no different in this regard, so shield the last } of each \thanks
% that ends a line with a % and do not let a space in before the next \thanks.
% Spaces after \IEEEmembership other than the last one are OK (and needed) as
% you are supposed to have spaces between the names. For what it is worth,
% this is a minor point as most people would not even notice if the said evil
% space somehow managed to creep in.



% The paper headers
%\markboth{IEEE Signal Processing Letters,~Vol.~XX, No.~X, Month~2017~(TEMPLATE)}%
        \markboth{}{}%
%{Shell \MakeLowercase{\textit{et al.}}: Bare Demo of IEEEtran.cls for IEEE Journals}
% The only time the second header will appear is for the odd numbered pages
% after the title page when using the twoside option.
% 
% *** Note that you probably will NOT want to include the author's ***
% *** name in the headers of peer review papers.                   ***
% You can use \ifCLASSOPTIONpeerreview for conditional compilation here if
% you desire.




% If you want to put a publisher's ID mark on the page you can do it like
% this:
%\IEEEpubid{0000--0000/00\$00.00~\copyright~2015 IEEE}
% Remember, if you use this you must call \IEEEpubidadjcol in the second
% column for its text to clear the IEEEpubid mark.



% use for special paper notices
%\IEEEspecialpapernotice{(Invited Paper)}




% make the title area
\maketitle

% As a general rule, do not put math, special symbols or citations
% in the abstract or keywords.
\begin{abstract}
    We consider estimating the transition probability matrix of a finite-state
    finite-observation alphabet hidden Markov model with known observation probabilities.
    The main contribution is a two-step algorithm; a method of moments estimator
    (formulated as a convex optimization problem) followed by a single iteration of
    a Newton-Raphson maximum likelihood estimator. The two-fold contribution of this
    letter is, firstly, to theoretically show that the proposed estimator is consistent
    and asymptotically efficient, and secondly, to numerically show that the method is
    computationally less demanding than conventional methods -- in particular for large
    data sets. 
\end{abstract}

% Note that keywords are not normally used for peerreview papers.
\begin{IEEEkeywords}
    Hidden Markov models, method of moments, maximum likelihood, system identification
\end{IEEEkeywords}






% For peer review papers, you can put extra information on the cover
% page as needed:
% \ifCLASSOPTIONpeerreview
% \begin{center} \bfseries EDICS Category: 3-BBND \end{center}
% \fi
%
% For peerreview papers, this IEEEtran command inserts a page break and
% creates the second title. It will be ignored for other modes.
\IEEEpeerreviewmaketitle

\section{Introduction}
% The very first letter is a 2 line initial drop letter followed
% by the rest of the first word in caps.
% 
% form to use if the first word consists of a single letter:
% \IEEEPARstart{A}{demo} file is ....
% 
% form to use if you need the single drop letter followed by
% normal text (unknown if ever used by the IEEE):
% \IEEEPARstart{A}{}demo file is ....
% 
% Some journals put the first two words in caps:
% \IEEEPARstart{T}{his demo} file is ....
% 
% Here we have the typical use of a "T" for an initial drop letter
% and "HIS" in caps to complete the first word.

\IEEEPARstart{T}{he} \emph{hidden Markov model} (HMM) has been applied in a diverse range
of fields, e.g., signal processing \cite{krishnamurthy_partially_2016}, gene sequencing
\cite{durbin_biological_1998, vidyasagar_hidden_2014} and speech recognition
\cite{rabiner_tutorial_1989}. The standard way of estimating the parameters of an HMM is
by employing a \emph{maximum likelihood} (ML) criterion. However, numerical ``hill
climbing'' algorithms for computing the ML estimate, such as direct maximization using
Newton-Raphson (and variants, e.g., \cite{cappe_inference_2005}) and the
\emph{expectation-maximization} (EM, e.g, \cite{dempster_maximum_1977,
rabiner_tutorial_1989}) algorithm are, in general, only guaranteed to converge to local
stationary points in the likelihood surface.  It is also known that these schemes can,
depending on the initial starting point of the algorithms, the shape of the likelihood
surface and the size of the data set, exhibit long run-times\comment{ (e.g.,
\cite{srebro_local_2007})}.

\comment{Even in the infinite-sample limit, EM, or other local-search methods,
    might take a very large number of steps to traverse these near-plateaus and
    converge. For this reason the conjecture does not directly imply
tractability.}

An alternative to ML criterion is to match moments of an HMM, resulting in a \emph{method
of moments} estimator (see, e.g. \cite{kay_fundamentals_1993} for details). In such
a method, observable correlations in the HMM data are related to the parameters of the
system. The correlations are empirically estimated and used in the inverted relations to
recover parameter estimates. A number of methods of moments for HMMs have been proposed in
the recent years, e.g., \cite{lakshminarayanan_non-negative_2010, anandkumar_method_2012,
    hsu_spectral_2012, kontorovich_learning_2013, anandkumar_tensor_2014,
mattila_recursive_2015, subakan_method_2015}. The main benefits over iterative ML schemes
are usually consistency and a shorter run-time, however, typically since only low-order
moments are considered, there is a loss of efficiency in the resulting estimate.

In the present letter, the problem of estimating the transition probabilities of
a finite discrete-time HMM with known sensor uncertainties, i.e., observation
matrix, is considered. This setup can be motivated in two ways: firstly, it can
be seen as the second step in a \emph{decoupling approach} to learning the HMM
parameters (see \cite{kontorovich_learning_2013}), or alternatively, by any
application where the sensor used to measure the system is designed/known to
the user. 

The main idea in this letter is a hybrid two-step algorithm based on combining the
advantages of the two aforementioned approaches. The first step uses a method of moments
estimator which requires a single pass over the data set (compared to iterative
algorithms, such as EM, that require multiple iterations over the data set). The second
step uses the method of moments estimate to initialize a non-iterative second-order direct
likelihood maximization procedure. This allows us to avoid resorting to ad hoc heuristics
for localizing a good starting point. More importantly, we show that it is
\emph{sufficient to perform only a single iteration} of the ML procedure to obtain an
asymptotically efficient estimate. Put differently, only two passes through the data set
are necessary in order to obtain an asymptotically efficient estimate.

To summarize, the main contributions of this letter are:
\begin{itemize}
\item a proposed two-step identification algorithm\comment{, for the system dynamics of 
    an HMM with known sensor uncertainties,} that exploits the benefits of
        both the method of moments approach (low computational burden and consistency) and
        direct likelihood maximization (high accuracy);
    \item we prove the consistency and asymptotic efficiency of the proposed
        estimator. Hence, the problem of only local convergence that may haunt
        iterative ML algorithms, such as EM, is shown to be
        avoided;
    \item numerical studies that show that the proposed method is up to an order of
        magnitude faster than the standard EM algorithm -- with the same resulting
        accuracy (when the EM iterations approach the global optimum of the likelihood
        function). Moreover, the run-time is, roughly, constant for a fixed data size,
        whereas the run-time of EM is highly dependent on the data (due to the number of
        iterations needed for convergence).
\end{itemize}

The outline of the remaining part of this letter is as follows. We first present a brief
overview of related work below.  \sref{sec:preliminaries} then poses the problem formally
and \sref{sec:algorithm} presents the algorithm. In \sref{sec:analysis} asymptotical
efficiency is proven, and \sref{sec:numerical_results} presents numerical studies.
\comment{\sref{sec:conclusion} concludes the paper with a brief summary.}

\subsection*{Related Work}
\label{sec:related_work}

HMM parameter estimation is now a classical area (with more than 50 years of literature).
There has recently been interest in the machine learning community for employing methods
of moments for HMMs.  The method presented in \cite[Appendix A]{hsu_spectral_2012}
demonstrates how to recover explicit estimates of the transition and observation matrices
by exploiting the special structure of the moments of an HMM. This method has been further
generalized and put in a tensor framework; see, e.g., \cite{anandkumar_method_2012},
\cite{anandkumar_tensor_2014} and references therein. The appealing attribute of these
methods is that they generate non-iterative estimates using simple linear algebra
operations (eigen and singular-value decompositions). However, the non-negativity and
sum-to-one properties of the estimated probabilities cannot be guaranteed.

There are a number of proposed methods of moments for HMMs formulated as
optimization problems (which allow constrains to be forced on the estimates),
e.g., \cite{lakshminarayanan_non-negative_2010},
\cite{kontorovich_learning_2013} and \cite{subakan_method_2015}. \comment{The
optimization formulation addresses the problem of possibly non-stochastic
estimates (i.e., estimates that do not fulfill the non-negativity and
sum-to-one constraints).} The identification problem is \emph{decoupled} in
\cite{kontorovich_learning_2013} into two stages: first an estimation of the
output parameters, and then a moment matching optimization problem. The
resulting optimization problem is related to the one in
\cite{lakshminarayanan_non-negative_2010} and to the problem in the present work. The
method we propose in this letter could be seen as a possible improvement of the
second step in the setting of \cite{kontorovich_learning_2013}.

In the general setting, hybrid approaches, such as the combination of EM and
direct likelihood maximization, and other attempts to accelerate EM has been
studied in, e.g., \cite{meilijson_fast_1989,fessler_space-alternating_1994}.
Iterative direct likelihood maximization for HMMs, as well as methods for
obtaining the necessary gradient and Hessian expressions, are treated in, e.g.,
\cite{lystig_exact_2002, cappe_recursive_2005, cappe_inference_2005,
turner_direct_2008, khreich_survey_2012, macdonald_numerical_2014}. The
combination of a method of moments and EM has, in the case of HMMs, been
considered in \cite{kontorovich_learning_2013}. 

\comment{
In \cite{lakshminarayanan_non-negative_2010}, non-negative matrix factorization is
employed to solve an optimization problem similar to ours. However, no
convergence results are given for the algorithms presented.
\cite{subakan_method_2015} deals with the special case of left-to-right and
Bakis HMMs as a two stage procedure where one stage involves a convex
optimization problem similar to the one in the present work.}


\comment{
    \cite{mattila_recursive_2015} presented how the algorithm in current work could
be employed in a recursive setting. There, convergence was not discussed and no
comparison to ML estimation was performed.
}

\section{Preliminaries and Problem Formulation}
\label{sec:preliminaries}

All vectors are column vectors unless transposed, $\mathds{1}$ denotes the vector of all
ones. The vector operator $\text{diag}:\mathbb{R}^n \rightarrow \mathbb{R}^{n\times n}$
gives the matrix where the vector has been put on the diagonal, and all other elements are
zero. $\| \cdot \|_F$ denotes the Frobenius norm of a matrix. The element at row $i$ and
column $j$ of a matrix is $[\cdot]_{ij}$, and the element at position $i$ of a vector is
$[\cdot]_i$.  Inequalities ($>, \geq, \leq, <$) between vectors or matrices should be
interpreted elementwise. The indicator function $\ind{\cdot}$ takes the value 1 if the
expression $\cdot$ is fulfilled and 0 otherwise. Let $\to_p$ and $\to_d$ denote
convergence in probability and in distribution, respectively, and let $\mathcal{O}_p$ and
$\op$ be stochastic-order symbols. $\sim$ denotes ``distributed according to''.

\subsection{Problem Formulation}

Consider a discrete-time finite-state \emph{hidden Markov model} (HMM) on the
state space $\mathcal{X} = \{1, 2, \dots, X\}$ with the transition probability
matrix
\begin{equation}
    [P]_{ij} = \Pr{x_{k+1} = j | x_k = i}.
\end{equation} 
Observations are made from the set $\mathcal{Y} = \{1, 2, \dots, Y\}$ according
to the observation probability matrix
\begin{equation}
    [B]_{ij} = \Pr{y_k = j | x_k = i}.
\end{equation}
These matrices are row-stochastic, i.e., the elements in each row sum to one.
Denote the initial distribution as $\pi_0$ and the stationary distribution as
$\pi_\infty$.

The HMM moments are joint probabilities of tuples of observations. The second
order moments can be represented by $Y \times Y$ matrices $M_k$ with elements
\begin{equation}
    [M_k]_{ij} = \Pr{y_{k} = i, y_{k+1} = j}.
    \label{eq:Mk_def}
\end{equation}
The following equation relates the second order moments and the system
parameters,
\begin{equation}
    M_k = B^T \diag{(P^T)^k \pi_0} PB,
    \label{eq:2nd_moments}
\end{equation}
and is the key to the method of moments formulation of the problem.

As we are interested in the asymptotic behaviour, we make the assumption that the initial
distribution $\pi_0$ is known to us -- its influence will anyway diminish over time. The
most important assumption we make is that the observation probabilities $B$ are known.
There are three motivations for this assumption: i) it admits the problem to a convex
formulation, ii) it holds in any real-world application where the sensor is designed by
the user, and iii) our method can be seen as an intermediate step of the \emph{decoupling}
approach in \cite{kontorovich_learning_2013}.  The identification problem we consider is,
hence,
\begin{problem}
    Consider an HMM with known initial distribution $\pi_0$ and known observation matrix
    $B$. The HMM is initialized according to $\pi_0$ and a sequence of observations $y_0,
    y_1, \dots, y_N$ is obtained. Given the sequence of $N+1$ observations
    $\{y_k\}_{k=0}^N$, estimate the transition matrix $P$.
    \label{pr:problem3}
\end{problem}


\comment{
The complete identification problem for HMMs is to estimate all the parameters
needed to specify the HMM, i.e., the transition matrix $P$, the observation
matrix $B$ and the initial distribution $\pi_0$, from a given set of
observations. In our case, as we consider the setup of a sequence of
consecutive observations, we are interested in the asymptotic behaviour of the
estimates and hence make the assumption that the initial distribution $\pi_0$
is known. We also make the assumption that the sensor uncertainties are known
to the user. There are three motivations for this assumption: i) is that it
admits the problem to a convex formulation, ii) is that it holds in any
real-world application where the sensor is designed by the user, and iii) is
that our method can be seen as an intermediate step of the \emph{decoupling}
approach in \cite{kontorovich_learning_2013}. Hence, the problem we consider
is:
}

\section{Asymptotically Efficient Two-Step Algorithm}
\label{sec:algorithm}

In this section, we outline the two-step algorithm which is the main contribution of this
letter.

\comment{
In this section, we outline the algorithm we propose. Recall that in the method
of moments, parameters of the system are related to the observable moments. The
moments are then empirically estimated and the relations are used in reverse to
yield estimates of the parameters. This can be formulated as an optimization
problem, where the objective is to minimize the mismatch between the observed
moments and the analytical expression \eref{eq:2nd_moments}.
}

\subsection*{Step 1. Initial Method of Moments Estimate}

In light of \eref{eq:Mk_def}, use the empirical moments estimate
\begin{equation}
    [\hat M_\infty]_{ij} = \frac{1}{N} \sum_{k=0}^{N-1} \ind{y_k = i,
    y_{k+1} = j},
    \label{eq:moment_estimator}
\end{equation}
for the (stationary) second order moments. \comment{Notice that we
consider only one realization, and hence cannot hope to estimate $M_k$ for all
$k$. However, as the processes reaches stationarity, we will be able to
estimate the stationary moments using the above estimator.}


In the moment matching optimization problem, we need to impose the constraint that the
transition matrix is a valid stochastic matrix, that is: the non-negativity and sum-to-one
properties of its rows. We require that the transition matrix of the HMM is ergodic
(aperiodic and irreducible). This implies, first of all, that $\pi_\infty$ is the right
eigenvector of $P^T$ and therefore satisfies the condition $\pi_\infty = P^T \pi_\infty$,
and secondly, that $\pi_\infty$ has strictly positive entries. We therefore, also, include
in the optimization problem a polytopic bound $\polylb$ on $\pi_\infty$ such that for
a vector $x \in \polylb \Rightarrow x > 0$.\footnote{This polyhedron can, for example, be
    obtained if it is possible to \emph{a priori} lower bound the elements of the
    transition matrix $P$ using another matrix $L$. In particular, this is possible since
    then the stationary distribution $\pi_\infty$ lies in a polyhedron $\polylb$ spanned
    by the normalized (i.e., non-negative and with elements that sum to one) columns of
    the matrix $(I-L^T)^{-1}$ -- see \cite{courtois_polyhedra_1985} for details.}

To summarize, estimating the transition matrix $P$ involves solving the optimization
problem (as the limit is taken in equation \eref{eq:2nd_moments} towards stationarity):
\begin{align}
    \min_{\pi_\infty, P} & \quad \|\hat M_\infty - B^T \diag{\pi_\infty} PB \|_F^2 \notag \\
    \text{s.t.} & \quad P \geq 0, \;\;\; \quad \pii \geq 0, \notag \\
                & \quad P \mathds{1} = \mathds{1}, \quad \mathds{1}^T \pi_\infty = 1, \notag \\
                & \quad \pi_\infty \in \polylb, \quad \pi_\infty = P^T \pi_\infty.
    \label{eq:nl_matching} % Non-linear moment matching
\end{align}
This is, in general, a non-convex optimization problem. The lemma below shows
that convex optimization techniques can be used to solve the problem.

\begin{lemma}
    The solution of problem \eref{eq:nl_matching} is obtainable by solving the convex
    problem
    %\begin{empheq}[box=\fbox]{align}
    \begin{align}
        \min_A \quad & \|\hat M_\infty - B^T A B\|_F^2 \notag \\
        \text{s.t.} \quad & A \geq 0, \mathds{1}^T A \mathds{1} = 1, \notag \\
                          & A \mathds{1} \in \underaccent{\bar}{\Pi},
        A\mathds{1} = A^T \mathds{1},
        \label{eq:convex_mom}
    %\end{empheq}
    \end{align}
    and using \eref{eq:recover_pi} and \eref{eq:recover_P}, see below, to recover
    $\pi_\infty$ and $P$ from the variable $A$.
    \label{thrm:convex_equivalence}
\end{lemma}
\begin{proof}
In problem \eref{eq:convex_mom}, we identify the product $\diag{\pii} P$ in
problem \eref{eq:nl_matching} as a new parameter $A$, i.e.,
\begin{equation}
    A = \diag{\pii} P,
    \label{eq:definition_A}
\end{equation}
and optimize over its elements instead of over $\pii$ and $P$ jointly.  Notice
that it is possible to recover $\pii$ and $P$ from $A$ as follows: Firstly,
recover $\pii$ from
\begin{equation}
    A \mathds{1} = \diag{\pii} P \mathds{1} = \pii,
    \label{eq:recover_pi}
\end{equation}
employing the fact that $P \mathds{1} = \mathds{1}$.
Secondly, recover $P$ from
\begin{equation}
    \diag{\pii}^{-1} A = \diag{\pii}^{-1} \diag{\pii} P = P.
    \label{eq:recover_P}
\end{equation}
The lemma follows by noting that the cost functions in problems
\eref{eq:nl_matching} and \eref{eq:convex_mom} are the same, and then mapping
feasible solutions between the two problems.
\end{proof}

Solving problem \eref{eq:convex_mom} requires only a single pass over the data to obtain
$\hat M_\infty$, and then solving a data-size independent convex (quadratic) optimization
problem to compute an estimate of the transition matrix $P$. The trade-off compared to ML
estimation\comment{ or direct likelihood maximization,}, which requires multiple iterations
over the observation data set, is of course between estimation accuracy and computational
cost: the method of moments outlined above employs only the second order moments and will
hence have disregarded some of the information in the observed data. 

\subsection*{Step 2. Single Newton-Raphson Step}

We propose to exploit the trade-off by first obtaining an estimate of $P$ using
the convex method of moments \eref{eq:convex_mom}, and then taking \emph{a
single} Newton-Raphson step on the likelihood function to increase the accuracy
of the estimate.

The (log-)likelihood function of the observed data is
\begin{equation}
    l_{N}(\theta) = \log \Pr{ \, \{y_k\}_{k=0}^N \, | \, x_0 \sim \pi_0; \, \theta \,},
\end{equation}
where $\theta$ is a parametrization of the transition matrix $P$. Denote the estimate
resulting from the method of moments \eref{eq:convex_mom} as $\thetaMM$. Then a single
Newton-Raphson step is performed as follows:\footnote{We assume that parametrization handles the
constraints, if not, then the Newton-Raphson step can be formulated as a constrained
quadratic program.}
\begin{equation}
    %\thetaNR = \thetaMM - \big[\nabla_\theta^2 l_N(\hat \theta)\big]^{-1} \; \nabla_\theta l_N(\hat \theta),
    \thetaNR = \thetaMM - \big[ \hesslik(\thetaMM)\big]^{-1} \; \gradlik(\thetaMM),
    \label{eq:newton_step}
\end{equation}
where the gradient $\nabla_\theta l_N(\hat \theta)$ and Hessian $\nabla_\theta^2 l_N(\hat
\theta)$ can be computed recursively -- see e.g., \cite{lystig_exact_2002,
cappe_recursive_2005, cappe_inference_2005, turner_direct_2008, khreich_survey_2012,
macdonald_numerical_2014}.

Compared to direct maximization of the likelihood function using the
Newton-Raphson method (see, e.g., \cite{cappe_inference_2005,
turner_direct_2008}), this procedure is non-iterative and hence, the gradient
and Hessian \emph{need only to be computed once}.

\comment{We will in the next section show that this
yields an asymptotically efficient estimator, and in the subsequent section
show by numerical simulations that the computational burden is lower than for
EM.}

\section{Analysis}
\label{sec:analysis}

In this section we analyze the properties of the proposed algorithm. First we state the
assumptions.

\begin{assumption}
    The transition matrix $P$ has positive elements. The observation matrix $B$
    is given, has full rank and is positive. There is a polytopic bound on
    $\pii$ such that all components of $\pii$ are strictly greater than zero.
\end{assumption}

The following lemma establishes (strong) consistency of the method of moments procedure.
\begin{lemma}
    The estimates of $P$ and $\pi_\infty$ obtained using \eref{eq:recover_pi}
    and \eref{eq:recover_P} from problem \eqref{eq:convex_mom} with the
    estimator $\hat M_\infty$ in \eref{eq:moment_estimator}, converge to their
    true values as the number of observations $N \rightarrow \infty$ with
    probability one.
    \label{thrm:convergence_P_pi}
\end{lemma}
\begin{proof}[Proof (outline)]
The lemma follows by showing
\begin{enumerate}
    \item that the estimate $\hat M_\infty$ converges to $M_\infty$ (using a 
        law of large numbers, \cite[Theorem 14.2.53]{cappe_inference_2005});
    \item that the solution $\hat A$ of the optimization problem converges to
        $A$ (follows by the fundamental theorem of statistical learning
        \cite[Lemma 1.1]{campi_system_2006} and the convexity of the cost
        function \cite[Theorem 10.8]{rockafellar_convex_1970});
    \item that the solution of the optimization problem $\hat A$ can be
        uniquely mapped to $P$ and $\pi_\infty$.
\end{enumerate}
Full details are available in the supplementary material.
\end{proof}

\begin{figure*}[ht!]
    %\captionsetup[subfigure]{justification=centering}
    \centering
\begin{subfigure}[]{0.49\linewidth}
    \centering
    \begin{tikzpicture}[xscale=1.05]
      \begin{axis}[
          %title=Accuracy,
          width=\linewidth, % Scale the plot to \linewidth
          grid=major, % Display a grid
          grid style={dashed,gray!30}, % Set the style
          xlabel=samples, % Set the labels
          ylabel=RMSE,
          x post scale=0.9, % Decrease width of figure
          y post scale=0.51, % Decrease height of figure
          %enlarge x limits=false, % Don't have any space between coordinate
                                  % axis and the plot
          xmode=log,
          ymode=log,
          legend pos=north east,
          legend style={font=\fontsize{6}{5}\selectfont}, % \fontsize{<size>}{<baselineskip>}\selectfont
          cycle list name=black white,
        ]

        \addplot table[x=N,y=mom,col sep=comma]{data/benchmark_X5_Y5_median.csv}; 
        \addplot table[x=N,y=em,col sep=comma]{data/benchmark_X5_Y5_median.csv}; 
        \addplot table[x=N,y=em_mom,col sep=comma]{data/benchmark_X5_Y5_median.csv}; 
        \addplot table[x=N,y=em_true,col sep=comma]{data/benchmark_X5_Y5_median.csv}; 
        \addplot table[x=N,y=newton,col sep=comma]{data/benchmark_X5_Y5_median.csv}; 

        \legend{MM, EM, EM-MM, EM-True, 2S}
      \end{axis}
    \end{tikzpicture}
    %\caption{Root mean squared error (RMSE) of the estimated transition matrix.}
    \label{fig:accuracy}
\end{subfigure}
\begin{subfigure}[]{0.49\linewidth}
    \centering
    \vspace{-0.25cm}
    \begin{tikzpicture}[xscale=1.05]
      \begin{axis}[
          %title=Time,
          width=\linewidth, % Scale the plot to \linewidth
          grid=major, % Display a grid
          grid style={dashed,gray!30}, % Set the style
          xlabel=samples, % Set the labels
          ylabel={time [sec]},
          x post scale=0.9, % Decrease width of figure
          y post scale=0.51, % Decrease height of figure
          %enlarge x limits=false, % Don't have any space between coordinate
                                  % axis and the plot
          xmode=log,
          ymode=log,
          ymax=1e3, % Make sure we get the 10^1 mark and minor ticks
          legend pos=north west,
          %legend style={font=\fontsize{8}{5}\selectfont}, % \fontsize{<size>}{<baselineskip>}\selectfont
          legend style={font=\fontsize{6}{5}\selectfont}, % \fontsize{<size>}{<baselineskip>}\selectfont
          cycle list name=black white,
        ]

        \addplot table[x=N,y=mom_time,col sep=comma]{data/benchmark_X5_Y5_median.csv}; 
        \addplot table[x=N,y=em_time,col sep=comma]{data/benchmark_X5_Y5_median.csv}; 
        \addplot table[x=N,y=em_mom_time,col sep=comma]{data/benchmark_X5_Y5_median.csv}; 
        \addplot table[x=N,y=em_true_time,col sep=comma]{data/benchmark_X5_Y5_median.csv}; 
        \addplot table[x=N,y=newton_time,col sep=comma]{data/benchmark_X5_Y5_median.csv}; 

        %\legend{MM, EM, EM-MM, EM-True, 2S}
      \end{axis}
    \end{tikzpicture}
    %\caption{Run-time.}
    \label{fig:runtime}
\end{subfigure}
    \caption{RMSE and run-time simulation data.}
    \label{fig:numerical}
    %\vspace{-0.5cm}
\end{figure*}

Next, we provide the main theorem of the letter.
\begin{theorem}
    The estimate $\thetaNR$ obtained by the two-step algorithm
    \eref{eq:convex_mom}-\eref{eq:newton_step} is asymptotically efficient, i.e., as $N
    \to \infty$,
    \begin{equation}
        \sqrt{N}(\thetaNR - \theta^*) \to_d \mathcal{N}(0, \Fisher^{-1}(\theta^*)),
    \end{equation}
    where $\mathcal{N}$ is a normal distribution, $\theta^*$ corresponds to the
    true parameters and $\Fisher$ is the Fisher information matrix.
    \label{thrm:as_eff}
\end{theorem}
\begin{proof}[Proof (outline)]
    The theorem follows by showing that
    \begin{enumerate}
        \item the estimate $\hat M_\infty$ follows a central limit theorem
            \cite[Corollary 5]{jones_markov_2004}, and
            using this, concluding that $\hMii = M_\infty
            + \mathcal{O}_p(N^{-1/2})$ \cite[Appendix A]{pollard_convergence_1984};
        \item this order in probability can be propagated through the
            optimization problem \eref{eq:convex_mom} to obtain a similar order
            on $\hat P$ and $\hat \pi_\infty$ \cite[Theorem
            2.1]{daniel_stability_1973};
        \item verifying that certain regularity conditions hold to ensure
            that we have a central limit theorem for the gradient and
            a law of large numbers for the Hessian matrix of the log-likelihood
            function \cite[Theorems 12.5.5 and 12.5.6]{cappe_inference_2005};
        \item verifying by explicit computation that the single Newton-Raphson
            step yields an asymptotically efficient estimator.
    \end{enumerate}
    Again, full details are available in the supplementary material.
    %\blindmathtrue
    %\blindtext

\end{proof}

\section{Numerical Evaluation}
\label{sec:numerical_results}

In this section, we evaluate the performance of the proposed two-step algorithm and
compare it to the standard EM algorithm for ML estimation. The EM implementation of Matlab
R2015a was employed (modified as to account for the fact that the observation matrix is
assumed known). The first step of the proposed algorithm, i.e., solving the convex
optimization problem \eref{eq:convex_mom}, was performed using the CVX package
\cite{grant_cvx_2014}. The second step, i.e., the single Newton-Raphson update
\eref{eq:newton_step}, can be implemented in (at least) two ways. The first is to
recursively compute the gradient and Hessian as explained in, e.g.,
\cite{lystig_exact_2002, cappe_recursive_2005, cappe_inference_2005, turner_direct_2008,
khreich_survey_2012, macdonald_numerical_2014}. The second, and the one we opted for, is
to use \emph{automatic differentiation} (AD, e.g., \cite{griewank_evaluating_2008}). We
interfaced Matlab to the ForwardDiff.jl-package in Julia \cite{revels_forward-mode_2016}
in our implementation. A small regularization term was added to the Hessian. Each
simulation was run on an Intel Xeon CPU at 3.1 GHz.

\comment{
    We used random systems with the transition and observation matrices generated
    as
    \begin{equation}
        P = \text{normalize\_rows}\big\{I + \frac{1}{X} \; \mathcal{U}[0,1]\big\}
    \end{equation}
    and
    \begin{equation}
    B = \text{normalize\_rows}\big\{\begin{bmatrix}I & 0 \end{bmatrix} + \frac{1}{Y} \; \mathcal{U}[0,1]\big\},
    \end{equation}
    where $I$ is the identity matrix, normalize\_rows makes sure that the row sum
    is one and $\mathcal{U}$ denotes (a matrix of) uniformly distributed numbers.
    For the sake of robustness, we used as an elementwise lower bound one tenth of
    the minimum element of the true stationary distribution of each system.
}

\begin{figure}[b!]
\centering
\includegraphics{data/box_plot}
\caption{Each box contains 100 simulations. \emph{(Left)} RMSE of the proposed algorithm
at different data sizes. \emph{(Right)} RMSE at $5\!\cdot\!10^5$ samples (one outlier not
seen).}
  \label{fig:box_plots}
\end{figure}

We sampled observations from randomly generated systems of size $X = Y = 5$.  Notice that
there are a total of 20 unknown parameters (i.e., elements of $P$) to estimate for such
systems.  We used an elementwise lower bound $\polylb$ of one tenth of the minimum element
of the true stationary distribution of each system. We compared the performance of the
proposed two-step algorithm (2S), to the estimate resulting from the method of moments
(MM), as well as, the EM algorithm started in three different initial points: a random
point (EM), the method of moments estimate (EM-MM) and the true parameter values
(EM-True). 

\fref{fig:numerical} presents the median over 100 simulations for each batch size of,
left, the root mean squared error (RMSE) and, right, the run-time. \comment{We stress the
logarithmic scale being used.}  \fref{fig:box_plots} presents box plots of, left, the
RMSEs of the proposed algorithm at various data sizes and, right, the RMSEs of the
compared algorithms for $5\cdot 10^5$ samples. All boxes contain 100 simulations. Three
things can be noted from the figures. 

Firstly, in the left plot of \fref{fig:numerical}, the loss of accuracy resulting from
only using the second order moments (compared to all moments in EM) is apparent from the
distance between the MM-curve and the EM-curves. This can also be seen in the right plot
of \fref{fig:box_plots}.

Secondly, also in the left plot of \fref{fig:numerical}, we see that the asymptotics
become valid around $10^5$ samples which takes the estimate resulting from the proposed
two-step method down to the accuracy of EM. The same conclusion is indicated by the left
plot of \fref{fig:box_plots}, where the number of observed outliers drop. These occurred
when the Hessian was not negative definite -- a result of the initial estimate not being
sufficiently close to the maximum of the likelihood function. Note that this can be
detected prior to employing the method.

Thirdly, in the right plot of \fref{fig:numerical}, it can be see that the run-times of
the compared algorithms differ by up to an order of magnitude. It should moreover be noted
that the run-time of the proposed algorithm is more or less constant for a fixed data size
(i.e., independent of the system and the observations), whereas the run-time of EM is
highly dependent on the data (due to the number of iterations needed to converge): The
maximum run-times for $5 \cdot 10^5$ observations were 1083, 480, 166 seconds for EM,
EM-MM and EM-True, respectively, whereas for the proposed method it was 54 seconds.

\comment{
We compare the algorithm to the Matlab implementation of the EM algorithm for
HMMs. Since EM is a local search algorithm, the choice of starting point is
crucial, in two senses: we are only guaranteed to converge locally, and
depending on the shape of the likelihood surface, convergence might be slow
(i.e. require many iterations). Also, each iteration of EM is computationally heavy
when the batch size of observations is large. The method of moments estimate is
obtained using only one pass over the data and then by solving a convex
optimization problem whose complexity is independent of the number of samples
(it depends only on the dimension of the system). 
}

%One possible application of our purposed method is to
%generate a starting point for a local likelihood maximization scheme (e.g. EM), that
%hopefully is located close enough to the true maximum, lowering the amount of
%iterations required to reach a specified accuracy. And thus, in turn, reduce
%the runtime of the estimation procedure.


\section{Conclusion}
\label{sec:conclusion}

This letter has proposed and analyzed a two-step algorithm for identification
of HMMs with known sensor uncertainties. A method of moments was combined with
direct likelihood maximization to exploit the benefits of both approaches:
lower computational cost and consistency in the former, and accuracy in the
later. Theoretical guarantees were given for asymptotic efficiency and
numerical simulations showed that the algorithm can yield the same accuracy as
the standard EM algorithm, but in up to an order of magnitude less time.

% An example of a floating figure using the graphicx package.
% Note that \label must occur AFTER (or within) \caption.
% For figures, \caption should occur after the \includegraphics.
% Note that IEEEtran v1.7 and later has special internal code that
% is designed to preserve the operation of \label within \caption
% even when the captionsoff option is in effect. However, because
% of issues like this, it may be the safest practice to put all your
% \label just after \caption rather than within \caption{}.
%
% Reminder: the "draftcls" or "draftclsnofoot", not "draft", class
% option should be used if it is desired that the figures are to be
% displayed while in draft mode.
%
%\begin{figure}[!t]
%\centering
%\includegraphics[width=2.5in]{myfigure}
% where an .eps filename suffix will be assumed under latex, 
% and a .pdf suffix will be assumed for pdflatex; or what has been declared
% via \DeclareGraphicsExtensions.
%\caption{Simulation results for the network.}
%\label{fig_sim}
%\end{figure}

% Note that the IEEE typically puts floats only at the top, even when this
% results in a large percentage of a column being occupied by floats.


% An example of a double column floating figure using two subfigures.
% (The subfig.sty package must be loaded for this to work.)
% The subfigure \label commands are set within each subfloat command,
% and the \label for the overall figure must come after \caption.
% \hfil is used as a separator to get equal spacing.
% Watch out that the combined width of all the subfigures on a 
% line do not exceed the text width or a line break will occur.
%
%\begin{figure*}[!t]
%\centering
%\subfloat[Case I]{\includegraphics[width=2.5in]{box}%
%\label{fig_first_case}}
%\hfil
%\subfloat[Case II]{\includegraphics[width=2.5in]{box}%
%\label{fig_second_case}}
%\caption{Simulation results for the network.}
%\label{fig_sim}
%\end{figure*}
%
% Note that often IEEE papers with subfigures do not employ subfigure
% captions (using the optional argument to \subfloat[]), but instead will
% reference/describe all of them (a), (b), etc., within the main caption.
% Be aware that for subfig.sty to generate the (a), (b), etc., subfigure
% labels, the optional argument to \subfloat must be present. If a
% subcaption is not desired, just leave its contents blank,
% e.g., \subfloat[].


% An example of a floating table. Note that, for IEEE style tables, the
% \caption command should come BEFORE the table and, given that table
% captions serve much like titles, are usually capitalized except for words
% such as a, an, and, as, at, but, by, for, in, nor, of, on, or, the, to
% and up, which are usually not capitalized unless they are the first or
% last word of the caption. Table text will default to \footnotesize as
% the IEEE normally uses this smaller font for tables.
% The \label must come after \caption as always.
%
%\begin{table}[!t]
%% increase table row spacing, adjust to taste
%\renewcommand{\arraystretch}{1.3}
% if using array.sty, it might be a good idea to tweak the value of
% \extrarowheight as needed to properly center the text within the cells
%\caption{An Example of a Table}
%\label{table_example}
%\centering
%% Some packages, such as MDW tools, offer better commands for making tables
%% than the plain LaTeX2e tabular which is used here.
%\begin{tabular}{|c||c|}
%\hline
%ones & Two\\
%\hline
%Three & Four\\
%\hline
%\end{tabular}
%\end{table}


% Note that the IEEE does not put floats in the very first column
% - or typically anywhere on the first page for that matter. Also,
% in-text middle ("here") positioning is typically not used, but it
% is allowed and encouraged for Computer Society conferences (but
% not Computer Society journals). Most IEEE journals/conferences use
% top floats exclusively. 
% Note that, LaTeX2e, unlike IEEE journals/conferences, places
% footnotes above bottom floats. This can be corrected via the
% \fnbelowfloat command of the stfloats package.

% if have a single appendix:
%\appendix[Proof of the Zonklar Equations]
% or
%\appendix  % for no appendix heading
% do not use \section anymore after \appendix, only \section*
% is possibly needed

% use appendices with more than one appendix
% then use \section to start each appendix
% you must declare a \section before using any
% \subsection or using \label (\appendices by itself
% starts a section numbered zero.)
%

% Can use something like this to put references on a page
% by themselves when using endfloat and the captionsoff option.
\ifCLASSOPTIONcaptionsoff
  \newpage
\fi



% trigger a \newpage just before the given reference
% number - used to balance the columns on the last page
% adjust value as needed - may need to be readjusted if
% the document is modified later
%\IEEEtriggeratref{8}
% The "triggered" command can be changed if desired:
%\IEEEtriggercmd{\enlargethispage{-5in}}

% references section

% can use a bibliography generated by BibTeX as a .bbl file
% BibTeX documentation can be easily obtained at:
% http://mirror.ctan.org/biblio/bibtex/contrib/doc/
% The IEEEtran BibTeX style support page is at:
% http://www.michaelshell.org/tex/ieeetran/bibtex/

% Push the references to a separate page
\clearpage

% Balance the references
\balance

\bibliographystyle{IEEEtran}
% argument is your BibTeX string definitions and bibliography database(s)
\bibliography{IEEEabrv,../rob_references}

% Unbalance for the appendix
\clearpage
\nobalance

%
% <OR> manually copy in the resultant .bbl file
% set second argument of \begin to the number of references
% (used to reserve space for the reference number labels box)
%\begin{thebibliography}{1}

%\bibitem{IEEEhowto:kopka}
%H.~Kopka and P.~W. Daly, \emph{A Guide to \LaTeX}, 3rd~ed.\hskip 1em plus
%  0.5em minus 0.4em\relax Harlow, England: Addison-Wesley, 1999.

%\end{thebibliography}

% biography section
% 
% If you have an EPS/PDF photo (graphicx package needed) extra braces are
% needed around the contents of the optional argument to biography to prevent
% the LaTeX parser from getting confused when it sees the complicated
% \includegraphics command within an optional argument. (You could create
% your own custom macro containing the \includegraphics command to make things
% simpler here.)
%\begin{IEEEbiography}[{\includegraphics[width=1in,height=1.25in,clip,keepaspectratio]{mshell}}]{Michael Shell}
% or if you just want to reserve a space for a photo:

%\begin{IEEEbiography}{Michael Shell}
%Biography text here.
%\end{IEEEbiography}

% if you will not have a photo at all:
%\begin{IEEEbiographynophoto}{John Doe}
%Biography text here.
%\end{IEEEbiographynophoto}

% insert where needed to balance the two columns on the last page with
% biographies
%\newpage

%\begin{IEEEbiographynophoto}{Jane Doe}
%Biography text here.
%\end{IEEEbiographynophoto}

% You can push biographies down or up by placing
% a \vfill before or after them. The appropriate
% use of \vfill depends on what kind of text is
% on the last page and whether or not the columns
% are being equalized.

%\vfill

% Can be used to pull up biographies so that the bottom of the last one
% is flush with the other column.
%\enlargethispage{-5in}

% Include (or not) the appendix
%\input{nm_app.tex}


\clearpage

\appendices

Here, we provide details of the proofs stated in the paper.

\section{Proof of \lref{thrm:convex_equivalence}}

To show the equivalence, we first establish that the mappings between $P$ and
$\pii$, and $A$ are one-to-one using the following lemma.
\begin{lemma}
    The mappings \eref{eq:definition_A}, \eref{eq:recover_pi} and
    \eref{eq:recover_P} between $P$ and $\pi_\infty$, and $A$ are one-to-one.
    \label{lemma:unique_mapping}
\end{lemma}
\begin{proof}
    Recall that $\pii$ has strictly positive elements.  Firstly, given $P$ and
    $\pii$, the equation \eref{eq:definition_A} yields a single $A$. Secondly,
    assume $\diag{\pi_\infty} P = A = \diag{\tilde \pi_\infty} \tilde P$, where
    $\tilde P$ and $P$ are row-stochastic.  Multiplying the equation by $\ones$
    from the right yields $\pi_\infty = \tilde \pi_\infty$. Then, multiplying
    from the left by $\diag{\pi_\infty}^{-1}$ yields $P = \tilde P$.

    Thirdly, equations \eref{eq:recover_pi} and \eref{eq:recover_P} yield
    unique $P$ and $\pii$ given an $A$. Fourthly, assume $A$ and $\tilde A$
    both map to $P$ and $\pii$, i.e., $A\ones = \pii = \tilde A \ones$, and
    $\diag{A\ones}^{-1} A = P = \diag{\tilde A\ones}^{-1} \tilde A$.
    Multiplying the last equation by $\diag{A\ones}$ yields $A = \tilde A$.
\end{proof}
Then we note that the cost functions are the same in the two formulations
\eref{eq:nl_matching} and \eref{eq:convex_mom} of the problem.  Secondly, we
map feasible solutions between the two problems.
\subsubsection{Solution of \eref{eq:nl_matching} $\Rightarrow$ Solution of
\eref{eq:convex_mom}}

Assume $P$ and $\pii$ are optimal for \eref{eq:nl_matching}, and define $A
= \diag{\pii} P$. Then
\begin{itemize}
    \item $P \geq 0, \pii \geq 0 \Rightarrow A \geq 0$,
    \item $\ones^T A \ones = \ones^T \diag{\pii} P \ones = \pii^T \ones = 1$,
    \item $A\ones = \diag{\pii} P \ones = \diag{\pii} \ones = \pii \in \polylb$,
    \item $\pii = P^T \pii \Rightarrow \diag{\pii} \ones = P^T \diag{\pii} \ones
        \overset{P\ones = \ones}{\Rightarrow} \diag{\pii} P \ones
        = [\diag{\pii}P]^T \ones \Rightarrow A \ones = A^T \ones$.
\end{itemize}

\subsubsection{Solution of \eref{eq:convex_mom} $\Rightarrow$ Solution of
\eref{eq:nl_matching}}

Assume $A$ is optimal for \eref{eq:convex_mom}. Let $\pii = A \ones$ and $P =
\diag{A\ones}^{-1} A$. Note that $\diag{A \ones}^{-1}$ is well-defined since $A
\ones \in \polylb$, i.e., $A \ones > 0$. Then
\begin{itemize}
    \item $A \geq 0 \Rightarrow \pii \geq 0 \text{ and } P \geq 0$,
    \item Since for any vector $x$ with all non-zero elements it holds that $[\diag{x}^{-1} x]_i
        = [\diag{x}^{-1}]_{ii} [x]_i = [x]_i^{-1} [x]_i = 1$, i.e.,
        $\diag{x}^{-1} x = \ones$, we have that $\ones = \diag{A\ones}^{-1}
        A \ones
        = P \ones$,
    \item $\ones^T A \ones = 1 \Rightarrow \ones^T \pii = 1$,
    \item $\pii = A \ones \in \polylb$,
    \item $A \ones = A^T \ones \Rightarrow \pii = A^T \ones$. Then, again
        employing that $\diag{x}^{-1} x = 1$ for any vector with all non-zero
        elements; $\pii = A^T \ones = A^T \diag{A\ones}^{-1} A \ones
        = [\diag{A\ones}^{-1} A]^T A \ones = P^T \pii$.
\end{itemize}

%\hfill$\square$

\section{Proof of \lref{thrm:convergence_P_pi}}

Here, we provide a sequence of lemmas that give the details on that 
\begin{enumerate}
    \item[A.] the estimate $\hat M_\infty$ converges to $M_\infty$,
    \item[B.] the solution $\hat A$ of the optimization problem
        \eref{eq:convex_mom} converges to the true parameter $A$,
    \item[C.] the solution $\hat A$ of the optimization problem
        \eref{eq:convex_mom} can be converted
        uniquely to $\hat P$ and $\hat \pi_\infty$.
\end{enumerate}

\subsection{Convergence of $\hat M_\infty$}

\begin{lemma}
    Let the sequence of $N$ observations from the HMM with initial
    distribution $\pi_0$ be the set $\{y_k\}_{k=0}^N$ and form the empirical
    estimate \eref{eq:moment_estimator} of the second order moments.  Then the
    estimate converges, i.e.,
    \begin{equation}
        \hat M_\infty \rightarrow B^T \diag{\pi_\infty} P B
    \end{equation}
    with probability one as the number of observations $N \rightarrow \infty$.
    \label{lemma:convergence_M}
\end{lemma}
Before we prove the above lemma, we first introduce two auxiliary lemmas.
\begin{lemma}
    Let $x_k$ be the state of an HMM and $y_k$ the corresponding observation.
    Then the process defined by the tuplet $(x_k, y_k, y_{k-1})$ is a Markov
    chain. 
    \label{lemma:hmm_as_mc}
\end{lemma}
\begin{proof}
    It can be checked that the Markov property is satisfied.
    \comment{
    Consider the state probability conditioned on the history:
    \begin{align}
        \Pr{&x_k = p, y_{k} = q, y_{k-1} = r | \notag \\
        & x_{k-1} = \xi_{k-1}, y_{k-1}
        = \eta_{k-1}, y_{k-2} = \eta_{k-2}, \dots, \notag \\
    & x_1 = \xi_1, y_1 = \eta_1, y_0 = \eta_0}  \notag \\[0.3cm]
    = \Pr{&y_{k} = q, y_{k-1} = r | x_k = p, \notag \\
        & x_{k-1} = \xi_{k-1}, y_{k-1}
        = \eta_{k-1}, y_{k-2} = \eta_{k-2}, \dots, \notag \\
        & x_1 = \xi_1, y_1 = \eta_1, y_0 = \eta_0} \quad \times  \notag \\
        \Pr{&x_k = p | \notag \\
        & x_{k-1} = \xi_{k-1}, y_{k-1}
        = \eta_{k-1}, y_{k-2} = \eta_{k-2}, \dots, \notag \\
        & x_1 = \xi_1, y_1 = \eta_1, y_0 = \eta_0}  \notag \\[0.3cm]
        = \Pr{&y_k = q | x_k = p}  \times \notag \\
        \Pr{&y_{k-1} = r | y_{k-1} = \eta_{k-1}} \Pr{x_k=p|x_{k-1}=\xi_{k-1}}  \notag \\[0.3cm]
        = \Pr{&y_k = q | x_k = p, x_{k-1} = \xi_{k-1}}  \times \notag \\
        \Pr{&y_{k-1} = r | y_{k-1} = \eta_{k-1}} \Pr{x_k=p|x_{k-1}=\xi_{k-1}}  \notag \\[0.3cm]
        = \Pr{&y_k = q, x_k = p | x_{k-1} = \xi_{k-1}}  \times \notag \\
        \Pr{&y_{k-1} = r | y_{k-1} = \eta_{k-1}} \notag \\[0.3cm]
        = \Pr{&y_k = q, x_k = p | x_{k-1} = \xi_{k-1}, y_{k-1}=\eta_{k-1}}  \times \notag \\
    \Pr{&y_{k-1} = r | y_{k-1} = \eta_{k-1}, x_{k-1}=\xi_{k-1}} \notag \\[0.3cm]
        =\Pr{&x_k = p, y_k = q, y_{k-1} = r | x_{k-1}=\xi_{k-1},
    y_{k-1}=\eta_{k-1} }  \notag \\[0.3cm]
        = \Pr{&x_k = p, y_k = q, y_{k-1} = r | \notag \\
            &x_{k-1}=\xi_{k-1},
y_{k-1}=\eta_{k-1}, y_{k-2}=\eta_{k-2} }. 
    \end{align}
    This shows that the Markov property is satisfied.
    }
\end{proof}
The above lemma allows us to recast the HMM into a Markov chain so that we can
leverage convergence results related to Markov chains. The following lemma
guarantees necessary properties of this new Markov chain:
\begin{lemma}
    If the transition and observation matrices of the HMM referred to in
    Lemma~\ref{lemma:hmm_as_mc} have all elements strictly positive, then the
    Markov chain $(x_k, y_k, y_{k-1})$ is irreducible and aperiodic.
    \label{lemma:ergodic_hmm_as_mc}
\end{lemma}
\begin{proof}
    The transition matrix of the lumped Markov chain consists of
    multiplications between elements of the $P$ and $B$ matrices (which are
    strictly positive) and zeros (whenever the common observation is not
    shared).
    
    It can be shown that any state can reach any state with positive
    probability in at most two steps ($\Rightarrow$ irreducibility).
    Furthermore, it can be shown that any state with the same two observations
    has positive probability of returning to itself in one step ($\Rightarrow$
    aperiodicity).
\end{proof}

With these two lemmas, we are now ready to prove \lref{lemma:convergence_M}.

\comment{
\begin{proof}[(Original) Proof of \lref{lemma:convergence_M}]
    \comment{
    We begin by calculating the expected value of the estimate.
    \begin{align}
        \mathbb{E}\{\hat M_\infty\} &= \mathbb{E}\Big\{ \frac{1}{N} \sum_{k=0}^{N-1}
    e_{y_k} e_{y_{k+1}}^T \Big\} \notag \\
    &= \frac{1}{N} \sum_{k=0}^{N-1} \mathbb{E}\{e_{y_k} e_{y_{k+1}}^T \}
    \notag \\
    &= \frac{1}{N} \sum_{k=0}^{N-1} B^T \diag{(P^T)^k \pi_0} PB \notag \\
    &= B^T \Bigg( \frac{1}{N} \sum_{k=0}^{N-1} \diag{(P^T)^k \pi_0} \Bigg) PB
    \notag \\
    &\rightarrow B^T \diag{\pi_\infty} PB,
    \end{align}
    as $N \rightarrow \infty$.  Here we employed that the expression in the big
    parenthesis is a C\'esaro sum. It is well-known that if the summand is
    convergent, then so is the C\'esaro mean \cite{wikipedia_cesaro_2016}. Since the vector $(P^T)^k \pi_0$
    will converge to $\pi_\infty$ as $k \rightarrow \infty$, under the
    aperiodicity and irreducibility assumptions on the transition matrix $P$,
    the whole expression inside the parenthesis will converge to
    $\diag{\pi_\infty}$, yielding the wanted expression. We also employed that
    \begin{align}
        \E{e_{y_k} e_{y_{k+1}}^T} &= \sum_{i=1}^Y \sum_{j=1}^Y \Pr{y_k = i,
    y_{k+1} = j} e_i e_j^T \notag \\
    &= \sum_{i=1}^Y \sum_{j=1}^Y [M_k]_{ij} e_i e_j^T \notag \\
    &= M_k \notag \\
    &= B^T \diag{(P^T)^k \pi_0} PB,
   \end{align}
   where \eref{eq:2nd_moments} was used. 
   }
    
   Consider the limit of the $(i,j)$th element of the estimate:
   \begin{align}
        \lim_{N\to \infty} [\hat M_\infty]_{ij} &= \lim_{N\to \infty} \frac{1}{N} \sum_{k=0}^{N-1}
        \ind{y_k = i, y_{k+1} = j} \notag \\
        &= \lim_{N\to \infty} \sum_{l=1}^X \frac{1}{N} \sum_{k=0}^{N-1} \ind{y_k = i, y_{k+1} = j,
        x_{k+1} = l} \notag \\
        &= \sum_{l=1}^X \lim_{N\to \infty} \frac{1}{N} \sum_{k=0}^{N-1} \ind{y_k = i, y_{k+1} = j,
        x_{k+1} = l} \notag \\
        &\rightarrow \sum_{l=1}^X [\tilde \pi_\infty]_{(i,j,l)} \notag \\
        &= \sum_{l=1}^X \lim_{k \to \infty} \Pr{y_k = i, y_{k+1} = j, x_{k+1} = l} \notag \\
        &= \lim_{k \to \infty} \sum_{l=1}^X \Pr{y_k = i, y_{k+1} = j, x_{k+1} = l} \notag \\
        &= \lim_{k \to \infty} \Pr{y_k = i, y_{k+1} = j} \notag \\
        &=\textcolor{gray}{\lim_{k \to \infty} [M_k]_{ij}} \notag \\
        &=\textcolor{gray}{\lim_{k \to \infty} [B^T \diag{(P^T)^k \pi_0} PB]_{ij}} \notag \\
        &= [B^T \diag{\pi_\infty} PB]_{ij} \notag \\
        &= [M_\infty]_{ij},
   \end{align}
    with probability one, where $\tilde \pi_\infty$ denotes the stationary
    distribution of the Markov chain $(y_k, y_{k+1}, x_{k+1})$. Here we
    employed a law of large numbers for Markov chains to say that the empirical
    average of the indicator function converges to the stationary distribution \cite[Theorem
    14.2.53]{cappe_inference_2005}, and that the state distribution converges to the
    stationary distribution, regardless of the initial distribution \cite[Proposition
    14.1.11]{cappe_inference_2005}. For this, we need the chain $(y_k, y_{k+1},
    x_{k+1})$ to be irreducible and aperiodic, which it is according to
    \lref{lemma:ergodic_hmm_as_mc}.
\end{proof}
}

\begin{proof}[Proof of \lref{lemma:convergence_M}]
    Denote the state of the lumped Markov chain as $z_k = (y_k, y_{k+1},
    x_{k+1})$ and let the functions $f_{ij}$ be given by
    \begin{equation}
        f_{ij}(z_k) = \ind{y_k = i, y_{k+1} = j}.
    \end{equation}
    Then,
    \begin{align*}
        \lim_{N\rightarrow\infty} [\hMii]_{ij} &= \lim_{N\to \infty} \frac{1}{N} \sum_{k=0}^{N-1}
        \ind{y_k = i, y_{k+1} = j} \\
        &= \lim_{N\to \infty} \frac{1}{N} \sum_{k=0}^{N-1} f_{ij}(z_k) \\
        &\rightarrow \sum_{z \in \mathcal{Z}} \tilde \pi_\infty(z) f_{ij}(z) \\
        &= \sum_{x\in\mathcal{X}}\tilde \pi_\infty((i,j,x)) \\
        &= \sum_{x\in\mathcal{X}} \lim_{k\to\infty} \Pr{(y_k, y_{k+1}, x_{k+1})
        = (i,j,x)} \\
        &= \lim_{k\to\infty} \Pr{(y_k, y_{k+1}) = (i,j)} \\
        &= [B^T \diag{\pi_\infty} PB]_{ij} \\
        &= [M_\infty]_{ij},
    \end{align*}
    with probability one, where $\mathcal{Z} = \mathcal{Y} \times \mathcal{Y}
    \times \mathcal{X}$ and $\tilde \pi_\infty$ is the stationary distribution of
    $z_k$. Here, we used that $z_k$ is aperiodic and irreducible
    (\lref{lemma:ergodic_hmm_as_mc}) and the strong law of large numbers for
    Markov chains \cite[Theorem 14.2.53]{cappe_inference_2005}.
\end{proof}

\subsection{Convergence of optimization solution}

As guaranteed by \lref{lemma:convergence_M}, the estimate $\hat M_\infty$ will
converge to $M_\infty$ with probability one. The next step is to
show that the solution(s) of the optimization problem \eref{eq:convex_mom}
converges to the optimal value, i.e., that $\hat A \rightarrow A$, as $\hat
M_\infty \rightarrow M_\infty$. The following lemma guarantees that there is
a unique solution to the optimization problem and allows us to write \emph{the}
minimizer from here on.
\begin{lemma}
    Under the assumption that $B$ has full rank, the minimizer of the
    optimization problem \eref{eq:convex_mom} is unique. 
    \label{lemma:unique_minimizer}
\end{lemma}
\begin{proof}
    We check that the Hessian of the cost function in \eref{eq:convex_mom} is
    positive definite to ensure strict convexity (see, e.g.,
    \cite{boyd_convex_2004}). Note that the cost is of the form (see equation
    \eref{eq:vectorization_of_cost} below for details)
    \begin{align}
        f(x) &= \| Qx + q \|_2^2 \notag \\
             &= x^T Q^T Q x + 2 q^T Q x + q^T q,
            \label{eq:quad_cost}
    \end{align}
    where $Q = B^T \otimes B^T$, which has the Hessian
    \begin{equation}
        \nabla^2 f(x) = 2 Q^T Q.
    \end{equation}
    Positive definiteness of the Hessian is, in this case, equivalent to
    \begin{align}
        x^T \big[\nabla^2 f(x)\big] x > 0 &\quad\forall x \neq
        0 & \Leftrightarrow \notag \\
        x^T \big[2Q^T Q] x > 0 &\quad\forall x \neq 0 & \Leftrightarrow \notag \\
        2 (Qx)^T (Q x) > 0 &\quad\forall x \neq 0 & \Leftrightarrow \notag \\
        2 \| Qx \|^2 > 0 &\quad\forall x \neq 0 & \Leftrightarrow \notag \\
        Qx \neq 0 &\quad\forall x \neq 0 & \Leftrightarrow \notag \\
        \ker Q  = \{0&\}. &
        \label{eq:strict_convex}
    \end{align}
    Since (see, e.g., \cite{horn_topics_1991}) % Problem 4.2.21
    \begin{equation}
        \rank{B^T \otimes B^T} = \rank{B^T}\rank{B^T} = X^2,
    \end{equation}
    we see that the $Y^2 \times X^2$ matrix $B^T \otimes B^T$ has full column
    rank. By \eref{eq:strict_convex}, this implies uniqueness since, then, the
    cost function is strictly convex.
\end{proof}

The next lemma says that the sequence of minimizers of the approximate
optimization problems will converge to the minimizer of the true optimization
problem.
\begin{lemma}
    Let $\hat A$ be the minimizer of the optimization problem
    \eref{eq:convex_mom} using $\hat M_\infty$ from equation
    \eref{eq:moment_estimator}, and let $A$ be the minimizer of the optimization problem
    \eref{eq:convex_mom} using instead the true $M_\infty$.
    Then $\hat A$ converges with probability one to $A$ as $\hat M_\infty$ tends to
    $M_\infty$ as in \lref{lemma:convergence_M}.
    \label{lemma:epigraphical_convergence}
\end{lemma}
To be able to prove Lemma~\ref{lemma:epigraphical_convergence}, we will make
use of another two additional lemmas. The first one provides results regarding
how well a minimizer of an approximate cost function is with respect to the
minimizer of the true cost function. In summary, it says that if we have
uniform convergence of the cost function, the minimizer of the approximate cost
function will converge to the minimizer of the true cost function, if the
parameter set is compact.
\begin{lemma}
    Consider a family of random functions $f_k(\theta) : \Theta \to \mathbb{R}$, where
    $\Theta$ is a compact subset of some Euclidean space. Let $\theta_k
    = \argmin_{\theta \in \Theta} f_k(\theta)$. If $f_k(\theta)$ tends
    uniformly to a continuous (on $\Theta$) deterministic limit $\bar
    f(\theta)$ with probability one, i.e.,
    \begin{equation}
        \sup_{\theta \in \Theta} | f_k(\theta) - \bar f(\theta) | \to 0
        \label{eq:limit_assumption}
    \end{equation}
    with probability one as $k \to \infty$, then with $\Theta^* = \{ \theta \in \Theta \text{ s.t. } \theta
            \text{ minimizes } \bar f(\theta) \}$, we have that
            \begin{equation}
                \inf_{\theta \in \Theta^*} \| \theta_k - \theta \| \to 0
            \end{equation}
            with probability one as $k \to \infty$.
    \label{lemma:convergence_minimizers}
\end{lemma}
\begin{proof}
    This follows from Lemma 1.1 of \cite{campi_system_2006} by restricting to
    the realizations where \eref{eq:limit_assumption} hold. Similar results also appear in
    \cite{pollard_asymptotics_1993} and \cite[Ch. 24]{gourieroux_statistics_1995}.
\end{proof}
The second auxiliary lemma needed states roughly that if we have pointwise
convergence with probability one of a sequence of convex functions, then we
also have uniform convergence over compact sets with probability one.
\begin{lemma}
    Suppose $f_k(\theta)$ is a sequence of convex random functions defined on
    an open convex set $\mathcal{S}$ of some Euclidean space, which converges
    pointwise in $\theta$ with probability one to some $\bar f(\theta)$.
    Then
    \begin{equation}
        \sup_{\theta \in \Theta} | f_k(\theta) - \bar f(\theta) |
        \label{eq:sup_fk_f}
    \end{equation}
    tends to zero with probability one as $k \to \infty$, for each compact subset
    $\Theta$ of $\mathcal{S}$.
    \label{lemma:pointwise_to_uniform}
\end{lemma}
\begin{proof}
    This lemma follows from \cite[Theorem 10.8]{rockafellar_convex_1970} by restricting to
    the set of realizations where pointwise convergence holds, which has probability one
    by assumption.
\end{proof}
Combining these two lemmas allows us to provide proof for
Lemma~\ref{lemma:epigraphical_convergence}.
\begin{proof}[Proof of Lemma \ref{lemma:epigraphical_convergence}]
    The cost function in problem \eref{eq:convex_mom} is strictly convex (see
    proof of Lemma~\ref{lemma:unique_minimizer}). The set
    of feasible parameters is compact and convex. From
    \lref{lemma:convergence_M}, we know that $\hat M_\infty$ converges with
    probability one. Since the cost function is a continuous mapping of $\hat
    M_\infty$, we conclude that the cost function converges pointwise with
    probability one. Hence, the conditions of
    Lemma~\ref{lemma:pointwise_to_uniform} are fulfilled.  This in turn
    fulfills the conditions of Lemma~\ref{lemma:convergence_minimizers} which
    allows us to conclude that $\hat A$ will tend to $A$ with probability one.
\end{proof}

\subsection{Convergence of $\hat P$ and $\hat \pi_0$}

From Lemma~\ref{lemma:epigraphical_convergence}, we know that $\hat
A \rightarrow A$ as the number of samples tends to infinity. Since the mapping
from $A$ to $P$ and $\pii$ is unique (see \lref{lemma:unique_mapping}), we
conclude that the estimates of $P$ and $\pii$ also will converge. In summary,
this concludes the proof of \thrmref{thrm:convergence_P_pi}.

\section{Proof of \thrmref{thrm:as_eff}}

Parts of the proof are inspired by \cite{kontorovich_learning_2013}.

\subsubsection{Central limit theorem for $\hMii$}

We will show that a central limit theorem holds for the estimates. For this, we
employ the following theorem from \cite{jones_markov_2004}:
\begin{theorem}[Corollary 5 of \cite{jones_markov_2004}]
    Consider a uniformly ergodic Markov chain on $\mathcal{X}$ with stationary
    distribution $\pi_\infty$. Suppose $\mathbb{E}_{\pi_\infty} f^2(x)
    < \infty$, where $f:\mathcal{X}\to\mathbb{R}$. Then for any initial distribution, as $N \to \infty$,
    \begin{equation}
        \sqrt{N}(\bar f_N - \mathbb{E}_{\pi_\infty} f) \to_d \mathcal{N}(0,
        \sigma^2_f),
    \end{equation}
    where $\bar f_N = N^{-1} \sum_{k=1}^N f(x_k)$ and $\sigma_f^2 < \infty$ is
    a constant.
    \label{thrm:general_clt}
\end{theorem}

As in the proof of \lref{lemma:convergence_M}, denote the state of the lumped
Markov chain as $z_k = (y_k, y_{k+1}, x_{k+1}) \in \mathcal{Z}
= \mathcal{Y}\times\mathcal{Y}\times\mathcal{X}$ and let the functions $f_{ij}$
be given by
\begin{equation}
    f_{ij}(z_k) = \ind{y_k = i, y_{k+1} = j}.
\end{equation}
Then, as guaranteed by \thrmref{thrm:general_clt} ($z_k$ is uniformly ergodic
since it is finite -- see \cite[Example 1]{jones_markov_2004}),
\begin{equation}
    \sqrt{N} \Bigg( \frac{1}{N} \sum_{k=0}^{N-1} f_{ij}(z_k)
    - \sum_{z\in\mathcal{Z}} \tilde \pi_\infty(z) f_{ij}(z) \Bigg) \to_d
    \mathcal{N}(0, \sigma_{ij}^2),
\end{equation}
or by changing back to the original variables,
\begin{align}
    \sqrt{N} \Bigg( \frac{1}{N} &\sum_{k=0}^{N-1} \ind{y_k = i, y_{k+1} = j}
    \notag \\
    &- \lim_{k\to\infty} \Pr{y_k = i, y_{k+1} = j} \Bigg) \notag \\
    &\to_d \mathcal{N}(0, \sigma_{ij}^2),
\end{align}
i.e.,
\begin{equation}
    \sqrt{N} \Big( [\hMii]_{ij} - [M_\infty]_{ij} \Big) \to_d \mathcal{N}(0, \sigma_{ij}^2),
    \label{eq:hMii_convergence_in_p}
\end{equation}
where $\sigma_{ij}^2 < \infty$ are constants.

\subsubsection{$\sqrt{N}$-consistency of $\hat M_\infty$} We now establish that
$\hMii$ is a $\sqrt{N}$-consistent estimator using the above result and the
following lemma.
\begin{lemma}[Appendix A, \cite{pollard_convergence_1984}]
    If a sequence of random variables $Z_N$ and a constant $z_0$ tend in
    distribution to another random variable $Z$ (as $N \to \infty$) according
    to
    \begin{equation}
        \sqrt{N}(Z_N - z_0) \to_d Z,
    \end{equation}
    then
    \begin{equation}
        Z_n - z_0 = \Op(N^{-1/2}).
    \end{equation}
\end{lemma}
Leveraging the above lemma, we conclude that
\begin{equation}
    [\hMii]_{ij} = [M_\infty]_{ij} + \mathcal{O}_p(N^{-1/2}).
    \label{eq:ordo_p_theta}
\end{equation}
This, by definition, means that for every $\varepsilon > 0$, we can find
a constant $c_{ij}(\varepsilon)$ such that for all $N$ sufficiently large,
\begin{equation}
    \PR{\sqrt{N} \big| [\hMii]_{ij} - [M_\infty]_{ij} \big| > c_{ij}(\varepsilon)} < \varepsilon.
\end{equation}

\comment{
\subsubsection{(Original) $\sqrt{N}$-consistency of $\hat M_\infty$} We now establish that
$\hMii$ is a $\sqrt{N}$-consistent estimator using the above result.

\begin{theorem}[Slutsky's theorem]
    % Convergence of Stochastic Processes, David Pollard. p. 62 (alt. App. A)
    If a sequence of random variables $X_n$ tends in distribution to another
    random variable $X$, i.e., $X_n \to_d X$, then $X_n = \mathcal{O}_p(1)$.
    \label{thrm:slutsky}
\end{theorem}

\comment{
In our setting, we employ \thrmref{thrm:slutsky} for an estimator $\hat \theta$
that follows a central limit theorem,
\begin{equation}
    \sqrt{N}(\hat \theta - \theta_0) \to_d \mathcal{N}(0, \sigma^2),
\end{equation}
i.e., $\hMii$ as in equation \eref{eq:hMii_convergence_in_p}, to conclude that
\begin{equation}
    \hat \theta = \theta_0 + \mathcal{O}_p(N^{-1/2}).
    \label{eq:ordo_p_theta}
\end{equation}
}
We employ \thrmref{thrm:slutsky} in conjunction with equation
\eref{eq:hMii_convergence_in_p} to conclude that 
\begin{equation}
    \sqrt{N} \Big( [\hMii]_{ij} - [M_\infty]_{ij} \Big) = \mathcal{O}_p(1),
\end{equation}
or equivalently that
\begin{equation}
    [\hMii]_{ij} = [M_\infty]_{ij} + \mathcal{O}_p(N^{-1/2}).
    \label{eq:ordo_p_theta}
\end{equation}
This, by definition, means that for every $\varepsilon > 0$, we can find
a constant $c_{ij}(\varepsilon)$ such that for $N$ sufficiently large,
\begin{equation}
    \PR{\sqrt{N} \big| [\hMii]_{ij} - [M_\infty]_{ij} \big| > c_{ij}(\varepsilon)} < \varepsilon.
\end{equation}
}

\subsubsection{$\sqrt{N}$-consistency of $\hat A$}
\label{ssec:sqrtN_consistencyA}
We now propagate the
$\sqrt{N}$-consistency of $\hMii$ to the variable $\hat A$ through the
optimization problem \eref{eq:convex_mom}.

First note that problem \eref{eq:convex_mom} can be rewritten on the standard
form for a \emph{quadratic program} (QP),
\begin{align}
    \min_x \quad & \frac{1}{2} x^T Q x - q^T x \notag \\
    \text{s.t.} \quad & Gx \leq g,  \notag \\
                      & Dx = d,
    \label{eq:nominal_QP}
\end{align}
where $Q$ is a positive definite matrix. In particular, using the
identity (for arbitrary matrices $A$, $B$ and $C$ of appropriate dimensions, see, e.g.,
\cite{horn_topics_1991}) % p. 267 and Lemma 4.3.1
\begin{equation}
    \vec{ABC} = (C^T \otimes A) \vec{B},
\end{equation}
we have that 
\begin{align*}
    \| \hat M_\infty &- B^T A B \|_F^2 = \| \vec{\hMii - B^T A B} \|_2^2 \\
                                      &= \| \vec{\hMii} - \vec{B^T AB} \|_2^2 \\
                                      &= \| \vec{\hMii} - (B \otimes B)^T \vec{A} \|_2^2 \\
                                      &= \vec{\hMii}^T \vec{\hMii} \\
                                      &- 2 \vec{\hMii}^T (B \otimes B)^T \vec{A} \\
                                      &+ \vec{A}^T (B \otimes B) (B
    \otimes B)^T \vec{A}, \stepcounter{equation}\tag{\theequation}
    \label{eq:vectorization_of_cost}
\end{align*}
so that,
\begin{align}
    Q &= 2 (B \otimes B) (B \otimes B)^T, \\
    \hat q &= 2 (B \otimes B) \vec{\hMii},
\end{align}
where $x = \vec{A}$, and the constant term has been dropped. The constraints can
similarly be translated by vectorization, e.g., $\ones^T A \ones = 1$
translates to $\ones^T \vec{A} = 1$, i.e., $\ones^T x = 1$.

The uncertainty in this problem, resulting from the estimation procedure, lies
in the estimate of the moments $\hMii$. Note that this only influences the cost
function -- not the constraints. We now ask ourselves how the uncertainty in
$\hMii$ propagates through the QP into our variable of interest $\hat A$.

Denote the minimizer of the nominal problem \eref{eq:nominal_QP}, where
$M_\infty$ is used instead of $\hMii$, as $x^*$ (= $\vec{A}$) and let
\begin{equation}
    \hat x^* = \argmin_x \frac{1}{2} x^T Q x - \hat q^T x,
\end{equation}
subject to the same constraints as in problem \eref{eq:nominal_QP}. Then
\cite[Theorem 2.1]{daniel_stability_1973} provides the following bound on the distance
between the solution of the nominal QP and the solution of the perturbed QP:
\begin{equation}
    \|x^* - \hat x^*\|_2 \leq \frac{\delta}{\lambda - \delta} (1
    + \|x^*\|_2),
    \label{eq:qp_bound}
\end{equation}
where $\delta = \|q - \hat q\|_2$ and $\lambda$ is the smallest eigenvalue
of $Q$.\footnote{$\lambda > \delta$ holds if $N$ is large
    enough (so that $\delta$ is small enough).}

Let $\sigma_1(\cdot)$ denote the largest singular value, then we note that (for
every $\varepsilon > 0$)
\begin{align*}
    \delta &= \|q-\hat q\|_2 \notag \\
                   &= \| 2(B\otimes B) \vec{M_\infty} - 2(B\otimes B) \vec{\hMii} \|_2 \\
                   &= 2\, \| (B\otimes B) (\vec{M_\infty} - \vec{\hMii}) \|_2 \\
                   &\leq 2\,\sigma_1(B \otimes B) \; \|\vec{M_\infty} - \vec{\hMii}\|_2 \\
                   &\leq 2\,\sigma_1(B \otimes B) \; \|\vec{M_\infty} - \vec{\hMii}\|_1 \\
                   &= 2\,\sigma_1(B \otimes B) \; \sum_{i,j\in\mathcal{Y}}\big| [M_\infty]_{ij} - [\hMii]_{ij}\big| \\
                   &\leq 2\,\sigma_1(B \otimes B) \; \sum_{i,j\in\mathcal{Y}} \frac{c_{ij}(\varepsilon)}{\sqrt{N}} \\
                   &\leq \frac{1}{\sqrt{N}} \; 2 \, \sigma_1(B \otimes B) \; Y^2 \max_{i,j} c_{ij}(\varepsilon) \\
                   &\overset{\text{def.}}{=} \frac{1}{\sqrt{N}} \; K(\varepsilon),
                    \stepcounter{equation}\tag{\theequation}
\end{align*}
with probability greater than $1-\varepsilon$, where $c_{ij}(\varepsilon)$ are
the constants in the stochastic order \eref{eq:ordo_p_theta}. Also note that
\begin{equation}
    \|x^*\|_2 = \| \vec{A} \|_2 \leq \| \ones_{X^2} \|_2 = \sqrt{X^2} = X,
\end{equation}
due to the sum-to-one constraint of $A$.

Hence, for every $\varepsilon > 0$ (and $N$ sufficiently large), we have in the
bound \eref{eq:qp_bound}, that
\begin{align}
    \|A - \hat A\|_F &= \|\vec{A} - \vec{\hat A}\|_2 \notag \\
                     &= \|x^* - \hat x^*\|_2 \notag \\
                     &\leq \frac{\delta}{\lambda - \delta} (1 + \|x^*\|_2) \notag \\
                     &\leq \frac{\delta}{\lambda} \; (1 + \|x^*\|_2) \notag \\
                     &\leq \frac{1}{\sqrt{N}} \; \frac{K(\varepsilon) \; (1
+ X)}{\lambda}
    \label{eq:A_hatA_bound}
\end{align}
with probability greater than $1 - \varepsilon$. This shows that\comment{\footnote{Note
    that for $A \in \mathbb{R}^{X\times Y}$, $\|A\|_F \leq M \Rightarrow |a_{ij}| \leq M$ and $|a_{ij}| \leq M \Rightarrow
\|A\|_F \leq \sqrt{XY} M$, so that a stochastic bound on the Frobenius norm is
equivalent to a stochastic elementwise bound.}}
\begin{equation}
    \hat A = A + \mathcal{O}_p(N^{-1/2}).
\end{equation}

\subsubsection{$\sqrt{N}$-consistency of $\hat \pi_\infty$ and $\hat P$}
\label{ssec:error_propagation}

Again, for any $\varepsilon > 0$, we have using equation \eref{eq:recover_pi} that
\begin{align}
    \|\pi_\infty - \hat \pi_\infty \|_2 &= \| A\ones - \hat A \ones \|_2 \notag \\
                                        &= \| (A - \hat A) \ones \|_2 \notag \\
                                        &\leq \Big(\max_{\|y\|_2 = 1} \|(A-\hat A)y\|_2\Big) \|\ones\|_2 \notag \\
                                        &\leq \|A-\hat A\|_F \|\ones\|_2 \notag \\
                                        &= \|A-\hat A\|_F \sqrt{X} \notag \\
                                        &\leq \frac{1}{\sqrt{N}} \; \frac{K(\varepsilon) \; (1
+ X)\sqrt{X}}{\lambda}
\label{eq:pi_hatPi_bound}
\end{align}
holds with probability greater than $1-\varepsilon$.

Continuing, equations \eref{eq:recover_pi} and \eref{eq:recover_P} tell us that
\begin{equation}
    \|P - \hat P\|_F = \|\diag{A \ones}^{-1} A - \diag{\hat A \ones}^{-1} \hat A\|_F.
\end{equation}
We will use that, for two invertible diagonal matrices $D_1$ and $D_2$, and
arbitrary matrices $X$ and $Y$, it holds that
\begin{align}
    \|D_1^{-1}X &- D_2^{-1}Y\|_F = \|D_1^{-1}D_2^{-1} (D_2 X - D_1 Y) \|_F \notag \\
                                &\leq \|D_1^{-1} D_2^{-1}\|_F \| D_2 X - D_1 Y \|_F \notag \\
                                &\leq \|D_1^{-1}\|_F \|D_2^{-1}\|_F \| D_2 X - D_1 Y + D_1 X - D_1 X \|_F \notag \\
                                &= \|D_1^{-1}\|_F \|D_2^{-1}\|_F \| (D_2 - D_1) X + D_1 (X-Y) \|_F \notag \\
                                &\leq \|D_1^{-1}\|_F \|D_2^{-1}\|_F \notag \\
                                &\quad\quad \times \Big( \|
    D_2 - D_1\|_F \|X\|_F + \|D_1\|_F \|X-Y \|_F \Big). 
\end{align}
This yields
\begin{align}
    \|P - \hat P\|_F \leq \|&\diag{A \ones}^{-1}\|_F \|\diag{\hat A \ones}^{-1}\|_F \notag \\
    &\times \Big( \|\hat A \ones - A \ones\|_2 \|A\|_F + \|A\ones\|_2 \|A - \hat A\|_F \Big),
\end{align}
where the first factor is bounded
by a constant due to the ergodicity assumptions (the stationary distribution
has strictly positive elements) and the terms in the parenthesis have trivial
bounds, or can be bounded using equations \eref{eq:A_hatA_bound} and
\eref{eq:pi_hatPi_bound}. Hence, for any
$\varepsilon > 0$, we have that $\|P - \hat P\|_F
\leq \frac{\text{constant}}{\sqrt{N}}$ with probability greater than
$1-\varepsilon$, or equivalently that,
\begin{equation}
    \hat P = P + \mathcal{O}_p(N^{-1/2}).
    \label{eq:Op_P}
\end{equation}

\subsubsection{$\Delta$-method}

Assume that the parametrization of the transition matrix is continuous and differentiable,
and denote by $\theta^*$ the true parameters. We can then propagate relation
\eref{eq:Op_P} to the parameters $\theta$ to obtain
\begin{equation}
    \hat \theta_\text{MM} = \theta^* + \mathcal{O}_p(N^{-1/2}),
\end{equation}
using the $\Delta$-method -- in particular, the first part of the proof of Theorem 3.1 in
\cite{vaart_asymptotic_1998}.  

\subsubsection{Regularity of the likelihood function}
\label{ssec:regularity_loglik}
Denote the Fisher information matrix as $I_F(\theta^*)$. Then Theorems
12.5.5 and 12.5.6 of \cite{cappe_inference_2005} guarantee that we have
a central limit theorem for the score function and a law of large numbers for
the observed information matrix, as follows:
\begin{align}
    N^{-1/2} \nabla_\theta l_N(\theta^*) &\rightarrow_d \mathcal{N}(0,
    I_F(\theta^*), \\
    N^{-1} \nabla_\theta^2 l_N(\theta^*) &\rightarrow_p -I_F(\theta^*),
\end{align}
as $N \rightarrow \infty$, since our chain is finite and $P, B > 0$.

\comment{
We need to verify the assumptions of the theorems:
\begin{itemize}
    \item[12.0.1] (i) and (ii) follows from that we consider finite HMMs
        (existence of transition/observation densities).
        (iii) from the assumption of $P > 0$ and $B > 0$.
    \item[12.2.1] (i) from $P > 0$. (ii) from $B > 0$.
    \item[12.3.1] follows from finite HMM. for $b^+$, $g_\theta \leq 1$, and
        $b^- > 0$ due to $B > 0$, so $\log b^-$ is bounded.
    \item[12.5.1] the parametrization is direct (every element of $P$ is one
        element of $\theta$). (i) are ok by direct, or exponential parameterization
        (composition of continuous functions) -- everything is infinitely
        differentiable. (ii) and (iii) note that $\nabla_\theta \log q_\theta
        \sim \frac{1}{q_\theta} \nabla_\theta q_\theta$ and since $q_\theta
        > 0$ and $\nabla_\theta q_\theta = 1$ (direct parametrization), it is
        ok. Same goes for the second derivative. (iv) all
        elements of $g_\theta(x,y) = B_{x,y}$ are bounded by 1. 
\end{itemize}
}

\subsubsection{Asymptotic efficiency by Newton-Raphson}

\begin{lemma}
    Let $\thetainit = \theta^* + \mathcal{O}_p(N^{-1/2})$. Then, one
    Newton-Raphson step starting in $\thetainit$ on $l_N(\theta)$ gives an
    asymptotically efficient estimator, i.e., with 
    \begin{equation}
        \hat \theta_\text{NR} = \thetainit - \big[\nabla_\theta^2 l_N(
        \thetainit)\big]^{-1} \; \nabla_\theta l_N(\thetainit),
    \end{equation}
    we get
    \begin{equation}
        \sqrt{N}(\hat \theta_\text{NR} - \theta^*) \rightarrow_d
        \mathcal{N}(0, I_F^{-1}(\theta^*)).
    \end{equation}
    \label{lemma:newton_efficient}
\end{lemma}
(Proof on next page)
\clearpage
\begin{proof}
    We have that
    \begin{align}
        \sqrt{N}(\thetaNR - \theta^*) &= \sqrt{N}(\thetainit - \theta^*)
        - \sqrt{N} \big[\hesslik(\thetainit) \big]^{-1} \gradlik(\thetainit)
        \notag \\
        &= \sqrt{N}(\thetainit - \theta^*) - N^{-1/2} \big[ -\Fisher(\theta^*)
    + \op(1) \big]^{-1} \big[ \gradlik(\theta^*)
        + \hesslik(\theta^*)(\thetainit - \theta^*)
    + \op(1) \big] \notag \\
    &= \sqrt{N}(\thetainit - \theta^*) - \big[ - \Fisher(\theta^*)
+ \op(1)\big]^{-1} \Big[ N^{-1/2} \gradlik(\theta^*)
    + \sqrt{N}[-\Fisher(\theta^*) + \op(1)](\thetainit - \theta^*) + \op(1)
\Big] \notag \\
&= \sqrt{N}(\thetainit - \theta^*) + \Fisher^{-1}(\theta^*) \Big[ N^{-1/2}
\gradlik(\theta^*) - \Fisher(\theta^*) \sqrt{N} (\thetainit - \theta^*) \Big]
+ \op(1) \notag \\
&= \Fisher^{-1}(\theta^*) N^{-1/2} \gradlik(\theta^*) + \op(1) \notag \\
&\to_d \mathcal{N}(0, \Fisher^{-1}(\theta^*) \Fisher(\theta^*)
\Fisher^{-T}(\theta^*) \notag \\
&= \mathcal{N}(0, \Fisher^{-1}(\theta^*).
    \end{align}
\end{proof}

\lref{lemma:newton_efficient}, together with the results of
subsections~\ref{ssec:error_propagation} and \ref{ssec:regularity_loglik}, conclude
the proof of \thrmref{thrm:as_eff} by taking $\thetainit = \hat \theta_{MM}$.

\comment{
\section{Random Systems}

Details on how the random systems were generated?
}


% that's all folks
\end{document}


