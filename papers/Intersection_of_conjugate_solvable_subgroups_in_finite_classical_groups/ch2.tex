\chapter{Definitions and preliminaries}

\section{Notation and basic definitions}
\label{secnot}

All group actions we use are right actions. For example, the action of a linear transformation $g$ of a vector space $V$ on $v \in V$ is $(v)g \in V.$

%If $G$ is a finite group, then ${\bf F}(G)$ is its {\bf Fitting subgroup}, which is the unique maximal normal nilpotent subgroup.

We write $GL(V)=GL(V, \mathbb{F})$ for the {\bf general linear group}, which is the group of all invertible linear transformations of  a vector space $V$ over a field $\mathbb{F}.$ 

 Let $p$ be a prime and $q=p^f$,  $f \in \mathbb{N}$. Denote a finite field of size $q$ by $\mathbb{F}_q$, its algebraic closure by $\overline{\mathbb{F}_q}$ and the multiplicative group of $\mathbb{F}_q$ by $\mathbb{F}_q^*.$ 
Throughout, unless stated otherwise, $V=\mathbb{F}_{q^{\bf u}}^n$ denotes a vector space of dimension $n$ over $\mathbb{F}_{q^{\bf u}}$ with ${\bf u} \in \{1,2\}.$  



We reserve the letter $\beta$ for a {\bf basis} of $V$. A basis is an ordered set. Let $\beta =\{v_1, \ldots, v_n\}$ be such a basis and let ${\bf u}=1$. If $g \in GL(V)$, then $g_{\beta}$ denotes the $n \times n$ matrix such that 
$$(v_i)g=\sum_{j=1}^{n} (g_{\beta})_{i,j} \cdot v_j.$$ Here $A_{i,j}$ is the $(i,j)$ entry of a matrix $A$.   We denote the group $\{g_{\beta} \mid g \in GL(V)\}$ by $GL_n(q, \beta)$ or simply $GL_n(q)$ when $\beta$ is understood. If $X \subseteq GL(V)$, then $X_{\beta}$ is $\{g_{\beta} \mid g \in X \}.$  It is easy to see that  $GL(V)$ and $GL_n(q)$ are isomorphic, and the map $g \mapsto g_{\beta}$ is an isomorphism. Since matrices from $GL_n(q)$ act on $V=\mathbb{F}_q^n$ by right multiplication, we  refer to them as linear transformations. In what
follows, we make no essential distinction between the groups $GL(V)$ and $GL_n(q)$ and use $GL(V)$ or $GL_n(q)$ depending on which one is more suitable for our purpose.  



%\medskip

We fix the following notation.
\medskip\\
\begin{longtable}{ p{7em} p{27em} }
${\bf F}(G)$ & {\bf Fitting subgroup} \index{Fitting subgroup} of a finite group $G$ (unique maximal \\ & normal  nilpotent subgroup);\\
$O_{\pi}(G)$ & unique maximal normal $\pi$-subgroup for a set of primes $\pi$;\\
$Z(G)$ & {\bf center} of a group $G$;\\
$g^G$ & conjugacy class of $g \in G$;\\
$A \rtimes B$ & semidirect product of groups $A$ and $B$ with $A$ normal;\\
 $\Sym(n)$ &  symmetric group of degree $n$; \\
$\sgn(\pi)$ &  {\bf sign} of a permutation $\pi$;\\
$M_n(\mathbb{F})$& algebra of all $n \times n$ matrices over $\mathbb{F}$;\\
 $\diag(\alpha_1, \ldots, \alpha_n)$ &  {\bf diagonal} matrix with entries $\alpha_1, \ldots, \alpha_n$ on its diagonal;  \\ 
 $\diag(\alpha, \ldots, \alpha)$ &  {\bf scalar} matrix, or simply a {\bf scalar};  \\  
 $\diag[g_1, \ldots, g_k]$ &  {\bf block-diagonal} matrix with blocks $g_1, \ldots, g_k$ on its diagonal;  \\
$\per(\sigma)$ & {\bf permutation matrix} corresponding to $\sigma \in \Sym(n);$\\
$g^{\top}$ & {\bf transpose} of a matrix $g$;\\
$\det(g)$ &  {\bf determinant} of a matrix $g$;\\
$\Det(H)$ & $\{\det(h) \mid h \in H \}$ for $H \le GL(V)$;\\
$g \otimes h$ &  {\bf Kronecker product} \index{Kronecker product}   
$\begin{pmatrix}
g \cdot h_{1,1}     &  \ldots & g \cdot h_{1,m}  \\
\ldots           &   \ldots   & \ldots   \\
g \cdot h_{m,1}     & \ldots     & g \cdot h_{m,m}      
\end{pmatrix} \in GL_{nm}(q)$ \\ & for $g \in GL_n(q)$ and $h \in GL_m(q)$; \\
$SL(V)$ & {\bf special linear group} $\{g \in GL(V) \mid \det(g)=1\}$;\\
$D(G)$ & subgroup of all diagonal matrices of a matrix group $G$;\\
$RT(G)$ & subgroup of all upper-triangular %(right-triangular)
 matrices of a matrix group $G$;\\
$p'$ & set of all primes except $p$;\\
$(a,b)$ & greatest common divisor of integers $a$ and $b$;\\
$\delta_{ij}$ & {\bf Kronecker delta}, $\delta_{ij}=1$ if $i=j$ and $\delta_{ij}=0$ otherwise.
\end{longtable}

\medskip


%We write $\diag(\alpha_1, \ldots, \alpha_n)$ to denote the {\bf diagonal} matrix with entries $\alpha_1, \ldots, \alpha_n$ on its diagonal.   The matrix $\diag(\alpha, \ldots, \alpha)$ (and the linear transformation represented by such a matrix) is a {\bf scalar} matrix (transformation), or simply a {\bf scalar}. If $g_1, \ldots, g_k \in GL_m(q)$, then we denote a {\bf block-diagonal} matrix in $GL_{mk}(q)$ with blocks $g_1, \ldots, g_k$ on its diagonal by $\diag[g_1, \ldots, g_k].$ If $\sigma \in \Sym(n),$ then $\per(\sigma) \in GL_n(q)$ is the {\bf permutation } matrix corresponding to $\sigma.$ For $H \ge GL_n(q)$ denote $\Det(H)=\{\det(h) \mid h \in H\} \le \mathbb{F}_q^*$ where $\det(h)$ is the {\bf determinant} of a matrix $h.$ Denote the {\bf special linear group} $\{g \in GL(V) \mid \det(g)=1\}$ by $SL_n(V)$ (we use notation $SL_n(q, \beta)$ and $SL_n(q)$ the same way it described above for $GL(V)$).

%Let $A \in GL_n(q)$ and let $B=(\beta_{ij}) \in GL_m(q).$ The {\bf Kronecker product} $A \otimes B \in GL_{nm}(q)$ is %the  $(nm \times nm)$ matrix    
%$$\begin{pmatrix}
%A \beta_{11}     &  \ldots & A \beta_{1m}  \\
%\ldots           &   \ldots   & \ldots   \\
%A \beta_{1m}     & \ldots     & A \beta_{mm}      
%\end{pmatrix}.$$

A map $g: V \to V$ is an $\mathbb{F}$-{\bf semilinear} \index{semilinear} transformation of $V$ if there exists $\sigma(g) \in \Aut(\mathbb{F})$ such that for all $u,v \in V$ and $\lambda \in \mathbb{F}$,
$$(u+v)g=ug +vg \text{ and } (\lambda v)g=\lambda^{\sigma(g)}(vg).$$
We write $\GL(V)=\GL(V, \mathbb{F})$ for the {\bf general semilinear group}, which is the group of all invertible $\mathbb{F}$-semilinear transformations of $V$. It is easy to see that $\sigma(gh)=\sigma(g)\sigma(h)$ for $g,h \in \GL(V, \mathbb{F}).$ Let $\beta$ be a basis of $V$. As each element of $GL(V, \mathbb{F})$ is determined by its action on $\beta$,  each  $g \in \GL(V, \mathbb{F})$ is determined by its action on $\beta$ and $\sigma(g).$ If $\alpha \in \Aut(\mathbb{F})$, then $\phi_{\beta}(\alpha)$ denotes the unique  $g \in \GL(V, \mathbb{F})$ such that $\sigma(g)=\alpha$ and $(v_i)g=v_i$ for all $v_i \in \beta.$ So
\begin{equation}
\label{defphibet}
\left(\sum_{i=1}^n \lambda_i v_i \right) \phi_{\beta}(\alpha)= \sum_{i=1}^n\lambda_i^{\alpha} v_i.
\end{equation}
 If $\mathbb{F}= \mathbb{F}_{q^{\bf u}}$ and $\alpha \in \Aut(\mathbb{F})$ is such that $\lambda^{\alpha}=\lambda^p$ for all $\lambda \in \mathbb{F}$,  then we denote $\phi_{\beta}(\alpha)$ by $\phi_{\beta}$ \index{$\phi_{\beta}$} or simply $\phi$ when $\beta$ is understood. It is routine to check (see \cite[\S 2.2]{kleidlieb}) that  $$\GL(V, \mathbb{F}_q)= GL(V, \mathbb{F}_q) \rtimes \langle \phi \rangle \cong GL_n(q, \beta) \rtimes \langle \phi \rangle.$$ We denote $GL_n(q, \beta) \rtimes \langle \phi \rangle$ by $\GL_n(q, \beta)$ or simply $\GL_n(q)$ when $\beta$ is understood. In what follows,
we make no essential distinction between the groups $\GL(V )$ and $\GL_n(q)$ and use $\GL(V )$ or
$\GL_n (q)$ depending on which one is more suitable for our purpose.

For a basis $\beta$ of $V$ let $\iota_{\beta}: GL_n(q, \beta) \to GL_n(q, \beta)$ be the inverse-transpose map 
$$\iota_{\beta}: g \mapsto (g^{-1})^{\top}.$$ Therefore, $\langle \iota_{\beta} \rangle$ acts on $GL_n(q, \beta)$ and we define 
$$A(n,q):=\GL_n(q) \rtimes \langle \iota_{\beta} \rangle$$ by letting $\iota_{\beta}$  commute with $\phi_{\beta}.$

It is convenient to view  the symmetric group 
as a group of permutation matrices. We define the {\bf wreath product} \index{wreath product}  of  $X \le GL_n(q)$ and a  group of permutation matrices $Y \le GL_m(q)$ as the matrix group $X \wr Y \le GL_{nm}(q)$ obtained
by replacing the entries 1 and 0  in every matrix in $Y$ by
arbitrary matrices in $X$ and by zero $(n \times n)$ matrices respectively. 

Let $A$ be an $(nm\times nm)$ matrix. We can view $A$ as the matrix
$$\begin{pmatrix}
A_{11}      &  \ldots & A_{1m}  \\
\ldots           &   \ldots   & \ldots   \\
A_{m1}     & \ldots     & A_{mm}      
\end{pmatrix}$$
where the $A_{ij}$ are $(n \times n)$ matrices. The vector $(A_{i1}, \ldots, A_{im})$ is the $i$-th $(n \times n)$-{\bf row} \index{$(n \times n)$-row} of $A.$

%Let $G$ be a group of matrices, denote its center,  subgroup of diagonal matrices and  subgroup of upper-triangular (right-triangular) matrices  by $Z(G)$, $D(G)$ and $RT(G)$ respectively. 

\medskip

 Let $\mathbb{F}$ be a field and let $G$ be a group. An $\mathbb{F}$-{\bf representation} \index{representation} of $G$ is a homomorphism $$\mathfrak{X}: G \to GL(V,\mathbb{F})$$ with $V=\mathbb{F}^n$ for some $n \in  \mathbb{N}$. By linear extension, $\mathfrak{X}$ determines an $\mathbb{F}$-representation of the {\bf group algebra} $\mathbb{F}[G]$, \index{group algebra} which is an algebra homomorphism from $\mathbb{F}[G]$ to $M_n(\mathbb{F})$ denoted by the same letter $\mathfrak{X}.$ Therefore, the action via $\mathfrak{X}$ makes $V$  an $\mathbb{F}[G]$-module.

 If $W \le V$ is an  $\mathbb{F}[G]$-submodule, then $W$ is a $G$-{\bf invariant} subspace of $V$, sometimes we state this fact as $(W)G=W.$ If there exists  a non-zero $G$-invariant subspace of $V$, then   $V$ is  a {\bf reducible} $\mathbb{F}[G]$-module, $\mathfrak{X}$ is a {\bf reducible} representation, and $G$ is a {\bf reducible} group. Otherwise $V$, $\mathfrak{X}$ and $G$ are {\bf irreducible}. \index{irreducible!group} \index{irreducible!module} \index{irreducible!representation} A subgroup of $\GL_n(q)$ is irreducible if it   stabilises (as a group of semilinear transformations) no non-zero proper subspace of $V$. 

 A representation $\mathfrak{X}$ (respectively, a module $V$ and a group $G$) is {\bf completely reducible} if $V$ is a direct sum of $\mathbb{F}[G]$-irreducible submodules. If $V$ is a completely reducible $\mathbb{F}[G]$-module and $M$ is an irreducible $\mathbb{F}[G]$-module, then the sum of those $\mathbb{F}[G]$-submodules of $V$ which are isomorphic to $M$ is the  {$M$-{\bf homogeneous}} {\bf component} $M(V)$ of $G$ on $V$.   If $V=M(V)$ for some irreducible $M$, then $V$ is a {\bf homogeneous} $\mathbb{F}[G]$-module. \index{homogeneous!module} \index{homogeneous!component}

Let  $\mathbb{E}$ be a field extension of $\mathbb{F}$. Then  $(G)\mathfrak{X} \le GL_n(\mathbb{F}) \le GL_n(\mathbb{E}),$ so $\mathfrak{X}$ can be viewed as an $\mathbb{E}$-representation of $G$ which we denote by $\mathfrak{X}^\mathbb{E}.$ A representation $\mathfrak{X}$ (and a group $G$) is {\bf absolutely irreducible} if $\mathfrak{X}^\mathbb{E}$ is irreducible for every field $\mathbb{E} \supseteq \mathbb{F}.$



Let $V$ be an irreducible $\mathbb{F}[G]$-module. If $V$ has a direct sum decomposition 
\begin{equation}\label{imprdecomp}
V=V_1 \oplus \ldots \oplus V_k \text{ for } k>1
\end{equation}
such that for each $i=1, \ldots, k$ and $g \in G$ there exists $j \in \{1, \ldots, k\}$ (unique, since $g$ is invertible) with 
$$(V_i)g=V_j,$$
then $G$ is {\bf imprimitive} and $\{V_1, \ldots, V_k\}$ is a system of imprimitivity of $G$. If $G$ has no system of imprimitivity, then it is {\bf primitive}. \index{primitive}

If $G \le GL(V,\mathbb{F})$, then we assume $(g)\mathfrak{X}=g$ for $g \in G$, unless stated otherwise. Abusing our notation, we denote the subalgebra $A$ of $M_n(\mathbb{F})$ generated by $X \subseteq M_n(\mathbb{F})$ by $\mathbb{F}[X].$ It should not be confusing since for $G \le GL_n(\mathbb{F})$ (complete) reducibility, (absolute) irreducibility, primitivity and other properties of representations we use do not depend on the choice of definition of $\mathbb{F}[G]$  (since $A=(\mathbb{F}[G])\mathfrak{X}$ for the group algebra  $\mathbb{F}[G]$).

\section{Classical forms and groups}
\label{clsec}

Let ${\bf f}$ be a map from $V \times V$ to $\mathbb{F}_{q^{\bf u}}.$ The map ${\bf f}$ is {\bf non-degenerate} \index{form!non-degenerate} if, for every $v \in V \backslash \{0\},$ the  maps $V \to \mathbb{F}$ given by $x \mapsto {\bf f}(x,v)$ and $x \mapsto {\bf f}(v,x)$ are non-zero. If ${\bf f}$ is fixed, then we write $(v,w)$ instead of ${\bf f}(v,w)$ for convenience. The vectors $v$, $w$ are mutually {\bf orthogonal} if $(v,w)=(w,v)=0$. A set of vectors $\{v_1, \ldots, v_n\}$ is {\bf orthonormal} if $(v_i,v_j)=0$  and $(v_i,v_i)=1$ for $i,j \in \{1, \ldots, n\}$ such that  $i\ne j$. 

Let ${\bf u}=2$, so $V=(\mathbb{F}_{q^2})^n$. A {\bf unitary form} \index{form!unitary} is a map ${\bf f}$ from $V \times V$ to $\mathbb{F}_{q^{2}}$ such that for all $u,v,w \in V$ and $\lambda \in \mathbb{F}_{q^{2}}$ the following hold:
\begin{itemize}
\item 
$(u+v,w)=(u,w)+(v,w) \text{ and } (\lambda u, v)=\lambda(u,v);$
\item $(u,v)=(v,u)^q$. 
\end{itemize}


Let ${\bf u}=1$, so $V=(\mathbb{F}_{q})^n$.  A {\bf symplectic form} \index{form!symplectic} is a map ${\bf f}$ from $V \times V$ to $\mathbb{F}_{q}$ such that for all $u,v,w \in V$ and $\lambda \in \mathbb{F}_{q}$ the following hold:
\begin{itemize}
\item 
$(u+v,w)=(u,w)+(v,w) \text{ and } (\lambda u, v)=\lambda(u,v);$
\item $(u,v)=-(v,u)$;
\item $(u,u)=0.$ 
\end{itemize}

Let ${\bf f}$ be a non-degenerate unitary (symplectic) form. The pair $(V,{\bf f})$ is a {\bf unitary (symplectic) space}. %Unitary ans symplectic spaces are examples of {\bf classical spaces }(see \cite[ \S 2.1]{lieb} for the definition. Authors of \cite{lieb} use the term "classical geometry", but we prefer the word "space").
 Two unitary (symplectic) spaces $(V_1, {\bf f}_1)$ and  $(V_2, {\bf f}_2)$ are {\bf isometric} if there exists an isomorphism of vector spaces $\varphi: V_1 \to V_2$ such that $${\bf f}_1(v,u)={\bf f}_2((v)\varphi,(u)\varphi)$$
for every $v$ and $u$ from $V_1$. Such $\varphi$ is an {\bf isometry.} \index{isometry} A {\bf similarity} \index{similarity} of unitary (symplectic) spaces $(V_1, {\bf f}_1)$ and  $(V_2, {\bf f}_2)$ is an isomorphism of vector spaces $\varphi: V_1 \to V_2$ such that there exists $\lambda \in \mathbb{F}_{q^{\bf u}}$ with 
\begin{equation}
\label{simillambda}
{\bf f}_1(v,u)=\lambda{\bf f}_2((v)\varphi,(u)\varphi)
\end{equation}
for every $v$ and $u$ from $V_1$. 

Let us fix ${\bf f}$ to be either  identically zero, or a non-degenerate unitary or symplectic form on $V$ for the rest of the section. Let $W$ be a subspace of $V$. If the restriction ${\bf f}_{W}$ of ${\bf f}$ to $W$ is non-degenerate, then $W$ is a {\bf non-degenerate subspace} \index{subspace!non-degenerate} of $V$. If ${\bf f}_{W}=0,$ then $W$ is a {\bf totally isotropic subspace} \index{subspace!totally isotropic} of $V.$ 

Two subspaces $U$ and $W$ of $V$ are {\bf orthogonal} if $(u,w)=0$ for all $u \in U$ and all $w \in W.$ We write $U \bot W$ for the direct sum of orthogonal subspaces. The {\bf orthogonal complement } $W^{\bot}$ of $W$ in $V$ is 
$$\{v \in V \mid (v,u)=0 \text{ for all } u \in W \}.$$   More details about spaces with forms can be found in \cite[\S 2.1]{kleidlieb}.
 
 Let $I(V, {\bf f})$ and $\Delta(V, {\bf f})$ be the group of all ${\bf f}$-isometries and all ${\bf f}$-similarities from $V$ to itself respectively. By definition, $I(V, {\bf f})$ and $\Delta(V, {\bf f})$ are subgroups of $GL(V)$, so $\Sigma(V, {\bf f}):=SL(V) \cap I(V, {\bf f})$ is well-defined. It is easy to see that if ${\bf f}$ is identically zero, then $\Sigma(V, {\bf f})=SL(V)$ and $I(V, {\bf f})=\Delta(V, {\bf f})=GL(V).$ 

%If $\bf f$ is a non-degenerate unitary or symplectic form, then $\Sigma(V, {\bf f}),$ $I(V, {\bf f})$ and $\Delta(V, {\bf f})$ are also defined uniquely 
 %We define $GU(V,\mathbb{F}_{q^2})=GU(V)$ and $Sp(V, \mathbb{F}_q)=Sp(V)$ to be $I(V, {\bf f})$ for ${\bf f}$ unitary and symplectic respectively. 
% since, by the following lemmas, in each of two cases, all such spaces $(V,{\bf f})$ are isometric. 

All non-degenerate unitary (respectively symplectic) spaces of the same dimension over $\mathbb{F}_{q^{\bf u}}$ are isometric by the following lemmas.

\begin{Lem}[{\cite[Propositions 2.3.1 and 2.3.2]{kleidlieb}}]\label{unibasisl}
Let ${\bf f}$ be unitary.
\begin{enumerate}[font=\normalfont]
\item The space $(V,{\bf f})$ has an orthonormal basis.
\item The space $(V,{\bf f})$ has a basis 
\begin{equation}\label{unibasis}
\begin{cases}
\{f_1, \ldots, f_m, e_1, \ldots, e_m\}, & \text{ if $n=2m$} \\
\{f_1, \ldots, f_m, x, e_1, \ldots, e_m\}, & \text{ if $n=2m+1$ }
\end{cases}
\end{equation}
where $(e_i,e_j)=(f_i,f_j)=0,$ $(e_i,f_j)=\delta_{ij}$ and $(e_i,x)=(f_i,x)=0$ for all $i,j,$ 
and $(x,x)=1.$
\end{enumerate}
\end{Lem}

\begin{Lem}[{\cite[Proposition 2.4.1]{kleidlieb}}]\label{sympbasisl}
Let ${\bf f}$ be symplectic.
The dimension $n$ of $V$ is even and the space $(V,{\bf f})$ has a basis 
\begin{equation}\label{sympbasis}
\{f_1, \ldots, f_m, e_1, \ldots, e_m\}, 
\end{equation}
where $2m=n$,  $(e_i,e_j)=(f_i,f_j)=0$ and $(e_i,f_j)=\delta_{ij}$ for all $i,j.$
\end{Lem}

Hence, for a non-degenerate unitary or symplectic space $(V, {\bf f}),$ the groups $\Sigma(V, {\bf f}),$ $I(V, {\bf f})$ and $\Delta(V, {\bf f})$ are also defined uniquely (up to conjugation in $GL(V)$) by $\dim V$ and $q.$ 



An ${\bf f}$-{\bf semisimilarity} \index{semisimilarity} is $g \in \GL_n(q^{\bf u})$ such that there exist $\lambda \in \mathbb{F}_{q^{\bf u}}^*$ and $\alpha \in \Aut (\mathbb{F}_{q^{\bf u}})$ satisfying
\begin{equation}
\label{GAMlambda}{\bf f}(vg,ug)=\lambda{\bf f}(v,u)^{\alpha} \text{ for all } v,u \in V.
\end{equation}
By \cite[Lemma 2.1.2]{kleidlieb}, $\alpha$ is determined uniquely by $g$, and $\alpha=\sigma(g).$ We denote the group of ${\bf f}$-semisimilarities of $V$ by $\Gamma(V, {\bf f}).$ It is easy to see that 
$$\Delta(V, {\bf f}) \le \Gamma(V, {\bf f}).$$


\begin{Def}
\label{taudef}
By \cite[Lemma 2.1.2]{kleidlieb}, if ${\bf f}$ is non-degenerate, then the $\lambda$ in \eqref{simillambda} and \eqref{GAMlambda} are uniquely determined by $g$. Moreover, there exists a homomorphism $\tau:\Delta(V, {\bf f}) \to \mathbb{F}_{q^{\bf u}}^*$ satisfying
$\tau(g)=\lambda.$
\end{Def}


We say that we work on the case {\bf L}, {\bf U} or {\bf S} \index{case {\bf L}, {\bf U}, {\bf S}} when ${\bf f}$ is identically zero, unitary or symplectic respectively. We summarise  notation for the groups $\Sigma,$ $I$, $\Delta$ and $\Gamma$ in Table \ref{classnot}. For more details on classical groups and the  equalities claimed in the table  see \cite[\S 2.1]{kleidlieb}. 

\begin{table}[h] 
\centering
\caption{Notation for classical groups}
\begin{tabular}{|c|c|c|c|} 
\hline
case                                      &            & notation & terminology                        \\ \hline
\multirow{3}{*}{{\bf L}} & $\Sigma$   & $SL(V)$  & \multirow{3}{*}{linear groups}     \\ \cline{2-3}
                                          & $I=\Delta$ & $GL(V)$  &                                    \\ \cline{2-3}
           & $\Gamma$ & $\GL(V)$  &                                    \\ \hline
\multirow{3}{*}{\bf U} & $\Sigma$   & $SU(V)$  & \multirow{3}{*}{unitary groups}    \\ \cline{2-3}
                                          & $I$        & $GU(V)$  &                                    \\ \cline{2-3}
& $\Gamma$        & $\GU(V)$  &                                    \\ \hline
\multirow{3}{*}{\bf S} & $\Sigma=I$ & $Sp(V)$    & \multirow{3}{*}{symplectic groups} \\ \cline{2-3}
                                          & $\Delta$   & $GSp(V)$   &                                    \\ \cline{2-3}
& $\Gamma$        & $\GS(V)$  &                                    \\ \hline
\end{tabular}
\label{classnot}
\end{table} 

Denote the identity $(n\times n)$ matrix by $I_n$ \index{$I_n$} and let $J_{2k}$ \index{$J_{2k}$} be the matrix
\begin{equation*}
\begin{pmatrix}
    & I_k\\
 -I_k &     
\end{pmatrix}.
\end{equation*} 
For $g \in GL_n(q^{\bf u})$ let $\overline{g}$ be  the matrix obtained from $g$ by taking every entry to the $q$-th power (so if ${\bf u}=1$, then $\overline{g}=g$). 
We write $g^{\dagger}$ for $(\overline{g}^{\top})^{-1}$ and $X^{\dagger}$ for $\{g^{\dagger} \mid g \in X\}$, where $X \subseteq GL_n(q^{\bf u}).$

Fix a basis $\beta=\{v_1, \ldots , v_n\}$ of $V$ and denote by ${\bf f}_{\beta}$ the matrix whose $(i,j)$ entry is ${\bf f}(v_i,v_j).$ By fixing the basis, we identify $I(V, {\bf f})$ and $\Delta(V, {\bf f})$ with the matrix groups
\begin{equation}\label{unimatr}
\{g \in GL_n(q^{\bf u}) \mid g {\bf f}_{\beta}\overline{g}^{\top}={\bf f}_{\beta}\} \text{ and } \{g \in GL_n(q^{\bf u}) \mid g {\bf f}_{\beta}\overline{g}^{\top}=\lambda{\bf f}_{\beta}, \lambda \in \mathbb{F}_{q^{\bf u}}^*\}
\end{equation}
 respectively; we identify $\Gamma(V, {\bf f})$ with the subgroup  $\Gamma(V, {\bf f})_{\beta} \le \GL(V, \beta)$ of ${\bf f}$-semisimilarities.


 Denote the group of matrices representing the isometries from $I(V, {\bf f})$ with respect to a basis $\beta$ such that ${\bf f}_{\beta}=\Phi$ by $GU_n(q,\Phi)$ (respectively $Sp_n(q,\Phi)$)  or $GU_n(q, \beta)$ (respectively $Sp_n(q, \beta)$).  We write  $GU_n(q)$ (respectively $Sp_n(q)$) instead of $GU_n(q, I_n)$ (respectively $Sp_n(q, J_n)$) for simplicity; we use similar notation for $\Sigma(V, {\bf f})$, $\Delta(V, {\bf f})$ and $\Gamma(V, {\bf f})$ in cases {\bf U} and {\bf S}. We also use $GL_n^{\varepsilon}(q)$ with $\varepsilon \in \{+, -\}$ where $GL_n^+(q)=GL_n(q)$ and $GL_n^-(q)=GU_n(q).$ 

\medskip

Note the following observations and notation:
\begin{itemize}
\item In some literature $\Delta(V, {\bf f})$ for case {\bf S} is denoted by $CSp(V)$ and called the ``conformal symplectic group''.
\item If $\beta$ is as in Lemmas \ref{unibasisl} and \ref{sympbasisl} for cases {\bf U} and {\bf S} respectively, then $\phi_{\beta} \in \Gamma(V, {\bf f})$ and $\Gamma(V, {\bf f})_{\beta}= \Delta(V, \bf{f})_{\beta} \rtimes \langle \phi_{\beta} \rangle$. 
\item The group   $\Delta(V, {\bf f})$ for case {\bf U} is omitted in Table \ref{classnot} since here $$\Delta(V, {\bf f})=I(V, {\bf f}) \cdot \mathbb{F}_q^*.$$ Therefore, 
  $\Delta(V, \bf{f})_{\beta} \rtimes \langle \phi_{\beta} \rangle$ and $I(V, \bf{f})_{\beta} \rtimes \langle \phi_{\beta} \rangle$ (and their maximal solvable subgroups) coincide modulo scalars. It is more convenient for us to work with $I(V, \bf{f})_{\beta} \rtimes \langle \phi_{\beta} \rangle$, so in what follows we abuse notation by letting   $$\Gamma(V, {\bf f})= I(V, {\bf f}) \rtimes \langle \phi_{\beta}\rangle$$
for an orthonormal basis $\beta$ in case {\bf U}.
\item If $\Sigma(V, {\bf f}) \le G \le \Gamma(V, {\bf f}),$ then $G$ is solvable if and only if $\Sigma(V, {\bf f})$ is solvable since $\Delta(V, {\bf f})/\Sigma(V, {\bf f})$ and $\Gamma(V, {\bf f})/\Delta(V, {\bf f})$ are abelian. Therefore, such $G$ is solvable if and only if either $n=1$ or  $\Sigma(V, {\bf f})$ is one of the following groups: $SL_2(q)=Sp_2(q)\cong SU_2(q) $ for $q \in \{2,3\}$,   $SU_3(2).$ 
We often write ``$G$ is not solvable'' where we ignore these groups.
\end{itemize}










We state  a particular case of Witt's Lemma, which we use later. For a proof see \cite[\S 20]{asch}.

\begin{Lem}
\label{witt}
 Assume that $(V_1,{\bf f}_1)$, $(V_2,{\bf f}_2)$ are isometric unitary (symplectic) spaces and  $W_i$ is a subspace of $V_i$ for $i=1,2.$ If there is an isometry $g$ from $(W_1,{\bf f}_1)$ to $(W_2,{\bf f}_2),$ then $g$ extends to an isometry from $(V_1,{\bf f}_1)$ to $(V_2,{\bf f}_2).$  
\end{Lem}


\section{Algebraic groups}
\label{algsec}
 In this section we state necessary notation and results on algebraic groups.  
Informally speaking, an algebraic group is a group that is an algebraic variety such that  the multiplication and inversion operations are morphisms (polynomial maps) of the variety. To avoid a long series of definitions on varieties we use the fact that an affine algebraic group over an algebraically closed field of positive characteristic is isomorphic (as an algebraic group, which means that there exists a group isomorphism $\varphi$ such that $\varphi$ and $\varphi^{-1}$ are also  morphisms of the corresponding varieties) to a linear group \cite[p. 63]{hump}. Hence we state definitions and results in terms of linear groups.  Our standard references are \cite[Chapter 1]{carter}, \cite[Chapter 1]{gorlyo} and \cite{hump}.

\begin{Def}
Let $\overline{\mathbb{F}}$ be the algebraic closure of the field of size $p.$
\begin{itemize}
\item     The {\bf Zariski topology} on $GL_n(\overline{\mathbb{F}})$ is the topology defined by the condition that the {\bf closed sets} are the solution sets of systems of polynomial equations in matrix entries and the function $g \mapsto (\det g)^{-1}$ for $g \in GL_n(\overline{\mathbb{F}}).$ An $\overline{\mathbb{F}}$-{\bf linear algebraic group} (which we abbreviate to $\overline{\mathbb{F}}${\bf -algebraic group} or just { \bf algebraic group}) is a closed subgroup $\overline{K}$ of  $GL_n(\overline{\mathbb{F}})$ for some $n.$ The Zariski topology on $\overline{K}$ is the topology inherited from that of $GL_n(\overline{\mathbb{F}})$.
\item The {\bf connected component containing the identity element} \index{connected component} (in Zariski topology) of $\overline{K}$ is denoted by $\overline{K}^0.$
\item A {\bf torus} \index{torus} is an algebraic group \index{algebraic group} isomorphic to the direct product of finitely many copies of $GL_1(\overline{\mathbb{F}}).$ A {\bf subtorus} of an algebraic group $\overline{K}$ is a closed subgroup of $\overline{K}$ which
is a torus. A {\bf maximal torus} \index{torus!maximal} of $\overline{K}$ is a subtorus of $\overline{K}$ not contained in any other
subtorus of $\overline{K}$.
\item A {\bf Frobenius endomorphism} \index{Frobenius endomorphism} of  $\overline{K}$ is a surjective endomorphism $\sigma$ of $\overline{K}$
whose {\bf fixed point subgroup} $\overline{K}_{\sigma}$ is finite.
\item If $\overline{K}$ is nontrivial and connected but has no proper closed connected normal subgroup, then $\overline{K}$ is a {\bf simple} algebraic group. \index{algebraic group!simple}
\end{itemize}
\end{Def}

We are interested in simple algebraic groups since most  finite classical groups appear as fixed point subgroups (or their normal subgroups) for suitable simple algebraic groups and Frobenius endomorphisms. The classification of simple algebraic groups is based on the classification of their root systems. A simple algebraic group has an irreducible reduced root system. We do not define root systems here, but use their labels, see \cite[Chapter 1]{carter} for details. 


%For our purpose, it is enough to state the labels of the root systems giving rise to classical simple algebraic groups matching them to the corresponding groups. 

A simple algebraic group is not  uniquely determined by its root system.
\begin{Th}[{\cite[Theorem 1.10.4]{gorlyo}}]
Let $\overline{\mathbb{F}}$ be the algebraic closure of the field of size $p.$ Let $\Sigma$ be an irreducible reduced root system. There exist simple algebraic groups $\overline{K}_u=\overline{K}_u(\Sigma)$ and $\overline{K}_a=\overline{K}_a(\Sigma)$ over $\overline{\mathbb{F}},$ unique up to isomorphism of algebraic groups, with the following properties:
\begin{itemize}
\item $\Sigma$ is the root system of both $\overline{K}_u$ and $\overline{K}_a;$
\item for every simple algebraic group $\overline{K}$ over $\overline{\mathbb{F}}$ with root system isomorphic to $\Sigma$ there exist  surjective homomorphisms of algebraic groups   $\overline{K}_u \to \overline{K}\to \overline{K}_a;$
\item $Z(\overline{K}_u)$ is finite and $Z(\overline{K}_a)=1.$
\end{itemize} 
\end{Th}
We call $\overline{K}_u$ and $\overline{K}_a$ the {\bf universal} \index{algebraic group!universal} and {\bf adjoint} \index{algebraic group!adjoint} simple algebraic group of type $\Sigma$ respectively.

Let ${\bf q}:GL_n(\overline{\mathbb{F}}) \to GL_n(\overline{\mathbb{F}})$ for $q=p^f$ be the map taking each entry of a  matrix to its $q$-th power and let ${\bf g}:GL_n(\overline{\mathbb{F}}) \to GL_n(\overline{\mathbb{F}})$ be the inverse-transpose map. The maps {\bf q} and {\bf qg} are Frobenius morphisms. We collect information about certain classical simple algebraic groups and their fixed point subgroups in Tables \ref{algAC} and \ref{algSIG}. For this information on all classical simple algebraic groups see \cite[Theorem 1.10.7]{gorlyo} and \cite[\S 1.19]{carter}. The labels  $A_l$ and $C_l$, for a positive integer $l$, are {\bf types} of irreducible reduced root systems. 

\begin{table}[] 
\centering
\caption{Simple algebraic groups with root systems $A_l$ and $C_l$}
\begin{tabular}{|l|l|l|} 
\hline
$\Sigma$ & $\overline{K}_u$  \phantom{ \LARGE G}     & $\overline{K}_a$        \\ \hline
$A_l$    & $SL_{l+1}(\overline{\mathbb{F}})$ & $PGL_{l+1}(\overline{\mathbb{F}})$ \\ \hline
$C_l$    & $Sp_{2l}(\overline{\mathbb{F}})$  & $PGSp_{2l}(\overline{\mathbb{F}})$ \\ \hline
\end{tabular}
\label{algAC}
\end{table}
\begin{table}[] 
\centering
\caption{Fixed point subgroups}
\begin{tabular}{|l|l|l|l|}  
\hline
$\Sigma$ & $\sigma$ & $(\overline{K}_u)_{\sigma}$ \phantom{ \LARGE G} & $(\overline{K}_a)_{\sigma}$ \\ \hline
$A_l$    & {\bf q}  & $SL_{l+1}(q)$    & $PGL_{l+1}(q)$   \\ \hline
$A_l$    & {\bf qg} & $SU_{l+1}(q)$    & $PGU_{l+1}(q)$   \\ \hline
$C_l$    & {\bf q}  & $Sp_{2l}(q)$     & $PGSp_{2l}(q)$   \\ \hline
\end{tabular}
\label{algSIG}
\end{table}


  


%%%%%%%%%%%%%%%%%%%%%%%%%%%%%%%%%%%%%%%%%%%%%%%%%%%%%%%%%%%%%%%%%%%%%%%%%%%%%%%%%%%%%%%%%%%%%%  

\section{Miscellaneous results}
\label{missec}

 We begin by stating two classical results on the structure of solvable linear groups.
\begin{Lem}[{\cite[\S 18, Theorem 5]{sup}}] \label{supirr}
An irreducible solvable subgroup of $GL_n(q)$ is either primitive, or conjugate in $GL_n(q)$ to a subgroup of the wreath product
$S \wr \Gamma$
where $S$ is a primitive solvable subgroup of $GL_m(q)$ and $\Gamma$ is a transitive solvable
subgroup of the symmetric group $\Sym(k)$ and $km = n$. In particular, an irreducible maximal solvable  subgroup of 
$GL_n(q)$ is either primitive, or conjugate in $GL_n(q)$
to  $S \wr \Gamma$, where $S$ is a  primitive maximal solvable subgroup of $GL_m(q)$
and $\Gamma$ is a  transitive maximal solvable subgroup of $\Sym(k)$.
\end{Lem}

\begin{Lem}[{\cite[\S 18, Theorem 3]{sup}}]
\label{supreduce}
Let  $H$ be a
subgroup of $GL_n(q)$. In a suitable basis $\beta$ of $V$, the matrices $g\in H$ have the
shape
\begin{gather}\label{stup}
\begin{pmatrix}
g_k     & g_{k,(k-1)} & \ldots & g_{k,1}  \\
0          &    g_{k-1}  & \ldots & g_{(k-1),1}  \\
\ldots     & \ldots     & \ldots & \ldots     \\
      0    &   0        & \ldots & g_1 
\end{pmatrix}
\end{gather}
where the mapping $\gamma_i: H \to GL_{n_i}(q)$, $g \mapsto g_i$ is an irreducible 
representation of $H$ of degree $n_i$, and $g_{i,j}$ is an $(n_i \times n_j)$ matrix over $\mathbb{F}_q$, and $n _1 + \ldots + n _k = n$.
 The group $H$  is solvable if and only if all the
groups
$$H_i ={\mathrm{Im} } (\gamma_i) \text{ for } i = 1, \ldots, k,$$
are solvable.
\end{Lem}

 Now we state three technical lemmas about solvable linear groups. We use them and ideas from their proofs many times throughout  this work. 

\begin{Lem}\label{supSL}
Let $H \le G \le GL_n(q).$ Assume $\Det(H)=\Det(G).$ 
\begin{enumerate}[font=\normalfont]
\item $H \cdot (SL_n(q) \cap G)=G$.
\item If $g \in G$, then there exists $g_1 \in SL_n(q) \cap G$ such that $H^g=H^{g_1}.$ 
\end{enumerate}
\end{Lem}
\begin{proof}
Let $g \in G$. Since $\Det(H)=\Det(G)$, there exists $h \in H$ such that $\det(h)=\det(g).$ Therefore, 
$h^{-1}g \in SL_n(q) \cap G$, so $g=h \cdot (h^{-1}g)\in H \cdot (SL_n(q) \cap G).$

Let $g_1=h^{-1}g,$ so $H^{g_1}=g^{-1}hHh^{-1}g=g^{-1}Hg=H^g.$
\end{proof}

\begin{Lem}\label{irrtog}
Let $H$ be a subgroup of $GL_n(q)$ of shape \eqref{stup}. If for every $H_i$ there exists $x_i \in GL_{n_i}(q)$ (respectively $SL_{n_i}(q)$) such that the intersection $H_i \cap H_i^{x_i}$ consists of upper triangular matrices, then there exist $x,y \in GL_n(q)$ (respectively $SL_n(q)$) such that 
$$(H\cap H^x) \cap (H\cap H^x)^y \le D(GL_n(q)).$$
\end{Lem}
\begin{proof}
Let $x=\diag(x_k, \ldots, x_1)$ and let $y$ be $$\diag(\sgn(\sigma),1 \ldots,1) \cdot \per(\sigma)$$ where  $$\sigma =(1,n)(2, n-1) \ldots ([n/2], [n/2+3/2]).$$ Here $[r]$ is the integer part of a positive number $r$. Since $\det(\per(\sigma))=\sgn(\sigma),$ $\det(y)=1.$ Since
$H\cap H^x$ consists of upper triangular matrices and $(H\cap H^x)^y$ consists of lower triangular matrices, 
\begin{equation*}
(H\cap H^x) \cap (H\cap H^x)^y \le D(GL_n(q)). %\qedhere
\end{equation*} 
If $x_i \in SL_{n_i}(q)$, then $x \in SL_n(q).$ % Notice that
%\begin{equation*}
%(H\cap H^x) \cap (H\cap H^x)^{y} \le D(GL_n(q)). \qedhere
%\end{equation*}  
\end{proof}




\begin{Def}
Let $\{v_1, v_2, \ldots,  v_n \}$ be a basis of $V$. The equality 
$$v=\alpha_1v_1+ \alpha_2v_2+  \ldots+ \alpha_nv_n $$ 
is the {\bf decomposition} \index{decomposition} of $v \in V$ with respect to $\{v_1, v_2, \ldots,  v_n \}$.
 \end{Def}


\begin{Lem}\label{diag}
Assume that matrices in $H\le GL_n(q)$ have shape \eqref{stup} with respect to the  basis $$\{v_1, v_2, \ldots,  v_n \}$$ and $n_1 < n,$ so $H$ stabilises $$U=\langle v_{n-n_1+1}, \ldots, v_n \rangle.$$  Then there exists $z \in SL_n(q)$ such that $D(GL_n(q)) \cap H^z \le Z(GL_n(q)).$
\end{Lem} 
\begin{proof}
Let $m=n_1.$ Define vectors 
\begin{equation*}
\begin{split}
u_1 & = v_1 + \ldots + v_{n-m} + v_{n-m+1}  \\
u_2 & = v_1 + \ldots + v_{n-m} + v_{n-m +2}\\
u_3 & = v_1 + \ldots + v_{n-m}  + v_{n-m +3}\\
\vdots \\
u_m & = v_1 + \ldots + v_{n-m} + v_{n}.
\end{split}
\end{equation*}
Let $z \in GL_n(q)$ be such that 
\begin{equation}\label{ui}
(v_{n-m+i})z=u_i, \text{ } i=1, \ldots, m.
\end{equation}
 Such  $z$ exists since $u_1, \ldots, u_m$ are linearly independent and we can assume $z\in SL_n(q)$ since $m<n.$ So \eqref{ui} implies that 
$$Uz = \langle u_1, \ldots, u_m \rangle $$ is $H^z$-invariant.

Let $g \in D(GL_n(q)) \cap H^z.$ So $g$ is $\text{diag}(\alpha_1, \ldots , \alpha_n)$ with respect to the basis $\{v_1, v_2, \ldots, v_{n}\}$. Thus,
$$(u_j)g = \alpha_1 v_1 + \ldots + \alpha_{n-m} v_{n-m} + \alpha_{n-m+j} v_{n-m+j}. $$
Also, $(u_j)g$ must lie in $Uz,$ since $g \in H^z,$ so 
$$(u_j)g = \beta_1 u_1 + \beta_2 u_2 + \ldots + \beta_m u_m.$$
But the decomposition of $(u_j)g$ with respect to    $\{v_1, v_2, \ldots, v_{n}\}$ does not contain $v_{n-m +i}$ for $i \ne j$, so 
$$\beta_1 = \beta_2 = \ldots =\beta_{j-1} =\beta_{j+1}= \ldots = \beta_m =0$$ and $(u_j)g= \beta_j u_j$. Thus, 
$$\alpha_1 = \alpha_2 = \ldots =  \alpha_{n-m}=\alpha_{n-m+j}$$
for $0<j\le m.$
Therefore, $g$ is scalar and lies in $Z(GL_n(q)).$
\end{proof}

 The next three lemmas provide information on $\Gamma(V, {\bf f})$ and its subgroups. Here ${\bf f}$ is unitary or symplectic; by default, we  assume that such a form ${\bf f}$ is non-degenerate. 


\begin{Lem}[{\cite[(5.5)]{asch}}] \label{ashb}
Let $H \le \Gamma(V, {\bf f})$, with {\bf f} unitary or symplectic, be irreducible. Let $L$ be a non-scalar normal subgroup of $H$ contained in $GL(V)$. Let $\{V_i \mid 1 \le i\le k\}$ be the homogeneous components of $L$ on
$V$ and assume $k > 1$. One of the following holds:
\begin{enumerate}[font=\normalfont]
\item  $$ \displaystyle V=\mathop{\bot}_{1 \le i\le k} V_i$$ 
 with $V_i$ non-degenerate and isometric to $V_j$ for each $1 \le i \le j \le k $;
\item  $$ \displaystyle V=\mathop{\bot}_{1 \le i\le k/2} U_i$$ with $U_i=V_{2i-1} \oplus V_{2i}$ where $U_i$ is non-degenerate and isometric to $U_j$ for 
$1 \le i \le j \le k / 2$,
and $V_i$ is totally isotropic for each $1 \le i \le k$.
\end{enumerate}
\end{Lem}


\begin{Lem}
\label{uniGamsdp}
Let ${\bf f}$ be a non-degenerate unitary or symplectic form on $V.$ If $\beta$ is a basis of $V$ such that ${\bf f}_{\beta}^{\phi_{\beta}}={\bf f}_{\beta},$ then 
$\Gamma(V,{\bf f})_{\beta}= \Delta(V,{\bf f})_{\beta} \rtimes \langle \phi_{\beta} \rangle.$ 
\end{Lem}
\begin{proof}
Clearly, $\Delta(V,{\bf f})_{\beta} \cap \langle \phi_{\beta} \rangle=1,$ so it suffices to show that $\phi_{\beta}$ normalises $\Delta(V,{\bf f})_{\beta}$ and  is a semisimilarity of $(V, {\bf f}).$ Let $g \in \Delta(V,{\bf f})_{\beta}$, so 
$$g {\bf f}_{\beta} \overline{g}^{\top}=\lambda {\bf f}_{\beta}$$
for some $\lambda \in \mathbb{F}_q^*.$ Therefore,
$$g^{\phi_{\beta}} {\bf f}_{\beta} \overline{(g^{\phi_{\beta}})}^{\top}=g^{\phi_{\beta}} {\bf f}_{\beta}^{\phi_{\beta}} \overline{(g^{\phi_{\beta}})}^{\top}=(g {\bf f}_{\beta} \overline{g}^{\top})^{\phi_{\beta}}= (\lambda {\bf f}_{\beta})^{\phi_{\beta}}= \lambda^p {\bf f}_{\beta}, $$
and $g^{\phi_{\beta}} \in \Delta(V,{\bf f})_{\beta}.$  

Let $v,u \in V$ have coefficients $(\alpha_1, \ldots, \alpha_n)$ and $(\delta_1, \ldots, \delta_n)$ with respect to $\beta$ respectively. Therefore,
\begin{align*}(v\phi_{\beta},u \phi_{\beta}) & =(\alpha_1^p, \ldots, \alpha_n^p){\bf f}_{\beta}(\overline{\delta_1}^p, \ldots, \overline{\delta_n}^p)^{\top}\\
& =(\alpha_1^p, \ldots, \alpha_n^p){\bf f}_{\beta}^{\phi_{\beta}}(\overline{\delta_1}^p, \ldots, \overline{\delta_n}^p)^{\top}\\
& =(v,u)^p,
\end{align*}
so $\phi_{\beta}$ is a semisimilarity.
\end{proof}





\begin{Lem}\label{unist}
Recall that $q=p^f.$ Let $H \le \Gamma(V, {\bf f})$ with ${\bf f}$ unitary or symplectic. There exists a basis $\beta$ such that 
${\bf f}_{\beta}$ is
\begin{equation}\label{fst}
\begin{aligned}
& \left(
\begin{smallmatrix}
        & & & & & & &    & I_{n_1} \\
        & & & & & & &  \reflectbox{$\ddots$}  &\\
        & & & & & &I_{n_k} &    & \\
        & & &I_{n_{k+1}} & & & &    &\\ 
        & & & &\ddots & & &    &\\
        & & & & &I_{n_{k+l}} & &    &\\ 
        & &I_{n_k} & & & & &    & \\
        &\reflectbox{$\ddots$} & & & & & &    &\\
I_{n_1} & & & & & & &    & 
\end{smallmatrix} \right)  \text{ or } \\
 & \left(
\begin{smallmatrix}
        & & & & & & &    & I_{n_1} \\
        & & & & & & &  \reflectbox{$\ddots$}  &\\
        & & & & & &I_{n_k} &    & \\
        & & &J_{n_{k+1}} & & & &    &\\ 
        & & & &\ddots & & &    &\\
        & & & & &J_{n_{k+l}} & &    &\\ 
        & &-I_{n_k} & & & & &    & \\
        &\reflectbox{$\ddots$} & & & & & &    &\\
-I_{n_1} & & & & & & &    & 
\end{smallmatrix} \right) 
\end{aligned}
\end{equation}
 in cases {\bf U} and {\bf S} respectively. Moreover, if $\varphi \in H_{\beta},$ then $\varphi=({\phi_{\beta}})^j g$ with $$j \in \{1, \ldots, {\bf u}f-1\}$$ and 
\begin{equation}\label{gst}
g=\Scale[0.9]{ \begin{pmatrix}
{\tau(g)\gamma_{1}(g)^{\dagger}}&* &\multicolumn{1}{l|}{*} &* & \ldots & *&* &*    &*  \\
        & \ddots &\multicolumn{1}{l|}{} & & &\ddots & &    &\\
0        & & \multicolumn{1}{l|}{\tau(g)\gamma_{k}(g)^{\dagger}}&* &* &* &\ldots &*    &* \\  \cline{1-6}
        & & \multicolumn{1}{l|}{} &\gamma_{k+1}(g) & & \multicolumn{1}{l|}{0}  &* & \ldots   &*\\ 
        & &  \multicolumn{1}{l|}{}& &\ddots &\multicolumn{1}{l|}{}       & & \ddots   &\\
        & &  \multicolumn{1}{l|}{}&0 & &\multicolumn{1}{l|}{\gamma_{k+l}(g)}    & * & \ldots   & *\\ \cline{4-9} 
        & & & & &\multicolumn{1}{l|}{} & \gamma_{k}(g) &  *  &* \\
        & & & & &\multicolumn{1}{l|}{} & & \ddots   &*\\
0        & & & & &\multicolumn{1}{l|}{} &0 &    & \gamma_{1}(g)\\  
\end{pmatrix}}
\end{equation}
where $\tau(g)$ is as in Definition $\ref{taudef}$, $\gamma_i$ is a homomorphism from $H$ to $\GL_{n_i}(q^{\bf u})$ if $i \le k$, and from $H$ to $\GU_{n_i}(q)$ or $\GS_{n_i}(q)$, in cases {\bf U} and {\bf S} respectively, if $i>k.$ Furthermore, $\gamma_i(H)$ is an irreducible subgroup of $\GL_{n_i}(q^{\bf u})$ for every $i$ and $\gamma_i(H \cap GL_n(q^{\bf u})) \le GL_{n_i}(q^{\bf u}).$ 
\end{Lem}
\begin{proof}
If $H$ is an irreducible subgroup of $\GL(V,\mathbb{F}_{q^{\bf u}})$, then by Lemmas \ref{unibasisl} and \ref{sympbasisl} we can take ${\bf f}_{\beta}$ to be $I_n$ or $J_n$ in cases {\bf U} and {\bf S}, and there is nothing  to prove. So assume that there is a proper $H$-invariant subspace $W$ of $V=\mathbb{F}_{q^{\bf u}}^n$ on which $H$ acts irreducibly. Therefore, $W$ is either non-degenerate or totally isotropic. If $V$ has no totally isotropic $H$-invariant subspace, then $V$ is the direct sum of pairwise orthogonal $H$-invariant  non-degenerate subspaces, so $k=0$ and the lemma follows.

Assume that $W$ is totally isotropic.  By Lemma \ref{witt} we can assume that $V$ has a basis $\beta$ as in \eqref{unibasis} such that $$W= \langle e_{(n-n_1+1)}, \ldots, e_{n} \rangle$$
where $n_1=\dim W$. %so ${\bf f}_{\beta}=J_n.$ 
Let $U$ be the subspace spanned by 
$$\beta \backslash \{f_{(n-n_1+1)}, \ldots, f_{n}, e_{(n-n_1+1)}, \ldots, e_{n}\}.$$ Notice that $U$ is non-degenerate. Let $\beta_2:=\{v_1, \ldots, v_{n-2n_1}\}$ be a basis of $U$ such that $\beta_2$ is orthonormal in case {\bf U} and as in \eqref{sympbasis} in case {\bf S}. 

Let us define a basis  $$\beta_1:=\{f_{(n-n_1+1)}, \ldots, f_{n}, v_1, \ldots, v_{n-2n_1},   e_{(n-n_1+1)}, \ldots, e_{n}\}.$$%Here  $\underline{x}=x$ as in \eqref{unibasis} if $n$ is odd, and  an empty entry if $n$ is even.    
Hence
 \begin{equation*}
{\bf f}_{\beta_1}=
\begin{pmatrix}
  & & I_{n_1}\\
  &\Phi&\\
(-1)^{\bf u}I_{n_1} & &
\end{pmatrix}
\end{equation*} 
with $\Phi$ equal to $I_{n-2n_1}$ and $J_{n-2n_1}$ in cases {\bf U} and {\bf S} respectively. Since $H$ stabilises $W$, it also stabilises $W^{\bot}=\langle v_1, \ldots, v_{n-2n_1},   e_{(n-n_1+1)}, \ldots, e_{n} \rangle$.  By Lemma \ref{uniGamsdp}, if $\varphi \in H_{\beta_1},$ then $\varphi= (\phi_{\beta_1})^j g$ with $g \in GU_n(q, {\bf f}_{\beta_1})$ or $GSp_n(q,{\bf f}_{\beta_1})$ respectively, so, by \eqref{unimatr},   
\begin{equation*}
g=
\begin{pmatrix}
{\tau(g){(g_W})^{\dagger}}&* &*   \\
   0     & g_1 &* \\
0        &0 & g_W    
\end{pmatrix},
\end{equation*}
where $g_1$ is an $(n-2n_1) \times (n-2n_1)$ matrix. If $n=2n_1$, then the lemma follows. We proceed by induction on $n-2n_1$ using the case $n-2n_1=0$ as the base. 

Assume that $n>2n_1.$ Since $g {\bf f}_{\beta_1} \overline{g}^{\top}=\tau(g){\bf f}_{\beta_1},$ 
$$g_1 \Phi \overline{g_1}^{\top}=\tau(g)\Phi.$$
Thus, $g_1$ is a similarity of $$(\langle v_1, \ldots,  v_{n-2n_1} \rangle,{\bf f_1} ),$$
and $$(\phi_{\beta_2})^j g_1 \in \Gamma (\langle v_1, \ldots,  v_{n-2n_1} \rangle,{\bf f_1} )$$
where   ${\bf f_1}$ is the restriction of ${\bf f}$ to $\langle v_1, \ldots,  v_{n-2n_1} \rangle$. Notice that $(\phi_{\beta_2})^j g_1$ is the restriction of $\varphi$ to $W^{\bot}/W.$  So there exists a homomorphism $\psi$  from $H_{\beta_1}$ to $\GU_{n-2n_1}(q)$ in case {\bf U} and $\GS_{n-2n_1}(q)$ in case {\bf S} defined by 
$\psi:g \mapsto (\phi_{\beta_2})^j g_1.$ Applying induction to $\psi(H_{\beta_1}),$ we obtain  the lemma. 
\end{proof}

 The following lemma plays an important role in our proof of Theorems \ref{theorem}, \ref{theoremGU} and \ref{theoremSp}.

\begin{Lem}
\label{scfield}
Let  $\Gamma \in \{\GL_n(q), \GU_n(q), \GS_n(q)\}.$ Let $n\ge 2$ and let $q$ be such that $\Gamma$ is not solvable. Let $\beta$ be a basis of $V$ such that ${\bf f}_{\beta}^{\phi_{\beta}}={\bf f}_{\beta}$ and let $\phi=\phi_{\beta}$ If $H \le \Gamma $ and  $H \cap GL_n(q^{\bf u})$ consists of scalar matrices, then there exists $b \in \Gamma \cap GL_n(q^{\bf u})$ such that every element of $H^b$ has shape $\phi^i g$ for some $i \in \{1, \ldots, {\bf u} f\}$ and $g \in Z(GL_n(q^{\bf u})).$   
\end{Lem}
\begin{proof}
 Let $Z=Z(GL_n(q^{\bf u}) \cap \Gamma).$ Notice that $\Gamma/Z$ is almost simple. Let $G_0$ and $\hat{G}$ be the socle of $\Gamma/Z$ and the group of inner-diagonal automorphisms of $G_0$ respectively. Therefore, $\hat{G}=(\Gamma \cap GL_n(q^{\bf u}))/Z.$ Without loss of generality, we may assume $Z \le H.$ Observe $H\cap GL_n(q^{\bf u}) = Z,$ so $H/Z$ is cyclic and consists of field automorphisms of $G_0.$ Let $\varphi \in H$ be such that $\langle Z \varphi \rangle=H/Z.$   By Lemma \ref{uniGamsdp},
$$\Gamma=(\Gamma \cap GL_n(q^{\bf u})) \rtimes \langle \phi \rangle,$$
so $\varphi \in \phi^i (\Gamma \cap GL_n(q^{\bf u}))$ for some  $i \in \{1, \ldots, {\bf u} f\}$ and $Z\varphi \in (Z\phi^i) \hat{G}$.

By \cite[(7-2)]{conjaut}, $Z \varphi$ and $Z \phi^i$ are conjugate in $\hat{G}$, so   there exists  $Zb \in \Gamma/Z \cap PGL_n(q^{\bf u})$ such that $(Z\varphi)^{Zb}=Z\phi^i$ for some $i \in \{1, \ldots, {\bf u} f\}.$  Therefore, $H^b = Z\langle \varphi^b \rangle = Z\langle \phi^i \rangle = \langle\phi^i \rangle Z.$
\end{proof}


\begin{Lem}\label{al}
For every prime power $q=p^f$ there  exists $\alpha \in \mathbb{F}_{q^2}$ such that $\alpha+ \alpha^q=1.$
\end{Lem}
\begin{proof}
If $p \ne 2$, then  $2^{-1} \in \mathbb{F}_q^{*}$, so $2^{-1} + (2^{-1})^q=2^{-1} + 2^{-1}=1.$

Let $p=2.$ Let $y \in \overline{\mathbb{F}_2}$ be a root of polynomial $x^q+x+1=0$. Hence 
$$y^{q^2}=(y^q)^q=(y+1)^q=y^q+1=y+1+1=y,$$
so $y \in \mathbb{F}_{q^2}.$
\end{proof} 

\begin{Lem}
\label{pj10}
Let $\eta$ be a generator of $\mathbb{F}_{q^2}^*$ and let $\theta=\eta^{q-1}.$ If $\theta^{p^j-1}=1$ for some $j \in \{0,1, \ldots, 2f-1\}$, then $j=0.$
\end{Lem}
\begin{proof}
Notice that $|\theta|=p^f+1.$ Let $j \in \{1, \ldots, 2f\}$ be minimal such that $p^f+1$ divides $p^j-1.$ Hence $p^f+1$ divides $(p^{2f}-1,p^j-1)=p^{(2f,j)}-1.$ Therefore, $(2f,j)>f$, so $j=2f.$
\end{proof}


\begin{Th}[Clifford's Theorem]\label{cliff}
 Let $H$ be a normal subgroup of a finite group $G$ and let $V$ be an irreducible $\mathbb{F}[G]$-module for an arbitrary field $\mathbb{F}.$ Let $W$ be an irreducible $\mathbb{F}[H]$-submodule of $V$. 
\begin{itemize}
\item[$(1)$] $V=W_1 \oplus \ldots \oplus W_k$ where $W_i$ is an irreducible $\mathbb{F}[H]$-submodule of $V$ and each $W_i$ has the form $(W)g_i$ for some $g_i \in G.$
\item[$(2)$] If $L_i$ for $i=1, \ldots, t$ is a homogeneous component of $H$ on $V$  and $t>1$, then $$V=L_1 \oplus \ldots \oplus L_t$$ and $\{L_1, \ldots, L_t\}$ is a system of imprimitivity for $G$. 
\end{itemize} 
\end{Th}
\begin{proof}
See \cite[\S 16]{sup}.
\end{proof}




%%%%%%%%%%%%%%%%%%%%%%%%%%%%%%%%%%%%%%%%%%%%%%%%%%%%%%%%%%%%%%%%%%%%%%%%%%%%%%%%%%%%%%%%%%%%%%%%%%%%%

\section{Singer cycles}
\label{sinsec}

\begin{Def} A  {\bf Singer cycle} \index{Singer cycle} of $GL_n(q)$ is a cyclic subgroup of order $q^n - 1$.
\end{Def}

While  Singer cycles are well known, many related statements are  ``folklore" and are often stated without  proof or  reference.
% Sometimes such a behavior leads to confusions. For example, in a number of publication, where the author is willing to provide a reference for the fact that all Singer cycles are conjugate in $GL_n(q),$ a wrong reference appears, namely \cite[p. 187]{hupp}. This reference takes the reader to a proof of several statements about Singer cycles, but not to the needed one.
Therefore, for completeness, we include statements with a proof and a reference for the earliest proof we found.       


\begin{Lem}[{\cite{singer38}, \cite[Chapter II, \S 7]{hupp}}] \label{singex} 
 A Singer cycle always exists.
\end{Lem}
\begin{proof}
A field $\mathbb{F}_{q^n}$ can be considered as an $n$-dimensional vector space $V=\mathbb{F}_q^n$ over $\mathbb{F}_q.$  Right multiplication by a generator of  $\mathbb{F}_{q^n}^*$ determines a bijective linear map from $V$ to itself of order $q^n-1$. So  $\mathbb{F}_{q^n}^*$  is isomorphic to a cyclic subgroup of $GL_n(q)$ of order $q^n-1$. 
\end{proof}

Moreover, the action of $\mathbb{F}_{q^n}^*$ on the set $V \backslash \{0\}$ is {\bf regular}, since $x(x^{-1}y)=y$ ({\bf transitivity}) and $xy=y$ if and only if $x=1$ ({\bf semiregularity}) for $x,y \in \mathbb{F}_{q^n}^*$.

%If a group is cyclic with  generator $x$, then we sometimes  denote its group algebra over $\mathbb{F}$ by $\mathbb{F}[x]$ instead of $\mathbb{F}[\langle x \rangle]$ for simplicity.


\begin{Lem}\label{por}
If $x \in GL_n(q),$ then $|x| \le q^n-1$.
\end{Lem}
\begin{proof}
 Let $\chi_x(t)$ be the characteristic polynomial of $x$, so by the Cayley--Hamilton Theorem $$\chi_x(x)=0.$$ Therefore,  the dimension of the subalgebra $\mathbb{F}_q[\langle x \rangle]$ of $M_n(q)$, generated by $x$, is at most 
$$\deg \chi_x(t) \le n.$$ So $|\mathbb{F}_q[\langle x \rangle]|\le|\mathbb{F}_q|^n=q^n$ and $|x|\le q^n-1.$
\end{proof}

\begin{Lem}\label{irr}
If $T \le GL_n(q)$ is a Singer cycle, then $T$ is irreducible.
\end{Lem}
\begin{proof}
Let $x$ be a generator of $T$. Obviously, $(q^n-1,p)=1$. By Maschke's Theorem $T$ is completely reducible and $x$ is conjugate to a block-diagonal matrix $$y=\diag [y_1, \ldots, y_k],$$
where $y_i \in GL_{n_i}(q)$, $\sum_{i=1}^k n_{i}=n$ and each $y_i$ generates an irreducible subgroup of $GL_{n_i}(q)$. Therefore, $$|x|=|y|={\rm lcm}(|y_1|, \ldots, |y_k|)\le \prod_{i=1}^k|y_i|.$$
By Lemma \ref{por} $|y_i| \le q^{n_i}-1,$ so  
$$q^n-1=|x|\le \prod_{i=1}^k (q^{n_i}-1),$$
which is true only if $k=1,$ so $T$ is irreducible.
\end{proof}


\begin{Lem}\label{field}
If $T \le GL_n(q)$ is a Singer cycle, then $T \cup \{0\}$  is a field with the usual matrix addition and multiplication.
\end{Lem}
\begin{proof}
Let $x$ be a generator of $T$ and let $\mu_x(t)$ be the minimal polynomial of $x.$ If $0 \ne v \in V$, then $\{v,vx,vx^2, \ldots, vx^{\deg{\mu_x(t)}-1}\}$ spans an $x$-invariant subspace, so, since $x$ acts irreducibly on $V$ by Lemma \ref{irr},  $\deg{\mu_x(t)}=n.$ Therefore, $\mathbb{F}_q[\langle x \rangle]$ has dimension $n$ and $|\mathbb{F}_q[\langle x \rangle]|=q^n,$ so $\mathbb{F}_q[\langle x \rangle]=T \cup \{0\},$ since $\mathbb{F}_q[\langle x \rangle]$ contains $T$ by definition. Thus, $T \cup \{0\}$ is closed under taking sums and multiplication, every non-zero element has an inverse, and the remaining axioms of a field are evident.      
\end{proof}



\begin{Th}[Noether-Skolem Theorem]
Let $R$ be a simple Artinian ring with center $F$ and let $A$, $B$ be simple subalgebras of $R$ which contain $F$ and are finite-dimensional over it. If $\phi: A \to B$ is an isomorphism  which fixes $F$ elementwise,  then there exists an invertible $x \in R$ such that $(a)\phi=x^{-1}ax$ for all $a \in A.$
\end{Th}
\begin{proof}
See \cite[Chapter 4]{herst}.  
\end{proof}  



\begin{Th}
\label{isoextth}
Let $ \psi :{F}_1 \to {F}_2$ be a field isomorphism. If $E_1$ and $E_2$ are isomorphic algebraic field extensions of $F_1$ and $F_2$ respectively, then there is an isomorphism $\phi : E_1 \to E_2$ such that $\phi |_{_{F_1}}=\psi.$ 
\end{Th}
\begin{proof}
See \cite[Theorem 3.20]{extth}.  
\end{proof}  

\begin{Lem}\label{singconj} 
All Singer cycles are conjugate in $GL_n(q).$ 
\end{Lem}
\begin{proof}
Let $T_1$ and $T_2$ be  Singer cycles, and let $A_i$   be the $\mathbb{F}_q$-subalgebra of $M_n(q)$ generated by $T_i$. By Lemma \ref{field}, $A_1$ and  $A_2$ are fields of order $q^n$ and each of them contains  the subalgebra of scalar matrices $F=Z(M_n(q)) \cong \mathbb{F}_q.$ By Theorem \ref{isoextth} there is an isomorphism $\phi: A_1 \to A_2$ fixing $F$ elementwise. Therefore, by the Noether-Skolem Theorem, there exists $x \in GL_n(q)$ such that $A_1^x=A_2$ and in particular $T_1^x=T_2.$
\end{proof} 


%\begin{Cor}\label{reg}
%A Singer cycle $T$ of $GL_n(q)$ acts regularly on the set of non-zero vectors of $V=\mathbb{F}_q^n.$    
%\end{Cor}
%\begin{proof}
%Follows directly from Lemma \ref{conj} and the construction of a Singer cycle given in Lemma \ref{exist}. 
%\end{proof}

\begin{Lem}\label{centr} 
If $A$ is an abelian regular permutation group of degree $n$, then $C_{\Sym(n)}(A)=A.$
\end{Lem}
\begin{proof}
Let $c \in C:= C_{\Sym(n)}(A)$ be such that $1^c=1$ and let $a \in A$ be such that $1^a=i.$ Notice
$$i^c=1^{ac}=1^{ca}=1^{a}=i,$$ so $c$ is trivial. Therefore, 
$$|C|=|C:\Fix_{C}(1)|.$$
By the Fundamental Counting Principle, $|C:\Fix_{C}(1)|$ is the size of the orbit of $C$ containing $1,$ so $|C:\Fix_{C}(1)| \le n=|A|$.
 The claim follows from the inclusion $A \le C$. 
\end{proof}

\begin{Lem}%\label{sinprim}
A Singer cycle is primitive as a linear group.
\end{Lem}
\begin{proof}
Assume that $T$ has a system of imprimitivity
$V_1 \oplus \ldots \oplus V_k,$
so $k \ne 1$ divides $n$ and for each $t \in T$ and $i \in \{1,\ldots, k\}$
 $$(V_i)t=V_j$$
for some $j \in \{1,\ldots k\}.$ Let $v_1$ and $v_2$ be non-zero vectors from $V_1$ and $V_2$ respectively. Since $T$ acts regularly on the set of non-zero vectors, there exists $t \in T$ such that $(v_1)t=v_1+v_2$. Therefore, $(v_1)t$ does not lie in any $V_i$ which contradicts the assumption.  
\end{proof}

\begin{Lem}\label{sin2}
Let $T \le GL_n(q)$ be a Singer cycle, let $g \in GL_n(q)$, and let $\phi$ be the field automorphism such that $(h^{\phi})_{i,j}=(h_{i,j}^p)$ for $h \in GL_n(q)$. Let  $\psi=g \phi^j$, where $$j \in \{0,1, \ldots , f-1\}.$$    If $T^{\psi}=T$, then $\psi$ acts on $T \cup \{0\}$ as a field automorphism.
\end{Lem}
\begin{proof}
Let $x \in \{g, \phi\}.$ It suffices to show that  $x$ induces an isomorphism of fields between $T \cup \{0\}$ and $T^x \cup \{0\}$. Since the action induced by $x$ is clearly a group isomorphism, we only need to show that $x$ preserves sums.

If $x=g \in GL(n,q)$ then the statement follows from properties of matrix addition and multiplication. If $x=\phi$, then  \begin{equation*}(t_1 +t_2)^x=((t_1 +t_2)(i,j))^p=(t_1(i,j))^p+(t_2(i,j))^p=t_1^x+t_2^x. \qedhere
\end{equation*}


\end{proof}




\begin{Th}[{\cite[Chapter II, \S 7]{hupp}}]\label{sin1} 
If $T$ is a Singer cycle of $GL_n(q)$, then $$C_{GL_n(q)}(T) = T$$
and $N_{GL_n(q)} (T) /T$ is cyclic of order $n$. Moreover, $N_{GL_n(q)} (T)=T \rtimes \langle \varphi \rangle$, where $t^{\varphi}=t^q$ for $t \in T.$ 
\end{Th}
\begin{proof}
%The lemma follows by \cite[p. 187]{hupp},  \cite[Section 1.2]{hiss}.
Consider $T$ as a cyclic regular subgroup of $\Sym(\mathbb{F}_q^n \backslash \{0\}).$  Lemma \ref{centr} implies the first claim  since $GL_n(q)$ is isomorphic to a subgroup of  $\Sym(\mathbb{F}_q^n \backslash \{0\})$.

Let $h \in N_{GL_n(q)} (T)$, so $h$ induces a field automorphism $\varphi$ of $K=T \cup \{0\}$ by Lemma \ref{sin2}. Since $h \in GL_n(q)$, $\varphi$ stabilises the subfield $F \cong \mathbb{F}_q$ of $K$ consisting of scalar matrices, so $|\varphi|$ divides $|\mathbb{F}_{q^n}:\mathbb{F}_q|=n$ and $t^h=t^{\varphi}=t^{q^{(n/|\varphi|)}}.$ If $h_1$ and $h_2$ induce the same automorphism $\varphi$, then $h_2h_1^{-1} \in C_{GL_n(q)}(T)=T,$ so 
$$|N_{GL(n,q)} (T) /T|\le |\mathbb{F}_{q^n}:\mathbb{F}_q|=n.$$

Now we show that every such automorphism is induced by some element of $GL_n(q).$ Recall that we can identify $T$ with $\mathbb{F}_{q^n}^*$ acting on $\mathbb{F}_{q^n}$ as on an $n$-dimensional vector space over $\mathbb{F}_q$ by multiplication, so we can consider vectors in $\mathbb{F}_q^n$ as elements of $\mathbb{F}_{q^n}.$   % We again identify $T$ with $\mathbb{F}_{q^n}^*$ and $\mathbb{F}_q^n \backslash \{0\}.$ 
Let $\varphi \in \Gal(\mathbb{F}_{q^n}:\mathbb{F}_q),$ so for $v_1, v_2 \in \mathbb{F}_q^n,$ $\lambda \in \mathbb{F}_q$ 
$$(v_1+ \lambda v_2) \varphi = (v_1) \varphi+ \lambda^{\varphi} (v_2)\varphi=(v_1) \varphi+ \lambda (v_2)\varphi.$$

Therefore, $\varphi$ acts linearly on $V=\mathbb{F}_q^n$, so $\varphi \in GL_n(q).$  
\end{proof}






%\section{Subgroups of a Singer cycle}

\begin{Lem}\label{irrsub}
A proper subgroup $C$ of a Singer cycle $T \le GL_n(q)$ is irreducible if and only if $|C|$ does not divide $q^r-1$ for every proper divisor $r$ of $n$. 
\end{Lem}
\begin{proof}
Since all subgroups of a cyclic group are cyclic, $C=\langle \sigma \rangle$ for some $\sigma \in T.$

Assume that $C$ is reducible, so there is a non-zero proper  subspace $V_1$ of $V=\mathbb{F}_q^n$ such that $$(V_1)\sigma=V_1.$$ Therefore, $V_1 \backslash \{0\}$ is a collection of $m$ orbits of $C$ acting on all non-zero vectors. Since $T$ is regular on that set, $C$ is semiregular and all orbits have size $|C|.$ Thus, $$q^r-1 =|V_1 \backslash \{0\}|=m|C|$$ for some  $r<n$. Thus, $|C|$ divides $q^r-1$ and $r$ divides $n$ by Clifford's Theorem. 

Let $r$ be the smallest divisor of $n$ such that $|C|$ divides $q^r-1.$ Then $C$ lies in the unique subgroup $S=\langle x^s \rangle$ of $T$ of order $q^r-1$, where $x$ is a generator of $T$ and $s=(q^n-1)/(q^r-1).$ Since $S$ is  unique, $S \cup \{0\}$ is a subfield of $T \cup \{0\}$. Therefore, $\dim (\mathbb{F}_q[\langle \sigma \rangle]) \le r$, which implies $\deg \mu_{\sigma}(t) \le r$ and  
$$\{v,v\sigma,v\sigma^2, \ldots, v\sigma^{\deg{\mu_{\sigma}(t)}-1}\}$$
spans a proper $\sigma$-invariant subspace of $V.$
\end{proof}



\begin{Lem}\label{semis}
An irreducible cyclic subgroup  of $GL_n(q)$ is  contained in some Singer cycle. %There exist irreducible cyclic subgroup of $GL_n(q)$ of order $m$ if and only if $m$ divides $q^n-1$ and $m$ does not divide 
\end{Lem}
\begin{proof}
Let $C=\langle \sigma \rangle$ be an irreducible cyclic subgroup  of $GL_n(q)$. For  non-zero $v \in V$,  as in the proof of Lemma \ref{field}, $$\{v,v\sigma,v\sigma^2, \ldots, v\sigma^{\deg{\mu_{\sigma}(t)}-1}\}$$
spans a $\sigma$-invariant subspace of $V.$ Hence $\deg{\mu_{\sigma}(t)}=n$ and
$\dim (\mathbb{F}_q[\langle \sigma \rangle])=n.$   Moreover, all non-zero matrices in $\mathbb{F}_q[\langle \sigma \rangle]$ are invertible. Indeed,
assume that $$\alpha_0 + \alpha_1 \sigma + \ldots + \alpha_{n-1}\sigma^{n-1}$$
is non-zero and non-invertible. Therefore, there exists a non-zero $v \in V$ such that 
$$v(\alpha_0 + \alpha_1 \sigma + \ldots + \alpha_{n-1}\sigma^{n-1})=0,$$
so $$\{v, v \sigma, \ldots , v\sigma^{n-2}\}$$ 
spans an $\sigma$-invariant subspace. This subspace is spanned by $n-1$ vectors, so it is the zero subspace since $\sigma$ is irreducible. Hence $v=0$, which  contradicts  our choice.   


Thus, $\mathbb{F}_q[\langle \sigma \rangle]$ is a field, so $C$ is a subgroup of its multiplicative group which is a Singer cycle.
\end{proof}

\begin{Cor}\label{semis1}
There exists a cyclic irreducible subgroup of order $m$ in $GL_n(q)$ if
and only if $m$ divides $q^n - 1$ but $m$ does not divide $q^d - 1$ for every positive
integer $d < n$.
\end{Cor}
\begin{proof}
The corollary follows from Lemmas \ref{irrsub} and \ref{semis} and the fact that 
\begin{equation*}(q^n-1,q^d-1)=q^{(n,d)}-1.
 \qedhere 
\end{equation*}  
\end{proof}

\begin{Lem}[{\cite[(2.6)]{sim}}]
A proper subgroup $C$ of a Singer cycle $T \le GL_n(q)$ is primitive if and only if $|C|$ does not divide $r(q^{n/r}-1)$ for every prime divisor $r$ of $n$. 
\end{Lem}
\begin{proof}
If $C$ is reducible, then $|C|$ divides $q^{n/r}-1$ for some prime $r$ by Lemma \ref{irrsub}. So let $C$ be irreducible.

Suppose $C$ has a system of imprimitivity $\{V_1, \ldots, V_k\},$ where $k>1.$ Then $C$ permutes the $V_i$ transitively because of irreducibility. So $k$ divides $|C|$ and $n$, but $k$ is not divisible by $p$. Let $\pi: C \to \Sym(k)$ be the homomorphism arising from the action of $C$ on  
$\{V_1, \ldots, V_k\},$ so $\pi(C)$ is cyclic of order $k.$ If $k$ is not prime, then let $k/r=l>1$ for some prime $r$ and let $\langle h \rangle=\pi(C).$ So $\{V_1, \ldots, V_k\}$ is a disjoint union of  $\langle h^r \rangle$-orbits $\{V_{i_1}, \ldots, V_{i_l}\}$ for $i=1, \ldots, r.$ Define $U_i=V_{i_1} \oplus \ldots \oplus V_{i_l}.$ Now
$$\{U_1, \ldots, U_r\}$$ is a $C$-system of imprimitivity. Therefore, $C$ always has a system of imprimitivity of prime size and we assume $k=r$, so $\dim V_i =n/r$. Consider the stabiliser $D$ of $V_1$ in $C$, so $D=\text{ker }\pi$  consists of all $g \in C$ such that $(V_1)g=V_1.$ Hence $|D|=|C|/|\text{Im }\pi|=|C|/r.$ Since $V$ is a faithful irreducible $\mathbb{F}_q[C]$-module, $V_1$ must be a faithful irreducible $\mathbb{F}_q[D]$-module. Therefore, $|D|$ divides $q^{n/r}-1$ by Lemma \ref{semis} and $|C|$ divides $r(q^{n/r}-1).$

Suppose  that  $r$ is a prime divisor of $n$ such that $|C|$ divides ${r(q^{n/r}-1)}.$ If $r$ does not divide $|C|,$ then $|C|$ divides $q^{n/r}-1,$ so $C$ is reducible. Let $r$ divide $|C|$ and let $D$ be the unique subgroup of index $r$ in $C,$ so $C/D=\{D,Dc, \ldots, Dc^{r-1}\}$ for some  $c \in C.$ Since $|D|$ divides $q^{n/r}-1$, by Corollary \ref{semis1} the dimension of an irreducible $\mathbb{F}_q[D]$-submodule $V_1$ of $V$ is at most $n/r$ and, by the proof of Lemma \ref{irrsub}, it divides $n/r.$
Since $D$ is normal in $C$, $(V_1)g$ is also an irreducible $\mathbb{F}_q[D]$-submodule of $V$ for every $g \in C$. Therefore, either $V_1=(V_1)g$ or $V_1 \cap (V_1)g=0$ because the intersection is also a $\mathbb{F}_q[D]$-submodule and $V_1$ is irreducible. Thus, the sum $V_1 + (V_1)g$ is direct. Notice that for every $g \in C$ there exists $d \in D$ and $i \in \{0, \ldots, r-1\}$ such that  $g = dc^i.$ Hence $(V_i)g$ has the form $(V_1)c^i.$  It is easy to see that $$V_1 \oplus V_1c \oplus \ldots \oplus V_1c^{r-1} $$
is a $C$-invariant subspace. Therefore, since $C$ is irreducible, 
$$V=V_1 \oplus V_1c \oplus \ldots \oplus V_1c^{r-1}.$$
Hence $\dim V_1 = n/r$ and $C$ permutes the set $\{V_1, V_1c, \ldots,V_1c^{r-1}\}$, so it is a system of imprimitivity. 
\end{proof}
 

\begin{Lem}
If $C$ is an irreducible subgroup of a Singer cycle $T \le GL_n(q)$ then $$C_{GL_n(q)}(C)=T \text{ and } N_{GL_n(q)}(C)=N_{GL_n(q)}(T).$$
\end{Lem}
\begin{proof}
Let $C= \langle \sigma \rangle.$  Since $\mathbb{F}_q[\langle \sigma \rangle]$ is a subalgebra of $T \cup \{0\}$, every non-zero element of $\mathbb{F}_q[\langle \sigma \rangle]$ is invertible, so $\mathbb{F}_q[\langle \sigma \rangle]$ is a subfield of $T \cup \{0\}$. Therefore, $\mathbb{F}_q[\langle \sigma \rangle]^*$ has order $q^m-1$ for some divisor $m$ of $n$. Lemma \ref{irrsub} implies that  $|\sigma|$ does not divide $q^m-1$ for any proper divisor $m$ of $n$, so $m=n.$ Thus, every element of $T$ can be represented as a linear combination of powers of $\sigma.$ Hence every element centralising $\sigma$ must centralise $T$ and every element normalising $\sigma$ must normalise $T.$ The inclusions 
$$C_{GL_n(q)}(C) \ge T \text{ and } N_{GL_n(q)}(C) \ge N_{GL_n(q)}(T)$$
are straightforward.
\end{proof}

%\begin{Lem}
%Let $M$ be  a subgroup of the normaliser $N$ of a Singer cycle $T \le GL_n(q)$. If $A:=M\cap T$ is reducible, then $M$ is also reducible.
%\end{Lem}
%\begin{proof}
%   Let $r$ be minimal such that $1<r<n$ and $|A|$ divides $q^r-1$, it exists by Lemma \ref{irrsub}
%since $A$ is reducible. Therefore, since $T$ is cyclical and there is the unique subgroup of each  order dividing $q^n-1,$ $A$ lies in the subfield of $T\cup \{0\}$ of order $q^r.$
%\end{proof}

\begin{Lem}\label{sin}
Let $T \le GL_n(q)$ be a Singer cycle and let $t_1, t_2 \in T$. Let $N$ be the normaliser of $T$ in $GL_n(q).$
If $t_2 =t_1^g$ for $g \in GL_n(q),$ then there exists $g_1 \in N$ such that $t_2=t_1^{g_1}.$
\end{Lem}
\begin{proof}
Let $\overline{G}$ be the algebraic group $GL_n( \overline{\mathbb{F}}),$ where $\overline{\mathbb{F}}$ is the algebraic closure of the field $\mathbb{F}_q$, and let $$\sigma : \overline{G} \to \overline{G}$$
be the Frobenius map $$ (a_{ij}) \mapsto (a_{ij}^q).$$
 By Lemma \ref{sin1}, $T$ is the set  of $\sigma$-fixed points of some maximal $\sigma$-stable torus $\overline{T} $ of  $\overline{G}.$ 
Since all maximal tori are conjugate in $\overline{G}$, there exists $h \in \overline{G}$ such that $\overline{T}=\overline{D}^h$, where $\overline{D}$ is the maximal torus consisting of diagonal matrices. Since $T$ is cyclic,  \cite[Lemma 1.2 and Proposition 2.1]{buturl} imply that  $T=(\overline{D}_{\sigma w})^{h},$ where $w$ is a permutation matrix representing a cycle of length $n$ and acting on $\overline{D}$ by conjugation. For $\alpha \in \Aut(\overline{G})$ and $\alpha$-invariant subgroup $\overline{H} \le \overline{G}$   we denote the subgroup of $\alpha$-invariant elements by $\overline{H}_{\alpha}.$ Without loss of generality, we can assume that $w$ represents the cycle $(1, 2, \ldots, n).$ Therefore, by \cite[Lemma 1.3]{buturl},
\begin{equation}\label{sopr1}
\overline{D}_{\sigma w}= \{\diag(\lambda, \lambda^q, \ldots, \lambda^{q^{n-1}}) \mid \lambda^{q^n-1} = 1, \lambda \in \overline{\mathbb{F}}\}  
\end{equation}
and
\begin{equation*}%\label{sopr2}
N=(\overline{D}_{\sigma w} \rtimes \langle w \rangle)^h.
\end{equation*}

Since $t_1$ and $t_2$ are conjugate, they have the same eigenvalues, so $t_1^{h^{-1}}$ and $t_2^{h^{-1}}$ are diagonal matrices in $\overline{D}_{\sigma w}$ with the same entries up to permutation. Therefore,
\begin{align*}t_1^{h^{-1}} & = \diag(\beta, \beta^q, \ldots, \beta^{q^{n-1}})\\
t_2^{h^{-1}} &  =  \diag(\delta, \delta^q, \ldots, \delta^{q^{n-1}})
\end{align*}
where $\beta, \delta \in \mathbb{F}_{q^n}^*$ and $\delta = \beta^{q^k}$ for some $1\le k<n.$ Thus, 
$$t_2^{h^{-1}}=\diag( \beta^{q^{k}}, \beta^{q^{k+1}}, \ldots, \beta^{q^{n-1}}, \beta, \beta^{q}, \ldots, \beta^{q^{k-1}}  ); \text{  } \beta^{q^n-1} = 1.$$
So $t_1^{h^{-1}}$ and $t_2^{h^{-1}}$ are conjugate by a power of $w$ which lies in $N^{h^{-1}}$. Therefore, $t_1$ and $t_2$ are conjugate in $N$. 
\end{proof}

Notice that, by \eqref{sopr1}, if $t \in GL_n(q)$ generates a Singer cycle, then $\det(t)$ generates $\mathbb{F}_q^*,$ so $\langle t \rangle SL_n(q)=GL_n(q)$ by Lemma \ref{supSL}.


%%%%%%%%%%%%%%%%%%%%%%%%%%%%%%%%%%%%%%%%%%%%%%%%%%%%%%%%%%%%%%%%%%%%%%%%%%%%%%%%%%%%%%%%%%%%%%%%%5

\section{Fixed point ratios and elements of prime order}
\label{fprsec}

\begin{Def}
If a group $G$ acts on a set $\Omega$, then   $C_{\Omega}(x)$ is the set of points in $\Omega$  fixed
by   $x \in G$. If $G$ and $\Omega$ are finite, then  the {\bf fixed point ratio} \index{fixed point ratio} of $x$, 
denoted  by $\fpr(x)$, is the proportion of points in $\Omega$ fixed by $x$, i.e. $\fpr(x) = |C_{\Omega}(x)|/|\Omega|.$
\end{Def}

For completeness, we include a proof of the following well-known result. 

\begin{Lem}\label{fpr}
If  $G$ acts transitively on a set $\Omega$  and $H$ is a point stabiliser,  then 
$$\fpr(x)= \frac{|x^{G} \cap H|}{|x^G|}$$ for  $x \in G.$
\end{Lem}

\begin{proof}

Let $\{1=g_1, g_2, \ldots , g_k\}$ be a right transversal for  $H$ in $G$. The action is transitive,
so  $\{H, H^{g_2}, \ldots , H^{g_k}\}$ is the set of stabilisers of all points. Observe
$$|C_{\Omega}(x)|=|\{i \in [1, \ldots ,k ] \mid x \in H^{g_i}\}|= \frac{|\{g \mid x^{g^{-1}} \in H\}|}{|H|}=  \frac{|x^G \cap H||C_{G}(x)|}{|H|}.$$
On the other hand, $|\Omega|=|G:H|=\frac{|x^G||C_{G}(x)|}{|H|}.$ Thus,
\begin{equation*}
\fpr(x)=\frac{|C_{\Omega}(x)|}{|\Omega|}=\frac{|x^G \cap H|}{|x^G|}. \qedhere 
\end{equation*}
\end{proof}

 In \cite{fpr, fpr2, fpr3, fpr4} Burness studies   fixed point ratios in classical groups. Recall some  observations from \cite{fpr}.
Let a group $G$ act faithfully   on the set $\Omega$ of right cosets of a subgroup $H$ of $G.$  
Let $Q(G, c)$ be the probability that
a randomly chosen $c$-tuple of points in $\Omega$ is not a base for $G$, so $G$ admits a base of size $c$ if and
only if $Q(G, c) < 1$. Of course, a $c$-tuple is not a  base if and only if it is fixed by  $x \in G$ of prime order, and the probability that a random $c$-tuple is fixed
by $x$ is equal to $\fpr(x)^c$. Let $\mathscr{P}$ be the set of elements of prime order in ${G}$, and let ${x}_1, \ldots, {x}_k$
be representatives for the ${G}$-classes of elements in $\mathscr{P}$. Since  fixed point
ratios are constant on conjugacy classes (see Lemma \ref{fpr}), 
\begin{equation}\label{ver}
Q(G,c) \le \sum_{{x} \in \mathscr{P}}\fpr({x})^c = \sum_{i=1}^{k}|{x_i}^{{G}}|\cdot\fpr({x}_i)^c=:\widehat{Q}(G,c).
\end{equation}

 
\begin{Lem}[{\cite[Lemma 2.1]{burness}}]
\label{fprAB}
Let $G$ act faithfully and transitively on $\Omega$ and let $H$ be a point stabiliser. If  $x_1,\ldots,x_k$ represent distinct $G$-classes such that $\sum_{i=1}^k |x_i^G \cap  H| \le A$ and $|x_i^G| \ge B$ for all $i \in \{1, \ldots, k\},$ then
$$\sum_{i=1}^m |x_i^G| \cdot \fpr (x_i)^c \le B \cdot (A/B)^c.$$ 
for all $c \in \mathbb{N}.$
\end{Lem}


If there exists $\xi \in \mathbb{R}$ such that $\fpr({x})\le |{x}^{{G}}|^{-\xi}$ for every  ${x} \in \mathscr{P}$,  then  
$$\widehat{Q}(G,c)\le \sum_{i=1}^{k}|{x}_i^{{G}}|^{1-c\xi}.$$



\begin{Def}
Let $\mathscr{C}$ be the set of conjugacy classes of prime order elements in ${G}$.
For $t \in \mathbb{R},$ 
$$\eta_G(t):= \sum_{C \in \mathscr{C}}|C|^{-t}.$$
If $Z(G)=1,$ then there exists $T_G \in \mathbb{R}$ such that $\eta_G(T_G) = 1.$
\end{Def}





\begin{Lem}\label{11}
	If $G$ acts faithfully and transitively on $\Omega$ and  $\fpr({x})\le|{x}^{{G}}|^{-\xi}$ for all  ${x} \in \mathscr{P}$ and $T_G < c\xi - 1$, then $b(G) \le c$.
\end{Lem}
\begin{proof}
We follow the proof of \cite[Proposition 2.1]{burness}. Let ${x}_1, \ldots, {x}_k$ be representatives of the ${G}$-classes
of prime order elements in ${G}$. %, and let $Q(G, c)$ be the probability that a randomly chosen
%$c$-tuple of points in $\Omega$ is not a base for $G$. Evidently, $G$ admits a base of size $c$ if and only if
%$Q(G, c) < 1.$ 
By \eqref{ver}, 
$$Q(G,c) \le  \sum_{i=1}^{k}|{x_i}^{{G}}|\cdot \fpr({x}_i)^c \le \eta_G(c\xi - 1).$$
The result follows since $\eta_G(t) < 1$ for all $t>T_G.$
\end{proof}

%\begin{Def} \label{algdef}
We fix the following notation for the rest of the section. Let $\overline{G}$ be an adjoint simple algebraic group of  type $A_{n-1}$ or $C_{n/2}$ over the algebraic closure of ${\mathbb{F}_p}$. Let $\overline{G}_{\sigma}=\{g \in \overline{G} \mid g^{\sigma}=g\}$ where $\sigma$ is a Frobenius morphism of $\overline{G}.$ Let $\overline{G}$ be such that $G_0=O^{p'}(\overline{G}_{\sigma})'$ is a finite simple group. Here  $O^{p'}(G)$ is the subgroup of a finite group $G$ generated by all $p$-elements of $G$. Therefore, $\overline{G}_{\sigma}=PGL_n^{\varepsilon}(q)$ and $G_0=PSL_n^{\varepsilon}(q)$ for type $A_{n-1}$; also $\overline{G}_{\sigma}=PGSp_n(q)$ and $G_0=PSp_n(q)'$ for type $C_{n/2}.$ Let  $G$ be a finite almost simple  group with  socle $G_0.$  
%\end{Def}

As proved in \cite[Proposition 2.2]{burness},  if  $n \ge 6$, then $T_G$ exists and $T_G<1/3$. 
Thus, if for such $G$  
\begin{equation}\label{0} 
\fpr(x)<|x^G|^{-\frac{4}{3c}}
\end{equation}
 for all $x \in \mathscr{P} $, then $\xi \ge {4}/{(3c)}$ and $c\xi -1 \ge 1/3 >T_G$ and $G$ has a base of size $c$.

Therefore, Lemma \ref{11} allows us to estimate the base size by calculating bounds for $|x^G|$ and $|x^G \cap H|$ for elements $x$ of prime order. 


\begin{Def}
\label{H1EE0}
Let $A$ be a group and let $\varphi: A \to A$ be a homomorphism. Let $H^1(\varphi,A)$  denote the set of equivalence classes of $A$ corresponding to the equivalence relation:
$$x \sim y \text{ if and only if } y=z^{-1}xz^{\varphi} \text{ for some } z \in A.$$
Let $x \in \overline{G}_{\sigma}$ and let $E=C_{\overline{G}}(x).$ Notice that $\sigma$ induces a homomorphism $\sigma : E/E^0 \to E/E^0$, where $E^0$ is the connected component of $E$ containing $1.$ Let $H^1(\sigma, E/E^0)$ be the set of equivalence classes of elements of $E/E^0$ corresponding to that induced homomorphism.
\end{Def}

%\begin{Lem}[{\cite[I, 2.7]{spriste}}]
% If $x \in \overline{G}$, then $x^{\overline{G}} \cap \overline{G}_{\sigma}$ is a union of precisely $|H^1 (\sigma, E/E^0 )|$ distinct
%$\overline{G}_{\sigma}$-conjugacy classes, where $E=C_{\overline{G}}(x).$ 
%\end{Lem}


\begin{Def}
\label{nudef}
Let $x \in PGL(V)=PGL_n(q)$. Let $\overline{\mathbb{F}}$ be the algebraic closure of $\mathbb{F}_q$, and let $\overline{V}=\overline{\mathbb{F}} \otimes V.$ Let $\hat{x}$ be the preimage of $x$ in $GL(n,q).$ Define
$$ \nu_{V, \overline{\mathbb{F}}}(x):= \min\{\dim [\overline{V}, \lambda \hat{x}] : \lambda \in \overline{\mathbb{F}}^* \}. $$ Here $[V,g]$ for a vector space $V$ and $g \in GL(V)$ is the commutator in $V \rtimes GL(V)$.  Therefore, $\nu_{V, \overline{\mathbb{F}}}(x)$ is  the minimal codimension of an eigenspace of $\hat{x}$  on $\overline{V}$.  Sometimes we denote this number by $\nu(x)$ and $\nu_{V, \overline{\mathbb{F}}}(\hat{x})$. 
\end{Def}

\begin{Lem}[{\cite[Lemma 3.11]{fpr2}}]\label{prost}  %\label{prost}
Let $x \in PGL^{\varepsilon}_n(q)$ have  prime order $r.$ One of the following holds: 
\begin{enumerate}[font=\normalfont]
\item $x$ lifts to  $\hat{x} \in GL^{\varepsilon}_n(q)$ of order $r$ such that $|x^{PGL^{\varepsilon}_n(q)}|=|\hat{x}^{GL^{\varepsilon}_n(q)}|;$ \label{prost1}
\item $r$ divides both $q-\varepsilon$ and $n$, and $x$ is $PGL_n(\overline{\mathbb{F}})$-conjugate to the image of $$\diag[I_{n/r}, \omega I_{n/r}, \ldots , \omega^{r-1} I_{n/r} ],$$ where $\omega \in \overline{\mathbb{F}}$ is a primitive $r$-th root of unity. \label{prost2}
\end{enumerate} 
\end{Lem}

\begin{Rem}
Lemma 3.11 from \cite{fpr2} is formulated for   all classical groups, but only for $r \ne 2$. It is easy to see  from its proof  that the condition $r\ne 2$ is necessary only for orthogonal and symplectic cases; if $x \in PGL^{\varepsilon}_n(q)$ then the statement is true for arbitrary prime $|x|.$  
\end{Rem}



%In the following lemmas we use notation from Definition \ref{algdef}.
\begin{Lem}\label{xGoGs}
Let $x \in \overline{G}_{\sigma}$ have prime order.
\begin{enumerate}[font=\normalfont]
\item If $x$ is semisimple, then $x^{\overline{G}_{\sigma}}=x^{G_0}.$
\item If $x$ is unipotent and $G_0=PSL_n^{\varepsilon}(q)$, then $|x^{\overline{G}_{\sigma}}| \le \min\{n,p\}|x^{G_0}|.$
\item If $x$ is unipotent, $p\ne 2$ and $G_0=PSp_n(q)$, then $|x^{\overline{G}_{\sigma}}| \le 2|x^{G_0}|.$  
\end{enumerate} 
\end{Lem}
\begin{proof}
See \cite[4.2.2(j)]{gorlyo} for the proof of $(1)$ and \cite[Lemma 3.20]{fpr2} for $(2)$ and $(3).$
\end{proof}
Notice that if $p=2$, $(n,q)\ne (4,2)$ and $\overline{G}_{\sigma}$ is symplectic, then $\overline{G}_{\sigma}=G_0.$

\begin{Lem}
Let $x \in G$ have prime order $r$ and $s:=\nu(x).$
\begin{enumerate}[font=\normalfont]
\item In case ${G_0}=PSL_n^{\varepsilon}(q)$:
\begin{equation}\label{5uni}
 |x^G| > 
\begin{cases}
  \frac{1}{2t} \left(\frac{q}{q+1} \right)^{as/(n-s)} q^{ns} \ge \frac{1}{2t} \left(\frac{q}{q+1} \right)^{as/(n-s)} q^{n^2/2}  & \text{ for } s \ge n/2;\\
  \frac{1}{2t} \left(\frac{q}{q+1} \right)^{a} q^{2s(n-s)} \ge  
\frac{1}{2t} \left(\frac{q}{q+1} \right)^{a}  q^{(3/8)n^2}  & \text{ for }  n/4 \le s < n/2,
\end{cases}
\end{equation}
where $t = \min\{r,n\}$ and $a=(1/2)(1- \varepsilon 1)$.
\item In case $G_0=PSp_n(q)'$:
\begin{equation}\label{5simp}
|x^G| > 
  \frac{1}{8} \left(\frac{q}{q+1} \right) \max(q^{s(n-s)}, q^{(ns/2)}).
\end{equation}
\end{enumerate}
\end{Lem}
\begin{proof}
The statement follows by Lemma \ref{xGoGs} and \cite[Propositions 3.22 and 3.36, Lemmas 3.34 and 3.38]{fpr2}.
\end{proof}



%%%%%%%%%%%%%%%%%%%%%%%%%%%%%%%%%%%%%%%%%%%%%%%%%%%%%%%%%%%%%%%%%%%%%%%%%%%%%%%%%%%%%%%%%%%

\section{Computations using {\sf GAP} and {\sc Magma}}
\label{gapsec}

We use the  computer algebra systems {\sf GAP} \cite{GAP4} and {\sc Magma} \cite{magma}  to check the inequality $b_S(S \cdot(SL_n(q^{\bf u}) \cap G))\le c$ for  particular cases where $S \le G$ is a maximal solvable subgroup of $G \in \{GL_n(q), GU_n(q), GSp_n(q) \}$ and $q$ and $n$ are small. In this section we discuss how the statement $b_S(G)\le c$ can be checked for given  $S \le G$ and integer $c$, and how to construct desired $S$ and $G$ using {\sf GAP} and {\sc Magma}.

Since a nontrivial transitive permutation group on $\Omega$ has base size $c>0$ if and only if the stabiliser of  $\omega \in \Omega$ acting on $\Omega \backslash \{\omega\}$ has base size $c-1$, it is enough to check that the action of the stabiliser of $S$ in $G/S_G$ on $(\Omega \backslash \{S\})^{c-1}$ has a regular point. Here $\Omega$ is the set of right cosets of $S$ in $G$. The following code in {\sf GAP} checks the existence of such an orbit  and returns {\sf true} if $b_S(G) \le c$: 

\lstset{
basicstyle=\ttfamily}  
{\small
\begin{lstlisting}
 gs:=RightCosets(G,S);;
 hom:=Action(G,gs,OnRight);
 SS:=Stabilizer(hom,1);
 m:=Size(SS);
 k:=Order(G)/Order(S);
 Orb:=OrbitsDomain(SS,Arrangements([2..k],c-1),OnTuples);;
 OrbReg:=Filtered(Orb,x->(Size(x)=m));;
 Size(OrbReg)>0;
\end{lstlisting}}

 This procedure works well when $|G:S|$ is relatively small. For example, if $|G:S|$ is at most $4000$, then this code executes in {\sf GAP} 4.11.1 in at most 345 seconds using the default memory allocation of 256 MB  on a machine with a 2.6 GHz processor. Another way to establish  $b_S(G)\le c$, that we use in most situations, is to find $a_1, \ldots, a_{c-1}$ such that 
\begin{equation}
\label{gapint}
S \cap S^{a_1} \cap \ldots \cap S^{a_{c-1}} =S_G=\cap_{g \in G} S^g.
\end{equation}
 The function {\sf Random}$(G)$, in both {\sf GAP} and {\sc Magma}, allows us to find such $a_i$ in most cases we consider. 

A more difficult  task is to construct desired (or all up to conjugation) maximal solvable subgroups of $G$.  Generating sets for primitive maximal solvable subgroups of $GL_n(q)$ for small $n$ (and for subgroups containing primitive maximal solvable subgroups of $GL_n(q)$ in the general case) can be found in \cite{short} and \cite[\S 21]{sup}. The function {\sf IrreducibleSolvableGroupMS}$(n,p,i)$ in the {\sf GAP} package {\sf PrimGrp} \cite{PrimGrp} realises the results of \cite{short}. It returns a representative of the $i$-th conjugacy class of irreducible solvable subgroups of $GL_n(p)$, where $n>1$, $p$ is a prime, and $p^n < 256$.   While constructing a specific subgroup can be difficult, it is usually not necessary:  for $M>S$ if
\begin{equation}
\label{gapintM}
M \cap M^{a_1} \cap \ldots \cap M^{a_{c-1}} =S_G,
\end{equation}
then \eqref{gapint} holds, so it is enough to construct an overgroup $M$ of $S$ such that \eqref{gapintM} holds.   The function {\sf ClassicalMaximals} in  {\sc Magma} realises the results of \cite{maxlow}. It  constructs all maximal subgroups of a classical group ($GL_n(q)$, $GU_n(q)$ or $GSp_n(q)$ for example) up to conjugation for $n \le 17$. If a maximal subgroup $M$ is too big and \eqref{gapintM} does not hold, then we use the function {\sf MaximalSubgroups} to obtain all maximal subgroups of $M$ up to conjugation and check if \eqref{gapintM} holds for them. In practice, at most three iterations are needed to obtain \eqref{gapintM} for some overgroup of the solvable subgroup under investigation.



As an example, consider $G=GU_7(2)$ and check in  {\sc Magma} that \eqref{gapintM} with $c=3$ holds for all irreducible maximal subgroups:
{\small
\begin{lstlisting}
> G:=GU(7, 2);
> M:=ClassicalMaximals("U",7,2:classes:={2..9}, general:=true);
> #M;
2
> #ClassicalMaximals("U",7,2:classes:={2..9}, general:=true,
>   novelties:=true);
0
> for i in [1..#M] do
>    H := M[i];
>    x := Random(G);
>    K := M[i]^x;
>    repeat
>       y := Random(G);
>       L := M[i]^y;
>       I := H meet K meet L;
>    until #I eq 3;
>    "Now shown intersection is central for i = ", i;
> end for;
Now shown intersection is central for i =  1
Now shown intersection is central for i =  2 
\end{lstlisting}}

Hence the intersection of three conjugates of an irreducible  maximal subgroup has order $3$ which is exactly the order of $Z(GU_7(2)).$ Since $M[i]$ contains $Z(GU_7(2)),$ 
$$M[i] \cap M[i]^x \cap M[i]^y =Z(GU_7(2))$$
for $i=1,2.$

We often write ``$b_S(G)<c$ is verified by computation''\index{computation} to imply that the statement is verified using one of the procedures described above.   

We often carry out calculations in {\sc Magma} of the following kind: given specified building blocks, we construct block-diagonal matrices to define a subgroup $S$ of $GL_n(q)$ where $n \le 4$ and $q$ is small, often 2; we then show  that the intersection of a specific number of conjugates of $S$ satisfies particular order bounds.  We often write ``calculations show'' to summarise such routine calculations.

