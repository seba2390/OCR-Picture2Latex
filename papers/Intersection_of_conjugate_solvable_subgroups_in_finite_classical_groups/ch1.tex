\chapter{Introduction}

\section{Statement of the problem}

Consider some property $\Psi$ of a finite group inherited by all its subgroups.  Important examples of  such a property are  the following:
\begin{itemize}
\item cyclicity; 
\item commutativity; 
\item nilpotence; 
\item solvability.
\end{itemize}
A natural question arises: how large is a normal $\Psi$-subgroup in an arbitrary finite group $G$? A more precise formulation of this question is the following:

\begin{Que}\label{que1}
Given a finite group $G$ with $\Psi$-subgroup $H$ of index $n$, is it true that $G$ has a normal $\Psi$-subgroup whose index is bounded by some function $f(n)?$
\end{Que}

Since the kernel of the action of $G$ on the set of right cosets of $H$ by right multiplication is a subgroup of $H$ and such an action provides a homomorphism to the symmetric group $\Sym(n),$   it always suffices to take $f(n)=n!$ for every such $\Psi$. We are interested in  stronger bounds, in particular those of shape $f(n)=n^c$ for some constant $c.$



Babai, Goodman and Pyber \cite{bab} prove some related results and state several conjectures.
In particular, they prove that
if a finite group $G$ has a cyclic subgroup ${C}$ of index $n$, then $ \cap_{g \in G} {{C}}^g$ has index at most $n^7.$
They also conjectured that the bound $n^2-n$ holds  and showed that it is best possible. Lucchini \cite{luccini} and, independently,  Kazarin and  Strunkov \cite{kaz} proved this, so for cyclicity the question is resolved. 

\begin{Th}
If a finite group $G$ has a cyclic subgroup ${C}$ of index $n$, then $ \cap_{g \in G} {{C}}^g$ has index at most $n^2-n.$
\end{Th}

The following theorem about commutativity follows from results  by Chermak and  Delgado \cite{cher}:
\begin{Th}
Let $G$ be a finite group. If $G$ has an abelian subgroup of index $n$, then 
 it has a normal abelian subgroup of index at most $n^2.$
\end{Th}
\noindent While this bound is not best possible, it is the best of shape $n^c.$

Zenkov \cite{zen21} proved the following  when $\Psi$ is nilpotence.

\begin{Th}
Let $G$ be a finite group and let ${\bf F}(G)$ be its maximal normal nilpotent subgroup.  If $G$ has a nilpotent subgroup of index $n$, then 
$|G:{\bf F}(G)| \le n^3.$
\end{Th}
Babai, Goodman and Pyber \cite{bab}   proved the following statement.

\begin{Th}\label{indexBab}
There is an absolute constant $c$ such that, if a finite group $G$ has a
solvable subgroup of index $n$, then $G$ has a solvable normal subgroup of index
at most $n^c$.
\end{Th}

\noindent Although their proof does not yield an explicit value, they  conjectured  that $c \le 7$.

This conjecture is closely related to  \cite[Problem 17.41 b)]{kt}:

\begin{Prob}\label{prob}
Let $H$ be a solvable subgroup of a finite group $G$ that has no nontrivial solvable
normal subgroups.
 Do there always exist five conjugates of $H$ whose intersection is trivial?
\end{Prob}

Before we explain how Problem \ref{prob} is related to Question \ref{que1}, we need to introduce some notation. Problem \ref{prob} can be reformulated using the notion of {\bf base size}.

\begin{Def}\label{def1}
Assume that a finite group $G$ acts on a set $\Omega.$ A point $\alpha \in \Omega$ is  $G${\bf-regular}   
 if its stabiliser in $G$ is trivial.
Define the action of  $G$ on $\Omega^k$ by 
$$(\alpha_1, \ldots,\alpha_k)g = (\alpha_1g,\ldots,\alpha_kg).$$
If $G$ acts faithfully and transitively on $\Omega$, then the minimal number $k$ such that the set $\Omega^k$ contains a 
$G$-regular point is the
{\bf base size} \index{base size} of $G$ and is denoted by  $b(G).$ For a positive integer $m$, a regular point in $\Omega^m$  is a {\bf base} \index{base} for the action of $G$ on $\Omega.$
Denote the number of $G$-regular orbits on $\Omega^m$   by $\Reg(G,m)$ (this number is 0 if $m < b(G)$).
If  $G$ acts by  right multiplication on the set $\Omega$ of right cosets of a subgroup $H$, then $G/H_G$ acts faithfully and transitively on $\Omega.$ (Here $H_G=\cap_{g \in G} H^g.$) In this case, we denote 
 %$b(G/H_G)$ and $\Reg(G/H_G,m)$ by $b_H(G)$ and $\Reg_H(G,m)$ respectively.
$$b_H(G):=b(G/H_G) \text{ and } \Reg_H(G,m):=\Reg(G/H_G,m).$$
\end{Def}

Therefore, for $G$ and $H$ as in Problem \ref{prob}, the existence of five conjugates of $H$ whose intersection is trivial is equivalent to the statement that $b_H(G)\le 5.$ Notice that $5$ is the best possible bound for $b_H(G)$ since $b_H(G)=5$ if $G=\Sym(8)$ and $H=\Sym(4) \wr \Sym(2).$ This can be easily verified. In fact, there are infinitely many examples with $b_H(G)=5$, for example see \cite[Remark 8.3]{burPS}.   



 Let  $G$ act transitively on $\Omega$ and let $H$ be a point stabiliser, so $|\Omega|=|G:H|.$ If $(\beta_1 , \ldots , \beta_n )$ is a base for the natural action of $G/H_G$ on $\Omega$, then $$|(\beta_1 , \ldots , \beta_n )^G| \le |\Omega| \cdot (|\Omega|-1) \ldots (|\Omega|-n+1)<|\Omega|^n=|G:H|^n.$$
Therefore,
$$|G:H_G|<|G:H|^n,$$
and if Problem \ref{prob} has a positive answer, then $c \le 5$ in Theorem \ref{indexBab}.

\medskip

A finite group $G$ is {\bf almost simple} if $$G_0 \le G \le \Aut(G_0)$$
for some non-abelian simple group $G_0.$

Problem \ref{prob} is essentially reduced to the case when $G$ is almost simple  by Vdovin  \cite{vd}. We introduce some notation before stating the reduction theorem.



Let $A$ and $B$ be subgroups of $G$ such that $B \trianglelefteq A.$ Then $N_G(A/B) := N_G(A) \cap N_G(B)$ is the {\bf normaliser} of $A/B$ in $G$. If $x\in N_G(A/B)$, then $x$ induces an
automorphism of $A/B$ by $Ba \mapsto Bx^{-1}ax.$ Thus, there exists a homomorphism $N_G(A/B) \to \Aut(A/B).$ The image of $N_G(A/B)$ under this homomorphism is
denoted by $\Aut_G(A/B)$ and is the {\bf group of $G$-induced automorphisms of $A/B$}. 






\begin{Th}[\cite{vd}]\label{sved}
Let $G$ be a finite group and let
$$\{1\}=G_0<G_1< G_2< \ldots < G_n=G $$
be a composition series of $G$ which is a refinement of a chief series. We identify non-abelian $G_i/G_{i-1}$ with the isomorphic normal subgroup of $\Aut_G(G_i/G_{i-1})$.   Assume that for
some $k$ the following condition holds: If $G_i/G_{i-1}$ is non-abelian, then for
every solvable subgroup $T$ of $\Aut_G(G_i/G_{i-1})$  
$$b_T(T \cdot (G_i/G_{i-1})) \le k \mbox{ and } \Reg_T(T \cdot (G_i/G_{i-1}),k)\ge5.$$
Then $b_H(G)\le k$ for every maximal solvable subgroup $H$ of $G$.  
\end{Th}


\begin{Rem}
The formulation of Theorem \ref{sved} in \cite{vd}  differs from ours. Specifically, the condition there  is the following: 
\medskip
\begin{quoting}[leftmargin=\parindent]
If $G_i/G_{i-1}$ is non-abelian, then for
every solvable subgroup $T$ of $\Aut_G(G_i/G_{i-1})$  
$$b_T(\Aut_G(G_i/G_{i-1})) \le k \mbox{ and } \Reg_T(\Aut_G(G_i/G_{i-1}),k)\ge5.$$
\end{quoting}
But the proof uses our formulation of the condition. An updated version of \cite{vd} is available on the arXiv; see the link in the Bibliography. 
\end{Rem}

In particular, Theorem \ref{sved} implies that, in order to solve Problem \ref{prob}, it is sufficient to prove $$\Reg_H(G,5) \ge 5$$ for every almost simple group $G$ and each of its maximal solvable subgroups $H$. 

Our main goal is to study Problem \ref{prob} for almost simple groups. In particular, we  focus on the almost simple classical groups.
 
%\begin{Lem}\cite[Lemma 3]{bay}
% Let $H < G$ and $b_H(G) \le 4$. Then $\Reg_H(G, 5) \ge 5$.
%\end{Lem}

\section{Review of existing literature}

 %Although, in this thesis, we study intersections of solvable subgroups,  we make use of results on intersections of abelian subgroups in the proof of our main result. So, let us start this section with  a  review of important results on intersection of various subgroups in finite groups.
The intersection of various subgroups in finite groups has been studied since the middle of the 20th century,  and associated results have proved useful in the study of group structure. For example, intersections of Sylow subgroups of a finite group are closely connected to representations of the group \cite{michler, strunSP}. Let us mention some important results on intersections of Sylow, nilpotent and abelian subgroups of finite groups.  While not directly applicable to  Problem \ref{prob}, they help to establish background and context. 
 
Let $\pi$ be a set of primes and let $p$ be a prime. A finite group $G$ is $\pi$-solvable if 
none of its non-abelian composition factors has order divisible by a prime from  $\pi$. If $\pi=\{p\}$ and $G$ satisfies this property, then $G$ is $p$-solvable.  Passman \cite{passman} proved that if  a finite group $G$ is $p$-solvable and $P$ is a Sylow $p$-subgroup of $G$, then there exist $x,y \in G$ such that $P \cap P^x \cap P^y$ is  the unique largest normal $p$-subgroup of $G$. Zenkov \cite{ZenP} generalised this statement to an arbitrary finite group.  Vdovin \cite{vdovinReg} and Dolfi \cite{dolfi} independently proved that if $G$ is $\pi$-solvable and $H$ is a solvable Hall $\pi$-subgroup (a $\pi$-subgroup of index  coprime to all primes in $\pi$), then there exist $x,y \in G$ such that $H \cap H^x \cap H^y \le {\bf F}(G)$. Recently, Zenkov \cite{zen21}  proved that if $N$ is a nilpotent subgroup of a finite group $G$, then there exists $x,y \in G$ such that $N \cap N^x \cap N^y \le {\bf F}(G)$.  We use the following related result of Zenkov  \cite{zen}.

\begin{Th}
\label{zenab}
If $A$ and $B$ are abelian subgroups of a finite group $G$, then there exists $x \in G$ such that $A \cap B^x \le {\bf F}(G).$
\end{Th}



\bigskip

Let us now discuss the  progress on Problem \ref{prob} for almost simple groups. In particular, we are interested in bounds for $b_S(G)$ and $\Reg_S(G,5)$ for an almost simple $G$ and its maximal solvable subgroup $S$. The following lemma is useful here.

\begin{Lem}[{\cite[Lemma 3]{bay}}] \label{base4}
Let $G$ be a finite group. If $H \le G$ and $b_H(G)\le 4$, then $\Reg_H(G,5)\ge 5.$
\end{Lem}




  If $G$ is almost simple, $S$ is a maximal solvable subgroup of $G$, and  $S \le H \le G$, then $b_S(G) \le b_H(G).$ Indeed, if $H^{a_1} \cap \ldots \cap H^{a_c}=1$ for $a_i \in G,$ then $$S^{a_1} \cap \ldots \cap S^{a_c}=1.$$ 

%We now present some definitions related to maximal subgroups of almost simple groups.

\begin{Def}
\label{nonstdef}
Let $G$ be a finite almost simple classical group over $\mathbb{F}_q,$ where $q=p^f$ and  $p$ is prime, with socle $G_0$ and natural module $V$. A maximal subgroup $H$ of $G$ not containing $G_0$ is a {\bf subspace subgroup} \index{subspace subgroup} if every
maximal subgroup $M$ of $G_0$ containing $H \cap G_0$ either acts reducibly on $V$ or $(G_0,M, p) = (Sp_{2m}(q)',O^{\pm}_{2m}(q), 2)$.   
%\begin{enumerate}
%\item $M$ is the stabiliser in $G_0$ of a proper non-zero subspace $U$ of $V$, where $U$ is totally
%singular, non-degenerate or, if $G_0$ is orthogonal and $p = 2$, a non-singular $1$-space ($U$ can be
%any subspace if $G_0 = PSL(V )$).
%\item $M = O^{\pm}_{2m}(q)$ if $(G_0, p) = (Sp_{2m}(q), 2)$.
%\end{enumerate}
A faithful transitive action of $G$ on a set $\Omega$ is a {\bf subspace action} \index{action!subspace} if the $G$-stabiliser of a point in
$\Omega$ is a subspace subgroup of $G$.  Non-subspace subgroups and actions are defined accordingly.
\end{Def}


\begin{Def}
Let $G$ be a finite almost simple group with socle $G_0$. A primitive  
action of $G$ on a set $\Omega$ (so the $G$-stabiliser of a point is a maximal subgroup of $G$) is {\bf standard} \index{action!standard} if one of the following holds:
\begin{enumerate}
\item $G_0 = A_n$ and $\Omega$ is an orbit of subsets or partitions of $\{1, \ldots,n\}$.
\item $G$ is a classical group in a subspace action.
\end{enumerate}
\end{Def}

  Liebeck and Shalev \cite{lieb} proved the following conjecture of Cameron and Kantor \cite{Camer}:  if $G$ is an almost simple finite group and $H\le G$ is maximal, then there exists an absolute constant $c$ such that $b_H(G)\le  c$ unless $(G,H)$ lies in a prescribed list of exceptions. The exceptions arise when the action of $G$ on the set of right cosets of $H$  is {\bf standard}. Below we discuss  results specifying bounds for $b_H(G)$ relevant to our study.  



\bigskip

\subsection*{Symmetric groups}


\begin{Th}[{\cite{bay}}]
Let $G$ be a finite almost simple group with socle isomorphic to an alternating group $\mathrm{Alt}(n)$ for $n\ge 5.$ If $H$ is a maximal solvable subgroup of $G$, then $\Reg_H(G,5)\ge 5$.
\end{Th}

 The proof uses a constructive and inductive approach and exploits  the following result of Burness, Guralnick and Saxl \cite{prim11}.

\begin{Th}
Let $G$ be  $\Sym(n)$ or $\mathrm{Alt}(n)$ and let $H<G$ be maximal.  Assume that $H$ acts primitively on $\{1, \ldots ,n\}$ and does not contain $\mathrm{Alt}(n)$. Then $b_H(G) \le 3$ for all $n \ge 11.$
\end{Th} 




\medskip

\subsection*{Classical groups}

  Burness \cite{fpr} obtains information on fixed point ratios of elements of prime order in classical groups in a non-standard action.
The fixed point ratio data underpins the {\bf probabilistic method} used in \cite{burness} to obtain the following result. We describe the probabilistic method in Chapter 2 since we use it in our proofs. 

\begin{Th}
\label{bernclass}
If $G$ is a finite almost simple classical group in a faithful primitive non-standard action with  point stabiliser $H$, then either $b_H(G) \le 4$, or $G = U_6 (2) \cdot 2$, $H = U_4 (3) \cdot 2^2$ and $b(G) = 5$.
\end{Th}

Roughly speaking, Theorem \ref{bernclass} is true for maximal subgroups $H \notin \mathcal{C}_1$ (with some exceptions). Here $\mathcal{C}_i$ for $i=1, \ldots, 8$ are Aschbacher's classes introduced in \cite{asch}   and described in \cite[\S 2.1]{maxlow}  and \cite[Chapter 4]{kleidlieb}.  If $H \in \mathcal{C}_1,$ then it stabilises a subspace (or a pair of subspaces) of the natural module of $G$. Tables 2 and 3 in \cite{burness} contain detailed information on $b_H(G)$ for $n\le 5$ and $H$ from distinct Aschbacher's classes.

\medskip

\subsection*{Exceptional groups of Lie type}

\begin{Th}[{\cite[Theorem 1]{bls}}]
\label{intexc}
Let $G$ be a finite almost simple group of exceptional Lie type, and let $\Omega$ be
a primitive faithful $G$-set. Then $b(G) \le 6$.
\end{Th}

 The proof is based on the probabilistic method.

\medskip



\subsection*{Sporadic groups}


We summarise the results of \cite{spor} and \cite{mon}.

\begin{Th}
\label{intspor}
 Let $G$ be a finite almost simple sporadic group and let $\Omega$ be a
faithful primitive $G$-set with  point stabiliser $H$. One of the following
holds:
\begin{enumerate}[font=\normalfont]
\item $b(G)=2$;
\item $(G, H, b(G))$ is listed in  {\rm \cite[Table 1 and 2]{spor}}; in most cases $b(G)\le 4$, $b(G)=5$ in $12$ cases, $b(G)=6$ in four cases, $b(G)=7$ in one case;
\item $G$ is the Baby Monster, $H = 2^{2+10+20}.(M_{22} : 2 \times S_3)$, $b(G)=3$. 
\end{enumerate}
\end{Th} 

 The proof uses  probabilistic, character-theoretic and computational methods. 
 
 Recently Burness \cite{burspor} proved the following.

\begin{Th}
 If $G$ is a finite almost simple group with sporadic socle and $H$ is a solvable subgroup, then $b_H(G) \le 3.$ 
\end{Th} 

The proof uses computational methods, unless the socle is isomorphic to the Monster or Baby Monster groups where the probabilistic method is used. %Additionally, unless the socle is Baby Monster, Burness establishes if $b_H(G) =2$ for all solvable subgroups of $G.$ 

\medskip

\subsection*{Primitive non-standard actions of $G$ with $b_H(G)>5$}

By Theorems \ref{intexc} and \ref{intspor}, if $G_0$ is exceptional or sporadic, then $b_H(G)\le 7$ for all maximal subgroups $H <G$ with equality only in one case. The following theorem lists all cases with $b_H(G)=6.$  

\begin{Th}[{\cite[Theorem 5.15]{sfa}}]
\label{b6except}
If $G$ is a finite almost simple  group in a faithful primitive non-standard action with point stabiliser $H$, then $b(G) = 6$ if and only if one of
the following holds:
\begin{enumerate}[font=\normalfont]
\item $(G, H) = (M_{23}, M_{22}), (Co_3, McL.2), (Co_2,PSU_6(2).2)$,\\
or $(Fi_{22}.2, 2.PSU_6(2).2);$
\item $G_0 = E_7(q)$ and $H = P_7$;
\item $G_0 = E_6(q)$ and $H = P_1$ or $P_6$.
\end{enumerate}
\end{Th} 
  Each $P_i$ is a maximal {\bf parabolic} subgroup; for details see the discussion before \cite[Theorem 3]{bls}. Therefore,  if $G_0$ is exceptional or sporadic, then either $b_H(G) \le 5$ or $(G,H)$ is listed in Theorems \ref{intspor}  and \ref{b6except}.


\subsection*{Maximal subgroups that are solvable}

Sometimes a maximal subgroup of an almost simple group is solvable.  An explicit list is given by Li and Zhang \cite{maxsolprim}. Recently, Burness \cite{burPS} proved the following.
\begin{Th}
\label{bur2020}
Let $G$ be a finite almost simple  group with socle $G_0$. If a maximal subgroup $H <G $ is solvable, then
  $b_H(G) \le 5$, with equality if and only if one of the following holds:
\begin{itemize}
\item[(a)] $G = \Sym (8)$ and $H = \Sym(4) \wr \Sym (2)$;
\item[(b)] $G_0 = PSL_4(3)$ and $H=P_2$;
\item[(c)] $G_0 = PSU_5(2)$ and $H = P_1$.
\end{itemize}
\end{Th}
The proof exploits both the probabilistic method and computation. Although Theorem \ref{bur2020} does not establish $\Reg_H(G,5)\ge 5$  when $b_H(G)=5$, it can be done routinely by computation.


\section{Main results}

As is clear from the above results, if $G$ is sporadic or exceptional of Lie type, then $b_H(G) \le 5$ for every maximal subgroup $H$ of $G$ apart from a short list of exceptions where  $b_H(G)$ is 6 or 7. If $G$ is classical of Lie type and $H \in \mathcal{C}_1$, then $b_H(G)$ can be arbitrarily large since the order
of $G$ is not always bounded by a fixed polynomial function of the degree of the action.  In particular, as the following lemma shows, $b_H(G)$ is not bounded by a constant.
\begin{Lem}
If $G$ acts faithfully on $\Omega$ and $d=|\Omega|,$ then $b(G) \ge \log_d|G|$.
\end{Lem}
\begin{proof}
Let $B \in \Omega^{b(G)}$ be a base. Every element of $G$ is uniquely determined by its action on $B$. Indeed, if $Bx=By$ for $x,y \in G$, then $Bxy^{-1}=B$ and $x=y$ since $B$ is a regular point. Hence $|G|\le d^{b(G)}.$ 
\end{proof}
    Therefore,  if a maximal solvable subgroup lies only in a $\mathcal{C}_1$-subgroup of $G,$ then one cannot solve Problem \ref{prob} simply by studying the corresponding problem for maximal subgroups.  


We study the situation when $G_0$ is a simple classical group of Lie type isomorphic to $PSL_n(q),$ $PSU_n(q)$ or $PSp_n(q)'$ for some $(n,q)$ and $G$ is an almost simple classical group with  socle isomorphic to $G_0.$ In particular, we identify $G_0$ with its group of  inner automorphisms, so 
$$G_0 \le G \le \Aut (G_0).$$ 
Here $PSp_n(q)'$ is the commutator subgroup of $PSp_n(q)'.$ If $n \ge 4$ and $q \ge 3$, then $PSp_n(q)$ is simple, but $PSp_4(2)=Sp_4(2) \cong \Sym(6)$, so  $PSp_4(2)' \cong \mathrm{Alt}(6)$ is simple. We write $PSp_n(q)'$ to include this group.

Our main result is the following.
\begin{mmainth}
\label{fulltheorem}
Let $G$ be a finite almost simple group with socle isomorphic to $PSL_n(q),$ $PSU_n(q)$ or $PSp_n(q)'.$ If $S \le G$ is solvable, then $\Reg_S(G,5) \ge 5$. In particular, $b_S(G) \le 5.$   
\end{mmainth}
%If $q$ is even, then $PSp_4(q)$ has a graph automorphism \cite[Proposition 12.3.3]{carter}. %Further,  we assume that $G$ has the restrictions above.


%\begin{Rem}
%Almost simple groups with classical socle containing neither graph nor graph-field automorphisms can be considered as groups of semisimilarities of corresponding geometrical structures (see Section \ref{secnot} for details). We use these geometrical properties in our proof of the main results. Since we faced time constraints in completing this work, we did not address groups containing graph or graph-field automorphisms here, but expect to do so using similar techniques in related publications.         
%\end{Rem}


\begin{Rem}
Classical groups of Lie type are naturally divided into four classes: linear, unitary, symplectic and orthogonal groups. Although we believe that our approach could be successfully applied to orthogonal groups, we expect that their consideration will require much more technical work than needed for the other classes because 
of the greater complexity of their structure.     
\end{Rem}

If $X$ is $\GL_n(q),$ $\GU_n(q)$ or $\GS_n(q)$ (see Section \ref{secnot} for definitions) and $N$ is the subgroup of all scalar matrices in $X$, then
$X/N$ is isomorphic to a subgroup of $\Aut(G_0)$ of index at most 2 where $G_0$ is equal to $PSL_n(q),$ $PSU_n(q)$ and $PSp_n(q)'$ respectively. Precisely, the corresponding index is 2  if $G_0=PSL_n(q)$ with $n \ge 3$ or $G_0=PSp_4(q)'$  with $q$ even.  If $G_0=PSL_n(q)$ and $n\ge 3,$ then $\Aut(G_0)$ is isomorphic to $A(n,q)/N$ where $A(n,q)=\GL_n(q) \rtimes \langle \iota \rangle$ and $\iota$ is the inverse-transpose map on $GL_n(q).$





We obtain the Main Theorem as a corollary of the following   theorems. Each of the theorems provide additional details depending on $G_0$.
\begin{mainthA}
\label{theorem}
Let $X=\GL_n(q)$, $n \ge 2$ and $(n,q)$ is neither $(2,2)$ nor $(2,3).$ If $S$ is a maximal solvable subgroup of $X$, 
 then $\Reg_S(S \cdot SL_n(q),5)\ge 5$, in particular $b_S(S \cdot SL_n(q)) \le 5.$
\end{mainthA}

\begin{mainthA}
\label{theoremGR}
Let $n\ge 3.$ If $S$ is a maximal solvable subgroup of $A(n,q)$ not contained in $\Gamma L_n(q),$ then one of the following holds:
\begin{enumerate}
\item[$(1)$] $b_S(S \cdot SL_n(q))\le 4$;
\item[$(2)$] $(n,q)=(4,3)$, $S$ is the normaliser in $A(n,q)$ of the stabiliser in $\GL_n(q)$ of a $2$-dimensional subspace of $V$, $b_S(S \cdot SL_n(q))=5$ and $\Reg_S(S \cdot SL_n(q),5)\ge 5.$
\end{enumerate}
\end{mainthA}


\begin{mainthB}\label{theoremGU}
Let $X=\GU_n(q)$, $n \ge 3$ and $(n,q)$ is not  $(3,2).$ If $S$ is a maximal solvable subgroup of $X$, 
 then one of the following holds:
\begin{enumerate}
\item[$(1)$]  $b_S(S \cdot SU_n(q)) \le 4,$ so $\Reg_S(S \cdot SU_n(q),5)\ge 5$;
\item[$(2)$]   $(n,q)=(5,2)$ and $S$ is the stabiliser in $X$ of a totally isotropic subspace of dimension $1$, $b_S(S \cdot SU_n(q)) =5$ and $\Reg_S(S \cdot SU_n(q),5)\ge 5$. 
\end{enumerate}
\end{mainthB}

\begin{mainthC}\label{theoremSp}
Let $X=\GS_n(q)$ and  $n \ge 4$. If $S$ is a maximal solvable subgroup of $X$, 
 then  $b_S(S \cdot Sp_n(q)) \le 4,$ so $\Reg_S(S \cdot Sp_n(q),5)\ge 5$. 
\end{mainthC}

\begin{mainthC}\label{theoremSpGR}
Let $q$ be even and let ${A}=\Aut(PSp_4(q)')$. If ${S} \le {A}$ is a maximal solvable subgroup, then $b_{{S}}({S} \cdot Sp_4(q)') \le 4,$ so $\Reg_S(S \cdot Sp_n(q)',5)\ge 5$.
\end{mainthC}



\begin{proof}[Proof of Main Theorem]
Let $G_0 \le G \le \Aut (G_0)$ and let $S\le G$ be solvable. Let $H$ be a maximal solvable subgroup of $  \Aut(G_0)$ containing $S$. By Theorems \ref{theorem}, \ref{theoremGR}, \ref{theoremGU},  \ref{theoremSp} and \ref{theoremSpGR},
$$\Reg_H(H \cdot G_0,5)\ge 5,$$
so there exist $x_{(i,1)},x_{(i,2)},x_{(i,3)},x_{(i,4)},x_{(i,5)} \in G_0$ for $i \in\{1, \ldots, 5\}$ such that
$$\omega_i=(Hx_{(i,1)},Hx_{(i,2)},Hx_{(i,3)},Hx_{(i,4)},Hx_{(i,5)})$$ 
are $H \cdot G_0$-regular points   in $\Omega^5=\{Hx \mid, x \in H \cdot G_0 \}^5$, and the $\omega_i$ lie in distinct orbits. 
We claim that $$w_i'=(Sx_{(i,1)},Sx_{(i,2)},Sx_{(i,3)},Sx_{(i,4)},Sx_{(i,4)}) \text{ for } i \in \{1, \ldots, 5\}$$ lie in distinct $G$-regular orbits in $(\Omega')^5=\{Sx \mid x \in G\}^5.$ Indeed, $$S^{x_{(i,1)}} \cap S^{x_{(i,2)}} \cap S^{x_{(i,3)}} \cap S^{x_{(i,4)}} \cap S^{x_{(i,5)}} \le H^{x_{(i,1)}} \cap H^{x_{(i,2)}} \cap H^{x_{(i,3)}} \cap H^{x_{(i,4)}} \cap H^{x_{(i,5)}}=1,$$ so $\omega_i'$ are regular. Assume that $\omega_1' g = \omega_2'$ for some $g \in G.$ Therefore,
$$(Sx_{(1,i)})g=Sx_{(2,i)} \text{ for } i \in \{1, \ldots, 5\}$$
and 
$$g \in \cap_{i=1}^5 (x_{(1,i)}^{-1}S x_{(2,i)}) \subseteq \cap_{i=1}^5 (x_{(1,i)}^{-1}H x_{(2,i)})= \emptyset$$
where the last equality holds since $\omega_1$ and $\omega_2$ lie in distinct $H \cdot G_0$-orbits. Hence $\omega_1'$ and $\omega_2'$ lie in distinct $G$-orbits. The same argument shows that all of the $\omega_i'$ lie in distinct $G$-orbits, so $\Reg_S(G,5)\ge 5$.
\end{proof}


\section{Summary of contents }

 Chapter 2 is devoted to notation, definitions and preliminary results.  We present notation and definitions for classical groups and forms in Sections \ref{secnot} and \ref{clsec},  and  briefly introduce algebraic groups in Section \ref{algsec}. In Section \ref{missec} we collect technical results that play significant roles in the proof of Theorems \ref{theorem} -- \ref{theoremSpGR}. These include lemmas on the structure of maximal solvable subgroups of classical groups, and the subgroups stabilising certain structures, such as a subspace of the natural module or a decomposition of the natural module into direct sum of subspaces. Section \ref{sinsec} is devoted to  Singer cycles -- cyclic subgroups of $GL_n(q)$ of order $q^n-1$ -- and their normalisers. Such subgroups play an important role in the structure of irreducible solvable linear groups. In Section \ref{fprsec} we describe the probabilistic method we mentioned earlier and give the necessary information on fixed point ratios of elements of prime order (modulo scalars) of classical groups. Finally, in Section \ref{gapsec} we describe the computational methods and software we used.   




 In Chapter \ref{ch2} we obtain upper bounds for $b_S(L)$ where $L$ is $GL_n(q)$, $GU_n(q)$ or $GSp_n(q)$ and $S$ is a solvable irreducible subgroup of $L.$ Our results  are refinements of Theorem \ref{bernclass} in the sense that they provide better estimates for $b_H(G)$  for solvable $H$ not lying in a $\mathcal{C}_1$-subgroup of $G \le L/Z(L)$ in the cases described above. 
 In particular, with an explicit list of exceptions, we obtain $b_S(L) =2$ for $L=GL_n(q)$ and $b_S(L) \le 3$ for $L=GU_n(q)$ or $GSp_n(q).$  
 These estimates form an important part of our proof of the main results and are necessary since the bound $b_S(L) \le 4$  from Theorem \ref{bernclass} is not sufficient for the proof. As a ``basic'' case we take the situation when $S$ is a primitive (for $L=GL_n(q)$) or quasi-primitive (for $L=GU_n(q)$ or $GSp_n(q)$) maximal solvable subgroup, so we first study such subgroups. We use the probabilistic method based on fixed point ratios for elements of prime orders to obtain the bounds for $b_S(L)$  for primitive and quasi-primitive $S.$ We do not explicitly  construct  $x,y \in L$ such that $S \cap S^x \cap S^y \le Z(L).$ Nevertheless, the reduction of the remaining cases to this case is constructive in most situations. We illustrate this point for linear groups, so $L=GL_n(q).$ If an irreducible subgroup of $L$ is not (quasi-)primitive, then it must stabilise a nontrivial decomposition of the natural module into a direct sum of subspaces having specified shapes.   In particular, if $S$ is an imprimitive maximal solvable group of $GL_n(q)$, then it is a wreath product of a linear primitive maximal solvable group of smaller degree $S_1 \le GL_m(q)$ and a group of permutations $\Gamma \le \Sym(k)$ where $n=mk$ (see Lemma \ref{supirr}).  If we know $x_1 \in SL_m(q)$ such that $S_1 \cap S_1^{x_1} \le Z(GL_m(q))$, then the proof of Theorem   \ref{irred} can be used to construct explicitly   $x \in SL_n(q)$ such that $S \cap S^x \le Z(GL_n(q)).$


  In Chapter \ref{ch3} we consider the general case  where $S$ is a maximal solvable subgroup of $X$ or $A$. Since for subgroups $S$ stabilising no subspace of the natural module Theorems \ref{theorem} -- \ref{theoremSpGR} follow (with some exceptions) by Theorem \ref{bernclass}, the main obstacle is the situation when $S$ lies in a maximal $\mathcal{C}_1$.   Our strategy is to combine effectively the results of Chapter \ref{ch2} and the structure of $S$. In particular, we use the fact that $S$ stabilises a non-zero proper subspace $U$ of the natural module, so $S^x$ must stabilise $(U)x.$ %See Lemma \ref{diag} for an example of usage of this fact.
  Our proof  is mostly constructive, we again illustrate it in the case of Theorem \ref{theorem} for simplicity.  If $S\le \GL_n(q)$ is reducible, then, in some basis, matrices of $S\cap GL_n(q)$ are upper-block-diagonal (see Lemma \ref{supreduce}) with blocks forming irreducible solvable subgroups $S_i$ of smaller degree $n_i$ where $i=1, \ldots, k$ for some $k$ and $n=\sum_{i=1}^k n_i.$ If we know  $x_i \in SL_{n_i}(q)$ such that $S_i \cap S_i^{x_i} \le Z(GL_{n_i}(q))$, then the proof of Theorem \ref{theorem} can be used to construct $5$ distinct regular orbits of the action of $G/S_G$ on $\Omega^5.$ 

