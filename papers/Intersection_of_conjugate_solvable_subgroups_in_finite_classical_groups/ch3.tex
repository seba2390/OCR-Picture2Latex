\chapter{Intersection of conjugate irreducible solvable subgroups}
\label{ch2}

 Recall from the introduction that $b_S(G)$ is the minimal number such that there exist $$x_1, \ldots, x_{b_S(G)} \in G  \text{ with } S^{x_1} \cap \ldots \cap S^{x_{b_S(G)}}=S_G$$ where $S_G= \cap _{g \in G}S^g$.  Let ${G}$ be  $GL_n(q),$ $GU_n(q)$ or $GSp_n(q)$ in cases {\bf L}, {\bf U} and {\bf S} respectively and let $S$ be an irreducible maximal solvable subgroup of $G$.  The goal of this chapter is to obtain upper bounds for $b_S(S \cdot (SL_n(q^{\bf u}) \cap {G}))$. These bounds play an important role in the proof of Theorems \ref{theorem}, \ref{theoremGU} and \ref{theoremSp} in Chapter \ref{ch3}.  While, with some exceptions,  $b_S(S \cdot (SL_n(q^{\bf u}) \cap {G})) \le 4$ follows by Theorem \ref{bernclass}, it is not sufficient for our purposes. In this chapter we prove that  $b_S(S \cdot (SL_n(q^{\bf u}) \cap {G})) \le 2$ in case {\bf L} and $b_S(S \cdot (SL_n(q^{\bf u}) \cap {G}))\le 3$ in cases {\bf U} and {\bf S} with a detailed list of exceptions. 


\section{Primitive and quasi-primitive subgroups} 


    We start our study with a special case:  $S$ is a primitive maximal solvable  subgroup for the case {\bf L} and $S$ is quasi-primitive solvable for cases {\bf U} and {\bf S}. In the next section  we use these results to obtain bounds for $b_S(S \cdot (SL_n(q^{\bf u}) \cap {G}))$ where $S$ is irreducible.  

%Recall from the introduction that we are interested in $b_S(S \cdot (SL_n(q^{\bf u}) \cap {G}))$ for a maximal solvable subgroup $S$ of ${G}$ where ${G}$ is $GL_n(q),$ $GU_n(q)$ or $GSp_n(q)$ in cases {\bf L}, {\bf U} and {\bf S} respectively.  We start the study with the case when $S$ is a maximal solvable primitive subgroup for the case {\bf L} and $S$ is quasi-primitive solvable for cases {\bf U} and {\bf S} in this section and generalise our results to $S$ irreducible in the following section. 

\begin{Def}
Let $H \le GL(V).$ An irreducible $\mathbb{F}_q[H]$-module $V$ is {\bf quasi-primitive} \index{quasi-primitive} if it is a homogeneous $\mathbb{F}_q[N]$-module for all $N \trianglelefteq H.$ A subgroup $H$ of $GL(V)$ is {\bf quasi-primitive} if $V$ is a {quasi-primitive}
$\mathbb{F}_q[H]$-module.
\end{Def}

We use Lemma \ref{supirr} to extend our results from a primitive subgroup  to an irreducible subgroup in the case {\bf L}. In cases {\bf U} and {\bf S}, Lemma \ref{ashb} does not guarantee that an irreducible subgroup of ${G}$ lies in the wreath product of a primitive subgroup of a general unitary  or symplectic group of smaller degree with a subgroup of a symmetric group. However, it gives us a decomposition of $V$ which allows us to use induction if $S$ is not quasi-primitive.

To prove results about $b_S(SL_n(q^{\bf u}) \cap {G})$, we need  information about  primitive and quasi-primitive solvable groups, upper bounds for $|S|$ and lower bound for $\nu(x)$ (the codimension of a largest eigenspace of $x$, see Definition \ref{nudef}), where $x$ is a prime order element of the image of $S$ in $PGL_n(q^{\bf u}).$  This information is needed to apply the probabilistic method described in Section \ref{fprsec}.
A primitive subgroup of $GL_n(q)$ is quasi-primitive by Clifford's Theorem.  If $S \le GL_n(q)$ is solvable and { quasi-primitive}, then every  normal abelian subgroup of $S$ is cyclic by 
\cite[Lemma 0.5]{manz}. Such groups are  studied in \cite{manz}; we collect the main results in the following lemma.

\begin{Lem}[{\cite[Corollary 1.10]{manz}}] \label{olaf}
Suppose $S \le GL_n(q)$ is nontrivial solvable and every normal abelian subgroup of $S$ is cyclic. Let $F = {\bf F}(S)$ be the Fitting subgroup of $S$ and let $Z$ be the socle of the cyclic
group $Z(F)$. Set $C=C_S(Z)$. Then there exist normal subgroups $E$ and $T$ of $S$ satisfying the following: 
\begin{enumerate}[font=\normalfont]
\item $F=ET$, $Z=E \cap T$ and $T=C_F(E);$
%(2) A Sylow $q$-subgroup of $E$ is cyclic of order $q$ or extra-special of
%exponent $q$ or 4;\\
\item $E/Z = E_1/Z \times \ldots \times E_k/Z$ for chief factors $E_i/Z$ of $G$ with 
$E_i \le C_S(E_j)$ for  $i \ne j$;
\item For each $i$, $Z(E_i) = Z$, $|E_i/Z| = p_i^{2k_i}$ for a prime $p_i$ and an integer
$k_i$, and $E_i= O_{p_i'} (Z) \cdot F_i$ for an extra-special group $F_i = O_{p_i} (E_i) \trianglelefteq S$ of order $p_i^{2k_i+1};$
\item There exists $U \le  T$ of index at most $2$ with $U$ cyclic, $U \trianglelefteq S$ and
$C_T(U)=U$;
%(6) $S$ is nilpotent if and only if $S = T$;\\
\item $T = C_S(E)$ and $F = C_C(E/Z)$;
%(8) $E/Z \cong F/T$ is a completely reducible $S/F$-module and faithful
% $A/ F$ -module (possibly of mixed characteristic);\\
\item If $C_i$ is the centraliser of $E_i/Z$ in C, then $C/C_i$ is isomorphic to a subgroup of $ Sp_{2k_i}(p_i)$.
%(10) If every normal abelian subgroup of $S$ is central in $F$, then $T= Z(F)$
%is cyclic.
\end{enumerate}
\end{Lem}

\begin{Rem}\label{eqrem}
In the notation of Lemma \ref{olaf}, let $e$ be a positive integer such that $|E/Z|=e^2,$ so $e= \prod_{i=1}^k p_i^{k_i}.$ Since $E_i$ has the subgroup $F_i$ of order $p_i^{2k_i+1}$ and $|E_i/Z|=p_i^{2k_i},$ for each $p_i$ there must exist an element of order $p_i$ in $Z$ (and in $U$, since $Z \le U$). In other words, each $p_i$ divides $|U|.$ 
\end{Rem}

The following lemma collects properties of primitive maximal solvable subgroups from \cite[\S \S 19 -- 20]{sup} and \cite[\S 2.5]{short}.  For $A \le GL_m(q)$ and $B \le GL_n(q)$, $A \otimes B = \{a \otimes b \mid a \in A, b \in B\}$ where $a \otimes b$ is the Kronecker product defined in Section \ref{secnot}.  

\begin{Lem}\label{suplem}
Let $S \le GL_n(q)$ be a primitive maximal solvable subgroup. Then $S$
admits the unique chain of subgroups 
\begin{equation}\label{ryad}
S \trianglerighteq C \trianglerighteq F \trianglerighteq A
\end{equation}
 where $A$ is the unique maximal abelian normal subgroup,  $C =C_S(A)$, and  $F$ is the full preimage in $C$ of the maximal  abelian normal subgroup of $S/A$ contained in $C/A.$ The following hold: 
\begin{enumerate}[font=\normalfont]
\item[$a)$] $A$ is the multiplicative group of a field extension $K$ of the field of scalar matrices $ \Delta = \{\alpha I_n: \alpha \in \mathbb{F}_q\}$ and $m:=|K : \Delta|$ divides  $n$.
\item[$b)$]  $S/C$ is isomorphic to a subgroup of the Galois group of the extension $K:\Delta,$ so 
$$S=C.\sigma,$$
where $\sigma$ is cyclic of order dividing $m$. 
\item[$c)$] $C \le GL_e(K)$ where $e=n/m$.
\item[$d)$] If
\begin{equation}\label{r}
 e=p_1^{l_1} \cdot \ldots \cdot p_t^{l_t},
\end{equation}
where $t \in \mathbb{N},$ and the $p_i$ are distinct primes, then each $p_i$ divides $|A|=q^m-1.$
\item[$e)$] $|F/A|=e^2$ and $F/A$ is the direct product of elementary abelian groups. 
\item[$f)$] If $Q_i$ is the full preimage in $F$ of the Sylow $p_i$-subgroup of $F/A,$ then
$$Q_i= \langle u_1 \rangle \langle v_1 \rangle \ldots \langle u_{l_i} \rangle \langle v_{l_i} \rangle A,$$
where 
\begin{equation}\label{quv}
[u_j,v_j]=\eta_j, \text{ } \eta_j^{p_i}=1, \text{ } \eta_j \ne 1, \text{ } \eta_j \in A; \text{ } u_j^{p_i},v_j^{p_i} \in A
\end{equation}
and elements from distinct pairs $(u_i,v_i)$ commute.
\item[$g)$] In a suitable $K$-basis of $K^e$, 
$$F=\tilde{Q_1} \otimes \ldots \otimes \tilde{Q_t},$$
where $\tilde{Q_i}$ is an absolutely irreducible subgroup of $GL_{p_i^{l_i}}(K)$ isomorphic to $Q_i.$
\item[$h)$] If $N=N_{GL_e(K)}(F)$ and $N_i=N_{G_i}(\tilde{Q_i}),$ then
 $$N=N_1 \otimes \ldots \otimes N_t.$$
\item[$i)$] $N_i/\tilde{Q_i}$ is isomorphic to a completely reducible subgroup of $Sp_{2l_i}(p_i).$
\end{enumerate}
\end{Lem}

\begin{Rem}\label{pderem}
The structure of a quasi-primitive solvable group is similar to that of a primitive maximal solvable one.  Nevertheless, we need Lemma \ref{suplem} to obtain better estimates for $|S|$ in the case {\bf L} and to deal with  particular cases when $n$ is small. Our notation in Lemma \ref{suplem} is not consistent with that of \cite{sup}, but similar to that of \cite{short}.
\end{Rem}

\begin{Th}[{\cite[Theorem 3.5]{manz}}] \label{lowboundsol}
If $S$ is a completely reducible solvable  subgroup of $GL_n(q)$, then $|S| < q^{9n/4}/2.8.$
\end{Th}


Let $S$ be a primitive maximal solvable  subgroup of $GL_n(q)$. If $A,$ $F$ and $C$ are as in  \eqref{ryad}, then by Lemma \ref{suplem}
\begin{equation*}
\begin{split}
|A|&=q^m-1;\\
|F:A|&=(n/m)^2=e^2;\\
|S:C|&\le m.\\
\end{split}
\end{equation*}
By  Lemma \ref{suplem} i) and Theorem \ref{lowboundsol}
\begin{equation}
\label{CFbound}
|C:F| \le 
\begin{cases}
\prod_{i=1}^t |Sp_{2l_i}(p_i)|;\\
\prod_{i=1}^t ((p_i^{2l_i})^{9/4}/2.8)<e^{9/2},
\end{cases}
\end{equation}
where $e=n/m$ is as in \eqref{r}. Notice that 
$$|Sp_{2l_i}(p_i)|=p_i^{l_i^2} \prod_{j=1}^{l_i}(p_i^{2j}-1)\le p_i^{l_i^2} \prod_{j=1}^{l_i}p_i^{2j}=(p_i^{l_i})^{(2l_i+1)}. $$
Denote $\log_2(e)$ by $l$, so $l_i \le l$ for $i=1, \ldots, t.$ Therefore,
$$|C:F| \le \prod_{i=1}^t (p_i^{l_i})^{(2l+1)}=e^{(2l+1)} $$ 
and
\begin{equation*}
|S|\le (q^m-1)\frac{n^2}{m^2}m \cdot \min \{ e^{(2l+1)}, e^{9/2}\}=(q^m-1)m \cdot \min \{e^{2(l+1)}, e^{13/2}\},
\end{equation*}
so
\begin{equation}\label{H}
|S/Z(GL_n(q))|\le \left( \frac{q^m-1}{q-1} \right) m\cdot \min \{e^{2(l+1)}, e^{13/2}\}.
\end{equation}





\begin{Lem}\label{simpcycl}
Let $C$ be a non-scalar cyclic subgroup of $G \in \{Sp_{n}(q), GU_n(q)\}$ such that $V$ is $\mathbb{F}_{q^{\bf u}}[C]$-homogeneous. Recall that ${\bf u}=2$ if $G=GU_n(q)$ and ${\bf u}=1$ otherwise.  If $W \subseteq V$ is a $\mathbb{F}_{q^{\bf u}}[C]$-irreducible $C$-invariant subspace with $\dim W = m,$ then  $|C|$ divides $(q^{\bf u})^{m/2}+1.$ Moreover, $m$ is even if $G=Sp_{n}(q)$ and $m$ is odd if  $G=GU_n(q)$.
\end{Lem}
\begin{proof}
Since $W$ is $\mathbb{F}_{q^{\bf u}}[C]$-irreducible, it is either non-degenerate or totally isotropic. If $W$ is non-degenerate, then the lemma follows by \cite[Satz 4 and 5]{zing}. 

Let $W$ be totally isotropic. We consider here the proof for $G=Sp_{n}(q),$ the proof for $GU_n(q)$ is  analogous. By \cite[(5.2)]{asch}, we can assume that there exist $W_1, W_2 \subseteq V$ such that $W_1=W$ and $W_2$ is totally isotropic and $\mathbb{F}_{q^{\bf u}}[C]$-irreducible, and $W_1 \oplus W_2$ is non-degenerate. 

Let $g \in Sp(W_1 \oplus W_2)$ be the restriction of a generator of $C$ to $W_1 \oplus W_2.$ Since $V$ is $\mathbb{F}_{q^{\bf u}}[C]$-homogeneous, there exist bases $\beta_i$ of $W_i$ for $i=1,2$ (let $\beta=\beta_1 \cup \beta_2$) such that 
$$g_{\beta}=
\begin{pmatrix}
g_1 & 0\\
0 & g_1
\end{pmatrix} 
\text{ and } 
{({\bf f}|_{_{W_1 \oplus W_2}})}_{\beta}=
\begin{pmatrix}
0 & A \\
-A^{\top} & 0
\end{pmatrix}
$$   for some $g_1, A \in GL_m(q).$ Here $\langle g_1 \rangle \le GL_m(q)$ is irreducible. Since $g$ is an isometry of $W_1 \oplus W_2,$ $$g_{\beta}{({\bf f}|_{_{W_1 \oplus W_2}})}_{\beta}(g_{\beta})^{\top}={({\bf f}|_{_{W_1 \oplus W_2}})}_{\beta},$$ so $g_1 A g_1^{\top}=A$ and $g_1^A=(g_1^{-1})^{\top}.$ Therefore, the set of eigenvalues of $g_1$ is closed under taking inverses. Moreover, the multiplicity of the eigenvalue $\mu$ is equal to the multiplicity of the eigenvalue $\mu^{-1}$ for every $\mu \in \overline{\mathbb{F}_q}^*.$

By \cite[Lemma 1.3]{buturl}, $g_1$ is conjugate in $GL_m(\overline{\mathbb{F}_q})$ to 
$$\diag(\lambda,\lambda^q, \ldots, \lambda^{q^{m-1}}), \text{ where } \lambda^{q^m-1}=1.$$
Clearly, $|\lambda| = |g_1|=|C|.$ Let $r$ be the minimal natural number such that $\lambda^{q^r}=\lambda^{-1}.$ If $r=0$, so $\lambda=\lambda^{-1}=\pm 1,$ then $g_1= \lambda I_m$ and $C$ is a group of scalars, since it is homogeneous. Assume $r>0$, so $\lambda^{q^r+1}=1$ and $(q^r+1)$ divides $(q^m-1).$  
Since $\langle g_1 \rangle$ is irreducible, $|g_1|$ does not divide $q^l-1$ for every proper divisor $l$ of $m.$ Notice that $|g_1|$ divides $(q^{2r}-1),$ so $|g_1|$ divides $(q^{2r}-1, q^m-1)=q^{(2r,m)}-1$ which is  divisible by $q^r+1.$ Therefore, $(2r,m)>r$ and $2r=m.$ Hence $m$ is even and $|C|$ divides $q^{m/2}+1.$
\end{proof}

\begin{Cor}\label{simpcyclcor}
Let $C$ be a non-scalar cyclic subgroup of $GSp_{n}(q)$ such that $V$ is $\mathbb{F}_{q}[C]$-homogeneous.   If $W \subseteq V$ is a $\mathbb{F}_{q}[C]$-irreducible  submodule of dimension $m,$ then $m$ is even and $|C|$ divides $(q^{m/2}+1)(q-1).$
\end{Cor} 
\begin{proof}
Let  $C=\langle c \rangle$ and $\tau(c)=\lambda \in \mathbb{F}_q$ where $\tau$ is as in Definition \ref{taudef}. So
$$(uc,vc)=\lambda (u,v) \text{ for all } u,v \in V.$$ Notice that
$$(uc^{|\lambda|},vc^{|\lambda|})=\lambda^{|\lambda|} (u,v) =(u,v) \text{ for all } u,v \in V.$$
Therefore, $c^{|\lambda|} \in Sp_n(q).$ Let $c_1$ be the restriction of $c$ to $W$, so $c=\diag[c_1, \ldots, c_1]$ in some basis of $V$ since $V$ is $\mathbb{F}_q[C]$-homogeneous.  

We claim that $\langle c_1^{|\lambda|} \rangle$ is an irreducible subgroup of $GL(W).$ Assume the  opposite, so there exists a $\langle c_1^{|\lambda|} \rangle$-invariant subspace of $W$ of dimension $r$ dividing $m$. Hence $r$ is even and $|c_1|/{|\lambda|}$ divides $(q^{r/2}+1)$ by Lemma   \ref{simpcycl} since $c^{|\lambda|} \in Sp_n(q).$ Also $|\lambda|$ divides $(q-1)$ and $(q-1)$ divides $(q^{r/2}-1),$ so $|c_1|=|\lambda|\cdot|c^{|\lambda|}|$ divides $q^r-1.$ By Lemma \ref{irrsub}, $\langle c_1 \rangle$ is a reducible subgroup of $GL(W)$ which is a contradiction. 

Thus, $W$ is $\langle c_1^{|\lambda|} \rangle$-irreducible and $|c^{|\lambda|}|$ divides $(q^{m/2}+1)$ by Lemma \ref{simpcycl}, so $|C|$ divides $(q^{m/2}+1)(q-1).$
\end{proof}

We adopt the notation of Lemma \ref{olaf} in the following statement.

\begin{Lem}\label{uniqorder}
Let $G \in \{GSp_{n}(q), GU_n(q)\}$. Let $S$ be a quasi-primitive solvable  subgroup of  $G$. Let $W$ be an $m$-dimensional  irreducible $U$-submodule of $V$ and let $e$ be a positive integer such that $e^2=|E/Z|$.  The following hold:
\begin{enumerate}[font=\normalfont]
\item $em$  divides $n$;
\item $|S| \le \min\{|U|^2e^{13/2}/2, |U|me^{13/2} \};$
%\item $|U|\le q^m+1$ or $n$ is even, $m$ divides $n/2$ and $|U|\le q^{2m}-1;$
\item if $e=1$, then $n=m$ and $S$ is a subgroup of the normaliser of a Singer cycle of $GL_n(q^{\bf u});$
\item if $m=1$ then $|S| \le |Z(G)|e^{13/2}.$
\end{enumerate}
\end{Lem}
\begin{proof}
Since $EU \trianglelefteq S,$ (1) follows by Clifford's Theorem and \cite[Corollary 2.6]{manz}.

It is easy to see that 
$$|S|=|S/C|\cdot |T| \cdot|C/F| \cdot|F/T|.$$
By the proof of \cite[Corollary 3.7]{manz}, $|S/C| \cdot |T| \le |U|^2,$ and $|C/F| \le e^{9/2}/2$ and $|F/T|=e^2,$ which gives us the first bound of (2). To obtain the second bound, we claim that $|S/C|\le m.$ Indeed, the linear span ${\mathbb{F}_{q^{\bf u}}}[Z]$ is the field extension $K$ of the field of scalar matrices $\Delta=\mathbb{F}_{q^{\bf u}}\cdot I_n$ of degree $m_1 = \dim W_1,$ where $W_1 \le V$ is an irreducible $\mathbb{F}_{q^{\bf u}}[Z]$-module, so $m_1$ divides $n$ since $Z$ is homogeneous, and $m_1 \le m$ since $Z \le U$. Consider the map 
$$f:S \to \mathrm{Gal}(K/\Delta), \text{ } g \mapsto \sigma_g,$$
where $\sigma_g : K \to K,$ $x^{\sigma_g}=x^g$ for $x \in K.$ Since $\ker(f)=C,$
$$S/C \cong \mathrm{Im}(f) \le {\mathrm{Gal}}(K/\Delta), $$
so $|S/C|$ divides $m_1$ and the second bound follows.

%Consider an irreducible $U$-submodule $W$ of $V,$ so $W$ is either non-degenerate or totally isotropic.
%If $W$ is non-degenerate then by \cite[Satz 4]{zing} $m$ is odd and $|U|$ divides $q^m+1.$ If $W$ is totally singular  then $m$ divides $n/2$ by \cite[(5.2) and (5.3)]{asch}. Since the restriction $U_1$ of $U$ to $W$ is an irreducible abelian subgroup of $GL(W),$ $U_1$ is a subgroup of a Singer cycle and $|U|=|U_1| \le q^{2m}-1,$ which completes the proof of (3).

If $e=1$, then $F=T$ and $S$ is a subgroup of the normaliser of a Singer cycle of $GL_n(q^{\bf u})$ by \cite[Corollary 2.3]{manz}, so $U$ is self-centralising. By \cite[Lemma 2.2]{manz} $U$ is irreducible, so $m=n$ %, $|U| \le ((q^{\bf u})^{n/2}+1)$ by \cite[Satz 4, Satz 5]{zing}
 and (3) follows.

If $m=1,$ then $U \le Z(G)$ since $U$ is homogeneous, so  $|S/C|=1$ which implies (4). 
\end{proof}



%\begin{Lem}\label{nuuni}
%Let $S$ be a solvable quasi-primitive subgroup of $GU_n(q)$ containing $Z(GU_n(q))$. Let $H$ be  the image of $S$ under the natural homomorphism from $GU_n(q)$ to $$GU_n(q)/Z(GU_n(q))=PGU_n(q).$$
% Let $1 \ne x \in H$ have prime order $k$, so its preimage  $\hat{x}$ lies in $S \backslash Z(GL_n(q))$. Then $\nu(x) \ge n/4$. Furthermore, if $|\hat{x}|$ is prime, then the following is true:
%\begin{enumerate}[font=\normalfont]
%\item if $\hat{x} \in EU$, then $\nu(x) \ge n/2; $ 
%\item if $\hat{x}  \in C \backslash EU$, then $\nu(x) \ge n/4; $
%\item if $\hat{x}  \in S \backslash C$, then $\nu(x)=n-n/k \ge n/2. $ 
%\end{enumerate}
%\end{Lem}
%\begin{proof}
%If  \eqref{prost2} holds in Lemma \ref{prost}, then $\nu(x)=n-n/k\ge n/2.$

%Suppose \eqref{prost1} holds in Lemma \ref{prost}, so $\hat{x} \in S$ has order $k.$ The lemma follows from proofs of \cite[Lemma 2.4]{yang2} and \cite[Lemma 2.4]{yang1}.
%\end{proof}


 
\begin{Lem}\label{nuuni}
Let $S$ be a quasi-primitive solvable subgroup of $GL_n(q)$ and let $H$  be the image of $S$ under the natural homomorphism from $GL_n(q)$ to $PGL_n(q).$ If $x \in H$ has prime order, then $\nu(x) \ge n/4.$
\end{Lem}
\begin{proof}
Let $\hat{x}$ be a preimage of $x$ in $GL_n(q),$ so $\hat{x}=\mu g$, where $\mu \in Z(GL_n(q))$ and $g \in S \backslash \{1\}.$ Observe that $\nu(x)=\nu(\hat{x})=\nu(g),$ so it suffices to prove that $\nu(g)\ge n/4$ for all nontrivial $g \in S.$

If $g \in U$, then, since $U$ is abelian and $V$ is $U$-homogeneous, $g$ is conjugate in $GL_n(\overline{\mathbb{F}_q})$ to 
$$\diag(\lambda, \lambda^q, \ldots, \lambda^{q^{m_1-1}}, \ldots, \lambda, \lambda^q, \ldots, \lambda^{q^{m_1-1}}); \text{ } \lambda \in \overline{\mathbb{F}_q}$$
by \cite[Lemma 1.3]{buturl}. Here $m_1$ is the smallest possible integer such that  $\lambda^{q^{m_1}}=\lambda.$ Therefore, $\nu(g)=n-n/m_1\ge n/2.$ Moreover, if $z \in U$ is nontrivial, then $C_V(z)=\{0\}.$

Let $\lambda \in \overline{\mathbb{F}_q}^*.$ If $g \in S \backslash C,$ then 
$$[\lambda g, z]=[g,z] \in Z \backslash \{1\}$$ for some $z \in Z \le U.$ Notice 
$$C_V((\lambda g)^{-1}) \cap C_V(z^{-1} \lambda g z) \subseteq C_V([\lambda g, z])=\{0\}$$
and $\dim(C_V((\lambda g)^{-1}))= \dim(C_V((\lambda g)^{z})),$ so $ \dim (C_V((\lambda g))) \le n/2$ for every $\lambda \in \overline{\mathbb{F}_q}^*$. Hence $\nu(g)\ge n/2.$

If $g \in F \backslash T,$ then, by $(1)$ and $(5)$ of Lemma \ref{olaf}, there exists $h \in E$ such that $$[ \lambda g, h]= [g,h] \in Z \backslash \{1\}$$ and $\nu(g) \ge n/2$ as above.

If $g \in T \backslash U$, then $[\lambda g, u]= [g,u] \in U \backslash \{1\}$ for some $u \in U$ by $(4)$ of Lemma \ref{olaf}, so $\nu(g) \ge n/2$ as above.

If $g \in C \backslash F$, then $[\lambda g, h] \in E \backslash Z \subseteq F \backslash T$ for some $h \in E$ by $(5)$ of Lemma \ref{olaf}. Therefore, 
$$\dim (C_V([\lambda g, h])) \le n/2 \text{ and } \dim (C_V(\lambda g)) \le 3n/4$$
for every $\lambda \in \overline{\mathbb{F}_q}^*$,
so $\nu(g) \ge n/4.$
\end{proof}

The above result  holds for a primitive solvable  subgroup of $GL_n(q).$ However, we prefer to state it in the notation of Lemma \ref{suplem}.% While the following result can be obtained from \cite[Lemma 2.3]{seress},  we give a direct proof for completeness.


\begin{Lem}\label{6}
 Let $S$  be a primitive maximal  solvable subgroup of $GL_n(q)$ and let $H$  be the image of $S$ under the natural homomorphism from $GL_n(q)$ to $PGL_n(q).$ If $x \in H$ has prime order $k$, so its preimage  $\hat{x}$ lies in $S \backslash Z(GL_n(q))$, then %$\nu(x) \ge n/4$. Furthermore, if $|\hat{x}|$ is prime, then 
the following hold:
\begin{enumerate}[font=\normalfont]
\item if $\hat{x} \in A$, then $\nu(x) \ge n/2; $ \label{6it1}
\item if $\hat{x}  \in F \backslash A$, then $\nu(x) \ge n/2; $ \label{6it2}
\item if $\hat{x}  \in C \backslash F$, then $\nu(x) \ge n/4; $ \label{6it3}
\item if $\hat{x}  \in S \backslash C$, then $\nu(x)=n-n/k \ge n/2. $ \label{6it4}
\end{enumerate}
\end{Lem}
\begin{proof}

 
The proof of \eqref{6it1} -- \eqref{6it3} is  as in Lemma \ref{nuuni}.

 Let us prove \eqref{6it4}.  If  \eqref{prost2} holds in Lemma \ref{prost}, then $\nu(x)=n-n/k\ge n/2.$ 
Suppose \eqref{prost1} holds in Lemma \ref{prost}, so $\hat{x} \in S$ has order $k.$
 If $\hat{x}  \in S \backslash C$ then, by $b)$ of Lemma \ref{suplem} and \cite[7-2]{conjaut}, $\hat{x}$ is conjugate to a field automorphism $\sigma_1 \le \langle \sigma \rangle$ of $GL_e(K)$ of order $k$.
Such an element acts as a permutation with $(n/k)$ $k$-cycles on a suitable basis of $V$; therefore, 
\begin{equation*}
\nu(x)=n-n/k\ge n/2. \qedhere
\end{equation*} 
%
%Now consider the case \eqref{6it1}. If $\hat{x} \in A=K^*,$  then $\hat{x}$ is conjugate in $GL_n( \overline{\mathbb{F}_q})$ to 
%$$\diag(\lambda, \lambda^q, \ldots , \lambda^{q^{m_1-1}}, \lambda, \ldots \lambda^{q^{m_1-1}}); \text{ } \lambda \in \overline{F}.$$
%Here $m_1$ is the minimal divisor of $m$ such $\lambda^{q^{m_1}}=\lambda.$ Since $x$ is nontrivial, $m>1$. Therefore
%\begin{equation}\label{sinnu}
%\nu(x)=n-n/m_1\ge  n/2.
%\end{equation} 
%
%Assume now that $\hat{x} \in C \backslash A,$  
%so $\hat{x} \in N=N_1 \otimes \ldots \otimes N_t$
%and
%  $$\hat{x} = x_1 \otimes \ldots \otimes x_t,$$
%where $x_i \in N_i \le G_i.$ Denote $V_0=K^r,$ $V_i=K^{p_i^{l_i}}$ and 
%$$s_i: = \min\{\dim[\overline{V_i},\lambda x_i]: \lambda \in \overline{K}^*\}.$$
%By \cite[Lemma 3.7]{lieb}
%\begin{equation}\label{lieb1}
%\nu_{V_0, \overline{K}}(\hat{x})\ge \max \left\{\frac{r}{p_i^{l_i}}s_i; i =1, \ldots, t\right\}.
%\end{equation}
%By \cite[Corollary 3.1]{sup} $C_C(F)=A,$ so $C_{G_i}(\tilde{Q_i})=A.$ Since $\tilde{Q_i}$ is nilpotent $$\tilde{Q_i}=R_i \times A_i,$$ 
%where $A_i$ is the $p_i'$-part of $A$ and $R_i$ is the $p_i$-Sylow subgroup of $\tilde{Q_i}.$
%If $x_i \in N_i \backslash A,$ then it induces a nontrivial automorphism of $R_i$ by conjugation, since $C_{G_i}(R_i)=A.$


%Let $\lambda$ be an arbitrary non-zero scalar in the algebraic closure $\overline{K}$ of $K.$ If $x_i \in \tilde{Q_i}$ then there exists $g \in R_i$ such that 
%$[\lambda x_i,g]$ is a non-zero scalar in $\overline{K}$ by \eqref{quv}, so 
%$$C_{V_i}([\lambda x_i,g])=0.$$ As $[\lambda x_i, g]$ is a product of $(\lambda x_i)^{-1}$ and a conjugate of $\lambda x_i$, 
%$$\dim[\overline{V_i},\lambda x_i]\ge \dim V_i/2=p_i^{l_i}/2,$$
%so in this case $s_i \ge p_i^{l_i}/2.$

%If $x_i \in N_i \backslash \tilde{Q_i}$ then there exists $g_1 \in R_i$ such that $[\lambda x_i, g_1] \in R_i \backslash A.$ Hence there exists $g_2 \in R_i$ such that  $[\lambda x_i,g_1,g_2]$ is a non-zero scalar in $\overline{K}$,  so 
%$$C_{V_i}([\lambda x_i,g_1,g_2])=0.$$ As $[\lambda x_i, g_1,g_2]$ is a product of 4 conjugates of $\lambda x_i$ and $(\lambda x_i)^{-1}$,
%$$\dim[\overline{V_i},\lambda x_i]\ge \dim V_i/4=p_i^{l_i}/4,$$
%so in this case $s_i \ge p_i^{l_i}/4$.
%Therefore, by \eqref{lieb1}, if $\hat{x} \in F,$ so all $x_i \in \tilde{Q_i}$, then  
%$$\nu_{V_0, \overline{K}}(\hat{x})\ge r/2.$$ 
%In general for $\hat{x} \in C$ 
%$$\nu_{V_0, \overline{K}}(\hat{x})\ge r/4.$$
%
%By the proof of \cite[Lemma 4.2]{lieb}
%$$\nu_{V,\overline{\mathbb{F}_q}}(x)\ge m \cdot \nu_{V_0, \overline{K}}(\hat{x}),$$
%which concludes the proof of \eqref{6it2} and \eqref{6it3}.
\end{proof}

%\subsection{Maximal primitive solvable subgroups for $n \ge 6$}

%In this section we adopt notation from Lemma \ref{suplem}.

\begin{Th}\label{sch} 
 $\phantom{gg}$
\begin{enumerate}[font=\normalfont]
\item Let $n \ge 6. $ If $S$ is a primitive maximal solvable  subgroup of $GL_n(q)$, then $$b_S(S \cdot SL_n(q)) = 2.$$ 
\item Let $n \ge 2$ and $q$ be such that $(n,q)$ is not any of $(2,2)$, $(2,3)$ or $(3,2)$. If $S$ is a quasi-primitive maximal solvable  subgroup of $GU_n(q)$, then $$b_S(S \cdot SU_n(q)) \le 3.$$ 
\item Let $n \ge 6. $ If $S$ is a quasi-primitive maximal solvable subgroup of $GSp_n(q)$, then $$b_S(S \cdot Sp_n(q)) \le 3.$$ 
\end{enumerate}
\end{Th}
\begin{proof}

Let $\hat{G}$ be $S \cdot SL_n(q),$ $S \cdot SU_n(q)$ and $S \cdot Sp_n(q),$ for  cases $(1),$ $(2)$ and $(3)$ respectively.  Let $G=\hat{G}/Z(\hat{G})\le PGL_n(q^{\bf u})$ and let $H$ be $S/Z(\hat{G})\le G.$ Obviously, 
$$b_S(\hat{G})=b_H(G).$$

If $n \ge 6$ and for all $x \in G$ of prime order   
$$|x^G \cap H| < |x^G|^{(3c-4)/(3c)},$$
then $b_H(G) \le c$ by Lemma \ref{fpr}  and \eqref{0}. Therefore, if $n \ge 6$, then it suffices to show this inequality for $c=2$ in $(1)$ and for $c=3$ in cases $(2)$ and $(3)$. 

%Let $s=\nu(x) \ge n/2$. By \cite[Proposition 3.22, Lemma 3.34, Proposition 3.36, Lemma 3.38]{fpr2},
%\begin{equation}\label{5}
%|x^G| \ge \frac{1}{2t}q^{ns} \ge \frac{1}{2t}q^{n^2/2},
%\end{equation}
%where $t$ is $1$, $2$ or a prime divisor of $n.$

%If $n/2 > s \ge n/4$, then 
%\begin{equation}
%|x^G| \ge \frac{1}{2}q^{2s(n-s)} \ge  
%\frac{1}{2}q^{(3/8)n^2}
%\end{equation}
%by \cite[Lemma 3.38]{fpr2}.

 %By \cite[Propositions 3.22 and 3.36, Lemmas 3.34 and 3.38]{fpr2},
%we have the following bounds for $|x^G|$. In case $\hat{G}=S \cdot SL_n^{\varepsilon}(q)$:
%\begin{equation}\label{5uni}
%|x^G| > 
%\begin{cases}
%  \frac{1}{2t} \left(\frac{q}{q+1} \right)^{as/(n-s)} q^{ns} \ge \frac{1}{2t} \left(\frac{q}{q+1} \right)^{as/(n-s)} q^{n^2/2}  & \text{ for } s \ge n/2;\\
%  \frac{1}{2} \left(\frac{q}{q+1} \right)^{a} q^{2s(n-s)} \ge  
%\frac{1}{2} \left(\frac{q}{q+1} \right)^{a}  q^{(3/8)n^2}  & \text{ for }  n/4 \le s < n/2,
%\end{cases}
%\end{equation}
%where $t$ is $1$, $2$ or a prime divisor of $n$ and $a=(1/2)(1- \varepsilon 1)$. In case $\hat{G}=Sp_n(q)$:
%\begin{equation}\label{5simp}
%|x^G| > 
%  \frac{1}{4} \left(\frac{q}{q+1} \right) \max(q^{s(n-s)}, q^{(ns/2)}).
%\end{equation}


Let $s:=\nu(x).$ We use bounds  \eqref{5uni} and \eqref{5simp} for $|x^G|.$
In most cases the bound $|x^G \cap H| \le |H|$ is sufficient. 

Part $(1)$ of the lemma follows from \eqref{0}, Lemma \ref{6} and bounds  \eqref{5uni}, \eqref{H} for all $q$ for $n \ge 16$. The list of cases when these bounds  are not sufficient for $6 <n \le 15$ is finite. By  Lemma \ref{suplem} $d)$,   $(q^m-1)$ must be divisible by $p_i$ for all $p_i,$ $i=1, \ldots, t.$ Using this statement and the bound for $|C:F|$ obtained in \eqref{CFbound} by using the precise orders of the $Sp_{2l_i}(p_i)$, we reduce this list to  cases 1--6 in Table~\ref{tab}. %For $n=6$ there exists an  infinite sequence of such cases.
  For cases 2--6 the lemma is verified by computation. 
\begin{table}[h]
\centering
\caption{Exceptional cases in proof of Theorem \ref{sch} for $(1)$}
\label{tab}
\begin{tabular}{|l|l|l|l|}
\hline
\textbf{Case} & $n$ & $e$ & $q$   \\ \hline
\textbf{1}  & 6   & 1   & any   \\ \hline
\textbf{2}  & 6   & 2   & 3     \\ \hline
\textbf{3}  & 6   & 3   & 2,4 \\ \hline
\textbf{4}  & 7   & 1   & 2,3,4 \\ \hline
\textbf{5}  & 8   & 1   & 2     \\ \hline
\textbf{6}  & 8   & 8   & 3,5    \\ \hline
\end{tabular}
\end{table} 

 Consider the case $n=6,$ $e=1,$ so $S$ is the normaliser of a Singer cycle and $S \cdot SL_n(q)=GL_n(q).$ First, we find a better 	
estimate for $|x^G \cap {H}|. $ We  represent $\hat{x}$ as $(\lambda, j) \in \mathbb{F}_{q^n}^{*} \rtimes \mathbb{Z}_n,$
where $(\lambda, 0)^{(1, j)}=(\lambda^{q^j},0).$

If $k$ does not divide $n$, then $\hat{x} \in \mathbb{F}_{q^n}^*$ has order $k$ by Lemma \ref{prost}. Let $\hat{x}^g$  be an element of $\hat{x}^{GL_n(q)} \cap S$ for some $g \in GL_n(q).$
Then $\hat{x}^g$ can be written in the form $(\lambda_1, i) \in \mathbb{F}_{q^n}^{*}\rtimes \mathbb{Z}_n$ for suitable $\lambda_1 \in \mathbb{F}_{q^n}^*$ and $i \in \mathbb{Z}_n$, so $(\hat{x}^g)^k=(\lambda_2, ki)$ and $ki \equiv 0 \pmod{n}$. Thus, $i\equiv 0 \pmod{n}$ and $\hat{x}^g \in \mathbb{F}_{q^n}^*$ since $(k,n)=1$. %Now $\mathbb{F}_{q^n}^*$ is a maximal torus of $GL_n(q)$. 
By  Lemma \ref{sin},  if $\hat{x}, \hat{x}^g \in \mathbb{F}_{q^n}^*$, then there exists $g_1 \in S$ such
that $\hat{x}^g=\hat{x}^{g_1}.$ Let $g_1 =(\mu,s ) \in \mathbb{F}_{q^n}^{*}\rtimes \mathbb{Z}_n$, so $\hat{x}^{(\mu, s)}= \hat{x}^{(1,s)}$ since $\hat{x} \in \mathbb{F}_{q^n}^{*}.$ Thus, 
\begin{equation}\label{8}
|x^G \cap H| = |\hat{x}^{GL_n(q)} \cap S|\le |\mathbb{Z}_n|=n.
\end{equation}


If $k=3$, then   $(\lambda,j)^3=(\lambda^{q^{2(n-j)} +q^{n-j}+1},3j)=(\delta,0)$, where $\delta \in \mathbb{F}_q,$ so   $j=4,2,0.$ 
 If $j=4$, then $\lambda^{q^4+q^2+1}=\delta$. If 
$j=2$, then $\lambda^{q^8+q^4+1}=\delta$. Notice that $q^8 +q^4 +1= (q^6-1)q^2 + (q^4+q^2+1)$. Therefore, in both cases  $$\lambda^{q^4+q^2+1}=\delta.$$ Let $\theta$ be a generator of $\mathbb{F}_{q^n}^*$. Since  $(q^6-1)=(q^4+q^2+1)(q^2-1)$, we deduce that  $\lambda = \theta^{m(q+1)}$, where $m=1, \ldots, (q^4 +q^2+1)(q-1).$ 
If $j=0$, then 
$x$ is an element of  $\mathbb{F}_{q^n}^*/\mathbb{F}_q^*$, and there are only two such elements of order 3. 
 In total, the number of elements of order 3 in $H$  is at most $2(q^4+q^2+1)(q-1)+2$,
 so 
\begin{equation}\label{9}
|x^G \cap H|\le 2(q^4+q^2+1)(q-1)+2.
\end{equation}

If $k=2$, then $(\lambda, j)^2=(\lambda^{q^{(n-j)}+1}, 2j)=(\delta,0)$, where $\delta \in \mathbb{F}_q,$ so $j=3,0.$  If $j=3$, then $$\lambda^{q^3+1}=\delta.$$ Thus, $\lambda= \theta^{m\cdot(q^2+q+1)}$ where $m=1, \ldots, (q^3+1)(q-1).$
If $j=0$, then 
$x$ is an element of  $\mathbb{F}_{q^n}^*/\mathbb{F}_q^*$, and there is only one such element of order 2.
 In total,  the number of involutions in $H$ is at most  $(q^3+1)(q-1)+1$,
 so 
\begin{equation}\label{10}
|x^G \cap H|\le (q^3+1)(q-1)+1.
\end{equation}
 Bounds \eqref{8} -- \eqref{10} and \eqref{5uni} are sufficient for $n=6.$ 

Part $(2)$ of the lemma follows from \eqref{0}, Lemma \ref{nuuni}, bounds  \eqref{5uni}, Lemma \ref{simpcycl} and $(2)$ of Lemma \ref{uniqorder} for all $q$ and for $n \ge 10$. These bounds do not suffice when $6 \le n \le 9$ and $q=2$; here the lemma is verified by computation.  
For  $n \le 5$  part $(2)$ follows by \cite[Table 2]{burness} with a finite number of exceptions verified by computation. 

Part $(3)$ of the lemma follows from \eqref{0}, Lemma \ref{nuuni}, bounds  \eqref{5uni}, Corollary \ref{simpcyclcor} and $(2)$ of Lemma \ref{uniqorder} for all $q$ and for $n \ge 16$. The list of cases when these bounds are not sufficient for $6 \le n \le 14$ is finite. Using Remark \ref{pderem} we reduce this list to  $6 \le n \le 8$ and $q \in \{2,3,5,7\}$; here the lemma is verified by computation.  
\end{proof}

We use the notation of Lemma $\ref{olaf}$ in the following lemma.

\begin{Lem}
\label{c6small}
 Let $S \le \hat{G} \in \{GL_n(q), GU_n(q), GSp_n(q)\}$ be a quasi-primitive maximal solvable subgroup.  Recall that $q=p^f.$ If $e=n=r^l$ for some integer $l$ and prime $r$, then the following hold:
\begin{enumerate}[font=\normalfont]
\item $T=Z(F)=C_S(E)=Z(\hat{G})$;
\item $S=S_1 \cdot Z(\hat{G})$ where $S_1=S \cap GL_n(p^t)$,   $t$ divides $f$, and $S_1$ lies in the normaliser $M$ in $GL_n(p^t)$ of an absolutely irreducible symplectic-type subgroup of $GL_n(p^t)$. So $M$ is a maximal group of $\hat{G} \cap GL_n(p^t).$  
\end{enumerate} 
\end{Lem}
\begin{proof}
By \cite[Lemma 2.10]{manz}, $(1)$ follows. Let $W \le V$ be an irreducible $\mathbb{F}_{q^{\bf u}}[F]$-submodule. By Theorem \ref{olaf}, $F=O_{r'}(Z) \cdot F_1$ where $F_1$ is extra-special of order $r^{2l+1},$ so $W$ is a faithful irreducible $\mathbb{F}_{q^{\bf u}}[F_1]$-module. Therefore, by \cite[Proposition 4.6.3]{kleidlieb}, $\dim W=r^{l}$, and $F_1$ is an absolutely irreducible subgroup of $\hat{G} \cap GL_n(p^t)$  where $\mathbb{F}_{p^t}$ is the smallest field over which such a representation of $F_1$ can be realised. By  \cite[Theorem 2.4.12]{short}, $S=S_1 \cdot Z(\hat{G})$ where $S_1 \le N_{GL_n(p^t)}(F_1)=M.$ %so $S_1$ lies in a maximal subgroup $M \in \mathcal{C}_6$ of $\hat{G} \cap GL_n(p^t)$ (see  \cite[\S 4.6]{kleidlieb} for the definition of $\mathcal{C}_6$). 
\end{proof}


\subsection{ Primitive and quasi-primitive maximal solvable subgroups for $n \le 5$}\label{sec5}

Since Theorem \ref{sch} gives us sufficient results for $n \le 5$ in case {\bf U}, in this section we consider cases {\bf L} and {\bf S} only.

 In  case {\bf L}  we assume that $S$ is a  primitive maximal solvable subgroup of $GL_n(q)$ of  degree $n \le 5.$ The equation  $b_S(S \cdot SL_n(q))=2$ does not always hold for such $n$. However, in view of Lemmas %\ref{irrtog}, \ref{prtoirr} and \ref{diag} for every maximal primitive solvable subgroup $S$ it suffices  to find $x \in GL(n,q)$  such that
\ref{irrtog}, \ref{diag}   and \ref{prtoirr}, for every primitive maximal  solvable subgroup $S$ it suffices  to find $x \in SL_n(q)$  such that
\begin{equation*}%\label{uptr}
 S \cap S^x \le RT(GL_n(q))
\end{equation*}
 to prove that $b_M(M \cdot SL_n(q)) \le 5$ for every maximal solvable subgroup $M$.  Therefore, if $b_S(S \cdot SL_n(q))>2$, then we decide if there exists $x \in SL_n(q)$ such that  $S \cap S^x$ lies in $D(GL_n(q))$ or $RT(GL_n(q)).$ Recall that $D(GL_n(q))$ and $RT(GL_n(q))$ denote the subgroups of all diagonal and all upper-triangular matrices in $GL_n(q)$ respectively. In particular, we prove in this section that if $q>7$, then such $x$ always exists, so  $$b_M(M \cdot SL_n(q)) \le 5$$
for every maximal solvable subgroup $M$ of $GL_n(q)$, for every $n \ge 2$ and $q>7.$ Also we use this information in the proof of Theorem \ref{theorem} in Section \ref{secproof}.

In case {\bf S} we assume that $S$ is a quasi-primitive solvable subgroup of $GSp_n(q)$ such that $Z(GSp_n(q)) \le S$ and $S$ is not contained in any larger solvable subgroup of $GSp_n(q).$ Our aim is to prove that $b_S(S \cdot Sp_n(q)) \le 3,$ which is sufficient for the proof of Theorem \ref{theoremSp}.



Let us consider $n \in \{3,5\}$ first since %  case {\bf S} is not realised  for odd $n$, and
 we only need to deal with  case {\bf L}.

\medskip

{\bf Degree 3.} If $n=3$, then by \cite[\S 21.3]{sup} either $S$ is an absolutely irreducible subgroup such that $S/Z(GL_3(q))$ is isomorphic to $3^2.Sp_2(3)$ or $S$ is the normaliser of a Singer cycle.  The  first case arises only if 3 divides  $q-1$; here  $b_S(S \cdot SL_3(q))=2$  by \cite[Table 2]{burness} and Lemma \ref{c6small}.  In the second case $b_S(S \cdot SL_3(q))=2$ for $q>2$ by \cite[Table 2]{burness} and computation. 
If $q=2$ and $S$ is the normaliser of the Singer cycle generated by the matrix 
$$ \begin{pmatrix}
0    & 0 & 1  \\
1 &0    & 0  \\
  0&    1    &   1      
\end{pmatrix},$$ then computation shows that 
$$ S\cap S^x = \left\langle
 \begin{pmatrix}
1    & 1 & 1  \\
0 &1    & 0  \\
  1&    0    &   0      
\end{pmatrix} 
\right\rangle $$
has order 3 where $$
x= \begin{pmatrix}
1    & 0 & 1  \\
1 &1    & 1  \\
  0&    0    &   1      
\end{pmatrix}.$$
\medskip

{\bf Degree 5.} If $n=5$, then by \cite[\S 21.3]{sup} either $S$ is the normaliser of a Singer cycle or $S$ is as in Lemma \ref{c6small}. %In the first case $b_S(GL_5(q))=2$ by \cite[Table 2]{burness}.
 The  second case arises only if 5 divides  $q-1$. In both cases  $b_S(S \cdot SL_5(q))=2$  by \cite[Table 2]{burness}. 
\medskip

{\bf Degree 2.} Notice that $GSp_2(q)=GL_2(q).$ Let $\hat{G}$ be $GL_2(q)$.  If $q \in \{2,3\},$ then $GL_2(q)$ is solvable. If $q>9$, then $b_S(\hat{G}) \le 3$  by \cite[Table 2]{burness}. For $4 \le q \le 9$ the inequality  $b_S(\hat{G}) \le 3$ is verified by computation.

 Let us consider the case {\bf L} more closely now. If $n=2$ and $q>3$, then by \cite[\S 21.3]{sup} either $S$ is the normaliser of a Singer cycle, or $S$  is as in Lemma \ref{c6small}. The  second case arises only if $q$ is odd. For such $S$,  $b_S(S \cdot SL_2(q))=2$ if $q \ne 7$ and $b_S(S \cdot SL_2(7))=3$ by \cite[Table 2]{burness}. 

Suppose  that $S$  is the normaliser of a Singer cycle, so $S \cdot SL_2(q)=GL_2(q).$ First, let $q$ be odd and let $a \in \mathbb{F}_q$ have no square roots in $\mathbb{F}_q$ (there are $(q-1)/2$ such elements in $\mathbb{F}_q$ if $q$ is odd). Consider 
\begin{equation}\label{thesin}
 S_a= \left\{
\alpha \begin{pmatrix}
1    & 0  \\
      0    &   1       
\end{pmatrix} + 
\beta \begin{pmatrix}
0    & 1  \\
a    &  0       
\end{pmatrix}
\right\} \backslash \{0\}.
\end{equation}
 Notice that 
$$\det \begin{pmatrix}
\alpha    & \beta  \\
    a  \beta    & \alpha       
\end{pmatrix} =0$$ if, and only if, $a=(\alpha/ \beta)^2,$ so all matrices in $S_a$ are invertible. 
 Calculations show that $S_a$ is an abelian subgroup of $GL_2(q)$ of order $q^2-1$ and $S_a \cup \{0\}$ under usual matrix addition and multiplication is a field, so $S_a$ is a Singer cycle. Notice that 
$$\varphi = \begin{pmatrix}
-1    & 0  \\
0    &  1       
\end{pmatrix}$$
normalises $S_a.$ Therefore, we can view $S$ as   $N_{GL_2(q)}(S_a)=S_a \rtimes \langle \varphi \rangle$. Moreover, $N_{\GL_2(q)}(S_a)$ is solvable and 
$$N_{\GL_2(q)}(S_a) = S_a \rtimes \left\langle \phi \begin{pmatrix}
a^{-(p-1)/2}    & 0  \\
0    &  1       
\end{pmatrix} \right\rangle$$ 
where $\phi : \lambda v_i \mapsto \lambda^p v_i$ for $\lambda \in \mathbb{F}_q$ and $\{v_1, v_2\}$ is the basis of $V$ with respect to which matrices from $S_a$ have shape \eqref{thesin}. 

 It is easy to see that if there exist $a, b \in \mathbb{F}_q$ such that $a \ne b$
and neither $a$ nor $b$ has square roots in $\mathbb{F}_q$, then $$S_a \cap S_b \le Z(GL_2(q)).$$ 
If, in addition, $a \ne -b$, then calculations show that 
\begin{equation}
\label{2sindiagodd}
 N_{\GL_2(q)}(S_a) \cap N_{\GL_2(q)}(S_b) =\langle \varphi \rangle Z(GL_n(q)) \le D(GL_2(q)).
\end{equation} 
It is possible to find such $a$ and $b$ if $q>5,$ so in this case there exists $x \in SL_2(q)$ such that $$S \cap S^x \le D(GL_2(q)),$$ since all Singer cycles are conjugate in $GL_2(q)$ and $\Det(S_a)=\Det(GL_2(q)).$

Let $q$ be even and let $a \in \mathbb{F}_q $ be such that there are no roots of $x^2+x+a$ in $\mathbb{F}_q$ (there are $q/2$ such elements in $\mathbb{F}_q$). 
 Consider  
\begin{equation} \label{thesineven}
S_a= \left\{
\alpha \begin{pmatrix}
1    & 0  \\
      0    &   1       
\end{pmatrix} + 
\beta \begin{pmatrix}
0    & 1  \\
a    &  1       
\end{pmatrix}
\right\} \backslash \{0\}.
\end{equation} Notice that 
$$\det \begin{pmatrix}
\alpha    & \beta  \\
    a  \beta    & \alpha +\beta      
\end{pmatrix} =0$$ if, and only if, $a=(\alpha/ \beta)^2 +(\alpha/\beta),$ so all matrices in $S_a$ are invertible. 
 Calculations show that $S_a$ is an abelian subgroup of $GL_2(q)$ of order $q^2-1$ and $S_a \cup \{0\}$ under usual matrix addition and multiplication is a field, so $S_a$ is a Singer cycle. Notice that 
$$\varphi = \begin{pmatrix}
1    & 0  \\
1    &  1       
\end{pmatrix}$$
normalises $S_a.$  Therefore, we can view $S$ as   $N_{GL_2(q)}(S_a)=S_a \rtimes \langle \varphi \rangle$. Moreover, $N_{\GL_2(q)}(S_a)$ is solvable and 
$$N_{\GL_2(q)}(S_a)= (S_a) \rtimes \left\langle \phi \begin{pmatrix}
a^{-1}    & 0  \\
1    &  a^1       
\end{pmatrix} \right\rangle$$ 
where $\phi : \lambda v_i \mapsto \lambda^p v_i$ for $\lambda \in \mathbb{F}_q$ and $\{v_1, v_2\}$ is the basis of $V$ with respect  to which matrices from $S_a$ have shape \eqref{thesineven}.

 It is easy to see that if there exist $a, b \in \mathbb{F}_q$ such that $a \ne b$
and neither $x^2+x+a$ nor $x^2+x+b$ has roots in $\mathbb{F}_q$, then $$S_a \cap S_b \le Z(GL_2(q)).$$
Calculations show that  
\begin{equation}
\label{2sindiageven}
N_{\GL_2(q)}(S_a) \cap N_{\GL_2(q)}(S_b)=\langle \varphi \rangle Z(GL_n(q))
\end{equation} consists of  lower triangular matrices. It is possible to find such $a$ and $b$ if $q \ge 4,$ so in this case there exists $x \in SL_2(q)$ such that all matrices in $S \cap S^x$ are  lower triangular,   since all Singer cycles are conjugate in $GL_2(q)$ and $\Det(S_a)=\Det(GL_2(q)).$
\medskip



{\bf Degree 4.} We claim that $b_S(S \cdot SL_4(q))=2$ for all $q$ in case ${\bf L}$ and $b_S(S \cdot Sp_4(q))\le 3$ for all $q$ in case {\bf S}. We provide here the proof for  {\bf S}. The proof for {\bf L} is  analogous. 
\begin{comment}
 By Lemma \ref{suplem}, $S$ contains the unique maximal abelian normal subgroup $A$ such that $A$ is the multiplicative group of an extension field $K$ of the field of scalar matrices $\Delta$ and $m=|K:\Delta|$ divides $n$.  If $m=1$, then, by Lemma \ref{suplem}, $S/Z(GL_4(q))$ lies in the absolutely irreducible subgroup isomorphic to $2^4.Sp(4,2)$ and such a subgroup exists only if $q$ is odd. By \cite[Table 2]{burness}, $b_S(GL_4(q))=2$ in this case. %The case q=3....

Let $m\ge 2.$ Therefore, $S$ lies in either in $GL_2(q^2).2$ or in $GL_1(q^4).4$.  

If $q$ is even then $m=4$ by \cite[\S 20, Corollary 3.1]{sup} and $S=T \rtimes \langle \phi \rangle$ is the normaliser of a Singer cycle $T$. Therefore, $|S|=|T||\langle \phi \rangle|$, where $|T|=q^4-1$ and $|\langle \phi \rangle|=4$, so $\langle \phi \rangle$ is a $2$-Sylow subgroup of $S$ and it contains only one element of order $2$. 

  Let $G=PGL_4(q)$ and let $H$ be the image of $S$ in $G$ under the natural homomorphism. In this case 
 we claim  that $Q(G,2)<1$ in \eqref{ver}.  
Let $x \in G$ have  prime order. Denote by $k_{s,r,u}$ and $k_{s,r,s}$ the numbers of conjugate classes of unipotent and semisimple elements of prime order $r$ in $G$ such that $\nu(x)=s$ respectively. By \cite[Propositions 3.24 and 3.40]{fpr2}
\begin{equation}\label{40}
\begin{split}
k_{s,p,u}&\le p^{s/2};\\
k_{s,r,s} &\le 
\left\{ 
\begin{split}
q^s, \text{ if } s<n/2;\\
q^{s+1}, \text{ otherwise}.
\end{split}
\right.
\end{split}
\end{equation}

 If $r \ne 2$, then
\begin{equation} \label{41}
 |x^G \cap H|\le 4
\end{equation}
 by  Lemma \ref{sin} and 
\begin{equation}\label{42}
|x^G|>
\left\{ 
\begin{split}
&q^{2s(n-s)}/2,& \text{ if } s<n/2;\\
&q^{ns}/2, & \text{ otherwise}
\end{split}
\right.
\end{equation}
by \cite[Lemma 3.34 and Proposition 3.36 ]{fpr2}.


 If $r=2$ then, as  mentioned above, $x$ is conjugate to the image of $\phi^2.$ Represent $S$ in the form $\mathbb{F}_{q^n}^{*} \rtimes \mathbb{Z}_n$ and consider the subspace $W$ of $\mathbb{F}_q^n$ consisting of vectors fixed by $\phi^2$. By Lemma \ref{sin1}, $v^{\phi^2}=v^{(1,2)}=v^{q^2},$ therefore, this subspace consists of  $v\in \mathbb{F}_q^n=\mathbb{F}_{q^n}$ such that
$v^{q^2-1}=0,$  so $\dim W=2$ and  $\nu(x)=2$. 

Since all elements of order $2$ are conjugate in $H$, $$|x^G \cap H|=|x^H|=|(\phi^2)^S| =|S|/ |C_S(\phi^2)|.$$ Let us compute the order of $C_S(\phi^2).$ Assume that $\lambda \in T$,  $i=1, \ldots, 4$ and $ \lambda \phi^i$ centralises $\phi^2$, so
$$(\lambda \phi^i)^{\phi^2}= \lambda^{q^2}\phi^i=\lambda \phi^i.$$
Therefore, $\lambda^{q^2-1}=1$ and $i$ can be any integer from $1$ to $4$. Thus,   $|C_S(\phi^2)|=4(q^2-1)$, so $$|x^H|=q^2+1.$$

 By \cite[Proposition 3.22]{fpr2}, 
$$|x^G|>\frac{1}{2}\left(\frac{q}{q+1} \right) \max(q^{2s(n-s),q^{ns}})=\frac{1}{2}\left(\frac{q}{q+1} \right)(q^{2n}).$$ 
There are fewer than $\log(q^3+q^2+q+1)$ prime dividers of $q^3+q^2+q+1$. Since
\begin{equation*}
\begin{aligned}
Q(G,2)  & \le \sum_{{x} \in \mathscr{P}}\fpr({x})^2  = \sum_{{x} \in \mathscr{P}} \left(\frac{|x^G \cap H|}{|x^G|}\right) ^2\\  
  & <  \frac{(q^2+1)^2}{1/2(q/(q+1))q^8}  
+\log \left( \frac{q^4-1}{q-1} \right) \left(\left(\frac{4^2q}{1/2 \cdot q^6} \right) +\left(\frac{4^2 q^3}{1/2 \cdot q^8} \right) + \left(\frac{4^2q^4}{1/2 \cdot q^{12}} \right) \right),
\end{aligned}
\end{equation*}
 $Q(G,2)<1$ and $b_S(GL_4(q))=2$ for  $q \ge 4.$ If $q=2$ then $b_S(GL_4(q))=2$ is established using {\sf GAP}. 
%\end{equation*}

Now let $q$ be odd and $m=4$, so $S$ is a normaliser of a Singer cycle. Let us compute the number of elements $(\lambda \phi^i) \in S$ such that $(\lambda \phi^i)^2$ is scalar. Therefore, $$(\lambda \phi^i)^2=\lambda^{q^{4-i}+1}\phi^{2i}= \theta^{k(q^3+q^2+q+1)},$$
where $\theta$ is a generator of $T$ and $k= 1, \ldots , q-1$, so $\theta^{k(q^3+q^2+q+1)}$ is scalar. Thus, there are two cases: $i=4$ and $i=2$. In the first case $\lambda^2 = \theta^{k(q^3+q^2+q+1)},$ so there are $2(q-1)$ such elements.
In the second case $\lambda^{q^2+1} = \theta^{k(q^3+q^2+q+1)},$ so there are $(q^2+1)(q-1)$ such elements. Therefore, there are $q^2+2$ elements of order 2.

Thus, if $r=2$ then $|x^G \cap H|\le q^2+2$ 
and 
\begin{equation}\label{polup2}
|x^G|>
\left\{ 
\begin{split}
&q^{6}/2,& \text{ if } s=1;\\
&q^{8}/4, &\text{ if } s=2
\end{split}
\right.
\end{equation}
by \cite[Proposition 3.37]{fpr2}. Also by \cite[Table 3.8]{fpr2}, $s$ can be only $1$ or $2$ and $k_{1,2,s}=1,$ $k_{2,2,s}=2.$ 

For $r\ne 2$ we use \eqref{41} and \eqref{42}. Thus
\begin{equation*}
\begin{aligned}
Q(G,2) & \le \sum_{{x} \in \mathscr{P}}\fpr({x})^2 = \sum_{{x} \in \mathscr{P}} \left(\frac{|x^G \cap H|}{|x^G|}\right) ^2\\
 &<  \frac{(q^2+2)^2}{1/2q^6} + \frac{(q^2+2)^2}{1/4q^8}  
+\log\left(\frac{q^4-1}{q-1}\right) \left(\left(\frac{4^2q}{1/2 \cdot q^6} \right) +\left(\frac{4^2q^3}{1/2 \cdot q^8} \right) + \left(\frac{4^2q^4}{1/2 \cdot q^{12}} \right) \right),
\end{aligned}
\end{equation*}
so $Q(G,2)<1$ and $b_S(GL_4(q))=2$ for  $q \ge 5.$ If $q=3$ then $b_S(GL_4(q))=2$ is established using {\sf GAP}.

Now let $m=2$. By Lemma \ref{suplem}, $S$ has a chain of normal subgroups:
$$S \ge C \ge F \ge A,$$
where $C \le GL_2(q^2),$ so $S \le GL_2(q^2).2,$ and $C/A$ is isomorphic to a subgroup of $2^2.Sp(2.2).$ 
Therefore, $|S| \le 48(q^2-1)$ and $|H| \le 48(q+1).$ If $r \ne 2, 3$ then $r$ divides $q+1$ and, since $r$ does not divide $n$, by Lemma \ref{prost} $x$ has a preimage $\hat{x} \in S$ of order $r$. Since $A$ is a normal cyclic subgroup of order $q^2-1$, it contains the unique Sylow cyclic $r$-subgroup of $S.$ Therefore, the number of elements of order $r$ in $S$ is at most $r-1\le q+1-1=q,$ so $|x^G \cap H|\le q$. There are fewer than $\log(q+1)$ prime divisors of $q+1$, so using \eqref{42} 
\begin{equation*}
\begin{aligned}
Q_1&= \sum_{{x} \in \mathscr{P}; |x|>3}\fpr({x})^2 = \sum_{{x} \in \mathscr{P}; |x|>3} \left(\frac{|x^G \cap H|}{|x^G|}\right) ^2\\
&<    
\log(q+1) \left(q\left(\frac{q^2}{1/2 \cdot q^6} \right) +q^3\left(\frac{q^2}{1/2 \cdot q^8} \right) + q^4\left(\frac{q^2}{1/2 \cdot q^{12}} \right) \right).
\end{aligned}
\end{equation*}
If $r=2$ then we use the bounds $|x^G \cap H|\le H$ and \eqref{polup2}, so 
$$
Q_2 = \sum_{{x} \in \mathscr{P}; |x|=2}\fpr({x})^2 = \sum_{{x} \in \mathscr{P}; |x|=2} \left(\frac{|x^G \cap H|}{|x^G|}\right) ^2
<  \frac{(48(q+1))^2}{1/2q^6} + 2\frac{(48(q+1))^2}{1/4q^8}.
$$

Let $r=3$ and let $\hat{T}$ be a Sylow $3$-subgroup of $S$. Since $|S:C|=2$, $\hat{T}$ lies in $C$. Therefore, $\hat{T}=T_1 \times T_2$ where $|T_1|=3$ and $T_2$ is the Sylow $3$-subgroup of $A$, since $A$ centralises $C.$ Thus, $A$ normalises $\hat{T}$ and the number of Sylow $3$-subgroups in $S$ is at most $|S:A|=48.$ If $T$ is the image of $\hat{T}$ in $PGL_4(q)$ and 3 does not divide $(q+1)$, then $|T|=3$ and there are 2 elements of order 3 in $T$, so $|x^G \cap H| \le 48 \cdot 2.$ If 3 divides $(q+1)$, then there are $8$ elements in $T$ since $F$ is cyclic, so $|x^G \cap H| \le 48 \cdot 8.$ If $p\ne 3$ we use  \eqref{42} for $|x^G|$. If $p=3$ then by \cite[Proposition 3.22]{fpr2}
$$|x^G|>\frac{1}{2}\left(\frac{q}{q+1} \right) \max(q^{2s(n-s),q^{ns}}).$$ Therefore, if $p \ne 3$ and 3 divide $(q+1)$, then \eqref{40}  gives us
\begin{equation*}
\begin{aligned}
Q_{3} & = \sum_{{x} \in \mathscr{P}; |x|=3}\fpr({x})^2 = \sum_{{x} \in \mathscr{P}; |x|=2} \left(\frac{|x^G \cap H|}{|x^G|}\right) ^2\\
 & <  \left(q\left(\frac{(48 \cdot 8)^2}{1/2 \cdot q^6} \right) +q^3\left(\frac{(48 \cdot 8)^2}{1/2 \cdot q^8} \right) + q^4\left(\frac{(48 \cdot 8)^2}{1/2 \cdot q^{12}} \right) \right) 
\end{aligned}
\end{equation*}
 If $p \ne 3$ and 3 does not divide $(q+1)$ then   
$$Q_3 = \left(q\left(\frac{(48 \cdot 2)^2}{1/2  \cdot q^6} \right) +q^3\left(\frac{(48 \cdot 2)^2}{1/2  \cdot q^8} \right) + q^4\left(\frac{(48  \cdot 2)^2}{1/2  \cdot q^{12}} \right) \right). 
$$
If $p = 3$ then

$$Q_3 = \left(\left(\frac{(48 \cdot 2)^2}{1/2 (q/(q+1)) \cdot q^6} \right) +3\left(\frac{(48 \cdot 2)^2}{1/2 (q/(q+1)) \cdot q^8} \right) + 3^{3/2}\left(\frac{(48 \cdot 2)^2}{1/2 (q/(q+1)) \cdot q^{12}} \right) \right). 
$$  

Computations show that $Q(G,2)\le Q_1+Q_2+Q_3 <1$ for $q>11.$ If $q \le 11$ then  $b_S(GL_4(q))=2$ is established using {\sf GAP}.
\end{comment}




%%%%%%%%%%%%%%%%%%%%%%%%%%%%%%%%%%%%%%%%%%%%%%%%%%%%%%%%%%%%%%%%%%%%%%%%%%%%%%%%5%%%%

\medskip

Let $S$ be a quasi-primitive maximal solvable subgroup  of $GSp_4(q)$ and $\hat{G}=S \cdot Sp_4(q).$ We use notation from Lemmas \ref{olaf} and \ref{uniqorder}. The proof splits into several cases depending on values of $e$, $m,$ and $q$.

\medskip

\underline{\it Case {$e=4.$}} If $e=4$, then $m=1$, $q$ is odd by Remark \ref{eqrem} and $S$ is as in Lemma \ref{c6small}. By \cite[Table 2]{burness}, $b_S(S \cdot Sp_4(q)) \le 3$  for $q>3;$ for $q=3$ the statement  $b_S(S \cdot Sp_4(q)) \le 3$ is established by computation.


\medskip


 Let $G$ and $H$ be the image of $\hat{G}$ and $S$ in $PGSp_4(q)$ under the natural homomorphism respectively. In the remaining cases 
 we claim  that $Q(G,3)<1$ in \eqref{ver}.  
%Let $x \in G$ have  prime order. 
Denote by $k_{s,r}$ the number of conjugacy classes of  $x \in PGSp_4(q)$ of prime order $r$  with $\nu(x)=s$. By \cite[Propositions 3.24 and 3.40]{fpr2}
\begin{equation}\label{40s}
k_{s,p} \le p^{s/2} \text{ and }
k_{s,r}  \le 
\left\{ 
\begin{aligned}
&q^{\xi s}, &\text{ if } s<n/2;\\
&q^{\xi(s+1)}, &\text{ otherwise},
\end{aligned}
\right.
\end{equation}
where $r \ne p$, $\xi=1$ in the case {\bf L} and $\xi=1/2$ in the case {\bf S}. Let $A(s,r)$ and $B(s,r)$ be  lower bounds for $|x^{PGSp_4(q)}|$ and $|x^G|$ respectively. Let $C(s,r)$ be an upper bound for $|x^G \cap H|.$   Notice that $A(s,r)$, $B(s,r)$ and $C(s,r)$ also depend on $n$ and $q$. Therefore, 
\begin{equation}\label{QABC}
Q(G,3) \le \sum_{{x} \in \mathscr{P}}\fpr({x})^3 \le \sum_{r \text{ divides } |H|}
\sum_{s=1}^{n-1} k_{s,r} \cdot A(s,r) \left(\frac{C(s,r)}{B(s,r)}  \right)^3. 
\end{equation}

\medskip

\underline{\it Case {$q$} is even.} In this case $PGSp_n(q)$ is $PSp_n(q)$ by \cite[Proposition 2.4.4]{kleidlieb}, so $B(s,r)=A(s,r).$  If $q$ is even, then $e=1$ by Remark \ref{eqrem}, so $n=m$ and $S$ is a subgroup of the normaliser $N=T \rtimes \langle \varphi \rangle$ of a Singer cycle $T$ of $GL_4(q)$ by Lemma \ref{uniqorder}. Here $T$ and $\varphi$ are as in Lemma \ref{sin1}. 


 Therefore, $|S|$ divides $|T|\cdot|\langle \varphi \rangle|$, where $|T|=q^4-1$ and $|\langle \varphi \rangle|=4$, so $\langle \varphi \rangle$ is a $2$-Sylow subgroup of $N$ and it contains only one element of order $2$. 

 If $r \ne 2$, then
\begin{equation} \label{41s}
 C(r,s):= |x^G \cap H|\le 4
\end{equation}
 by  Lemma \ref{sin} and 
\begin{equation}\label{42s}
A(s,r)=B(s,r):=
(1/2) \max(q^{s(n-s)},q^{ns/2})
\end{equation}
by  \cite[Lemma 3.34 and Proposition 3.36]{fpr2}.

  By the proof of Lemma \ref{nuuni}, if $S$ is a subgroup of the normaliser of a Singer cycle of $GL_n(q)$, then $\nu(x) \ge n/2$ for all $x \in H,$ since $C=F=U.$ Therefore, $\nu(x) \in \{2,3\}$.


 If $r=2$ then, as  mentioned above, $x$ is conjugate to the image of $\varphi^2.$ Recall the identification of $\mathbb{F}_{q^n}$ with $\mathbb{F}_q^n$ from Lemma \ref{singex}. Represent $N$ in the form $\mathbb{F}_{q^n}^{*} \rtimes \mathbb{Z}_n$ and consider the subspace $W$ of $\mathbb{F}_q^n$ consisting of vectors fixed by $\varphi^2$. By Lemma \ref{sin1}, $v^{\varphi^2}=v^{(1,2)}=v^{q^2};$ therefore, $W$ consists of  $v\in \mathbb{F}_q^n \cong \mathbb{F}_{q^n}$ (as vector spaces) such that
$v^{q^2-1}=0,$  so $\dim W=2$ and  $\nu(x)=s=2$. 

Since all elements of order $2$ are conjugate in $H$, $$|x^G \cap H|=|x^H|\le|(\varphi^2)^N|= |N|/ |C_N(\varphi^2)|.$$ Let us compute the order of $C_N(\varphi^2).$ Assume that $\lambda \in T$  and $ \lambda \varphi^i$ centralises $\varphi^2$ where $i=1, \ldots, 4$, so
$$(\lambda \varphi^i)^{\varphi^2}= \lambda^{q^2}\varphi^i=\lambda \varphi^i.$$
Therefore, $\lambda^{q^2-1}=1$ and $i \in \{1, \ldots, 4\}$. Thus,   $|C_N(\varphi^2)|=4(q^2-1)$, so $$C(2,2):=|x^H|\le q^2+1.$$

 By \cite[Proposition 3.22]{fpr2}, 
$$A(2,2)>\frac{1}{4}\left(\frac{q}{q+1} \right) \max(q^{s(4-s)},q^{4s/2})=\frac{1}{4}\left(\frac{q}{q+1} \right)(q^{4}).$$ 
The intersection $T \cap Sp_4(q)$ has at most $q^2+1$ elements by Lemma \ref{simpcycl}. So there are fewer than $\log_2(q^2+1)$ distinct odd prime divisors of $|H|$. Thus,
\begin{equation*}
\begin{aligned}
Q(G,3)  & \le \sum_{r \text{ divides } |H|} \left(
\sum_{s=1}^{n-1} k_{s,r} \cdot A(s,r) \left(\frac{C(r,s)}{B(s,r)}  \right)^3 \right)\\  
  & <  \frac{(q^2+1)^3}{(1/4(q/(q+1))q^4)^2}  
+\log_2 \left( q^2+1 \right) \left(\frac{4^3 q^{3/2}}{((1/2) \cdot q^4)^2}  + \frac{4^3q^2}{((1/2) \cdot q^{6})^2}  \right),
\end{aligned}
\end{equation*}
so $Q(G,3)<1$ and $b_S(\hat{G})\le 3$ for  $q > 4.$ If $q=2, 4$ then $b_S(\hat{G})\le 3$ is established by computation. 

\medskip

\underline{\it Case ${q}$ is odd and ${e=1}$.} Now let $q$ be odd and $e=1$, so $m=4$ and $S$ is again a subgroup of the normaliser $N$ of a Singer cycle $T$ of $GL_4(q)$. Since $q$ is odd, there is no element of order $p$ in $H$. Let us compute the number of elements $(\lambda \phi^i) \in S\le N$ such that $(\lambda \varphi^i)^2$ is scalar, so $(\lambda \varphi^i)^2 \in Z(\hat{G})= Z(GSp_n(q)).$ Notice that $|Z(GSp_n(q))|=|Z(GL_n(q))|=q-1.$ Since $$(\lambda \phi^i)^2=\lambda^{q^{4-i}+1}\phi^{2i} \in Z(\hat{G}),$$
 there are two possibilities: $i=4$ and $i=2$. If $i=4$, then $\lambda^2 \in Z(\hat{G})$, so there are $2(q-1)$ such elements in $T$. 
In the second case $\lambda^{q^2+1} \in Z(\hat{G}),$ so there are $(q^2+1)(q-1)$ such elements. Therefore, there are at most $((q^2+1)+2)-1=q^2+2$ elements of order two in $H$.

Thus, if $r=2$, then $C(s,r):=|x^G \cap H|\le (q^2+2)$. Also by \cite[Table 3.8]{fpr2} (and since $\nu(x) \ge n/2$ by the proof of Lemma \ref{nuuni}), $s$ can be only  $2$ and $k_{2,2}=4.$ So
\begin{equation}\label{polup2s}
A(s,r)=B(s,r):=|x^G|> q^{4}/4
\end{equation}
by \cite[Proposition 3.37]{fpr2}. 

For $r\ne 2$ we use \eqref{41s} and \eqref{42s}. Thus
\begin{equation*}
\begin{aligned}
Q(G,3) & \le  \sum_{r \text{ divides } |H|} \left(
\sum_{s=1}^{n-1} k_{s,r} \cdot A(s,r) \left(\frac{C(s,r)}{B(s,r)}  \right)^3 \right)\\
 & <   \frac{4(q^2+2)^3}{((1/4)q^4)^2}  
+\log_2 \left( q^2+1 \right) \left(\frac{4^3 q^{3/2}}{(1/2 \cdot q^4)^2}  + \frac{4^3q^2}{(1/2 \cdot q^{6})^2}  \right),
\end{aligned}
\end{equation*}
so $Q(G,3)<1$ and $b_S(\hat{G})\le 3$ for  $q \ge 9.$ If $q<9$ then $b_S(\hat{G})\le 3$ is established by computation.

\medskip

\underline{\it Case ${q}$ is odd and ${e=2}$.}
 Since $$|S|= |S/C|\cdot|T/U |\cdot|U|\cdot|C/F |\cdot|F/T |,$$ by $(6)$ of Lemma \ref{olaf}, $|S|$ divides
$2 \cdot 2 \cdot (q^2-1) \cdot |Sp_2(2)| \cdot e^2.$
Therefore,  $|H|$ divides $96(q+1).$ Let $x \in H$ have prime order $r$.  Let $Q_1,$ $Q_2$ and $Q_3$ be  $$\sum_{{x} \in \mathscr{P}; r \mid (q+1)}\fpr({x})^3, \text{ } \sum_{{x} \in \mathscr{P}; r=2}\fpr({x})^3 \text{ and } \sum_{{x} \in \mathscr{P}; r=3}\fpr({x})^3$$ respectively, so $Q(G,3)\le Q_1 + Q_2 +Q_3$. We  find upper bounds for $Q_i,$ $i \in \{1,2,3\}.$

If $r \ne 2, 3$ then $r$ divides $q+1$ and, since $r$ does not divide $n$, by Lemma \ref{prost} $x$ has a preimage $\hat{x} \in U$ of order $r$. Since $U$ is a normal cyclic subgroup of order dividing $q+1$, its image in $PGL_n(q)$ contains the unique Sylow cyclic $r$-subgroup of $S.$ Therefore, the number of elements of order $r$ in $H$ is at most $r-1\le q+1-1=q,$ so $C(s,r):=|x^G \cap H|\le q$. There are fewer than $\log_2(q+1)$ prime divisors of $q+1$, so using \eqref{42s} for $A(s,r)$ and $B(s,r)$ and  using \eqref{40s} for $k_{s,r}$ we obtain
\begin{equation*}
\begin{aligned}
Q_1& \le \sum_{r \mid (q+1)} 
\sum_{s=1}^{n-1} k_{s,r} \cdot A(s,r) \left(\frac{C(s,r)}{B(s,r)}  \right)^3\\
&<    
\log_2(q+1) \left(q^{3/2}\left(\frac{q^3}{((1/2) \cdot q^4)^2} \right) + q^2\left(\frac{q^3}{((1/2) \cdot q^{6})^2} \right) \right).
\end{aligned}
\end{equation*}
%If $r=2$ then we use the bounds $|x^G \cap H|\le H$ and \eqref{polup2s}, so 
%$$
%Q_2 = \sum_{{x} \in \mathscr{P}; |x|=2}\fpr({x})^3 = \sum_{{x} \in \mathscr{P}; |x|=2} \left(\frac{|x^G \cap H|}{|x^G|}\right) ^3
%<  \frac{(2 \cdot 48(q+1))^3}{((1/4)q^3)^2} + 4\frac{(2 \cdot 48(q+1))^3}{((1/4)q^4)^2}.
%$$

 Our arguments  to estimate $Q_2$ and $Q_3$ are more complex and require more work. Our analysis splits into two subcases: $S$ is imprimitive and $S$ is primitive. We use results summarised in the following remark to find $C(s,r)$ for $r \in \{2,3\}.$

\medskip

\begin{Rem}\label{remmi}
Let $r\in \{2,3\}$. For  $L \le GL_n(q)$ let $$c_r(L)=|\{g \in L : g^r \in Z(GL_n(q))\}|.$$ Notice that $|x^G \cap H| \le c_r(S)/|Z(Sp_4(q))|$ for $x \in S/Z(Sp_4(q))$ of prime order. Here we compute $c_r(L)$ for some specific groups.  

By \cite[\S21, Theorem 6]{sup} and \cite[Chapter 5]{short}, a primitive maximal solvable subgroup of $GL_2(q)$ is conjugate to either the normaliser of a Singer cycle or to a certain subgroup of order $24(q-1).$ We follow \cite[Chapter 5]{short} and denote the normaliser of a Singer cycle of $GL_2(q)$ by $M_2$ and the primitive maximal solvable subgroup of order $24(q-1)$ by   $M_3$ and $M_4$ for $q \equiv 3 \bmod 4$ and $q \equiv 1 \bmod {4}$ respectively.  Explicit generating sets of $M_3$ and $M_4$ are listed in \cite[\S 5.2]{short}. It is routine to check that $M_3$ and $M_4$ contain $Z(GL_2(q))$, $c_2(M_i)=10(q-1)$ and $c_3(M_i)=9(q-1)$ for both $i=3,4.$ Notice that $c_3(M_2)=(3,q+1) \cdot (q-1),$ since all  $g \in M_2$ such that $g^3 \in Z(GL_2(q))$ lie in the Singer cycle which is a normal subgroup of $M_2.$ Using the same method as for the case $m=4$ (when $S$ lies in the normaliser of a Singer cycle), we obtain $c_2(M_2)=(q+3)(q-1)$.
\end{Rem}

\bigskip

{\bf Subcase 1.} Assume that $S$ is imprimitive, so there exists a system of imprimitivity
$$V=V_1 \oplus \ldots \oplus V_k.$$ Let $k$ be the maximum possible for $S,$ so $k \in \{2,4\}.$

Let $k=2,$ so $N:=\Stab_S(V_1)=\Stab_S(V_2)$ and $N$ is normal in $S$. Hence $V$ is $\mathbb{F}_q[N]$-homogeneous. Notice that, by \cite[\S 15, Lemma 5]{sup}, in some basis of $V$, $S$ must be a subgroup of $S_1 \wr \Sym(2)$ where $S_1=\Stab_S(V_1)|_{_{V_1}}$. So $N$ is not a group of scalar matrices, since in that case $k$ must be $4.$ Since $S$ is irreducible, $V_i$ is either totally isotropic  or non-degenerate  for both $i=1,2.$  If $V_i$ is totally isotropic, then $S$ lies in a maximal group of $GSp_4(q)$ of type $GL_2(q).2$ and $b_S(\hat{G}) \le 3$ by  \cite[Table 2]{burness}. If $V_i$ is non-degenerate, then  either $V_1 \bot V_2$ and  $S$ lies in a larger solvable subgroup $S_1 \wr \Sym(2) \cap GSp_4(q)$ (which is not quasi-primitive) of $Sp_4(q)$, which contradicts the assumption, or $V_2 \ne V_1^{\bot}.$

Assume that $V_2 \ne V_1^{\bot}$ and consider projection operators $\pi_1$ and $\pi_2$ on $V_1$ and $V_1^{\bot}$ respectively with respect to the decomposition $V=V_1 \oplus V_1^{\bot}.$ Notice that $(V_2)\pi_1$ is $\mathbb{F}_q[N]$-irreducible, so $(V_2)\pi_1=V_1$  since $V_2 \ne V_1^{\bot}.$  Therefore, $(V_2)\pi_2=V_1^{\bot}$ since otherwise $V_1 \cap V_2 \ne 0.$ 

Fix a basis $\beta_1=\{f_1, e_1\}$ of $V_1$ as in \eqref{sympbasis} and basis $\beta_2=\{f_1+w_1, e_1 +w_2\}$ of $V_2$, where $w_1, w_2 \in V_1^{\bot}$ such that $(f_1+w_1), (e_1 +w_2) \in V_2.$ Let ${\bf f}_1$ and ${\bf f}_2$ be the restrictions of {\bf f} to $V_1$ and $V_2$ respectively, so ${\bf f}_i$ is a non-degenerate symplectic form since $V_i$ is non-degenerate for $i=1,2.$  Denote   $({\bf f}_i)_{\beta_i}$ by $\Phi_i$ for $i=1,2.$ Let ${\bf f}(w_1,w_2)= \alpha \in \mathbb{F}_q$,  $\delta=1+\alpha$ and $\beta=\beta_1 \cup \beta_2$. Notice that 
$$\Phi_1=
\begin{pmatrix}
0&1 \\
-1&0
\end{pmatrix}; \text{ }
\Phi_2=
\begin{pmatrix}
0&\delta \\
-\delta&0
\end{pmatrix}; \text{ }
{\bf f}_{\beta}=
\begin{pmatrix}
\Phi_1&\Phi_1 \\
\Phi_1&\Phi_2
\end{pmatrix}.
$$
If $g \in N$, then, in basis $\beta$, $g=\diag[g_1, g_2]$ with $g_1 \in S_1 \le GSp_2(q),$ $g_2 \in GL_2(q).$ By \eqref{unimatr}, $g {\bf f}_{\beta}g^{\top}=\tau(g){\bf f}_{\beta},$ so 
$$
\begin{pmatrix}
g_1\Phi_1 g_1^{\top}&g_1\Phi_1 g_2^{\top} \\
g_2\Phi_1 g_1^{\top}&g_2\Phi_2 g_2^{\top}
\end{pmatrix}=\tau(g)
\begin{pmatrix}
\Phi_1&\Phi_1 \\
\Phi_1&\Phi_2
\end{pmatrix}.
$$
In particular, $$\tau(g)\Phi_1=g_1 \Phi_1 g_2^{\top}=g_1 \Phi_1g_1^{\top}(g_1^{\top})^{-1} g_2^{\top}=\tau(g)\Phi_1(g_1^{\top})^{-1} g_2^{\top},$$
so $(g_1^{\top})^{-1} g_2^{\top}=1$ and $g_1=g_2.$

%Consider  $h \in S \backslash N,$ so $h=\left( \begin{smallmatrix}0& h_1\\ h_2 &0 \end{smallmatrix} \right)$ and $h^2\in N.$ By \eqref{unimatr}, $h {\bf f}_{\beta}h^{\top}=\tau(g){\bf f}_{\beta},$ so 
%$$
%\begin{pmatrix}
%h_1\Phi_2 h_1^{\top}&h_1\Phi_1 h_2^{\top} \\
%h_2\Phi_1 h_1^{\top}&h_2\Phi_1 h_2^{\top}
%\end{pmatrix}= \tau(g)
%\begin{pmatrix}
%\Phi_1&\Phi_1 \\
%\Phi_1&\Phi_2
%\end{pmatrix}.
%$$
%In particular, $$\Phi_1=h_1 \Phi_1 h_2^{\top}=h_1 h_2^{-1}h_2 \Phi_1 g_2^{\top}=h_1h_2^{-1}\Phi_2,$$
%so $h_1h_2^{-1}=\Phi_1 \Phi_2^{-1}=\delta^{-1} I_2$ and $h_2=\delta h_1.$

Hence $N$ consists of matrices $\diag[g_1,g_1]$ with $g_1 \in S_1$ and $S_1$ is a primitive subgroup of $GSp_2(q)$ since otherwise $k=4.$ Therefore, $S_1$ is a subgroup of a primitive maximal solvable subgroup $M$ of $GL_2(q),$ so, by \cite[Chapter 5]{short}, either $M$ is $M_2$ or $M$ is  $M_i$ as in Remark \ref{remmi} with $i=3,4$  and has order $24(q-1).$ It is clear that $N$ is a subgroup of index $2$ in $S$, so, if $M=M_2$, then  $|H|$ divides $(q+1) \cdot 2 \cdot 2$. If $M$ is $M_i$ for $i=3,4$, then $|H|$ divides $24 \cdot 2$.

 Recall that $x \in H$ has prime order $r$ and $\hat{x}$ is a preimage of $x$ in $S$. Notice that if $r\ne 2,$ then $\hat{x} \in N.$ If $\hat{x} \in N$, so $\hat{x}=\diag[g_1,g_1]$ where $g_1 \in S_1,$ then $\nu(g_1)=1$ and $\nu(x)=2.$  If $r=2,$ then $\nu(x)=2$ and $k_{2,2}=4$ by \cite[Table 3.8]{fpr2}. 
So, using \eqref{polup2s} and $C(s,r) := |H|$,% we obtain 
$$
Q_2  \le \sum_{{x} \in \mathscr{P}; |x|=2}\fpr({x})^3
<   4\frac{|H|^3}{((1/4)q^4)^2}. 
$$


If $M$ is $M_2$, then either $p\ne 3$ or there is no element of order $3$ in $H$ since all elements of odd prime order lie in the Singer cycle. If $p \ne 3$, then it is clear that there are two elements of order $3$ in $H,$ so, using \eqref{42s},% we obtain
 $$Q_{3}  \le \sum_{{x} \in \mathscr{P}; |x|=3}\fpr({x})^3  \le 2^3/((1/2)q^4)^2.$$

If $M$ is $M_i$ with $i=3,4,$ then $|x^G \cap H|\le c_3(M_i)/(q-1)=9$ and there are at most two conjugacy classes of elements of order $3$ in $H$, since a Sylow $3$-subgroup of $H$ has order $3.$  If $p=3$, then by \cite[Proposition 3.22]{fpr2} and Lemma \ref{xGoGs} we can take
\begin{equation}\label{ptri}
2B(s,p)=A(s,p):=\frac{1}{2}\left(\frac{q}{q+1} \right) \max(q^{s(n-s)},q^{ns/2}).
\end{equation} If $p\ne 3$, then we use \eqref{polup2s}. Therefore,
  $$Q_{3}  \le \sum_{{x} \in \mathscr{P}; r=3}\fpr({x})^3 \le 2 \cdot 9^3 \cdot 8/((1/2)(q/(q+1))q^4)^2.$$ 
Computations show that $Q(G,3)\le Q_1+Q_2+Q_3 <1$ for $q>9.$ If $q \le 9$ then  $b_S(S \cdot Sp_4(q)) \le 3$ is established by computation. 

If $k=4$, then  $S$ is a group of monomial matrices in some basis of $V.$ Thus, $D=S \cap D(GL_n(q))$ is normal in $S$, so $V$  is $\mathbb{F}_q[D]$-homogeneous. Hence $D \le Z(\hat{G}).$ Therefore, if $a,b \in S$ correspond to the same permutation (every monomial matrix is a product of a diagonal matrix and a permutation matrix which are unique), then $ab^{-1} \in Z(\hat{G}).$ So $H \cong \Sym(4)$ and there are $8$ elements of order $3$ and $9$ elements of order $2$ in $H.$ Hence $$|x^G \cap H| \le 
\begin{cases}
 9 \text{ if } r=2;\\
8 \text{ if } r=3.
\end{cases}
$$   
These bounds and \eqref{polup2s} show that
$$
Q_2 
<   4\frac{9^3}{((1/4)q^4)^2},
$$ since, by \cite[Table 3.8]{fpr2}, $s$ can be only  $2$ (recall that $q$ is odd) and $k_{2,2}=4.$ 

We use \eqref{42s} and \eqref{ptri} when $p\ne 3$ and $p=3$ respectively.  Therefore, if $p \ne 3$, then \eqref{40s}  shows that
\begin{equation*}
\begin{aligned}
Q_{3} \le \left(q^{1/2}\left(\frac{8^3}{((1/2) \cdot q^3)^2} \right) +q^{3/2}\left(\frac{8^3}{((1/2) \cdot q^4)^2} \right) + q^2\left(\frac{8^3}{((1/2) \cdot q^{6})^2} \right) \right).
\end{aligned}
\end{equation*}
If $p = 3$, then

$$\Scale[0.99]{Q_3 \le \left(\left(\frac{(8)^3 \cdot 8}{(1/2 (q/(q+1)) \cdot q^3)^2} \right) +3\left(\frac{(8)^3 \cdot 8}{(1/2 (q/(q+1)) \cdot q^4)^2} \right) + 3^{3/2}\left(\frac{(8)^3 \cdot 8}{(1/2 (q/(q+1)) \cdot q^{6})^2} \right) \right).} 
$$ 

Computations show that $Q(G,3)\le Q_1+ Q_2+Q_3 <1$ for $q>5.$ If $q \le 5$ then  $b_S(Sp_4(q)) \le 3$ is established by computation.

\bigskip

{\bf Subcase 2.} Now let $S$ be primitive, so $S$ lies in a  primitive maximal solvable  subgroup $M$ of $GL_n(q).$ Since $e=2,$ $M=M_6$ (see \cite[\S 8.1]{short} for the definition). By \cite[Proposition 8.2.1]{short}, 
$$M=M_2 \otimes M_i$$
where $M_i$ is defined in Remark \ref{remmi} for $i=2,3,4.$ Recall also the values of $c_r(M_i)$ for $r=2,3$ from Remark \ref{remmi}.  Now, since $M_2 \otimes I_2$ and $I_2 \otimes M_i$ contain $Z(GL_4(q)),$ we deduce $c_2(M)=(q+3) \cdot 10(q-1)$  and $c_3(M)=(3, q+1) \cdot 9(q-1).$ So there are $9(3, q+1)$ elements $g$ of $M/Z(GL_4(q))$ such that $g^3=1$ and, therefore,  $9(3, q+1)-1$ elements of order $3$. Similarly, there are $10(q+3)-1$ elements of order $2$ in $M/Z(GL_4(q)).$ Since $H$ is isomorphic to a subgroup of $M/Z(GL_4(q))$,
$$|x^G \cap H| \le
\begin{cases}
10(q+3)-1 &\text{ if } r=2;\\
 9(3,q+1)-1 &\text{ if } r=3.
\end{cases}
$$ 

These bounds and \eqref{polup2s} show that
$$
Q_2 
\le   4\frac{(10(q+3)-1)^3}{((1/4)q^4)^2},
$$ since, by \cite[Table 3.8]{fpr2}, $s$ can be only  $2$ (recall that $q$ is odd) and $k_{2,2}=4.$ 

We use \eqref{42s} and \eqref{ptri} when $p\ne 3$ and $p=3$ respectively. Therefore, if $p \ne 3$, then \eqref{40s}  shows that
\begin{equation*}
\begin{aligned}
\Scale[0.99]{
Q_{3} \le  \left(q^{1/2}\left(\frac{(9(3,q+1)-1)^3}{(1/4 \cdot q^3)^2} \right) +q^{3/2}\left(\frac{(9(3,q+1)-1)^3}{(1/4 \cdot q^4)^2} \right) + q^2\left(\frac{(9(3,q+1)-1)^3}{(1/4 \cdot q^{6})^2} \right) \right).}
\end{aligned}
\end{equation*}
If $p = 3$, then
$$\Scale[0.99]{Q_3 \le \left(\left(\frac{(8)^3 \cdot 8}{(1/2 (q/(q+1)) \cdot q^3)^2} \right) +3\left(\frac{(8)^3 \cdot 8}{(1/2 (q/(q+1)) \cdot q^4)^2} \right) + 3^{3/2}\left(\frac{(8)^3 \cdot 8}{(1/2 (q/(q+1)) \cdot q^{6})^2} \right) \right).} 
$$  

Computations show that $Q(G,3)\le Q_1+Q_2+Q_3 <1$ for $q>11.$ If $q \le 11$, then  $b_S(Sp_4(q)) \le 3$ is established by computation.

\medskip

\medskip

Notice that $b_S(GL_n(q))\le 3$ in the case {\bf L} for all $n \le 5$ and a maximal solvable subgroup $S$ of $GL_n(q)$ with $(n,q)$ neither $(2,2)$ nor $(2,3)$ by \cite[Table 2]{burness}.

\medskip

 We now summarise the results of this section.
\begin{Th}
\label{sec311lem}
Let ${G}$ be  $GL_n(q),$ $GU_n(q)$ or $GSp_n(q)$ in cases {\bf L}, {\bf U} and {\bf S} respectively. Let $S$ be a primitive maximal solvable subgroup of $G$ in case ${\bf L}$, and let $S$ be a quasi-primitive maximal solvable subgroup of $G$  in cases {\bf U} and {\bf S}. In each case let $(n,q)$ be such that $G$ is not solvable.
\begin{itemize}
\item In case {\bf L}, either $b_S(S \cdot SL_n(q))=2$, or $b_S(S \cdot SL_n(q))=3$ and one of the following holds:
\begin{enumerate}
\item[$(1)$] $n =2$,  $q>3$ is odd, and $S$ is the normaliser of a Singer cycle. If $q>5$, then there exists $x\in GL_n(q)$ such that  $S \cap S^x \le D(GL_n(q));$ 
\item[$(2)$] $n =2$,  $q\ge 4$ is even, and $S$ is the normaliser of a Singer cycle. In this case there exists $x\in GL_n(q)$ such that  $S \cap S^x \le RT(GL_n(q));$
\item[$(3)$] $n=2$, $q=7$, and $S$  is an absolutely irreducible subgroup such that $S/Z(GL_n(q))$ is isomorphic to $2^2.Sp_2(2);$
\item[$(4)$] $n=3$, $q=2$,  and $S$ is the normaliser of a Singer cycle. 
\end{enumerate}  
\item In case {\bf U}, $b_S(S \cdot SU_n(q))\le 3.$
\item In case  {\bf S}, $b_S(S \cdot Sp_n(q)) \le 3.$
\end{itemize}
\end{Th}


\section{Imprimitive irreducible subgroups}

 We commence by obtaining a result about the  groups of monomial matrices in $GL_n(q)$ and $GU_n(q).$ Recall that by default we assume that $GU_n(q)$ is $GU_n(q, I_n)$, the general unitary group with respect to an orthonormal basis of $V$. We combine this result with those of the previous section to obtain 
an upper bound to $b_S(S \cdot (SL_n(q^{\bf u}) \cap G))$ for those maximal solvable subgroups $S$ of $G \in \{GL_n(q), GU_n(q), GSp_n(q)\}$ which are neither primitive nor quasi-primitive.   

\medskip

For $a \in \mathbb{F}_{q^{\bf u}}^*$ let $A(n), B(n,a)$ and $C(n,a)$  be the following $n \times n$ matrices:
\begin{equation}\label{igrek}
\begin{aligned}
A(n) & = \begin{pmatrix}
1      & -1     & 1      & -1     & \ldots & (-1)^{n+1}         \\
0      &   1    & -1     & 1      & \ldots &  (-1)^{n}         \\
0      &    0   & 1      & -1     & \ldots &  (-1)^{n-1}        \\
       &        &        & \ddots &\ddots  &          \\
       &        &        &        & 1      & -1        \\
       &        &        &        & 0      &  1      
\end{pmatrix}, \text{ so } \\
A(n)^{-1} & =
\begin{pmatrix}
1      & 1      & 0      & 0      &        &          \\
0      &   1    & 1      & 0      &        &           \\
0      &    0   & 1      & 1      &        &          \\
       &        &        & \ddots &\ddots  &          \\
       &        &        &        & 1      & 1        \\
       &        &        &        & 0      &  1      
\end{pmatrix};
\end{aligned}
\end{equation}
 
\begin{gather*}
B(n,a) = \begin{pmatrix}
a         & a          & 0          & 0       &0     & \ldots & 0         \\
a^2       & -a^2       & a          & 0       &0     & \ldots & 0         \\
a^3       & -a^3       & -a^2       & a       &0     & \ldots &           \\
          &            &            &\ddots   &      &\ddots  &           \\
a^{(n-2)} & -a^{(n-2)} & -a^{(n-3)} & \ldots  & -a^2 & a      & 0         \\
a^{(n-1)} &-a^{(n-1)}  &-a^{(n-2)}  &\ldots   & -a^3 & -a^2   & a         \\
a^{(n-1)} &-a^{(n-1)}  &-a^{(n-2)}  &\ldots   & -a^3 & -a^2   & -a        
\end{pmatrix};
\end{gather*} 
\begin{gather*}
C(n,a) = \begin{pmatrix}
\sqrt{a^2+1}      & a      & 0          &   0   & \ldots & 0         \\
\sqrt{a^2+1}      & (a+a^{-1})      & a^{-1}          &   0   & \ldots &  0         \\
\sqrt{a^2+1}      & (a+a^{-1})      & (a+a^{-1})          &   a   & \ldots &  0        \\
                  &        & \vdots     &\vdots &\vdots  &           \\
\sqrt{a^2+1}      & (a+a^{-1})      & \ldots     &(a+a^{-1})      & (a+a^{-1})      &  a^{(-1)^{n}}     \\ 
\alpha     & \delta      & \ldots     &\delta & \delta      &  \delta 
\end{pmatrix}.
%C_1(n) = \begin{pmatrix}
%1      & 0      & 0          &   0   & \ldots & 0  &0        \\
%1      & 0      & 1          &   1   & \ldots &  1 &0        \\
%1      & 1      & 0          &   1   & \ldots &  1 &0       \\
%       &        & \ddots     &\ddots &\ddots  &           \\
%1      & 1      & \ldots     &1      & 0      &  1 &0    \\ 
%1      & 1      & \ldots     &1      & 1      &  0      &0\\
%0      & 0      & \ldots     &0      & 0      &  0      &1
%\end{pmatrix}.
\end{gather*} 
Here $n\ge 3$ for $B(n,a)$ and $C(n,a).$ % $n \ge 4$ is even for $C_0(n)$ and $n \ge 4$ is odd for $C_1(n).$
 Denote by $B'(n,a)$ the matrix $B(n,a)\pi$, where $\pi$ is the permutation matrix for the permutation $(1,n)(2,n-1) \ldots ([n/2], [n/2+3/2]).$ If $n$ is even, then $\alpha=a$ and $\delta=\sqrt{a^2+1}$ in $C(n,a).$ If $n$ is odd, then $\alpha=1$ and $\delta= a^{-1}\sqrt{a^2+1}.$



\begin{Lem} \label{omnom}
Let $M_n(q)$ be the group of all monomial matrices in $GL_n(q)$ and $MU_n(q):=M_n(q^2) \cap GU_n(q)$. %Here $GU_n(q)$ is the group of unitary matrices with respect to form with matrix $I_n.$ 
\begin{enumerate}[font=\normalfont]
\item $M_n(q) \cap M_n(q)^{A(n)}=Z(GL_n(q))$. \label{omnom1}
\item If $q$ is odd and $a \in \mathbb{F}_{q^2}$ satisfies $a^{q+1}=2^{-1},$ then 
$$MU_n(q) \cap MU_n(q)^{B(n,a)} \cap MU_n(q)^{B'(n,a)} \le Z(GU_n(q))$$ for $n \ge 3.$ \label{omnom2}
\item If $q$ is even and $1 \ne a \in \mathbb{F}_q^*$ (so $q>2$), then
$$MU_n(q) \cap MU_n(q)^{C(n,a)}  \le Z(GU_n(q))$$ for $n \ge 3.$ \label{omnom3}
 \item If $q\ge 4$, then $b_{MU_2(q)}(GU_2(q))\le 3.$ \label{omnom4}
 \item If $n=4$, then $b_{MU_n(2)}(GU_n(2))=4.$ If $n>4$, then $b_{MU_n(2)}(GU_n(2))\le 3.$ \label{omnom5}
\end{enumerate}
\end{Lem}
\begin{proof}
\eqref{omnom1} Consider $g \in M_n(q) \cap M_n(q)^{A(n)},$ so $g=(\diag(d_1, \ldots, d_n)s)^{A(n)}$, where $s \in \Sym(n)$ and $d_i \in \mathbb{F}_q^{*}.$  If $s$ does not fix the point $n$, so $(n)s=l$ for $l<n$, then the last row of the matrix $g$ is equal to the $l$-th row of the matrix $A(n)$ multiplied by $d_l$, which contains more than one non-zero entry. Therefore, $s$ stabilises $n$.  Assume that $s$ stabilises the last ${n-j}$ points and $(j)s=i \le j.$
The  $j$-th row of $g$ is 
\begin{equation*}
\begin{split}
\bigl( \overbrace{0, \ldots, 0}^{i-1},  \overbrace{d_i,-d_i,  \ldots,(-1)^{j-i}d_i}^{j+1-i} , \ldots \bigr).
\end{split}
\end{equation*}
Therefore, $s$ must stabilise $j$ and, by induction, $s$ is trivial.  It is easy to check that the $j$-th row of  $g$ is 
\begin{equation*}
\begin{split}
\bigl( \overbrace{0, \ldots, 0}^{j-1},  d_{j},   
 ( d_{j+1} -d_{j})   ,  \ldots, (-1)^{k-j+1}((d_{j+1} - d_{j})) \bigr)
\end{split}
\end{equation*}
for all $1\le j < n$. Therefore,  $d_{i+1}=d_{i}$ for all $1\le i<n$, since $g \in M_n(q).$ So $g \in Z(GL(n,q)).$

\eqref{omnom2} Since $q$ is odd, there always exists $a \in \mathbb{F}_{q^2}$ such that $a^{q+1}=2^{-1}$. Indeed, let $\eta$ be a generator of $\mathbb{F}_{q^2}^*$ and $\theta=\eta^{q+1},$ so $\theta$ is a generator of $\mathbb{F}_q^*.$ Thus, $\theta^k=2^{-1}$ for some integer  $k$. Therefore, $(\eta^k)^{q+1}=2^{-1}.$
It is routine to check that  $B(n,a)$ and $B'(n,a)$ lie in $GU_n(q)$ for such $a.$ 

%For convenience we record that
%\begin{gather}
%B(n,a)^{-1} = \begin{pmatrix}
%a^q & a^{2q}    & a^{3q}     & \ldots  &\ldots & a^{q(n-2)}  &  a^{q(n-2)}      \\
%a^q & -a^{2q}   & -a^{3q}    & \ldots  &\ldots &-a^{q(n-2)}  & -a^{q(n-2)}       \\
%0   & a^q       & -a^{2q}    & \ldots  &\ldots &-a^{q(n-3)}  & -a^{q(n-3)}       \\
%    &  0        &  a^q       & \ldots  &\ldots &-a^{q(n-4)}  & -a^{q(n-4)}       \\
%    &           &            & \ddots  &       & \ddots      &                   \\
%0   &0          & 0          &\ldots   & a^q   & -a^{2q}     & -a^{2q}           \\
%0   &0          & 0          &\ldots   & 0     &  a^q        & -a^q              
%\end{pmatrix};
%\end{gather} 

%%%%%%%%%%%

Consider $g \in MU_n(q) \cap MU_n(q)^{B(n,a)}$,  so 
\begin{equation*}
g=\diag(g_1, \ldots, g_n)r,
\end{equation*}
where $r \in \Sym(n)$ and $g_i \in \mathbb{F}_{q^2}^{*}$. %Notice that $g_i^{q+1}=d_i^{q+1}=1$ for $1 \le i \le n$ since $MU_n(q) \le GU_n(q)$.

Let $\beta=\{v_1, \ldots, v_n\}$ be the orthonormal basis of $V$ such that $GU_n(q)=GU_n(q, {\bf f}_{\beta})$. Since $g$ is monomial, it stabilises the decomposition
$$\langle v_1\rangle \oplus \ldots \oplus \langle v_n \rangle.$$ Since $g \in GU_n(q)^{B(n,a)},$ it stabilises the decomposition 
$$\langle (v_1)B(n,a)\rangle \oplus \ldots \oplus \langle (v_n) B(n,a) \rangle.$$ We write $w_i$ for $(v_i)B(n,a).$ Notice that 
\begin{equation*}
w_i=
\begin{cases}
av_1 +a v_2 & \text{ if } i=1; \\
a^{i}v_1  -a^{i}v_2  -a^{i-1}v_3  \ldots   -a^2v_i + av_{i+1} & \text{ if } 1<i<n; \\
a^{n-1}v_1  -a^{n-1}v_2  -a^{n-2}v_3  \ldots   -a^2v_{n-1} - av_{n} & \text{ if } i=n.\\          
\end{cases}
\end{equation*}
Since $g$ is monomial, $w_i$ and $(w_i)g$ have the same number of non-zero entries (which is $i+1$ for $i \ne n$ and $n$ for $i=n$) in the decomposition with respect to $\beta.$ Therefore, $(w_i)g \in \langle w_i \rangle$ for $i <n-1,$ so $r$ must fix $\{1,2\}$ and points $3, \ldots, n.$  
Thus, $(w_{n-1})g$ is either $ \delta w_{n-1}$ or $\delta w_{n}$ for some $\delta \in \mathbb{F}_{q^2}.$ If $r$ fixes the point 1, then  $\delta=g_1=g_2= \ldots =g_{n-1}= \pm g_n;$ if $(1)r=2$, then 
$\delta=-g_1=-g_2=g_3= \ldots =g_{n-1}= \pm g_n.$  It is easy to see that $MU_n(q) \cap MU_n(q)^{B(n,q)}$  lies in 
\begin{equation*}
\begin{split}
\{ \diag((-1)^i \alpha,(-1)^i \alpha,\alpha, \ldots,\alpha, \pm \alpha) \cdot (1,2)^i \mid \alpha \in \mathbb{F}_{q^2};  i \in \{0,1\}\}.
\end{split}
\end{equation*}
Therefore, $MU_n(q) \cap MU_n(q)^{B(n,q)\pi}  =  (MU_n(q) \cap MU_n(q)^{B(n,q)})^{\pi}$ lies in 
\begin{equation*}
\begin{aligned}
 & \,   \{ \diag((-1)^i \alpha,(-1)^i \alpha,\alpha, \ldots,\alpha, \pm \alpha) \cdot (1,2)^i\mid \alpha \in \mathbb{F}_{q^2};  i \in \{0,1\}\}^{\pi} \\
 \subseteq & \, \{ \diag(\pm \alpha,\alpha, \ldots,\alpha,  (-1)^i \alpha,(-1)^i \alpha) \cdot (n,n-1)^i\mid \alpha \in \mathbb{F}_{q^2};  i \in \{0,1\}\}
\end{aligned}
\end{equation*} 
and $$MU_n(q) \cap MU_n(q)^{B(n,q)} \cap MU_n(q)^{B(n,q)\pi}\le Z(GU_n(q)).$$


 

\eqref{omnom3} Let $1 \ne a \in \mathbb{F}_q$. Since $\phi : \mathbb{F}_q \to \mathbb{F}_q$ mapping $x$ to $x^2$ is a Frobenius automorphism of $\mathbb{F}_q$, every element of $\mathbb{F}_q$ has a unique square root in $\mathbb{F}_q$. Therefore, the matrix $C(n,a)$ exists and lies in $GU_n(q).$ 

Suppose that $n\ge 3$ is odd and consider $g \in MU_n(q) \cap MU_n(q)^{C(n,a)}$, so 
\begin{equation*}
g=\diag(g_1, \ldots, g_n)r.
\end{equation*}
Let $\beta=\{v_1, \ldots, v_n\}$ be the orthonormal basis as in \eqref{omnom2}. Since $g$ is monomial, it stabilises the decomposition
$$\langle v_1\rangle \oplus \ldots \oplus \langle v_n \rangle.$$ Since $g \in MU_n(q)^{C(n,a)},$ it stabilises the decomposition 
$$\langle (v_1)C(n,a)\rangle \oplus \ldots \oplus \langle (v_n) C(n,a) \rangle.$$ We write $w_i$ for $(v_i)C(n,a).$ Notice that 
\begin{equation*}
\begin{aligned}
w_1 & =\sqrt{a^2+1} v_1 +a v_2;\\
w_n & =\alpha v_1 + \delta v_2+    \ldots  + \delta v_{n-1} + \delta v_{n},
\end{aligned}
\end{equation*}
and if $1<i<n$, then 
\begin{equation*}
w_i=
\begin{cases}
\sqrt{a^2+1} v_1 + (a+a^{-1})v_2 +   \ldots+   (a+a^{-1})v_i + av_{i+1} & \text{ if $i$ is odd;}  \\ 
\sqrt{a^2+1} v_1 + (a+a^{-1})v_2+    \ldots +  (a+a^{-1})v_i + -av_{i+1} & \text{ if $i$ is even.} \\
       
\end{cases}
\end{equation*}
Since $g$ is monomial, $w_i$ and $(w_i)g$ have the same number of non-zero entries (which is $i+1$ for $i \ne n$ and $n$ for $i=n$) in the decomposition with respect to $\beta.$ Therefore, $(w_i)g \in \langle w_i \rangle$ for $i <n-1,$ so $r$ must fix $\{1,2\}$ and points $3, \ldots, n.$
Assume that $(1)r=2,$ so $$(w_1)g= g_1 \sqrt{a^2+1}v_2 + g_2av_1.$$ Since $(w_1)g \in \langle w_1 \rangle,$  
$$g_1 \sqrt{a^2+1}v_2 + g_2av_1= \gamma (\sqrt{a^2+1} v_1 +a v_2)$$
for some $\gamma \in \mathbb{F}_{q^2}.$ Calculations show that $g_2(g_1)^{-1}=1+a^{-2.}$ Notice that $g_1^{q+1}=g_2^{q+1}=1$, since $g \in MU_n(q).$ Hence $(g_2(g_1)^{-1})^{q+1}$ must be 1. However,
$$(1+a^{-2})^{q+1}=(1+a^{-2})^2=1+a^{-4} \ne 1.$$
So $r$ must fix the points 1 and 2. 

  
Since $(w_{n-2})g \in \langle w_{n-2} \rangle$, we obtain $g_1= \ldots = g_{n-1}.$ Assume that $(w_{n-1})g= \gamma w_n$ for some $\gamma \in \mathbb{F}_{q^2}.$ Then $g_1= \gamma \sqrt{a^2+1} $ and $g_n= \gamma(\sqrt{a^2+1})^{-1}.$ Since $(g_i)^{q+1}=1$ for all $i=1, \ldots, n,$ 
$$\gamma^{q+1}(a^2+1)=\gamma^{q+1}(a^2+1)^{-1}=1.$$
Therefore, $a^2+1$ must be equal to $(a^2+1)^{-1},$ which is not true since $$(a^2+1)^2=a^4+1 \ne 1.$$
Thus, $(w_{n-1})g= \gamma w_{n-1}$ and $g$ is a scalar.

The proof of \eqref{omnom3} for even $n$ is  analogous to that for  odd $n$. 

\medskip

\eqref{omnom4} For $q=4,5$ the statement is verified by computation. For $q>5$ the statement follows from \cite[Table 2]{burness}.

\medskip

% For $n \le 15$ the statement is verified by computation. For larger $n$ we prove the statement by checking \eqref{0} for the elements of prime order of $H \le PGU_n(2)$, where $H$ is the image of $MU_n(2)$ in $G=PGU_n(2).$

% Sufficient bounds for $|x^G|$ and $|x^G \cap H|$ can be found in the proof of Propositions 2.5 and 2.6 in \cite{fpr3}. %We state below the cases of structure of $x$ and more precise location of the bounds in \cite{fpr3} for this cases.

%If $|x|=2,$ then a preimage $\hat{x}$ of $x$ in $GU_n(2)$ cannot be diagonal. {\bf Case 2.1} of the proof of \cite[Proposition 2.6]{fpr3} states the bounds, sufficient for $n \ge 9.$

%Let $|x|>3,$ so $\hat x$ is not diagonal, since all diagonal matrices in $GU_n(2)$ have order $3$. Since $(|x|,q+1)=1$, we obtain $|H^1(\sigma, E/E^0)|=1$ (see Definition \ref{H1EE0}) by \cite[Lemma 3.35]{fpr2}. The bounds stated in  {\bf Case 2.1} of the proof of \cite[Proposition 2.5]{fpr3} suffice for $n \ge 6.$ 

%Let $B \le G$ be the image of $D(GU_n(q))$ in $G$. Assume that $(x^G \cap H) \subseteq B$, so $|x|=3$. If $|H^1(\sigma, E/E^0)|=1$, then the bounds stated in {\bf Case 1.1.2} of the proof of \cite[Proposition 2.5]{fpr3} suffice for $n \ge 10.$ By the first paragraph of {\bf Case 1.2} of the proof of \cite[Proposition 2.5]{fpr3}, $|H^1(\sigma, E/E^0)|$ cannot be greater than $1$ for $x \in B$ of order $3$.

%Assume that $|x|=3$ and  $(x^G \cap H)$ does not lie in $B$. Here $|H^1(\sigma, E/E^0)| \in \{1,3\}$ by \cite[Lemma 3.35]{fpr2}. If $|H^1(\sigma, E/E^0)|=1$, then the bounds  stated for $(\varepsilon,q)=(-,2)$ in  {\bf Case 2.2} of the proof of \cite[Proposition 2.5]{fpr3} suffice for $n \ge 9.$ If $|H^1(\sigma, E/E^0)|=3$, then the bounds stated in {\bf Case 2.3} of the proof of \cite[Proposition 2.5]{fpr3} are  sufficient  for $n > 15.$  

\eqref{omnom5}  For $n < 7 $  the statement is verified by computation. Assume $n\ge 7.$ Let $a$ be a generator of $\mathbb{F}_4^*,$ so $a^2=a+1$ and $a^3=1.$ Let $\beta=\{v_1, \ldots, v_n\}$ be the orthonormal basis of $V$ such that $GU_n(q)=GU_n(q, {\bf f}_{\beta})$. Let $E(n,a) \in GU_n(2)$ be defined as follows. If $n$ is even, then
\begin{equation*}
(v_i)E(n,a)=w_i=
\begin{cases}
v_1 & \text{ if } i=1; \\
\sum_{j=2}^n v_j & \text{ if } i=2; \\
(a+1)v_i +a v_{i+1} + \sum_{j=i+2}^n v_j & \text{ if } i \text{ is odd and }3\le i\le n; \\
 a v_i +(a+1) v_{i+1} + \sum_{j=i+2}^n v_j & \text{ if } i \text{ is even and }3\le i\le n.
\end{cases}
\end{equation*}
If $n$ is odd, then 
\begin{equation*}
(v_i)E(n,a)=w_i=
\begin{cases}
\sum_{j=1}^n v_j & \text{ if } i=1; \\
(a+1)v_i +a v_{i+1} + \sum_{j=i+2}^n v_j & \text{ if } i \text{ is even and } 2\le i\le n; \\
 a v_i +(a+1) v_{i+1} + \sum_{j=i+2}^n v_j & \text{ if } i \text{ is odd and } 2\le i\le n.
\end{cases}
\end{equation*}
For example,  $E(8,a)$ is
$$
\begin{pmatrix}
1 & 0 & 0 & 0 & 0 & 0 & 0 & 0\\
0 & 1 & 1 & 1 & 1 & 1 & 1 & 1\\
0 & a+1 & a & 1 & 1 & 1 & 1 & 1\\
0 & a & a+1 & 1 & 1 & 1 & 1 & 1\\
0 & 0 & 0 & a+1 & a & 1 & 1 & 1\\
0 & 0 & 0 & a & a+1 & 1 & 1 & 1\\
0 & 0 & 0 & 0 & 0 & a+1 & a & 1\\
0 & 0 & 0 & 0 & 0 & a & a+1 & 1\\
\end{pmatrix}.$$ We obtain $E(7,a)$ by deleting the first row and the first column in $E(8,a).$ It is routine to verify that $E(n,a) \in GU_n(2).$

 Let $\pi \in GU_n(2)$ be the permutation matrix corresponding to the permutation
\begin{align*}
(1,n)(3,n-1)(5,n-2)(6, 7, \ldots, n-3) & \text{ if } n \text{ is even;}\\
(1,n)(3,n-1)(5,6, 7, \ldots, n-3) & \text{ if } n \text{ is odd.}
\end{align*}
We claim that $MU_n(2) \cap MU_n(2)^{E(n,a)} \cap MU_n(2)^{E(n,a)^{\pi}}\le Z(GU_n(2)).$ We prove this for even $n$; the proof is analogous for odd $n$. 


\medskip

Consider $g \in MU_n(2) \cap MU_n(2)^{E(n,a)}$,  so 
\begin{equation*}
g=\diag(g_1, \ldots, g_n)r,
\end{equation*}
where $r \in \Sym(n)$ and $g_i \in \mathbb{F}_{4}^{*}$.  Since $g$ is monomial, it stabilises the decomposition
$$\langle v_1\rangle \oplus \ldots \oplus \langle v_n \rangle.$$ Since $g \in MU_n(2)^{E(n,a)},$ it stabilises the decomposition 
$$\langle w_1\rangle \oplus \ldots \oplus \langle w_n \rangle.$$ 
Since $g$ is monomial, $w_i$ and $(w_i)g$ have the same number of non-zero entries in the decomposition with respect to $\beta.$ Therefore, $(w_i)g \in \langle w_{n-2}, w_{n-1}, w_{n} \rangle$ for $n-2 \le i \le n,$ so $r$ must fix $\{n-2,n-1,n\}.$ If $i \in \{n-4,n-3\},$ then $(w_i)g \in \langle w_{n-4}, w_{n-3} \rangle$, so $r$ must fix $\{n-4,\ldots,n\}$ and, therefore, $\{n-4, n-3\}$. Continuing this process, we obtain that $r$ fixes 
\begin{equation}
\label{monq2new}
\{1\},\{2,3\},\{4,5\}, \ldots, \{n-4, n-3\}, \{n-2, n-1, n\}.
\end{equation}
Now assume $g \in MU_n(2) \cap MU_n(2)^{E(n,a)^{\pi}}$. The above arguments show that $r$ must fix
$$\{(1)\pi\},\{(2)\pi,(3)\pi\},\{(4)\pi,(5)\pi\}, \ldots, \{(n-4)\pi, (n-3)\pi\}, \{(n-2)\pi, (n-1)\pi, (n)\pi\}$$ which are
$$\{n\},\{2,n-1\},\{4,n-2\}, \{7,8\}, \ldots, \{n-3, 6\}, \{1, 3, 4\}.$$ Combining this with \eqref{monq2new} we obtain that $r$ is a trivial permutation, so $g$ is diagonal.

Observe that $(w_2)g = (0, g_2, g_3, \ldots, g_n)$ with respect to $\beta$. Since  $ (w_2)g \in \langle w_2, w_3, w_4 \rangle$, 
$$g_4 = g_5 = \ldots =g_n.$$ So $g=\diag(g_1, g_2, g_3, \lambda, \ldots, \lambda)$ for some $\lambda \in \mathbb{F}_4^*.$  Let $u_i = (v_i)E(n,a)^{\pi}.$ Therefore, 
\begin{align*}
u_2 & =(1, \ldots, 1, 0)\\
u_{n-1} & = (1, a+1, 1, \ldots, 1, a, 0)\\
u_{4} & = (1, a, 1, \ldots, 1, a+1, 0)
\end{align*}
are the only vectors in $\{u_1, \ldots, u_n\}$ that have $n-1$ non-zero entries in the decomposition with respect to $\beta.$ Hence  $(u_2)g$ lies in $\langle u_2\rangle$, $\langle u_{n-1}\rangle$, or $\langle u_4 \rangle$ since $g \in MU_n(2)^{E(n,a)^{\pi}}$ and stabilises the decomposition
$$\langle u_1\rangle \oplus \ldots \oplus \langle u_n \rangle.$$ Notice that  $(u_2)g= (g_1, g_2, g_3, \lambda, \ldots, \lambda,0)$  Hence $(u_2)g \in \langle u_2 \rangle$ and $g_1=g_2=g_3=\lambda.$ So $g = \lambda I_n \in  Z(GU_n(q))$. Therefore, 
\begin{equation*}
MU_n(2) \cap MU_n(2)^{E(n,a)} \cap MU_n(2)^{E(n,a)^{\pi}}\le Z(GU_n(2)). \qedhere
\end{equation*}
\end{proof}

\begin{Rem} \label{omnomSL}
In Lemma \ref{omnom} each statement of \eqref{omnom1}-\eqref{omnom5} can be written as $$S \cap S^{x_1} \cap \ldots \cap S^{x_t}\le Z(\widehat{G})$$ for $x_i \in \widehat{G}$ with suitable $S$ and $\widehat{G}=GL_n(q)$ for \eqref{omnom1} and $\widehat{G}=GU_n(q)$ for \eqref{omnom2}--\eqref{omnom5}. In each case we can assume $x_i \in SL_n(q^{\bf u}) \cap \widehat{G}.$ Indeed, if $\det(x_i)\ne 1$, then $a_i=\diag(\det(x_i)^{-1},1\ldots, 1) \in S$ since $S$ is the group of all monomial matrices in $\hat{G},$ so $S^{a_ix_i}=S^{x_i}.$
\end{Rem}

\begin{Lem}\label{igrekl}
Let $H \le X \wr Y,$ where $X \le GL_m(q),$ $Y \le \Sym(k).$  
Let $A(k)=(y_{ij})$ be as in \eqref{igrek} and let $x_i$ for $i =1, \ldots, k$ be arbitrary elements of $X$.
Define $x \in GL_{mk}(q)$ to be
\begin{gather*}
\begin{pmatrix}
y_{11}x_1      & y_{12}x_1      & \ldots     & y_{1k}x_1            \\
y_{21}x_2      &  y_{22}x_2     & \ldots     & y_{2k}x_2             \\
\vdots      &             &            & \vdots               \\
y_{k 1}x_{k}   & y_{k 2}x_{k}   &  \ldots    & y_{k k}x_{k}               
\end{pmatrix},
\end{gather*} 
 Let $h=\diag[D_1, \ldots, D_k] \cdot s  \in H$, where $D_i \in X$ and $s \in Y,$ so $h$ is obtained from the permutation matrix $s$ by replacing $1$ in the $j$-th line by the $(m \times m)$ matrix $D_j$ for $j =1, \ldots, k$ and replacing each zero by an $(m\times m)$ zero matrix. 
If $h^x \in H$, then $s$ is trivial and $D_j^{x_j}=D_{j+1}^{x_{j+1}}$ for $j=1, \ldots, k-1.$ 
\end{Lem}
\begin{proof}
 %We say that rows with numbers $(j-1)m+1,  \ldots, jm$ of a matrix form its $j$th $m \times m$-row.
   If $s$ does not stabilise the point $k$, then there is more than one non-zero $(m \times m)$-block in the last $(m \times m)$-row of $h^{x}$ and, thus, $h^{x}$ does not lie in   $ X \wr Y.$
Assume that $s$ stabilises the last ${k-j}$ points and $(j)s=i \le j.$ The  $j$-th $(m \times m)$-row of $h^{x}$ is 
\begin{equation*}	
\begin{split}
\bigl( \overbrace{0, \ldots, 0}^{i-1},  \overbrace{{x_j}^{-1}(D_{j}){x_i},-{x_j}^{-1}(D_{j}){x_i},  \ldots,(-1)^{j-i} {x_j}^{-1}(D_{j}){x_i}}^{j+1-i}, \ldots \bigr).
\end{split}
\end{equation*}
Therefore, $h^x$ does not lie in $X \wr Y$ if $i \ne j,$ since the $j$-th  $(m \times m)$-row contains more than one non-zero $(m \times m)$-block in that case. So $i=j$ and the  $j$-th $(m \times m)$-row of $h^{x}$ is
\begin{equation*}
\begin{split}
\bigl( \overbrace{0, \ldots, 0}^{i-1},  (D_{j})^{x_j},   
( ( D_{j+1})^{x_{j+1}} -(D_{j})^{x_j}  ) ,  \ldots, (-1)^{k-j+1}((D_{j+1})^{x_{j+1}} - (D_{j})^{x_j}  ) \bigr).
\end{split}
\end{equation*}
%with $(i-1)$ zeros in the beginning.
 So, if $h^x \in H$, then $s$ stabilises $j$ and $D_{j}^{{x_j}}=D_{j+1}^{x_{j+1}}.$
\end{proof}

%If $x \in GL(n_1, q)$, $y \in GL(n_2,q)$ then 
%\begin{gather}
%x \otimes y:= 
%\begin{pmatrix}
%y_{11}x      & y_{12}x      & \ldots     & y_{1n_2}x            \\
%y_{21}x      &  y_{22}x     & \ldots     & y_{2n_2}x             \\
%\vdots      &             &            & \vdots               \\
%y_{n_2 1}x   & y_{n_2 2}x   &  \ldots    & y_{n_2 n_2}x               
%\end{pmatrix}.
%\end{gather} 
 

\begin{Lem}\label{prtoirr}
Let $T(GL_n(q))$ be one of the following subgroups: $Z(GL_n(q))$, $D(GL_n(q))$ or $RT(GL_n(q))$. Let $H$ be a subgroup of $GL_n(q)$ such that
$$H \le H_1 \wr \Sym(k), $$
where $n=mk$, $H_1 \le GL_m(q)$. If there exist $g_1, \ldots, g_b \in GL_m(q)$ ({respectively } $SL_m(q))$ such that 
$$H_1 \cap H_1^{g_1} \cap \ldots \cap H_1^{g_b} \le T(GL_m(q)),$$
then there exist   $x_1, \ldots, x_b \in GL_n(q)$ ({respectively} $SL_n(q))$ such that 
\begin{equation}\label{irreq}
H \cap H^{x_1} \cap \ldots \cap H^{x_b} \le T(GL_n(q)).
\end{equation}
\end{Lem}
\begin{proof}
Define $x_i$ to be 
\begin{gather*}
\begin{pmatrix}
y_{11}g_{i}      & y_{1 2}g_{i }      & \ldots     & y_{1 k}g_{i }            \\
y_{21}g_{i}      &  y_{22}g_{i}     & \ldots     & y_{2k}g_{i}             \\
\vdots      &             &            & \vdots               \\
y_{k 1}g_{i }   & y_{k 2}g_{i}   &  \ldots    & y_{kk}g_{i}               
\end{pmatrix},
\end{gather*} 
 where $y=(y_{ij})=A(n).$ Let us show that \eqref{irreq} holds for such $x_i.$  Let  $h=\diag[D_1, \ldots, D_k] \cdot s  \in H$, where $D_i \in H_1$ and $s \in \Sym(k).$
If $$h \in H \cap H^{x_1} \cap \ldots \cap H^{x_b}$$  then, by  Lemma \ref{igrekl}, $s$ is trivial and $D_i=D_j$ for all $1 \le i, j \le {k}$. Thus, $$D_i \in H_1 \cap H_1^{g_1} \cap \ldots \cap H_1^{g_b} \le T(GL_m(q))$$
and $h \in  T(GL_n(q)).$ 

Notice that $\det(y)=1$ and if $\det(g_i)=1,$ then 
\begin{equation*}
\det(x_i)=\det(g_i \otimes y)=\det(g_i)^k \cdot \det({y})^m=1. \qedhere
\end{equation*} 
\end{proof}

\begin{Cor}
Let $S$ be a maximal solvable subgroup of $GL_n(q)$ and assume that matrices in $S$ have shape \eqref{stup}; so $S_i \gamma_i(S)= P_{i} \wr \Gamma_i$,
where $P_{i}$ is a primitive solvable subgroup of $GL_{m_i}(q)$, $\Gamma_i$ is a transitive solvable
subgroup of the symmetric group $\Sym(k_i)$, and $k_im_i = n_i$. If, for every $P_{i}$, there exist $x_i \in SL_{m_i}(q)$ such that
\begin{equation*}%\label{cor29}
P_{i} \cap P_{i}^{x_i}\le RT(GL_{m_i}(q)),
\end{equation*}
then $b_S(S \cdot SL_n(q))\le5.$
\end{Cor}
\begin{proof}
The statement follows  from Lemmas \ref{irrtog},  \ref{prtoirr} and  \ref{diag}.
\end{proof}

%\begin{Rem}
%Since $RT(GL(m_i,q))$ is not normal in $GL(m_i,q)$ when $m_i \ge 2$, it may be  that $P_{i} \cap P_{i}^{x_i}$ does not lie in $RT(GL(m_i,q))$ but is conjugate to a subgroup of $RT(GL(m_i,q))$. Since the conclusion of the corollary holds for every conjugate of such subgroup, we may assume  % So $S$ is conjugate to the subgroup which satisfies the condition in the corollary and the conclusion is true for $S$. Therefore, we  assume in such cases 
%that $S$ satisfies \eqref{cor29}.
%\end{Rem}

\begin{Th}\label{irred}
Let $S$ be an irreducible maximal solvable  subgroup of $GL_n(q)$ with $n \ge 2$, and $(n,q)$ is neither  $(2,2)$ nor $(2,3)$. For every such $S$, $$b_S(S \cdot SL_n(q)) \le 3.$$ Moreover, $$b_S(S \cdot SL_n(q))=2$$ for all such $S$ except the following cases: 
\begin{enumerate}[font=\normalfont]
\item $n =2$,  $q>3$ is odd, and $S$ is the normaliser of a Singer cycle. If $q>5$, then there exists $x\in GL_n(q)$ such that  $S \cap S^x \le D(GL_n(q)).$ \label{irred11}
\item $n =2$,  $q\ge 4$ is even, and $S$ is the normaliser of a Singer cycle. In this case there exists $x\in GL_n(q)$ such that  $S \cap S^x \le RT(GL_n(q)).$ \label{irred12}
\item $n=2$, $q=7$, and $S$  is an absolutely irreducible subgroup such that $S/Z(GL_n(q))$ is isomorphic to $2^2.Sp_2(2).$ \label{irred13}
\item $n=3$, $q=2$,  and $S$ is the normaliser of a Singer cycle. In this case $b_S(GL_n(q))=3.$ \label{irred14} 
\item $n=4$, $q=3$, and $S=GL_2(3) \wr \Sym(2)$. In this case there exists $x \in SL_4(3)$ such that $S \cap S^x \le RT(GL_4(3)).$ \label{irred15} %(v) $n=4$, $q \ge 9$ is odd, and $S$ is conjugate to a subgroup of $GL(2,q^2).2.$ In this case there exists $x\in GL(n,q)$ such that  $S \cap S^x \le D(GL(n,q)).$\\ %(vi)  $n=4$, $q\ge 5$ is odd, and $S=S_1 \wr Sym(2)$, where $S_1$ is the normalizer of a Singer cycle of $GL(2,q)$. In this case there exists $x\in GL(n,q)$ such that  $S \cap S^x \le D(GL(n,q)).$\\
\end{enumerate} 
\end{Th}
\begin{proof}
Let $S$ be an  irreducible maximal solvable subgroup of $GL_n(q).$ The statement  follows by Lemmas \ref{supirr} and \ref{prtoirr}, Theorem \ref{sch} and Section  \ref{sec5}
%there exists $x$ such that 
%\begin{equation}\label{ut}
%S \cap S^x \le RT(GL(n,q))
%\end{equation}
 for all cases except groups  conjugate to $S_1 \wr \Gamma$, where 
$n=kl$, $\Gamma$ is a transitive maximal  solvable subgroup of $\Sym(l)$, and $S_1$ is  one of the following groups:
\begin{enumerate}[label=\alph*)]
\item $k=3, q=2$ and $S_1$ is the normaliser of a Singer cycle of $GL_3(2);$ \label{irred1}
\item $k=2$, $q=2,3$ and $S_1=GL_2(q);$ 
\item $k=2$, $q=3$ and $S_1=GL_2(3)$; 
\item $k=2$, $q=7$ and $S_1/Z(GL_2(7))$ is isomorphic to $2^2.Sp_2(2);$ \label{irred4}
\item $k=2$, $q>3$ is odd and $S_1$ is the normaliser of a Singer cycle of $GL_2(q);$ \label{irred6}
\item $k=2$, $q$ is even and $S_1$ is the normaliser of a Singer cycle of $GL_2(q).$ \label{irred5}

\end{enumerate}
If $l \ge 3$, then $b_S(S \cdot SL_n(q))=2$ by \cite[Theorem 3.1]{james}. % Also by the remark after \cite[Theorem 1.4]{james}, if $l=2$ and $q$ is even then $b_S(GL_n(q))=2$ for all such $S,$ in particular for case \ref{irred5}.

We verify by computation that $b_S(S \cdot SL_n(q))=2$ for $l=2$ for all cases \ref{irred1} -- \ref{irred4} except  $S_1=GL_2(3)$. If $S_1=GL_2(3)$, then there exists $x \in SL_4(3)$ such that $S \cap S^x \le RT(GL_4(3))$. 
%Therefore, the only cases when there is no  $x$ such that \eqref{ut} holds are cases 1) -- 4) with $l=1$, so $S=S_1$ is primitive. 
%\medskip

Consider  case \ref{irred6}: so we assume $S=S_1 \wr \Sym(2)\ \le GL_4(q)$, where $S_1$ is the normaliser of a Singer cycle $S_a$ in $GL_2(q)$ and $q>3$ is odd, as in \eqref{thesin}. Hence $S \cdot SL_n(q)=GL_n(q)$ by Lemma \ref{supSL} since the determinant of a generator of a Singer cycle generates $\mathbb{F}_q^*$. % Let $G= PGL(4,q)$ and $H$ is the image of $S$ in $H$ under the natural homomorphism. In
%this case we claim that $Q(G, 2) < 1$ in \eqref{ver}. Let $x \in H$ has a prime order $r.$ If $r \ne 2$, so $r$ does not divide $n$, then by Lemma \ref{prost} $x$ has a preimage $\hat{x}$ of order $r$. Since $r \ne 2$, $\hat{x}$ lies in $T_1=T \times T,$ which is a maximal torus of $GL(4,q)$.  so $|x^G \cap H| \le 8$ by Lemma \ref{sin}. bounds for $|x^G|$ and $k_{s,r,s}$ are given by \eqref{40} and \eqref{42}. 

%If $r=2$ then we use bounds $|x^G \cap H|<|H|=4(q+1)$

Let $$s=
\begin{pmatrix}
0 & 0& 1 & 0\\
0 & 0& 0 & 1\\
1 & 0& 0 & 0\\
0 & 1& 0 & 0
\end{pmatrix},
$$  
so  $g \in S$ has shape 
$$
\begin{pmatrix}
\alpha_1' & \beta_1& 0 & 0\\
a \beta_1' &  \alpha_1 & 0 & 0 \\
0 & 0& \alpha_2' & \beta_2\\
0 & 0& a \beta_2' & \alpha_2
\end{pmatrix} \cdot s^i,
$$
where $i =0,1,$ $\alpha_j' = \pm \alpha_j$, $\beta_j' = \pm \beta_j$ and $\alpha_j'\beta_j' = \alpha_j \beta_j.$ Consider $g^y$ where $y=A(n)$ is as in \eqref{igrek}. First let $i=0,$ so 
$$g^y = \Scale[0.95]{
\begin{pmatrix}
\alpha_1' +a \beta_1' &\alpha_1+ \beta_1 - \alpha_1' -a\beta_1' & \alpha_1' +a\beta_1'- \alpha_1 - \beta_1 & \alpha_1 + \beta_1 -\alpha_1' -a\beta_1'\\
a \beta_1' &  \alpha_1 -a\beta_1' & a\beta_1' - \alpha_1 + \alpha_2' & \alpha_1 + \beta_2 -a\beta_1' - \alpha_2' \\
0 & 0& \alpha_2' +a\beta_2' & \alpha_2+\beta_2-\alpha_2' -a\beta_2'\\
0 & 0& a \beta_2' & \alpha_2 - a\beta_2'
\end{pmatrix}}.
$$ 
Assume that $g^y \in S,$ so ${g^y}_{1,3}={g^y}_{1,4}={g^y}_{2,3}={g^y}_{2,4}=0.$ Thus, $0={g^y}_{2,3}+{g^y}_{2,4}=\beta_2.$ Also, ${g^y}_{1,2}= - {g^y}_{1,3} = 0$, but, since the left upper $(2 \times 2)$ block must lie in $N_{GL_2(q)}(S_a)$, 
$$({g^y}_{1,2})a = \pm {g^y}_{2,1},$$
so $0 ={g^y}_{2,1} = a\beta_1'$ and $\beta_1=0.$ Therefore, ${g^y}_{1,2}= \alpha_1 -\alpha_1',$ so
$$\alpha_1 =\alpha_1'.$$ Now, since $\beta_2=0,$ ${g^y}_{3,4} = \pm ({g^y}_{4,3}) a^{-1}=0,$  
$$\alpha_2=\alpha_2'$$
and since ${g^y}_{2,4}=0$ we obtain $\alpha_1=\alpha_2,$ so $g^y$ is a scalar.  

Now let $i=1$, so 
$$g^y = \Scale[0.95]{
\begin{pmatrix}
0 & 0 & \alpha_1' + a\beta_1'& \alpha_1 +\beta_1 - \alpha_1' - a\beta_1'\\
\alpha_2' &  \beta_2'-\alpha_2' & a\beta_1' - \beta_2' + \alpha_2' & \alpha_1 - a\beta_1' +a\beta_2 - \alpha_2' \\
a \beta_2' + \alpha_2' & \alpha_2 + \beta_2 -a\beta_2' -\alpha_2'& a\beta_2' + \alpha_2 - \alpha_2 -\beta_2 & \alpha_2+ \beta_2 - a\beta_2' \alpha_2'\\
a\beta_2' & \alpha_2 -a \beta_2'& a\beta_2' - \alpha_2 & \alpha_2 - a\beta_2'
\end{pmatrix}}.
$$ 
If $g^y \in  S$, then ${g^y}_{2,1}={g^y}_{2,2}=0.$ So $\alpha_2 = \beta_2= 0$ which  contradicts the invertibility of $g.$
Therefore, $S \cap S^y  \le Z(GL_4(q)).$ 

Consider \ref{irred5},  so we  assume $S=S_1 \wr \Sym(2)\ \le GL_4(q)$, where $S_1$ is the normaliser of a Singer cycle $S_a$ in $GL_2(q)$ as in \eqref{thesineven}. If $q>2$, so we can choose $a\ne 1$, then arguments similar to those in case \ref{irred6} show that $$S \cap S^y \le Z(GL_n(q)).$$ For $q=2$ the statement $b_S(GL_4(q))=2$ is verified  by computation.
\end{proof}




%In the next theorem fix $GU_n(q)$ to be the group $\eqref{unimatr}$ with $\beta$ orthonormal, so ${\bf f}_{\beta}=I_n.$ 
\begin{Th} \label{uniind}
Let $Z(GU_m(q)) \le H \le GU_m(q)$. Assume that there exist  $a,b \in GU_m(q)$ such that 
$$H \cap H^a \cap H^b \le Z (GU_m(q)).$$
Let $M=(\mathbb{F}_{q^2}^*)^{q-1} \wr \Gamma$ for $\Gamma \le \Sym(k)$, so $M$ is a subgroup of monomial matrices in $GU_k(q).$ Assume that there exist  $x,y \in GU_k(q)$ such that 
$$M \cap M^x \cap M^y \le Z (GU_k(q)).$$
Denote 
\begin{align*}
&X=I_m \otimes x & & A= a \otimes I_k\\
&Y=I_m \otimes y & & B= b \otimes I_k.
\end{align*} 
If $n=mk$ and $S = H \wr \Gamma \le GU_n(q)$, then 
$$S \cap S^{AX} \cap S^{BY} \le Z(GU_{n}(q)).$$
\end{Th}
\begin{proof}
Consider $h \in S\cap S^{AX},$ so $h=g^{AX}$ where $g  \in S.$ Hence 
$$g^A = \diag[g_1, \ldots , g_k] \cdot \pi,$$ where $g_i \in H^a$ and $\pi=I_m \otimes \pi_1 $ for some $\pi_1 \in \Sym(k).$ If 
\begin{equation*}
x=
\begin{pmatrix}
x_{11} & \ldots & x_{1k}\\
\vdots & & \vdots \\
x_{k1} & \ldots & x_{kk}
\end{pmatrix}, \text{ then }
x^{-1}=
\begin{pmatrix}
x_{11}^q & \ldots & x_{k1}^q\\
\vdots & & \vdots \\
x_{1k}^q & \ldots & x_{kk}^q
\end{pmatrix},
\end{equation*}
since $x \in GU_k(q)$, and
\begin{equation*}
X=
\begin{pmatrix}
x_{11} I_m & \ldots & x_{1k} I_m\\
\vdots & & \vdots \\
x_{k1} I_m & \ldots & x_{kk} I_m
\end{pmatrix}. 
\end{equation*}
Here $x_{ij} \in \mathbb{F}_{q^2.}$ The $i$-th $(k\times k)$-row of $X^{-1}g^A$  is equal to 
\begin{equation}\label{garow}
(x_{(1)\pi_1^{-1}i}^q g_{(1)\pi_1^{-1}}, \ldots, x_{(k)\pi_1^{-1}i}^q g_{(k)\pi_1^{-1}}).
\end{equation} 
Let  $j$ be such that the $(i,j)$-th $(m \times m)$-block of $h$ is not zero (there is only one such $j$ for given $i$ since $h \in S$). Consider the system of linear equations with variables $Z_1, \ldots, Z_k \in H^a$
\begin{equation}\label{sys}
\begin{split}
&x_{11}Z_1+x_{21}  Z_2 + \ldots + x_{k1}Z_k=0\\
&\vdots\\
&\underline{x_{1j} Z_1+x_{2j}Z_2 + \ldots + x_{kj}Z_k=0} \\
& \vdots\\
&x_{1k}Z_1+x_{2k}  Z_2 + \ldots + x_{kk}Z_k=0,\\
\end{split}
\end{equation}
where we exclude the (underlined) $j$-th equation. Thus,  \eqref{sys} consist of $k-1$ linearly independent equations. If we fix $Z_k$ to be some matrix from $GL_n(q^2)$, then $Z_i$ for $i=1, \ldots, k-1$ are  determined uniquely. It is routine to check that 
$$(x_{1j}^q D, \ldots, x_{kj}^q D) \text{ where } D\in GL_n(q^2)$$
is a solution for the system \eqref{sys}. 

Notice that the row \eqref{garow} must be a solution of \eqref{sys}, since $X^{-1}g^AX=h \in S.$ Therefore, by fixing $Z_k$ to be $x_{kj}^q D_i:=x_{(k)\pi_1^{-1}i}^q g_{(k)\pi_1^{-1}}$, we obtain 
$$(x_{(1)\pi_1^{-1}i}^q g_{(1)\pi_1^{-1}}, \ldots, x_{(k)\pi_1^{-1}i}^q g_{(k)\pi_1^{-1}})=(x_{1j}^q D_i, \ldots, x_{kj}^q D_i)$$
for some $D_i \in \alpha H^a$, $\alpha \in \mathbb{F}_{q^2}^*$, since $g_i \in H^a.$ Thus, %$g_i=g_1$ for $i=1, \ldots, k$ and 
$$h=\diag[h_1, \ldots, h_k] \cdot \sigma$$
where $h_i = D_i$. Therefore, $\alpha^{q+1}=1$ and $D_i \in H^a$, since $h_i \in GU_m(q).$ So $h_i\in H \cap H^a$ and $\sigma \in I_m \otimes \Sym(k).$ 

Assume that $h \in S\cap S^{BY},$  so $h =(g')^{BY}$ for $g' \in S$. The same argument as above shows that 
$$h=\diag[h_1, \ldots, h_k] \cdot \sigma$$
where $h_i \in H \cap H^b$ and $\sigma \in I_m \otimes \Sym(k).$ 

Therefore, if $h \in S \cap S^{AX} \cap S^{BY}$ then  $h_i = \lambda_i I_m \in H \cap H^a \cap H^b$ for some $\lambda_i \in \mathbb{F}_{q^2}^*$ with $\lambda_i^{q+1}=1$. So $g^A, g'^B \in  I_m \otimes M$ and 
\begin{equation*}
h \in  I_m \otimes (M \cap M^x \cap M^y) \le Z(GU_n(q)). \qedhere
\end{equation*} 
\end{proof}

\begin{Rem}\label{uniindSL}
If $a,b \in SU_m(q)$ and $x,y \in SU_k(q)$ in Theorem \ref{uniind}, then $AX, BY \in SU_n(q),$ since $AX=a \otimes x$ and $BY=b \otimes y.$
\end{Rem}

\begin{comment}
\begin{Lem} \label{m2isotr}
Let $H$ be an irreducible subgroup of $GU_{2m}(q)$ such that $H$ stabilises decomposition 
$$V=V_1 \oplus V_2$$
with totally isotropic $V_1$ and $\dim V_i=m.$ Denote $ \Stab_H(V_1)|_{V_1} \le GL_m(q^2)$ by $H_1.$ 
If there exist $a,b \in GL_m(q^2)$ such that 
$$H_1 \cap H_1^a \cap H_1^b \le Z(GL_m(q^2)),$$
then there exist $A, B \in GU_{2m}(q)$ such that
$$H \cap H^A \cap H^B \le Z(GU_n(q)).$$ 
\end{Lem}
\begin{proof}
Let $\alpha \in \mathbb{F}_{q^2}^*$ be such that $\alpha + \alpha^q=0.$ Such $\alpha$ always exists. Indeed,
if $q$ is even then $\alpha$ can be an arbitrary element from $\mathbb{F}_q^*.$ Assume that $q$ is odd and $\eta$ is a generator of $\mathbb{F}_{q^2},$ so  $\eta^{(q^2-1)/2}=-1$ is the unique element of order 2 in $\mathbb{F}_{q^2}^*.$ Let $\alpha =\eta^{(q+1)/2},$ therefore $\alpha^{q-1}=-1$ and $\alpha^q=-\alpha$, so $\alpha+\alpha^q=0.$

Let $\beta$ be a basis as in \eqref{unibasis}. Since $V_1$ is totally isotropic and by Lemma \ref{witt} we can assume that $V_1=\langle f_1, \ldots, f_m \rangle.$ Every $v \in V$ has unique decomposition $v=v_1+v_2,$ $v_i \in V_i.$ Let us define the projection operators $\pi_i: V \to V_i$ by $\pi(v)=v_i$ for $i=1,2$. Notice that $$(f_i,e_j)=(f_i, \pi_1(e_j)+\pi_2(e_j))=(f_i,\pi_2(e_j))$$
since $V_1$ is totally isotropic. Also 
$(\pi_2(e_i), \pi_2(e_j))=0$
since $V_2$ is totally isotropic. Therefore the shape {\bf f} has matrix ${\bf f}_{\beta_1}= J_{2m}$ with respect to the basis $ \beta_1=\{f_1, \ldots, f_m, \pi_2(e_1), \ldots, \pi_2(e_m)\}.$ In other words, we can assume  that $V_2= \langle e_1, \ldots, e_m \rangle$ in the first place.  

Therefore there is basis $\beta= \{ f_1, \ldots, f_m, e_1, \ldots, e_m \}$ of $V$ such that ${\bf f}_{\beta}=J_{2m}$ and $V_1=\langle f_1, \ldots, f_m \rangle,$ $V_2=\langle e_1, \ldots, e_m \rangle.$ Notice that matrices 
\begin{align*}
A:=
\begin{pmatrix}
\alpha a& a \\
0 & (\alpha a)^*
\end{pmatrix} \text{ and }
B:=
\begin{pmatrix}
\alpha b& b \\
0 & (\alpha b)^*
\end{pmatrix}
\end{align*}
lie in $GU_n(q, {\bf f}_{\beta}).$
We claim that $H_{\beta} \cap H_{\beta}^A \cap H_{\beta}^B \le Z(GU_n(q, {\bf f}_{\beta})).$ Consider
$g \in H_{\beta} \cap H_{\beta}^A,$ so $g=h^A$ for some $h \in H_{\beta}.$
If $h \in H \backslash ( \Stab_H(V_1))_{\beta},$ then, since $H$ stabilises the decomposition $V=V_1 \oplus V_2,$ $h$ is of the shape 
$$
\begin{pmatrix}
0& h_1\\
h_1^*& 0
\end{pmatrix}
$$
where $h_1 \in GL_m(q^2).$ Calculations show that $$h^A=
\begin{pmatrix}
\alpha (\overline{a}^{\top}) h^* a & (\overline{a}^{\top})h^*a+ \alpha\alpha^*a^{-1}ha^*\\
\alpha^{q+1}(\overline{a}^{\top})ha & \alpha^q(\overline{a}^{\top})ha
\end{pmatrix},
$$ so $g$ does not stabilise the decomposition since both $m\times m$ blocks in the  second ($m\times m)$-row are non-zero matrices. Therefore $h \in (Stab_H(V_1))_{\beta}$ and 
$$ h=
\begin{pmatrix}
h_1& 0\\
0& h_1^*
\end{pmatrix}
$$
where $h_1 \in H_1.$ Calculations show that  $$h^A=
\begin{pmatrix}
h_1^a & \alpha^{-1}(h_1^a-(h_1^a)^*)\\
0 & (h_1^a)^*
\end{pmatrix},
$$ 
so $g \in H_{\beta} \cap H_{\beta}^A$ if and only if $h_1^a=(h_1^a)^*$ and $g=\diag (h_1^a, h_1^a)$ for $h_1 \in H_1.$

The same argument for $g \in H_{\beta} \cap H_{\beta}^B$ shows that if $g \in H_{\beta} \cap H_{\beta}^A \cap H_{\beta}^B$ then $g=\diag(g_1,g_1)$ for $g_1 \in H_1 \cap H_1^a \cap H_1^b$, so $g \in Z(GU_{2m}(q, {\bf f}_{\beta})).$
\end{proof}
\end{comment}


\begin{Lem} \label{m2isotr}
Let $k \in \{2,4,6,8\}$. Let $H$ be an irreducible subgroup of $GU(V)$  that stabilises the decomposition 
$$V=V_1 \oplus \ldots \oplus V_k$$
as in $(2)$ of Lemma $\ref{ashb}$, so each $V_i$ is totally isotropic  and $\dim V_i=m,$ where $n=km.$ Denote $ \Stab_H(V_1)|_{_{V_1}} \le GL(V_1)$ by $H_1.$ 
If there exist $a,b \in GL(V_1)$ $(\text{respectively } SL(V_1))$ such that 
$$H_1 \cap H_1^a \cap H_1^b \le Z(GL(V_1)),$$
then there exist $A, B \in GU(V)$ $(\text{respectively } SU(V))$ such that
$$H \cap H^A \cap H^B \le Z(GU(V)).$$ 
\end{Lem}
\begin{proof}
Let $\alpha \in \mathbb{F}_{q^2}^*$ be such that $\alpha + \alpha^q=0.$ Such $\alpha$ always exists. Indeed,
if $q$ is even, then $\alpha$ can be an arbitrary element of $\mathbb{F}_q^*.$ Assume that $q$ is odd and $\eta$ is a generator of $\mathbb{F}_{q^2}^*,$ so  $\eta^{(q^2-1)/2}=-1$ is the unique element of order 2 in $\mathbb{F}_{q^2}^*.$ Let $\alpha =\eta^{(q+1)/2},$ therefore, $\alpha^{q-1}=-1$ and $\alpha^q=-\alpha$, so $\alpha+\alpha^q=0.$

Assume $k=2.$ Let $\beta$ be a basis as in \eqref{unibasis}. Since $V_1$ is totally isotropic, we can assume that $V_1=\langle f_1, \ldots, f_m \rangle$ by Lemma \ref{witt}. Every $v \in V$ has a unique decomposition $v=v_1+v_2,$ $v_i \in V_i.$ Define the projection operators $\pi_i: V \to V_i$ by $(v)\pi_i=v_i$ for $i=1,2$. Notice that $$(f_i,e_j)=(f_i, (e_j)\pi_1+(e_j)\pi_2)=(f_i,(e_j)\pi_2)$$
since $V_1$ is totally isotropic. Also 
$((e_i)\pi_2, (e_j)\pi_2)=0$
since $V_2$ is totally isotropic. Therefore, the form {\bf f} has matrix ${\bf f}_{\beta_1}= J_{2m}$ with respect to the basis $ \beta_1=\{f_1, \ldots, f_m, (e_1)\pi_2, \ldots, (e_m)\pi_2\}.$ In other words, we can assume  that $$V_2= \langle e_1, \ldots, e_m \rangle.$$  

Therefore, applying the  above argument to each $U_i=V_{2i-1} \oplus V_{2i}$ for $i \in \{1,2,3\}$, we obtain  a basis $$\beta= \{ f_{11}, \ldots, f_{1m}, e_{11}, \ldots, e_{1m}, \ldots, f_{(k/2)1}, \ldots, f_{(k/2)m}, e_{(k/2)1}, \ldots, e_{(k/2)m} \}$$ of $V$ such that ${\bf f}_{\beta}=J_{2m} \otimes I_{k/2}$ and $V_{2i-1}=\langle f_{i1}, \ldots, f_{im} \rangle,$ $V_{2i}=\langle e_{i1}, \ldots, e_{im} \rangle.$ 

Recall that $g^{\dagger}=(\overline{g}^{\top})^{-1}$ for $g \in GU_n(q).$ For $x \in GL_m(q^2),$ denote by $X(k,x)$ the initial $(k \times k)$-submatrix  of the  matrix
\begin{equation*}
X(x)= \begin{pmatrix} 
\alpha x& \multicolumn{1}{c|}{x} & \alpha x& \multicolumn{1}{c|}{x}&0&\multicolumn{1}{c|}{0}&0&0\\
0&\multicolumn{1}{c|}{(\alpha x)^{\dagger}}& 0&\multicolumn{1}{c|}{0}&0&\multicolumn{1}{c|}{0}&0&0\\ \cline{1-2}
0& -x& 0 &\multicolumn{1}{c|}{x}&0&\multicolumn{1}{c|}{0}&0&0\\
0 & 0&x^{\dagger}&\multicolumn{1}{c|}{-(\alpha x)^{\dagger}}&x^{\dagger}&\multicolumn{1}{c|}{-(\alpha x)^{\dagger}}&0&0\\ \cline{1-4}
0&0&0&0&\alpha x&\multicolumn{1}{c|}{x}&0&0\\
0& (\alpha x)^{\dagger}&0&-(\alpha x)^{\dagger}&0&\multicolumn{1}{c|}{(\alpha x)^{\dagger}}&x^{\dagger}&-(\alpha x)^{\dagger}\\ \cline{1-6}
0&0&0&0&0&0&\alpha x& x\\
0&0&0&0&x^{\dagger}&-(\alpha x)^{\dagger}&0& (\alpha x)^{\dagger}
\end{pmatrix}.
\end{equation*}

It is routine to check that $X(k,x) \in GU_n(q, {\bf f}_{\beta})$. If $g \in H_{\beta} \cap H_{\beta}^{X(k,a)},$  then $g$ stabilises both 
\begin{equation}\label{2m2dec1}
V=V_1 \oplus \ldots \oplus V_k,
\end{equation}
and
\begin{equation}\label{2m2dec2}
V=(V_1){X(k,a)} \oplus \ldots \oplus (V_k) {X(k,a)}.
\end{equation}

Let $k=8$ and $v \in V$. Since $g$ stabilises \eqref{2m2dec1}, $(v)g$ and $v$ have the same number of non-zero projections on the $V_i$.  Hence $g$ stabilises $(V_2)X(k,a)$ and $V_2$ because $(V_2)X(k,a)$ is the only subspace in \eqref{2m2dec2} which has only one non-zero projection on the $V_i$. Therefore, $g$ stabilises  $V_1$ and $(V_2)X(k,a)$ because they are the only subspaces which are not orthogonal to $V_2$  and $(V_2)X(k,a)$ respectively in decompositions \eqref{2m2dec1} and \eqref{2m2dec2}. Since $g$ stabilises $V_2$, it stabilises $(V_3)X(k,a)$, so $g$ also stabilises $V_3$, $V_4$ and $(V_4)X(k,a).$ Since $g$ stabilises $(V_4)X(k,a)$, it stabilises $V_5 \oplus V_6$, so it stabilises $(V_5)X(k,a)$ and $(V_6)X(k,a).$ Now it is easy to see that $g$ must stabilise $V_5$ and $V_6.$ Since $g$ stabilises $(V_7)X(k,a) \oplus (V_8)X(k,a)$, it stabilises $(V_8)X(k,a),$ so $g$ stabilises $V_8$ and $V_7.$ Therefore, $g$ stabilises all subspaces in \eqref{2m2dec1} and \eqref{2m2dec2}, so $g=\diag[g_1, g_1^{\dagger}, \ldots, g_{k/2}, g_{k/2}^{\dagger}]$ with $g_i \in H_1.$ 

Since $X(x)^{-1}={\bf f}_{\beta}\overline{X(x)}^{\top}{\bf f}_{\beta},$ 
\begin{equation*}
X(x)^{-1}= \Scale[0.93]{\begin{pmatrix} 
(\alpha x)^{-1}& \multicolumn{1}{c|}{\overline{x}^{\top}} & 0& \multicolumn{1}{c|}{-\overline{x}^{\top}}&0&\multicolumn{1}{c|}{0}&0&0\\
0&\multicolumn{1}{c|}{(\overline{\alpha x})^{\top}}& 0&\multicolumn{1}{c|}{0}&0&\multicolumn{1}{c|}{0}&0&0\\ \cline{1-2}
0& \overline{x}^{\top^{\phantom{1}}}& -(\alpha x)^{-1} &\multicolumn{1}{c|}{\overline{x}^{\top}}&-(\alpha x)^{-1}&\multicolumn{1}{c|}{0}&0&0\\
0 & (\overline{\alpha x})^{\top}&x^{-1}&\multicolumn{1}{c|}{0}&0&\multicolumn{1}{c|}{0}&0&0\\ \cline{1-4}
0&0&-(\alpha x)^{-1}&0&(\alpha x)^{-1}&\multicolumn{1}{c|}{\overline{x}^{\top}}&-(\alpha x)^{-1}&0\\
0& 0&x^{-1}&0&0&\multicolumn{1}{c|}{(\overline{\alpha x})^{\top}}&x^{-1}&0\\ \cline{1-6}
0&0&0&0&-(\alpha x)^{-1}&0&(\alpha x)^{-1}& \overline{x}^{\top}\\
0&0&0&0&x^{-1}&0&0& (\overline{\alpha x})^{\top}
\end{pmatrix}}.
\end{equation*}

A similar argument to the above shows that if $h =g^{X(k,a)^{-1}} \in H_{\beta} \cap H_{\beta}^{X(k,a)^{-1}}$, then $h=\diag[h_1, h_1^{\dagger}, \ldots, h_{k/2}, h_{k/2}^{\dagger}]$ with $h_i \in H_1.$ Calculations show that if the equation
$h^{X(k,a)}=g$  holds, then 
\begin{equation*}
\begin{cases}
g_i &  =  g(2i-1,2i-1)= h_i^a \text{ for } i \in \{1,2,3,4\} \\
0 & =  g(1,2)=(\alpha a)^{-1} h_1 a + \overline{a}^{\top} h_1^{\dagger} (\alpha a)^{\dagger}= {\alpha}^{-1}(h_1^a - (h_1^a)^{\dagger});\\
0 & =  g(1,3)=(\alpha a)^{-1} h_1 (\alpha a) - \overline{a}^{\top}h_2^{\dagger} a^{\dagger}=h_1^a - (h_2^a)^{\dagger};\\
0 & =  g(1,5)= - \overline{x}^{\top} h_2^{\dagger} a^{\dagger} +(\alpha a)^{-1} h_3 (\alpha a)= h_3^a - (h_2^a)^{\dagger};\\
0 & =  g(8,5)= a^{-1} h_3 (\alpha a) + (\overline{\alpha a})^{\top}h_4^{*} a^{\dagger}= \alpha (h_3^a - (h_4^a)^{\dagger}).
\end{cases}
\end{equation*}
So $h_1^a=g_1=g_1^{\dagger} \ldots = g_{k/2}= g_{k/2}^{\dagger}$ and $g_1 \in H_1 \cap H_1^a.$ 

 If $g \in H_{\beta} \cap H_{\beta}^{X(k,a)} \cap H_{\beta}^{X(k,b)},$ then the same argument   with $a$ replaced by $b$ shows that 
$g=\diag[g_1, \ldots, g_1]$ and $g_1 \in H_{1} \cap H_{1}^{a} \cap H_{1}^{b},$ so $g \in Z(GU_{2m}(q, {\bf f}_{\beta}))$.

The proof for $k \in \{2,4,6\}$ is analogous.

Calculations show that if $x \in SL_m(q)$, then 
$$\det(X(k,x))=\begin{cases}
1 & \text{ for } k=4,8;\\
(-1)^m & \text{ for } k=2,6.
\end{cases}
$$ 
Consider $A=\diag[\alpha I_m, \alpha^{\dagger} I_m, I_m, \ldots, I_m] \in GU_n(q, {\bf f}_{\beta})$. Notice that $\det(A)=(-1)^m.$ Repeating the arguments above, one can show that  $$H_{\beta} \cap H_{\beta}^{AX(k,a)} \cap H_{\beta}^{AX(k,b)} \le Z(GU_n(q, {\bf f}_{\beta})).$$  Notice that $X(k,a), X(k,b) \in SU_n(q, {\bf f}_{\beta})$ for $k=4,8$ and $AX(k,a), AX(k,b) \in SU_n(q, {\bf f}_{\beta})$ for $k=2,6.$ 
\end{proof}

\begin{comment}
\begin{Lem}
Let $H$ be an irreducible subgroup of $GU(V)$ such that $H$ stabilises decomposition 
$$V=V_1 \oplus \ldots \oplus V_k$$
as in (2) of Lemma \ref{ashb}, so $k$ is even, $V_i$ is totally isotropic  and $\dim V_i=m,$ so $n=km.$ Denote $ \Stab_H(V_1)|_{V_1} \le GL(V_1)$ by $H_1.$ Let $\Phi$ be the permutation matrix for the permutation $$\sigma=(1,2)(2,3) \ldots (k-1, k)$$ and let $M = M_k(q) \cap GU_k(q,\Phi).$ 
If there exist $a,b \in GL(V_1)$ and $x,y \in GU_k(q,\Phi)$ such that 
$$H_1 \cap H_1^a \cap H_1^b \le Z(V_1) \text{ and } M \cap M^x \cap M^y \le Z(GU_k(q,\Phi))$$
then $b_H(GU(V))\le 3.$
\end{Lem}
\begin{proof}
Let us fix a basis $\beta_1$ of $V_1$. By \cite[\S 15 , Lemma 5]{sup} for odd $i=3, \ldots, k-1$ we can choose a basis $\beta_i$ of $V_i$ such that $(H_i)_{\beta_i}=(H_1)_{\beta_1}$. By the proof of Lemma \ref{m2isotr}, it is possible to choose a basis $\beta_i$ for even $i=2, \ldots, k$ such that  if $\beta =\cup_{i=1}^k \beta_i,$ then 
\begin{equation*}
{\bf f}_{\beta}=
\begin{pmatrix}
0   & I_m    &          &          & \\
I_m & 0      &    &          & \\  
    &  & \ddots   &    & \\
    &        &    & 0        & I_m \\
    &        &          &   I_m    & 0        
\end{pmatrix}.
\end{equation*} 



then $H_{\beta}$ is a subgroup of $(H_1)_{\beta} \wr \Gamma$ where $\Gamma \le \Sym(k).$ Notice that $\Gamma \le \Sym(k) \cap GU_k(q, \Phi)$ and 

Denote 
\begin{align*}
&X=I_m \otimes x & & A= \begin{pmatrix} a & 0 \\ 0 & a^{\dagger}  \end{pmatrix} \otimes I_{k/2}\\
&Y=I_m \otimes y & & B= \begin{pmatrix} b & 0 \\ 0 & b^{\dagger}  \end{pmatrix} \otimes I_k/2.
\end{align*}  Consider $h \in H\cap H^{AX},$ so $h=g^{AX},$ $g  \in H.$ Hence 
$$g^A = \diag(g_1, \ldots , g_k) \cdot \pi,$$ where $g_i \in H^a$ and $\pi=I_m \otimes \pi_1 $ for some $\pi_1 \in \Sym(k).$ If 
\begin{equation*}
x=
\begin{pmatrix}
x_{11} & \ldots & x_{1k}\\
\vdots & & \vdots \\
x_{k1} & \ldots & x_{kk}
\end{pmatrix}, \text{ then }
x^{-1}=
\begin{pmatrix}
x_{11}^q & \ldots & x_{k1}^q\\
\vdots & & \vdots \\
x_{1k}^q & \ldots & x_{kk}^q
\end{pmatrix},
\end{equation*}
since $x \in GU_n(q)$, and
\begin{equation*}
X=
\begin{pmatrix}
x_{11} I_m & \ldots & x_{1k} I_m\\
\vdots & & \vdots \\
x_{k1} I_m & \ldots & x_{kk} I_m
\end{pmatrix}. 
\end{equation*}
Here $x_{ij} \in \mathbb{F}_{q^2.}$ The $i$-th $(m\times m)$-row of $X^{-1}g^A$  is equal to 
\begin{equation}\label{garow}
(x_{(1)\pi_1^{-1}i}^q g_{(1)\pi_1^{-1}}, \ldots, x_{(k)\pi_1^{-1}i}^q g_{(k)\pi_1^{-1}}).
\end{equation} 

Consider the system of linear equations with variables $Z_1, \ldots, Z_k \in H^a$
\begin{equation}\label{sys}
\begin{split}
&x_{11}Z_1+x_{21}  Z_2 + \ldots x_{k1}Z_k=0\\
&\vdots\\
&\underline{x_{1j} Z_1+x_{2j}Z_2 + \ldots x_{kj}Z_k=0} \\
& \vdots\\
&x_{1k}Z_1+x_{2k}  Z_2 + \ldots x_{kk}Z_k=0\\
\end{split},
\end{equation}
where we exclude the $j$-th equation for $j$ such that the $(i,j)$-th $(m \times m)$-block of $h$ is not zero (there is only one such $j$ for given $i$ since $h \in S$). Thus,  \eqref{sys} consist of $k-1$ linearly independent equations, so if we fix $Z_k$ to be some matrix from $GL_n(q^2)$, then $Z_i$ for $i=1, \ldots, k-1$ are  determined uniquely. It is routine to check that 
$$(x_{1j}^q D, \ldots, x_{kj}^q D), \text{ } D\in GL_n(q^2)$$
is a solution for the system \eqref{sys}. 

Notice that the row \eqref{garow} must be a solution of \eqref{sys}, since $X^{-1}g^AX=h \in S.$ Therefore, by fixing $Z_k$ to be $x_{kj}^q D_i:=x_{(k)\pi_1^{-1}i}^q g_{(k)\pi_1^{-1}}$, we obtain 
$$(x_{(1)\pi_1^{-1}i}^q g_{(1)\pi_1^{-1}}, \ldots, x_{(k)\pi_1^{-1}i}^q g_{(k)\pi_1^{-1}})=(x_{1j}^q D_i, \ldots, x_{kj}^q D_i)$$
for some $D_i \in \alpha H^a$, $\alpha \in \mathbb{F}_{q^2}^*$, since $g_i \in H^a.$ Thus, %$g_i=g_1$ for $i=1, \ldots, k$ and 
$$h=\diag(h_1, \ldots, h_k) \cdot \sigma$$
such that $h_i = D_i$. Therefore $\alpha^{q+1}=1$ and $D_i \in H^a$, since $h_i \in GU_m(q).$ So, $h_i\in H \cap H^a$ and $\sigma \in I_m \otimes \Sym(k).$ 

Assume that $h \in S\cap S^{BY},$  so $h =(g')^{BY}$ for $g' \in S$. The same argument as above shows that 
$$h=\diag(h_1, \ldots, h_k) \cdot \sigma$$
such that $h_i \in H \cap H^b$, $\sigma \in I_m \otimes \Sym(k).$ 

Therefore, if $h \in S \cap S^{AX} \cap S^{BY}$ then  $h_i = \lambda_i I_m \in H \cap H^a \cap H^b$ for some $\lambda_i \in (\mathbb{F}_{q^2}^*)^{q-1}$. So $g^A, g'^B \in  I_m \otimes M$ and 
$$h \in  I_m \otimes (M \cap M^x \cap M^y) \le Z(GU_n(q)).$$
\end{proof}
\end{comment}

\begin{Lem}\label{isklwr}
Let $n=mk$ for integers $m \ge 3$ and $k \ge 2$. 
\begin{enumerate}[font=\normalfont]
\item If $S=GU_2(2) \wr \Sym(k)$, then  $b_S(S \cdot SU_{2k}(2)) \le 3.$ \label{isklwr1}
\item If $S=GU_2(3) \wr \Sym(k)$, then $b_S(S \cdot SU_{2k}(3))\le 3.$ \label{isklwr2}
\item If $S=GU_3(2) \wr \Sym(k)$, then $b_S(S \cdot SU_{3k}(2))\le 3.$ \label{isklwr3}
\item Let $N$ be a  quasi-primitive maximal solvable  subgroup of $GU_m(2)$. If $S=N \wr \Sym(k)$ with $k \in \{2,3,4\}$, then $b_S(S \cdot SU_{km}(2))\le 3.$ \label{isklwr4}
\item Let $N$ be a  quasi-primitive maximal  solvable  subgroup of $GU_m(3)$. If $S=N \wr \Sym(2)$, then $b_S(S \cdot SU_{2m}(3))\le 3.$ \label{isklwr5}
\end{enumerate}
\end{Lem}
\begin{proof}
Notice that $S \cdot SU_n(q)=GU_n(q)$ for \eqref{isklwr1} -- \eqref{isklwr2}.  

\eqref{isklwr1} Notice that $\lambda^{q+1}=1$ for $q=2$ and $\lambda \in \mathbb{F}_{q^2}^*.$
Therefore, since  a row $v$ of a matrix in $GU_n(q)$ satisfies ${\bf f}(v,v)=1,$
every matrix in $GU_2(2)$ is  monomial. Thus, $$S=GU_2(2) \wr \Sym(k)\le MU_{2k}(q)$$ and  the statement for $k>2$ follows by \eqref{omnom5} of Lemma \ref{omnom}. The case $k=2$ is verified by computation.

\medskip

\eqref{isklwr2} For $k \in \{2,3\}$ we verify the statement by computation, so assume $k \ge 4.$ Let $\beta =\{v_{11}, v_{12}, v_{21}, v_{22}, \ldots, v_{k1}, v_{k2}\}$ be an orthonormal basis of $V$ such that $S$ stabilises the decomposition $V_1 \oplus \ldots \oplus V_k$ with $V_i=\langle v_{i1}, v_{i2} \rangle.$ Define a basis $\beta_1=\{w_{11}, w_{12}, w_{21}, w_{22}, \ldots, w_{k1}, w_{k2}\}$ by the following rule:
\begin{equation*}
\begin{split}
(w_{11}, w_{21}, \ldots, w_{k1})&=(v_{12}, v_{21}, \ldots, v_{k1})B(k,a); \\
(w_{12}, w_{22}, \ldots, w_{k2})&=(v_{22}, v_{32}, \ldots, v_{k2}, v_{12})B(k,a).
\end{split}
\end{equation*}
Here $a$ and $B(k,a)$ are as in \eqref{omnom2} of Lemma \ref{omnom}. Denote the change-of-basis matrix from $\beta_1$ to $\beta$ by $y$. For example, if $k=4$, then  
\begin{equation*}
y=
\begin{pmatrix}
a& \multicolumn{1}{c|}{}&a&\multicolumn{1}{c|}{}&&\multicolumn{1}{c|}{}&& \\
& \multicolumn{1}{c|}{}&&\multicolumn{1}{c|}{a}&&\multicolumn{1}{c|}{a}&& \\ \hline%{1-8}
a^2& \multicolumn{1}{c|}{} &-a^2&\multicolumn{1}{c|}{}&a&\multicolumn{1}{c|}{}&& \\
  &\multicolumn{1}{c|}{} &&\multicolumn{1}{c|}{a^2}&&\multicolumn{1}{c|}{-a^2}&&a \\\hline%{1-8}
a^3&\multicolumn{1}{c|}{} &-a^3&\multicolumn{1}{c|}{} &-a^2&\multicolumn{1}{c|}{}&a& \\
 &\multicolumn{1}{c|}{a}&&\multicolumn{1}{c|}{a^3}&&\multicolumn{1}{c|}{-a^3}&&-a^2 \\\hline%{1-8}
a^3&\multicolumn{1}{c|}{} &-a^3&\multicolumn{1}{c|}{}&-a^2&\multicolumn{1}{c|}{}&-a& \\
& \multicolumn{1}{c|}{-a}&&\multicolumn{1}{c|}{a^3}&&\multicolumn{1}{c|}{-a^3}&&-a^2 \\ 
\end{pmatrix}.
\end{equation*}
We use blanks instead of zeroes in the matrix. It is routine to verify that $\beta_1$ is orthonormal, so $y \in GU_{2k}(3)$. Observe $g \in S \cap S^y$ stabilises the decompositions
$$V_1 \oplus \ldots \oplus V_k \text{ and  } W_1 \oplus \ldots \oplus W_k, $$
where $W_i=\langle w_{i1}, w_{i2} \rangle.$

Notice that $w_{11}$ has non-zero entries only in two $V_i$-s, so $(w_{11})g$ also must have non-zero entries only in two $V_i$-s. It is easy to see that a vector from $W_j$ for $j>1$ has non-zero entries  in at least three $V_i$-s. Thus, $g$ stabilises $W_1.$ The same argument shows that $g$ must stabilise $W_i$ for $i=1, \ldots, k-2.$ Notice that $(w_{ij})g$ lies either in $\langle w_{i1} \rangle$ or $\langle w_{i2} \rangle$ for $i=1, \ldots, k-2,$ since otherwise it would have non-zero entries in more $V_i$-s than $w_{ij}$.

Assume that $(w_{11})g \in \langle w_{12}\rangle$. Therefore, $(w_{12})g \in \langle w_{11}\rangle$, so either
 $$(V_1)g=V_2, (V_2)g=V_3, (V_3)g=V_1,$$
 or
 $$(V_1)g=V_3, (V_2)g=V_2, (V_3)g=V_1.$$
In both cases $(w_{22})g$ cannot lie in either  $\langle w_{21} \rangle$ or $\langle w_{22} \rangle$, which is a contradiction. So $g$ stabilises $\langle w_{11} \rangle$ and, therefore, it stabilises $\langle w_{12} \rangle$, since $\langle w_{12} \rangle$ is the orthogonal complement of $\langle w_{11} \rangle$ in $W_1.$ Therefore, $g$ stabilises $V_1,$ $V_2$ and $V_3.$ The same argument shows that $g$
stabilises $V_1, \ldots, V_{k-1},$ so $g$ stabilises $V_k$ as well. Thus, $g$ stabilises $$\langle w_{11} \rangle, \langle w_{12} \rangle, \ldots, \langle w_{(k-2)1} \rangle, \langle w_{(k-2)2} \rangle,$$ which implies that $g$ stabilises   $\langle v_{11} \rangle, \langle v_{12} \rangle, \ldots, \langle v_{k1} \rangle, \langle v_{k2} \rangle$. So $$g=\diag(g_{11}, g_{12}, \ldots, g_{k1}, g_{k2}).$$
Since $(w_{11})g \in \langle w_{11} \rangle$, $g_{11}=g_{21}.$ Applying the same argument to all $w_{ij}$ for $i=1, \ldots, k-2$ and $j= 1,2$ we obtain that $g$ is scalar.

\medskip

\eqref{isklwr3} For $k \le 3$ the statement is verified by computation, so assume $k \ge 4.$ Fix $$\beta=\{v_{11}, v_{12}, v_{13}, v_{21}, \ldots, v_{k3}\}$$ to be the initial orthonormal basis, so $g \in S$ stabilises the decomposition
\begin{equation}\label{dec1}
V=V_1 \oplus \ldots \oplus V_k,
\end{equation}
where $V_i=\langle v_{i1}, v_{i2}, v_{i3} \rangle$.
 Let $x$ be the permutation matrix for the permutation $(1,2, \ldots, n)$ where $n=3k.$ Consider $g \in S \cap S^x$. We claim that $g$ is monomial. Indeed, since $g \in S^x$, it stabilises the decomposition
\begin{equation} \label{dec2}
V=(V_1)x \oplus \ldots \oplus (V_k)x=\langle v_{12}, v_{13}, v_{21} \rangle \oplus \ldots \oplus \langle v_{k2}, v_{k3}, v_{11} \rangle,
\end{equation} 
so it permutes subspaces $\langle v_{11} \rangle, \langle v_{21} \rangle, \ldots, \langle v_{k1} \rangle$ and  $\langle v_{12}, v_{13} \rangle, \langle v_{22}, v_{23} \rangle, \ldots, \langle v_{k2}, v_{k3} \rangle.$ Thus,  $g$ consists of $(1 \times 1)$ and $(2 \times 2)$ blocks which lie in $GU_1(2)$ and $GU_2(2)=MU_2(2)$, respectively. 

Define a basis $\beta_1=\{w_{11}, w_{12}, w_{13}, w_{21}, \ldots, w_{k3}\}$ as follows:
\begin{equation*}
\begin{split}
w_{11}&= (\sum_{i=1}^{k-1} \sum_{j=1}^3 v_{ij}) +v_{k1} +((1+(-1)^k)/2)v_{k2}  \\ 
w_{12}&= v_{11} + v_{12} + v_{k3}\\
w_{13}&= v_{11} + v_{21} + v_{k3}\\
w_{s1}&= v_{11} + v_{(s-1)3} + v_{k3}\\
w_{s2}&= v_{11} + v_{s2} + v_{k3}\\
w_{s3}&= v_{11} + v_{(s+1)1} + v_{k3}\\
w_{k1}&= v_{11} + v_{(k-1)3} + v_{k3}\\
w_{k2}&= v_{11} + v_{k2} + v_{k3}\\
w_{k3}&= ((1+(-1)^k)/2)v_{12} +v_{13} + (\sum_{i=2}^{k} \sum_{j=1}^3 v_{ij}). \\
\end{split}
\end{equation*}
Here $1<s<k.$  Denote the change-of-basis matrix from $\beta_1$ to $\beta$ by $y$. For example, if $k=3$, then
\begin{equation*}
y=
\begin{pmatrix}
1&1& \multicolumn{1}{c|}{1}&1&1&\multicolumn{1}{c|}{1}&1&0&0 \\
1&1& \multicolumn{1}{c|}{0}&0&0&\multicolumn{1}{c|}{0}&0&0&1 \\
1&0& \multicolumn{1}{c|}{0}&1&0&\multicolumn{1}{c|}{0}&0&0&1 \\ \cline{1-9}
1&0& \multicolumn{1}{c|}{1}&0&0&\multicolumn{1}{c|}{0}&0&0&1 \\
1&0& \multicolumn{1}{c|}{0}&0&1&\multicolumn{1}{c|}{0}&0&0&1 \\
1&0& \multicolumn{1}{c|}{0}&0&0&\multicolumn{1}{c|}{0}&1&0&1 \\\cline{1-9}
1&0& \multicolumn{1}{c|}{0}&0&0&\multicolumn{1}{c|}{1}&0&0&1 \\
1&0& \multicolumn{1}{c|}{0}&0&0&\multicolumn{1}{c|}{0}&0&1&1 \\
0&0& \multicolumn{1}{c|}{1}&1&1&\multicolumn{1}{c|}{1}&1&1&1 \\
\end{pmatrix}.
\end{equation*}

It is routine to verify that $\beta_1$ is orthonormal, so $y \in GU_{3k}(2)$. If $g \in S \cap S^x \cap S^y$, then $g$ stabilises decompositions \eqref{dec1}, \eqref{dec2} and 
\begin{equation*}
V=W_1 \oplus \ldots \oplus W_k,
\end{equation*}
where $W_i=\langle w_{i1}, w_{i2}, w_{i3} \rangle.$ Since $g$ is monomial, $w_{ij}$ and $(w_{ij})g$ have the same number of non-zero entries in the decomposition with respect to $\beta.$ Therefore, $(w_{11})g$ can lie either in $W_1$ or in $W_k.$ Assume that $(w_{11})g \in W_k,$ so $(W_1)g=W_k.$ If a vector in $W_k$ has the same number of non-zero entries  in the decomposition with respect to $\beta$ as $w_{11}$, then its first $1+((1+(-1)^k)/2)$ entries are zero. So $g$ must permute subspace $\langle v_{11}, v_{12} \rangle$ with $\langle v_{k2}, v_{k3} \rangle$ for $k$ odd (respectively $\langle v_{11} \rangle$ with $\langle v_{k3} \rangle$ for $k$ even) which contradicts the fact that $g$ stabilises decompositions \eqref{dec1} and \eqref{dec2}. Therefore, $g$ stabilises $W_1$ and $\langle w_{11} \rangle$ in particular. It is easy to see now that $g$ stabilises  $\langle w_{12} \rangle$ and $\langle w_{13} \rangle$, since $g$ stabilises \eqref{dec1}. Thus, $g$ stabilises $V_1$, $V_k$ and $V_2$, so it stabilises  $\langle v_{11} \rangle$,
$\langle v_{12} \rangle$, $\langle v_{13} \rangle$, $\langle v_{21} \rangle$, $\langle v_{k3} \rangle$. Using the same argument, we obtain that $g$ is diagonal. Since $g$ stabilises $\langle w_{11} \rangle$ and $\langle w_{k3} \rangle$, all non-zero entries of $g$ must be equal, so $g$ is scalar. 





\medskip

\eqref{isklwr4}-\eqref{isklwr5}  For  $n<12$ we verify the statement by computation. For larger $n$ we prove the statement by checking \eqref{0} with $c=3$ for the elements of prime order of $H \le PGU_n(q)$, where $H$ and $G$ are the images of $S$ and $S \cdot SU_n(q)$ respectively in $PGU_n(q).$  

 Let $B$ be the image in $G$ of the block-diagonal subgroup $$GU_m(q) \times \ldots \times GU_m(q) \le GU_{km}(q).$$ 
Let $x \in H$ have prime order.

 If  $(|x| > 3$ for $k \in \{3,4\})$ or $(|x| > 2$ for $k=2)$, then $x^G \cap H \subseteq B.$ 
In this case there exists a preimage $\hat{x}=\diag[\hat{x_1}, \ldots ,\hat{x_k}]$ of $x$ in $GU_{km}(q)$ such that $\hat{x_i} \in N.$  If $x_i\ne 1$, then $\nu(x_i)\ge m/4$ by Lemma \ref{nuuni}. We can assume that $x$ is such that the number $l(x)$ of $x_i$ not equal to 1 is maximal for elements in $x^G \cap H.$ Therefore, 
$$\nu(x) \ge (1/4)l(x)m \text{ and } |x^G \cap H| \le \tbinom{k}{l(x)} |N|^{l(x)}.$$
These bounds together with bounds from Lemma \ref{uniqorder} for $|N|$ and \eqref{5uni} for $|x^G|$ are  sufficient to show \eqref{0} holds for $mk \ge 12.$

\medskip
Now consider the case where $x^G \cap H$ is not a subset of $B$. For such $x$, we use the bounds for $|x^G|$ and $|x^G \cap H|$ given in \cite[Propositions 2.5 and 2.6]{fpr3}. These propositions give corresponding bounds when $H=(GU_m(q) \wr \Sym(k))/Z(GU_n(q)),$ so they are applicable in our situation. For $mk \ge 12$, these bounds  are sufficient to show that \eqref{0} holds. 

We briefly outline how to extract the corresponding bounds. The proofs of the propositions split into several cases depending on $|x|,$ $m$, $k$ and 
$|H^1(\sigma, E/E^0)|$ (see Definition \ref{H1EE0}). Notice that if $x$ is semisimple, then $|H^1(\sigma, E/E^0)|=(|x|,q+1)$ by \cite[Lemma 3.35]{fpr}. 


 Assume that $k=2$, so $|x|=2.$ If $q=2$, then we use the bounds from {\bf Case 2.2} of the proof of  \cite[Proposition 2.6]{fpr3} for unipotent $x$. 
If $q=3$, then we use bounds from {\bf Case 2.4} of the proof of  \cite[Proposition 2.5]{fpr3} for semisimple $x.$

Assume that $k \in \{3,4\}$, so $q=2.$ If $|x|=2$, then we use the bounds from {\bf Case 2.2} of the proof of  \cite[Proposition 2.6]{fpr3} for unipotent $x$. 
Let  $|x|=3$. We use bounds  from {\bf Case 2.2} (if $|H^1(\sigma, E/E^0)|=1$) and {\bf Case 2.3} (if $|H^1(\sigma, E/E^0)|=3$)  of the proof of  \cite[Proposition 2.5]{fpr3}. 
\end{proof}

\begin{Th} \label{irredGU}
Let $(n,q)$ be such that $GU_n(q)$ is not solvable. If $S$ is an irreducible maximal solvable  subgroup of $GU_n(q)$,  then $b_S(S \cdot SU_n(q)) \le 3$ or $(n,q)=(4,2)$. If $(n,q)=(4,2)$  and $b_S(GU_n(q))>3$, then  $S$ is conjugate to $GU_1(q) \wr \Sym(4)=MU_4(2)$ (so $S \cdot SU_n(q)=GU_n(q)$) and $b_S(GU_n(q))=4.$
\end{Th}
\begin{proof}
Let us fix $q$ and consider a minimal counterexample to the statement $$b_S(S \cdot SU_n(q)) \le 3,$$ so $(n,S)$ is such that  $n$ is the smallest integer satisfying the conditions of the theorem: namely,  $GU_n(q)$ is not solvable and $GU_n(q)$ has an irreducible maximal solvable  subgroup $S$ with $b_S(S \cdot SU_n(q))>3.$ 
By Theorem \ref{sch}, $S$ is not quasi-primitive, so  $S$ has a normal subgroup $L$ such that $V$ is not homogeneous as $\mathbb{F}_q[L]$-module. Therefore, $S$ and $L$ satisfy the conditions of Lemma \ref{ashb}. So $S$ stabilises a decomposition
\begin{equation}\label{decirr}
V=V_1 \oplus \ldots \oplus V_k, \text{ } k \ge 2
\end{equation} 
such that $(1)$ or $(2)$ of Lemma \ref{ashb} holds. % Consider all decompositions of $V$ stabilised by $S$ such that $(1)$ or $(2)$ of Lemma \ref{ashb} holds. 
  Let us fix \eqref{decirr} to be such a decomposition with  the largest possible $k.$  

If $(1)$ of Lemma \ref{ashb} holds, then consider $S_1:= \Stab_S(V_1)|_{_{V_1}} \le GU(V_1).$  By Clifford's  Theorem $S_1$ acts irreducibly on $V_1.$ Notice that we can assume $S_1$ to be quasi-primitive in that case. Indeed, if $S_1$ is not quasi-primitive, then $S_1$ stabilises a decomposition  
$$V_1=V_{11} \oplus \ldots \oplus V_{1t}$$ for some $t \ge 2$ such that  $(1)$ or $(2)$ of Lemma \ref{ashb} holds. Therefore, since $S$ is irreducible, it stabilises the  decomposition 
$$V=V_{11} \oplus \ldots \oplus V_{1t} \oplus \ldots \oplus V_{k1} \oplus \ldots \oplus V_{kt},$$
for which  $(1)$ or $(2)$ of Lemma \ref{ashb} holds contradicting the maximality of $k$ in \eqref{decirr}.

  If $\dim V_1=1$ then $S$ can be represented as a group of monomial matrices with respect to an orthonormal basis. By Lemma \ref{omnom}, this is possible if and only if $(n,q)=(4,2)$ and $S=MU_4(q).$
Assume $\dim V_1=m \ge 2.$ If $q\in \{2,3\}$ and $S$ is conjugate to a subgroup of one of the groups listed in Lemma \ref{isklwr}, then we obtain a contradiction. If $S$ is not conjugate to a subgroup of a group from Lemma \ref{isklwr}, then $GU_m(q)$ is not solvable. Therefore, $S_1\le GU(V_1)$ satisfies the condition of the theorem  
 and $b_{S_1}(S_1 \cdot SU(V_1)) \le 3$ since $(n,S)$ is a minimal counterexample. Thus, there exist $a,b \in SU(V_1)$ such that $S_1 \cap S_1^a \cap S_1^b \le Z(GU(V_1)).$ Applying Theorem \ref{uniind} and Lemma \ref{omnom} we obtain 
$b_S(S \cdot SU_n(q))\le 3.$ %This is a contradiction to the existence of counterexample.

Finally, let us assume part $(2)$ of  Lemma \ref{ashb} holds. Let $U_i=V_{2i-1} \oplus V_{2i}$, so 
$$V=U_1 \bot \ldots \bot U_{k/2}$$
 and $S$ transitively permutes the $U_i$. Indeed, since $S$ acts on $V$ by isometries, $(V_{2i-1})g$ and $(V_{2i})g$ cannot be mutually orthogonal for $g \in S$, which is possible if and only if      $(V_{2i-1})g$ and $(V_{2i})g$ lie in the same $U_j$ for some $j =1, \ldots, k/2.$ Transitivity follows from the irreducibility of $S.$ Consider $S_1:= \Stab_S(U_1)|_{_{U_1}}.$ 
Notice that $\dim U_1 = 2m \ge 2.$ 

If $(2m, q) \in \{(2,2), (2,3)\}$, then, since $S$ is a maximal solvable subgroup of $GU(V),$ $S$ must be conjugate to $GU_{2m}(q) \wr \Gamma$ with $\Gamma \le \Sym(k/2)$. In this case the theorem follows by Lemma \ref{isklwr}. Otherwise $GU(U_1)$ is not solvable. If $k>8$, then  $S_1\le GU(U_1)$ satisfies the condition of the theorem 
 and $b_{S_1}(GU(U_1)) \le 3$ since $(n,S)$ is a minimal counterexample. Thus, there exist $a,b \in SU(U_1)$ such that $S_1 \cap S_1^a \cap S_1^b \le Z(GU(U_1))$. Applying Theorem \ref{uniind} and Lemma \ref{omnom} we obtain 
$b_S(S \cdot SU_n(q))\le 3.$
Let $k \le 8$. Consider $P:= \Stab_S(V_1)|_{_{V_1}} \le GL(V_1),$ so $P$ is an irreducible solvable  subgroup of $GL(V_1)$ and  there exist $a,b \in SL(V_1)$ such that $P \cap P^a \cap P^b \le Z(GL(V_1))$ by Theorem \ref{irred}. Applying Lemma \ref{m2isotr} 
we obtain $b_S(S \cdot SU(V))\le 3,$ which contradicts the assumption.
%Since $U_i$ are non-degenerate, $S_1 \le GU(U_1)$ satisfies the conditions of the theorem. If $k/2>1$ then $\dim U_i < n$ and we obtain a contradiction in the same way as in the case $(1).$ Therefore $k=2$ and $U_1=V$ and the existence of a counterexample contradicts to   Lemmas \ref{irred} and  \ref{m2isotr}.
\end{proof}

\begin{Th}\label{sympirr}
Let $n \ge 2$. If $S \le GSp_n(q)$ is an irreducible maximal solvable  subgroup, then one of the following holds:
\begin{enumerate}[font=\normalfont]
\item  there exist $x,y \in Sp_n(q)$ such that $S \cap S^x \cap S^y \le Z(GSp_n(q))$; \label{sympirr1}
\item \label{sympirr2} $n=4$, $q \in \{2,3\}$, and $S$ is the stabiliser of decomposition $V =V_1 \bot V_2$ with $V_i$ non-degenerate and there 
exist $x,y,z \in Sp_n(q)$ such that $S \cap S^x \cap S^y \cap S^z \le Z(GSp_n(q))$; 
\item $n=2$, $q \in \{2,3\},$ and $S=GSp_2(q).$ \label{sympirr3}
\end{enumerate}
\end{Th}
\begin{proof}
The following is verified by computation: if $n=4$ and $q \in \{2,3\},$ then either $b_S(S \cdot Sp_n(q)) \le 3$ or $S$ is as in \eqref{sympirr2}.

Assume that $n$ is minimal such that there exists a counterexample to the theorem: namely,  $(S,n,q)$  is such that  $b_S(Sp_n(q))>3$ and neither  \eqref{sympirr2} nor \eqref{sympirr3}  hold.

If $S$ is quasi-primitive, then it is not a counterexample by Theorem \ref{sch} and Section \ref{sec5}.

 Assume that $S$ is not quasi-primitive, so $S$ has a normal subgroup $L$ such that $V$ is not $\mathbb{F}_q[L]$-homogeneous by Lemma \ref{ashb}. Therefore, $S$ stabilises a decomposition 
\begin{equation}
\label{BCvisp}
V=V_1 \oplus \ldots \oplus V_k
\end{equation}
such that $\dim V_i=m$ for $i \in \{1, \ldots, k\},$ $k>1$ and one of the following holds:
\begin{itemize}
\item[{\bf Case 1.}] $V=V_1 \bot \ldots \bot V_k$ with $V_i$ non-degenerate for $i=1, \ldots, k;$
\item[{\bf Case 2.}] $V=U_1 \bot \ldots \bot U_{k/2}$ with $U_i=V_{2i-1} \oplus V_{2i}$ non-degenerate and $V_i$ totally isotropic. 
\end{itemize} Let us fix \eqref{BCvisp} to be such a decomposition with
the largest possible $k$.   {\bf Case 1} splits into two subcases: $k \ge 3$ and $k=2$. 

\medskip

{\bf Case (1.1).} Assume that the $V_i$ are non-degenerate and $k\ge 3$ in \eqref{BCvisp}. Let $H$ be $\Stab_S(V_1)|_{_{V_1}} \le GSp_m(q).$ Notice that $H$ is an irreducible maximal solvable  subgroup of $GSp_m(q).$ If $H$ is as $S$ in \eqref{sympirr2}, then $V_i = V_{i1} \bot V_{i2}$ with non-degenerate $V_{ij}$ and $S$ stabilises the decomposition $$V=(V_{11} \bot V_{12}) \bot \ldots \bot (V_{k1} \bot V_{k2}).$$   Since $m<n$ and $H$ is not a counterexample, we can assume that either there exist $x_1, x_2 \in Sp_m(q)$ such that $H \cap H^{x_1} \cap H^{x_2} \le Z(GSp_m(q))$ or  $H=GSp_2(q)$ with $q \in \{2,3\}$. In the latter case we take $x_1=x_2=I_2.$ 

Let $\{v_1, u_1, \ldots, v_k, u_k\}$ be a basis of a $2k$-dimensional vector space over $\mathbb{F}_q.$ Let $y_1$ and $z_1$ be the matrices of the linear transformations of this space defined by the formulae:
\begin{equation*}
\begin{split}
(v_i)y_1 &=v_i - v_{i+1} \text{ for } i \in \{1, \ldots, k-1\};\\
(v_k)y_1 &= v_k;\\
(u_i)y_1 &= \sum_{j=1}^i u_i \text{ for } i \in \{1, \ldots, k\};
\end{split}
\end{equation*}
and  
\begin{align*}
(v_1)z_1 &=v_1; &(u_1)z_1 &= u_1+v_2;\\
(v_i)z_1 &=v_i - v_{i+1} \text{ for } i \in \{1, \ldots, k-2\}; &(u_i)z_1 &= \sum_{j=1}^i u_i \text{ for } i \in \{2, \ldots, k-1\};\\
(v_{k-1})z_1 &=v_1 + u_{k}; &(u_k)z_1 &= u_k;\\
(v_k)z_1 &= v_1 + v_k + \sum_2^{k-1}u_i.
\end{align*} 
For example, if $k=4$, then 
\begin{equation*} y_1=
\begin{pmatrix}
1      & \multicolumn{1}{c|}{0}& -1& \multicolumn{1}{c|}{0}&0  &\multicolumn{1}{c|}{0}&0&0   \\
0      & \multicolumn{1}{c|}{1}& 0 & \multicolumn{1}{c|}{0}&0  &\multicolumn{1}{c|}{0}&0&0 \\ \cline{1-8}
0      & \multicolumn{1}{c|}{0}& 1 & \multicolumn{1}{c|}{0}& -1&\multicolumn{1}{c|}{0}&0&0   \\
0      & \multicolumn{1}{c|}{1}& 0 & \multicolumn{1}{c|}{1}& 0 &\multicolumn{1}{c|}{0}&0&0   \\\cline{1-8}
0      & \multicolumn{1}{c|}{0}& 0 & \multicolumn{1}{c|}{0}& 1 &\multicolumn{1}{c|}{0}&-1&0 \\       
0      & \multicolumn{1}{c|}{1}& 0 & \multicolumn{1}{c|}{1}& 0 &\multicolumn{1}{c|}{1}&0&0   \\\cline{1-8}
0      & \multicolumn{1}{c|}{0}& 0 & \multicolumn{1}{c|}{0}& 0 &\multicolumn{1}{c|}{0}& 1 &        0\\
0      & \multicolumn{1}{c|}{1}& 0 & \multicolumn{1}{c|}{1}& 0 &\multicolumn{1}{c|}{1}& 0 &        1
\end{pmatrix}, \text{ }
z_1=
\begin{pmatrix}
1      & \multicolumn{1}{c|}{0}& 0 & \multicolumn{1}{c|}{0}&0  &\multicolumn{1}{c|}{0}&0&0   \\
0      & \multicolumn{1}{c|}{1}& 1 & \multicolumn{1}{c|}{0}&0  &\multicolumn{1}{c|}{0}&0&0 \\ \cline{1-8}
0      & \multicolumn{1}{c|}{0}& 1 & \multicolumn{1}{c|}{0}& -1&\multicolumn{1}{c|}{0}&0&0   \\
1      & \multicolumn{1}{c|}{0}& 0 & \multicolumn{1}{c|}{1}& 0 &\multicolumn{1}{c|}{0}&0&0   \\\cline{1-8}
0      & \multicolumn{1}{c|}{0}& 0 & \multicolumn{1}{c|}{0}& 1 &\multicolumn{1}{c|}{0}&0&1 \\       
1      & \multicolumn{1}{c|}{0}& 0 & \multicolumn{1}{c|}{1}& 0 &\multicolumn{1}{c|}{1}&0&0   \\\cline{1-8}
1      & \multicolumn{1}{c|}{0}& 0 & \multicolumn{1}{c|}{1}& 0 &\multicolumn{1}{c|}{1}& 1 &        0\\
0      & \multicolumn{1}{c|}{0}& 0 & \multicolumn{1}{c|}{0}& 0 &\multicolumn{1}{c|}{0}& 0 &        1
\end{pmatrix}.
\end{equation*}
Let $\beta_i$ be a basis of $V_i$ of shape \eqref{sympbasis} for $i \in \{1, \ldots, k\}$ and let $\beta$ be $\beta_1 \cup \ldots \cup \beta_k$. Let $y=I_{m/2} \otimes y_1$ and $z=I_{m/2} \otimes z_1$.  It is routine to check that $y,z \in Sp_n(q, {\bf f}_{\beta}).$

Let $W_i=V_i(x_1 \otimes I_k)y$ for $i \in \{1, \ldots, k\}.$ Consider $g \in S \cap S^{(x_1 \otimes I_k)y},$ so $g$ stabilises decompositions $V=V_1 \bot \ldots \bot V_k$ and $V=W_1 \bot \ldots \bot W_k.$ Notice that $W_i$ has non-zero projection on exactly $i+1$ of the $V_j$ for $i \in \{1, \ldots, k-2\}.$ So $g$ cannot map such $W_i$ to others and, therefore, $g$ stabilises subspaces $W_1, \ldots, W_{k-2}$ and $\{W_{k-1}, W_k\}.$ Thus, $g$ stabilises $\{V_1, V_2\}$ and $V_3, \ldots, V_k.$ Notice that $\dim(V_1 \cap W_1)=m/2$ and $\dim(V_2 \cap W_1)=0.$ So $(V_1)g=V_1$ since $(V_1 \cap W_1)g=(V_1)g \cap (W_1)g=(V_1)g \cap W_1 \ne \{0\}.$ The same argument for $V_n \cap W_n$ shows that $(W_n)g=W_n$. Hence $g$ stabilises all $V_i$ and $W_i$ for $i \in \{1, \ldots, k\}$; in particular $$g= \diag[g_1, \ldots, g_k]$$
for $g_i \in H.$

Now let us show that  $g_i \in H \cap H^{x_1}$ for all $i \in \{1, \ldots, k\}.$ Since $g \in S \cap S^{(x_1 \otimes I_k)y},$ $g=h^{(x_1 \otimes I_k)y},$ where $h \in S \cap S^{((x_1 \otimes I_k)y)^{-1}}.$ The same arguments as above show that $h = \diag[h_1, \ldots, h_k]$ with $h_i \in H.$ Denote $h_i^{x_1}$ by $\hat{h}_i$ and $h^{x_1 \otimes I_k}=\diag[\hat{h}_1, \ldots, \hat{h}_k]$ by $\hat{h}$, so $g= \hat{h}^y.$ Let $\hat{h}_i = 
\left( \begin{smallmatrix}
h_{(i,1)} & h_{(i,2)}\\
h_{(i,3)} & h_{(i,4)}
\end{smallmatrix} \right),
$ where $h_{(i,j)} \in GL_{m/2}(q).$

Consider the last $(m \times m)$-row of $g=\hat{h}^y.$ Calculations show that it is $$(0, \ldots, 0, g_k)=(A_1, A_2, \ldots, A_k)$$ with 
{\small
\begin{equation*}
\begin{aligned}
A_i= & \begin{pmatrix} 0 & h_{(k,2)} \\ 0& h_{(k,4)}-h_{(k-1,4)} \end{pmatrix} & \text{ for } i \in \{1, \ldots, k-2\}, \\
 A_{k-1}= &  
 \begin{pmatrix} 0 & h_{(k,2)} \\ -h_{(k-1,3)}& h_{(k,3)}-h_{(k-1,4)} \end{pmatrix},& \\
 A_k = & \begin{pmatrix} h_{(k,1)} & h_{(k,2)} \\ h_{(k,3)}-h_{(k-1,3)}& h_{(k,4)} \end{pmatrix}.&
\end{aligned}
\end{equation*}
}
%{\small
%\begin{equation*} \left( 
%\begin{aligned}
%\begin{pmatrix} 0 & h_{(k,2)} \\ 0& h_{(k,4)}-h_{(k-1,4)} \end{pmatrix}, \ldots, \begin{pmatrix} 0 & h_{(k,2)} \\ 0& h_{(k,4)}-h_{(k-1,4)} \end{pmatrix}, %& \\ 
% \begin{pmatrix} 0 & h_{(k,2)} \\ -h_{(k-1,3)}& h_{(k,3)}-h_{(k-1,4)} \end{pmatrix}, \begin{pmatrix} h_{(k,1)} & h_{(k,2)} \\ h_{(k,3)}-h_{(k-1,3)}& h_{(k,4)} \end{pmatrix}
%\end{aligned}
% \right).
%\end{equation*}
%}
 So, $h_{(k-1,4)}=h_{(k,4)}$; $h_{(k,2)}=h_{(k-1,3)}=0$ and $\hat{h}_k=g_k.$
Consider the $(k-1)$-th $(m \times m)$-row of $g=\hat{h}^y.$ As above, we obtain
\begin{align*} h_{(k-2,4)}&=h_{(k-1,4)};\\ h_{(k-2,1)}&=h_{(k-1,1)};\\ h_{(k-2,3)}&=h_{(k-1,2)}=0
\end{align*} and $\hat{h}_{k-1}=g_{k-1}.$

Continuing in the same way we obtain for all $i,j \in \{1, \ldots, k\}:$
\begin{equation}\label{hhati4}
\begin{split}
\hat{h}_i&=g_i;\\
h_{(i,1)}&=h_{(j,1)};\\
h_{(i,4)}&=h_{(j,4)}.
\end{split}
\end{equation} 
Also $$
\hat{h}_1 = \begin{pmatrix} h_{(1,1)} & h_{(1,2)} \\ 0& h_{(1,4)}\end{pmatrix}; \text{ }
\hat{h}_k = \begin{pmatrix} h_{(k,1)} & 0 \\ h_{(k,3)} & h_{(k,4)}\end{pmatrix}$$ { and }
$$\hat{h}_i = \begin{pmatrix} h_{(i,1)} & 0 \\ 0 & h_{(i,4)}\end{pmatrix} \text{ for } 1<i<k. 
$$ Hence $g_i \in H \cap H^{x_1}.$ 

Assume now that $g \in S \cap S^{(x_2 \otimes I_k)z},$ $g=t^{(x_2 \otimes I_k)z},$ where $t \in S \cap S^{((x_2 \otimes I_k)z)^{-1}}.$ Using similar arguments to above, we obtain 
$$g=\diag [g_1, \ldots, g_k]= \diag[\hat{t}_1, \ldots, \hat{t}_k],$$
where $\hat{t}_i \in H^{x_2}$ is defined analogously to $\hat{h}_i \in H^{x_1}.$ In addition,
\begin{equation}\label{thati1}
 t_{(k-1,4)}=t_{(k,1)} \text{ and } t_{(k,3)}=t_{(1,2)}=0.
\end{equation}


Therefore, if $g \in S \cap S^{(x_1 \otimes I_k)y} \cap S^{(x_2 \otimes I_k)z},$ then, by \eqref{hhati4} and \eqref{thati1},  $g = \diag[g_1, \ldots, g_k]$ and $g_i= \diag[\delta,\delta] \in  H \cap H^{x_1} \cap H^{x_2}$  with $\delta \in GL_{m/2}(q)$ for all $ i \in \{1, \ldots, k\}.$ If $m=2$, then $g$ is scalar, so
$S \cap S^{(x_1 \otimes I_k)y} \cap S^{(x_2 \otimes I_k)z}\le Z(GSp_n(q))$.
If $m>2$, then $H$ is not a counterexample, so  $g_i \in H \cap H^{x_1} \cap H^{x_2} \le Z(GSp_m(q))$ and $g \in Z(GSp_n(q, {\bf f}_{\beta})).$

\medskip

{\bf Case (1.2).} Assume that the $V_i$ are non-degenerate and $k= 2$ in \eqref{BCvisp}. Let $H$ be $\Stab_S(V_1)|_{_{V_1}} \le GSp_m(q).$ Thus, either $H$ is not a counterexample, so there exist $x_1, x_2 \in Sp_m(q)$ such that $H \cap H^{x_1} \cap H^{x_2} \le Z(GSp_m(q))$, or  $H=GSp_2(q)$ with $q \in \{2,3\}$. The latter was discussed at the beginning of the proof. Assume the former holds.   
Let 
\begin{equation*} y= I_{m/2} \otimes
\begin{pmatrix}
1      & \multicolumn{1}{c|}{0}& -1& 0   \\
0      & \multicolumn{1}{c|}{1}& 0 & 0 \\ \cline{1-4}
0      & \multicolumn{1}{c|}{0}& 1 & 0   \\
0      & \multicolumn{1}{c|}{1}& 0 & 1
\end{pmatrix}, \text{ }
z= I_{m/2} \otimes
\begin{pmatrix}
1      & \multicolumn{1}{c|}{0}& 0 & 0   \\
0      & \multicolumn{1}{c|}{1}& 1 & 0 \\ \cline{1-4}
0      & \multicolumn{1}{c|}{0}& 1 & 0   \\
1      & \multicolumn{1}{c|}{0}& 0 & 1
\end{pmatrix}.
\end{equation*}
It is routine to check that $y,z \in Sp_n(q, {\bf f}_{\beta}).$
Denote $(V_i)y$ by $W_i$ and $(V_i)z$ by $U_i$ for $i=1,2.$ We claim that if $g \in S \cap S^{(x_1 \otimes I_2)y} \cap S^{(x_2 \otimes I_2)z}$, then $g$ stabilises $V_i$, $i=1,2.$ Assume the opposite, so $(V_1)g=(V_2).$ Therefore, $(W_1)g=W_2$ and $(U_1)g=U_2.$ Thus, 
$$(V_1 \cap W_1)g= (V_1)g \cap (W_1)g= (V_2 \cap W_2)$$ 
and 
$$(V_1 \cap U_1)g= (V_1)g \cap (U_1)g= (V_2 \cap U_2).$$
 Notice that $(V_2 \cap W_2)=(V_2 \cap U_2)$ but $(V_1 \cap W_1) \ne (V_1 \cap U_1)$ which is a contradiction since $g$ is invertible. Therefore, $g= \diag[g_1, g_2]$ where $g_i \in H.$  Also, $g=h^y$ where $h \in S^{y^{-1}} \cap S^{(x_1 \otimes I_2)}$ and $g=t^z$ where $t \in S^{z^{-1}} \cap S^{(x_2 \otimes I_2)}$. It is routine to check, using arguments as above, that $h=\diag[h_1, h_2]$ with $h_i \in H^{x_1}$ and $t=\diag[t_1,t_2]$ with $t_i \in H^{x_2}.$

Now calculations as in {\bf Case (1.1)} show that 
$$
g=
\begin{pmatrix}
h_{(1,1)}      & \multicolumn{1}{c|}{h_{(1,2)}}& 0& 0   \\
0      & \multicolumn{1}{c|}{h_{(1,4)}}& 0 & 0 \\ \cline{1-4}
0      & \multicolumn{1}{c|}{0}& h_{(2,1)} & 0   \\
0      & \multicolumn{1}{c|}{0}& h_{(2,3)} & h_{(2,4)}
\end{pmatrix}=
\begin{pmatrix}
t_{(1,1)}      & \multicolumn{1}{c|}{0}& 0& 0   \\
t_{(1,3)}      & \multicolumn{1}{c|}{t_{(1,4)}}& 0 & 0 \\ \cline{1-4}
0      & \multicolumn{1}{c|}{0}& t_{(2,1)} & 0   \\
0      & \multicolumn{1}{c|}{0}& t_{(2,3)} & t_{(2,4)}
\end{pmatrix}
$$
for some $h_{(i,j)}, t_{(i,j)} \in GL_{m/2}(q)$  with 
\begin{align*}
h_{(1,1)}=h_{(2,1)};& &t_{(1,1)}=t_{(2,4)}; \\
h_{(1,4)}=h_{(2,4)};& &t_{(1,4)}=t_{(2,1)}.
\end{align*}
So $$
g=
\begin{pmatrix}
g_{(1,1)}      & \multicolumn{1}{c|}{0}& 0& 0   \\
0      & \multicolumn{1}{c|}{g_{(1,1)}}& 0 & 0 \\ \cline{1-4}
0      & \multicolumn{1}{c|}{0}& g_{(1,1)} & 0   \\
0      & \multicolumn{1}{c|}{0}& g_{(2,3)} & g_{(1,1)}
\end{pmatrix}; \text{ }
g_1 =\begin{pmatrix}
g_{(1,1)}      & {0}   \\
0      & {g_{(1,1)}}
\end{pmatrix}; \text{ }
g_2=
\begin{pmatrix}
 g_{(1,1)} & 0   \\
 g_{(2,3)} & g_{(1,1)}
\end{pmatrix}
$$
with $g_1, g_2 \in H \cap H^{x_1} \cap H^{x_2}.$ Since $H \cap H^{x_1} \cap H^{x_2} \le Z(GSp_m(q))$, we obtain $g \in Z(GSp_n(q)).$ 

\medskip

{\bf Case 2.} Assume that the $V_i$ are totally isotropic, so $(2)$ of Lemma \ref{ashb} holds. If $k>2$, then $S$ stabilises the decomposition $V=U_1 \bot \ldots \bot U_{k/2}$ with $U_i$ non-degenerate, so {\bf Case 1} applies.

 Now assume $k=2.$
Let $H$ be $\Stab_S(V_1)|_{_{V_1}} \le GL_m(q).$ Thus, by Theorem \ref{irred}, either $H=GL_2(q)$ with $q \in \{2,3\}$ or  there exist $x_1, x_2 \in GL_m(q)$ such that $H \cap H^{x_1} \cap H^{x_2} \le Z(GL_m(q)).$ In the first case the theorem is verified by computation, so assume that the second case holds.

Fix $\beta$ to be a basis of $V$ as in \eqref{sympbasis} with $\langle f_1, \ldots, f_m \rangle=V_1$ and $\langle e_1, \ldots, e_m \rangle=V_2$. Let $y$ be $I_m \otimes \left( \begin{smallmatrix} 1& -1\\ 0& 1 \end{smallmatrix} \right),$ and let $X_i$ be $\diag[x_i, (x_i^{-1})^{\top}]$ for $i=1,2.$ Notice that $y, X_1, X_2 \in Sp_n(q,{\bf f}_{\beta}).$ Consider $g \in S \cap S^{X_1 y} \cap S^{X_2}.$ By the proof of Lemma \ref{igrekl}, $g=\diag[g_1,g_1]$ with $g_1 \in H \cap H^{x_1} \cap H^{x_2},$ so $g \in Z(GL_n(q)) \cap GSp_n(q)=Z(GSp_n(q)).$
\end{proof}

