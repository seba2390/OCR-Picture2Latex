\chapter{The general case}
\label{ch3}
%We prove the main results for linear, unitary and symplectic case in Sections $\ref{ch3}.1$, $\ref{ch3}.2$ and $\ref{ch3}.3$ respectively.
 We prove Theorems (\ref{theorem}-\ref{theoremGR}), \ref{theoremGU} and (\ref{theoremSp}--\ref{theoremSpGR}) in Sections $\ref{ch3}.1$, $\ref{ch3}.2$ and $\ref{ch3}.3$ respectively. If $S \cap \GL_n(q^{\bf u})$ is irreducible, then  their proofs are either consequences of Theorem \ref{bernclass} or of the results obtained in Chapter \ref{ch2}.   So the main obstacle  is the situation when $S\cap \GL_n(q^{\bf u})$ stabilises a non-zero proper subspace of $V$. Our general strategy is to obtain three or four  conjugates of $S$ such that their intersection consists of elements of shape $(\phi_{\beta})^j \diag(\alpha_1, \ldots, \alpha_n)$ for some basis $\beta$ of $V$ (here $\phi_{\beta}$ is as defined in \eqref{defphibet}), and then use a technique similar to that used in the proof of Lemma \ref{diag} to construct another conjugate of $S$ such that the intersection of all of these conjugates consists of scalar matrices. This task is  particularly tricky  when $q \in \{2,3\}$ and  leads to case-by-case considerations. In case ${\bf L}$, since in general $b_S(S \cdot SL_n(q)) \le 5$, we also construct five distinct regular orbits in $\Omega^5$ to show that $\Reg_S(S \cdot SL_n(q),5)\ge 5$.  In cases {\bf U} and {\bf S} (apart from the situation $(n,q)=(5,2)$ verified by computation in Theorem \ref{theoremGU}) we show that $b_S(S \cdot (SL_n(q^{\bf u}) \cap X))\le 4$; Lemma \ref{base4} now implies that $\Reg_S(S \cdot (SL_n(q^{\bf u}) \cap X),5)\ge 5$. 

  

\section{Linear groups}\label{secproof}

We prove Theorems \ref{theorem} and \ref{theoremGR} in Sections \ref{sec411} and \ref{sec412} respectively.

\subsection{Solvable subgroups contained in $\GL_n(q)$}
\label{sec411}

%\begin{Lem}\cite[Lemma 3]{bay} \label{base4}
%If $S \le G$ and $b_S(G)\le 4$ then $\Reg_S(G,5)\ge 5.$
%\end{Lem}
In this section $S$ is a maximal solvable subgroup of $\GL_n(q),$ $G=S \cdot SL_n(q)$, $H=S/Z(GL_n(q))$ and $\overline{G}=G/Z(GL_n(q)).$ Our goal is to prove the following theorem.

\begin{T1}
Let $X=\GL_n(q)$, $n \ge 2$ and $(n,q)$ is neither $(2,2)$ nor $(2,3).$ If $S$ is a maximal solvable subgroup of $X$, 
 then $\Reg_S(S \cdot SL_n(q),5)\ge 5$, in particular $b_S(S \cdot SL_n(q)) \le 5.$
\end{T1}



Before we start the proof, let us discuss the structure of a maximal solvable subgroup $S \le \GL_n(q)$ and fix some notation. 

Let $0< V_1 \le V$ be such that $(V_1)S=V_1$ and $V_1$ has no non-zero proper $S$-invariant subspace. It is easy to see that $S$ acts semilinearly on $V_1$. Let $$\gamma_1:S \to \GL_{n_1}(q),$$ where $n_1=\dim V_1,$ be the homomorphism defined by $\gamma_1:g \mapsto g|_{_{V_1}}.$  Since $V_1$ is $S$-invariant, $S$ acts (semilinearly) on $V/V_1.$ Let $V_1 < V_2 \le V$ be such that $(V_2/V_1)S=V_2/V_1$ and $V_2/V_1$ has no non-zero proper $S$-invariant subspace.  Observe that $S$ acts semilinearly on $V_2/V_1$. Let $\gamma_2:S \to \GL_{n_2}(q)$, where $n_2=\dim (V_2/V_1),$ be the homomorphism defined by $\gamma_2:g \mapsto g|_{_{(V_2/V_1)}}.$ 
Continuing this procedure we obtain the chain of subspaces 
\begin{equation}
\label{Gchain}
0=V_0<V_1< \ldots < V_k=V
\end{equation}
 and a sequence of homomorphisms $\gamma_1, \ldots, \gamma_k$ such that $V_{i}/V_{i-1}$ is $S$-invariant and has no  non-zero proper $S$-invariant  subspaces and $\gamma_i(S) \le \GL_{n_i}(q)$ is the restriction of $S$ to $V_{i}/V_{i-1}$ for $i \in \{1, \ldots, k\}.$ 
 
 
 \begin{Lem}
 \label{GammairGL}
 If $k=1$, then $S \cap GL_n(q)$ lies in an irreducible solvable subgroup of $GL_n(q).$
 \end{Lem}
 \begin{proof}
 Since $k=1,$  $V$ has no  non-zero proper $S$-invariant subspace. Let $M$ be $S \cap GL_n(q).$ Assume that $M$ is reducible, so there exists $0<U_1 <V$ such that $(U_1)M=M$ and $U_1$ is $\mathbb{F}_q[M]$-irreducible. Let $\varphi \in S$ be such that $\varphi M$ is a generator of $S/M.$  Let $U_2$ be $(U_1)\varphi,$ so, for $g \in M$,
$$(U_2)g=(U_1)\varphi g=(U_1)g^{\varphi^{-1}}\varphi=(U_1)\varphi=U_2$$
since $g^{\varphi^{-1}} \in M.$  Thus, $U_2$ is $M$-invariant and $U_1 \cap U_2=\{0\}$ since $(U_1 \cap U_2)M=(U_1 \cap U_2)$, $U_1$ is $\mathbb{F}_q[M]$-irreducible and $U_1 \ne U_2$. Here $U_1 \ne U_2$ since otherwise $V$ has an $S$-invariant non-zero proper subspace and $k>1$. Let $M_i$ be the restriction of $M$ on $U_i$. Since $M_2=M_1^{\varphi},$ $U_2$ is $\mathbb{F}_q[M]$-irreducible. Let $m=\dim U_1$. For $i \in \{1, \ldots, n/m\}$, the same argument shows  that $(U_1) \varphi^i$ is an $\mathbb{F}_q[M]$-irreducible submodule of $V$ and 
$$(U_1 \oplus \ldots \oplus U_{i-1}) \cap U_i=\{0\}.$$ So $M$ stabilises the decomposition
$$V=U_1 \oplus \ldots \oplus U_{n/m}.$$  In particular, $M$ lies in an imprimitive irreducible maximal solvable subgroup of $GL_n(q).$
 \end{proof}
 
 We start the proof of Theorem \ref{theorem} with the case $k=1.$
 
 
\begin{Th}
Theorem {\rm \ref{theorem}} holds for $k=1.$ 
\end{Th}
\begin{proof}
Let $M = S \cap GL_n(q)$. By Lemma \ref{GammairGL}, $M$ lies in an irreducible solvable subgroup of $GL_n(q).$
If $M$ is not a subgroup of one of the groups listed in  \eqref{irred11} -- \eqref{irred15} of Theorem \ref{irred}, then there exists $x \in SL_n(q)$ such that $M \cap M^x \le Z(GL_n(q)).$ So $H \cap H^{\overline{x}}$ is a cyclic subgroup of $\overline{G}$ and by Theorem \ref{zenab} there exists $\overline{y} \in \overline{G}$ such that 
$$(H \cap H^{\overline{x}}) \cap (H \cap H^{\overline{x}})^{\overline{y}}=1.$$ Hence $b_S(G) \le 4$ and $\Reg_S(G,5) \ge 5$ by Lemma \ref{base4}.
If  $M$ is a subgroup of one of the groups in \eqref{irred13} -- \eqref{irred15} of Theorem \ref{irred}, then $S=M$ and $b_S(G) \le 3.$ If  $M$ is a subgroup of one of the groups in \eqref{irred11} -- \eqref{irred12}, then $b_S(G) \le 3$ by \cite[Table 3]{burness} for $q>4$ and by computation for $q=4.$ So $\Reg_S(G,5) \ge 5$ by Lemma \ref{base4}.
\end{proof}


 For the rest of the section we assume that $k>1.$ Our proof for $q=2$ naturally splits into two cases.  If $q=2$ and $S_0$ is the normaliser of a Singer cycle of $GL_3(2)$, then there is no $x, y \in GL_3(2)$ such that $S_0^{x} \cap S_0^{y}$ is contained in $RT(GL_3(2))$ (see Theorem \ref{irred}). So, in Theorem \ref{lem41}, we assume that if $q=2$, then there is no  $i \in \{1, \ldots, k\}$ such that $\gamma_i(S)$ is the normaliser of a Singer cycle of $GL_3(2).$ In Theorem \ref{lem42} we address the case where there exists such an $i$.


\begin{Th}
\label{lem41}
Let $k>1.$ Theorem {\rm \ref{theorem}} holds if
\begin{itemize}
\item $q \ge 3,$ or
\item  $q=2$ and there is no  $i \in \{1, \ldots, k\}$ such that $\gamma_i(S)$ is the normaliser of a Singer cycle of $GL_3(2).$
\end{itemize}
\end{Th} 
\begin{proof}

Since $k>1,$ there exists a nontrivial $S$-invariant subspace $U<V$ of dimension $m<n$. Let $M$ be $S \cap GL_n(q).$ We fix  
\begin{equation}
\label{basGLG}
\beta=\{v_1, v_2, \ldots, v_{n-m+1}, v_{n-m+2}, \ldots, v_n \} 
\end{equation}
such that the last $ \left( \sum_{j=1}^i n_j \right)$  
 vectors in $\beta$ form a basis of $V_i$. So  $g \in M$ has  shape 
\begin{equation}
\label{stupG}
\begin{pmatrix}
\gamma_k(g)     & * & \ldots & * & * \\
0          &    \gamma_{k-1}(g)  & \ldots & * & * \\
     &     & \ddots & \ddots  &    \\
   0    &    \ldots& 0 & \gamma_{2}(g) & * \\
      0    &   \ldots        & \ldots& 0 & \gamma_{1}(g) 
\end{pmatrix}
\end{equation}
 with respect to $\beta.$ Recall that $q=p^f$, so if $f=1,$ then $\GL_n(q)=GL_n(q)$ and $S=M.$ Let  $\phi \in \GL_n(q)$ be such that \begin{equation}
\label{phidef}
\phi : \lambda v_i \mapsto \lambda^p v_i \text{ for all } i \in \{1, \ldots, n\} \text{ and } \lambda \in \mathbb{F}_q.
\end{equation}
% There are cases:\\
%{\bf A.} $n_i \ne 2$ for all $i =1, \ldots , k$ in \eqref{stup};\\
%{\bf B.} $n_i=2$ at least for two distinct $i$;\\
%{\bf C.} $n_i=2$ exactly for one $i$.

Assume that $f>1.$  Let $x= \diag[x_k, \ldots, x_1]$, where $x_i\in SL_{n_i}(q)$ is such that $\gamma_i(M) \cap (\gamma_i(M))^{x_i} \le RT(GL_{n_i}(q))$. Such $x_i$ exist by Theorem \ref{irred} since $\gamma_i(M)$ lies in an irreducible solvable subgroup of $GL_{n_i}(q)$ by Lemma \ref{GammairGL}. Let $y$ be as in the proof of Lemma \ref{irrtog}. Recall that  we defined $\gamma_i$ only on $S$, but it is easy to see that $\gamma_i$ can be extended to $\Stab_{\GL_n(q)}(V_i, V_{i-1})$ since $\gamma_i$ is the restriction on $V_i/V_{i-1}.$

 
  We present the following piece of the proof as a proposition for easy reference. 
\begin{Prop} 
\label{clm44}
There exists $\beta$ as in \eqref{basGLG} such that for every $\varphi \in (S \cap S^x) \cap (S \cap S^x)^y$, 
$$\varphi=\phi^j g \text{ for } j \in \{0,1, \ldots,f-1\}$$
where $g \in GL_n(q)$ is diagonal.
 Moreover, if  $\gamma_i(M) \cap \gamma_i(M)^{x_i} \le Z(GL_{n_i}(q)),$ then $\gamma_i(g)$ is scalar. 
\end{Prop}
\begin{proof}
Consider $\varphi \in S \cap S^x.$ If $n_i=2$ and $\gamma_i(S)$ lies in the normaliser of a Singer cycle of $GL_2(q)$ in $\GL_2(q)$, then by \eqref{2sindiagodd} and \eqref{2sindiageven} we can choose $x_i \in SL_2(q)$ such that $\gamma_i(S) \cap \gamma_i(S)^{x_i}\le RT(GL_2(q))$.  If $\varphi$ acts linearly on $V_i/V_{i-1},$ then $\varphi$ acts linearly on $V$, so $S \cap S^x \le GL_n(q)$.   Otherwise, by Lemma \ref{scfield}, there exists a basis of $V_i/V_{i-1}$ such that $\gamma_i(\varphi)= \phi^{j_i} g_i$ with scalar $g_i$ and $j_i \in \{0,1, \ldots , f-1 \}.$ Hence we can choose $\beta$ in \eqref{basGLG} such that if $\varphi \in S \cap S^x,$ then $\gamma_i(\varphi)= \phi^{j_i} g_i$ with scalar $g_i.$ Since $(\phi^{j_1})^{-1} \varphi$ acts on $V_1\ne 0$ linearly,   $(\phi^{j_1})^{-1} \varphi \in GL_n(q),$ so $j_i=j_l$ for all $i,l \in \{1, \ldots, k\}$
and 
\begin{equation}
\label{phitri}
\varphi = \phi^j 
\begin{pmatrix}
g_k     & * & \ldots & * & * \\
0          &    g_{k-1}  & \ldots & * & * \\
     &     & \ddots & \ddots  &    \\
   0    &    \ldots& 0 & g_{2} & * \\
      0    &   \ldots        & \ldots& 0 & g_{1}
\end{pmatrix}
\end{equation}
 with $g_i \in Z(GL_{n_i}(q))$ if $n_i \ne 2$ and $g_i \in RT(GL_2(q))$ if $n_i=2.$ 

%Let $i \in \{1, \ldots, k\}$ be such that $V_i=\langle v_{j_1},v_{j_1+1}, v_{j_1+2}, \ldots, v_n \rangle$. Since $\varphi \in S \cap S^x$,  $\varphi$ stabilises $V_i$, $(V_i)x$ and $\varphi =\varphi_1^x$ for some $\varphi \in S.$ So
%$(v_{j_1+1})\varphi \in V_1.$ On the other hand 
%$$(v_{j_1+1})\varphi=(v_{j_1+1})x^{-1}\varphi_1 x=(v_{j_t+1})\varphi_1 x,
%$$ so
%$$(v_{j_1+1})\varphi \in \langle v_{j_t},v_{j_t+1}, v_{j_t+2}, \ldots, v_n \rangle x =\langle v_{j_1+1},v_{j_t+1}, v_{j_t+2}, \ldots, v_n \rangle.$$ 
%So
%\begin{equation}
%\label{2diag2}
%(\varphi) \gamma_i= \phi^j 
%\begin{pmatrix}
%\lambda_1 & \lambda_2\\
%0 & \lambda_3
%\end{pmatrix}
%\end{equation}
% for some $j \in \{0, 1, \ldots, f-1\}$ and $\alpha_1, \ldots, \alpha_3 \in \mathbb{F}_q$ since $v_{j_1} \notin \langle v_{j_1+1},v_{j_t+1}, v_{j_t+2}, \ldots, v_n \rangle.$ Similar arguments show that \eqref{2diag2} holds for all $i$ such that $V_i=\langle v_{j_l},v_{j_l+1}, v_{j_l+2}, \ldots, v_n \rangle$ with $l \in \{2, \ldots, t\}.$
%Therefore, \eqref{phitri} holds with  with $g_i \in Z(GL_{n_i}(q))$ if $n_i \ne 2$ and $g_i \in RT(GL_2(q))$ if $n_i=2.$ 


Since $\varphi \in (S \cap S^x)^y,$
$$
\varphi = \phi^j 
\begin{pmatrix}
g_1'     & 0 & \ldots & 0 & 0 \\
*          &    g_{k-1}'  & \ldots & 0 & 0 \\
     &     & \ddots & \ddots  &    \\
   *    &    \ldots& * & g_{2}' & 0 \\
      *    &   \ldots        & \ldots& * & g_{1}'
\end{pmatrix}
$$ with $g_i$ either scalar or lower-triangular. So 
\begin{equation}
\label{phidiagG}
\varphi =\phi^j g \text{ where } g=\diag[g_k, \ldots, g_1]
\end{equation}
 with $g_i$ scalar if $n_i \ne 2$ and $g_i$ diagonal otherwise. 
\end{proof}

\medskip

We now resume our proof of Theorem \ref{lem41}.
Let $\Omega$ be the set of right $S$-cosets in $G$ and let $\overline{\Omega}$ be the set of right $H$-cosets in $\overline{G}$, where the action is given by  right multiplication.  Since $S$ is maximal solvable and $$S_{G}:=\cap_{g \in G} S^g=Z(GL_n(q)),$$ $\Reg_S(G,5)$ is the number of $\overline{G}$-regular orbits on  $\overline{\Omega}^5$.   Therefore, $$\overline{\omega}=(H\overline{g_1},H\overline{g_2},H\overline{g_3},H\overline{g_4},H\overline{g_5}) \in \overline{\Omega}^5$$ is regular under the action of $\overline{G}$ if and only if the stabiliser of $$ \omega =(Sg_1, Sg_2, Sg_3,Sg_4, Sg_5) \in \Omega^5$$ under the action of $G$ is equal to $Z(GL_n(q))$, where $g_i$ is a preimage in $\GL_n(q)$ of $\overline{g_i}.$  
%In addition, we can induce the action of $PGL_n(q)$ on $\Omega$ by 
%$$(Sg) \overline{h}= Sgh,$$
%so, since $S \ge Z(GL_n(q))$, $\omega \in \Omega^5$ is regular under the action of $PGL_n(q)$ if and only if the stabiliser of $\omega$ under the action of $GL_n(q)$ is $Z(GL_n(q)).$ 
In our proof we say that   $\omega \in \Omega^5$ is regular if it is stabilised only by elements from $Z(GL_n(q))$, so $\overline{\omega}$ is regular in terms of Definition \ref{def1} under the induced  action of $\overline{G}.$  

The proof of Theorem \ref{lem41} splits into five cases. To show $\Reg_S(G,5) \ge 5$, in each case we find five regular orbits in $\Omega^5$ or show that $b_S(G)\le 4,$ so $\Reg_S(G,5) \ge 5$ by Lemma \ref{base4}. Recall that $n_i = \dim V_i/V_{i-1}$ for $i \in \{1, \ldots, k\}.$  Different cases arise according to the number of $n_i$ for $i \in \{1, \ldots, k\}$ which equal 2  and in what rows of  $g \in M=S\cap GL_n(q)$ the submatrix $\gamma_i(g)$ is located for $n_i=2.$ We now specify the cases.
\begin{description}[before={\renewcommand\makelabel[1]{\bfseries ##1}}]
\item[{\bf Case 1.}] Either $f>1$, or $f=1$ and the number of $i \in \{1, \ldots, k\}$ with $n_i=2$ is not one;
\item[{\bf Case 2.}] $f=1, $ $n$ is even, there exist exactly one $i \in \{1, \ldots, k\}$ with $n_i=2$ and $\gamma_i(g)$ appears in rows  $\{n/2,n/2+1\}$ of $g \in M$ for such $i$;
\item[{\bf Case 3.}] $f=1, $ $n$ is odd, there exist exactly one $i \in \{1, \ldots, k\}$ with $n_i=2$ and $\gamma_i(g)$ appears in rows  $\{(n+1)/2,(n+1/2)+1\}$ of $g \in M$ for such $i$;
\item[{\bf Case 4.}] $f=1, $ $n$ is odd, there exist exactly one $i \in \{1, \ldots, k\}$ with $n_i=2$ and $\gamma_i(g)$ appears in rows  $\{(n-1)/2,(n+1)/2+1\}$ of $g \in M$ for such $i$;
\item[{\bf Case 5.}] $f=1,$ there exist exactly one $i \in \{1, \ldots, k\}$ with $n_i=2$ and none of {\bf Cases 2 -- 4} holds. 
\end{description}

%In {\bf Case 1} we deal with the situation when either $f>1$, or $f=1$ and the number of $n_i$ equal to $2$ is not one. {\bf Cases 2--5} consider the situation when $f=1$ and there is exactly one $n_i$ such that $n_i=2.$

Before we proceed with the proof of Theorem \ref{lem41}, let us resolve two situations which both arise often in our analysis of these cases and can be readily settled.

\begin{Prop}
\label{ni1}
Let $S$ be a maximal solvable subgroup of $\GL_n(q),$ let $\beta$ be as in \eqref{basGLG},  $n \ge 2$ and $(n,q)$ is neither $(2,2)$ nor $(2,3).$ If  all $V_i$ in \eqref{Gchain} are such that $n_i=1,$ then there exist $x,y,z \in SL_n(q)$ such that $S \cap S^x \cap S^y \cap S^z \le Z(GL_n(q)).$ 
\end{Prop}
\begin{proof}
Let $\sigma =(1, n)(2, n - 1) \ldots ([n/2], [n/2 + 3/2]) \in \Sym(n).$ We define $x,y,z \in SL_n(q)$ as follows:
\begin{itemize}
 \item $x = \diag(\sgn(\sigma), 1\ldots, 1) \cdot \per (\sigma);$ 
\item $(v_i)y=v_i$ for $i \in \{1, \ldots, n-1\}$ and $(v_n)y= \sum_{i=1}^n v_i$;
\item $(v_i)z=v_i$ for $i \in \{1, \ldots, n-1\}$ and $(v_n)z= \theta v_1 + \sum_{i=2}^n v_i$, 
\end{itemize}
where $\theta$ is a generator of $\mathbb{F}_q^*.$ Let $\varphi \in S \cap S^x \cap S^y \cap S^z.$ Since $\varphi \in S \cap S^x,$ it stabilises $\langle v_i \rangle $ for all $i \in \{1, \ldots, n\}.$ So $\varphi =\phi^j \diag(\alpha_1, \ldots, \alpha_n)$ for $j \in \{0,1, \ldots, f-1\}$ and $\alpha_i \in \mathbb{F}_q^*.$ Since $\varphi \in S^y,$ it stabilises $(V_1)y =\langle v_n \rangle y= \langle v_1 + \ldots +v_n \rangle.$ So $\alpha_1=\alpha_i$ for $i \in \{2, \ldots, n\}.$  Since $\varphi \in S^z,$ it stabilises $(V_1)z =\langle v_n \rangle z= \langle \theta v_1 + v_2 + \ldots +v_n \rangle.$ So $\theta^{p^j}=\theta$ and $j=0.$ Hence $\varphi \in Z(GL_n(q)).$ 
\end{proof}

\begin{Prop}
\label{n3GL}
Fix a basis $\beta$ as in \eqref{basGLG}. Let $n \in \{2,3\}$ and $(n,q)$ is neither $(2,2)$ nor $(2,3)$. Let $S$ be a maximal solvable subgroup of $\GL_n(q).$ If $S$ stabilises a  non-zero proper subspace $V_1$ of $V$, then there exist $x,y,z \in SL_n(q)$ such that 
 $$S \cap S^x \cap S^y \cap S^z \le Z(GL_n(q)).$$
\end{Prop}
\begin{proof}
We can assume that $V_1$ has no non-zero proper $S$-invariant subspaces. Let $V_i$ be as in \eqref{Gchain}.

If $n=2$, then $\dim V_1=1$ and $n_1=n_2=1$, so the statement follows by Proposition \ref{ni1} 


Assume $n=3$.  If $k=3,$ so $n_i=1$ for all $i$, then the statement follows by Proposition \ref{ni1}, so we assume $k=2.$

Let $n_1=1,$ so $n_2=2.$ Let $f>1.$ Assume that $\gamma_2(S)$ is a subgroup of the normaliser of a Singer cycle of $GL_2(q)$ in $\GL_2(q).$ Therefore, by \eqref{2sindiagodd} and \eqref{2sindiageven}, there exists $x_1 \in SL_2(q)$ such that $\gamma_2(S) \cap (\gamma_2(S))^{x_1} \le \langle \varphi \rangle Z(GL_2(q))$ where $\varphi = \diag(-1,1)$ if $q$ is odd and $\varphi = \begin{pmatrix}
1 & 0\\
1 & 1
\end{pmatrix}$ if $q$ is even. Hence, if $x=\diag[x_1,1],$ then $S \cap S^x \le GL_3(q)$ and matrices in $S \cap S^x$ have shape 
$$
 \begin{pmatrix}
\alpha_1 \varphi & *\\
0 & \alpha_2
\end{pmatrix}
$$ with $\alpha_1, \alpha_2 \in \mathbb{F}_q^*.$ Let $y=\per((1,2,3))$ and 
$$
z=
\begin{cases}
\begin{pmatrix}
1&0&0\\
0&1&0\\
1&1&1
\end{pmatrix} \text{ if $q$ is odd;}\\
\begin{pmatrix}
0&1&0\\
0&0&1\\
1&0&0
\end{pmatrix} \text{ if $q$ is even.}\\
\end{cases}
$$
Calculations show that $S \cap S^x \cap S^y \cap S^z \le Z(GL_n(q)).$

If $\gamma_2(S)$ does not normalise a Singer cycle of $GL_2(q),$ then, by Theorem \ref{irred}, 
   there exists $x_1 \in SL_2(q)$ such that $\gamma_2(S) \cap (\gamma_2(S))^{x_1} \cap GL_2(q) \le Z(GL_2(q)).$ Let $x=\diag[x_1,1]$ so $\varphi \in S \cap S^x$ has shape \eqref{phitri} with $g_2 \in Z(GL_2(q))$ and $g_1 \in \mathbb{F}_q^*.$ In particular, let 
$$
\varphi =\phi^j
\begin{pmatrix}
\alpha_1 & 0 & \delta_1\\
 0&     \alpha_1 & \delta_2\\
0 & 0& \alpha_2  
\end{pmatrix}.
$$
 Let $y=
\begin{pmatrix}
0&1&0\\
1&0&0\\
\theta&1&1
\end{pmatrix}$ and consider $\varphi \in (S \cap S^x) \cap (S \cap S^x)^y.$ Since $\varphi \in S^y,$ it stabilises $$(V_1)y=\langle v_3 \rangle y= \langle \theta v_1 +v_2 +v_3 \rangle.$$ Therefore,
$$(\theta v_1 +v_2 +v_3)\varphi = \theta^{p^j} \alpha_1 v_1 + \alpha_1 v_2 + (\alpha_2 +\delta_1 +\delta_2)v_3 \in \langle \theta v_1 +v_2 +v_3 \rangle,$$
so $\theta^{p^j} \alpha_1= \theta \alpha_1$ and $j=0$. Thus, $(S \cap S^x) \cap (S \cap S^x)^y \le GL_3(q).$ Now calculations  show that $(S \cap S^x) \cap (S \cap S^x)^y \le Z(GL_3(q)).$

Now assume $f=1$ (we  continue to assume that $n_1=1$). Let $\sigma=(1,3) \in \Sym(3)$ and $x=\diag(\sgn(\sigma),1,1) \cdot {\rm perm}(\sigma) \in SL_3(q)$. Matrices from $S \cap S^x$ have shape  
  \begin{equation*} 
  \begin{pmatrix}
*    & 0 & 0  \\
* &*    & *  \\
  0&    0    &   *      
\end{pmatrix}.
\end{equation*}
Let   
  \begin{equation*} 
y=  \begin{pmatrix}
0    & -1 & 0  \\
1 &0    & 0  \\
  1&    1    &   1      
\end{pmatrix} \in SL_3(q).
\end{equation*}
It is easy to check that $$(S\cap S^x) \cap (S\cap S^x)^y \le Z(GL_3(q)).$$

To complete the proof, we assume  $n_1=2,$ so $n_2=1.$ We consider the cases  $f>1$ and $f=1$ separately. First assume $f>1.$ Assume that $\gamma_1(S)$ is a subgroup of the normaliser of a Singer cycle of $GL_2(q)$ in $\GL_2(q).$ Therefore, by \eqref{2sindiagodd} and \eqref{2sindiageven}, there exists $x_1 \in SL_2(q)$ such that $\gamma_1(S) \cap (\gamma_1(S))^{x_1} \le \langle \varphi \rangle Z(GL_2(q))$ where $\varphi = \diag(-1,1)$ if $q$ is odd and $\varphi = \begin{pmatrix}
1 & 0\\
1 & 1
\end{pmatrix}$ if $q$ is even. Hence, if $x=\diag[1,x_1],$ then $S \cap S^x \le GL_3(q)$ and matrices in $S \cap S^x$ have shape 
$$
 \begin{pmatrix}
\alpha_1 & *\\
0 & \alpha_2 \varphi
\end{pmatrix}
$$ with $\alpha_1, \alpha_2 \in \mathbb{F}_q^*.$ Let 
$$
y=
\begin{cases}
\begin{pmatrix}
0&-1&0\\
1&0&0\\
1&1&1
\end{pmatrix} \text{ if $q$ is odd;}\\
\begin{pmatrix}
0&0&1\\
1&1&1\\
1&0&0
\end{pmatrix} \text{ if $q$ is even.}\\
\end{cases}
$$ Calculations show that $(S \cap S^x) \cap (S \cap S^x)^y \le Z(GL_3(q)).$ 

If $\gamma_1(S)$ does not normalise a Singer cycle of $GL_2(q),$ then, by Theorem \ref{irred}, 
   there exists $x_1 \in SL_2(q)$ such that $\gamma_1(S) \cap (\gamma_1(S))^{x_1} \cap GL_2(q) \le Z(GL_2(q)).$ Let $x=\diag[1,x_1]$ so $\varphi \in S \cap S^x$ has shape \eqref{phitri} with $g_1 \in Z(GL_2(q))$ and $g_2 \in \mathbb{F}_q^*.$ In particular, let 
$$
\varphi = \phi^j
\begin{pmatrix}
\alpha_1 & \delta_1 & \delta_2\\
 0&     \alpha_2 & 0\\
0 & 0& \alpha_2  
\end{pmatrix}.
$$ Notice that $S \cap S^x$ stabilises $\langle v_2\rangle$, $\langle v_3\rangle$ and $\langle v_2+v_3\rangle.$
 Let $y=
\begin{pmatrix}
0&0&-1\\
1&0&0\\
0&1&\theta
\end{pmatrix}$ and consider $\varphi \in (S \cap S^x) \cap (S \cap S^x)^y.$ Since $\varphi \in (S \cap S^x)^y,$ it stabilises $$\langle  v_3 \rangle y= \langle  v_2 + \theta v_3 \rangle.$$ Therefore,
$$(v_2 + \theta v_3)\varphi =  \alpha_2 v_2 + \theta^{p^j} \alpha_2 v_3 \in \langle  v_2 + \theta v_3 \rangle,$$
so $\theta^{p^j} \alpha_2= \theta \alpha_2$ and $j=0$. Thus, $(S \cap S^x) \cap (S \cap S^x)^y \le GL_3(q).$ Now calculations  show that $(S \cap S^x) \cap (S \cap S^x)^y \le Z(GL_3(q)).$
  
Finally, assume $n_1=2$ and $f=1$. Let $x \in SL_3(q)$ be $\diag(-1,1, 1) \per(\sigma)$ with $\sigma=(1,3)$ and let 
  \begin{equation*} 
y=  \begin{pmatrix}
1    & 0 & 0  \\
1 &1    & 1 \\
  0&    0    &   1      
\end{pmatrix} \in SL_3(q).
\end{equation*}
Calculations show that 
\begin{equation*} (S\cap S^x) \cap (S\cap S^x)^y \le Z(GL_3(q)).
 \qedhere
\end{equation*}
\end{proof} 

We now consider {\bf Cases 1} -- {\bf 5}, resolving each in turn via a proposition. 



\begin{Prop}
\label{case1prop}
Theorem $\ref{lem41}$ holds in {\bf Case 1}.
\end{Prop} 
\begin{proof}
In this case, there are four conjugates of $M=S \cap GL_n(q)$ whose intersection lies in $D(GL_n(q)).$ Indeed, if $f>1,$ or $f=1$ and there is no $i \in \{1, \ldots, k\}$ (here $k$ is as in \eqref{Gchain}) such that $n_i=2$, then 
 there exist $x,y \in SL_n(q)$ such that 
$$(M \cap M^x) \cap (M \cap M^x)^y \le D(GL_n(q))$$
by Proposition \ref{clm44}  or Lemma \ref{irred}  and Lemma \ref{irrtog}.


Assume $f=1$ (so $S=M$) and there exists at least two $i \in \{1, \ldots, k\}$ such that $n_i=2.$ Let $t\ge 2$ be the number of such $i$-s and let $j_1, \ldots, j_t \in \{1, \ldots, n\}$ be such that    $(2 \times 2)$ blocks (corresponding to $V_i/V_{i-1}$ of dimension $2$) on the diagonal in matrices of $M$ occur in the rows $$(j_1, j_1+1), (j_2, j_2 +1), \ldots, (j_t, j_t+1).$$ 
Let $\tilde{x}=\diag(\sgn(\sigma), 1, \ldots, 1) \cdot {\rm perm}(\sigma) \in SL_n(q),$ where  $$\sigma=(j_1, j_1+1, j_2, j_2 +1, \ldots, j_t, j_t+1)(j_1, j_1+1).$$  Let $x= \diag[x_k, \ldots, x_1]\tilde{x}$, where $x_i\in SL_{n_i}(q)$ is such that $\gamma_i(M) \cap (\gamma_i(M))^{x_i} \le RT(GL_{n_i}(q))$ if $n_i \ne 2$ and $x_i$ is the identity matrix if $n_i=2.$ Such $x_i$ exist by Lemma \ref{irred}. If $y$ is as in Lemma \ref{irrtog}, then calculations show that  
$$(M \cap M^x) \cap (M \cap M^x)^y \le D(GL_n(q)).$$








%Thus, by Lemma \ref{diag}*, there exists $z \in SL_n(q)$ such that 
%$$(S,Sx,Sy,Sxy,Sz)$$ is a regular point in $\Omega^5$. 

Let $0<U <V$ be an $S$-invariant subspace and $\dim U=m.$ In general, we take $U=V_i$ for some $i<k$ (in most cases it is sufficient to take $U=V_1$). We fix $\beta$ to be as in \eqref{basGLG} such that Proposition \ref{clm44} holds for $f>1.$ So 
$$U = \langle v_{n-m+1}, v_{n-m+2}, \ldots, v_n \rangle.$$ Since in the proof we reorder $\beta,$ let us fix a second notation for  $$v_{n-m+1}, v_{n-m+2}, \ldots, v_n,$$ namely 
$w_1, \ldots, w_m$
respectively. So $U = \langle w_1, \ldots, w_m \rangle.$

  Our proof splits into the following four  subcases:
\begin{description}[before={\renewcommand\makelabel[1]{\bfseries ##1}}]
\item[{\bf Case (1.1)}] $n-m\ge 2 $ and $m \ge 2$;
\item[{\bf Case (1.2)}] $n-m=m=1$;
\item[{\bf Case (1.3)}] $n-m \ge 2$ and $m=1$;
\item[{\bf Case (1.4)}] $n-m=1$  and $m \ge 2$.
\end{description} 



{\bf Case (1.1).} Let $n-m\ge 2 $ and $m \ge 2$, so $n \ge 4$. We claim that there exist $z_i \in SL_n(q)$ for $i=1, \ldots, 5$ such that the points $\omega_i=(S,Sx,Sy,Sxy,Sz_i)$ lie in distinct regular orbits. 

 Recall that, if either $f=1$ or  $f>1$ and  $\gamma_i(S)$ normalises a Singer cycle of $GL_2(q)$ for some $i \in \{1, \ldots, k\}$, then $$(S \cap S^x) \cap (S \cap S^x)^y \le GL_n(q).$$ Also, we can assume that $n_l \ge 2$ for some $l \in \{1, \ldots, k\}$ in \eqref{Gchain}: otherwise Theorem~\ref{lem41} follows by Proposition \ref{ni1}. Let $s \in \{1, \ldots, n\}$ be such that $$V_l = \langle v_s, v_{s+1}, \ldots, v_{s+n_l-1}, V_{l-1} \rangle.$$

 If $(S \cap S^x) \cap (S \cap S^x)^y$ does not lie in $GL_n(q)$, then   $\varphi \in (S \cap S^x) \cap (S \cap S^x)^y$ stabilises $W=\langle v_s, v_{s+1}, \ldots, v_{s+n_l-1} \rangle$ and the restriction $\varphi|_{_{W}}$ is $\gamma_l(\varphi)=\phi^j g_l$ where $g_l =\lambda I_{n_l}$ for some $\lambda \in \mathbb{F}_q^*$ by \eqref{phidiagG}. We now relabel vectors in $\beta$ as follows: $v_s, v_{s+1}, \ldots, v_{s+n_l-1}$ becomes $v_1, v_{2}, \ldots, v_{n_l}$ respectively; $v_1, v_{2}, \ldots, v_{n_l}$  becomes $v_s, v_{s+1}, \ldots, v_{s+n_l-1}$  respectively;  the remaining labels are unchanged.



  Let $z_1, \ldots, z_5 \in SL_n(q)$ be such that $$(w_{j})z_i=u_{(i,j)}$$ for $i=1, \ldots, 5$ and $j=1, \ldots, m,$ where

\begingroup
\allowdisplaybreaks
\begin{align*}
& \left\{ 
\begin{aligned}
u_{(1,1)} & =\theta v_1+v_2;\\
u_{(1,2)} & =\theta v_1+v_3+v_4;\\
u_{(1,2+r)} & =\theta v_1+v_{4+r}\phantom{;} \text{ for } r \in \{1, \ldots, m-3\};\\
u_{(1,m)} & =\theta v_1 +\sum_{r=2+m}^{n}v_r;
\end{aligned}
\right. \\
& \left\{ 
\begin{aligned}
u_{(2,1)} & =\theta v_1+v_3;\\
u_{(2,2)} & =\theta v_1+v_2+v_4;\\
u_{(2,2+r)} & =\theta v_1+v_{4+r}\phantom{;} \text{ for } r \in \{1, \ldots, m-3\};\\
u_{(2,m)} & =\theta v_1 +\sum_{r=2+m}^{n}v_r;\\
\end{aligned}
\right. \\
& \left\{ 
\begin{aligned}
u_{(3,1)} & =\theta v_1+v_4;\\
u_{(3,2)} & =\theta v_1+v_2+v_3;\\
u_{(3,2+r)} & =\theta v_1+v_{4+r}\phantom{;} \text{ for } r \in \{1, \ldots, m-3\}; \\
u_{(3,m)} & =\theta v_1 +\sum_{r=2+m}^{n}v_r;
\end{aligned}
\right. \stepcounter{equation}\tag{\theequation} \label{orb}  \\
& \left\{ 
\begin{aligned}
u_{(4,1)} & =v_2+v_3;\\
u_{(4,2)} & =\theta v_1+v_2+v_4;\\
u_{(4,2+r)} & =v_2+v_{4+r}\phantom{;} \text{ for } r \in \{1, \ldots, m-3\};\\
u_{(4,m)} & =v_2 +\sum_{r=2+m}^{n}v_r;
\end{aligned}
\right. \\
& \left\{ 
\begin{aligned}
u_{(5,1)} & =v_2+v_4;\\
u_{(5,2)} & =\theta v_1+v_2+v_3;\\
u_{(5,2+r)} & =v_2+v_{4+r}\phantom{;} \text{ for } r \in \{1, \ldots, m-3\};\\
u_{(5,m)} & =v_2 +\sum_{r=2+m}^{n}v_r.
\end{aligned}
\right.  
\end{align*}
\endgroup

Recall that $\theta$ is a generator of $\mathbb{F}_q^*.$ Such $z_i$ always exist since $m<n$ and $u_{(i,1)}, \ldots, u_{(i,m)}$ are linearly independent  for every $i =1 ,\ldots, 5$. 
Consider $\varphi \in (S \cap S^x) \cap (S \cap S^x)^y \cap S^{z_1}$, so $\varphi=\phi^jg$ as in \eqref{phidiagG} with $g=\diag({\alpha_1, \ldots, \alpha_n})$ for $\alpha_i \in \mathbb{F}_q^*.$ Notice that $\varphi$ stabilises the subspace $Uz_1.$ 
Therefore, 
$$(u_{(1,1)})\varphi =
\begin{cases}
 \theta^{p^j} \alpha_1 v_1 + \alpha_2 v_2\\
 \mu_1 u_{(1,1)} + \ldots + \mu_m u_{(1,m)}
\end{cases}
$$ for $\mu_i \in \mathbb{F}_q^*.$ The first line does not contain $v_i$ for $i>2$, so $\mu_i=0$ for $i>1$ and $(u_{(1,1)})\varphi=\mu_1 u_{(1,1)}= \mu_1(\theta v_1 +v_2).$ Therefore, $\mu_1= \alpha_2.$  Assume  that $\varphi \notin GL_n(q)$, so, by the arguments before \eqref{orb}, $\alpha_1 =\alpha_2$,  $\theta^{p^j-1}=1$ and $j=0$ which is a contradiction. Thus, we can assume $\varphi=g \in GL_n(q).$
 The same arguments as in the proof of Lemma \ref{diag} show that $$(u_{(1,j)} )g = \delta_j u_{(1,j)} \text{ for all } j\in \{1,\ldots, m\}$$ for some $\delta_j \in \mathbb{F}_q,$. Since all $u_{(1,j)}$ have $v_1$ in the decomposition \eqref{orb}, all $\delta_j$ are equal and $g$ is a scalar matrix, so $\omega_1$ is a regular point in $\Omega^5.$ It is routine to check that every $\omega_i \in \Omega^5$ is regular. 

Assume that $\omega_1$ and $\omega_2$ lie in the same orbit in $\Omega^5$, so there exists $\varphi \in \GL_n(q)$ such that $\omega_1 \varphi =\omega_2$. This implies $$\varphi \in (S \cap S^x) \cap (S \cap S^x)^y \cap z_1^{-1}Sz_2.$$
In particular, $\varphi= \phi^j g$ with $g =\diag(\alpha_1, \ldots, \alpha_n) \in D(GL_n(q))$ and $g=z_1^{-1}\psi z_2$, $\psi \in S$. Consider $(u_{(1,1)})\varphi$.  Firstly, 
\begin{equation}\label{orr}
(u_{(1,1)})\varphi=(\theta v_1+v_2)\varphi= \theta^{p^j} \alpha_1 v_1 + \alpha_2 v_2;
\end{equation}
on the other hand,
$$(u_{(1,1)})\varphi =(u_{(1,1)})z_1^{-1}\psi z_2=(w_{1})\psi z_2=(\eta_{1} w_{1} + \ldots +\eta_{m} w_{m})z_2 \in Uz_2,$$
for some $\eta_i \in \mathbb{F}_q$. So $(u_{(1,1)})\varphi= \delta u_{(2,2)}$ for some $\delta \in \mathbb{F}_q$ since there are no $v_i$ for $i>2$ in decomposition \eqref{orr}. However, 
$$\theta^{p^j} \alpha_1 v_1 + \alpha_2 v_2=\delta u_{(2,2)}$$
if and only if $(u_{(1,1)})\varphi=0$, which contradicts the fact that $\varphi$ is invertible. Hence  there is no such $\varphi$, so points $\omega_1$ and $\omega_2$ are in  distinct regular orbits on $\Omega^5.$ The same arguments show that all $\omega_i$ lie in  distinct regular orbits on $\Omega^5.$   

\medskip

{\bf Case (1.2).} If $n-m=m=1$, then $n=2$ and  Theorem \ref{lem41} follows by Proposition \ref{n3GL}.
% consists of upper triangular matrices and $S \cap S^x \le D(GL_2(q))$, where $x=\diag(\sgn(\sigma),1) \cdot {\rm per}(\sigma) \in SL_2(q)$ and  $\sigma=(1,2)$. Thus, $$b_S(G)\le 3$$ by Lemma \ref{diag}, so $\Reg_{S}(G,5) \ge 5$ by Lemma \ref{base4}.  

\medskip

{\bf Case (1.3).} Let $m=1$ and $n-m \ge 2$, so $n \ge 3.$ %so $k=2$ in \eqref{stup} and $S_1=Im(\gamma_1)$ is a maximal irreducible solvable subgroup of $GL(n-1,q).$
If $n=3$, then Theorem \ref{lem41} follows by Proposition \ref{n3GL}, so we assume that $m=1$ and $n \ge 4.$ If $k\ge 4$ in \eqref{stup}, then $U = \langle v_{n_1+n_2+1}, \ldots, v_n \rangle$ is an $S$-invariant subspace with $m=n_3 + \ldots + n_k \ge 2$ and $n-m=n_1 +n_2 \ge 2$, so the proof as in {\bf Case (1.1)} using \eqref{orb} works.

 Assume that $k=3.$ If $n_3 \ge 2$, then    $U = \langle v_{n_3+1}, \ldots, v_n \rangle$ is an $S$-invariant subspace with $m=n_1 + n_2 \ge 2$ and $n-m=n_3 \ge 2$, so the proof as in {\bf Case (1.1)} using \eqref{orb} also works.

Let $n_3=1.$ Since $m=1$, $n_1=1.$ If $(n_2,q)=(2,2)$ or $(n_2,q)=(2,3)$, then Theorem~\ref{lem41} is verified by computation. Otherwise, by Theorem \ref{irred} and Lemma \ref{scfield}, there exist $\tilde{x}_1,\tilde{x}_2 \in SL_{n_2} (q)$  such that $$\gamma_2(S) \cap (\gamma_2(S))^{\tilde{x}_1} \cap (\gamma_2(S))^{\tilde{x}_2} \le \langle \phi^i \rangle Z(GL_{n_2}(q))$$ for some $i \in \{0, 1, \ldots, f-1\}$. Let $x_1, x_2 \in GL_n(q)$ be the matrices $\diag[1,\tilde{x}_1,1]$ and $\diag[1,\tilde{x}_2,1]$ respectively.  Let $\tilde{x}= \diag(\sgn(\sigma),1 \ldots, 1) \cdot \per(\sigma) \in SL_n(q)$ with $\sigma=(1,n)$. Calculations show that $(S \cap S^{x_1} \cap S^{x_2 \tilde{x}}) \cap GL_n(q) \le D(GL_n(q)).$ So, with respect to a basis $\beta$ as in \eqref{basGLG},  $\varphi \in S \cap S^{x_1} \cap S^{x_2 \tilde{x}}$ has shape
$$\phi^j \diag(\alpha_1, \alpha_2, \ldots, \alpha_2, \alpha_3)$$ with $j \in \{0,1, \ldots, f-1\}$ and $\alpha_1, \alpha_2, \alpha_3 \in \mathbb{F}_q^*.$ Let 
$$y=
\begin{pmatrix}
1& 0      & 0\ldots  & 0 &0 \\
0& 1      & 0 \ldots  & 0 &0\\
&       &\ddots &  &  &\\
&       & & \ddots &  &\\
0& 0      & 0& \ldots & 1 &0 \\
1& \theta & 1& \ldots & 1 &1
\end{pmatrix}.$$
Calculations show that $S \cap S^{x_1} \cap S^{x_2 \tilde{x}} \cap S^y \le Z(GL_n(q)).$ 

Now let $k=2$, so $n_2=n-m=n-1$ and $n_2>2$. If $(n,q)$ is $(4,2)$ or $(5,3)$, then Theorem \ref{lem41} is verified by  computation.  Otherwise,  by Theorem \ref{irred} and Lemma \ref{scfield},
there exists $x \in SL_n(q)$ such that $S \cap S^x$ consists of elements of  shape (with respect to a basis $\beta$ as in \eqref{basGLG})
\begin{equation*} 
\phi^j
  \begin{pmatrix}
\alpha_1 &0 & \dots    & 0 & \delta_1  \\
0 & \alpha_1 & \dots  & 0 & \delta_2 \\
 &  & \ddots  &  &  \\
 0& \dots & 0  & \alpha_{1} & \delta_{n-1} \\
0 & 0 & \dots  & 0 & \alpha_2 \\
\end{pmatrix}
\end{equation*}
where $\alpha_i, \delta_i \in \mathbb{F}_q$ and $j \in \{0,1, \ldots, f-1\}.$
Let $\sigma =(1, n)(2, n - 1) \ldots ([n/2], [n/2 + 3/2]) \in \Sym(n).$ Let $$y = \per(\sigma) \cdot \diag(\sgn(\sigma), 1, \ldots, 1, \theta, \theta^{-1}).$$ Calculations show that $(S \cap S^x) \cap (S \cap S^x)^y \le Z(GL_n(q)).$
%Let $z \in SL_n(q)$ be such that $(v_i)z=v_{i+1}$ for $1\le i <n-1$, $(v_{n-1})z=v_1$ and $(v_n)z= \sgn(\sigma)( v_1 + \ldots + v_n),$ where $\sigma:=(1,n-1)(2,n-2) \ldots ([n/2],[n/2]+1).$   
%Let $g=z^{-1}hz \in(S\cap S^x) \cap (S\cap S^x)^z$ with $h \in (S\cap S^x).$ For $i \ne n$
%$$(v_i)g = \alpha_i v_{i} + \beta_i v_{n}$$
%since $g \in (S\cap S^x)$ and 
%$$(v_i)g= (v_i)z^{-1}hz =(v_{i-1})hz= (\alpha_{i-1}' v_{i-1} + \beta_{i-1}' v_{n})z=\alpha_{i-1}'v_{i} + \beta_{i-1}'(v_1 +\ldots + v_n).$$ Here  $i-1$ is read modulo $(n-1),$ so if $i=1$, then $i-1=n-1.$ Therefore, $\beta_i=0$ and $g$ is diagonal. Since $g$ stabilises 
%$\langle(v_n)z \rangle= \langle v_1+ \ldots +v_n \rangle$, $g$ is scalar, so
%$$(S\cap S^x) \cap (S\cap S^x)^z \le Z(GL_n(q)).$$
Thus, $b_S(G)\le 4$ and $\Reg_{S}(G,5) \ge 5.$
 

\medskip

{\bf Case (1.4).} Now let $n-m=1$  and $m \ge 2$,  so $n\ge 3.$ If $n=3$, then Theorem \ref{lem41} follows by Proposition \ref{n3GL}, so we assume $n>3.$
If $k>2$, then the proof is  as in  {\bf Case (1.3)}, so we assume $k=2.$
If $(n,q)$ is $(4,2)$ or $(5,3)$, then Theorem \ref{lem41} is verified by computation. 
 Otherwise,  by Theorem \ref{irred} and Lemma \ref{scfield}, there exists $x \in SL_n(q)$ such that $S \cap S^x$ consists of elements of  shape (with respect to a basis $\beta$ as in \eqref{basGLG})
\begin{equation*} 
\phi^j
  \begin{pmatrix}
\alpha_1 &\delta_2 & \dots    & \delta_{n-1} & \delta_n  \\
0 & \alpha_2 & \dots  & 0 & 0 \\
 &  & \ddots  &  &  \\
 0& \dots & 0  & \alpha_{2} & 0 \\
0 & 0 & \dots  & 0 & \alpha_2 \\
\end{pmatrix}
\end{equation*}
where $\alpha_i, \delta_i \in \mathbb{F}_q$ and $j \in \{0,1, \ldots, f-1\}.$
Let $\sigma =(1, n)(2, n - 1) \ldots ([n/2], [n/2 + 3/2]) \in \Sym(n).$ Let $$y = \per(\sigma) \cdot \diag(\sgn(\sigma), 1, \ldots, 1, \theta, \theta^{-1}).$$ Calculations show that $(S \cap S^x) \cap (S \cap S^x)^y \le Z(GL_n(q)).$ Thus, $b_S(G)\le 4$ and $\Reg_{S}(G,5) \ge 5.$ This concludes the proof of Proposition \ref{case1prop}.
\end{proof}

\bigskip

Recall that $f=1$ for {\bf Cases 2 -- 5}, so $\GL_n(q)=GL_n(q)$ and $S=M.$ Therefore, $\beta$ and $\gamma_i$ are as in Lemma \ref{supreduce} and matrices from $S$ have shape \eqref{stup}. Denote $\gamma_i(S)$ by $S_i$.



\begin{Prop}
\label{case2prop}
 Theorem $\ref{lem41}$ holds in {\bf Case 2}.
\end{Prop} 
\begin{proof}


%{\bf Only one $n_i=2$. $q >3$.} 
 Recall that $n$ is even, there exists exactly one $i \in \{1, \ldots, k\}$ with $n_i=2$, and $\gamma_i(g)$ appears in rows  $\{n/2,n/2+1\}$ of $g \in M$ for such $i$.  Let $x= \diag[x_k, \ldots, x_1]$, where $x_i\in SL_{n_i}(q)$ is such that $S_i \cap S_i^{x_i} \le RT(GL_{n_i}(q))$ if $n_i \ne 2$ and $x_i$ is the identity matrix if $n_i=2.$ If $y$ is as in Lemma \ref{irrtog}, then   calculations show that  
$$(S \cap S^x) \cap (S \cap S^x)^y$$ consists of matrices of  shape 
\begin{equation}\label{sered} 
  \begin{pmatrix} 
\alpha_1 &0       & \dots          &              &               &              &\dots  &0  \\
0        & \ddots &                &              &               &              &       &  \\
         &        & \alpha_{n/2-1} &              &               &              &       &   \\
         &        &                &\alpha_{n/2}  &\beta_{n/2}    &              &       &   \\
         &        &                &\beta_{n/2+1} &\alpha_{n/2+1} &              &       &   \\
         &        &                &              &               &\alpha_{n/2+2}&       & \\
         &        &                &              &               &              &\ddots & 0 \\
 0       & \dots  &                &              &               &     \dots    &  0    & \alpha_n 
\end{pmatrix}.
\end{equation}
Assume that $n \ge 8$. We take as the $S$-invariant subspace $U$ the subspace with  basis $\{v_{n/2+2}, \ldots, v_n\}$, so
$m=n/2-1 \ge 3$ and $n-m=n/2+1\ge 5.$  Let us rename some basis vectors for convenience, so denote vectors 
$$v_1, \ldots v_{n/2-1}, v_{n/2+2}, \ldots, v_n$$ by $$w_1, w_2, \ldots, w_{n-2}$$ respectively.
Let  $z_1, \ldots, z_5 \in SL_n(q)$ be such that $$(v_{n-m+j})z_i=u_{(i,j)}$$ for $i=1, \ldots, 5$ and $j=1, \ldots, m,$ where

\begin{gather}\label{orb21}
\left\{ 
\begin{aligned}
u_{(1,1)} & =v_{n/2}+w_1+w_2;\\
u_{(1,2)} & =v_{n/2+1}+w_1+w_3;\\
u_{(1,2+r)} & =w_1+w_{3+r}\phantom{;} &\text{ for } r \in \{1,\ldots, m-3\};\\
u_{(1,m)} & =w_1 +\sum_{r=2+m}^{n-2}w_r;
\end{aligned}
\right.
\end{gather}
\begin{gather}\label{orb212}
\left\{ 
\begin{aligned}
u_{(2,1)} & =v_{n/2}+w_2+w_3;\\
u_{(2,2)} & =v_{n/2+1}+w_2+w_4;\\
u_{(2,2+r)} & =w_2+w_{(2,r)}\phantom{;} &\text{ for } r \in \{1,\ldots, m-3\};\\
u_{(2,m)} & =w_2 +\sum_{r=m-2}^{n-5}w_{(2,r)}.\\
\end{aligned}
\right. 
\end{gather}
Here $w_{(2,1)}, \ldots, w_{(2,n-5)}$ are equal to $w_1, w_5, \ldots, w_{n-2}$ respectively.  
\begin{gather}
\left\{ 
\begin{aligned}
u_{(3,1)} & =v_{n/2}+w_3+w_4;\\
u_{(3,2)} & =v_{n/2+1}+w_3+w_5;\\
u_{(3,2+r)} & =w_3+w_{(3,r)}\phantom{;} &\text{ for } r \in \{1, \ldots, m-3\};\\
u_{(3,m)} & =w_3 +\sum_{r=m-2}^{n-5}w_{(3,r)}.
\end{aligned}
\right.
\end{gather}
Here $w_{(3,1)}, \ldots, w_{(3,n-5)}$ are equal to $w_1, w_2,  w_6, \ldots, w_{n-2}$ respectively.  
\begin{gather} 
\left\{ 
\begin{aligned}\label{orb22}
u_{(4,1)} & =v_{n/2}+w_4+w_5;\\
u_{(4,2)} & =v_{n/2+1}+w_4+w_6;\\
u_{(4,2+r)} & =w_4+w_{(4,r)}\phantom{;} &\text{ for } r \in \{1, \ldots, m-3\};\\
u_{(4,m)} & =w_4 +\sum_{r=m-2}^{n-5}w_{(4,r)}.
\end{aligned}
\right. 
\end{gather}
Here $w_{(4,1)}, \ldots, w_{(4,n-5)}$ are equal to $w_1, w_2, w_3,  w_7, \ldots, w_{n-2}$ respectively.  
\begin{gather} 
\left\{ 
\begin{aligned}\label{orb23}
u_{(5,1)} & =v_{n/2}+w_5+w_6;\\
u_{(5,2)} & =v_{n/2+1}+w_5+w_1;\\
u_{(5,3)} & =w_5+w_{(5,1)}+w_{(5,2)};\\
u_{(5,3+r)} & =w_5+w_{(5,r+2)}\phantom{;} && \text{ for } r \in \{1, \ldots, m-4\};\\
u_{(5,m)} & =w_5 +\sum_{r=m-1}^{n-5}w_{(5,r)}\phantom{;} && \text{ if } m>3;\\
u_{(5,m)} & =w_5 +\sum_{r=m-2}^{n-5}w_{(5,1)}\phantom{;} && \text{ if } m=3.
\end{aligned}
\right. 
\end{gather}
Here $w_{(5,1)}, \ldots, w_{(5,n-5)}$ are equal to $w_2, w_3, w_4, w_7, \ldots, w_{n-2}$ respectively.

 If $m=3$, then there are no  
$u_{(i,2+r)}$ for   $r \in \{1, \ldots, m-3\}$ and $i \in \{1, \ldots, 4\}$; and no $u_{(5,3+r)}$ for   $r \in \{1, \ldots, m-4\}$. Also if $m=3$, we define $u_{(i,3)}$   by $u_{(i,m)}$ in \eqref{orb21} -- \eqref{orb23}.   If $m=4$, then there are no  
$u_{(5,3+r)}$ for   $r \in [1, m-4]$ and $u_{(5,4)}$ is defined by $u_{(5,m)}$ in \eqref{orb23}.

 Let $$\omega_i=(S,Sx,Sy,Sxy,Sz_i).$$ We first show  that the  $\omega_i$ are regular points of $\Omega^5$. Consider $\omega_1$. The regularity of the remaining points can be shown using the same arguments. Let $t \in (S \cap S^x) \cap (S \cap S^x)^y \cap S^{z_1}$, so it takes shape \eqref{sered} for some $\alpha_i, \beta_i \in \mathbb{F}_q$ and stabilises the subspace $Uz_1.$ Thus
\begin{equation}\label{b2s1}
(u_{(1,1)})t=\alpha_{n/2} v_{n/2} + \beta_{n/2} v_{n/2+1} +\alpha_1 w_1 + \alpha_2 w_2,
\end{equation}
since $t \in (S \cap S^x) \cap (S \cap S^x)^y$ and 
\begin{equation}\label{b2s2}
(u_{(1,1)})t=\eta_1 u_{(1,1)} +\eta_2 u_{(1,2)} + \ldots + \eta_m u_{(1,m)},
\end{equation}
since $t$ stabilises  $Uz_1.$ There is no $w_i$ for $i \ge 3$ in decomposition \eqref{b2s1}, so in \eqref{b2s2} we obtain
$$(u_{(1,1)})t=\eta_1 u_{(1,1)}=\alpha_1 u_{(1,1)},$$
in particular $\beta_{n/2}=0$. The same arguments show that 
$$(u_{(1,i)})t=\alpha_1 u_{(1,i)},$$
for the remaining $i \in \{1, \ldots , m\},$ so $t$ is scalar. Therefore, $\omega_1$ is regular.

Now we claim that the $\omega_i$ lie in distinct orbits of $\Omega^5.$ Here we prove that $\omega_1$ does not lie in the orbits containing $\omega_2$ or $\omega_5$; the remaining cases are similar.  First assume that $\omega_1 g =\omega_2$ for $g \in GL_n(q),$ so 
$$g \in (S \cap S^x) \cap (S \cap S^x)^y \cap z_1^{-1}Sz_2.$$
Therefore, $g$ has shape \eqref{sered} and $g=z_1^{-1}hz_2$ for $h \in S,$ so 
\begin{equation}\label{b2s3}
(u_{(1,1)})g=\alpha_{n/2} v_{n/2} + \beta_{n/2} v_{n/2+1} +\alpha_1 w_1 + \alpha_2 w_2,
\end{equation}
and 
\begin{equation}\label{b2s4}
(u_{(1,1)})g=(u_{(1,1)})z_1^{-1}hz_2=(v_{n-m+1})hz_2 = \delta_1 u_{(2,1)} + \ldots + \delta_m u_{(2,m)}.
\end{equation}
At least one of $\alpha_{n/2}$ and $\beta_{n/2}$ in \eqref{b2s3} is non-zero, since $g$ is invertible,   so at least one of $\delta_1$ and $\delta_2$ is non-zero, since only $u_{(2,1)}$ and $u_{(2,2)}$ have $v_{n/2}$ or $v_{n/2+1}$ in the decomposition \eqref{orb212}.  On the other hand, $(u_{(1,1)})g$ in decomposition \eqref{b2s3} does not contain $w_3$ and $w_4$, so $\delta_1$ and $\delta_2$ must be zero, which is a contradiction. Thus, such $g$ does not exist, so $\omega_1$ and $\omega_2$ lie in distinct $\Omega^5$-orbits.

Assume now      that $\omega_1 g =\omega_5$ for $g \in GL_n(q),$ so 
$$g \in (S \cap S^x) \cap (S \cap S^x)^y \cap z_1^{-1}Sz_5.$$
Therefore, $g$ has shape \eqref{sered} and $g=z_1^{-1}hz_5$ for $h \in S,$ so 
$(u_{(1,1)})g$ has decomposition \eqref{b2s3}
and 
\begin{equation}\label{b2s5}
(u_{(1,1)})g=(u_{(1,1)})z_1^{-1}hz_5=(v_{n-m+1})hz_5 = \delta_1 u_{(5,1)} + \ldots + \delta_m u_{(5,m)}.
\end{equation}
At least one of $\alpha_{n/2}$ and $\beta_{n/2}$ in \eqref{b2s3} is non-zero, since $g$ is invertible,   so at least one of $\delta_1$ and $\delta_2$ is non-zero, since only $u_{(5,1)}$ and $u_{(5,2)}$ have $v_{n/2}$ or $v_{n/2+1}$ in the decomposition \eqref{orb23}.  On the other hand, $(u_{(1,1)})g$ in decomposition \eqref{b2s3} does not contain $w_6$, so $\delta_1=1.$ Also, there is no $w_{(5,j)}$ for $j>1$ in \eqref{b2s3}, so only $\delta_2$ can be non-zero in \eqref{b2s5}, but \eqref{b2s3} does not contain $w_5$ and $\delta_2$ must be zero, which  contradicts the invertibility of $g$. Thus, such $g$ does not exist, so $\omega_1$ and $\omega_5$ lie in distinct $\Omega^5$-orbits.

Now let $n < 8,$ so $n=4$ or $n=6$ since $n$ is even. In both cases $k=n-1$, since there must be only one  $(2 \times 2)$ block, so $n_{n/2}=2$ and $n_i=1$ for the remaining $i.$ Therefore, if $x=\diag(\sgn(\sigma),1 \ldots, 1) \per(\sigma)$ with $\sigma = (1, 2, \ldots, n )$, then $S \cap S^x \le RT(GL_n(q))$ and $$(S \cap S^x) \cap (S \cap S^x)^y \le D(GL_n(q)),$$
where $y$ is as in {\bf Case 1}. The rest of the proof is as in {\bf Case 1}. 
\end{proof}


\begin{Prop}
\label{case3prop}
 Theorem $\ref{lem41}$ holds in {\bf Case 3}.
\end{Prop}
\begin{proof}

 Recall that $f=1, $ $n$ is odd, there exists exactly one $i \in \{1, \ldots, k\}$ with $n_i=2$, and $\gamma_i(g)$ appears in rows  $\{(n+1)/2,(n+1/2)+1\}$ of $g \in M$ for such $i$. Let $s := (n + 1)/2$. Let $x= \diag[x_k, \ldots, x_1]$, where $x_i\in SL_{n_i}(q)$ such that $S_i \cap S_i^{x_i} \le RT(GL_{n_i}(q))$ if $n_i \ne 2$ and $x_i$ is the identity matrix if $n_i=2.$ If $y$ is as in Lemma \ref{irrtog}, then  calculations show that  
$$(S \cap S^x) \cap (S \cap S^x)^y$$ consists of matrices of shape 
\begin{equation}\label{sered2} 
  \left(\begin{smallmatrix} 
\alpha_1 &0       & \dots              &                  &                 &                   &  & \dots &0  \\
0        & \ddots &                    &                  &                 &                   &       & &  \\
         &        & \alpha_{s-2} &                  &                 &                   &       & &   \\
         &        &                    &\alpha_{s-1}&\beta_{s-1}&                   &       & &   \\
         &        &                    &                  &\alpha_{s} &                   &               &              &       &   \\
         &        &                    &                  &\beta_{s+1}&\alpha_{s+1} &              &       &   \\
         &        &                    &                    &               &                   &\alpha_{s+2}&       & \\
         &        &                    &                    &               &                   &              &\ddots & 0 \\
 0       & \dots  &                    &                    &               &                   &     \dots    &  0    & \alpha_n 
\end{smallmatrix} \right).
\end{equation}
Assume that $n \ge 9$. We take as the $S$-invariant subspace $U$ the subspace with  basis $\{v_{(n+1)/2}, \ldots, v_n\}$, so
$m=(n+1)/2 \ge 5$ and $n-m=(n-1)/2\ge 4.$  Let us rename some basis vectors for convenience, so denote vectors 
$$v_1, \ldots, v_{(n-1)/2}, v_{(n+2)/2}, \ldots, v_n$$ by $$w_1, w_2, \ldots, w_{n-3}$$ respectively.
Let  $z_1, \ldots, z_5 \in SL_n(q)$ be such that $$(v_{n-m+j})z_i=u_{(i,j)}$$ for $i=1, \ldots, 5$ and $j=1, \ldots, m,$ where
\begin{gather}\label{orb2r1}
\left\{ 
\begin{aligned}
u_{(1,1)} & = v_{s-1}+w_1;\\
u_{(1,2)} & = v_{s+1}+w_1;\\
u_{(1,3)} & =v_{s}+w_1+w_2;\\
u_{(1,3+r)} & =w_1+w_{2+r}\phantom{;} &\text{ for } r \in \{1, \ldots, m-4\};\\
u_{(1,m)} & =w_1 +\sum_{r=m-1}^{n-3}w_r; \, \, \, \, \, \, \,
\end{aligned}
\right.
\end{gather}
\begin{gather}\label{orb2r12}
\left\{ 
\begin{aligned}
u_{(2,1)} & =v_{s-1 }+w_2;\\
u_{(2,2)} & = v_{s+1}+w_2;\\
u_{(2,3)} & =v_{s}+w_2+w_3;\\
u_{(2,3+r)} & =w_2+w_{(2,r)}\phantom{;} &\text{ for } r \in \{1, \ldots, m-4\};\\
u_m^2 & =w_2 +\sum_{r=m-3}^{n-5}w_{(2,r)}.\\
\end{aligned}
\right. 
\end{gather}
Here $w_{(2,1)}, \ldots, w_{(2,n-5)}$ are equal to $w_1, w_4, \ldots, w_{n-3}$ respectively.  
\begin{gather}
\left\{ 
\begin{aligned}
u_{(3,1)} & =v_{s-1 }+w_3;\\
u_{(3,2)} & = v_{s+1}+w_3;\\
u_{(3,3)} & =v_{s}+w_3+w_4;\\
u_{(3,3+r)} & =w_3+w_{(3,r)}\phantom{;} &\text{ for } r \in \{1, \ldots, m-4\};\\
u_{(3,m)} & =w_3 +\sum_{r=m-3}^{n-5}w_{(3,r)}.
\end{aligned}
\right.
\end{gather}
Here $w_{(3,1)}, \ldots, w_{(3,n-5)}$ are equal to $w_1, w_2,  w_5, \ldots, w_{n-3}$ respectively.  
\begin{gather} 
\left\{ 
\begin{aligned}\label{orb2r2}
u_{(4,1)} & =v_{s-1 }+w_4;\\
u_{(4,2)} & = v_{s+1}+w_4;\\
u_{(4,3)} & =v_{s}+w_4+w_5;\\
u_{(4,3+r)} & =w_4+w_{(4,r)}\phantom{;} &\text{ for } r \in \{1,\ldots, m-4\};\\
u_{(4,m)} & =w_4 +\sum_{r=m-3}^{n-5}w_{(4,r)}.
\end{aligned}
\right. 
\end{gather}
Here $w_{(4,1)}, \ldots, w_{(4,n-5)}$ are equal to $w_1, w_2, w_3,  w_6, \ldots, w_{n-3}$ respectively.  
\begin{gather} 
\left\{ 
\begin{aligned}\label{orb2r3}
u_{(5,1)} & =v_{s-1 }+w_5;\\
u_{(5,2)} & = v_{s+1}+w_5;\\
u_{(5,3)} & =v_{s}+w_5+w_6;\\
u_{(5,3+r)} & =w_5+w_{(5,r)}\phantom{;} &\text{ for } r \in \{1, \ldots, m-4\};\\
u_{(5,m)} & =w_5 +\sum_{r=m-3}^{n-5}w_{(5,r)}.
\end{aligned}
\right. 
\end{gather}
Here $w_{(5,1)}, \ldots, w_{(5,n-5)}$ are equal to $w_1, w_2, w_3, w_4, w_7, \ldots, w_{n-3}$ respectively.

Let $$\omega_i=(S,Sx,Sy,Sxy,Sz_i).$$ We first show  that the $\omega_i$ are regular points of $\Omega^5$. Consider $\omega_1$. The regularity of the remaining points can be shown using the same arguments. Let $t \in (S \cap S^x) \cap (S \cap S^x)^y \cap S^{z_1}$, so it takes shape \eqref{sered2} for some $\alpha_i, \beta_i \in \mathbb{F}_q$ and stabilises the subspace $Uz_1.$ Thus
\begin{equation}\label{b2sp1}
(u_{(1,1)})t=\alpha_{s-1} v_{s-1} +  \beta_{s-1}v_{s}+\alpha_1 w_1,
\end{equation}
since $t \in (S \cap S^x) \cap (S \cap S^x)^y$ and 
\begin{equation}\label{b2sp2}
(u_{(1,1)})t=\eta_1 u_{(1,1)} +\eta_2 u_{(1,2)} + \ldots + \eta_m u_{(1,m)},
\end{equation}
since $t$ stabilises  $Uz_1.$ There is no $v_{s+1}$ and $w_i$ for $i \ge 2$ in decomposition \eqref{b2sp1}, so in \eqref{b2sp2} we obtain
$$(u_{(1,1)})t=\eta_1 u_{(1,1)}=\alpha_1 u_{(1,1)},$$
in particular $\beta_{s-1}=0$. The same arguments show that $\beta_{s+1}=0$ and
$$(u_{(1,i)})t=\alpha_1 u_{(1,i)},$$
for the remaining $i \in \{1, \ldots , m\},$
so $t$ is scalar. Therefore, $\omega_1$ is regular.

Now we claim that the $\omega_i$ lie in distinct orbits of $\Omega^5.$ Here we prove that $\omega_1$ does not lie in the orbit containing $\omega_2$; the   remaining cases are similar.  Assume that $\omega_1 g =\omega_2$ for $g \in GL_n(q),$ so 
$$g \in (S \cap S^x) \cap (S \cap S^x)^y \cap z_1^{-1}Sz_2.$$
Therefore, $g$ has shape \eqref{sered2} and $g=z_1^{-1}hz_2$ for $h \in S,$ so 
\begin{equation}\label{b2sp3}
(u_{(1,3)})g=\alpha_{s} v_{s} +\alpha_1 w_1+\alpha_2 w_2,
\end{equation}
and 
\begin{equation}\label{b2sp4}
(u_{(1,3)})g=(u_{(1,3)})z_1^{-1}hz_2=(v_{n-m+3})hz_2 = \delta_1 u_{(2,1)} + \ldots + \delta_m u_{(2,m)}.
\end{equation}
Since $\alpha_{s}$  is non-zero and there are no $v_{s \pm 1}$ in \eqref{b2sp3}, $\delta_3$ is non-zero, so \eqref{b2sp3}  must contain a term  $ \delta_3 w_3$. Thus, such $g$ does not exist. Therefore, $\omega_1$ and $\omega_2$ lie in  distinct orbits. 

Now let $n<9.$ If $n=3,$ then Theorem~\ref{lem41} follows by Proposition \ref{n3GL}. So $n=5$ or $n=7.$
Assume that $n_i=1$ for some $i\le k$. So there is a $(1\times 1)$ block on the line $j \le n$. Let $\tilde{x}$ be $\diag(\sgn(\sigma), 1, \ldots, 1) \cdot \per(\sigma)$ with $\sigma=(s+1,j)$ if $j>s+1$ and $\sigma=(j,s,s+1)$ if $j<s.$ It is easy to see that 
$$S \cap S^{x \tilde x}\le RT(GL_n(q)),$$
so the rest of the proof is as in {\bf Case 1}. If $n=5$ then $s+1=4,$ so there is a $(1 \times 1)$ block on the $n$-th line. If $n=7,$ then  $s+1=5,$ so there are  $(1 \times 1)$ blocks in the rows $n-1$ and $n$, since there must be only one  $(2 \times 2)$ block.       
\end{proof}


\begin{Prop}
\label{case4prop}
 Theorem $\ref{lem41}$ holds in {\bf Case 4}.
\end{Prop}
\begin{proof}

Recall that $f=1, $ $n$ is odd, there exists exactly one $i \in \{1, \ldots, k\}$ with $n_i=2$, and $\gamma_i(g)$ appears in rows  $\{(n-1)/2,(n+1)/2+1\}$ of $g \in M$ for such $i$.   Let $s:=(n-1)/2+1$. Let $x= \diag[x_k, \ldots, x_1]$, where $x_i\in SL_{n_i}(q)$ is such that $S_i \cap S_i^{x_i} \le RT(GL_{n_i}(q))$ if $n_i \ne 2$ and $x_i$ is the identity matrix if $n_i=2.$ If $y$ is as in Lemma \ref{irrtog}, then  calculations show that  
$$(S \cap S^x) \cap (S \cap S^x)^y$$ consists of matrices of shape 
\begin{equation}\label{seredl} 
  \left(\begin{matrix} 
\alpha_1 &0       & \dots              &                  &                 &             \dots &0  \\
0        & \ddots &                    &                  &                 &                   &   \\
         &        & \alpha_{s-1}       &                  &                 &                   &   \\
         &        & \beta_{s-1}        &\alpha_{s}        &\beta_{s+1}      &                   &   \\
         &        &                    &                  &\alpha_{s+1   }  &                   &   \\
         &        &                    &                  &                 &\ddots             & 0  \\
 0       & \dots  &                    &                  &                 &     0             &\alpha_{n} 
\end{matrix} \right).
\end{equation}
Assume that $n \ge 7$. We take as the $S$-invariant subspace $U$ the subspace with the basis $\{v_{s+1}, \ldots, v_n\}$, so
$m=(n-1)/2 \ge 3$ and $n-m=(n+1)/2\ge 4.$  Let us rename some basis vectors for convenience, so denote vectors 
$$v_1, \ldots, v_{s-2}, v_{s+2}, \ldots, v_n$$ by $$w_1, w_2, \ldots, w_{n-3}$$ respectively.
Let  $z_1, \ldots, z_5 \in SL_n(q)$ be such that $$(v_{n-m+j})z_i=u_{(i,j)}$$ for $i=1, \ldots, 5$ and $j=1, \ldots, m,$ where
\begin{gather}\label{orb2l1}
\left\{ 
\begin{aligned}
u_{(1,1)} & = v_{s}+w_1;\\
u_{(1,2)} & = v_{s}+v_{s-1}+v_{s+1}+w_2;\\
u_{(1,2+r)} & =v_{s+1}+w_{2+r}\phantom{;} &\text{ for } r \in \{1, \ldots, m-3\};\\
u_{(1,m)} & =v_{s+1} +\sum_{r=m-1}^{n-3}w_r;
\end{aligned}
\right.
\end{gather}
\begin{gather}\label{orb2l12}
\left\{ 
\begin{aligned}
u_{(2,1)} & =v_{s}+w_1;\\
u_{(2,2)} & = v_{s}+v_{s-1}+v_{s+1} +w_3;\\
u_{(2,2+r)} & =v_{s+1}+w_{(2,r)}\phantom{;} &\text{ for } r \in \{1, \ldots, m-3\};\\
u_{(2,m)} & =v_{s+1} +\sum_{r=m-2}^{n-5}w_{(2,r)}.\\
\end{aligned}
\right. 
\end{gather}
Here $w_{(2,1)}, \ldots, w_{(2,n-5)}$ are equal to $w_2, w_4, \ldots, w_{n-3}$ respectively.  
\begin{gather}
\left\{ 
\begin{aligned}
u_{(3,1)} & =v_{s}+w_1;\\
u_{(3,2)} & = v_{s}+v_{s-1}+v_{s+1}+w_4;\\
u_{(3,2+r)} & =v_{s+1}+w_{(3,m)}\phantom{;} &\text{ for } r \in \{1, \ldots, m-3\};\\
u_{(3,m)} & =v_{s+1} +\sum_{r=m-2}^{n-5}w_{(3,r)}.
\end{aligned}
\right.
\end{gather}
Here $w_{(3,1)}, \ldots, w_{(3,n-5)}$ are equal to $w_2, w_3,  w_5, \ldots, w_{n-3}$ respectively.  
\begin{gather} 
\left\{ 
\begin{aligned}\label{orb2l2}
u_{(4,1)} & =v_{s}+w_2;\\
u_{(4,2)} & = v_{s}+v_{s-1}+v_{s+1}+w_3;\\
u_{(4,2+r)} & =v_{s+1}+w_{(4,r)}\phantom{;} &\text{ for } r \in \{1, \ldots, m-3\};\\
u_{(4,m)} & =v_{s+1} +\sum_{r=m-2}^{n-5}w_{(4,r)}.
\end{aligned}
\right. 
\end{gather}
Here $w_{(4,1)}, \ldots, w_{(4,n-5)}$ are equal to $w_1, w_4, \ldots, w_{n-3}$ respectively.  
\begin{gather} 
\left\{ 
\begin{aligned}\label{orb2l3}
u_{(5,1)} & =v_{s}+w_2;\\
u_{(5,2)} & = v_{s}+v_{s-1}+v_{s+1}+w_4;\\
u_{(5,2+r)} & =v_{s+1}+w_{(5,r)}\phantom{;} &\text{ for } r \in \{1, \ldots, m-3\};\\
u_{(5,m)} & =v_{s+1} +\sum_{r=m-2}^{n-5}w_{(5,r)}.
\end{aligned}
\right. 
\end{gather}
Here $w_{(5,1)}, \ldots, w_{(5,n-5)}$ are equal to $w_1, w_3, w_5, \ldots, w_{n-3}$ respectively.

Let $$\omega_i=(S,Sx,Sy,Sxy,Sz_i).$$ We  first show that the $\omega_i$ are regular points of $\Omega^5$. Consider $\omega_1$. The regularity of the remaining points can be shown using the same arguments. Let $t \in (S \cap S^x) \cap (S \cap S^x)^y \cap S^{z_1}$, so it takes shape \eqref{seredl} for some $\alpha_i, \beta_i \in \mathbb{F}_q$ and stabilises the subspace $Uz_1.$ Thus
\begin{equation}\label{b2l1}
(u_{(1,1)})t= \beta_{s-1}v_{v-1}+\alpha_s v_s +\beta_{s+1}v_{v+1} + \alpha_1 w_1,
\end{equation}
since $t \in (S \cap S^x) \cap (S \cap S^x)^y$ and 
\begin{equation}\label{b2l2}
(u_{(1,1)})t=\eta_1 u_{(1,1)} +\eta_2 u_{(1,2)} + \ldots + \eta_m u_{(1,m)},
\end{equation}
since $t$ stabilises  $Uz_1.$ There is no  $w_i$ for $i \ge 2$ in decomposition \eqref{b2l1}, so in \eqref{b2l2} we obtain
$$(u_{(1,1)})t=\eta_1 u_{(1,1)}=\alpha_s u_{(1,1)},$$
in particular $\beta_{s-1}=\beta_{s+1}=0$. The same arguments show that 
$$(u_{(1,i)})t=\alpha_s u_{(1,i)},$$
for the remaining $i \in \{1, \ldots , m\},$
so $t$ is scalar. Therefore, $\omega_1$ is regular.

Now we claim that the $\omega_i$ lie in distinct orbits of $\Omega^5.$ Here we prove that $\omega_1$ does not lie in the orbits containing $\omega_2$ and $\omega_4$; the remaining cases are similar.  Assume that $\omega_1 g =\omega_2$ for $g \in GL_n(q),$ so 
$$g \in (S \cap S^x) \cap (S \cap S^x)^y \cap z_1^{-1}Sz_2.$$
Therefore, $g$ has shape \eqref{seredl} and $g=z_1^{-1}hz_2$ for $h \in S,$ so 
\begin{equation}\label{b2l3}
(u_{(1,2)})g=\alpha_{s}v_{s}+\beta_{s-1}v_{s-1}+\beta_{s+1}v_{s+1} +\alpha_{s-1}v_{s-1}+\alpha_{s+1}v_{s+1} + \alpha_2 w_2,
\end{equation}
and 
\begin{equation}\label{b2l4}
(u_{(1,2)})g=(u_{(1,2)})z_1^{-1}hz_2=(v_{n-m+3})hz_2 = \delta_1 u_{(2,1)} + \ldots + \delta_m u_{(2,m)}.
\end{equation}
Since there are no $w_1$ and $w_3$ in \eqref{b2l3}, $\delta_1=\delta_2=0$. Thus, $\alpha_s=0$ in \eqref{b2l3}, which is a contradiction, since $g$ must be invertible, so such $g$ does not exist. Therefore, $\omega_1$ and $\omega_2$ lie in  distinct orbits. 

Assume that $\omega_1 g =\omega_4$ for $g \in GL_n(q),$ so 
$$g \in (S \cap S^x) \cap (S \cap S^x)^y \cap z_1^{-1}Sz_4.$$
Therefore, $g$ has shape \eqref{seredl} and $g=z_1^{-1}hz_4$ for $h \in S,$ so 
\begin{equation}\label{b2l13}
(u_{(1,1)})g=\alpha_{s}v_{s}+\beta_{s-1}v_{s-1}+\beta_{s+1}v_{s+1}  +\alpha_1 w_1,
\end{equation}
and 
\begin{equation}\label{b2l14}
(u_{(1,1)})g=(u_{(1,1)})z_1^{-1}hz_4=(v_{n-m+3})hz_4 = \delta_1 u_{(4,1)} + \ldots + \delta_m u_{(4,m)}.
\end{equation}
Since there are no $w_2$ and $w_3$ in \eqref{b2l13}, $\delta_1=\delta_2=0$. Thus, $\alpha_s=0$ in \eqref{b2l13}, which is a contradiction, since $g$ must be invertible, so such $g$ does not exist. Therefore, $\omega_1$ and $\omega_4$ lie in  distinct orbits. 

Now let $n \le 5.$ If $n=3,$ then Theorem~\ref{lem41} follows by Proposition \ref{n3GL}, so consider the case $n=5.$
Assume that $n_i=1$ for some $i\le k$. So there is a $(1\times 1)$ block on the line $j \le n$. Let $\tilde{x}$ be $\diag(\sgn(\sigma),1, \ldots, 1) \cdot \per(\sigma)$ with $\sigma=(s+1,j)$ if $j>s+1$ and  $\sigma=(j,s,s+1)$ if $j<s.$ It is easy to see that 
$$S \cap S^{x \tilde x}\le RT(GL_n(q)),$$
so the rest of the proof is  as in {\bf Case 1}. Since $n=5$,  $s-1=2,$ so there is a $(1 \times 1)$ block on the first line. 
\end{proof}

%\\\\\\\\\\\\\\\\\\\\\\\\\\\\\\\\\\\\\\\\\\\\\\\\\\\\\\\\\\\\\
%\\\\\\\\\\\\\\\\\\\\\\\\\\\\\\\\\\\\\\\\\\\\\\\\\\\\\\\\\\\\\





\begin{Prop}
\label{case5prop}
Theorem $\ref{lem41}$ holds in {\bf Case 5}.
\end{Prop}
\begin{proof}
 Let $i \in \{1, \ldots, k\}$  be  such that $n_i=2$. Our proof  splits into three subcases:
\begin{description}[before={\renewcommand\makelabel[1]{\bfseries ##1}}]
\item[{\bf Case (5.1)}] $n\ge 9$ and $\gamma_i(g)$ appears in rows  $\{l,l+1\}$ of $g \in M$ where $l>n/2$ if $n$ is even and $l>(n+1)/2$ if $n$ is odd;
\item[{\bf Case (5.2)}] $n \ge 9$ and $\gamma_i(g)$ appears in rows $\{l-1,l\},$ where $l \le n/2$ if $n$ is even and $l \le (n-1)/2$ if $n$ is odd;
\item[{\bf Case (5.3)}] $n<9$.
\end{description} 

{\bf Case (5.1).} Consider the case when  the only $(2\times 2)$ block is in  rows $\{l,l+1\},$ where $l>n/2$ if $n$ is even and $l>(n+1)/2$ if $n$ is odd. Denote $s:=n-l$. Let $x= \diag[x_k, \ldots, x_1]$, where $x_i\in SL_{n_i}(q)$ is such that $S_i \cap S_i^{x_i} \le RT(GL_{n_i}(q))$ if $n_i \ne 2$ and $x_i$ is the identity matrix if $n_i=2.$ If $y$ is as in Lemma \ref{irrtog}, then  calculations show that  
$$(S \cap S^x) \cap (S \cap S^x)^y$$ consists of matrices of shape 
{
\begin{equation}\label{ugol} 
  \left(\begin{smallmatrix} 
\alpha_1&0     &\dots         &            &              &      &           &             &            &\dots  &0  \\
0       &\ddots&              &            &              &      &           &             &            &       & \\
        &      &\alpha_{s-1}  &            &              &      &           &             &            &       &   \\
        &      &              &\alpha_{s}&\beta_{s}       &      &           &             &            &       &   \\
        &      &              &            &\alpha_{s+1  }&      &           &             &            &       &   \\
        &      &              &            &              &\ddots&           &             &            &       &   \\
        &      &              &            &              &      &\alpha_{l} &             &            &       &      \\
        &      &              &            &              &      &\beta_{l+1}&\alpha_{j+1} &            &       &   \\
        &      &              &            &              &      &           &             &\alpha_{l+2}&       & \\
        &      &              &            &              &      &           &             &            &\ddots & 0 \\
0       &\dots &              &            &              &      &           &             &     \dots  &  0    & \alpha_n 
\end{smallmatrix} \right).
\end{equation}
}
Assume that $n \ge 9$. We take as the $S$-invariant subspace $U$ the subspace with  basis $\{v_{l}, \ldots, v_n\}$, so
$m\ge 2$ and $n-m\ge [n/2] \ge 4.$  Let us rename some basis vectors for convenience, so denote vectors 
$$v_1, \ldots, v_{s-1}, v_{s+2}, \ldots, v_{l-1}, v_{l+2}, \ldots, v_{n}$$ by $$w_1, w_2, \ldots, w_{n-4}$$ respectively.
Let $z_1, \ldots, z_5 \in SL_n(q)$ be such that $$(v_{n-m+j})z_i=u_{(i,j)}$$ for $i=1, \ldots, 5$ and $j=1, \ldots, m,$ where
\begin{gather}\label{orb2t1}
\left\{ 
\begin{aligned}
u_{(1,1)} & = v_{s} + v_{l+1};\\
u_{(1,2)} & = v_{s}+ v_{s+1}+v_{l}+w_1;\\
u_{(1,2+r)} & =v_{s+1}+w_{1+r}\phantom{;} &\text{ for } r \in \{1, \ldots, m-3\};\\
u_{(1,m)} & =v_{s+1}+\sum_{r=m-1}^{n-4}w_r;
\end{aligned}
\right.
\end{gather}
\begin{gather}%\label{orb2t12}
\left\{ 
\begin{aligned}
u_{(2,1)} & = v_{s} + v_{l+1};\\
u_{(2,2)} & = v_{s}+v_{s+1}+v_{l}+w_2;\\
u_{(2,2+r)} & =v_{s+1}+w_{(2,r)}\phantom{;} &\text{ for } r \in \{1, \ldots, m-3\};\\
u_{(2,m)} & =v_{s+1}+\sum_{r=m-2}^{n-5}w_{(2,r)}.
\end{aligned}
\right. 
\end{gather}
Here $w_{(2,1)}, \ldots, w_{(2,n-5)}$ are equal to $w_1, w_3, \ldots, w_{n-4}$ respectively.  
\begin{gather}
\left\{ 
\begin{aligned}
u_{(3,1)} & = v_{s} + v_{l+1};\\
u_{(3,2)} & = v_{s}+ v_{s+1}+v_{l}+w_3;\\
u_{(3,2+r)} & =v_{s+1}+w_{(3,r)}\phantom{;} &\text{ for } r \in \{1, \ldots, m-3\};\\
u_{(3,m)} & =v_{s+1}+\sum_{r=m-2}^{n-5}w_{(3,r)}.
\end{aligned}
\right.
\end{gather} 
Here $w_{(3,1)}, \ldots, w_{(3,n-5)}$ are equal to $w_1, w_2, w_4 \ldots, w_{n-4}$ respectively.  
\begin{gather} 
\left\{ 
\begin{aligned}%\label{orb2t2}
u_{(4,1)} & = v_{s} + v_{l+1};\\
u_{(4,2)} & = v_{s}+ v_{s+1}+v_{l}+w_4;\\
u_{(4,2+r)} & =v_{s+1}+w_{(4,r)}\phantom{;} &\text{ for } r \in \{1, \ldots, m-3\};\\
u_{(4,m)} & =v_{s+1}+\sum_{r=m-2}^{n-5}w_{(4,r)}.
\end{aligned}
\right. 
\end{gather}
Here $w_{(4,1)}, \ldots, w_{(4,n-5)}$ are equal to $w_1, w_2, w_3, w_5 \ldots, w_{n-4}$ respectively.  
\begin{gather} 
\left\{ 
\begin{aligned}\label{orb2t3}
u_{(5,1)} & = v_{s} + v_{l+1};\\
u_{(5,2)} & = v_{s}+ v_{s+1}+v_{l}+w_5;\\
u_{(5,2+r)} & =v_{s+1}+w_{(5,r)}\phantom{;} &\text{ for } r \in \{1, \ldots, m-3\};\\
u_{(5,m)} & =v_{s+1}+\sum_{r=m-2}^{n-5}w_{(5,r)}.
\end{aligned}
\right. 
\end{gather}
Here $w_{(5,1)}, \ldots, w_{(5,n-5)}$ are equal to $w_1, w_2, w_3, w_4, w_6, \ldots, w_{n-4}$ respectively.

 Notice that $n \ge 9$ and $n-m \ge 3$ is necessary for such a definition of $u_{(i,j)}$. 

Let $$\omega_i=(S,Sx,Sy,Sxy,Sz_i).$$ We first show that the $\omega_i$ are regular points of $\Omega^5$. Consider $\omega_1$.  The regularity of the remaining points can be shown using the same arguments. Let $t \in (S \cap S^x) \cap (S \cap S^x)^y \cap S^{z_1}$, so it takes shape \eqref{ugol} for some $\alpha_i, \beta_i \in \mathbb{F}_q$ and stabilises the subspace $Uz_1.$ Thus
\begin{equation}\label{b2t1}
(u_{(1,1)})t=\alpha_{s}v_s +\beta_s v_{s+1} + \alpha_{l+1} v_{l+1} +\beta_{l+1}v_l,
\end{equation}
since $t \in (S \cap S^x) \cap (S \cap S^x)^y$ and 
\begin{equation}\label{b2t2}
(u_{(1,1)})t=\eta_1 u_{(1,1)} +\eta_2 u_{(1,2)} + \ldots + \eta_m u_{(1,m)},
\end{equation}
since $t$ stabilises  $Uz_1.$ There is no $w_i$ for $i \ge 1$ in decomposition \eqref{b2t1}, so in \eqref{b2t2} we obtain
$$(u_{(1,1)})t=\eta_1 u_{(1,1)}=\alpha_s u_{(1,1)},$$
in particular $\beta_{s}=\beta_{l+1}=0$, so $t$ is diagonal. Therefore,
\begin{equation}\label{2b2t2}
(u_{(1,2)})t=\alpha_{s}v_s +\alpha_{s+1}v_{s+1}+ \alpha_{l} v_{l} +\alpha_{1}w_1,
\end{equation}
must also lie in $\langle u_{(1,1)}, \ldots, u_{(1,m)} \rangle$. There are no $v_{l+1}$ and $w_i$ for $i \ge 2$ in \eqref{2b2t2}, so 
$$(u_{(1,2)})t= \alpha_s u_{(1,1)}.$$ If $i>2$ then
 the same arguments show that 
$$(u_{(1,i)})t=\alpha_{s+1} u_{(1,i)}=\alpha_s u_{(1,i)},$$
for the remaining $i \in \{1, \ldots , m\},$ so $t$ is scalar. Therefore, $\omega_1$ is regular.

Now we claim that  the $\omega_i$ lie in distinct orbits of $\Omega^5.$ Here we prove that $\omega_1$ and $\omega_2$  lie in distinct orbits;  the remaining cases are similar.  Assume that $\omega_1 g =\omega_2$ for $g \in GL_n(q),$ so 
$$g \in (S \cap S^x) \cap (S \cap S^x)^y \cap z_1^{-1}Sz_2.$$
Therefore, $g$ has shape \eqref{ugol} and $g=z_1^{-1}hz_2$ for $h \in S,$ so 
\begin{equation}\label{b2t3}
(u_{(1,2)})g=\alpha_{s}v_s +\beta_s v_{s+1} + \alpha_{s+1} v_{s+1} +\alpha_{l}v_l+\alpha_1 w_1,
\end{equation}
and 
\begin{equation*}%\label{b2t4}
(u_{(1,2)})g=(u_{(1,2)})z_1^{-1}hz_2=(v_{n-m+2})hz_2 = \delta_1 u_{(2,1)} + \ldots + \delta_m u_{(2,m)}.
\end{equation*}
Since there is no $w_2$ in \eqref{b2t3}, $\delta_2$ is zero. Therefore, there must be no $v_l$ in \eqref{b2t3}, so $\alpha_l$ must be zero, which contradicts the existence of such invertible $g.$ Thus, $\omega_1$ and $\omega_2$ lie in  distinct orbits. 

\medskip

{\bf Case (5.2).} Consider the case when $n \ge 9$ and the only $(2\times 2)$ block is in  rows $\{l-1,l\},$ where $l \le n/2$ if $n$ is even and $l \le (n-1)/2$ if $n$ is odd. We take as the $S$-invariant subspace $U$ the subspace with  basis $\{v_{l+1}, \ldots, v_n\}$, so
$m\ge [n/2]$ and $n-m \ge 2.$  If $n-m \ge 3$, then the proof is  as in {\bf Case (5.1).} 

 Consider the case  $n-m=2,$ so $n_k=2$. Let $x= \diag[x_k, \ldots, x_1]$, where $x_i\in SL_{n_i}(q)$ is such that $S_i \cap S_i^{x_i} \le RT(GL_{n_i}(q))$ if $n_i \ne 2$ and $x_i$ is the identity matrix if $n_i=2.$ Let $y$ be as in Lemma \ref{irrtog}. If $k=2$ in \eqref{stup}, then there exists $x_1 \in SL_{n_1}(q)$ such that $S_1 \cap S_1^{x_1} \le D(GL_{n_1}(q))$, by %Lemma \ref{sch}, Lemma \ref{prtoirr} and Section \ref{sec5}
Theorem \ref{irred}, since $n_1=n-2\ge 7.$ Thus, $(S\cap S^x) \cap (S\cap S^x)^y \le D(GL_n(q))$ and the rest of the proof is as in {\bf Case 1}.

Let $k>2$ and $n_1 \ne 1$. So $n_1 >2$, since the only $(2\times 2)$ block is in  rows $\{l-1,l\},$ where $l \le n/2$. % there is only one $i$ such that $n_i=2,$ 
We can take $U$ to be the subspace with  basis $\{v_{n -n_1+1}, \ldots, v_n\}$. Therefore, $n-m\ge 3$  and the proof is the same as the one using \eqref{orb2t1} -- \eqref{orb2t3}. 

Let $k>2$ and $n_1=1.$ Denote $\diag(-1,1\ldots,1) \cdot \per((2,n))$ by $\tilde{x}$. Calculations show that  $S \cap S^{x \tilde{x}} \le RT(GL_n(q))$, so $(S \cap S^{x \tilde{x}}) \cap (S \cap S^{x \tilde{x}})^y\le D(GL_n(q))$. The rest of the proof is  as in {\bf Case 1}, since we can take $U=V_{k-1}=\langle v_3, \ldots, v_n \rangle,$ so $n-m=2,$ $m\ge 2.$

\medskip

{\bf Case (5.3).}  Now let $n<9.$ If $n=3,$ then Theorem~\ref{lem41} follows by Proposition \ref{n3GL}, so $n \in \{4,5,6,7,8\}$.

Assume that $n_i=1$ for some $i\le k$. So there is a $(1\times 1)$ block in the row $j \le n$ of $g \in S$. Let $\tilde{x}$ be $\diag(\sgn(\sigma), 1, \ldots, 1) \cdot \per(\sigma)$ with $\sigma=(s+1,j)$ if $j>s+1$ and  $\sigma=(j,s,s+1)$ if $j<s.$ It is easy to see that 
$$S \cap S^{x \tilde x}\le RT(GL_n(q)),$$
so the rest of proof is  as in {\bf Case 1}.

If we exclude the cases previously resolved, then we obtain the following list of possibilities:\medskip\\
$n=5, k=2, n_1=2, n_2=3;$\\
$n=5, k=2, n_1=3, n_2=2;$\\
$n=6, k=2, n_1=2, n_2=4;$\\
$n=6, k=2, n_1=4, n_2=2;$\\
$n=7, k=2, n_1=2, n_2=5;$\\
$n=7, k=2, n_1=5, n_2=2;$\\
$n=8, k=2, n_1=2, n_2=6;$\\
$n=8, k=2, n_1=6, n_2=2;$\\
$n=8, k=3, n_1=2, n_2=3, n_3=3;$\\
$n=8, k=3, n_1=3, n_2=3, n_3=2.$
\medskip

If $(n,q) \ne (6,3),$ then, by Lemma \ref{irred}, for all $i$ such that $n_i\ne 2$  there exist $x_i \in SL_{n_i}(q)$ such that $S_i \cap S_i^{x_i} \le D(GL_{n_i}(q))$. Let $x=\diag[x_k, \ldots, x_1]$, where $x_i$ is the identity matrix  if $n_i=2$. It is easy to check directly that $$(S \cap S^x) \cap (S \cap S^x)^y \le D(GL_n(q))$$
for all cases above, so the rest of the proof is  as in {\bf Case 1}.

  For $(n,q)=(6,3)$ the result is established by computation. This concludes the proof of Proposition \ref{case5prop}.
\end{proof}
  
 Theorem \ref{lem41}  now follows from     Propositions \ref{case1prop} -- \ref{case5prop}.
\end{proof}


\begin{Th}
\label{lem42}
If $k>1$,  $q=2$ and  there exists  $i \in \{1, \ldots, k\}$ such that $\gamma_i(S)$ is the normaliser of a Singer cycle of $GL_3(2),$ then Theorem {\rm \ref{theorem}} holds.
\end{Th} 
\begin{proof}
Notice that $GL_n(2)=SL_n(2),$ so $G=GL_n(2).$  %If $S$ is irreducible, then $b_S(GL_n(q)) \le 3$ by Lemma \ref{irred}, so $\Reg_S(G,5) \ge 5$ by Lemma \ref{base4}.


Since $k>1,$ $S$ stabilises a nontrivial invariant subspace $U$ of dimension $m<n$. Assume that matrices in $S$ take shape \eqref{stup} in the  basis $$\{v_1, v_2, \ldots, v_{n-m+1}, v_{n-m+2}, \ldots, v_n \}.$$

The main difference from the case $q>2$ is  that if $S_0$ is the normaliser of a Singer cycle of $GL_3(2)$, then there is no $x, y \in GL_3(2)$ such that $S_0^{x} \cap S_0^{y}$ is contained in $RT(GL_3(2))$ (see Theorem \ref{irred}). Since all Singer cycles are conjugate in $GL_3(2),$ we assume that $S_0$ is the normaliser of $$
\left\langle
\begin{pmatrix}
0&0&1\\
1&0&0\\
0&1&1
\end{pmatrix}
 \right\rangle.
$$ If $$x_0=\begin{pmatrix}
0&1&0\\
1&0&1\\
0&0&1
\end{pmatrix},
$$ then 
\begin{equation*}
S_0 \cap S_0^{x_0} = P =  \left\{
\begin{pmatrix}
1 & 1& 1\\
0& 1 & 0\\
1 & 0 & 0
\end{pmatrix},
\begin{pmatrix}
0 & 0& 1\\
0& 1 & 0\\
1 & 1 & 1
\end{pmatrix},
\begin{pmatrix}
1 & 0& 0\\
0& 1 & 0\\
0 & 0 & 1
\end{pmatrix} \right\}. 
\end{equation*} 


By Lemma \ref{irred} for $n_i \ge 3$   there exist $x_i\in GL_{n_i}(2)$ such that $S_i \cap S_i^{x_i} \le Z(GL_{n_i}(2))$ if $S_i$ is not conjugate to $S_0$ and $S_i \cap S_i^{x_0} \le P$ if $S_i$ is conjugate to $S_0$. Notice that $Z(GL_{n_i}(2))=1.$  Let $x=\diag[x_k, \ldots, x_1]$, where $x_i$ is an identity matrix if $n_i = 2.$  Therefore, matrices in $S \cap S^x$ are upper triangular except for $(2 \times 2)$ and  $(3 \times 3)$ blocks on the diagonal. Let $y$ be the permutation matrix corresponding to the permutation $$(1,n)(2, n-1) \ldots (n/2, n/2+1)$$ if $n$ is even, and $$(1,n)(2, n-1) \ldots ((n-1)/2, (n+1)/2)$$ if $n$ is odd.

 Fix some $(3\times 3)$ block on the diagonal of matrices in $S$ such that $S_i=S_0$ is in this block, so $P$ is in this block in $S \cap S^x$.
 This block intersects  one, two or three blocks in $(S \cap S^x)^y.$  
If it intersects  at least two blocks, then the  matrices in $(S \cap S^x) \cap (S \cap S^x)^y$ have the following restriction  to the chosen $(3 \times 3)$ block:
$$
\begin{pmatrix}
*&*&0\\
*&*&*\\
*&*&*
\end{pmatrix}. 
$$ 
Since  the only such matrix in $P$ is the identity, every matrix in $(S \cap S^x) \cap (S \cap S^x)^y$ has the identity submatrix in this block.  

If the chosen $(3 \times 3)$ block intersects  a bigger block in $(S \cap S^x)^y$, then it must lie in the block with  $S_i \cap S_i^{x_i}$ for some $i$ such that $n_i>3$ and  all such matrices are scalar.

Let the chosen $(3 \times 3)$ block intersect another $(3 \times 3)$ block in $(S \cap S^x)^y$ which consists of matrices in $P$. By Lemma \ref{irred}, $b_{S_0}(GL_3(2))=3,$ so there exists $y_0$ such that $$S_0 \cap S_0^{x_0} \cap S_0^{y_0} \le Z(GL_{3}(2)).$$ Let $\tilde{y}=\diag[1, \ldots, 1, y_0, 1, \ldots, 1]$ where $y_0$ is in the chosen block. Therefore,
 $$(S \cap S^x) \cap (S \cap S^x)^{y\tilde{y}}$$
consists of matrices which are diagonal except, possibly, for $(2 \times 2)$ blocks.

The rest of the proof is  as in Theorem \ref{lem41}. 
\end{proof}

Theorem \ref{theorem} now follows by Theorems \ref{lem41} and \ref{lem42}.

\subsection{Solvable subgroups not contained in $\GL_n(q)$}
\label{sec412}

%\chapter{Graph Automorphisms}


Recall that $V=\mathbb{F}_q^n$ and let $\beta=\{v_1, \ldots, v_n\}$ be a basis of $V$. Let $\Gamma= \Gamma L _n(q)= GL_n(q) \rtimes \langle  \phi_{\beta} \rangle$
and $A=A(V)=A(n,q):=\Gamma \rtimes \langle \iota_{\beta} \rangle$ where $\iota_{\beta}$ is the inverse-transpose  map of $GL_n(q)$ with respect to $\beta.$  Our goal is to prove the following theorem.

\begin{T4}
Let $n\ge 3.$ If $S$ is a maximal solvable subgroup of $A(n,q)$ not contained in $\Gamma,$ then one of the following holds:
\begin{enumerate}
\item[$(1)$] $b_S(S \cdot SL_n(q))\le 4$;
\item[$(2)$] $(n,q)=(4,3)$, $S$ is the normaliser in $A(n,q)$ of the stabiliser in $\GL_n(q)$ of a $2$-dimensional subspace of $V$, $b_S(S \cdot SL_n(q))=5$ and $\Reg_S(S \cdot SL_n(q),5)\ge 5.$
\end{enumerate}
\end{T4} 

Before we start the proof, let us discuss the structure of a maximal solvable subgroup
$S$ of $A(n,q)$ and fix some notation. In this section, we assume  $S$ is not contained in $\Gamma.$

Consider the action of $\Gamma$ on the set $\Omega_1$ of subspaces of $V$ of dimension $m<n.$ This action is transitive and equivalent to the action of $\Gamma$ on the set
$$\Omega_1'=\{\mathrm{Stab}_{\Gamma}(U) \mid U<V, \dim U= m\}$$ by conjugation. Let $\Omega_2$ be the set of subspaces of $V$ of dimension $n-m$ and let  $$\Omega_2'=\{\mathrm{Stab}_{SL_n(q)}(W) \mid W<V, \dim W= n-m\}.$$ It is easy to see that $\Gamma$ acts on $\Omega=\Omega_1 \cup \Omega_2$ with orbits $\Omega_1$ and $\Omega_2$ (respectively, on $\Omega'=\Omega_1' \cup \Omega_2'$ with orbits $\Omega_1'$ and $\Omega_2'$). We can extend this action to an action of $A$ on $\Omega'$ by conjugation which is equivalent to the following action of $A$ on $\Omega$:  for $U,W \in \Omega \text{ and } \varphi \in A$,
$$ U\varphi =W \text{ if and only if } 
(\mathrm{Stab}_{\Gamma}(U))^{\varphi}=\mathrm{Stab}_{\Gamma}(W).$$ 
In particular, if $U= \langle v_{n-m+1}, v_{n-m+2}, \ldots, v_n \rangle$, then $U\iota_{\beta}=\langle v_1, \ldots, v_{n-m}\rangle:=U'$ of dimension $n-m.$ Moreover, if $\varphi=\iota_{\beta}g$ with $g \in \Gamma$ and $h \in \Gamma,$ then 
$$(Uh)\varphi=U\iota_{\beta}h^{\iota_{\beta}}g=U'h^{\iota_{\beta}}g.$$  So elements of $A\backslash \Gamma$ permute $\Omega_1$ and $\Omega_2.$

Let us now define the action of $A$ on the pairs of subspaces $(U,W)$ of $V$ with $\dim U=m \le n/2$ and $\dim W=n-m$ where either
\begin{itemize}
\item  $U<W$, or
\item $U\cap W= \{0\}.$
\end{itemize}
 Here we let $(U,W) \varphi =(U\varphi, W \varphi).$ The pair $(U, W)$ is not ordered, but for convenience we usually list first the subspace of smaller dimension.  Notice that this action is equivalent to the action of $A$ by conjugation on 
 \begin{itemize}
\item {$\{\mathrm{Stab}_{\Gamma}((U,W)) \mid U \le W < V, \dim U=m, \dim W=n-m\}$};
\item $\{\mathrm{Stab}_{\Gamma}((U,W)) \mid U , W < V, U \cap W = \{0\}, \dim U=m, \dim W=n-m\}$ 
 \end{itemize}
respectively
where $\mathrm{Stab}_{\Gamma}((U,W))=\mathrm{Stab}_{\Gamma}(U) \cap \mathrm{Stab}_{\Gamma}(W).$ 



\begin{Def}
\label{deftypepm}
Let $M$ be the stabiliser in $A$ of a pair $(U,W)$ of subspaces of $V$ where $\dim U=m \le n/2$, $\dim W=n-m$. Assume that $M$ is a maximal subgroup of $A.$
\begin{itemize}
\item If $U \le W$, then $M$ is a maximal subgroup {\bf of type} $P_{m,n-m};$
\item If $U\cap W=\{0\}$, then $M$ is a maximal subgroup {\bf of type} $GL_m(q) \oplus GL_{n-m}(q).$
\end{itemize}
\end{Def}
Let $S$  be a maximal solvable subgroup of $A$ not contained in $\Gamma$ such that $S \le M$ where $M$ is a maximal subgroup of $A$ contained in Aschbacher's class $\mathit{C_1}.$ By \cite[\S 4.1]{kleidlieb}, $M$ is as in Definition \ref{deftypepm}.
In \cite[\S 4.1]{kleidlieb} the type $P_{m,n-m}$ is used only when $m<n/2;$ when $m=n/2$ such $M$ are labelled $P_m$. We let $m \le n/2$ and use the label $P_{m,n-m}$ since it allows us to use more uniform statements.

%Let us discuss some structure of $S$. Let $m_1$ be the minimal such that there exist a maximal subgroup $M_1 \le A$ of type $P_{m_1, n-m_1}$ or $GL_{m_1}(q) \oplus GL_{n-m_1}(q).$ 
Let us fix $M\ge S$ as above and let $\beta=\{v_1, \ldots, v_n\}$ be a basis of $V$ such that $U=\langle v_{n-m+1}, \ldots, v_n \rangle$ and 
$$W= \begin{cases}
\{v_{m+1}, \ldots, v_n\} &\text{ if } M \text{ is of type } P_{m, n-m};\\
\{v_1, \ldots, v_{n-m}\} &\text{ if } M \text{ is of type } GL_{m}(q) \oplus GL_{n-m}(q).
\end{cases} 
$$ 
For such $\beta$ we say that it is {\bf associated} with $(U,W).$
 By Definition \ref{deftypepm}, $M$ is the normaliser of $\mathrm{Stab}_{\Gamma}(U,W)$ in $A$. Therefore, with respect to $\beta$,
\begin{equation}
\label{Mstr}
M =\begin{cases}
\mathrm{Stab}_{\Gamma}(U,W) \rtimes \langle \iota_{\beta}a(n,m) \rangle & \text{ if } M \text{ is of type } P_{m, n-m};\\
\mathrm{Stab}_{\Gamma}(U,W) \rtimes \langle \iota_{\beta} \rangle & \text{ if } M \text{ is of type } GL_{m}(q) \oplus GL_{n-m}(q)
\end{cases}
\end{equation}
where $a(n,m) \in GL_n(q)$ is 
\begin{equation}
\label{adef}
\begin{pmatrix}
0 & 0 & I_{m} \\
0 & I_{n-2m} & 0 \\
I_{m} & 0 & 0
\end{pmatrix}.
\end{equation}

\begin{Lem}
\label{GRmmim2}
Let $n\ge 3$ and let $S \le A$ be as above. If $m$ is the least integer such that $S$ lies in $M$ as in \eqref{Mstr}, then 
 $\tilde{S}=(S \cap \Gamma)$ acts  on $U$ irreducibly.
\end{Lem} 
\begin{proof}
Let $\beta$ and $M$ be as in \eqref{Mstr} and let $P$ be $\mathrm{Stab}_{\Gamma}(U,W),$ so 
$$M=P \rtimes \langle \iota_{\beta} a\rangle$$
where $a$ is $a(n,m)$ if $M$ is of type $P_{m,n-m}$ and $a=I_n$ otherwise.


Assume that $\tilde{S}$ acts  reducibly on $U$, so  $\tilde{S}$ stabilises $U_1 <U$ of dimension $s<m.$  Assume that $U_1$ is a minimal such subspace, so $\tilde{S}$ stabilises no proper non-zero subspace of $U_1.$ Let $\varphi \in S \backslash \tilde{S}$ be such that
$S=\langle \tilde{S}, \varphi \rangle$, so $\varphi=\iota_{\beta}a \cdot g$ with $g \in P.$  Hence $\tilde{S}$ stabilises $W_1=U_1 \varphi$ of dimension  $n-s$. Notice that $U_1<W_1$ if $M$ is of type $P_{m,n-m}$ and $W_1 \cap U_1=0$ otherwise.
Since $\varphi^2 \in \Gamma,$ we obtain $\varphi^2 \in \tilde{S}$ and   $$W_1\varphi=U_1\varphi^2=U_1.$$
Therefore, $S$ normalises the stabiliser of $(U_1, W_1)$ in $\Gamma,$ so $S$ lies in a maximal subgroup of $A$ of type $P_{s, n-s}$ or $GL_{s}(q) \oplus GL_{n-s}(q)$  which contradicts the assumption of the lemma. 
\end{proof}

Notice that if we take $U=W=V$, then the proof of Lemma \ref{GRmmim2} implies the following statement.
\begin{Lem}
\label{GRirred2}
If $S \le A$ is not contained  in a maximal subgroup of $A$ from the class $\mathit{C_1}$, then $\tilde{S}=(S \cap \Gamma)$ acts irreducibly on $V$.
\end{Lem}

\begin{Th}
Theorem {\rm \ref{theoremGR}} holds if $S$ is not contained  in a maximal subgroup of $A$ from the class $\mathit{C_1}$.
\end{Th}
\begin{proof}
 Lemmas \ref{GammairGL}  and \ref{GRirred2}  imply that $\hat{S}:=S \cap GL_n(q)$ lies in an irreducible solvable subgroup of $GL_n(q)$. By Theorem \ref{irred}, either there exists $x \in SL_n(q)$ such that $\hat{S} \cap \hat{S}^x \le GL_n(q)$, or $\hat{S}$ lies on one of the groups in $(4)$ -- $(5)$ of Theorem \ref{irred}. In the latter case the statement is verified by computation. 

Otherwise, $\overline{S} \cap \overline{S}^{\overline{x}}$ is an abelian subgroup of $G=(S \cdot SL_n(q))/Z(GL_n(q))$ where $\overline{\phantom{G}}:S \cdot SL_n(q) \to G$ is the natural homomorphism.  By Theorem \ref{zenab}, there exist $\overline{y} \in G$ such that $\overline{S} \cap \overline{S}^{\overline{x}} \cap (\overline{S} \cap \overline{S}^{\overline{x}})^{\overline{y}}=1.$ So $b_S(S \cdot SL_n(q)) \le 4$ and the statement follows. 
\end{proof}

For the rest of the section we assume that $S$ lies in a maximal subgroup of $A$ from the class $\mathit{C_1}$.

 Let $m_1$ be minimal such that $S$ lies in a maximal subgroup $M_1<A$ of type $P_{m_1, n-m_1}$ or $GL_{m_1}(q) \oplus GL_{n-m_1}(q)$ stabilising subspaces $(U_1,W_1)$. Let $V_1$ be $V$;  let $V_2=W_1/U_1$ and $n_2=\dim V_2$. Let $\beta$ be a basis associated with $(U_1,W_1).$ Notice that elements from $\mathrm{Stab}_{\Gamma}(U_1,W_1)$ induce semilinear transformations on $V_2$, and $\iota_{\beta}$ induces the inverse-transpose map on $GL(V_2).$  Hence there exists a homomorphism 
$$\psi: S \to A(V_2) $$ mapping $x \in S$ to the element it induces on $V_2.$ Denote $\psi(S)$ by $S|_{_{V_2}}.$ Let $m_2$ be minimal such that $S|_{_{V_2}}$ lies in a maximal subgroup $M_2<A(V_2)$ of type $P_{m_2, n_2-m_2}$, or $GL_{m_2}(q) \oplus GL_{n_2-m_2}(q)$ stabilising subspaces $(U_2/U_1,W_2/U_1)$ of $V_2$.

 Repeating the arguments above we obtain a chain of subspaces 
\begin{equation}
\label{GRUs}
 0=U_0<U_1< \ldots < U_k<V \text { with } \dim U_i/U_{i-1} =m_i \text{ for } i \in\{1 \ldots, k\},
\end{equation} 
 subspaces $W_i$  for $i \in \{0, 1, \ldots, k\}$ where $W_0=V$, and groups $S|_{V_i}$ where $V_i=W_{i-1}/U_{i-1}$ for $i \in \{1, \ldots, k+1\}.$ Here $S|_{V_{i}}$ stabilises $(U_{i}/U_{i-1}, W_{i}/U_{i-1})$ for $i\in \{1, \ldots, k\}$ and $S|_{V_{k+1}}$ stabilises no subspace of $V_{k+1}.$ If $U_k=W_k,$ then $V_{k+1}$ is a zero space and  $S|_{V_{k+1}}$ is trivial.  Let $\beta=\{v_1, \ldots, v_n\}$ be a basis of $V$ such that  $\beta_i=\{v_1 +U_{i-1}, \ldots, v_n +U_{i-1}\} \cap V_i$ is a basis associated with $(U_{i}/U_{i-1}, W_{i}/U_{i-1})$ for $i\in \{1, \ldots, k\}.$

Let $\varphi \in S$. Hence $\varphi|_{_{V_i}}$ lies in $M_i$ for $i=1, \ldots, k$, so $$\varphi|_{_{V_i}}= (\iota_{\beta_i} a_i)^l \cdot g_i \text{ with } l\in \{0,1\} \text{ and } g_i \in \mathrm{Stab}_{\Gamma L (V_i)}((U_{i}/U_{i-1}, W_{i}/U_{i-1})),$$
where $a_i$ is $a(n_i,m_i)$  as in \eqref{adef} if $M_i$ is of type $P_{m_i, n_i-m_i}$ and $a_i$ is $I_{n_i}$ otherwise. Let $a \in GL_n(q)$ be such that  $a|_{_{V_k}}=a_k$ and 
\begin{equation}
\label{adefGR}
a|_{_{V_i}}= \begin{cases}
\begin{pmatrix}
a|_{_{V_{i+1}}} & 0\\
0 & I_{m_{i}}
\end{pmatrix} &\text{ if } M_i \text{ is of type } GL_{m_i}(q) \oplus GL_{n_i-m_i}(q)\\
\begin{pmatrix}
0 & 0 & I_{m_i}\\
0 & a|_{_{V_{i+1}}} & 0\\
I_{m_i} & 0 & 0
\end{pmatrix} &\text{ if } M_i \text{ is of type } P_{m_i, n_i -m_i}
\end{cases}
\end{equation}
 for $i \in \{1, \ldots, k-1\}.$  Therefore, 
\begin{equation}
\label{GRvarphishape}
\varphi = (\iota_{\beta}a)^l \cdot (\phi_{\beta})^j \cdot g,
\end{equation}
where $g \in GL_n(q)$ and $g|_{_{V_i}}$ stabilises $(U_{i}/U_{i-1}, W_{i}/U_{i-1}).$ More specifically, let $s \le k$ be the number of $i \in \{1, \ldots, k\}$ such that $M_i$ is of type $P_{m_i, n_i-m_i}$ and let $i_1 < \ldots < i_s$ be the corresponding $i$-s. Therefore,
\begin{equation}
\label{GRguppdiag}
g= \begin{pmatrix}
g_{i_1} '&* &\ldots &\ldots &\ldots & &* \\
 & \ddots &* &\ldots &\ldots& \ldots &* \\
 & &g_{i_s}' &* &\ldots & &* \\
 & & &g_{k+1} &* & &* \\
 & & & & g_k &* & *\\
 & & & & & \ddots &*\\
& & & & & & g_1
\end{pmatrix}
\end{equation} 
where $g_i, g_i' \in GL_{m_i}(q)$ and $g_{k+1} \in GL_{n_{k+1}}(q).$


\begin{example}
Let $k=3,$ and let $M_1$ and $M_3$ be of types $P_{m_i, n_i-m_i}$ for $i=1$ and $i=3$ respectively. Let $M_2$ be of type $GL_{m_2}(q) \oplus GL_{n_2-m_2}(q).$ Then 
$$
a=\begin{pmatrix}
 & & & & &  I_{m_1} \\
 &{\cellcolor{gray!50}} &{\cellcolor{gray!50}} &{\cellcolor{gray!50}} I_{m_3} &{\cellcolor{gray!20}} & \\
 & {\cellcolor{gray!50}} &{\cellcolor{gray!50}} I_{n_3-2m_3} &{\cellcolor{gray!50}} &{\cellcolor{gray!20}} & \\
 &{\cellcolor{gray!50}} I_{m_3} &{\cellcolor{gray!50}} & {\cellcolor{gray!50}} &{\cellcolor{gray!20}} & \\
 & {\cellcolor{gray!20}} & {\cellcolor{gray!20}} &{\cellcolor{gray!20}} &{\cellcolor{gray!20}} I_{m_2} & \\
I_{m_1} & & & & & 
\end{pmatrix}
$$
and $$\varphi = (\iota_{\beta}a)^l \cdot (\phi_{\beta})^j \cdot \begin{pmatrix}
 g_1 '&* &* &* &* &* \\
 & g_3' &* &* &0 &* \\
 & &g_4 &* &0 &* \\
 & & &g_3 &0 &* \\
 & & & & g_2 &* \\
 & & & & & g_1
\end{pmatrix}, $$
where $g_1, g_1' \in GL_{m_1}(q),$ $g_2 \in GL_{m_2}(q),$ $g_3, g_3' \in GL_{m_3}(q),$ $g_4 \in GL_{n_3-2m_3}(q).$
\end{example}

\begin{Lem}
\label{GRdiag}
Let $n \ge 3$. Let  $S$ be a maximal solvable subgroup of $A$. Assume that $S$ is contained in a maximal subgroup of $A$ of type $P_{m,n-m}$ or $GL_m(q) \oplus GL_{n-m}(q)$ for some $m \le n/2.$ Let $\beta$ be as described after \eqref{GRUs}.
\begin{enumerate}
\item[$(1)$] If none of $(m_i, q)$ and $(n_i,q)$ lies in $\{(2,2), (2,3)\},$ then there exist $x, y \in SL_n(q)$ such that if $\varphi \in (S \cap S^x \cap S^y) \cap \Gamma,$ then 
\begin{equation}
\label{GRvarphdiagc1}
\varphi=(\phi_{\beta})^j \cdot \diag[g_{i_1}, \ldots,g_{i_s}, g_{k+1}, g_k, \ldots,  g_1]=(\phi_{\beta})^j \cdot \diag(\alpha_1, \ldots, \alpha_n)
\end{equation}
 where $\alpha_i \in \mathbb{F}_q^*$ and $j \in \{0,1, \ldots, f-1\}.$ Moreover, $g_i, g_i' \in Z(GL_{m_i}(q))$ for $i \in \{1, \ldots, k\}$ and $g_{k+1} \in Z(GL_{n_{k+1}}(q))$.
\item[$(2)$] If $q \in \{2,3\}$ and at least one of $m_i$ or $n_i$ is $2$, then there exist $x,y \in SL_n(q)$ such that if $g \in S \cap S^x \cap S^y \cap GL_n(q)$, then $$g=\diag[g_{i_1}, \ldots,g_{i_s}, g_{k+1}, g_k, \ldots,  g_1]$$ with $g_i, g_i' \in GL_{m_i}(q)$ for $i \in \{1, \ldots, k\},$ $g_{k+1} \in GL_{n_{k+1}}(q)$, and one of the following holds:
\begin{enumerate} 
\item[$(2a)$]  for each $i \in \{1, \ldots, k+1\}$, either  $g_i$, $g_i'$ are  scalar matrices over $\mathbb{F}_q$ or $m_i=2$ and  $g_i$, $g_i'$ are upper-triangular matrices in $GL_2(q);$
\item[$(2b)$] there exist exactly one $j \in \{1, \ldots, k+1\} \backslash \{i_1, \ldots, i_s\}$ such that $g_j \in GL_2(q),$ and $g_i$, $g_i'$ are scalar for $i \in \{1, \ldots, t\} \backslash \{j\}.$
\end{enumerate}
\end{enumerate}
\end{Lem}
\begin{proof}
We start with the proof of $(1),$ so neither $(m_1,q)$ nor $(n_2,q)$ lies in $\{(2,2), (2,3)\}.$ 
Let us fix $q$ and assume that $n$ is minimal such that there exists $S\le A$ for which the statement of the lemma does not hold. Let $K$ be $S|_{_{U_1}} \le A(U_1)$ and let $R$ be $S|_{_{V_2}} \le A(V_2).$

We claim that there exist $x_2, y_2 \in SL_{n_2}(q)$ such that $(1)$ holds for $R \le A(V_2)$. Indeed, if $k>1$, then $R$ stabilises $(U_2,W_2)$ and $R$ is not a counterexample to the lemma since $n_2<n.$ If $k=1$, then $\tilde{R}=R \cap \Gamma$ acts irreducibly on $V_2$ by Lemma \ref{GRirred2}.  Hence $R \cap GL_{n_2}(q)$ lies in an irreducible maximal solvable subgroup of $GL_{n_2}(q)$ by  Lemma \ref{lem41}. By Theorem \ref{irred}, there exist $x_2,y_2 \in SL(V_2)$ such that $\tilde{R} \cap \tilde{R}^{x_2} \cap \tilde{R}^{y_2}= Z(GL(V_2)).$ Now the claim follows by Lemma \ref{scfield}.

By the same argument, $\tilde{K}=K \cap \Gamma L_{m_1}(q)$ acts irreducibly on $U_1$, and $K \cap GL_{m_1}(q)$ lies in an irreducible maximal solvable subgroup of $GL_{m_1}(q)$.

Our proof of $(1)$ splits into two cases: when $U_1\cap W_1=\{0\}$ and $U_1 \le W_1$ respectively.
\medskip

{\bf Case (1.1).} Let $U_1 \cap W_1= \{0\},$ so $V=U_1 \oplus W_1$, $n_2=n-m_1$ and $S$ lies in a maximal subgroup $M_1$ of $A$ of type $GL_{m_1}(q) \oplus GL_{n-m_1}(q).$ Recall that, by \eqref{Mstr}, elements in $M_1$ have shape 
\begin{equation}
\label{GROpleldiag}
(\iota_{\beta})^l \cdot (\phi_{\beta})^j \cdot \diag[h_2, h_1]
\end{equation}
with $l \in \{0,1\}$, $j \in \{0, 1, \ldots, f-1\}$, $h_2 \in GL_{n_2}(q)$ and $h_1 \in GL_{m_1}(q).$   By Theorem \ref{irred}, there exist $x_1, y_1 \in GL_{m_1}(q)$ such that 
$$K \cap K^{x_1} \cap K^{y_1} \cap GL_{m_1}(q) \le Z(GL_{m_1}(q)).$$ So, by Lemma \ref{scfield}, we can assume that $$K \cap K^{x_1} \cap K^{y_1} \cap  \Gamma L_{m_1}(q) \le \langle \phi_{\beta_1} \rangle Z(GL_{m_1}(q)),$$ where $\beta_1$ is the basis of $U_1$ consisting of the last $m_1$ vectors of $\beta.$ Let $x=\diag[x_2, x_1]$ and let $y=\diag[y_2, y_1],$ so $x,y \in SL_n(q).$

Consider $h \in S \cap S^x \cap S^y \cap \Gamma.$ By \eqref{GROpleldiag},
$$h= (\phi_{\beta})^j \cdot \diag[h_2,h_1],$$ so
$$h|_{_{V_2}}=(\phi_{\beta_2})^j \cdot h_2 \le (R \cap R^{x_2} \cap R^{y_2}) \cap \Gamma L_{n_2}(q)$$ and 
$$h|_{_{U_1}}=(\phi_{\beta_1})^j \cdot h_1 \le (K \cap K^{x_1} \cap K^{y_1}) \cap \Gamma L_{m_1}(q),$$ 
where $\beta_2=\beta \backslash \beta_1.$ Therefore, $h_1$ is a scalar matrix and  $h_2$ is as $\varphi$ in \eqref{GRvarphdiagc1}, so $h$ has the shape claimed by the lemma.

\medskip

{\bf Case (1.2).} Let $U_1 \le W_1,$ so $n_2=n-2m_1$ and $S$ lies in a maximal subgroup $M_1$ of $A$ of type $P_{m_1, n-m_1}.$ Recall that, by \eqref{Mstr}, elements in $M_1$ have shape 
\begin{equation}
\label{GRPmnmdiag}
(\iota_{\beta} a)^l \cdot (\phi_{\beta})^j \cdot 
\begin{pmatrix}
h_1' & * & *\\
 0   & h_2 & *\\
 0   & 0 & h_1
\end{pmatrix}
\end{equation}
with $l \in \{0,1\}$, $j \in \{0, 1, \ldots, f-1\}$, $h_2 \in GL_{n_2}(q)$ and $h_1, h_1' \in GL_{m_1}(q).$ Let $K'$ be the restriction of $S$ on $V/W_1$ and let $\hat{K}$ and $\hat{K'}$ be $K \cap GL_{m_1}(q)$ and $K' \cap GL_{m_1}(q)$ respectively.

 Recall that $S=\langle \tilde{S}, \varphi \rangle,$ where $\varphi = \iota_{\beta} a \cdot g$ with $g \in \tilde{M_1}=M_1 \cap \GL_n(q),$ so 
$$g= (\phi_{\beta})^{j_g} \cdot 
\begin{pmatrix}
g_3 & * & * \\
0  & g_2 & *\\
0 & 0 & g_1 \\
\end{pmatrix}
$$ with $g_1, g_3 \in GL_{m_1}(q),$ $g_2 \in GL_{n_2}(q)$ and $j_g \in \{0,1, \ldots, f-1\}.$ 

Since $(S \cap GL_n(q))^{\varphi}=S \cap GL_n(q),$  $$\hat{K}'=(\hat{K}^{\top})^{\phi^{j_g} g_3}.$$ Recall that $\hat{K}$ lies in an irreducible maximal solvable subgroup $T$ of $GL_{m_1}(q).$  Let $T'=(T^{\top})^{\phi^{j_g} g_3},$ so $\hat{K'}\le T'.$ 

By Theorem \ref{irred}, there exist $x_1, y_1 \in SL_{m_1}(q)$ such that $$T \cap T^{x_1} \cap T^{y_1}\le Z(GL_{m_1}(q)).$$ Let $x_1'= ((x_1^{-1})^{\top})^{\phi^{j_g} g_3}$ and let $y_1'= ((y_1^{-1})^{\top})^{\phi^{j_g} g_3}$,  so $$T' \cap T'^{x_1'} \cap T'^{x_1'} \le Z(GL_{m_1}(q)).$$  If $T$ is not as $S$ in  $(1)$ -- $(5)$ of Theorem \ref{irred}, then we assume $y_1=y_1'=I_{m_1}.$ So, by Lemma \ref{scfield}, we can assume that 
\begin{align*}
K \cap K^{x_1} \cap K^{y_1} \cap \Gamma L_{m_1}(q) \le \langle \phi_{\beta_1} \rangle Z(GL_{m_1}(q));\\
K' \cap (K')^{x_1} \cap (K')^{y_1} \cap \Gamma L_{m_1}(q) \le \langle \phi_{\beta_3} \rangle Z(GL_{m_1}(q)),
\end{align*}
 where $\beta_1$ is the basis of $U_1$ consisting of the last $m_1$ vectors of $\beta$ and $\beta_{3}=\{v_1 +W_1, \ldots, v_{m_1} + W_1\}$ is a basis of $V/W_1.$


If $T$ is as $S$ in  $(1)$ -- $(5)$ of Theorem \ref{irred}, then we claim that $T$ and $T'$ are conjugate by an element of $GL_{m_1}(q).$ Indeed, if $T$ is as $S$ in  $(1), (2) , (4),$ then both $T$ and $T'$ are normalisers of Singer cycles which are conjugate by Lemma \ref{singconj}. If $T$ is as $S$ in  $(3),$ then $T$ and $T'$ are conjugate by \cite[\S 21, Theorem 6 $(1)$]{sup}.  If $T$ is as $S$ in  $(5),$ then $T=(T^{\top})^{\phi^{j_g}},$ so $T'=T^{g_3}$ with $g_3 \in GL_{m_1}(q).$ Notice that $\Det(T)=\mathbb{F}_q^*$:  it is easy to check directly if $T$ is as $S$ in $(3)$ and $(5)$ of Theorem \ref{irred}; also recall that the determinant of a generator of a Singer cycle generates  $\mathbb{F}_q^*$. Hence, for a given $\lambda \in \mathbb{F}_q^*$, there exists $t \in GL_n(q)$ such that $\det(t)= \lambda$ and $T'=T^t.$



Let $x= \diag[x_1', x_2, x_1]$, let 
$$y=
\begin{pmatrix}
0 & 0 & t_2 y_1 \\
0 & y_2 & 0 \\
t_1 y_1'& 0 & 0
\end{pmatrix}$$
where $t_1, t_2 \in GL_{m_1}(q)$ are such that $\det(y)=1$, and 
$$T^{t_1}=T' \text{ and } (T')^{t_2}=T \text{ for } T \text{ as } S \text{ in } (1)-(5) \text{ of Theorem \ref{irred}.}$$
 So $x,y \in SL_n(q).$ 

Consider $h \in S \cap S^x \cap S^y \cap \Gamma.$ Using \eqref{GRPmnmdiag}, we obtain that elements in $S^y$ have shape 
\begin{equation}
\label{GRSy}
(\iota_{\beta} a)^l \cdot (\phi_{\beta})^j \cdot 
\begin{pmatrix}
h_1' & 0 & 0\\
 *   & h_2 & 0\\
 *   & * & h_1
\end{pmatrix}
\end{equation}
with $l \in \{0,1\}$, $j \in \{0, 1, \ldots, f-1\}$, $h_2 \in GL_{n_2}(q)$ and $h_1, h_1' \in GL_{m_1}(q).$ Therefore,  
by \eqref{GRPmnmdiag},
$$h= (\phi_{\beta})^j \cdot \diag[h_1',h_2,h_1],$$ so
\begin{align*}
h|_{_{V_2}} & =(\phi_{\beta_2})^j \cdot h_2 \le (R \cap R^{x_2} \cap R^{y_2}) \cap \Gamma L_{n_2}(q);\\  
h|_{_{U_1}} & =(\phi_{\beta_1})^j \cdot h_1 \le (K \cap K^{x_1} \cap K^{y_1}) \cap \Gamma L_{m_1}(q);\\
h|_{_{V/W_1}} & =(\phi_{\beta_3})^j \cdot h_1' \le (K' \cap (K')^{x_1'} \cap (K')^{y_1'}) \cap \Gamma L_{m_1}(q)
\end{align*} 
where $\beta_2=\{v_{m_1+1} +U_1, \ldots, v_{n-m_1}+U_1\}$ is a basis of $V_2.$ Therefore, $h_1$ and $h_1'$ are scalar matrices, and  $h_2$ is as $\varphi$ in \eqref{GRvarphdiagc1}, so $h$ has the shape claimed by the lemma.
 This concludes the proof of part $(1)$ of Lemma \ref{GRdiag}.

\bigskip

Now we start the proof of part $(2)$ of Lemma \ref{GRdiag}. So $q \in \{2,3\}$ and at least one of $m_i$ and $n_i$ is $2$ for $i \in \{1, \ldots, k\}$. Notice that $f=1$, so $\phi_{\beta}$ is trivial and $\Gamma =GL_n(q)$. Therefore, matrices in $S \cap \Gamma$ are block-upper-triangular and have shape \eqref{GRguppdiag} where 
\begin{align*}
g_i \in S_i & :=S|_{_{U_i/U_{i-1}}} \cap GL_{m_i} (q)  &&\text{ for } i\in \{1, \ldots, k\};\\
g_{k+1} \in S_k & := S|_{_{V_{n_{k+1}}}} \cap GL_{n_{k+1}} (q); &&\\
g_i \in S_i' & :=S|_{_{V_i/W_{i}}} \cap GL_{m_i} (q) && \text{ for } i\in \{i_1, \ldots, i_s\}.
 \end{align*}
Denote $GL_{m_i}(q)$ by $G_i$ for $i \in \{1, \ldots, k\}$ and $GL_{n_{k+1}}(q)$ by $G_{k+1}.$ By Lemma \ref{GRirred2}, $S_i$ and $S_i'$ are irreducible solvable subgroup of $G_i$ for $i \in \{1, \ldots, k+1\}$. By Theorem \ref{irred}, for $i \in \{1, \ldots, k+1\}$ one of the following holds:
\begin{enumerate}[label=(\roman*)]
\item $G_i=GL_2(q);$ \label{RGlistA} 
\item there exists $x_i \in G_i$ of determinant $1$ such that $S_i \cap S_i^{x_i} \le Z(G_i);$ \label{RGlistB}
\item $(G_i, S_i) \in \{(GL_4(3),GL_2(3) \wr \Sym(2)), (GL_3(2), N_{GL_3(2)}(T))\}$ where $T$ is a Singer cycle of $GL_3(2)$ and there exists $x_i, z_i \in G_i$ such that $S_i \cap S_i^{x_i} \cap S_i^{z_i} \le Z(G_i)$.\label{RGlistC}
\end{enumerate}
The same statement is true for $S_i'$ and we denote corresponding conjugating elements by $x_i'$ and $z_i'$. If $G_i=GL_2(q)$, then we take $x_i=x_i'=I_2.$

 If condition \ref{RGlistC}  holds for $S_i,$ then, by Theorem \ref{irred}, as discussed in {\bf Case~(1.2)}, $S_i'$ is conjugate to $S_i$  and for a given $\lambda \in \mathbb{F}_q^*$ there exists  $t \in G_i$ such that $\det(t)=1$ and $S_i'=S_i^t.$ 

Let us define $y_i, y_i' \in G_i$ for $i \in \{1, \ldots, k+1\}.$ % Assume first that $G_i=GL_4(3)$ and $S_i=GL_2(3) \wr \Sym(2).$ 
 If $M_i$ is of type $GL_{m_i}(q) \oplus GL_{n_i-m_i}(q)$ or $i=k+1$ then let $y_i \in G_i$ be such that $S_i \cap S_i^{x_i} \cap S_i^{y_i} \le Z(G_i).$ If $M_i$ is of type $P_{m_i, n_i-m_i},$ then let $y_i, y_i' \in G_i$ be such that 
\begin{align*}
S_i \cap S_i^{x_i} \cap (S_i')^{y_i'} & \le Z(G_i)\\
S_i' \cap (S_i')^{x_i'} \cap (S_i)^{y_i} & \le Z(G_i).
\end{align*} 

If conditions \ref{RGlistA} or \ref{RGlistB} hold for $S_i$ (respectively $S_i'$), then we take $y_i$ and $y_i'$ to be the identity matrix in $G_i.$ Let 
$$y=\diag[y_{i_1}', \ldots, y_{i_s}', y_{k+1}, \ldots, y_k] \cdot a.$$
If $\det (y) \ne 1$, then $q=3$ and $\det(y)=-1.$ If so, then we pick  $i \in \{1, \ldots, k+1\}$  such that $G_i=GL_2(3)$ and change $y_i$ to be $\diag(-1,1)$ which is equivalent to multiplying a line of $y$ by $-1.$  Therefore, we can assume that $y \in SL_n(q).$   

Let $r$ be the number of $G_i$ and $G_i'$ equal to $GL_2(q)$. Our proof of $(2)$ splits into two cases: 
when $r \ge 2$
and $r=1$ respectively.


\medskip

{\bf Case (2.1).}  
  Let    $(2 \times 2)$ blocks (corresponding to the $S_i$ and the $S_i'$ lying in $GL_2(q)$) on the diagonal in matrices of $\tilde{S}=S \cap \Gamma$ occur in the rows $$(j_1, j_1+1), (j_2, j_2 +1), \ldots, (j_r, j_r+1).$$ 
Let $\tilde{x}=\diag(\sgn(\sigma), 1, \ldots, 1) \cdot {\rm perm}(\sigma) \in SL_n(q),$ where  $$\sigma=(j_1, j_1+1, j_2, j_2 +1, \ldots, j_r, j_r+1)(j_1, j_1+1).$$  
Let $x= \diag[x_{i_1}',\ldots, x_{i_s}', x_{k+1}  \ldots, x_1]\tilde{x}$. Notice that $\det x=1.$ Calculations show that if $h \in \tilde{S} \cap \tilde{S}^x \cap \tilde{S}^y$, then 
$$h=\diag[h'_{i_1}, \ldots, h'_{i_s}, h_{k+1}, \ldots, h_k]$$
where
\begin{itemize}
\item $h_i \in S_i \cap S_i^{x_i} \le Z(G_i)$ (respectively $h_i' \in S_i' \cap (S_i')^{x_i} \le Z(G_i)$) if condition \ref{RGlistB} holds for $S_i$ (respectively $S_i'$);
\item $h_i \in S_i \cap S_i^{x_i} \cap (S_i')^{y_i'} \le Z(G_i)$ (respectively $h_i' \in S_i' \cap (S_i')^{x'_i} \cap (S_i)^{y_i} \le Z(G_i)$) if condition \ref{RGlistC} holds for $S_i$ (respectively $S_i'$);
\item $h_i \in GL_2(q)$ (respectively $h_i' \in GL_2(q)$) is upper-triangular if condition \ref{RGlistA} holds
$S_i$ (respectively $S_i'$).
\end{itemize}
 Therefore, $h_i$ and $h_i'$ are either scalar or upper-triangular, so  $(2a)$   of Lemma \ref{GRdiag} holds.

\medskip

{\bf Case (2.2).} Assume that the number $r$ of $G_i$ and $G_i'$ equal to $GL_2(q)$ is $1$. Notice that if $G_i'=GL_2(q)$, then $G_i=GL_2(q)$ and $r \ge 2.$ Hence there exists unique $j \in \{1, \ldots, k+1\} \backslash \{i_1, \ldots, i_s\}$ such that $G_j$ is $GL_2(q)$.  Let $y_j=I_2$  and let $x_i, x'_i, y_i, y'_i$ for $i \in \{1, \ldots, k\} \backslash \{j\}$  be as defined before {\bf Case (2.1)}. Let $x=\diag[x_{i_1}',\ldots, x_{i_s}', x_{k+1}  \ldots, x_1]$ and $y=\diag[y_{i_1}', \ldots, y_{i_s}', y_{k+1}, \ldots, y_k] \cdot a$. It is easy to see that $(2b)$   of Lemma \ref{GRdiag} holds.  
\end{proof}

Now we prove Theorem \ref{theoremGR}.

\begin{proof}[Proof of Theorem {\rm \ref{theoremGR}}]
Let $U=U_1$, $W=W_1$ and $m=m_1.$ If  $Q \le V$ has dimension $r$, then we write 
$$Q= \left \langle 
\begin{pmatrix}
u_1 \\
 \vdots \\
u_r
\end{pmatrix}
\right \rangle,$$
where $u_1, \ldots, u_r \in V$ form a basis of $Q$.

The proof splits into two cases: when $(1)$ and $(2)$ of Lemma \ref{GRdiag} holds respectively.

\medskip

{\bf Case 1.} Assume that $(1)$ of Lemma \ref{GRdiag} holds. We study two subcases:
\begin{description}[before={\renewcommand\makelabel[1]{\bfseries ##1}}]
\item[{\bf Case (1.1)}] $m_i=1$ for $i \in \{1, \ldots, k\}$ and $n_{k+1} \in \{0,1\}$;
\item[{\bf Case (1.2)}] $m_i\ge 2$ for some $i \in \{1, \ldots, k\}$ or $n_{k+1}\ge 2.$
\end{description}

\medskip

{\bf Case (1.1).} Assume that $m_i=1$ for $i \in \{1, \ldots, k\}$ and $n_{k+1} \in \{0,1\}.$ Let $x=a$ where $a$ is as in \eqref{adefGR}, so $\det x = \pm 1$. Notice that if $\varphi \in S^x,$ then $\varphi$  has shape \eqref{GRvarphishape} with 
\begin{equation}
\label{GRginsy}
g= \begin{pmatrix}
g_{i_1}' & & & & & & \\
* & \ddots & & & & &\\
* & *&g_{i_s}' &  & & &\\
* & &* &g_{i_{k+1}} & & & \\
* &\ldots & &* & g_{k} & & \\
* &\ldots &\ldots& \ldots &*  & \ddots &\\
* &\ldots &\ldots &\ldots & &* &  g_{1}
\end{pmatrix}.
\end{equation}
So, if $\varphi \in S \cap S^x,$ then $g$ is diagonal.

 Let $y ,z \in SL_n(q)$ be as in the proof of Proposition \ref{ni1}. Let us show that if $\varphi \in S \cap S^x \cap S^y$, then $\varphi \in S \cap \Gamma.$ Assume that $\varphi \notin \Gamma,$ so $l=1$ in  \eqref{GRvarphishape}.  Since $S$ stabilises $(U,W),$ $S^y$ stabilises $(U,W)y$. Therefore, $$(U,W)y \varphi =(U,W)y$$
and $Uy=Wy\varphi$ since $\dim Uy= \dim Wy\varphi.$ With respect to $\beta$,
$$Uy= \left \langle 
\begin{pmatrix}
1, \ldots, 1 \\
\end{pmatrix}
\right \rangle.$$ On the other hand, 
$$Uy=Wy \varphi =W y (\iota_{\beta} a) \cdot (\phi_{\beta})^j \cdot g =W \iota_{\beta} y^{\iota_{\beta}}a\cdot (\phi_{\beta})^j \cdot g= W'y^{\iota_{\beta}}a\cdot (\phi_{\beta})^j \cdot g,$$
where $W'$ is spanned by $\beta \backslash (W \cap \beta).$ 

 If $U \cap W = \{0\},$ then $W'=U.$ It is now easy to see that 
$$W'y^{\iota_{\beta}}a\cdot (\phi_{\beta})^j \cdot g = 
\begin{cases} \left \langle 
\begin{pmatrix}
1,0, \ldots, 0 \\
\end{pmatrix}
\right \rangle \text{ if } M_2 \text{ is of type } GL_1(q) \oplus GL_{n_2-1}(q); \\
\left \langle 
\begin{pmatrix}
0, \ldots,1, 0 \\
\end{pmatrix}
\right \rangle \text{ if } M_2 \text{ is of type } P_{1, n_2-1}
\end{cases} $$
since $g$ is diagonal. So, $Uy \ne Wy \varphi$ which is a contradiction, Hence $\varphi \in \Gamma.$

If $U\le W$, then $W'= \left \langle 
\begin{pmatrix}
1, 0, \ldots, 0 \\
\end{pmatrix}
\right \rangle$, $W'y^{\iota_{\beta}}= \left \langle 
\begin{pmatrix}
1, 0, \ldots, -1 \\
\end{pmatrix}
\right \rangle$ and  $$W'y^{\iota_{\beta}}a\cdot (\phi_{\beta})^j \cdot g=\left \langle 
\begin{pmatrix}
\alpha_1, 0, \ldots,0, \alpha_2 \\
\end{pmatrix}
\right \rangle$$ for some $\alpha_1, \alpha_2 \in \mathbb{F}_q^*.$ So, $Uy \ne Wy \varphi$ which is a contradiction. Hence $\varphi \in \Gamma.$

Therefore, $S\cap S^x \cap S^y \cap S^z = \tilde{S}\cap \tilde{S}^x \cap \tilde{S}^y \cap \tilde{S}^z \le Z(GL_n(q))$ by Proposition \ref{ni1}. Notice that if $\det a=-1$, then we can take $x= \diag(-1, 1, \ldots, 1) \cdot a$ and the argument above still works, so we can assume $x,y,z \in SL_n(q).$

\medskip

{\bf Case (1.2).}  Let $x,y$ be as in $(1)$ of Lemma \ref{GRdiag}. Assume that  $m_i\ge 2$ for some $i \in \{1, \ldots, k\}$ or $n_{k+1}\ge 2.$ So, if $\varphi \in (S \cap S^x \cap S^y) \cap  \Gamma$, then  there exists $r \in \{1, \ldots, n\}$ such that  $\alpha_r =\alpha_{r+1}$ in \eqref{GRvarphdiagc1}. We choose $r$ to be minimal such that $\alpha_r =\alpha_{r+1}$ for all such $\varphi.$

 By \eqref{GROpleldiag}, \eqref{GRPmnmdiag} and \eqref{GRSy}, if $\varphi \in S \cap S^y$, then 
\begin{equation}
\label{GRvarphisy}
\varphi = (\iota_{\beta} a )^l \cdot (\phi_{\beta}^j) \cdot g
\end{equation}
where $l \in \{0,1\},$ $j \in \{0, 1, \ldots, f-1\},$
$$a = \begin{cases}
I_n  &  \text{ if } M_1 \text{ is of type } GL_m(q) \oplus GL_{n-m(q)};\\
a(n,m) & \text{ if } M_1 \text{ is of type } P_{m, n-m}

\end{cases}
$$
and
$$g = 
\begin{cases}
\diag[g_2, g_1] &  \text{ if } M_1 \text{ is of type } GL_m(q) \oplus GL_{n-m(q)};\\
\diag[g_1', g_2, g_1] & \text{ if } M_1 \text{ is of type } P_{m, n-m}.
\end{cases}$$
Here $g_1 \in GL_m(q)$, $g_2 \in GL_{n-m}(q)$ in the first option and  $g_1, g_1' \in GL_m(q)$, ${g_2 \in GL_{n-2m}(q)}$ in the second. Our consideration of {\bf Case (1.2)} splits into two subcases:
when  $U \ne W$
and  $U=W$ respectively.

\medskip

{\bf Case (1.2.1).} Assume that $U \ne W$. Let $\theta$ be a generator of $\mathbb{F}_q^*$ and let $z \in GL_n(q)$ be defined as follows:
\begin{equation}
\label{GRzdef1}
\begin{aligned}
(v_i)z & = v_i &&\text{ for } i \in \{1, \ldots, n-m\}; \\
(v_{n-m+1})z & =  \sum_{i=1}^{n-m} v_i -v_r + \theta v_{r} + v_{n-m+1}; && \text{ if } r \le n-m\\
(v_{n-m+1})z & =  \sum_{i=1}^{n-m-1} v_i  + \theta v_{n-m} + v_{n-m+1}; && \text{ if } r \ge n-m+1\\
(v_i)z & =  v_{n-m} + v_{i} && \text{ for } i \in \{n-m+2, \ldots, n\}.
\end{aligned} 
\end{equation}

Let $\varphi \in S \cap S^x \cap S^y \cap S^z$, so $\varphi$ has shape \eqref{GRvarphisy}.   Assume that $\varphi \notin \Gamma,$ so $l=1.$ Since $\varphi \in S^z$, it stabilises $(U,W)z,$ so $(U,W)z \varphi =(U,W)z$ and, therefore, 
$$Uz=Wz \varphi = W'z^{\iota_{\beta}}a\cdot (\phi_{\beta})^j \cdot g,$$
where $W'$ is spanned by $\beta \backslash (W \cap \beta).$  

 With respect to $\beta$,
$$Uz= \left \langle 
\left(\begin{array}{ccccccc|ccccc}
1 & \ldots & 1 & \theta &1 & \ldots & 1     &             1 & & &  \\
1 & \ldots & 1 &  1    &1 & \ldots & 1       &           & 1 & &  \\
\vdots &   &   &       &  &        &\vdots    &          & &\ddots & \\
1 & \ldots & 1 &  1    &1 & \ldots & 1         &         & & & 1 
\end{array}\right)
\right \rangle$$
where $\theta$ in the first line is either in the $r$-th or $(m-n)$-th column, and the part after the vertical line forms $I_m.$

 If $U \cap W=\{0\},$ then $W'=U$ and it is easy to see that $Wz \varphi = W'z^{\iota_{\beta}}a\cdot (\phi_{\beta})^j \cdot g= U,$ since $g$ stabilises $U$ by \eqref{GRvarphisy}. So, $Uz \ne Wz \varphi$ which is a contradiction. Hence $\varphi \in \Gamma.$

If $U \le W$, then $W'= \langle (I_m \mid 0_{ m \times (n-m)}) \rangle$, so 
$$W'z^{\iota_{\beta}}a= \left \langle 
(A \mid 0_{n \times (n-2m)} \mid I_n)
\right \rangle$$
where $A$ is $m \times m$ matrix with entries $-1$ or $- \theta$, and $-\theta$ can occur at most once.
Therefore, $W'z^{\iota_{\beta}}a \cdot g = \langle (A g_1'\mid0_{m \times (n-2m)}\mid g_1) \rangle$. Notice that $n-2m\ge 1$ since $U \ne W.$ So, $Uz \ne Wz \varphi$ which is a contradiction. Hence $\varphi \in \Gamma.$

Therefore, $\varphi = (\phi_{\beta})^j \cdot \diag(\alpha_1, \ldots, \alpha_n)$ as in $(1)$ of Lemma
\ref{GRdiag}. Since $\varphi \in S \cap S^z \cap \Gamma,$ it stabilises $U$ and $Uz.$  

Consider $((v_{n-m+1})z) \varphi.$ First, let $r\le n-m,$ so %$m \ge 2$ and
\begin{equation*}((v_{n-m+1})z) \varphi=
\begin{cases}
\left(\sum_{\substack{i \in \{1, \ldots, n-m\}\\ i \ne r}} \alpha_i v_i \right) + \alpha_r \theta^{p^j} v_r + \alpha_{n-m+1} v_{n-m+1};\\
\sum_{i=1}^m \delta_i (v_{n-m+i})z
\end{cases}
\end{equation*}
for some $\delta_i \in \mathbb{F}_q.$ Since $((v_{n-m+1})z) \varphi$ has no $v_i$ for $i \in \{n-m+2, \ldots, n\}$ in the decomposition with respect to $\beta$, 
$$\delta_2 = \ldots = \delta_m =0$$
and $((v_{n-m+1})z) \varphi= \delta_1 ((v_{n-m+1})z).$ Hence 
$$\delta_1 = \alpha_r \theta^{p^j-1}=\alpha_{r+1} = \alpha_i \text{ for } i \in \{1, \ldots, n-m+1\} \backslash \{r,r+1\},$$
so $j=0$ and $\varphi \in Z(GL_n(q)).$

Now let $r\ge n-m+1,$ so $m\ge 2$ and 
\begin{equation*}((v_{n-m+1})z) \varphi=
\begin{cases}
\left(\sum_{i=1}^{n-m-1} \alpha_i v_i \right) + \alpha_{n-m} \theta^{p^j} v_{n-m} + \alpha_{n-m+1} v_{n-m+1};\\
\sum_{i=1}^m \delta_i (v_{n-m+i})z
\end{cases}
\end{equation*}
for some $\delta_i \in \mathbb{F}_q.$ Since $((v_{n-m+1})z) \varphi$ has no $v_i$ for $i \in \{n-m+2, \ldots, n\}$ in the decomposition with respect to $\beta$, 
$$\delta_2 = \ldots = \delta_m =0$$
and $((v_{n-m+1})z) \varphi= \delta_1 ((v_{n-m+1})z).$ Hence 
$$\delta_1 = \alpha_{n-m} \theta^{p^j-1}=\alpha_{n-m+1} = \alpha_i \text{ for } i \in \{1, \ldots, n-m-1\}.$$
The same arguments for $((v_{n-m+2})z) \varphi$ show that $\alpha_{n-m+2}=\alpha_{n-m},$ so, since $\alpha_{n-m+1}=\alpha_{n-m+2}$ by $(1)$ of Lemma \ref{GRdiag}, we obtain $\theta^{p^j-1}=1.$ Hence $j=0$ and $\varphi \in Z(GL_n(q)).$
 



\medskip
{\bf Case (1.2.2).} Assume that $U=W$, so $m=n/2$ and $M$ is of type $P_{n/2,n/2}.$ In particular, $n \ge 4,$ since $n \ge 3$ by the assumption of the theorem.  Let $\theta$ be a generator of $\mathbb{F}_q^*$ and let $z \in SL_n(q)$ be defined as follows: 
\begin{equation}
\label{GRzdef2}
\begin{aligned}
(v_i)z= &v_i &&\text{ for } i \in \{1, \ldots, n-m\}; \\
(v_{m-n+1})z= & \theta v_{n-m} + v_{m-n+1}; &&\\
(v_i)z= & v_{n-m} + v_{i} && \text{ for } i \in \{n-m+2, \ldots, n\}.
\end{aligned} 
\end{equation}
  Let $\varphi \in S \cap S^x \cap S^y \cap S^z$, so $\varphi$ has shape \eqref{GRvarphisy}.  Assume that $\varphi \notin \Gamma,$ so $l=1.$ Since $\varphi \in S^z$, it stabilises $(U,W)z,$ so $(U,W)z \varphi =(U,W)z$ and, therefore, 
$$Uz=Wz \varphi = W'z^{\iota_{\beta}}a\cdot (\phi_{\beta})^j \cdot g,$$
where $W'$ is spanned by $\beta \backslash (W \cap \beta).$  

 With respect to $\beta$,
$$Uz= \left \langle 
\left(\begin{array}{cccc|cccc}
0 & \ldots & 0 &  \theta & 1 & & &  \\
0 & \ldots & 0 &  1      &   & 1 & &  \\
\vdots &  &  & \vdots &  & &\ddots & \\
0 & \ldots & 0 & 1 &  & & & 1 
\end{array}\right)
\right \rangle$$
where the part after the vertical line forms $I_m.$

Observe $W'= \langle (I_m \mid 0_{m \times m}) \rangle$, so 
$$W'z^{\iota_{\beta}}a= \left \langle 
\left(\begin{array}{cccc|cccc}
0 & \ldots & 0 &  0 & 1 & & &  \\
\vdots &  &  & \vdots &   &\ddots & & \\
0 & \ldots & 0 &  0      &  &  & 1 &   \\
- \theta & -1 & \ldots  & -1 &  & & & 1 
\end{array}\right)
\right \rangle,$$
and $W'z^{\iota_{\beta}}a \cdot g = \langle A\mid g_1 \rangle$ where 
$$A=\left(\begin{array}{cccc}
0 & \ldots & 0 &  0   \\
\vdots &  &  & \vdots \\
0 & \ldots & 0 &  0   \\
\alpha_1 & \alpha_2 & \ldots  & \alpha_{m} 
\end{array}\right)
$$ is an $m \times (n-m)$ matrix with $\alpha_i \in \mathbb{F}_q.$ So $Uz \ne Wz \varphi$ which is a contradiction. Hence $\varphi \in \Gamma.$
Therefore, $\varphi = (\phi_{\beta})^j \cdot \diag(\alpha_1, \ldots, \alpha_n)$ as in $(1)$ of Lemma
\ref{GRdiag}. In particular, $\alpha_1 = \ldots = \alpha_m$ and $\alpha_{m+1}= \ldots = \alpha_n.$ The same arguments as in the {\bf Case (1.2.1)} applied to $((v_{m+1})z) \varphi$ and $((v_{m+2})z) \varphi$ shows that $\varphi \in Z(GL_n(q)).$

\bigskip

{\bf Case 2.}  Assume that $(2)$ of Lemma \ref{GRdiag} holds.
 For $n \in \{3,4\}$ the theorem follows by computation, so  we may assume that $n \ge 5$.


We adopt notation from the proof of Lemma \ref{GRdiag}, in particular $x,y$, $S_i$, $G_i$.
Let $(2 \times 2)$ blocks (corresponding to the $S_i$ and the $S_i'$ lying in $GL_2(q)$) on the diagonal in matrices of $\tilde{S}=S \cap \Gamma$ occur in the rows $$(j_1, j_1+1), (j_2, j_2 +1), \ldots, (j_r, j_r+1).$$ Let $\Lambda=\Lambda_1 \cup \Lambda_2$ where 
$\Lambda_1=\{j_1, j_2, \ldots, j_r\}$ and $\Lambda_2=\{j_1+1, j_2+1, \ldots, j_r+1\}.$
 Let $U=U_1$, $W=W_1$ and $m=m_1.$  
 Notice, that if $\varphi \in S^y$, then it has shape \eqref{GRvarphishape} where $g$ has shape \eqref{GRginsy} with $g_i, g_i' \in GL_{m_i}$ for $i \in \{1, \ldots, k\}$ and $g_{k+1} \in GL_{n_{k+1}}(q).$ So, if $\varphi \in S \cap S^y$, then it has shape \eqref{GRvarphishape} with 
 \begin{equation}
 \label{GRgdiagcase2}
 g = \diag[g_{i_1}', \ldots, g_{i_s}', g_{k+1}, g_k, \ldots, g_1].
 \end{equation}
 
 Our consideration of {\bf Case 2} splits into two subcases: when $(2a)$ and $(2b)$ of Lemma \ref{GRdiag} holds respectively.

\medskip

{\bf Case (2.1).} Assume that $(2a)$ of Lemma \ref{GRdiag} holds. We consider two subcases: when $m \ge 2$ and $m=1$.

{\bf Case (2.1.1).} Assume that $m \ge 2.$ Let $z \in SL_n(q)$ be defined as follows
\begin{equation}
\label{GRzdefq23}
\begin{aligned}
(v_i)z & =  v_i &&\text{ for } i \in \{1, \ldots, n-m\}; \\
%(v_{n-m+1})z= & \sum_{i =1}^{n-m} v_i + v_{n-m+1} && \text{ if } (a) \text{ holds};\\
%(v_{n-m+1})z= & \sum_{i \in \{1, \ldots, n-m\} \backslash \Lambda_1} v_i + v_{n-m+1} && \text{ if } (b) \text{ holds}; \\
(v_{n-m+1})z & =  \left(\underset{i \in \{1, \ldots, n-m\} \backslash \{j_1\}}{\sum} v_i \right) + v_{n-m+1};\\
(v_{n-m+2})z & =  \left(\underset{i \in \{1, \ldots, n-m\} \backslash \Lambda_2}{\sum} v_i \right)   + v_{n-m+2};\\
(v_i)z & =  \sum_{j=1}^{n-m} v_{j} + v_{i} && \text{ for } i \in \{n-m+2, \ldots, n\}.
\end{aligned} 
\end{equation}

Let $\varphi \in S \cap S^x \cap S^y \cap S^z$, so $\varphi$ has shape \eqref{GRvarphishape} where $g$ has shape \eqref{GRgdiagcase2}.   Assume that $\varphi \notin \Gamma,$ so $l=1.$ Since $\varphi \in S^z$, it stabilises $(U,W)z,$ so $(U,W)z \varphi =(U,W)z$ and, therefore, 
$$Uz=Wz \varphi = W'z^{\iota_{\beta}}a \cdot g,$$
where $W'$ is spanned by $\beta \backslash (W \cap \beta).$  

 With respect to $\beta$,
$$Uz= \left \langle 
\left(\begin{array}{ccc|ccccc}
\lambda_1 & \ldots & \lambda_{n-m}   &           1 & & & & \\
\mu_1     & \ldots & \mu_{n-m}       &           & 1 & & & \\
  1       & \ldots &  1              &           &   &1 & &  \\
\vdots    &        &\vdots           &           &  & &\ddots & \\
1         & \ldots &  1              &           &  & & & 1 
\end{array}\right)
\right \rangle$$
where $\lambda_i, \mu_i \in \{0,1\}$ according to \eqref{GRzdefq23}, so for each $i \in \{1, \ldots, n-m\}$ at least one of $\lambda_i$ and $\mu_i$ is $1$,  and the part after the vertical line forms $I_m.$

 If $U \cap W=\{0\},$ then $W'=U$ and it is easy to see that $Wz \varphi = W'z^{\iota_{\beta}}a\cdot (\phi_{\beta})^j \cdot g= U,$ since $g$ stabilises $U$ by \eqref{GRvarphishape}. So, $Uz \ne Wz \varphi$ which is a contradiction. Hence $\varphi \in \Gamma.$

If $U \le W$, then $W'= \langle (I_m  \mid  0_{(m \times n-m)}) \rangle$, so 
$$W'z^{\iota_{\beta}}a= \left \langle 
(A \mid 0_{m \times (n-2m)} \mid I_n)
\right \rangle$$
where $A$ is $m \times m$ matrix with entries $-1$ ,$-\lambda_i$ and $- \mu_i$.
Therefore, $$W'z^{\iota_{\beta}}a \cdot g = \langle (A g_1'\mid 0_{m \times (n-2m)}\mid g_1) \rangle.$$ Notice that $n-2m\ge 1,$ since otherwise  $U = W,$ so $m=2$ and $n=4.$ Thus, $Uz \ne Wz \varphi$ which is a contradiction. Hence $\varphi \in \Gamma.$

Therefore, $\varphi = g= \diag[g_{i_1}, \ldots,g_{i_s}, g_{k+1}, g_k, \ldots,  g_1]$ as in $(a)$ of $(2)$ of Lemma
\ref{GRdiag}. Specifically, let $(v_i)\varphi =\alpha_i v_i$ for $i \in \{1, \ldots, n\} \backslash \Lambda_1$ and let $(v_i)\varphi = \alpha_i v_i + \gamma_i v_{i+1}$ for $i \in \Lambda_1$ with $\alpha_i, \gamma_i \in \mathbb{F}_q.$

 Since $\varphi \in S \cap S^z \cap \Gamma,$ it stabilises $U$ and $Uz.$  
Therefore, $((v_{n-m+2})z) \varphi$ is
\begin{equation*}
\label{vnm2zph}
\left(\underset{i \in \{1, \ldots, n-m\} \backslash \Lambda_2}{\sum} \alpha_i v_i \right) +
\left(\underset{i \in \{1, \ldots, n-m\} \cap \Lambda_2}{\sum} \gamma_{i-1} v_i \right) + \alpha_{n-m+2} v_{n-m+2},
\end{equation*}
and
$$((v_{n-m+2})z) \varphi= \sum_{i=1}^m \delta_i (v_{n-m+i})z$$
for some $\delta_i \in \mathbb{F}_q.$ Since $((v_{n-m+2})z) \varphi$ does not contain $v_i$ for $i \in \{n-m+1,n-m+3, \ldots, n\}$ in the decomposition with respect to $\beta$, 
$$\delta_1= \delta_3 = \ldots = \delta_m =0$$
and $((v_{n-m+1})z) \varphi= \delta_1 ((v_{n-m+1})z).$ Hence 
$$\gamma_{j_1} = \ldots = \gamma_{j_r}=0,$$
so $\varphi= \diag(\alpha_1, \ldots, \alpha_n)$ where 
\begin{equation}
\label{GRalphasvnm2}
\alpha_i=\alpha_{n-m+2} \text{ for } i \in \{1, \ldots, n-m\} \backslash \Lambda_2.
\end{equation} 

Consider 
\begin{equation*}((v_{n-m+1})z) \varphi=
\begin{cases}
\underset{i \in \{1, \ldots, n-m\} \backslash \{j_1\}}{\sum} \alpha_i v_i + \alpha_{n-m+1} v_{n-m+1} +\underline{\gamma_{n-m+1} v_{n-m+1}}\\
\sum_{i=1}^m \delta_i (v_{n-m+i})z
\end{cases}
\end{equation*}
for some $\delta_i \in \mathbb{F}_q.$ The underlined part is present only if $m=2.$ Since $((v_{n-m+1})z) \varphi$ contains neither $v_{j_1}$ (notice that $j_1>n-m$ since if $(2a)$ of Lemma \ref{GRdiag} holds, then $r \ge 2$) nor $v_i$ for $i \in \{n-m+3, \ldots, n\}$ in the decomposition with respect to $\beta$, 
$$\delta_2 = \ldots = \delta_m =0.$$
Here $\delta_2=0$ since $((v_{n-m+2})z) $ contains $v_{j_1}$ in the decomposition with respect to $\beta$ and $((v_{n-m+1})z)\varphi$ does not.
Thus $((v_{n-m+1})z) \varphi= \delta_1 ((v_{n-m+1})z)$ and 
$$\alpha_i = \alpha_{n-m+1} \text{ for } i \in \{1, \ldots, n-m\} \backslash \{j_1\}.$$
Combined with \eqref{GRalphasvnm2}, it implies $\varphi \in Z(GL_n(q)).$ 

\medskip

{\bf Case (2.1.2).} Assume that $m_1=1$ and let $t$ be the smallest $i \in \{2, \ldots, k\}$ such that $m_i\ge 2.$ Such $t$ exists since otherwise $m_i=1$ for $i \in \{1, \ldots, k\}$ and $n_{k+1}=2,$ so $(2b)$ of Lemma \ref{GRdiag} holds. Let $d_i=\sum_{j=1}^i m_j$ for $i \in \{1, \ldots, k\}$. Let $z \in SL_n(q)$ be defined as follows:
\begin{equation}
\label{GRzdefq23mr}
\begin{aligned}
(v_i)z & =  v_i &&\text{ for } i \in \{1, \ldots, n-d_t\}; \\
(v_{n-d_t+1})z & =  \left(\underset{i \in \{1, \ldots, n-d_t\} \backslash \{j_1\}}{\sum} v_i \right) + v_{n-d_t+1};\\
(v_{n-d_t+2})z & =  \left(\underset{i \in \{1, \ldots, n-d_t\} \backslash \Lambda_2}{\sum} v_i \right)   + v_{n-d_t+2};\\
(v_i)z & =  v_i && \text{ for } i \in \{n-d_{t}+3, \ldots, n-1\};\\
(v_n)z & =  \sum_{i=1}^n v_i.
\end{aligned} 
\end{equation}
 Let $\varphi \in S \cap S^x \cap S^y \cap S^z$, so $\varphi= (\iota_{\beta}a)^l \cdot g$ where $g=\diag[g_{i_1},\ldots, g_{i_s}, g_{k+1}, \ldots, g_1]$ with $g_i$ and $g_i'$  as in \eqref{GRguppdiag}.  
 Assume that $\varphi \notin \Gamma,$ so $l=1.$ Since $\varphi \in S^z$, it stabilises $(U,W)z,$ so $(U,W)z \varphi =(U,W)z$ and, therefore, 
$$Uz=Wz \varphi = W'z^{\iota_{\beta}}a \cdot g,$$
where $W'$ is spanned by $\beta \backslash (W \cap \beta).$  

 With respect to $\beta$,
$$Uz= \langle (1, \ldots, 1) \rangle.$$


 If $U \cap W=\{0\},$ then $W'=U$ and it is easy to see that $Wz \varphi = W'z^{\iota_{\beta}}a\cdot g= U,$ since $g$ stabilises $U$. So $Uz \ne Wz \varphi$ which is a contradiction. Hence $\varphi \in \Gamma.$

If $U \le W$, then $W'= \langle (1 , 0,\ldots, 0) \rangle$, so 
$$W'z^{\iota_{\beta}}a \cdot g= \langle 
(1 , 0, \ldots, 0, -1,-1, 0 \ldots, 0,  1)a \cdot g
 \rangle$$
where $-1$ is in the ${n-d_t+1}$ and ${n-d_t+2}$ entries. Notice that 
\begin{equation}
\label{GR212zeros}
(1 , 0, \ldots, 0, -1,-1, 0 \ldots, 0,  1)=(u_{i_1}', \ldots, u_{i_s}', u_{k+1}, \ldots, u_1)
\end{equation}
where $u_i, u_i' \in \mathbb{F}_q^{m_i}$ for $i \in \{1, \ldots, k\}$ and  $u_{k+1} \in \mathbb{F}_q^{n_{k+1}}.$ Here $u_1=u_1'=1$, $u_t=(-1,-1)$ and all other $u_i, u_i'$ are zero vectors. There is at least one such zero vector in \eqref{GR212zeros} since $n \ge 5.$ Notice that,  for a given $i \in \{1, \ldots, k+1\},$  either $a$ fixes $u_i$ and $u_i'$ or $a$ permutes them. Hence 
\begin{align*}
(u_{i_1}', \ldots, u_{i_s}', u_{k+1}, \ldots, u_1)ag & =(w_{i_1}', \ldots, w_{i_s}', w_{k+1}, \ldots, w_1)g \\
& =(w_{i_1}'g_{i_1}', \ldots, w_{i_s}'g_{i_s}', w_{k+1}g_{k+1}, \ldots, w_1g_1)
\end{align*}
where at least one of $w_i$ and $w_i'$ (and hence at least one one of $w_ig_i$ and $w_i'g_i'$) equals to a zero vector. Thus, since $Uz$ contains no non-zero vector with zero entries, $Uz \ne Wz \varphi$ which is a contradiction. Hence $\varphi =g\in \Gamma.$

The same arguments as  in {\bf Case (2.1.1)}, now applied to $(v_{n-d_t+1})z$ and $(v_{n-d_t+2})z$ instead of $(v_{n-m+1})z$ and $(v_{n-m+2})z,$ show that $g=\diag(\alpha_1, \ldots, \alpha_n)$ for some $\alpha_i \in \mathbb{F}_q.$ Since $g \in S^z \cap \Gamma,$ it stabilises $Uz$ so
$$(v_n)z g = \sum_i^n \alpha_i v_i = \alpha (v_n)z= \alpha \sum_i^n v_i$$
for some $\alpha \in \mathbb{F}_q^*.$ Therefore, $\alpha=\alpha_1 = \ldots = \alpha_n$, $g$ is scalar and $\varphi \in Z(GL_n(q)).$

\bigskip

{\bf Case (2.2).} Assume that $(2b)$ of Lemma \ref{GRdiag} holds. We consider two subcases: when $m=2$ and $m \ne 2$.

\medskip

{\bf Case (2.2.1).} Assume that $m=2,$ so if $g \in S \cap S^x \cap S^y \cap GL_n(q),$ then $g_1 \in GL_2(q)$ and $g_i$, $g_i'$ are scalar for $i \in \{2, \ldots, k+1\}.$ We may assume $U \cap W=0$ since otherwise the number of $G_i$ and $G_i'$ equal to $GL_2(q)$ is at least 2 and $(2a)$  of Lemma \ref{GRdiag} holds. Let $z \in SL_n(q)$ be defined as follows:
\begin{equation}
\label{GRzdefcase221}
\begin{aligned}
(v_i)z & =  v_i &&\text{ for } i \in \{1, \ldots, n-2\}; \\
(v_{n-1})z & =  \sum_{i=2}^{n-2} v_i + v_{n-1};\\
(v_{n})z & =  v_1+ \sum_{i=3}^{n-2} v_i   + v_{n}.\\
\end{aligned} 
\end{equation}
Let $\varphi \in S \cap S^x \cap S^y \cap S^z$, so $\varphi= (\iota_{\beta}a)^l \cdot g$ where $g=\diag[g_{i_1},\ldots, g_{i_s}, g_{k+1}, \ldots, g_1]$ where $g_i$ and $g_i'$ are as in \eqref{GRguppdiag}.  
 Assume that $\varphi \notin \Gamma,$ so $l=1.$ Since $\varphi \in S^z$, it stabilises $(U,W)z,$ so $(U,W)z \varphi =(U,W)z$ and, therefore, 
$$Uz=Wz \varphi = W'z^{\iota_{\beta}}a \cdot g,$$
where $W'$ is spanned by $\beta \backslash (W \cap \beta).$  

 With respect to $\beta$,
$$Uz= \left \langle 
\left(\begin{array}{ccccc|cc}
0  & 1& 1 & \ldots & 1   &           1 &0  \\
1  & 0& 1     & \ldots & 1       &    0       & 1 
\end{array}\right)
\right \rangle.$$
 Since $U \cap W=\{0\},$ we obtain $W'=U$ and it is easy to see that $Wz \varphi = W'z^{\iota_{\beta}}a\cdot g= U,$ since $g$ stabilises $U$. So, $Uz \ne Wz \varphi$ which is a contradiction. Hence $\varphi =g\in \Gamma.$ Therefore, $g=\diag[A,g_1]$ where $A=\diag(\alpha_1, \ldots, \alpha_{n-2})$ for some $\alpha_i \in \mathbb{F}_q^*$ and $$g_1 = \left( \begin{matrix} \delta_1 & \delta_2 \\ \delta_3 & \delta_4  \end{matrix} \right) \in GL_2(q).$$ Since $g \in S^z$, it stabilises $Uz.$
 
 Consider 
\begin{equation}
\label{GRc221vn1z}
 ((v_{n-1})z)g =
 \begin{cases}
 \sum_{i=2}^{n-2} \alpha_i v_i + \delta_1 v_{n-1} + \delta_2 v_n;\\
 \lambda_1 (v_{n-1})z + \lambda_2 (v_{n})z
 \end{cases}
 \end{equation}
 for some $\lambda_{i} \in \mathbb{F}_q.$ Since there is no $v_1$ in the first line of \eqref{GRc221vn1z}, $\lambda_2=0,$ so
 $$\alpha_2 = \ldots = \alpha_{n-2}=\delta_1 \text{ and } \delta_2=0.$$
 The same arguments for $((v_{n})z)g$ show that $\delta_3=0$ and, since $n \ge 5$,
 $$\alpha_1=\alpha_{n-2}=\delta_4.$$
 Hence $g$ is scalar and $\varphi \in Z(GL_n(q)).$
 
 \medskip
 
 {\bf Case (2.2.2).} Assume $m\ne 2.$  
  We may assume that $n_{k+1}\ne 2$. Indeed, if $n_{k+1}=2$, then  $n_{k}=3$ since $m_k\le n_k/2$ and there is only one $G_i$ equal to $GL_2(q)$ for $i \in \{1, \ldots, k+1\}.$ Therefore, $$S_k=S|_{_{V_k}}\cap GL(V_k)=GL_2(q) \times GL_1(q) \le GL(V_k)=GL_3(q)$$ and, using computation, we obtain that  there are $x_k, y_k \in SL_n(q)$ such that $S_k \cap S_k^{x_k} \cap S_k^{y_k} \le Z(GL_3(q))$, so the conclusion of  $(1)$ of Lemma \ref{GRdiag} holds and the theorem  holds by {\bf Case  1}. 
  
  Therefore, $j_1\ge 3$ where $j_1$ is as defined in the beginning of {\bf Case  2}, so the $(2 \times 2)$ block (corresponding to the $S_i$ lying in $GL_2(q)$) on the diagonal in matrices of $\tilde{S}=S \cap \Gamma$ occurs in the rows $(j_1, j_1+1)$. Recall that if $h \in  S \cap S^x \cap S^y \cap  GL_n(q)$, then $$h= \diag[\alpha_1, \ldots, \alpha_{j_1-1},A, \alpha_{j_1+2}, \ldots, \alpha_n]$$ where $\alpha_i \in \mathbb{F}_q^*$ and $A= \left( \begin{smallmatrix} \delta_1 & \delta_2 \\ \delta_3 & \delta_4  \end{smallmatrix} \right) \in GL_2(q).$ Let $z \in SL_n(q)$ be defined as follows:
\begin{equation}
\label{GRzdefcase222}
\begin{aligned}
(v_i)z & =  v_i &&\text{ for } i \in \{1, \ldots, n\} \backslash \{j_1, j_1+1, n\}; \\
(v_{j_1})z & =  v_1 + v_{j_1};\\
(v_{j_1+1})z & =  v_2 + v_{j_1+1};\\
(v_{n})z & =   \sum_{i=1}^{n-m} v_i   + v_{n}.\\
\end{aligned} 
\end{equation}
Let $\varphi \in S \cap S^x \cap S^y \cap S^z$, so $\varphi= (\iota_{\beta}a)^l \cdot g$ with $g=\diag[g_{i_1},\ldots, g_{i_s}, g_{k+1}, \ldots, g_1]$ where $g_i$ and $g_i'$ are as in \eqref{GRguppdiag}.  
 Assume that $\varphi \notin \Gamma,$ so $l=1.$ Since $\varphi \in S^z$, it stabilises $(U,W)z,$ so $(U,W)z \varphi =(U,W)z$ and, therefore, 
$$Uz=Wz \varphi = W'z^{\iota_{\beta}}a \cdot g,$$
where $W'$ is spanned by $\beta \backslash (W \cap \beta).$  

 With respect to $\beta$,
$$Uz= \left \langle 
\left(\begin{array}{ccc|ccccc}
 0        & \ldots & 0               &           1 & & &  \\
\vdots    &        &\vdots           &           & \ddots  & &  \\
0         & \ldots & 0               &            & &\ddots &  \\
1         & \ldots &  1              &           &  & & 1 
\end{array}\right)
\right \rangle$$
where the part after the vertical line forms $I_m.$

 If $U \cap W=\{0\},$ then $W'=U$ and it is easy to see that $Wz \varphi = W'z^{\iota_{\beta}}a \cdot g= U,$ since $g$ stabilises $U$ by \eqref{GRvarphisy}. So $Uz \ne Wz \varphi$ which is a contradiction. Hence $\varphi \in \Gamma.$
 
 If $U \le W,$ then 
 \begingroup
\allowdisplaybreaks
 \begin{align*}
 Wz\varphi & = W'z^{\iota_{\beta}}ag \\ & = 
 \left\langle I_m\mid 0_{m \times (n-m)} \right\rangle \, z^{\iota_{\beta}}ag \\
& =\Scale[0.95]{\left \langle \left(\begin{array}{ccccc|cccccc|cccc}
1 &  &   &        &          & 0 & \ldots 0 &-1 & 0 & 0 & \ldots 0 &       0  & 0 & \ldots &  0 \\         
  &1 &   &        &          & 0 & \ldots 0 & 0 &-1 & 0 & \ldots 0 &       0  & 0 & \ldots &  0 \\
  &  &1  &        &          & 0 & \ldots 0 & 0 &0  & 0 & \ldots 0 &       -1 & 0 & \ldots &  0 \\
  &  &   & \ddots &          & \vdots &     &   &   &   &          &   \vdots &   &        &    \\
  &  &   &        &1         & 0 & \ldots 0 & 0 &0  & 0 & \ldots 0 &       -1 & 0 & \ldots &  0 \\ 
\end{array}\right)
\right \rangle}\, a g\\
& = \Scale[0.95]{ \left \langle \left(\begin{array}{cccc|cccccc|ccccc}
 0  & 0 & \ldots &  0          & 0 & \ldots 0 &-1 & 0 & 0 & \ldots 0 &       1 &  &   &        & \\         
 0  & 0 & \ldots &  0        & 0 & \ldots 0 & 0 &-1 & 0 & \ldots 0 &         &1 &   &        &   \\
 -1 & 0 & \ldots &  0         & 0 & \ldots 0 & 0 &0  & 0 & \ldots 0 &         &  &1  &        &  \\
  &   \vdots &   &        &            & \vdots &     &   &   &   &           &  &   & \ddots &  \\
  -1 & 0 & \ldots &  0        & 0 & \ldots 0 & 0 &0  & 0 & \ldots 0 &        &  &   &        &1  \\ 
\end{array}\right)
\right \rangle}\, g
 \end{align*}
 \endgroup
 where $-1$ in the first row is in the $j_1$ entry, and $-1$ in the second row is in the $j_1+1$ entry. The result of the action of $g$ on the first two rows is 
 $$\langle 0_{2\times 1}, \ldots, 0_{2\times 1}, -g_t, 0_{2\times 1}, \ldots, 0_{2\times 1}, g_1^{(1,2)} \rangle $$ where $t \in \{1, \ldots, k+1\}$ is such that $G_t=GL_2(q)$ and $g_1^{(1,2)}$ is the matrix formed by the first two rows of $g_1.$ It is easy to see that  two such vectors cannot lie in $Uz$, so  $Uz \ne Wz \varphi$ which is a contradiction. Hence $\varphi =g\in \Gamma.$ 
 
 Therefore $g=\diag[A,g_t, B]$ where $$A=\diag(\alpha_1, \ldots, \alpha_{j_1-1}), \text{ }B=\diag(\alpha_{j_1+2}, \ldots, \alpha_{n}),$$  for some $\alpha_i \in \mathbb{F}_q^*$ and $$g_t = \left( \begin{matrix} \delta_1 & \delta_2 \\ \delta_3 & \delta_4  \end{matrix} \right) \in GL_2(q).$$ Since $g \in S^z$, it stabilises $Uz.$
 Notice that $S \cap GL_n(q)$ also stabilises $\langle v_{j_1}, \ldots, v_n \rangle,$ so $g \in S^z \cap GL_n(q)$ stabilises $\langle v_{j_1}, \ldots, v_n \rangle z$. 
 
 Consider 
\begin{equation}
\label{GR222vj1}
((v_{j_1})z)g= 
 \begin{cases}
 \alpha_1 v_1 + \delta_1 v_{j_1} + \delta_2 v_{j_1+1};\\
 \sum_{i=j_1}^n \lambda_i v_i
 \end{cases}
\end{equation}
 for some $\lambda_i \in \mathbb{F}_q.$ Since the first line of \eqref{GR222vj1} contains no terms with $v_2$ and $v_i$ for $i \ge j_1+2$, we obtain   
 $((v_{j_1})z)g= \alpha (v_{j_1})z $ for some $\alpha \in \mathbb{F}_q^*.$ Therefore, $\delta_2=0$ and $\alpha= \alpha_1 = \delta_1.$ The same arguments applied to $(v_{j_1+1})z$ show that
 $\delta_4=0$ and $\alpha_2=\delta_3.$ 
 
   The same arguments applied to $(v_{n})z$ show that
$\alpha= \alpha_1 = \ldots =\alpha_n,$ so $g$ is scalar and $\varphi \in Z(GL_n(q)).$
\end{proof}


\section{Unitary groups}

In this section $S$ is a maximal solvable subgroup of $\GU_n(q)=GU_n(q) \rtimes \langle \phi_{\beta}\rangle$ where $\beta$ is an orthonormal basis of $(V, {\bf f}).$ Our goal is to prove the following theorem.

\begin{T2}
Let $X=\GU_n(q)$, $n \ge 3$ and $(n,q)$ is not equal to $(3,2).$ If $S$ is a maximal solvable subgroup of $X$, 
 then one of the following holds:
\begin{itemize}
\item  $b_S(S \cdot SU_n(q)) \le 4,$ so $\Reg_S(S \cdot SU_n(q),5)\ge 5$;
\item   $(n,q)=(5,2)$ and $S$ is the stabiliser in $X$ of a totally isotropic subspace of dimension $1$, $b_S(S \cdot SU_n(q)) =5$ and $\Reg_S(S \cdot SU_n(q),5)\ge 5$. 
\end{itemize}
\end{T2}

Recall that $g^{\dagger}=(\overline{g}^{\top})^{-1}$ for $g \in GL_n(q^{\bf u}),$ see the discussion after Definition \ref{taudef} for details. To prove  Theorem \ref{theoremGU}, we need the following lemma.

\begin{Lem}\label{starc}
Let $(n,q, {\bf u})$ be such that $GL_n(q^{\bf u})$ is not solvable. If $S$ is an irreducible maximal solvable  subgroup of $GL_n(q^{\bf u})$, then there exist $x, y \in SL_n(q^{\bf u})$ such that $$S \cap S^x \cap (S^{\dagger})^y \le Z(GL_n(q^{\bf u})).$$
\end{Lem} 
\begin{proof}
If $b_S(S \cdot SL_n(q^{\bf u}))=2,$ then there exists $x \in SL_n(q^{\bf u})$ such that $$S \cap S^x \le Z(GL_n(q^{\bf u})),$$ so $y$ can be arbitrary. Therefore, it suffices to consider cases \eqref{irred11}--\eqref{irred15} from Theorem \ref{irred} only.
 In cases \eqref{irred11}, \eqref{irred12} and \eqref{irred14}, $S$ is the normaliser of a Singer cycle, so $S \cdot SL_n(q^{\bf u})=GL_n(q^{\bf u})$. Since all Singer cycles are conjugate in $GL_n(q^{\bf u}),$ $S^{\dagger}=S^g$ for some $g \in GL_n(q^{\bf u})$, so the statement follows by Theorem \ref{irred}. 
In  case \eqref{irred15} $S^{\dagger}=S,$ so the statement follows since $b_S(S \cdot SL_4(3))\le 3$ by Theorem \ref{irred}.
In  case \eqref{irred13} the statement is verified by computation.  
\end{proof}


%First we will show that {\bf if} for any maximal irreducible solvable subgroup $S \in GU_m(q)$ (with $m\ge 2$ and $q \ge 4$) there exist  $y,z \in GU_m(q)$ such that $S \cap S^y \cap S^z \le Z(GU_m(q))$, {\bf then}, for any maximal solvable subgroup $S$ of $GU_n(q)$ with $q \ge 4$, the base size $b_S(GU_n(q))$ is at most 4. 


\begin{Lem} \label{lemn4uni}
Theorem {\rm \ref{theoremGU}} holds for $n=3.$
\end{Lem}
\begin{proof}
If $S$  stabilises no non-zero proper subspace of $V$, then the statement follows by 
\cite[Theorem 1.1]{burness}.

\medskip

Assume that $S$ stabilises $U<V$ and $S$  stabilises no non-zero proper subspace of $U$, so  $U$ is either totally isotropic or non-degenerate.

 If $U$ is totally isotropic, then $\dim U=1$ since a maximal totally isotropic subspace of a non-degenerate unitary space of dimension $n$  has dimension $[n/2]$.  By Lemma \ref{unist}, there exists a basis $\beta=\{f,v,e\}$ such that ${\bf f}_{\beta}$ is the permutation matrix for the permutation $(1,3)$ and all elements in $S_{\beta}$ have shape $\phi^j g$ with
\begin{equation*}
g=\begin{pmatrix}
\alpha_1^{\dagger} &* &*\\
 0 & \alpha_2 &*\\
0& 0 & \alpha_1
\end{pmatrix}
\end{equation*}
where $j \in \{0,1 , \ldots, 2f-1\}$, $\alpha_i \in \mathbb{F}_{q^2}^*$ and $\alpha_2^{q+1}=1.$
 Let $\eta$ be a generator of $\mathbb{F}_{q^2}^*.$ For $q$  even let $\delta=1$, for $q$ odd let $\delta=\eta^{-(q+1)/2},$ so $\delta \cdot \delta^{\dagger}=\delta^{1-q}=-1.$
The matrix $x=\diag(\delta^{\dagger}, 1 \ldots, 1, \delta){\bf f}_{\beta}$
lies in $SU_n(q,{\bf f}_{\beta})$. It is routine to check that if $\varphi \in S_{\beta} \cap S_{\beta}^x$, then $\varphi= \phi^j g$ with $g=\diag(\alpha_1^{\dagger}, \alpha_2, \alpha_1)$.
%$S_{\beta} \cap S_{\beta}^x$ is abelian and, by \cite[Theorem 1]{zen}, $b_S(S \cdot SU_n(q))\le 4.$
Let $\alpha \in \mathbb{F}_{q^2}$ be such that $\alpha + \alpha^q =1.$ It exists by Lemma \ref{al}. Let $\theta \in \mathbb{F}_{q^2}$ be $\eta^{q-1}$ and let $y, z \in SU_3(q,{\bf f}_{\beta})$  be $$
\begin{pmatrix}
1 & 0& 0\\
 -1 & 1 &0\\
-\alpha& 1 & 1
\end{pmatrix} \text{ and }
\begin{pmatrix}
1 & 0& 0\\
 -\theta^{-1} & 1 &0\\
-\alpha& \theta & 1
\end{pmatrix} 
$$ respectively. If $\varphi \in S_{\beta} \cap S_{\beta}^x \cap S_{\beta}^y,$ then $\varphi$ stabilises $\langle e \rangle y= \langle e +v - \alpha f \rangle,$ so $\alpha_1=\alpha_2$ and $\varphi=\phi^j \alpha_1 I_3.$ If $\varphi \in S_{\beta} \cap S_{\beta}^x \cap S_{\beta}^y  \cap S_{\beta}^z,$ then $\varphi$ stabilises $\langle e \rangle z= \langle e +\theta v - \alpha f \rangle,$ so $\theta^{p^j}\alpha_1= \theta \alpha_1$. Thus, $\theta^{p^j-1}=1$ and $j=0$ by Lemma \ref{pj10}, so $\varphi \in Z(GU_3(q)).$

 


 Assume $U$ is non-degenerate, so $S$ stabilises $U^{\bot}$ and we can assume that $\dim U=1$. Let $\beta=\{f,e,v\},$ where $\{f,e\}$ is a basis of $U^{\bot}$ as in \eqref{unibasis} and $U= \langle v\rangle$.
Let $x,y,z \in SU_3(q, {\bf f}_{\beta})$ be
$$
\begin{pmatrix}
1 & 0& 0\\
 -\alpha & 1 &1\\
-1& 0 & 1
\end{pmatrix},
\begin{pmatrix}
1 & -\alpha & 1\\
 0 & 1 &0\\
0& -1 & 1
\end{pmatrix}  \text{ and }
\begin{pmatrix}
1 & 0 & 0\\
 -\alpha & 1 &\theta^{-1}\\
-\theta & 0 & 1
\end{pmatrix} 
$$ 
respectively.
%For the rest of this section let $S$ be a maximal solvable subgroup of $GU_n(q)$ and let $G$ be $S \cdot SL_n(q).$ 
If $\varphi \in S_{\beta} \cap S_{\beta}^x \cap S_{\beta}^y  \cap S_{\beta}^z,$ then $\varphi$ stabilises $\langle v \rangle$, $\langle v-f \rangle$, $\langle v-e \rangle$ and $\langle v - \theta f \rangle.$ Arguments as in the previous case show that $\varphi \in Z(GU_3(q)).$
\end{proof}


\begin{Lem}
\label{3conjM}
Let $M=S \cap GU_n(q).$ If $S$ stabilises no non-zero proper subspace of $V$, then there exist $y,z \in SU_n(q)$ such that $M \cap M^y \cap M^z \le Z(GU_n(q))$ unless $(n,q)=(4,2)$ and $M=MU_4(2)$ is as defined in  Theorem $\ref{irredGU}$. 
\end{Lem}
\begin{proof}
If $M \le GU_n(q)$ is irreducible, then such $y,z$ exist by Theorem \ref{irredGU}. Assume that $M$ is reducible. The same arguments as in the proof of Lemma \ref{GammairGL} show that $M$ is completely reducible. If $V$ is not $\mathbb{F}_{q^2}[M]$-homogeneous, then $S$ (and $M$) stabilises a decomposition of $V$ as in Lemma \ref{ashb}, and such $y,z$ exist by the proof of Theorem \ref{irredGU}. If $V$ is  $\mathbb{F}_{q^2}[M]$-homogeneous, then $M$ stabilises a decomposition as in    Lemma \ref{ashb} by \cite[(5.2) and (5.3)]{asch}, and such $y,z$ exist by the proof of Theorem \ref{irredGU}.
\end{proof}


\begin{Th}
\label{GU4sp}
Theorem {\rm \ref{theoremGU}} holds for $n \ge 4$ if $S$ stabilises no non-zero proper subspace of $V$.
\end{Th}
\begin{proof}
 The result follows by \cite[Theorem 1.1]{burness} unless $n=4$ and $S$ lies in a maximal subgroup of type $Sp_4(q)$ as in \cite[Table 1]{burness}. We now consider this outstanding case. 




Let $n=4$ and $M=S \cap GU_n(q).$
If $S$ stabilises a decomposition of $V$ as in Lemma \ref{ashb}, then the statement follows by \cite[Table 2]{burness}.  Hence we can assume that if $N\le M$ is normal in $S$, then $V$ is $\mathbb{F}_{q^2}[N]$-homogeneous. In particular, every characteristic abelian  subgroup of $M$ is cyclic by \cite[Lemma 0.5]{manz}. 

Assume that $M$ is reducible, so $M$ stabilises non-zero $W<V$ such that $W$ is $\mathbb{F}_{q^2}[M]$-irreducible and $W$ is either non-degenerate or totally isotropic. If $V$ is not $\mathbb{F}_{q^2}[M]$-homogeneous, then $S$ stabilises a decomposition as in Lemma \ref{ashb} which contradicts the assumption above, so   $V$ is  $\mathbb{F}_{q^2}[M]$-homogeneous. Therefore, if $\dim W=1,$ then $M$ is a group of scalars, so $S/Z(GU_n(q))$ is cyclic and $b_S(S \cdot GU_4(q)) \le 2$ by Theorem \ref{zenab}. Hence we may assume that $\dim W=2$ and $W$ is either totally isotropic or non-degenerate.

First assume that $\dim W=2$ and $W$ is totally isotropic. By \cite[(5.2)]{asch}, 
$$V=W_1 \oplus W_2$$ where $W_i$ is a $M$-invariant submodule of $V$ isometric to $W$, so we can assume $W_1=W.$ Let $\beta$ be a basis as in \eqref{unibasis} corresponding to this decomposition of $V$. Let $M_1 \le GL_2(q^2)$ be the restriction of $M$ in $W.$ By Theorem \ref{irred}, either there exists $x_1 \in SL_2(q^2)$ such that $M_1 \cap M_1^{x_1} \le Z(GL_2(q))$ or $M_1$ is a subgroup of the normaliser of a Singer cycle in $GL_2(q^2).$ If $x_1$ as above exists, then $M \cap M^x\le Z(GU_n(q))$ where $x_{\beta}=\diag[x_1, x_1^{\dagger}],$ since $V$ is $\mathbb{F}_q[M]$-homogeneous. Therefore, $b_S(S \cdot SU_4(q)) \le 4$ by Theorem \ref{zenab}.

Let $M_1$ be  a subgroup of the normaliser of a Singer cycle in $GL_2(q^2).$ Since $M  \cong M_1,$ it has a maximal abelian normal subgroup $A$ of index at most $2$, which is also characteristic. Hence $V$ is $\mathbb{F}_{q^2}[A]$-homogeneous and the dimension of an irreducible $\mathbb{F}_{q^2}[A]$-submodule of $V$ is odd by Lemma \ref{simpcycl}, so $A$ is a group of scalars. So $M$ is cyclic modulo scalars and we obtain $b_S(S \cdot SU_4(q)) \le 4$ by applying Theorem \ref{zenab} twice.



Now let us assume that either  $\dim W=2$ and $W$ is non-degenerate or $M$ is irreducible (here we let $W=V$, so $\dim W=4$). Let $m=\dim W.$  Since every characteristic abelian subgroup of $M$ is cyclic, $M$ satisfies the conditions of \cite[Corollary 1.4]{manz}. In particular, in the notation of Lemma \ref{olaf}, the following hold: 
\begin{enumerate}[font=\normalfont]
\item $F=ET$, $Z=E \cap T$ and $T=C_F(E);$
\item a Sylow subgroup of $E$ is either cyclic of prime order or extra-special;
\item there exists $U \le  T$ of index at most $2$ with $U$ cyclic and characteristic in $M$, and
$C_T(U)=U$;
\item $EU = C_F(U)$ is characteristic in $M$.
\end{enumerate}
Since $U$ is characteristic in $M$, $V$ is $\mathbb{F}_{q^2}[U]$-homogeneous, so, by Lemma \ref{simpcycl}, $U$ is a group of scalars, $T=U$ and $M=C=C_M(U).$ Let $e$ be such that $e^2=|E/Z|.$ Let $0<L \le W$ be an $\mathbb{F}_{q^2}[EU]$-submodule. By \cite[Corollary 2.6]{manz}, $$m=e \cdot \dim L.$$
 Thus, $e \in \{1,2,4\}$, so $E$ is either cyclic or an extra-special $2$-group. By the proof of $(vii)$ and $(ix)$ of \cite[Corollary 1.10]{manz}, 
$F=C_M(E/Z)$ and $M/F$ is trivial for $e=1$ and  isomorphic to a subgroup of $Sp_e(2)$ for $e \in \{2,4\}.$

If $e=1,$ then $F=U$ is self-centralising (since the centraliser of the Fitting subgroup of a solvable group lies in the Fitting subgroup) and $W$ is  $\mathbb{F}_{q^2}[U]$-irreducible by \cite[Lemma 2.2]{manz}, which is a contradiction, since $U$ is a group of scalars. Therefore, $e \in \{2,4\}.$   

If $e=4,$ then, by the proof of Lemma \ref{c6small}, $M=M_1 \cdot Z(GU_4(q))$ and $M_1$ lies in the normaliser of a symplectic-type subgroup of $GU_4(p^t)$ for some $t \le f$. Hence  $b_M(M \cdot SU_4(q)) \le 2$ for $q>3$ by \cite[Table 2]{burness} and $b_S(S \cdot SU_4(q))\le 4$ by Theorem \ref{zenab}. For $q \le 3$ the statement is verified by computation.

Let $e=2.$ Therefore, $|M|=|U|\cdot|E/Z| \cdot |M/F|$ divides $$(q+1) \cdot e^2 \cdot |Sp_2(2)|=24(q+1).$$ So $|S|$ divides $24(q+1) \cdot 2f$ and $|S/Z(GU_4(q))|$ divides $48f.$ We claim that $$\hat{Q}((S \cdot SU_4(q)/Z(GU_4(q)),4)<1$$ where $\hat{Q}(G,c)$ is as in \eqref{ver} and $H=S/Z(GU_4(q))$. By Lemma \ref{fprAB}, if  $x_1,\ldots,x_k$ represent distinct $G$-classes such that $\sum_{i=1}^k |x_i^G \cap  H| \le A$ and $|x_i^G| \ge B$ for all $i \in \{1, \ldots, k\},$ then
$$\sum_{i=1}^m |x_i^G| \cdot \fpr (x_i)^c \le B \cdot (A/B)^c.$$ 
We take $A= 48 f \ge |H| \ge \sum_{i=1}^k |x_i^G \cap  H|.$ For elements in $PGU_4(q)$ of prime order with $s=\nu(x) \in \{1,2,3\}$ we use \eqref{5uni} as a lower bound for $|x_i^G|$. If $x \in H \backslash PGU_4(q)$ has prime order, then we use the corresponding bound for $|x^G|$ in \cite[Corollary 3.49]{fpr2}. We take $B$ to be the smallest of these bounds for $|x_i^G|.$ For $q \ge 5$, such $A$ and $B$ are sufficient to obtain $$\hat{Q}((S \cdot SU_4(q)/Z(GU_4(q)),4)<1,$$ so $b_S(S \cdot SU_4(q)) \le 4.$ For $q \le 4$ the theorem is verified by computation.
\end{proof}







\begin{Th}
\label{lem421}
Theorem {\rm \ref{theoremGU}} holds for  $n \ge 4$ if $S$ stabilises a non-zero proper subspace of $V$. 
\end{Th}
\begin{proof}
The proof proceeds in two steps. In {\bf Step 1} we obtain three conjugates of $S$ such that elements of their intersection have shape $\phi_{\beta} g$ for some basis $\beta$ of $V$ where $g \in GU_n(q, {\bf f}_{\beta})$ is diagonal or has  few non-zero entries not on the diagonal. In {\bf Step 2} we find a fourth conjugate of $S$ such that the intersection of the four is a group of scalars.

\subsection*{Step 1} 
 Fix a basis $\beta$ of the unitary space $(V, {\bf f})$ as in Lemma \ref{unist}, so ${\bf f}_{\beta}$ is as in \eqref{fst} and elements of $S$ take shape $\phi_{\beta}^{j}g$ with $g$ as in \eqref{gst} and $j \in \{0, 1, \ldots, 2f-1\}$. We consider $S$ as a subgroup of $\GU_n(q,{\bf f}_{\beta}).$ Let $M$ be $S \cap GU_n(q,{\bf f}_{\beta}).$  We obtain three conjugates of $S$ such that their intersection consists of elements $\phi_{\beta}^{j}g$ where $g$ is diagonal with respect to $\beta.$




Let $\gamma_i$ be as in Lemma \ref{unist}. 
%Let $x$ be the matrix ${\bf f}_{\beta}.$
%\begin{equation}
%\begin{pmatrix}
%        & & & & & & &    & I_{n_1} \\
%        & & & & & & &  \reflectbox{$\ddots$}  &\\
%        & & & & & &I_{n_k} &    & \\
%        & & & & &I_{n_{k+1}} & &    &\\ 
%        & & & &\reflectbox{$\ddots$} & & &    &\\
%        & & &I_{n_k+l} & & & &    &\\ 
%        & &I_{n_k} & & & & &    & \\
%        &\reflectbox{$\ddots$} & & & & & &    &\\
%I_{n_1} & & & & & & &    & 
%\end{pmatrix}.
%\end{equation}
Observe that ${\bf f}_{\beta}{\bf f}_{\beta}\overline{{\bf f}_{\beta}}^{\top}={\bf f}_{\beta},$ so ${\bf f}_{\beta} \in GU_n(q,{\bf f}_{\beta}).$  Notice that $\det({\bf f}_{\beta})=(-1)^{n_1 + \ldots + n_k}.$ %If $\sum_{i=1}^k n_i$ is even, then $\det(x)=1 \in \Det(S).$
 If $\sum_{i=1}^k n_i$ is odd, then one of the $n_r$ is odd for some $r \in \{1, \ldots, k\}.$ Let $\delta$ be as in the proof of Lemma \ref{lemn4uni}, so $\delta \delta^{\dagger}=-1.$ Notice that 
\begin{equation*} \label{hdetJn}
h=\diag[I_{n_1}, \ldots, I_{n_{r-1}}, \delta^{\dagger}I_{n_r}, I_{n_r+1}, \ldots, I_{n_r+1}, \delta I_{n_r},  I_{n_{r-1}}, \ldots, I_{n_1} ] \in GU_n(q,{\bf f}_{\beta})
\end{equation*}
 has determinant $\det({\bf f}_{\beta})$. %and normalises $M.$
  In particular, $x=h{\bf f}_{\beta} \in SU_n(q)$.
  It is easy to see that if $g \in M,$ so it has shape \eqref{gst}, then
\begin{equation*}%\label{antigst}
g^x=
\begin{pmatrix}
\gamma_{1}(g)& &\multicolumn{1}{l|}{0} & &  & & &    &0  \\
    *    & \ddots &\multicolumn{1}{l|}{} & & & & &    &\\
*        &* & \multicolumn{1}{l|}{\gamma_{k}(g)}& & & & &    & \\  \cline{1-6}
 *       &\ldots & \multicolumn{1}{l|}{*} &\gamma_{k+1}(g) & & \multicolumn{1}{l|}{0}  & &    &\\ 
  *      &\ldots &  \multicolumn{1}{l|}{*}& &\ddots &\multicolumn{1}{l|}{}       & &   &\\
   *     &\ldots &  \multicolumn{1}{l|}{*}&0 & &\multicolumn{1}{l|}{\gamma_{k+l}(g)}    &  &    & \\ \cline{4-9} 
    *    &\ldots & & & &\multicolumn{1}{l|}{*} & {{\gamma_{k}(g)}^{\dagger}}  &    &0 \\
    *    &\ldots & & & &\multicolumn{1}{l|}{*} &* & \ddots   &\\
*        &\ldots & & & &\multicolumn{1}{l|}{*} &* & *   & {{\gamma_{1}(g)}^{\dagger}} \\  
\end{pmatrix}.
\end{equation*}



Let $q>3.$ Notice that by Lemma \ref{GammairGL}, if $N$ is a solvable subgroup of $\GL_n(q)$  stabilising no  non-zero proper subspace, then $N \cap GL_n(q)$ lies in an irreducible maximal  solvable subgroup of $GL_n(q).$ Therefore, by Lemmas \ref{3conjM}  and   \ref{starc}  there exist $y_i, z_i \in SL_{n_i}(q^2)$ for $i=1, \ldots, k$ and $y_i, z_i \in SU_{n_i}(q)$ for $i =k+1, \ldots, k+l$ such that 
\begin{equation}\label{smint}
\gamma_i(M) \cap \gamma_i(M)^{y_i} \cap (\gamma_i(M)^{\dagger})^{z_i}  \le Z(GL_{n_i}(q^2)). 
\end{equation}
Notice that $\gamma_i(M)^{\dagger} = \gamma_i(M)$ for $i=k+1, \ldots, k+l$. Denote by $y$ and $z$ the block-diagonal matrices 
\begin{equation}
\label{yzdef}
\begin{split}
&\diag[y_1^{\dagger}, \ldots, y_k^{\dagger}, y_{k+1}, \ldots, y_{k+l}, y_k, \ldots, y_1]  \text{ and }\\
&\diag[z_1^{\dagger}, \ldots, z_k^{\dagger}, z_{k+1}, \ldots, z_{k+l}, z_k, \ldots, z_1]
\end{split}
\end{equation}
 respectively. It is routine to check that $y,z \in SU_n(q,{\bf f}_{\beta}).$

 %  By Lemma \ref{primconj} for $i = 1, \ldots, k$ there exist $\tilde{g}_i \in GL_n(q^2)$ such that 
%$$(\gamma_i(S)^{\dagger})^{\tilde{g}_i}=\gamma_i(S).$$  Denote by $\tilde{g}$ the block-diagonal matrix
%\begin{equation}\label{gd1}
%\diag(\tilde{g}_1, \ldots, \tilde{g}_k, I_{n_{k+1}}, \ldots, I_{n_{k+l}}, \tilde{g}_k^{\dagger}, \ldots, \tilde{g}_1^{\dagger}) \in GU_n(q).
%\end{equation}

 Therefore, if $g \in M \cap M^{xz},$ then $g$ is the block-diagonal matrix
\begin{equation}\label{gd1}
\diag [g_1^{\dagger}, \ldots, g_k^{\dagger}, g_{k+1}, \ldots, g_{k+l}, g_k, \ldots, g_1],
\end{equation}
where $g_i \in \gamma_i(M) \cap (\gamma_i(M)^{\dagger})^{z_i}$ for $i=1, \ldots, k+l.$ 
Thus, if $g \in M \cap M^{y} \cap M^{xz}$, then $g$ has shape \eqref{gd1} where
$$g_i \in \gamma_i(M) \cap \gamma_i(M)^{y_i} \cap (\gamma_i(M)^{\dagger})^{z_i} \le Z(GL_{n_i}(q^2)) \text{ for } i=1, \ldots, k+l.$$ 
So, by Lemma \ref{scfield}, we can assume that elements in $\gamma_i(S) \cap \gamma_i(S)^{y_i} \cap (\gamma_i(S)^{\dagger})^{z_i}$ have shape $\phi^j g_i$ with $g_i \in Z(GL_{n_i}(q^2)).$ Thus, if 
$ \varphi \in S \cap S^{y} \cap S^{xz},$ then $\varphi= \phi^jg$ with $g$ as in \eqref{gd1} and $g_i \in Z(GL_{n_i}(q^2)).$
 Denote $S \cap S^{y} \cap S^{xz}$ by $\tilde{S}$ and $M \cap \tilde{S}$ by $\tilde{M}.$



\medskip

If $q \in \{2,3\}$, then it may be that $\gamma_{k+i}(M) \in \{GU_2(q), GU_3(2), MU_4(2)\}.$ Recall that $MU_n(q)$ is defined in Lemma \ref{omnom}.  In view of Theorem \ref{irredGU}, and since $GU_2(q)$ and $GU_3(2)$ are solvable, elements $y_{k+i}$ and $z_{k+i}$ as in \eqref{smint} do not exist. If there is more than one such $\gamma_{k+i}(M)$, say 
\begin{equation*}% \label{mugamma}
\gamma_{k+i_1}(M), \ldots, \gamma_{k+i_{\mu}}(M),
\end{equation*}
then we join them in pairs, and there is one such group without pair if $\mu$ is odd. Let $H_1, H_2 \in \{GU_2(q), GU_3(2), MU_4(2)\}$ and let $\nu_j$ for $j \in \{1,2\}$ be the corresponding degree of $H_i,$ so $H_i \le GU_{\nu_j}(q).$ Let $$H=H_1 \times H_2=\{\diag[h_1, h_2] \mid h_j \in H_j\} \le GU_{\nu_1+ \nu_2}(q).$$  Computations show that $$b_H(H \cdot SU_{\nu_1+ \nu_2}(q)) \le 3.$$ Therefore, we can assume that there is at most one such $\gamma_{k+i}(S),$ so $\mu \le 1$. Denote the degree of such $\gamma_{k+i}(S)$ by $\nu$, so $2 \le \nu \le 4.$  Repeating the argument above for the rest of $\gamma_{i}(M)$ and $\gamma_{i}(S)$, we obtain that if $\varphi \in \tilde{S},$ then $\varphi =\phi^j g$ with $g$ as in \eqref{gd1} and either all $g_i \in Z(GL_{n_i}(q^2))$ (if $\mu=0$) or  all but one  $g_i \in Z(GL_{n_i}(q^2))$ and one $g_i$ (for $i>k$) is a 
$(\nu \times \nu)$ matrix (if $\mu=1$).   

\begin{Rem}%\label{rem2}
\label{remni1}
It may  be  that some  $\gamma_{k+i}(M)$ have degree 1, so $\gamma_{k+i}(S) \le \GU_1(q).$ %These subgroups already form a diagonal subgroup, but it is useful for the second step to 
We can treat them together. Indeed, assume that $\gamma_{k+i}(M)$ has degree 1 for $i=1, \ldots, \zeta \le l$. Define $\gamma_{k+1}{'} : S \to \GU_{\zeta}(q)$ by 
$$\gamma_{k+1}{'}(\phi^j g)= \phi^j \diag(\gamma_1(g), \ldots, \gamma_{\zeta}(g))$$ for  $g \in M$ and $j \in \{0,1, \ldots, 2f-1\}.$
 Hence the group $T=\gamma_{k+1}{'}(M)$, consisting of block-diagonal matrices, 
is an abelian subgroup of $T \cdot SU_{\zeta}(q)=GU_{\zeta}(q).$ If $q>3$, then by Theorem \ref{zenab} there exists $g \in SU_{\zeta}(q)$ such that 
$T \cap T^g\le {\bf F}(GU_{\zeta}(q))=Z(GU_{\zeta}(q)),$ where ${\bf F}(G)$ is the Fitting subgroup of a finite group $G$. So \eqref{smint} holds for $\gamma_{k+1}{'}(M)$ and  we can replace $\gamma_1, \ldots, \gamma_{\zeta}$ with 
$\gamma_{k+1}{'}$ of degree $\zeta$. 
Finally, suppose $q \le 3$. Notice that $$T \rtimes \langle \phi \rangle < ((GU_1(q))^{(1/2 +(-1)^{{\zeta}-1}/2)} \times (GU_2(q))^{[{\zeta}/2]}) \rtimes \langle \phi \rangle,$$ so if ${\zeta}>1$, then $S$ is not a maximal solvable subgroup of $GU_n(q)$ since $GU_2(q)$ is solvable.
 Therefore, we can assume that there is at most one $\gamma_{k+i}(S)$ of degree 1 in every case, so ${\zeta} \le 1.$
\end{Rem}

\bigskip

 We summarise the outcome of {\bf Step 1}. Let $\mu$ be the number of $i \in \{1, \ldots, l\}$ such that $\gamma_{k+i}(M) \in \{GU_2(2), GU_2(3), GU_3(2), MU_4(2)\}.$  We may assume $\mu \in \{0,1\}.$ In particular, $\mu=0$ if $q>3.$  There exist $x,y \in GU_n(q)$ such that if $\varphi \in \tilde{S}=S\cap S^x \cap S^y$, then    $\varphi =\phi^j g$ with $g$ as in \eqref{gd1} and either all $g_i \in Z(GL_{n_i}(q^2))$ (if $\mu=0$) or  all but one  $g_i \in Z(GL_{n_i}(q^2))$ and one $g_i$ (for $i>k$) is a 
$(\nu \times \nu)$ matrix (if $\mu=1$). Notice that $\nu$ is 2, 3, or 4 if the corresponding $\gamma_{k+i}(M)$ is $GU_2(q)$,  $GU_3(2)$ and  $MU_4(2)$ respectively. 



 
\subsection*{Step 2} We now find a fourth conjugate of $S$ such that its intersection with $\tilde{S}$ lies in  $Z(GU_n(q)).$ Let $\varphi$ be an element of $\tilde{S}.$


Assume that $S$ is such that $\mu=0.$
First we slightly modify the basis $\beta$ from the first step. Recall that $\beta$ is such that 
$\bf f_{\beta}$ is as in \eqref{fst}. Therefore, 
\begin{equation*}
\begin{aligned}
\beta  =  & \; \{f_1^{1}, \ldots, f_{n_1}^1, \ldots, f_1^k, \ldots, f_{n_k}^k, \\ & \; x_1^1, \ldots, x_{n_{k+1}}^1, \ldots, x_1^l, \ldots, x_{n_{k+l}}^l,\\ & \; e_{1}^{k}, \ldots, e_{n_k}^k, \ldots, e_{1}^1, \ldots, e_{n_1}^1 \},
\end{aligned}
\end{equation*}
where $(e_i^j,f_i^j)=1$ and  every other pair of vectors from $\beta$ is mutually orthogonal.  Let 
\begin{equation}\label{basisW}
\begin{aligned}
U_i & =\langle x_1^i, \ldots, x_{n_{k+i}}^i \rangle, &\text{ } i&=1, \ldots, l;\\
W_i & =\langle f_1^i, \ldots, f_{n_{i}}^i,  e_{1}^i, \ldots, e_{n_{i}}^i \rangle, &\text{ } i&=1, \ldots, k. 
\end{aligned}
\end{equation}
Thus, $$V=(W_1 \bot \ldots \bot W_k) \bot (U_1 \bot \ldots \bot U_l),$$
where $W_i$, $U_i$ are $\tilde{S}$-invariant subspaces 
 and $\gamma_{k+i}(S) \le \GU(U_i)$ for $i=1, \ldots, l$.  By Lemma \ref{unibasisl}, we can choose for $U_i$ the basis 
\begin{equation}\label{basisU}
\beta_{1i}=
\begin{cases} 
\{f_1^{k+i}, \ldots, f_{m_i}^{k+i}, e_{m_i}^{k+i}, \ldots, e_1^{k+i}\}, & \text{ if $n_{k+i}=2m_i$}; \\
\{f_1^{k+i}, \ldots, f_{m_i}^{k+i}, x^{k+i}, e_{m_i}^{k+i}, \ldots, e_1^{k+i}\}, & \text{ if $n_{k+i}=2m_i+1$. } 
\end{cases}
\end{equation} 
 By the first step
$$\gamma_{k+i}(\tilde{M}) \le  Z(GU(U_i)),$$
so, by Lemmas  \ref{uniGamsdp} and \ref{scfield}, $\gamma_i(\varphi)=\phi_{\beta_{1i}}^jg_i$ with $g_i \in Z(GU(U_i)).$

Now we renumber the basis vectors of the $W_i$ from \eqref{basisW} and basis vectors of the $U_i$ from \eqref{basisU} to obtain the basis 
$$\beta_1=\{f_1, \ldots, f_m, x_1, \ldots, x_t, e_m, \ldots, e_1\},$$
where $m=\left( \sum_{i=1}^{k}n_k +\sum_{i=1}^l m_{i} \right)$ and $t$ is the number of odd $n_{k+i}$ for $i=1, \ldots, l.$ In more detail, to obtain $\beta_1$ from $\beta$, we apply the following procedure:
\begin{itemize}
\item  replace bases of $U_i$ as in \eqref{basisW} by those as in \eqref{basisU}, denote new basis by $\beta_{1/3}$;
\item rearrange vectors as follows: first write down the $f^i_j$ in the order they occur in $\beta_{1/3},$ then do the same with the $x^i$ and then write the $e^i_j$ in the order opposite to the $f^i_j$ (so if $f^i_j$ is the $t$-th entry of $\beta_{1/3},$ then $e^i_j$ is the $(n-t+1)$-th entry of $\beta_{1/3}$). Denote new basis by $\beta_{2/3}$;
\item relabel the $f$-vectors with just one index in the order they occur, do the same with the $x$-vectors and label  the  $e$-vectors such that $(f_i,e_i)=1.$
\end{itemize}
We illustrate this procedure in the following example. 
\begin{example}
Let $k=2$, $l=2$, $n_1=1,$ $n_2=2,$ $n_3=2$ and $n_4=3.$ So
\begin{equation*}
\begin{split}
U_1& = \langle x_1^1, x_2^1 \rangle= \langle f_1^3, e_1^3 \rangle\\
U_2& = \langle x_1^2, x_2^2,x_3^2 \rangle= \langle f_1^4, x^4, e_1^4 \rangle \\
W_1& = \langle f_1^1, e_1^1 \rangle \\
W_2& = \langle f_1^2, f_2^2, e_1^2, e_2^2 \rangle
\end{split}
\end{equation*}
and $$\beta=\{f_1^1, f_1^2, f_2^2, x_1^1, x_2^1, x_1^2, x_2^2, x_3^2, e_1^2,e_2^2,e_1^1\}.$$ Hence 
$$\beta_{1/3}=\{f_1^1, f_1^2, f_2^2, f_1^3, e_1^3, f_1^4, x^4, e_1^4, e_1^2,e_2^2,e_1^1\}$$
and 
$$\beta_{2/3}=\{f_1^1, f_1^2, f_2^2, f_1^3, f_1^4, x^4,  e_1^4, e_1^3, e_2^2, e_1^2, e_1^1\}.$$ The relabelling is 
\begin{equation}
\begin{array}{cccccccccccccc}
{\beta_{2/3}}&= \{& f_1^1,& f_1^2,& f_2^2,& f_1^3,& f_1^4,& x^4,&  e_1^4,& e_1^3,& e_2^2,& e_1^2,& e_1^1&\}\\
\downarrow& & \downarrow&\downarrow&\downarrow&\downarrow&\downarrow&\downarrow&\downarrow&\downarrow&\downarrow&\downarrow&\downarrow& \\
\beta_{1}&= \{& f_1,& f_2,& f_3,& f_4,& f_5,& x_1,&  e_5,& e_4,& e_3,& e_2,& e_1&\}.\\
\end{array}
\end{equation}
\end{example}

We now resume the proof of Theorem \ref{lem421}. 
 Notice that $\varphi \in \tilde{S}_{\beta_1}$ has shape $(\phi_{\beta_1})^jg$ with $g$ as in \eqref{gd1} and $g_i \in Z(GL_{n_i}(q^2))$. For simplicity we omit the subscripts and consider $S$ and $\tilde{S}$ as subgroups in $\GU_n(q, {\bf f}_{\beta_1}).$ Let $\phi^jg \in \tilde{S},$ so 
\begin{equation}\label{gdiag}
g=\diag(\alpha_1^{\dagger}, \ldots,  \alpha_m^{\dagger},\delta_1, \ldots, \delta_t, \alpha_m, \ldots, \alpha_1).
\end{equation}
If 
 \begin{equation*}
 U_i =
\begin{cases} 
\langle f_s, \ldots, f_{s+m_i}, e_{s+m_i}, \ldots, e_s\rangle, & \text{ for $n_{k+i}=2m_i$},  \\
\langle f_s, \ldots, f_{s+m_i}, x_r, e_{s+m_i}, \ldots, e_s\rangle, & \text{ for $n_{k+i}=2m_i+1$, } 
\end{cases}
\end{equation*}
 then  $\alpha_s= \ldots =\alpha_{s+m_i}=\delta_r$  and  $\alpha_s^{q+1}=1$ since $g$ is scalar on each $U_i$ by the first step. 
If 
 \begin{equation}\label{Weq}
 W_i = 
\langle f_s, \ldots, f_{s+n_i}, e_{s+n_i}, \ldots, e_s\rangle,   \\
\end{equation}
 then  $\alpha_s= \ldots =\alpha_{s+n_i}$ since $g$ is scalar on  $ \langle e_{s+n_i}, \ldots, e_s\rangle$ by the first step.




\begin{Rem}\label{x1} %Assume that $t>0.$
 If  $\alpha_i=\alpha_i^{\dagger}=\alpha_1$ for $i=1, \ldots, m$, then $g$ is not scalar if and only if ${\zeta}=1$ in Remark \ref{remni1}. So, if there exists $\gamma_s(S)$ of degree $1$, then we can assume, without loss of generality, that $\delta_1$ is the corresponding entry (so $\gamma_s(S)$ acts on $\langle x_1\rangle$). Therefore, if  $\alpha_i=\alpha_i^{\dagger}=\delta_1$ for $i=1, \ldots, m$, then $g=\delta_1 I_n \in Z(GU_n(q,{\bf f}_{\beta_1})).$ 
\end{Rem}





The remainder of our proof of {\bf Step 2} splits into 3 cases:
\begin{description}[before={\renewcommand\makelabel[1]{\bfseries ##1}}]
\item[{\bf Case 1.}] $\mu=0$, $k>0;$
\item[{\bf Case 2.}] $\mu=0$, $k=0;$
\item[{\bf Case 3.}] $\mu=1.$
\end{description}
Each case splits into two or three subcases depending on other parameters. In  {\bf Cases 1} and {\bf 2} we show $b_S(S \cdot SU_n(q))\le 4.$ In  {\bf Case 3} we show  $b_S(S \cdot SU_n(q))\le 4$ unless $n$ is small ($q \in \{2,3\}$ here since $\mu=1$). For small $n$ the statement of Theorem \ref{theoremGU} is verified by computation; we identify these values of $n$ in {\bf Case 3}.


\subsection*{Case 1.} Let $\mu=0$  and $k>0.$ So there is a totally singular $S$-invariant subspace 
$$V_1=\langle e_1, \ldots, e_{n_1} \rangle.$$
Recall that $n_i$ is the degree of $\gamma_i(S)$ for $i \in \{1, \ldots, k+l\}.$ Let $\alpha \in \mathbb{F}_{q^2}$ be such that  
$\alpha+\alpha^q=1,$ it exists by Lemma \ref{al}.

\medskip


The three subcases we consider correspond to the following situations:
\begin{description}[before={\renewcommand\makelabel[1]{\bfseries ##1}}]
\item[{\bf Case (1.1)}] $\dim W_i=2$ for $W_i$ in \eqref{basisW} and $i=1, \ldots, k$;
\item[{\bf Case (1.2)}] Condition of {\bf Case (1.1)} does not hold and $l=0$;
\item[{\bf Case (1.3)}] Condition of {\bf Case (1.1)} does not hold and $l>0$.
\end{description} 

\medskip

{\bf Case (1.1).} Assume that  $\dim W_i=2$ for $W_i$ in \eqref{basisW} and $i=1, \ldots, k.$
Let $\eta$ be a generator of $\mathbb{F}_{q^2}^*$ and let $\theta=\eta^{q-1}.$ We redefine $y$ from \eqref{yzdef} to $$\diag[A^{\dagger}, y_{k+1}, \ldots, y_{k+l},A]$$
where 
$$
A=
\begin{pmatrix}
1 & 0   & 0   & \ldots  & 0 \\
0 & 1      & 0   & \ldots  & 0 \\
  &        & \ddots &   &  \\
0 & \ldots & 0      & 1 & 0\\
1 & \ldots & \ldots & 1 & 1\\
\end{pmatrix}.
$$ It is easy to see that $y \in SU_n(q, {\bf f}_{\beta}).$ Let $x,z \in SU_n(q, {\bf f}_{\beta_1})$ be as in {\bf Step 1}, so $\varphi \in \tilde{S}$ has shape $\phi^j g$ with $g$ as in \eqref{gdiag}. Since $S$ stabilises $\langle e_1 \rangle$, $S^y$ stabilises $\langle e_1 \rangle y= \langle e_1 + \ldots +e_k \rangle.$ Therefore,
$$((e_1)y)\varphi=(e_1 + \ldots + e_k)\phi^j g=\alpha_1 e_1 + \ldots + \alpha_k e_k= \lambda (e_1 + \ldots + e_k)$$  
for some $\lambda \in \mathbb{F}_{q^2}^*,$ so $\alpha_1 = \ldots = \alpha_k.$

Let $k\ge 2$. We claim that there exists  $a \in SU_n(q, {\bf f}_{\beta_1})$ such that 
\begin{equation}
\begin{aligned}
(e_1)a =   & \sum_{i=3}^m e_i +\theta e_2+ e_1 + \underline{x_1 -\alpha f_1}; &  (f_1)a & =f_1;  \\
(e_{2})a = & \phantom{(} e_{2}; &  (f_{2})a & =f_{2}- \theta^{-1}f_1; &&   \\
(e_{i})a = & \phantom{(} e_{i};&  (f_{i})a & =f_{i} -  f_{1}; && i \in \{3, \ldots, m \}   \\
\underline{(x_1)a =}&\underline{  \phantom{(}x_1-f_1}, 
\end{aligned}
\end{equation}
and $a$ stabilises all other vectors from $\beta_1.$ Here the underlined part is  in the formula only if ${\zeta}=1$ and $x_1$ is as in Remark \ref{x1}. In other words, if $n_{k+i}>1$  for all $i=1, \ldots, l,$ then we  omit the underlined part.
It is routine to check that $\det(a)=1$ and $a$ is an isometry of $(V, {\bf f}),$ so $a \in SU_n(q,{\bf f}_{\beta_1}).$

We claim that $\tilde{S} \cap S^a \le Z(GU_n(q)).$ Let $\varphi= \phi^jg \in \tilde{S} \cap S^a$, where $g$ is as in \eqref{gdiag}. Observe that $S$ stabilises  $\langle e_1\rangle$, so $S^a$ stabilises $\langle e_1\rangle a.$ Therefore, 
\begin{equation}
((e_1)a)\phi^j g=
\begin{cases}
\sum_{i=3}^m\alpha_i e_i +\theta^{p^j}\alpha_2 e_2 + \alpha_1e_1 + \underline{\delta_1 x_1 -\alpha_1^{\dagger} \alpha f_1}\\
\lambda (e_1)a
\end{cases}
\end{equation}
for some $\lambda \in \mathbb{F}_{q^2}^*.$  Thus, $$\lambda=\alpha_1=\theta^{p^j-1}\alpha_2 =\underline{\alpha_1^{\dagger}} =\alpha_{3}=\ldots=\alpha_m=\underline{\delta_1}$$
and $\theta^{p^j-1}=1,$ so $j=0$ and $\varphi =g \in Z(GU_n(q, {\bf f}_{\beta})).$

\medskip

Let $k=1.$ We can assume that $n_{k+1}\ge 2$. Indeed, if  $n_{k+i}=1$ for all $i \in \{1, \ldots, l\},$ then,  by Remark \ref{remni1}, $l=1$, so $n=3$ and Theorem \ref{theoremGU} follows by Lemma \ref{lemn4uni}. Thus, $\langle e_2, f_2 \rangle \subseteq U_1$ and $\alpha_2=\alpha_2^{\dagger}.$ We claim that there exists  $a \in SU_n(q, {\bf f}_{\beta_1})$ such that 
\begin{equation}
\label{465}
\begin{aligned}
(e_1)a & =    \sum_{i=3}^m e_i +\theta e_2+ e_1 +f_2-\theta f_1 + \underline{x_1 -\alpha f_1}; \\  (f_1)a & =f_1;  \\
(e_{2})a & =  \phantom{(} e_{2}-f_1; \\  (f_{2})a & =f_{2}- \theta^q f_1; &&   \\
(e_{i})a  & = \phantom{(} e_{i}; && i \in \{3, \ldots, m \}\\  (f_{i})a & =f_{i} -  f_{1}; && i \in \{3, \ldots, m \}   \\
\underline{(x_1)a} & \underline{ \; = \phantom{(}x_1-f_1}, 
\end{aligned}
\end{equation}
and $a$ stabilises all other vectors from $\beta_1.$ Here the underlined part is in the formula only if ${\zeta}=1$ and $x_1$ is as in Remark \ref{x1}. %In other words, if $n_{k+i}>1$  for all $i=1, \ldots, l,$ then one  omits the underlined part.
It is routine to check that $\det(a)=1$ and $a$ is an isometry of $(V, {\bf f}),$ so $a \in SU_n(q,{\bf f}_{\beta_1}).$ 

We claim that $\tilde{S} \cap S^a \le Z(GU_n(q)).$ Let $\varphi= \phi^jg \in \tilde{S} \cap S^a$, where $g$ is as in \eqref{gdiag}. Observe that $S$ stabilises  $\langle e_1\rangle$, so $S^a$ stabilises $\langle e_1\rangle a.$ Therefore, 
\begin{equation}
((e_1)a)\phi^j g=
\begin{cases}
\sum_{i=3}^m\alpha_i e_i +\theta^{p^j}\alpha_2 e_2 + \alpha_1e_1 + \alpha_2 f_2 -\theta^{p^j}\alpha_1^{\dagger} f_1+ \underline{\delta_1 x_1 -\alpha_1^{\dagger} \alpha f_1}\\
\lambda (e_1)a
\end{cases}
\end{equation}
for some $\lambda \in \mathbb{F}_{q^2}^*.$  Thus, $$\lambda=\alpha_1=\alpha_2=\theta^{p^j-1}\alpha_2=\alpha_{3}=\ldots=\alpha_m=\underline{\delta_1}$$
and $\theta^{p^j-1}=1,$ so $j=0$ and $\varphi =g \in Z(GU_n(q, {\bf f}_{\beta})).$

\medskip

%{\bf Step 2.1.1} Assume that $l=0$ (so there is no $U_i$) and  $\dim W_i=2$ for $W_i$ in \eqref{basisW} and $i=1, \ldots, k.$ Notice that $S$ consists of upper triangular matrices and $J_n \in GU_n(q, {\bf f}_{\beta_1}).$ If $\det(J_n)=-1,$ then let $h$ be as in \eqref{hdetJn}, otherwise let $h=I_n.$ It is easy to see that $S \cap S^{hJ_n}$ is abelian, so $b_S(G)\le 4$ by \cite[Theorem 1]{zen}. 

{\bf Case (1.2).} Assume that $l=0$ (so there is no $U_i$) and there exists $r \in \{1, \ldots, k\}$ such that $\dim W_r \ge 4$. So 
 $$W_r = \langle f_s, \ldots, f_{s+n_r}, e_{s}, \ldots, e_{s+n_r}\rangle,  \text{ for some } s \text{ and } n_r \ge 2.$$
In particular, $\alpha_s=\alpha_{s+1}.$ Let $\chi=\eta^{(q+1)/2},$ so $\chi +\chi^q=0$ and $\chi^{-q}=-\chi^{-1}.$ We claim that there exists  $a \in SU_n(q, {\bf f}_{\beta_1})$ such that 
\begin{align*}
(e_s)a= & \phantom{(} f_s + \theta f_{s+1} + \sum_{i \notin \{s,s+1\}}^m f_i + \chi e_s; & (f_s)a= & -\chi^{-1}f_s; & \\
(e_{s+1})a=  & \phantom{(} \chi e_{s+1} +\theta^q f_s; & (f_{s+1})a= & -\chi^{-1}f_{s+1}; & \\
(e_i)a= & \phantom{(} e_i+\chi^{-1} f_s; & (f_i)a= & \phantom{(} f_i &  \text{ for } i \ne s.
\end{align*}
It is routine to check that $\det(a)=1$ and $a$ is an isometry of $(V, {\bf f}),$ so $a \in SU_n(q,{\bf f}_{\beta_1}).$ 

We claim that $\tilde{S} \cap S^a \le Z(GU_n(q)).$ Let $\varphi= \phi^jg \in \tilde{S} \cap S^a$, where $g$ is as in \eqref{gdiag}. Notice that $S$ stabilises $E= \langle e_1, \ldots, e_m \rangle$, so $S^a$ stabilises  $Ea$. Therefore, 
\begin{equation}\label{sl22}
((e_s)a)\varphi=
\begin{cases}
\phantom{(} \alpha_s^{\dagger} f_s + \theta^{p^j} \alpha_{s+1}^{\dagger} f_{s+1} + \sum_{i \notin \{s,s+1\}}^m \alpha_i^{\dagger} f_i + \chi^{p^j} \alpha_s e_s\\
\eta_1 (e_1)a + \ldots + \eta_m (e_m)a.
\end{cases}
\end{equation}
Since $((e_s)a)g$ does not have terms with $e_i$ for $i \ne s$ in the first line of \eqref{sl22}, 
$((e_s)a)g=\eta_s(e_s)a,$ so 
\begin{equation}\label{sl22qe}
\eta_s=\chi^{p^j-1} \alpha_s=\alpha_s^{\dagger}=\theta^{p^j-1}\alpha_{s+1}^{\dagger} =\alpha_1^{\dagger} = \ldots= \alpha_{s-1}^{\dagger}  = \alpha_{s+1}^{\dagger} = \ldots = \alpha_m^{\dagger}
\end{equation}
and $\theta^{p^j-1}=1.$ Hence $j=0$ and $\alpha_s=\alpha_s^{\dagger}$ by \eqref{sl22qe}, so $g$ is scalar and $\tilde{S} \cap S^a\le Z(GU_n(q,{\bf f}_{\beta_1})).$

%Consider 
%\begin{equation}\label{sl22e}
%((e_{s+1})a)g=
%\begin{cases}
%\alpha_{s+1} e_{s+1} - \alpha_s^{\dagger} f_s \\
%\eta_1 (e_1)a + \ldots +\eta_m (e_m)a.
%\end{cases}
%\end{equation}
%Since $((e_{s+1})a)g$ does not have $e_i$ for $i \ne s+1$ in the first line of \eqref{sl22e}, 
%$((e_{s+1})a)g=\eta_{s+1}(e_{s+1})a,$ so 
%$\alpha_{s+1}=\alpha_s^{\dagger}$. Also, by \eqref{Weq}, $\alpha_s=\alpha_{s+1}$. Therefore
%$\alpha_s^{\dagger}=\alpha_s$ and together with \eqref{sl22qe} it proves that $g$ is scalar.

\medskip

{\bf Case (1.3).} Assume $l >0$ and there exists $i \in \{1, \ldots, k\}$ such that $\dim W_i \ge 4$. So 
 $$W_i = \langle f_s, \ldots, f_{s+n_i}, e_{s}, \ldots, e_{s+n_i}\rangle,  \text{ for some } s \text{ and } n_i \ge 2.$$
In particular, $\alpha_s=\alpha_{s+1}.$ Let $r= n_1 + \ldots + n_k.$ 
We claim that there exists  $a \in SU_n(q, {\bf f}_{\beta_1})$ such that 
\begin{align*}
(e_s)a= & (\sum_{i=r+1}^m e_i)+ e_s + \theta f_{s+1} + \sum_{i \notin \{s,s+1\}}^r f_i + \underline{(x_1-\alpha f_s)}; & (f_s)a= & f_s; & \\
\end{align*}
\begin{align*}
(e_{s+1})a=  & \phantom{(}  e_{s+1} -\theta^q f_s; & (f_{s+1})a= & f_{s+1}; & \\
(e_i)a= & \phantom{(} e_i-f_s; & (f_i)a= & \phantom{(} f_i; & & \text{ for } i \in \{1, \ldots, r\} \backslash \{s, s+1\}; \\
(e_i)a= & \phantom{(} e_i; & (f_i)a= & \phantom{(} f_i -f_s; & & \text{ for } i \in \{r+1, \ldots, m\};\\
\underline{(x_1)a=} & \underline{\phantom{(} x_1-f_s}, & & & 
\end{align*}
and $a$ stabilises all other vectors from $\beta_1.$ Here the underlined part is in the formula only if ${\zeta}=1$ and $x_1$ is as in Remark \ref{x1}. %In other words, if $n_{k+i}>1$  for all $i=1, \ldots, l,$ then one  omits the underlined part.
  It is routine to check that $\det(a)=1$ and $a$ is an isometry of $(V, {\bf f}),$ so $a \in SU_n(q,{\bf f}_{\beta_1}).$

We claim that $\tilde{S} \cap S^a \le Z(GU_n(q)).$ Let $\varphi= \phi^jg \in \tilde{S} \cap S^a$, where $g$ is as in \eqref{gdiag}. Observe that $S$ stabilises  $E=\langle e_1, \ldots, e_r \rangle$, so $S^a$ stabilises $Ea.$ Therefore, $((e_s)a)\varphi$ is
\begin{equation}\label{2pres}
(\sum_{i=r+1}^m \alpha_i e_i)+ \alpha_s e_s + \theta^{p^j} \alpha_{s+1}^{\dagger} f_{s+1} + \sum_{i \notin \{s,s+1\}}^r \alpha_i^{\dagger} f_i + \underline{(\delta_1 x_1-\alpha^{p^j} \alpha_s^{\dagger} f_s)}
\end{equation}
and
$$((e_s)a)\varphi= \eta_1 (e_1)a+ \ldots +\eta_{r}(e_{r})a.$$
Since $\eta_i(e_i)a$ for $i \in \{1, \ldots, r\} \backslash \{s\}$ has $\eta_i$ as a coefficient for $e_i$ with respect to $\beta_1$ and $((e_s)a)\phi$ has $0$ as these coefficients (see  \eqref{2pres}), $\eta_i=0$ for all $i\ne s.$  Thus, $$\eta_s= \alpha_s=\theta^{p^j-1} \alpha_{s+1}^{\dagger} =\alpha_{r+1}=\ldots=\alpha_m=\underline{\delta_1}.$$
Observe $l>0,$ so $m>r$ or ${\zeta}=1,$ so $\alpha_s^{\dagger}=\alpha_s.$ Therefore,  $\theta^{p^j-1}=1$ since $\alpha_s=\alpha_{s+1},$ so $j=0.$  Hence $\alpha_i=\alpha_s$ for $i \le r$ by \eqref{2pres} and  $g$ is scalar, so $\tilde{S} \cap S^a\le Z(GU_n(q,{\bf f}_{\beta_1})).$





\subsection*{Case 2.} Let $\mu=0$ and $k=0.$ So $S$   stabilises no non-zero singular subspace. Choose $U$
% $U=(U)S$, a proper non-degenerate subspace of  $V,$
 to be one of the $U_i$ such that $\dim U = \max_{i \in \{1, \ldots, l\}} \{\dim U_i\}.$ Therefore, $V=U\bot U^{\bot}$ and $(U^{\bot})S=U^{\bot}.$ Without loss of generality, we can assume  that 
\begin{equation}
\label{U1dx}
U=U_1=\langle f_1, \ldots, f_{d}, \underline{x}, e_d, \ldots, e_1 \rangle,
\end{equation}
where $d=[\dim U /2]$ and $\{\underline{x}\}=\{x_1, \ldots, x_t\} \cap U$. If $\dim U$ is even, then  $\{\underline{x}\}$ is empty   and we read \eqref{U1dx} without $\underline{x}.$ If $\dim U$ is odd, then we assume that $\underline{x}=x_t.$ So 
$$U^{\bot}=\langle f_{d+1}, \ldots, f_{m}, {x_1, \ldots,x_s }, e_{m}, \ldots, e_{d+1} \rangle,$$
where $s=t-1$ if $\dim U$ is odd and $s=t$ otherwise. Define $x_1$ as in Remark \ref{x1}. Notice that if $\phi^j g \in \tilde{S},$ so $g$ has shape \eqref{gdiag}, %or \eqref{GalmostDiag}
 then $\alpha_i^{\dagger} = \alpha_i$ for $i \in \{1, \ldots, m\}$ since $g$ acts on $U_r$ containing $e_i$ and $f_i$ as a scalar.

If  $\dim U_i=1$ for all $i=1, \ldots, l$, then $M$ is abelian and, by Theorem \ref{zenab}, there exists $y \in SU_n(q)$ such that $M \cap M^y \le Z(GU_n(q)).$ Thus, $(S \cap S^y)/Z(GU_n(q))$ is an abelian subgroup of $(S \cdot SU_n(q))/Z(GU_n(q))$ and, by Theorem \ref{zenab}, there is $z \in SU_n(q)$ such that $(S \cap S^y) \cap (S \cap S^y)^z =Z(GL_n(q))$. So we can assume $\dim U \ge 2.$


%We claim that there is such $a \in GU_n(q)$ that 
%\begin{align*}
%&(e_1)a =\sum_{i=r+1}^m e_i +e_1 + \underline{x -\alpha f_{1}}& & (e_{r+1})a =e_{r+1}  \\
%&(e_{2})a =\sum_{i=r+1}^m e_{i}+ e_{2} & &(e_{r+2})a=e_{r+2} \\
%& \vdots & & \vdots\\
%&(e_{r})a =\sum_{i=r+1}^m e_{i}+ e_{r} & &(e_{m})a =e_{m}  \\
%&(f_1)a=f_1 & &(f_{r+1})a=f_{r+1} - \sum_{i=1}^r f_{i} + \underline{x_1 -\alpha e_{r+1}} \\
%& \vdots & & \vdots\\
%& (f_{r})a=f_{r} & &(f_{m})a=f_{m} - \sum_{i=1}^r f_{i}\\
%&\underline{(x)a= x_1-e_{r+1}},&&\underline{(x_1)a=x-f_{1}}& 
%\end{align*}
%and $a$ stabilises all other vectors from $\beta_1.$ Here underline part is present in the formula only if such $x$ and $x_1$ are present in $\beta_1.$ In other words, if $n_{k+i}>1$  for all $i=1, \ldots, l,$ then one should read above formulas without underlined part on the right and if $r$ is even then one should read above formulas without underlined part on the left. Also, $\alpha \in F_{q^2}$ is such that  
%$\alpha+\alpha^q=1,$ which exists by Lemma \ref{al}. It is routine to check that $a$ is an isometry of $(V, {\bf f}),$ so $a \in GU_n(q).$


\medskip


The two subcases we consider correspond to the following situations:
when $d=m$ and  $d<m$ respectively.

\medskip

{\bf Case (2.1).} Let $d=m,$ so $V=U \bot \langle x_1 \rangle.$

Assume $d\ge 2,$ so $\alpha_1=\alpha_2.$ 
We claim that there exists  $a \in SU_n(q, {\bf f}_{\beta_1})$ such that 
\begin{align*}
(e_1)a= &\phantom{(}  e_1; & (f_1)a= & f_1 +(x_1 - \alpha e_1); & \\
(e_{2})a=  & \phantom{(}  e_{2}; & (f_{2})a= & f_{2}+\theta^q(x_1 - \alpha e_2); & \\
(x_1)a= & \phantom{(} x_1 -e_1 -\theta e_2, &  & & & 
\end{align*}
and $a$ stabilises all other vectors from $\beta_1.$   It is routine to check that $\det(a)=1$ and $a$ is an isometry of $(V, {\bf f}),$ so $a \in SU_n(q,{\bf f}_{\beta_1}).$

We claim that $\tilde{S} \cap S^a \le Z(GU_n(q)).$ Let $\varphi= \phi^jg \in \tilde{S} \cap S^a$, where $g$ is as in \eqref{gdiag}. Observe that $S$ stabilises  $\langle x_1\rangle$, so $S^a$ stabilises $\langle x_1\rangle a.$ Therefore, 
\begin{equation*}
((x_1)a)\varphi= \delta_1 x_1 + \alpha_1 e_1 + \theta^{p^j} \alpha_2 e_2= \lambda ((x_1)a) 
\end{equation*}
 for some $\lambda \in \mathbb{F}_{q^2}^*.$ Hence $\lambda= \delta_1 =\alpha_1 = \theta^{p^j-1} \alpha_2,$
$g$ is scalar and $j=0$ since $\alpha_1=\alpha_2.$ So $\tilde{S} \cap S^a\le Z(GU_n(q,{\bf f}_{\beta_1})).$

Assume that $d=1,$ so $\dim U \le 3$. If $\dim U=2$, then $n=3$ and this case is considered in Lemma \ref{lemn4uni}, so we may assume $\dim U=3.$ We claim that there exists  $a \in SU_n(q, {\bf f}_{\beta_1})$ such that 
\begin{align*}
(e_1)a= &\phantom{(}  e_1+x_1 - \alpha f_1; & (f_1)a= & \alpha f_1  - e_1 - \theta \underline{x}; & \\
(\underline{x})a=  & \phantom{(} \theta^q e_{1} + \theta^q \alpha^q f_1; & (x_1)a= & e_1- \alpha f_{1}+ x_1 +\theta \underline{x}. & 
\end{align*}
   It is routine to check that $\det(a)=1$ and $a$ is an isometry of $(V, {\bf f}),$ so $a \in SU_n(q,{\bf f}_{\beta_1}).$ We claim that $\tilde{S} \cap S^a \le Z(GU_n(q)).$ Let $\varphi= \phi^jg \in \tilde{S} \cap S^a$, where $g$ is as in \eqref{gdiag}. Observe that $S$ stabilises  $\langle x_1\rangle$, so $S^a$ stabilises $\langle x_1\rangle a.$ Therefore, 
\begin{equation*}
((x_1)a)\varphi= \alpha_1 e_1 - \alpha^{p^j} \alpha_1 f_1+ \delta_1 x_1 + \alpha_1  + \theta^{p^j} \alpha_1 \underline{x} = \lambda ((x_1)a) 
\end{equation*}
 for some $\lambda \in \mathbb{F}_{q^2}^*.$ Hence $\lambda= \delta_1 =\alpha_1 = \theta^{p^j-1} \alpha_1,$
$g$ is scalar and $j=0$. So $\tilde{S} \cap S^a\le Z(GU_n(q,{\bf f}_{\beta_1})).$

\medskip

{\bf Case (2.2).} Let $d<m.$

Assume $d\ge 2,$ so $\alpha_1=\alpha_2.$ We claim that there exists  $a \in SU_n(q, {\bf f}_{\beta_1})$ such that 
\begingroup
\allowdisplaybreaks
\begin{align*}
(e_1)a & =  (\sum_{i=d+1}^m e_i)+ e_1  + \underline{(x_1-\alpha f_1)}; \\ (f_1)a & =  f_1; & \\
(e_{2})a & =    (\sum_{i=d+1}^m \theta e_i) + e_{2}; \\ (f_{2})a & =  f_{2}; & \\
(e_i)a & =  \phantom{(} e_i; & & \text{ for } i \in \{3, \ldots, d\}; \\  (f_i)a & =  f_i; & & \text{ for } i \in \{3, \ldots, d\}; \\
(e_i)a & =  \phantom{(} e_i; & & \text{ for }  i \in \{d+1, \ldots, m\};\\ (f_i)a & =  \phantom{(} f_i -f_1-\theta^q f_2; & & \text{ for }  i \in \{d+1, \ldots, m\};\\
\underline{(x_1)a} & \underline{\; =\phantom{(} x_1-f_1}, & & & 
\end{align*} 
\endgroup
and $a$ stabilises all other vectors from $\beta_1.$ It is routine to check that $\det(a)=1$ and $a$ is an isometry of $(V, {\bf f}),$ so $a \in SU_n(q,{\bf f}_{\beta_1}).$ We claim that $\tilde{S} \cap S^a \le Z(GU_n(q)).$ Let $\varphi= \phi^jg \in \tilde{S} \cap S^a$, where $g$ is as in \eqref{gdiag}. Observe that $S$ stabilises  $U$, so $S^a$ stabilises $Ua.$ Therefore, 
\begin{equation*}
((e_1)a)\varphi=
\begin{cases}
(\sum_{i=d+1}^m \alpha_i e_i)+ \alpha_1 e_1 + \underline{(\delta_1 x_1-\alpha^{p^j} \alpha_1 f_1)}\\
\eta_1 (e_1)a+ \ldots +\eta_{d}(e_{d})a + \lambda (\underline{x})a+ \mu_1(f_1)a+ \ldots \mu_d(f_d)a.
\end{cases}
\end{equation*}
 for some $\lambda, \eta_i, \mu_i \in \mathbb{F}_{q^2}^*.$ 
Since there is no $\underline{x},$ or $e_i$ and $f_i$ for $1<i\le d$ in  the first line of the formula, $$\lambda=\eta_2 =\ldots =\eta_r=\mu_2= \ldots=\mu_d=0,$$
so $\alpha_1=\alpha_{r+1}= \ldots= \alpha_{m}=\underline{\delta_1}.$

The same arguments show that $((e_2)a)\varphi=\lambda((e_2)a)$ for some $\lambda \in \mathbb{F}_{q^2}^*.$ So $$((e_2)a)\varphi=(\sum_{i=d+1}^m \theta^{p^j} \alpha_i e_i)+ \alpha_2 e_2=\lambda((e_2)a)$$
and $\lambda=\alpha_2= \theta^{p^j-1}\alpha_{d+1}.$ 
Hence $\theta^{p^j-1}\alpha_{d+1}=\alpha_{d+1}$ since $\alpha_1=\alpha_2,$ so $j=0$ and $g$ is scalar.
 Hence $\tilde{S} \cap S^a\le Z(GU_n(q,{\bf f}_{\beta_1})).$

Assume $d=1.$ Let $a \in SU_n(q, {\bf f}_{\beta_1}) $ be defined by \eqref{465}. Notice that $\langle e_2, f_2 \rangle \subseteq U_2.$ Observe that $S$ stabilises  $U$, so $S^a$ stabilises $Ua.$
Therefore,  $((e_1)a)\varphi$ is
\begin{equation*}
(\sum_{i=3}^m \alpha_i e_i)+ \theta^{p^j} \alpha_2 e_2+ \alpha_1 e_1 +\alpha_2 f_2 -\theta^{p^j} \alpha_1 f_1 + \underline{(\delta_1 x_1-\alpha^{p^j} \alpha_1 f_1)}
\end{equation*}
and
$$((e_1)a)\varphi= \eta_1 (e_1)a+  \lambda (\underline{x})a+ \mu_1(f_1)a$$
 for some $\lambda \in \mathbb{F}_{q^2}^*.$ Thus,
$$\eta_1 = \alpha_1= \alpha_2 = \theta^{p^j-1}\alpha_2= \alpha_3 = \ldots =\alpha_m = \underline{\delta_1},$$ 
 so $j=0$ and $g$ is scalar. Hence $\tilde{S} \cap S^a\le Z(GU_n(q,{\bf f}_{\beta_1})).$





\subsection*{Case 3.} Let $\mu =1,$ so $q \in \{2,3\}$. Without loss of generality, we can assume that $\gamma_{k+l}(M) \in \{GU_2(q), GU_3(2), MU_4(2)\}.$  Let $\{v_1, \ldots, v_{\nu}\}$ be an orthonormal basis of $U_l$, so $2 \le \nu \le 4$. For the remaining $W_i$ and $U_i$ we change basis as in \eqref{basisU}, so  
 \begin{equation} \label{betamu}\beta_1=\{f_1, \ldots, f_m, x_1, \ldots, x_t,v_1, \ldots, v_{\nu}, e_m, \ldots, e_1\},
\end{equation}
and $n=2m+t+ \nu.$
 Denote $n_1 + \ldots + n_k$ by $r,$ so $r\le m.$ Notice that the subspace 
$$E= \langle e_1, \ldots, e_r \rangle$$
is $S$-invariant. Let $\varphi \in \tilde{S}$, so, by {\bf Step 1}, $\varphi=\phi^j g$ with $j \in \{0,1\}$ and
\begin{equation*}%\label{GalmostDiag}
\begin{aligned}
&(e_i)g =  \alpha_i e_i & &\text{ for } i \in \{1, \ldots, m\};\\
&(f_i)g =  \alpha_i^{\dagger} f_i & &\text{ for } i \in \{1, \ldots, m\};\\
&(x_i)g =  \delta_i x_i & &\text{ for } i \in \{1, \ldots, t\}; \\
&(v_i)g =  \lambda_{i1} v_1 +\ldots + \lambda_{i \nu} v_{\nu} & & \text{ for } i \in \{1, \ldots, \nu \},
\end{aligned}
\end{equation*}
for $\alpha_i$, $\delta_i,$ $\lambda_{ji} \in \mathbb{F}_{q^2}.$ Let $\alpha \in \mathbb{F}_{q^2}^*$ be such that $\alpha + \alpha^q=1$ and $\alpha \notin \mathbb{F}_q.$ It is easy to verify existence of such $\alpha$ for $q \in \{2,3\}$ by computation.

\medskip


The three subcases we consider correspond to the following situations:
when  $m=r>0$,
  $m>r>0$,
and  $m \ge r=0$ respectively.


\medskip

{\bf Case (3.1).} Let $m=r>0$, so $l \le 2$ and $l=2$ if and only if $\dim U_1=1,$ so $\zeta=1$ and $U_1= \langle x_1 \rangle.$ If $n \ge 3 \nu+1$, then $r \ge \nu$ (recall that $2 \le \nu \le 4$). For smaller $n$, Theorem \ref{theoremGU} is verified by computation, so  we assume $r \ge \nu.$

We claim that there exists  $a \in SU_n(q, {\bf f}_{\beta_1})$ such that 
\begin{small}
\begin{equation*}%\label{a31}
\begin{aligned}
&(e_1)a =\sum_{s=2}^r f_s +e_1 +(v_1 -\alpha f_1)+ \underline{x_1 -\alpha f_1};& & (f_1)a=f_1 \\
&(e_{i})a = e_{i} - f_1 +(v_i - \alpha f_i); & &(f_{i})a=f_{i}; & &  i\in \{2, \ldots, \nu \}  \\
&(e_{i})a =e_{i} -f_1; & &(f_{i})a=f_{i}; && i\in \{\nu+1, \ldots, r \}   \\
&\underline{(x_1)a=x_1-f_1};&& (v_i)a= v_i-f_i;&& i\in \{1, \ldots, \nu \} 
\end{aligned}
\end{equation*}
\end{small}\\
and $a$ stabilises all other vectors in $\beta_1.$ Here the underlined part is in the formula only if ${\zeta}=1$ and $x_1$ is as in Remark \ref{x1}.  It is routine to check that $\det(a)=1$ and $a$ is an isometry of $(V, {\bf f}),$ so $a \in SU_n(q,{\bf f}_{\beta_1}).$ Let $\varphi \in \tilde{S} \cap S^a$.
 Notice that $S$ stabilises $E= \langle e_1, \ldots, e_r \rangle$, so $S^a$ stabilises  $Ea$. Therefore, $((e_1)a)\varphi$ is
\begin{equation}\label{sl31}
\sum_{i=2}^r\alpha_i^{\dagger} f_i  +\alpha_1 e_1 + (\lambda_{11} v_1 +\ldots + \lambda_{1 \nu} v_{\nu} -\alpha_1^{\dagger}\alpha^{p^j} f_1)+ \underline{\delta_1 x_1 -\alpha_1^{\dagger} \alpha f_1}
\end{equation}
and
$$((e_1)a)\varphi =\eta_1 (e_1)a + \ldots +\eta_r (e_r)a.$$
Since $((e_1)a)\varphi$ does not have $e_i$ for $i \ne 1$ in  \eqref{sl31}, 
$((e_1)a)\varphi=\eta_1(e_1)a,$ so 
$$
\begin{cases}
\eta_1=\alpha_1= \alpha_2^{\dagger}= \ldots = \alpha_r^{\dagger}=\lambda_{11}=\alpha_1^{\dagger}\alpha^{p^j-1}=\underline{\delta_1};\\
\lambda_{12}= \ldots = \lambda_{1 \nu}=0.
\end{cases}
$$ Therefore, in particular, $g$ stabilises $\langle v_1 \rangle,$ a non-degenerate subspace, so $\lambda_{11}^{\dagger}=\lambda_{11}$ and $$\alpha_1^{\dagger}=\alpha_1 = \ldots = \alpha_r=\alpha_1^{\dagger}\alpha^{p^j-1}.$$ Hence $\alpha^{p^j-1}=1$ and $j=0$ since $\alpha \notin \mathbb{F}_q.$ The same arguments for $((e_i)a)\varphi$ with $i=2, \ldots , \nu$ show that $\lambda_{ii}=\alpha_1$ and $\lambda_{ij}=0$ for $j \ne i.$ Therefore, $g$ is scalar and $\tilde{S} \cap S^a \le Z(GU_n(q,{\bf f}_{\beta_1})).$

%{\bf Step 2.3.2} Assume that $m>r$ and $r \ge \nu.$
%We claim that there exists  $a \in GU_n(q, {\bf f}_{\beta_1})$ such that
%\begin{equation*}%\label{a14}
%{
%\begin{aligned}
%&(e_1)a =\sum_{s=r+1}^m e_s +e_1 +(v_1 -\alpha f_1)+ \underline{x_1 -\alpha f_1};& & (f_1)a=f_1; \\
%&(e_{i})a =\sum_{s=r+1}^m e_{s}+ e_{i} +(v_i - \alpha f_i); & &(f_{i})a=f_{i}; & &  i\in \{2, \ldots, \nu \}  \\
%&(e_i)a =\sum_{s=r+1}^m e_{s}+ e_{i};  & & (f_{i})a=f_{i}; & &  i\in \{\nu+1, \ldots, r \}  \\
%&(e_{i})a =e_{i}; & &(f_{i})a=f_{i} - \sum_{s=1}^r f_{s}; && i\in \{r+1, \ldots, m \}   \\
%&\underline{(x_1)a=x_1-f_1};&& (v_i)a= v_i-f_i;&& i\in \{1, \ldots, \nu \}
%\end{aligned}
%}
%\end{equation*}
%and $a$ stabilises the remaining vectors in $\beta_1.$   Here the underlined part is present in the formula only if ${\zeta}=1$ and $x_1$ is as in Remark \ref{x1}.  It is routine to check that $\det(a)=1$ and $a$ is an isometry of $(V, {\bf f}),$ so $a \in SU_n(q,{\bf f}_{\beta_1}).$
%We claim that $\tilde{S} \cap S^a \le Z(GU_n(q)).$ Let $\varphi \in \tilde{S} \cap S^a$.
% Notice that $S$ stabilises subspace the $E$, so $S^a$ stabilises $(E)a.$ Therefore, 
%\begin{equation}\label{2pres4}
%((e_1)a)\varphi=
%\begin{cases}
%\sum_{i=r+1}^m\alpha_i e_i +\alpha_1e_1 + (\lambda_{11} v_1 +\ldots + \lambda_{1 \nu} v_{\nu} -\alpha_1^{\dagger} \alpha^{p^j} f_1)+ \underline{\delta_1 x_1 -\alpha_1^{\dagger} \alpha^{p^j-1} f_1}\\
%\eta_1 (e_1)a+ \ldots +\eta_r(e_r)a.
%\end{cases}
%\end{equation}
%Since $\eta_i(e_i)a$ for $i>1$ has $\eta_i$ as a coordinate for $e_i$ with respect to $\beta_1$ and $((e_1)a)\varphi$ has $0$ as these coordinates (see the first line of \eqref{2pres4}), $\eta_i=0$ for all $i>1.$  Thus, 
%\begin{equation*}
%\begin{cases}
%\eta_1=\alpha_1= {\alpha_1^{\dagger} \alpha^{p^j-1}} =\alpha_{r+1}=\ldots=\alpha_m= \lambda_{11}=\underline{\delta_1}\\
%\lambda_{12}= \ldots = \lambda_{1 \nu}=0.
%\end{cases}
%\end{equation*}
%Therefore, in particular, $g$ stabilises $\langle v_1 \rangle,$ which is non-degenerate subspace, so $\lambda_{11}^{\dagger}=\lambda_{11}$ and $$\alpha_1^{\dagger}=\alpha_1=\alpha_1^{\dagger}\alpha^{p^j-1}.$$ Hence $j=0.$ The same arguments for $((e_i)a)\varphi$ with $i=2, \ldots, r$ show that
% $$\lambda_11=\ldots = \lambda_{\nu \nu}=\alpha_i=\alpha_1=\underline{\delta_1}$$ for all $i=1, \ldots, m$ and $\lambda_{is}=0$ if $i \ne s$ for $i,s \in \{1, \ldots, \nu\}.$ So $g$ is scalar and $\tilde{S} \cap S^a\le Z(GU_n(q,{\bf f}_{\beta_1})).$

\medskip

{\bf Case (3.2).} Assume that $m>r>0$. Recall $\mu \le 1$ and Remark \ref{remni1}; thus, if $\nu=2$, then $m \ge \nu$ for $n \ge 6$; if $\nu \in \{3,4\}$, then $m \ge \nu$ for $n \ge 9$.       For smaller $n$, Theorem \ref{theoremGU} is verified by computation, so we assume $m \ge \nu$.
    
We claim that there exists  $a \in SU_n(q, {\bf f}_{\beta_1})$ such that 

\begin{equation*}%\label{a33}
\begin{aligned}
&(e_1)a =\sum_{s=n_1+1}^m e_s +e_1 + (v_1 - \alpha f_{1}) + \underline{(x_1 -\alpha f_1)} ; \\ & (f_{1})a=f_{1}; & &  \\
&(e_i)a = e_i + (v_i - \alpha f_i); & &  i\in \{2, \ldots,  \nu \}  \\  & (f_{i})a=f_{i}; & &  i\in \{2, \ldots,  \nu \}  \\
&(e_i)a =e_i; & &  i\in \{\nu+1, \ldots, m \}  \\    & (f_{i})a=f_{i} -\delta_{(i>n_1)} f_1 ; & &  i\in \{\nu+1, \ldots, m \}  \\
&\underline{(x_1)a=x_1-f_{1}};\\ & (v_i)a= v_i-f_{i};&& i\in \{1, \ldots, \nu \}  
\end{aligned}
\end{equation*}
and $a$ stabilises all other vectors from $\beta_1.$ Here the underlined part is  in the formula only if ${\zeta}=1$ and $x_1$ is as in Remark \ref{x1}.  It is routine to check that $\det(a)=1$ and $a$ is an isometry of $(V, {\bf f}),$ so $a \in SU_n(q,{\bf f}_{\beta_1}).$
We claim that $\tilde{S} \cap S^a \le Z(GU_n(q)).$ Let $\varphi \in \tilde{S} \cap S^a$.
 Observe that $S$ stabilises  $E=\langle e_1, \ldots, e_{n_1} \rangle$, so $S^a$ stabilises $Ea.$ Therefore,  $((e_1)a)\varphi$ is
\begin{equation*}%\label{2pres33}
\sum_{i=n_1+1}^m\alpha_i e_i +\alpha_1 e_1 +(\lambda_{11} v_1 +\ldots + \lambda_{1 \nu} v_{\nu} -\alpha_1^{\dagger} \alpha^{p^j} f_1)+ \underline{\delta_1 x_1 -\alpha_1^{\dagger} \alpha^{p^j-1} f_1} 
\end{equation*}
and
$$((e_1)a)\varphi=\eta_1 (e_1)a+ \ldots +\eta_r(e_r)a.$$
Since $\eta_i(e_i)a$ for $i>1$ has $\eta_i$ as a coefficient for $e_i$ with respect to $\beta_1$ and $((e_1)a)\varphi$ has $0$ as these coefficients, $\eta_i=0$ for all $i>1.$  Thus, 
\begin{equation*}%\label{alphas}
\begin{cases}
\eta_1=\alpha_1=  \alpha_{n_1+1}=\ldots=\alpha_m=\lambda_{11}=\alpha^{p^j-1}\alpha_1^{\dagger}= \underline{\delta_1}\\
\lambda_{12} = \ldots = \lambda_{1 \nu}=0.
\end{cases}
\end{equation*}
Therefore, in particular, $g$ stabilises $\langle v_1 \rangle,$ a non-degenerate subspace, so $\lambda_{11}^{\dagger}=\lambda_{11}$ and $\alpha_1^{\dagger}=\alpha_1$. Hence $\alpha^{p^j-1}=1$ and $j=0.$




The same argument for $((e_{i})a)g$ with $i=r+2, \ldots, r+ \nu$ shows that $\lambda_{ii}=\alpha_1$ for $i \in \{1, \ldots, \nu\}$ and $\lambda_{ij}=0$ for $i \ne j.$ Therefore, $g$  is scalar by Remark \ref{x1}. 

\medskip

{\bf Case (3.3).} Assume that $r=0$. Recall $\mu \le 1$ and Remark \ref{remni1}; thus, if $\nu$ is $2$, $3$ or $4$, then $m \ge \nu$ for $n \ge 6$, $10$ and $12$  respectively.     For smaller $n$, Theorem \ref{theoremGU} is verified by computation, so we assume $m \ge \nu$.    Let $U$ be one of  $\{U_1, ..., U_{l - 1}\}$ with maximum dimension.   So we can assume $$U=U_1 = \langle f_1, \ldots, f_d, \underline{x}, e_d, \ldots, e_1 \rangle$$
where $d$ and $\underline{x}$ are defined as in \eqref{U1dx}.


Assume $d=1.$
We claim that there exists  $a \in SU_n(q, {\bf f}_{\beta_1})$ such that 
\begin{align*}%\label{a33}
&(e_1)a =\sum_{s=3}^m e_s + \alpha^{\dagger}e_2 +e_1 +f_2 -\alpha^{\dagger}f_1 + (v_1 - \alpha f_{1}) + \underline{(x_1 -\alpha f_1)} ; & & (f_{1})a=f_{1}; & &  
\end{align*}
\begin{align*}
&(e_2)a = e_2 + (v_2 - \alpha f_2); & & (f_{2})a=f_{2}- \alpha^{-1}f_1; & &    \\
&(e_i)a =e_i+ v_i - \alpha f_i + \alpha f_1;   & & (f_{i})a=f_{i} - f_1 ; & &  i\in \{3, \ldots, \nu \}  \\
&(e_i)a =e_i;   & & (f_{i})a=f_{i} - f_1 ; & &  i\in \{\nu+1, \ldots, m \}  \\
&\underline{(x_1)a=x_1-f_{1}};&& (v_i)a= v_i-f_{i};&& i\in \{1, \ldots, \nu \}  
\end{align*}
and $a$ stabilises all other vectors from $\beta_1.$ Here the underlined part is  in the formula only if ${\zeta}=1$ and $x_1$ is as in Remark \ref{x1}.  It is routine to check that $\det(a)=1$ and $a$ is an isometry of $(V, {\bf f}),$ so $a \in SU_n(q,{\bf f}_{\beta_1}).$
We claim that $\tilde{S} \cap S^a \le Z(GU_n(q)).$ Let $\varphi \in \tilde{S} \cap S^a$.


%Notice that $\alpha_i^{\dagger}=\alpha_i$ for $i>r$ since $S$ acts as a scalar on $U_s$ for $s=1, \ldots, l-1$ by the first step. In particular, $\alpha_1^{\dagger}=\alpha_1.$ The same arguments for $((e_i)a)\varphi$ with $i=2, \ldots, r$ show  that 
%$$\alpha_1 = \ldots= \alpha_r = \alpha_{r+1}=\ldots=\alpha_m.$$
%Let $W=E \oplus (\bot_{i=1}^{l-1} U_i )$ if $\zeta=0$ and $W=E \oplus (\bot_{i=2}^{l-1} U_i )$ if $\zeta=1$ (so $\dim U_1=1$). Notice that $(W)S=W.$ First let $\zeta=1.$ Consider 

Observe that $S$ stabilises  $U$, so $S^a$ stabilises $Ua.$ Therefore, 
\begin{equation*}%\label{2pres332}
((e_{1})a)\varphi=
\begin{cases}
\begin{aligned}
\sum_{s=3}^m \alpha_s e_s + (\alpha^{\dagger})^{p^j} \alpha_2 e_2 +& \alpha_1 e_1 +\alpha_2 f_2 -(\alpha^{\dagger})^{p^j} \alpha_1 f_1 + \\ +& (\sum_{i=1}^{\nu}\lambda_{1i} v_i) -\alpha_1 \alpha^{p^j} f_1)+ \underline{\delta_1 x_1 -\alpha_1 \alpha^{p^j-1} f_1}
\end{aligned} \\
\eta_1 (e_1)a+\mu_{r+1}(f_{r+1})+ \lambda \underline{x}.
\end{cases}
\end{equation*}
Thus, $$
\begin{cases}
\eta_1=\alpha_1=\alpha_2=(\alpha^{\dagger})^{p^j-1} \alpha_2 = \alpha_3= \ldots = \alpha_m=\lambda_{11}= \underline{\delta_1}\\
\lambda_{12}= \ldots = \lambda_{1 \nu}=0.
\end{cases}
$$
 Hence $\alpha^{p^j-1}=1$ and $j=0.$
The same argument for $((e_{i})a)g$ with $i=r+2, \ldots, r+ \nu$ shows that $\lambda_{ii}=\alpha_1$ for $i \in \{1, \ldots, \nu\}$ and $\lambda_{ij}=0$ for $i \ne j.$ Therefore, $g$  is scalar by Remark \ref{x1}. 

Assume $d \ge 2.$ 
We claim that there exists  $a \in SU_n(q, {\bf f}_{\beta_1})$ such that 
\begin{align*}
&(e_1)a =\sum_{s=d+1}^m e_s  +e_1  + (v_1 - \alpha f_{1}) + \underline{(x_1 -\alpha f_1)} ; & & (f_{1})a=f_{1}; \\ &  (v_1)a=v_1 -f_1 - \alpha^q f_2; \\
&(e_2)a =  e_2  +\alpha v_1 - \alpha f_1 + v_2 - 2 \cdot \alpha f_2; & & (f_{2})a=f_{2}; \\ &  (v_2)a= v_2-f_2;
\end{align*}
\begin{align*}  
&(e_i)a =e_i+ v_i - \alpha f_i + \delta_{i>d} \alpha f_1 ;   & & (f_{i})a=f_{i} - \delta_{i>d} f_1 ; & &  i\in \{3, \ldots, \nu \}  \\
&(e_i)a =e_i;   & & (f_{i})a=f_{i} - \delta_{i>d} f_1 ; & &  i\in \{\nu+1, \ldots, m \}  \\
&\underline{(x_1)a=x_1-f_{1}};&& (v_i)a= v_i-f_{i} + \delta_{i>d} f_1;&& i\in \{3, \ldots, \nu \}  
\end{align*}
and $a$ stabilises all other vectors from $\beta_1.$ Here the underlined part is in the formula only if ${\zeta}=1$ and $x_1$ is as in Remark \ref{x1}; $\delta_{i>d}$ is $1$ if $i>d$ and $0$ otherwise.  It is routine to check that $\det(a)=1$ and $a$ is an isometry of $(V, {\bf f}),$ so $a \in SU_n(q,{\bf f}_{\beta_1}).$
We claim that $\tilde{S} \cap S^a \le Z(GU_n(q)).$ Let $\varphi \in \tilde{S} \cap S^a$.

Observe that $S$ stabilises  $U$, so $S^a$ stabilises $Ua.$ Therefore, $((e_{1})a)\varphi$ is
\begin{equation*}%\label{2pres332}
\sum_{s=d+1}^m \alpha_s e_s +  \alpha_1 e_1  + (\sum_{i=1}^{\nu}\lambda_{1i} v_i) -\alpha_1 \alpha^{p^j} f_1+ \underline{\delta_1 x_1 -\alpha_1 \alpha^{p^j-1} f_1} 
\end{equation*}
and
$$((e_{1})a)\varphi=\sum_i^{d} \eta_i (e_i)a+ \sum_i^{d}\mu_{i}(f_{i})+ \lambda \underline{x}.$$
Since $\eta_i(e_i)a$ for $i>1$ has $\eta_i$ as a coefficient for $e_i$ with respect to $\beta_1$ and $((e_1)a)\varphi$ has $0$ as these coefficients in the first line of the formula above, $\eta_i=0$ for all $i>1.$ The same arguments for $\mu_i$ and $\lambda$ shows that $\lambda=0$ and $\mu_i=0$ for $i>1.$ Thus, $$
\begin{cases}
\eta_1=\alpha_1 = \alpha_{d+1}= \ldots = \alpha_m=\lambda_{11}= \underline{\delta_1}\\
\lambda_{12}= \ldots = \lambda_{1 \nu}=0.
\end{cases}
$$
So $g$ stabilises $\langle v_1 \rangle$ and its orthogonal complement $\langle v_2, \ldots, v_{\nu}  \rangle$ in $\langle v_1, \ldots, v_{\nu}  \rangle$. In particular $\lambda_{i1}=0$ for $i \in \{2, \ldots, \nu\}.$    Recall that $\alpha_1=\alpha_2,$ since $d \ge 2.$ Consider 
\begin{equation*}%\label{2pres332}
((e_{2})a)\varphi=
\begin{cases}
 \alpha_1 e_2  + \alpha^{p^j} \lambda_{11} - \alpha^{p^j} \alpha_1 f_1 + (\sum_{i=2}^{\nu}\lambda_{2i} v_i) - 2 \cdot \alpha_1 \alpha^{p^j} f_1 \\
\sum_i^{d} \eta_i (e_i)a+ \sum_i^{d}\mu_{i}(f_{i})+ \lambda \underline{x}.
\end{cases}
\end{equation*}
The same arguments as above show that 
$$
\begin{cases}
\alpha_1 = \alpha^{p^j-1} \lambda_{11} = \lambda_{22} \\ 
\lambda_{23}= \ldots = \lambda_{2 \nu}=0.
\end{cases}
$$
 Hence $\alpha^{p^j-1}=1$ and $j=0.$
The same argument for $((e_{i})a)g$ with $i=3, \ldots,  \nu$ shows that $\lambda_{ii}=\alpha_1$ for $i \in \{1, \ldots, \nu\}$ and $\lambda_{ij}=0$ for $i \ne j.$ Therefore, $g$  is scalar and $\tilde{S} \cap S^a \le Z(GU_n(q)).$


\medskip
Hence in all cases there exist four conjugates of $S$ in $G$ which intersect in a group of scalars, except when $$(n,q)=(5,2) \text{ with } k=l=1, n_1=1, n_2=3.$$ Here $b_S(S \cdot SU_n(q))=5$ and $\Reg_S(S \cdot GU_n(q),5)\ge 5$ are verified by computation. This already arises in {\bf Case (3.1)}. This concludes the proof of Theorem \ref{lem421}.
\end{proof}

Theorem \ref{theoremGU} now follows by  Lemma \ref{lemn4uni} and Theorems \ref{GU4sp} and  \ref{lem421}.

%%%%%%%%%%%%%%%%%%%%%%%%%%%%%%%%%%%%%%%%%%%


\section{Symplectic groups}

We prove Theorems \ref{theoremSp} and \ref{theoremSpGR} in Sections \ref{sec431} and \ref{sec432} respectively.

\subsection{Solvable subgroups contained in $\GS_n(q)$.} 
\label{sec431}
Here $S$ is a maximal solvable subgroup of $\GS_n(q)$ where $n\ge 4.$
Our goal  is to prove the following theorem.

\begin{T3}
Let $X=\GS_n(q)$ and  $n \ge 4$. If $S$ is a maximal solvable subgroup of $X$, 
 then  $b_S(S \cdot Sp_n(q)) \le 4,$ so $\Reg_S(S \cdot Sp_n(q),5)\ge 5$.
\end{T3}


 If $(n,q)=(4,2),$ then   Theorem \ref{theoremSp} is verified by computation.  


\begin{Lem}
\label{3conjMSP}
Let $M= S \cap GSp_n(q).$ If $S$  stabilises no non-zero proper subspace of $V$, then there exist $y,z \in Sp_n(q)$ such that $M \cap M^y \cap M^z \le Z(GSp_n(q))$ unless $n=4$, $q \in \{2,3\}$ and $M$ is defined as $S$ in \eqref{sympirr2} of Theorem $\ref{sympirr}$. 
\end{Lem}
\begin{proof}
If $M \le Sp_n(q)$ is irreducible, then such $y,z$ exist by Theorem \ref{sympirr}. Assume that $M$ is reducible. The same arguments as in the proof of Lemma \ref{GammairGL} show that $M$ is completely reducible.  If $V$ is not $\mathbb{F}_q[M]$-homogeneous, then $S$ (and $M$) stabilises a decomposition of $V$ as in Lemma \ref{ashb}, and such $y,z$ exist by the proof of Theorem \ref{sympirr}. If $V$ is  $\mathbb{F}_q[M]$-homogeneous, then $M$ stabilises a decomposition as in    Lemma \ref{ashb} by \cite[(5.2) and (5.3)]{asch}, and such $y,z$ exist by the proof of Theorem \ref{sympirr}. 
\end{proof}

\begin{Th}
\label{SpSirr41}
Theorem {\rm \ref{theoremSp}} holds if $S$ stabilises no non-zero proper subspaces of $V$.
\end{Th}
\begin{proof}
If $f=1,$ then the theorem follows by Theorem \ref{sympirr}, so we assume $f>1.$ It follows by Theorem \ref{bernclass} unless  $S$ lies in a maximal subgroup $H$ of $S \cdot Sp_4(q)$ such that the action of $S \cdot Sp_4(q)$ on right cosets of $H$ is a standard action. Hence one of the following holds (see Definition \ref{nonstdef} and \cite[Table 1]{burness}):
\begin{itemize}
 \item[$(a)$] $q=2^f$ and $H$ is of type $O_n^{\epsilon}(q)$;
\item[$(b)$] $n=4$ and $H$ is the stabiliser of a decomposition $V= V_1 \bot V_2$ with non-degenerate $V_i$ of dimension $2$;
\item[$(c)$] $n=4$ and $H$ is the normaliser in $\GS_4(q)$ of a field extension of the field of scalar matrices.  
\end{itemize}

 First, assume that $(a)$ holds, so $q=2^f$ with $f>1$ and $H$ is a group of semisimilarities of $V$ with respect to a non-degenerate quadratic form $Q:V \to V.$ Let ${\bf f}_{Q}$ be defined by 
\begin{equation}
\label{fpolarQ}
{\bf f}_{Q}(u,v)=Q(u + v) - Q(u) - Q(v) \text{ for all } u,v \in V.
\end{equation}  By \cite[Table 4.8.A]{kleidlieb}, ${\bf f}_{Q}={\bf f}$. By \cite[Proposition 2.5.3]{kleidlieb},  there exists a basis 
$$\beta = \{f_1, \ldots, f_m, e_1, \ldots, e_m\}$$ as in Lemma \ref{sympbasisl} such that
\begin{itemize}
\item if $\epsilon=+,$ then $Q(f_i)=Q(e_i)=0$ for $i \in \{1, \ldots, m\}$;
\item if $\epsilon=-,$ then $Q(f_i)=Q(e_i)=0$ for $i \in \{1, \ldots, m-1\},$ $Q(f_m)=\mu$ and $Q(e_m)=1$.
\end{itemize}
Here $\mu \in \mathbb{F}_q^*$ is such that the polynomial $x^2+x+ \mu$ is irreducible over $\mathbb{F}_q.$ 

By Theorem \ref{sympirr}, there exist $x,y \in Sp_n(q)$ such that 
$$S \cap S^x \cap S^y \cap GSp_n(q) \le Z(GSp_n(q)).$$
Therefore, by Lemma \ref{scfield}, we may assume that if $\varphi \in S \cap S^x \cap S^y,$ then $\varphi= (\phi_{\beta})^j \cdot \lambda I_n$ for some $\lambda \in \mathbb{F}_q^*$ and $j \in \{0,1, \ldots, f-1\}.$

Let $\theta$ be a generator of $\mathbb{F}_q^*$ and let $z \in Sp_n(q)$ be defined as follows:
\begin{align*}
&(e_1)z=e_1 + \theta f_1;  && (f_1)z=f_1;\\
&(e_i)z=e_i; && (f_i)z=f_i & \text{ for } i \in \{2, \ldots, m\}.
\end{align*}

Notice that $H^z$ consists of semisimilarities of $V$ with respect to the quadratic form $Q_1$ defined by the rule $Q_1(v)=Q((v)z^{-1})$ for all $v \in V.$ Let us show that if $\varphi \in S \cap S^x \cap S^y$ is not a scalar, then it is not a semisimilarity with respect to $Q_1.$ Indeed, if $\varphi$ is a semisimilarity with respect to $Q_1,$ then 
\begin{equation}
\label{Qformsp}
Q_1((e_1 +\theta f_1)\varphi) = \delta Q_1 (e_1 + \theta f_1)^{\sigma}
\end{equation}
for some $\lambda \in \mathbb{F}_q^*$ and $\sigma \in \Aut (\mathbb{F}_q).$ 
Observe $$Q_1(e_1 +\theta f_1)=Q((e_1 +\theta f_1)z^{-1})=Q(e_1)=0.$$
On the other hand 
\begingroup
\allowdisplaybreaks
\begin{align*}
Q_1((e_1 +\theta f_1)\varphi)& =  Q_1(\lambda (e_1 +\theta^{p^j} f_1)) \\ & =  \lambda^2 Q_1(e_1 +\theta^{p^j} f_1) \\ & =   \lambda^2 Q((e_1 +\theta^{p^j} f_1)z^{-1})\\ &=  \lambda^2 Q((e_1 +\theta f_1)z^{-1} + ((\theta^{p^j}- \theta)f_1)z^{-1}) \\ & =  \lambda^2 Q(e_1  + (\theta^{p^j}- \theta)f_1)\\ & =\lambda^2 (\theta^{p^j}- \theta). 
\end{align*}
\endgroup
The last equality is obtained using \eqref{fpolarQ}. Hence \eqref{Qformsp} holds only if $j=0$ and $\varphi$ is scalar. Therefore, $S \cap S^x \cap S^y \cap S^z \le Z(GSp_n(q)).$ %end of red

\medskip


Now assume  that $(b)$ holds, so $S$ stabilises a decomposition $V= V_1 \bot V_2$ with $V_i= \langle e_i, f_i \rangle$, where $\beta =\{e_1, f_1, e_2, f_2\}$ with $e_i$ and $f_i$ as in \eqref{sympbasis}. Let $y,z \in Sp_4(q)$ be as in {\bf Case 1b} of the proof of Theorem \ref{sympirr}. Denote $(V_i)y$ and $(V_i)z$ by $W_i$ and $U_i$ respectively for $i \in \{1,2\}.$ Let $\theta$ be a generator of $\mathbb{F}_q^*$ and let $a \in Sp_4(q, {\bf f}_{\beta})$
be $$\begin{pmatrix}
1      & \multicolumn{1}{c|}{0}& 0& 0   \\
0      & \multicolumn{1}{c|}{1}& \theta & 0 \\ \cline{1-4}
0      & \multicolumn{1}{c|}{0}& \theta & 0   \\
0      & \multicolumn{1}{c|}{1}& 0 & \theta^{-1}
\end{pmatrix}.$$
Consider $\varphi \in S \cap S^y \cap S^z \cap S^a,$ so $\varphi = \phi^j g$ with $j \in \{0,1, \ldots, f-1\}$ and $g \in GSp_4(q, {\bf f}_{\beta})$ by Lemma \ref{uniGamsdp}. By {\bf Case 1b} of the proof of Theorem \ref{sympirr}, $\varphi$ stabilises $V_i,$ $W_i$ and $U_i$ for each $i.$ Therefore, $\varphi$ stabilises 
$\langle e_1 \rangle=V_1 \cap U_1$, $\langle f_1 \rangle=V_1 \cap W_1$, $\langle e_2 \rangle=V_2 \cap W_2$.  So $\varphi$ stabilises $\langle f_1 + e_2\rangle \subseteq (V_1)z$ and $\langle f_1 + \theta e_2\rangle \subseteq (V_1)a$. Let $(e_1)g= \lambda_1 e_1,$ $(f_1)g= \lambda_2 f_1$ and $(e_2)g= \lambda_3 e_2$. Therefore,  $(f_1 + e_2)\varphi = \lambda_2 f_1+ \lambda_3 e_2$ and $\lambda_2=\lambda_3$. Also,    $(f_1 + \theta e_2)\varphi = \lambda_2 f_1+ \theta^{p^j}\lambda_3 e_2,$ so $\theta^{p^j-1}=1$ and $j=0.$ In particular, $S \cap S^y \cap S^z \cap S^a \le M \cap M^y \cap M^z$. Therefore, $\varphi$ is scalar since  $M \cap M^y \cap M^z \le Z(GSp_4(q))$ by {\bf Case 1b} of the proof of Theorem \ref{sympirr}.

\medskip

Finally, assume that $(c)$ holds, so $S$ lies in the normaliser in $\GS_4(q)$ of a field extension of the field of scalar matrices. Thus, $S \le R=GL_2(q^2)\rtimes \langle \psi \rangle$ and 
$M \le GL_2(q^2).2= GL_2(q^2)\rtimes \langle \psi^f \rangle$ where $\psi^2 = \phi$ and $\psi^f \in GSp_4(q).$ 

Assume that $M$ lies in the normaliser in  $R$ of a Singer cycle of $GL_2(q^2).$ By \eqref{2sindiagodd} and \eqref{2sindiageven}, there exists $x \in SL_2(q^2)\le Sp_4(q)$ such that $$S \cap S^x \le  GL_2(q).2 \le GSp_4(q),$$ so $S \cap S^x \le M.$ By Lemma \ref{3conjMSP}, there exist $y,z$ such that $M \cap M^y \cap M^z \le Z(GSp_4(q)),$ so 
$$(S \cap S^x) \cap S^y \cap S^z \le Z(GSp_4(q)).$$

Assume that $M$ does not lie in the normaliser in  $R$ of a Singer cycle of $GL_2(q^2)$ and let $M_1=M \cap GL_2(q).$ By Theorem \ref{irred}, there exists $x \in SL_2(q^2)$ such that $M_1 \cap M_1^x \le Z(GL_4(q))$. Hence $S \cap S^x \le Z(GL_2(q^2)) \rtimes \langle \psi \rangle.$ Let $N$ be  $S \cap S^x$.  So $|N/Z(GSp_4(q))|$ divides $(q+1)\cdot 2f$ and 
\begin{equation}
\label{Asp4}
A=|N/Z(GSp_4(q)) \cap PGSp_4(q)|
\end{equation} divides $(q+1)\cdot 2f$.   


We claim that $\hat{Q}((N \cdot Sp_4(q)/Z(GSp_4(q)),2)<1$ where  $\hat{Q}(G,c)$ is as in \eqref{ver}. Denote $N/Z(GSp_4(q))$ by $H$. By Lemma \ref{fprAB}, if  $x_1,\ldots,x_k$ represent distinct $G$-classes such that $\sum_{i=1}^k |x_i^G \cap  H| \le A$ and $|x_i^G| \ge B$ for all $i \in \{1, \ldots, k\},$ then
$$\sum_{i=1}^m |x_i^G| \cdot \fpr (x_i)^c \le B \cdot (A/B)^c.$$
We take $A$ as in \eqref{Asp4} since $A \ge |H|.$  Lemma \ref{6} implies that  $\nu(g)\ge n/2 =2$ for $g \in N \cap GSp_4(q)$. For elements in $PGSp_4(q)$ of prime order with $s=\nu(x) \in \{2,3\}$  we use \eqref{5uni} as a lower bound for $|x_i^G|$. If $x \in H \backslash PGSp_4(q)$ has prime order, then we use the corresponding bound for $|x^G|$ in \cite[Corollary 3.49]{fpr2}. We take $B$ to be
the smallest of these bounds for $|x_i^G|.$ Such $A$ and $B$ are sufficient to obtain $\hat{Q}((N \cdot Sp_4(q)/Z(GSp_4(q)),2)<1$  for $q>4.$ Hence $b_S(S \cdot SU_4(q)) \le 4$. For $q=4$ the statement is verified by computation. 
\end{proof}







\begin{Th}
\label{Spqgt3}
Theorem {\rm \ref{theoremSp}} holds for $q > 3$ if $S$ stabilises a non-zero proper subspace of $V$. 
\end{Th}
\begin{proof} The proof proceeds in two steps. In {\bf Step 1} we obtain three conjugates of $S$ such that elements of their intersection have special shape. In {\bf Step 2} we find a fourth conjugate of $S$ such that the intersection of the four is a group of scalars.

\subsection*{Step 1} This is similar to the first step of the proof of Theorem \ref{theoremGU}. Fix a basis $\beta$ of  $(V, {\bf f})$ as in Lemma \ref{unist}, so ${\bf f}_{\beta}$ is as in \eqref{fst} and elements of $S$ take shape $\phi^j g$ with $g$ as in \eqref{gst} and $j \in \{0,1, \ldots, f-1\}$. We consider $S$ as a subgroup of $\GS_n(q,{\bf f}_{\beta})$ and  let $M=S \cap GSp_n(q,{\bf f}_{\beta}).$ We obtain three conjugates of $S$ such that their intersection consists of elements $\phi^j g$ where $g$ is diagonal with respect to $\beta.$




Let $\gamma_i$ be as in Lemma \ref{unist}.
Let $x$ be the matrix 
\begin{equation}\label{exSP}
 \left(
\begin{smallmatrix}
        & & & & & & &    & I_{n_1} \\
        & & & & & & &  \reflectbox{$\ddots$}  &\\
        & & & & & &I_{n_k} &    & \\
        & & &I_{n_{k+1}} & & & &    &\\ 
        & & & &\ddots & & &    &\\
        & & & & &I_{n_k+l} & &    &\\ 
        & &-I_{n_k} & & & & &    & \\
        &\reflectbox{$\ddots$} & & & & & &    &\\
-I_{n_1} & & & & & & &    & 
\end{smallmatrix} \right).
\end{equation}
Observe that $x{\bf f}_{\beta}{x}^{\top}={\bf f}_{\beta},$ so $x \in Sp_n(q,{\bf f}_{\beta}).$ It is easy to see that if $g \in M,$ so it has shape \eqref{gst}, then $g^x$ has shape \eqref{antigstSp}.




%Let $q>3.$
 Notice that by the proof of Theorem \ref{theorem}, if $N$ is a solvable subgroup of $\GL_n(q)$  stabilising no  non-zero proper subspace, then $N \cap GL_n(q)$ lies in an irreducible maximal  solvable subgroup of $GL_n(q).$ Therefore, by  Theorem  \ref{irred} and Lemmas \ref{starc} and \ref{3conjMSP},  there exist $y_i, z_i \in GL_{n_i}(q)$ for $i=1, \ldots, k$ and $y_i, z_i \in Sp_{n_i}(q)$ for $i \in \{k+1, \ldots, k+l\}$ such that 
\begin{equation}\label{smintSp}
\begin{split}
\gamma_i(M) \cap \gamma_i(M)^{y_i} \cap (\gamma_i(M)^{\dagger})^{z_i} & \le Z(GL_{n_i}(q)) \text{ for } i \in\{1, \ldots, k\};\\
\gamma_i(M) \cap \gamma_i(M)^{y_i} \cap \gamma_i(M)^{z_i} & \le Z(GL_{n_i}(q)) \text{ for } i \in \{k+1, \ldots, k+l \} .
\end{split}
\end{equation}
\begin{figure}[t]
\begin{equation}\label{antigstSp}
\Scale[0.95]{\begin{pmatrix}
\gamma_{1}(g)& &\multicolumn{1}{l|}{0} & &  & & &    &0  \\
    *    & \ddots &\multicolumn{1}{l|}{} & & & & &    &\\
*        &* & \multicolumn{1}{l|}{\gamma_{k}(g)}& & & & &    & \\  \cline{1-6}
 *       &\ldots & \multicolumn{1}{l|}{*} &\gamma_{k+1}(g) & & \multicolumn{1}{l|}{0}  & &    &\\ 
  *      &\ldots &  \multicolumn{1}{l|}{*}& &\ddots &\multicolumn{1}{l|}{}       & &   &\\
   *     &\ldots &  \multicolumn{1}{l|}{*}&0 & &\multicolumn{1}{l|}{\gamma_{k+l}(g)}    &  &    & \\ \cline{4-9} 
    *    &\ldots & & & &\multicolumn{1}{l|}{*} & {{\tau(g) \gamma_{k}(g)}^{\dagger}}  &    &0 \\
    *    &\ldots & & & &\multicolumn{1}{l|}{*} &* & \ddots   &\\
*        &\ldots & & & &\multicolumn{1}{l|}{*} &* & *   & {{\tau(g) \gamma_{1}(g)}^{\dagger}} \\  
\end{pmatrix}}
\end{equation}
\end{figure}
 Denote by $y$ and $z$ the block-diagonal matrices 
\begin{equation}\label{yizi}
\begin{split}
&\diag[y_1^{\dagger}, \ldots, y_k^{\dagger}, y_{k+1}, \ldots, y_{k+l}, y_k, \ldots, y_1]  \text{ and }\\
&\diag[z_1^{\dagger}, \ldots, z_k^{\dagger}, z_{k+1}, \ldots, z_{k+l}, z_k, \ldots, z_1]
\end{split}
\end{equation}
 respectively. It is routine to check that $y,z \in Sp_n(q,{\bf f}_{\beta}).$

 %  By Lemma \ref{primconj} for $i = 1, \ldots, k$ there exist $\tilde{g}_i \in GL_n(q^2)$ such that 
%$$(\gamma_i(S)^{\dagger})^{\tilde{g}_i}=\gamma_i(S).$$  Denote by $\tilde{g}$ the block-diagonal matrix
%\begin{equation}\label{gd1}
%\diag(\tilde{g}_1, \ldots, \tilde{g}_k, I_{n_{k+1}}, \ldots, I_{n_{k+l}}, \tilde{g}_k^{\dagger}, \ldots, \tilde{g}_1^{\dagger}) \in GU_n(q).
%\end{equation}

 Therefore, if $g \in M \cap M^{xz},$ then $g$ is the block-diagonal matrix
\begin{equation}\label{gd1Sp}
\diag [\tau(g)g_1^{\dagger}, \ldots, \tau(g)g_k^{\dagger}, g_{k+1}, \ldots, g_{k+l}, g_k, \ldots, g_1],
\end{equation}
where $g_i \in \gamma_i(M) \cap (\gamma_i(M)^{\dagger})^{z_i}$ for $i\in \{1, \ldots, k+l\}.$ 



Thus, if $g \in M \cap M^{y} \cap M^{xz}$, then $g$ has shape \eqref{gd1Sp} where
\begin{equation*}
\begin{split}
g_i \in \gamma_i(M) \cap \gamma_i(M)^{y_i} \cap (\gamma_i(M)^{\dagger})^{z_i} & \le Z(GL_{n_i}(q)) \text{ for } i \in \{1, \ldots, k\}; \\
g_i \in \gamma_i(M) \cap \gamma_i(M)^{y_i} \cap \gamma_i(M)^{z_i} & \le Z(Sp_{n_i}(q)) \text{ for } i \in \{k+1, \ldots, k+l\}. 
\end{split}
\end{equation*} In particular, $g$ is 
\begin{equation}\label{gd1Spscal}
\Scale[0.97]{
\diag [\tau(g)\alpha_1^{\dagger} I_{n_1}, \ldots, \tau(g)\alpha_{k}^{\dagger} I_{n_k}, \alpha_{k+1} I_{n_{k+1}}, \ldots, \alpha_{k+l} I_{n_{k+l}}, \alpha_{k} I_{n_k}, \ldots, \alpha_{1} I_{n_1}]},
\end{equation}
where $\alpha_i \in \mathbb{F}_{q}$ for $i \in \{1, \ldots, k+l\}$  and $\alpha_i^{\dagger}=\alpha_i^{-1}$ for $i \in \{1, \ldots, k\}$.
By Lemma \ref{scfield}, we can assume that elements in $\gamma_i(S) \cap \gamma_i(S)^{y_i} \cap (\gamma_i(S)^{\dagger})^{z_i}$ for $i \le k$ and in $\gamma_i(S) \cap \gamma_i(S)^{y_i} \cap (\gamma_i(S))^{z_i}$ for $i>k$ have shape $\phi^j g_i$ with $g_i \in Z(GL_{n_i}(q)).$ Thus, if $\varphi \in S \cap S^y \cap S^{xz},$ then $\varphi = \phi^j_{\beta} g$ with $g$ as in \eqref{gd1Spscal}.
  Denote $S \cap S^{y} \cap S^{xz}$ by $\tilde{S}.$


 
\subsection*{Step 2} We now find a fourth conjugate of $S$ such that its intersection with $\tilde{S}$ lies in  $Z(GSp_n(q)).$

Recall that $\beta$ is such that 
$\bf f_{\beta}$ is as in \eqref{fst}. Therefore,
 $$\beta= \beta_{(1,1)} \cup \ldots \cup \beta_{(1,k)} \cup \beta_{k+1} \cup \ldots \cup \beta_{k+l} \cup \beta_{(2,1)} \cup \ldots \cup \beta_{(2,k)},$$
where 
\begin{equation}\label{betaij}
\begin{split}
\beta_{(1,i)} & =\{f_1^{i}, \ldots, f_{n_i}^i\} \text{ for } i \in \{1, \ldots, k\}; \\
\beta_{(2,i)} & =\{e_1^{i}, \ldots, e_{n_i}^i\} \text{ for } i \in \{1, \ldots, k\}; \\
\beta_{i} & =\{f_1^{i}, \ldots, f_{n_i/2}^i,e_1^{i}, \ldots, e_{n_i/2}^{i}\} \text{ for } i \in \{k+1, \ldots, k+l\},
\end{split}
\end{equation}
and $(f_i^j,e_i^j)=1$ for all $i,j.$ All other pairs of vectors from $\beta$ are orthogonal. For simplicity we relabel vectors $f_i^j$ in $\beta$ in the order they appear in $\beta$ using just one index, so $f_i^j$ becomes 
\begin{equation*}
\begin{aligned}
& f_{(\sum_{t=0}^{j-1} n_t +i)} & \text{ if } j \le k+1; \\
& f_{(\sum_{t=0}^{k} n_t +\sum_{t=k+1}^{j-1} (n_t/2) +i)} & \text{ if } j > k+1. 
\end{aligned}
\end{equation*}
We relabel the $e_i^j$ such that $(f_i,e_i)=1.$

If $\varphi \in \tilde{S}$, so $\varphi = \phi^j g$ with $g$ as in \eqref{gd1Spscal}, then let $\delta_i \in \mathbb{F}_q$ be such that $(e_i)g=\delta_i e_i$ for $i \in\{1, \ldots, n/2\}$ (so $\delta_i$ is some $\alpha_j$ from \eqref{gd1Spscal}).
 Let $\theta$ be a generator of $\mathbb{F}_q^*.$ 


The remainder of the proof splits into two cases: 
when $k \ge 1$ and
 $k=0.$
  In each   we show that $b_S(S \cdot SU_n(q))\le 4.$ 



\subsection*{Case 1} Let $k\ge 1$. This step splits into two  subcases. In the first  $n_i=1$ for all $i \in \{1, \ldots, k\}$; in the second  there exists $i \in \{1, \ldots, k\}$ such that $n_i \ge 2.$

\medskip

{\bf Case (1.1).} Let $n_i=1$ for all $i \in \{1, \ldots, k\}$.
We redefine $y$ in \eqref{yizi} to be
$$\diag [A^{\dagger}, y_{k+1}, \ldots, y_{k+l}, A]$$
where $$
A=
\begin{pmatrix}
1 & 0   & 0   & \ldots  & 0 \\
0 & 1      & 0   & \ldots  & 0 \\
  &        & \ddots &   &  \\
0 & \ldots & 0      & 1 & 0\\
1 & \ldots & \ldots & 1 & 1\\
\end{pmatrix}.
$$ It is easy to see that $y \in Sp_n{(q, {\bf f}_{\beta})}.$ Let $x,z \in Sp_n{(q, {\bf f}_{\beta})}$
be as in {\bf Step 1}, so $\varphi \in \tilde{S}$ has shape $\phi^j g$ with $g$ as in \eqref{gd1Spscal}. Since $S$ stabilises $\langle e_1 \rangle,$ $S^y$ stabilises $\langle e_1 \rangle y= \langle e_1 + \ldots + e_k \rangle.$ Therefore, 
$$((e_1)y)\varphi = (e_1 + \ldots + e_k)\varphi^j g = \alpha_1 e_1 + \ldots + \alpha_k e_k= \lambda(e_1 + \ldots + e_k)$$
for some $\lambda \in \mathbb{F}_q^*,$ so $\alpha_1 = \ldots = \alpha_k.$


Assume $k \ge 2.$ 
Consider $a \in GL_n(q)$ such that
\begin{equation*}
\begin{aligned}
&(e_1)a =\sum_{i=3}^{n/2} e_i +e_1 + \theta e_2 + {f_1};& & (f_1)a=f_1; \\
&(e_{2})a =  e_2; & &(f_{2})a=f_{2} - \theta f_1; & &   \\
&(e_{i})a =e_{i};  & &(f_{i})a=f_{i} -  f_{1}; && i\in \{3, \ldots, n/2 \}.   \\
\end{aligned}
\end{equation*}
 It is routine to check that $a$ is an isometry of $(V, {\bf f}),$ so we can consider $a$ as an element of $Sp_n(q, {\bf f}_{\beta}).$

We claim that $\tilde{S} \cap S^a \le Z(GSp_n(q,{\bf f}_{\beta})).$ Let $\varphi =\phi^j g \in \tilde{S} \cap S^a$.  Since $S$ stabilises the subspace $\langle e_1 \rangle$,  $S^a$ stabilises $\langle e_1 \rangle a.$  Therefore, 
\begin{equation*}
((e_1)a)\varphi=
\begin{cases}
\sum_{i=m+1}^{n/2}\delta_i e_i +\alpha_1e_1 + \theta^{p^j} \alpha_1 e_2+ \tau(g)\alpha_1^{\dagger}f_1\\
\lambda (e_1)a
\end{cases}
\end{equation*}
for some $\lambda \in \mathbb{F}_q.$
Hence 
$$\alpha_1=\theta^{p^j-1}\alpha_1=\tau(g)\alpha_1^{\dagger}=\delta_{m+1}= \ldots= \delta_{n/2}.$$
Therefore, $j=0$,  $\alpha_1= \ldots = \alpha_{k+l}$ and $\alpha_i=\tau(g)\alpha_i^{\dagger}$ for all $i,$ so $g$ is scalar  and $\tilde{S} \cap S^a \le Z(GSp_n(q,{\bf f}_{\beta})).$


Assume $k=1,$ so $l\ge 1$ (otherwise $n=2$) and $\{f_2,e_2\} \subseteq \beta_{k+1}.$ In particular, if $\varphi = \phi^j g \in \tilde{S}$, then $(f_2)g=(e_2)g=\alpha_2.$ Consider $a \in GL_n(q)$ such that
\begin{equation*}
\begin{aligned}
&(e_1)a =\sum_{i=3}^{n/2} e_i +e_1 + \theta e_2 + {f_2};& & (f_1)a=f_1; \\
&(e_{2})a =  e_2+f_1; & &(f_{2})a=f_{2} - \theta f_1; & &   \\
&(e_{i})a =e_{i};  & &(f_{i})a=f_{i} -  f_{1}; && i\in \{3, \ldots, n/2 \}.   \\
\end{aligned}
\end{equation*}
 It is routine to check that $a$ is an isometry of $(V, {\bf f}),$ so we can consider $a$ as an element of $Sp_n(q, {\bf f}_{\beta}).$

We claim that $\tilde{S} \cap S^a \le Z(GSp_n(q,{\bf f}_{\beta})).$ Let $\varphi =\phi^j g \in \tilde{S} \cap S^a$.  Since $S$ stabilises the subspace $\langle e_1 \rangle$,  $S^a$ stabilises $\langle e_1 \rangle a.$  Therefore, 
\begin{equation*}
((e_1)a)\varphi=
\begin{cases}
\sum_{i=m+1}^{n/2}\delta_i e_i +\alpha_1e_1 + \theta^{p^j} \alpha_2 e_2+ \alpha_2 f_1;\\
\lambda (e_1)a
\end{cases}
\end{equation*}
for some $\lambda \in \mathbb{F}_q.$
Hence 
$$\alpha_1=\alpha_2=\theta^{p^j-1}\alpha_2=\delta_{m+1}= \ldots= \delta_{n/2}.$$
Therefore, $j=0$,  $\alpha_1= \ldots = \alpha_{k+l}$ and $\alpha_i=\tau(g)\alpha_i^{\dagger}$ for all $i,$ so $g$ is scalar  and $\tilde{S} \cap S^a \le Z(GSp_n(q,{\bf f}_{\beta})).$

\medskip

{\bf Case (1.2).} Denote $r:= \sum_{i=1}^k n_i.$ Let $n_i \ge 2$ for some $i \le k,$ so $\delta_s= \delta_{s+1}$ for some $s<r.$ Consider $a \in GL_n(q)$ such that
\begin{equation*}
\begin{aligned}
&(e_s)a =f_s + \theta f_{s+1} + \sum_{i \notin \{s,s+1\}}^{n/2} f_i +e_s;& & (f_s)a=f_s; \\
&(e_{s+1})a =  e_{s+1}+\theta f_s; & &(f_{s+1})a=f_{s+1}; & &   \\
&(e_{i})a =e_{i} +f_s;  & &(f_{i})a=f_{i}; && \Scale[0.95]{i \in \{3, \ldots, n/2 \} \backslash \{s, s+1\}}.   \\
\end{aligned}
\end{equation*}
 It is routine to check that $a$ is an isometry of $(V, {\bf f}),$ so we can consider $a$ as an element of $Sp_n(q, {\bf f}_{\beta}).$

We claim that $\tilde{S} \cap S^a \le Z(GSp_n(q,{\bf f}_{\beta})).$ Let $\varphi =\phi^j g \in \tilde{S} \cap S^a$.  Since $S$ stabilises the subspace $W=\langle e_1 + \ldots +e_r \rangle$,  $S^a$ stabilises $W a.$  Therefore, 
\begin{equation*}
((e_s)a)\varphi=
\begin{cases}
\tau(g) \delta_s^{\dagger} f_s + \theta^{p^j}\tau(g) \delta_s^{\dagger} f_{s+1} + \sum_{i \notin \{s,s+1\}}^{n/2} \tau(g) \delta_i^{\dagger} f_i +\delta_s e_s;\\
\eta_1 (e_1)a+ \ldots +\eta_r(e_r)a
\end{cases}
\end{equation*}
for some $\eta_1, \ldots, \eta_r \in \mathbb{F}_q.$
Since $((e_1)a)\varphi$ does not have $e_i$ for $i \ne s$ in the first line of the equation above, $((e_s )a)\varphi = \eta_s (e_s )a$, so
$$\alpha_s=\tau(g)\alpha_s^{\dagger}=\theta^{p^j-1}\tau(g) \delta_s^{\dagger}=\tau(g)\delta_{i}^{\dagger}$$
for $i \in \{3, \ldots, n/2 \} \backslash \{s, s+1\}.$
Therefore, $j=0$,  $\alpha_1= \ldots = \alpha_{k+l}$ and $\alpha_i=\tau(g)\alpha_i^{\dagger}$ for all $i,$ so $g$ is scalar  and $\tilde{S} \cap S^a \le Z(GSp_n(q,{\bf f}_{\beta})).$




\subsection*{Case 2.} Let $k=0,$ so $l \ge 2.$ Denote $s:=n_1/2.$ Hence $\{f_{s+1},e_{s+1}\} \subseteq \beta_{2}.$ In particular, if $\varphi = \phi^j g \in \tilde{S}$, then $(f_{s+1})g=(e_{s+1})g=\alpha_2.$
Consider $a \in GL_n(q)$ such that
\begin{align*}
(e_1)a & =\sum_{i=s+1}^{n/2} e_i +e_1 + \theta f_{s +1} ;&  (f_1)a & =f_1; \\
(e_{s+1})a & =  e_{s+1}+\theta f_1; & (f_{s +1})a & =f_{s +1} -  f_1; 
\end{align*}
\begin{align*}
&(e_{i})a =e_{i};  & &(f_{i})a=f_{i}; && i\in \{2, \ldots, n_1/2 \}.   \\
&(e_{i})a =e_{i};  & &(f_{i})a=f_{i} -  f_{1}; && i\in \{s+2, \ldots, n/2 \}.   
\end{align*}
 It is routine to check that $a$ is an isometry of $(V, {\bf f}),$ so we can consider $a$ as an element of $Sp_n(q, {\bf f}_{\beta}).$

We claim that $\tilde{S} \cap S^a \le Z(GSp_n(q,{\bf f}_{\beta})).$ Let $\varphi =\phi^j g \in \tilde{S} \cap S^a$.  Since $S$ stabilises the subspace $W=\langle e_1 + \ldots + e_{s}, f_1 + \ldots + f_{s}  \rangle$,  $S^a$ stabilises $W a.$  Therefore, 
\begin{equation*}
((e_1)a)\varphi=
\begin{cases}
 \sum_{i=s +1}^{n/2}  \delta_i e_i + \alpha_1 e_1 + \theta^{p^j} \delta_{s+1} f_{s+1};\\
\eta_1 (e_1)a+ \ldots +\eta_{s}(e_{s})a + \mu_1 (f_1)a+ \ldots +\mu_{s}(f_{s})a
\end{cases}
\end{equation*}
for some $\eta_1, \ldots, \eta_{s}, \mu_1, \ldots, \mu_{s}  \in \mathbb{F}_q.$
Since $((e_1)a)\varphi$ does not have terms with $e_i$ for $1<i  \le {s}$ and $f_i $ for $1 \le i \le {s}$ in the first line of the equation above, $((e_1 )a)\varphi = \eta_1 (e_1 )a$, so
$$\alpha_1=\alpha_2=\theta^{p^j-1}\alpha_2=\delta_{i}$$
for $i \in \{s+1, \ldots, n/2 \}.$
Therefore, $j=0$,  $\alpha_1= \ldots = \alpha_{k+l}$ and $\alpha_i=\tau(g)\alpha_i^{\dagger}$ for all $i,$ so $g$ is scalar  and $\tilde{S} \cap S^a \le Z(GSp_n(q,{\bf f}_{\beta})).$
\end{proof}

We have now proved Theorem \ref{theoremSp} for $q > 3.$



\begin{Rem} \label{symprem23} Equation \eqref{smintSp} does not always hold for 
$q \in \{2,3\}$. In particular, it does not hold in each of the following cases: 
\begin{enumerate}[label=(\alph*)]
\item $\gamma_i(S)=GL_2(q)$ for $i \in \{1, \ldots, k\};$ \label{q23gl2}
\item $\gamma_i(S)=GSp_2(q)$ for $i \in \{k+1, \ldots, k+l\};$ \label{q23gsp2}
\item $\gamma_i(S)$ is the stabiliser in $GSp_4(q)$ of the decomposition $V=V_1 \bot V_2$ with $V_1$ and $V_2$ non-degenerate of dimension $2$  for $i \in \{k+1, \ldots, k+l\}.$ Recall that $\beta_i=\{f_1^i,f_2^i,e_1^i,e_2^i\}$ and let $V_r= \langle f_r^i, e_r^i \rangle$ for $r=1,2.$ \label{q23sp2wr2}
\end{enumerate}
\end{Rem}

The following two lemmas  are verified by computation. Let $Q$, $R$ and $T$ be $\gamma_i(S)$   from (a), (b) and (c) of Remark \ref{symprem23} respectively.

\begin{Lem}\label{smSp23}
Let $q \in \{2,3\},$ $S \le Sp_n(q)$ is a maximal solvable subgroup and $\beta$ is a basis of $V$ as in Lemma $\ref{unist}$, so matrices in $S$ have shape \eqref{gst}. Specifically, let one of the following hold:
\begin{itemize}
\item $k=0$, $l=2$, $\gamma_i(S)$ is $R$   for both $i=1,2$, so $n=4$;
\item $k=0$, $l=2$, $\gamma_i(S)$ is $T$  for both $i=1,2$, so $n=8$;  
\item $k=0$, $l=2$,  $\gamma_1(S)$ is $R$, $\gamma_2(S)$ is $T$, so $n=6$;
\item $k=1$, $l=1$,  $\gamma_1(S)$ is $Q$, $\gamma_2(S)$ is $R$, so $n=6$;
\item $k=1$, $l=1$,  $\gamma_1(S)$ is $Q$, $\gamma_2(S)$ is $T$, so $n=8$;
\item $k=2$, $l=0$,  $\gamma_i(S)$ is $Q$, for both $i=1,2$, so $n=8$.
\end{itemize} 
Then there exist $x, y \in Sp_n(q)$ such that 
$$S \cap S^x \cap S^y \le Z(Sp_n(q)).$$
\end{Lem}

\begin{Lem}\label{Spq2c23}
Let  $S \le Sp_4(q)$ be a maximal solvable subgroup and let $\beta$ be a basis of $V$ as in Lemma $\ref{unist}$, so matrices in $S$ have shape \eqref{gst}. 
\begin{enumerate}[font=\normalfont]
\item Let $k=0,$ $l=1$ and let $S=T$. If $q=3$, then there exist $x,y \in Sp_4(3)$ such that 
$$S \cap S^x \cap S^y = \left\langle 2I_4, \left( \begin{smallmatrix}
1&0&1&0\\
0&1&0&0\\
0&0&1&0\\
0&0&0&1
\end{smallmatrix} \right)
 \right\rangle.$$ For instance,  $$x=\left( \begin{smallmatrix}
0&1&2&1\\
2&2&1&0\\
0&0&2&1\\
0&0&2&0
\end{smallmatrix} \right), \;  y=\left( \begin{smallmatrix}
0&2&0&2\\
0&0&2&0\\
0&0&1&2\\
1&1&1&1
\end{smallmatrix} \right).$$
\item Let $k=1$, $l=0$ and $\gamma_1(S)=GL_2(q).$ If $q=3,$ then there exist $x,y \in Sp_4(3)$ such that 
$$S \cap S^x \cap S^y = \left\langle \left( \begin{smallmatrix}
1&0&0&0\\
1&2&0&0\\
0&0&1&1\\
0&0&0&2
\end{smallmatrix} \right), \left( \begin{smallmatrix}
2&2&0&0\\
2&1&0&0\\
0&0&2&2\\
0&0&2&1
\end{smallmatrix} \right)
 \right\rangle.$$ For instance,  $$x=\left( \begin{smallmatrix}
0&0&1&0\\
0&0&0&1\\
1&0&0&0\\
0&1&0&0
\end{smallmatrix} \right), \;  y=\left( \begin{smallmatrix}
2&1&0&0\\
1&0&2&2\\
2&0&1&2\\
1&2&1&1
\end{smallmatrix} \right).$$
\item If $q \in \{2,3\}$, then, in both  $(1)$ and $(2)$, there exist $x,y,z \in Sp_n(q)$ such that 
$$S \cap S^x \cap S^y \cap S^z =Z(Sp_n(q)).$$ 
\end{enumerate}
\end{Lem}


\begin{Th}
\label{Spq23}
Theorem {\rm\ref{theoremSp}} holds for
 $q \in \{2,3\}$ if $S$ stabilises a non-zero proper subspace of $V$.
\end{Th}
\begin{proof}
Notice that $\GS_n(q)=GSp_n(q).$ As in the case $q>3$, in {\bf Step 1}  we  obtain three conjugates of $S$ in $S \cdot Sp_n(q)$ such that their intersection consists of diagonal matrices and matrices which have few non-zero entries not on the diagonal. In {\bf Step 2} we find a fourth conjugate of $S$ such that the intersection of the four is a group of scalars.


\medskip

{\bf Step 1.} 
We commence with a technical definition.
\begin{Def}
Let $\beta=\{v_1, \ldots, v_n\}$ be a basis of a vector space $V$ over a field $\mathbb{F}$. Let $g \in GL(V)$, so $g_{\beta} \in GL_n(\mathbb{F}).$ We label the rows and columns of $g_{\beta}$ by corresponding basis vectors, so the $i$-th row (column) is labelled by $v_i.$ If $\widehat{\beta}$ is a subset of $\beta$, then the {\bf restriction} $\widehat{g_{\beta}}$  of $g_{\beta}$ to  $\widehat{\beta}$ is the matrix in $GL_{|\widehat{\beta}|}(\mathbb{F})$  obtained from $g_{\beta}$ by taking only the entries lying on the intersections of rows and columns labelled by vectors in $\widehat{\beta}.$ If $h\in GL_{|\widehat{\beta}|}(\mathbb{F}),$ then the $(h, \widehat{\beta})$-{\bf replacement} of $g_{\beta}$ is the matrix obtained from $g_{\beta}$ by replacing the entries lying on the intersections of rows and columns labelled by vectors in $\hat{\beta}$ by corresponding entries of $h.$

For example, if $n=4,$ $\beta=\{v_1,v_2,v_3,v_4\},$ $\widehat{\beta}=\{v_2,v_4\}$, 
\begin{equation*}
g=
\left( \begin{array}{c>{\columncolor{gray!20}}cc>{\columncolor{gray!20}}c}
g_{11} & g_{12} & g_{13} & g_{14}\\
\rowcolor{gray!20}
g_{21} &{\cellcolor{gray!50}} g_{22} & g_{23} &{\cellcolor{gray!50}} g_{24}\\
g_{31} & g_{32}& g_{33} & g_{34}\\
\rowcolor{gray!20}
g_{41} &{\cellcolor{gray!50}} g_{42} & g_{43} &{\cellcolor{gray!50}} g_{44}
\end{array} \right)
\text{ and } 
h=
\begin{pmatrix}
h_{11} & h_{12}\\
h_{21} & h_{22}
\end{pmatrix},
\end{equation*}
then the restriction of $g$ to $\widehat{\beta}$ and the $(h, \widehat{\beta})$-{replacement} of $g$ are
$$
\begin{pmatrix}
g_{22} & g_{24}\\
g_{42} & g_{44}
\end{pmatrix},
\text{ and } 
\left( \begin{array}{c>{\columncolor{gray!20}}cc>{\columncolor{gray!20}}c}
g_{11} & g_{12} & g_{13} & g_{14}\\
\rowcolor{gray!20}
g_{21} &{\cellcolor{gray!50}} h_{11} & g_{23} &{\cellcolor{gray!50}} h_{12}\\
g_{31} & g_{32}& g_{33} & g_{34}\\
\rowcolor{gray!20}
g_{41} &{\cellcolor{gray!50}} h_{21} & g_{43} &{\cellcolor{gray!50}} h_{22}
\end{array} \right) 
$$ respectively.
\end{Def}

We claim that we can assume that there is at most one $i \in \{1, \ldots, k+l\}$ such that $\gamma_i(S)$ is one of the groups in Remark \ref{symprem23}. Assume  $r<s$ are the only elements of $\{1, \ldots, k+l\}$ such that  $\gamma_r(S)$ and $\gamma_s(S)$ are as in Remark \ref{symprem23}. Recall that $\beta$ is as in \eqref{betaij}. If $i \in \{1, \ldots, k\}$, then denote $\beta_i= \beta_{(1,i)} \cup \beta_{(2,i)}.$ Let $\hat{\beta}$ be $\beta_r \cup \beta_s$ if $r>k$ and $\beta_{(1,r)} \cup \beta_s \cup \beta_{(2,r)}$ if $r<k.$ Consider $g \in S$ and notice that its restriction $\widehat{g}$ to $\widehat{\beta}$ lies in $GSp_{|\widehat{\beta}|}(q, {\bf f}_{\widehat{\beta}}),$ where ${\bf f}_{\widehat{\beta}}$ is the restriction of ${\bf f}_{\beta}$ to $\langle \widehat{\beta} \rangle$ with respect to the basis $\widehat{\beta}.$ If $\widehat{S}$ is the group consisting of restrictions to $\widehat{\beta}$ for all $g \in S$, then, as is easy to see, $\widehat{S}$ is one of the groups in Lemma \ref{smSp23}. For example, if $g$ is as in \eqref{gst}, $r=1$ and $s=k+l$, then $\widehat{g}$ is constructed by the dark gray blocks of the following matrix
$$
\left( \begin{array}{>{\columncolor{gray!20}}ccccc>{\columncolor{gray!20}}ccc>{\columncolor{gray!20}}c}
\rowcolor{gray!20}
{\cellcolor{gray!50}} {\tau(g)\gamma_{1}(g)^{\dagger}}&* &{ *  } & * & \ldots &{\cellcolor{gray!50}} *&* &*    &{\cellcolor{gray!50}} *  \\
        & \ddots &{} & & &\ddots & &    &\\
0        & & {\tau(g)\gamma_{k}(g)^{\dagger}}&* &* &* &\ldots &*    &* \\  
        & &  &\gamma_{k+1}(g) & & {0}  &* & \ldots   &*\\ 
        & &  & &\ddots &       & & \ddots   &\\
\rowcolor{gray!20}
{\cellcolor{gray!50}}   & &  &0 & &{\cellcolor{gray!50}} {\gamma_{k+l}(g)}    & * & \ldots   &{\cellcolor{gray!50}} *\\  
        & & & & & & \gamma_{k}(g) &  *  &* \\
        & & & & & & & \ddots   &*\\
\rowcolor{gray!20}
{\cellcolor{gray!50}} 0        & & & & &{\cellcolor{gray!50}} &0 &    &{\cellcolor{gray!50}} \gamma_{1}(g)\\ 
\end{array} \right).  
$$
Let $h \in Sp_{|\widehat{\beta}|}(q, {\bf f}_{\widehat{\beta}})$ and let $t$ be the $(h, \widehat{\beta})$-{replacement} of $I_n.$ It is routine to check that $t \in Sp_n(q, {\bf f}_{\beta})$ and the restriction of  $g^t$ to $\widehat{\beta}$ is $\widehat{g}^{{h}}.$  Let $x$ be the matrix \eqref{exSP}. Notice that if $\widehat{S}$, $\widehat{S^x}$ and $\widehat{x}$ are the restrictions of ${S}$, $S^x$ and $x$ to $\widehat{\beta}$ respectively, then $\widehat{S^x}=\widehat{S}^{\widehat{x}}.$ Therefore, by Lemma \ref{smSp23}, there exist $\hat{y}, \hat{z} \in Sp_{|\widehat{\beta}|}(q, {\bf f}_{\widehat{\beta}})$ such that 
$$\widehat{S} \cap \widehat{S}^{\widehat{x} \widehat{y}} \cap \widehat{S}^{\widehat{z}}\le Z(GSp_{|\widehat{\beta}|}(q, {\bf f}_{\widehat{\beta}})).$$
 For $i \ne r,j$ define $y_i$ and $z_i$ as in \eqref{smintSp}. Let $y$ and $z$ be the $(\widehat{y}, \widehat{\beta})$-{replacement} and $(\widehat{z}, \widehat{\beta})$-{replacement} of  matrices from \eqref{yizi} respectively. It is routine to check that $y,z \in Sp_n(q,{\bf f}_{\beta}).$ Observe now that 
$\tilde{S}= S \cap S^{xy} \cap S^{z}$ is a group of diagonal matrices.

If there is more than one such pair $(r,s),$ then the same corrections of $y$ and $z$ for each pair can be done. Therefore, we can assume that there is at most one $s \in \{1, \ldots, k+l\}$ such that $\gamma_s(S)$ is one of the groups in Remark \ref{symprem23}. If there is no such $s$, then {\bf Step 2} of the proof for $q>3$ implies the theorem, so assume that such $s$ exists.

\medskip

{\bf Step 2.} Let $s \in \{1, \ldots, k+l\}$ be such that $\gamma_s(S)$ is $Q$, $R$ or $T$ as defined after Remark \ref{symprem23}. 

\medskip Since the only diagonal matrix in $Sp_n(2)$ is $I_n$, it is enough to obtain four conjugates of $S$ in $Sp_n(2)$ such that their intersection is a group of diagonal matrices. Therefore, if $\gamma_s(S) \in \{GL_2(q), Sp_2(q) \wr \Sym(2)\},$ then by Lemma  \ref{Spq2c23} and Theorem \ref{sympirr}
 using the construction in {\bf Step 1} we obtain $x,y,z \in Sp_n(q)$ such that 
$$S \cap S^x \cap S^y \cap S^z =\{1\}.$$
Hence for $q=2$ we only need to consider the situation $\gamma_s(S)=R$. 

 We consider three distinct cases -- when  $\gamma_s(S)$ is $R$, $T$ and $Q$ respectively.

\medskip

{\bf Case 1.}
First assume $\gamma_s(S)$ is  $R$, so $s>k.$ Without loss of generality, we can assume $s=k+1.$  Let $\beta$ be as in {\bf Step 2} of the proof for $q>3$. Let $r$ be such that restriction of matrices from $S$ to vectors $\{f_r, e_r\}$ is $\gamma_s(S).$ We relabel vectors in $\beta$ as follows:
\begin{itemize}
\item $f_r$ and $e_r$ become $f$ and $e$ respectively;
\item if $i<r$, then $f_i$ and $e_i$ remain $f_i$ and $e_i$ respectively;
\item if $i>r$, then  $f_i$ and $e_i$ become $f_{i-1}$ and $e_{i-1}$ respectively.
\end{itemize}
Therefore, since $\alpha^{\dagger}=\alpha$ for $\alpha \in \mathbb{F}_q$ with $q \in \{2,3\},$ a matrix $g \in \tilde{S}$ has shape 
$$\diag[\tau(g)\alpha_1 I_1, \ldots, \tau(g) \alpha_k I_{n_{k}}, \Lambda, \alpha_{k+2} I_{n_{k+2}}, \ldots, \alpha_{k+l} I_{n_{k+l}}, \alpha_k I_{n_k}, \ldots, \alpha_1 I_{1} ],$$
where 
\begin{equation*}
\Lambda =
\begin{pmatrix}
 \lambda_1 & \lambda_2  \\
 \lambda_3 & \lambda_4  \\
\end{pmatrix}.
\end{equation*}

Let $k>0$ and $n_1 \ge 2.$ Now $W=\langle e_1, \ldots, e_{n_1} \rangle$ is an $S$-invariant subspace. Let $m=n_1$ and  let $a \in GL_n(q)$ be such that
\begingroup
\allowdisplaybreaks
\begin{align*}
%\begin{aligned}
&(e_1)a =\sum_{i=m+1}^{(n-2)/2} e_i +e_1 + e; \\ & (f_1)a=f_1; \\
&(e_2)a =\sum_{i=m+1}^{(n-2)/2} e_i +e_2 + f -f_1; \\ & (f_2)a=f_2; \\
&(e_{j})a = \sum_{i=m+1}^{(n-2)/2}e_i + e_j; & &  j\in \{3, \ldots, m \} \stepcounter{equation}\tag{\theequation} \label{aSp2n12} \\ &(f_{j})a=f_{j}; & &  j\in \{3, \ldots, m \}  \\
&(e_{j})a =e_{j};  && j\in \{m+1, \ldots, (n-2)/2 \}   \\ &(f_{j})a=f_{j} - \sum_{i=1}^m f_{i}; && j\in \{m+1, \ldots, (n-2)/2 \}   \\
& (e)a=e+f_2; \\ & (f)a=f-f_1.
%\end{aligned}
\end{align*}
\endgroup
It is routine to check that $a \in Sp_n(q, {\bf f}_{\beta})$. Consider $g \in \tilde{S} \cap S^a.$  Since $S$ stabilises the subspace $W$,  $S^a$ stabilises $(W)a.$ As in the proof for $q>3,$ let $\delta_i \in \mathbb{F}_q$ be such that $(e_i)g=\delta_i e_i$ for $i \in\{1, \ldots, (n-1)/2\}.$  Therefore, 
\begin{equation}\label{Wisotrq23n12}
((e_1)a)g=
\begin{cases}
\sum_{i=m+1}^{(n-2)/2}\delta_i e_i +\alpha_1e_1 + \lambda_3 f +\lambda_4 e;\\
\eta_1 (e_1)a+ \ldots +\eta_m(e_m)a
\end{cases}
\end{equation}
for some $\eta_1, \ldots, \eta_m \in \mathbb{F}_q.$ Observe that $((e_1)a)g$ does not have $e_i$ for $1<i  \le m$  in the first line of \eqref{Wisotrq23n12}, so $((e_1 )a)g = \eta_1 (e_1 )a$; thus $\lambda_2=0$ and
$$\lambda_4=\alpha_1=\delta_{m+1}= \ldots= \delta_{(n-2)/2}.$$
The same argument for $((e_2)a)g$ shows that $\lambda_2=0$ and $\tau(g)\alpha_1=\lambda_1=\alpha_1.$ 
Therefore, $g=\alpha_1 I_n$, so $g \in Z(GSp_n(q, {\bf f}_{\beta})).$


Let $k=1$ and $n_1 = 1.$ So $\langle e_1\rangle$ is an $S$-invariant subspace of $V$. If $n=4,$ then  Theorem \ref{theoremSp} is verified by computation. So we can assume that $n>4.$ 
Thus, $l \ge 2$ and 
$$W= \langle f_2 \ldots, f_{(n-2)/2}, e_2, \ldots, e_{(n-2)/2}, e_1 \rangle $$
is $S$-invariant.
Let $a \in GL_n(q)$ be such that
\begin{equation}\label{aSp2n11}
\begin{aligned}
&(e_1)a =\sum_{i=2}^{(n-2)/2} e_i +e_1 + e; & & (f_1)a=f_1; \\
&(e_2)a = e_2 + f -f_1; & & (f_2)a=f_2 -f_1; \\
&(e_{j})a =e_{j};  & &(f_{j})a=f_{j} - f_1; && j\in \{2, \ldots, (n-2)/2 \}   \\
& (e)a=e+f_2; & & (f)a=f-f_1.
\end{aligned}
\end{equation}
It is routine to check that $a \in Sp_n(q, {\bf f}_{\beta})$. Consider $g \in \tilde{S} \cap S^a.$  Since $S$ stabilises the  subspaces $\langle e_1 \rangle$ and $W$,  $S^a$ stabilises $\langle (e_1)a \rangle$ and $(W)a.$ Therefore, 
\begin{equation}\label{Wisotrq23n11}
((e_1)a)g=
\sum_{i=2}^{(n-2)/2}\delta_i e_i +\alpha_1e_1 + \lambda_3 f +\lambda_4 e= \eta (e_1)a
\end{equation}
for some $\eta \in \mathbb{F}_q.$ Hence $\delta_2= \ldots = \delta_{(n-2)/2}=\lambda_4=\alpha_1$ and $\lambda_3=0.$
In the same way 
\begin{equation}\label{Wisotrq23n11f}
((e_2)a)g=
\begin{cases}
\delta_2 e_2 + \lambda_1 f +\lambda_2 e - \tau(g)\alpha_1 f_1;\\
\eta_1 (e_1)a+ \sum_{i=2}^{(n-2)/2} \eta_i (e_i)a+ \sum_{i=2}^{(n-2)/2} \xi_i (f_i)a
\end{cases}
\end{equation}
for some $\eta_i, \xi_i \in \mathbb{F}_q.$ Since $((e_2)a)g$ does not have $e_i$ for $i>2$ and $f_j$ for $j>1$ in the first line of \eqref{Wisotrq23n11f}, $((e_2)a)g= \eta_2 (e_2)a$. Therefore, 
$\tau(g)\alpha_1=\lambda_1=\alpha_1$ and $\lambda_2=0,$ so $g= \alpha_1 I_n \in Z(Sp_n(q, {\bf f}_{\beta})).$

Let $k \ge 2$ and $n_1 = 1.$ So $\langle e_1\rangle$ and $W=\langle e_1, e_2, \ldots, e_{(n_2/2+1)} \rangle $ are $S$-invariant subspaces of $V$. Let $a$ be as in \eqref{aSp2n11}. Consider $g \in \tilde{S} \cap S^a.$  Since $S$ stabilises the subspaces $\langle e_1 \rangle$ and $W$,  $S^a$ stabilises $\langle (e_1)a \rangle$ and $(W)a.$ Therefore, \eqref{Wisotrq23n11} holds, so $\delta_2= \ldots = \delta_{(n-2)/2}=\lambda_4=\alpha_1$ and $\lambda_3=0.$
In the same way 
\begin{equation}\label{Wisotrq23n11fk2}
((e_2)a)g=
\begin{cases}
\delta_2 e_2 + \lambda_1 f +\lambda_2 e - \alpha_1 f_1;\\
\eta_1 (e_1)a+ \sum_{i=2}^{(n_2/2+1)} \eta_i (e_i)a
\end{cases}
\end{equation}
for some $\eta_i \in \mathbb{F}_q.$  Since $((e_2)a)g$ does not have $e_i$ for $i>2$  in the first line of \eqref{Wisotrq23n11fk2}, $((e_2)a)g= \eta_2 (e_2)a$. Therefore, 
$\lambda_1=\alpha_1$ and $\lambda_2=0,$ so $g= \alpha_1 I_n \in Z(GSp_n(q, {\bf f}_{\beta})).$

Let $k=0,$ so $g \in S$ is a block-diagonal matrix with blocks in $\gamma_i(S) \le GSp_{n_i}(q).$ Let $W= \langle f_1, \ldots, f_{n_2/2}, e_1, \ldots, e_{n_2/2} \rangle$. If $n=4,$ then  Theorem \ref{theoremSp} follows by Lemma \ref{smSp23}, so let $n \ge 6.$ We can assume that $n_2 \ge 4.$ Indeed, if  $n_i=2$ for $i \in \{1, \ldots, l\},$ then we can consider $S_1=\diag [\gamma_2(S), \gamma_3(S)] \le \diag[GSp_2(q), GSp_2(q)]$ as a subgroup in $Sp_4(q)$. By Lemma \ref{smSp23}, $b_{S_1}(Sp_4(q))\le 3.$ We redefine $\gamma_2(g)$ to be $\diag[GSp_2(q), GSp_2(q)]$, so now $n_2=4.$   Let $m=n_2/2$ and let $a \in Sp_n(q, {\bf f}_{\beta})$ be defined by \eqref{aSp2n12}.  Arguments similar to the case $(k>0, n_1 \ge 2)$ imply $g \in Z(GSp_n(q, {\bf f}_{\beta})).$



\medskip

{\bf Case 2.} Let $q=3$ and $\gamma_s(S)$ is $T$ as defined  after Remark \ref{symprem23}.
Without loss of generality, we can assume $s=k+1.$  Let $\beta$ be as in {\bf Step 2} of the proof for $q>3$. Let $r$ be such that the restriction of matrices from $S$ to vectors $\{f_r, f_{r+1}, e_r, e_{r+1}\}$ is $\gamma_s(S).$ We relabel vectors in $\beta$ as follows:
\begin{itemize}
\item $f_r$, $f_{r+1}$, $e_r$ and  $e_{r+1}$ become $f$, $f_0$, $e$ and $e_0$ respectively;
\item if $i<r$, then $f_i$ and $e_i$ remain $f_i$ and $e_i$ respectively;
\item if $i>r$, then  $f_i$ and $e_i$ become $f_{i-2}$ and $e_{i-2}$ respectively.
\end{itemize}
Let  ${y}_s, {z}_s \in Sp_4(q)$ be such that $\gamma_s(S) \cap \gamma_s(S)^{{y}_s} \cap \gamma_s(S)^{{z}_s}$ is as in $(1)$ of Lemma \ref{Spq2c23}. For $i \ne s$ define $y_i$ and $z_i$ as in \eqref{smintSp}. Define $y$ and $x$ as in \eqref{yizi} and $\tilde{S}$ as in {\bf Step 1} of the proof for $q>3.$


 Therefore, $g \in \tilde{S}$ has shape 
$$\diag[\tau(g)\alpha_1 I_1, \ldots, \tau(g)\alpha_k I_{n_{k}}, \Lambda, \alpha_{k+2} I_{n_{k+2}}, \ldots, \alpha_{k+l} I_{n_{k+l}}, \alpha_k I_{n_k}, \ldots, \alpha_1 I_{1} ],$$
where $$
\Lambda = 
\left( \begin{smallmatrix}
\lambda_1&0&\lambda_2&0\\
0&\lambda_1&0&0\\
0&0&\lambda_1&0\\
0&0&0&\lambda_1
\end{smallmatrix} \right) \in
 \left\langle 2I_4, \left( \begin{smallmatrix}
1&0&1&0\\
0&1&0&0\\
0&0&1&0\\
0&0&0&1
\end{smallmatrix} \right)
 \right\rangle$$ and $\lambda_i \in \mathbb{F}_3$ for $i \in \{1,2\}.$
Let $W= \langle e_1, \ldots, e_{n_1} \rangle$ if $k>0$ and $$W= \langle f_1, \ldots, f_{n_{1}/2},e_1, \ldots, e_{n_1/2} \rangle$$ if $k=0.$ Let $m=n_1$ for $W$ totally singular and $m=n_1/2$ for $W$ non-degenerate. Let $a \in GL_n(q)$ be such that 
\begingroup
\allowdisplaybreaks
\begin{align*}
%\begin{aligned}
&(e_1)a =\sum_{i=m+1}^{(n-4)/2} e_i +e_1 + f +\underline{f_1}; \\ & (f_1)a=f_1; \\
&(e_{j})a =\sum_{i=m+1}^{(n-4)/2} e_i +e_j; && j\in \{2, \ldots, m \} \\ &(f_{j})a=f_{j};  && j\in \{2, \ldots, m \}   \\
&(e_{j})a =e_j;  && j\in \{m+1, \ldots, (n-4)/2 \} \stepcounter{equation}\tag{\theequation}  \label{aSp4qq3} \\ &(f_{j})a=f_{j} -\sum_{i=1}^{m} f_i;  && j\in \{m+1, \ldots, (n-4)/2 \}   \\
& (e)a=e+f_1; \\ & (f)a=f; \\
& (e_0)a=e_0; \\ & (f_0)a=f_0.
%\end{aligned}
\end{align*}
\endgroup
Here the underlined part is present only in the case $W$ is totally isotropic. It is routine to check that $a \in Sp_n(q, {\bf f}_{\beta}).$   Since $S$ stabilises the subspace $W$,  $S^a$ stabilises $(W)a.$ %First assume that $W$ is totally isotropic, so $W = \langle e_1, \ldots, e_m \rangle$.
 Therefore, 
\begin{equation}\label{Wisotrq3n14}
((e_1)a)g=
\begin{cases}
\sum_{i=m+1}^{(n-4)/2}\delta_i e_i +\alpha_1e_1 + \lambda_1 f + \lambda_2 e + \underline{\tau(g) \alpha_1 f_1};\\
\sum_{i=1}^{m} \eta_i (e_i)a+ \sum_{i=1}^{m} \xi_i (f_i)a
\end{cases}
\end{equation}
for some $\eta_i, \xi_i \in \mathbb{F}_q.$ Here all $\xi_i=0$ if $W$ is totally isotropic.
Since $((e_1)a)g$ does not have $e_i$ for $1<i  \le m$ and $f_i$ for $1 \le i \le m$ (for $W$ non-degenerate) in the first line of \eqref{Wisotrq3n14}, $((e_1 )a)g = \eta_1 (e_1 )a$, so $\lambda_2=0$ and
$$\alpha_1=\tau(g)\alpha_1=\lambda_1=\delta_{m+1}= \ldots= \delta_{(n-4)/2}.$$
Therefore, $g=\alpha_1I_n$ and  $g \in Z(GSp_n(q, {\bf f}_{\beta})).$

\medskip

{\bf Case 3.} Let $q=3$ and $\gamma_j(S)=GL_2(q),$ so $s \le k.$
If $k+l=1$, then Theorem \ref{theoremSp} follows by $(3)$ of Lemma \ref{Spq2c23}. Let $\beta$ be as in {\bf Step 2} of the proof for $q>3$. Let $r$ be such that the restriction of matrices of $S$ to vectors $\{ e_r, e_{r+1}\}$ is $\gamma_s(S).$ We relabel vectors in $\beta$ as follows:
\begin{itemize}
\item $f_r$, $f_{r+1}$, $e_r$ and  $e_{r+1}$ become $f$, $f_0$, $e$ and $e_0$ respectively;
\item if $i<r$, then $f_i$ and $e_i$ remain $f_i$ and $e_i$ respectively;
\item if $i>r$, then  $f_i$ and $e_i$ become $f_{i-2}$ and $e_{i-2}$ respectively.
\end{itemize}
Let  ${y}_s, {z}_s \in Sp_4(q)$ be such that $\gamma_s(S) \cap \gamma_s(S)^{{y}_s} \cap \gamma_j(S)^{{z}_s}$ is as in $(2)$ of Lemma \ref{Spq2c23}. For $i \ne s$ define $y_i$ and $z_i$ as in \eqref{smintSp}. Define $y$ and $x$ as in \eqref{yizi} and $\tilde{S}$ as in {\bf Step 1} of the proof for $q>3.$
Therefore, $g \in \tilde{S}$ has shape 
\begin{multline*}
\diag[\tau(g)\alpha_1 I_{n_1}, \ldots, \tau(g)\alpha_{s-1} I_{n_{s-1}}, \Lambda_1,\tau(g)\alpha_{s+1} I_{n_{s+1}}, \ldots , \tau(g)\alpha_k I_{n_{k}}, \\
 \alpha_{k+1} I_{n_{k+1}}, \ldots, \alpha_{k+l} I_{n_{k+l}}, \\
 \alpha_k I_{n_k}, \ldots, \alpha_{s+1} I_{n_{s+1}}, \Lambda_2, \alpha_{s-1} I_{n_{s-1}},  \ldots, \alpha_1 I_{n_1} ],
\end{multline*}
where $$
\diag[\Lambda_1, \Lambda_2] = 
\left( \begin{smallmatrix}
\lambda_1&\lambda_2&0&0\\
\lambda_3&\lambda_4&0&0\\
0&0&\lambda_5&\lambda_6\\
0&0&\lambda_7&\lambda_8
\end{smallmatrix} \right) \in
 \left\langle \left( \begin{smallmatrix}
1&0&0&0\\
1&2&0&0\\
0&0&1&1\\
0&0&0&2
\end{smallmatrix} \right), \left( \begin{smallmatrix}
2&2&0&0\\
2&1&0&0\\
0&0&2&2\\
0&0&2&1
\end{smallmatrix} \right)
 \right\rangle$$ and $\lambda_i \in \mathbb{F}_3$ for $i \in \{1, \ldots, 8\}.$ Notice that if $\lambda_2=\lambda_6=0$, then $\diag[\Lambda_1, \Lambda_2]$ is the scalar matrix $\alpha I_n$ with $\alpha \in \mathbb{F}_3^*,$ so $\tau(g)=\alpha^2=1.$

Assume $s>1$ and let $W=\langle e_1, \ldots, e_{m} \rangle$, where $m=n_1.$  Let $a \in GL_n(q)$ be such that 
\begin{equation*}%\label{aGL22qq3}
\begin{aligned}
&(e_1)a =\sum_{i=m+1}^{(n-4)/2} e_i +e_1 +e+ f; & & (f_1)a=f_1; \\
&(e_{j})a = e_j;  & &(f_{j})a=f_{j};  && j\in \{2, \ldots, m \}   \\
&(e_{j})a =e_j;  & &(f_{j})a=f_{j} - f_1;  && j\in \{m+1, \ldots, (n-4)/2 \}   \\
& (e)a=e+f_1; & & (f)a=f-f_1; \\
& (e_0)a=e_0; & & (f_0)a=f_0.
\end{aligned}
\end{equation*}
 It is routine to check that $a \in Sp_n(q, {\bf f}_{\beta}).$   Since $S$ stabilises  $W$,  $S^a$ stabilises $(W)a.$ %First assume that $W$ is totally isotropic, so $W = \langle e_1, \ldots, e_m \rangle$.
 Therefore, 
\begin{equation}\label{Wisotrsn1}
((e_1)a)g=
\begin{cases}
\sum_{i=m+1}^{(n-4)/2}\delta_i e_i +\alpha_1e_1 + \lambda_5 e+ \lambda_6 e_0 +\lambda_1 f + \lambda_2 f_0;\\
\sum_{i=1}^{m} \eta_i (e_i)a
\end{cases}
\end{equation}
for some $\eta_i \in \mathbb{F}_q.$
Since $((e_1)a)g$ does not have $e_i$ for $1<i  \le m$  in the first line of \eqref{Wisotrsn1}, $((e_1 )a)g = \eta_1 (e_1 )a$, so $\lambda_2=\lambda_6=0$ and
$$\alpha_1=\lambda_1=\lambda_5=\delta_{m+1}= \ldots= \delta_{(n-4)/2}.$$
Therefore, $g=\alpha_1I_n$ and  $g \in Z(GSp_n(q, {\bf f}_{\beta})).$

Assume $s=1$ and let $W =\langle e,e_0 \rangle.$ Let $a \in GL_n(q)$  be such that 
\begin{equation*}
\begin{aligned}
&(e)a =e+ f; & & (f)a=f; \\
&(e_{0})a =\sum_{i=1}^{(n-4)/2} e_i + e_0;  & &(f_{0})a=f_{0};  &&   \\
&(e_{j})a =e_j;  & &(f_{j})a=f_{j} - f_0;  && j\in \{1, \ldots, (n-4)/2 \}.   \\
\end{aligned}
\end{equation*}
It is routine to check that $a \in Sp_n(q, {\bf f}_{\beta}).$   Since $S$ stabilises  $W$,  $S^a$ stabilises $(W)a.$ %First assume that $W$ is totally isotropic, so $W = \langle e_1, \ldots, e_m \rangle$.
 Therefore, 
\begin{equation}\label{Wisotrsis1}
((e)a)g=
\begin{cases}
 \lambda_5 e+ \lambda_6 e_0 +\lambda_1 f + \lambda_2 f_0;\\
 \eta_1 (e)a + \eta_2 (e_0)a
\end{cases}
\end{equation}
for some $\eta_1, \eta_2\in \mathbb{F}_q.$ Since $((e)a)g$ does not have $e_i$ for $1 \le i  \le (n-4)/2$  in the first line of \eqref{Wisotrsis1}, $((e )a)g = \eta_1 (e )a$, so $\lambda_2=\lambda_6=0$ and $\diag[\Lambda_1, \Lambda_2]$ is the scalar matrix $\lambda_1 I_4.$ In the same way 
\begin{equation}\label{Wisotrsis12}
((e_0)a)g=
\begin{cases}
 \sum_{i=1}^{(n-4)/2} \delta_i e_i + \lambda_1 e_0;\\
 \eta_1 (e)a + \eta_2 (e_0)a
\end{cases}
\end{equation}
for some $\eta_1, \eta_2\in \mathbb{F}_q.$ Since $((e_0)a)g$ does not have $e$ or $f$  in the first line of \eqref{Wisotrsis12}, $((e )a)g = \eta_1 (e )a$, so 
$\lambda_1=\delta_i$ for $i \in {1, \ldots, (n-4)/2}$ and $g = \lambda_1 I_n \in Z(GSp_n(q, {\bf f}_{\beta}))$ since $\tau(g)=1.$
\end{proof}

Theorem \ref{theoremSp} now follows by Theorems \ref{SpSirr41}, \ref{Spqgt3} and \ref{Spq23}.

\subsection{Solvable subgroups not contained in $\GS_n(q)$.}
\label{sec432}

If $q=2^f$, then $Sp_4(q)$ has a graph-field automorphism $\psi$ of order $2f$; see \cite[\S 12.3]{carsim} for details. If $\beta$ is a basis of $V$ as in Lemma \ref{unist}, then we can assume that $\psi^2$ is $\phi_{\beta}$ by \cite[Proposition 12.3.3]{carsim}. 



\begin{T5}
Let $q$ be even and let ${A}=\Aut(PSp_4(q)')$. If ${S} $ is a maximal solvable subgroup of  {A}, then $b_{{S}}({S} \cdot Sp_4(q)') \le 4,$ so $\Reg_S(S \cdot Sp_n(q)',5)\ge 5$.
\end{T5}
\begin{proof}
For $q=2$ the statement is verified by computation.

 Assume $q=2^f$ with $f>1.$ Let $\theta$ be a generator of $\mathbb{F}_q^*.$ By \cite[Proposition 2.4.3]{kleidlieb}, $\Delta =Sp_4(q) \times \langle \theta I_4 \rangle$ where $\Delta$ is as defined before Lemma \ref{unibasisl}. Therefore, $$\Aut(Sp_4(q)) \cong Sp_4(q) \rtimes \langle \psi \rangle,$$ and we identify these two groups. Denote $\Gamma:=Sp_4(q) \rtimes \langle \psi^2 \rangle,$ so $\Gamma = Sp_4(q) \rtimes \langle \phi \rangle.$
 
  If $S$ lies in $\Gamma$, then the statement follows by Theorem \ref{theoremSp}.
 
 Assume that $S$ does not lie in $\Gamma$, so $S$ is in a maximal subgroup $H$ of $A$ not contained in $\Gamma$. For a description of such maximal subgroups see \cite[\S 14]{asch} and \cite[Table 8.14]{maxlow}. If $H$ is a non-subspace subgroup, then the statement follows by Theorem \ref{bernclass}. If $H$ is a subspace subgroup, then $H$ is solvable by \cite[Table 8.14]{maxlow}, so $S=H$ and $b_S(S \cdot Sp_4(q))\le 3$ by \cite[Lemma 5.8]{burPS}. 
\end{proof}

%%%%%%%%%%%%%%%%%%%%%%%%%%%%%%%%%%%%%%%%%%%%%%%%%%%%%%%%%%%
%%%%%%%%%%%%%%%%%%%%%%%%%%%%%%%%%%%%%%%%%%%%%%%%%%%%%%%%%%%
%%%%%%%%%%%%%%%%%%%%%%%%%%%%%%%%%%%%%%%%%%%%%%%%%%%%%%%%%%%



