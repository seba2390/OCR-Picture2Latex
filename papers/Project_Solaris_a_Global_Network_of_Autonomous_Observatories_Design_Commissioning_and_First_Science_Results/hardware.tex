% !TEX root = Solaris_PASP.tex

\section{System components}

\label{sec:HardwareComponents}

The Solaris network's design prerequisites define the detailed requirements that need to be met by the individual observatories. Though the major hardware components are off-the shelf products, they need to be integrated in such a way that the systems operate autonomously. The architecture of the system is presented on the diagram in Fig.~\ref{fig:SolarisOverviewDiagram}. A photographic overview of the components after installation is shown in Fig. \ref{img:HardwareOverview}. The proposed architecture is characterized by the following features.

\begin{itemize}
\item Ability to control the power supply state of all components. In a case, when a hardware reset is necessary, it can be performed by the system automatically or manually by the user. 
\item Power isolation. Sensitive and expensive components, such as the CCD camera and the telescope, are separated from components that are exposed to lightning strikes and power surges, e.g. weather stations, antennas, etc. The use of online UPS units additionally increases the level of power safety. Additionally, all components installed outdoor are equipped with surge arrestors. Fiber is used for data transfer wherever possible.
\item Focus on security. It is crucial for the dome to be closed whenever it is not safe to operate. Our system implements many safety features that supervise the dome. The dome is closed automatically during daytime, bad weather (the rain sensor is hardwired to the controller), in case of a loss of communication with the dome controller (both software errors or hardware problems) and when there is no active internet access on site. All sites but CASLEO offer a stable and reliable internet connection. SAAO even has a backup link. CASLEO, especially during winter, suffers from occasional power losses that eventually lead to major communication problems. Therefore, to be on the safe side, all sites treat the internet access property as one of the operating conditions.
\end{itemize}

Below we present an overview of the hardware setup. A much more detailed  description can be found in \cite{Kozlowski2014}.

\begin{figure*}[htb!]
\begin{center}
\includegraphics[width=\textwidth]{img/SolarisOverviewDiagram.eps}
\caption{Low-level architecture of a Solaris node. The diagram presents an overview of all components used in the observatory along with the most important communication channels that interconnect the elements of the system. A single server is used to control and integrate all subsystems (Tab. \ref{tab:ComputerHardware}). Original graphic from \cite{Kozlowski2014}.}
\label{fig:SolarisOverviewDiagram}
\end{center}
\end{figure*}


\begin{figure*}[htb!]
\centering
\includegraphics[width = \textwidth]{img/site_new.eps}
\caption{Solaris hardware overview. 
1. ASA DDM-160 telescope mount.
2. 0.5-m optical tube assembly.
3. Flatfield screen and FlatMount lifting manipulator.
4. Air-conditioner internal unit.
5. Dome segment motors.
6. PLC HVAC and supervision system.
7. 19-inch rack cabinet:
a - KVM remote console with keyboard and LCD monitor,
b - network router and switch,
c - control PC,
d - UPS units,
e - power supplies and accessories.
8. Imaging train:
a - Andor iKon-L CCD camera,
b - ASA field rotator,
c - FLI filter wheel.
9. Mirror covers.
10. Vaisala WXT-520 weather station.
11. Mobotix external surveillance camera.
12. Reinhardt weather station.
13. Meinberg GPS antenna.
14. Mobotix internal surveillance camera.
15. SBIG all-sky camera with custom \twopiskytm cloud detection module. Graphic based on original from \cite{Kozlowski2014}.}
\label{img:HardwareOverview}
\end{figure*}


\subsection{Astronomical equipment}

\paragraph{Telescope, mount and accessories.}
Astrosysteme Austria (ASA) mounts and optical tubes have been chosen for the project. Solaris-1, -2 and -4 are Ritchey--Chr\`etien telescopes, Solaris-3 is a Schmidt-Cassegrain design with a field corrector, all riding on ASA DDM-160 direct drive mounts. The DDM-160 mounts are installed on modified piers that allow the telescope to observe past the meridian significantly longer than in case of a classical design. The loading capacity is 300 kg. According to the manufacturer's specification the pointing RMS should be better than 8 arc seconds and the tracking precision should be better than 0.25 arc sec. RMS during 5 minutes - all thanks to hight resolution (0.007 arc sec) incremental encoders. Figure \ref{fig:TrackingTest} shows tracking test results. The maximum slewing speed is 13 degrees/s. Unfortunately, the mounts have a USB interface (based on a FTDI chip) making the setup very sensitive to communication errors. The optical tube assembly is fitted with a focuser, mirror covers and a field rotator -- all motorized and controlled via dedicated software -- Autoslew. Autoslew has a graphical user interface and handles the configuration of the mount. The user can control the parameters of the PID controllers (tabulated for different slewing speeds) and even the parameters of filters that are used in the control loop. The pointing model can be created manually or with the help of dedicated software (Sequence by ASA) that automates the process.

\begin{figure}[htb!]
\centering
\includegraphics[width=\columnwidth]{img/tracking.eps}
\caption{RA and DEC position recorded during a 7 hour run on Solaris-2, no position corrections have been made during this time. Wind speed 12-15 km h$^{-1}$.}
\label{fig:TrackingTest}
\end{figure}

\paragraph{Imaging train.}
All four telescopes are equipped with Andor iKon-L CCD cameras that are based on e2V CCD42-40 chips and fitted with four stage thermoelectric cooling that cools the CCD down to -70\degree~Celsius. The camera's shutter is connected to a GPS card that records the opening and closure times of the shutter providing a very precise time stamp that is then saved in the image header. A Fingerlakes Instruments filter wheel with $\phi$50 mm Johnson (UBVRI) and Sloan (u'g'r'i'z') filter is installed as well. All four imaging trains are identical with the exception of Solaris-3 that has an additional field corrector and Solaris-1 that is fitted with a spectrograph.

\paragraph{\'Echelle spectrograph.}

In 2013, Baches, a prototype \'echelle spectrograph has been tested on the Solaris-4 telescope \citep{Kozlowski2014}. The spectrograph body is 290x100x52 mm in size and weighs less than 1.5 kg, making it a very compact instrument, even after including the spectroscopic and guide cameras. Both are very conveniently attached to the instrument.  Internally, the instrument consists of a collimator lens, a 63 l/mm 73$\degree$ \'echelle grating, a cross-dispersing diffraction grating and an objective. The instrument is optimized for f/10 input beams and cameras based on the KAF-1603 CCD chips. After a successful test campaign, a final production version of Baches has been installed permanently on the Solaris-1 telescope in South Africa -- Fig. \ref{fig:BachesSet}. To avoid the need of manual instrument changes (photometry remains the main observing mode of the telescope), the imaging train has been fitted with a custom designed guide and acquisition module (GAM). The unit includes an internal flip mirror allowing for remote selection of the desired instrument: CCD camera or spectrograph. The spectrograph setup also includes a remote calibration unit (RCU) that is used to feed light from quartz and thorium-argon calibration lamps to the spectrograph via a fibre. After commissioning, the spectroscopic mode has been thoroughly tested during a dedicated observing campaign \citep{Kozlowski2016}.

\begin{figure}[htb!]
\begin{center}
\includegraphics[width=\columnwidth]{img/BachesSet.eps}
\caption{Solaris-1 imaging train: 1 -- BACHES spectrograph, 2 -- GAM, 3 -- guide camera, 4 -- filter wheel, 5 -- photometric CCD camera, 6 -- spectroscopic camera (only mounting flange is visible), 7 -- spectrograph control cabling and fiber. Graphic based on original from  \cite{Kozlowski2014}.}
\label{fig:BachesSet}
\end{center}
\end{figure}

\label{sec:spectrograph}

\subsection{Computer hardware}
\label{ssec:ComputerHardware}

Each observatory is controlled by a single server-grade computer fitted with a GPS card and a multi-port serial card (Tab. \ref{tab:ComputerHardware}). Fiber is used where needed  (Fig.~\ref{fig:SolarisOverviewDiagram}) and all components have UPS backup power. Practice shows that USB connections can be unstable, especially when large data throughput is required. To maximize robustness, the following has been taken into account: USB data cables short as possible, respecting the standards (e.g. microUSB and similar have cable length limits), use of tested configurations (e.g. the Icron USB-fibre extender), use of reliable power supplies with proper cabling (cable gauge  adjusted according to the current, voltage requirements and length) and proper grounding of all equipment.

\begin{deluxetable*}{rp{13cm}}
%\tablewidth{0pt}
%\tabletypesize{\scriptsize}
\tablecaption{Computer hardware as of January 2017.}
\tablehead{\colhead{component} & \colhead{description}}
\startdata			
PC  				&  SuperMicro server-grade motherboard, Intel Xeon with 32GB RAM, 200 GB SSD (RAID 1), 12 TB storage \\ 
computer access	& KVM with integrated LCD monitor and keyboard \\
time source		&  high precision PCI Express GPS receiver with a dedicated antenna and external hardware time event capture  \\ 
internet access		&  1Gbit fiber to ethernet converter, 1Gbit router and 24-port 100Mbit ethernet switch  \\ 
power backup 		&  two on-line UPS units: 3000VA and 1500VA  \\ 
power distribution	&  two power distribution units, network enabled \\
power supplies		&  astronomical equipment, incl. telescope with focuser and mirror covers, CCD camera and imaging train components, flat field screen and FlatMount, IP surveillance cameras, weather monitoring devices, emergency lighting
\enddata
\label{tab:ComputerHardware}
\end{deluxetable*}

\subsection{Dome and infrastructure}

\paragraph{Dome.}
Clamshell domes manufactured by Baader Planetarium GmbH have been selected for the project. The 3,5-m domes are made of  fibre-reinforced plastic and are self-sustained structures consisting of a cylindrical dome base and four motorized segments that allow the dome to be opened fully and provide an unobscured view of the entire sky. The sandwiched structure is well insulated. The concrete footing precisely matches the external circumference of the dome and is separated from the telescope's pier base to prevent the transfer of vibrations from the dome onto the telescope. 

\paragraph{Weather stations.}
Real time weather information is provided by two weather stations that measure temperature, relative humidity (redundant), wind speed and direction, precipitation (redundant) and the cloud base. One of the rain sensors is additionally hard-wired to the dome controller for extra security. 

\paragraph{All-sky camera.}

To provide precise cloud coverage information all sites have been equipped with an SBIG all-sky camera and a \twopiskytm add-on module that provides the cloud detection functionality. The device analyses the all-sky images and determines the cloud coverage based on photometric measurements. The results are used as input by the control system. \twopiskytm also provides a web interface that gives access to the configuration options and image database (Fig. \ref{fig:AllSkyImages}). The approach is sensitive to high altitude clouds that are not detected by the simple temperature-based  cloud sensors. 

\begin{figure*}[htb!]
\begin{center}
\includegraphics[width=0.49\textwidth]{img/AllSkyImage_bad.eps}
\includegraphics[width=0.49\textwidth]{img/AllSkyImage_good.eps}
\caption{Sample all sky images obtained with the all-sky monitoring system installed in Colmplejo Astronomico El Leoncito in Argentina. Each of the 60 second exposures shows different cloud coverages. Green circles denote stars properly detected and identified in the catalog, red circles denote stars obscured by clouds.}
\label{fig:AllSkyImages}
\end{center}
\end{figure*}


\paragraph{Flatfield screen and FlatMount.}
For camera calibration purposes (nonlinearity, shutter effects and general troubleshooting), all sites have been equipped with 0.6x0.6-m electroluminescent flatfield screens that are mounted on a dedicated manipulator called the FlatMount. The device can raise the screen during calibration and lower it during normal observing so that the sky view is not obscured. 

\paragraph{Surveillance cameras.}

All sites are under video surveillance. IP cameras are installed inside the domes (monochrome, high sensitivity, fisheye lens) and oudoors to overlook the domes (color cameras with narrow-angle lenses). 

\paragraph{Building management system.}
\mbox{}\newline \owtm~is a PLC-based  building management system that handles the heating, ventilation and air conditioning of the dome, emergency closure and rack cooling. This approach was chosen to popularize industry standards in the world of small astronomical observatories. The design of the PLC system is illustrated in Fig. \ref{fig:PLCdesign}. 

\begin{figure}[htb!]
\begin{center}
\includegraphics[width=\columnwidth]{img/PLCdesignPASP.eps}
\caption{Building Management System design overview. \cite{Kozlowski2014}.}
\label{fig:PLCdesign}
\end{center}
\end{figure}

\paragraph{Lightning protection, grounding and electrical system.}
Proper grounding and lighting protection is extremely important in remote, high altitude locations. All Solaris sites have a carefully designed TT-type electrical network with proper grounding and lightning protection based on bentonite, copper rods and deep rock drillings. The Solaris-4 site in Argentina, due to its solitude in the area, has a dedicated 10.5-m lightning protection mast. The mast can influence the quality of photometry, but it's active cross-section is minimal and is therefore not taken into account during observing planning.


