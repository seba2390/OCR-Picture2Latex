% !TEX root = Solaris_PASP.tex

\section{Introduction}
\label{sec:Introduction}

Various exoplanet detection methods exist that are sensitive to different star-planet configurations. In the past, exoplanets orbiting both components of a binary system (circumbinary planets) have been searched for using, for example, radial velocities (RVs), including the iodine cell technique applied to double-lined spectroscopic binaries \citep{Konacki2005,Konacki2009}, direct imaging \citep{Currie2014,Kraus2014,Bonavita2016}, or microlensing \citep{Bennett2016}. The first transiting circumbinary planet has been discovered by Doyle et al. (2011) with several more to follow since then, including the popular Kepler-47 system \citep{Orosz2012} or a quadruple star PH-1 \citep{Schwamb2013}.

The main goal of Project Solaris is to conduct an observing campaign in the search for circumbinary planets using a dedicated network of telescopes that can supply high cadence and high precision photometry data for (a) the utilization of the eclipse timing method and for  (b) characterization of eclipsing binary stars. The secondary goal of the Project is to increase the competence in the design and construction of robotic telescopes, associated hardware components and software. The name \textit{Solaris} is a tribute to Stanis\l aw Lem's identically titled novel published in 1961. In his book the famous Polish writer describes a fictional planet that exists in a binary system. To date, 25 planetary systems (31 planets, 5 multi planetary systems) have been detected strictly using timing methods (exoplanet.eu, June 2017), some of which are not widely accepted by the community due to low credibility of the results.  Possibly, more exoplanets will be discovered using transit timing variation in the Kepler data \citep{Borucki2010,Koch2010}.  

The timing technique with respect to exoplanets dates back to 1992, when the first exoplanet was discovered: PSR~1257~12b \citep{Wolszczan1992, Konacki2003}. Variations in the otherwise precisely periodic pulse signal have been identified as the footprint of the presence of an additional body in the system. The motion of the pulsar around the common center of mass and the finite speed of light cause the pulses appear to the observer being too early or too late, which is known as the light-time effect \citep{Sterken2005a,Sterken2005b}. 
In case of eclipsing binary stars, the regular eclipses act as the carrier signal instead of pulsar pulses. An additional body in the system will cause variations in timing of eclipses. The resulting light-time orbit can be derived from long-span photometric data.

%BAD
%The ASAS\footnote{All Sky Automated Survey} Catalogue of Variable Stars  \citep[ACVS;][]{Pojmanski1997} contains 50122 variable stars, including 11062 eclipsing binaries. 5374 of these are classified as contact binaries, 2939 as semi-detached binaries and 2749 as detached binaries. 60 stars have been selected from the last group based on eclipse width, eclipse depth, maximum magnitude, orbital period the availability of nearby reference stars and the expected timing precision. An initial eclipse timing analysis of selected binaries from the ACVS \citep{Kozlowski2011} revealed a few interesting systems and served as a test-bed for software development. To extrapolate these results onto a dedicated survey, \cite{Sybilski2010} conducted numerical simulations that showed that it is possible to detect exoplanets orbiting eclipsing binary systems using a network of 0.5-m telescopes achieving 0.1-1.0 s precision in timing. To accomplish this task, a global network of identical 0.5-m telescopes has been proposed. Other means of gathering an amount of data suitable for eclipse timing analysis are highly impractical. As an example, covering eclipses of 50 targets would require applying for time on different globally-dristributed telescopes in a very strict scheduling mode generating high costs and nontrivial logistical problems if traveling to the observatories would be necessary. Being a positive side effect, precise photometric data gathered throughout the project, when coupled with spectroscopy, will help in determining stellar parameters with very high precision. Lastly, a well established network of autonomous observatories can be equipped with other instruments, e.g. spectrographs.

The first Automated Photoelectric Telescopes have been built in the mid 1960's. Since then more and more telescopes operate without direct human supervision \citep{Castro2010}, sometimes during extended periods of time. Instruments attached to telescopes tend to be more complex and need to fulfill the most demanding requirements of astrophysicists in terms of efficiency, speed, precision and stability. In many cases manual operation of modern telescope systems is not even possible. This applies not only to the largest instruments but also to distributed networks of smaller telescopes, such as Project Solaris, that operates a global network of autonomous telescopes. In Sec \ref{sec:NetworkDescription} we describe the global network, locations of the observatories. In Sec. \ref{sec:HardwareComponents} we describe the hardware components of the individual sites. Section \ref{sec:SoftwareArchitecture} describes the software that is used to control, manage and operate the entire network. In Sec. \ref{sec:Operation} we focus on the operation of the network to date including major problems encountered during the installation and operation phases. We present scientific commissioning results in Sec. \ref{sec:FirstResults} and we summarize in Sec. \ref{sec:Summary}. 

\subsection{Autonomous Observatories and Existing Telescope Networks}
\label{ssec:ExistingNetworks}

Types of observatories based on their operation mode have been defined by \cite{Gelderman2001}. Observatories can be remote, unmanned, robotic and fully autonomous. This nomenclature refers mainly to the way observations are executed, i.e. how advanced are the scheduling algorithms. From the system's engineering point of view, however, the proposed classification is not complete. This becomes more evident when the the telescope is treated as a robot with two (usually) degrees of freedom that operates in a controlled environment. A detailed elaboration on the nomenclature is provided in Sec. \ref{sec:SoftwareArchitecture}. Two or more observatories located in different sites that operate within the same framework or are governed by the same institution constitute a network of observatories. In 1956, 12 satellite tracker stations were deployed marking the beginning of the era of observatory networks  \citep{Whipple1956}. Since then many networks comprising telescopes with a wide range of apertures have been designed and commissioned. Table \ref{tab:Networks} lists selected networks of telescopes that have inspired us during the design process. 

\begin{deluxetable*}{@{}p{4cm}p{2cm}p{6cm}l@{}}
%\tablewidth{0pt}
\tablecaption{Selected networks of telesocpes.}
\tablehead{\colhead{Project} & \colhead{Infrastructure} & \colhead{Comments} & \colhead{Reference}}
\startdata			
%Whole Earth Telescope (WET) & multiple, 1-m & campaigns aimed at monitoring variabilities of all sorts; not built as a network, rather a consortium of observatories & \cite{Nather1990}\\
Probing Lensing Anomalies NETwork (PLANET) & 3 x 1-m & worldwide network that discovered several microlensing phenomena & \cite{Albrow1998} \\
 RoboNET & 3 x 2-m & Hawaii, La Palma and Australia, partially owned by LCOGT & \cite{Tsapras2009}\\
 Robotic Optical Transient Search Experiment (ROTSE-III)  & multiple 0.5-m & globally distributed, aimed at the detection od optical transients & \cite{Akerlof2003} \\
 The Kilodegree Extremely Little Telescope (KELT)  & two sites with wide field 80 mm lenses  & northern and southern Hemisphere sites, dedicated to search for transiting exoplanets around bright stars   & \cite{Pepper2007}\\
 Hungarian-made Automated Telescope Network (HATSouth) & 6 units & unit consists of four 0.18-m f/2.8 optical telescopes on a common mount that have a combined field of view of 8.5\degree x 8.5\degree; dedicated to detect transiting exoplanets & \cite{Bakos2013} \\
 Las Cumbres Observatory Global Telescope Network (LCOGT) & 2 x 2-m, 17 x 1-m, multiple 0.4-m & 7 locations in both hemispheres, dedicated for professional research and citizen science projects & \cite{Brown2013} \\
 Master-II & 7 x twin 0.4-m &  dedicated to observing optical counterparts of gamma-ray bursts (GRBs) &  \cite{Gorbovskoy2013} \\
 RAPid Telescope for Optical Response (RAPTOR) & 2 arrays & optical transients monitoring, spectroscopy & \cite{White2004} \\
 Wide Angle Search for Planets  (SuperWASP)& 2 arrays & array consists of eight wide-field cameras on an equatorial mount, 482 square degrees total field of view, exoplanetary transits & \cite{Pollacco2006} \\
 MOnitoring NEtwork of Telescopes  (MONET) & 2 x 1.5-m & time available to schools, photometry and spectroscopy & \cite{Bischoff2006} \\
 Pi of the Sky & two arrays & multiple telephoto lens and custom CCD cameras &  \cite{Wawrzaszek2010}\\
Skynet Robotic Telescope Network & 9 sites & global consortium of robotic telescopes that use the Skynet job-queuing & \cite{Reichart2008} 
\enddata
\label{tab:Networks}
\end{deluxetable*}


Apart from telescope networks, a large amount of single autonomous telescopes operate all around the globe - both professional and amateur.
