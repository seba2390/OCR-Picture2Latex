% !TEX root = Solaris_PASP.tex

\section{Network Description}
\label{sec:NetworkDescription}

%To realize the scientific goal of Project Solaris as described in. Sec. \ref{sec:Introduction}, we have designed and established a global network of autonomous observatories in the Southern Hemisphere. Below we present very general prerequisites that served as guidelines throughout the design and installation phases. In the course of the project these guidelines have been specified in more detail and applied in concrete solutions that are described later in this paper.


%\subsection{Design Prerequisites}
%\label{ssec:DesignPrerequisites}
%The goal of the Solaris Project determines the conditions that must be satisfied by the proposed network of autonomous telescopes.
%\begin{enumerate}
%\item Global coverage. The network must be designed in such a way that a selected celestial target can be observed continuously or nearly continuously assuming favorable weather conditions over a period of 24 hours and more. This implies even longitudinal spacing and close latitudes of the sites that are characterized by very good or excellent observing conditions throughout the year with local support staff.
%\item Fully autonomous system. Due to the global coverage of the network and very limited human resources, it is crucial that the observatories operate autonomously as outlined in Sec. \ref{ssec:ExistingNetworks}. 
%\item Compatibility across the network. This is crucial for integration of the network into one system. Individual observatories need to be designed in such a way that they are autonomous and independent of each other (even in case of physical proximity) but at the same time easy to consolidate.
%\item Emphasis on photometry and timing precision. This implies concrete requirements in terms of hardware and software engineering. Adequate devices must be available to properly calibrate the equipment and monitor any abnormalities that may influence photometric and timing errors. 
%\item Standardization and modularity. From the systems engineering point of view, all components of the network should be identical across all sites and allow easy replacement and/or upgrade. This facilitates not only regular maintenance but also reduces downtime and the cost of operating the network. 
%\end{enumerate}

\subsection{Remote sites}
\label{ssec:RemoteSites}
The Solaris observatories are located in the Republic of South Africa, Australia and Argentina. Figure \ref{fig:SolarisMap_setellite} gives a graphical overview. All of them lie within less than 1\degree\ difference in latitude. The nighttime coverage is shown in Fig.~\ref{fig:NetworkCoverage}. The plot shows a theoretical result that takes into account only the Sun's position at the respective sites. Actual object observability will be determined by its coordinates and the weather conditions.

%\begin{figure*}[ht]
%\begin{center}
%\includegraphics[width=\textwidth]{img/SolarisMapSPIE.png}
%\caption{World map with Solaris sites. Longitudinal separations between the sites, counting from Solaris-1 towards the east are: 128\degree 15', 141\degree 38' and 90\degree 7'. Original source Wikimedia Commons.}
%\label{fig:SolarisMap}
%\end{center}
%\end{figure*}

\begin{figure*}[htb!]
\begin{center}
\includegraphics[width=\textwidth]{img/Sites_Satellite.eps}
\caption{Birdseye views of the three observatories with exact Solaris sites marked along with several important waypoints at the South African Astronomical Observatory (SAAO, South Africa), Siding Spring Observatory (SSO, Australia) and Complejo Astron\'omico El Leoncito (CALSEO, Argentina). North is up on all images.  Background image source google.com.}
\label{fig:SolarisMap_setellite}
\end{center}
\end{figure*}

\begin{figure}[htb!]
\begin{center}
\includegraphics[width=\columnwidth]{img/NetworkCoverage.eps}
\caption{Network nighttime coverage throughout the year. Nighttime occurs whenever the Sun is below -18\degree ~altitude.  Plots indicate the number of hours that nighttime between respective sites overlaps (positive values) or has a gap (negative values) per every 24 hours. Magenta, green and blue colors represent the following site pairs: SSO - SAAO, SAAO - CASLEO and CASLEO - SSO, respectively. The red line represents the sum of the gaps in coverages for the entire network. The network covers permanent nighttime from the end of March till mid-September, 46\% of the year. During the southern hemisphere summer the total gap in coverage reaches 5 hours per day. The largest gap occurs between CASLEO and SSO due to the largest longitudinal separation.}
\label{fig:NetworkCoverage}
\end{center}
\end{figure}

The \textbf{Solaris-1 and Solaris-2} telescopes are located on the premises of the South African Astronomical Observatory near Sutherland in the Hantam Karoo at an elevation of 1842 m AMSL. They share the plateau with many other telescopes among which are the Southern African Large Telescope, a station of the Birmingham Solar Oscillations Network (BiSON), the Kilodegree Extremely Little Telescope (KELT-South), LCOGT, Monet, SuperWasp, Master. SAAO's infrastructure is very well developed and managed with very good technical support. 

\textbf{Solaris-3} is located at the border of the Warrumbungle National Park, near Coonabarabran in New South Wales, Australia at 1165 m AMSL elevation. This volcano crater location posed construction difficulties at the same time being a very picturesque area. The observatory is home to the  3.9-m Anglo-Australian Telescope (AAT), Faulkes South, HAT-South, ROTSE, UK Schmidt Telescope (UKST), the The Automated Patrol Telescope. The technical support staff at the mountain is very professional. 

\textbf{Solaris-4} is located in the El Leoncito National Park. The Complejo Astron\'omico El Leoncito observatory is located in the San Juan province. The site comprises the main observatory buildings including the Jorge Sahade 2.15-m telescope and a remote location at a higher elevation of 2552 m AMSL, 7 km away - this is where Solaris-4 has been built. This site has been particularly challenging in terms of customs regulations and bad road conditions in the area (Fig. \ref{fig:casleo}). The network's headquarters is located in Toru\'n, Poland.

\begin{figure*}[htb!]
\begin{center}
\includegraphics[width=\textwidth]{img/CASLEOPano_s.eps}
\caption{Solaris-4 site, CASLEO, Argentina. The site is accessible only with a proper 4x4, especially during and after rainfall. The steel cables visible on the right are guy-wires that support the lightning mast.}
\label{fig:casleo}
\end{center}
\end{figure*}
