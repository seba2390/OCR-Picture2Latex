% !TEX root = Solaris_PASP.tex
\section{Operation}
\label{sec:Operation}

\subsection{Weather statistics}

As described in the previous sections all three Solaris sites are equipped with weather monitoring equipment. This not only permits autonomous observing but  also allows us to analyze weather measurement data that is stored in the database. Figure \ref{fig:weatherSAAO} illustrates how different weather parameters influence the observability at SAAO. Relative humidity has the most significant contribution to the non-observing night time (27.0\%), followed by cloud cover (27.3\%) and wind speed (7.7\%). In practice the total count of night time hours with favourable weather conditions will be somewhat smaller due to several factors. One of them is the hysteresis (easing time) that is associated with each observing parameter independently and prevents the system from repeatedly closing and opening the  dome during unsure weather conditions. Secondly, the cloud base contribution is underestimated. Due to the operating principle of the cloud sensor, hight altitude clouds can be sometimes undetected which leads to worse quality data being collected.

\begin{figure*}[htb!]
\includegraphics[width=\textwidth]{img/SLR1Summary_RH84.eps}
\caption{Observing conditions analysis for SAAO during 800 days staring 30-04-2013. The top panel shows the amount of night hours per day with weather parameters within allowable limits. Dark hours are defined such that the Sun is lower than 18 degrees below the horizon. A total of 62.0\% of night time had favourable observing conditions. Gaps in the data that are the effect of system downtime or other abnormalities are not taken into account. The three lower panels show the contribution of relative humidity (RH), wind speed and cloud base into the non-observing time. Two wind speed parameters are used - average wind speed and maximum wind speed (the limits are 10 and 12 ms$^{-1}$, respectively). Precipitation is not analysed as it is a subset of the cloud contribution. Current observing limits related to ambient conditions are listed in Tab. \ref{tab:ObservingLimits}}.
\label{fig:weatherSAAO}
\end{figure*}

\subsubsection{Observing limits}

Autonomous operation requires well defined observing limits to ensure that observations are only carried out during good weather conditions. The limits are, in general, more conservative than they would be for a human observer to maintain a larger safety margin in case of unexpected equipment failures. Not only ambient weather conditions are taken into account. Safe operation requires that parameters such as rack temperature, humidity, UPS battery capacity are monitored as well. Table \ref{tab:ObservingLimits} lists the most important parameters and their allowed values for Solaris-1. Most limits are identical throughout the sites. Cloud base is an exception due to different altitudes of the observatories that need to be accounted for.



\begin{deluxetable}{lcc}
%\tablewidth{0pt}
\tabletypesize{\scriptsize}
\tablecaption{Example observing limits for Solaris-1. If a parameter value falls out of the allowable range it is considered to be in this state for the amount of time defined by the easing time value.}
\tablehead{\colhead{Variable} & \colhead{Limits} & \colhead{Easing time (min)}}
\startdata			
Temperature & $-1<T<30$ deg. C & 15 \\
Ambient RH & $RH<84$\%  & 15 \\
Minimum wind speed & $v_{min}<9$ ms$^{-1}$ & 25 \\
Average wind speed &  $v_{avg}<10$ ms$^{-1}$ & 25 \\
Maximum wind speed &  $v_{max}<12$ ms$^{-1}$ & 25 \\
Rain intensity & $i=0$ & 30 \\
Cloud base altitude & $h>3000$ m & 10 \\
Rack RH & $RH < 83$\%& 10 \\
Rack temperature & $0<T < 32$ deg. C& 10 \\
UPS battery capacity & $c > 80$\% & 10
\enddata
\label{tab:ObservingLimits}
\end{deluxetable}

\subsection{Observing workflow}

Observing workflow is an ordered list of steps that defines the behavior of the system during daytime, twilight and nighttime. An overview of the daily workflow is shown in Fig. \ref{fig:Workflow}. The system is in sleeping state during daytime. During this stage leftover data from the previous night is uploaded to the the servers in Poland. All equipment remains switched on. At a defined moment before sunset the system starts to prepare for observing. The photometric CCD camera is cooled down to its operating temperature of -70 deg. C. Sets of bias and dark frames are acquired. After the Sun sets and the light sensor's output drops below the defined threshold, the dome opens and flatfields are acquired. The order of filters and amount of flatfield frames to be acquired are defined in a configuration file. The exposure times are adjusted automatically to guarantee that the frames are properly exposed. If for a given filter the exposure time is too short (the shutter becomes evident in such case) the system delays the execution of the remaining flatfields. As the sky darkens, the remaining flatfields are acquired. After the flatfielding procedure is complete, the dome closes and the system waits for twilight to end. Once this happens, the observing queue is populated and the observing loop starts. The queue can be populated in two ways -- from a manually created observing program  in form of an xml file or from a cloud server. In the first case the user needs to make sure that the selected targets are observable in the order defined in the file. This mode is used when targets that are not in the global observing schedule need to be observed, usually for one night or part of a night. Normally, the observing queue is fed with targets from the cloud server. This service provides small chunks of the observing queue that are generated upon request for the specific site multiple times over the night. This scheduler makes sure that long-term project goals are accomplished in the global scale. Each element in the observing queue includes the name of the program, target identifier, field coordinates (that usually differ from the target coordinates), exposure time, filter band and a boolean value that decides whether exposure time should be automatically modified based on pixel counts from previous exposures. This is a very useful feature that helps in optimizing the exposure time and compensates for zenith distance and seeing changes throughout the night. Another functionality implemented in the system is astrometric frame solving and position correction. This feature is used to correct pointing errors and precisely position the field on the CCD frame, e.g. when one or more very bright stars need to positioned just outside the field of view. At all times all sensor data is read out processed and if necessary, the dome is closed an the observing queue paused. The observing queue is cleared at morning twilight and the dome closes. Morning flatfields are taken before sunrise following the same procedure as evening flatfields. Once this is done, sets of bias and dark frames are acquired. After that the system warms-up the CCD camera, enters sleeping state and remains in it during the day. All operations described above are executed asynchronously whenever possible. Multiple distributed security features on different levels overlook the safety of the system.

\begin{figure}[htb!]
\centering
\includegraphics[width=\columnwidth]{img/workflow.eps}
\caption{Solaris high-level observing workflow. Stages in the diagram are described in text.}
\label{fig:Workflow}
\end{figure}


\subsection{Faults and major problems}
\label{sec:Faults}
We have experienced a number of equipment failures and problems. Probably the most notable ones concern the Andor CCD cameras. All four required servicing at least once at some point of the project. Considering the process of handling and shipping the camera from the observatory to the manufacturer and back, going through customs, etc., every failure of the camera meant a 2-3 month downtime in the operation of the observatory. Other hardware problems concerned the Vaisala weather transmitter (communication module replacement), Moxa serial card and fiber-ethernet converters. The PC internals in Solaris-3 required replacement due to problems with the motherboard. We have also experienced multiple UPS failures in Solaris-3, one of which involved a totally burnt unit. Luckily the security system isolated power from the equipment and nothing else had been damaged.

In January 2013 a severe bushfire passed through the Warrumbungle National Park threatening the Siding Spring Observatory. Solaris-3, as well as all other domes on the mouton survived the threat without damage. A few buildings and offices, however, have been destroyed. Since the service visit following the bushfire in early 2013, Solaris-3 has been operating for four years without the need for another maintenance visit.
In July 2013 a hail storm passed through the South African Astronomical Observatory. Two air-conditioning units (outdoor) required repair. 
%
%\begin{figure}[ht]
%\includegraphics[width=\columnwidth]{img/bushfire.png}
%\caption{ Solaris-3 under bushfire threat in January 2013.}
%\label{fig:bushfire}
%\end{figure}


