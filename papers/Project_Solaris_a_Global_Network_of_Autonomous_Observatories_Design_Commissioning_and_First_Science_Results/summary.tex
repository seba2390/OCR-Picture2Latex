% !TEX root = Solaris_PASP.tex
\section{Summary}
\label{sec:Summary}

In this paper we have presented a global network of autonomous telescopes called Solaris. We have described the motivation that pushed us towards designing and building this scientific infrastructure consisting of four observatories located in the Southern Hemisphere. A thorough description of the network as a whole has been provided along with many technical details of the individual observatory components. Our design approach assumed the use of off-the-shelf components whenever possible, industrial standards and technologies wherever practical. Even the best design with carefully selected hardware will be useless without proper integration. In this paper we have described our software engineering approach, assumptions and prerequisites that led us to designing and building a custom observatory control software suite that allows the network to operate efficiently with minimal human intervention. Our modern approach and the use of industry approved computer science technologies allow the network to operate in such a way that reliability, availability and absolute efficiency metrics are in the high 90\% range. The autonomy of the system starts with observation scheduling and job distribution, through observatory control, image acquisition and ends with data transfer to the project's headquarters. Thanks to a small but dedicated team of professionals with profound experience in astrophysics, computer science and engineering we have reached the expected goals of the project. We have selected a set of eight targets that were used for scientific commissioning of the network. The goal was to verify that certain observing tasks can be accomplished with the network. Individual observing capabilities were verified during exoplanetary transit observations. Data has been reduced and analyzed with available software packages and our own software. Fitting transit models revealed residuals RMS values between 2.9 and 3.6 mmag. These have been compared with results available in the literature. We have also investigated three objects that are taking part in the timing campaign -- KZ Hya, RR Cae and SOL-0023. For these we have computed initial models and confirmed that formal photometric errors agree with RMS scatter. The commissioning campaign also included an interesting eclipsing sdB system. This particular target was used as the testbed for photometric precision and cadence. Again, using available software tools we were able to identify and distinguish pulsation semiamplitudes down to 2.1 mmag and construct a model of the system with the help of RV data available in the literature. Finally, we have presented a full detached eclipsing binary model based on photometric data obtained with three telescopes observing the same target in a continuous mode. Thanks to the global network we were able to cover more than four cycles in 55 hours with small gaps. RV measurements from the Chiron spectrograph supplement photometric data. We conclude that the commissioning results satisfy the project's assumptions.

Apart from showing the scientific potential of the network we have also described lessons learned from building and operating the network, major faults and problems that had to be resolved. 

Instrumental projects, such as Solaris, are mainly engineering undertakings, especially during the design and construction phases. Although most components of the observatories are off-the-shelf devices, their integration requires significant effort that builds expertise. If properly managed and financed, this know-how can gain commercial value. Several products that have been developed for the project have reached a level of maturity that allowed them to be commercialized via spin-out companies that were established by the project's team members. These products include the Abot software suite\footnote{http://sybillatechnologies.com} that runs the entire network, 2$\pi$Sky\footnote{http://www.cilium.pl}, the embedded cloud monitoring system, ObservatoryWatch, the PLC supervision system and several smaller software and hardware components that have been found to have market value as parts of larger systems. 

We believe that this is the proper way of running scientific instrumental projects, even in astrophysics, that initially seem not to have commercializable value. In fact, this approach brings benefit to the scientific part of the undertaking when financing of the project ceases. When this happens, further development is still possible thanks to private funding and cooperation with industry partners that initially derived from the scientific project. 

\acknowledgments

We are grateful to the technical and administration staff in SAAO, SSO and CASLEO for their help during the construction and commissioning of the telescopes.
This work is supported by the European Research Council through a Starting Grant, the National Science Centre through grant 5813/B/H03/2011/40, the Polish Ministry of Science and Higher Education through grant 2072/7.PR/2011/2 and the Foundation for Polish Science through \textit{Idee dla Polski} funding scheme. K.G.H. acknowledges support provided by the National Science Center through grant 2016/21/B/ST9/01613. P.S. acknowledges support provided by the National Science Center through grant 2011/03/N/ST9/03192. M.R. acknowledges support provided by the National Science Center through grant 2011/01/N/ST9/02209.

