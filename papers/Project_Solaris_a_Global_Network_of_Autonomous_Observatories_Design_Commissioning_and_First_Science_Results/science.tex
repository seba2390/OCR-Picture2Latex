% !TEX root = Solaris_PASP.tex

\section{Scientific commissioning}
\label{sec:FirstResults}
In this section we present scientific results obtained with the Solaris telescopes during the commissioning phase. We demonstrate capabilities of individual telescopes and the network as a whole in delivering high quality scientific data. We focus mostly on photometry, but for selected cases we combine it with spectroscopy to present complete astrophysical models of eclipsing binaries.

\subsection{Photometric results}

Data reduction poses many challenges, especially if it is automated. Currently, a custom data reduction pipeline is being tested and verified. Part of the commissioning data presented in this paper has been reduced with this pipeline to demonstrate its possibilities. Unlike off-the shelf software packages, the dedicated pipeline is tailored to efficiently reduce data gathered by the Solaris telescopes and take full advantage of their capabilities such as shutter effects and camera nonlinearity modeling and SQL database interface (over 2M frames have been acquired).
The list of objects observed photometrically comprises exoplanetary transits and eclipsing binaries of various types and is presented in Tab. \ref{tab:campaign}. The range of $V$ magnitudes of the observed objects spans 8.99 to 14.40.

\begin{deluxetable*}{llrll}
\tablecaption{Targets observed with the Solaris network during commissioning and testing phase.}
\tablehead{\colhead{Object ID} & \colhead{J2000 Coordinates} & \colhead{V mag} & \colhead{Telescope} & \colhead{Comments}}
\startdata
Wasp-4b 			& \ra{23}{34}{15}{06}  \de{--42}{03}{41}{10}	& 12.5   & SLR1 &  exoplanet \\
Wasp-64b 		& \ra{06}{44}{27}{61}  \de{--32}{51}{30}{25} 	& 12.29 & SLR4 &  exoplanet \\
Wasp-98b 		& \ra{03}{53}{42}{91}  \de{--34}{19}{41}{50}	& 13.00 & SLR3 &  exoplanet \\
PG1336-0118		& \ra{13}{38}{48}{15}  \de{--02}{01}{49}{10}	& 13.30 & SLR2 & close eclipsing binary with pulsating subdwarf component\\
RR Cae			& \ra{04}{21}{05}{56}  \de{--48}{39}{07}{02}	& 14.40 & SLR3 & white and red dwarf eclipsing binary with mass transfer \\
KZ Hya			& \ra{10}{50}{54}{08}  \de{--25}{21}{14}{71}	& 10.06 & SLR3 & short-period high amplitude pulsating variable \\
SOL-0023			& \multicolumn{2}{c}{undisclosed}				& SLR1	  & eclipsing binary \\					
J024946-3825.6	& \ra{02}{49}{45}{90}  \de{--38}{25}{36}{00}	& 13.30 & SLR1, SLR3, SLR4 & eclipsing binary, SOL-0132
\enddata
\label{tab:campaign}
\end{deluxetable*}

\subsubsection{Exoplanet transits}
In this section we present exoplanetary transits obtained using Solaris-1, Solaris-3 and Solaris-4 telescopes, covering all three sites of the network. Presented data sets have been reduced using the AstroImageJ (AiJ) data reduction package \citep{Collins2016}. AstroImageJ provides a very intuitive and user friendly graphical interface that greatly simplifies photometric data reduction. Its interactive capabilities are especially convenient for reducing data gathered in short observing runs. Transit fitting was done using the EXOFAST code\citep{Eastman2013, Eastman2010, Wright2014} - and IDL package for transit modeling that implements routines allowing simultaneous or separate light curve transit and radial velocity fitting. The biggest strength of EXOFAST, however, is its ability to characterize the parameter uncertainties using a differential evolution Markov chain Monte Carlo method. Exoplanet transits were selected using the Exoplanet Transit Database \citep{Brat2010} that also provided ephemerides and visibility information.

\paragraph{WASP-4b}
Wasp-4b is a hot Jupiter transiting a $V = 12.5$ mag star discovered by the SuperWASP-South observatory and CORALIE collaboration. The transit was recorded with the Solaris-1 telescope in SAAO on the evening of October 11th, 2015 in the $V$ band. The exposure time was fixed at 59 seconds. A master bias frame wes subtracted from our raw science images, then a median twilight flatfield frame was used to remove image inhomogenities. Aperture photometry was used with fixed star apertures and sky background annulus. Apertures were automatically recentered using the center-of-light method \citep{Howell2006} implemented in AiJ. Occasionally the aperture matching and recentering algorithm required manual intervention. The host star is reported to be a G7V main sequence star with $T_{\textrm{eff}} = 5500\pm150$K, log$g = 4.3\pm 0.2$, [M/H] = $0.0\pm0.2$ \citep{Wilson2008}. The remaining priors used as input for EXOFAST include period, inclination, RV semi-amplitude and semimajor axis - all taken from the aforecited discovery paper. The time of transit prior has been estimated using AiJ, a circular orbit was assumed. The period value was kept fixed since only one transit was observed. Photometric data, obtained model and residuals are plotted in Fig. \ref{fig:wasp-4}. The system's parameters along with errors are listed in Tab. \ref{tab:wasp-4}.  The fit's RMS is 3.6 mmag. \cite{Wilson2008} published RMS scatter values of 15.3, 2.7 and 1.8 mmag for data obtained with WASP-S, the 2.0-m Faulkes Telescope South and 1.2-m Euler Telescope, respectively. With a fully developed data reduction pipeline we expect even better photometric precision. 

\begin{figure}[htbp]
\centering
\includegraphics[width=\columnwidth]{img/SLR1_WASP-4b_V_EXO.eps}
\caption{Normalized and light curve data for WASP-4b with best fit transit model overlaid.}
\label{fig:wasp-4}
\end{figure}

\begin{deluxetable}{lcc}
%\tablewidth{0pt}
%\tabletypesize{\normalsize}
\tablecaption{Median values and 68\% confidence interval for Wasp-4b obtained with EXOFAST.}
\tablehead{\colhead{~~~Parameter} & \colhead{Units} & \colhead{Value}}
\startdata
\sidehead{Stellar Parameters:}
                           ~~~$M_{*}$\dotfill &Mass (\msun)\dotfill & $0.962_{-0.069}^{+0.073}$\\
                         ~~~$R_{*}$\dotfill &Radius (\rsun)\dotfill & $0.970_{-0.027}^{+0.030}$\\
                     ~~~$L_{*}$\dotfill &Luminosity (\lsun)\dotfill & $0.78_{-0.11}^{+0.12}$\\
                         ~~~$\rho_*$\dotfill &Density (cgs)\dotfill & $1.485_{-0.068}^{+0.069}$\\
              ~~~$\log(g_*)$\dotfill &Surface gravity (cgs)\dotfill & $4.447\pm0.017$\\
              ~~~$\teff$\dotfill &Effective temperature (K)\dotfill & $5510\pm150$\\
                              ~~~$\feh$\dotfill &Metalicity\dotfill & $-0.00\pm0.20$\\
\sidehead{Planetary Parameters:}
                              ~~~$P$\dotfill &Period (days)\dotfill & $1.3382281\pm0.0000030$\\
                       ~~~$a$\dotfill &Semi-major axis (AU)\dotfill & $0.02346_{-0.00057}^{+0.00058}$\\
                           ~~~$R_{P}$\dotfill &Radius (\rj)\dotfill & $1.515_{-0.052}^{+0.057}$\\
           ~~~$T_{eq}$\dotfill &Equilibrium Temperature (K)\dotfill & $1707\pm47$\\
               ~~~$\fave$\dotfill &Incident flux (\fluxcgs)\dotfill & $1.93_{-0.20}^{+0.22}$\\
\sidehead{Primary Transit Parameters:}
                ~~~$T_C$\dotfill &Time of transit (\bjdtdb)\dotfill & $2457307.34753\pm0.00023$\\
~~~$R_{P}/R_{*}$\dotfill &Radius of planet in stellar radii\dotfill & $0.1605\pm0.0033$\\
     ~~~$a/R_{*}$\dotfill &Semi-major axis in stellar radii\dotfill & $5.198_{-0.080}^{+0.079}$\\
              ~~~$u_1$\dotfill &linear limb-darkening coeff\dotfill & $0.520_{-0.059}^{+0.063}$\\
           ~~~$u_2$\dotfill &quadratic limb-darkening coeff\dotfill & $0.222_{-0.056}^{+0.055}$\\
                      ~~~$i$\dotfill &Inclination (degrees)\dotfill & $88.50_{-1.0}^{+0.93}$\\
                           ~~~$b$\dotfill &Impact Parameter\dotfill & $0.136_{-0.084}^{+0.090}$\\
                         ~~~$\delta$\dotfill &Transit depth\dotfill & $0.0258\pm0.0011$\\
                ~~~$T_{FWHM}$\dotfill &FWHM duration (days)\dotfill & $0.0814_{-0.0014}^{+0.0013}$\\
          ~~~$\tau$\dotfill &Ingress/egress duration (days)\dotfill & $0.01355_{-0.00040}^{+0.00049}$\\
                 ~~~$T_{14}$\dotfill &Total duration (days)\dotfill & $0.0950\pm0.0014$\\
      ~~~$P_{T}$\dotfill &A priori non-grazing transit prob\dotfill & $0.1615_{-0.0025}^{+0.0026}$\\
                ~~~$P_{T,G}$\dotfill &A priori transit prob\dotfill & $0.2233_{-0.0034}^{+0.0035}$\\
                            ~~~$F_0$\dotfill &Baseline flux\dotfill & $1.4802\pm0.0012$\\
\sidehead{Secondary Eclipse Parameters:}
              ~~~$T_{S}$\dotfill &Time of eclipse (\bjdtdb)\dotfill & $2457308.01664_{-0.00023}^{+0.00024}$
\enddata
\label{tab:wasp-4}
\end{deluxetable}


\paragraph{WASP-64b}
The discovery of Wasp-64b has been reported by \cite{Gillon2013}. It is a $1.271\rj$  and $1.271\mj$ giant planet in a short ($a=0.02648$ AU, $P = 1.5732918$ d) orbit around a $V = 12.3$ mag G7-type dwarf. The transit has been recorded using the Solaris-4 telescope in CASLEO on December 10th, 2015. 313 frames were acquired in the $V$-band with varying auto-adjusted exposure time between 20 and 40 seconds. Priors for EXOFAST were taken from the aforecited discovery paper. Photometric data, obtained model and residuals are plotted in Fig. \ref{fig:wasp-64}. The system's parameters along with errors are listed in Tab. \ref{tab:wasp-64}. The fit's RMS is 2.6 mmag. 

\begin{figure}[htb!]
\centering
\includegraphics[width=\columnwidth]{img/SLR4_WASP-64b_V_EXO.eps}
\caption{Normalized and light curve data for Wasp-64b with best fit transit model overlaid.}
\label{fig:wasp-64}
\end{figure}

\begin{deluxetable}{lcc}
%\tablewidth{0pt}
%\tabletypesize{\scriptsize}
\tablecaption{Median values and 68\% confidence interval for Wasp-64b obtained with EXOFAST.}
\tablehead{\colhead{~~~Parameter} & \colhead{Units} & \colhead{Value}}
\startdata
\sidehead{Stellar Parameters:}
                           ~~~$M_{*}$\dotfill &Mass (\msun)\dotfill & $0.959_{-0.061}^{+0.063}$\\
                         ~~~$R_{*}$\dotfill &Radius (\rsun)\dotfill & $1.039_{-0.044}^{+0.045}$\\
                     ~~~$L_{*}$\dotfill &Luminosity (\lsun)\dotfill & $0.89_{-0.13}^{+0.15}$\\
                         ~~~$\rho_*$\dotfill &Density (cgs)\dotfill & $1.21_{-0.12}^{+0.13}$\\
              ~~~$\log(g_*)$\dotfill &Surface gravity (cgs)\dotfill & $4.387\pm0.030$\\
              ~~~$\teff$\dotfill &Effective temperature (K)\dotfill & $5500\pm150$\\
                              ~~~$\feh$\dotfill &Metalicity\dotfill & $-0.08\pm0.11$\\
\sidehead{Planetary Parameters:}
                              ~~~$P$\dotfill &Period (days)\dotfill & $1.5732917\pm0.0000015$\\
                       ~~~$a$\dotfill &Semi-major axis (AU)\dotfill & $0.02610\pm0.00056$\\
                           ~~~$R_{P}$\dotfill &Radius (\rj)\dotfill & $1.15_{-0.15}^{+0.14}$\\
           ~~~$T_{eq}$\dotfill &Equilibrium Temperature (K)\dotfill & $1672_{-53}^{+55}$\\
               ~~~$\fave$\dotfill &Incident flux (\fluxcgs)\dotfill & $1.77_{-0.21}^{+0.24}$\\
\sidehead{Primary Transit Parameters:}
                ~~~$T_C$\dotfill &Time of transit (\bjdtdb)\dotfill & $2457366.71230_{-0.00100}^{+0.00095}$\\
~~~$R_{P}/R_{*}$\dotfill &Radius of planet in stellar radii\dotfill & $0.114_{-0.015}^{+0.013}$\\
     ~~~$a/R_{*}$\dotfill &Semi-major axis in stellar radii\dotfill & $5.40_{-0.18}^{+0.19}$\\
              ~~~$u_1$\dotfill &linear limb-darkening coeff\dotfill & $0.511_{-0.062}^{+0.063}$\\
           ~~~$u_2$\dotfill &quadratic limb-darkening coeff\dotfill & $0.226_{-0.055}^{+0.053}$\\
                      ~~~$i$\dotfill &Inclination (degrees)\dotfill & $86.569\pm0.097$\\
                           ~~~$b$\dotfill &Impact Parameter\dotfill & $0.323_{-0.014}^{+0.015}$\\
                         ~~~$\delta$\dotfill &Transit depth\dotfill & $0.0129_{-0.0031}^{+0.0030}$\\
                ~~~$T_{FWHM}$\dotfill &FWHM duration (days)\dotfill & $0.0882_{-0.0034}^{+0.0036}$\\
          ~~~$\tau$\dotfill &Ingress/egress duration (days)\dotfill & $0.0113_{-0.0015}^{+0.0013}$\\
                 ~~~$T_{14}$\dotfill &Total duration (days)\dotfill & $0.0995\pm0.0040$\\
      ~~~$P_{T}$\dotfill &A priori non-grazing transit prob\dotfill & $0.1642_{-0.0061}^{+0.0065}$\\
                ~~~$P_{T,G}$\dotfill &A priori transit prob\dotfill & $0.2060\pm0.0075$\\
                            ~~~$F_0$\dotfill &Baseline flux\dotfill & $0.20836_{-0.00044}^{+0.00041}$\\
\sidehead{Secondary Eclipse Parameters:}
              ~~~$T_{S}$\dotfill &Time of eclipse (\bjdtdb)\dotfill & $2457367.49895_{-0.00100}^{+0.00095}$
\enddata
\label{tab:wasp-64}
\end{deluxetable}


\paragraph{Wasp-98b}

Wasp-98b was discovered by \cite{Hellier2014}. It is a $0.83\mj$ hot Jupier orbiting a G7-type star. Photometric data, obtained model and residuals are plotted in Fig. \ref{fig:wasp-98}. The system's parameters along with errors are listed in Tab. \ref{tab:wasp-98}.  The fit's RMS is  2.9 mmag. 

\begin{figure}[htb!]
\centering
\includegraphics[width=\columnwidth]{img/SLR3_WASP-98b_V_EXO.eps}
\caption{Normalized and light curve data for Wasp-98b with best fit transit model overlaid.}
\label{fig:wasp-98}
\end{figure}

\begin{deluxetable}{lcc}
%\tablewidth{0pt}
%\tabletypesize{\scriptsize}
\tablecaption{Median values and 68\% confidence interval for Wasp-98b obtained with EXOFAST.}
\tablehead{\colhead{~~~Parameter} & \colhead{Units} & \colhead{Value}}
\startdata
\sidehead{Stellar Parameters:}
                           ~~~$M_{*}$\dotfill &Mass (\msun)\dotfill & $0.840_{-0.050}^{+0.055}$\\
                         ~~~$R_{*}$\dotfill &Radius (\rsun)\dotfill & $0.754\pm0.020$\\
                     ~~~$L_{*}$\dotfill &Luminosity (\lsun)\dotfill & $0.488_{-0.061}^{+0.069}$\\
                         ~~~$\rho_*$\dotfill &Density (cgs)\dotfill & $2.77_{-0.14}^{+0.15}$\\
              ~~~$\log(g_*)$\dotfill &Surface gravity (cgs)\dotfill & $4.609\pm0.018$\\
              ~~~$\teff$\dotfill &Effective temperature (K)\dotfill & $5560\pm140$\\
                              ~~~$\feh$\dotfill &Metalicity\dotfill & $-0.601_{-0.10}^{+0.098}$\\
\sidehead{Planetary Parameters:}
                              ~~~$P$\dotfill &Period (days)\dotfill & $2.9626401_{-0.0000014}^{+0.0000013}$\\
                       ~~~$a$\dotfill &Semi-major axis (AU)\dotfill & $0.03808_{-0.00078}^{+0.00082}$\\
                           ~~~$R_{P}$\dotfill &Radius (\rj)\dotfill & $1.199\pm0.032$\\
           ~~~$T_{eq}$\dotfill &Equilibrium Temperature (K)\dotfill & $1193\pm31$\\
               ~~~$\fave$\dotfill &Incident flux (\fluxcgs)\dotfill & $0.460_{-0.046}^{+0.050}$\\
\sidehead{Primary Transit Parameters:}
                ~~~$T_C$\dotfill &Time of transit (\bjdtdb)\dotfill & $2457305.13972\pm0.00049$\\
~~~$R_{P}/R_{*}$\dotfill &Radius of planet in stellar radii\dotfill & $0.163476_{-0.000100}^{+0.000097}$\\
     ~~~$a/R_{*}$\dotfill &Semi-major axis in stellar radii\dotfill & $10.87\pm0.19$\\
              ~~~$u_1$\dotfill &linear limb-darkening coeff\dotfill & $0.410_{-0.056}^{+0.060}$\\
           ~~~$u_2$\dotfill &quadratic limb-darkening coeff\dotfill & $0.266_{-0.054}^{+0.052}$\\
                      ~~~$i$\dotfill &Inclination (degrees)\dotfill & $86.324_{-0.091}^{+0.092}$\\
                           ~~~$b$\dotfill &Impact Parameter\dotfill & $0.697_{-0.012}^{+0.011}$\\
                         ~~~$\delta$\dotfill &Transit depth\dotfill & $0.026724_{-0.000033}^{+0.000032}$\\
                ~~~$T_{FWHM}$\dotfill &FWHM duration (days)\dotfill & $0.0607\pm0.0015$\\
          ~~~$\tau$\dotfill &Ingress/egress duration (days)\dotfill & $0.02043_{-0.00054}^{+0.00055}$\\
                 ~~~$T_{14}$\dotfill &Total duration (days)\dotfill & $0.0811\pm0.0016$\\
      ~~~$P_{T}$\dotfill &A priori non-grazing transit prob\dotfill & $0.0769_{-0.0013}^{+0.0014}$\\
                ~~~$P_{T,G}$\dotfill &A priori transit prob\dotfill & $0.1070\pm0.0019$\\
                            ~~~$F_0$\dotfill &Baseline flux\dotfill & $1.18441\pm0.00054$\\
\sidehead{Secondary Eclipse Parameters:}
              ~~~$T_{S}$\dotfill &Time of eclipse (\bjdtdb)\dotfill & $2457306.62104\pm0.00049$
\enddata
\label{tab:wasp-98}
\end{deluxetable}	

\subsubsection{Timing targets}

The main goal of Project Solaris is to monitor and detect timing variations of eclipsing binaries. In this section we present photometric measurements of eclipsing and pulsating binaries and demonstrate the achievable photometric precision. 
\paragraph{KZ Hya}

KZ Hya or HD94033 is a SX Phoenicis type pulsating variable. Sx Phe stars are metal-poor Type II population variables that pulsate with periods between 1.0 and 1.75 h. 
\citep{McNamara1995}. These pulsations can act as a convenient beacon and reveal additional information about the pulsator and its possible companion. Indeed, \cite{Kim2007} show that an 24.77 year eccentric ($e = 0.25$) orbit is visible in the O--C data. Assuming that the primary has a mass of 0.9\msun, the minimum mass of the unseen companion is  0.66\msun$\sin i$. This target has been selected for a follow-up campaign. At this time we present an example V-band light curve of a single epoch obtained with the Solaris-3 telescope -- Fig. \ref{fig:KZHya}. Frequencies from \cite{Kim2007} have been fitted to the data. Residuals have an rms of 5 mmag, but the short term precision is 1 mmag. KZ Hya will be investigated in much more detail in an upcoming publication.

\begin{figure}[htb!]
\center
\includegraphics[width=\columnwidth]{img/plot_lc_KzHya1.eps}
\caption{KZ Hya, V-band data for one epoch with (top panel). Residuals after fitting nine frequencies from \cite{Kim2007}.}
\label{fig:KZHya}
\end{figure}


\paragraph{RR Cae}

RR Cae is a $V=14.4$ mag short period dwarf-M-dwarf eclipsing binary. \cite{Maxted2007} have shown that no meaningful O--C variations are present in the data spanning 10 years. A more recent study by \cite{Qian2012}, however, reveals a periodic signal in O--C measurements. After ruling out several other effects that might have been the reason for the 11.9 year variations, the authors conclude that a circumbinary planet is responsible for the detected signal. RR Cae is therefore an interesting target that is present in our long-term timing campaign. An example light curve around the primary eclipse is shown in Fig. \ref{fig:RRCae}.

\begin{figure}[htb!]
\centering
\includegraphics[width=\columnwidth]{img/plot_lc_RRCaeli.eps}
\caption{RR Cae primary eclipse recorded in the I band. Data has been phased with $P=0.303704$ d and spans 62 days from March 11th, 2015.}
\label{fig:RRCae}
\end{figure}


\paragraph{SOL-0023}

SOL-0023 is a bright, $\sim 2.5$ d eclipsing binary that is a very promising target in our eclipse timing campaign. Several mmag precision is achievable regularly with the Solaris telescopes. A sample light curve centered around the primary eclipse is shown in Fig. \ref{fig:SOL_0023}.

\begin{figure}[htb!]
\centering
\includegraphics[width=\columnwidth]{img/plot_lc_SOL-0023.eps}
\caption{SOL-0023 phased V-band light curve around the primary eclipse used for eclipse timing analysis.}
\label{fig:SOL_0023}
\end{figure}

\subsubsection{Eclipsing binary with a pulsator}

\paragraph{PG 1336-018}

PG 1336-018 (NY Vir) is a $V=13.3$ mag eclipsing B-type subdwarf (sdB). With $P=0.1010174$ d it is one of the shortest-period eclipsing binary known. Additionally, the system shows pulsations of the type found in sdB pulsators. \cite{Hu2007} present a detailed evolutionary study of the system. Its history is complex: in the past the system underwent the common-envelope stage with a companion of unknown type; after mass transfer, the system evolved into what we observe now - a binary pulsating star with a M-type dwarf companion. \cite{Kilkenny1998} studied this binary initially and identified pulsations with periods 184 s and 141 s and semi-amplitudes 10 and 5 mmag, respectively. Data was collected with the University of Cape Town (UCT) 1-m telescope using a CCD and photometer. The authors expected additional pulsation frequencies with semi-amplitudes below 3 mmag. These have been identified in the follow-up study \citep{Kilkenny2003} that was based on a multi-site (Whole Earth Telescope) observing campaign. Authors detected and identified 28 frequencies down to the semi-amplitude of 0.5 mmag. The most recent study by \cite{Vuckovic2007} presents a complete astrophysical model of the system based on photometric and spectroscopic data obtained with the VLT (UVES and ULTRACAM). PG 1336-016 was observed with the Solaris-2 telescope on the night of April 24th 2015. 650 20-second exposures were acquired without any filter. The decision not to use filters was made in order to increase photon counts and keep the exposure time at 20 s to obtain good temporal resolution and satisfactory SNR of the target and comparison stars. Frames were reduced with AiJ using a similar approach as for exoplanetary transits. Relative fluxes obtained with AiJ were loaded into \textsc{phoebe} \citep{Prsa2005} and \textsc{jktebop} \citep{sou04a,sou04b}, where an initial model was fitted. Priors for physical parameters were adopted from \cite{Vuckovic2007}. Once a satisfactory fit was obtained, residuals were analyzed with the Period04 software package \citep{Lenz2005}. Several frequencies have been identified in the data (Fig. \ref{fig:24hFourier}). Monte Carlo analysis was performed to obtain uncertainties of the frequencies and corresponding semiaamplitudes. Table \ref{tab:24hfrequencies} lists four frequencies with highest semiamplitudes that have been used to remove oscillations from the data set. Figure \ref{fig:24hperiodogram} illustrates a 30-minute long set of data points with the fitted oscillation model. Once cleaned, a second iteration fit was done with \textsc{phoebe} and \textsc{jktebop} to obtain a final set of parameters. Uncertainties of parameters have been computed using \textsc{jktabsdim}. The resulting light curve and model are shown in Fig. \ref{fig:24h}.
This target has been chosen to demonstrate the capabilities of the network to acquire high cadence for asteroseismology. The obtained astrophysical model, constructed primarily based on new data with priors from the literature, is in agreement with what can be found in previous works.


\begin{figure}[htb!]
\centering
\includegraphics[width=\columnwidth]{img/PG1336_Fourier.eps}
\includegraphics[width=\columnwidth]{img/PG1336_Fourier_close.eps}
\caption{PG1336-0118 frequency analysis. Top panel shows amplitude spectrum in the entire computed range. Bottom panel shows more detail in the 5.5 mHz frequency range.}
\label{fig:24hFourier}
\end{figure}


\begin{figure}[htb!]
\centering
\includegraphics[width=\columnwidth]{img/PG1336_pulse.eps}
\caption{Part of residuals of PG1336-018 and fitted oscillation model with four identified frequencies.}
\label{fig:24hperiodogram}
\end{figure}


\begin{figure*}[htb!]
\centering
\includegraphics[width=\textwidth]{img/PG1336_all.eps}
\caption{PG1336-0118 Solaris-2 light curve and models. The first panel shows the observed magnitude and the initial eclipsing binary model obtained with PHOEBE. The second panel shows residuals of the above fit. Points have been connected with lines to better visualize pulsations. The third panel shows the data from the first panel with pulsations removed based on the frequencies, amplitudes and phases identified during the analysis of the residuals. An updated model is over-plotted on the data. Residuals of this model are shown in the last panel. Visible trends are caused by imperfect modeling of the reflection effects.}
\label{fig:24h}
\end{figure*}


\begin{deluxetable*}{cccccc}
\tablewidth{0pt}
\tabletypesize{\scriptsize}
\tablecaption{Four frequencies identified in the residuals' spectrum of PG1336-018. Obtained values are the result of 1000 Monte Carlo simulations computed using Period04.}
\tablehead{\colhead{F\#} & \colhead{$f$ ($\mu$ Hz)} & \colhead{$1/f (s)$} & \colhead{$\sigma f$ ($\mu$ Hz)} & \colhead{semiampl. (mmag)} &  \colhead{$\sigma$ semiampl. (mmag)} }
\startdata
F1	& 5436.2	 	& 183.95	& 0.6 		& 21.7 		& 0.4\\
F2	& 5540		& 180.50		& 6		& 2.1		& 0.4\\
F3	& 5894		& 169.66		& 6		& 2.1		& 0.4\\
F4	& 6719		& 148.83		& 5  		& 2.1 	& 0.4
\enddata
\label{tab:24hfrequencies}
\end{deluxetable*}


\begin{deluxetable}{rlrr}
\tablewidth{0pt}

\tablecaption{Solutions for eclipsing binaries. Formal errors are noted directly under the parameter value.}
%For effective temperatures ($T_1$ and $T_2$) only the error for $T_2$ is given, since $T_1$ is fixed in the fitting process.
\tablehead{\colhead{Parameter}	&	\colhead{Unit}	&	\colhead{PG1336-018}	&	\colhead{J024946-3825.6}}
\startdata	
	$T_0$	&	(JD)	&	2457140.323152	&	2457353.37820 	\\ 		 \vspace{0.1cm}
		&		&	0.000008	&		0.00004 \\		
	$P$	&	(d)	&	0.101004	&	0.463220	\\		 \vspace{0.1cm}
		&		&	 0.000012	&	0.000022	\\		
	$K_1$	&	(km s$^{-1}$) 	&	78.6	&	124.3	\\		 \vspace{0.1cm}
		&		&	0.6	&	2.1	\\		
	$K_2$	&	(km s$^{-1}$) 	&	300	&	162.3	\\		 \vspace{0.1cm}
		&		&	2	&	2.6	\\		
	$e$	&		&	0	&	0	\\		 \vspace{0.1cm}
		&		&	-	&	-	\\		
	$i$	&	(deg.)	&	77.88	&	81.0	\\		 \vspace{0.1cm}
		&		&	0.27	&	1.0	\\		
	$a$	&	(R$_\sun$)	&	 0.461	&	2.66	\\		 \vspace{0.1cm}
		&		&	 0.006 	&	0.03	\\		
	$\omega$	&	(\arcdeg)	&	-	&	-	\\		 \vspace{0.1cm}
		&		&	-	&	-	\\		
	$v_\gamma$	&	(km s$^{-1}$) 	&	-	&	21.4	\\		 \vspace{0.1cm}
		&		&	-	&	0.5	\\		
	RMS RV1	&	(km s$^{-1}$) 	&	-	&	2.2	\\ 		 \vspace{0.1cm}
	RMS RV2	&	(km s$^{-1}$) 	&	-	&	1.4	\\		
	$T_1$	&	(K)	&	32850	&	4100	\\		 \vspace{0.1cm}
		&		&	-	&	350	\\		
	$T_2$	&	(K)	&	3100	&	3475	\\		 \vspace{0.1cm}
	~	&		&	-	&	350	\\		
%	Mratio	&		&		&		\\		
%		&		&		&		\\		
	$M_1$	&	(M$_\sun$)	&	0.468	&	0.664	\\		 \vspace{0.1cm}
		&		&	0.008	&	0.025	\\		
	$M_2$	&	(M$_\sun$)	&	0.123	&	0.509	\\		 \vspace{0.1cm}
		&		&	0.002	&	0.019	\\		
	$R_1$	&	(R$_\sun$)	&	0.1448	&	0.590	\\		 \vspace{0.1cm}
		&		&	0.0021	&	0.027	\\		
	$R_2$	&	(R$_\sun$)	&	0.1543	&	0.518	\\		 \vspace{0.1cm}
		&		&	0.0022	&	0.027	\\		
	log $g_1$	&	(cm s$^{-1}$)	&	5.787	&	4.72	\\		 \vspace{0.1cm}
		&		&	0.012	&	0.04	\\		
	log $g_2$	&	(cm s$^{-1}$)	&	5.150	&	4.72	\\		 \vspace{0.1cm}
		&		&	 0.012	&	0.04	\\		
	$v_{synchr, 1}$	&	(km s$^{-1}$) 	&	72.5	&	64	\\		 \vspace{0.1cm}
		&		&	1.0	&	3	\\		
	$v_{synchr, 2}$	&	(km s$^{-1}$) 	&	 77.3	&	56.6	\\		 \vspace{0.1cm}
		&		&	1.1	&	3.0	\\		
	log $L_1$	&	(L$_\sun$)	&	1.343	&	-1.05	\\		 \vspace{0.1cm}
		&		&	0.015	&	 0.15	\\		
	log $L_2$	&	(L$_\sun$)	&	-2.70	&	-1.45	\\		 \vspace{0.1cm}
		&		&	0.09	&	0.18	\\		
	$M_{bol,1}$	&	(mag)	&	1.392	&	7.4	\\		 \vspace{0.1cm}
		&		&	0.038	&	0.4	\\		
	$M_{bol,2}$	&	(mag)	&	11.51	&	 8.4	\\		 \vspace{0.1cm}
		&		&	 0.23	&	0.5	\\		

\enddata
\label{tab:ModelsSolutions}
\end{deluxetable}

\subsubsection{Complete model - J024946-3825.6}

J024946-3825.6 or SOL-0132, also known as 1RXS\footnote{1-st ROSAT X Survey.} J024946.0-382540, is a $V=11.7$ mag eclipsing detached binary from the ASAS catalog with an orbital period of 0.46322 days. It has been observed photometrically with SLR1, SLR3 and SLR4 telescopes. Spectra were obtained with the SMARTS 1.5m telescope and the Chiron spectrograph at the Cerro Tololo Inter-American Observatory in Chile. Apart from entries in stellar catalogs, this binary is not present in the literature. Herein we present a full model of this binary star based on photometry and spectroscopy that has been derived using a range of custom and publicly available software packages. 

\textbf{Photometry.} Photometric data has been collected during a long $\sim$55h run between November 27th and 29th, 2015 with the Solaris telescopes demonstrating the power of a global telescope network. Two gaps that have been caused by unfavorable weather conditions are present in the data set. The end of November is the time where the gap in the network coverage reaches its peak value (Fig. \ref{fig:NetworkCoverage}), hence it is not the optimal time to conduct campaigns requiring continuous coverage. Despite this, however,  we have been able to observe three primary and four secondary eclipses in just 2.3 days, an equivalent of $\sim$5 full orbital periods of the system. Altogether, multicolor photometry includes $\sim$3500 observations in V and I bands. A sample of I-band data reduced with AiJ is shown in Fig. \ref{fig:J024946_55h}. For modeling purposes I and V band data has been reduced using our custom pipeline and translated so that the levels agree with SLR3.

  
\begin{figure*}[htb!]
\includegraphics[width=\textwidth]{img/SOL-0132_55h_new.eps}
\caption{J024946 55-h observing campaign with the Solaris telescope network: Solaris-1 (orange), Solaris-3 (blue) and Solaris-4 (green), I band - raw output from AiJ.}
\label{fig:J024946_55h}
\end{figure*}

 \textbf{Spectroscopy}. 25 echelle spectra have been acquired between September 26th and October 22nd, 2015. Data has been reduced using \textsc{iraf}. RVs were computed using two methods: fitting a Gaussian profile to H-alpha emission for both stars and one-dimensional cross-correlation for the primary component (absorption lines from the secondary component are practically invisible in the spectra). An initial orbital fit was obtained using our custom \textsc{v2fit} code \citep{Konacki2010} using RVs obtained with both methods. This allowed us to verify that both methods lead to agreeing RV values. Errors of individual measurements were increased in quadrature during the fitting process to get $\chi^2\approx1$, a procedure that guarantees that values of parameters' errors are not underestimated. Then, photometric data along with RVs and initial orbital parameters were loaded into \textsc{phoebe}.


 \textbf{Modeling.} \textsc{phoebe} allows one to simultaneously fit photometric and spectroscopic data and, among many features, can also work with stellar spots. The physics behind the stellar spot functionality is rather simple -- the surface of the star that features a spot has lower (or higher) emergent intensity than spot-less parts of the star. The shape of J024946's light curve immediately reveals that stellar spots are present on at least one star. Moreover, the system shows activity in the X-ray domain and emission in the spectral region of H-$\alpha$ line, confirming that stellar spots are indeed very likely to be present. Following this logic, we have derived a model of J024946 that mimics the stellar spot with one large spot on the primary component.

Since the influence of a secondary component on the spectra is barely noticeable we have not succeeded in spectral disentangling and applied spectral analysis for co-added spectra to obtain atmospheric parameters of primary component. For this purpose, we used the software package \textit{Spectroscopy Made Easy} \citep[hereafter SME,][]{val96, val98}. Since line blending becomes more severe in the blue, what makes continuum placement and thus derived parameters less accurate \citep{val05}, we analyzed only 6 orders of the co-added spectrum which cover the wavelength region from 5992 to 6358 \AA. The list obtained from the Vienna Atomic Line Database \citep[VALD,][]{pis95, kup99} was used as atomic line data with the initial values described by \citet{val05} and \citet{kur92} model atmospheres. We used the value of $\log g$ derived from \textsc{phoebe} analysis and fixed it during spectral analysis, fitting $T_{\mathrm{eff}}$, $[M/H]$ and $v\sin i$ for every spectral order. Final values of $T_{\mathrm{eff}}$, $[M/H]$ were calculated as a mean value of the results from separate orders. The variance between orders was treated as the uncertainty of every parameter. The value of $T_{\mathrm{eff1}}$ obtained from spectral analysis was then fixed in \textsc{phoebe}, which was used in order to calculate $T_{\mathrm{eff2}}$. The resulting models are shown in Fig. \ref{fig:J024946_models}.

Absolute values of stellar and orbital parameters with its uncertainties were calculated with \textsc{jktabsdim} \citep{sou04a, sou04b} and are presented in Tab. \ref{tab:ModelsSolutions}. Since the value of \textit{v sin i} from spectral analysis was consistent with the value of rotational velocity calculated under the assumption of tidal locking $v_{syn}$ we adopted the latter value as a final one.

Yonsei-Yale \citep[hereafter YY, ][]{yi01} tracks and isochrones were used in order to check evolutionary status and age of the system. Evolutionary tracks calculated for given masses and the value of metallicity derived from spectral analysis ($[M/H] = -0.4$) are presented in Fig. \ref{fig:J024946_tracks} and imply that J024946-3825.6 is a main-sequence system.  Age estimation was based on fitting the isochrones of given metallicity to the observational data using three relationships: $M_\mathrm{bol}$ - mass, $\log T_\mathrm{eff}$ - mass, $\log g$ - mass and is presented in Fig. \ref{fig:J024946_iso}. As can be seen from the presented plots, stellar evolution models fail to reproduce properties like temperature and radius (thus $\log g$) obtained from our analysis. It is related to the known issue of low-mass stars discrepancy with stellar evolution models \citep[e.g.,][]{chab07,mor09, hel11} - objects are larger and cooler than expected. Studies \citep{fei12} show that model radii under-predict observed values up to a dozen percent. However model calculations are in good agreement with the observational mass - luminosity (thus $M_\mathrm{bol}$) relationship. Assuming that studied stars are 10$\%$ larger than model predictions, we re-calculated stellar parameters for the values of radius of $R_1-10\%$ and $R_2-10\%$ and corresponding effective temperatures (keeping constant value of luminosity) and presented it in Fig. \ref{fig:J024946_iso} using red color. It is clearly seen that the new values reproduce stellar models and the system age is $\sim$ 7 Gyr. 
  
\begin{figure}[htb!]
\centering
\includegraphics[angle=-90,width=\columnwidth]{img/SOL0132_track_4100_incl.eps}
\caption{YY evolutionary tracks for components of J024946-3825.6. Blue color represents a track calculated for primary component mass and system metallicity $[M/H]$ = -0.4, while red - secondary�s track.}
\label{fig:J024946_tracks}
\end{figure}

\begin{figure}[htb!]
\centering
\includegraphics[width=\columnwidth]{img/SOL0132_iso_kompil_4100_incl_m10proc_pTemp.eps}
\caption{YY isochrones for components of J024946-3825.6 calculated assuming system metallicity of $[M/H]$ = -0.4. Black symbols represent initial values of $log g$ and $T_{eff}$, red - recalculated after decreasing stellar radii (see text).}
\label{fig:J024946_iso}
\end{figure}

\begin{figure}[htb!]
\centering
\includegraphics[width=\columnwidth,clip=true]{img/plot_lc_V_SOL-0132.eps}
\includegraphics[width=\columnwidth]{img/plot_lc_I_SOL-0132.eps}
\includegraphics[width=\columnwidth]{img/plot_rv_SOL-0132.eps}
\caption{J024946-3825.6 photometric (V and I bands) and spectroscopic data - measurements, models and residuals.}
\label{fig:J024946_models}
\end{figure}



\begin{figure}[htb!]
\centering
\includegraphics[width=\columnwidth]{img/wasp-4-avar.eps}
\includegraphics[width=\columnwidth]{img/wasp-64-avar.eps}
\includegraphics[width=\columnwidth]{img/wasp-98-avar.eps}
\caption{Allan variance plots of transit residuals.}
\label{fig:transits-avar}
\end{figure}

\begin{figure}[htb!]
\centering
\includegraphics[width=\columnwidth]{img/SOL-0023-avar.eps}
\caption{Allan variance plots of SOL-0023, one of the timing targets.}
\label{fig:sol-0023-avar}
\end{figure}

\subsubsection{Photometric precision} 
In the previous sections we have demonstrated concrete evidence that proves the capability of the Solaris Network to acquire high quality data that can be used to tangle astrophysical problems of that require photometric measurements. Exoplanetary transits reduced with AiJ served as an initial test of the individual capabilities of the telescopes. Using off-the-shelf software allowed us to eliminate potential data reduction errors and prepare a benchmark for testing a dedicated pipeline. We follow an approach that is  similar to what has bee presented in \cite{Swift2015}, where the Minerva system's capabilities are benchmarked with the help of transits. The remaining targets were used to further demonstrate the precision achievable with the network. One of the best ways to quantify the photometric quality is to analyze the scatter of the O-C values that are a product of model fitting. This has been done for the following systems: Wasp-4b, Wasp-64b, Wasp-98b, SOL-0132 (V and I bands, three sites), PG-1338 and SOL-0023. The results are presented in Tab. \ref{tab:precision} and represent a wide range of cadence values from 8 to 500 s. In case of transits the fit RMS' values are comparable with the formal photometric errors that were computed during the data reduction process. Similarly, SOL-0023 shows good accordance between these values. It is worth noting that the formal errors themselves have small scatter. SOL-0023's data has been reduced using our dedicated pipeline. In case of SOL-0132 and and PG-1336 the average formal errors are smaller than the fit's RMS values. This has its origin in the modeling process. In both cases astrophysical models are computed and these have their limitations. In this case, however, the obtained astrophysical parameters and their formal errors serve as the measure of the data quality. It is important to measure and understand the sources of photometric errors but the end product is the key factor in quantifying the strength of the hardware and software medley. The precision in mass determination is 1.6 and 1.7 \% for PG-1336, 3.8 and  3.7 \% for SOL-0132. Radial velocity measurements' quality is the main factor, but for stellar radii, it is photometry that plays the crucial role. We obtain 1.4 \% precision for radii of the PG1336 components 4.5 and 5.2\% for SOL-0132. The latter is a very interesting case and has been described in detail in the previous section.


Allan variance has been computed for transits and SOL-0023. The plots are presented in Figs \ref{fig:transits-avar}-\ref{fig:sol-0023-avar}. White noise equivalents with appropriate standard deviation values are over-plotted to show the main components of the O-C scatter. In case of transits, 1 mmag precision is achievable in 30-minute timescales, in case of SOL-0023 this timescale is equivalent to 0.5 mmag precision. The last case is particularly interesting in terms of atmospheric scintillation due to the short 2.5-6 s exposure times. Based on the formula provided by \cite{Osborn2015}, the scintillation component accounts for 4.7 to 3.0 mmag of the final photometric residuals. 
The goal of the project is to precisely characterize a limited number of eclipsing binaries, predominately for eclipse timing. The pool of targets is ca. 300 binaries. During most nights no more than 1-2 targets (i.e. fields) per night are scheduled on a particular telescope. For this reasons, the performance of the network telescopes can be best tested on specific, know from literature targets that are typically characterized with high cadence (or relatively high), high precision observations (transits, pulsations, eclipsing binaries). The nature of photometric data reduction from systems such as Solaris (narrow field of view, 13-21 arcmin) is very specific and significantly different from wide-angle survey-type systems. Targets have been carefully chosen so that comparison stars are present in the field. Very often, target fields required an offset to avoid bright stars being saturated and at the same time move good comparison stars into the field. Hence the need for online astrometry and constant monitoring of the field. Solaris telescopes are high-end optical systems that produce flat fields, PSFs are well sampled with a high-grade CCD camera, very low read noise and negligible dark current. Exposure times for most targets need to updated online based on seeing conditions to react to changes in the observing conditions, keep the pixel counts in the desired range and avoid overexposure of field stars. These intrinsic properties of the Solaris systems influence the way data is handled and also scheduled. Wide-field systems usually have an abundance of comparison stars that are evenly spread across the field. This is not the case for Solaris. The final photometric errors strongly depend on the field and exposure time and, as demonstrated, are in the low mmag range and are consistent with modeling results. 


\begin{deluxetable}{lcccc}
\tablewidth{0pt}
\tabletypesize{\scriptsize}								
\tablecaption{Formal photometric errors and fit RMS values for targets with computed models.}									
\tablehead{\colhead{Object} &\colhead{SLR} & \colhead{Cadence} & \colhead{Photometric error} & \colhead{Fit RMS} \\
				            &			     & \colhead{(s)}	     & \colhead{avg/min/max (mmag)}					  & \colhead{(mmag)}}								
\startdata									
Wasp-4b  		&1	& 137 		& 2.9 / 2.8 / 3.0  & 3.6  \\
Wasp-64b		&4	& 124 		& 2.1 / 1.7 / 3.1  & 2.6  \\
Wasp-98b		&3	& 229 		& 3.5 / 3.4 / 3.6  & 2.9  \\
SOL-0132 (V) 	&1	& 500		& 1.9 / 1.6 / 3.6  &        \\
			&3	& 46 - 56		& 5.7 / 4.3 / 8.6  & 8.7	   \\
			&4	& 140 - 155	& 2.6 / 1.7 / 3.3  &	    \\
SOL-0132 (I) 	&1	& 85 - 150		& 3.1 / 2.6 / 7.4  &        \\
			&3	& 26 - 31		& 10.0 / 7.4 / 18.0  & 10	   \\
			&4	& 140 - 155	& 4.9 / 3.3 / 8.2  &	    \\
PG-1336 		& 2	& 25			& 2.8 / 2.4 / 5.0 &  7.2 \\
SOL-0023		& 1	& 8			& 5.4 / 3.7 / 7.9 &  6.8
\enddata									
\label{tab:precision}									
\end{deluxetable}



%
%\begin{figure*}[htbp]
%\includegraphics[width=\textwidth]{img/SOL-0132_55h.eps}
%\caption{J024946-3825.6, 55-hour coverage with the Solaris Telescope Network.}
%\label{fig:24h}
%\end{figure*}
