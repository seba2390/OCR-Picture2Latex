\documentclass{article}[12pt]

\usepackage{amsmath}
\usepackage{amssymb}
\usepackage{amsthm}
\usepackage[pdftex]{hyperref} %normal processing
%\usepackage[backref]{hyperref} %show cited in refs
%\usepackage{refcheck} %show actual labels
%\nocite{*}


\newtheorem{theorem}{Theorem}
\newtheorem{lemma}[theorem]{Lemma}
\newtheorem{cor}[theorem]{Corollary}
\newtheorem{fact}[theorem]{Fact}
\newtheorem{claim}[theorem]{Claim}
\newtheorem{conj}[theorem]{Conjecture}
\newtheorem{definition}{Definition}[section]
\newtheorem{Prob}[theorem]{Problem}
\newtheorem{example}{Example}[section]
\newtheorem{remark}{Remark}
\newtheorem{obs}{Observation}

\title{\bf
A Note on Upper Bounds\\
for Some Generalized Folkman Numbers\footnote{
Supported by the National Natural Science
Foundation (11361008).}}

\author{Xiaodong Xu\\[-0.1ex]
\small Guangxi Academy of Sciences\\[-0.6ex]
\small Nanning 530007, P.R. China\\[-0.6ex]
\small {\tt xxdmaths@sina.com}\\[1.3ex]\and
Meilian Liang\\[-0.1ex]
\small School of Mathematics and Information Science\\[-0.6ex]
\small Guangxi University, Nanning 530004, P.R. China\\[-0.6ex]
\small {\tt gxulml@163.com}\\[1.3ex]\and
Stanis{\l}aw Radziszowski\\[-0.1ex]
\small Department of Computer Science\\[-0.6ex]
\small Rochester Institute of Technology, Rochester, NY 14623\\[-0.6ex]
\small {\tt spr@cs.rit.edu}\\[3.3ex]
}

\date{\today}

\begin{document}
\maketitle
\thispagestyle{empty}
   
\begin{abstract}
We present some new constructive upper bounds
based on product graphs
for generalized vertex Folkman numbers. They
lead to new upper bounds for some
special cases of generalized edge Folkman numbers,
including $F_e(K_3,K_4-e; K_5) \leq 27$ and
$F_e(K_4-e,K_4-e; K_5) \leq 51$. The latter
bound follows from a construction of a
$K_5$-free graph on 51 vertices, for which
every coloring of its edges with two colors
contains a monochromatic $K_4-e$.
\end{abstract}

\bigskip
\noindent
{\bf Keywords:} Folkman number, Ramsey number\\
{\bf AMS classification subjects:} 05C55, 05C35

\section{Folkman numbers} \label{Folkman}

Let $r, s, a_1, \cdots, a_r$ be integers such
that $r \ge 2$,
$s >  \max \{a_1, \cdots, a_r \}$ and
$\min \{a_1, \cdots, a_r \} \ge 2$.
We write $G \rightarrow (a_1 ,\cdots ,a_r)^v$
(resp. $G \rightarrow (a_1 ,\cdots ,a_r)^e$)
if for every
$r$-coloring of $V(G)$ (resp. $E(G)$), there exists
a monochromatic $K_{a_i}$ in $G$ for some color
$i \in \{1, \cdots, r\}$.
The Ramsey number $R(a_1, \cdots, a_r)$
is defined as the smallest integer $n$ such that
$K_n \rightarrow (a_1, \cdots, a_r)^e$.
The sets of vertex and edge Folkman graphs are defined as
$$\mathcal{F}_v(a_1, \cdots, a_r; s)=\{G \;|\; G \rightarrow
(a_1, \cdots, a_r)^v \textrm{ and } K_s \not\subseteq G\},\textrm{ and}$$
$$\mathcal{F}_e(a_1, \cdots, a_r;s)=\{G\;|\; G \rightarrow
(a_1, \cdots, a_r)^e \textrm{ and } K_s \not\subseteq G\},$$
respectively, and the vertex and edge Folkman numbers
are defined as the smallest orders of graphs in these sets,
namely
$$F_v(a_1, \cdots, a_r;s)=\min\{|V(G)|\;|\;G\in
\mathcal{F}_v(a_1, \cdots, a_r;s) \},\textrm{ and}$$
$$F_e(a_1, \cdots, a_r;s)=\min\{|V(G)|\;|\;G\in
\mathcal{F}_e(a_1, \cdots, a_r;s) \}.$$

The generalized vertex and edge Folkman numbers,
$F_v(H_1, \cdots, H_r;H)$ and $F_e(H_1, \cdots, H_r;H)$,
are defined analogously by considering arrowing graphs
$H_i$ while avoiding $H$, instead of arrowing complete
graphs $K_{a_i}$ while avoiding $K_s$.
The edge Folkman number
$F_e(a_1, \cdots, a_r; k)$ can be seen as a generalization
of the classical Ramsey number $R(a_1, \cdots, a_r)$,
since for $k > R(a_1, \cdots, a_r)$ we clearly have
$F_e(a_1, \cdots, a_r; k) = R(a_1, \cdots, a_r).$

\medskip
In 1970, Folkman \cite{Folkman} proved that
for positive integers $k$ and $a_1, \cdots, a_r$,
$F_v (a_1, \cdots, a_r; k)$  and $F_e (a_1, a_2; k)$
exist if and only if $k > \max \{a_1, \cdots$, $a_r\}$.
Folkman's method did not work for
edge colorings for more than two colors.
The existence of $F_e(a_1, \cdots, a_r; k)$
was proved by Ne\v{s}et\v{r}il and R\"{o}dl
in 1976 \cite{NesetrilRodl}.
Folkman numbers have been studied by many
other authors, in particular in
\cite{drr14,RT,Nkolev2008,LiLin2017a,
LinLi2012a,RodlRS,XuShao2010a,XLSW}.
The current authors studied chromatic variations
of Folkman numbers \cite{XLR0}, and some
existence questions for
$F_v(H_1, \cdots, H_r;H)$ and $F_e(H_1, \cdots, H_r;H)$
\cite{XLR1}.

Perhaps the most wanted Folkman number is $F_e(3,3;4)$,
for which the currently best known bounds are 20
\cite{BikovNenov2016a} and 785 (an unpublished
improvement from 786 obtained by Kauffman, Wickus and the third
author, for more information about the upper bound see
\cite{XLR1}). Further improvements of the
bounds on $F_e(3,3;4)$ seem very difficult,
but some insights can be made into similar
cases involving almost complete graphs $K_k-e$.

For vertex-disjoint graphs $G$ and $H$, their {\em join}
$G+H$ has the vertices $V(G) \cup V(H)$ and edges
$E(G) \cup E(H) \cup E(G,H)$, where $E(G,H)$ is the set
of all possible edges between $V(G)$ and $V(H)$.
Let us also denote $K_k-e$ by $J_k$.
In \cite{XLR1}, we proved
the existence of
$F_e(K_{k+1},K_{k+1};J_{k+2})$ and
$F_v(K_k,K_k;J_{k+1})$, for all $k \ge 3$.
In the same paper we discussed the existence
of some generalized Folkman numbers, especially
in the cases of the form $F_e(K_3,K_3;H)$ for some
small graphs $H$. The latter
includes proofs of nonexistence of the numbers
$F_e(K_3,K_3;J_4)$, $F_e(K_3,K_3;K_2+3K_1)$ and
$F_e(K_3,K_3;K_1+P_4)$, and poses some open cases,
like that for $F_e(K_3,K_3;K_1+C_4)$.

In Section 2 we overview some of the prior constructions
and related upper bounds, and we present our new
constructions. They lead to some new concrete upper bounds,
presented in Section 3, for some special cases including
$F_e(K_3,J_4; K_5) \leq 27$ and
$F_e(J_4,J_4; K_5) \leq 51$.

\input body1.tex
\input ref1.tex
\end{document}
