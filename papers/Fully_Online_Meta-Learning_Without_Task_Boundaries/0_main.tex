
\documentclass{article} % For LaTeX2e
\usepackage{iclr2022_conference,times}

% Optional math commands from https://github.com/goodfeli/dlbook_notation.
%%%%% NEW MATH DEFINITIONS %%%%%

\usepackage{amsmath,amsfonts,bm}
\usepackage{xifthen}

% Highlight a newly defined term
\newcommand{\newterm}[1]{{\bf #1}}

\def\eps{{\epsilon}}


% Utility for ticks 
\newcommand{\cmark}{\ding{51}}%
\newcommand{\xmark}{\ding{55}}%

% Theorem styles 
\theoremstyle{definition}
\newtheorem{theorem}{Theorem}[section]
\newtheorem{definition}{Definition}[section]
% \newtheorem{remark}{Remark}[theorem] %numbered remark
\newtheorem*{remark}{Remark} %unnumbered remark
\newtheorem{lemma}{Lemma}[section]
\newtheorem{prop}{Proposition}[section]
\newtheorem{corollary}{Corollary}[theorem]
\newtheorem{conjecture}{Conjecture}[section]
\newtheorem{assumption}{Assumption}[section]

\newtheorem{manualtheoreminner}{Theorem}
\newenvironment{manualtheorem}[1]{%
  \renewcommand\themanualtheoreminner{#1}%
  \manualtheoreminner
}{\endmanualtheoreminner}


% Math helper - standard function
\DeclareMathOperator*{\argmax}{arg\,max}
\DeclareMathOperator*{\argmin}{arg\,min}
\DeclareMathOperator{\support}{support}
\DeclareMathOperator{\MAX}{MAX}
\DeclareMathOperator{\term}{\texttt{term}}
\DeclareMathOperator*{\logsumexp}{log-sum-exp}
\DeclareMathOperator*{\TV}{TV}
\newcommand{\norm}[1]{\left\lVert#1\right\rVert}
\DeclarePairedDelimiter\set\{\}
\DeclarePairedDelimiter\abs{\lvert}{\rvert}%
\newcommand*{\mytop}{\mathrel{\scalebox{0.5}{$\top$}}}
\newcommand*{\mybot}{\mathrel{\scalebox{0.5}{$\bot$}}}
\newcommand*{\mydiese}{\mathrel{\scalebox{0.5}{$\#$}}}
\newcommand*{\myplus}{\mathrel{\scalebox{0.5}{$+$}}}
\newcommand*{\myminus}{\mathrel{\scalebox{0.5}{$-$}}}
\newcommand*{\bmg}{\bm{\gamma}}
\newcommand*{\bml}{\bm{\lambda}}

% MDP notation
\renewcommand{\S}{\mathcal{S}}
\newcommand{\X}{\mathcal{X}}
\newcommand{\A}{\mathcal{A}}
\newcommand{\T}{\mathcal{T}}
\newcommand{\M}{\mathcal{M}}
\newcommand{\B}{\mathcal{B}}
\newcommand{\Bset}{\mathfrak{B}}
\newcommand{\Dist}{\mathscr{P}}
\newcommand{\D}{\mathcal{D}}
\newcommand{\Real}{\mathbb{R}}
\renewcommand{\P}{\mathcal{P}}
\newcommand{\E}{\mathop{\mathbb{E}}}
\renewcommand{\H}{\mathcal{H}}
% \newcommand{\R}{\mathcal{R}}
% \newcommand{\C}{\mathcal{C}}

% Extended MDP notation
\newcommand{\Pstar}{p^{\star}}
\newcommand{\Rstar}{\bm{r}^{\star}}
\newcommand{\Cstar}{C^{\star}}
% \newcommand{\rmax}{\textsc{Rmax}}
\newcommand{\rmax}{r_{\mytop}}
\newcommand{\cmax}{\textsc{Cmax}}

\newcommand{\mstar}{m^{\star}}
\newcommand{\mhat}{\hat{m}}
\newcommand{\mopt}{m^{\star}}

\newcommand{\Phat}{\hat{p}}
\newcommand{\Rhat}{\hat{\bm{r}}}
\newcommand{\Chat}{\hat{C}}

% Math helper - custom function
\newcommand{\expwrtpi}[1]{\E_{\pi} [\sum_{t=0}^{\infty} \gamma^t #1(s_t, a_t)]}
\newcommand{\expangle}[1]{\langle #1  \rangle}

% helper function for return and constraints

% for value function, takes arguments:
% #1: policy 
% #2: the function of interest, R or C_i
% #3 (optional): the MDP for which this is estimated
\newcommand{\V}[3]{ %
    \ifthenelse{\isempty{#3}}%
    {V^{#1}(#2)}% #3 is empty 
    {V^{#1}_{#3}(#2)}%
}

\newcommand{\Q}[3]{
    \ifthenelse{\isempty{#3}}
    {Q^{#1}(#2)}% #3 is empty 
    {Q^{#1}_{#3}(#2)}%
}


\newcommand{\Adv}[3]{
    \ifthenelse{\isempty{#3}}
    {A^{#1}(#2)}% #3 is empty 
    {A^{#1}_{#3}(#2)}%
}

% careful diff notation
% 1: pi
% 2: R/C
% 3: M
\newcommand{\J}[3]{
    \ifthenelse{\isempty{#3}}
    {\mathcal{J}^{#1}_{#2}}% #3 is empty -> eg V^{\pi}(x ; R)
    {\mathcal{J}^{#1}_{#3,#2}}% -? eg V^{\pi}_{M}(x ; C)
    % {J_{#2}(#1)}% #3 is empty 
    % {J_{#2}(#1, #3)} %
}



\newcommand{\MRkern}{%
  \mkern-6.5mu
  \mathchoice{}{}{\mkern0.2mu}{\mkern0.5mu}%
}

% for value function, takes arguments:
% #1: policy 
% #2: the function of interest, R or C_i
% #3 (optional): the MDP for which this is estimated
% #4: variables to be given input (x) or (x,a)
\newcommand{\val}[4]{ %
    \ifthenelse{\isempty{#3}}%
    {v^{#1}_{#2}(#4)}% #3 is empty -> eg V^{\pi}(x ; R)
    {v^{#1}_{#3,#2}(#4)}% -? eg V^{\pi}_{M}(x ; C)
    % {V^{#1}_{#3}(#4 ;#2)}% -? eg V^{\pi}_{M}(x ; C)
    % {V_{#2}(#4 ; #1)}% #3 is empty -> eg V_R(x ; \pi)
    % {V_{#2}(#4 ;#1, #3)}% -? eg V_C(x ; \pi, M)
    % {#2 \MRkern V^{#1}_{#3}(#4)}% -? eg V^{\pi}_{M}(x ; C) # combines the letter V and R together
}

\newcommand{\qval}[4]{
    \ifthenelse{\isempty{#3}}
    {q^{#1}_{#2}(#4)}% #3 is empty -> eg V^{\pi}(x ; R)
    {q^{#1}_{#3,#2}(#4)}% -? eg V^{\pi}_{M}(x ; C)
    % {Q^{#1}(#4 ; #2)}% #3 is empty -> eg Q^{\pi}(x,a ; R)
    % {Q^{#1}_{#3}(#4 ;#2)}% -? eg Q^{\pi}_{M}(x,a ; C)
    % {Q_{#2}(#4 ; #1)}% #3 is empty -> eg Q_R(x,a ; \pi)
    % {Q_{#2}(#4 ;#1, #3)}% -? eg Q_C(x,a ; \pi, M)
}
\DeclareMathOperator*{\advantage}{Adv}

\newcommand{\adv}[4]{
    \ifthenelse{\isempty{#3}}
    {\advantage^{#1}_{#2}(#4)}% #3 is empty -> eg V^{\pi}(x ; R)
    {\advantage^{#1}_{#3,#2}(#4)}% -? eg V^{\pi}_{M}(x ; C)
    % {A^{#1}(#4 ; #2)}% #3 is empty -> eg Q^{\pi}(x,a ; R)
    % {A^{#1}_{#3}(#4 ;#2)}% -? eg Q^{\pi}_{M}(x,a ; C)
    % {A_{#2}(#4 ; #1)}% #3 is empty -> eg A_R(x,a ; \pi)
    % {A_{#2}(#4 ;#1, #3)}% -? eg A_C(x,a ; \pi, M)
}




\newcommand{\ci}{C}

\newcommand{\pib}{\pi_{b}}
\newcommand{\piopt}{\pi^{*}}
\newcommand{\pie}{\pi_{t}}

\newcommand{\lR}{\lambda_{R}}
\newcommand{\lC}{\lambda_{C}}
\newcommand{\ephi}{e_{\phi}}

\newcommand{\pr}{\text{Pr}}
\newcommand{\IS}{\text{IS}}
\newcommand{\CI}{\text{CI}}


% SPIBB symbols 
\newcommand{\EpsPib}{(\pi_b, e, \epsilon)}

\usepackage{hyperref}
\usepackage{url}


\usepackage[utf8]{inputenc} % allow utf-8 input
\usepackage[T1]{fontenc}    % use 8-bit T1 fonts
\usepackage{hyperref}       % hyperlinks
\usepackage{url}            % simple URL typesetting
\usepackage{booktabs}       % professional-quality tables
\usepackage{amsfonts}       % blackboard math symbols
\usepackage{nicefrac}       % compact symbols for 1/2, etc.
\usepackage{microtype}      % microtypography
\usepackage{xcolor}         % colors


% \usepackage[pdftex]{graphicx}          
% \usepackage{comment}
\usepackage{color}
\usepackage{colortbl}
% \usepackage{geometry}
\usepackage[english]{babel}
\usepackage{dsfont}
\usepackage{algorithm}
\usepackage{algorithmicx}
\usepackage{mathtools}
\usepackage{adjustbox}
\usepackage{import}
\usepackage[noend]{algpseudocode}
\usepackage{relsize}
\usepackage{subcaption}
\usepackage{multirow}
\usepackage{booktabs}
\usepackage{array}
\usepackage{enumitem}
\usepackage[low-sup]{subdepth}
\usepackage{xspace}
\usepackage{bm}
\usepackage[font=small,labelfont=bf]{caption}
\usepackage{varwidth}
\newsavebox\tmpbox


\title{Fully Online Meta-Learning Without Task Boundaries}
\iclrfinalcopy

\author{Jathushan Rajasegaran$^{1}$, Chelsea Finn$^{2}$, Sergey Levine$^{1}$\\
$^{1}$UC Berkeley, $^{2}$Stanford University \\
}

% The \author macro works with any number of authors. There are two commands
% used to separate the names and addresses of multiple authors: \And and \AND.
%
% Using \And between authors leaves it to \LaTeX{} to determine where to break
% the lines. Using \AND forces a linebreak at that point. So, if \LaTeX{}
% puts 3 of 4 authors names on the first line, and the last on the second
% line, try using \AND instead of \And before the third author name.

\newcommand{\fix}{\marginpar{FIX}}
\newcommand{\new}{\marginpar{NEW}}
\newcommand{\ours}{FOML\xspace}
% \newcommand{\rj}[1]{\textcolor{blue}{#1}}
\DeclareMathOperator*{\argminB}{argmin}
%\iclrfinalcopy % Uncomment for camera-ready version, but NOT for submission.
\begin{document}


\maketitle

\begin{abstract}
While deep networks can learn complex functions such as classifiers, detectors, and trackers, many applications require models that continually adapt to changing input distributions, changing tasks, and changing environmental conditions. Indeed, this ability to continuously accrue knowledge and use past experience to learn new tasks quickly in continual settings is one of the key properties of an intelligent system. For complex and high-dimensional problems, simply updating the model continually with standard learning algorithms such as gradient descent may result in slow adaptation. Meta-learning can provide a powerful tool to accelerate adaptation yet is conventionally studied in batch settings. In this paper, we study how meta-learning can be applied to tackle online problems of this nature, simultaneously adapting to changing tasks and input distributions and meta-training the model in order to adapt more quickly in the future. Extending meta-learning into the online setting presents its own challenges, and although several prior methods have studied related problems, they generally require a discrete notion of tasks, with known ground-truth task boundaries. Such methods typically adapt to each task in sequence, resetting the model between tasks, rather than adapting continuously across tasks. In many real-world settings, such discrete boundaries are unavailable, and may not even exist. To address these settings, we propose a Fully Online Meta-Learning (FOML) algorithm, which does not require any ground truth knowledge about the task boundaries and stays fully online without resetting back to pre-trained weights. Our experiments show that FOML was able to learn new tasks faster than the state-of-the-art online learning methods on Rainbow-MNIST, CIFAR100 and CELEBA datasets.
\end{abstract}



\section{Introduction}

Scientific literature is most commonly available in the form of PDFs, which pose challenges for accessibility \citep{NielsenPDFStillUnfit, Bigham2016AnUT}. When researchers, students, and other individuals who are blind or low vision (BLV) interact with scientific PDFs through screen readers, the availability of document structure tags, labeled reading order, labeled headers, and image alt-text are necessary to facilitate these interactions. However, these features must be painstakingly added by authors using proprietary software tools, and as a result, are often missing from papers. Low vision or dyslexic readers who interact with PDFs through screen magnification or text-to-speech may also find the complexity of certain academic paper PDF formats challenging, e.g., non-linear layout can interrupt the flow of text in a magnifying tool. Inaccessible paper PDFs can lead to high cognitive overload, frustration, and abandonment of reading for BLV readers. 

Unfortunately, we find that the majority of scientific PDFs lack basic accessibility features. We estimate based on a sample of \numpdfs PDFs from multiple fields of study that only around \percaccessible of paper PDFs released in the last decade satisfy all of the aforementioned accessibility requirements. 
Accessibility challenges for academic PDFs are largely due to three factors: (1) the complexity of the PDF file format, which make it less amenable to certain accessibility features, (2) the dearth of tools, especially non-proprietary tools, for creating accessible PDFs, and (3) the dependency on volunteerism from the community with minimal support or enforcement \citep{Bigham2016AnUT}. The intent of the PDF file format is to support faithful visual representation of a document for printing, a goal that is inherently divergent from that of document representation for the purposes of accessibility. Though some professional organizations like the Association for Computing Machinery (ACM) have encouraged PDF accessibility through standards and writing guidelines,\footnote{\href{https://www.acm.org/publications/authors/submissions}{https://www.acm.org/publications/authors/submissions}} uptake among academic publishers and disciplines more broadly has been limited. 

While policy changes help, the fact remains that most academic PDFs produced today, and historically, are inaccessible, yet remain as the dominant way to read those papers. A long-range solution will necessitate buy-in from multiple stakeholders---publishers, authors, readers, technologists, granting agencies, and the like. But in the interim, there are technological solutions that can be offered as a sort of ``band-aid'' to the problem. We use this paper to offer an in-depth qualitative and quantitative description of the problem as it stands, and to introduce one such technological solution: the \scially system that automatically extracts semantic information from paper PDFs and re-renders this content in the form of an accessible HTML document. Though the process is imperfect and can introduce errors, we demonstrate the ability of the rendered HTMLs to reduce cognitive load and facilitate in-paper navigation and interactions for BLV users. 

The goals and contributions of this paper are three-fold:

\begin{enumerate}
    \item We characterize the state of academic-paper PDF accessibility by estimating the degree of adherence to accessibility criteria for papers published in the last decade (2010--2019), and describe correlations between year, field of study, PDF typesetting software, and PDF accessibility.
    \item We propose an automated approach for extracting the content of academic PDFs and displaying this content in a more accessible HTML document format. We build a prototype that re-renders 12 million PDFs in HTML, and describe the design decisions, features, and quality of the renders (assessed as faithfulness to the source PDF). We perform expert grading of the rendered HTML and report an error analysis. A demo of our system is available at \href{https://scia11y.org/}{scia11y.org}, which makes available 1.5M HTML renders of open access PDFs.
    \item We conduct an exploratory user study with \numusers BLV scholars to better understand the challenges they experience when reading academic papers and how our proposed tool might augment their current workflow. During the study, we ask users to interact with the prototype and offer feedback for its improvement. We perform open coding of interviews to identify existing reading challenges, coping mechanisms, as well as positive and negative responses to prototype features. We summarize the findings of this user study into a set of design recommendations.
\end{enumerate}

Our analysis reveals that PDF accessibility adherence is low across all fields of study. Of the five accessibility criteria we assess, only \percaccessible of the PDFs we assess demonstrate full compliance. Though compliance for several criteria seems to be increasing over time, author awareness and contribution to accessibility remains low, as Alt-text has the lowest compliance of the five criteria at between 5--10\% (Alt-text is the only criterion of the five that \textit{requires} author intervention in all cases using current tools). We also find that typesetting software is strongly associated with accessibility compliance, with LaTeX and publishing software like Arbortext APP producing low compliance PDFs, while Microsoft Word is generally associated with higher compliance.


\begin{figure}[t!]
    \centering
    \includegraphics[width=\textwidth]{figures/pipeline.png}
    \caption{A schematic for creating the \scially HTML render from a paper PDF. Starting with the raw two-column PDF on the left, S2ORC \citep{lo-wang-2020-s2orc} is used to extract title, authors, abstract, section headers, body text, and references. S2ORC also identifies links between inline citations and references to figures and table objects. DeepFigures \citep{Siegel2018ExtractingSF} is used to extract figures and tables, along with their captions. The output of these two models are merged with metadata from the Semantic Scholar API. Heuristics are used to construct a table of contents, to insert figures and tables in the appropriate places in the text, and to repair broken URLs. We add HTML headers as illustrated (header tags for sections, paragraph tags for body text, and figure tags for figures and tables); highlighted components (table of contents and links in references) are not in the PDF and novel navigational features that we introduce to the HTML render. An example HTML render of parts of a paper document is show to the right (actual render is single column, which is split here for presentation).}
    \label{fig:pipeline}
    \Description{A schematic diagram showing the components of the SciA11y pipeline. An image of a paper PDF is on the left. Red boxes on the PDF image highlight the text components from the paper, with an arrow pointing to a box that says "S2ORC extracts: title, authors, abstract, section headers, body paragraphs, and references." A blue box on the PDF image highlights a figure, with an arrow pointing to a box that says "DeepFigures extracts: figures, figure captions, tables, and table titles/captions." A box below "S2ORC extracts" and "DeepFigures extracts" says "Additional content: metadata from Semantic Scholar API, table of contents, figures and tables inserted at first mention, and links between references and text." Arrows from all three boxes point into a larger box that describes the SciA11y prototype, where HTML tags are inserted around various blocks of text extracted from the PDF. On the right of all this is a screen capture of an example HTML render, showing how the semantic content from the PDF is represented as a single-column HTML page for easy reading.}
\end{figure}

To offset the reading challenges of inaccessible papers for BLV researchers, we propose and test the \scially system for rendering academic PDFs into accessible HTML documents. As shown in Figure~\ref{fig:pipeline}, our prototype integrates several machine learning text and vision models to extract the structure and semantic content of papers. The content is represented as an HTML document with headings and links for navigation, figures and tables, as well as other novel features to assist in document structure understanding. Our evaluation of the \scially system identifies common classes of extraction problems, and finds that though many papers exhibit some extraction errors, the majority (55\%) have no major problems that impact readability, and another 32\% have only some problems that impact readability.

Through our user study, we identify numerous challenges faced by BLV users when reading paper PDFs, including some that affect the whole document or limit navigation, and many that affect the ability of the reader to understand text or various elements of a paper like math content or tables. Responses to \scially were positive; participants especially liked navigation features such as headings, the table of contents, and bidirectional links between inline citations and references. Of the extraction errors in \scially, missed or incorrectly extracted headings were the most problematic, as these impact the user's ability to navigate between sections and fully trust the system. All users reported being likely to use the system in the future. When asked how the system might be integrated into their workflow, one participant replied ``I think it would become the workflow.'' Another participant said, ``for unaccessible PDFs, this is life-changing.'' We condense these findings into a set of recommendations for designing and engineering accessible reading systems (Section~\ref{sec:designrecs}). Most importantly, documents should be structured to match a reader's mental model, objects should be properly tagged, and care should be taken to reduce the reader's cognitive load and increase trust in the system. Features that emulate the external memory that visual layout provides to sighted users can be especially beneficial.

This paper is organized as follows. Following a description of related work in Section \ref{sec:related_work}, we first provide a meta-scientific analysis of the current state of academic PDF accessibility in Section \ref{sec:sos}. In Section \ref{sec:pdf2html}, we document our pipeline for converting PDF to HTML and describe the \scially prototype for rendering papers. An evaluation of HTML render quality and faithfulness is provided in Section \ref{sec:evaluation}. Section \ref{sec:user_study} describes our user study and findings. 
We recognize that no PDF extraction system is perfect, and many open research challenges remain in improving these systems. However, based on our findings, we believe \scially can dramatically improve screen reader navigation of most papers compared to PDFs, and is well-positioned to assist BLV researchers with many of their most common reading use cases. Our hope is that a system such as \scially can improve BLV researcher access to the content of academic papers, and that these design recommendations can be leveraged by others to create better, more faithful, and ultimately more usable tools and systems for scholars in the BLV community.

\section{Related Work}

Online meta-learning brings together ideas from online learning, meta learning, and continual learning, with the aim of adapting quickly to each new task while \emph{simultaneously} learning how to adapt even more quickly in the future. We discuss these three sets of approaches next.


\noindent \textbf{Meta Learning:} Meta learning methods try to learn the high-level context of the data, to behave well on new tasks (\emph{Learning to learn}). These methods involve learning a metric space~\citep{koch2015siamese, vinyals2016matching, snell2017prototypical, yang2017learning}, gradient based updates~\citep{finn2017model, li2017meta, park2019meta, nichol2018first, nichol2018reptile}, or some specific architecture designs~\citep{santoro2016meta, munkhdalai2017meta, ravi2016optimization}.
In this work, we are mainly interested in gradient based meta learning methods for online learning. MAML~\citep{finn2017model} and its variants~\citep{nichol2018first, nichol2018reptile, li2017meta, park2019meta, antoniou2018train} first meta train the models in such a way that the meta parameters are close to the optimal task specific parameters (good initialization). This way, adaptation becomes faster when fine tuning from the meta parameters. However, directly adapting this approach into an online setting will require more relaxation on online learning assumptions, such as access to task boundaries and resetting back and froth from meta parameters. Our method does not require knowledge of task boundaries.




\noindent \textbf{Online Learning:} Online learning methods update their models based on the stream of data sequentially. There are various works on online learning using linear models~\citep{cesa2006prediction}, non-linear models with kernels~\citep{kivinen2004online, jin2010online}, and deep neural networks~\citep{zhou2012online}. Online learning algorithms often simply update the model on the new data, and do not consider the past knowledge of the previously seen data to do this online update more efficiently. However, the online meta learning framework, allow us to keep track of previously seen data and with the ``meta'' knowledge we can update the online weights to the new data more faster and efficiently.
\noindent \textbf{Continual Learning:} 
A number of prior works on continual learning have addressed catastrophic forgetting~\citep{mccloskey1989catastrophic,li2017learning,ratcliff1990connectionist, rajasegaran2019random, rajasegaran2020itaml}, removing the need to store all prior data during training. Our method does not address catastrophic forgetting for the meta-training phase, because we must still store all data so as to ``replay'' it for meta-training, though it may be possible to discard or sub-sample old data (which we leave to future work). However, our adaptation process is fully online. A number of works perform meta-learning for better continual learning, i.e. learning good continual learning strategies~\citep{al2017continuous,nagabandi2018deep,javed2019meta,harrison2019continuous,he2019task,beaulieu2020learning}. However, these prior methods still perform batch-mode meta-training, while our method also performs the meta-training itself incrementally online, without task boundaries.


The closest work to ours is the follow the meta-leader (FTML) method~\citep{finn19a} and other online meta-learning methods~\citep{yao2020online}. FTML is a varaint of MAML that finetunes to each new task in turn, resetting to the meta-trained parameters between every task. While this effectively accelerates acquisition of new tasks, it requires ground truth knowledge of task boundaries and, as we show in our experiments, our approach outperforms FTML \emph{even when FTML has access to task boundaries and our method does not}. Note that the memory requirements for such methods increase with the number of adaptation gradient steps, and this limitation is also shared by our approach. Online-within-online meta-learning~\cite{denevi2019online} also aims to accelerate online updates by leveraging prior tasks, but still requires knowledge of task boundaries. MOCA~\cite{harrison2020continuous} instead aims to \emph{infer} the task boundaries. In contrast, our method does not even attempt to find the task boundaries, but directly adapts without them. A number of related works also address continual learning via meta-learning, but with the aim of minimizing catastrophic forgetting~\cite{gupta2020maml, caccia2020online}. Our aim is not to address catastrophic forgetting. Our method also meta-trains from small datasets for thousands of tasks, whereas prior continual learning approaches typically focus on settings with fewer larger tasks (e.g., 10-100 tasks).

\section{Method}
Fig.~\ref{fig:framework} presents the illustration of the proposed \frameworkName.
In this section,  
we start by providing the problem definition of online CIL. Then, we describe the definition of the online prototype, the proposed online prototype equilibrium, and the proposed adaptive prototypical feedback. Finally, we propose an online prototype learning framework.

\subsection{Problem Definition}
Formally, online CIL considers a continuous sequence of tasks from a single-pass data stream $\mathfrak{D}=\left\{\mathcal{D}_1, \ldots, \mathcal{D}_T \right\} $, where $\mathcal{D}_t = \left\{ x_{i}, y_{i} \right\} ^{N_t}_{i=1} $ is the dataset of task $t$, and $T$ is the total number of tasks. Dataset $\mathcal{D}_t$ contains $N_t$ labeled samples, $y_{i}$ is the class label of sample $x_{i}$ and $y_{i} \in \mathcal{C}_t$, where $\mathcal{C}_t$ is the class set of task $t$ and the class sets of different tasks are disjoint. 
For replay-based methods, a memory bank is used to store a small subset of seen data, and we also maintain a memory bank $\mathcal{M}$ in our method.
At each time step of task $t$, the model receives a mini-batch data $X \cup X^\mathrm{b}$ for training, where $X$ and $X^\mathrm{b}$ are drawn from the i.i.d distribution $\mathcal{D}_t$ and the memory bank $\mathcal{M}$, respectively. 
Moreover, we adopt the single-head evaluation setup~\cite{EWC}, where a unified classifier must choose labels from all seen classes at inference due to unavailable task identifiers. 
The goal of online CIL is to train a unified model on data seen only once while predicting well on both new and old classes.

\subsection{Online Prototype Definition}
Prior to introducing the online prototypes, we first present the network architecture of our \frameworkName. Suppose that the model consists of three components: an encoder network $f$, a projection head $g$, and a classifier $\varphi$. Each sample $x$ in incoming data $X$ (a mini-batch data from new classes) is mapped to a projected vectorial embedding (representation) $\mathbf{z}$ by encoder $f$ and projector $g$:
\begin{align}
\label{eq:cal_z}
    \mathbf{z} = g(f(\operatorname{aug}(x);\theta_f);\theta_g),
\end{align}
where $\operatorname{aug}$ represents the data augmentation operation, $\theta_f$ and $\theta_g$ represent the parameters of $f$ and $g$, respectively, and $\mathbf{z}$ is $\ell_2$-normalized. 
Similar to Eq.~\eqref{eq:cal_z}, we use $\mathbf{z}^\mathrm{b}$ to denote the representation of replay data $X^\mathrm{b}$ (a mini-batch data from seen classes in the memory bank). 

At each time step of task $t$, the online prototype of each class is defined as the mean representation in a mini-batch:
\begin{align}
\label{eq:cal_p}
    \mathbf{p}_i = \frac{1}{n_i}\sum\nolimits_j\mathbf{z}_j\cdot \mathbbm{1}\{y_j = i\},
\end{align}
where $n_i$ is the number of samples for class $i$ in a mini-batch, and $\mathbbm{1}$ is the indicator function. 
We can get a set of $K$ online prototypes  in $X$, $\mathcal{P} = \left\{ \mathbf{p}_{i} \right\} ^{K}_{i=1}$, and a set of $K^\mathrm{b}$ online prototypes in $X^\mathrm{b}$, $\mathcal{P}^\mathrm{b} = \left\{ \mathbf{p}_i^\mathrm{b} \right\} ^{K^\mathrm{b}}_{i=1}$.
Note that $K = |\mathcal{P}| \leq |\mathcal{C}_t|$ and $K^\mathrm{b} = |\mathcal{P}^\mathrm{b}| \leq \sum_{i=1}^{t}|\mathcal{C}_i| $, where $|\cdot|$ denotes the cardinal number.



\subsection{Online Prototype Equilibrium}
The introduced online prototypes can provide representative features and avoid class-unrelated information.  
These characteristics are exactly the key to counteracting shortcut learning in online CL.
Besides, maintaining the discrimination among seen classes is also essential to mitigate catastrophic forgetting.
Based on these, we attempt to learn representative features of each class by pulling online prototypes $\mathcal{P}$ and their augmented views $\widehat{\mathcal{P}}$ closer in the embedding space, and learn discriminative features between classes by pushing online prototypes of different classes away, formally defined as a contrastive loss:
\begin{align}
\label{eq:proto_infoNCE}
    \ell(\mathcal{P},\widehat{\mathcal{P}})\!=\!
    % \frac{-1}{K}
    \frac{-1}{|\mathcal{P}|}\sum_{i=1}^{|\mathcal{P}|}\!\log\! 
    \tfrac
    {\exp \big(\tfrac{{\mathbf{p}_i^\mathrm{T} \widehat{\mathbf{p}}_i}}{\tau}\big)}
    {
    \sum\limits_{j} \exp \big(\tfrac{{\mathbf{p}_i^\mathrm{T} \widehat{\mathbf{p}}_j}}{\tau}\big)
    +\!
    \sum\limits_{\substack{j \neq i}} \exp \big(\tfrac{{\mathbf{p}_i^\mathrm{T} \mathbf{p}_j}}{\tau}\big) 
    },
\end{align}
where 
$\tau$ is the temperature hyper-parameter, 
$\mathcal{P}$ and $\widehat{\mathcal{P}}$ are $\ell_2$-normalized. To compute the contrastive loss across all positive pairs in both $(\mathcal{P}, \widehat{\mathcal{P}})$ and $(\widehat{\mathcal{P}}, \mathcal{P})$, we define $\mathcal{L}_{\mathrm{pro}}$ as the final contrastive loss over online prototypes:
\begin{align}
    \mathcal{L}_{\mathrm{pro}}(\mathcal{P},\widehat{\mathcal{P}}) = 
    \frac{1}{2}
    \left[\ell(\mathcal{P}, \widehat{\mathcal{P}}) + \ell(\widehat{\mathcal{P}}, \mathcal{P})\right].
\end{align}



Considering the learning of new classes and the consolidation of learned knowledge simultaneously in online CL, we propose Online Prototype Equilibrium (\methodname) to 
learn representative and discriminative features on both new and seen classes by employing $\mathcal{L}_{\mathrm{pro}}$:
\begin{equation}
    \begin{aligned}
    \mathcal{L}_{\mathrm{\methodname}}
    &=
    \mathcal{L}^{\mathrm{new}}_{\mathrm{pro}}(\mathcal{P},\widehat{\mathcal{P}}) + \mathcal{L}^{\mathrm{seen}}_{\mathrm{pro}}(\mathcal{P}^\mathrm{b},\widehat{\mathcal{P}}^\mathrm{b}),
    \end{aligned}
\end{equation}
where
$\mathcal{L}^{\mathrm{new}}_{\mathrm{pro}}$ focuses on learning knowledge from \emph{new} classes, and $\mathcal{L}^{\mathrm{seen}}_{\mathrm{pro}}$ is dedicated to preserving learned knowledge of all \emph{seen} classes.
\textit{This process is similar to a zero-sum game, 
and \methodname aims to achieve the equilibrium to play a win-win game.}
Concretely,
as the model learns, the knowledge of new classes is gained and added to the prototypes over the memory bank $\mathcal{M}$, causing $\mathcal{L}^{\mathrm{seen}}_{\mathrm{pro}}$ gradually changes to the equilibrium that separates all seen classes well, including new ones. 
This variation is crucial to mitigate forgetting and is consistent with the goal of CIL.



\subsection{Adaptive Prototypical Feedback} 
Although \methodname can bring an overall equilibrium, it tends to treat each class \emph{equally}. 
In fact, the degree of confusion varies among classes, 
and the model should focus purposefully on confused classes to consolidate learned knowledge. 
To this end, we propose Adaptive Prototypical Feedback (\dataaugname) with the feedback of online prototypes to sense the classes that are prone to be misclassified and then enhance their decision boundaries.
 
For each class pair in the memory bank $\mathcal{M}$, \dataaugname calculates the distances between online prototypes of all seen classes from the previous time step, showing the class confusion status by these distances. The closer the two prototypes are, the easier to be misclassified. 
Based on this analysis, 
our idea is to enhance the boundaries for those classes. Therefore, we convert the prototype distance matrix to a probability distribution $P$ over the classes via a symmetric Gaussian kernel, defined as follows:
\begin{align}
\label{eq:cal_d}
    P_{i, j} \propto \exp (-\| \mathbf{p}_i^\mathrm{b} - \mathbf{p}_j^\mathrm{b} \|_2^2),
\end{align}
where $i,j \in \{ 1, \ldots, |\mathcal{P}^\mathrm{b}| \}$ and $i \neq j$. 
Then, 
all probabilities are normalized to a probability mass function that sums to one.
\dataaugname returns probabilities to $\mathcal{M}$ for guiding the next sampling process and enhancing decision boundaries of easily misclassified classes. 


Our adaptive prototypical feedback is implemented as a sampling-based mixup. Specifically, 
\dataaugname adaptively selects more samples from easily misclassified classes in $\mathcal{M}$ for mixup~\cite{Mixup} according to the probability distribution $P$. 
Considering not over-penalizing the equilibrium of current online prototypes, we introduce a two-stage sampling strategy for replay data $X^\mathrm{b}$ of size $m$. 
First, we select $n_{\mathrm{\dataaugname}}$ samples  
with $P$, and a larger $P_{a,b}$ means more sampling from classes $a$ and $b$. Here, $n_{\mathrm{\dataaugname}} = \alpha \cdot m$, and $\alpha$ is the ratio of \dataaugname.
Second, the remaining $m-n_{\mathrm{\dataaugname}}$ samples are uniformly randomly selected from the entire memory bank to avoid the model only focusing on easily misclassified classes and disrupting the established equilibrium. 




\subsection{Overall Framework of \frameworkName}
The overall structure of \frameworkName is shown in Fig.~\ref{fig:framework}. \frameworkName comprises two key components based on proposed online prototypes: Online Prototype Equilibrium (\methodname) and Adaptive Prototypical Feedback (\dataaugname). 
With the two components, 
the model can learn representative features against shortcut learning, and 
all seen classes maintain discriminative when learning new classes. 
However, classes may not be compact, because the online prototypes cannot cover full instance-level information.
To further achieve intra-class compactness, 
we employ supervised contrastive learning~\cite{SupCL} to learn instance-wise representations:
\begin{equation}
\begin{aligned}
    \mathcal{L}_{\mathrm{INS}}
    &=
    \sum_{i=1}^{2N} \frac{-1}{\left|I_i\right|} \sum_{j \in I_i} \log \frac{\exp \left(\mathrm{sim}(\mathbf{z}_i, \mathbf{z}_j) / \tau^{\prime}\right)}{\sum\limits_{k \neq i} \exp \left(\mathrm{sim}(\mathbf{z}_i, \mathbf{z}_k) / \tau^{\prime}\right)}
    \\
    &+
    \sum_{i=1}^{2m} \frac{-1}{\left|I_i^{\mathrm{b}}\right|} \sum_{j \in I_i^{\mathrm{b}}} \log \frac{\exp (\mathrm{sim}(\mathbf{z}_i^{\mathrm{b}}, \mathbf{z}_j^{\mathrm{b}}) / \tau^{\prime})}{\sum\limits_{k \neq i} \exp \left(\mathrm{sim}(\mathbf{z}_i^{\mathrm{b}}, \mathbf{z}_k^{\mathrm{b}}) / \tau^{\prime}\right)},
\end{aligned}
\end{equation}
where $I_i=\left\{j \in\{1, \ldots, 2 N\} \mid j \neq i, y_j=y_i\right\}$ and $I_i^\mathrm{{b}}=\left\{j \in\{1, \ldots, 2m\} \mid j \neq i, y_j^\mathrm{b}=y_i^\mathrm{b}\right\}$ are the set of positive samples indexes to $\mathbf{z}_i$ and $\mathbf{z}_i^\mathrm{{b}}$, respectively. $y_i^\mathrm{b}$ is the class label of input $x_i^\mathrm{b}$ from $X^\mathrm{b}$. $N$ is the batch size of $X$. $\tau^{\prime}$ is the temperature hyperparameter.
The similarity function $\mathrm{sim}$ is computed in the same way as Eq.~(9) in OCM~\cite{OCM}.

Thus, the total loss of our \frameworkName framework is given as:
\begin{align}
    \mathcal{L}_{\mathrm{\frameworkName}}=\mathcal{L}_{\mathrm{\methodname}} + \mathcal{L}_{\mathrm{INS}} + \mathcal{L}_{\mathrm{CE}},
\end{align}
where $\mathcal{L}_{\mathrm{CE}} = \mathrm{CE}(y^\mathrm{b}, \varphi(f(\operatorname{aug}(x^\mathrm{b}))))$ is the cross-entropy loss; see Appendix~\ref{appendix:algorithm} for detailed training algorithms.

Following other replay-based methods~\cite{ER, SCR, OCM}, we update the memory bank in each time step by uniformly randomly selecting samples from $X$ to push into $\mathcal{M}$ and, if $\mathcal{M}$ is full, pulling an equal number of samples out of $\mathcal{M}$.


\section{Experimental Evaluation}

Our experiments focus on online meta-learning of image classification tasks. In these settings, an effective algorithm should adapt to changing tasks as quickly as possible, rapidly identify when the task has changed, and adjust its parameters accordingly. Furthermore, a successful algorithm should make use of past task shifts to meta-learn effectively, thus accelerating the speed with which it adapts to future task changes. In all cases, the algorithm receives one data point at a time, and the task changes periodically. In order to allow for a comparison with prior methods, which generally assume known task boundaries, the data stream is partitioned into discrete tasks, but our algorithm is not aware of which datapoint belongs to which tasks or where the boundaries occur. The prior methods, in contrast, \emph{are} provided with this information, thus giving them an advantage. We first describe the specific task formulations, and then the prior methods that we compare to.

Online meta-learning algorithms should adapt to each task as quickly as possible, and also use data from past tasks to accelerate acquisition of future tasks. Therefore, we report our results as a learning curve, with one axis corresponding to the number of seen tasks, and the other axis corresponding to the cumulative error rate on that task. This error rate is computed using a held-out validation data for each task after the adaptation that task. 

We evaluate prior online meta-learning methods and baselines on three different datasets (Rainbow-MNIST, CIFAR100 and CELEBA). TOE (train on everything), TFS (train from scratch), FTL (follow the leader), FTML (follow the meta-leader)~\citep{finn19a}, LwF~\citep{li2017learning}, iCaRL~\citep{rebuffi2016icarl} and MOCA~\citep{harrison2020continuous} are the baseline methods we compare against our method FOML. See Section 4 for more detailed description of these methods.

\noindent \textbf{Datasets:} We compare TOE, TFS, FTL, FTML, LwF, iCaLR and FOML on three different datasets. Rainbow-MNIST~\citep{finn19a} was created by changing the background color, scale and rotation of the MNIST dataset. It includes 7 different background colors, 2 scales (full and half) and 4 different rotations. This leads to a total of 56 number of tasks. Each individual task is to classify the images into 10 classes. We use the same partition with 900 samples per each task, as in prior work~\citep{finn19a}. However, this task contains relatively simple images, and only 56 tasks. To create a much longer task sequence with significantly more realistic images, we modified the CIFAR-100 and CELEBA datasets to create an online meta-learning benchmark, which we call online-CIFAR100 and online-CELEBA, respectively. Every task is a randomly sampled set of classes, and the goal is to classify whether two images in this set belongs to same class or not. Specifically, each task corresponds to 5 classes, and every datapoint consists of a \emph{pair} of images, each corresponding to one of the 5 classes for that task. The goal is to predict whether the two images belong to the same class or not. Note that different tasks are not mutually exclusive, which \emph{in principle} should favor a TOE-style method, since meta-learning is known to underperform with non-mutually-exclusive tasks~\citep{yin2019meta}. To make sure the data distribution changes smoothly over tasks, we only change a subset of the classes between consecutive tasks. This allow us to create a very large number of tasks (1200 for online-CIFAR100), and evaluate our method and the baselines on much longer and more realistic task sequences.

\begin{figure*}[!t]
\centering
    \begin{subfigure}{0.32\textwidth}
        \centering
        \includegraphics[width=0.95\textwidth]{figs/exp4x.pdf}
    \end{subfigure} 
    \begin{subfigure}{0.32\textwidth}
        \centering
        \includegraphics[width=0.95\textwidth]{figs/exp_cifar_1_main.pdf}
    \end{subfigure}
    \begin{subfigure}{0.32\textwidth}
        \centering
        \includegraphics[width=0.95\textwidth]{figs/exp_celeba_1_main.pdf}
    \end{subfigure}
    \caption{\footnotesize{ \emph{Comparison between online algorithms:} We compare our method with baselines and prior approaches, including TFS (Train from Scratch), TOE (Train on Everything), FTL (Follow the Leader) and FTML (Follow the Meta Leader). \textbf{a:} Performance relative to the number of tasks seen over the course of online training on the Rainbow-MNIST dataset. As the number of task increases, FOML achieves lower error rates compared to other methods. We also compare our method with continual learning baselines: LwF~\citep{li2017learning}, iCaRL~\citep{rebuffi2016icarl} and MOCA~\citep{harrison2020continuous}. MOCA~\cite{harrison2020continuous} archive similar performance to ours at the end of the learning, but FOML makes significantly faster progress. \textbf{b:} Error rates on the Online-CIFAR100 dataset. Note that FOML not only achieves lower error rates on average, but also reaches the lowest error (of around 17\%) more quickly than the other methods. \textbf{c:} Performance of FOML on the CELEBA dataset. This dataset contains more than 1000 classes, and we follow the same protocol as in Online-CIFAR100 experiments. Our method, FOML, learns more quickly on this task as well.}}
    \label{fig:main_plots}

\end{figure*}





\noindent \textbf{Implementation Details:} We use a simple 4 layer convolutional neural network with 8,16,32,64 filters at each layer for both experiments. However, for the CIFAR-100 experiments, a Siamese version of the same network is used. All the methods were trained via a cross-entropy loss, with their best performing hyper-parameters, on a single NVIDA-2080 GPU machine. Please see Appendix~\ref{sec:A1} for more details on hyper-parameters and network architecture.


\noindent \textbf{Results on Rainbow-MNIST:} As shown in Fig~\ref{fig:main_plots}, FOML attains the lowest error rate on most tasks in Rainbow-MNIST, except a small segment in the very beginning. The performance of TFS is similar performance across all the tasks, and does not improve over time. This is because it resets its weights every time it encounters a new task, and therefore cannot not gain any advantage from past data. TOE has larger error rates at the start, but as we add more data into the buffer, TOE is able to improve. On the other hand, both FTL and FTML start with similar performance, but FTML achieve much lower error rates at the end of the sequence compared to FTL, consistently with prior work~\citep{finn19a}. The final error rates of FOML are around 10\%, and it reaches this performance significantly faster than FTML, after less than 20 tasks. Note that FTML also has access to task boundaries, while FOML does not.

\noindent \textbf{Results on Online-CIFAR100 and Online-CELEBA:} We use a Siamese network for this experiment, where each image is fed into a 7-layer convolutional network, and each branch outputs a 128 dimensional embedding vector. A difference of these vectors are fed into a fully connected layer for the final classification. Each task contains data from 5 classes, and each new task introduces three new classes, and retains two of the classes from the previous task, providing a degree of temporal coherence while still requiring each algorithm to handle persistent shift in the task. Fig~\ref{fig:main_plots} shows the error rates of various online learning methods, where each method is trained over a sequence of 1200 tasks for online-CIFAR100. All the methods start with initial error rates of 50\%. The tasks are not mutually exclusive, so in principle TOE can attain good performance, but it makes the slowest progress among all the methods, suggesting that simple pretraining is not sufficient to \emph{accelerate} learning. FTL uses a similar pre-training strategy as TOE. However it has an adaptation stage where the model is fine-tuned on the new task. This allows it to make slower progress.
As expected from prior work~\citep{finn19a}, the meta-learning procedure used by FTML allows it to make faster progress than FTL. However, FOML makes faster progress on average, and achieves the lowest final error rate ($\sim15\%$) after sequence of 1200 tasks. 

\vspace{-0.2cm}
\subsection{Ablation Studies}
\vspace{-0.2cm}
We perform various ablations by varying the number of online parameters used for the meta-update $K$, importance of meta-model to analysis the properties of our method.

\noindent \textbf{Number of online parameters used for the meta-update:} Our method periodically updates the online weights and meta weights. The meta-updates involves taking $K$ recent online parameters and updating the meta model via MAML gradient. Therefore, meta-updates depend on the trajectory of the online parameters. In this experiment, we investigate how the performance of FOML changes as we vary the number of parameters used for the meta-update ($K$ in Algorithm~\ref{alg:OML}). Fig~\ref{fig:ablations} shows the performance of our method with various values of $K$: $K=[1,2,3,5,10]$. We can see that the performance improves when we update the meta parameters over longer trajectory of online parameters (larger $K$). We speculate that this is due to the longer sequences providing a clearer signal for how the meta-parameters influence online updates over many steps.

\noindent \textbf{Importance of meta update:} FOML keeps track of separate online parameters and meta-parameters, and each of them is updated via corresponding updates. However, only the online parameters $\phi$ are used for evaluation, while the meta-parameters $\theta$ only influence them via the regularizer, and have no direct effect on evaluation performance. This might raise the question: how important is the contribution of the meta-parameters to the performance of the algorithm during online training? We train a model with and without meta-updates, and the performance is shown in Fig~\ref{fig:ablations}. None that, the model without meta-updates is identical to our method, except that the meta-updates themselves are not performed. We can clearly see that the model trained with meta-updates preforms much better than a model trained without meta-updates. The model trained without meta-updates generally does not improve significantly as more tasks are seen, while the model trained with meta-updates improves with each task, and reaches significantly lower final error. This shows that, even though $\theta$ and $\phi$ are decoupled and only connected via a regularization, the meta-learning component of our method really is critical for its good performance.

\begin{figure}[!t]
    \centering
    \begin{minipage}[]{0.45\textwidth}
        \includegraphics[width=0.95\columnwidth]{figs/test5.pdf}
    \end{minipage}
    \begin{minipage}[]{0.45\textwidth}
        \includegraphics[width=0.95\columnwidth]{figs/test7.pdf}
    \end{minipage}
    \caption{\footnotesize{\emph{Ablation experiments:} \textbf{a)} We vary the number of online updates $K$ used before the meta-update, to see how it affects the performance of our method. The performance of FOML improves as the number of online updates is increased. \textbf{b)} This experiment shows how FOML performs with and without meta updates, to confirm that the meta-training is indeed an essential component of our method. With meta-updates, FOML learns more quickly, and performance improves with more tasks.}}
    \label{fig:ablations}
\end{figure}



\vspace{-0.4cm}
\section{Conclusion}
\vspace{-0.4cm}
We presented FOML, a MAML-based algorithm for online meta-learning that does not require ground truth knowledge of task boundaries, and does not require resetting the parameter vector back to the meta-learned parameters for every task. FOML is conceptually simple, maintaining just two parameter vectors over the entire online adaptation process: a vector of online parameters $\phi$, which are updated continually on each new batch of datapoints, and a vector of meta-parameters $\theta$, which are updated correspondingly with meta-updates to accelerate the online adaptation process, and influence the online updates via a regularizer. We find that even a relatively simple task sampling scheme that selects datapoints at random from a buffer of all seen data enables effective meta-training that accelerates the speed with which FOML can adapt to each new task, and we find that FOML reaches a final performance that is comparable to or better than baselines and prior methods, while learning to adapt quickly to new tasks significantly faster. While our work focuses on supervised classification problems, a particularly exciting direction for future work is to extend such online meta-learning methods to other types of online supervision that may be more readily available, including self-supervision and prediction, so that models equipped with online meta-learning can continually improve as they see more of the world.



\bibliography{iclr2022_conference}
\bibliographystyle{iclr2022_conference}

\appendix
\newpage

\section{Appendix}

\subsection{Implementation Details-RainbowMNIST}
\label{sec:A1}
All the images from Rainbow-MNIST are in $28 \times 28$ pixel resolution and with 3 channels. We used a 4 layer convolutional neural network for these experiments, with each layer having [32,32,64,64] number of filters. After every convolution ReLU activation and max pooling is applied to the features. At the last layer we take an average pooling. Finally, 64 dimensional features are passed through fully connected layer with a sigmoid activation. Since the Rainbow-MNIST uses MNIST data to classify, the final layer has 10 neurons, and the objective is to correctly classify the digits of different scales and colors into 10 classes (actual numbers irrespective of the rotation and background color). All the models were trained with 50 gradient updates.

\vspace{-0.4cm}
\subsection{Implementation Details-ONLINE-CIFAR100}
\vspace{-0.2cm}
\label{sec:A2}
All the images from CIFAR100 are in $32 \times 32$ pixel resolution and with 3 channels. We used a similar 7 layer convolutional neural network in a siamese network architecture for these experiments, with each layer having [32,32,32,64,64,64,128] number of filters. After every convolution ReLU activation and batchnorm is applied to the features. Every other layers have max-pooling. At the last layer we take an average pooling. The L2 distance between two images are calculated and a fully connected layer is used to classify the two images as same class or not.

\vspace{-0.4cm}
\subsection{Baseline Methods}
\vspace{-0.2cm}
\label{sec:A3}
\textbf{TFS:} Every time this model sees a new task (This needs to know the task boundary), the model resets to new sets of random weights. After that the model is trained only using the data from the new task. At the same time, we also rest the optimizer state to remove any effect of momentum from previous updates. The model is trained with Adam optimizer with a learning rate of 0.001. 

\textbf{TOE:} Although TOE optimize the parameters on the all available task, training the model using all the available data is practically impossible. Therefore, we sample datapoints from the buffer and update the model parameters. However, since the number of data is increasing over time, the model also needs be updated on a good representative data of the online stream. Therefore, we increase the number of gradient updates over time. For ONLINE-CIFAR100 experiment, the number of gradients updates are increased by 10 for every 100 tasks. The model is trained with Adam optimizer with a learning rate of 0.001. 

\textbf{FTL:} This method, first pretrains a model on the past data, and then fine-tune it on the very recent samples. Similar to TOE, we first train a set of weights using the datapoints in the data buffer. After that, the model is fine-tuned on the correct task data. However, after this adaptation and evaluation the fine-tuned weights are simply discarded and the pretraining starts from the initial weights of the adaptation process. Adam optimizer with same configuration is used to train this model. 

\textbf{FTML:} The FTML updates the meta-parameters using MAML style update. For that, we trained the FTML with 5 inner-loop updates (each inner-loop have task-specific data for the inner-loop adaptation). After the meta-update, the meta-parameters are used as the initialization for the inner-loop adaptation, and similar to FTL the adapted weights are simply discarded after evaluation. The inner-loop is trained with SGD with a learning rate of 0.001 and the outer-loop is trained with a learning rate of 0.0005.

\textbf{FOML:} Our method also have two sets of parameter vectors, thus two different optimizers. We use Adam for both online and meta updates with a learning rate of 0.001 for both cases. Since, our online updates share the meta-knowledge via the regularization term, the $\beta_1$ controls how much pull is applied to the meta parameters by the online parameters. We use 0.01 for $\beta_1$. Similarly the pull on meta-parameters by the online parameters is controlled by $\beta_2$, and we use 0.001 in our experiments.

\vspace{-0.4cm}
\subsection{Meta updates}
\vspace{-0.2cm}
\label{sec:A4}

The full meta update we employ in our complete implementation includes an additional regularization term, in addition to the loss, such that the full meta-update is given by the following equation:
\begin{align*}
    \theta = \theta - \alpha_2 \nabla_{\theta} \left\{ \mathcal{L}(\phi^j_{\theta}; \mathcal{D}_{val}^m) + \beta_2 \sum_{k=0}^K\mathcal{R}(\theta, \phi^{j-k}_{\theta}) \right\} 
\end{align*}
Note that the gradient of $\mathcal{L}(\phi^j_{\theta}; \mathcal{D}_{val}^m)$ with respect to $\theta$ also contains gradients (with respect to $\theta$) for each $\mathcal{R}(\theta, \phi^{j-k}_{\theta})$, so the additional second term simply increases the weights on these gradients by an additional $\beta_2$ factor. While we found this to work well in our experiments, we did not ablate this design decision in detail, and we do not believe it is essential to good performance.


\end{document}
