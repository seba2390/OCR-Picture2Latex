%%%%%%%%%%%%%%%%%%%%%%%%%%%%%%%%%%%%%%%%%%%%%%%%%%%%%%%%%%%%%%%%%%%%%%%%%%%%%%%%
%2345678901234567890123456789012345678901234567890123456789012345678901234567890
%        1         2         3         4         5         6         7         8
%\documentclass[a4paper,10pt,twocolumn]{IEEEtran}
%%\documentclass[letterpaper, 10pt,conference]{ieeeconf}  % Comment this line out if you need a4paper
%%\documentclass[a4paper, 10pt, conference]{ieeeconf}      % Use this line for a4 paper
%
%\IEEEoverridecommandlockouts                              % This command is only needed if
%                                                          % you want to use the \thanks command
%
%\overrideIEEEmargins                                      % Needed to meet printer requirements.
%
%% See the \addtolength command later in the file to balance the column lengths
%% on the last page of the document

\documentclass[letterpaper, 10pt, conference]{ieeeconf}  % Comment this line out if you need a4paper

%\documentclass[a4paper, 10pt, conference]{ieeeconf}      % Use this line for a4 paper

\IEEEoverridecommandlockouts                              % This command is only needed if
                                                          % you want to use the \thanks command

\overrideIEEEmargins



%%%%%%%%%%%%%%%%%%%%%%%%%%%%%%%%%%%%%%%%%%%%%%%%%%%%%%%%%%%%%%%%%%%%%%%%%%%%%
\usepackage{amsmath}
\usepackage{amssymb}
\usepackage{amsfonts}
%\usepackage{amsthm}
\usepackage{mathptmx} % assumes new font selection scheme installed
\usepackage{times}
\usepackage{graphicx}
%\usepackage{relsize}
%\usepackage{color}
%\usepackage{refcheck}

%\setlength{\textwidth}{185truemm}
%\setlength{\textheight}{235truemm}
%\setlength{\oddsidemargin}{-10truemm}
%\setlength{\evensidemargin}{-2mm}
%\setlength{\topmargin}{-7truemm}
\DeclareMathAlphabet{\bit}{OML}{cmm}{b}{it}
\newtheorem{lem}{Lemma}
\newtheorem{thm}{Theorem}
\newtheorem{definition}{Definition}
%\newtheorem{proposition}{Proposition}
\newtheorem{corollary}{Corollary}
%\newtheorem{heuristic}{Heuristic}


\newtheorem{example}{Example}


%\documentclass[12pt,onecolumn]{article}
%%\documentclass[12pt,draftcls,onecolumn]{IEEEtran}
%%\documentclass[a4paper, 12pt, onecolumn, conference]{ieeeconf}
%%\IEEEoverridecommandlockouts                              % This command is only
%%                                                          % needed if you want to
%%                                                          % use the \thanks command
%%\overrideIEEEmargins
%%\addtolength{\topmargin}{0mm}
%
%\setlength{\textwidth}{160truemm}
%\setlength{\textheight}{235truemm}
%\setlength{\oddsidemargin}{1truemm}
%\setlength{\evensidemargin}{1.6mm}
%\setlength{\topmargin}{-9.3truemm}


%\usepackage{amsmath}
%\usepackage{amssymb}
%\usepackage{amsfonts}
%\usepackage{graphicx}
%\usepackage{color}
%\usepackage{times}
%%\usepackage{mathptmx} % assumes new font selection scheme installed
%\usepackage{datetime}
%\usepackage{framed}

%\setlength{\textwidth}{165truemm}
%\setlength{\textheight}{240truemm}
%\setlength{\oddsidemargin}{-2truemm}
%\setlength{\evensidemargin}{1.6mm}
%\setlength{\topmargin}{-15truemm}

\usepackage{datetime}
% The following packages can be found on http:\\www.ctan.org
%\usepackage{graphics} % for pdf, bitmapped graphics files
%\usepackage{epsfig} % for postscript graphics files
%\usepackage{mathptmx} % assumes new font selection scheme installed
%\usepackage{times} % assumes new font selection scheme installed
%\usepackage{amsmath} % assumes amsmath package installed
%\usepackage{amssymb}  % assumes amsmath package installed
\usepackage{framed} % for pdf, bitmapped graphics files
\usepackage{graphicx} % for pdf, bitmapped graphics files
%\usepackage{epsfig} % for postscript graphics files
%\usepackage{txfonts} % assumes new font selection scheme installed
%\usepackage{mathptmx} % assumes new font selection scheme installed
%\usepackage{times} % assumes new font selection scheme installed
\usepackage{amsmath} % assumes amsmath package installed
\usepackage{amssymb}  % assumes amsmath package installed
%%%%%%%%%%%%%%%%%%%%%%%%%%%%%%%%%%%%%%%%%%%%%%%%%%%%%%%%%%%%%%%%%%%%%%%%%%%%%%%


%\newtheorem{lem}{Lemma}
%\newtheorem{thm}{Theorem}
%\newtheorem{proposition}{Proposition}
%\newtheorem{corollary}{Corollary}
%\newtheorem{heuristic}{Heuristic}

%\def\proof{\paragraph*{\bf Proof}}
\def\endproof{{$\blacksquare$\bigskip}}


\def\fa{\mathfrak{a}}
\def\fb{\mathfrak{b}}

%%%%%%%%%%%%%%%%%%%%%%%%%%%%%%%%%%%%%%%%%%%%%%%%%%%%%%%%%%%%%%%%%%%%%%%%%%%%%%%
\def\<{\leqslant}           % nice less than or equal to sign
\def\>{\geqslant}           % nice larger than or equal to sign
\def\div{\mathrm{div}}         % divergence
\def\endremark{{$\blacktriangle$\bigskip}}
%%%%%%%%%%%%%%%%%%%%%%%%%%%%%%%%%%%%%%%%%%%%%%%%%%%%%%%%%%%%%%%%%%%%%%%%%%%%%%%
\def\d{\partial}
\def\wh{\widehat}
\def\wt{\widetilde}


\def\Re{\mathrm{Re}}   % real part
\def\Im{\mathrm{Im}}   % imaginary part

%%%%%%%%%%%%%%%%%%%%%%%%%%%%%%%%%%%%%%%%%%%%%%%%%%%%%%%%%%%%%%%%%%%%%%%%%%%%%%%
\def\col{\mathrm{vec}}   % vectorization of matrices
\def\cH{\mathcal{H}}   % Hardy space
\def\mA{\mathbb{A}}    % space of real antisymmetric matrices
\def\mK{\mathbb{K}}    % class of controllers
\def\mZ{\mathbb{Z}}    % set of integers
\def\mN{\mathbb{N}}    % set of positive integers
\def\mR{\mathbb{R}}    % real line
\def\mC{\mathbb{C}}    % complex plane

%%%%%%%%%%%%%%%%%%%%%%%%%%%%%%%%%%%%%%%%%%%%%%%%%%%%%%%%%%%%%%%%%%%%%%%%%%%%%%%
\def\llaw{{[\![}}       % probability law
\def\rlaw{{]\!]}}       % probability law
\def\Tr{\mathrm{Tr}}       % matrix trace
\def\rT{\mathrm{T}}        % matrix transpose
\def\rB{\rm{B}}        % matrix transpose
\def\rF{\mathrm{F}}        % matrix transpose
\def\diam{\diamond}       % matrix trace
%%%%%%%%%%%%%%%%%%%%%%%%%%%%%%%%%%%%%%%%%%%%%%%%%%%%%%%%%%%%%%%%%%%%%%%%%%%%%%%
\def\bS{\mathbf{S}}
\def\bSigma{{\bf \Sigma}}
\def\bXi{{\bf \Xi}}
\def\bOmega{{\bf \Omega}}
\def\sGamma{{\bf \Gamma}}
\def\sPi{{\bf \Pi}}

\def\bP{\mathbf{P}}    % probability
\def\bE{\mathbf{E}}    % expectation

\def\bK{\mathbf{K}}    % cumulant

%%%%%%%%%%%%%%%%%%%%%%%%%%%%%%%%%%%%%%%%%%%%%%%%%%%%%%%%%%%%%%%%%%%%%%%%%%%%%%%
\def\[[[{[\![\![}
\def\]]]{]\!]\!]}

\def\bra{{\langle}}
\def\ket{{\rangle}}

\def\Bra{\left\langle}
\def\Ket{\right\rangle}

\def\dbra{\langle\!\!\langle}
\def\dket{\rangle\!\!\rangle}

\def\dBra{\big\langle\!\!\big\langle}
\def\dKet{\big\rangle\!\!\big\rangle}


\def\dBRA{\Big\langle\!\!\!\Big\langle}
\def\dKET{\Big\rangle\!\!\!\Big\rangle}


%%%%%%%%%%%%%%%%%%%%%%%%%%%%%%%%%%%%%%%%%%%%%%%%%%%%%%%%%%%%%%%%%%%%%%%%%%%%%%%

\def\re{\mathrm{e}}        % number e
\def\rd{\mathrm{d}}        % differential

%%%%%%%%%%%%%%%%%%%%%%%%%%%%%%%%%%%%%%%%%%%%%%%%%%%%%%%%%%%%%%%%%%%%%%%%%%%%%%%

\def\fL{{\mathfrak L}}
\def\fM{{\mathfrak M}}
\def\fN{{\mathfrak N}}
\def\fO{{\mathfrak O}}

\def\cL{{\mathcal L}}

%%%%%%%%%%%%%%%%%%%%%%%%%%%%%%%%%%%%%%%%%%%%%%%%%%%%%%%%%%%%%%%%%%%%%%%%%%%%%%%
\def\bQ{\mathbf{Q}}

\def\bV{\mathbf{V}}
\def\bA{\mathbf{A}}
\def\bB{\mathbf{B}}
\def\bC{\mathbf{C}}
\def\bD{\mathbf{D}}

\def\bc{\mathbf{c}}
\def\bb{\mathbf{b}}
\def\bN{\mathbf{N}}
\def\bR{\mathbf{R}}

\def\bJ{\mathbf{J}}





\def\br{\mathbf{r}}
\def\x{\times}
\def\ox{\otimes}

\def\fA{{\mathfrak A}}
\def\fB{{\mathfrak B}}
\def\fC{{\mathfrak C}}
\def\fE{{\mathfrak E}}
\def\fF{{\mathfrak F}}
\def\fG{{\mathfrak G}}
\def\fS{\mathfrak{S}}
\def\fP{{\mathfrak P}}
\def\fQ{{\mathfrak Q}}

\def\mG{\mathbb{G}}
\def\mL{\mathbb{L}}
\def\cZ{{\mathcal Z}}
\def\fZ{{\mathfrak Z}}

\def\W{\mathbb{W}}
\def\mP{\mathbb{P}}

\def\om{{\ominus}}


\def\Var{\mathbf{Var}}

\def\bO{\mathbf{0}}
\def\bone{\mathbf{1}}
\def\bzero{\mathbf{0}}
\def\sK{\mathsf{K}}
\def\sL{\mathsf{L}}
\def\sM{\mathsf{M}}
\def\sH{\mathsf{H}}
\def\sA{\mathsf{A}}

\def\sB{\mathsf{B}}

\def\sC{\mathsf{C}}

\def\sD{\mathsf{D}}


\def\sX{\mathsf{X}}

\def\sY{\mathsf{Y}}

\def\sU{\mathsf{U}}

\def\sV{\mathsf{V}}

\def\sG{\mathsf{G}}






\def\bH{\mathbf{H}}
\def\bh{\mathbf{h}}

\def\bI{\mathbf{I}}
\def\bT{\mathbf{T}}
\def\bU{\mathbf{U}}
\def\bu{\mathbf{u}}

\def\cF{\mathcal{F}}
\def\cW{{\mathcal W}}
\def\fW{{\mathfrak W}}
\def\cU{\mathcal{U}}
\def\cX{\mathcal{X}}
\def\cY{\mathcal{Y}}
\def\fX{\mathfrak{X}}
\def\fU{\mathfrak{U}}
\def\mF{\mathbb{F}}
\def\cK{\mathcal{K}}


\def\cJ{{\mathcal J}}
\def\cD{{\mathcal D}}

\def\cM{\mathcal{M}}
\def\cV{\mathcal{V}}
\def\cC{\mathcal{C}}
\def\cR{\mathcal{R}}

\def\sB{{\sf B}}
\def\sC{{\sf C}}
\def\sE{{\sf E}}
\def\sP{{\sf P}}

\def\cG{\mathcal{G}}
\def\cI{\mathcal{I}}
\def\cP{{\mathcal P}}
\def\cQ{{\mathcal Q}}
\def\fY{\mathfrak{Y}}

\def\cA{\mathcal{A}}
\def\cB{\mathcal{B}}
\def\cE{\mathcal{E}}
\def\cov{\mathbf{cov}}
\def\var{\mathbf{var}}



\def\cN{\mathcal{N}}
\def\cS{\mathcal{S}}
\def\cT{\mathcal{T}}

\def\bG{\mathbf{G}}
\def\bL{\mathbf{L}}



\def\F{\mathbb{F}}
\def\mM{\mathbb{M}}
\def\mX{\mathbb{X}}
\def\mY{\mathbb{Y}}
\def\mU{\mathbb{U}}
\def\mV{\mathbb{V}}

\def\im{\mathrm{im\,}}
\def\ker{\mathrm{ker\,}}

\def\mH{\mathbb{H}}
\def\mS{\mathbb{S}}
\def\mT{\mathbb{T}}
\def\mZ{\mathbb{Z}}




\def\veps{\varepsilon}
\def\eps{\epsilon}
\def\Ups{\Upsilon}

\def\sn{|\!|\!|}

\def\an{|\!|\!|}


\def\diag{\mathop{\mathrm{diag}}}    % diagonal matrix
\def\blockdiag{\mathop{\rm blockdiag}}    % diagonal matrix
\def\esssup{\mathop{\mathrm{ess\, sup}}}    % diagonal matrix
\def\essinf{\mathop{\mathrm{ess\, inf}}}    % diagonal matrix
%%%%%%%%%%%%%%%%%%%%%%%%%%%%%%%%%%%%%%%%%%%%%%%%%%%%%%%%%%%%%%%%%%%%%%%%%%%%%%%%%%%%%%%%%%%%%%%%%%%

%\title{\bf\Large  Convergence to equilibrium in Smoluchowski and stochastic  Hamiltonian systems with fractal-like energy landscapes on multidimensional tori}


%%%%%%%%%%%%%%%%%%%%%%%%%%%%%%%%%%%%%%%%%%%%%%%%%%%%%%%%%%%%%%%%%%%%%%%%%%%%%%%%%%%%%%%%%%%%%%%%%%%
%\title{\LARGE \bf
%Energy-related Properties of Translation Invariant Lattice Networks of Interacting Linear
%Systems in the Spatio-temporal Frequency Domain$^*$}

\title{\LARGE \bf
Spatio-temporal Transfer Function Conditions of Positive Realness for Translation Invariant Lattice Networks of Interacting Linear
Systems$^*$}

%Coherent Quantum
%Observers, Coupling-Estimation Inequalities and Dispersion Relations in Networks of Quantum Harmonic Oscillators$^*$

\usepackage{datetime}

\author{%Working draft by: \\
Igor G. Vladimirov$^{\dagger}$,
\qquad
Ian R. Petersen$^{\dagger}$\\%
%\today, \currenttime% <-this % stops a space
\thanks{$^*$This work is supported by the Australian Research Council under grant DP160101121.}
\thanks{$^\dagger$Research School of Engineering, College of Engineering and Computer Science, Australian National University, ACT 2601, Canberra, Australia.
{\tt igor.g.vladimirov@gmail.com}, {\tt i.r.petersen@gmail.com}.
}
}

\pagestyle{plain}
\begin{document}
\maketitle
\thispagestyle{empty}
%\pagestyle{empty}

\begin{abstract}
This paper is concerned with networks of interacting linear systems at sites of a multidimensional lattice. The systems are governed by linear ODEs with constant coefficients driven by external inputs, and their internal dynamics and coupling with the other component systems are translation invariant. Such systems occur, for example, in  finite-difference models of large-scale flexible structures manufactured from homogeneous materials. Using the spatio-temporal transfer function of this translation invariant network, we establish conditions for its positive realness in the sense of energy dissipation. The latter is formulated in terms of block Toeplitz bilinear forms of the input and output variables of the composite system. We also discuss quadratic stability of the network in isolation from the environment
%, links with multivariate sampling theorems  in the case of finite-range coupling,
and phonon theoretic dispersion relations.
%
%We also apply this result to a class of systems which are organised as a feedback loop with a linear part and a static nonlinearity.
\end{abstract}

%\begin{keywords}
%Quantum systems,
%risk-sensitive criteria,
%quadratic-exponential functionals,
%uncertain quantum states,
%quantum relative entropy,
%robustness to state uncertainty.
%\end{keywords}
%{\it MSC codes} ---
%90B10,   	%Network models, deterministic
%37L60,%   	Lattice dynamics
%37K05.%   	Hamiltonian structures, symmetries, variational principles, conservation laws
%81S25,   	%Quantum stochastic calculus
%\and
%81S30       % Phase-space methods including Wigner distributions, etc.
%81S05,       %Canonical quantization, commutation relations and statistics
%81S22,       % Open systems, reduced dynamics, master equations, decoherence
%81P16,   	%Quantum state spaces, operational and probabilistic concepts
%81P40,   	%Quantum coherence, entanglement, quantum correlations
%81Q93,   	%Quantum control
%81Q10,   	%Selfadjoint operator theory in quantum theory, including spectral analysis
%60G15,   	% Gaussian processes
%93E20      % Optimal stochastic control

\begin{keywords}
Translation invariant networks,
spatio-temporal frequency domain,
energy balance relations. %dissipation,


\emph{MSC Codes} ---
93C05,   	%Linear systems
90B10,   	%Network models, deterministic
37L60,      %Lattice dynamics
37K05,      %Hamiltonian structures, symmetries, variational principles, conservation laws
93C80.   	%Frequency-response methods
\end{keywords}

%\paragraph*{Key words and phrases}
%Smoluchowski equation, stochastic Hamiltonian system, Fokker-Planck-Kolmogorov equation, multidimensional torus, invariant measure, relaxation dynamics, convergence rate, relative entropy, fractal energy landscape,  scaling laws.

%%%%%%%%%%%%%%%%%%%%%%%%%%%%%%%%%%%%%%%%%%%%%%%%%%%%%%%%%%%%%%%%%%%%%%%%%%%%%%%%%%%%%%%%%%%%%%%%%%%
\section{INTRODUCTION}
%%%%%%%%%%%%%%%%%%%%%%%%%%%%%%%%%%%%%%%%%%%%%%%%%%%%%%%%%%%%%%%%%%%%%%%%%%%%%%%%%%%%%%%%%%%%%%%%%%%

Complex physical systems can be viewed as a large number of relatively simple subsystems whose collective behaviour is a cumulative effect of their interaction rather than a particular individual  structure. Spatially homogeneous states of matter are modelled as identical building blocks which interact with each other in a translationally invariant fashion. A natural example of large-scale composite systems with translational symmetry is provided by crystalline solids, where spatially periodic arrangements of constituent particles result from their interaction and play an important role in their thermodynamic and mechanical properties (including the heat transfer and wave propagation) studied in the phonon theory  \cite{S_1990}.

Modern engineering exploits
translation  invariant interconnections  in drone swarming, vehicle  platooning and artificially fabricated metamaterials   \cite{VBSH_2006}, such as split ring resonator arrays with unusual electrodynamic characteristics (a negative refraction index). Nontrivial input-output properties of such networks of systems (natural or artificial) are not merely a ``sum'' of individual internal dynamics of their constituent blocks  and come from a specific structure of energy flows through the translation invariant interaction.

The energy balance relations,  which reflect the energy conservation and dissipation in isolated and open systems (for example, due to electrical resistance and mechanical friction), significantly affect the behaviour
of physical systems and play an increasingly important role in control design \cite{OVMM_2001,OVME_2002,VJ_2014}. These equations involve the internal energy and the work done on the system (which are represented in the dissipativity theory \cite{W_1972} in terms of storage and  supply rate functions). Work is modelled by using a bilinear form of the input and output variables, which are interpreted as the generalised force and velocity  respectively. For linear time-invariant systems with a finite-dimensional internal state,  the properties of being passive, positive real or negative imaginary (in the case of position variables instead of the velocity as the output) admit criteria in the form of linear matrix inequalities for the transfer functions in the frequency domain or the state-space matrices themselves \cite{PL_2010,XPL_2010}.

The present paper is concerned with similar conditions for networks of interacting linear systems at sites of a multidimensional lattice. The composite system is governed by an infinite set of linear ODEs with constant coefficients driven by external inputs, and their internal dynamics and coupling with the other component systems are translation invariant. These ODEs have block Toeplitz state-space matrices and can be represented in the spatio-temporal frequency domain by using appropriately modified transfer functions of several variables.
Such systems arise, for example, as finite-difference approximations of PDEs for large-scale flexible structures made of spatially homogeneous materials. Using the spatio-temporal transfer function of this translation invariant network, we establish conditions for its positive realness in the sense of energy dissipation. The latter is formulated in terms of block Toeplitz bilinear forms of the input and output variables of the composite system. The multivariate  Laplace and Fourier  transform techniques, which are used for this purpose, are similar to those for distributed control systems in the classical and quantum settings \cite{SVP_2015,VP_2014}.

The paper is organised as follows.
Section~\ref{sec:sys} describes the class of translation invariant networks under consideration.
Section~\ref{sec:freq} represents the network dynamics in terms of the spatial Fourier transforms of its signals.
Section~\ref{sec:bal} discusses energy balance relations in the case of bilinear supply rate and quadratic storage functions.
Section~\ref{sec:pass} establishes conditions for passivity of the network in terms of its spatio-temporal transfer function, and also discusses quadratic stability bounds for a dissipative network in isolation from the environment.
%Section~\ref{sec:stab} discusses quadratic stability bounds for a dissipative network in isolation from the environment.
Section~\ref{sec:phon} considers phonon theoretic dispersion relations for the isolated network.
Section~\ref{sec:conc} provides concluding remarks.

%Section~\ref{sec:posreal} specifies a class of nonlinear open systems with an affine dependence on the input.






%These ideas are taken from the classical macroscopic scale to the quantum level in quantum metamaterials \cite{QSMGH_2011} which are organised as one, two or three-dimensional \cite{Z_2012} periodic arrays of coherently coupled quantum systems. The latter form a fully quantum composite system which does not involve measurements.
%Active research into this kind of artificial materials is inspired by qualitatively new  properties  of light-matter interaction which %do not occur naturally and
%unveil previously hidden resources.
%%, such as overcoming the diffraction limit and invisibility cloaking (at least in the microwave range) \cite{?}.
%An  example is provided by artificial crystals of atoms trapped at nodes of an optical lattice which can be controlled by external fields and used for entanglement generation \cite{CBFFFRCIP_2013} or as a quantum memory \cite{NDMRLLWJ_2010}.
%
%In light of the emerging technology of quantum metamaterials,  the present paper is concerned with the modelling and analysis of the dynamics of translation invariant networks of interacting linear quantum stochastic
%systems which represent open quantum harmonic oscillators. In particular, we are concerned with physical realizability (PR) and mean square performance of this class of large-scale quantum systems from the viewpoint of quantum linear systems theory \cite{P_2010}.
%%
%In the present setting, the quantum systems form a one-dimensional chain or are associated with sites of a multidimensional lattice and are governed by a set of coupled linear quantum stochastic differential equations (QSDEs) in the framework of the Hudson-Parthasarathy noncommutative version of the Ito calculus \cite{HP_1984,P_1992}. Although these QSDEs look similar to the classical Ito SDEs widely used in linear stochastic control  theory \cite{AM_1989,KS_1972}, their coefficients %can not be arbitrary and
%must satisfy certain PR conditions \cite{JNP_2008,NJP_2009,SP_2012}.
%%
%The PR constraints are closely related with the postulate of a unitary evolution \cite{M_1998,S_1994} for isolated quantum systems (for example, those formed from a system of interest and its environment which may involve other quantum systems and external fields). Such an evolution preserves canonical commutation relations (CCRs) between quantum variables and is specified by the ``energetics'' of the underlying system and its interaction with the surroundings.
%%
%The dynamics of interconnected quantum stochastic systems are governed by QSDEs whose drift and diffusion terms are expressed in terms of scattering, coupling and Hamiltonian operators \cite{GJ_2009}.  For linear quantum stochastic systems,   the PR conditions reflect dynamic equivalence to an open quantum harmonic oscillator \cite{EB_2005,GZ_2004} whose dynamic variables satisfy CCRs. Its  Hamiltonian is quadratic and the coupling operator  is linear with respect to the dynamic variables. The resulting PR conditions are organised as quadratic constraints on the state-space matrices of the QSDEs \cite{JNP_2008,NJP_2009,SP_2012}.
%%
%Using the previous results on PR of linear QSDEs, we take advantage of the specific structure of coupled QSDEs for translation  invariant quantum networks with nearest neighbour interaction. This allows PR constraints to be established here in the form of matrix algebraic equations for the parameters of an individual building block of the network and its coupling to its neighbours and external fields.
%%
%Adapting the performance criteria used in the Coherent Quantum Linear Quadratic Gaussian (CQLQG) control/filtering problems \cite{NJP_2009,VP_2013a,VP_2013b}, we also discuss mean square functionals with block Toeplitz weighting matrices whose structure reflects the translation invariance of the quantum network. Under a stability condition, we compute the steady-state value of such a  functional per site for unboundedly increasing fragments of the network. This corresponds to the thermodynamic limit in equilibrium statistical mechanics \cite{R_1978}. The results of the paper can be used for the development of decentralised CQLQG controllers for large-scale quantum networks.
%%The analysis employs spatial Fourier transforms (and related $z$-transforms) which are applied to the  quantum processes of component systems in a finite (but arbitrarily large) fragment of the network. This corresponds to the role Green's functions and their Fourier transforms play in linear partial differential equations with constant coefficients \cite{E_1998}. %,V_1971}.
%%%The same circle of ideas underlies dispersion relations for the propagation of plane waves in homogeneous media.
%%
%%%This is carried out by considering an unboundedly increasing fragment of the network, which corresponds to the thermodynamic limit in statistical mechanics of interacting particle systems \cite{R_1978}. The finite fragment considerations employ spatial Fourier transforms
%%%
%%%We also mention that stability analysis problems
%%%
%%%We also mention that large-scale systems and, in particular,  networked control systems is an area of intensive research in many different non-quantum contexts such as communication networks,   vehicle platooning (or formation flying), to mention a few. Some
%%%
%%%The study employs spatial Fourier transforms similar to those used in the theory of linear partial differential equations with constant coefficients \cite{E_1998}.
%%
%%
%The paper is organized as follows. Section \ref{sec:chain} describes a one-dimensional chain of interacting quantum systems. Section~\ref{sec:ztrans} introduces spatial Fourier transforms of %the underlying
%quantum processes. These are employed  in Section~\ref{sec:PR} to establish PR conditions for the governing QSDEs. % which govern the quantum network.
%Section \ref{sec:quadro} computes mean square performance functionals with block Toeplitz weights in the thermodynamic limit. % per site of the network.
%Section \ref{sec:2dlattice} outlines an extension of the results to the multivariate case.% by considering two-dimensional lattices of quantum systems.

%!!!!!!!!!!!!!!!!!!!!!!!!!!!!!!!!!!!!!!!!!!
%
%Port-Hamiltonian systems \cite{VJ_2014} constitute an important class of open dynamical systems whose structure reflects the energetics of the internal dynamics of a physical system and its interaction with the environment (which is accompanied, in particular, with heat exchange and electromechanical work). The energy balance for such systems is formulated in terms of differential or integral relations which involve the system Hamiltonian together with the input and output consisting of  effort (force or voltage) and flow (velocity or current) variables. The conservation and dissipation of energy provides a natural mechanism for achieving stability of port-Hamiltonian plants by their interconnection with such controllers without the mediation from digital signal processing. Although this approach was present already in early control devices (such as the centrifugal governors for steam engines \cite{M_1868}), a similar control-by-interconnection paradigm is used in modern control for classical systems \cite{OVMM_2001,OVME_2002} and the emerging field of coherent quantum control \cite{DP_2010,JG_2010,ZJ_2011b}.
%
%The dynamic variables of a port-Hamiltonian system are usually formed from the generalised coordinates and momenta in accordance with the symplectic structure matrix which specifies the Poisson bracket. In combination with the Hamiltonian,  the Poisson bracket describes the time derivative of a smooth function of the system variables in isolation from the environment. For an open port-Hamiltonian system, the evolution of such a function acquires additional terms including those due to the energy dissipation and system-environment coupling. In application  to the Hamiltonian itself,  this leads to an  energy balance relation in terms of which the properties of  passivity and positive realness can be formulated \cite{?}. These properties are also of interest for more general open systems with similar interpretations of the input and output variables.
%
%In the present paper, we consider a class of nonlinear systems whose dynamics are governed by ordinary differential equations with an affine dependence on the input. Since it is impossible to exhaust all scenarios of nonlinearity, we represent the nonlinear dependencies (at least approximately)  by linear combinations of a finite number of basis functions forming an algebra. This approach belongs to the general framework of Galerkin methods \cite{?} for the numerical solution of partial differential equations. The Galerkin-type approximation, which is used here, allows the energy balance equations to be represented in terms of the coefficients of the basis functions. In particular, conditions of positive realness for such systems take the form of linear matrix inequalities.
%
%
%
%
%
%
%%It is well-known that the Smoluchowski SDE \cite{?} provides an important class of stochastic systems with a closed-form invariant measure. The dynamics of such systems are organized as a noise-corrupted gradient descent for a potential function. Due to this particular structure, the solutions of the Smoluchowski SDE are used in the simulated annealing (SA) algorithm \cite{?} for finding global minima of multiextremal functions. The presence of diffusion in this algorithm (which is closely related to the idea of thermalization in statistical mechanics of interacting particle systems)  enables its sample paths to avoid being trapped in local minima of the potential. Due to the stochastic nature of SA, the neighbourhoods of local minima become metastable (rather than absorbing) states in the phase space   in contrast to the original deterministic version of the gradient descent method.
%
%%The fact, that the drift term of the Smoluchowski SDE is the antigradient of a potential, allows such a system to be interpreted as a dissipative dynamical system which interacts with a heat bath whose absolute temperature specifies the level of noise. A more realistic class of models of open systems in contact with a noisy environment is provided by stochastic Hamiltonian systems. Their dynamics  in the augmented position-momentum space  are governed by SDEs whose drift consists of the Hamiltonian vector field and a Langevin damping term \cite{?}. In the framework of nonrelativistic mechanics, the Hamiltonian is the sum of a position-dependent quadratic form of the generalised momenta (representing the kinetic energy) and the potential energy which is a function of the generalised coordinates. Although stochastic Hamiltonian models do not capture all physical effects, they  are used in molecular dynamics simulations,  including the modelling of protein folding \cite{?}.
%
%%In the latter context, a natural set of generalised coordinates is provided by dihedral, bond and torsion  angles, which quantify the mutual orientation of peptide bonds in the protein backbone and, considered together,  specify the conformation  of the whole molecule \cite{?} (with the other degrees of freedom, such as the bond lengths, being ignored due to their relatively small variations). The angular nature of the generalised coordinates in this case makes it relevant to consider Smoluchowski and stochastic Hamiltonian models on multidimensional tori. This is convenient from the analytic point of view in that all the functions of the position variables become periodic (thus making Fourier methods natural), the compactness of the torus eliminates many improper integrals from consideration,  and the periodicity simplifies spatial integration by parts. At the same time, more complicated probability distributions on the torus, such as the von Mises-Tikhonov distributions \cite{?}, have to be used instead of, for example,  Gaussian distributions which are typical  for linear stochastic systems (with quadratic Hamiltonians) in Euclidean spaces.
%
%
%%Proteins are known to manifest a remarkable ability of folding to their native conformation (where they can properly carry out their biological functions) in a relatively fast fashion despite the high dimension of the phase space (with hundreds or even thousands of coordinates). This problem is addressed by Levinthal's paradox \cite{?} which suggests that the path of the protein macromolecule conformation  through intermediate states towards equilibrium must be well-organized and selective in order to enable reversibly denatured conformations (which can be far from the equilibrium) to avoid the exhaustive sampling of the whole space. This seems to be  particularly so for multiextremal potential energy landscapes with many metastable traps created by local minima. However, such locations may, in fact, facilitate the protein folding process by forming a hierarchical staircase which works like a spatially distributed sink leading down to the native state. This consideration is reflected in the funnel-like energy landscape hypothesis \cite{?} and has attracted an active research interest to random walks on fractal networks as models for protein folding pathways \cite{LBO_2010}.
%
%
%
%
%%Motivated by the above developments on the protein folding problem, the present paper is concerned with the influence of fractal-like properties of the potential energy landscape on the rate of convergence  to the invariant measure in Smoluchowski and stochastic Hamiltonian systems on multidimensional tori.
%
%
%The paper is organised as follows.
%Section~\ref{sec:portham} describes port-Hamiltonian systems for completeness.
%Section~\ref{sec:posreal} specifies a class of nonlinear open systems with an affine dependence on the input.
%Section~\ref{sec:Gal} describes Galerkin  approximations of the nonlinearities.
%
%
%
%!!!!!!!!!!!!!!!!!!!!!!!!!!!!!!!!!!!!!!!!!
%

%%%%%%%%%%%%%%%%%%%%%%%%%%%%%%%%%%%%%%%%%%%%%%%%%%%%%%%%%%%%%%%%%%%%%%%%%%%%%%%%%%%%%%%%%%%%%%%%%%%
\section{TRANSLATION INVARIANT NETWORKS%BEING CONSIDERED
}\label{sec:sys}
%%%%%%%%%%%%%%%%%%%%%%%%%%%%%%%%%%%%%%%%%%%%%%%%%%%%%%%%%%%%%%%%%%%%%%%%%%%%%%%%%%%%%%%%%%%%%%%%%%%

We consider a translation invariant network of coupled linear systems at sites of the $\nu$-dimensional integer lattice $\mZ^{\nu}$. For any spatial index $k \in \mZ^{\nu}$, the $k$th system is endowed with an $\mR^n$-valued  vector $x_k$ of time-varying state variables (for example, the generalised positions and velocities). Associated with the $k$th lattice site are vectors $u_k$ and $y_k$  of external input and output variables which take values in $\mR^m$ and $\mR^r$, respectively, and also vary in time (unless indicated otherwise, vectors are organised as columns).
The states and outputs of these systems are governed by an infinite  set of coupled ODEs
\begin{align}
\label{xj}
  \dot{x}_j
  & =
  \sum_{k \in \mZ^\nu}
  (A_{j-k} x_k + B_{j-k} u_k),\\
\label{yj}
  y_j
  & =
  \sum_{k\in \mZ^\nu}
  (C_{j-k} x_k + D_{j-k} u_k)
\end{align}
for all $j\in \mZ^\nu$, where
$\dot{(\,)}$ is the time derivative (the time arguments are often omitted for brevity). Their right-hand sides are organised as convolutions, with the matrices $A_\ell \in \mR^{n\x n}$, $B_\ell\in \mR^{n\x m}$, $C_\ell\in \mR^{r\x n}$, $D_\ell \in \mR^{r\x m}$ depending on the relative location $\ell \in \mZ^\nu$ of the lattice sites in accordance with the translation invariance of the individual dynamics  of the systems and their coupling.
By assembling the inputs, states and outputs into the infinite-dimensional vectors $u:= (u_k)_{k \in \mZ^\nu}$, $x:= (x_k)_{k \in \mZ^\nu}$, $y:= (y_k)_{k \in \mZ^\nu}$, the set of ODEs (\ref{xj}) and (\ref{yj}) can be written as
\begin{align}
\label{x}
  \dot{x}
  & =
  A x + B u,\\
\label{y}
  y
  & =
  Cx+Du,
\end{align}
where the matrices $A:= (A_{j-k})_{j,k \in \mZ^\nu}$, $B:= (B_{j-k})_{j,k \in \mZ^\nu}$, $C:= (C_{j-k})_{j,k \in \mZ^\nu}$,  $D:= (D_{j-k})_{j,k \in \mZ^\nu}$ are block Toeplitz in the sense of the additive group structure of the lattice $\mZ^\nu$. The ODE (\ref{x}) is understood as a Volterra integral equation (of the second kind)
\begin{equation}
\label{xint}
    x(t)
    =
    \int_0^t
    (A x(\tau) + B u(\tau))
    \rd \tau
\end{equation}
whose solution is given by
\begin{equation}
\label{xsol}
    x(t)
    =
    \re^{tA} x(0) + \int_0^t \re^{(t-\tau) A} B u(\tau)\rd \tau,
    \qquad
    t\>0.
\end{equation}
For completeness, we note that (\ref{x})--(\ref{xsol}) can be endowed with a rigorous meaning as follows. To this end,
the state-space matrices in (\ref{xj}) and (\ref{yj}) are assumed to be square summable:
\begin{equation}
\label{ABCDgood}
    \sum_{\ell\in \mZ^\nu}
    \left\|
  \begin{bmatrix}
    A_\ell & B_\ell\\
    C_\ell & D_\ell
  \end{bmatrix}
    \right\|^2
%    (\|A_k\|^2 +
%    \|B_k\|^2 +
%    \|C_k\|^2 +
%    \|D_k\|^2)
    <+\infty,
\end{equation}
where $\|\cdot\|$ is an arbitrary matrix norm (whose particular choice is irrelevant in this case).  The fulfillment of (\ref{ABCDgood}) allows the Fourier series
\begin{equation}
\label{ABCD}
  \begin{bmatrix}
    \sA(\sigma) & \sB(\sigma)\\
    \sC(\sigma) & \sD(\sigma)
  \end{bmatrix}
  :=
  \sum_{\ell\in \mZ^\nu}
  \re^{-i\ell^{\rT}\sigma}
  \begin{bmatrix}
    A_\ell & B_\ell\\
    C_\ell & D_\ell
  \end{bmatrix},
\end{equation}
%\begin{align}
%\label{cA}
%    \sA(\sigma)
%    & :=
%    \sum_{k\in \mZ^\nu}
%    \re^{-ik^{\rT}\sigma}
%    A_k,\\
%\label{cB}
%    \sB(\sigma)
%    & :=
%    \sum_{k\in \mZ^\nu}
%    \re^{-ik^{\rT}\sigma}
%    B_k,\\
%\label{cC}
%    \sC(\sigma)
%    & :=
%    \sum_{k\in \mZ^\nu}
%    \re^{-ik^{\rT}\sigma}
%    C_k,\\
%\label{cD}
%    \sD(\sigma)
%    & :=
%    \sum_{k\in \mZ^\nu}
%    \re^{-ik^{\rT}\sigma}
%    D_k
%\end{align}
to be defined (in a blockwise fashion)
for almost all $\sigma \in \mR^\nu$ as the $L^2$-limit of appropriate partial sums over finite subsets of the lattice forming an exhausting sequence. Since the matrix-valued  functions $\sA$, $\sB$, $\sC$, $\sD$ (with values in $\mC^{n\x n}$, $\mC^{n\x m}$, $\mC^{r\x n}$, $\mC^{r\x m}$, respectively)
are $2\pi$-periodic with respect to each of their arguments, then, without loss of generality,  they can be considered on a $\nu$-dimensional torus $\mT^\nu$ (where $\mT$ can be  identified with the interval $[-\pi,\pi)$). These functions are Hermitian in the sense that $\overline{\sA(\sigma)} = \sA(-\sigma)$ (and similarly for the other functions $\sB$, $\sC$, $\sD$) for all $\sigma \in \mR^\nu$, where $\overline{(\cdot )}$ is the complex conjugate.  Furthermore, let
$
    \|u\|_2
    :=
    \sqrt{\sum_{k \in \mZ^\nu} |u_k|^2}$,
$
    \|x\|_2
    := \sqrt{\sum_{k \in \mZ^\nu} |x_k|^2}$,
$\|y\|_2
    :=
    \sqrt{\sum_{k \in \mZ^\nu} |y_k|^2}
$
denote the norms of the input, state and output of the network at a fixed but otherwise arbitrary moment of time $t\>0$ in the corresponding Hilbert spaces
\begin{equation}
\label{fU_fX_fY}
    \fU
    := L^2(\mZ^\nu, \mR^m),
    \
     \fX
     :=
     L^2(\mZ^\nu,\mR^n),
     \
     \fY
     :=
     L^2(\mZ^\nu, \mR^r)
\end{equation}
of appropriately dimensioned square summable real vector-valued functions $f:= (f_k)_{k \in \mZ^\nu}$ and $g:=(g_k)_{k \in \mZ^\nu}$ on the lattice $\mZ^\nu$ with the inner product $\bra f, g\ket := \sum_{k\in \mZ^\nu} f_k^\rT g_k$. Now, suppose the state-space matrices in (\ref{x}) and (\ref{y})  specify bounded linear operators $A: \fX \to \fX$, $B:\fU\to \fX$, $C:\fX\to \fY$, $D:\fU\to \fY$. This is equivalent to their $L^2$-induced operator norms being finite:
\begin{align}
\label{Anorm}
    \|A\|_{\infty}
    & =
    \esssup_{\sigma \in \mT^\nu}
    \|\sA(\sigma)\|
    <+\infty,\\
\label{Bnorm}
    \|B\|_{\infty}
    & =
    \esssup_{\sigma \in \mT^\nu}
    \|\sB(\sigma)\|
    <+\infty,\\
\label{Cnorm}
    \|C\|_{\infty}
    & =
    \esssup_{\sigma \in \mT^\nu}
    \|\sC(\sigma)\|
    <+\infty,\\
\label{Dnorm}
    \|D\|_{\infty}
    & =
    \esssup_{\sigma \in \mT^\nu}
    \|\sD(\sigma)\|
    <+\infty,
\end{align}
where the essential supremum is applied to the operator  norms of the matrices in (\ref{ABCD}).
In particular, if the matrices  $A_\ell$, $B_\ell$, $C_\ell$, $D_\ell$  vanish for all $\ell \in \mZ^\nu$ with $|\ell|$ large enough (so that each of the component systems in (\ref{xj}) and (\ref{yj}) is coupled with a finite number of the other systems in the network), then the functions $\sA$, $\sB$, $\sC$, $\sD$ in (\ref{ABCD}) are multivariate trigonometric polynomials, and the conditions (\ref{Anorm})--(\ref{Dnorm}) are satisfied in this case.  %(which will be discussed in Section~\ref{sec:fin}).
In general, the fulfillment of (\ref{Anorm}) and (\ref{Bnorm}) guarantees that the operator exponential $\re^{tA}$ and the product $\re^{(t-\tau)A}B$ in (\ref{xsol}) are bounded block Toeplitz   operators acting on the Hilbert space $\fX$ and from $\fU$ to $\fX$, respectively, with
\begin{equation}
\label{bounds}
    \|\re^{tA}\|_{\infty} \< \re^{\|A\|_{\infty} t},
    \qquad
    \|\re^{(t-\tau)A} B\|_{\infty} \< \re^{\|A\|_{\infty}(t-\tau)}\|B\|_{\infty}
\end{equation}
for any $t\> \tau \> 0$, in view of
the submultiplicativity of the operator norm and the fact that block Toeplitz matrices form an algebra.
Therefore, if the initial network state is square summable, that is,
\begin{equation}
\label{x0norm}
    \|x(0)\|_2 < +\infty,
\end{equation}
and the network input is locally absolutely integrable with respect to time in the sense that
\begin{equation}
\label{ugood}
    \int_0^T
    \|u(t)\|_2
    \rd t
    <+\infty,
    \qquad
    T>0,
\end{equation}
then these properties are inherited by the subsequent states of the network. Indeed, a combination of (\ref{xsol}) with (\ref{bounds})--(\ref{ugood}) leads to
 \begin{equation}
\label{xgood}
    \|x(t)\|_2
    \<
%    \|x(0)\|_2
%    +
%    \int_0^T
%    \|\dot{x}(t)\|_2
%    \rd t\\
%\nonumber
%    & \<
%    \|x(0)\|_2
%    +
%    \|A\|_{\infty}
%    \int_0^T
%    \|x(t)\|_2
%    \rd t
%    +
%    \|B\|_{\infty}
%    \int_0^T
%    \|u(t)\|_2
%    \rd t\\
    \re^{\|A\|_{\infty}t}
    \Big(
        \|x(0)\|_2
        +
    \|B\|_{\infty}
    \int_0^t
    \|u(\tau)\|_2
    \rd \tau
    \Big)
    <+\infty
\end{equation}
for all $t\>0$, and hence,
 \begin{align}
\nonumber
    \int_0^T
    \|x(t)\|_2
    \rd t
    \< &
    T
    \re^{\|A\|_{\infty}T}
    \Big(
        \|x(0)\|_2\\
\label{xgood1}
        & +
    \|B\|_{\infty}
    \int_0^T
    \|u(\tau)\|_2
    \rd \tau
    \Big)
    <+\infty
\end{align}
for any $T>0$.
%\begin{align}
%\nonumber
%    (\|x\|_2^2)^{^\centerdot}
%    & =
%    2\bra x, \dot{x}\ket\\
%\nonumber
%    & =
%    2\bra x, Ax + Bu\ket\\
%\nonumber
%    & \<
%    2\|x\|_2\|Ax\|_2
%    +
%    2\bra x, Bu\ket\\
%\nonumber
%    & \<
%    2\|A\|_\infty\|x\|_2^2
%    +
%    \|x\|_2^2 + \|Bu\|_2^2\\
%\nonumber
%    & \<
%    (2\|A\|_\infty + 1)\|x\|_2^2
%    +
%    \|B\|_\infty^2\|u\|_2^2,
%\end{align}
%where use is made of (\ref{x}) and the triangle inequality.
%, thus
%leading to the upper bound
%\begin{align}
%\nonumber
%    \|x(T)\|_2^2
%    \< &
%    \re^{(2\|A\|_\infty + 1)T}\\
%\label{xgood}
%    & \x \Big(
%        \|x(0)\|_2^2
%        +
%        \|B\|_\infty^2
%        \int_0^T
%        \|u(t)\|_2^2
%        \rd t
%    \Big)
%    <+\infty
%\end{align}
%for all $T>0$ .
Together with (\ref{y}),  (\ref{Cnorm}), (\ref{Dnorm}),  the property (\ref{xgood}) implies that the output norm $\|y(t)\|_2$ is finite for almost all $t>0$ since so is $\|u(t)\|_2$ in view of (\ref{ugood}). Furthermore, due to (\ref{xgood1}), the network output is also locally absolutely integrable with respect to time:
\begin{align}
\nonumber
    \int_0^T
    \|y(t)\|_2
    \rd t
    \< &
    \|C\|_{\infty}
    \int_0^T
    \|x(t)\|_2
    \rd t\\
\label{ygood}
    & +
    \|D\|_{\infty}
    \int_0^T
    \|u(t)\|_2
    \rd t
    <+\infty
\end{align}
for any $T>0$. In Section~\ref{sec:bal}, we will replace (\ref{ugood}) with a stronger condition on the network inputs in order to guarantee time-local square integrability for the outputs instead of (\ref{ygood}).




%%%%%%%%%%%%%%%%%%%%%%%%%%%%%%%%%%%%%%%%%%%%%%%%%%%%%%%%%%%%%%%%%%%%%%%%%%%%%%%%%%%%%%%%%%%%%%%%%%%
\section{NETWORK DYNAMICS IN THE SPATIAL FREQUENCY DOMAIN}\label{sec:freq}
%%%%%%%%%%%%%%%%%%%%%%%%%%%%%%%%%%%%%%%%%%%%%%%%%%%%%%%%%%%%%%%%%%%%%%%%%%%%%%%%%%%%%%%%%%%%%%%%%%%


The preservation of the spatial square summability (\ref{xgood}) (provided the input satisfies (\ref{ugood})) allows the network dynamics (\ref{x}) and (\ref{y}) to be represented in the spatial frequency domain as
\begin{align}
\label{X}
  \d_t X(t,\sigma)
  & =
  \sA(\sigma) X(t,\sigma) + \sB(\sigma) U(t,\sigma),\\
\label{Y}
  Y(t,\sigma)
  & =
  \sC(\sigma) X(t,\sigma)+\sD(\sigma) U(t,\sigma).
\end{align}
Here, similarly to (\ref{ABCD}), the functions $U$, $X$, $Y$ on $\mR_+\x \mT^\nu$ with values in $\mC^m$, $\mC^n$, $\mC^r$ are the Fourier transforms of the input, state and output of the network:
\begin{align}
\label{uDFT}
    U(t,\sigma)
    & :=
    \sum_{\ell\in \mZ^\nu}
    \re^{-i\ell^{\rT}\sigma}
    u_\ell(t),\\
\label{xDFT}
    X(t,\sigma)
    & :=
    \sum_{\ell\in \mZ^\nu}
    \re^{-i\ell^{\rT}\sigma}
    x_\ell(t),\\
\label{yDFT}
    Y(t,\sigma)
    & :=
    \sum_{\ell\in \mZ^\nu}
    \re^{-i\ell^{\rT}\sigma}
    y_\ell(t)
\end{align}
for almost all $\sigma \in \mT^\nu$ at time $t\> 0$. For any given $\sigma$, the equations (\ref{X}) and (\ref{Y}) describe an autonomous system (which is independent of the other systems in this parametric family with different values of $\sigma$) with a finite-dimensional internal state $X(\cdot, \sigma)$.  Their solution can be represented in terms of the Laplace transform over the time variable as
\begin{align}
\nonumber
  \sX(s,\sigma&)
   :=
  \int_{0}^{+\infty}
  \re^{-st}
  X(t,\sigma)
  \rd t\\
\label{cX}
    =&
    (sI_n - \sA(\sigma))^{-1} (X(0,\sigma)+ \sB(\sigma)\sU(s,\sigma)),\\
\nonumber
  \sY(s,\sigma&)
  :=
  \int_{0}^{+\infty}
  \re^{-st}
  Y(t,\sigma)
  \rd t\\
\nonumber
  = &
  \sC(\sigma) \sX(s,\sigma)+\sD(\sigma) \sU(s,\sigma)\\
\label{cY}
  = &
  \sC(\sigma)
  (sI_n - \sA(\sigma))^{-1} X(0,\sigma)
  +
  F(s,\sigma)\sU(s,\sigma),
\end{align}
where
\begin{equation}
\label{cU}
  \sU(s,\sigma)
   :=
  \int_{0}^{+\infty}
  \re^{-st}
  U(t,\sigma)
  \rd t.
\end{equation}
Here,
\begin{equation}
\label{F}
  F(s,\sigma)
  :=
  \sC(\sigma)(sI_n - \sA(\sigma))^{-1} \sB(\sigma) + \sD(\sigma)
\end{equation}
is the spatio-temporal transfer function of the network with values in $\mC^{r\x m}$. In view of (\ref{Anorm}), the relations (\ref{cX})--(\ref{F}) are valid for all $s\in \mC$ satisfying
\begin{equation}
\label{sgood}
  \Re s
  >
  \max
  \Big(
    \esssup_{\sigma \in \mT^\nu}
    \ln \br(\re^{\sA(\sigma)}),\,
  \limsup_{t\to +\infty}
    \frac{    \ln
    \|u(t)\|_2}{t}
  \Big),
\end{equation}
where $\br(\cdot)$ denotes the spectral radius of a square matrix (so that $\ln\br(\re^N) = \max \Re\fS(N)$
is the largest real part of the eigenvalues of $N$,
with $\fS(N)$ denoting its spectrum).
The presence of the upper limit in (\ref{sgood}) makes the integral in (\ref{cU}) convergent in the Hilbert space $L^2(\mT^\nu, \mC^m)$ due to the Parseval identity
$
    \|U(t,\cdot)\|_2
     :=
    \sqrt{
    \int_{\mT^\nu}
    |U(t,\sigma)|^2
    \rd \sigma}
    =
    (2\pi)^{\nu/2}
    \|u(t)\|_2
$
for any $t\>0$. In the case when the input vanishes, so that the network is effectively isolated from the environment,   (\ref{X}) reduces to a homogeneous linear ODE
$
    \d_t X(t,\sigma) = \sA(\sigma)X(t,\sigma)
$.
Its solution
$
    X(t,\sigma) = \re^{t\sA(\sigma)} X(0,\sigma)
$,
considered for different values of $\sigma \in \mT^\nu$,
admits a direct analogy (formulated in system theoretic terms) with the phonon theory \cite{S_1990} of collective oscillations  in
spatially periodic arrangements of atoms in
crystalline solids. More precisely, for a given $\sigma \in  \mT^\nu$, let $z \in \mC^n$ be an eigenvector of the matrix $\sA(\sigma)$ associated with its eigenvalue $s\in \mC$. % ($\sA(\sigma)z = s z$.
Then the functions
\begin{equation}
\label{phon}
    x_k(t)
    =
    \Re(\re^{st + ik^{\rT} \sigma} z),
    \qquad
    k \in \mZ^\nu,\
    t\>0,
\end{equation}
satisfy the ODEs (\ref{xj}) with $u=0$. Indeed, substitution of the latter equality and (\ref{phon}) into the right-hand side of (\ref{xj}) yields
$
      \sum_{k \in \mZ^\nu}
  A_{j-k} x_k(t)
   =
  \Re
  \big(
  \re^{st + ij^{\rT} \sigma}   \sum_{k \in \mZ^\nu}\re^{-i(j-k)^{\rT} \sigma}A_{j-k}z
  \big)
  =
  \Re
  \big(
  \re^{st + ij^{\rT} \sigma}   \sA(\sigma)z
  \big)
  =
  \Re
  \big(
  s
  \re^{st + ij^{\rT} \sigma}   z
  \big) = \dot{x}_j(t)
$.
%\begin{align*}
%      \sum_{k \in \mZ^\nu}
%  A_{j-k} &x_k(t)
%   =
%%  \sum_{k \in \mZ^\nu}
%%  A_{j-k}
%%  \Re(\re^{st + ik^{\rT} \sigma} z)\\
%%  & =
%  \Re
%  \Big(
%  \re^{st + ij^{\rT} \sigma}   \sum_{k \in \mZ^\nu}\re^{-i(j-k)^{\rT} \sigma}A_{j-k}z
%  \Big)\\
%  & =
%  \Re
%  \big(
%  \re^{st + ij^{\rT} \sigma}   \sA(\sigma)z
%  \big)
%  =
%  \Re
%  \big(
%  s
%  \re^{st + ij^{\rT} \sigma}   z
%  \big) = \dot{x}_j(t).
%\end{align*}
For purely imaginary eigenvalues $s = i\omega$ of the matrix $\sA(\sigma)$ (with $\omega \in \mR$), the solutions (\ref{phon}) describe persistent oscillations of frequency $\omega$ in the network, which are organised as (hyper) plane waves of length $\frac{2\pi}{|\sigma|}$ in $\mR^\nu$. Their wavefronts are orthogonal to the vector $\sigma$ and move (along $\sigma$) at constant phase velocity $-\frac{\omega}{|\sigma|}$, with the direction depending on the sign of $\omega$. Similarly to the dispersion relations of the phonon theory, the spectral structure of such oscillations  in the network is represented by the multi-valued map $\mT^\nu\ni \sigma\mapsto\omega \in \mR$ (of the wave vector to the frequency), which  will be discussed in Section~\ref{sec:phon}.


%%%%%%%%%%%%%%%%%%%%%%%%%%%%%%%%%%%%%%%%%%%%%%%%%%%%%%%%%%%%%%%%%%%%%%%%%%%%%%%%%%%%%%%%%%%%%%%%%%%
\section{SUPPLY RATE AND ENERGY BALANCE% RELATIONS
}\label{sec:bal}
%%%%%%%%%%%%%%%%%%%%%%%%%%%%%%%%%%%%%%%%%%%%%%%%%%%%%%%%%%%%%%%%%%%%%%%%%%%%%%%%%%%%%%%%%%%%%%%%%%%

A generalised model for the work done by the input $u$ on the network is provided by a supply rate \cite{W_1972} at time $t\>0$ in the form
\begin{align}
\nonumber
    S(t)&:=
    \bra
        u(t),Gy(t)
    \ket
    =
    \sum_{j,k\in \mZ^\nu}
    u_j(t)^{\rT}
    G_{j-k}
    y_k(t)\\
\nonumber
    & =
    \frac{1}
    {(2\pi)^{\nu}}
    \int_{\mT^\nu}
    U(t,\sigma)^*
    \sG(\sigma)
    Y(t,\sigma)
    \rd \sigma\\
\label{S}
    & =
    \frac{1}
    {(2\pi)^{\nu}}
    \bra
        U(t,\cdot),
        \sG Y(t,\cdot)
    \ket,
\end{align}
where the Parseval identity is applied to the Fourier transforms $U$ and $Y$ from (\ref{uDFT}) and (\ref{yDFT}).
The quantity $S(t)$ is a bilinear function of the current input and output of the network
which is specified by a block Toeplitz matrix $G:= (G_{j-k})_{j,k\in \mZ^\nu}$, where $G_\ell \in \mR^{m\x r}$ are given matrices satisfying
\begin{equation}
\label{Ggood}
  \sum_{\ell \in \mZ^\nu}
  \|G_\ell\|^2 < +\infty,
\end{equation}
with the Fourier transform
\begin{equation}
\label{cG}
    \sG(\sigma)
    :=
    \sum_{\ell\in \mZ^\nu}
    \re^{-i\ell^{\rT}\sigma}
    G_\ell.
\end{equation}
It is assumed that the matrix $G$ describes a bounded linear operator from the output space $\fY$ in (\ref{fU_fX_fY}) to the input space $\fU$ in (\ref{fU_fX_fY}), so that
\begin{equation}
\label{Gnorm}
    \|G\|_{\infty}
    =
    \esssup_{\sigma \in \mT^\nu}
    \|\sG(\sigma)\|
    <+\infty,
\end{equation}
similarly to the operator norms in (\ref{Anorm})--(\ref{Dnorm}).
For example, the condition (\ref{Gnorm}) holds in the
case of a standard supply rate, when $u$ and $y$ consist of the corresponding force and velocity variables (with $r=m$) and $G$ is the identity operator. Returning to the general case, substitution of (\ref{Y}) into (\ref{S}) leads to
\begin{align}
\nonumber
    S(t)
     =&
    \frac{1}
    {(2\pi)^{\nu}}
    \int_{\mT^\nu}
    \Re
    \big(
    U(t,\sigma)^*
    \sG(\sigma)\\
\nonumber
    & \x
    (\sC(\sigma) X(t,\sigma)+\sD(\sigma) U(t,\sigma))
    \big)
    \rd \sigma\\
\nonumber
    = &
    \frac{1}
    {2(2\pi)^{\nu}}
    \int_{\mT^\nu}
    Z(t,\sigma)^*\\
\nonumber
    & \x
    {\begin{bmatrix}
        0 & \sC(\sigma)^* \sG(\sigma)^*\\
        \sG(\sigma) \sC(\sigma) & \sG(\sigma) \sD(\sigma) + \sD(\sigma)^* \sG(\sigma)^*
    \end{bmatrix}}\\
\label{W1}
    & \x
    Z(t,\sigma)
    \rd \sigma,
\end{align}
where
\begin{equation}
\label{Z}
    Z(t,\sigma)
    :=
      \begin{bmatrix}
        X(t,\sigma)\\
        U(t,\sigma)
    \end{bmatrix}
    =
    \sum_{k \in \mZ^\nu}
    \re^{-ik^{\rT}\sigma}
      \begin{bmatrix}
        x_k(t)\\
        u_k(t)
    \end{bmatrix}
\end{equation}
is the Fourier transform of the augmented state-input pair of the network at time $t\>0$.
Now, let $V:= (V_{j-k})_{j,k\in \mZ^\nu}$ be a symmetric block Toeplitz matrix, which is specified by a square summable matrix-valued function $\mZ^\nu \ni \ell\mapsto V_\ell \in \mR^{n\x n}$  on the lattice  satisfying $
    V_{-\ell} = V_\ell^{\rT}
$. Then the corresponding Fourier transform
\begin{equation}
\label{cV}
    \sV(\sigma)
    :=
    \sum_{\ell\in \mZ^\nu}
    \re^{-i\ell^{\rT}\sigma}
    V_\ell
\end{equation}
takes values in the subspace of complex Hermitian matrices of order $n$ and satisfies
\begin{equation}
\label{Vsymm}
  \sV(-\sigma)
  =
  \sV(\sigma)^{\rT}
  =
  \overline{\sV(\sigma)},
  \qquad
  \sigma \in \mT^\nu.
\end{equation}
Under the additional condition
\begin{equation}
\label{Vnorm}
    \|V\|_{\infty}
    =
    \esssup_{\sigma \in \mT^\nu}
    \|\sV(\sigma)\|
    <+\infty,
\end{equation}
the matrix $V$ describes a self-adjoint operator on the Hilbert space $\fX$ in (\ref{fU_fX_fY}). This gives rise to a quadratic form of the network state at time $t\>0$:
\begin{align}
\nonumber
    H(t)
    & :=
    \frac{1}{2}
    \bra
        x(t),
        Vx(t)
    \ket
    =
    \frac{1}{2}
    \sum_{j,k\in \mZ^\nu}
    x_j(t)^{\rT}
    V_{j-k}
    x_k(t)\\
\label{H}
    & = \frac{1}{(2\pi)^\nu}
    \int_{\mT^\nu}
    \sH(t,\sigma)
    \rd \sigma,
\end{align}
where
\begin{equation}
\label{Hsigma}
    \sH(t,\sigma)
    :=
    \frac{1}{2}
      X(t,\sigma)^*
    \sV(\sigma)
    X(t,\sigma).
\end{equation}
Here, the Parseval identity is applied to the Fourier transform $X$ from  (\ref{xDFT}), whose contribution to $H(t)$ at a given $\sigma \in \mT^\nu$ is quantified by the real-valued quantity $\sH(t,\sigma)$ in (\ref{Hsigma}).
The spatial frequency domain representation (\ref{X}) of the network state dynamics allows the time derivative of $H$ in (\ref{H}) to be computed as
\begin{equation}
\label{Hdot}
    \dot{H}(t)
    =
    \frac{1}{(2\pi)^\nu}
    \int_{\mT^\nu}
    \d_t \sH(t,\sigma)
    \rd \sigma,
\end{equation}
with
\begin{align}
\nonumber
    \d_t &\sH(t,\sigma)
    =
    \Re
    \big(
    X(t,\sigma)^*
    \sV(\sigma)
    \d_t X(t,\sigma)
    \big)    \\
\nonumber
    &=
    \Re
    \big(
    X(t,\sigma)^*
    \sV(\sigma)
    (\sA(\sigma) X(t,\sigma) + \sB(\sigma) U(t,\sigma))
    \big)    \\
\label{Hsigmadot}
    &=
    \frac{1}{2}
    Z(t,\sigma)^*
    {\small\begin{bmatrix}
        \sA(\sigma)^* \sV(\sigma) + \sV(\sigma) \sA(\sigma) & \sV(\sigma) \sB(\sigma)\\
        \sB(\sigma)^* \sV(\sigma) & 0
    \end{bmatrix}}
    Z(t,\sigma),
\end{align}
where use is made of (\ref{Z}).
If $H(t)$ in (\ref{H}) describes the internal energy (usually   referred to as the Hamiltonian) of the network at the current moment of time $t\>0$, then the difference
\begin{equation}
\label{diff}
    S(t) - \dot{H}(t)
    =
    \frac{1}
    {2(2\pi)^{\nu}}
    \int_{\mT^\nu}
    Z(t,\sigma)^*
    N(\sigma)
    Z(t,\sigma)
    \rd \sigma
\end{equation}
can be interpreted as the energy dissipation rate. This quantity is part of the
supply rate in (\ref{S}) which is not accounted for by the rate of change of the internal energy in (\ref{Hdot}).  The complex Hermitian matrix
%\begin{equation}
%\label{N}
%    N(\sigma)
%    :=
%      {\small\begin{bmatrix}
%        -\sA(\sigma)^* \sV(\sigma) - \sV(\sigma) \sA(\sigma) & \sC(\sigma)^* \sG(\sigma)^*-\sV(\sigma) \sB(\sigma)\\
%        \sG(\sigma) \sC(\sigma)-\sB(\sigma)^* \sV(\sigma)    & \sG(\sigma) \sD(\sigma) + \sD(\sigma)^* \sG(\sigma)^*
%    \end{bmatrix}}
%\end{equation}
\begin{align}
\nonumber
    N(\sigma)
    := &
      \left[{\begin{matrix}
        -\sA(\sigma)^* \sV(\sigma) - \sV(\sigma) \sA(\sigma)\\
        \sG(\sigma) \sC(\sigma)-\sB(\sigma)^* \sV(\sigma)
    \end{matrix}}\right.\\
\label{N}
      & \qquad\qquad\qquad
      \left.{\begin{matrix}
        \sC(\sigma)^* \sG(\sigma)^*-\sV(\sigma) \sB(\sigma)\\
        \sG(\sigma) \sD(\sigma) + \sD(\sigma)^* \sG(\sigma)^*
    \end{matrix}}
    \right]
\end{align}
%\begin{equation}
%\label{N}
%    N(\sigma)
%    :=
%      {\begin{bmatrix}
%        -\sA^* \sV - \sV \sA & \sC^* \sG^*-\sV \sB\\
%        \sG \sC-\sB^* \sV    & \sG \sD + \sD^* \sG^*
%    \end{bmatrix}}
%\end{equation}
of order $n+m$ in (\ref{diff}) is obtained by using (\ref{W1}), (\ref{Hdot}) and (\ref{Hsigmadot}),
and satisfies
\begin{equation}
\label{Nsymm}
  N(-\sigma)
  =
  N(\sigma)^{\rT}
  =
  \overline{N(\sigma)},
  \qquad
  \sigma \in \mT^\nu,
\end{equation}
similarly to (\ref{Vsymm}).
% (with the dependence on $\sigma$ being omitted for brevity).
In order for the network model to correspond to a real physical system, the dissipation rate in (\ref{diff}) has to be nonnegative in view of the total energy conservation, and hence,
\begin{equation}
\label{diss}
    S(t) \> \dot{H}(t).
\end{equation}
A sufficient condition for this inequality to hold  for arbitrary admissible inputs $u$ and  initial network states $x(0)$ is positive semi-definiteness of the matrix (\ref{N}):
\begin{equation}
\label{Npos}
  N(\sigma) \succcurlyeq 0,
  \qquad
  {\rm almost\ all}\
  \sigma \in \mT^\nu.
\end{equation}
This condition is also necessary under an additional controllability assumption.
%%%%%%%%%%%%%%%%%%%%%%%%%%%%%%%%%%%%%%%%%%%%%%%%%%%%%%%%%%%%%%%%%%%%%%%%%%%%%%%%%%%%%%%%%%%%%%%%%%%
\begin{thm}
\label{th:Npos}
Suppose the matrix pair $(\sA(\sigma), \sB(\sigma))$ for the network in (\ref{xj}) and (\ref{yj}), given by (\ref{ABCD}), is controllable for any $\sigma\in \mT^\nu$. Then the dissipation rate in (\ref{diff}) is nonnegative for any square summable initial network states and locally integrable inputs in (\ref{ugood}) if and only if the matrix (\ref{N}) satisfies (\ref{Npos}).\hfill$\square$
\end{thm}
%%%%%%%%%%%%%%%%%%%%%%%%%%%%%%%%%%%%%%%%%%%%%%%%%%%%%%%%%%%%%%%%%%%%%%%%%%%%%%%%%%%%%%%%%%%%%%%%%%%
\begin{proof}
In view of (\ref{diff}), the fulfillment of (\ref{Npos}) implies (\ref{diss}) regardless of the controllability of $(\sA,\sB)$, thus proving the sufficiency of (\ref{Npos}), mentioned above. In order to show the necessity, let $\zeta: \mT^\nu\to \mC^{n+m}$ be a measurable  function which is Hermitian (that is, $\overline{\zeta(\sigma)} = \zeta(-\sigma)$ for any $\sigma\in \mT^\nu$) and such that $\zeta(\sigma)$ is a unit eigenvector
associated with the smallest eigenvalue
\begin{equation}
\label{Nmin}
    \lambda(\sigma)
    :=
    \lambda_{\min}(N(\sigma))
\end{equation}
of the Hermitian  matrix $N(\sigma)$ in (\ref{N}), so that, in accordance with (\ref{Nsymm}),
\begin{equation}
\label{zeta}
  N(\sigma)\zeta(\sigma) = \lambda(\sigma)\zeta(\sigma).
\end{equation}
Note that, due to (\ref{Nsymm}), the function $\lambda$ in (\ref{Nmin}) is symmetric.
Therefore, if the above controllability condition is satisfied, then for any time horizon $t>0$, there exists an admissible  network input $u$ on the time interval $[0,t]$ such that the vector (\ref{Z}) satisfies
\begin{equation}
\label{Zmin}
    Z(t,\sigma)
    =
    \left\{
    \begin{matrix}
    \zeta(\sigma) & {\rm if}\ \lambda(\sigma)<0\\
    0 & {\rm otherwise}
    \end{matrix}
    \right.
\end{equation}
for any $\sigma \in \mT^\nu$.
For such an input, in view of (\ref{Nmin})--(\ref{Zmin}), the energy dissipation rate (\ref{diff}) takes the form
\begin{equation}
\label{diffmin}
    S(t) - \dot{H}(t)
    =
    \frac{1}
    {2(2\pi)^{\nu}}
    \int_{\mT^\nu}
    \min(\lambda(\sigma),\, 0)
    \rd \sigma.
\end{equation}
If the property (\ref{Npos}) does not hold, then the set $\{\sigma \in \mT^\nu:\ \lambda(\sigma)<0\}$ is of positive Lebesgue measure, and the right-hand side of (\ref{diffmin})  is negative, thus contradicting (\ref{diss}). This proves that, under the controllability assumption on the pair $(\sA,\sB)$, the condition (\ref{Npos}) is necessary for the network dissipativity.
%
%
%The right-hand side of (\ref{diffmin}) is nonnegative if and only if the set $\{\sigma \in \mT^\nu:\ \lambda(\sigma)<0\}$ is of zero Lebesgue measure, which, due to (\ref{Nmin}), is equivalent to (\ref{Npos}), thus proving the necessity of the latter condition for the network dissipativity (\ref{diss}) under the controllability assumption on the pair $(\sA,\sB)$.
\end{proof}
%%%%%%%%%%%%%%%%%%%%%%%%%%%%%%%%%%%%%%%%%%%%%%%%%%%%%%%%%%%%%%%%%%%%%%%%%%%%%%%%%%%%%%%%%%%%%%%%%%%

If the self-adjoint operator $V$ is positive semi-definite (which is equivalent to that the corresponding Fourier transform $\sV$ in (\ref{cV}) satisfies $\sV(\sigma) \succcurlyeq 0$ for almost all $\sigma \in \mT^\nu$), and the network is initialized at zero state $x(0)=0$, then the integration of both parts of (\ref{diss}) over a time interval $[0,T]$ leads to
\begin{equation}
\label{Wintpos}
  \int_0^T S(t)\rd t
  \>
  H(T)-H(0) = H(T)\> 0
\end{equation}
for any $T>0$.
These relations hold for any network input $u$ such that $\|u(t)\|_2$ is a locally square integrable function of time $t\>0$ in the sense that
\begin{equation}
\label{ugoodT}
    \int_0^T
    \|u(t)\|_2^2
    \rd t
    =
    \int_0^T
    \sum_{k\in \mZ^\nu}
    |u_k(t)|^2
    \rd t
    <+\infty,
    \qquad
    T>0.
\end{equation}
The condition (\ref{ugoodT}) is stronger than (\ref{ugood}) and guarantees finiteness of the work by such an input on the network over any bounded  time interval. More precisely, (\ref{S}) implies that
\begin{align}
\nonumber
  \Big(\int_0^T
  |S(t)|
  \rd t
  \Big)^2
  & \<
  \int_0^T
  \|u(t)\|_2^2\rd t
  \int_0^T
  \|Gy(t)\|_2^2\rd t  \\
\label{Wgood}
  & \<
  \|G\|_{\infty}^2
  \int_0^T
  \|u(t)\|_2^2\rd t
  \int_0^T
  \|y(t)\|_2^2\rd t
\end{align}
in view of the Cauchy-Bunyakovsky-Schwarz inequality and the boundedness (\ref{Gnorm}) of the operator $G$. Here, the time-local square integrability of the input allows (\ref{ygood}) to be enhanced as
\begin{align}
\nonumber
    \frac{1}{2}
  \int_0^T&
  \|y(t)\|_2^2\rd t
  \<
    \int_0^T
    (
    \|Cx(t)\|_2^2 +
    \|Du(t)\|_2^2
    )
    \rd t\\
\label{ygood1}
  \< &
    \|C\|_{\infty}^2
    \int_0^T
    \|x(t)\|_2^2
    \rd t
    +
    \|D\|_{\infty}^2
    \int_0^T
    \|u(t)\|_2^2
    \rd t,
\end{align}
which is obtained by applying the inequality $\frac{1}{2}\|\alpha + \beta\|^2 \< \|\alpha\|^2 + \|\beta\|^2$ in an arbitrary Hilbert space to the right-hand side of (\ref{y}) and using the boundedness (\ref{Cnorm}) and (\ref{Dnorm}) of the operators $C$ and $D$. Also, (\ref{xgood}) implies that
 \begin{align}
\nonumber
    \|x(t)\|_2^2
    & \<
    2
    \re^{2\|A\|_{\infty}t}
    \Big(
        \|x(0)\|_2^2
        +
    \|B\|_{\infty}^2
        \Big(
    \int_0^t
    \|u(\tau)\|_2
    \rd \tau
    \Big)^2
    \Big)\\
\label{xnormup}
    & \<
    2
    \re^{2\|A\|_{\infty}t}
    \Big(
        \|x(0)\|_2^2
        +
        t
    \|B\|_{\infty}^2
    \int_0^t
    \|u(\tau)\|_2^2
    \rd \tau
    \Big),
\end{align}
where the Cauchy-Bunyakovsky-Schwarz inequality is used again. The local square integrability of $\|x(t)\|_2$ as a function of time $t\>0$ follows from (\ref{xnormup})  in view of (\ref{x0norm}) and (\ref{ugoodT}). In combination with (\ref{ygood1}), this ensures the time-local square integrability of the network output: $  \int_0^T
  \|y(t)\|_2^2\rd t <+\infty$ for any $T>0$. From the last property, (\ref{ugoodT}) and (\ref{Wgood}), it follows that the supply rate $S(t)$ is indeed locally integrable with respect to time.





%%%%%%%%%%%%%%%%%%%%%%%%%%%%%%%%%%%%%%%%%%%%%%%%%%%%%%%%%%%%%%%%%%%%%%%%%%%%%%%%%%%%%%%%%%%%%%%%%%%
\section{NETWORK PASSIVITY CONDITIONS IN THE SPATIAL FREQUENCY DOMAIN}\label{sec:pass}
%%%%%%%%%%%%%%%%%%%%%%%%%%%%%%%%%%%%%%%%%%%%%%%%%%%%%%%%%%%%%%%%%%%%%%%%%%%%%%%%%%%%%%%%%%%%%%%%%%%

The relations (\ref{Wintpos}) imply that the network is passive %\cite{?}
in the sense that any time-locally square integrable input $u$ in (\ref{ugoodT}) performs a nonnegative work:
\begin{equation}
\label{W}
    W(T)
    :=
  \int_0^T S(t)\rd t
  \>
  0,
  \qquad
  T>0.
\end{equation}
The passivity
can also be considered irrespective of a specific internal energy (storage) function (provided the initial network state is zero). For what follows, the network input $u$ is assumed to be time-space square summable in the sense that
\begin{equation}
\label{ugood1}
    \sn u \sn
    :=
    \sqrt{
    \int_0^{+\infty}
    \|u(t)\|_2^2
    \rd t}
    =
    \sqrt{
    \int_0^{+\infty}
    \sum_{k \in \mZ^\nu}
    |u_k(t)|^2
    \rd t}
    <+\infty,
\end{equation}
where
the norm $\sn \cdot \sn$ is associated with the inner product
$
    \dbra f, g \dket
    :=
    \int_0^{+\infty}
    \bra
        f(t),
        g(t)
    \ket
    =
    \int_0^{+\infty}
    \sum_{k \in \mZ^\nu}
    f_k(t)^* g_k(t)
    \rd t
$
for real or complex vector-valued functions $f(t):= (f_k(t))_{k \in \mZ^\nu}$ and $g(t):=(g_k(t))_{k \in \mZ^\nu}$ of the time and space variables. By the Parseval identity, (\ref{ugood1}) yields
$    \int_{\mR \x \mT^{\nu}}
    |\sU(i\omega, \sigma)|^2
    \rd \omega \rd \sigma
    =
    (2\pi)^{\nu+1}
    \sn u \sn^2
$,
where the function $\sU$ is given by (\ref{cU}). Although (\ref{ugood1}) is a stronger condition than (\ref{ugoodT}), the input $u$ produces the same work $W(T)$ in (\ref{W}) over a given time interval $[0,T]$  as its ``truncated'' version
\begin{equation}
\label{PiT}
\Pi_T(u)(t)
:=
\left\{
\begin{matrix}
u(t) & {\rm if}\ 0\< t\< T\\
0    & {\rm otherwise}
\end{matrix}
\right.
\end{equation}
(with $\Pi_T$ being an orthogonal projection operator for any $T>0$).
This property follows from causality of the network as an input-output operator $u\mapsto y$ and from the fact that the   supply rate $S(t)$ in (\ref{S}) is a bilinear function of $u(t)$ and $y(t)$.
%%%%%%%%%%%%%%%%%%%%%%%%%%%%%%%%%%%%%%%%%%%%%%%%%%%%%%%%%%%%%%%%%%%%%%%%%%%%%%%%%%%%%%%%%%%%%%%%%%%
\begin{thm}
\label{th:pass}
Suppose the network, governed by (\ref{xj}) and (\ref{yj}), satisfies (\ref{ABCDgood}) and (\ref{Anorm})--(\ref{Dnorm}), and is endowed with the supply rate (\ref{S}) subject to (\ref{Ggood}) and (\ref{Gnorm}). Also, suppose the matrix $\sA(\sigma)$ in (\ref{ABCD}) is Hurwitz for any $\sigma \in \mT^\nu$. Then the network is passive in the sense of (\ref{W})  for zero initial states and arbitrary time-locally square integrable inputs $u$ in (\ref{ugoodT}) if and only if its
spatio-temporal transfer function (\ref{F}) satisfies
\begin{equation}
\label{FGpos}
  E(\omega, \sigma)
  :=
  F(i\omega,\sigma)^*\sG(\sigma)^*
  +
  \sG(\sigma)F(i\omega,\sigma)
  \succcurlyeq
  0
\end{equation}
for almost all $(\omega,\sigma) \in \mR\x \mT^\nu$. Here, the function $\sG$ is given by (\ref{cG}).
\hfill$\square$
\end{thm}
%%%%%%%%%%%%%%%%%%%%%%%%%%%%%%%%%%%%%%%%%%%%%%%%%%%%%%%%%%%%%%%%%%%%%%%%%%%%%%%%%%%%%%%%%%%%%%%%%%%
\begin{proof}
Assuming that the network is initialized at zero, consider the work $W(T)$ up until a given time horizon $T>0$. Since an arbitrary admissible input $u$ (in the sense of (\ref{ugoodT})) can be replaced with $\Pi_T(u)$ in (\ref{PiT}) without affecting the work, we will assume, without loss of generality, that $u$ vanishes beyond the time interval $[0,T]$, and hence, so also does the supply rate $S$ in (\ref{S}). For any such input, substitution of (\ref{S}) into (\ref{W}) leads to
\begin{align}
\nonumber
    W(&T)
    =
    \int_0^{+\infty}
    S(t)\rd t
    =
    \dbra
        u, Gy
    \dket\\
%\nonumber
%    = &
%    \frac{1}
%    {(2\pi)^{\nu}}
%    \int_{\mT^\nu}
%    \Big(
%    \int_0^{+\infty}
%    U(t,\sigma)^*
%    \sG(\sigma)
%    Y(t,\sigma)
%    \rd t
%    \Big)
%    \rd \sigma\\
\nonumber
    = &
    \frac{1}
    {(2\pi)^{\nu+1}}
    \int_{\mT^\nu}
    \Big(
    \int_{-\infty}^{+\infty}
    \sU(i\omega,\sigma)^*
    \sG(\sigma)
    \sY(i\omega,\sigma)
    \rd \omega
    \Big)
    \rd \sigma\\
\nonumber
    = &
    \frac{1}
    {(2\pi)^{\nu+1}}
    \int_{\mR\x \mT^\nu}
    \sU(i\omega,\sigma)^*
    \sG(\sigma)
    F(i\omega,\sigma)
    \sU(i\omega,\sigma)
    \rd \omega
    \rd \sigma\\
\label{work}
    = &
    \frac{1}
    {2(2\pi)^{\nu+1}}
    \int_{\mR\x \mT^\nu}
    \sU(i\omega,\sigma)^*
    E(\omega,\sigma)
    \sU(i\omega,\sigma)
    \rd \omega
    \rd \sigma,
\end{align}
where the function $F$ in (\ref{F}) is well-defined on the set $(i\mR)\x \mT^\nu$ (with $i\mR$ the imaginary axis) due to the matrix $\sA(\sigma)$ being Hurwitz for all $\sigma \in \mT^\nu$.
Here, in view of  (\ref{sgood}), the Parseval identity is used together with the Fourier transforms (\ref{cY}), (\ref{cU}) and the Hermitian matrix $E(\omega,\sigma) \in \mC^{m\x m}$ in (\ref{FGpos}).
In view of (\ref{work}), the positive semi-definiteness of $E$ almost everywhere in $\mR\x \mT^{\nu}$ implies that $W(T)\> 0$ for all admissible inputs vanishing outside the time interval $[0,T]$. The necessity of the matrix inequality in (\ref{FGpos}) for the network passivity can be obtained by letting $T\to +\infty$ and considering all possible time-space square summable inputs satisfying (\ref{ugood1}).
%
%
%\begin{align}
%\nonumber
%    W(T)
%    & =
%    \frac{1}
%    {(2\pi)^{\nu}}
%    \int_{\mT^\nu}
%    \Big(
%    \int_0^T
%    U(t,\sigma)^*
%    \sG(\sigma)
%    Y(t,\sigma)
%    \rd t
%    \Big)
%    \rd \sigma\\
%\nonumber
%    & =
%    \frac{1}
%    {(2\pi)^{\nu+1}}
%    \int_{\mT^\nu}
%    \Big(
%    \int_0^T
%    U(t,\sigma)^*
%    \sG(\sigma)
%    Y(t,\sigma)
%    \rd t
%    \Big)
%    \rd \sigma\\
%\end{align}
%where use is made of the inverse Fourier transforms of (\ref{cY}) and (\ref{cU}) over the time variable:
%\begin{align}
%\label{sUinv}
%    U(t,\sigma)
%    &=
%    \frac{1}{2\pi}
%    \int_\mR
%    \re^{i\alpha t}
%    \sU(i\alpha,\sigma)
%    \rd \alpha,\\
%\label{sYinv}
%    Y(t,\sigma)
%    &=
%    \frac{1}{2\pi}
%    \int_\mR
%    \re^{i\beta t}
%    \sY(i\beta,\sigma)
%    \rd \beta
%\end{align}
\end{proof}
%%%%%%%%%%%%%%%%%%%%%%%%%%%%%%%%%%%%%%%%%%%%%%%%%%%%%%%%%%%%%%%%%%%%%%%%%%%%%%%%%%%%%%%%%%%%%%%%%%%

Note that in the case of $r=m$ and the standard supply rate mentioned above,  $\sG$ in (\ref{cG}) is the identity matrix, and (\ref{FGpos}) reduces to
\begin{equation}
\label{Fposreal}
  F(i\omega,\sigma) +
  F(i\omega,\sigma)^*
  \succcurlyeq
  0,
  \qquad
  \omega \in \mR,\
  \sigma \in \mT^\nu,
\end{equation}
which is a network counterpart of the positive real property. % \cite{?}.
On the other hand, in an extended setting, the block Toeplitz matrix $G$ can be formed from differential operators $G_\ell(\d_t)$ with respect to time (whose entries  are, for example, polynomials of $\d_t$). In this case, Theorem~\ref{th:pass} is modified by replacing the function $\sG$ in (\ref{FGpos}), given by (\ref{cG}), with the Fourier-Laplace transform
\begin{equation}
\label{cGgen}
    \sG(s,\sigma)
    :=
%    \int_0^{+\infty}
    \sum_{\ell\in \mZ^\nu}
    \re^{-i\ell^{\rT}\sigma}
    G_\ell(s).
%    \rd t.
\end{equation}
In particular, if $r=m$ and the extended operator $G$ acts on the network output as $Gy = \dot{y}$, then (\ref{cGgen}) yields $\sG(s,\sigma) = sI_m$. In accordance with the structure  of the supply rate (\ref{S}),  this describes the setting when $u$ and $y$ consist of the corresponding force and position (rather than velocity) variables.  In this case,
 (\ref{FGpos})  takes the form
\begin{equation}
\label{Fnegimag0}
  i\omega (F(i\omega,\sigma) -
  F(i\omega,\sigma)^*)
  \succcurlyeq
  0,
  \qquad
  \omega \in \mR,\
  \sigma \in \mT^\nu.
\end{equation}
Since the spatio-temporal transfer function $F$ in (\ref{F}) satisfies $F(-i\omega, -\sigma) = \overline{F(i\omega,\sigma)}$ for all $\omega \in \mR$, $\sigma \in \mR^\nu$, and the complex conjugation of a Hermitian matrix preserves positive semi-definiteness, then  (\ref{Fnegimag0}) is equivalent to
\begin{equation}
\label{Fnegimag}
    \frac{1}{i}
  (F(i\omega,\sigma) -
  F(i\omega,\sigma)^*)
  \preccurlyeq
  0,
  \qquad
  \omega \in \mR_+,\
  \sigma \in \mT^\nu.
\end{equation}
Similarly to (\ref{Fposreal}), the condition (\ref{Fnegimag}) is a network version of the negative-imaginary property \cite{PL_2010,XPL_2010}.
%
%
%
%%%%%%%%%%%%%%%%%%%%%%%%%%%%%%%%%%%%%%%%%%%%%%%%%%%%%%%%%%%%%%%%%%%%%%%%%%%%%%%%%%%%%%%%%%%%%%%%%%%%%
%%\section{QUADRATIC STABILITY OF THE NETWORK IN ISOLATION FROM THE ENVIRONMENT}\label{sec:stab}
%%%%%%%%%%%%%%%%%%%%%%%%%%%%%%%%%%%%%%%%%%%%%%%%%%%%%%%%%%%%%%%%%%%%%%%%%%%%%%%%%%%%%%%%%%%%%%%%%%%%%
%
%%In order for the spatio-temporal transfer function $F$ of the network in (\ref{F}) to be well-defined on the set $(i\mR)\x \mT^\nu$ (with $i\mR$ the imaginary axis in the complex plane), which is used as an assumption in Theorem~\ref{th:pass} of the previous section, it is sufficient that the matrix $\sA(\sigma)$ in (\ref{ABCD}) is Hurwitz for all $\sigma \in \mT^\nu$.
In the case when the matrix $\sA(\sigma)$ in (\ref{ABCD}) is Hurwitz for all $\sigma \in \mT^\nu$ (as assumed in Theorem~\ref{th:pass}), there exists a positive definite Hermitian $\mC^{n\x n}$-valued function $\sV$ in (\ref{cV}) on the torus $\mT^\nu$ satisfying (\ref{Vsymm}), (\ref{Vnorm}) and such that
\begin{equation}
\label{ALI}
    \sA(\sigma)^* \sV(\sigma) +\sV(\sigma) \sA(\sigma) \prec 0,
    \qquad
    \sigma \in \mT^\nu.
\end{equation}
In isolation from the environment (when $u=0$), the quadratic stability of the network can be formulated by enhancing the positive definiteness as
\begin{equation}
\label{Vgood}
    \mu
    :=
    \essinf_{\sigma \in \mT^\nu}
    \lambda_{\min}(\sV(\sigma))
    >0.
\end{equation}
Then
the corresponding block Toeplitz matrix $V$ describes a positive definite self-adjoint operator on $L^2(\mZ^\nu, \mR^n)$ whose inverse $V^{-1}$ is also     such an operator and its norm is related to the quantity $\mu$ in (\ref{Vgood}) by
\begin{equation}
\label{Vinvnorm}
    \|V^{-1}\|_{\infty}
    =
    \frac{1}{\mu}.
\end{equation}
Furthermore, the Hamiltonian $H$, associated with $V$ by (\ref{H}), admits the following bounds  in terms of the standard $L^2$-norm of the network state:
\begin{equation}
\label{Hbounds}
    \frac{\mu}{2}
    \|x(t)\|_2^2
    \<
    H(t)
    \<
    \frac{\|V\|_{\infty}}{2}
    \|x(t)\|_2^2.
\end{equation}
Similarly to finite-dimensional settings,
the dissipativity (\ref{diss}) (or a nonstrict version of the inequality  (\ref{ALI})) implies that the Hamiltonian of the isolated network does not increase with time (that is, $\dot{H}\< 0$), which, in combination with (\ref{Hbounds}) leads to
\begin{equation}
\label{cond}
    \|x(t)\|_2^2
    \<
    \frac{2}{\mu} H(t)
    \<
    \frac{2}{\mu} H(0)
      \< \frac{\|V\|_\infty}{\mu} \|x(0)\|_2^2,
    \quad
    t\>0
\end{equation}
%Here, use is also made of the upper bound for the Hamiltonian
%$$
%    H(0) \< \frac{1}{2}\|V\|_{\infty} \|x(0)\|_2^2,
%$$
% which follows from (\ref{Vnorm}) and (\ref{H}).
(similar bounds, arising from quadratic Hamiltonians, are also used for quantum harmonic oscillators \cite[Eq. (22)]{P_2014}).
In view of (\ref{Vinvnorm}), the factor
$
%    \kappa(V)
%    :=
    \frac{\|V\|_\infty}{\mu} = \|V\|_{\infty}\|V^{-1}\|_{\infty}
    \> 1
$
on the right-hand side of (\ref{cond}) is the condition number of the operator $V$ (which quantifies how far $V$ is from scalar operators).



%The interaction is  arranged in a coherent (measurement-free) fashion and is  mediated by quantum fields which propagate through quantum channels shown as directed edges in  Fig.~\ref{fig:chain}.
%%==============================================================================
%\begin{figure}[htbp]
%\centering
%\unitlength=0.55mm
%\linethickness{0.4pt}
%\begin{picture}(105.00,76.00)
%    \multiput(0, 40)(40, 0){3}{
%        \framebox(20,20)[cc]{{\small$F$}}
%        \put(-10,35){\vector(0,-1){15}}
%        \put(-10,0){\vector(0,-1){15}}
%
%        \put(-20,5){\vector(-1,0){20}}
%        \put(20,5){\vector(-1,0){20}}
%        \put(-40,15){\vector(1,0){20}}
%        \put(0,15){\vector(1,0){20}}
%    }
%    \put(13,80){\makebox(0,0)[cc]{{\small$w_{k-1}$}}}
%    \put(53,80){\makebox(0,0)[cc]{{\small$w_{k}$}}}
%    \put(93,80){\makebox(0,0)[cc]{{\small$w_{k+1}$}}}
%
%    \put(13,20){\makebox(0,0)[cc]{{\small$r_{k-1}$}}}
%    \put(53,20){\makebox(0,0)[cc]{{\small$r_{k}$}}}
%    \put(93,20){\makebox(0,0)[cc]{{\small$r_{k+1}$}}}
%
%    \put(-9,60){\makebox(0,0)[cc]{{\scriptsize$y_{k-2}^+ = u_{k-1}^+$}}}
%    \put(-9,40){\makebox(0,0)[cc]{{\scriptsize$u_{k-2}^-=y_{k-1}^-$}}}
%
%    \put(-23,52){\makebox(0,0)[cc]{$\vdots$}}
%    \put(127,52){\makebox(0,0)[cc]{$\vdots$}}
%
%
%    \put(33,60){\makebox(0,0)[cc]{{\scriptsize$y_{k-1}^+=u_k^+$}}}
%    \put(33,40){\makebox(0,0)[cc]{{\scriptsize$u_{k-1}^-=y_k^- $}}}
%
%    \put(73,60){\makebox(0,0)[cc]{{\scriptsize$y_k^+=u_{k+1}^+$}}}
%    \put(73,40){\makebox(0,0)[cc]{{\scriptsize$u_k^-=y_{k+1}^- $}}}
%
%    \put(114,60){\makebox(0,0)[cc]{{\scriptsize$y_{k+1}^+=u_{k+2}^+$}}}
%    \put(114,40){\makebox(0,0)[cc]{{\scriptsize$u_{k+1}^-=y_{k+2}^- $}}}
%
%\end{picture}\vskip-13mm
%\caption{A fragment of a one-dimensional chain of linear quantum stochastic systems with nearest neighbour interaction. The subsystems and fields are numbered from left to right. The ``$+$'' and ``$-$'' superscripts indicate field  propagation in the  positive and negative directions, respectively.}
%\label{fig:chain}
%\end{figure}
%%==============================================================================
%This infinite interconnection is a translation invariant quantum feedback network which, in practice, can be implemented as its finite fragment.
%%
%Suppose the fragment of the chain consists of $N$ building blocks which are numbered by integers $k = 0, \ldots, N-1$.
%The $k$th block is a linear quantum stochastic system with ``rightward'' input and  output $u_k^+$ and $y_k^+$ of common dimension $m_+$ and ``leftward''  input and  output $u_k^-$ and $y_k^-$ of common dimension $m_-$; see Fig.~\ref{fig:block}.
%%==============================================================================
%\begin{figure}[htbp]
%\centering
%\unitlength=0.4mm
%\linethickness{0.4pt}
%\begin{picture}(70.00,51.00)
%    \put(20, 17){
%        \framebox(20,20)[cc]{{\small$F$}}
%        \put(-10,35){\vector(0,-1){15}}
%        \put(-10,0){\vector(0,-1){15}}
%
%        \put(-20,5){\vector(-1,0){20}}
%        \put(20,5){\vector(-1,0){20}}
%        \put(-40,15){\vector(1,0){20}}
%        \put(0,15){\vector(1,0){20}}
%        \put(-10,40){\makebox(0,0)[cc]{{\small$w$}}}
%
%        \put(-10,-20){\makebox(0,0)[cc]{{\small$r$}}}
%
%        \put(-45,15){\makebox(0,0)[cc]{{\small$u^+$}}}
%        \put(-45,5){\makebox(0,0)[cc]{{\small$y^-$}}}
%
%        \put(25,15){\makebox(0,0)[cc]{{\small$y^+$}}}
%        \put(25,5){\makebox(0,0)[cc]{{\small$u^-$}}}
%    }
%\end{picture}
%\caption{A common building block of the chain.}
%\label{fig:block}
%\end{figure}
%%==============================================================================
%The system  is endowed with an $n$-dimensional vector $x_k$ of dynamic variables and is coupled to external input fields  which are modelled by an $m_0$-dimensional  quantum Wiener process $w_k$ on a boson Fock space \cite{P_1992}. As shown in Figs.~\ref{fig:chain}, \ref{fig:block},  the systems can also have external output fields $r_k$. However, they will be taken into consideration elsewhere. The entries of the vectors $x_k$, $w_k$, $u_k^{\pm}$, $y_k^{\pm}$  are self-adjoint operators on appropriate complex separable Hilbert spaces evolving in time in accordance with the Heisenberg picture of quantum dynamics \cite{M_1998}. Unless specified otherwise, vectors are organised as columns. The joint quantum Ito table of the quantum Wiener processes $w_k$ is given by
%\begin{equation}
%\label{ww}
%    \rd w_j \rd w_k^{\rT} = \delta_{jk}\Omega\rd t,
%    \qquad
%    \Omega := I_{m_0} + i J.
%\end{equation}
%Here, the transpose $(\cdot)^{\rT}$ acts on vectors of operators as if the latter were scalars. Also,  $i:= \sqrt{-1}$ is the imaginary unit,
%$\delta_{jk}$ is the Kronecker delta, $I_{m_0}$ is the identity matrix of order $m_0$, and
%$J$ is a real antisymmetric matrix of order $m_0$. The matrix $J$ has spectral radius $\br(J)\< 1$ (thus ensuring the positive semi-definiteness  of the quantum Ito matrix $\Omega \succcurlyeq 0$) and
%specifies the cross commutations between the boson fields as
%\begin{equation}
%\label{wCCR}
%    [\rd w_j, \rd w_k^{\rT}]
%    =
%    2i
%    \delta_{jk}
%    J\rd t,
%\end{equation}
%where $[\alpha, \beta^{\rT}]:= \alpha\beta^{\rT} - (\beta\alpha^{\rT})^{\rT}$ is the commutator matrix.
%Usually, the quantum noise dimension $m_0$ is even, and
%%\begin{equation*}
%%\label{J}
%$    J = I_{m_0/2} \ox
%        {\scriptsize\begin{bmatrix}
%        0 & 1\\
%        -1  & 0
%    \end{bmatrix}}
%%\bJ,
%%    \qquad
%%    \bJ
%%    :=
%%    {\small\begin{bmatrix}
%%        0 & 1\\
%%        -1  & 0
%%    \end{bmatrix}},
%%\end{equation*}
%$,
%where $\ox$ is the Kronecker product of matrices. In view of (\ref{ww}), (\ref{wCCR}), the quantum Wiener processes $w_k$ for different component systems are uncorrelated and commuting.
%Now, the $k$th system in the chain is governed by the following linear QSDEs with constant coefficients
%\begin{align}
%\label{dx}
%    \rd x_k
%    &=
%    A x_k \rd t + B \rd w_k + E \rd u_k,\\
%\label{dy}
%    \rd y_k
%    &=
%    C x_k \rd t + D \rd w_k.
%\end{align}
%Here,
%\begin{equation}
%\label{uy}
%    u_k
%    :=
%    {\small\begin{bmatrix}
%        u_k^+\\
%        u_k^-
%    \end{bmatrix}},
%    \qquad
%    y_k :=
%    {\small\begin{bmatrix}
%        y_k^+\\
%        y_k^-
%    \end{bmatrix}}
%\end{equation}
%are the input and output quantum processes, each having dimension
%$
%    m:= m_++m_-
%$,
%and
%\begin{equation}
%\label{CDE}
%    C :=
%    {\small\begin{bmatrix}
%        C_+ \\
%        C_-
%    \end{bmatrix}},
%    \qquad
%    D
%    :=
%    {\small\begin{bmatrix}
%        D_{+} \\
%        D_{-}
%    \end{bmatrix}},
%    \qquad
%    E :=
%    {\small\begin{bmatrix}
%        E_+ & E_-
%    \end{bmatrix}},
%\end{equation}
%where $A \in \mR^{n\x n}$, $B\in \mR^{n\x m_0}$, $C_{\pm} \in \mR^{m_{\pm} \x n}$, $D_{\pm} \in \mR^{m_{\pm} \x m_0}$, $E_{\pm}\in \mR^{n\x m_{\pm}}$  are appropriately dimensioned real matrices. In view of the interconnection of the systems in the chain (see Fig.~\ref{fig:chain}),  the QSDEs (\ref{dx}), (\ref{dy}) are complemented by the algebraic relations for the rightward and leftward inputs and outputs of the adjacent systems:
%\begin{equation}
%\label{connect}
%    y_{k-1}^+ = u_k^+,
%    \qquad
%    u_k^- = y_{k+1}^-.
%\end{equation}
%Furthermore, the  set of equations (\ref{dx}), (\ref{dy}), (\ref{connect}) for the fragment of the chain must be equipped with boundary conditions for the $0$th and $(N-1)$th systems. As a variant of such conditions, it can be assumed that the boundary inputs $u_0^+$ and $u_{N-1}^-$ are additional uncorrelated and commuting quantum Wiener processes.
%%$    u_0^+ := w_+$ and $
%%    u_{N-1}^- := w_-
%%$.
%%with quantum Ito tables
%%$$
%%    \rd w_{\pm} \rd w_{\pm} = \Omega_{\pm} \rd t,
%%    \quad
%%    \Omega_{\pm} := I_{m_{\pm}} + i J_{\pm},
%%    \quad
%%    J_{\pm}:= I_{m_{\pm}/2} \ox \bJ,
%%$$
%%where $J_{\pm}$ are real antisymmetric matrices of order $m_{\pm}$ whose structure is similar to that of $J$ in (\ref{J}) (and so, the dimensions $m_{\pm}$ are also even).
%However, in order to simplify the analysis at this stage, we will use the periodic boundary conditions (PBCs)
%\begin{equation}
%\label{PBC}
%    u_0^+ = y_{N-1}^+,
%    \qquad
%    u_{N-1}^- = y_0^-,
%\end{equation}
%as shown in Fig.~\ref{fig:PBC}.
%%==============================================================================
%\begin{figure}[htbp]
%\centering
%\unitlength=0.45mm
%\linethickness{0.4pt}
%\begin{picture}(105.00,90.00)
%    \multiput(0, 40)(40, 0){3}{
%        \framebox(20,20)[cc]{{\small$F$}}
%        \put(-10,30){\vector(0,-1){10}}
%%        \put(-10,0){\vector(0,-1){10}}
%
%        \put(-20,5){\vector(-1,0){20}}
%        \put(20,5){\vector(-1,0){20}}
%        \put(-40,15){\vector(1,0){20}}
%        \put(0,15){\vector(1,0){20}}
%    }
%    \put(13,75){\makebox(0,0)[cc]{{\small$w_0$}}}
%    \put(53,75){\makebox(0,0)[cc]{{\small$w_1$}}}
%    \put(93,75){\makebox(0,0)[cc]{{\small$w_2$}}}
%
%%    \put(13,25){\makebox(0,0)[cc]{$r_0$}}
%%    \put(53,25){\makebox(0,0)[cc]{$r_1$}}
%%    \put(93,25){\makebox(0,0)[cc]{$r_2$}}
%
%    \put(-7,60){\makebox(0,0)[cc]{{\scriptsize$u_0^+$}}}
%    \put(-7,40){\makebox(0,0)[cc]{{\scriptsize$y_0^-$}}}
%
%
%
%    \put(33,60){\makebox(0,0)[cc]{{\scriptsize$y_0^+=u_1^+$}}}
%    \put(33,40){\makebox(0,0)[cc]{{\scriptsize$u_0^-=y_1^- $}}}
%
%    \put(73,60){\makebox(0,0)[cc]{{\scriptsize$y_1^+=u_2^+$}}}
%    \put(73,40){\makebox(0,0)[cc]{{\scriptsize$u_1^-=y_2^- $}}}
%
%    \put(111,60){\makebox(0,0)[cc]{{\scriptsize$y_2^+$}}}
%    \put(111,40){\makebox(0,0)[cc]{{\scriptsize$u_2^-$}}}
%
%    \put(122,55){\vector(0,1){30}}
%    \put(122,85){\vector(-1,0){140}}
%    \put(-18,85){\vector(0,-1){30}}
%
%    \put(-18,45){\vector(0,-1){30}}
%    \put(-18,15){\vector(1,0){140}}
%    \put(122,15){\vector(0,1){30}}
%
%\end{picture}\vskip-8mm
%\caption{An illustration of  the PBCs for a fragment of the chain with $N = 3$.}
%\label{fig:PBC}
%\end{figure}
%%==============================================================================
%In this case, the fragment of the chain acquires a ring topology, % \cite{B_1998},
%and the interconnection rules (\ref{connect}) take a more unified form
%\begin{equation}
%\label{connectPBC}
%    u_k
%    =
%    {\small\begin{bmatrix}
%        I_{m_+} & 0\\
%        0 & 0
%    \end{bmatrix}}
%    y_{k-1}
%    +
%    {\small\begin{bmatrix}
%        0 & 0\\
%        0 & I_{m_-}
%    \end{bmatrix}}
%    y_{k+1},
%    \ \ \,
%    k=0, \ldots, N-1,\!\!
%\end{equation}
%where $k\pm 1$  are calculated modulo $N$. This representation is convenient for the harmonic analysis of the quantum network in the spatial frequency domain.
%
%%%%%%%%%%%%%%%%%%%%%%%%%%%%%%%%%%%%%%%%%%%%%%%%%%%%%%%%%%%%%%%%%%%%%%%%%%%%%%%%%%%%%%%%%%%%%%%%%%%%
%\section{Spatial Fourier transforms}\label{sec:ztrans}
%%%%%%%%%%%%%%%%%%%%%%%%%%%%%%%%%%%%%%%%%%%%%%%%%%%%%%%%%%%%%%%%%%%%%%%%%%%%%%%%%%%%%%%%%%%%%%%%%%%%
%\vskip-1mm
%
%Consider the $z$-transform of the quantum processes $x_k$, $w_k$, $u_k$, $y_k$ in (\ref{dx})--(\ref{uy}) over the ``spatial'' subscript $k = 0, \ldots, N-1$ which numbers the systems in the fragment of the chain:
%\begin{align}
%\label{XW}
%    X_z(t)
%    & :=
%    \sum_{k=0}^{N-1} z^{-k}x_k(t),
%    \quad
%    W_z(t)
%    :=
%    \sum_{k=0}^{N-1} z^{-k}w_k(t),\\
%\label{U}
%    U_z(t)
%    & :=
%    {\small\begin{bmatrix}
%        U_z^+(t)\\
%        U_z^-(t)
%    \end{bmatrix}},
%    \qquad
%    U_z^{\pm}(t)
%    :=
%    \sum_{k=0}^{N-1} z^{-k} u_k^\pm(t),\\
%%\label{Y}
%    Y_z(t)
%    & :=
%    {\small\begin{bmatrix}
%        Y_z^+(t)\\
%        Y_z^-(t)
%    \end{bmatrix}},
%    \qquad\
%    Y_z^{\pm}(t)
%    :=
%    \sum_{k=0}^{N-1}
%    z^{-k}
%    y_k^{\pm}(t),
%\end{align}
%where $z$ is a nonzero complex parameter.  Note that, being linear combinations of self-adjoint operators with complex coefficients, the entries of the vectors $X_z(t)$, $W_z(t)$, $U_z(t)$, $Y_z(t)$  are not self-adjoint.
%By applying $z$-transforms to the linear QSDEs (\ref{dx}), (\ref{dy}) and using (\ref{XW})--(\ref{Y}), it follows that the quantum processes $X_z$, $W_z$, $U_z$, $Y_z$, as functions of time $t$, satisfy QSDEs with the same coefficients:
%\begin{align}
%\label{dX}
%    \rd X_z
%    &=
%    A X_z \rd t + B \rd W_z + E \rd U_z,\\
%\label{dY}
%    \rd Y_z
%    &=
%    C X_z \rd t + D \rd W_z,
%\end{align}
%where the time arguments are omitted for brevity.
%Furthermore, in the framework of the PBCs (\ref{PBC}), the equalities (\ref{connectPBC}) imply that the $z$-transforms $U_z$, $Y_z$ in (\ref{U}), (\ref{Y}) are related by
%\begin{align}
%\nonumber
%    U_z
%    =&
%    {\small\begin{bmatrix}
%        I_{m_+} & 0\\
%        0 & 0
%    \end{bmatrix}}
%    \left(
%        z^{-1}Y_z + \left(1 - z^{-N}\right)y_{N-1}
%    \right)\\
%\nonumber
%    & +
%    {\small\begin{bmatrix}
%        0 & 0\\
%        0 & I_{m_-}
%    \end{bmatrix}}
%    \left(
%        z Y_z + z\left(z^{-N}-1\right)y_0
%    \right)\\
%\label{UY}
%    = &
%    {\small\begin{bmatrix}
%        z^{-1}I_{m_+} & 0\\
%        0 & zI_{m_-}
%    \end{bmatrix}}
%    Y_z
%    +
%    \left(1 - z^{-N}\right)
%    {\small\begin{bmatrix}
%        y_{N-1}^+\\
%        -zy_0^-
%    \end{bmatrix}}.
%\end{align}
%In particular, the boundary outputs $y_0^-$ and $y_{N-1}^+$ of the chain fragment make no contribution  to (\ref{UY}) when $z$ belongs to the set of $N$th roots of unity
%\begin{equation}
%\label{roots}
%    \mU_N:=
%    \big\{
%        \re^{2\pi i\ell/N}:\
%        \ell = 0, \ldots, N-1
%    \big\}.
%\end{equation}
%In this case, the quantum processes $X_z$, $W_z$, $U_z$, $Y_z$ in (\ref{XW})--(\ref{Y}), as functions of $\ell$ in (\ref{roots}),  become the spatial discrete Fourier transforms (DFT) of the quantum processes $x_k$, $w_k$,  $u_k$,  $y_k$  over $k = 0, \ldots, N-1$, with $2\pi\ell/N$ playing the role of a wavenumber.
%%
%Recall that the set $\mU_N$ is a multiplicative group which is isomorphic to the additive group of residues modulo $N$. Since $\mU_N$ is a subset of the  unit circle
%$$
%    \mU:= \{z \in \mC:\ |z| = 1\}
%$$
%in the complex plane, the inversion in $\mU_N$ is equivalent to the complex conjugation: $z^{-1} = \overline{z}$.
%%
%In view of (\ref{UY}), for any $z\in \mU_N$, the quantum processes $U_z$, $Y_z$  are related by a static (time-independent) unitary transformation as
%\begin{equation}
%\label{K}
%    U_z
%    =
%    {\small\begin{bmatrix}
%        z^{-1}Y_z^+\\
%        zY_z^-
%    \end{bmatrix}}
%    =
%    K_z
%    Y_z,
%    \qquad
%    K_z
%    :=
%    {\small\begin{bmatrix}
%        z^{-1}I_{m_+} & 0\\
%        0 & zI_{m_-}
%    \end{bmatrix}},
%\end{equation}
%where use is also made of (\ref{Y}). The relationship (\ref{K}) describes two feedback loops with phase shift factors $z^{\pm1}$ (see Fig.~\ref{fig:blockz})
%%==============================================================================
%\begin{figure}[htbp]
%\centering
%\unitlength=0.45mm
%\linethickness{0.4pt}
%\begin{picture}(60.00,70.00)
%    \put(20, 25){
%        \framebox(20,20)[cc]{{\small$F$}}
%        \put(-15,40){\framebox(10,10)[cc]{{\small$z^{-1}$}}}
%        \put(-15,-30){\framebox(10,10)[cc]{{\small$z$}}}
%        \put(-10,28){\vector(0,-1){8}}
%%        \put(-10,0){\vector(0,-1){8}}
%
%        \put(-20,5){\vector(-1,0){20}}
%        \put(20,5){\vector(-1,0){20}}
%
%        \put(-40,15){\vector(1,0){20}}
%        \put(0,15){\vector(1,0){20}}
%        \put(20,15){\vector(0,1){30}}
%        \put(20,45){\vector(-1,0){25}}
%        \put(-15,45){\vector(-1,0){25}}
%        \put(-40,45){\vector(0,-1){30}}
%
%        \put(-40,5){\vector(0,-1){30}}
%        \put(-40,-25){\vector(1,0){25}}
%        \put(-5,-25){\vector(1,0){25}}
%        \put(20,-25){\vector(0,1){30}}
%
%        \put(-10,32){\makebox(0,0)[cc]{{\small$W_z$}}}
%
%%        \put(-10,-13){\makebox(0,0)[cc]{$R_z$}}
%
%        \put(-45,15){\makebox(0,0)[rc]{{\small$U_z^+$}}}
%        \put(-45,5){\makebox(0,0)[rc]{{\small$Y_z^-$}}}
%
%        \put(25,15){\makebox(0,0)[lc]{{\small$Y_z^+$}}}
%        \put(25,5){\makebox(0,0)[lc]{{\small$U_z^-$}}}
%    }
%\end{picture}
%\caption{A block diagram of the QSDEs (\ref{dX}), (\ref{dY}) combined with (\ref{K}).}
%\label{fig:blockz}
%\end{figure}
%%==============================================================================
%and  allows $U_z$ to be eliminated from the QSDE (\ref{dX}) by substituting $\rd Y_z$ from (\ref{dY}) into $\rd U_z = K_z \rd Y_z$, which  yields  the QSDE
%\begin{equation}
%\label{dX1}
%    \rd X_z
%    =
%    \sA_z X_z \rd t + \sB_z \rd W_z,
%\end{equation}
%with
%\begin{equation}
%\label{cAB}
%    \sA_z := A + EK_z  C,
%    \qquad
%    \sB_z := B + E K_z D.
%\end{equation}
%This procedure can be  justified by the elimination of edges in a quantum feedback network in the zero time delay limit \cite{GJ_2009}.
%%
%Since the diagonal matrix $K_z$ in (\ref{K}) satisfies
%$
%    K_{1/z} = \overline{K_z}
%$ for any $z$ on the unit circle, and the state-space matrices of the QSDEs (\ref{dx}), (\ref{dy}) are real, then (\ref{cAB}) implies that
%\begin{equation}
%\label{star}
%    \sA_z^* = \sA_{1/z}^{\rT},
%    \quad
%    \sB_z^* = \sB_{1/z}^{\rT},
%    \qquad
%    z \in \mU,
%\end{equation}
%where $(\cdot)^*:= (\overline{(\cdot)})^{\rT}$ is the complex conjugate transpose of a matrix.
%Note that $W_z(t)$ in (\ref{XW}), as functions of time $t$,   are quantum Wiener processes whose  joint quantum Ito table, in view of (\ref{ww}),  is computed as
%\begin{equation}
%\label{WW}
%    \rd W_z \rd W_v^{\dagger}
%    \!=\!\!
%    \sum_{j,k=0}^{N-1}
%    z^{-j}v^k
%    \rd w_j \rd w_k^{\rT}
%    \!=\!\!
%    \sum_{k=0}^{N-1}\!
%    \Big(\frac{v}{z}\Big)^k
%        \Omega \rd t
%    \!=\!
%    N \delta_{zv}\Omega \rd t\!
%\end{equation}
%for any roots of unity $z, v \in \mU_N$ from (\ref{roots}), with $(\cdot)^{\dagger}:= ((\cdot)^{\#})^{\rT}$ the transpose
%of the entry-wise adjoint $(\cdot)^{\#}$ of a matrix of operators.
%  Here, use is also made of the bilinearity of the commutator, the group property of $\mU_N$, and the identity
%$
%    \sum_{k=0}^{N-1}
%    \zeta^k
%    =
%    N \delta_{\zeta 1}
%%    \qquad
%%    \zeta \in \mU_N.
%%
%%    \left\{
%%        \begin{matrix}
%%            0 & {\rm if} & \zeta \in \mU_N\setminus \{1\}        \\
%%            N & {\rm if} &\zeta = 1
%%        \end{matrix}
%%    \right..
%$ for all $\zeta \in \mU_N$.
%Note that the right-hand side of  (\ref{WW}) is nonzero only for $z=v$, and hence, $W_z$ and $W_v$ are uncorrelated and commuting for $z\ne v$.
%In accordance with (\ref{wCCR}), the relation (\ref{WW}) implies
%\begin{align}
%\label{WCCR}
%    [\rd W_z, \rd W_v^{\dagger}]
%    =
%    2i \delta_{zv} N  J \rd t.
%%    =
%%    \left\{
%%        \begin{matrix}
%%            NJ & {\rm if}\ z = v\\
%%            0 & {\rm otherwise}
%%        \end{matrix}
%%    \right..
%\end{align}
%Therefore, instead of the original system of coupled QSDEs  (\ref{dx}), (\ref{dy}), (\ref{connect})
%(whose number increases with the chain fragment size $N$), we have obtained the algebraically closed QSDEs (\ref{dY}),  (\ref{dX1}) parameterized by $z \in \mU$. For any fixed  $z$, these QSDEs describe a  linear quantum stochastic system $F_z$ with the quadruple of state-space matrices $\sA_z$, $\sB_z$, $C$, $D$ in (\ref{cAB}) and input and output fields $W_z$ and $Y_z$.
%%; see Fig.~\ref{fig:Fourier}.
%%%==============================================================================
%%\begin{figure}[htbp]
%%\centering
%%\unitlength=0.4mm
%%\linethickness{0.4pt}
%%\begin{picture}(70.00,20.00)
%%    \put(20, 0){
%%        \framebox(20,20)[cc]{$F_z$}
%%
%%        \put(-40,10){\vector(1,0){20}}
%%        \put(0,10){\vector(1,0){20}}
%%
%%        \put(-45,10){\makebox(0,0)[cc]{$W_z$}}
%%        \put(25,10){\makebox(0,0)[cc]{$Y_z$}}
%%    }
%%\end{picture}
%%\caption{
%%    The system $F_z$, parameterized by the roots of unity $z \in \mU_N$ in (\ref{roots}) and  governed by the QSDEs (\ref{dY}), (\ref{dX1}).
%%}
%%\label{fig:Fourier}
%%\end{figure}
%%%==============================================================================
%This family of quantum systems (in which every member can be regarded independently of the others) encodes the dynamics of the whole network. In fact, the systems $F_z$, considered for different values of $z$, are analogous to the independent spatial modes of vibration in  the phonon theory of crystal lattices \cite{S_1990}.
%



%%%%%%%%%%%%%%%%%%%%%%%%%%%%%%%%%%%%%%%%%%%%%%%%%%%%%%%%%%%%%%%%%%%%%%%%%%%%%%%%%%%%%%%%%%%%%%%%%%%%
%\section{FINITE-RANGE COUPLING AND THE MULTIVARIATE SAMPLING THEOREM}\label{sec:fin}
%%%%%%%%%%%%%%%%%%%%%%%%%%%%%%%%%%%%%%%%%%%%%%%%%%%%%%%%%%%%%%%%%%%%%%%%%%%%%%%%%%%%%%%%%%%%%%%%%%%%
%
%In the case of finite-range coupling mentioned in Section~\ref{sec:sys}, when the matrices  $A_\ell$, $B_\ell$, $C_\ell$, $D_\ell$ in (\ref{xj}) and (\ref{yj})  vanish beyond a discrete (hyper) cube
%\begin{equation}
%\label{cube}
%    \Lambda
%    :=
%    \Big\{
%        \ell:= (\ell_k)_{1\< k\< \nu} \in \mZ^\nu:\
%     \max_{1\< k\< \nu}|\ell_k|\< L
%     \Big\}
%%\label{cube}
%%    & = \big(
%%        [-L,L]\bigcap \mZ
%%    \big)^\nu
%\end{equation}
%(whose edge halflength $L$  is
%a nonnegative integer),  so that
%\begin{equation}
%\label{fin}
%      \begin{bmatrix}
%    A_\ell & B_\ell\\
%    C_\ell & D_\ell
%  \end{bmatrix}
%  =
%  0,
%  \qquad
%  \ell \in \mZ^\nu\setminus \Lambda,
%\end{equation}
%each of the subsystems is coupled with at most $(2L+1)^\nu -1$ of its neighbours in the network. In this case,
%the functions $\sA$, $\sB$, $\sC$, $\sD$ in (\ref{ABCD}) are trigonometric polynomials which can be recovered uniquely on the whole torus $\mT^\nu$ from their values at the grid
%\begin{equation}
%\label{grid}
%    \Sigma
%    :=
%    \Big\{
%        \frac{2\pi}{2L+1}
%        \ell:\
%        \ell \in \Lambda
%    \Big\}
%    =
%    \frac{2\pi}{2L+1}
%    \Lambda
%%    \big(
%%        [-L,L]
%%        \bigcap
%%        \mZ
%%    \big)^\nu
%\end{equation}
%of $(2L+1)^\nu$ points with stepsize $\frac{2\pi}{2L+1}$ along each coordinate axis. This is the content of the multivariate version \cite{PM_1962} of the Whittaker-Nyquist-Kotel'nikov-Shannon sampling theorem \cite{J_1977} and its circular counterpart \cite{S_1979}. More precisely, the restrictions of the functions $\sA$, $\sB$, $\sC$, $\sD$ in (\ref{ABCD}) to the grid $\Sigma$ in (\ref{grid}) are related to the matrices $A_\ell$, $B_\ell$, $C_\ell$, $D_\ell$ on the cube $\ell \in \Lambda$ in (\ref{cube}) by the discrete Fourier transform whose inverse is
%\begin{equation}
%\label{DFT}
%  {\begin{bmatrix}
%    A_\ell & B_\ell\\
%    C_\ell & D_\ell
%  \end{bmatrix}}
%  =
%  \frac{1}{(2L+1)^\nu}
%  \sum_{\sigma\in \Sigma}
%  \re^{i\ell^{\rT}\sigma}
%  {\begin{bmatrix}
%    \sA(\sigma) & \sB(\sigma)\\
%    \sC(\sigma) & \sD(\sigma)
%  \end{bmatrix}},
%  \quad
%  \ell \in \Lambda.
%\end{equation}
%In view of (\ref{fin}),
%substitution of (\ref{DFT}) into (\ref{ABCD}) leads to the relation
%\begin{align}
%\nonumber
%  {\begin{bmatrix}
%    \sA(\sigma) & \sB(\sigma)\\
%    \sC(\sigma) & \sD(\sigma)
%  \end{bmatrix}}
%  & =
%  \sum_{\ell\in \Lambda}
%  \re^{-i\ell^{\rT}\sigma}
%  \begin{bmatrix}
%    A_\ell & B_\ell\\
%    C_\ell & D_\ell
%  \end{bmatrix}\\
%\nonumber
%    & =
%  \frac{1}{(2L+1)^\nu}
%  \sum_{\ell\in \Lambda,\gamma\in \Sigma}
%    \re^{i\ell^{\rT}(\gamma-\sigma)}
%  {\begin{bmatrix}
%    \sA(\gamma) & \sB(\gamma)\\
%    \sC(\gamma) & \sD(\gamma)
%  \end{bmatrix}}\\
%\label{ABCDfin}
%    & =
%  \sum_{\gamma\in \Sigma}
%  \phi(\gamma-\sigma)
%  {\begin{bmatrix}
%    \sA(\gamma) & \sB(\gamma)\\
%    \sC(\gamma) & \sD(\gamma)
%  \end{bmatrix}},
%\end{align}
%which holds for any $\sigma\in \mT^\nu$. Here, the kernel function of the summation is computed as
%\begin{equation}
%\label{phi}
%    \phi(\tau)
%    :=
%  \frac{1}{(2L+1)^\nu}
%  \sum_{\ell\in \Lambda}
%    \re^{i\ell^{\rT}\tau}
%    =
%    \prod_{j=1}^\nu
%    \frac{\sin((L+\frac{1}{2})\tau_j)}{(2L+1)\sin\frac{\tau_j}{2}}
%\end{equation}
%for any $\tau:= (\tau_j)_{1\< j\< \nu}\in \mR^\nu$. The rightmost fraction in (\ref{phi}) is extended by continuity to $1$ at $\tau_j=0$, so that   $\phi(0)=1$ in accordance with the interpolation property of (\ref{ABCDfin}) with respect to the grid $\Sigma$ in (\ref{grid}).


%%%%%%%%%%%%%%%%%%%%%%%%%%%%%%%%%%%%%%%%%%%%%%%%%%%%%%%%%%%%%%%%%%%%%%%%%%%%%%%%%%%%%%%%%%%%%%%%%%%
\section{PHONON THEORETIC
DISPERSION RELATIONS}\label{sec:phon}
%%%%%%%%%%%%%%%%%%%%%%%%%%%%%%%%%%%%%%%%%%%%%%%%%%%%%%%%%%%%%%%%%%%%%%%%%%%%%%%%%%%%%%%%%%%%%%%%%%%

We will now consider a class of translation invariant networks, which, in isolation from the environment, manifest phonon-like dynamics mentioned in Section~\ref{sec:freq}. Suppose the state vectors  $x_k$  of the component systems consist of the conjugate position and momentum variables which are assembled  into $\frac{n}{2}$-dimensional vectors $q_k$ and $p_k$, respectively (with $n$ being even):
\begin{equation}
\label{xqp}
    x_k
    =
    \begin{bmatrix}
        q_k\\
        p_k
    \end{bmatrix},
    \qquad
    k \in \mZ^\nu.
\end{equation}
Accordingly, the network is assumed to have the following Hamiltonian:
\begin{equation}
\label{Hqp}
    H
    :=
    \frac{1}{2}
    \sum_{j\in \mZ^\nu}
    \Big(
        \|p_j\|_{M^{-1}}^2
        +
        q_j^{\rT}
        \sum_{k\in \mZ^\nu}
        K_{j-k}
        q_k
    \Big).
\end{equation}
Here, $M$ is a real positive definite symmetric mass matrix of order $\frac{n}{2}$, and use is made of a weighted Euclidean norm $\|v\|_L := \sqrt{v^{\rT} L v} = |\sqrt{L}v|$ of a vector $v$ specified by such a matrix $L$. Also,
$K:=(K_{j-k})_{j,k\in \mZ^\nu}$ is a symmetric block Toeplitz stiffness operator whose blocks $K_{\ell} \in \mR^{\frac{n}{2}\x \frac{n}{2}}$ specify the potential energy part of the Hamiltonian. In view of (\ref{xqp}) and (\ref{Hqp}),  the blocks of the corresponding
block Toeplitz matrix $V$ in (\ref{H}) are given by
\begin{equation}
\label{VKM}
    V_{\ell}
    =
    \begin{bmatrix}
        K_\ell & 0\\
        0 & \delta_{0\ell}M^{-1}
    \end{bmatrix},
    \qquad
    \ell \in \mZ^\nu,
\end{equation}
where
$\delta_{jk}$ is the Kronecker delta. In application to the isolated network, the Hamiltonian equations of motion take the form
\begin{equation}
\label{qpdot}
    \dot{x}_j
    =
    \begin{bmatrix}
        \dot{q}_j\\
        \dot{p}_j
    \end{bmatrix}
%    \\
%\nonumber
    =
    J \d_{x_j}H
    =
    \begin{bmatrix}
        \d_{p_j}H\\
        -\d_{q_j}H
    \end{bmatrix}
    =
    \begin{bmatrix}
        M^{-1} p_j\\
        -\sum_{k\in \mZ^\nu}
    K_{j-k} q_k
    \end{bmatrix}
%\label{qdot}
%    \dot{q}_j
%    & =
%    M^{-1} p_j,\\
%\label{pdot}
%    \dot{p}_j
%    & =
%    -\sum_{k\in \mZ^\nu}
%    K_{j-k} q_k,
\end{equation}
for all $j\in \mZ^\nu$,
where
\begin{equation}
\label{J}
    J:=
    \begin{bmatrix}
        0 & 1 \\
        -1 & 0
    \end{bmatrix}
    \ox I_{n/2}
\end{equation}
is the symplectic structure matrix associated with the partitioning (\ref{xqp}), with $\ox$ the Kronecker product of matrices. A comparison of (\ref{qpdot}) with (\ref{xj}) allows the blocks of the matrix $A$ in (\ref{x}) to be expressed in terms of (\ref{VKM}) and (\ref{J}) as
\begin{equation}
\label{AKM}
  A_{\ell}
  =
  J V_{\ell}
  =
    \begin{bmatrix}
        0 & \delta_{0\ell}M^{-1}\\
        -K_\ell & 0
    \end{bmatrix},
    \qquad
    \ell \in \mZ^\nu.
\end{equation}
Upon substituting (\ref{AKM}) into (\ref{ABCD}),
the function $\sA$ takes the form
\begin{equation}
\label{sAKM}
  \sA(\sigma)
  =
    \begin{bmatrix}
        0 & M^{-1}\\
        -\sK(\sigma) & 0
    \end{bmatrix},
\end{equation}
where
\begin{equation}
\label{sK}
    \sK(\sigma)
    :=
    \sum_{\ell\in \mZ^\nu}
    \re^{-i\ell^{\rT}\sigma}
    K_\ell.
\end{equation}
Since $M\succ 0$, then $s\in \mC$ is an eigenvalue of the matrix $\sA(\sigma)$ in (\ref{sAKM}) if and only if $s^2$ is an eigenvalue of $-M^{-1/2}\sK(\sigma)M^{-1/2}$. Therefore, if the stiffness operator $K$ is positive semi-definite (so that $0$ delivers a minimum to the potential energy), then $\sA(\sigma)$ has a purely imaginary spectrum:
\begin{equation}
\label{spec}
    \fS(\sA(\sigma))
    =
    \big\{
        \pm i\sqrt{\lambda}:\
        \lambda \in \fS(\sK(\sigma)M^{-1})
    \big\}
\end{equation}
for any $\sigma \in \mT^\nu$. Such a network is organised as an infinite-dimensional spring-mass system whose vibrations are linear superpositions of ``basis'' phonons given by (\ref{phon}) with $s=\pm i \sqrt{\lambda}$ for different values of $\sigma \in \mT^\nu$. The matrix $\sK(\sigma)M^{-1} = \sqrt{M}M^{-1/2}\sK(\sigma)M^{-1/2}M^{-1/2}$ in (\ref{spec}),  which is isospectral to $M^{-1/2}\sK(\sigma)M^{-1/2}\succcurlyeq 0$, corresponds to the stiffness-to-mass ratio for one-mode harmonic oscillators. The dynamic properties of phonons with large wavelengths $\frac{2\pi}{|\sigma|}$ (manifesting themselves in highly correlated motions of distant subsystems in the network) depend on the asymptotic behaviour of the matrix $\sK(\sigma)$ in (\ref{sK}) in a small neighbourhood of $\sigma = 0$.

%The phonons in the network have bounded phase velocities if $\sK$ is a continuously differentiable function


%%%%%%%%%%%%%%%%%%%%%%%%%%%%%%%%%%%%%%%%%%%%%%%%%%%%%%%%%%%%%%%%%%%%%%%%%%%%%%%%%%%%%%%%%%%%%%%%%%%
\begin{thm}
\label{th:phon}
Suppose the isolated translation invariant network (\ref{qpdot}) with the Hamiltonian (\ref{Hqp}) has a positive semi-definite stiffness operator $K$ satisfying
\begin{align}
\label{K0}
    \sum_{\ell \in \mZ^\nu}
    K_{\ell} & = 0,\\
%\label{K1}
%    \sum_{\ell \in \mZ^\nu}
%    \ell \ox K_{\ell} & = 0,\\
\label{K2}
  \sum_{\ell \in \mZ^\nu}
  |\ell|^2
  \|K_\ell\|
  & <+\infty.
\end{align}
Then the phase velocities of the phonons in the network are uniformly bounded:
\begin{equation}
\label{speed}
  %\sup_{\sigma \in \mT^\nu\setminus\{0\},\ \omega \in \mR_+:\ i\omega \in \fS(\sA(\sigma))}
  \sup
  \Big\{
        \frac{\omega}{|\sigma|}:\
        \sigma
        \in
        \mT^\nu\setminus\{0\},\,
        \omega
        \in
        \mR_+,\,
        i\omega
        \in
        \fS(\sA(\sigma))
  \Big\}
  <+\infty.
\end{equation}
\hfill$\square$
\end{thm}
%%%%%%%%%%%%%%%%%%%%%%%%%%%%%%%%%%%%%%%%%%%%%%%%%%%%%%%%%%%%%%%%%%%%%%%%%%%%%%%%%%%%%%%%%%%%%%%%%%%
\begin{proof}
The condition (\ref{K2}) ensures twice continuous differentiability of the function $\sK$ in (\ref{sK}).
Since $K \succcurlyeq 0$ is equivalent to $\sK(\sigma)\succcurlyeq 0$ for all $\sigma \in \mT^\nu$, then (\ref{K0}) implies that $\sK(0)=0$. Hence, $\sK$ has zero Frechet derivative at the origin: $\sK'(0) = 0$ (that is, $\sum_{\ell \in \mZ^\nu}\ell \ox K_{\ell} = 0$). Indeed, otherwise, the leading term of the linearly truncated Taylor series expansion $\sK(\sigma) = \sum_{j=1}^\nu \sigma_j \d_{\sigma_j}\sK(\sigma)\big|_{\sigma = 0} +o(|\sigma|)$ would fail  to be a positive semi-definite matrix for arbitrary $\sigma:= (\sigma_k)_{1\< k\<\nu}$. Therefore, the quadratically truncated Taylor series expansion of $\sK$ in a small neighbourhood of the origin reduces to
\begin{align}
\nonumber
    \sK(\sigma)
    & =
    \frac{1}{2}
    \sum_{j,k=1}^\nu
    \sigma_j \sigma_k
    \Gamma_{jk}
%    \d_{\sigma_j}\d_{\sigma_k}
%    \sK(\sigma)\big|_{\sigma = 0}
    +
    o(|\sigma|^2)\\
\label{sKquad}
    & =
    |\sigma|^2
    \left(
    \frac{1}{2}
    \sum_{j,k=1}^\nu
    \theta_j \theta_k
    \Gamma_{jk}
    +
    o(1)
    \right),
    \quad
    {\rm as}\ \sigma \to 0.
\end{align}
Here, we have associated with $\sigma\ne 0$ the unit direction vector     $\theta
    :=
    (\theta_k)_{1\< k\< \nu}$ by
\begin{equation}
\label{theta}
    \theta
    :=
    \frac{\sigma}{|\sigma|},
\end{equation}
and the matrices $\Gamma_{jk} \in \mR^{\frac{n}{2}\x \frac{n}{2}}$ specify the second Frechet derivative of the function $\sK$  as
\begin{equation}
\label{Gammajk}
    \Gamma_{jk}
    :=
    \d_{\sigma_j}\d_{\sigma_k}
    \sK(\sigma)
    \big|_{\sigma = 0}
    =
    -
    \sum_{\ell \in \mZ^\nu}
    \ell_j\ell_k
    K_{\ell}
\end{equation}
in view of (\ref{sK}). The leading term     $\sum_{j,k=1}^\nu
    \sigma_j \sigma_k
    \Gamma_{jk}$ in (\ref{sKquad}) is a real positive semi-definite symmetric matrix of order $\frac{n}{2}$ for any $\sigma \in \mR^\nu$. Now, since the spectrum (as a set-valued function) of a matrix depends continuously on it \cite{H_2008,HJ_2007}, then, in view of (\ref{spec}),
\begin{equation}
\label{ommax}
  \sup_{\sigma \in \mT^\nu}
  \br(\sA(\sigma))
  =
  \sqrt{\max_{\sigma \in \mT^\nu}
  \lambda_{\max}(\sK(\sigma)M^{-1})}
  <+\infty,
\end{equation}
where $\lambda_{\max}(\cdot)$ is the largest eigenvalue of a matrix with a real spectrum. Hence, the left-hand side of (\ref{speed}) can be unbounded only because of the asymptotic behaviour of $\br(\sA(\sigma))$ as $\sigma \to 0$  (which corresponds to long-wave phonons). However, it follows from (\ref{spec}) and (\ref{sKquad})--(\ref{Gammajk}) that
\begin{align}
\nonumber
  &\limsup_{\sigma\to 0}
  \frac{\br(\sA(\sigma))}{|\sigma|}
  =
  \limsup_{\sigma\to 0}
  \frac{\sqrt{\lambda_{\max}(\sK(\sigma)M^{-1})}}{|\sigma|}  \\
\label{speed0}
  & =
  \max_{\theta\in \mR^\nu:\ |\theta| = 1}
  \sqrt{
  \frac{1}{2}
  \lambda_{\max}
  \Big(
      \sum_{j,k=1}^\nu
    \theta_j \theta_k
    \Gamma_{jk}
    M^{-1}
    \Big)
  }
   <+\infty.
%  \Big\{
%        \frac{\omega}{|\sigma|}:\
%        \omega
%        \in
%        \mR_+,\,
%        i\omega
%        \in
%        \fS(\sA(\sigma))
%  \Big\}
\end{align}
A combination of (\ref{ommax}) and (\ref{speed0}) leads to
%the uniform boundedness of the phonon phase velocities in
(\ref{speed}).
\end{proof}
%%%%%%%%%%%%%%%%%%%%%%%%%%%%%%%%%%%%%%%%%%%%%%%%%%%%%%%%%%%%%%%%%%%%%%%%%%%%%%%%%%%%%%%%%%%%%%%%%%%

The speed of propagation of disturbances is described in physics literature  in terms of their group velocity $\d_{\sigma}\omega$ (see, for example, \cite{LL_1980}). Note that the differentiability  is ensured for those phonons whose frequency $\omega$ corresponds to a non-degenerate  eigenvalue (of multiplicity one) of the matrix $\sK(\sigma)M^{-1}$ in (\ref{spec}).
%, which is why  we will use the phase velocities instead.
%%%%%%%%%%%%%%%%%%%%%%%%%%%%%%%%%%%%%%%%%%%%%%%%%%%%%%%%%%%%%%%%%%%%%%%%%%%%%%%%%%%%%%%%%%%%%%%%%%%%
%\section{TRANSLATION INVARIANT SPARSE FEEDBACK WITH COLLOCATION}\label{sec:feed}
%%%%%%%%%%%%%%%%%%%%%%%%%%%%%%%%%%%%%%%%%%%%%%%%%%%%%%%%%%%%%%%%%%%%%%%%%%%%%%%%%%%%%%%%%%%%%%%%%%%%
%
%Since, as mentioned in Section~\ref{sec:sys}, block Toeplitz matrices form an algebra, a feedback, described by such an operator $\Phi:= (\Phi_{j-k})_{j,k\in \mZ^\nu}$  (mapping the output $y$  to the input $u=\Phi y$), with $\Phi_\ell \in \mR^{m\x r}$, leads to a translation invariant network. This feedback is static (memoryless) since the current input depends only on the current output. The resulting closed-loop network is  governed by
%\begin{equation}
%  \dot{x}
%  =
%  \cA x,
%\end{equation}
%where
%\begin{equation}
%\label{cA}
%    \cA := A + B(\cI - \Phi D)^{-1} \Phi C
%\end{equation}
%is also a block Toeplitz matrix, with $\cI$ denoting the identity operator.
%
%
%%%%%%%%%%%%%%%%%%%%%%%%%%%%%%%%%%%%%%%%%%%%%%%%%%%%%%%%%%%%%%%%%%%%%%%%%%%%%%%%%%%%%%%%%%%%%%%%%%%%
%\section{A LINEAR QUADRATIC REGULATOR PROBLEM}\label{sec:LQR}
%%%%%%%%%%%%%%%%%%%%%%%%%%%%%%%%%%%%%%%%%%%%%%%%%%%%%%%%%%%%%%%%%%%%%%%%%%%%%%%%%%%%%%%%%%%%%%%%%%%%
%
%%%%%%%%%%%%%%%%%%%%%%%%%%%%%%%%%%%%%%%%%%%%%%%%%%%%%%%%%%%%%%%%%%%%%%%%%%%%%%%%%%%%%%%%%%%%%%%%%%%%
%\section{APPLICATION TO THIN PLATE VIBRATIONS}\label{sec:plate}
%%%%%%%%%%%%%%%%%%%%%%%%%%%%%%%%%%%%%%%%%%%%%%%%%%%%%%%%%%%%%%%%%%%%%%%%%%%%%%%%%%%%%%%%%%%%%%%%%%%%
%
We will now provide an example which illustrates the condition (\ref{K0}).
Consider
the movement of a thin isotropic plate of density $\rho>0$ and bending stiffness $\beta>0$, governed by the Kirchhoff-Love equation \cite{R_2007}:
\begin{equation}
\label{PDE}
    \rho\d_t^2 w = -\frac{\beta}{2} \Delta^2 w +f.
\end{equation}
Here, $w(t,\xi)$ is the local deviation of the plate from the equilibrium (unbent) position, which is a real-valued  function of time $t$ and the two-dimensional vector $\xi:= {\small\begin{bmatrix}\xi_1\\ \xi_2\end{bmatrix}} \in \mR^2$ of Cartesian coordinates on the plane, and $\Delta:= \d_{\xi_1}^2 + \d_{\xi_2}^2$ is the Laplacian over the spatial variables. Also, $f(t,\xi)$ is the density of the external force acting transversally on the plate. A finite-difference scheme  for the numerical solution of the PDE (\ref{PDE}) with spatial step size $h>0$ can be viewed as a translation invariant network on the two-dimensional integer lattice $\mZ^2$. By denoting $\mho:= (\mho_{jk})_{j,k\in \mZ}$ the spatial discretization of the solution of (\ref{PDE}) at time $t\>0$ (so that $\mho_{jk}(t)$ approximates $w(t,jh,kh)$), the corresponding functions
\begin{equation}
\label{qpmho}
    q:= \mho,
    \qquad
    p:= \rho \d_t \mho
\end{equation}
on the set $\mR_+\x \mZ^2$
are evolved according to the Hamilton equations
\begin{equation}
\label{qpmhodot}
    \dot{q}
    =
    \frac{1}{\rho} p,
    \qquad
    \dot{p}
    =
    -\frac{\beta}{2}L^2 q + g.
\end{equation}
Here, $g:= (f(t,jh,kh))_{j,k\in \mZ}$ is a discretization of the forcing term,
and $L$ is a self-adjoint Toeplitz operator which approximates the Laplacian $\Delta$ as
\begin{equation}
\label{Lmho}
    (L\mho)_{jk}
    :=
    \frac{1}{h^2}
    (
        q_{j+1,k}+q_{j-1,k}
        +
        q_{j,k+1}+q_{j,k-1}
        -
        4q_{jk}
    ).
\end{equation}
A comparison of (\ref{qpmho}) and (\ref{qpmhodot}) with (\ref{qpdot}) shows that the scalar $\rho$ plays the role of the mass matrix $M$, and $\frac{\beta}{2}L^2$ is the stiffness operator $K$. The latter implies that the Fourier transforms (\ref{sK}) and $
    \sL(\sigma)
    :=
    \sum_{\ell\in \mZ^2}
    \re^{-i\ell^{\rT}\sigma}
    L_\ell
$ are related by
$\sK = \frac{\beta}{2}\sL^2$, and hence, $K$ inherits the property (\ref{K0}) from the discretized Laplacian $L$ in (\ref{Lmho}), since $\sL(0) = \sum_{\ell \in \mZ^2} L_\ell = 0$.



%%%%%%%%%%%%%%%%%%%%%%%%%%%%%%%%%%%%%%%%%%%%%%%%%%%%%%%%%%%%%%%%%%%%%%%%%%%%%%%%
%\section{PORT-HAMILTONIAN SYSTEMS}
%\label{sec:portham}
%%%%%%%%%%%%%%%%%%%%%%%%%%%%%%%%%%%%%%%%%%%%%%%%%%%%%%%%%%%%%%%%%%%%%%%%%%%%%%%%
%
%Consider a port-Hamiltonian system with $n$ degrees of freedom described by generalised positions $q_1, \ldots, q_n$ and the corresponding momenta $p_1, \ldots, p_n$ which are assembled into the vectors
%\begin{equation}
%%\label{xqp}
%    x
%    :=
%    \begin{bmatrix}
%        q\\
%        p
%    \end{bmatrix},
%    \qquad
%    q
%    :=
%    \begin{bmatrix}
%        q_1\\
%        \vdots\\
%        q_n
%    \end{bmatrix},
%    \qquad
%    p
%    :=
%    \begin{bmatrix}
%        p_1\\
%        \vdots\\
%        p_n
%    \end{bmatrix}.
%\end{equation}
%The internal energy of the system is specified by a Hamiltonian $H: \mR^n\x \mR^n\to \mR$ which is a twice continuously differentiable  function of the positions and momenta. In isolation from the environment, the Hamiltonian system is governed by the equations
%\begin{equation}
%\label{xdotiso}
%    \dot{x}
%    =
%    \begin{bmatrix}
%        \dot{q}\\
%        \dot{p}
%    \end{bmatrix}
%    =
%    \begin{bmatrix}
%        \d_p H\\
%        -\d_q H
%    \end{bmatrix}
%    =
%    J
%    H'.
%\end{equation}
%Here, $\dot{(\ )}$ denotes the time derivative, and $H' := \d_xH$, where $\d_q(\cdot)$, $\d_p(\cdot)$ and $\d_x(\cdot):= {\small     \begin{bmatrix}
%        \d_q(\cdot)\\
%        \d_p(\cdot)
%    \end{bmatrix}
%}$ are the corresponding gradient operators.
%
%In this case, the time derivative of a continuously differentiable function $f: \mR^n\x \mR^n\to \mR$ along the phase trajectory of the system in $\mR^{2n}$ is computed as
%\begin{equation}
%\label{fdotiso}
%    f(x)^{^\centerdot}
%    =
%    f'^{\rT} JH'
%    =
%    \{f,H\},
%\end{equation}
%in view of (\ref{xdotiso}),
%where $\{\varphi, \psi\}:= \varphi'^{\rT} J\psi'$ is the Poisson bracket associated with (\ref{J}). Here, $(\cdot)^{\rT}$ is the matrix transpose, and, unless indicated otherwise, vectors are organised as columns. Since the Hamiltonian $H$ is a conserved quantity for the isolated system in view of $\{H,H\}=0$ (due to the matrix $J$ being antisymmetric), then so also is any function of $H$. In order to take into account energy dissipation and interaction with the environment, the dynamics of the open port-Hamiltonian system acquires additional terms:
%\begin{equation}
%%\label{xdot}
%    \dot{x}
%    =
%    (J-R)H'(x)
%    +
%    g(x)w.
%\end{equation}
%In this ODE, the matrix-valued function $g:\mR^{2n}\to \mR^{2n\x m}$ comes from the coupling of the system with its environment which is represented by a time-varying  $\mR^m$-valued quantity $w$ interpreted as an external force acting on the system \cite{?}. Also, $R$ is a real positive semi-definite symmetric matrix of order $2n$ which quantifies the energy dissipation (for example, due to mechanical friction or electrical resistance). More precisely, in accordance with (\ref{xdot}), the internal energy of the open system evolves as
%\begin{align}
%\nonumber
%    \dot{H}
%    & =
%    H'^{\rT}\dot{x}\\
%\nonumber
%    & =
%    H'^{\rT}
%    ((J-R)H'
%    +
%    gw)\\
%%\label{Hdot}
%    & =
%    z^{\rT}w
%    -\|H'\|_R^2,
%\end{align}
%where $\|v\|_R:=\sqrt{v^{\rT} R v} = |\sqrt{R}v|$ is a semi-norm of a vector $v\in \mR^{2n}$, and
%\begin{equation}
%\label{z}
%    z := g(x)^{\rT}H'(x)
%\end{equation}
%is an $\mR^m$-valued output of the system whose entries are interpreted as velocity variables. The quantity $z^{\rT}w$ in (\ref{Hdot}) describes the work done by the environment on the system per unit time and is referred to as the (energy) supply rate, while $\|H'\|_R^2$ is the energy dissipation rate. Note that the energy balance equation (\ref{Hdot}) remains valid if the system output $z$ in (\ref{z}) has a more general form:
%\begin{equation}
%\label{zw}
%    z := g(x)^{\rT}H'(x) + Kw,
%\end{equation}
%where $K$ is an arbitrary real antisymmetric matrix of order $m$ (so that $w^{\rT}Kw=0$ for any $w$).  In a more general case, the matrices $J$, $R$ and $K$ in (\ref{Hdot}) and (\ref{zw}) can be state-dependent (subject to the symmetry and antisymmetry conditions mentioned above).
%
%%%%%%%%%%%%%%%%%%%%%%%%%%%%%%%%%%%%%%%%%%%%%%%%%%%%%%%%%%%%%%%%%%%%%%%%%%%%%%%%
%\section{POSITIVE REAL INPUT-OUTPUT SYSTEMS}
%\label{sec:posreal}
%%%%%%%%%%%%%%%%%%%%%%%%%%%%%%%%%%%%%%%%%%%%%%%%%%%%%%%%%%%%%%%%%%%%%%%%%%%%%%%%
%
%We will be concerned with a class of input-output systems with an $\mR^n$-valued state $x$, and $\mR^m$-valued input and  output $w$ and $z$ governed by the equations
%\begin{align}
%\label{xw}
%    \dot{x}
%    & =
%    f(x) + g(x)w,\\
%\label{zxw}
%    z
%    & = \varphi(x) + \psi(x)w
%\end{align}
%with an affine dependence on the input. Here,
%$f$, $g$, $\varphi$, $\psi$ are functions on the state space $\mR^n$ with values in $\mR^n$, $\mR^{n\x m}$, $\mR^m$, $\mR^{m\x m}$, respectively. Assuming, as before, that $w$ and $z$ consist of the effort and flow variables, a continuously differentiable function $V:\mR^n\to \mR$ is a candidate for describing the internal energy of the system (\ref{xw}), (\ref{zxw}) similarly to (\ref{Hdot}), if it satisfies the dissipation inequality
%\begin{align}
%\nonumber
%    \dot{V}
%    & =
%    V'(x)^{\rT}\dot{x}\\
%\nonumber
%    & =
%    V'(x)^{\rT}
%    (f(x)
%    +
%    g(x)w)\\
%\label{Vdot}
%    & \<
%    z^{\rT}w
%    =
%    w^{\rT}(\varphi(x) + \psi(x)w)
%\end{align}
%for all $x\in \mR^n$ and $w\in \mR^m$. By completing the squares, (\ref{Vdot}) is equivalent to
%\begin{align}
%\nonumber
%    0
%    & \<
%    z^{\rT}w- \dot{V}\\
%\nonumber
%    & =
%        w^{\rT}(\varphi + \psi w) -     V'^{\rT}
%    (f
%    +
%    gw)
%\\
%\nonumber
%    & =
%        w^{\rT}\bS(\psi)w
%        +
%        w^{\rT}(\varphi -g^{\rT}V')
%        -f^{\rT}V'\\
%\label{PDI0}
%    & =
%    \begin{bmatrix}
%      w \\
%      1
%    \end{bmatrix}^{\rT}
%    \begin{bmatrix}
%      \bS(\psi) & \frac{1}{2}(\varphi -g^{\rT}V') \\
%      \frac{1}{2}(\varphi -g^{\rT}V')^{\rT} & -f^{\rT}V'
%    \end{bmatrix}
%    \begin{bmatrix}
%      w \\
%      1
%    \end{bmatrix},
%\end{align}
%where $\bS(N):= \frac{1}{2}(N+N^{\rT})$ denotes the symmetrizer of matrices. A sufficient condition for (\ref{PDI0}) is
%\begin{equation}
%\label{pos}
%    \begin{bmatrix}
%      \bS(\psi) & \frac{1}{2}(\varphi -g^{\rT}V') \\
%      \frac{1}{2}(\varphi -g^{\rT}V')^{\rT} & -f^{\rT}V'
%    \end{bmatrix}
%    \succcurlyeq 0
%\end{equation}
%for all $x\in \mR^n$. Furthermore, if the map $\psi: \mR^n \to \mR^{m\x m}$ satisfies
%\begin{equation}
%\label{psipos}
%    \bS(\psi(x))\succ 0,
%    \qquad
%    x\in \mR^n,
%\end{equation}
%then the fulfillment of the inequality in (\ref{PDI0}) for any $w\in \mR^m$ is equivalent to
%\begin{align}
%\nonumber
%    0
%    & \<
%    \inf_{w\in \mR^m}
%    (z^{\rT}w- \dot{V})\\
%\nonumber
%    & =
%    \inf_{w\in \mR^m}
%    \big(
%        w^{\rT}(\varphi + \psi w) -     V'^{\rT}
%    (f
%    +
%    gw)
%    \big)\\
%\nonumber
%    & =
%    \inf_{w\in \mR^m}
%    \big(
%        w^{\rT}\bS(\psi)w
%        +
%        w^{\rT}(\varphi -g^{\rT}V')
%    \big)
%        -f^{\rT}V'\\
%\label{PDI}
%    & =
%    -\frac{1}{4}
%    \|\varphi -g^{\rT}V'\|_{\bS(\psi)^{-1}}^2
%    -f^{\rT}V',
%\end{align}
%whose right-hand side is the Schur complement \cite{HJ_2007} of the block $\bS(\psi)$ in the matrix (\ref{pos}).
%Note that in the case of the port-Hamiltonian system in (\ref{xdot}) and (\ref{zw}) with $V:=H$, the infimum in (\ref{PDI})  reduces to $\|H'\|_R^2$ regardless of whether the condition (\ref{psipos}) is satisfied. For a given system in (\ref{xw}) and (\ref{zxw}), the verification of positive realness  requires finding a solution $V$ of the partial differential inequalities (PDIs) (\ref{pos}) or (\ref{PDI}). By making $\varphi -g^{\rT}V'$ vanish (as in the port-Hamiltonian case),  each of these PDIs reduces to
%\begin{equation}
%\label{posred}
%  f^{\rT}V'\< 0
%\end{equation}
%for a narrower class of functions $V$ satisfying the differential inclusion
%\begin{equation}
%\label{V'}
%    V'(x)
%    \in
%    \ker (g(x)^{\rT})
%    +
%    g(x)(g(x)^{\rT} g(x))^{-1} \varphi(x),
%    \quad
%    x\in \mR^n,
%\end{equation}
%where $\ker(\cdot)$ is the null space of a matrix, and $g(x)$ is assumed to be of full column rank (that is, $g(x)^{\rT} g(x)\succ 0$ and hence, $m\< n$) for all $x \in \mR^n$. In the case when $f=Ax$ and $\varphi=Cx$ are linear functions of the state $x$, and $g = B \in \mR^{n\x m}$ and $\psi= D \in \mR^{m\x m}$ are constant matrices, with $A\in \mR^{n\x n}$ and $C \in \mR^{m\x n}$, the system in (\ref{xw}) and (\ref{zxw}) becomes linear:
%\begin{align}
%\label{xwlin}
%    \dot{x}
%    & =
%    Ax + Bw,\\
%\label{zxwlin}
%    z
%    & = Cx + Dw.
%\end{align}
%In this case, by assuming (in accordance with (\ref{psipos})) that the matrix $D$ satisfies
%\begin{equation}
%\label{Dpos}
%    \Delta:= \bS(D)\succ 0,
%\end{equation}
%it follows that the fulfillment of the  PDI (\ref{PDI}) can be achieved for a quadratic function
%\begin{equation}
%\label{VE}
%  V(x):= \frac{1}{2} x^{\rT} E x,
%  \qquad
%  x\in \mR^n,
%\end{equation}
%with a real symmetric  \emph{energy matrix} $E$ of order $n$. More precisely, under the condition (\ref{Dpos}),  the inequality (\ref{PDI}) takes the form
%\begin{align}
%\nonumber
%    0
%    & \<
%    -\frac{1}{4}
%    \|Cx -B^{\rT}Ex\|_{\Delta^{-1}}^2
%    -(Ax)^{\rT}Ex\\
%\label{PDI1}
%    & =
%    -
%    x^{\rT}
%    \Big(
%        \bS(A^{\rT} E)
%        +
%        \frac{1}{4}
%        (C -B^{\rT}E)^{\rT}\Delta^{-1}(C -B^{\rT}E)
%    \Big)
%    x
%\end{align}
%and holds for all $x\in \mR^n$ if and only if the matrix $E$ satisfies the algebraic Riccati inequality (ARI)
%\begin{equation}
%\label{ARI}
%    A^{\rT} E + EA
%        +
%        \frac{1}{2}
%        (C^{\rT} -EB)\Delta^{-1}(C -B^{\rT}E)
%        \preccurlyeq
%        0.
%\end{equation}
%Therefore, in the linear-quadratic case (\ref{xwlin})--(\ref{VE}), the fulfillment of the PDI (\ref{PDI}) over the state space $\mR^n$ (which involves infinitely many input-output values) reduces to an algebraic matrix inequality (\ref{ARI}) whose verification requires only a finite number of calculations.


%%%%%%%%%%%%%%%%%%%%%%%%%%%%%%%%%%%%%%%%%%%%%%%%%%%%%%%%%%%%%%%%%%%%%%%%%%%%%%%%
%\section{\bf Galerkin approximation of nonlinearities}
%\label{sec:Gal}
%%%%%%%%%%%%%%%%%%%%%%%%%%%%%%%%%%%%%%%%%%%%%%%%%%%%%%%%%%%%%%%%%%%%%%%%%%%%%%%%
%
%The idea of finite-dimensional parameterization of the positive real property can be extended from the linear-quadratic case to nonlinear systems. For this purpose, we assume that the functions $f$, $g$, $\varphi$, $\psi$ in (\ref{xw}) and (\ref{zxw}) are organised as linear combinations
%\begin{align}
%\label{fg}
%    f(x) & := \sum_{k=1}^N f_k b_k(x),
%    \qquad\
%    g(x)  := \sum_{k=1}^N g_k b_k(x),\\
%\label{phipsi}
%    \varphi(x) & := \sum_{k=1}^N \varphi_k b_k(x),
%    \qquad
%    \psi(x)  := \sum_{k=1}^N \psi_k b_k(x)
%\end{align}
%of $N$ linearly independent functions $b_k:\mR^n\to \mR$ with appropriately dimensioned coefficients  $f_k\in \mR^n$, $g_k\in \mR^{n\x m}$, $\varphi_k\in \mR^m$, $\psi_k\in \mR^{m\x m}$. For example, the basis functions $b_1,\ldots,b_N$ in (\ref{fg}), (\ref{phipsi}) can be quasi-monomials of the form $x^{\alpha}\re^{\beta^{\rT} x}\cos(\gamma^{\rT}x)$ or $x^{\alpha}\re^{\beta^{\rT} x}\sin(\gamma^{\rT}x)$, where $\beta, \gamma \in \mR^n$, and the multiindex notation $x^{\alpha}:= \prod_{j=1}^n x_j^{\alpha_j}$ is used for any $x:=(x_j)_{1\< j\< n}\in \mR^n$ and $\alpha:= (\alpha_j)_{1\< j\< n} \in \mZ_+^n$, with $\mZ_+:= \{0,1,2,\ldots\}$ denoting the set of nonnegative integers. Quasi-polynomials \cite{?}, obtained as  linear combinations of the quasi-monomials, form an algebra under the multiplication (which is also closed under the differentiation with respect to $x$).

%%%%%%%%%%%%%%%%%%%%%%%%%%%%%%%%%%%%%%%%%%%%%%%%%%%%%%%%%%%%%%%%%%%%%%%%%%%%%%%%
%\section{Smoluchowski equation on a torus}
%%%%%%%%%%%%%%%%%%%%%%%%%%%%%%%%%%%%%%%%%%%%%%%%%%%%%%%%%%%%%%%%%%%%%%%%%%%%%%%%
%
%Consider a version of the Smoluchowski SDE \cite{?} on an $n$-dimensional torus $\mT^n$ (with $\mT$ obtained by wrapping  up the half-open interval $[0,2\pi)$):
%\begin{equation}
%\label{SSDE}
%    \rd X(t) = -\nabla \varphi(X(t))\rd t + \sigma \rd W(t),
%\end{equation}
%Here, $X:= (X(t))_{t\> 0}$ is a $\mT^n$-valued Markov diffusion process, $\varphi: \mT^n \to \mR$ is a twice-continuously differentiable   function with the gradient $\nabla \varphi:=(\nabla_k f)_{1\< k \< n}: \mT^n \to \mR^n$. Also,  $\sigma > 0$ is a positive scalar parameter, and $W$ is an $n$-dimensional standard Wiener process. The SDE (\ref{SSDE}), which is understood in the Ito sense, describes a noise-corrupted gradient descent for the function $\varphi$.   The latter can be interpreted as a potential self-energy of a dissipative dynamical system in contact with a heat bath at absolute temperature $T>0$ which specifies the noise level $\sigma$ as
%$$
%    \sigma := \sqrt{2k_{\rB}T},
%$$
%with $k_{\rB}$ the Boltzmann constant. This relationship is motivated by the fact that the invariant probability measure for the Smoluchowski SDE (\ref{SSDE}) is absolutely continuous (with respect to the $n$-variate Lebesgue measure $\lambda_n$ on the torus $\mT^n$) with the Gibbs-Boltzmann probability density function (PDF)
%\begin{equation}
%\label{inv}
%    f_*(x) = \frac{\re^{-\beta \varphi(x)}}{Z(\beta)},
%    \qquad
%    x \in \mT^n.
%\end{equation}
%Here,
%\begin{equation}
%\label{Sbeta}
%    \beta := \frac{2}{\sigma^2} = \frac{1}{k_{\rB} T}
%\end{equation}
%is an auxiliary parameter, and the normalization constant
%$$
%    Z(\beta):= \int_{\mT^n} \re^{-\beta \varphi(x)}
%    \rd x
%$$
%is the statistical mechanical partition function \cite{?} associated with the potential $\varphi$. The invariant PDF $f_*$ is an  equilibrium point of the Fokker-Planck-Kolmogorov partial differential equation (PDE)
%\begin{equation}
%\label{SFPKE}
%    \d_t f= \cL^{\dagger}(f) := \div( f\nabla \varphi) + \frac{\sigma^2}{2}\Delta f
%    =
%    f\Delta \varphi
%    +
%    \nabla \varphi^{\rT}\nabla f   + \frac{\sigma^2}{2}\Delta f
%\end{equation}
%for the Smoluchowski SDE, with $\div(\cdot)$ and $\Delta(\cdot)$ the divergence and Laplace operators, respectively. The differential operator $\cL^{\dagger}$ is the adjoint   of the infinitesimal generator $\cL$ of the Markov diffusion $X$ given by
%$$
%    \cL(g) = -\nabla \varphi^{\rT} \nabla g + \frac{\sigma^2}{2}\Delta g.
%$$
%
%%%%%%%%%%%%%%%%%%%%%%%%%%%%%%%%%%%%%%%%%%%%%%%%%%%%%%%%%%%%%%%%%%%%%%%%%%%%%%%%
%\section{\bf The FPKE for the Smoluchowski SDE in the frequency domain}
%%%%%%%%%%%%%%%%%%%%%%%%%%%%%%%%%%%%%%%%%%%%%%%%%%%%%%%%%%%%%%%%%%%%%%%%%%%%%%%%
%
%All functions on the torus $\mT^n$  are periodic with respect to their (spatial) coordinates, and hence,  can be represented by Fourier series (under an additional  assumption that they are square integrable over the torus). More precisely, let
%\begin{align}
%    \varphi(x)
%    & = \sum_{\lambda \in \mZ^n} \varphi_{\lambda}\re^{i\lambda^{\rT}x},\\
%    f(t,x)
%    & = \sum_{\lambda \in \mZ^n} f_{\lambda}(t)\re^{i\lambda^{\rT}x}
%\end{align}
%denote the Fourier series for the potential $\varphi$ and the PDF $f$ at time $t\>0$, with $\mZ^n$ the $n$-dimensional integer lattice. Since the functions $\varphi$ and $f$ are real-valued, their Fourier coefficients satisfy
%$$
%    \varphi_{-\lambda} = \overline{\varphi_{\lambda}},
%    \qquad
%    f_{-\lambda} = \overline{f_{\lambda}},
%    \qquad
%    \lambda \in \mZ^n,
%$$
%where $\overline{(\cdot)}$ denotes the complex conjugate. Also, the normalization condition for $f(t,\cdot)$, as a PDF with respect to the Lebesgue measure over the torus, implies that
%$$
%    f_0(t) = (2\pi)^{-n}\int_{\mT^n}f(t,x)\rd x =(2\pi)^{-n},
%    \qquad
%    t\>0.
%$$
%From (\ref{SFPKE}), it follows that the other Fourier coefficients of the PDF $f$ satisfy a denumerable  system of ODEs
%$$
%    \dot{f}_{\lambda} = -\lambda^{\rT}\sum_{\mu \in \mZ^n\setminus \{0\}} f_{\lambda-\mu}\varphi_{\mu}\mu - \frac{\sigma^2}{2}|\lambda|^2 f_{\lambda},
%    \qquad
%    \lambda \in \mZ^n\setminus \{0\}.
%$$
%
%
%
%%%%%%%%%%%%%%%%%%%%%%%%%%%%%%%%%%%%%%%%%%%%%%%%%%%%%%%%%%%%%%%%%%%%%%%%%%%%%%%%
%\section{STOCHASTIC HAMILTONIAN SYSTEM}
%%%%%%%%%%%%%%%%%%%%%%%%%%%%%%%%%%%%%%%%%%%%%%%%%%%%%%%%%%%%%%%%%%%%%%%%%%%%%%%%
%
% Consider a stochastic Hamiltonian system with rotational degrees of freedom accommodated by position space $\mT^n$ which is the $n$-dimensional torus,  where $\mT$ is implemented as the interval $[0,2\pi)$. The position of the system on the torus $\mT^n$ is specified by a vector  of angular coordinates $q:= (q_k)_{1\< k\< n} \in \mT^n$, with the corresponding angular rates comprising the generalised velocity vector $\dot{q} = (\dot{q}_k)_{1\< k\< n} \in \mR^n$. The kinetic energy of the system is a position-dependent quadratic form of the vector of momenta
%\begin{equation}
%\label{pqdot}
%    p:=
%    \frac{1}{2}
%    \d_{\dot{q}} (\|\dot{q}\|_{M(q)}^2)
%    =
%    M(q)\dot{q}
%\end{equation}
%(see, for example, \cite{VJ_2014}). Here,
%$M(q)$ is a real positive definite symmetric matrix which, in the case of angular coordinates being considered,
%is interpreted as the local tensor of inertia, and $\|v\|_N:= |\sqrt{N}v| = \sqrt{v^{\rT}N v}$ denotes the Euclidean norm of a vector $v$ associated with a weighting matrix $N=N^{\rT}\succ 0$. The system Hamiltonian  $H: \mT^n \x \mR^n \to \mR$ on the phase space $\mT^n \x \mR^n$ is the sum of the kinetic energy and
%the potential energy $V: \mT^n \to \mR$:
%\begin{equation}
%\label{H}
%    H(q,p) := V(q) + \frac{1}{2} \|p\|_{M(q)^{-1}}^2,
%    \qquad
%    q \in \mT^n,\
%    p \in \mR^n.
%\end{equation}
%Both functions $V$ and $M$ are $2\pi$-periodic with respect to each of the angular coordinates $q_1, \ldots, q_n$ and are assumed to be twice continuously differentiable. The position $Q(t)$ and the momentum $P(t)$ of the system  at time $t\>0$  evolve  according to the equations
%\begin{align}
%\label{SH1}
%    \dot{Q}
%    &= M(Q)^{-1}P,\\
%\label{SH2}
%    \rd P
%    & =
%    -(\d_q H(X) + f(Q)P )\rd t
%    +
%    \sqrt{D(Q)}\rd W,
%\end{align}
%the first of which is an ODE, while the second one is an Ito SDE driven by an $n$-dimensional standard Wiener process $W$. Here, $f:\mT^n\to \mR^{n\x n}$ is a matrix-valued function which specifies the Langevin viscous damping force $-f(Q)P$ and satisfies the dissipativity condition
%\begin{equation}
%\label{visc}
%  S(q):= \bS(f(q)M(q))\succcurlyeq 0,
%  \qquad
%  q\in \mT^n,
%\end{equation}
%where $\bS(N):= \frac{1}{2}(N+N^{\rT})$ is the symmetrizer of matrices.
%Also, $D:\mT^n\to \mS_n^+$ describes the diffusion matrix of the SDE (\ref{SH2}). The $\mT^n\x \mR^n$-valued state process
%\begin{equation}
%\label{XQP}
%    X
%    :=
%    \begin{bmatrix}
%        Q\\
%        P
%    \end{bmatrix}
%\end{equation}
%satisfies the following SDE which is obtained by combining (\ref{SH1}) with (\ref{SH2}):
%\begin{equation}
%\label{dX}
%    \rd X =
%    \left(
%        J H'(X)
%        -
%        \begin{bmatrix}
%          0 \\
%          1
%        \end{bmatrix}
%        \ox
%        f(Q)P
%    \right)
%        \rd  t
%        +
%        \begin{bmatrix}
%          0 \\
%          1
%        \end{bmatrix}
%        \ox
%        \sqrt{D(Q)}\rd W.
%\end{equation}
%Here, $\ox$ is the Kronecker product of matrices,  and $J$ is the symplectic structure matrix
%\begin{equation}
%%\label{J}
%    J:=
%    \begin{bmatrix}
%        0 & 1 \\
%        -1 & 0 \\
%    \end{bmatrix}
%    \ox I_n.
%\end{equation}
%Also,
%$(\cdot)'$ denotes the gradient of a function with respect to all its variables, so that $H' = \begin{bmatrix} \d_q H \\ \d_p H \end{bmatrix}$ consists of the gradients of $H$ over the positions and momenta, where $\d_q H = V'(q) -\frac{1}{2}(p^{\rT}M^{-1} (\d_{q_k}M) M^{-1}p)_{1\< k\< n}$ and $\d_p H = M^{-1}p$ in view of (\ref{H}).  The centrifugal terms in $\d_qH$ come from the dependence of $M$ on $q$, and $-V'(q)$ describes the potential force field. Application of the Ito lemma \cite{?} to the Hamiltonian $H(X)$, evaluated at the process (\ref{XQP}), leads to the SDE
%\begin{align}
%\nonumber
%    \rd H
%    =&
%    H'^{\rT}\rd X
%    +
%    \frac{1}{2} \bra \d_p^2 H, D\ket\rd t\\
%\nonumber
%    =&
%    H'^{\rT}
%        \left(
%        J
%        -
%        \begin{bmatrix}
%          0 & 0 \\
%          0 & \bA(fM)
%        \end{bmatrix}
%        -
%        \begin{bmatrix}
%          0 & 0 \\
%          0 & S(Q)
%        \end{bmatrix}
%    \right)
%    H'
%        \rd  t\\
%\nonumber
%        &
%        +
%        \d_p H(X)^{\rT}
%        \sqrt{D}\rd W
%        +
%        \frac{1}{2} \bra \d_p^2 H, D\ket\rd t\\
%\label{dH}
%        =&
%        (\bra M^{-1}, D\ket-\|M^{-1} P\|_S^2)\rd t
%        +
%        \d_p H^{\rT}
%        \sqrt{D}\rd W,
%\end{align}
%where the functions $D$, $f$, $M$, $S$ are evaluated at the position process $Q$.
%Since the martingale part $\d_p H^{\rT}
%        \sqrt{D}\rd W$ does not contribute to the time derivative of the mean value of $H$, then
%\begin{align*}
%    (\bE H)^{^\centerdot}
%    & =
%    \bra M^{-1}, D\ket-\|M^{-1} P\|_S^2\\
%    &  =
%    \bE \Bra M^{-1},\,  D-\bE(PP^{\rT}\mid Q)M^{-1} S\Ket,
%\end{align*}
%where use is made of the tower property of conditional expectations \cite{?}, and $\bE(PP^{\rT}\mid Q)$ is the matrix of conditional second-order moments of $P$ given $Q$.
%
%
%The function $V$ can be interpreted as the potential energy of a dissipative dynamical system in contact with a heat bath at temperature $T>0$ which specifies the noise level $\sigma$ in (\ref{SSDE}) as
%$
%    \sigma = \sqrt{2k_{\rB}T}
%$,
%with $k_{\rB}$ the Boltzmann constant. This interpretation is motivated by the fact \cite{KS_1991} that the invariant probability measure for the Smoluchowski SDE (\ref{SSDE}) is absolutely continuous %(with respect to the $n$-variate Lebesgue measure on the torus $\mT^n$)
%with the Gibbs-Boltzmann PDF
%\begin{equation}
%\label{inv}
%    f_*(x) = \frac{\re^{-\beta V(x)}}{Z(\beta)},
%    \qquad
%    x \in \mT^n,\
%    \beta := \frac{2}{\sigma^2}.
%\end{equation}
%Here,
%$
%    Z(\beta):= \int_{\mT^n} \re^{-\beta V(x)}
%    \rd x
%$ is the statistical mechanical partition function \cite{ME_1981} of the auxiliary parameter $\beta = \frac{1}{k_{\rB} T}$,
% associated with the potential $V$.
%
%
%
%
%The positive parameters $\gamma$   and $\sigma$ in (\ref{SH}) quantify the levels of Langevin damping in the system and the noise from the environment which, as before, is modelled as a heat bath. The FPKE for the state PDF $f: \mR_+ \x \mT^n \x \mR^n \to \mR_+$ of the stochastic Hamiltonian system  in the position-momentum space $\mT^n \x \mR^n$ takes the form
%\begin{align}
%\nonumber
%    \d_t f
%    &=
%    -\div
%    \left(
%        f
%        \begin{bmatrix}
%            \nabla_p H\\
%            -\nabla_q H - \gamma \nabla_p H
%        \end{bmatrix}
%    \right)
%    +
%    \frac{\sigma^2}{2}
%    \Delta_p f\\
%\label{SHfdot}
%    & =
%    \{H,f\}
%    +
%    \gamma
%    \left(\div_p(f\nabla_p H)
%    +
%    \frac{\sigma^2}{2\gamma}
%    \Delta_p f\right),
%\end{align}
%where $\div_p(\cdot)$ and $\Delta_p(\cdot)$ denote the divergence and Laplace operators  with respect to the momentum variables, and
%$$
%    \{\varphi, \psi\} := \varphi'^{\rT} J \psi' = \d_q \varphi^{\rT}\d_p \psi - \d_p \varphi^{\rT}\d_q \psi
%$$
%is the Poisson bracket of differentiable functions $\varphi$ and $\psi$ on the position-momentum space $\mT^n\x \mR^n$ with the symplectic structure (\ref{J}). The corresponding invariant PDF is another Gibbs-Boltzmann density
%\begin{equation}
%\label{SHinv}
%    f_*(x) = \frac{\re^{-\beta H(x)}}{Z(\beta)},
%    \qquad
%    x \in \mT^n\x \mR^n,
%\end{equation}
%where
%$$
%    \beta := \frac{2\gamma}{\sigma^2} = \frac{1}{k_{\rB}T},
%$$
%and
%$$
%    Z(\beta):= \int_{\mT^n \x \mR^n}
%    \re^{-\beta H(x)}
%    \rd x.
%$$
%and the Langevin term in the drift of the SDE in (\ref{SH}) becomes proportional to the vector  $\dot{Q}$ of generalised velocities:
%$$
%    -\gamma \nabla_p H(X) = -\gamma M(Q)^{-1}P = -\gamma \dot{Q},
%$$
%thus modelling the viscous damping force.
%
%$$
%    f_*(x)
%    =
%    \frac{\re^{-\beta V(q)}}{U(\beta)}
%    \frac{(2\pi)^{-n/2} \re^{-\frac{\beta}{2}\|p\|_{M(q)^{-1}}^2}}{\sqrt{\det (M(q)/\beta) }}
%$$
%where
%$$
%    U(\beta):= \int_{\mT^n} \re^{-\beta V(q)}\rd q
%$$
%If
%Therefore, at equilibrium, the conditional probability distribution of the momentum $P$, given $Q$, is Gaussian with zero mean and covariance matrix $$
%    \cov(P \mid Q) = \frac{1}{\beta}M(Q)
%$$
%Now consider the evolution of the unconditional PDF $g: \mR_+ \x \mT^n \to \mR_+$ of the position $Q$ on the torus and the conditional PDF $h:  \mR^n \x \mT^n\x \mR_+\to \mR_+$ of the momentum $P$ (conditioned on $Q$), which  factorize their  joint PDF $f$ as
%$$
%    f(t,q,p) = g(t,q)h(p\mid q,t),
%    \qquad
%    t \> 0,\ q\in \mT^n,\
%    p \in \mR^n.
%$$
%
%
%
%
%
%$$
%    \d_t p = -\div(f p) + \div^2(D p)/2
%$$
%where
%$$
%    D(x)
%    :=
%    g(x)g(x)^{\rT}
%$$
%is the diffusion matrix
%
%Let $\xi$ be a stationary point of the PDF $p$, that is, the gradient of $p$ with respect to the spatial variables vanishes at $\xi$:
%$$
%    \d_x p(t,\xi) = 0
%$$
%If the Hessian $p'':= (\d_{x_j}\d_{x_k} p)_{1\< j,k\<n}$ is nonsingular at $\xi$, that is,
%$$
%    \det p''(t,\xi)\ne 0,
%$$
%then $\xi$ is locally smooth, with its time derivative  being computed as
%$$
%    \mathop{\xi}^{\centerdot} = -(p''(t,\xi))^{-1}\d_t \d_x p(t,\xi),
%$$
%where $\d_t \d_x p:= (\d_t\d_{x_k}p)_{1\< k\< n}$. The corresponding derivative of the  PDF value coincides with the partial time derivative:
%$$
%    p(t,\xi)^{\cdot} = \d_t p + \d_x p^{\rT}\mathop{\xi}^{\centerdot} = \d_t p.
%$$




%%%%%%%%%%%%%%%%%%%%%%%%%%%%%%%%%%%%%%%%%%%%%%%%%%%%%%%%%%%%%%%%%%%%%%%%%%%%%%%
\section{CONCLUSION}
\label{sec:conc}
%%%%%%%%%%%%%%%%%%%%%%%%%%%%%%%%%%%%%%%%%%%%%%%%%%%%%%%%%%%%%%%%%%%%%%%%%%%%%%%


We have considered a class of translation invariant networks of interacting linear systems at sites of a multidimensional lattice.
%The systems are governed by linear ODEs with constant coefficients driven by external inputs, and their internal dynamics and coupling with the other component systems are translation invariant. Such are, for example, finite-difference models of large-scale flexible structures manufactured from homogeneous materials.
Energy-based input-output properties of such a network (including passivity, positive realness and the negative imaginary property)  have been considered in terms of its transfer function representation in the spatio-temporal frequency domain. We have also discussed quadratic stability of the network in the isolated regime, %connections with multivariate sampling theorems  in the case of finite-range coupling,
and dispersion relations for phonons in the isolated network. The results of the paper are applicable to quadratic regulation problems for large flexible structures with a sparse collocation of sensors and actuators.
%Furthermore, we have also outlined a linear quadratic regulator problem for the network with a translation invariant feedback.

% The input-output properties of such a network is encoded by the spatio-temporal transfer function, we establish conditions for its positive realness in the sense of energy dissipation. The latter is formulated in terms of block Toeplitz bilinear forms of the input and output variables of the composite system.
%
%
%We have considered a class of port-Hamiltonian systems with a Galerkin-type representation of the Hamiltonian and coupling operators in the form of linear combinations of nonlinear basis functions satisfying an algebraic property. Sufficient conditions have been developed for the positive real property of such a system in the sense of energy dissipation. This result has been applied to systems which are organised as a feedback loop with a linear part and a static nonlinearity.







%%%%%%%%%%%%%%%%%%%%%%%%%%%%%%%%%%%%%%%%%%%%%%%%%%%%%%%%%%%%%%%%%%%%%%%%%%%%%%%
\begin{thebibliography}{99}
%==============================================================================

%\bibitem{BH_1998}
%D.S.Bernstein, and W.M.Haddad,
%LQG control with an
%$H^{\infty}$ performance bound: a Riccati equation approach,
%\textit{IEEE Trans.
%Automat. Contr.}, vol. 34, no. 3, 1989, pp. 293--305.

%%==============================================================================
%\bibitem{B_1986}
%W.M.Boothby,
%\textit{An Introduction to Differentiable Manifolds and Riemannian Geometry}, 2nd Ed.,
%Academic Press, London, 1986.
%%==============================================================================
%\bibitem{DP_2010}
%D.Dong, and I.R.Petersen, Quantum control theory and applications: a survey,
%\emph{IET Contr. Theor. Appl.}, vol. 4, no. 12, 2010, pp. 2651--2671.

%%==============================================================================

%\bibitem{EB_2005}
%S.C.Edwards, and V.P.Belavkin,
%Optimal quantum filtering and
%quantum feedback control,
%arXiv:quant-ph/0506018v2, August 1,  2005.
%%==============================================================================
%\bibitem{E_1998}
%L.C.Evans,
%\textit{Partial Differential Equations},
%American Mathematical Society, Providence, 1998.

%%==============================================================================

%\bibitem{GZ_2004}
%C.W.Gardiner, and P.Zoller,
%\textit{Quantum Noise}.
%Springer, Berlin, 2004.
%%==============================================================================
%\bibitem{G_2006}
%M.de~Gosson,
%\textit{Symplectic Geometry and Quantum Mechanics},
%Birk\-h\"{a}user, Basel, 2006.
%%==============================================================================
%\bibitem{G_1977}
%H.W.Guggenheimer,
%\textit{Differential Geometry},
%Dover, New York, 1977.
%%==============================================================================
\bibitem{H_2008}
N.J.Higham,
\emph{Functions of Matrices}, SIAM, 2008.
%%==============================================================================
\bibitem{HJ_2007}
R.A.Horn, and C.R.Johnson,
\textit{Matrix Analysis},
Cambridge
University Press, New York, 2007.
%%==============================================================================
%\bibitem{HM_1994}
%U.Helmke, and J.B.Moore,
%\textit{Optimization and Dynamical Systems},
%Springer, London, 1994.
%%==============================================================================
%\bibitem{JG_2010}
%M.R.James, and J.E.Gough,
%Quantum dissipative systems and feedback control design by interconnection,
% \emph{IEEE Trans. Autom. Contr.},  vol. 55, no. 8, pp. 1806--1821.

%%==============================================================================
%\bibitem{JNP_2008}
%M.R.James, H.I.Nurdin, and I.R.Petersen,
%$H^{\infty}$ control of
%linear quantum stochastic systems,
%\textit{IEEE Trans.
%Automat. Contr.}, vol. 53, no. 8, 2008, pp. 1787--1803.
%
%%==============================================================================
%\bibitem{J_1977}
%A.J.Jerri, The Shannon sampling theorem – its various extensions and applications:
%a tutorial review, \textit{Proc. IEEE}, vol. 65, no. 11, 1977, pp. 1565--1596.
%%==============================================================================
%%==============================================================================
%\bibitem{KS_1972}
%H.Kwakernaak, and R.Sivan,
%\textit{Linear Optimal Control Systems},
%Wiley, New York, 1972.
%%==============================================================================
\bibitem{LL_1980}
L.D.Landau, and E.M.Lifshitz, \textit{Statistical Physics},  3rd Ed.,  Butterworth-Heinemann, Oxford, 1980.
%%==============================================================================
%\bibitem{LBO_2010}
%G.Lois, J.Blawzdziewicz, C.S.O'Hern,
%Protein folding on rugged energy landscapes: Conformational diffusion on fractal networks,
%\emph{Phys. Rev. E}, vol. 81, 051907 (2010).
%%==============================================================================
%\bibitem{M_1988}
%J.R.Magnus,
%\textit{Linear Structures},
%Oxford University Press, New York, 1988.
%%==============================================================================
%\bibitem{M_1868}
%J.C.Maxwell, On governors,
%\textit{Proc. Roy. Soc. Lond.}, vol. 16,  1868,   pp. 270--283.
%
%%%==============================================================================
%\bibitem{M_1998}
%E.Merzbacher,
%\textit{Quantum Mechanics}, 3rd Ed.,
%Wiley, New York, 1998.
%%==============================================================================
%
%\bibitem{NJP_2009}
%H.I.Nurdin, M.R.James, and I.R.Petersen,
%Coherent quantum LQG
%control,
%\textit{Automatica}, vol.  45, 2009, pp. 1837--1846.
%
%%==============================================================================
\bibitem{OVMM_2001}
R.Ortega, A.J.van der Schaft, I.Mareels, and B.Maschke,  Putting energy back in control,
 \textit{IEEE Contr. Sys.},
vol. 21, no. 2, pp. 18--33, 2001.
%%==============================================================================
\bibitem{OVME_2002}
R.Ortega, A.van der Schaft, B.Maschke, and G.Escobar, Interconnection and damping assignment passivity-based control
of port-controlled Hamiltonian systems, \textit{Automatica}, vol. 38, no. 4, 2002, pp. 585--596.
%%==============================================================================
%\bibitem{P_1992}
%K.R.Parthasarathy,
%\textit{An Introduction to Quantum Stochastic Calculus},
%Birk\-h\"{a}user, Basel, 1992.
%%==============================================================================

%\bibitem{Par_2010}
%K.R.Parthasarathy,
%What is a Gaussian state?
%\textit{Comm. Stoch. Anal.}, vol. 4, no. 2, 2010, pp. 143--160.

%%==============================================================================
%\bibitem{P_2010}
%I.R.Petersen,
%Quantum linear systems theory,
%Proc. 19th Int. Symp. Math. Theor. Networks Syst., Budapest, Hungary, July 5--9, 2010, pp.  2173--2184.

%%==============================================================================
\bibitem{P_2014}
I.R.Petersen,
A direct coupling coherent quantum observer,
2014 IEEE Conference on Control Applications (CCA),
8--10 October 2014 (arXiv:1408.0399v2 [quant-ph], 8 August 2014).

%%==============================================================================
\bibitem{PL_2010}
I.R.Petersen, and A.Lanzon,
Feedback control of negative-imaginary systems,
\textit{IEEE Contr. Sys.}, vol. 30, no. 5, 2010, pp. 54--72.


%%==============================================================================
%\bibitem{PM_1962}
%D.P.Petersen, and D.Middleton,
%Sampling and reconstruction of wave-number-limited functions in $N$-dimensional Euclidean
%spaces, \textit{Inform. Contr.}, vol. 5, 1962, pp. 279--323.

%%%==============================================================================
%\bibitem{PBGM_1962}
%L.S.Pontryagin, V.G.Boltyanskii,
%R.V.Gamkrelidze, and E.F. Mishchenko,
%\emph{The Mathematical Theory of Optimal Processes},
%Wiley, New York, 1962.
%==============================================================================
\bibitem{R_2007}
J.N.Reddy, \textit{Theory and Analysis of Elastic Plates and Shells}, 2nd Ed., CRC Press, Taylor and Francis, London, 2007.
%==============================================================================
%\bibitem{SIS_2004}
%A.Serafini, F.Illuminati, and S.De Siena,
%Symplectic invariants, entropic measures and correlation of Gaussian states,
%\textit{J. Phys. B: At. Mol. Opt. Phys.}, vol. 37, 2004, pp. L21--L28.
%==============================================================================
%
%\bibitem{SP_2009}
%A.J.Shaiju, and I.R.Petersen,
%On the physical realizability of
%general linear quantum stochastic differential equations with
%complex coefficients,
%Proc. Joint 48th IEEE Conf. Decision Control \&
%28th Chinese Control Conf.,
%Shanghai, P.R. China, December 16--18, 2009, pp. 1422--1427.


\bibitem{SVP_2015}
A.Kh.Sichani, I.G.Vladimirov, and I.R.Petersen,
Covariance dynamics and entanglement in translation invariant linear quantum stochastic networks
IEEE 54th Annual Conference on Decision and Control (CDC),
December 15-18, 2015,  Osaka, Japan,
pp. 7107--7112.

%%==============================================================================
%
%\bibitem{S_2000}
%R.Simon,
%Peres-Horodecki separability criterion for continuous variable systems,
%\textit{Phys. Rev. Lett.},
%vol. 84, no. 12, 2000, pp. 2726--2729.

%
%%==============================================================================
%%
%\bibitem{SIG_1998}
%R.E.Skelton, T.Iwasaki, and K.M.Grigoriadis,
%\textit{A Unified Algebraic Approach to Linear Control Design},
%Taylor \& Francis, London, 1998.
%%==============================================================================
\bibitem{S_1990}
G.P.Srivastava, \emph{The Physics of Phonons}, Taylor \& Francis, 1990.
%%==============================================================================
%\bibitem{S_1979}
%H.Stark, Sampling theorems in polar coordinates, \textit{J. Opt. Soc. Am.}, vol. 69, no.
%11, 1979, pp. 1519--1525.
%%==============================================================================
%\bibitem{SW_1997}
%H.J.Sussmann, and J.C.Willems,
%300 years of optimal control: from the brachystochrone to the maximum principle,
%\textit{Control Systems}, vol. 17, no. 3, 1997, pp. 32--44.
%%==============================================================================
\bibitem{VJ_2014}
A. van der Schaft, and D. Jeltsema,  Port-Hamiltonian Systems Theory: An Introductory
Overview, \emph{ Foundations and Trends
R in Systems and Control}, vol. 1, no. 2--3, 2014,  pp. 173--378.
%==============================================================================
\bibitem{VBSH_2006}
V.Veselago, L.Braginsky, V.Shklover, and C.Hafner,
Negative refractive index metamaterials,
\emph{J. Comp. Theor. Nanosci.}, vol. 3, 2006, pp. 1--30.

%
%==============================================================================
%\bibitem{V_1971}
%V.S.Vladimirov,
%\textit{Equations of Mathematical Physics},
%M.Dekker,
%New York, 1971.
%%==============================================================================
%\bibitem{VP_2010a}
%I.G.Vladimirov, and I.R.Petersen,
%Minimum relative entropy state transitions in linear stochastic
%systems: the continuous time case,
%Proc. 19th Int. Symp. Math. Theor. Networks Syst., Budapest, Hungary, July 5--9,  2010, pp.  51--58.
%%==============================================================================
%\bibitem{VP_2010b}
%I.G.Vladimirov, and I.R.Petersen,
%Hardy-Schatten norms of systems, output energy cumulants and linear quadro-quartic  Gaussian control,
%Proc. 19th Int. Symp. Math. Theor. Networks Syst., Budapest, Hungary, July 5--9,  2010, pp.  2383--2390.
%%==============================================================================
%\bibitem{VP_2011a}
%I.G.Vladimirov, and I.R.Petersen,
%A quasi-separation principle and Newton-like scheme for coherent quantum LQG control, 	
%18th IFAC World Congress, Milan, Italy, 28 August--2 September, 2011, pp. 4721--4727 (preprint:  arXiv:1010.3125v2 [quant-ph], 15 April, 2011).
%%==============================================================================
%\bibitem{VP_2011b}
%I.G.Vladimirov, and I.R.Petersen,
%A dynamic programming approach to finite-horizon coherent quantum LQG control, 	
%Australian Control Conference, Melbourne, 10--11 November, 2011 (preprint:  		arXiv:1105.1574v2 [quant-ph], 3 August, 2011).

%%%==============================================================================
\bibitem{VP_2014}
I.G.Vladimirov, and I.R.Petersen,
Physical realizability and mean square performance of translation
invariant networks of interacting linear quantum stochastic systems,
21st
Int. Symp. Math. Theor. Networks Syst.,
Groningen, The Netherlands,
July 7--11, 2014,
pp.  1881--1888.
%%==============================================================================

%%==============================================================================
\bibitem{W_1972}
J.C.Willems, Dissipative dynamical systems. Part I: general
theory, Part II: linear
systems with quadratic supply rates, \emph{Arch. Rational Mech. Anal.}, vol. 45, no. 5, 1972, pp.
321--351, 352--393.
%%==============================================================================

%\bibitem{W_1936}
%J.Williamson,
%On the algebraic problem concerning the normal forms of linear dynamical systems,
%\textit{Am. J. Math.}, vol. 58, no. 1, 1936, pp. 141--163.
%%==============================================================================
%\bibitem{W_1937}
%J.Williamson,
%On the normal forms of linear canonical transformations in dynamics,
%\textit{Am. J. Math.}, vol. 59, no. 3, 1937, pp. 599--617.
%%==============================================================================
\bibitem{XPL_2010}
J.Xiong, I.R.Petersen, and A.Lanzon,
A negative imaginary lemma and the stability of
interconnections of linear negative imaginary systems,
\textit{IEEE Trans. Automat. Contr.},
vol. 55, no. 10, 2010, pp. 2342--2347.
%%==============================================================================
%\bibitem{ZJ_2011b}
%G.Zhang, and M.R.James,
%Direct and indirect couplings in coherent feedback control of linear quantum systems,
%\textit{IEEE Trans. Automat. Contr.},
%vol. 56, no. 7, 2011, pp. 1535--1550.

\end{thebibliography}

%\appendix
%
%%%%%%%%%%%%%%%%%%%%%%%%%%%%%%%%%%%%%%%%%%%%%%%%%%%%%%%%%%%%%%%%%%%%%%%%%%%%%%%%
%\subsection{Block Toeplitz matrices and Fourier transforms}\label{sec:oper}
%%%%%%%%%%%%%%%%%%%%%%%%%%%%%%%%%%%%%%%%%%%%%%%%%%%%%%%%%%%%%%%%%%%%%%%%%%%%%%%%
%\renewcommand{\theequation}{A\arabic{equation}}
%\setcounter{equation}{0}
%%%%%%%%%%%%%%%%%%%%%%%%%%%%%%%%%%%%%%%%%%%%%%%%%%%%%%%%%%%%%%%%%%%%%%%%%%%%%%%%%
%
%
%Let $\alpha_{\ell}$ and $\beta_{\ell}$ be real or complex matrices of compatible dimensions $p\x q$ and $q\x r$, respectively, depending on a spatial index  $\ell \in \mZ^\nu$. Such matrix-valued functions specify infinite-dimensional matrices $\alpha :=(\alpha_{j-k})_{j,k\in \mZ^\nu}$ and $\beta :=(\beta_{j-k})_{j,k\in \mZ^\nu}$  which are block Toeplitz in the sense of the additive group structure of the $\nu$-dimensional integer lattice.
%The product
%$$
%    \gamma:= (\gamma_{j-k})_{j,k\in \mZ^\nu} := \alpha \beta
%$$
%of such matrices is also block Toeplitz, and its blocks are given by the convolution
%\begin{equation}
%\label{conv}
%    \gamma_j = \sum_{k\in \mZ^\nu}\alpha_k \beta_{j-k},
%    \qquad
%    j\in \mZ^\nu.
%\end{equation}
%These series are summable, for example, if both $\alpha_\ell$ and $\beta_\ell$ are square summable:
%$$
%    \sum_{k \in \mZ^\nu}
%    (\|\alpha_k\|^2 + \|\beta_k\|^2)<+\infty,
%$$
%where $\|\cdot\|$ is a matrix norm (whose particular choice is irrelevant here). This square summability makes the Fourier series
%\begin{align}
%\label{Aseries}
%    A(\sigma)
%    & :=
%    \sum_{k\in \mZ^\nu}\alpha_k \re^{-ik^{\rT}\sigma},\\
%\label{Bseries}
%    B(\sigma)
%    & :=
%    \sum_{k\in \mZ^\nu}\beta_k \re^{-ik^{\rT}\sigma}
%\end{align}
%well-defined for almost all $\sigma \in \mT^\nu$ on the $\nu$-dimensional torus as the $L^2$-limits of appropriate partial sums over finite subsets of the lattice $\mZ^\nu$ forming an exhausting sequence. The block Toeplitz matrices $\alpha$ and $\beta$ describe bounded linear operators on the corresponding Hilbert spaces
%\begin{align*}
%    L^2(\mZ^\nu,\mC^q)
%    := &
%    \Big\{
%        \xi := (\xi_k)_{k\in \mZ^\nu}:\\
%    & \|\xi\|_2:= \sqrt{\sum_{k \in \mZ^\nu} |\xi_k|^2}<+\infty
%    \Big\},\\
%    L^2(\mZ^\nu,\mC^r)
%    := &
%    \Big\{
%        \eta := (\eta_k)_{k\in \mZ^\nu}:\ \\
%    & \|\eta\|_2:= \sqrt{\sum_{k \in \mZ^\nu} |\eta_k|^2}<+\infty
%    \Big\}
%\end{align*}
%of square summable vector-valued functions on $\mZ^\nu$ with values in $\mC^q$ and $\mC^r$, respectively,  if their $L^2$-induced operator norms are finite:
%\begin{align*}
%    \|\alpha\|_{\infty}
%    & :=
%    \sup_{\xi \in L^2(\mZ^\nu,\mC^q)}
%    \frac{\|\alpha \xi\|_2}{\|\xi\|_2}\\
%    & =
%    \esssup_{\sigma \in \mT^\nu}
%    \|A(\sigma)\|
%    <+\infty,\\
%    \|\beta\|_{\infty}
%    & :=
%    \sup_{\eta \in L^2(\mZ^\nu,\mC^r)\setminus \{0\}}
%    \frac{\|\beta \eta\|_2}{\|\eta\|_2}\\
%    & =
%    \esssup_{\sigma \in \mT^\nu}
%    \|B(\sigma)\|
%    <+\infty.
%\end{align*}
%Due to submultiplicativity  of the operator norms, $\|\gamma\|_{\infty} \< \|\alpha\|_{\infty} \|\beta\|_{\infty}$,  which is in accordance with the fact that the Fourier series for the convolution in (\ref{conv}) is the pointwise product
%$$
%    \sum_{k \in \mZ^\nu}\gamma_k \re^{-ik^{\rT}\sigma} = A(\sigma)B(\sigma)
%$$
%of the Fourier transforms (\ref{Aseries}) and (\ref{Bseries}). Also, the Parseval identity

\end{document} 