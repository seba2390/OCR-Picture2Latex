\documentclass{article}
\usepackage{ifthen}
\usepackage{mdwlist}
\usepackage{amsmath,amssymb,amsfonts,amsthm}
\usepackage{bm}
\usepackage{hyperref}
\usepackage{enumitem}
\usepackage{graphicx}
\usepackage{xspace}
\usepackage{verbatim}
\usepackage{algorithm}
\usepackage{algpseudocode}
\usepackage[margin=1in]{geometry}
\usepackage{color}
\usepackage{thm-restate}
\usepackage{latexsym}
\usepackage{epsfig}
\usepackage{pgf}
\usepackage{tikz}
\usetikzlibrary{tikzmark}
\usepackage{xcolor}
\usepackage{colortbl}
% OLD PREAMBLE:

% \usepackage{jsen}
% \usepackage{cite}
% \usepackage{amsmath,amssymb,amsfonts, bbm, mathtools}
% \usepackage{algorithm,algorithmic}
% \usepackage{graphicx}
% \usepackage{textcomp}
% \usepackage{wrapfig}
% \usepackage{xfrac}
% \usepackage{stackengine}
% \usepackage{subfigure}
% \def\delequal{\mathrel{\ensurestackMath{\stackon[1pt]{=}{\scriptstyle\Delta}}}}



% \usepackage{color, soul}
% \newcommand{\hlt}[1]{\hl{#1}}
% \newcommand{\red}[1]{\textcolor{red}{#1}}

% \def\BibTeX{{\rm B\kern-.05em{\sc i\kern-.025em b}\kern-.08em
%     T\kern-.1667em\lower.7ex\hbox{E}\kern-.125emX}}
% \markboth{\journalname, VOL. XX, NO. XX, XXXX 2017}
% {Author \MakeLowercase{\textit{et al.}}: Preparation of Papers for IEEE TRANSACTIONS and JOURNALS (February 2017)}
% \definecolor{abstractbg}{rgb}{0.89804,0.94510,0.83137}
% \setlength{\fboxrule}{0pt}
% \setlength{\fboxsep}{0pt}

% NEW PREAMBLE:


\usepackage{amsmath,amsfonts,amssymb,bbm, amsthm, xfrac}
\usepackage{algorithmic}
\usepackage{algorithm}
\usepackage{array, multirow}
% \usepackage[caption=false,font=normalsize,labelfont=sf,textfont=sf]{subfig}
\usepackage{caption, subcaption}
\usepackage{textcomp}
\usepackage{stfloats}
\usepackage{url}
\usepackage{verbatim}
\usepackage{graphicx}
\usepackage{cite}
\usepackage{caption}
\usepackage{subcaption}
\hyphenation{}

\theoremstyle{plain}
\newtheorem{theorem}{Theorem}

\usepackage{color, soul}
\newcommand{\hlt}[1]{\hl{#1}}
\newcommand{\red}[1]{\textcolor{red}{#1}}


\newcommand{\cnote}[1]{\red{#1}}

\title{Which Distribution Distances are Sublinearly Testable?}

\author {
Constantinos Daskalakis\thanks{Supported by NSF CCF-1617730, CCF-1650733, and ONR N00014-12-1-0999.}\\
EECS \& CSAIL, MIT\\
\tt{costis@csail.mit.edu}
\and
Gautam Kamath\thanks{Supported by NSF CCF-1617730, CCF-1650733, and ONR N00014-12-1-0999. Part of this work was done while the author was an intern at Microsoft Research New England.} \\
EECS \& CSAIL, MIT\\
\tt{g@csail.mit.edu}
\and
John Wright\thanks{Supported by NSF grant CCF-6931885.} \\
Physics, MIT\\
\tt{jswright@mit.edu}
}
\begin{document}
\maketitle
\begin{abstract}
%!TEX root = ms.tex
\begin{abstract}
%Recently, the problem of detecting and categorizing semantic relationship mentions from a given context has received a significant amount of attention.
% Extracting entity relationships for types of interests from text is important for understanding massive text corpora.
Relation extraction is a fundamental task in information extraction.
% Most existing systems for relation extraction heavily rely on manual labeling by human experts to create training data---a process that is costly, non-scalable, and hardly portable across different corpora.
% Most existing principles heavily rely on annotations given by human experts, which is costly and time-consuming.
Most existing methods have heavy reliance on annotations labeled by human experts, which are costly and time-consuming.
%To break the bottleneck of labeled data, knowledge bases like Freebase have been utilized to provide \ds. However, for many domain specific corpora, \ds is either non-existent or insufficient, while other related information like domain specific patterns is available and could be used. In this paper, we combined different types of information to provide \hs and perform relation extraction.
% To break this bottleneck, knowledge bases have been utilized to generate noisy annotations and provide \ds, while for many domain specific corpora, it's either non-existent or insufficient. 
% Therefore, we proposed a novel framework to supervise relation extraction model with knowledge bases and more,   combined different types of information to provide \hs and perform relation extraction.
% To reduce human labeling effort, two kinds of methods are studied by prior work: 
% (1) taking a small set of human-crafted patterns (instead of fully-annotated sentences) as ``\textit{weak}" supervision; and (2) leveraging freely available relation information from external knowledge bases as ``\textit{distant}" supervision. 
% However, both methodologies encounter challenges in dealing with domain-specific corpora. 
% On one hand, 
% Weak supervision, guided by domain knowledge, can generate high-quality but \textit{limited} amount of labeled data, due to its dependence on substantial human effort, 
% On the other hand, 
% whereas distant supervision can automatically produce a large amount of labeled data but the labels so generated may not be ``\textit{perfect}" for individual mentions.
% To overcome this drawback, we conduct relation extractor learning under annotations generated by heterogeneous information (e.g., knowledge base and domain heuristics), which are noisy but require l, and is referred as \hs. 
To overcome this drawback, we propose a novel framework, \our, to conduct relation extractor learning using annotations from heterogeneous information source, e.g., knowledge base and domain heuristics.
% In this paper, we study how to leverage \textit{heterogeneous supervision} (\ie, combining weak supervision and distant supervision) to perform relation extraction in an \textit{effecdtive} way. 
These annotations, referred as \hs, often conflict with each other, which brings a new challenge
 to the original relation extraction task: how to infer the true label from noisy labels for a given instance.
% And the challenges here are relation extraction task itself and more, resolving the conflicts among \hs.
% A key challenge is how to integrate the two kinds of sources while eliminating the conflicts among them in a trustworthy manner.
% In this paper, We propose a novel framework, \our, to jointly conducts relation extractor learning and context-aware truth discovery.
% Specifically, to resolve conflicts among \hs, true label discovery is conducted in a context-aware manner, while context information, or text feature, also serves as the backbone of relation extraction.
Identifying context information as the backbone of both relation extraction and true label discovery, we adopt embedding techniques to learn the distributed representations of context, which bridges all components with mutual enhancement in an iterative fashion.
% and allows them to enhance each other. 
%Specifically, we adopted an embedding method to learn distributed representations of text features, relation types and supervisions, which bridges the context-aware truth discovery module and the relation extraction module.
Extensive experimental results demonstrate the superiority of \our over the state-of-the-art.
\end{abstract} 
\end{abstract}

Reinforcement learning has achieved great success in areas such as Game-playing \citep{silver2018general,vinyals2019grandmaster}, robotics \cite{kober2013reinforcement}, large language models \citep{ouyang2022training}, etc.
However, due to safety concerns or physical limitations, in some real-world reinforcement learning problems, we must consider additional constraints that may influence the optimal policy and the learning process \citep{garcia2015comprehensive}.
% For example, a robotic arm must not take actions that may cause harm to itself or the environments.
A standard framework to handle such cases is the constrained Markov Decision Process (CMDP) \citep{altman1999constrained}.
Within the CMDP framework, the agent has to maximize
the expected cumulative reward while
obeying a finite number of constraints, which are usually in the form of expected cumulative cost criteria.

However, we are sometimes concerned with the problem with a continuum of constraints.
For example,
the constraints we meet might be time-evolving or subject to uncertain parameters, which
cannot be formulated as an ordinary CMDP
(see Examples \ref{Example_Time_Evolving} and  \ref{Example_Uncertain}).
In this paper we would study a generalized CMDP  
to address the above problem.  Because the constraints are not only infinite-number but also lie
in a continuous set,
the generalization is not trivial. Fortunately, we find that we can borrow the idea behind semi-infinite programming (SIP) \citep{remez1934determination, hettich1993semi} to deal with the semi-infinite constraints.
Accordingly, we propose \emph{semi-infinitely constrained Markov decision processes} (SICMDPs)
as a novel complement to the ordinary CMDP framework.
%More specifically,  an SICMDP model %, we consider 
%contains a continuum of constraints whereas an ordinary CMDP contains a finite number of constraints. 

%This generalization is natural but not trivial. However, we can brows the idea  
%The idea is quite natural and can be backtracked
%to the practice of extending linear programming to linear semi-infinite programming (LSIP) %\cite{remez1934determination, GobernaLSIO1998}.
%In addition, 
%As a complementary approach to the ordinary CMDP framework, 
%SICMDP can be used to model these problems  which cannot be described by a finite number of constraints
%that are not covered by .
%For example,
%the restrictions we consider can be time-evolving or subject to uncertain parameters
%, thus
%cannot be described by a finite number of constraints but a continuum of constraints 
%(see Examples \ref{Example_Time_Evolving} and  \ref{Example_Uncertain}).

We also present two reinforcement learning algorithms to solve SICMDPs called SI-CRL and SI-CPO, respectively.
SI-CRL is a model-based reinforcement learning algorithm designed for tabular cases, and SI-CPO is a policy optimization algorithm for non-tabular cases.
% and analyze its performance both theoretically and empirically.
The main challenge is that we need to deal with a continuum of constraints, thus reinforcement learning algorithms for ordinary CMDPs do not work anymore.
In SI-CRL, we tackle this difficulty by first transforming the reinforcement learning problem to an equivalent LSIP problem, which can then be solved using methods in the LSIP literature like the dual exchange methods \citep{Hu1990,reemtsen1998numerical}.
In SI-CPO, we resort to the idea of cooperative stochastic approximation developed in \cite{lan2020algorithms, wei2020comirror}.
As far as we know, we are the first to introduce tools from semi-infinitely programming (SIP) into the reinforcement learning community for solving constrained reinforcement learning problems.

% To the best of our knowledge, we are the first to apply tools from semi-infinitely programming (SIP) to solve reinforcement learning problems.
Furthermore, we give theoretical analysis for both SI-CRL and SI-CPO.
We decompose the error of SI-CRL into two parts: the statistical error from approximating the true SICMDP with an offline dataset and the optimization error due to the fact that the solution of the LSIP problem obtained by the dual exchange method is inexact.
On the optimization side, we show that the iteration complexity of SI-CRL is $O\left(\left\{\mathrm{diam}(Y)L\sqrt{|\gS|^2|\gA|m}/\left[(1-\gamma)\epsilon\right]\right\}^m\right)$.
On the statistical side, we show that the sample complexity of SI-CRL is $\widetilde O\left(\frac{|S|^2|A|^2}{\epsilon^2(1-\gamma)^3}\right)$ if the offline dataset is generated by a generative model, and $\widetilde O\left(\frac{|S||A|}{\nu_{\min} \epsilon^2(1-\gamma)^3}\right)$ if the dataset is generated by a probability measure $\nu$ as considered in \cite{chen2019information}.
Here $\widetilde O$ means that all logarithm terms are discarded.
For SI-CPO, things become a little more complicated because other than the statistical error and the optimization error, we also need to consider the function approximation error, which comes from imperfect policy parametrizations.
It is shown if the function approximation error can be controlled to $O(\epsilon)$ order, the iteration complexity of SI-CPO is $\widetilde{O}\left(\frac{1}{\epsilon^2(1-\gamma)^6}\right)$ and the sample complexity of SI-CPO is $\widetilde{O}(\frac{1}{\epsilon^4(1-\gamma)^{10}})$.
Here our iteration complexity bound is equivalent to a typical $\widetilde O(1/\sqrt{T})$ global convergence rate.

We perform a set of numerical experiments to illustrate the SICMDP model and validate our proposed algorithms.
Specifically, we examine two numerical examples, namely the discharge of sewage and ship route planning.
Through the discharge of sewage example, we show the advantage of the SICMDP framework over the CMDP baseline obtained by naive discretization in modeling realistic sequential decision-making problems.
Moreover, we demonstrate the effectiveness of the SI-CRL and SI-CPO algorithms in such tabular environments. 
In the ship route planning example, we illustrate the benefits of the SICMDP framework and the ability of the SI-CPO algorithm to address complex continuous control tasks involving continuous state spaces with modern deep reinforcement learning techniques.

% In summary, our contributions are listed as follows.
% First, we present the SICMDP model, which can be viewed as a generalization of the ordinary CMDP model.
% Second, we propose an algorithm to perform reinforcement learning for SICMDPs, which is called SI-CRL, and we believe that we are the first to apply tools from SIP
% to solve reinforcement learning problems.
% Third, we give a theoretical analysis of SI-CRL and identify both its sample complexity and iteration complexity.
% In addition, we perform numerical experiments to illustrate the SICMDP model and validate the SI-CRL algorithm.
% \{This paragraph can be removed!!! \}





\section{Preliminaries}\label{chpt:preliminiaries}
In this chapter we will introduce some of the mathematical background and notation needed for this thesis. In particular, we will shortly introduce the differential geometric description of spacetime in Section \ref{sec:spacetime_geometry} and give an introduction to the notion of global hyperbolicity and its connection to Green- and normally-hyperbolic operators in Section \ref{sec:global_hyperbolicity}. In a bit more detail, we will introduce the notion of differential forms and give explicit definitions, also in terms of an index based notation, in Section \ref{sec:differential_forms}. For completeness, in Section \ref{sec:cat-theory}, we present basic definitions of category theory. The reader familiar with these topics can safely skip this chapter and refer to it when interested in the chosen conventions.
%
%
%
%
%%%%%%
%%SPACTIME GEOMETRY
%%%%%
%
%
%
\subsection{Spacetime geometry}\label{sec:spacetime_geometry}
In GR, the universe is mathematically described as a four dimensional \emph{spacetime}, consisting of a smooth, four dimensional manifold \gls{M} (assumed to be Hausdorff, connected, oriented, time-oriented and para-compact) and a Lorentzian metric $g$. We will assume the signature of the Lorentzian metric $g$ to be $(-,+,+,+)$. The Levi-Civita connection on $(\M,g)$ is as usual denoted by \gls{nabla}.
Throughout this thesis, we treat spacetime as fixed, implementing a gravitational background determined classically by Einstein's field equations. Hence, we neglect any back-reaction of the fields on the metric, both in the quantum and the classical case. In that sense, we treat the fields as \emph{test fields}.\par
For the basic mathematical theory regarding Lorentzian manifolds, we refer to the literature: An introduction to the topic with an emphasis on the physical application in GR is for example given in \cite{wald_GR} and \cite{carroll_spacetime-and-gr}.
Here, we will shortly recap the notion of a tangent space and tangent bundle and generalize to the notion of a vector bundle which we will use in the general description of normally hyperbolic operators and differential forms.
In the following, we generalize the setting to an arbitrary smooth manifold $\N$ of dimension $N$ with either Lorentzian or Riemannian metric $k$.\par
%
%
A \emph{tangent vector} $v_x$ at point $x \in \N$ is a linear map $v_x : C^\infty(\N , \IR) \to \IR$ that obeys the Leibniz rule, that is, for $f,g \in C^\infty (\N,\IR)$ it holds $v_x(fg) = f(x)v_x(g) + v_x(f)g(x)$.
We define the \emph{tangent space} \gls{TxN} of $\N$ at $x$ as the real $N$-dimensional vector space of all tangent vectors at point $x$.
The disjoint union of all tangent spaces is called the \emph{tangent bundle} \gls{TN} of $\N$ and is itself a manifold of dimension $2N$. A \emph{vector field} is a map $v: \N \to T\N$ such that $v(x) \in T_x\N$.
The respective dual spaces, that is the space of all linear functionals, the \emph{co-tangent space} and the \emph{co-tangent bundle}, are denoted by \gls{TsxN} and \gls{TsN} respectively.\par
%
For Lorentzian manifolds, we call a tangent vector $v$ at $x \in \N$ \emph{timelike} if $k_{\mu \nu} v^\mu v^\nu < 0$, \emph{spacelike} if $k_{\mu \nu} v^\mu v^\nu > 0$ and \emph{null} (or lightlike) if $k_{\mu \nu} v^\mu v^\nu = 0$. At every point $x \in \N$, we define the set of all \emph{causal}, that is, either timelike or null, tangent vectors in the tangent space at $x$. This set is called the \emph{light cone} at $x$ and it is split up into two distinct parts, one that we call the future light cone, and one that we call the past light cone at $x$. Since we assume the manifold to be time orientable, there exists a smooth vector field $t$ that is timelike at every $x \in \N$. Given this time orientation, we identify the future (past) light cone with the set of tangent vectors $v \in T_x\N$ such that $k_{\mu\nu} v^\mu t^\nu < 0$ (respectively $> 0$). Therefore, a tangent vector $v$ at $x$ is called \emph{future directed} (past directed) if it lies in the future (past) light cone at $x$.\\
Accordingly, a curve $\gamma : I \to \N$ is called timelike (spacelike, null, causal, future or past directed) if its tangent vector $\dot{\gamma}$ is timelike (spacelike, null, causal, future or past directed) at every $x \in \N$.  For every point $x \in \N$ we define the \emph{causal future/past} \gls{causalfuturepast} of $x$ as the set of all points $q \in \N$ that can be reached by a future directed causal curve originating in $x$. For any subset $S \in \N$ we define $J^\pm (S) = \bigcup_{x \in S} J^\pm(x)$ and $J(S) = J^+(S) \cup J^- (S)$. Finally, the future/past domain of dependence $\gls{futurepastdomainofdependence}$ of a set $S \subset \N$ is the set of all points $x \in \N$ such that every inextendible causal curve through $x$ intersects $S$. The \emph{domain of dependence} \gls{domainofdependence} of $S$ is the union of the future and past domain of dependence of the set $S$.
For more details on the causal structure of spacetime we refer to for example \cite[Chapter 8]{wald_GR}.\par
%
%
%
The notion of tangent bundles can be generalized to the notion of a vector bundle. Instead of ``attaching'' the vector spaces $T_x \N$ to every point $x$ of the manifold, we allow for the occurrence of arbitrary vector spaces, called the fibres of the vector bundle. A vector bundle then consists of the base manifold, in our case $\N$, the total space and a map $\pi$ from the total space to the base manifold, that can be locally trivialized. At each point of the base manifold, the pre-image of $\pi$ is the fibre of the vector bundle. To be precise we define, following \cite{rudolph_schmidt}:
\begin{definition}[Vector bundle]
	A smooth \emph{vector bundle} over $\N$ is a tuple $\gls{vectorbundle} = (E,\N, \pi)$, where $E$ is a smooth manifold and $\pi : E \to \N$ is a smooth surjective map satisfying:
	\begin{enumerate}
		\item For every $x \in \N$, $\pi^{-1}(x)$ is a vector space, called the fibre of the bundle at point $x$.
		\item There exists a finite dimensional vector space $F$, an open covering $\left\{ U_\alpha\right\}_\alpha$ of $\N$ and a family of diffeomorphisms $\chi_\alpha : \pi^{-1}(U_\alpha) \to U_\alpha \times F$ such that for all $\alpha$ it holds $\chi_\alpha \comp \text{pr}_1 =  \restr{\pi}{\pi^{-1}(U_\alpha)}$ and for every $x \in \N$ the map $\text{pr}_2 \comp \restr{\chi_\alpha}{\pi^{-1}(x)} : \pi^{-1}(x) \to F$ is linear.
	\end{enumerate}
\end{definition}
Here, the maps $\text{pr}_1$ and $\text{pr}_2$ denote the projection onto the first respectively second component of an element in $U_\alpha \times F$. The properties graphically mean that \emph{locally}, the vector bundle ``looks like" the product of the base manifold with the fibre. The tuples $(U_\alpha, \chi_\alpha)$ are called \emph{local trivializations} of the vector bundle. Like for vector spaces, we can define the sum and product of vector bundles, by using the according vector space definitions on the fibres of the bundle.\par
Let $\mathfrak{X}, \mathfrak{Y}$ be vector bundles over $\N$ with fibres $X_x$ and $Y_x$ at $x \in \N$. We denote by \gls{whitneysum} the \emph{Whitney sum} of the two vector bundles - the vector bundle over $\N$ whose fibres are given by the direct sum $X_x \oplus Y_x$. Similarly, one obtains the local trivializations of the Whitney sum from the trivializations of $\mathfrak{X}, \mathfrak{Y}$ and direct sums.\par
Accordingly, let $\mathfrak{X}, \mathfrak{Y}$ be vector bundles over $\N$ and $\widetilde{\N}$, with fibres $X_x$ and $Y_{\tilde{x}}$ at $x \in \N$, $\tilde{x} \in \widetilde{\N}$ respectively. We denote by \gls{outerproductbundle} the \emph{outer product} of the two vector bundles - the vector bundle over $\N \times \widetilde{\N}$ whose fibres are given by the tensor products $X_x \otimes Y_x$. Similarly, one obtains the local trivializations of the outer product from the trivializations of $\mathfrak{X}, \mathfrak{Y}$ and tensor products. \par
%
Finally, we generalize the notion of vector fields:
\begin{definition}[Sections of vector bundles]
Let $\mathfrak{X}=(E,\N,\pi)$ be a vector bundle with fibres $X_x=\pi^{-1}(x)$ at $x \in \N$. A \emph{smooth section} of the vector bundle is a smooth map $\gamma : \N \to E$ such that $\gamma(x) \in X_x$ for all $x \in \N$. The \emph{vector space of smooth sections} of $\mathfrak{X}$ is denoted by \gls{gammax}, the one with compactly supported sections is as usual denoted by \gls{gammaxzero}.
\end{definition}
In this language, a vector field $v$ is just a smooth section of the tangent bundle of a manifold, $v \in \Gamma(T\N)$. One may therefore identify the physical notion of fields with smooth sections of vector bundles. This point of view will be used to define the notion of differential forms in Section \ref{sec:differential_forms}.\par
In this thesis, we usually are interested in complex valued functions (or sections in general). Therefore, we view all occurring vector bundles as complex, in the sense that we take two distinct copies of the vector bundle, one representing the real, one the imaginary part of the bundle. A section of that complex vector bundle is just a pair of two sections of the real vector bundle under consideration. From now, if not specified explicitly, we will view all vector bundles, including the tangent bundle $T\N$, as complex vector bundles. Accordingly, smooth sections of those bundles will in general be complex valued.
%
%
%
%
%
%
%
%
%%%%%%%
%%PARTIAL DIFFERENTIAL OPERATORS AND GLOBAL HYPERBOLICITY
%%%%%%%
%
%
%
\subsection{Partial differential operators and global hyperbolicity}\label{sec:global_hyperbolicity}
When dealing with field theories, whether classical or quantum, one is, of course, interested in the dynamics of the fields. These are usually described by some partial differential equation, often of second order. In the following, we give a short introduction to the theory of certain partial differential operators acting on smooth sections of a vector bundle over the spacetime $(\M,g)$.\par
%
As we have seen, these smooth sections are generalizations of the notion of a field.  In the following, let $\mathfrak{X}$ denote a vector bundle over the manifold $\M$ and let $P: \Gamma(\mathfrak{X}) \to \Gamma(\mathfrak{X})$ be a partial differential operator acting on smooth sections of the bundle. As in the case of flat spacetime, we are interested in basic questions regarding the differential equation $Pf = j$, for example: Can we formulate a (globally) well posed initial value problem? Does the differential equation possess (unique) solutions? To answer these questions, we will now restrict to the case where $P$ is linear and of second order, as it is often the case in physical applications. One can show that for a certain class of such operators, namely normally hyperbolic partial differential operators of second order, we can rigorously treat these questions.\par
Choosing local coordinates $x=(x_\mu)$ on $\M$ and a local trivialization of $\mathfrak{X}$, a linear partial differential operator of second order is called \emph{normally hyperbolic} if it takes the form
\begin{align}
	P = - \sum_{\mu,\nu} g^{\mu \nu} \partial_\mu \partial_\nu + \sum_{\alpha} A_\alpha (x) \partial_\alpha + B(x) \formspace,
\end{align}
where $A_\alpha$ and $B$ are matrix-valued coefficients depending smoothly on the coordinate $x$ (see. \cite[Chapter 1.5]{baer_ginoux_pfaeffle}). One can also formulate a coordinate independent definition in terms of the principal symbol, which we will not present here (see for example \cite[Section 1.5]{baer_ginoux_pfaeffle} ). \par
%
Normally hyperbolic operators possess unique fundamental solutions (see for example the fundamental solutions to the wave operator as noted in Lemma \ref{lem:fundamental_solution_wave_operator}). These fundamental solutions fulfill certain physically important properties, such as a finite propagation speed smaller than the speed of light. Furthermore, specifying the initial data on some space-like hypersurface $X \in  \M$ specifies a unique solution on the domain of dependence $D(X)$ of $X$. Due to these properties, one often calls normally hyperbolic operators just \emph{wave operators}. But to state a \emph{globally} well posed initial value problem for a wave equation, we need to restrict the class of spacetimes $\M$ under consideration to those that possess space-like hypersurfaces $X$ whose domain of dependence is all of the spacetime, $D(X) = \M$. This leads to the notion of \emph{globally hyperbolic} spacetimes:
\begin{definition}[Global Hyperbolicity]
	A spacetime $\M$ is called \emph{globally hyperbolic} if there exists a Cauchy surface $\gls{sigma}$ in $\M$.
\end{definition}
\noindent Here, a Cauchy surface is a space-like hypersurface $\Sigma \subset \M$ such that every inextendible causal curve $\gamma$ intersects $\Sigma$ exactly once. One can show that Cauchy surfaces fulfill the desired property mentioned above, that is,  $D(\Sigma) = \M$. Furthermore, one can show that any globally hyperbolic spacetime $\M$ is foliated by a one-parameter family $\left\{ \Sigma_t \right\}_t$ of Cauchy surfaces (see for example \cite[Theorem 8.3.14]{wald_GR}). \par
In physical applications, one often finds the dynamics of a theory to be described by wave operators. Most prominently, the Klein-Gordon operator $(\square + m^2)$ acting on scalar fields, or its generalization, the wave operator acting on differential forms introduced in Section \ref{sec:differential_forms}, is normally hyperbolic. But there are also important physical field theories that are not described by wave operators, such as the Proca field treated in this thesis. It turns out that the Proca operator (see Definition \ref{def:proca_operator}) is a so called \emph{Green-hyperbolic} operator. These are again partial differential operators $P$ of second order acting on smooth sections of some vector bundle, such that $P$ (and its dual $P'$) posses fundamental solutions. Obviously, normally hyperbolic operators are Green-hyperbolic, but the opposite is not true. One can generalize some results obtained by studying normally hyperbolic operators to Green-hyperbolic operators. An introduction to this topic is given in \cite{baer_green-hyperbolic}, where it is also shown that the Proca operator is Green-hyperbolic but not normally hyperbolic.\par
For our application, the notion of Green-hyperbolicity is not of vast importance, but it is worth mentioning that there exists a more detailed mathematical background on the treatment of such operators.
A very detailed description of normally hyperbolic operators on Lorentzian manifolds, including proofs of the above statements regarding the initial value problem and the existence of fundamental solutions, is given in \cite{baer_ginoux_pfaeffle}, also with an overview of quantization. A shorter introduction to the topic is for example treated in \cite{baer-ginoux_classical-and-quantum-fields}, also with a description of quantization.
%
%
%
%
%
%
%%%
%
%
%
%%
%%%%%%%%%
%%%DIFFERENTIAL FORMS
%%%%%%%%
%
%
%
\subsection{Differential forms}\label{sec:differential_forms}
%
%
Differential forms provide an elegant, coordinate independent description of calculus on smooth manifolds. In particular, they generalize the notion of line- and volume-integrals that are known from analysis. Differential forms play a remarkable role in physics, as one can argue that they indeed describe fundamental physical entities. As an example, instead of viewing a classical force as a vector, one can think of it, more closely related to experiments, as a differential one-form that assigns a scalar to a tangent vector of a curve. This scalar is the (infinitesimal) work associated with the force along the curve. Also, differential forms allow for an elegant geometric description of field theories, for example the Maxwell and Proca field theories that we encounter in this thesis. In Maxwell's classical theory of electromagnetism, instead of viewing the electric and magnetic field (which are conceptually just forces) as the fundamental physical entities, one introduces the \emph{vector potential}, a one-form, consisting of the scalar electric potential and the vector potential associated with the magnet field. Experiments like the Aharonov-Bohm experiment allow for an interpretation of the vector potential as the fundamental physical object, rather than the associated electromagnetic field. \\
Even more fundamentally, the two main theories of physics, General Relativity and the Standard Model of particle physics, are field theories. They are deeply connected to a geometric interpretation and can be elegantly described using differential forms. \par
%
%
Despite of all this, differential forms are usually not part of the standard curriculum of physicists. We shall therefore introduce the basic aspects and definitions regarding differential forms that are used in this thesis. For a more detailed introduction we refer to the literature: For example \cite[Chapter 2 and 4]{rudolph_schmidt} or \cite[Appendix B]{wald_GR} provide introductions to the topic.\par
%
%
In the following, let $\N$ denote a smooth $N$-dimensional manifold, assumed to be Hausdorff, connected, oriented and para-compact, with either Lorentzian or Riemannian metric $k$ and Levi-Civita connection $\nabla$. For a Lorentzian manifold we use the sign convention $(-,+,\dots,+)$ of the metric $k$. The number of negative eigenvalues of $k$ is denoted by $s$, so $s=0$ for a Riemannian manifold and, in our convention, $s=1$ for a Lorentzian manifold.
Later, we will specify to a four dimensional (globally hyperbolic) spacetime consisting of a four dimensional manifold $\M$ with Lorentzian metric $g$ and Cauchy surface $\Sigma$ with induced Riemannian metric $h$.
%
We define:
\begin{definition}[Differential form]
	Let $p\in \{0,1,\dots,N\}$. A \emph{differential form} $\omega$ of degree $p$, or $p$-form for short, on the manifold $\N$ is an anti-symmetric tensor field of rank $(0,p)$. That is, at every point $x \in \N$, $\omega_x$ is an anti-symmetric multi-linear map
	\begin{align}
	\omega_x : \underbrace{T_x \N \times T_x \N \times \cdots \times T_x \N}_{p\text{-times}} \to \IR \formspace.
	\end{align}
	We denote the vector space\footnote{Naturally, addition and scalar multiplication are defined point-wise.} of $p$-forms on $\N$ by $\gls{omegap}$, the space with compactly supported ones by \gls{omegapz}.
\end{definition}
As an example, a zero-form $f \in \Omega^0(\N)$ is just a $C^\infty$-function from $\N$ to $\IR$, hence we can identify $\Omega^0(\N) = C^\infty (\N, \IR)$. A one-form $A \in \Omega^1(\N)$ is nothing more than a co-vector field and in a physical context usually denoted in local coordinates by $A_\mu$. Note, that alternatively one can directly define a $p$-form as a smooth section of the $p$-th exterior product of the co-tangent bundle and hence identify $\Omega^p(\N) = \Gamma \big( \largewedge^k T^*\N\big)$. As mentioned in Section \ref{sec:spacetime_geometry}, we view the tangent bundle as a complex bundle. Therefore, the sections of that bundle will be complex valued functionals. In that fashion, we will usually view the spaces $\Omega^p(\N)$ as complex valued differential forms.\par
%
Next we define the basic operations, besides addition and scalar multiplication, that one can perform on differential forms.
%
\begin{definition}[Exterior product]
	Let $A \in \Omega^p(\N)$ be a $p$-form and  $B\in \Omega^q(\N)$ a $q$-form on $\N$. \\
	The \emph{exterior product} $\gls{wedge}:\Omega^p(\N) \times \Omega^q(\N) \to \Omega^{p+q} (\N)$ is defined by
	\begin{align}
	(A \wedge B)_{\mu_1\dots\mu_p \nu_1\dots\nu_q} = \frac{(p+q)!}{p!q!}\, A_{[\mu_1 \dots \mu_p} B_{\nu_1\dots\nu_q]} \formspace,
	\end{align}
	where the anti-symmetrization of a tensor $T$ is given through
	\begin{align}
	T_{[\mu_1\dots\mu_p]} = \frac{1}{p!} \sum\limits_{\sigma\in S_N }\textrm{sgn}(\sigma) T_{\sigma(\mu_1)\dots\sigma(\mu_p)} \formspace.
	\end{align}
\end{definition}
Here, $S_N$ denotes the symmetric group\footnote{Usually the symmetric group is defined as the set of permutations of $\{1,2,\dots,N\}$ but we chose the index to run over $\{0,1,\dots,N-1\}$, identifying the time component with zero rather then one.} of degree $N$, consisting of permutations of the set $\{0,1,\dots,N-1\}$.
With this notion of multiplication, point-wise addition and scalar multiplication, the space $\gls{omega} \coloneqq \bigoplus_{p = 0}^\infty \Omega^p(\N) = \bigoplus_{p = 0}^N \Omega^p(\N)$ becomes an algebra, usually called the Grassmann- or \emph{exterior algebra} of differential forms on $\N$. We have used that obviously $\Omega^k(\N) =0$ for $k >N$ due to the anti-symmetrization.\par
Furthermore, we find a notion of how to \emph{pullback} differential forms on manifolds to another manifold, for example the pullback of a differential form on the spacetime $\M$ to differential forms on its Cauchy surface $\Sigma$. Given a $C^\infty$-map $\psi: \widetilde{\N} \to \N$, where $\N, \widetilde{\N}$ are manifolds, we can naturally define the pullback of a function $f \in \Omega^0(\N)$ to a function $(\psi^* f) \in \Omega^0(\widetilde{\N})$ by composing $f$ with $\psi$:
\begin{align}
\psi^* f \coloneqq f \comp \psi \formspace.
\end{align}
\newpage
With the pullback of functions defined, we can define how to \emph{push forward}, or carry along, vector fields on $\widetilde{\N}$ to vector fields on $\N$: Let $f\in \Omega^0(\N)$ and $\tilde{v} \in \Gamma(T\widetilde{\N})$ and $\tilde{x} \in \widetilde{\N}$. Then
\begin{align}
(\psi_* \tilde{v})_{\psi(\tilde{x})} (f) \coloneqq \tilde{v}_{\tilde{x}}(\psi^* f)
\end{align}
defines the vector field $(\psi_* v) \in \Gamma(T\N)$. With these basic operations at hand, we can generalize to define the pullback of differential forms:
\begin{definition}[Pullback]\label{def:pullback}
	Let $\N, \widetilde{\N}$ be manifolds of dimension $N,\widetilde{N}$ respectively, and let $\psi: \widetilde{\N} \to \N$ be a smooth map. Then, $\psi$ defines an algebra homomorphism $\psi^* : \Omega(\N) \to  \Omega(\widetilde{\N})$,
	called the \emph{pullback} of differential forms. For $\omega \in \Omega^p(\N)$, $\tilde{x} \in \widetilde{\N}$ and $\tilde{v}_i \in T_x \widetilde{\N}$, $i=1,2,\dots,p$, it is defined by
	\begin{align}
	\left( \psi^* \omega \right)_{\tilde{x}}  (\tilde{v}_1,\tilde{v}_2,\dots,\tilde{v}_p) \coloneqq \omega_{\psi(\tilde{x})} (\psi_* \tilde{v}_1, \dots , \psi_* \tilde{v}_p) \formspace.
	\end{align}
\end{definition}
%
%
%
%
On the exterior algebra we find a duality, provided by the Hodge operator:
\begin{definition}[Hodge dual]
	The hodge star operator $\gls{hodge}: \Omega^p(\N) \to \Omega^{N-p}(\N)$ is defined through
	\begin{align}
	B \wedge *A = \frac{1}{p!} B^{\mu_1\dots\mu_p}A_{\mu_1\dots\mu_p} \dvolk \formspace,
	\end{align}
	which yields the coordinate representation
	\begin{align}
	(*A)_{\mu_{p+1}\dots\mu_N} = \frac{\detk}{p!} \, \epsilon_{\mu_1\dots\mu_N} A^{\mu_1\dots\mu_p} \formspace.
	\end{align}
\end{definition}
Here, \gls{levicivita} denotes the fully antisymmetric tensor of rank $N$ (Levi-Civita symbol) satisfying $\epsilon_{12,\dots,N} =1$ and the \emph{volume element} \gls{dvolk} is defined by
\begin{align}
\left( \gls{dvolk} \right)_{\alpha_1\dots\alpha_N} = \detk \, \epsilon_{\alpha_1\dots\alpha_N} \formspace.
\end{align}
In a sense, the volume element describes how the curvature of the manifold deforms a unit volume.
The duality follows from the important property of the Hodge operator as stated in the following lemma:
\begin{lemma}
	Let $*$ denote the Hodge star operator on the exterior algebra $\Omega(\N) $. It holds that
	\begin{align}
	** = (-1)^{s+p(N-p)} \, \mathbbm{1} \formspace,
	\end{align}
	which is trivially equivalent to $*^{-1} = (-1)^{s+p(N-p)} \, *$.
\end{lemma}
\begin{proof}
	Let $A \in \Omega^p(\N)$ be a $p$-form on $\N$. Then:
	\begin{align}
	(*{*A})_{\mu_1 \dots \mu_p}
	&= \frac{\detk \, \detk}{p! \, (N-p)!} \; \epsilon_{\alpha_{p+1}\dots\alpha_N \mu_1 \dots \mu_p}\;\epsilon^{\alpha_{1}\dots\alpha_N}\;A_{\alpha_1\dots\alpha_p} \notag\\
	&= (-1)^{p(N-p)} \frac{\detk \, \detk}{p! \, (N-p)!} \; \epsilon_{\alpha_{p+1}\dots\alpha_N \mu_1 \dots \mu_p}\;\epsilon^{\alpha_{p+1}\dots\alpha_{N}\alpha_1\dots\alpha_p}\;A_{\alpha_1\dots\alpha_p}  \notag\\
	&= (-1)^{s+p(N-p)} \delta\indices{^{[\alpha_{1}}_{\mu_{1}}}\, \dots \, \delta\indices{^{\alpha_p ] }_{\mu_p}} \;A_{\alpha_1\dots\alpha_p} \notag\\
	&=  (-1)^{s+p(N-p)}\;A_{\mu_1\dots\mu_p} \formspace
	\end{align}
	We have used Lemma \ref{lem:epsilon_contraction} and, in the last step, that the anti-symmetrization is absorbed by contraction because $A$ is antisymmetric.
\end{proof}
%
%
%
%
%
Furthermore, we can equip the exterior algebra with a differentiable structure, introducing the notion of the exterior derivative.
\begin{definition}[Exterior derivative]
	The \emph{exterior derivative} $\gls{d}:\Omega^p(\N) \to \Omega^{p+1} (\N)$ is defined by the following properties:
	\begin{enumerate}
		\item $d$ is linear
		\item $d$ obeys a graded Leibniz rule: Let $A \in \Omega^p(\N)$ and  $B\in \Omega^q(\N)$, then
		\begin{align}
		d(A \wedge B) = dA \wedge B + (-1)^p \, A \wedge dB
		\end{align}
		\item $d$ is nilpotent, that is,  $d^2 = 0$.
	\end{enumerate}
	In local coordinates, this is equivalent to the representation
	\begin{align}
	(dA)_{\mu \alpha_1\dots\alpha_p} = (p+1)\, \nabla_{[\mu}A_{\alpha_1\dots\alpha_p]} \formspace.
	\end{align}
\end{definition}
An important property of the exterior derivative is that it commutes (or rather intertwines its action) with pullbacks (see \cite[Proposition 4.1.7]{rudolph_schmidt}).
A $p$-form $\omega \in \Omega^p(\N)$ is called \emph{exact} if there is a $(p-1)$-form $\alpha \in \Omega^{p-1}(\N)$ such that $\omega = d\alpha$. We call $\omega$ \emph{closed} if $d \omega =0$. Accordingly, the space of closed $p$-forms is denoted by \gls{omegapd}, the space of exact ones by \gls{domegap}. As usual, the ones with compact support are denoted by a subscript zero. Note, that every exact form is closed, using that $d$ is by definition nilpotent, but the reverse is in general not true. It does hold, however, on certain manifolds with trivial topology, such as Minkowski spacetime. This is expressed in the so called Poincar\'e-Lemma (see for example \cite[Chapter 4]{bott_tu}) based on the study of de Rham cohomology.\par
%
Moreover, $N$-forms can naturally be integrated. Using local coordinates and a partition of unity, we define the integral of $N$-forms via the well known integration on $\IR^N$:
\begin{definition}[Integration on manifolds]
	Let $\left\{U_\alpha, \psi_\alpha\right\}_\alpha$ be an atlas of the manifold $\N$ and $\left\{\chi_\alpha\right\}_\alpha$ a partition of unity subordinate to the locally finite open cover $\left\{U_\alpha\right\}_\alpha$. Let $x^\mu_{(\alpha)}$ be a coordinate basis of $\psi$ on $U_\alpha$. For any $N$-form $\omega \in \Omega^N_0(\M)$ we define the integral
	\begin{align}
	\int\limits_{\N} \omega &\coloneqq \sum_{\alpha} \int\limits_{\psi_\alpha (U_\alpha)} w(x_{(\alpha)}^0,\dots,x_{(\alpha)}^1)\; dx_{(\alpha)}^0 \cdots dx_{(\alpha)}^{N-1} \formspace,
	\end{align}
	where $w$ are the components of $\omega$ in the coordinates $x_{(\alpha)}^\mu$, that is $\omega = w dx_{(\alpha)}^0 \wedge \cdots \wedge dx_{(\alpha)}^{N-1}$.
	This definition is independent of the choice of the atlas and the partition of unity (see \cite[Proposition 3.3]{bott_tu}).
\end{definition}
With integration at our disposal, we present an important theorem regarding the integration of exact differential forms:
\begin{theorem}[Stoke's Theorem]\label{thm:stokes}
	Let $\N$ be an oriented manifold of dimension $N$ and let its boundary $\partial \N$ be endowed with the induced orientation. Let $\gls{inclusionmap} : \partial \N \hookrightarrow \N$ be the inclusion operator.
	Let $\omega \in \Omega^{N-1}_0(\N)$ be a compactly supported $(N-1)$-form on $\N$. Then it holds
	\begin{align}
	\int\limits_\N d\omega = \int\limits_{\partial \N} i^*\omega \formspace.
	\end{align}
\end{theorem}
\begin{proof}
	A proof is given in most of the introductory literature on differential geometry (see for example \cite[Chapter 17, Theorem 2.1]{lang}).
	Note that one can equivalently formulate Stoke's theorem on a \emph{compact} manifold but for {arbitrary} (that is, in general not compactly supported) $(N-1)$-forms on the manifold (see for example \cite[Theorem 4.2.14]{rudolph_schmidt}). This will be of importance in later calculations.
\end{proof}
%
Furthermore, we can define a bilinear map on $\Omega^p(\N)$ using the integration of $N$-forms:
\begin{definition}
	Let $A,B \in \Omega^p(\N)$ such that their supports have a compact intersection. Define the bilinear map $\gls{innerprod} : \Omega^p(\N) \times \Omega^p(\N) \to \IC$ by
	\begin{align}
	\langle A, B \rangle_\N \coloneqq  \int_{\N } A \wedge * B = \int_{\N } A_{\mu_1 \dots \mu_p}B^{\mu_1 \dots \mu_p}\,\dvolk \formspace.
	\end{align}
\end{definition}
Since by definition $A \wedge * B$ is a compactly supported $N$-form, this is well defined. We may sometimes refer to $\langle \cdot , \cdot \rangle_\N$ as an inner product for simplicity, even though it is not positive definite.
%
%
%
%
%
Using the exterior derivative, we define the interior or co-derivative:
\begin{definition}[Interior derivative]
	The \emph{interior derivative} $\gls{delta} : \Omega^p(\N) \to \Omega^{p-1}(\N)$ is defined by
	\begin{align}
	\delta \coloneqq (-1)^{s+1+N(p-1)}\, {*{d*}} \formspace.
	\end{align}
	From the defining properties of $d$ and $*$ it follows $\delta^2 =0$.
\end{definition}
Here, $s$ again denotes the number of negative eigenvalues of the metric $k$ of $\N$. In accordance with our nomenclature, we call a $p$-form $\omega$ co-exact if there exists a $\alpha \in \Omega^{p+1}(\N)$ such that $\omega = \delta \alpha$ and co-closed if $\delta \omega = 0$. Accordingly, the spaces of co-closed and co-exact $p$-forms are denoted by \gls{omegapdelta} and \gls{deltaomegap} respectively.\par
Using the exterior and interior derivative we define the partial differential operator:
\begin{definition}[D'Alembert Operator]
	The d'Alembert (or Laplace - de Rham) operator $\gls{dalembert}: \Omega^p(\N) \to \Omega^{p}(\N)$ is defined by
	\begin{align}
	\square \coloneqq \delta d +d \delta \formspace.
	\end{align}
\end{definition}
By definition of the exterior and interior derivative, it is easy to show that $\square$ commutes with both $d$ and $\delta$:
\begin{align}
\square d &= (\delta d + d \delta )d \notag \\
&= d \delta d \notag \\
&= d (\delta d + d \delta) \formspace,
\end{align}
and analogously for $\delta$.
The d'Alembert operator, and its generalization to $(\square + m^2)$ for some constant $m > 0$, are important examples for a normally hyperbolic differential operators (see Section \ref{sec:global_hyperbolicity}) and we may therefore sometimes just refer to them as \emph{wave operators}.\par
The sign convention in the definition of the exterior derivative is chosen such that on any Lorentzian or Riemannian manifold the interior derivative is formally adjoint to the exterior derivative, that is,  for $A \in \Omega^{p}(\N)$ and $B \in \Omega^{p+1}(\N)$ it holds that
\begin{align}
\langle dA , B \rangle_{\N} = \langle A , \delta B \rangle_\N \formspace,
\end{align}
which leads to a representation in local coordinates of the Manifold given by:
\begin{align}
(\delta A)_{\mu_2\dots\mu_p} = - \nabla^{\mu_1}A_{\mu_1\dots\mu_p} \formspace.
\end{align}
To see that this is consistent, let $A \in \Omega^{p-1}(\N)$ and $B \in \Omega^{p}(\N)$ such that their supports have compact intersection.
We obtain, using Stoke's Theorem \ref{thm:stokes}:
\begin{align}
0 &= \int \limits_{\partial \N} i^* (A \wedge *B) \notag\\
&= \int \limits_{\N} d(A \wedge *B)  \notag\\
&= \int \limits_{\N} dA \wedge *B + (-1)^{p-1} A \wedge d{*B} \notag\\
&= \int \limits_{\N} dA \wedge *B + (-1)^{p-1} A \wedge *{*^{-1}}\underbrace{d{*B}}_{\textrm{is a } (N-p+1) \textrm{ form.}} \notag\\
&= \int \limits_{\N} dA \wedge *B + (-1)^{p-1}(-1)^{s+(N-p+1)(N-N+p-1)} A \wedge *{*d{*B}} \notag\\
&= \int \limits_{\N} dA \wedge *B + (-1)^{p+(1-p)(p-1)} A \wedge *\delta B \formspace.
\end{align}
It can easily be proven by induction that $\big(p+(1-p)(p-1)\big)$ is odd for any $p \in \IN$, which yields the result
\begin{align}
\langle dA , B \rangle_{\N} = \langle A , \delta B \rangle_\N \formspace.
\end{align}
The definitions stated above thus fulfill the requirement of formal adjointness of the exterior and interior derivate on an arbitrary Lorentzian or Riemannian manifold $\N$.
In local coordinates we use a partial integration to obtain
\begin{align}
\langle dA , B \rangle_\N &= \int \limits_{\N} dA \wedge * B \notag\\
%&= \int \limits_{\N} \frac{1}{p!} (dA)^{\alpha_1\dots\alpha_p}\,B_{\alpha_1 \dots \alpha_p} \, \dvolk \notag\\
&= \int \limits_{\N}  \frac{p}{p!} \nabla^{[\alpha_1}A^{\alpha_2\dots\alpha_p]}\,B_{\alpha_1 \dots \alpha_p} \, \dvolk \notag\\
&= \int \limits_{\N}  \frac{1}{(p-1)!} \nabla^{\alpha_1}A^{\alpha_2\dots\alpha_p}\,B_{\alpha_1 \dots \alpha_p} \, \dvolk \notag\\
&= - \int \limits_{\N}  \frac{1}{(p-1)!} A^{\alpha_2\dots\alpha_p}\, \nabla^{\alpha_1}B_{\alpha_1 \dots \alpha_p} \, \dvolk \notag\\
&= \langle A, \delta B \rangle_\N \formspace,
\end{align}
which yields
\begin{align}
-\nabla^{\alpha_1}B_{\alpha_1 \dots \alpha p} = (\delta B)_{\alpha_2 \dots \alpha_p}\formspace.
\end{align}
On the four dimensional spacetime $(\M,g)$ the definitions of the Hodge star operator and the interior derivative simplify, such that
\begin{align}
*_{(\M)}*_{(\M)} &= (-1)^{p+1} \mathbbm{1} \\
\delta_{(\M)} &= *_{(\M)}{d_{(\M)}*_{(\M)}} \formspace ,
\end{align}
holds on the spacetime $(\M,g)$ and
\begin{align}
*_{(\Sigma)}*_{(\Sigma)} &= \mathbbm{1} \\
\delta_{(\Sigma)} &= (-1)^p *_{(\Sigma)}{d_{(\Sigma)}*_{(\Sigma)}}
\end{align}
holds on  $(\Sigma,h)$. In the following we will drop the subscript ${(\M)}$, since we will perform all the calculations on a four dimensional spacetime, except when explicitly noted (for example with a subscript $(\Sigma)$).
%
%
%
%
%
%
%
%
%%%%%%
%%CATEGORY THEORY
%%%%%%
\subsection{Category theory}\label{sec:cat-theory}
The description of Quantum Field Theory on Curved Spacetimes (QFTCS) in the framework of \name{Brunetti}, \name{Fredenhagen} and \name{Verch} \cite{Brunetti_Fredenhagen_Verch} is based on category theory. In this thesis, we will not go into detail on those categorical aspects, however we will need some basic definitions to formulate the theory rigorously, that is namely the notion of a category and that of covariant functors, since, in the used framework, the generally covariant QFTCS is a functor.\par
Here, we present definitions given in \cite[Appendix A.1]{baer_ginoux_pfaeffle} and refer to the appropriate literature for details. We define:
\begin{definition}[Category]
	A \emph{category} $\mathsf{Cat}$ consists of the following:
	\begin{enumerate}
		\item a class $\mathsf{Obj}_\mathsf{Cat}$ whose members are called \emph{objects},
		\item a set $\mathsf{Mor}_\mathsf{Cat}(A,B)$, for any two objects $A,B \in \mathsf{Obj}_\mathsf{Cat}$, whose elements are called \emph{morphisms},
		\item for any three objects $A,B,C \in \mathsf{Obj}_\mathsf{Cat}$ there is a map
		\begin{align}
\mathsf{Mor}_\mathsf{Cat}(B,C) \times \mathsf{Mor}_\mathsf{Cat}(A,B) &\to \mathsf{Mor}_\mathsf{Cat}(A,C) \notag\\
(\psi,\phi) &\mapsto \psi \comp \phi
		\end{align}
		called the composition of morphisms subject to the relations:\vspace{4mm}
		\begin{enumerate}[label=(\arabic*)]
			\item for non equal pairs $(A,B)$, $(A',B')$ of objects, the sets $\mathsf{Mor}_\mathsf{Cat}(A,B)$ and $\mathsf{Mor}_\mathsf{Cat}(A',B')$ are disjoint,
			\item for every object $A$ there exists a morphism $\text{id}_A \in \mathsf{Mor}_\mathsf{Cat}(A,A)$ such that it holds for all objects $B$, morphisms $\psi \in \mathsf{Mor}_\mathsf{Cat}(B,A)$ and $\phi \in \mathsf{Mor}_\mathsf{Cat}(A,B)$
			\begin{align}
				\text{id}_A \comp \psi &= \psi \quad \text{and}\\
				\phi \comp \text{id}_A &= \phi \quad,
			\end{align}
			\item the composition law is associative, that is for an objects $A,B,C,D$ and any morphisms $\psi \in \mathsf{Mor}_\mathsf{Cat}(A,B)$, $\phi \in \mathsf{Mor}_\mathsf{Cat}(B,C)$ and $\chi \in \mathsf{Mor}_\mathsf{Cat}(C,D)$ it holds
			\begin{align}
				(\chi \comp \phi) \comp \psi = \chi \comp (\phi \comp \psi) \formspace.
			\end{align}
		\end{enumerate}
	\end{enumerate}
\end{definition}
%
%
%
\begin{definition}[Functor]
	Let $\mathsf{Cat1}$ and $\mathsf{Cat2}$ be categories. A \emph{covariant functor} $\mathscr{A}: \mathsf{Cat1} \to \mathsf{Cat2}$ consists of the map $\mathscr{A} : \mathsf{Obj}_\mathsf{Cat1} \to \mathsf{Obj}_\mathsf{Cat2}$ and maps $\mathscr{A}: \mathsf{Mor}_\mathsf{Cat1}(A,B) \to \mathsf{Mor}_\mathsf{Cat2}\big(\mathscr{A}(A),\mathscr{A}(B)\big)$ for any two objects $A,B \in \mathsf{Obj}_\mathsf{Cat1}$ such that
	\begin{enumerate}
		\item {the composition is preserved, that is for all objects $A,B,C \in \mathsf{Obj}_\mathsf{Cat1}$ and for any morphisms $\psi \in \mathsf{Mor}_\mathsf{Cat1}(A,B)$ and $\phi \in \mathsf{Mor}_\mathsf{Cat1}(B,C)$ it holds
		\begin{align}
			\mathscr{A}(\phi \comp \psi) = \mathscr{A}(\phi) \comp \mathscr{A}(\psi) \formspace,
		\end{align}}
		\item{
			$\mathscr{A}$ maps identities to identities, that is for any object $A \in \mathsf{Obj}_\mathsf{Cat1}$ it holds
			\begin{align}
				\mathscr{A}(\text{id}_\mathsf{A}) = \text{id}_{\mathscr{A}(A)} \formspace.
			\end{align}
			}
	\end{enumerate}
\end{definition}
%
%
%
%
%
%
%
%
%
%
%
%
%%%%%%
%%SIGN CONVENTIONS
%%%%%%
%
%
\subsection{Sign conventions}\label{sec:sign_conventions}
At certain points throughout this chapter we have had a freedom of choice regarding the signs of some entities, in particular the sign of the signature of the Lorentzian metric $g$ and that of the interior derivative $\delta$. Though at this stage the choice can be made arbitrarily, we want to make it in a way that in the end allows us to make certain physical interpretations on some parameters. More precisely, we want to interpret the parameter $m$ of the Klein-Gordon equation\footnote{or its generalization on $p$-forms} $(\square + m^2) f = 0$ for a zero-form $f \in \Omega^0(\M)$ as a mass in the physical sense. With the chosen sign convention for $\delta$ we find, using ${\delta}f = 0$:
\begin{align}
	\square f
	&= (\delta d + d \delta) f \notag\\
	&= \delta d f \notag\\
	&= - \nabla^\mu \nabla_\mu f \formspace.
\end{align}
In the following heuristic (local) argument we see
\begin{align}
	\square + m^2
	&= -\nabla^\mu \nabla_\mu + m^2 \notag\\
	&\sim \partial_t^2 + \sum_i \partial_i^2 + m^2\notag\\
	&\sim -E^2 + \abs{\vector{p}}^2 + m^2
\end{align}
which yields the correct relativistic relation of energy, momentum and mass according to $E^2 = \abs{\vector{p}}^2 + m^2$.
A similar calculation holds for the Klein-Gordon operator generalized to act on one-forms. If we had found a ``wrong'' relation between energy, momentum and mass, we would have had to adapt the chosen signs. Usually one chooses the sign of the metric and the interior derivative such that they are in some sense mathematically convenient (although one might disagree with another one's choice). We have made the choice of the metric, such that the Cauchy surfaces become Riemannian rather that ``anti-Riemannian'' (with an all minus signature), which seems more natural to some. Also, a lot of the used references on spacetime geometry (in particular the book by \name{Wald} \cite{wald_GR}) use this sign convention, which makes the application of certain formulas easier. As mentioned, the sign of the interior derivative was chosen such that it is formally adjoint to the exterior derivative (with respect the specified inner product) on all Lorentzian and Riemannian manifolds. It seemed convenient for the actual calculations to fix the sign regardless of the signature of the metric of the underlying manifold. One could equivalently have fixed the opposite sign, yielding the two derivatives to be skew-adjoint, which is also done in the literature. However, in the end, one has one freedom left to make the energy-momentum-mass relation work: that is the sign in front of the mass in the Klein-Gordon equation and all other wave equations accordingly. Hence, one regularly also finds the Klein-Gordon equation to be defined with a flipped sign of the mass term. But for our case, we want the mass $m$ in any wave equation to appear with a positive sign.
%
%

\section{Upper Bounds for Identity Testing}
\label{sec:ones-ub}
In this section, we prove the following theorems for identity testing.
\begin{theorem}\label{thm:ones-csq-h}
There exists an algorithm for identity testing between $p$ and $q$ distinguishing the cases:
\begin{itemize}
\item $\dxs(p,q) \leq \ve^2$;
\item $\dh(p,q) \geq \ve$.
\end{itemize}
The algorithm uses $O\left(\frac{n^{1/2}}{\ve^2}\right)$ samples.
\end{theorem}

\begin{theorem}\label{thm:ones-tv}
There exists an algorithm for identity testing between $p$ and $q$ distinguishing the cases:
\begin{itemize}
\item $\dlt(p,q) \leq \frac{\ve}{\sqrt{n}}$;
\item $\dtv(p,q) \geq \ve$.
\end{itemize}
The algorithm uses $O\left(\frac{n^{1/2}}{\ve^2}\right)$ samples.
\end{theorem}

\begin{theorem}\label{thm:ones-h}
There exists an algorithm for identity testing between $p$ and $q$ distinguishing the cases:
\begin{itemize}
\item $\dlt(p,q) \leq \frac{\ve^2}{\sqrt{n}}$;
\item $\dh(p,q) \geq \ve$.
\end{itemize}
The algorithm uses $O\left(\frac{n^{1/2}}{\ve^2}\right)$ samples.
\end{theorem}

We prove Theorem~\ref{thm:ones-csq-h} in Section~\ref{sec:id-csq-h}, and Theorems~\ref{thm:ones-tv} and~\ref{thm:ones-h} in Section~\ref{sec:id-lt}.

\subsection{Identity Testing with Hellinger Distance and $\chi^2$-Tolerance}
\label{sec:id-csq-h}

We prove Theorem~\ref{thm:ones-csq-h} by analyzing Algorithm~\ref{alg:testing}.
We will set $c_1 = \frac{1}{100}, c_2 = \frac{6}{25}$, and let $C$ be a sufficiently large constant.
\begin{algorithm}[h]
\caption{$\chi^2$-close versus Hellinger-far testing algorithm}\label{alg:testing}
\begin{algorithmic}[1]
\State \textbf{Input:} $\ve$; an explicit distribution $q$; sample access to a distribution $p$
\State Implicitly define $\mathcal{A} \leftarrow \{i:q_i \geq c_1\ve^2/n\}$, $\mathcal{\bar A} \leftarrow [n] \setminus \mathcal{A}$
\State Let $\hat p$ be the empirical distribution\footnote{The empirical distribution is defined by taking a set of samples and normalizing the counts such that the result forms a probability distribution.} from drawing $m_1 = \Theta(1/\ve^2)$ samples from $p$
\If {$\hat p(\mathcal{\bar A}) \geq \frac34 c_2\ve^2$} \label{ln:light-test}
\State \Return \reject \label{ln:early-reject}
\EndIf
\State Draw a multiset $S$ of $\mathrm{Poisson}(m_2)$ samples from $p$, where $m_2 = C\sqrt{n}/\ve^2$
\State Let $N_i$ be the number of occurrences of the $i$th domain element in $S$
\State Let $S'$ be the set of domain elements observed in $S$
\State $Z \leftarrow \sum_{i \in S' \cap \mathcal{A}} \frac{(N_i - m_2q_i)^2 - N_i}{m_2q_i} + m_2 (1 - q(S' \cap \mathcal{A}))$ \label{ln:statistic}
\If {$Z \leq \frac{3}{2}m_2\ve^2$}
\State \Return \accept
\Else 
\State \Return \reject
\EndIf 
\end{algorithmic}
\end{algorithm}

We note that the sample and time complexity are both $O(\sqrt{n}/\ve^2)$.
We draw $m_1 + m_2 = \Theta(\sqrt{n}/\ve^2)$ samples total.
All steps of the algorithm only involve inspecting domain elements where a sample falls, and it runs linearly in the number of such elements.
Indeed, Step~\ref{ln:statistic} of the algorithm is written in an unusual way in order to ensure the running time of the algorithm is linear.

We first analyze the test in Step \ref{ln:light-test} of the algorithm.
Folklore results state that with probability at least $99/100$, this preliminary test will reject any $p$ with $p(\mathcal{\bar A}) \geq c_2 \ve^2$, it will not reject any $p$ with $p(\mathcal{\bar A}) \leq \frac{c_2}{2} \ve^2$, and behavior for any other $p$ is arbitrary.
Condition on the event the test does not reject for the remainder of the proof.
Note that since both thresholds here are $\Theta(\ve^2)$, it only requires $m_1 = \Theta(1/\ve^2)$ samples, rather than the ``non-extreme'' regime, where we would require $\Theta(1/\ve^4)$ samples.

\begin{remark}
We informally refer to this ``extreme'' versus ``non-extreme'' regime in distribution testing.
To give an example of what we mean in these two cases, consider distinguishing $Ber(1/2)$ from $Ber(1/2 + \ve)$.
The complexity of this problem is $\Theta(1/\ve^2)$, and we consider this to be in the non-extreme regime.
On the other hand, distinguishing $Ber(\ve)$ from $Ber(2\ve)$ has a sample complexity of $\Theta(1/\ve)$, and we consider this to be in the extreme regime.
\end{remark}

We justify that any $p$ which may be rejected in Step \ref{ln:early-reject} (i.e., any $p$ such that $p(\mathcal{\bar A}) > \frac{c_2}{2} \ve^2$) has the property that $\dxs(p,q) > \ve^2$ (in other words, we do not wrongfully reject any $p$).

Consider a $p$ such that $p(\mathcal{\bar A}) \geq \frac{c_2}{2}\ve^2$.
Note that $\dxs(p, q) \geq \dxs(p_\mathcal{\bar A}, q_\mathcal{\bar A})$, which we lower bound as follows:
\begin{align*}
\dxs(p_\mathcal{\bar A}, q_\mathcal{\bar A})
&= \sum_{i \in \mathcal{\bar A}} \frac{(p_i - q_i)^2}{q_i} \\
&\geq \frac{n}{c_1 \ve^2} \sum_{i \in \mathcal{\bar A}} (p_i - q_i)^2 \\
&\geq \frac{n}{c_1 \ve^2} \cdot \frac{1}{n} \left(\sum_{i \in \mathcal{\bar A}} (p_i - q_i) \right)^2  \\
&\geq \frac{n}{c_1 \ve^2} \frac{\ve^4\left(\frac{c_2}{2} - c_1\right)^2}{n} \\
&= \frac{\left(\frac{c_2}{2} - c_1\right)^2}{c_1}\ve^2
\end{align*}
The first inequality is by the definition of $\mathcal{\bar A}$, the second is by Cauchy-Schwarz, and the third is since $p(\mathcal{\bar A}) \geq \frac{c_2}{2}\ve^2$ and $q(\mathcal{\bar A}) \leq c_1\ve^2$.
By our setting of $c_1$ and $c_2$, this implies that $\dxs(p, q) > \ve^2$, and we are not rejecting any $p$ which should be accepted.

For the remainder of the proof, we will implicitly assume that $p(\mathcal{\bar A}) \leq c_2 \ve^2$.


Let
$$Z' = \sum_{i \in \mathcal{A}} \frac{(N_i - m_2 q_i)^2 - N_i}{m_2q_i}.$$

Note that the statistic $Z$ can be rewritten as follows:
\begin{align*}
Z &= \sum_{i \in S' \cap \mathcal{A}} \frac{(N_i - m_2q_i)^2 - N_i}{m_2q_i} + m_2 (1 - q(S' \cap \mathcal{A})) \\
  &= \sum_{i \in S' \cap \mathcal{A}} \frac{(N_i - m_2q_i)^2 - N_i}{m_2q_i} + \sum_{i \in \mathcal{A} \setminus S'} m_2 q_i + m_2  q(\mathcal{\bar A}) \\
  &= \sum_{i \in S' \cap \mathcal{A}} \frac{(N_i - m_2q_i)^2 - N_i}{m_2q_i} + \sum_{i \in \mathcal{A} \setminus S'} \frac{(N_i - m_2q_i)^2 - N_i}{m_2q_i} + m_2  q(\mathcal{\bar A}) \\
  &= Z' + m_2 q(\mathcal{\bar A})
\end{align*}

We proceed by analyzing $Z'$.
First, note that it has the following expectation and variance:
\begin{align}
\E[Z'] &= m_2 \cdot \sum_{i \in \mathcal{A}} \frac{(p_i - q_i)^2}{q_i} = m_2 \cdot \dxs(p_\mathcal{A}, q_\mathcal{A}) \label{eqn:mean} \\
\Var[Z'] &= \sum_{i \in \mathcal{A}} \left[2\frac{p_i^2}{q_i^2} + 4m_2 \cdot \frac{p_i \cdot (p_i - q_i)^2}{q_i^2}\right] \label{eqn:variance}
\end{align}
These properties are proven in Section A of~\cite{AcharyaDK15}.

We require the following two lemmas, which state that the mean of the statistic is separated in the two cases, and that the variance is bounded.
The proofs largely follow the proofs of two similar lemmas in~\cite{AcharyaDK15}.
\begin{lemma}
\label{lem:means}
If $\dxs(p,q) \leq \ve^2$, then $\E[Z'] \leq m_2 \ve^2$. 
If $\dh(p,q) \geq \ve$, then $\E[Z'] \geq (2 - c_1 - c_2)m_2 \ve^2$.
\end{lemma}
\begin{proof}
The former case is immediate from (\ref{eqn:mean}).

For the latter case, note that
$$\dh^2(p,q) = \dh^2(p_\mathcal{A}, q_\mathcal{A}) + \dh^2(p_\mathcal{\bar A}, q_\mathcal{\bar A}).$$
We upper bound the latter term as follows:
\begin{align*}
\dh^2(p_\mathcal{\bar A}, q_\mathcal{\bar A}) 
&\leq \dtv(p_\mathcal{\bar A}, q_\mathcal{\bar A}) \\
&= \frac12 \sum_{i \in \mathcal{\bar A}} |p_i - q_i| \\
&\leq \frac12 \left(p(\mathcal{\bar A}) + q(\mathcal{\bar A})\right) \\
&\leq \left(\frac{c_1 + c_2}{2}\right)\ve^2 \\
\end{align*}
The first inequality is from Proposition \ref{prop:distanceinequalities}, and the third inequality is from our prior condition that $p(\mathcal{\bar A}) \leq c_2 \ve^2$.

Since $\dh^2(p,q) \geq \ve^2$, this implies $\dh^2(p_\mathcal{A}, q_\mathcal{A}) \geq \left(1 - \frac{c_1 + c_2}{2}\right)\ve^2$.
Proposition \ref{prop:distanceinequalities} further implies that $\dxs(p_\mathcal{A}, q_\mathcal{A}) \geq \left(2 - c_1 - c_2\right)\ve^2$.
The lemma follows from (\ref{eqn:mean}).
\end{proof}


\begin{lemma}
\label{lem:vars}
If $\dxs(p,q) \leq \ve^2$, then $\Var[Z'] = O(m_2^2 \ve^4)$. 
If $\dh(p,q) \geq \ve$, then $\Var[Z'] \leq O(\E[Z']^2)$.
The constant in both expressions can be made arbitrarily small with the choice of the constant $C$.
\end{lemma}
\begin{proof}
We bound the terms of (\ref{eqn:variance}) separately, starting with the first.

\begin{align}
2\sum_{i \in \mathcal{A}} \frac{p_i^2}{q_i^2} &= 2\sum_{i \in \mathcal{A}} \left(\frac{(p_i - q_i)^2}{q_i^2} + \frac{2p_iq_i - q_i^2}{q_i^2}\right) \nonumber \\
                                             &= 2\sum_{i \in \mathcal{A}} \left(\frac{(p_i - q_i)^2}{q_i^2} + \frac{2q_i(p_i - q_i) + q_i^2}{q_i^2}\right) \nonumber\\
                                             &\leq 2n + 2\sum_{i \in \mathcal{A}} \left(\frac{(p_i - q_i)^2}{q_i^2} + 2\frac{(p_i - q_i)}{q_i}\right) \nonumber\\
                                             &\leq 4n + 4\sum_{i \in \mathcal{A}} \frac{(p_i - q_i)^2}{q_i^2} \nonumber\\
                                             &\leq 4n + \frac{4n}{c_1\ve^2} \sum_{i \in \mathcal{A}} \frac{(p_i - q_i)^2}{q_i}\nonumber\\
                                             &= 4n + \frac{4n}{c_1\ve^2}\frac{E[Z']}{m_2} \nonumber\\
                                             &\leq 4n + \frac{4}{c_1C}\sqrt{n} E[Z']\label{eq:first-var-term-in}
\end{align}
The second inequality is the AM-GM inequality, the third inequality uses that $q_i \geq \frac{c_1\ve^2}{n}$ for all $i \in \mathcal{A}$, the last equality uses \eqref{eqn:mean}, and the final inequality substitutes a value $m_2 \geq C\frac{\sqrt{n}}{\ve^2}$.

The second term can be similarly bounded:
\begin{align*}
4m_2 \sum_{i \in \mathcal{A}} \frac{p_i(p_i - q_i)^2}{q_i^2} &\leq 4m_2 \left(\sum_{i \in \mathcal{A}} \frac{p_i^2}{q_i^2}\right)^{1/2}\left(\sum_{i \in \mathcal{A}} \frac{(p_i - q_i)^4}{q_i^2}\right)^{1/2} \\
                                                          &\leq 4m_2 \left(4n + \frac{4}{c_1C}\sqrt{n} E[Z'] \right)^{1/2}\left(\sum_{i \in \mathcal{A}} \frac{(p_i - q_i)^4}{q_i^2}\right)^{1/2} \\
                                                          &\leq 4m_2 \left(2\sqrt{n} + \frac{2}{\sqrt{c_1C}}n^{1/4} E[Z']^{1/2}\right)\left(\sum_{i \in \mathcal{A}} \frac{(p_i - q_i)^2}{q_i}\right) \\
                                                          &= \left(8\sqrt{n} + \frac{8}{\sqrt{c_1C}}n^{1/4} E[Z']^{1/2}\right)E[Z'].
\end{align*}
The first inequality is Cauchy-Schwarz, the second inequality uses (\ref{eq:first-var-term-in}), the third inequality uses the monotonicity of the $\ell_p$ norms, and the equality uses~\eqref{eqn:mean}.

Combining the two terms, we get
$$\Var[Z'] \leq 4n + \left(8 + \frac{4}{c_1C}\right)\sqrt{n} \E[Z']  + \frac{8}{\sqrt{c_1C}}n^{1/4} \E[Z']^{3/2}  .$$

We now consider the two cases in the statement of our lemma.
\begin{itemize}
\item
When $\dxs(p,q) \leq \ve^2$, we know from Lemma~\ref{lem:means} that $\E[Z'] \leq m_2 \ve^2$. 
Combined with a choice of $m_2 \geq C \frac{\sqrt{n}}{\ve^2}$ and the above expression for the variance, this gives:
\begin{align*}
\Var[Z']
& \leq \frac{4}{C^2}m_2^2\ve^4 + \left(\frac{8}{C} + \frac{4}{c_1C^2}\right)m_2^2 \ve^4+ \frac{8}{C\sqrt{c_1}}m_2^2 \ve^4 \\
& = \left(\frac{8}{C} + \frac{8}{C\sqrt{c_1}} + \frac{4}{C^2} + \frac{4}{c_1C^2} \right)m_2^2 \ve^4 = O(m_2^2 \ve^4).
\end{align*}

\item When $\dh(p,q) \geq \ve$, Lemma~\ref{lem:means} and  $m_2 \geq C\frac{\sqrt{n}}{\ve^2}$ give:
$$\E[Z'] \geq (2 - c_1 - c_2) m_2 \ve^2 \geq C(2 - c_1 - c_2) \sqrt{n}.$$

Similar to before, combining this with our expression for the variance we get:
\begin{align*}
\Var[Z']
&\leq \left(\frac{8}{C(2 -c_1 - c_2)} + \frac{8}{C\sqrt{c_1 (2 - c_1 - c_2)}} +  \frac{4}{C^2(2-c_1-c_2)^2} + \frac{4}{C^2c_1(2 - c_1 - c_2)} \right) \E[Z']^2 \\
&= O(\E[Z']^2).\qedhere
\end{align*}
\end{itemize}
\end{proof}

To conclude the proof, we consider the two cases.
\begin{itemize}
\item Suppose $\dxs(p,q) \leq \ve^2$. 
By Lemma~\ref{lem:means} and the definition of $\mathcal{A}$, we have that $\E[Z] \leq (1 + c_1)m_2\ve^2$. 
By Lemma~\ref{lem:vars}, $\Var[Z] = O(m_2^2\ve^4)$.
Therefore, for constant $C$ sufficiently large, Chebyshev's inequality implies $\Pr(Z > \frac32 m_2 \ve^2) \leq 1/10$.
\item
Suppose $\dh(p,q) \geq \ve$.
By Lemma~\ref{lem:means}, we have that $\E[Z'] \geq (2 - c_1 - c_2)m_2\ve^2$. 
By Lemma~\ref{lem:vars}, $\Var[Z'] = O(\E[Z']^2)$.
Therefore, for constant $C$ sufficiently large, Chebyshev's inequality implies $\Pr(Z' < \frac32 m_2 \ve^2) \leq 1/10$.
Since $Z \geq Z'$, $\Pr(Z < \frac32 m_2 \ve^2) \leq 1/10$ as well.
\end{itemize}

\subsection{Identity Testing with $\ell_2$ Tolerance}
\label{sec:id-lt}
In this section, we sketch the algorithms required to achieve $\ell_2$ tolerance for identity testing.
Since the algorithms and analysis are very similar to those of Algorithm 1 of~\cite{AcharyaDK15} and Algorithm~\ref{alg:testing}, the full details are omitted.

First, we prove Theorem~\ref{thm:ones-tv}.
The algorithm is Algorithm 1 of~\cite{AcharyaDK15}, but instead of testing on $p$ and $q$, we instead test on $p^{+\frac12}$ and $q^{+\frac12}$, as defined in Proposition~\ref{prop:mixing}.
By this proposition, this operation preserves total variation and $\ell_2$ distance, up to a factor of $2$, and also makes it so that the minimum probability element of $q^{+\frac12}$ is at least $1/2n$. 
In the case where $\dlt(p,q) \leq \frac{\ve}{\sqrt{n}}$, we have the following upper bound on $\E[Z]$:
$$\E[Z'] = m \sum_{i \in \mathcal{\bar A}} \frac{(p_i - q_i)^2}{q_i} \leq O\left( m \cdot n \cdot \dlt^2(p,q)\right) \leq O(m_2 \ve^2).$$
This is the same bound as in Lemma 2 of~\cite{AcharyaDK15}.
The rest of the analysis follows identically to that of Algorithm 1 of~\cite{AcharyaDK15}, giving us Theorem~\ref{thm:ones-tv}.

Next, we prove Theorem~\ref{thm:ones-h}.
We observe that Algorithm~\ref{alg:testing} as stated can be considered as $\ell_2$-tolerant instead of $\chi^2$-tolerant, if desired.
First, we do not wrongfully reject any $p$ (i.e., those with $\dlt(p,q) \leq \frac{\ve^2}{\sqrt{n}}$) in Step~\ref{ln:early-reject}.
This is because we reject in this step if there is $\geq \Omega(\ve^2)$ total variation distance between $p$ and $q$ (witnessed by the set $\mathcal{\bar A}$), which implies that $p$ and $q$ are far in $\ell_2$-distance by Proposition~\ref{prop:ltinequalities}.
It remains to prove an upper bound on $\E[Z']$ in the case where $\dlt(p,q) \leq \frac{\ve^2}{\sqrt{n}}$.
$$\E[Z'] = m_2 \dxs(p,q) = m_2 \sum_{i \in \mathcal{\bar A}} \frac{(p_i - q_i)^2}{q_i} \leq O\left( m_2 \cdot \left(\frac{n}{\ve^2}\right)\cdot \dlt^2(p,q)\right) \leq O(m_2 \ve^2).$$
We note that this is the same bound as in Lemma~\ref{lem:means}.
With this bound on the mean, the rest of the analysis is identical to that of Theorem~\ref{thm:ones-csq-h}, giving us Theorem~\ref{thm:ones-h}.

\section{Upper Bounds for Equivalence Testing}
\label{sec:twos-ub}
In this section, we prove the following theorems for equivalence testing.

\begin{theorem}\label{thm:twos-tv}
There exists an algorithm for equivalence testing between $p$ and $q$ distinguishing the cases:
\begin{itemize}
\item $\dlt(p,q) \leq \frac{\ve}{2\sqrt{n}}$ 
\item $\dtv(p,q) \geq \ve$
\end{itemize}
The algorithm uses $O\left(\max\left\{\frac{n^{2/3}}{\ve^{4/3}}, \frac{n^{1/2}}{\ve^2}\right\}\right)$ samples.
\end{theorem}

\begin{theorem}\label{thm:twos-h}
There exists an algorithm for equivalence testing between $p$ and $q$ distinguishing the cases:
\begin{itemize}
\item $\dlt(p,q) \leq \frac{\ve^2}{32\sqrt{n}}$ 
\item $\dh(p,q) \geq \ve$
\end{itemize}
The algorithm uses $O\left(\min\left\{\frac{n^{2/3}}{\ve^{8/3}}, \frac{n^{3/4}}{\ve^2}\right\}\right)$ samples.
\end{theorem}



Consider drawing~$\mathrm{Poisson}(m)$ samples from two unknown distributions $p = (p_1, \ldots, p_n)$ and $q = (q_1, \ldots, q_n)$.
Given the resulting histograms~$\bX$ and~$\bY$, \cite{ChanDVV14} define the following statistic:
\begin{equation}\label{eq:hel-statistic}
\bZ = \sum_{i=1}^n \frac{(\bX_i - \bY_i)^2 - \bX_i - \bY_i}{\bX_i + \bY_i}.
\end{equation}
This can be viewed as a modification to the empirical triangle distance applied to~$\bX$ and~$\bY$.
Both of our equivalence testing upper bounds will be obtained by appropriate thresholding of the statistic $\bZ$.

The organization of this section is as follows.
In Section~\ref{sec:eq-prelim}, we prove some basic properties of $\bZ$.
In Section~\ref{sec:eq-tv}, we prove Theorem~\ref{thm:twos-tv}.
In Section~\ref{sec:eq-h}, we prove Theorem~\ref{thm:twos-h}.

\subsection{Some facts about $\mathbf Z$}
\label{sec:eq-prelim}
Chan et al.~\cite{ChanDVV14} give the following expressions for the mean and variance of~$\bZ$.

\begin{proposition}[\cite{ChanDVV14}]\label{prop:mean-var}
Consider the function
\begin{equation*}
f(x) = \left(1 - \frac{1- e^{-x}}{x}\right).
\end{equation*}
Then for any subset $A\subseteq [n]$,
\begin{equation}\label{eq:z-mean}
\E[\bZ_A] = \sum_{i \in A} \frac{(p_i-q_i)^2}{p_i+q_i} m \cdot f(m(p_i + q_i)).
\end{equation}
As a result, $\bZ$ is mean-zero when $p= q$.  Furthermore,
\begin{equation*}
\Var[\bZ] \leq 2 \min\{m, n\} + \sum_{i=1}^n 5m \frac{(p_i - q_i)^2}{p_i + q_i}.
\end{equation*}
\end{proposition}
\noindent
Applying Proposition~\ref{prp:didnt-know-it-was-hellinger}, we immediately have the following corollary.
\begin{corollary}\label{cor:var-hel}
$\displaystyle
\Var[\bZ] \leq 2 \min\{m, n\} +  20m \dh(p,q)^2.
$
\end{corollary}

Without the corrective factor of $f(m(p_i + q_i))$,
Equation~\eqref{eq:z-mean} would just be~$m$ times the triangle distance between~$p$ and~$q$.
Our goal then is to understand the function $f(x)$ and how it affects this quantity.
Aside from the removable discontinuity at $x=0$, $f$ is a monotonically increasing function,
and for $x > 0$, it is strictly bounded between~$0$ and~$1$.
Furthermore, for~$x > 0$ there are roughly two ``regimes" that $f(x)$ exhibits:
when $x < 1$, where $f(x)$ is well-approximated by $x/2$,
and when $x \geq 1$, where $f(x)$ is ``morally the constant one,'' slowly increasing from~$e^{-1}$ to~$1$.
In fact, we have the following explicit bound on~$f(x)$.
\begin{fact}\label{fact:f-upper}
For all $x > 0$, $f(x) \leq \min\{1, x\}.$
\end{fact}
\noindent
In terms of $f(m(p_i + q_i))$, these regimes correspond to whether $p_i + q_i$ is less than or greater than~$\frac{1}{m}$.
Hence, the expression for the mean of~$\bZ$ (i.e.\ Equation~\eqref{eq:z-mean} for $A = [n]$)
splits in two: those terms for ``large" $p_i + q_i$ look roughly like the triangle distance (times~$m$),
and those terms for ``small" $p_i + q_i$ look roughly like the $\ell_2^2$ distance (times~$m^2$).
This is why we have given ourselves the flexibility to consider subsets~$A$ of the domain.

We will now prove several upper and lower bounds on $\E[\bZ_A]$,
based in part on whether we will apply them in the large or small $p_i + q_i$ regime.
Let us begin with a pair of upper bounds.

\begin{proposition}\label{prop:well-conditioned-upper}
Suppose for every $i \in A$, $p_i + q_i \geq \delta$. Then
\begin{equation*}
\E[\bZ_A] \leq \frac{m}{\delta} \dlt^2(p_A, q_A).
\end{equation*}
\end{proposition}
\begin{proof}
Because $f(x) \leq 1$ for all $x > 0$,
\begin{equation*}
\E[\bZ_A]
= \sum_{i \in A} \frac{(p_i-q_i)^2}{p_i+q_i} m \cdot f(m(p_i + q_i))
\leq  \sum_{i \in A} \frac{(p_i-q_i)^2}{p_i+q_i} m
\leq \frac{m}{\delta} \sum_{i \in A} (p_i-q_i)^2
=  \frac{m}{\delta} \dlt^2(p_A, q_A).\qedhere
\end{equation*}
\end{proof}

\begin{proposition}\label{prop:general-upper}
$\displaystyle
\E[\bZ] \leq m^2 \dlt^2(p, q).
$
\end{proposition}
\begin{proof}
Let $L$ be the set of $i$ such that $m(p_i + q_i) \geq 1$.
Then $\E[\bZ] = \E[\bZ_L] + \E[\bZ_{\overline{L}}]$, and by Proposition~\ref{prop:well-conditioned-upper},
$\E[\bZ_L] \leq m^2 \dlt^2(p_L, q_L)$.
On the other hand, by Fact~\ref{fact:f-upper}, $f(x) \leq x$, and therefore
\begin{equation*}
\E[\bZ_{\overline{L}}]
= \sum_{i \in \overline{L}} \frac{(p_i-q_i)^2}{p_i+q_i} m \cdot f(m(p_i + q_i))
\leq \sum_{i \in \overline{L}} (p_i-q_i)^2 m^2
= m^2 \dlt^2(p_{\overline{L}}, q_{\overline{L}}).
\end{equation*}
The proof is completed by noting that $\dlt^2(p_L, q_L) + \dlt^2(p_{\overline{L}}, q_{\overline{L}}) = \dlt^2(p, q).$
\end{proof}

Now we give a pair of lower bounds.

\begin{proposition}\label{prop:pretty-much-a-trivial-lower-bound-i-dont-know-what-to-tell-you}
Suppose for every $i \in A$, $m(p_i + q_i) \geq 1$.  Then
\begin{equation*}
\E[\bZ_A] \geq \frac{2m}{3} \dh^2(p_A,q_A).
\end{equation*}
\end{proposition}
\begin{proof}
Because $f(x)$ is monotonically increasing and $f(1) = 1/e$,
\begin{equation*}
\E[\bZ_A]
= m \sum_{i \in A} \frac{(p_i-q_i)^2}{p_i+q_i} f(m(p_i+q_i))
\geq m \sum_{i \in A} \frac{(p_i-q_i)^2}{p_i+q_i} f(1)
\geq \frac{2m}{e}\dh^2(p_A,q_A),
\end{equation*}
where the first step is by Proposition~\ref{prop:mean-var} and the last is by Proposition~\ref{prp:didnt-know-it-was-hellinger}.
The result follows from $e \leq 3$.
\end{proof}

The next proposition is essentially the second half of the proof of Lemma~$4$ from~\cite{ChanDVV14}.

\begin{proposition}\label{prop:their-bound}
For any subset~$A$,
\begin{equation*}
\E[\bZ_A] \geq \left(\frac{4m^2}{2|A| + m\cdot(p(A) + q(A))}\right)\cdot \dtv^2(p_A,q_A),
\end{equation*}
where we write $p(A) = \sum_{i \in A} p(i)$ and likewise for~$q(A)$.
\end{proposition}
\begin{proof}
Consider the function $g(x) = x f(x)^{-1}$.
Then $g(x) \leq 2+x$ for nonnegative~$x$.
Furthermore,
\begin{equation*}
\frac{(p_i - q_i)^2}{g(m(p_i + q_i))} = \frac{(p_i-q_i)^2}{m(p_i+q_i)} \left(1 - \frac{1-e^{-m(p_i+q_i)}}{m(p_i + q_i)}\right),
\end{equation*}
which, from Proposition~\ref{prop:mean-var}, is $\frac{1}{m^2} \cdot \E[\bZ_{\{i\}}]$.
As a result,
\begin{multline*}
\dtv^2(p_A,q_A)
= \frac14 \left(\sum_{i \in A} |p_i - q_i|\right)^2
= \frac14 \left(\sum_{i \in A} |p_i - q_i|\cdot \frac{\sqrt{g(m(p_i+q_i))}}{\sqrt{g(m(p_i+q_i))}}\right)^2\\
\leq \frac14\left(\sum_{i \in A} \frac{(p_i - q_i)^2}{g(m(p_i + q_i))}\right) \cdot \left(\sum_{i \in A} g(m(p_i + q_i))\right)
\leq \frac{1}{4m^2} \cdot \E[\bZ_A] \cdot(2|A| + m\cdot (p(A) + q(A))),
\end{multline*}
where the first inequality is Cauchy-Schwarz. Rearranging finishes the proof.
\end{proof}


\subsection{Equivalence Testing with Total Variation Distance}
\label{sec:eq-tv}
In this section, we prove Theorem~\ref{thm:twos-tv}. 
We will take the number of samples to be 
\begin{equation}\label{eq:l1-max}
m = \max\left\{C\cdot \frac{n^{2/3}}{\epsilon^{4/3}}, C^{3/2}\cdot \frac{n^{1/2}}{\epsilon^2}\right\},
\end{equation}
where~$C$ is some constant which can be taken to be~$10^{10}$.
 
Rather than drawing samples from~$p$ or~$q$,
our algorithm draws samples from~$p^{+1/2}$ and~$q^{+1/2}$.
By Proposition~\ref{prop:mixing}, we have the following guarantees in the two cases:
\begin{equation*}
\text{(Case 1):}~\dlt(p^{+1/2},q^{+1/2}) \leq \frac{\epsilon}{4 \sqrt{n}},
\qquad
\text{(Case 2):}~\dtv(p^{+1/2},q^{+1/2}) \geq \frac{\epsilon}{2}.
\end{equation*}
Furthermore, for any $i \in [n]$, we know the $i$-th coordinates of $p^{+1/2}$ and~$q^{+1/2}$ are both at least~$\frac{1}{2n}$.
Henceforth, we will write~$p'$ and~$q'$ for~$p^{+1/2}$ and~$q^{+1/2}$, respectively.

In Case 1, if we apply Proposition~\ref{prop:well-conditioned-upper} with $A = [n]$ and $\delta = \frac{1}{n}$
and Proposition~\ref{prop:general-upper},
\begin{equation*}
\E[\bZ]
\leq \min\{m^2, mn\} \cdot \dlt^2(p',q')
\leq \min\{m^2, mn\} \cdot \frac{\epsilon^2}{16 n}
\leq \frac{m^2}{4(2m + 2n)} \cdot \epsilon^2.
\end{equation*}
On the other hand, in Case 2, applying Proposition~\ref{prop:their-bound} with $A = [n]$,
\begin{equation*}
\E[\bZ] \geq \frac{4m^2}{2m + 2n} \cdot \dtv(p',q')^2 \geq \frac{m^2}{2m + 2n} \cdot \epsilon^2.
\end{equation*}
Our algorithm therefore thresholds~$\bZ$ on the value $\frac{5 m^2}{8(2m +2n)} \epsilon^2$,
outputting ``close" if it's below this value and ``far" otherwise.

The two bounds in~\eqref{eq:l1-max} meet when $C^3 \epsilon^{-4} = n$,
which is exactly when $m = n$.
When $m \leq n$, the first bound applies, and when $m > n$ the second bound applies.
As a result, we will split our analysis into the two cases.

\begin{lemma}
The tester succeeds in the $m \leq n$ case of Theorem~\ref{thm:twos-tv}.
\end{lemma}
\begin{proof}
By Corollary~\ref{cor:var-hel}
\begin{equation*}
\Var[\bZ] \leq 2 \min\{m, n\} +  20m \dh(p',q')^2
\leq 22m,
\end{equation*}
where we used the fact that $\dh(p',q') \leq 1$.
In Case 1, by Chebyshev's inequality,
\begin{equation*}
\Pr\left[\bZ \geq \frac{5 m^2}{8(2m +2n)} \epsilon^2\right]
\leq \frac{\Var[\bZ]}{\left(\frac{3m^2}{8(2m +2n)} \epsilon^2\right)^2}
= O\left(\frac{m}{\frac{m^4}{n^2} \epsilon^4}\right)
= O\left(\frac{n^2}{m^3 \epsilon^4}\right).
\end{equation*}
In Case 2,
\begin{equation*}
\Pr\left[\bZ \leq \frac{5 m^2}{8(2m +2n)} \epsilon^2\right]
\leq \frac{64 \Var[\bZ]}{9\E[\bZ]^2}
= O\left(\frac{m}{\frac{m^4}{n^2} \epsilon^4}\right)
= O\left(\frac{n^2}{m^3 \epsilon^4}\right).
\end{equation*}
Both of these bounds can be made arbitrarily small constants by setting~$C$ sufficiently large.
\end{proof}

\begin{lemma}
The tester succeeds in the $m \geq n$ case of Theorem~\ref{thm:twos-tv}.
\end{lemma}
\begin{proof}
We first consider Case 1.
By Proposition~\ref{prop:mean-var},
\begin{equation*}
\Var[\bZ]
\leq 2 \min\{m, n\} + \sum_{i=1}^n 5m \frac{(p_i' - q_i')^2}{p_i' + q_i'}
\leq 2 n + 5 m n \dlt^2(p',q')
\leq 2 n + \tfrac{5}{16} m \epsilon^2.
\end{equation*}
Then, we have that
\begin{equation*}
\Pr\left[\bZ \geq \frac{5 m^2}{8(2m +2n)} \epsilon^2\right]
\leq \frac{\Var[\bZ]}{\left(\frac{3m^2}{8(2m +2n)} \epsilon^2\right)^2}
= O\left(\frac{n}{m^2 \epsilon^4} + \frac{m\epsilon^2}{m^2 \epsilon^4}\right)
= O\left(\frac{n}{m^2 \epsilon^4} + \frac{1}{m \epsilon^2}\right).
\end{equation*}
Next, we focus on Case 2.
Write $L$ for the set of $i \in [n]$ such that $m(p_i' + q_i') \geq 1$.
Then $\dh^2(p_{\overline{L}}',q_{\overline{L}}') \leq \frac12 \sum_{i \in \overline{L}} (p_i' + q_i') \leq n/2m$.
As a result, 
by Corollary~\ref{cor:var-hel}
\begin{equation*}
\Var[\bZ] \leq 2 \min\{m, n\} +  20m \dh^2(p',q')
\leq 12 n + 20m \dh^2(p_L',q_L'). 
\end{equation*}
By Proposition~\ref{prop:pretty-much-a-trivial-lower-bound-i-dont-know-what-to-tell-you},
$\E[\bZ] \geq \frac{2m}{3} \dh^2(p_L',q_L')$.
Hence, 
\begin{align*}
\Pr\left[\bZ \leq \frac{5 m^2}{8(2m +2n)} \epsilon^2\right]
&\leq \frac{64 \Var[\bZ]}{9\E[\bZ]^2}
= O\left(\frac{n}{\E[\bZ]^2} + \frac{m \dh^2(p_L',q_L')}{\E[\bZ]^2}\right)\\
&= O\left(\frac{n}{\E[\bZ]^2} + \frac{1}{\E[\bZ]}\right)
= O\left(\frac{n}{m^2 \epsilon^4} + \frac{1}{m \epsilon^2}\right).
\end{align*}
Both of these bounds can be made arbitrarily small constants by setting~$C$ sufficiently large.
\end{proof}



\subsection{Equivalence Testing with Hellinger Distance}
\label{sec:eq-h}

In this section, we prove Theorem~\ref{thm:twos-h}. 
We will take the number of samples to be 
\begin{equation*}
m = \min\left\{C\cdot \frac{n^{2/3}}{\epsilon^{8/3}}, C^{3/4}\cdot \frac{n^{3/4}}{\epsilon^2}\right\},
\end{equation*}
where~$C$ is some constant which can be taken to be~$10^{10}$.

Rather than drawing samples from~$p$ or~$q$,
our algorithm draws samples from~$p^{+\delta}$ and~$q^{+\delta}$
for $\delta = \epsilon^2/32$.
By Proposition~\ref{prop:mixing}, we have the following guarantees in the two cases:
\begin{equation*}
\text{(Case 1):}~\dlt(p,q) \leq \frac{\epsilon^2}{32 \sqrt{n}},
\qquad
\text{(Case 2):}~\dh(p,q) \geq \frac{1}{2} \epsilon.
\end{equation*}
Furthermore, for any $i \in [n]$, we know the $i$-th coordinates of $p^{+\delta}$ and~$q^{+\delta}$ are both at least~$\frac{\epsilon^2}{32n}$.
Henceforth, we will write~$p'$ and~$q'$ for~$p^{+\delta}$ and~$q^{+\delta}$, respectively.


The two bounds meet when $C^{3/4}\epsilon^{-2} = n^{1/4}$,
which is exactly when $m = n$.
When $m \leq n$, the first bound applies, and when $m > n$ the second bound applies.
As a result, we will split our analysis into the two cases.

\begin{lemma}
The tester succeeds in the $m \leq n$ case of Theorem~\ref{thm:twos-h}.
\end{lemma}
\begin{proof}
In Case 1, if we apply Proposition~\ref{prop:general-upper},
\begin{equation*}
\E[\bZ]
\leq m^2 \cdot  \dlt^2(p',q')
\leq \frac{m^2 \epsilon^4}{32^2 n}.
\end{equation*}
On the other hand, in Case 2, applying Proposition~\ref{prop:their-bound} with $A = [n]$,
\begin{equation*}
\E[\bZ]
\geq \left(\frac{4m^2}{2n+2m}\right) \cdot \dtv(p',q')^2
\geq \left(\frac{4m^2}{2n+2m}\right) \cdot \dh(p',q')^4
\geq \frac{m^2\epsilon^4}{16n}.
\end{equation*}
Our algorithm therefore thresholds~$\bZ$ on the value $ \frac{m^2\epsilon^4}{128n}$,
outputting ``close" if it's below this value and ``far" otherwise.

By Corollary~\ref{cor:var-hel}
\begin{equation*}
\Var[\bZ] \leq 2 \min\{m, n\} +  20m \dh(p',q')^2
\leq 22m,
\end{equation*}
where we used the fact that $\dh(p',q') \leq 1$.
In Case 1,
\begin{equation*}
\Pr\left[\bZ \geq \frac{m^2\epsilon^4}{128n}\right]
\leq \frac{\Var[\bZ]}{\left(\frac{m^2\epsilon^4}{256n}\right)^2}
= O\left(\frac{m}{\frac{m^4}{n^2} \epsilon^8}\right)
= O\left(\frac{n^2}{m^3 \epsilon^8}\right).
\end{equation*}
In Case 2,
\begin{equation*}
\Pr\left[\bZ \leq \frac{m^2\epsilon^4}{128n}\right]
\leq \frac{64 \Var[\bZ]}{49\E[\bZ]^2}
= O\left(\frac{m}{\frac{m^4}{n^2} \epsilon^8}\right)
= O\left(\frac{n^2}{m^3 \epsilon^8}\right).
\end{equation*}
Both of these bounds can be made arbitrarily small constants by setting~$C$ sufficiently large.
\end{proof}

\begin{lemma}
The tester succeeds in the $m > n$ case of Theorem~\ref{thm:twos-h}.
\end{lemma}
\begin{proof}
In Case 1, if we apply Proposition~\ref{prop:well-conditioned-upper} with $A = [n]$ and $\delta = \frac{\epsilon^2}{16n}$
and Proposition~\ref{prop:general-upper},
\begin{equation*}
\E[\bZ]
\leq \min\left\{m^2, 16\frac{mn}{\epsilon^2}\right\} \cdot \dlt^2(p',q')
\leq \min\left\{m^2, 16\frac{mn}{\epsilon^2}\right\} \cdot \frac{\epsilon^4}{32^2 n}
= \min\left\{\frac{m^2\epsilon^4}{32^2 n}, \frac{m\epsilon^2}{64}\right\}.
\end{equation*}
Case 2 is more complicated.
We will need to define the set of ``large" coordinates
$L = \{i : m (p_i' + q_i') \geq 1\}$
and the set of ``small" coordinates $S = [n] \setminus L$.
Applying Proposition~\ref{prop:their-bound} to~$S$, we have
\begin{equation*}
\E[\bZ_S] \geq \left(\frac{4m^2}{2|S| + m\cdot(p'(S) + q'(S))}\right)\cdot \dtv^2(p_S',q_S')
\geq \frac{4m^2}{3n} \dtv^2(p_S',q_S'),
\end{equation*}
where $m\cdot(p'(S)+q'(S)) \leq n$ by the definition of~$S$.
If we also apply Proposition~\ref{prop:pretty-much-a-trivial-lower-bound-i-dont-know-what-to-tell-you} to~$L$,
we get
\begin{equation*}
\E[\bZ] = \E[\bZ_S] + \E[\bZ_L]
\geq \frac{4m^2}{3n} \dtv^2(p_S',q_S') + \frac{2m}{3} \dh^2(p_L',q_L')
\geq \min\left\{\frac{m^2\epsilon^4}{48n}, \frac{m\epsilon^2}{12}\right\},
\end{equation*}
where the last step follows
because $\dh^2(p_S',q_S') + \dh^2(p_L',q_L') = \dh^2(p',q')$ and $\dtv^2(p_S',q_S') \geq \dh^4(p_S',q_S')$.
As a result, we threshold~$\bZ$ on the value
\begin{equation*}
\frac{1}{2} \cdot \min\left\{\frac{m^2\epsilon^4}{48n}, \frac{m\epsilon^2}{12}\right\},
\end{equation*}
outputting ``close" if it's below this value and ``far" otherwise.

In Case 1, by Proposition~\ref{prop:mean-var},
\begin{equation*}
\Var[\bZ]
\leq 2 \min\{m, n\} + \sum_{i=1}^m 5m \frac{(p_i' - q_i')^2}{p_i' + q_i'}
\leq 2 n + \frac{80 m n}{\epsilon^2} \Vert p' - q' \Vert_2^2
\leq 2 n + \frac{5}{64}m \epsilon^2.
\end{equation*}
Hence, by Chebyshev's inequality,
\begin{multline*}
\Pr\left[\bZ \geq\frac{1}{2} \cdot \min\left\{\frac{m^2\epsilon^4}{48n}, \frac{m\epsilon^2}{12}\right\}\right]
\leq \frac{\Var[\bZ]}{(\frac{1}{8} \cdot \min\left\{\frac{m^2\epsilon^4}{48n}, \frac{m\epsilon^2}{12}\right\})^2}\\
\leq O\left(\frac{n}{(\frac{m^2 \epsilon^4}{n})^2} + \frac{n}{(m \epsilon^2)^2}
	+ \frac{m\epsilon^2}{(\frac{m^2 \epsilon^4}{n})^2} + \frac{m\epsilon^2}{(m\epsilon^2)^2}\right)\\
= O\left(\frac{n^3}{m^4 \epsilon^8} + \frac{n}{m^2 \epsilon^4}
	+ \frac{n^2}{m^3 \epsilon^6} + \frac{1}{m\epsilon^2}\right).
\end{multline*}
This can be made an arbitrarily small constant by setting~$C$ sufficiently large.


In Case 2, by Corollary~\ref{cor:var-hel},
\begin{equation}\label{eq:gonna-split}
\Pr\left[\bZ \leq \frac{\E[\bZ]}{2} \right]
\leq \frac{4 \Var[\bZ]}{\E[\bZ]^2}
\leq \frac{8 n + 80 m\dh(p',q')^2}{\E[\bZ]^2}.
\end{equation}
Because $\dh(p',q')^2 = \dh^2(p_S',q_S') + \dh^2(p_L',q_L')$,
either $\dh^2(p_S',q_S')$ or $\dh^2(p_L',q_L')$ is at least $\frac{1}{2}\dh^2(p',q')$.
Suppose that $\dh^2(p_S',q_S') \geq \frac{1}{2}\dh^2(p',q')$.
We note that
\begin{equation*}
m \dh^2(p_S',q_S')
= \frac{m}{2} \sum_{i \in S} (\sqrt{p_i'} - \sqrt{q_i'})^2
\leq \frac{m}{2} \sum_{i \in S} |p_i' + q_i'|
\leq \frac{n}{2},
\end{equation*}
by the definition of~$S$.
Thus,
\begin{equation*}
\eqref{eq:gonna-split}
\leq \frac{8n + 160 m \dh^2(p_S',q_S')}{(\frac{4m^2}{3n}\dtv^2(p_S',q_S'))^2}
\leq \frac{88n}{(\frac{4m^2}{3n}\dtv^2(p_S',q_S'))^2}
= O\left(\frac{n^3}{m^4 \dtv^4(p_S',q_S')}\right)
\leq O\left(\frac{n^3}{m^4 \epsilon^8}\right),
\end{equation*}
where the last step used the fact that $\dtv(p_S',q_S') \geq \dh^2(p_S',q_S') \geq \frac{1}{2}\dh^2(p',q') \geq \frac{1}{2}\epsilon^2$.

In the case when $\dh^2(p_L',q_L') \geq \frac{1}{2} \dh^2(p',q')$,
\begin{equation*}
\eqref{eq:gonna-split}
\leq \frac{8n + 160 m \dh^2(p_L',q_L')}{(\frac{2m}{3} \dh^2(p_L',q_L'))^2}
= O\left(\frac{n}{m^2 \dh^4(p_L',q_L')} + \frac{1}{m \dh^2(p_L',q_L')}\right)
\leq O\left(\frac{n}{m^2 \epsilon^4} + \frac{1}{m \epsilon^2}\right).
\end{equation*}
This can be made an arbitrarily small constant by setting~$C$ sufficiently large.
\end{proof}

\section{Upper Bounds Based on Estimation}
\label{sec:est-ub}

We start by showing a simple meta-algorithm -- in short, it says that if a testing problem is well-defined (i.e., has appropriate separation between the cases) and we can estimate one of the distances, it can be converted to a testing algorithm.
\begin{theorem}\label{thm:est-ub}
Suppose there exists an $m(n, \ve)$-sample algorithm which, given sample access to distributions $p$ and $q$ over $[n]$ and parameter $\ve$, estimates some distance $d(p,q)$ up to an additive $\ve$ with probability at least $2/3$.
Consider distances $d_X(\cdot, \cdot), d_Y(\cdot, \cdot)$ and $\ve_1, \ve_2 > 0$ such that $ d_Y(p,q) \geq \ve_2 \rightarrow d_X(p,q) > 3\ve_1/2$ and $d_X(p,q) \leq \ve_1 \rightarrow d_Y(p,q) < 2\ve_2/3$, and $d(\cdot, \cdot)$ is either $d_X(\cdot, \cdot)$ or $d_Y(\cdot, \cdot)$.

Then there exists an algorithm for equivalence testing between $p$ and $q$ distinguishing the cases:
\begin{itemize}
\item $d_X(p,q) \leq \ve_1$;
\item $d_Y(p,q) \geq \ve_2$.
\end{itemize}
The algorithm uses either $m(n, O(\ve_1))$ or $m(n, O(\ve_2))$ samples, depending on whether $d = d_X$ or $d_Y$.
\end{theorem}
\begin{proof}
Suppose that $d = d_X$, the other case follows similarly.
Using the $m(n, \ve_1/4)$ samples, obtain an estimate $\hat \tau$ of $d_X(p,q)$, accurate up to an additive $\ve_1/4$.
If $\hat \tau \leq 5\ve_1/4$, output that $d_X(p,q) \leq \ve_1$, else output that $d_Y(p,q) \geq \ve_2$. 
Conditioning on the correctness of the estimation algorithm, correctness for the case when $d_X(p,q) \leq \ve_1$ is immediate, and correctness for the case when $d_Y(p,q) \geq \ve_2$ follows from the separation between the cases.
\end{proof}

It is folklore that a distribution over $[n]$ can be $\ve$-learned in $\ell_2$-distance with $O(1/\ve^2)$ samples (see, i.e., \cite{ChanDVV14, Waggoner15} for a reference).
By triangle inequality, this implies that we can estimate the $\ell_2$ distance between $p$ and $q$ up to an additive $O(\ve)$ with $O(1/\ve^2)$ samples, leading to the following corollary.

\begin{corollary}\label{cor:l2-est}
There exists an algorithm for equivalence testing between $p$ and $q$ distinguishing the cases:
\begin{itemize}
\item $d(p,q) \leq f(n, \ve)$;
\item $\dlt(p,q) \geq \ve$,
\end{itemize}
where $d(\cdot, \cdot)$ is a distance and $f(n, \ve)$ is such that $\dlt(p,q) \geq \ve \rightarrow d(p,q) \geq 3f(n, \ve)/2$ and $d(p,q) \leq f(n, \ve) \rightarrow \dlt(p,q) \leq 2\ve/3$.  
The algorithm uses $O(1/\ve^2)$ samples.
\end{corollary}

Finally, we note that total variation distance between $p$ and $q$ can be additively estimated up to a constant using $O(n/\log n)$ samples~\cite{LehmannC06, ValiantV11b, JiaoHW16}, leading to the following corollary:
\begin{corollary}\label{cor:tv-est}
For constant $\ve > 0$, there exists an algorithm for equivalence testing between $p$ and $q$ distinguishing the cases:
\begin{itemize}
\item $\dtv(p,q) \leq \ve^2/4$;
\item $\dh(p,q) \geq \ve/\sqrt{2}$.
\end{itemize}
The algorithm uses $O(n/\log n)$ samples.
\end{corollary}

\section{Lower Bounds}
\label{sec:lb}
We start with a simple lower bound, showing that identity testing with respect to KL divergence is impossible.
A similar observation was made in~\cite{BatuFRSW00}.
\begin{theorem}\label{thm:untestable}
No finite sample test can perform identity testing between $p$ and $q$ distinguishing the cases:
\begin{itemize}
\item $p = q$;
\item $\dkl(p,q) \geq \ve^2$.
\end{itemize}
\end{theorem}
\begin{proof}
Simply take $q = (1, 0)$ and let~$p$ be either $(1, 0)$ or $(1-\delta, \delta)$, for~$\delta > 0$ tending to zero.
Then $p = q$ in the first case and $\dkl(p,q) = \infty$ in the second, but distinguishing between these two possibilities for~$p$
takes $\Omega(\delta^{-1})\rightarrow \infty$ samples.
\end{proof}

Next, we prove our lower bound for KL tolerant identity testing.

\begin{theorem}\label{thm:ones-lb}
There exist constants $0 < s < c$, such that any algorithm for identity testing between $p$ and $q$ distinguishing the cases:
\begin{itemize}
\item $\dkl(p,q) \leq s$;
\item $\dtv(p,q) \geq c$;
\end{itemize}
requires $\Omega(n/\log n)$ samples.
\end{theorem}
\begin{proof}
Let $q = (\tfrac{1}{n}, \ldots, \tfrac{1}{n})$ be the uniform distribution.
Let $R(\cdot, \cdot)$ denote the \emph{relative earthmover distance} (see~\cite{ValiantV10a} for the definition).
By Theorem~$1$ of~\cite{ValiantV10a},
for any $\delta < \frac{1}{4}$
there exist sets of distributions~$\calC$ and~$\calF$ (for \emph{close} and \emph{far})
such that:
\begin{itemize}
\item For every $p \in \calC$, $R(p, q) = O(\delta | \log \delta|)$.
\item For every $p \in \calF$ there exists a distribution~$r$ which is uniform over~$n/2$ elements such that $R(p, r) = O(\delta | \log \delta|)$.
\item Distinguishing between $p \in \calC$ and $p \in \calF$ requires $\Omega(\frac{\delta n}{\log(n)})$ samples.
\end{itemize}
Now, if $p \in \calC$ then
\begin{equation*}
\dkl(p,q)
= \sum_{i=1}^n p_i \log\left(\frac{p_i}{1/n}\right)
= \log(n) - H(p)
\leq O(\delta |\log(\delta)|),
\end{equation*}
where $H(p)$ is the Shannon entropy of~$p$,
and here we used the fact that $|H(p) - H(q)| \leq R(p, q)$, which follows from Fact~$5$ of~\cite{ValiantV10a}.
On the other hand, if $q \in \calF$, let~$r$ be the corresponding distribution which is uniform over~$n/2$ elements.
Then
\begin{equation*}
\frac{1}{2}
= \dtv(p,q)
\leq \dtv(q,p) + \dtv(p,r)
\leq \dtv(q,p) + O(\delta | \log \delta|),
\end{equation*}
where we used the triangle inequality
and the fact that $\dtv(p,r) \leq R(p, r)$ (see~\cite{ValiantV10a} page 4).
As a result, if we set~$\delta$ to be some small constant,
$s = O(\delta |\log(\delta)|)$,
and $c = \frac{1}{2} - O(\delta | \log\delta|)$,
then this argument shows that distinguish $\dkl(p,q) \leq s$ versus $\dtv(p,q) \geq c$
requires $\Omega(n/\log n)$ samples.
\end{proof}

Finally, we conclude with our lower bound for $\chi^2$-tolerant equivalence testing.

\begin{theorem}\label{thm:twos-lb}
There exists a constant $\ve > 0$ such that any algorithm for equivalence testing between $p$ and $q$ distinguishing the cases:
\begin{itemize}
\item $\dxs(p,q) \leq \ve^2/4$;
\item $\dtv(p,q) \geq \ve$;
\end{itemize}
requires $\Omega(n/\log n)$ samples.
\end{theorem}
\begin{proof}
We reduce the problem of distinguishing $\dh(p,q) \leq \frac{1}{\sqrt{48}} \epsilon$ from $\dtv(p,q) \geq 3\epsilon$ to this.
Define the distributions
\begin{equation*}
p' = \frac{2}{3} p + \frac{1}{3} q, \qquad q' = \frac{1}{3} p + \frac{2}{3} q.
\end{equation*}
Then $m$ samples from~$p'$ and~$q'$ can be simulated by $m$ samples from~$p$ and~$q$.
Furthermore,
\begin{equation*}
\dh(p',q') \leq \frac{1}{\sqrt{48}} \epsilon, \qquad \dtv(p',q') = \frac{1}{3} \dtv(p,q) \geq \epsilon,
\end{equation*}
where we used the fact that Hellinger distance satisfies the data processing inequality.
But then, in the ``close" case,
\begin{equation*}
\dxs(p',q')
= \sum_{i=1}^n \frac{(p'_i - q'_i)^2}{q'_i}
\leq 3 \sum_{i=1}^n \frac{(p'_i - q'_i)^2}{p'_i + q'_i}
\leq 12 \dh^2(p',q') \leq \frac{1}{4} \epsilon^2,
\end{equation*}
where we used the fact that $p'_i \leq 2q'_i$ and Proposition~\ref{prp:didnt-know-it-was-hellinger}.
Hence, this problem, which requires $\Omega(n/\log n)$ samples (by the relationship between total variation and Hellinger distance, and the lower bound for testing total variation-close versus -far of~\cite{ValiantV10a}), reduces to the problem in the proposition, and so that requires $\Omega(n/\log n)$ samples as well.
\end{proof}

\bibliographystyle{alpha}
\bibliography{biblio}
\appendix
\section{Proof of Proposition~\ref{prop:distanceinequalities}}
\label{sec:distanceinequalities}
Recall that we will prove this for restrictions of probability distributions to subsets of the support -- in other words, we do not assume $\sum_{i \in S} p_i = \sum_{i \in S} q_i = 1$, we only assume that $\sum_{i \in S} p_i \leq 1$ and $\sum_{i \in S} q_i \leq 1$.
\paragraph{$\dh^2(p_S,q_S) \leq \dtv(p_S,q_S):$}
\begin{align*}
\dh^2(p_S,q_S) &= \frac12 \sum_{i \in S} (\sqrt{p_i} - \sqrt{q_i})^2 \\
&\leq \frac12 \sum_{i \in S} |\sqrt{p_i} - \sqrt{q_i}|(\sqrt{p_i} + \sqrt{q_i}) \\
&= \frac12 \sum_{i \in S} |p_i - q_i| \\
&= \dtv(p_S,q_S).
\end{align*}

\paragraph{$\dtv(p_S,q_S) \leq \sqrt{2}\dh(p_S,q_S):$}
\begin{align*}
\dtv^2(p_S,q_S) &= \frac14 \left(\sum_{i \in S} \left|p_i - q_i\right|\right)^2 \\
&= \frac14 \left(\sum_{i \in S} \left|\sqrt{p_i} -\sqrt{q_i}\right|(\sqrt{p_i} + \sqrt{q_i})\right)^2 \\
&\leq \frac14 \left(\sum_{i \in S} \left|\sqrt{p_i} -\sqrt{q_i}\right|^2\right)\left(\sum_{i \in S}(\sqrt{p_i} + \sqrt{q_i})^2\right) \\
&\leq \dh^2(p_S, q_S) \cdot \frac12 \left(\sum_{i \in S}(\sqrt{p_i} + \sqrt{q_i})^2\right) \\
&= \dh^2(p_S, q_S) \cdot \left(\sum_{i \in S}p_i + \sum_{i \in S} q_i - \dh^2(p_S,q_S)\right) \\
&\leq \dh^2(p_S,q_S) \cdot \left(2 - \dh^2(p_S,q_S)\right) \\
&\leq 2 \dh^2(p_S,q_S).
\end{align*}
Taking the square root of both sides gives the result.
The second inequality is Cauchy-Schwarz.

\paragraph{$2\dh^2(p_S, q_S) \leq \sum_{i \in S} (q_i - p_i) + \dkl(p_S, q_S):$}
\begin{align*}
2 \dh^2(p_S, q_S) 
&= \sum_{i \in S} (q_i + p_i) - 2\sum_{i \in S} \sqrt{p_i q_i} \\
&= \sum_{i \in S} (q_i + p_i) - 2\left(\left(\sum_{j \in S} p_j\right)\sum_{i \in S} \frac{p_i}{\sum_{j \in S} p_j} \sqrt{\frac{q_i}{p_i}}\right) \\
&\leq \sum_{i \in S} (q_i + p_i) - 2\left(\left(\sum_{j \in S} p_j\right)\exp\left(\frac12 \sum_{i \in S} \frac{p_i}{\sum_{j \in S} p_j} \log{\frac{q_i}{p_i}}\right)\right) \\
&\leq \sum_{i \in S} (q_i + p_i) - 2\left(\left(\sum_{j \in S} p_j\right)\left(1 + \frac12 \sum_{i \in S} \frac{p_i}{\sum_{j \in S} p_j} \log{\frac{q_i}{p_i}}\right)\right) \\
&= \sum_{i \in S} (q_i - p_i) - \left(\sum_{i \in S} p_i \log{\frac{q_i}{p_i}} \right)\\
&= \sum_{i \in S} (q_i - p_i) + \dkl(p_S, q_S).
\end{align*}
The first inequality is Jensen's, and the second is $1 + x \leq \exp(x)$.

\paragraph{$\dkl(p_S, q_S) \leq \sum_{i \in S} (p_i - q_i) +  \dxs(p_S, q_S):$}
\begin{align*}
\dkl(p_S, q_S) 
&= \left(\sum_{j \in S} p_j\right)\left(\sum_{i \in S} \frac{p_i}{\sum_{j \in S} p_j} \log{\frac{p_i}{q_i}}\right) \\
&\leq \left(\sum_{j \in S} p_j\right)\left(\log {\frac{1}{\sum_{j\in S} p_j}\sum_{i \in S} \frac{p_i^2}{q_i}} \right) \\
&= \left(\sum_{j \in S} p_j\right) \left(\log{ \left(\frac{1}{\sum_{j\in S} p_j} \left(\dxs(p_S, q_S) + 2\sum_{i \in S} p_i - \sum_{i \in S} q_i\right)\right)} \right) \\
&= \left(\sum_{j \in S} p_j\right) \left(\log{  \left(2 + \frac{1}{\sum_{j\in S} p_j} \left(\dxs(p_S, q_S)  - \sum_{i \in S} q_i\right)\right)} \right) \\
&\leq \left(\sum_{j \in S} p_j\right) \left(1 + \frac{1}{\sum_{j\in S} p_j} \left(\dxs(p_S, q_S)  - \sum_{i \in S} q_i\right)\right) \\
&=\sum_{i \in S} (p_i - q_i) +  \dxs(p_S, q_S).
\end{align*}
The first inequality is Jensen's, and the second is $1 + x \leq \exp(x)$.

\end{document}
