\documentclass{article}
\usepackage{ifthen}
\usepackage{mdwlist}
\usepackage{amsmath,amssymb,amsfonts,amsthm}
\usepackage{bm}
\usepackage{hyperref}
\usepackage{enumitem}
\usepackage{graphicx}
\usepackage{xspace}
\usepackage{verbatim}
\usepackage{algorithm}
\usepackage{algpseudocode}
\usepackage[margin=1in]{geometry}
\usepackage{color}
\usepackage{thm-restate}
\usepackage{latexsym}
\usepackage{epsfig}
\usepackage{pgf}
\usepackage{tikz}
\usetikzlibrary{tikzmark}
\usepackage{xcolor}
\usepackage{colortbl}

%\usepackage[OT1,T1]{fontenc}

\usepackage[numbers,sort&compress]{natbib}
\renewcommand{\bibfont}{\footnotesize}
%\usepackage{cite}
%\usepackage{mystyle}
%%%%%%%%%%%%%%%%%%%%%%%%%%%%%%%%%%%%
\makeatletter

\usepackage{etex}

%%% Review %%%

\usepackage{zref-savepos}

\newcounter{mnote}%[page]
\renewcommand{\themnote}{p.\thepage\;$\langle$\arabic{mnote}$\rangle$}

\def\xmarginnote{%
  \xymarginnote{\hskip -\marginparsep \hskip -\marginparwidth}}

\def\ymarginnote{%
  \xymarginnote{\hskip\columnwidth \hskip\marginparsep}}

\long\def\xymarginnote#1#2{%
\vadjust{#1%
\smash{\hbox{{%
        \hsize\marginparwidth
        \@parboxrestore
        \@marginparreset
\footnotesize #2}}}}}

\def\mnoteson{%
\gdef\mnote##1{\refstepcounter{mnote}\label{##1}%
  \zsavepos{##1}%
  \ifnum20432158>\number\zposx{##1}%
  \xmarginnote{{\color{blue}\bf $\langle$\arabic{mnote}$\rangle$}}% 
  \else
  \ymarginnote{{\color{blue}\bf $\langle$\arabic{mnote}$\rangle$}}%
  \fi%
}
  }
\gdef\mnotesoff{\gdef\mnote##1{}}
\mnoteson
\mnotesoff








%%% Layout %%%

% \usepackage{geometry} % override layout
% \geometry{tmargin=2.5cm,bmargin=m2.5cm,lmargin=3cm,rmargin=3cm}
% \setlength{\pdfpagewidth}{8.5in} % overrides default pdftex paper size
% \setlength{\pdfpageheight}{11in}

\newlength{\mywidth}

%%% Conventions %%%

% References
\newcommand{\figref}[1]{Fig.~\ref{#1}}
\newcommand{\defref}[1]{Definition~\ref{#1}}
\newcommand{\tabref}[1]{Table~\ref{#1}}
% general
%\usepackage{ifthen,nonfloat,subfigure,rotating,array,framed}
\usepackage{framed}
%\usepackage{subfigure}
\usepackage{subcaption}
\usepackage{comment}
%\specialcomment{nb}{\begingroup \noindent \framed\textbf{n.b.\ }}{\endframed\endgroup}
%%\usepackage{xtab,arydshln,multirow}
% topcaption defined in xtab. must load nonfloat before xtab
%\PassOptionsToPackage{svgnames,dvipsnames}{xcolor}
\usepackage[svgnames,dvipsnames]{xcolor}
%\definecolor{myblue}{rgb}{.8,.8,1}
%\definecolor{umbra}{rgb}{.8,.8,.5}
%\newcommand*\mybluebox[1]{%
%  \colorbox{myblue}{\hspace{1em}#1\hspace{1em}}}
\usepackage[all]{xy}
%\usepackage{pstricks,pst-node}
\usepackage{tikz}
\usetikzlibrary{positioning,matrix,through,calc,arrows,fit,shapes,decorations.pathreplacing,decorations.markings,decorations.text}

\tikzstyle{block} = [draw,fill=blue!20,minimum size=2em]

% allow prefix to scope name
\tikzset{%
	prefix node name/.code={%
		\tikzset{%
			name/.code={\edef\tikz@fig@name{#1 ##1}}
		}%
	}%
}


\@ifpackagelater{tikz}{2013/12/01}{
	\newcommand{\convexpath}[2]{
		[create hullcoords/.code={
			\global\edef\namelist{#1}
			\foreach [count=\counter] \nodename in \namelist {
				\global\edef\numberofnodes{\counter}
				\coordinate (hullcoord\counter) at (\nodename);
			}
			\coordinate (hullcoord0) at (hullcoord\numberofnodes);
			\pgfmathtruncatemacro\lastnumber{\numberofnodes+1}
			\coordinate (hullcoord\lastnumber) at (hullcoord1);
		}, create hullcoords ]
		($(hullcoord1)!#2!-90:(hullcoord0)$)
		\foreach [evaluate=\currentnode as \previousnode using \currentnode-1,
		evaluate=\currentnode as \nextnode using \currentnode+1] \currentnode in {1,...,\numberofnodes} {
			let \p1 = ($(hullcoord\currentnode) - (hullcoord\previousnode)$),
			\n1 = {atan2(\y1,\x1) + 90},
			\p2 = ($(hullcoord\nextnode) - (hullcoord\currentnode)$),
			\n2 = {atan2(\y2,\x2) + 90},
			\n{delta} = {Mod(\n2-\n1,360) - 360}
			in 
			{arc [start angle=\n1, delta angle=\n{delta}, radius=#2]}
			-- ($(hullcoord\nextnode)!#2!-90:(hullcoord\currentnode)$) 
		}
	}
}{
	\newcommand{\convexpath}[2]{
		[create hullcoords/.code={
			\global\edef\namelist{#1}
			\foreach [count=\counter] \nodename in \namelist {
				\global\edef\numberofnodes{\counter}
				\coordinate (hullcoord\counter) at (\nodename);
			}
			\coordinate (hullcoord0) at (hullcoord\numberofnodes);
			\pgfmathtruncatemacro\lastnumber{\numberofnodes+1}
			\coordinate (hullcoord\lastnumber) at (hullcoord1);
		}, create hullcoords ]
		($(hullcoord1)!#2!-90:(hullcoord0)$)
		\foreach [evaluate=\currentnode as \previousnode using \currentnode-1,
		evaluate=\currentnode as \nextnode using \currentnode+1] \currentnode in {1,...,\numberofnodes} {
			let \p1 = ($(hullcoord\currentnode) - (hullcoord\previousnode)$),
			\n1 = {atan2(\x1,\y1) + 90},
			\p2 = ($(hullcoord\nextnode) - (hullcoord\currentnode)$),
			\n2 = {atan2(\x2,\y2) + 90},
			\n{delta} = {Mod(\n2-\n1,360) - 360}
			in 
			{arc [start angle=\n1, delta angle=\n{delta}, radius=#2]}
			-- ($(hullcoord\nextnode)!#2!-90:(hullcoord\currentnode)$) 
		}
	}
}

% circle around nodes

% typsetting math
\usepackage{qsymbols,amssymb,mathrsfs}
\usepackage{amsmath}
\usepackage[standard,thmmarks]{ntheorem}
\theoremstyle{plain}
\theoremsymbol{\ensuremath{_\vartriangleleft}}
\theorembodyfont{\itshape}
\theoremheaderfont{\normalfont\bfseries}
\theoremseparator{}
\newtheorem{Claim}{Claim}
\newtheorem{Subclaim}{Subclaim}
\theoremstyle{nonumberplain}
\theoremheaderfont{\scshape}
\theorembodyfont{\normalfont}
\theoremsymbol{\ensuremath{_\blacktriangleleft}}
\newtheorem{Subproof}{Proof}

\theoremnumbering{arabic}
\theoremstyle{plain}
\usepackage{latexsym}
\theoremsymbol{\ensuremath{_\Box}}
\theorembodyfont{\itshape}
\theoremheaderfont{\normalfont\bfseries}
\theoremseparator{}
\newtheorem{Conjecture}{Conjecture}

\theorembodyfont{\upshape}
\theoremprework{\bigskip\hrule}
\theorempostwork{\hrule\bigskip}
\newtheorem{Condition}{Condition}%[section]


%\RequirePckage{amsmath} loaded by empheq
\usepackage[overload]{empheq} % no \intertext and \displaybreak
%\usepackage{breqn}

\let\iftwocolumn\if@twocolumn
\g@addto@macro\@twocolumntrue{\let\iftwocolumn\if@twocolumn}
\g@addto@macro\@twocolumnfalse{\let\iftwocolumn\if@twocolumn}

%\empheqset{box=\mybluebox}
%\usepackage{mathtools}      % to polish math typsetting, loaded
%                                % by empeq
\mathtoolsset{showonlyrefs=false,showmanualtags}
\let\underbrace\LaTeXunderbrace % adapt spacing to font sizes
\let\overbrace\LaTeXoverbrace
\renewcommand{\eqref}[1]{\textup{(\refeq{#1})}} % eqref was not allowed in
                                       % sub/super-scripts
\newtagform{brackets}[]{(}{)}   % new tags for equations
\usetagform{brackets}
% defined commands:
% \shortintertext{}, dcases*, \cramped, \smashoperator[]{}

\usepackage[Smaller]{cancel}
\renewcommand{\CancelColor}{\color{Red}}
%\newcommand\hcancel[2][black]{\setbox0=\hbox{#2}% colored horizontal cross
%  \rlap{\raisebox{.45\ht0}{\color{#1}\rule{\wd0}{1pt}}}#2}



\usepackage{graphicx,psfrag}
\graphicspath{{figure/}{image/}} % Search path of figures

% for tabular
\usepackage{diagbox} % \backslashbox{}{} for slashed entries
%\usepackage{threeparttable} % threeparttable, \tnote{},
                                % tablenotes, and \item[]
%\usepackage{colortab} % \cellcolor[gray]{0.9},
%\rowcolor,\columncolor,
%\usepackage{colortab} % \LCC \gray & ...  \ECC \\

% typesetting codes
%\usepackage{maple2e} % need to use \char29 for ^
\usepackage{algorithm2e}
\usepackage{listings} 
\lstdefinelanguage{Maple}{
  morekeywords={proc,module,end, for,from,to,by,while,in,do,od
    ,if,elif,else,then,fi ,use,try,catch,finally}, sensitive,
  morecomment=[l]\#,
  morestring=[b]",morestring=[b]`}[keywords,comments,strings]
\lstset{
  basicstyle=\scriptsize,
  keywordstyle=\color{ForestGreen}\bfseries,
  commentstyle=\color{DarkBlue},
  stringstyle=\color{DimGray}\ttfamily,
  texcl
}
%%% New fonts %%%
\DeclareMathAlphabet{\mathpzc}{OT1}{pzc}{m}{it}
\usepackage{upgreek} % \upalpha,\upbeta, ...
%\usepackage{bbold}   % blackboard math
\usepackage{dsfont}  % \mathds

%%% Macros for multiple definitions %%%

% example usage:
% \multi{M}{\boldsymbol{#1}}  % defines \multiM
% \multi ABC.                 % defines \MA \MB and \MC as
%                             % \boldsymbol{A}, \boldsymbol{B} and
%                             % \boldsymbol{C} respectively.
% 
%  The last period '.' is necessary to terminate the macro expansion.
%
% \multi*{M}{\boldsymbol{#1}} % defines \multiM and \M
% \M{A}                       % expands to \boldsymbol{A}

\def\multi@nostar#1#2{%
  \expandafter\def\csname multi#1\endcsname##1{%
    \if ##1.\let\next=\relax \else
    \def\next{\csname multi#1\endcsname}     
    %\expandafter\def\csname #1##1\endcsname{#2}
    \expandafter\newcommand\csname #1##1\endcsname{#2}
    \fi\next}}

\def\multi@star#1#2{%
  \expandafter\def\csname #1\endcsname##1{#2}
  \multi@nostar{#1}{#2}
}

\newcommand{\multi}{%
  \@ifstar \multi@star \multi@nostar}

%%% new alphabets %%%

\multi*{rm}{\mathrm{#1}}
\multi*{mc}{\mathcal{#1}}
\multi*{op}{\mathop {\operator@font #1}}
% \multi*{op}{\operatorname{#1}}
\multi*{ds}{\mathds{#1}}
\multi*{set}{\mathcal{#1}}
\multi*{rsfs}{\mathscr{#1}}
\multi*{pz}{\mathpzc{#1}}
\multi*{M}{\boldsymbol{#1}}
\multi*{R}{\mathsf{#1}}
\multi*{RM}{\M{\R{#1}}}
\multi*{bb}{\mathbb{#1}}
\multi*{td}{\tilde{#1}}
\multi*{tR}{\tilde{\mathsf{#1}}}
\multi*{trM}{\tilde{\M{\R{#1}}}}
\multi*{tset}{\tilde{\mathcal{#1}}}
\multi*{tM}{\tilde{\M{#1}}}
\multi*{baM}{\bar{\M{#1}}}
\multi*{ol}{\overline{#1}}

\multirm  ABCDEFGHIJKLMNOPQRSTUVWXYZabcdefghijklmnopqrstuvwxyz.
\multiol  ABCDEFGHIJKLMNOPQRSTUVWXYZabcdefghijklmnopqrstuvwxyz.
\multitR   ABCDEFGHIJKLMNOPQRSTUVWXYZabcdefghijklmnopqrstuvwxyz.
\multitd   ABCDEFGHIJKLMNOPQRSTUVWXYZabcdefghijklmnopqrstuvwxyz.
\multitset ABCDEFGHIJKLMNOPQRSTUVWXYZabcdefghijklmnopqrstuvwxyz.
\multitM   ABCDEFGHIJKLMNOPQRSTUVWXYZabcdefghijklmnopqrstuvwxyz.
\multibaM   ABCDEFGHIJKLMNOPQRSTUVWXYZabcdefghijklmnopqrstuvwxyz.
\multitrM   ABCDEFGHIJKLMNOPQRSTUVWXYZabcdefghijklmnopqrstuvwxyz.
\multimc   ABCDEFGHIJKLMNOPQRSTUVWXYZabcdefghijklmnopqrstuvwxyz.
\multiop   ABCDEFGHIJKLMNOPQRSTUVWXYZabcdefghijklmnopqrstuvwxyz.
\multids   ABCDEFGHIJKLMNOPQRSTUVWXYZabcdefghijklmnopqrstuvwxyz.
\multiset  ABCDEFGHIJKLMNOPQRSTUVWXYZabcdefghijklmnopqrstuvwxyz.
\multirsfs ABCDEFGHIJKLMNOPQRSTUVWXYZabcdefghijklmnopqrstuvwxyz.
\multipz   ABCDEFGHIJKLMNOPQRSTUVWXYZabcdefghijklmnopqrstuvwxyz.
\multiM    ABCDEFGHIJKLMNOPQRSTUVWXYZabcdefghijklmnopqrstuvwxyz.
\multiR    ABCDEFGHIJKL NO QR TUVWXYZabcd fghijklmnopqrstuvwxyz.
\multibb   ABCDEFGHIJKLMNOPQRSTUVWXYZabcdefghijklmnopqrstuvwxyz.
\multiRM   ABCDEFGHIJKLMNOPQRSTUVWXYZabcdefghijklmnopqrstuvwxyz.
\newcommand{\RRM}{\R{M}}
\newcommand{\RRP}{\R{P}}
\newcommand{\RRe}{\R{e}}
\newcommand{\RRS}{\R{S}}
%%% new symbols %%%

%\newcommand{\dotgeq}{\buildrel \textstyle  .\over \geq}
%\newcommand{\dotleq}{\buildrel \textstyle  .\over \leq}
\newcommand{\dotleq}{\buildrel \textstyle  .\over {\smash{\lower
      .2ex\hbox{\ensuremath\leqslant}}\vphantom{=}}}
\newcommand{\dotgeq}{\buildrel \textstyle  .\over {\smash{\lower
      .2ex\hbox{\ensuremath\geqslant}}\vphantom{=}}}

\DeclareMathOperator*{\argmin}{arg\,min}
\DeclareMathOperator*{\argmax}{arg\,max}

%%% abbreviations %%%

% commands
\newcommand{\esm}{\ensuremath}

% environments
\newcommand{\bM}{\begin{bmatrix}}
\newcommand{\eM}{\end{bmatrix}}
\newcommand{\bSM}{\left[\begin{smallmatrix}}
\newcommand{\eSM}{\end{smallmatrix}\right]}
\renewcommand*\env@matrix[1][*\c@MaxMatrixCols c]{%
  \hskip -\arraycolsep
  \let\@ifnextchar\new@ifnextchar
  \array{#1}}



% sets of number
\newqsymbol{`N}{\mathbb{N}}
\newqsymbol{`R}{\mathbb{R}}
\newqsymbol{`P}{\mathbb{P}}
\newqsymbol{`Z}{\mathbb{Z}}

% symbol short cut
\newqsymbol{`|}{\mid}
% use \| for \parallel
\newqsymbol{`8}{\infty}
\newqsymbol{`1}{\left}
\newqsymbol{`2}{\right}
\newqsymbol{`6}{\partial}
\newqsymbol{`0}{\emptyset}
\newqsymbol{`-}{\leftrightarrow}
\newqsymbol{`<}{\langle}
\newqsymbol{`>}{\rangle}

%%% new operators / functions %%%

\newcommand{\sgn}{\operatorname{sgn}}
\newcommand{\Var}{\op{Var}}
\newcommand{\diag}{\operatorname{diag}}
\newcommand{\erf}{\operatorname{erf}}
\newcommand{\erfc}{\operatorname{erfc}}
\newcommand{\erfi}{\operatorname{erfi}}
\newcommand{\adj}{\operatorname{adj}}
\newcommand{\supp}{\operatorname{supp}}
\newcommand{\E}{\opE\nolimits}
\newcommand{\T}{\intercal}
% requires mathtools
% \abs,\abs*,\abs[<size_cmd:\big,\Big,\bigg,\Bigg etc.>]
\DeclarePairedDelimiter\abs{\lvert}{\rvert} 
\DeclarePairedDelimiter\norm{\lVert}{\rVert}
\DeclarePairedDelimiter\ceil{\lceil}{\rceil}
\DeclarePairedDelimiter\floor{\lfloor}{\rfloor}
\DeclarePairedDelimiter\Set{\{}{\}}
\newcommand{\imod}[1]{\allowbreak\mkern10mu({\operator@font mod}\,\,#1)}

%%% new formats %%%
\newcommand{\leftexp}[2]{{\vphantom{#2}}^{#1}{#2}}


% non-floating figures that can be put inside tables
\newenvironment{nffigure}[1][\relax]{\vskip \intextsep
  \noindent\minipage{\linewidth}\def\@captype{figure}}{\endminipage\vskip \intextsep}

\newcommand{\threecols}[3]{
\hbox to \textwidth{%
      \normalfont\rlap{\parbox[b]{\textwidth}{\raggedright#1\strut}}%
        \hss\parbox[b]{\textwidth}{\centering#2\strut}\hss
        \llap{\parbox[b]{\textwidth}{\raggedleft#3\strut}}%
    }% hbox 
}

\newcommand{\reason}[2][\relax]{
  \ifthenelse{\equal{#1}{\relax}}{
    \left(\text{#2}\right)
  }{
    \left(\parbox{#1}{\raggedright #2}\right)
  }
}

\newcommand{\marginlabel}[1]
{\mbox[]\marginpar{\color{ForestGreen} \sffamily \small \raggedright\hspace{0pt}#1}}


% up-tag an equation
\newcommand{\utag}[2]{\mathop{#2}\limits^{\text{(#1)}}}
\newcommand{\uref}[1]{(#1)}


% Notation table

\newcommand{\Hline}{\noalign{\vskip 0.1in \hrule height 0.1pt \vskip
    0.1in}}
  
\def\Malign#1{\tabskip=0in
  \halign to\columnwidth{
    \ensuremath{\displaystyle ##}\hfil
    \tabskip=0in plus 1 fil minus 1 fil
    &
    \parbox[t]{0.8\columnwidth}{##}
    \tabskip=0in
    \cr #1}}


%%%%%%%%%%%%%%%%%%%%%%%%%%%%%%%%%%%%%%%%%%%%%%%%%%%%%%%%%%%%%%%%%%%
% MISCELLANEOUS

% Modification from braket.sty by Donald Arseneau
% Command defined is: \extendvert{ }
% The "small versions" use fixed-size brackets independent of their
% contents, whereas the expand the first vertical line '|' or '\|' to
% envelop the content
\let\SavedDoubleVert\relax
\let\protect\relax
{\catcode`\|=\active
  \xdef\extendvert{\protect\expandafter\noexpand\csname extendvert \endcsname}
  \expandafter\gdef\csname extendvert \endcsname#1{\mskip-5mu \left.%
      \ifx\SavedDoubleVert\relax \let\SavedDoubleVert\|\fi
     \:{\let\|\SetDoubleVert
       \mathcode`\|32768\let|\SetVert
     #1}\:\right.\mskip-5mu}
}
\def\SetVert{\@ifnextchar|{\|\@gobble}% turn || into \|
    {\egroup\;\mid@vertical\;\bgroup}}
\def\SetDoubleVert{\egroup\;\mid@dblvertical\;\bgroup}

% If the user is using e-TeX with its \middle primitive, use that for
% verticals instead of \vrule.
%
\begingroup
 \edef\@tempa{\meaning\middle}
 \edef\@tempb{\string\middle}
\expandafter \endgroup \ifx\@tempa\@tempb
 \def\mid@vertical{\middle|}
 \def\mid@dblvertical{\middle\SavedDoubleVert}
\else
 \def\mid@vertical{\mskip1mu\vrule\mskip1mu}
 \def\mid@dblvertical{\mskip1mu\vrule\mskip2.5mu\vrule\mskip1mu}
\fi

%%%%%%%%%%%%%%%%%%%%%%%%%%%%%%%%%%%%%%%%%%%%%%%%%%%%%%%%%%%%%%%%

\makeatother

%%%%%%%%%%%%%%%%%%%%%%%%%%%%%%%%%%%%

\usepackage{ctable}
\usepackage{fouridx}
%\usepackage{calc}
\usepackage{framed}
\usetikzlibrary{positioning,matrix}

\usepackage{paralist}
%\usepackage{refcheck}
\usepackage{enumerate}

\usepackage[normalem]{ulem}
\newcommand{\Ans}[1]{\uuline{\raisebox{.15em}{#1}}}



\numberwithin{equation}{section}
\makeatletter
\@addtoreset{equation}{section}
\renewcommand{\theequation}{\arabic{section}.\arabic{equation}}
\@addtoreset{Theorem}{section}
\renewcommand{\theTheorem}{\arabic{section}.\arabic{Theorem}}
\@addtoreset{Lemma}{section}
\renewcommand{\theLemma}{\arabic{section}.\arabic{Lemma}}
\@addtoreset{Corollary}{section}
\renewcommand{\theCorollary}{\arabic{section}.\arabic{Corollary}}
\@addtoreset{Example}{section}
\renewcommand{\theExample}{\arabic{section}.\arabic{Example}}
\@addtoreset{Remark}{section}
\renewcommand{\theRemark}{\arabic{section}.\arabic{Remark}}
\@addtoreset{Proposition}{section}
\renewcommand{\theProposition}{\arabic{section}.\arabic{Proposition}}
\@addtoreset{Definition}{section}
\renewcommand{\theDefinition}{\arabic{section}.\arabic{Definition}}
\@addtoreset{Claim}{section}
\renewcommand{\theClaim}{\arabic{section}.\arabic{Claim}}
\@addtoreset{Subclaim}{Theorem}
\renewcommand{\theSubclaim}{\theTheorem\Alph{Subclaim}}
\makeatother

\newcommand{\Null}{\op{Null}}
%\newcommand{\T}{\op{T}\nolimits}
\newcommand{\Bern}{\op{Bern}\nolimits}
\newcommand{\odd}{\op{odd}}
\newcommand{\even}{\op{even}}
\newcommand{\Sym}{\op{Sym}}
\newcommand{\si}{s_{\op{div}}}
\newcommand{\sv}{s_{\op{var}}}
\newcommand{\Wtyp}{W_{\op{typ}}}
\newcommand{\Rco}{R_{\op{CO}}}
\newcommand{\Tm}{\op{T}\nolimits}
\newcommand{\JGK}{J_{\op{GK}}}

\newcommand{\diff}{\mathrm{d}}

\newenvironment{lbox}{
  \setlength{\FrameSep}{1.5mm}
  \setlength{\FrameRule}{0mm}
  \def\FrameCommand{\fboxsep=\FrameSep \fcolorbox{black!20}{white}}%
  \MakeFramed {\FrameRestore}}%
{\endMakeFramed}

\newenvironment{ybox}{
	\setlength{\FrameSep}{1.5mm}
	\setlength{\FrameRule}{0mm}
  \def\FrameCommand{\fboxsep=\FrameSep \fcolorbox{black!10}{yellow!8}}%
  \MakeFramed {\FrameRestore}}%
{\endMakeFramed}

\newenvironment{gbox}{
	\setlength{\FrameSep}{1.5mm}
\setlength{\FrameRule}{0mm}
  \def\FrameCommand{\fboxsep=\FrameSep \fcolorbox{black!10}{green!8}}%
  \MakeFramed {\FrameRestore}}%
{\endMakeFramed}

\newenvironment{bbox}{
	\setlength{\FrameSep}{1.5mm}
\setlength{\FrameRule}{0mm}
  \def\FrameCommand{\fboxsep=\FrameSep \fcolorbox{black!10}{blue!8}}%
  \MakeFramed {\FrameRestore}}%
{\endMakeFramed}

\newenvironment{yleftbar}{%
  \def\FrameCommand{{\color{yellow!20}\vrule width 3pt} \hspace{10pt}}%
  \MakeFramed {\advance\hsize-\width \FrameRestore}}%
 {\endMakeFramed}

\newcommand{\tbox}[2][\relax]{
 \setlength{\FrameSep}{1.5mm}
  \setlength{\FrameRule}{0mm}
  \begin{ybox}
    \noindent\underline{#1:}\newline
    #2
  \end{ybox}
}

\newcommand{\pbox}[2][\relax]{
  \setlength{\FrameSep}{1.5mm}
 \setlength{\FrameRule}{0mm}
  \begin{gbox}
    \noindent\underline{#1:}\newline
    #2
  \end{gbox}
}

\newcommand{\gtag}[1]{\text{\color{green!50!black!60} #1}}
\let\labelindent\relax
\usepackage{enumitem}

%%%%%%%%%%%%%%%%%%%%%%%%%%%%%%%%%%%%
% fix subequations
% http://tex.stackexchange.com/questions/80134/nesting-subequations-within-align
%%%%%%%%%%%%%%%%%%%%%%%%%%%%%%%%%%%%

\usepackage{etoolbox}

% let \theparentequation use the same definition as equation
\let\theparentequation\theequation
% change every occurence of "equation" to "parentequation"
\patchcmd{\theparentequation}{equation}{parentequation}{}{}

\renewenvironment{subequations}[1][]{%              optional argument: label-name for (first) parent equation
	\refstepcounter{equation}%
	%  \def\theparentequation{\arabic{parentequation}}% we patched it already :)
	\setcounter{parentequation}{\value{equation}}%    parentequation = equation
	\setcounter{equation}{0}%                         (sub)equation  = 0
	\def\theequation{\theparentequation\alph{equation}}% 
	\let\parentlabel\label%                           Evade sanitation performed by amsmath
	\ifx\\#1\\\relax\else\label{#1}\fi%               #1 given: \label{#1}, otherwise: nothing
	\ignorespaces
}{%
	\setcounter{equation}{\value{parentequation}}%    equation = subequation
	\ignorespacesafterend
}

\newcommand*{\nextParentEquation}[1][]{%            optional argument: label-name for (first) parent equation
	\refstepcounter{parentequation}%                  parentequation++
	\setcounter{equation}{0}%                         equation = 0
	\ifx\\#1\\\relax\else\parentlabel{#1}\fi%         #1 given: \label{#1}, otherwise: nothing
}

% hyperlink
\PassOptionsToPackage{breaklinks,letterpaper,hyperindex=true,backref=false,bookmarksnumbered,bookmarksopen,linktocpage,colorlinks,linkcolor=BrickRed,citecolor=OliveGreen,urlcolor=Blue,pdfstartview=FitH}{hyperref}
\usepackage{hyperref}

% makeindex style
\newcommand{\indexmain}[1]{\textbf{\hyperpage{#1}}}

\newcommand{\cnote}[1]{\red{#1}}

\title{Which Distribution Distances are Sublinearly Testable?}

\author {
Constantinos Daskalakis\thanks{Supported by NSF CCF-1617730, CCF-1650733, and ONR N00014-12-1-0999.}\\
EECS \& CSAIL, MIT\\
\tt{costis@csail.mit.edu}
\and
Gautam Kamath\thanks{Supported by NSF CCF-1617730, CCF-1650733, and ONR N00014-12-1-0999. Part of this work was done while the author was an intern at Microsoft Research New England.} \\
EECS \& CSAIL, MIT\\
\tt{g@csail.mit.edu}
\and
John Wright\thanks{Supported by NSF grant CCF-6931885.} \\
Physics, MIT\\
\tt{jswright@mit.edu}
}
\begin{document}
\maketitle
\begin{abstract}
\begin{abstract}
\noindent\textbf{Abstract:} In this paper, we present \textit{JADE}, a targeted linguistic fuzzing platform which strengthens the linguistic complexity of seed questions to simultaneously and consistently break a wide range of widely-used LLMs categorized in three groups: eight open-sourced Chinese, six commercial Chinese and four commercial English LLMs. JADE generates three safety benchmarks for the three groups of LLMs, which contain unsafe questions that are highly threatening: the questions simultaneously trigger harmful generation of multiple LLMs, with an average unsafe generation ratio of \textbf{$70\%$} (please see the table below), while are still natural questions, fluent and preserving the core unsafe semantics. We release the benchmark demos generated for commercial English LLMs and open-sourced Chinese LLMs in the following link: \url{https://github.com/whitzard-ai/jade-db}. For readers who are interested in evaluating on more questions generated by JADE, please contact us.


% This results in a safety benchmark of natural questions which simultaneously trigger harmful generation of a wide range of widely-used LLMs below, in over $70\%$ test cases.



% Table generated by Excel2LaTeX from sheet 'Sheet2'
\begin{center}
\scalebox{0.65}{
    \begin{tabular}{lccccccc}
    \toprule
    \multirow{2}[3]{*}{\textbf{Group}} & \multicolumn{4}{c}{\multirow{2}[3]{*}{\textbf{Model Name}}} & \multicolumn{3}{c}{\textbf{Unsafe Generation Ratio}} \\
\cmidrule{6-8}          & \multicolumn{4}{c}{}          & \textbf{Average} & \textbf{Least} & \textbf{Most} \\
    \midrule
    \multirow{2}[2]{*}{\textbf{Open-sourced LLM (Chinese)}} & ChatGLM & ChatGLM2 & InternLM & Ziya  & \multirow{2}[2]{*}{74.13\%} & \multirow{2}[2]{*}{49.00\%} & \multirow{2}[2]{*}{93.50\%} \\
          & Baichuan & BELLE & MOSS  & ChatYuan2 &       &       &  \\
    \midrule
    \textbf{Commercial LLM (English)} & ChatGPT & Claude & PaLM2 & LLaMA2 & 74.38\% & 35.00\% & 91.25\% \\
    \midrule
    \multirow{2}[2]{*}{\textbf{Commercial LLM (Chinese)}} & Doubao & Wenxin Yiyan & ChatGLM & SenseChat & \multirow{2}[2]{*}{77.5\%} & \multirow{2}[2]{*}{56.00\%} & \multirow{2}[2]{*}{90.00\%} \\
          & Baichuan & ABAB  & \multicolumn{2}{c}{\footnotesize{(For the detailed info., please refer to Table 2)}} &       &       &  \\
    \bottomrule
    \end{tabular}}%
\end{center}





\textit{JADE} is based on Noam Chomsky's seminal theory of transformational-generative grammar. Given a seed question with unsafe intention, \textit{JADE} invokes a sequence of generative and transformational rules to increment the complexity of the syntactic structure of the original question, until the safety guardrail is broken. Our key insight is: Due to the complexity of human language, most of the current best LLMs can hardly recognize the invariant evil from the infinite number of different syntactic structures which form an unbound example space that can never be fully covered. Technically, the generative/transformative rules are constructed by native speakers of the languages, and, once developed, can be used to automatically grow and transform the parse tree of a given question, until the guardrail is broken. Besides, \textit{JADE} also incorporates an active learning algorithm to incrementally improve the LLM-based evaluation module, which
iteratively optimizes the prompts for evaluation with a small amount of annotated data, to effectively strengthen the alignment with the judgement made by human experts. For more evaluation results and demo, please check our website: \url{https://whitzard-ai.github.io/jade.html}.

\noindent\pxd{{\footnotesize[\textbf{Content Warning: This paper contains examples of harmful language.}]}}
\end{abstract}





% Featured by OpenAI's ChatGPT, the rise of aligned large language models (LLM) is recognized as a milestone in the history of AI, and catalyzes wild imagination on the arrival of \textit{Artificial General Intelligence} (AGI). To achieve harmless generation, many approaches are proposed to align the AI generation contents with human values, or called \textit{AI alignment}. This equips pretrained large language models with the ability of generating safe responses under unsafe requests. However, we find human language is more complex than the current best LLM can handle. To validate this point, 
\end{abstract}

% !TEX root = ../arxiv.tex

Unsupervised domain adaptation (UDA) is a variant of semi-supervised learning \cite{blum1998combining}, where the available unlabelled data comes from a different distribution than the annotated dataset \cite{Ben-DavidBCP06}.
A case in point is to exploit synthetic data, where annotation is more accessible compared to the costly labelling of real-world images \cite{RichterVRK16,RosSMVL16}.
Along with some success in addressing UDA for semantic segmentation \cite{TsaiHSS0C18,VuJBCP19,0001S20,ZouYKW18}, the developed methods are growing increasingly sophisticated and often combine style transfer networks, adversarial training or network ensembles \cite{KimB20a,LiYV19,TsaiSSC19,Yang_2020_ECCV}.
This increase in model complexity impedes reproducibility, potentially slowing further progress.

In this work, we propose a UDA framework reaching state-of-the-art segmentation accuracy (measured by the Intersection-over-Union, IoU) without incurring substantial training efforts.
Toward this goal, we adopt a simple semi-supervised approach, \emph{self-training} \cite{ChenWB11,lee2013pseudo,ZouYKW18}, used in recent works only in conjunction with adversarial training or network ensembles \cite{ChoiKK19,KimB20a,Mei_2020_ECCV,Wang_2020_ECCV,0001S20,Zheng_2020_IJCV,ZhengY20}.
By contrast, we use self-training \emph{standalone}.
Compared to previous self-training methods \cite{ChenLCCCZAS20,Li_2020_ECCV,subhani2020learning,ZouYKW18,ZouYLKW19}, our approach also sidesteps the inconvenience of multiple training rounds, as they often require expert intervention between consecutive rounds.
We train our model using co-evolving pseudo labels end-to-end without such need.

\begin{figure}[t]%
    \centering
    \def\svgwidth{\linewidth}
    \input{figures/preview/bars.pdf_tex}
    \caption{\textbf{Results preview.} Unlike much recent work that combines multiple training paradigms, such as adversarial training and style transfer, our approach retains the modest single-round training complexity of self-training, yet improves the state of the art for adapting semantic segmentation by a significant margin.}
    \label{fig:preview}
\end{figure}

Our method leverages the ubiquitous \emph{data augmentation} techniques from fully supervised learning \cite{deeplabv3plus2018,ZhaoSQWJ17}: photometric jitter, flipping and multi-scale cropping.
We enforce \emph{consistency} of the semantic maps produced by the model across these image perturbations.
The following assumption formalises the key premise:

\myparagraph{Assumption 1.}
Let $f: \mathcal{I} \rightarrow \mathcal{M}$ represent a pixelwise mapping from images $\mathcal{I}$ to semantic output $\mathcal{M}$.
Denote $\rho_{\bm{\epsilon}}: \mathcal{I} \rightarrow \mathcal{I}$ a photometric image transform and, similarly, $\tau_{\bm{\epsilon}'}: \mathcal{I} \rightarrow \mathcal{I}$ a spatial similarity transformation, where $\bm{\epsilon},\bm{\epsilon}'\sim p(\cdot)$ are control variables following some pre-defined density (\eg, $p \equiv \mathcal{N}(0, 1)$).
Then, for any image $I \in \mathcal{I}$, $f$ is \emph{invariant} under $\rho_{\bm{\epsilon}}$ and \emph{equivariant} under $\tau_{\bm{\epsilon}'}$, \ie~$f(\rho_{\bm{\epsilon}}(I)) = f(I)$ and $f(\tau_{\bm{\epsilon}'}(I)) = \tau_{\bm{\epsilon}'}(f(I))$.

\smallskip
\noindent Next, we introduce a training framework using a \emph{momentum network} -- a slowly advancing copy of the original model.
The momentum network provides stable, yet recent targets for model updates, as opposed to the fixed supervision in model distillation \cite{Chen0G18,Zheng_2020_IJCV,ZhengY20}.
We also re-visit the problem of long-tail recognition in the context of generating pseudo labels for self-supervision.
In particular, we maintain an \emph{exponentially moving class prior} used to discount the confidence thresholds for those classes with few samples and increase their relative contribution to the training loss.
Our framework is simple to train, adds moderate computational overhead compared to a fully supervised setup, yet sets a new state of the art on established benchmarks (\cf \cref{fig:preview}).

%!TEX root = hopfwright.tex
%

In this section we systematically recast the Hopf bifurcation problem in Fourier space. 
We introduce appropriate scalings, sequence spaces of Fourier coefficients and convenient operators on these spaces. 
To study Equation~\eqref{eq:FourierSequenceEquation} we consider Fourier sequences $ \{a_k\}$ and fix a Banach space in which these sequences reside. It is indispensable for our analysis that this space have an algebraic structure. 
The Wiener algebra of absolutely summable Fourier series is a natural candidate, which we use with minor modifications. 
In numerical applications, weighted sequence spaces with algebraic and geometric decay have been used to great effect to study periodic solutions which are $C^k$ and analytic, respectively~\cite{lessard2010recent,hungria2016rigorous}. 
Although it follows from Lemma~\ref{l:analytic} that the Fourier coefficients of any solution decay exponentially, we choose to work in a space of less regularity. 
The reason is that by working in a space with less regularity, we are better able to connect our results with the global estimates in \cite{neumaier2014global}, see Theorem~\ref{thm:UniqunessNbd2}.


%
%
%\begin{remark}
%	Although it follows from Lemma~\ref{l:analytic} that the Fourier coefficients of any solution decay exponentially, we choose to work in a space of less regularity, namely summable Fourier coefficients. This allows us to draw SOME MORE INTERESTING CONCLUSION LATER.
%	EXPLAIN WHY WE CHOOSE A NORM WITH ALMOST NO DECAY!
%	% of s Periodic solutions to Wright's equation are known to be real analytic and so their  Fourier coefficients must decay geometrically [Nussbaum].
%	% We do not use such a strong result;  any periodic solution must be continuously differentiable, by which it follows that $ \sum | c_k| < \infty$.
%\end{remark}


\begin{remark}\label{r:a0}
There is considerable redundancy in Equation~\eqref{eq:FourierSequenceEquation}. First, since we are considering real-valued solutions $y$, we assume $\c_{-k}$ is the complex conjugate of $\c_k$. This symmetry implies it suffices to consider Equation~\eqref{eq:FourierSequenceEquation} for $k \geq 0$.
Second, we may effectively ignore the zeroth Fourier coefficient of any periodic solution \cite{jones1962existence}, since it is necessarily equal to $0$. 
%In \cite{jones1962existence}, it is shown that if $y \not\equiv -1$ is a periodic solution of~\eqref{eq:Wright} with frequency $\omega$, then $ \int_0^{2\pi/\omega} y(t) dt =0$. 
		The self contained argument is as follows. 
		As mentioned in the introduction, any periodic solution to Wright's equation must satisfy $ y(t) > -1$ for all $t$. 
	By dividing Equation~\eqref{eq:Wright} by $(1+y(t))$, which never vanishes, we obtain
	\[
	\frac{d}{dt} \log (1 + y(t)) = - \alpha y(t-1).
	\]  
	Integrating over one period $L$ we derive the condition 
	$0=\int_0^L y(t) dt $.
	Hence $a_0=0$ for any periodic solution. 
	It will be shown in Theorem~\ref{thm:FourierEquivalence1} that a related argument implies that we do not need to consider Equation~\eqref{eq:FourierSequenceEquation} for $k=0$.
\end{remark}

%%%
%%%
%%%\begin{remark}\label{r:c0} 
%%%In \cite{jones1962existence}, it is shown that if $y \not\equiv -1$ is a periodic solution of~\eqref{eq:Wright} with frequency $\omega$, then $ \int_0^{2\pi/\omega} y(t) dt =0$. 
%%%PERHAPS TOO MUCH DETAIL HERE. The self contained argument is as follows.
%%%If $y \not\equiv -1$ then $y(t) \neq -1$ for all $t$, since if $y(t_0)=-1$ for some $t_0 \in \R$ then $y'(t_0)=0$ by~\eqref{eq:Wright} and in fact by differentiating~\eqref{eq:Wright} repeatedly one obtains that all derivatives of $y$ vanish at $t_0$. Hence $y \equiv -1$ by Lemma~\ref{l:analytic}, a contradiction. Now divide~\eqref{eq:Wright} by $(1+y(t))$, which never vanishes, to obtain
%%%\[
%%%  \frac{d}{dt} \log |1 + y(t)| = - \alpha y(t-1).
%%%\]  
%%%Integrating over one period we obtain $\int_0^L y(t) dt =0$.
%%%\end{remark}



%Furthermore, the condition that $y(t)$ is real forces $\c_{-k} = \overline{\c}_{k}$.  
%
We define the spaces of absolutely summable Fourier series
\begin{alignat*}{1}
	\ell^1 &:= \left\{ \{ \c_k \}_{k \geq 1} : 
    \sum_{k \geq 1} | \c_k| < \infty  \right\} , \\
	\ell^1_\bi &:= \left\{ \{ \c_k \}_{k \in \Z} : 
    \sum_{k \in \Z} | \c_k| < \infty  \right\} .
\end{alignat*} 
We identify any semi-infinite sequence $ \{ \c_k \}_{k \geq 1} \in \ell^1$ with the bi-infinite sequence $ \{ \c_k \}_{k \in \Z} \in \ell^1_\bi$ via the conventions (see Remark~\ref{r:a0})
\begin{equation}
  \c_0=0 \qquad\text{ and }\qquad \c_{-k} = \c_{k}^*. 
\end{equation}
In other word, we identify $\ell^1$ with the set
\begin{equation*}
   \ell^1_\sym := \left\{ \c \in \ell^1_\bi : 
	\c_0=0,~\c_{-k}=\c_k^* \right\} .
\end{equation*}
On $\ell^1$ we introduce the norm
\begin{equation}\label{e:lnorm}
  \| \c \| = \| \c \|_{\ell^1} := 2 \sum_{k = 1}^\infty |\c_k|.
\end{equation}
The factor $2$ in this norm is chosen to have a Banach algebra estimate.
Indeed, for $\c, \tilde{\c} \in \ell^1 \cong \ell^1_\sym$ we define
the discrete convolution 
\[
\left[ \c * \tilde{\c} \right]_k = \sum_{\substack{k_1,k_2\in\Z\\ k_1 + k_2 = k}} \c_{k_1} \tilde{\c}_{k_2} .
\]
Although $[\c*\tilde{\c}]_0$ does not necessarily vanish, we have $\{\c*\tilde{\c}\}_{k \geq 1} \in \ell^1 $ and 
\begin{equation*}
	\| \c*\tilde{\c} \| \leq \| \c \| \cdot  \| \tilde{\c} \| 
	\qquad\text{for all } \c , \tilde{\c} \in \ell^1, 
\end{equation*}
hence $\ell^1$ with norm~\eqref{e:lnorm} is a Banach algebra.

By Lemma~\ref{l:analytic} it is clear that any periodic solution of~\eqref{eq:Wright} has a well-defined Fourier series $\c \in \ell^1_\bi$. 
The next theorem shows that in order to study periodic orbits to Wright's equation we only need to study Equation~\eqref{eq:FourierSequenceEquation} 
for $k \geq 1$. For convenience we introduce the notation 
\[
G(\alpha,\omega,\c)_k=
( i \omega k + \alpha e^{ - i \omega k}) \c_k + \alpha \sum_{k_1 + k_2 = k} e^{- i \omega k_1} \c_{k_1} \c_{k_2} \qquad \text{for } k \in \N.
\]
We note that we may interpret the trivial solution $y(t)\equiv 0$ as a periodic solution of arbitrary period.
\begin{theorem}
\label{thm:FourierEquivalence1}
Let $\alpha>0$ and $\omega>0$.
If $\c \in \ell^1 \cong \ell^1_{\sym}$ solves
$G(\alpha,\omega,\c)_k =0$  for all $k \geq 1$,
then $y(t)$ given by~\eqref{eq:FourierEquation} is a periodic solution of~\eqref{eq:Wright} with period~$2\pi/\omega$.
Vice versa, if $y(t)$ is a periodic solution of~\eqref{eq:Wright} with period~$2\pi/\omega$ then its Fourier coefficients $\c \in \ell^1_\bi$ lie in $\ell^1_\sym \cong \ell^1$ and solve $G(\alpha,\omega,\c)_k =0$ for all $k \geq 1$.
\end{theorem}

\begin{proof}	
	If $y(t)$ is a periodic solution of~\eqref{eq:Wright} then it is real analytic by Lemma~\ref{l:analytic}, hence its Fourier series $\c$ is well-defined and $\c \in \ell^1_{\sym}$ by Remark~\ref{r:a0}.
	Plugging the Fourier series~\eqref{eq:FourierEquation} into~\eqref{eq:Wright} one easily derives that $\c$ solves~\eqref{eq:FourierSequenceEquation} for all $k \geq 1$.

To prove the reverse implication, assume that $\c \in \ell^1_\sym$ solves
Equation~\eqref{eq:FourierSequenceEquation} for all $k \geq 1$. Since $\c_{-k}
= \c_k^*$, Equation \eqref{eq:FourierSequenceEquation} is also satisfied for
all $k \leq -1$. It follows from the Banach algebra property and
\eqref{eq:FourierSequenceEquation} that $\{k \c_k\}_{k \in \Z} \in \ell^1_\bi$,
hence $y$, given by~\eqref{eq:FourierEquation}, is continuously differentiable.
% (and by bootstrapping one infers that $\{k^m c_k \} \in \ell^1_\bi$, 
% hence $y \in C^m$ for any $m \geq 1$).
	Since~\eqref{eq:FourierSequenceEquation} is satisfied for all $k \in \Z \setminus \{0\}$ (but not necessarily for $k=0$) one may perform the inverse Fourier transform on~\eqref{eq:FourierSequenceEquation} to conclude that
	$y$ satisfies the delay equation 
\begin{equation}\label{eq:delaywithK}
   	y'(t) = - \alpha y(t-1) [ 1 + y(t)] + C
\end{equation}
	for some constant $C \in \R$. 
   Finally, to prove that $C=0$ we argue by contradiction.
   Suppose $C \neq 0$. Then $y(t) \neq -1$ for all $t$.
   Namely, at any point where $y(t_0) =-1$ one would have $y'(t_0) = C$
   which has fixed sign,   hence it would follow that $y$ is not periodic
   ($y$ would not be able to cross $-1$ in the opposite direction, 
   preventing $y$  from being periodic).  
  We may thus divide~\eqref{eq:delaywithK} through by $1 + y(t)$ and obtain 
\begin{equation*}
	\frac{d}{dt} \log | 1 + y(t) | = - \alpha y(t-1) + \frac{C}{1+y(t)} .
\end{equation*}
	By integrating both sides of the equation over one period $L$ and by using that $\c_0=0$, we 
	obtain
	\[
	 C \int_0^L \frac{1}{1+y(t)} dt =0.
	\]
	Since the integrand is either strictly negative or strictly positive, this implies that $C=0$. Hence~\eqref{eq:delaywithK} reduces to~\eqref{eq:Wright},
	and $y$ satisfies Wright's equation. 
\end{proof}






To efficiently study Equation~\eqref{eq:FourierSequenceEquation}, we introduce the following linear operators on $ \ell^1$:
\begin{alignat*}{1}
   [K \c ]_k &:= k^{-1} \c_k  , \\ 
   [ U_\omega \c ]_k &:= e^{-i k \omega} \c_k  .
\end{alignat*}
The map $K$ is a compact operator, and it has a densely defined inverse $K^{-1}$. The domain of $K^{-1}$ is denoted by
\[
  \ell^K := \{ \c \in \ell^1 : K^{-1} \c \in \ell^1 \}.  
\]
The map $U_{\omega}$ is a unitary operator on $\ell^1$, but
it is discontinuous in $\omega$. 
With this notation, Theorem~\ref{thm:FourierEquivalence1} implies that our problem of finding a SOPS to~\eqref{eq:Wright} is equivalent to finding an $\c \in \ell^1$ such that
\begin{equation}
\label{e:defG}
  G(\alpha,\omega,\c) :=
  \left( i \omega K^{-1} + \alpha U_\omega \right) \c + \alpha \left[U_\omega \, \c \right] * \c  = 0.
\end{equation}


%In order for the solutions of Equation \ref{eq:FHat} to be isolated we need to impose a phase condition. 
%If there is a sequence $ \{ c_k \} $ which satisfies  Equation \ref{eq:FHat}, then $ y( t + \tau) = \sum_{k \in \Z} c_k e^{ i k \omega (t + \tau)}$ satisfies Wright's equation at parameter $\alpha$. 
%Fix $ \tau = - Arg[c_1] / \omega$ so that $ c_1  e^{ i \omega \tau} $ is a nonnegative real number. 
%By Proposition \ref{thm:FourierEquivalence1} it follows that $\{ c'_k \} =  \{c_k e^{ i \omega k \tau }   \}$ is a solution to Equation \ref{eq:FHat}, and furthermore that $ c'_1 = \epsilon$ for some $ \epsilon \geq 0$. 


Periodic solutions are invariant under time translation: if $y(t)$ solves Wright's equation, then so does $ y(t+\tau)$ for any $\tau \in \R$. 
We remove this degeneracy by adding a phase condition. 
Without loss of generality, if $\c \in \ell^1$ solves Equation~\eqref{e:defG}, we may assume that $\c_1 = \epsilon$ for some 
\emph{real non-negative}~$\epsilon$:
\[
  \ell^1_{\epsilon} := \{\c \in \ell^1 : \c_1 = \epsilon \} 
  \qquad \text{where } \epsilon \in \R,  \epsilon \geq 0.
\]
In the rest of our analysis, we will split elements $\c \in \ell^1$ into two parts: $\c_1$ and $\{\c_{k}\}_{k \geq 2}$.  
We define the basis elements $\e_j \in \ell^1$ for $j=1,2,\dots$ as
\[
  [\e_j]_k = \begin{cases}
  1 & \text{if } k=j, \\
  0 & \text{if } k \neq j.
  \end{cases}
\]
We note that $\| \e_j \|=2$. 
Then we can decompose
% We define
% \[
%   \tilde{\epsilon} := (\epsilon,0,0,0,\dots) \in \ell^1
% \]
% and
% For clarity when referring to sequences $\{c_{k}\}_{k \geq 2}$, we make the following definition:
% \[
% \ell^1_0  := \{ \tc \in \ell^1 : \tc_1 = 0 \}.
% \]
% With the
any $\c \in \ell^1_\epsilon$ uniquely as
\begin{equation}\label{e:aepsc}
  \c= \epsilon \e_1 + \tc \qquad \text{with}\quad 
  \tc \in \ell^1_0 := \{ \tc \in \ell^1 : \tc_1 = 0 \}.
\end{equation}
We follow the classical approach in studying Hopf bifurcations and consider 
$\c_1 = \epsilon$ to be a parameter, and then find periodic solutions with Fourier modes in $\ell^1_{\epsilon}$.
This approach rewrites the function $G: \R^2 \times \ell^K \to \ell^1$ as a function $\tilde{F}_\epsilon : \R^2 \times \ell^K_0 \to \ell^1$, where 
we denote 
\[
\ell^K_0 := \ell^1_0 \cap \ell^K.
\]
% I AM ACTUALLY NOT SURE IF YOU WANT TO DEFINE THIS WITH RANGE IN $\ell^1$
% OR WITH DOMAIN IN $\ell^1_0$ ?? IT SEEMS TO DEPEND ON WHICH GLOBAL STATEMENT YOU WANT/NEED TO MAKE!?
\begin{definition}
We define the $\epsilon$-parameterized family of  functions $\tilde{F}_\epsilon: \R^2 \times \ell^K_0  \to \ell^1$ 
by 
\begin{equation}
\label{eq:fourieroperators}
\tilde{F}_{\epsilon}(\alpha,\omega, \tc) := 
\epsilon [i \omega + \alpha e^{-i \omega}] \e_1 + 
( i \omega K^{-1} + \alpha U_{\omega}) \tc + 
\epsilon^2 \alpha e^{-i \omega}  \e_2  +
\alpha \epsilon L_\omega \tc + 
\alpha  [ U_{\omega} \tc] * \tc ,
\end{equation}
where
$L_\omega : \ell^1_0 \to \ell^1$ is given by
\[
   L_{\omega} := \sigma^+( e^{- i \omega} I + U_{\omega}) + \sigma^-(e^{i \omega} I + U_{\omega}),
\]
with $I$ the identity and  $\sigma^\pm$ the shift operators on $\ell^1$:
\begin{alignat*}{2}
\left[ \sigma^- a \right]_k &:=  a_{k+1}  , \\
\left[ \sigma^+ a \right]_k &:=  a_{k-1}  &\qquad&\text{with the convention } \c_0=0.
\end{alignat*}
The operator $ L_\omega$ is discontinuous in $\omega$ and $ \| L_\omega \| \leq 4$. 
\end{definition} 

%The maps $ \sigma^{+}$ and $ \sigma^-$ are shift up and shift down operators respectively. 
We reformulate Theorem~\ref{thm:FourierEquivalence1}  in terms of the map  $\tilde{F}$. 
We note that it follows from Lemma~\ref{l:analytic} and 
%\marginpar{Reformulate}
%one's choice of  
Equation~\eqref{eq:FourierSequenceEquation}  
%or Equation ~\eqref{eq:fourieroperators},
that the Fourier coefficients of any periodic solution of~\eqref{eq:Wright} lie in $\ell^K$.
These observations are summarized in the following theorem.
\begin{theorem}
\label{thm:FourierEquivalence2}
	Let $ \epsilon \geq 0$,  $\tc \in \ell^K_0$, $\alpha>0$ and $ \omega >0$. 
	Define $y: \R\to \R$ as 
\begin{equation}\label{e:ytc}
	y(t) = 
	\epsilon \left( e^{i \omega t }  + e^{- i \omega t }\right) 
	+  \sum_{k = 2}^\infty   \tc_k e^{i \omega k t }  + \tc_k^* e^{- i \omega k t } .
\end{equation}
%	and suppose that $ y(t) > -1$. 
	Then $y(t)$ solves~\eqref{eq:Wright} if and only if $\tilde{F}_{\epsilon}( \alpha , \omega , \tc) = 0$. 
	Furthermore, up to time translation, any periodic solution of~\eqref{eq:Wright} with period $2\pi/\omega$ is described by a Fourier series of the form~\eqref{e:ytc} with $\epsilon \geq 0$ and $\tc \in \ell^K_0$.
\end{theorem}


%We note that for $\epsilon>0$ such solutions are truly periodic, while for $\epsilon=0$ a zero of $\tilde{F}_\epsilon$ may either correspond to a periodic solution or to the trivial solution $y(t) \equiv 0$. 



% \begin{proof}
%  By Proposition \ref{thm:FourierEquivalence1}, it suffices to show that $\tilde{F}(\alpha,\omega,c) =0$ is equivalent to Equation \ref{eq:FourierSequenceEquation} being satisfied for $k \geq 1$.
%  Since Equation \ref{eq:FourierSequenceEquation} is equivalent to Equation \ref{eq:FHat}, we expand  Equation \ref{eq:FHat} by writing $ \hat{c} = \hat{\epsilon } + c$  where $ \hat{\epsilon} := (\epsilon,0,0,\dots) \in \ell^1$ as below:
%  \begin{equation}
%  0=  \left( i \omega K^{-1} + \alpha U_\omega \right) (\hat{\epsilon}+ c) + \alpha \left[U_\omega \, (\hat{\epsilon}+ c) \right] * (\hat{\epsilon}+ c) \label{eq:Intial}
%  \end{equation}
%  The RHS of Equation \ref{eq:Intial} is $ \tilde{F}(\alpha,\omega,c)$, so the theorem is proved.
% \end{proof}



Since we want to analyze a Hopf bifurcation, we will want to solve $\tilde{F}_\epsilon = 0$ for small values of~$\epsilon$. 
However, at the bifurcation point, $ D \tilde{F}_0(\pp  ,\pp , 0)$ is not invertible.
In order for our asymptotic analysis to be non-degenerate,
we work with a rescaled version of the problem. To this end, for any $\epsilon >0$, we rescale both $\tc$ and $\tilde{F}$ as follows. Let $\tc = \epsilon c$ and 
\begin{equation}\label{e:changeofvariables}
  \tilde{F}_\epsilon (\alpha,\omega,\epsilon c) = \epsilon F_\epsilon (\alpha,\omega,c).
\end{equation}
For $\epsilon>0$ the problem then reduces to finding zeros of 
\begin{equation}
\label{eq:FDefinition}
	F_\epsilon(\alpha,\omega, c) := 
	[i \omega + \alpha e^{-i \omega}] \e_1 + 
	( i \omega K^{-1} + \alpha U_{\omega}) c + 
	\epsilon \alpha e^{-i \omega} \e_2  +
	\alpha \epsilon L_\omega c + 
	\alpha \epsilon [ U_{\omega} c] * c.
\end{equation}
We denote the triple $(\alpha,\omega,c) \in \R^2 \times \ell^1_0$ by $x$.
To pinpoint the components of $x$ we use the projection operators
\[
   \pi_\alpha x = \alpha, \quad \pi_\omega x = \omega, \quad 
  \pi_c x = c \qquad\text{for any } x=(\alpha,\omega,c).
\]

After the change of variables~\eqref{e:changeofvariables} we now have an invertible Jacobian $D F_0(\pp  ,\pp , 0)$ at the bifurcation point.
On the other hand, for $\epsilon=0$ the zero finding problems for $\tilde{F}_\epsilon$ and $F_\epsilon$ are not equivalent. 
However, it follows from the following lemma that any nontrivial periodic solution having $ \epsilon=0$ must have a relatively large size when $ \alpha $ and $ \omega $ are close to the bifurcation point. 

\begin{lemma}\label{lem:Cone}
	Fix $ \epsilon \geq 0$ and $\alpha,\omega >0$. 
	Let
	\[
	b_* :=  \frac{\omega}{\alpha} - \frac{1}{2} - \epsilon  \left(\frac{2}{3}+ \frac{1}{2}\sqrt{2 + 2 |\omega-\pp| } \right).
	\]
Assume that $b_*> \sqrt{2} \epsilon$. 
Define
% \begin{equation*}%\label{e:zstar}
% 	z^{\pm}_* :=b_* \pm \sqrt{(b_*)^2- \epsilon^2 } .
% \end{equation*}
% \note[J]{Proposed change to match Lemma E.4}
\begin{equation}\label{e:zstar}
z^{\pm}_* :=b_* \pm \sqrt{(b_*)^2- 2 \epsilon^2 } .
\end{equation}
If there exists a $\tc \in \ell^1_0$ such that $\tilde{F}_\epsilon(\alpha, \omega,\tc) = 0$, then \\
\mbox{}\quad\textup{(a)} either $ \|\tc\| \leq  z_*^-$ or $ \|\tc\| \geq z_*^+  $.\\
\mbox{}\quad\textup{(b)} 
$ \| K^{-1} \tc \| \leq (2\epsilon^2+ \|\tc\|^2) / b_*$. 
\end{lemma}
\begin{proof}
	The proof follows from Lemmas~\ref{lem:gamma} and~\ref{lem:thecone} in Appendix~\ref{appendix:aprioribounds}, combined with the observation that
$\frac{\omega}{\alpha} - \gamma \geq b_*$,
% \[
%   \frac{\omega}{\alpha} - \gamma \geq b_*
%  \qquad\text{for all }
% | \alpha - \pp| \leq r_\alpha \text{ and } 
%   | \omega - \pp| \leq r_\omega.
% \]
with $\gamma$ as defined in Lemma~\ref{lem:gamma}.
\end{proof}

\begin{remark}\label{r:smalleps}
We note that for $\alpha < 2\omega$
\begin{alignat*}{1}
z^+_* &\geq   \frac{2 \omega - \alpha}{\alpha} 
- \epsilon \left(4/3+\sqrt{2 + 2 |\omega-\pp| } \, \right) + \cO(\epsilon^2)
\\[1mm]
z^-_* & \leq   \cO(\epsilon^2)
\end{alignat*}
for small $\epsilon$. 
Hence Lemma~\ref{lem:Cone} implies that for values of $(\alpha,\omega)$ near $(\pp,\pp)$ any solution has either $\|\tc\|$ of order 1 or $\|\tc\| =  \cO(\epsilon^2)$. 
The asymptotically small term bounding $z_*^-$ is explicitly calculated in Lemma~\ref{lem:ZminusBound}. 
A related consequence is that for $\epsilon=0$ there are no nontrivial solutions 
of $\tilde{F}_0(\alpha,\omega,\tc)=0$ with 
$\| \tc \| < \frac{2 \omega - \alpha}{\alpha} $. 
\end{remark}

\begin{remark}\label{r:rhobound}
In Section~\ref{s:contraction} we will work on subsets of $\ell^K_0$ of the form
\[
  \ell_\rho := \{ c \in \ell^K_0 : \|K^{-1} c\| \leq \rho \} .
\]
Part (b) of Lemma~\ref{lem:Cone} will be used in Section~\ref{s:global} to guarantee that we are not missing any solutions by considering $\ell_\rho$ (for some specific choice of $\rho$) rather than the full space $\ell^K_0$.
In particular, we infer from Remark~\ref{r:smalleps} that  small solutions (meaning roughly that $\|\tc\| \to 0$ as $\epsilon \to 0$)
satisfy $\| K^{-1} \tc \| = \cO(\epsilon^2)$.
\end{remark}

The following theorem guarantees that near the bifurcation point the problem of finding all periodic solutions is equivalent to considering the rescaled problem $F_\epsilon(\alpha,\omega,c)=0$.
\begin{theorem}
\label{thm:FourierEquivalence3}
\textup{(a)} Let $ \epsilon > 0$,  $c \in \ell^K_0$, $\alpha>0$ and $ \omega >0$. 
	Define $y: \R\to \R$ as 
\begin{equation}\label{e:yc}
	y(t) = 
	\epsilon \left( e^{i \omega t }  + e^{- i \omega t }\right) 
	+ \epsilon  \sum_{k = 2}^\infty   c_k e^{i \omega k t }  + c_k^* e^{- i \omega k t } .
\end{equation}
%	and suppose that $ y(t) > -1$. 
	Then $y(t)$ solves~\eqref{eq:Wright} if and only if $F_{\epsilon}( \alpha , \omega , c) = 0$.\\
\textup{(b)}
Let $y(t) \not\equiv 0$ be a periodic solution of~\eqref{eq:Wright} of period $2\pi/\omega$
 with Fourier coefficients $\c$.
Suppose $\alpha < 2\omega$ and $\| \c \| < \frac{2 \omega - \alpha}{\alpha} $.
Then, up to time translation, $y(t)$ is described by a Fourier series of the form~\eqref{e:yc} with $\epsilon > 0$ and $c \in \ell^K_0$.
\end{theorem}

\begin{proof}
Part (a) follows directly from Theorem~\ref{thm:FourierEquivalence2} and the  change of variables~\eqref{e:changeofvariables}.
To prove part (b) we need to exclude the possibility that there is a nontrivial solution with $\epsilon=0$. The asserted bound on the ratio of $\alpha$ and $\omega$ guarantees, by Lemma~\ref{lem:Cone} (see also Remark~\ref{r:smalleps}), that indeed $\epsilon>0$ for any nontrivial solution. 
\end{proof}

We note that in practice (see Section~\ref{s:global}) a bound on $\| \c \|$ is derived from a bound on $y$ or $y'$ using Parseval's identity.

\begin{remark}\label{r:cone}
It follows from Theorem~\ref{thm:FourierEquivalence3} and Remark~\ref{r:smalleps} that for values of $(\alpha,\omega)$ near $(\pp,\pp)$ any reasonably bounded solution satisfies $\| c\| =  O(\epsilon)$ as well as $\|K^{-1} c \| = O(\epsilon)$ asymptotically (as $\epsilon \to 0$).
These bounds will be made explicit (and non-asymptotic) for specific choices of the parameters in Section~\ref{s:global}.
\end{remark}

% We are able to rule out such large amplitude solutions using global estimates such as those in \cite{neumaier2014global}.
% Hence, near the bifurcation point, the problem of describing periodic solutions of~\eqref{eq:Wright} reduces to studying the family of zeros finding problems $F_\epsilon=0$.





%Specifically, if a solution having $ \epsilon = 0$ does in fact correspond to a nontrivial periodic solution and $\alpha  < 2\omega $, then $ \| \tilde{c} \| > 2 \omega \alpha^{-1} -1$. 
%%PERHAPS THIS NEEDS A FORMULATION AS A THEOREM AS WELL?
%%IN OTHER WORDS: ARE WE SURE WE HAVE FOUND ALL ZEROS OF $\tilde{F}_0$, I.E. ALL SOLUTIONS WITH $\epsilon=0$ NEAR THE BIFURCATION POINT? AFTER RESCALING THESE ARE INVISIBLE?
%%THERE SHOULD BE A STATEMENT ABOUT THIS SOMEWHERE! EITHER HERE OR SOME





We finish this section by defining a curve of approximate zeros $\bx_\epsilon$ of $F_\epsilon$ 
(see \cite{chow1977integral,hassard1981theory}). 
%(see \cite{chow1977integral,morris1976perturbative,hassard1981theory}). 


\begin{definition}\label{def:xepsilon}
Let
\begin{alignat*}{1}
	\balpha_\epsilon &:= \pp + \tfrac{\epsilon^2}{5} ( \tfrac{3\pi}{2} -1)  \\
	\bomega_\epsilon &:= \pp -  \tfrac{\epsilon^2}{5} \\
	\bc_\epsilon 	 &:= \left(\tfrac{2 - i}{5}\right) \epsilon \,  \e_2 \,.
\end{alignat*}
We define the approximate solution 
$ \bx_\epsilon := \left( \balpha_\epsilon , \bomega_\epsilon  , \bc_\epsilon \right)$
for all $\epsilon \geq 0$.
\end{definition}

We leave it to the reader to verify that both 
 $F_\epsilon(\pp,\pp,\bc_{\epsilon})=\cO(\epsilon^2)$ and $F_\epsilon(\bx_\epsilon)=\cO(\epsilon^2)$.
%%%	
%%%	
%%%	}{Better like this?}
%%%\annote[J]{ $F_\epsilon(\bx_0)=\cO(\epsilon^2)$ and $F_\epsilon(\bx_\epsilon)=\cO(\epsilon^2)$.}{I think we'd still need the $ \bar{c}_\epsilon$ term in $\bar{x}_0$ to be of order $ \epsilon$.}
%%%\remove[JB]{We show in Proposition A.1
%%%%\ref{prop:ApproximateSolutionWorks} 
%%% that any $ x \in \R^2 \times \ell^1_0$ which is $ \cO(\epsilon^2)$ close to $ \bar{x}_\epsilon $ will yield the estimate $F_\epsilon(x) = \cO(\epsilon^2)$.
%%%Hence choosing $\{ \pp , \pp, \bar{c}_\epsilon\}$ as our approximate solution would also have been a natural choice for performing an $\cO(\epsilon^2)$ analysis and would have simplified several of our calculations.
%%%However,} 
%%%
We choose to use the more accurate approximation 
for the $ \alpha$ and $ \omega $ components to improve our final quantitative results. 














%
% Values for $ (\alpha, \omega,c)$ which approximately solve $\tilde{F}(\alpha,\omega,c) = 0$  are computed in  \cite{chow1977integral,morris1976perturbative,hassard1981theory} and are as follows:
%  \begin{eqnarray}
%  \tilde{\alpha}( \epsilon) &:=& \pi /2 + \tfrac{\epsilon^2}{5} ( \tfrac{3\pi}{2} -1) \nonumber \\
%  \tilde{\omega}( \epsilon) &:=& \pi /2 -  \tfrac{\epsilon^2}{5} \label{eq:ScaleApprox} \\
%  \tc(\epsilon) 	  &:=& \{ \left(\tfrac{2 - i}{5}\right)  \epsilon^2 , 0,0, \dots \} \nonumber
%  \end{eqnarray}
% In Appendix \ref{sec:OperatorNorms} we illustrate an alternative method for deriving this approximation.
%
%
%
%
% We want to solve $ \tilde{F}(\alpha , \omega, \hat{c}) =0$ for small values of $ \epsilon$.
% However $ D \tilde{F}(\alpha , \omega , c)$ is not invertible at $ ( \pp , \pp , 0)$ when $ \epsilon = 0$.
% In order for our asymptotic analysis to be non-degenerate, we need to make the change of variables $ c \mapsto \epsilon c$.
% Under this change of variables, we define the function $ F$ below so that $ \tilde{F}(\alpha , \omega , \epsilon c) =\epsilon  F( \alpha , \omega , c)$.
%
%
%
% \begin{definition}
% Construct an $\epsilon$-parameterized family of densely defined functions  $F : \R^2 \oplus \ell^1 / \C \to \ell^1$ by:
% \begin{equation}
% \label{eq:FDefinition}
% 	F(\alpha,\omega, c) :=
% 	[i \omega + \alpha e^{-i \omega}]_1 +
% 	( i \omega K^{-1} + \alpha U_{\omega}) c +
% 	[\epsilon \alpha e^{-i \omega}]_2  +
% 	\alpha \epsilon L_\omega c +
% 	\alpha \epsilon [ U_{\omega} c] * c.
% \end{equation}
% \end{definition}

%%
%%
%%\begin{corollary}
%%	\label{thm:FourierEquivalence3}
%%	Fix $ \epsilon > 0$, and $ c \in \ell^1 / \C $, and $ \omega >0$. Define $y: \R\to \R$ as 
%%	\[
%%	y(t) = 
%%	\epsilon \left( e^{i \omega t }  + e^{- i \omega t }\right) 
%%	+  \epsilon  \left( \sum_{k = 2}^\infty   c_k e^{i \omega k t }  + \overline{c}_k e^{- i \omega k t } \right) 
%%	\]
%%	and suppose that $ y(t) > -1$. 
%%	Then $y(t)$ solves Wright's equation at parameter $ \alpha > 0 $ if and only if $ F( \alpha , \omega , c) = 0$ at parameter $ \epsilon$. 
%%	
%%	
%%	
%%\end{corollary}
%%
%%
%%\begin{proof}
%%	Since $ \tilde{F}(\alpha,\omega, \epsilon c) = \epsilon F( \alpha , \omega , c)$, the result follows from Theorem \ref{thm:FourierEquivalence2}.
%%\end{proof}

% If we can find $(\alpha , \omega, c)$ for which $ F( \alpha , \omega,c)=0$ at parameter $\epsilon$, then $ \tilde{F}(\alpha ,\omega, c)=0$.
% By Theorem \ref{thm:FourierEquivalence2} this amounts to finding a periodic solution to Wright's equation.
% Lastly, because we have performed the change of variables $ c \mapsto \epsilon c$, we need to  apply this change of variables to our approximate solution as well.
%
% \begin{definition}
% 	Define the approximate solution $ x( \epsilon) = \left\{ \alpha(\epsilon ) , \omega ( \epsilon ) , c(\epsilon) \right\}$ as below,  where $c(\epsilon) = \{ c_2( \epsilon) , 0 ,0 , \dots\} $.
% 	We may also write $ x_\epsilon = x(\epsilon) $.
% 	\begin{eqnarray}
% 	\alpha( \epsilon) &:=& \pi /2 + \tfrac{\epsilon^2}{5} ( \tfrac{3\pi}{2} -1) \nonumber \\
% 	\omega( \epsilon) &:=& \pi /2 -  \tfrac{\epsilon^2}{5} \label{eq:Approx} \\
% 	c_2(\epsilon) 	  &:=& \left(\tfrac{2 - i}{5}\right) \epsilon \nonumber
% 	\end{eqnarray}
%
% \end{definition}

\section{Upper Bounds for Identity Testing}
\label{sec:ones-ub}
In this section, we prove the following theorems for identity testing.
\begin{theorem}\label{thm:ones-csq-h}
There exists an algorithm for identity testing between $p$ and $q$ distinguishing the cases:
\begin{itemize}
\item $\dxs(p,q) \leq \ve^2$;
\item $\dh(p,q) \geq \ve$.
\end{itemize}
The algorithm uses $O\left(\frac{n^{1/2}}{\ve^2}\right)$ samples.
\end{theorem}

\begin{theorem}\label{thm:ones-tv}
There exists an algorithm for identity testing between $p$ and $q$ distinguishing the cases:
\begin{itemize}
\item $\dlt(p,q) \leq \frac{\ve}{\sqrt{n}}$;
\item $\dtv(p,q) \geq \ve$.
\end{itemize}
The algorithm uses $O\left(\frac{n^{1/2}}{\ve^2}\right)$ samples.
\end{theorem}

\begin{theorem}\label{thm:ones-h}
There exists an algorithm for identity testing between $p$ and $q$ distinguishing the cases:
\begin{itemize}
\item $\dlt(p,q) \leq \frac{\ve^2}{\sqrt{n}}$;
\item $\dh(p,q) \geq \ve$.
\end{itemize}
The algorithm uses $O\left(\frac{n^{1/2}}{\ve^2}\right)$ samples.
\end{theorem}

We prove Theorem~\ref{thm:ones-csq-h} in Section~\ref{sec:id-csq-h}, and Theorems~\ref{thm:ones-tv} and~\ref{thm:ones-h} in Section~\ref{sec:id-lt}.

\subsection{Identity Testing with Hellinger Distance and $\chi^2$-Tolerance}
\label{sec:id-csq-h}

We prove Theorem~\ref{thm:ones-csq-h} by analyzing Algorithm~\ref{alg:testing}.
We will set $c_1 = \frac{1}{100}, c_2 = \frac{6}{25}$, and let $C$ be a sufficiently large constant.
\begin{algorithm}[h]
\caption{$\chi^2$-close versus Hellinger-far testing algorithm}\label{alg:testing}
\begin{algorithmic}[1]
\State \textbf{Input:} $\ve$; an explicit distribution $q$; sample access to a distribution $p$
\State Implicitly define $\mathcal{A} \leftarrow \{i:q_i \geq c_1\ve^2/n\}$, $\mathcal{\bar A} \leftarrow [n] \setminus \mathcal{A}$
\State Let $\hat p$ be the empirical distribution\footnote{The empirical distribution is defined by taking a set of samples and normalizing the counts such that the result forms a probability distribution.} from drawing $m_1 = \Theta(1/\ve^2)$ samples from $p$
\If {$\hat p(\mathcal{\bar A}) \geq \frac34 c_2\ve^2$} \label{ln:light-test}
\State \Return \reject \label{ln:early-reject}
\EndIf
\State Draw a multiset $S$ of $\mathrm{Poisson}(m_2)$ samples from $p$, where $m_2 = C\sqrt{n}/\ve^2$
\State Let $N_i$ be the number of occurrences of the $i$th domain element in $S$
\State Let $S'$ be the set of domain elements observed in $S$
\State $Z \leftarrow \sum_{i \in S' \cap \mathcal{A}} \frac{(N_i - m_2q_i)^2 - N_i}{m_2q_i} + m_2 (1 - q(S' \cap \mathcal{A}))$ \label{ln:statistic}
\If {$Z \leq \frac{3}{2}m_2\ve^2$}
\State \Return \accept
\Else 
\State \Return \reject
\EndIf 
\end{algorithmic}
\end{algorithm}

We note that the sample and time complexity are both $O(\sqrt{n}/\ve^2)$.
We draw $m_1 + m_2 = \Theta(\sqrt{n}/\ve^2)$ samples total.
All steps of the algorithm only involve inspecting domain elements where a sample falls, and it runs linearly in the number of such elements.
Indeed, Step~\ref{ln:statistic} of the algorithm is written in an unusual way in order to ensure the running time of the algorithm is linear.

We first analyze the test in Step \ref{ln:light-test} of the algorithm.
Folklore results state that with probability at least $99/100$, this preliminary test will reject any $p$ with $p(\mathcal{\bar A}) \geq c_2 \ve^2$, it will not reject any $p$ with $p(\mathcal{\bar A}) \leq \frac{c_2}{2} \ve^2$, and behavior for any other $p$ is arbitrary.
Condition on the event the test does not reject for the remainder of the proof.
Note that since both thresholds here are $\Theta(\ve^2)$, it only requires $m_1 = \Theta(1/\ve^2)$ samples, rather than the ``non-extreme'' regime, where we would require $\Theta(1/\ve^4)$ samples.

\begin{remark}
We informally refer to this ``extreme'' versus ``non-extreme'' regime in distribution testing.
To give an example of what we mean in these two cases, consider distinguishing $Ber(1/2)$ from $Ber(1/2 + \ve)$.
The complexity of this problem is $\Theta(1/\ve^2)$, and we consider this to be in the non-extreme regime.
On the other hand, distinguishing $Ber(\ve)$ from $Ber(2\ve)$ has a sample complexity of $\Theta(1/\ve)$, and we consider this to be in the extreme regime.
\end{remark}

We justify that any $p$ which may be rejected in Step \ref{ln:early-reject} (i.e., any $p$ such that $p(\mathcal{\bar A}) > \frac{c_2}{2} \ve^2$) has the property that $\dxs(p,q) > \ve^2$ (in other words, we do not wrongfully reject any $p$).

Consider a $p$ such that $p(\mathcal{\bar A}) \geq \frac{c_2}{2}\ve^2$.
Note that $\dxs(p, q) \geq \dxs(p_\mathcal{\bar A}, q_\mathcal{\bar A})$, which we lower bound as follows:
\begin{align*}
\dxs(p_\mathcal{\bar A}, q_\mathcal{\bar A})
&= \sum_{i \in \mathcal{\bar A}} \frac{(p_i - q_i)^2}{q_i} \\
&\geq \frac{n}{c_1 \ve^2} \sum_{i \in \mathcal{\bar A}} (p_i - q_i)^2 \\
&\geq \frac{n}{c_1 \ve^2} \cdot \frac{1}{n} \left(\sum_{i \in \mathcal{\bar A}} (p_i - q_i) \right)^2  \\
&\geq \frac{n}{c_1 \ve^2} \frac{\ve^4\left(\frac{c_2}{2} - c_1\right)^2}{n} \\
&= \frac{\left(\frac{c_2}{2} - c_1\right)^2}{c_1}\ve^2
\end{align*}
The first inequality is by the definition of $\mathcal{\bar A}$, the second is by Cauchy-Schwarz, and the third is since $p(\mathcal{\bar A}) \geq \frac{c_2}{2}\ve^2$ and $q(\mathcal{\bar A}) \leq c_1\ve^2$.
By our setting of $c_1$ and $c_2$, this implies that $\dxs(p, q) > \ve^2$, and we are not rejecting any $p$ which should be accepted.

For the remainder of the proof, we will implicitly assume that $p(\mathcal{\bar A}) \leq c_2 \ve^2$.


Let
$$Z' = \sum_{i \in \mathcal{A}} \frac{(N_i - m_2 q_i)^2 - N_i}{m_2q_i}.$$

Note that the statistic $Z$ can be rewritten as follows:
\begin{align*}
Z &= \sum_{i \in S' \cap \mathcal{A}} \frac{(N_i - m_2q_i)^2 - N_i}{m_2q_i} + m_2 (1 - q(S' \cap \mathcal{A})) \\
  &= \sum_{i \in S' \cap \mathcal{A}} \frac{(N_i - m_2q_i)^2 - N_i}{m_2q_i} + \sum_{i \in \mathcal{A} \setminus S'} m_2 q_i + m_2  q(\mathcal{\bar A}) \\
  &= \sum_{i \in S' \cap \mathcal{A}} \frac{(N_i - m_2q_i)^2 - N_i}{m_2q_i} + \sum_{i \in \mathcal{A} \setminus S'} \frac{(N_i - m_2q_i)^2 - N_i}{m_2q_i} + m_2  q(\mathcal{\bar A}) \\
  &= Z' + m_2 q(\mathcal{\bar A})
\end{align*}

We proceed by analyzing $Z'$.
First, note that it has the following expectation and variance:
\begin{align}
\E[Z'] &= m_2 \cdot \sum_{i \in \mathcal{A}} \frac{(p_i - q_i)^2}{q_i} = m_2 \cdot \dxs(p_\mathcal{A}, q_\mathcal{A}) \label{eqn:mean} \\
\Var[Z'] &= \sum_{i \in \mathcal{A}} \left[2\frac{p_i^2}{q_i^2} + 4m_2 \cdot \frac{p_i \cdot (p_i - q_i)^2}{q_i^2}\right] \label{eqn:variance}
\end{align}
These properties are proven in Section A of~\cite{AcharyaDK15}.

We require the following two lemmas, which state that the mean of the statistic is separated in the two cases, and that the variance is bounded.
The proofs largely follow the proofs of two similar lemmas in~\cite{AcharyaDK15}.
\begin{lemma}
\label{lem:means}
If $\dxs(p,q) \leq \ve^2$, then $\E[Z'] \leq m_2 \ve^2$. 
If $\dh(p,q) \geq \ve$, then $\E[Z'] \geq (2 - c_1 - c_2)m_2 \ve^2$.
\end{lemma}
\begin{proof}
The former case is immediate from (\ref{eqn:mean}).

For the latter case, note that
$$\dh^2(p,q) = \dh^2(p_\mathcal{A}, q_\mathcal{A}) + \dh^2(p_\mathcal{\bar A}, q_\mathcal{\bar A}).$$
We upper bound the latter term as follows:
\begin{align*}
\dh^2(p_\mathcal{\bar A}, q_\mathcal{\bar A}) 
&\leq \dtv(p_\mathcal{\bar A}, q_\mathcal{\bar A}) \\
&= \frac12 \sum_{i \in \mathcal{\bar A}} |p_i - q_i| \\
&\leq \frac12 \left(p(\mathcal{\bar A}) + q(\mathcal{\bar A})\right) \\
&\leq \left(\frac{c_1 + c_2}{2}\right)\ve^2 \\
\end{align*}
The first inequality is from Proposition \ref{prop:distanceinequalities}, and the third inequality is from our prior condition that $p(\mathcal{\bar A}) \leq c_2 \ve^2$.

Since $\dh^2(p,q) \geq \ve^2$, this implies $\dh^2(p_\mathcal{A}, q_\mathcal{A}) \geq \left(1 - \frac{c_1 + c_2}{2}\right)\ve^2$.
Proposition \ref{prop:distanceinequalities} further implies that $\dxs(p_\mathcal{A}, q_\mathcal{A}) \geq \left(2 - c_1 - c_2\right)\ve^2$.
The lemma follows from (\ref{eqn:mean}).
\end{proof}


\begin{lemma}
\label{lem:vars}
If $\dxs(p,q) \leq \ve^2$, then $\Var[Z'] = O(m_2^2 \ve^4)$. 
If $\dh(p,q) \geq \ve$, then $\Var[Z'] \leq O(\E[Z']^2)$.
The constant in both expressions can be made arbitrarily small with the choice of the constant $C$.
\end{lemma}
\begin{proof}
We bound the terms of (\ref{eqn:variance}) separately, starting with the first.

\begin{align}
2\sum_{i \in \mathcal{A}} \frac{p_i^2}{q_i^2} &= 2\sum_{i \in \mathcal{A}} \left(\frac{(p_i - q_i)^2}{q_i^2} + \frac{2p_iq_i - q_i^2}{q_i^2}\right) \nonumber \\
                                             &= 2\sum_{i \in \mathcal{A}} \left(\frac{(p_i - q_i)^2}{q_i^2} + \frac{2q_i(p_i - q_i) + q_i^2}{q_i^2}\right) \nonumber\\
                                             &\leq 2n + 2\sum_{i \in \mathcal{A}} \left(\frac{(p_i - q_i)^2}{q_i^2} + 2\frac{(p_i - q_i)}{q_i}\right) \nonumber\\
                                             &\leq 4n + 4\sum_{i \in \mathcal{A}} \frac{(p_i - q_i)^2}{q_i^2} \nonumber\\
                                             &\leq 4n + \frac{4n}{c_1\ve^2} \sum_{i \in \mathcal{A}} \frac{(p_i - q_i)^2}{q_i}\nonumber\\
                                             &= 4n + \frac{4n}{c_1\ve^2}\frac{E[Z']}{m_2} \nonumber\\
                                             &\leq 4n + \frac{4}{c_1C}\sqrt{n} E[Z']\label{eq:first-var-term-in}
\end{align}
The second inequality is the AM-GM inequality, the third inequality uses that $q_i \geq \frac{c_1\ve^2}{n}$ for all $i \in \mathcal{A}$, the last equality uses \eqref{eqn:mean}, and the final inequality substitutes a value $m_2 \geq C\frac{\sqrt{n}}{\ve^2}$.

The second term can be similarly bounded:
\begin{align*}
4m_2 \sum_{i \in \mathcal{A}} \frac{p_i(p_i - q_i)^2}{q_i^2} &\leq 4m_2 \left(\sum_{i \in \mathcal{A}} \frac{p_i^2}{q_i^2}\right)^{1/2}\left(\sum_{i \in \mathcal{A}} \frac{(p_i - q_i)^4}{q_i^2}\right)^{1/2} \\
                                                          &\leq 4m_2 \left(4n + \frac{4}{c_1C}\sqrt{n} E[Z'] \right)^{1/2}\left(\sum_{i \in \mathcal{A}} \frac{(p_i - q_i)^4}{q_i^2}\right)^{1/2} \\
                                                          &\leq 4m_2 \left(2\sqrt{n} + \frac{2}{\sqrt{c_1C}}n^{1/4} E[Z']^{1/2}\right)\left(\sum_{i \in \mathcal{A}} \frac{(p_i - q_i)^2}{q_i}\right) \\
                                                          &= \left(8\sqrt{n} + \frac{8}{\sqrt{c_1C}}n^{1/4} E[Z']^{1/2}\right)E[Z'].
\end{align*}
The first inequality is Cauchy-Schwarz, the second inequality uses (\ref{eq:first-var-term-in}), the third inequality uses the monotonicity of the $\ell_p$ norms, and the equality uses~\eqref{eqn:mean}.

Combining the two terms, we get
$$\Var[Z'] \leq 4n + \left(8 + \frac{4}{c_1C}\right)\sqrt{n} \E[Z']  + \frac{8}{\sqrt{c_1C}}n^{1/4} \E[Z']^{3/2}  .$$

We now consider the two cases in the statement of our lemma.
\begin{itemize}
\item
When $\dxs(p,q) \leq \ve^2$, we know from Lemma~\ref{lem:means} that $\E[Z'] \leq m_2 \ve^2$. 
Combined with a choice of $m_2 \geq C \frac{\sqrt{n}}{\ve^2}$ and the above expression for the variance, this gives:
\begin{align*}
\Var[Z']
& \leq \frac{4}{C^2}m_2^2\ve^4 + \left(\frac{8}{C} + \frac{4}{c_1C^2}\right)m_2^2 \ve^4+ \frac{8}{C\sqrt{c_1}}m_2^2 \ve^4 \\
& = \left(\frac{8}{C} + \frac{8}{C\sqrt{c_1}} + \frac{4}{C^2} + \frac{4}{c_1C^2} \right)m_2^2 \ve^4 = O(m_2^2 \ve^4).
\end{align*}

\item When $\dh(p,q) \geq \ve$, Lemma~\ref{lem:means} and  $m_2 \geq C\frac{\sqrt{n}}{\ve^2}$ give:
$$\E[Z'] \geq (2 - c_1 - c_2) m_2 \ve^2 \geq C(2 - c_1 - c_2) \sqrt{n}.$$

Similar to before, combining this with our expression for the variance we get:
\begin{align*}
\Var[Z']
&\leq \left(\frac{8}{C(2 -c_1 - c_2)} + \frac{8}{C\sqrt{c_1 (2 - c_1 - c_2)}} +  \frac{4}{C^2(2-c_1-c_2)^2} + \frac{4}{C^2c_1(2 - c_1 - c_2)} \right) \E[Z']^2 \\
&= O(\E[Z']^2).\qedhere
\end{align*}
\end{itemize}
\end{proof}

To conclude the proof, we consider the two cases.
\begin{itemize}
\item Suppose $\dxs(p,q) \leq \ve^2$. 
By Lemma~\ref{lem:means} and the definition of $\mathcal{A}$, we have that $\E[Z] \leq (1 + c_1)m_2\ve^2$. 
By Lemma~\ref{lem:vars}, $\Var[Z] = O(m_2^2\ve^4)$.
Therefore, for constant $C$ sufficiently large, Chebyshev's inequality implies $\Pr(Z > \frac32 m_2 \ve^2) \leq 1/10$.
\item
Suppose $\dh(p,q) \geq \ve$.
By Lemma~\ref{lem:means}, we have that $\E[Z'] \geq (2 - c_1 - c_2)m_2\ve^2$. 
By Lemma~\ref{lem:vars}, $\Var[Z'] = O(\E[Z']^2)$.
Therefore, for constant $C$ sufficiently large, Chebyshev's inequality implies $\Pr(Z' < \frac32 m_2 \ve^2) \leq 1/10$.
Since $Z \geq Z'$, $\Pr(Z < \frac32 m_2 \ve^2) \leq 1/10$ as well.
\end{itemize}

\subsection{Identity Testing with $\ell_2$ Tolerance}
\label{sec:id-lt}
In this section, we sketch the algorithms required to achieve $\ell_2$ tolerance for identity testing.
Since the algorithms and analysis are very similar to those of Algorithm 1 of~\cite{AcharyaDK15} and Algorithm~\ref{alg:testing}, the full details are omitted.

First, we prove Theorem~\ref{thm:ones-tv}.
The algorithm is Algorithm 1 of~\cite{AcharyaDK15}, but instead of testing on $p$ and $q$, we instead test on $p^{+\frac12}$ and $q^{+\frac12}$, as defined in Proposition~\ref{prop:mixing}.
By this proposition, this operation preserves total variation and $\ell_2$ distance, up to a factor of $2$, and also makes it so that the minimum probability element of $q^{+\frac12}$ is at least $1/2n$. 
In the case where $\dlt(p,q) \leq \frac{\ve}{\sqrt{n}}$, we have the following upper bound on $\E[Z]$:
$$\E[Z'] = m \sum_{i \in \mathcal{\bar A}} \frac{(p_i - q_i)^2}{q_i} \leq O\left( m \cdot n \cdot \dlt^2(p,q)\right) \leq O(m_2 \ve^2).$$
This is the same bound as in Lemma 2 of~\cite{AcharyaDK15}.
The rest of the analysis follows identically to that of Algorithm 1 of~\cite{AcharyaDK15}, giving us Theorem~\ref{thm:ones-tv}.

Next, we prove Theorem~\ref{thm:ones-h}.
We observe that Algorithm~\ref{alg:testing} as stated can be considered as $\ell_2$-tolerant instead of $\chi^2$-tolerant, if desired.
First, we do not wrongfully reject any $p$ (i.e., those with $\dlt(p,q) \leq \frac{\ve^2}{\sqrt{n}}$) in Step~\ref{ln:early-reject}.
This is because we reject in this step if there is $\geq \Omega(\ve^2)$ total variation distance between $p$ and $q$ (witnessed by the set $\mathcal{\bar A}$), which implies that $p$ and $q$ are far in $\ell_2$-distance by Proposition~\ref{prop:ltinequalities}.
It remains to prove an upper bound on $\E[Z']$ in the case where $\dlt(p,q) \leq \frac{\ve^2}{\sqrt{n}}$.
$$\E[Z'] = m_2 \dxs(p,q) = m_2 \sum_{i \in \mathcal{\bar A}} \frac{(p_i - q_i)^2}{q_i} \leq O\left( m_2 \cdot \left(\frac{n}{\ve^2}\right)\cdot \dlt^2(p,q)\right) \leq O(m_2 \ve^2).$$
We note that this is the same bound as in Lemma~\ref{lem:means}.
With this bound on the mean, the rest of the analysis is identical to that of Theorem~\ref{thm:ones-csq-h}, giving us Theorem~\ref{thm:ones-h}.

\section{Upper Bounds for Equivalence Testing}
\label{sec:twos-ub}
In this section, we prove the following theorems for equivalence testing.

\begin{theorem}\label{thm:twos-tv}
There exists an algorithm for equivalence testing between $p$ and $q$ distinguishing the cases:
\begin{itemize}
\item $\dlt(p,q) \leq \frac{\ve}{2\sqrt{n}}$ 
\item $\dtv(p,q) \geq \ve$
\end{itemize}
The algorithm uses $O\left(\max\left\{\frac{n^{2/3}}{\ve^{4/3}}, \frac{n^{1/2}}{\ve^2}\right\}\right)$ samples.
\end{theorem}

\begin{theorem}\label{thm:twos-h}
There exists an algorithm for equivalence testing between $p$ and $q$ distinguishing the cases:
\begin{itemize}
\item $\dlt(p,q) \leq \frac{\ve^2}{32\sqrt{n}}$ 
\item $\dh(p,q) \geq \ve$
\end{itemize}
The algorithm uses $O\left(\min\left\{\frac{n^{2/3}}{\ve^{8/3}}, \frac{n^{3/4}}{\ve^2}\right\}\right)$ samples.
\end{theorem}



Consider drawing~$\mathrm{Poisson}(m)$ samples from two unknown distributions $p = (p_1, \ldots, p_n)$ and $q = (q_1, \ldots, q_n)$.
Given the resulting histograms~$\bX$ and~$\bY$, \cite{ChanDVV14} define the following statistic:
\begin{equation}\label{eq:hel-statistic}
\bZ = \sum_{i=1}^n \frac{(\bX_i - \bY_i)^2 - \bX_i - \bY_i}{\bX_i + \bY_i}.
\end{equation}
This can be viewed as a modification to the empirical triangle distance applied to~$\bX$ and~$\bY$.
Both of our equivalence testing upper bounds will be obtained by appropriate thresholding of the statistic $\bZ$.

The organization of this section is as follows.
In Section~\ref{sec:eq-prelim}, we prove some basic properties of $\bZ$.
In Section~\ref{sec:eq-tv}, we prove Theorem~\ref{thm:twos-tv}.
In Section~\ref{sec:eq-h}, we prove Theorem~\ref{thm:twos-h}.

\subsection{Some facts about $\mathbf Z$}
\label{sec:eq-prelim}
Chan et al.~\cite{ChanDVV14} give the following expressions for the mean and variance of~$\bZ$.

\begin{proposition}[\cite{ChanDVV14}]\label{prop:mean-var}
Consider the function
\begin{equation*}
f(x) = \left(1 - \frac{1- e^{-x}}{x}\right).
\end{equation*}
Then for any subset $A\subseteq [n]$,
\begin{equation}\label{eq:z-mean}
\E[\bZ_A] = \sum_{i \in A} \frac{(p_i-q_i)^2}{p_i+q_i} m \cdot f(m(p_i + q_i)).
\end{equation}
As a result, $\bZ$ is mean-zero when $p= q$.  Furthermore,
\begin{equation*}
\Var[\bZ] \leq 2 \min\{m, n\} + \sum_{i=1}^n 5m \frac{(p_i - q_i)^2}{p_i + q_i}.
\end{equation*}
\end{proposition}
\noindent
Applying Proposition~\ref{prp:didnt-know-it-was-hellinger}, we immediately have the following corollary.
\begin{corollary}\label{cor:var-hel}
$\displaystyle
\Var[\bZ] \leq 2 \min\{m, n\} +  20m \dh(p,q)^2.
$
\end{corollary}

Without the corrective factor of $f(m(p_i + q_i))$,
Equation~\eqref{eq:z-mean} would just be~$m$ times the triangle distance between~$p$ and~$q$.
Our goal then is to understand the function $f(x)$ and how it affects this quantity.
Aside from the removable discontinuity at $x=0$, $f$ is a monotonically increasing function,
and for $x > 0$, it is strictly bounded between~$0$ and~$1$.
Furthermore, for~$x > 0$ there are roughly two ``regimes" that $f(x)$ exhibits:
when $x < 1$, where $f(x)$ is well-approximated by $x/2$,
and when $x \geq 1$, where $f(x)$ is ``morally the constant one,'' slowly increasing from~$e^{-1}$ to~$1$.
In fact, we have the following explicit bound on~$f(x)$.
\begin{fact}\label{fact:f-upper}
For all $x > 0$, $f(x) \leq \min\{1, x\}.$
\end{fact}
\noindent
In terms of $f(m(p_i + q_i))$, these regimes correspond to whether $p_i + q_i$ is less than or greater than~$\frac{1}{m}$.
Hence, the expression for the mean of~$\bZ$ (i.e.\ Equation~\eqref{eq:z-mean} for $A = [n]$)
splits in two: those terms for ``large" $p_i + q_i$ look roughly like the triangle distance (times~$m$),
and those terms for ``small" $p_i + q_i$ look roughly like the $\ell_2^2$ distance (times~$m^2$).
This is why we have given ourselves the flexibility to consider subsets~$A$ of the domain.

We will now prove several upper and lower bounds on $\E[\bZ_A]$,
based in part on whether we will apply them in the large or small $p_i + q_i$ regime.
Let us begin with a pair of upper bounds.

\begin{proposition}\label{prop:well-conditioned-upper}
Suppose for every $i \in A$, $p_i + q_i \geq \delta$. Then
\begin{equation*}
\E[\bZ_A] \leq \frac{m}{\delta} \dlt^2(p_A, q_A).
\end{equation*}
\end{proposition}
\begin{proof}
Because $f(x) \leq 1$ for all $x > 0$,
\begin{equation*}
\E[\bZ_A]
= \sum_{i \in A} \frac{(p_i-q_i)^2}{p_i+q_i} m \cdot f(m(p_i + q_i))
\leq  \sum_{i \in A} \frac{(p_i-q_i)^2}{p_i+q_i} m
\leq \frac{m}{\delta} \sum_{i \in A} (p_i-q_i)^2
=  \frac{m}{\delta} \dlt^2(p_A, q_A).\qedhere
\end{equation*}
\end{proof}

\begin{proposition}\label{prop:general-upper}
$\displaystyle
\E[\bZ] \leq m^2 \dlt^2(p, q).
$
\end{proposition}
\begin{proof}
Let $L$ be the set of $i$ such that $m(p_i + q_i) \geq 1$.
Then $\E[\bZ] = \E[\bZ_L] + \E[\bZ_{\overline{L}}]$, and by Proposition~\ref{prop:well-conditioned-upper},
$\E[\bZ_L] \leq m^2 \dlt^2(p_L, q_L)$.
On the other hand, by Fact~\ref{fact:f-upper}, $f(x) \leq x$, and therefore
\begin{equation*}
\E[\bZ_{\overline{L}}]
= \sum_{i \in \overline{L}} \frac{(p_i-q_i)^2}{p_i+q_i} m \cdot f(m(p_i + q_i))
\leq \sum_{i \in \overline{L}} (p_i-q_i)^2 m^2
= m^2 \dlt^2(p_{\overline{L}}, q_{\overline{L}}).
\end{equation*}
The proof is completed by noting that $\dlt^2(p_L, q_L) + \dlt^2(p_{\overline{L}}, q_{\overline{L}}) = \dlt^2(p, q).$
\end{proof}

Now we give a pair of lower bounds.

\begin{proposition}\label{prop:pretty-much-a-trivial-lower-bound-i-dont-know-what-to-tell-you}
Suppose for every $i \in A$, $m(p_i + q_i) \geq 1$.  Then
\begin{equation*}
\E[\bZ_A] \geq \frac{2m}{3} \dh^2(p_A,q_A).
\end{equation*}
\end{proposition}
\begin{proof}
Because $f(x)$ is monotonically increasing and $f(1) = 1/e$,
\begin{equation*}
\E[\bZ_A]
= m \sum_{i \in A} \frac{(p_i-q_i)^2}{p_i+q_i} f(m(p_i+q_i))
\geq m \sum_{i \in A} \frac{(p_i-q_i)^2}{p_i+q_i} f(1)
\geq \frac{2m}{e}\dh^2(p_A,q_A),
\end{equation*}
where the first step is by Proposition~\ref{prop:mean-var} and the last is by Proposition~\ref{prp:didnt-know-it-was-hellinger}.
The result follows from $e \leq 3$.
\end{proof}

The next proposition is essentially the second half of the proof of Lemma~$4$ from~\cite{ChanDVV14}.

\begin{proposition}\label{prop:their-bound}
For any subset~$A$,
\begin{equation*}
\E[\bZ_A] \geq \left(\frac{4m^2}{2|A| + m\cdot(p(A) + q(A))}\right)\cdot \dtv^2(p_A,q_A),
\end{equation*}
where we write $p(A) = \sum_{i \in A} p(i)$ and likewise for~$q(A)$.
\end{proposition}
\begin{proof}
Consider the function $g(x) = x f(x)^{-1}$.
Then $g(x) \leq 2+x$ for nonnegative~$x$.
Furthermore,
\begin{equation*}
\frac{(p_i - q_i)^2}{g(m(p_i + q_i))} = \frac{(p_i-q_i)^2}{m(p_i+q_i)} \left(1 - \frac{1-e^{-m(p_i+q_i)}}{m(p_i + q_i)}\right),
\end{equation*}
which, from Proposition~\ref{prop:mean-var}, is $\frac{1}{m^2} \cdot \E[\bZ_{\{i\}}]$.
As a result,
\begin{multline*}
\dtv^2(p_A,q_A)
= \frac14 \left(\sum_{i \in A} |p_i - q_i|\right)^2
= \frac14 \left(\sum_{i \in A} |p_i - q_i|\cdot \frac{\sqrt{g(m(p_i+q_i))}}{\sqrt{g(m(p_i+q_i))}}\right)^2\\
\leq \frac14\left(\sum_{i \in A} \frac{(p_i - q_i)^2}{g(m(p_i + q_i))}\right) \cdot \left(\sum_{i \in A} g(m(p_i + q_i))\right)
\leq \frac{1}{4m^2} \cdot \E[\bZ_A] \cdot(2|A| + m\cdot (p(A) + q(A))),
\end{multline*}
where the first inequality is Cauchy-Schwarz. Rearranging finishes the proof.
\end{proof}


\subsection{Equivalence Testing with Total Variation Distance}
\label{sec:eq-tv}
In this section, we prove Theorem~\ref{thm:twos-tv}. 
We will take the number of samples to be 
\begin{equation}\label{eq:l1-max}
m = \max\left\{C\cdot \frac{n^{2/3}}{\epsilon^{4/3}}, C^{3/2}\cdot \frac{n^{1/2}}{\epsilon^2}\right\},
\end{equation}
where~$C$ is some constant which can be taken to be~$10^{10}$.
 
Rather than drawing samples from~$p$ or~$q$,
our algorithm draws samples from~$p^{+1/2}$ and~$q^{+1/2}$.
By Proposition~\ref{prop:mixing}, we have the following guarantees in the two cases:
\begin{equation*}
\text{(Case 1):}~\dlt(p^{+1/2},q^{+1/2}) \leq \frac{\epsilon}{4 \sqrt{n}},
\qquad
\text{(Case 2):}~\dtv(p^{+1/2},q^{+1/2}) \geq \frac{\epsilon}{2}.
\end{equation*}
Furthermore, for any $i \in [n]$, we know the $i$-th coordinates of $p^{+1/2}$ and~$q^{+1/2}$ are both at least~$\frac{1}{2n}$.
Henceforth, we will write~$p'$ and~$q'$ for~$p^{+1/2}$ and~$q^{+1/2}$, respectively.

In Case 1, if we apply Proposition~\ref{prop:well-conditioned-upper} with $A = [n]$ and $\delta = \frac{1}{n}$
and Proposition~\ref{prop:general-upper},
\begin{equation*}
\E[\bZ]
\leq \min\{m^2, mn\} \cdot \dlt^2(p',q')
\leq \min\{m^2, mn\} \cdot \frac{\epsilon^2}{16 n}
\leq \frac{m^2}{4(2m + 2n)} \cdot \epsilon^2.
\end{equation*}
On the other hand, in Case 2, applying Proposition~\ref{prop:their-bound} with $A = [n]$,
\begin{equation*}
\E[\bZ] \geq \frac{4m^2}{2m + 2n} \cdot \dtv(p',q')^2 \geq \frac{m^2}{2m + 2n} \cdot \epsilon^2.
\end{equation*}
Our algorithm therefore thresholds~$\bZ$ on the value $\frac{5 m^2}{8(2m +2n)} \epsilon^2$,
outputting ``close" if it's below this value and ``far" otherwise.

The two bounds in~\eqref{eq:l1-max} meet when $C^3 \epsilon^{-4} = n$,
which is exactly when $m = n$.
When $m \leq n$, the first bound applies, and when $m > n$ the second bound applies.
As a result, we will split our analysis into the two cases.

\begin{lemma}
The tester succeeds in the $m \leq n$ case of Theorem~\ref{thm:twos-tv}.
\end{lemma}
\begin{proof}
By Corollary~\ref{cor:var-hel}
\begin{equation*}
\Var[\bZ] \leq 2 \min\{m, n\} +  20m \dh(p',q')^2
\leq 22m,
\end{equation*}
where we used the fact that $\dh(p',q') \leq 1$.
In Case 1, by Chebyshev's inequality,
\begin{equation*}
\Pr\left[\bZ \geq \frac{5 m^2}{8(2m +2n)} \epsilon^2\right]
\leq \frac{\Var[\bZ]}{\left(\frac{3m^2}{8(2m +2n)} \epsilon^2\right)^2}
= O\left(\frac{m}{\frac{m^4}{n^2} \epsilon^4}\right)
= O\left(\frac{n^2}{m^3 \epsilon^4}\right).
\end{equation*}
In Case 2,
\begin{equation*}
\Pr\left[\bZ \leq \frac{5 m^2}{8(2m +2n)} \epsilon^2\right]
\leq \frac{64 \Var[\bZ]}{9\E[\bZ]^2}
= O\left(\frac{m}{\frac{m^4}{n^2} \epsilon^4}\right)
= O\left(\frac{n^2}{m^3 \epsilon^4}\right).
\end{equation*}
Both of these bounds can be made arbitrarily small constants by setting~$C$ sufficiently large.
\end{proof}

\begin{lemma}
The tester succeeds in the $m \geq n$ case of Theorem~\ref{thm:twos-tv}.
\end{lemma}
\begin{proof}
We first consider Case 1.
By Proposition~\ref{prop:mean-var},
\begin{equation*}
\Var[\bZ]
\leq 2 \min\{m, n\} + \sum_{i=1}^n 5m \frac{(p_i' - q_i')^2}{p_i' + q_i'}
\leq 2 n + 5 m n \dlt^2(p',q')
\leq 2 n + \tfrac{5}{16} m \epsilon^2.
\end{equation*}
Then, we have that
\begin{equation*}
\Pr\left[\bZ \geq \frac{5 m^2}{8(2m +2n)} \epsilon^2\right]
\leq \frac{\Var[\bZ]}{\left(\frac{3m^2}{8(2m +2n)} \epsilon^2\right)^2}
= O\left(\frac{n}{m^2 \epsilon^4} + \frac{m\epsilon^2}{m^2 \epsilon^4}\right)
= O\left(\frac{n}{m^2 \epsilon^4} + \frac{1}{m \epsilon^2}\right).
\end{equation*}
Next, we focus on Case 2.
Write $L$ for the set of $i \in [n]$ such that $m(p_i' + q_i') \geq 1$.
Then $\dh^2(p_{\overline{L}}',q_{\overline{L}}') \leq \frac12 \sum_{i \in \overline{L}} (p_i' + q_i') \leq n/2m$.
As a result, 
by Corollary~\ref{cor:var-hel}
\begin{equation*}
\Var[\bZ] \leq 2 \min\{m, n\} +  20m \dh^2(p',q')
\leq 12 n + 20m \dh^2(p_L',q_L'). 
\end{equation*}
By Proposition~\ref{prop:pretty-much-a-trivial-lower-bound-i-dont-know-what-to-tell-you},
$\E[\bZ] \geq \frac{2m}{3} \dh^2(p_L',q_L')$.
Hence, 
\begin{align*}
\Pr\left[\bZ \leq \frac{5 m^2}{8(2m +2n)} \epsilon^2\right]
&\leq \frac{64 \Var[\bZ]}{9\E[\bZ]^2}
= O\left(\frac{n}{\E[\bZ]^2} + \frac{m \dh^2(p_L',q_L')}{\E[\bZ]^2}\right)\\
&= O\left(\frac{n}{\E[\bZ]^2} + \frac{1}{\E[\bZ]}\right)
= O\left(\frac{n}{m^2 \epsilon^4} + \frac{1}{m \epsilon^2}\right).
\end{align*}
Both of these bounds can be made arbitrarily small constants by setting~$C$ sufficiently large.
\end{proof}



\subsection{Equivalence Testing with Hellinger Distance}
\label{sec:eq-h}

In this section, we prove Theorem~\ref{thm:twos-h}. 
We will take the number of samples to be 
\begin{equation*}
m = \min\left\{C\cdot \frac{n^{2/3}}{\epsilon^{8/3}}, C^{3/4}\cdot \frac{n^{3/4}}{\epsilon^2}\right\},
\end{equation*}
where~$C$ is some constant which can be taken to be~$10^{10}$.

Rather than drawing samples from~$p$ or~$q$,
our algorithm draws samples from~$p^{+\delta}$ and~$q^{+\delta}$
for $\delta = \epsilon^2/32$.
By Proposition~\ref{prop:mixing}, we have the following guarantees in the two cases:
\begin{equation*}
\text{(Case 1):}~\dlt(p,q) \leq \frac{\epsilon^2}{32 \sqrt{n}},
\qquad
\text{(Case 2):}~\dh(p,q) \geq \frac{1}{2} \epsilon.
\end{equation*}
Furthermore, for any $i \in [n]$, we know the $i$-th coordinates of $p^{+\delta}$ and~$q^{+\delta}$ are both at least~$\frac{\epsilon^2}{32n}$.
Henceforth, we will write~$p'$ and~$q'$ for~$p^{+\delta}$ and~$q^{+\delta}$, respectively.


The two bounds meet when $C^{3/4}\epsilon^{-2} = n^{1/4}$,
which is exactly when $m = n$.
When $m \leq n$, the first bound applies, and when $m > n$ the second bound applies.
As a result, we will split our analysis into the two cases.

\begin{lemma}
The tester succeeds in the $m \leq n$ case of Theorem~\ref{thm:twos-h}.
\end{lemma}
\begin{proof}
In Case 1, if we apply Proposition~\ref{prop:general-upper},
\begin{equation*}
\E[\bZ]
\leq m^2 \cdot  \dlt^2(p',q')
\leq \frac{m^2 \epsilon^4}{32^2 n}.
\end{equation*}
On the other hand, in Case 2, applying Proposition~\ref{prop:their-bound} with $A = [n]$,
\begin{equation*}
\E[\bZ]
\geq \left(\frac{4m^2}{2n+2m}\right) \cdot \dtv(p',q')^2
\geq \left(\frac{4m^2}{2n+2m}\right) \cdot \dh(p',q')^4
\geq \frac{m^2\epsilon^4}{16n}.
\end{equation*}
Our algorithm therefore thresholds~$\bZ$ on the value $ \frac{m^2\epsilon^4}{128n}$,
outputting ``close" if it's below this value and ``far" otherwise.

By Corollary~\ref{cor:var-hel}
\begin{equation*}
\Var[\bZ] \leq 2 \min\{m, n\} +  20m \dh(p',q')^2
\leq 22m,
\end{equation*}
where we used the fact that $\dh(p',q') \leq 1$.
In Case 1,
\begin{equation*}
\Pr\left[\bZ \geq \frac{m^2\epsilon^4}{128n}\right]
\leq \frac{\Var[\bZ]}{\left(\frac{m^2\epsilon^4}{256n}\right)^2}
= O\left(\frac{m}{\frac{m^4}{n^2} \epsilon^8}\right)
= O\left(\frac{n^2}{m^3 \epsilon^8}\right).
\end{equation*}
In Case 2,
\begin{equation*}
\Pr\left[\bZ \leq \frac{m^2\epsilon^4}{128n}\right]
\leq \frac{64 \Var[\bZ]}{49\E[\bZ]^2}
= O\left(\frac{m}{\frac{m^4}{n^2} \epsilon^8}\right)
= O\left(\frac{n^2}{m^3 \epsilon^8}\right).
\end{equation*}
Both of these bounds can be made arbitrarily small constants by setting~$C$ sufficiently large.
\end{proof}

\begin{lemma}
The tester succeeds in the $m > n$ case of Theorem~\ref{thm:twos-h}.
\end{lemma}
\begin{proof}
In Case 1, if we apply Proposition~\ref{prop:well-conditioned-upper} with $A = [n]$ and $\delta = \frac{\epsilon^2}{16n}$
and Proposition~\ref{prop:general-upper},
\begin{equation*}
\E[\bZ]
\leq \min\left\{m^2, 16\frac{mn}{\epsilon^2}\right\} \cdot \dlt^2(p',q')
\leq \min\left\{m^2, 16\frac{mn}{\epsilon^2}\right\} \cdot \frac{\epsilon^4}{32^2 n}
= \min\left\{\frac{m^2\epsilon^4}{32^2 n}, \frac{m\epsilon^2}{64}\right\}.
\end{equation*}
Case 2 is more complicated.
We will need to define the set of ``large" coordinates
$L = \{i : m (p_i' + q_i') \geq 1\}$
and the set of ``small" coordinates $S = [n] \setminus L$.
Applying Proposition~\ref{prop:their-bound} to~$S$, we have
\begin{equation*}
\E[\bZ_S] \geq \left(\frac{4m^2}{2|S| + m\cdot(p'(S) + q'(S))}\right)\cdot \dtv^2(p_S',q_S')
\geq \frac{4m^2}{3n} \dtv^2(p_S',q_S'),
\end{equation*}
where $m\cdot(p'(S)+q'(S)) \leq n$ by the definition of~$S$.
If we also apply Proposition~\ref{prop:pretty-much-a-trivial-lower-bound-i-dont-know-what-to-tell-you} to~$L$,
we get
\begin{equation*}
\E[\bZ] = \E[\bZ_S] + \E[\bZ_L]
\geq \frac{4m^2}{3n} \dtv^2(p_S',q_S') + \frac{2m}{3} \dh^2(p_L',q_L')
\geq \min\left\{\frac{m^2\epsilon^4}{48n}, \frac{m\epsilon^2}{12}\right\},
\end{equation*}
where the last step follows
because $\dh^2(p_S',q_S') + \dh^2(p_L',q_L') = \dh^2(p',q')$ and $\dtv^2(p_S',q_S') \geq \dh^4(p_S',q_S')$.
As a result, we threshold~$\bZ$ on the value
\begin{equation*}
\frac{1}{2} \cdot \min\left\{\frac{m^2\epsilon^4}{48n}, \frac{m\epsilon^2}{12}\right\},
\end{equation*}
outputting ``close" if it's below this value and ``far" otherwise.

In Case 1, by Proposition~\ref{prop:mean-var},
\begin{equation*}
\Var[\bZ]
\leq 2 \min\{m, n\} + \sum_{i=1}^m 5m \frac{(p_i' - q_i')^2}{p_i' + q_i'}
\leq 2 n + \frac{80 m n}{\epsilon^2} \Vert p' - q' \Vert_2^2
\leq 2 n + \frac{5}{64}m \epsilon^2.
\end{equation*}
Hence, by Chebyshev's inequality,
\begin{multline*}
\Pr\left[\bZ \geq\frac{1}{2} \cdot \min\left\{\frac{m^2\epsilon^4}{48n}, \frac{m\epsilon^2}{12}\right\}\right]
\leq \frac{\Var[\bZ]}{(\frac{1}{8} \cdot \min\left\{\frac{m^2\epsilon^4}{48n}, \frac{m\epsilon^2}{12}\right\})^2}\\
\leq O\left(\frac{n}{(\frac{m^2 \epsilon^4}{n})^2} + \frac{n}{(m \epsilon^2)^2}
	+ \frac{m\epsilon^2}{(\frac{m^2 \epsilon^4}{n})^2} + \frac{m\epsilon^2}{(m\epsilon^2)^2}\right)\\
= O\left(\frac{n^3}{m^4 \epsilon^8} + \frac{n}{m^2 \epsilon^4}
	+ \frac{n^2}{m^3 \epsilon^6} + \frac{1}{m\epsilon^2}\right).
\end{multline*}
This can be made an arbitrarily small constant by setting~$C$ sufficiently large.


In Case 2, by Corollary~\ref{cor:var-hel},
\begin{equation}\label{eq:gonna-split}
\Pr\left[\bZ \leq \frac{\E[\bZ]}{2} \right]
\leq \frac{4 \Var[\bZ]}{\E[\bZ]^2}
\leq \frac{8 n + 80 m\dh(p',q')^2}{\E[\bZ]^2}.
\end{equation}
Because $\dh(p',q')^2 = \dh^2(p_S',q_S') + \dh^2(p_L',q_L')$,
either $\dh^2(p_S',q_S')$ or $\dh^2(p_L',q_L')$ is at least $\frac{1}{2}\dh^2(p',q')$.
Suppose that $\dh^2(p_S',q_S') \geq \frac{1}{2}\dh^2(p',q')$.
We note that
\begin{equation*}
m \dh^2(p_S',q_S')
= \frac{m}{2} \sum_{i \in S} (\sqrt{p_i'} - \sqrt{q_i'})^2
\leq \frac{m}{2} \sum_{i \in S} |p_i' + q_i'|
\leq \frac{n}{2},
\end{equation*}
by the definition of~$S$.
Thus,
\begin{equation*}
\eqref{eq:gonna-split}
\leq \frac{8n + 160 m \dh^2(p_S',q_S')}{(\frac{4m^2}{3n}\dtv^2(p_S',q_S'))^2}
\leq \frac{88n}{(\frac{4m^2}{3n}\dtv^2(p_S',q_S'))^2}
= O\left(\frac{n^3}{m^4 \dtv^4(p_S',q_S')}\right)
\leq O\left(\frac{n^3}{m^4 \epsilon^8}\right),
\end{equation*}
where the last step used the fact that $\dtv(p_S',q_S') \geq \dh^2(p_S',q_S') \geq \frac{1}{2}\dh^2(p',q') \geq \frac{1}{2}\epsilon^2$.

In the case when $\dh^2(p_L',q_L') \geq \frac{1}{2} \dh^2(p',q')$,
\begin{equation*}
\eqref{eq:gonna-split}
\leq \frac{8n + 160 m \dh^2(p_L',q_L')}{(\frac{2m}{3} \dh^2(p_L',q_L'))^2}
= O\left(\frac{n}{m^2 \dh^4(p_L',q_L')} + \frac{1}{m \dh^2(p_L',q_L')}\right)
\leq O\left(\frac{n}{m^2 \epsilon^4} + \frac{1}{m \epsilon^2}\right).
\end{equation*}
This can be made an arbitrarily small constant by setting~$C$ sufficiently large.
\end{proof}

\section{Upper Bounds Based on Estimation}
\label{sec:est-ub}

We start by showing a simple meta-algorithm -- in short, it says that if a testing problem is well-defined (i.e., has appropriate separation between the cases) and we can estimate one of the distances, it can be converted to a testing algorithm.
\begin{theorem}\label{thm:est-ub}
Suppose there exists an $m(n, \ve)$-sample algorithm which, given sample access to distributions $p$ and $q$ over $[n]$ and parameter $\ve$, estimates some distance $d(p,q)$ up to an additive $\ve$ with probability at least $2/3$.
Consider distances $d_X(\cdot, \cdot), d_Y(\cdot, \cdot)$ and $\ve_1, \ve_2 > 0$ such that $ d_Y(p,q) \geq \ve_2 \rightarrow d_X(p,q) > 3\ve_1/2$ and $d_X(p,q) \leq \ve_1 \rightarrow d_Y(p,q) < 2\ve_2/3$, and $d(\cdot, \cdot)$ is either $d_X(\cdot, \cdot)$ or $d_Y(\cdot, \cdot)$.

Then there exists an algorithm for equivalence testing between $p$ and $q$ distinguishing the cases:
\begin{itemize}
\item $d_X(p,q) \leq \ve_1$;
\item $d_Y(p,q) \geq \ve_2$.
\end{itemize}
The algorithm uses either $m(n, O(\ve_1))$ or $m(n, O(\ve_2))$ samples, depending on whether $d = d_X$ or $d_Y$.
\end{theorem}
\begin{proof}
Suppose that $d = d_X$, the other case follows similarly.
Using the $m(n, \ve_1/4)$ samples, obtain an estimate $\hat \tau$ of $d_X(p,q)$, accurate up to an additive $\ve_1/4$.
If $\hat \tau \leq 5\ve_1/4$, output that $d_X(p,q) \leq \ve_1$, else output that $d_Y(p,q) \geq \ve_2$. 
Conditioning on the correctness of the estimation algorithm, correctness for the case when $d_X(p,q) \leq \ve_1$ is immediate, and correctness for the case when $d_Y(p,q) \geq \ve_2$ follows from the separation between the cases.
\end{proof}

It is folklore that a distribution over $[n]$ can be $\ve$-learned in $\ell_2$-distance with $O(1/\ve^2)$ samples (see, i.e., \cite{ChanDVV14, Waggoner15} for a reference).
By triangle inequality, this implies that we can estimate the $\ell_2$ distance between $p$ and $q$ up to an additive $O(\ve)$ with $O(1/\ve^2)$ samples, leading to the following corollary.

\begin{corollary}\label{cor:l2-est}
There exists an algorithm for equivalence testing between $p$ and $q$ distinguishing the cases:
\begin{itemize}
\item $d(p,q) \leq f(n, \ve)$;
\item $\dlt(p,q) \geq \ve$,
\end{itemize}
where $d(\cdot, \cdot)$ is a distance and $f(n, \ve)$ is such that $\dlt(p,q) \geq \ve \rightarrow d(p,q) \geq 3f(n, \ve)/2$ and $d(p,q) \leq f(n, \ve) \rightarrow \dlt(p,q) \leq 2\ve/3$.  
The algorithm uses $O(1/\ve^2)$ samples.
\end{corollary}

Finally, we note that total variation distance between $p$ and $q$ can be additively estimated up to a constant using $O(n/\log n)$ samples~\cite{LehmannC06, ValiantV11b, JiaoHW16}, leading to the following corollary:
\begin{corollary}\label{cor:tv-est}
For constant $\ve > 0$, there exists an algorithm for equivalence testing between $p$ and $q$ distinguishing the cases:
\begin{itemize}
\item $\dtv(p,q) \leq \ve^2/4$;
\item $\dh(p,q) \geq \ve/\sqrt{2}$.
\end{itemize}
The algorithm uses $O(n/\log n)$ samples.
\end{corollary}

\section{Lower Bounds}
\label{sec:lb}
We start with a simple lower bound, showing that identity testing with respect to KL divergence is impossible.
A similar observation was made in~\cite{BatuFRSW00}.
\begin{theorem}\label{thm:untestable}
No finite sample test can perform identity testing between $p$ and $q$ distinguishing the cases:
\begin{itemize}
\item $p = q$;
\item $\dkl(p,q) \geq \ve^2$.
\end{itemize}
\end{theorem}
\begin{proof}
Simply take $q = (1, 0)$ and let~$p$ be either $(1, 0)$ or $(1-\delta, \delta)$, for~$\delta > 0$ tending to zero.
Then $p = q$ in the first case and $\dkl(p,q) = \infty$ in the second, but distinguishing between these two possibilities for~$p$
takes $\Omega(\delta^{-1})\rightarrow \infty$ samples.
\end{proof}

Next, we prove our lower bound for KL tolerant identity testing.

\begin{theorem}\label{thm:ones-lb}
There exist constants $0 < s < c$, such that any algorithm for identity testing between $p$ and $q$ distinguishing the cases:
\begin{itemize}
\item $\dkl(p,q) \leq s$;
\item $\dtv(p,q) \geq c$;
\end{itemize}
requires $\Omega(n/\log n)$ samples.
\end{theorem}
\begin{proof}
Let $q = (\tfrac{1}{n}, \ldots, \tfrac{1}{n})$ be the uniform distribution.
Let $R(\cdot, \cdot)$ denote the \emph{relative earthmover distance} (see~\cite{ValiantV10a} for the definition).
By Theorem~$1$ of~\cite{ValiantV10a},
for any $\delta < \frac{1}{4}$
there exist sets of distributions~$\calC$ and~$\calF$ (for \emph{close} and \emph{far})
such that:
\begin{itemize}
\item For every $p \in \calC$, $R(p, q) = O(\delta | \log \delta|)$.
\item For every $p \in \calF$ there exists a distribution~$r$ which is uniform over~$n/2$ elements such that $R(p, r) = O(\delta | \log \delta|)$.
\item Distinguishing between $p \in \calC$ and $p \in \calF$ requires $\Omega(\frac{\delta n}{\log(n)})$ samples.
\end{itemize}
Now, if $p \in \calC$ then
\begin{equation*}
\dkl(p,q)
= \sum_{i=1}^n p_i \log\left(\frac{p_i}{1/n}\right)
= \log(n) - H(p)
\leq O(\delta |\log(\delta)|),
\end{equation*}
where $H(p)$ is the Shannon entropy of~$p$,
and here we used the fact that $|H(p) - H(q)| \leq R(p, q)$, which follows from Fact~$5$ of~\cite{ValiantV10a}.
On the other hand, if $q \in \calF$, let~$r$ be the corresponding distribution which is uniform over~$n/2$ elements.
Then
\begin{equation*}
\frac{1}{2}
= \dtv(p,q)
\leq \dtv(q,p) + \dtv(p,r)
\leq \dtv(q,p) + O(\delta | \log \delta|),
\end{equation*}
where we used the triangle inequality
and the fact that $\dtv(p,r) \leq R(p, r)$ (see~\cite{ValiantV10a} page 4).
As a result, if we set~$\delta$ to be some small constant,
$s = O(\delta |\log(\delta)|)$,
and $c = \frac{1}{2} - O(\delta | \log\delta|)$,
then this argument shows that distinguish $\dkl(p,q) \leq s$ versus $\dtv(p,q) \geq c$
requires $\Omega(n/\log n)$ samples.
\end{proof}

Finally, we conclude with our lower bound for $\chi^2$-tolerant equivalence testing.

\begin{theorem}\label{thm:twos-lb}
There exists a constant $\ve > 0$ such that any algorithm for equivalence testing between $p$ and $q$ distinguishing the cases:
\begin{itemize}
\item $\dxs(p,q) \leq \ve^2/4$;
\item $\dtv(p,q) \geq \ve$;
\end{itemize}
requires $\Omega(n/\log n)$ samples.
\end{theorem}
\begin{proof}
We reduce the problem of distinguishing $\dh(p,q) \leq \frac{1}{\sqrt{48}} \epsilon$ from $\dtv(p,q) \geq 3\epsilon$ to this.
Define the distributions
\begin{equation*}
p' = \frac{2}{3} p + \frac{1}{3} q, \qquad q' = \frac{1}{3} p + \frac{2}{3} q.
\end{equation*}
Then $m$ samples from~$p'$ and~$q'$ can be simulated by $m$ samples from~$p$ and~$q$.
Furthermore,
\begin{equation*}
\dh(p',q') \leq \frac{1}{\sqrt{48}} \epsilon, \qquad \dtv(p',q') = \frac{1}{3} \dtv(p,q) \geq \epsilon,
\end{equation*}
where we used the fact that Hellinger distance satisfies the data processing inequality.
But then, in the ``close" case,
\begin{equation*}
\dxs(p',q')
= \sum_{i=1}^n \frac{(p'_i - q'_i)^2}{q'_i}
\leq 3 \sum_{i=1}^n \frac{(p'_i - q'_i)^2}{p'_i + q'_i}
\leq 12 \dh^2(p',q') \leq \frac{1}{4} \epsilon^2,
\end{equation*}
where we used the fact that $p'_i \leq 2q'_i$ and Proposition~\ref{prp:didnt-know-it-was-hellinger}.
Hence, this problem, which requires $\Omega(n/\log n)$ samples (by the relationship between total variation and Hellinger distance, and the lower bound for testing total variation-close versus -far of~\cite{ValiantV10a}), reduces to the problem in the proposition, and so that requires $\Omega(n/\log n)$ samples as well.
\end{proof}

\bibliographystyle{alpha}
\bibliography{biblio}
\appendix
\section{Proof of Proposition~\ref{prop:distanceinequalities}}
\label{sec:distanceinequalities}
Recall that we will prove this for restrictions of probability distributions to subsets of the support -- in other words, we do not assume $\sum_{i \in S} p_i = \sum_{i \in S} q_i = 1$, we only assume that $\sum_{i \in S} p_i \leq 1$ and $\sum_{i \in S} q_i \leq 1$.
\paragraph{$\dh^2(p_S,q_S) \leq \dtv(p_S,q_S):$}
\begin{align*}
\dh^2(p_S,q_S) &= \frac12 \sum_{i \in S} (\sqrt{p_i} - \sqrt{q_i})^2 \\
&\leq \frac12 \sum_{i \in S} |\sqrt{p_i} - \sqrt{q_i}|(\sqrt{p_i} + \sqrt{q_i}) \\
&= \frac12 \sum_{i \in S} |p_i - q_i| \\
&= \dtv(p_S,q_S).
\end{align*}

\paragraph{$\dtv(p_S,q_S) \leq \sqrt{2}\dh(p_S,q_S):$}
\begin{align*}
\dtv^2(p_S,q_S) &= \frac14 \left(\sum_{i \in S} \left|p_i - q_i\right|\right)^2 \\
&= \frac14 \left(\sum_{i \in S} \left|\sqrt{p_i} -\sqrt{q_i}\right|(\sqrt{p_i} + \sqrt{q_i})\right)^2 \\
&\leq \frac14 \left(\sum_{i \in S} \left|\sqrt{p_i} -\sqrt{q_i}\right|^2\right)\left(\sum_{i \in S}(\sqrt{p_i} + \sqrt{q_i})^2\right) \\
&\leq \dh^2(p_S, q_S) \cdot \frac12 \left(\sum_{i \in S}(\sqrt{p_i} + \sqrt{q_i})^2\right) \\
&= \dh^2(p_S, q_S) \cdot \left(\sum_{i \in S}p_i + \sum_{i \in S} q_i - \dh^2(p_S,q_S)\right) \\
&\leq \dh^2(p_S,q_S) \cdot \left(2 - \dh^2(p_S,q_S)\right) \\
&\leq 2 \dh^2(p_S,q_S).
\end{align*}
Taking the square root of both sides gives the result.
The second inequality is Cauchy-Schwarz.

\paragraph{$2\dh^2(p_S, q_S) \leq \sum_{i \in S} (q_i - p_i) + \dkl(p_S, q_S):$}
\begin{align*}
2 \dh^2(p_S, q_S) 
&= \sum_{i \in S} (q_i + p_i) - 2\sum_{i \in S} \sqrt{p_i q_i} \\
&= \sum_{i \in S} (q_i + p_i) - 2\left(\left(\sum_{j \in S} p_j\right)\sum_{i \in S} \frac{p_i}{\sum_{j \in S} p_j} \sqrt{\frac{q_i}{p_i}}\right) \\
&\leq \sum_{i \in S} (q_i + p_i) - 2\left(\left(\sum_{j \in S} p_j\right)\exp\left(\frac12 \sum_{i \in S} \frac{p_i}{\sum_{j \in S} p_j} \log{\frac{q_i}{p_i}}\right)\right) \\
&\leq \sum_{i \in S} (q_i + p_i) - 2\left(\left(\sum_{j \in S} p_j\right)\left(1 + \frac12 \sum_{i \in S} \frac{p_i}{\sum_{j \in S} p_j} \log{\frac{q_i}{p_i}}\right)\right) \\
&= \sum_{i \in S} (q_i - p_i) - \left(\sum_{i \in S} p_i \log{\frac{q_i}{p_i}} \right)\\
&= \sum_{i \in S} (q_i - p_i) + \dkl(p_S, q_S).
\end{align*}
The first inequality is Jensen's, and the second is $1 + x \leq \exp(x)$.

\paragraph{$\dkl(p_S, q_S) \leq \sum_{i \in S} (p_i - q_i) +  \dxs(p_S, q_S):$}
\begin{align*}
\dkl(p_S, q_S) 
&= \left(\sum_{j \in S} p_j\right)\left(\sum_{i \in S} \frac{p_i}{\sum_{j \in S} p_j} \log{\frac{p_i}{q_i}}\right) \\
&\leq \left(\sum_{j \in S} p_j\right)\left(\log {\frac{1}{\sum_{j\in S} p_j}\sum_{i \in S} \frac{p_i^2}{q_i}} \right) \\
&= \left(\sum_{j \in S} p_j\right) \left(\log{ \left(\frac{1}{\sum_{j\in S} p_j} \left(\dxs(p_S, q_S) + 2\sum_{i \in S} p_i - \sum_{i \in S} q_i\right)\right)} \right) \\
&= \left(\sum_{j \in S} p_j\right) \left(\log{  \left(2 + \frac{1}{\sum_{j\in S} p_j} \left(\dxs(p_S, q_S)  - \sum_{i \in S} q_i\right)\right)} \right) \\
&\leq \left(\sum_{j \in S} p_j\right) \left(1 + \frac{1}{\sum_{j\in S} p_j} \left(\dxs(p_S, q_S)  - \sum_{i \in S} q_i\right)\right) \\
&=\sum_{i \in S} (p_i - q_i) +  \dxs(p_S, q_S).
\end{align*}
The first inequality is Jensen's, and the second is $1 + x \leq \exp(x)$.

\end{document}
