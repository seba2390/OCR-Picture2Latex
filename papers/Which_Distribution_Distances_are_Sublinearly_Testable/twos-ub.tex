\section{Upper Bounds for Equivalence Testing}
\label{sec:twos-ub}
In this section, we prove the following theorems for equivalence testing.

\begin{theorem}\label{thm:twos-tv}
There exists an algorithm for equivalence testing between $p$ and $q$ distinguishing the cases:
\begin{itemize}
\item $\dlt(p,q) \leq \frac{\ve}{2\sqrt{n}}$ 
\item $\dtv(p,q) \geq \ve$
\end{itemize}
The algorithm uses $O\left(\max\left\{\frac{n^{2/3}}{\ve^{4/3}}, \frac{n^{1/2}}{\ve^2}\right\}\right)$ samples.
\end{theorem}

\begin{theorem}\label{thm:twos-h}
There exists an algorithm for equivalence testing between $p$ and $q$ distinguishing the cases:
\begin{itemize}
\item $\dlt(p,q) \leq \frac{\ve^2}{32\sqrt{n}}$ 
\item $\dh(p,q) \geq \ve$
\end{itemize}
The algorithm uses $O\left(\min\left\{\frac{n^{2/3}}{\ve^{8/3}}, \frac{n^{3/4}}{\ve^2}\right\}\right)$ samples.
\end{theorem}



Consider drawing~$\mathrm{Poisson}(m)$ samples from two unknown distributions $p = (p_1, \ldots, p_n)$ and $q = (q_1, \ldots, q_n)$.
Given the resulting histograms~$\bX$ and~$\bY$, \cite{ChanDVV14} define the following statistic:
\begin{equation}\label{eq:hel-statistic}
\bZ = \sum_{i=1}^n \frac{(\bX_i - \bY_i)^2 - \bX_i - \bY_i}{\bX_i + \bY_i}.
\end{equation}
This can be viewed as a modification to the empirical triangle distance applied to~$\bX$ and~$\bY$.
Both of our equivalence testing upper bounds will be obtained by appropriate thresholding of the statistic $\bZ$.

The organization of this section is as follows.
In Section~\ref{sec:eq-prelim}, we prove some basic properties of $\bZ$.
In Section~\ref{sec:eq-tv}, we prove Theorem~\ref{thm:twos-tv}.
In Section~\ref{sec:eq-h}, we prove Theorem~\ref{thm:twos-h}.

\subsection{Some facts about $\mathbf Z$}
\label{sec:eq-prelim}
Chan et al.~\cite{ChanDVV14} give the following expressions for the mean and variance of~$\bZ$.

\begin{proposition}[\cite{ChanDVV14}]\label{prop:mean-var}
Consider the function
\begin{equation*}
f(x) = \left(1 - \frac{1- e^{-x}}{x}\right).
\end{equation*}
Then for any subset $A\subseteq [n]$,
\begin{equation}\label{eq:z-mean}
\E[\bZ_A] = \sum_{i \in A} \frac{(p_i-q_i)^2}{p_i+q_i} m \cdot f(m(p_i + q_i)).
\end{equation}
As a result, $\bZ$ is mean-zero when $p= q$.  Furthermore,
\begin{equation*}
\Var[\bZ] \leq 2 \min\{m, n\} + \sum_{i=1}^n 5m \frac{(p_i - q_i)^2}{p_i + q_i}.
\end{equation*}
\end{proposition}
\noindent
Applying Proposition~\ref{prp:didnt-know-it-was-hellinger}, we immediately have the following corollary.
\begin{corollary}\label{cor:var-hel}
$\displaystyle
\Var[\bZ] \leq 2 \min\{m, n\} +  20m \dh(p,q)^2.
$
\end{corollary}

Without the corrective factor of $f(m(p_i + q_i))$,
Equation~\eqref{eq:z-mean} would just be~$m$ times the triangle distance between~$p$ and~$q$.
Our goal then is to understand the function $f(x)$ and how it affects this quantity.
Aside from the removable discontinuity at $x=0$, $f$ is a monotonically increasing function,
and for $x > 0$, it is strictly bounded between~$0$ and~$1$.
Furthermore, for~$x > 0$ there are roughly two ``regimes" that $f(x)$ exhibits:
when $x < 1$, where $f(x)$ is well-approximated by $x/2$,
and when $x \geq 1$, where $f(x)$ is ``morally the constant one,'' slowly increasing from~$e^{-1}$ to~$1$.
In fact, we have the following explicit bound on~$f(x)$.
\begin{fact}\label{fact:f-upper}
For all $x > 0$, $f(x) \leq \min\{1, x\}.$
\end{fact}
\noindent
In terms of $f(m(p_i + q_i))$, these regimes correspond to whether $p_i + q_i$ is less than or greater than~$\frac{1}{m}$.
Hence, the expression for the mean of~$\bZ$ (i.e.\ Equation~\eqref{eq:z-mean} for $A = [n]$)
splits in two: those terms for ``large" $p_i + q_i$ look roughly like the triangle distance (times~$m$),
and those terms for ``small" $p_i + q_i$ look roughly like the $\ell_2^2$ distance (times~$m^2$).
This is why we have given ourselves the flexibility to consider subsets~$A$ of the domain.

We will now prove several upper and lower bounds on $\E[\bZ_A]$,
based in part on whether we will apply them in the large or small $p_i + q_i$ regime.
Let us begin with a pair of upper bounds.

\begin{proposition}\label{prop:well-conditioned-upper}
Suppose for every $i \in A$, $p_i + q_i \geq \delta$. Then
\begin{equation*}
\E[\bZ_A] \leq \frac{m}{\delta} \dlt^2(p_A, q_A).
\end{equation*}
\end{proposition}
\begin{proof}
Because $f(x) \leq 1$ for all $x > 0$,
\begin{equation*}
\E[\bZ_A]
= \sum_{i \in A} \frac{(p_i-q_i)^2}{p_i+q_i} m \cdot f(m(p_i + q_i))
\leq  \sum_{i \in A} \frac{(p_i-q_i)^2}{p_i+q_i} m
\leq \frac{m}{\delta} \sum_{i \in A} (p_i-q_i)^2
=  \frac{m}{\delta} \dlt^2(p_A, q_A).\qedhere
\end{equation*}
\end{proof}

\begin{proposition}\label{prop:general-upper}
$\displaystyle
\E[\bZ] \leq m^2 \dlt^2(p, q).
$
\end{proposition}
\begin{proof}
Let $L$ be the set of $i$ such that $m(p_i + q_i) \geq 1$.
Then $\E[\bZ] = \E[\bZ_L] + \E[\bZ_{\overline{L}}]$, and by Proposition~\ref{prop:well-conditioned-upper},
$\E[\bZ_L] \leq m^2 \dlt^2(p_L, q_L)$.
On the other hand, by Fact~\ref{fact:f-upper}, $f(x) \leq x$, and therefore
\begin{equation*}
\E[\bZ_{\overline{L}}]
= \sum_{i \in \overline{L}} \frac{(p_i-q_i)^2}{p_i+q_i} m \cdot f(m(p_i + q_i))
\leq \sum_{i \in \overline{L}} (p_i-q_i)^2 m^2
= m^2 \dlt^2(p_{\overline{L}}, q_{\overline{L}}).
\end{equation*}
The proof is completed by noting that $\dlt^2(p_L, q_L) + \dlt^2(p_{\overline{L}}, q_{\overline{L}}) = \dlt^2(p, q).$
\end{proof}

Now we give a pair of lower bounds.

\begin{proposition}\label{prop:pretty-much-a-trivial-lower-bound-i-dont-know-what-to-tell-you}
Suppose for every $i \in A$, $m(p_i + q_i) \geq 1$.  Then
\begin{equation*}
\E[\bZ_A] \geq \frac{2m}{3} \dh^2(p_A,q_A).
\end{equation*}
\end{proposition}
\begin{proof}
Because $f(x)$ is monotonically increasing and $f(1) = 1/e$,
\begin{equation*}
\E[\bZ_A]
= m \sum_{i \in A} \frac{(p_i-q_i)^2}{p_i+q_i} f(m(p_i+q_i))
\geq m \sum_{i \in A} \frac{(p_i-q_i)^2}{p_i+q_i} f(1)
\geq \frac{2m}{e}\dh^2(p_A,q_A),
\end{equation*}
where the first step is by Proposition~\ref{prop:mean-var} and the last is by Proposition~\ref{prp:didnt-know-it-was-hellinger}.
The result follows from $e \leq 3$.
\end{proof}

The next proposition is essentially the second half of the proof of Lemma~$4$ from~\cite{ChanDVV14}.

\begin{proposition}\label{prop:their-bound}
For any subset~$A$,
\begin{equation*}
\E[\bZ_A] \geq \left(\frac{4m^2}{2|A| + m\cdot(p(A) + q(A))}\right)\cdot \dtv^2(p_A,q_A),
\end{equation*}
where we write $p(A) = \sum_{i \in A} p(i)$ and likewise for~$q(A)$.
\end{proposition}
\begin{proof}
Consider the function $g(x) = x f(x)^{-1}$.
Then $g(x) \leq 2+x$ for nonnegative~$x$.
Furthermore,
\begin{equation*}
\frac{(p_i - q_i)^2}{g(m(p_i + q_i))} = \frac{(p_i-q_i)^2}{m(p_i+q_i)} \left(1 - \frac{1-e^{-m(p_i+q_i)}}{m(p_i + q_i)}\right),
\end{equation*}
which, from Proposition~\ref{prop:mean-var}, is $\frac{1}{m^2} \cdot \E[\bZ_{\{i\}}]$.
As a result,
\begin{multline*}
\dtv^2(p_A,q_A)
= \frac14 \left(\sum_{i \in A} |p_i - q_i|\right)^2
= \frac14 \left(\sum_{i \in A} |p_i - q_i|\cdot \frac{\sqrt{g(m(p_i+q_i))}}{\sqrt{g(m(p_i+q_i))}}\right)^2\\
\leq \frac14\left(\sum_{i \in A} \frac{(p_i - q_i)^2}{g(m(p_i + q_i))}\right) \cdot \left(\sum_{i \in A} g(m(p_i + q_i))\right)
\leq \frac{1}{4m^2} \cdot \E[\bZ_A] \cdot(2|A| + m\cdot (p(A) + q(A))),
\end{multline*}
where the first inequality is Cauchy-Schwarz. Rearranging finishes the proof.
\end{proof}


\subsection{Equivalence Testing with Total Variation Distance}
\label{sec:eq-tv}
In this section, we prove Theorem~\ref{thm:twos-tv}. 
We will take the number of samples to be 
\begin{equation}\label{eq:l1-max}
m = \max\left\{C\cdot \frac{n^{2/3}}{\epsilon^{4/3}}, C^{3/2}\cdot \frac{n^{1/2}}{\epsilon^2}\right\},
\end{equation}
where~$C$ is some constant which can be taken to be~$10^{10}$.
 
Rather than drawing samples from~$p$ or~$q$,
our algorithm draws samples from~$p^{+1/2}$ and~$q^{+1/2}$.
By Proposition~\ref{prop:mixing}, we have the following guarantees in the two cases:
\begin{equation*}
\text{(Case 1):}~\dlt(p^{+1/2},q^{+1/2}) \leq \frac{\epsilon}{4 \sqrt{n}},
\qquad
\text{(Case 2):}~\dtv(p^{+1/2},q^{+1/2}) \geq \frac{\epsilon}{2}.
\end{equation*}
Furthermore, for any $i \in [n]$, we know the $i$-th coordinates of $p^{+1/2}$ and~$q^{+1/2}$ are both at least~$\frac{1}{2n}$.
Henceforth, we will write~$p'$ and~$q'$ for~$p^{+1/2}$ and~$q^{+1/2}$, respectively.

In Case 1, if we apply Proposition~\ref{prop:well-conditioned-upper} with $A = [n]$ and $\delta = \frac{1}{n}$
and Proposition~\ref{prop:general-upper},
\begin{equation*}
\E[\bZ]
\leq \min\{m^2, mn\} \cdot \dlt^2(p',q')
\leq \min\{m^2, mn\} \cdot \frac{\epsilon^2}{16 n}
\leq \frac{m^2}{4(2m + 2n)} \cdot \epsilon^2.
\end{equation*}
On the other hand, in Case 2, applying Proposition~\ref{prop:their-bound} with $A = [n]$,
\begin{equation*}
\E[\bZ] \geq \frac{4m^2}{2m + 2n} \cdot \dtv(p',q')^2 \geq \frac{m^2}{2m + 2n} \cdot \epsilon^2.
\end{equation*}
Our algorithm therefore thresholds~$\bZ$ on the value $\frac{5 m^2}{8(2m +2n)} \epsilon^2$,
outputting ``close" if it's below this value and ``far" otherwise.

The two bounds in~\eqref{eq:l1-max} meet when $C^3 \epsilon^{-4} = n$,
which is exactly when $m = n$.
When $m \leq n$, the first bound applies, and when $m > n$ the second bound applies.
As a result, we will split our analysis into the two cases.

\begin{lemma}
The tester succeeds in the $m \leq n$ case of Theorem~\ref{thm:twos-tv}.
\end{lemma}
\begin{proof}
By Corollary~\ref{cor:var-hel}
\begin{equation*}
\Var[\bZ] \leq 2 \min\{m, n\} +  20m \dh(p',q')^2
\leq 22m,
\end{equation*}
where we used the fact that $\dh(p',q') \leq 1$.
In Case 1, by Chebyshev's inequality,
\begin{equation*}
\Pr\left[\bZ \geq \frac{5 m^2}{8(2m +2n)} \epsilon^2\right]
\leq \frac{\Var[\bZ]}{\left(\frac{3m^2}{8(2m +2n)} \epsilon^2\right)^2}
= O\left(\frac{m}{\frac{m^4}{n^2} \epsilon^4}\right)
= O\left(\frac{n^2}{m^3 \epsilon^4}\right).
\end{equation*}
In Case 2,
\begin{equation*}
\Pr\left[\bZ \leq \frac{5 m^2}{8(2m +2n)} \epsilon^2\right]
\leq \frac{64 \Var[\bZ]}{9\E[\bZ]^2}
= O\left(\frac{m}{\frac{m^4}{n^2} \epsilon^4}\right)
= O\left(\frac{n^2}{m^3 \epsilon^4}\right).
\end{equation*}
Both of these bounds can be made arbitrarily small constants by setting~$C$ sufficiently large.
\end{proof}

\begin{lemma}
The tester succeeds in the $m \geq n$ case of Theorem~\ref{thm:twos-tv}.
\end{lemma}
\begin{proof}
We first consider Case 1.
By Proposition~\ref{prop:mean-var},
\begin{equation*}
\Var[\bZ]
\leq 2 \min\{m, n\} + \sum_{i=1}^n 5m \frac{(p_i' - q_i')^2}{p_i' + q_i'}
\leq 2 n + 5 m n \dlt^2(p',q')
\leq 2 n + \tfrac{5}{16} m \epsilon^2.
\end{equation*}
Then, we have that
\begin{equation*}
\Pr\left[\bZ \geq \frac{5 m^2}{8(2m +2n)} \epsilon^2\right]
\leq \frac{\Var[\bZ]}{\left(\frac{3m^2}{8(2m +2n)} \epsilon^2\right)^2}
= O\left(\frac{n}{m^2 \epsilon^4} + \frac{m\epsilon^2}{m^2 \epsilon^4}\right)
= O\left(\frac{n}{m^2 \epsilon^4} + \frac{1}{m \epsilon^2}\right).
\end{equation*}
Next, we focus on Case 2.
Write $L$ for the set of $i \in [n]$ such that $m(p_i' + q_i') \geq 1$.
Then $\dh^2(p_{\overline{L}}',q_{\overline{L}}') \leq \frac12 \sum_{i \in \overline{L}} (p_i' + q_i') \leq n/2m$.
As a result, 
by Corollary~\ref{cor:var-hel}
\begin{equation*}
\Var[\bZ] \leq 2 \min\{m, n\} +  20m \dh^2(p',q')
\leq 12 n + 20m \dh^2(p_L',q_L'). 
\end{equation*}
By Proposition~\ref{prop:pretty-much-a-trivial-lower-bound-i-dont-know-what-to-tell-you},
$\E[\bZ] \geq \frac{2m}{3} \dh^2(p_L',q_L')$.
Hence, 
\begin{align*}
\Pr\left[\bZ \leq \frac{5 m^2}{8(2m +2n)} \epsilon^2\right]
&\leq \frac{64 \Var[\bZ]}{9\E[\bZ]^2}
= O\left(\frac{n}{\E[\bZ]^2} + \frac{m \dh^2(p_L',q_L')}{\E[\bZ]^2}\right)\\
&= O\left(\frac{n}{\E[\bZ]^2} + \frac{1}{\E[\bZ]}\right)
= O\left(\frac{n}{m^2 \epsilon^4} + \frac{1}{m \epsilon^2}\right).
\end{align*}
Both of these bounds can be made arbitrarily small constants by setting~$C$ sufficiently large.
\end{proof}



\subsection{Equivalence Testing with Hellinger Distance}
\label{sec:eq-h}

In this section, we prove Theorem~\ref{thm:twos-h}. 
We will take the number of samples to be 
\begin{equation*}
m = \min\left\{C\cdot \frac{n^{2/3}}{\epsilon^{8/3}}, C^{3/4}\cdot \frac{n^{3/4}}{\epsilon^2}\right\},
\end{equation*}
where~$C$ is some constant which can be taken to be~$10^{10}$.

Rather than drawing samples from~$p$ or~$q$,
our algorithm draws samples from~$p^{+\delta}$ and~$q^{+\delta}$
for $\delta = \epsilon^2/32$.
By Proposition~\ref{prop:mixing}, we have the following guarantees in the two cases:
\begin{equation*}
\text{(Case 1):}~\dlt(p,q) \leq \frac{\epsilon^2}{32 \sqrt{n}},
\qquad
\text{(Case 2):}~\dh(p,q) \geq \frac{1}{2} \epsilon.
\end{equation*}
Furthermore, for any $i \in [n]$, we know the $i$-th coordinates of $p^{+\delta}$ and~$q^{+\delta}$ are both at least~$\frac{\epsilon^2}{32n}$.
Henceforth, we will write~$p'$ and~$q'$ for~$p^{+\delta}$ and~$q^{+\delta}$, respectively.


The two bounds meet when $C^{3/4}\epsilon^{-2} = n^{1/4}$,
which is exactly when $m = n$.
When $m \leq n$, the first bound applies, and when $m > n$ the second bound applies.
As a result, we will split our analysis into the two cases.

\begin{lemma}
The tester succeeds in the $m \leq n$ case of Theorem~\ref{thm:twos-h}.
\end{lemma}
\begin{proof}
In Case 1, if we apply Proposition~\ref{prop:general-upper},
\begin{equation*}
\E[\bZ]
\leq m^2 \cdot  \dlt^2(p',q')
\leq \frac{m^2 \epsilon^4}{32^2 n}.
\end{equation*}
On the other hand, in Case 2, applying Proposition~\ref{prop:their-bound} with $A = [n]$,
\begin{equation*}
\E[\bZ]
\geq \left(\frac{4m^2}{2n+2m}\right) \cdot \dtv(p',q')^2
\geq \left(\frac{4m^2}{2n+2m}\right) \cdot \dh(p',q')^4
\geq \frac{m^2\epsilon^4}{16n}.
\end{equation*}
Our algorithm therefore thresholds~$\bZ$ on the value $ \frac{m^2\epsilon^4}{128n}$,
outputting ``close" if it's below this value and ``far" otherwise.

By Corollary~\ref{cor:var-hel}
\begin{equation*}
\Var[\bZ] \leq 2 \min\{m, n\} +  20m \dh(p',q')^2
\leq 22m,
\end{equation*}
where we used the fact that $\dh(p',q') \leq 1$.
In Case 1,
\begin{equation*}
\Pr\left[\bZ \geq \frac{m^2\epsilon^4}{128n}\right]
\leq \frac{\Var[\bZ]}{\left(\frac{m^2\epsilon^4}{256n}\right)^2}
= O\left(\frac{m}{\frac{m^4}{n^2} \epsilon^8}\right)
= O\left(\frac{n^2}{m^3 \epsilon^8}\right).
\end{equation*}
In Case 2,
\begin{equation*}
\Pr\left[\bZ \leq \frac{m^2\epsilon^4}{128n}\right]
\leq \frac{64 \Var[\bZ]}{49\E[\bZ]^2}
= O\left(\frac{m}{\frac{m^4}{n^2} \epsilon^8}\right)
= O\left(\frac{n^2}{m^3 \epsilon^8}\right).
\end{equation*}
Both of these bounds can be made arbitrarily small constants by setting~$C$ sufficiently large.
\end{proof}

\begin{lemma}
The tester succeeds in the $m > n$ case of Theorem~\ref{thm:twos-h}.
\end{lemma}
\begin{proof}
In Case 1, if we apply Proposition~\ref{prop:well-conditioned-upper} with $A = [n]$ and $\delta = \frac{\epsilon^2}{16n}$
and Proposition~\ref{prop:general-upper},
\begin{equation*}
\E[\bZ]
\leq \min\left\{m^2, 16\frac{mn}{\epsilon^2}\right\} \cdot \dlt^2(p',q')
\leq \min\left\{m^2, 16\frac{mn}{\epsilon^2}\right\} \cdot \frac{\epsilon^4}{32^2 n}
= \min\left\{\frac{m^2\epsilon^4}{32^2 n}, \frac{m\epsilon^2}{64}\right\}.
\end{equation*}
Case 2 is more complicated.
We will need to define the set of ``large" coordinates
$L = \{i : m (p_i' + q_i') \geq 1\}$
and the set of ``small" coordinates $S = [n] \setminus L$.
Applying Proposition~\ref{prop:their-bound} to~$S$, we have
\begin{equation*}
\E[\bZ_S] \geq \left(\frac{4m^2}{2|S| + m\cdot(p'(S) + q'(S))}\right)\cdot \dtv^2(p_S',q_S')
\geq \frac{4m^2}{3n} \dtv^2(p_S',q_S'),
\end{equation*}
where $m\cdot(p'(S)+q'(S)) \leq n$ by the definition of~$S$.
If we also apply Proposition~\ref{prop:pretty-much-a-trivial-lower-bound-i-dont-know-what-to-tell-you} to~$L$,
we get
\begin{equation*}
\E[\bZ] = \E[\bZ_S] + \E[\bZ_L]
\geq \frac{4m^2}{3n} \dtv^2(p_S',q_S') + \frac{2m}{3} \dh^2(p_L',q_L')
\geq \min\left\{\frac{m^2\epsilon^4}{48n}, \frac{m\epsilon^2}{12}\right\},
\end{equation*}
where the last step follows
because $\dh^2(p_S',q_S') + \dh^2(p_L',q_L') = \dh^2(p',q')$ and $\dtv^2(p_S',q_S') \geq \dh^4(p_S',q_S')$.
As a result, we threshold~$\bZ$ on the value
\begin{equation*}
\frac{1}{2} \cdot \min\left\{\frac{m^2\epsilon^4}{48n}, \frac{m\epsilon^2}{12}\right\},
\end{equation*}
outputting ``close" if it's below this value and ``far" otherwise.

In Case 1, by Proposition~\ref{prop:mean-var},
\begin{equation*}
\Var[\bZ]
\leq 2 \min\{m, n\} + \sum_{i=1}^m 5m \frac{(p_i' - q_i')^2}{p_i' + q_i'}
\leq 2 n + \frac{80 m n}{\epsilon^2} \Vert p' - q' \Vert_2^2
\leq 2 n + \frac{5}{64}m \epsilon^2.
\end{equation*}
Hence, by Chebyshev's inequality,
\begin{multline*}
\Pr\left[\bZ \geq\frac{1}{2} \cdot \min\left\{\frac{m^2\epsilon^4}{48n}, \frac{m\epsilon^2}{12}\right\}\right]
\leq \frac{\Var[\bZ]}{(\frac{1}{8} \cdot \min\left\{\frac{m^2\epsilon^4}{48n}, \frac{m\epsilon^2}{12}\right\})^2}\\
\leq O\left(\frac{n}{(\frac{m^2 \epsilon^4}{n})^2} + \frac{n}{(m \epsilon^2)^2}
	+ \frac{m\epsilon^2}{(\frac{m^2 \epsilon^4}{n})^2} + \frac{m\epsilon^2}{(m\epsilon^2)^2}\right)\\
= O\left(\frac{n^3}{m^4 \epsilon^8} + \frac{n}{m^2 \epsilon^4}
	+ \frac{n^2}{m^3 \epsilon^6} + \frac{1}{m\epsilon^2}\right).
\end{multline*}
This can be made an arbitrarily small constant by setting~$C$ sufficiently large.


In Case 2, by Corollary~\ref{cor:var-hel},
\begin{equation}\label{eq:gonna-split}
\Pr\left[\bZ \leq \frac{\E[\bZ]}{2} \right]
\leq \frac{4 \Var[\bZ]}{\E[\bZ]^2}
\leq \frac{8 n + 80 m\dh(p',q')^2}{\E[\bZ]^2}.
\end{equation}
Because $\dh(p',q')^2 = \dh^2(p_S',q_S') + \dh^2(p_L',q_L')$,
either $\dh^2(p_S',q_S')$ or $\dh^2(p_L',q_L')$ is at least $\frac{1}{2}\dh^2(p',q')$.
Suppose that $\dh^2(p_S',q_S') \geq \frac{1}{2}\dh^2(p',q')$.
We note that
\begin{equation*}
m \dh^2(p_S',q_S')
= \frac{m}{2} \sum_{i \in S} (\sqrt{p_i'} - \sqrt{q_i'})^2
\leq \frac{m}{2} \sum_{i \in S} |p_i' + q_i'|
\leq \frac{n}{2},
\end{equation*}
by the definition of~$S$.
Thus,
\begin{equation*}
\eqref{eq:gonna-split}
\leq \frac{8n + 160 m \dh^2(p_S',q_S')}{(\frac{4m^2}{3n}\dtv^2(p_S',q_S'))^2}
\leq \frac{88n}{(\frac{4m^2}{3n}\dtv^2(p_S',q_S'))^2}
= O\left(\frac{n^3}{m^4 \dtv^4(p_S',q_S')}\right)
\leq O\left(\frac{n^3}{m^4 \epsilon^8}\right),
\end{equation*}
where the last step used the fact that $\dtv(p_S',q_S') \geq \dh^2(p_S',q_S') \geq \frac{1}{2}\dh^2(p',q') \geq \frac{1}{2}\epsilon^2$.

In the case when $\dh^2(p_L',q_L') \geq \frac{1}{2} \dh^2(p',q')$,
\begin{equation*}
\eqref{eq:gonna-split}
\leq \frac{8n + 160 m \dh^2(p_L',q_L')}{(\frac{2m}{3} \dh^2(p_L',q_L'))^2}
= O\left(\frac{n}{m^2 \dh^4(p_L',q_L')} + \frac{1}{m \dh^2(p_L',q_L')}\right)
\leq O\left(\frac{n}{m^2 \epsilon^4} + \frac{1}{m \epsilon^2}\right).
\end{equation*}
This can be made an arbitrarily small constant by setting~$C$ sufficiently large.
\end{proof}
