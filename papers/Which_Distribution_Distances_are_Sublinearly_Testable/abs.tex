Given samples from an unknown distribution $p$ and a description of a distribution $q$, are $p$ and $q$ close or far?
This question of ``identity testing'' has received significant attention in the case of testing whether $p$ and $q$ are equal or far in total variation distance.
However, in recent work~\cite{ValiantV11a, AcharyaDK15, DaskalakisP17}, the following questions have been been critical to solving problems at the frontiers of distribution testing:
\begin{itemize}
\item Alternative Distances: Can we test whether $p$ and $q$ are far in other distances, say Hellinger?
\item Tolerance: Can we test when $p$ and $q$ are \emph{close}, rather than equal? And if so, close in which distances?
\end{itemize}

Motivated by these questions, we characterize the complexity of distribution testing under a variety of distances, including total variation, $\ell_2$, Hellinger, Kullback-Leibler, and $\chi^2$.
For each pair of distances $d_1$ and $d_2$, we study the complexity of testing if $p$ and $q$ are close in $d_1$ versus far in $d_2$, with a focus on identifying which problems allow \emph{strongly} sublinear testers (i.e., those with complexity $O(n^{1 - \gamma})$ for some $\gamma > 0$ where $n$ is the size of the support of the distributions $p$ and $q$).
We provide matching upper and lower bounds for each case.
We also study these questions in the case where we only have samples from $q$ (equivalence testing), showing qualitative differences from identity testing in terms of when tolerance can be achieved.
Our algorithms fall into the classical paradigm of $\chi^2$-statistics, but require crucial changes to handle the challenges introduced by each distance we consider.
Finally, we survey other recent results in an attempt to serve as a reference for the complexity of various distribution testing problems.
