\begin{figure*}[h!]
  \begin{subfigure}{\textwidth}
    \centering
      \includegraphics[width=0.30\textwidth]{liver/1clusters.pdf}
      \includegraphics[width=0.30\textwidth]{liver/2clusters.pdf}
      \includegraphics[width=0.30\textwidth]{liver/all-clusters.pdf}\\
     \caption{Liver: 345 instances, 6 features}
     \label{fig:liver}
   \end{subfigure}

   \begin{subfigure}{\textwidth}
      \centering
        \includegraphics[width=0.30\textwidth]{heart-statlog/1clusters.pdf}
        \includegraphics[width=0.30\textwidth]{heart-statlog/2clusters.pdf}
        \includegraphics[width=0.30\textwidth]{heart-statlog/all-clusters.pdf}\\
       \caption{Heart-Statlog: 270 instances, 13 features.}
       \label{fig:heart}
    \end{subfigure}
    \begin{subfigure}{\textwidth}
      \centering
        \includegraphics[width=0.30\textwidth]{sonar/1clusters.pdf}
        \includegraphics[width=0.30\textwidth]{sonar/2clusters.pdf}
        \includegraphics[width=0.30\textwidth]{sonar/all-clusters.pdf}\\
       \caption{Sonar: 351 instances, 60 features.}
       \label{fig:sonar}
    \end{subfigure}
    \begin{subfigure}{\textwidth}
      \centering
        \includegraphics[width=0.30\textwidth]{ionosphere/1clusters.pdf}
        \includegraphics[width=0.30\textwidth]{ionosphere/2clusters.pdf}
        \includegraphics[width=0.30\textwidth]{ionosphere/all-clusters.pdf}\\
       \caption{Ionosphere: 209 instances, 33 features.}
       \label{fig:ionosphere}
    \end{subfigure}
  \caption{We report the results for 1 cluster (left) and 2 clusters (middle), and for 1 to 40 clusters we report the maximal mean accuracy obtained with all possible values of $L$ with a given number of clusters (right). The black line in the first two pictures marks the dimension of the input space.}
  \label{fig:uci}
\end{figure*}