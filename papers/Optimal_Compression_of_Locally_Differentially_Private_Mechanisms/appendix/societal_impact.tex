\section{Societal impact}
\label{appendix:societal_impacts}

Collecting large datasets allows us to build better machine learning models which can facilitate our lives in many different ways. However, harnessing data from devices can expose their users to privacy risks. Research into differential privacy can help to minimize these risks. At present, our work is mostly theoretical in nature as there are a few unsolved questions. In particular, for large $\varepsilon$ the computational complexity of our approach may be too expensive to be practical.
% It could enable large-scale smart infrastructure and IoT applications, providing a profound and positive impact on power-grid efficiency, traffic, health-monitoring, medical diagnoses, carbon emissions, and many other areas. A fundamental understanding of distributed learning, or federated learning and analytics, can also benefit many different fields of study such as neuroscience, medicine, economics, and social networks, where statistical tools are often used to analyze information that is generated and processed in large networks. 
% While the above vision is expected to generate many social opportunities, it presents a number of unprecedented challenges. 
%However, this presents a number of challenges. 
%First, massive amounts of data need to be collected by, and transferred across, resource-constrained devices.  Second, serious concerns such as data privacy and security should be rigorously addressed.

%Our work tackles the above challenges simultaneously by providing a generic solution that optimally compresses any privatization scheme and achieves the fundamental trade-off between communication, privacy, and accuracy. This enables us to utilize a larger and richer set of privatization tools previously developed in differential privacy literature without worrying about the bandwidth constraints of local devices. 


%Broadly speaking, the impact of our work, for now, is mostly theoretical in nature. There are some unsolved questions with respect to the computational complexity of our work and these need to be addressed before our methodology can be adopted widely.

%As a potential benefit of our work, the optimal solution that achieves communication-privacy-accuracy trade-off could provide incentives to industry for adopting privacy-preserve 

% Even more broadly, our work advances the current state-of-the-art in a number of areas of statistics and computer science. Indeed, our work builds on a long line of fundamental research in information theory, statistics, and theoretical computer science, extending them in non-trivial ways.
