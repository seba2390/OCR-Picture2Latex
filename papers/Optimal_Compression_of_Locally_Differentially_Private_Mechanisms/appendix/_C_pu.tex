\section{Preliminary on \texorpdfstring{$\PrivUnit$}{PrivUnit}}
\label{appendix:privunit}

First, we briefly recap the $\PrivUnit$ mechanism ($\qPU$) proposed in \cite{BDFKR2018}. $\PrivUnit$ is a private sampling scheme when the input alphabet $\cX$ is the $d-$dimensional unit $\ell_2$ sphere $\sphere^{d-1}$. More formally, given a vector $\svbx \in \sphere^{d-1}$, $\PrivUnit$ (see Algorithm \ref{alg:privunit}) draws a vector $\svbz$ from a spherical cap $\{\svbz \in \sphere^{d-1} \mid
\lan \svbz, \svbx  \ran \ge \gamma\}$ with probability $p_0 \geq 1/2$ or from its complement $\{\svbz \in \sphere^{d-1} \mid \lan \svbz, \svbx \ran < \gamma\}$  with probability $1 - p_0$, where
$\gamma \in [0, 1]$ and $p_0$ are constants that trade accuracy and privacy. In other words, the conditional density $\qPU(\svbz | \svbx)$ is:
\begin{align}\label{eq:qpu}
    \qPU(\rvbz | \rvbx) = 
    \begin{cases}
      p_0 \times \dfrac{2}{A(1,d)I_{1-\gamma^2}(\frac{d-1}{2},\frac{1}{2})} & \text{if}\ \langle \rvbx, \rvbz \rangle \geq \gamma
      \\[10pt]
      (1-p_0) \times \dfrac{2}{2A(1,d)- A(1,d)I_{1-\gamma^2}(\frac{d-1}{2},\frac{1}{2})} & \text{otherwise}
    \end{cases}
\end{align}
where $A(1,d)$ denotes the area of $\sphere^{d-1}$ and $I_x(a,b)$ denotes the regularized incomplete beta function.


\begin{algorithm}[h]
\caption{Privatized Unit Vector: $\PrivUnit$}
\label{alg:privunit}
\begin{algorithmic}
\Require $\svbx \in \sphere^{d-1}$, $\gamma \in [0,1]$, $p_0 \ge 1/2$.
\State Draw random vector $\svbz$ according to the distribution
\State \begin{equation}
	  \label{eqn:w-flip-mechanism}
	  \svbz = \begin{cases}
	    \mbox{uniform on } \{\svbz \in \sphere^{d-1} \mid
	    \lan \svbz, \svbx 
	    \ran \ge \gamma\} & \mbox{with~probability~} p_0 \\
	    \mbox{uniform on } \{\svbz \in \sphere^{d-1} \mid
	    \lan \svbz, \svbx 
	    \ran < \gamma \} & \mbox{otherwise}.
	  \end{cases}
\end{equation}
\State Set $\alpha = \frac{d-1}{2}$, $\tau = \frac{1+\gamma}{2}$, and
\begin{equation}
  m_{\texttt{pu}} = \frac{(1 - \gamma^2)^\alpha}{2^{d-2} (d - 1)}
  \left[\frac{p_0}{B(\alpha,\alpha) - B(\tau; \alpha,\alpha)}
    - \frac{1 - p_0}{B(\tau; \alpha, \alpha)}\right]
  \label{eqn:norm-of-W}
\end{equation}
\Return $\hat{\svbx}^{\texttt{pu}} = \frac{\svbz}{m_{\texttt{pu}}}$
\end{algorithmic}
\end{algorithm}


Given its inputs $\svbx, \gamma,$ and $p_0$, Algorithm \ref{alg:privunit} returns an estimator $\hat{\svbx}^{\texttt{pu}} \coloneqq \svbz/m_{\texttt{pu}}$ which is differentially private and unbiased where $m_{\texttt{pu}}$ is a scaling factor. The choice of $\gamma$ described in Theorem \ref{thm:PrivUnitPriv} ensures differential privacy and the choice of the scaling factor $m$ described in \eqref{eqn:norm-of-W} ensures unbiasedness where
\begin{align}
  B(x;\alpha,\beta) \coloneqq \int_{0}^x
  t^{\alpha - 1} (1- t)^{\beta-1}dt
  ~~ \mbox{where} ~~
  B(\alpha,\beta) \coloneqq
  B(1;\alpha,\beta) = \frac{\Gamma(\alpha)
    \Gamma(\beta)}{\Gamma(\alpha+\beta)} \label{eq:incomplete_beta}
\end{align}
denotes the incomplete beta function. 
% At times, for convenience, we will view $m$ defined in \eqref{eqn:norm-of-W} as a function of $p$ for a given $d$ and $\gamma$ i.e.,
% \begin{align}
%     m(p; d, \gamma) \coloneqq \frac{(1 - \gamma^2)^\alpha}{2^{d-2} (d - 1)}
%   \left[\frac{p}{B(\alpha,\alpha) - B(\tau; \alpha,\alpha)}
%     - \frac{1 - p}{B(\tau; \alpha, \alpha)}\right] \label{eq:function_m}
% \end{align}
% Often, for notational convenience, when the context is clear and $d$ and $\gamma$ are fixed, we will omit the dependence of $m$ on $d$ and $\gamma$.


% \subsection{Properties of $\PrivUnit$}
\subsection{\texorpdfstring{$\PrivUnit$}{PrivUnit} is a differentially private mechanism}
\label{subsubsec:privunit_dp}
The following theorem borrowed from \cite{BDFKR2018} describes the choice of $\gamma$ and provides the precise associated differential privacy guarantee of the $\PrivUnit$ mechanism.
\begin{theorem}[\cite{BDFKR2018}]
  \label{thm:PrivUnitPriv}
  Let $\gamma \in [0,1]$ and $p_0 = \frac{e^{\varepsilon_0}}{1 +
    e^{\varepsilon_0}}$. Then algorithm \emph{$\PrivUnit(\cdot, \gamma, p_0)$}
  is $\varepsilon = (\bar{\varepsilon}+\varepsilon_0)$-differentially private whenever $\gamma \ge 0$ is
  such that
  \begin{align}
  \begin{aligned}\label{eqn:sufficient-gamma}
    %  \begin{subequations}
    %   \label{eqn:small-suff-gamma}
      \bar{\varepsilon} & \ge
      \log\frac{ 1+\gamma \cdot \sqrt{ 2(d-1) / \pi } }{
        {\lp 1 - \gamma \cdot \sqrt{ 2(d-1) / \pi } \rp_+} },
      ~~~ \mbox{i.e.} ~~~
      \gamma \le \frac{e^{\bar{\varepsilon}} - 1}{e^{\bar{\varepsilon}} + 1} \sqrt{\frac{\pi}{2(d-1)}},\\
    %  \end{subequations}\\
     or\\
    %  \begin{subequations}
    %   \label{eqn:big-suff-gamma}
       \bar{\varepsilon} & \ge 1/2 \log(d) + \log 6 - \frac{d - 1}{2}
       \log (1 - \gamma^2) + \log \gamma
       ~~ \mbox{and} ~~
       \gamma \ge \sqrt{\frac{2}{d}}.
    %  \end{subequations}
  \end{aligned}
  \end{align}
\end{theorem}

Here, $\varepsilon$ can be viewed as the total privacy budget. Typically, $\mu$ fraction of this budget is allocated for the spherical cap threshold $\gamma$ and $1-\mu$ fraction is allocated to the probability parameter $p_0$ with which a particular spherical cap is chosen i.e., $\bar{\varepsilon} = \mu \varepsilon$ and $\varepsilon_0 = (1-\mu) \varepsilon$ for some $\mu \in [0,1]$. While the parameter $\mu$ can be optimized over as described in \cite{FT21}, we will view it as a constant for convenience. Our results on $\MRC$ and $\MMRC$ simulating $\PrivUnit$ can be easily extended to the setup where $\mu$ needs to be optimized over.

\subsection{\texorpdfstring{$\PrivUnit$}{PrivUnit} is unbiased and order-optimal}
The following lemma borrowed from \cite{BDFKR2018} shows that the output of the $\PrivUnit$ mechanism  (a) is unbiased, (b) has a bounded norm, and (c) has order-optimal utility.

\begin{proposition}[\cite{BDFKR2018}]
 Let $\hat{\svbx}^{\texttt{pu}}$ = \emph{$\PrivUnit(\svbx,\gamma, p_0)$} for some $\svbx \in \sphere^{d-1}$, $\gamma
  \in [0,1]$, and $p_0 \in [1/2, 1]$. Then, $\E[\hat{\svbx}^{\texttt{pu}}] = \svbx$. Further, assume that $0 \le \varepsilon \le d$. Then, there exists a numerical constant $c < \infty$ such that if
  $\gamma$ saturates either of the two inequalities
  \eqref{eqn:sufficient-gamma}, then $\gamma \gtrsim \min\{\varepsilon / \sqrt{d},
  \sqrt{\varepsilon / d}\}$, and 
  \begin{equation*}
    \ltwo{\hat{\svbx}^{\texttt{pu}}}
    \le c \cdot \sqrt{\frac{d}{\varepsilon} \vee
      \frac{d}{(e^\varepsilon - 1)^2}}.
  \end{equation*}
  Additionally, $\E[\ltwo{\hat{\svbx}^{\texttt{pu}} - \svbx}^2] \lesssim
  \frac{d}{\varepsilon} \vee \frac{d}{(e^\varepsilon - 1)^2}$.
 \label{proposition:ltwo-utility}
\end{proposition}

% \subsection{The conditional distribution of the output of the $\PrivUnit$ mechanism}

% The randomness in the estimator $x(\svbz)$ obtained from the $\PrivUnit(\svbx,\gamma, p)$ mechanism comes from $\svbz$. Therefore, we obtain a convenient expression for the conditional distribution of $\svbz$ conditioned on $\svbx$ i.e., $\qPU(\svbz | \svbx)$.

% Recall from \eqref{eqn:sufficient-gamma} that $\gamma$ is a function of $\varepsilon$ and $d$ and therefore, $m$ from \eqref{eqn:norm-of-W} can be viewed as a function of $p$, $d$ and $\varepsilon$. Further, as described in Section \ref{subsubsec:privunit_dp}, when the budget split parameter $\mu$ is known, $p$ can viewed as a function of $\varepsilon$. Define $\msf{Cap}_{\varepsilon, d}(\svbx) = \{\svbz | \langle \svbx, \svbz \rangle \geq \gamma\}$. Then, the conditional distribution $\qPU(\svbz | \svbx)$ can be written as follows:
% \begin{align}
%     \qPU(\svbz | \svbx) \propto 
%     \begin{cases}
%       c_1(\varepsilon, d) & \text{if}\ \svbz \in \msf{Cap}_{\varepsilon, d}(\svbx)
%       \\[10pt]
%       c_2(\varepsilon, d) & \text{if}\ \svbz \notin \msf{Cap}_{\varepsilon, d}(\svbx)
%     \end{cases} \label{eq:privunit_density}
% \end{align}
% where $c_1(\varepsilon, d)$ and $c_2(\varepsilon, d)$ are some functions of $\varepsilon$ and $d$. 





% \subsection{Relationship between the debiasing scaling factors and mean squared errors}
% \begin{proposition}\label{proposition:mse_wrt_privunit}
%   Let $\check{q}(\svbz|\svbx)$ denote some privatization mechanism which debiases $\svbz \in \cZ$ by scaling it with $1/\check{m}$ i.e., $\check{x}(\svbz) \coloneqq \svbz/\check{m}$. If $m - \check{m} \leq \lambda\cdot\check{m}$, then 
%   \begin{align}
%     \Expectation_{\svbz \sim \check{q}(\svbz|\svbx)} \big[ \lV  \check{x}(\svbz) - \svbx \rV^2_2  \big] 
%     & \leq  \lp 1+\lambda \rp^2  \Expectation_{\svbz \sim q^{\varepsilon}_\msf{PrivUnit}(\svbz | \svbx)}\big[\|x(\svbz) - \svbx\|^2\big] \\
%     & \qquad + 2(1+\lambda)(2+\lambda) \sqrt{\Expectation_{\svbz \sim q^{\varepsilon}_\msf{PrivUnit}(\svbz | \svbx)}\big[\|x(\svbz) - \svbx\|^2\big]} + (2+\lambda)^2
% \end{align}
% \end{proposition}
% \begin{proof}
% We will start by upper bounding $1/m$ in terms of $\Expectation_{\svbz \sim q^{\varepsilon}_\msf{PrivUnit}(\svbz | \svbx)}\big[\|x(\svbz) - \svbx\|^2\big]$. First, observe that
% \begin{align}
%     \|x(\svbz) - \svbx\|  \stackrel{(a)}{\geq} \lV x(\svbz)\rV - \lV \svbx \rV  \stackrel{(b)}{\geq}
%     \frac{1}{m}  - 1  \label{eq:mrc_privunit_variance_1}
% \end{align}
% where $(a)$ follows from the triangle inequality and $(b)$ follows because $\|x(\svbz)\| = 1/m$ and $\|\svbx\| \leq 1$. Next, we have
% \begin{align}
%     \frac{1}{m} = \frac{1}{m} - 1 + 1 \stackrel{(a)}{\leq} \sqrt{\Expectation_{\svbz \sim q^{\varepsilon}_\msf{PrivUnit}(\svbz | \svbx)}\big[\|{x}(\svbz) - \svbx\|^2\big]} + 1 \label{eq:mrc_privunit_variance_2}
% \end{align}
% where $(a)$ follows from \eqref{eq:mrc_privunit_variance_1}. We will now upper bound $\Expectation_{\svbz \sim \check{q}(\svbz|\svbx)}[\| \check{x}(\svbz) - \svbx\|^2]$. We have
% \begin{align}
%     \Expectation_{\svbz \sim \check{q}(\svbz|\svbx)}[\| \check{x}(\svbz) - \svbx\|^2] 
%     & = \Expectation_{\svbz \sim \check{q}(\svbz|\svbx)} [\| \check{x}(\svbz)\|^2] + \lV \svbx \rV^2_2 + 2\lan \Expectation_{\svbz \sim \check{q}(\svbz|\svbx)} [\check{x}(\svbz)], \svbx\ran \\
%     & \stackrel{(a)}{\leq} \Expectation_{\svbz \sim \check{q}(\svbz|\svbx)} [\| \check{x}(\svbz)\|^2] + \lV \svbx \rV^2_2 + 2\sqrt{ \Expectation_{\svbz \sim \check{q}(\svbz|\svbx)} [\lV \check{x}(\svbz) \rV^2] \cdot \lV \svbx \rV^2 } \\
%     & \stackrel{(b)}{\leq} \lp\frac{1}{\check{m}}\rp^2  + 1 + \frac{2}{\check{m}}\\
%     & \stackrel{(c)}{\leq} \lp\frac{1+\lambda}{m}\rp^2 + 1 + \frac{2(1+\lambda)}{m} \\
%     & \stackrel{(d)}{\leq}
%      \lp 1+\lambda \rp^2  \Expectation_{\svbz \sim q^{\varepsilon}_\msf{PrivUnit}(\svbz | \svbx)}\big[\|x(\svbz) - \svbx\|^2\big] + 2(1+\lambda)(2+\lambda) \sqrt{\Expectation_{\svbz \sim q^{\varepsilon}_\msf{PrivUnit}(\svbz | \svbx)}\big[\|x(\svbz) - \svbx\|^2\big]} \\
%     & \qquad + (2+\lambda)^2
% \end{align}
% where $(a)$ follows from Cauchy–Schwarz inequality, $(b)$ follows because $\|\check{x}(\svbz)\| = 1/m$ and $\|\svbx\| \leq 1$, (c) follows from because $m - \check{m} \leq \lambda\cdot\check{m}$, and $(d)$ follows using \eqref{eq:mrc_privunit_variance_2} and some simple manipulations.
% \end{proof}

\subsection{\texorpdfstring{$\PrivUnit$}{PrivUnit} is a cap-based mechanism}
The randomness in the estimator $\hat{\svbx}^{\texttt{pu}}$ obtained from the $\PrivUnit(\svbx,\gamma, p_0)$ mechanism comes from $\svbz$. Therefore, we obtain a convenient expression for the conditional distribution of $\svbz$ conditioned on $\svbx$ i.e., $\qPU(\svbz | \svbx)$. Define $\msf{Cap}_{\svbx} \coloneqq \{\svbz | \langle \svbx, \svbz \rangle \geq \gamma\}$. Recall from \eqref{eqn:sufficient-gamma} that $\gamma$ is a function of $\varepsilon$ and $d$. 
% and therefore, $m_{\texttt{pu}}$ from \eqref{eqn:norm-of-W} can be viewed as a function of $p_0$, $d$ and $\varepsilon$. 
Further, as described in Section \ref{subsubsec:privunit_dp}, when the budget split parameter $\mu$ is known, $p_0$ can viewed as a function of $\varepsilon$. Then, the conditional distribution $\qPU(\svbz | \svbx)$ in \eqref{eq:qpu} can be written as follows:
\begin{align}
    \qPU(\svbz | \svbx) = 
    \begin{cases}
      c_1(\varepsilon, d) & \text{if}\ \svbz \in \msf{Cap}_{\svbx}
      \\[10pt]
      c_2(\varepsilon, d) & \text{if}\ \svbz \notin \msf{Cap}_{\svbx}
    \end{cases} \label{eq:privunit_density}
\end{align}
where $c_1(\varepsilon, d) = p_0 \times \dfrac{2}{A(1,d)I_{1-\gamma^2}(\frac{d-1}{2},\frac{1}{2})}$ and $c_2(\varepsilon, d) = (1-p_0) \times \dfrac{2}{2A(1,d)- A(1,d)I_{1-\gamma^2}(\frac{d-1}{2},\frac{1}{2})}$ are functions of $\varepsilon$ and $d$. 

Further, $\Probability_{\svbz \sim \Unif(\cZ)}\lp \svbz \in \cZ_{\svbx} \rp = \frac{I_{1-\gamma^2}(\frac{d-1}{2},\frac{1}{2})}{2}$. Therefore,
\begin{align}
    \frac{c_1(\varepsilon, d)}{c_2(\varepsilon, d)} \times \Probability_{\svbz \sim \Unif(\cZ)}\lp \svbz \in \cZ_{\svbx} \rp = \frac{p_0 \times (2 - I_{1-\gamma^2}(\frac{d-1}{2},\frac{1}{2}))}{2(1-p_0)} \stackrel{(a)}{\geq} \frac{p_0}{2(1-p_0)} \stackrel{(b)}{\geq} \frac{1}{2}
\end{align}
where $(a)$ follows because $I_{1-\gamma^2}(\frac{d-1}{2},\frac{1}{2}) \leq 1$ and $(b)$ follows because $p_0 \geq 1/2$.
