\usepackage{bm}
\usepackage{color}
\usepackage{booktabs} 
\usepackage{nicefrac}
\usepackage{float}
\usepackage{caption}
\usepackage{subcaption}
\usepackage{mathtools}
\usepackage{soul}
\mathtoolsset{showonlyrefs}
\usepackage{enumitem}
\usepackage{multirow}
\usepackage{scalerel}
\usepackage{url}
% \usepackage[multiple]{footmisc}
% \usepackage{hyperref}

% \hypersetup{colorlinks=true,linkbordercolor=red,linkcolor=blue}
\usepackage{xcolor}
\usepackage[colorlinks = true,
            linkcolor = blue,
            urlcolor  = blue,
            citecolor = blue,
            anchorcolor = blue]{hyperref}

%\usepackage{subfig}
\usepackage{mathtools}
\usepackage{amsmath,mathrsfs,amsfonts,amssymb,amsthm}
\usepackage{thmtools,thm-restate}
\newtheorem{lemma}{Lemma}[section]
\newtheorem{theorem}{Theorem}[section]
\newtheorem{proposition}{Proposition}[section]
\newtheorem{assumption}{Assumption}[section]
\newtheorem{corollary}{Corollary}[section]
\newtheorem{definition}{Definition}[section]
\newtheorem{remark}{Remark}[section]
\newtheorem{claim}{Claim}[section]
\newtheorem{fact}{Fact}[section]
%\newtheorem{example}{Example}[section]
% \newtheorem{conj}{Conjecture}[section]
% \newtheorem{proofpart}{Part}
%\makeatletter
%\@addtoreset{proofpart}{theorem}
%\makeatother
% \newtheorem{proof}{Proof}[section]
% \usepackage[ruled,vlined]{algorithm2e}
\usepackage{tcolorbox}
\usepackage{microtype}      % microtypography
\usepackage{xcolor}         % colors
\usepackage[normalem]{ulem}
\usepackage[none]{hyphenat}
\usepackage{breakcites}
% \usepackage{algorithm}
\usepackage[noend]{algpseudocode}
\usepackage[ruled,vlined]{algorithm2e}
\newcommand\mycommfont[1]{\footnotesize\ttfamily\textcolor{blue}{#1}}
\SetCommentSty{mycommfont}
\SetKwInput{KwInput}{Input}                % Set the Input
\SetKwInput{KwOutput}{Output}  
\SetKwInput{KwInitialization}{Initialization}

\usepackage{dsfont}
\allowdisplaybreaks
% If you use BibTeX in apalike style, activate the following line:
%\bibliographystyle{apalike}
\newcommand{\textBlue}[1]{{\leavevmode\color{blue}#1}} % blue color over text and math
\newcommand{\textRed}[1]{{\leavevmode\color{red}#1}} % red color over text and math

% % maximum numbers of floats at top and bottom of the page, as well as total
% \setcounter{topnumber}{20}
% \setcounter{bottomnumber}{20}
% \setcounter{totalnumber}{20}

% % maximum fraction of page for floats at top and bottom
% \renewcommand{\topfraction}{.9}
% \renewcommand{\bottomfraction}{.9}
% \renewcommand{\dbltopfraction}{.9}

% % minimum fraction of text page for text, of float page for floats
% \renewcommand{\textfraction}{0}
% \renewcommand{\floatpagefraction}{.8}
% \renewcommand{\dblfloatpagefraction}{.8}

% % separation between floats and text
% \addtolength{\textfloatsep}{-1.5ex}
% % % separation between floats and caption
% \addtolength{\abovecaptionskip}{-1.5ex}

% \clubpenalty=0
% \widowpenalty=0
% \displaywidowpenalty=0

% % whitespace below and above display equations
% \abovedisplayskip=0pt
% \abovedisplayshortskip=0pt
% \belowdisplayskip=0pt
% \belowdisplayshortskip=0pt

% \allowdisplaybreaks[2]
% \interdisplaylinepenalty=2500

