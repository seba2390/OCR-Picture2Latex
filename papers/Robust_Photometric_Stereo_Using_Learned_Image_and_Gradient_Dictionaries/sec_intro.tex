%!TEX root = PSviaDL.tex

Photometric stereo \cite{woodham1980} is a method for estimating the normal vectors of an object from images of the object under varying lighting conditions. Since its inception, a significant amount of work has been done extending photometric stereo to more general conditions. This body of work has been divided into two primary areas. Uncalibrated photometric stereo seeks to solve the photometric stereo problem when the lighting directions are unknown \cite{hayakawa1994,belhumeur1999,yuille1999,georghiades2003}, while robust photometric stereo algorithms attempt to estimate the normal vectors of an object when the surface violates the assumptions of the underlying model (usually, the Lambertian reflectance model). In this work, we are primarily concerned with the latter problem.
%chandraker2005,alldrin2007,shi2010,favaro2012,wu2013,queau2015

The Lambertian reflectance model states that the observed intensity of a point on a surface is linearly proportional to the direction the surface is illuminated and the object's normal vectors \cite{woodham1980}. While this assumption holds in some cases, shadows, specularities, and other non-idealities can cause this model to break down. A variety of techniques have been developed to compensate for these corruptions. Several works seek to model non-Lambertian effects as outliers and employ a framework to estimate these outliers and discard them from the data, leaving only Lambertian data behind \cite{barsky2003,chandraker2007,verbiest2008,yu2010,wu2010}. Of particular interest in this category are the more recent works by Wu \textit{et al}. \cite{wu2011} and Ikehata \textit{et al}. \cite{ikehata2012}. These works model the outliers as a sparse matrix, casting the problem as a matrix completion problem, and use robust PCA or sparse regression, respectively, to solve for some Lambertian representation of the data. Other works have proposed more complicated reflectance models to account for non-Lambertian effects, eliminating the need to discard non-ideal data \cite{oren1995,hertzmann2005,alldrin2007_2,chung2008,alldrin2008,seitz2010,higo2010,shi2012,chandraker2013}.  The current state-of-the-art in this category are the works by Ikehata \textit{et al}. \cite{ikehata2014} and Shi \textit{et al}. \cite{shi2014}.


In this work, we propose two new approaches for photometric stereo that are robust to noisy data. Our methods utilize a dictionary learning model \cite{elad2006image,aharon2006rm} to handle non-idealities and impose some adaptive structure on the data. Our approach is motivated in part by the recent success of dictionary learning in other imaging domains \cite{ravishankar2011mr,ravishankar2016lassi}.
We demonstrate the viability and performance of our methods on several datasets with varying degrees of non-ideality---including the recently proposed DiLiGenT dataset \cite{shi2016}---comparing them to the performance of state-of-the-art methods. In particular, we investigate the ability of our methods to handle general, non-sparse errors and noise.
%kreutz2003dictionary
%ravishankar2016low

The rest of this paper is organized as follows. Section 2 introduces the photometric stereo problem. Section 3 presents our dictionary learning based formulations and details their implementation. Finally, in Section 4, we demonstrate the performance of our methods on a variety of datasets.

