%!TEX root = PSviaDL.tex

The Lambertian reflectance model states that, given an image of a \emph{Lambertian} object, the light intensity observed at point $(x,y)$ on the surface satisfies
\begin{equation} \label{ps1}
I(x,y) = \rho(x,y) \ell^T n(x,y),
\end{equation}
where $I(x,y)$ is the image intensity, $\ell \in \mathbb{R}^3$ is the direction of the light source incident on the surface, $n(x,y)$ is the normal vector of the surface, and $\rho (x,y)$ is the surface albedo---a measure of the reflectivity of the surface. 

If we fix the position of a camera facing our surface and vary the position of the light source over $d$ unique locations, we can write $d$ equations of the form \eqref{ps1}, which can be stacked to form the system of equations
\begin{equation} \label{ps2}
[I_1(x,y)~\ldots~I_d(x,y)]^T = [\ell_1~\ldots~\ell_d]^T \rho(x,y) n(x,y).
\end{equation}
Assuming each of the $d$ images has size $m_1 \times m_2$, Equation \eqref{ps2} can then be solved $m_1 m_2$ times to obtain the normal vectors of the object at each point on its surface. These equations can also be combined into a single matrix equation. Indeed, let us define the observation matrix
\begin{equation} \label{ddef}
Y := \left [\textbf{vec}(I_1)~\ldots~\textbf{vec}(I_d) \right ] \in \mathbb{R}^{m_1 m_2 \times d},
\end{equation}
where $\textbf{vec}(I_j) := [I_j(1,1)~\ldots~I_j(m_1,m_2)]^T$. Assuming our light source is at infinity and there is no variation in illumination from point to point on our object, one can succinctly express \eqref{ps2} as
\begin{equation} \label{ps3}
Y = NL,
\end{equation}
where $N = [\rho(1,1) n(1,1)~\ldots~\rho(m_1,m_2) n(m_1,m_2)]^T \in \mathbb{R}^{m_1 m_2 \times 3}$ and $L = [\ell_1~\ldots~\ell_d] \in \mathbb{R}^{3 \times d}$. For simplicity, we assume that $\|\ell_k\|_2=1$, and, without loss of generality, we assume that $n(x,y)$ are \emph{unit} normals.

Given $d \geq 3$ images and their corresponding light directions, one can solve \eqref{ps3} exactly to obtain the normal vector matrix $N$, from which one can compute the full 3D representation of the underlying surface \cite{simchony1990}.

In theory, \eqref{ps3} should hold exactly for a Lambertian surface, but, in practice, due to noise and other non-idealities, one only expects that $Y \approx NL$. In the latter case, one can instead collect $d > 3$ measurements and solve the overdetermined least squares problem
\begin{equation} \label{eq:ls}
\min_{N} \ \left \| Y - NL \right \|_F^2,
\end{equation}
which has the convenient closed-form solution $\hat{N} = YL^\dagger$, where $\dagger$ denotes the Moore-Penrose pseudoinverse.
