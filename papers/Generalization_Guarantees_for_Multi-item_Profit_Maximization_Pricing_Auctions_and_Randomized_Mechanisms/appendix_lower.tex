

\begin{theorem}
	\label{thm:lower}
	For a class of auctions $\pazocal{M}$, let $N_{\pazocal{M}}(\epsilon, \delta)$ be the number of samples required to ensure that for any distribution $\pazocal{D}$, with probability at least $1-\delta$ over the draw of a set of samples of size $N_{\pazocal{M}}(\epsilon, \delta)$ from $\pazocal{D}$, for all auctions $M \in \pazocal{M}$, average profit is $\epsilon$-close to expected profit. 
	\begin{enumerate}
		\item If $\pazocal{M}$ is the set of AMAs or $\lambda$-auctions, then $N_{\pazocal{M}}(\epsilon, \delta) \geq \frac{n^m - n}{2}.$
		\item If $\pazocal{M}$ is the set of VVCAs, then $N_{\pazocal{M}}(\epsilon, \delta) \geq 2^m - 2.$
	\end{enumerate}
\end{theorem}

First, we prove part 1 and then we prove part 2.

\bigskip\emph{Lower Bound on Sample Complexity for $\lambda$-Auctions.}
We prove that $N_{\pazocal{M}}(\epsilon, \delta) \geq \frac{n^m-n}{2}$ samples are required to ensure that for any distribution $\pazocal{D}$, with probability at least $1-\delta$ over the draw of a set of samples of size $N_{\pazocal{M}}(\epsilon, \delta)$ from $\pazocal{D}$, for all $\lambda$-auctions $M \in \pazocal{M}$, average profit is $\epsilon$-close to expected profit.
 Since $\lambda$-auctions are a subset of AMAs, this lower bound applies to AMAs as well.

To prove Theorem~\ref{thm:AMAlower}, we construct a set $V$ of $n$-bidder, $m$-item valuation functions taking values in $\{0,1\}$ where, under each valuation function, each bidder is interested in a specific subset of items, and these subsets are all pairwise disjoint. Moreover, $|V| = n^m - n$. The high level idea is to show that for any subset $H$ of $V$, there exists a $\lambda$-auction that has high profit over valuation functions in $H$, but low profit on the valuation functions in $V \setminus H$. Theorem~\ref{thm:AMA_high_low} describes $V$ in more detail. Now suppose that the distribution over the bidders' valuation functions is the uniform distribution over $V$. This means that if a set of samples consist of only a small subset of $V$, then we cannot guarantee that every profit function will achieve average profit over the set of samples which is close to its expected profit over the distribution, as we require.

We now present Theorem~\ref{thm:AMA_high_low}, wherein we describe the set $V$ of valuation functions which we will use to prove Theorem~\ref{thm:AMAlower}.

\begin{theorem}\label{thm:AMA_high_low}
For any $n,m \geq 2$ and any $\beta \in (0,1)$, there exists a set of $N = n^m-n$ $n$-bidder, $m$-item additive valuation functions $V = \left\{\vec{v}^1, \dots, \vec{v}^N\right\}$ such that for any $H \subseteq V$, there exists a $\lambda$-auction $M_H$ with profit 0 on $\vec{v}^i$ if $\vec{v}^i \not\in H$ and profit at least $2-2\beta$ on $\vec{v}^i$ otherwise.
\end{theorem}

\begin{proof}
We define the set $V = \left\{\vec{v}^1, \dots, \vec{v}^N\right\}$ of $n$-bidder, $m$-item additive valuation functions, where \[\vec{v}^j = \left(v_1^j(\vec{e}_1), \dots, v_1^j(\vec{e}_m), \dots, v_n^j(\vec{e}_1) \dots, v_n^j(\vec{e}_m)\right),\] with $N = n^m - n$. Recall that every allocation $Q$ is written as $\left(\vec{q}_1,\dots,  \vec{q}_n\right)$ where $\vec{q}_1,\dots,  \vec{q}_n$ are disjoint subsets of the $m$ items being auctioned. First, let $\hat{Q}^j$ be the allocation where bidder $j$ receives all $m$ items. Next, let $\tilde{Q}^1, \dots, \tilde{Q}^N$ be a fixed ordering of the $n^m-n$ allocations where all $m$ items are allocated except $\hat{Q}^1, \dots, \hat{Q}^n$. Let the bundles allocated to the $n$ bidders in $\tilde{Q}^\ell$ be $\tilde{\vec{q}}^{\ell}_1, \dots, \tilde{\vec{q}}^{\ell}_n$ and let $S_\ell$ be the set of bidders who are allocated some item in allocation $\tilde{Q}^{\ell}$. In other words, $S_{\ell} = \left\{j \ | \ \tilde{\vec{q}}^{\ell}_j \not=\vec{0}\right\}$. For a sanity check, notice that $\sum_{i \in S_{\ell}} \tilde{\vec{q}}^{\ell}_i = \vec{1}$.

We will now define the valuation vectors $\left\{\vec{v}^1, \dots, \vec{v}^N\right\}$ in terms of this set of special allocations $\tilde{Q}^1, \dots, \tilde{Q}^N$, so each vector $\vec{v}^{\ell}$ depends on the allocation $\tilde{Q}^{\ell}$. Specifically, we define $\vec{v}^\ell$ for $\ell \in [N]$ as follows.
If $i \not\in S_\ell$ $\left( \text{i.e., }\tilde{\vec{q}}^\ell_i = \vec{0}\right)$, set $v_i^\ell(\vec{e}_j) = 0$ for all $j \in [m]$. Otherwise, set \[v_i^\ell(\vec{e}_j) = \begin{cases} 0 &\text{if } \tilde{\vec{q}}^\ell_i[j] = 0\\
1 &\text{if } \tilde{\vec{q}}^\ell_i[j] = 1
\end{cases}.\]

We proceed to prove that for any subset $H \subseteq V$, there exists a $\lambda$-auction with 0 profit on all valuation functions in $V \setminus H$ and at least $2-2\beta$ profit on all valuation functions in $H$. To define this $\lambda$-auction, we set the $\lambda$ terms such that \[\lambda\left(Q\right) = \begin{cases}
0 &\text{if } Q = \tilde{Q}^\ell \text{ for some } \vec{v}^\ell \in H\\
1 - \beta &\text{otherwise}\end{cases}.\]


\begin{lemma}\label{lemma:n_bid_high}
If $\vec{v}^\ell \in H$, then the profit on $\vec{v}^\ell$ is at least $2-2\beta$.
\end{lemma}

\begin{proof}[Proof of Lemma~\ref{lemma:n_bid_high}]
First, note that $\sum_{i = 1}^n v_i^\ell\left(\tilde{\vec{q}}_i^{\ell}\right) + \lambda\left(\tilde{Q}^\ell\right) = m$, and for all allocations $Q = \left(\vec{q}_1, \dots, \vec{q}_n\right) \not = \tilde{Q}^\ell$, $\sum_{i = 1}^n v_i^\ell\left(\vec{q}_i\right) + \lambda\left(Q\right) \leq m-1+1-\beta < m$. Therefore, the $\lambda$-auction allocation is $\tilde{Q}^\ell$.

In order to analyze the profit of this $\lambda$-auction, we must understand the payments of each bidder, which means that we must investigate what the outcome of this $\lambda$-auction would be without any one bidder's participation. To this end, suppose $i \in S_\ell$, so bidder $i$ is allocated some item in $\tilde{Q},$ i.e., $\tilde{\vec{q}}_i^{\ell} \not= \vec{0}$. Then $\sum_{j \not= i} v_j^\ell\left(\tilde{\vec{q}}_j^\ell\right) + \lambda\left(\tilde{Q}^\ell\right) = m-\left|\left|\tilde{\vec{q}}_i^{\ell}\right|\right|_1$ because bidder $i$'s valuation for the bundle $\tilde{\vec{q}}_i^{\ell}$ is exactly $\left|\left|\tilde{\vec{q}}_i^{\ell}\right|\right|_1$.

By construction, no bidder receives all $m$ items in $\tilde{Q}^\ell$, so we know that there exists some $i' \in S_\ell, i' \not = i$. With this fact in mind, let $\tilde{Q}^{\ell,-i} = \left(\tilde{\vec{q}}^{\ell, -i}_1, \dots, \tilde{\vec{q}}^{\ell,-i}_n\right)$ be the allocation where all bidders in $S_\ell$ are allocated the same items as they are in $\tilde{Q}^\ell$ and bidder $i$ receives the empty set. This is one possible allocation of the $\lambda$-auction without bidder $i$'s participation, and therefore the social welfare of the other bidders will be at least as high under this allocation as it would be in the true allocation of the $\lambda$-auction without bidder $i$'s participation. By construction, $\lambda\left(\tilde{Q}^{\ell,-i}\right) = 1 - \beta$. Therefore, $\sum_{\ell \not= i} v_j^\ell \left(\tilde{\vec{q}}_j^{\ell,-i}\right) + \lambda\left(\tilde{Q}^{\ell,-i}\right) = m-\left|\left|\tilde{\vec{q}}_i^{\ell}\right|\right|_1 + 1 - \beta$ which means that bidder $i$ must pay at least $\left(m-\left|\left|\tilde{\vec{q}}_i^{\ell}\right|\right|_1 + 1 - \beta\right) - \left(m-\left|\left|\tilde{\vec{q}}_i^{\ell}\right|\right|_1\right) = 1 - \beta.$ We know that $|S_\ell|\geq 2$, i.e., there are at least 2 bidders who receive a non-empty bundle and therefore must pay at least $1 - \beta$, so the profit of this $\lambda$-auction is at least $2-2\beta$.
\end{proof}

\begin{lemma}\label{lemma:n_bid_low} If $\vec{v}^\ell \not\in H$, then the profit on $\vec{v}^\ell$ is 0.
\end{lemma}

\begin{proof}[Proof of Lemma~\ref{lemma:n_bid_low}]
First, note that $\sum_{i = 1}^n v_i^\ell\left(\tilde{\vec{q}}_i^{\ell}\right) + \lambda\left(\tilde{Q}^\ell\right) = m + 1-\beta$, and for all allocations $Q = \left(\vec{q}_1, \dots, \vec{q}_n\right) \not= \tilde{Q}^\ell$, $\sum_{i = 1}^n v_i^\ell\left(\vec{q}_i\right) + \lambda\left(Q\right) \leq m-1 + 1-\beta < m$, so the $\lambda$-auction allocation is $\tilde{Q}^\ell$. Now, suppose $i \in S_\ell$. Then $\sum_{j \not= i} v_j^\ell\left(\tilde{\vec{q}}_j^\ell\right) + \lambda\left(\tilde{Q}^\ell\right) = m - \left|\left|\tilde{\vec{q}}_i^{\ell}\right|\right|_1 + 1 - \beta.$ Since bidder $i$ is the only bidder with nonzero valuations for the items in $\tilde{\vec{q}}_i^{\ell}$ under $\vec{v}^\ell$, any allocation $\tilde{Q}^{\ell,-i}$ without his participation will have social welfare at most $\sum_{j \not= i} v_j^\ell\left(\tilde{\vec{q}}_j^{\ell,-i}\right) + \lambda\left(\tilde{Q}^{\ell,-i}\right) \leq m - \left|\left|\tilde{\vec{q}}_i^{\ell}\right|\right|_1 + 1 - \beta.$ Therefore, bidder $i$ pays nothing.

Of course, for any bidder $i \not\in S_\ell$, her presence in the auction makes no difference on the resulting allocation because her valuation function under $\vec{v}^\ell$ is 0 on all items, so he pays nothing as well. Therefore, the profit on $\vec{v}^\ell$ is 0.
\end{proof}

Putting Lemmas~\ref{lemma:n_bid_high} and~\ref{lemma:n_bid_low} together, we have the desired result.
\end{proof}

We now use Theorem~\ref{thm:AMA_high_low} to prove Theorem~\ref{thm:AMAlower}.

\begin{theorem}\label{thm:AMAlower}
For any $\epsilon \in (0,1)$, there exists a distribution $\pazocal{D}$ and a $\lambda$-auction $M^*$ such that, with probability 1 over the draw of a set of samples $\sample$ of size at most $\frac{n^m-n}{2}$, \[\frac{1}{|\sample|}\sum_{\vec{v} \in \sample} {\normalfont \profit}_{M^*} \left(\vec{v}\right) - \E_{\vec{v}\sim \pazocal{D}}\left[ {\normalfont \profit}_{M^*}\left(\vec{v}\right)\right] > \epsilon.\]
\end{theorem}

\begin{proof}
Let $\beta = 1 - \epsilon$ and let $V$ be the set of valuation functions proven to exist in Theorem~\ref{thm:AMA_high_low} corresponding to $\beta$ (i.e. for any $H \subseteq V$, there exists a $\lambda$-auction $M_H$ with profit 0 on $\vec{v}$ if $\vec{v} \in H$ and profit at least $2-2\beta$ on $\vec{v}$ otherwise). Let $\pazocal{D}$ be the uniform distribution on $V$.

Suppose that $\sample$ is a set of at most $\frac{n^m-n}{2}$ samples. Of course, $\sample \subseteq V$, so let $M^*$ be the $\lambda$-auction with 0 profit on every valuation function not in the set of samples and profit at least $2 - 2\beta$ on every valuation function in the set of samples. We know that $M^*$ exists due to Theorem~\ref{thm:AMA_high_low}.

Notice that the average empirical profit of $M^*$ on $\sample$ is at least $2 - 2\beta$. Meanwhile, the probability, on a random draw $\vec{v} \sim \pazocal{D}$ that ${\normalfont \profit}_{M^*}\left(\vec{v}\right)$ is 0 is exactly the probability that $\vec{v} \not\in \sample$. Given that the set of training examples has measure $\frac{|\sample|}{n^m - n}\leq\frac{1}{2},$ we have that \begin{align*}
&\frac{1}{|\sample|}\sum_{\vec{v} \in \sample} {\normalfont \profit}_{M^*} \left(\vec{v}\right) - \E_{\vec{v}\sim \pazocal{D}}\left[ {\normalfont \profit}_{M^*}\left(\vec{v}\right)\right] \geq 2-2\beta - (2-2\beta)\Pr_{\vec{v}\sim\pazocal{D}}\left[\vec{v} \in \sample\right]\\
>\text{ }&2-2\beta - (1-\beta) = 1-\beta = \epsilon.
\end{align*}
as desired.
\end{proof}

\bigskip\emph{Lower Bound on Sample Complexity for VVCAs.}We now prove that it is not possible to learn over the set of VVCA profit function under and arbitrary distribution with subexponential sample complexity. In particular, we prove that no algorithm can learn over the class of $n$-bidder, $m$-item VVCA profit functions with sample complexity $2^m - 2$. This holds even when the bidders' valuation functions are additive.

The format of this proof similar to that of Theorem~\ref{thm:AMAlower}. Namely, we construct a set $V$ of $n$-bidder, $m$-item valuation functions such that $|V| = 2^m - 2$. We then show that for any subset $H$ of $V$, there exists a VVCA that has high profit over valuation functions in $H$, but low profit on the valuation functions in $V \setminus H$. The set $V$ is described in more detail in Theorem~\ref{thm:VVCA_high_low}. As described in Theorem~\ref{thm:AMAlower}, this immediately implies hardness for learning over the uniform distribution on $V$. Given the parallel proof structure, we present Theorem~\ref{thm:VVCA_high_low} and refer the reader to Theorem~\ref{thm:AMAlower} to see how it implies hardness for learning.

\begin{theorem}\label{thm:VVCA_high_low}
For any $m \geq 2$ and any $\beta \in (0,1)$, there exists a set of $N = 2^m-2$ 2-bidder additive valuation functions $V = \{\vec{v}^1, \dots, \vec{v}^N\}$ such that for any $H \subseteq V$, there exists a VVCA with profit 0 on $\vec{v}^i$ if $\vec{v}^i \in V$ and profit $1-\beta$ on $\vec{v}^i$ if $\vec{v}^i \not\in V$.
\end{theorem}

\begin{proof}
We define the set $V = \{\vec{v}^1, \dots, \vec{v}^N\}$ of 2-bidder valuation functions, where \[\vec{v}^j = (v_1^j(\vec{e}_1), \dots, v_1^j(\vec{e}_m), v_2^j(\vec{e}_1) \dots, v_2^j(\vec{e}_m)),\] with $N = 2^m - 2$. Recall that every allocation vector $Q$ can be written as $\left(\vec{q}_1, \vec{q}_2\right)$ where $\vec{q}_1$ and $\vec{q}_2$ are disjoint subsets of the $m$ items being auctioned. In order to define the valuation functions in $V$, we define $\tilde{\vec{q}}^1, \dots, \tilde{\vec{q}}^N$ to be a arbitrary, fixed ordering of the vectors in the set $\{0,1\}^m \setminus \{\vec{0}, \vec{1}\}$. We will define each valuation function in $V$ in terms of this ordering. In particular, let $\tilde{Q}^\ell = \left(\vec{1} - \tilde{\vec{q}}^\ell, \tilde{\vec{q}}^\ell\right)$ be the allocation where bidder 1 receives $\vec{1} - \tilde{\vec{q}}^\ell$ and bidder 2 receives $\tilde{\vec{q}}^\ell$. Finally, let $\vec{v}^\ell$ for $\ell \in [N]$ be defined as follows:

\[v_1^\ell(\vec{e}_i) = \begin{cases} 1 &\text{if } \tilde{\vec{q}}^\ell[i] = 0\\
0 &\text{otherwise}
\end{cases} \text{ and } v_2^\ell(\vec{e}_i) = \begin{cases} 1 &\text{if } \tilde{\vec{q}}^\ell[i] = 1\\
0 &\text{otherwise}
\end{cases}.\]

Clearly, if $w_1=w_2=1$ and $\lambda_1(Q) = \lambda_2(Q) = 0$ for all $Q \in \Qset$, then the VVCA allocation on any $\vec{v}^\ell \in S$ is the one in which bidder 2 receives $\vec{1} - \tilde{\vec{q}}^\ell$ and bidder 1 receives $\tilde{\vec{q}}^\ell$. This has a social welfare of $m$, whereas any other allocation has a social welfare at most $m-1$.

We claim that for any $H \subseteq V$, there exists a VVCA with profit 0 on $\vec{v}^i$ if $\vec{v}^i \in H$ and profit $1-\beta$ on $\vec{v}^i$ if $\vec{v}^i \not\in H$. The VVCA has bidder weights $w_1 = w_2 = 1$, and for all $\vec{v}^\ell \in H$, we set $\lambda_1(\tilde{Q}^\ell) = c_{1,\vec{1} - \tilde{\vec{q}}^\ell} = c_{2, \tilde{\vec{q}}^\ell} = \lambda_2(\tilde{Q}^\ell) = 0$. Otherwise, we set $\lambda_i(Q) = (1 - \beta)/2$ for each $i \in \{1,2\}$.

\begin{lemma}\label{lem:VVCA_high} If $\vec{v}^\ell \in H$, then the profit on $\vec{v}^\ell$ is $1-\beta$.
\end{lemma}

\begin{proof}[Proof of Lemma~\ref{lem:VVCA_high}]
First, note that $v_1(\vec{1} - \tilde{\vec{q}}^\ell) + v_2(\tilde{\vec{q}}^\ell) +\lambda_1(\tilde{Q}^\ell) + \lambda_2(\tilde{Q}^\ell) = m$, and for all allocations $Q = \left(\vec{q}_1, \vec{q}_2\right) \not = \tilde{Q}^\ell$, $v_1(\vec{q}_1) + v_2(\vec{q}_2) + \lambda_1(Q) + \lambda_2(Q) \leq m-1 + 1-\beta$. Therefore, the VVCA allocation is $\tilde{Q}^\ell$. However, this is neither bidder 1 nor bidder 2's favorite weighted allocation, since $v_1(\vec{1} - \tilde{\vec{q}}^\ell) + \lambda_1(\tilde{Q}^\ell) = |\vec{1} - \tilde{\vec{q}}^\ell| < v_1(\vec{1}) + c_{1,\vec{1}} = |\vec{1} - \tilde{\vec{q}}^\ell| + (1-\beta) /2$ and $v_2(\tilde{\vec{q}}^\ell) + \lambda_2(\tilde{Q}^\ell) = |\tilde{\vec{q}}^\ell|< v_2(\vec{1}) + c_{2,\vec{1}} = |\tilde{\vec{q}}^\ell| + (1-\beta) /2$. This follows from the fact that $\tilde{\vec{q}}^\ell \not = \vec{1}$ and $\vec{1} - \tilde{\vec{q}}^\ell \not = \vec{1}$ for all $\ell \in [N]$, so it must be that $c_{1,\vec{1}} = c_{2,\vec{1}} = (1 - \beta)/2.$

Since $|\vec{1} - \tilde{\vec{q}}^\ell|$ and $|\tilde{\vec{q}}^\ell|$ are bidder 1 and 2's highest valuations for any allocation, respectively, and because $(1-\beta)/2$ is the highest value of any $\lambda$ term, $v_1(\vec{1}) + c_{1,\vec{1}}$ and $v_2(\vec{1}) + c_{2,\vec{1}}$ are the maximum weighted valuation that either bidder has for any allocation under this VVCA. Therefore, the profit of this VVCA on $\vec{v}_\ell$ is $|\tilde{\vec{q}}^\ell| + |\vec{1} - \tilde{\vec{q}}^\ell| + 1-\beta - |\tilde{\vec{q}}^\ell| - |\vec{1} - \tilde{\vec{q}}^\ell| = 1-\beta$.
\end{proof}

\begin{lemma}\label{lem:VVCA_low} If $\vec{v}^\ell \not \in H$, then the profit on that valuation function pair is 0.
\end{lemma}

\begin{proof}[Proof of Lemma~\ref{lem:VVCA_low}]
First, note that $v_1(\vec{1} - \tilde{\vec{q}}^\ell) + v_2(\tilde{\vec{q}}^\ell) +\lambda_1(\tilde{Q}^\ell) + \lambda_2(\tilde{Q}^\ell) = m+1-\beta$, and for all allocations $Q = \left(\vec{q}_1, \vec{q}_2\right) \not= \tilde{Q}^\ell$, $v_1(\vec{q}_1) + v_2(\vec{q}_2) + \lambda_1(Q) + \lambda_2(Q) \leq m-1 + 1-\beta < m + 1-\beta$, so the AMA allocation is $\tilde{Q}^\ell$. Moreover, for all allocations $Q = (\vec{q}_1, \vec{q}_2)$, $v_1(\vec{1} - \tilde{\vec{q}}^\ell) + \lambda_1(\tilde{Q}^\ell) = |\vec{1} - \tilde{\vec{q}}^\ell| + (1-\beta)/2 \geq v_1(\vec{q}_1) + \lambda_1(Q)$ and  $v_2(\tilde{\vec{q}}^\ell) + \lambda_2(\tilde{Q}^\ell) = |\tilde{\vec{q}}^\ell| + (1-\beta)/2 \geq v_2(\vec{q}_2) + \lambda_2(Q)$. Therefore, both bidders receive one of their favorite weighted allocations, so the profit is 0.
\end{proof}
\end{proof}