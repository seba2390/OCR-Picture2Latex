
\begin{figure}
	\centering
	\begin{subfigure}{0.45\textwidth}
		\includegraphics{2shatter1}\centering
		\caption{Illustration of an affine function $f^{(1)} : \R \to \R$ where $f^{(1)}\left(x^{(1)}\right)$ is larger than the witness $z^{(1)}$ and $f^{(1)}\left(x^{(2)}\right) < z^{(2)}$.}
	\end{subfigure}\qquad
	\begin{subfigure}{0.45\textwidth}
		\includegraphics{2shatter2}\centering
		\caption{Illustration of another affine function $f^{(2)} : \R \to \R$. Here, $f^{(2)}\left(x^{(1)}\right) < z^{(1)}$ and $f^{(2)}\left(x^{(2)}\right) > z^{(2)}$.}
	\end{subfigure}\qquad
	\begin{subfigure}{0.45\textwidth}
		\includegraphics{2shatter3}\centering
		\caption{Illustration of a third function $f^{(3)}$. Here, $f^{(3)}\left(x^{(1)}\right) > z^{(1)}$ and $f^{(3)}\left(x^{(2)}\right) > z^{(2)}$.}
	\end{subfigure}\qquad
	\begin{subfigure}{0.45\textwidth}
		\includegraphics{2shatter4}\centering
		\caption{Illustration of one last function $f^{(4)}$. Here, $f^{(4)}\left(x^{(1)}\right) < z^{(1)}$ and $f^{(4)}\left(x^{(2)}\right) < z^{(2)}$.}
	\end{subfigure}
	\caption{The two points $x^{(1)}$ and $x^{(2)}$ can be shattered by the set $\pazocal{F}$ of affine functions mapping $\R$ to $\R$.}\label{fig:shattering}
\end{figure}


\begin{corollary}\label{cor:add_apx}
	Let $M^* \in \cM$ be the mechanism that maximizes expected profit over the distribution over buyers' values. For any $\delta \in (0,1)$, with probability at least $1-\delta$ over the draw of a set of samples $\sample$ of size $N$ from the distribution over buyers' values, the difference between the expected profit of $\hat{M}_{\rho}$ and expected profit of $M^*$ is at most $\rho + \epsilon_{\pazocal{M}}\left(N, \frac{\delta}{2}\right) + U\sqrt{\frac{1}{2N} \ln \frac{4}{\delta}}.$
\end{corollary}

\begin{proof}
Let $\pazocal{M}(\sample)$ be the mechanism in $\pazocal{M}$ that maximizes empirical profit over $\sample$. With probability at least $1-\delta$,
\begin{align}
 {\normalfont \profit}_{\pazocal{D}}\left(\hat{M}_{\rho}\right) + \epsilon_{\pazocal{M}}\left(N, \frac{\delta}{2}\right) &\geq   {\normalfont \profit}_{\sample}\left(\hat{M}_{\rho}\right)\label{ineq:1a} \\
&\geq {\normalfont \profit}_{\sample}\left(\pazocal{M}(\sample)\right) - \rho\label{ineq:2a}\\ 
&\geq  {\normalfont \profit}_{\sample}\left({M^*}\right) - \rho\label{ineq:3a} \\
& \geq {\normalfont \profit}_{\pazocal{D}}\left({M^*}\right) -  U \sqrt{\frac{2\ln(4/\delta)}{2N}}-\rho.\label{ineq:4a}
\end{align}


Inequality~\eqref{ineq:1a} follows from standard uniform convergence bounds: with probability at least $1-\delta/2$, \[\left|{\normalfont \profit}_{\dist}\left(\hat{M}_{\rho}\right) - {\normalfont \profit}_{\sample}\left(\hat{M}_{\rho}\right)\right| \leq \epsilon_{\pazocal{M}}\left(N, \frac{\delta}{2}\right).\] Inequality~\eqref{ineq:2a} follows from the fact that $\hat{M}_{\rho}$ has empirical profit that is within an additive factor of $\rho$ from empirically optimal over the set of samples, or in other words, ${\normalfont \profit}_\sample\left(\hat{M}_{\rho}\right) \geq {\normalfont \profit}_\sample\left(\pazocal{M}(\sample)\right) - \rho$.  Inequality~\eqref{ineq:3a} follows because $\pazocal{M}(\sample)$ is the empirical profit maximizer (i.e., it maximizes ${\normalfont \profit}_\pazocal{S}\left(M\right)$). Finally, inequality~\eqref{ineq:4a} is a result of Hoeffding's inequality, which guarantees that with probability at least $1-\delta/2$, ${\normalfont \profit}_{\sample}\left(M^*\right) \geq {\normalfont \profit}_{\pazocal{D}}\left(M^*\right) -  U \sqrt{\frac{2\ln(4/\delta)}{2N}}$.

Rearranging, we get that \[ {\normalfont \profit}_{\pazocal{D}}\left(\hat{M}_{\rho}\right) \geq {\normalfont \profit}_{\pazocal{D}}\left(M^*\right) - \epsilon_{\pazocal{M}}\left(N, \frac{\delta}{2}\right) - U \sqrt{\frac{2\ln(4/\delta)}{2N}} - \rho,\] as claimed.
\end{proof}

\begin{restatable}{corollary}{multApx}\label{cor:mult_apx}
	Let $\pazocal{M}$ be a mechanism class and let $M^* \in \pazocal{M}$ be a mechanism with maximum expected profit. Given a set of samples $\sample$, let $\hat{M}_{\alpha}$ be a mechanism in $\pazocal{M}$ with empirical profit that is at least an $\alpha$-fraction of the empirically optimal: $\sum_{\vec{v} \in \sample} \profit_{\hat{M}_{\alpha}}\left(\vec{v}\right) \geq \alpha \cdot \max_{M \in \mclass} \sum_{\vec{v} \in \sample} \profit_{M}\left(\vec{v}\right) $. With probability at least $1-\delta$ over the draw $\sample \sim \dist^N$, the difference between the expected profit of $\hat{M}_{\alpha}$ and an $\alpha$-fraction of the expected profit of $M^*$ is at most $ \epsilon_{\pazocal{M}}\left(N, \frac{\delta}{2}\right) + U\alpha \sqrt{\frac{\ln(4/\delta)}{2N}}$: \[ \E_{\vec{v}\sim \dist}\left[{\normalfont \profit}_{\hat{M}_{\alpha}}(\vec{v})\right] \geq \alpha \cdot \E_{\vec{v}\sim \dist}\left[{\normalfont \profit}_{M^*}(\vec{v})\right] - \epsilon_{\pazocal{M}}\left(N, \frac{\delta}{2}\right) - U \alpha \sqrt{\frac{2\ln(4/\delta)}{2N}}.\]
\end{restatable}

\begin{proof}
Let $\sample = \left\{\vec{v}^1, \dots, \vec{v}^N\right\}$ be a set of samples of buyer valuations.
With probability at least $1-\delta$,
\begin{align*}
 &{\normalfont \profit}_{\pazocal{D}}\left(\hat{M}_{\alpha}\right) + \epsilon_{\pazocal{M}}\left(N, \frac{\delta}{2}\right) \geq   {\normalfont \profit}_{\sample}\left(\hat{M}_{\alpha}\right) \geq \alpha \cdot \max_{M \in \mclass} \profit_{\sample}(M)\\
 \geq  \text{ }&\alpha \cdot {\normalfont \profit}_{\sample}\left({M^*}\right) \geq \alpha \cdot {\normalfont \profit}_{\pazocal{D}}\left({M^*}\right) -  U \alpha \sqrt{\frac{2\ln(4/\delta)}{2N}}.
\end{align*}

These inequalities follow for the same reasons as in the proof of Corollary~\ref{cor:add_apx}.
\end{proof}

\begin{lemma}[\citet{Shalev14:Understanding}, Lemma A.2]\label{lem:log_ineq}
Let $a \geq 1$ and $b > 0$. Then $x < a\log x + b$ implies that $x < 4a \log (2a) + 2b$.
\end{lemma}

\begin{lemma}\label{lem:kappa_bnded_2pt}
No matter which parameters the mechanism designer chooses in $\pspace$ or $\pspace'$, if all buyers simultaneously choose the tariff and the number of units $(q_1, \dots, q_n)$ that maximize their utilities, then $\sum_{j = 1}^n q_j \leq \kappa$.
\end{lemma}

\begin{proof}
We prove this lemma for non-anonymous prices, and the lemma for anonymous prices follow since they are a special case of non-anonymous prices. For a contradiction, suppose there exists a set of buyers' values $\vec{v}$ and a non-anonymous menu of two-part tariffs with parameters in $\pspace'$ such that if $t_j$ is the tariff that buyer $j$ chooses and $q_j$ is the number of units he chooses, $\sum_{j = 1}^n q_j > \kappa$. Since the mechanisms are profit non-negative, we know that $\sum_{j = 1}^n p_{1,j}^{(t_j)} \cdot \textbf{1}_{\{q_j \geq 1\}} + p_{2,j}^{(t_j)} \cdot q_j - c\left(Q\right) \geq 0$, where $Q = (q_1, \dots, q_n)$. We also know that each buyer's value for the units he bought is greater than the price: $\sum_{j = 1}^n v_j(q_j) \geq \sum_{j = 1}^n p_{1,j}^{(t_j)} \cdot \textbf{1}_{\{q_j \geq 1\}} + p_{2,j}^{(t_j)} \cdot q_j$. Therefore, $\sum_{j = 1}^n v_j(q_j) - c(Q) \geq 0$. However, this contradicts Assumption~\ref{assumption:unit_cap}, so the lemma holds.
\end{proof}

\twoPart*

\begin{proof}
A length-$\ell$ menu of two-part tariffs is defined by $2\ell$ parameters. The first $2$ parameters (denoted $\left(p_0^{(1)}, p_1^{(1)}\right)$) define the first tariff in the menu, the second $2$ parameters (denoted $\left(p_0^{(2)}, p_1^{(2)}\right)$) define the second tariff in the menu, and so on. Buyer $j$ will prefer to buy $q \geq 1$ units using $i^{th}$ menu entry (defined by the parameters $\left(p_0^{(i)}, p_1^{(i)}\right)$)  so long as
$v_j(q)  - \left(p_0^{(i)} + p_1^{(i)}q\right)> v_j(q')  - \left(p_0^{(i')} + p_1^{(i')}q'\right)$ for any $i' \not=i$ and $q' \not=q$. In total, these inequalities define $O\left(n\left(\kappa \ell\right)^2\right)$ hyperplanes in $\R^{2\ell}$. In any region defined by these hyperplanes, the menu entries and quantities demanded by all $n$ buyers are fixed. In any such region, profit is linear in the fixed fees and unit prices.

In the case of non-anonymous reserve prices, the same argument holds, except that every length-$\ell$ menu of two-part tariffs is defined by $2n\ell$ parameters: for each buyer, we must set the fixed fee and unit price for each of the $\ell$ menu entries.
\end{proof}

\begin{lemma}\label{lem:kappa_bnded_NL}
No matter which parameters the mechanism designer chooses in $\pspace$ or $\pspace'$, if all buyers simultaneously choose the bundles that maximize their utilities, then $\sum_{j = 1}^n q_j[i] \leq \kappa_i$ for all $i \in [m]$.
\end{lemma}

\begin{proof}
We prove this lemma for non-anonymous prices, and the lemma for anonymous prices follow since they are a special case of non-anonymous prices. For a contradiction, suppose there exists a set of buyers' values $\vec{v}$ and a non-anonymous non-linear pricing mechanism with parameters in $\pspace'$ such that if $\vec{q}_j$ is the bundle buyer $j$ chooses, $\sum_{j = 1}^n q_j[i] > \kappa_i$ for some $i \in [m]$. Since the mechanisms are profit non-negative, we know that $\sum_{j = 1}^n p_j(\vec{q}_j)- c\left(Q\right) \geq 0$, where $Q = \left(\vec{q}_1, \dots, \vec{q}_n\right)$. We also know that each buyer's value for the units he bought is greater than the price: $\sum_{j = 1}^n v_j(\vec{q}_j) \geq \sum_{j = 1}^n p_j(\vec{q}_j)$. Therefore, $\sum_{j = 1}^n v_j(\vec{q}_j) - c(Q) \geq 0$. However, this contradicts Assumption~\ref{assumption:unit_cap}, so the lemma holds.
\end{proof}

\begin{definition}[Profit non-negative non-linear pricing mechanisms]\label{def:PNN_NL}
	In the case of anonymous prices (respectively, non-anonymous), let $\pspace$ (respectively, $\pspace'$) be the set of mechanism parameters such that for each buyer $j \in [n]$ and each allocation $Q = \left(\vec{q}_1, \dots, \vec{q}_n\right)$, the seller's utility is non-negative: $\sum_{j = 1}^n p_j(\vec{q}_j) - c\left(Q\right) \geq 0.$ The set of profit non-negative non-linear pricing mechanisms is defined by parameters in $\pspace$ (respectively, $\pspace'$).
\end{definition}

\nonlinear*

\begin{proof}
We begin by analyzing the case where there are anonymous prices. By Lemma~\ref{lem:kappa_bnded_NL}, the mechanism designer might as well set the price of any bundle $\vec{q}$ such that $q[i] \geq \kappa_i$ for some $i \in [m]$ to $\infty$. Therefore, every non-linear pricing mechanism is defined by $d = \prod_{i = 1}^m \left(\kappa_i+1\right)$ parameters because that is the number of different bundles and there is a price per bundle. Buyer $j$ will prefer the bundle corresponding to the quantity vector $\vec{q}$ over the bundle corresponding to the quantity vector $\vec{q}'$ if $v_j(\vec{q}) - p(\vec{q}) \geq v_j(\vec{q}') - p(\vec{q}')$. Therefore, there are at most $\prod_{i = 1}^m \left(\kappa_i+1\right)^2$ hyperplanes in $\R^{d}$ determining each buyer's preferred bundle --- one hyperplane per pair of bundles. This means that there are a total of $n\prod_{i = 1}^m \left(\kappa_i+1\right)^2$ hyperplanes in $\R^{d}$ such that in any one region induced by these hyperplanes, the bundles demanded by all $n$ buyers are fixed and profit is linear in the prices of these $n$ bundles.

In the case of non-anonymous prices, the same argument holds, except that every non-linear pricing mechanism is defined by $n\prod_{i = 1}^m \left(\kappa_i+1\right)$ parameters --- one parameter per bundle-buyer pair.
\end{proof}

\begin{definition}[Additively decomposable non-linear pricing mechanisms] Additively decomposable non-linear pricing mechanisms are a subset of non-linear pricing mechanisms where the prices are additive over the items. Specifically, if the prices are anonymous, there exist $m$ functions $p^{(i)}:[\kappa_i] \to \R$ for all $i \in [m]$ such that for every quantity vector $\vec{q}$, $p(\vec{q}) = \sum_{i: q[i] \geq 1} p^{(i)}(q[i])$. If the prices are non-anonymous, there exist $nm$ functions $p^{(i)}_j:[\kappa_i] \to \R$ for all $i \in [m]$ and $j \in [n]$ such that for every quantity vector $\vec{q}$, $p_j(\vec{q}) = \sum_{i: q[i] \geq 1} p^{(i)}_j(q[i])$.\end{definition}
\begin{restatable}{lemma}{addNonlinear}\label{lem:nonlinear_additive}
Let $\pazocal{M}$ and $\pazocal{M}'$ be the classes of additively decomposable non-linear pricing mechanisms with anonymous and non-anonymous prices, respectively. Then
$\mclass$ is \[\left(\sum_{i = 1}^m (\kappa_i+1), n\prod_{i = 1}^m \left(\kappa_i+1\right)^2\right)\text{-delineable}\] and $\mclass'$ is $\left(n\sum_{i = 1}^m \left(\kappa_i+1\right), n\prod_{i = 1}^m \left(\kappa_i+1\right)^2\right)$-delineable.
\end{restatable}

\begin{proof}
In the case of anonymous prices, any additively decomposable non-linear pricing mechanism is defined by $d = \sum_{i = 1}^m (\kappa_i+1)$ parameters. As in the proof of Lemma~\ref{lem:nonlinear}, there are a total of $n\prod_{i = 1}^m (\kappa_i+1)^2$ hyperplanes in $\R^{d}$ such that in any one region induced by these hyperplanes, the bundles demanded by all $n$ buyers are fixed and profit is linear in the prices of these $n$ bundles.

In the case of non-anonymous prices, the same argument holds, except that every non-linear pricing mechanism is defined by $n\sum_{i = 1}^m (\kappa_i+1)$ parameters --- one parameter per item, quantity, and buyer tuple.
\end{proof}

\itemAdd*

\begin{proof}
In the case of anonymous prices, every item-pricing mechanisms is defined by $m$ prices $\vec{p} \in \R^m$, so the parameter space is $\R^m$. Let $j_i$ be the buyer with the highest value for item $i$. We know that item $i$ will be bought so long as $v_{j_i}(\vec{e}_i) \geq p(\vec{e}_i)$. Once the items bought are fixed, profit is linear. Therefore, there are $m$ hyperplanes splitting $\R^m$ into regions where profit is linear.

In the case of non-anonymous prices, the parameter space is $\R^{nm}$ since there is a price per buyer and per item. The items each buyer $j$ is willing to buy is defined by $m$ hyperplanes: $v_j(\vec{e}_i) \geq p_j(\vec{e}_i)$. So long as these preferences are fixed, profit is a linear function of the prices. Therefore, there are $nm$ hyperplanes splitting $\R^{nm}$ into regions where profit is linear.
\end{proof}

\secondPrice*

\begin{proof}
For a given valuation vector $\vec{v}$, let $j_i$ be the highest bidder for item $i$ and let $j_i'$ be the second highest bidder.
Under anonymous prices, item $i$ will be bought so long as $v_{j_i}(\vec{e}_i) \geq p(\vec{e}_i)$. If buyer $j_i$ buys item $i$, his payment depends on whether or not $v_{j_i'}(\vec{e}_i) \geq p(\vec{e}_i)$. Therefore, there are $t = 2m$ hyperplanes splitting $\R^m$ into regions where profit is linear.
In the case of non-anonymous prices, the only difference is that the parameter space is $\R^{nm}$. 
\end{proof}


\begin{definition}[Mixed-bundling auctions with reserve prices (MBARPs)]\label{def:MBARP}
	MBARPs are defined by a parameter $\gamma \geq 0$ and $m$ reserve prices $p\left(\vec{e}_1\right), \dots, p\left(\vec{e}_m\right)$. Let $\lambda$ be a function such that $\lambda\left(Q\right) = \gamma$ if some buyer receives the grand bundle under allocation $Q$ and 0 otherwise. For an allocation $Q$, let $\vec{q}_Q$ be the items not allocated. Given a valuation vector $\vec{v}$, the MBARP allocation is \[Q^* = \left(\vec{q}_1^*, \dots, \vec{q}_n^*\right) = \text{argmax}\left\{\sum_{j = 1}^n v_j\left(\vec{q}_j\right) + \sum_{i : q_Q[i] = 1} p\left(\vec{e}_i\right) + \lambda\left(Q\right) - c\left(Q\right)\right\}.\] 
	Using the notation \[Q^{-j} = \left(\vec{q}_1^{-j}, \dots, \vec{q}_n^{-j}\right) = \text{argmax}\left\{ \sum_{\ell \not = j} v_\ell\left(\vec{q}_\ell\right) + \sum_{i : q_Q[i] = 1} p\left(\vec{e}_i\right) + \lambda\left(Q\right) - c\left(Q\right)\right\},\] buyer $j$ pays \[\sum_{\ell \not= j} v_\ell\left(\vec{q}_\ell^{-j}\right) + \sum_{i : q_{Q^{-j}}[i] = 1} p\left(\vec{e}_i\right) + \lambda\left(Q^{-j}\right) - c\left(Q^{-j}\right) - \sum_{\ell \not= j} v_\ell\left(\vec{q}^*_\ell\right) - \sum_{i : q_{Q^*}[i] = 1} p\left(\vec{e}_i\right) - \lambda\left(Q^*\right) + c\left(Q^*\right).\]
\end{definition}
 
 \MBARP*
 
 \begin{proof}
An MBARP is defined by $m+1$ parameters since there is one reserve per item and one allocation boost. Let $K = (n+1)^m$ be the total number of allocations. 
Fix some valuation vector $\vec{v}$. We claim that the allocation of any MBARP is determined by at most $(n+1)K^2$ hyperplanes in $\R^{m+1}$. To see why this is, let $Q^k = \left(\vec{q}_1^k, \dots, \vec{q}_n^k\right)$ and $Q^{\ell} = \left(\vec{q}_1^{\ell}, \dots, \vec{q}_n^{\ell}\right)$ be any two allocations and let $\vec{q}_{Q^k}$ and $\vec{q}_{Q^{\ell}}$ be the bundles of items not allocated. Consider the ${K \choose 2}$ hyperplanes defined as \[\sum_{i = 1}^n v_i\left(\vec{q}_i^{\ell}\right) + \sum_{j : q_{Q^{\ell}}[i] = 1} p\left(\vec{e}_i\right) + \lambda\left(Q^{\ell}\right) - c\left(Q^{\ell}\right) = \sum_{i = 1}^n v_i\left(\vec{q}_i^{k}\right) + \sum_{j : q_{Q^{k}}[i] = 1} p\left(\vec{e}_i\right) + \lambda\left(Q^{k}\right) - c\left(Q^{k}\right).\] In the intersection of these ${K \choose 2}$ hyperplanes, the allocation of the MBARP is fixed.

By a similar argument, it is straightforward to see that $K^2$ hyperplanes determine the allocation of any MBARP in this restricted space without any one bidder's participation. This leads us to a total of $(n+1)K^2$ hyperplanes which partition the space of MBARP parameters in a way such that for any two parameter vectors in the same region, the auction allocations are the same, as are the allocations without any one bidder's participation.  Once these allocations are fixed, profit is a linear function in this parameter space.
\end{proof}

\begin{definition}[Affine maximizer auction]\label{def:AMA}
	An AMA is defined by a weight per buyer $w_j \in \R_{> 0}$ and a boost per allocation $\lambda\left(Q\right) \in \R_{\geq 0}$. The AMA allocation $Q^*$ is the one which maximizes the weighted social welfare, i.e., $Q^* =   \left(\vec{q}_1^*, \dots, \vec{q}_n^*\right) = \text{argmax}\left\{\sum_{j = 1}^n w_jv_j\left(\vec{q}_j\right) + \lambda\left(Q\right) - c\left(Q\right)\right\}.$ Using the notation \[Q^{-j} = \left(\vec{q}_1^{-j}, \dots, \vec{q}_n^{-j}\right) = \text{argmax}\left\{ \sum_{\ell \not= j} w_{\ell}v_{\ell}\left(\vec{q}_{\ell}\right) + \lambda\left(Q\right) - c\left(Q\right)\right\},\] each buyer $j$ pays 
	\[\frac{1}{w_j}\left[ \sum_{\ell \not= j} w_{\ell}v_{\ell}\left(\vec{q}_{\ell}^{-j}\right) + \lambda\left(Q^{-j}\right) - c\left(Q^{-j}\right)-\left(\sum_{\ell \not= j} w_{\ell} v_{\ell}\left(\vec{q}^*_{\ell}\right) +\lambda\left(Q^*\right) - c\left(Q^*\right)\right)\right].\]
	\end{definition}

\AMA*

\begin{proof}
Let $K = (n+1)^m$ be the total number of allocations and let $\vec{p}$ be a parameter vector where the first $n$ components correspond to the bidder weights $w_j$ for $j \in [n]$, the next $n$ components correspond to $1/w_j$ for $j \in [n]$, the next $2{n \choose 2}$ components correspond to $w_i/w_j$ for all $i \not= j$, the next $K$ components correspond to $\lambda(Q)$ for every allocation $Q$, and the final $nK$ components correspond to $\lambda(Q)/w_j$ for all allocations $Q$ and all bidders $j \in [n]$. In total, the dimension of this parameter space is at most $2n + 2n^2 + K + nK = O(nK)$. Let $\vec{v}$ be a valuation vector. We claim that this parameter space can be partitioned using $t = (n+1)K^2$ hyperplanes into regions where in any one region $\pspace'$, there exists a vector $\vec{k}$ such that $\profit_{\vec{v}}(\vec{p}) = \vec{k} \cdot \vec{p}$ for all $\vec{p} \in\pspace'$.

To this end, an allocation $Q = \left(\vec{q}_1, \dots, \vec{q}_n\right)$ will be the allocation of the AMA so long as $\sum_{i = 1}^n w_i v_i\left(\vec{q}_i\right) + \lambda(Q) - c(Q) \geq \sum_{i = 1}^n w_i v_i\left(\vec{q}_i'\right) + \lambda\left(Q'\right) - c(Q')$ for all allocations $Q' = \left(\vec{q}_1', \dots, \vec{q}_n'\right) \not= Q$. Since the number of different allocations is at most $K$, the allocation of the auction on $\vec{v}$ is defined by at most $K^2$ hyperplanes in $\R^{d}$. Similarly, the allocations $Q^{-1}, \dots, Q^{-n}$ are also determined by at most $K^2$ hyperplanes in $\R^{d}$.
Once these allocations are fixed, profit is a linear function of this parameter space.

The proof for VVCAs follows the same argument except that we redefine the parameter space to consist of vectors where the first $n$ components correspond to the bidder weights $w_j$ for $j \in [n]$, the next $n$ components correspond to $1/w_j$ for $j \in [n]$, the next $2{n \choose 2}$ components correspond to $w_i/w_j$ for all $i \not= j$, the next $K' = n2^m$ components correspond to the bidder-specific bundle boosts $c_{j,\vec{q}}$ for every quantity vector $\vec{q}$ and bidder $j \in [n]$, and the final $nK'$ components correspond to $c_{k,\vec{q}}/w_j$ for every quantity vector $\vec{q}$ and every pair of bidders $j,k \in [n]$. The dimension of this parameter space is at most $2n + 2n^2 + K' + nK' \leq 2K' + nK' + K' + nK' = O(nK')$.

Finally, the proof for $\lambda$-auctions follows the same argument as the proof for AMAs except there are zero bidder weights. Therefore, the parameter space consists of vectors with $K$ components corresponding to $\lambda(Q)$ for every allocation $Q$.
\end{proof} 

\begin{lemma}\label{lem:expectation}
For all $\vec{v} \in \domain$ and all $M \in \mclass$, $\profit_M(\vec{v}) = \E_{\vec{z}}\left[\profit_{M} '\left(\vec{v}, \vec{z}\right)\right]$.
\end{lemma}

\begin{proof} By definition of $\profit_m'$,
\begin{align*}
&\E_{\vec{z}}\left[\profit_{M} '\left(\vec{v}, \vec{z}\right)\right]\\
=\text{ }&\E_{\vec{z}}\left[p_{\vec{v}} - c\left(\sum_{j: z[j] < \phi_{\vec{v}}[j]} \vec{e}_j\right)\right]\\
=\text{ }& p_{\vec{v}} - \sum_{\vec{r} \in \{0,1\}^m}c\left(\vec{r}\right)\prod_{j: r[j] = 1}\Pr\left[z[j] < \phi_{\vec{v}}[j] \right]\prod_{j: r[j] = 0}\Pr\left[z[j] \geq \phi_{\vec{v}}[j] \right]\\
=\text{ }& p_{\vec{v}} - \sum_{\vec{r} \in \{0,1\}^m}c\left(\vec{r}\right)\prod_{j: r[j] = 1}\phi_{\vec{v}}[j]\prod_{j: r[j] = 0}\left(1-\phi_{\vec{v}}[j]\right).
\end{align*}

From the other direction, \begin{align*}
\profit_{M} \left(\vec{v}\right) &= p_{\vec{v}} - \E_{\vec{q} \sim \vec{\phi}_{\vec{v}}}\left[c(\vec{q})\right]\\
&= p_{\vec{v}} - \sum_{\vec{r} \in \{0,1\}^m}c\left(\vec{r}\right)\prod_{j: r[j] = 1}\Pr\left[q[j] = 1 \right]\prod_{j: r[j] = 0}\Pr\left[q[j] = 0 \right]\\
&= p_{\vec{v}} - \sum_{\vec{r} \in \{0,1\}^m}c\left(\vec{r}\right)\prod_{j: r[j] = 1}\phi_{\vec{v}}[j]\prod_{j: r[j] = 0}\left(1-\phi_{\vec{v}}[j]\right).
\end{align*} Therefore, $\profit_M(\vec{v}) = \E_{\vec{z}}\left[\profit_{M} '\left(\vec{v}, \vec{z}\right)\right]$.
\end{proof}

\lotteryEquiv*

\begin{proof}
We know that with probability at least $1-\delta$ over the draw of a sample \[\left\{\left(\vec{v}^{(1)}, \vec{z}^{(1)}\right), \dots, \left(\vec{v}^{(N)}, \vec{z}^{(N)}\right)\right\} \sim \left(\dist \times U([0,1])^{m}\right)^N,\] for all mechanisms $M \in \mclass$, \begin{align*}&\left|\frac{1}{N} \sum_{j = 1}^N \profit_{M} '\left(\vec{v}^{(j)}, \vec{z}^{(j)}\right) - \E_{\vec{v}, \vec{z} \sim \dist \times U([0,1])^{m}}\left[\profit_M'(\vec{v}, \vec{z})\right] \right|\\
= \text{ }&O\left(U \sqrt{\frac{Pdim(\mclass')}{N}} + U \sqrt{\frac{\log(1/\delta)}{N}}\right).\end{align*} We also know from Lemma~\ref{lem:expectation} that \[\E_{\vec{v}, \vec{z} \sim \dist \times U([0,1])^{m}}\left[\profit_M'(\vec{v}, \vec{z})\right] = \E_{\vec{v} \sim \dist}\left[\profit_M(\vec{v})\right].\] Therefore, the theorem statement holds.
\end{proof}

\lottery*

\begin{proof}
	A length-$\ell$ lottery menu is defined by $\ell(m+1)$ parameters. The first $m+1$ parameters (denoted $\left(\phi^{(1)}[1], \dots, \phi^{(1)}[m], p^{(1)}\right)$) define the first lottery in the menu, the second $m+1$ parameters (denoted $\left(\phi^{(2)}[1], \dots, \phi^{(2)}[m], p^{(2)}\right)$) define the second lottery in the menu, and so on. The buyer will prefer the $j^{th}$ menu entry (defined by the parameters $\left(\phi^{(j)}[1], \dots, \phi^{(j)}[m], p^{(j)}\right)$)  so long as
	$\vec{v} \cdot \vec{\phi}^{(j)} - p^{(j)}> \vec{v} \cdot \vec{\phi}^{(k)}- p^{(k)}$ for any $k \not=j$. In total, these inequalities define ${\ell+1 \choose 2}$ hyperplanes in $\R^{\ell(m+1)}$. In any region defined by these hyperplanes, the menu entry that the buyer prefers is fixed. Next, for each menu entry $\left(\vec{\phi}^{(k)}, p^{(k)}\right)$, there are $m$ hyperplanes determining the vector $\sum_{j: w[j] < \phi^{(k)}[j]} \vec{e}_j$, and thus the cost  $c \left(\sum_{j: w[j] < \phi^{(k)}[j]} \vec{e}_j\right)$. These vectors have the form $w[j] = \phi^{(k)}[j].$ Thus, there are a total of $\ell m$ hyperplanes determining the costs. Let $\hyp$ be the union of all $(\ell+1)^2 + m \ell$ hyperplanes. Within any connected component of $\R^{\ell(m+1)}\setminus \hyp$, the menu entry that the buyer buys is fixed and for each menu entry, $c \left(\sum_{j: w[j] < \phi^{(k)}[j]} \vec{e}_j\right)$ is fixed. Therefore, profit is a linear function of the prices $p^{(1)}, \dots, p^{(\ell)}$.
\end{proof}

\subsection{Additional lottery results}\label{sec:additional_lotteries}

\paragraph{Lotteries for a unit-demand buyer.} Recall that if the buyer is unit-demand, then for any bundle $\vec{q} \in \{0,1\}^m$, $v_1\left(\vec{q}\right) = \max_{i : q[i] \geq 1} v_1\left(\vec{e}_i\right)$. We assume that under a lottery $\left(\phi^{(j)}, p^{(j)}\right)$ with a unit-demand buyer, the buyer will only receive one item, and the probability that item is item $i$ is $\phi^{(j)}[i]$. Thus, we assume that $\sum_{i = 1}^m \phi^{(j)}[i] \leq 1$. Since $v_1(\vec{e}_i)\cdot \phi^{(j)}[i]$ is their value for item $i$ times the probability they get that item, their expected utility is $\sum_{i = 1}^m v_1(\vec{e}_i) \cdot \phi^{(j)}[i] - p^{(j)}$, as in the case with an additive buyer. Therefore, the following theorem follows by the exact same proof as Lemma~\ref{lem:lottery}.

\begin{theorem}
Let $\mclass'$ be the class of functions defined in Section~\ref{sec:delineable}.4. Then $\mclass'$ is \[\left(\ell\left(m+1\right), \left(\ell+1\right)^2 + m\ell\right)\text{-delineable}.\]
\end{theorem}

\paragraph{Lotteries for multiple unit-demand or additive buyers.} In order to generalize to multi-buyer settings, we assume that there are $n$ units of each item for sale and that each buyer will receive at most one unit of each item. The buyers arrive simultaneously and each will buy the lottery that maximizes her expected utility. Thus, the following is a corollary of Lemma~\ref{lem:lottery}.

\begin{theorem}
Let $\mclass'$ be the class of functions defined in Section~\ref{sec:delineable}.4. Then $\mclass'$ is \[\left(\ell\left(m+1\right), n\left(\left(\ell+1\right)^2 + m\ell\right)\right)\text{-delineable.}\]
\end{theorem}