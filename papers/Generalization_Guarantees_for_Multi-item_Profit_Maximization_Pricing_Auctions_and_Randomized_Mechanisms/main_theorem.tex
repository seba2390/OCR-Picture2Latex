We provide generalization bounds for a variety of mechanism classes. These guarantees bound the difference between the expected profit and average empirical profit of any mechanism in the class.

\begin{definition}\label{def:gen_guar}
A \emph{generalization guarantee} for a mechanism class $\pazocal{M}$ is a function $\epsilon_{\cM} : \Z_{\geq 1} \times (0,1) \to \R_{\geq 0}$ defined such that for any $\delta \in (0,1)$, any $N \in \Z_{\geq 1}$, and any distribution $\dist$ over buyers' values, with probability at least $1-\delta$ over the draw of a set $\sample \sim \dist^N$, for any $M \in \pazocal{M}$, the difference between the average profit of $M$ over $\sample$ and the expected profit of $M$ over $\dist$ is at most $\epsilon_{\pazocal{M}}(N, \delta)$:
\[\Pr_{\sample \sim \dist^N} \left[\exists M \in \cM \text{ such that } \left|\frac{1}{N}\sum_{\vec{v} \in \sample} \profit_M\left(\vec{v}\right) - \E_{\vec{v} \sim \dist}\left[\profit_M\left(\vec{v}\right)\right]\right| > \epsilon_{\cM}(N, \delta) \right] < \delta.\]\end{definition} 


Generalization guarantees allow the mechanism designer to relate the expected profit of a mechanism in $\pazocal{M}$ which achieves maximum average profit over the set of samples to the expected profit of an optimal mechanism in $\pazocal{M}$. We summarize this connection in the following remark.

\begin{remark}\label{rem:opt}
For a set of samples $\sample \sim \dist^N$, let $\hat{M} = \argmax_{M \in \cM}\left\{\sum_{\vec{v} \in \sample} \profit_M(\vec{v})\right\}$ maximize average profit over $\sample$ and let $M^* = \argmax_{M \in \cM}\left\{\E_{\vec{v} \sim \dist}\left[\profit_M(\vec{v})\right]\right\}$ maximize expected profit. Then $\Pr_{\sample \sim \dist^N} \left[\E_{\vec{v} \sim \dist}\left[\profit_{M^*}\left(\vec{v}\right) - \profit_{\hat{M}}\left(\vec{v}\right)\right] > 2\epsilon_{\cM}\left(N, \delta\right)\right] < \delta.$
\end{remark}



Similar bounds also hold for mechanisms with approximately optimal average profit over the samples (see Corollaries~\ref{cor:add_apx} and \ref{cor:mult_apx} in Appendix~\ref{APP:MAIN}).

\subsection{General structure for sample-based mechanism design}

Our general theorem uses structure shared by a variety of mechanism classes to characterize the function $\epsilon_{\pazocal{M}}\left(N, \delta\right)$.
Our results apply broadly to \emph{parameterized} sets $\mclass$ of mechanisms where every mechanism in $\mclass$ is defined by a vector $\vec{p}\in \R^d$, such as a vector of prices. Our guarantees apply to mechanism classes where for every valuation  $\vec{v} \in \domain$, the profit as a function of the parameters $\vec{p}$, denoted $\profit_{\vec{v}}\left(\vec{p}\right)$, is piecewise linear. We illustrate this property via several simple examples.

\begin{example}\label{ex:2PT}
In a \emph{two-part tariff,} there are multiple units (i.e., copies) of an item for sale. The seller sets an \emph{upfront fee} $p_1$ and a \emph{price per unit} $p_2$. Here, we consider the simple case where there is a single buyer. If the buyer buys $t \geq 1$ units, he pays $p_1 + p_2\cdot t$. Two-part tariffs have been studied extensively \citep{Oi71:Disneyland, Feldstein72:Equity, Wilson93:Nonlinear} and are prevalent throughout daily life. For example, health clubs often require an upfront membership fee plus a fee per month. Amusement parks often require an entrance fee with an additional payment per ride.
In many cities, purchasing a public transportation card requires an upfront fee and an additional cost per ride.  \citet{Balcan20:Efficient} showed how to learn two-part tariffs that maximize average revenue over a training set.

Suppose there are $\kappa$ units of the item for sale. The buyer will buy  $t \in [\kappa]$ units so long as $v_1\left(t\right) - \left(p_1 + p_2 \cdot t\right) >  v_1\left(t'\right) - \left(p_1 + p_2 \cdot t'\right)$ for all $t' \not= t$ and $v_1\left(t\right) - \left(p_1 + p_2 \cdot t\right) > 0$. Therefore, there are at most ${\kappa + 1 \choose 2}$ hyperplanes splitting $\R^2$ into regions such that within any one region, the number of units bought is fixed, in which case profit is linear in $p_1$ and $p_2$.
\begin{figure}
	\includegraphics{2PT_detailed}\centering
	\caption{Partition of the two-part tariff parameter space into piecewise-linear regions. There are four units for sale and one buyer with values $v_1(1) = 6$, $v_1(2) = 9$,  $v_1(3) = 11$, and $v_1(4) = 12$. The buyer will buy one unit in the top orange region where $v_1(1) - p_1 - p_2 > v_1(i) - p_1 - i \cdot p_2$ for all $i \in \{2, 3, 4\}$ and $v_1(1) - p_1 - p_2 > 0$. The buyer will buy two units in the second-to-the-top blue region, three units in the second-to-the-bottom green region, and four units in the bottom red region.
	}
	\label{fig:2PT_detailed}
\end{figure}
See Figure~\ref{fig:2PT_detailed} for an illustration.
\end{example}

\begin{example}\label{ex:item_pricing}
Under an \emph{item-pricing mechanism}, there are multiple items, multiple buyers, and a single unit of each item for sale. 
Under anonymous prices, the seller sets a price $p_i$ per item $i$. There is an arbitrary ordering on the buyers such that the first buyer buys the bundle that maximizes his utility, then the next buyer buys the bundle of remaining items that maximizes his utility, and so on.
Buyer $j$ will prefer bundle $\vec{q}_1 \in \{0,1\}^m$ over $\vec{q}_2$ if $v_j(\vec{q}_1) - \sum_{i: q_1[i] = 1} p_i > v_j(\vec{q}_2) - \sum_{i: q_2[i] = 1} p_i$, so his preference ordering over bundles is determined by these ${2^m \choose 2}$ hyperplanes. Once the buyers' preference orderings are fixed, the bundles they buy are fixed. In any region of the price space where the purchased bundles are fixed, profit is a linear in the prices.
\begin{figure}
	\includegraphics{pricing_Partition}\centering
	\caption{Partition of the \emph{item-pricing} parameter space into piecewise-linear regions when there are two buyers, two items, and buyer 1 comes before buyer 2 in the ordering. Buyer 1's value for item 1 is $v_1(1,0) = 2$, her value for item 2 is $v_1(0, 1) = 1$, and her value for both items is $v_1(1,1) = 2.5$. Buyer 2's values are $v_2(1, 0) = 0, v_2(0, 1) = 1$, and $v_2(1, 1) = 1$. In the orange region, buyer 1 will buy item 1 because $v_1(1, 0) - p_1 > v_1(0,1) - p_2$, $v_1(1, 0) - p_1 > v_1(1,1) - (p_1 + p_2)$, and $v_1(1, 0) - p_1 > 0$. Buyer 2 will not buy item 2 because $v_2(0, 1) - p_2 < 0$. In the red region, neither buyer will buy any item. In the blue region, buyer 1 will buy item 1 and buyer 2 will buy item 2. In the green region, buyer 1 will buy item 2 and buyer 2 will not buy anything. Finally, in the white region, buyer 1 will buy both items.
	}
	\label{fig:pricing}
\end{figure}
See Figure~\ref{fig:pricing} for an illustration.
\end{example}

We analyze the ``complexity'' of the partition splitting $\R^d$ into regions where $\profit_{\vec{v}}\left(\vec{p}\right)$ is linear.
\begin{definition}[$\left(d,t\right)$-delineable]\label{def:delineable} A mechanism class $\mclass$ is \emph{$\left(d,t\right)$-delineable} if:
\begin{enumerate}
\item The class $\mclass$ consists of mechanisms parameterized by vectors $\vec{p}$ from a set $\pspace \subseteq \R^d$; and
\item For any valuation vector $\vec{v} \in \domain$, there is a set $\hyp$ of $t$ hyperplanes such that for any connected component $\pspace'$ of $\pspace \setminus \hyp$, $\profit_{\vec{v}}\left(\vec{p}\right)$ is linear over $\pspace'.$ (As is standard, $\pspace \setminus \hyp$ indicates set removal.)
\end{enumerate}
\end{definition}
\begin{figure}
	\centering
	\begin{subfigure}{0.21\textwidth}
		\includegraphics{fig1} \centering
		\caption{A partition by hyperplanes.}
		\label{fig:overlay1}
	\end{subfigure}\qquad
	\begin{subfigure}{0.21\textwidth}
		\includegraphics{fig2}\centering
		\caption{Another partition by hyperplanes.}
		\label{fig:overlay2}
	\end{subfigure}\qquad
	\begin{subfigure}{0.21\textwidth}
		\includegraphics{fig3}\centering
		\caption{Overlay of partitions (a) and (b).}
		\label{fig:overlay3}
	\end{subfigure}\qquad
	\begin{subfigure}{0.21\textwidth}
		\includegraphics{fig4}\centering
		\caption{A further subdivision of each region.}
		\label{fig:subpartition}
	\end{subfigure}
	\caption{Illustrations of the proof of Lemma~\ref{lem:main_pdim}.}
	\label{fig:overlay}
\end{figure}
For example, in Figures~\ref{fig:overlay1} and \ref{fig:overlay2}, there are four connected components.
We relate delineability to the mechanism class's intrinsic complexity using \emph{pseudo-dimension}.
\subsection{Pseudo-dimension}

Pseudo-dimension is a well-studied tool used to measure the complexity of a function class.
Pseudo-dimension captures the following intuition: functions in a ``complex'' class should be able to fit complex patterns.
We first introduce the notion of \emph{shattering} for general function classes.

\begin{definition}
Let $\fclass$ be a set of functions $f:\pazocal{A} \to \R$ with an abstract domain $\pazocal{A}$. We say that $z^{\left(1\right)}, \dots, z^{\left(N\right)} \in \R$ \emph{witness the shattering} of $\sample= \left\{x^{\left(1\right)}, \dots, x^{\left(N\right)}\right\} \subseteq \cA$ by $\pazocal{F}$ if for all $T \subseteq \sample$, there is a function $f_T \in \pazocal{F}$ such that for all $x^{\left(i\right)} \in T$, $f_T\left(x^{\left(i\right)}\right) \leq z^{\left(i\right)}$ and for all $x^{\left(i\right)} \not\in T$, $f_T\left(x^{\left(i\right)}\right) > z^{\left(i\right)}$.
\end{definition}


Figure~\ref{fig:shattering} in Appendix~\ref{APP:MAIN} provides a visualization. The larger the set a function class can shatter, the more complex that function class is, an intuition formalized by pseudo-dimension.

\begin{definition}[\citet{Pollard84:Convergence}]
Let $\fclass$ be a set of functions $f:\pazocal{A} \to \R$ and let $\sample \subseteq \pazocal{A}$ be the largest set that can be shattered by $\fclass$. The \emph{pseudo-dimension} of $\fclass$, denoted $\pdim(\fclass)$, is  $|\sample|$.
\end{definition}

In the language of mechanism design, let $\sample= \left\{\vec{v}^{\left(1\right)}, \dots, \vec{v}^{\left(N\right)}\right\}$ be a subset of $\domain$. We say that $z^{\left(1\right)}, \dots, z^{\left(N\right)} \in \R$ \emph{witness} the shattering of $\sample$ by $\pazocal{M}$ if for all $T \subseteq \sample$, there is a mechanism $M_T \in \pazocal{M}$ such that for all $\vec{v}^{\left(i\right)} \in T$, $\profit_{M_T}\left(\vec{v}^{\left(i\right)}\right) \leq z^{\left(i\right)}$ and for all $\vec{v}^{\left(i\right)} \not\in T$, $\profit_{M_T}\left(\vec{v}^{\left(i\right)}\right) > z^{\left(i\right)}$. The pseudo-dimension of $\pazocal{M}$, denoted $\pdim\left(\mclass\right)$, is the size of the largest set that is shatterable by $\pazocal{M}$.

\citet{Pollard84:Convergence} and \citet{Dudley87:Universal} provide generalization guarantees in terms of pseudo-dimension, which we describe below in the language of mechanism design.
\begin{theorem}\label{thm:pdim}
For any mechanism class $\mclass$, let $U$ be the maximum profit of any mechanism in $\mclass$ over the support of $\dist$. There is a generalization guarantee $\epsilon_{\cM} : \Z_{\geq 1} \times (0,1) \to \R_{\geq 0}$ defined such that \[\epsilon_{\pazocal{M}}\left(N, \delta\right) = 120U\sqrt{\frac{\pdim(\mclass)}{N}} + 4U \sqrt{\frac{2\ln(4/\delta)}{N}}.\]
\end{theorem}


\subsection{General theorem for sample-based mechanism design}

In the following theorem, which is our main theorem, we relate pseudo-dimension to delineability.

\begin{theorem}\label{thm:main_pdim}
Let $\mclass$ be a $\left(d, t\right)$-delineable mechanism class. Given a distribution $\dist$ over buyers' values, let $U$ be the maximum profit of any mechanism in $\mclass$ over the support of $\dist$. Then \[\epsilon_{\pazocal{M}}\left(N, \delta\right) = 120U\sqrt{\frac{9d \log (4dt)}{N}} + 4U \sqrt{\frac{2\ln(4/\delta)}{N}}\] is a generalization guarantee for $\mclass$.
\end{theorem}

\begin{proof}
This theorem follows directly from the following lemma.
\end{proof}

\begin{lemma}\label{lem:main_pdim}
If $\mclass$ is a mechanism class that is $\left(d, t\right)$-delineable, then $\pdim(\mclass) \leq 9d \log (4dt)$.
\end{lemma}

\begin{proof}
For any set $\sample = \left\{\vec{v}^{(1)}, \dots, \vec{v}^{(N)}\right\}$ of valuation vectors and real values $z^{(1)}, \dots, z^{(N)} \in \R$, we show that there is a partitioning of the parameter space into at most  $dN^d\cdot d(Nt)^d$ regions such that for all $\vec{p}$ in any one region and all $\vec{v}^{(i)}$, either $\profit_{\vec{v}^{(i)}}\left(\vec{p}\right) \leq z^{(i)}$ or $\profit_{\vec{v}^{(i)}}\left(\vec{p}\right) > z^{(i)}$. We will then use this fact to bound $\pdim(\mclass)$.
To this end, let $\hyp^{(i)}$ be the set of $t$ hyperplanes such
that for any connected component $\pspace'$ of $\pspace \setminus \hyp^{(i)}$, $\profit_{\vec{v}^{(i)}}\left(\vec{p}\right)$ is linear over $\pspace'.$ Let $\pspace_1, \dots, \pspace_{\tau}$ be the connected components
of $\pspace \setminus \left(\bigcup_{i = 1}^N \hyp^{(i)}\right)$. For each set $\pspace_j$ and each $i \in [N]$, $\pspace_j$ is contained in a single connected component of $\pspace \setminus \hyp^{(i)}$, which means that $\profit_{\vec{v}^{(i)}}\left(\vec{p}\right)$ is linear over $\pspace_j.$
(See Figures~\ref{fig:overlay1}-\ref{fig:overlay3} for illustrations.)
Since $\left|\hyp^{(i)}\right|\leq t$ for all $i \in [N]$, $\tau < d(Nt)^d$~\citep[][Theorem 1]{Buck43:Partition}.

For any region $\pspace_j$ and $\vec{v}^{(i)} \in \sample$,
let  $\vec{a}_j^{(i)} \in \R^d$ and $b_j^{(i)} \in \R$ be defined
such that $\profit_{\vec{v}^{(i)}}\left(\vec{p}\right) = \vec{a}_j^{(i)} \cdot \vec{p} + b_j^{(i)}$ for all $\vec{p} \in \pspace_j$. On one side of the
hyperplane $\vec{a}_j^{(i)} \cdot \vec{p} + b_j^{(i)} = z^{(i)}$, $\profit_{\vec{v}^{(i)}}\left(\vec{p}\right) \leq z^{(i)}$ and on the other
side, $\profit_{\vec{v}^{(i)}}\left(\vec{p}\right) > z^{(i)}$.
Let $\hyp_{\pspace_j}$ be all $N$ hyperplanes for all $N$ samples, i.e., $\hyp_{\pspace_j} = \left\{\vec{a}_j^{(i)} \cdot \vec{p} + b_j^{(i)} = z^{(i)} : i \in [N]\right\}.$ In any
connected component $\pspace'$ of $\pspace_j \setminus \hyp_{\pspace_j}$ (illustrated in Figure~\ref{fig:subpartition}), for all $i \in [N]$, $\profit_{\vec{v}^{(i)}}\left(\vec{p}\right)$ is either greater than $z^{(i)}$ or
less than $z^{(i)}$  for all $\vec{p} \in \pspace'$. The number of
connected components of $\pspace_j \setminus \hyp_{\pspace_j}$ is at most $dN^d$.
Thus, the total number of regions where
for all $i \in [N]$, $\profit_{\vec{v}^{(i)}}\left(\vec{p}\right)$ is either greater
than $z^{(i)}$ or less than $z^{(i)}$ is at most $dN^d\cdot d(Nt)^d$.

We now use this fact to bound $\pdim(\mclass)$. Suppose $\pdim(\mclass) = \bar{N}$, so there is a set \[\{\vec{v}^{(1)}, \dots, \vec{v}^{(\bar{N})}\}\] that is shattered by $\mclass$ with witnesses $z^{(1)}, \dots, z^{(\bar{N})} \in \R$. For any $T \subseteq [\bar{N}]$, there is a parameter vector $\vec{p}_T \in \pspace$ such that $\profit_{\vec{p}_T}\left(\vec{v}^{(i)}\right) \geq z^{(i)}$ if and only if $i\in T$. Let $\pspace^* = \left\{\vec{p}_T : T \subseteq [\bar{N}]\right\}$. There are $k \leq d\bar{N}^d\cdot d(\bar{N}t)^d$ regions $\cP_1, \dots, \cP_k$ where for each region $\cP_j$ and each $i \in [\bar{N}]$, either $\profit_{\vec{v}^{(i)}}\left(\vec{p}\right)\geq z^{(i)}$ for all $\vec{p} \in \cP_j$ or $\profit_{\vec{v}^{(i)}}\left(\vec{p}\right) < z^{(i)}$.
At most one
vector in $\pspace^*$ can come from any one region. This means that $|\pspace^*| = 2^{\bar{N}} < d\bar{N}^d\cdot d(\bar{N}t)^d$. The result follows from Lemma~\ref{lem:log_ineq}.
\end{proof} 
