We studied profit maximization when the mechanism designer has a set of samples from the distribution over buyers' values.  We identified structural similarities of mechanism classes including non-linear pricing mechanisms, generalized VCG mechanisms such as affine maximizer auctions, and lotteries: profit is a piecewise-linear function of the mechanism class's parameters. These similarities led us to a general theorem that gives generalization bounds for a broad range of mechanism classes. It offers the first generalization guarantees for many important classes and also matches and improves over many existing bounds.
Finally, we provided guarantees for optimizing a fundamental tradeoff in sample-based mechanism design: more complex mechanisms have higher average profit over the samples than simpler mechanisms, but require more samples to avoid overfitting.

An important direction for future research is the development of learning algorithms for multi-item profit maximization. Learning algorithms have been proposed for several of the mechanism classes we consider, including two-part tariffs~\citep{Balcan20:Efficient}, affine maximizer auctions~\citep{Sandholm15:Automated}, and item-pricing mechanisms~\citep[][who also provide algorithms for other multi-item mechanism classes]{Cai17:Learning}. A line of research also provides learning algorithms for single-item profit maximization~\citep{Devanur16:Sample,Hartline16:Non,Gonczarowski17:Efficient,Guo19:Settling}.

Another direction is to use tools such as Rademacher complexity to provide generalization bounds for non-worst-case distributions beyond item-independent distributions (the focus of Section~\ref{SEC:DATA}). For example, suppose any buyer's values for any items are correlated, but his values are independent of any other buyer's values. Can the bounds in this paper be improved?
