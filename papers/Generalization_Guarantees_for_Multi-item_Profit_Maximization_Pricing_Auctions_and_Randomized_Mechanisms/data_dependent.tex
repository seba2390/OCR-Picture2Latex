
In this section, we provide stronger results when the buyers' values are additive and drawn from \emph{item-independent} distributions, which means that for all $i_1, i_2 \in [n]$ and $j ,j' \in [m]$, buyer $i_1$'s values for items $j$ and $j'$ are independent, but her values may be correlated with buyer $i_2$'s values. We also require that the mechanism class's profit functions decompose additively. For example, under item-pricing mechanisms, the profit decomposes into the profit obtained from selling item 1, plus the profit obtained by selling item 2, and so on. Surprisingly, our bounds do not depend on the number of items and under anonymous prices, they do not depend on the number of buyers either.



To prove distribution-dependent generalization guarantees, we use Rademacher complexity~\citep{Bartlett02:Rademacher,Koltchinskii01:Rademacher}. In contrast, pseudo-dimension implies bounds that are worst-case over the distribution.
We prove that it is impossible to obtain guarantees that are independent of the number of items using pseudo-dimension alone (Theorem~\ref{thm:pdim_lower}).

\begin{definition}\label{def:DD_gen_guar}
A \emph{distribution-dependent generalization guarantee} for a mechanism class $\pazocal{M}$ and a distribution $\dist$ over buyers' values is a function $\epsilon_{\cM}^\dist : \Z_{\geq 1} \times (0,1) \to \R_{\geq 0}$ defined such that for any sample size $N \in \Z_{\geq 1}$ and any $\delta \in (0,1)$, with probability at least $1-\delta$ over the draw of a set $\sample \sim \dist^N$, for any mechanism $M$ in $\pazocal{M}$, the difference between the average profit of $M$ over $\sample$ and the expected profit of $M$ over $\dist$ is at most $\epsilon_{\pazocal{M}}^\dist(N, \delta)$. In other words,
\[\Pr_{\sample \sim \dist^N} \left[\exists M \in \cM \text{ such that } \left|\frac{1}{N}\sum_{\vec{v} \in \sample} \profit_M\left(\vec{v}\right) - \E_{\vec{v} \sim \dist}\left[\profit_M\left(\vec{v}\right)\right]\right| > \epsilon_{\cM}^{\dist}(N, \delta) \right] < \delta.\]\end{definition}
The generalization guarantee $\epsilon_{\cM}$ is worst case in that in holds for any distribution $\dist$. In contrast, the distribution-dependent bound $\epsilon_{\cM}^{\dist}$ may be much tighter when the distribution is ``well-behaved''.

We now define Rademacher complexity, which measures the ability of a class of mechanism profit functions to fit random noise. Intuitively, more complex classes should fit random noise better than simple classes. 
The \emph{empirical Rademacher complexity} of $\pazocal{M}$ with respect to $\sample= \left\{\vec{v}^{\left(1\right)}, \dots, \vec{v}^{\left(N\right)}\right\}$ is  \[\erad\left(\pazocal{M}\right) = \E_{\vec{\sigma}}\left[\sup_{M \in \pazocal{M}} \frac{1}{N} \sum_{i = 1}^N \sigma_i \cdot \profit_M\left(\vec{v}^{(i)}\right) \right],\] where $\sigma_i \sim U\left(\left\{-1,1\right\}\right)$. Classic learning-theoretical results~\citep{Bartlett02:Rademacher,Koltchinskii01:Rademacher} imply the distribution-dependent generalization bound $\epsilon_{\cM}^{\dist}(N, \delta) = 2\E_{\sample \sim \dist^N}\left[\erad\left(\pazocal{M}\right)\right] + U \sqrt{\frac{2\ln \left(2/\delta\right)}{N}},$ where $U$ is the maximum profit of any mechanism in $\mclass$ over the support of $\dist$.
It is well-known that Rademacher complexity and pseudo-dimension are connected as follows.

\begin{restatable}{lemma}{pseudimerad}[\cite{Pollard84:Convergence, Dudley87:Universal}]\label{lem:pdim2erad}
	For any mechanism class $\mclass$ and any set of samples $\sample$ of size $N$, $\erad(\pazocal{M}) = O\left(U \sqrt {\frac{\pdim(\pazocal{M})}{N}}\right).$
\end{restatable}

We show that if the profit functions of a class $\mclass$ decompose additively into simpler functions, then we can bound $\erad\left(\mclass\right)$ using the Rademacher complexity of those simpler functions. We use this to prove tighter bounds for several mechanism classes under additive buyers with values drawn from item-independent distributions. This includes product distributions, a setting that has been studied extensively~\citep[e.g.,][]{Hart12:Approximate, Cai17:Learning, Yao14:n, Cai16:Duality, Babaioff17:Menu}.
 Formally, a mechanism class $\mclass$ \emph{decomposes additively} if for all $M \in \mclass$, there are $T$ functions $f_{1, M}, \dots, f_{T, M}$ such that $\profit_{M}\left(\cdot\right) = f_{1, M}\left(\cdot\right) + \cdots + f_{T, M}\left(\cdot\right)$.

\begin{corollary}\label{cor:data}
Suppose that $\mclass$ is a set of additively decomposable mechanisms. Moreover, suppose that for all $M \in \mclass$, the range of $f_{i, M}$ over the support of $\dist$ is $[0, U_i]$ and that the class $\left\{f_{i, M} : M \in \mclass\right\}$ is $\left(d_i,t_i\right)$-delineable.
For any set $\sample \sim \dist^N$, \[\erad\left(\mclass\right) \leq 180\sum_{i = 1}^T U_i \sqrt{\frac{d_i\log \left(4d_it_i\right)}{N}}.\]
\end{corollary}

\begin{proof}
This follows from Theorem~\ref{thm:main_pdim}, Lemma~\ref{lem:pdim2erad}, and the fact that for any sets $\pazocal{G}$ and $\pazocal{G}'$ of functions with a domain $\cA$ and any $\sample \subseteq \cA$, $\erad\left(\left\{g + g' : g \in \pazocal{G}, g' \in \pazocal{G}'\right\}\right) \leq \erad\left(\pazocal{G}\right) + \erad\left(\pazocal{G}'\right)$.
\end{proof}

We now instantiate Corollary~\ref{cor:data} for several mechanism classes. The proofs are in Appendix~\ref{APP:DATA}.

\begin{restatable}{lemma}{secondPriceProduct}\label{lem:second_price_product}
Let $\mclass$ and $\mclass'$ be the sets of second-price auctions with anonymous and non-anonymous reserves. Suppose the buyers are additive, $\dist$ is item-independent, and the cost function is additive. For any set $\sample \sim \dist^N$, $\erad\left(\mclass\right) \leq 180U\sqrt{1/N}$ and $\erad\left(\mclass'\right) \leq 180U\sqrt{n \log (4n)/N}$.
\end{restatable}

\begin{restatable}{lemma}{itemProduct}\label{lem:item_product}
Let $\mclass$ and $\mclass'$ be the sets of anonymous and non-anonymous item-pricing mechanisms. Suppose the buyers are additive, $\dist$ is item-independent, and the cost function is additive. For any set of samples $\sample \sim \dist^N$, $\erad\left(\mclass\right) \leq 180U\sqrt{1/N}$ and $\erad\left(\mclass'\right) \leq 180U\sqrt{n \log (4n)/N}$.
\end{restatable}

We prove similar guarantees for menus of item lotteries (Lemma~\ref{lem:lottery_product}).
Finally, we provide lower bounds showing that one could not prove the generalization guarantees implied by Lemmas~\ref{lem:second_price_product} and \ref{lem:item_product}---which do not depend on the number of items---using pseudo-dimension alone.
\begin{restatable}{theorem}{pdimLower}\label{thm:pdim_lower}
Let $\pazocal{M}$ and $\pazocal{M}'$ be the classes of anonymous and non-anonymous item-pricing mechanisms. Then $\pdim\left(\pazocal{M}\right) \geq m$ and $\pdim\left(\pazocal{M}'\right) \geq nm$. The same holds if $\pazocal{M}$ and $\pazocal{M}'$ are the classes of second-price auctions with anonymous and non-anonymous reserves.
\end{restatable}