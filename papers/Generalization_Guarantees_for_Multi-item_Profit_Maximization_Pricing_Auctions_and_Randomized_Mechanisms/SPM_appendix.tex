\itemSPM*

\begin{proof}
This theorem follows from the fact that $\mclass_k$ is $(km, nm)$-delineable. Every mechanism in $\mclass_k$ is defined by $km$ parameters, one price per item per price group, and for every buyer $j$, the items they are willing to buy are defined by the $m$ hyperplanes $v_j(\vec{e}_i) = p_j(\vec{e}_i)$ for every item $i$. Therefore, the theorem follows from Theorems~\ref{thm:pdim} and \ref{thm:main_pdim}, and by multiplying $\delta$ with $w(k)$.
\end{proof}

\bigskip\emph{Two-part tariffs.} Let $\pazocal{M}$ be the class of anonymous two-part tariff menus, by which we mean the union of all length-$\ell$ menus of two-part tariffs with anonymous prices. Similarly, let $\pazocal{M}'$ be the class of non-anonymous two-part tariff menus. For a given menu $M$ of two-part tariffs, let $\ell_M$ be the length of its menu.

\begin{theorem}\label{thm:2part_nonuniform}
Let $w: \N \to [0,1]$ be a weight function such that $\sum w(i) \leq 1$. Then for any $\delta \in (0,1)$, with probability at least $1-\delta$ over the draw of a set of samples of size $N$ from $\pazocal{D}$, for any mechanism $M \in \pazocal{M}$, the difference between the average profit of $M$ over the set of samples and the expected profit of $M$ over $\pazocal{D}$ is 
\[O\left(U\sqrt{\frac{\ell_M \log(n\kappa\ell_M)}{N}} + U\sqrt{\frac{1}{N} \log \frac{1}{\delta \cdot w(\ell_M)}}\right).\] Also, with probability at least $1-\delta$ over the draw of a set of samples of size $N$ from $\pazocal{D}$, for any mechanism $M \in \pazocal{M}'$, the difference between the average profit of $M$ over the set of samples and the expected profit of $M$ over $\pazocal{D}$ is at most \[O\left(U\sqrt{\frac{n\ell_M \log(n\kappa\ell_M)}{N}} + U\sqrt{\frac{1}{N} \log \frac{1}{\delta \cdot w(\ell_M)}}\right).\]
\end{theorem}

\bigskip\emph{$\Qset$-boosted AMAs.} For an AMA $M$, let $\Qset_M$ be the set of all allocations $Q$  such that $\lambda\left(Q\right) > 0$.
\begin{theorem}\label{thm:Oboosted_SPM}
	Let $\pazocal{M}$ be the class of AMAs and let $w$ be a weight function that maps sets of allocations $\Qset$ to $[0,1]$ such that $\sum w\left(\Qset\right) \leq 1$. With probability $1-\delta$ over $\sample \sim \pazocal{D}^N$, for any $M \in \pazocal{M}$, \[\left|\profit_{\sample}\left(M\right) - \profit_{\dist}\left(M\right)\right| \leq 360U \sqrt{\frac{nm\left(n+|\Qset_M|\right)\log(4n)}{N}} + 4U\sqrt{\frac{2}{N}\ln\frac{4}{\delta \cdot w\left(\Qset_M\right)}}.\]
\end{theorem}

\bigskip\emph{$\Qset$-boosted $\lambda$-auctions.} For the next theorem, given a $\lambda$-auction $M$, let $\Qset_M$ be the set of all allocations $Q$ such that $\lambda(Q) > 0$.

\begin{theorem}\label{thm:Oboosted_lambda_nonuniform}
Let $\pazocal{M}$ be the class of $\lambda$-auctions and let $w$ be a weight function which maps sets of allocations $\Qset$ to $[0,1]$ such that $\sum w(\Qset) \leq 1$. Then for any $\delta \in (0,1)$, with probability at least $1-\delta$ over the draw of a set of samples of size $N$ from $\pazocal{D}$, for any mechanism $M \in \pazocal{M}$, the difference between the average profit of $M$ over the set of samples and the expected profit of $M$ over $\pazocal{D}$ is at most \[O\left(U\sqrt{\frac{|\Qset_M| \log(n|\Qset_M|)}{N}} + U\sqrt{\frac{1}{N} \log \frac{1}{\delta \cdot w(\Qset_M)}}\right).\]
\end{theorem}

\bigskip\emph{Menu lotteries.} Let $\pazocal{M}$ be the class of lottery menus, by which we mean the union of all length-$\ell$ lottery menus. For a given lottery menu $M$, let $\ell_M$ be the length of its menu.

\begin{theorem}\label{thm:lottery_nonuniform}
Let $w: \N \to [0,1]$ be a weight function such that $\sum w(i) \leq 1$. Then for any $\delta \in (0,1)$, with probability at least $1-\delta$ over the draw of a set of samples of size $N$ from $\pazocal{D}$, for any mechanism $M \in \pazocal{M}$, the difference between the average profit of $M$ over the set of samples and the expected profit of $M$ over $\pazocal{D}$ is 
\[O\left(U\sqrt{\frac{\ell_M \log(n\ell_M)}{N}} + U\sqrt{\frac{1}{N} \log \frac{1}{\delta \cdot w(\ell_M)}}\right).\]
\end{theorem}