We now instantiate Theorem~\ref{thm:version2}.

\begin{restatable}{theorem}{itemUnit}\label{thm:item_pricing_unit}
Let $\pazocal{M}$ and $\pazocal{M}'$ be the classes of item-pricing mechanisms with anonymous prices and non-anonymous prices. If the buyers are unit-demand, then $\mclass$ is  $\left(m, nm^2, 1\right)$-divisible and $\mclass'$ is $\left(nm,nm^2,1\right)$-divisible. Also, $\mclass$ and $\mclass'$ are $\left(m+1\right)$- and $\left(nm+1\right)$-dimensionally linearly separable over $\left\{0,1\right\}^m$ and $[n]^m$. Therefore, \[\pdim\left(\mclass\right) = O\left(\min \left\{m^2, m\log\left(nm\right)\right\}\right) \text{ and } \pdim\left(\mclass'\right) = O\left(nm \log (nm)\right).\]
\end{restatable}

\begin{proof}
We begin with anonymous reserves. Let $f_{\vec{p}}^{(1)}: \domain \to \{0,1\}^m$ be defined so that the $i^{th}$ component is 1 if and only if item $i$ is sold. For each buyer $j$, there are ${m \choose 2}$ hyperplanes defining their preference ordering on the items: $v_j(\vec{e}_i) - p(\vec{e}_i) = v_j(\vec{e}_k) - p(\vec{e}_k)$ for all $i \not = k$. This gives a total of at most $t_1 = nm^2$ hyperplanes splitting $\R^m$ into regions where $f_{\vec{v}}^{(1)}(\vec{p})$ is constant. Next, we can write $f_{\vec{p}}^{(2)}(\vec{v}, \vec{\alpha}) = \vec{\alpha} \cdot \vec{p}$, which is always linear, so we may set $t_2 = 1$.

Under non-anonymous reserve prices, let $f_{\vec{p}}^{(1)}: \domain \to \{0,1\}^{nm}$ be defined so that for every buyer $j$ and every item $i$, there is a component of $f_{\vec{p}}^{(1)}(\vec{v})$ that is 1 if and only if buyer $j$ receives item $i$. As with anonymous prices, there are $t_1 = nm^2$ hyperplanes splitting $\R^{nm}$ into regions where $f_{\vec{v}}^{(1)}(\vec{p})$ is constant. Next, we can write $f_{\vec{p}}^{(2)}(\vec{v}, \vec{\alpha}) = \vec{\alpha} \cdot \vec{p}$, which is always linear, so we may set $t_2 = 1$.

\citet{Morgenstern16:Learning} proved that $\mclass$ and $\mclass'$ are $(m+1)$- and $(nm+1)$-dimensionally linearly separable over $\{0,1\}^m$ and $[n]^m$, respectively.
\end{proof}


When prices are anonymous, if $ n < 2^m$, Theorem~\ref{thm:item_pricing_unit} improves on the pseudo-dimension bound of $O\left(m^2\right)$ \citet{Morgenstern16:Learning} gave for this class, and otherwise it matches their bound. When the prices are non-anonymous our bound improves on their bound of $O\left(nm^2 \log n\right)$.

\begin{restatable}{theorem}{itemGeneral}\label{thm:item_pricing_general}
Let $\pazocal{M}$ and $\pazocal{M}'$ be the classes of item-pricing mechanisms with anonymous prices and non-anonymous prices, respectively. If the buyers have general values, then $\mclass$ is $\left(m, n2^{2m}, 1\right)$-divisible and $\mclass'$ is $\left(nm,n2^{2m},1\right)$-divisible. Also, $\mclass$ is $\left(m+1\right)$-dimensionally linearly separable over $\left\{0,1\right\}^m$ and $\mclass'$ is $\left(nm+1\right)$-dimensionally linearly separable over $[n]^m$. Thus, $Pdim\left(\mclass\right) = O\left(m^2\right)$ and $Pdim\left(\mclass'\right) = O\left(nm\left(m + \log n\right)\right)$.
\end{restatable}
%

\begin{proof}
We begin with anonymous reserves. Let $f_{\vec{p}}^{(1)}: \domain \to \{0,1\}^m$ be defined so that the $i^{th}$ component is 1 if and only if item $i$ is sold. For each buyer $j$, there are ${2^m \choose 2}$ hyperplanes defining their preference ordering on the bundles: $v_j(\vec{q}) - \sum_{i: q[i] = 1} p(\vec{e}_i) = v_j(\vec{q}') - \sum_{i: q'[i] = 1} p(\vec{e}_i)$ for all $\vec{q}, \vec{q}' \in \{0,1\}^m$. This gives a total of at most $t_1 = n2^{2m}$ hyperplanes splitting $\R^m$ into regions where $f_{\vec{v}}^{(1)}(\vec{p})$ is constant. Next, we can write $f_{\vec{p}}^{(2)}(\vec{v}, \vec{\alpha}) = \vec{\alpha} \cdot \vec{p}$, which is always linear, so we may set $t_2 = 1$.

Under non-anonymous reserve prices, let $f_{\vec{p}}^{(1)}: \domain \to \{0,1\}^{nm}$ be defined so that for every buyer $j$ and every item $i$, there is a component of $f_{\vec{p}}^{(1)}(\vec{v})$ that is 1 if and only if buyer $j$ receives item $i$. As with anonymous prices, there are $t_1 = n2^{2m}$ hyperplanes splitting $\R^{nm}$ into regions where $f_{\vec{v}}^{(1)}(\vec{p})$ is constant. Next, we can write $f_{\vec{p}}^{(2)}(\vec{v}, \vec{\alpha}) = \vec{\alpha} \cdot \vec{p}$, which is always linear, so we may set $t_2 = 1$.

\citet{Morgenstern16:Learning} proved that $\mclass$ and $\mclass'$ are $(m+1)$- and $(nm+1)$-dimensionally linearly separable over $\{0,1\}^m$ and $[n]^m$, respectively.
\end{proof}


When there are anonymous prices, the number of hyperplanes in the partition is large, so considering the hyperplane partition does not help us. As a result, Theorem~\ref{thm:item_pricing_general} implies the same bound \citet{Morgenstern16:Learning} gave. In the case of non-anonymous prices, analyzing the hyperplane partition gives a better bound than their bound of $O\left(nm^2 \log n\right)$.

In Theorem~\ref{thm:second_price_sep}, we use Theorem~\ref{thm:version2} to prove pseudo-dimension bounds of $O\left(m \log m\right)$ and $O\left(nm \log nm\right)$ for the classes of second price auctions for additive buyers with anonymous and non-anonymous reserves, respectively. In Theorem~\ref{thm:item_pricing_add_sep}, we prove the same for item-pricing mechanisms. We thus answer the open question by \citet{Morgenstern16:Learning}. These bounds match those implied by Lemmas~\ref{lem:item_pricing_add} and \ref{lem:second_price}.


\begin{theorem}\label{thm:second_price_sep}
Let $\pazocal{M}$ and $\pazocal{M}'$ be the classes of anonymous and non-anonymous second price item auctions. Then $\mclass$ is $\left(m, m, m\right)$-divisible and $\mclass'$ is $(nm, m, m)$-divisible. Also, $\mclass$ and $\mclass'$ are $(m+1)$- and $(nm+1)$-dimensionally linearly separable over $\{0,1\}^m$ and $[n]^m$. Therefore, $Pdim(\mclass) = O(m \log m)$ and $Pdim(\mclass') = O(nm \log (nm))$.
\end{theorem}

\begin{proof}
We begin with anonymous reserves. For a given valuation vector $\vec{v}$, let $j_i$ be the highest buyer for item $i$ and let $j_i'$ be the second highest buyer. Let $f_{\vec{p}}^{(1)}: \domain \to \{0,1\}^m$ be defined so that the $i^{th}$ component is 1 if and only if item $i$ is sold. There are $t_1 = m$ hyperplanes splitting $\R^m$ into regions where $f_{\vec{v}}^{(1)}(\vec{p})$ is constant: the $i^{th}$ component of $f_{\vec{v}}^{(1)}(\vec{p})$ is 1 if and only if $v_{j_i}(\vec{e}_i) \geq p(\vec{e}_i)$. Next, we can write $f_{\vec{p}}^{(2)}(\vec{v}, \vec{\alpha}) = \sum_{i: \alpha[i] = 1 } \max\left\{v_{j_i'}(\vec{e}_i), p(\vec{e}_i)\right\} - c(\vec{\alpha})$, which is linear so long as either $v_{j_i'}(\vec{e}_i) <p(\vec{e}_i)$ or $v_{j_i'}(\vec{e}_i) \geq  p(\vec{e}_i)$ for all $i \in [m]$. Therefore, there are $t_2 = m$ hyperplanes $\hyp_2$ such that for any connected component $\pspace'$ of $\pspace \setminus \hyp_2$, $f_{\vec{v}, \vec{\alpha}}^{(2)}(\vec{p})$ is linear over all $\vec{p} \in \pspace'.$


Under non-anonymous reserve prices, let $f_{\vec{p}}^{(1)}: \domain \to \{0,1\}^{nm}$ be defined so that for every buyer $j$ and every item $i$, there is a component of $f_{\vec{p}}^{(1)}(\vec{v})$ that is 1 if and only if buyer $j$ receives item $i$. There are $t_1 = m$ hyperplanes splitting $\R^{nm}$ into regions where $f_{\vec{v}}^{(1)}(\vec{p})$ is constant: for every item $i$, the component corresponding to buyer $j_i$ is 1 if and only if $v_{j_i}(\vec{e}_i) \geq p_j(\vec{e}_i)$. Next, we can write $f_{\vec{p}}^{(2)}(\vec{v}, \vec{\alpha}) = \sum_{i: \alpha[i] = 1 } \max\left\{v_{j_i'}(\vec{e}_i), p_{j_i}(\vec{e}_i)\right\} - c(\vec{\alpha})$, which is linear so long as either $v_{j_i'}(\vec{e}_i) <p_{j_i}(\vec{e}_i)$ or $v_{j_i'}(\vec{e}_i) \geq  p_{j_i}(\vec{e}_i)$ for all $i \in [m]$. Therefore, there are $t_2 = m$ hyperplanes $\hyp_2$ such that for any connected component $\pspace'$ of $\pspace \setminus \hyp_2$, $f_{\vec{v}, \vec{\alpha}}^{(2)}(\vec{p})$ is linear over all $\vec{p} \in \pspace'.$

\citet{Morgenstern16:Learning} proved that $\mclass$ and $\mclass'$ are $(m+1)$- and $(nm+1)$-dimensionally linearly separable over $\{0,1\}^m$ and $[n]^m$, respectively.
\end{proof}

\begin{theorem}\label{thm:item_pricing_add_sep}
Let $\pazocal{M}$ and $\pazocal{M}'$ be the classes of item-pricing mechanisms with anonymous prices and non-anonymous prices, respectively. If the buyers are additive, then $\mclass$ is $(m, m, 1)$-divisible and $\mclass'$ is $(nm,nm,1)$-divisible. Also, $\mclass$ and $\mclass'$ are $(m+1)$- and $(nm+1)$-dimensionally linearly separable over $\{0,1\}^m$ and $[n]^m$. Therefore, $Pdim(\mclass) = O(m \log m)$ and $Pdim(\mclass') = O(nm \log (nm))$.
\end{theorem}

\begin{proof}
We begin with anonymous reserves. For a given valuation vector $\vec{v}$, let $j_i$ be the buyer with the highest valuation for item $i$. Let $f_{\vec{p}}^{(1)}: \domain \to \{0,1\}^m$ be defined so that the $i^{th}$ component is 1 if and only if item $i$ is sold. There are $t_1 = m$ hyperplanes splitting $\R^m$ into regions where $f_{\vec{v}}^{(1)}(\vec{p})$ is constant: the $i^{th}$ component of $f_{\vec{v}}^{(1)}(\vec{p})$ is 1 if and only if $v_{j_i}(\vec{e}_i) \geq p(\vec{e}_i)$. Next, we can write $f_{\vec{p}}^{(2)}(\vec{v}, \vec{\alpha}) = \vec{\alpha} \cdot \vec{p}$, which is always linear, so we may set $t_2 = 1$.

Under non-anonymous reserve prices, let $f_{\vec{p}}^{(1)}: \domain \to \{0,1\}^{nm}$ be defined so that for every buyer $j$ and every item $i$, there is a component of $f_{\vec{p}}^{(1)}(\vec{v})$ that is 1 if and only if buyer $j$ receives item $i$. There are $t_1 = nm$ hyperplanes splitting $\R^{nm}$ into regions where $f_{\vec{v}}^{(1)}(\vec{p})$ is constant: $v_{j}(\vec{e}_i) = p_{j}(\vec{e}_i)$ for all $i$ and all $j$. Next, we can write $f_{\vec{p}}^{(2)}(\vec{v}, \vec{\alpha}) = \vec{\alpha} \cdot \vec{p}$, which is always linear, so we may set $t_2 = 1$.

\citet{Morgenstern16:Learning} proved that $\mclass$ and $\mclass'$ are $(m+1)$- and $(nm+1)$-dimensionally linearly separable over $\{0,1\}^m$ and $[n]^m$, respectively.
\end{proof}