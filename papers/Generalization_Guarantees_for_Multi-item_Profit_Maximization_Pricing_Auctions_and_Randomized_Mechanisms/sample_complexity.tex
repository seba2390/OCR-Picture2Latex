Generalization guarantees can be inverted to provide \emph{sample complexity guarantees.} A sample complexity guarantee bounds the number of samples required to ensure a desired bound on generalization. In other words, suppose that for some $\delta \in (0,1)$ and some $\epsilon >0$, the mechanism designer requires that with probability at least $1-\delta$, for all mechanisms in the class $\pazocal{M}$, average profit is $\epsilon$-close to expected profit. By choosing $N$ so that $\epsilon_{\pazocal{M}}(N, \delta) < \epsilon$, we can be assured that this will be the case. This leads to the following well-known remark.

\begin{remark}\label{remark:sample_comp}
Let $\pazocal{M}$ be a mechanism class with a generalization guarantee $\epsilon_{\pazocal{M}}(N, \delta) = U\sqrt{\frac{C_1}{N}} + 4U\sqrt{\frac{2}{N}\log\frac{C_2}{\delta}}$ for some $C_1, C_2 \in \R$ and let $\dist$ be the distribution over buyers' values. Then for $\delta \in (0,1)$ and $\epsilon > 0$, a sample size of
$N = \frac{2U^2}{\epsilon^2}\left(C_1 + 32\log\frac{C_2}{\delta}\right)$ is sufficient to ensure that with probability at least $1-\delta$ over the draw of a set $\sample \sim \dist^N$, for any mechanism $M \in \pazocal{M}$, the difference between the average profit of $M$ over $\sample$ and the expected profit of $M$ over $\dist$ is at most $\epsilon$.
\end{remark}

For every mechanism class $\pazocal{M}$ we study, $\epsilon_{\pazocal{M}}(N, \delta) = U\sqrt{\frac{C_1}{N}} + 4U\sqrt{\frac{2}{N}\log\frac{C_2}{\delta}}$ for some $C_1, C_2 \in \R$, so Remark~\ref{remark:sample_comp} applies. Notice that the number of samples required to ensure a strong generalization guarantee for the classes of AMAs, VVCAs, and $\lambda$-auctions is expononential in $m$. We prove that this exponential factor is, in fact, necessary.

\begin{theorem}
\label{thm:lower}
For a class of auctions $\pazocal{M}$, let $N_{\pazocal{M}}(\epsilon, \delta)$ be the number of samples required to ensure that for any distribution $\pazocal{D}$, with probability at least $1-\delta$ over the draw of a set of samples of size $N_{\pazocal{M}}(\epsilon, \delta)$ from $\pazocal{D}$, for all auctions $M \in \pazocal{M}$, average profit is $\epsilon$-close to expected profit. 
\begin{enumerate}
\item If $\pazocal{M}$ is the set of AMAs or $\lambda$-auctions, then $N_{\pazocal{M}}(\epsilon, \delta) \geq \frac{n^m - n}{2}.$
\item If $\pazocal{M}$ is the set of VVCAs, then $N_{\pazocal{M}}(\epsilon, \delta) \geq 2^m - 2.$
\end{enumerate}
\end{theorem}

We prove Theorem~\ref{thm:lower} in Appendix~\ref{app:lower}.
Thus, delineability implies not only generalization guarantees but also sample complexity guarantees which are nearly tight for AMAs and VVCAs.