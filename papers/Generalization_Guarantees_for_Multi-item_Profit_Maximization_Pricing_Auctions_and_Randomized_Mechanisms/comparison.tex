We compare our results to prior research that provides generalization bounds for some of the mechanisms we study. 
\citet{Morgenstern16:Learning} studied ``simple'' multi-item pricing mechanisms and second-price auctions.
See Tables~\ref{tab:results_pricing} and \ref{tab:results_auctions} and Appendix~\ref{sec:divisible} for a comparison.

\citet{Syrgkanis17:Sample} provided bounds specifically for the mechanism that maximizes average revenue over the samples, whereas our bounds apply to every mechanism in a given class. This is important when exactly optimizing average revenue is intractable. To illustrate their bounds, let $\hat{M}$ be the anonymous item-pricing mechanism maximizing average revenue over  $N$ samples. \citet{Syrgkanis17:Sample} proved that with probability $1-\delta$, $|\profit_{\dist}(\hat{M}) - \max_{M \in \mclass}\profit_{\dist}(M)| = O((U/\delta)\sqrt{m\log(nN)/N})$. When $\dist$ is item-independent, our bound $O(U\sqrt{\log(1/\delta)/N})$ is an improvement. Otherwise, our bound $O(U \sqrt{m \log (m)/N} + U\sqrt{\log(1/\delta)/N})$ is incomparable. \citet{Syrgkanis17:Sample} proved a similar bound for non-anonymous prices (see Table~\ref{tab:results_pricing}) which is also incomparable.

\citet{Cai17:Learning} provided learning algorithms for buyers with values drawn from product distributions.
For additive and unit-demand buyers with values bounded in $[0,H]$, we match their guarantees, which are based on those of~\citet{Morgenstern16:Learning}. They also study buyers with XOS, constrained additive, and subadditive values in which case our results do not provide an improvement.
For example, for XOS and constrained additive buyers, \citet{Cai17:Learning} provided algorithms which return item-pricing mechanisms with entry fees. Our results would imply pessimistic bounds for this class due to the exponential number of parameters. To circumvent this, their proofs use specific structural properties exhibited by bidders with product distributions, whereas the primary focus of this paper is to provide a general theory applicable to many different mechanisms and buyer types.

\citet{Medina17:Revenue} studied a different model than ours where items are defined by feature vectors and the seller has access to a bid predictor mapping feature vectors to bids.

Among other results, \citet[][Section 6.1]{Devanur17:Sample} proved that for the class $\mclass$ of second price item auctions with non-anonymous reserves, $N = O((U/\epsilon)^2(n \log (U/\epsilon) + \log (1/\delta)))$ samples are sufficient to ensure that with probability $1-\delta$, for all $M \in \pazocal{M}$, $|\profit_{\sample}(M) - \profit_{\dist}(M)| \leq \epsilon$. Our Lemma~\ref{lem:second_price} implies $O((U/\epsilon)^2(n \log n + \log (1/\delta)))$ samples are sufficient, which is incomparable.

\citet{Gonczarowski18:Sample} studied a setting where there are $n$ buyers with additive, independent values in the interval $[0,H]$ for $m$ items, as well as a generalization to Lipschitz valuations. They proved that poly$(n, m, H, 1/\epsilon)$ samples are sufficient to learn an approximately incentive compatible mechanism with $\epsilon$-approximately optimal revenue. From a computation perspective, it is not known how to efficiently find an $\epsilon$-approximately optimal mechanism in this setting where the number of types is exponential in the number of items.
In contrast, our guarantees apply uniformly to any mechanism from within a variety of parameterized classes, so the seller can use our guarantees to bound the expected profit of the mechanism he obtains via \emph{any} optimization procedure.
However, there may not be a mechanism in these classes with nearly optimal revenue.