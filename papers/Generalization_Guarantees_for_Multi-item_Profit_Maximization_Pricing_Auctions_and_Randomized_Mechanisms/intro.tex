The design of profit-maximizing mechanisms is a fundamental problem with diverse applications including Internet retailing, advertising markets, and strategic sourcing.
This problem has traditionally been studied under the assumption that there is a joint distribution from which the buyers' values are drawn and that the mechanism designer knows this distribution in advance. This assumption has led to groundbreaking theoretical results in the single-item setting~\citep{Myerson81:Optimal}, but transitioning from theory to practice is challenging because the true distribution over buyers' values is typically unknown. Moreover, in the dramatically more challenging multi-item setting, the support of the distribution alone is
often doubly exponential (even if there were just a single buyer with a finite type space\footnote{When each buyer's values are independent from every other buyer's values, the number of support points is $nk^{2^m}$, where $n$ is the number of buyers, $k$ is the number of discrete value levels a buyer can assign to a bundle, and $m$ is the number of items.  This is because each of the $2^m$ bundles can take any of $k$ values.  With correlated valuations, the prior has $k^{2^{nm}}$ support points.}), so obtaining and storing the distribution is typically impossible.

We relax this strong assumption and instead assume that the mechanism designer only has a set of independent samples from the distribution~\citep{Likhodedov04:Boosting,Likhodedov05:Approximating,Sandholm15:Automated}. 
This type of \emph{sample-based mechanism design} reflects current industry practices since many companies---such as online ad exchanges~\citep{He14:Practical, Medina17:Revenue}, sponsored search platforms~\citep{Edelman07:Internet, Tang17:Reinforcement}, and travel companies~\citep{Yee15:Aerosolve}---use historical purchase data to adjust the sales mechanism.

In most multi-item settings, the form of the revenue-maximizing mechanism is still a mystery. Therefore, rather than use the samples to uncover \emph{the} optimal mechanism, much of the literature on sample-based mechanism design suggests that we first fix a reasonably expressive mechanism class and then use the samples to optimize over the class. If, however, the mechanism class is complex and the number of samples is not sufficiently large, a mechanism with high average profit over the set of samples may have low expected profit on the actual unknown distribution: \emph{overfitting} has occurred. This motivates an important question in sample-based mechanism design:

\smallskip

\noindent\emph{Given a set of samples and a mechanism class $\mclass$, what is the difference between the average profit over the samples and the expected profit on the unknown distribution for any mechanism in $\mclass$?}

\smallskip

\noindent If this difference is small, the mechanism in $\mclass$ that maximizes average profit over the set of samples nearly maximizes expected profit over the distribution as well.


We present a general theory for deriving \emph{generalization guarantees} in multi-item settings.
A generalization guarantee for a mechanism class $\mclass$ bounds the difference between the average profit over the samples and expected profit for any mechanism in $\mclass$.
These bounds can be applied no matter how the mechanism designer optimizes over the class, using an automated or manual approach. Optimization algorithms for many of the mechanisms we study have been developed in prior research~\citep{Sandholm15:Automated,Cai17:Learning,Balcan20:Efficient}. 

This paper is part of a line of research that studies how learning theory can be used to design and analyze mechanisms. Most of these papers have studied only single-parameter settings~\citep{Balcan05:Mechanism,Balcan08:Reducing,Elkind07:Designing,Cole14:Sample, Huang15:Making, Mohri14:Learning, Morgenstern15:Pseudo, Roughgarden15:Ironing,Devanur16:Sample, Hartline16:Non,Gonczarowski17:Efficient,  Bubeck17:Online, Alon17:Submultiplicative, Guo19:Settling}. In contrast, we focus on multi-item mechanism design, as have recent papers by~\citet{Morgenstern16:Learning,Syrgkanis17:Sample, Medina17:Revenue, Cai17:Learning}, and \citet{Gonczarowski18:Sample}.

\subsection{Our contributions}

Our contributions come in two interrelated parts.

\paragraph{A general theory that unifies diverse mechanism classes.}
We uncover a structural property shared by a wide variety of mechanisms which allows us to prove generalization guarantees: for any fixed set of bids, profit is a piecewise linear function of the mechanism's parameters. Our main theorem provides generalization bounds for any class exhibiting this structure. We relate the complexity of the partition splitting the parameter space into linear portions to the intrinsic complexity of the mechanism class, which we quantify using \emph{pseudo-dimension.} In turn, pseudo-dimension bounds imply generalization bounds. We prove that many seemingly disparate mechanisms share this structure, and thus our main theorem yields learnability guarantees. By contrast, previous research on multi-item mechanism design focused on deriving guarantees for a few mechanism classes that are ``simple'' by design~\citep{Morgenstern16:Learning,Syrgkanis17:Sample}.
\begin{table}
	\scriptsize
			\begin{tabularx}{\textwidth}{L{0.85} L{1.15} L{1.1} L{.9}}
				\toprule
				\textbf{Category} & \textbf{Mechanism class} & \textbf{Valuations}& \textbf{Result}\\\midrule
				Pricing mechanisms & Item-pricing mechanisms & General, unit-demand, additive & Lemmas~\ref{lem:item_pricing_add}, \ref{lem:item_product}, \ref{thm:item_pricing_unit}, \ref{thm:item_pricing_general}\\\cmidrule{2-4}
				& Two-part tariffs & General & Lemma~\ref{lem:2part}\\\cmidrule{2-4}
				& Non-linear pricing mechanisms & General & Lemmas~\ref{lem:nonlinear}, \ref{lem:nonlinear_additive}\\\midrule
				Auctions & Second-price auctions with reserves & Additive & Lemmas~\ref{lem:second_price}, \ref{lem:second_price_product}, \ref{thm:second_price_sep}\\\cmidrule{2-4}
				& Affine maximizer auctions & General& Lemma~\ref{lem:AMA}\\\cmidrule{2-4}
				& Virtual valuation combinatorial auctions & General& Lemma~\ref{lem:AMA}\\\cmidrule{2-4}
				& Mixed-bundling auctions with reserves & General& Lemma~\ref{lem:MBARP}\\\midrule
				Randomized mechanisms & Lotteries & Additive, unit-demand & Lemmas~\ref{lem:lottery}, \ref{lem:lottery_product}\\\bottomrule
			\end{tabularx}
			\caption{Some of the main mechanism classes we analyze.}\label{tab:classes}
\end{table}
Table~\ref{tab:classes} summarizes some of the main mechanism classes we analyze and Tables~\ref{tab:results_lotteries}, \ref{tab:results_pricing}, and \ref{tab:results_auctions} summarize our bounds.

Our main theorem applies to lotteries, a general representation of randomized mechanisms which generate higher expected revenue than deterministic mechanisms in many settings~\citep{Conitzer03:Applications, Dobzinski09:Power}.
We also provide guarantees for \emph{item-pricing mechanisms} where each item has a price and buyers buy their utility-maximizing bundles. Additionally, we study \emph{multi-part tariffs}, where there is an upfront fee and a price per unit. These tariffs and other non-linear pricing mechanisms have been studied in economics for decades~\citep{Oi71:Disneyland, Feldstein72:Equity, Wilson93:Nonlinear}. 
Our main theorem also applies to many auction classes, such as \emph{second price auctions} and well-studied generalized VCG auctions including \emph{affine maximizer auctions (AMAs)} and \emph{mixed-bundling auctions}~\citep{Sandholm15:Automated, Roberts79:Characterization, Lavi03:Towards, Dobzinski08:Characterizations, Jehiel07:Mixed}. Under AMAs, revenue is not piecewise-linear in the original parameter space, but we show it is piecewise-linear in a higher-dimensional space.

A key challenge we face is the sensitivity of these mechanisms to small changes in their parameters. For example, changing the price of a good can cause a steep drop in profit if the buyer no longer wants to buy it. Meanwhile, for many well-understood function classes in machine learning, there is a close connection between the distance in parameter space between two parameter vectors and the distance in function space between the two corresponding functions. Since profit functions do not exhibit this predictable behavior, we must carefully analyze the structure of the mechanisms we study in order to derive our generalization guarantees.

\paragraph{Data-dependent generalization guarantees.} We strengthen our main theorem when the distribution over buyers' values is ``well-behaved,'' proving generalization guarantees that are independent of the number of items for item-pricing mechanisms, second price auctions with reserves, and lottery mechanisms. Under anonymous prices, our bounds do not depend on the number of buyers either. These guarantees hold when the buyers are additive with values drawn from \emph{item-independent distributions} (buyer $i_1$'s value for item $j$ is independent from her value for item $j'$, but her value for item $j$ may be arbitrarily correlated with buyer $i_2$'s value for item $j$). Buyers with item-independent value distributions have been studied extensively in prior research~ \citep{Cai17:Learning, Yao14:n, Cai16:Duality, Goldner16:Prior, Babaioff17:Menu, Chawla07:Algorithmic, Hart12:Approximate}. This could model buyers at, for example, antique auctions and art auctions (as long as there are no collections to try to assemble or the collections are sold as atomic lots).


\begin{table}
	\scriptsize
	\begin{tabularx}{\textwidth}{L{1} L{1} L{1} L{1}}
		\toprule
		Valuations & Auction class & Our bounds & Prior bounds\\\midrule
		Additive or unit-demand  & Length-$\ell$ lottery menu & $U\sqrt{\ell m \log (\ell m)/N}$ & N/A\\\midrule
		Additive, item-independent\footnotemark[1] &  Length-$\ell$ item lottery menu  &$U\sqrt{\ell \log \ell /N}$ & N/A\\\bottomrule
	\end{tabularx}
	\noindent\par
	{\center
		\scriptsize
		\footnotemark[1]~~{Additive cost function}
	}
	
	\caption{Generalization bounds in big-$\tilde O$ notation for lotteries. The maximum profit achievable by any mechanism in the class over the support of the buyers' valuation distribution is $U$. There are $m$ items, $N$ samples, and the cost function is general unless otherwise noted.}\label{tab:results_lotteries}
\end{table}

\begin{table}
	\scriptsize
	\begin{tabularx}{\textwidth}{L{0.55} L{1} L{0.8} L{1.25} L{1.4}}
		\toprule
		\textbf{Valuations} & \textbf{Mechanism class} & \textbf{Price class} & \textbf{Our bounds} & \textbf{Prior bounds} \\\midrule
		General & \multirow{2}{*}{\specialcell{Length-$\ell$ menus of\\two-part tariffs\\over $\kappa$ units}} & Anonymous & $U\sqrt{\ell \log (\kappa n\ell)/N}$ &N/A\\\cmidrule{3-5}
		& & Non-anonymous & $U\sqrt{n\ell \log (\kappa n\ell)/N}$ &N/A\\\cmidrule{2-5}
		& \specialcell{Non-linear pricing} & Anonymous & $U\sqrt{m\prod_{i = 1}^m (\kappa_i + 1)/N}$\footnotemark[3] &N/A\\\cmidrule{3-5}
		& & Non-anonymous & $U\sqrt{nm\prod_{i = 1}^m (\kappa_i + 1)/N}$\footnotemark[3] &N/A\\\cmidrule{2-5}
		& \multirow{2}{*}{\specialcell{Additively\\decomposable\\non-linear pricing}} & Anonymous & $U\sqrt{m\sum_{i = 1}^m \kappa_i/N}$\footnotemark[3] &N/A\\\cmidrule{3-5}
		& & Non-anonymous & $U\sqrt{nm\sum_{i = 1}^m \kappa_i/N}$\footnotemark[3] &N/A\\\cmidrule{2-5}
		& Item-pricing & Anonymous & $U\sqrt{m^2/N}$& $U\sqrt{m^2/N}$\footnotemark[4]\\\cmidrule{3-5}
		& & Non-anonymous & $U\sqrt{nm(m + \log n)/N}$ & $U\sqrt{nm^2\log n/N}$\footnotemark[4]\\\midrule
		Unit-demand & Item-pricing & Anonymous & $U\sqrt{m \cdot \min\{m, \log (nm)\}/N}$& $U\sqrt{m^2/N}$\footnotemark[4]\\\cmidrule{3-5}
		& & Non-anonymous & $U\sqrt{nm \log (nm)/N}$ & $U\sqrt{nm^2\log n/N}$\footnotemark[4]\\\midrule
		Additive & Item-pricing & Anonymous & $U\sqrt{m \log m/N}$& $U\sqrt{m \log m/N}$\footnotemark[4], $\left(U/\delta\right)\sqrt{m\log\left(nN\right)/N}$\footnotemark[2]\\\cmidrule{3-5}
		& & Non-anonymous & $U\sqrt{nm \log (nm)/N}$ & $U\sqrt{nm \log (nm)/N}$\footnotemark[4], $\left(U/\delta\right)\sqrt{nm\log \left(N\right)/N}$\footnotemark[2]\\\midrule
		\multirow{2}{*}{\specialcell{Additive,\\item-\\independent\footnotemark[1]}} & Item-pricing & Anonymous & $U\sqrt{1/N}$& $U\sqrt{m \log m/N}$\footnotemark[4], $\left(U/\delta\right)\sqrt{m\log\left(nN\right)/N}$\footnotemark[2]\\\cmidrule{3-5}
		& & Non-anonymous& $U\sqrt{n \log n/N}$ & $U\sqrt{nm \log (nm)/N}$\footnotemark[4], $\left(U/\delta\right)\sqrt{nm\log \left(N\right)/N}$\footnotemark[2]\\\bottomrule
	\end{tabularx}
	\noindent\par
	{\center
		\scriptsize
		\footnotemark[1]~~{Additive cost function};\quad
		\footnotemark[3]~~{$\kappa_i$ is an upper bound on the number of units available of item $i$};\quad
		\footnotemark[4]~~{\citet{Morgenstern16:Learning}};\quad
		\footnotemark[2]~~{\citet{Syrgkanis17:Sample}. The probability these bounds fail to hold is $\delta$. In all other bounds, $\delta$ appears in a log so we suppress it using big-$\tilde O$ notation.}
	}
	
	\caption{Generalization bounds in big-$\tilde O$ notation for pricing mechanisms. We denote the maximum profit achievable by any mechanism in the class over the support of the buyers' valuation distribution by $U$. There are $m$ items, $n$ buyers, and $N$ samples. The cost function is general unless otherwise noted.}\label{tab:results_pricing}
\end{table}
\begin{table}
	\scriptsize
	\begin{tabularx}{\textwidth}{L{1} L{1} L{.9} L{1.1}}
		\toprule
		\textbf{Valuations} & \textbf{Auction class} & \textbf{Our bounds} & \textbf{Prior bounds}\\\midrule
		General & AMAs and $\lambda$-auctions & $U\sqrt{n^{m+1}m\log n/N}$ & $cU\sqrt{m/N} n^{m+2}\left(n^2 + \sqrt{n^m}\right)$\footnotemark[5]\footnotemark[6]\\\cmidrule{2-4}
		& VVCAs & $U\sqrt{n^2m2^m\log n/N}$ & $cU\sqrt{m/N} n^{m+2}\left(n^2 + \sqrt{n^m}\right)$\footnotemark[5]\footnotemark[6]\\\cmidrule{2-4}
		&MBARPs &$U \sqrt{m(\log n + m)/N}$&$U \sqrt{m^3 \log n/N}$\footnotemark[5]\\\midrule
		Additive & Second price item auctions with anonymous reserve prices &$U\sqrt{m \log m/N}$ &$U\sqrt{m \log m/N}$\footnotemark[4]\\\cmidrule{2-4}
		& Second price item auctions with non-anonymous reserve prices &$U\sqrt{nm \log (nm)/N} $ &$U\sqrt{nm \log (nm)/N}$\footnotemark[4]\\\midrule
		\multirow{2}{*}{\specialcell{Additive, item-independent\footnotemark[1]}} & Second price item auctions with anonymous reserve prices &$U\sqrt{1/N}$ &$U\sqrt{m \log m/N}$\footnotemark[4]\\\cmidrule{2-4}
		& Second price item auctions with non-anonymous reserve prices &$U\sqrt{n \log n/N} $ &$U\sqrt{nm \log (nm)/N}$\footnotemark[4]\\\bottomrule
	\end{tabularx}
	\noindent\par
	{\center
		\scriptsize
		\footnotemark[1]~~{Additive cost function};\quad
		\footnotemark[6]~~{The value of $c > 1$ depends on the range of the auction parameters};\quad
		\footnotemark[4]~~{\citet{Morgenstern16:Learning}};\quad
		\footnotemark[5]~~{\citet{Balcan16:Sample}}.
	}
	\caption{Generalization bounds in big-$\tilde O$ notation for auctions. We denote the maximum profit achievable by any mechanism in the class over the support of the buyers' valuation distribution by $U$. There are $m$ items, $n$ buyers, and $N$ samples. The cost function is general unless otherwise noted.}\label{tab:results_auctions}
\end{table}%



\subsection{Related research}\label{sec:related}
\subsubsection{Sample-based mechanism design}
Sample-based mechanism design was introduced in the context of \emph{automated mechanism design}
(AMD), where the goal is to design algorithms that take as input information about a set of buyers
and return a mechanism that maximizes an objective such as revenue~\citep{Conitzer02:Mechanism,Sandholm03:Automated,Conitzer04:Self}. The input information about the buyers in early AMD was an explicit description of the distribution over their valuations. Later, sample-based mechanism design was introduced where the input is a set of samples from this distribution~\citep{Likhodedov04:Boosting,Likhodedov05:Approximating,Sandholm15:Automated}. Those papers also introduced the idea of searching for a high-revenue mechanism in a parameterized space where any parameter vector yields a mechanism that satisfies the individual rationality and incentive-compatibility constraints. They did not provide generalization guarantees.

\citet{Balcan05:Mechanism, Balcan08:Reducing} were the first to study the connection between learning theory and revenue maximization. They showed how to use an algorithm $\pazocal{A}$ that returns a high-revenue, manipulable mechanism in order to find a high-revenue, incentive-compatible mechanism. They study settings with unrestricted supply, whereas we primarily focus on settings with limited supply.
	
	More recent research has provided generalization guarantees when there is limited supply, with a particular focus on single-parameter settings~\citep{Alon17:Submultiplicative, Elkind07:Designing,Cole14:Sample, Huang15:Making, Mohri14:Learning, Morgenstern15:Pseudo, Roughgarden15:Ironing, Bubeck17:Online, Chawla14:Mechanism}.  \citet{Devanur16:Sample,Gonczarowski17:Efficient,Guo19:Settling}, and \citet{Hartline16:Non} provide computationally efficient algorithms for learning nearly-optimal single-item auctions in various settings. In contrast, we study multi-parameter settings. Our bounds do not apply to the state-of-the-art in this direction by \citet{Guo19:Settling} because their approach does not involve optimizing over a mechanism class with continuously tunable parameters.

\citet{Balcan14:Learning} drew on classic tools from learning theory to provide algorithms and generalization guarantees for the related problem of learning agents' preferences. Their analysis made connections to the concept of \emph{generalized linear functions} from the structured prediction literature~\citep{Collins00:Discriminative}. Their algorithms predict the future purchases of utility-maximizing agents.


\citet{Morgenstern16:Learning} later used this concept of generalized linear functions to provide sample complexity guarantees for multi-item revenue maximization.
They provide a technique for bounding a mechanism class's pseudo-dimension that requires two steps, described at a high level here and in detail in Appendix~\ref{APP:STRUCTURED}. First, one must show that for any mechanism in the class, its allocation function is a $d$-dimensional linear function for some $d \in \Z$. Next, fixing a set of samples and an allocation per sample, one must bound the pseudo-dimension of the set of revenue functions across all mechanisms that induce those allocations.
	
The guarantees presented in this paper offer several advantages over Morgenstern and Roughgarden's approach. First, our main theorem depends on a structural property---the piecewise-linear form of the revenue function---that is not defined in terms of any learning theory concept (such as generalized linear functions or pseudo-dimension) and thus can be more readily applied.
Moreover, in several cases, \citet{Morgenstern16:Learning} proved loose guarantees using structured prediction; in their appendix, they used a first-principles approach to prove stronger guarantees.
Their structured prediction proof technique requires them to bound the total number of allocations a mechanism class can induce on a set of samples. Their bound is a bit loose, and we are able to tighten it using the techniques we develop in this paper, as we detail in Appendix~\ref{APP:STRUCTURED}.
By combining our analysis techniques with tools from structured prediction, we are able to match the tighter bounds  that \citet{Morgenstern16:Learning}, which answer an open question they posed. Finally, we apply our guarantees to a wide variety of mechanism classes, both simple and complex, whereas \citet{Morgenstern16:Learning} applied their guarantees to three mechanism classes that are ``simple'' by design: item-pricing mechanisms, grand-bundle-pricing mechanisms (where the grand bundle is sold as a single unit), and second-price item auctions.

 \citet{Syrgkanis17:Sample} also suggests a general technique for providing generalization guarantees which he applies to several ``simple'' mechanism classes: the same three as \citet{Morgenstern16:Learning} as well as single-item \emph{$t$-level auctions}~\citep{Morgenstern15:Pseudo}. His generalization guarantees apply only to \emph{empirical revenue maximization} algorithms, which return the mechanism in a class that maximizes average revenue over the samples. This is in contrast to our bounds (and those by \citet{Morgenstern16:Learning}) which apply uniformly to every mechanism in a given class. This is crucial when empirical revenue maximization is not computationally feasible. Another advantage of our bounds is that they grow logarithmically in $\frac{1}{\delta}$ where $\delta$ is the probability that the bound fails to hold (as do those by \citet{Morgenstern16:Learning}). In contrast, the bounds by \citet{Syrgkanis17:Sample} grow linearly in $\frac{1}{\delta}$.

In Section~\ref{SEC:COMPARISON}, we provide more details on how our results compare to those by \citet{Morgenstern16:Learning} and \citet{Syrgkanis17:Sample}, as well as a detailed comparison of our results to other papers on the sample complexity of multi-item revenue maximization~\citep{Balcan16:Sample, Medina17:Revenue, Cai17:Learning,Gonczarowski18:Sample}.

\subsubsection{Dynamic mechanism design} Dynamic pricing is a  similar but distinct problem from ours where prices are adjusted over a finite time horizon and the consumer demand function is unknown~\citep[e.g.,][all of whom study single-item settings]{Araman09:Dynamic,Besbes09:Dynamic,Broder12:Dynamic}. The goal is typically to minimize regret (the difference between the cumulative profit of the best prices in hindsight and that of the chosen prices).


\subsubsection{Approximation guarantees}\label{sec:apx}
Many mechanisms we analyze can guarantee approximately-optimal revenue.

\paragraph{Item-pricing mechanisms.} For a single unit-demand buyer with a bounded value distribution, item-pricing mechanisms can yield a constant fraction of optimal revenue~\citep{Chawla07:Algorithmic}. Moreover, given multiple unit-demand buyers and constraints on which allocations are feasible, item-pricing mechanisms provide a constant-factor approximation~ \citep{Chawla10:Multi}.

For an additive buyer with independent values, item-pricing mechanisms provide a $O(\log^2 m)$ fraction of optimal revenue~\citep{Hart12:Approximate}, later improved to $O(\log m)$~\citep{Li13:Revenue}.
The better of an item-pricing mechanism or selling the grand bundle as a single unit provides a constant-factor approximation in this setting~\citep{Babaioff14:Simple} and generalizations thereof~\citep{Bateni15:Revenue,Eden21:Simple}\footnote{For multiple additive buyers, a VCG mechanism with bidder entries fees achieves a constant-factor approximation~\citep{Yao14:n}.}. Similarly, for a subadditive buyer, the better of an item-pricing mechanism and a more general bundling mechanism provides a constant-factor approximation~\citep{Rubinstein15:Simple}.

For multiple buyers with XOS values over independent items, a non-anonymous item-pricing mechanism or an anonymous item-pricing mechanism with an entry fee is a constant-factor approximation~\citep{Cai17:Simple}. For  subadditive buyers, the approximation is $O(\log\log m)$~\citep{Dutting20:Log} and a single random price is a $O(2^{\sqrt{\log m \log\log m}})$ approximation~\citep{Balcan08:Item}.

\paragraph{Two-part tariffs.} For buyers with additive values up to a matroid feasibility constraint, a sequential variant of two-part tariffs provides a constant-factor approximation~\citep{Chawla16:Mechanism}.


\paragraph{Lotteries.}
For a single additive buyer with independent values, \citet{Babaioff17:Menu} proved that a lottery menu of length $(\log m / \epsilon)^{O(m)}$ is a $(1-\epsilon)$-factor approximation. For a single unit-demand buyer with independent item values, \citet{Kothari19:Approximation} introduce the notion of \emph{symmetric menu complexity}, which is the number of menu entries up to permutations of the items. A quasi-polynomial symmetric menu complexity suffices to guarantee a $(1-\epsilon)$ approximation.

