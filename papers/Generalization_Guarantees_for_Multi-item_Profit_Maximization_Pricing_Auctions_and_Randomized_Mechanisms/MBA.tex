\subsection{Mixed bundling auctions}\label{app:MBA}

Mixed bundling auctions (MBAs) are defined by a single parameter $\gamma$. They correspond to a $\lambda$-auction where $\lambda(Q) = \gamma$ if some buyer receives the grand bundle under allocation $Q$ and 0 otherwise. The class of MBAs is particularly simple, and we prove an even tighter bound on the Rademacher complexity of MBAs than that guaranteed by Theorem~\ref{thm:main_pdim}. Our analysis requires us to understand how the profit of a $\gamma$-MBA on a single bidding instance changes as a function of $\gamma$. We take advantage of this function's structural properties, first uncovered by \citet{Jehiel07:Mixed}: no matter the number of buyers and no matter the number of items, there exists an easily characterizable value $\gamma^*$ such that the function in question is increasing as $\gamma$ grows from 0 to $\gamma^*$, and then it is non-increasing as $\gamma$ grows beyond $\gamma^*$. This is depicted in Figure~\ref{fig:revGraphn}.
\begin{figure}
  \centering
  \includegraphics[scale=1]{rev_graph}
  \caption{An example of the $\gamma$-MBA revenue of a single bidding instance as $\gamma$ varies.}
  \label{fig:revGraphn}
\end{figure}
Intuitively, $\gamma^*$ represents the number at which $\gamma$ has grown so large that the MBA has morphed into a second price auction on the grand bundle. As a result, no matter how much larger $\gamma$ grows beyond $\gamma^*$, the value of $\gamma$ no longer factors into the profit function. This simple structure allows us to prove the strong generalization guarantee described in Theorem~\ref{thm:MBA}.

\begin{theorem}\label{thm:MBA}
Let $\pazocal{M}$ be the class of MBAs. Then $\pdim(\mclass) = 2$.
\end{theorem}

\begin{proof} 
First, we show that the pseudo-dimension of the class of $n$-buyer, $m$-item MBAs is at most 2. Let $\sample = \left\{\vec{v}^{(1)}, \dots, \vec{v}^{(N)}\right\}$ be a set of $n$-buyer valuation functions that can be shattered by a set $\Gamma$ of $2^N$ MBAs. This means that there exist $N$ witnesses $z^{(1)}, \dots, z^{(N)}$ such that each MBA in $\Gamma$ induces a binary labeling of the samples $\vec{v}^{(j)}$ of $\sample$ (whether the profit of the MBA on $\vec{v}^{(j)}$ is at least $z_j$ or strictly less than $z^{(j)}$). Since $\sample$ is shatterable, we can thus label $\sample$ in every possible way using MBAs in $\Gamma$.

Now, fix one sample $\vec{v}^{(i)} \in \sample$. We denote the profit of the $\gamma$-MBA on $\vec{v}^{(i)}$ as a function of $\gamma$ as ${\normalfont \profit}_{\vec{v}^{(i)}}(\gamma)$. From Lemma~\ref{lem:rev_struct}, we know that there exists $\gamma^*_i \in [0,\infty)$, such that ${\normalfont \profit}_{\vec{v}^{(i)}}(\gamma)$ is non-decreasing on the interval $[0,\gamma^*_i]$ and non-increasing on the interval $(\gamma^*_i, \infty)$. Therefore, there exist two thresholds $t^{(1)}_i \in [0,\gamma^*_i]$ and $t^{(2)}_i \in (\gamma^*_i, \infty) \cup \{\infty\}$ such that ${\normalfont \profit}_{\vec{v}^{(i)}}(\gamma)$ is below its threshold for $\gamma \in [0,t^{(1)}_i)$, above its threshold for $\gamma \in (t^{(1)}_i, t^{(2)}_i)$, and below its threshold for $\gamma \in (t^{(2)}_i, \infty)$. Now, merge these thresholds for all $N$ samples on the real line and consider the interval $(t_1,t_2)$ between two adjacent thresholds. The binary labeling of the samples in $\sample$ on this interval is fixed. In other words, for any sample $\vec{v}^{(j)} \in \sample$, ${\normalfont \profit}_{\vec{v}^{(j)}}(\gamma)$ is either at least $z^{(j)}$ or strictly less than $z^{(j)}$ for all $\gamma \in (t_1,t_2)$. There are at most $2N+1$ intervals between adjacent thresholds, so at most $2N+1$ different binary labelings of $\sample$. Since we assumed $\sample$ is shatterable, it must be that $2^N \leq 2N+1$, so $N \leq 2.$

Finally, we show that the pseudo-dimension of the class of $n$-buyer, $m$-item MBAs is at least 2 by constructing a set $\sample = \left\{\vec{v}^{(1)}, \vec{v}^{(2)}\right\}$ that can be shattered by the set of MBAs. To construct this set of samples $\sample$, let \[v_1^{(1)}\left(\vec{q}\right) = v_2^{(1)}\left(\vec{q}\right) = \begin{cases} 0 &\text{if } ||\vec{q}||_1 < \lfloor m/2 \rfloor\\
3 &\text{if } \lfloor m/2 \rfloor \leq ||\vec{q}||_1 \end{cases} \text{ and } v_1^{(2)}\left(\vec{q}\right) = v_2^{(2)}\left(\vec{q}\right) = \begin{cases} 0 &\text{if } ||\vec{q}||_1 < \lfloor m/2 \rfloor\\
3 &\text{if } \lfloor m/2 \rfloor \leq ||\vec{q}||_1 < m\\
4 &\text{if } ||\vec{q}||_1 = m. \end{cases}\] Finally, let buyers 3 through $n$ have all-zero valuations in both $\vec{v}^{(1)}$ and $\vec{v}^{(2)}$ and let the cost function be 0 for all allocations.

Now, let $z^{(1)} = 3$ and $z^{(2)} = 4$. We define four MBAs parameterized by the coefficients $\gamma_1 = 0, \gamma_2 = 1.5, \gamma_3 = 1.75, \gamma_4  = 2.5.$ It is easy to check that this set of MBAs shatters $\sample$, witnessed by $z^{(1)}$ and $z^{(2)}$. For example, see Table~\ref{tab:shattering}.
\begin{table}\centering
{\begin{tabular}{lll}
\textbf{$\gamma$ value} & \textbf{Profit on $\vec{v}^1$} & \textbf{Profit on $\vec{v}^2$} \\\hline
0                 & $0\leq z^{(1)}$                               & $2 \leq z^{(2)}$                               \\
1.5                & $3 \leq z^{(1)}$                               & $5 > z^{(2)}$                            \\
1.75                  & $3.5 > z^{(1)}$                               & $5.5 > z^{(2)}$
\\
2.5                & $5 > z^{(1)}$                               & $4 \leq z^{(2)}$                               \\
\end{tabular}}
\caption{Example of a shattered set of size 2\label{tab:shattering}}
\end{table}

The generalization guarantee follows from Theorem~\ref{thm:pdim}.
\end{proof}


\begin{lemma}\label{lem:rev_struct}
For a valuation vector $\vec{v}$, let $\profit_{\vec{v}}(\gamma)$ be the profit of the $\gamma$-MBA on $\vec{v}$ as a function of $\gamma$. There exists $\gamma^* \in [0,\infty)$ such that ${\normalfont \profit}_{\vec{v}}(\gamma)$ is non-decreasing on the interval $[0,\gamma^*]$ and non-increasing on the interval $(\gamma^*, \infty)$.
\end{lemma}

For additive buyers, this lemma is implied by Theorem 1 in the paper by \citet{Jehiel07:Mixed} which provides the derivative of $\profit_{\vec{v}}(\gamma)$. The techniques used by \citet{Jehiel07:Mixed} extend immediately to general buyers as well, as we show here.

\begin{proof}[Proof of Lemma~\ref{lem:rev_struct}]
We will show that ${\normalfont \profit}_{\vec{v}}$ can be decomposed into simple components, each of which can be easily analyzed on its own, and by combining these analyses, we prove the lemma statement. Suppose $Q^* = \left(\vec{q}_1^*, \dots, \vec{q}_n^*\right)$ is the resulting allocation of a certain $\gamma$-MBA $M$ and $Q^{-i} = \left(\vec{q}_1^{-i}, \dots, \vec{q}_n^{-i}\right)$ is the boosted social-welfare maximizing allocation without buyer $i$'s participation. More explicitly, $Q^* = \text{argmax} \left\{\sum_{i = 1}^n v_i\left(\vec{q}_i\right) + \lambda\left(Q\right) - c(Q) \right\}$ and $Q^{-i} = \text{argmax} \left\{\sum_{k \not=i} v_k\left(\vec{q}_k\right) + \lambda\left(Q\right) - c(Q)\right\}$, where $\lambda\left(Q\right)$ is set according to the MBA allocation boosting rule for all $Q$. Then buyer $i$ pays \[p_{i, \vec{v}}\left(\gamma\right) = v_i\left(\vec{q}^*_i\right) - \left[\sum_{j = 1}^n v_j\left(\vec{q}^*_j\right) + \lambda\left(Q^*\right) - c(Q^*) - \left(\sum_{j \not= i} v_j\left(\vec{q}^{-i}_j\right) + \lambda\left(Q^{-i}\right) - c(Q^{-i}) \right)\right].\] This means that \begin{align*}&{\normalfont \profit}_{\vec{v}}(\gamma) = \sum_{i = 1}^n p_{i, \vec{v}}\left(\gamma\right)\\
= \text{ } &(1 - n)\sum_{i = 1}^n v_i\left(\vec{q}_i^*\right) - n\left(\lambda\left(Q^*\right) - c\left(Q^*\right)\right) + \sum_{i = 1}^n \sum_{j \not= i} v_j\left(\vec{q}_j^{-i}\right) + \lambda\left(Q^{-i}\right) - c\left(Q^{-i}\right).\end{align*}

The profit function can be split into $n+1$ functions:
$f_{i,\vec{v}}(\gamma) = \sum_{j \not= i} v_j\left(\vec{q}^{-i}_j\right) + \lambda\left(Q^{-i}\right) - c\left(Q^{-i}\right)$ for $i \in \{1, \dots, n\}$ and $g_{\vec{v}}(\gamma) = (1 - n)\sum_{i = 1}^n v_i\left(\vec{q}_i^*\right) - n\left(\lambda\left(Q^*\right) - c\left(Q^*\right)\right).$ We claim that $f_{i,\vec{v}}(\gamma)$ is continuous for all $i$, whereas $g_{\vec{v}}(\gamma)$ has at most one discontinuity. This means that ${\normalfont \profit}_{\vec{v}}(\gamma) = \sum_{i = 1}^n f_{i,\vec{v}}(\gamma) + g_{\vec{v}}(\gamma)$ has at most one discontinuity as well. Moreover, the slope of $\sum_{i = 1}^n f_{i,\vec{v}}(\gamma)$ is between zero and $n$, whereas the slope of $g_{\vec{v}}(\gamma)$ is zero until its discontinuity, and then is $-n$. Therefore, the slope of ${\normalfont \profit}_{\vec{v}}(\gamma)$ is at least zero before its discontinuity and at most zero after its discontinuity. This is enough to prove the lemma statement.

To see why these properties are true for the functions $f_{i,\vec{v}}(\gamma)$, first let $\tilde{Q}^{-i} = \left(\tilde{\vec{q}}^{-i}_1, \dots, \tilde{\vec{q}}^{-i}_n\right)$ be the VCG allocation without buyer $i$, i.e., $\tilde{Q}^{-i} = \text{argmax} \left\{\sum_{k \not=i} v_k\left(\vec{q}_k\right) - c(Q)\right\}$. If one buyer is allocated the grand bundle in allocation $\tilde{Q}^{-i}$, then this allocation will only be more valuable as $\gamma$ grows, so $\tilde{Q}^{-i} = \text{argmax} \left\{\sum_{k \not=i} v_k\left(\vec{q}_k\right) + \lambda\left(Q\right) - c(Q)\right\}$ for all values of $\gamma$, which means that $f_{i,\vec{v}}(\gamma) = \sum_{j \not= i} v_j\left(\tilde{\vec{q}}_j^{-i}\right) + \lambda\left(\tilde{Q}^{-i}\right) - c\left(\tilde{Q}^{-i}\right) = \sum_{j \not= i} v_j\left(\tilde{\vec{q}}_j^{-i}\right) + \gamma - c\left(\tilde{Q}^{-i}\right)$ for all values of $\gamma$ as well. Clearly, in this case, $f_{i,\vec{v}}(\gamma)$ is increasing and continuous. Otherwise, using the notation $c^1$ to denote the cost of producing the grand bundle, we know there exists some value $\gamma_i$ such that $\sum_{j \not= i} v_j\left(\tilde{\vec{q}}_j^{-i}\right) + \lambda\left(\tilde{Q}^{-i}\right) - c\left(\tilde{Q}^{-i}\right) = \sum_{j \not= i} v_j\left(\tilde{\vec{q}}_j^{-i}\right) - c\left(\tilde{Q}^{-i}\right) \geq \max_{k \not= i} \left\{v_k\left(\vec{1}\right) + \gamma - c^1\right\}$ if $\gamma \leq \gamma_i$ and
$\sum_{j \not= i} v_j\left(\tilde{\vec{q}}_j^{-i}\right) - c\left(\tilde{Q}^{-i}\right) < \max_{k \not= i} \left\{v_k\left(\vec{1}\right) + \gamma - c^1\right\}$ if $\gamma > \gamma_i.$ This means that $\tilde{Q}^{-i}$ is the allocation of the $\gamma$-MBA without buyer $i$'s participation for $\gamma \leq \gamma_i$, and the allocation of the $\gamma$-MBA without buyer $i$'s participation for $\gamma > \gamma_i$ is the one where the highest buyer for the grand bundle (excluding buyer $i$) wins the grand bundle. Therefore, \[f_{i,\vec{v}}(\gamma) = \begin{cases} \sum_{j \not= i} v_j\left(\tilde{\vec{q}}_j^{-i}\right) - c\left(\tilde{Q}^{-i}\right) & \text{if } \gamma \leq \gamma_i\\ \max_{k \not= i} \left\{v_k\left(\vec{1}\right) + \gamma - c^1\right\} &\text{if } \gamma > \gamma_i.\end{cases}\] Notice that $\sum_{j \not= i} v_j\left(\tilde{\vec{q}}_j^{-i}\right) - c\left(\tilde{Q}^{-i}\right) = \max_{k \not= i} \left\{v_k\left(\vec{1}\right) + \gamma_i - c^1\right\}$, so $f_{i,\vec{v}}(\gamma)$ is continuous. Finally, it is clear that the slope of each $f_{i,\vec{v}}(\gamma)$ is between 0 and 1, so the slope of $\sum_{i = 1}^n f_{i,\vec{v}}(\gamma)$ is between 0 and $n$.

Similarly, let $\tilde{Q} = \left(\tilde{\vec{q}}_1, \dots, \tilde{\vec{q}}_n\right)$ be the allocation of the VCG mechanism run on $\vec{v}$. Then there exists some $\gamma^*$ such that $\tilde{Q}$ is the allocation of the $\gamma$-MBA for $\gamma \leq \gamma^*$ and the allocation of the $\gamma$-MBA for $\gamma > \gamma^*$ is the one where the highest bidder for the grand bundle wins the grand bundle. More explicitly, $\sum_{i=1}^n v_i\left(\tilde{\vec{q}}_i\right) + \lambda\left(\tilde{Q}\right) - c\left(\tilde{Q}\right) \geq \max_{k \in [n]}\left\{v_k\left(\vec{1}\right) + \gamma - c^1\right\}$ if $\gamma \leq \gamma^*$ and
$\sum_{i=1}^n v_i\left(\tilde{\vec{q}}_i\right) + \lambda\left(\tilde{Q}\right) - c\left(\tilde{Q}\right) < \max_{k \in [n]}\left\{v_k\left(\vec{1}\right) + \gamma - c^1\right\}$ if $\gamma > \gamma^*$. Therefore, \[g_{\vec{v}}(\gamma) = \begin{cases} (1-n)\sum_{i=1}^n v_i\left(\tilde{\vec{q}}_i\right) - n\left(\lambda\left(\tilde{Q}\right) - c\left(\tilde{Q}\right)\right) & \text{if } \gamma \leq \gamma^*\\ (1-n)\max \left\{v_k\left(\vec{1}\right)\right\} - n\left(\gamma - c^1\right) &\text{if } \gamma > \gamma^*.\end{cases}\] Therefore, $g_{\vec{v}}(\gamma)$ has at most one discontinuity, which falls at $\gamma^*$. Moreover, the slope of $g_{\vec{v}}(\gamma)$ is 0 for $\gamma < \gamma^*$ and $-n$ for $\gamma > \gamma^*$. As described, these properties of $f_{i,\vec{v}}(\gamma)$ and $g_{\vec{v}}(\gamma)$ are enough to show that the lemma statement holds.
\end{proof}