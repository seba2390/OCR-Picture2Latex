We now show that a diverse array of mechanism classes are delineable, so we can apply Theorem~\ref{thm:main_pdim}.
We warm up with Examples~\ref{ex:2PT} and \ref{ex:item_pricing}, which imply the following lemmas.

\begin{lemma}\label{lem:easy2part}
The class of two-part tariffs for one buyer and $\kappa$ units of an item is $\left(2,{\kappa + 1 \choose 2}\right)$-delineable.
\end{lemma}


\begin{lemma}
The class of anonymous item-pricing mechanisms is $\left(m, n{2^m \choose 2}\right)$-delineable.
\end{lemma}

\subsubsection{Non-linear pricing mechanisms.}\label{sec:non_linear}
Non-linear pricing mechanisms are used to sell multiple units of a set of items. We make the following natural assumption which says that as the number of units in an allocation grows, the cost will eventually exceed the buyers' welfare. 
\begin{assumption}\label{assumption:unit_cap}
There is a cap $\kappa_i \in \Z$ per item $i$ such that it costs more to produce $\kappa_i$ units of item $i$ than the buyers will pay.
In other words, for all $\vec{v}$ in the support of $\dist$ and all allocations $Q = \left(\vec{q}_1, \dots, \vec{q}_n\right)$, if there exists an item $i$ such that $\sum_{j = 1}^n q_j[i] > \kappa_i$, then $\sum_{j = 1}^n v_j\left(\vec{q}_j\right) - c\left(Q\right) < 0$.
\end{assumption}



\paragraph{Menus of two-part tariffs.} Menus of two-part tariffs are a generalization of Example~\ref{ex:2PT}. The seller offers the buyers $\ell$ different two-part tariffs and each buyer chooses the tariff and number of units that maximizes his utility. For example, consumers often choose among various membership tiers---typically with a larger upfront fee and lower future payments---for health clubs,  wholesale  stores,  amusement  parks, credit cards, and cellphone plans.
Under non-anonymous prices, let $\left(p_{1, j}^{(1)}, p_{2, j}^{(1)}\right), \dots, \left(p_{1, j}^{(\ell)}, p_{2, j}^{(\ell)}\right)$ be the menu of two-part tariffs that the seller offers to buyer $j$. Here, $p_{1,j}^{(i)}$ is the upfront fee of the $i^{th}$ tariff and $p_{2,j}^{(i)}$ is the price per unit. Under anonymous prices, $p_{1,1}^{(i)} = \cdots = p_{1,n}^{(i)}$ and $p_{2,1}^{(i)} = \cdots = p_{2,n}^{(i)}$. Each buyer  chooses the tariff $t_j \in [\ell]$ and the number of units $q_j \geq 1$ maximizing his utility, and pays $p_{1,j}^{(t_j)} + p_{2,j}^{(t_j)} \cdot q_j$. In this context, an allocation is  a vector $Q = \left(q_1, \dots, q_n\right)$ where $q_j \in \Z_{\geq 0}$ is the number of units buyer $j$ buys.

 We make the natural assumption that the seller will not choose prices that result in negative profit. In other words, he will select a \emph{profit non-negative menu of two-part tariffs}, formalized below.

\begin{definition}\label{def:PNN_2PT}
For anonymous prices (respectively, non-anonymous), let $\pspace \subseteq \R^{2\ell}$ (respectively, $\pspace' \subseteq \R^{2n\ell}$) be the set of prices where no matter which tariff each buyer chooses and no matter how many units he buys, the seller will obtain non-negative profit. In other words, for each buyer $j \in [n]$, each tariff $t_j \in [\ell]$, and each allocation $Q = \left(q_1, \dots, q_n\right)$, $\sum_{j = 1}^n p_{1,j}^{(t_j)} \cdot \textbf{1}_{\{q_j \geq 1\}} + p_{2,j}^{(t_j)} \cdot q_j - c\left(Q\right) \geq 0.$ The set of \emph{profit non-negative menus of two-part tariffs} is defined by parameters in $\pspace$ (resp., $\pspace'$).
\end{definition}

Under Assumption~\ref{assumption:unit_cap}, no matter which parameters the seller chooses in $\pspace$ or $\pspace'$, if all buyers simultaneously choose the tariff and the number of units $(q_1, \dots, q_n)$ that maximize their utilities, then $\sum_{j = 1}^n q_j \leq \kappa$. See Lemma~\ref{lem:kappa_bnded_2pt}  for the proof. This allows us to prove the following lemma.

\begin{restatable}{lemma}{twoPart}\label{lem:2part}
Let $\pazocal{M}$ and $\pazocal{M}'$ be the classes of anonymous and non-anonymous profit non-negative length-$\ell$ menus of two-part tariffs. Under Assumption~\ref{assumption:unit_cap}, $\mclass$ is $\left(2\ell, n\left(\kappa \ell\right)^2\right)$-delineable and $\mclass'$ is $\left(2n\ell, n\left(\kappa \ell\right)^2\right)$-delineable.
\end{restatable}


\paragraph{General non-linear pricing mechanisms.} We study general non-linear pricing mechanisms under Wilson's \emph{bundling interpretation} \citep{Wilson93:Nonlinear}: if the prices are anonymous, there is a price per quantity vector $\vec{q}$ denoted $p\left(\vec{q}\right)$. The buyers simultaneously choose the bundles maximizing their utilities. If the prices are non-anonymous, there is a price per vector $\vec{q}$ and buyer $j \in [n]$ denoted $p_j\left(\textbf{q}\right)$. These general non-linear pricing mechanisms include \emph{multi-part tariffs} as a special case.
Without assumptions, the parameter space infinite-dimensional since the seller could set prices for every bundle $\vec{q} \in \Z_{\geq 0}^m$. In Lemma~\ref{lem:kappa_bnded_NL}, we show that under Assumption~\ref{assumption:unit_cap}, no buyer will choose a bundle $\vec{q}$ with $q[i] > \kappa_i$ for any $i \in [m]$ if the seller chooses a \emph{profit non-negative non-linear pricing mechanism}. The definition is similar to Definition~\ref{def:PNN_2PT} and is in Appendix~\ref{APP:MAIN} (Definition~\ref{def:PNN_NL}).


\begin{restatable}{lemma}{nonlinear}\label{lem:nonlinear}
Let $\pazocal{M}$ and $\pazocal{M}'$ be the classes of anonymous and non-anonymous profit non-negative non-linear pricing mechanisms. Under Assumption~\ref{assumption:unit_cap},
$\mclass$ is $\left(K, nK^2\right)$-delineable and $\mclass'$ is $\left(nK, nK^2\right)$-delineable.
\end{restatable}

We prove polynomial bounds when prices are additive over items (Lemma~\ref{lem:nonlinear_additive}).

\subsubsection{Item-pricing mechanisms.}
We now apply Theorem~\ref{thm:main_pdim} to anonymous and non-anonymous item-pricing mechanisms. Unlike non-linear pricing, there is only one unit of each item for sale. 
Under anonymous prices, the seller sets a price per item. Under non-anonymous prices, there is a buyer-specific price per item. We make the common assumption~\citep[e.g.,][]{Feldman15:Combinatorial, Babaioff14:Simple, Cai16:Duality} that there is a fixed, arbitrary ordering on the buyers such that the first buyer arrives and buys the bundle that maximizes his utility, then the next buyer arrives and buys the bundle of remaining items that maximizes his utility, and so on.

\begin{restatable}{lemma}{itemAdd}\label{lem:item_pricing_add}
Let $\pazocal{M}$ (resp., $\pazocal{M}'$) be the class of item-pricing mechanisms with anonymous (resp., non-anonymous) prices. For additive buyers, $\mclass$ is $\left(m, m\right)$-delineable and $\mclass'$ is $\left(nm,nm\right)$-delineable.
\end{restatable}

In Appendix~\ref{APP:STRUCTURED}, we connect the hyperplane structure we investigate in this paper to the structured prediction literature in machine learning~\citep{Collins00:Discriminative}, thus strengthening our generalization bounds for item-pricing mechanisms under buyers with unit-demand and general valuations and answering an open question by \citet{Morgenstern16:Learning}.


\subsubsection{Auctions.} We now present applications of Lemma~\ref{lem:main_pdim} to auctions in single-unit settings.

\paragraph{Second price item auctions with reserves.} We study additive buyers in this setting. Under non-anonymous reserves, there is a price $p_j\left(\vec{e}_i\right)$ for each item $i$ and buyer $j$. The buyers submit bids on the items. For each item $i$, the highest bidder $j$ wins the item if her bid is above $p_j\left(\vec{e}_i\right)$. She pays the maximum of the second highest bid and $p_j\left(\vec{e}_i\right)$. Under anonymous reserves, $p_1\left(\vec{e}_i\right)  = \cdots = p_n\left(\vec{e}_i\right)$.
\begin{restatable}{lemma}{secondPrice}\label{lem:second_price}
Let $\pazocal{M}$ and $\pazocal{M}'$ be the classes of anonymous and non-anonymous second price item auctions. Then $\mclass$ is $\left(m, m\right)$-delineable and $\mclass'$ is $\left(nm, m\right)$-delineable.
\end{restatable}

In Section~\ref{SEC:COMPARISON}, we compare these results with those of prior research~\citep{Morgenstern16:Learning,Devanur16:Sample,Syrgkanis17:Sample}.

\paragraph{Mixed bundling auctions with reserve prices (MBARPs).}
MBARPs~\citep{Jehiel07:Mixed,Tang12:Mixed} are a VCG generalization.
Intuitively, the MBARP enlarges the set of agents to include the seller, whose values are defined by reserve prices. The auction boosts the social welfare of any allocation where the grand bundle is allocated and then runs the VCG over this larger set of buyers.
Formally, MBARPs are defined by a parameter $\gamma \geq 0$ and reserves $p\left(\vec{e}_1\right), \dots, p\left(\vec{e}_m\right)$. Let $\lambda$ be a function such that $\lambda\left(Q\right) = \gamma$ if some buyer receives the grand bundle under allocation $Q$ and 0 otherwise. For an allocation $Q$, let $\vec{q}_Q$ be the items not allocated. The MBARP allocation is \[\text{argmax}\left\{\sum_{j = 1}^n v_j\left(\vec{q}_j\right) + \sum_{i : q_Q[i] = 1} p\left(\vec{e}_i\right) + \lambda\left(Q\right) - c\left(Q\right)\right\}.\] The payments are defined as in the VCG mechanism (see Definition~\ref{def:MBARP} in Appendix~\ref{APP:MAIN}).

\begin{restatable}{lemma}{MBARP}\label{lem:MBARP}
Let $\pazocal{M}$ be the set of MBARPs. Then $\mclass$ is $\left(m+1, (n+1)^{2m+1}\right)$-delineable.
\end{restatable}



\emph{Mixed-bundling auctions}~\citep{Jehiel07:Mixed} are MBARPs with no reserve prices. We provide a stronger, specialized guarantee for this class in Appendix~\ref{app:MBA}.


\paragraph{Affine maximizer auctions (AMAs).} AMAs are the
only \emph{ex post} truthful mechanisms over unrestricted
value domains~\citep{Roberts79:Characterization} and  under
natural assumptions, every truthful multi-item auction
is an ``almost'' AMA, that is, an AMA for sufficiently high values~\citep{Lavi03:Towards}.\footnote{Surprisingly, even when the buyers have additive values, AMAs can generate higher revenue than running a separate Myerson auction for each item~\citep{Sandholm15:Automated}.}
An AMA is defined by a weight per buyer $w_j \in \R_{> 0}$ and a boost per allocation $\lambda\left(Q\right) \in \R_{\geq 0}$. Its allocation maximizes the weighted social welfare $\sum_{j = 1}^n w_jv_j\left(\vec{q}_j\right) + \lambda\left(Q\right) - c\left(Q\right).$ The payments have the same form as the VCG payments (see Definition~\ref{def:AMA} in Appendix~\ref{APP:MAIN}).
A virtual valuation combinational auction (VVCA) \citep{Likhodedov04:Boosting} is an AMA where each $\lambda\left(Q\right)$ is split into $n$ terms such that $\lambda\left(Q\right) = \sum_{j = 1}^n \lambda_j\left(Q\right)$ where $\lambda_j\left(Q\right) = c_{j,\vec{q}}$ for all allocations $Q$ that give buyer $j$ exactly bundle $\vec{q}$. Finally, $\lambda$-auctions~\citep{Jehiel07:Mixed} are defined such that $w_1 = \cdots = w_n = 1$.

\begin{restatable}{lemma}{AMA}\label{lem:AMA}
Let $\pazocal{M}$, $\mclass'$, and $\mclass''$ be the classes of AMAs, VVCAs, and $\lambda$-auctions, respectively. Letting $t = \left(n+1\right)^{2m+1}$, we have that $\mclass$ is $\left(2n(n+1) + (n+1)^{m+1},t\right)$-delineable, $\mclass'$ is $\left(n2^m(3 + 2n), t\right)$-delineable, and $\mclass''$ is $\left(\left(n+1\right)^m, t\right)$-delineable.
 \end{restatable}


Lemma~\ref{lem:AMA} implies that exponentially-many samples are sufficient to avoid overfitting. In Appendix~\ref{app:lower}, we prove an exponential number of samples is also necessary.


\subsubsection{Lotteries.}
Lotteries are randomized mechanisms which typically have higher  revenue than deterministic mechanisms.  We analyze a single additive buyer and generalize to unit-demand buyers and multiple buyers in Appendix~\ref{sec:additional_lotteries}.
A \emph{length-$\ell$ lottery menu} is a set $M = \{(\vec{\phi}^{(0)}, p^{(0)}), (\vec{\phi}^{(1)}, p^{(1)}), \dots, (\vec{\phi}^{(\ell)}, p^{(\ell)})\} \subseteq \R^m \times \R$, where $\vec{\phi}^{\left(0\right)} = \vec{0}$ and $p^{\left(0\right)}= 0$. Under the lottery $\left(\vec{\phi}^{(j)}, p^{(j)}\right)$, the buyer pays $p^{(j)}$ and receives each item $i$ with probability $\phi^{(j)}[i]$. For a buyer with values $\vec{v}$, let $\left(\vec{\phi}_{\vec{v}}, p_{\vec{v}}\right) \in M$ be the lottery that maximizes the his expected utility and let $\vec{q} \sim \phi_{\vec{v}}$ denote the allocation. The expected profit is $\profit_{M} \left(\vec{v}\right) = p_{\vec{v}} - \E_{\vec{q} \sim \vec{\phi}_{\vec{v}}}\left[c\left(\vec{q}\right)\right]].$
The challenge in bounding the pseudo-dimension of the class $\mclass$ of these lotteries is that $\E_{\vec{q} \sim \vec{\phi}_{\vec{v}}}\left[c\left(\vec{q}\right)\right]$ is not piecewise linear in $\vec{\phi}^{\left(0\right)}, \dots, \vec{\phi}^{\left(\ell\right)}$. Instead, we bound the pseudo-dimension of a related class $\mclass'$ and show that optimizing over $\mclass'$ amounts to optimizing over $\mclass$ itself. To motivate $\mclass'$, note that if $\vec{z} \sim U\left([0,1]^m\right)$, then $\Pr_{\vec{z}}[z[j] \leq \phi_{\vec{v}}[j]] = \phi_{\vec{v}}[j]$, so $\E_{\vec{q} \sim \vec{\phi}_{\vec{v}}}\left[c\left(\vec{q}\right)\right] = \E_{\vec{z}}\left[c\left(\sum_{j: z[j] < \phi_{\vec{v}}[j]} \vec{e}_j\right)\right]$. For $M \in \mclass$, we define $\profit_{M} '\left(\vec{v}, \vec{z}\right) := p_{\vec{v}} - c\left(\sum_{j: z[j] < \phi_{\vec{v}}[j]} \vec{e}_j\right)$ and $\mclass' = \left\{\profit_M' : M \in \mclass\right\}$. The class $\mclass'$ is delineable: for any $\left(\vec{v}, \vec{z}\right)$, buyer's chosen lottery and the bundle $\sum_{j: z[j] < \phi_{\vec{v}}[j]} \vec{e}_j$ are determined by hyperplanes.

\begin{restatable}{lemma}{lottery}\label{lem:lottery}
The class $\mclass'$ is $\left(\ell\left(m+1\right), \left(\ell+1\right)^2 + m\ell\right)$-delineable.
\end{restatable}


The following lemma guarantees that optimizing over $\mclass'$ amounts to optimizing over $\mclass$ itself.

\begin{restatable}{lemma}{lotteryEquiv} \label{lem:lottery_equiv}
With probability $1-\delta$ over $\left(\vec{v}^{\left(1\right)}, \vec{z}^{\left(1\right)}\right), \dots, \left(\vec{v}^{\left(N\right)}, \vec{z}^{\left(N\right)}\right) \sim \dist \times U[0,1]^{m},$ for all $M \in \mclass$, \[\left|\frac{1}{N} \sum_{i = 1}^N \profit_{M} '\left(\vec{v}^{\left(i\right)}, \vec{z}^{\left(i\right)}\right) - \E_{\vec{v} \sim \dist}[\profit_M\left(\vec{v}\right)] \right| \leq 120U\sqrt{\frac{\pdim(\mclass')}{N}} + 4U \sqrt{\frac{2\ln(4/\delta)}{N}}.\]
\end{restatable}

This section demonstrates that a wide variety of mechanism classes $\cM$ are delineable. Therefore, Theorem~\ref{thm:main_pdim} immediately implies a generalization bound $\epsilon_{\cM}(N, \delta)$ for a diverse array of mechanisms.