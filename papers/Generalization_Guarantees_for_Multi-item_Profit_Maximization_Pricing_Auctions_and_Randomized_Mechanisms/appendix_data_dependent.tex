\begin{lemma}\label{lem:product}
Let $\domain = \domain_1 \times \cdots \times \domain_d$. Let $\fclass = \left\{f_{\vec{p}} : \vec{p} \in \pspace\right\}$ be a set of functions mapping $\domain$ to $\R$, parameterized by a set $\pspace = \pspace_1 \times \cdots \times \pspace_d$.
Suppose for $i \in [d]$, there exists a class $\fclass_i = \left\{f_p^{(i)} : p \in \pspace_i\right\}$ of functions mapping $\domain_i$ to $\R$ such that for any $\vec{p} = \left(p[1], \dots, p[d]\right) \in \pspace$, $f_{\vec{p}}$ decomposes additively as $f_{\vec{p}}\left(v_1, \dots, v_d\right) = \sum_{i = 1}^d f^{(i)}_{p[i]}\left(v_i\right)$. Then \[\sup_{\vec{v} \in \domain, \vec{p} \in \pspace} f_{\vec{p}}(\vec{v}) = \sum_{i = 1}^d \sup_{v \in \domain_i, p \in \pspace_i} f^{(i)}_{p}\left(v\right).\]
\end{lemma}

\begin{proof}
Recall that for any set $A \subseteq \R$, $s = \sup A$ if and only if:
\begin{enumerate}
\item For all $\epsilon > 0$, there exists $a \in A$ such that $a > s - \epsilon$, and
\item For all $a \in A$, $a \leq s$.
\end{enumerate}

Let $t_i = \sup_{v \in \domain_i, p \in \pspace_i} f^{(i)}_{p}\left(v\right)$ and let $t = \sum_{i=1}^d t_i$. We will show that $t = \sup_{\vec{v} \in \domain, \vec{p} \in \pspace} f_{\vec{p}}(\vec{v}).$

First, we will show that condition (1) holds. In particular, we want to show that for all $\epsilon > 0$, there exists $\vec{v} \in \domain$ and $\vec{p} \in \pspace$ such that $f_{\vec{p}}(\vec{v}) > t - \epsilon$. Since $t_i = \sup_{v \in \domain_i, p \in \pspace_i} f^{(i)}_{p}\left(v\right)$, we know that there exists $v_i \in \domain_i, p_i \in \pspace$ such that $ f^{(i)}_{p_i }\left(v_i\right) > t_i - \epsilon / d$. Therefore, letting $\vec{p} = \left(p_1, \dots, p_d\right)$, we know that $f_{\vec{p}}\left(v_1, \dots, v_d\right) = \sum_{i = 1}^d f^{(i)}_{p_i}\left(v_i\right) > \sum_{i = 1}^d t_i - \epsilon = t-\epsilon.$ Since $\left(v_1, \dots, v_d\right) \in \domain$ and $\left(p_1, \dots, p_d\right) \in \pspace$, we may conclude that condition (1) holds.

Next, we will show that condition (2) holds. In particular, we want to show that for all $\vec{v} \in \domain$ and $\vec{p} \in \pspace$, $f_{\vec{p}}(\vec{v}) \leq t$. We know that $f^{(i)}_{p[i]}\left(v[i]\right) \leq t_i$, which means that $f_{\vec{p}}(\vec{v}) = \sum_{i = 1}^d f^{(i)}_{p[i]}\left(v[i]\right) \leq \sum_{i = 1}^d t_i =t.$ Therefore, condition (2) holds.
\end{proof}

\secondPriceProduct*

\begin{proof}
We begin with anonymous second-price auctions, which are parameterized by a set $\pspace \subset \R^m$. Without loss of generality, we may write $\pspace = \pspace_1 \times \cdots \times \pspace_m$, where $\pspace_i \subset \R$. Given a valuation vector $\vec{v}$ and an item $i$, let $\vec{v}(i) \in \R^n$ be all $n$ buyers' values for item $i$. Let $\profit_p(\vec{v}(i))$ be the profit obtained by selling item $i$ with a reserve price of $p$. Notice that for any $\vec{p} \in \pspace$, $\profit_{\vec{p}}(\vec{v}) = \sum_{i = 1}^m \profit_{p[i]}(\vec{v}(i))$. Let $\domain_i$ be the support of the distribution over $\vec{v}(i)$ and let $U_i = \sup_{p \in \pspace_i, \vec{v}(i) \in \domain_i} \profit_{p}(\vec{v}(i))$. Next, let
$\domain$ be the support of $\dist$. By definition, since $U$ is the maximum profit achievable via second price auctions over valuation vectors from $\domain$, we may write $U = \sup_{\vec{v} \in \domain, \vec{p} \in \pspace}\profit_{\vec{p}}(\vec{v})$.  Since $\dist$ is item-independent, we know that $\domain = \domain_1 \times \cdots \times \domain_m$. Therefore, we may apply Lemma~\ref{lem:product}, which tells us that $U = \sum_{i = 1}^m U_i$. Finally, each class of functions $\left\{\profit_{p} : p \in \pspace_i\right\}$ is $(1, 2)$-delineable, since for $\vec{v}(i) \in \domain_i$, $\profit_{\vec{v}(i)}(p)$ is linear so long as $p$ is larger than the largest component of $\vec{v}(i)$, between the second largest and largest component of $\vec{v}(i)$, or smaller than the second largest component of $\vec{v}(i)$. By Corollary~\ref{cor:data}, we may conclude that for any set of samples $\sample \sim \dist^N$, $\erad(\mclass) \leq O\left(U\sqrt{1/N}\right)$.

The bound on $\erad(\mclass')$ follows by almost the exact same logic, except for a few adjustments. First of all, the class is defined by $nm$ parameters coming from some set $\pspace \subseteq \R^{nm}$, since there are $n$ non-anonymous prices per item. Without loss of generality, we assume $\pspace = \pspace_1 \times \cdots \times \pspace_m$, where $\pspace_i \subseteq \R^n$ is the set of non-anonymous prices for item $i$. Given a set of non-anonymous prices $\vec{p} \in \R^n$ for item $i$, let $\profit_{\vec{p}}(\vec{v}(i))$ be the profit of selling the item the bidders defined by $\vec{v}(i)$ given the reserve prices $\vec{p}$. Notice that $\profit_{\vec{v}(i)}(\vec{p})$ is linear so long as for each bidder $j$, $p[j]$ is either larger than their value for item $i$ or smaller than their value. Thus, the set $\left\{\profit_{\vec{p}} : \vec{p} \in \pspace_i\right\}$ is $(n,n)$-delineable. Defining each $U_i$ in the same way as before, Lemma~\ref{lem:product} guarantees that $U = \sum_{i = 1}^m U_i$. Therefore, by Corollary~\ref{cor:data}, we may conclude that for any set of samples $\sample \sim \dist^N$, $\erad(\mclass') \leq O\left(U\sqrt{n \log n/N}\right)$.
\end{proof}

\itemProduct*

\begin{proof}
We begin with anonymous item-pricing mechanisms, which are parameterized by a set $\pspace \subset \R^m$. Without loss of generality, we may write $\pspace = \pspace_1 \times \cdots \times \pspace_m$, where $\pspace_i \subset \R$. Given a valuation vector $\vec{v}$ and an item $i$, let $\vec{v}(i) \in \R^n$ be all $n$ buyers' values for item $i$. Let $\profit_p(\vec{v}(i))$ be the profit obtained by selling item $i$ at a price of $p$, i.e., $\profit_p(\vec{v}(i)) = \textbf{1}_{\left\{||\vec{v}(i)||_{\infty} \geq p\right\}} (p - c(\vec{e}_i))$. Notice that for any $\vec{p} \in \pspace$, $\profit_{\vec{p}}(\vec{v}) = \sum_{i = 1}^m \profit_{p[i]}(\vec{v}(i))$. Let $\domain_i$ be the support of the distribution over $\vec{v}(i)$ and let $U_i = \sup_{p \in \pspace_i, \vec{v}(i) \in \domain_i} \profit_{p}(\vec{v}(i))$. Next, let
$\domain$ be the support of $\dist$. By definition, since $U$ is the maximum profit achievable via item-pricing mechanisms over valuation vectors from $\domain$, we may write $U = \sup_{\vec{v} \in \domain, \vec{p} \in \pspace}\profit_{\vec{p}}(\vec{v})$.  Since $\dist$ is item-independent, we know that $\domain = \domain_1 \times \cdots \times \domain_m$. Therefore, we may apply Lemma~\ref{lem:product}, which tells us that $U = \sum_{i = 1}^m U_i$. Finally, each class of functions $\left\{\profit_{p} : p \in \pspace_i\right\}$ is $(1, 1)$-delineable, since for $\vec{v}(i) \in \domain_i$, $\profit_{\vec{v}(i)}(p)$ is linear so long as $||\vec{v}(i)||_{\infty} \leq p$ or $||\vec{v}(i)||_{\infty} > p$. By Corollary~\ref{cor:data}, we may conclude that for any set of samples $\sample \sim \dist^N$, $\erad(\mclass) \leq O\left(U\sqrt{1/N}\right)$.

The bound on $\erad(\mclass')$ follows by almost the exact same logic, except for a few adjustments. First of all, the class is defined by $nm$ parameters coming from some set $\pspace \subseteq \R^{nm}$, since there are $n$ non-anonymous prices per item. Without loss of generality, we assume $\pspace = \pspace_1 \times \cdots \times \pspace_m$, where $\pspace_i \subseteq \R^n$ is the set of non-anonymous prices for item $i$. Given a set of non-anonymous prices $\vec{p} \in \R^n$ for item $i$, let $\profit_{\vec{p}}(\vec{v}(i))$ be the profit of selling the item to the buyers defined by $\vec{v}(i)$ given the prices $\vec{p}$. Notice that $\profit_{\vec{v}(i)}(\vec{p})$ is linear so long as for each buyer $j$, $p(\vec{e}_j)$ is either larger than their value for item $i$ or smaller than their value. Thus, the set $\left\{\profit_{\vec{p}} : \vec{p} \in \pspace_i\right\}$ is $(n,n)$-delineable. Defining each $U_i$ in the same way as before, Lemma~\ref{lem:product} in Appendix~\ref{APP:DATA} guarantees that $U = \sum_{i = 1}^m U_i$. Therefore, by Corollary~\ref{cor:data}, we may conclude that for any set of samples $\sample \sim \dist^N$, $\erad(\mclass') \leq O\left(U\sqrt{n \log n/N}\right)$.
\end{proof}

\medskip\emph{Menus of item lotteries.} A \emph{length-$\ell$ item lottery menu} is a set of $\ell$ lotteries per item. The menu for item $i$ is $M_i = \left\{\left(\phi_i^{\left(0\right)}, p_i^{\left(0\right)}\right), \left(\phi_i^{\left(1\right)}, p_i^{\left(1\right)}\right), \dots, \left(\phi_i^{\left(\ell\right)}, p_i^{\left(\ell\right)}\right)\right\}$, where $\phi_i^{\left(0\right)}= p_i^{\left(0\right)}= 0$. The buyer chooses a lottery $\left(\phi_i^{\left(j_i\right)}, p_i^{\left(j_i\right)}\right)$ per menu $M_i$, pays $\sum_{i = 1}^m p^{\left(j_i\right)}$, and receives each item $i$ with probability $\phi_i^{\left(j_i\right)}$. 

\begin{restatable}{lemma}{lotteryProduct}\label{lem:lottery_product}
	Let $\mclass$ be the set of length-$\ell$ item lottery menus. If the buyer is additive, $\dist$ is item-independent, and the cost function is additive, then for any set $\sample \sim \dist^N$, $\erad\left(\mclass\right) \leq 180\sqrt{2\ell \log\left(8\ell^3\right)}$.
\end{restatable}

\begin{proof}
For a given menu $M = \left(M_1, \dots, M_m\right)$ of item lotteries, let $\profit_{M_i}(\vec{v})$ be the profit achieved from menu $M_i$. Since the cost function is additive, \[\profit_{M_i} (\vec{v}) = p_{i,\vec{v}} - \E_{q \sim \phi_{i, \vec{v}}}\left[c(q)\right] = p_{i,\vec{v}} - c(\vec{e}_i)\cdot \phi_{i, \vec{v}},\] where $(p_{i, \vec{v}}, \phi_{i, \vec{v}})$ is the lottery in $M_i$ that maximizes the buyer's utility.
Notice that $\profit_{M}(\vec{v}) = \sum_{i = 1}^m \profit_{M_i}(v(\vec{e}_i))$. Let $\domain_i$ be the support of the distribution $\dist_i$ over $v(\vec{e}_i)$ and let $U_i = \sup_{M_i, v(\vec{e}_i) \in \domain_i} \profit_{M_i}(v(\vec{e}_i))$. By definition, since $U$ is the maximum profit achievable via item menus over valuation vectors from $\domain$, we may write $U = \sup_{\vec{v} \in \domain, M \in \mclass}\profit_{M}(\vec{v})$.  Since $\dist$ is a product distribution, we know that $\domain = \domain_1 \times \cdots \times \domain_m$. Therefore, we may apply Lemma~\ref{lem:product}, which tells us that $U = \sum_{i = 1}^m U_i$. Finally, for each $i \in [n]$, the class of all single-item lotteries $M_i$ is $(2\ell, \ell^2)$-delineable, since for $v(\vec{e}_i) \in \domain_i$, the lottery the buyer chooses depends on the ${\ell + 1 \choose 2}$ hyperplanes $\phi_i^{(j)}v(\vec{e}_i) - p_i^{(j)} = \phi_i^{(j')}v(\vec{e}_i) - p_i^{(j')}$ for $j, j' \in \{0, \dots, \ell\}$, and once the lottery is fixed, $\profit_{M_i} (\vec{v})$ is a linear function.
\end{proof}

\pdimLower*

\begin{proof}
Let $\pazocal{M}$ be the class of item-pricing mechanisms with anonymous prices. We construct a set $\sample$ of $m$ single-buyer, $m$-item valuation vectors that can be shattered by $\pazocal{M}$. Let $\vec{v}^{(i)}$ be valuation vector where $v_1^{(i)}(\vec{e}_i) = 3$ and $v_1^{(i)}(\vec{e}_j) = 0$ for all $j \not= i$ and let $\sample = \left\{\vec{v}^{(1)}, \dots, \vec{v}^{(m)}\right\}$. For any $T \subseteq [m]$, let $M_T$ be the mechanism defined such that the price of item $i$ is 2 if $i \in T$ and otherwise, its price is 0. If $i \in T$, then $\profit_{M_T}(\vec{v}^{(i)}) = 2$ and otherwise, $\profit_{M_T}(\vec{v}^{(i)}) = 0$. Therefore, the targets $z^{(1)} = \dots = z^{(m)} = 1$ witness the shattering of $\sample$ by $\pazocal{M}$. This example also proves that the pseudo-dimension of the class of second-price auctions with anonymous reserve prices is also at least $m$, since in the single-buyer case, this class is identical to $\pazocal{M}$.

Next, let $\pazocal{M}'$ be the class of item-pricing mechanisms with non-anonymous prices. We construct a set $\sample$ of $nm$ $n$-buyer, $m$-item valuation vectors that can be shattered by $\pazocal{M}'$. For $i \in [m]$ and $j \in [n]$, let $\vec{v}^{(i,j)}$ be valuation vector where $v_j^{(i,j)}(\vec{e}_i) = 3$ and $v_{j'}^{(i,j)}(\vec{e}_{i'}) = 0$ for all $(i', j') \not= (i,j)$. Let $\sample = \left\{\vec{v}^{(i,j)}\right\}_{i \in [m], j \in [n]}$. For any $T \subseteq [m] \times [n]$, let $M_T$ be the mechanism defined such that the price of item $i$ for buyer $j$ is 2 if $(i,j) \in T$ and otherwise, it is 0. If $(i,j) \in T$, then $\profit_{M_T}(\vec{v}^{(i,j)}) = 2$ and otherwise, $\profit_{M_T}(\vec{v}^{(i,j)}) = 0$. Therefore, the targets $z^{(i,j)} = 1$ for all $i\in [m], j \in [n]$ witness the shattering of $\sample$ by $\pazocal{M}$. This example with the prices as reserve prices also proves that the pseudo-dimension of the class of second-price auctions with non-anonymous reserve prices is at least $nm$.
\end{proof}