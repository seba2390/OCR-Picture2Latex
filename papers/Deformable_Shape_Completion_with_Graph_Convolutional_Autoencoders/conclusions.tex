\section{Conclusions and future work}
This paper introduces a novel graph-convolutional method for shape completion. Its important properties include a model robust to non-rigid deformations, small sample complexity when training, and the ability to reconstruct any style of missing data. Evaluations indicate this is a promising first step towards shape completion from real-world scans, and the analysis reveals directions for future work.
Firstly, exploring a representation that disentangles shape and pose would allow for more control in the completion and likely improve dynamic fusion results.
Secondly, for initialization we require correspondences between the partial and canonical shape model. Although we show resilience to poor correspondences, improving this initialization for noisy real-world data would be beneficial.
Finally, the proposed formulation assumes the desired shape topology (i.e. vertex connectivity) is known when decoding shapes. We leave to future work the task of completion with unknown topology.
