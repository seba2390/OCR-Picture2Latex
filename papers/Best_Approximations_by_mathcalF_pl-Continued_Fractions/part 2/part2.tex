\documentclass[12pt]{elsarticle}
%\documentclass[12pt,reqno]{elsarticle}
\usepackage{amsfonts,amsmath,oldgerm,amssymb,amscd}
\usepackage{lineno}
%\linenumbers
%\usepackage{blindtext}
%\pagewiselinenumbers

\usepackage{amsthm}
%\newtheorem{defi}{Definition}
%\newtheorem{notation}{Notation}
\newtheorem{theorem}{Theorem}[section]
\newtheorem{lemma}[theorem]{Lemma}
\newtheorem{corollary}[theorem]{Corollary}
\newtheorem{proposition}[theorem]{Proposition}
%\usepackage{chngcntr}
%\usepackage{apptools}
%\AtAppendix{\counterwithin{}{section}}

\usepackage{chngcntr}
\newdefinition{remark}{Remark}
\newproof{pf}{Proof}
\theoremstyle{definition}
\newtheorem{defi}{Definition}
\newtheorem{notation}[theorem]{Notation}
\newtheorem{example}{Example}
%\numberwithin{equation}{section}
\usepackage{mathtools}
\usepackage{amsfonts}
\newcommand{\field}[1]{\mathbb{#1}}          \newcommand{\Q}{\field{Q}}
\newcommand{\N}{\field{N}}
\newcommand{\R}{\field{R}}                   \newcommand{\Z}{\field{Z}}
\newcommand{\C}{\field{C}}                   \newcommand{\A}{\field{A}}
\newcommand{\F}{\field{F}}
\newcommand{\h}{\field{H}}
\newcommand{\im}{\rm Im}
\newcommand{\ord}{{\rm ord}}
\newcommand{\mA}{{\mathcal A}}
\newcommand{\mB}{{\mathcal B}}
\newcommand{\mC}{{\mathcal C}}
\newcommand{\mF}{{\mathcal F}}
\newcommand{\mX}{{\mathcal X}}
\newcommand{\mXe}{{\mathcal X}_{2^n}}
\newcommand{\mXp}{{\mathcal X}_{p^l}}
\newcommand{\mY}{{\mathcal Y}}
\newcommand{\mZ}{{\mathcal Z}}
\newcommand{\mO}{{\mathcal O}}
\newcommand{\f}{{\mathcal F}}
\newcommand{\fpn}{{\mathcal F}_{p^l}}
\newcommand{\m}{{\mathcal {M}_{n}}}
\newcommand{\mf}{\mathfrak}
\newcommand{\us}{\underset}
\newcommand{\os}{\overset}
\newcommand{\ep}{\epsilon}
\newcommand{\card}{{\rm card}}
\newcommand{\Span}{{\textnormal{span}}}
\newcommand{\id}{{\textnormal{\bf Id}}}
\newcommand{\rank}{{\rm rank}}
\newcommand{\Lie}{{\rm Lie}}
\newcommand{\U}{{\rm U}}
\newcommand{\M}{{\rm M}}
\newcommand{\aP}{{\rm P}}
\newcommand{\nullity}{{\rm Nullity}}
\renewcommand{\a}{\alpha}                      \renewcommand{\b}{\beta}
\renewcommand{\d}{\delta}                       \newcommand{\D}{\Delta}
\newcommand{\e}{\varepsilon}                    \newcommand{\g}{\gamma}
\newcommand{\G}{{\rm G}}                      \renewcommand{\l}{\lambda}
\renewcommand{\L}{\Lambda}
\newcommand{\ol}{\overline}
\newcommand{\n}{\nabla}
\newcommand{\var}{\varphi}                      \newcommand{\s}{\sigma}
\newcommand{\Sig}{\Sigma} \renewcommand{\t}{\tau}
\renewcommand{\th}{\theta}
\renewcommand{\dim}{\textnormal{dim}}
\renewcommand{\ker}{\textnormal{ker}}
\newcommand{\sign}{\textnormal{sign}}
\renewcommand{\o}{\omega} \newcommand{\z}{\zeta}
\newcommand{\fg}{\mathfrak     g}     \newcommand{\fp}{\mathfrak    p}
\newcommand{\fk}{\mathfrak     k}     \newcommand{\fh}{\mathfrak    h}
\newcommand{\fu}{\mathfrak u} \newcommand {\ft}{\mathfrak t}

\newcommand{\bs}{\backslash} \newcommand{\ra}{\rightarrow}

\textwidth 6.2 in
\textheight 8.5 in
\topmargin -1 cm
%\leftmargin -0.5 cm
%\rightmargin 1 cm

\clearpage
\thispagestyle{empty}
\newtheorem{ab}{}%[chapter]

\begin{document}
	\title{ Best Approximations by $\mathcal{F}_{p^l}$-Continued Fractions} 
	
	\author[sk]{S. Kushwaha}\corref{cor1}  
	\ead{seema28k@gmail.com}
	\author[rvt]{R. Sarma} 
	\ead{ritumoni@maths.iitd.ac.in}

	
	\cortext[cor1]{Corresponding author}
	\address[sk]{Department of Applied Sciences, Indian Institute of Information Technology Allahabad, Prayagraj, India}
	\address[rvt]{Department of Mathematics,
		Indian Institute of Technology Delhi, India}

	\begin{abstract}
		
		In this article, for a certain subset $\mathcal{X}$ of the extended set of rational numbers, we introduce the notion of {\it best $\mathcal{X}$-approximations} of a real number.  The notion of best $\mathcal{X}$-approximation  is analogous to that of best rational approximation.   We explore these approximations with the help of $\mathcal{F}_{p^l}$-continued fractions, where $p$ is a prime and $l\in\mathbb{N}$, we show that the convergents of the $\mathcal{F}_{p^l}$-continued fraction expansion  of a real number $x$ satisfying certain maximal conditions are exactly the best $\mathcal{F}_{p^l}$-approximations of $x$.
	
	\end{abstract}
	\maketitle	
	\section{Introduction}


Let $p$ be a prime and $l\in \N.$	A finite continued fraction of the form $$\frac{1}{0+}~\frac{{p^l}}{b+}~\frac{\epsilon_{1} }{a_{1}+}~\frac{\epsilon_{2}}{a_{2}+}~\cdots\frac{\epsilon_{n}}{a_{n}}~~(n\ge 0)$$
or
an infinite continued fraction of the form
$$\frac{1}{0+}~\frac{{p^l}}{b+}~\frac{\epsilon_{1} }{a_{1}+}~\frac{\epsilon_{2}}{a_{2}+}~\cdots\frac{\epsilon_{n}}{a_{n}+}\cdots$$ 
is called an $\mF_{{p^l}}$-{\it continued fraction},
 where $b$ is an integer co-prime to $p$ and for $i\ge1$,  $a_i\in\N$ and  $\ep_i\in\{\pm1\}$  with certain conditions.  A precise definition of an $\f_{{p^l}}$-continued fraction is stated in Section 3 introduced by Kushwaha et al. \cite{seemafnpart1}.
  In fact, this family of continued fractions  arises from a family of graphs $\f_{{p^l}}$ which are similar to the Farey graph. The value of a finite $\f_{p^l}$-continued fraction is a member of the set 
 \begin{equation}\label{X_n}
 		\mathcal{X}_{p^l}=\left\{\frac{x}{y}:~x,y\in\mathbb{Z},~ y>0,~\mathrm{gcd}(x,y)=1~\textnormal{and}~{p^l}|y\right\}\cup\{\infty\}
\end{equation}   
 which is the vertex set of $\f_{p^l}$.
 The important fact is that every real number has an $\f_{p^l}$-continued fraction expansion.

 An element  $u/v$ of $\mX_{p^l}$ is called a \textit{best $\mX_{{p^l}}$-approximation} of $x\in\R$, if for every $u'/v'\in\mX_{{p^l}}$ different from $u/v$ with	$0< v' \le v$, we have $|vx-u|<|v'x-u'|$.  

These approximations have been discussed in \cite{seema2,seema} for $p^l=3,2$ respectively. In these papers, authors have achieved results analogous to the classical one, that is,  convergents of the continued fraction of a real number characterize the best approximations of the real number.

 In several recent papers \cite{lucax_tribnocci1,lucax_twofibonacci,primepowers,rep}, the following problem was investigated. Let $U$
be some interesting set of positive integers. What can one say about the
square-free integers $d$ such that the  first (or, the second) coordinate $X\,(\textnormal{respectively, }Y)$ of a solution to the Pell equation $X^2-dY^2=1$ is a member of the set $U$. The first author has applied best $\mX_{2^l}$-approximations to solve certain conditional Pell equations \cite{seema_pell} which is a special case of the above mentioned problem. We strongly believe that a complete generalization of work in \cite{seema2,seema} will be very useful and in this article, we deal with the relation of best $\mX_{p^l}$-approximations and $\f_{p^l}$-convergents.

Note that a real number may have more than one $\f_{p^l}$-continued fraction expansions; this fact was observed in \cite{seema2} for $p^l=3$. 
For instance, the set of $\f_5$-continued fraction expansions of $11/40$ is as follows
\begin{align*}
&	\Big\{\frac{1}{0+}~\frac{5}{1+}~\frac{1}{2+}~\frac{1}{1+}~\frac{1}{1+}~\frac{1}{1},~ \frac{1}{0+}~\frac{5}{1+}~\frac{1}{2+}~\frac{1}{2+}~\frac{-1}{2},~\\
&\frac{1}{0+}~\frac{5}{1+}~\frac{1}{2+}~\frac{1}{1+}~\frac{1}{2},
\frac{1}{0+}~\frac{5}{1+}~\frac{1}{3+}~\frac{-1}{2+}~\frac{1}{1},~\frac{1}{0+}~\frac{5}{1+}~\frac{1}{3+}~\frac{-1}{3},~\\ \tag*{(2)}\label{manyexpansions}
&\frac{1}{0+}~\frac{5}{2+}~\frac{-1}{2+}~\frac{-1}{2+}~\frac{1}{2}, \frac{1}{0+}~\frac{5}{2+}~\frac{-1}{2+}~\frac{-1}{3+}~\frac{-1}{2},
\frac{1}{0+}~\frac{5}{2+}~\frac{-1}{2+}~\frac{-1}{2+}~\frac{1}{1+}~\frac{1}{1}
\Big\}.
\end{align*}
When $p=3$, the longest $\f_{p^l}$-continued fraction is unique (see \cite{seema2}) and this was helpful to achieve the approximation results. But this is false for $p\ge5.$ In the above example,  $ \frac{1}{0+}~\frac{5}{1+}~\frac{1}{2+}~\frac{1}{1+}~\frac{1}{1+}~\frac{1}{1}$ and $	\frac{1}{0+}~\frac{5}{2+}~\frac{-1}{2+}~\frac{-1}{2+}~\frac{1}{1+}~\frac{1}{1} $ are both longest $\f_5$-continued fractions of 11/40.  Besides, not every longest $\f_{p^l}$-continued fraction is helpful to describe best $\mX_{p^l}$-approximations. In fact, best approximations are described by $\f_{p^l}$-continued fractions with maximum $+1$ (see Theorem 4.9 and 4.10).

	Finally, we show that for any real number $x$ which is not in $\Q\setminus\mX_{p^l},$ every convergent of the $\f_{p^l}$-continued fraction of $x$ with maximum $+1$ is a best $\mX_{p^l}$-approximation of $x$  and conversely. 
	%In the last section of this article, namely, Appendix A, we describe  a few terminologies and discuss a few diagrams which will be helpful to follow our main results.
	
	\section{Preliminaries}

In this section, we recall certain definitions and results on $\f_{p^l}$-continued fractions from \cite{seemafnpart1}. Now onwards, $N$ denotes a positive integer of the form $p^l,$ where $p$ is a prime and $l$ is a natural number.
\begin{defi}
Given $N=p^l,$	a finite continued fraction of the form $$\frac{1}{0+}~\frac{N}{b+}~\frac{\epsilon_{1} }{a_{1}+}~\frac{\epsilon_{2}}{a_{2}+}~\cdots\frac{\epsilon_{n}}{a_{n}}~~(n\ge 0)$$
	or
	an infinite continued fraction of the form
	$$\frac{1}{0+}~\frac{N}{b+}~\frac{\epsilon_{1} }{a_{1}+}~\frac{\epsilon_{2}}{a_{2}+}~\cdots\frac{\epsilon_{n}}{a_{n}+}\cdots$$ 
	is called an $\mF_{N}$-{\it continued fraction} if $b$ is an integer co-prime to $N$, and for $i\ge1$, $a_i\in\N$ and $\ep_i\in\{\pm 1\}$ such that the following conditions hold:
	\begin{enumerate}
		\item $a_i+\ep_{i+1}\ge1$;
		\item $a_i+\ep_i\ge1$;
		\item $\mathrm{gcd}(p_i,q_i)=1,$ where $p_i=a_i p_{i-1}+\ep_i p_{i-2}$, $q_i=a_i q_{i-1}+\ep_i q_{i-2}$, $(p_{-1},q_{-1})=(1,0)$ and $(p_0,q_0)=(b,N)$. 
	\end{enumerate}  
\end{defi} 
	For $i\ge1,$ the value $p_i/q_i$ of the the expression
	$$\frac{1}{0+}~\frac{N}{b+}~\frac{\epsilon_{1} }{a_{1}+}~\frac{\epsilon_{2}}{a_{2}+}~\cdots\frac{\epsilon_{i}}{a_{i}}$$
	is called the {\it $i$-th $\f_N$-convergent} of the continued fraction. The sequence $\{\frac{p_i}{q_i}\}_{i\ge0}$ is called the {\it sequence of $\f_N$-convergents}. The expression 
	$\frac{\epsilon_{i}}{a_{i}+}~\frac{\epsilon_{i+1}}{a_{i+1}+}\cdots $ is called the {\it $i$-th fin}. Let $y_i$ denote the $i$-th fin, that is,  $y_i=\frac{\epsilon_{i}}{a_{i}+}~\frac{\epsilon_{i+1}}{a_{i+1}+}\cdots $, then $\ep_{i}=\sign(y_{i})$.
	 The following theorem assets certain properties of $\f_N$-continued fractions. 
	\begin{theorem}\cite[Theorem 3.2]{seemafnpart1}\label{distinctconvergents}
		Suppose $x= \frac{1}{0+}~\frac{N}{b+}~ \frac{\ep_1}{a_1 +}~ \frac{\ep_2}{a_2 +}~ \frac{\ep_3}{a_3 +}~\cdots
		$ is an $\f_N$-continued fraction with the sequence of convergents $\{p_i/q_i\}_{i\ge -1}$. For $i\ge1$, let  
		$y_i$ be the $i$-th \it{fin} of the continued fraction.  Then
		\begin{enumerate}
			\item 	for  $i\ge1,$ $a_ip_{i-1}\not \equiv -\ep_{i}p_{i-2}\mod p$;
			\item 	the sequence $\{q_i\}_{i\ge -1}$ is strictly increasing;
			\item $\dfrac{p_i}{q_i}\ne \dfrac{p_j}{q_j}$ for $i\ne j$;
			\item for $i\ge1$, $|y_i|\le1$;
			\item for $i\ge0,$ $x=\dfrac{x_{i+1}p_i+\ep_{i+1}p_{i-1}}{x_{i+1}q_i+\ep_{i+1}q_{i-1}},$ where $x_{i+1}=\dfrac{1}{|y_{i+1}|}.$
		\end{enumerate}
	\end{theorem}
	
	 An $\f_N$-continued fractions is arising from the graph $\f_N,$ where
 the  vertex set is  $\mathcal{X}_N$ (as defined in Equation \eqref{X_n})  and vertices ${p}/{q}$ and ${r}/{s}$, are adjacent in $\mathcal{F}_N$ if and only if  $$rq-sp=\pm N.$$ 
		If $P$ and $Q$ are adjacent in $\f_N$ we write $P\sim_N Q$. Note that the graph $\f_{1}$ is the Farey graph 
and for every $N\in\N$, the graph $\f_N$ is isomorphic to a subgraph of the Farey graph.


	 \begin{figure}[!h]
	 	%	\begin{minipage}{.9\textwidth}
	 	\centering
	 	\includegraphics[width=.8\linewidth]{f1,1.eps}
	 	\caption{A few vertices and edges of the Farey graph in [-1,1]}\label{Fig:farey}
	 	%	\end{minipage}\hfill
	 \end{figure}
	\noindent Edges of $\f_{N}$ are represented as hyperbolic geodesics in the upper-half plane $$\mathcal{U}=\{z\in\C: \im(z)>0\},$$
	that is, as Euclidean semicircles or half lines perpendicular to the real line. Figure 1 is a display of a few edges of the Farey graph in the interval [-1,1].
Since edges of the Farey graph do not cross each other, and $\f_N$ is embedded in the Farey graph, we have the following result.
	
	\begin{proposition}\cite[Corollary 2.2]{seemafnpart1}\label{nocrossing}
		No two edges cross in $\f_N$.
	\end{proposition}
%	\begin{theorem}\label{connectednessthm} If $N=p^l$, where $p$ is a prime and $l\in\N\cup\{0\},$ the graph $\f_N$ is connected. 
%	\end{theorem}
%	\begin{proof}
%		For $N=1$, $\f_N$ is  the Farey graph, which is connected. Suppose $N=p^l,$ $l\in\N.$ The proof is by induction on the value of the denominator of the given vertex. Suppose $x/(p^ly)$ is a vertex in $\f_{p^l}$. If $y=1$ then  $\infty \sim_{p^l}x/(p^ly)$ as $p^ly-0x=p^l$. Now assume that the result holds true for every vertex in $\f_{p^l}$ with the denominator less than $p^ly$ and there is a path from $\infty$ to each such vertex. Since $\mathrm{gcd}(x,y)=1$, there exist $r,s\in\Z$ with $ry-sx=1$. If $s\ge y$, replace $r$ and $s$ by $r+kx$ and $s+ky$ respectively, for a suitable value of $k\in\Z$ such that $0<s<y$ as well as $ry-sx=1$. Observe that either $\frac{r}{p^ls}$ or $\frac{x-r}{p^l(y-s)}$ is a vertex in $\f_{p^l}$ which is  adjacent to $x/(p^ly)$ having a smaller denominator and the result follows.
%	  \end{proof}
%	
%	\begin{remark}\label{increasingpath}In the proof of Theorem \ref{connectednessthm}, we can see that there is a path from $\infty$ to $P\in\mX_{p^l}$ given by
%		$$\infty\to P_0\to P_1\to\cdots\to P_n$$
%	with $q_{i-1}<q_{i},$	where $P_i=p_i/q_i\in\mX_{p^l}$  for each $0\le i\le n$ and $P=P_n.$
%	\end{remark}
%	
%\begin{remark}\label{psl_action}
%
%Suppose $a/(bp^l)$ and  $c/(dp^l)$ are two adjacent vertices in $\f_{p^l}.$ 
%We claim that there exists $\gamma\in \textnormal{PSL}(2,\Z)$  such that $\gamma(\frac{a}{bp^l})=1/0$ and $\gamma(\frac{c}{dp^l})=m/p^l$ for some integer $m$ co-prime to $p.$ Since $\mathrm{gcd}(a,bp^l)=1,$ there exist  integers $A$ and $B$ such that $Aa+Bbp^l=1$. Set $\gamma=\begin{pmatrix}
%A & B\\
%-bp^l& a
%\end{pmatrix}\in\textnormal{PSL}(2,\Z)$, then
%$\gamma(\frac{a}{p^lb})=1/0 \textnormal{ and } \gamma(\frac{c}{p^ld})=m/p^l,$
%where $m=Ac+Bdp^l$. 
%In fact, $\gamma(\mX_{p^l})=\mX_{p^l}$ and $P\sim_{p^l} Q \Rightarrow \gamma(P)\sim_{p^l} \gamma(Q).$
%Therefore, a path 
%	$$P_0\to P_1\to P_2\to\cdots\to P_n,$$
%	in $\f_{p^l}$ can be transformed to a path 
%	$$\infty=P'_0\to m/p^l=P'_1\to P'_2\cdots\to P'_n$$
%	by a suitable element $\gamma$ of $\textnormal{PSL}(2,\Z).$
%\end{remark}	
%
%	\begin{theorem} For every $l\in\N,$ the graph 
%		$\mathcal{F}_{2^l}$ is a tree whereas  $\f_{p^l}$ is not a tree for each odd prime $p$.
% 	\end{theorem}
%	\begin{proof}
%		To show that $\f_{2^l}$ is a tree, we have to show that there is no circuit in $\mathcal{F}_{2^l}$. We know that a vertex adjacent to $\infty$ is of the form $b/2^l$ for some odd integer $b.$ Thus by Remark \ref{psl_action}, without loss of generality, assume that 
%		$$\infty\to P_0=\frac{m}{2^l}\to P_1\to\cdots\to P_n=\frac{m+2}{2^l}\to \infty$$ 
%		is a circuit of minimal length in $\mathcal{F}_{2^l}$. We have $
%		m/2^l<(m+1)/2^{l}<(m+2)/2^l.$
%		By Corollary \ref{nocrossing}, two edges do not cross in $\mathcal{F}_{2^l}$ and so there exists a positive integer $i$ such that $$P_i<(m+1)/2^{l}<P_{i+1}.$$
%	Observe that 
%	$m\sim_1 (m+1)$ and $ 2^l P_i\sim_1 2^l P_{i+1}$ in the Farey graph with $$m<2^lP_i<(m+1)<2^lP_{i+1}.$$
%	Hence, we get a contradiction as no two edges cross in the Farey graph and the result follows. Now suppose $p$ is an odd prime. Then
%		$\infty\to \frac{1}{p^l}\to\frac{2}{p^l}\to\infty$
%		forms a circuit in $\f_{p^l}$ and hence $\f_{p^l}$ is not a tree.
%	\end{proof}	
%
%	\begin{remark}\label{existence_m,n}
%		Suppose $N\in\N.$	Let $p,q$ be two distinct primes dividing $N.$ Then there exist two consecutive integers $0<A,B<N$ such that  $A$ and $B$ are not co-prime to $N.$
%	\end{remark}
%
%	\begin{proof}
%		Since $p$ and $q$ are distinct primes, there exist two positive integers $m,n$ such that
%		$$mp-nq=\pm1.$$  If $n\ge p,$ we  replace $m,n$ by $m-kq,n-kp$ respectively,
%		where $k$ is an integer such that $0<n-kp<p.$ Then $0<(m-kq)p<N$. Set $A=(m-kq)p$ and $B=(n-kp)q,$ then  $A$ and $B$ are consecutive positive integers less than $N$ which are not co-prime to $N$.
%	\end{proof}
%	
%	
%	\begin{theorem}
%		If $N\in\N$ has (at least) two distinct prime divisors then $\f_N$ is disconnected. 
%	\end{theorem}
%	\begin{proof}Let $p$ and $q$ be two distinct primes dividing $N.$ By Remark \ref{existence_m,n}, there exist two consecutive integers $0<A,B<N$ such that  $A$ and $B$ are not co-prime to $N.$ Without loss of generality, we assume that $A<B.$ Note that $\mathrm{gcd}(A,N)\ne1$ and $\mathrm{gcd}(B,N)\ne1$ so that $A/N,~B/N\not\in\mX_N.$ Suppose $x\in\mX_N$ is such that $A/N<x<B/N.$ If the graph is connected, there is a path from $\infty$ to $x$ 
%		$$\infty\to P_0\to P_1\to \cdots \to P_r,$$ for some natural number $r$ and $x=P_r.$ Without loss of generality, we may assume that $(A-1)/N\in \mX_N.$ Then there exists a positive integer $k<r$ such that $P_k<A/N<P_{k+1}.$ Thus, in the Farey graph, $NP_k\sim_1 N P_{k+1}$ and $(A-1)<NP_k<A<NP_{k+1}$. This is a contradiction as $(A-1)$ is adjacent to $A$ in the Farey graph so that the edge $(A-1)\sim_1 A$ crosses the edge $NP_k\sim_1 N P_{k+1}$ in the Farey graph. Thus, the result follows.
%		\end{proof}

	



%For $i\ge1,$ the value $p_i/q_i$ of the the expression
%	$$\frac{1}{0+}~\frac{N}{b+}~\frac{\epsilon_{1} }{a_{1}+}~\frac{\epsilon_{2}}{a_{2}+}~\cdots\frac{\epsilon_{i}}{a_{i}}$$
% is called the {\it $i$-th $\f_N$-convergent} of the continued fraction. The sequence $\{\frac{p_i}{q_i}\}_{i\ge0}$ is called the {\it sequence of $\f_N$-convergents}. The expression 
%	 $\frac{\epsilon_{i}}{a_{i}+}~\frac{\epsilon_{i+1}}{a_{i+1}+}\cdots $ is called the {\it $i$-th fin}. Let $y_i$ denote the $i$-th fin, that is,  $y_i=\frac{\epsilon_{i}}{a_{i}+}~\frac{\epsilon_{i+1}}{a_{i+1}+}\cdots $, then $\ep_{i}=\sign(y_{i})$.
%	\begin{remark}Condition 3 in Definition 2 guarantees that the $i$-th $\f_N$-convergent $p_i/q_i$  is an element of $\mX_N$ for each $i\ge0.$ Note  that $\mathrm{gcd}(p_i,q_i)=1$ if and only if $\mathrm{gcd}(p_i,N)=1$. 
%	\end{remark} 
%	
%	For $N=2,3$, the above continued fractions have been studied in \cite{seema,seema2} and they are referred to as $\mathcal{F}_{1,2}$-continued fraction and $\mathcal{F}_{1,3}$-continued fraction, respectively. An $\f_N$-continued fraction is closely related to a semi-regular continued fraction. A semi-regular continued fraction, when it is finite, is expressed as
%$$ a_0 + \frac{\epsilon_1}{a_1 +}~ \frac{\epsilon_2}{a_2 +}~
%\frac{\ep_3}{a_3 +}~ \cdots\frac{\ep_n}{a_n  }$$
%and when infinite,  as
%$$ a_0 + \frac{\epsilon_1}{a_1 +}~ \frac{\epsilon_2}{a_2 +}~
%\frac{\ep_3}{a_3 +}~ \cdots\frac{\ep_n}{a_n+}\cdots,$$  where $a_0\in\mathbb{Z}$, $\epsilon_i\in\{\pm1\}$,  and $a_i\in\mathbb{N}$ with $a_i+\ep_{i+1}\geq1$  for $i\geq1$.
%
%	The following result about semi-regular continued fractions is well known.  For more details on semi-regular continued fractions, see \cite{kraai,perron,seema_conversion}.
%	\begin{proposition}\label{relationwithttail}
%		Suppose $x$ and $y_n$ $(n\ge1)$ are real numbers such that for every
%		$n\ge1$,
%		$$a_0 + \frac{\ep_1}{a_1 +}~ \frac{\ep_2}{a_2 +}~ \frac{\ep_3}{a_3 +}~\cdots
%		\frac{\ep_n}{a_n +y_{n+1} }$$
%		is a semi-regular continued fraction having value $x$.
%		\begin{enumerate}
%	\item If $y_n= \frac{\ep_{n}}{a_{n} +}~ \frac{\ep_{n+1}}{a_{n+1} +}~\cdots
%		\frac{\ep_{n+k}}{a_{n+k} +}~\cdots$, then $\ep_{n}y_n\in
%		[\frac{1}{a_{n}},1]$, $n\ge0.$
%			\item If $r_i/s_i$ is the $i$-th convergent of the continued fraction for $i\ge0$ then $r_{i+1}s_{i}-r_{i}s_{i+1}=\pm1$ and
%			\begin{equation}\label{reln.w.tail}
%			x=\frac{r_i+y_{i+1}r_{i-1}}{s_i+y_{i+1}s_{i-1}}.
%			\end{equation}
%	\item The sequence $\{s_i\}_{i\ge0}$ is monotonically increasing if and only if $a_i+\ep_i\ge 1$ for $i\ge1$. 	
%	\end{enumerate}
%	
%	\end{proposition}
% 

	
%	\begin{proof}Suppose $p$ is a prime dividing $N.$ Observe that $\mathrm{gcd}(p_i,q_i)=1$ if and only if $\mathrm{gcd}(p_i,N)=1$. By Definition 2, $\mathrm{gcd}(p_i,q_i)=1$ for $i\ge1.$ Thus $\mathrm{gcd}(p_i,p)=1$ and so
%		$p_i\not\equiv 0\mod p$, equivalently, $a_ip_{i-1 }\not\equiv -\ep_ip_{i-2} \mod p,$ which is Statement 1. Since  $q_i=a_i q_{i-1}+\ep_i q_{i-2},$ and $a_i+\ep_i\ge1,$ by induction, we can see that  $\{q_i\}_{i\ge -1}$ is strictly increasing. Statement 3 is clear from Statement 2, and the fact that $\mathrm{gcd}(p_i,q_i)=1.$ Statement 4 and 5 directly follow from Proposition \ref{relationwithttail}. 	
%	\end{proof}
%\noindent For the rest part of this paper, we will write $a_i\not\equiv -\ep_ip_{i-2}p^{-1}_{i-1 } \mod p,$ instead of $a_ip_{i-1 }\not\equiv -\ep_ip_{i-2} \mod p,$ where $p^{-1}_{i-1}$ is the inverse of $p_{i-1}$ in $\Z/p\Z.$  
Here, we recall a few definitions which will help us to show that every irrational number has a unique $\f_N$-continued fraction expansion.
 	 \begin{defi} Let $x\in\mX_N.$ Suppose $\Theta_n\equiv\infty=P_{-1}\to P_0\to P_1\to\cdots\to P_{n-1}\to P_n,$ where $x=P_n,$ is such that no vertex is repeated (i.e., $P_i\ne P_j$ for $-1\le i\ne j\le n$).  
	 Let $Q\in \mX_N$ such that $Q\ne P_i$ for $-1\le i\le n$ and $x\sim_N Q$. 
  	   If $P_{n-1}<Q<x$ or $x<Q<P_{n-1}$, then the edge $x\to Q$ is called a {\it direction changing edge} from $x$ relative to $\Theta_n$ (see Figure 2). 
  	 
  If $P_{n-1}<x<Q \textnormal{ or } P_{n-1}>x>Q$, then the edge $x\to Q$ is called a {\it direction retaining edge} from $x$ relative to $\Theta_n$ (see Figure 3).
  		
  	\end{defi}
  	  \begin{figure}[h!]
  	 	\centering
  	 	\includegraphics[scale= 1.5]{direction_change1.jpg}\caption{Direction changing edge $P_n\sim_N Q$}
  	 \end{figure}   
  	 \begin{figure}[h!]
  	 	\centering
  	 	  	 	\includegraphics[scale= 1.5]{direction_retain1.jpg}
  	 	\caption{Direction retaining edge $P_n\sim_N Q$}		
  	 \end{figure}
%   	\begin{figure}[!h]
%   	%	\begin{minipage}{.9\textwidth}
%   	\centering
%   	\includegraphics[width=.5\linewidth]{edge_numbering.eps}
%   	\caption{Ordering of edges relative to $\Theta_i$ and $\Theta_{i+1}$}\label{Fig:crossing}
%   	%	\end{minipage}\hfill
%   \end{figure}
%  	\noindent We order the edges  according to the size of the radii of the edges (semi circle) such that the first edge has the longest radius.  In Figure 4, $P_i$ denote the $i$-th vertex of a path $\Theta_i$. The edge $P_i\sim_N P_{i+1}$ is direction retaining  and $P_{i+1}$ lies on the second semicircle emanating from $P_i$ relative to $\Theta_i$.   The dotted edges emanating from $P_i$ are showing direction changing edges relative to $\Theta_i$ with their ordering.
%  	Similarly for $\Theta_{i+1}.$
%  	
%  
  	
  	
  	
  		
  		 
 
  	 
  	 
%  	 	Let $N\in\N$ and $Q\in\mX_N.$ Suppose there is a path from $\infty$ to $Q$ which is given by $$\infty\to P_0\to P_1\to P_2\to\cdots\to P_n=Q,$$ where $n\ge1$ so that at least two edges are required to join $\infty$ to $Q$. Denote this path by $\Theta_n$.  Set $P_{-1}=\infty$.
%  	\begin{defi}
%   The vertex $Q$ is called {\it the direction retaining with respect to $\Theta_n$} if either $P_{n-2}<P_{n-1}<Q$ or  $P_{n-2}>P_{n-1}>Q$ (Figure 2) and {\it the direction changing with respect to $\Theta_n$} if $P_{n-2}<Q<P_{n-1}$ or  $P_{n-2}>Q>P_{n-1}$ (Figure 3).
%  	If $Q<P_{n-2}<P_{n-1}$ or $P_{n-1}<P_{n-2}<Q$, then $Q$ is neither direction retaining nor direction changing with respect to $\Theta_n$.	
%  	\end{defi}  
%  	   \begin{figure}[h]
%  	  	\centering
%  	  		 	\includegraphics[scale= 1.5]{drr.eps}\\
%  	  	\vspace{-5mm}	\caption{Direction retaining}
%  	  	\vspace{0mm}
%  	  		\includegraphics[scale= 1.5]{drc.eps}\\	\vspace{-5mm}	\caption{Direction changing}
%  	  \end{figure}
%  	  
%  	  
%  	  In fact, a path from infinity to $Q$ with respect to which $Q$ is neither direction retaining nor direction changing will not appear in this note. When the path from $\infty$ to a vertex $Q$ is clear, we simply write ``$Q$ is a direction retaining vertex" and ``$Q$ is a direction changing vertex" (without mentioning the path).
%  	   Suppose a path $\Theta_n$ is given by
%  	  $$\Theta_n\equiv\infty\to P_0\to P_1\to P_2\to\cdots\to P_n.$$ Then it is clear for $0\le i \le n$, the path denoted by $\Theta_i$ is a part of the path  $\Theta_n$ 
%  	  from infinity to $P_i$. 
%  	
%  		  The semicircle joining $P_i$ and $P_{i+1}$ is said to be the direction changing
%  	  (respectively, direction retaining) semicircle with respect to the path $\Theta_i$ if $P_{i+1}$ is a
%  	  direction changing vertex (respectively, direction retaining vertex). 
%  	  \begin{defi}
%  	  	If the farthest (in Euclidean sense) direction changing vertex adjacent to $P_i$ but different from $P_{i-1}$ is $Q_1$, then the edge
%  	  	joining $P_i$ to $Q_1$ is called the first direction changing semicircle emanating from $P_i$. 
%  	  	Similarly, the direction changing edge joining $P_i$ to the farthest vertex different from $P_{i-1}$ and $Q_1$ 
%  	  	is called the second direction changing semicircle emanating from $P_i$ with respect to $\Theta_i$. Thus, for
%  	  	any $k\ge 1$, we define inductively the $k$-th direction changing semicircle emanating from $P_i$ with respect to $\Theta_i$.
%  	  \end{defi}
%  	  \noindent Similarly, we order the direction retaining semicircles emanating from $P_i$ with respect to $\Theta_i$, where the largest 
%  	  semicircle is the first member.
  	   
  	   
  	   
  	   
  	   
  	   
  	
%	
%\begin{proposition}\label{semicircles}
%	Suppose, for  $x\in\mX_N\setminus\{\infty\},$ there is a path $$\Theta_n\equiv\infty\to P_0=b/N\to P_1\to P_2\to\cdots\to P_n$$
%	with $q_i<q_{i+1},$ where $P_i=p_i/q_i$ for each $i\ge-1$, $P_{-1}=1/0$ and  $x=P_n\ne b/N$.
%	Then for every  $1\le i\le n$, there exist a positive integer $a_{i}$ and an integer $\ep_{i}\in\{1,-1\}$ such that
%	\begin{enumerate}
%		\item $p_{i}=a_{i} p_{i-1}+\ep_{i}p_{i-2},~
%		q_{i}=a_{i} q_{i-1}+\ep_{i}q_{i-2};$
%		
%		\item $\ep_1=-1$ if and only if $x<\frac{b}{N}$  (and hence, $P_1<P_0=\frac{b}{N}$);
%		\item $\ep_{i}=-1$ if and only if $P_{i-1}\sim_NP_{i}$ is direction retaining relative to $\Theta_{i-1}\equiv\infty\to P_0\to\cdots\to P_{i-1}$;
%		\item  Suppose $N=p^l$ for some prime $p.$ Then $P_{i}$ lies in the $k$-th semicircle emanating from $P_{i-1}$ relative to $\Theta_{i-1}$ if and only if 
%		$$a_{i}=\left\{
%		\begin{array}{ll}
%		
%		pm, & \mbox{ if } k=(p-1)m,\, m\ge1\\\\
%		pm+t, \textnormal{ or } pm+(t+1), & \mbox{ if } k=(p-1)m+t, 0<t<p-1, m\ge0.
%		\end{array}
%		\right.$$
%		
%		\end{enumerate}
%\end{proposition}

%\begin{proof}
%	Since $ P_{i-1}\sim_N P_{i}$ and $P_{i-2}\sim_N P_{i-1}$,
%	\begin{eqnarray} p_{i}q_{i-1}-q_{i}p_{i-1}&=& Ne_{i-1}\label{eq1}\\
%	p_{i-1}q_{i-2}-q_{i-1}p_{i-2}&=& Ne_{i-2}\label{eq2},
%	\end{eqnarray}
%	where $e_{i-1},e_{i-2}=\pm1$
%	so that
%	$p_{i}q_{i-1}\equiv -e_{i-2}e_{i-1} p_{i-2}q_{i-1}\mod p_{i-1}.$ Since $p_{i-1}$ and $q_{i-1}$ are co-prime, we have
%	$p_{i}\equiv - e_{i-2}e_{i-1}p_{i-2}\mod p_{i-1}$ and hence
%	$p_{i}=a_{i} p_{i-1}-e_{i-2}e_{i-1} p_{i-2}$ for some $a_{i}\in\Z$. Set $\ep_{i}=-e_{i-2}e_{i-1}.$  
%	Substitute the value of $p_i$ and $\ep_i$ in \eqref{eq1} and   \eqref{eq2}, we get $q_{i}=a_{i} q_{i-1}-e_{i-2}e_{i-1} q_{i-2}$. By hypothesis, $q_i>q_{i-1}>0$ and hence $a_i\in\N$.
%
%The second statement follows from the equality:  $P_1=P_0+\frac{\ep_1}{Na_1}$ and the third statement is a consequence of the following identity:
%	\begin{eqnarray}\label{retaining}
%	\frac{p_{i}}{q_{i}}-\frac{p_{i-1}}{q_{i-1}}=-\ep_{i}\frac{q_{i-2}}{q_{i}} \left(\frac{p_{i-1}}{q_{i-1}}-\frac{p_{i-2}}{q_{i-2}}\right).
%	\end{eqnarray}
%		Let $p$ be a prime such that $p\mid N.$	Suppose $R={r}/{s}$ is a vertex in $\mathcal{F}_{N}$  which is adjacent to $P_{i-1}$ with $q_{i-1}<s$ and $\Theta_{i-1}$ can be
%extended to a non-self-intersecting path from $\infty$ to $R$, given by 
%$$\infty\to P_0=b/N\to P_1\to P_2\to\cdots\to P_{i-1}\to R.$$ Then by the first statement, $r=ap_{i-1}+\ep p_{i-2}$ for some positive integer $a$ and $\ep\in\{\pm1\}.$
%Since $\mathrm{gcd}(p,r)=1$, we have
%	$$r\not\equiv 0~ \mod~ p$$
%	$$ap_{i-1}\not\equiv -\ep p_{i-2 }~ \mod ~p$$
%	\begin{equation}
%	a\not\equiv -\ep p_{i-1}^{-1}p_{i-2}~ \mod~ p.
%	\end{equation}
%	Next, suppose  $P_i'={p_i'}/{q_i'}$ is another vertex such that $P_{i-1}\sim_N P_{i}'$ with $P_i$ and $P_i'$ lying in the same side of $P_{i-1}$ and the path $\Theta_{i-1}$ can be extended to $P_i'$ which is not self-intersecting. Then for some $a_i,a_i'\in\N$ and $\ep_i\in\{\pm1\}$,\\
%	
%		\begin{tabular}{ccc}
%		$\begin{array}{c}	p_{i}= a_{i} p_{i-1}+\ep_{i}p_{i-2},\\
%			q_{i}= a_{i} q_{i-1}+\ep_{i}q_{i-2}
%		\end{array}$
%		
%		& and &
%		
%		$\begin{array}{c}	p_{i}'= a_{i}' p_{i-1}+\ep_{i}p_{i-2},\\
%		q_{i}'= a_{i}' q_{i-1}+\ep_{i}q_{i-2}.
%		\end{array}$\\
%	\end{tabular}\\
%	Thus,
%	\begin{eqnarray}
%	P_i-P_i'&=& \frac{\ep_i q_{i-1}}{q_i'}\left( \frac{p_{i-1}}{q_{i-1}}-\frac{p_{i-2}}{q_{i-2}} \right) (a_i-a_i')\nonumber\\
%	\label{ordering}
%	&=& \frac{q_{i-1}}{q_i'} (P_i-P_{i-1})(a_i'-a_i) \hskip 1cm(\textnormal{using \eqref{retaining}})
%	\end{eqnarray}
%	The fourth statement follows from Equation \eqref{ordering} by considering all possibilities for $a_i$ and $a_i'$ satisfying $a_i,a_i'\not\equiv-\ep_i p_{i-1}^{-1}p_{i-2}\mod p$.
%\end{proof}





% The path $\Theta_n\equiv\infty\to P_0\to P_1\to\cdots\to P_n$ is called {\it well directed} if it has the following property
%  	 for each $i$ $(1\le i\le n)$: the edge $P_i\sim_N P_{i+1}$ is direction changing relative to $\Theta_i$ whenever the rank of $P_{i-1}\sim P_i$ relative to $\Theta_{i-1}$ is $1$ (every edge following a rank one edge is direction changing).
%  	 
\begin{defi} Suppose $n\in\N.$
	A path from infinity to a vertex $x$ in $\f_N$ given by
	$$\Theta_n\equiv\infty=P_{-1}\rightarrow P_0\rightarrow P_1\rightarrow\cdots \to  P_n,$$	where $P_i=p_i/q_i$ and $q_{i}<q_{i+1}$ for $i\ge-1$ and $x=P_n$, is called a \textit{well directed path} if  $P_{i+1}\sim_NP_{i+2}$ ($0\le i\le n-2$) is direction changing relative to  $\Theta_{i+1}$ whenever  $P_{i-1}\sim_N P_{i+1}$.
\end{defi}
%Observe, $P_{i-1}\sim_N P_{i+1}$ implies that $P_{i-1}\to P_i\to P_{i+1}\to P_{i-1}$  is a triangle in the graph. For the case $n=1,$ the path $\Theta_1\equiv\infty\rightarrow P_0\rightarrow P_1$ is always well directed as $\infty$ is never adjacent to $ P_1$. 
%Now recall that $\f_{2^l}$, for $l\ge1$, is a tree, and so the path from infinity to a vertex $x$ in $\f_{2^l}$ is well directed.  Figure 5 is displaying  two paths from $\infty$ to $5/21$ in $\f_{3},$ namely, $\Theta_2$ and $\Theta_3'$, and  given by
% \begin{figure}[!h]
% 	%	\begin{minipage}{.9\textwidth}
% 	\centering
% 	\includegraphics[width=.5\linewidth]{pink-blue.eps}
% 	\caption{Paths from $\infty$ to $5/21$ in $\f_{3}$}\label{Fig:pinkblue}
% 	%	\end{minipage}\hfill
% \end{figure}
%\begin{eqnarray}
%\Theta_2&\equiv&\infty\to 1/3 \to 2/9 \to 5/21,\label{path2}\\
%\Theta_3'&\equiv&\infty\to 1/3\to 1/6\to 2/9 \to 5/21.\label{path1}
%\end{eqnarray} 
%Path \eqref{path1} is not well directed since $1/3\sim_3 2/9$ and $2/9\sim_3 5/21$ is direction retaining relative to $\Theta_1$   whereas path \eqref{path2} is well directed.
%
%\begin{proposition}\label{welldirectedpathexist}
%	For every $x\in\mX_{p^l},$ there is a well directed path from $\infty$ to $x$ in $\f_{p^l}.$
%\end{proposition}
%\begin{proof}
%	Suppose $x\in\mX_{p^l}$.
%	Then by Remark \ref{increasingpath}, there is a path  from $\infty$ to $x$ $$\Theta_n\equiv\infty\rightarrow P_0\rightarrow P_1\rightarrow\cdots \rightarrow P_n=x,$$
%	where $P_i=p_i/q_i,\,-1\le i\le n$ and $\{q_i\}_{i\ge-1}$ is increasing. Note that $q_1>q_0=N$ and hence the path $\Theta_1\equiv\infty\to P_0\to P_1$ is well directed. Suppose $i$ is the smallest positive integer such that $1\le i\le n$ and $\Theta_{i+1}$ is not well directed, then 
%	$P_{i-2}\sim_{p^l} P_{i}$ and $P_{i}\sim_{p^l}P_{i+1}$ is direction retaining relative to $\Theta_{i}$. By dropping $P_{i-1}$ from the path, we get a path
%	$$\hat{\Theta}_{n-1}\equiv\infty\rightarrow P_0\rightarrow\dots\rightarrow P_{i-2}\to P_i\to P_{i+1}\to\cdots\to P_n.$$
%	Then the path  $\hat{\Theta}_{i-1}$ is well directed since $P_{i}$ lies at least on the second semicircle relative to $\Theta_{i-2}=\hat{\Theta}_{i-2}.$ Proceeding in this way, we can construct a well directed path from $\infty$ to $x$ in finitely many steps.
%\end{proof}
%	\begin{remark}\label{remark1}Suppose a path from infinity to a vertex $x$ in $\f_N$ is given by
%		$$\Theta_n\equiv\infty\rightarrow P_0\rightarrow P_1\rightarrow\cdots P_k\rightarrow\cdots\rightarrow P_n,$$
%		where $P_i=p_i/q_i,\,i\ge-1$ and $x=P_n$. Then by Proposition \ref{semicircles} and the above discussion, we have that $a_{i+1}=1$ if and only if $P_{i-1}\sim_N P_{i+1}.$ 
%		Thus, if the path is well directed and $a_{i+1}=1$ for some $i\ge1$, then $\ep_{i+2}=1.$ 	
%	\end{remark}
%	\begin{example}
%			In Figure 6, we are considering two paths, in $\f_{25}$:
%			$\infty\to 1/25\to3/50\to8/125$ (the red path) and  $\infty\to 1/25\to1/50\to2/75$ (the green path), both the paths are well directed.
%			\begin{figure}[!h]
%				%	\begin{minipage}{.9\textwidth}
%				\centering
%				\includegraphics[width=.5\linewidth]{path25.eps}
%				\caption{A few vertices  in $\f_{25}$}
%				%	\end{minipage}\hfill
%			\end{figure}
%		In this diagram, the colored numbers $1,2,\dots$ are denoting the numbering of the semicircle emanating from a vertex relative to the path. In the set up of Proposition \ref{semicircles}, for the red path $\ep_1=1,a_1=2$  and $\ep_2=1,$ $a_2=2$ whereas for the green $\ep_1=-1,a_1=2$ and   $\ep_2=1,$ $a_2=1$. Observe that the continued fraction  $$\frac{1}{0+}~\frac{25}{1+}~\frac{1}{2+}~\frac{1}{2}=8/125$$
%			describes the red path and the green path is described by
%		 $$\frac{1}{0+}~\frac{25}{1+}~\frac{-1}{2+}~\frac{1}{1}=2/75.$$
%		
%			
%			Next we establish a correspondence between $\f_{N}$-continued fractions and well directed paths in $\f_{N}$.
%			
%		
%	\end{example}
	\begin{theorem}\cite[Theorem 3.5]{seemafnpart1}\label{main1} Let $p^l$ be a fixed natural number, where $p$ is a prime and $l\in \N$.
		\begin{enumerate}
			
			\item Suppose $x\in\mX_{p^l}$.  Then every well directed path defines a finite $\f_{p^l}$-continued fraction of $x$.
			\item The value of every finite $\f_{p^l}$-continued fraction belongs to $\mX_{p^l}$ and the continued fraction defines a well directed path in $\f_{p^l}$ from $\infty$ to its value with the convergents as  vertices in the path.
			\item 	Every real number has an $\mathcal{F}_{p^l}$-continued fraction expansion.
		\end{enumerate}
	\end{theorem}
%	\begin{proof}
%		Suppose the well directed path from $\infty$ to $x\in\mX_N$ is given by $$\infty\rightarrow P_0\rightarrow P_1\rightarrow\cdots P_k\rightarrow\cdots\rightarrow P_n,$$ where $P_k=\frac{p_k}{q_k}\in\mX_N$ for $k\ge-1$ and $x=P_n$.
%		By definition, $P_0={b}/{N}$ for some integer $b$ co-prime to $N$. 
%		We complete the proof by induction on the distance of $x$ from $\infty$ through the given  path.	By induction hypothesis, any vertex $P_i=\frac{p_{i}}{q_{i}}$ on the path having distance $i+1$ ($1\le i\le k$) from $\infty$ is defined by an $\f_N$-continued fraction
%		$$\frac{p_{i}}{q_{i}} = \frac{1}{0+}~\frac{N}{b+}~\frac{\epsilon_{1} }{a_{1}+}~\frac{\epsilon_{2}}{a_{2}+}\cdots\frac{\epsilon_{i}}{a_{i}}.$$
%		Since $P_{k-1}\sim_N P_{k}$ and $ P_{k}\sim_N P_{k+1}$, by  Proposition \ref{semicircles}, we have
%		\begin{eqnarray*}
%			p_{k+1}&=&a_{k+1} p_{k}+\ep_{k+1}p_{k-1},\\
%			q_{k+1}&=&a_{k+1} q_{k}+\ep_{k+1}q_{k-1}.
%		\end{eqnarray*}
%		Since $p_{k+1}$ and $q_{k+1}$ satisfy the same recurrence relation with the initial condition $(p_{-1},q_{-1})=(1,0)$ and  $(p_{0},q_{0})=(b,N)$, we have $$P_{k+1}=\frac{p_{k+1}}{q_{k+1}}=  \frac{1}{0+}~\frac{N}{b+}~\frac{\epsilon_{1}}
%		{a_{1}+}~\frac{\epsilon_{2}}{a_{2}+}\cdots\frac{\epsilon_{k}}{a_{k}+}~\frac{\epsilon_{k+1}}{a_{k+1}}.$$
%		Since denominators $q_i$'s are increasing, we have $\ep_i+a_i\ge1$ for $i\ge1$. Using Remark \ref{remark1}, we get $a_i+\ep_{i+1}\ge1$ for each $i\ge1$ as the path is well directed.
%		Thus a well directed path from $\infty$ to $x$ defines  a finite $\f_N$-continued fraction of $x$ given by
%		$$x= \frac{1}{0+}~\frac{N}{b+}~\frac{\epsilon_{1} }{a_{1}+}~\frac{\epsilon_{2}}{a_{2}+}~\cdots\frac{\epsilon_{n}}{a_{n}}.$$
%		
%	\noindent	To prove the second statement, suppose  $$\frac{1}{0+}~\frac{N}{b+}~\frac{\epsilon_{1} }{a_{1}+}~\frac{\epsilon_{2}}{a_{2}+}~\cdots\frac{\epsilon_{n}}{a_{n}}$$ is an $\f_N$-continued fraction.
%		Let $P_0={b}/{N}$ and let for each $1\le i\le n$, $P_{i}=\frac{1}{0+}~\frac{N}{b+}~\frac{\epsilon_{1} }{a_{1}+}~\frac{\epsilon_{2}}{a_{2}+}~\cdots\frac{\epsilon_{i}}{a_{i}}.$ Then $P_0\in \mX_N$ and $\infty\to P_0$ is well directed. By induction hypothesis $P_1, P_2,\dots, P_{k-1}\in\mX_N$ and $\infty\to P_0\to P_1\to \cdots \to P_{k-1}$ is well directed. That $P_k\in \mX_N$ and $P_{k-1}\sim_N P_k$ follow from  Theorem \ref{distinctconvergents} and since $a_i+\ep_i\ge1$ and $a_i+\ep_{i+1}\ge1$, the path $\infty\to P_0\to P_1\to \cdots \to P_{k}$ is well directed.
%		\end{proof}
%		The following corollary of Theorem \ref{main1} follows from Proposition \ref{welldirectedpathexist}.
%		\begin{corollary}
%			Suppose $x\in\mathcal{X}_{p^l}$ for some prime $p$ and a positive integer $l.$ There is a finite $\mathcal{F}_{p^l}$-continued fraction expansion of $x.$
%		\end{corollary}
%	\noindent Since $\f_{2^l}$ ($l\ge1$) is a tree, we have another corollary of Theorem \ref{main1}.
%		\begin{corollary}
%			Suppose $x\in\mathcal{X}_{2^l}$ for some positive integer $l.$ There is a unique finite $\mathcal{F}_{2^l}$-continued fraction expansion of $x.$
%		\end{corollary}
	



%\section{An Algorithm to Find an $\f_{p^l}$-Continued Fraction expansion}
%
%Let $p$ be a prime and $l\in\N$. In $\f_{p^l}$,   
%any path from $\infty$ is via $a/p^l$ for some $a\in \Z$, where $\mathrm{gcd}(a,p)=1.$ We have seen that there is a well directed path from $\infty$ to every $x\in\mX_{p^l}.$ Suppose $a\in\Z$  and $x\in\mX_{p^l}$ are such that $\textnormal{gcd}(a,p)=1=\textnormal{gcd}(a+1,p)$   and $a/p^l<x<(a+1)/p^l.$ It is not difficult to see that any well directed path from $\infty$ to $x$ is via $a/p^l$ or $(a+1)/p^l.$ 
% For certain points there are well directed paths from $\infty$ to $x$ via both $a/p^l$ and $(a+1)/p^l.$ For instance, the following paths in $\f_5$  from $\infty$ to $11/40$ are well directed and the former is via $1/5$ and the latter is via $2/5:$
%$$\Theta\equiv\infty\to 1/5\to 3/10\to 4/15\to 11/40$$
%and $$\Theta'\equiv\infty\to 2/5\to 3/10\to 4/15\to 11/40.$$
%Now,  observe that $7/20\in[1/5,2/5]$ and 
%$\infty\to 2/5\to 7/20$ is the only well directed path in $\f_5$ from $\infty$ to $7/20$. Thus,  there is no well directed path from $\infty$ to $7/20$ via $1/5$.
\section{ $\f_{p^l}$-Continued Fractions with maximum +1}
\begin{defi}
	Suppose $ R_1,R_2\in\mX_{p^l}$ are such that $R_1\sim_{p^l} R_2$ in $\f_{{p^l}}$, where $R_i=r_i/s_i$ with $\mathrm{gcd}(r_i,s_i)=1$ for $i=1,2$. Then $R_1\oplus R_2$ denotes a rational number  $r/s$ where $r=r_1+r_2$ and $s=s_1+s_2$ and $R_2\ominus R_1$ denotes a rational number  $r'/s'$ where $r'=r_2-r_1$ and $s'=s_2-s_1.$ Operations $\oplus$ and $\ominus$ are referred to as \textit{Farey sum}\index{Farey sum} and \textit{Farey difference}\index{Farey difference} of two rational numbers. 
\end{defi}
%Observe, $r$ and $s$  (similarly, $r'$ and $s'$) need not be co-prime. We show, if $a/p^l\oplus(a+1)/p^l\not\in\mathcal{X}_{p^l},$  a well directed path from $\infty$ to $x$ is either via $a/p^l$ or $(a+1)/p^l$ but not via both.
%
%\begin{lemma}\label{p0}
%	Let $p$ be a prime and $l\in\N$. Suppose  $x\in\mathcal{X}_{p^l}$ is 
%	  such that $a/p^l<x<(a+1)/p^l,$ where $a/p^l,~(a+1)/p^l\in\mathcal{X}_{p^l}$. If $a/p^l\oplus(a+1)/p^l\not\in\mathcal{X}_{p^l},$ then a well directed path from $\infty$ to $x$ is via a unique vertex $b/p^l,$ where 
%	$$b=\left\{
%	\begin{array}{ll}
%	a, &\mbox{if }  x<a/p^l\oplus(a+1)/p^l\\\\
%a+1, &\mbox{if }  x>a/p^l\oplus(a+1)/p^l.
%	\end{array}
%	\right.$$
%
%\end{lemma}
%\begin{proof}
%
% Suppose $a/p^l\oplus(a+1)/p^l\not\in\mathcal{X}_{p^l}$ and $x<a/p^l\oplus(a+1)/p^l.$ We claim that there is no well directed path from $\infty$ to $x$ via $(a+1)/p^l.$ Suppose $$\infty\to P_0\to P_1\to P_2\to\cdots\to P_n,$$
% where $P_0=(a+1)/p^l$ and $P_n=x,$
%	is a well directed path. Then there is an edge, say $P_m\sim_{p^l}P_{m+1}$, 
%	   such that $$a/p^l<P_{m+1}<a/p^l\oplus(a+1)/p^l<P_{m}<(a+1)/p^l.$$
%	  Thus, in the Farey graph,  $a\sim_1 a+1$ and $p^lP_m\sim_1 p^l P_{m+1}$ with $a<p^lP_{m+1}<a+1<p^lP_{m}$ and we get a contradiction as no two edges cross in the Farey graph.
%	  The other case when  $x>a/p^l\oplus(a+1)/p^l$ is similar, hence the result follows.
%	  \end{proof}
%
%\begin{corollary}\label{cor2}
%	If $x\in\mX_{2^l},$  the well directed path from $\infty$ to $x$ is via $b/2^l,$ where	$b=
%	2\lfloor 2^{l-1}x\rfloor+1.$	
%\end{corollary}
%
%\begin{corollary}
%	Suppose $x\in\mathcal{X}_{3^l}$ such that $a/3^l<x<(a+1)/3^l,$ where $a/3^l,(a+1)/3^l\in\mX_{3^l}$. Then a well directed path from $\infty$ to $x$ is via a unique vertex $b/3^l,$ where
%	$$b=\left\{
%	\begin{array}{ll}
%	a, &\mbox{if }  x<a/3^l\oplus(a+1)/3^l\\\\
%	a+1, &\mbox{if }  x>a/3^l\oplus(a+1)/3^l.
%	\end{array}
%	\right.$$	
%\end{corollary}


%Let $x\in\mX_{p^l},$ where $p$ is a prime and $l$ is a positive integer.  Here, we formulate an algorithm to
%find an  $\f_{p^l}$-continued fraction of $x$. 
%
%
%\begin{theorem} \label{relation} Given any $x\in\mX_{p^l}$, an $\f_{p^l}$-continued fraction expansion $$\frac{1}{0+}~\frac{p^l}{b+}~\frac{\epsilon_{1} }{a_{1}+}~\frac{\epsilon_{2}}{a_{2}+}~\cdots\frac{\epsilon_{n}}{a_{n}}$$of $x$ is obtained as follows:
%	
%	$$b=\left\{
%		\begin{array}{ll}
%	\lfloor p^l x\rfloor, &\mbox{if } 							
%	
%	 (\lfloor p^l x\rfloor+1,p)\neq1\\\\
%	 	\lfloor p^l x\rfloor+1, &\mbox{if } 							
%	 	(\lfloor p^l x\rfloor,p)\neq1\\\\
%	
%	\lfloor p^l x\rfloor,& \mbox{ if } (\lfloor p^l x\rfloor,p)=1=(\lfloor p^l x\rfloor+1,p)  \textnormal{ and }  x<	\frac{\lfloor p^l x\rfloor}{p^l}\oplus\frac{\lfloor p^l x\rfloor+1}{p^l}\\\\
%	 
%	\lfloor p^l x\rfloor+1, &  \mbox{ if } (\lfloor p^l x\rfloor,p)=1=(\lfloor p^l x\rfloor+1,p) \textnormal{ and } 	x>	\frac{\lfloor p^l x\rfloor}{p^l}\oplus\frac{\lfloor p^l x\rfloor+1}{p^l}.
%	\end{array}
%	\right.$$	
%	
%	
%	Set $y_1=p^lx-b$,
%	
%	\begin{enumerate}
%		\item   $\ep_{i}=\sign (y_i);$\\
%		\item  \label{partialquotients}$a_i=\lfloor (\frac{1}{|y_i|}+1)\rfloor$ or $a_i=\lceil (\frac{1}{|y_i|}-1)\rceil$ or $a_i=\frac{1}{|y_i|}$ if $\frac{1}{|y_i|}\in \N$, 
%		
%			such that $a_i\not\equiv -\ep_i p_{i-2}p_{i-1}^{-1}\mod p$ and $a_i+\ep_i\ge1;$ \\
%		
%		
%		
%		\item    $y_{i+1}=\frac{1}{|y_i|}-a_i$. \label{tail-relation}
%	\end{enumerate}
%		In fact, $n$ is the smallest non-negative integer for which $y_{n+1}=0$.
%		
%\end{theorem}
%\begin{proof}
%	Set $b$ as given by Lemma \ref{p0} and $y_1 = p^lx-b$. Applying algorithm steps on $y_1$, we get $y_1=\frac{\epsilon_{1} }{a_{1}+}~\frac{\epsilon_{2}}{a_{2}+}~\cdots\frac{\epsilon_{n}}{a_{n}}.$	Let $y_i$ be the $i$-th fin of the continued fraction expansion of $y_1$, namely,
%	\begin{equation*}
%	y_i=\frac{\ep_i}{a_i+}~\frac{\ep_{i+1}}{a_{i+1}+}\cdots\frac{\ep_n}{a_n}.
%	\end{equation*}
%	Then 
%	$$y_1=\frac{\ep_1}{a_1+y_{2}};$$ in a similar way, for $2\le i \le n$, we also have
%	\begin{equation}\label{aaa}
%	y_i=\frac{\ep_i}{a_i+y_{i+1}}.
%	\end{equation}
%	
%	\noindent	Note that $\ep_i = \sign(y_i)$ if and only if  $a_i+\ep_{i+1}\ge1$. By Proposition \ref{distinctconvergents} (4), $|y_{i+1}|\leq 1$ and $a_i$ is a positive integer. This implies that  
%	$$a_i=\lfloor (\frac{1}{|y_i|}+1)\rfloor \textnormal{ or } \lceil (\frac{1}{|y_i|}-1)\rceil \textnormal{ or } \frac{1}{|y_i|},$$
%	where the last possibility is feasible only when $1/|y_i|\in\N$.
%	Now suppose $p_i/q_i$ denotes the $i$-th convergent of the following continued fraction  $$\frac{1}{0+}~\frac{p^l}{b+}~\frac{\epsilon_{1} }{a_{1}+}~\frac{\epsilon_{2}}{a_{2}+}~\cdots\frac{\epsilon_{n}}{a_{n}},$$ $p_0=b,q_0=p^l$ and $i\ge1.$ 
%	Since $\lceil (\frac{1}{|y_i|}-1)\rceil,\frac{1}{|y_i|} $ and  $\lfloor (\frac{1}{|y_i|}+1)\rfloor$ are consecutive integers, at least one of them is not congruent to  
%	$-\ep_ip_{i-2}p_{i-1}^{-1}$ modulo $p.$ Set this value as $a_i.$ If $a_i+\ep_i\ge1$ then we are done.
%	If $a_i+\ep_i=0$ then we claim that $a_{i-1}$ has two possibilities satisfying all the conditions and the other choice of $a_{i-1}$ gives that the new $a_i$ is not congruent to  
%	$-\ep_ip_{i-2}p_{i-1}^{-1}$ modulo $p$ and $a_i+\ep_i\ge1$. 
%	To see this, suppose $a_i+\ep_i=0$ and $-\ep_{i-1}p_{i-3}p_{i-2}^{-1}\equiv a \mod p.$ Then we know that $a_i=1,\ep_i=-1$ and $a_{i-1}\not\equiv a \mod p$.   Let $c=a_{i-1}$ then we claim that the other choice for $a_{i-1}$ is $c-1$. Since $a_i+\ep_i=0$, $P_{i}=P_{i-1}\ominus P_{i-2}$ so that $p_i=(c-1)p_{i-2}+\ep_{i-1}p_{i-3}$ with $p_i\not\equiv0\mod p,$ hence $c-1\not\equiv a \mod p,$ which produces $\epsilon_i=1$ and we are done.	The last claim is clear from the fact that the denominator in \eqref{aaa} is non-zero for $1\le i \le n$ and $y_n = \frac{\ep_n}{a_n}$ by definition, giving $y_{n+1}=0$.
%\end{proof}
%
%\begin{corollary}
%	Given any $x\in\mathcal{X}_{2^l}$, the $\f_{2^l}$-continued fraction expansion $$x=\frac{1}{0+}~\frac{2^l}{b+}~\frac{\epsilon_{1} }{a_{1}+}~\frac{\epsilon_{2}}{a_{2}+}~\cdots\frac{\epsilon_{n}}{a_{n}},$$ is obtained as follows:
%	$b=2\lfloor2^{l-1} x\rfloor+1$ and for $(1\le i\le n)$, setting $y_1=2^lx-b$,
%	\begin{enumerate}[(1)]
%		\item  $ a_{i}=2\big\lfloor\frac{1}{2}\left(1+\frac{1}{|y_{i}|}\right)\big\rfloor$, 
%	%	\label{partialquotients}
%		\item   $\ep_{i}=\sign (y_i),$
%		\item    $y_{i+1}=\frac{1}{|y_i|}-a_i$. %\label{tail-relation}
%	\end{enumerate}
%	In fact, $n$ is the smallest non-negative integer for which $y_{n+1}=0$.
%	
%\end{corollary}




%\begin{proof}
%%	 For a proof, we refer to \cite[Theorem 5.1]{seema}
%%	 	To prove this, we note that \eqref{reln.w.tail2} implies that \eqref{reln.w.tail} holds (the analogue of Proposition \ref{relationwithttail} for $\f$-continued fractions) and hence we have
%%	 	\begin{equation*}%\label{reln.w.tail3}
%%	 	\left| x-\frac{p_n}{q_n} \right| =\frac{2 |y_{n+1}|}{q_n (q_n+y_{n+1}q_{n-1})}.
%%	 	\end{equation*}
%%	 	The right side above converges to $0$ because $q_n$ are monotonically increasing integers (Theorem \ref{distinctconvergents}) and $|y_n|\leq 1$ as $|y_n|=T^{(n-1)}(|y_1|)$. This completes the proof of the theorem.
%We can see that the $n$-th iteration of the algorithm given in Theorem \ref{relation} on $y_1=p^lx-b$  yields the relation (for $n\geq1$)
%\begin{equation}\label{reln.w.tail2}
%x=\frac{1}{0+}~\frac{p^l}{b+}~\frac{\epsilon_{1} }{a_{1}+}~\frac{\epsilon_{2}}{a_{2}+}~\cdots\frac{\epsilon_{n}}{a_{n}+y_{n+1}},
%\end{equation}
%we need to show that the infinite continued fraction
%$$\frac{1}{0+}~\frac{p^l}{b+}~\frac{\epsilon_{1} }{a_{1}+}~\frac{\epsilon_{2}}{a_{2}+}~\cdots\frac{\epsilon_{n}}{a_{n}+}~\cdots$$
%converges to $x$. Let $\{\frac{p_n}{q_n}\}_{n\ge0}$ with $p_0/q_0=b/p^l$ denotes the sequence of convergents of the continued fraction obtained in \eqref{reln.w.tail2}.
%	 		To prove that the sequence $\{\frac{p_n}{q_n}\}_{n\ge0}$ converges to $x$, we use  Proposition \ref{distinctconvergents}(5) and we have
%	 		\begin{equation*}%\label{reln.w.tail3}
%	 		\left| x-\frac{p_n}{q_n} \right| =\frac{p^l |y_{n+1}|}{q_n (q_n+y_{n+1}q_{n-1})}.
%	 		\end{equation*}
%	 		The right side above converges to $0$ because $q_n$ are monotonically increasing integers (Theorem \ref{distinctconvergents}) and $|y_{n+1}|\leq 1$. This completes the proof of the theorem.
%\end{proof}




\begin{lemma}\label{twoways}
	Let $P,Q\in\mX_{p^l}$ be two adjacent vertices in $\f_{p^l}$ and let
	$$P= P_1\to P_2\to\cdots\to P_{n+1}=Q$$ be a path in $\f_{p^l}$ such that $q_i<q_{i+1},~1\le i\le n,$ where $P_i=p_i/q_i,~\mathrm{gcd}(p_i,q_i)=1.$ Then $n\le 2$ and if $n=2,$  then $P_2=Q\ominus P.$
\end{lemma}
Since every element of $\mX_{p^l}$ has finite $\f_{{p^l}}$-continued fraction expansions, we have the following definition.

\begin{defi}Suppose $x\in\mX_{p^l}.$ An $\f_{p^l}$-continued fraction of $x$ not ending with $1/1$ is said to be an {\it $\f_{p^l}$-continued fraction with maximum $+1$} if it has maximum number of positive partial numerators excluding $\ep_1$, the first partial numerator, among all its $\f_{p^l}$-continued fraction expansions.
	
	An infinite $\f_{p^l}$-continued fraction
	$$\frac{1}{0+}~\frac{p^l}{b+}~\frac{\epsilon_{1} }{a_{1}+}~\frac{\epsilon_{2}}{a_{2}+}~\cdots\frac{\epsilon_{n}}{a_{n}+}\cdots$$ is said to be an \textit{ $\f_{p^l}$-continued fraction with maximum $+1$} if 
	$$\frac{1}{0+}~\frac{p^l}{b+}~\frac{\epsilon_{1} }{a_{1}+}~\frac{\epsilon_{2}}{a_{2}+}~\cdots\frac{\epsilon_{i}}{a_{i}}$$
	is an  $\f_{p^l}$-continued fraction with maximum $+1$ of the $i$-th convergent unless $(\ep_i,a_i)=(1,1).$
	\end{defi} 
 
%\begin{defi}The path 
%	$\infty\to P_0\to P_1\to\cdots\to x,$
%	corresponding to an $\f_{p^l}$-continued fraction (finite or infinite) with maximum $+1$ is a well directed path from $\infty$ to $x$ having maximum direction changing edges excluding the edge $P_0\sim_{p^l} P_1$ and $x$ is not a direction changing vertex lying on the first circle. We call it a well directed path with maximum direction changing edges from $\infty$ to $x$.
%	
%\end{defi}
%  

\begin{defi}
	The path associated to an $\mX_{p^l}$-continued fraction with maximum $+1$ is called a {\it well directed path with maximum direction changing edges}.
\end{defi}	
%   \begin{remark} Suppose
%   	$\frac{1}{0+}~\frac{p^l}{b+}~\frac{\epsilon_{1} }{a_{1}+}~\frac{\epsilon_{2}}{a_{2}+}~\cdots\frac{\epsilon_{n}}{a_{n}+}\cdots$
%   	is an $\f_{p^l}$-continued fraction with maximum $+1.$ Then for certain $i\ge1,$
%   	$$\frac{1}{0+}~\frac{p^l}{b+}~\frac{\epsilon_{1} }{a_{1}+}~\frac{\epsilon_{2}}{a_{2}+}~\cdots\frac{\epsilon_{i}}{a_{i}}$$ 
%   	need not be the $\f_{p^l}$-continued fraction with maximum $+1$ of the $i$-th convergent. In fact, if it is not, then $\ep_i/a_i=1/1$ so that $$\frac{1}{0+}~\frac{p^l}{b+}~\frac{\epsilon_{1} }{a_{1}+}~\frac{\epsilon_{2}}{a_{2}+}~\cdots\frac{\epsilon_{i-1}}{(a_{i-1}+1)}$$ is the $\f_{p^l}$-continued fraction with maximum $+1$ of the $i$-th convergent.
%   \end{remark}

In the following subsections, we discuss uniqueness of $\f_{p^l}$-continued fractions with maximum $+1.$

\subsection{Uniqueness of $\f_{p^l}$-Continued Fractions for  $x\in\mX_{p^l}$ or $x\in \R\setminus\Q$ }

\vspace{2mm}
Suppose $P$ and $R$ are adjacent vertices in $\f_{p^l}$. Then for $k\ge1,$ $\underbrace{P\oplus\cdots\oplus P}_{k\textnormal{-times}}\oplus R$ is a rational number of the form $\frac{ku+r}{kv+s}$, where $P=u/v$ and $R=r/s$ with $\mathrm{gcd}(u,v)=1=\mathrm{gcd}(r,s).$ For $k\ge2,$ we denote it by $(\oplus_k P)\oplus R$. 
\vspace{2mm}

\begin{lemma}\label{k-thfareysum}
	Suppose $P,R\in\mX_{p^l}$ are adjacent in $\f_{p^l}.$ Then there exists a natural number $k<p$ such that $(\oplus_k P)\oplus R\not\in\mX_{p^l}$.
\end{lemma}
\begin{proof}
	Let $P=u/v$ and $R=r/s$ be in $\mX_{p^l}$, where $\mathrm{gcd}(u,p)=1=\mathrm{gcd}(r,p).$ Suppose $\bar{x}$ is the congruence class of $x$ modulo $p.$ Then $$\{\overline{r},\overline{u+r},\overline{2u+r},\dots, \overline{(p-1)u+r}\}=\{\overline{0},\overline{1},\dots, \overline{p-1}\}.$$
	Hence, there is a positive integer $k<p$ such that $\overline{ku+r}=\overline{0}$ and the result follows.
\end{proof}	

	\begin{proposition}\label{nonuniquehalfpoint}
	Let $x\in\mX_{p^l}$. If $x=\lfloor p^lx\rfloor/p^l\oplus (\lfloor p^lx\rfloor+1)/p^l$, where $\mathrm{gcd}(\lfloor p^lx\rfloor,p)=1=\mathrm{gcd}(\lfloor p^lx\rfloor+1,p)$. Then $\f_{p^l}$-continued fraction expansion of $x$ with maximum $+1$ is not unique. 
\end{proposition}
\begin{proof}
Let $p_i/q_i$ be a sequence of $\f_{p^l}$-convergents of a real number. Then $q_i<q_{i+1}$. Therefore, the only possible choices of  $\f_{p^l}$-continued fractions of 	$x$ are $\frac{1}{0+}~\frac{p^l}{\lfloor p^lx\rfloor+}~\frac{+1 }{2}$ and $\frac{1}{0+}~\frac{p^l}{(\lfloor p^lx\rfloor+1)+}~\frac{-1 }{2}$, which $\f_{p^l}$-continued fractions with maximum $+1.$
\end{proof}
	If $ x\in\R\setminus\Q$ or $x\in\mX_{p^l}$. Let $P_0=a/p^l$ and $Q_0=(a+1)/p^l$ be  vertices in $ \f_{p^l}$ such that $P_0<x<P_0\oplus Q_0$. Then there is a well directed path from $\infty$ to $x$ via $P_0$ (by Lemma \cite[Lemma 4.1 ]{seemafnpart1}.  In fact, we have the following:
\begin{lemma}\label{lemma_uniquemaximumflips}
	Suppose $ x\in\R\setminus\Q$ or $x\in\mX_{p^l}$. Let $P_0=a/p^l$ and $Q_0=(a+1)/p^l$ be  vertices in $ \f_{p^l}$ such that $P_0<x<P_0\oplus Q_0$.
Then there is a unique  well directed path from $\infty$ to $x$ via $P_0$ having  maximum direction changing edges. 
\end{lemma}
\begin{proof}	Suppose $x\in\R$ and $x\not\in\Q\setminus\mX_{p^l}$ and there are two well directed paths from $\infty$ to $x$ with maximum direction changing edges via $P_0$. Let $\{P_i\}_{i\ge0}$ and $\{P_i'\}_{i\ge0}$ be the corresponding sequences of vertices, respectively. 
Now suppose  $P_i=P_i',~0\le i\le k-1,$ and $P_k\ne P_k'$. Without loss of generality, we may assume that $P_{k-1}<P_k<P_k'$ so that $P_{k-1}<P_k\le P_{k-1}\oplus P_{k}'<P_k'.$ First, we claim that  $x\ne P_{k-1}\oplus P_{k}'.$ If $ P_{k-1}\oplus P_{k}'\not\in\mX_{p^l}$, then $x\ne P_{k-1}\oplus P_{k}'$ as $x\not\in\Q\setminus\mX_{p^l}$. If $P_{k-1}\oplus P_{k}'\in\mX_{p^l}$  and $x=P_{k-1}\oplus P_{k}'$, then 
	$$\infty\to P_0\to\cdots\to P_{k-1}\to x$$ is the only path with maximal direction changing edges. This contradicts our assumption. Now we claim that $P_k= P_{k-1}\oplus P_k'$. Suppose $P_k\ne P_{k-1}\oplus P_k'$. If $P_k<x<P_{k-1}\oplus P_k'$, then any path through $P_k'$ in the direction of $x$ is not well directed. Similarly, if $P_{k-1}\oplus P_k'<x<P_k'$, then there is no well directed path to $x$ through $P_k$. Thus,  $P_k= P_{k-1}\oplus P_k'.$ Now, observe that the following path 
	\begin{equation}\label{extraflips}
	\infty\to P_0'\to\cdots\to P_{k-1}'\to P_k'\to P_k\to P_{k+1}\to\cdots\to x,
	\end{equation}
	where $P_k= P_{k-1}\oplus P_k'$, is a path from $\infty$ to $x$ having two additional direction changing edges and hence,  not well directed. Therefore, $P_{k+1}=P_{k}\oplus P_{k}'=\oplus_2P_{k-1}\oplus P_k'$ and $P_{k+2}$ is direction retaining with respect to the path
	$$\infty\to P_0'\to\cdots P_{k-1}'\to P_k'\to P_k\to P_{k+1}.$$
	Now consider the path
	\begin{equation}\label{extraflips1}
	\infty\to P_0'\to\cdots P_{k-1}'\to P_k'\to P_{k+1}=(\oplus_2P_{k-1}\oplus P_k')\to P_{k+2}\to\cdots\to x
	\end{equation}
	having two additional direction changing edges. Again, the path is not well directed so that $P_{k+2}=\oplus_3 P_{k-1}\oplus P_k'$. By repeating this argument, we get $P_{k+i-1}=\oplus_i P_{k-1}\oplus P_k'$ for $1\le i\le p-1$ which contradicts Lemma \ref{k-thfareysum}. 
\end{proof}
\begin{remark}\label{remark_maximumdchange}	Suppose $P_0=a/p^l,P_0'=(a+1)/p^l\in \mX_{p^l}$ and $x\in\R$ are such that $P_0<x<P_0\oplus P_0'$. Let   $P_0\sim_{p^l} P$ and $P_0\sim_{p^l} P'$ be two consecutive edges emanating from $P_0$ such that $P'<x<P.$ Then the following statements are easy to observe:
	\begin{enumerate}
		\item For some positive integer $k,$ $P=(\oplus_k P_0)\oplus P_0'$ and there is a well directed path from $\infty$ to $x$ through $P_0'$ if and only if for each $i,$ $1\le i\le k,$ $(\oplus_i P_0)\oplus P_0'\in\mX_{p^l}.$
		\item If there is a well directed path from $\infty$ to $x$ through $P_0'$ then the path is via $P.$
	\end{enumerate}
	
\end{remark}

\begin{proposition}\label{prop_maxdchngP0}
	Suppose $P_0=a/p^l,P_0'=(a+1)/p^l\in \mX_{p^l}$ and $x\in\R$ are such that $P_0<x< P_0'$. Then the well directed path from $\infty$ to $x$ with maximum direction changing edges is via $P_0$ if and only if $P_0<x<P_0\oplus P_0'$.
\end{proposition}
\begin{proof}
	Let $P_0<x<P_0\oplus P_0'$. By Lemma \ref{lemma_uniquemaximumflips}, there is a unique well directed path from $\infty$ to $x$ via $P_0$ with maximum direction changing edges, and  assume that the path is given by
	\begin{equation}\label{pathfromP0}
\Theta\equiv\infty\to P_0\to P_1\to P_2\to\cdots\to x.
	\end{equation}
	Let  $k$ be the number of direction changing edges in this path excluding $P_0\sim_{p^l} P_1.$
	Now suppose there is a well directed path from $\infty$ to $x$ via $P_0'$. By Remark \ref{remark_maximumdchange}, the path via $P_0'$ is through $P_1$ and $P_0<x<P_1$ so that
	\begin{equation}\label{pathfromP0'}
\Theta'\equiv	\infty\to P_0'\to\cdots\to P_r'\to P_1\to P_2\to\cdots\to x
	\end{equation}
	is a well directed path with maximum direction changing edges through $P_0'.$ Since $P_0<x<P_1<P_0\oplus P_0'$ the edge $P_1\to P_2$ is direction changing relative to $\Theta_1$ in path \eqref{pathfromP0} and the edge $P_1\to P_2$ is direction retaining relative to $\Theta_{r+1}'$ in path \eqref{pathfromP0'}. Further, note that the edge $P_i'\to P_{i+1}'$ in the path \eqref{pathfromP0'} is direction retaining relative to $\Theta_{i}'$ for $0\le i \le r.$ Thus, the path \eqref{pathfromP0'} has $k-1$ direction changing edges, which is a contradiction. So a well directed path from $\infty$ to $x$ with maximum direction changing edges is only via $P_0.$
\end{proof}
 We summarize Lemma \ref{lemma_uniquemaximumflips}, Remark \ref{remark_maximumdchange} and Proposition \ref{nonuniquehalfpoint} and { \ref{prop_maxdchngP0} in the following theorem.
\begin{theorem}\label{uniquemaximumflips}Suppose $x\in\R\setminus\Q$ or $x\in\mX_{p^l}$ such that $x\ne\lfloor p^lx\rfloor/p^l\oplus (\lfloor p^lx\rfloor+1)/p^l$ $x\not\in\Q\setminus\mX_{p^l}.$ Then 
	 there is a unique  well directed path with  maximum direction changing edges from $\infty$ to $x$. Consequently, an $\f_{p^l}$-continued fraction expansion of $x$ with maximum $+1$   is unique.
\end{theorem} 
 
 

%\noindent From the proof of Proposition \ref{prop_maxdchngP0}, we make the following observations.\vspace{3mm}
 
%\noindent Let  $x\in\mX_{p^l}\cap[a/p^l,(a+1)/p^l]$ for some integer $a.$  Suppose the path from $\infty$ to $x$ with maximum direction changing edges is given by
% 	$$\infty\to P_0\to P_1\to\cdots\to P_n=x.$$
% 	\begin{enumerate}
% 		\item $P_0$ is either $a/p^l$ or $(a+1)/p^l.$ When $a/p^l$ and $(a+1)/p^l$ are both in $\mX_{p^l},$ then   $$P_0=a/p^l \Leftrightarrow |x-\frac{a}{p^l}|\le|x-\frac{a+1}{p^l}|.$$
% 		\item At the $i$-th stage for $i\ge1$, suppose there are two vertices adjacent to $P_{i-1}$ in $\f_{p^l}$, say $P$ and $P'$ such that there are well directed paths from $\infty$ to $x$ through  $P$ as well as through $P'$ and $P_{i-1}<P'<P.$ Then
% 		\begin{enumerate}
% 			\item  $P'<x<P.$
% 			\item $P\sim_{p^l}P'.$
% 			\item 	$P_i=P.$
% 		\end{enumerate}
% 		
% 	\end{enumerate}
% 	 The above discussion suggests the following algorithm to find a well directed path from $\infty$ to $x\in\mX_{p^l}$ with maximum direction changing edges.
 \begin{theorem}\label{pathwithmaximumflips}
 	Suppose $x\in\R.$ The well directed path $\infty\to P_0\to P_1\to\cdots\to P_n\to\cdots\to x$   in $\f_{p^l}$ with maximum direction changing edges can be obtained by the following steps:
 	\begin{enumerate}
 		\item $P_0\sim_{p^l}\infty$ is chosen so that $|P_0-x|$ is the least possible.

 		\item For $i\ge1,$ 
 		if $P, P'\in\mX_{p^l}$ are such that  $P_{i-1}\sim_{p^l} P,$ $P_{i-1}\sim_{p^l} P'$,  $P\sim_{p^l} P'$ and $P_{i-1}<P'<x<P$. Then $P_i=P.$ 
 	\end{enumerate}
 \end{theorem}
 
 Now we utilize the algorithm given in Theorem \ref{pathwithmaximumflips} for finding an $\f_{p^l}$-continued fraction of $x\in\R$ with maximum $+1$.
 \begin{corollary} \label{algoformaximumflips} Given any $x\in\R$, an $\f_{p^l}$-continued fraction expansion $$\frac{1}{0+}~\frac{p^l}{b+}~\frac{\epsilon_{1} }{a_{1}+}~\frac{\epsilon_{2}}{a_{2}+}~\cdots\frac{\epsilon_{n}}{a_{n}}\cdots$$ of $x$ with maximum $+1$ is obtained as follows:
 	$$b=\left\{
 	\begin{array}{ll}
 	\lfloor p^l x\rfloor, &\mbox{if } 							
 	
 	(\lfloor p^l x\rfloor+1,p)\neq1\\\\
 	\lfloor p^l x\rfloor+1, &\mbox{if } 							
 	(\lfloor p^l x\rfloor,p)\neq1\\\\
 	
 	\lfloor p^l x\rfloor,& \mbox{ if } (\lfloor p^l x\rfloor,p)=1=(\lfloor p^l x\rfloor+1,p)  \textnormal{ and }  x<	\frac{\lfloor p^l x\rfloor}{p^l}\oplus\frac{\lfloor p^l x\rfloor+1}{p^l}\\\\
 	
 	
 	
 	
 	\lfloor p^l x\rfloor+1, &  \mbox{ if } (\lfloor p^l x\rfloor,p)=1=(\lfloor p^l x\rfloor+1,p) \textnormal{ and } 	x>	\frac{\lfloor p^l x\rfloor}{p^l}\oplus\frac{\lfloor p^l x\rfloor+1}{p^l}.
 	\end{array}
 	\right.$$	
 	
  Set $y_1=p^lx-b$, 
 	%\begin{eqnarray}
 	%\textnormal{ where }
 	\begin{enumerate}
 		\item   $\ep_{i}=\sign (y_i);$\\
 		\item \begin{enumerate}
 			\item Suppose $1/|y_i|\in\N$ then 
 			$$a_i=\left\{ \begin{array}{lr}
 			\frac{1}{|y_i|}-1, & 	\textnormal{if } \frac{1}{|y_i|}-1\not\equiv -\ep_i p_{i-2}p_{i-1}^{-1}\mod p\\\\
 			1/|y_i|, & \textnormal{otherwise}.
 			\end{array}\right.$$
 			
 		
 			\item Suppose $1/|y_i|\not\in\N$ then 
 			$$a_i=\left\{ \begin{array}{lr}
 			\lfloor (\frac{1}{|y_i|})\rfloor, & 	\textnormal{if } \lfloor (\frac{1}{|y_i|})\rfloor\not\equiv -\ep_i p_{i-2}p_{i-1}^{-1}\mod p\\\\
 			\lfloor (\frac{1}{|y_i|}+1)\rfloor, & \textnormal{otherwise}.
 			\end{array}\right.$$
 			
 		\end{enumerate}
 		
 		\item    $y_{i+1}=\frac{1}{|y_i|}-a_i$. 
 	\end{enumerate}
 	\end{corollary}
 	\subsection{Non-uniqueness of $\f_{p^l}$-Continued fraction of $x\in\Q\setminus\mX_{p^l}$}
 In this subsection, we show that an $\f_{p^l}$-continued fraction expansion with maximum $+1$ of an element of $\Q\setminus\mX_{p^l}$  is not unique. In fact,  there are exactly two such expansions.
 	%	\begin{example}\label{maximumflips} Recall Example \ref{manyexpansions} for all  $\f_5$-continued fraction expansions of $11/40$. We see that $$\frac{1}{0+}~\frac{5}{1+}~\frac{1}{2+}~\frac{1}{1+}~\frac{1}{1+}~\frac{1}{1}$$
 	%	is the $\f_5$-continued fraction expansion of $11/40$ with maximum $+1.$
 %	\end{example}
 	
  Observe that 
  	$x\in\Z$ if and only if $x=\frac{p^l\lfloor{x}\rfloor+1}{p^l}\oplus \frac{p^l\lfloor{x}\rfloor-1}{p^l}.$ For $0\le k\le l-1,$ set $$\frac{1}{p^k}\dot{\Z}=\{a/p^{k}: a\in\Z \textnormal{ and } \mathrm{gcd}(a,p)=1\}.$$ 
  	For $0\le k\le l-1,$
  		$x=a/p^k\in\frac{1}{p^k}\dot{\Z}$ if and only if $x=p^{l-k}a/p^l=\frac{p^{l-k}a+1}{p^l}\oplus \frac{p^{l-k}a-1}{p^l}.$	
  		
  	We adopt $\frac{1}{p^k}\dot{\Z}$ as a notation for the set $\{a/p^k:a\in\Z \textnormal{ and } \mathrm{gcd}(a,p)=1\}$ to differentiate it from $\frac{1}{p^k}{\Z}=\{a/p^k:a\in\Z\}.$
  	Here we record a lemma to generalize the above observations for each rational number which is not in $\mX_{p^l}.$
  \begin{lemma}\label{fareytwoways}
  	Let $a/b$ and $c/d$ be adjacent vertices in the Farey graph with $0<d<b.$ Suppose a reduced rational $x/y$ is adjacent to $a/b$ with $0<y<b$ then either $x/y=c/d$ or $x/y=(a-c)/(b-d).$
  \end{lemma}	
  \begin{lemma}\label{uniquefareysum} 
  	Suppose $x=r/(p^ks)\in\Q\setminus\mX_{p^l}$ with $p\nmid s$, $\mathrm{gcd}(r,p^ks)=1$ and $0\le k<l$. Then there exists a unique pair of vertices $R_1,R_2\in\mX_{p^l}$ such that $R_1\sim_{p^l} R_2$, and $r/(p^ks)=R_1\oplus R_2$, where  $R_i=r_i/(p^ls_i)$ with $\mathrm{gcd}(r_i,p^ls_i)=1$ for $i=1,2$ and  $0<s_1\le s_2$.
  	%$3p=r_1+ r_2$ and  $3q=3s_1+ 3s_2$, where $R_i=r_i/3s_i, i=1,2.$
  \end{lemma}
  \begin{proof}
 It is enough to show that the following system has a solution
  	\begin{eqnarray*}
  	r_1+r_2&=& p^{l-k}r\\
  	s_1+s_2&=& s
  	\end{eqnarray*}
such that $	r_1s_2-r_2s_1=\pm1.$ Since $\mathrm{gcd}(p^{l-k}r,s)=1$, there exist integers
 $t,m,$ with $0<m<s$ so that $mp^{l-k} r-t s=\pm1.$ If $m\le s-m$, then set $r_1=t$ and $s_1=m$, else set $r_1=p^{l-k}r-t$ and $s_1=s-m.$
 The uniqueness follows from Lemma \ref{fareytwoways} and the fact that $\f_{p^l}$ is a subgraph of the Farey graph.
  \end{proof}
\begin{notation}
	For $x\in \Q\setminus \mX_{p^l}$, we will continue to use the notation $R_1, R_2$ with the properties stated in Lemma  \ref{uniquefareysum}. Let $N_x\in\N\cup\{0\}$ be such that the distance of $R_1$ from $\infty$ along the well directed path  with maximum direction changing edges is $N_x+1$.
\end{notation}

 By putting $s=2$ in Lemma \ref{uniquefareysum}, we get the following corollary:
  \begin{corollary}\label{coro_halfinteger}
  	Let $p>2$ be a prime and  $\frac{1}{2p^k}\dot{\Z}=\{a/(2p^{k}):a\in\Z \textnormal{ and }\mathrm{gcd}(a,2p)=1\}$ where $0\le k<l$.  Suppose $x\in\frac{1}{2p^k}\dot{\Z}$, then there exists a unique integer $t$ such that $x=t/p^l\oplus(t+1)/p^l.$
  \end{corollary}
  Now set \begin{equation}\label{Bp}\mathcal{B}_{p^l}=\left\{
  \begin{array}{ll} 
  \cup_{k=0}^{l-1}\frac{1}{2^k}\dot{\Z}, & \textnormal{ if } p=2\\\\
  
  \cup_{k=0}^{l-1}(\frac{1}{p^k}\dot{\Z}\cup\frac{1}{2p^k}\dot{\Z}), & \textnormal{ if } p\ne2.
  \end{array}\right.
   \end{equation}
  
 \begin{corollary}\label{dist_B_p}
If $x\in\mathcal{B}_{p^l},$ then $N_x=0.$	
 \end{corollary}
  \begin{proposition}\label{forNx}
  For $0\le k\le l-1,$	suppose $x=r/(p^ks)\in\Q\setminus\mX_{p^l}$ and $x=R_1\oplus R_2$. Let 
  	$\infty\to P_0\to P_1\to\cdots \to P_{N_x-1}\to P_{N_x}=R_1$ be the  well directed path from $\infty$ to $R_1$ with maximum direction changing edges, 
   then the  well directed path from $\infty$ to $R_2$ with maximum direction changing edges is $$\infty\to P_0\to P_1\to\cdots \to P_{N_x-1}\to P_{N_x}=R_1\to R_2.$$
   Further, 	the path from $\infty$ to $R_2$ extends to a well directed path from $\infty$ to $x.$ 
  \end{proposition}

%  \begin{proof}
%  	The proof is by induction on the distance of $R_1$ from $\infty$ through the well directed path with maximum direction changing edges. If the distance of $R_1$ from $\infty$ is one, so that $N_x=0,$ there is nothing to prove. If the distance of $R_1$ from $\infty$ is more than one then by Corollary \ref{dist_B_p}, $x\not\in\mathcal{B}_{p^l}$. Suppose the well directed paths with maximum direction changing edges from $\infty$ to $R_1$ and $R_2$ are respectively given by
%  	
%  	\begin{center}
%  		$	\infty\to P_0\to P_1\to\cdots\to P_{N_x-1}\to R_1$
%  	\end{center}
%  	and
%  	\begin{center}
%  		$\infty\to P_0'\to P_1'\to\cdots\to P_{N_x-1}'\to \cdots\to R_2,$
%  	\end{center}
%  	where $P_0=P_0'$ (since $R_1\sim_{p^l} R_2$ and $x\not\in\mathcal{B}_{p^l}$).  Let $\gamma\in \textnormal{PSL}(2,\Z)$ be such that
%  	$\gamma(P_0)=\infty.$ 
%  	Then $\gamma(x)=\gamma(R_1)\oplus \gamma(R_2)$ so that well directed paths with maximum direction changing edges from $\infty$ to $\gamma(R_1)$ and $\gamma(R_2$) are 
%  	$$\infty\to \g(P_1)\to\cdots\to \g(P_{N_x-1})\to \g(R_1)$$ and
%  	$$\infty\to \g(P_1')\to\cdots\to \g(P_{N_x-1}')\to \cdots\to \g(R_2),$$
%  	respectively. By induction hypothesis
%  	$$\gamma(P_i)=\gamma(P_i'),~1\le i\le N_x-1,$$
%  	and hence $P_i=P_i'$ for $0\le i\le N_x-1.$
%  \end{proof}
  

 

%  \begin{proof}
%  	The path with maximum direction changing edges from $\infty$ to $R_2$ is  given by
%  	$$\infty\to P_0\to P_1\to\cdots\to P_{N_x-1}\to  R_1\to  R_2\ominus R_1\to R_2.$$
%  Without loss of generality, we can assume that
%  	$R_1<R_2<R_2\ominus R_1,$
%  	so that $x$ lies between $R_1$ and $R_2$. Note that $R_2=P_{N_x+2}$ lies on the first semicircle emanating from $P_{N_x+1}$. Then any path from $R_2$ (extending the above path) in the direction of $x$ is not well directed. Hence the result follows.
%  \end{proof}
  
  \begin{remark}  \label{longest cf1}  We have seen that $x\in\R$ has a finite $\f_{p^l}$-continued fraction if and only if $x\in\mX_{p^l}.$  Suppose $x\in\Q\setminus\mX_{p^l}$ so that its $\f_{p^l}$-continued fraction is infinite. The value of the infinite continued fraction $\frac{1}{2+}~\frac{-1}{2+}~\frac{-1}{2+}\cdots$ is 1. Here, we write two infinite $\f_{p^l}$-continued fraction expansions of $x\in\Q\setminus\mX_{p^l}.$ Suppose $x\not\in\mathcal{B}_{p^l}$. Then $N_x>0$ and the $\f_{p^l}$-continued fractions with maximum $+1$ of $R_1$ and $R_2$ are given by
%  	\begin{eqnarray*}
%  		R_1&= &\frac{1}{0+}~\frac{p^l}{b+}~\frac{\epsilon_{1} }{a_{1}+}~\frac{\epsilon_{2}}{a_{2}+}~\cdots\frac{\epsilon_{N_x}}{a_{N_x}};\\
%  		R_2&=& \frac{1}{0+}~\frac{p^l}{b+}~\frac{\epsilon_{1} }{a_{1}+}~\frac{\epsilon_{2}}{a_{2}+}~\cdots\frac{\epsilon_{N_x}}{a_{N_x}+}~\frac{\ep_{N_x+1}}{a_{N_x+1}},\\
%  	\end{eqnarray*}
  
  $$	R_1= \frac{1}{0+}~\frac{p^l}{b+}~\frac{\epsilon_{1} }{a_{1}+}~\frac{\epsilon_{2}}{a_{2}+}~\cdots\frac{\epsilon_{N_x}}{a_{N_x}} {\text{ and }}
  	R_2= \frac{1}{0+}~\frac{p^l}{b+}~\frac{\epsilon_{1} }{a_{1}+}~\frac{\epsilon_{2}}{a_{2}+}~\cdots\frac{\epsilon_{N_x}}{a_{N_x}+}~\frac{\ep_{N_x+1}}{a_{N_x+1}},\\
  $$
  	so that the following expressions are  $\f_{p^l}$-continued fractions of $x$
  	
  	\begin{subequations}\label{rationalnotinX}
  		\begin{equation}\label{rational1}
  		\frac{1}{0+}~\frac{p^l}{b+}~\frac{\epsilon_{1} }{a_{1}+}~\frac{\epsilon_{2}}{a_{2}+}~\frac{\ep_3}{a_3+}~\cdots\frac{\epsilon_{N_x}}{a_{N_x}+}~\frac{\ep_{N_x+1}}{a_{N_x+1}+y_{N_x+2}},
  		\end{equation}
  		\begin{equation}\label{rational2}
  		\frac{1}{0+}~\frac{p^l}{b+}~\frac{\epsilon_{1} }{a_{1}+}~\frac{\epsilon_{2}}{a_{2}+}~\cdots\frac{\epsilon_{N_x}}{a_{N_x}+}~\frac{\ep_{N_x+1}}{(a_{N_x+1}+2)-y_{N_x+2}},
  		\end{equation}
  			where $y_{N_x+2}=\frac{1}{2+}~\frac{-1}{2+}~\frac{-1}{2+}\cdots.$	 	\end{subequations} 	
  \label{longest cf2}
  	Suppose $x\in \frac{1}{p^i}\dot{\Z} $ for some $0\le i< l,$ then
  	\begin{subequations}
  		\begin{equation}\label{interger_cfr1}
  		x=\frac{1}{0+}~\frac{p^l}{(\lfloor{p^lx}\rfloor-1)+y};
  		\end{equation}
  		\begin{equation}\label{interger_cfr2}
  		x=\frac{1}{0+}~\frac{p^l}{(\lfloor{p^lx}\rfloor+1)-y},
  		\end{equation}
  	\end{subequations}
  	where $y=\frac{1}{2+}~\frac{-1}{2+}~\frac{-1}{2+}\cdots.$	
    	Similarly,  if $x\in \frac{1}{2p^i}\dot{\Z}$ for some $0\le i<l$ with  $p\ne2$, then
  	\begin{subequations}
  		\begin{equation}\label{halfinterger_cfr1}
  		x=\frac{1}{0+}~\frac{p^l}{\lfloor{p^lx}\rfloor+}~\frac{1}{3-y};
  		\end{equation}
  		\begin{equation}\label{halfinterger_cfr2}
  		x=\frac{1}{0+}~\frac{p^l}{(\lfloor{p^lx}\rfloor+1)+}~\frac{-1}{3-y},
  		\end{equation}
  	\end{subequations}
  	where $y=\frac{1}{2+}~\frac{-1}{2+}~\frac{-1}{2+}\cdots.$	
  \end{remark}

Using the fact that edges do not cross in $\f_{p^l}$, one can observe the following result.
  \begin{proposition}
  		Every $x\in\Q\setminus\mX_{p^l}$ has exactly two $\f_{p^l}$-continued fraction expansions  of $x$ with maximum $+1.$
  	\end{proposition}
%  	\begin{proof}
%  		First, we claim that the continued fraction expansions of $x\in\Q\setminus\mX_{p^l}$ given by Remark \ref{longest cf1} are $\f_{p^l}$-continued fractions  of $x$ with maximum $+1$ in each case. Since arguments are similar, we deal with only continued fractions described by \eqref{rational1} and \eqref{rational2} in Remark \ref{longest cf1}. 
%  		Suppose $P_i$ and $P_i'$ denote the $i$-th convergent of continued fractions given by Equations \eqref{rational1} and \eqref{rational2}, respectively. Then $$ \frac{1}{0+}~\frac{p^l}{b+}~\frac{\epsilon_{1} }{a_{1}+}~\frac{\epsilon_{2}}{a_{2}+}~\cdots\frac{\epsilon_{i}}{a_{i}}$$ is the  $\f_{p^l}$-continued fraction of $P_{N_x}$ with maximum $+1$. Since $P_{N_x+2}\ominus P_{N_x+1}=R_{1}\oplus R_{2}\not\in\mX_{p^l},$ $P_{N_x+2}$ has its $\f_{p^l}$-continued fraction with maximum $+1$. Now observe that for $i>N_x+1$, $$P_{i+1}\ominus P_i=P_{N_x+2}\ominus P_{N_x+1}=R_1\oplus R_2,$$
%  		so that $P_{i+1}\ominus P_i\not\in\mX_{p^l}$. The only alternative well directed path from $\infty$ to $P_{i+1}$ via $P_i$ is through $P_{i+1}\ominus P_i$ (provided $P_{i+1}\ominus P_i\in\mX_{p^l}$). But $P_{i+1}\ominus P_i\notin\mX_{p^l}$ and so there is no alternative path. Thus, $ \frac{1}{0+}~\frac{p^l}{b+}~\frac{\epsilon_{1} }{a_{1}+}~\frac{\epsilon_{2}}{a_{2}+}~\cdots\frac{\epsilon_{i}}{a_{i}}$ is the $\f_{p^l}$-continued fraction of $P_i$ with maximum $+1$   for each $i\ge N_x.$
%  		A similar argument holds for each $P_i'$ and we achieve our first claim.
%  		
%  		Now we claim that no other $\f_{p^l}$-continued fraction converging to $x$ is with maximum $+1$. Suppose an $\f_{p^l}$-continued fraction, other than the above two, with the sequence of convergents $Q_i$ converges to $x$. Suppose $i\in\N$ is such that $P_{i}=P_i'=Q_{i}$, then $P_j=P_j'=Q_j$ for all $j\le i.$ Suppose $i_0=max\{i~|~P_{i}=P_i'=Q_{i}\}$, then $i_0\le N_x$. Then $x$ lies between $Q_{i_0+1}$ and $P_{i_0+1}$. Without loss of generality, we assume that $Q_{i_0+1}<P_{i_0+1}$ so that $Q_{i_0+1}<x<P_{i_0+1}$.
%  		Then $Q_{i_0+1}= P_{i_0}\oplus P_{i_0+1}$ and maximality of direction changing edges in a path via $Q_{i_0+1}$ forces that the path
%  		$$\infty\to P_0\to\cdots\to P_{i_0}\to P_{i_0+1}\to Q_{i_0+1}\to Q_{i_0+2}\to\cdots\to x$$
%  		is not well directed. In that case, $Q_{i_0+2}=P_{i_0+1}\oplus Q_{i_0+1}=(\oplus_2 P_{i_0+1})\oplus P_{i_0},$ and again, the path
%  			$$\infty\to P_0\to\cdots\to P_{i_0}\to P_{i_0+1}\to Q_{i_0+2}\to Q_{i_0+3}\to\cdots\to x$$
%  			is not well directed. Repeating the process and
%  		 using Lemma \eqref{k-thfareysum}, there exists a positive integer $k$ such that
%  		\begin{equation}\label{newpathQi}
%  		\infty\to P_0\to\cdots\to P_{i_0}\to P_{i_0+1}\to Q_{i_0+k}\to Q_{i_0+k+1}\to\cdots\to x
%  		\end{equation}
%  		is well directed. Note that, 
%  		$$\infty\to P_0\to\cdots\to P_{i_0}\to P_{i_0+1}\to Q_{i_0+k}\to\cdots\to Q_{i_0+k+r}$$
%  		have more direction changing edges than
%  		$$\infty\to Q_0\to \cdots\to Q_{i_0}\to Q_{i_0+1}\to\cdots\to Q_{i_0+k}\to\cdots\to Q_{i_0+k+r},$$
%  		 for each $r\ge 0$ and we get a contradiction. 	\end{proof}
  	
   	
  	\section{Best Approximations and Convergents}
  	Suppose $x$ is a real number. A reduced rational number $u/v$ is a best rational approximation of $x$ if $|vx-u|<|v'x-u'|$ for every $u'/v'\ne u/v$ with $0<v'\le v.$ %Note that  $|vx-u|<|v'x-u'|\implies|x-\frac{u}{v}|<|x-\frac{u'}{v'}|$ as $0<v'\le v.$
  	 Best rational approximations of a real number are described by the convergents of the regular continued fraction. Here, we introduce best  $\mX_{p^l}$-approximations  of a real number.
  
  	\begin{defi}
  		A rational number $u/v\in \mX_{p^l}$ is called a {\it best $\mX_{p^l}$-approximation of $x\in\R$}, if for every $u'/v'\in\mX_{p^l}$ different from $u/v$ with
  		$0< v' \le v$, we have $|vx-u|<|v'x-u'|$.
  	\end{defi}
  	
%  	Suppose $x\in\mX_{p^l}$ and its $\f_{p^l}$-continued fraction with maximum $+1$ is given by
%  	$$ \frac{1}{0+}~\frac{p^l}{b+}~\frac{\epsilon_{1} }{a_{1}+}~\frac{\epsilon_{2}}{a_{2}+}~\cdots\frac{\epsilon_{n}}{a_{n}}.$$ 
%  	If $\ep_n/a_n=1/1,$ then the continued fraction 
%  	\begin{equation}\label{length_n-1}
%  	\frac{1}{0+}~\frac{p^l}{b+}~\frac{\epsilon_{1} }{a_{1}+}~\frac{\epsilon_{2}}{a_{2}+}~\cdots\frac{\epsilon_{n-1}}{a_{n-1}+1}
%  	\end{equation}
%  	is also an $\f_{p^l}$-continued fraction expansion of $x.$ We know that 
%  	$a_{n-1}\not\equiv a \mod p,$  where $a\equiv-\ep_{n-1}p^{-1}_{n-2}p_{n-3}\mod p$. Suppose $a_n=1$ and $\ep_n=1,$ then $P_n=P_{n-1}\oplus P_{n-2}$ so that $p_n=(a_{n-1}+1)p_{n-2}+\ep_{n-1}p_{n-3}.$ Thus  $a_{n-1}+1\not\equiv a\mod p$. 
%  	\begin{proposition}
%  		Let $x\in\mX_{p^l}$ be such that its $\f_{p^l}$-continued fraction with maximum $+1$ is given by
%  		$$ \frac{1}{0+}~\frac{p^l}{b+}~\frac{\epsilon_{1} }{a_{1}+}~\frac{\epsilon_{2}}{a_{2}+}~\cdots\frac{\epsilon_{n}}{a_{n}},$$ where  $\ep_n/a_n=1/1.$ 
%  		 Suppose the number of $\ep_i$ taking value $+1$, for $i\ge 2,$  is $k+1$. Then it has a unique $\f_{p^l}$-continued fraction expansion having exactly  $k$ many $+1$ and the last partial quotient is not $1/1.$
%  	
%  	\end{proposition}
%  	\begin{proof}Observe that $$ \frac{1}{0+}~\frac{p^l}{b+}~\frac{\epsilon_{1} }{a_{1}+}~\frac{\epsilon_{2}}{a_{2}+}~\cdots\frac{\epsilon_{n-1}}{(a_{n-1}+1)}$$
%  		is an $\f_{p^l}$-expansion of $x$ with $k$ many partial numerators as $+1$ and the last partial denominator is not $1/1.$  The remaining proof follows from Lemma \ref{twoways}.
%  		\end{proof}
%  			
%  	An $\f_{p^l}$-continued fraction with maximum $+1$ not ending with $1/1$, is called a {\it proper} $\f_{p^l}$-continued fraction. Obviously, 
%  	for $x\in\R\setminus \mX_{p^l}$, 
%  	every $\f_{p^l}$-continued fraction with maximum $+1$ is proper (since the continued fraction is infinite). Thus, for an irrational number there is a unique proper $\f_{p^l}$-continued fraction expansion.
  	\begin{lemma}\label{sandwich}
  		Let $\{\frac{p_i}{q_i}\}_{i\ge0}$ be a  sequence of $\f_{p^l}$-convergents of $x\in\R$. If $u/v\in\mX_{p^l}$, then 
  		there exists a unique
  		solution $(\alpha,\beta)\in\Z\times\Z$ of the following system of equations 
  		\begin{equation}\label{systemofequations}
  		\begin{pmatrix} u\\v \end{pmatrix}= \alpha \begin{pmatrix} p_{n+1}\\q_{n+1}\end{pmatrix}+\beta\begin{pmatrix} p_n\\q_n\end{pmatrix}.
  		\end{equation}
  	\end{lemma}
  	\begin{proof} Since the determinant of the coefficient matrix of Eqs.\eqref{systemofequations} is $\pm p^l\ne0$ and the system has a unique solution. In fact, $\alpha=\pm (uq_n-vp_n)/p^l$ and
  		$\beta=\pm (vp_{n+1}-uq_{n+1})/p^l$. Note that $p^l$ divides $v,q_n$ and $q_{n+1}$ so that $(\alpha,\beta)\in\Z\times\Z$.
  	\end{proof}
  	%	The following propositions follow from Theorem \ref{distinctconvergents} and \cite[Proposition 6.1]{seema}.
  	
  	\begin{proposition}\label{appposition}
  		Let $x$ and $\{p_i/q_i\}_{i=0}^M$ be the sequence of its $\f_{p^l}$-convergents with maximum $+1.$ Then $u/v\in \mX_{p^l}$ with $q_{M-1}\le v<q_M$ is not its best $\mX_{p^l}$-approximation. 
  	\end{proposition}
  \begin{proof}
  	By Lemma \ref{sandwich},  $u/v=(p_{M}-p_{M-1})/(q_{M}-q_{M-1})$ (since $v>0,$ ). Thus, $|vx-u|=\frac{p^l}{q_M}=|q_{M-1}x-p_{M-1}|$. Note that $q_{M-1}<v$. Therefore, $u/v$ is not a best approximation.
  \end{proof}
  		\begin{proposition}\label{tail1}
  		Let $y_i$ be the $i$-th fin  of  $\f_{p^l}$-continued fraction of $x\not\in\Q\setminus\mX_{p^l}$ with maximum +1 . Then $|y_i|<1$ for every $i\ge1$.
  		\end{proposition}
  		\begin{proof}
  			By Theorem \ref{distinctconvergents}, for each $i\ge1,$ $|y_i|\le1$  and if $|y_i|=1$ for some $i\ge1$ then $x$ is a rational number. 
  		  By Corollary \ref{algoformaximumflips}, $y_{i+1}=1/|y_i|-a_i$ so that  $a_i=2$ for being $\f_{p^l}$-continued fraction with $+1$ which is infinite. We get a contradiction as $x\not\in\Q\setminus\mX_{p^l}.$
  		\end{proof}
  	
  		\begin{corollary}\label{tail2} 
  			 Let  $1/y_i=\pm2$ be the $i$-th fin  of the $\f_{p^l}$-continued fraction of $x\notin\Q\setminus \mX_{p^l}$ with maximum +1  for some $i\ge1$. Then $x\in\mX_{p^l}$. 
  		\end{corollary}
  		\begin{proof}
  			By Proposition \ref{tail1}, $y_{i+1}\ne\pm1$ and so $a_i=2.$ Thus, $y_{i+1}=0$ and $x\in\mX_{p^l}$. 
  		\end{proof}
  	
  		\begin{proposition}\label{rationalwithfareysum}
  			Suppose $x\in \Q\setminus(\mX_{p^l}\cup \mathcal{B}_{p^l}),$ where $\mathcal{B}_{p^l}$ is given by Equation \eqref{Bp}. 
  			If $y_i$ is the $i$-th fin of a  $\f_{p^l}$-continued fraction of $x$ with maximum +1  and $N_x$ is as in \eqref{rational1} and \eqref{rational2}, then  $y_i\neq\pm1$ and $y_{i-1}\ne\pm1/2$ for each $1\le  i\le N_x+1$ .
  			\end{proposition}
  			\begin{proof}We can see that $y_{N_x+1}\ne\pm1$ in each case.	Suppose $y_i=\pm1$ for some $i< N_x+1$. Then by Theorem \ref{algoformaximumflips}, $a_i=1$, (with $\ep_i=1$)  or $a_i=2$. If $a_i=1$, then $y_{i+1}=0$ which contradicts that $x\not\in\mX_{p^l}$. If $a_i=2$, then $y_{i+k}=-1, \forall k\ge1$  which contradicts that $y_{N_x+1}\neq\pm1$. The other part of the proposition can be proved by a similar argument.
  			\end{proof}
  	
  		
  		\begin{lemma}\label{bestappX} 
  			If $u/v\in\mX_{p^l}$ is  a best $\mX_{p^l}$-approximation of $r/(p^ls)\in\mX_{p^l}$, then $v\le p^ls$.
  		\end{lemma}
  		\begin{proof}
  			Suppose $v>p^ls$. Then $|vx-u|\ge0=|p^ls x-r|$ so that we get a contradiction. 
  		\end{proof}
  		\begin{lemma}\label{noapproximation}
  			Suppose $x\in \Q\setminus\mX_{p^l}$ with $x=R_1\oplus R_2$ as in Lemma \ref{uniquefareysum}. Let $u/v$ be a best $\mX_{p^l}$-approximation of $x$. Then $v\le p^ls_1$, where $R_1=r_1/(p^ls_1)$ and $R_2=r_2/(p^ls_2)$. Moreover, $x\in\mathcal{B}_{p^l}$ has no best $\mX_{p^l}$-approximation.
  		\end{lemma}	
  		\begin{proof} Suppose $x=r/(p^is)\in \Q\setminus\mX_{p^l}$ with $\mathrm{gcd}(r,s)=1=\mathrm{gcd}(r,p)$ and $0\le i\le l-1.$ 
  			Suppose $v>p^ls_1$. Observe that $p^{l-i}r={r_1+r_2}$ and $p^ls={p^ls_1+p^ls_2}$. Hence $|vx-u|\geq1/s=|p^ls_1 x-r_1|$ so that $u/v$ is not a best $\mX_{p^l}$-approximation of $x$.
  			
  		For $x=a/p^i\in\cup_{i=0}^{l-1}\frac{1}{p^i}\dot{\Z}$, the result follows by the observation: $p^lx=p^{l-i}a\in\Z$ so that $1=|p^l x -(p^lx-1)|=|p^l x - (p^lx+1)|.$ 
  			Similarly, let $x=a/(2p^i)\in\cup_{i=0}^{l-1}\frac{1}{2p^i}\dot{\Z},$ then $p^l x=(2m+1)/2$ for some non zero integer $m$ co-prime to $p$ so that $\lfloor p^lx\rfloor=m$. Thus the result follows by observing that $1/(2p^i)=|p^lx-\lfloor p^l x\rfloor |=|p^lx-(\lfloor p^lx\rfloor +1)|.$\end{proof}
  		\begin{example}
  			Let $x=1/5$. Then $x=\frac{4}{25}\oplus\frac{6}{25}$ in $\f_{25}$ so that $|25x-4|=1=|25x-6|.$ Therefore, $1/5$ has no best $\mX_{25}$-approximation.
  		\end{example}
  		
  		\begin{corollary}\label{constantdifference}
  			Suppose $x=r/s\in\Q\setminus\mX_{p^l}$ and $\{p_i/q_i\}$ is the sequence of convergents of one of the $\f_{p^l}$-continued fractions of $x$ with maximum $+1$. Suppose  $N_x$ is as in Remark \ref{longest cf1}. 
  			Then for $k\ge N_x$,	$|q_kx-p_k|=1/s.$
  		\end{corollary}
  		
  			
  		\begin{example}	Recall Example \ref{manyexpansions}, the  $\f_{5}$-continued fraction expansion of $11/40$ with maximum +1 is
  			$\frac{1}{0+}~\frac{5}{1+}~\frac{1}{2+}~\frac{1}{1+}~\frac{1}{2}$ and the corresponding convergents are given by
  			$$\Big\{\frac{1}{5},\frac{3}{10},\frac{4}{15},\frac{11}{40}\Big\}.$$
  			In fact, this list of convergents is the complete set of best $\mX_5$-approximations of $11/40$. 
  			
  		\end{example}	
  		\begin{example}
  			The $\f_5$-continued fraction expansion of $1/\pi$ with maximum +1 is
  				$$\frac{1}{0+}~\frac{5}{2+}~\frac{-1}{2+}~\frac{1}{2+}~\frac{1}{5}~\frac{-1}{2}~\frac{-1}{2+(\frac{355+113\pi}{78\pi-245})}$$
  			The first six $\f_5$-convergents are
  			$\Big\{\frac{2}{5},\frac{3}{10}, \frac{8}{25}, \frac{43}{135}, \frac{78}{245}, \frac{113}{355} \Big\}.$
  			One can verify that every element of this set is a best $\mX_5$-approximation of $1/\pi$.
  		\end{example}	
  \begin{example}
  The two $\f_5$-continued fraction expansions of ${7}/{27}$ with maximum $+1$  are 
  	$$\frac{1}{0+}~\frac{5}{1+}~\frac{1}{3+}~\frac{1}{2+}~\frac{1}{1+}~\frac{1}{3+y}
  \textnormal{ and }
  	\frac{1}{0+}~\frac{5}{1+}~\frac{1}{3+}~\frac{1}{2+}~\frac{1}{1+}~\frac{1}{1+}~\frac{1}{2+y},$$
  	where $y=\frac{-1}{2+}~\frac{-1}{2+}~\frac{-1}{2+}\cdots.$
  	 The corresponding convergents are
  	\begin{equation*}
  	\left\{\frac{1}{5},\frac{4}{15},\frac{9}{35},\frac{13}{50},\frac{48}{185},\cdots\right\}~\mathrm{and}~
  	\left\{\frac{1}{5},\frac{4}{15},\frac{9}{35},\frac{13}{50},\frac{22}{85},\cdots\right\}, 
  	\end{equation*}
  	respectively. Observe that $\{1/5,4/15,9/35,13/50\}$ is 
  	the set of common convergents of the two $\f_{5}$-continued fraction expansions of $7/27$. Again, we can see that these are the only best $\mX_5$-approximations of $7/27$. 
  \end{example} 	
  	We have a couple of theorems relating convergents of an $\f_{p^l}$-continued fraction expansion of a real number $x$ with maximum $+1$ and best 	$\mX_{p^l}$-approximations of $x$.
  		
  	
  		\begin{theorem}
  			Every best $\mX_{p^l}$-approximation of a real number is a convergent of its $\f_{p^l}$-continued fraction with maximum +1.
  		\end{theorem}
  		\begin{proof} 
  		Suppose $x\in\R.$	Let $\{P_i=\frac{p_i}{q_i}\}_{i\ge0}$ be a sequence of convergents of an $\f_{p^l}$-continued fraction of $x$ with maximum $+1$. Suppose $y_i$ is the $i$-th fin of the same $\f_{p^l}$-continued fraction of $x$. For convenience, set $x_i=1/|y_i|$.
  			Let $u/v\in\mX_{p^l}$ be a best $\mX_{p^l}$-approximation of $x$. Then for some $n \geq 1$,
  			$q_{n-1} \le v< q_{n}.$ If $x\in\mX_{p^l}$, by Lemma \ref{bestappX}, $v\le q_M$, where $x=p_M/q_M$. Observe that if $v=q_M,$ then $u=p_M$. If $v\ne q_M,$ then by Proposition \ref{appposition}, $v\le q_{M-1}$ so that $n\le M-1$. By Proposition \ref{tail1}, $x_{n+1}>1$ for $n\le M-1$. 
  			For $x\notin \mX_{p^l}$, we note that $x_i\ne1~\forall i\ge1$ unless $x\in\Q\setminus\mX_{p^l}$ and if $x\in\Q\setminus\mX_{p^l}$, then by Lemma \ref{noapproximation}, $N_x \geq 1$ and $n<N_x$ so that $x_{n+1}\ne1$ (Proposition \ref{rationalwithfareysum}). Thus, we can assume that $x_{n+1}>1$ in each case.
  			
  		 Suppose $u/v$ is not an $\f_{p^l}$-convergent of the  $\f_{p^l}$-continued fraction of $x$ with maximum $+1$.  By Lemma \ref{sandwich}, 
  			$v=\beta q_{n-1}+\alpha q_n, u=\beta p_{n-1}+\alpha p_n$ for $\alpha,\beta\in\Z$ with $|\beta|\ge1$ and by Theorem \ref{distinctconvergents}, 
  			\begin{eqnarray}
  				|vx-u|=\dfrac{p^l |\beta x_{n+1}-\ep_{n+1}\alpha|}{x_{n+1}q_n+\ep_{n+1}q_{n-1}}.\label{exp1}
  			\end{eqnarray}
  			Note that $|q_nx-p_n|=\dfrac{p^l }{x_{n+1}q_n+\ep_{n+1}q_{n-1}}$. We will show
  			that the numerator in \eqref{exp1} is strictly bigger than $p^l$ which contradicts that $\frac{u}{v}$ is a best $\mX_{p^l}$-approximation of $x$.
  			
  		\noindent	\textbf{Case 1.} Suppose $|\beta|=1$. Since $v>0,$  $u/v=(p_{n}-p_{n-1})/(q_{n}-q_{n-1})=P_{n}\ominus P_{n-1}.$ Note that the continued fraction is with maximum +1 and $u/v$ is not a convergent implies that $x$ lies between $P_n$ and $P_{n-1}$ so that $\ep_{n+1}=1$, which gives that
  			$$|\beta x_{n+1}-\ep_{n+1}\alpha|=|x_{n+1}+\ep_{n+1}|>1.$$
%  			equality holds if and only if $x_{n+1}=2.$ Now suppose $x_{n+1}=2$ then by Proposition \ref{tail2} and Proposition \ref{rationalwithfareysum}, $x\in\mX_{p^l}$ and $P_{n+1}=x.$ Consequently,
%  			$|vx-u|=|q_nx-p_n|$
%  			which contradicts that $u/v$ is a best $\mX_{p^l}$-approximation of $x$.
%  			
  			\textbf{Case 2.} Suppose $\beta\ge2$.
  			Then  $q_n \le v=\beta q_{n-1}+\alpha q_{n} < q_{n+1}=\epsilon_{n+1}q_{n-1}+a_{n+1}q_{n}$. Hence
  			$1-\beta < \alpha \leq a_{n+1}-1$ (since $q_{n-1}>0$). Using $|y_{n+2}|\le1$, $a_{n+1}\le x_{n+1}+1$, we have
  			$1-\beta < \alpha \leq x_{n+1}$.
  			These bounds on $\alpha$ imply $\beta x_{n+1}-\ep_{n+1}\alpha > 1$.
  			
  		\noindent	\textbf{Case 3.} Suppose $\beta\le -2$. Then
  			$0<v=\beta q_{n-1}+\alpha q_{n} < \epsilon_{n+1}q_{n-1}+a_{n+1}q_{n}$. Hence
  			$1\leq \alpha\le\ep_{n+1}-\beta+a_{n+1}-1$ (since $\frac{q_{n-1}}{q_n}<1$). 	These bounds on $\alpha$ imply $\beta x_{n+1}-\ep_{n+1}\alpha<-1$ unless  $\ep_{n+1}=-1$ and $\alpha=-\beta+a_{n+1}-2$. 
  			
  			Now,
  			suppose $\ep_{n+1}=-1$ and $\alpha=-\beta+a_{n+1}-2,$ then $u=\beta(p_n-p_{n-1})+(a_{n+1}-2)p_n$. Since $\ep_{n+1}=-1$,  $a_{n+1}\ge2.$ First, we consider $a_{n+1}\ge3$, then  $x_{n+1}\ge2$. If $x_{n+1}>2$ then  $\beta x_{n+1}-\ep_{n+1}\alpha<-1$. If $x_{n+1}=2$, then by Proposition \ref{tail2}, $x\in\mX_{p^l}$ and $P_{n+1}=x$ so that $a_{n+1}=2$ but we have considered that $a_{n+1}\ge3$. Thus inequality $\beta x_{n+1}-\ep_{n+1}\alpha<-1$ holds. Now consider the remaining case, $a_{n+1}=2,$ then $u/v=(p_{n+1}-p_n)/(q_{n+1}-q_n).$ Using a similar argument as in Case 1, we get the inequality.
  		\end{proof}
  		
  	\begin{proposition}\label{nobestappoffareysum}
  		Let $x\in\mX_{p^l}$. If $x=\lfloor p^lx\rfloor/p^l\oplus (\lfloor p^lx\rfloor+1)/p^l$, where $\mathrm{gcd}(\lfloor p^lx\rfloor,p)=1=\mathrm{gcd}(\lfloor p^lx\rfloor+1,p)$. Then $x$ has no best $\mX_{p^l}$-approximation other than itself.
  	\end{proposition}
  	\begin{proof}Suppose $r/s\in\mX_{p^l}$ is a best $\mX_{p^l}$-approximation of $x$. If $x\ne r/s$, then $s=p^l$. Observe that	$|p^lx-\lfloor p^lx\rfloor|=\frac{1}{2}=|p^lx-(\lfloor p^lx\rfloor+1)|$ and so $r/p^l$ is not a best $\mX_{p^l}$-approximation of $x.$	\end{proof}
  	\begin{example}
  		Take $3/10\in\mX_5$  with	$|5.3/10-1|=1/2=|5.3/10-2|$ so that $3/10$ has no best approximation other than itself.
  	\end{example}
  		\begin{theorem} Suppose $x\in\R$. Then
  			\begin{enumerate}
  				\item If $x\not\in\Q\setminus{\mX_{p^l}}$ and $x\ne\lfloor p^lx\rfloor/p^l\oplus (\lfloor p^lx\rfloor+1)/p^l$, then every convergent of the  $\f_{p^l}$-continued fraction of $x$ with maximum $+1$ is a best $\mX_{p^l}$-approximation of $x$.
  				
  				\item For $x\in\Q\setminus\mX_{p^l}$ or $x=\lfloor p^lx\rfloor/p^l\oplus (\lfloor p^lx\rfloor+1)/p^l$, an $\f_{p^l}$-convergent is a best $\mX_{p^l}$-approximation of $x$ if and only if it is a convergent of both  the  $\f_{p^l}$-continued fractions of $x$ with maximum $+1$.
  			\end{enumerate}
  		\end{theorem}
  		
  		
  		\begin{proof}			
  			Suppose $x\notin\Q\setminus \mX_{p^l}$ and $x\ne\lfloor p^lx\rfloor/p^l\oplus (\lfloor p^lx\rfloor+1)/p^l$.  Now let $\{\frac{p_k}{q_k}\}_{k=0}^M$ be the sequence of $\f_{p^l}$-convergents, where $M$ is finite if and only if $x\in\mX_{p^l}$.
  			Note that  $\frac{p_0}{q_0}=\frac{b}{p^l}$, where 	$b$ is given by Corollary \ref{algoformaximumflips}. This is clearly a best $\mX_{p^l}$-approximation of $x$.
  			
  		
  			Assume that, for $0 \le k \le n$, $p_{k}/q_{k}$  is a best $\mX_{p^l}$-approximation of $x$. Now we show that $p_{n+1}/q_{n+1}$ is a best $\mX_{p^l}$-approximation of $x$. When $M$ is finite and $n=M-1$, $p_M/q_M$ is a best $\mX_{p^l}$-approximation as $q_M x-p_M=0$. 	Thus, assume $n \geq 0$ is an integer with the restriction that $n<M-1$ when $M$ is finite.
  			
  			
  			For any $u/v\in\mX_{p^l}$ different from $p_{n+1}/q_{n+1}$ with $0<v\le q_{n}$, we have $|vx-u|>|q_{n}x-p_{n}|\ge|q_{n+1}x-p_{n+1}|$. Next assume $q_{n}< v\le q_{n+1}$. By Theorem \ref{distinctconvergents} (5),
  			\begin{align*}
			&|q_{n+1}x-p_{n+1}| =\frac{p^l}{x_{n+2}q_{n+1}+\epsilon_{n+2}q_{n}}. 
  	%	&	|vx-u| %&=\frac{|x_{n+2}(p_{n+1}q-q_{n+1}p)+\epsilon_{n+2}(p_{n}q-q_{n}p)|}{x_{n+2}q_{n+1}+\epsilon_{n+2}q_{n}}.
  			\end{align*}
  		By Lemma \ref{sandwich},
  			$u= \beta p_{n}+\alpha p_{n+1}$, $v= \beta q_{n}+\alpha q_{n+1}$ for some $\alpha,\beta\in\Z$. % with $|\beta|\ge 2$.
  			Thus,
  			\begin{align}|vx-u|=\frac{p^l| \beta x_{n+2}-\alpha \epsilon_{n+2}|}{x_{n+2}q_{n+1}+\ep_{n+2}q_n}.\label{scaledistance}\end{align}
  			Now, we will show that the numerator in \eqref{scaledistance} is greater than $p^l$. 	The proof of part (1) will be complete if we show that
  			\begin{equation}\label{greaterthan1}
  			| \beta x_{n+2}-\alpha \epsilon_{n+2}|>1.
  			\end{equation}
  			
  			\textbf{Case 1.} Suppose $|\beta|=1$. Then
  			$	| \beta x_{n+2}-\alpha \epsilon_{n+2}|=|x_{n+2}+\ep_{n+2}|,$
  			and $u/v=(p_{n+1}-p_n)/(q_{n+1}-q_n).$ The definition of well directed path with maximum direction changing edges forces that either $\ep_{n+2}=1$ or $P_{n+1}=x.$
  			Thus we have $|x_{n+2}+\ep_{n+2}|>1.$
  			
  			
  			\textbf{Case 2.} Suppose $\beta\ge2$. Since $v\le q_{n+1}$, we have
  			$(\alpha-1)q_{n+1}\le -\beta q_{n} <0$. Since $\alpha\in\Z$, we have
  			$\alpha\leq0$. Again, since $q_{n}< \beta q_{n}+\alpha q_{n+1}$, we have
  			$\frac{-\alpha}{\beta-1}< \frac{q_{n}}{q_{n+1}}$.
  			Hence, $\alpha> 1-\beta$ (since $q_{n}/q_{n+1}<1$). Thus we have shown $1-\beta<\alpha\le0$.
  			Using these bounds and the fact that $x_{n+2}> 1$ (by Proposition \ref{tail1}), inequality  \eqref{greaterthan1} follows.
  			
  			\textbf{Case 3.} Suppose $\beta\le -2$. Since $v>0$, $\alpha\geq 1$.
  			Since $v\le q_{n+1}$, $\frac{-\beta}{\alpha-1}\ge \frac{q_{n+1}}{q_{n}}$ so that $\alpha\le -\beta$ (since $q_{n+1}/q_{n}>1$). These bounds on $\alpha$ and $\beta$ implies inequality \eqref{greaterthan1} unless $\a=-\b$ and $\ep_{n+2}=-1$. Now suppose $\a=-\b$ and $\ep_{n+2}=-1$. Then $p/q=(p_{n+1}-p_n)/(q_{n+1}-q_n),$ we have discussed this possibility in Case 1. Thus we have $|x_{n+2}+\ep_{n+2}|>1.$
  		
  			
  			\medskip
  			To prove the second assertion of the theorem, let $x\in \Q\setminus \mX_{p^l}$. By Lemma \ref{noapproximation} and Proposition \ref{nobestappoffareysum}, if $x\in\mathcal{B}_{p^l}$ or $x=\lfloor p^lx\rfloor/p^l\oplus (\lfloor p^lx\rfloor+1)/p^l$, then $x$ has no best $\mX_{p^l}$-approximation. We assume that $x\ne\lfloor p^lx\rfloor/p^l\oplus (\lfloor p^lx\rfloor+1)/p^l$ $x\not\in\mathcal{B}_{p^l}$  so that $N_x\ge1$. 
  			Suppose $k\le N_x$. Then, by Proposition \ref{rationalwithfareysum}, we have $x_{k+1}>1$. Now the result follows by the same argument used in the first part.
  			
  		The	converse follows from Corollary \ref{constantdifference}.
  			%, suppose $p_k/q_k,$ is not a common convergent of the two $\f_{p^l}$-continued fractions of $x$ with maximum $+1$. Then $k> N_x\ge 0$ and $|q_k x-p_k|=|q_{N_x} x - p_{N_x}|$ (by Corollary \ref{constantdifference}). The remaining case $ q_0=p^l$ ($N_x=0,q_{N_x}=q_k$)   is completed by Lemma \ref{noapproximation} as $x\in\mathcal{B}_{p^l}$. 
  		\end{proof}
%  	\noindent	\textbf{Conclusion}\\
%  	\noindent	We have shown that every real number has an $\f_{p^l}$-continued fraction expansion and such an expansion of a number need not be unique. For an irrational number $x,$ the proper $\f_{p^l}$-continued fraction expansion is unique and its convergents are best $\mX_{p^l}$-approximations of $x$.  %Thus, this article generalizes the results obtained in \cite{seema2,seema}.
%\appendix  	\renewcommand\thefigure{A.\arabic{figure}}    
%%\setcounter{remark}{0}   
%
% %\counterwithin{figure}{section}  
% 
%\section{} In this appendix, we describe  a few terminologies and discuss a few diagrams which will be helpful to follow our main results.
%\setcounter{figure}{0} 
%
%By {\it Crossing of two edges}, we mean the intersection of two edges excluding the real line  (see Figure A.1) and two edges { \it meet} only on the real line (see Figure A.2).
%\vspace{-2mm}
%\begin{figure}[!h]
%	\begin{minipage}{0.495\textwidth}
%		\centering
%		\includegraphics[width=.5\linewidth]{crossing.eps}
%			\vspace{-5mm}
%		\caption{Crossing of two edges}\label{Fig:crossing}
%	\end{minipage}\hfill
%	\begin {minipage}{0.485\textwidth}
%	\centering
%	\includegraphics[width=.5\linewidth]{meeting1.eps}
%		\vspace{-5mm}
%	\caption{Meeting of two edges at $P_2$}\label{Fig:meeting}
%\end{minipage}
%\end{figure}
%
%
%\begin{figure}[!h]
%	\centering
%	\includegraphics[scale= 1.5]{crossing_over_Q.eps}\\
%	\vspace{-5mm}
%	\caption{An edge crossing over $Q$}
%\end{figure}
%Further, an edge joining $P_1$ and $P_2$ crosses over a real number $Q$ if $Q$ lies between $P_1$ and $P_2$ (see Figure A.3).
%\begin{figure}[!htb]
%	\centering
%	\includegraphics[scale= 1.5]{f1,2.eps}\\
%	\vspace{-10mm}
%	\caption{A few edges of $\f_2$ in $[0,1]$}
%\end{figure}
%Figure A.4 is displaying a few edges of $\f_2$ in the interval [0,1] and note that the graph is a  tree. Figure A.5 is a display  of a few edges of $\f_3$ in the interval [0,1].   In Figure A.5, we can see three circuits: $\infty\to 1/3\to 2/3 \to\infty$; $1/3\to 1/6\to 2/9\to 1/3$; and $2/3\to 5/6\to 7/9\to 2/3.$
%
%\begin{figure}[!htb]
%	\centering
%	\includegraphics[scale= 1.0]{f1,3_diagram.eps}\\
%	\vspace{-10mm}
%	\caption{A few edges of $\f_3$ in $[0,1]$}
%\end{figure}
%	\begin{figure}[!htb]
%		\centering
%		\includegraphics[scale= 0.8]{f6.eps}\\
%		\vspace{-10mm}
%		\caption{ A few edges of $\f_6$ in $[0,1]$}
%	\end{figure}
%
%
%	\begin{figure}[!htb]
%		\centering
%		\includegraphics[scale= 0.4]{f6_1.eps}\\
%		\vspace{-10mm}
%		\caption{ Showing images of $P$ and $Q$ in $\f_1$ under $x\mapsto6x$}
%	\end{figure}
%		Here, we discuss an example which shows that when $N$ is not a prime power then $\f_N$ need not be connected. Figure A.6 is a display of a few edges of $\f_6$ in the interval $[0,1]$. We will show that $5/12$ is not connected to $\infty$ in $\f_6$.	
%	Suppose a path from $\infty$ via $1/6$ (or $5/6$) reaches to $5/12.$ Then there is an edge $P\to Q$ in the path such that $1/6<P<1/3<Q.$ Looking at the image of this path in the Farey graph (under the map $x\mapsto 6x$), we get two intersecting edges, $1\to2$ and $6P\to6Q,$ which is not possible (see Figure A.7). Hence, there is no path connecting infinity to 5/12.\\
%	\noindent{\bf Remark A.1} In a well directed path, say $ \infty\rightarrow P_0\rightarrow P_1\rightarrow\cdots P_k\rightarrow\cdots\rightarrow P_n=x,$ each vertex $P_i$ (edge $P_{i-1}\to P_i$) is either direction retaining or direction changing. To see this, suppose the assertion does not hold for some $j\le n-1,$ or $P_j\to P_{j+1}$ is neither direction changing nor retaining. Then either $P_{j+1}<P_{j-1}<P_{j}$ or $P_{j}<P_{j-1}<P_{j+1}.$ Since two edges do not cross, $P_{j+1}$ is one of the $P_i$, $0\le i<(j-1).$ Thus the path we have considered fails to have  vertices with increasing denominators, hence not  well directed.
	
	 


%\section*{References}
  		\bibliographystyle{plain}
  		\bibliography{bib}
  		%\bibitem{paper1} Chuy Xiong. Discussion on Mechanical Learning and Learning Machine, arxiv.org, 2016. \\ \htmladdnormallink{http://arxiv.org/pdf/1602.00198.pdf}{http://arxiv.org/pdf/1602.00198.pdf}


\bibitem{pedro}  Pedro Domingos. The Master Algorithm, Talks at Google. \\ \htmladdnormallink{https://plus.google.com/117039636053462680924/posts/RxnFUqbbFRc}{https://plus.google.com/117039636053462680924/posts/RxnFUqbbFRc}

\bibitem{valiant}  L. Valiant. A theory of the learnable. Communications of the ACM, 27, 1984. \\
\htmladdnormallink{http://web.mit.edu/6.435/www/Valiant84.pdf}
{http://web.mit.edu/6.435/www/Valiant84.pdf}

\bibitem{hinton} E. G. Hinton. Learning multiple layers of representation, Trends in Cognitive Sciences, Vol. 11, pp 428-434.. \htmladdnormallink{http://www.cs.toronto.edu/~hinton/absps/tics.pdf}{http://www.cs.toronto.edu/~hinton/absps/tics.pdf}

\bibitem{mkrot} Markus Krötzsch, František Simancík, Ian Horrocks. A Description Logic Primer, arxiv.org, 2013. 
\htmladdnormallink{http://arxiv.org/pdf/1201.4089.pdf}{http://arxiv.org/pdf/1201.4089.pdf}

\bibitem{dlog} wikipedia. Description logic. 
\htmladdnormallink{http://en.wikipedia.org/wiki/Description\_logic}{http://en.wikipedia.org/wiki/Description\_logic}

\bibitem{gal}  Yarin Gal, Uncertainty in Deep Learning (PhD Thesis) \\
 \htmladdnormallink{http://mlg.eng.cam.ac.uk/yarin/thesis/thesis.pdf}{http://mlg.eng.cam.ac.uk/yarin/thesis/thesis.pdf}


  	\end{document}
  	
  	
 