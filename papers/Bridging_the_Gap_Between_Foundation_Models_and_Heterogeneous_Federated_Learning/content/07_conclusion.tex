% \section{Conclusion}
% We propose RaFFM addressing the challenges when deploy FMs to resource-heterogeneous FL systems. RaFFM introduced specialized FM compression algorithms for edge-FL system that allows scaling down the FM to edge constraints. The experiments demonstrate RaFFM's capability to optimize resource utilization during FL's training and application phases, showing its potential for resource-efficient federated learning. 
% Additionally, we also show the flexibility of RaFFM equipped with PEFT method to fine-tune LLM with significant acceleration. However, we addressing the issue that some foundation models, even after compressed, e.g., Llama-7B, might not be able to run on resource-constrained edge devices FL. It requires advancement in hardware and our future work to make it possible.  

\section{Conclusion}
% We presented RaFFM, a novel approach designed to tackle the inherent challenges associated with deploying FMs in resource-heterogeneous FL. RaFFM integrates advanced FM compression algorithms tailored for edge-FL systems, facilitating the adaptation of these models to the constraints inherent to edge devices. Our experimental results underscore the prowess of RaFFM in optimizing resource utilization across both training and application phases of FL, thereby underscoring its potential to foster a more resource-efficient FL paradigm.
We propose RaFFM addressing the challenges when deploy FMs to resource-heterogeneous FL systems. RaFFM introduced specialized FM compression algorithms for edge-FL system that allows scaling down the FM to edge constraints. The experiments demonstrate RaFFM's capability to optimize resource utilization during FL's life cycle, showing its potential for resource-efficient FL. 
Moreover, the flexibility of RaFFM allows for accelerated LLM fine-tuning in FL with PEFT. Nevertheless, it is essential to recognize the limitations of our approach. Notably, certain foundation models, even post-compression – such as Llama-7B – remain unsuitable for deployment on resource-constrained edge devices in FL settings. Addressing this limitation necessitates advancements in both hardware technology and algorithmic strategies, marking a promising avenue for our future research endeavors.