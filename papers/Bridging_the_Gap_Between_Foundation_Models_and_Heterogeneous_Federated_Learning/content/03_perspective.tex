\section{Motivation for Federated Foundation Models}
\label{sec:prospective}

In this section, we discuss the various challenges that motivate the development of Federated Foundation Models (FFMs), covering aspects such as data privacy, model performance, communication cost, scalability, deployment, personalization and real-time adaptation, and bias reduction. As shown in Figure~\ref{fig:FFM overview}, These existing challenges highlight the potential advantages of combining Foundation Models and Federated Learning for a wide range of applications and scenarios.


\textbf{Data privacy.}
The widespread deployment of AI in society generates vast amounts of data (e.g., images collected by cameras in smartphone applications), presenting potential resources for optimizing and specializing Foundation Models.
However, privacy concerns have limited the use of private data for FM optimization. 
FFMs offer significant improvements in data privacy by incorporating Federated Learning (FL), enabling FM optimization on private data. 
By optimizing FM tasks~(e.g., pre-training, fine-tuning, and prompt engineering) on local data without sharing raw information, FFMs comply with data protection regulations and preserve user privacy. This approach is particularly beneficial when sensitive data, such as medical records or personal communications, must be used to improve model performance without compromising confidentiality.


\textbf{Model performance.}
% Combining Foundation Models and Federated Learning offers mutual benefits that boost model performance. For Foundation Models, integrating Federated Learning allows access to a broader range of data for optimization tasks, such as prompt engineering, in-context learning, instruction tuning, fine-tuning, and pre-training. This expanded data access enables the development of more accurate and efficient AI systems that can better serve users in diverse scenarios.
% On the other hand, Federated Learning can also benefit from the integration of Foundation Models. FL typically suffers from the challenges of Non-IID and biased data, which may result in performance degradation for models trained under such conditions. However, Foundation Models contain comprehensive knowledge and can be easily specialized for various FL private tasks. By leveraging the advanced capabilities of Foundation Models, Federated Learning can overcome the limitations associated with Non-IID and biased data, leading to improved performance across different tasks and domains.
Combining FMs and FL provides benefits to FMs, boosting their performance. FMs gain access to a broader range of data for optimization tasks such as prompt engineering, in-context learning, instruction tuning, fine-tuning, and pre-training. This expanded data access enables the development of more accurate and efficient AI systems better suited for users in diverse scenarios. This combination benefits FL, as well. FL can overcome challenges associated with Non-IID~(Non-Identical Independent Distributed) and biased data by leveraging the advanced capabilities of Foundation Models, leading to improved performance across different tasks and domains.



\textbf{Cost.}
FFMs reduce communication costs by sharing only model updates between devices and the central server, significantly saving bandwidth and communication costs for transmitting raw data. Additionally, FFMs can potentially reduce the labor cost associated with collecting and managing data in a central location, as data is generated and used locally at edge devices. This efficiency makes FFMs a more practical and cost-effective solution for training and deploying FMs.




\textbf{Scalability.}
% Current FMs, especially large language models, often face scalability limitations due to the limited computational power at the edge. To mitigate these limitations, many Foundation Models are run centrally and provide API access for users. However, this approach can lead to capacity constraints and API congestion, hindering the widespread deployment of Foundation Models.
% In the near future, advancements in computational power available to users may enable Foundation Models to run locally on edge devices. The scalable nature of Federated Learning makes it an ideal framework for combining with Foundation Models, as it can accommodate numerous devices with varying computational capabilities. By integrating Federated Learning principles, Foundation Models can take advantage of these advancements in computational power and become more scalable, enabling broader deployment and improved performance across various tasks and domains.
Current FMs, especially large language models, often face scalability limitations due to limited computational power at the edge. Many FMs are run centrally and provide API access for users, which can lead to capacity constraints and API congestion. In the near future, advancements in computational power may enable FMs to run locally on edge devices. FL's scalable nature makes it an ideal framework for combining with FMs, accommodating numerous devices with varying computational capabilities. By integrating FL principles, FMs can leverage advancements in computational power, becoming more scalable and enabling broader deployment and improved performance across various tasks and domains.



\textbf{Deployment.}
% Federated Foundation Models offer potential advantages in terms of deployment, especially when it comes to reducing latency and enhancing user experience. When Foundation Models are run centrally and provide API access for users, latency can be an issue due to network communication between the user's device and the central server hosting the model. This latency can be affected by network conditions, server load, and other factors.
% In contrast, Federated Foundation Models can be deployed and run locally on edge devices, which has the potential to reduce latency by eliminating the need for network communication. This allows for faster response times and a more seamless user experience when interacting with the model.
% However, it is essential to consider the available computational resources on edge devices when deploying Foundation Models locally. As discussed in the Scalability section, advancements in computational power will be crucial for enabling the local deployment of Foundation Models on a wide range of devices, ensuring efficient and effective performance across various tasks and domains.
FFMs offer potential advantages in deployment, particularly in reducing latency and enhancing user experience. Running FMs centrally with API access for users can result in latency issues due to network communication between the user's device and the central server hosting the model. In contrast, FFMs can be deployed and run locally on edge devices, potentially reducing latency by eliminating network communication. This allows for faster response times and a more seamless user experience when interacting with the model. However, available computational resources on edge devices must be considered when deploying FMs locally. As discussed in the Scalability section, advancements in computational power will be crucial for enabling local deployment on a wide range of devices, ensuring efficient and effective performance across various tasks and domains.


\textbf{Personalization and real-time adaptation.}
FFMs facilitate a high degree of personalization by leveraging the decentralized nature of FL. By training on diverse, user-generated data, FMs can be tailored to individual preferences and requirements, offering more personalized and context-aware solutions across various tasks and domains.
A key advantage of FFMs is their ability to adapt in real-time as new personalized data becomes available from edge devices. This continuous learning capability ensures that the models remain up-to-date with users' evolving needs and preferences, further enhancing their personalization.
The focus on personalization in FFMs leads to improved performance and greater user satisfaction. By providing AI solutions that dynamically adapt to user-specific needs, FFMs enable more effective and engaging user experiences across a wide range of applications and domains.


% FFMs facilitate a high degree of personalization by leveraging the decentralized nature of FL. By training on diverse, user-generated data, FMs can be tailored to individual preferences and requirements, offering more personalized and context-aware solutions across various tasks and domains.
% One key advantage of Federated Foundation Models is their ability to adapt in real-time as new personalized data becomes available from edge devices. This continuous learning capability ensures that the models stay up-to-date with the evolving needs and preferences of users, further enhancing their personalization.
% The focus on personalization in Federated Foundation Models results in improved performance and greater satisfaction for users who interact with these models in their daily lives. By providing AI solutions that can dynamically adapt to user-specific needs, Federated Foundation Models enable more effective and engaging user experiences across a wide range of applications and domains.

\textbf{Bias reduction.}
% Federated Foundation Models contribute to bias reduction in AI systems by incorporating diverse data from decentralized sources, leading to more inclusive and fair AI solutions. The models learn from a wide variety of users and become more aware of the nuances and complexities of real-world scenarios, resulting in more informed and less biased decisions across tasks and domains.
% Moreover, the privacy-preserving nature of Federated Learning encourages more users to participate in the training process, further diversifying the data and knowledge incorporated into Foundation Models. This results in models better equipped to handle and minimize biases, providing fairer and more equitable AI solutions for all users.
FFMs contribute to bias reduction in AI systems by incorporating diverse data from decentralized sources, resulting in more inclusive and fair AI solutions. The models learn from various users, increasing their awareness of the nuances and complexities of real-world scenarios, and leading to more informed and less biased decisions across tasks and domains.
Additionally, the privacy-preserving nature of FL encourages more users to participate in the training process, further diversifying the data and knowledge incorporated into FMs. This results in models better equipped to handle and minimize biases, providing fairer and more equitable AI solutions for all users.

\textbf{Continual/Lifelong learning.} 
% Continual or Lifelong Learning refers to the ability of AI models to learn and adapt to new tasks and information over time without forgetting previously acquired knowledge. This learning paradigm aims to mimic human learning, which is characterized by the continuous accumulation and integration of knowledge throughout an individual's lifetime. The benefits of continual/lifelong learning include better generalization across tasks, efficient transfer learning, and more robust adaptation to new and changing environments~\cite{}.
% Recent research in the field of continual/lifelong learning has focused on techniques such as experience replay, elastic weight consolidation, and meta-learning, which enable models to maintain and update their knowledge efficiently~\cite{}. Integrating these approaches with Foundation Models could unlock new opportunities for developing more versatile and adaptive AI systems. In the context of Federated Foundation Models, continual/lifelong learning could be particularly beneficial as it enables models to learn from diverse and distributed data sources continuously while preserving privacy. Exploring the combination of federated learning and continual/lifelong learning could lead to the development of advanced AI models that can effectively leverage distributed data while adapting to new tasks and challenges throughout their lifespan.
FMs combined with FL provide an ideal platform for continual lifelong learning. This combination facilitates the continuous adaptation and improvement of models by harnessing decentralized and diverse data sources, leading to more versatile and effective AI systems. As advancements in edge computing power become more prevalent, the realization of continual lifelong learning in Foundation Models will soon be within reach. This progress will enable AI models to learn and grow throughout their lifespan, unlocking new possibilities for AI research and practical applications in various domains. By embracing continual lifelong learning, FFMs can help create more adaptive, efficient, and personalized AI systems that can dynamically adjust to user-specific needs and preferences, ultimately benefiting users from all walks of life.







In summary, FFMs offer a promising approach to address many challenges and limitations associated with traditional, centralized machine learning. By integrating FL into FM optimization, we can create more efficient, personalized, privacy-preserving, and inclusive AI systems. This opens up new possibilities for AI research and practical applications, making AI more accessible and beneficial to users from all walks of life.








