\subsection{Differentiation through Blackbox Root-Finding}

\paragraph{Gradient relations}
Here we derive the gradient relations as discussed in Sec. \ref{sec:black-box-inverse-optimization}. This is a generalization of the usual inverse function theorem in 1D to the case of scalar bijections conditioned on some parameter $\bm \theta$. For reasons of brevity in the main text and precision in the proof, we used a different notation in Sec. \ref{sec:black-box-inverse-optimization} than we will use here. 
\begin{theorem}
\label{thm:backward-gradients}
Let 
\begin{align*}
    \alpha \colon (x; \bm \theta) \mapsto y; \qquad
    \beta \colon (y; \bm \theta) \mapsto x
\end{align*}
be $\mathcal{C}^{2}$ functions satisfying
\begin{align} \label{eq:inverse-condition}
    \alpha(\beta(y; \bm \theta); \bm \theta) = y; \qquad
    \beta(\alpha(x; \bm \theta); \bm \theta) = x
\end{align}
and 
\begin{align}
    \partial_{x} \alpha(x; \bm \theta) > 0; \qquad
    \partial_{y} \beta(y; \bm \theta) > 0.
\end{align}
Let further
\begin{align}
    g_{\alpha}(x,\bm \theta) = \log \left| \frac{\partial \alpha(x; \bm \theta) }{\partial x } \right|; \quad g_{\beta}(y,\bm u) = \log \left| \frac{\partial \beta(y; \bm u) }{\partial y } \right|.
\end{align}
For $y = \alpha(x; \bm \theta)$, we have the following gradient relations
\begin{align}
    \frac{\partial \beta(y; \bm \theta)}{\partial y} &= \left(\frac{\partial \alpha(x; \bm \theta)}{\partial x}\right)^{-1} \label{eq:grad-out-in} \\
    \frac{\partial \beta(y; \bm \theta)}{\partial \bm \theta} &= - \left(\frac{\partial \alpha(x; \bm \theta)}{\partial x}\right)^{-1} \frac{\partial \alpha(x; \bm \theta)}{\partial \bm \theta} \label{eq:grad-out-param} \\
    \frac{\partial g_{\beta}(y; \bm \theta)}{\partial y} &= - \left(\frac{\partial \alpha(x; \bm \theta)}{\partial x}\right)^{-1} \frac{\partial g_{\alpha}(x; \bm \theta)}{\partial x} \label{eq:grad-density-in} \\
    \frac{\partial g_{\beta}(y; \bm \theta)}{\partial \bm \theta} &= \left(\frac{\partial \alpha(y; \bm \theta)}{\partial x}\right)^{-1} \left(  \frac{\partial g_{\alpha}(x; \bm \theta)}{\partial x} \frac{\partial \alpha(x; \bm \theta)}{\partial \bm \theta} - \frac{\partial^{2} \alpha(x; \bm \theta)}{\partial \bm \theta \partial x} \right) \label{eq:grad-density-param}
\end{align}
\end{theorem}
\begin{proof}
    We have
    \begin{align}
        \frac{d}{d x} \beta(\alpha(x; \bm \theta); \bm \theta) = \frac{\partial \beta(\alpha(x; \bm \theta); \bm \theta)}{\partial y} \frac{\partial \alpha(x; \bm \theta)}{\partial x} = \frac{d x}{d x} = 1.
    \end{align}
    and thus
    \begin{align}
        \frac{\partial \beta(\alpha(x; \bm \theta); \bm \theta)}{\partial y}  &= \left( \frac{\partial \alpha(x; \bm \theta)}{\partial x} \right)^{-1}.
    \end{align}
    
    Now 
    \begin{align}
        \frac{d}{d \bm \theta} \beta(\alpha(x; \bm \theta); \bm \theta) = \frac{\partial \beta(\alpha(x; \bm \theta); \bm \theta)}{\partial y} \frac{\partial \alpha(x; \bm \theta)}{\partial \bm \theta} + \frac{\partial \beta(\alpha(x; \bm \theta); \bm \theta)}{\partial \bm \theta} = \frac{\partial x}{\partial \bm \theta} = 0.
    \end{align}
    which gives
    \begin{align}
        \frac{\partial \beta(\alpha(x; \bm \theta); \bm \theta)}{\partial \bm \theta} 
        &= \frac{\partial \beta(\alpha(x; \bm \theta); \bm \theta)}{\partial y} \frac{\partial \alpha(x; \bm \theta)}{\partial \bm \theta}\\
        &= \left( \frac{\partial \alpha(x; \bm \theta)}{\partial x} \right)^{-1} \frac{\partial \alpha(x; \bm \theta)}{\partial \bm \theta}
    \end{align}
    
    Combining
    \begin{align}
        \frac{d }{d x} \frac{d }{d x} \beta(\alpha(x; \bm \theta); \bm \theta) &= \frac{d }{d x} \frac{d x}{d x} = 0
    \end{align}
    and 
    \begin{align}
        \frac{d }{d x} \frac{d }{d x} \beta(\alpha(x; \bm \theta); \bm \theta) &= \frac{d}{dx} \frac{\partial \beta(\alpha(x; \bm \theta); \bm \theta)}{\partial y} \frac{\partial \alpha(x; \bm \theta)}{\partial x} \\
        &= \frac{\partial^{2} \beta(\alpha(x; \bm \theta); \bm \theta)}{\partial y\partial y} \left(\frac{\partial \alpha(x; \bm \theta)}{\partial x} \right)^{2} + \frac{\partial \beta(\alpha(x; \bm \theta); \bm \theta)}{\partial y} \frac{\partial^{2} \alpha(x; \bm \theta)}{\partial x \partial x} 
    \end{align}
    we obtain
    \begin{align}
    \frac{\partial^{2} \beta(\alpha(x; \bm \theta); \bm \theta)}{\partial y \partial y}
          &= - \left(\frac{\partial \alpha(x; \bm \theta)}{\partial x} \right)^{-2} \frac{\partial \beta(\alpha(x; \bm \theta); \bm \theta)}{\partial y} \frac{\partial^{2} \alpha(x; \bm \theta)}{\partial x \partial x} \\
          &= - \left(\frac{\partial \alpha(x; \bm \theta)}{\partial x} \right)^{-3} \frac{\partial^{2} \alpha(x; \bm \theta)}{\partial x \partial x}\\
          &= - \left(\frac{\partial \alpha(x; \bm \theta)}{\partial x} \right)^{-2} \frac{\partial}{\partial x} \log \left ( \frac{\partial \alpha(x; \bm \theta)}{\partial x}  \right)\\
          &= - \left(\frac{\partial \alpha(x; \bm \theta)}{\partial x} \right)^{-2} \frac{\partial g_{\alpha}(x; \bm \theta)}{\partial x}
    \end{align}
    and thus
    \begin{align}
        \frac{\partial}{\partial y} g_{\beta}(\alpha(x; \bm \theta); \bm \theta) 
        &= \left(\frac{\partial \beta(\alpha(x; \bm \theta); \bm \theta)}{\partial y} \right)^{-1} \frac{\partial^{2} \beta(\alpha(x; \bm \theta); \bm \theta)}{\partial y \partial y} \\
        &= - \left(\frac{\partial \alpha(x; \bm \theta)}{\partial x} \right)^{-1} \frac{\partial g_{\alpha}(x; \bm \theta)}{\partial x}
    \end{align}
    Finally, combining
    \begin{align}
        \frac{d }{d \bm \theta} \frac{d }{d x} \beta(\alpha(x; \bm \theta); \bm \theta)  &= \frac{d }{d \bm \theta} \frac{d x}{d x} = 0
    \end{align}
    and
    \begin{align}
        \frac{d }{d \bm \theta} \frac{d }{d x} \beta(\alpha(x; \bm \theta); \bm \theta) 
        &= \frac{d}{d\bm \theta} \frac{\partial \beta(\alpha(x; \bm \theta); \bm \theta)}{\partial y} \frac{\partial \alpha(x; \bm \theta)}{\partial x} \\
        &= \frac{\partial^{2} \beta(\alpha(x; \bm \theta); \bm \theta)}{\partial y\partial y} \frac{\partial \alpha(x; \bm \theta)}{\partial \bm \theta} \frac{\partial \alpha(x; \bm \theta)}{\partial x}  \\
        &\quad + \frac{\partial \beta(\alpha(x; \bm \theta); \bm \theta)}{\partial y} \frac{\partial^{2} \alpha(x; \bm \theta)}{\partial \bm \theta \partial x} \nonumber  \\
        &\quad + \frac{\partial^{2} \beta(\alpha(x; \bm \theta); \bm \theta)}{\partial \bm \theta  \partial y} \frac{\partial \alpha(x; \bm \theta)}{\partial x} \nonumber 
    \end{align}
    gives us
    \begin{align}
        \frac{\partial^{2}\beta(\alpha(x; \bm \theta); \bm \theta)}{\partial \bm \theta \partial y}
        &= - \frac{\partial^{2} \beta(\alpha(x; \bm \theta); \bm \theta)}{\partial y\partial y} \frac{\partial \alpha(x; \bm \theta)}{\partial \bm \theta} \\
        &\quad - \frac{\partial \beta(\alpha(x; \bm \theta); \bm \theta)}{\partial y} \frac{\partial^{2} \alpha(x; \bm \theta)}{\partial \bm \theta \partial x} \left(\frac{\partial \alpha(x; \bm \theta)}{\partial x}\right)^{-1} \nonumber \\
        &= \left(\frac{\partial \alpha(x; \bm \theta)}{\partial x} \right)^{-2} \frac{\partial g_{\alpha}(x; \bm \theta)}{\partial x} \frac{\partial \alpha(x; \bm \theta)}{\partial \bm \theta} \\
        &\quad - \left(\frac{\partial \alpha(x; \bm \theta)}{\partial x} \right)^{-2} \frac{\partial^{2} \alpha(x; \bm \theta)}{\partial \bm \theta \partial x}  \nonumber\\
        &= \left(\frac{\partial \alpha(x; \bm \theta)}{\partial x} \right)^{-2} \left(\frac{\partial g_{\alpha}(x; \bm \theta)}{\partial x} \frac{\partial \alpha(x; \bm \theta)}{\partial \bm \theta}  - \frac{\partial^{2} \alpha(x; \bm \theta)}{\partial \bm \theta \partial x} \right)
    \end{align}
    And thus we get
    \begin{align}
        \frac{\partial}{\partial \bm \theta} g_{\beta}(\alpha(x; \bm \theta); \bm \theta) 
        &= \left( \frac{\partial \beta(\alpha(x; \bm \theta); \bm \theta)}{\partial y}  \right)^{-1} \frac{\partial^{2}\beta(\alpha(x; \bm \theta); \bm \theta)}{\partial \bm \theta \partial y} \\
        &= \left(\frac{\partial \alpha(x; \bm \theta)}{\partial x} \right)^{-1} \left(\frac{\partial g_{\alpha}(x; \bm \theta)}{\partial x} \frac{\partial \alpha(x; \bm \theta)}{\partial \bm \theta}  - \frac{\partial^{2} \alpha(x; \bm \theta)}{\partial \bm \theta \partial x} \right)
    \end{align}
\end{proof}

\paragraph{Additional derivations for forces}

\begin{align}
    d^3_{x} \beta 
    &= d_{x} \left( 
        \partial_{y}^{2} \beta \left(\partial_{x} \alpha \right)^2 
        + \partial_{y} \beta \partial_{x}^{2} \alpha 
    \right) \\
    &= \partial_{y}^{3} \beta \left(\partial_{x} \alpha \right)^3 
    + 2  \partial_{y}^{2} \beta \partial_{x} \alpha \partial^2_{x} \alpha
    + \partial^2_{y} \beta \partial_{x} \alpha \partial_{x}^{2} \alpha
    + \partial_{y} \beta \partial_{x}^{3} \alpha\\
    &= \partial_{y}^{3} \beta \left(\partial_{x} \alpha \right)^3 
    + 3  \partial_{y}^{2} \beta \partial_{x} \alpha \partial^2_{x} \alpha
    + \partial_{y} \beta \partial_{x}^{3} \alpha = 0\\
    \implies&\\
    \partial_{y}^{3} \beta &= -\left(\partial_{x} \alpha \right)^{-3}\left[ 3  \partial_{y}^{2} \beta \partial_{x} \alpha \partial^2_{x} \alpha
    + \partial_{y} \beta \partial_{x}^{3} \alpha \right] \\
     &= -\left(\partial_{x} \alpha \right)^{-4}\left[ 3   \partial_{x} g_{\alpha}   \partial^2_{x} \alpha
    + \partial_{x}^{3} \alpha \right] \\
\end{align}

\begin{align}
    d_{\bm \theta} d^2_{x} \beta 
    &= d_{\bm \theta} \left( 
        \partial_{y}^{2} \beta \left(\partial_{x} \alpha \right)^2 
        + \partial_{y} \beta \partial_{x}^{2} \alpha 
    \right) \\
    &= 
        \partial_{\bm \theta} \partial_{y}^{2} \beta \left(\partial_{x} \alpha \right)^2 
        + \partial_{y}^{3} \beta  \partial_{\bm \theta} \alpha \left(\partial_{x} \alpha \right)^2 
        + \partial_{y}^{2} \beta 2 \left(\partial_{x} \alpha \right) \partial_{\bm \theta} \partial_{x} \alpha
        \\ &\quad
        + \partial_{\bm \theta} \partial_{y} \beta \partial_{x}^{2} \alpha 
        + \partial^2_{y} \beta \partial_{\bm \theta} \alpha \partial_{x}^{2} \alpha 
        + \partial_{y} \beta \partial_{\bm \theta} \partial_{x}^{2} \alpha = 0
     \\
    \implies&\\
    \partial_{\bm \theta} \partial_{y}^{2} \beta &= - \left(\partial_{x} \alpha \right)^{-2} \Bigg[ 
        \partial_{y}^{3} \beta  \partial_{\bm \theta} \alpha \left(\partial_{x} \alpha \right)^2 
        + \partial_{y}^{2} \beta 2 \left(\partial_{x} \alpha \right) \partial_{\bm \theta} \partial_{x} \alpha
        \\ &\qquad\qquad\qquad
        + \partial_{\bm \theta} \partial_{y} \beta \partial_{x}^{2} \alpha 
        + \partial^2_{y} \beta \partial_{\bm \theta} \alpha \partial_{x}^{2} \alpha 
        + \partial_{y} \beta \partial_{\bm \theta} \partial_{x}^{2} \alpha
    \Bigg] 
\end{align}