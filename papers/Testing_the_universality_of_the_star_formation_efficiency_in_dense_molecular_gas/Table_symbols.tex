\begin{table*}
\tiny
\scalebox{0.92}[0.92]{
\centering
\begin{threeparttable}
\caption{Definition of each notation used in the paper\label{list_symbols}}
\begin{tabular}{|l|l|l|}
\hline
Quantity & Notation & Meaning \\
\hline
\multirow{2}{*}{Surface area}  & $A_{\rm mol}^{\rm detect}$ & Area where the H$^{13}$CO$^+$ or H$^{13}$CN emission has been detected.\\
                                 & $A_{\rm Herschel}^{A_{\rm V}>8}$ & Area above $A_{\rm V}$=8 according to the $Herschel$ column density map in each observed cloud \\
\hline
\multirow{2}{*}{Cloud radius}  & $R_{\rm mol}^{\rm detect}$ & Equivalent radius of the area where H$^{13}$CO$^+$ or H$^{13}$CN emission has been detected in each cloud.\\
                                 & $R_{\rm Herschel}^{A_{\rm V}>8}$ & Equivalent radius of the area above $A_{\rm V}$=8 in each observed cloud (according to the $Herschel$ column density map). \\
\hline
\multirow{3}{*}{Velocity width}  & $dV_{\rm FWHM}$ & FWHM velocity width at each pixel\\
                         & $dV_{\rm mol}^{\rm detect}$ & FWHM velocity width estimated from the spectrum averaged over the area where H$^{13}$CO$^+$ or H$^{13}$CN emission has been detected.  \\
                         & $dV_{\rm mol}^{A_{\rm V}>8}$ & Scaled FWHM velocity width using $dV_{\rm mol}^{A_{\rm V}>8}$=$dV_{\rm mol}^{\rm detect}$ $\left(\frac{R_{\rm Herschel}^{A_{\rm V}>8}}{R_{\rm mol}^{\rm detect}}\right)^{0.5}$ \\
\hline
\multirow{5}{*}{Mass} & $M_{\rm VIR,mol}^{\rm detect}$           &  Virial mass of the portion of each cloud where H$^{13}$CO$^+$ or H$^{13}$CN emission has been detected. \\
         & $M_{\rm VIR,mol}^{A_{\rm V}>8}$  &   Virial mass of the region above $A_{\rm V}$=8 in each observed cloud.   \\
         & $M_{\rm Herschel}^{\rm mol-detect}$  &  Dense gas mass estimated from the Herschel column density map of the area where H$^{13}$CO$^+$ or H$^{13}$CN emission has been detected.  \\
         & $M_{\rm Herschel}^{A_{\rm V}>8}$ &  Dense gas mass estimated from the Herschel column density map for the area above $A_{\rm V}$=8 in each observed cloud.   \\
         & $M_{\rm dense,mol}$  & Dense gas mass estimated from molecular luminosity conversion factor ($M_{\rm dense,mol}$=$\alpha_{\rm Herschel-mol}^{\rm fit}$$L_{\rm mol}$). \\
\hline
\multirow{2}{*}{Virial ratio} & $\mathcal{R}_{\rm VIR,mol}^{\rm detect,}$           &  Virial mass ratio of the portion of the cloud where the H$^{13}$CO$^+$ or H$^{13}$CN emission has been detected $(\mathcal{R}_{\rm VIR,mol}^{\rm detect}={M_{\rm VIR,mol}^{\rm detect}}/{M_{\rm Herschel}^{\rm mol-detect}})$. \\
                                         & $\mathcal{R}_{\rm VIR,mol}^{A_{\rm V}>8}$    &  Virial mass ratio of the region above $A_{\rm V}$=8 in each observed cloud $(\mathcal{R}_{\rm VIR,mol}^{A_{\rm V}>8}={M_{\rm VIR,mol}^{A_{\rm V}>8}}/{M_{\rm Herschel}^{A_{\rm V}>8}})$.  \\
\hline
\multirow{2}{*}{Star formation rate}  &  SFR$_{\rm YSO}$       &  Star formation rate estimated from the number count of YSOs. \\
                              &  SFR$_{\rm prestellar}$   & Star formation rate estimated from the number count of prestellar cores.  \\
\hline
\multirow{7}{*}{HCN conversion factor}  & $\alpha_{\rm HCN}$  & Conversion factor from HCN(1--0) luminosity to dense gas mass. \\
& $\alpha_{\rm Herschel-HCN}$  & Empirical $\alpha_{\rm HCN}$  factor derived for the target nearby clouds using {\it Herschel} mass estimates as references ($\alpha_{\rm Herschel-HCN}$=$M_{\rm Herschel}^{A_{\rm V}>8}$/$L_{\rm HCN}$). \\
                             & $\alpha_{\rm Herschel-HCN}^{\rm fit}$  & $\alpha_{\rm HCN}$ conversion factor obtained from the relation $\alpha_{\rm Herschel-HCN}^{\rm fit}$=496$\times$ $G_{\rm 0}^{-0.24}$.   \\
                             & $\alpha_{\rm Herschel-HCN}^{\rm fit ^\prime}$  & $\alpha_{\rm HCN}$ conversion factor obtained from the relation $\alpha_{\rm Herschel-HCN}^{\rm fit^{\prime}}$=0.13$\times G_{\rm 0}^{-0.095} \times \alpha_{\rm Herschel-HCN}^{\rm fit}$.   \\
                             & $\alpha_{\rm GS04-HCN}$  & $\alpha_{\rm HCN}$ conversion factor assumed in \citet{Gao04b}.  \\
                             & $\alpha_{\rm Wu05-HCN}$  & $\alpha_{\rm HCN}$ conversion factor obtained in \citet{Wu05}.  \\
                             & $\alpha_{\rm GB12-HCN}$  & $\alpha_{\rm HCN}$ conversion factor obtained in \citet{Garcia12}.  \\
                             & $\alpha_{\rm Usero-HCN}$  & $\alpha_{\rm HCN}$ conversion factor obtained in \citet{Usero15}.  \\
\hline
\multirow{2}{*}{${\rm HCO^+}$ conversion factor}                             & $\alpha_{\rm Herschel-HCO^+}$  & $\alpha_{\rm HCO^+}$ conversion factor obtained toward each observed sub region in the present study ($\alpha_{\rm Herschel-HCO^+}$=$M_{\rm Herschel}^{A_{\rm V}>8}$/$L_{\rm HCO^+}$). \\
                             & $\alpha_{\rm Herschel-HCO^+}^{\rm fit}$  & $\alpha_{\rm HCO^+}$ conversion factor obtained by the relation $\alpha_{\rm Herschel-HCO^+}^{\rm fit}$=689$\times$ $G_{\rm 0}^{-0.24}$.   \\
\hline
\end{tabular}
\end{threeparttable}
}
\end{table*}
