\begin{table*}
\centering
\begin{threeparttable}
\caption{Observations\label{obs_parameters}}
\begin{tabular}{lccc}
\hline
Region & Aquila &  Ophiuchus   & Orion B  \\
\hline
Distance & 260 pc\tnote{*}  & 139 pc\tnote{$\dag$}  & 400 pc\tnote{$\ddag$} \\
Telescope & IRAM 30m & MOPRA 22m   & Nobeyama 45m  \\
Receiver & EMIR & 3mm & TZ  \\
Correlator & FTS50 & MOPS  & SAM45  \\
Obs. period &  18 -- 29 Dec 2014 & 26 July -- 2 Aug 2015 & 7--21 May 2015  \\
                    &  2--9 Sep 2015     & &   \\
Mapping area &  0.42 deg$^2$ ($\sim$8.7 pc$^2$) & 0.21 deg$^2$ ($\sim$1.2 pc$^2$) & 0.14 deg$^2$ ($\sim$6.8 pc$^2$)  \\
$\theta_{\rm ant}$ at 86 GHz &  28$\arcsec$.6 ($\sim$0.04 pc) &39$\arcsec$.0 ($\sim$0.03 pc) & 19$\arcsec$.1 ($\sim$0.04 pc)   \\
$dV$ at 86 GHz &  0.15 km/s & 0.10 km/s & 0.023 km/s   \\
\hline
\end{tabular}
\begin{tablenotes}
\item[*] The distance to the Aquila molecular complex is  under debate \citep{Drew97, Dzib10,Ortiz-Leon16}. In this paper, we adopt a distance of 260 pc for Aquila according to \citet{Maury11} and \citet{Konyves15}. 
\item[$\dag$] See \citet{Mamajek08}.
\item[$\ddag$] See \citet{Gibb08}. 
\end{tablenotes}
\end{threeparttable}
\end{table*}
