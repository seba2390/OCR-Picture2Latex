\section{Conclusions and next steps}\label{sec:conclusions}
% definition of MIL
In this paper we present a novel objective function, the \emph{Missing Information Loss} (MIL), specifically designed for handling unobserved user-item interactions in implicit feedback datasets. In particular, MIL explicitly forbids treating missing user-item interactions as positive or negative feedback.
% What it does
We demonstrate that, thanks to the functional form of the MIL function, the ranking of unseen items is almost entirely left to the low--rank process, rather than forcing unobserved items to be at the tail of the recommendation (\emph{i.e.}, MIL does not force a zero predicted preference for unobserved user-item interactions). 

% Metric results
Extensive experiments with Matrix Factorization and Denoising Autoencoders conducted on three datasets, show that \textsc{MIL} models demonstrate competitive performance when compared with other traditional losses such as cross-entropy or the multinomial log-likelihood. 
% Best performing models 
% Analysis of recommendations
In addition, we study the distribution of the recommendations and observe that the reported metric performance takes place while recommending popular items less frequently (up to a $20 \%$ decrease with respect to the best competing method). Indeed, \textsc{MIL} models sharply increase the recommendation of medium--tail items, while almost linearly expanding the appearance of long--tail items with the ranking position in the list of recommendations. Such expansion results in up to a $50 \%$ increase of long--tail recommendations, a feature of utmost importance for industries with a large catalogue of items. 

% Future work
Future lines of research may involve the incorporation of negative feedback, or the usage of \textsc{MIL} in temporal--aware Recommender Systems (such as those using Recurrent Neural Networks).  
In addition, we hope that the results here reported  will bring forward first-principle mathematical derivations of the \textsc{MIL} function, so that the vast family of possible polynomials modelling the missing information term can be reduced, or even extended with more suitable functions. 