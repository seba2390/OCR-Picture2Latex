\documentclass[sigconf]{acmart}

\settopmatter{printacmref=false} % Removes citation information below abstract
\renewcommand\footnotetextcopyrightpermission[1]{} % removes footnote with conference information in first column
\pagestyle{plain} % removes running headers

\usepackage{booktabs} % For formal tables
\usepackage{multirow}
\usepackage{amsmath}
\usepackage{color}
\usepackage{arydshln }


%Conference
\acmConference[WSDM]{The Twelfth International Conference on Web Search and Data Mining}{February 11--15}{Melbourne, Australia} 
\acmYear{2019}

\begin{document}
\title[Missing Information Loss]{A Missing Information Loss for implicit feedback datasets}

\author{Juan Ar\'evalo}
\affiliation{%
  \institution{BBVA Data \& Analytics}
}
\email{juanmaria.arevalo@bbvadata.com}

\author{Juan Ram\'on Duque}
\affiliation{%
  \institution{BBVA Data \& Analytics}
}
\email{juanramon.duque@bbvadata.com}

\author{Marco Creatura}
\affiliation{%
  \institution{BBVA Data \& Analytics}
}
\email{marco.creatura@bbvadata.com}

% The default list of authors is too long for headers}
\renewcommand{\shortauthors}{Ar\'evalo, Duque and Creatura}

% Some useful commands
\newcommand{\MFsquare}{\textsc{MF-square}}
\newcommand{\MFmil}{\textsc{MF-mil}}
\newcommand{\MFce}{\textsc{MF-CE}}
\newcommand{\CEpointlinsig}{\textsc{CE$_{\rm Point}$ lin-sig}}
\newcommand{\CEpointsigsig}{\textsc{CE$_{\rm Point}$ sig-sig}}
\newcommand{\CEpairlinsig}{\textsc{CE$_{\rm Pair}$ lin-sig}}
\newcommand{\CEpairsigsig}{\textsc{CE$_{\rm Pair}$ sig-sig}}
\newcommand{\MULTItanhlin}{\textsc{MULTI tanh-lin}}
\newcommand{\MILlinsig}{\textsc{MIL lin-sig}}
\newcommand{\MILsigsig}{\textsc{MIL sig-sig}}


\begin{abstract}
% missing values and negative feedback
Latent factor models for Recommender Systems with implicit feedback typically treat unobserved user-item interactions (\emph{i.e.} missing information) as negative feedback. This is frequently done 
either through negative sampling (point--wise loss) or with a ranking loss function (pair-- or list--wise estimation). 
% Common objective functions allow zero prediction
Since a zero preference recommendation is a valid solution for most common objective functions, 
regarding unknown values as actual zeros results in users 
having a zero preference recommendation  for most of the available items. 

% MIL
In this paper we propose a novel objective function, the \emph{Missing Information Loss} (MIL), 
that explicitly forbids treating unobserved user-item interactions as positive or negative feedback. 
% application to AE and metrics
We apply this loss to both traditional Matrix Factorization and user--based Denoising Autoencoder, and compare it with other established objective functions such as cross--entropy (both point-- and pair--wise) or the recently proposed multinomial log-likelihood. MIL achieves competitive performance in ranking--aware metrics when applied to three datasets.
% towards long-tail recommendations
Furthermore, we show that such a relevance in the recommendation is obtained while displaying popular items less frequently (up to a $20 \%$ decrease with respect to the best competing method). This debiasing from the recommendation of popular items favours the appearance of infrequent items (up to a $50 \%$ increase of long--tail recommendations), a valuable feature for Recommender Systems with a large catalogue of products. 
\end{abstract}

\begin{CCSXML}
<ccs2012>
<concept>
<concept_id>10002951.10003317.10003347.10003350</concept_id>
<concept_desc>Information systems~Recommender systems</concept_desc>
<concept_significance>500</concept_significance>
</concept>
</ccs2012>
\end{CCSXML}

\ccsdesc[500]{Information systems~Recommender systems}

\keywords{Collaborative Filtering, Autoencoders, Implicit Feedback, Missing Information}

\setcopyright{None}

\maketitle

\setlength{\abovecaptionskip}{0pt}
\setlength{\belowcaptionskip}{-10pt}

% Put all the sections with inputs

Reinforcement learning has achieved great success in areas such as Game-playing \citep{silver2018general,vinyals2019grandmaster}, robotics \cite{kober2013reinforcement}, large language models \citep{ouyang2022training}, etc.
However, due to safety concerns or physical limitations, in some real-world reinforcement learning problems, we must consider additional constraints that may influence the optimal policy and the learning process \citep{garcia2015comprehensive}.
% For example, a robotic arm must not take actions that may cause harm to itself or the environments.
A standard framework to handle such cases is the constrained Markov Decision Process (CMDP) \citep{altman1999constrained}.
Within the CMDP framework, the agent has to maximize
the expected cumulative reward while
obeying a finite number of constraints, which are usually in the form of expected cumulative cost criteria.

However, we are sometimes concerned with the problem with a continuum of constraints.
For example,
the constraints we meet might be time-evolving or subject to uncertain parameters, which
cannot be formulated as an ordinary CMDP
(see Examples \ref{Example_Time_Evolving} and  \ref{Example_Uncertain}).
In this paper we would study a generalized CMDP  
to address the above problem.  Because the constraints are not only infinite-number but also lie
in a continuous set,
the generalization is not trivial. Fortunately, we find that we can borrow the idea behind semi-infinite programming (SIP) \citep{remez1934determination, hettich1993semi} to deal with the semi-infinite constraints.
Accordingly, we propose \emph{semi-infinitely constrained Markov decision processes} (SICMDPs)
as a novel complement to the ordinary CMDP framework.
%More specifically,  an SICMDP model %, we consider 
%contains a continuum of constraints whereas an ordinary CMDP contains a finite number of constraints. 

%This generalization is natural but not trivial. However, we can brows the idea  
%The idea is quite natural and can be backtracked
%to the practice of extending linear programming to linear semi-infinite programming (LSIP) %\cite{remez1934determination, GobernaLSIO1998}.
%In addition, 
%As a complementary approach to the ordinary CMDP framework, 
%SICMDP can be used to model these problems  which cannot be described by a finite number of constraints
%that are not covered by .
%For example,
%the restrictions we consider can be time-evolving or subject to uncertain parameters
%, thus
%cannot be described by a finite number of constraints but a continuum of constraints 
%(see Examples \ref{Example_Time_Evolving} and  \ref{Example_Uncertain}).

We also present two reinforcement learning algorithms to solve SICMDPs called SI-CRL and SI-CPO, respectively.
SI-CRL is a model-based reinforcement learning algorithm designed for tabular cases, and SI-CPO is a policy optimization algorithm for non-tabular cases.
% and analyze its performance both theoretically and empirically.
The main challenge is that we need to deal with a continuum of constraints, thus reinforcement learning algorithms for ordinary CMDPs do not work anymore.
In SI-CRL, we tackle this difficulty by first transforming the reinforcement learning problem to an equivalent LSIP problem, which can then be solved using methods in the LSIP literature like the dual exchange methods \citep{Hu1990,reemtsen1998numerical}.
In SI-CPO, we resort to the idea of cooperative stochastic approximation developed in \cite{lan2020algorithms, wei2020comirror}.
As far as we know, we are the first to introduce tools from semi-infinitely programming (SIP) into the reinforcement learning community for solving constrained reinforcement learning problems.

% To the best of our knowledge, we are the first to apply tools from semi-infinitely programming (SIP) to solve reinforcement learning problems.
Furthermore, we give theoretical analysis for both SI-CRL and SI-CPO.
We decompose the error of SI-CRL into two parts: the statistical error from approximating the true SICMDP with an offline dataset and the optimization error due to the fact that the solution of the LSIP problem obtained by the dual exchange method is inexact.
On the optimization side, we show that the iteration complexity of SI-CRL is $O\left(\left\{\mathrm{diam}(Y)L\sqrt{|\gS|^2|\gA|m}/\left[(1-\gamma)\epsilon\right]\right\}^m\right)$.
On the statistical side, we show that the sample complexity of SI-CRL is $\widetilde O\left(\frac{|S|^2|A|^2}{\epsilon^2(1-\gamma)^3}\right)$ if the offline dataset is generated by a generative model, and $\widetilde O\left(\frac{|S||A|}{\nu_{\min} \epsilon^2(1-\gamma)^3}\right)$ if the dataset is generated by a probability measure $\nu$ as considered in \cite{chen2019information}.
Here $\widetilde O$ means that all logarithm terms are discarded.
For SI-CPO, things become a little more complicated because other than the statistical error and the optimization error, we also need to consider the function approximation error, which comes from imperfect policy parametrizations.
It is shown if the function approximation error can be controlled to $O(\epsilon)$ order, the iteration complexity of SI-CPO is $\widetilde{O}\left(\frac{1}{\epsilon^2(1-\gamma)^6}\right)$ and the sample complexity of SI-CPO is $\widetilde{O}(\frac{1}{\epsilon^4(1-\gamma)^{10}})$.
Here our iteration complexity bound is equivalent to a typical $\widetilde O(1/\sqrt{T})$ global convergence rate.

We perform a set of numerical experiments to illustrate the SICMDP model and validate our proposed algorithms.
Specifically, we examine two numerical examples, namely the discharge of sewage and ship route planning.
Through the discharge of sewage example, we show the advantage of the SICMDP framework over the CMDP baseline obtained by naive discretization in modeling realistic sequential decision-making problems.
Moreover, we demonstrate the effectiveness of the SI-CRL and SI-CPO algorithms in such tabular environments. 
In the ship route planning example, we illustrate the benefits of the SICMDP framework and the ability of the SI-CPO algorithm to address complex continuous control tasks involving continuous state spaces with modern deep reinforcement learning techniques.

% In summary, our contributions are listed as follows.
% First, we present the SICMDP model, which can be viewed as a generalization of the ordinary CMDP model.
% Second, we propose an algorithm to perform reinforcement learning for SICMDPs, which is called SI-CRL, and we believe that we are the first to apply tools from SIP
% to solve reinforcement learning problems.
% Third, we give a theoretical analysis of SI-CRL and identify both its sample complexity and iteration complexity.
% In addition, we perform numerical experiments to illustrate the SICMDP model and validate the SI-CRL algorithm.
% \{This paragraph can be removed!!! \}






\section{The \MakeLowercase{i}W\MakeLowercase{inr}NFL model}
\label{sec:model}

In this section we are going to present the data we used to develop our in-game probability model as well as the design details of {\method}. 

{\bf Data: }In order to perform our analysis we utilize a dataset collected from NFL's Game Center for all the regular season games between the seasons 2009 and 2016. 
We access the data using the Python {\tt nflgame} API \cite{nflgame}. 
The dataset includes detailed play-by-play information for every game that took place during these seasons. 
This information is used to obtain the state of the game that will drive the design of {\method}. 
In total, we collected information for 2,048 regular season games and a total of 338,294 snaps/plays. 

{\bf Model: }
{\method} is based on a logistic regression model that calculates the probability of the home team winning given the current status of the game as: 

\begin{equation}
\Pr(H=1| \mathbf{x})= \frac{\exp(\mathbf{\weight}^T\cdot\mathbf{x})}{1+\exp(\mathbf{\weight}^T\cdot\mathbf{x})}
\label{eq:reg}
\end{equation}
where $H$ is the dependent random variable of our model representing whether the home team wins or not, $\mathbf{x}$ is the vector with the independent variables, while the coefficient vector $\mathbf{\weight}$ includes the weights for each independent variable and is estimated using the corresponding data.  
For a game of infinite duration a linear model could be a very good approximation.  
However, the boundary effects from the finite duration of a game create several non-linearities \cite{winston2012mathletics}.  
For this reason, we enhance our model - using the same set of features - with a Support Vector Machine classifier with radial kernel for the last three minutes of regulation.  
In order to obtain a probability output from the SVM classifier, we further use Platt's scaling \cite{platt1999probabilistic}: 

\begin{equation}
\Pr(H=1| \mathbf{x})= \frac{1}{1+\exp{(Af(x)+B)}}
\label{eq:platt}
\end{equation}
where $f(x)$ is the uncalibrated value produced by the SVM classifier: 

\begin{equation}
f(x) = \sum_{i} (\alpha_i y_i k(\mathbf{x}_i\cdot\mathbf{x}))+ b
\label{eq:svm}
\end{equation}
where $k(\mathbf{x},\mathbf{x}')$ is the kernel used for the SVM.   
Figure \ref{fig:iwinrNFL} depicts the simple flow chart of {\method}. 


\begin{figure}[t]
\begin{center}
\includegraphics[scale=0.35]{plots/iwinrNFL.pdf}%\vspacecap
 \caption{{\method} includes a linear and a non-linear component.}
 \label{fig:iwinrNFL}
\end{center}
\end{figure}

In order to describe the status of the game we use the following variables:

\begin{enumerate}
\item {\bf Ball Possession Team:} This binary feature captures whether the home or the visiting team has the ball possession
\item {\bf Score Differential:} This feature captures the current score differential (home - visiting)
\item {\bf Timeouts Remaining:} This feature is represented by two independent variables - one for the home and one for the away team - and they capture the number of timeouts remaining for each of the teams
%\item {\bf Quarter:} This feature captures the current quarter of the game
%\item {\bf Time Remaining:} This feature captures the time (in seconds) remaining for the current quarter to end
\item {\bf Time Elapsed: } This feature captures the time elapsed since the beginning of the game
\item {\bf Down:} This feature represents the down of the team in possession
\item {\bf Field Position:} This feature captures the distance covered by the team in possession from their own yard line
\item {\bf Yards-to-go:} This variables represents the number of yards needed for a first down
\item {\bf Ball Possession Time: } This variable captures the time that the offensive unit of the home team is on the field 
\item {\bf Ranking Differential: } This variable represents the difference of the win percentage for the two team (home - visiting)
\end{enumerate}

The last independent variable is representative of the power ranking difference between the two teams. 
Most of the existing models that include such a variable are using the Vegas line spread for each game.  
We choose not to do so for the following reason.  
The objective of the Vegas line is not to predict game outcomes but rather distribute money across the different bets.  
Exactly because of this objective the line is changing during the week before the game.  
While this line can change due to new information for the competing teams (e.g., injury updates), the line is mainly changing when a particular team has accumulated the majority of the bets. 
In this case it will also be hard to choose which line to use (e.g., the opening, the closing or some average of them).  
Therefore, we choose to use the win percentage differential of the two teams as an indicator of their strength (even though this has its own issues given the uneven schedule in NFL).  
However, note that if one would like to use the point spread as a variable this can be easily incorporated in the model. 
Table \ref{tab:iwinrnfl} presents the coefficients of the logistic regression model of {\method} with standardized independent variables for better comparisons. 


\begin{table}[ht]
\begin{center}
\def\sym#1{\ifmmode^{#1}\else\(^{#1}\)\fi}
\begin{tabular}{l*{1}{c}}
\toprule
                    &\multicolumn{1}{c}{(1)}\\
                    &\multicolumn{1}{c}{Winner}\\
\midrule
Possession Team (H)         &      0.41\sym{***}\\
                    &     (49.19)         \\
\addlinespace
Score Differential           &      3.59\sym{***}\\
                    &    (247.34)         \\
\addlinespace
Home Timeouts           &     0.12\sym{***}\\
                    &      (8.74)         \\
\addlinespace
Away Timeouts           &     -0.11\sym{***}\\
                    &    (-12.47)         \\
\addlinespace
Ball Possession Time  &     -0.05.\\
                    &    (-1.66)         \\
\addlinespace
Time Lapsed       &   -0.05.\\
                    &      (-1.66)         \\
\addlinespace
Down                &   -0.01         \\
                    &      (0.04)         \\
\addlinespace
Field Position            &   0.02\sym{**} \\
                    &      (2.71)         \\
\addlinespace
Yards-to-go                &  -0.01         \\
                    &      (0.23)         \\
\addlinespace
Rating differential         &       0.75\sym{***}\\
                    &     (80.47)         \\
\addlinespace
Intercept            &       0.57\sym{*}\\
                    &    (2.09)         \\
\midrule
Observations        &      338,294         \\
\bottomrule
\multicolumn{2}{l}{\footnotesize \textit{t} statistics in parentheses}\\
\multicolumn{2}{l}{\footnotesize \sym{$_.$} \(p<0.1\), \sym{*} \(p<0.05\), \sym{**} \(p<0.01\), \sym{***} \(p<0.001\)}\\
\end{tabular}
\end{center}
\caption{Standardized logisitic regression coefficients for {\method}.}
\label{tab:iwinrnfl}
\end{table}


As we can see, as one might have expected the current scoring differential exhibits the strongest correlation with the in-game win probability.  
The only factors that do not appear to be statistically significant predictors of the dependent variable are the down and the yards-to-go. 
Even though the corresponding coefficients are negative as one might have expected (e.g., being at an earlier down gives you more chances to advance the ball), they are not significant in estimating the win probability. 
On the contrary, all else being equal timeouts appear to be quiet important since they can help a team stop the clock, while teams with better win percentage appear to have an advantage as well, since this can be a sign of a better team. 
In the following section we provide a detailed evaluation of {\method}.

\section{Experimental protocols}\label{sec:protocols}
\subsection{Datasets}

We use the 
MovieLens--20M\footnote{http://grouplens.org/datasets/movielens}
and Netflix\footnote{http://www.netflixprize.com} explicit feedback  datasets.
As both  of these contain explicit ratings, we create binary preferences
by keeping ratings $\ge\!4$, which we interpret as positive feedback ($p_{ui}=1$).
Furthermore, we only keep users with at least 5 views.
Validation and test sets are obtained randomly, selecting a $10~\%$ of the original dataset for each set. We denote such datasets \textsc{ML20M} and \textsc{Netflix}.

In addition, we explore models performance on the Last.fm\footnote{https://www.upf.edu/web/mtg/lastfm360k} dataset~\cite{Celma:Springer2010}, an implicit feedback dataset consisting of tuples (user, artist, plays), that contains top artists by user. In order to make the comparison with the above datasets more straightforward, we binarize play counts and interpret them as implicit preference data. Next, we filter out artist with less than 50 distinct listening users, and user with less than 20 artists in their listening history. In the following, we name this dataset \textsc{Lastfm}.

\setlength{\belowcaptionskip}{5pt}
\begin{table}[htb]
\begin{tabular}{c c c c c}
 Dataset & \#users & \#items & \#pairs & \#pairs$_{\rm test}$ \\
\hline
\textsc{ML20M} & 136,7k & 20,3k & 7,99M & 1,0M \\
\textsc{Netflix}  & 463,4k & 17,7k & 45,5M & 5,7M \\
\textsc{Lastfm}  & 350,2k & 24,6k & 12,8M & 1,6M \\
\hline
\end{tabular}
\caption{Statistics of the datasets after preprocessing.}
\label{table:datasets}
\end{table}

The statistics of the training set after such  processing, as well as the number of user-item interactions in test, are presented in Table~\ref{table:datasets}. 
%We also represent the item popularity distribution in training, validation and test sets in Figure~\ref{fig:ditribution_dataset} for the \textsc{ML20M},  \textsc{Netflix}, \textsc{Lastfm} datasets, from top to bottom, respectively. As expected, validation and test distributions  follow the distribution of the training set. 

% \begin{figure}
%     \centering
%     \includegraphics[width=.7\linewidth]{figures/distributions_dataset.png}
%     \caption{Normalized distributions of user-item interactions in train (green), validation (red) and test (blue) sets for the  \textsc{ML20M},  \textsc{Netflix} and \textsc{Last-fm} datasets (from top to bottom, respectively).}
%     \label{fig:ditribution_dataset}
% \end{figure}

\subsection{Evaluation metrics}\label{subsec:metrics}
\setlength{\belowcaptionskip}{-10pt}
Given the set of adopted items in test, $\mathcal{I}_u^{\rm t}$, and the ranked list of predicted preferences, 
the relevance of a recommendation at position $k$ is given by ${\rm rel}_{ui}(k)$--${\rm rel}(k)$ from here on--, which equals $1$ if user $u$ adopted item $i$ in the test set, $0$ otherwise. In the calculation of metrics, we remove items observed in  training and validation from the list of recommendations. 
Next, we detail the metrics used for model evaluation.

\paragraph{Recall} It does not account for the relative ordering of the recommendation, and we defined it as~\cite{liang:2018:VAE}
\begin{equation}
{\rm Recall}@k = \frac{\sum_{s=1}^k {\rm rel}(s)}
{\mathcal{N}_u(k)}.
\end{equation}
Here, $\mathcal{N}_u(k) = \min\left(k,|\mathcal{I}_u^{\rm t}|\right)$, with $|\mathcal{I}_u^{\rm t}|$ the number of items adopted by user $u$ in testing. The final recall is averaged across all users in testing. 

\paragraph{Normalized Discount Cumulative Gain} In contrast to recall metric, the Discount Cumulative Gain (DCG) performs a logarithmic discount according to the position of a recommendation, that is
\begin{equation}
{\rm DCG}@k = \sum_{s=1}^k \frac{{\rm rel}(s)}
{\log_2(s+1)}.
\end{equation}
This quantity can be normalized by the Ideal DCG, 
\begin{equation}
{\rm IDCG}@k = \sum_{s=1}^{\mathcal{N}_u(k)} \frac{1}{\log_2(s+1)}.
\end{equation}
Finally, NDCG$@k= {\rm DCG}@k/{\rm IDCG}@k$, which we average across all users in the test set.

\paragraph{Novelty} Following reference~\cite{Vargas:2011:Novelty_diversity}, we define a novelty-weighted DCG score as
\begin{equation}\label{eq:nov-ndcg}
{\rm Nov\!-\!DCG}@k = 
-\sum_{s=1}^k \frac{{\rm rel}(s)\times \ln{\nu(i)}}
{\log_2(s+1)}.
\end{equation}
Here, $\nu(i)$ is the frequency of occurrences of item $i$  normalized to the total interactions in training. The corresponding novelty-weighted IDCG would be
\begin{equation}
{\rm Nov\!-\!IDCG}@k = 
\sum_{s=1}^{\mathcal{N}_u(k)} 
\frac{\max_{i\in\mathcal{I}_u}\left(-\ln \nu(i)\right)}
{\log_2(s+1)}.
\end{equation}
In other words, the highest DCG is obtained by ranking the most novel items (among those relevant to the user) in descending order. 
\subsection{Implementation details}\label{subsec:implementation}
The implementation of our model is performed in TensorFlow~\cite{tensorflow2015-whitepaper}.
%\footnote{The code will be available online at
%\url{https://github.com/bbvadata/RecApp}.
%}. 
The model can be trained in both CPU or GPU. 
When GPU is enabled, the use of queues to feed the tensors greatly speeds up the training.
% batch size and epoch
We set the batch size to $100$, and train every DAE model for $120$k iterations, so as to ensure proper convergence. For MF models we use $180$k iterations.
% number of neurons
The number of neurons is $200$ in all DAE experiments; for MF models, since the large number of users makes them prone to overfit, we train the models with $100$ and $200$ neurons and take the best performing model.
% weights and biases initialization
Weight matrices are initialized with random uniform values whose amplitude is computed as described by Glorot~\emph{et al.}~\cite{Xavier_initialization}. 
%Generally speaking, we discourage the use of random normal initialization without truncation.
For the biases we use a truncated random normal initialization with a standard deviation of $10^{-3}$. 
% Optimizer
Models are trained with Adam optimizer~\cite{Kingma2014AdamAM} and a learning rate of $10^{-3}$.
% gradient clip
%Additionally, we clip gradients whenever they exceed a norm of $1$, and apply batch normalization~\cite{icml2015_ioffe15_batch-norm} at every training step.

% Size of T and P sets
Concerning negative sampling in point and pair--wise schemes, 
%since we train with a batch size different form one, 
we fix the size of the target sets for every user (sets $\mathcal{T}_u$ and $\mathcal{P}_u$ for point and pair--wise learning, respectively, see subsection~\ref{subsec:losses}).
In particular, we make such sets proportional to the median number of items adopted by users, except for the multinomial loss, where all items are utilized~\cite{Liang:2016:CoFactor}.
The proportionality factors are hyper-parameters fine--tuned with the validation set, swapping the values $\{1,\,5,\,10,\,50,\,100,\,150\}$. We find a factor of $50$ or $100$ to provide the best results.
%(see additional comments in subsection~\ref{subsec:baseline_models}).

% Max norm regularization
%Regarding max--norm regularization, we notice that an asymmetric max--norm regularization might be required for the encoder/decoder weight matrices, depending on the activations and objective functions utilized for modeling. We swap $\alpha_{\rm enc}$ and $\alpha_{\rm dec}$ values in $\{0.0, 0.05, 0.1, 0.2, 0.3, 0.5, 1.0\}$, where $\alpha=0.0$ means no regularization.

% Regularization: noise and weight-decay
We add noise to the input vector of the AE~\cite{Vincent:2008:ECRF-AE, Wu:2016:CDAE-topN} using drop-out~\cite{liang:2018:VAE}. We fix the level of noise at $0.5$. Competitive performance is achieved after normalizing the AE input vector.
% Regularization of MIL is smaller
For DAE models, we swap the $L_2$ regularization strength $\lambda\in[10^{-7}-10^{-4}]$, while for MF models we take the form in equation~(\ref{eq:l2_reg_scaled}) with $\lambda\in[10^{0}-10^{3}]$, which provides a more stable training for MF models\footnote{Recall the different scales of the $\lambda$ factor in equations (\ref{eq:L2_reg}) and (\ref{eq:l2_reg_scaled}).}. In general we find that MIL models require smaller $\lambda$ factors than cross-entropy or multinomial--based models. This is expected, as the level of weight--decay regularization in equations~(\ref{eq:L2_reg}) and (\ref{eq:l2_reg_scaled}) depends on the value of the loss, which is smaller for MIL models.

\subsection{Baseline models}\label{subsec:baseline_models}
We implement the objective functions described in subsection \ref{subsec:losses} on a user-based DAE~\cite{Sedhain:2015:Autorec, Wu:2016:CDAE-topN} and compare the results with the MIL function. We also compare them with traditional Matrix Factorization with Weight Regularization~\cite{HuKoren:2008:CF_implicit}. In the following, we provide details on the training of the different models.

\textbf{Weight-Regularized Matrix Factorization} WRMF~\cite{HuKoren:2008:CF_implicit} is a linear factorization model  trained with square loss and weight decay. We use negative sampling with a sampling ratio of $100$ and $\lambda\sim 5-10$ (as obtained in the validation set). We call this model \MFsquare. In addition, we train WRMF models with MIL and point--wise cross--entropy losses, applying a sigmoid function at the output, so as to ensure $\hat{p}_{ui}\in(0,1)$. In these cases, we find that a sampling ratio of $100$ and $\lambda=50-500$ provide best results. We name these models \MFmil\, and \MFce, respectively.

\textbf{\emph{Denoising Autoencoder models}}

\textbf{Cross-entropy loss} For the cross-entropy loss defined in equations~(\ref{eq:cross-entropy}), (\ref{eq:point-wise}) and (\ref{eq:pair-wise}), we use linear--sigmoid and sigmoid--sigmoid activations at the encoder and decoder, respectively. We name the DAEs models with cross-entropy loss and point--wise estimation \CEpointlinsig\,and \CEpointsigsig; and those with pair--wise,  \CEpairlinsig \,and \CEpairsigsig. In order to prevent numerical instabilities, we ensure that the output preferences are in $[\varepsilon, 1-\varepsilon]$, with $\varepsilon=10^{-5}$. Regarding negative sampling, we find that the best sampling ratio is $50\times {\rm median}(\mathcal{I}_u)$ and $100\times {\rm median}(\mathcal{I}_u)$  for point and pair--wise estimation, respectively. Best weight-decay regularization is found to be $\lambda=2\cdot 10^{-5}$.

% Comparison to CDAE
The closest model to these baselines is the Collaborative Denoising AE (CDAE)~\cite{Wu:2016:CDAE-topN}, although for the sake of simplicity, in the present paper we do not include the user embedding of CDAE. 
% bad performance of BPR
Similar to CDAE, we find that  pair--wise learning does not achieve competitive results at the top of the ranked list~\cite{Wu:2016:CDAE-topN, liang:2018:VAE}. 

\textbf{Multinomial loss} AEs trained with a multinomial log-likelihood have recently been  introduced by Lian et al~\cite{liang:2018:VAE}, either applied to DAEs or Variational AEs (VAE) with partial regularization. Here, we focus on the multi-DAE modeling with $\tanh$-linear activations\footnote{
We use the actual implementation provided at \url{https://github.com/dawenl/vae_cf} to verify that the activation used at the decoder of multi-DAE is linear, although the original writing~\cite{liang:2018:VAE} suggests a $\tanh$ non-linearity for the decoder.
}, and name this baseline \MULTItanhlin. Our implementation exactly reproduces that of~\cite{liang:2018:VAE} when using $\lambda=2\cdot 10^{-5}$, input noise of $0.5$ and without applying negative sampling. 

\textbf{Missing Information loss} We apply the \textsc{MIL} function defined in equation~(\ref{eq:mil_def}) to linear-sigmoid and sigmoid-sigmoid DAEs. We name these models \MILlinsig\, and \MILsigsig, respectively. Best hyper-parameters of the loss turn out to be $A_{\rm MI}=10^6,\, \gamma_{\rm MI}=10$ and $\gamma_{+}=1$, after grid search the pairs $\left(A_{\rm MI}, \gamma_{\rm MI}\right)\in\{
(5\cdot10^1, 2)$, $(10^3, 4)$, $(2\cdot10^4, 6)$, $(5\cdot10^5, 10)$, $(1\cdot10^6, 10)$, $(5\cdot10^6, 10)$, $(5\cdot10^9, 15)\}$, and $\gamma_{+}=1$ or $2$. In addition, we use a sampling ratio of $50$ and $\lambda\in(10^{-6}, 10^{-5})$.

\begin{table}[t!]
\centering
\caption{Voice conversion \& F0 manipulation results. MOS results are reported with 95\% confidence interval. VDE, and FFE are reported for F0 manipulation while PER, WER, EER, and MOS are reported for voice conversion. Notice, for VDE, and FFE higher is the better since F0 was flattened.}
\label{tab:conv}

\resizebox{1\columnwidth}{!}{
\begin{tabular}{c@{~} | c@{~} | c@{~}c@{~} | c@{~} | c@{~} ||  c@{~}c@{~} }
\toprule
\multirow{2}{*}{Dataset} & \multirow{2}{*}{Method} & \multicolumn{4}{c||}{Voice Conversion} & \multicolumn{2}{c}{F0 Manipulation} \\
\cmidrule{3-8}
& & PER~$\downarrow$ & WER~$\downarrow$ & EER~$\downarrow$ & MOS~$\uparrow$ & VDE~$\uparrow$ & FFE~$\uparrow$ \\
\midrule
VCTK & GT  & 17.16 & 4.32 & 3.25 & 4.11$\pm$0.29 & -- & -- \\
\midrule 
\multirow{3}{*}{LJ}
% & ASR-TTS   & 50.74  & --     & 66.08 & 32.96 & 1.46 \\
& CPC       & 22.22 	& 16.11 		& 0.46 		& 3.57$\pm$0.15 		& \bf 46.68 & \bf 48.71\\
& HuBERT    & \bf 19.09 & \bf 12.23 & \bf 0.31  & \bf 3.71$\pm$0.24 & 39.20 		& 48.42\\
& VQ-VAE    & 40.88 	& 36.96 		& 9.65 		& 2.90$\pm$0.17 		& 10.54 	& 12.08 \\
\midrule 
\multirow{3}{*}{VCTK} 
% & ASR-TTS   & 68.88  & --    & 41.77 & 13.55 & 6.48 \\
& CPC       &  23.58 		& 15.98 		& \bf 4.83  &  3.42 $\pm$ 0.24 		& \bf 25.29 & \bf 26.97 \\
& HuBERT    &  \bf 20.85 	& \bf 12.72 & 6.01  		& \bf  3.58 $\pm$ 0.28 	& 23.46 	& 26.67 \\
& VQ-VAE    & 36.88  		& 29.44 		& 11.56 		& 3.08 $\pm$ 0.34 		& 7.03  	& 7.80  \\
\bottomrule
\end{tabular}}
\vspace{-0.4cm}
\end{table}

\vspace{-0.1cm}
\section{Results}
\vspace{-0.1cm}
Our results cover
% We report results for 
three different settings: (i) speech reconstruction experiments; (ii) speaker conversion and F0 manipulation; (iii) bitrate analysis with subjective tests for speech codec evaluation. We employ two datasets: LJ~\cite{ljspeech17} single speaker dataset and VCTK~\cite{vctk} multi-speaker dataset. All datasets were resampled to a 16kHz sample rate.

% \paragraph*{Implementation Details.}
% \smallskip
\noindent{\bf Implementation Details\quad} 
\label{sec:impl}
We follow the same setup as in~\cite{lakhotia2021generative}. For CPC, we used the model from~\cite{Riviere2020}, which was trained on a ``clean'' 6k hour sub-sample of the LibriLight dataset~\cite{Kahn2020,Riviere2020}. We extract a downsampled representation from an intermediate layer with a 256-dimensional embedding and a hop size of 160 audio samples. For HuBERT we used a \textsc{Base} 12 transformer-layer model trained for two iterations~\cite{hsu2020hubert} on 960 hours of LibriSpeech corpus~\cite{Panayotov2015}. 
% This model encodes every 320 raw audio samples into a 768-dimensional vector. 
This model downsamples the raw audio $\times320$ into a sequence of 768-dimensional vectors. Similarly to~\cite{lakhotia2021generative}, activations were extracted from the sixth layer.

%CPC: We use a dictionary of 100 units, leading to a bitrate of 700bps.
%HuBERT: A dictionary of 100 units is used, leading to a bitrate of 350bps. 
%VQVE: The VQ-VAE discrete code operates at a bitrate of 800bps.
% For both CPC and HuBERT, the k-means algorithm is applied to convert continuous frames to discrete codes, using the LibriSpeech clean-100h~\cite{Panayotov2015} dataset. 
For CPC and HuBERT, the k-means algorithm is trained on LibriSpeech clean-100h~\cite{Panayotov2015} dataset to convert continuous frames to discrete codes. We quantize both learned representations with $K=100$ centroids. Leading to a bitrate of 700bps for CPC and 350bps for HuBERT.

% VQ-VAE
Similarly to CPC models, we trained the VQ-VAE content encoder model on the ``clean'' 6K hours subset from the LibriLight dataset. We use an encoder operating on the raw signal to extract discrete units, similar to~\cite{jukebox}. In addition, ``random restarts'' were performed when the mean usage of a codebook vector fell below a predetermined threshold. Finally, we used HiFiGAN (architecture and objective) as the decoder instead of a simple convolutional decoder, as it improved the overall audio quality. This model encodes the raw audio into a sequence of discrete tokens from 256 possible tokens~\cite{garbacea2019low} with a hop size of 160 raw audio samples. The VQ-VAE discrete code operates at a bitrate of 800bps. We additionally experimented with 100 discrete units for VQ-VAE, however results were the best for 256. This finding is consistent with~\cite{garbacea2019low}.

% verification model
The speaker verification network uses the architecture proposed in~\cite{heigold2016end}. It was trained on the VoxCeleb2~\cite{voxceleb2} dataset, achieving a 7.4\% Equal Error Rate (EER) for speaker verification on the test split of the VoxCeleb1~\cite{Nagrani17} dataset.

% pitch
Only a single F0 representation is considered across all evaluated models, trained on the VCTK dataset.
% The F0 is extracted from the raw audio using YAAPT~\cite{yaapt} algorithm, using a window size of 20ms and a 5ms hop. 
The F0 is extracted from the raw audio using a window size of 20ms and a 5ms hop. 
As a result, the F0 sequence is sampled at 200Hz. 
% We apply the quantization described at Sec.~\ref{sec:method}, using a pitch codebook of $K'=20$ tokens and an encoder that downsamples the pitch by $\times16$. 
The quantization described at Sec.~\ref{sec:method}, is applied using an F0 codebook of $K'=20$ tokens and an encoder that downsamples the signal by $\times16$. Hence, the discrete F0 representation is sampled at 12.5Hz, leading to a bitrate of 65bps. The final bitrate of the evaluated codecs is the sum of the pitch code bitrate with the content code bitrate.

% \paragraph*{Evaluation Metrics}
% \smallskip
\noindent{\bf Evaluation Metrics\quad} 
We consider both subjective and objective evaluation metrics. For subjective tests, we report the Mean Opinion Scores (MOS). In which human evaluators rate the naturalness of audio samples on a scale of 1--5. Each experiment, included 50 randomly selected samples rated by 30 raters. For objective evaluation, we consider: (i) Equal Error Rate~(EER) as an automatic speaker verification metric obtained using a pre-trained speaker verification network. We report EER between test utterances and enrolled speakers; (ii) Voicing Decision Error (VDE)~\cite{nakatani2008method}, which measures the portion of frames with voicing decision error; (iii) F0 Frame Error (FFE)~\cite{chu2009reducing}, measures the percentage of frames that contain a deviation of more than 20\% in pitch value or have a voicing decision error; (iv) Word Error Rate (WER) and Phoneme Error Rate (PER), proxy metrics to the intelligibility of the generated audio. We used a pre-trained ASR network~\cite{baevski2020wav2vec} on both reconstructed and converted samples to calculate both metrics. %To generate target phonemes, the g2p-en~\cite{g2pE2019} Grapheme2Phoneme module was used.

% \vspace{-0.1cm}
% \smallskip
\noindent{\bf Reconstruction \& Conversion}
% \vspace{-0.1cm}
We start by reporting the reconstruction performance. Results are summarized in Table~\ref{tab:recon}. When considering the intelligibility of the reconstructed signal HuBERT reaches the lowest PER and WER scores across all models, where both CPC and HuBERT are superior to VQ-VAE. However, when considering F0 reconstruction VQ-VAE outperforms both HuBERT and CPC by a significant margin. This results are somewhat intuitive, bearing in mind VQ-VAE objective is to fully reconstruct the input signal. In terms of subjective evaluation, all models reach similar MOS scores, with one exception of CPC on LJ. 

%Notice, since the same F0 units are used for each method, this result implies the VQ-VAE units contain some information about the F0 of the signal, enabling better reconstruction. Regarding speaker information, the CPC gets the lowest EER. 

To better evaluate the disentanglement properties of each method with respect to speaker identity and F0, we conducted an additional set of experiments aiming at speaker conversion and F0 manipulation. For voice conversion, we converted each test utterance into five random target speakers. Next, we employed a speaker verification network, which extracts \emph{d-vector} representation to evaluate speaker-converted utterances' similarity to real speaker utterances (low error-rate indicates good conversion), providing measurement to the speaker identity's disentanglement from the evaluated coding method. The error-rate is reported between converted test utterances and enrolled speakers. For the LJ speech single speaker dataset, we converted samples from the VCTK dataset to the single speaker and enrolled all VCTK speakers together with the single speaker. Results are summarized in Table~\ref{tab:conv} (left). Unlike resynthesis results, on voice conversion CPC and HuBERT outperform VQ-VAE on both LJ and VCTK datasets, indicating VQ-VAE contains more information about the speaker in the encoded units, hence producing more artifacts. Notice, this also affects WER, PER, and the overall subjective quality (MOS). 

Next, to evaluate the presence of F0 in the discrete units, we flattened the F0 units before synthesizing the signal and calculated VDE and FFE with respect to the original F0 values. F0 flattening was done by setting the speakers' mean F0 value across all voiced frames. In this experiment, we expected units that contain F0 information to be better at F0 reconstruction over disentangled units. Results are summarized in Table~\ref{tab:conv} (right). Notice VQ-VAE can still reconstruct the F0 almost at the same level as when using the original F0 as conditioning (5.2 vs 7.03, and 5.59 vs 7.8), in contrast to CPC and HuBERT.

\begin{figure}[t!]
\centering
\includegraphics[width=0.65\columnwidth, trim={50 20 70 20}]{figures/codec_2.pdf}
% \caption{MUSHRA subjective listening test results as a function of bitrate per second for various methods. Purple dots denote the baseline methods, and green dots the proposed SSL based method.} 
\caption{MUSHRA subjective quality results as a function of bitrate per second. Purple dots denote the baseline methods, and green dots the proposed SSL based method.} 
\label{fig:codec}
\vspace{-0.5cm}
\end{figure}

% \vspace{-0.1cm}
% \smallskip
\noindent{\bf Speech Codec}
Our final experiment evaluates the obtained speech units as a low bitrate speech codec. 
% Therefore, we evaluate how the performance varies as a function of the number of discrete units. Changing the number of units is equivalent to varying the bitrate of the encoded signal. 
We use a subjective MUSHRA-type listening test~\cite{series2014method} to measure the perceived quality of the proposed speech codec with regard to its bitrate constraints. In MUSHRA evaluations, listeners are presented with a labeled uncompressed signal for reference, a set of test samples to rate, a copy of the uncompressed reference, and a low-quality anchor. Listeners are asked to rate each test utterance and the copy of the uncompressed reference with respect to the labeled reference in a scale of 1-100.

The experiment is performed on the VCTK dataset~\cite{vctk}. For evaluation, we used 20 utterances from 5 speakers. The set of speakers in the test data is disjoint with those in the training data. For this experiment, HuBERT models with 50, 100, and 200 units were trained as described in Sec.~\ref{sec:impl}. For comparison, we included other speech codecs in our evaluation: Opus~\cite{valin2012definition} wideband at 9 kbps VBR, Codec2~\cite{rowe2011codec} at 2.4 kbps and LPCNet~\cite{valin2019real} operating at 1.6 kbps. The LPCNet model was trained from scratch on the VCTK dataset following the experimental setup in~\cite{valin2019real}. The VQ-VAE model employs the HiFiGAN decoder trained on the LibriLight dataset to match the amount of data reported in~\cite{garbacea2019low}. We compressed the anchor sample with Speex~\cite{valin2016speex} at 4 kbps as a low anchor. Fig.~\ref{fig:codec} depicts the results. HuBERT with 50 units reaches the best MUSHRA score while its bitrate is only 365bps, which is significantly lower than the baseline methods.

\section{Conclusions and next steps}\label{sec:conclusions}

% definition of MIL
In this paper we present a novel objective function, the \emph{Missing Information Loss} (MIL), specifically designed for handling unobserved user-item interactions in implicit feedback datasets. In particular, MIL explicitly forbids treating missing user-item interactions as positive or negative feedback.
% What it does
We demonstrate that, thanks to the functional form of the MIL function, the ranking of unseen items is almost entirely left to the low--rank process, rather than forcing unobserved items to be at the tail of the recommendation (\emph{i.e.}, MIL does not force a zero predicted preference for unobserved user-item interactions). 

% Metric results
Extensive experiments with Matrix Factorization and Denoising Autoencoders conducted on three datasets, show that \textsc{MIL} models demonstrate competitive performance when compared with other traditional losses such as cross-entropy or the multinomial log-likelihood. 
% Best performing models 
% Analysis of recommendations
In addition, we study the distribution of the recommendations and observe that the reported metric performance takes place while recommending popular items less frequently (up to a $20 \%$ decrease with respect to the best competing method). Indeed, \textsc{MIL} models sharply increase the recommendation of medium--tail items, while almost linearly expanding the appearance of long--tail items with the ranking position in the list of recommendations. Such expansion results in up to a $50 \%$ increase of long--tail recommendations, a feature of utmost importance for industries with a large catalogue of items. 

% Future work
Future lines of research may involve the incorporation of negative feedback, or the usage of \textsc{MIL} in temporal--aware Recommender Systems (such as those using Recurrent Neural Networks).  
In addition, we hope that the results here reported  will bring forward first-principle mathematical derivations of the \textsc{MIL} function, so that the vast family of possible polynomials modelling the missing information term can be reduced, or even extended with more suitable functions. 

\begin{acks}
We would like to thank the continuous support and careful reading of the manuscript by the \emph{Edge} guild within BBVA Data \& Analytics, specially J. Garc\'ia Santamar\'ia and J. A. Rodr\'iguez Serrano. 
\end{acks}

%\bibliographystyle{ACM-Reference-Format}
\bibliographystyle{unsrt}
%\bibliography{sigproc} 

\documentclass[sigconf]{acmart}

\settopmatter{printacmref=false} % Removes citation information below abstract
\renewcommand\footnotetextcopyrightpermission[1]{} % removes footnote with conference information in first column
\pagestyle{plain} % removes running headers

\usepackage{booktabs} % For formal tables
\usepackage{multirow}
\usepackage{amsmath}
\usepackage{color}
\usepackage{arydshln }


%Conference
\acmConference[WSDM]{The Twelfth International Conference on Web Search and Data Mining}{February 11--15}{Melbourne, Australia} 
\acmYear{2019}

\begin{document}
\title[Missing Information Loss]{A Missing Information Loss for implicit feedback datasets}

\author{Juan Ar\'evalo}
\affiliation{%
  \institution{BBVA Data \& Analytics}
}
\email{juanmaria.arevalo@bbvadata.com}

\author{Juan Ram\'on Duque}
\affiliation{%
  \institution{BBVA Data \& Analytics}
}
\email{juanramon.duque@bbvadata.com}

\author{Marco Creatura}
\affiliation{%
  \institution{BBVA Data \& Analytics}
}
\email{marco.creatura@bbvadata.com}

% The default list of authors is too long for headers}
\renewcommand{\shortauthors}{Ar\'evalo, Duque and Creatura}

% Some useful commands
\newcommand{\MFsquare}{\textsc{MF-square}}
\newcommand{\MFmil}{\textsc{MF-mil}}
\newcommand{\MFce}{\textsc{MF-CE}}
\newcommand{\CEpointlinsig}{\textsc{CE$_{\rm Point}$ lin-sig}}
\newcommand{\CEpointsigsig}{\textsc{CE$_{\rm Point}$ sig-sig}}
\newcommand{\CEpairlinsig}{\textsc{CE$_{\rm Pair}$ lin-sig}}
\newcommand{\CEpairsigsig}{\textsc{CE$_{\rm Pair}$ sig-sig}}
\newcommand{\MULTItanhlin}{\textsc{MULTI tanh-lin}}
\newcommand{\MILlinsig}{\textsc{MIL lin-sig}}
\newcommand{\MILsigsig}{\textsc{MIL sig-sig}}


\begin{abstract}
% missing values and negative feedback
Latent factor models for Recommender Systems with implicit feedback typically treat unobserved user-item interactions (\emph{i.e.} missing information) as negative feedback. This is frequently done 
either through negative sampling (point--wise loss) or with a ranking loss function (pair-- or list--wise estimation). 
% Common objective functions allow zero prediction
Since a zero preference recommendation is a valid solution for most common objective functions, 
regarding unknown values as actual zeros results in users 
having a zero preference recommendation  for most of the available items. 

% MIL
In this paper we propose a novel objective function, the \emph{Missing Information Loss} (MIL), 
that explicitly forbids treating unobserved user-item interactions as positive or negative feedback. 
% application to AE and metrics
We apply this loss to both traditional Matrix Factorization and user--based Denoising Autoencoder, and compare it with other established objective functions such as cross--entropy (both point-- and pair--wise) or the recently proposed multinomial log-likelihood. MIL achieves competitive performance in ranking--aware metrics when applied to three datasets.
% towards long-tail recommendations
Furthermore, we show that such a relevance in the recommendation is obtained while displaying popular items less frequently (up to a $20 \%$ decrease with respect to the best competing method). This debiasing from the recommendation of popular items favours the appearance of infrequent items (up to a $50 \%$ increase of long--tail recommendations), a valuable feature for Recommender Systems with a large catalogue of products. 
\end{abstract}

\begin{CCSXML}
<ccs2012>
<concept>
<concept_id>10002951.10003317.10003347.10003350</concept_id>
<concept_desc>Information systems~Recommender systems</concept_desc>
<concept_significance>500</concept_significance>
</concept>
</ccs2012>
\end{CCSXML}

\ccsdesc[500]{Information systems~Recommender systems}

\keywords{Collaborative Filtering, Autoencoders, Implicit Feedback, Missing Information}

\setcopyright{None}

\maketitle

\setlength{\abovecaptionskip}{0pt}
\setlength{\belowcaptionskip}{-10pt}

% Put all the sections with inputs

Reinforcement learning has achieved great success in areas such as Game-playing \citep{silver2018general,vinyals2019grandmaster}, robotics \cite{kober2013reinforcement}, large language models \citep{ouyang2022training}, etc.
However, due to safety concerns or physical limitations, in some real-world reinforcement learning problems, we must consider additional constraints that may influence the optimal policy and the learning process \citep{garcia2015comprehensive}.
% For example, a robotic arm must not take actions that may cause harm to itself or the environments.
A standard framework to handle such cases is the constrained Markov Decision Process (CMDP) \citep{altman1999constrained}.
Within the CMDP framework, the agent has to maximize
the expected cumulative reward while
obeying a finite number of constraints, which are usually in the form of expected cumulative cost criteria.

However, we are sometimes concerned with the problem with a continuum of constraints.
For example,
the constraints we meet might be time-evolving or subject to uncertain parameters, which
cannot be formulated as an ordinary CMDP
(see Examples \ref{Example_Time_Evolving} and  \ref{Example_Uncertain}).
In this paper we would study a generalized CMDP  
to address the above problem.  Because the constraints are not only infinite-number but also lie
in a continuous set,
the generalization is not trivial. Fortunately, we find that we can borrow the idea behind semi-infinite programming (SIP) \citep{remez1934determination, hettich1993semi} to deal with the semi-infinite constraints.
Accordingly, we propose \emph{semi-infinitely constrained Markov decision processes} (SICMDPs)
as a novel complement to the ordinary CMDP framework.
%More specifically,  an SICMDP model %, we consider 
%contains a continuum of constraints whereas an ordinary CMDP contains a finite number of constraints. 

%This generalization is natural but not trivial. However, we can brows the idea  
%The idea is quite natural and can be backtracked
%to the practice of extending linear programming to linear semi-infinite programming (LSIP) %\cite{remez1934determination, GobernaLSIO1998}.
%In addition, 
%As a complementary approach to the ordinary CMDP framework, 
%SICMDP can be used to model these problems  which cannot be described by a finite number of constraints
%that are not covered by .
%For example,
%the restrictions we consider can be time-evolving or subject to uncertain parameters
%, thus
%cannot be described by a finite number of constraints but a continuum of constraints 
%(see Examples \ref{Example_Time_Evolving} and  \ref{Example_Uncertain}).

We also present two reinforcement learning algorithms to solve SICMDPs called SI-CRL and SI-CPO, respectively.
SI-CRL is a model-based reinforcement learning algorithm designed for tabular cases, and SI-CPO is a policy optimization algorithm for non-tabular cases.
% and analyze its performance both theoretically and empirically.
The main challenge is that we need to deal with a continuum of constraints, thus reinforcement learning algorithms for ordinary CMDPs do not work anymore.
In SI-CRL, we tackle this difficulty by first transforming the reinforcement learning problem to an equivalent LSIP problem, which can then be solved using methods in the LSIP literature like the dual exchange methods \citep{Hu1990,reemtsen1998numerical}.
In SI-CPO, we resort to the idea of cooperative stochastic approximation developed in \cite{lan2020algorithms, wei2020comirror}.
As far as we know, we are the first to introduce tools from semi-infinitely programming (SIP) into the reinforcement learning community for solving constrained reinforcement learning problems.

% To the best of our knowledge, we are the first to apply tools from semi-infinitely programming (SIP) to solve reinforcement learning problems.
Furthermore, we give theoretical analysis for both SI-CRL and SI-CPO.
We decompose the error of SI-CRL into two parts: the statistical error from approximating the true SICMDP with an offline dataset and the optimization error due to the fact that the solution of the LSIP problem obtained by the dual exchange method is inexact.
On the optimization side, we show that the iteration complexity of SI-CRL is $O\left(\left\{\mathrm{diam}(Y)L\sqrt{|\gS|^2|\gA|m}/\left[(1-\gamma)\epsilon\right]\right\}^m\right)$.
On the statistical side, we show that the sample complexity of SI-CRL is $\widetilde O\left(\frac{|S|^2|A|^2}{\epsilon^2(1-\gamma)^3}\right)$ if the offline dataset is generated by a generative model, and $\widetilde O\left(\frac{|S||A|}{\nu_{\min} \epsilon^2(1-\gamma)^3}\right)$ if the dataset is generated by a probability measure $\nu$ as considered in \cite{chen2019information}.
Here $\widetilde O$ means that all logarithm terms are discarded.
For SI-CPO, things become a little more complicated because other than the statistical error and the optimization error, we also need to consider the function approximation error, which comes from imperfect policy parametrizations.
It is shown if the function approximation error can be controlled to $O(\epsilon)$ order, the iteration complexity of SI-CPO is $\widetilde{O}\left(\frac{1}{\epsilon^2(1-\gamma)^6}\right)$ and the sample complexity of SI-CPO is $\widetilde{O}(\frac{1}{\epsilon^4(1-\gamma)^{10}})$.
Here our iteration complexity bound is equivalent to a typical $\widetilde O(1/\sqrt{T})$ global convergence rate.

We perform a set of numerical experiments to illustrate the SICMDP model and validate our proposed algorithms.
Specifically, we examine two numerical examples, namely the discharge of sewage and ship route planning.
Through the discharge of sewage example, we show the advantage of the SICMDP framework over the CMDP baseline obtained by naive discretization in modeling realistic sequential decision-making problems.
Moreover, we demonstrate the effectiveness of the SI-CRL and SI-CPO algorithms in such tabular environments. 
In the ship route planning example, we illustrate the benefits of the SICMDP framework and the ability of the SI-CPO algorithm to address complex continuous control tasks involving continuous state spaces with modern deep reinforcement learning techniques.

% In summary, our contributions are listed as follows.
% First, we present the SICMDP model, which can be viewed as a generalization of the ordinary CMDP model.
% Second, we propose an algorithm to perform reinforcement learning for SICMDPs, which is called SI-CRL, and we believe that we are the first to apply tools from SIP
% to solve reinforcement learning problems.
% Third, we give a theoretical analysis of SI-CRL and identify both its sample complexity and iteration complexity.
% In addition, we perform numerical experiments to illustrate the SICMDP model and validate the SI-CRL algorithm.
% \{This paragraph can be removed!!! \}






\section{The \MakeLowercase{i}W\MakeLowercase{inr}NFL model}
\label{sec:model}

In this section we are going to present the data we used to develop our in-game probability model as well as the design details of {\method}. 

{\bf Data: }In order to perform our analysis we utilize a dataset collected from NFL's Game Center for all the regular season games between the seasons 2009 and 2016. 
We access the data using the Python {\tt nflgame} API \cite{nflgame}. 
The dataset includes detailed play-by-play information for every game that took place during these seasons. 
This information is used to obtain the state of the game that will drive the design of {\method}. 
In total, we collected information for 2,048 regular season games and a total of 338,294 snaps/plays. 

{\bf Model: }
{\method} is based on a logistic regression model that calculates the probability of the home team winning given the current status of the game as: 

\begin{equation}
\Pr(H=1| \mathbf{x})= \frac{\exp(\mathbf{\weight}^T\cdot\mathbf{x})}{1+\exp(\mathbf{\weight}^T\cdot\mathbf{x})}
\label{eq:reg}
\end{equation}
where $H$ is the dependent random variable of our model representing whether the home team wins or not, $\mathbf{x}$ is the vector with the independent variables, while the coefficient vector $\mathbf{\weight}$ includes the weights for each independent variable and is estimated using the corresponding data.  
For a game of infinite duration a linear model could be a very good approximation.  
However, the boundary effects from the finite duration of a game create several non-linearities \cite{winston2012mathletics}.  
For this reason, we enhance our model - using the same set of features - with a Support Vector Machine classifier with radial kernel for the last three minutes of regulation.  
In order to obtain a probability output from the SVM classifier, we further use Platt's scaling \cite{platt1999probabilistic}: 

\begin{equation}
\Pr(H=1| \mathbf{x})= \frac{1}{1+\exp{(Af(x)+B)}}
\label{eq:platt}
\end{equation}
where $f(x)$ is the uncalibrated value produced by the SVM classifier: 

\begin{equation}
f(x) = \sum_{i} (\alpha_i y_i k(\mathbf{x}_i\cdot\mathbf{x}))+ b
\label{eq:svm}
\end{equation}
where $k(\mathbf{x},\mathbf{x}')$ is the kernel used for the SVM.   
Figure \ref{fig:iwinrNFL} depicts the simple flow chart of {\method}. 


\begin{figure}[t]
\begin{center}
\includegraphics[scale=0.35]{plots/iwinrNFL.pdf}%\vspacecap
 \caption{{\method} includes a linear and a non-linear component.}
 \label{fig:iwinrNFL}
\end{center}
\end{figure}

In order to describe the status of the game we use the following variables:

\begin{enumerate}
\item {\bf Ball Possession Team:} This binary feature captures whether the home or the visiting team has the ball possession
\item {\bf Score Differential:} This feature captures the current score differential (home - visiting)
\item {\bf Timeouts Remaining:} This feature is represented by two independent variables - one for the home and one for the away team - and they capture the number of timeouts remaining for each of the teams
%\item {\bf Quarter:} This feature captures the current quarter of the game
%\item {\bf Time Remaining:} This feature captures the time (in seconds) remaining for the current quarter to end
\item {\bf Time Elapsed: } This feature captures the time elapsed since the beginning of the game
\item {\bf Down:} This feature represents the down of the team in possession
\item {\bf Field Position:} This feature captures the distance covered by the team in possession from their own yard line
\item {\bf Yards-to-go:} This variables represents the number of yards needed for a first down
\item {\bf Ball Possession Time: } This variable captures the time that the offensive unit of the home team is on the field 
\item {\bf Ranking Differential: } This variable represents the difference of the win percentage for the two team (home - visiting)
\end{enumerate}

The last independent variable is representative of the power ranking difference between the two teams. 
Most of the existing models that include such a variable are using the Vegas line spread for each game.  
We choose not to do so for the following reason.  
The objective of the Vegas line is not to predict game outcomes but rather distribute money across the different bets.  
Exactly because of this objective the line is changing during the week before the game.  
While this line can change due to new information for the competing teams (e.g., injury updates), the line is mainly changing when a particular team has accumulated the majority of the bets. 
In this case it will also be hard to choose which line to use (e.g., the opening, the closing or some average of them).  
Therefore, we choose to use the win percentage differential of the two teams as an indicator of their strength (even though this has its own issues given the uneven schedule in NFL).  
However, note that if one would like to use the point spread as a variable this can be easily incorporated in the model. 
Table \ref{tab:iwinrnfl} presents the coefficients of the logistic regression model of {\method} with standardized independent variables for better comparisons. 


\begin{table}[ht]
\begin{center}
\def\sym#1{\ifmmode^{#1}\else\(^{#1}\)\fi}
\begin{tabular}{l*{1}{c}}
\toprule
                    &\multicolumn{1}{c}{(1)}\\
                    &\multicolumn{1}{c}{Winner}\\
\midrule
Possession Team (H)         &      0.41\sym{***}\\
                    &     (49.19)         \\
\addlinespace
Score Differential           &      3.59\sym{***}\\
                    &    (247.34)         \\
\addlinespace
Home Timeouts           &     0.12\sym{***}\\
                    &      (8.74)         \\
\addlinespace
Away Timeouts           &     -0.11\sym{***}\\
                    &    (-12.47)         \\
\addlinespace
Ball Possession Time  &     -0.05.\\
                    &    (-1.66)         \\
\addlinespace
Time Lapsed       &   -0.05.\\
                    &      (-1.66)         \\
\addlinespace
Down                &   -0.01         \\
                    &      (0.04)         \\
\addlinespace
Field Position            &   0.02\sym{**} \\
                    &      (2.71)         \\
\addlinespace
Yards-to-go                &  -0.01         \\
                    &      (0.23)         \\
\addlinespace
Rating differential         &       0.75\sym{***}\\
                    &     (80.47)         \\
\addlinespace
Intercept            &       0.57\sym{*}\\
                    &    (2.09)         \\
\midrule
Observations        &      338,294         \\
\bottomrule
\multicolumn{2}{l}{\footnotesize \textit{t} statistics in parentheses}\\
\multicolumn{2}{l}{\footnotesize \sym{$_.$} \(p<0.1\), \sym{*} \(p<0.05\), \sym{**} \(p<0.01\), \sym{***} \(p<0.001\)}\\
\end{tabular}
\end{center}
\caption{Standardized logisitic regression coefficients for {\method}.}
\label{tab:iwinrnfl}
\end{table}


As we can see, as one might have expected the current scoring differential exhibits the strongest correlation with the in-game win probability.  
The only factors that do not appear to be statistically significant predictors of the dependent variable are the down and the yards-to-go. 
Even though the corresponding coefficients are negative as one might have expected (e.g., being at an earlier down gives you more chances to advance the ball), they are not significant in estimating the win probability. 
On the contrary, all else being equal timeouts appear to be quiet important since they can help a team stop the clock, while teams with better win percentage appear to have an advantage as well, since this can be a sign of a better team. 
In the following section we provide a detailed evaluation of {\method}.

\section{Experimental protocols}\label{sec:protocols}
\subsection{Datasets}

We use the 
MovieLens--20M\footnote{http://grouplens.org/datasets/movielens}
and Netflix\footnote{http://www.netflixprize.com} explicit feedback  datasets.
As both  of these contain explicit ratings, we create binary preferences
by keeping ratings $\ge\!4$, which we interpret as positive feedback ($p_{ui}=1$).
Furthermore, we only keep users with at least 5 views.
Validation and test sets are obtained randomly, selecting a $10~\%$ of the original dataset for each set. We denote such datasets \textsc{ML20M} and \textsc{Netflix}.

In addition, we explore models performance on the Last.fm\footnote{https://www.upf.edu/web/mtg/lastfm360k} dataset~\cite{Celma:Springer2010}, an implicit feedback dataset consisting of tuples (user, artist, plays), that contains top artists by user. In order to make the comparison with the above datasets more straightforward, we binarize play counts and interpret them as implicit preference data. Next, we filter out artist with less than 50 distinct listening users, and user with less than 20 artists in their listening history. In the following, we name this dataset \textsc{Lastfm}.

\setlength{\belowcaptionskip}{5pt}
\begin{table}[htb]
\begin{tabular}{c c c c c}
 Dataset & \#users & \#items & \#pairs & \#pairs$_{\rm test}$ \\
\hline
\textsc{ML20M} & 136,7k & 20,3k & 7,99M & 1,0M \\
\textsc{Netflix}  & 463,4k & 17,7k & 45,5M & 5,7M \\
\textsc{Lastfm}  & 350,2k & 24,6k & 12,8M & 1,6M \\
\hline
\end{tabular}
\caption{Statistics of the datasets after preprocessing.}
\label{table:datasets}
\end{table}

The statistics of the training set after such  processing, as well as the number of user-item interactions in test, are presented in Table~\ref{table:datasets}. 
%We also represent the item popularity distribution in training, validation and test sets in Figure~\ref{fig:ditribution_dataset} for the \textsc{ML20M},  \textsc{Netflix}, \textsc{Lastfm} datasets, from top to bottom, respectively. As expected, validation and test distributions  follow the distribution of the training set. 

% \begin{figure}
%     \centering
%     \includegraphics[width=.7\linewidth]{figures/distributions_dataset.png}
%     \caption{Normalized distributions of user-item interactions in train (green), validation (red) and test (blue) sets for the  \textsc{ML20M},  \textsc{Netflix} and \textsc{Last-fm} datasets (from top to bottom, respectively).}
%     \label{fig:ditribution_dataset}
% \end{figure}

\subsection{Evaluation metrics}\label{subsec:metrics}
\setlength{\belowcaptionskip}{-10pt}
Given the set of adopted items in test, $\mathcal{I}_u^{\rm t}$, and the ranked list of predicted preferences, 
the relevance of a recommendation at position $k$ is given by ${\rm rel}_{ui}(k)$--${\rm rel}(k)$ from here on--, which equals $1$ if user $u$ adopted item $i$ in the test set, $0$ otherwise. In the calculation of metrics, we remove items observed in  training and validation from the list of recommendations. 
Next, we detail the metrics used for model evaluation.

\paragraph{Recall} It does not account for the relative ordering of the recommendation, and we defined it as~\cite{liang:2018:VAE}
\begin{equation}
{\rm Recall}@k = \frac{\sum_{s=1}^k {\rm rel}(s)}
{\mathcal{N}_u(k)}.
\end{equation}
Here, $\mathcal{N}_u(k) = \min\left(k,|\mathcal{I}_u^{\rm t}|\right)$, with $|\mathcal{I}_u^{\rm t}|$ the number of items adopted by user $u$ in testing. The final recall is averaged across all users in testing. 

\paragraph{Normalized Discount Cumulative Gain} In contrast to recall metric, the Discount Cumulative Gain (DCG) performs a logarithmic discount according to the position of a recommendation, that is
\begin{equation}
{\rm DCG}@k = \sum_{s=1}^k \frac{{\rm rel}(s)}
{\log_2(s+1)}.
\end{equation}
This quantity can be normalized by the Ideal DCG, 
\begin{equation}
{\rm IDCG}@k = \sum_{s=1}^{\mathcal{N}_u(k)} \frac{1}{\log_2(s+1)}.
\end{equation}
Finally, NDCG$@k= {\rm DCG}@k/{\rm IDCG}@k$, which we average across all users in the test set.

\paragraph{Novelty} Following reference~\cite{Vargas:2011:Novelty_diversity}, we define a novelty-weighted DCG score as
\begin{equation}\label{eq:nov-ndcg}
{\rm Nov\!-\!DCG}@k = 
-\sum_{s=1}^k \frac{{\rm rel}(s)\times \ln{\nu(i)}}
{\log_2(s+1)}.
\end{equation}
Here, $\nu(i)$ is the frequency of occurrences of item $i$  normalized to the total interactions in training. The corresponding novelty-weighted IDCG would be
\begin{equation}
{\rm Nov\!-\!IDCG}@k = 
\sum_{s=1}^{\mathcal{N}_u(k)} 
\frac{\max_{i\in\mathcal{I}_u}\left(-\ln \nu(i)\right)}
{\log_2(s+1)}.
\end{equation}
In other words, the highest DCG is obtained by ranking the most novel items (among those relevant to the user) in descending order. 
\subsection{Implementation details}\label{subsec:implementation}
The implementation of our model is performed in TensorFlow~\cite{tensorflow2015-whitepaper}.
%\footnote{The code will be available online at
%\url{https://github.com/bbvadata/RecApp}.
%}. 
The model can be trained in both CPU or GPU. 
When GPU is enabled, the use of queues to feed the tensors greatly speeds up the training.
% batch size and epoch
We set the batch size to $100$, and train every DAE model for $120$k iterations, so as to ensure proper convergence. For MF models we use $180$k iterations.
% number of neurons
The number of neurons is $200$ in all DAE experiments; for MF models, since the large number of users makes them prone to overfit, we train the models with $100$ and $200$ neurons and take the best performing model.
% weights and biases initialization
Weight matrices are initialized with random uniform values whose amplitude is computed as described by Glorot~\emph{et al.}~\cite{Xavier_initialization}. 
%Generally speaking, we discourage the use of random normal initialization without truncation.
For the biases we use a truncated random normal initialization with a standard deviation of $10^{-3}$. 
% Optimizer
Models are trained with Adam optimizer~\cite{Kingma2014AdamAM} and a learning rate of $10^{-3}$.
% gradient clip
%Additionally, we clip gradients whenever they exceed a norm of $1$, and apply batch normalization~\cite{icml2015_ioffe15_batch-norm} at every training step.

% Size of T and P sets
Concerning negative sampling in point and pair--wise schemes, 
%since we train with a batch size different form one, 
we fix the size of the target sets for every user (sets $\mathcal{T}_u$ and $\mathcal{P}_u$ for point and pair--wise learning, respectively, see subsection~\ref{subsec:losses}).
In particular, we make such sets proportional to the median number of items adopted by users, except for the multinomial loss, where all items are utilized~\cite{Liang:2016:CoFactor}.
The proportionality factors are hyper-parameters fine--tuned with the validation set, swapping the values $\{1,\,5,\,10,\,50,\,100,\,150\}$. We find a factor of $50$ or $100$ to provide the best results.
%(see additional comments in subsection~\ref{subsec:baseline_models}).

% Max norm regularization
%Regarding max--norm regularization, we notice that an asymmetric max--norm regularization might be required for the encoder/decoder weight matrices, depending on the activations and objective functions utilized for modeling. We swap $\alpha_{\rm enc}$ and $\alpha_{\rm dec}$ values in $\{0.0, 0.05, 0.1, 0.2, 0.3, 0.5, 1.0\}$, where $\alpha=0.0$ means no regularization.

% Regularization: noise and weight-decay
We add noise to the input vector of the AE~\cite{Vincent:2008:ECRF-AE, Wu:2016:CDAE-topN} using drop-out~\cite{liang:2018:VAE}. We fix the level of noise at $0.5$. Competitive performance is achieved after normalizing the AE input vector.
% Regularization of MIL is smaller
For DAE models, we swap the $L_2$ regularization strength $\lambda\in[10^{-7}-10^{-4}]$, while for MF models we take the form in equation~(\ref{eq:l2_reg_scaled}) with $\lambda\in[10^{0}-10^{3}]$, which provides a more stable training for MF models\footnote{Recall the different scales of the $\lambda$ factor in equations (\ref{eq:L2_reg}) and (\ref{eq:l2_reg_scaled}).}. In general we find that MIL models require smaller $\lambda$ factors than cross-entropy or multinomial--based models. This is expected, as the level of weight--decay regularization in equations~(\ref{eq:L2_reg}) and (\ref{eq:l2_reg_scaled}) depends on the value of the loss, which is smaller for MIL models.

\subsection{Baseline models}\label{subsec:baseline_models}
We implement the objective functions described in subsection \ref{subsec:losses} on a user-based DAE~\cite{Sedhain:2015:Autorec, Wu:2016:CDAE-topN} and compare the results with the MIL function. We also compare them with traditional Matrix Factorization with Weight Regularization~\cite{HuKoren:2008:CF_implicit}. In the following, we provide details on the training of the different models.

\textbf{Weight-Regularized Matrix Factorization} WRMF~\cite{HuKoren:2008:CF_implicit} is a linear factorization model  trained with square loss and weight decay. We use negative sampling with a sampling ratio of $100$ and $\lambda\sim 5-10$ (as obtained in the validation set). We call this model \MFsquare. In addition, we train WRMF models with MIL and point--wise cross--entropy losses, applying a sigmoid function at the output, so as to ensure $\hat{p}_{ui}\in(0,1)$. In these cases, we find that a sampling ratio of $100$ and $\lambda=50-500$ provide best results. We name these models \MFmil\, and \MFce, respectively.

\textbf{\emph{Denoising Autoencoder models}}

\textbf{Cross-entropy loss} For the cross-entropy loss defined in equations~(\ref{eq:cross-entropy}), (\ref{eq:point-wise}) and (\ref{eq:pair-wise}), we use linear--sigmoid and sigmoid--sigmoid activations at the encoder and decoder, respectively. We name the DAEs models with cross-entropy loss and point--wise estimation \CEpointlinsig\,and \CEpointsigsig; and those with pair--wise,  \CEpairlinsig \,and \CEpairsigsig. In order to prevent numerical instabilities, we ensure that the output preferences are in $[\varepsilon, 1-\varepsilon]$, with $\varepsilon=10^{-5}$. Regarding negative sampling, we find that the best sampling ratio is $50\times {\rm median}(\mathcal{I}_u)$ and $100\times {\rm median}(\mathcal{I}_u)$  for point and pair--wise estimation, respectively. Best weight-decay regularization is found to be $\lambda=2\cdot 10^{-5}$.

% Comparison to CDAE
The closest model to these baselines is the Collaborative Denoising AE (CDAE)~\cite{Wu:2016:CDAE-topN}, although for the sake of simplicity, in the present paper we do not include the user embedding of CDAE. 
% bad performance of BPR
Similar to CDAE, we find that  pair--wise learning does not achieve competitive results at the top of the ranked list~\cite{Wu:2016:CDAE-topN, liang:2018:VAE}. 

\textbf{Multinomial loss} AEs trained with a multinomial log-likelihood have recently been  introduced by Lian et al~\cite{liang:2018:VAE}, either applied to DAEs or Variational AEs (VAE) with partial regularization. Here, we focus on the multi-DAE modeling with $\tanh$-linear activations\footnote{
We use the actual implementation provided at \url{https://github.com/dawenl/vae_cf} to verify that the activation used at the decoder of multi-DAE is linear, although the original writing~\cite{liang:2018:VAE} suggests a $\tanh$ non-linearity for the decoder.
}, and name this baseline \MULTItanhlin. Our implementation exactly reproduces that of~\cite{liang:2018:VAE} when using $\lambda=2\cdot 10^{-5}$, input noise of $0.5$ and without applying negative sampling. 

\textbf{Missing Information loss} We apply the \textsc{MIL} function defined in equation~(\ref{eq:mil_def}) to linear-sigmoid and sigmoid-sigmoid DAEs. We name these models \MILlinsig\, and \MILsigsig, respectively. Best hyper-parameters of the loss turn out to be $A_{\rm MI}=10^6,\, \gamma_{\rm MI}=10$ and $\gamma_{+}=1$, after grid search the pairs $\left(A_{\rm MI}, \gamma_{\rm MI}\right)\in\{
(5\cdot10^1, 2)$, $(10^3, 4)$, $(2\cdot10^4, 6)$, $(5\cdot10^5, 10)$, $(1\cdot10^6, 10)$, $(5\cdot10^6, 10)$, $(5\cdot10^9, 15)\}$, and $\gamma_{+}=1$ or $2$. In addition, we use a sampling ratio of $50$ and $\lambda\in(10^{-6}, 10^{-5})$.

\begin{table}[t!]
\centering
\caption{Voice conversion \& F0 manipulation results. MOS results are reported with 95\% confidence interval. VDE, and FFE are reported for F0 manipulation while PER, WER, EER, and MOS are reported for voice conversion. Notice, for VDE, and FFE higher is the better since F0 was flattened.}
\label{tab:conv}

\resizebox{1\columnwidth}{!}{
\begin{tabular}{c@{~} | c@{~} | c@{~}c@{~} | c@{~} | c@{~} ||  c@{~}c@{~} }
\toprule
\multirow{2}{*}{Dataset} & \multirow{2}{*}{Method} & \multicolumn{4}{c||}{Voice Conversion} & \multicolumn{2}{c}{F0 Manipulation} \\
\cmidrule{3-8}
& & PER~$\downarrow$ & WER~$\downarrow$ & EER~$\downarrow$ & MOS~$\uparrow$ & VDE~$\uparrow$ & FFE~$\uparrow$ \\
\midrule
VCTK & GT  & 17.16 & 4.32 & 3.25 & 4.11$\pm$0.29 & -- & -- \\
\midrule 
\multirow{3}{*}{LJ}
% & ASR-TTS   & 50.74  & --     & 66.08 & 32.96 & 1.46 \\
& CPC       & 22.22 	& 16.11 		& 0.46 		& 3.57$\pm$0.15 		& \bf 46.68 & \bf 48.71\\
& HuBERT    & \bf 19.09 & \bf 12.23 & \bf 0.31  & \bf 3.71$\pm$0.24 & 39.20 		& 48.42\\
& VQ-VAE    & 40.88 	& 36.96 		& 9.65 		& 2.90$\pm$0.17 		& 10.54 	& 12.08 \\
\midrule 
\multirow{3}{*}{VCTK} 
% & ASR-TTS   & 68.88  & --    & 41.77 & 13.55 & 6.48 \\
& CPC       &  23.58 		& 15.98 		& \bf 4.83  &  3.42 $\pm$ 0.24 		& \bf 25.29 & \bf 26.97 \\
& HuBERT    &  \bf 20.85 	& \bf 12.72 & 6.01  		& \bf  3.58 $\pm$ 0.28 	& 23.46 	& 26.67 \\
& VQ-VAE    & 36.88  		& 29.44 		& 11.56 		& 3.08 $\pm$ 0.34 		& 7.03  	& 7.80  \\
\bottomrule
\end{tabular}}
\vspace{-0.4cm}
\end{table}

\vspace{-0.1cm}
\section{Results}
\vspace{-0.1cm}
Our results cover
% We report results for 
three different settings: (i) speech reconstruction experiments; (ii) speaker conversion and F0 manipulation; (iii) bitrate analysis with subjective tests for speech codec evaluation. We employ two datasets: LJ~\cite{ljspeech17} single speaker dataset and VCTK~\cite{vctk} multi-speaker dataset. All datasets were resampled to a 16kHz sample rate.

% \paragraph*{Implementation Details.}
% \smallskip
\noindent{\bf Implementation Details\quad} 
\label{sec:impl}
We follow the same setup as in~\cite{lakhotia2021generative}. For CPC, we used the model from~\cite{Riviere2020}, which was trained on a ``clean'' 6k hour sub-sample of the LibriLight dataset~\cite{Kahn2020,Riviere2020}. We extract a downsampled representation from an intermediate layer with a 256-dimensional embedding and a hop size of 160 audio samples. For HuBERT we used a \textsc{Base} 12 transformer-layer model trained for two iterations~\cite{hsu2020hubert} on 960 hours of LibriSpeech corpus~\cite{Panayotov2015}. 
% This model encodes every 320 raw audio samples into a 768-dimensional vector. 
This model downsamples the raw audio $\times320$ into a sequence of 768-dimensional vectors. Similarly to~\cite{lakhotia2021generative}, activations were extracted from the sixth layer.

%CPC: We use a dictionary of 100 units, leading to a bitrate of 700bps.
%HuBERT: A dictionary of 100 units is used, leading to a bitrate of 350bps. 
%VQVE: The VQ-VAE discrete code operates at a bitrate of 800bps.
% For both CPC and HuBERT, the k-means algorithm is applied to convert continuous frames to discrete codes, using the LibriSpeech clean-100h~\cite{Panayotov2015} dataset. 
For CPC and HuBERT, the k-means algorithm is trained on LibriSpeech clean-100h~\cite{Panayotov2015} dataset to convert continuous frames to discrete codes. We quantize both learned representations with $K=100$ centroids. Leading to a bitrate of 700bps for CPC and 350bps for HuBERT.

% VQ-VAE
Similarly to CPC models, we trained the VQ-VAE content encoder model on the ``clean'' 6K hours subset from the LibriLight dataset. We use an encoder operating on the raw signal to extract discrete units, similar to~\cite{jukebox}. In addition, ``random restarts'' were performed when the mean usage of a codebook vector fell below a predetermined threshold. Finally, we used HiFiGAN (architecture and objective) as the decoder instead of a simple convolutional decoder, as it improved the overall audio quality. This model encodes the raw audio into a sequence of discrete tokens from 256 possible tokens~\cite{garbacea2019low} with a hop size of 160 raw audio samples. The VQ-VAE discrete code operates at a bitrate of 800bps. We additionally experimented with 100 discrete units for VQ-VAE, however results were the best for 256. This finding is consistent with~\cite{garbacea2019low}.

% verification model
The speaker verification network uses the architecture proposed in~\cite{heigold2016end}. It was trained on the VoxCeleb2~\cite{voxceleb2} dataset, achieving a 7.4\% Equal Error Rate (EER) for speaker verification on the test split of the VoxCeleb1~\cite{Nagrani17} dataset.

% pitch
Only a single F0 representation is considered across all evaluated models, trained on the VCTK dataset.
% The F0 is extracted from the raw audio using YAAPT~\cite{yaapt} algorithm, using a window size of 20ms and a 5ms hop. 
The F0 is extracted from the raw audio using a window size of 20ms and a 5ms hop. 
As a result, the F0 sequence is sampled at 200Hz. 
% We apply the quantization described at Sec.~\ref{sec:method}, using a pitch codebook of $K'=20$ tokens and an encoder that downsamples the pitch by $\times16$. 
The quantization described at Sec.~\ref{sec:method}, is applied using an F0 codebook of $K'=20$ tokens and an encoder that downsamples the signal by $\times16$. Hence, the discrete F0 representation is sampled at 12.5Hz, leading to a bitrate of 65bps. The final bitrate of the evaluated codecs is the sum of the pitch code bitrate with the content code bitrate.

% \paragraph*{Evaluation Metrics}
% \smallskip
\noindent{\bf Evaluation Metrics\quad} 
We consider both subjective and objective evaluation metrics. For subjective tests, we report the Mean Opinion Scores (MOS). In which human evaluators rate the naturalness of audio samples on a scale of 1--5. Each experiment, included 50 randomly selected samples rated by 30 raters. For objective evaluation, we consider: (i) Equal Error Rate~(EER) as an automatic speaker verification metric obtained using a pre-trained speaker verification network. We report EER between test utterances and enrolled speakers; (ii) Voicing Decision Error (VDE)~\cite{nakatani2008method}, which measures the portion of frames with voicing decision error; (iii) F0 Frame Error (FFE)~\cite{chu2009reducing}, measures the percentage of frames that contain a deviation of more than 20\% in pitch value or have a voicing decision error; (iv) Word Error Rate (WER) and Phoneme Error Rate (PER), proxy metrics to the intelligibility of the generated audio. We used a pre-trained ASR network~\cite{baevski2020wav2vec} on both reconstructed and converted samples to calculate both metrics. %To generate target phonemes, the g2p-en~\cite{g2pE2019} Grapheme2Phoneme module was used.

% \vspace{-0.1cm}
% \smallskip
\noindent{\bf Reconstruction \& Conversion}
% \vspace{-0.1cm}
We start by reporting the reconstruction performance. Results are summarized in Table~\ref{tab:recon}. When considering the intelligibility of the reconstructed signal HuBERT reaches the lowest PER and WER scores across all models, where both CPC and HuBERT are superior to VQ-VAE. However, when considering F0 reconstruction VQ-VAE outperforms both HuBERT and CPC by a significant margin. This results are somewhat intuitive, bearing in mind VQ-VAE objective is to fully reconstruct the input signal. In terms of subjective evaluation, all models reach similar MOS scores, with one exception of CPC on LJ. 

%Notice, since the same F0 units are used for each method, this result implies the VQ-VAE units contain some information about the F0 of the signal, enabling better reconstruction. Regarding speaker information, the CPC gets the lowest EER. 

To better evaluate the disentanglement properties of each method with respect to speaker identity and F0, we conducted an additional set of experiments aiming at speaker conversion and F0 manipulation. For voice conversion, we converted each test utterance into five random target speakers. Next, we employed a speaker verification network, which extracts \emph{d-vector} representation to evaluate speaker-converted utterances' similarity to real speaker utterances (low error-rate indicates good conversion), providing measurement to the speaker identity's disentanglement from the evaluated coding method. The error-rate is reported between converted test utterances and enrolled speakers. For the LJ speech single speaker dataset, we converted samples from the VCTK dataset to the single speaker and enrolled all VCTK speakers together with the single speaker. Results are summarized in Table~\ref{tab:conv} (left). Unlike resynthesis results, on voice conversion CPC and HuBERT outperform VQ-VAE on both LJ and VCTK datasets, indicating VQ-VAE contains more information about the speaker in the encoded units, hence producing more artifacts. Notice, this also affects WER, PER, and the overall subjective quality (MOS). 

Next, to evaluate the presence of F0 in the discrete units, we flattened the F0 units before synthesizing the signal and calculated VDE and FFE with respect to the original F0 values. F0 flattening was done by setting the speakers' mean F0 value across all voiced frames. In this experiment, we expected units that contain F0 information to be better at F0 reconstruction over disentangled units. Results are summarized in Table~\ref{tab:conv} (right). Notice VQ-VAE can still reconstruct the F0 almost at the same level as when using the original F0 as conditioning (5.2 vs 7.03, and 5.59 vs 7.8), in contrast to CPC and HuBERT.

\begin{figure}[t!]
\centering
\includegraphics[width=0.65\columnwidth, trim={50 20 70 20}]{figures/codec_2.pdf}
% \caption{MUSHRA subjective listening test results as a function of bitrate per second for various methods. Purple dots denote the baseline methods, and green dots the proposed SSL based method.} 
\caption{MUSHRA subjective quality results as a function of bitrate per second. Purple dots denote the baseline methods, and green dots the proposed SSL based method.} 
\label{fig:codec}
\vspace{-0.5cm}
\end{figure}

% \vspace{-0.1cm}
% \smallskip
\noindent{\bf Speech Codec}
Our final experiment evaluates the obtained speech units as a low bitrate speech codec. 
% Therefore, we evaluate how the performance varies as a function of the number of discrete units. Changing the number of units is equivalent to varying the bitrate of the encoded signal. 
We use a subjective MUSHRA-type listening test~\cite{series2014method} to measure the perceived quality of the proposed speech codec with regard to its bitrate constraints. In MUSHRA evaluations, listeners are presented with a labeled uncompressed signal for reference, a set of test samples to rate, a copy of the uncompressed reference, and a low-quality anchor. Listeners are asked to rate each test utterance and the copy of the uncompressed reference with respect to the labeled reference in a scale of 1-100.

The experiment is performed on the VCTK dataset~\cite{vctk}. For evaluation, we used 20 utterances from 5 speakers. The set of speakers in the test data is disjoint with those in the training data. For this experiment, HuBERT models with 50, 100, and 200 units were trained as described in Sec.~\ref{sec:impl}. For comparison, we included other speech codecs in our evaluation: Opus~\cite{valin2012definition} wideband at 9 kbps VBR, Codec2~\cite{rowe2011codec} at 2.4 kbps and LPCNet~\cite{valin2019real} operating at 1.6 kbps. The LPCNet model was trained from scratch on the VCTK dataset following the experimental setup in~\cite{valin2019real}. The VQ-VAE model employs the HiFiGAN decoder trained on the LibriLight dataset to match the amount of data reported in~\cite{garbacea2019low}. We compressed the anchor sample with Speex~\cite{valin2016speex} at 4 kbps as a low anchor. Fig.~\ref{fig:codec} depicts the results. HuBERT with 50 units reaches the best MUSHRA score while its bitrate is only 365bps, which is significantly lower than the baseline methods.

\section{Conclusions and next steps}\label{sec:conclusions}

% definition of MIL
In this paper we present a novel objective function, the \emph{Missing Information Loss} (MIL), specifically designed for handling unobserved user-item interactions in implicit feedback datasets. In particular, MIL explicitly forbids treating missing user-item interactions as positive or negative feedback.
% What it does
We demonstrate that, thanks to the functional form of the MIL function, the ranking of unseen items is almost entirely left to the low--rank process, rather than forcing unobserved items to be at the tail of the recommendation (\emph{i.e.}, MIL does not force a zero predicted preference for unobserved user-item interactions). 

% Metric results
Extensive experiments with Matrix Factorization and Denoising Autoencoders conducted on three datasets, show that \textsc{MIL} models demonstrate competitive performance when compared with other traditional losses such as cross-entropy or the multinomial log-likelihood. 
% Best performing models 
% Analysis of recommendations
In addition, we study the distribution of the recommendations and observe that the reported metric performance takes place while recommending popular items less frequently (up to a $20 \%$ decrease with respect to the best competing method). Indeed, \textsc{MIL} models sharply increase the recommendation of medium--tail items, while almost linearly expanding the appearance of long--tail items with the ranking position in the list of recommendations. Such expansion results in up to a $50 \%$ increase of long--tail recommendations, a feature of utmost importance for industries with a large catalogue of items. 

% Future work
Future lines of research may involve the incorporation of negative feedback, or the usage of \textsc{MIL} in temporal--aware Recommender Systems (such as those using Recurrent Neural Networks).  
In addition, we hope that the results here reported  will bring forward first-principle mathematical derivations of the \textsc{MIL} function, so that the vast family of possible polynomials modelling the missing information term can be reduced, or even extended with more suitable functions. 

\begin{acks}
We would like to thank the continuous support and careful reading of the manuscript by the \emph{Edge} guild within BBVA Data \& Analytics, specially J. Garc\'ia Santamar\'ia and J. A. Rodr\'iguez Serrano. 
\end{acks}

%\bibliographystyle{ACM-Reference-Format}
\bibliographystyle{unsrt}
%\bibliography{sigproc} 

\documentclass[sigconf]{acmart}

\settopmatter{printacmref=false} % Removes citation information below abstract
\renewcommand\footnotetextcopyrightpermission[1]{} % removes footnote with conference information in first column
\pagestyle{plain} % removes running headers

\usepackage{booktabs} % For formal tables
\usepackage{multirow}
\usepackage{amsmath}
\usepackage{color}
\usepackage{arydshln }


%Conference
\acmConference[WSDM]{The Twelfth International Conference on Web Search and Data Mining}{February 11--15}{Melbourne, Australia} 
\acmYear{2019}

\begin{document}
\title[Missing Information Loss]{A Missing Information Loss for implicit feedback datasets}

\author{Juan Ar\'evalo}
\affiliation{%
  \institution{BBVA Data \& Analytics}
}
\email{juanmaria.arevalo@bbvadata.com}

\author{Juan Ram\'on Duque}
\affiliation{%
  \institution{BBVA Data \& Analytics}
}
\email{juanramon.duque@bbvadata.com}

\author{Marco Creatura}
\affiliation{%
  \institution{BBVA Data \& Analytics}
}
\email{marco.creatura@bbvadata.com}

% The default list of authors is too long for headers}
\renewcommand{\shortauthors}{Ar\'evalo, Duque and Creatura}

% Some useful commands
\newcommand{\MFsquare}{\textsc{MF-square}}
\newcommand{\MFmil}{\textsc{MF-mil}}
\newcommand{\MFce}{\textsc{MF-CE}}
\newcommand{\CEpointlinsig}{\textsc{CE$_{\rm Point}$ lin-sig}}
\newcommand{\CEpointsigsig}{\textsc{CE$_{\rm Point}$ sig-sig}}
\newcommand{\CEpairlinsig}{\textsc{CE$_{\rm Pair}$ lin-sig}}
\newcommand{\CEpairsigsig}{\textsc{CE$_{\rm Pair}$ sig-sig}}
\newcommand{\MULTItanhlin}{\textsc{MULTI tanh-lin}}
\newcommand{\MILlinsig}{\textsc{MIL lin-sig}}
\newcommand{\MILsigsig}{\textsc{MIL sig-sig}}


\begin{abstract}
% missing values and negative feedback
Latent factor models for Recommender Systems with implicit feedback typically treat unobserved user-item interactions (\emph{i.e.} missing information) as negative feedback. This is frequently done 
either through negative sampling (point--wise loss) or with a ranking loss function (pair-- or list--wise estimation). 
% Common objective functions allow zero prediction
Since a zero preference recommendation is a valid solution for most common objective functions, 
regarding unknown values as actual zeros results in users 
having a zero preference recommendation  for most of the available items. 

% MIL
In this paper we propose a novel objective function, the \emph{Missing Information Loss} (MIL), 
that explicitly forbids treating unobserved user-item interactions as positive or negative feedback. 
% application to AE and metrics
We apply this loss to both traditional Matrix Factorization and user--based Denoising Autoencoder, and compare it with other established objective functions such as cross--entropy (both point-- and pair--wise) or the recently proposed multinomial log-likelihood. MIL achieves competitive performance in ranking--aware metrics when applied to three datasets.
% towards long-tail recommendations
Furthermore, we show that such a relevance in the recommendation is obtained while displaying popular items less frequently (up to a $20 \%$ decrease with respect to the best competing method). This debiasing from the recommendation of popular items favours the appearance of infrequent items (up to a $50 \%$ increase of long--tail recommendations), a valuable feature for Recommender Systems with a large catalogue of products. 
\end{abstract}

\begin{CCSXML}
<ccs2012>
<concept>
<concept_id>10002951.10003317.10003347.10003350</concept_id>
<concept_desc>Information systems~Recommender systems</concept_desc>
<concept_significance>500</concept_significance>
</concept>
</ccs2012>
\end{CCSXML}

\ccsdesc[500]{Information systems~Recommender systems}

\keywords{Collaborative Filtering, Autoencoders, Implicit Feedback, Missing Information}

\setcopyright{None}

\maketitle

\setlength{\abovecaptionskip}{0pt}
\setlength{\belowcaptionskip}{-10pt}

% Put all the sections with inputs

Reinforcement learning has achieved great success in areas such as Game-playing \citep{silver2018general,vinyals2019grandmaster}, robotics \cite{kober2013reinforcement}, large language models \citep{ouyang2022training}, etc.
However, due to safety concerns or physical limitations, in some real-world reinforcement learning problems, we must consider additional constraints that may influence the optimal policy and the learning process \citep{garcia2015comprehensive}.
% For example, a robotic arm must not take actions that may cause harm to itself or the environments.
A standard framework to handle such cases is the constrained Markov Decision Process (CMDP) \citep{altman1999constrained}.
Within the CMDP framework, the agent has to maximize
the expected cumulative reward while
obeying a finite number of constraints, which are usually in the form of expected cumulative cost criteria.

However, we are sometimes concerned with the problem with a continuum of constraints.
For example,
the constraints we meet might be time-evolving or subject to uncertain parameters, which
cannot be formulated as an ordinary CMDP
(see Examples \ref{Example_Time_Evolving} and  \ref{Example_Uncertain}).
In this paper we would study a generalized CMDP  
to address the above problem.  Because the constraints are not only infinite-number but also lie
in a continuous set,
the generalization is not trivial. Fortunately, we find that we can borrow the idea behind semi-infinite programming (SIP) \citep{remez1934determination, hettich1993semi} to deal with the semi-infinite constraints.
Accordingly, we propose \emph{semi-infinitely constrained Markov decision processes} (SICMDPs)
as a novel complement to the ordinary CMDP framework.
%More specifically,  an SICMDP model %, we consider 
%contains a continuum of constraints whereas an ordinary CMDP contains a finite number of constraints. 

%This generalization is natural but not trivial. However, we can brows the idea  
%The idea is quite natural and can be backtracked
%to the practice of extending linear programming to linear semi-infinite programming (LSIP) %\cite{remez1934determination, GobernaLSIO1998}.
%In addition, 
%As a complementary approach to the ordinary CMDP framework, 
%SICMDP can be used to model these problems  which cannot be described by a finite number of constraints
%that are not covered by .
%For example,
%the restrictions we consider can be time-evolving or subject to uncertain parameters
%, thus
%cannot be described by a finite number of constraints but a continuum of constraints 
%(see Examples \ref{Example_Time_Evolving} and  \ref{Example_Uncertain}).

We also present two reinforcement learning algorithms to solve SICMDPs called SI-CRL and SI-CPO, respectively.
SI-CRL is a model-based reinforcement learning algorithm designed for tabular cases, and SI-CPO is a policy optimization algorithm for non-tabular cases.
% and analyze its performance both theoretically and empirically.
The main challenge is that we need to deal with a continuum of constraints, thus reinforcement learning algorithms for ordinary CMDPs do not work anymore.
In SI-CRL, we tackle this difficulty by first transforming the reinforcement learning problem to an equivalent LSIP problem, which can then be solved using methods in the LSIP literature like the dual exchange methods \citep{Hu1990,reemtsen1998numerical}.
In SI-CPO, we resort to the idea of cooperative stochastic approximation developed in \cite{lan2020algorithms, wei2020comirror}.
As far as we know, we are the first to introduce tools from semi-infinitely programming (SIP) into the reinforcement learning community for solving constrained reinforcement learning problems.

% To the best of our knowledge, we are the first to apply tools from semi-infinitely programming (SIP) to solve reinforcement learning problems.
Furthermore, we give theoretical analysis for both SI-CRL and SI-CPO.
We decompose the error of SI-CRL into two parts: the statistical error from approximating the true SICMDP with an offline dataset and the optimization error due to the fact that the solution of the LSIP problem obtained by the dual exchange method is inexact.
On the optimization side, we show that the iteration complexity of SI-CRL is $O\left(\left\{\mathrm{diam}(Y)L\sqrt{|\gS|^2|\gA|m}/\left[(1-\gamma)\epsilon\right]\right\}^m\right)$.
On the statistical side, we show that the sample complexity of SI-CRL is $\widetilde O\left(\frac{|S|^2|A|^2}{\epsilon^2(1-\gamma)^3}\right)$ if the offline dataset is generated by a generative model, and $\widetilde O\left(\frac{|S||A|}{\nu_{\min} \epsilon^2(1-\gamma)^3}\right)$ if the dataset is generated by a probability measure $\nu$ as considered in \cite{chen2019information}.
Here $\widetilde O$ means that all logarithm terms are discarded.
For SI-CPO, things become a little more complicated because other than the statistical error and the optimization error, we also need to consider the function approximation error, which comes from imperfect policy parametrizations.
It is shown if the function approximation error can be controlled to $O(\epsilon)$ order, the iteration complexity of SI-CPO is $\widetilde{O}\left(\frac{1}{\epsilon^2(1-\gamma)^6}\right)$ and the sample complexity of SI-CPO is $\widetilde{O}(\frac{1}{\epsilon^4(1-\gamma)^{10}})$.
Here our iteration complexity bound is equivalent to a typical $\widetilde O(1/\sqrt{T})$ global convergence rate.

We perform a set of numerical experiments to illustrate the SICMDP model and validate our proposed algorithms.
Specifically, we examine two numerical examples, namely the discharge of sewage and ship route planning.
Through the discharge of sewage example, we show the advantage of the SICMDP framework over the CMDP baseline obtained by naive discretization in modeling realistic sequential decision-making problems.
Moreover, we demonstrate the effectiveness of the SI-CRL and SI-CPO algorithms in such tabular environments. 
In the ship route planning example, we illustrate the benefits of the SICMDP framework and the ability of the SI-CPO algorithm to address complex continuous control tasks involving continuous state spaces with modern deep reinforcement learning techniques.

% In summary, our contributions are listed as follows.
% First, we present the SICMDP model, which can be viewed as a generalization of the ordinary CMDP model.
% Second, we propose an algorithm to perform reinforcement learning for SICMDPs, which is called SI-CRL, and we believe that we are the first to apply tools from SIP
% to solve reinforcement learning problems.
% Third, we give a theoretical analysis of SI-CRL and identify both its sample complexity and iteration complexity.
% In addition, we perform numerical experiments to illustrate the SICMDP model and validate the SI-CRL algorithm.
% \{This paragraph can be removed!!! \}






\section{The \MakeLowercase{i}W\MakeLowercase{inr}NFL model}
\label{sec:model}

In this section we are going to present the data we used to develop our in-game probability model as well as the design details of {\method}. 

{\bf Data: }In order to perform our analysis we utilize a dataset collected from NFL's Game Center for all the regular season games between the seasons 2009 and 2016. 
We access the data using the Python {\tt nflgame} API \cite{nflgame}. 
The dataset includes detailed play-by-play information for every game that took place during these seasons. 
This information is used to obtain the state of the game that will drive the design of {\method}. 
In total, we collected information for 2,048 regular season games and a total of 338,294 snaps/plays. 

{\bf Model: }
{\method} is based on a logistic regression model that calculates the probability of the home team winning given the current status of the game as: 

\begin{equation}
\Pr(H=1| \mathbf{x})= \frac{\exp(\mathbf{\weight}^T\cdot\mathbf{x})}{1+\exp(\mathbf{\weight}^T\cdot\mathbf{x})}
\label{eq:reg}
\end{equation}
where $H$ is the dependent random variable of our model representing whether the home team wins or not, $\mathbf{x}$ is the vector with the independent variables, while the coefficient vector $\mathbf{\weight}$ includes the weights for each independent variable and is estimated using the corresponding data.  
For a game of infinite duration a linear model could be a very good approximation.  
However, the boundary effects from the finite duration of a game create several non-linearities \cite{winston2012mathletics}.  
For this reason, we enhance our model - using the same set of features - with a Support Vector Machine classifier with radial kernel for the last three minutes of regulation.  
In order to obtain a probability output from the SVM classifier, we further use Platt's scaling \cite{platt1999probabilistic}: 

\begin{equation}
\Pr(H=1| \mathbf{x})= \frac{1}{1+\exp{(Af(x)+B)}}
\label{eq:platt}
\end{equation}
where $f(x)$ is the uncalibrated value produced by the SVM classifier: 

\begin{equation}
f(x) = \sum_{i} (\alpha_i y_i k(\mathbf{x}_i\cdot\mathbf{x}))+ b
\label{eq:svm}
\end{equation}
where $k(\mathbf{x},\mathbf{x}')$ is the kernel used for the SVM.   
Figure \ref{fig:iwinrNFL} depicts the simple flow chart of {\method}. 


\begin{figure}[t]
\begin{center}
\includegraphics[scale=0.35]{plots/iwinrNFL.pdf}%\vspacecap
 \caption{{\method} includes a linear and a non-linear component.}
 \label{fig:iwinrNFL}
\end{center}
\end{figure}

In order to describe the status of the game we use the following variables:

\begin{enumerate}
\item {\bf Ball Possession Team:} This binary feature captures whether the home or the visiting team has the ball possession
\item {\bf Score Differential:} This feature captures the current score differential (home - visiting)
\item {\bf Timeouts Remaining:} This feature is represented by two independent variables - one for the home and one for the away team - and they capture the number of timeouts remaining for each of the teams
%\item {\bf Quarter:} This feature captures the current quarter of the game
%\item {\bf Time Remaining:} This feature captures the time (in seconds) remaining for the current quarter to end
\item {\bf Time Elapsed: } This feature captures the time elapsed since the beginning of the game
\item {\bf Down:} This feature represents the down of the team in possession
\item {\bf Field Position:} This feature captures the distance covered by the team in possession from their own yard line
\item {\bf Yards-to-go:} This variables represents the number of yards needed for a first down
\item {\bf Ball Possession Time: } This variable captures the time that the offensive unit of the home team is on the field 
\item {\bf Ranking Differential: } This variable represents the difference of the win percentage for the two team (home - visiting)
\end{enumerate}

The last independent variable is representative of the power ranking difference between the two teams. 
Most of the existing models that include such a variable are using the Vegas line spread for each game.  
We choose not to do so for the following reason.  
The objective of the Vegas line is not to predict game outcomes but rather distribute money across the different bets.  
Exactly because of this objective the line is changing during the week before the game.  
While this line can change due to new information for the competing teams (e.g., injury updates), the line is mainly changing when a particular team has accumulated the majority of the bets. 
In this case it will also be hard to choose which line to use (e.g., the opening, the closing or some average of them).  
Therefore, we choose to use the win percentage differential of the two teams as an indicator of their strength (even though this has its own issues given the uneven schedule in NFL).  
However, note that if one would like to use the point spread as a variable this can be easily incorporated in the model. 
Table \ref{tab:iwinrnfl} presents the coefficients of the logistic regression model of {\method} with standardized independent variables for better comparisons. 


\begin{table}[ht]
\begin{center}
\def\sym#1{\ifmmode^{#1}\else\(^{#1}\)\fi}
\begin{tabular}{l*{1}{c}}
\toprule
                    &\multicolumn{1}{c}{(1)}\\
                    &\multicolumn{1}{c}{Winner}\\
\midrule
Possession Team (H)         &      0.41\sym{***}\\
                    &     (49.19)         \\
\addlinespace
Score Differential           &      3.59\sym{***}\\
                    &    (247.34)         \\
\addlinespace
Home Timeouts           &     0.12\sym{***}\\
                    &      (8.74)         \\
\addlinespace
Away Timeouts           &     -0.11\sym{***}\\
                    &    (-12.47)         \\
\addlinespace
Ball Possession Time  &     -0.05.\\
                    &    (-1.66)         \\
\addlinespace
Time Lapsed       &   -0.05.\\
                    &      (-1.66)         \\
\addlinespace
Down                &   -0.01         \\
                    &      (0.04)         \\
\addlinespace
Field Position            &   0.02\sym{**} \\
                    &      (2.71)         \\
\addlinespace
Yards-to-go                &  -0.01         \\
                    &      (0.23)         \\
\addlinespace
Rating differential         &       0.75\sym{***}\\
                    &     (80.47)         \\
\addlinespace
Intercept            &       0.57\sym{*}\\
                    &    (2.09)         \\
\midrule
Observations        &      338,294         \\
\bottomrule
\multicolumn{2}{l}{\footnotesize \textit{t} statistics in parentheses}\\
\multicolumn{2}{l}{\footnotesize \sym{$_.$} \(p<0.1\), \sym{*} \(p<0.05\), \sym{**} \(p<0.01\), \sym{***} \(p<0.001\)}\\
\end{tabular}
\end{center}
\caption{Standardized logisitic regression coefficients for {\method}.}
\label{tab:iwinrnfl}
\end{table}


As we can see, as one might have expected the current scoring differential exhibits the strongest correlation with the in-game win probability.  
The only factors that do not appear to be statistically significant predictors of the dependent variable are the down and the yards-to-go. 
Even though the corresponding coefficients are negative as one might have expected (e.g., being at an earlier down gives you more chances to advance the ball), they are not significant in estimating the win probability. 
On the contrary, all else being equal timeouts appear to be quiet important since they can help a team stop the clock, while teams with better win percentage appear to have an advantage as well, since this can be a sign of a better team. 
In the following section we provide a detailed evaluation of {\method}.

\section{Experimental protocols}\label{sec:protocols}
\subsection{Datasets}

We use the 
MovieLens--20M\footnote{http://grouplens.org/datasets/movielens}
and Netflix\footnote{http://www.netflixprize.com} explicit feedback  datasets.
As both  of these contain explicit ratings, we create binary preferences
by keeping ratings $\ge\!4$, which we interpret as positive feedback ($p_{ui}=1$).
Furthermore, we only keep users with at least 5 views.
Validation and test sets are obtained randomly, selecting a $10~\%$ of the original dataset for each set. We denote such datasets \textsc{ML20M} and \textsc{Netflix}.

In addition, we explore models performance on the Last.fm\footnote{https://www.upf.edu/web/mtg/lastfm360k} dataset~\cite{Celma:Springer2010}, an implicit feedback dataset consisting of tuples (user, artist, plays), that contains top artists by user. In order to make the comparison with the above datasets more straightforward, we binarize play counts and interpret them as implicit preference data. Next, we filter out artist with less than 50 distinct listening users, and user with less than 20 artists in their listening history. In the following, we name this dataset \textsc{Lastfm}.

\setlength{\belowcaptionskip}{5pt}
\begin{table}[htb]
\begin{tabular}{c c c c c}
 Dataset & \#users & \#items & \#pairs & \#pairs$_{\rm test}$ \\
\hline
\textsc{ML20M} & 136,7k & 20,3k & 7,99M & 1,0M \\
\textsc{Netflix}  & 463,4k & 17,7k & 45,5M & 5,7M \\
\textsc{Lastfm}  & 350,2k & 24,6k & 12,8M & 1,6M \\
\hline
\end{tabular}
\caption{Statistics of the datasets after preprocessing.}
\label{table:datasets}
\end{table}

The statistics of the training set after such  processing, as well as the number of user-item interactions in test, are presented in Table~\ref{table:datasets}. 
%We also represent the item popularity distribution in training, validation and test sets in Figure~\ref{fig:ditribution_dataset} for the \textsc{ML20M},  \textsc{Netflix}, \textsc{Lastfm} datasets, from top to bottom, respectively. As expected, validation and test distributions  follow the distribution of the training set. 

% \begin{figure}
%     \centering
%     \includegraphics[width=.7\linewidth]{figures/distributions_dataset.png}
%     \caption{Normalized distributions of user-item interactions in train (green), validation (red) and test (blue) sets for the  \textsc{ML20M},  \textsc{Netflix} and \textsc{Last-fm} datasets (from top to bottom, respectively).}
%     \label{fig:ditribution_dataset}
% \end{figure}

\subsection{Evaluation metrics}\label{subsec:metrics}
\setlength{\belowcaptionskip}{-10pt}
Given the set of adopted items in test, $\mathcal{I}_u^{\rm t}$, and the ranked list of predicted preferences, 
the relevance of a recommendation at position $k$ is given by ${\rm rel}_{ui}(k)$--${\rm rel}(k)$ from here on--, which equals $1$ if user $u$ adopted item $i$ in the test set, $0$ otherwise. In the calculation of metrics, we remove items observed in  training and validation from the list of recommendations. 
Next, we detail the metrics used for model evaluation.

\paragraph{Recall} It does not account for the relative ordering of the recommendation, and we defined it as~\cite{liang:2018:VAE}
\begin{equation}
{\rm Recall}@k = \frac{\sum_{s=1}^k {\rm rel}(s)}
{\mathcal{N}_u(k)}.
\end{equation}
Here, $\mathcal{N}_u(k) = \min\left(k,|\mathcal{I}_u^{\rm t}|\right)$, with $|\mathcal{I}_u^{\rm t}|$ the number of items adopted by user $u$ in testing. The final recall is averaged across all users in testing. 

\paragraph{Normalized Discount Cumulative Gain} In contrast to recall metric, the Discount Cumulative Gain (DCG) performs a logarithmic discount according to the position of a recommendation, that is
\begin{equation}
{\rm DCG}@k = \sum_{s=1}^k \frac{{\rm rel}(s)}
{\log_2(s+1)}.
\end{equation}
This quantity can be normalized by the Ideal DCG, 
\begin{equation}
{\rm IDCG}@k = \sum_{s=1}^{\mathcal{N}_u(k)} \frac{1}{\log_2(s+1)}.
\end{equation}
Finally, NDCG$@k= {\rm DCG}@k/{\rm IDCG}@k$, which we average across all users in the test set.

\paragraph{Novelty} Following reference~\cite{Vargas:2011:Novelty_diversity}, we define a novelty-weighted DCG score as
\begin{equation}\label{eq:nov-ndcg}
{\rm Nov\!-\!DCG}@k = 
-\sum_{s=1}^k \frac{{\rm rel}(s)\times \ln{\nu(i)}}
{\log_2(s+1)}.
\end{equation}
Here, $\nu(i)$ is the frequency of occurrences of item $i$  normalized to the total interactions in training. The corresponding novelty-weighted IDCG would be
\begin{equation}
{\rm Nov\!-\!IDCG}@k = 
\sum_{s=1}^{\mathcal{N}_u(k)} 
\frac{\max_{i\in\mathcal{I}_u}\left(-\ln \nu(i)\right)}
{\log_2(s+1)}.
\end{equation}
In other words, the highest DCG is obtained by ranking the most novel items (among those relevant to the user) in descending order. 
\subsection{Implementation details}\label{subsec:implementation}
The implementation of our model is performed in TensorFlow~\cite{tensorflow2015-whitepaper}.
%\footnote{The code will be available online at
%\url{https://github.com/bbvadata/RecApp}.
%}. 
The model can be trained in both CPU or GPU. 
When GPU is enabled, the use of queues to feed the tensors greatly speeds up the training.
% batch size and epoch
We set the batch size to $100$, and train every DAE model for $120$k iterations, so as to ensure proper convergence. For MF models we use $180$k iterations.
% number of neurons
The number of neurons is $200$ in all DAE experiments; for MF models, since the large number of users makes them prone to overfit, we train the models with $100$ and $200$ neurons and take the best performing model.
% weights and biases initialization
Weight matrices are initialized with random uniform values whose amplitude is computed as described by Glorot~\emph{et al.}~\cite{Xavier_initialization}. 
%Generally speaking, we discourage the use of random normal initialization without truncation.
For the biases we use a truncated random normal initialization with a standard deviation of $10^{-3}$. 
% Optimizer
Models are trained with Adam optimizer~\cite{Kingma2014AdamAM} and a learning rate of $10^{-3}$.
% gradient clip
%Additionally, we clip gradients whenever they exceed a norm of $1$, and apply batch normalization~\cite{icml2015_ioffe15_batch-norm} at every training step.

% Size of T and P sets
Concerning negative sampling in point and pair--wise schemes, 
%since we train with a batch size different form one, 
we fix the size of the target sets for every user (sets $\mathcal{T}_u$ and $\mathcal{P}_u$ for point and pair--wise learning, respectively, see subsection~\ref{subsec:losses}).
In particular, we make such sets proportional to the median number of items adopted by users, except for the multinomial loss, where all items are utilized~\cite{Liang:2016:CoFactor}.
The proportionality factors are hyper-parameters fine--tuned with the validation set, swapping the values $\{1,\,5,\,10,\,50,\,100,\,150\}$. We find a factor of $50$ or $100$ to provide the best results.
%(see additional comments in subsection~\ref{subsec:baseline_models}).

% Max norm regularization
%Regarding max--norm regularization, we notice that an asymmetric max--norm regularization might be required for the encoder/decoder weight matrices, depending on the activations and objective functions utilized for modeling. We swap $\alpha_{\rm enc}$ and $\alpha_{\rm dec}$ values in $\{0.0, 0.05, 0.1, 0.2, 0.3, 0.5, 1.0\}$, where $\alpha=0.0$ means no regularization.

% Regularization: noise and weight-decay
We add noise to the input vector of the AE~\cite{Vincent:2008:ECRF-AE, Wu:2016:CDAE-topN} using drop-out~\cite{liang:2018:VAE}. We fix the level of noise at $0.5$. Competitive performance is achieved after normalizing the AE input vector.
% Regularization of MIL is smaller
For DAE models, we swap the $L_2$ regularization strength $\lambda\in[10^{-7}-10^{-4}]$, while for MF models we take the form in equation~(\ref{eq:l2_reg_scaled}) with $\lambda\in[10^{0}-10^{3}]$, which provides a more stable training for MF models\footnote{Recall the different scales of the $\lambda$ factor in equations (\ref{eq:L2_reg}) and (\ref{eq:l2_reg_scaled}).}. In general we find that MIL models require smaller $\lambda$ factors than cross-entropy or multinomial--based models. This is expected, as the level of weight--decay regularization in equations~(\ref{eq:L2_reg}) and (\ref{eq:l2_reg_scaled}) depends on the value of the loss, which is smaller for MIL models.

\subsection{Baseline models}\label{subsec:baseline_models}
We implement the objective functions described in subsection \ref{subsec:losses} on a user-based DAE~\cite{Sedhain:2015:Autorec, Wu:2016:CDAE-topN} and compare the results with the MIL function. We also compare them with traditional Matrix Factorization with Weight Regularization~\cite{HuKoren:2008:CF_implicit}. In the following, we provide details on the training of the different models.

\textbf{Weight-Regularized Matrix Factorization} WRMF~\cite{HuKoren:2008:CF_implicit} is a linear factorization model  trained with square loss and weight decay. We use negative sampling with a sampling ratio of $100$ and $\lambda\sim 5-10$ (as obtained in the validation set). We call this model \MFsquare. In addition, we train WRMF models with MIL and point--wise cross--entropy losses, applying a sigmoid function at the output, so as to ensure $\hat{p}_{ui}\in(0,1)$. In these cases, we find that a sampling ratio of $100$ and $\lambda=50-500$ provide best results. We name these models \MFmil\, and \MFce, respectively.

\textbf{\emph{Denoising Autoencoder models}}

\textbf{Cross-entropy loss} For the cross-entropy loss defined in equations~(\ref{eq:cross-entropy}), (\ref{eq:point-wise}) and (\ref{eq:pair-wise}), we use linear--sigmoid and sigmoid--sigmoid activations at the encoder and decoder, respectively. We name the DAEs models with cross-entropy loss and point--wise estimation \CEpointlinsig\,and \CEpointsigsig; and those with pair--wise,  \CEpairlinsig \,and \CEpairsigsig. In order to prevent numerical instabilities, we ensure that the output preferences are in $[\varepsilon, 1-\varepsilon]$, with $\varepsilon=10^{-5}$. Regarding negative sampling, we find that the best sampling ratio is $50\times {\rm median}(\mathcal{I}_u)$ and $100\times {\rm median}(\mathcal{I}_u)$  for point and pair--wise estimation, respectively. Best weight-decay regularization is found to be $\lambda=2\cdot 10^{-5}$.

% Comparison to CDAE
The closest model to these baselines is the Collaborative Denoising AE (CDAE)~\cite{Wu:2016:CDAE-topN}, although for the sake of simplicity, in the present paper we do not include the user embedding of CDAE. 
% bad performance of BPR
Similar to CDAE, we find that  pair--wise learning does not achieve competitive results at the top of the ranked list~\cite{Wu:2016:CDAE-topN, liang:2018:VAE}. 

\textbf{Multinomial loss} AEs trained with a multinomial log-likelihood have recently been  introduced by Lian et al~\cite{liang:2018:VAE}, either applied to DAEs or Variational AEs (VAE) with partial regularization. Here, we focus on the multi-DAE modeling with $\tanh$-linear activations\footnote{
We use the actual implementation provided at \url{https://github.com/dawenl/vae_cf} to verify that the activation used at the decoder of multi-DAE is linear, although the original writing~\cite{liang:2018:VAE} suggests a $\tanh$ non-linearity for the decoder.
}, and name this baseline \MULTItanhlin. Our implementation exactly reproduces that of~\cite{liang:2018:VAE} when using $\lambda=2\cdot 10^{-5}$, input noise of $0.5$ and without applying negative sampling. 

\textbf{Missing Information loss} We apply the \textsc{MIL} function defined in equation~(\ref{eq:mil_def}) to linear-sigmoid and sigmoid-sigmoid DAEs. We name these models \MILlinsig\, and \MILsigsig, respectively. Best hyper-parameters of the loss turn out to be $A_{\rm MI}=10^6,\, \gamma_{\rm MI}=10$ and $\gamma_{+}=1$, after grid search the pairs $\left(A_{\rm MI}, \gamma_{\rm MI}\right)\in\{
(5\cdot10^1, 2)$, $(10^3, 4)$, $(2\cdot10^4, 6)$, $(5\cdot10^5, 10)$, $(1\cdot10^6, 10)$, $(5\cdot10^6, 10)$, $(5\cdot10^9, 15)\}$, and $\gamma_{+}=1$ or $2$. In addition, we use a sampling ratio of $50$ and $\lambda\in(10^{-6}, 10^{-5})$.

\begin{table}[t!]
\centering
\caption{Voice conversion \& F0 manipulation results. MOS results are reported with 95\% confidence interval. VDE, and FFE are reported for F0 manipulation while PER, WER, EER, and MOS are reported for voice conversion. Notice, for VDE, and FFE higher is the better since F0 was flattened.}
\label{tab:conv}

\resizebox{1\columnwidth}{!}{
\begin{tabular}{c@{~} | c@{~} | c@{~}c@{~} | c@{~} | c@{~} ||  c@{~}c@{~} }
\toprule
\multirow{2}{*}{Dataset} & \multirow{2}{*}{Method} & \multicolumn{4}{c||}{Voice Conversion} & \multicolumn{2}{c}{F0 Manipulation} \\
\cmidrule{3-8}
& & PER~$\downarrow$ & WER~$\downarrow$ & EER~$\downarrow$ & MOS~$\uparrow$ & VDE~$\uparrow$ & FFE~$\uparrow$ \\
\midrule
VCTK & GT  & 17.16 & 4.32 & 3.25 & 4.11$\pm$0.29 & -- & -- \\
\midrule 
\multirow{3}{*}{LJ}
% & ASR-TTS   & 50.74  & --     & 66.08 & 32.96 & 1.46 \\
& CPC       & 22.22 	& 16.11 		& 0.46 		& 3.57$\pm$0.15 		& \bf 46.68 & \bf 48.71\\
& HuBERT    & \bf 19.09 & \bf 12.23 & \bf 0.31  & \bf 3.71$\pm$0.24 & 39.20 		& 48.42\\
& VQ-VAE    & 40.88 	& 36.96 		& 9.65 		& 2.90$\pm$0.17 		& 10.54 	& 12.08 \\
\midrule 
\multirow{3}{*}{VCTK} 
% & ASR-TTS   & 68.88  & --    & 41.77 & 13.55 & 6.48 \\
& CPC       &  23.58 		& 15.98 		& \bf 4.83  &  3.42 $\pm$ 0.24 		& \bf 25.29 & \bf 26.97 \\
& HuBERT    &  \bf 20.85 	& \bf 12.72 & 6.01  		& \bf  3.58 $\pm$ 0.28 	& 23.46 	& 26.67 \\
& VQ-VAE    & 36.88  		& 29.44 		& 11.56 		& 3.08 $\pm$ 0.34 		& 7.03  	& 7.80  \\
\bottomrule
\end{tabular}}
\vspace{-0.4cm}
\end{table}

\vspace{-0.1cm}
\section{Results}
\vspace{-0.1cm}
Our results cover
% We report results for 
three different settings: (i) speech reconstruction experiments; (ii) speaker conversion and F0 manipulation; (iii) bitrate analysis with subjective tests for speech codec evaluation. We employ two datasets: LJ~\cite{ljspeech17} single speaker dataset and VCTK~\cite{vctk} multi-speaker dataset. All datasets were resampled to a 16kHz sample rate.

% \paragraph*{Implementation Details.}
% \smallskip
\noindent{\bf Implementation Details\quad} 
\label{sec:impl}
We follow the same setup as in~\cite{lakhotia2021generative}. For CPC, we used the model from~\cite{Riviere2020}, which was trained on a ``clean'' 6k hour sub-sample of the LibriLight dataset~\cite{Kahn2020,Riviere2020}. We extract a downsampled representation from an intermediate layer with a 256-dimensional embedding and a hop size of 160 audio samples. For HuBERT we used a \textsc{Base} 12 transformer-layer model trained for two iterations~\cite{hsu2020hubert} on 960 hours of LibriSpeech corpus~\cite{Panayotov2015}. 
% This model encodes every 320 raw audio samples into a 768-dimensional vector. 
This model downsamples the raw audio $\times320$ into a sequence of 768-dimensional vectors. Similarly to~\cite{lakhotia2021generative}, activations were extracted from the sixth layer.

%CPC: We use a dictionary of 100 units, leading to a bitrate of 700bps.
%HuBERT: A dictionary of 100 units is used, leading to a bitrate of 350bps. 
%VQVE: The VQ-VAE discrete code operates at a bitrate of 800bps.
% For both CPC and HuBERT, the k-means algorithm is applied to convert continuous frames to discrete codes, using the LibriSpeech clean-100h~\cite{Panayotov2015} dataset. 
For CPC and HuBERT, the k-means algorithm is trained on LibriSpeech clean-100h~\cite{Panayotov2015} dataset to convert continuous frames to discrete codes. We quantize both learned representations with $K=100$ centroids. Leading to a bitrate of 700bps for CPC and 350bps for HuBERT.

% VQ-VAE
Similarly to CPC models, we trained the VQ-VAE content encoder model on the ``clean'' 6K hours subset from the LibriLight dataset. We use an encoder operating on the raw signal to extract discrete units, similar to~\cite{jukebox}. In addition, ``random restarts'' were performed when the mean usage of a codebook vector fell below a predetermined threshold. Finally, we used HiFiGAN (architecture and objective) as the decoder instead of a simple convolutional decoder, as it improved the overall audio quality. This model encodes the raw audio into a sequence of discrete tokens from 256 possible tokens~\cite{garbacea2019low} with a hop size of 160 raw audio samples. The VQ-VAE discrete code operates at a bitrate of 800bps. We additionally experimented with 100 discrete units for VQ-VAE, however results were the best for 256. This finding is consistent with~\cite{garbacea2019low}.

% verification model
The speaker verification network uses the architecture proposed in~\cite{heigold2016end}. It was trained on the VoxCeleb2~\cite{voxceleb2} dataset, achieving a 7.4\% Equal Error Rate (EER) for speaker verification on the test split of the VoxCeleb1~\cite{Nagrani17} dataset.

% pitch
Only a single F0 representation is considered across all evaluated models, trained on the VCTK dataset.
% The F0 is extracted from the raw audio using YAAPT~\cite{yaapt} algorithm, using a window size of 20ms and a 5ms hop. 
The F0 is extracted from the raw audio using a window size of 20ms and a 5ms hop. 
As a result, the F0 sequence is sampled at 200Hz. 
% We apply the quantization described at Sec.~\ref{sec:method}, using a pitch codebook of $K'=20$ tokens and an encoder that downsamples the pitch by $\times16$. 
The quantization described at Sec.~\ref{sec:method}, is applied using an F0 codebook of $K'=20$ tokens and an encoder that downsamples the signal by $\times16$. Hence, the discrete F0 representation is sampled at 12.5Hz, leading to a bitrate of 65bps. The final bitrate of the evaluated codecs is the sum of the pitch code bitrate with the content code bitrate.

% \paragraph*{Evaluation Metrics}
% \smallskip
\noindent{\bf Evaluation Metrics\quad} 
We consider both subjective and objective evaluation metrics. For subjective tests, we report the Mean Opinion Scores (MOS). In which human evaluators rate the naturalness of audio samples on a scale of 1--5. Each experiment, included 50 randomly selected samples rated by 30 raters. For objective evaluation, we consider: (i) Equal Error Rate~(EER) as an automatic speaker verification metric obtained using a pre-trained speaker verification network. We report EER between test utterances and enrolled speakers; (ii) Voicing Decision Error (VDE)~\cite{nakatani2008method}, which measures the portion of frames with voicing decision error; (iii) F0 Frame Error (FFE)~\cite{chu2009reducing}, measures the percentage of frames that contain a deviation of more than 20\% in pitch value or have a voicing decision error; (iv) Word Error Rate (WER) and Phoneme Error Rate (PER), proxy metrics to the intelligibility of the generated audio. We used a pre-trained ASR network~\cite{baevski2020wav2vec} on both reconstructed and converted samples to calculate both metrics. %To generate target phonemes, the g2p-en~\cite{g2pE2019} Grapheme2Phoneme module was used.

% \vspace{-0.1cm}
% \smallskip
\noindent{\bf Reconstruction \& Conversion}
% \vspace{-0.1cm}
We start by reporting the reconstruction performance. Results are summarized in Table~\ref{tab:recon}. When considering the intelligibility of the reconstructed signal HuBERT reaches the lowest PER and WER scores across all models, where both CPC and HuBERT are superior to VQ-VAE. However, when considering F0 reconstruction VQ-VAE outperforms both HuBERT and CPC by a significant margin. This results are somewhat intuitive, bearing in mind VQ-VAE objective is to fully reconstruct the input signal. In terms of subjective evaluation, all models reach similar MOS scores, with one exception of CPC on LJ. 

%Notice, since the same F0 units are used for each method, this result implies the VQ-VAE units contain some information about the F0 of the signal, enabling better reconstruction. Regarding speaker information, the CPC gets the lowest EER. 

To better evaluate the disentanglement properties of each method with respect to speaker identity and F0, we conducted an additional set of experiments aiming at speaker conversion and F0 manipulation. For voice conversion, we converted each test utterance into five random target speakers. Next, we employed a speaker verification network, which extracts \emph{d-vector} representation to evaluate speaker-converted utterances' similarity to real speaker utterances (low error-rate indicates good conversion), providing measurement to the speaker identity's disentanglement from the evaluated coding method. The error-rate is reported between converted test utterances and enrolled speakers. For the LJ speech single speaker dataset, we converted samples from the VCTK dataset to the single speaker and enrolled all VCTK speakers together with the single speaker. Results are summarized in Table~\ref{tab:conv} (left). Unlike resynthesis results, on voice conversion CPC and HuBERT outperform VQ-VAE on both LJ and VCTK datasets, indicating VQ-VAE contains more information about the speaker in the encoded units, hence producing more artifacts. Notice, this also affects WER, PER, and the overall subjective quality (MOS). 

Next, to evaluate the presence of F0 in the discrete units, we flattened the F0 units before synthesizing the signal and calculated VDE and FFE with respect to the original F0 values. F0 flattening was done by setting the speakers' mean F0 value across all voiced frames. In this experiment, we expected units that contain F0 information to be better at F0 reconstruction over disentangled units. Results are summarized in Table~\ref{tab:conv} (right). Notice VQ-VAE can still reconstruct the F0 almost at the same level as when using the original F0 as conditioning (5.2 vs 7.03, and 5.59 vs 7.8), in contrast to CPC and HuBERT.

\begin{figure}[t!]
\centering
\includegraphics[width=0.65\columnwidth, trim={50 20 70 20}]{figures/codec_2.pdf}
% \caption{MUSHRA subjective listening test results as a function of bitrate per second for various methods. Purple dots denote the baseline methods, and green dots the proposed SSL based method.} 
\caption{MUSHRA subjective quality results as a function of bitrate per second. Purple dots denote the baseline methods, and green dots the proposed SSL based method.} 
\label{fig:codec}
\vspace{-0.5cm}
\end{figure}

% \vspace{-0.1cm}
% \smallskip
\noindent{\bf Speech Codec}
Our final experiment evaluates the obtained speech units as a low bitrate speech codec. 
% Therefore, we evaluate how the performance varies as a function of the number of discrete units. Changing the number of units is equivalent to varying the bitrate of the encoded signal. 
We use a subjective MUSHRA-type listening test~\cite{series2014method} to measure the perceived quality of the proposed speech codec with regard to its bitrate constraints. In MUSHRA evaluations, listeners are presented with a labeled uncompressed signal for reference, a set of test samples to rate, a copy of the uncompressed reference, and a low-quality anchor. Listeners are asked to rate each test utterance and the copy of the uncompressed reference with respect to the labeled reference in a scale of 1-100.

The experiment is performed on the VCTK dataset~\cite{vctk}. For evaluation, we used 20 utterances from 5 speakers. The set of speakers in the test data is disjoint with those in the training data. For this experiment, HuBERT models with 50, 100, and 200 units were trained as described in Sec.~\ref{sec:impl}. For comparison, we included other speech codecs in our evaluation: Opus~\cite{valin2012definition} wideband at 9 kbps VBR, Codec2~\cite{rowe2011codec} at 2.4 kbps and LPCNet~\cite{valin2019real} operating at 1.6 kbps. The LPCNet model was trained from scratch on the VCTK dataset following the experimental setup in~\cite{valin2019real}. The VQ-VAE model employs the HiFiGAN decoder trained on the LibriLight dataset to match the amount of data reported in~\cite{garbacea2019low}. We compressed the anchor sample with Speex~\cite{valin2016speex} at 4 kbps as a low anchor. Fig.~\ref{fig:codec} depicts the results. HuBERT with 50 units reaches the best MUSHRA score while its bitrate is only 365bps, which is significantly lower than the baseline methods.

\section{Conclusions and next steps}\label{sec:conclusions}

% definition of MIL
In this paper we present a novel objective function, the \emph{Missing Information Loss} (MIL), specifically designed for handling unobserved user-item interactions in implicit feedback datasets. In particular, MIL explicitly forbids treating missing user-item interactions as positive or negative feedback.
% What it does
We demonstrate that, thanks to the functional form of the MIL function, the ranking of unseen items is almost entirely left to the low--rank process, rather than forcing unobserved items to be at the tail of the recommendation (\emph{i.e.}, MIL does not force a zero predicted preference for unobserved user-item interactions). 

% Metric results
Extensive experiments with Matrix Factorization and Denoising Autoencoders conducted on three datasets, show that \textsc{MIL} models demonstrate competitive performance when compared with other traditional losses such as cross-entropy or the multinomial log-likelihood. 
% Best performing models 
% Analysis of recommendations
In addition, we study the distribution of the recommendations and observe that the reported metric performance takes place while recommending popular items less frequently (up to a $20 \%$ decrease with respect to the best competing method). Indeed, \textsc{MIL} models sharply increase the recommendation of medium--tail items, while almost linearly expanding the appearance of long--tail items with the ranking position in the list of recommendations. Such expansion results in up to a $50 \%$ increase of long--tail recommendations, a feature of utmost importance for industries with a large catalogue of items. 

% Future work
Future lines of research may involve the incorporation of negative feedback, or the usage of \textsc{MIL} in temporal--aware Recommender Systems (such as those using Recurrent Neural Networks).  
In addition, we hope that the results here reported  will bring forward first-principle mathematical derivations of the \textsc{MIL} function, so that the vast family of possible polynomials modelling the missing information term can be reduced, or even extended with more suitable functions. 

\begin{acks}
We would like to thank the continuous support and careful reading of the manuscript by the \emph{Edge} guild within BBVA Data \& Analytics, specially J. Garc\'ia Santamar\'ia and J. A. Rodr\'iguez Serrano. 
\end{acks}

%\bibliographystyle{ACM-Reference-Format}
\bibliographystyle{unsrt}
%\bibliography{sigproc} 

\documentclass[sigconf]{acmart}

\settopmatter{printacmref=false} % Removes citation information below abstract
\renewcommand\footnotetextcopyrightpermission[1]{} % removes footnote with conference information in first column
\pagestyle{plain} % removes running headers

\usepackage{booktabs} % For formal tables
\usepackage{multirow}
\usepackage{amsmath}
\usepackage{color}
\usepackage{arydshln }


%Conference
\acmConference[WSDM]{The Twelfth International Conference on Web Search and Data Mining}{February 11--15}{Melbourne, Australia} 
\acmYear{2019}

\begin{document}
\title[Missing Information Loss]{A Missing Information Loss for implicit feedback datasets}

\author{Juan Ar\'evalo}
\affiliation{%
  \institution{BBVA Data \& Analytics}
}
\email{juanmaria.arevalo@bbvadata.com}

\author{Juan Ram\'on Duque}
\affiliation{%
  \institution{BBVA Data \& Analytics}
}
\email{juanramon.duque@bbvadata.com}

\author{Marco Creatura}
\affiliation{%
  \institution{BBVA Data \& Analytics}
}
\email{marco.creatura@bbvadata.com}

% The default list of authors is too long for headers}
\renewcommand{\shortauthors}{Ar\'evalo, Duque and Creatura}

% Some useful commands
\newcommand{\MFsquare}{\textsc{MF-square}}
\newcommand{\MFmil}{\textsc{MF-mil}}
\newcommand{\MFce}{\textsc{MF-CE}}
\newcommand{\CEpointlinsig}{\textsc{CE$_{\rm Point}$ lin-sig}}
\newcommand{\CEpointsigsig}{\textsc{CE$_{\rm Point}$ sig-sig}}
\newcommand{\CEpairlinsig}{\textsc{CE$_{\rm Pair}$ lin-sig}}
\newcommand{\CEpairsigsig}{\textsc{CE$_{\rm Pair}$ sig-sig}}
\newcommand{\MULTItanhlin}{\textsc{MULTI tanh-lin}}
\newcommand{\MILlinsig}{\textsc{MIL lin-sig}}
\newcommand{\MILsigsig}{\textsc{MIL sig-sig}}


\begin{abstract}
% missing values and negative feedback
Latent factor models for Recommender Systems with implicit feedback typically treat unobserved user-item interactions (\emph{i.e.} missing information) as negative feedback. This is frequently done 
either through negative sampling (point--wise loss) or with a ranking loss function (pair-- or list--wise estimation). 
% Common objective functions allow zero prediction
Since a zero preference recommendation is a valid solution for most common objective functions, 
regarding unknown values as actual zeros results in users 
having a zero preference recommendation  for most of the available items. 

% MIL
In this paper we propose a novel objective function, the \emph{Missing Information Loss} (MIL), 
that explicitly forbids treating unobserved user-item interactions as positive or negative feedback. 
% application to AE and metrics
We apply this loss to both traditional Matrix Factorization and user--based Denoising Autoencoder, and compare it with other established objective functions such as cross--entropy (both point-- and pair--wise) or the recently proposed multinomial log-likelihood. MIL achieves competitive performance in ranking--aware metrics when applied to three datasets.
% towards long-tail recommendations
Furthermore, we show that such a relevance in the recommendation is obtained while displaying popular items less frequently (up to a $20 \%$ decrease with respect to the best competing method). This debiasing from the recommendation of popular items favours the appearance of infrequent items (up to a $50 \%$ increase of long--tail recommendations), a valuable feature for Recommender Systems with a large catalogue of products. 
\end{abstract}

\begin{CCSXML}
<ccs2012>
<concept>
<concept_id>10002951.10003317.10003347.10003350</concept_id>
<concept_desc>Information systems~Recommender systems</concept_desc>
<concept_significance>500</concept_significance>
</concept>
</ccs2012>
\end{CCSXML}

\ccsdesc[500]{Information systems~Recommender systems}

\keywords{Collaborative Filtering, Autoencoders, Implicit Feedback, Missing Information}

\setcopyright{None}

\maketitle

\setlength{\abovecaptionskip}{0pt}
\setlength{\belowcaptionskip}{-10pt}

% Put all the sections with inputs

\input{intro}

\input{model}

\input{protocols}

\input{results}

\section{Conclusions and next steps}\label{sec:conclusions}

% definition of MIL
In this paper we present a novel objective function, the \emph{Missing Information Loss} (MIL), specifically designed for handling unobserved user-item interactions in implicit feedback datasets. In particular, MIL explicitly forbids treating missing user-item interactions as positive or negative feedback.
% What it does
We demonstrate that, thanks to the functional form of the MIL function, the ranking of unseen items is almost entirely left to the low--rank process, rather than forcing unobserved items to be at the tail of the recommendation (\emph{i.e.}, MIL does not force a zero predicted preference for unobserved user-item interactions). 

% Metric results
Extensive experiments with Matrix Factorization and Denoising Autoencoders conducted on three datasets, show that \textsc{MIL} models demonstrate competitive performance when compared with other traditional losses such as cross-entropy or the multinomial log-likelihood. 
% Best performing models 
% Analysis of recommendations
In addition, we study the distribution of the recommendations and observe that the reported metric performance takes place while recommending popular items less frequently (up to a $20 \%$ decrease with respect to the best competing method). Indeed, \textsc{MIL} models sharply increase the recommendation of medium--tail items, while almost linearly expanding the appearance of long--tail items with the ranking position in the list of recommendations. Such expansion results in up to a $50 \%$ increase of long--tail recommendations, a feature of utmost importance for industries with a large catalogue of items. 

% Future work
Future lines of research may involve the incorporation of negative feedback, or the usage of \textsc{MIL} in temporal--aware Recommender Systems (such as those using Recurrent Neural Networks).  
In addition, we hope that the results here reported  will bring forward first-principle mathematical derivations of the \textsc{MIL} function, so that the vast family of possible polynomials modelling the missing information term can be reduced, or even extended with more suitable functions. 

\begin{acks}
We would like to thank the continuous support and careful reading of the manuscript by the \emph{Edge} guild within BBVA Data \& Analytics, specially J. Garc\'ia Santamar\'ia and J. A. Rodr\'iguez Serrano. 
\end{acks}

%\bibliographystyle{ACM-Reference-Format}
\bibliographystyle{unsrt}
%\bibliography{sigproc} 

\input{MIL.bbl}
\end{document}

\end{document}

\end{document}

\end{document}
