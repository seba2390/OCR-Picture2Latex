\documentclass[sigconf]{acmart}

\settopmatter{printacmref=false} % Removes citation information below abstract
\renewcommand\footnotetextcopyrightpermission[1]{} % removes footnote with conference information in first column
\pagestyle{plain} % removes running headers

\usepackage{booktabs} % For formal tables
\usepackage{multirow}
\usepackage{amsmath}
\usepackage{color}
\usepackage{arydshln }


%Conference
\acmConference[WSDM]{The Twelfth International Conference on Web Search and Data Mining}{February 11--15}{Melbourne, Australia} 
\acmYear{2019}

\begin{document}
\title[Missing Information Loss]{A Missing Information Loss for implicit feedback datasets}

\author{Juan Ar\'evalo}
\affiliation{%
  \institution{BBVA Data \& Analytics}
}
\email{juanmaria.arevalo@bbvadata.com}

\author{Juan Ram\'on Duque}
\affiliation{%
  \institution{BBVA Data \& Analytics}
}
\email{juanramon.duque@bbvadata.com}

\author{Marco Creatura}
\affiliation{%
  \institution{BBVA Data \& Analytics}
}
\email{marco.creatura@bbvadata.com}

% The default list of authors is too long for headers}
\renewcommand{\shortauthors}{Ar\'evalo, Duque and Creatura}

% Some useful commands
\newcommand{\MFsquare}{\textsc{MF-square}}
\newcommand{\MFmil}{\textsc{MF-mil}}
\newcommand{\MFce}{\textsc{MF-CE}}
\newcommand{\CEpointlinsig}{\textsc{CE$_{\rm Point}$ lin-sig}}
\newcommand{\CEpointsigsig}{\textsc{CE$_{\rm Point}$ sig-sig}}
\newcommand{\CEpairlinsig}{\textsc{CE$_{\rm Pair}$ lin-sig}}
\newcommand{\CEpairsigsig}{\textsc{CE$_{\rm Pair}$ sig-sig}}
\newcommand{\MULTItanhlin}{\textsc{MULTI tanh-lin}}
\newcommand{\MILlinsig}{\textsc{MIL lin-sig}}
\newcommand{\MILsigsig}{\textsc{MIL sig-sig}}


\begin{abstract}
% missing values and negative feedback
Latent factor models for Recommender Systems with implicit feedback typically treat unobserved user-item interactions (\emph{i.e.} missing information) as negative feedback. This is frequently done 
either through negative sampling (point--wise loss) or with a ranking loss function (pair-- or list--wise estimation). 
% Common objective functions allow zero prediction
Since a zero preference recommendation is a valid solution for most common objective functions, 
regarding unknown values as actual zeros results in users 
having a zero preference recommendation  for most of the available items. 

% MIL
In this paper we propose a novel objective function, the \emph{Missing Information Loss} (MIL), 
that explicitly forbids treating unobserved user-item interactions as positive or negative feedback. 
% application to AE and metrics
We apply this loss to both traditional Matrix Factorization and user--based Denoising Autoencoder, and compare it with other established objective functions such as cross--entropy (both point-- and pair--wise) or the recently proposed multinomial log-likelihood. MIL achieves competitive performance in ranking--aware metrics when applied to three datasets.
% towards long-tail recommendations
Furthermore, we show that such a relevance in the recommendation is obtained while displaying popular items less frequently (up to a $20 \%$ decrease with respect to the best competing method). This debiasing from the recommendation of popular items favours the appearance of infrequent items (up to a $50 \%$ increase of long--tail recommendations), a valuable feature for Recommender Systems with a large catalogue of products. 
\end{abstract}

\begin{CCSXML}
<ccs2012>
<concept>
<concept_id>10002951.10003317.10003347.10003350</concept_id>
<concept_desc>Information systems~Recommender systems</concept_desc>
<concept_significance>500</concept_significance>
</concept>
</ccs2012>
\end{CCSXML}

\ccsdesc[500]{Information systems~Recommender systems}

\keywords{Collaborative Filtering, Autoencoders, Implicit Feedback, Missing Information}

\setcopyright{None}

\maketitle

\setlength{\abovecaptionskip}{0pt}
\setlength{\belowcaptionskip}{-10pt}

% Put all the sections with inputs

% !TEX root = ../arxiv.tex

Unsupervised domain adaptation (UDA) is a variant of semi-supervised learning \cite{blum1998combining}, where the available unlabelled data comes from a different distribution than the annotated dataset \cite{Ben-DavidBCP06}.
A case in point is to exploit synthetic data, where annotation is more accessible compared to the costly labelling of real-world images \cite{RichterVRK16,RosSMVL16}.
Along with some success in addressing UDA for semantic segmentation \cite{TsaiHSS0C18,VuJBCP19,0001S20,ZouYKW18}, the developed methods are growing increasingly sophisticated and often combine style transfer networks, adversarial training or network ensembles \cite{KimB20a,LiYV19,TsaiSSC19,Yang_2020_ECCV}.
This increase in model complexity impedes reproducibility, potentially slowing further progress.

In this work, we propose a UDA framework reaching state-of-the-art segmentation accuracy (measured by the Intersection-over-Union, IoU) without incurring substantial training efforts.
Toward this goal, we adopt a simple semi-supervised approach, \emph{self-training} \cite{ChenWB11,lee2013pseudo,ZouYKW18}, used in recent works only in conjunction with adversarial training or network ensembles \cite{ChoiKK19,KimB20a,Mei_2020_ECCV,Wang_2020_ECCV,0001S20,Zheng_2020_IJCV,ZhengY20}.
By contrast, we use self-training \emph{standalone}.
Compared to previous self-training methods \cite{ChenLCCCZAS20,Li_2020_ECCV,subhani2020learning,ZouYKW18,ZouYLKW19}, our approach also sidesteps the inconvenience of multiple training rounds, as they often require expert intervention between consecutive rounds.
We train our model using co-evolving pseudo labels end-to-end without such need.

\begin{figure}[t]%
    \centering
    \def\svgwidth{\linewidth}
    \input{figures/preview/bars.pdf_tex}
    \caption{\textbf{Results preview.} Unlike much recent work that combines multiple training paradigms, such as adversarial training and style transfer, our approach retains the modest single-round training complexity of self-training, yet improves the state of the art for adapting semantic segmentation by a significant margin.}
    \label{fig:preview}
\end{figure}

Our method leverages the ubiquitous \emph{data augmentation} techniques from fully supervised learning \cite{deeplabv3plus2018,ZhaoSQWJ17}: photometric jitter, flipping and multi-scale cropping.
We enforce \emph{consistency} of the semantic maps produced by the model across these image perturbations.
The following assumption formalises the key premise:

\myparagraph{Assumption 1.}
Let $f: \mathcal{I} \rightarrow \mathcal{M}$ represent a pixelwise mapping from images $\mathcal{I}$ to semantic output $\mathcal{M}$.
Denote $\rho_{\bm{\epsilon}}: \mathcal{I} \rightarrow \mathcal{I}$ a photometric image transform and, similarly, $\tau_{\bm{\epsilon}'}: \mathcal{I} \rightarrow \mathcal{I}$ a spatial similarity transformation, where $\bm{\epsilon},\bm{\epsilon}'\sim p(\cdot)$ are control variables following some pre-defined density (\eg, $p \equiv \mathcal{N}(0, 1)$).
Then, for any image $I \in \mathcal{I}$, $f$ is \emph{invariant} under $\rho_{\bm{\epsilon}}$ and \emph{equivariant} under $\tau_{\bm{\epsilon}'}$, \ie~$f(\rho_{\bm{\epsilon}}(I)) = f(I)$ and $f(\tau_{\bm{\epsilon}'}(I)) = \tau_{\bm{\epsilon}'}(f(I))$.

\smallskip
\noindent Next, we introduce a training framework using a \emph{momentum network} -- a slowly advancing copy of the original model.
The momentum network provides stable, yet recent targets for model updates, as opposed to the fixed supervision in model distillation \cite{Chen0G18,Zheng_2020_IJCV,ZhengY20}.
We also re-visit the problem of long-tail recognition in the context of generating pseudo labels for self-supervision.
In particular, we maintain an \emph{exponentially moving class prior} used to discount the confidence thresholds for those classes with few samples and increase their relative contribution to the training loss.
Our framework is simple to train, adds moderate computational overhead compared to a fully supervised setup, yet sets a new state of the art on established benchmarks (\cf \cref{fig:preview}).


Online convex optimization with memory has emerged as an important and challenging area with a wide array of applications, see \citep{lin2012online,anava2015online,chen2018smoothed,goel2019beyond,agarwal2019online,bubeck2019competitively} and the references therein.  Many results in this area have focused on the case of online optimization with switching costs (movement costs), a form of one-step memory, e.g., \citep{chen2018smoothed,goel2019beyond,bubeck2019competitively}, though some papers have focused on more general forms of memory, e.g., \citep{anava2015online,agarwal2019online}. In this paper we, for the first time, study the impact of feedback delay and nonlinear switching cost in online optimization with switching costs. 

An instance consists of a convex action set $\mathcal{K}\subset\mathbb{R}^d$, an initial point $y_0\in\mathcal{K}$, a sequence of non-negative convex cost functions $f_1,\cdots,f_T:\mathbb{R}^d\to\mathbb{R}_{\ge0}$, and a switching cost $c:\mathbb{R}^{d\times(p+1)}\to\mathbb{R}_{\ge0}$. To incorporate feedback delay, we consider a situation where the online learner only knows the geometry of the hitting cost function at each round, i.e., $f_t$, but that the minimizer of $f_t$ is revealed only after a delay of $k$ steps, i.e., at time $t+k$.  This captures practical scenarios where the form of the loss function or tracking function is known by the online learner, but the target moves over time and measurement lag means that the position of the target is not known until some time after an action must be taken. 
To incorporate nonlinear (and potentially nonconvex) switching costs, we consider the addition of a known nonlinear function $\delta$ from $\mathbb{R}^{d\times p}$ to $\mathbb{R}^d$ to the structured memory model introduced previously.  Specifically, we have
\begin{align}
c(y_{t:t-p}) = \frac{1}{2}\|y_t-\delta(y_{t-1:t-p})\|^2,    \label{e.newswitching}
\end{align}
where we use $y_{i:j}$ to denote either $\{y_i, y_{i+1}, \cdots, y_j\}$ if $i\leq j$, or  $\{y_i, y_{i-1}, \cdots, y_j\}$ if $i > j$ throughout the paper. Additionally, we use $\|\cdot\|$ to denote the 2-norm of a vector or the spectral norm of a matrix.

In summary, we consider an online agent that interacts with the environment as follows:
% \begin{inparaenum}[(i)] 
\begin{enumerate}%[leftmargin=*]
    \item The adversary reveals a function $h_t$, which is the geometry of the $t^\mathrm{th}$ hitting cost, and a point $v_{t-k}$, which is the minimizer of the $(t-k)^\mathrm{th}$ hitting cost. Assume that $h_t$ is $m$-strongly convex and $l$-strongly smooth, and that $\arg\min_y h_t(y)=0$.
    \item The online learner picks $y_t$ as its decision point at time step $t$ after observing $h_t,$  $v_{t-k}$.
    \item The adversary picks the minimizer of the hitting cost at time step $t$: $v_t$. 
    \item The learner pays hitting cost $f_t(y_t)=h_t(y_t-v_t)$ and switching cost $c(y_{t:t-p})$ of the form \eqref{e.newswitching}.
\end{enumerate}

The goal of the online learner is to minimize the total cost incurred over $T$ time steps, $cost(ALG)=\sum_{t=1}^Tf_t(y_t)+c(y_{t:t-p})$, with the goal of (nearly) matching the performance of the offline optimal algorithm with the optimal cost $cost(OPT)$. The performance metric used to evaluate an algorithm is typically the \textit{competitive ratio} because the goal is to learn in an environment that is changing dynamically and is potentially adversarial. Formally, the competitive ratio (CR) of the online algorithm is defined as the worst-case ratio between the total cost incurred by the online learner and the offline optimal cost: $CR(ALG)=\sup_{f_{1:T}}\frac{cost(ALG)}{cost(OPT)}$.

It is important to emphasize that the online learner decides $y_t$ based on the knowledge of the previous decisions $y_1\cdots y_{t-1}$, the geometry of cost functions $h_1\cdots h_t$, and the delayed feedback on the minimizer $v_1\cdots v_{t-k}$. Thus, the learner has perfect knowledge of cost functions $f_1\cdots f_{t-k}$, but incomplete knowledge of $f_{t-k+1}\cdots f_t$ (recall that $f_t(y)=h_t(y-v_t)$).

Both feedback delay and nonlinear switching cost add considerable difficulty for the online learner compared to versions of online optimization studied previously. Delay hides crucial information from the online learner and so makes adaptation to changes in the environment more challenging. As the learner makes decisions it is unaware of the true cost it is experiencing, and thus it is difficult to track the optimal solution. This is magnified by the fact that nonlinear switching costs increase the dependency of the variables on each other. It further stresses the influence of the delay, because an inaccurate estimation on the unknown data, potentially magnifying the mistakes of the learner. 

The impact of feedback delay has been studied previously in online learning settings without switching costs, with a focus on regret, e.g., \citep{joulani2013online,shamir2017online}.  However, in settings with switching costs the impact of delay is magnified since delay may lead to not only more hitting cost in individual rounds, but significantly larger switching costs since the arrival of delayed information may trigger a very large chance in action.  To the best of our knowledge, we give the first competitive ratio for delayed feedback in online optimization with switching costs. 

We illustrate a concrete example application of our setting in the following.

\begin{example}[Drone tracking problem]
\label{example:drone} \emph{
Consider a drone with vertical speed $y_t\in\mathbb{R}$. The goal of the drone is to track a sequence of desired speeds $y^d_1,\cdots,y^d_T$ with the following tracking cost:}
\begin{equation}
    \sum_{t=1}^T \frac{1}{2}(y_t-y^d_t)^2 + \frac{1}{2}(y_t-y_{t-1}+g(y_{t-1}))^2,
\end{equation}
\emph{where $g(y_{t-1})$ accounts for the gravity and the aerodynamic drag. One example is $g(y)=C_1+C_2\cdot|y|\cdot y$ where $C_1,C_2>0$ are two constants~\cite{shi2019neural}. Note that the desired speed $y_t^d$ is typically sent from a remote computer/server. Due to the communication delay, at time step $t$ the drone only knows $y_1^d,\cdots,y_{t-k}^d$.}

\emph{This example is beyond the scope of existing results in online optimization, e.g.,~\cite{shi2020online,goel2019beyond,goel2019online}, because of (i) the $k$-step delay in the hitting cost $\frac{1}{2}(y_t-y_t^d)$ and (ii) the nonlinearity in the switching cost $\frac{1}{2}(y_t-y_{t-1}+g(y_{t-1}))^2$ with respective to $y_{t-1}$. However, in this paper, because we directly incorporate the effect of delay and nonlinearity in the algorithm design, our algorithms immediately provide constant-competitive policies for this setting.}
\end{example}


\subsection{Related Work}
This paper contributes to the growing literature on online convex optimization with memory.  
Initial results in this area focused on developing constant-competitive algorithms for the special case of 1-step memory, a.k.a., the Smoothed Online Convex Optimization (SOCO) problem, e.g., \citep{chen2018smoothed,goel2019beyond}. In that setting, \citep{chen2018smoothed} was the first to develop a constant, dimension-free competitive algorithm for high-dimensional problems.  The proposed algorithm, Online Balanced Descent (OBD), achieves a competitive ratio of $3+O(1/\beta)$ when cost functions are $\beta$-locally polyhedral.  This result was improved by \citep{goel2019beyond}, which proposed two new algorithms, Greedy OBD and Regularized OBD (ROBD), that both achieve $1+O(m^{-1/2})$ competitive ratios for $m$-strongly convex cost functions.  Recently, \citep{shi2020online} gave the first competitive analysis that holds beyond one step of memory.  It holds for a form of structured memory where the switching cost is linear:
$
    c(y_{t:t-p})=\frac{1}{2}\|y_t-\sum_{i=1}^pC_iy_{t-i}\|^2,
$
with known $C_i\in\mathbb{R}^{d\times d}$, $i=1,\cdots,p$. If the memory length $p = 1$ and $C_1$ is an identity matrix, this is equivalent to SOCO. In this setting, \citep{shi2020online} shows that ROBD has a competitive ratio of 
\begin{align}
    \frac{1}{2}\left( 1 + \frac{\alpha^2 - 1}{m} + \sqrt{\Big( 1 + \frac{\alpha^2 - 1}{m}\Big)^2 + \frac{4}{m}} \right),
\end{align}
when hitting costs are $m$-strongly convex and $\alpha=\sum_{i=1}^p\|C_i\|$. 


Prior to this paper, competitive algorithms for online optimization have nearly always assumed that the online learner acts \emph{after} observing the cost function in the current round, i.e., have zero delay.  The only exception is \citep{shi2020online}, which considered the case where the learner must act before observing the cost function, i.e., a one-step delay.  Even that small addition of delay requires a significant modification to the algorithm (from ROBD to Optimistic ROBD) and analysis compared to previous work. 

As the above highlights, there is no previous work that addresses either the setting of nonlinear switching costs nor the setting of multi-step delay. However, the prior work highlights that ROBD is a promising algorithmic framework and our work in this paper extends the ROBD framework in order to address the challenges of delay and non-linear switching costs. Given its importance to our work, we describe the workings of ROBD in detail in Algorithm~\ref{robd}. 

\begin{algorithm}[t!]
  \caption{ROBD \citep{goel2019beyond}}
  \label{robd}
\begin{algorithmic}[1]
  \STATE {\bfseries Parameter:} $\lambda_1\ge0,\lambda_2\ge0$
  \FOR{$t=1$ {\bfseries to} $T$}
  \STATE {\bfseries Input:} Hitting cost function $f_t$, previous decision points $y_{t-p:t-1}$
  \STATE $v_t\leftarrow\arg\min_yf_t(y)$
  \STATE $y_t\leftarrow\arg\min_yf_t(y)+\lambda_1c(y,y_{t-1:t-p})+\frac{\lambda_2}{2}\|y-v_t\|^2_2$
  \STATE {\bfseries Output:} $y_t$
  \ENDFOR
   
\end{algorithmic}
\end{algorithm}

Another line of literature that this paper contributes to is the growing understanding of the connection between online optimization and adaptive control. The reduction from adaptive control to online optimization with memory was first studied in \citep{agarwal2019online} to obtain a sublinear static regret guarantee against the best linear state-feedback controller, where the approach is to consider a disturbance-action policy class with some fixed horizon.  Many follow-up works adopt similar reduction techniques \citep{agarwal2019logarithmic, brukhim2020online, gradu2020adaptive}. A different reduction approach using control canonical form is proposed by \citep{li2019online} and further exploited by \citep{shi2020online}. Our work falls into this category.  The most general results so far focus on Input-Disturbed Squared Regulators, which can be reduced to online convex optimization with structured memory (without delay or nonlinear switching costs).  As we show in \Cref{Control}, the addition of delay and nonlinear switching costs leads to a significant extension of the generality of control models that can be reduced to online optimization. 

\section{Experimental protocols}\label{sec:protocols}
\subsection{Datasets}

We use the 
MovieLens--20M\footnote{http://grouplens.org/datasets/movielens}
and Netflix\footnote{http://www.netflixprize.com} explicit feedback  datasets.
As both  of these contain explicit ratings, we create binary preferences
by keeping ratings $\ge\!4$, which we interpret as positive feedback ($p_{ui}=1$).
Furthermore, we only keep users with at least 5 views.
Validation and test sets are obtained randomly, selecting a $10~\%$ of the original dataset for each set. We denote such datasets \textsc{ML20M} and \textsc{Netflix}.

In addition, we explore models performance on the Last.fm\footnote{https://www.upf.edu/web/mtg/lastfm360k} dataset~\cite{Celma:Springer2010}, an implicit feedback dataset consisting of tuples (user, artist, plays), that contains top artists by user. In order to make the comparison with the above datasets more straightforward, we binarize play counts and interpret them as implicit preference data. Next, we filter out artist with less than 50 distinct listening users, and user with less than 20 artists in their listening history. In the following, we name this dataset \textsc{Lastfm}.

\setlength{\belowcaptionskip}{5pt}
\begin{table}[htb]
\begin{tabular}{c c c c c}
 Dataset & \#users & \#items & \#pairs & \#pairs$_{\rm test}$ \\
\hline
\textsc{ML20M} & 136,7k & 20,3k & 7,99M & 1,0M \\
\textsc{Netflix}  & 463,4k & 17,7k & 45,5M & 5,7M \\
\textsc{Lastfm}  & 350,2k & 24,6k & 12,8M & 1,6M \\
\hline
\end{tabular}
\caption{Statistics of the datasets after preprocessing.}
\label{table:datasets}
\end{table}

The statistics of the training set after such  processing, as well as the number of user-item interactions in test, are presented in Table~\ref{table:datasets}. 
%We also represent the item popularity distribution in training, validation and test sets in Figure~\ref{fig:ditribution_dataset} for the \textsc{ML20M},  \textsc{Netflix}, \textsc{Lastfm} datasets, from top to bottom, respectively. As expected, validation and test distributions  follow the distribution of the training set. 

% \begin{figure}
%     \centering
%     \includegraphics[width=.7\linewidth]{figures/distributions_dataset.png}
%     \caption{Normalized distributions of user-item interactions in train (green), validation (red) and test (blue) sets for the  \textsc{ML20M},  \textsc{Netflix} and \textsc{Last-fm} datasets (from top to bottom, respectively).}
%     \label{fig:ditribution_dataset}
% \end{figure}

\subsection{Evaluation metrics}\label{subsec:metrics}
\setlength{\belowcaptionskip}{-10pt}
Given the set of adopted items in test, $\mathcal{I}_u^{\rm t}$, and the ranked list of predicted preferences, 
the relevance of a recommendation at position $k$ is given by ${\rm rel}_{ui}(k)$--${\rm rel}(k)$ from here on--, which equals $1$ if user $u$ adopted item $i$ in the test set, $0$ otherwise. In the calculation of metrics, we remove items observed in  training and validation from the list of recommendations. 
Next, we detail the metrics used for model evaluation.

\paragraph{Recall} It does not account for the relative ordering of the recommendation, and we defined it as~\cite{liang:2018:VAE}
\begin{equation}
{\rm Recall}@k = \frac{\sum_{s=1}^k {\rm rel}(s)}
{\mathcal{N}_u(k)}.
\end{equation}
Here, $\mathcal{N}_u(k) = \min\left(k,|\mathcal{I}_u^{\rm t}|\right)$, with $|\mathcal{I}_u^{\rm t}|$ the number of items adopted by user $u$ in testing. The final recall is averaged across all users in testing. 

\paragraph{Normalized Discount Cumulative Gain} In contrast to recall metric, the Discount Cumulative Gain (DCG) performs a logarithmic discount according to the position of a recommendation, that is
\begin{equation}
{\rm DCG}@k = \sum_{s=1}^k \frac{{\rm rel}(s)}
{\log_2(s+1)}.
\end{equation}
This quantity can be normalized by the Ideal DCG, 
\begin{equation}
{\rm IDCG}@k = \sum_{s=1}^{\mathcal{N}_u(k)} \frac{1}{\log_2(s+1)}.
\end{equation}
Finally, NDCG$@k= {\rm DCG}@k/{\rm IDCG}@k$, which we average across all users in the test set.

\paragraph{Novelty} Following reference~\cite{Vargas:2011:Novelty_diversity}, we define a novelty-weighted DCG score as
\begin{equation}\label{eq:nov-ndcg}
{\rm Nov\!-\!DCG}@k = 
-\sum_{s=1}^k \frac{{\rm rel}(s)\times \ln{\nu(i)}}
{\log_2(s+1)}.
\end{equation}
Here, $\nu(i)$ is the frequency of occurrences of item $i$  normalized to the total interactions in training. The corresponding novelty-weighted IDCG would be
\begin{equation}
{\rm Nov\!-\!IDCG}@k = 
\sum_{s=1}^{\mathcal{N}_u(k)} 
\frac{\max_{i\in\mathcal{I}_u}\left(-\ln \nu(i)\right)}
{\log_2(s+1)}.
\end{equation}
In other words, the highest DCG is obtained by ranking the most novel items (among those relevant to the user) in descending order. 
\subsection{Implementation details}\label{subsec:implementation}
The implementation of our model is performed in TensorFlow~\cite{tensorflow2015-whitepaper}.
%\footnote{The code will be available online at
%\url{https://github.com/bbvadata/RecApp}.
%}. 
The model can be trained in both CPU or GPU. 
When GPU is enabled, the use of queues to feed the tensors greatly speeds up the training.
% batch size and epoch
We set the batch size to $100$, and train every DAE model for $120$k iterations, so as to ensure proper convergence. For MF models we use $180$k iterations.
% number of neurons
The number of neurons is $200$ in all DAE experiments; for MF models, since the large number of users makes them prone to overfit, we train the models with $100$ and $200$ neurons and take the best performing model.
% weights and biases initialization
Weight matrices are initialized with random uniform values whose amplitude is computed as described by Glorot~\emph{et al.}~\cite{Xavier_initialization}. 
%Generally speaking, we discourage the use of random normal initialization without truncation.
For the biases we use a truncated random normal initialization with a standard deviation of $10^{-3}$. 
% Optimizer
Models are trained with Adam optimizer~\cite{Kingma2014AdamAM} and a learning rate of $10^{-3}$.
% gradient clip
%Additionally, we clip gradients whenever they exceed a norm of $1$, and apply batch normalization~\cite{icml2015_ioffe15_batch-norm} at every training step.

% Size of T and P sets
Concerning negative sampling in point and pair--wise schemes, 
%since we train with a batch size different form one, 
we fix the size of the target sets for every user (sets $\mathcal{T}_u$ and $\mathcal{P}_u$ for point and pair--wise learning, respectively, see subsection~\ref{subsec:losses}).
In particular, we make such sets proportional to the median number of items adopted by users, except for the multinomial loss, where all items are utilized~\cite{Liang:2016:CoFactor}.
The proportionality factors are hyper-parameters fine--tuned with the validation set, swapping the values $\{1,\,5,\,10,\,50,\,100,\,150\}$. We find a factor of $50$ or $100$ to provide the best results.
%(see additional comments in subsection~\ref{subsec:baseline_models}).

% Max norm regularization
%Regarding max--norm regularization, we notice that an asymmetric max--norm regularization might be required for the encoder/decoder weight matrices, depending on the activations and objective functions utilized for modeling. We swap $\alpha_{\rm enc}$ and $\alpha_{\rm dec}$ values in $\{0.0, 0.05, 0.1, 0.2, 0.3, 0.5, 1.0\}$, where $\alpha=0.0$ means no regularization.

% Regularization: noise and weight-decay
We add noise to the input vector of the AE~\cite{Vincent:2008:ECRF-AE, Wu:2016:CDAE-topN} using drop-out~\cite{liang:2018:VAE}. We fix the level of noise at $0.5$. Competitive performance is achieved after normalizing the AE input vector.
% Regularization of MIL is smaller
For DAE models, we swap the $L_2$ regularization strength $\lambda\in[10^{-7}-10^{-4}]$, while for MF models we take the form in equation~(\ref{eq:l2_reg_scaled}) with $\lambda\in[10^{0}-10^{3}]$, which provides a more stable training for MF models\footnote{Recall the different scales of the $\lambda$ factor in equations (\ref{eq:L2_reg}) and (\ref{eq:l2_reg_scaled}).}. In general we find that MIL models require smaller $\lambda$ factors than cross-entropy or multinomial--based models. This is expected, as the level of weight--decay regularization in equations~(\ref{eq:L2_reg}) and (\ref{eq:l2_reg_scaled}) depends on the value of the loss, which is smaller for MIL models.

\subsection{Baseline models}\label{subsec:baseline_models}
We implement the objective functions described in subsection \ref{subsec:losses} on a user-based DAE~\cite{Sedhain:2015:Autorec, Wu:2016:CDAE-topN} and compare the results with the MIL function. We also compare them with traditional Matrix Factorization with Weight Regularization~\cite{HuKoren:2008:CF_implicit}. In the following, we provide details on the training of the different models.

\textbf{Weight-Regularized Matrix Factorization} WRMF~\cite{HuKoren:2008:CF_implicit} is a linear factorization model  trained with square loss and weight decay. We use negative sampling with a sampling ratio of $100$ and $\lambda\sim 5-10$ (as obtained in the validation set). We call this model \MFsquare. In addition, we train WRMF models with MIL and point--wise cross--entropy losses, applying a sigmoid function at the output, so as to ensure $\hat{p}_{ui}\in(0,1)$. In these cases, we find that a sampling ratio of $100$ and $\lambda=50-500$ provide best results. We name these models \MFmil\, and \MFce, respectively.

\textbf{\emph{Denoising Autoencoder models}}

\textbf{Cross-entropy loss} For the cross-entropy loss defined in equations~(\ref{eq:cross-entropy}), (\ref{eq:point-wise}) and (\ref{eq:pair-wise}), we use linear--sigmoid and sigmoid--sigmoid activations at the encoder and decoder, respectively. We name the DAEs models with cross-entropy loss and point--wise estimation \CEpointlinsig\,and \CEpointsigsig; and those with pair--wise,  \CEpairlinsig \,and \CEpairsigsig. In order to prevent numerical instabilities, we ensure that the output preferences are in $[\varepsilon, 1-\varepsilon]$, with $\varepsilon=10^{-5}$. Regarding negative sampling, we find that the best sampling ratio is $50\times {\rm median}(\mathcal{I}_u)$ and $100\times {\rm median}(\mathcal{I}_u)$  for point and pair--wise estimation, respectively. Best weight-decay regularization is found to be $\lambda=2\cdot 10^{-5}$.

% Comparison to CDAE
The closest model to these baselines is the Collaborative Denoising AE (CDAE)~\cite{Wu:2016:CDAE-topN}, although for the sake of simplicity, in the present paper we do not include the user embedding of CDAE. 
% bad performance of BPR
Similar to CDAE, we find that  pair--wise learning does not achieve competitive results at the top of the ranked list~\cite{Wu:2016:CDAE-topN, liang:2018:VAE}. 

\textbf{Multinomial loss} AEs trained with a multinomial log-likelihood have recently been  introduced by Lian et al~\cite{liang:2018:VAE}, either applied to DAEs or Variational AEs (VAE) with partial regularization. Here, we focus on the multi-DAE modeling with $\tanh$-linear activations\footnote{
We use the actual implementation provided at \url{https://github.com/dawenl/vae_cf} to verify that the activation used at the decoder of multi-DAE is linear, although the original writing~\cite{liang:2018:VAE} suggests a $\tanh$ non-linearity for the decoder.
}, and name this baseline \MULTItanhlin. Our implementation exactly reproduces that of~\cite{liang:2018:VAE} when using $\lambda=2\cdot 10^{-5}$, input noise of $0.5$ and without applying negative sampling. 

\textbf{Missing Information loss} We apply the \textsc{MIL} function defined in equation~(\ref{eq:mil_def}) to linear-sigmoid and sigmoid-sigmoid DAEs. We name these models \MILlinsig\, and \MILsigsig, respectively. Best hyper-parameters of the loss turn out to be $A_{\rm MI}=10^6,\, \gamma_{\rm MI}=10$ and $\gamma_{+}=1$, after grid search the pairs $\left(A_{\rm MI}, \gamma_{\rm MI}\right)\in\{
(5\cdot10^1, 2)$, $(10^3, 4)$, $(2\cdot10^4, 6)$, $(5\cdot10^5, 10)$, $(1\cdot10^6, 10)$, $(5\cdot10^6, 10)$, $(5\cdot10^9, 15)\}$, and $\gamma_{+}=1$ or $2$. In addition, we use a sampling ratio of $50$ and $\lambda\in(10^{-6}, 10^{-5})$.

%!TEX ROOT = ../../centralized_vs_distributed.tex

\section{{\titlecap{the centralized-distributed trade-off}}}\label{sec:numerical-results}

\revision{In the previous sections we formulated the optimal control problem for a given controller architecture
(\ie the number of links) parametrized by $ n $
and showed how to compute minimum-variance objective function and the corresponding constraints.
In this section, we present our main result:
%\red{for a ring topology with multiple options for the parameter $ n $},
we solve the optimal control problem for each $ n $ and compare the best achievable closed-loop performance with different control architectures.\footnote{
\revision{Recall that small (large) values of $ n $ mean sparse (dense) architectures.}}
For delays that increase linearly with $n$,
\ie $ f(n) \propto n $, 
we demonstrate that distributed controllers with} {few communication links outperform controllers with larger number of communication links.}

\textcolor{subsectioncolor}{Figure~\ref{fig:cont-time-single-int-opt-var}} shows the steady-state variances
obtained with single-integrator dynamics~\eqref{eq:cont-time-single-int-variance-minimization}
%where we compare the standard multi-parameter design 
%with a simplified version \tcb{that utilizes spatially-constant feedback gains
and the quadratic approximation~\eqref{eq:quadratic-approximation} for \revision{ring topology}
with $ N = 50 $ nodes. % and $ n\in\{1,\dots,10\} $.
%with $ N = 50 $, $ f(n) = n $ and $ \tau_{\textit{min}} = 0.1 $.
%\autoref{fig:cont-time-single-int-err} shows the relative error, defined as
%\begin{equation}\label{eq:relative-error}
%	e \doteq \dfrac{\optvarx-\optvar}{\optvar}
%\end{equation}
%where $ \optvar $ and $ \optvarx $ denote the the optimal and sub-optimal scalar variances, respectively.
%The performance gap is small
%and becomes negligible for large $ n $.
{The best performance is achieved for a sparse architecture with  $ n = 2 $ 
in which each agent communicates with the two closest pairs of neighboring nodes. 
This should be compared and contrasted to nearest-neighbor and all-to-all 
communication topologies which induce higher closed-loop variances. 
Thus, 
the advantage of introducing additional communication links diminishes 
beyond}
{a certain threshold because of communication delays.}

%For a linear increase in the delay,
\textcolor{subsectioncolor}{Figure~\ref{fig:cont-time-double-int-opt-var}} shows that the use of approximation~\eqref{eq:cont-time-double-int-min-var-simplified} with $ \tilde{\gvel}^* = 70 $
identifies nearest-neighbor information exchange as the {near-optimal} architecture for a double-integrator model
with ring topology. 
This can be explained by noting that the variance of the process noise $ n(t) $
in the reduced model~\eqref{eq:x-dynamics-1st-order-approximation}
is proportional to $ \nicefrac{1}{\gvel} $ and thereby to $ \taun $,
according to~\eqref{eq:substitutions-4-normalization},
making the variance scale with the delay.

%\mjmargin{i feel that we need to comment about different results that we obtained for CT and DT double-intergrator dynamics (monotonic deterioration of performance for the former and oscillations for the latter)}
\revision{\textcolor{subsectioncolor}{Figures~\ref{fig:disc-time-single-int-opt-var}--\ref{fig:disc-time-double-int-opt-var}}
show the results obtained by solving the optimal control problem for discrete-time dynamics.
%which exhibit similar trade-offs.
The oscillations about the minimum in~\autoref{fig:disc-time-double-int-opt-var}
are compatible with the investigated \tradeoff~\eqref{eq:trade-off}:
in general, 
the sum of two monotone functions does not have a unique local minimum.
Details about discrete-time systems are deferred to~\autoref{sec:disc-time}.
Interestingly,
double integrators with continuous- (\autoref{fig:cont-time-double-int-opt-var}) ad discrete-time (\autoref{fig:disc-time-double-int-opt-var}) dynamics
exhibits very different trade-off curves,
whereby performance monotonically deteriorates for the former and oscillates for the latter.
While a clear interpretation is difficult because there is no explicit expression of the variance as a function of $ n $,
one possible explanation might be the first-order approximation used to compute gains in the continuous-time case.
%which reinforce our thesis exposed in~\autoref{sec:contribution}.

%\begin{figure}
%	\centering
%	\includegraphics[width=.6\linewidth]{cont-time-double-int-opt-var-n}
%	\caption{Steady-state scalar variance for continuous-time double integrators with $ \taun = 0.1n $.
%		Here, the \tradeoff is optimized by nearest-neighbor interaction.
%	}
%	\label{fig:cont-time-double-int-opt-var-lin}
%\end{figure}
}

\begin{figure}
	\centering
	\begin{minipage}[l]{.5\linewidth}
		\centering
		\includegraphics[width=\linewidth]{random-graph}
	\end{minipage}%
	\begin{minipage}[r]{.5\linewidth}
		\centering
		\includegraphics[width=\linewidth]{disc-time-single-int-random-graph-opt-var}
	\end{minipage}
	\caption{Network topology and its optimal {closed-loop} variance.}
	\label{fig:general-graph}
\end{figure}

Finally,
\autoref{fig:general-graph} shows the optimization results for a random graph topology with discrete-time single integrator agents. % with a linear increase in the delay, $ \taun = n $.
Here, $ n $ denotes the number of communication hops in the ``original" network, shown in~\autoref{fig:general-graph}:
as $ n $ increases, each agent can first communicate with its nearest neighbors,
then with its neighbors' neighbors, and so on. For a control architecture that utilizes different feedback gains for each communication link
	(\ie we only require $ K = K^\top $) we demonstrate that, in this case, two communication hops provide optimal closed-loop performance. % of the system.}

Additional computational experiments performed with different rates $ f(\cdot) $ show that the optimal number of links increases for slower rates: 
for example, 
the optimal number of links is larger for $ f(n) = \sqrt{n} $ than for $ f(n) = n $. 
\revision{These results are not reported because of space limitations.}

\section{Conclusions and next steps}\label{sec:conclusions}

% definition of MIL
In this paper we present a novel objective function, the \emph{Missing Information Loss} (MIL), specifically designed for handling unobserved user-item interactions in implicit feedback datasets. In particular, MIL explicitly forbids treating missing user-item interactions as positive or negative feedback.
% What it does
We demonstrate that, thanks to the functional form of the MIL function, the ranking of unseen items is almost entirely left to the low--rank process, rather than forcing unobserved items to be at the tail of the recommendation (\emph{i.e.}, MIL does not force a zero predicted preference for unobserved user-item interactions). 

% Metric results
Extensive experiments with Matrix Factorization and Denoising Autoencoders conducted on three datasets, show that \textsc{MIL} models demonstrate competitive performance when compared with other traditional losses such as cross-entropy or the multinomial log-likelihood. 
% Best performing models 
% Analysis of recommendations
In addition, we study the distribution of the recommendations and observe that the reported metric performance takes place while recommending popular items less frequently (up to a $20 \%$ decrease with respect to the best competing method). Indeed, \textsc{MIL} models sharply increase the recommendation of medium--tail items, while almost linearly expanding the appearance of long--tail items with the ranking position in the list of recommendations. Such expansion results in up to a $50 \%$ increase of long--tail recommendations, a feature of utmost importance for industries with a large catalogue of items. 

% Future work
Future lines of research may involve the incorporation of negative feedback, or the usage of \textsc{MIL} in temporal--aware Recommender Systems (such as those using Recurrent Neural Networks).  
In addition, we hope that the results here reported  will bring forward first-principle mathematical derivations of the \textsc{MIL} function, so that the vast family of possible polynomials modelling the missing information term can be reduced, or even extended with more suitable functions. 

\begin{acks}
We would like to thank the continuous support and careful reading of the manuscript by the \emph{Edge} guild within BBVA Data \& Analytics, specially J. Garc\'ia Santamar\'ia and J. A. Rodr\'iguez Serrano. 
\end{acks}

%\bibliographystyle{ACM-Reference-Format}
\bibliographystyle{unsrt}
%\bibliography{sigproc} 

\documentclass[sigconf]{acmart}

\settopmatter{printacmref=false} % Removes citation information below abstract
\renewcommand\footnotetextcopyrightpermission[1]{} % removes footnote with conference information in first column
\pagestyle{plain} % removes running headers

\usepackage{booktabs} % For formal tables
\usepackage{multirow}
\usepackage{amsmath}
\usepackage{color}
\usepackage{arydshln }


%Conference
\acmConference[WSDM]{The Twelfth International Conference on Web Search and Data Mining}{February 11--15}{Melbourne, Australia} 
\acmYear{2019}

\begin{document}
\title[Missing Information Loss]{A Missing Information Loss for implicit feedback datasets}

\author{Juan Ar\'evalo}
\affiliation{%
  \institution{BBVA Data \& Analytics}
}
\email{juanmaria.arevalo@bbvadata.com}

\author{Juan Ram\'on Duque}
\affiliation{%
  \institution{BBVA Data \& Analytics}
}
\email{juanramon.duque@bbvadata.com}

\author{Marco Creatura}
\affiliation{%
  \institution{BBVA Data \& Analytics}
}
\email{marco.creatura@bbvadata.com}

% The default list of authors is too long for headers}
\renewcommand{\shortauthors}{Ar\'evalo, Duque and Creatura}

% Some useful commands
\newcommand{\MFsquare}{\textsc{MF-square}}
\newcommand{\MFmil}{\textsc{MF-mil}}
\newcommand{\MFce}{\textsc{MF-CE}}
\newcommand{\CEpointlinsig}{\textsc{CE$_{\rm Point}$ lin-sig}}
\newcommand{\CEpointsigsig}{\textsc{CE$_{\rm Point}$ sig-sig}}
\newcommand{\CEpairlinsig}{\textsc{CE$_{\rm Pair}$ lin-sig}}
\newcommand{\CEpairsigsig}{\textsc{CE$_{\rm Pair}$ sig-sig}}
\newcommand{\MULTItanhlin}{\textsc{MULTI tanh-lin}}
\newcommand{\MILlinsig}{\textsc{MIL lin-sig}}
\newcommand{\MILsigsig}{\textsc{MIL sig-sig}}


\begin{abstract}
% missing values and negative feedback
Latent factor models for Recommender Systems with implicit feedback typically treat unobserved user-item interactions (\emph{i.e.} missing information) as negative feedback. This is frequently done 
either through negative sampling (point--wise loss) or with a ranking loss function (pair-- or list--wise estimation). 
% Common objective functions allow zero prediction
Since a zero preference recommendation is a valid solution for most common objective functions, 
regarding unknown values as actual zeros results in users 
having a zero preference recommendation  for most of the available items. 

% MIL
In this paper we propose a novel objective function, the \emph{Missing Information Loss} (MIL), 
that explicitly forbids treating unobserved user-item interactions as positive or negative feedback. 
% application to AE and metrics
We apply this loss to both traditional Matrix Factorization and user--based Denoising Autoencoder, and compare it with other established objective functions such as cross--entropy (both point-- and pair--wise) or the recently proposed multinomial log-likelihood. MIL achieves competitive performance in ranking--aware metrics when applied to three datasets.
% towards long-tail recommendations
Furthermore, we show that such a relevance in the recommendation is obtained while displaying popular items less frequently (up to a $20 \%$ decrease with respect to the best competing method). This debiasing from the recommendation of popular items favours the appearance of infrequent items (up to a $50 \%$ increase of long--tail recommendations), a valuable feature for Recommender Systems with a large catalogue of products. 
\end{abstract}

\begin{CCSXML}
<ccs2012>
<concept>
<concept_id>10002951.10003317.10003347.10003350</concept_id>
<concept_desc>Information systems~Recommender systems</concept_desc>
<concept_significance>500</concept_significance>
</concept>
</ccs2012>
\end{CCSXML}

\ccsdesc[500]{Information systems~Recommender systems}

\keywords{Collaborative Filtering, Autoencoders, Implicit Feedback, Missing Information}

\setcopyright{None}

\maketitle

\setlength{\abovecaptionskip}{0pt}
\setlength{\belowcaptionskip}{-10pt}

% Put all the sections with inputs

% !TEX root = ../arxiv.tex

Unsupervised domain adaptation (UDA) is a variant of semi-supervised learning \cite{blum1998combining}, where the available unlabelled data comes from a different distribution than the annotated dataset \cite{Ben-DavidBCP06}.
A case in point is to exploit synthetic data, where annotation is more accessible compared to the costly labelling of real-world images \cite{RichterVRK16,RosSMVL16}.
Along with some success in addressing UDA for semantic segmentation \cite{TsaiHSS0C18,VuJBCP19,0001S20,ZouYKW18}, the developed methods are growing increasingly sophisticated and often combine style transfer networks, adversarial training or network ensembles \cite{KimB20a,LiYV19,TsaiSSC19,Yang_2020_ECCV}.
This increase in model complexity impedes reproducibility, potentially slowing further progress.

In this work, we propose a UDA framework reaching state-of-the-art segmentation accuracy (measured by the Intersection-over-Union, IoU) without incurring substantial training efforts.
Toward this goal, we adopt a simple semi-supervised approach, \emph{self-training} \cite{ChenWB11,lee2013pseudo,ZouYKW18}, used in recent works only in conjunction with adversarial training or network ensembles \cite{ChoiKK19,KimB20a,Mei_2020_ECCV,Wang_2020_ECCV,0001S20,Zheng_2020_IJCV,ZhengY20}.
By contrast, we use self-training \emph{standalone}.
Compared to previous self-training methods \cite{ChenLCCCZAS20,Li_2020_ECCV,subhani2020learning,ZouYKW18,ZouYLKW19}, our approach also sidesteps the inconvenience of multiple training rounds, as they often require expert intervention between consecutive rounds.
We train our model using co-evolving pseudo labels end-to-end without such need.

\begin{figure}[t]%
    \centering
    \def\svgwidth{\linewidth}
    \input{figures/preview/bars.pdf_tex}
    \caption{\textbf{Results preview.} Unlike much recent work that combines multiple training paradigms, such as adversarial training and style transfer, our approach retains the modest single-round training complexity of self-training, yet improves the state of the art for adapting semantic segmentation by a significant margin.}
    \label{fig:preview}
\end{figure}

Our method leverages the ubiquitous \emph{data augmentation} techniques from fully supervised learning \cite{deeplabv3plus2018,ZhaoSQWJ17}: photometric jitter, flipping and multi-scale cropping.
We enforce \emph{consistency} of the semantic maps produced by the model across these image perturbations.
The following assumption formalises the key premise:

\myparagraph{Assumption 1.}
Let $f: \mathcal{I} \rightarrow \mathcal{M}$ represent a pixelwise mapping from images $\mathcal{I}$ to semantic output $\mathcal{M}$.
Denote $\rho_{\bm{\epsilon}}: \mathcal{I} \rightarrow \mathcal{I}$ a photometric image transform and, similarly, $\tau_{\bm{\epsilon}'}: \mathcal{I} \rightarrow \mathcal{I}$ a spatial similarity transformation, where $\bm{\epsilon},\bm{\epsilon}'\sim p(\cdot)$ are control variables following some pre-defined density (\eg, $p \equiv \mathcal{N}(0, 1)$).
Then, for any image $I \in \mathcal{I}$, $f$ is \emph{invariant} under $\rho_{\bm{\epsilon}}$ and \emph{equivariant} under $\tau_{\bm{\epsilon}'}$, \ie~$f(\rho_{\bm{\epsilon}}(I)) = f(I)$ and $f(\tau_{\bm{\epsilon}'}(I)) = \tau_{\bm{\epsilon}'}(f(I))$.

\smallskip
\noindent Next, we introduce a training framework using a \emph{momentum network} -- a slowly advancing copy of the original model.
The momentum network provides stable, yet recent targets for model updates, as opposed to the fixed supervision in model distillation \cite{Chen0G18,Zheng_2020_IJCV,ZhengY20}.
We also re-visit the problem of long-tail recognition in the context of generating pseudo labels for self-supervision.
In particular, we maintain an \emph{exponentially moving class prior} used to discount the confidence thresholds for those classes with few samples and increase their relative contribution to the training loss.
Our framework is simple to train, adds moderate computational overhead compared to a fully supervised setup, yet sets a new state of the art on established benchmarks (\cf \cref{fig:preview}).


Online convex optimization with memory has emerged as an important and challenging area with a wide array of applications, see \citep{lin2012online,anava2015online,chen2018smoothed,goel2019beyond,agarwal2019online,bubeck2019competitively} and the references therein.  Many results in this area have focused on the case of online optimization with switching costs (movement costs), a form of one-step memory, e.g., \citep{chen2018smoothed,goel2019beyond,bubeck2019competitively}, though some papers have focused on more general forms of memory, e.g., \citep{anava2015online,agarwal2019online}. In this paper we, for the first time, study the impact of feedback delay and nonlinear switching cost in online optimization with switching costs. 

An instance consists of a convex action set $\mathcal{K}\subset\mathbb{R}^d$, an initial point $y_0\in\mathcal{K}$, a sequence of non-negative convex cost functions $f_1,\cdots,f_T:\mathbb{R}^d\to\mathbb{R}_{\ge0}$, and a switching cost $c:\mathbb{R}^{d\times(p+1)}\to\mathbb{R}_{\ge0}$. To incorporate feedback delay, we consider a situation where the online learner only knows the geometry of the hitting cost function at each round, i.e., $f_t$, but that the minimizer of $f_t$ is revealed only after a delay of $k$ steps, i.e., at time $t+k$.  This captures practical scenarios where the form of the loss function or tracking function is known by the online learner, but the target moves over time and measurement lag means that the position of the target is not known until some time after an action must be taken. 
To incorporate nonlinear (and potentially nonconvex) switching costs, we consider the addition of a known nonlinear function $\delta$ from $\mathbb{R}^{d\times p}$ to $\mathbb{R}^d$ to the structured memory model introduced previously.  Specifically, we have
\begin{align}
c(y_{t:t-p}) = \frac{1}{2}\|y_t-\delta(y_{t-1:t-p})\|^2,    \label{e.newswitching}
\end{align}
where we use $y_{i:j}$ to denote either $\{y_i, y_{i+1}, \cdots, y_j\}$ if $i\leq j$, or  $\{y_i, y_{i-1}, \cdots, y_j\}$ if $i > j$ throughout the paper. Additionally, we use $\|\cdot\|$ to denote the 2-norm of a vector or the spectral norm of a matrix.

In summary, we consider an online agent that interacts with the environment as follows:
% \begin{inparaenum}[(i)] 
\begin{enumerate}%[leftmargin=*]
    \item The adversary reveals a function $h_t$, which is the geometry of the $t^\mathrm{th}$ hitting cost, and a point $v_{t-k}$, which is the minimizer of the $(t-k)^\mathrm{th}$ hitting cost. Assume that $h_t$ is $m$-strongly convex and $l$-strongly smooth, and that $\arg\min_y h_t(y)=0$.
    \item The online learner picks $y_t$ as its decision point at time step $t$ after observing $h_t,$  $v_{t-k}$.
    \item The adversary picks the minimizer of the hitting cost at time step $t$: $v_t$. 
    \item The learner pays hitting cost $f_t(y_t)=h_t(y_t-v_t)$ and switching cost $c(y_{t:t-p})$ of the form \eqref{e.newswitching}.
\end{enumerate}

The goal of the online learner is to minimize the total cost incurred over $T$ time steps, $cost(ALG)=\sum_{t=1}^Tf_t(y_t)+c(y_{t:t-p})$, with the goal of (nearly) matching the performance of the offline optimal algorithm with the optimal cost $cost(OPT)$. The performance metric used to evaluate an algorithm is typically the \textit{competitive ratio} because the goal is to learn in an environment that is changing dynamically and is potentially adversarial. Formally, the competitive ratio (CR) of the online algorithm is defined as the worst-case ratio between the total cost incurred by the online learner and the offline optimal cost: $CR(ALG)=\sup_{f_{1:T}}\frac{cost(ALG)}{cost(OPT)}$.

It is important to emphasize that the online learner decides $y_t$ based on the knowledge of the previous decisions $y_1\cdots y_{t-1}$, the geometry of cost functions $h_1\cdots h_t$, and the delayed feedback on the minimizer $v_1\cdots v_{t-k}$. Thus, the learner has perfect knowledge of cost functions $f_1\cdots f_{t-k}$, but incomplete knowledge of $f_{t-k+1}\cdots f_t$ (recall that $f_t(y)=h_t(y-v_t)$).

Both feedback delay and nonlinear switching cost add considerable difficulty for the online learner compared to versions of online optimization studied previously. Delay hides crucial information from the online learner and so makes adaptation to changes in the environment more challenging. As the learner makes decisions it is unaware of the true cost it is experiencing, and thus it is difficult to track the optimal solution. This is magnified by the fact that nonlinear switching costs increase the dependency of the variables on each other. It further stresses the influence of the delay, because an inaccurate estimation on the unknown data, potentially magnifying the mistakes of the learner. 

The impact of feedback delay has been studied previously in online learning settings without switching costs, with a focus on regret, e.g., \citep{joulani2013online,shamir2017online}.  However, in settings with switching costs the impact of delay is magnified since delay may lead to not only more hitting cost in individual rounds, but significantly larger switching costs since the arrival of delayed information may trigger a very large chance in action.  To the best of our knowledge, we give the first competitive ratio for delayed feedback in online optimization with switching costs. 

We illustrate a concrete example application of our setting in the following.

\begin{example}[Drone tracking problem]
\label{example:drone} \emph{
Consider a drone with vertical speed $y_t\in\mathbb{R}$. The goal of the drone is to track a sequence of desired speeds $y^d_1,\cdots,y^d_T$ with the following tracking cost:}
\begin{equation}
    \sum_{t=1}^T \frac{1}{2}(y_t-y^d_t)^2 + \frac{1}{2}(y_t-y_{t-1}+g(y_{t-1}))^2,
\end{equation}
\emph{where $g(y_{t-1})$ accounts for the gravity and the aerodynamic drag. One example is $g(y)=C_1+C_2\cdot|y|\cdot y$ where $C_1,C_2>0$ are two constants~\cite{shi2019neural}. Note that the desired speed $y_t^d$ is typically sent from a remote computer/server. Due to the communication delay, at time step $t$ the drone only knows $y_1^d,\cdots,y_{t-k}^d$.}

\emph{This example is beyond the scope of existing results in online optimization, e.g.,~\cite{shi2020online,goel2019beyond,goel2019online}, because of (i) the $k$-step delay in the hitting cost $\frac{1}{2}(y_t-y_t^d)$ and (ii) the nonlinearity in the switching cost $\frac{1}{2}(y_t-y_{t-1}+g(y_{t-1}))^2$ with respective to $y_{t-1}$. However, in this paper, because we directly incorporate the effect of delay and nonlinearity in the algorithm design, our algorithms immediately provide constant-competitive policies for this setting.}
\end{example}


\subsection{Related Work}
This paper contributes to the growing literature on online convex optimization with memory.  
Initial results in this area focused on developing constant-competitive algorithms for the special case of 1-step memory, a.k.a., the Smoothed Online Convex Optimization (SOCO) problem, e.g., \citep{chen2018smoothed,goel2019beyond}. In that setting, \citep{chen2018smoothed} was the first to develop a constant, dimension-free competitive algorithm for high-dimensional problems.  The proposed algorithm, Online Balanced Descent (OBD), achieves a competitive ratio of $3+O(1/\beta)$ when cost functions are $\beta$-locally polyhedral.  This result was improved by \citep{goel2019beyond}, which proposed two new algorithms, Greedy OBD and Regularized OBD (ROBD), that both achieve $1+O(m^{-1/2})$ competitive ratios for $m$-strongly convex cost functions.  Recently, \citep{shi2020online} gave the first competitive analysis that holds beyond one step of memory.  It holds for a form of structured memory where the switching cost is linear:
$
    c(y_{t:t-p})=\frac{1}{2}\|y_t-\sum_{i=1}^pC_iy_{t-i}\|^2,
$
with known $C_i\in\mathbb{R}^{d\times d}$, $i=1,\cdots,p$. If the memory length $p = 1$ and $C_1$ is an identity matrix, this is equivalent to SOCO. In this setting, \citep{shi2020online} shows that ROBD has a competitive ratio of 
\begin{align}
    \frac{1}{2}\left( 1 + \frac{\alpha^2 - 1}{m} + \sqrt{\Big( 1 + \frac{\alpha^2 - 1}{m}\Big)^2 + \frac{4}{m}} \right),
\end{align}
when hitting costs are $m$-strongly convex and $\alpha=\sum_{i=1}^p\|C_i\|$. 


Prior to this paper, competitive algorithms for online optimization have nearly always assumed that the online learner acts \emph{after} observing the cost function in the current round, i.e., have zero delay.  The only exception is \citep{shi2020online}, which considered the case where the learner must act before observing the cost function, i.e., a one-step delay.  Even that small addition of delay requires a significant modification to the algorithm (from ROBD to Optimistic ROBD) and analysis compared to previous work. 

As the above highlights, there is no previous work that addresses either the setting of nonlinear switching costs nor the setting of multi-step delay. However, the prior work highlights that ROBD is a promising algorithmic framework and our work in this paper extends the ROBD framework in order to address the challenges of delay and non-linear switching costs. Given its importance to our work, we describe the workings of ROBD in detail in Algorithm~\ref{robd}. 

\begin{algorithm}[t!]
  \caption{ROBD \citep{goel2019beyond}}
  \label{robd}
\begin{algorithmic}[1]
  \STATE {\bfseries Parameter:} $\lambda_1\ge0,\lambda_2\ge0$
  \FOR{$t=1$ {\bfseries to} $T$}
  \STATE {\bfseries Input:} Hitting cost function $f_t$, previous decision points $y_{t-p:t-1}$
  \STATE $v_t\leftarrow\arg\min_yf_t(y)$
  \STATE $y_t\leftarrow\arg\min_yf_t(y)+\lambda_1c(y,y_{t-1:t-p})+\frac{\lambda_2}{2}\|y-v_t\|^2_2$
  \STATE {\bfseries Output:} $y_t$
  \ENDFOR
   
\end{algorithmic}
\end{algorithm}

Another line of literature that this paper contributes to is the growing understanding of the connection between online optimization and adaptive control. The reduction from adaptive control to online optimization with memory was first studied in \citep{agarwal2019online} to obtain a sublinear static regret guarantee against the best linear state-feedback controller, where the approach is to consider a disturbance-action policy class with some fixed horizon.  Many follow-up works adopt similar reduction techniques \citep{agarwal2019logarithmic, brukhim2020online, gradu2020adaptive}. A different reduction approach using control canonical form is proposed by \citep{li2019online} and further exploited by \citep{shi2020online}. Our work falls into this category.  The most general results so far focus on Input-Disturbed Squared Regulators, which can be reduced to online convex optimization with structured memory (without delay or nonlinear switching costs).  As we show in \Cref{Control}, the addition of delay and nonlinear switching costs leads to a significant extension of the generality of control models that can be reduced to online optimization. 

\section{Experimental protocols}\label{sec:protocols}
\subsection{Datasets}

We use the 
MovieLens--20M\footnote{http://grouplens.org/datasets/movielens}
and Netflix\footnote{http://www.netflixprize.com} explicit feedback  datasets.
As both  of these contain explicit ratings, we create binary preferences
by keeping ratings $\ge\!4$, which we interpret as positive feedback ($p_{ui}=1$).
Furthermore, we only keep users with at least 5 views.
Validation and test sets are obtained randomly, selecting a $10~\%$ of the original dataset for each set. We denote such datasets \textsc{ML20M} and \textsc{Netflix}.

In addition, we explore models performance on the Last.fm\footnote{https://www.upf.edu/web/mtg/lastfm360k} dataset~\cite{Celma:Springer2010}, an implicit feedback dataset consisting of tuples (user, artist, plays), that contains top artists by user. In order to make the comparison with the above datasets more straightforward, we binarize play counts and interpret them as implicit preference data. Next, we filter out artist with less than 50 distinct listening users, and user with less than 20 artists in their listening history. In the following, we name this dataset \textsc{Lastfm}.

\setlength{\belowcaptionskip}{5pt}
\begin{table}[htb]
\begin{tabular}{c c c c c}
 Dataset & \#users & \#items & \#pairs & \#pairs$_{\rm test}$ \\
\hline
\textsc{ML20M} & 136,7k & 20,3k & 7,99M & 1,0M \\
\textsc{Netflix}  & 463,4k & 17,7k & 45,5M & 5,7M \\
\textsc{Lastfm}  & 350,2k & 24,6k & 12,8M & 1,6M \\
\hline
\end{tabular}
\caption{Statistics of the datasets after preprocessing.}
\label{table:datasets}
\end{table}

The statistics of the training set after such  processing, as well as the number of user-item interactions in test, are presented in Table~\ref{table:datasets}. 
%We also represent the item popularity distribution in training, validation and test sets in Figure~\ref{fig:ditribution_dataset} for the \textsc{ML20M},  \textsc{Netflix}, \textsc{Lastfm} datasets, from top to bottom, respectively. As expected, validation and test distributions  follow the distribution of the training set. 

% \begin{figure}
%     \centering
%     \includegraphics[width=.7\linewidth]{figures/distributions_dataset.png}
%     \caption{Normalized distributions of user-item interactions in train (green), validation (red) and test (blue) sets for the  \textsc{ML20M},  \textsc{Netflix} and \textsc{Last-fm} datasets (from top to bottom, respectively).}
%     \label{fig:ditribution_dataset}
% \end{figure}

\subsection{Evaluation metrics}\label{subsec:metrics}
\setlength{\belowcaptionskip}{-10pt}
Given the set of adopted items in test, $\mathcal{I}_u^{\rm t}$, and the ranked list of predicted preferences, 
the relevance of a recommendation at position $k$ is given by ${\rm rel}_{ui}(k)$--${\rm rel}(k)$ from here on--, which equals $1$ if user $u$ adopted item $i$ in the test set, $0$ otherwise. In the calculation of metrics, we remove items observed in  training and validation from the list of recommendations. 
Next, we detail the metrics used for model evaluation.

\paragraph{Recall} It does not account for the relative ordering of the recommendation, and we defined it as~\cite{liang:2018:VAE}
\begin{equation}
{\rm Recall}@k = \frac{\sum_{s=1}^k {\rm rel}(s)}
{\mathcal{N}_u(k)}.
\end{equation}
Here, $\mathcal{N}_u(k) = \min\left(k,|\mathcal{I}_u^{\rm t}|\right)$, with $|\mathcal{I}_u^{\rm t}|$ the number of items adopted by user $u$ in testing. The final recall is averaged across all users in testing. 

\paragraph{Normalized Discount Cumulative Gain} In contrast to recall metric, the Discount Cumulative Gain (DCG) performs a logarithmic discount according to the position of a recommendation, that is
\begin{equation}
{\rm DCG}@k = \sum_{s=1}^k \frac{{\rm rel}(s)}
{\log_2(s+1)}.
\end{equation}
This quantity can be normalized by the Ideal DCG, 
\begin{equation}
{\rm IDCG}@k = \sum_{s=1}^{\mathcal{N}_u(k)} \frac{1}{\log_2(s+1)}.
\end{equation}
Finally, NDCG$@k= {\rm DCG}@k/{\rm IDCG}@k$, which we average across all users in the test set.

\paragraph{Novelty} Following reference~\cite{Vargas:2011:Novelty_diversity}, we define a novelty-weighted DCG score as
\begin{equation}\label{eq:nov-ndcg}
{\rm Nov\!-\!DCG}@k = 
-\sum_{s=1}^k \frac{{\rm rel}(s)\times \ln{\nu(i)}}
{\log_2(s+1)}.
\end{equation}
Here, $\nu(i)$ is the frequency of occurrences of item $i$  normalized to the total interactions in training. The corresponding novelty-weighted IDCG would be
\begin{equation}
{\rm Nov\!-\!IDCG}@k = 
\sum_{s=1}^{\mathcal{N}_u(k)} 
\frac{\max_{i\in\mathcal{I}_u}\left(-\ln \nu(i)\right)}
{\log_2(s+1)}.
\end{equation}
In other words, the highest DCG is obtained by ranking the most novel items (among those relevant to the user) in descending order. 
\subsection{Implementation details}\label{subsec:implementation}
The implementation of our model is performed in TensorFlow~\cite{tensorflow2015-whitepaper}.
%\footnote{The code will be available online at
%\url{https://github.com/bbvadata/RecApp}.
%}. 
The model can be trained in both CPU or GPU. 
When GPU is enabled, the use of queues to feed the tensors greatly speeds up the training.
% batch size and epoch
We set the batch size to $100$, and train every DAE model for $120$k iterations, so as to ensure proper convergence. For MF models we use $180$k iterations.
% number of neurons
The number of neurons is $200$ in all DAE experiments; for MF models, since the large number of users makes them prone to overfit, we train the models with $100$ and $200$ neurons and take the best performing model.
% weights and biases initialization
Weight matrices are initialized with random uniform values whose amplitude is computed as described by Glorot~\emph{et al.}~\cite{Xavier_initialization}. 
%Generally speaking, we discourage the use of random normal initialization without truncation.
For the biases we use a truncated random normal initialization with a standard deviation of $10^{-3}$. 
% Optimizer
Models are trained with Adam optimizer~\cite{Kingma2014AdamAM} and a learning rate of $10^{-3}$.
% gradient clip
%Additionally, we clip gradients whenever they exceed a norm of $1$, and apply batch normalization~\cite{icml2015_ioffe15_batch-norm} at every training step.

% Size of T and P sets
Concerning negative sampling in point and pair--wise schemes, 
%since we train with a batch size different form one, 
we fix the size of the target sets for every user (sets $\mathcal{T}_u$ and $\mathcal{P}_u$ for point and pair--wise learning, respectively, see subsection~\ref{subsec:losses}).
In particular, we make such sets proportional to the median number of items adopted by users, except for the multinomial loss, where all items are utilized~\cite{Liang:2016:CoFactor}.
The proportionality factors are hyper-parameters fine--tuned with the validation set, swapping the values $\{1,\,5,\,10,\,50,\,100,\,150\}$. We find a factor of $50$ or $100$ to provide the best results.
%(see additional comments in subsection~\ref{subsec:baseline_models}).

% Max norm regularization
%Regarding max--norm regularization, we notice that an asymmetric max--norm regularization might be required for the encoder/decoder weight matrices, depending on the activations and objective functions utilized for modeling. We swap $\alpha_{\rm enc}$ and $\alpha_{\rm dec}$ values in $\{0.0, 0.05, 0.1, 0.2, 0.3, 0.5, 1.0\}$, where $\alpha=0.0$ means no regularization.

% Regularization: noise and weight-decay
We add noise to the input vector of the AE~\cite{Vincent:2008:ECRF-AE, Wu:2016:CDAE-topN} using drop-out~\cite{liang:2018:VAE}. We fix the level of noise at $0.5$. Competitive performance is achieved after normalizing the AE input vector.
% Regularization of MIL is smaller
For DAE models, we swap the $L_2$ regularization strength $\lambda\in[10^{-7}-10^{-4}]$, while for MF models we take the form in equation~(\ref{eq:l2_reg_scaled}) with $\lambda\in[10^{0}-10^{3}]$, which provides a more stable training for MF models\footnote{Recall the different scales of the $\lambda$ factor in equations (\ref{eq:L2_reg}) and (\ref{eq:l2_reg_scaled}).}. In general we find that MIL models require smaller $\lambda$ factors than cross-entropy or multinomial--based models. This is expected, as the level of weight--decay regularization in equations~(\ref{eq:L2_reg}) and (\ref{eq:l2_reg_scaled}) depends on the value of the loss, which is smaller for MIL models.

\subsection{Baseline models}\label{subsec:baseline_models}
We implement the objective functions described in subsection \ref{subsec:losses} on a user-based DAE~\cite{Sedhain:2015:Autorec, Wu:2016:CDAE-topN} and compare the results with the MIL function. We also compare them with traditional Matrix Factorization with Weight Regularization~\cite{HuKoren:2008:CF_implicit}. In the following, we provide details on the training of the different models.

\textbf{Weight-Regularized Matrix Factorization} WRMF~\cite{HuKoren:2008:CF_implicit} is a linear factorization model  trained with square loss and weight decay. We use negative sampling with a sampling ratio of $100$ and $\lambda\sim 5-10$ (as obtained in the validation set). We call this model \MFsquare. In addition, we train WRMF models with MIL and point--wise cross--entropy losses, applying a sigmoid function at the output, so as to ensure $\hat{p}_{ui}\in(0,1)$. In these cases, we find that a sampling ratio of $100$ and $\lambda=50-500$ provide best results. We name these models \MFmil\, and \MFce, respectively.

\textbf{\emph{Denoising Autoencoder models}}

\textbf{Cross-entropy loss} For the cross-entropy loss defined in equations~(\ref{eq:cross-entropy}), (\ref{eq:point-wise}) and (\ref{eq:pair-wise}), we use linear--sigmoid and sigmoid--sigmoid activations at the encoder and decoder, respectively. We name the DAEs models with cross-entropy loss and point--wise estimation \CEpointlinsig\,and \CEpointsigsig; and those with pair--wise,  \CEpairlinsig \,and \CEpairsigsig. In order to prevent numerical instabilities, we ensure that the output preferences are in $[\varepsilon, 1-\varepsilon]$, with $\varepsilon=10^{-5}$. Regarding negative sampling, we find that the best sampling ratio is $50\times {\rm median}(\mathcal{I}_u)$ and $100\times {\rm median}(\mathcal{I}_u)$  for point and pair--wise estimation, respectively. Best weight-decay regularization is found to be $\lambda=2\cdot 10^{-5}$.

% Comparison to CDAE
The closest model to these baselines is the Collaborative Denoising AE (CDAE)~\cite{Wu:2016:CDAE-topN}, although for the sake of simplicity, in the present paper we do not include the user embedding of CDAE. 
% bad performance of BPR
Similar to CDAE, we find that  pair--wise learning does not achieve competitive results at the top of the ranked list~\cite{Wu:2016:CDAE-topN, liang:2018:VAE}. 

\textbf{Multinomial loss} AEs trained with a multinomial log-likelihood have recently been  introduced by Lian et al~\cite{liang:2018:VAE}, either applied to DAEs or Variational AEs (VAE) with partial regularization. Here, we focus on the multi-DAE modeling with $\tanh$-linear activations\footnote{
We use the actual implementation provided at \url{https://github.com/dawenl/vae_cf} to verify that the activation used at the decoder of multi-DAE is linear, although the original writing~\cite{liang:2018:VAE} suggests a $\tanh$ non-linearity for the decoder.
}, and name this baseline \MULTItanhlin. Our implementation exactly reproduces that of~\cite{liang:2018:VAE} when using $\lambda=2\cdot 10^{-5}$, input noise of $0.5$ and without applying negative sampling. 

\textbf{Missing Information loss} We apply the \textsc{MIL} function defined in equation~(\ref{eq:mil_def}) to linear-sigmoid and sigmoid-sigmoid DAEs. We name these models \MILlinsig\, and \MILsigsig, respectively. Best hyper-parameters of the loss turn out to be $A_{\rm MI}=10^6,\, \gamma_{\rm MI}=10$ and $\gamma_{+}=1$, after grid search the pairs $\left(A_{\rm MI}, \gamma_{\rm MI}\right)\in\{
(5\cdot10^1, 2)$, $(10^3, 4)$, $(2\cdot10^4, 6)$, $(5\cdot10^5, 10)$, $(1\cdot10^6, 10)$, $(5\cdot10^6, 10)$, $(5\cdot10^9, 15)\}$, and $\gamma_{+}=1$ or $2$. In addition, we use a sampling ratio of $50$ and $\lambda\in(10^{-6}, 10^{-5})$.

%!TEX ROOT = ../../centralized_vs_distributed.tex

\section{{\titlecap{the centralized-distributed trade-off}}}\label{sec:numerical-results}

\revision{In the previous sections we formulated the optimal control problem for a given controller architecture
(\ie the number of links) parametrized by $ n $
and showed how to compute minimum-variance objective function and the corresponding constraints.
In this section, we present our main result:
%\red{for a ring topology with multiple options for the parameter $ n $},
we solve the optimal control problem for each $ n $ and compare the best achievable closed-loop performance with different control architectures.\footnote{
\revision{Recall that small (large) values of $ n $ mean sparse (dense) architectures.}}
For delays that increase linearly with $n$,
\ie $ f(n) \propto n $, 
we demonstrate that distributed controllers with} {few communication links outperform controllers with larger number of communication links.}

\textcolor{subsectioncolor}{Figure~\ref{fig:cont-time-single-int-opt-var}} shows the steady-state variances
obtained with single-integrator dynamics~\eqref{eq:cont-time-single-int-variance-minimization}
%where we compare the standard multi-parameter design 
%with a simplified version \tcb{that utilizes spatially-constant feedback gains
and the quadratic approximation~\eqref{eq:quadratic-approximation} for \revision{ring topology}
with $ N = 50 $ nodes. % and $ n\in\{1,\dots,10\} $.
%with $ N = 50 $, $ f(n) = n $ and $ \tau_{\textit{min}} = 0.1 $.
%\autoref{fig:cont-time-single-int-err} shows the relative error, defined as
%\begin{equation}\label{eq:relative-error}
%	e \doteq \dfrac{\optvarx-\optvar}{\optvar}
%\end{equation}
%where $ \optvar $ and $ \optvarx $ denote the the optimal and sub-optimal scalar variances, respectively.
%The performance gap is small
%and becomes negligible for large $ n $.
{The best performance is achieved for a sparse architecture with  $ n = 2 $ 
in which each agent communicates with the two closest pairs of neighboring nodes. 
This should be compared and contrasted to nearest-neighbor and all-to-all 
communication topologies which induce higher closed-loop variances. 
Thus, 
the advantage of introducing additional communication links diminishes 
beyond}
{a certain threshold because of communication delays.}

%For a linear increase in the delay,
\textcolor{subsectioncolor}{Figure~\ref{fig:cont-time-double-int-opt-var}} shows that the use of approximation~\eqref{eq:cont-time-double-int-min-var-simplified} with $ \tilde{\gvel}^* = 70 $
identifies nearest-neighbor information exchange as the {near-optimal} architecture for a double-integrator model
with ring topology. 
This can be explained by noting that the variance of the process noise $ n(t) $
in the reduced model~\eqref{eq:x-dynamics-1st-order-approximation}
is proportional to $ \nicefrac{1}{\gvel} $ and thereby to $ \taun $,
according to~\eqref{eq:substitutions-4-normalization},
making the variance scale with the delay.

%\mjmargin{i feel that we need to comment about different results that we obtained for CT and DT double-intergrator dynamics (monotonic deterioration of performance for the former and oscillations for the latter)}
\revision{\textcolor{subsectioncolor}{Figures~\ref{fig:disc-time-single-int-opt-var}--\ref{fig:disc-time-double-int-opt-var}}
show the results obtained by solving the optimal control problem for discrete-time dynamics.
%which exhibit similar trade-offs.
The oscillations about the minimum in~\autoref{fig:disc-time-double-int-opt-var}
are compatible with the investigated \tradeoff~\eqref{eq:trade-off}:
in general, 
the sum of two monotone functions does not have a unique local minimum.
Details about discrete-time systems are deferred to~\autoref{sec:disc-time}.
Interestingly,
double integrators with continuous- (\autoref{fig:cont-time-double-int-opt-var}) ad discrete-time (\autoref{fig:disc-time-double-int-opt-var}) dynamics
exhibits very different trade-off curves,
whereby performance monotonically deteriorates for the former and oscillates for the latter.
While a clear interpretation is difficult because there is no explicit expression of the variance as a function of $ n $,
one possible explanation might be the first-order approximation used to compute gains in the continuous-time case.
%which reinforce our thesis exposed in~\autoref{sec:contribution}.

%\begin{figure}
%	\centering
%	\includegraphics[width=.6\linewidth]{cont-time-double-int-opt-var-n}
%	\caption{Steady-state scalar variance for continuous-time double integrators with $ \taun = 0.1n $.
%		Here, the \tradeoff is optimized by nearest-neighbor interaction.
%	}
%	\label{fig:cont-time-double-int-opt-var-lin}
%\end{figure}
}

\begin{figure}
	\centering
	\begin{minipage}[l]{.5\linewidth}
		\centering
		\includegraphics[width=\linewidth]{random-graph}
	\end{minipage}%
	\begin{minipage}[r]{.5\linewidth}
		\centering
		\includegraphics[width=\linewidth]{disc-time-single-int-random-graph-opt-var}
	\end{minipage}
	\caption{Network topology and its optimal {closed-loop} variance.}
	\label{fig:general-graph}
\end{figure}

Finally,
\autoref{fig:general-graph} shows the optimization results for a random graph topology with discrete-time single integrator agents. % with a linear increase in the delay, $ \taun = n $.
Here, $ n $ denotes the number of communication hops in the ``original" network, shown in~\autoref{fig:general-graph}:
as $ n $ increases, each agent can first communicate with its nearest neighbors,
then with its neighbors' neighbors, and so on. For a control architecture that utilizes different feedback gains for each communication link
	(\ie we only require $ K = K^\top $) we demonstrate that, in this case, two communication hops provide optimal closed-loop performance. % of the system.}

Additional computational experiments performed with different rates $ f(\cdot) $ show that the optimal number of links increases for slower rates: 
for example, 
the optimal number of links is larger for $ f(n) = \sqrt{n} $ than for $ f(n) = n $. 
\revision{These results are not reported because of space limitations.}

\section{Conclusions and next steps}\label{sec:conclusions}

% definition of MIL
In this paper we present a novel objective function, the \emph{Missing Information Loss} (MIL), specifically designed for handling unobserved user-item interactions in implicit feedback datasets. In particular, MIL explicitly forbids treating missing user-item interactions as positive or negative feedback.
% What it does
We demonstrate that, thanks to the functional form of the MIL function, the ranking of unseen items is almost entirely left to the low--rank process, rather than forcing unobserved items to be at the tail of the recommendation (\emph{i.e.}, MIL does not force a zero predicted preference for unobserved user-item interactions). 

% Metric results
Extensive experiments with Matrix Factorization and Denoising Autoencoders conducted on three datasets, show that \textsc{MIL} models demonstrate competitive performance when compared with other traditional losses such as cross-entropy or the multinomial log-likelihood. 
% Best performing models 
% Analysis of recommendations
In addition, we study the distribution of the recommendations and observe that the reported metric performance takes place while recommending popular items less frequently (up to a $20 \%$ decrease with respect to the best competing method). Indeed, \textsc{MIL} models sharply increase the recommendation of medium--tail items, while almost linearly expanding the appearance of long--tail items with the ranking position in the list of recommendations. Such expansion results in up to a $50 \%$ increase of long--tail recommendations, a feature of utmost importance for industries with a large catalogue of items. 

% Future work
Future lines of research may involve the incorporation of negative feedback, or the usage of \textsc{MIL} in temporal--aware Recommender Systems (such as those using Recurrent Neural Networks).  
In addition, we hope that the results here reported  will bring forward first-principle mathematical derivations of the \textsc{MIL} function, so that the vast family of possible polynomials modelling the missing information term can be reduced, or even extended with more suitable functions. 

\begin{acks}
We would like to thank the continuous support and careful reading of the manuscript by the \emph{Edge} guild within BBVA Data \& Analytics, specially J. Garc\'ia Santamar\'ia and J. A. Rodr\'iguez Serrano. 
\end{acks}

%\bibliographystyle{ACM-Reference-Format}
\bibliographystyle{unsrt}
%\bibliography{sigproc} 

\documentclass[sigconf]{acmart}

\settopmatter{printacmref=false} % Removes citation information below abstract
\renewcommand\footnotetextcopyrightpermission[1]{} % removes footnote with conference information in first column
\pagestyle{plain} % removes running headers

\usepackage{booktabs} % For formal tables
\usepackage{multirow}
\usepackage{amsmath}
\usepackage{color}
\usepackage{arydshln }


%Conference
\acmConference[WSDM]{The Twelfth International Conference on Web Search and Data Mining}{February 11--15}{Melbourne, Australia} 
\acmYear{2019}

\begin{document}
\title[Missing Information Loss]{A Missing Information Loss for implicit feedback datasets}

\author{Juan Ar\'evalo}
\affiliation{%
  \institution{BBVA Data \& Analytics}
}
\email{juanmaria.arevalo@bbvadata.com}

\author{Juan Ram\'on Duque}
\affiliation{%
  \institution{BBVA Data \& Analytics}
}
\email{juanramon.duque@bbvadata.com}

\author{Marco Creatura}
\affiliation{%
  \institution{BBVA Data \& Analytics}
}
\email{marco.creatura@bbvadata.com}

% The default list of authors is too long for headers}
\renewcommand{\shortauthors}{Ar\'evalo, Duque and Creatura}

% Some useful commands
\newcommand{\MFsquare}{\textsc{MF-square}}
\newcommand{\MFmil}{\textsc{MF-mil}}
\newcommand{\MFce}{\textsc{MF-CE}}
\newcommand{\CEpointlinsig}{\textsc{CE$_{\rm Point}$ lin-sig}}
\newcommand{\CEpointsigsig}{\textsc{CE$_{\rm Point}$ sig-sig}}
\newcommand{\CEpairlinsig}{\textsc{CE$_{\rm Pair}$ lin-sig}}
\newcommand{\CEpairsigsig}{\textsc{CE$_{\rm Pair}$ sig-sig}}
\newcommand{\MULTItanhlin}{\textsc{MULTI tanh-lin}}
\newcommand{\MILlinsig}{\textsc{MIL lin-sig}}
\newcommand{\MILsigsig}{\textsc{MIL sig-sig}}


\begin{abstract}
% missing values and negative feedback
Latent factor models for Recommender Systems with implicit feedback typically treat unobserved user-item interactions (\emph{i.e.} missing information) as negative feedback. This is frequently done 
either through negative sampling (point--wise loss) or with a ranking loss function (pair-- or list--wise estimation). 
% Common objective functions allow zero prediction
Since a zero preference recommendation is a valid solution for most common objective functions, 
regarding unknown values as actual zeros results in users 
having a zero preference recommendation  for most of the available items. 

% MIL
In this paper we propose a novel objective function, the \emph{Missing Information Loss} (MIL), 
that explicitly forbids treating unobserved user-item interactions as positive or negative feedback. 
% application to AE and metrics
We apply this loss to both traditional Matrix Factorization and user--based Denoising Autoencoder, and compare it with other established objective functions such as cross--entropy (both point-- and pair--wise) or the recently proposed multinomial log-likelihood. MIL achieves competitive performance in ranking--aware metrics when applied to three datasets.
% towards long-tail recommendations
Furthermore, we show that such a relevance in the recommendation is obtained while displaying popular items less frequently (up to a $20 \%$ decrease with respect to the best competing method). This debiasing from the recommendation of popular items favours the appearance of infrequent items (up to a $50 \%$ increase of long--tail recommendations), a valuable feature for Recommender Systems with a large catalogue of products. 
\end{abstract}

\begin{CCSXML}
<ccs2012>
<concept>
<concept_id>10002951.10003317.10003347.10003350</concept_id>
<concept_desc>Information systems~Recommender systems</concept_desc>
<concept_significance>500</concept_significance>
</concept>
</ccs2012>
\end{CCSXML}

\ccsdesc[500]{Information systems~Recommender systems}

\keywords{Collaborative Filtering, Autoencoders, Implicit Feedback, Missing Information}

\setcopyright{None}

\maketitle

\setlength{\abovecaptionskip}{0pt}
\setlength{\belowcaptionskip}{-10pt}

% Put all the sections with inputs

% !TEX root = ../arxiv.tex

Unsupervised domain adaptation (UDA) is a variant of semi-supervised learning \cite{blum1998combining}, where the available unlabelled data comes from a different distribution than the annotated dataset \cite{Ben-DavidBCP06}.
A case in point is to exploit synthetic data, where annotation is more accessible compared to the costly labelling of real-world images \cite{RichterVRK16,RosSMVL16}.
Along with some success in addressing UDA for semantic segmentation \cite{TsaiHSS0C18,VuJBCP19,0001S20,ZouYKW18}, the developed methods are growing increasingly sophisticated and often combine style transfer networks, adversarial training or network ensembles \cite{KimB20a,LiYV19,TsaiSSC19,Yang_2020_ECCV}.
This increase in model complexity impedes reproducibility, potentially slowing further progress.

In this work, we propose a UDA framework reaching state-of-the-art segmentation accuracy (measured by the Intersection-over-Union, IoU) without incurring substantial training efforts.
Toward this goal, we adopt a simple semi-supervised approach, \emph{self-training} \cite{ChenWB11,lee2013pseudo,ZouYKW18}, used in recent works only in conjunction with adversarial training or network ensembles \cite{ChoiKK19,KimB20a,Mei_2020_ECCV,Wang_2020_ECCV,0001S20,Zheng_2020_IJCV,ZhengY20}.
By contrast, we use self-training \emph{standalone}.
Compared to previous self-training methods \cite{ChenLCCCZAS20,Li_2020_ECCV,subhani2020learning,ZouYKW18,ZouYLKW19}, our approach also sidesteps the inconvenience of multiple training rounds, as they often require expert intervention between consecutive rounds.
We train our model using co-evolving pseudo labels end-to-end without such need.

\begin{figure}[t]%
    \centering
    \def\svgwidth{\linewidth}
    \input{figures/preview/bars.pdf_tex}
    \caption{\textbf{Results preview.} Unlike much recent work that combines multiple training paradigms, such as adversarial training and style transfer, our approach retains the modest single-round training complexity of self-training, yet improves the state of the art for adapting semantic segmentation by a significant margin.}
    \label{fig:preview}
\end{figure}

Our method leverages the ubiquitous \emph{data augmentation} techniques from fully supervised learning \cite{deeplabv3plus2018,ZhaoSQWJ17}: photometric jitter, flipping and multi-scale cropping.
We enforce \emph{consistency} of the semantic maps produced by the model across these image perturbations.
The following assumption formalises the key premise:

\myparagraph{Assumption 1.}
Let $f: \mathcal{I} \rightarrow \mathcal{M}$ represent a pixelwise mapping from images $\mathcal{I}$ to semantic output $\mathcal{M}$.
Denote $\rho_{\bm{\epsilon}}: \mathcal{I} \rightarrow \mathcal{I}$ a photometric image transform and, similarly, $\tau_{\bm{\epsilon}'}: \mathcal{I} \rightarrow \mathcal{I}$ a spatial similarity transformation, where $\bm{\epsilon},\bm{\epsilon}'\sim p(\cdot)$ are control variables following some pre-defined density (\eg, $p \equiv \mathcal{N}(0, 1)$).
Then, for any image $I \in \mathcal{I}$, $f$ is \emph{invariant} under $\rho_{\bm{\epsilon}}$ and \emph{equivariant} under $\tau_{\bm{\epsilon}'}$, \ie~$f(\rho_{\bm{\epsilon}}(I)) = f(I)$ and $f(\tau_{\bm{\epsilon}'}(I)) = \tau_{\bm{\epsilon}'}(f(I))$.

\smallskip
\noindent Next, we introduce a training framework using a \emph{momentum network} -- a slowly advancing copy of the original model.
The momentum network provides stable, yet recent targets for model updates, as opposed to the fixed supervision in model distillation \cite{Chen0G18,Zheng_2020_IJCV,ZhengY20}.
We also re-visit the problem of long-tail recognition in the context of generating pseudo labels for self-supervision.
In particular, we maintain an \emph{exponentially moving class prior} used to discount the confidence thresholds for those classes with few samples and increase their relative contribution to the training loss.
Our framework is simple to train, adds moderate computational overhead compared to a fully supervised setup, yet sets a new state of the art on established benchmarks (\cf \cref{fig:preview}).


Online convex optimization with memory has emerged as an important and challenging area with a wide array of applications, see \citep{lin2012online,anava2015online,chen2018smoothed,goel2019beyond,agarwal2019online,bubeck2019competitively} and the references therein.  Many results in this area have focused on the case of online optimization with switching costs (movement costs), a form of one-step memory, e.g., \citep{chen2018smoothed,goel2019beyond,bubeck2019competitively}, though some papers have focused on more general forms of memory, e.g., \citep{anava2015online,agarwal2019online}. In this paper we, for the first time, study the impact of feedback delay and nonlinear switching cost in online optimization with switching costs. 

An instance consists of a convex action set $\mathcal{K}\subset\mathbb{R}^d$, an initial point $y_0\in\mathcal{K}$, a sequence of non-negative convex cost functions $f_1,\cdots,f_T:\mathbb{R}^d\to\mathbb{R}_{\ge0}$, and a switching cost $c:\mathbb{R}^{d\times(p+1)}\to\mathbb{R}_{\ge0}$. To incorporate feedback delay, we consider a situation where the online learner only knows the geometry of the hitting cost function at each round, i.e., $f_t$, but that the minimizer of $f_t$ is revealed only after a delay of $k$ steps, i.e., at time $t+k$.  This captures practical scenarios where the form of the loss function or tracking function is known by the online learner, but the target moves over time and measurement lag means that the position of the target is not known until some time after an action must be taken. 
To incorporate nonlinear (and potentially nonconvex) switching costs, we consider the addition of a known nonlinear function $\delta$ from $\mathbb{R}^{d\times p}$ to $\mathbb{R}^d$ to the structured memory model introduced previously.  Specifically, we have
\begin{align}
c(y_{t:t-p}) = \frac{1}{2}\|y_t-\delta(y_{t-1:t-p})\|^2,    \label{e.newswitching}
\end{align}
where we use $y_{i:j}$ to denote either $\{y_i, y_{i+1}, \cdots, y_j\}$ if $i\leq j$, or  $\{y_i, y_{i-1}, \cdots, y_j\}$ if $i > j$ throughout the paper. Additionally, we use $\|\cdot\|$ to denote the 2-norm of a vector or the spectral norm of a matrix.

In summary, we consider an online agent that interacts with the environment as follows:
% \begin{inparaenum}[(i)] 
\begin{enumerate}%[leftmargin=*]
    \item The adversary reveals a function $h_t$, which is the geometry of the $t^\mathrm{th}$ hitting cost, and a point $v_{t-k}$, which is the minimizer of the $(t-k)^\mathrm{th}$ hitting cost. Assume that $h_t$ is $m$-strongly convex and $l$-strongly smooth, and that $\arg\min_y h_t(y)=0$.
    \item The online learner picks $y_t$ as its decision point at time step $t$ after observing $h_t,$  $v_{t-k}$.
    \item The adversary picks the minimizer of the hitting cost at time step $t$: $v_t$. 
    \item The learner pays hitting cost $f_t(y_t)=h_t(y_t-v_t)$ and switching cost $c(y_{t:t-p})$ of the form \eqref{e.newswitching}.
\end{enumerate}

The goal of the online learner is to minimize the total cost incurred over $T$ time steps, $cost(ALG)=\sum_{t=1}^Tf_t(y_t)+c(y_{t:t-p})$, with the goal of (nearly) matching the performance of the offline optimal algorithm with the optimal cost $cost(OPT)$. The performance metric used to evaluate an algorithm is typically the \textit{competitive ratio} because the goal is to learn in an environment that is changing dynamically and is potentially adversarial. Formally, the competitive ratio (CR) of the online algorithm is defined as the worst-case ratio between the total cost incurred by the online learner and the offline optimal cost: $CR(ALG)=\sup_{f_{1:T}}\frac{cost(ALG)}{cost(OPT)}$.

It is important to emphasize that the online learner decides $y_t$ based on the knowledge of the previous decisions $y_1\cdots y_{t-1}$, the geometry of cost functions $h_1\cdots h_t$, and the delayed feedback on the minimizer $v_1\cdots v_{t-k}$. Thus, the learner has perfect knowledge of cost functions $f_1\cdots f_{t-k}$, but incomplete knowledge of $f_{t-k+1}\cdots f_t$ (recall that $f_t(y)=h_t(y-v_t)$).

Both feedback delay and nonlinear switching cost add considerable difficulty for the online learner compared to versions of online optimization studied previously. Delay hides crucial information from the online learner and so makes adaptation to changes in the environment more challenging. As the learner makes decisions it is unaware of the true cost it is experiencing, and thus it is difficult to track the optimal solution. This is magnified by the fact that nonlinear switching costs increase the dependency of the variables on each other. It further stresses the influence of the delay, because an inaccurate estimation on the unknown data, potentially magnifying the mistakes of the learner. 

The impact of feedback delay has been studied previously in online learning settings without switching costs, with a focus on regret, e.g., \citep{joulani2013online,shamir2017online}.  However, in settings with switching costs the impact of delay is magnified since delay may lead to not only more hitting cost in individual rounds, but significantly larger switching costs since the arrival of delayed information may trigger a very large chance in action.  To the best of our knowledge, we give the first competitive ratio for delayed feedback in online optimization with switching costs. 

We illustrate a concrete example application of our setting in the following.

\begin{example}[Drone tracking problem]
\label{example:drone} \emph{
Consider a drone with vertical speed $y_t\in\mathbb{R}$. The goal of the drone is to track a sequence of desired speeds $y^d_1,\cdots,y^d_T$ with the following tracking cost:}
\begin{equation}
    \sum_{t=1}^T \frac{1}{2}(y_t-y^d_t)^2 + \frac{1}{2}(y_t-y_{t-1}+g(y_{t-1}))^2,
\end{equation}
\emph{where $g(y_{t-1})$ accounts for the gravity and the aerodynamic drag. One example is $g(y)=C_1+C_2\cdot|y|\cdot y$ where $C_1,C_2>0$ are two constants~\cite{shi2019neural}. Note that the desired speed $y_t^d$ is typically sent from a remote computer/server. Due to the communication delay, at time step $t$ the drone only knows $y_1^d,\cdots,y_{t-k}^d$.}

\emph{This example is beyond the scope of existing results in online optimization, e.g.,~\cite{shi2020online,goel2019beyond,goel2019online}, because of (i) the $k$-step delay in the hitting cost $\frac{1}{2}(y_t-y_t^d)$ and (ii) the nonlinearity in the switching cost $\frac{1}{2}(y_t-y_{t-1}+g(y_{t-1}))^2$ with respective to $y_{t-1}$. However, in this paper, because we directly incorporate the effect of delay and nonlinearity in the algorithm design, our algorithms immediately provide constant-competitive policies for this setting.}
\end{example}


\subsection{Related Work}
This paper contributes to the growing literature on online convex optimization with memory.  
Initial results in this area focused on developing constant-competitive algorithms for the special case of 1-step memory, a.k.a., the Smoothed Online Convex Optimization (SOCO) problem, e.g., \citep{chen2018smoothed,goel2019beyond}. In that setting, \citep{chen2018smoothed} was the first to develop a constant, dimension-free competitive algorithm for high-dimensional problems.  The proposed algorithm, Online Balanced Descent (OBD), achieves a competitive ratio of $3+O(1/\beta)$ when cost functions are $\beta$-locally polyhedral.  This result was improved by \citep{goel2019beyond}, which proposed two new algorithms, Greedy OBD and Regularized OBD (ROBD), that both achieve $1+O(m^{-1/2})$ competitive ratios for $m$-strongly convex cost functions.  Recently, \citep{shi2020online} gave the first competitive analysis that holds beyond one step of memory.  It holds for a form of structured memory where the switching cost is linear:
$
    c(y_{t:t-p})=\frac{1}{2}\|y_t-\sum_{i=1}^pC_iy_{t-i}\|^2,
$
with known $C_i\in\mathbb{R}^{d\times d}$, $i=1,\cdots,p$. If the memory length $p = 1$ and $C_1$ is an identity matrix, this is equivalent to SOCO. In this setting, \citep{shi2020online} shows that ROBD has a competitive ratio of 
\begin{align}
    \frac{1}{2}\left( 1 + \frac{\alpha^2 - 1}{m} + \sqrt{\Big( 1 + \frac{\alpha^2 - 1}{m}\Big)^2 + \frac{4}{m}} \right),
\end{align}
when hitting costs are $m$-strongly convex and $\alpha=\sum_{i=1}^p\|C_i\|$. 


Prior to this paper, competitive algorithms for online optimization have nearly always assumed that the online learner acts \emph{after} observing the cost function in the current round, i.e., have zero delay.  The only exception is \citep{shi2020online}, which considered the case where the learner must act before observing the cost function, i.e., a one-step delay.  Even that small addition of delay requires a significant modification to the algorithm (from ROBD to Optimistic ROBD) and analysis compared to previous work. 

As the above highlights, there is no previous work that addresses either the setting of nonlinear switching costs nor the setting of multi-step delay. However, the prior work highlights that ROBD is a promising algorithmic framework and our work in this paper extends the ROBD framework in order to address the challenges of delay and non-linear switching costs. Given its importance to our work, we describe the workings of ROBD in detail in Algorithm~\ref{robd}. 

\begin{algorithm}[t!]
  \caption{ROBD \citep{goel2019beyond}}
  \label{robd}
\begin{algorithmic}[1]
  \STATE {\bfseries Parameter:} $\lambda_1\ge0,\lambda_2\ge0$
  \FOR{$t=1$ {\bfseries to} $T$}
  \STATE {\bfseries Input:} Hitting cost function $f_t$, previous decision points $y_{t-p:t-1}$
  \STATE $v_t\leftarrow\arg\min_yf_t(y)$
  \STATE $y_t\leftarrow\arg\min_yf_t(y)+\lambda_1c(y,y_{t-1:t-p})+\frac{\lambda_2}{2}\|y-v_t\|^2_2$
  \STATE {\bfseries Output:} $y_t$
  \ENDFOR
   
\end{algorithmic}
\end{algorithm}

Another line of literature that this paper contributes to is the growing understanding of the connection between online optimization and adaptive control. The reduction from adaptive control to online optimization with memory was first studied in \citep{agarwal2019online} to obtain a sublinear static regret guarantee against the best linear state-feedback controller, where the approach is to consider a disturbance-action policy class with some fixed horizon.  Many follow-up works adopt similar reduction techniques \citep{agarwal2019logarithmic, brukhim2020online, gradu2020adaptive}. A different reduction approach using control canonical form is proposed by \citep{li2019online} and further exploited by \citep{shi2020online}. Our work falls into this category.  The most general results so far focus on Input-Disturbed Squared Regulators, which can be reduced to online convex optimization with structured memory (without delay or nonlinear switching costs).  As we show in \Cref{Control}, the addition of delay and nonlinear switching costs leads to a significant extension of the generality of control models that can be reduced to online optimization. 

\section{Experimental protocols}\label{sec:protocols}
\subsection{Datasets}

We use the 
MovieLens--20M\footnote{http://grouplens.org/datasets/movielens}
and Netflix\footnote{http://www.netflixprize.com} explicit feedback  datasets.
As both  of these contain explicit ratings, we create binary preferences
by keeping ratings $\ge\!4$, which we interpret as positive feedback ($p_{ui}=1$).
Furthermore, we only keep users with at least 5 views.
Validation and test sets are obtained randomly, selecting a $10~\%$ of the original dataset for each set. We denote such datasets \textsc{ML20M} and \textsc{Netflix}.

In addition, we explore models performance on the Last.fm\footnote{https://www.upf.edu/web/mtg/lastfm360k} dataset~\cite{Celma:Springer2010}, an implicit feedback dataset consisting of tuples (user, artist, plays), that contains top artists by user. In order to make the comparison with the above datasets more straightforward, we binarize play counts and interpret them as implicit preference data. Next, we filter out artist with less than 50 distinct listening users, and user with less than 20 artists in their listening history. In the following, we name this dataset \textsc{Lastfm}.

\setlength{\belowcaptionskip}{5pt}
\begin{table}[htb]
\begin{tabular}{c c c c c}
 Dataset & \#users & \#items & \#pairs & \#pairs$_{\rm test}$ \\
\hline
\textsc{ML20M} & 136,7k & 20,3k & 7,99M & 1,0M \\
\textsc{Netflix}  & 463,4k & 17,7k & 45,5M & 5,7M \\
\textsc{Lastfm}  & 350,2k & 24,6k & 12,8M & 1,6M \\
\hline
\end{tabular}
\caption{Statistics of the datasets after preprocessing.}
\label{table:datasets}
\end{table}

The statistics of the training set after such  processing, as well as the number of user-item interactions in test, are presented in Table~\ref{table:datasets}. 
%We also represent the item popularity distribution in training, validation and test sets in Figure~\ref{fig:ditribution_dataset} for the \textsc{ML20M},  \textsc{Netflix}, \textsc{Lastfm} datasets, from top to bottom, respectively. As expected, validation and test distributions  follow the distribution of the training set. 

% \begin{figure}
%     \centering
%     \includegraphics[width=.7\linewidth]{figures/distributions_dataset.png}
%     \caption{Normalized distributions of user-item interactions in train (green), validation (red) and test (blue) sets for the  \textsc{ML20M},  \textsc{Netflix} and \textsc{Last-fm} datasets (from top to bottom, respectively).}
%     \label{fig:ditribution_dataset}
% \end{figure}

\subsection{Evaluation metrics}\label{subsec:metrics}
\setlength{\belowcaptionskip}{-10pt}
Given the set of adopted items in test, $\mathcal{I}_u^{\rm t}$, and the ranked list of predicted preferences, 
the relevance of a recommendation at position $k$ is given by ${\rm rel}_{ui}(k)$--${\rm rel}(k)$ from here on--, which equals $1$ if user $u$ adopted item $i$ in the test set, $0$ otherwise. In the calculation of metrics, we remove items observed in  training and validation from the list of recommendations. 
Next, we detail the metrics used for model evaluation.

\paragraph{Recall} It does not account for the relative ordering of the recommendation, and we defined it as~\cite{liang:2018:VAE}
\begin{equation}
{\rm Recall}@k = \frac{\sum_{s=1}^k {\rm rel}(s)}
{\mathcal{N}_u(k)}.
\end{equation}
Here, $\mathcal{N}_u(k) = \min\left(k,|\mathcal{I}_u^{\rm t}|\right)$, with $|\mathcal{I}_u^{\rm t}|$ the number of items adopted by user $u$ in testing. The final recall is averaged across all users in testing. 

\paragraph{Normalized Discount Cumulative Gain} In contrast to recall metric, the Discount Cumulative Gain (DCG) performs a logarithmic discount according to the position of a recommendation, that is
\begin{equation}
{\rm DCG}@k = \sum_{s=1}^k \frac{{\rm rel}(s)}
{\log_2(s+1)}.
\end{equation}
This quantity can be normalized by the Ideal DCG, 
\begin{equation}
{\rm IDCG}@k = \sum_{s=1}^{\mathcal{N}_u(k)} \frac{1}{\log_2(s+1)}.
\end{equation}
Finally, NDCG$@k= {\rm DCG}@k/{\rm IDCG}@k$, which we average across all users in the test set.

\paragraph{Novelty} Following reference~\cite{Vargas:2011:Novelty_diversity}, we define a novelty-weighted DCG score as
\begin{equation}\label{eq:nov-ndcg}
{\rm Nov\!-\!DCG}@k = 
-\sum_{s=1}^k \frac{{\rm rel}(s)\times \ln{\nu(i)}}
{\log_2(s+1)}.
\end{equation}
Here, $\nu(i)$ is the frequency of occurrences of item $i$  normalized to the total interactions in training. The corresponding novelty-weighted IDCG would be
\begin{equation}
{\rm Nov\!-\!IDCG}@k = 
\sum_{s=1}^{\mathcal{N}_u(k)} 
\frac{\max_{i\in\mathcal{I}_u}\left(-\ln \nu(i)\right)}
{\log_2(s+1)}.
\end{equation}
In other words, the highest DCG is obtained by ranking the most novel items (among those relevant to the user) in descending order. 
\subsection{Implementation details}\label{subsec:implementation}
The implementation of our model is performed in TensorFlow~\cite{tensorflow2015-whitepaper}.
%\footnote{The code will be available online at
%\url{https://github.com/bbvadata/RecApp}.
%}. 
The model can be trained in both CPU or GPU. 
When GPU is enabled, the use of queues to feed the tensors greatly speeds up the training.
% batch size and epoch
We set the batch size to $100$, and train every DAE model for $120$k iterations, so as to ensure proper convergence. For MF models we use $180$k iterations.
% number of neurons
The number of neurons is $200$ in all DAE experiments; for MF models, since the large number of users makes them prone to overfit, we train the models with $100$ and $200$ neurons and take the best performing model.
% weights and biases initialization
Weight matrices are initialized with random uniform values whose amplitude is computed as described by Glorot~\emph{et al.}~\cite{Xavier_initialization}. 
%Generally speaking, we discourage the use of random normal initialization without truncation.
For the biases we use a truncated random normal initialization with a standard deviation of $10^{-3}$. 
% Optimizer
Models are trained with Adam optimizer~\cite{Kingma2014AdamAM} and a learning rate of $10^{-3}$.
% gradient clip
%Additionally, we clip gradients whenever they exceed a norm of $1$, and apply batch normalization~\cite{icml2015_ioffe15_batch-norm} at every training step.

% Size of T and P sets
Concerning negative sampling in point and pair--wise schemes, 
%since we train with a batch size different form one, 
we fix the size of the target sets for every user (sets $\mathcal{T}_u$ and $\mathcal{P}_u$ for point and pair--wise learning, respectively, see subsection~\ref{subsec:losses}).
In particular, we make such sets proportional to the median number of items adopted by users, except for the multinomial loss, where all items are utilized~\cite{Liang:2016:CoFactor}.
The proportionality factors are hyper-parameters fine--tuned with the validation set, swapping the values $\{1,\,5,\,10,\,50,\,100,\,150\}$. We find a factor of $50$ or $100$ to provide the best results.
%(see additional comments in subsection~\ref{subsec:baseline_models}).

% Max norm regularization
%Regarding max--norm regularization, we notice that an asymmetric max--norm regularization might be required for the encoder/decoder weight matrices, depending on the activations and objective functions utilized for modeling. We swap $\alpha_{\rm enc}$ and $\alpha_{\rm dec}$ values in $\{0.0, 0.05, 0.1, 0.2, 0.3, 0.5, 1.0\}$, where $\alpha=0.0$ means no regularization.

% Regularization: noise and weight-decay
We add noise to the input vector of the AE~\cite{Vincent:2008:ECRF-AE, Wu:2016:CDAE-topN} using drop-out~\cite{liang:2018:VAE}. We fix the level of noise at $0.5$. Competitive performance is achieved after normalizing the AE input vector.
% Regularization of MIL is smaller
For DAE models, we swap the $L_2$ regularization strength $\lambda\in[10^{-7}-10^{-4}]$, while for MF models we take the form in equation~(\ref{eq:l2_reg_scaled}) with $\lambda\in[10^{0}-10^{3}]$, which provides a more stable training for MF models\footnote{Recall the different scales of the $\lambda$ factor in equations (\ref{eq:L2_reg}) and (\ref{eq:l2_reg_scaled}).}. In general we find that MIL models require smaller $\lambda$ factors than cross-entropy or multinomial--based models. This is expected, as the level of weight--decay regularization in equations~(\ref{eq:L2_reg}) and (\ref{eq:l2_reg_scaled}) depends on the value of the loss, which is smaller for MIL models.

\subsection{Baseline models}\label{subsec:baseline_models}
We implement the objective functions described in subsection \ref{subsec:losses} on a user-based DAE~\cite{Sedhain:2015:Autorec, Wu:2016:CDAE-topN} and compare the results with the MIL function. We also compare them with traditional Matrix Factorization with Weight Regularization~\cite{HuKoren:2008:CF_implicit}. In the following, we provide details on the training of the different models.

\textbf{Weight-Regularized Matrix Factorization} WRMF~\cite{HuKoren:2008:CF_implicit} is a linear factorization model  trained with square loss and weight decay. We use negative sampling with a sampling ratio of $100$ and $\lambda\sim 5-10$ (as obtained in the validation set). We call this model \MFsquare. In addition, we train WRMF models with MIL and point--wise cross--entropy losses, applying a sigmoid function at the output, so as to ensure $\hat{p}_{ui}\in(0,1)$. In these cases, we find that a sampling ratio of $100$ and $\lambda=50-500$ provide best results. We name these models \MFmil\, and \MFce, respectively.

\textbf{\emph{Denoising Autoencoder models}}

\textbf{Cross-entropy loss} For the cross-entropy loss defined in equations~(\ref{eq:cross-entropy}), (\ref{eq:point-wise}) and (\ref{eq:pair-wise}), we use linear--sigmoid and sigmoid--sigmoid activations at the encoder and decoder, respectively. We name the DAEs models with cross-entropy loss and point--wise estimation \CEpointlinsig\,and \CEpointsigsig; and those with pair--wise,  \CEpairlinsig \,and \CEpairsigsig. In order to prevent numerical instabilities, we ensure that the output preferences are in $[\varepsilon, 1-\varepsilon]$, with $\varepsilon=10^{-5}$. Regarding negative sampling, we find that the best sampling ratio is $50\times {\rm median}(\mathcal{I}_u)$ and $100\times {\rm median}(\mathcal{I}_u)$  for point and pair--wise estimation, respectively. Best weight-decay regularization is found to be $\lambda=2\cdot 10^{-5}$.

% Comparison to CDAE
The closest model to these baselines is the Collaborative Denoising AE (CDAE)~\cite{Wu:2016:CDAE-topN}, although for the sake of simplicity, in the present paper we do not include the user embedding of CDAE. 
% bad performance of BPR
Similar to CDAE, we find that  pair--wise learning does not achieve competitive results at the top of the ranked list~\cite{Wu:2016:CDAE-topN, liang:2018:VAE}. 

\textbf{Multinomial loss} AEs trained with a multinomial log-likelihood have recently been  introduced by Lian et al~\cite{liang:2018:VAE}, either applied to DAEs or Variational AEs (VAE) with partial regularization. Here, we focus on the multi-DAE modeling with $\tanh$-linear activations\footnote{
We use the actual implementation provided at \url{https://github.com/dawenl/vae_cf} to verify that the activation used at the decoder of multi-DAE is linear, although the original writing~\cite{liang:2018:VAE} suggests a $\tanh$ non-linearity for the decoder.
}, and name this baseline \MULTItanhlin. Our implementation exactly reproduces that of~\cite{liang:2018:VAE} when using $\lambda=2\cdot 10^{-5}$, input noise of $0.5$ and without applying negative sampling. 

\textbf{Missing Information loss} We apply the \textsc{MIL} function defined in equation~(\ref{eq:mil_def}) to linear-sigmoid and sigmoid-sigmoid DAEs. We name these models \MILlinsig\, and \MILsigsig, respectively. Best hyper-parameters of the loss turn out to be $A_{\rm MI}=10^6,\, \gamma_{\rm MI}=10$ and $\gamma_{+}=1$, after grid search the pairs $\left(A_{\rm MI}, \gamma_{\rm MI}\right)\in\{
(5\cdot10^1, 2)$, $(10^3, 4)$, $(2\cdot10^4, 6)$, $(5\cdot10^5, 10)$, $(1\cdot10^6, 10)$, $(5\cdot10^6, 10)$, $(5\cdot10^9, 15)\}$, and $\gamma_{+}=1$ or $2$. In addition, we use a sampling ratio of $50$ and $\lambda\in(10^{-6}, 10^{-5})$.

%!TEX ROOT = ../../centralized_vs_distributed.tex

\section{{\titlecap{the centralized-distributed trade-off}}}\label{sec:numerical-results}

\revision{In the previous sections we formulated the optimal control problem for a given controller architecture
(\ie the number of links) parametrized by $ n $
and showed how to compute minimum-variance objective function and the corresponding constraints.
In this section, we present our main result:
%\red{for a ring topology with multiple options for the parameter $ n $},
we solve the optimal control problem for each $ n $ and compare the best achievable closed-loop performance with different control architectures.\footnote{
\revision{Recall that small (large) values of $ n $ mean sparse (dense) architectures.}}
For delays that increase linearly with $n$,
\ie $ f(n) \propto n $, 
we demonstrate that distributed controllers with} {few communication links outperform controllers with larger number of communication links.}

\textcolor{subsectioncolor}{Figure~\ref{fig:cont-time-single-int-opt-var}} shows the steady-state variances
obtained with single-integrator dynamics~\eqref{eq:cont-time-single-int-variance-minimization}
%where we compare the standard multi-parameter design 
%with a simplified version \tcb{that utilizes spatially-constant feedback gains
and the quadratic approximation~\eqref{eq:quadratic-approximation} for \revision{ring topology}
with $ N = 50 $ nodes. % and $ n\in\{1,\dots,10\} $.
%with $ N = 50 $, $ f(n) = n $ and $ \tau_{\textit{min}} = 0.1 $.
%\autoref{fig:cont-time-single-int-err} shows the relative error, defined as
%\begin{equation}\label{eq:relative-error}
%	e \doteq \dfrac{\optvarx-\optvar}{\optvar}
%\end{equation}
%where $ \optvar $ and $ \optvarx $ denote the the optimal and sub-optimal scalar variances, respectively.
%The performance gap is small
%and becomes negligible for large $ n $.
{The best performance is achieved for a sparse architecture with  $ n = 2 $ 
in which each agent communicates with the two closest pairs of neighboring nodes. 
This should be compared and contrasted to nearest-neighbor and all-to-all 
communication topologies which induce higher closed-loop variances. 
Thus, 
the advantage of introducing additional communication links diminishes 
beyond}
{a certain threshold because of communication delays.}

%For a linear increase in the delay,
\textcolor{subsectioncolor}{Figure~\ref{fig:cont-time-double-int-opt-var}} shows that the use of approximation~\eqref{eq:cont-time-double-int-min-var-simplified} with $ \tilde{\gvel}^* = 70 $
identifies nearest-neighbor information exchange as the {near-optimal} architecture for a double-integrator model
with ring topology. 
This can be explained by noting that the variance of the process noise $ n(t) $
in the reduced model~\eqref{eq:x-dynamics-1st-order-approximation}
is proportional to $ \nicefrac{1}{\gvel} $ and thereby to $ \taun $,
according to~\eqref{eq:substitutions-4-normalization},
making the variance scale with the delay.

%\mjmargin{i feel that we need to comment about different results that we obtained for CT and DT double-intergrator dynamics (monotonic deterioration of performance for the former and oscillations for the latter)}
\revision{\textcolor{subsectioncolor}{Figures~\ref{fig:disc-time-single-int-opt-var}--\ref{fig:disc-time-double-int-opt-var}}
show the results obtained by solving the optimal control problem for discrete-time dynamics.
%which exhibit similar trade-offs.
The oscillations about the minimum in~\autoref{fig:disc-time-double-int-opt-var}
are compatible with the investigated \tradeoff~\eqref{eq:trade-off}:
in general, 
the sum of two monotone functions does not have a unique local minimum.
Details about discrete-time systems are deferred to~\autoref{sec:disc-time}.
Interestingly,
double integrators with continuous- (\autoref{fig:cont-time-double-int-opt-var}) ad discrete-time (\autoref{fig:disc-time-double-int-opt-var}) dynamics
exhibits very different trade-off curves,
whereby performance monotonically deteriorates for the former and oscillates for the latter.
While a clear interpretation is difficult because there is no explicit expression of the variance as a function of $ n $,
one possible explanation might be the first-order approximation used to compute gains in the continuous-time case.
%which reinforce our thesis exposed in~\autoref{sec:contribution}.

%\begin{figure}
%	\centering
%	\includegraphics[width=.6\linewidth]{cont-time-double-int-opt-var-n}
%	\caption{Steady-state scalar variance for continuous-time double integrators with $ \taun = 0.1n $.
%		Here, the \tradeoff is optimized by nearest-neighbor interaction.
%	}
%	\label{fig:cont-time-double-int-opt-var-lin}
%\end{figure}
}

\begin{figure}
	\centering
	\begin{minipage}[l]{.5\linewidth}
		\centering
		\includegraphics[width=\linewidth]{random-graph}
	\end{minipage}%
	\begin{minipage}[r]{.5\linewidth}
		\centering
		\includegraphics[width=\linewidth]{disc-time-single-int-random-graph-opt-var}
	\end{minipage}
	\caption{Network topology and its optimal {closed-loop} variance.}
	\label{fig:general-graph}
\end{figure}

Finally,
\autoref{fig:general-graph} shows the optimization results for a random graph topology with discrete-time single integrator agents. % with a linear increase in the delay, $ \taun = n $.
Here, $ n $ denotes the number of communication hops in the ``original" network, shown in~\autoref{fig:general-graph}:
as $ n $ increases, each agent can first communicate with its nearest neighbors,
then with its neighbors' neighbors, and so on. For a control architecture that utilizes different feedback gains for each communication link
	(\ie we only require $ K = K^\top $) we demonstrate that, in this case, two communication hops provide optimal closed-loop performance. % of the system.}

Additional computational experiments performed with different rates $ f(\cdot) $ show that the optimal number of links increases for slower rates: 
for example, 
the optimal number of links is larger for $ f(n) = \sqrt{n} $ than for $ f(n) = n $. 
\revision{These results are not reported because of space limitations.}

\section{Conclusions and next steps}\label{sec:conclusions}

% definition of MIL
In this paper we present a novel objective function, the \emph{Missing Information Loss} (MIL), specifically designed for handling unobserved user-item interactions in implicit feedback datasets. In particular, MIL explicitly forbids treating missing user-item interactions as positive or negative feedback.
% What it does
We demonstrate that, thanks to the functional form of the MIL function, the ranking of unseen items is almost entirely left to the low--rank process, rather than forcing unobserved items to be at the tail of the recommendation (\emph{i.e.}, MIL does not force a zero predicted preference for unobserved user-item interactions). 

% Metric results
Extensive experiments with Matrix Factorization and Denoising Autoencoders conducted on three datasets, show that \textsc{MIL} models demonstrate competitive performance when compared with other traditional losses such as cross-entropy or the multinomial log-likelihood. 
% Best performing models 
% Analysis of recommendations
In addition, we study the distribution of the recommendations and observe that the reported metric performance takes place while recommending popular items less frequently (up to a $20 \%$ decrease with respect to the best competing method). Indeed, \textsc{MIL} models sharply increase the recommendation of medium--tail items, while almost linearly expanding the appearance of long--tail items with the ranking position in the list of recommendations. Such expansion results in up to a $50 \%$ increase of long--tail recommendations, a feature of utmost importance for industries with a large catalogue of items. 

% Future work
Future lines of research may involve the incorporation of negative feedback, or the usage of \textsc{MIL} in temporal--aware Recommender Systems (such as those using Recurrent Neural Networks).  
In addition, we hope that the results here reported  will bring forward first-principle mathematical derivations of the \textsc{MIL} function, so that the vast family of possible polynomials modelling the missing information term can be reduced, or even extended with more suitable functions. 

\begin{acks}
We would like to thank the continuous support and careful reading of the manuscript by the \emph{Edge} guild within BBVA Data \& Analytics, specially J. Garc\'ia Santamar\'ia and J. A. Rodr\'iguez Serrano. 
\end{acks}

%\bibliographystyle{ACM-Reference-Format}
\bibliographystyle{unsrt}
%\bibliography{sigproc} 

\documentclass[sigconf]{acmart}

\settopmatter{printacmref=false} % Removes citation information below abstract
\renewcommand\footnotetextcopyrightpermission[1]{} % removes footnote with conference information in first column
\pagestyle{plain} % removes running headers

\usepackage{booktabs} % For formal tables
\usepackage{multirow}
\usepackage{amsmath}
\usepackage{color}
\usepackage{arydshln }


%Conference
\acmConference[WSDM]{The Twelfth International Conference on Web Search and Data Mining}{February 11--15}{Melbourne, Australia} 
\acmYear{2019}

\begin{document}
\title[Missing Information Loss]{A Missing Information Loss for implicit feedback datasets}

\author{Juan Ar\'evalo}
\affiliation{%
  \institution{BBVA Data \& Analytics}
}
\email{juanmaria.arevalo@bbvadata.com}

\author{Juan Ram\'on Duque}
\affiliation{%
  \institution{BBVA Data \& Analytics}
}
\email{juanramon.duque@bbvadata.com}

\author{Marco Creatura}
\affiliation{%
  \institution{BBVA Data \& Analytics}
}
\email{marco.creatura@bbvadata.com}

% The default list of authors is too long for headers}
\renewcommand{\shortauthors}{Ar\'evalo, Duque and Creatura}

% Some useful commands
\newcommand{\MFsquare}{\textsc{MF-square}}
\newcommand{\MFmil}{\textsc{MF-mil}}
\newcommand{\MFce}{\textsc{MF-CE}}
\newcommand{\CEpointlinsig}{\textsc{CE$_{\rm Point}$ lin-sig}}
\newcommand{\CEpointsigsig}{\textsc{CE$_{\rm Point}$ sig-sig}}
\newcommand{\CEpairlinsig}{\textsc{CE$_{\rm Pair}$ lin-sig}}
\newcommand{\CEpairsigsig}{\textsc{CE$_{\rm Pair}$ sig-sig}}
\newcommand{\MULTItanhlin}{\textsc{MULTI tanh-lin}}
\newcommand{\MILlinsig}{\textsc{MIL lin-sig}}
\newcommand{\MILsigsig}{\textsc{MIL sig-sig}}


\begin{abstract}
% missing values and negative feedback
Latent factor models for Recommender Systems with implicit feedback typically treat unobserved user-item interactions (\emph{i.e.} missing information) as negative feedback. This is frequently done 
either through negative sampling (point--wise loss) or with a ranking loss function (pair-- or list--wise estimation). 
% Common objective functions allow zero prediction
Since a zero preference recommendation is a valid solution for most common objective functions, 
regarding unknown values as actual zeros results in users 
having a zero preference recommendation  for most of the available items. 

% MIL
In this paper we propose a novel objective function, the \emph{Missing Information Loss} (MIL), 
that explicitly forbids treating unobserved user-item interactions as positive or negative feedback. 
% application to AE and metrics
We apply this loss to both traditional Matrix Factorization and user--based Denoising Autoencoder, and compare it with other established objective functions such as cross--entropy (both point-- and pair--wise) or the recently proposed multinomial log-likelihood. MIL achieves competitive performance in ranking--aware metrics when applied to three datasets.
% towards long-tail recommendations
Furthermore, we show that such a relevance in the recommendation is obtained while displaying popular items less frequently (up to a $20 \%$ decrease with respect to the best competing method). This debiasing from the recommendation of popular items favours the appearance of infrequent items (up to a $50 \%$ increase of long--tail recommendations), a valuable feature for Recommender Systems with a large catalogue of products. 
\end{abstract}

\begin{CCSXML}
<ccs2012>
<concept>
<concept_id>10002951.10003317.10003347.10003350</concept_id>
<concept_desc>Information systems~Recommender systems</concept_desc>
<concept_significance>500</concept_significance>
</concept>
</ccs2012>
\end{CCSXML}

\ccsdesc[500]{Information systems~Recommender systems}

\keywords{Collaborative Filtering, Autoencoders, Implicit Feedback, Missing Information}

\setcopyright{None}

\maketitle

\setlength{\abovecaptionskip}{0pt}
\setlength{\belowcaptionskip}{-10pt}

% Put all the sections with inputs

\input{intro}

\input{model}

\input{protocols}

\input{results}

\section{Conclusions and next steps}\label{sec:conclusions}

% definition of MIL
In this paper we present a novel objective function, the \emph{Missing Information Loss} (MIL), specifically designed for handling unobserved user-item interactions in implicit feedback datasets. In particular, MIL explicitly forbids treating missing user-item interactions as positive or negative feedback.
% What it does
We demonstrate that, thanks to the functional form of the MIL function, the ranking of unseen items is almost entirely left to the low--rank process, rather than forcing unobserved items to be at the tail of the recommendation (\emph{i.e.}, MIL does not force a zero predicted preference for unobserved user-item interactions). 

% Metric results
Extensive experiments with Matrix Factorization and Denoising Autoencoders conducted on three datasets, show that \textsc{MIL} models demonstrate competitive performance when compared with other traditional losses such as cross-entropy or the multinomial log-likelihood. 
% Best performing models 
% Analysis of recommendations
In addition, we study the distribution of the recommendations and observe that the reported metric performance takes place while recommending popular items less frequently (up to a $20 \%$ decrease with respect to the best competing method). Indeed, \textsc{MIL} models sharply increase the recommendation of medium--tail items, while almost linearly expanding the appearance of long--tail items with the ranking position in the list of recommendations. Such expansion results in up to a $50 \%$ increase of long--tail recommendations, a feature of utmost importance for industries with a large catalogue of items. 

% Future work
Future lines of research may involve the incorporation of negative feedback, or the usage of \textsc{MIL} in temporal--aware Recommender Systems (such as those using Recurrent Neural Networks).  
In addition, we hope that the results here reported  will bring forward first-principle mathematical derivations of the \textsc{MIL} function, so that the vast family of possible polynomials modelling the missing information term can be reduced, or even extended with more suitable functions. 

\begin{acks}
We would like to thank the continuous support and careful reading of the manuscript by the \emph{Edge} guild within BBVA Data \& Analytics, specially J. Garc\'ia Santamar\'ia and J. A. Rodr\'iguez Serrano. 
\end{acks}

%\bibliographystyle{ACM-Reference-Format}
\bibliographystyle{unsrt}
%\bibliography{sigproc} 

\input{MIL.bbl}
\end{document}

\end{document}

\end{document}

\end{document}
