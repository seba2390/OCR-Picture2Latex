\documentclass[aps,twocolumn,prl,tightenlines,floatfix,superscriptaddress]{revtex4-1}

%\usepackage[breaklinks=true]{hyperref}
\usepackage{graphicx}
\usepackage[english]{babel}
\usepackage{amsmath}
\usepackage{amsfonts}
\usepackage{amssymb}
\usepackage{times}

\newcommand{\D}{\mathrm{D}}
\newcommand{\p}{\partial}
\newcommand{\Tr}{\mathrm{Tr}}
\renewcommand{\d}{\textrm{d}}
\newcommand{\be}{\begin{equation}}
\newcommand{\ee}{\end{equation}}
\newcommand{\e}{\mathrm{e}}
\renewcommand{\thefigure}{{S}\arabic{figure}}


\begin{document}


\title{Supplementary Information\\
Exotic superfluidity and pairing phenomena in atomic Fermi
  gases in mixed dimensions}

\author{ Leifeng Zhang }
\author{Yanming Che }
\affiliation{Department of Physics and Zhejiang Institute of Modern
  Physics, Zhejiang University, Hangzhou, Zhejiang 310027, China}
\affiliation{Synergetic Innovation Center of Quantum Information and
  Quantum Physics, Hefei, Anhui 230026, China} 

\author{Jibiao Wang }
\affiliation{Department of Physics and Zhejiang Institute of Modern
  Physics, Zhejiang University, Hangzhou, Zhejiang 310027, China}
\affiliation{Synergetic Innovation Center of Quantum Information and
  Quantum Physics, Hefei, Anhui 230026, China} 
\affiliation{TianQin Research Center \& School of Physics and Astronomy, Sun Yat-Sen University (Zhuhai Campus), Zhuhai, Guangdong 519082, China} 

\author{Qijin Chen}
\email[Corresponding author: ]{qchen@uchicago.edu}
\affiliation{Department of Physics and Zhejiang Institute of Modern
  Physics, Zhejiang University, Hangzhou, Zhejiang 310027, China}
\affiliation{Synergetic Innovation Center of Quantum Information and
  Quantum Physics, Hefei, Anhui 230026, China} 
\affiliation{James Franck Institute, University of Chicago, Chicago, Illinois
  60637, USA}


\date{\today}


\begin{abstract}
Here we present extra plots which may help with the understanding of the main text.
\end{abstract}

%\pacs{03.75.Ss,03.75.Hh,37.10.Jk,67.85.-d,74.25.Dw}

\maketitle


\section{Superfluid transition $T_c$ as a function of $1/k_Fa$ for $t/E_F=1$}

Shown in Fig.~\ref{fig:Tc-d} is $T_c$ as a function of $1/k_Fa$ for
$t/E_F=1$, with a series of values of $k_Fd$, as labeled. As a basis
for comparison, we also included the $T_c$ curve from a simple
isotropic 3D Fermi gas, labeled ``3D''. For this large $t=E_F$, the
best Fermi surface match occurs near $k_Fd=1$. Here we only show the
curves with $k_Fd<1$, which do not intersect the 3D curve. The $T_c$
curve splits for small $d$, giving way to pair density wave ground
states. In the shaded area, the system is unstable at
$T_c$. Intermediate temperature superfluid exists for $k_Fd \ge 0.3$.

\begin{figure}[t]
%  \centerline{\includegraphics[clip,width=3.2in] {Tc-inva_p0_d_srep.eps}}
  \centerline{\includegraphics[clip,width=3.2in] {Fig_S1.eps}}
  \caption{ Behavior of $T_c$ as functions of $1/k_Fa$ at fixed
    $t/E_F = 1$, but for different value of $k_Fd \le 1$, as
    labeled. The $T_c$ solution in shaded regions is unstable against
    phase separation.}
\label{fig:Tc-d}
%\vspace*{-2ex}
\end{figure}


\section{Effects of a band dispersion for pairs}

To check the effect of a band dispersion for the pairs on $T_c$, we
performed $T_c$ calculations using both parabolic and band dispersions
for the $\hat{z}$ direction of spin-up fermions. The result is shown
in Fig.~\ref{fig:qz-band}, for $t/E_F=0.05$ and $k_Fd=4$. It is
evident that the two curves overlap with each other for $1/k_Fa<0$,
and only a minor quantitative difference arises in the BEC regime.

\begin{figure}[b]
%  \centerline{\includegraphics[clip,width=3.2in] {Mixp0t0.05d4Tc-InvkFa-qz_comp_srep.eps}}
  \centerline{\includegraphics[clip,width=3.2in] {Fig_S2.eps}}
  \caption{Comparison between two $T_c$ solutions as a function of
    $1/k_Fa$ using a parapolic dispersion (black) and a band
    dispersion (red) for the $q_z$ contribution of the pair. Here
    $t/E_F = 0.05$ and $k_Fd=4$, as labeled. }
\label{fig:qz-band}
\end{figure}


\end{document}
