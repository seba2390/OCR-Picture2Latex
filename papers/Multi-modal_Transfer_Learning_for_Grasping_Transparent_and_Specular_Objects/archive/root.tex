\documentclass{article}
\pdfminorversion=5 
\pdfcompresslevel=9
\pdfobjcompresslevel=2

\usepackage{corl_2019} % initial submission
% \usepackage[final]{corl_2019} % Uncomment for the camera-ready ``final'' version
\usepackage{booktabs}
\usepackage{caption}
\usepackage{subcaption}
\usepackage{graphicx}
\usepackage{amsmath}

\newcommand{\todo}[1]{\textcolor[rgb]{0.1,0.5,0.1}{\small {~[TODO]~}{#1}}}
\newcommand{\meta}[1]{\textcolor[rgb]{0.2,0.2,0.7}{\small {~[META]~}{#1}}}
\newcommand{\comment}[1]{\textcolor[rgb]{0.2,0.2,0.7}{\small {~[COMMENT]~}{#1}}}

\newcommand{\fig}[1]{Fig.~\ref{#1}}
\newcommand{\tbl}[1]{Table~\ref{#1}}
\newcommand{\algo}[1]{Algorithm~\ref{#1}}
\newcommand{\sect}[1]{Sec.~\ref{#1}}
\newcommand{\etal}[0]{{\em et al.~}}
\newcommand{\eg}[0]{{\em e.g.,~}}
\newcommand{\ie}[0]{{\em i.e.,~}}
\newcommand{\etc}[0]{{\em etc.\xspace}}

\title{Transfer Learning for Multi-modal Perception on Grasping Transparent and Specular Objects}

% The \author macro works with any number of authors. There are two
% commands used to separate the names and addresses of multiple
% authors: \And and \AND.
%
% Using \And between authors leaves it to LaTeX to determine where to
% break the lines. Using \AND forces a line break at that point. So,
% if LaTeX puts 3 of 4 authors names on the first line, and the last
% on the second line, try using \AND instead of \And before the third
% author name.

% NOTE: authors will be visible only in the camera-ready (ie, when using the option 'final'). 
% 	For the initial submission the authors will be anonymized.

\author{
  Thomas Weng\\
%   Carnegie Mellon University\\
  \texttt{tweng@cmu.edu} \\
  %% examples of more authors
%   \And
%   Amith Pallankize\\
%   Affiliation \\
%   Address \\
%   \texttt{email} \\
  \AND
  David Held\\
%   Carnegie Mellon University\\
  Address \\
  \texttt{email} \\
  %% \And
  %% Coauthor \\
  %% Affiliation \\
  %% Address \\
  %% \texttt{email} \\
  %% \And
  %% Coauthor \\
  %% Affiliation \\
  %% Address \\
  %% \texttt{email} \\
}


\begin{document}
\maketitle

%===============================================================================

\begin{abstract}
State-of-the-art object grasping methods rely on depth sensing to plan robust grasps, but commercially available depth sensors fail to detect transparent and specular objects.
To reach comparable grasping performance on such objects, we introduce a method for learning a multimodal perception model by bootstrapping from an existing uni-modal model. 
This transfer learning approach requires only paired multi-modal data for training, and requires no ground-truth labels nor real grasp attempts.
Our experiments show that our method outperforms depth-based models on transparent, specular, and opaque objects.
\end{abstract}

% Two or three meaningful keywords should be added here
\keywords{Multimodal perception, transfer learning, grasping} 

% !TEX root = 0-qqQQmain.tex

\section{Introduction}


The success of the collider particle physics program, whose main
player today is the Large Hadron Collider (LHC) at CERN, relies
heavily on our ability to model with high precision and accuracy the
scattering of high energetic protons in Quantum Chromodynamics (QCD).
Thanks to asymptotic freedom and the factorization properties of QCD,
this intrinsically non-perturbative problem can be treated with
perturbative methods, supplemented by non-perturbative information about
the distribution of partons in the proton. Within this picture, an important
role is played by higher order perturbative QCD calculations, which allow
for a reliable and precise description of a wide range of collider processes
and observables.

Thanks to a concerted effort in the high-energy community over the
last few years, it is currently possible to compute predictions for
many interesting reactions to second order in the strong coupling
expansion, $i.e.$ to what is usually referred to as
next-to-next-to-leading order (NNLO). 
This has required, on the one hand, 
major advances in computational techniques
for multi-loop scattering amplitudes~\cite{Tkachov:1981wb,Chetyrkin:1981qh,Hodges:2009hk,Gluza:2010ws,Ita:2015tya,Larsen:2015ped,Bohm:2017qme,Badger:2016uuq,vonManteuffel:2014ixa,Peraro:2016wsq,Peraro:2019svx,Guan:2019bcx,Pak:2011xt,Abreu:2019odu,Heller:2021qkz,Kotikov:1990kg,Bern:1993kr,Remiddi:1997ny,Gehrmann:1999as,Papadopoulos:2014lla,Dixon:1996wi,Henn:2013pwa,Primo:2016ebd,Goncharov,Remiddi:1999ew,Goncharov:2001iea,Goncharov:2010jf,Brown:2008um,Ablinger:2013cf,Panzer:2014caa,Duhr:2011zq,Duhr:2012fh,Duhr:2019tlz},
which, notably, have recently made it possible to compute
various $2 \to 3$ processes up to two loops in QCD~\cite{Badger:2017jhb,Abreu:2017hqn,Abreu:2018aqd,Abreu:2018zmy,Abreu:2018jgq,Abreu:2019rpt,Abreu:2020cwb,Chicherin:2018yne,Chicherin:2019xeg,Chawdhry:2020for,DeLaurentis:2020qle,Chawdhry:2018awn,Abreu:2020xvt,Agarwal:2021grm,Badger:2021nhg,Abreu:2021fuk,Agarwal:2021vdh,Chawdhry:2021mkw,Badger:2021imn,Gehrmann:2015bfy,Papadopoulos:2015jft,Gehrmann:2018yef,Chicherin:2018mue,Chicherin:2020oor}.
On the other hand, the use of these amplitudes to perform phenomenological
studies for the relevant processes at NNLO~\cite{Chawdhry:2019bji,Kallweit:2020gcp,Chawdhry:2021hkp,Czakon:2021mjy} has required the
development of so-called subtraction or slicing frameworks~\cite{GehrmannDeRidder:2005cm,Czakon:2010td,Caola:2017dug,Magnea:2018hab,Herzog:2018ily,DelDuca:2016ily,Cacciari:2015jma,Catani:2007vq,Gaunt:2015pea,Boughezal:2015dva}
to properly deal with the intricate IR divergences that appear in QCD reactions. 

Beyond NNLO, predictions at third order in the
perturbative couplings, i.e.\ at N$^3$LO, are
known only for a handful of important LHC processes~\cite{Anastasiou:2015vya,Duhr:2019kwi,
Dulat:2018bfe,Mistlberger:2018etf,Dreyer:2016oyx,Dreyer:2018qbw,
Billis:2021ecs,Chen:2021isd,Chen:2021vtu}. 
In particular, N$^3$LO results are currently available only for reactions that
require at most three-point three-loop integrals.
Given the remarkable success of the
experimental program at the LHC, it is desirable to extend these
calculations to more complex processes. A particularly interesting
one is di-jet production.  In fact, jets are ubiquitous at
hadron colliders, so understanding their dynamics is of great
interest.
Moreover, di-jet production is the first massless $2\to2$
process that has a non-trivial colour structure. This makes it an
ideal ground for studying the structure of perturbative QCD. For
example, it is by now well known that when four or more coloured
partons interact, starting at the three-loop order, non-trivial
colour correlations can affect the pattern of IR divergences,
generating new structures~\cite{Almelid:2015jia} beyond the standard dipole
formula~\cite{Sterman:2002qn,Aybat:2006wq,Aybat:2006mz,Becher:2009cu,Gardi:2009qi,Becher:2009qa,Dixon:2009gx}. Also, the
non-trivial colour structure may create subtle violations of the
factorization framework that is at the very core of theoretical
predictions at hadronic
colliders~\cite{Catani:2011st,Forshaw:2012bi,Forshaw:2006fk,Becher:2021zkk}.
This makes jet production at hadronic colliders an extremely interesting
process to investigate at higher orders. 

A key ingredient for the study of jet production at N$^3$LO is
provided by the virtual three-loop corrections to the scattering
amplitudes for the production of two jets in massless QCD.  Modulo
crossings, there are three main partonic channels that need to be
computed: four-gluon scattering, the scattering of two quarks and two
gluons, and the scattering of four quarks.  All ingredients necessary
for the calculation of the two-loop QCD corrections to these processes have
been known for a long time~\cite{Smirnov:1999gc,Tausk:1999vh,Glover:2001af,Anastasiou:2002zn,Glover:2003cm}, which 
have made it possible to compute the relevant scattering amplitudes~\cite{Anastasiou:2000kg,Bern:2003ck,Glover:2004si,DeFreitas:2004kmi}. Also, in view of extending these
calculations to three loops, results for the two loop helicity
amplitudes up to order $\epsilon^2$ have been
obtained~\cite{Ahmed:2019qtg}.  For what concerns the three loop
results, instead, the relevant master integrals have been computed in
ref.~\cite{Henn:2020lye}, and have then been used to obtain the first
three loop results for $2 \to 2$ scattering amplitudes in
supersymmetric theories~\cite{Henn:2016jdu,Henn:2019rgj}.  More
recently also the first three loop corrections to the production of
two photons in full QCD have been obtained~\cite{Caola:2020dfu}.

In this paper, we move one step further and consider one of the three classes of partonic processes  listed above, 
namely the scattering of four massless quarks. This particular process is interesting not only
because it allows us, for the first time, to check the full structure of IR divergences at three loops in QCD, 
but also because it involves two external spinor structures. In fact, this property makes the use of the standard form factors
method for the calculation of the helicity amplitudes particularly cumbersome, due to the fact that the  $\gamma$-algebra does not
close in  $d$ space-time dimensions. For the calculation of the helicity amplitudes we then make use of
a different approach, recently described in refs~\cite{Peraro:2019cjj,Peraro:2020sfm}, which allows us to calculate the helicity amplitudes
in a simpler way, corresponding to the 't Hooft-Veltman scheme (tHV)~\cite{tHooft:1972tcz} for the processes considered here.\footnote{See also ref.~\cite{Heller:2020owb} for an application of similar ideas in the case of a chiral theory, and ref.~\cite{Chen:2019wyb} for an alternative approach.} 
In doing this, we also expose some
subtleties in the usual approach to compute helicity amplitudes in 
tHV with the standard form factor method.
The rest of the paper is organised as follows. 
We start in section~\ref{Kinematics} by establishing the notation for the calculation of the
fundamental partonic channel $q\bar{q} \to Q \bar{Q}$, from which all other channels can by obtained by crossing.
We continue in section~\ref{The Scattering Amplitude},
where we describe the colour and tensor decomposition  of the scattering amplitude, and show how to 
efficiently compute the helicity amplitudes without considering evanescent Lorentz structures.
In section~\ref{computation} we provide details on our computational set up,
and in section~\ref{subtraction} we discuss the renormalisation and the infrared structure of the three-loop scattering amplitudes.
In section~\ref{results} we discuss our final results for the main partonic channel.
In section~\ref{extra_results} we then explain how to obtain all other partonic channels from our calculation, both for quarks of equal and different flavour.
Finally we conclude in section~\ref{conclusions}.  In 
appendices~\ref{sec:appB} and~\ref{sec:appA}
we provide some details on the structure of infrared divergences up at three loops, in particular focusing on the explicit derivation of the quadrupole terms, which appear for the first time with the scattering of at least four coloured partons at three loops.
In appendix~\ref{sec:appC}, we review the analytical continuation of the amplitudes to different regions of phase space.

\section{Related Work}
\label{sec:related}

\subsection{Sensing Transparent and Specular Objects}

Sensing transparent and specular objects is a well-studied challenge in the computer vision community. 
Ihrke \etal\cite{ihrke2010transparent} provide a survey of recent approaches to transparent and specular object reconstruction.
Curless \etal\cite{curless1995better} perform space-time analysis on structured light sensing to achieve better triangulation on transparent objects.
Structured light sensing can also be paired with additional equipment like polarization lenses, light fields, or immersion in fluorescent or refractive liquids to detect transparent objects.
While structured light sensing is the closest to commercial sensing, the survey also presents methods that improve on multi-view stereo matching to detect transparent and specular objects.

Light field photography for depth reconstruction is another direction for detecting specular and transparent objects~\cite{6619203,wetzstein2011hand}. 
Light field photography has been used in robotics by 
Oberlin \etal\cite{oberlintime} applied light field photography to robot manipulation tasks like grasping non-Lambertian objects under running water.
However, this method requires capturing a dense set of images in a 3D volume over the scene of interest at both training and test time to construct suitable synthetic images for grasping.
In comparison, our proposed method requires a single, static RGB-D sensor, resulting in faster and simpler training and deployment.

Commercial RGB-D sensors (\eg~Intel RealSense, Microsoft Kinect, PrimeSense) use structured-light or time-of-flight techniques to estimate depth.
These techniques fail on transparent and specular surfaces, either allowing light emitted by the sensor to pass through or scattering it by reflection.
IR stereo and cross-modal stereo techniques have been used to improve depth reconstruction, but the reconstruction quality is still not comparable to that of Lambertian, or diffusely reflective, objects~\cite{alhwarin2014ir, chiu2011improving, mahler2018guest}. 
Lysenkov \etal\cite{lysenkov2013recognition, lysenkov2013pose} painted over transparent objects to create a dataset of paired transparent and opaque objects, but this approach scales poorly for objects with arbitrary geometries and material properties.
Our proposed method is able to use conventional RGB-D sensors without hardware and environmental modifications by combining depth and color information. 

\subsection{Grasp Synthesis}

Grasp synthesis refers to the problem of finding a stable robotic grasp for a given object and is a longstanding research problem in robotics. 
Approaches to grasp synthesis can be classified into analytic and empirical methods; see Bohg \etal\cite{bohg2013data} for a survey.
Analytic approaches use physics-based contact models to compute force closure on an object, using the shape and estimated pose of the target object~\cite{miller2003automatic, ten2017grasp, watkins2018multi}, but work poorly in the real world due to noisy sensing, simplified assumptions of contact physics, and difficulty in placing contact points accurately.

Empirical approaches, on the other hand, learn to predict the quality of grasp candidates from data on a diverse set of objects, images, and grasp attempts collected through human labeling~\cite{saxena2008robotic, jiang2011efficient, doi:10.1177/0278364914549607, 7139361}, self-supervision~\cite{pinto2016supersizing, levine2018learning}, or simulated data~\cite{depierre2018jacquard, doi:10.1177/0278364919859066, gualtieri2016high, mahler2017dex, satish2019policy}. 
Saxena \etal\cite{saxena2008robotic} trained a classifier on human-labeled RGB images to predict grasp points, triangulated the points on stereo RGB images, and demonstrated successful grasps on a limited set of household objects, including some transparent and specular objects.
However, the predicted grasp points for transparent and specular objects were limited to grasps on points where stereo triangulation was successful. 
The Cornell Grasping Dataset~\cite{jiang2011efficient}, consisting of 1k RGB-D images of objects and human-labeled grasps parameterized as an oriented bounding box, has been used to train many deep learning-based grasping methods~\cite{doi:10.1177/0278364914549607, doi:10.1177/0278364919859066, 7139361}. 
Self-supervised methods such as those by Pinto and Gupta~\cite{pinto2016supersizing} or Levine \etal\cite{levine2018learning} forego the need for human labels by training a robot to grasp directly from real grasp attempts, but these methods require tens of thousands of attempts to converge.

Recently, approaches trained on data gathered in simulation have demonstrated state-of-the-art performance.
The Jacquard dataset by Amaury \etal\cite{depierre2018jacquard} uses a grasp specification similar to the Cornell Grasping Dataset, contains simulated objects and grasp attempts, and has been successfully used for training by Morrison \etal's GG-CNN~\cite{doi:10.1177/0278364919859066}. 
Mahler \etal\cite{mahler2017dex} developed GQCNN, which was trained on a dataset of simulated grasps generated using analytic model, representing a hybrid empirical and analytic approach.  

As we will show, these depth-only grasping approaches fail on transparent and reflective objects. 
Note that GG-CNN could be modified to incorporate RGB images, which could potentially be used to grasp transparent and specular objects after training on simulated images (such as those in the Jacquard dataset~\cite{depierre2018jacquard}); however, such performance has not been demonstrated; this method has only been demonstrated for depth-based grasping of opaque objects.
In this work, we build upon the fully convolutional version of GQCNN (FC-GQCNN) proposed by Satish \etal\cite{satish2019policy}, but our method is agnostic to the specific network architecture used.
Our method does not require any real-world grasps or labeled data but instead relies on supervision transfer from \hl{a pre-trained depth network to obtain a multi-modal grasping method. The pre-trained depth network also may not require real-world grasps or human labels; for example, FC-GQCNN is trained entirely on simulated grasps.}

\subsection{Cross-modal Transfer Learning}
Supervision transfer has been explored in the past for tasks such as image classification and object detection~\cite{gupta2016cross, hoffman2016cross, li2018cross}.  These approaches are typically used to transfer image-based networks trained on ImageNet~\cite{deng2009imagenet} to depth-based or RGB-D based classification or detection networks.  To our knowledge, such approaches have not been used previously in the context of multi-modal grasping.  We show that such an approach can lead to greatly improved performance for grasping transparent and reflective objects, and can even improve performance on some opaque objects.

\section{Dynamic Alignment Networks}
\label{subsec:alignment}


In this section, we present our novel type of network architecture: the Convolutional Dynamic Alignment Networks (CoDA-Nets). For this, we first introduce Dynamic Alignment Units (DAUs) as the basic building blocks of CoDA-Nets and discuss two of their key properties in sec.~\ref{subsec:align_units}. Concretely, we show that these units linearly transform their inputs with dynamic (input-dependent) weight vectors and, additionally, that they are biased to align these weights with the input during optimisation. 
We then discuss how DAUs can be used for classification (sec.~\ref{subsec:classification}) and how we build performant networks out of multiple layers of convolutional DAUs (sec.~\ref{subsec:coda}). Importantly, the resulting \emph{linear decompositions} of the network outputs are optimised to align with discriminative patterns in the input, making them highly suitable for interpreting the network predictions. 

In particular, we structure this section around the following \textbf{three important properties} (\colornum{P1-P3}) of the DAUs:
\\[.25em]
\colornum{P1: Dynamic linearity.} The DAU output $o$ is computed as a dynamic (input-dependent) linear transformation of the input $\vec x$, such that \mbox{$o=\vec w(\vec x)^T\vec x=\sum_jw_j(\vec x)x_j$}. 
Hence, 
$o$ can be decomposed into contributions  
from individual input dimensions, which are given by $w_j(\vec x)x_j$ for dimension $j$.
\\[.5em]
\colornum{P2: Alignment maximisation.} Maximising the average output of a single DAU over a set of inputs $\vec x_i$ 
maximises the alignment between inputs $\vec x_i$ and the weight vectors $\vec w(\vec x_i)$. As the modelling capacity of $\vec w(\vec x)$ is restricted, $\vec w(\vec x)$ will encode the most frequent patterns in the set of inputs $\vec x_i$.
\\[.5em]
\colornum{P3: Inheritance.} When combining multiple DAU layers to form a \mbox{Dynamic} Alignment Network (DA-Net), the properties \colornum{P1} and \colornum{P2} are \emph{inherited}. In particular, DA-Nets are dynamic linear (\colornum{P1}) and maximising the last layer's output induces an output maximisation in the constituent DAUs (\colornum{P2}).
\\[.5em]
These properties increase the interpretability
of a DA-Net, such as a CoDA-Net (sec.~\ref{subsec:coda}) for the following reasons.
First, the output of a DA-Net can be decomposed into contributions from the individual input dimensions, similar to linear models (cf.~Fig.~\ref{fig:teaser}, \colornum{P1} and \colornum{P3}).
Second, we note that optimising a neural network for classification applies a maximisation to the outputs of the last layer for every sample. 
This maximisation aligns the dynamic weight vectors $\vec w(\vec x)$ of the constituent DAUs of the DA-Net with their respective inputs (cf.~Fig.~\ref{fig:alignment}, \colornum{P2} and \colornum{P3}).

 Importantly, the weight vectors will align with the \emph{discriminative} patterns in their inputs when optimised for classification as we show in sec.~\ref{subsec:classification}.
As a result, the model-inherent contribution maps of CoDA-Nets are optimised to align well with \emph{discriminative input patterns} in the input image 
and the interpretability of our models thus forms part of the global optimisation procedure.
\begin{figure}[t!]
    \centering
    \hspace{-.25em}
    \begin{subfigure}[b]{0.48\textwidth}
    \includegraphics[width=\textwidth]{resources/DAUs-v3.pdf}
     \end{subfigure}
    \caption{\small
        For different inputs $\vec x$, we visualise the linear weights and contributions (for the single layer, see eq.~\eqref{eq:contrib_1}, for the CoDA-Net eq.~\eqref{eq:contrib}) for the ground truth label $l$ and the strongest non-label output $z$. 
    As can be seen, the weights align well with the input images.
    The first three rows are based on a single DAU layer, the last three on a 5 layer CoDA-Net. The first two samples (rows) per model are correctly classified and the last one is misclassified. }
    \label{fig:alignment}
\end{figure}
\subsection{Dynamic Alignment Units}
\label{subsec:align_units}
We define the Dynamic Alignment Units (DAUs) by
\begin{align}
    \label{eq:au}
    \text{DAU}(\vec x) = g(\mat a \mat b\vec x +\vec b)^T \vec x = \vec w(\vec x)^T\, \vec x\quad \textbf{.}
\end{align}
% 
Here, $\vec x\in\mathbb R^{d}$ is an input vector, $\mat a\in\mathbb R^{d\times r}$ and $\mat b \in \mathbb R^{r\times d}$ are trainable transformation matrices, $\vec b\in\mathbb R^{d}$ a trainable bias vector, and \mbox{$g(\vec u)=\alpha(||\vec u||)\vec u$} is a non-linear function that scales the norm of its input. {In contrast to using a single matrix $\mat m \in\mathbb R^{d\times d}$, using $\mat{ab}$ allows us to control the maximum rank $r$ of the transformation and to reduce the number of parameters}; we will hence refer to $r$ as the rank of a DAU. 
%
As can be seen by the right-hand side of eq.~\eqref{eq:au}, the DAU linearly transforms the input $\vec x$ (\colornum{P1}). At the same time, given the quadratic form ($\vec x^T\mat B^T\mat A^T\vec x$) and the  rescaling function $\alpha(||\vec u||)$, the output of the DAU is a non-linear function of its input. In this work, we focus our analysis on 
two choices for $g(\vec u)$ in particular\footnote{
In preliminary experiments we observed comparable behaviour over a range of different normalisation functions such as, e.g., L1 normalisation.}, namely rescaling to unit norm ($\text{L2}$) and the squashing function ($\text{SQ}$, see \cite{sabour2017dynamic}):
\begin{align}
    \label{eq:nonlin}
    \text{L2}(\vec u) = \frac{\vec u}{||\vec u||_2} \;\;\text{and}\;\;
    \text{SQ}(\vec u) = \text{L2}(\vec u) \times \frac{||\vec u||^2_2}{1+||\vec u||_2^2}
\end{align}
Under these rescaling functions, the norm of the weight vector is upper-bounded: $||\vec w(\vec x)|| \leq 1$. Therefore, the output of the DAUs is upper-bounded by the norm of the input:
\begin{align}
    \text{DAU}(\vec x) = 
    ||\vec w(\vec x)|| \hspace{.2em} ||\vec x|| \cos(\angle(\vec x, \vec w(\vec x)))\leq ||\vec x||
    \label{eq:bound}
\end{align}
As a corollary, for a given input $\vec x_i$, the DAUs can only achieve this upper bound if $\vec x_i$ is an eigenvector (EV) of the linear transform $\mat{AB}\vec x+ \vec b$. Otherwise, the cosine in eq.~\eqref{eq:bound} will not be maximal\footnote{
Note that $\vec w(\vec x)$ is proportional to $\mat{ab}\vec x + \vec b$. The cosine in eq.~\eqref{eq:bound}, in turn, is maximal if and only if $\vec w(\vec x_i)$ is proportional to $\vec x_i$ and thus, by transitivity, if $\vec x_i$ is proportional to $\mat{ab}\vec x_i + \vec b$. This means that $\vec x_i$ has to be an EV of $\mat{ab}\vec x +\vec b$ to achieve maximal output.}. 
As can be seen in eq.~\eqref{eq:bound}, maximising the average output of a DAU over a set of inputs $\{\vec x_i|\,i=1, ..., n\}$
maximises the alignment between $\vec w(\vec x)$ and $\vec x$ (\colornum{P2}).
In particular, it optimises the parameters of the DAU such that the \emph{most frequent input patterns} are encoded as EVs in the linear transform $\mat{ab}\vec x + \vec b$, similar to an $r$-dimensional PCA decomposition ($r$ the rank of $\mat{ab}$). In fact, as discussed in the supplement, the optimum of the DAU maximisation solves a low-rank matrix approximation~\cite{eckart1936approximation} problem similar to singular value decomposition.
\begin{figure}[t!]
    \centering
    \includegraphics[height=6.5em]{resources/evs.pdf}
    \caption{\small Eigenvectors (EVs) of \tmat{AB} after maximising the output of a rank-3 DAU over a set of noisy samples of 3 MNIST digits. Effectively, the DAUs encode the most frequent components in their EVs, similar to a principal component analysis (PCA).
    }
    \label{fig:EVs}
\end{figure}
%
As an illustration of this property, in Fig.~\ref{fig:EVs} we show the 3 EVs\footnote{Given $r=3$, the EVs maximally span a 3-dimensional subspace.} of matrix $\mat{ab}$ (with rank $r=3$, bias $\vec b=\vec 0$) after optimising a DAU over a set of $n$ noisy samples of 3 specific MNIST~\cite{lecun2010mnist} images; for this, we used $n=3072$ and zero-mean Gaussian noise. As expected, the EVs of \tmat{ab} encode the original, noise-free images, since this on average maximises the alignment (eq.~\eqref{eq:bound}) between the weight vectors $\vec w(\vec x_i)$ and the input samples $\vec x_i$ over the dataset.
%
%

\subsection{DAUs for classification}
\label{subsec:classification}
{DAUs can be used directly for classification by applying $k$ DAUs in parallel to obtain an output \mbox{$\hat{\vec y}(\vec x)=\left[\text{DAU}_1(\vec x), ..., \text{DAU}_k(\vec x)\right]$}. 
Note that this is a linear transformation $\hat{\vec y}(\vec x)$$=$$\mat W(\vec x) \vec x$, with each row in $\mat w$$\in$$\mathbb R^{k \times d}$ corresponding to the weight vector $\vec w_j^T$ of a specific DAU $j$.
In particular, consider 
a dataset $\mathcal D = \{(\vec x_i, \vec y_i)|\, \vec x_i\in\mathbb R^d, \vec y_i\in\mathbb R^k\}$ of $k$ classes with `one-hot' encoded labels $\vec y_i$ for the inputs $\vec x_i$.
To optimise the DAUs as classifiers on $\mathcal D$,} we can apply a sigmoid non-linearity to each DAU output and optimise the loss function $\mathcal L = \sum_i\text{BCE}(\sigma(\hat{\vec y}_i), \vec y_i)$, where \text{BCE} denotes the binary cross-entropy and $\sigma$ applies the sigmoid function to each entry in $\hat{\vec y}_i$. Note that for a given sample, \text{BCE} either maximises (DAU for correct class) or minimises (DAU for incorrect classes) the output of each DAU. Hence, this classification loss will still maximise the (signed) cosine between the weight vectors $\vec w(\vec x_i)$ and $\vec x_i$. 

To illustrate this property, in Fig.~\ref{fig:alignment} (top) we show the weights $\vec w(\vec x_i)$ for several samples of the digit `3' after optimising the DAUs for classification on a noisy MNIST dataset; the first two are correctly classified, the last one is misclassified as a `5'. As can be seen, the weights align with the respective input (the weights for different samples are different). However,  different parts of the input are either positively or negatively correlated with a class, which is reflected in the weights: for example, the extended stroke on top of the `3' in the misclassified sample is assigned \emph{negative weight} and, since the background noise is \emph{uncorrelated} with the class labels, it is not represented in the weights. 

In a classification setting, the DAUs {thus} encode \emph{the most frequent discriminative patterns} in the linear transform $\mat{ab}\vec x + \vec b$ such that the dynamic weights $\vec w(\vec x)$ align well with these patterns.
Additionally, since the output for class $j$ is a linear transformation of the input (\colornum{P1}), we can compute the contribution vector $\vec s_j$ containing the per-pixel contributions to this output by the element-wise product ($\odot$)
\begin{align}
\label{eq:contrib_1}
    \vec s_j(\vec x_i) = \vec w_j(\vec x_i)\odot\vec x_i\quad ,
\end{align}
 see Figs.~\ref{fig:teaser} and
\ref{fig:alignment}. 
Such linear decompositions constitute the model-inherent `explanations' which we evaluate in sec.~\ref{sec:results}.
\subsection{Convolutional Dynamic Alignment Networks}
\label{subsec:coda}
The modelling capacity of a single layer of DAUs is limited, similar to a single linear classifier. However, DAUs can be used as the basic building block for deep convolutional neural networks, which yields powerful classifiers. Importantly, in this section we show that such a Convolutional Dynamic Alignment Network (CoDA-Net) inherits the properties (\colornum{P3}) of the DAUs by maintaining both the dynamic linearity (\colornum{P1}) as well as the alignment maximisation (\colornum{P2}). For a convolutional dynamic alignment layer, each filter is modelled by a DAU, similar to dynamic local filtering layers~\cite{jia2016dynamic}. Note that the output of such a layer is also a dynamic linear transformation of the input to that layer, since a convolution is equivalent to a linear layer with certain constraints on the weights, cf.~\cite{convlin}. We include the implementation details in the supplement.
Finally, at the end of this section, we highlight an important difference between output maximisation and optimising for classification with the {BCE} loss. In this context we discuss the effect of \emph{temperature scaling} and present the loss function we optimise in our experiments.

\myparagraph{Dynamic linearity (\colornum{P1}).} In order to see that the linearity is maintained, we note that the successive application of multiple layers of DAUs also results in a dynamic linear mapping. Let $\mat W_l$ denote the linear transformation matrix produced by a layer of DAUs and let $\vec a_{l-1}$ be the input vector to that layer; as mentioned before, each row in the matrix $\mat w_l$ corresponds to the weight vector of a single DAU\footnote{
Note that this also holds for convolutional DAU layers. Specifically, each row in the matrix $\mat w_l$ corresponds to a single DAU applied to exactly one spatial location in the input and the input with spatial dimensions is vectorised to yield $\vec a_{l-1}$. For further details, we kindly refer the reader to~\cite{convlin} and the implementation details in the supplement of this work.}. As such, the output of this layer is given by 
\begin{align}
    \vec a_l = \mat W_l (\vec a_{l-1}) \vec a_{l-1}\quad .
\end{align}
In a network of DAUs, the successive linear transformations can thus be collapsed. In particular, \emph{for any pair of activation vectors} $\vec{a}_{l_1}$ and $\vec{a}_{l_2}$ with ${l_1}<{l_2}$, the vector $\vec{a}_{l_2}$ can 
    be expressed as a linear transformation of $\vec{a}_{l_1}$:
\begin{align}
\label{eq:collapse}
    \vec{a}_{l_2} &= \mat{W}_{{l_1}\rightarrow {l_2}} \left(\vec{a}_{l_1}\right)\vec{a}_{l_1} \quad 
        \\{with} \quad \mat{W}_{{l_1}\rightarrow {l_2}}\left(\vec{a}_{l_1}\right) &= \textstyle\prod_{k={l_1}+1}^{l_2} \mat{W}_k \left(\vec{a}_{k-1}\right)\quad \text{.}
\end{align}
For example, the matrix $\mat W_{0\rightarrow L}(\vec{a}_0 = \vec{x}) = \mat W(\vec{x})$ models the linear transformation from the input to the output space, see Fig.~\ref{fig:teaser}.
Since this linearity holds between any two layers, the $j$-th entry of any activation vector $\vec a_l$ in the network can be decomposed into input contributions via:
    \begin{align}
    \label{eq:contrib}
        \vec{s}_{j}^l(\vec x_i) = \left[\mat W_{0\rightarrow l} (\vec{x}_i)\right]_j^T \odot \vec x_i\quad \text{,}
    \end{align}
    with $[\mat W]_j$ the $j$-th row in the matrix.
%

\myparagraph{Alignment maximisation (\colornum{P2}).}
Note that the output of a CoDA-Net is bounded independent of the network parameters: since each DAU operation can---independent of its parameters---at most reproduce the norm of its input (eq.~\eqref{eq:bound}), the linear concatenation of these operations necessarily also has an upper bound which does not depend on the parameters.
Therefore, in order to achieve maximal outputs on average (e.g., the class logit over the subset of images of that class), all DAUs in the network need to produce weights $\vec w (\vec a_l)$ that align well with the class features. In other words, the weights will align with discriminative patterns in the input.
For example, in Fig.~\ref{fig:alignment} (bottom), we visualise the `global matrices' $\mat W_{0\rightarrow L}$ and the corresponding contributions (eq.~\eqref{eq:contrib}) for a $L=5$ layer CoDA-Net. As before, the weights align with discriminative patterns in the input and do not encode the uninformative noise.
%
%
%
%

\myparagraph[0]{Temperature scaling and loss function.} 
\begin{figure}[t]
    \centering
    \includegraphics[width=.45\textwidth]{resources/Temperature_qualitative.pdf}
    \caption{\small By lowering the upper bound (cf.~eq.~\eqref{eq:bound}), the correlation maximisation in the DAUs can be emphasised.
    We show contribution maps for a model trained with different temperatures.
    }
    \label{fig:scaling}
\end{figure}
So far we have assumed that minimising the {BCE} loss for a given sample is equivalent to applying a maximisation or minimisation loss to the individual outputs of a CoDA-Net. While this is in principle correct, {BCE} introduces an additional, non-negligible effect: \emph{saturation}. Specifically, it is possible for a CoDA-Net to achieve a low {BCE} loss without the need to produce well-aligned weight vectors. As soon as the classification accuracy is high and the outputs of the networks are large, the gradient---and therefore the \emph{alignment pressure}---will vanish. This effect can, however, easily be mitigated:
 as discussed in the previous paragraph, the output of a CoDA-Net is upper-bounded \textit{independent of the network parameters}, since each individual DAU in the network is upper-bounded. 
By scaling the network output with a temperature parameter $T$ such that 
    $\hat{\vec y} (\vec x) = T^{-1} \mat W_{0\rightarrow L}(\vec x)\,\vec x$, 
we can explicitly decrease this upper bound and thereby increase the \emph{alignment pressure} in the DAUs by avoiding the early saturation due to {BCE}.
In particular, the lower the upper bound is, the stronger the induced DAU output maximisation should be, since the network needs to accumulate more signal to obtain large class logits (and thus a negligible gradient). This is indeed what we observe both qualitatively, cf.~Fig.~\ref{fig:scaling}, and quantitatively, cf.~Fig.~\ref{fig:localisation} (right column).
Alternatively, the representation of the network's computation as a linear mapping allows to directly regularise what properties these linear mappings should fulfill. For example, we show in the supplement that by regularising the absolute values of the matrix $\mat W_{0\rightarrow L}$, we can induce sparsity in the signal alignments, which can lead to sharper heatmaps.
%
The overall loss for an input $\vec x_i$ and the target vector $\vec y_i$ is thus computed as 
    \begin{align}
        \label{eq:loss}
        \mathcal{L}(\vec x_i, \vec y_i) &= 
        \text{BCE}(\sigma(T^{-1} \mat W_{0\rightarrow L}(\vec x_i)\,\vec{x}_i + {\vec{b}}_0)\,,\, \vec{y}_i) \\&+ 
        \lambda \langle | \mat W_{0\rightarrow L}(\vec x_i) |\rangle\quad \text{.}
    \end{align}
    Here, $\lambda$ is the strength of the regularisation, $\sigma$ applies the sigmoid activation to each vector entry,
    ${\vec{b}}_0$ is a fixed bias term, and $\langle|\mat W_{0\rightarrow L}(\vec x_i)|\rangle$ refers to the mean over the absolute values of 
        all entries in the matrix $\mat W_{0\rightarrow L}(\vec x_i)$.
%
%
%
\subsection{Implementation details}
\label{subsec:details}
\myparagraph[-.25]{Shared matrix \tmat b.} In our experiments, we opted to share the matrix $\mat b\in \mathbb R^{r\times d}$ between all DAUs in a given layer. This increases parameter efficiency by having the DAUs share a common $r$-dimensional subspace and still fixes the maximal rank of each DAU to the chosen value of $r$. 

\myparagraph[-.25]{Input encoding.} 
In sec.~\ref{subsec:align_units}, we showed that the norm-weighted cosine similarity between the dynamic weights and the layer inputs is optimised and the output of a DAU is at most the norm of its input. This favours pixels with large RGB values, since these have a larger norm and can thus produce larger outputs in the maximisation task. To mitigate this bias, we add the negative image as three additional color channels and thus encode each pixel in the input %is encoded 
as 
\mbox{[$r$, $g$, $b$, $1-r$, $1-g$, $1-b$]}, with $r, g, b\in [0, 1]$.

\documentclass[../absorber.tex]{subfiles}
\begin{document}

To demonstrate the effectiveness of the absorber region, we present results from a variety of 2-dimensional simulations of a laser incident on an overdense plasma.  Simulations were done using \textsc{Osiris},\cite{Fonseca2002} where an absorbing region has been implemented.

\subsection{Simulation setup}
In the simulations, an intense 1-$\mu$m plane-wave laser with normalized amplitude $a_0=3$ and 3~ps in duration (2.9-ps flat envelope with 0.13-ps rise and fall ramps) is incident on uniform plasma with density $n=10n_c$ (where $n_c$ is the critical density).  The exponential ramp has a scale length of 3~$\mu$m and begins at $x=-27.6$~$\mu$m.  The critical density is then located at $x_c=-6.9$~$\mu$m, and we define time $t=0$ to be when the leading edge of the laser pulse would arrive at $x_c$ if traveling at speed $c$. The laser is focused to the critical surface and is launched from the left wall.  The plasma skin depth is $c/\omega_p = 50.3$~nm and $c/\omega_0 = 159.2$~nm.  See Fig.~\ref{fig:laser}(a) for a schematic.

The simulations used periodic boundary conditions in the second dimension ($y$), and the laser was polarized with its electric field in the simulation plane (p-polarized). The simulation dimensions were kept constant in the $y$-direction,  3.2~$\mu$m, and in the $x$-direction were either 923.9 or 1597.8~$\mu$m for truncated and causally separated runs, respectively.  Square cells of size 0.2~$c/\omega_0$ were used, resulting in a simulation domain of 50197$\times$100 cells for the simulation with the largest length in $x$ (29025$\times$100 cells otherwise).  The time step was 0.141~$\omega_0^{-1}$.  The electron (ion) species had 64 (16) particles per cell, and each species used cubic interpolation with an initial temperature of 0.1~keV.  We employed a static load balancing routine\cite{Fonseca2013} at initialization to distribute processing elements in an optimal configuration, and the particle push time was delayed until the laser neared the plasma.

\begin{figure}
\includegraphics[width=\linewidth]{figures/fig-1-schematic.jpeg}
\caption{\label{fig:laser} (a)~Simulation schematic, showing the full box size. The box is truncated at 150~$\mu$m when the absorber is in use. (b)~Laser Poynting flux incident at the plasma critical interface, reflected Poynting flux measured 387~$\mu$m to the left of the critical interface and forward electron energy flux measured over the diagnostic region.  All quantities are synced up in time for better visualization.  Percentages represent integrated energy flux as a fraction of the total incident energy.}
\end{figure}

% \begin{figure}
%     \centering
%     \captionsetup{width=0.98\linewidth}
%      \includegraphics[width=\linewidth]{figures/actual-figures/fig-1-schematic.png}
%     \caption{(a)~Simulation schematic, showing the full box size. The box is truncated at 150~$\mu$m when the absorber is in use. (b)~Laser Poynting flux incident at the plasma critical interface, reflected Poynting flux measured 387~$\mu$m to the left of the critical interface and electron energy flux measured over the diagnostic region.  All quantities are synced up in time for better visualization.  Percentages represent integrated energy flux as a fraction of the total incident energy.}
%     \label{fig:laser}
% \end{figure}

In Fig.~\ref{fig:laser}(b) we show the temporal laser profile, as well as the reflected Poynting flux and
%???how do you define the reflected Poynting flux???
transmitted particle energy flux.  The reflected Poynting flux is calculated by measuring the total Poynting flux 380~$\mu$m before the critical-density interface, then subtracting the known incident laser flux.  Both the Poynting and energy fluxes plotted in Fig.~\ref{fig:laser}(b) are translated in time to line up with the incident laser light.  To diagnose the forward momentum and energy flux deep in the plasma, we choose a diagnostic region 48--64~$\mu$m into the uniform plasma over which we average the particle data in space.  The energy flux is defined as $\int (\gamma-1)m_ec^2 \mathbf{p}/\gamma\,d\mathbf{p}$ for electron mass $m_e$.  In order to avoid particle refluxing from either boundary in the $x$-direction, a 746-$\mu$m vacuum region (computationally inexpensive because of the static load balancing) is placed to the left of the plasma upramp, and the uniform-density plasma is extended to the right a distance of 824~$\mu$m (computationally expensive).  This ensures that any particles reflected from the right boundary region will be causally separated from the diagnostic region for the duration of the simulation (for a time $2\times 760\,\mu$m$/c\approx5$~ps).  The $p_x$-$x$ phasespace is shown in Fig.~\ref{fig:px-x-a} at 3.7~ps after the laser was incident on the critical interface, with the diagnostic region marked by dashed lines.  Note the large size of the plasma required compared to the diagnostic region location, along with the very hot return current reflecting off the right simulation boundary---even though a thermal particle boundary is being used.

\begin{figure}
\includegraphics[width=\linewidth]{figures/fig-2-p1x1-a.jpeg}
\caption{\label{fig:px-x-a} The $p_x$-$x$ phasespace for the causally separated simulation (single run, not averaged).  The dashed lines indicate the diagnostic region, but the plasma has to be much larger in length to be causally separated from the hot return current reflecting off the right boundary.}
\end{figure}

% \begin{figure}
%     \centering
%     \captionsetup{width=0.98\linewidth}
%      \includegraphics[width=\linewidth]{figures/actual-figures/fig-2-p1x1-a.png}
%     \caption{The $p_x$-$x$ phasespace for the causally separated simulation (single run, not averaged).  The dashed lines indicate the diagnostic region, but the plasma has to be much larger in length to be causally separated from the hot return current reflecting off the right boundary.}
%     \label{fig:px-x-a}
% \end{figure}

We ran the simulations until 2~ps after the laser had finished hitting the plasma.  In all cases, hot particles were split into two after reaching a $\gamma$ of 1.4, 1.5, 1.6, 1.7, and 1.8 (i.e., very energetic particles were eventually split into 32 smaller particles); the splitting routine was executed every 10 time steps.  Contact the corresponding author for information about the source code and input files used for these simulations.

The particle acceleration mechanisms in these types of simulations are stochastic; therefore, we expect and indeed do observe large differences in particle statistics due to slightly different simulation configurations.  For example, we performed the causally separated simulation three different times with varied random number seeds and observed a factor of 2--4 variation in particle number in the tail of the momentum distribution over the diagnostic region.  For this reason we performed the simulations presented in this paper three times with different random number seeds.  Unless otherwise noted, visualizations presented here are of data averaged over three different runs; this averaging gives increased confidence that any observed deviations from the causally separated run are due to the particle boundary conditions.


\subsection{Effect of the absorber boundary condition}

To greatly reduce computation time and resources, we desire to shrink the simulation region shown in Fig.~\ref{fig:px-x-a}, but preserve the behavior from the causally separated run.  We truncate the plasma at a distance of 150~$\mu$m (29025 cells in $x$) and vary the length of the absorber, where each absorber is designed to stop all hot particles 5~$\mu$m short of the right boundary.  For all results shown here we use the linearly varying absorber from Sec.~\ref{sec:linear} and calculate the local temperature via Eq.~(\ref{Eq:lin-int}).  We quote the mean free path for each absorber, which as shown in Fig.~\ref{fig:f-and-h-lin} is 26\% of the entire absorber length.  We used an energy threshold of 6 times the local thermal velocity and re-emitted stopped particles at the local temperature.  Stopping was performed every time step for both electrons and ions
%???Why the ions??? Explain that ions had large energy
to give a large number of stopping loops for a fast particle traversing the absorbing region.  Particles are typically stopped every $\sim3$ time steps, but we perform a stopping loop every time step to more accurately assess the different methods.  Though it is much more important to use an absorber for electrons than for ions, we observed a sufficient number of hot ions reaching the thermal boundary to warrant stopping ions as well.  Stopping loops were delayed until hot particles approached the absorber region.  Particle recombination (for electrons) was executed every 5 time steps over the absorbing region; this dramatically reduces the simulation runtime as hot particles that have been split into 32 smaller particles are all stopped over a very short distance.

\begin{figure}
\includegraphics[width=\linewidth]{figures/fig-3-p1x1-all-123.jpeg}
\caption{\label{fig:px-x-1.1} The $p_x$-$x$ phasespace (single runs, not averaged) for the causally separated ($\lambda=\infty$), absorber, and no absorber ($\lambda=0$) simulations 1.1~ps after the incident laser.  A hot reflux of electrons is already shown to be entering the dashed diagnostic region for the truncated run with no absorber.}
\end{figure}

% \begin{figure}
%     \centering
%     \captionsetup{width=0.98\linewidth}
%      \includegraphics[width=\linewidth]{figures/actual-figures/fig-3-p1x1-all-123.png}
%     \caption{The $p_x$-$x$ phasespace (single runs, not averaged) for the causally separated ($\lambda=\infty$), absorber, and no absorber ($\lambda=0$) simulations 1.1~ps after the incident laser.  A hot reflux of electrons is already shown to be entering the dashed diagnostic region for the truncated run with no absorber.}
%     \label{fig:px-x-1.1}
% \end{figure}

The $p_x$-$x$ phasespaces for the causally separated ($\lambda=\infty$, where we are zooming in on a particular region), absorber (with $\lambda=100\,c/\omega_p$) and no absorber/truncated ($\lambda=0$) simulations are shown in Figs.~\ref{fig:px-x-1.1} and \ref{fig:px-x-3.7} at two different times.  After just 1.1~ps, a hot reflux of electrons is visible in the truncated run [see Fig.~\ref{fig:px-x-1.1}(c)] that has already entered the diagnostic region.  These refluxing electrons are seen to completely overwhelm the simulation late in time [see Fig.~\ref{fig:px-x-3.7}(c)], while the simulation with the absorber [see Fig.~\ref{fig:px-x-3.7}(b)] is able to maintain an appropriate return current.  These plots are not averaged over three simulations, so sizeable variations within the pre-plasma are expected for the causally separated run due to differences in random number initialization with a different box size [note that the phasespace in the density upramp and surrounding region are identical in Figs.~\ref{fig:px-x-1.1}(b) and (c)].

\begin{figure}
\includegraphics[width=\linewidth]{figures/fig-3-p1x1-all-210.jpeg}
\caption{\label{fig:px-x-3.7} The $p_x$-$x$ phasespace (single runs, not averaged) for the causally separated ($\lambda=\infty$), absorber, and no absorber ($\lambda=0$) simulations 3.7~ps after the incident laser.  The refluxing electrons for the truncated run have completely altered the particle phasespace; the returning hot electrons cyclically interact with the laser and re-enter the plasma, artificially heating the bulk plasma to a much higher temperature than in the casually separated or absorbing runs.}
\end{figure}

% \begin{figure}
%     \centering
%     \captionsetup{width=0.98\linewidth}
%      \includegraphics[width=\linewidth]{figures/actual-figures/fig-3-p1x1-all-210.png}
%     \caption{The $p_x$-$x$ phasespace (single runs, not averaged) for the causally separated ($\lambda=\infty$), absorber, and no absorber ($\lambda=0$) simulations 3.7~ps after the incident laser.  The refluxing electrons for the truncated run have completely altered the particle phasespace; the returning hot electrons cyclically interact with the laser and re-enter the plasma, artificially heating the bulk plasma to a much higher temperature than in the casually separated or absorbing runs.}
%     \label{fig:px-x-3.7}
% \end{figure}

To better visualize temporal behavior, we plot the electron energy flux in the $x$ direction as a function of time and space for the causally separated, absorber, and no absorber simulations in Fig.~\ref{fig:s1-t}.  For the causally separated simulation, a steady stream of energy flux is observed to the right of the critical-density interface, which is slowly pushed forward in time.  Energetic electrons are also seen to escape to the left as the plasma expands.  This expansion is enhanced after the laser turns off.  When using the absorber with $\lambda=100\,c/\omega_p$, the energy flux looks qualitatively very similar to the causally separated run, except that the energy flux quickly decreases to zero in the absorber region.  In contrast, the truncated simulation ($\lambda=0$) shows that a large fraction of the forward energy flux is reflected from the right boundary (especially visible at 0.8~ps), so much so that it dramatically reduces the overall energy flux as it travels backward.  Once the first reflux arrives back to the laser-plasma interface at around 1.5~ps, the forward energy flux is then permanently altered.  This change in physics, as the hot return current interacts with and is accelerated by the laser, is the primary issue that the absorber is able to eliminate.  Finally, this hot reflux of electrons is also visible in the blue negative energy flux after the laser turns off in the truncated run.

\begin{figure}[htp]
\includegraphics[width=\linewidth]{figures/fig-4-s1-t.jpeg}
\caption{\label{fig:s1-t} Forward particle energy flux as a function of position and time for three different cases.  For the truncated simulation ($\lambda=0$), the forward energy flux can be seen to be neutralized by a refluxing current emitted from the boundary.  The absorber effectively reduces the particle energy flux before the simulation boundary without a reflux current.}
\end{figure}

% \begin{figure}
%     \centering
%     \captionsetup{width=0.98\linewidth}
%      \includegraphics[width=\linewidth]{figures/actual-figures/fig-4-s1-t.png}
%     \caption{Forward particle energy flux as a function of position and time for three different cases.  For the truncated simulation ($\lambda=0$), the forward energy flux can be seen to be neutralized by a refluxing current emitted from the boundary.  The absorber effectively reduces the particle energy flux before the simulation boundary without a reflux current.}
%     \label{fig:s1-t}
% \end{figure}

We also examine energy conservation (fields plus particles) across the simulation region when the absorber boundary condition is in use.  To do this we compute the integral of energy density over a specific domain ($V$) and add the energy flux through the left and right boundaries of that domain ($\partial V$):
\begin{equation} \label{Eq:energy}
    \int_V U\,dV + \oint_{\partial V} \mathbf{S} \cdot d\mathbf{A},
\end{equation}
where $U$ is the energy density [$E^2/8\pi + B^2/8\pi + \sum (\gamma-1)m_e c^2$] and $\mathbf{S}$ is the energy flux [$\mathbf{E}\times \mathbf{B}/4\pi + \sum (\gamma-1)m_ec^2 \mathbf{p}/\gamma$].  We compute a running sum of this value over the simulation time (which should remain at zero) and then divide by the maximum energy present in the simulation box at any given time.  This gives a good measure of the energy conservation of the code, although it is not perfect since we only use data reported every 401 time steps (0.3~ps).  In Fig.~\ref{fig:energy} we plot Eq.~(\ref{Eq:energy}) as a function of time, where the right-hand side of volume $V$ (i.e., the location of $\partial V$ on the right) is given by the $x$ coordinate displayed for an absorber with mean free path $\lambda=100\,c/\omega_p$.  We can see that to the left of the absorber (dashed line), the deviation in the coarsely computed energy conservation is less than 1.4\%.  However, by including the absorber region we see that a large fraction of the energy is steadily removed as energetic particles are stopped.  Once again, it is this extended slowing of the particle beam that allows for an appropriate return current to develop, causing plasma to return back into the main simulation region.

\begin{figure}
\includegraphics[width=\linewidth]{figures/fig-5-energy-conservation-edited.jpeg}
\caption{\label{fig:energy} The scaled deviation in energy conservation [see Eq.~(\ref{Eq:energy})] as a function of time, including all points to the left of a given $x$ value (single run, not averaged).  To the left of the absorbing region, energy is well conserved ($<$1.4\% error), but in the absorbing region energy is steadily removed as particles are stopped.}
\end{figure}

% \begin{figure}
%     \centering
%     \captionsetup{width=0.98\linewidth}
%      \includegraphics[width=\linewidth]{figures/actual-figures/fig-5-energy-conservation-edited.png}
%     \caption{The scaled deviation in energy conservation [see Eq.~(\ref{Eq:energy})] as a function of time, including all points to the left of a given $x$ value (single run, not averaged).  To the left of the absorbing region, energy is well conserved ($<$1.4\% error), but in the absorbing region energy is steadily removed as particles are stopped.}
%     \label{fig:energy}
% \end{figure}

\subsection{Variation of absorber parameters}

As mentioned in Sec.~\ref{sec:absorber} and Appendix~\ref{app:temp}, there are a variety of options for implementing the absorber region. 
%??? We should be clearer???
When determining the energy threshold and re-emission temperature of the stopped particles, we can calculate the background temperature dynamically by weighting the distribution function with the proper velocity to some power,
%or its fourth root 
%to determine the energy threshold and re-emission temperature of stopped particles
or we can simply specify a constant value to use. Using a lower power (such as the fourth root) for the proper velocity will emphasize the bulk over a hot tail; more details are given in Appendix~\ref{app:temp}.  However, using the fourth-root temperature never improved the absorber performance for the simulations shown here, so we calculate the temperature in each cell as given by Eq.~(\ref{Eq:lin-int}).

We can also use the hazard function probability defined in Sec.~\ref{sec:hazard} or the linearly varying probability defined in Sec.~\ref{sec:linear} to stop the particles.  In our tests these two choices produce similar results, but overall the linearly varying absorber maintained the proper response for a longer time.  The main reason for this is that due to the periodicity in $y$, simulations using the hazard-function absorber exhibited a large and increasing transverse temperature in the absorbing region; the hazard-function absorber preferentially stops particles with large forward momentum, allowing energetic particles to stream transversely and for some accelerating/reflecting fields to develop (see last paragraph of Sec.~\ref{sec:hazard}).  For this reason we use the linearly varying absorber in this paper, which stops particles as a function of the magnitude of the velocity and not just the longitudinal component.

% Both the hazard and linearly varying absorber schemes appear to efficiently stop particles.  However, in many of the quasi-1D simulations that we performed, the linearly varying absorber was able to maintain the proper response for longer than the hazard function absorber.  This is because the hazard function relies on particles streaming primarily in one direction.  However, with only 100 cells in the transverse direction, many energetic particles that stream towards the boundary are also very hot in the transverse direction.  Both electrons and ions can develop very large transverse momentum in the absorbing region as they drift backward as part of the return current, and that energy will not be removed by the absorber.  The linearly varying absorber relies on the absolute magnitude of a particle's velocity, and hence any transverse momentum in the absorbing region will be kept small.  For this reason we recommend using the linearly varying absorber with a calculated threshold and re-emission temperature.

We compare a combination of absorbers in Fig.~\ref{fig:variation}, where we show the $p_x$ momentum phasespace for all electrons in the diagnostic region at two different times.  Although all absorber schemes appear to perform equally well early in time, the return current is clearly hotter when constant values of the energy threshold and re-emission temperature are given.  For the static temperature simulation, we set the absorber to stop particles with energy greater than 0.6~keV and to re-emit particles at 0.1~keV (the original plasma temperature); in contrast the dynamic absorber stops particles moving at more than 6 times the locally computed thermal velocity.  Using a static temperature performs poorly because, as seen even in the absorbing region of Fig.~\ref{fig:px-x-3.7}(b), the plasma heats up significantly in response to the energetic electron beam.  Particles stopped and re-emitted at the original temperature are not moving fast enough to provide the necessary return current, and a nonphysical potential develops that accelerates electrons backward with too much energy.  Calculating the local temperature instead allows the absorber to accurately compensate for this dynamic behavior.

Although not shown here, we performed a series of simulations varying the mean free path of the absorber by factors of two between $\lambda=0.1\,c/\omega_p$ and $\lambda=200\,c/\omega_p$.  We observed that if the absorber had a mean free path $\lambda \gtrsim 6\,c/\omega_p$, it was able to closely match the causally separated momentum distribution when averaged over three separate runs.  However, individual simulations with $\lambda \lesssim 20\,c/\omega_p$ exhibited slightly greater variability in comparison to the causally separated data.  In our simulations, an absorber with a mean free path of $6\,c/\omega_p$ performed only $\sim30$ stopping events before nearly all particles were stopped, which was sufficient for a laser 3~ps in duration with $a_0=3$.  However, care must be taken for lasers of longer duration or higher intensity; Fig.~\ref{fig:variation} shows that some absorbers can perform well (a)~initially, but (b)~eventually fail due to the large amount of energetic particles striking the absorber.  Thus $\lambda \gtrsim \bigO(10\,c/\omega_p)$ gives a reasonable estimate of the appropriate mean free path, but the absorber length should be verified for each individual simulation.

% In our simulations the hot electron beam exhibited a temperature of 3~MeV, and the background plasma in the absorbing region had an average thermal velocity of 0.15~$c$.  Equation~(\ref{Eq:mfp}) then gives (ignoring errors from relativistic inaccuracies) a predicted mean free path of $\lambda \approx 80\,c/\omega_p$.  Equation~(\ref{Eq:mfp}) thus gives a reasonable estimate of appropriate mean free path and can be used as a guideline for absorber length---especially for long-time laser-plasma interactions---but a much shorter absorber may be similarly effective depending on the simulation.

% In these simulations, the hot electron beam exhibited a temperature of 3~MeV, and the background plasma in the absorbing region had an average thermal velocity of 0.15~$c$, giving a predicted mean free path (ignoring errors from relativistic inaccuracies) of $\lambda \approx 80\,c/\omega_p$.  In Fig.~\ref{fig:variation} we show the $p_x$ phasespace averaged over electrons in the diagnostic region for a wide range of the mean free path both for (a)~a single set of runs performed and (b)~the sets of three runs averaged together.  For these simulations, when averaged together all absorber lengths appear to perform consistently well, though perhaps greater variance is observed for mean free paths shorter than $\sim100\,c/\omega_p$ in Fig.~\ref{fig:variation}(a).  In addition, our tests using longer lasers showed that absorbers with shorter mean free paths failed at earlier than those with longer mean free paths.  Thus, again, the absorber parameters should be verified for each unique simulation, though Eq.~(\ref{Eq:mfp}) can be used as a guideline for the appropriate mean free path length, especially for long-time simulations.

\begin{figure}
\includegraphics[width=\linewidth]{figures/fig-6-avg-p1.jpeg}
\caption{\label{fig:variation} The $p_x$ phasespace for all electrons in the region 48--64~$\mu$m into the constant-density plasma for various schemes (a)~1.5~ps and (b)~3.7~ps after the laser was incident on the plasma.  Though the performance of all shown absorbers is nearly identical early in time, either using a static temperature threshold and re-emission or using a very short absorber gives improper results later in time.}
\end{figure}

% \begin{figure}
%     \centering
%     \captionsetup{width=0.98\linewidth}
%      \includegraphics[width=\linewidth]{figures/actual-figures/fig-6-avg-p1.png}
%     \caption{The $p_x$ phasespace for all electrons in the region 48--64~$\mu$m into the constant-density plasma for various schemes (a)~1.5~ps and (b)~3.7~ps after the laser was incident on the plasma.  Though the performance of all shown absorbers is nearly identical early in time, either using a static temperature threshold and re-emission or using a very short absorber gives improper results later in time.}
%     \label{fig:variation}
% \end{figure}

% In Sec.~\ref{sec:concept-mfp} we discussed the appropriate mean free path to use for the absorber, given by Eq.~(\ref{Eq:mfp}) as a function of the beam energy and thermal velocity of the background plasma.  We performed various simulations using the linearly varying absorber with a dynamically calculated threshold and re-emission temperature and varied the mean free path of the absorber.  In these simulations, the hot electron beam exhibited a temperature of 3~MeV, and the background plasma in the absorbing region had an average thermal velocity of 0.15~$c$, giving a predicted mean free path (ignoring errors from relativistic inaccuracies) of $\lambda \approx 80\,c/\omega_p$.  In Fig.~\ref{fig:variation} we show the $p_x$ phasespace averaged over electrons in the diagnostic region for a wide range of the mean free path both for (a)~a single set of runs performed and (b)~the sets of three runs averaged together.  For these simulations, when averaged together all absorber lengths appear to perform consistently well, though perhaps greater variance is observed for mean free paths shorter than $\sim100\,c/\omega_p$ in Fig.~\ref{fig:variation}(a).  In addition, our tests using longer lasers showed that absorbers with shorter mean free paths failed at earlier than those with longer mean free paths.  Thus, again, the absorber parameters should be verified for each unique simulation, though Eq.~(\ref{Eq:mfp}) can be used as a guideline for the appropriate mean free path length, especially for long-time simulations.

% We show in Fig.~\ref{fig:p1} the $p_x$ momentum phasespace for all electrons in the diagnostic region (48--64~$\mu$m into the uniform-density plasma) at two different simulation times, with and without the absorber.  A causally separated run is also given to show the ideal response.  We see that if a traditional thermal boundary condition is used instead of the absorbing boundary, a stream of hot electrons reflected from the right simulation boundary comes back through the main plasma body at early times.  These hot electrons eventually arrive at the laser-plasma interface, are again heated, and return as a forward current through the plasma.  This cycle results in an artificially hot plasma due to the improper boundary.  However, good agreement with the causally separated plasma is obtained when the absorbing boundary condition is used.

% Another visualization of this same effect is shown in Fig.~\ref{fig:p1x1}, which shows the $p_x$-$x$ phasespace for a broader range of the simulation space at two separate times.  Without the absorber, a very hot return current is visible in Fig.~\ref{fig:p1x1}(e) at early times, which results in a much hotter plasma overall late in time [Fig.~\ref{fig:p1x1}(f)].  Employing the absorber, however, allows the truncated simulation to maintain the low-temperature return current characteristic of the causally separated simulation.  The beginning of the absorbing region is indicated in Fig.~\ref{fig:p1x1}(c)--(d) by a dashed line, past which hot electrons are seen to gradually be cooled until near the simulation boundary.

% \begin{figure}
%     \centering

%     \captionsetup{width=0.98\linewidth}
%     \begin{subfigure}[b]{0.49\linewidth}
%          \centering
%          \includegraphics[width=\linewidth]{figures/p1x1-0-123.png}
%          \label{fig:p1x1-a}
%      \end{subfigure}
%      \hfill
%     \begin{subfigure}[b]{0.49\linewidth}
%          \centering
%          \includegraphics[width=\linewidth]{figures/p1x1-0-210.png}
%          \label{fig:p1x1-b}
%      \end{subfigure}
%     \begin{subfigure}[b]{0.49\linewidth}
%          \centering
%          \includegraphics[width=\linewidth]{figures/p1x1-20-123.png}
%          \label{fig:p1x1-c}
%      \end{subfigure}
%      \hfill
%     \begin{subfigure}[b]{0.49\linewidth}
%          \centering
%          \includegraphics[width=\linewidth]{figures/p1x1-20-210.png}
%          \label{fig:p1x1-d}
%      \end{subfigure}
%     \begin{subfigure}[b]{0.49\linewidth}
%          \centering
%          \includegraphics[width=\linewidth]{figures/p1x1-00-123.png}
%          \label{fig:p1x1-e}
%      \end{subfigure}
%      \hfill
%     \begin{subfigure}[b]{0.49\linewidth}
%          \centering
%          \includegraphics[width=\linewidth]{figures/p1x1-00-210.png}
%          \label{fig:p1x1-f}
%      \end{subfigure}
%     \caption{The $p_x$-$x$ phasespace for electrons in the broader simulation space at two different times for simulations that are (a)--(b) causally separated (actual right simulation boundary extended to 824~$\mu$m), (c)--(d) with the absorber (dashed line shows start of absorbing region) and (e)--(f) without the absorber.  Note that when truncating the simulation space without the absorber, a hot return current is present at early times, which translates to a much hotter overall plasma late in time.}
%     \label{fig:p1x1}
% \end{figure}

% \begin{figure}
%     \centering

%     \captionsetup{width=0.98\linewidth}
%     \begin{subfigure}[b]{0.98\linewidth}
%          \centering
%          \includegraphics[width=\linewidth]{figures/avg-lin-haz-runs-p1-135.png}
%          \label{fig:lin-haz-p1-a}
%      \end{subfigure}
%      \begin{subfigure}[b]{0.98\linewidth}
%          \centering
%          \includegraphics[width=\linewidth]{figures/avg-lin-haz-runs-p1-210.png}
%          \label{fig:lin-haz-p1-b}
%      \end{subfigure}
%     \caption{The $p_x$ phasespace for all electrons in the region 48--64~$\mu$m into the constant-density plasma (a)~1.47~ps and (b)~3.72~ps after the laser was incident on the plasma for the two different absorber schemes.  As for temperature calculation, performance is seen to be significantly worse when constant values are used.}
%     \label{fig:lin-haz-p1}
% \end{figure}

% \begin{figure}
%     \centering

%     \captionsetup{width=0.98\linewidth}
%     \begin{subfigure}[b]{0.98\linewidth}
%          \centering
%          \includegraphics[width=\linewidth]{figures/not-avg-2-lambda-p1-210.png}
%          \label{fig:lambda-p1-a}
%      \end{subfigure}
%      \begin{subfigure}[b]{0.98\linewidth}
%          \centering
%          \includegraphics[width=\linewidth]{figures/avg-lambda-p1-210.png}
%          \label{fig:lambda-p1-b}
%      \end{subfigure}
%     \caption{The $p_x$ phasespace for all electrons in the region 48--64~$\mu$m into the constant-density plasma 3.72~ps after the laser was incident on the plasma as a function of stopping distance for (a)~one particular set of runs and (b)~all runs averaged together.  Some greater deviations from the causally separated spectrum are observed for mean free paths shorter than $\sim100\,c/\omega_p$, though all runs seem to perform about equally when averaged together.}
%     \label{fig:lambda-p1}
% \end{figure}

\subsection{Best practices}

Here we make a few notes on best practices for performing simulations with the extended absorbing boundary condition.  We found it important to also causally separate the vacuum boundary (where the laser is injected) from the laser-plasma interface.  Even with absorbing particle boundary conditions at this vacuum boundary, most energetic particles that reached the vacuum boundary were immediately reflected back into the simulation space.  This is a combined effect of the laser potential at the wall and the electric field buildup from exiting particles (a nonnegligible number of particles are accelerated toward the laser from the pre-plasma region).  Refluxing from the vacuum boundary leads to a modified distribution at the laser-plasma interaction region, which then artificially inflates the forward electron energy flux in the target.

% In addition, for simulations 
% %of a target with finite size in the transverse direction, 
% with a finite width laser the boundary in the transverse directions also need to 
% %we advise extending out the vacuum boundary to be 
% be large enough causally separate itself from the interaction region.
% %that dimension as well.

For simulations with a finite-width laser, absorber regions can also be placed at the transverse simulation edges to correctly handle the large flux of relativistic electrons expelled transversely from the laser spot.  However, the effectiveness of the absorber relies on having a large number of particles in each cell (for calculating the temperature).  If absorbers are placed at the transverse simulation boundaries, they may overlap with near-vacuum regions in and before the pre-plasma.
%???Not sure I understand what you are trying to say????
Thus for finite-size-target simulations with multiple absorbers, we found it is useful to transition the absorbers positioned along the transverse boundaries to stop and re-emit particles based on a static (rather than dynamically calculated) temperature in those near-vacuum regions.

Finally, the start of the absorber region should be located a reasonable distance away from where accurate plasma measurements are expected.  For example, when comparing Figs.~\ref{fig:px-x-3.7}(a) and \ref{fig:px-x-3.7}(b), the phasespace immediately in front of the absorbing region in (b) does not exactly mimic the causally separated phasespace in (a).  Examining the particle phasespace for irregularities near the absorber region can help determine the appropriate distance at which to measure plasma quantities.

\subsection{Future work}

The implementation described here, though effective, is by no means a comprehensive treatment or unique solution to the reflux problem.  Here we list some ideas that could be used to iterate on our proposed solution.  Particles could be re-emitted from a distribution that is hotter in the return direction than in the forward direction, assisting in establishing the appropriate return current.  Particles could be stopped preferentially based on their direction of motion.  We employed absorbers for both ions and electrons in these simulations, but the ion response and stopping could be explored in greater detail for long-time simulations.  Alternatives that are more computationally expensive could include applying a drag force to energetic particles over the length of the entire absorber or calculating the re-emission temperature from a position located before the absorber region.  Last, it may be possible to develop a thermal bath boundary where particles are re-emitted from a distribution determined from a region somewhere inside the plasma.

\end{document}
% \input{5-discuss.tex}
\section{Conclusions and Future Work}\label{section-conclusion}
In this work, we have systematically studied different key notions and results concerning anti-unification of unordered goals, i.e. sets of atoms. We have defined different anti-unification operators and we have studied several desirable characteristics for a common generalization, namely optimal cardinality (lcg), highest $\tau$-value (msg) and variable dataflow optimizations. For each case we have provided detailed worst-case time complexity results and proofs. An interesting case arises when one wants to minimize the number of generalization variables or constrain the generalization relations so as they are built on injective substitutions. In both cases, computing a relevant generalization becomes an NP-complete problem, results that we have formally established.
In addition, we have proven that an interesting abstraction -- namely $k$-swap stability which was introduced in earlier work -- can be computed in polynomially bounded time, a result that was only conjectured in  earlier work. 

Our discussion of dataflow optimization in Section~\ref{section-relation-2} essentially corresponds to a reframing of what authors of related work sometimes call the \textit{merging} operation in rule-based anti-unification approaches as in~\cite{Baumgartner2017}. Indeed, if the "store" manipulated by these approaches contains two anti-unification problems with variables generalizing the same terms, then one can "merge" the two variables to produce their most specific generalization. If the merging is exhaustive, this technique results in a generalization with as few different variables as possible. In this work we isolated dataflow optimization from that specific use case and discussed it as an anti-unification problem in its own right.

While anti-unification of goals in logic programming is not in itself a new subject, to the best of our knowledge our work is the first systematic treatment of the problem in the case where the goals are not sequences but unordered sets. Our work is motivated by the need for a practical (i.e. tractable) generalization algorithm in this context. The current work provides the theoretical basis behind these abstractions, and our concept of $k$-swap stability is a first attempt that is worth exploring in work on clone detection such as~\cite{clones}. 

Other topics for further work include adapting the $k$-swap stable abstraction from the $\preceq^\iota$ relation to dealing with the $\sqsubseteq^\iota$ relation. 
A different yet related topic in need of further research is the question about what anti-unification relation is best suited for what applications. For example, in our own work centered around clone detection in Constraint Logic Programming, anti-unification is seen as a way to measure the distance amongst predicates in order to guide successive syntactic transformations. Which generalization relation is best suited to be applied at a given moment and whether this depends on the underlying constraint context remain open questions that we plan to investigate in the future. 

%The main results of this paper are the polynomial algorithms solving specific anti-unification problems, along with several worst-case time complexity results and proofs. 

% have made efforts to extend the classical anti-unification concepts to the case where the artefacts to generalize are unordered goals. We have done this by considering different levels of atomic abstraction through different generalization relations. W





%Throughout the paper, we have introduced four generalization relations. Figure~\ref{fig-interconnexion} shows how the four relations are linked on a conceptual level. $\sqsubseteq$ is the most general relation as generalization is defined with any substitution. Restricting the definition to injective substitutions or to renamings yields more specific relations, the intersection of which is relation $\preceq^\iota$ where variables are generalized through injective renamings. 

%\begin{figure}[htbp]
%	\begin{center}
%		\begin{tikzpicture}[x=0.75pt,y=0.75pt,yscale=-1,xscale=1]
%		%uncomment if require: \path (0,300); %set diagram left start at 0, and has height of 300
%		
%		%Shape: Ellipse [id:dp330479544492589] 
%		\draw   (150.05,144.64) .. controls (150.05,72.29) and (186.09,13.64) .. (230.55,13.64) .. controls (275,13.64) and (311.05,72.29) .. (311.05,144.64) .. controls (311.05,216.99) and (275,275.64) .. (230.55,275.64) .. controls (186.09,275.64) and (150.05,216.99) .. (150.05,144.64) -- cycle ;
%		%Shape: Ellipse [id:dp3140532351606715] 
%		\draw   (165,87.62) .. controls (165,64.99) and (193.75,46.64) .. (229.22,46.64) .. controls (264.69,46.64) and (293.45,64.99) .. (293.45,87.62) .. controls (293.45,110.26) and (264.69,128.61) .. (229.22,128.61) .. controls (193.75,128.61) and (165,110.26) .. (165,87.62) -- cycle ;
%		%Shape: Ellipse [id:dp9580468020324391] 
%		\draw   (261.25,53.46) .. controls (285.77,54.29) and (304.38,90.37) .. (302.81,134.06) .. controls (301.24,177.74) and (280.08,212.49) .. (255.56,211.67) .. controls (231.03,210.85) and (212.43,174.76) .. (214,131.08) .. controls (215.57,87.39) and (236.72,52.64) .. (261.25,53.46) -- cycle ;
%		
%		% Text Node
%		\draw (172,266) node   {$\sqsubseteq $};
%		% Text Node
%		\draw (235,216) node   {$\preceq $};
%		% Text Node
%		\draw (170,125) node   {$\sqsubseteq^\iota $};
%		% Text Node
%		\draw (254,92) node   {$\preceq^\iota $};
%		\end{tikzpicture}
%	\end{center}
%	\caption{The interconnexions of four generalization relations}
%	\label{fig-interconnexion}
%\end{figure}

%Figure~\ref{fig-interconnexion} shows how the four relations are linked on a conceptual level. When needed in concrete applications, the right generalization operator (or an abstraction) should be used; this of course depends on whether or not the atomic structure should be generalized and the variable dataflow preserved. 

%Future work will focus on the use of such generalization operators in the purpose of applying synctatic transformations on predicates in such a way that the structural distance between them decreases; such a synctatic distance can be evaluated over the most specific generalization of the predicates under scrutiny.

% \section{Citations}
% \label{sec:citations}

% 	Citations can be made using either \textbackslash citep\{\} or \textbackslash citet\{\}, depending from the appropriateness. To avoid the citation moving to the next line, it is often a good practice to replace the space before with a tilde (\~{}) character.
% 	Example 1: ``CoRL is the best conference ever, as discussed in~\citep{Calandra2016}.``
% 	Example 2: ``\citet{Calandra2016} proved, both theoretically and numerically, that CoRL is the best conference ever.``
	
% The maximum paper length is 8 pages excluding references and acknowledgements, and 10 pages including references and acknowledgements

\clearpage
% The acknowledgments are automatically included only in the final version of the paper.
% \acknowledgments{If a paper is accepted, the final camera-ready version will (and probably should) include acknowledgments. All acknowledgments go at the end of the paper, including thanks to reviewers who gave useful comments, to colleagues who contributed to the ideas, and to funding agencies and corporate sponsors that provided financial support.}

%===============================================================================

% no \bibliographystyle is required, since the corl style is automatically used.
\bibliography{references}  % .bib

\end{document}
