\documentclass{article}
\pdfminorversion=5 
\pdfcompresslevel=9
\pdfobjcompresslevel=2

\usepackage{corl_2019} % initial submission
% \usepackage[final]{corl_2019} % Uncomment for the camera-ready ``final'' version
\usepackage{booktabs}
\usepackage{caption}
\usepackage{subcaption}
\usepackage{graphicx}
\usepackage{amsmath}

\newcommand{\todo}[1]{\textcolor[rgb]{0.1,0.5,0.1}{\small {~[TODO]~}{#1}}}
\newcommand{\meta}[1]{\textcolor[rgb]{0.2,0.2,0.7}{\small {~[META]~}{#1}}}
\newcommand{\comment}[1]{\textcolor[rgb]{0.2,0.2,0.7}{\small {~[COMMENT]~}{#1}}}

\newcommand{\fig}[1]{Fig.~\ref{#1}}
\newcommand{\tbl}[1]{Table~\ref{#1}}
\newcommand{\algo}[1]{Algorithm~\ref{#1}}
\newcommand{\sect}[1]{Sec.~\ref{#1}}
\newcommand{\etal}[0]{{\em et al.~}}
\newcommand{\eg}[0]{{\em e.g.,~}}
\newcommand{\ie}[0]{{\em i.e.,~}}
\newcommand{\etc}[0]{{\em etc.\xspace}}

\title{Transfer Learning for Multi-modal Perception on Grasping Transparent and Specular Objects}

% The \author macro works with any number of authors. There are two
% commands used to separate the names and addresses of multiple
% authors: \And and \AND.
%
% Using \And between authors leaves it to LaTeX to determine where to
% break the lines. Using \AND forces a line break at that point. So,
% if LaTeX puts 3 of 4 authors names on the first line, and the last
% on the second line, try using \AND instead of \And before the third
% author name.

% NOTE: authors will be visible only in the camera-ready (ie, when using the option 'final'). 
% 	For the initial submission the authors will be anonymized.

\author{
  Thomas Weng\\
%   Carnegie Mellon University\\
  \texttt{tweng@cmu.edu} \\
  %% examples of more authors
%   \And
%   Amith Pallankize\\
%   Affiliation \\
%   Address \\
%   \texttt{email} \\
  \AND
  David Held\\
%   Carnegie Mellon University\\
  Address \\
  \texttt{email} \\
  %% \And
  %% Coauthor \\
  %% Affiliation \\
  %% Address \\
  %% \texttt{email} \\
  %% \And
  %% Coauthor \\
  %% Affiliation \\
  %% Address \\
  %% \texttt{email} \\
}


\begin{document}
\maketitle

%===============================================================================

\begin{abstract}
State-of-the-art object grasping methods rely on depth sensing to plan robust grasps, but commercially available depth sensors fail to detect transparent and specular objects.
To reach comparable grasping performance on such objects, we introduce a method for learning a multimodal perception model by bootstrapping from an existing uni-modal model. 
This transfer learning approach requires only paired multi-modal data for training, and requires no ground-truth labels nor real grasp attempts.
Our experiments show that our method outperforms depth-based models on transparent, specular, and opaque objects.
\end{abstract}

% Two or three meaningful keywords should be added here
\keywords{Multimodal perception, transfer learning, grasping} 

\section{Motivation and Objectives}
%Anti-unification is an important topic in Logic Programming. It refers to the process of computing some expression $G$, called a generalization, that captures common structure amongst a set of syntactic expressions $E$. In this work, we study the anti-unification of logical \textit{goals}, a goal being a set of atomic structures. To simplify our approach the set $E$ will be composed \textit{two goals}, but our results can easily be extended to the more general case where any number of goals need to be generalized. 

Anti-unification refers to the process of generalizing two (or more) program objects $S$ into a single, more general, program object that captures some of the structure that is common to all the objects in $S$. In a classical logic programming context, the atom $p(X,Y)$ can thus be seen as a generalization of both the atoms $p(f(A), U)$ and $p(f(g(B)),h(C))$, thanks to the variables $X$ and $Y$. 

Anti-unification constitutes a useful tool in various contexts ranging from program analysis techniques (including partial evaluation, refactoring, automatic theorem proving, program transformation, formal verification and test-case generation~\cite{au-applications,calculus-constr,DESCHREYE1999231,lg-gs,under-implication}) to automated reasoning \cite{ilp-theory-and-methods,Muggleton90efficientinduction} or analogy making~\cite{analogy-making}, supercompilation~\cite{Sorensen95analgorithm} and even plagiarism detection~\cite{clones}. Many of these static techniques are executed on programs written in the form of (constraint) Horn clauses, a formalism that has been praised for its ability to capture a program's essence in a quite universal and straightforward manner~\cite{horn-clauses-intermediate-representation}. 

In the introductive example above, the presence of variables $X$ and $Y$ conceptually allows concrete instances (i.e. less general objects) to harbor any value at the positions corresponding to the variable positions. The generalization process is indeed usually achieved by ``forgetting'' parts of the objects to generalize (either by replacing sub-objects with variables or by dropping them altogether): the less syntactic information in an object, the more general it is. Most anti-unification methods are thus steered by a \textit{variabilization} algorithm determining how to ``forget'' object parts when necessary while keeping (common) parts in the generalization. Therefore, in general one is typically interested in computing what is often called a most specific generalization (or synonymously least general generalization), that is a generalization that captures a maximal amount of shared structure. With the atoms of the example above, the common generalization $p(f(X), Y)$ is in that regard a \textit{better} anti-unification result than $p(X,Y)$, as it exhibits more common structure (namely the use of functor $f$). As this example hints, ``better'' results are often obtained at the cost of more complex anti-unification algorithms. In that regard, computing more specific generalizations often boils down to performing some kind of optimization in the variabilization process. 

In a classical approach where goals are \textit{ordered} sequences of atoms, a goal $G$ is more general than some other goal $G'$ if $G'$ can be obtained by applying on $G$ some substitution $\theta$, being a mapping from variables to values. $G$ then typically harbors more variables than $G'$, making it a less instantiated, thus more general, version of $G'$. In that case, $G$ and $G'$ are related by the $\theta$-subsumption relation from~\cite{plotkin}, often considered to be a foundation of Inductive Logic Programming where anti-unification is used as a way to learn a general hypothesis from specific examples~\cite{ilp-theory-and-methods}. As the name may suggest, looking for a generalization that is common to a group of program artefacts (be it terms, atoms, goals or even predicates as a whole) is referred to as anti-unification due to it being the dual operation of unification. Both can, in fact, be applied in similar contexts. Such applications of (anti-)unification include program transformation techniques for partial deduction \cite{Gallagher:1993:TSL:154630.154640,DESCHREYE1999231}, fold/unfold routines \cite{DBLP:journals/csur/PettorossiP98}, invariant generation~\cite{DBLP:conf/synasc/KovacsJ05} and reuse of proofs~\cite{unranked-2-order-au,calculus-constr}. 

The study of anti-unification so far has mainly been focused on such ordered goals. However, many applications require goals to be defined as (\textit{unordered}) sets of atoms. It is the case, for instance, when considering the most declarative semantics of logic programs~\cite{lp-semantics,clp-semantics,horn-clauses-intermediate-representation}. Having a clear overview of anti-unification operators computing most specific generalizations for unordered goals (sometimes called \textit{linear} generalizations) in logic programs is necessary for generalization-driven semantic
% todo our own
clone detection with programs composed of constraint Horn clauses~\cite{clones,DBLP:conf/ppdp/MesnardPV16}. Indeed, generalization operators allow to quantify a certain amount of structural similarity between different predicate definitions by highlighting what parts these have in common. In~\cite{clones}, this quantitative similarity measurement is used as an indication of which semantic-preserving program transformation should be applied next in order to ultimately assess whether two programs (or predicates) are semantic clones. A quite similar approach has already been taken in the case of ordered goals in~\cite{au-applications}, an obvious application of this being plagiarism detection. 

Directing our interest towards unordered goals also has the advantage of broadening the traditional anti-unification theories usually rooted in a setting where logic programming is based on operational semantics, by extending the theories to the more general area of Constraint Logic Programming (CLP), unordered goals being a crucial ingredient of the CLP(X) framework. The fixpoint semantics of CLP programs are indeed typically defined with no regard to the order of appearance of the atoms in a clause's body~\cite{clp-semantics}. While CLP is interesting in its own right, it is also considered a serious candidate for representing abstract \textit{algorithmic knowledge}, rather than mere computations, in a quite universal manner~\cite{horn-clauses-intermediate-representation}. In that regard, focusing on unordered goals could pave the way for performing anti-unification at the algorithmic level rather than at the level of language-specific operations. 

The topic of anti-unification in the case of unordered goals has ocasionally come up in studies focussed on related fields such as \textit{equational} anti-unification, encompassing theories specified by commutativity or associative-commutativity axioms. The topic has been treated for first-order theories~\cite{order-sorted} as well as higher-order variants~\cite{kutsia_2020}. The latter work applies to the first-order case as well and provides polynomial algorithms for variants of anti-unification for unordered input. A grammar-based approach to equational anti-unification including commutative theories, called E-generalization, was introduced in~\cite{e-generalization} and refined with a working implementation in~\cite{e-generalization-improved}. The authors of~\cite{unranked-2-order-au} elaborate a \textit{rigid anti-unification} algorithm that can apply to unordered (and so-called \textit{unranked}) theories by instantiating a parameter called rigidity function, a direct application of which being the computation of longest common substrings. The algorithms described in all of these works can be used to compute what we will call $\sqsubseteq$-common generalizations below in the present paper. Although none of these works develop a general (non-equational) taxonomy allowing to extend the results beyond that simple setting, nor discusses variable- or injectivity-based variants of anti-unification operators, %-- both being concepts that will show central in the present paper -- 
their usages do point out other interesting (and recent) applications of anti-unification when focused on unordered goals, namely detection of recursion schemes in functional programs (as explained in~\cite{BARWELL2018669}) and techniques for learning bugfixes from software code repositories (an example being~\cite{rolim2018learning}). 

Anti-unification techniques that are adapted for CLP(X) have been defined in~\cite{gen}, but its focus is set on a polynomial abstraction procedure for a specific case where terms cannot be generalized (only variables can) and where generalization has to be carried out through injective substitutions. 
%
While~\cite{gen} provides useful insights and results, it lacks a more general and in-depth study of the used generalization operator. In this work we broaden, generalize and complete the latter work by providing a detailed and systematic study of generalization operators and their characteristics in the context of CLP. 
%

The main contributions of the present work are the following. In Section~\ref{section-preliminaries} we define relations close to the well-known $\theta$-subsumption in an effort of adapting this notion to the case of unordered goals. As will be illustrated throughout the paper, our adaption of anti-unification to unordered goals makes the usual subsumption techniques unusable. In Section~\ref{section-relation-1} we reframe the problem of looking for a most general/largest generalization as an optimization problem, parametrized by the \textit{generalization operator} (or anti-unification strategy) and \textit{variabilization function} (responsible for introducing variables in the resulting generalization) at hand.  We will see that given two unordered goals as input, searching for such generalizations can be done in polynomial time. The algorithms, as well as their worst-case time complexities, are detailed throughout the development of our anti-unification framework. 
%We study and characterize problem statements 
%and related algorithms, as well as their computability, 
%for several incarnations of the anti-unification problem in this setting. Indeed, 
%This new approach constitutes an in-depth study of anti-unification in the presence of unordered goals. 
%Its novelty comes from two main aspects. First, the definition of a general framework in which 
 In Section~\ref{section-relation-2} we provide an in-depth examination of several key variations of the anti-unification problem, namely variable generalization (where no terms are allowed to be generalized), injective generalization (where the generalizing substitutions need to be injective) and dataflow optimization (where the number of generalizing variables needs to be minimized) -- the latter of which is proved to make the anti-unification statement NP-hard. Finally, addressing this last problem more in depth in Section~\ref{section-relation-3} we revisit a tractable abstraction that was introduced in~\cite{gen} but we provide for the first time a formal proof of its worst-case complexity, showing that the approximation can effectively be computed in polynomially bounded time. With the exception of this last result, the proofs of propositions, lemmas and theorems are provided in the Appendices.


%In this work, we complete the work of~\cite{gen} by providing 
% -- a setting which corresponds to one of the few generalization contexts that we aim to further develop hereunder. The present work can, as such, be seen as a continuation of~\cite{gen}. To complete the claims of that related work we will also prove one of its results which, to our knowledge, has remained a conjecture so fa@r.



%The remainder of the paper is organized as follows. Some of the main concepts are formally introduced in Section~\ref{section-preliminaries}, where we provide a definition for most specific generalization and largest common generalization in our context. We then see in Section~\ref{section-relation-1} that given two unordered goals as input, searching for such generalizations can be done in polynomial time. The algorithms, as well as their worst-case time complexities, are detailed throughout the development of our anti-unification framework.
%Then, in Section~\ref{section-relation-2}, we will address \textit{dataflow optimization}, the process of minimizing the number of variables introduced in the anti-unification operation. We show this problem to be NP-hard, even when the generalization relation at hand is built upon injective substitutions rather than ordinary substitutions. 
%%
%In Section~\ref{section-relation-3} we  
%further study the injectivity-based anti-unification of unordered goals. 
% based on results that have been exposed in related work in that specific context~\cite{gen}. 
% as a continuation of related work 
%We revisit a tractable abstraction that was introduced  in~\cite{gen} but we provide for the first time a formal proof of its worst-case complexity, showing that the abstraction can effectively be computed in polynomially bounded time. We conclude in Section~\ref{section-conclusion}.
\section{Related Works}

\paragraph{Self-supervised Learning} reviously, many works\cite{ViT, DeiT} relied on abundant labeled datasets to achieve promising results. However, annotating this data requires a large number of human labors. Therefore, how to effectively capture useful semantics embedded in the abundance of data available on the Internet is currently a hot topic.
In recent times, self-supervised learning has witnessed tremendous growth in computer vision, following remarkable achievements in natural language processing. These methods cater to diverse inputs, including images\cite{simclr,BEiT,MAE,ibot}, videos\cite{2022MoQuad,hu2021contrast}, and multi-modality inputs\cite{CLIP,2021LearningTBP}. They capture rich semantic information by creating effective proxy tasks, such as contrastive learning and masked image modeling, in large amounts of unlabeled data. In comparison to supervised learning\cite{DeiT,DeiT-v2,DEiT-v3}, these self-supervised learning approaches have gradually outperformed them in numerous downstream tasks and possess immense potential to become the principal pre-training paradigm.

\paragraph{Feature Pyramid} Utilizing multi-level features has been extensively studied in previous years, and one of the most famous applications is the Feature Pyramid Network (FPN)\cite{FPN}. This technique has been widely used in many dense tasks such as object detection and semantic segmentation to improve the model's perception of objects of different scales. Incorporating FPN into existing designs in many works\cite{maskrcnn, upernet} has led to significant improvements. However, the multi-level feature fusion module only accepts features of different scales as input, limiting its adaptation to isotropic architectures such as ViT\cite{ViT}, in which features from different layers are of the same scale. In masked image modeling, most approaches choose ViT as their encoder due to the masked patch prediction task. Therefore, there are few works exploring multi-level feature fusion in this domain. Even though some works~\cite{ConvMAE,itpn} aim to explore multi-level fusion in masked image modeling, their applications are still limited to the traditional hierarchical architecture and do not address the issue of being biased toward low-level details for these pixel-based methods.

\section{Method}
\label{sec:method}
We introduce the main model architecture in this section. Each subsection forms the foundation for the next one. Section~\ref{sec:din} describes an attention based network inspired by \cite{zhou2018deep} that exploits the correlation between the current ranking task and the entire historical sequence of items for which the user has expressed interest. Section~\ref{sec:rnn} details an elaborate Recurrent Neural Net backbone that can efficiently handle a batch of uneven sized sequences, and how it is used to capture more recent user interactions with the search engine. The output embedding of the attention network is simply fed as an input into every timestamp of the recurrence. Finally we discuss how to build an actor-critic style reinforcement learning model on top of the RNN structure in section~\ref{sec:ddpg}. 

\subsection{Attention for Long-Term Session Sequence}
\label{sec:din}
As mentioned in section~\ref{subsec:rw:user}, the attention network in the Deep Interest Network (DIN) model is a natural way to incorporate user history into personalized recommendation. To adapt to the search ranking setting, we introduce Search Ranking Deep Interest Network (DIN-S), which makes the following adjustments on top of DIN:
\begin{itemize}
    \item Query side features are introduced alongside the focus item features, to participate in attention with historical item sequence.
    \item To account for the possibility that the current query request is not correlated with any of the user's past actioins, a zero score is appended to the regular attention scores before getting the softmax weights. This is illustrated by the zero attention unit in Figure~\ref{fig:din}.
\end{itemize}https://www.overleaf.com/project/5fed74e5fac2600586cb62da
% The only difference in personalized search is to add query related features to the attend list. 
\begin{figure}
    \centering
    \includegraphics[width=\linewidth]{DIN.png}
    \centering
    \caption{DIN-S architecture.}
    \label{fig:din}
\vspace{-5pt}
\end{figure}

The overall architecture of DIN-S is outlined in Diagram~\ref{fig:din}. Due to the nature of algorithmic iterations within an industrial setting, DIN-S is not only one of our quality comparison baselines, but also one of the major components in our proposed final architecture.
We have also tried DIEN \cite{zhou2019deep} and other follow up works. Despite the better results reported in papers, we found little incremental improvement in our own systems. However our design of the RNN backbone model shares some similarity with the approach in DIEN, and indeed both ours and the DIEN work use attention and RNN together. A key difference, however, is that our RNN training data and algorithmic design uses all features available in previous interactions by the user, including both item or query/user one-sided as well as two-sided features. By contrast, the RNN (GRU) in DIEN appears to be just an extension of the co-existing Attention network, taking mostly item-side only categorical features.

\subsection{RNN for Near-Term Sequential Search}
\label{sec:rnn}
In order to compare with the baseline DIN-S (attention + MLP) model fairly and conveniently, we build a so-called RNN backbone that can wrap around any base model architecture. The logic is outline in the bottom half of Diagram~\ref{fig:rnn}. To summarize, for any base model $M$, the RNN backbone introduces a new feature vector $H_t$, the hidden state, concatenated to the output of $M$. The output of each time iteration of the RNN model is another vector $H_{t+1}$, which serves both as the input to downstream networks, as well as the hidden state input for the next time iteration. 

\subsubsection{Contiguous Session-Based Data Format}
\label{subsubsec:data_format}
While DIN-S can be trained in a pointwise / pairwise fashion, our implementation of RNN tries to pool all relevant information together in the data by 
\begin{itemize}
    \item arranging all items within a query session in a single training example. In our case we used tsv (tab-separated values) format. Thus the number of columns in each row is variable, depending on the number of items under the session.
    \item placing all query sessions belonging to the same user contiguously to ensure they are loaded altogether in a mini-match.
\end{itemize}
\begin{figure}
    \centering
    \includegraphics[width=\linewidth]{DDPG-UserSessionTSV.png}
    \centering
    \caption{User Session Input Format. $B$ stands for batch size. Each row represents a single TSV row in the input data. The numbers of columns = number of query features + num of items $\times$ number of item features.}
    \label{fig:user_session_tsv}
\vspace{-5pt}
\end{figure}
As illustrated in Figure~\ref{fig:user_session_tsv}, each mini-batch thus contains a 4d tensor $B$ whose elements are indexed by $(u, t, i, f)$, which stand for users, sessions (time-ordered), items, and features respectively. We assume that features are all dense or have been converted to fixed width dense format, through either embedding sum-pooling or 
attention-pooling from the DIN-S base model. We will use $B_{u, t}$ to denote the 2d slices of $B$ containing all $(i, f)$ values. 
% \vspace{-20pt}

\subsubsection{RNN Model Architecture}
\begin{figure}
    \centering
    \includegraphics[width=\linewidth]{RNN.png}
    \centering
    \caption{RNN architecture.}
    \label{fig:rnn}
\end{figure}
Let $\omega_{u, t}, H_{u, t}$ stand for the regular output and hidden state output of the RNN network for user $u$ and session $t$. The RNN network can thus be described by a function $F$ with the following signature
\begin{align} \label{eq:rnn_kernel}
    (\omega_{u, t+1}, H_{u, t+1}) = F(B_{u, t + 1}, H_t).
\end{align}
This is the most general form of an RNN network. All RNN variants such as LSTM, GRU obey the above signature of $F$.

Recall now $B_{u, t}$ is a 2d tensor, with dimension given by (the number of items, number of features). The same is true of the output tensor $\omega_{u,t}$. The hidden state $H_{u, t}$ however has \textbf{no item dimension}: it is a fixed width vector for a given user after a given session. For simplicity, our choice of $F$ simply computes $H_{u, t}$ as a weighted average of the output $\omega_{u, t}$. More precisely,
\begin{align}
    \omega_{u, t+1}, S_{u, t+1} &= \rm{GRU}(\omega_{u, t}, S_{u, t}) \\
    H_{u, t+1, f} &= \frac{1}{|\cC_{u, t}|} \sum_{j \in \cC_{u, t}} \omega_{u, t+1, j, f}. 
\end{align}
Here $\cC_{u, t}$ stands for the set of items in user $u$'s session $t$ that were purchased. Those sessions without purchases are excluded from our training set, since under the above framework, 
\begin{enumerate}
    \item the user hidden state would not be updated;
    \item the final pairwise training label contains no information.
\end{enumerate} 
The vast majority of the remaining sessions contain exactly 1 purchased item.

This completes the description of the RNN evolution of the state vector under a listwise input format, where all items in a session are used. For training efficiency, however, we adopt a pairwise setup, where 2 items are sampled from each query session, and exactly one of them has been purchased. Thus we can think of each session as consisting of exactly 2 items. Since the hidden state is a weighted average over only the purchased items, pairwise sampling preserves all the information for the hidden state vector in a single RNN step, provided the session contains only a single purchased item, which is more than $90\%$ of the cases.

Lastly, the RNN model outputs a single logit $\eta_{u, t, i}$ for each item $i$ chosen within the user session $(u, t)$, by passing the RNN output vector $\omega_{u, t, i} \in \R^d$ of dimension $d$ through a multi-layer perception $P$ of dimensions $[1024, 256, 64, 1]$:
\begin{align} \label{eq:rnn_output}
    \eta_{u, t, i} = P(O(u, t, i)),  \quad P: \R^d \to \R.
\end{align}
The corresponding label is a binary indicator $\lambda_{u, t, i} \in \{1, 0\}$, which denotes whether the item was purchased or not.

\subsubsection{Pairwise Loss} \label{subsec:rnn:pairwise}
Unlike clicks or mouse hover actions, each page session in e-commerce search typically receives at most one \textbf{purchase}. Thus we are confronted with severe positive and negative label imbalance. To address this problem, we choose pairwise loss in our modeling, which samples a purchased item from the current session at random, and matches it with a random item that is viewed or clicked but not purchased. 

The exact sampling procedure is described in Algorithm~\ref{alg:pairwise_sampling}. Note that as long as the session is non-empty, the procedure will always output a pair. There are occasional edge cases when all items are purchased, in which case we output two purchased items. Alternatively, such perfect sessions can be filtered from the training set.

\begin{algorithm}[t]
\caption{Pairwise sampling from a query session.}
\label{alg:pairwise_sampling}
\begin{algorithmic}[1]
\REQUIRE{a list of $N > 0$ labels: $\lambda_1, \ldots, \lambda_N \in \{0, 1\}$}
\ENSURE{two indices $1 \leq a, b \leq N$, s.t. $\lambda_a = \max \lambda_i$ and $\lambda_b = \min \lambda_i$}
\STATE Compute $\lambda_{\min} := \min_i \lambda_i$ and $\lambda_{\max} := \max_i \lambda_i$.
\STATE Construct the list of admissible pairs $A := \{(a, b) \in [N]^2: \lambda_a = \lambda_{\max}, \lambda_b = \lambda_{\min}\}$ 
\STATE Output a uniformly random element $(a, b)$ from $A$.
\end{algorithmic}
\end{algorithm}
% Our choice of pairwise loss is motivated by the disparity between positive and negative examples within each session: typically users only make a single purchase out of a page of 10 results. 
The final loss function on an input session $(u, t)$ is given by the following standard sigmoid cross entropy formula:

\begin{align} \label{eq:logloss}
    &\cL(B_{u, t}, \lambda_{u, t}) = -\lambda_{u, t} \log \sigma(\eta_{u, t}) - (1-\lambda_{u, t}) \log (1 - \sigma(\eta_{u, t}))
\end{align}
where 
\begin{itemize}
\item $B_{u, t}$ stands for all the features available to the model for a given user session $(u, t)$.
\item $\lambda_{u, t} := \lambda_{u, t, a, b} = \frac{\lambda_{u, t, a}}{\lambda_{u, t, a} + \lambda_{u, t, b}} \in \{0, 1\}$, depending on whether a purchase was made on item $a$ or $b$ within the user session $(u, t)$. 
\item $\eta_{u, t} := \eta_{u, t, a, b} = \eta_{u, t, a} - \eta_{u, t, b}$ is simply the difference between the model outputs for the two items $a$ and $b$, which can be interpreted as the log-odds that the purchase was made on the first item.
\item $a, b$ are a pair of random item indices within the current session, chosen according to Algorithm~\ref{alg:pairwise_sampling}, where item $a$ is purchased while item $b$ is not.
\item $\sigma(\eta_{u, t})$ transforms the pairwise logit $\eta_{u, t}$ through the sigmoid function $\sigma: x \mapsto (1 + e^{-x})^{-1}$, and can be interpreted as the model predicted probability that item $a$ is purchased, given exactly one of item $a$ and $b$ is purchased.
\end{itemize}

% \\
%     &\frac{\ell_{u, t, i}}{\ell_{u, t, i} + \ell_{u, t, j}} \log \frac{e_{u, t, i}}{e_{u, t, i} + e_{u, t, j}} + \frac{\ell_{u, t, j}}{\ell_{u, t, i} + \ell_{u, t, j}} \log \frac{e_{u, t, j}}{e_{u, t, i} + e_{u, t, j}}
\subsubsection{Knapsack Sequence Packing}
Since the numbers of historical sessions vary widely across different users, the naive implementation of the above 4d representation can be computationally quite wasteful due to excessive zero padding. We thus adopt a knapsack strategy (Algorithm~\ref{alg:knapsack_rnn}) to fit multiple short user session sequences into the maximum length seen in the current mini-batch. 

\begin{algorithm}[t]
\caption{Parallel RNN via Knapsack Packing}
\label{alg:knapsack_rnn}
\begin{algorithmic}[1]
\REQUIRE{a list of $N$ (user, session) indices: $\cI = \{(u_1, 1), \ldots, (u_1, T_1), \ldots, (u_n, T_n)\}$}
\REQUIRE{input feature vectors associated with each (user, session) pair: $\{B_{u, t} \in \R^D: (u, t) \in \cI\}$}
\REQUIRE{an expensive RNN kernel $\tilde{O}: \R^{2D} \to \R^{2D}$}
\ENSURE{efficient computation of $\{\omega_{u, t} := O(B_{u, t}): (u, t) \in \cI\}$}
\STATE Apply the greedy knapsack strategy (Algorithm~\ref{alg:greedy_knapsack}) to get a mapping $m: (u, t) \mapsto (u', t')$, as well as the 2d array $S := \{S_{u', t'}\}$ that encodes the starting positions of the subsequences.
\STATE Construct a new input features $B'$ according to $B'_{u', t'} = B_{u, t}$.
\STATE Zero pad the missing entries of $B'$, for vectorized processing.
\STATE Compute $\omega' := O'(B', S)$ for all packed users in parallel.
\STATE Rerrange $\omega'_{u', t'}$ into the original user sequences $\omega_{u, t}$ via the inverse map $m^{-1}: (u', t') \mapsto (u, t)$.
\end{algorithmic}
\end{algorithm}
To break down Algorithm~\ref{alg:knapsack_rnn}, we introduce a few terminologies:
\begin{definition}
For a given RNN kernel $\tilde{O}: \R^D \times \R^D \to \R^D \times \R^D$, its associated \textbf{sequence map} $O: \R^{D \times T} \to \R^{D \times T}$, $(B_1, \ldots, B_T) \mapsto (\omega_1, \ldots, \omega_T)$ is given inductively by 
\begin{align*}
    (\omega_1, H_{u, 1}) &:= \tilde{O}(B_{u, 0}, H_{u, 0}) \\
    (\omega_{t+1}, H_{u, t+1})  &:= \tilde{O}(\omega_t, H_{u, t}) \quad \text{for } t \leq T -1.
\end{align*}
The initial hidden state is typically chosen to be the all zero vector: $H_{u, 0} = \vec{0}$.
\end{definition}

Note that after applying the knapsack packing Algorithm~\ref{alg:greedy_knapsack}, the maximum length of all the sequences stays the same. The total number of sequences, however, is reduced, by an average factor of 20x. As a result, some new sequence now contains multiple old sequences, arranged contiguously from the left. In such cases, we do not want the hidden states to propagate across sequences. Thus we introduce the following extended RNN sequence map that takes into account the old sequence boundary information:
\begin{definition}
Given an RNN kernel $\tilde{O}$ as above, and a 2d indicator array $\{S_t \in \{0, 1\}: 1 \leq t \leq T\}$ denoting the starting positions of sub-sequences within each user sequence, the \textbf{boundary-aware sequence map} 
\begin{align*}
    O': \R^{D \times T} \times  \{0, 1\}^{D \times T} \to \R^{D \times T}, \quad (B_1, \ldots, B_T) \mapsto (\omega_1, \ldots, \omega_T)
\end{align*} 
is defined via the following inductive formula
\begin{align*}
    (\omega_1, H_{u, 1}) &:= \tilde{O}(B_{u, 0}, H_{u, 0}) \\
    (\omega_{t + 1}, H_{u, t + 1}) &:= 
    \begin{cases}
      \tilde{O}(B_{u, t}, H_{u, 0}) & \text{if  $S_{t+1} = 1$ }\\
      \tilde{O}(\omega_t, H_{u, t}) & \text{otherwise}
    \end{cases}  
\end{align*}
\end{definition}
 
% TODO(jyj): check table stats to ensure logical soundness here.
% The numbers of items per session typically do not vary by much, and have a mean around 10; indeed we count users' next page request under the same query as a separate session. Thus there is little marginal benefit to apply knapsack packing along the item dimension. Furthermore, under the pairwise training strategy (Section~\ref{subsec:rnn:pairwise}, each session consists of exactly 2 items, making the item dimension already uniform across sessions. 

Overall the session knapsack strategy saves about 20x compute and speed up CPU training time by about 3x. Note that during online serving, knapsack is not needed since we deal with one user at a time.
\begin{algorithm}[t]
\caption{Greedy Knapsack Sequence Packing.}
\label{alg:greedy_knapsack}
\begin{algorithmic}[1]
\REQUIRE{A nonempty index set $\cI := \{(u_1, 1), \ldots, (u_1, T_1), \ldots, (u_n, T_n)\}$.}
\ENSURE{an index map $m: \cI \to \cU' \times [T']$, where $|\cU'| \leq n$ is the packed user index set and $T' \leq \max_i T_i$.}
\ENSURE{a 2d array $S_{u', t'}$ indicating the start positions of subsequences in each packed user sequence.}  
\STATE Set $T' := \max_{i=1}^n T_i$.
\STATE Initialize $U := [n] \setminus \{i\}$.
\STATE Initialize the list of knapsacks $\cK \leftarrow []$.
\STATE Initialize $S_{u', t'}$ to the all $0$ 2d array.
\WHILE {$U \neq \emptyset$}
    \STATE Pop a longest user sequence from $U$, say $u_j$.
    \IF {$T_j + \sum_{k \in \cK_i} T_k < T'$ for some $i \leq |\cK|$}
        \STATE Define $m(j, \ell) := (i, \ell + \sum_{k \in \cK_i} T_k)$ for $\ell < T_j$.
        \STATE Set $S_{i, \sum_{k \in \cK_i} T_k} \leftarrow 1$.
        \STATE Append $j$ to the end of $K_i$.
    \ELSE
        \STATE Append $[j]$ to the end of $\cK$.
        \STATE Define $m(j, \ell) := (|\cK|, \ell)$
        \STATE Set $S_{|\cK|, 1} \leftarrow 1$.
    \ENDIF
\ENDWHILE

\end{algorithmic}
\end{algorithm}
\subsection{DDPG for Near-Term Future Sessions}
\label{sec:ddpg}

While attention and RNN are capable of leveraging past sequential data, they fall short of predicting or optimizing future user behavior several steps in advance. This is not surprising because the former are essentially trained in a supervised approach, where the target is simply the next session. To optimize trajectories of several future sessions, we naturally turn to the vast repertoire of reinforcement learning (RL) techniques. 

As mentioned in Section~\ref{subsec:rw:rnn}, unlike the vast majority of RL literature in search and recommendation, our trajectory of agent (ranker) / environment (user) interaction is not confined within a single query session. Instead the user continues to type new queries, over a span of weeks or months. Thus the environment changes from one session to the next. However a key assumption here is that the different manifestations of the environment (user) share an underlying preference theme, as a single user's shopping tastes are strongly correlated across multiple shopping categories or intents.

Another important difference between our sequential session setup and the single session setup in other works is that each step of S3DDPG needs to rank a list of tens or hundreds of items, rather than just picking the top K from the remaining candidate pool. Due to the combinatorial explosion associated with ranking tasks, it becomes infeasible to treat the set of permutations of the items as our action space. Instead we take the vector output of the RNN network, along with the actor network prediction, as the action, which lives in a continuous space.
% Table~\ref{tab:modeling_differences} summarizes the key areas where S3DDPG departs from existing RL work in search and recommendation.

% \begin{table}[htbp]
% \centering
% \caption{Modeling Difference}
% \small
% \begin{tabular}{c|c|c}
% \hline
% Key Areas & Our Work & Typical Other Work \\
% \hline
% Single Step Task & \begin{tabular}{@{}c@{}}Independent \\ Ranking\end{tabular} & \begin{tabular}{@{}c@{}} Masked Top-K \\ Retrieval \end{tabular} \\
% \hline
% Explicit Actions & 10! permutations & Candidate Items \\
% \hline
% Implicit Actions & RNN output vector & Candidate Items \\
% \hline
% \begin{tabular}{@{}c@{}} Environment \\
% (User Intent) \end{tabular} & Dynamic & Static \\
% \hline
% \end{tabular}
% \label{tab:modeling_differences}
% \end{table}


\subsubsection{S3DDPG Network}

\begin{figure}
    \centering
    % \includegraphics[width=\linewidth]{DDPG.png}
    \includegraphics[scale=0.5]{ddpg_new.png}
    \centering
    \caption{S3DDPG architecture.}
    \label{fig:ddpg}
\end{figure}

Finally we come to our reinforcement-learning based ranking framework, which is depicted in Diagram~\ref{fig:ddpg}. The bottom half of the network consists of the RNN structure described in the previous subsection. The reinforcement learning part takes the regular RNN output (i.e., non-hidden state related) as the input, and is similar to the actor/critic framework. We closely follow the logic of DDPG network \cite{lillicrap2015continuous}.

The actor network has the same structure as the final MLP layers in RNN that takes the intermediate embedding to per-item logit. The latter is thus used also as the final ranking score for each item within a single query session. 

The critic network (also known as the Q-network) is a separate multi-layer perceptron, $Q: \R^d \to \R$, taking \textbf{a pair of RNN outputs} $\omega_{u, t, a}, \omega_{u, t, b} \in \R^d$ to a single scalar logit. 
\begin{align*}
    q_{u, t} := Q(\omega_{u, t, a}, \omega_{u, t, b}) \in \R.
\end{align*}

$Q$ is introduced here to approximate the following maximal cumulative discounted long term reward:
\begin{align*}
    q_{u, t} \sim \sup_{\eta_{u, t}, \ldots, \eta_{u, T}} \sum_{s = t}^T \gamma^{s - t} r(\eta_{u,  t}, \lambda_{u, t}).
\end{align*}

Here the supremum is taken over all trajectories starting at session $t$, and the reward $r(\eta_{u,t},\lambda_{u, t}) =: r_{u, t}$ is simply given by the opposite of the sigmoid cross entropy loss $\cL(B_{u, t}, \lambda)$ (See \eqref{eq:logloss}):
% reproduced here for convenience:
\begin{align} \label{eq:reward_definition}
    r(\eta_{u, t}, \lambda_{u, t}) = \lambda_{u, t} \log \sigma(\eta_{u, t}) + (1 - \lambda_{u, t}) \log \sigma(1 - \eta_{u, t}).
\end{align}
The time horizon $T$ itself is also random in general.

To summarize, we have introduced three networks and their associated output layers so far
\begin{itemize}
    \item $\omega_{u, t, a}, \omega_{u, t, b} \in \R^d$ are the output vectors of the RNN network for the chosen item pair.
    \item $\eta_{u, t} = P(\omega_{u, t, a}) - P(\omega_{u, t, b})$ is the scalar output of the actor network, which has the interpretation of log-odds of the first item being purchased.
    \item $q_{u, t} = Q(\omega_{u, t, a}, \omega_{u, t, b})$ is the scalar output of the critic (Q) network for the pair.
\end{itemize}


The critic (Q) network differs significantly from the actor network $P$ in that the input consists of pairs of items. Thus unlike $\eta_{u, t}$, it is not anti-symmetric under swapping of the item pair.


It is interesting to note that the original supervised loss function $\cL(\eta, \lambda)$ has been re-purposed as the reward in the Q-network. The actual loss functions are defined next.
% At this stage, we have not defined the loss function of our S3DDPG model yet, which we proceed to do next.

\subsubsection{Loss Functions}
There are two loss functions in the S3DDPG framework. The first of these two, the temporal difference (TD) loss, is well-known since the first DQN paper \cite{mnih2013playing}. It aims to enforce the Bellman's equation on the Q-values:
\begin{align} \label{eq:bellman}
    q_{u, t} = \sup_{\eta_{u, t}} r(\eta_{u, t}, \lambda_{u, t}) + \gamma q_{u, t + 1}.
\end{align}
Here $\gamma$ is a discount factor, which is set to $0.8$ throughout our experiments. The associated TD loss would then be
\begin{align} \label{eq:dqn_td_loss}
    \cL^{\text{DQN}}_{\text{TD}} (B_{u, t}, \lambda_{u, t}) := \sum_{u \in \cU} \sum_{t=1}^{T - 1} (q_{u, t} - \sup_\eta \left\{r_t(\eta, \lambda) - \gamma q_{u, t+1}\right\})^2.
\end{align}
Here $\cU$ stands for all the users in the training data, and $T$ implicitly depends on the choice of $u$. 

% As mentioned in Table~\ref{tab:modeling_differences}, however,
As mentioned in Section~\ref{sec:ddpg}, however, our action space is either combinatorially explosive ($10!$), or continuous $\R^d$. Thus it is unclear how to compute the supremum on the right hand side. Instead we simply drop the supremum operator and consider the following weakened Bellman equation
\begin{align} \label{eq:weak_bellman}
    q_{u, t} = r(\eta_{u, t}, \lambda_{u, t}) + \gamma q_{u, t + 1}, \quad q_{u, T} = 0.
\end{align}
The TD loss thus aims to minimize the sum-of-square error between the two sides of the equation above:
\begin{align} \label{eq:td_loss}
    \cL_{\text{TD}}(B_{u, t}, \lambda_{u, t}) := \sum_{u \in \cU} \sum_{t=1}^{T - 1} (q_{u, t} - r_{u, t} - \gamma q_{u, t + 1})^2.
\end{align}
The problem with the above weakened TD loss \eqref{eq:td_loss} is that by itself, it is under-specified. Indeed, $r_{u, t} = r(\eta_{u, t}, \lambda_{u, t})$ can take on any (negative) value without affecting $\cL_{\text{TD}}$, since the extra degrees of freedom in $q_{u, t}$ can easily compensate for its wild moves. By contrast, the original TD Loss (for DQN) \eqref{eq:dqn_td_loss} eliminates this extra degree of freedom by taking the supremum over all actions $\eta_{u, t}$. 

To make the training loss fully specified, we thus introduce a second loss term, the policy gradient (PG) loss, which seeks to maximize the cumulative Q-value over the RNN and critic network model parameters. 
\begin{align}
    \cL_{\text{PG}}(B_{u, t}, \lambda_{u, t}) := \sum_{u \in \cU} \sum_{t=1}^T q_{u, t}, \quad q_{u, t} = Q(O(B_{u, t})).
\end{align}
where recall $q_{u, t} = Q(O(B_{u, t, a}), O(B_{u, t, b}))$ for the chosen positive / negative item pair. Note that since the actor network also depends on the RNN network parameters, the PG loss also indirectly optimizes over the action space. Furthermore, since $q_{u, t}$ are very closely tied with the supervised reward function $r_t$, by maximizing $q_{u, t}$, we are implicitly also maximizing the original supervised reward.


% This prediction is then fed into a policy gradient loss, which takes into account the logged user feedback label, which in our case is the binary signal of whether the current item has been purchased. The exact policy gradient loss is given by the formula below
% \begin{align}
%     \rm{PGLoss} = \sum_{t=1}^T \gamma^t \mathbb{E} Q_t, \quad Q_t := Q(\omega_{u, t}, P_{u, t}) 
% \end{align}

% The critic network takes the RNN output and actor output (the scalar logit value) as its input, and tries to simulate the Q value associated with the Bellman equation, which is a vector, $\{Q_t: 1 \leq t \leq T\}$, indexed by the session ids. The resulting Q value output is then used to compute a temporal difference loss which measures the deviation from the exact Bellman equation.
% \begin{align}
%     \rm{TDLoss} = \sum_{t=1}^{T-1} (Q_t - \gamma Q_{t+1} - r_t)^2.
% \end{align}

As is standard in DQN and DDPG, we also add the so-called target Q-network \cite{mnih2015human}, denoted by $\tilde{Q}$, that differs from the original Q-network only by one time-step, which is useful for stabilizing its learning. In other words, the exact weight updates are given by,
\begin{align}
    Q &\leftarrow Q + \alpha \nabla_Q (\sum_{t = 1}^{T - 1} Q(\omega_t) - \gamma \tilde{Q}(\omega_{t+1}) - r_t) \\
    \tilde{Q} &\leftarrow Q ,
\end{align}
where $\alpha$ is the effective learning rate that depends on the actual 1st order optimizer used.


% TODOs:
% 1. Put a formula about target Q-network here.
% 2. Put a formula about GRU here.
% 3. Put a formula about 


% One distinguishing feature of DDPG from DQN is that besides Q-learning, there is also policy learning, expressed through the policy gradient loss. The latter tries to maximizes the expected discounted cumulative reward over available actions given the current state. This is naturally done through gradient descent in our ranking problem since the parameter space is continuous, where exact discrete maximization is highly infeasible.
% Due to the instability of optimizing both the target and the source in Q loss, we use two copies of the Q network, and periodically copy the source version (at $t$) to the target (at $t + 1$).

We have also tried two versions of the actor networks, but the difference in evaluation metrics is small (about 0.04\% in session AUC), thus was discarded for simplicity and training efficiency.

Another important way S3DDPG differs from traditional DDPG implmementation is the relation between the two losses and weight updates. In the original proposal \cite{lillicrap2015continuous}, the actor and critic network weights are updated separately by the PG and TD losses:
\begin{align*}
    P \leftarrow P + \alpha \nabla_P \cL_{\text{PG}}, \qquad Q \leftarrow Q + \alpha \nabla_Q \cL_{\text{TD}}.
\end{align*}
However we cannot get the model to converge under this gradient update schedule. Instead we simply take the sum of the two losses $\cL_{\text{PG}} + \cL_{\text{TD}}$, and update all the network weights according to
\begin{align*}
    O, P, Q \leftarrow \alpha \nabla_{O, P, Q} (\cL_{\text{PG}} + \cL_{\text{TD}}).
\end{align*}


\section{Online Incremental Update}
To capitalize on the underlying RNN modeling framework, we perform incremental update when the model is served online, so that the most recent user interactions can be captured by the model to update the user states. The overall architecture and its relation to offline training is summarized in Figure~\ref{fig:incremental_update}. The offline trained model can be divided into two sets of network parameters: 
\begin{itemize}
    \item The user state aggregation network takes the hidden states associated with all the items in the session, along with their corresponding labels, and perform average pooling to obtain a fixed size updated user state. If the session contains no purchase action, we do not update the user state.
    \item The remaining network take in the usual input features, along with the user state, to output predictions for each item.
\end{itemize}
The first of these is sent to an online incremental update component. While the latter goes directly to the neural network scorer.

The online serving component is roughly divided into three modules. At the center is the search engine itself, which is in charge of distributing and receiving features. 

When a user types in a query, the associated user context features, including query text, user's basic profile information, as well as user's historical actions, are all sent to the search engine. The search engine then relays this information to the neural network predictor, which in turn computes the predicted scores as well as the hidden state for each item, all of which are sent back to the search engine. Finally if the user makes any purchase in the current session, the new user state is updated to be the average of hidden states from the purchase items. 

\begin{figure}
    \centering
    \includegraphics[width=\linewidth]{DDPG-IncrementalUpdate.png}
    \caption{Real-Time Incremental Update Pipeline.}
    \label{fig:incremental_update}
\vspace{10pt}
\end{figure}

% TODOs:
% 1. add some diversity time series plot.
% 2. add UV value time series plot.
% 3. show some arena side by side to show diversity.
% 4. 

\documentclass[../absorber.tex]{subfiles}
\begin{document}

To demonstrate the effectiveness of the absorber region, we present results from a variety of 2-dimensional simulations of a laser incident on an overdense plasma.  Simulations were done using \textsc{Osiris},\cite{Fonseca2002} where an absorbing region has been implemented.

\subsection{Simulation setup}
In the simulations, an intense 1-$\mu$m plane-wave laser with normalized amplitude $a_0=3$ and 3~ps in duration (2.9-ps flat envelope with 0.13-ps rise and fall ramps) is incident on uniform plasma with density $n=10n_c$ (where $n_c$ is the critical density).  The exponential ramp has a scale length of 3~$\mu$m and begins at $x=-27.6$~$\mu$m.  The critical density is then located at $x_c=-6.9$~$\mu$m, and we define time $t=0$ to be when the leading edge of the laser pulse would arrive at $x_c$ if traveling at speed $c$. The laser is focused to the critical surface and is launched from the left wall.  The plasma skin depth is $c/\omega_p = 50.3$~nm and $c/\omega_0 = 159.2$~nm.  See Fig.~\ref{fig:laser}(a) for a schematic.

The simulations used periodic boundary conditions in the second dimension ($y$), and the laser was polarized with its electric field in the simulation plane (p-polarized). The simulation dimensions were kept constant in the $y$-direction,  3.2~$\mu$m, and in the $x$-direction were either 923.9 or 1597.8~$\mu$m for truncated and causally separated runs, respectively.  Square cells of size 0.2~$c/\omega_0$ were used, resulting in a simulation domain of 50197$\times$100 cells for the simulation with the largest length in $x$ (29025$\times$100 cells otherwise).  The time step was 0.141~$\omega_0^{-1}$.  The electron (ion) species had 64 (16) particles per cell, and each species used cubic interpolation with an initial temperature of 0.1~keV.  We employed a static load balancing routine\cite{Fonseca2013} at initialization to distribute processing elements in an optimal configuration, and the particle push time was delayed until the laser neared the plasma.

\begin{figure}
\includegraphics[width=\linewidth]{figures/fig-1-schematic.jpeg}
\caption{\label{fig:laser} (a)~Simulation schematic, showing the full box size. The box is truncated at 150~$\mu$m when the absorber is in use. (b)~Laser Poynting flux incident at the plasma critical interface, reflected Poynting flux measured 387~$\mu$m to the left of the critical interface and forward electron energy flux measured over the diagnostic region.  All quantities are synced up in time for better visualization.  Percentages represent integrated energy flux as a fraction of the total incident energy.}
\end{figure}

% \begin{figure}
%     \centering
%     \captionsetup{width=0.98\linewidth}
%      \includegraphics[width=\linewidth]{figures/actual-figures/fig-1-schematic.png}
%     \caption{(a)~Simulation schematic, showing the full box size. The box is truncated at 150~$\mu$m when the absorber is in use. (b)~Laser Poynting flux incident at the plasma critical interface, reflected Poynting flux measured 387~$\mu$m to the left of the critical interface and electron energy flux measured over the diagnostic region.  All quantities are synced up in time for better visualization.  Percentages represent integrated energy flux as a fraction of the total incident energy.}
%     \label{fig:laser}
% \end{figure}

In Fig.~\ref{fig:laser}(b) we show the temporal laser profile, as well as the reflected Poynting flux and
%???how do you define the reflected Poynting flux???
transmitted particle energy flux.  The reflected Poynting flux is calculated by measuring the total Poynting flux 380~$\mu$m before the critical-density interface, then subtracting the known incident laser flux.  Both the Poynting and energy fluxes plotted in Fig.~\ref{fig:laser}(b) are translated in time to line up with the incident laser light.  To diagnose the forward momentum and energy flux deep in the plasma, we choose a diagnostic region 48--64~$\mu$m into the uniform plasma over which we average the particle data in space.  The energy flux is defined as $\int (\gamma-1)m_ec^2 \mathbf{p}/\gamma\,d\mathbf{p}$ for electron mass $m_e$.  In order to avoid particle refluxing from either boundary in the $x$-direction, a 746-$\mu$m vacuum region (computationally inexpensive because of the static load balancing) is placed to the left of the plasma upramp, and the uniform-density plasma is extended to the right a distance of 824~$\mu$m (computationally expensive).  This ensures that any particles reflected from the right boundary region will be causally separated from the diagnostic region for the duration of the simulation (for a time $2\times 760\,\mu$m$/c\approx5$~ps).  The $p_x$-$x$ phasespace is shown in Fig.~\ref{fig:px-x-a} at 3.7~ps after the laser was incident on the critical interface, with the diagnostic region marked by dashed lines.  Note the large size of the plasma required compared to the diagnostic region location, along with the very hot return current reflecting off the right simulation boundary---even though a thermal particle boundary is being used.

\begin{figure}
\includegraphics[width=\linewidth]{figures/fig-2-p1x1-a.jpeg}
\caption{\label{fig:px-x-a} The $p_x$-$x$ phasespace for the causally separated simulation (single run, not averaged).  The dashed lines indicate the diagnostic region, but the plasma has to be much larger in length to be causally separated from the hot return current reflecting off the right boundary.}
\end{figure}

% \begin{figure}
%     \centering
%     \captionsetup{width=0.98\linewidth}
%      \includegraphics[width=\linewidth]{figures/actual-figures/fig-2-p1x1-a.png}
%     \caption{The $p_x$-$x$ phasespace for the causally separated simulation (single run, not averaged).  The dashed lines indicate the diagnostic region, but the plasma has to be much larger in length to be causally separated from the hot return current reflecting off the right boundary.}
%     \label{fig:px-x-a}
% \end{figure}

We ran the simulations until 2~ps after the laser had finished hitting the plasma.  In all cases, hot particles were split into two after reaching a $\gamma$ of 1.4, 1.5, 1.6, 1.7, and 1.8 (i.e., very energetic particles were eventually split into 32 smaller particles); the splitting routine was executed every 10 time steps.  Contact the corresponding author for information about the source code and input files used for these simulations.

The particle acceleration mechanisms in these types of simulations are stochastic; therefore, we expect and indeed do observe large differences in particle statistics due to slightly different simulation configurations.  For example, we performed the causally separated simulation three different times with varied random number seeds and observed a factor of 2--4 variation in particle number in the tail of the momentum distribution over the diagnostic region.  For this reason we performed the simulations presented in this paper three times with different random number seeds.  Unless otherwise noted, visualizations presented here are of data averaged over three different runs; this averaging gives increased confidence that any observed deviations from the causally separated run are due to the particle boundary conditions.


\subsection{Effect of the absorber boundary condition}

To greatly reduce computation time and resources, we desire to shrink the simulation region shown in Fig.~\ref{fig:px-x-a}, but preserve the behavior from the causally separated run.  We truncate the plasma at a distance of 150~$\mu$m (29025 cells in $x$) and vary the length of the absorber, where each absorber is designed to stop all hot particles 5~$\mu$m short of the right boundary.  For all results shown here we use the linearly varying absorber from Sec.~\ref{sec:linear} and calculate the local temperature via Eq.~(\ref{Eq:lin-int}).  We quote the mean free path for each absorber, which as shown in Fig.~\ref{fig:f-and-h-lin} is 26\% of the entire absorber length.  We used an energy threshold of 6 times the local thermal velocity and re-emitted stopped particles at the local temperature.  Stopping was performed every time step for both electrons and ions
%???Why the ions??? Explain that ions had large energy
to give a large number of stopping loops for a fast particle traversing the absorbing region.  Particles are typically stopped every $\sim3$ time steps, but we perform a stopping loop every time step to more accurately assess the different methods.  Though it is much more important to use an absorber for electrons than for ions, we observed a sufficient number of hot ions reaching the thermal boundary to warrant stopping ions as well.  Stopping loops were delayed until hot particles approached the absorber region.  Particle recombination (for electrons) was executed every 5 time steps over the absorbing region; this dramatically reduces the simulation runtime as hot particles that have been split into 32 smaller particles are all stopped over a very short distance.

\begin{figure}
\includegraphics[width=\linewidth]{figures/fig-3-p1x1-all-123.jpeg}
\caption{\label{fig:px-x-1.1} The $p_x$-$x$ phasespace (single runs, not averaged) for the causally separated ($\lambda=\infty$), absorber, and no absorber ($\lambda=0$) simulations 1.1~ps after the incident laser.  A hot reflux of electrons is already shown to be entering the dashed diagnostic region for the truncated run with no absorber.}
\end{figure}

% \begin{figure}
%     \centering
%     \captionsetup{width=0.98\linewidth}
%      \includegraphics[width=\linewidth]{figures/actual-figures/fig-3-p1x1-all-123.png}
%     \caption{The $p_x$-$x$ phasespace (single runs, not averaged) for the causally separated ($\lambda=\infty$), absorber, and no absorber ($\lambda=0$) simulations 1.1~ps after the incident laser.  A hot reflux of electrons is already shown to be entering the dashed diagnostic region for the truncated run with no absorber.}
%     \label{fig:px-x-1.1}
% \end{figure}

The $p_x$-$x$ phasespaces for the causally separated ($\lambda=\infty$, where we are zooming in on a particular region), absorber (with $\lambda=100\,c/\omega_p$) and no absorber/truncated ($\lambda=0$) simulations are shown in Figs.~\ref{fig:px-x-1.1} and \ref{fig:px-x-3.7} at two different times.  After just 1.1~ps, a hot reflux of electrons is visible in the truncated run [see Fig.~\ref{fig:px-x-1.1}(c)] that has already entered the diagnostic region.  These refluxing electrons are seen to completely overwhelm the simulation late in time [see Fig.~\ref{fig:px-x-3.7}(c)], while the simulation with the absorber [see Fig.~\ref{fig:px-x-3.7}(b)] is able to maintain an appropriate return current.  These plots are not averaged over three simulations, so sizeable variations within the pre-plasma are expected for the causally separated run due to differences in random number initialization with a different box size [note that the phasespace in the density upramp and surrounding region are identical in Figs.~\ref{fig:px-x-1.1}(b) and (c)].

\begin{figure}
\includegraphics[width=\linewidth]{figures/fig-3-p1x1-all-210.jpeg}
\caption{\label{fig:px-x-3.7} The $p_x$-$x$ phasespace (single runs, not averaged) for the causally separated ($\lambda=\infty$), absorber, and no absorber ($\lambda=0$) simulations 3.7~ps after the incident laser.  The refluxing electrons for the truncated run have completely altered the particle phasespace; the returning hot electrons cyclically interact with the laser and re-enter the plasma, artificially heating the bulk plasma to a much higher temperature than in the casually separated or absorbing runs.}
\end{figure}

% \begin{figure}
%     \centering
%     \captionsetup{width=0.98\linewidth}
%      \includegraphics[width=\linewidth]{figures/actual-figures/fig-3-p1x1-all-210.png}
%     \caption{The $p_x$-$x$ phasespace (single runs, not averaged) for the causally separated ($\lambda=\infty$), absorber, and no absorber ($\lambda=0$) simulations 3.7~ps after the incident laser.  The refluxing electrons for the truncated run have completely altered the particle phasespace; the returning hot electrons cyclically interact with the laser and re-enter the plasma, artificially heating the bulk plasma to a much higher temperature than in the casually separated or absorbing runs.}
%     \label{fig:px-x-3.7}
% \end{figure}

To better visualize temporal behavior, we plot the electron energy flux in the $x$ direction as a function of time and space for the causally separated, absorber, and no absorber simulations in Fig.~\ref{fig:s1-t}.  For the causally separated simulation, a steady stream of energy flux is observed to the right of the critical-density interface, which is slowly pushed forward in time.  Energetic electrons are also seen to escape to the left as the plasma expands.  This expansion is enhanced after the laser turns off.  When using the absorber with $\lambda=100\,c/\omega_p$, the energy flux looks qualitatively very similar to the causally separated run, except that the energy flux quickly decreases to zero in the absorber region.  In contrast, the truncated simulation ($\lambda=0$) shows that a large fraction of the forward energy flux is reflected from the right boundary (especially visible at 0.8~ps), so much so that it dramatically reduces the overall energy flux as it travels backward.  Once the first reflux arrives back to the laser-plasma interface at around 1.5~ps, the forward energy flux is then permanently altered.  This change in physics, as the hot return current interacts with and is accelerated by the laser, is the primary issue that the absorber is able to eliminate.  Finally, this hot reflux of electrons is also visible in the blue negative energy flux after the laser turns off in the truncated run.

\begin{figure}[htp]
\includegraphics[width=\linewidth]{figures/fig-4-s1-t.jpeg}
\caption{\label{fig:s1-t} Forward particle energy flux as a function of position and time for three different cases.  For the truncated simulation ($\lambda=0$), the forward energy flux can be seen to be neutralized by a refluxing current emitted from the boundary.  The absorber effectively reduces the particle energy flux before the simulation boundary without a reflux current.}
\end{figure}

% \begin{figure}
%     \centering
%     \captionsetup{width=0.98\linewidth}
%      \includegraphics[width=\linewidth]{figures/actual-figures/fig-4-s1-t.png}
%     \caption{Forward particle energy flux as a function of position and time for three different cases.  For the truncated simulation ($\lambda=0$), the forward energy flux can be seen to be neutralized by a refluxing current emitted from the boundary.  The absorber effectively reduces the particle energy flux before the simulation boundary without a reflux current.}
%     \label{fig:s1-t}
% \end{figure}

We also examine energy conservation (fields plus particles) across the simulation region when the absorber boundary condition is in use.  To do this we compute the integral of energy density over a specific domain ($V$) and add the energy flux through the left and right boundaries of that domain ($\partial V$):
\begin{equation} \label{Eq:energy}
    \int_V U\,dV + \oint_{\partial V} \mathbf{S} \cdot d\mathbf{A},
\end{equation}
where $U$ is the energy density [$E^2/8\pi + B^2/8\pi + \sum (\gamma-1)m_e c^2$] and $\mathbf{S}$ is the energy flux [$\mathbf{E}\times \mathbf{B}/4\pi + \sum (\gamma-1)m_ec^2 \mathbf{p}/\gamma$].  We compute a running sum of this value over the simulation time (which should remain at zero) and then divide by the maximum energy present in the simulation box at any given time.  This gives a good measure of the energy conservation of the code, although it is not perfect since we only use data reported every 401 time steps (0.3~ps).  In Fig.~\ref{fig:energy} we plot Eq.~(\ref{Eq:energy}) as a function of time, where the right-hand side of volume $V$ (i.e., the location of $\partial V$ on the right) is given by the $x$ coordinate displayed for an absorber with mean free path $\lambda=100\,c/\omega_p$.  We can see that to the left of the absorber (dashed line), the deviation in the coarsely computed energy conservation is less than 1.4\%.  However, by including the absorber region we see that a large fraction of the energy is steadily removed as energetic particles are stopped.  Once again, it is this extended slowing of the particle beam that allows for an appropriate return current to develop, causing plasma to return back into the main simulation region.

\begin{figure}
\includegraphics[width=\linewidth]{figures/fig-5-energy-conservation-edited.jpeg}
\caption{\label{fig:energy} The scaled deviation in energy conservation [see Eq.~(\ref{Eq:energy})] as a function of time, including all points to the left of a given $x$ value (single run, not averaged).  To the left of the absorbing region, energy is well conserved ($<$1.4\% error), but in the absorbing region energy is steadily removed as particles are stopped.}
\end{figure}

% \begin{figure}
%     \centering
%     \captionsetup{width=0.98\linewidth}
%      \includegraphics[width=\linewidth]{figures/actual-figures/fig-5-energy-conservation-edited.png}
%     \caption{The scaled deviation in energy conservation [see Eq.~(\ref{Eq:energy})] as a function of time, including all points to the left of a given $x$ value (single run, not averaged).  To the left of the absorbing region, energy is well conserved ($<$1.4\% error), but in the absorbing region energy is steadily removed as particles are stopped.}
%     \label{fig:energy}
% \end{figure}

\subsection{Variation of absorber parameters}

As mentioned in Sec.~\ref{sec:absorber} and Appendix~\ref{app:temp}, there are a variety of options for implementing the absorber region. 
%??? We should be clearer???
When determining the energy threshold and re-emission temperature of the stopped particles, we can calculate the background temperature dynamically by weighting the distribution function with the proper velocity to some power,
%or its fourth root 
%to determine the energy threshold and re-emission temperature of stopped particles
or we can simply specify a constant value to use. Using a lower power (such as the fourth root) for the proper velocity will emphasize the bulk over a hot tail; more details are given in Appendix~\ref{app:temp}.  However, using the fourth-root temperature never improved the absorber performance for the simulations shown here, so we calculate the temperature in each cell as given by Eq.~(\ref{Eq:lin-int}).

We can also use the hazard function probability defined in Sec.~\ref{sec:hazard} or the linearly varying probability defined in Sec.~\ref{sec:linear} to stop the particles.  In our tests these two choices produce similar results, but overall the linearly varying absorber maintained the proper response for a longer time.  The main reason for this is that due to the periodicity in $y$, simulations using the hazard-function absorber exhibited a large and increasing transverse temperature in the absorbing region; the hazard-function absorber preferentially stops particles with large forward momentum, allowing energetic particles to stream transversely and for some accelerating/reflecting fields to develop (see last paragraph of Sec.~\ref{sec:hazard}).  For this reason we use the linearly varying absorber in this paper, which stops particles as a function of the magnitude of the velocity and not just the longitudinal component.

% Both the hazard and linearly varying absorber schemes appear to efficiently stop particles.  However, in many of the quasi-1D simulations that we performed, the linearly varying absorber was able to maintain the proper response for longer than the hazard function absorber.  This is because the hazard function relies on particles streaming primarily in one direction.  However, with only 100 cells in the transverse direction, many energetic particles that stream towards the boundary are also very hot in the transverse direction.  Both electrons and ions can develop very large transverse momentum in the absorbing region as they drift backward as part of the return current, and that energy will not be removed by the absorber.  The linearly varying absorber relies on the absolute magnitude of a particle's velocity, and hence any transverse momentum in the absorbing region will be kept small.  For this reason we recommend using the linearly varying absorber with a calculated threshold and re-emission temperature.

We compare a combination of absorbers in Fig.~\ref{fig:variation}, where we show the $p_x$ momentum phasespace for all electrons in the diagnostic region at two different times.  Although all absorber schemes appear to perform equally well early in time, the return current is clearly hotter when constant values of the energy threshold and re-emission temperature are given.  For the static temperature simulation, we set the absorber to stop particles with energy greater than 0.6~keV and to re-emit particles at 0.1~keV (the original plasma temperature); in contrast the dynamic absorber stops particles moving at more than 6 times the locally computed thermal velocity.  Using a static temperature performs poorly because, as seen even in the absorbing region of Fig.~\ref{fig:px-x-3.7}(b), the plasma heats up significantly in response to the energetic electron beam.  Particles stopped and re-emitted at the original temperature are not moving fast enough to provide the necessary return current, and a nonphysical potential develops that accelerates electrons backward with too much energy.  Calculating the local temperature instead allows the absorber to accurately compensate for this dynamic behavior.

Although not shown here, we performed a series of simulations varying the mean free path of the absorber by factors of two between $\lambda=0.1\,c/\omega_p$ and $\lambda=200\,c/\omega_p$.  We observed that if the absorber had a mean free path $\lambda \gtrsim 6\,c/\omega_p$, it was able to closely match the causally separated momentum distribution when averaged over three separate runs.  However, individual simulations with $\lambda \lesssim 20\,c/\omega_p$ exhibited slightly greater variability in comparison to the causally separated data.  In our simulations, an absorber with a mean free path of $6\,c/\omega_p$ performed only $\sim30$ stopping events before nearly all particles were stopped, which was sufficient for a laser 3~ps in duration with $a_0=3$.  However, care must be taken for lasers of longer duration or higher intensity; Fig.~\ref{fig:variation} shows that some absorbers can perform well (a)~initially, but (b)~eventually fail due to the large amount of energetic particles striking the absorber.  Thus $\lambda \gtrsim \bigO(10\,c/\omega_p)$ gives a reasonable estimate of the appropriate mean free path, but the absorber length should be verified for each individual simulation.

% In our simulations the hot electron beam exhibited a temperature of 3~MeV, and the background plasma in the absorbing region had an average thermal velocity of 0.15~$c$.  Equation~(\ref{Eq:mfp}) then gives (ignoring errors from relativistic inaccuracies) a predicted mean free path of $\lambda \approx 80\,c/\omega_p$.  Equation~(\ref{Eq:mfp}) thus gives a reasonable estimate of appropriate mean free path and can be used as a guideline for absorber length---especially for long-time laser-plasma interactions---but a much shorter absorber may be similarly effective depending on the simulation.

% In these simulations, the hot electron beam exhibited a temperature of 3~MeV, and the background plasma in the absorbing region had an average thermal velocity of 0.15~$c$, giving a predicted mean free path (ignoring errors from relativistic inaccuracies) of $\lambda \approx 80\,c/\omega_p$.  In Fig.~\ref{fig:variation} we show the $p_x$ phasespace averaged over electrons in the diagnostic region for a wide range of the mean free path both for (a)~a single set of runs performed and (b)~the sets of three runs averaged together.  For these simulations, when averaged together all absorber lengths appear to perform consistently well, though perhaps greater variance is observed for mean free paths shorter than $\sim100\,c/\omega_p$ in Fig.~\ref{fig:variation}(a).  In addition, our tests using longer lasers showed that absorbers with shorter mean free paths failed at earlier than those with longer mean free paths.  Thus, again, the absorber parameters should be verified for each unique simulation, though Eq.~(\ref{Eq:mfp}) can be used as a guideline for the appropriate mean free path length, especially for long-time simulations.

\begin{figure}
\includegraphics[width=\linewidth]{figures/fig-6-avg-p1.jpeg}
\caption{\label{fig:variation} The $p_x$ phasespace for all electrons in the region 48--64~$\mu$m into the constant-density plasma for various schemes (a)~1.5~ps and (b)~3.7~ps after the laser was incident on the plasma.  Though the performance of all shown absorbers is nearly identical early in time, either using a static temperature threshold and re-emission or using a very short absorber gives improper results later in time.}
\end{figure}

% \begin{figure}
%     \centering
%     \captionsetup{width=0.98\linewidth}
%      \includegraphics[width=\linewidth]{figures/actual-figures/fig-6-avg-p1.png}
%     \caption{The $p_x$ phasespace for all electrons in the region 48--64~$\mu$m into the constant-density plasma for various schemes (a)~1.5~ps and (b)~3.7~ps after the laser was incident on the plasma.  Though the performance of all shown absorbers is nearly identical early in time, either using a static temperature threshold and re-emission or using a very short absorber gives improper results later in time.}
%     \label{fig:variation}
% \end{figure}

% In Sec.~\ref{sec:concept-mfp} we discussed the appropriate mean free path to use for the absorber, given by Eq.~(\ref{Eq:mfp}) as a function of the beam energy and thermal velocity of the background plasma.  We performed various simulations using the linearly varying absorber with a dynamically calculated threshold and re-emission temperature and varied the mean free path of the absorber.  In these simulations, the hot electron beam exhibited a temperature of 3~MeV, and the background plasma in the absorbing region had an average thermal velocity of 0.15~$c$, giving a predicted mean free path (ignoring errors from relativistic inaccuracies) of $\lambda \approx 80\,c/\omega_p$.  In Fig.~\ref{fig:variation} we show the $p_x$ phasespace averaged over electrons in the diagnostic region for a wide range of the mean free path both for (a)~a single set of runs performed and (b)~the sets of three runs averaged together.  For these simulations, when averaged together all absorber lengths appear to perform consistently well, though perhaps greater variance is observed for mean free paths shorter than $\sim100\,c/\omega_p$ in Fig.~\ref{fig:variation}(a).  In addition, our tests using longer lasers showed that absorbers with shorter mean free paths failed at earlier than those with longer mean free paths.  Thus, again, the absorber parameters should be verified for each unique simulation, though Eq.~(\ref{Eq:mfp}) can be used as a guideline for the appropriate mean free path length, especially for long-time simulations.

% We show in Fig.~\ref{fig:p1} the $p_x$ momentum phasespace for all electrons in the diagnostic region (48--64~$\mu$m into the uniform-density plasma) at two different simulation times, with and without the absorber.  A causally separated run is also given to show the ideal response.  We see that if a traditional thermal boundary condition is used instead of the absorbing boundary, a stream of hot electrons reflected from the right simulation boundary comes back through the main plasma body at early times.  These hot electrons eventually arrive at the laser-plasma interface, are again heated, and return as a forward current through the plasma.  This cycle results in an artificially hot plasma due to the improper boundary.  However, good agreement with the causally separated plasma is obtained when the absorbing boundary condition is used.

% Another visualization of this same effect is shown in Fig.~\ref{fig:p1x1}, which shows the $p_x$-$x$ phasespace for a broader range of the simulation space at two separate times.  Without the absorber, a very hot return current is visible in Fig.~\ref{fig:p1x1}(e) at early times, which results in a much hotter plasma overall late in time [Fig.~\ref{fig:p1x1}(f)].  Employing the absorber, however, allows the truncated simulation to maintain the low-temperature return current characteristic of the causally separated simulation.  The beginning of the absorbing region is indicated in Fig.~\ref{fig:p1x1}(c)--(d) by a dashed line, past which hot electrons are seen to gradually be cooled until near the simulation boundary.

% \begin{figure}
%     \centering

%     \captionsetup{width=0.98\linewidth}
%     \begin{subfigure}[b]{0.49\linewidth}
%          \centering
%          \includegraphics[width=\linewidth]{figures/p1x1-0-123.png}
%          \label{fig:p1x1-a}
%      \end{subfigure}
%      \hfill
%     \begin{subfigure}[b]{0.49\linewidth}
%          \centering
%          \includegraphics[width=\linewidth]{figures/p1x1-0-210.png}
%          \label{fig:p1x1-b}
%      \end{subfigure}
%     \begin{subfigure}[b]{0.49\linewidth}
%          \centering
%          \includegraphics[width=\linewidth]{figures/p1x1-20-123.png}
%          \label{fig:p1x1-c}
%      \end{subfigure}
%      \hfill
%     \begin{subfigure}[b]{0.49\linewidth}
%          \centering
%          \includegraphics[width=\linewidth]{figures/p1x1-20-210.png}
%          \label{fig:p1x1-d}
%      \end{subfigure}
%     \begin{subfigure}[b]{0.49\linewidth}
%          \centering
%          \includegraphics[width=\linewidth]{figures/p1x1-00-123.png}
%          \label{fig:p1x1-e}
%      \end{subfigure}
%      \hfill
%     \begin{subfigure}[b]{0.49\linewidth}
%          \centering
%          \includegraphics[width=\linewidth]{figures/p1x1-00-210.png}
%          \label{fig:p1x1-f}
%      \end{subfigure}
%     \caption{The $p_x$-$x$ phasespace for electrons in the broader simulation space at two different times for simulations that are (a)--(b) causally separated (actual right simulation boundary extended to 824~$\mu$m), (c)--(d) with the absorber (dashed line shows start of absorbing region) and (e)--(f) without the absorber.  Note that when truncating the simulation space without the absorber, a hot return current is present at early times, which translates to a much hotter overall plasma late in time.}
%     \label{fig:p1x1}
% \end{figure}

% \begin{figure}
%     \centering

%     \captionsetup{width=0.98\linewidth}
%     \begin{subfigure}[b]{0.98\linewidth}
%          \centering
%          \includegraphics[width=\linewidth]{figures/avg-lin-haz-runs-p1-135.png}
%          \label{fig:lin-haz-p1-a}
%      \end{subfigure}
%      \begin{subfigure}[b]{0.98\linewidth}
%          \centering
%          \includegraphics[width=\linewidth]{figures/avg-lin-haz-runs-p1-210.png}
%          \label{fig:lin-haz-p1-b}
%      \end{subfigure}
%     \caption{The $p_x$ phasespace for all electrons in the region 48--64~$\mu$m into the constant-density plasma (a)~1.47~ps and (b)~3.72~ps after the laser was incident on the plasma for the two different absorber schemes.  As for temperature calculation, performance is seen to be significantly worse when constant values are used.}
%     \label{fig:lin-haz-p1}
% \end{figure}

% \begin{figure}
%     \centering

%     \captionsetup{width=0.98\linewidth}
%     \begin{subfigure}[b]{0.98\linewidth}
%          \centering
%          \includegraphics[width=\linewidth]{figures/not-avg-2-lambda-p1-210.png}
%          \label{fig:lambda-p1-a}
%      \end{subfigure}
%      \begin{subfigure}[b]{0.98\linewidth}
%          \centering
%          \includegraphics[width=\linewidth]{figures/avg-lambda-p1-210.png}
%          \label{fig:lambda-p1-b}
%      \end{subfigure}
%     \caption{The $p_x$ phasespace for all electrons in the region 48--64~$\mu$m into the constant-density plasma 3.72~ps after the laser was incident on the plasma as a function of stopping distance for (a)~one particular set of runs and (b)~all runs averaged together.  Some greater deviations from the causally separated spectrum are observed for mean free paths shorter than $\sim100\,c/\omega_p$, though all runs seem to perform about equally when averaged together.}
%     \label{fig:lambda-p1}
% \end{figure}

\subsection{Best practices}

Here we make a few notes on best practices for performing simulations with the extended absorbing boundary condition.  We found it important to also causally separate the vacuum boundary (where the laser is injected) from the laser-plasma interface.  Even with absorbing particle boundary conditions at this vacuum boundary, most energetic particles that reached the vacuum boundary were immediately reflected back into the simulation space.  This is a combined effect of the laser potential at the wall and the electric field buildup from exiting particles (a nonnegligible number of particles are accelerated toward the laser from the pre-plasma region).  Refluxing from the vacuum boundary leads to a modified distribution at the laser-plasma interaction region, which then artificially inflates the forward electron energy flux in the target.

% In addition, for simulations 
% %of a target with finite size in the transverse direction, 
% with a finite width laser the boundary in the transverse directions also need to 
% %we advise extending out the vacuum boundary to be 
% be large enough causally separate itself from the interaction region.
% %that dimension as well.

For simulations with a finite-width laser, absorber regions can also be placed at the transverse simulation edges to correctly handle the large flux of relativistic electrons expelled transversely from the laser spot.  However, the effectiveness of the absorber relies on having a large number of particles in each cell (for calculating the temperature).  If absorbers are placed at the transverse simulation boundaries, they may overlap with near-vacuum regions in and before the pre-plasma.
%???Not sure I understand what you are trying to say????
Thus for finite-size-target simulations with multiple absorbers, we found it is useful to transition the absorbers positioned along the transverse boundaries to stop and re-emit particles based on a static (rather than dynamically calculated) temperature in those near-vacuum regions.

Finally, the start of the absorber region should be located a reasonable distance away from where accurate plasma measurements are expected.  For example, when comparing Figs.~\ref{fig:px-x-3.7}(a) and \ref{fig:px-x-3.7}(b), the phasespace immediately in front of the absorbing region in (b) does not exactly mimic the causally separated phasespace in (a).  Examining the particle phasespace for irregularities near the absorber region can help determine the appropriate distance at which to measure plasma quantities.

\subsection{Future work}

The implementation described here, though effective, is by no means a comprehensive treatment or unique solution to the reflux problem.  Here we list some ideas that could be used to iterate on our proposed solution.  Particles could be re-emitted from a distribution that is hotter in the return direction than in the forward direction, assisting in establishing the appropriate return current.  Particles could be stopped preferentially based on their direction of motion.  We employed absorbers for both ions and electrons in these simulations, but the ion response and stopping could be explored in greater detail for long-time simulations.  Alternatives that are more computationally expensive could include applying a drag force to energetic particles over the length of the entire absorber or calculating the re-emission temperature from a position located before the absorber region.  Last, it may be possible to develop a thermal bath boundary where particles are re-emitted from a distribution determined from a region somewhere inside the plasma.

\end{document}
% \input{5-discuss.tex}
\section{Conclusion}
In this study, we take the first step to systematically explore multi-level feature fusion for the isotropic architecture, such as ViT, in masked image modeling. Initially, we recognize that pixel-based MIM approaches tend to excessively rely on low-level features from shallow layers to complete the pixel value reconstruction task by a pilot experiment. We then apply a simple and intuitive multi-level feature fusion to two pixel-based MIM approaches, MAE and PixMIM, and observe significant improvements in both, gradually closing the performance gap with these approaches by using an extra heavy tokenizer. Finally, we conduct an extensive analysis of multi-level feature fusion and find that it can suppress high-frequency information and flatten the loss landscape. We believe that this work can provide the community with a fresh perspective on these pixel-based MIM approaches and continue to rejuvenate this kind of simple and efficient self-supervised learning paradigm.

% \section{Citations}
% \label{sec:citations}

% 	Citations can be made using either \textbackslash citep\{\} or \textbackslash citet\{\}, depending from the appropriateness. To avoid the citation moving to the next line, it is often a good practice to replace the space before with a tilde (\~{}) character.
% 	Example 1: ``CoRL is the best conference ever, as discussed in~\citep{Calandra2016}.``
% 	Example 2: ``\citet{Calandra2016} proved, both theoretically and numerically, that CoRL is the best conference ever.``
	
% The maximum paper length is 8 pages excluding references and acknowledgements, and 10 pages including references and acknowledgements

\clearpage
% The acknowledgments are automatically included only in the final version of the paper.
% \acknowledgments{If a paper is accepted, the final camera-ready version will (and probably should) include acknowledgments. All acknowledgments go at the end of the paper, including thanks to reviewers who gave useful comments, to colleagues who contributed to the ideas, and to funding agencies and corporate sponsors that provided financial support.}

%===============================================================================

% no \bibliographystyle is required, since the corl style is automatically used.
\bibliography{references}  % .bib

\end{document}
