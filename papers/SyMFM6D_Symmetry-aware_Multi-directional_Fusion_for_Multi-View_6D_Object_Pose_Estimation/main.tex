
\documentclass[letterpaper, 10 pt, journal, twoside]{IEEEtran}

\usepackage{amsmath}
\usepackage{amssymb}
\usepackage{bbm}
\usepackage{booktabs}
\usepackage{comment}
\usepackage{dblfloatfix}
\usepackage{graphicx} 
\usepackage{multirow}
\usepackage{microtype}
\usepackage{siunitx}

\sisetup{
detect-weight = true,
locale = US,
mode = text,
}

\usepackage[nobreak]{cite}

\let\labelindent\relax 
\usepackage[inline]{enumitem}
\newcommand{\enumlabel}{\textbf{(\roman*)}}

\usepackage{array}
\newcolumntype{M}[1]{>{\centering\arraybackslash}m{#1}}

\usepackage{subcaption}
\captionsetup{font=footnotesize}

\usepackage[breaklinks, hidelinks]{hyperref}  

\usepackage[capitalize]{cleveref}
\crefname{section}{Sec.}{Secs.}
\Crefname{section}{Section}{Sections}
\Crefname{table}{Table}{Tables}
\crefname{table}{Tab.}{Tabs.}

\newcommand{\bb}{\boldsymbol}  
\newcommand{\tb}{\textbf}  
\newcommand{\tu}{\underline}
\newcommand{\norm}[1]{\left\lVert#1\right\rVert}

\clubpenalty = 10000
\widowpenalty = 10000
\displaywidowpenalty = 10000

\title{
SyMFM6D: Symmetry-aware Multi-directional Fusion for Multi-View 6D Object Pose Estimation
}

\author{
Fabian Duffhauss$^{1, 2}$, 
Sebastian Koch$^{1, 3}$, 
Hanna Ziesche$^{1}$, 
Ngo Anh Vien$^{1}$, and
Gerhard Neumann$^{4}$% 
\thanks{$^{1}$Fabian Duffhauss, Sebastian Koch, Hanna Ziesche, and Ngo Anh Vien are with the Bosch Center for Artificial Intelligence, Renningen, Germany
        {\tt\footnotesize Fabian.Duffhauss@bosch.com}}%
\thanks{$^{2}$Fabian Duffhauss is with the University of Tübingen, Tübingen, Germany}%
\thanks{$^{3}$Sebastian Koch is with the Ulm University, Ulm, Germany}%
\thanks{$^{4}$Gerhard Neumann is with the Institute for Anthropomatics and Robotics, Karlsruhe Institute of Technology, Karlsruhe, Germany
        {\tt\footnotesize Gerhard.Neumann@kit.edu}}%
\thanks{Source code, datasets, and implementation details are available at \url{https://github.com/boschresearch/SyMFM6D}.}%
}

\begin{document}
\bstctlcite{IEEEexample:BSTcontrol}

\maketitle

\begin{abstract}
%\medskip
%\centering \textcolor{red}{Write the abstract last}
Silicon-compatible short- and mid-wave infrared emitters are highly sought-after for on-chip monolithic integration of electronic and photonic circuits to serve a myriad of applications in sensing and communication. To address this longstanding challenge, GeSn semiconductors have been proposed as versatile building blocks for silicon-integrated optoelectronic devices. In this regard, this work demonstrates light-emitting diodes (LEDs) consisting of a vertical PIN double heterostructure  p-Ge$_{0.94}$Sn$_{0.06}$/i-Ge$_{0.91}$Sn$_{0.09}$/n-Ge$_{0.95}$Sn$_{0.05}$ grown epitaxially on a silicon wafer using germanium interlayer and multiple GeSn buffer layers. The emission from these GeSn LEDs at variable diameters in the 40-120 $\mu$m range is investigated under both DC and AC operation modes. The fabricated LEDs exhibit a room temperature emission in the extended short-wave range centered around 2.5 $\mu$m under an injected current density as low as 45 A/cm$^2$.  By comparing the photoluminescence and electroluminescence signals, it is demonstrated that the LED emission wavelength is not affected by the device fabrication process or heating during the LED operation. Moreover, the measured optical power was found to increase monotonically as the duty cycle increases indicating that the DC operation yields the highest achievable optical power. The LED emission profile and bandwidth are also presented and discussed. 
\end{abstract}
\section{Introduction}

Scientific literature is most commonly available in the form of PDFs, which pose challenges for accessibility \citep{NielsenPDFStillUnfit, Bigham2016AnUT}. When researchers, students, and other individuals who are blind or low vision (BLV) interact with scientific PDFs through screen readers, the availability of document structure tags, labeled reading order, labeled headers, and image alt-text are necessary to facilitate these interactions. However, these features must be painstakingly added by authors using proprietary software tools, and as a result, are often missing from papers. Low vision or dyslexic readers who interact with PDFs through screen magnification or text-to-speech may also find the complexity of certain academic paper PDF formats challenging, e.g., non-linear layout can interrupt the flow of text in a magnifying tool. Inaccessible paper PDFs can lead to high cognitive overload, frustration, and abandonment of reading for BLV readers. 

Unfortunately, we find that the majority of scientific PDFs lack basic accessibility features. We estimate based on a sample of \numpdfs PDFs from multiple fields of study that only around \percaccessible of paper PDFs released in the last decade satisfy all of the aforementioned accessibility requirements. 
Accessibility challenges for academic PDFs are largely due to three factors: (1) the complexity of the PDF file format, which make it less amenable to certain accessibility features, (2) the dearth of tools, especially non-proprietary tools, for creating accessible PDFs, and (3) the dependency on volunteerism from the community with minimal support or enforcement \citep{Bigham2016AnUT}. The intent of the PDF file format is to support faithful visual representation of a document for printing, a goal that is inherently divergent from that of document representation for the purposes of accessibility. Though some professional organizations like the Association for Computing Machinery (ACM) have encouraged PDF accessibility through standards and writing guidelines,\footnote{\href{https://www.acm.org/publications/authors/submissions}{https://www.acm.org/publications/authors/submissions}} uptake among academic publishers and disciplines more broadly has been limited. 

While policy changes help, the fact remains that most academic PDFs produced today, and historically, are inaccessible, yet remain as the dominant way to read those papers. A long-range solution will necessitate buy-in from multiple stakeholders---publishers, authors, readers, technologists, granting agencies, and the like. But in the interim, there are technological solutions that can be offered as a sort of ``band-aid'' to the problem. We use this paper to offer an in-depth qualitative and quantitative description of the problem as it stands, and to introduce one such technological solution: the \scially system that automatically extracts semantic information from paper PDFs and re-renders this content in the form of an accessible HTML document. Though the process is imperfect and can introduce errors, we demonstrate the ability of the rendered HTMLs to reduce cognitive load and facilitate in-paper navigation and interactions for BLV users. 

The goals and contributions of this paper are three-fold:

\begin{enumerate}
    \item We characterize the state of academic-paper PDF accessibility by estimating the degree of adherence to accessibility criteria for papers published in the last decade (2010--2019), and describe correlations between year, field of study, PDF typesetting software, and PDF accessibility.
    \item We propose an automated approach for extracting the content of academic PDFs and displaying this content in a more accessible HTML document format. We build a prototype that re-renders 12 million PDFs in HTML, and describe the design decisions, features, and quality of the renders (assessed as faithfulness to the source PDF). We perform expert grading of the rendered HTML and report an error analysis. A demo of our system is available at \href{https://scia11y.org/}{scia11y.org}, which makes available 1.5M HTML renders of open access PDFs.
    \item We conduct an exploratory user study with \numusers BLV scholars to better understand the challenges they experience when reading academic papers and how our proposed tool might augment their current workflow. During the study, we ask users to interact with the prototype and offer feedback for its improvement. We perform open coding of interviews to identify existing reading challenges, coping mechanisms, as well as positive and negative responses to prototype features. We summarize the findings of this user study into a set of design recommendations.
\end{enumerate}

Our analysis reveals that PDF accessibility adherence is low across all fields of study. Of the five accessibility criteria we assess, only \percaccessible of the PDFs we assess demonstrate full compliance. Though compliance for several criteria seems to be increasing over time, author awareness and contribution to accessibility remains low, as Alt-text has the lowest compliance of the five criteria at between 5--10\% (Alt-text is the only criterion of the five that \textit{requires} author intervention in all cases using current tools). We also find that typesetting software is strongly associated with accessibility compliance, with LaTeX and publishing software like Arbortext APP producing low compliance PDFs, while Microsoft Word is generally associated with higher compliance.


\begin{figure}[t!]
    \centering
    \includegraphics[width=\textwidth]{figures/pipeline.png}
    \caption{A schematic for creating the \scially HTML render from a paper PDF. Starting with the raw two-column PDF on the left, S2ORC \citep{lo-wang-2020-s2orc} is used to extract title, authors, abstract, section headers, body text, and references. S2ORC also identifies links between inline citations and references to figures and table objects. DeepFigures \citep{Siegel2018ExtractingSF} is used to extract figures and tables, along with their captions. The output of these two models are merged with metadata from the Semantic Scholar API. Heuristics are used to construct a table of contents, to insert figures and tables in the appropriate places in the text, and to repair broken URLs. We add HTML headers as illustrated (header tags for sections, paragraph tags for body text, and figure tags for figures and tables); highlighted components (table of contents and links in references) are not in the PDF and novel navigational features that we introduce to the HTML render. An example HTML render of parts of a paper document is show to the right (actual render is single column, which is split here for presentation).}
    \label{fig:pipeline}
    \Description{A schematic diagram showing the components of the SciA11y pipeline. An image of a paper PDF is on the left. Red boxes on the PDF image highlight the text components from the paper, with an arrow pointing to a box that says "S2ORC extracts: title, authors, abstract, section headers, body paragraphs, and references." A blue box on the PDF image highlights a figure, with an arrow pointing to a box that says "DeepFigures extracts: figures, figure captions, tables, and table titles/captions." A box below "S2ORC extracts" and "DeepFigures extracts" says "Additional content: metadata from Semantic Scholar API, table of contents, figures and tables inserted at first mention, and links between references and text." Arrows from all three boxes point into a larger box that describes the SciA11y prototype, where HTML tags are inserted around various blocks of text extracted from the PDF. On the right of all this is a screen capture of an example HTML render, showing how the semantic content from the PDF is represented as a single-column HTML page for easy reading.}
\end{figure}

To offset the reading challenges of inaccessible papers for BLV researchers, we propose and test the \scially system for rendering academic PDFs into accessible HTML documents. As shown in Figure~\ref{fig:pipeline}, our prototype integrates several machine learning text and vision models to extract the structure and semantic content of papers. The content is represented as an HTML document with headings and links for navigation, figures and tables, as well as other novel features to assist in document structure understanding. Our evaluation of the \scially system identifies common classes of extraction problems, and finds that though many papers exhibit some extraction errors, the majority (55\%) have no major problems that impact readability, and another 32\% have only some problems that impact readability.

Through our user study, we identify numerous challenges faced by BLV users when reading paper PDFs, including some that affect the whole document or limit navigation, and many that affect the ability of the reader to understand text or various elements of a paper like math content or tables. Responses to \scially were positive; participants especially liked navigation features such as headings, the table of contents, and bidirectional links between inline citations and references. Of the extraction errors in \scially, missed or incorrectly extracted headings were the most problematic, as these impact the user's ability to navigate between sections and fully trust the system. All users reported being likely to use the system in the future. When asked how the system might be integrated into their workflow, one participant replied ``I think it would become the workflow.'' Another participant said, ``for unaccessible PDFs, this is life-changing.'' We condense these findings into a set of recommendations for designing and engineering accessible reading systems (Section~\ref{sec:designrecs}). Most importantly, documents should be structured to match a reader's mental model, objects should be properly tagged, and care should be taken to reduce the reader's cognitive load and increase trust in the system. Features that emulate the external memory that visual layout provides to sighted users can be especially beneficial.

This paper is organized as follows. Following a description of related work in Section \ref{sec:related_work}, we first provide a meta-scientific analysis of the current state of academic PDF accessibility in Section \ref{sec:sos}. In Section \ref{sec:pdf2html}, we document our pipeline for converting PDF to HTML and describe the \scially prototype for rendering papers. An evaluation of HTML render quality and faithfulness is provided in Section \ref{sec:evaluation}. Section \ref{sec:user_study} describes our user study and findings. 
We recognize that no PDF extraction system is perfect, and many open research challenges remain in improving these systems. However, based on our findings, we believe \scially can dramatically improve screen reader navigation of most papers compared to PDFs, and is well-positioned to assist BLV researchers with many of their most common reading use cases. Our hope is that a system such as \scially can improve BLV researcher access to the content of academic papers, and that these design recommendations can be leveraged by others to create better, more faithful, and ultimately more usable tools and systems for scholars in the BLV community.

\section{Related Work}
\label{sec:related_work}
% In this section, we review the related work, which includes graph neural networks, and robust graph neural networks. 

\subsection{Graph Neural Networks}
Graph Neural Networks (GNNs) have shown their great power in modeling graph structured data for various applications~\cite{wang2019semi,wang2018cross,zhao2020semi,dai2021say,zhao2021graphsmote}.
To generalize neural networks for graphs, two categories of GNNs are proposed, i.e., spectral-based~\cite{bruna2013spectral,henaff2015deep,kipf2016semi,levie2018cayleynets} and spatial-based~\cite{velivckovic2017graph,hamilton2017inductive,chen2018fastgcn,chiang2019cluster}. \citeauthor{bruna2013spectral} \cite{bruna2013spectral} first propose spectral-based GNNs by defining graph convolution with spectral graph theory. For instance, GCN~\cite{kipf2016semi} simplifies the convolutional operation by using the first order approximation. Spatial-based graph convolution is defined in spatial domain, which updates node representation by aggregating its neighbors' representations \cite{niepert2016learning,gilmer2017neural,hamilton2017inductive}. 
For example, self-attention of neighbor nodes is leveraged in graph attention network (GAT) \cite{velivckovic2017graph}. Moreover, various spatial methods are proposed to solve the scalability issue~\cite{chen2018fastgcn,chiang2019cluster} and learn deeper GNNs~\cite{chen2020simple}.  Recently, to alleviate the problem of lacking labeled nodes, many efforts are taken to explore GNNs using self-supervision, which aims to learn better node representations with pretext tasks~\cite{sun2019multi,li2018deeper,kim2021find,zhu2020self,jin2020self,dai2021towards}. For instance, superGAT~\cite{kim2021find} deploys edge prediction in GAT to guide the learning of attention for better representations. SE-GNN~\cite{dai2021towards} deploys contrastive learning to benefit the similarity modeling for self-explainable GNN.

Inspired by the great success of GNNs, methods that construct graphs and adopt GNNs for data without explicit relational structure are also explored~\cite{henaff2015deep,chen2019multi,jiang2019semi,dai2021nrgnn}. Generally, a graph would be built based on certain rules~\cite{henaff2015deep,chen2019multi} or be learned in an end-to-end model~\cite{jiang2019semi,dai2021nrgnn}. Our RS-GNN is inherently different from these methods as we eliminate/down-weight the noisy edges and predict the missing edges for robust GNNs on noisy graphs with limited labels. 

\subsection{Robust GNNs}
Although GNNs have obtained great achievements, they are vulnerable to adversarial attacks~\cite{wu2019adversarial,dai2018adversarial,zugner2018adversarial,zugner2019adversarial}. Based on the objective, the adversarial attacks on GNNs can be split into two categories, i.e., targeted attack~\cite{dai2018adversarial,zugner2018adversarial} and non-targeted attack~\cite{zugner2019adversarial}. Targeted attack methods aim to degrade the performance of the GNNs on target nodes. 
For instance, \textit{nettack}~\cite{zugner2018adversarial} adds adversarial perturbations to a graph to attack targeted nodes. Non-targeted attack aims to reduce the overall performance of GNNs. For example, \textit{metattack}~\cite{zugner2019adversarial} poisons the graph globally to achieve non-targeted attack with meta-learning. To defend against adversarial attacks, many efforts are taken recently~\cite{zhu2019robust,wu2019adversarial,entezari2020all,jin2020graph,tang2020transferring,zhang2020gnnguard}. \cite{wu2019adversarial} prune the perturbed edges based on Jaccard similarity of node features. Another preprocessing method by low-rank approximation of adjacent matrix is investigated~\cite{entezari2020all}. Pro-GNN~\cite{jin2020graph} is the most similar work to ours, which learns a clean graph structure by low-rank constraint. However, they only tackle the adversarial edges and their computational cost is very large due to the direct learning of the graph and the sparse low-rank constraint.
This work is inherently different from these methods as: (i) we study a novel problem of developing robust GNN for both noisy graphs and label sparsity issues; and (ii) the proposed RS-GNN simultaneously tackles the two issues by learning an link predictor to 
down-weight noisy edges and connecting nodes with high similarity to facilitate message-passing; 
and (iii) RS-GNN uses link predictor instead of direct graph learning to save computational cost. 
\section{Method}
Fig.~\ref{fig:framework} presents the illustration of the proposed \frameworkName.
In this section,  
we start by providing the problem definition of online CIL. Then, we describe the definition of the online prototype, the proposed online prototype equilibrium, and the proposed adaptive prototypical feedback. Finally, we propose an online prototype learning framework.

\subsection{Problem Definition}
Formally, online CIL considers a continuous sequence of tasks from a single-pass data stream $\mathfrak{D}=\left\{\mathcal{D}_1, \ldots, \mathcal{D}_T \right\} $, where $\mathcal{D}_t = \left\{ x_{i}, y_{i} \right\} ^{N_t}_{i=1} $ is the dataset of task $t$, and $T$ is the total number of tasks. Dataset $\mathcal{D}_t$ contains $N_t$ labeled samples, $y_{i}$ is the class label of sample $x_{i}$ and $y_{i} \in \mathcal{C}_t$, where $\mathcal{C}_t$ is the class set of task $t$ and the class sets of different tasks are disjoint. 
For replay-based methods, a memory bank is used to store a small subset of seen data, and we also maintain a memory bank $\mathcal{M}$ in our method.
At each time step of task $t$, the model receives a mini-batch data $X \cup X^\mathrm{b}$ for training, where $X$ and $X^\mathrm{b}$ are drawn from the i.i.d distribution $\mathcal{D}_t$ and the memory bank $\mathcal{M}$, respectively. 
Moreover, we adopt the single-head evaluation setup~\cite{EWC}, where a unified classifier must choose labels from all seen classes at inference due to unavailable task identifiers. 
The goal of online CIL is to train a unified model on data seen only once while predicting well on both new and old classes.

\subsection{Online Prototype Definition}
Prior to introducing the online prototypes, we first present the network architecture of our \frameworkName. Suppose that the model consists of three components: an encoder network $f$, a projection head $g$, and a classifier $\varphi$. Each sample $x$ in incoming data $X$ (a mini-batch data from new classes) is mapped to a projected vectorial embedding (representation) $\mathbf{z}$ by encoder $f$ and projector $g$:
\begin{align}
\label{eq:cal_z}
    \mathbf{z} = g(f(\operatorname{aug}(x);\theta_f);\theta_g),
\end{align}
where $\operatorname{aug}$ represents the data augmentation operation, $\theta_f$ and $\theta_g$ represent the parameters of $f$ and $g$, respectively, and $\mathbf{z}$ is $\ell_2$-normalized. 
Similar to Eq.~\eqref{eq:cal_z}, we use $\mathbf{z}^\mathrm{b}$ to denote the representation of replay data $X^\mathrm{b}$ (a mini-batch data from seen classes in the memory bank). 

At each time step of task $t$, the online prototype of each class is defined as the mean representation in a mini-batch:
\begin{align}
\label{eq:cal_p}
    \mathbf{p}_i = \frac{1}{n_i}\sum\nolimits_j\mathbf{z}_j\cdot \mathbbm{1}\{y_j = i\},
\end{align}
where $n_i$ is the number of samples for class $i$ in a mini-batch, and $\mathbbm{1}$ is the indicator function. 
We can get a set of $K$ online prototypes  in $X$, $\mathcal{P} = \left\{ \mathbf{p}_{i} \right\} ^{K}_{i=1}$, and a set of $K^\mathrm{b}$ online prototypes in $X^\mathrm{b}$, $\mathcal{P}^\mathrm{b} = \left\{ \mathbf{p}_i^\mathrm{b} \right\} ^{K^\mathrm{b}}_{i=1}$.
Note that $K = |\mathcal{P}| \leq |\mathcal{C}_t|$ and $K^\mathrm{b} = |\mathcal{P}^\mathrm{b}| \leq \sum_{i=1}^{t}|\mathcal{C}_i| $, where $|\cdot|$ denotes the cardinal number.



\subsection{Online Prototype Equilibrium}
The introduced online prototypes can provide representative features and avoid class-unrelated information.  
These characteristics are exactly the key to counteracting shortcut learning in online CL.
Besides, maintaining the discrimination among seen classes is also essential to mitigate catastrophic forgetting.
Based on these, we attempt to learn representative features of each class by pulling online prototypes $\mathcal{P}$ and their augmented views $\widehat{\mathcal{P}}$ closer in the embedding space, and learn discriminative features between classes by pushing online prototypes of different classes away, formally defined as a contrastive loss:
\begin{align}
\label{eq:proto_infoNCE}
    \ell(\mathcal{P},\widehat{\mathcal{P}})\!=\!
    % \frac{-1}{K}
    \frac{-1}{|\mathcal{P}|}\sum_{i=1}^{|\mathcal{P}|}\!\log\! 
    \tfrac
    {\exp \big(\tfrac{{\mathbf{p}_i^\mathrm{T} \widehat{\mathbf{p}}_i}}{\tau}\big)}
    {
    \sum\limits_{j} \exp \big(\tfrac{{\mathbf{p}_i^\mathrm{T} \widehat{\mathbf{p}}_j}}{\tau}\big)
    +\!
    \sum\limits_{\substack{j \neq i}} \exp \big(\tfrac{{\mathbf{p}_i^\mathrm{T} \mathbf{p}_j}}{\tau}\big) 
    },
\end{align}
where 
$\tau$ is the temperature hyper-parameter, 
$\mathcal{P}$ and $\widehat{\mathcal{P}}$ are $\ell_2$-normalized. To compute the contrastive loss across all positive pairs in both $(\mathcal{P}, \widehat{\mathcal{P}})$ and $(\widehat{\mathcal{P}}, \mathcal{P})$, we define $\mathcal{L}_{\mathrm{pro}}$ as the final contrastive loss over online prototypes:
\begin{align}
    \mathcal{L}_{\mathrm{pro}}(\mathcal{P},\widehat{\mathcal{P}}) = 
    \frac{1}{2}
    \left[\ell(\mathcal{P}, \widehat{\mathcal{P}}) + \ell(\widehat{\mathcal{P}}, \mathcal{P})\right].
\end{align}



Considering the learning of new classes and the consolidation of learned knowledge simultaneously in online CL, we propose Online Prototype Equilibrium (\methodname) to 
learn representative and discriminative features on both new and seen classes by employing $\mathcal{L}_{\mathrm{pro}}$:
\begin{equation}
    \begin{aligned}
    \mathcal{L}_{\mathrm{\methodname}}
    &=
    \mathcal{L}^{\mathrm{new}}_{\mathrm{pro}}(\mathcal{P},\widehat{\mathcal{P}}) + \mathcal{L}^{\mathrm{seen}}_{\mathrm{pro}}(\mathcal{P}^\mathrm{b},\widehat{\mathcal{P}}^\mathrm{b}),
    \end{aligned}
\end{equation}
where
$\mathcal{L}^{\mathrm{new}}_{\mathrm{pro}}$ focuses on learning knowledge from \emph{new} classes, and $\mathcal{L}^{\mathrm{seen}}_{\mathrm{pro}}$ is dedicated to preserving learned knowledge of all \emph{seen} classes.
\textit{This process is similar to a zero-sum game, 
and \methodname aims to achieve the equilibrium to play a win-win game.}
Concretely,
as the model learns, the knowledge of new classes is gained and added to the prototypes over the memory bank $\mathcal{M}$, causing $\mathcal{L}^{\mathrm{seen}}_{\mathrm{pro}}$ gradually changes to the equilibrium that separates all seen classes well, including new ones. 
This variation is crucial to mitigate forgetting and is consistent with the goal of CIL.



\subsection{Adaptive Prototypical Feedback} 
Although \methodname can bring an overall equilibrium, it tends to treat each class \emph{equally}. 
In fact, the degree of confusion varies among classes, 
and the model should focus purposefully on confused classes to consolidate learned knowledge. 
To this end, we propose Adaptive Prototypical Feedback (\dataaugname) with the feedback of online prototypes to sense the classes that are prone to be misclassified and then enhance their decision boundaries.
 
For each class pair in the memory bank $\mathcal{M}$, \dataaugname calculates the distances between online prototypes of all seen classes from the previous time step, showing the class confusion status by these distances. The closer the two prototypes are, the easier to be misclassified. 
Based on this analysis, 
our idea is to enhance the boundaries for those classes. Therefore, we convert the prototype distance matrix to a probability distribution $P$ over the classes via a symmetric Gaussian kernel, defined as follows:
\begin{align}
\label{eq:cal_d}
    P_{i, j} \propto \exp (-\| \mathbf{p}_i^\mathrm{b} - \mathbf{p}_j^\mathrm{b} \|_2^2),
\end{align}
where $i,j \in \{ 1, \ldots, |\mathcal{P}^\mathrm{b}| \}$ and $i \neq j$. 
Then, 
all probabilities are normalized to a probability mass function that sums to one.
\dataaugname returns probabilities to $\mathcal{M}$ for guiding the next sampling process and enhancing decision boundaries of easily misclassified classes. 


Our adaptive prototypical feedback is implemented as a sampling-based mixup. Specifically, 
\dataaugname adaptively selects more samples from easily misclassified classes in $\mathcal{M}$ for mixup~\cite{Mixup} according to the probability distribution $P$. 
Considering not over-penalizing the equilibrium of current online prototypes, we introduce a two-stage sampling strategy for replay data $X^\mathrm{b}$ of size $m$. 
First, we select $n_{\mathrm{\dataaugname}}$ samples  
with $P$, and a larger $P_{a,b}$ means more sampling from classes $a$ and $b$. Here, $n_{\mathrm{\dataaugname}} = \alpha \cdot m$, and $\alpha$ is the ratio of \dataaugname.
Second, the remaining $m-n_{\mathrm{\dataaugname}}$ samples are uniformly randomly selected from the entire memory bank to avoid the model only focusing on easily misclassified classes and disrupting the established equilibrium. 




\subsection{Overall Framework of \frameworkName}
The overall structure of \frameworkName is shown in Fig.~\ref{fig:framework}. \frameworkName comprises two key components based on proposed online prototypes: Online Prototype Equilibrium (\methodname) and Adaptive Prototypical Feedback (\dataaugname). 
With the two components, 
the model can learn representative features against shortcut learning, and 
all seen classes maintain discriminative when learning new classes. 
However, classes may not be compact, because the online prototypes cannot cover full instance-level information.
To further achieve intra-class compactness, 
we employ supervised contrastive learning~\cite{SupCL} to learn instance-wise representations:
\begin{equation}
\begin{aligned}
    \mathcal{L}_{\mathrm{INS}}
    &=
    \sum_{i=1}^{2N} \frac{-1}{\left|I_i\right|} \sum_{j \in I_i} \log \frac{\exp \left(\mathrm{sim}(\mathbf{z}_i, \mathbf{z}_j) / \tau^{\prime}\right)}{\sum\limits_{k \neq i} \exp \left(\mathrm{sim}(\mathbf{z}_i, \mathbf{z}_k) / \tau^{\prime}\right)}
    \\
    &+
    \sum_{i=1}^{2m} \frac{-1}{\left|I_i^{\mathrm{b}}\right|} \sum_{j \in I_i^{\mathrm{b}}} \log \frac{\exp (\mathrm{sim}(\mathbf{z}_i^{\mathrm{b}}, \mathbf{z}_j^{\mathrm{b}}) / \tau^{\prime})}{\sum\limits_{k \neq i} \exp \left(\mathrm{sim}(\mathbf{z}_i^{\mathrm{b}}, \mathbf{z}_k^{\mathrm{b}}) / \tau^{\prime}\right)},
\end{aligned}
\end{equation}
where $I_i=\left\{j \in\{1, \ldots, 2 N\} \mid j \neq i, y_j=y_i\right\}$ and $I_i^\mathrm{{b}}=\left\{j \in\{1, \ldots, 2m\} \mid j \neq i, y_j^\mathrm{b}=y_i^\mathrm{b}\right\}$ are the set of positive samples indexes to $\mathbf{z}_i$ and $\mathbf{z}_i^\mathrm{{b}}$, respectively. $y_i^\mathrm{b}$ is the class label of input $x_i^\mathrm{b}$ from $X^\mathrm{b}$. $N$ is the batch size of $X$. $\tau^{\prime}$ is the temperature hyperparameter.
The similarity function $\mathrm{sim}$ is computed in the same way as Eq.~(9) in OCM~\cite{OCM}.

Thus, the total loss of our \frameworkName framework is given as:
\begin{align}
    \mathcal{L}_{\mathrm{\frameworkName}}=\mathcal{L}_{\mathrm{\methodname}} + \mathcal{L}_{\mathrm{INS}} + \mathcal{L}_{\mathrm{CE}},
\end{align}
where $\mathcal{L}_{\mathrm{CE}} = \mathrm{CE}(y^\mathrm{b}, \varphi(f(\operatorname{aug}(x^\mathrm{b}))))$ is the cross-entropy loss; see Appendix~\ref{appendix:algorithm} for detailed training algorithms.

Following other replay-based methods~\cite{ER, SCR, OCM}, we update the memory bank in each time step by uniformly randomly selecting samples from $X$ to push into $\mathcal{M}$ and, if $\mathcal{M}$ is full, pulling an equal number of samples out of $\mathcal{M}$.


\section{Experiments}

To demonstrate the performance of our method in comparison to related approaches, we perform extensive experiments on four very challenging datasets.



\subsection{Datasets}

The {\bf YCB-Video dataset} \cite{posecnn} contains a total of 133,827 RGB-D images showing 92 scenes composed of three to nine objects from the 21 Yale-CMU-Berkeley (YCB) objects \cite{ycb}.
Additionally, there are 80,000 synthetic non-sequential \mbox{RGB-D} frames showing a random subset of the YCB objects placed at random positions.

However, most frames from YCB-Video are very similar because they originate from videos with 30 frames per second recorded by a handheld camera that was moved slowly. The videos also do not show the scene from all sides but just from similar perspectives. Furthermore, the scenes do not include strong occlusions, and hence, most object poses are simple to estimate from a single perspective. 
Therefore, we additionally consider the recently proposed photorealistic synthetic datasets {\bf MV-YCB FixCam} and {\bf MV-YCB WiggleCam} \cite{mv6d} as they contain much more difficult scenes with strong occlusions and diverse camera perspectives.
Both datasets depict 8,333 cluttered scenes composed of eleven non-symmetric YCB objects which are randomly arranged so that strong occlusions occur. Each scene is photorealistically rendered from three very different perspectives providing 24,999 RGB-D images with accurate ground truth annotations. Unlike FixCam which uses fixed camera positions while providing accurate camera poses, WiggleCam has varying camera poses which are inaccurately annotated on purpose.

Since FixCam and WiggleCam contain only non-symmetric objects, we created an additional photorealistic synthetic dataset with symmetric and non-symmetric objects called {\bf MV-YCB SymMovCam} using Blender with physically based rendering and domain randomization as in \cite{mv6d}. It also depicts 8,333 cluttered scenes, but they are composed of 8 -- 16 objects randomly chosen from the 21 YCB objects which results in very strong occlusions. For each scene, we created four cameras at changing positions around the scene with the restriction that in each quadrant there is only one camera so that the perspectives are very distinct. This results in a total of 33,332 annotated RGB-D images.



\subsection{Training Procedure}

For training our model in single-view mode on YCB-Video, we randomly use the synthetic and real images of YCB-Video with a ratio of 4:1. Since consecutive real frames are very similar, we consider only every seventh real frame. For training a multi-view model, we start from the corresponding single-view checkpoint and continue training with batches of real YCB-Video frames. 
For training on FixCam and WiggleCam we follow \cite{mv6d} and use random permutations of the three available camera views. For SymMovCam, we take a random subset of three views from the available four views.



\subsection{Evaluation Metrics}
\label{sec_eval_metrics}

We evaluated our method using the area-under-curve (AUC) metrics for ADD-S and \mbox{ADD(-S)} and the precision metrics ADD-S \textless ~\SI{2}{cm} and \mbox{ADD(-S)} \textless ~\SI{2}{cm} as these metrics are most commonly used in related work \cite{cosypose, ffb6d, mv6d}. 



\subsection{Baseline Methods}

We compare our methods with many established and some very recent methods namely 
DenseFusion \cite{densefusion}, CosyPose \cite{cosypose}, PVN3D \cite{pvn3d}, FFB6D \cite{ffb6d}, ES6D \cite{es6d}, and MV6D \cite{mv6d}.


\subsection{Results on YCB-Video}

\begin{table}[b]
    \centering
\begin{tabular}{l|ccc}
    \toprule
                           &    ADD-S  &    ADD(-S) \\\midrule
DenseFusion (per-pixel)    &     91.2  &     82.9   \\ 
DenseFusion (iterative)    &     93.2  &     86.1   \\
CosyPose                   &     89.8  &     84.5   \\
PVN3D                      &     95.5  &     91.8   \\     
FFB6D                      &     96.6  &     92.7   \\     
ES6D                       &     93.6  &     89.0   \\   
SyMFM6D                    & \tb{96.8} & \tb{94.1}  \\ 
\bottomrule
\end{tabular}
    \caption{Single-view results on YCB-Video using the AUC metrics for ADD-S and \mbox{ADD(-S)}. The best results are printed in bold.}
    \label{tab_ycbv_sv}
\end{table}

\cref{tab_ycbv_sv} compares the single-view performance of our SyMFM6D network with all baseline methods using the AUC of ADD-S and \mbox{ADD(-S)} on YCB-Video. Please note that MV6D corresponds to PVN6D in the single-view scenario. The results show that our approach copes very well with the dynamic camera setup of YCB-Video while outperforming all methods significantly. On the symmetry-aware \mbox{ADD(-S)} AUC metric, SyMFM6D outperforms the current state-of-the-art FFB6D by even \SI{1.5}{\%}. 
Please note that unlike DenseFusion (iterative) and CosyPose, our approach does not perform computationally expensive post processing or iterative refinement procedures.


To examine the effect of our symmetry-aware training procedure, we provide an object-wise evaluation of the three best single-view methods on YCB-Video in \cref{fig_ycb_sv_objects}. Please note that in single-view mode, our model architecture is the same as FFB6D except for our novel symmetry-aware loss function. 
The results show that not only most symmetric objects (highlighted in bold) are estimated more accurate but also most non-symmetric objects.
This indicates that there is a synergy effect which improves the keypoint detection for non-symmetric objects due to an improvement of the keypoint detection for symmetric objects.

\begin{table}[tbp]
    \vspace{2mm}
    \centering
    \begin{tabular}{l|ccc}
        \toprule 
        Object class  		   &   PVN3D  &   FFB6D  &  SyMFM6D  \\\midrule
        Master chef can        &    80.5  &    80.6  &\tb{80.7} \\
        Cracker box            &    94.8  &    94.6  &\tb{94.9} \\
        Sugar box              &    96.3  &\tb{96.6} &\tb{96.6} \\
        Tomato soup can        &    88.5  &\tb{89.6} &    87.9  \\
        Mustard bottle         &    96.2  &    97.0  &\tb{97.8} \\
        Tuna fish can          &    89.3  &    88.9  &\tb{92.3} \\
        Pudding box            &\tb{95.7} &    94.6  &    93.3  \\
        Gelatin box            &    96.1  &\tb{96.9} &    96.1  \\
        Potted meat can        &    88.6  &    88.1  &\tb{90.0} \\
        Banana                 &    93.7  &    94.9  &\tb{95.2} \\
        Pitcher base           &    96.5  &    96.9  &\tb{97.5} \\
        Bleach cleanser        &    93.2  &\tb{94.8} &    93.9  \\
        \tb{Bowl}              &    90.2  &    96.3  &\tb{96.4} \\
        Mug                    &    95.4  &    94.2  &\tb{95.7} \\
        Power drill            &    95.1  &    95.9  &\tb{96.4} \\
        \tb{Wood block}        &    90.4  &    92.6  &\tb{95.2} \\
        Scissors               &    92.7  &    95.7  &\tb{95.8} \\
        Large marker           &\tb{91.8} &    89.1  &    90.0  \\
        \tb{Large clamp}       &    93.6  &    96.8  &\tb{96.9} \\
        \tb{Extra large clamp} &    88.4  &\tb{96.0} &    95.3  \\
        \tb{Foam brick}        &    96.8  &    97.3  &\tb{97.6} \\\midrule
        ALL                    &    91.8  &    92.7  &\tb{94.1} \\\bottomrule
    \end{tabular} 
	\caption{Single-view results on YCB-Video evaluated for each object class individually using the \mbox{ADD(-S)} AUC metric. Symmetric objects and the best results are printed in bold.}
	\label{fig_ycb_sv_objects}
\end{table}

\cref{fig_ycbv_sv} shows a visualization of three scenes of YCB-Video with 6D pose ground truth, predictions of FFB6D, and predictions of our SyMFM6D network using only the depicted view. It can be seen that both FFB6D and SyMFM6D estimate very accurate poses as the scenes of YCB-Video contain only a few objects and not many occlusions. However, SyMFM6D predicts even more accurate poses than FFB6D due to our proposed symmetry-aware training procedure.

\begin{figure*}[htbp]
        \vspace{2mm}
	\centering
	\begin{minipage}{0.24\textwidth}
		\centering
		\textbf{Original View}
	\end{minipage}%
	\begin{minipage}{0.24\textwidth}
		\centering
		\textbf{Ground Truth}
	\end{minipage}%
	\begin{minipage}{0.24\textwidth}
		\centering
		\textbf{FFB6D}
	\end{minipage}%
	\begin{minipage}{0.24\textwidth}
		\centering
		\textbf{SyMFM6D}
	\end{minipage}% 
	\setkeys{Gin}{width=0.24\linewidth}
	\includegraphics{figures/ycb_sv/0055_001042_rgb_1800.png}\,%
	\includegraphics{figures/ycb_sv/0055_001042_gt_1800.png}\,%
	\includegraphics{figures/ycb_sv/0055_001042_FFB6D_1view_1800.png}\,%
	\includegraphics{figures/ycb_sv/0055_001042_SyMV6D_16sym_1view.png}\,%
	\vspace{0.7mm}
	
	\includegraphics{figures/ycb_sv/0051_000636_rgb_864.png}\,%
	\includegraphics{figures/ycb_sv/0051_000636_gt_864.png}\,%
	\includegraphics{figures/ycb_sv/0051_000636_FFB6D_1view_864.png}\,%
	\includegraphics{figures/ycb_sv/0051_000636_SyMV6D_16sym_1view_864.png}\,%
	\vspace{0.7mm}
	
	\includegraphics{figures/ycb_sv/0058_000553_rgb_2461.png}\,%
	\includegraphics{figures/ycb_sv/0058_000553_gt_2461.png}\,%
	\includegraphics{figures/ycb_sv/0058_000553_FFB6D_1view_2461.png}\,%
	\includegraphics{figures/ycb_sv/0058_000553_SyMV6D_16sym_1view.png}\,%
	\caption{Comparison of 6D pose predictions on single frames of the YCB-Video dataset.}
	\label{fig_ycbv_sv}
\end{figure*}

\cref{tab_ycbv_mv} compares our multi-view results with all multi-view baseline methods on YCB-Video using three and five input views.
We see that our approach with disabled symmetry training procedure already outperforms all previous multi-view methods significantly. Enabling the symmetry awareness further improves the results slightly. However, 
using more views does not improve the accuracy as most views of YCB-Video are very similar in which case additional views do not provide beneficial information while the learning problem of fusing different views becomes slightly harder.

\begin{table}[!hbt]
    \tabcolsep=1.35mm
    \centering
\begin{tabular}{l|cc|cc}
    \toprule 
                  & \multicolumn{2}{c|}{ADD-S}
                                          & \multicolumn{2}{c}{ADD(-S)} \\
                  &  3 views  &  5 views  &  3 views  &  5 views  \\\midrule
CosyPose          &     92.3  &     93.4  &     87.7  &     88.8  \\
MV6D              &     91.2  &     91.1  &     85.6  &     84.0  \\
SyMFM6D (no sym)  &     95.2  &     95.2  &     91.5  &     91.4  \\
SyMFM6D           & \tb{95.4} & \tb{95.4} & \tb{91.7} & \tb{91.6} \\
\bottomrule
\end{tabular}
    \caption{Quantitative multi-view results on YCB-Video. The best results are printed in bold.}
    \label{tab_ycbv_mv}
\end{table}



\subsection{Results on MV-YCB FixCam, WiggleCam and SymMovCam}

We show the quantitative results on the datasets MV-YCB FixCam, MV-YCB WiggleCam, and MV-YCB SymMovCam in \cref{tab_fixCam_wiggleCam}. It includes a comparison with two modified CosyPose (CP) versions with and without known camera poses as presented by \cite{mv6d}.
Our SyMFM6D network yields the best results on all metrics on all three datasets. This shows that SyMFM6D copes very well with the strong occlusions in the datasets. The results on WiggleCam are just slightly worse than on FixCam which demonstrates that our approach is robust towards inaccurately known camera poses.

On the novel SymMovCam dataset, our method outperforms the baselines by a much larger margin than on FixCam and WiggleCam. This is due to the symmetric objects in the datasets on which the keypoint estimation of the baseline methods is inaccurate. The results also prove that our approach is robust to very dynamic camera setups where the cameras are mounted at varying positions.


\begin{table*}[h]
	\tabcolsep=1.0mm
	\centering
	\begin{tabular}{r|cccccc|cccccc|ccccc}
    \toprule
                           &     \multicolumn{6}{c|}{MV-YCB FixCam}                        &      \multicolumn{6}{c|}{MV-YCB WiggleCam}                    &      \multicolumn{5}{c}{MV-YCB SymMovCam}            \\
                           &  PVN3D   &   FFB6D  &   CP   &    CP    &   MV6D   &   Ours   &  PVN3D   & FFB6D    &   CP   &   CP     &  MV6D    &  Ours    &  PVN3D   & FFB6D    &   Ours   &    MV6D  &  Ours    \\ 
    Number of views        &   1      &     1    &   3    &    3     &   3      &    3     &    1     & 1        &   3    &   3      &   3      &    3     &    1     &    1     &     1    &     3    &    3     \\
    Known cam poses        &\checkmark&\checkmark&$\times$&\checkmark&\checkmark&\checkmark&\checkmark&\checkmark&$\times$&\checkmark&\checkmark&\checkmark&\checkmark&\checkmark&\checkmark&\checkmark&\checkmark\\
    \midrule                                                                                      
    ADD-S AUC              &  81.3    &   82.3   &  90.8  &   91.9   &  96.9    &\tb{97.3} &   80.8   &   81.9   & 90.0   &  91.3    &    96.2  &\tb{96.7} &   75.0   &   79.9   &   80.6   &   92.8   & \tb{94.2}\\
    ADD(-S) AUC            &  74.9    &   76.3   &  82.4  &   84.6   &  94.8    &\tb{95.6} &   74.0   &   75.5   & 81.0   &  83.4    &    93.0  &\tb{94.2} &   68.5   &   75.6   &   76.7   &   88.7   & \tb{91.6}\\
    ADD-S \textless   ~\SI{2}{cm} &  82.1    &   83.6   &  92.9  &   93.0   &  98.8    &\tb{98.9} &   82.0   &   83.4   & 92.3   &  92.6    &    98.7  &\tb{98.8} &   77.2   &   81.1   &   81.9   &   96.3   & \tb{96.6}\\
    ADD(-S) \textless ~\SI{2}{cm} &  73.0    &   74.8   &  80.6  &   82.4   &  96.5    &\tb{96.8} &   72.4   &   74.0   & 78.9   &  81.6    &\tb{96.0} &\tb{96.0} &   64.5   &   74.5   &   76.3   &   91.6   & \tb{93.6}\\
    \bottomrule
	\end{tabular}
	\caption{Quantitative results on the datasets MV-YCB FixCam (left), MV-YCB WiggleCam (middle), and MV-YCB SymMovCam (right). The baseline CosyPose (CP) uses PVN3D as backend network as described in \cite{mv6d}. The best results per dataset are printed in bold.}
	\label{tab_fixCam_wiggleCam}
\end{table*}



\subsection{Keypoint Visualization}

\cref{fig_ycbv_sv_keypoints} shows predicted keypoints of FFB6D and SyMFM6D in a YCB-Video scene. We additionally visualize the keypoint proposals of each object in individual colors.
The resulting predicted keypoints are white, the target keypoints are black. You can see that both FFB6D and SyMFM6D predict very accurate keypoints on all non-symmetric objects. However, FFB6D fails to predict accurate keypoints on the large clamp which has one discrete rotational symmetry. This shortcoming of FFB6D is also apparent on other symmetric objects. We believe that this is caused by the ambiguities of the object poses resulting in ambiguous target keypoints which results in averaging over the multiple solutions given by the symmetry. Therefore, the training loss is minimized when predicting keypoints on the symmetric axis rather than predicting them on the desired target locations. SyMFM6D in contrast overcomes this problem by our novel symmetry-aware training procedure as it can be seen in \cref{fig_ycbv_sv_keypoints_SyMFM6D}.

\begin{figure}[!tbh]
  \centering
\begin{subfigure}[b]{0.49\columnwidth}
  \includegraphics[width=1.0\columnwidth]{figures/0054_000204_FFB6D_1view_keypoints_cropped.jpg}
   \caption{FFB6D}
   \label{fig_ycbv_sv_keypoints_FFB6D}
\end{subfigure}
\begin{subfigure}[b]{0.49\columnwidth}
  \centering
  \includegraphics[width=1.0\columnwidth]{figures/0054_000204_SyMFM6D_16sym_1view_keypoints_BestSym_cropped.jpg}
   \caption{SyMFM6D}
   \label{fig_ycbv_sv_keypoints_SyMFM6D}
   \end{subfigure}
	\caption{Visualization of the predicted keypoints on single frames of the YCB-Video dataset.} 
   \label{fig_ycbv_sv_keypoints}
\end{figure}


\subsection{Implementation Details and Runtime}

We trained our network up to seven days on four NVIDIA Tesla V100 GPUs with \SI{32}{GB} of memory. 
The network architecture of our SyMFM6D approach has 3.5 million trainable parameters and requires about \SI{46}{ms} for processing a single RGB-D image on a single GPU. 
Mean shift clustering and least-squares fitting for computing a 6D pose require additional \SI{14}{ms} per object. 
Please visit our previously mentioned GitHub repository for code, datasets, and further details.

 We propose a novel commonsense reasoning challenge, \textsc{RiddleSense}, which requires complex commonsense skills for reasoning about creative and counterfactual questions, coming with a large multiple-choice QA dataset.  
 We systematically evaluate recent commonsense reasoning methods over the proposed \textsc{RiddleSense} dataset, and find that the best model is still far behind human performance, suggesting that there is still much space for commonsense reasoning methods to improve.
 We hope \textsc{RiddleSense} can serve as a benchmark dataset for future research targeting complex commonsense reasoning and computational creativity.


\section*{Acknowledgements}
This research is supported in part by the Office of the Director of National Intelligence (ODNI), Intelligence Advanced Research Projects Activity (IARPA), via Contract No. 2019-19051600007, the DARPA MCS program under Contract No. N660011924033 with the United States Office Of Naval Research, the Defense Advanced Research Projects Agency with award W911NF-19-20271, and NSF SMA 18-29268. The views and conclusions contained herein are those of the authors and should not be interpreted as necessarily representing the official policies, either expressed or implied, of ODNI, IARPA, or the U.S. Government. We would like to thank all the collaborators in USC INK research lab and the reviewers for their constructive feedback on the work.

{\small
\bibliographystyle{IEEEtran}
\bibliography{IEEEabrv}
}

\end{document}
