%%%%%%%%%%%%%%%%%%%%%%%%%%%%%%%%%%%%%%%%%%%%%%%%%%%%%%%%%%%%%%%%%%%%%%%%%%%%%%%%
%2345678901234567890123456789012345678901234567890123456789012345678901234567890
%        1         2         3         4         5         6         7         8

% Comment this line out if you need a4paper
%\documentclass[letterpaper, 10 pt, conference]{ieeeconf}  
\documentclass[letterpaper, 10 pt, journal, twoside]{IEEEtran} 

% Use this line for a4 paper
%\documentclass[a4paper, 10pt, conference]{ieeeconf}      

% This command is only needed if 
% you want to use the \thanks command
\IEEEoverridecommandlockouts                              

% Needed to meet printer requirements.
%\overrideIEEEmargins                                      

% See the \addtolength command later in the file to balance the column lengths
% on the last page of the document


%%XXX: Packages
\usepackage[normalem]{ulem}
\usepackage[table,usenames,dvipsnames]{xcolor}
\usepackage{extarrows}
\let\labelindent\relax
\usepackage[inline]{enumitem}
\usepackage{cite}

% Math
\usepackage{amsmath,amssymb,amsfonts,dsfont} % math
\usepackage{mathtools, bm}
\usepackage{algorithm,algorithmicx,listings}
\usepackage[noend]{algpseudocode}

% New command for R
\newcommand{\RR}{\mathbb{R}}

% Figures
\usepackage{graphicx}
\usepackage{tabularx}
%\usepackage{multirow,multicol,rotating,diagbox}
%\usepackage{booktabs}
%\usepackage{makecell}
\usepackage[font={small}]{caption}   %onehalfspacing
\usepackage{subcaption}
%\usepackage[font={small}]{subcaption}
%\setlength{\belowcaptionskip}{-3.5pt}
%\setlength{\abovecaptionskip}{3pt}
\captionsetup[algorithm]{font=small}
\usepackage[breaklinks=true, colorlinks, bookmarks=true, citecolor=Black, urlcolor=Violet,linkcolor=Black]{hyperref}

%%%%%%%%%%%%%%%%%%%%%%%%%%%%%%%%%%%%%%%%%%%%%%%%%%%%%%%%%%%%%%%%%%%%%%%%%
%%XXX: Commands
\def\liminf{\mathop{\lim\inf}\limits}	% EXAMPLE: \liminf_n A_n
\def\limsup{\mathop{\lim\sup}\limits}	%
\def\argmin{\mathop{\arg\min}\limits}	%
\def\argmax{\mathop{\arg\max}\limits}	%
% Write above and below equal sign
\newcommand{\longeq}[2]{\xlongequal[\!#2\!]{\!#1\!}}
% #1 = top; #2 = bottom; #3 = inequality (<,>,\leq,\geq)
\newcommand{\longineq}[3]{\overset{#1}{\underset{#2}{#3}}}
\newcommand{\indicator}{\mathds{1}}
\DeclareMathOperator{\tr}{tr}
\DeclareMathOperator{\per}{\mathbf{per}}
\def\deg{^{\circ}}
\def\negquad{\mkern-18mu}             % negative quad space
\def\negqquad{\mkern-36mu}            % negative qquad space
\newcommand{\txbx}[1]{\boxed{\text{#1}}}
\newcommand{\TODO}[1]{{\color{red}#1}}
\newcommand{\prl}[1]{\mathopen{}\left(#1\right)\mathclose{}}
\newcommand{\brl}[1]{\mathopen{}\left[#1\right]\mathclose{}}
\newcommand{\crl}[1]{\mathopen{}\left\{#1\right\}\mathclose{}}

\newtheorem{theorem}{Theorem}
\newtheorem{proposition}{Proposition}
\newtheorem{corollary}{Corollary}[theorem]
\newtheorem{definition}{Definition}
\newtheorem{assumption}{Assumption}
%\newtheorem*{assumption*}{Assumption}
\newtheorem{remark}{Remark}
%\newtheorem*{problem*}{Problem}
\newtheorem{problem}{Problem}
\newtheorem{lemma}{Lemma}


\title{
Robust Stereo Visual Inertial Odometry for Fast Autonomous Flight
}


\author{Ke Sun, Kartik Mohta, Bernd Pfrommer, Michael Watterson, Sikang Liu, \\
Yash Mulgaonkar, Camillo J. Taylor, and Vijay Kumar*
\thanks{Manuscript received: September, 10, 2017; Revised December, 1, 2017; Accepted December, 27, 2017. This paper was recommended for publication by Editor Cyrill Stachniss upon evaluation of the Associate Editor and Reviewers' comments. This work was supported in part by DARPA grants HR001151626 and HR0011516850, ARL grant W911NF-08-2-0004, and a NASA Space Technology Research Fellowship awarded to Michael Watterson.}%Use only for final RAL version
\thanks{*The authors are with GRASP Lab, University of Pennsylvania, Philadelphia, PA 19104, USA, {\tt\small\{sunke, kmohta, pfrommer, wami, sikang, yashm, cjtaylor, kumar\}@seas.upenn.edu}.}
\thanks{Digital Object Identifier (DOI): see top of this page.}
}


\begin{document}

\maketitle
%\thispagestyle{empty}
%\pagestyle{empty}
% Paper headers 
\markboth{IEEE Robotics and Automation Letters. Preprint Version. Accepted December, 2017}
{Sun \MakeLowercase{\textit{et al.}}: Robust Stereo Visual Inertial Odometry for Fast Autonomous Flight}  



\begin{abstract}
In recent years, vision-aided inertial odometry for state estimation has matured significantly. However, we still encounter challenges in terms of improving the computational efficiency and robustness of the underlying algorithms for applications in autonomous flight with micro aerial vehicles in which it is difficult to use high quality sensors and powerful processors because of constraints on size and weight. In this paper, we present a filter-based stereo visual inertial odometry that uses the Multi-State Constraint Kalman Filter (MSCKF)~\cite{mourikis2007multi}. Previous work on stereo visual inertial odometry has resulted in solutions that are computationally expensive. We demonstrate that our Stereo Multi-State Constraint Kalman Filter (S-MSCKF) is comparable to state-of-art monocular solutions in terms of computational cost, while providing significantly greater robustness. We evaluate our S-MSCKF algorithm and compare it with state-of-art methods including OKVIS, ROVIO, and VINS-MONO on both the EuRoC dataset, and our own experimental datasets demonstrating fast autonomous flight with maximum speed of $17.5$m/s in indoor and outdoor environments. Our implementation of the S-MSCKF is available at \url{https://github.com/KumarRobotics/msckf_vio}.
\end{abstract}
\begin{IEEEkeywords}
Localization; Aerial Systems: Perception and Autonomy; SLAM
\end{IEEEkeywords}


\section{Introduction}
\label{sec:Introduction}


The goal in top-$\size$ recommendation is to recommend to each
consumer a small set of $\size$ items from a large collection of
items~\cite{cremonesi2010performance}.  For example, Netflix may want
to recommend $\size$ appealing movies to each consumer.  Collaborative
Filtering (CF)~\cite{herlocker2002empirical,lee2012comparative} is a
common top-$\size$ recommendation method.  CF infers user interests by
analyzing partially observed user-item interaction data, such as user
ratings on movies or historical purchase
logs~\cite{kanagal2012supercharging}. The main assumption in CF is that
users with similar interaction patterns have similar interests.


Standard CF methods for top-$\size$ recommendation focus on making  suggestions  that accurately reflect the user's preference history. However, as  observed in previous work,  CF recommendations are generally biased toward  popular items, leading to a rich get richer effect~\cite{vargas2014improving,steck2011item}.  The major reasons for this are \textit{popularity bias} and \textit{sparsity} of CF interaction data (detailed in Section~\ref{sec:related-work}). In a nutshell, to maintain  accuracy, recommendations are generated from the dense regions of the data,  where the popular items lie.  

However,  accurately suggesting popular items, may not be satisfactory for the consumers. For example, in Netflix, an accuracy-focused movie recommender may recommend ``Star Wars: The Force Awakens'' to users who have seen ``Star Wars: Rogue One''.  But, those users are probably already aware of ``The Force Awakens''. Considering additional factors, such as novelty of recommendations,  can lead to more effective suggestions~\cite{cremonesi2010performance,Castells2015,zhang2008avoiding,ziegler2005improving,zhang2012auralist}. 
%Second, accuracy-focused models typically achieve a   overall item-space coverage across their recommendations,  whereas high item-space coverage helps providers of the items increase revenue
%, users satisfaction since they are  likely already aware of or can find these items on their own.  

Focusing on popular items also adversely affects the satisfaction of  the providers of the items. This is because  accuracy-focused models typically achieve a  low overall item space coverage across their recommendations, whereas   high item space coverage helps providers of the items increase their revenue~\cite{vargas2014improving,Castells2015,adomavicius2011maximizing,anderson2006thelongtail, yin2012challenging,adomavicius2012improving}.
%accuracy-focused models typically achieve a

In contrast to the relatively small number of popular items, there are copious  {\it long-tail\/} items that have fewer observations (e.g., ratings) available. More precisely,  using the Pareto  principle (i.e.,~the $80/20$ rule),  long-tail items can be defined as items that generate the lower $20\%$ of observations~\cite{yin2012challenging}. Experimentally we found that these items correspond to almost $85\%$ of the items in several datasets (Sections~\ref{sec:Notation} and \ref{sec:Experiments}). %Table~\ref{tab:DatasetStatsticsSmall})


As previously shown, one way to improve the novelty of top-$\size$ sets is to recommend interesting long-tail items~\cite{cremonesi2010performance,ge2010beyond}.  The intuition  is that since they have fewer observations available,  they are more likely to be unseen~\cite{Kaminskas:2016:DSN:3028254.2926720}.  
 %For example, in online commerce,  newly added items are long-tail items that are yet to be discovered.  
Moreover, long-tail item promotion also results in higher overall coverage of the item space%, which increases profits for providers of the items
~\cite{vargas2014improving,Castells2015,zhang2008avoiding,zhang2012auralist,adomavicius2011maximizing,anderson2006thelongtail,yin2012challenging,jambor2010optimizing}. Because long-tail promotion reduces accuracy~\cite{steck2011item}, there are trade-offs to be explored.


%original submitted to ICDE
%This work studies three aspects of top-$\size$ recommendation: accuracy, novelty, and item-space coverage, and examines their trade-offs. In most previous work, predictions of a base recommendation system are re-ranked to handle their trade-offs~\cite{adomavicius2012improving,jambor2010optimizing,zhang2013personalize,wang2009portfolio}. Due to performance considerations, however, these techniques are not customized per user. For example,  parameters that balance the trade-off between novelty and accuracy are cross-validated at a global level.  This can be detrimental since users have varying preferences for  objectives such as long-tail novelty. We explore how to  automatically infer  user  preference for long-tail novelty, and how to leverage  it to correct  the popularity bias in standard recommender models. Our work does not rely on any additional contextual data, although such data, if available, can help promote newly-added long-tail items~\cite{agarwal2009regression,Saveski:2014:ICR:2645710.2645751}.

This work studies three aspects of top-$\size$ recommendation: accuracy, novelty, and item space coverage, and examines their trade-offs. In most previous work, predictions of a base recommendation algorithm are \textit{re-ranked} to handle these trade-offs~\cite{adomavicius2012improving,jambor2010optimizing,zhang2013personalize,wang2009portfolio}. The re-ranking models are computationally efficient but suffer from two drawbacks. First, due to performance considerations,  parameters that balance the trade-off between novelty and accuracy  are not customized per user. Instead they are cross-validated at a global level.  This can be detrimental since users have varying preferences for  objectives such as long-tail novelty. Second,  the re-ranking methods are often limited to a specific base recommender  that may be sensitive to dataset density. 
As a result, the datasets are pruned and the problem is studied in dense settings~\cite{adomavicius2012improving,ho2014likes}; but real world  scenarios are often sparse~\cite{kanagal2012supercharging,liu2017experimental}.   
% Because  dataset density can impact the performance of most base recommenders (like R-SVD), which in turn affects the performance of the re-ranking model, 

\iffalse
We address these limitations by directly inferring  user  preference for long-tail novelty  from interaction data.  This  allows us to customize the re-ranking  per user, and design a \textit{generic} framework, which resolves the second problem. In particular, since the long-tail novelty preferences are estimated independently of any base  recommender model, we can  plug-in an appropriate base recommender w.r.t. the dataset sparsity.% including ones that are more suitable for sparse settings.  

Modelling  user  preference for  long-tail novelty using only item popularity statistics, e.g., the average popularity of rated items as in~\cite{jugovac2017efficient}, disregards additional information like whether the user found the item interesting and the long-tail preferences of other users  of the items. \iffalse To incorporate them, we introduce the notion of  \emph{item long-tail importance}. Both  user long-tail preferences and item long-tail importance are dependent:  a user has high preference for discovering long-tail items if she is interested in important long-tail items, and an item that is associated with many of these kinds of users is likely to be more important.  We propose a joint optimization framework to directly learn,  from interaction data, both the users' long-tail preferences and the  items' long-tail importance. \fi
We propose an optimization approach that  incorporates  this information and  directly learns,  from interaction data, the users' long-tail novelty preferences.

Next, we use these learned preferences  to design a  top-$\size$ recommendation framework thats is generic, and provides customized balance between accuracy, novelty, and coverage. We refer to it as framework as GANC.  Using GANC, we design a novel algorithm, {\it Ordered Sampling-based Locally Greedy (OSLG)\/}, that relies on the learned long-tail novelty preferences  to scalably correct for popularity bias. Our work does not rely on any additional contextual data, although such data, if available, can help promote newly-added long-tail items~\cite{agarwal2009regression,Saveski:2014:ICR:2645710.2645751}. In summary:
\fi

We address the first limitation by directly inferring  user  preference for long-tail novelty  from interaction data.   Estimating these  preferences  using only item popularity statistics, e.g., the average popularity of rated items as in~\cite{jugovac2017efficient}, disregards additional information, like whether the user found the item interesting or the long-tail preferences of other users  of the items. We propose an approach that  incorporates  this information and  learns the users' long-tail novelty preferences from interaction data.

This approach allows us to customize the re-ranking  per user, and  design a \textit{generic} re-ranking framework, which resolves the second limitation of prior work. In particular, since the long-tail novelty preferences are estimated independently of any base recommender, we can  plug-in an appropriate one w.r.t. different factors, such as the dataset sparsity.

Our top-$\size$ recommendation framework, \textbf{GANC}, is \textbf{G}eneric, and provides customized balance between \textbf{A}ccuracy, \textbf{N}ovelty, and \textbf{C}overage. % Moreover, based on the learned long-tail novelty preferences, we also design a novel algorithm, {\it Ordered Sampling-based Locally Greedy (OSLG)\/}, that relies on the learned long-tail novelty preferences  to scalably correct for popularity bias. 
Our work does not rely on any additional contextual data, although such data, if available, can help promote newly-added long-tail items~\cite{agarwal2009regression,Saveski:2014:ICR:2645710.2645751}. In summary:

%Consider  the following toy example:
\vspace{-0.2cm}
\begin{table}[htb]
\centering
\scriptsize
%\small
\begin{tabular}{ccccccc} 
%\toprule
%&\multirow{2}{*}{}&\multicolumn{7}{c}{Ratings}\\
& & \cellcolor{blue!35}$w_1$ &\cellcolor{blue!18} $w_2$ & $\dots$ &\cellcolor{blue!8} $w_{89}$  &\cellcolor{blue!8} $w_{99}$   
\\
&   &$i_1$&$i_2$&$\dots$&$i_{89}$&$i_{90}$\\ 
\cmidrule(r){3-7} 	 
%\midrule
\cellcolor{red!35}$\theta_1$  &$u_1 $   &5 &   & $\dots$ &  &   \\
\cellcolor{red!28}$\theta_2$  &$u_2$     &5 &    & $\dots$ &  &  \\
 $\theta_3=?$  &$\bf u_3$  &5 &  &   $\dots$ &  &  \\
\cellcolor{red!10}$\theta_4$ & $u_4$  &  &5   & $\dots$ & &\\ 
\cellcolor{red!10}$\theta_5$ & $u_5$  &  & 5  & $\dots$ & &\\ 
$\theta_6=?$  & $\bf u_6$ & &5  &      $\dots$& &  \\ 
 & & $\hdots$  &$\hdots$   &$\hdots$   &$\hdots$   &$\hdots$  \\
%\midrule 
\cmidrule(r){3-7} 	 
\multicolumn{2}{c}{item pop.}  & 3  & 3  & $\dots$ &50&60\\  
%\bottomrule
%$ f_i$    &3  &3  &1  &3  &1  &2  \\  \hline
\end{tabular}
%#.
\caption{Simplified user-item interaction data. The user long-tail novelty preference ($\theta_u$), item long-tail importance weight ($w_i$) are highlighted. Darker colors indicate larger values. } \label{tab:example}
\end{table} 
\vspace{-0.2cm}
\begin{example}  
In Table~\ref{tab:example}, we are interested in estimating $\theta_3$ and $\theta_6$,  the long-tail preference of users $u_3$ and $u_6$ who have each rated a single movie. Additional ratings for other users  are not included here.  Considering only rating information, we observe $i_1$ and $i_2$ are  equally popular $|\mathcal{U}_{i_1}^{\trainset}| = |\mathcal{U}_{i_2}^{\trainset}|=3$, and $r_{31}=5$ and $r_{62}=5$. Using Eq.~\ref{eq:tfidf-risk}  we have $\theta_3 = \theta_6$. However, if we were given the long-tail preferences of the each item's user set, specifically that $u_1$ and $u_2$ have high long-tail preference (darker red), while $u_4$ and $u_5$ have lower long-tail preference (lighter red), we could conclude $i_1$ is a more important long-tail item compared to $i_2$ (indicated by a darker blue shade for $w_1$), and we expect  $\theta_3 \geq \theta_6$.

% On the other hand, if we knew that $u_4$ and $u_5$ have lower long-tail preference, we could conclude $i_2$ is a  less significant long-tail item. Therefore, However, if we  consider the long-tail preferences of other users, we may reason differently.    We need another variable $w_i$ which captures this information. 
%we would conclude that $u_3$ has higher long-tail preference compared to $u_6$, since the users $i_1$ is a more prominent long-tail item. 

% Relying only  on item popularity information, we would  conclude   $u_3$ and $u_6$ have equal long-tail preference, since $i_1$ and $i_2$ are  equally popular. However, considering  the second column,  long-tail preference of users,  long-tail importance for each item,  which captures the long-tail preference of its users. Since  that  both users of $i_1$ have high long-tail preference while  the users of $i_2$ have lower preference,  we may conclude $i_1$ is a more important long-tail item compared to $i_2$. Therefore, $u_3$'s long-tail preference should be at least as large as $u_6$'s preference. Specifically, consider two  items $i_1$ and $i_2$, with the following rating data: $i_1=\{u_1:5, u_2:5, u_3:5 \}$, $i_2=\{u_4:5, u_5:5, u_6:5\}$.  

%Table~\ref{tab:example} shows  simplified rating data. We want an estimate of the long-tail preference of $u_3$ and $u_6$, who have each  rated a single movie.  Relying only  on movie popularity information, we would  conclude   $u_3$ and $u_6$ have similar long-tail preference, since $m_1$ and $m_2$ are  equally popular. However, considering the long-tail preferences of other users of those movies, we may reason differently: since $u_1$ and $u_2$ have high long-tail preference, and $u_4$ and $u_5$ have low long-tail preference, $m_1$ is a more prominent long-tail item compared to $m_2$. Therefore, it is likely that $u_3$ has higher long-tail preference compared to $u_6$.considering the long-tail preferences of other users of those movies, we may reason differently.  For example, 
\label{ex:running}
\end{example}



%------------------------------

\iffalse
\begin{example}
Table~\ref{tab:example} shows rating data for a simplified system. %Note the user-item interaction matrix is sparse.
For this example, we define popular movies as those that have received  three or more ratings; $\{m_1, m_2, m_4\}$ are popular and  $\{m_3, m_5, m_6\}$ are niche movies. We observe $u_1$ and $u_3$  have rated relatively popular movies (risk-averse) while $u_2$ and $u_4$ have rated niche movies (risk-loving). 
\label{ex:running}
\end{example}

\begin{table}[htb]
\centering
\scriptsize
\begin{tabular}{ccccccc} 
\toprule
			&$m_1$ &$m_2$   &$m_3$    &$m_4$   &$m_5$ &$m_6$  \\ \hline 
$u_1 $ &5  &4  & - &-  &-  &-   \\
$u_2$  &-  &-  &-  &-  &5  &5   \\
$u_3$  &-  &4  &-  &5  &-  &-   \\
$u_4$  &-  &-  &3  &-  &-  &4   \\ 
$u_5$  &5  &-  &-  &3  &-  &-   \\ 
$u_6$  &4  &2  &-  &4  &-  &-   \\ 
\bottomrule
%$ f_i$    &3  &3  &1  &3  &1  &2  \\  \hline
\end{tabular}
\caption{User-Movie rating data} \label{tab:example}
\end{table}

It is essential to consider consumer characteristics in designing recommender systems so that they promote long-tail items to the right group of users and spread demand evenly between hit and niche items.  

\fi





%------------------------------
\iffalse
\begin{table}[htb]
\centering
\scriptsize
\begin{tabular}{ccccccc} 
\toprule
			&$m_1$ &$m_2$   &$m_3$    &$m_4$   &$m_5$ &$m_6$  \\ \hline 
$u_1 $ &\textbf{5}  & \textbf{4}  &\textcolor{gray}{ 1.2} &-  &-  &-   \\
$u_2$  &-  &-  &-  &-  & \textbf{5}  &\textbf{5}   \\
$u_3$  &-  &\textbf{4}  &-  &\textbf{5}  &-  &-   \\
$u_4$  &-  &-  &\textbf{3}  &-  &-  &\textbf{4}   \\ 
$u_5$  &\textbf{5}  &-  &-  &\textbf{3}  &-  &-   \\ 
$u_6$  &\textbf{4}  &\textbf{2}  &-  &\textbf{4}  &-  &-   \\ 
\bottomrule
%$ f_i$    &3  &3  &1  &3  &1  &2  \\  \hline
\end{tabular}
\caption{User-Movie rating data} \label{tab:example}
\end{table}
% $\mathcal{P}^1= \{ \mathcal{P}_1^1 \{i_1,i_2,i_3\}, \mathcal{P}_2^1:\{i_2,i_3,i_5\}  \}$
 %$\mathcal{P}^2= \{ \mathcal{P}_1^2: \{i_1,i_2,i_3\}, \mathcal{P}_2^2:\{i_2,i_5,i_6\}  \}$
 %$\mathcal{P}^3= \{ \mathcal{P}_1^3: \{i_7,i_8,i_9\}, \mathcal{P}_2^3:\{i_{10},i_{11},i_{12}\}  \}$
\begin{table}[htb]
\centering
\tiny
\begin{tabular}{ccc} 
\toprule
		&$u_1$&$u_2$  \\ \hline 
$\mathcal{P}^1 $ & $\{i_1,i_2,i_3\}$ & $\{i_2,i_3,i_5\} $ \\
$\mathcal{P}^2$ & $\{i_1,i_2,i_3\}$ & $\{i_2,i_5,i_6\} $ \\
$\mathcal{P}^3$ & $\{i_7,i_8,i_9\}$ & $\{i_{10},i_{11},i_{12} \}$ \\
\bottomrule
%$ f_i$    &3  &3  &1  &3  &1  &2  \\  \hline
\end{tabular}
\caption{Top-$\size$ allocations to users.} \label{tab:paretoExamples}
\end{table}
\fi


\iffalse
When considering long-tail items, it is important to consider consumers' willingness  to explore niche or unpopular items and their propensity towards similar items. In particular, they can be characterized by their  {\it risk degree\/} and {\it focusing degree\/}, respectively.  We compute these estimates  based on historical rating information. The following example further describes these notions in the context of movie rating data. 

\begin{example}  
Table~\ref{tab:example} shows rating data for a simplified system with $6$ users, $6$ movies, and $3$ genres. $m_i^{j}$ implies that movie $m_i$ belongs to genre $j$. Note the user-item interaction matrix is sparse. 
  For this setting, we define popular movies as those that have received  three or more ratings; $\{m_1, m_2, m_4\}$ are popular and  $\{m_3, m_5, m_6\}$ are niche movies. We now profile the users according to their risk and focusing degree. E.g., $u_1$ has rated relatively popular movies belonging to the same genre (risk-averse, high focusing degree); $u_2$ has rated niches movies in the same genre (risk-loving, high focusing degree); $u_3$ has rated popular movies in two different genres (risk-averse, low focusing degree), and $u_4$ has rated niches movies in two different genres (risk-loving, low focusing degree). 
\label{ex:running}
\end{example}
\begin{table}[htb]
\centering
\tiny
\begin{tabular}{ccccccc} 
\toprule
			&$m_1^{1}$ &$m_2^{1}$   &$m_3^{2}$    &$m_4^{3}$   &$m_5^{3}$ &$m_6^{3}$  \\ \hline 
$u_1 $ &5  &4  &-  &-  &-  &-   \\
$u_2$  &-  &-  &-  &-  &5  &5   \\
$u_3$  &-  &4  &-  &5  &-  &-   \\
$u_4$  &-  &-  &3  &-  &-  &4   \\ 
$u_5$  &5  &-  &-  &3  &-  &-   \\ 
$u_6$  &4  &2  &-  &4  &-  &-   \\ 
\bottomrule
%$ f_i$    &3  &3  &1  &3  &1  &2  \\  \hline
\end{tabular}
\caption{User-Movie rating data} \label{tab:example}
\end{table}
It is essential to consider these consumer characteristics in designing recommender systems so that they promote long-tail items to the right group of users and spread demand evenly between the hit and niche items.  
\fi
\iffalse
\begin{center}
\begin{figure*}[tp]
%\scalebox{0.5}{%
\resizebox{1\textwidth}{!}{%
%\small%\addtolength{\tabcolsep}{5pt}% below sums to 8
\begin{tabularx}{1.5\textwidth}{>{\hsize=2.5\hsize}X>{\hsize=2.5\hsize}X>{\hsize=0.5\hsize}X>{\hsize=0.5\hsize}X>{\hsize=0.5\hsize}X>{\hsize=0.5\hsize}X>{\hsize=0.5\hsize}X>{\hsize=0.5\hsize}X}
    \multirow{12}{*}{\includegraphics[scale=0.3]{codeForExample/popularity-movie.png}} & \multirow{12}{*}{\includegraphics[scale=0.3]{codeForExample/scatterplot.png}} & & & & & & \\
%   & &               &       &       &       &       &       \\
    & &\multicolumn{1}{l|}{}               &$m_1^{g1}$   	&$m_2^{g1}$    	&$m_3^{g2}$    &$m_4^{g2}$      &$m_5^{g3}$    \\ \cline{3-8}%\hline
    & &\multicolumn{1}{l|}{u1}          &5  &5  &-  &-   &-  \\
    & &\multicolumn{1}{l|}{u2}    		&-  &-  &4  &4  &5  \\
    & &\multicolumn{1}{l|}{u3}   			&1  &2  &1  &-  &-   \\
    & &\multicolumn{1}{l|}{u4}     		&1  &-  &-  &-  &-  \\
    & &               &       &       &       &       &       \\
    & &               &       &       &       &       &       \\
    & &               &       &       &       &       &       \\
    & &               &       &       &       &       &	\\
    \\
\end{tabularx}}
\caption{User-Movie interaction data a) Popularity-Movie histogram b)Movie genres/clusters c) User-Movie rating data} \label{fig:example}
\end{figure*}
\end{center}
\fi



%We propose a novel approach that allows us to  promote long-tail items in a targeted manner, thereby improving the novelty of top-$\size$ sets, the overall item-space coverage across recommendations, while maintaining reasonable levels of accuracy.

%Next, we integrate these learned preferences  in a generic  top-$\size$ recommendation framework to provide customized balance between accuracy and coverage.

%sequentially make recommendations, while adjusting its parameters with regard to the set of top-$\size$ recommendations made so far. However, since  sequential parameter updates  cause  scalability issues, we propose a sampling based algorithm. This variant of our framework, called {\it Ordered Sampling-based Locally Greedy (OSLG)\/},  allows us to  correct for the popularity bias in recommendations with regard to individual user long-tail preferences. 

%ICDE submission
%Our framework differs with  prior work in the following aspects:  unlike~\cite{adomavicius2011maximizing,adomavicius2012improving,zhang2013personalize,ho2014likes},  the long-tail preference personalization in our framework is learned rather than optimized using cross-validation or parameter tuning. In other words, our personalization method is independent of the underlying base  recommendation models.  Moreover, our framework is  generic. This enables us to  plug-in several base recommenders, and evaluate their  effectiveness without requiring  extensive tuning for the accuracy and coverage trade-off. 


%\vspace{-2.8pt}
\begin{itemize}

\item  We examine various measures for estimating user long-tail novelty preference in Section~\ref{sec:lt-pref} and formulate an optimization problem  to directly learn users' preferences for long-tail  items from interaction data in Section~\ref{sec:learning-lt-pref}. %In addition, we introduce several heuristics for measuring the user preference for less common items from historical rating data.% 

\item  We integrate the user preference estimates into GANC %, a generic re-ranking framework that provides customized balance between accuracy, novelty, and coverage 
(Section~\ref{sec:RiskbasedReranking}), and  introduce {\it Ordered Sampling-based Locally Greedy (OSLG)\/}, a scalable algorithm that relies  on user long-tail preferences to correct the popularity bias (Section~\ref{sec:optimizationAlgorithm}).
%We introduce OSLG, a scalable algorithm that relies  on user long-tail preferences to  maximize item space coverage \textcolor{red}{while maintaining acceptable levels of accuracy} (Section~\ref{sec:optimizationAlgorithm}).

\item   We conduct an extensive empirical study and evaluate performance from  accuracy, novelty, and coverage perspectives (Section~\ref{sec:Experiments}).  We use five  datasets with varying density and difficulty levels. %:  Netflix, MovieTweetings, and MovieLens (100K, 1M, 10M). 
  In contrast to most related work,  our evaluation considers realistic settings that include a large number of infrequent  items and users. %This enables us to study the impact of  data density on the performance trade-offs of several  state of the art top-$\size$ recommendation algorithms. %   %,  and use the all-items ranking protocol~\cite{steck2013evaluation,vargas2014improving}, where performance is measured using all items with train data. to evaluate the performance of several  state of the art top-$\size$ recommendation algorithms 
 
\item Our empirical results confirm that the performance of re-ranking models is impacted by the underlying   base recommender and the dataset density. Our generic approach enables us to easily incorporate a suitable base recommender to devise an effective solution for both dense and sparse settings. In dense settings, we use the same base recommender as existing re-ranking approaches, and we outperform them in accuracy and coverage metrics. For sparse settings, we plug-in a more suitable base recommender, and devise an effective solution that is competitive with existing top-$\size$ recommendation methods in accuracy and novelty. 

%Directly estimating the long-tail novelty preferences allows us to customize re-ranking per user, and  devise a generic framework.   
 
\end{itemize}

Section~\ref{sec:related-work} describes related work. Section~\ref{sec:conclusion} concludes.

\section{Related Work}
\label{sec: Related Work}
In this section, we introduce two representative diversity estimators and discuss the difficulties they meet when handling MMOPs. Subsequently, some existing multi-modal multi-objective optimization algorithms are reviewed.
\subsection{Review of diversity estimators}
\subsubsection{Density in SPEA2}
In SPEA2 \cite{SPEA2}, each solution is assigned a density value which is used to calculate its fitness value. Eq. (\ref{eq: Density in SPEA2}) gives the density of a solution $\boldsymbol{x}$.
\begin{equation}
	\textit{Density} (\boldsymbol{x}) = \frac{1}{\sigma_k(\boldsymbol{x}) + 2},
	\label{eq: Density in SPEA2}
\end{equation}
where $\sigma_k(\boldsymbol{x})$ is the distance from $\boldsymbol{x}$ to its $k$-th nearest neighbor in the objective space. In SPEA2, $k$ is set to the square root of the total number of solutions in the current population as a general parameter setting.

Notice that in SPEA2, higher density means worse diversity in the objective space.

\subsubsection{Crowding distance}
Crowding distance is proposed along with the NSGA-II algorithm\cite{NSGAII} to preserve the diversity of the population in the objective space. The crowding distance of a solution $\boldsymbol{x}$ is given by the average side length of the hypercube constructed by its left and right neighbors in each objective. More precisely, for each objective, the left and right neighbors of $\boldsymbol{x}$ are the solutions at the left and right positions of $\boldsymbol{x}$ for that objective (i.e., in the list obtained by sorting the population in an increasing order of the objective values of that objective). The crowding distance of all boundary solutions (i.e., best solutions in any objectives) are set to $\infty$ to ensure that they are always selected. In NSGA-II, larger crowding distance values indicate better diversity. Formally, Eq. (\ref{eq: crowding distance}) calculates the crowding distance for a solution $\boldsymbol{x}$.
\begin{equation}
	\textit{Crowding-Distance} (\boldsymbol{x}) =
	\begin{cases}
		\infty                                                                     & ,\boldsymbol{x} \text{ is a boundary solution} \\
		\frac{1}{M}\sum_{m=1}^M[f_m(\boldsymbol{x}_{rm})-f_m(\boldsymbol{x}_{lm})] & ,\text{otherwise}
	\end{cases},
	\label{eq: crowding distance}
\end{equation}
where $M$ refers to the number of objectives, and $\boldsymbol{x}_{lm}$ and $\boldsymbol{x}_{rm}$ are the left and right neighbors of solution $\boldsymbol{x}$ regarding the $m$-th objective, respectively.
\subsection{Difficulties when handling MMOPs}
In most diversity estimators in MOEAs, the solution distribution in the decision space is out of consideration, which makes them inefficient on MMOPs. As we have discussed in Section \ref{sec: Introduction}, in MMOPs, equivalent solutions have the same or almost the same objective values. Consequently, they are usually not preferable in terms of diversity (in the objective space). For this reason, diversity estimators used in MOEAs are often responsible for the loss of equivalent solutions when tackling MMOPs. Fig. \ref{fig: Difficulty when handling MMOPs} gives an example when a diversity estimator such as crowding distance produces undesirable effects. In Fig. \ref{fig: Difficulty when handling MMOPs}, $A$ and $B$ are two Pareto optimal solutions on different (but equivalent) Pareto subsets (i.e., the upper and lower dash lines in (a)). Although $A$ and $B$ have similar objective values, the decision maker may want to keep both of them since they represent different implementations (i.e., they are different in the decision space). However, a diversity estimator tends to assign bad diversity values to them due to the small difference between their objective values. As a result, some of them are likely to be removed. From this example, we can see that solutions in different regions in the decision space should be considered separately when estimating solution diversity for MMOPs. Following this idea, we propose a niching diversity estimation method in Section \ref{sec: Proposed method}.

\begin{figure}
	\centering
	\includegraphics[width=.75\textwidth]{figures/RelatedWork/Difficulties}
	\caption{Explanation of the diversity loss in the decision space caused by diversity estimators when handling an MMOP. The dash lines in (a) and (b) denote the Pareto set and Pareto front, respectively.}
	\label{fig: Difficulty when handling MMOPs}
\end{figure}

\subsection{Multi-modal multi-objective optimization algorithms}
\label{sec: Existing multi-modal multi-objective optimization algorithms}
In most state-of-the-art multi-modal multi-objective evolutionary algorithms (MMEAs), the diversity in the decision space is maintained by niching strategies. Some MMEAs extend existing niching strategies in MOEAs to enable them to maintain the diversity in the objective space as well as in the decision space. For example, in \cite{OmniOptimizer}, Deb and Tiwari proposed one of the first MMEA called Omni-optimizer which modifies the crowding distance to measure the diversity in the decision space and the objective space simultaneously. Yue et. al. proposed a particle swarm optimizer named MO\_Ring\_PSO\_SCD \cite{MO_Ring_PSO_SCD} which adopts a similar modified crowding distance and a ring topology to create a niche structure. The DNEA algorithm \cite{DNEA} applies the fitness sharing \cite{Sharing} to both decision and objective spaces and combines them into a single sharing function. Some MMEAs are proposed with dedicated niching strategies in the decision space. Tanabe et. al. proposed a decomposition-based MMEA called MOEA/D-AD\cite{MOEAD_AD} where multiple solutions can be assigned to a weight vector, and a newly generated solution only competes with other solutions which are assigned to the same weight vector and neighboring to that solution in the decision space. In our previous study \cite{MOEAD_MM}, we proposed another decomposition-based MMEA which utilizes a clearing strategy in the decision space. Some MMEAs such as the algorithms proposed in \cite{DBSCAN_MMEA} and \cite{MMOEADC} use clustering approaches to maintain the niching structure in the decision space.
\section{Filter Description}
\label{sec: filter description}
In the description of the filter setup, we follow the convention in~\cite{mourikis2007multi}. The IMU state is defined as,
\begin{equation*}
\mathbf{x}_{I} = 
\left(
{}^I_G \mathbf{q}^\top \quad 
\mathbf{b}_g^\top \quad 
{}^G\mathbf{v}^\top_I \quad 
\mathbf{b}_a^\top \quad
{}^G\mathbf{p}^\top_I \quad
{}^I_C \mathbf{q}^\top \quad
{}^I\mathbf{p}^\top_C
\right)^\top
\end{equation*}
where the quaternion ${}^I_G \mathbf{q}$ represents the rotation from the inertial frame to the body frame. In our configuration, the body frame is set to be the IMU frame. The vectors ${}^G\mathbf{v}_I \in \RR^3$ and ${}^G\mathbf{p}_I \in \RR^3$ represent the velocity and position of the body frame in the inertial frame. The vectors $\mathbf{b}_g \in \RR^3$ and $\mathbf{b}_a \in \RR^3$ are the biases of the measured angular velocity and linear acceleration from the IMU. Finally the quaternion, ${}^I_C \mathbf{q}$ and ${}^I\mathbf{p}_C \in \RR^3$ represent the relative transformation between the camera frame and the body frame. Without loss of generality, the left camera frame is used assuming the extrinsic parameters relating the left and right cameras are known. Using the true IMU state would cause singularities in the resulting covariance matrices because of the additional unit constraint on the quaternions in the state vector. Instead, the error IMU state, defined as,
\begin{equation*}
\tilde{\mathbf{x}}_{I} = 
\left(
{}^I_G \tilde{\bm{\theta}}^\top \quad 
\tilde{\mathbf{b}}_g^\top \quad 
{}^G\tilde{\mathbf{v}}^\top_I \quad 
\tilde{\mathbf{b}}_a^\top \quad
{}^G\tilde{\mathbf{p}}^\top_I \quad
{}^I_C \tilde{\bm{\theta}}^\top \quad
{}^I\tilde{\mathbf{p}}^\top_C
\right)^\top
\end{equation*}
is used with standard additive error used for position, velocity, and biases (e.g. ${}^G\tilde{\mathbf{p}}_I = {}^G\mathbf{p}_I-{}^G\hat{\mathbf{p}}_I$). For the quaternions, the error quaternion $\delta\mathbf{q} = \mathbf{q}\otimes\hat{\mathbf{q}}^{-1}$ is related to the error state as,
\begin{equation*}
\delta\mathbf{q} \approx
\left(
\frac{1}{2} {}^G_I\tilde{\bm{\theta}}^\top \quad 1
\right)^\top
\end{equation*}
where ${}^G_I\tilde{\bm{\theta}} \in \RR^3$ represents a
small angle rotation. With such a representation, the dimension of orientation error is reduced to $3$ enabling proper presentation of its uncertainty. Ultimately $N$ camera states are considered together in the state vector, so the entire error state vector would be,
\begin{equation*}
\tilde{\mathbf{x}} = 
\left(
\tilde{\mathbf{x}}_I^\top \quad
\tilde{\mathbf{x}}_{C_1}^\top \quad
\cdots \quad 
\tilde{\mathbf{x}}_{C_N}^\top
\right)^\top
\end{equation*}
where each camera error state is defined as,
\begin{equation*}
\tilde{\mathbf{x}}_{C_i} = 
\left(
{}^{C_i}_G\tilde{\bm{\theta}}^\top \quad
{}^G\tilde{\mathbf{p}}_{C_i}^\top
\right)^\top
\end{equation*}
In order to maintain bounded computational complexity, some camera states have to be marginalized once the number of camera states reaches a preset limit. Discussions of how to choose camera states to marginalize can be found in Section \ref{subsec: filter update mechanism}. 
\subsection{Process Model}
\label{subsec: process model}
The continuous dynamics of the estimated IMU state is,
\begin{equation}
\label{eq: estimated state dynamics}
\begin{gathered}
{}^I_G\dot{\hat{\mathbf{q}}} = \frac{1}{2}\Omega(\hat{\bm{\omega}}) {}^I_G\hat{\mathbf{q}}, \quad
\dot{\hat{\mathbf{b}}}_g = \mathbf{0}_{3\times 1}, \\
{}^G\dot{\hat{\mathbf{v}}} = C\left({}^I_G\hat{\mathbf{q}}\right)^\top \hat{\mathbf{a}} + {}^G\mathbf{g}, \\
\dot{\hat{b}}_a = \mathbf{0}_{3\times 1}, \quad
{}^G\dot{\hat{\mathbf{p}}}_I = {}^G\hat{\mathbf{v}}, \\
{}^I_C\dot{\hat{\mathbf{q}}} = \mathbf{0}_{3\times 1}, \quad
{}^I\dot{\hat{\mathbf{p}}}_C = \mathbf{0}_{3\times 1}
\end{gathered}
\end{equation}
where $\hat{\bm{\omega}} \in \RR^3$ and $\hat{\mathbf{a}} \in \RR^3$ are the IMU measurements for angular velocity and acceleration respectively with biases removed, i.e,
\begin{equation*}
\hat{\bm{\omega}} = \bm{\omega}_m - \hat{\mathbf{b}}_g, \quad
\hat{\mathbf{a}} = \mathbf{a}_m - \hat{\mathbf{b}}_a
\end{equation*}
Meanwhile,
\begin{equation*}
\Omega\left(\hat{\bm{\omega}}\right) = 
\begin{pmatrix}
-[\hat{\bm{\omega}}_\times] & \bm{\omega} \\
-\bm{\omega}^\top & 0
\end{pmatrix}
\end{equation*}
where $[\hat{\bm{\omega}}_\times]$ is the skew symmetric matrix of $\hat{\bm{\omega}}$. $C(\cdot)$ in Eq.~\eqref{eq: estimated state dynamics} is the function converting quaternion to the corresponding rotation matrix. Based on Eq.~\eqref{eq: estimated state dynamics}, the linearized continuous dynamics for the error IMU state follows,
\begin{equation}
\label{eq: error state dynamics}
\dot{\tilde{\mathbf{x}}}_I = 
\mathbf{F} \tilde{\mathbf{x}}_I + 
\mathbf{G} \mathbf{n}_I
\end{equation}
where $\mathbf{n}_I^\top = \left(\mathbf{n}_g^\top \; \mathbf{n}_{wg}^\top \; \mathbf{n}_a^\top \; \mathbf{n}_{wa}^\top\right)^\top$. The vectors $\mathbf{n}_g$ and $\mathbf{n}_a$ represent the Gaussian noise of the gyroscope and accelerometer measurement, while $\mathbf{n}_{wg}$ and $\mathbf{n}_{wa}$ are the random walk rate of the gyroscope and accelerometer measurement biases. $\mathbf{F}$ and $\mathbf{G}$ are shown in Appendix~\ref{sec: error state dynamics}.

To deal with discrete time measurement from the IMU, we apply a $4^{th}$ order Runge-Kutta numerical integration of Eq.~\eqref{eq: estimated state dynamics} to propagate the estimated IMU state. To propagate the uncertainty of the state, the discrete time state transition matrix of Eq.~\eqref{eq: error state dynamics} and discrete time noise covairance matrix need to be computed first, 
\begin{equation*}
\begin{gathered}
\bm{\Phi}_k = \bm{\Phi}(t_{k+1}, t_k) = 
\exp\left(\int_{t_k}^{t_{k+1}} \mathbf{F}(\tau)d\tau\right) \\
\mathbf{Q}_k = \int_{t_k}^{t_{k+1}} \bm{\Phi}(t_{k+1},\tau)\mathbf{G}\mathbf{Q}\mathbf{G}\bm{\Phi}(t_{k+1},\tau)^\top d\tau
\end{gathered}
\end{equation*}
where $\mathbf{Q} = \mathbb{E}\left[\mathbf{n}_I^{}\mathbf{n}_I^\top\right]$ is the continuous time noise covariance matrix of the system. Then the propagated covariance of the IMU state is,
\begin{equation*}
\mathbf{P}_{II_{k+1|k}} = \bm{\Phi}_k\mathbf{P}_{II_{k|k}}\bm{\Phi}_k^\top + \mathbf{Q}_k
\end{equation*}
By partioning the covariance of the whole state as,
\begin{equation*}
\mathbf{P}_{k|k} = 
\begin{pmatrix}
\mathbf{P}_{II_{k|k}} & \mathbf{P}_{IC_{k|k}} \\
\mathbf{P}_{IC_{k|k}}^\top & \mathbf{P}_{CC_{k|k}}
\end{pmatrix}
\end{equation*}
the full uncertainty propagation can be represented as,
\begin{equation*}
\mathbf{P}_{k+1|k} = 
\begin{pmatrix}
\mathbf{P}_{II_{k+1|k}} & \bm{\Phi}_k \mathbf{P}_{IC_{k|k}} \\
\mathbf{P}_{IC_{k|k}}^\top \bm{\Phi}_k^\top & \mathbf{P}_{CC_{k|k}}
\end{pmatrix}
\end{equation*}
When new images are received, the state should be augmented with the new camera state. The pose of the new camera state can be computed from the latest IMU state as,
\begin{equation*}
{}^C_G\hat{\mathbf{q}} = {}^C_I\hat{\mathbf{q}} \otimes {}^I_G\hat{\mathbf{q}}, \quad
{}^G\hat{\mathbf{p}}_C = {}^G\hat{\mathbf{p}}_C + C\left({}^I_G\hat{\mathbf{q}}\right)^\top {}^I\hat{\mathbf{p}}_C
\end{equation*}
And the augmented covariance matrix is,
\begin{equation}
\label{eq: state covariance augmentation}
\mathbf{P}_{k|k} = 
\begin{pmatrix}
\mathbf{I}_{21+6N} \\ \mathbf{J}
\end{pmatrix}
\mathbf{P}_{k|k}
\begin{pmatrix}
\mathbf{I}_{21+6N} \\ \mathbf{J}
\end{pmatrix}^\top
\end{equation}
where $\mathbf{J}$ is shown in Appendix~\ref{sec: state augmentation jacobian}.
\subsection{Measurement Model}
\label{subsec: measurement model}
Consider the case of a single feature $f_j$ observed by the stereo cameras with pose $\left({}^{C_i}_G\mathbf{q}, {}^G\mathbf{p}_{C_i}\right)$. Note that the stereo cameras have different poses, represented as $\left({}^{C_{i,1}}_G\mathbf{q}, {}^G\mathbf{p}_{C_{i,1}}\right)$ and $\left({}^{C_{i,2}}_G\mathbf{q}, {}^G\mathbf{p}_{C_{i,2}}\right)$ for left and right cameras respectively, at the same time instance. Although the state vector only contains the pose of the left camera, the pose of the right camera can be easily obtained using the extrinsic parameters from the calibration. The stereo measurement, $\mathbf{z}_i^j$, is represented as,
\begin{equation}
\mathbf{z}_i^j = 
\begin{pmatrix}
u_{i, 1}^j \\ v_{i, 1}^j \\ 
u_{i, 2}^j \\ v_{i, 2}^j
\end{pmatrix} 
= 
\begin{pmatrix}
\frac{1}{{}^{C_{i, 1}}Z_j} & \mathbf{0}_{2\times 2} \\
\mathbf{0}_{2\times 2} & \frac{1}{{}^{C_{i, 2}}Z_j}
\end{pmatrix}
\begin{pmatrix}
{}^{C_{i, 1}}X_j \\ {}^{C_{i, 1}}Y_j \\
{}^{C_{i, 2}}X_j \\ {}^{C_{i, 2}}Y_j
\end{pmatrix}
\label{eq: stereo measurement}
\end{equation}
Note that the dimension of $\mathbf{z}_i^j$ can be reduced to $\mathbb{R}^3$ assuming the stereo images are properly rectified. However, by representing $\mathbf{z}_i^j$ in $\mathbb{R}^4$, it is no longer required that the observations of the same feature on the stereo images are on the same image plane, which removes the necessity for stereo rectification. In Eq.~\eqref{eq: stereo measurement}, $\left({}^{C_{i, k}}X_j\  {}^{C_{i, k}}Y_j\  {}^{C_{i, k}}Z_j \right)^\top$, $k\in\{1, 2\}$, are the positions of the feature, $f_j$, in the left and right camera frame, $C_{i, 1}$ and $C_{i, 2}$, which are related to the camera pose by,
\begin{equation*}
\begin{gathered}
{}^{C_{i, 1}}\mathbf{p}_j = 
\begin{pmatrix}
{}^{C_{i, 1}}X_j \\ {}^{C_{i, 1}}Y_j \\ {}^{C_{i, 1}}Z_j
\end{pmatrix} = 
C\left({}^{C_{i, 1}}_G\mathbf{q}\right)
\left({}^G\mathbf{p}_j-{}^G\mathbf{p}_{C_{i, 1}}\right) \\
\begin{aligned}
{}^{C_{i, 2}}\mathbf{p}_j &= 
\begin{pmatrix}
{}^{C_{i, 2}}X_j \\ {}^{C_{i, 2}}Y_j \\ {}^{C_{i, 2}}Z_j
\end{pmatrix} = 
C\left({}^{C_{i, 2}}_G\mathbf{q}\right)
\left({}^G\mathbf{p}_j-{}^G\mathbf{p}_{C_{i, 2}}\right) \\
&= C\left({}^{C_{i, 2}}_{C_{i, 1}}\mathbf{q}\right)
\left({}^{C_{i, 1}}\mathbf{p}_j - 
{}^{C_{i, 1}}\mathbf{p}_{C_{i, 2}}\right)
\end{aligned}
\end{gathered}
\end{equation*} 
The position of the feature in the world frame, ${}^G\mathbf{p}_j$, is 
computed using the least square method given in~\cite{mourikis2007multi} based on the current estimated camera poses. Linearizing the measurement model at the current estimate, the residual of the measurement can be approximated as,
\begin{equation}
\label{eq: error measurement model}
\mathbf{r}^j_i = 
\mathbf{z}_i^j - \hat{\mathbf{z}}_i^j = 
\mathbf{H}_{C_i}^j\tilde{\mathbf{x}}_{C_i} + 
\mathbf{H}_{f_i}^j{}^G\tilde{\mathbf{p}}_{j} + 
\mathbf{n}_i^j
\end{equation}
where $\mathbf{n}_i^j$ is the noise of the measurement. The measurement Jacobian $\mathbf{H}_{C_i}^j$ and $\mathbf{H}_{f_i}^j$ are shown in Appendix~\ref{sec: measurement jacobian}.

By stacking multiple observations of the same feature $f_j$, we have,
\begin{equation*}
%\label{eq: stacked error measurement model}
\mathbf{r}^j = 
\mathbf{H}_{\mathbf{x}}^j\tilde{\mathbf{x}} + 
\mathbf{H}_f^j {}^G\tilde{\mathbf{p}}_j + 
\mathbf{n}^j
\end{equation*}
As pointed out in~\cite{mourikis2007multi}, since ${}^G\mathbf{p}_j$ is computed using the camera poses, the uncertainty of ${}^G\mathbf{p}_j$ is, therefore, correlated with the camera poses in the state. In order to ensure the uncertainty of ${}^G\mathbf{p}_j$ does not affect the residual, the residual in Eq.~\eqref{eq: error measurement model} is projected to the null space, $\mathbf{V}$, of $\mathbf{H}_f^j$ , i.e.
\begin{equation}
\label{eq: null space error measurement model}
\mathbf{r}^j_o 
= \mathbf{V}^\top \mathbf{r}^j
= \mathbf{V}^\top \mathbf{H}_{\mathbf{x}}^j\tilde{\mathbf{x}} +
\mathbf{V}^\top \mathbf{n}^j
= \mathbf{H}_{\mathbf{x}, o}^j\tilde{\mathbf{x}} + 
\mathbf{n}^j_o
\end{equation}
Based on Eq.~\eqref{eq: null space error measurement model}, the update step of the EKF can be carried out in a standard way.

















\subsection{Observability Constraint}
\label{subsec: observability constraint}
As has been shown in~\cite{huang2010observability, li2013high}, the EKF-based VIO for 6-DOF motion estimation has four unobservable directions corresponding to global position and rotation along the gravity axis, i.e. yaw angle. A naive implementation of EKF VIO will gain spurious information on yaw. This is due to the fact that the linearizing point of the process and measurement step are different at the same time instant.

There are different methods for maintaining the consistency of the filter, including the First Estimate Jacobian EKF (FEJ-EKF)~\cite{huang2010observability}, the Observability Constrained EKF (OC-EKF)~\cite{hesch2012observability}, and Robocentric Mapping Filter~\cite{castellanos2007robocentric}. In our implementation, OC-EKF is applied for two reasons as discussed in~\cite{huang2010observability},
\begin{enumerate*}[label=(\roman*)]
\item unlike FEJ-EKF, OC-EKF does not heavily depend on an accurate initial estimation,
\item comparing to Robocentric Mapping Filter, camera poses in the state vector can be represented with respect to the inertial frame instead of the latest IMU frame so that the uncertainty of the existing camera states in the state vector is not affected by the uncertainty of the latest IMU state during the propagation step.
\end{enumerate*}

\subsection{Filter Update Mechanism}
\label{subsec: filter update mechanism}
Two types of delayed measurement updates are described in~\cite{mourikis2007multi}. The measurement update step is executed when either the algorithm loses a feature or the number of camera poses in the state reaches the limit. In our implementation, we inherit the same update mechanism with modifications for real-time considerations. As suggested in~\cite{mourikis2007multi}, one third of the camera states are marginalized once the buffer is full, which can cause sudden jumps in computational load in real-time implementations. It is desired that one camera state is marginalized at each time step in order to average out the computation. However, removing the observation of a feature at one camera state is not practical in the MSCKF framework since the observation contains no information about the relative transformation between the camera states. Mathematically, this is due to the fact that the null space of $\mathbf{H}_{f_i}^j$ in Eq.~\eqref{eq: error measurement model} is a subspace of the null space of $\mathbf{H}_{C_i}^j$ (see Appendix~\ref{sec: nullify measurement jacobian}), which results in a trivial measurement model based on Eq.~\eqref{eq: error measurement model}. 

In our implementation, two camera states are removed every other update step. All feature observations obtained at the two camera states are used for measurement update. Note that because of the reason mentioned above, the two stereo measurements of the two camera states are only useful if they are of the same feature. It is understood that such frequent removal of camera states can cause some valid observations to be ignored. In practice, we found that the estimation performance is barely affected although fewer observations are used. To select the two camera states to be removed, we apply a keyframe selection strategy similar to the two-way marginalization method proposed in~\cite{shen2014initialization}.Based on the relative motion between the second latest camera state and its previous one, either the second latest or the oldest camera state is selected for removal. The selection procedure is executed twice to find two camera states to remove. Note that the latest camera state is always kept since it has the measurements for the newly detected features.
\subsection{Image Processing Frontend}
\label{subsec: image processing frontend}
In our implementation, the FAST~\cite{trajkovic1998fast} feature detector is employed for its efficiency. Existing features are tracked temporally using the KLT optical flow algorithm~\cite{lucas1981iterative}. It is shown in~\cite{paulcomparative} that descriptor-based methods for temporal feature tracking are better than KLT-based methods in accuracy. In our experiments, we find that descriptor-based methods require much more CPU resource with small gain in accuracy, making such methods less favorable in our application. Note that we also use the KLT optical flow algorithm for stereo feature matching, which further saves computation compared to descriptor-based methods. Empirically, corner features with depths greater than $1$m can be reliably matched across the stereo images using KLT tracking with a $20$cm baseline stereo configuration. Two types of outlier rejection procedure are implemented in the image processing frontend. A 2-point RANSAC is applied to remove outliers in temporal tracking. In addition, a circular matching similar to~\cite{kitt2010visual} is performed between the previous and current stereo image pairs to further remove outliers generated in the feature tracking and stereo matching steps.










\section{Experiments}\label{sec:expts}

\iffalse
\begin{center}
 \begin{tabular}{|| c | c | c ||} 
 \hline
 Task & RNN(Relu) & RNN(Tanh)  \\ [0.5ex] 
 \hline\hline
  Counter-2 & 0.15 & 0.1\\
  Counter-3 & 0.02 & 0.02\\
  Counter-4 & 0.06 & 0.02 \\
  Boolean-3 & 0.99 & 0.756 \\
  Boolean-5 & 0.0 & 0.766\\
  Shuffle-2 & 0.966 & 0.60\\
  Shuffle-4 & 0.599 & 0.202\\
  Shuffle-6 & 0.348 & 0.338\\ [1ex]
 \hline
\end{tabular}
\end{center}
\fi 


%\iffalse
\begin{figure}[!ht]
\centering
\iffalse
\includegraphics[width=.45\textwidth]{../ESN_RNN_normalizedseq/RNN_2.png}
\includegraphics[width=.45\textwidth]{../ESN_RNN_normalizedseq/RNN_4.png}
\includegraphics[width=.45\textwidth]{../ESN_RNN_normalizedseq/RNN_8.png}
\fi
%\iffalse
\begin{subfigure}
  \centering
  \includegraphics[width=0.5\linewidth]{ESN_RNN_normalizedseq/RNN_2.png}
%          \vspace{-2\baselineskip}
  \caption{Data dimension: 2}
\end{subfigure}
\begin{subfigure}
  \centering
  \includegraphics[width=0.5\linewidth]{ESN_RNN_normalizedseq/RNN_4.png}
%\vspace{-2\baselineskip}
  \caption{Data dimension: 4}
\end{subfigure}%\hfill
\begin{subfigure}
  \centering
  \includegraphics[width=0.5\linewidth]{ESN_RNN_normalizedseq/RNN_8.png}
%\vspace{-2\baselineskip}
  \caption{Data dimension: 8}
\end{subfigure}%
%\fi
\caption{Invertibitiliy of RNNs at random initialization: Checking behavior of inversion error with number of neurons and the sequence length at different data dimensions.}
\label{fig:RNN_inver}
\end{figure}
%\fi 
\textbf{RNN inversion at random initialization.} We consider a randomly initialized RNN, with the entries of the weights $\mathbf{W}$ and $\mathbf{A}$ randomly picked from the distribution $\mathcal{N}(0, 1)$. Sequences are generated i.i.d. from normal distribution i.e. for each sequence, $\bx^{(i)} \sim N(0, \mathbf{I})$ for each $i \in [L]$. We use SGD with batch size 128, momentum $0.9$ and learning rate $0.1$ to compute the linear matrix $\obW^{[L]}$ so that $\norm[0]{\obW^{[L]} \mathbf{h}^{(L)} - [\bx^{(1)}, \ldots, \bx^{(L)}]}^2$ is minimized. We compute the following two quantities on the test dataset, containing $1000$ sequences: average $L_2$ error given by $\mathbb{E}_{\bx} \frac{\norm[0]{\obW^{[L]} \mathbf{h}^{(L)} - [\bx^{(1)}, \ldots, \bx^{(L)}]}}{\norm[0]{[\bx^{(1)}, \ldots, \bx^{(L)}]}}$ and average $L_\infty$ error given by $\mathbb{E}_{\bx} \norm[0]{\obW^{[L]} \mathbf{h}^{(L)} - [\bx^{(1)}, \ldots, \bx^{(L)}]}_{\infty}$. We plot both the quantities for different settings of data dimension $d$, sequence length $L$ and the number of neurons $m$. $L$ takes values from the set $\{2, 4, 6\}$, $d$ takes from $\{2, 4, 8\}$ and $m$ takes from $\{500, 1000, 2000, 5000, 10000\}$ (Figure~\ref{fig:RNN_inver}). The trends support our bounds in Theorem~\ref{thm:Invertibility_ESN_outline}, i.e. the error increases with increasing $L$ and decreases with increasing $m$. Note that the data distribution is different from the one assumed in normalized sequence Def.~\ref{def:normalized_seq}. It was easier to conduct experiments in the current data setting and a similar statement as Thm.~\ref{thm:Invertibility_ESN_outline} can be given.



\textbf{Performance of RNNs on different regular languages. } We check the performance of RNNs on the formal language recognition task for a wide variety of regular languages. We follow the set-up in \cite{BhattamishraAG20} who conducted experiments on LSTMs etc. but not on RNNs.

%\section{Regular languages} \label{sec:regular}
We consider the regular languages as considered in \cite{BhattamishraAG20}.
Tomita grammars \cite{tomita:cogsci82} contain 7
regular languages representable by DFAs of small
sizes, a popular benchmark for evaluating recurrent models (see references in \cite{BhattamishraAG20}). 
We reproduce the definitions of the Tomita grammars from there verbatim:
Tomita Grammars are 7 regular langauges defined on the alphabet $\Sigma = \{0, 1\}$.
Tomita-1 has the regular expression $1^\ast$.
Tomita-2 is defined by the regular expression $(10)^\ast$.
Tomita-3 accepts the strings where odd number
of consecutive 1s are always followed by an even
number of $0$'s. Tomita-4 accepts the strings that
do not contain three consecutive $0$'s. In Tomita-5 only
the strings containing an even number of $0$'s and
even number of $1$'s are allowed. In Tomita-6 the
difference in the number of $1$'s and $0$'s should be
divisible by 3 and finally, Tomita-7 has the regular
expression $0^\ast 1^\ast 0^\ast 1^\ast$. 

We also check the performance of RNNs on $\mathrm{Parity}$, which contains all languages with strings of the form $(w_1, \ldots, w_L)$ s.t. $w_1 + \ldots + w_L = 1 \mod 2$. Languages $\mathcal{D}_n$ are recursively defined as the set of all strings of the form $(0w1)^{\ast}$, where $w \in \mathcal{D}_{n-1}$, with $\mathcal{D}_0$ containing only $\epsilon$, the empty word. Other languages considered are $(00)^{\ast}$, $(0101)^{\ast}$ and $(00)^{\ast}(11)^{\ast}$. Table~\ref{table:regular} shows the number of examples in train and test data, the range of the length of the strings in the language, and the test accuracy of the RNNs with activation functions $\relu$ and $\tanh$ on the regular languages mentioned above. 

\begin{center}
\begin{table}[!ht]
\centering
\begin{tabular}{|| c | c | c | c | c ||} 
 \hline
 Task & No. of Training/Test examples  & Range of length of strings & RNN(Relu) & RNN(Tanh)  \\ [0.5ex] 
 \hline\hline
 Tomita 1 & 50/100 & [2, 50] & 1.0 & 1.0 \\
 Tomita 2 & 25/50 & [2, 50] & 1.0 & 1.0 \\
 Tomita 3 & 10000/2000 & [2, 50] & 1.0 & 1.0 \\
 Tomita 4 & 10000/2000 & [2, 50] & 1.0 & 1.0 \\
 Tomita 5 & 10000/2000 & [2, 50] & 1.0 & 1.0\\
 Tomita 6 & 10000/2000 & [2, 50] & 1.0 & 1.0\\
 Tomita 7 & 10000/2000 & [2, 50] & 0.259 & 0.99\\
 Parity & 10000/2000 & [2, 50] & 1.0 & 1.0\\
 $\mathcal{D}_2$ & 10000/2000 & [2, 100] & 1.0 & 1.0 \\
 $\mathcal{D}_3$ & 10000/2000 & [2, 100] & 0.99 & 1.0\\
 $\mathcal{D}_4$ & 10000/2000 & [2, 100] & 1.0 & 0.99\\
 $(00)^{\ast}$ & 250/50 & [2, 500] & 1.0 & 1.0\\
 $(0101)^{\ast}$ & 125/25 & [4, 500] & 0.99 & 1.0 \\
 $(00)^{\ast}(11)^{\star}$ & 10000/2000 & [2, 200] & 0.99 & 1.0 
 %\\Dyck-1 & 0.99 & \textbf{0.91} 
 \\[1ex]
 \hline
\end{tabular}
\caption{Performance of RNNs on different regular languages.}
\label{table:regular}
\end{table}
\end{center}

%The test languages consist of Tomita grammars which constitute a popular benchmark, and a number of other languages including $\mathsf{PARITY}$ and cover a variety of capabilities needed to recognize regular languages.
%The details of the regular languages above can be found in \cite{BhattamishraAG20}; we only note that Tomita languages constitute a popular benchmark. 
We vary $m$, the dimension of the hidden state, in the range $[3, 32]$, used RMSProp optimizer~\cite{hinton2014coursera} with the smoothing constant $\alpha = 0.99$ and varied the learning rate in the range $[10^{-2}, 10^{-3}]$. For each language
we train models corresponding to each language
for $100$ epochs and a batch size of $32$. We experimented with two different activations $\relu$ and $\tanh$. 
%The best test accuracies achieved on different languages are given in table~\ref{table:regular} and these were all achieved for $m=32$.
In all but one case (Tomita 7 with ReLU) the test accuracies with near-perfect. This was the case across runs. Tomita 7 results could perhaps be improved by more extensive hyperparameter tuning. 
We train and test on strings of length up to 50, and in a few cases strings of larger lengths (when the number of strings in the language is small). 
%Details are in Appendix~\ref{sec:regular}.




\iffalse
\begin{figure*}

\centering
\includegraphics[width=.45\textwidth]{../ESN_RNN_normalizedseq/RNN_2.png}
\includegraphics[width=.45\textwidth]{../ESN_RNN_normalizedseq/RNN_4.png}
\includegraphics[width=.5\textwidth]{../ESN_RNN_normalizedseq/RNN_8.png}
\caption{Invertibility of RNNs at initialization}
%\label{fig:figure3}
\end{figure*}
%\FloatBarrier


\begin{center}
\begin{table}[!ht]
\centering
\begin{tabular}{|| c | c | c ||} 
 \hline
 Task & RNN(Relu) & RNN(Tanh)  \\ [0.5ex] 
 \hline\hline
 Tomita 1 & 1.0 & 1.0 \\
 Tomita 2 & 1.0 & 1.0 \\
 Tomita 3 & 1.0 & 1.0 \\
 Tomita 4 & 1.0 & 1.0 \\
 Tomita 5 & 1.0 & 1.0\\
 Tomita 6 & 1.0 & 1.0\\
 Tomita 7 & 0.259 & 0.99\\
 Parity & 1.0 & 1.0\\
 $\mathcal{D}_1$ & 1.0 & 1.0 \\
 $\mathcal{D}_2$ & 0.99 & 1.0\\
 $\mathcal{D}_4$ & 1.0 & 0.99\\
 $(aa)^{\ast}$ & 1.0 & 1.0\\
 $(abab)^{\ast}$ & 0.99 & 1.0 \\
 $(aa)^{\ast}(bb)^{\star}$ & 0.99 & 1.0 
 %\\Dyck-1 & 0.99 & \textbf{0.91} 
 \\[1ex]
 \hline
\end{tabular}
\caption{Performance of RNNs on different regular languages.}
\label{table:regular}
\end{table}
\end{center}
\fi

\section{Conclusion}
\label{sec:conclusion}
This paper presents a generic top-$\size$ recommendation framework for  trading-off accuracy, novelty, and coverage. To achieve this, we profile the users according to their preference for long-tail novelty. We examine various measures, and formulate an optimization problem to learn these user preferences from interaction data.  We then integrate the user preference estimates in our generic framework, GANC.  Extensive experiments on several datasets confirm that there are trade-offs between accuracy, coverage, and novelty. Almost all re-ranking models increase coverage and novelty at the cost of accuracy. However, existing re-ranking models typically rely on rating prediction models, and are hence more effective in dense settings. Using a generic approach, we can easily incorporate a suitable base accuracy recommender to devise an effective solution for both sparse and dense settings.  %Our results  also indicate there is no single method that outperforms other methods in all metrics. However our techniques obtain a significant improvement in coverage, while  . 
Although we integrated the  long-tail novelty preference estimates into a re-ranking framework, their use-case is not limited to these frameworks. In  the future, we intend to explore the temporal and topical dynamics of long-tail novelty preference, particularly in settings where contextual information is  available.  
%We achieve these objectives without using any additional contextual information.


\iffalse
While we focused on promoting long-tail items across users, we did not consider diversity of individual top-$\size$ recommendations, a factor that has been shown to positively affect consumer satisfaction. This is one direction for future work. Moreover, the sequential setting  in our work, creates a dependency between different batches, where,  the items recommended to a batch of users, depends on those recommended to previous batches. This dependency is created through the parameter $\mathbf{f}$, that is updated every time a top-$\size$ set  is allocated to a batch of users. A future direction for our work is to estimate a distribution over $\mathbf{f}$ that allows us to independently solve the problem for each user, leading to improvements across all performance metrics, including recommendation time. 

We design algorithms that take advantage of the structure in the value functions to obtain both efficient and scalable solutions. 
We design an algorithm that takes advantage of the structure in the value functions to obtain both efficient and scalable solutions. 

\textcolor{red}{Our  sequential  algorithms can be applied for batch recommendation contexts,~e.g., personalized email marketing, where based on prior interaction data between users and items,  a new round of recommendations must be sent to all users in the system.  However, the independent coverage algorithms lift the sequential setting restrictions and allow it be applied for re-ranking the output of base recommender in any setting. }A future direction for our work is to incorporate explicit diversity metrics in the framework. 
\fi


%We have a presented a submodular maximization framework to systematically trade-off relevance and diversity in recommendations to individual users and coverage across the item-space. This ensures both consumer and producer satisfaction. We model users according to their risk and focusing degrees and promote long-tail items to the right group of consumers. Consequently, we obtain a significant improvement in coverage while maintaining reasonable levels of user satisfaction. Furthermore, our methods are able to achieve a more balanced distribution across the set of recommended items. In the future, we plan to investigate the effect of using alternative base recommender systems. 

%Future Work
%However most of these methods assume that the ratings are missing at random (MAR). Since our method of generating recommendations is based on the completed matrix, assuming MAR might introduce additional bias, we will use methods which assume that the ratings at missing not at random (MNAR),explored in~\cite{steck2010training, icml2014c2_hernandez-lobatob14}. 	 
%Long Tail %Recently, authors in~\cite{cremonesi2010performance} conducted extensive experiments to evaluate the performances of various matrix factorization-based algorithms and neighborhood models on the task of recommending long tail items. Their experimental results show that long tail recommendation leads to a decrease in accuracy for all algorithms. They also showed that for this task, SVD outperforms other state-of-the-art algorithms. 


% This command serves to balance the column lengths
% on the last page of the document manually. It shortens
% the textheight of the last page by a suitable amount.
% This command does not take effect until the next page
% so it should come on the page before the last. Make
% sure that you do not shorten the textheight too much.
%\addtolength{\textheight}{-12cm}   



\section*{APPENDIX}
\appendices
\section{}
\label{sec: error state dynamics}
The $\mathbf{F}$ and $\mathbf{G}$ in Eq.~\eqref{eq: error state dynamics} are,
\begin{equation*}
\mathbf{F} = 
\begin{pmatrix}
-\lfloor\hat{\bm{\omega}}{}_{\times}\rfloor & -\mathbf{I}_3 & 
\mathbf{0}_{3\times 3} & \mathbf{0}_{3\times 3} & \mathbf{0}_{3\times 3} \\
\mathbf{0}_{3\times 3} & \mathbf{0}_{3\times 3} & \mathbf{0}_{3\times 3} & 
\mathbf{0}_{3\times 3} & \mathbf{0}_{3\times 3} \\
-C\left({}^I_G\hat{\mathbf{q}}\right)^\top\lfloor\hat{\mathbf{a}}{}_{\times}\rfloor & 
\mathbf{0}_{3\times 3} & \mathbf{0}_{3\times 3} & 
-C\left({}^I_G\hat{\mathbf{q}}\right)^\top & \mathbf{0}_{3\times 3} \\
\mathbf{0}_{3\times 3} & \mathbf{0}_{3\times 3} & \mathbf{0}_{3\times 3} & 
\mathbf{0}_{3\times 3} & \mathbf{0}_{3\times 3} \\
\mathbf{0}_{3\times 3} & \mathbf{0}_{3\times 3} & \mathbf{I}_3 & 
\mathbf{0}_{3\times 3} & \mathbf{0}_{3\times 3} \\
\mathbf{0}_{3\times 3} & \mathbf{0}_{3\times 3} & \mathbf{0}_{3\times 3} & 
\mathbf{0}_{3\times 3} & \mathbf{0}_{3\times 3} \\
\mathbf{0}_{3\times 3} & \mathbf{0}_{3\times 3} & \mathbf{0}_{3\times 3} & 
\mathbf{0}_{3\times 3} & \mathbf{0}_{3\times 3}
\end{pmatrix}
\end{equation*}
and, 
\begin{equation*}
\mathbf{G} = 
\begin{pmatrix}
-\mathbf{I}_3 & \mathbf{0}_{3\times 3} & 
\mathbf{0}_{3\times 3} & \mathbf{0}_{3\times 3} \\
\mathbf{0}_{3\times 3} & \mathbf{I}_3 & 
\mathbf{0}_{3\times 3} & \mathbf{0}_{3\times 3} \\
\mathbf{0}_{3\times 3} & \mathbf{0}_{3\times 3} & 
-C\left({}^I_G\hat{\mathbf{q}}\right)^\top & \mathbf{0}_{3\times 3} \\
\mathbf{0}_{3\times 3} & \mathbf{0}_{3\times 3} & 
\mathbf{0}_{3\times 3} & \mathbf{0}_{3\times 3} \\
\mathbf{0}_{3\times 3} & \mathbf{0}_{3\times 3} & 
\mathbf{0}_{3\times 3} & \mathbf{I}_3 \\
\mathbf{0}_{3\times 3} & \mathbf{0}_{3\times 3} & 
\mathbf{0}_{3\times 3} & \mathbf{0}_{3\times 3} \\
\mathbf{0}_{3\times 3} & \mathbf{0}_{3\times 3} & 
\mathbf{0}_{3\times 3} & \mathbf{0}_{3\times 3}
\end{pmatrix}
\end{equation*}

\section{}
\label{sec: state augmentation jacobian}
The state augmentation Jacobian, $\mathbf{J}$, given in Eq.~\eqref{eq: state covariance augmentation}, is of the form,
\begin{equation*}
\mathbf{J} = 
\begin{pmatrix}
\mathbf{J}_I & \mathbf{0}_{6\times 6N}
\end{pmatrix}
\end{equation*}
where $\mathbf{J}_I$ is,
\begin{equation*}
\mathbf{J}_I = 
\begin{pmatrix}
C\left({}^I_G\hat{\mathbf{q}}\right) & \mathbf{0}_{3\times 9} & 
\mathbf{0}_{3\times 3} & \mathbf{I}_3 & \mathbf{0}_{3\times 3} \\
-C\left({}^I_G\hat{\mathbf{q}}\right)^\top \lfloor{}^I\hat{\mathbf{p}}_C {}_{\times}\rfloor & 
\mathbf{0}_{3\times 9} & \mathbf{I}_3 & \mathbf{0}_{3\times 3} & 
\mathbf{I}_{3}
\end{pmatrix}
\end{equation*}
Note that $\mathbf{J}_I$ given above corrects the typo in Eq. (16) of~\cite{mourikis2007multi}. 

\section{}
\label{sec: measurement jacobian}
Following the chain rule, $\mathbf{H}_{C_i}^j$ and $\mathbf{H}_{f_i}^j$, in Eq.~\eqref{eq: error measurement model}, can be computed as,
\begin{equation}
\label{eq: measurement jacobian}
\begin{gathered}
\mathbf{H}_{C_i}^j = 
\frac{\partial \mathbf{z}_i^j}{\partial {}^{C_{i,1}}\mathbf{p}_j} \cdot 
\frac{\partial {}^{C_{i,1}}\mathbf{p}_j}{\partial \mathbf{x}_{C_{i,1}}} + 
\frac{\partial \mathbf{z}_i^j}{\partial {}^{C_{i,2}}\mathbf{p}_j} \cdot 
\frac{\partial {}^{C_{i,2}}\mathbf{p}_j}{\partial \mathbf{x}_{C_{i,1}}} \\
\mathbf{H}_{f_i}^j = 
\frac{\partial \mathbf{z}_i^j}{\partial {}^{C_{i,1}}\mathbf{p}_j} \cdot 
\frac{\partial {}^{C_{i,1}}\mathbf{p}_j}{\partial {}^G\mathbf{p}_j} +
\frac{\partial \mathbf{z}_i^j}{\partial {}^{C_{i,2}}\mathbf{p}_j} \cdot 
\frac{\partial {}^{C_{i,2}}\mathbf{p}_j}{\partial {}^G\mathbf{p}_j} 
\end{gathered}
\end{equation}
where,
\begin{equation}
\label{eq: measurment jacobian expression}
\begin{gathered}
\frac{\partial \mathbf{z}_i^j}{\partial {}^{C_{i,1}}\mathbf{p}_j} = 
\frac{1}{{}^{C_{i, 1}}\hat{Z}_j}
\begin{pmatrix}
1 & 0 & -\frac{{}^{C_{i, 1}}\hat{X}_j}{{}^{C_{i, 1}}\hat{Z}_j} \\
0 & 1 & -\frac{{}^{C_{i, 1}}\hat{Y}_j}{{}^{C_{i, 1}}\hat{Z}_j} \\
0 & 0 & 0 \\
0 & 0 & 0 
\end{pmatrix} \\
\frac{\partial \mathbf{z}_i^j}{\partial {}^{C_{i,2}}\mathbf{p}_j} = 
\frac{1}{{}^{C_{i, 2}}\hat{Z}_j}
\begin{pmatrix}
0 & 0 & 0 \\
0 & 0 & 0 \\
1 & 0 & -\frac{{}^{C_{i, 2}}\hat{X}_j}{{}^{C_{i, 1}}\hat{Z}_j} \\
0 & 1 & -\frac{{}^{C_{i, 2}}\hat{Y}_j}{{}^{C_{i, 1}}\hat{Z}_j} 
\end{pmatrix} \\
\frac{\partial {}^{C_{i,1}}\mathbf{p}_j}{\partial \mathbf{x}_{C_{i,1}}} = 
\begin{pmatrix}
\lfloor{}^{C_{i,1}}\hat{\mathbf{p}}_j{}_{\times}\rfloor & 
-C\left({}^{C_{i,1}}_G\hat{\mathbf{q}}\right)
\end{pmatrix} \\
\frac{\partial {}^{C_{i,1}}\mathbf{p}_j}{\partial {}^G\mathbf{p}_j} = 
C\left({}^{C_{i,1}}_G\hat{\mathbf{q}}\right) \\
\frac{\partial {}^{C_{i,2}}\mathbf{p}_j}{\partial \mathbf{x}_{C_{i,1}}} = 
C\left({}^{C_{i,1}}_{C_{i,2}}\mathbf{q}\right)^\top
\begin{pmatrix}
\lfloor{}^{C_{i,1}}\hat{\mathbf{p}}_j{}_{\times}\rfloor & 
-C\left({}^{C_{i,1}}_G\hat{\mathbf{q}}\right)
\end{pmatrix} \\
\frac{\partial {}^{C_{i,2}}\mathbf{p}_j}{\partial {}^G\mathbf{p}_j} = 
C\left({}^{C_{i,1}}_{C_{i,2}}\mathbf{q}\right)^\top
C\left({}^{C_{i,1}}_G\hat{\mathbf{q}}\right)
\end{gathered}
\end{equation}

\section{}
\label{sec: nullify measurement jacobian}
By defining the following short-hand notation from Eq.~\eqref{eq: measurment jacobian expression}
\begin{equation*}
\begin{gathered}
\frac{\partial \mathbf{z}_i^j}{\partial {}^{C_{i,1}}\mathbf{p}_j} = 
\begin{pmatrix}
\mathbf{J}_1 \\ \mathbf{0}
\end{pmatrix}, \quad
\frac{\partial \mathbf{z}_i^j}{\partial {}^{C_{i,2}}\mathbf{p}_j} = 
\begin{pmatrix}
\mathbf{0} \\ \mathbf{J}_2
\end{pmatrix}\\
\frac{\partial {}^{C_{i,1}}\mathbf{p}_j}{\partial \mathbf{x}_{C_{i,1}}} = 
\mathbf{H}_1, \quad 
\frac{\partial {}^{C_{i,1}}\mathbf{p}_j}{\partial {}^G\mathbf{p}_j} = 
\mathbf{H}_2, \quad
C\left({}^{C_{i,1}}_{C_{i,2}}\mathbf{q}\right) = 
\mathbf{R}\ ,
\end{gathered}
\end{equation*}
the measurement Jacobian in Eq.~\eqref{eq: measurement jacobian} can be compactly written as
\begin{equation*}
\mathbf{H}_{C_i}^j = 
\begin{pmatrix}
\mathbf{J}_1 \mathbf{H}_1 \\
\mathbf{J}_2 \mathbf{R}^\top \mathbf{H}_1
\end{pmatrix},\quad
\mathbf{H}_{f_i}^j =
\begin{pmatrix}
\mathbf{J}_1 \mathbf{H}_2 \\
\mathbf{J}_2 \mathbf{R}^\top \mathbf{H}_2
\end{pmatrix}\ .
\end{equation*}
Assuming $\mathbf{v} = \left(\mathbf{v}_1^\top,\ \mathbf{v}_2^\top\right)^\top\in\mathbb{R}^4$ is the left null space of $\mathbf{H}_{f_i}^j$, then,
\begin{equation*}
\mathbf{v}^\top \mathbf{H}_{f_i}^j  = 
\left(\mathbf{v}_1^\top \mathbf{J}_1 + 
\mathbf{v}_2^\top\mathbf{J}_2\mathbf{R}^\top\right) 
\mathbf{H}_2 = \mathbf{0}
\end{equation*}
Since $\mathbf{H}_2 = C\left({}^{C_{i,1}}_G\hat{\mathbf{q}}\right)$ is a rotation matrix, $\text{rank}\left(\mathbf{H}_2\right) = 3$ which implies that $\mathbf{v}_1^\top \mathbf{J}_1 + \mathbf{v}_2^\top\mathbf{J}_2\mathbf{R}^\top = \mathbf{0}$. With such property, it immediately follows that $\mathbf{v}$ is also the left null space of $\mathbf{H}_{C_i}^j$, 
\begin{equation*}
\mathbf{v}^\top \mathbf{H}_{C_i}^j = 
\left(\mathbf{v}_1^\top \mathbf{J}_1 + 
\mathbf{v}_2^\top\mathbf{J}_2\mathbf{R}^\top\right) 
\mathbf{H}_1 = \mathbf{0}
\end{equation*}
Therefore, a singe stereo measurement cannot be directly used for measurement update.




%\section*{ACKNOWLEDGMENT}

% Bibliography
%==================================================================%
\bibliographystyle{IEEEtran}
\bibliography{ref}
%==================================================================%


\end{document}
