\section{Filter Description}
\label{sec: filter description}
In the description of the filter setup, we follow the convention in~\cite{mourikis2007multi}. The IMU state is defined as,
\begin{equation*}
\mathbf{x}_{I} = 
\left(
{}^I_G \mathbf{q}^\top \quad 
\mathbf{b}_g^\top \quad 
{}^G\mathbf{v}^\top_I \quad 
\mathbf{b}_a^\top \quad
{}^G\mathbf{p}^\top_I \quad
{}^I_C \mathbf{q}^\top \quad
{}^I\mathbf{p}^\top_C
\right)^\top
\end{equation*}
where the quaternion ${}^I_G \mathbf{q}$ represents the rotation from the inertial frame to the body frame. In our configuration, the body frame is set to be the IMU frame. The vectors ${}^G\mathbf{v}_I \in \RR^3$ and ${}^G\mathbf{p}_I \in \RR^3$ represent the velocity and position of the body frame in the inertial frame. The vectors $\mathbf{b}_g \in \RR^3$ and $\mathbf{b}_a \in \RR^3$ are the biases of the measured angular velocity and linear acceleration from the IMU. Finally the quaternion, ${}^I_C \mathbf{q}$ and ${}^I\mathbf{p}_C \in \RR^3$ represent the relative transformation between the camera frame and the body frame. Without loss of generality, the left camera frame is used assuming the extrinsic parameters relating the left and right cameras are known. Using the true IMU state would cause singularities in the resulting covariance matrices because of the additional unit constraint on the quaternions in the state vector. Instead, the error IMU state, defined as,
\begin{equation*}
\tilde{\mathbf{x}}_{I} = 
\left(
{}^I_G \tilde{\bm{\theta}}^\top \quad 
\tilde{\mathbf{b}}_g^\top \quad 
{}^G\tilde{\mathbf{v}}^\top_I \quad 
\tilde{\mathbf{b}}_a^\top \quad
{}^G\tilde{\mathbf{p}}^\top_I \quad
{}^I_C \tilde{\bm{\theta}}^\top \quad
{}^I\tilde{\mathbf{p}}^\top_C
\right)^\top
\end{equation*}
is used with standard additive error used for position, velocity, and biases (e.g. ${}^G\tilde{\mathbf{p}}_I = {}^G\mathbf{p}_I-{}^G\hat{\mathbf{p}}_I$). For the quaternions, the error quaternion $\delta\mathbf{q} = \mathbf{q}\otimes\hat{\mathbf{q}}^{-1}$ is related to the error state as,
\begin{equation*}
\delta\mathbf{q} \approx
\left(
\frac{1}{2} {}^G_I\tilde{\bm{\theta}}^\top \quad 1
\right)^\top
\end{equation*}
where ${}^G_I\tilde{\bm{\theta}} \in \RR^3$ represents a
small angle rotation. With such a representation, the dimension of orientation error is reduced to $3$ enabling proper presentation of its uncertainty. Ultimately $N$ camera states are considered together in the state vector, so the entire error state vector would be,
\begin{equation*}
\tilde{\mathbf{x}} = 
\left(
\tilde{\mathbf{x}}_I^\top \quad
\tilde{\mathbf{x}}_{C_1}^\top \quad
\cdots \quad 
\tilde{\mathbf{x}}_{C_N}^\top
\right)^\top
\end{equation*}
where each camera error state is defined as,
\begin{equation*}
\tilde{\mathbf{x}}_{C_i} = 
\left(
{}^{C_i}_G\tilde{\bm{\theta}}^\top \quad
{}^G\tilde{\mathbf{p}}_{C_i}^\top
\right)^\top
\end{equation*}
In order to maintain bounded computational complexity, some camera states have to be marginalized once the number of camera states reaches a preset limit. Discussions of how to choose camera states to marginalize can be found in Section \ref{subsec: filter update mechanism}. 
\subsection{Process Model}
\label{subsec: process model}
The continuous dynamics of the estimated IMU state is,
\begin{equation}
\label{eq: estimated state dynamics}
\begin{gathered}
{}^I_G\dot{\hat{\mathbf{q}}} = \frac{1}{2}\Omega(\hat{\bm{\omega}}) {}^I_G\hat{\mathbf{q}}, \quad
\dot{\hat{\mathbf{b}}}_g = \mathbf{0}_{3\times 1}, \\
{}^G\dot{\hat{\mathbf{v}}} = C\left({}^I_G\hat{\mathbf{q}}\right)^\top \hat{\mathbf{a}} + {}^G\mathbf{g}, \\
\dot{\hat{b}}_a = \mathbf{0}_{3\times 1}, \quad
{}^G\dot{\hat{\mathbf{p}}}_I = {}^G\hat{\mathbf{v}}, \\
{}^I_C\dot{\hat{\mathbf{q}}} = \mathbf{0}_{3\times 1}, \quad
{}^I\dot{\hat{\mathbf{p}}}_C = \mathbf{0}_{3\times 1}
\end{gathered}
\end{equation}
where $\hat{\bm{\omega}} \in \RR^3$ and $\hat{\mathbf{a}} \in \RR^3$ are the IMU measurements for angular velocity and acceleration respectively with biases removed, i.e,
\begin{equation*}
\hat{\bm{\omega}} = \bm{\omega}_m - \hat{\mathbf{b}}_g, \quad
\hat{\mathbf{a}} = \mathbf{a}_m - \hat{\mathbf{b}}_a
\end{equation*}
Meanwhile,
\begin{equation*}
\Omega\left(\hat{\bm{\omega}}\right) = 
\begin{pmatrix}
-[\hat{\bm{\omega}}_\times] & \bm{\omega} \\
-\bm{\omega}^\top & 0
\end{pmatrix}
\end{equation*}
where $[\hat{\bm{\omega}}_\times]$ is the skew symmetric matrix of $\hat{\bm{\omega}}$. $C(\cdot)$ in Eq.~\eqref{eq: estimated state dynamics} is the function converting quaternion to the corresponding rotation matrix. Based on Eq.~\eqref{eq: estimated state dynamics}, the linearized continuous dynamics for the error IMU state follows,
\begin{equation}
\label{eq: error state dynamics}
\dot{\tilde{\mathbf{x}}}_I = 
\mathbf{F} \tilde{\mathbf{x}}_I + 
\mathbf{G} \mathbf{n}_I
\end{equation}
where $\mathbf{n}_I^\top = \left(\mathbf{n}_g^\top \; \mathbf{n}_{wg}^\top \; \mathbf{n}_a^\top \; \mathbf{n}_{wa}^\top\right)^\top$. The vectors $\mathbf{n}_g$ and $\mathbf{n}_a$ represent the Gaussian noise of the gyroscope and accelerometer measurement, while $\mathbf{n}_{wg}$ and $\mathbf{n}_{wa}$ are the random walk rate of the gyroscope and accelerometer measurement biases. $\mathbf{F}$ and $\mathbf{G}$ are shown in Appendix~\ref{sec: error state dynamics}.

To deal with discrete time measurement from the IMU, we apply a $4^{th}$ order Runge-Kutta numerical integration of Eq.~\eqref{eq: estimated state dynamics} to propagate the estimated IMU state. To propagate the uncertainty of the state, the discrete time state transition matrix of Eq.~\eqref{eq: error state dynamics} and discrete time noise covairance matrix need to be computed first, 
\begin{equation*}
\begin{gathered}
\bm{\Phi}_k = \bm{\Phi}(t_{k+1}, t_k) = 
\exp\left(\int_{t_k}^{t_{k+1}} \mathbf{F}(\tau)d\tau\right) \\
\mathbf{Q}_k = \int_{t_k}^{t_{k+1}} \bm{\Phi}(t_{k+1},\tau)\mathbf{G}\mathbf{Q}\mathbf{G}\bm{\Phi}(t_{k+1},\tau)^\top d\tau
\end{gathered}
\end{equation*}
where $\mathbf{Q} = \mathbb{E}\left[\mathbf{n}_I^{}\mathbf{n}_I^\top\right]$ is the continuous time noise covariance matrix of the system. Then the propagated covariance of the IMU state is,
\begin{equation*}
\mathbf{P}_{II_{k+1|k}} = \bm{\Phi}_k\mathbf{P}_{II_{k|k}}\bm{\Phi}_k^\top + \mathbf{Q}_k
\end{equation*}
By partioning the covariance of the whole state as,
\begin{equation*}
\mathbf{P}_{k|k} = 
\begin{pmatrix}
\mathbf{P}_{II_{k|k}} & \mathbf{P}_{IC_{k|k}} \\
\mathbf{P}_{IC_{k|k}}^\top & \mathbf{P}_{CC_{k|k}}
\end{pmatrix}
\end{equation*}
the full uncertainty propagation can be represented as,
\begin{equation*}
\mathbf{P}_{k+1|k} = 
\begin{pmatrix}
\mathbf{P}_{II_{k+1|k}} & \bm{\Phi}_k \mathbf{P}_{IC_{k|k}} \\
\mathbf{P}_{IC_{k|k}}^\top \bm{\Phi}_k^\top & \mathbf{P}_{CC_{k|k}}
\end{pmatrix}
\end{equation*}
When new images are received, the state should be augmented with the new camera state. The pose of the new camera state can be computed from the latest IMU state as,
\begin{equation*}
{}^C_G\hat{\mathbf{q}} = {}^C_I\hat{\mathbf{q}} \otimes {}^I_G\hat{\mathbf{q}}, \quad
{}^G\hat{\mathbf{p}}_C = {}^G\hat{\mathbf{p}}_C + C\left({}^I_G\hat{\mathbf{q}}\right)^\top {}^I\hat{\mathbf{p}}_C
\end{equation*}
And the augmented covariance matrix is,
\begin{equation}
\label{eq: state covariance augmentation}
\mathbf{P}_{k|k} = 
\begin{pmatrix}
\mathbf{I}_{21+6N} \\ \mathbf{J}
\end{pmatrix}
\mathbf{P}_{k|k}
\begin{pmatrix}
\mathbf{I}_{21+6N} \\ \mathbf{J}
\end{pmatrix}^\top
\end{equation}
where $\mathbf{J}$ is shown in Appendix~\ref{sec: state augmentation jacobian}.
\subsection{Measurement Model}
\label{subsec: measurement model}
Consider the case of a single feature $f_j$ observed by the stereo cameras with pose $\left({}^{C_i}_G\mathbf{q}, {}^G\mathbf{p}_{C_i}\right)$. Note that the stereo cameras have different poses, represented as $\left({}^{C_{i,1}}_G\mathbf{q}, {}^G\mathbf{p}_{C_{i,1}}\right)$ and $\left({}^{C_{i,2}}_G\mathbf{q}, {}^G\mathbf{p}_{C_{i,2}}\right)$ for left and right cameras respectively, at the same time instance. Although the state vector only contains the pose of the left camera, the pose of the right camera can be easily obtained using the extrinsic parameters from the calibration. The stereo measurement, $\mathbf{z}_i^j$, is represented as,
\begin{equation}
\mathbf{z}_i^j = 
\begin{pmatrix}
u_{i, 1}^j \\ v_{i, 1}^j \\ 
u_{i, 2}^j \\ v_{i, 2}^j
\end{pmatrix} 
= 
\begin{pmatrix}
\frac{1}{{}^{C_{i, 1}}Z_j} & \mathbf{0}_{2\times 2} \\
\mathbf{0}_{2\times 2} & \frac{1}{{}^{C_{i, 2}}Z_j}
\end{pmatrix}
\begin{pmatrix}
{}^{C_{i, 1}}X_j \\ {}^{C_{i, 1}}Y_j \\
{}^{C_{i, 2}}X_j \\ {}^{C_{i, 2}}Y_j
\end{pmatrix}
\label{eq: stereo measurement}
\end{equation}
Note that the dimension of $\mathbf{z}_i^j$ can be reduced to $\mathbb{R}^3$ assuming the stereo images are properly rectified. However, by representing $\mathbf{z}_i^j$ in $\mathbb{R}^4$, it is no longer required that the observations of the same feature on the stereo images are on the same image plane, which removes the necessity for stereo rectification. In Eq.~\eqref{eq: stereo measurement}, $\left({}^{C_{i, k}}X_j\  {}^{C_{i, k}}Y_j\  {}^{C_{i, k}}Z_j \right)^\top$, $k\in\{1, 2\}$, are the positions of the feature, $f_j$, in the left and right camera frame, $C_{i, 1}$ and $C_{i, 2}$, which are related to the camera pose by,
\begin{equation*}
\begin{gathered}
{}^{C_{i, 1}}\mathbf{p}_j = 
\begin{pmatrix}
{}^{C_{i, 1}}X_j \\ {}^{C_{i, 1}}Y_j \\ {}^{C_{i, 1}}Z_j
\end{pmatrix} = 
C\left({}^{C_{i, 1}}_G\mathbf{q}\right)
\left({}^G\mathbf{p}_j-{}^G\mathbf{p}_{C_{i, 1}}\right) \\
\begin{aligned}
{}^{C_{i, 2}}\mathbf{p}_j &= 
\begin{pmatrix}
{}^{C_{i, 2}}X_j \\ {}^{C_{i, 2}}Y_j \\ {}^{C_{i, 2}}Z_j
\end{pmatrix} = 
C\left({}^{C_{i, 2}}_G\mathbf{q}\right)
\left({}^G\mathbf{p}_j-{}^G\mathbf{p}_{C_{i, 2}}\right) \\
&= C\left({}^{C_{i, 2}}_{C_{i, 1}}\mathbf{q}\right)
\left({}^{C_{i, 1}}\mathbf{p}_j - 
{}^{C_{i, 1}}\mathbf{p}_{C_{i, 2}}\right)
\end{aligned}
\end{gathered}
\end{equation*} 
The position of the feature in the world frame, ${}^G\mathbf{p}_j$, is 
computed using the least square method given in~\cite{mourikis2007multi} based on the current estimated camera poses. Linearizing the measurement model at the current estimate, the residual of the measurement can be approximated as,
\begin{equation}
\label{eq: error measurement model}
\mathbf{r}^j_i = 
\mathbf{z}_i^j - \hat{\mathbf{z}}_i^j = 
\mathbf{H}_{C_i}^j\tilde{\mathbf{x}}_{C_i} + 
\mathbf{H}_{f_i}^j{}^G\tilde{\mathbf{p}}_{j} + 
\mathbf{n}_i^j
\end{equation}
where $\mathbf{n}_i^j$ is the noise of the measurement. The measurement Jacobian $\mathbf{H}_{C_i}^j$ and $\mathbf{H}_{f_i}^j$ are shown in Appendix~\ref{sec: measurement jacobian}.

By stacking multiple observations of the same feature $f_j$, we have,
\begin{equation*}
%\label{eq: stacked error measurement model}
\mathbf{r}^j = 
\mathbf{H}_{\mathbf{x}}^j\tilde{\mathbf{x}} + 
\mathbf{H}_f^j {}^G\tilde{\mathbf{p}}_j + 
\mathbf{n}^j
\end{equation*}
As pointed out in~\cite{mourikis2007multi}, since ${}^G\mathbf{p}_j$ is computed using the camera poses, the uncertainty of ${}^G\mathbf{p}_j$ is, therefore, correlated with the camera poses in the state. In order to ensure the uncertainty of ${}^G\mathbf{p}_j$ does not affect the residual, the residual in Eq.~\eqref{eq: error measurement model} is projected to the null space, $\mathbf{V}$, of $\mathbf{H}_f^j$ , i.e.
\begin{equation}
\label{eq: null space error measurement model}
\mathbf{r}^j_o 
= \mathbf{V}^\top \mathbf{r}^j
= \mathbf{V}^\top \mathbf{H}_{\mathbf{x}}^j\tilde{\mathbf{x}} +
\mathbf{V}^\top \mathbf{n}^j
= \mathbf{H}_{\mathbf{x}, o}^j\tilde{\mathbf{x}} + 
\mathbf{n}^j_o
\end{equation}
Based on Eq.~\eqref{eq: null space error measurement model}, the update step of the EKF can be carried out in a standard way.

















\subsection{Observability Constraint}
\label{subsec: observability constraint}
As has been shown in~\cite{huang2010observability, li2013high}, the EKF-based VIO for 6-DOF motion estimation has four unobservable directions corresponding to global position and rotation along the gravity axis, i.e. yaw angle. A naive implementation of EKF VIO will gain spurious information on yaw. This is due to the fact that the linearizing point of the process and measurement step are different at the same time instant.

There are different methods for maintaining the consistency of the filter, including the First Estimate Jacobian EKF (FEJ-EKF)~\cite{huang2010observability}, the Observability Constrained EKF (OC-EKF)~\cite{hesch2012observability}, and Robocentric Mapping Filter~\cite{castellanos2007robocentric}. In our implementation, OC-EKF is applied for two reasons as discussed in~\cite{huang2010observability},
\begin{enumerate*}[label=(\roman*)]
\item unlike FEJ-EKF, OC-EKF does not heavily depend on an accurate initial estimation,
\item comparing to Robocentric Mapping Filter, camera poses in the state vector can be represented with respect to the inertial frame instead of the latest IMU frame so that the uncertainty of the existing camera states in the state vector is not affected by the uncertainty of the latest IMU state during the propagation step.
\end{enumerate*}

\subsection{Filter Update Mechanism}
\label{subsec: filter update mechanism}
Two types of delayed measurement updates are described in~\cite{mourikis2007multi}. The measurement update step is executed when either the algorithm loses a feature or the number of camera poses in the state reaches the limit. In our implementation, we inherit the same update mechanism with modifications for real-time considerations. As suggested in~\cite{mourikis2007multi}, one third of the camera states are marginalized once the buffer is full, which can cause sudden jumps in computational load in real-time implementations. It is desired that one camera state is marginalized at each time step in order to average out the computation. However, removing the observation of a feature at one camera state is not practical in the MSCKF framework since the observation contains no information about the relative transformation between the camera states. Mathematically, this is due to the fact that the null space of $\mathbf{H}_{f_i}^j$ in Eq.~\eqref{eq: error measurement model} is a subspace of the null space of $\mathbf{H}_{C_i}^j$ (see Appendix~\ref{sec: nullify measurement jacobian}), which results in a trivial measurement model based on Eq.~\eqref{eq: error measurement model}. 

In our implementation, two camera states are removed every other update step. All feature observations obtained at the two camera states are used for measurement update. Note that because of the reason mentioned above, the two stereo measurements of the two camera states are only useful if they are of the same feature. It is understood that such frequent removal of camera states can cause some valid observations to be ignored. In practice, we found that the estimation performance is barely affected although fewer observations are used. To select the two camera states to be removed, we apply a keyframe selection strategy similar to the two-way marginalization method proposed in~\cite{shen2014initialization}.Based on the relative motion between the second latest camera state and its previous one, either the second latest or the oldest camera state is selected for removal. The selection procedure is executed twice to find two camera states to remove. Note that the latest camera state is always kept since it has the measurements for the newly detected features.
\subsection{Image Processing Frontend}
\label{subsec: image processing frontend}
In our implementation, the FAST~\cite{trajkovic1998fast} feature detector is employed for its efficiency. Existing features are tracked temporally using the KLT optical flow algorithm~\cite{lucas1981iterative}. It is shown in~\cite{paulcomparative} that descriptor-based methods for temporal feature tracking are better than KLT-based methods in accuracy. In our experiments, we find that descriptor-based methods require much more CPU resource with small gain in accuracy, making such methods less favorable in our application. Note that we also use the KLT optical flow algorithm for stereo feature matching, which further saves computation compared to descriptor-based methods. Empirically, corner features with depths greater than $1$m can be reliably matched across the stereo images using KLT tracking with a $20$cm baseline stereo configuration. Two types of outlier rejection procedure are implemented in the image processing frontend. A 2-point RANSAC is applied to remove outliers in temporal tracking. In addition, a circular matching similar to~\cite{kitt2010visual} is performed between the previous and current stereo image pairs to further remove outliers generated in the feature tracking and stereo matching steps.









