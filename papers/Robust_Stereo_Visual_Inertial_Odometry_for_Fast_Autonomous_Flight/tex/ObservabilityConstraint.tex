\subsection{Observability Constraint}
\label{subsec: observability constraint}
As has been shown in~\cite{huang2010observability, li2013high}, the EKF-based VIO for 6-DOF motion estimation has four unobservable directions corresponding to global position and rotation along the gravity axis, i.e. yaw angle. A naive implementation of EKF VIO will gain spurious information on yaw. This is due to the fact that the linearizing point of the process and measurement step are different at the same time instant.

There are different methods for maintaining the consistency of the filter, including the First Estimate Jacobian EKF (FEJ-EKF)~\cite{huang2010observability}, the Observability Constrained EKF (OC-EKF)~\cite{hesch2012observability}, and Robocentric Mapping Filter~\cite{castellanos2007robocentric}. In our implementation, OC-EKF is applied for two reasons as discussed in~\cite{huang2010observability},
\begin{enumerate*}[label=(\roman*)]
\item unlike FEJ-EKF, OC-EKF does not heavily depend on an accurate initial estimation,
\item comparing to Robocentric Mapping Filter, camera poses in the state vector can be represented with respect to the inertial frame instead of the latest IMU frame so that the uncertainty of the existing camera states in the state vector is not affected by the uncertainty of the latest IMU state during the propagation step.
\end{enumerate*}
