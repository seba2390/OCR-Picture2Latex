\subsection{EuRoC Dataset}
\label{subsec: euroc dataset}
The EuRoC datasets were collected with a VI sensor~\cite{nikolic2014synchronized} on a MAV, which includes synchronized $20$Hz stereo images and $200$Hz IMU messages. The aggressive rotation and significant lighting change make the dataset challenging for vision-based state estimation. 
\begin{figure}[htp]
\centering
\begin{subfigure}[b]{0.4\textwidth}
\includegraphics[width=\textwidth]{figures/euroc_accuracy_benchmark.pdf}
\caption{•}
\label{fig: euroc accuracy benchmark}
\end{subfigure}
\begin{subfigure}[b]{0.4\textwidth}
\includegraphics[width=\textwidth]{figures/euroc_cpu_benchmark.pdf}
\caption{•}
\label{fig: euroc cpu benchmark}
\end{subfigure}
\caption{(a) Root Mean Square Error (RMSE) and (b) average CPU load of OKVIS, ROVIO, VINS-MONO, and the proposed method on the EuRoC dataset. The parameters used for each method are the same as the values given in the corresponding Github repositories. Statistics are averaged over five runs on each dataset. For VINS-MONO and S-MSCKF, the frontend and backend are run as separate ROS nodes. The lighter color represents the CPU usage of the frontend while the darker color represents the backend. Note that the backend of VINS-MONO is run at $10$hz because of limited CPU power.}
\label{fig: euroc benchmark}
\end{figure}
We compare our results on the EuRoC dataset with three representative VIO systems, OKVIS (stereo-optimization), ROVIO (monocular-filter), and VINS-MONO (monocular-optimization). Including the proposed method, the four visual inertial solutions are different combinations of monocular, stereo, filter-based, and optimization-based methods, which may provide insights into the pros and cons of the various approaches. For the monocular approaches, only the images from the left camera are used.  

Figure~\ref{fig: euroc benchmark} shows the Root Mean Square Error (RMSE) and the average CPU load of the four methods on the Dataset. The CPU load is measured with NUC6i7KYK equipped with quad-core i76770HQ 2.6Hz. The proposed method does not work properly on \verb!V2_03_difficult!. The reason is that we use the KLT optical flow algorithm~\cite{lucas1981iterative} for both temporal feature tracking and stereo matching to improve efficiency. The continuous inconsistency in brightness between the stereo images in \verb!V2_03_difficult! causes failures in the stereo feature matching, which then results in divergence of the filter. On the remaining datasets, the accuracy of the four different approaches is similar except ROVIO has larger error in the machine hall datasets which may be caused by the larger scene depth compared to the Vicon room datasets. 
For the CPU usage, the filter-based methods, both monocular and stereo, are more efficient compared with optimization based methods, which makes the filter-based approaches favorable for on-board real-time application. Between OKVIS and VINS-MONO, OKVIS has more CPU usage mainly because it uses Harris corner detector~\cite{harris1988combined} and BRISK~\cite{leutenegger2011brisk} descriptor for both temporal and stereo matching. Also, the backend of OKVIS is run at the fastest possible rate comparing to $10$Hz fixed in VINS-MONO. In the proposed S-MSCKF, around $80\%$ of the computation is caused by the frontend including feature detection, tracking and matching. The filter itself takes about $10\%$ of one core at $20$hz. Our proposed method provides a good compromise between accuracy and computational efficiency. 