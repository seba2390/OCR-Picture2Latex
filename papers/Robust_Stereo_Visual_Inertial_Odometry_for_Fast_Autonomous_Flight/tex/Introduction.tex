\section{Introduction}
\label{sec: introduction}
% What is the problem addressed in this paper?
% Why it is important?
\IEEEPARstart{A}{ccurate} and robust state estimation is of crucial importance for robot autonomy and in particular for micro aerial vehicles (MAVs), where correct pose estimation is essential for stabilizing the robot in the air.

% What is the solution? VIO, need no GPS and has low weight
The solution of combining visual information from cameras and measurements from an Inertial Measurement Unit (IMU), usually referred to as Visual Inertial Odometry (VIO), is popular because it can perform well in GPS-denied environments and, compared to lidar based approaches, requires only a small and  lightweight sensor package, making it the preferred technique for MAV platforms.

% Needs to also be robust
In scenarios such as search and rescue or first response, MAVs have to operate in a wide range of environments that pose challenges to VIO algorithms such as drastically varying lighting conditions, uneven illumination, low texture scenes, and abrupt changes in attitude due to wind gusts or aggressive maneuvering. Thus the VIO not only has to be accurate, it must also be robust.

% Must be low CPU power
In order to achieve full autonomy, all software components, from low-level sensor drivers to high-level planning algorithms, have to run onboard in real-time on a computer with similar computational power to a laptop because of the limited payload of a MAV such as the one shown in Figure~\ref{fig: fla robot}. The requirement to share onboard resources with other components puts additional pressure on the VIO algorithm to be as efficient as possible and importantly, to not produce excessive intermittent spikes in CPU consumption.

\begin{figure}[tp]
\centering
\includegraphics[width=0.4\textwidth]{figures/fla_robot.png}
\caption{The 3kg {\sc Falcon} robot with an onboard Intel NUC5i7RYH computer,  synchronized stereo cameras and IMU, a laser scanner, and a downward facing lidar. Note that only the stereo cameras and the IMU are used for state estimation.}
\label{fig: fla robot}
\end{figure}

% What is the solution proposed in this paper?
% What is the contribution of this paper?
In this paper, we propose a filter-based stereo VIO solution to address such challenges, mostly because they are generally more computationally efficient than competing optimization-based methods. Among the filtering approaches, we choose as a starting point the state-of-art MSCKF~\cite{mourikis2007multi,li2013high,li2013optimization} algorithm for its accuracy and consistency. A stereo configuration is preferred over the recently more popular monocular solutions because of its robustness to different environments and motion. Contradicting the widely held belief that stereo vision-based estimation incurs much higher compute cost than monocular approaches, we demonstrate that the proposed stereo VIO is able to achieve similar or even higher efficiency than state-of-art monocular solutions. Our main contributions are:
\begin{itemize}
\item To the best of our knowledge, the presented work is the first open-source filter-based stereo VIO that can run onboard on a laptop-class computer without GPU acceleration. 
\item We provide detailed experimental comparisons between the proposed S-MSCKF and state-of-art open-source VIO solutions including OKVIS~\cite{leutenegger2015keyframe}, ROVIO~\cite{bloesch2015robust}, and VINS-MONO~\cite{vins-mono} in accuracy, efficiency and robustness. The comparison is performed on both the EuRoC~\cite{burri2016euroc} dataset and the fast flight dataset using {\sc Falcon} shown in Figure~\ref{fig: fla robot}. 
\item The fast flight dataset is publicly available at \url{https://github.com/KumarRobotics/msckf_vio/wiki}.
\end{itemize}

% Structure of the paper.
%
% Bernd: We are looking at 8 pages here, there is no overview needed. I know it's commonly done, but is it necessary? Your call.
%
The rest of the paper is organized as follows: Section~\ref{sec: related work} discusses the related work. Section~\ref{sec: filter description} presents the mathematical details of the proposed algorithm. In Section~\ref{sec: experiments}, we compare our work with different state-of-the-art works in VIO on different datasets and demonstrate the performance of the proposed S-MSCKF with a fully autonomous flight through unstructured and unknown environments. Finally, Section~\ref{sec: conclusion} concludes the paper.