\section{Related Work}
\label{sec: related work}
There is extensive literature on visual odometry, ranging from pure vision-based methods~\cite{forster2014svo, Geiger2011IV, mur2017orb, engel2017direct, engel2014lsd}, loosely-coupled VIO solutions~\cite{weiss2012real, shen2014multi}, to the recently more popular tightly-coupled VIO solutions~\cite{leutenegger2015keyframe, yang2017monocular, usenko2016direct, forster2015imu, mourikis2007multi, wu2015square, bloesch2015robust, kelly2011visual, tsotsos2015robust} which will be the major focus of this paper. 

Existing tightly-coupled VIO solutions can, in general, be classified into optimization-based (e.g. \cite{leutenegger2015keyframe, yang2017monocular, usenko2016direct}) and filter-based approaches (e.g. \cite{mourikis2007multi, tsotsos2015robust, bloesch2015robust}). Optimization-based methods obtain the optimal estimate by jointly minimizing the residual using the measurements from both IMU data and images. Most of these systems, such as~\cite{yang2017monocular}, use sparse features obtained from images as measurements. Methods like this are also called indirect methods. Usenko et al.~\cite{usenko2016direct} and Forster et al.~\cite{forster2015imu} propose a direct method which minimizes photometric error directly in order to exploit more information from the images. The literature shows that optimization-based approaches are able to achieve high accuracy. However, such methods require significant computational resources because of the iterative optimization process, although recent efficient solvers (e.g.~\cite{kaess2012isam2, ceres-solver}) can be run in real time online.

 In contrast, filter-based approaches, which generally use the Extended Kalman Filter (EKF)~\cite{mourikis2007multi}, or the Uncented Kalman Filter~\cite{kelly2011visual}, are much more efficient while achieving accuracy comparable to optimization-based approaches. Huang et al.~\cite{huang2010observability}, Li et al.~\cite{li2013high}, and Hesch et al.~\cite{hesch2014consistency}  also propose the First Estimate Jacobian (FEJ) and the Observability Constraint (OC) to improve consistency of VIO in the filter framework, which in turn improves the estimation accuracy. In  more recent work~\cite{bloesch2015robust, xing2017photometric}, the direct method is used in the filter-based VIO framework to further improve accuracy and robustness.

Only a few VIO solutions are designed for stereo or multi-camera system~\cite{lupton2012visual, leutenegger2015keyframe, usenko2016direct, paulcomparative} compared to the vast amount of work on monocular systems. This can be attributed in part to costs associated with processing additional images and matching features.
In \cite{lupton2012visual}, datasets are collected using stereo visual inertial configuration with cameras running at $6.25$hz, which are then processed offline. The stereo VIO in~\cite{lupton2012visual} is more a proof of concept of IMU pre-integration and does not lend itself to practical implementation. Leutenegger et al.~\cite{leutenegger2015keyframe} propose a more complete optimization framework for multi-camera VIO that is able to run in real time. Usenko~\cite{usenko2016direct} introduces direct methods into stereo VIO in order to further improve accuracy. All three solutions are optimization-based approaches requiring powerful CPUs to operate in real time. More recently, Paul et al.~\cite{paulcomparative} proposed a filter-based stereo VIO based on the square root inverse filter~\cite{wu2015square}, which demonstrates the possibility of operating a stereo VIO online efficiently, even on a mobile device. Since the implementations of~\cite{usenko2016direct} and~\cite{wu2015square} are not open-sourced, they are not used for comparison in this paper.

