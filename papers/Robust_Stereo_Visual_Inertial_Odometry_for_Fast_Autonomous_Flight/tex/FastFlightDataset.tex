\subsection{Fast Flight Dataset}
\label{subsec: fast flight dataset}
To further test the robustness of the proposed S-MSCKF, the algorithm is evaluated on four fast flight datasets with top speeds of $5$m/s, $10$m/s, $15$m/s, and $17.5$m/s respectively collected over an airport runway. During each run, the quadrotor is commanded to go to a waypoint $300$m ahead and return to the starting point. Our configuration includes two forward-looking PointGrey CM3-U3-13Y3M-CS cameras\footnote{https://www.ptgrey.com} running at $40$Hz with resolution $960\times 800$ and one VectorNav VN-100 Rugged IMU\footnote{http://www.vectornav.com/products/vn-100} running at $200$Hz. The whole sensor suite is synchronized based on the trigger signal from the IMU. To achieve proper image exposure under varying lighting conditions, the camera's internal auto-exposure is disabled and replaced by a fast-adapting external controller that maintains constant average image brightness. The controller uses only the left image for brightness measurement, but then applies identical shutter times and gains to both cameras simultaneously. Figure~\ref{fig: fast flight example img} shows some example images of the datasets. The dataset is publicly available at \url{https://github.com/KumarRobotics/msckf_vio/wiki}.

\begin{figure}[t]
\centering
\begin{subfigure}[b]{0.23\textwidth}
\includegraphics[width=\textwidth]{figures/example_img_runway.png}
\caption{•}
\label{fig: fast flight example img runway}
\end{subfigure}
\begin{subfigure}[b]{0.23\textwidth}
\includegraphics[width=\textwidth]{figures/example_img_acceleration.png}
\caption{•}
\label{fig: fast flight example img acceleration}
\end{subfigure}
\caption{Example images in the fast flight datasets. (a) images when the quadrotor is hovering. (b) images when the quadrotor is accelerating.}
\label{fig: fast flight example img}
\end{figure}

Figure~\ref{fig: fast flight benchmark} compares the accuracy and CPU usage of different VIO solutions on the fast flight dataset. The result of ROVIO is omitted in the comparison since it has significant drift in scale which results in much lower accuracy compared to other methods. The accuracy is evaluated by computing the RMSE of estimated and GPS position only in the $x$ and $y$ directions after proper alignment in both time and yaw. From the experiments, it can be observed that the S-MSCKF achieves the lowest CPU usage while maintaining similar accuracy comparing with other solutions. 

Note that compared to the experiments with the EuRoC dataset, the proposed method spends more computational effort on the image processing frontend.  One cause is the higher image frequency and resolution, while the other is that the aggressive flight induces shorter feature lifetime which then requires more frequent new feature detection. Figure~\ref{fig: max speed dataset} shows the aligned trajectories and speed profiles in the dataset with top speed at $17.5$m/s.

\begin{figure}[htp]
\centering
\begin{subfigure}[b]{0.4\textwidth}
\includegraphics[width=\textwidth]{figures/fast_flight_accuracy_benchmark.pdf}
\caption{•}
\label{fig: fast flight accuracy benchmark}
\end{subfigure}
\begin{subfigure}[b]{0.4\textwidth}
\includegraphics[width=\textwidth]{figures/fast_flight_cpu_benchmark.pdf}
\caption{•}
\label{fig: fast flight cpu benchmark}
\end{subfigure}
\caption{(a) RMSE and (b) CPU load of OKVIS, VINS-MONO, and the proposed method averaged over five runs on each dataset. As in the EuRoC dataset test, the CPU load of VINS-MONO and our method is shown as combinations of front and back end. The backend of VINS-MONO is run at $10$hz.}
\label{fig: fast flight benchmark}
\end{figure}

\begin{figure}[htp]
\centering
\begin{subfigure}[b]{0.22\textwidth}
\includegraphics[width=\textwidth]{figures/fast_flight_trajectory.pdf}
\caption{}
\label{fig: fast flight trajectory}
\end{subfigure}
\begin{subfigure}[b]{0.22\textwidth}
\includegraphics[width=\textwidth]{figures/fast_flight_trajectory_origin.pdf}
\caption{}
\label{fig: fast flight trajectory origin}
\end{subfigure}\\
\begin{subfigure}[b]{0.22\textwidth}
\includegraphics[width=\textwidth]{figures/fast_flight_trajectory_goal.pdf}
\caption{}
\label{fig: fast flight trajectory goal}
\end{subfigure}
\begin{subfigure}[b]{0.22\textwidth}
\includegraphics[width=\textwidth]{figures/fast_flight_max_speed.pdf}
\caption{}
\label{fig: fast flight max speed}
\end{subfigure}
\caption{(a) Aligned trajectories, (b) the starting point, (c) the goal location, and (d) speed profiles in the dataset with top speed at $17.5$m/s. }
\label{fig: max speed dataset}
\end{figure}

