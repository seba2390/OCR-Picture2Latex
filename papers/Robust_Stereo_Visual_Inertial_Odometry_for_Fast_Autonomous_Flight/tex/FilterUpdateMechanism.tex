\subsection{Filter Update Mechanism}
\label{subsec: filter update mechanism}
Two types of delayed measurement updates are described in~\cite{mourikis2007multi}. The measurement update step is executed when either the algorithm loses a feature or the number of camera poses in the state reaches the limit. In our implementation, we inherit the same update mechanism with modifications for real-time considerations. As suggested in~\cite{mourikis2007multi}, one third of the camera states are marginalized once the buffer is full, which can cause sudden jumps in computational load in real-time implementations. It is desired that one camera state is marginalized at each time step in order to average out the computation. However, removing the observation of a feature at one camera state is not practical in the MSCKF framework since the observation contains no information about the relative transformation between the camera states. Mathematically, this is due to the fact that the null space of $\mathbf{H}_{f_i}^j$ in Eq.~\eqref{eq: error measurement model} is a subspace of the null space of $\mathbf{H}_{C_i}^j$ (see Appendix~\ref{sec: nullify measurement jacobian}), which results in a trivial measurement model based on Eq.~\eqref{eq: error measurement model}. 

In our implementation, two camera states are removed every other update step. All feature observations obtained at the two camera states are used for measurement update. Note that because of the reason mentioned above, the two stereo measurements of the two camera states are only useful if they are of the same feature. It is understood that such frequent removal of camera states can cause some valid observations to be ignored. In practice, we found that the estimation performance is barely affected although fewer observations are used. To select the two camera states to be removed, we apply a keyframe selection strategy similar to the two-way marginalization method proposed in~\cite{shen2014initialization}.Based on the relative motion between the second latest camera state and its previous one, either the second latest or the oldest camera state is selected for removal. The selection procedure is executed twice to find two camera states to remove. Note that the latest camera state is always kept since it has the measurements for the newly detected features.