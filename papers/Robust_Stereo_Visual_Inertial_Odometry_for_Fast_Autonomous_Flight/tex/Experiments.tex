\section{Experiments}
\label{sec: experiments}
In order to evaluate the proposed method, three different kinds of experiments were performed. First, the proposed method was compared with  state-of-art visual inertial odometry algorithms including OKVIS~\cite{leutenegger2015keyframe}, ROVIO~\cite{bloesch2015robust}, and VINS-MONO~\cite{vins-mono} on the EuRoC dataset~\cite{burri2016euroc}. Second, we demonstrate the robustness of the proposed algorithm on high speed flights reaching maximum speeds of $17.5$m/s on a runway environment. In both of the experiments, the loop closure functionality of VINS-MONO is disabled in order to just compare the odometry of different approaches. Note that although all of the algorithms used in the comparison are capable of estimating extrinsic parameters between the IMU and camera frames online, the offline calibration parameters are provided in the experiments for optimal performance. Finally, we show a representative application of the proposed S-MSCKF in an experiment that combines estimation, with control and planning for autonomous flight in an unstructured and unknown environment which includes a warehouse, a wooded area and a runway.
\subsection{EuRoC Dataset}
\label{subsec: euroc dataset}
The EuRoC datasets were collected with a VI sensor~\cite{nikolic2014synchronized} on a MAV, which includes synchronized $20$Hz stereo images and $200$Hz IMU messages. The aggressive rotation and significant lighting change make the dataset challenging for vision-based state estimation. 
\begin{figure}[htp]
\centering
\begin{subfigure}[b]{0.4\textwidth}
\includegraphics[width=\textwidth]{figures/euroc_accuracy_benchmark.pdf}
\caption{•}
\label{fig: euroc accuracy benchmark}
\end{subfigure}
\begin{subfigure}[b]{0.4\textwidth}
\includegraphics[width=\textwidth]{figures/euroc_cpu_benchmark.pdf}
\caption{•}
\label{fig: euroc cpu benchmark}
\end{subfigure}
\caption{(a) Root Mean Square Error (RMSE) and (b) average CPU load of OKVIS, ROVIO, VINS-MONO, and the proposed method on the EuRoC dataset. The parameters used for each method are the same as the values given in the corresponding Github repositories. Statistics are averaged over five runs on each dataset. For VINS-MONO and S-MSCKF, the frontend and backend are run as separate ROS nodes. The lighter color represents the CPU usage of the frontend while the darker color represents the backend. Note that the backend of VINS-MONO is run at $10$hz because of limited CPU power.}
\label{fig: euroc benchmark}
\end{figure}
We compare our results on the EuRoC dataset with three representative VIO systems, OKVIS (stereo-optimization), ROVIO (monocular-filter), and VINS-MONO (monocular-optimization). Including the proposed method, the four visual inertial solutions are different combinations of monocular, stereo, filter-based, and optimization-based methods, which may provide insights into the pros and cons of the various approaches. For the monocular approaches, only the images from the left camera are used.  

Figure~\ref{fig: euroc benchmark} shows the Root Mean Square Error (RMSE) and the average CPU load of the four methods on the Dataset. The CPU load is measured with NUC6i7KYK equipped with quad-core i76770HQ 2.6Hz. The proposed method does not work properly on \verb!V2_03_difficult!. The reason is that we use the KLT optical flow algorithm~\cite{lucas1981iterative} for both temporal feature tracking and stereo matching to improve efficiency. The continuous inconsistency in brightness between the stereo images in \verb!V2_03_difficult! causes failures in the stereo feature matching, which then results in divergence of the filter. On the remaining datasets, the accuracy of the four different approaches is similar except ROVIO has larger error in the machine hall datasets which may be caused by the larger scene depth compared to the Vicon room datasets. 
For the CPU usage, the filter-based methods, both monocular and stereo, are more efficient compared with optimization based methods, which makes the filter-based approaches favorable for on-board real-time application. Between OKVIS and VINS-MONO, OKVIS has more CPU usage mainly because it uses Harris corner detector~\cite{harris1988combined} and BRISK~\cite{leutenegger2011brisk} descriptor for both temporal and stereo matching. Also, the backend of OKVIS is run at the fastest possible rate comparing to $10$Hz fixed in VINS-MONO. In the proposed S-MSCKF, around $80\%$ of the computation is caused by the frontend including feature detection, tracking and matching. The filter itself takes about $10\%$ of one core at $20$hz. Our proposed method provides a good compromise between accuracy and computational efficiency. 
\subsection{Fast Flight Dataset}
\label{subsec: fast flight dataset}
To further test the robustness of the proposed S-MSCKF, the algorithm is evaluated on four fast flight datasets with top speeds of $5$m/s, $10$m/s, $15$m/s, and $17.5$m/s respectively collected over an airport runway. During each run, the quadrotor is commanded to go to a waypoint $300$m ahead and return to the starting point. Our configuration includes two forward-looking PointGrey CM3-U3-13Y3M-CS cameras\footnote{https://www.ptgrey.com} running at $40$Hz with resolution $960\times 800$ and one VectorNav VN-100 Rugged IMU\footnote{http://www.vectornav.com/products/vn-100} running at $200$Hz. The whole sensor suite is synchronized based on the trigger signal from the IMU. To achieve proper image exposure under varying lighting conditions, the camera's internal auto-exposure is disabled and replaced by a fast-adapting external controller that maintains constant average image brightness. The controller uses only the left image for brightness measurement, but then applies identical shutter times and gains to both cameras simultaneously. Figure~\ref{fig: fast flight example img} shows some example images of the datasets. The dataset is publicly available at \url{https://github.com/KumarRobotics/msckf_vio/wiki}.

\begin{figure}[t]
\centering
\begin{subfigure}[b]{0.23\textwidth}
\includegraphics[width=\textwidth]{figures/example_img_runway.png}
\caption{•}
\label{fig: fast flight example img runway}
\end{subfigure}
\begin{subfigure}[b]{0.23\textwidth}
\includegraphics[width=\textwidth]{figures/example_img_acceleration.png}
\caption{•}
\label{fig: fast flight example img acceleration}
\end{subfigure}
\caption{Example images in the fast flight datasets. (a) images when the quadrotor is hovering. (b) images when the quadrotor is accelerating.}
\label{fig: fast flight example img}
\end{figure}

Figure~\ref{fig: fast flight benchmark} compares the accuracy and CPU usage of different VIO solutions on the fast flight dataset. The result of ROVIO is omitted in the comparison since it has significant drift in scale which results in much lower accuracy compared to other methods. The accuracy is evaluated by computing the RMSE of estimated and GPS position only in the $x$ and $y$ directions after proper alignment in both time and yaw. From the experiments, it can be observed that the S-MSCKF achieves the lowest CPU usage while maintaining similar accuracy comparing with other solutions. 

Note that compared to the experiments with the EuRoC dataset, the proposed method spends more computational effort on the image processing frontend.  One cause is the higher image frequency and resolution, while the other is that the aggressive flight induces shorter feature lifetime which then requires more frequent new feature detection. Figure~\ref{fig: max speed dataset} shows the aligned trajectories and speed profiles in the dataset with top speed at $17.5$m/s.

\begin{figure}[htp]
\centering
\begin{subfigure}[b]{0.4\textwidth}
\includegraphics[width=\textwidth]{figures/fast_flight_accuracy_benchmark.pdf}
\caption{•}
\label{fig: fast flight accuracy benchmark}
\end{subfigure}
\begin{subfigure}[b]{0.4\textwidth}
\includegraphics[width=\textwidth]{figures/fast_flight_cpu_benchmark.pdf}
\caption{•}
\label{fig: fast flight cpu benchmark}
\end{subfigure}
\caption{(a) RMSE and (b) CPU load of OKVIS, VINS-MONO, and the proposed method averaged over five runs on each dataset. As in the EuRoC dataset test, the CPU load of VINS-MONO and our method is shown as combinations of front and back end. The backend of VINS-MONO is run at $10$hz.}
\label{fig: fast flight benchmark}
\end{figure}

\begin{figure}[htp]
\centering
\begin{subfigure}[b]{0.22\textwidth}
\includegraphics[width=\textwidth]{figures/fast_flight_trajectory.pdf}
\caption{}
\label{fig: fast flight trajectory}
\end{subfigure}
\begin{subfigure}[b]{0.22\textwidth}
\includegraphics[width=\textwidth]{figures/fast_flight_trajectory_origin.pdf}
\caption{}
\label{fig: fast flight trajectory origin}
\end{subfigure}\\
\begin{subfigure}[b]{0.22\textwidth}
\includegraphics[width=\textwidth]{figures/fast_flight_trajectory_goal.pdf}
\caption{}
\label{fig: fast flight trajectory goal}
\end{subfigure}
\begin{subfigure}[b]{0.22\textwidth}
\includegraphics[width=\textwidth]{figures/fast_flight_max_speed.pdf}
\caption{}
\label{fig: fast flight max speed}
\end{subfigure}
\caption{(a) Aligned trajectories, (b) the starting point, (c) the goal location, and (d) speed profiles in the dataset with top speed at $17.5$m/s. }
\label{fig: max speed dataset}
\end{figure}


\subsection{Autonomous Flight in Unstructured Environments}
\label{subsec: fla field test}
The proposed S-MSCKF has been thoroughly tested in various field experiments. In this section, we show an example of a fully autonomous flight where the robot has to first navigate through a wooded area, then look for an entrance into a warehouse, find a target, then return to the starting point. This experiment is illustrative since it includes a combination of different kinds of environments as well as common challenges for vision-based estimation including feature poverty, aggressive maneuvers, and significant changes in lighting conditions during indoor-outdoor transitions. 

Figure \ref{fig: fla experiment iv} shows the global laser point cloud and round-trip trajectory overlaid on the Google satellite map.  Note that, during the experiment, the laser measurement is used for mapping only. The state estimation is solely based on the stereo cameras and IMU as the sensor configuration given in Section~\ref{subsec: fast flight dataset}. Over $700$m round-trip trajectory, the final drift is around $3$m, which is less than $0.5\%$ of the total traveled distance despite the combination of various challenges along the flight. More details of this trial can be found in the supplementary video\footnote{https://youtu.be/jxfJFgzmNSw}

\begin{figure}[htp]
\centering
\includegraphics[width=0.45\textwidth]{figures/fla_experiment_iv_map.png}
\caption{The global map and round-trip trajectory overlaid on the Google satellite map in an fully autonomous flight experiment. The \textcolor{blue}{blue}, \textcolor{red}{red}, and \textcolor{yellow}{yellow} dots represents the staring point, goal location, and the only entrance of the warehouse respectively. The global laser point cloud is registered using the estimation produced by the S-MSCKF. Over $700$m trajectory, the final drift is around $3$m under $0.5\%$ of the total traveled distance.}
\label{fig: fla experiment iv}
\end{figure}