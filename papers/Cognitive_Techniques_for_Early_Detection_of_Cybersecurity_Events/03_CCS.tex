\label{sec:ccs}
This section describes our approach to detect cyber-attacks mimicking the cognitive capabilities of humans. Oxford dictionary defines cognition~\cite{cognition_oxford} as \textit{``the mental action or process of acquiring knowledge and understanding through thought, experience, and the senses''}. Our cognitive strategy involves acquiring knowledge from various intelligence sources, combining them into an existing knowledge graph (which is already populated with information about attack patterns, previous attacks, tools used for attacks, etc.) and using this knowledge graph to reason over the data from multiple traditional and non-traditional sensors to detect cybersecurity events. 

A novel feature of our framework is its ability to assimilate information from dynamic textual sources and combine it with malware behavioral information, detecting known and unknown attacks. The main challenge with the textual sources is that they are meant for human consumption and hence the information can be incomplete. Moreover, the text is written for a specific audience who already has some knowledge about the topic. For instance, if the target audience of an article is a security analyst, the line \textit{``Wannacry is a new ransomware.''} carries more knowledge than the textual information. Some thoughts readily inferred by a human who reads it may be the following. \textit{``wannacry is a ransomware'',  ``wannacry tries to encrypt sensitive files'', ``It uses some encryption tool to encrypt these sensitive files'', ``A downloaded program may have initiated the encryption'', ``Either downloaded keys or randomly generated keys are used for encryption'', ``Encryption can increase the processor usage in the victim machine'', ``It modifies many sensitive files''}, etc. Our cognitive approach addresses this issue by remembering the experiences or knowledge in a knowledge graph, and combining it with new and potentially incomplete textual knowledge using standard reasoning techniques. 

Most cybersecurity events follow a pattern which we call an intrusion kill chain. Lockheed Martin~\cite{hutchins2011intelligence} defines it with the following seven steps.
\begin{itemize}
	\item {\bf Reconnaissance:} Gathering information about the target and various existing attacks (e.g., port scanning, collecting public information on hardware/software used, etc.)
	\item {\bf Weaponization:} Combining a specific trojan (software to provide remote access to a victim machine) with an exploit (software to get first unauthorized access to the victim machine, often exploiting vulnerabilities). Trojans and exploits are chosen taking the knowledge from Reconnaissance into consideration.
	\item {\bf Delivery:} Deliver the weaponized payload to the victim machine. (e.g., email attachments, removable media, HTML pages, etc.)
	\item {\bf Exploitation:} Execution of the weaponized payload on the victim machine.
	\item {\bf Installation:} Once the exploitation is successful, the attacker gains easier access to victim machine by installing the trojan attached. 
	\item {\bf Command and Control (C2):} The trojan installed on the victim machine can connect to a Command and Control machine and get ready to receive various commands to be executed on the victim machine. Often APTs use such a strategy. 
	\item {\bf Actions on Objectives:} The final step is to carry out different malicious actions on the victim machine. For example, a ransomware starts searching and encrypting sensitive files. 
\end{itemize} 

Different attacks use one or more of these seven steps during their execution. However, many attacks have fewer steps. For example, some attacks are self-contained such that there is no requirement of a command and control setup. Attacks often use the same tools or similar techniques during their execution. Newer attacks could be permutations of tools/techniques used in different stages in older attacks. We use this understanding to detect known and potentially unknown attacks. Some examples are attackers using the same port scanning tools like \textit{``nmap''} for reconnaissance or  using the same exploit \textit{``Eternal Blue''} in the Weaponization stage for different major attacks like \textit{Wannacry, NotPetya, Retefe, etc.}
Thus, combining information about different known attacks, fusing it with incomplete textual information and reasoning over them will help us to detect newer attacks. 

Consider an example in which a blog reported a new ransomware which uses \textit{nmap} for reconnaissance, and \textit{Eternal Blue} for exploitation. Our knowledge graph is already populated with common information including that ``\textit{Eternal Blue} uses mal-formed SMB packets for exploitation'', ``a generic ransomware modifies sensitive files'', ``ransomware increases the processor utilization'', etc. Let's also consider that sensors detected sensitive file modifications, mal-formed SMB packets, and the nmap port scan. This data alone cannot conclusively detect the presence of a ransomware attack because these can happen for other reasons also (eg. files are modified intentionally by the user, incorrect SMB packets due to a bad network, etc.). However, when we register the information from the blog that \textit{a new ransomware uses Eternal Blue for exploitation} it will act as the missing piece of a jigsaw puzzle to indicate the presence of an attack with better certainty. 

\subsection{Attack Model}

To constrain the system, we make some assumptions about the attacker. The attacker does not have complete inside knowledge of the system being attacked which implies that he will perform some level of probing or reconnaissance. Another assumption we make is that not all attacks are brand new. The attackers will reuse published (in security blogs, dark market, intelligence sharing formats like STIX, etc.) vulnerabilities in software/systems to perform different mal-activities like DoS (Denial-of-Service), data ex-filtration, unauthorized access, etc. Finally, we assume that our framework will have enough traditional sensors to detect basic behaviors in a network (Snort), host (HIDS), etc. 

We categorize attackers into script kiddies, intermediate and advanced state actors. Often script kiddies use existing known techniques and try permutations of known tools for intrusion. Intermediate attackers, on the other hand, modify known attacks or tools significantly and try to evade direct detection, but attack behaviors remain the same.  The state actors or experts mine for new vulnerabilities and come up with brand new attacks. Our system tries to defend effectively against the first two categories, but it will be hard to defend the third category until information about those attacks is added to the knowledge graph.

