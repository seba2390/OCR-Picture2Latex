\label{sec:SystemArchitecture}
New attack techniques are reported on a daily basis.  
In this paper, we propose a cognitive framework to detect cybersecurity events amalgamating information from traditional sensors, dynamic online textual information and knowledge graphs. The system architecture for our framework is described in Figure~\ref{fig:ccs}. The three major input sources to our framework are the dynamic information from textual sources, input from traditional sensors, and inputs from human experts. The Intel-Aggregate module captures information from these sources, converts them to semantic web OWL representation and delivers them to the CCS (Cognitive CyberSecurity) module. The CCS module is the brain of our framework where actionable intelligence will be generated to assist the security analyst. The various components are described in detail below.

\begin{figure}	
	\includegraphics[width=\columnwidth]{arch.png}
	\caption{Cognitive CyberSecurity Architecture}	
	\label{fig:ccs}
\end{figure}

\subsection{CCS Framework Inputs}
\label{sec:CCSFrameworkInputs}
The first input is from textual sources. This input can either be structured information in formats like STIX, TAXII, etc. (from threat intelligence sources like US-CERT, Talos, etc.) or plain text from sources like blogs, twitter, Reddit posts, and dark-web posts. Part of a sample threat intelligence in STIX format shared by US-CERT on \textit{wannacry} is presented in Figure~\ref{fig:wannacry_stix}. We use an off-the-shelf Named-Entity Recognizer (NER) trained on cybersecurity text from Joshi et al.~\cite{joshi2013extracting} for extracting entities from plain text. The next input is from traditional network sensors (Snort, Bro, etc.) and host sensors (Host intrusion detection systems, file monitoring modules, process monitoring modules, firewalls, etc.). We use the logs from these sensors as input to our system. Finally, human experts can define specific rules to detect complex behaviors or complex attacks. Input from human experts is vital because most organizations maintain policies or standards for the activities in their networks. For example, organizations may use a white-list policy for inbound IP connections (ie. Only IP addresses from a specific list are accepted by default). In such cases, a simple firewall rule will block unauthorized accesses. However, it would add value to a security analyst to know if there is a sudden spike in the inbound accesses from illegal IP addresses even though they are blocked (Simple blocking techniques cannot stop a motivated attacker and he will find a way in. Perhaps an analyst considers such activities as a precursor). Moreover, today an analyst's intuitions are used for identifying potential intrusions. Hence if we can capture these intuitions we can make our framework better. All these inputs are sent to the Intel-Aggregate module for further processing.

\begin{figure}	
	\includegraphics[width=\columnwidth]{wannacry_stix.png}
	\caption{STIX for Wannacry Ransomware}	
	\label{fig:wannacry_stix}
\end{figure}


\subsection{Cognitive CyberSecurity Module}

The Cognitive CyberSecurity (CCS) module is the brain of our framework which generates actionable intelligence from the input data. The core of this module is knowledge representation. We use an extension of UCO (Unified CyberSecurity Ontology) using a W3C standard OWL format to represent knowledge in the domain. We extend UCO such that it can reason over the inputs from various network sensors like Snort, IDS, etc. and information from the cyber-Kill chain. We also use SWRL (Semantic Web Rule Language) to specify rules between entities. For instance, SWRL rules are used to specify that an attack would be detected if different stages in the kill chain are identified for a specific IP address. The information in the knowledge graph is general such that experts can easily add new knowledge to it. Experts can directly use different known techniques as indicators because how indicators can be detected (directly from sensors or using complex analysis of these sensors) is already present in the knowledge graph. For example, if an expert mentions port scan as an indicator, 
the reasoner will automatically infer that Snort can detect it and looks for Snort alerts. 
The statistical analytics sub-module and graph analytic sub-module check for anomalous activities using standard techniques like frequency analysis, clustering techniques, etc. Any standard technique can be utilized to generate indicators as far as they generate standard OWL triplets to be fed to the CCS knowledge graph. A standard reasoner is also part of the CCS module which will reason over the knowledge graph generating actionable intelligence. A concrete proof of concept implementation for this model is described in Section~\ref{sec:Evaluation}.

\subsection{Intel-Aggregate Module}

The Intel-Aggregate (IA) Module is responsible for the conversion of various traditional and non-traditional network sensor inputs to the standard semantic web OWL format. Various inputs to our system are mentioned in Section~\ref{sec:CCSFrameworkInputs}. However, they will produce outputs in different formats and will be incompatible with our framework's knowledge graph (represented using UCO). UCO has defined various entities, classes, etc. and, to be consistent with this knowledge graph, these inputs need to be transformed. This module assimilates all inputs to the cognitive framework, maps them to UCO classes and entities, and generates their corresponding well-formed OWL statements. The IA module will be part of all the sensors which are attached to the framework.
