Cybersecurity threats and associated costs to organizations are surging. About 23,000~\cite{pandasecurity} new malware samples are produced daily and a company's average cost for a data breach is about \$3.4 million (according to Microsoft~\cite{thebestvpn}). 
Many corporate enterprises fell prey to cybersecurity attacks because of the time gap between the exploit becoming public and patching their systems. For example, consider the time line of the \textit{wannacry} ransomware, which uses an SMB vulnerability in Windows 7 machines for gaining access to victim systems. On March 14 2017, Microsoft Security Bulletin~\cite{ms017bulletin} and Cisco NGFW published this vulnerability. A hacker group, \textit{Shadow Brokers}~\cite{ciscowannacry}, released a set of vulnerabilities including Eternal Blue and Double Pulsar on April 14, 2017.

By Mid-May, the actual \textit{wannacry} ransomware, which exploits the SMB vulnerability using Eternal Blue started spreading\footnote{\url{https://en.wikipedia.org/wiki/WannaCry_ransomware_attack}}. Had enterprises patched their systems or properly configured their intrusion detection systems to detect the vulnerability, the wide-scale spread of this ransomware, which infected over two hundred thousand victims, could have been prevented. Such incidents reveal how critical it is to be aware of newly reported vulnerabilities and attacks. However, due to the avalanche of threat intelligence information from sources like chat forums, security bulletins, blogs and reports, security analysts find it difficult to keep track of all such information. 

Variations of the same cyber-attack is another concern for attack detection. When attacks become popular, security companies release signatures to detect them. To evade detection, attackers often modify the attack or use different modes to attack. An example is the Petya ransomware\footnote{\url{https://blog.checkpoint.com/2016/04/11/decrypting-the-petya-ransomware/}} attack, which was discovered in 2016. It spread via email attachments and infected computers running Windows, overwriting the Master Boot Record (MBR), installing a custom boot loader, and forcing a reboot. The custom boot-loader encrypts all Master-File-Table (MFT) records, rendering the complete file system unreadable. The attack did not succeed in large-scale infections. However, another attack resurfaced in 2017 sharing significant code with Petya. Instead of using email attachments to spread, the new attack, named NotPetya, used Eternal Blue to spread. %While it is debatable whether NotPetya is actually a variant of the same attack from the same attackers, it demonstrated similar behavior as Petya. 

Advanced Persistent Threats (APTs) are another class of attacks that are difficult to detect.
APT attacks tend to be sophisticated and the attackers are persistent over a long period of time \cite{li2011detailed}\cite{sood2013targeted}. The attackers gain illegal access to an organization's network and may go undetected for a significant time period, with the complete scope of attack remaining unknown. Unlike other common threats, such as viruses and trojans, APTs are implemented in multiple stages \cite{sood2013targeted}. The stages broadly include a reconnaissance (or surveillance) of the target network or hosts, gaining illegal access \& payload delivery, and execution of malicious programs \cite{bhatt2014towards}. Although these steps remain the same, the specific vulnerabilities used to perform them might change from one APT to another. Hence an important challenge when creating new approaches to detect threats (or APTs) is designing systems that can easily adapt to the evolving threat landscape, and detect attacks early (i.e., "left of exploit" in the "Cyber Kill Chain".) 

Modern Security Information and Event Management (SIEM) systems emerged when early security monitoring systems like Intrusion Detection Systems (IDS), and Intrusion Detection and Prevention Systems (IDPS) started to flood the security analyst with alerts. LogRhythm, Splunk, IBM, and AlienVault are just a few of the commercially~\cite{gartnersiem} available SIEM systems. A typical SIEM collects security log events from a large array of machines in a big enterprise, aggregates this data centrally, and does simple analyses to provide security analysts with information. However, despite ingesting large volumes of host/network sensor data, SIEM reports are hard to understand, noisy, and not actionable~\cite{netwrixlimits}. Noise in SIEM reports bothers 81\% of users as reported in a survey~\cite{netwrixinfo} on SIEM efficiency.  What is missing in such systems is the integration of threat intelligence from disparate sources and efficient interpretation of data using known intelligence. This can reduce false positives and improve the current state of the art in this domain. Equally important, it reduces the cognitive load on the analyst, because the system can fuse threat intelligence with observed data to detect attacks early, ideally before the exploit has started.

In this paper, we describe a cognitive strategy to assimilate and process information from a wide variety of traditional and non-traditional sources. 
A key challenge with textual sources like blogs and security bulletins, is their inherent incompleteness. They are often written for specific audiences and do not explain or define what each term means. For example, an excerpt from the Microsoft security bulletin is ``The most severe of the vulnerabilities could allow remote code execution if an attacker sends specially crafted messages to a Microsoft Server Message Block 1.0 (SMBv1) server.''. Since this is intended for experts in computer science, the rest of the text will not describe what a remote code execution or SMB server is.

To fill this gap, we use the Unified Cybersecurity Ontology \cite{syed2016uco} to represent the cybersecurity domain knowledge. It provides a common ontology for information from a disparate stream of sources and gives a combined/unified view of the data. Concepts and standards from different intelligent sources like STIX \cite{barnum2012standardizing}, CVE \cite{mell2002use}, CCE, CVSS, CAPEC, CYBOX, KillChain, and STUCCO can be represented directly using UCO. 

We have developed a proof of concept system that ingests information from textual sources, combines it with knowledge about the state of the system as observed by host and network sensors, and reasons over them to detect known and potentially unknown attacks. We developed multiple agents, including a process monitoring agent, a file monitoring agent and a snort agent, that run on respective machines and deliver knowledge to the Cognitive CyberSecurity (CCS) module. The CCS module then reasons over the data and knowledge to detect cybersecurity events and reports them to the security analyst using a dashboard interface as described in section~\ref{sec:Evaluation}. We also developed a custom ransomware program, similar to wannacry, to test and evaluate the capabilities of our prototype system. Its design and working are described in detail in section~\ref{sec:CustomRansomwareDesign}. We build upon our earlier work in this domain \cite{more2012knowledge}.

In the following sections, we describe in detail the various approaches that are similar to our work (section~\ref{sec:RelatedWork}), a detailed description of CCS and how it works (section~\ref{sec:ccs}), the system architecture (section~\ref{sec:SystemArchitecture}) and evaluation of our system using a simulated ransomware attack (section~\ref{sec:Evaluation}) to showcase the effectiveness of the approach.