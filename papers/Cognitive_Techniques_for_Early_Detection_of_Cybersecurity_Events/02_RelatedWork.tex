\label{sec:RelatedWork}
\subsection{Security \& Event Management}
As the complexity of threats and APTs has grown, several companies have released commercial platforms or security information and event management (SIEM) systems that integrate information from different sources. A typical SIEM has a number of features such as managing logs from disparate sources, correlation analysis of various events, and mechanisms to alert system administrators \cite{swift2006practical}. IBM's QRadar, for example, can manage logs, detect anomalies, assess vulnerabilities,  and perform forensic analysis of known incidents \cite{qradar}. Its threat intelligence comes from IBMs X-Force \cite{nicolett2017magic}. Cisco's Talos \cite{talos} is another threat intelligence system. Many SIEMs\footnote{\url{https://www.gartner.com/reviews/market/security-information-event-management/compare/logrhythm-vs-logpoint-vs-splunk}}, such as LogRhythm, Splunk, AlienVault, Micro Focus, McAfee, LogPoint, Dell Technologies (RSA), Elastic, Rapid 7 and Comodo, exist in the market with capabilities like real-time monitoring, threat intelligence, behavior profiling, data and user monitoring, application monitoring, log management and analytics. 

\subsection{Ontology based Systems}

Obrst et. al \cite{obrst2012developing} detail a process to design an ontology for the cybersecurity domain. The study is based on the diamond model that defines malicious activity \cite{ingle2010organizing}. Ontologies are constructed in a three tier architecture consisting of a domain-specific ontology at the lowest layer, a mid-level ontology that clusters and defines multiple domains together and an upper-level ontology that is defined to be as universal as possible. Multiple ontologies designed later-on have used the above mentioned process. Oltramari et. al \cite{oltramari2015computational} created CRATELO as a three layered ontology to characterize different network security threats. The layers include an ontology for secure operations (OSCO) that combines different domain ontologies, a security-related middle ontology (SECCO) that extends security concepts, and the DOLCE ontology \cite{masolo2002wonderweb} at the higher level. In Oltramari et. al \cite{oltramari2014building}, a simplified version of the DOLCE ontology (DOLCE-SPRAY) is used to show how a SQL injection attack can be detected. Ben-Asher et. al. \cite{ben2015ontology} designed a hybrid ontology-based model combining a network packet-centric ontology representing network-traffic with an adaptive cognitive agent that learns how humans make decisions while defending against malicious attacks. The agent is based on instance-based learning theory, using reinforcement learning to improve decision making through experience.
Gregio et. al. \cite{gregio2014ontology} discusses a comprehensive ontology to define malware behavior.

Each of these systems and ontologies looks at a narrow subset of information, such as network traffic or host system information, while SIEM products do not use the vast capabilities and benefits of an ontological approach and systems to reason using them.  In this regard, Cognitive CyberSecurity (CCS) takes a larger and comprehensive view of security threats by integrating information from multiple existing ontologies as well as network \& host-based sensors (including system information) to first create a single representative view of the data for system administrators and then provide a framework to reason across these various sources of data.

This paper significantly improves our previous work~\cite{more2012knowledge} in this domain, where semantic rules were used to detect cyber attacks. CCS uses the Unified Cybersecurity Ontology (UCO) which is a sophisticated and STIX compliant  Ontology to represent the knowledge in this domain. Current extensions to it help linking standard cyber kill chain phases to various host behaviors and network behaviors that are detected by traditional sensors like Snort and monitoring agents. Unlike our previous work, such extensions allow our technique to assimilate incomplete textual sources  for cybersecurity event detection in a cognitive manner. 