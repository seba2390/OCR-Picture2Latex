\label{sec:Evaluation}
Evaluation of a system by detecting real attacks is difficult. Various challenges involve getting access to an actual malware, finding a safe execution environment within existing laws, etc. Hence, we developed a sample ransomware which is safe but displays typical malicious behaviors.

In this section, we describe the proof of concept implementation of our approach, an evaluation network on which we ran the custom ransomware, and discuss the effectiveness of our cognitive approach using the simulated network. 

\subsection{Proof of Concept Cognitive CyberSecurity System}

Our proof of concept implementation has an architecture similar to some real-world systems like Symantec Data Center Security and Crowd Strike. Our system has a CCS module (the master node) which detects various cybersecurity events and a configurable set of cognitive agents which run on host systems collecting various statistics, as shown in Figure~\ref{fig:ccs_arch}. 

\begin{figure}	
	\includegraphics[width=\columnwidth]{ccs_arch.png}
	\caption{Proof of Concept CCS Architecture}	
	\label{fig:ccs_arch}
\end{figure}

\subsubsection{Cognitive Agents}
\label{sec:CognitiveAgents}

A full agent is a combination of the Intel-Aggregator module and a traditional sensor. The Intel-Aggregator module is developed in such a way that it can be customized to work with multiple traditional sensors collecting and sending information to the CCS module (in OWL format) for further processing. In our proof of concept to detect ransomware attacks, we used a process monitoring agent, a file monitoring agent and a Snort agent as enumerated below.
\begin{itemize}
	\item \textit{Process Monitoring Agent:} This agent combines a  custom process monitor and an IA module. It will run on all host machines in the network and monitor different processes in the machine, their parent hierarchy, and various statistics like memory usage, CPU usage etc. The agent converts all this information into OWL format using the IA module and reports them to the CCS module.
	\item \textit{File Monitoring Agent:} A custom file monitor is attached to an IA module to develop this agent. Similar to a process monitoring agent, the file monitoring agent also runs on all host machines aggregating various file-related statistics. Monitoring all files is a cumbersome task. Hence, we maintain a list of sensitive files, file locations, and suspicious files. The suspicious ones are all those new files generated by a new process, large files downloaded from the Internet, files copied from mass storage devices, etc. Various statistics which are sent to the CCS module include the process which modified the file, size, how it was created, etc. 
	\item \textit{Snort Agent:} Like the name suggests, this agent is a combination of a traditional snort and IA module. This agent will read the output log file of a snort and generates OWL consistent with the CCS module's knowledge graph.
\end{itemize}

\subsubsection{CCS Module}

The CCS module is the brain of our approach which uses cognitive analytics to detect attacks using information from cognitive agents. We implemented it using an Apache Fuseki server configured with a SWRL and Jena reasoner. This module will get inputs from all the agents enumerated, reason over them and feed output to a cognitive dashboard (a custom website). The dashboard will dynamically display various statistics, IP information, detected activities, etc. On detection of a full-scale attack, the dashboard will pop attack alerts as shown in Figure~\ref{fig:ccs_dashboard} (dashboard screenshot).
\begin{figure}	
	\includegraphics[width=\columnwidth]{ccs_dashboard.jpg}
	\caption{CCS Dashboard}	
	\label{fig:ccs_dashboard}
\end{figure}

\subsection{Custom Ransomware Design} 
\label{sec:CustomRansomwareDesign}

Our custom ransomware targets Windows 7 machines which have CVE-2017-0143 vulnerability~\cite{CVE-2017-0143} (buffer overflow related to SMB protocol). We use the exploit from metasploit (for this CVE) to get access to the victim machine. Once the custom ransomware gets access to the victim machine, it downloads the malware script from the attacker machine to the victim machine. The downloaded malware script performs the following functions.
\begin{enumerate}
	\item Download an executable to encrypt files
	\item Download a public key from the attacker machine
	\item The script enumerates the sensitive files/folders in the victim's machine. The malware script contains a compiled list of potential locations (eg. Default Thunderbird email client storage location, default Outlook location, documents folder, default downloads folder, etc.). Our ransomware avoids system files because it hampers the booting process.
	\item All the selected files are split into chunks and a random key is generated for each chunk
	\item Files from each chunk are encrypted using the AES algorithm and the corresponding random key. (AES is used because RSA implementations are not normally used to encrypt big files).
	\item The corresponding raw data files are deleted securely.
	\item A file is created with all encrypted file locations and corresponding random keys used for their AES encryption.
	\item This newly generated file is encrypted using RSA and the downloaded attacker public key.
	\item The raw text file with chunk info is then deleted.
	
\end{enumerate} 
\begin{figure*}[h]	
	\includegraphics[width=\textwidth]{timeline.png}
	\caption{Proof of Concept ransomware attack timeline}	
	\label{fig:AttackTimeline}
\end{figure*}
\subsection{Proof of Concept Network Architecture}
\label{sec:POCNetwork}
To deploy our system and infect it using the custom ransomware described in section~\ref{sec:CustomRansomwareDesign}, we created a network with three different machines as shown in Figure~\ref{fig:AttackNetwork}. 

\begin{figure}	
	\includegraphics[width=\columnwidth]{AttackNetwork.png}
	\caption{Proof of Concept Network Architecture}	
	\label{fig:AttackNetwork}
\end{figure}

\subsubsection{Attack Machine}
The attack machine is an Ubuntu 16 loaded with custom scripts and a webserver. The attack script is responsible for scanning the network for vulnerable machines and identifying their IP addresses. Once the IP list is compiled, it will begin the attack by sending mal-formed SMB packets and get access to the victim machine. The next step is to download the ransomware script from the attack machine to the victim machine and start running it. The machine will also run a webserver which hosts the ransomware script, encryption software, etc. along with a mechanism to generate, send and save public-private key pairs for each of the requested IP addresses. 
%Next section will describe the custom ransomware and its working.

\subsubsection{Victim Machine}
In this proof of concept, we are using the exploit for CVE-2017-0143, as described in section~\ref{sec:CustomRansomwareDesign} which targets Windows 7 machines. Hence, we choose a fresh installation of Windows 7 SP1 as the victim machine. The only additional software we install on it are the file monitoring and process monitoring agents (described in Section~\ref{sec:CognitiveAgents}). We also added some valuable files into some of the folders like Documents, Pictures, etc. 

\subsubsection{CCS Master Machine}
The core detection techniques are installed in the CCS master machine. Apart from this, we run Snort and snort agent in this machine (Snort could also be run on another machine because the Snort agent will take care of sending it to the CCS Master module). The CCS module includes two components. First, the Fuseki server loaded with the modified UCO Ontology and related SWRL rules. Standard reasoners are also part of the Fuseki server. The CCS module is the second component which extracts new information related to new attacks, host machine activities, etc. and updates the CCS dashboard dynamically.

\subsection{Proof of Concept Timeline}

We implemented the associated modules of CCS and created a network described in Section~\ref{sec:POCNetwork}. Our next goal is to check if we can detect the ransomware attack or not. Our knowledge graph is updated with common knowledge (the cyber-kill chain and knowledge mentioned in section~\ref{sec:ccs}) about cyber attacks. We demonstrate that even such simple information could be used for detecting newer attacks using this proof of concept. 
%For exampleithout the information ``Wannacry is a ransomware'' and ``Wannacry uses Malformed SMB packets to exploit''. 
Figure~\ref{fig:AttackTimeline} depicts the timeline of the attack performed and the actions from our CCS module. Each step in it is detailed below. 

\begin{itemize}
	\item {\bf Step 1: }Attacker performs a port scan on the victim machine using Nmap
	\item {\bf Step 2: }Snort detects port scan and reports to the CCS module. 
	\item {\bf Step 3: }Attacker uses the attack script to exploit the victim machine (using ``Eternal Blue'')
	\item {\bf Step 4: }Snort detects mal-formed SMB packets in the network.
	\item {\bf Step 5: }On successful exploitation, the attacker injects malware into the victim machine.
	\item {\bf Step 6: }The first attack script starts running the malware from the victim machine.
	\item {\bf Step 7: }As described in section~\ref{sec:CustomRansomwareDesign}, the malware now initiates downloads of encryption software, keys, etc. 
	\item {\bf Step 8: }Encryption software and keys get downloaded into the victim machine. The next task from the malware is the detection of sensitive files and their encryption using the downloaded tool.
	\item {\bf Step 9: }Snort detects downloads from unknown / potentially bad IP addresses.
	\item {\bf Step 10: }The file monitoring agent will detect new files downloaded from the Internet.
	\item {\bf Step 11: }While performing encryption, the malware modifies many sensitive files and the file monitoring agent reports it to the CCS module.
	\item {\bf Step 12: }When encryption is performed on larger files, the processor usage showed larger values and the process monitoring agent reports it to the CCS module.
\end{itemize}

In this test, the CCS Knowledge graph has the information about a new ransomware attack from textual sources. The new information from textual sources are \textit{``Wannacry is a ransomware''} and \textit{``Wannacry uses Malformed SMB packets to exploit''}. In the attack timeline, at Step 2, Snort will report a port scan which will be inferred by the CCS module as a potential reconnaissance step. At step 4, when Snort detects some mal-formed packets in the network, it is not conclusive to tell that it is an attack. It could just be some error packets. Subsequent steps, Step 9 through Step 12,  detect downloads from unknown sources, sensitive file modification, increased processor usage, etc. From the knowledge graph, we know the typical characteristics of a ransomware's ``Action-on-objective''; a lot of sensitive file modification, high processor usage, etc. However, these can also happen because of normal usage (for example, the user manually modifying files, downloading and running applications from the Internet, etc.). The presence of these indicators taken alone cannot be used to detect a ransomware. However, the CCS system already knows about a new ransomware attack using malformed SMB packets from textual sources and when combined with data from various sensors, the CCS system infers that the WannaCry attack is happening in the system and it is displayed on the dashboard as shown in Figure~\ref{fig:ccs_dashboard}. %detected indicators and 