\documentclass[11pt,oneside,fleqn,reqno,titlepage]{article}
\usepackage{amsmath}
\usepackage{amssymb}
\usepackage{amsthm}
\usepackage{natbib} \citeindextrue
\usepackage{comment}
\usepackage{url}
\usepackage{bbm}
\usepackage{bm}
\usepackage{multirow}
\usepackage{relsize}
\usepackage[pdftex]{graphicx}
\usepackage{color}




% Preamble =================================

\parskip 9pt
\textwidth 6.6in
\hoffset=-0.76in


\textheight 9.125in
%\headheight 0.25in
\headheight 0.30in
\headsep 0.375in
%\headsep 0.400in
\topmargin -0.5in
%\footheight 0.2in
%\footskip 0.5in
%\voffset=-0.50in



% To control the side margins.

%\evensidemargin = 0.0in
%\oddsidemargin = 0.0in
%\textwidth = 6.5in


\raggedbottom

%For better placement of figures
\setcounter{topnumber}{5}
\renewcommand{\topfraction}{.99}
\setcounter{bottomnumber}{5}
\renewcommand{\bottomfraction}{.99}
\setcounter{totalnumber}{10}
\renewcommand{\textfraction}{.0000001}
\renewcommand{\floatpagefraction}{.01}

%\DeclareMathOperator{\Prob}{Prob}


%Macro ==============================
%
% Define the sub-numbering within an eqnarray
%
\newcommand{\doublespace}{\addtolength{\baselineskip}{.25\baselineskip}}
\newcommand{\doublespacetwo}{\addtolength{\baselineskip}{1.0\baselineskip}}
\newcommand{\doublespacerestore}{\addtolength{\baselineskip}{1.25\baselineskip}}

%
% If you use doublespace, then switch to single space, and then you
% want to switch back, use this command (using /doublespace doesn't work.
%
\newcommand{\restoredoublespace}{\addtolength{\baselineskip}{1.\baselineskip}}
\newcommand{\regspace}{\addtolength{\baselineskip}{0.25\baselineskip}}
\newcommand{\singlespace}{\addtolength{\baselineskip}{-.5\baselineskip}}
\newcommand{\dense}{\renewcommand{\arraystretch}{1}}
\newcommand{\medium}{\renewcommand{\arraystretch}{1.5}}
\newcommand{\loose}{\renewcommand{\arraystretch}{2}}
\newcommand{\lindent}{\addtolength{\leftmargin}{\labelsep}}
\newcommand{\lnormal}{\addtolength{\leftmargin}{-1\labelsep}}
\newcounter{stepno}
\newcommand{\horizontalline}{\noindent \mbox{}\hrulefill\mbox{}}


\def\changemargin#1#2{\list{}{\rightmargin#2\leftmargin#1}\item[]}
\let\endchangemargin=\endlist

%
% Define the environments for definition, theorem, lamma etc.
%
\newtheorem{definition}{Definition}[section]
\newtheorem{defn}{Definition}[section]
\newtheorem{prty}{Property}[section]
\newtheorem{thm}{Theorem}[section]
\newtheorem{lemma}{Lemma}[section]
\newtheorem{corol}{Corollary}[section]
\newtheorem{corollary}{Corollary}
\newtheorem{prtn}{Proposition}[section]
\newtheorem{prop}{Proposition}[section]
\newtheorem{exam} {Example}[section]

\newtheorem{proposition}{Proposition}
\newtheorem{theorem}{Theorem}
\newtheorem{property}{Property}

\newcommand{\E}{\mathbb{E}}
\newcommand{\bn}{\begin{eqnarray}}
\newcommand{\en}{\end{eqnarray}}
\newcommand{\bns}{\begin{eqnarray*}}
\newcommand{\ens}{\end{eqnarray*}}
\newcommand{\defvarbegin}{\begin{quotation}\vspace{-15pt}\begin{tabbing}}
\newcommand{\defvarend}  {\end{tabbing}\vspace{-10pt}\end{quotation}}

\newcommand{\bnarray}{\begin{equation}\begin{array}{rcll}}
\newcommand{\enarray}{\end{array}\end{equation}}

\newcommand{\textwrap}{\parbox[t]{5.0in}}
\newcommand{\textwrapbig}{\parbox[t]{5.5in}}
\newcommand{\textwrapbigger}{\parbox[t]{5.75in}}
\newcommand{\textwrapsmall}{\parbox[t]{4.5in}}


\newcommand{\barr}{\begin{array}}
\newcommand{\earr}{\end{array}}


%\newcommand{\beginlist}{\begin{list}}
%\newcommand{\endlist}{\end{list}}

\newcounter{cnum}
\newenvironment{cond}[1]
{\begin{list}{(#1.\arabic{cnum})}{\usecounter{cnum}}}{\end{list}}


%
% Define the environment for steps in algorithm
%
\newcommand{\beginalg}{\setcounter{stepno}{1}
                \begin{list}{\bf Step~\arabic{stepno}}
                         {\usecounter{stepno}\settowidth{\labelwidth}{\bf Step~9m}
                \addtolength{\leftmargin}{2\parindent}}
                }
\newcommand{\eg}{\end{list}}



\newcommand{\imp}{{~\Rightarrow~}}
\newcommand{\goto}{\!\rightarrow\!}
\newcommand{\err}{{\varepsilon}}
\newcommand{\as}{~\mbox{a.s.}}
\newcommand{\sv}{\vspace{-0.1in}}
\newcommand{\eqnu}[2]{~~~~~~~~~~~~~~ \mbox{({#1}{#2})} }


\newcommand{\dt}{$\bullet ~$}
\newcommand{\remark}{{\noindent \bf Remark:}~}

%\begin{namelist}{xx} where {xx} gives the width.  This is used for defining variables.  After this, you use:
%
%  \item[Variable name]  Explanation.
%

\newcommand{\namelistlabel}[1]{\mbox{#1}\hfil}
\newenvironment{namelist}[1]{%
\begin{list}{}
        {
          \let\makelabel\namelistlabel
          \settowidth{\labelwidth}{#1}
          \setlength{\leftmargin}{1.1\labelwidth}
        }
     }{%
\end{list}}


\def \define{\begin{quote}\begin{itemize}}
\def \enddefine{\end{itemize}\end{quote}}

%Define a new dot product operator
\newcommand{\dotprod}{{\scriptscriptstyle \stackrel{\bullet}{{}}}}

\newlength{\boxedparwidth} \setlength{\boxedparwidth}{0.95\textwidth}
\newenvironment{boxedtext}%
  {\begin{center} \begin{tabular}{|@{\hspace{.15in}}c@{\hspace{.15in}}|}
                 \hline \\ \begin{minipage}[t]{\boxedparwidth}
                 \setlength{\parindent}{0.0in}}%
   {\end{minipage} \\ \\ \hline \end{tabular} \end{center}}
%Define my ``Example'' counter
\newcounter{example}
\setcounter{example}{0}

%Notation ============================

\def \rhooperator{\mathlarger{\mathlarger{\rho}}}
\newcommand{\Argmax}{{\rm Arg}\max}
\newcommand{\argmax}{{\rm arg}\max}
\newcommand{\Argmin}{{\rm Arg}\min}
\newcommand{\argmin}{{\rm arg}\min}


\def \atilde{{\tilde a}}
\def \btilde{{\tilde b}}
\def \ctilde{{\tilde c}}
\def \dtilde{{\tilde d}}
\def \etilde{{\tilde e}}
\def \ftilde{{\tilde f}}
\def \gtilde{{\tilde g}}
\def \htilde{{\tilde h}}
\def \itilde{{\tilde i}}
\def \jtilde{{\tilde j}}
\def \ktilde{{\tilde k}}
\def \ltilde{{\tilde l}}
\def \mtilde{{\tilde m}}
\def \ntilde{{\tilde n}}
\def \otilde{{\tilde o}}
\def \ptilde{{\tilde p}}
\def \qtilde{{\tilde q}}
\def \rtilde{{\tilde r}}
\def \stilde{{\tilde s}}
\def \ttilde{{\tilde t}}
\def \utilde{{\tilde u}}
\def \vtilde{{\tilde v}}
\def \wtilde{{\tilde w}}
\def \xtilde{{\tilde x}}
\def \ytilde{{\tilde y}}
\def \ztilde{{\tilde z}}

\def \Atilde{{\tilde A}}
\def \Btilde{{\tilde B}}
\def \Ctilde{{\tilde C}}
\def \Dtilde{{\tilde D}}
\def \Etilde{{\tilde E}}
\def \Ftilde{{\tilde F}}
\def \Gtilde{{\tilde G}}
\def \Htilde{{\tilde H}}
\def \Itilde{{\tilde I}}
\def \Jtilde{{\tilde J}}
\def \Ktilde{{\tilde K}}
\def \Ltilde{{\tilde L}}
\def \Mtilde{{\tilde M}}
\def \Ntilde{{\tilde N}}
\def \Otilde{{\tilde O}}
\def \Ptilde{{\tilde P}}
\def \Qtilde{{\tilde Q}}
\def \Rtilde{{\tilde R}}
\def \Stilde{{\tilde S}}
\def \Ttilde{{\tilde T}}
\def \Utilde{{\tilde U}}
%\def \Vtilde{{\tilde V}}
\def \Vtilde{\stackrel{\sim\;}{V}}
\def \Vtildenew{\stackrel{\sim\;}{V}}
\def \Wtilde{{\tilde W}}
\def \Xtilde{{\tilde X}}
\def \Ytilde{{\tilde Y}}
\def \Ztilde{{\tilde Y}}
\def \Omegatilde{{\tilde \Omega}}
\def \omegatilde{{\tilde \omega}}
\def \Omegaforecast {\Omegahat}
\def \nutilde{{\tilde \nu}}
\def \mutilde{{\tilde \mu}}
\def \pitilde{{\tilde \pi}}
\def \rhotilde{{\tilde \rho}}

\def \alphahat{\hat \alpha}

\def \ahat{\hat a}
\def \bhat{\hat b}
\def \chat{\hat c}
\def \dhat{\hat d}
\def \ehat{\hat e}
\def \fhat{\hat f}
\def \ghat{\hat g}
\def \hhat{\hat h}
\def \ihat{\hat i}
\def \jhat{\hat j}
\def \khat{\hat k}
\def \lhat{\hat l}
\def \mhat{\hat m}
\def \nhat{\hat n}
\def \ohat{\hat o}
\def \phat{\hat p}
\def \qhat{\hat q}
\def \rhat{\hat r}
\def \shat{\hat s}
\def \that{\hat t}
\def \uhat{\hat u}
\def \vhat{\hat v}
\def \what{\hat w}
\def \xhat{\hat x}
\def \yhat{\hat y}
\def \zhat{\hat z}

\def \Ahat{\hat A}
\def \Bhat{\hat B}
\def \Chat{\hat C}
\def \Dhat{\hat D}
\def \Ehat{\hat E}
\def \Fhat{\hat F}
\def \Ghat{\hat G}
\def \Hhat{\hat H}
\def \Ihat{\hat I}
\def \Lhat{\hat L}
\def \Nhat{\hat N}
\def \Phat{\hat P}
\def \Qhat{\hat Q}
\def \Rhat{\hat R}
\def \Shat{\hat S}
\def \Uhat{\hat U}
\def \Vhat{\hat V}
\def \What{\widehat W}
\def \Xhat{\hat X}

\def \rhohat{\hat \rho}
\def \nuhat{\hat \nu}
\def \nubar{\bar \nu}
\def \taubar{\bar \tau}
\def \thetabar{\bar \theta}
\def \deltabar{\bar \delta}
\def \deltahat{\hat \delta}
\def \Omegahat{\hat \Omega}
\def \omegahat{\hat \omega}
\def \Thetahat{\hat \Theta}



\def \abar{\bar a}
\def \bbar{\bar b}
\def \cbar{\bar c}
\def \dbar{\bar d}
\def \ebar{\bar e}
\def \fbar{\bar f}
\def \gbar{\bar g}
\def \hbar{\bar h}
\def \ibar{\bar i}
\def \jbar{\bar j}
\def \kbar{\bar k}
\def \lbar{\bar l}
\def \mbar{\bar m}
\def \nbar{\bar n}
\def \obar{\bar o}
\def \pbar{\bar p}
\def \qbar{\bar q}
\def \rbar{\bar r}
\def \sbar{\bar s}
\def \tbar{\bar t}
\def \ubar{\bar u}
\def \vbar{\bar v}
\def \wbar{\bar w}
\def \xbar{\bar x}
\def \ybar{\bar y}
\def \zbar{\bar z}


\def \Cbar{\bar C}
\def \Ebar{\bar E}
\def \Fbar{\bar F}
\def \Qbar{\bar Q}
\def \Rbar{\bar R}
\def \Sbar{\bar S}
\def \Xbar{\bar X}
\def \Vbar{\bar V}

\def \mubar{\bar \mu}
\def \sigmabar{\bar \sigma}

\def \Acal{{\cal A}}
\def \Bcal{{\cal B}}
\def \Ccal{{\cal C}}
\def \Dcal{{\cal D}}
\def \Ecal{{\cal E}}
\def \Fcal{{\cal F}}
\def \Gcal{{\cal G}}
\def \Hcal{{\cal H}}
\def \Ical{{\cal I}}
\def \Jcal{{\cal J}}
\def \Kcal{{\cal K}}
\def \Lcal{{\cal L}}
\def \Mcal{{\cal M}}
\def \Ncal{{\cal N}}
\def \Ocal{{\cal O}}
\def \Pcal{{\cal P}}
\def \Qcal{{\cal Q}}
\def \Rcal{{\cal R}}
\def \Scal{{\cal S}}
\def \Tcal{{\cal T}}
\def \Ucal{{\cal U}}
\def \Vcal{{\cal V}}
\def \Wcal{{\cal W}}
\def \Xcal{{\cal X}}
\def \Ycal{{\cal Y}}
\def \Zcal{{\cal Z}}

\def \Ycalbar {{\bar \Ycal}}

\def \Acalhat{{\hat \Acal}}
\def \Bcalhat{{\hat \Bcal}}
\def \Ccalhat{{\hat \Ccal}}
\def \Dcalhat{{\hat \Dcal}}
\def \Ecalhat{{\hat \Ecal}}
\def \Fcalhat{{\hat \Fcal}}
\def \Gcalhat{{\hat \Gcal}}
\def \Hcalhat{{\hat \Hcal}}
\def \Icalhat {\hat \Ical}
\def \Jcalhat{{\hat \Jcal}}
\def \Kcalhat{{\hat \Kcal}}
\def \Lcalhat{{\hat \Lcal}}
\def \Mcalhat{{\hat \Mcal}}
\def \Ncalhat{{\hat \Ncal}}
\def \Ocalhat{{\hat \Ocal}}
\def \Pcalhat{{\hat \Pcal}}
\def \Qcalhat{{\hat \Qcal}}
\def \Rcalhat {\hat \Rcal}
\def \Scalhat{{\hat \Scal}}
\def \Tcalhat{{\hat \Tcal}}
\def \Ucalhat{{\hat \Ucal}}
\def \Vcalhat{{\hat \Vcal}}
\def \Wcalhat{{\hat \Wcal}}
\def \Xcalhat{{\hat \Xcal}}
\def \Ycalhat{{\hat \Ycal}}
\def \Zcalhat{{\hat \Zcal}}

\def \Acaltilde{{\tilde \Acal}}


%\newcommand{\singlespace}{\addtolength{\baselineskip}{+0.5\baselineskip}}
%\newcommand{\doublespace}{\addtolength{\baselineskip}{.25\baselineskip}}



%\def\endremark{{\ \hfill\hbox{%
%      \vrule width1.0ex height1.0ex
%    }\parfillskip 0pt}\par}

%\newmdtheoremenv{theo}{Theorem}

%% The lineno packages adds line numbers. Start line numbering with
%% \begin{linenumbers}, end it with \end{linenumbers}. Or switch it on
%% for the whole article with \linenumbers.


%\journal{European Journal of Operations Research}

\begin{document}


\title{The Parametric Cost Function Approximation: A new approach for multistage stochastic programming}
\author{Warren B. Powell \\ Department of Operations Research and Financial Engineering \\
        Saeed Ghadimi \\ University of Waterloo}
\date{\today}

\maketitle

\clearpage

\begin{abstract}
The most common approaches for solving multistage stochastic programming problems in the research literature have been to either use value functions (``dynamic programming") or scenario trees (``stochastic programming") to approximate the impact of a decision now on the future.  By contrast, common industry practice is to use a deterministic approximation of the future which is easier to understand and solve, but which is criticized for ignoring uncertainty.  We show that a parameterized version of a deterministic optimization model can be an effective way of handling uncertainty without the complexity of either stochastic programming or dynamic programming.  We present the idea of a parameterized deterministic optimization model, and in particular a deterministic lookahead model, as a powerful strategy for many complex stochastic decision problems.  This approach can handle complex, high-dimensional state variables, and avoids the usual approximations associated with scenario trees or value function approximations.  Instead, it introduces the offline challenge of designing and tuning the parameterization.  We illustrate the idea by using a series of application settings, and demonstrate its use in a nonstationary energy storage problem with rolling forecasts.\\ \\
{\bf{Keywords:}} Stochastic Optimization, Policy Search, Stochastic Programming, Approximate Dynamic Programming, Simulation-Optimization, Parametric Cost Function Approximation
\end{abstract}

%A widely used heuristic method for solving multistage stochastic linear programming problems, as arises in many inventory and resource allocation problems, is to use a deterministic rolling horizon procedure which has been modified to handle uncertainty (e.g. buffer stocks, schedule slack). This approach has been criticized for its use of a deterministic approximation of a stochastic problem, which is the major motivation for using stochastic approximations of the future within the context of stochastic programming.  We recast this debate by identifying both deterministic and stochastic approaches as policies for solving a stochastic base model, which may be a simulator or the real world.  Stochastic lookahead models (stochastic programming) require a range of approximations to keep the problem tractable.  By contrast, so-called deterministic models are actually parametrically modified cost function approximations which use parametric adjustments to the objective function and/or the constraints.  These parameters are then optimized in a stochastic base model which does not require making any of the types of simplifications required by stochastic programming.  We formalize this strategy and describe simulation-based optimization algorithms to optimize the parameters.

%\end{abstract}
%We formalize this strategy and call it a parametric cost function approximation where parameterized modifications of costs or constraints can be adaptively learned using the principles of policy search. We will also present stochastic algorithms for tuning the policy in a simulator.

%%%%%%%%%%%%%%%%%%%%%%%%%%%
%%%%%%%%%%%%%%%%%%%%%%%%%%%

\thispagestyle{empty}

\singlespace

\doublespacerestore

\clearpage
\setcounter{page}{1}
\pagestyle{plain}


\section{Introduction}
There is an extensive history in the academic literature of solving sequential decision problems over time, under uncertainty, using the idea of approximating the future using sampled scenario trees, dating to the original paper by \cite{Da55}. Since this time thousands of papers have been written using this approach for making decisions under uncertainty.

A parallel literature has evolved estimating the value of being in a downstream state using statistical approximations estimated using machine learning that have evolved under names such as approximate (or adaptive) dynamic programming, neuro-dynamic programming, and (more recently) reinforcement learning (although this is an umbrella for an entire family of methods).  In the arena of stochastic linear programs, we can approximate the value of being in a state using Bender's cuts, a method known as stochastic dual dynamic programming.  Other methods have evolved under the heading of robust optimization (when used in the setting of sequential decision problems), which optimizes over an uncertainty set.

All of these methods are computationally demanding which has sharply limited their use in practice.  At the same time, when used in the context of complex operational problems (our primary focus in this paper), all of these methods are solving approximations of lookahead models which means that none of them are offering optimal or even asymptotically optimal policies (an optimal solution of an approximate problem is not an optimal policy).

By contrast, there has been a long history in industry of using deterministic optimization models to make decisions that are then implemented in a stochastic setting.  Grid operators use deterministic forecasts of wind, solar and loads to plan energy generation \citep{wallace2003stochastic}; utilities use rolling estimates to plan the storage of natural gas \citep{Lai2008}; airlines use deterministic estimates of flight times to schedule aircraft and crews \citep{lan2006planning}; and retailers use deterministic estimates of demands and travel times to plan inventories \citep{harrison1999multi}. There is an extensive literature using deterministic lookahead models for dynamic vehicle routing problems \citep{pillac2013}. All of us use the solution of deterministic shortest path problems produced by in-vehicle navigation systems to plan our route through stochastic, dynamic traffic networks.  These models have been widely criticized by the research community for not accounting for uncertainty, but this criticism ignores the creative use of parametric modifications that help these models account for uncertainty.

%We make the case that these previous approaches ignore the true problem that is being solved, which is always stochastic.  The so-called ``deterministic models'' used in industry are almost always parametrically modified deterministic approximations, where the modifications are designed to handle uncertainty.  Both the ``deterministic models'' and the ``stochastic models'' (formulated using the framework of stochastic programming) are examples of lookahead policies to solve a stochastic optimization problem with the goal of finding the best policy which is typically tested using a simulator, but which may be field tested in an online environment (the real world).

While these parameterized policies have been dismissed as industrial heuristics in the stochastic optimization community, we argue that they parallel parametric functions that have been studied in statistics for over 100 years and widely used in practice.  The field of parametric statistics requires that a knowledgeable human choose the structure of a parameterized model, which might be linear or nonlinear in the parameters (this includes neural networks), and then uses data and algorithms to choose the parameters.  Our use of parameterized deterministic models similarly requires a domain expert to design the parameterized optimization problem, after which algorithms have to be used to find the best values of the parameters.  With parametric statistics, we need a training dataset to estimate the parameters.  With a parameterized policy, we need a model of the problem that is captured in a dataset, although it is possible to do online training in the field.

In this paper, we characterize these modified deterministic models as {\it parametric cost function approximations} (CFAs) which puts them into the same category as other parameterized policies that have been used in computer science for solving simpler problems under the umbrella of ``policy search.'' Policy search is the same as stochastic search applied to the context of tuning parameterized policies for sequential decision problems.  The difference is that using a parameterized optimization model for a policy allows us to solve dramatically more complex problems than we can with simple analytical functions that are the focus of the policy search literature (\cite{Sutton2018}, \cite{Robots}, \cite{Levine2013}).  Instead of controlling a robot or computer game, parametric cost function approximations can be used to dispatch thousands of trucks, schedule an airline, manage a network of hydroelectric reservoirs, or plan energy generation for a power grid.

Our paper takes these ideas into the arena of multistage stochastic programming problems, which has been dominated in the research community by methods based on lookahead policies using scenario trees or, for special cases, Benders' decomposition (\cite{birge2011introduction,sen2014multistage,bayraksan2009assessing,zhao2013multi}).  Our use of modified linear programs is new to the policy search literature, where ``policies'' are typically parametric models such as linear models (``affine policies''), structured nonlinear models (such as $(s,S)$ policies for inventories) or neural networks (such as \cite{HanE16}). These more classical ``policies'' are limited to scalar or low-dimensional problems which could not be applied to the domain of high-dimensional resource allocation problems.

There are two dimensions to our approach:
\begin{description}
\item{1)} The design of the parameterized lookahead model which serves as the policy for making decisions.
\item{2)} The optimization of the parameters so that the policy performs as well as policy in expectation.
\end{description}
The process of designing the parameterization involves the same art as the design of any statistical model or parametric policy; it requires exploiting the structure of the problem which would be associated with a particular problem domain.  The optimization of the parameters is a more classical algorithmic challenge which we formulate and solve as a stochastic optimization problem (more precisely, a stochastic search problem).  Both dimensions are challenging, but the end result is a model that is no harder to solve than a classical deterministic lookahead model.  By contrast, policies based on stochastic programming produce stochastic lookahead models that are much harder to solve in the field than a parameterized deterministic model.  In addition, we argue below that a parameterized deterministic model may easily produce higher quality solutions.

We feel that classical stochastic programs based on scenario trees suffer from several limitations:
\begin{itemize}
    \item They are much harder to solve than deterministic lookahead models.
    \item Two-stage stochastic programs, which are the most widely used approach in stochastic programming, require a number of modeling approximations, including:
    \begin{itemize}
        \item The replacement of fully sequential, multistage stochastic decision problems with two-stage approximations.
        \item The need to use a sample of outcomes, and often a fairly small sample.
        \item The inability to model interactions between decisions and exogenous information (relevant in some applications).
    \end{itemize}
    \item There are many problems where the effect of uncertainty on the behavior of the policy is well known.  Yet, two-stage stochastic programs using modest numbers of scenarios do not provide any mechanism to capture this intuition.
\end{itemize}

Methods based on value function approximations, which includes SDDP (\cite{Pereira1991}, \cite{birge2011introduction}, \cite{Kall2009}), and approximate dynamic programming (\cite{PowellADP2011}, \cite{Bertsekas2017}) represent an alternative approach for optimizing over a stochastic lookahead model.  There are many complex problems where ADP is not a viable method for solving the full model, but might be a useful approach for solving an approximate stochastic lookahead model.  We revisit this issue later.

%Our interest is in problems for which these methods would not be appropriate.  Later we provide guidance regarding problems that lend themselves to value function approximations (of any form) versus a full optimization over a planning horizon, which is the focus of this paper.

This paper formalizes the idea that an effective way to solve certain classes of complex stochastic optimization problems is to shift the modeling of uncertainty from an approximate lookahead model to the stochastic base model, typically implemented as a simulator but which might also be the real world. Tuning a model in a stochastic simulator makes it possible to handle arbitrarily complex dynamics, avoiding the many approximations that are standard in stochastic programming. This strategy also means that we transition from methods that are hard to solve in the field, to methods that are (relatively) easy to solve in the field, but which require serious research in the laboratory to design and tune model parameterizations.

%The CFAs make it possible to exploit structural properties using domain knowledge.  For example, we could solve a stochastic shortest path problem minimizing the risk of arriving late to determine when to leave, or we can solve a deterministic approximation and then add in a buffer to leave early.  Airlines solve large deterministic models to schedule aircraft, but include buffers to account for weather delays (which have to be tuned through field experience).  Grid operators solve deterministic approximations of the future, but schedule reserve generation to handle uncertainties such as generator failures or, increasingly, variations in energy from wind and solar.

The parametric CFA makes it possible to incorporate problem structure for handling uncertainty. Some examples include:
\begin{itemize}
\item Supply chains handle uncertainty by introducing buffer stocks.
\item Hospitals can handle uncertainty in blood donations and the demand for blood by maintaining supplies of O-minus blood, which can be used by anyone.
\item People will use the estimated time for the shortest path when driving to a destination, but will then leave early to accommodate the uncertainty in the travel times.
\item Grid operators handle uncertainty in generator failures, as well as uncertainty in energy from wind and solar, by requiring reserve generator capacity.
\end{itemize}
Central to our approach is the ability to manage uncertainty by recognizing effective strategies for responding to unexpected events.  We would argue that this structure is apparent in many settings, especially in complex resource allocation problems. We offer that our approach represents an interesting, and very practical, alternative to classical stochastic programming.  We also argue that this approach is a perfectly valid form of stochastic optimization that may easily outperform methods based on solutions of approximate stochastic lookahead models.

%We illustrate the parametric CFA using a number of settings.  We then use an inventory problem that arises in energy storage to address a widely overlooked issue, which is the handling of time-dependent problems where inventory decisions have to be made in the presence of capacity constraints and rolling forecasts.  The presence of rolling forecasts is quite common in many resource allocation problems, but there is only a small number of papers that recognize the proper way to model rolling forecasts in the context of a sequential decision problem, and these are limited to very simple problems.

Our presentation is organized as follows.  Section \ref{sec:canonicalmodel} provides a canonical model for sequential decision problems and describes two strategies for designing policies: policy search, which searches over classes of policies (and policies within a class) to find the ones that work best over time (section \ref{sec:policysearchpolicies} provides more detail), and policies based on lookahead models which make good decisions by creating the best approximation of the impact of a decision now on the future (section \ref{sec:lookaheadpolicies} provides more detail).  Basic cost function approximations fall within the policy search class, but we will be considering hybrids that combine deterministic lookaheads (from the lookahead class) with parameterizations (from the policy search class).  Section \ref{sec:examplecfas} provides a series of examples of cost function approximations, divided between parameterizations of the objective function, and parameterizations of the constraints.  Then, section \ref{sec:cfaenergystorage} illustrates the idea of a parameterized, deterministic lookahead policy using the setting of a time-dependent energy storage problem using rolling forecasts, a common modeling device which has been largely overlooked in the research literature.  Section \ref{sec:closingremarks} offers some closing remarks.


%The modeling framework and an overview of the different classes of policies are given in Section 2. We then formally introduce the parametric CFA approach in Section 3. Algorithms for optimizing policy parameters together with their convergence properties and some theoretical results about structure of the optimization problem in the CFA approach are presented in Section 4. We then specialize the parametric CFA approach for an energy storage application and present some numerical experiments for solving this problem in Section 5. Finally, we conclude the paper in Section 6.
%%%%%%%%%%%%%%%%%%%%%%%%%%%
%%%%%%%%%%%%%%%%%%%%%%%%%%%
\section{Canonical model and solution strategies}
\label{sec:canonicalmodel}
We begin by writing a sequential decision problem as the sequence
\bns
(S_0,x_0,W_1, \ldots, S_t, x_t, W_{t+1}, \ldots, S_T)
\ens
where $S_t$ is our state at time $t$ which includes physical state variables $R_t$ (this captures physical resources such as inventories or the location on a graph), information $I_t$ (this could be prices, speeds, weather), and beliefs $B_t$ which captures what we know about uncertain quantities and parameters.  Let $C_t(S_t,x_t)$ be the cost we incur given what we know in $S_t$ and our decision $x_t$.  $W_{t+1}$ is information that arrives after we make a decision, which can depend on $S_t$ and/or $x_t$.  We make decisions $x_t$ using a method we call a {\it policy} that we write as $X^\pi(S_t)$.  The state variable evolves according to a transition function
\bns
S_{t+1} = S^M(S_t,x_t = X^\pi(S_t), W_{t+1}),
\ens
where the transition can span updating inventories, tracking the movement of a vehicle, updating prices and weather, or updating estimates or beliefs about uncertain quantities and parameters.

Our goal is to find a policy $\pi$ that solves
\begin{equation}
	\min_{\pi \in \Pi}  \E \left\{ \sum^T_{t=0} C_t(S_t, X^{\pi}_t(S_t)) \: \bigg| \: S_0  \right\}, \label{eq:baseobjective}
\end{equation}
where $S_{t+1} = S^M(S_t,X^\pi_t(S_t),W_{t+1})$ and given an information model for $(S_0, W_1, \ldots, W_T)$.  We will sometimes refer to \eqref{eq:baseobjective} as the base model, which will typically be a simulator but is sometimes the real world.

This canonical model is the foundation for a wide range of sequential decision problems.  \cite{Powell2019} describes two fundamental strategies for designing policies: policy search, where we search over classes of functions for making decisions to find the function that works the best over time, and policies based on lookahead approximations which combine the immediate cost of a decision plus an approximation of costs incurred as a result of the decision.

Each of these two strategies can be applied in two different ways, creating four classes of policies which encompass every possible method for making decisions \citep{Powell2019}:
\begin{itemize}
    \item Policy search strategies:
    \begin{description}
        \item{1)} Policy function approximations (PFAs) - These are analytical functions that map states to decisions.
        \item{2)} Cost function approximations (CFAs) - These are parameterized static or single-period optimization problems that yield a decision.
    \end{description}
    \item Policies based on lookahead approximations:
    \begin{description}
        \item{3)} Policies based on value function approximations (VFAs) - A value function approximation estimates the future cost from the next state we land in after making decision $x_t$ now.
        \item{4)} Policies based on approximations of direct lookahead models (DLAs) - Here we form an approximate model over some horizon, which we solve to make a decision now. Lookahead models can be divided into two broad classes:
        \begin{itemize}
            \item Deterministic lookahead models.
            \item Stochastic lookahead models.
        \end{itemize}
    \end{description}
\end{itemize}
We can also create hybrids, as we will below when we create a parameterized, deterministic lookahead policy that is a hybrid DLA/CFA.

Policy function approximations have received considerable attention in computer science under the banner of ``policy search''  \citep{ng2000pegasus,peshkin2000learning,hu2007evolutionary,mannor2003cross}.  Parameterized policies such as order-up-to inventory ordering policies have been studied since 1960 \citep{Clark1960}, along with a host of other specially structured policies in dynamic programming (\cite{putermanmarkov}, \cite{PowellRLSO}[Chapter 14]). Policy search uses classical stochastic search methods applied to parameterized analytical functions for making decisions, drawing on an extensive literature in derivative-based and derivative-free methods dating to 1951. We review this literature in section \ref{sec:algorithmsforpolicysearch}.

Policies based on value functions and value function approximations have been investigated extensively, building on the foundation of Bellman's equation \citep{putermanmarkov,Bertsekas2017} or Hamilton-Jacobi equations \citep{Kirk2012,stengel1986,Sontag1998,sethi2019,lewis2012}.  Value function approximations have been studied under names such as approximate dynamic programming \citep{PowellADP2011,Bertsekas2017} with applications in trucking \citep{SiDaGe09}, rail \citep{BeChPo2014}, and health \citep{Bartroff2010}; reinforcement learning  \citep{sutton1998reinforcement,Sutton2018} with applications in robotics \citep{SiBaPo04,Robots} and games \citep{Fu2017};  adaptive dynamic programming \citep{LewisVrabie2009,Wang2009}, neuro-dynamic programming \citep{Neuro_DP} and heuristic dynamic programming \citep{SiBaPo04}.  The stochastic programming community has developed the idea of approximating value functions using Bender's cuts under the heading of stochastic dual dynamic programming (SDDP) \citep{Pereira1991,Shapiro2014} with applications in hydroelectric planning \citep{Shapiro2011,Philpott2000}.

Direct lookahead policies represent an umbrella for a number of strategies that have been studied under names such as model predictive control \citep{camacho2013model} and stochastic programming \citep{birge2011introduction,Kall2009,bayraksan2009assessing,zhao2013multi,sen2014multistage}.  This approach has been developed in many books and thousands of papers, with applications that include unit commitment (\cite{jin2011modeling}), hydroelectric planning (\cite{carpentier2015managing}), and transportation (\cite{lium2009study}).

Another form of direct lookahead policy has evolved more recently using robust optimization which replaces scenario trees with uncertainty sets \citep{Ben-Tal2009a,Wiesemann2014}.  \cite{Ben-Tal2005} illustrates robust optimization as a lookahead policy for an inventory problem. See \cite{BertsimasBrown2011} for a review of other applications of robust optimization.

Cost function approximations, however, have been largely overlooked by the academic research community.  CFAs are parameterized optimization models, where the specific parameterization is designed (as would be the case with any parametric model in machine learning) to incorporate intuition into how uncertainty would affect the solution.  The use of parameterized optimization models has been widely used in industry, but in an ad hoc manner.

There is one example of a cost function approximation which has received extensive attention from the research literature.  The problem of finding the best performer out of a discrete set of alternatives (drugs, products, ads) is known as the multiarmed bandit problem.  A widely studied class of policies are known as upper confidence bounding, first introduced by \cite{Lai1985}.  A simple version (introduced by \cite{Ka93}) is given by
\bn
X^{UCB}(S_t|\theta) &=& \argmax_{x} \big(\mubar^n_x + \theta \sigmabar^n_x\big), \label{eq:ucbpolicy}
\en
where $\mubar^n_x$ is the current estimate of the performance of discrete alternative $x\in\{x_1, \ldots, x_M\}$ (e.g. the expected sales from advertising product $x$), and $\sigmabar^n_x$ is the standard deviation of $\mubar^n_x$ (note that $S_t = B_t = (\mubar^n_x, \sigmabar^n_x)$).  There is by now an extensive literature proving various regret bounds on the performance of the policy $X^{UCB}(S_t)$ (see e.g. \cite{Bubeck2012}).  This is particularly important for this paper, since $X^{UCB}(S_t)$ is, first and foremost, a class of parametric cost function approximation given that there is an imbedded ``$\argmax_x$'' within the policy.  The regret bounds that have been derived for UCB policies (of which \eqref{eq:ucbpolicy} is just one example) represents a rare set of provable bounds for the quality of a CFA policy.

PFAs and VFAs tend to be limited to problems that are relatively simple, or which enjoy special structure that can be exploited to estimate the required function (the policy or the value function).  Stochastic lookaheads have attracted considerable attention in the research literature, but relatively little of this work has made its way into practice. For this reason the most common approach used in practice for more complex decision problems is a deterministic approximation, which may be either static or single-period optimization models, or deterministic lookaheads.

In this paper, we want to shine a light on the power of using parameterized deterministic models, particularly for the complex problems that often arise in real applications.  We are not going to argue that this is a panacea that can replace all other methods, but we do feel that it is a powerful and overlooked strategy that belongs alongside widely studied (but rarely used) methods such as stochastic programming or approximate dynamic programming.

We next provide an overview of the policies based on policy search where we cover PFAs and static/single-period CFAs (section \ref{sec:policysearchpolicies}), and policies based on lookahead approximations where we cover policies based on VFAs and deterministic or stochastic DLAs (section \ref{sec:lookaheadpolicies}).  Ultimately we are going to take the idea of policy search that originated with PFAs, and apply it to the idea of parameterized optimization models with special emphasis on parameterized, deterministic lookaheads.  We are going to argue that this is often going to be a more effective strategy in practice for complex sequential decision problems than policies based on VFAs or stochastic DLAs.


\section{Policies based on policy search}
\label{sec:policysearchpolicies}
We begin with the principle of creating parameterized policies (broadly defined), which we divide between policy function approximations and cost function approximations.


\subsection{Policy function approximations (PFAs)}
\label{sec:pfas}
Policy function approximations are analytical functions that map states to decisions.  The functions can be lookup tables, parametric functions (linear or nonlinear), or nonparametric (in particular, locally parametric).  PFAs (using any of a wide range of approximation strategies) have been widely studied in the computer science literature under the umbrella of policy search (see e.g., \cite{Sutton2018}[Chapter 13], \cite{hadjiyiannis2011efficient,lillicrap2015continuous,levine2014learning}).  Although limited to relatively simple problems, the fundamental idea of policy search is quite powerful, and an idea that we are going to exploit.

An example of a PFA is a linear decision rule (also known as an affine policy) which can be written
\bn
X^\pi(S_t|\theta) = \sum_{f\in\Fcal} \theta_f \phi_f(S_t). \label{eq:linearpfa}
\en
where $\phi_f(S_t)$ is a feature drawn from the information in $S_t$ and $\theta_f$ is the coefficient for that feature.

We might wish to use a nonlinear model to choose the price $x^{bid}_t$ to bid for a set of keywords to maximize ad-clicks such as
\bns
X^\pi(S_t|\theta) = \frac{e^{\sum_{f\in\Fcal}\theta_f \phi_f(S_t)}}{1+e^{\sum_{f\in\Fcal}\theta_f \phi_f(S_t)}}.
\ens
More generally, we could represent a policy using a neural network, in which case $\theta$ might have many thousands (or millions) of parameters.

We have to choose the type of function $f\in\Fcal$ (for a neural network, $f$ would specify the network structure, number of layers and nodes per layer). For a given function type $f$ we have to choose $\theta\in\Theta^f$.

The choice of function $f$ is the art of policy search (just as it is with machine learning).  If this were a machine learning problem where we are given a training dataset $(x^n,y^n)_{n=1}^N$, we would be solving
\bn
\min_{(f\in\Fcal, \theta\in\Theta^f)}\sum_{n=1}^N(y^n - f(x^n|\theta))^2.\label{eq:machinelearning}
\en
Policy search for PFAs would be written similarly
\bn
\min_{\pi =(f,\theta) \in (\Fcal,\Theta^f)} \E\left\{\sum_{t=0}^T C_t(S_t,X^\pi_t(S_t|\theta^)) \big| S_0\right\}, \label{eq:objfunpolicies}
\en
where $S_{t+1}=S^M(S_t,x_t=X^\pi(S_t|\theta),W_{t+1}$) and where we are given a model for \linebreak $(S_0,W_1, \ldots, W_T)$.  We note that machine learning \eqref{eq:machinelearning} requires a training dataset while policy search for sequential decision problems in \eqref{eq:objfunpolicies} requires a model of the decision problem (costs, constraints, transition function and exogenous information model).

The comparison with machine learning hints at the limitation of PFAs: they only work for relatively simple decision problems.  For example, the response $y$ is typically scalar, although it might be a low dimensional vector.  We could never use a PFA to, say, schedule machines, dispatch a fleet of trucks, find a shortest path or optimize flows in a supply chain.





\subsection{Cost function approximations (CFAs)}
\label{sec:cfas}

The second class of policy is cost function approximations (CFAs) which are parametrically modified optimization problems.  Eventually we are going to include optimization problems that extend into the future, but for now we limit ourselves to static or single-period optimization models. We emphasize that with the notable exception of upper confidence bounding policies (equation \eqref{eq:ucbpolicy}) for multiarmed bandit problems, CFAs have received virtually no attention in the research literature.

We can create a CFA by modifying either the objective function or the constraints.  For this reason, we begin by defining
\bns
\Cbar^\pi_t(S_t,x_t|\theta) &=& \textwrap{the modified objective function as determined by the policy $\pi$, where $\theta$ represents the tunable parameters,}\\
\Xcal^\pi_t(\theta) &=& \textwrap{the modified set of constraints (that is, the feasible region) determined by policy $\pi$, with tunable parameters $\theta$.}
\ens
We might modify the objective function with a linear correction factor which we could write
\bn
\Cbar^\pi_t(S_t,x_t|\theta) = C(S_t,x_t) + \sum_{f\in\Fcal} \theta_f \phi_f(S_t,x_t). \label{eq:linearobjcorrection}
\en
As an illustration of how constraints can be modified, assume that we start with linear constraints
\bn
A_t x_t &=& b_t, \\
x_t & \leq & u_t,\\
x_t & \geq & 0.
\en
We might modify these using
\bn
A^\pi_t(\theta^a) \xtilde_t &   =  &  \theta^b \odot b_t + \theta^c, \label{eq:constraintcorrection1}\\
x_t                         & \leq &  u_t-\theta^u,\label{eq:constraintcorrection2}\\
x_t                         & \geq &  0 + \theta^\ell. \label{eq:constraintcorrection3}
\en
where $\theta^b \otimes b_t$ is the element by element product of the vector $b_t$ with the similarly dimensioned vector of coefficients $\theta^b$, plus a shift vector $\theta^c$.  We can enter schedule slack by parameterizing the matrix $A^\pi_t(\theta^a)$.  We then reduce the upper bounds $u_t$ by a shift vector $\theta^u$, and possibly raise the lower bounds by $\theta^\ell$.  Our constraints are now parameterized by the (possibly high-dimensional) vector $\theta = (\theta^a, \theta^b, \theta^c, \theta^\ell, \theta^u)$.

A parametric CFA policy can then be written
\bn
X^{CFA}(S_t|\theta) = \argmin_{x_t\in\Xcal^\pi_t(\theta)} \Cbar^\pi(S_t,x_t|\theta). \label{eq:cfa}
\en
We provide examples of CFAs in section \ref{sec:examplecfas}.


\subsection{Algorithms for policy search}
\label{sec:algorithmsforpolicysearch}
PFAs and CFAs both involve tuning a vector $\theta$.  Once we have the structure of the policy $f\in\Fcal$, which is typically chosen by a knowledgeable human guided by intuition, we tune $\theta$ using
\bn
\min_{\theta} \E\left\{\sum_{t=0}^T C_t(S_t,X^\pi_t(S_t|\theta)) \big| S_0\right\}. \label{eq:objfuntheta}
\en
The tools for optimizing the parameters using \eqref{eq:objfuntheta} fall under the broad umbrella of stochastic search which can be approached using both derivative-based algorithms, with a literature that dates to \cite{RoMo51}, and derivative-free algorithms, with a literature that dates to \cite{Box1951}.

The derivative-based stochastic optimization literature is extensive, beginning with the body of research building off of \cite{RoMo51} in the 1950s and later (see, e.g., \cite{Dv56}), initially for unconstrained problems.  A separate literature evolved in the context of constrained stochastic gradient problems (\cite{Sh79}, \cite{Ermoliev1983}).  In addition, the simulation-optimization community (see \cite{fu2015handbook}) has developed powerful tools for taking derivatives of simulations (see \cite{glasserman1991gradient}, \cite{ho1992discrete}, \cite{kushner2003stochastic}, \cite{cao2008stochastic});  a nice tutorial is given in \cite{chau2014simulation}. Much of this literature focuses on derivatives of discrete event simulations, but there is an equally extensive literature on methods based on numerical derivatives such as SPSA (\cite{spall2005introduction,NesSpo17,GhaLan12}).  More recently is work on derivatives of parameterized policies for discrete dynamic programs from the reinforcement learning literature under the umbrella of the policy gradient theorem (\cite{SuMcSi2000}, \cite{Sutton2018}[Chapter 13]).

There is a parallel literature in derivative-free algorithms for stochastic search which is equally extensive.  This literature spans active learning problems \citep{Settles2010}, multiarmed bandit problems \citep{gittins2011}, and optimal learning \citep{PoRy2012}. See \cite{PowellRLSO}[Chapter 7] for an overview of this rich field.

While both PFAs and CFAs require parameter tuning, the characteristics of the tuning problems for PFAs and CFAs tend to be quite different.  It is well known that scaling is a major issue in stochastic search.  Consider the linear decision rule in equation \eqref{eq:linearpfa}.  The scaling of each coefficient $\theta_f$ depends heavily on the characteristics of the feature $\phi_f(S_t)$.  By contrast, the coefficients $\theta$ used in a parametric CFA tend to be scaled by the structure of the deterministic optimization model.  In section \ref{sec:cfaenergystorage}, we demonstrate a parametric CFA for a stochastic inventory control problem where the optimal coefficients are all equal to 1.0 if the forecasts are perfect.  Given imperfect forecasts, the optimal coefficients all appear to be in the interval $[0,2]$.



\section{Policies based on lookahead approximation}
\label{sec:lookaheadpolicies}
%CFAs are widely used in industry for complex problems such as scheduling energy generation or planning supply chains, but they have not been studied formally in the research literature.
    %In special cases, PFAs and CFAs may produce optimal policies, although generally we are looking for the best within a class.

The second strategy for creating policies is to construct policies based on approximations of the downstream impact of a decision $x_t$ made while in state $S_t$.  An optimal policy can be written
\bn
X^*_t(S_t) = \argmin_{x_t \in {\cal X}_t} \left(C_t(S_t,x_t) + \E \left\{ \min_{\pi\in\Pi} \E \left\{\sum_{t'=t+1}^T C_{t'}(S_{t'},X^\pi_{t'}(S_{t'})) \big| S_{t+1}\right\} \big| S_t,x_t \right\}\right)\hspace{-.04in}. \label{eq:optimalpolicylookahead}
\en
Equation \eqref{eq:optimalpolicylookahead} is called a lookahead policy.  Not surprisingly this is computationally intractable for any realistic problem (this includes all the problems that we are interested in).

Just as we divided the policy search strategy into two classes (PFAs and CFAs), there are two classes of policies that we can use to approximate equation \eqref{eq:optimalpolicylookahead}.  These are policies based on value function approximations (VFA policies) and policies based on approximations of the direct lookahead approximations (DLA policies).  We describe these in more detail in sections \ref{sec:vfapolicy} and \ref{sec:dlapolicy} below.




\subsection{Policies based on value function approximations}
\label{sec:vfapolicy}
Equation \eqref{eq:optimalpolicylookahead} is basically Bellman's equation although it is more conventional to write
\bn
X^*_t(S_t) = \argmin_{x_t \in {\cal X}_t} \big(C_t(S_t,x_t) + \E \left\{V_{t+1}(S_{t+1})|S_t,x_t\right\}\big) \label{eq:optimalpolicylookahead2}
\en
where
\bn
V_{t+1}(S_{t+1}) = \min_{\pi\in\Pi} \E \left\{\sum_{t'=t+1}^T C_{t'}(S_{t'},X^\pi_{t'}(S_{t'})) \big| S_{t+1}\right\}. \label{eq:optimalpolicylookahead3}
\en
Alternatively, we can write the expression for value functions recursively using
\bn
V_t(S_t) = \min_{x_t}\big(C_t(S_t,x_t) + E  \left\{V_{t+1}(S_{t+1})|S_t,x_t\right\}\big).\label{eq:optimalpolicylookahead4}
\en
Equation \eqref{eq:optimalpolicylookahead4} is the most common way of writing Bellman's equation, but it is mathematically equivalent to equation \eqref{eq:optimalpolicylookahead} assuming that we can compute $V_t(S_t)$ using either \eqref{eq:optimalpolicylookahead3} or \eqref{eq:optimalpolicylookahead4}, which is never the case for the problems that we are interested in.

When $x_t$ is a vector, it is customary to eliminate the expectation in \eqref{eq:optimalpolicylookahead2} and \eqref{eq:optimalpolicylookahead4} by using the post-decision state variable $S^x_t$ (see \cite{PowellADP2011} and \cite{Shapiro2011}).  We then replace the post-decision value function $V^x_t(S^x_t)$ (which we could never compute) with an approximation $\Vbar^x_t(S^x_t)$, giving us the policy
\bn
X^{VFA}_t(S_t) = \argmin_{x_t}\big(C_t(S_t,x_t) + \Vbar^x_t(S^x_t)\big),\label{eq:optimalpolicylookahead5}
\en
where $\Vbar^x_t(S^x_t)$ might be a linear model, separable piecewise linear functions, or Benders cuts.

While this approach has attracted considerable attention in the literature (see e.g., \cite{powell2004learning,PowellADP2011,bertsekas2011dynamic, Sutton2018}), it is limited to a surprisingly narrow set of problems.  For example, while SDDP has been widely studied in the stochastic programming community, applications are generally limited to fairly simple resource allocation problems such as hydroelectric planning problems \citep{Shapiro2011,Philpott2012} which is known as a single layer resource allocation problem (water is the only resource).  For example, you could never use SDDP for dynamic vehicle routing problems, complex inventory planning problems, or dynamic shortest path problems.

As an indication of how easy it is to ``break'' approximate dynamic programming, ADP (or SDDP) is very effective for solving a blood management problem as long as the surgeries requiring blood having to be completed at a particular point in time (this might be some time during a week).  If we have elective surgeries that can be delayed, creating effective value functions becomes dramatically more difficult.

%Other examples of single layer resource allocation problems arise in the optimizing of fleets of trucks moving loads that must be moved at a single point in time \citep{SiDaGe09}, fleets of locomotives moving trains that also have to be moved at a single point in time \citep{BeChPo2014}.  If we allow the loads of freight or trains to be delayed, the challenge of estimating value function approximations becomes much more complicated.

Even more complex problems include dynamic vehicle routing, where you have to optimize the movement of vehicles and the timing of deliveries, or scheduling machines to complete a series of tasks.  In short, value function approximations are effective when there is structure (such as linearity or convexity) that can be exploited.  Later, we are going to illustrate a CFA in the solution of a time-dependent inventory problem with rolling forecasts.  Rolling forecasts create complex, high-dimensional state variables that are completely intractable using methods based on value function approximations.




\subsection{Policies based on direct lookahead approximations}
\label{sec:dlapolicy}
The most commonly used approach used to solve complex time-dependent problems is to solve an approximate lookahead model on a rolling basis.  Here we first create an approximate sequential decision problem that we are going to use as an approximate model of the future.  We represent this approximate sequential decision problem as
\bns
(\Stilde_{tt},\xtilde_{tt},\Wtilde_{tt}, \ldots, \Stilde_{tt'},\xtilde_{tt'},\Wtilde_{t,t'+1}, \ldots),
\ens
where $\Stilde_{tt'}$, $\xtilde_{tt'}$ and $\Wtilde_{t,t'+1}$ are approximations of $S_{t'}$, $x_{t'}$ and $W_{t'}$ for a decision we are making at time $t$.

While there are a variety of strategies for approximating lookahead models, the two that have received the most attention are:
\begin{itemize}
    \item{Deterministic lookaheads} - This is the approach most widely used in practice, but it has a substantial academic following under the umbrella ``model predictive control.'' Using a deterministic lookahead model reduces  equation \eqref{eq:optimalpolicylookahead} to
    \bn
    X^{DLA}_t(S_t) &=& \argmin_{\xtilde_{tt}} \left(C(\Stilde_{tt},\xtilde_{tt}) + \min_{\xtilde_{t,t+1},\ldots,\xtilde_{t,t+H}} \sum_{t'=t+1}^{t+H} C(\Stilde_{tt'},\xtilde_{tt'})\right) \nonumber\\
               &=& \argmin_{\xtilde_{tt},\xtilde_{t,t+1},\ldots,\xtilde_{t,t+H}}  \sum_{t'=t}^{t+H} C(\Stilde_{tt'},\xtilde_{tt'}). \label{eq:deterministicdla}
    \en
    Equation \eqref{eq:deterministicdla} is so widely used it is known under a number of names including rolling horizon procedure, receding horizon procedure, model predictive control, or deterministic direct lookahead (\cite{sethi1991theory,camacho2013model}, \cite{PowellRLSO}[Chapter 19]).
%
    \item{Stochastic programming} - First introduced by  \cite{Da55}, the most common approach is to replace the fully sequential decision problem in the future with a two-stage approximation which means our sequence of decisions and information looks like
    \bns
    \big(\xtilde_{tt}, (\Wtilde_{t,t+1}(\omega),\Wtilde_{t,t+2}(\omega), \ldots, \Wtilde_{t,t+H}(\omega)),(\xtilde_{t,t+1}(\omega),\xtilde_{t,t+1}(\omega),\ldots, \xtilde_{t,t+H}(\omega))\big).
    \ens
    This model assumes we make a single decision now, $\xtilde_{tt}$, then observe a complete sample path of realizations:
    \bns
    (\Wtilde_{t,t+1}(\omega),\Wtilde_{t,t+2}(\omega), \ldots, \Wtilde_{t,t+H}(\omega)),
    \ens
    and then make a complete set of decisions for each sample path $\omega$:
    \bns
    (\xtilde_{t,t+1}(\omega),\xtilde_{t,t+1}(\omega),\ldots, \xtilde_{t,t+H}(\omega)).
    \ens
    This approach insures that the decision now, $x_t=\xtilde_{tt}$, does not depend on what outcome happens in the future, but future decisions, $\xtilde_{tt'}$ for $t' > t$ are allowed to see the entire history of future information.  Next we create a set of scenarios of $\Wtilde_{tt'}(\omega)$ that we denote $\Omegatilde_t$.  Our policy is then written
    \bn
    \hspace{-.15in}X^\pi_t(S_t) = \argmin_{\xtilde_{tt},(\xtilde_{tt'}(\omega))_{t'=t+1}^{t+H},\forall \omega\in\Omega_{[t,t+H]}} \left(\Ctilde(\Stilde_{tt},\xtilde_{tt}) + \hspace{-.05in} \sum_{\omega\in\Omega_[t,t+H]} \Ctilde(\Stilde_{tt'}(\omega),\xtilde_{tt'}(\omega))\right)\hspace{-.04in}. \label{eq:twostagepolicy}
    \en
    The optimization problem in \eqref{eq:twostagepolicy} is typically around $|\Omegatilde_t|$ times bigger than the deterministic problem in equation \eqref{eq:deterministicdla}, but at least it is solvable.
\end{itemize}
The deterministic lookahead, which is also known as ``model predictive control'' in the optimal control literature, is largely dismissed by the stochastic optimization community as little more than a ``deterministic approximation.''  By contrast, the policy based on the two-stage stochastic program in \eqref{eq:twostagepolicy} has appeared in thousands of academic publications, often without recognizing that it is a suboptimal policy for the original optimization problem in equation \eqref{eq:baseobjective}.  See \cite{PowellUnitComm2019} for a discussion of the limitation of scenario trees for the stochastic unit commitment problem.

\subsection{Discussion}
Policies based on solving lookahead models depend on the accuracy of the model to produce good decisions.  The problem is that the solution of full, multistage stochastic decision problems is inherently intractable, forcing the use of very strong approximations such as two-stage stochastic programs.

We propose to extend the idea of a parametric cost function approximation, which we first introduced for state or single-period problems (in section \ref{sec:cfas}), to deterministic lookahead models.  Then, instead of depending on developing an accurate stochastic lookahead model, we exploit structure in the problem to parameterize the deterministic lookahead model to produce behaviors that make the optimal solution more robust.  We then depend on the tuning using a realistic, stochastic simulator (equation \eqref{eq:baseobjective}) to produce the best values of the parameters.

Some advantages of this approach include:
\begin{itemize}
    \item The tuning is done in a realistic simulator (equation \eqref{eq:baseobjective}) that does not need to make simplifications such as an exogenous information process that is independent of decisions.
    \item The simulator can capture any level of detail in the dynamics of the system.
    \item The parameterization of the policy can exploit structure and the modeler's intuition about how uncertainty is likely to affect the solution (an assumption that is made in virtually all parametric models in machine learning).
    \item The resulting policy will generally have the same computational demands as a classical (unparameterized) deterministic lookahead, which would be much easier than solving any stochastic lookahead model.
%    \item Particularly important is that we never have to worry about whether the approach will be used in practice; there are many settings where a parameterized deterministic lookahead is already being used, but may be improved with either a better parameterization, in addition to tuning of the parameters.
    \item This idea has been widely used in industry in an ad-hoc manner.  Specifically, industrial applications will insert parameters without a) recognizing that they are creating a class of policy that is a solution (albeit a suboptimal one) to the optimization problem \eqref{eq:baseobjective} and b) without recognizing that the parameters need to be tuned using the framework of \eqref{eq:baseobjective}.
\end{itemize}
At the same time, we have to consider:
\begin{itemize}
    \item We need to have the intuition into how uncertainty changes the solution we would get from a deterministic lookahead model.
    \item Despite over 60 years of research into stochastic search, parameter tuning remains difficult, but the difficult part is in the research lab where it belongs, not in the field.
\end{itemize}

A major goal of this paper is bring to the attention of the research community in stochastic optimization that a parameterized deterministic lookahead is as valid an approach to the stochastic optimization in \eqref{eq:baseobjective} as any policy based on a stochastic lookahead.  We believe that there are problems where the parameterized deterministic lookahead, in addition to its computational advantages, may outperform a two-stage stochastic program in terms of its ability to solve \eqref{eq:baseobjective}.




\section{Examples of cost function approximations}
\label{sec:examplecfas}
Recall that there are two ways to parameterize an optimization problem: in the objective function, as we did in equation \eqref{eq:linearobjcorrection}, and in the constraints as we did in equations \eqref{eq:constraintcorrection1} - \eqref{eq:constraintcorrection3}.  In this section we are going to provide more concrete examples.



\subsection{CFAs for dynamic assignment problems}

The truckload trucking industry matches drivers to loads over time.  Let
\bns
x_{td\ell} &=& \textwrap{1 if we assign driver $d$ to load $\ell$ at time $t$, 0 otherwise,}\\
c_{td\ell} &=& \textwrap{the contribution of assigning driver $d\in\Dcal_t$ to load $\ell\in\Lcal_t$ at time $t$, including the revenue generated by the load, the cost of moving empty to the load, as well as penalties for late pickup or delivery.}
\ens
We can perform a myopic assignment of drivers to loads by solving
\bn
X^{Assign}(S_t) = \argmax_{x_t} \sum_{d\in\Dcal_t} \sum_{\ell\in\Lcal_t} c_{td\ell} x_{td\ell}. \label{eq:cfadriver}
\en
A potential problem with a myopic policy is that there may be loads that are not assigned and are then held in the hope that a driver may be found to move the load at a later time.  However, the load may be in a location where we do not traditionally have drivers.  We can create an artificial incentive.  Let
\bns
\tau_{t\ell} &=& \textwrap{the amount of time that load $\ell$ has been held at time $t$.}
\ens
Now consider the following modified optimization problem
\bn
X^{Assign}(S_t|\theta) = \argmin_{x_t} \sum_{d\in\Dcal_t} \sum_{\ell\in\Lcal_t} (c_{td\ell} -\theta \tau_{t \ell})x_{td\ell}. \label{eq:cfadriver2}
\en
$X^{Assign}(S_t|\theta)$ is now a parameterized cost function approximation with a modified objective function.


\subsection{A dynamic shortest path problem}
\label{sec:dynamicshortestpath}
Everyone is familiar with the process of navigation systems repeatedly solving shortest path problems to a destination as it receives updates to estimates of travel times around the network.  This is, of course, a fully sequential decision problem that can be modeled as a dynamic program.  We can model the problem as a sequential decision problem.

Let $R_{t}$ be the location of the traveler at time $t$, and let $\cbar_{tij}$ be our estimate of the cost of traversing link $(i,j)$ given what we know at time $t$.  The estimates $\cbar_t$ evolve over time according to
\bn
\cbar_{t+1,ij} = \cbar_{t,ij} + \delta \cbar_{t+1,ij}. \label{eq:shortestpathdynamictransition1}
\en
where $\delta \cbar_{t+1,ij}$ is the change in the estimate of $\cbar_{tij}$ given new observations of traffic.

We assume that at each time $t$ we are at an intersection where we have to make a decision given by
\bns
x_{tij} &=& \left\{\begin{tabular}{cl} 1 & \mbox{if we traverse link $i$ to $j$ when we are at $i$ at time $t$,}\\
                                      0 & \mbox{otherwise.} \end{tabular} \right.
\ens
We make this decision using a policy $X^\pi(S_t)$, where the state $S_t$ is given by
\bns
S_t = (R_t, \cbar_t).
\ens
Models of shortest path problems typically overlook the need to include the vector of estimates $\cbar_t$ in the state variable.  This is precisely why we cannot solve this problem (at least not optimally) using classical shortest path algorithms.

Our challenge is to then find the best policy $X^\pi(S_t)$ that solves
\bn
\min_\pi F^\pi(\theta) = \E \left\{\sum_{t=0}^T \sum_{(i,j)} \chat_{tij}X^\pi(S_t|\theta)|S_0\right\}, \label{eq:shortestpathdynamicobjective}
\en
where $\chat_{tij}$ is the actual cost we experience traversing link $(i,j)$ at time $t+1$.

A natural strategy is to fix the vector of estimates of link costs $\cbar_t$ and solve a shortest path problem to the destination, updating the shortest path as $\cbar_t$ evolves to $\cbar_{t+1}$ (and the traveler transitions to a new node).  This is a lookahead policy based on a lookahead model that uses fixed estimates $\cbar_t$ rather than modeling their stochastic evolution.
%Figure \ref{fig:cfadynamicshortestpath} illustrates the use of a lookahead policy for this problem.
%\begin{figure}[tb]
%\begin{center}
%\begin{tabular}{c}
%\includegraphics[width=5.0in]{cfadynamicshortestpath}
%\end{tabular}
%\caption{Illustration of rolling solution of deterministic shortest path problems using costs $\ctilde^\pi_t(\theta)$.}
%\label{fig:cfadynamicshortestpath}
%\end{center}
%\end{figure}

The question is: Can we do better?  A limitation of the classical approach of solving sequences of deterministic lookaheads is that it fails to recognize that some links can have long tails, which introduces the risk of arriving late.  An alternative is to replace $\cbar_{tij}$ with the $\theta$-percentile of the distribution for each link.  Let
\bns
\cbar^\pi_{t,ij}(\theta) &=& \textwrap{the $\theta$-percentile of the travel time for link $(i,j)$ given our estimate at time $t$.}
\ens
This means we still have a deterministic shortest path problem, but now we have to tune $\theta$ using \eqref{eq:shortestpathdynamicobjective} using our tools from stochastic search.





\begin{comment}

\subsection{Dynamic trading policy}
\label{sec:cfadynamictradingpolicy}
We are going to design a policy for trading financial assets.  Let
\bns
\Ical &=& \textwrap{the set of stocks we may hold a position in, with $i=0$ referring to cash,}\\
R_{ti} &=& \textwrap{our position (in shares) in a particular stock $i\in\Ical$, where $R_{ti}$ can be either positive (for a long position) or negative (for a short position), and where $R_{t,0}$ is the amount in cash,}\\
R_t &=& (R_{ti})_{i\in\Ical}.
\ens
Other information variables are
\bns
p_{ti} &=& \textwrap{the price of stock $i$,}\\
p_t &=& (p_{ti})_{i\in\Ical},\\
f_{tt'i} &=& \textwrap{the forecast, generated at time $t$, of the price of stock $i$ at time $t'$ over a horizon $t' = t, \ldots, t+H$,}\\
f_t &=& (f_{tt'i})_{i\in\Ical, t'=t, \ldots, t+H.}
\ens
Our decision variable is
\bns
x_{ti} &=& \textwrap{the number of shares that we trade for each of the stocks. We use $x_{ti}>0$ to represent the number of shares we buy for stock $i$, and  $x_{ti}<0$ to represent a selling decision.}
\ens
The decision is constrained by the requirement that we have enough cash on hand to finance the purchasing decisions, given by
\bns
\sum_{i=1}^M x_{ti}p_{ti} \le R_{t,0}.
\ens
We can now formulate a trading policy
\bn
X^\pi_t(S_t|\theta) \hspace{-.10in}&=&\hspace{-.10in} \argmax_{x_t} \left(\sum_{i=1}^M \left((R_{ti}+x_{ti})(\tilde{f}_{ti}(\theta)-p_{ti})-c^{trans}|x_{ti}|p_{ti}\right) - \rho(R_t+x_t)\right), \nonumber \\
& & \label{eq:tradingpolicy}
\en
where $\tilde{f}_{ti}(\theta)=\sum_{s=1}^H\theta_{s}f_{t,t+s,i}$ represents an overall prediction of the future price using all available forecasts with different horizons and a tunable parameter vector $\theta=(\theta_1,\dots,\theta_H)$. This policy maximizes a utility function that balances the trade-off between return and risk.

%It can be seen that for the risk function \eqref{eq:trading_risk}, the policy can be computed efficiently by solving a convex optimization problem.
\end{comment}



\section{An energy storage example with rolling forecasts}
\label{sec:cfaenergystorage}
One of the most overlooked modeling issues in operations research is the proper handling of rolling forecasts.  We use the setting of an energy storage system (depicted in figure \ref{fig:energysystemnew}) which draws energy from a wind farm or the power grid to serve a time-varying load, with a finite capacity storage device (and fixed transmission constraints) to help smooth the variations.
\begin{figure}[tb]
\begin{center}
    \includegraphics[width=4.5in]{energysystemnewnoimage}
    \caption{Energy storage system, including a renewable source (wind), energy from the grid at real-time prices, battery storage, and a load.}
    \label{fig:energysystemnew}
\end{center}
\end{figure}

Our energy system has some important characteristics that make it unusually difficult as a stochastic optimization problem:
\begin{itemize}
    \item The energy demands follow a highly time-dependent diurnal pattern (see figure \ref{fig:cfahourofday}(a)).
    \item The energy from wind is highly stochastic.  We have rolling forecasts, updated every hour, but these rolling forecasts evolve considerably over time, as indicated in figure \ref{fig:cfahourofday}(b).  Our rolling wind forecast data was obtained courtesy of PJM Interconnections.
    \item There is unlimited power available from the grid, but at highly stochastic prices.  We can buy from, and sell to, the grid.
    \item The battery has fixed capacity, and the transmission lines are also capacitated, which limits our ability to transmit and store power.  For this reason, the ability to anticipate surges and dips in wind energy requires that we be able to plan into the future.
\end{itemize}

\begin{figure}[b]
\center{
    \begin{tabular}{cc}
    \includegraphics[width = 2.70in]{cfaloadhourofday} &
    \includegraphics[width = 3.40in]{cfarollingforecast} \\
                    (a)                  &                    (b)
    \end{tabular}
    \caption{(a) Energy load by hour of day and (b) rolling forecast, updated hourly.}
    \label{fig:cfahourofday}
    }
\end{figure}

We use a {\it martingale model of forecast evolution} (MMFE) \citep{Heath1994,Graves1986} where forecasts (for energy from the wind farm and the demand) evolve according to
\bn
f^E_{t+1,t'} = f^E_{tt'} + \varepsilon^E_{t+1,t'}, \label{eq:rollingforecastE}\\
f^D_{t+1,t'} = f^D_{tt'} + \varepsilon^D_{t+1,t'}. \label{eq:rollingforecastD}
\en
where $\varepsilon^E_{t+1,t'} \sim N(0,\sigma^2_E)$ and $\varepsilon^D_{t+1,t'} \sim N(0,\sigma^2_D)$ represents the exogenous change in the forecast of energy from the wind farm and the demand for time $t'$.


\subsection{A model of the energy system}

Our model consists of five elements: state variables $S_t$, decision variables $x_t$, exogenous information variables $W_{t+1}$, the transition function $S_{t+1}=S^M(S_t,x_t,W_{t+1})$, and the objective function.  These are given below:

{\noindent \bf State variables:}- The state of the system at time $t$ is all the information we need to model our system from time $t$ onward, which means the information need to compute costs and constraints, making a decision, and compute the transition function.  For the energy problem this information is:
\bns
D_t   &=& \textwrap{Demand (``load'') for power during hour $t$.}\\
E_t   &=& \textwrap{Energy generated from renewables (wind/solar) during hour $t$.}\\
R_t   &=& \textwrap{Amount of energy stored in the battery at time $t$.}\\
u_t   &=& \textwrap{Limit on how much generation can be transmitted at time $t$ (this is known in advance).}\\
p_t   &=& \textwrap{Price to be paid for energy drawn from the grid at time $t$.},\\
f^D_{tt'} &=& \textwrap{Forecast of $D_{t'}$ made at time $t$.}\\
f^E_{tt'} &=& \textwrap{Forecast of $E_{t'}$ made at time $t$.}
\ens
These variables make up our state variable:
\bns
S_t = (R_t, D_t, E_t,  (f^D_{tt'})_{t'\geq t}, (f^E_{tt'})_{t'\geq t}).
\ens
Rolling forecasts are widely used in dynamic models, but the recognition that the forecast itself belongs in the state variable has been recognized by only a small handful of authors, including \cite{Chen1999},\cite{Iida2006} and \cite{Lai2008}.



\noindent {\bf Decision variables:} - These are the flows between each of the elements of our energy system:
\bns
x_{t}      &=& \textwrap{Planned generation of energy during hour $t$ which consists of the following elements:}\\
x^{ED}_{t} &=& \textwrap{flow of energy from wind to demand,}\\
x^{EB}_{t} &=& \textwrap{flow of energy from wind to battery,}\\
x^{GD}_{t} &=& \textwrap{flow of energy from grid to demand,}\\
x^{GB}_{t} &=& \textwrap{flow of energy from grid to battery,}\\
x^{BD}_{t} &=& \textwrap{flow of energy from battery to demand.}
\ens
We would normally write out the constraints that these flows have to satisfy.  These consist of the flow conservation constraints, as well as upper bounds due to transmission constraints, in addition to nonnegativity constraints on all the variables except $x^{GB}_{t}$ since energy is allowed to flow both ways between the grid and the battery.  For compactness, we are going to represent the constraints using
\bns
A_t x_t &=& R_t,\\
x_t     & \leq & u_t,\\
x_t     & \geq & 0.
\ens


\noindent {\bf Exogenous information} - For the variables with forecasts (demand and wind energy), the exogenous information is the change in the forecast, or the deviation between forecast and actual:
\bns
\varepsilon^D_{t+1,\tau} &=& \textwrap{Change in the forecast of demand (for $\tau > 1$ periods in the future) that we first learn at time $t+1$, or the deviation between actual and forecast (for $\tau = 1$).}\\
\varepsilon^E_{t+1,\tau} &=& \textwrap{Change in the forecast of wind energy (for $\tau > 1$ periods in the future) that we first learn at time $t+1$,  or the deviation between actual and forecast (for $\tau = 1$).}
\ens
We assume that prices evolve purely exogenously with deviations:
\bns
\phat_{t+1} &=& \textwrap{Change in grid prices between $t$ and $t+1$.}
\ens
Our exogenous information is then
\bns
W_{t+1} = ((\varepsilon^D_{t+1,\tau},\varepsilon^E_{t+1,\tau})_{\tau \geq 1}, \phat_{t+1}).
\ens

\noindent {\bf Transition function} - The variables that evolve exogenously are
\bns
f^D_{t+1,t'} &=& f^D_{tt'} + \varepsilon^D_{t+1,t'-t-1},~~t'=t+2, \ldots,\\
D_{t+1}      &=& f^D_{t+1,t'} + \varepsilon^D_{t+1,1},\\
f^E_{t+1,t'} &=& f^E_{tt'} + \varepsilon^E_{t+1,t'-t-1},~~t'=t+2, \ldots,\\
E_{t+1}      &=& f^E_{t+1,t'} + \varepsilon^E_{t+1,1},\\
p_{t+1}      &=& p_t + \phat_{t+1}.
\ens
The energy in storage evolves according to
\bns
R_{t+1} = R_t  + x^{EB}_t + x^{GB}_t - x^{BD}_t.
\ens
These equations make up our transition function $S_{t+1} = S^M(S_t,x_t,W_{t+1})$.\\

\noindent {\bf Objective function} - Our single-period contribution function is
\bns
C(S_t,x_t) = p_t \big(x^{GB}_t + x^{GD}_t\big).
\ens
Our objective function, then, would be
\bn
\max_{\pi=(f\in\Fcal,\theta\in\Theta^f)} F^\pi(\theta) = \E\left\{\sum_{t=0}^T C(S_t,X^\pi(S_t|\theta))|S_0\right\}, \label{eq:cfaenergyobjective}
\en
where $S_{t+1} = S^M(S_t,x_t=X^\pi(S_t|\theta),W_{t+1})$, and given a model of the uncertainty that enters our system through the initial state $S_0$ and the exogenous information sequence $W_1, W_2, \ldots, W_T$.  As in the past, we can estimate this objective function by simulating our policy, which we present next.



\subsection{Designing a policy}


%xxx Be sure to discuss reasons why affine policies, VFAs and stochastic lookaheads will not work.\\
%xxx Point out that a full stochastic lookahead model has to model the evolution of forecasts, as well as the evolution of decisions.  Introduce $f^D_{t',t''}$ and $\xtilde_{t',t''}$ in the lookahead model.\\
%xxx Note that there is no pattern among $\theta_\tau$ across $\tau$.

There is a very small literature that addresses inventory problems while explicitly recognizing rolling forecasts $f_t = (f^D_{tt'},f^E_{tt'})_{t'\geq t}$.  \cite{Iida2006} shows that an order-up-to policy parameterized by $\theta(f_t)$ is optimal, but does not attempt to compute the multidimensional function $\theta(f_t)$.  \cite{Lai2008} considers price forecasts in the context of natural gas storage, formulating the dynamic program with $f_t$ in the state variable and showing that an order-up-to policy $\theta(f_t)$ is optimal  but also proposes an approximation based on supporting hyperplanes.  \cite{Chen1999} formulates an inventory problem with rolling forecasts and attempts to use approximate dynamic programming, but is limited to about a half dozen dimensions.  None of these papers considers the much more difficult problem of bounds on order quantities and storage, which makes the problem much harder, and invalidates the optimality proofs of order-up-to policies that can be written as $\theta(f_t)$ (they have to be time-dependent, and state-dependent $\theta_t(S_t)$ where $S_t$ includes all the elements of the state variable).

Drawing on the framework of a parametric cost function approximation, we are going to start with a classical deterministic lookahead model.  We begin by creating the decision variables for our lookahead model
\bns
\xtilde_{tt'} = (\xtilde^{ED}_{tt'},\xtilde^{EB}_{tt'},\xtilde^{GD}_{tt'},\xtilde^{GB}_{tt'},\xtilde^{BD}_{tt'}), ~t+1 \leq  t' \leq t+H,
\ens
which parallels the elements of $x_t$ for each time $t'$ in the future.

This is a time-dependent problem with complex interactions between the uncertain supply of wind energy, the price of energy from the grid, and the time-dependent nature of demands that have to be satisfied over a capacitated grid.  It seems natural to start by creating a policy based on a deterministic lookahead model given by:

{\small
\bn
X^{DLA}(S_t) \hspace{-.10in}& = &\hspace{-.10in} \argmax_{x_t,(\xtilde_{tt'},t'=t+1, \ldots, t+H)} \left(p_t (x^{GB}_t + x^{GD}_t) + \sum_{t'=t+1}^{t+H} \ptilde_{tt'} (\xtilde^{GB}_{tt'} + \xtilde^{GD}_{tt'}) \right) \nonumber \\
& &\label{eq:DLA0}
\en
}
subject to the following constraints: First, for time $t$ we have
\bn
x^{BD}_{t} - x^{GB}_{t} - x^{EB}_{t} & \leq & R_{t}, \label{eq:DLA1} \\
\Rtilde_{t,t+1}- (x^{GB}_{t} + x^{EB}_{t} - x^{BD}_{t}) &=& R_{t}, \label{eq:DLA2}\\
x^{ED}_{t} + x^{BD}_{t} + x^{GD}_{t} & = & D_{t},\label{eq:DLA3}\\
x^{EB}_{t} + x^{ED}_{t}  & \leq & E_{t}, \label{eq:DLA4} \\
x^{GD}_t, x^{EB}_t, x^{ED}_t, x^{BD}_t & \geq & 0, \label{eq:DLA5}
%\Rtilde_{tt} &=& R_t - (x^{BD}_{t} - x^{GB}_{t} - x^{EB}_{t}). \label{eq:DLA6}
\en
Then, for $t' = t+1, \ldots, t+H$ we have
\bn
%\Rtilde_{t,t'+1} &=& \Rtilde_{t,t'} - (\xtilde^{BD}_{tt'} - \xtilde^{GB}_{tt'} - \xtilde^{EB}_{tt'}),\label{eq:DLA6}\\
\xtilde^{BD}_{tt'} - \xtilde^{GB}_{tt'} - \xtilde^{EB}_{tt'} & \leq & \Rtilde_{tt'}, \label{eq:DLA6} \\
\Rtilde_{t,t'+1}- (\xtilde^{GB}_{tt'} + \xtilde^{EB}_{tt'} - \xtilde^{BD}_{tt'}) &=& \Rtilde_{tt'}, \label{eq:DLA7}\\
\xtilde^{ED}_{tt'} + \xtilde^{BD}_{tt'} + \xtilde^{GD}_{tt'} & = & f^D_{tt'},\label{eq:DLA8}\\
\xtilde^{EB}_{tt'} + \xtilde^{ED}_{tt'}  & \leq & f^E_{tt'}. \label{eq:DLA9}
\en

The weakness in this model is the forecasts of wind energy $f^E_{tt'}$ and demand $f^D_{tt'}$.  One idea would be to ``discount'' these forecasts by multiplying each forecast with coefficients $\theta^E_{t'-t}$ and $\theta^D_{t'-t}$ that depend on how far into the future we are trying to forecast.  Using this approach, we create a parameterized policy by replacing equations \eqref{eq:DLA8} and \eqref{eq:DLA9} with
\bn
\xtilde^{ED}_{tt'} + \xtilde^{BD}_{tt'} + \xtilde^{GD}_{tt'} & = & \theta^D_{t'-t}f^D_{tt'},\label{eq:DLA8a}\\
\xtilde^{EB}_{tt'} + \xtilde^{ED}_{tt'}  & \leq & \theta^E_{t'-t}f^E_{tt'}. \label{eq:DLA9a}
\en

%It is easy to criticize this approach as just a heuristic, deterministic lookahead.  This attitude completely ignores a) the intuition behind the parameterization and b) the fact that we are tuning $\theta$ (using equation \eqref{eq:cfaenergyobjective}) with a stochastic, dynamic model that does not make any of the approximations that we would need to create a solvable stochastic lookahead model.

The problem of tuning the parameter vector $\theta$ is not easy, but there are a number of strategies we can draw on.  We designed an algorithm \citep{Ghadimi2022} using a stochastic gradient algorithm based on Spall's simultaneous perturbation stochastic approximation algorithm \cite{spall2005introduction} which is well suited to problems with multidimensional parameters.  We then compared the performance of the optimized parametric policy against a base policy with $\theta = 1$.  The results are shown in figure \ref{fig:cfaenergysystemperformance} for optimized $\theta$ using a range of different starting points, where the left-most bar uses an initial starting point of $\theta^0 = 1$.  Most of the results show improvements of 20 to 50 percent.
\begin{figure}[tb]
\begin{center}
\begin{tabular}{c}
\includegraphics[width=3.5in]{cfaenergysystemperformance}
\end{tabular}
\caption{Relative performance of optimized parameterized lookahead policy to the performance of a deterministic lookahead with $\theta = 1$.}
\label{fig:cfaenergysystemperformance}
\end{center}
\end{figure}

We claim that the roughly 30 percent improvement is quite significant, given that it does not come at any additional computational cost in the field.  At the same time we observe that there is no alternative computational strategy that would be guaranteed to be better.  Two-stage stochastic programs do not even attempt to model the evolution of estimates of the forecasts.  Approximate dynamic programming would not be able to capture the complex nonlinearities of forecasts in the state variable, especially for the capacitated problem that we are solving.

\begin{comment}

This approach enjoys some important qualities:
\begin{itemize}
\item The policy naturally handles the complex dynamics of a highly time-dependent problem with capacity constraints on transmission and storage that emphasize the importance for planning into the future.
\item Although a time-dependent problem requires time-dependent behavior, the effect of incorporating a rolling forecast produces a stationary policy.
\item The dimensionality of the rolling forecast significantly complicates solutions based on Bellman's equation.  Yet, we could transition from modeling time in hourly increments (giving us a 24-dimensional forecast) to 5-minute increments (giving us a 288-dimensional forecast) without difficulty.
\item The policy can be computed very quickly, which would be particularly important if we decide to solve the problem every five minutes.
\end{itemize}


While this approach is both attractive (since it is easy to implement) and promising (see figure \ref{fig:cfaenergysystemperformance}), serious research issues remain:
\begin{itemize}
  \item While the parameterization that multiplies coefficients $\theta$ times the forecasts is intuitively appealing, other parameterizations are possible, such as ensuring that the energy in storage in future time periods stays above a minimum level (as a reserve) and below a maximum level (so we can store unexpected surges in wind).  Searching for the best parameterization is a challenging research problem.
%  \item Even if we accept the intuition of ``discounting'' forecasts, it seems reasonable to expect that $\theta_\tau$ (for either forecasted wind energy or demand) would trace a smooth function in $\tau$.  One set of optimized values for $\theta^E_\tau$ is shown in figure \ref{fig:optimizedthetas}, which does not show any pattern at all.  It would be interesting to understand if this is a result of noise in the stochastic search algorithm, or if it reflects the properties of the problem.
  \item Stochastic search remains a challenge. The objective function \eqref{eq:cfaenergyobjective} is nonconvex in $\theta$ (when we limit the search over $\theta$).
  \item It is easy to assume that we are going to optimize $\theta$ using a simulator, but a real challenge would be to optimize $\theta$ online (in the field), so that the system adapts to changing conditions.
\end{itemize}

\begin{figure}[tb]
\begin{center}
%\begin{tabular}{c}
\includegraphics[width=3.5in]{optimizedthetas}
\end{tabular}
\caption{Optimized values of $\theta_\tau$ for $tau$ time periods into the future.}
\label{fig:optimizedthetas}
\end{center}
\end{figure}





\begin{figure}[tb]
\center{
    \begin{tabular}{cc}
    \includegraphics[width = 2.25in]{cfaoptimumthetadeterministic} &
    \includegraphics[width = 2.25in]{cfaoptimumthetastochastic} \\
                    (a)                  &                    (b)
    \end{tabular}
    \caption{Objective vs. $\theta_\tau$ for (a) perfect forecasts and (b) stochastic forecasts.}
    \label{fig:cfaoptimumtheta}
    }
\end{figure}
\end{comment}


\section{Closing remarks}
\label{sec:closingremarks}
The major goal of this paper is to make the argument that a parameterized deterministic optimization model is a perfectly valid basis for a policy for (stochastic) sequential decision problems.  The research community needs to accept that tuning a parameterized policy using a stochastic base model such as that given in equation \eqref{eq:baseobjective} is a form of stochastic optimization, even if the policy requires solving a deterministic optimization problem.

Deterministic lookahead models are widely used in practice because they easily handle complexity, and are relatively easy to solve.  The use of parameterized cost function approximations enjoys several significant strengths, especially in the context of complex problems.  Some examples are:
\begin{itemize}
\item The parametric cost function approximations naturally handles the dynamics of a highly time-dependent problem with complicating constraints (as we encountered in the energy storage problem) that emphasize the importance for planning into the future.
\item Although a time-dependent problem requires time-dependent behavior, the effect of incorporating a rolling forecast produces a stationary policy.  The function $X^\pi(S_t)$ is not time dependent, and the vector $\theta$ is not time-dependent.  This property significantly simplifies the search process for $\theta$.
%\item The policy naturally captures the dependence on all the elements of the state variable $S_t$, eliminating the need to design a complex function such as a state-dependent order-up-to parameter $\theta_t(S_t)$.
\item Rolling forecasts arise in many settings, yet introduce complex stochastic interactions between forecasts and decisions in the future, which impact the decisions made now.  Capturing this in a stochastic lookahead policy is exceptionally difficult, but is quite easy in a simulator.
\item Parametric deterministic lookahead policies can capture complex state variables (the rolling forecast is just one example) much more easily than policies based on stochastic lookaheads.  Similarly, simulating complex state variables for the purpose of parameter tuning is also quite easy.
\item The parametric deterministic lookahead policy is generally easy to compute in the field.
\end{itemize}

While this approach is both attractive (since it is easy to implement) and promising (see figure \ref{fig:cfaenergysystemperformance}), serious research issues remain:
\begin{itemize}
  \item Designing the best parameterization is difficult, but closely parallels the challenges of model design in machine learning.  For our energy storage problem, we might say that multiplying coefficients $\theta$ times the forecasts is intuitively appealing, but other parameterizations are possible, such as ensuring that the energy in storage in future time periods stays above a minimum level (as a reserve) and below a maximum level (so we can store unexpected surges in wind).  We suspect that most industrial applications at best use intuitive parameterizations without the benefits of experimental testing in a simulator.
%  \item Even if we accept the intuition of ``discounting'' forecasts, it seems reasonable to expect that $\theta_\tau$ (for either forecasted wind energy or demand) would trace a smooth function in $\tau$.  One set of optimized values for $\theta^E_\tau$ is shown in figure \ref{fig:optimizedthetas}, which does not show any pattern at all.  It would be interesting to understand if this is a result of noise in the stochastic search algorithm, or if it reflects the properties of the problem.
  \item Stochastic search remains a challenge. For example, the objective function \eqref{eq:cfaenergyobjective} is nonconvex in $\theta$ (when we limit the search over $\theta$).  Simulating policies can also be quite noisy.
  \item While it is natural to assume that we can tune the parameters using a simulator, there will be many settings where a simulator is not available (or would not be trusted).  An important research challenge is to perform online parameter tuning in the field so that the policy adapts to changing conditions.
\end{itemize}

We hope that the thoughts in this paper encourage the stochastic optimization community to include parameterized deterministic models as valid policies for stochastic optimization problems.  This initiative is likely to be warmly endorsed by industry that is already implementing parameterized deterministic models, without the benefits of careful design of the parameterization and parameter tuning.


%{\bf Acknowledgments} \\ \\
%This research has been supported in part by the National Science Foundation, grant CMMI-1537427.

%\singlespace
%\bibliographystyle{../../agsm}
%\bibliography{ghadimilibrary,../../bib2/library}
\begin{thebibliography}{xx}

\harvarditem[Bartroff \& Lai]{Bartroff \& Lai}{2010}{Bartroff2010}
Bartroff, J. \& Lai, T.~L.  (2010), `{Approximate Dynamic Programming and Its
  Applications to the Design of Phase I Cancer Trials}', {\em Statistical
  Science} {\bf 25}(2),~245--257.

\harvarditem[Bayraksan \& Morton]{Bayraksan \&
  Morton}{2009}{bayraksan2009assessing}
Bayraksan, G. \& Morton, D.~P.  (2009), `Assessing solution quality in
  stochastic programs via sampling', {\em Tutorials in Operations Research}
  {\bf 5},~102--122.

\harvarditem[Ben-Tal et al.]{Ben-Tal, El-Ghaoui \&
  Nemirovski}{2009}{Ben-Tal2009a}
Ben-Tal, A., El-Ghaoui, L. \& Nemirovski, A.  (2009), `{Robust Optimization}',
  {\bf 53}(3),~464--501.

\harvarditem[Ben-Tal et al.]{Ben-Tal, Golany, Nemirovski \&
  Vial}{2005}{Ben-Tal2005}
Ben-Tal, A., Golany, B., Nemirovski, A. \& Vial, J.-p.  (2005),
  `{Retailer-Supplier Flexible Commitments Contracts: A Robust Optimization
  Approach}', {\em Manufacturing \& Service Operations Management} {\bf
  7}(3),~248--271.

\harvarditem[Bertsekas]{Bertsekas}{2011}{bertsekas2011dynamic}
Bertsekas, D.~P.  (2011), `Dynamic programming and optimal control 3rd
  edition', {\em Vol. II, Belmont, MA: Athena Scientific}.

\harvarditem[Bertsekas]{Bertsekas}{2017}{Bertsekas2017}
Bertsekas, D.~P.  (2017), {\em {Dynamic Programming and Optimal Control:
  Approximate Dynamic Programming}}, 4 edn, Athena Scientific, Belmont, MA.

\harvarditem[Bertsekas \& Tsitsiklis]{Bertsekas \& Tsitsiklis}{1996}{Neuro_DP}
Bertsekas, D.~P. \& Tsitsiklis, J.~N.  (1996), {\em Neuro-Dynamic Programming},
  Athena Scientific.

\harvarditem[Bertsimas et al.]{Bertsimas, Brown \&
  Caramanis}{2011}{BertsimasBrown2011}
Bertsimas, D., Brown, D.~B. \& Caramanis, C.  (2011), `{Theory and applications
  of robust optimization}', {\em SIAM Review} {\bf 53}(3),~464--501.

\harvarditem[Birge \& Louveaux]{Birge \& Louveaux}{2011}{birge2011introduction}
Birge, J.~R. \& Louveaux, F.  (2011), {\em Introduction to stochastic
  programming}, Springer Science \& Business Media.

\harvarditem[Bouzaiene-Ayari et al.]{Bouzaiene-Ayari, Cheng, Das, Fiorillo \&
  Powell}{2014}{BeChPo2014}
Bouzaiene-Ayari, B., Cheng, C., Das, S., Fiorillo, R. \& Powell, W.~B.  (2014),
  `{From Single Commodity to Multiattribute Models for Locomotive Optimization
  : A Comparison of Optimal Integer Programming and Approximate Dynamic
  Programming}', {\em Transportation Science} pp.~1--24.

\harvarditem[Box \& Wilson]{Box \& Wilson}{1951}{Box1951}
Box, G. E.~P. \& Wilson, K.~B.  (1951), `{On the Experimental Attainment of
  Optimum Conditions}', {\em Journal of the Royal Statistical Society Series B}
  {\bf 13}(1),~1--45.

\harvarditem[Bubeck \& Cesa-Bianchi]{Bubeck \& Cesa-Bianchi}{2012}{Bubeck2012}
Bubeck, S. \& Cesa-Bianchi, N.  (2012), `{Regret Analysis of Stochastic and
  Nonstochastic Multi-armed Bandit Problems}', {\em Foundations and Trends in
  Machine Learning} {\bf 5}(1),~1--122.

\harvarditem[Camacho \& Alba]{Camacho \& Alba}{2013}{camacho2013model}
Camacho, E.~F. \& Alba, C.~B.  (2013), {\em Model predictive control}, Springer
  Science \& Business Media.

\harvarditem[Cao]{Cao}{2008}{cao2008stochastic}
Cao, X.-R.  (2008), `Stochastic learning and optimization-a sensitivity-based
  approach', {\em IFAC Proceedings Volumes} {\bf 41}(2),~3480--3492.

\harvarditem[Carpentier et al.]{Carpentier, Gendreau \&
  Bastin}{2015}{carpentier2015managing}
Carpentier, P.-L., Gendreau, M. \& Bastin, F.  (2015), `Managing hydroelectric
  reservoirs over an extended horizon using benders decomposition with a memory
  loss assumption', {\em IEEE Transactions on Power Systems} {\bf
  30}(2),~563--572.

\harvarditem[Chau et al.]{Chau, Fu, Qu \& Ryzhov}{2014}{chau2014simulation}
Chau, M., Fu, M.~C., Qu, H. \& Ryzhov, I.~O.  (2014), Simulation optimization:
  a tutorial overview and recent developments in gradient-based methods, {\em
  in} `Proceedings of the 2014 Winter Simulation Conference', IEEE Press,
  pp.~21--35.

\harvarditem[Chen et al.]{Chen, Ruppert \& Shoemaker}{1999}{Chen1999}
Chen, V. C.~P., Ruppert, D. \& Shoemaker, C.~A.  (1999), `{Applying
  Experimental Design and Regression Splines to High-Dimensional
  Continuous-State Stochastic Dynamic Programming}', {\em Operations Research}
  {\bf 47}(1),~38--53.

\harvarditem[Clark \& Scarf]{Clark \& Scarf}{1960}{Clark1960}
Clark, A.~J. \& Scarf, H.  (1960), `{Optimal policies for a multi-echelon
  inventory problem}', {\em Management Science} {\bf 6}(4),~363--505.

\harvarditem[Dantzig]{Dantzig}{1955}{Da55}
Dantzig, G.~B.  (1955), `{Linear programming with uncertainty}', {\em
  Management Science} {\bf 1},~197--206.

\harvarditem[Deisenroth et al.]{Deisenroth, Neumann \& Peters}{2013}{Robots}
Deisenroth, M.~P., Neumann, G. \& Peters, J.  (2013), `A survey on policy
  search for robotics', {\em Foundations and Trends in Robotics} {\bf
  2}(1-2),~1--142.

\harvarditem[Dvoretzky]{Dvoretzky}{1956}{Dv56}
Dvoretzky, A.  (1956), {On Stochastic Approximation}, {\em in} J.~Neyman, ed.,
  `Proceedings 3rd Berkeley Symposium on Mathematical Statistics and
  Probability', University of California Press, pp.~39--55.

\harvarditem[Ermoliev]{Ermoliev}{1983}{Ermoliev1983}
Ermoliev, Y.~M.  (1983), `{Stochastic quasigradient methods and their
  application to system optimization}', {\em Stochastics} {\bf 9},~1--36.

\harvarditem[Fu]{Fu}{2017}{Fu2017}
Fu, M.~C.  (2017), {Markov Decision Processes, AlphaGo, and Monte Carlo Tree
  Search: Back to the Future}, {\em in} `TutORials in Operations Research',
  pp.~68--88.

\harvarditem[Fu]{Fu}{2015}{fu2015handbook}
Fu, M.~C., ed.  (2015), {\em Handbook of simulation optimization}, Springer.

\harvarditem[Ghadimi \& Lan]{Ghadimi \& Lan}{2013}{GhaLan12}
Ghadimi, S. \& Lan, G.  (2013), `Stochastic first- and zeroth-order methods for
  nonconvex stochastic programming', {\em SIAM Journal on Optimization} {\bf
  23(4)},~2341--2368.

\harvarditem[Ghadimi \& Powell]{Ghadimi \& Powell}{2022}{Ghadimi2022}
Ghadimi, S. \& Powell, W.~B.  (2022), {Stochastic Search for a Parametric Cost
  Function Approximation: Energy storage with rolling forecasts}, Technical
  report.

\harvarditem[Gittins et al.]{Gittins, Glazebrook \& Weber}{2011}{gittins2011}
Gittins, J., Glazebrook, K.~D. \& Weber, R.~R.  (2011), {\em {Multi-Armed
  Bandit Allocation Indices}}, John Wiley and Sons, New York.

\harvarditem[Glasserman]{Glasserman}{1991}{glasserman1991gradient}
Glasserman, P.  (1991), {\em Gradient estimation via perturbation analysis},
  Vol. 116, Springer Science \& Business Media.

\harvarditem[Graves et al.]{Graves, Meal, Dasu \& Qui}{1986}{Graves1986}
Graves, S.~C., Meal, H.~C., Dasu, S. \& Qui, Y.  (1986), {Two-Stage Production
  Planning in a Dynamic Environment}, {\em in} A.~S, S.~C \& S.~E, eds,
  `Multi-Stage Production Planning and Inventory Control, Lecture Notes in
  Economics and Mathematical Systems', number 266, Springer-Verlag, Berlin,
  pp.~9--43.

\harvarditem[Hadjiyiannis et al.]{Hadjiyiannis, Goulart \&
  Kuhn}{2011}{hadjiyiannis2011efficient}
Hadjiyiannis, M.~J., Goulart, P.~J. \& Kuhn, D.  (2011), `An efficient method
  to estimate the suboptimality of affine controllers', {\em IEEE Transactions
  on Automatic Control} {\bf 56}(12),~2841--2853.

\harvarditem[Han \& E]{Han \& E}{2016}{HanE16}
Han, J. \& E, W.  (2016), `Deep learning approximation for stochastic control
  problems', {\em arXiv preprint arXive:1611.07422}.

\harvarditem[Harrison \& Van~Mieghem]{Harrison \&
  Van~Mieghem}{1999}{harrison1999multi}
Harrison, J.~M. \& Van~Mieghem, J.~A.  (1999), `Multi-resource investment
  strategies: Operational hedging under demand uncertainty', {\em European
  Journal of Operational Research} {\bf 113}(1),~17--29.

\harvarditem[Heath \& Jackson]{Heath \& Jackson}{1994}{Heath1994}
Heath, D.~C. \& Jackson, P.~L.  (1994), `{Modeling the Evolution of Demand
  Forecasts With Application to Safety Stock Analysis In
  Production/Distribution Systems}', {\em IIE Transactions (Institute of
  Industrial Engineers)} {\bf 26}(3),~17--30.

\harvarditem[Ho]{Ho}{1992}{ho1992discrete}
Ho, Y.-C.  (1992), {\em Discrete event dynamic systems: analyzing complexity
  and performance in the modern world}, IEEEPress, New York.

\harvarditem[Hu et al.]{Hu, Fu, Ramezani \& Marcus}{2007}{hu2007evolutionary}
Hu, J., Fu, M.~C., Ramezani, V.~R. \& Marcus, S.~I.  (2007), `An evolutionary
  random policy search algorithm for solving markov decision processes', {\em
  INFORMS Journal on Computing} {\bf 19}(2),~161--174.

\harvarditem[Iida \& Zipkin]{Iida \& Zipkin}{2006}{Iida2006}
Iida, T. \& Zipkin, P.~H.  (2006), `{Approximate Solutions of a Dynamic
  Forecast-Inventory Model}', {\em Manufacturing \& Service Operations
  Management} {\bf 8}(4),~407--425.

\harvarditem[Jin et al.]{Jin, Ryan, Watson \& Woodruff}{2011}{jin2011modeling}
Jin, S., Ryan, S.~M., Watson, J.-P. \& Woodruff, D.~L.  (2011), `Modeling and
  solving a large-scale generation expansion planning problem under
  uncertainty', {\em Energy Systems} {\bf 2}(3-4),~209--242.

\harvarditem[Kaelbling]{Kaelbling}{1993}{Ka93}
Kaelbling, L.~P.  (1993), {\em {Learning in embedded systems}}, MIT Press,
  Cambridge, MA.

\harvarditem[Kall \& Wallace]{Kall \& Wallace}{2009}{Kall2009}
Kall, P. \& Wallace, S.~W.  (2009), {\em {Stochastic Programming}}, Vol.~10,
  John Wiley and Sons, Hoboken, NJ.

\harvarditem[Kirk]{Kirk}{2012}{Kirk2012}
Kirk, D.~E.  (2012), {\em {Optimal Control Theory: An introduction}}, Dover,
  New York.

\harvarditem[Kushner \& Yin]{Kushner \& Yin}{2003}{kushner2003stochastic}
Kushner, H.~J. \& Yin, G.  (2003), {\em Stochastic approximation and recursive
  algorithms and applications}, Vol.~35, Springer Science \& Business Media.

\harvarditem[Lai et al.]{Lai, Margot \& Secomandi}{2008}{Lai2008}
Lai, G., Margot, F. \& Secomandi, N.  (2008), `{An Approximate Dynamic
  Programming Approach to Benchmark Practice-based Heuristics for Natural Gas
  Storage Valuation}', {\em Working Paper, Tepper School of Business, Carnegie
  Mellon University.}

\harvarditem[Lai \& Robbins]{Lai \& Robbins}{1985}{Lai1985}
Lai, T.~L. \& Robbins, H.  (1985), `{Asymptotically efficient adaptive
  allocation rules}', {\em Advances in Applied Mathematics} {\bf 6}(1),~4--22.

\harvarditem[Lan et al.]{Lan, Clarke \& Barnhart}{2006}{lan2006planning}
Lan, S., Clarke, J.-P. \& Barnhart, C.  (2006), `Planning for robust airline
  operations: Optimizing aircraft routings and flight departure times to
  minimize passenger disruptions', {\em Transportation science} {\bf
  40}(1),~15--28.

\harvarditem[Levine \& Abbeel]{Levine \& Abbeel}{2014}{levine2014learning}
Levine, S. \& Abbeel, P.  (2014), Learning neural network policies with guided
  policy search under unknown dynamics, {\em in} `Advances in Neural
  Information Processing Systems', pp.~1071--1079.

\harvarditem[Levine \& Koltun]{Levine \& Koltun}{2013}{Levine2013}
Levine, S. \& Koltun, V.  (2013), `{Guided policy search}', {\em 30th
  International Conference on Machine Learning, ICML 2013} {\bf 28}(PART
  2),~1038--1046.

\harvarditem[Lewis \& Vrabie]{Lewis \& Vrabie}{2009}{LewisVrabie2009}
Lewis, F.~L. \& Vrabie, D.  (2009), `{Reinforcement Learning and Adaptive
  Dynamic Programming for Feedback Control}', {\em IEEE Circuits And Systems
  Magazine} {\bf 9}(3),~32--50.

\harvarditem[Lewis et al.]{Lewis, Vrabie \& Syrmos}{2012}{lewis2012}
Lewis, F.~L., Vrabie, D. \& Syrmos, V.~L.  (2012), {\em {Optimal Control}}, 3rd
  edn, John Wiley and Sons, Hoboken, NJ.

\harvarditem[Lillicrap et al.]{Lillicrap, Hunt, Pritzel, Heess, Erez, Tassa,
  Silver \& Wierstra}{2015}{lillicrap2015continuous}
Lillicrap, T.~P., Hunt, J.~J., Pritzel, A., Heess, N., Erez, T., Tassa, Y.,
  Silver, D. \& Wierstra, D.  (2015), `Continuous control with deep
  reinforcement learning', {\em arXiv preprint arXiv:1509.02971}.

\harvarditem[Lium et al.]{Lium, Crainic \& Wallace}{2009}{lium2009study}
Lium, A.-G., Crainic, T.~G. \& Wallace, S.~W.  (2009), `A study of demand
  stochasticity in service network design', {\em Transportation Science} {\bf
  43}(2),~144--157.

\harvarditem[Mannor et al.]{Mannor, Rubinstein \& Gat}{2003}{mannor2003cross}
Mannor, S., Rubinstein, R.~Y. \& Gat, Y.  (2003), The cross entropy method for
  fast policy search, {\em in} `ICML', pp.~512--519.

\harvarditem[Nesterov \& Spokoiny]{Nesterov \& Spokoiny}{2017}{NesSpo17}
Nesterov, Y. \& Spokoiny, V.  (2017), `Random gradient-free minimization of
  convex functions', {\em Foundations of Computational Mathematics} {\bf
  17}(2),~527--566.

\harvarditem[Ng \& Jordan]{Ng \& Jordan}{2000}{ng2000pegasus}
Ng, A.~Y. \& Jordan, M.  (2000), Pegasus: A policy search method for large mdps
  and pomdps, {\em in} `Proceedings of the Sixteenth conference on Uncertainty
  in artificial intelligence', Morgan Kaufmann Publishers Inc., pp.~406--415.

\harvarditem[Pereira \& Pinto]{Pereira \& Pinto}{1991}{Pereira1991}
Pereira, M.~F. \& Pinto, L. M. V.~G.  (1991), `{Multi-stage stochastic
  optimization applied to energy planning}', {\em Mathematical Programming}
  {\bf 52},~359--375.

\harvarditem[Peshkin et al.]{Peshkin, Kim, Meuleau \&
  Kaelbling}{2000}{peshkin2000learning}
Peshkin, L., Kim, K.-E., Meuleau, N. \& Kaelbling, L.~P.  (2000), Learning to
  cooperate via policy search, {\em in} `Proceedings of the Sixteenth
  conference on Uncertainty in artificial intelligence', Morgan Kaufmann
  Publishers Inc., pp.~489--496.

\harvarditem[Philpott \& {De Matos}]{Philpott \& {De
  Matos}}{2012}{Philpott2012}
Philpott, A.~B. \& {De Matos}, V.~L.  (2012), `{Dynamic sampling algorithms for
  multi-stage stochastic programs with risk aversion}', {\em European Journal
  of Operational Research} {\bf 218}(2),~470--483.

\harvarditem[Philpott et al.]{Philpott, Craddock \&
  Waterer}{2000}{Philpott2000}
Philpott, A.~B., Craddock, M. \& Waterer, H.  (2000), `{Hydro-electric unit
  commitment subject to uncertain demand}', {\em European Journal of
  Operational Research} {\bf 125}(2),~410--424.

\harvarditem[Pillac et al.]{Pillac, Gendreau, Gu{\'{e}}ret \&
  Medaglia}{2013}{pillac2013}
Pillac, V., Gendreau, M., Gu{\'{e}}ret, C. \& Medaglia, A.~L.  (2013), `{A
  review of dynamic vehicle routing problems}', {\em European Journal of
  Operational Research} {\bf 225}(1),~1--11.

\harvarditem[Powell]{Powell}{2011}{PowellADP2011}
Powell, W.~B.  (2011), {\em {Approximate Dynamic Programming: Solving the
  Curses of Dimensionality}}, 2 edn, John Wiley and Sons.

\harvarditem[Powell]{Powell}{2019{\em a}}{PowellUnitComm2019}
Powell, W.~B.  (2019{\em a}), `{Perspectives on Stochastic Unit Commitment:
  From Scenario Trees to Parametric Cost Function Approximations}'.

\harvarditem[Powell]{Powell}{2019{\em b}}{Powell2019}
Powell, W.~B.  (2019{\em b}), `A unified framework for stochastic
  optimization', {\em European Journal of Operational Research} {\bf
  275}(3),~795--821.

\harvarditem[Powell]{Powell}{2022}{PowellRLSO}
Powell, W.~B.  (2022), {\em {Reinforcement Learning and Stochastic
  Optimization: A unified framework for sequential decisions}}, john Wiley and
  Sons, New York.

\harvarditem[Powell \& Ryzhov]{Powell \& Ryzhov}{2012}{PoRy2012}
Powell, W.~B. \& Ryzhov, I.~O.  (2012), {\em {Optimal Learning}}, John Wiley
  and Sons, Hoboken, NJ.

\harvarditem[Powell et al.]{Powell, Ruszczy{\'n}ski \&
  Topaloglu}{2004}{powell2004learning}
Powell, W., Ruszczy{\'n}ski, A. \& Topaloglu, H.  (2004), `Learning algorithms
  for separable approximations of discrete stochastic optimization problems',
  {\em Mathematics of Operations Research} {\bf 29}(4),~814--836.

\harvarditem[Puterman]{Puterman}{2014}{putermanmarkov}
Puterman, M.~L.  (2014), {\em Markov decision processes: discrete stochastic
  dynamic programming}, John Wiley \& Sons.

\harvarditem[Robbins \& Monro]{Robbins \& Monro}{1951}{RoMo51}
Robbins, H. \& Monro, S.  (1951), `{A stochastic approximation method}', {\em
  The Annals of Mathematical Statistics} {\bf 22}(3),~400--407.

\harvarditem[Sen \& Zhou]{Sen \& Zhou}{2014}{sen2014multistage}
Sen, S. \& Zhou, Z.  (2014), `Multistage stochastic decomposition: a bridge
  between stochastic programming and approximate dynamic programming', {\em
  SIAM Journal on Optimization} {\bf 24}(1),~127--153.

\harvarditem[Sethi \& Sorger]{Sethi \& Sorger}{1991}{sethi1991theory}
Sethi, S. \& Sorger, G.  (1991), `A theory of rolling horizon decision making',
  {\em Annals of Operations Research} {\bf 29}(1),~387--415.

\harvarditem[Sethi]{Sethi}{2019}{sethi2019}
Sethi, S.~P.  (2019), {\em {Optimal Control Theory: Applications to Management
  Science and Economics}}, 3 edn, Springer-Verlag, Boston.

\harvarditem[Settles]{Settles}{2010}{Settles2010}
Settles, B.  (2010), `{Active Learning}', {\em Sciences-New York}.

\harvarditem[Shapiro]{Shapiro}{2011}{Shapiro2011}
Shapiro, A.  (2011), `{Analysis of stochastic dual dynamic programming
  method}', {\em European Journal of Operational Research} {\bf
  209}(1),~63--72.

\harvarditem[Shapiro et al.]{Shapiro, Dentcheva \&
  Ruszczy{\'{n}}ski}{2014}{Shapiro2014}
Shapiro, A., Dentcheva, D. \& Ruszczy{\'{n}}ski, A.  (2014), `{Lectures on
  stochastic programming: modeling and theory}', {\em Technology} p.~447.

\harvarditem[Shor]{Shor}{1979}{Sh79}
Shor, N.~K.  (1979), {\em {The Methods of Nondifferentiable Op[timization and
  their Applications}}, Naukova Dumka, Kiev.

\harvarditem[Si et al.]{Si, Barto, Powell \& Wunsch}{2004}{SiBaPo04}
Si, J., Barto, A.~G., Powell, W.~B. \& Wunsch, D.  (2004), {\em {Handbook of
  Learning and Approximate Dynamic Programming}}, Chichester, U.K.

\harvarditem[Simao et al.]{Simao, Day, George, Gifford, Powell \&
  Nienow}{2009}{SiDaGe09}
Simao, H.~P., Day, J., George, A.~P., Gifford, T., Powell, W.~B. \& Nienow, J.
  (2009), `{An Approximate Dynamic Programming Algorithm for Large-Scale Fleet
  Management: A Case Application}', {\em Transportation Science} {\bf
  43}(2),~178--197.

\harvarditem[Sontag]{Sontag}{1998}{Sontag1998}
Sontag, E.  (1998), `{Mathematical Control Theory, 2nd ed.}', {\em Springer}
  pp.~1--544.

\harvarditem[Spall]{Spall}{2003}{spall2005introduction}
Spall, J.~C.  (2003), {\em Introduction to stochastic search and optimization:
  estimation, simulation, and control}, Vol.~65, John Wiley \& Sons.

\harvarditem[Stengel]{Stengel}{1986}{stengel1986}
Stengel, R.~F.  (1986), {\em {Stochastic optimal control: theory and
  application}}, John Wiley and Sons, Hoboken, NJ.

\harvarditem[Sutton \& Barto]{Sutton \& Barto}{1998}{sutton1998reinforcement}
Sutton, R.~S. \& Barto, A.~G.  (1998), {\em Reinforcement learning: An
  introduction}, Vol.~1, MIT press Cambridge.

\harvarditem[Sutton \& Barto]{Sutton \& Barto}{2018}{Sutton2018}
Sutton, R.~S. \& Barto, A.~G.  (2018), {\em {Reinforcement Learning: An
  Introduction}}, 2nd edn, MIT Press, Cambridge, MA.

\harvarditem[Sutton et al.]{Sutton, McAllester, Singh \&
  Mansour}{2000}{SuMcSi2000}
Sutton, R.~S., McAllester, D., Singh, S.~P. \& Mansour, Y.  (2000), `{Policy
  gradient methods for reinforcement learning with function approximation}',
  {\em Advances in neural information processing systems} {\bf
  12}(22),~1057--1063.

\harvarditem[Wallace \& Fleten]{Wallace \& Fleten}{2003}{wallace2003stochastic}
Wallace, S.~W. \& Fleten, S.-E.  (2003), `Stochastic programming models in
  energy', {\em Handbooks in operations research and management science} {\bf
  10},~637--677.

\harvarditem[Wang et al.]{Wang, Zhang \& Liu}{2009}{Wang2009}
Wang, F.-y., Zhang, H. \& Liu, D.  (2009), `{Adaptive Dynamic Programming: An
  Introduction}', {\em IEEE Computational Intelligence Magazine} (May),~39--47.

\harvarditem[Wiesemann et al.]{Wiesemann, Kuhn \& Sim}{2014}{Wiesemann2014}
Wiesemann, W., Kuhn, D. \& Sim, M.  (2014), `{Distributionally Robust Convex
  Optimization}', {\em Operations Research} {\bf 62}(6),~1358--1376.

\harvarditem[Zhao et al.]{Zhao, Wang, Watson \& Guan}{2013}{zhao2013multi}
Zhao, C., Wang, J., Watson, J.-P. \& Guan, Y.  (2013), `Multi-stage robust unit
  commitment considering wind and demand response uncertainties', {\em IEEE
  Transactions on Power Systems} {\bf 28}(3),~2708--2717.

\end{thebibliography}

%\newpage


%\newpage
\section{Dataset Visualizations}
\label{sec:app_dataset_visuals}

%%%%%%
%%
%%
\subsection{Examples of each view class}
\newcommand{\BC}{0.33}
\setlength{\tabcolsep}{0.1cm}
\begin{figure}[!h]
\begin{tabular}{c c c c}
    PLAX  & PSAX & OTHER 
    \\
    \includegraphics[width=\BC\textwidth]{figures/small_appendix/Appendix_PLAX1.jpg}
    &
    \includegraphics[width=\BC\textwidth]{figures/small_appendix/Appendix_PSAX1.jpg}
    &
    \includegraphics[width=\BC\textwidth]{figures/small_appendix/Appendix_Other1.jpg}
    &
   
    \\
    
    \includegraphics[width=\BC\textwidth]{figures/small_appendix/Appendix_PLAX2.jpg}
    &
    \includegraphics[width=\BC\textwidth]{figures/small_appendix/Appendix_PSAX2.jpg}
    &
    \includegraphics[width=\BC\textwidth]{figures/small_appendix/Appendix_Other2.jpg}
    &
   
     \\
     
     \includegraphics[width=\BC\textwidth]{figures/small_appendix/Appendix_PLAX3.jpg}
    &
    \includegraphics[width=\BC\textwidth]{figures/small_appendix/Appendix_PSAX3.jpg}
    &
    \includegraphics[width=\BC\textwidth]{figures/small_appendix/Appendix_Other3.jpg}
    &
   
     \\
     
     \includegraphics[width=\BC\textwidth]{figures/small_appendix/Appendix_PLAX4.jpg}
    &
    \includegraphics[width=\BC\textwidth]{figures/small_appendix/Appendix_PSAX4.jpg}
    &
    \includegraphics[width=\BC\textwidth]{figures/small_appendix/Appendix_Other4.jpg}
    &
   
    \end{tabular}	
    \caption{Examples of images for each possible view label in our dataset. \emph{From left to right:} Four examples of peristernal long axis (PLAX) view, four examples of peristernal short axis (PSAX) view, and four examples of other kinds of view in our ``Other'' class. }
    \label{fig:VIEW_SAMPLES_APPENDIX}
\end{figure}

%%%%%%
%%
%%
\newpage
\subsection{Examples of each view for a Severe AS patient}
\newcommand{\BA}{0.33}
\setlength{\tabcolsep}{0.1cm}
\begin{figure}[!h]
\begin{tabular}{c c c c}
    PLAX  & PSAX & OTHER 
    \\
    \includegraphics[width=\BA\textwidth]{figures/small_appendix/SevereAS_11112007_PLAX1.jpg}
    &
    \includegraphics[width=\BA\textwidth]{figures/small_appendix/SevereAS_11112007_PSAX1.jpg}
    &
    \includegraphics[width=\BA\textwidth]{figures/small_appendix/SevereAS_11112007_Other1.jpg}
    &
    
    \\
    
    \includegraphics[width=\BA\textwidth]{figures/small_appendix/SevereAS_11112007_PLAX2.jpg}
    &
    \includegraphics[width=\BA\textwidth]{figures/small_appendix/SevereAS_11112007_PSAX2.jpg}
    &
    \includegraphics[width=\BA\textwidth]{figures/small_appendix/SevereAS_11112007_Other2.jpg}
    &
   
     \\
     
     \includegraphics[width=\BA\textwidth]{figures/small_appendix/SevereAS_11112007_PLAX3.jpg}
    &
    \includegraphics[width=\BA\textwidth]{figures/small_appendix/SevereAS_11112007_PSAX3.jpg}
    &
    \includegraphics[width=\BA\textwidth]{figures/small_appendix/SevereAS_11112007_Other3.jpg}
    &
  
    \end{tabular}	
    \caption{Examples of images from a patient with Severe AS in our dataset. \emph{From left to right:} Three examples of parasternal long axis (PLAX) view, three examples of parasternal short axis (PSAX) view, and three examples of other kinds of view in our ``Other'' class. }
    \label{fig:PatientSevereAS}
\end{figure}


%%%%%%
%%
%%
\newpage
\subsection{Examples of each view for a No AS patient}
\newcommand{\BB}{0.33}
\setlength{\tabcolsep}{0.1cm}
\begin{figure}[!h]
\begin{tabular}{c c c c}
    PLAX  & PSAX & OTHER 
    \\
    \includegraphics[width=\BB\textwidth]{figures/small_appendix/NoAS_1996889_PLAX1.jpg}
    &
    \includegraphics[width=\BB\textwidth]{figures/small_appendix/NoAS_1996889_PSAX1.jpg}
    &
    \includegraphics[width=\BB\textwidth]{figures/small_appendix/NoAS_1996889_Other1.jpg}
    &
    
    \\
    
    \includegraphics[width=\BB\textwidth]{figures/small_appendix/NoAS_1996889_PLAX2.jpg}
    &
    \includegraphics[width=\BB\textwidth]{figures/small_appendix/NoAS_1996889_PSAX2.jpg}
    &
    \includegraphics[width=\BB\textwidth]{figures/small_appendix/NoAS_1996889_Other2.jpg}
    &
   
     \\
     
     \includegraphics[width=\BB\textwidth]{figures/small_appendix/NoAS_1996889_PLAX3.jpg}
    &
    \includegraphics[width=\BB\textwidth]{figures/small_appendix/NoAS_1996889_PSAX3.jpg}
    &
    \includegraphics[width=\BB\textwidth]{figures/small_appendix/NoAS_1996889_Other3.jpg}
    &
  
    \end{tabular}	
    \caption{Examples of images from a patient with No AS in our dataset. \emph{From left to right:} Three examples of parasternal long axis (PLAX) view, three examples of parasternal short axis (PSAX) view, and three examples of other kinds of view in our ``Other'' class. }
    \label{fig:PatientNoAS}
\end{figure}



\newpage 
\section{Further Results}

\subsection{Assessment of ensembling}

Table~\ref{tab:best_single_checkpoint_VS_ensemble_FS_echo260} compares using a single checkpoint (one point estimate of neural network weight vector $\theta$) to using an ensemble of parameters aggregated from the last 25 checkpoints (one per epoch).

\begin{table}[!h]
    \centering
    \begin{tabular}{c|cccc|c}
    \textit{Diagnosis classification} & Split 1  & Split 2 & Split 3 & Split 4 & Average\\
    \hline
    Best single checkpoint  & 61.81 & 59.79 & 56.05 & 64.21 & 60.46\\
    Ensemble  & 62.95 & 61.03 & 56.58 & 63.84 & \textbf{61.13}
	\\ \hline
    \textit{View classification}  &   &  &  &  & 
    \\ \hline
    Best single checkpoint  & 93.03 & 93.24 & 92.39 & 93.79 & 93.11\\
    Ensemble  & 92.37 & 93.24 & 93.72 & 93.87 & \textbf{93.30}\\
    \end{tabular}
    \caption{Comparing best single checkpoint performance with ensemble performance on \textbf{Full-size \datasetName-156-52}}
    \label{tab:best_single_checkpoint_VS_ensemble_FS_echo260}
\end{table}


%%%%%%
%%
%%
\subsection{Patient-level diagnosis performance on bonus heldout set}

Table~\ref{tab:diagnosis classification patient unlabeled_heldout_174} examines the performance of the best labeled-set-only methods and MixMatch methods on the 174 patient studies that have diagnosis but no view labels.
 While the images used here were originally included in the unlabeled training set (which was used to train SSL methods like MixMatch), the diagnosis labels were not provided at all during training time. 
 We thus still believe this is an authentic test of generalization given the scarcity of labeled data available for our task.
 Of course, additional independent evaluation (especially from another institution) is needed.

\begin{table}[!h]
    \centering
    \begin{tabular}{l l l|rrrr|c}
    Pretrain & Method & Voting
    & Split 1  & Split 2 & Split 3 & Split 4 & average\\
    \hline
    & Basic WRN & Simple average & 76.73 & 75.25 & 76.87 & 81.88 & 77.68\\
    & Basic WRN & View-prioritized & 73.63 & 83.21 & 79.70 & 80.08 & 79.18\\
    %SSL & FS & MixMatch & Priority view + confidence & 94.58 & 84.17 & 77.50 & 92.5 & 87.19\\
    \hline
    & MixMatch & Simple average & 85.32 & 76.29 & 74.14 & 79.95 & 78.93\\
    view & MixMatch & Simple average & 83.36 & 77.96 & 75.61 & 81.37 & 79.58\\
    & MixMatch & View-prioritized & 83.27 & 83.76 & 82.34 & 82.83 & \textbf{83.05}\\
    view & MixMatch & View-prioritized & 82.53 & 86.15 & 79.62 & 83.27 & 82.89\\
    %view & MixMatch & LR with view-priority & 80.42 & 84.24 & 76.58 & 80.67 & 80.48\\
    %(MixMatch transfered) + MysteryMethod & NA & NA & NA\\ 
    \end{tabular}
    \caption{Patient-level AS Severity Diagnosis Classification on the \textbf{bonus heldout set} of 174 patients for whom we have diagnosis labels only (no view labels). We show balanced accuracy on models trained on each of the four folds on four \textbf{full-size \datasetName-156-52} dataset.
    }%endcaption
    \label{tab:diagnosis classification patient unlabeled_heldout_174}
\end{table}


%%%%%%
%%
%%
\subsection{Assessment of MixMatch hyperparameter sensitivity}

In Table~\ref{tab:MixMatch hyperparameters ablation study}, we consider four possible strategies for setting the hyperparameters of MixMatch, varying two  key settings for the weight on unlabeled loss $\lambda$. First, we vary whether the final value of $\lambda$ is set to its \emph{best} value among a grid of candidates (based on validation set performance), or \emph{fixed} to a constant.
Second, we vary whether $\lambda$ remains fixed over iterations throughout a training run, or is updated over iterations on a linear ramp schedule from 0 to its final target value. 

From this comparison, we see we consistent gains across splits (average gain across splits of over 1.6\% balanced accuracy) for using a delayed ramp up schedule with target value selected via grid search.

\begin{table}[!h]
    \centering
    \begin{tabular}{l l| rrrr | r}
    Final $\lambda$ value & $\lambda$ update schedule & Split 1  & Split 2 & Split 3 & Split 4 & Average\\
    \hline
    best on val & Delayed ramp-up  & 65.57 & 62.69 & 60.87 & 66.29 & 63.86\\
    best on val & Immediate ramp-up & 65.07 & 61.87 & 60.82 & 65.37 & 63.28\\
    best on val & Constant  & 65.03 & 61.52 & 58.87 & 65.22 & 62.66\\
    100 (fixed) & Constant & 63.94 & 61.79 & 58.87 & 64.35 & 62.24\\
    \end{tabular}
    \caption{Ablation study of different settings of the unlabeled loss weight $\lambda$ for MixMatch. AS severity diagnosis classification for individual images on the \textbf{full-size \datasetName-156-52} dataset. showing balanced accuracy averaged over the test sets from multiple folds (each fold’s test set contains all images from 52 patients). }%endcaption
    \label{tab:MixMatch hyperparameters ablation study}
\end{table}



%%%%%%
%%
%%
\subsection{Assessment of alternative view prioritization strategy using thresholding}


An anonymous reviewer suggested an alternative strategy for prioritizing images of relevant view.
The alternative strategy works as follows: for each image, we compute the predicted probability that the image is a ``relevant view'' (either PLAX and PSAX) by summing the probabilities of each view type.
However, instead of using this raw probability as a weight (as our chosen method does), we use a \emph{cutoff threshold} and simply average the diagnosis predictions of images whose relevant view probability is above the cutoff.
For each patient, we use the majority vote prediction of the diagnosis from the images of relevant views.
The value of the cutoff threshold is selected using the validation set to maximize balanced accuracy.

Table~\ref{tab:Suggested_Aggregation_Ablation} shows the performance of this strategy (``threshold-then-average'') on the full-size dataset.
Using this alternative prioritization strategy together with our suggested methodology for patient-level diagnosis (using MixMatch, pretraining on view), we find the average test set balanced accuracy is around 85.8\%, while the weighted average strategy in the main paper achieves over 90\% balanced accuracy. We take this as reasonably decisive evidence that a weighted average (rather than a simple cutoff) should be preferred.

\begin{table}[!h]
    \centering
    \begin{tabular}{l l l|rrrr|c}
    Pretrain & Method & Aggregation across images
    & Split 1  & Split 2 & Split 3 & Split 4 & average\\
    \hline
    & Basic WRN & Threshold-then-Average & 85.42 & 86.25 & 79.17 & 92.50 & 85.84 \\
    %SSL & FS & MixMatch & Priority view + confidence & 94.58 & 84.17 & 77.50 & 92.5 & 87.19\\
    & MixMatch & Threshold-then-Average & 83.33 & 84.17 & 77.50 & 94.58 & 84.90 \\
    view & MixMatch & Threshold-then-Averagen & 86.67 & 80.00 & 82.50 & 94.17 & 85.84\\
    %view & MixMatch & LR with view-priority & 87.08 & 82.08 & 85.00 & 88.75 & 85.73\\
    %(MixMatch transfered) + MysteryMethod & NA & NA & NA\\ 
    \end{tabular}
    \caption{Alternative view-prioritizing strategy for patient-level AS severity diagnosis classification on the \textbf{full-size \datasetName-156-52} dataset, showing balanced accuracy on the test set across multiple folds (each fold’s test set contains 52 patients).}
    %endcaption
    \label{tab:Suggested_Aggregation_Ablation}
\end{table}



%%%%%%
%%
%%
\subsection{ROC Curve of patient-level diagnosis: no AS vs. mild/moderate/severe AS}

Fig.~\ref{fig: No AS vs Some AS} shows receiver operating curves for several methods for the task of distinguishing no AS vs Some AS (which aggregates both the mild/moderate and severe levels in the 3-level diagnosis task of the main paper).

\begin{figure}[!h]
\begin{tabular}{c c}
	\includegraphics[width=0.43\textwidth]{figures/fold0_multitask_PatientLevel_NoVSSome_NormalizedPriorityStrategyClassProbabilityScore.pdf}
	&
    \includegraphics[width=0.43\textwidth]{figures/fold1_multitask_PatientLevel_NoVSSome_NormalizedPriorityStrategyClassProbabilityScore.pdf}
	\\
	(a) Split 1 & (b) Split 2
	\\
	\includegraphics[width=0.43\textwidth]{figures/fold2_multitask_PatientLevel_NoVSSome_NormalizedPriorityStrategyClassProbabilityScore.pdf}
	&
    \includegraphics[width=0.43\textwidth]{figures/fold3_multitask_PatientLevel_NoVSSome_NormalizedPriorityStrategyClassProbabilityScore.pdf}
	\\
	(c) Split 3 & (d) Split 4
\end{tabular}
    
\caption{ROC curves for binary diagnosis task (no AS vs ``mild/moderate/severe AS'') on \textbf{full-size \datasetName-156-52}.
    }%endcaption
    \label{fig: No AS vs Some AS}
\end{figure}

\section{Methodological Details}

\subsection{Image processing details}
\label{sec:removing_doppler}

\paragraph{Removing doppler images.}
In the raw data of all imagery available for an echocardiogram study, 
we obtained TIFF files that represent both cineloops and Doppler images.

We verified in our labeled set that all Doppler images have one of the following landscape aspect ratio $(831, 323)$, $(901, 384)$, $(901, 390)$, $(704, 305)$, $(831, 421)$, $(901, 469)$ or $(563, 294)$. Only the Dopplers have these aspect ratios. We thus filtered out Doppler completely via these aspect ratios. 

\paragraph{Downsizing}
The original images are provided as high-resolution TIFF format images (hundreds of pixels per side) of varying aspect ratios. Generally, we can expect that both view and diagnosis classifiers would perform better given higher-resolution input (and holding other factors the same). The main trade-off of processing higher-resolution images is increased runtime and memory requirements. In our preliminary experiments, we compared downsizing all images to a standard square aspect ratio at 3 possible sizes: 32x32, 64x64 and 128x128. We found that 64x64 achieves a good balance between model performance and computation cost. 
A prior study by \citet{madaniDeepEchocardiographyDataefficient2018} provides a more extensive study of optimal resolution size. The interested reader can refer to their work for more details. 


\subsection{Architecture Settings and Hyperparameters}
\label{sec:arch_and_hyperparameters}

\paragraph{Weighted cross-entropy for labeled loss}
To counteract the effect of class imbalance in the dataset, we use weighted cross-entropy for the labeled loss. For an input image $x$ whose true label $y$ indicates it belongs to class $c$, the weighted cross-entropy assumes the following form:
\begin{align}
\mathcal{L}^L(\theta, x) = - w_{c} \log \hat{p}_{c}(\theta, x),
\end{align}
where $\hat{p}_{c}$ is the predicted probability of class $c$. The weight $w_{c}$ is calculated using the training set statistics as follow:
\begin{align}
w_{c} = \frac{\prod_{k\neq c}{N_{k}}}{\sum_{j}\prod_{k \neq j}{N_{k}}}
\end{align}
where $N_{k}$ is the number of images of class $k$ in the training set.

\paragraph{Common architecture.}
Following~\citet{oliverRealisticEvaluationDeep2018}, for all considered methods, we use the \emph{same} backbone neural network architecture: a wide residual network~\citep{zagoruykoWideResidualNetworks2017} with 28 layers (WRN-28), which has total of 5,931,683 parameters.
This same network architecture is used in the original MixMatch evaluation~\citep{berthelotMixmatchHolisticApproach2019} with promising results.

\paragraph{Common training protocol.}
All SSL methods we consider follow the loss minimization framework with two primary losses (one for ``labeled'' data and one for ``unlabeled'' data) in Eq.~\eqref{eq:standard-SSL-loss-template}.
We allow every method to train for 32 epochs (where each epoch processes $2^{16}$ images, as in \citet{berthelotMixmatchHolisticApproach2019}).
Our preliminary experiments suggest that after 30 epochs all methods effectively converge in terms of validation balanced accuracy. 

\paragraph{Common regularization.}
For all methods, we expect performance will be vulnerable to overfitting, so we impose an L2-norm penalty on the weights $\theta$, also known as weight decay. Each method selects its preferred value of this penalty strength hyperparameter. We searched values in [0.0002, 0.002, 0.02].

\paragraph{Common optimization.}
We use ADAM \citep{kingma2014adam} to optimize each model.
Each method selects the value of the step size (learning rate) as a hyperparameter. We experimented with 0.002 and 0.0007
%HZ: 'performance being sensitive to learning rate' is very reasonable. But we don't have an ablation to back it. 
%We find performance is sensitive to the step size (learning rate) hyperparameter, so we perform a grid search and select the value that maximizes balanced accuracy on the validation set.

\paragraph{Hyperparameters for Pseudo-Label.}
Beyond the usual hyperparameters for our loss-minimization SSL framework, another important hyperparameter for pseudo-label is the threshold $\tau$. We find that performance is not very sensitive to the chosen $\tau$ value as long as it is within a certain range. We set $\tau$ to 0.95, as done in past literature that evaluates Pseudo-Label as an SSL method ~\citep{oliverRealisticEvaluationDeep2018,berthelotMixmatchHolisticApproach2019, berthelotRemixmatchSemisupervisedLearning2019, sohnFixmatchSimplifyingSemisupervised2020}.


\paragraph{Hyperparameters for VAT.}
Beyond the usual hyperparameters for our SSL framework, for VAT we need to select a value for $\epsilon$.
In \citet{miyatoVirtualAdversarialTraining2019}, the authors claimed that they can achieve superior performance by tuning only $\epsilon$ and fixing $\lambda$ to 1. In our experiment, we used the default $\lambda$ as in \cite{berthelotMixmatchHolisticApproach2019} and searched the value of $\epsilon$ in [2, 6, 18], together with learning rate and weight decay. We select the best hyperparameters using validation set performance. 


\paragraph{Hyperparameters for MixMatch.}
Beyond the usual hyperparameters for our SSL framework, the key hyperparameters for MixMatch include the number of augmentations $K$, the temperature $T>0$ used for sharpening, interpolation hyperparameter $\alpha$ and unlabeled loss coefficient $\lambda$. We set $K=2$, $T=0.5$, and $\alpha=0.75$ as done in \citet{berthelotMixmatchHolisticApproach2019}, and search for $\lambda$ in the range [10, 30, 75, 100, 130] using validation set. 

\paragraph{Hyperparameters for Multitask training.}
We searched $\gamma$, the hyperparameter that control the strength of the auxilliary view loss in Eq.~\eqref{eq:multitask}, in the range [10, 3, 1, 0.3, 0.1]. The best $\alpha$ is selected together with other hyperparameters on validation set. 



\end{document}

