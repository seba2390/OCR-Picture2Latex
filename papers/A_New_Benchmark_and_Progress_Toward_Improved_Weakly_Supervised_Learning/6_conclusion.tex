% ==================================================
\section{Conclusion}

In this work, conventional training methods and model features have been
demonstrated to solve a previously unsolved task by training a black-box
model to solve the Pentomino problem.  The All-Pairs problem is
introduced as a challenge to the research community by measuring the
limits of conventional model performance, introducing a model
advancement (TypeNet) to solve such problems, and measuring the limits
of TypeNet on the All-Pairs and conventional image classification
benchmarks.

The following extensions to the TypeNet model and its training may
prove useful or generate further insights: (1) filtering the data
generator output to study supervised and unsupervised curriculum
learning, (2) generating multi-scale statistics before
the final \textbf{fully-connected} layers, (3) annealing a $softmax$-type
activation during training to help the network seek better minima,
and (4) using the TypeNet structure in a residual architecture by adding the
post-\textbf{superposition} block of features back to the \textbf{conv}
block.  The hope is to direct research toward valuable investigations
and to promote a methodology of falsifiable scientific claims both by
falsifying previous claims and by making further claims which, if we believe
Popper \cite{popper59}, are likely to be false.
