%%% Local Variables:
%%% mode: latex
%%% TeX-master: "0.knowledge_and_techniques_matter"
%%% End:

% ==================================================
\section{The All-Pairs Problem} \label{all_pairs}
% -----------------------------------
\begin{figure}[h!]
\vskip 0.2in
\begin{center}
\scalebox{0.6}{\centerline{\includegraphics[width=\columnwidth]{images/all_pairs_survey}}}
\caption{All-Pairs examples from 2-2 on the left to 8-8 on the right.  The bottom row is $true$ and the top row is $false$.}
\label{example8}
\end{center}
\vskip -0.2in
\end{figure}


\subsection{Definition and Examples}

Extending the ideas in the Pentomino problem, we use anti-aliased white
symbols on a black background to construct the following new problem.
The \textit{N-K} All-Pairs problem contains \textit{2N} symbols from an
alphabet of \textit{K} choices.  Each example is \textit{true} if each of its
symbols pairs with a symbol of the same type without reuse,
and \textit{false} otherwise.  Symbols are positioned randomly with no
overlap.  Symbols are of similar scales, ranging from
10--18 pixels across, and have differing symmetries (for instance, some
are rotationally invariant, while others are not).  The exact structure
and variations of each symbol are given by the generator code supplied
online \cite{allpairs2018}.

Each symbol is shown below with the
number of unique ways it can appear, as configured in our experiments.  In contrast, the
Pentomino problem used 8
variations for each symbol. The symbols are used in the order given, so the 4-4 All-Pairs problem will use \textbf{circle}, \textbf{line}, \textbf{cross}, and \textbf{angle}.  For this work a $76\TMS76$ image is used for $N<6$ and a
$96\TMS96$ image is used for larger $N$.

\begin{table}[htp]
  \begin{center}
\scalebox{0.9}{
\begin{tabular}{r r l r r r l r}
id & name & examples & cardinality &id & name & examples & cardinality \\
\hline
1 & \textbf{circle}        & \inlinegraphics{images/strip_0.png}  & 165   & 10 & \textbf{box}          & \inlinegraphics{images/strip_9.png}  & 480   \\
2 & \textbf{line}          & \inlinegraphics{images/strip_1.png}  & 174   & 11 & \textbf{box-diagonal} & \inlinegraphics{images/strip_10.png} & 518   \\
3 & \textbf{cross}         & \inlinegraphics{images/strip_2.png}  & 45.3k & 12 & \textbf{barbell}      & \inlinegraphics{images/strip_11.png} & 78    \\
4 & \textbf{angle}         & \inlinegraphics{images/strip_3.png}  & 39k   & 13 & \textbf{dot-line}     & \inlinegraphics{images/strip_12.png} & 156   \\
5 & \textbf{3-star}        & \inlinegraphics{images/strip_4.png}  & 1.43M & 14 & \textbf{z}            & \inlinegraphics{images/strip_13.png} & 518   \\
6 & \textbf{theta}         & \inlinegraphics{images/strip_5.png}  & 20k   & 15 & \textbf{triangle-lid} & \inlinegraphics{images/strip_14.png} & 1036  \\
7 & \textbf{phi}           & \inlinegraphics{images/strip_6.png}  & 20k   & 16 & \textbf{dot-mid-line} & \inlinegraphics{images/strip_15.png} & 78    \\
8 & \textbf{2-circle}      & \inlinegraphics{images/strip_7.png}  & 7k& 17 & \textbf{hourglass}    & \inlinegraphics{images/strip_16.png} & 518   \\
9 & \textbf{circle-3star}  & \inlinegraphics{images/strip_8.png}  & 7.15M& 18 & \textbf{triangle}     & \inlinegraphics{images/strip_17.png} & 11.8k \\
\end{tabular}
}
\end{center}
\label{default}
\end{table}
\vskip -0.2in

\noindent
Figure \ref{example8} shows a \textit{true} and a \textit{false} example
for the 2-2 to 8-8 All-Pairs problem.  A data generator for All-Pairs is
used to generate on-demand, unique training examples (the 4-4
All-Pairs problem has approximately $10^{28}$ unique images), and a fixed
validation set is generated at the start of training.  The separability of the eighteen
symbols was confirmed by training a simple conv-net to 100\% test accuracy in 350k training samples.

\subsection{Comparison with Conventional Results}
Conventional algorithms from the literature have difficulty with the 4-4
All-Pairs problem, as shown in the following table.  Clearly, of the
hundreds of conventional, valuable DNN algorithms, there may exist some
that can solve the 4-4 problem.  One open challenge is to identify them
and extend training techniques to efficiently solve these types of problems.
Of the runs of each algorithm summarized below, none achieved more than
92\% test accuracy after training on 100M samples.  An expert human made one mistake in 100 samples for each of the All-Pairs problem from difficulty 4-4 to 7-7, taking 8-9 seconds to classify each image.  Humans use sequential attention and working memory to do the All-Pairs task, suggesting the task as a benchmark for building sequential models.
TypeNet consistently achieves 100\% test accuracy in the 4-4 All-Pairs
problem using 20k test samples.  

\begin{center}
\begin{tabular}{ c c c c c }
 algorithm & model size & normalized size & accuracy & std deviation \\
\hline
 TypeNet [$\times$10] & 918k & 1.0 & \textbf{1.000} & \textbf{0.000} \\
 Expert Human [$\times$1] & -- & -- & \textbf{0.990} & -- \\
 Relational Net [$\times$10] & \textbf{630k} & \textbf{0.7} & 0.867 & 0.078 \\
 \ifarxiv
 ConvNet (\S\ref{comptocnn}) [$\times$4] & 9.9M & 11 & 0.808 & 0.093    \\
 \fi
 Inception v3 [$\times$10] & 22M & 24 & 0.803 & 0.079    \\
 Resnet-34 [$\times$10] & 21M & 23 & 0.788 & 0.068   \\
 Resnet-18 [$\times$10] & 11M & 12 & 0.711 & 0.157    \\
 Vgg19 [$\times$6] & 139M & 151 & 0.509 & 0.002    \\
 Vgg16 [$\times$3] & 134M & 146 & 0.506 & 0.002    \\
\end{tabular}
\end{center}
