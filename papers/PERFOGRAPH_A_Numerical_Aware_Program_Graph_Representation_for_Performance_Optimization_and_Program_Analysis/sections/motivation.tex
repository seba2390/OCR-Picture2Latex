\vspace{-10pt}
\section{Motivation}
\label{sec:motivation}
\vspace{-10pt}
As stated in the related work section, program representations based on the intermediate representation are very effective in enabling DL models to automate the process of various optimizations. One such representation is \programl, whose performance surpasses other code representations, making it state-of-the-art in various optimizations and downstream tasks. However, despite its potential, it suffers from several limitations. To name a few: it is incapable of properly carrying out information regarding read and write operations to the memory location, has no support for aggregate data types, and discards numerical values. Listing \ref{lst:pglimit} shows a code snippet where a 3-dimensional array is defined.
Figure \ref{figs:pglimit} shows the \programl representation of this code snippet. For illustration purposes, instruction nodes and control flow edges are shown in blue, whereas red represents variable, constant nodes, and data flow edges. Green edges show the call graph.
As it can be seen, \programl fails to represent some critical information. 
For instance, code \mintinline{cpp}|float arr[2][3][4]| is converted to LLVM IR \mintinline{cpp}|[2 x [3 x [4 x float]]]*|, which is used to construct a node in \programl.
It eliminates the aggregate data structure information, like the array's dimension. Leaving it up to the DL model to infer the meaning behind the numbers in \mintinline{cpp}|[2 x [3 x [4 x float]]]*|.
Moreover, in this representation, only the type of numbers (e.g., \texttt{int8}, \texttt{float}) are considered, and the actual values of the numbers are not given attention.
The absence of numerical awareness limits the performance of \programl in applications where numerical values play an important role. A numerically aware representation can help understand and optimize operations involving numeric data types, constants, and expressions. There are also some anomalies in the way temporary variables are depicted. For example, in \ref{figs:pglimit}, we see the fourth \texttt{alloca} node allocates memory for a variable, and two \texttt{store} instructions are applied on two separate nodes representing the variable. Thus, the information about the first \texttt{store} instruction is not carried out properly when the second \texttt{store} instruction happens. In the following section, we will see how \ourtool effectively addresses many limitations in the current state-of-the-art program representation. 
\ourtool uses \programl representation as its initial graph and reconstructs the graphs by addressing the aforementioned limitations.

\begin{minipage}{0.5\textwidth}
\begin{lstlisting}[language=C++,
                   numbers=left,
                   stepnumber=1,
                   numbersep=10pt,
                   showspaces=false,
                   basicstyle=\small,
                   showstringspaces=false,
                   captionpos=b,
                   xleftmargin=.04\textwidth, xrightmargin=.2\textwidth,
                   caption=C++ code example.,
                   label={lst:pglimit},]
#include <stdio.h>

int main(int argc, char *argv[]) {
  int arr_state = 0;
  float arr[2][3][4] 
    = {{{1.0f,1.0f,1.0f,1.0f}, 
        {2.0f,2.0f,2.0f,2.0f}, 
        {3.0f,3.0f,3.0f,3.0f}},
        {{4.0f,4.0f,4.0f,4.0f}, 
        {5.0f,5.0f,5.0f,5.0f}, 
        {6.0f,6.0f,6.0f,6.0f}}};
    arr_state = 1;
    return 0;
}
\end{lstlisting}
\vspace{-1.5cm}
\end{minipage}
\begin{minipage}{0.5\textwidth}
     \begin{figure}[H]
    \centering
     \includegraphics[width=\textwidth]{images/pg_limit_example_v2.pdf}
    %\includegraphics[scale=0.35]{figures/fig5.jpg}
    \caption{ProGraML representation of Listing \ref{lst:pglimit}.}
    \label{figs:pglimit}
    \end{figure}
\end{minipage}