\section{Introduction}
\label{sec:introduction}
\vspace{-8pt}
In recent years, the remarkable success of machine learning has led to transformative advancements across numerous fields, including compiler optimization and program analysis. 
The applications include compiler heuristics prediction, optimization decisions, parallelism detection, etc. ~\cite{allamanis2018survey, cummins2022compilergym}.
The training process generally involves feeding program data as input and transforming it into a representation suitable for machine learning models. The selection of program representation is crucial, as it can significantly impact the performance of the machine learning model \cite{chen2019literature}.
With the development of graph neural networks (GNNs), an increasing number of graph representations of programs have been incorporated into GNN models for program analysis~\cite{allamanis2017learning, li2019graph, chen2022multi}.
One of the pioneering efforts in developing a comprehensive graph representation for programs is \programl ~\cite{cummins2020programl}.
\programl incorporates control, data, and call dependencies as integral components of a program's representation.
In contrast to prior sequential learning systems for code, \programl closely resembles the intermediate representations used by compilers, and the propagation of information through these graphs mimics the behavior of typical iterative data-flow analyses.
Despite the success that \programl has achieved, there are shortcomings in this current state-of-the-art program representation, especially for performance-oriented downstream tasks.
These limitations stem from neglecting numerical values available at compile time and the inadequate representation of aggregate data types.

In this paper, we present \ourtool to address the limitations of the current state-of-the-art program representation.
Additionally, we propose a novel way to embed numbers in programs in an elegant way so that our DL model will not face unknown numbers during inference time. Our experiments demonstrate that \ourtool sets new state-of-the-art results in numerous downstream tasks.
For example, in the Device Mapping downstream task, \ourtool yield error rates as low as 6\% and 10\% depending on the target hardware.
Moreover, \ourtool even outperforms the tools and models specially designed for specific tasks such as parallelism discovery.  

\vspace{-1pt}
Overall, the main contributions of this paper are:
\vspace{-3pt}
\begin{itemize}[leftmargin=15pt, itemsep=0pt, parsep=3pt, topsep=0pt]
    \item An enhanced compiler and language-agnostic program representation based on \programl that represents programs as graphs.
    \item The proposed representation supports aggregate data types and provides numerical awareness, making it highly effective for performance optimization tasks.
    %\item Evaluation of the proposed representation on common downstream tasks and outperforming state-of-the-art representations such as \programl.
    \item Evaluation of the proposed representation on common downstream tasks and exceeding the performance of \programl.
    \item Quantification of the proposed approach on a new set of downstream tasks such as parallelism discovery and configuration of NUMA systems.
\end{itemize}
\vspace{-2pt}

The rest of the paper is structured as follows: Section \ref{sec:related_works} presents the related works. In section \ref{sec:motivation}, we provide a motivational example, showing the limitations of \programl, the state-of-the-art program representation. This section is followed by section \ref{sec:approach} where we present our proposed representation \ourtool along with the novel way of embedding numerical values. In section \ref{sec:results}, experimental results on downstream tasks are provided, and finally, Section \ref{sec:conclusion} concludes the paper and discusses some future works.

\vspace{-8pt}