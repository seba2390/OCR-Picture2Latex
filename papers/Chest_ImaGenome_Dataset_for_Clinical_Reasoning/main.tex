\documentclass{article}

% if you need to pass options to natbib, use, e.g.:
%     \PassOptionsToPackage{numbers, compress}{natbib}
% before loading neurips_data_2021

\PassOptionsToPackage{square,comma,numbers,sort&compress}{natbib}

% ready for submission
\usepackage[preprint]{neurips_data_2021}

% to compile a preprint version, add the [preprint] option:
%     \usepackage[preprint]{neurips_data_2021}
% This will indicate that the work is currently under review.

% to compile a camera-ready version, add the [final] option:
%     \usepackage[final]{neurips_data_2021}

% to avoid loading the natbib package, add option nonatbib:
%    \usepackage[nonatbib]{neurips_data_2021}

% Submissions to the datasets and benchmarks are non-anonymous. If you do want to compile an anonymous version for other purposes, you can add the [anonymous] option:
%     \usepackage[anonymous]{neurips_data_2021}
% This will hide all author names.

\usepackage{wrapfig}
\usepackage[utf8]{inputenc} % allow utf-8 input
\usepackage[T1]{fontenc}    % use 8-bit T1 fonts
\usepackage{hyperref}       % hyperlinks
\usepackage{url}            % simple URL typesetting
\usepackage{booktabs}       % professional-quality tables
\usepackage{amsfonts}       % blackboard math symbols
\usepackage{nicefrac}       % compact symbols for 1/2, etc.
\usepackage{microtype}      % microtypography
\usepackage{algorithm2e}
\usepackage{hyperref}
\usepackage{graphicx}
%\usepackage{float}
\usepackage{multirow}
\usepackage[table,xcdraw]{xcolor}
\usepackage{subcaption}

\renewcommand{\topfraction}{1}
\renewcommand{\dbltopfraction}{1}
\renewcommand{\floatpagefraction}{1}
\renewcommand{\textfraction}{0}
\newcommand{\hideseg}[1]{} %hide

\usepackage{xspace}
%COMMENTS FOR ISMINI
\newcommand{\il}[1]{\textsf{\textbf{\color{magenta}{\small{[IL: #1]}}}}}
%COMMENTS FOR MEHDI
\newcommand{\mm}[1]{\textsf{\textbf{\color{blue}{\small{[MM: #1]}}}}}
\newcommand*{\myalign}[2]{\multicolumn{1}{#1}{#2}}


\title{Chest ImaGenome Dataset for Clinical Reasoning}

% \author[1,*]{Joy T. Wu}
% \author[2]{Nkechinyere N. Agu}
% \author[1,3]{Ismini Lourentzou}
% \author[4,$\dag$]{Arjun Sharma} 
% \author[5,$\dag$]{Joseph Paguio}
% \author[5,$\dag$]{Jasper Seth Yao}
% \author[6,$\dag$]{Edward Christopher Dee}
% \author[4,$\dag$]{William Mitchell}
% \author[1]{Satyananda Kashyap}
% \author[1]{Andrea Giovannini}
% \author[4]{Leo A. Celi}
% % & any others who helped
% \author[1,*]{Mehdi Moradi} 
% % order tbd -- happy to discuss

% \affil[1]{IBM Almaden Research Center, San Jose, CA 95120, USA}
% \affil[2]{Rensselaer Polytechnic Institute, Troy, NY 12180, USA}
% \affil[3]{Virginia Polytechnic Institute and State University, Blacksburg, VA 24061, USA}
% \affil[4]{MIT Critical Data, Cambridge, MA 02139, USA}
%\affil[5]{Albert Einstein Healthcare Network-Philadelphia Campus, PA 19141, USA}
% \affil[6]{MIT Harvard Medical School, Boston, MA 02115, USA}

% \affil[*]{Corresponding author(s): Joy T. Wu (research@joytywu.net), Mehdi Moradi (mmoradi@us.ibm.com)}

% \affil[$\dag$]{These authors contributed equally to this work}

% The \author macro works with any number of authors. There are two commands
% used to separate the names and addresses of multiple authors: \And and \AND.
%
% Using \And between authors leaves it to LaTeX to determine where to break the
% lines. Using \AND forces a line break at that point. So, if LaTeX puts 3 of 4
% authors names on the first line, and the last on the second line, try using
% \AND instead of \And before the third author name.

% \author[1,*]{Joy T. Wu}
% \author[2]{Nkechinyere N. Agu}
% \author[1,3]{Ismini Lourentzou}
% \author[4,$\dag$]{Arjun Sharma} 
% \author[5,$\dag$]{Joseph Alexander Paguio}
% \author[5,$\dag$]{Jasper Seth Yao}
% \author[6,$\dag$]{Edward Christopher Dee}
% \author[4,$\dag$]{William Mitchell}
% \author[1]{Satyananda Kashyap}
% \author[1]{Andrea Giovannini}
% \author[4]{Leo A. Celi}
% % & any others who helped
% \author[1,*]{Mehdi Moradi} 

%\hide{\thanks{Use footnote for providing further information
%    about author (webpage, alternative address)---\emph{not} for acknowledging funding agencies.}}

\author{Joy T. Wu\textsuperscript{1}, Nkechinyere N. Agu\textsuperscript{2}, Ismini Lourentzou\textsuperscript{3}, Arjun Sharma\textsuperscript{4}, Joseph A. Paguio\textsuperscript{5}, \and \textbf{Jasper S. Yao\textsuperscript{5}, Edward C. Dee\textsuperscript{6}, William Mitchell\textsuperscript{4}, Satyananda Kashyap\textsuperscript{1},} \and \textbf{Andrea Giovannini\textsuperscript{1}, Leo A. Celi\textsuperscript{4}, Mehdi Moradi\textsuperscript{1}} \\
\textsuperscript{1}{IBM Almaden Research Center, San Jose, CA 95120, USA}\\
\textsuperscript{2}{Rensselaer Polytechnic Institute, Troy, NY 12180, USA}\\
\textsuperscript{3}{Virginia Polytechnic Institute and State University, Blacksburg, VA 24061, USA}\\
\textsuperscript{4}{MIT Critical Data, Cambridge, MA 02139, USA}\\
\textsuperscript{5}{Albert Einstein Healthcare Network-Philadelphia Campus, PA 19141, USA}\\
\textsuperscript{6}{Harvard Medical School, Boston, MA 02115, USA} \\
  % examples of more authors
  % \And
  % Coauthor \\
  % Affiliation \\
  % Address \\
  % \texttt{email} \\
  % \AND
  % Coauthor \\
  % Affiliation \\
  % Address \\
  % \texttt{email} \\
  % \And
  % Coauthor \\
  % Affiliation \\
  % Address \\
  % \texttt{email} \\
  % \And
  % Coauthor \\
  % Affiliation \\
  % Address \\
  % \texttt{email} \\
}


\begin{document}

\maketitle

\begin{abstract}
  %In recent years, with the release of multiple large datasets, the automatic interpretation of chest X-ray (CXR) images with deep learning models have become feasible for specific abnormalities or for generating preliminary reports. However,   reports of performance reaching similar levels to that of radiologists, 
Despite the progress in automatic detection of radiologic findings from chest X-ray (CXR) images in recent years, a quantitative evaluation of the explainability of these models is hampered by the lack of locally labeled datasets for different findings. With the exception of a few expert-labeled small-scale datasets for specific findings, such as pneumonia and pneumothorax, most of the CXR deep learning models to date are trained on global "weak" labels extracted from text reports, or trained via a joint image and unstructured text learning strategy. Inspired by the Visual Genome effort in the computer vision community, we constructed the first Chest ImaGenome dataset with a scene graph data structure to describe $242,072$ images. Local annotations are automatically produced using a joint rule-based natural language processing (NLP) and atlas-based bounding box detection pipeline. Through a radiologist constructed CXR ontology, the annotations for each CXR are connected as an anatomy-centered scene graph, useful for image-level reasoning and multimodal fusion applications. Overall, we provide: i) $1,256$ combinations of relation annotations between $29$ CXR anatomical locations (objects with bounding box coordinates) and their attributes, structured as a scene graph per image, ii) over $670,000$ localized comparison relations (for improved, worsened, or no change) between the anatomical locations across sequential exams, as well as ii) a manually annotated gold standard scene graph dataset from $500$ unique patients.

%In our work, a joint rule-based natural language processing (NLP) and CXR atlas-based bounding box detection pipeline are used to automatically label 242072 frontal MIMIC CXRs locally. 
\end{abstract}

Reinforcement learning has achieved great success in areas such as Game-playing \citep{silver2018general,vinyals2019grandmaster}, robotics \cite{kober2013reinforcement}, large language models \citep{ouyang2022training}, etc.
However, due to safety concerns or physical limitations, in some real-world reinforcement learning problems, we must consider additional constraints that may influence the optimal policy and the learning process \citep{garcia2015comprehensive}.
% For example, a robotic arm must not take actions that may cause harm to itself or the environments.
A standard framework to handle such cases is the constrained Markov Decision Process (CMDP) \citep{altman1999constrained}.
Within the CMDP framework, the agent has to maximize
the expected cumulative reward while
obeying a finite number of constraints, which are usually in the form of expected cumulative cost criteria.

However, we are sometimes concerned with the problem with a continuum of constraints.
For example,
the constraints we meet might be time-evolving or subject to uncertain parameters, which
cannot be formulated as an ordinary CMDP
(see Examples \ref{Example_Time_Evolving} and  \ref{Example_Uncertain}).
In this paper we would study a generalized CMDP  
to address the above problem.  Because the constraints are not only infinite-number but also lie
in a continuous set,
the generalization is not trivial. Fortunately, we find that we can borrow the idea behind semi-infinite programming (SIP) \citep{remez1934determination, hettich1993semi} to deal with the semi-infinite constraints.
Accordingly, we propose \emph{semi-infinitely constrained Markov decision processes} (SICMDPs)
as a novel complement to the ordinary CMDP framework.
%More specifically,  an SICMDP model %, we consider 
%contains a continuum of constraints whereas an ordinary CMDP contains a finite number of constraints. 

%This generalization is natural but not trivial. However, we can brows the idea  
%The idea is quite natural and can be backtracked
%to the practice of extending linear programming to linear semi-infinite programming (LSIP) %\cite{remez1934determination, GobernaLSIO1998}.
%In addition, 
%As a complementary approach to the ordinary CMDP framework, 
%SICMDP can be used to model these problems  which cannot be described by a finite number of constraints
%that are not covered by .
%For example,
%the restrictions we consider can be time-evolving or subject to uncertain parameters
%, thus
%cannot be described by a finite number of constraints but a continuum of constraints 
%(see Examples \ref{Example_Time_Evolving} and  \ref{Example_Uncertain}).

We also present two reinforcement learning algorithms to solve SICMDPs called SI-CRL and SI-CPO, respectively.
SI-CRL is a model-based reinforcement learning algorithm designed for tabular cases, and SI-CPO is a policy optimization algorithm for non-tabular cases.
% and analyze its performance both theoretically and empirically.
The main challenge is that we need to deal with a continuum of constraints, thus reinforcement learning algorithms for ordinary CMDPs do not work anymore.
In SI-CRL, we tackle this difficulty by first transforming the reinforcement learning problem to an equivalent LSIP problem, which can then be solved using methods in the LSIP literature like the dual exchange methods \citep{Hu1990,reemtsen1998numerical}.
In SI-CPO, we resort to the idea of cooperative stochastic approximation developed in \cite{lan2020algorithms, wei2020comirror}.
As far as we know, we are the first to introduce tools from semi-infinitely programming (SIP) into the reinforcement learning community for solving constrained reinforcement learning problems.

% To the best of our knowledge, we are the first to apply tools from semi-infinitely programming (SIP) to solve reinforcement learning problems.
Furthermore, we give theoretical analysis for both SI-CRL and SI-CPO.
We decompose the error of SI-CRL into two parts: the statistical error from approximating the true SICMDP with an offline dataset and the optimization error due to the fact that the solution of the LSIP problem obtained by the dual exchange method is inexact.
On the optimization side, we show that the iteration complexity of SI-CRL is $O\left(\left\{\mathrm{diam}(Y)L\sqrt{|\gS|^2|\gA|m}/\left[(1-\gamma)\epsilon\right]\right\}^m\right)$.
On the statistical side, we show that the sample complexity of SI-CRL is $\widetilde O\left(\frac{|S|^2|A|^2}{\epsilon^2(1-\gamma)^3}\right)$ if the offline dataset is generated by a generative model, and $\widetilde O\left(\frac{|S||A|}{\nu_{\min} \epsilon^2(1-\gamma)^3}\right)$ if the dataset is generated by a probability measure $\nu$ as considered in \cite{chen2019information}.
Here $\widetilde O$ means that all logarithm terms are discarded.
For SI-CPO, things become a little more complicated because other than the statistical error and the optimization error, we also need to consider the function approximation error, which comes from imperfect policy parametrizations.
It is shown if the function approximation error can be controlled to $O(\epsilon)$ order, the iteration complexity of SI-CPO is $\widetilde{O}\left(\frac{1}{\epsilon^2(1-\gamma)^6}\right)$ and the sample complexity of SI-CPO is $\widetilde{O}(\frac{1}{\epsilon^4(1-\gamma)^{10}})$.
Here our iteration complexity bound is equivalent to a typical $\widetilde O(1/\sqrt{T})$ global convergence rate.

We perform a set of numerical experiments to illustrate the SICMDP model and validate our proposed algorithms.
Specifically, we examine two numerical examples, namely the discharge of sewage and ship route planning.
Through the discharge of sewage example, we show the advantage of the SICMDP framework over the CMDP baseline obtained by naive discretization in modeling realistic sequential decision-making problems.
Moreover, we demonstrate the effectiveness of the SI-CRL and SI-CPO algorithms in such tabular environments. 
In the ship route planning example, we illustrate the benefits of the SICMDP framework and the ability of the SI-CPO algorithm to address complex continuous control tasks involving continuous state spaces with modern deep reinforcement learning techniques.

% In summary, our contributions are listed as follows.
% First, we present the SICMDP model, which can be viewed as a generalization of the ordinary CMDP model.
% Second, we propose an algorithm to perform reinforcement learning for SICMDPs, which is called SI-CRL, and we believe that we are the first to apply tools from SIP
% to solve reinforcement learning problems.
% Third, we give a theoretical analysis of SI-CRL and identify both its sample complexity and iteration complexity.
% In addition, we perform numerical experiments to illustrate the SICMDP model and validate the SI-CRL algorithm.
% \{This paragraph can be removed!!! \}





\textbf{Related work}:
% Object detection related datasets/algo in non-medical domain
% Locally labeled CXR dataset
A few CXR datasets have localized abnormality annotations \cite{shih2019augmenting,filice2020crowdsourcing,jaeger2014two} that are curated manually. These are high quality gold standard ground truth datasets but tend to be smaller in scale (< 30,000 images) and have a narrow coverage, with typically only 1-2 labels. In addition, since most labeling efforts only have abnormality semantics attached, no direct relationships with the affected anatomical locations are available. 

%MEHDI: repeated concepts from above. I am removing the following: 

%The lack of anatomic semantics in the annotation is a limitation for complex multi-modal clinical reasoning work, e.g., differential diagnosis, since clinicians often integrate information along anatomical lines, and for downstream report generation tasks, which often requires describing not only the abnormality but also correctly communicate the location of the abnormalities (and medical devices) to the receiving clinicians. 

Two recent CXR datasets have labels for anatomies described in the reports. In \cite{datta2020dataset}, a small manually annotated dataset (2000 reports) included 10 abnormalities that are individually associated with 29 unique spatial locations (anatomies) at the report level. Another CXR dataset has automatically extracted abnormality and anatomy labels as disconnected concepts that are only correlated at the study level from  160,000 reports using a supervised NLP algorithm \cite{bustos2020padchest}. This was trained on a smaller set of manually annotated data. Neither datasets contain localized annotations for the associated CXR images, nor any comparison relation annotations between sequential exams, both of which are available in the Chest ImaGenome dataset. In Table \ref{tab:related}, we present a comparison of our Chest ImagGenome dataset with other datasets available in the literature.

% Table -- Kashyap

% MEdical imaging datasets to go here: Discussed that we will only focus on cxr datasets that are available for this paper. 
% \caption{\color{red} Kashyap, feel free to continue with the table. We should remove the questionmarks and add a line for our dataset (since all others are not graph). For longer text, using abbreviations and explaining them in the caption often works better. If fill in the values is not possible, it is better to remove the table altogether.}


\begin{table}[t!]
\caption{Summary of existing chest X-ray datasets}
\resizebox{\textwidth}{!}{%
\begin{tabular}{@{}lllllllll@{}}
\toprule
\textbf{Dataset} & \textbf{Annotation Level} & \textbf{Annotation Method} & \textbf{Num Labels} & \textbf{Anatomy Labeled} & \textbf{Graph} & \textbf{Dataset Size} & \textbf{Temporal Labels} & \textbf{Reports} \\ \midrule
SIIM-ACR Pneumothorax Segmentation \cite{filice2020crowdsourcing} & Segmentation & Manual + augmented & 1 & No & No & 12,047 & No & No \\
RSNA Pneumonia Detection Challenge   \cite{shih2019augmenting} & Bounding Boxes & Manual & 1 & No & No & 30,000 & No & No \\
Indiana University Chest X-ray collection \cite{demner2016preparing} & Global & Automated & 10 & No & No & 3,813 & No & Yes \\
NIH CXR dataset \cite{wang2017chestx} & Global & Automated & 14 & No & No & 112,120 & No & No \\
PLCO \cite{team2000prostate} & Global & Automated & 24 & Yes & No & 236,000 & Yes & No \\
Stanford CheXpert \cite{irvin2019chexpert} & Global & Automated & 14 & No & No & 224,316 & No & No \\
MIMIC-CXR \cite{johnson2019mimic} & Global & Automated & 14 & No & No & 377,110 & No & Yes \\
Dutta \cite{datta2020dataset} & Global & Manual & 10 & Yes & Yes & 2,000 & No & Yes \\
PadChest \cite{bustos2020padchest} & Global & Manual + automated & 297 & Yes & No & 160,868 & No & Yes \\
Montgomery County Chest X-ray   \cite{jaeger2014two} & Segmentation & Manual & 1 & Yes & No & 138 & No & No \\
Shenzen Hospital Chest X-ray   \cite{jaeger2014two} & Segmentation & Manual & 1 & Yes & No & 662 & No & No \\  \hline \hline
\textbf{Chest ImaGenome} & Bounding Boxes & Automated & 131 & Yes & Yes & 242,072 & Yes & Yes \\
\bottomrule
\end{tabular}%
}
\label{tab:related}
\vspace{-0.4cm}
\end{table}
% removed (Derived from MIMIC-CXR \cite{johnson2019mimic}) % makes table really small

The proposed segmentation-by-detection framework, as depicted in Figure \ref{fig:framework}, consists of a detection module and a segmentation module.
In detection stage, 2D slices (layered box) from the input volume are fed to the RPN. Based on the region proposals obtained from RPN, an attention model (block in orange) is formed. The input volume as well as the attention model are further processed in segmentation stage to get the refined anatomical segmentation. 
\vspace{1em} 

\begin{figure}[t]
\centering
\includegraphics[width=0.95\linewidth]{fig/framework.pdf}
\caption{Schematic representation of the segmentation-by-detection framework. The left part is the detection module while the segmentation module is followed on the right. The blue block denotes the input volume which is 3D ultrasound scan of femoral head. The output segmentation is in red.}
\label{fig:framework}
\end{figure}
% dana could you improve the figure. we can try to think together of better ways 

\noindent\textbf{Detection Module:} 
% dana : here you have to make the clarification that you have ground truth on the boxes (in implementation part)
The detection module follows an RPN architecture, a fully convolutional network which takes image slice as input and outputs object region candidates. 
We use the VGG-16 model as the backbone \cite{simonyan2014very} to learn convolutional features and an $3 \times 3$ spatial window to generate region proposals. At each sliding-window location, 9 anchors are predicted associated with different scales and aspect ratios. The last layer consists of a box-regression (reg) layer and a box-classification (cls) layer in parallel. The reg layer outputs 4 regression offsets, $ t = (t_x,t_y,t_w,t_h)$, denoting a scale-invariant translation as well as log-space height and width shift, where $x,y,w$ and $h$ specify two coordinates of the box center, width and height. The cls layer outputs two scores by softmax, related to probabilities of object and background for each proposal. We assign a positive label (of being object) to candidate which has an Intersection-over-Union (IoU) ratio higher than 0.7 with ground truth box. Note that an image slice may contain multiple object regions or none. 

The loss function of RPN follows the multi-task loss \cite{ren2015faster} which is defined as $L = L_{reg} + L_{cls}$. The regression loss, $L_{reg} = -\log p_{obj}$ is log loss and the classification loss,
\begin{equation} \label{eq:loss}
L_{cls} = \sum_{i \in \{x,y,w,h\}} smooth_{L_1} (t_i - t_i^*)
\end{equation}
is smooth $L_1$ loss where $t_i^*$ denotes the ground truth box for the target object. 
\vspace{1em}

\noindent\textbf{Segmentation Module:}
3D U-Net \cite{cciccek20163d} is utilized in the segmentation module as its outstanding performance in medical image segmentation. The u-shaped architecture consists of two paths: a contracting path, where each layer contains two $3\times3\times3$ convolutions followed by a rectified linear unit (ReLU) and then a max pooling, provides high resolution features. While, the symmetric expanding path for semantically richer features replaces max pooling with a upconvolution $2\times2\times2$ with stride of 2 in each dimension, and then two $3\times3\times3$ convolutions each followed by a ReLU. Skip connections between layers of equal resolution in the contracting path and the expanding path enables context information as well as precise localization.

Different from 3D U-Net, to incorporate the attention model detected by the RPN, our architecture takes as input both the volumetric image data and the candidate RoIs proposed by the RPN, concatenated as 3D volume. 
% dana not sure what you like to say below
% densely annotated
The attention model makes the network to focus on the potential RoIs and can reduce the interference of the surrounding noise.
The anatomical segmentation is then generated from a $1\times1\times1$ convolution which reduces the number of feature maps to the number of labels.  The energy function is computed by a pixel-wise softmax combined with the cross entropy loss.
% dana equation ??

\subsection{System and implementation Details}
The segmentation-by-detection approach adopts a cascade structure with two stages: detection and segmentation. The two networks are trained separately in an end-to-end manner. All the new layers are randomly initialized from zero-mean Gaussian distribution with standard deviations 0.01. Biases are initialized to 0. We use Caffe \cite{jia2014caffe} for the implementation and an NVIDIA Titan X GPU for training.

In the detection stage, we initialize the VGG-16 model by the pre-trained model for ImageNet classification \cite{russakovsky2015imagenet} and further fine-tune the model for our detection task. The input fed to the network are image slices with a fixed size of $184\times96$ and the corresponding ground truth boxes are generated from the annotation in the format of tight bounding boxes surrounding the segmentation contour (as illustrated in Figure \ref{fig:hip} (b), the boundary of white area). To optimize the energy function, stochastic gradient descent (SGD) is used. The global learning rate is set to 0.001, while a momentum of 0.9 and a weight decay of 0.0005 are used. The batch size is set to 256 and each mini-batch only contains the positive anchors for training. The region proposals are obtained from the reg path for each image slice. The attention model is then formed by concatenating all the detected regions, as binary masks, into a volume.

In the segmentation stage, we use the Adam optimizer \cite{kingma2014adam} to learn the network parameters. A global learning rate is set to 0.001 while the two momentum coefficients are set to 0.9 and 0.999 respectively. A batch size of 1 is used due to the memory constraints of the GPU. The network takes the volume data as well as the attention model as input. We train the network for a maximum of 30K iterations and reserve the learned weights with the best performance from every 1K iterations. 
\vspace{1em}

\noindent\textbf{Inference:}
At test time, the 2D slices from an input volume are first fed to the detection module. The attention model is obtained based on the output. Then the volume data as well as the attention model are fed to the segmentation module to get the pixel-wise prediction.



\section*{Data description}

The Chest ImaGenome dataset is committed to the PhysioNet repository in two main directories, one for the scene graphs that are automatically generated (``silver\_dataset''), and another for the 500 unique patient subset that was manually validated and corrected (``gold\_dataset''). Overall, $242,072$ scene graphs were automatically derived from $217,013$ unique CXR studies. The nodes and edges in the graph are defined in detail in Supplementary Table \ref{tab:define_nodes_edges}. On average 7 anatomical objects and 5 attributes are extracted from each study report. However, up to 29 anatomy objects can be detected in each CXR image with a percentage of misses < 0.02\% for most objects (See Table \ref{tab:object-detect} in Supplementary material). In addition, even without considering the related attribute(s), $678,543$ object-object comparison relations are extracted between anatomies across $128,468$ pairs of sequential CXR images. Detailed dataset characteristics are explained and provided in the PhysioNet repository (generate\_scenegraph\_statistics.ipynb). Figure \ref{fig:bbox-sample} shows an example of all the anatomical bounding boxes.

\vspace{-5pt}
\subsection*{Chest ImaGenome Scene Graph JSONs}
\vspace{-2pt}
%\noindent \textbf{Chest ImaGenome Scene Graph JSONs}: 
The `silver\_dataset/scene\_graph.zip' file is a directory that contains multiple JSON files, one for each scene graph. Each scene graph describes one frontal chest X-ray image. The structure for each scene graph JSON is described by components for easier explanation in Supplementary (Section \ref{jsonsg}). The first level of the JSON in Supplementary (\ref{json1}) describes the patient or study level information that may not be available in the image. The fields are: `image\_id' (dicom\_id in MIMIC-CXR), `viewpoint' (AP or PA), `patient\_id' (subject\_id in MIMIC-CXR), `study\_id' (study\_id in MIMIC-CXR), `gender' and `age\_decile' demographics (from MIMIC-CXR's metadata), `reason for exam' (patient history sentence(s) from the CXR reports with age removed), `StudyOrder' (the order of the CXR study for the patient, which is derived from chronologically ordering the DICOM timestamps), and `StudyDateTime; (from MIMIC's dicom metadata, which had been de-identified into the future).

% \hideseg{
% \vspace{-0.3cm}
% \begin{footnotesize}
% \begin{verbatim}
% {
%  `chest_imageimage_id': `10cd06e9-5443fef9-9afbe903-e2ce1eb5-dcff1097',
%  `viewpoint': `AP', `patient_id': 10063856, `study_id': 56759094,
%  `gender': `F', `age_decile': `50-60',
%  `reason_for_exam': `___F with hypotension.  Evaluate for pneumonia.',
%  `StudyOrder': 2, `StudyDateTime': `2178-10-05 15:05:32 UTC',
%  `objects': [ <...list of {} for each object...> ],
%  `attributes':[ <...list of {} for each object...> ],
%  `relationships':[ <...list of {} of comparison relationships between objects 
%  from sequential exams for the same patient...> ] 
% }
% \end{verbatim}
% \end{footnotesize}
% }

For each scene graph, there are 3 separate nested fields to describe the ``objects'' on the CXR images, the ``attributes'' related to the different objects as extracted from the corresponding reports, and ``relationships'' to describe comparison relations between sequential CXR images for the same patient. These 3 fields are a list of dictionaries, where the format of each dictionary is modeled after the respective JSONs in the Visual Genome dataset \cite{krishna2017visual}.

For objects, each dictionary has the format shown in Supplementary (\ref{json2}). The `object\_id' is unique across the whole dataset for the anatomical location on the particular image. Fields `x1', `y1', `x2', `y2', `width' and `height' are for a padded and resized 224x224 CXR frontal image, where coordinates `x1', `y1' are for the top left corner of the bounding box and `x2', `y2' are for the bottom right corner. The bounding box coordinates in the original image are denoted with `original\_*'. The remaining fields: `bbox\_name' is the name given to the anatomical location within the Chest ImaGenome dataset, and is useful for lookups in other parts of the scene graph JSON; `synsets' contain the UMLS CUI for the anatomical location concept; and the `name' is the UMLS name for that CUI \cite{bodenreider2004unified}. Note that CXRs are 2D images of a 3D structure so there are many overlying anatomical locations. A sample of 17 of the anatomical objects is plotted on a CXR as shown in Figure \ref{fig:bbox-sample}.
\vspace{-5pt}

\begin{figure}[!ht]
\centering
\includegraphics[scale=0.3]{figures/Figure_6_lung_mediastinum_clavicle_bboxes.pdf}
\caption{Sample CXR case with 17 overlaying clavicles, lung and mediastinum related anatomical bounding boxes (objects).}
\label{fig:bbox-sample}
\vspace{-12pt}
\end{figure}

% \hideseg{
% \vspace{-0.3cm}
% \begin{footnotesize}
% \begin{verbatim}
% {
%   `object_id': `10cd06e9-5443fef9-9afbe903-e2ce1eb5-dcff1097_right upper lung zone',
%   `x1': 48, `y1': 39, `x2': 111, `y2': 93,
%   `width': 63, `height': 54,
%   `bbox_name': `right upper lung zone',
%   `synsets': [`C0934570'],
%   `name': `Right upper lung zone',
%   `original_x1': 395, `original_y1': 532,
%   `original_x2': 1255, `original_y2': 1268,
%   `original_width': 860, `original_height': 736
% }
% \end{verbatim}
% \end{footnotesize}
% }

Each attribute dictionary, e.g., Supplementary  (\ref{json3}), aims to summarize all the CXR attribute descriptions for one anatomical location (`bbox\_name'). This means, for a particular CXR anatomical location, all the sentences describing attributes related to it have been grouped into the `phrases' field, where the order of sentences in the original report has been maintained. However, an anatomical location may not always be described or implied in the report. In that case, looking up dictionary[`bbox\_name'] will be False. The fields `synsets' and `name' are the same as in the objects' dictionaries, where they describe the UMLS CUI information for the anatomical location concept.
% \hideseg{
% \vspace{-0.3cm}
% \begin{footnotesize}
% \begin{verbatim}
% {
%   `right lung': True, `bbox_name': `right lung',
%   `synsets': [`C0225706'], `name': `Right lung',
%   `attributes': [[`anatomicalfinding|no|lung opacity',
%   `anatomicalfinding|no|pneumothorax',  `nlp|yes|normal'],
%   [`anatomicalfinding|no|pneumothorax']],
%   `attributes_ids': [[`CL556823', `C1963215;;C0032326', `C1550457'],
%   [`C1963215;;C0032326']],
%   `phrases': [`Right lung is clear without pneumothorax.', 
%   `No pneumothorax identified.'],
%   `phrase_IDs': [`56759094|10', `56759094|14'],
%   `sections': [`finalreport', `finalreport'],
%   `comparison_cues': [[], []],
%   `temporal_cues': [[], []],
%   `severity_cues': [[], []],
%   `texture_cues': [[], []],
%   `object_id': `10cd06e9-5443fef9-9afbe903-e2ce1eb5-dcff1097_right lung'
% }
% \end{verbatim}
% \end{footnotesize}
% }

The `attributes' field contains the relations between the anatomical location and the CXR attributes extracted from the respective sentences. Note that there can be multiple attributes extracted from each sentence. Therefore, the `attributes' field is a list of lists. The `attributes' in the lists follow the pattern of < categoryID | relation | label\_name >, where `categoryID' is the radiology semantic category the authors gave to the CXR concept in consultation with multiple radiologists, and relation is the NLP context relating the label\_name to the anatomical location as an attribute. If the relation is `no', then the `label\_name' is specifically negated in the sentence. If the relation is 'yes', then the `label\_name` is affirmed in the sentence. The order of the lists in the `attribute\_ids' field follow the lists in the `attributes' field and map each `label\_name' to UMLS CUIs. Thus, the way the Chest ImaGenome dataset is formulated, one can interpret a statement such as the `right lung' <has no> `lung opacity' as true in the extracted radiology knowledge graph, whereby each node has been mapped to an externally recognized ontology. 

The certainty of each relation in the CXR knowledge graph can be optionally further modified by the cues from the `severity\_cues' and `temporal\_cues' fields in each attribute dictionary. The severity cues can include `hedge', `mild', `moderate' or `severe', which are only assigned by co-occurrence at the sentence level. These extractions can benefit from future NLP improvement. Similarly, the temporal cues can modify the relation as either `acute' or `chronic' depending on clinical use cases.

The Chest ImaGenome categoryIDs can be used to differentiate the use case for different attributes:

$\bullet$ \textbf{anatomicalfinding} - findings of anatomies where there is some subjectivity in the grouping of the phrases used to extract the labels.
\vspace{-2pt}

$\bullet$ \textbf{disease} - descriptions that are more diagnostic level and often require patient information outside the image and most subjective to the reading radiologist's inference/impression.
\vspace{-2pt}

$\bullet$ \textbf{nlp} - normal / abnormal descriptions about different anatomical locations and can be subjective.
\vspace{-2pt}

$\bullet$ \textbf{technicalassessment} - image quality issues affecting interpretation of CXR observations.
\vspace{-2pt}

$\bullet$ \textbf{tubesandlines} - medical support devices where radiologists need to report any placement issues.
\vspace{-2pt}

$\bullet$ \textbf{devices}: medical devices where placement issues are less relevant
\vspace{-2pt}

$\bullet$ \textbf{texture} - these are only present in the 'texture\_cues' field, we kept a set of highly non-specific attributes (e.g. opacity, lucency, interstitial, airspace) that tend to form the initial most objective descriptions about what is observed in the images by radiologists. 

Finally, for comparison relationships, each dictionary has the format shown in Supplementary (\ref{json4}). Each relationship dictionary describes the comparison relation(s) relevant for only one anatomical location (`bbox\_name'). The `relationship\_id' uniquely identifies each comparison relationship between the object (`subject\_id') on the current exam and the object (`object\_id' for the same anatomical location) from the previous exam. The `predicate' and `synsets' are the UMLS CUIs for `relationship\_names', which is a list with usually one (but could be more) comparison relation type, which can be in [`comparison|yes|improved', `comparison|yes|worsened', `comparison|yes|no change']. The `attributes' field records the attributes that are related to the anatomical location as per the sentence from the original report (kept in the `phrase' field) that describes the comparison relationship.

% \hideseg{
% \vspace{-0.3cm}
% \begin{footnotesize}
% \begin{verbatim}
% {
%   `relationship_id': `56759094|7_54814005_C0929215_10cd06e9_4bb710ab',
%   `predicate': ``['No status change']'',
%   `synsets': [`C0442739'],
%   `relationship_names': [`comparison|yes|no change'],
%   `relationship_contexts': [1.0],
%   `phrase': `Compared with the prior radiograph, there is a persistent veil 
%   -like opacity\n over the left hemithorax, with a crescent of air surrounding 
%   the aortic arch,\n in keeping with continued left upper lobe collapse.',
%   `attributes': [`anatomicalfinding|yes|atelectasis',
%   `anatomicalfinding|yes|lobar/segmental collapse',
%   `anatomicalfinding|yes|lung opacity', `nlp|yes|abnormal'],
%   `bbox_name': `left upper lung zone',
%   `subject_id': `10cd06e9-5443fef9-9afbe903-e2ce1eb5-dcff1097_left upper lung zone',
%   `object_id': `4bb710ab-ab7d4781-568bcd6e-5079d3e6-7fdb61b6_left upper lung zone'
% }
% \end{verbatim}
% \end{footnotesize}
% }

% Not all the sentences in the MIMIC-CXR v2.0.0 reports have made it into the Chest ImaGenome dataset, which only contains sentences that have the specific objects, attributes or relations targeted by version 1.0.0 of the dataset. We provide the preprocessing steps (Preprocess_mimic_cxr_v2.0.0_reports.ipynb) done to index the sentences from the original text reports in the "utils" directory, the output of which is cxr-mimic-v2.0.0-processed-sentences_all.txt.

\vspace{-5pt}
\subsection*{CXR Scene Graphs Rendered in an Enriched RDF Format}
\vspace{-2pt}
%\textbf{CXR Scene Graphs Rendered in an enriched RDF Format}
Supplementary (\ref{json5}):
Radiology report sentences are fairly repetitive. Therefore, in the scene graph JSONS, one could see similar information described multiple times in different sentences for a study. In addition, in the MIMIC reports we worked with, each report could also have a preliminary read section (recorded by trainee radiologists - i.e., resident M.D.s) that comes before the final report section (approved by a fully trained and experienced radiologist). Therefore, occasionally, the extraction from the sentences near the beginning of a CXR report can be different from the conclusion sentences later in the report. To render the scene graphs easier for downstream utilization, we also provide post-processing utils (scenegraph\_postprocessing.py) to roll the annotations up to the study level for each relation. This is done by taking the last relation extracted for each anatomical location and attribute combinations for a report. The processing utils can either render the scene graphs in a tabular format or represent the information in a simpler enriched RDF format, which we used to generate the graph visualizations in Figure \ref{fig1.cxr_graph}. 

% \hideseg{
% \vspace{-0.3cm}
% \begin{footnotesize}
% \begin{verbatim}
% {
%  <study_id_i> : [
%                   [[node_id_1, node_type_1], [node_id_2, node_type_2], relation_name_A],
%                   [[node_id_1, node_type_1], [node_id_3, node_type_3], relation_name_B],
%                     ...
%                 ],
%  <study_id_i+1>:[
%                   [[node_id_1, node_type_1], [node_id_2, node_type_2], relation_name_A],
%                   [[node_id_1, node_type_1], [node_id_3, node_type_3], relation_name_B],
%                     ...
%                 ],
% }   
% \end{verbatim}
% \end{footnotesize}
% }

\vspace{-5pt}
\subsection*{Gold Standard Dataset Tables}
\vspace{-2pt}
%\noindent \textbf{Gold standard dataset tables}: 
We curated a manual gold standard evaluation dataset to measure the quality of the automatically derived annotations in the Chest ImaGenome dataset and for model benchmarking. Here we describe the three gold standard ground truth files in the ``gold\_dataset'' directory. They are in tabular format for ease of comparison purposes.

$\bullet$  \textit{\textbf{gold\_attributes\_relations\_500pts\_500studies1st.txt}} is the ground truth file which contains 21,594 object-to-attribute relations manually annotated for 3,042 sentences from the \textit{first} CXR study for 500 unique patients. The notebook `object-attribute-relation\_evaluation.ipynb' explains in detail how we it to calculate the performance of object-to-attribute relation extraction.

$\bullet$  \textit{\textbf{gold\_comparison\_relations\_500pts\_500studies2nd.txt}} is the ground truth file which contains 5,156 object-object (per attribute) comparison relations for 638 sentences from the \textit{second} CXR study for the same 500 unique patients. The notebook `object-object-comparison-relation\_evaluation.ipynb' uses it to calculate the performance for object-to-object-comparison relation extraction.

$\bullet$  The four \textit{\textbf{bbox\_coordinate\_annotations*.csv}} files contain the manually annotated bounding box coordinates for the objects on the corresponding 1,000 unique CXR images. The notebook `object-bbox-coordinates\_evaluation.ipynb' calculates the bounding box object detection performance using these ground truth files.

$\bullet$ Lastly, \textit{\textbf{final\_merging\_report\_and\_bbox\_ground\_truth.ipynb}} combines the manual text and anatomical bbox annotations as \textit{\textbf{gold\_object\_attribute\_with\_coordinates.txt}} and \textit{\textbf{gold\_object\_comparison\_with\_coordinates.txt}}.

Additional supporting files for measuring the performance of the silver dataset against the gold standard are described in Supplementary (Section \ref{gold_supp}):

% % Can also put this in supplementary material
% \noindent \textbf{gold\_all\_sentences\_500pts\_1000studies.txt} contains all the sentences tokenized from the original MIMIC-CXR reports that were used to create the gold standard dataset. We include this file because sentences with no relevant object, attribute or relation descriptions did not make it into the gold standard dataset. We renamed `subject\_id' from MIMIC-CXR dataset to `patient\_id' in Chest ImaGenome dataset to avoid confusion with field names for relationships in the scene graphs. Otherwise, the ids are unchanged. Sentences in the tokenized file are assigned to `history', `prelimread', or `finalreport' in the `section' column. The `sent\_loc' column contains the order of the sentences as in the original report. Minimal tokenization has been done to the sentences.

% \noindent \textbf{gold\_bbox\_scaling\_factors\_original\_to\_224x224.csv} contains the scaling `ratio' and the paddings (`left', `right', `top', and `bottom') added to square the image after resizing the original MIMIC-CXR dicoms to 224x224 sizes. These ratios were used to rescale the annotated coordinates for 224x224 images back to the original CXR image sizes.

% \noindent \textbf{auto\_bbox\_pipeline\_coordinates\_1000\_images.txt'} contains the bounding box coordinates that were automatically extracted by the Bbox pipeline for the different objects for images in the gold standard dataset. It is in tabular format like with the ground truth for easier evaluation purposes.

%!TEX root = main.tex
\section{Evaluation}
\label{sec:eval}

In this section, we evaluate the performance of our unsupervised Ordered Word Mover's Distance metric and supervised Multi-scale Sentence Matching model with factorized sentences as input. We apply our algorithms to semantic textual similarity estimation tasks and sentence pair paraphrase identification tasks, based on four datasets: STSbenchmark, SICK, MSRP and MSRvid. 

\subsection{Experimental Setup}
\label{subsec:setup}


\begin{table}[tb]
  \caption{Description of evaluation datasets.}
  \label{tab:datasets}
  \begin{tabular}{lllll}
    \toprule
    Dataset & Task & Train & Dev & Test\\
    \midrule
    STSbenchmark & Similarity scoring & $5748$ & $1500$ & $1378$ \\
    SICK & Similarity scoring & $4500$ & $500$ & $4927$ \\
    MSRP & Paraphrase identification & $4076$ & - & $1725$ \\
    MSRvid & Similarity scoring & $750$ & - & $750$ \\
    \bottomrule
  \end{tabular}
  \vspace{-2mm}
\end{table}

We will start with a brief description for each dataset:
\begin{itemize}
\item \textbf{STSbenchmark}\cite{cer2017semeval}: it is a dataset for semantic textual similarity (STS) estimation. The task is to assign a similarity score to each sentence pair on a scale of 0.0 to 5.0, with 5.0 being the most similar.

\item \textbf{SICK}\cite{marelli2014sick}: it is another STS dataset from the SemEval 2014 task 1. It has the same scoring mechanism as STSbenchmark, where 0.0 represents the least amount of relatedness and 5.0 represents the most.

\item \textbf{MSRvid}: the Microsoft Research Video Description Corpus contains 1500 sentences that are concise summaries on the content of a short video. Each pair of sentences is also assigned a semantic similarity score between 0.0 and 5.0. 

\item \textbf{MSRP}\cite{quirk2004monolingual}: the Microsoft Research Paraphrase Corpus is a set of 5800 sentence pairs collected from news articles on the Internet. Each sentence pair is labeled 0 or 1, with 1 indicating that the two sentences are paraphrases of each other.
\end{itemize}

Table \ref{tab:datasets} shows a detailed breakdown of the datasets used in evaluation.
For STSbenchmark dataset we use the provided train/dev/test split.
The SICK dataset does not provide development set out of the box, so we extracted 500 instances from the training set as the development set.
For MSRP and MSRvid, since their sizes are relatively small to begin with, we did not create any development set for them.

One metric we used to evaluate the performance of our proposed models on the task of semantic textual similarity estimation is the Pearson Correlation coefficient, commonly denoted by $r$. Pearson Correlation is defined as:
\begin{equation}
\label{eq:pearson}
 r = cov(X,Y) /( \sigma_X \sigma_Y),
\end{equation}
where $cov(X,Y)$ is the co-variance between distributions X and Y, and $\sigma_X$, $\sigma_Y$ are the standard deviations of X and Y.
The Pearson Correlation coefficient can be thought as a measure of how well two distributions fit on a straight line. Its value has range [-1, 1], where a value of 1 indicates that data points from two distribution lie on the same line with a positive slope.
% Due to this unique property, we believe the Pearson Correlation coefficient is a strong indicator of the performance of our metric. 

Another metric we utilized is the Spearman's Rank Correlation coefficient. Commonly denoted by $r_s$, the Spearman's Rank Correlation coefficient shares a similar mathematical expression with the Pearson Correlation coefficient, but it is applied to ranked variables.
Formally it is defined as \cite{wiki:spearman}:
\begin{equation}
\label{eq:spearman}
 \rho = cov(rg_X, rg_Y) / (\sigma_{rg_X} \sigma_{rg_Y}),
\end{equation}
where $rg_X$, $rg_Y$ denotes the ranked variables derived from $X$ and $Y$. $cov(rg_X,rg_Y)$, $\sigma_{rg_X}$, $\sigma_{rg_Y}$ corresponds to the co-variance and standard deviations of the rank variables. The term ranked simply means that each instance in X is ranked higher or lower against every other instances in X and the same for Y. We then compare the rank values of X and Y with \ref{eq:spearman}. Like the Pearson Correlation coefficient, the Spearman's Rank Correlation coefficient has an output range of [-1, 1], and it measures the monotonic relationship between X and Y. A Spearman's Rank Correlation value of 1 implies that as X increases, Y is guaranteed to increase as well.
The Spearman's Rank Correlation is also less sensitive to noise created by outliers compared to the Pearson Correlation.

For the task of paraphrase identification, the classification accuracy of label $1$ and the F1 score are used as metrics. 

In the supervised learning portion, we conduct the experiments on the aforementioned four datasets. We use training sets to train the models, development set to tune the hyper-parameters and each test set is only used once in the final evaluation. For datasets without any development set, we will use cross-validation in the training process to prevent overfitting, that is, use $10\%$ of the training data for validation and the rest is used in training. For each model, we carry out training for 10 epochs. We then choose the model with the best validation performance to be evaluated on the test set.  


\subsection{Unsupervised Matching with OWMD}
\label{subsec:eval-owmd}

To evaluate the effectiveness of our Ordered Word Mover's Distance metric, we first take an unsupervised approach towards the similarity estimation task on the STSbenchmark, SICK and MSRvid datasets. Using the distance metrics listed in Table \ref{tab:compare-pearson} and \ref{tab:compare-spearman}, we first computed the distance between two sentences, then calculated the Pearson Correlation coefficients and the Spearman's Rank Correlation coefficients between all pair's distances and their labeled scores. We did not use the MSRP dataset since it is a binary classification problem.


In our proposed Ordered Word Mover's Distance metric, distance between two sentences is calculated using the order preserving Word Mover's Distance algorithm. For all three datasets, we performed hyper-parameter tuning using the training set and calculated the Pearson Correlation coefficients on the test and development set. We found that for the STSbenchmark dataset, setting $\lambda_1=10$, $\lambda_2=0.03$ produces the most optimal result. For the SICK dataset, a combination of $\lambda_1=3.5$, $\lambda_2=0.015$ works best. And for the MSRvid dataset, the highest Pearson Correlation is attained when $\lambda_1=0.01$, $\lambda_2=0.02$.
We maintain a max iteration of 20 since in our experiments we found that it is sufficient for the correlation result to converge.
During hyper-parameter tuning we discovered that using the Euclidean metric along with $\sigma=10$ produces better results, so all OWMD results summarized in Table \ref{tab:compare-pearson} and \ref{tab:compare-spearman} are acquired under these parameter settings. Finally, it is worth mentioning that our OWMD metric calculates the distances using factorized versions of sentences, while all other metrics use the original sentences. Sentence factorization is a necessary preprocessing step for the OWMD metric.


We compared the performance of Ordered Word Mover's Distance metric with the following methods:

\begin{itemize}
\item \textbf{Bag-of-Words (BoW)}: in the Bag-of-Words metric, distance between two sentences is computed as the cosine similarity between the word counts of the sentences.

\item \textbf{LexVec}~\cite{salle2016enhancing}: calculate the cosine similarity between the  averaged 300-dimensional LexVec word embedding of the two sentences. 

\item \textbf{GloVe}~\cite{pennington2014glove}: calculate the cosine similarity between the averaged 300-dimensional GloVe 6B word embedding of the two sentences. 

\item \textbf{Fastext}~\cite{joulin2016bag}: calculate the cosine similarity between the averaged 300-dimensional Fastext word embedding of the two sentences. 

\item \textbf{Word2vec}~\cite{mikolov2013efficient}: calculate the cosine similarity between the averaged 300-dimensional Word2vec word embedding of the two sentences.

\item \textbf{Word Mover's Distance (WMD)}~\cite{kusner2015word}: estimating the semantic distance between two sentences by WMD introduced in Sec.~\ref{sec:owmd}.
\end{itemize} 


\begin{table}[tb]
  \caption{Pearson Correlation results on different distance metrics.}
  \label{tab:compare-pearson}
  \begin{tabular}{c|cc|cc|c}
    \toprule
    \multirow{2}{*}{Algorithm} & \multicolumn{2}{c}{STSbenchmark} & \multicolumn{2}{c}{SICK} & MSRvid\\ 
     & Test & Dev & Test & Dev & Test\\
    \midrule
    BoW & $0.5705$ & $0.6561$ & $0.6114$ & $0.6087$ & $0.5044$ \\
    LexVec & $0.5759$ & $0.6852$ & $0.6948$ & $\mathbf{0.6811}$ & $0.7318$\\
    GloVe & $0.4064$ & $0.5207$ & $0.6297$ & $0.5892$  & $0.5481$ \\
    Fastext & $0.5079$ & $0.6247$ & $0.6517$ & $0.6421$  & $0.5517$  \\
    Word2vec & $0.5550$ & $0.6911$ & $\mathbf{0.7021}$ & $0.6730$  & $0.7209$  \\
    WMD & $0.4241$ & $0.5679$ & $0.5962$ & $0.5953$  & $0.3430$  \\
    OWMD & $\mathbf{0.6144}$ & $\mathbf{0.7240}$ & $0.6797$ & $0.6772$  & $\mathbf{0.7519}$  \\
    \bottomrule
  \end{tabular}
  \vspace{-4mm}
\end{table}

\begin{table}[tb]
  \caption{Spearman's Rank Correlation results on different distance metrics.}
  \label{tab:compare-spearman}
  \begin{tabular}{c|cc|cc|c}
    \toprule
    \multirow{2}{*}{Algorithm} & \multicolumn{2}{c}{STSbenchmark} & \multicolumn{2}{c}{SICK} & MSRvid\\ 
     & Test & Dev & Test & Dev & Test\\
    \midrule
    BoW & $0.5592$ & $0.6572$ & $0.5727$ & $0.5894$ & $0.5233$ \\
    LexVec & $0.5472$ & $0.7032$ & $0.5872$ & $0.5879$ & $0.7311$\\
    GloVe & $0.4268$ & $0.5862$ & $0.5505$ & $0.5490$  & $0.5828$ \\
    Fastext & $0.4874$ & $0.6424$ & $0.5739$ & $0.5941$  & $0.5634$  \\
    Word2vec & $0.5184$ & $0.7021$ & $0.6082$ & $0.6056$  & $0.7175$  \\
    WMD & $0.4270$ & $0.5781$ & $0.5488$ & $0.5612$  & $0.3699$  \\
    OWMD & $\mathbf{0.5855}$ & $\mathbf{0.7253}$ & $\mathbf{0.6133}$ & $\mathbf{0.6188}$  & $\mathbf{0.7543}$  \\
    \bottomrule
  \end{tabular}
  \vspace{-2mm}
\end{table}


Table \ref{tab:compare-pearson} and Table \ref{tab:compare-spearman} compare the performance of different metrics in terms of the Pearson Correlation coefficients and the Spearman's Rank Correlation coefficients.
We can see that the result of our OWMD metric achieves the best performance on all the datasets in terms of the Spearman's Rank Correlation coefficients.
It also produced the best Pearson Correlation results on the STSbenchmark and the MSRvid dataset, while the performance on SICK datasets are close to the best.
This can be attributed to the two characteristics of OWMD. First, the input sentence is re-organized into a predicate-argument structure using the sentence factorization tree. Therefore, corresponding semantic units in the two sentences will be aligned roughly in order. Second, our OWMD metric takes word positions into consideration and penalizes disordered matches. Therefore, it will produce less mismatches compared with the WMD metric.

% On the SICK dataset, although the result of our metric falls slightly behind Word2vec, LexVec on the test set and Word2vec on the development set, we still believe that it is a superior metric because it produced competitive results across multiple datasets. 

% Table \ref{tab:compare-spearman} presents the Spearman's Rank Correlation coefficients acquired with the same distance metrics. We can observe that our OWMD metric achieves the highest correlation scores on all three datasets. Which proves once again that OWMD is a better distance metric for the task of semantic similarity detection.

\subsection{Supervised Multi-scale Semantic Matching}
\label{subsec:eval-multilayer}

\begin{table*}[tb]
  \caption{A comparison among different supervised learning models in terms of accuracy, F1 score, Pearson's $r$ and Spearman's $\rho$ on various test sets.}
  \label{tab:sts}
  \begin{tabular}{c|cc|cc|cc|cc}
    \toprule
    \multirow{2}{*}{Model} & \multicolumn{2}{c}{MSRP} & \multicolumn{2}{c}{SICK} & \multicolumn{2}{c}{MSRvid} & \multicolumn{2}{c}{STSbenchmark}\\ 
     & Acc.(\%) & F1(\%) & $r$ & $\rho$ & $r$ & $\rho$ & $r$ & $\rho$ \\
    \midrule
    MaLSTM & $66.95$ & $73.95$ & $0.7824$ & $0.71843$ & $0.7325$ & $0.7193$ & $0.5739$ & $0.5558$\\
    Multi-scale MaLSTM & $\mathbf{74.09}$ & $\mathbf{82.18}$ & $\mathbf{0.8168}$ & $\mathbf{0.74226}$ & $\mathbf{0.8236}$ & $\mathbf{0.8188}$ & $\mathbf{0.6839}$ & $\mathbf{0.6575}$\\
    \midrule
    HCTI & $73.80$ & $80.85$ & $0.8408$ & $0.7698$ & $\mathbf{0.8848}$ & $\mathbf{0.8763}$  & $\mathbf{0.7697}$ & $\mathbf{0.7549}$ \\
    Multi-scale HCTI & $\mathbf{74.03}$ & $\mathbf{81.76}$ & $\mathbf{0.8437}$ & $\mathbf{0.7729}$ & $0.8763$ & $0.8686$  & $0.7269$ & $0.7033$  \\
    \bottomrule
  \end{tabular}
  \vspace{-2mm}
\end{table*}

The use of sentence factorization can improve both existing unsupervised metrics and existing supervised models. 
% We extend the normal Siamese model to Fig. \ref{fig:network} to take advantage of different level of information in the factorized sentence. 
To evaluate how the performance of existing Siamese neural networks can be improved by our sentence factorization technique and the multi-scale Siamese architecture, we implemented two types of Siamese sentence matching models, HCTI \cite{mueller2016siamese} and MaLSTM \cite{shao2017hcti}. HCTI is a Convolutional Neural Network (CNN) based Siamese model, which achieves the best Pearson Correlation coefficient on STSbenchmark dataset in SemEval2017 competition (compared with all the other neural network models). MaLSTM is a Siamese adaptation of the Long Short-Term Memory (LSTM) network for learning sentence similarity. As the source code of HCTI is not released in public, we implemented it according to \cite{shao2017hcti} by Keras \cite{chollet2015keras}. With the same parameter settings listed in paper \cite{shao2017hcti} and tried our best to optimize the model, we got a Pearson correlation of 0.7697 (0.7833 in paper \cite{shao2017hcti}) in STSbencmark test dataset.

We extended HCTI and MaLSTM to our proposed Siamese architecture in Fig. \ref{fig:network}, namely the Multi-scale MaLSTM and the Multi-scale HCTI. To evaluate the performance of our models, the experiment is conducted on two tasks: 1) semantic textual similarity estimation based on the STSbenchmark, MSRvid, and SICK2014 datasets; 2) paraphrase identification based on the MSRP dataset.

Table \ref{tab:sts} shows the results of HCTI, MaLSTM and our multi-scale models on different datasets. Compared with the original models, our models with multi-scale semantic units of the input sentences as network inputs significantly improved the performance on most datasets. 
Furthermore, the improvements on different tasks and datasets also proved the general applicability of our proposed architecture.

Compared with MaLSTM, our multi-scaled Siamese models with factorized sentences as input perform much better on each dataset. For MSRvid and STSbenmark dataset, both Pearson's $r$ and Spearman's $\rho$ increase about $10\%$ with Multi-scale MaLSTM. Moreover, the Multi-scale MaLSTM achieves the highest accuracy and F1 score on the MSRP dataset compared with other models listed in Table \ref{tab:sts}.

There are two reasons why our Multi-scale MaLSTM significantly outperforms MaLSTM model. First, for an input sentence pair, 
we explicitly model their semantic units with the factorization algorithm.
%we explicitly model the different scales of semantics of them with the semantic units produced by our sentence factorization algorithm. 
Second, our multi-scaled network architecture is 
specifically designed
%specially adapted to 
for multi-scaled sentences representations. Therefore, it is able to explicitly match a pair of sentences at different granularities.

We also report the results of HCTI and Multi-scale HCTI in Table \ref{tab:sts}. For the paraphrase identification task, our model shows better accuracy and F1 score on MSRP dataset. For the semantic textual similarity estimation task, the performance varies across datasets. On the SICK dataset, the performance of Multi-scale HCTI is close to HCTI with slightly better Pearson' $r$ and Spearman's $\rho$. However, the Multi-scale HCTI is not able to outperform HCTI on MSRvid and STSbenchmark. HCTI is still the best neural network model on the STSbenchmark dataset, and the MSRvid dataset is a subset of STSbenchmark.
Although HCTI has strong performance on these two datasets, it performs worse than our model on other datasets.
% Overall, the experimental results demonstrated the superior applicability and generalizability of our proposed models.
Overall, the experimental results demonstrated the general applicability of our proposed model architecture, which performs well on various semantic matching tasks.

% \begin{table}[tb]
%   \caption{Results of Accuracy and F1 score on MSRP test dataset.}
%   \label{tab:MSRP result}
%   \begin{tabular}{lllll}
%     \toprule
%     Model & Acc.(\%) & F1(\%)  \\
%     \midrule
%     MaLSTM & $66.95$ & $73.95$ \\
%     Factorized MaLSTM & $\mathbf{74.09}$ & $\mathbf{82.18}$ \\
%     HCTI & $73.80$ & $80.85$ \\
%     Factorized HCTI & $\mathbf{74.03}$ & $\mathbf{81.76}$ \\
%     \bottomrule
%   \end{tabular}
%   \vspace{0mm}
% \end{table}


% \begin{table}[tb]
%   \caption{Results of Pearson's $r$ and Spearman's $\rho$ on SICK test dataset.}
%   \label{tab:SICK result}
%   \begin{tabular}{lllll}
%     \toprule
%     Model & r & $\rho$ \\
%     \midrule
%     MaLSTM & $0.7824$ & $0.71843$ \\
%     Factorized MaLSTM & $\mathbf{0.8168}$ & $\mathbf{0.74226}$ \\
%     HCTI & $0.8408$ & $\mathbf{0.7698}$ \\
%     Factorized HCTI & $\mathbf{0.8429}$ & $0.7676$ \\
%     \bottomrule
%   \end{tabular}
%   \vspace{0mm}
% \end{table}


% \begin{table}[tb]
%   \caption{Results of Pearson's $r$ and Spearman's $\rho$ on MSRvid test dataset.}
%   \label{tab:MSRvid result}
%   \begin{tabular}{lll}
%     \toprule
%     Model & r & $\rho$  \\
%     \midrule
%     MaLSTM & $0.7325$ & $0.7193$ \\
%     Factorized MaLSTM & $\mathbf{0.8236}$ & $\mathbf{0.8188}$ \\
%     HCTI & $\mathbf{0.8848}$ & $\mathbf{0.8763}$ \\
%     Factorized HCTI & $0.8763$ & $0.8686$ \\
%     \bottomrule
%   \end{tabular}
%   \vspace{0mm}
% \end{table}



% \begin{table}[tb]
%   \caption{Results of Pearson's $r$ and Spearman's $\rho$ on STSbenchmark test dataset.}
%   \label{tab:STSbenchmark result}
%   \begin{tabular}{lllll}
%     \toprule
%     Model & r & $\rho$ \\
%     \midrule
%     MaLSTM & $0.5739$ & $0.5558$ \\
%     Factorized MaLSTM & $\mathbf{0.6839}$ & $\mathbf{0.6575}$ \\
%     HCTI & $\mathbf{0.7697}$ & $\mathbf{0.7549}$ \\
%     Factorized HCTI & $0.7269$ & $0.7033$ \\
%     \bottomrule
%   \end{tabular}
%   \vspace{0mm}
% \end{table}




%\section*{Usage Notes}
\vspace{-5pt}
\section*{Clinical Applications}
% \textbf{Clinical applications}: 
\vspace{-5pt}
There are numerous clinical topics that may be explored for a dataset that links anatomic structures with individual abnormalities and simultaneously provides comparison relation annotations for sequential images. Monitoring the progression of pathologies that are visualized through chest imaging is the most unexplored clinical application of this dataset. In the in-patient setting, diagnosis and monitoring of pneumonia are typically performed through comparisons of sequential CXR images from admission\cite{kalil2016management}. The same management principle may apply to the evaluation of the progression of other diseases, such as pneumothorax, pulmonary edema, acute respiratory distress syndrome, or congestive heart failure \cite{henry2003bts, cardinale2014effectiveness, rubenfeld2012acute}. In the outpatient setting, surveillance of incidental pulmonary nodules, malignancies, tuberculosis, or interstitial lung disease is done through chest imaging in several-month intervals \cite{gould2013evaluation, koo2019chest, nahid2016official, hansell2015ct}. Furthermore, the methodological concepts of this dataset could be extended to other modes of imaging, such as computed tomography (CT), and magnetic resonance (MR) imaging, etc, further expanding the potential clinical utility of this project.


\textbf{Consistent dataset splits for performance reporting}: For reproducibility, we include splits for train, valid and test sets in the ``silver\_dataset/splits'' directory. The random data split was done at the patient level. We also included a file (images\_to\_avoid.csv) with image IDs (`dicom\_id') and `study\_id's for patients in the gold standard dataset, which should all be excluded from training and validation. 
%We expect all final benchmark reporting to be done on both the test set in the silver dataset and the manually annotated gold standard dataset.

As described, Chest ImaGenome has been constructed with multiple possible downstream tasks in mind. Here, we showcase two example tasks that can have the most immediate clinical applications, (i) outputting both the location and the type of CXR attribute for an image (Example Task 2) and (ii) comparing whether a location has worsened or improved across sequential exams (Example Task 1). Clinically, the two chosen types of tasks are the two most important ones for radiologists to report when interpreting CXRs. 


\textbf{Example Task 1: Change between sequential CXR exams.} CXRs are commonly repeatedly requested in the clinical workflow to assess for a myriad of attributes. Given a patient with sequential CXRs, the goal of this task is to automatically evaluate disease change over time based on two sequential CXR exams. We restricted the problem to a subset of the Chest ImaGenome dataset, i.e., to attributes related to congestive heart failure (CHF), as fluid management is one of the most routine clinical tasks for which CXRs can be ordered to guide the next steps (e.g. whether to give more intravenous fluid or give diuretics, etc). However, we note that users of this dataset can also explore comparison changes for other CXR attributes (e.g. pneumonia). Each CXR image is also associated with a bounding box that marks a localized area, e.g., ``left lung'' for specific anatomical finding (i.e., attribute), such as ``pulmonary edema/hazy opacity'', etc. In addition, the pair of CXR images is mapped to the comparison label that indicates whether the condition of the anatomical finding has improved or worsened. As a baseline example, we focus on change relations in the 'left lung' and 'right lung' objects that are related to the `pulmonary edema/hazy opacity' and `fluid overload/heart failure' attributes. The number of examples labeled in the training, validation and test data are $10,515$, $1,493$ and $2,987$, respectively. 
%is summarized in Table \ref{tab:change_dataset}. 
%Note that we also include a separate small gold standard sample that is validated by subject matter experts.
We design a siamese architecture (Figure \ref{fig:siamese} in Supplementary \ref{clinical_applications}) that first extracts the localized bounding box from each image and encodes the extracted image patches with a pre-trained ResNet101 autoencoder, denoted that is trained on several medical imaging datasets, e.g., NIH, CheXpert, and MIMIC datasets, etc. \cite{irvin2019chexpert,johnson2019mimic,wang2017chestx}. The autoencoder image representations are concatenated and passed through a dense layer with 128 neurons and ReLU activations, and a final classification layer. 
%The model architecture is implemented with TorchXRayVision \cite{Cohen2020xrv} and PyTorch Lighting \cite{falcon2019pytorch}. 
We train for $300$ epochs with cross-entropy, stochastic gradient descent, $1e-3$ learning rate, $0.1$ gradient clipping and $32$ batch size. We freeze the autoencoder weights and finetune the two last dense layers. On this challenging task of predicting change in localized anatomical findings between two sequential exams, we achieve an accuracy of $75.3\%$. %and $71.43\%$ on the test and gold test sets, respectively. 

 \textbf{Example Task 2: Localization of CXR attributes.} Knowing the anatomical location of non-specific findings/attributes on CXR images can help with narrowing down possible disease diagnoses and guide the next steps in requesting more specific imaging exams or treatment. To this end, we train a Faster R-CNN model \cite{ren2015faster} %using detectron2 \cite{wu2019detectron2}
to learn 18 anatomical locations within the dataset. We extract the 1024 dimension convolution feature vector of each anatomical region. We re-implement the state-of-the-art CheXGCN model \cite{chen2020label} %that uses a Graph Convolutional Network (GCN) model 
to learn the dependencies between attributes within the Chest X-ray. 
%In particular, the convolutions are replaced with the Faster R-CNN model. 
Similar to the work done by CheXGCN we model the correlation of the CXR attributes using a conditional probability (see Figure \ref{fig:gcn} in Supplementary \ref{clinical_applications}). We compare the results of the model with two baseline models, a Faster R-CNN model followed by a linear model without the GCN, and a Densenet model \cite{huang2017densely} without the Faster R-CNN to evaluate the effectiveness of the localized models. We focus on 9 common CXR attributes, which include lung opacity, pleural effusion, atelectasis, enlarged cardiac silhouette, pulmonary edema/hazy opacity, pneumothorax, consolidation, fluid overload/heart failure, pneumonia. The results of the experiments are shown in Table \ref{tab:attr_results} and the labels are ordered according to the attribute list above. 


\begin{table}[t!]
\centering
\caption{Anatomically localized CXR attribute detection (AUC scores). L1: Lung Opacity, L2: Pleural Effusion, L3: Atelectasis, L4: Enlarged Cardiac Silhouette, L5: Pulmonary Edema/Hazy Opacity, L6: Pneumothorax, L7: Consolidation, L8: Fluid Overload/Heart Failure, L9: Pneumonia.}
\resizebox{\textwidth}{!}{
\begin{tabular}{p{3cm}*{10}{p{0.8cm}}}
\toprule
Method & L1 &  L2 & L3 & L4 & L5 & L6 & L7 & L8 & L9 & \textbf{AVG}  \\
\midrule
Faster R-CNN & 0.84 & 0.89 & 0.77 & 0.85 & 0.87 & 0.77 & 0.75 & 0.81 & 0.71 & 0.80\\
GlobalView & \textbf{0.91} & \textbf{0.94} & 0.86 & 0.92 & 0.92 & \textbf{0.93} & 0.86 & 0.87 & 0.84 & 0.89\\
CheXGCN & 0.86 & 0.90 & \textbf{0.91} & \textbf{0.94} & \textbf{0.95} & 0.75 & \textbf{0.89} & \textbf{0.98} & \textbf{0.88} & \textbf{0.90}\\
\bottomrule 
\end{tabular}
}
\label{tab:attr_results}
\vspace{-15pt}
\end{table}

\hideseg{
\begin{table}[t!]
\centering
\caption{Change relation experiment: dataset statistics.}
%\resizebox{0.3\linewidth}{!}{%
\begin{tabular}{ccc} %p{0.45\linewidth}p{0.20\linewidth}p{0.20\linewidth}
\toprule
Data & \#Worsened & \#Improved \\ \midrule
\textbf{Train} & 5,802 & 4,713 \\
\textbf{Validation} & 808 & 685 \\
\textbf{Test} & 1,638 & 1,349 \\
%\textbf{Test (Gold)} & 51 & 47 \\
\bottomrule 
\end{tabular}
%}
\label{tab:change_dataset}
\vspace{-0.2cm}
\end{table}
}

\textbf{Dataset Limitations}: The Chest ImaGenome dataset came from only one U.S. hospital source. It is automatically generated and is limited by the performance of the NLP and the Bbox extraction pipelines. Furthermore, we cannot assume that all the clinically relevant CXR attributes are always described on every exam by the reporting radiologists. In fact, we have observed many implied object-attribute relation descriptions that are documented only in the form of comparisons (e.g. no change from previous) in short CXR reports. As such, even with perfect NLP extraction of object and attribute relations from individual reports, there would be missing information in the report knowledge graph constructed for some images. These technical areas are worth improving on in future research with more powerful NLP, image processing techniques and other graph-based techniques. Addressing missing relations will certainly improve this dataset too. Regardless, version 1.0.0 of the Chest ImaGenome dataset serves as a pioneering vision for a richer radiology imaging dataset.
% %\begin{figure*}[htb]
    \begin{multicols}{2}
        \begin{lstlisting}[language=C++, basicstyle=\scriptsize\ttfamily, frame=leftline]
#include <torch/torch.h>


struct FloatNetImpl : torch::nn::Module{
  FloatNetImpl() : linear(10, 2){ 
    register_module("linear", linear);
  }

  torch::Tensor forward(torch::Tensor x){
    x = linear(x);
    return torch::sigmoid(x);
  }

  
  
  
  
  
  torch::nn::Linear linear;
};
TORCH_MODULE(FloatNet);
        \end{lstlisting}
        \columnbreak
        \begin{lstlisting}[language=C++, basicstyle=\scriptsize\ttfamily, frame=leftline, tabsize=2]
#include <positnn/positnn>

template <typename P>
struct PositNet : Layer<P>{
  PositNet() : linear(10, 2){
    this->register_module(linear);
  }

  StdTensor<P> forward(StdTensor<P> x){
    x = linear.forward(x);
    return sigmoid.forward(x);
  }

  StdTensor<P> backward(StdTensor<P> x){
    x = sigmoid.backward(x);
    return linear.backward(x);
  }

  Linear<P> linear;
  Sigmoid<P> sigmoid;
};
        \end{lstlisting}
    \end{multicols}
    %\caption{Comparison of PyTorch (left) and the proposed framework PositNN (right).}
%\end{figure*}
% As a summary of the sections above, the Gaussian approximated smoothing solutions, whilst being more robust than SMC methods (and extensions thereof), lack the unbiasedness and convergence properties of these. This motivates the following contributions of this article:
\begin{enumerate}
    \item In Section~\ref{sec:auxiliary_samplers}, we show that, in the case of generalised Feynman--Kac models~\eqref{eq:gen-ssm} with Gaussian dynamics, the auxiliary proposals of \citet{titsias2018} recover the posterior distribution of an auxiliary LGSSM. We leverage this to reduce their time and space complexity to $\bigO(T)$ rather than $\bigO(T^3)$. We then use the generalised statistical regression framework of \citet{Tronarp2018iterative} to derive a new class of LGSSM auxiliary samplers for non-Gaussian latent dynamical systems.
    \item In Section~\ref{sec:PIT-sampling}, we introduce two parallel-in-time samplers for LGSSM pathwise smoothing distributions, one based on a prefix-sum implementation akin to \citet{Sarkka2021temporal}, and the other based on a divide-and-conquer recursion. We use these to sample from the LGSSM proposals, resulting in an overall $\bigO(\log T)$ MCMC algorithm on parallel hardware.
    \item In Section~\ref{sec:pgibbs_samplers}, we extend the construction of Section~\ref{sec:auxiliary_samplers} to the context of particle MCMC. This will allow us to tackle models for which Gaussian approximations are not practical or under-performing. We show that \citet{finke2021csmc} is recovered as a special case of our method.
    \item In Section~\ref{sec:experiments}, we illustrate the proposed methods on a series of examples from the SMC and Gaussian approximated inference literature. Special attention is paid to understanding their statistical as well as time and memory trade-offs.
\end{enumerate}
% \input{sections/conflicts}
% \section*{Figures \& Tables}

\begin{figure}[ht]
\centering
\includegraphics[width=\linewidth]{figures/rod_shere_disk_example_with_structures}
\caption{\textbf{A})~Principal-moments-of-inertia plot~\cite{sauer2003molecular} for molecules in the QMugs dataset. $NPR_x$ = $x$-th normalized principal moment, $I_x$ = $x$-th smallest principal moment of inertia.
\textbf{B})~Venn diagram showing overlap between QMugs and other well-known datasets with DFT-level computed properties: QM9~\cite{ramakrishnan2014quantum}, PubChemQC~\cite{nakata2017pubchemqc}, and ANI-1~\cite{smith2017anid}. Overlap was computed based on the uniqueness of the InChI representations of the contained molecules. Numbers do not add up to those reported in Table~\ref{tbl:desc} because of InChI strings that occur multiple times.}
\label{fig:rod_disk_sphere_venn}
\end{figure}


\begin{figure}[ht]
\centering
\includegraphics[width=\linewidth]{figures/rdkit_props}
\caption{Distribution of properties for the molecules contained in the QMugs dataset.}
\label{fig:molecule_props}
\end{figure}


\begin{figure}[ht]
\centering
\includegraphics[width=\linewidth]{figures/qmugs_pipeline}
\caption{Overview of the data generation process. Molecules were extracted from the ChEMBL database, standardized, and filtered, and starting conformers were generated using the RDKit software package. Metadynamics (MTD) simulations were performed using the GFN2-xTB semi-empirical method to generate three diverse conformations before final geometry optimization. Molecules that did not pass a series of geometric sanity checks were removed. DFT-level properties ($\omega$B97X-D/def2-SVP) were computed using the Psi4 software.}
\label{fig:pipeline}
\end{figure}


\begin{figure}[ht]
\centering
\includegraphics[width=\linewidth]{figures/geometry_validation}
\caption{
    (\textbf{A})~Distributions of mean pairwise RMSD of atom positions between conformations of each molecule in the QMugs dataset at different stages along the pipeline. While the $k$-means sampling process selects conformations that are, on average, more geometrically diverse than the average pair of structures generated by MTD simulations, geometry optimization reduces the geometrical diversity between the optimized conformers.
    (\textbf{B})~Change in atom positions during geometry optimization vs. mean pairwise RMSD of conformations before optimization. Molecules with initially more diverse conformations displayed a greater change in atom positions than those with initially less diverse conformations. 
    (\textbf{C})~Distribution of RMSD of structures prior to and after optimization with the semi-empirical GFN2-xTB method, and of structures optimized with the same approach vs. with $\omega$B97X-D/def2-SVP. The structures of three molecules with varying differences between the two methods are shown as illustrative examples (black and gray correspond to GFN2-xTB and $\omega$B97X-D/def2-SVP-optimized structures, respectively). For illustrative purposes, the example molecules are aligned on their substructures.
}
\label{fig:geometry_validation}
\end{figure}


\begin{figure}[ht]
\centering
\includegraphics[width=\linewidth]{figures/delta_molecular}
\caption{Comparison of molecular properties computed at the two levels of theory considered herein (GFN2-xTB, $\omega$B97X-D/def2-SVP) for the molecules contained in QMugs. The molecular formation energy $E_{\mathrm{form}}$ in (\textbf{A}) was calculated by subtracting the atomic $U_{\mathrm{Atom}}$ contributions from the  total molecular energies $U_{RT}$. Only the rotational constants $A$ are shown in (\textbf{C}) as their $B$ and $C$ counterparts showed highly similar values. $22$ conformations of small molecules show very large rotational constants and are not shown. RMSE and PCC for rotational constant $A$ are $845.834$ cm$^{-1}$ and $0.091$ respectively, if those structures are included. Abbreviations: RMSE, root mean squared error; PCC, Pearson's correlation coefficient.}
\label{fig:delta_molecular_props}
\end{figure}


\begin{figure}[ht]
\centering
\includegraphics[width=\linewidth]{figures/partial_charges}
\caption{Atom-type-specific partial charge correlations (GFN2-xTB, $\omega$B97X-D/def2-SVP) for the QMugs dataset (see ESI Table~1 for additional metrics)}
\label{fig:partial}
\end{figure}


\begin{figure}[ht]
\centering
\includegraphics[width=\linewidth]{figures/bond_orders}
\caption{Comparison of Wiberg bond orders between GFN2-xTB and $\omega$B97X-D/def2-SVP for the 15 most frequently occurring bond types in the QMugs dataset. The latter level of theory uses L\"owdin-orthogonalization. See ESI Table~2 for additional metrics. For bond types which occurred $>1$M times in the dataset, a randomly chosen sample of $1$M bonds is plotted.}
\label{fig:bond}
\end{figure}




\begin{table}[ht]
\caption{Descriptive statistics of the dataset reported herein in the context of other  DFT-level molecular datasets and the information provided by each. The number of molecules for PubChemQC corresponds to that available on the website of the project.~\cite{pubchemqc_website} Heavy atom averages are weighted by the number of conformations.}
\label{tbl:desc}
\centering
\resizebox{\textwidth}{!}{%
\begin{tabular}{@{}l>{\raggedleft\arraybackslash}p{2cm}>{\raggedleft\arraybackslash}p{2cm}>{\raggedleft\arraybackslash}p{2cm}p{5cm}P{2cm}P{2cm}p{3cm}@{}}
\toprule
\textbf{Dataset} &
  \textbf{Unique compounds} &
  \textbf{Total conformations} &
  \textbf{Heavy atoms max (mean)} &
  \textbf{Method} &
  \textbf{$\Delta$-learning possible} &
  \textbf{Wave functions} \\ \midrule
QM9       & $133,885$  & $133,885$   & $9$~~~($8.8$) & B3LYP/6-31G(2df,p)               & \xmark &  \xmark \\
ANI-1     & $57,462$   & $22,057,374$ & $8$~~~($7.1$) & $\omega$B97X/6-31G(d)              & \xmark  & \xmark \\
PubChemQC & $3,982,436$ & $3,982,436$  & $51$ ($14.1$) & B3LYP/6-31G(d)              & \xmark  & \xmark \\
QMugs     & $665,911$  & $1,992,984$  & $100$ ($30.6$) & GFN2-xTB + $\omega$B97X-D/def2-SVP & \cmark  & \cmark \\ \bottomrule
\end{tabular}
}
\end{table}


\begin{table}[ht]
\footnotesize
\caption{Calculated properties as stored in the SDFs of the QMugs data collection. Abbreviations: a.u., atomic units; vib., vibrational; rot., rotational; transl., translational. Properties that enable $\Delta$ machine learning are labelled with $\blacklozenge$.}
\label{tbl:properties}
\centering
\begin{tabular}{llllll}
\toprule
\textbf{Property}                                        & \textbf{Symbol}                               & \textbf{Unit}           & \textbf{Key}  & $\Delta$-ML        \\  \midrule
ChEMBL identifier                                               &         -                      &           -              & \texttt{CHEMBL\_ID} &                \\
Conformer identifier                                                 &     -                                          &                    -     & \texttt{CONF\_ID}      &              \\
Total energy                                             & $U_{RT}$            & \si{\hartree}                      & \texttt{GFN2:TOTAL\_ENERGY} & $\blacklozenge$                  \\
Internal atomic energy                             & $E_\mathrm{Atom}$                             & \si{\hartree}                      & \texttt{GFN2:ATOMIC\_ENERGY} &       \\
Formation energy                             & $E_\mathrm{Form}$           & \si{\hartree}                      & \texttt{GFN2:FORMATION\_ENERGY}   & $\blacklozenge$          \\
Total enthalpy                                           & $H_{RT}$                                      & \si{\hartree}                      & \texttt{GFN2:TOTAL\_ENTHALPY}    &        \\
Total free energy                                        & $G_{RT}$                                      & \si{\hartree}                      & \texttt{GFN2:TOTAL\_FREE\_ENERGY} &        \\
Dipole ($x$, $y$, $z$, total)                  & $\mu$                                         & D                       & \texttt{GFN2:DIPOLE}      & $\blacklozenge$                    \\
Quadrupole ($xx$, $xy$, $yy$, $xz$, $yz$, $zz$) & $Q_{ij}$                                      & D \si{\angstrom}                 & \texttt{GFN2:QUADRUPOLE}      &           \\
Rotational constants ($A$, $B$, $C$)           & $A$, $B$, $C$                              & \si{\centi\meter}$^{-1}$                     & \texttt{GFN2:ROT\_RONSTANTS}    & $\blacklozenge$              \\
Enthalpy (vib., rot., transl., total)          & $\Delta H$                                    & cal mol$^{-1}$          & \texttt{GFN2:ENTHALPY}     &               \\
Heat capacity (vib., rot., transl., total)    & $C_{V}$                                       & cal K$^{-1}$ mol$^{-1}$ & \texttt{GFN2:HEAT\_CAPACITY}  &             \\
Entropy (vib., rot., transl., and total)       & $\Delta S$                                    & cal K$^{-1}$ mol$^{-1}$ & \texttt{GFN2:ENTROPY}     &                \\
HOMO energy                                              & $E_\mathrm{HOMO}$                             & \si{\hartree}                      & \texttt{GFN2:HOMO\_ENERGY}       & $\blacklozenge$             \\
LUMO energy                                              & $E_\mathrm{LUMO}$                             & \si{\hartree}                      & \texttt{GFN2:LUMO\_ENERGY}     & $\blacklozenge$           \\
HOMO-LUMO gap                                            & $E_\mathrm{Gap}$                              & \si{\hartree}                      & \texttt{GFN2:HOMO\_LUMO\_GAP}  & $\blacklozenge$               \\
Fermi level                                              & $E_{\mathrm{Fermi}}$                                   & \si{\hartree}                      & \texttt{GFN2:FERMI\_LEVEL}   &             \\
Mulliken partial charges                                 & $\delta_{M}$                                  & \si{\elementarycharge}                       & \texttt{GFN2:MULLIKEN\_CHARGES} & $\blacklozenge$              \\
Covalent coordination number                           & $N_{\textrm{coord}}$                  & -                   &\texttt{GFN2:COVALENT\_COORDINATION\_NUMBER}  & \\
Molecular dispersion coefficient                           & $C_6$                                                & a.u.                            &\texttt{GFN2:DISPERSION\_COEFFICIENT\_MOLECULAR} & \\
Atomic dispersion coefficients                         & $C_6$                                                & a.u.                                &\texttt{GFN2:DISPERSION\_COEFFICIENT\_ATOMIC} & \\
Molecular polarizability                           & $\alpha(0)$                                                 & a.u.                                        &\texttt{GFN2:POLARIZABILITY\_MOLECULAR} &  \\
Atomic polarizabilities                            & $\alpha(0)$                                                 & a.u.                                        &\texttt{GFN2:POLARIZABILITY\_ATOMIC} &  \\
Wiberg bond orders                                    & $M_{AB}$                                 &             -            &\texttt{GFN2:WIBERG\_BOND\_ORDER}     & $\blacklozenge$            \\
Total Wiberg bond orders                               & $\sum_{A (A \neq B)} M_{AB}$                      &       -                  &\texttt{GFN2:TOTAL\_WIBERG\_BOND\_ORDER}  & $\blacklozenge$              \\
Total energy                                             & $U_{RT}$                                      & \si{\hartree}                      & \texttt{DFT:TOTAL\_ENERGY}     & $\blacklozenge$             \\
Total internal atomic energy                             & $E_\mathrm{Atom}$                             & \si{\hartree}                      & \texttt{DFT:ATOMIC\_ENERGY}    &         \\
Formation energy                             & $E_\mathrm{Form}$                             & \si{\hartree}                      & \texttt{DFT:FORMATION\_ENERGY}  & $\blacklozenge$           \\
Electrostatic potential                                  & $V_{ESP}$                                     & \si{\volt}                       & \texttt{DFT:ESP\_AT\_NUCLEI}   &         \\
L\"owdin partial charges                                 & $\delta_{L}$                                  & \si{\elementarycharge}                       & \texttt{DFT:LOWDIN\_CHARGES}     &       \\
Mulliken partial charges                                 & $\delta_{M}$                                  & \si{\elementarycharge}                       & \texttt{DFT:MULLIKEN\_CHARGES}   & $\blacklozenge$           \\
Rotational constants  ($A$, $B$, $C$)          & $A$, $B$, $C$                              & \si{\centi\meter}$^{-1}$                     & \texttt{DFT:ROT\_CONSTANTS}   & $\blacklozenge$              \\
Dipole ($x$, $y$, $z$, total)                  & $\mu$                                         & D                       & \texttt{DFT:DIPOLE}                     \\
Exchange correlation energy                              & $\hat{V}_{eN}$                                & \si{\hartree}                      & \texttt{DFT:XC\_ENERGY}   &              \\
Nuclear repulsion energy                                 & $\hat{V}_{eN}$                                & \si{\hartree}                      & \texttt{DFT:NUCLEAR\_REPULSION\_ENERGY} & \\
One-electron energy                               & $\hat{T}_{e}$                                 & \si{\hartree}                      & \texttt{DFT:ONE\_ELECTRON\_ENERGY}   &    \\
Two-electron energy                                      & $\hat{V}_{ee}$                                & \si{\hartree}                      & \texttt{DFT:TWO\_ELECTRON\_ENERGY} &     \\
HOMO energy                                              & $E_\mathrm{HOMO}$                             & \si{\hartree}                      & \texttt{DFT:HOMO\_ENERGY}    & $\blacklozenge$               \\
LUMO energy                                              & $E_\mathrm{LUMO}$                             & \si{\hartree}                      & \texttt{DFT:LUMO\_ENERGY}    & $\blacklozenge$               \\
HOMO-LUMO gap                                            & $E_\mathrm{Gap}$                              & \si{\hartree}                      & \texttt{DFT:HOMO\_LUMO\_GAP}   & $\blacklozenge$             \\
Mayer bond orders                                        & $M_{AB}$                                 &                -         & \texttt{DFT:MAYER\_BOND\_ORDER}   &           \\
Wiberg-L\"owdin bond orders                              & $W_{AB}$                                 &            -             & \texttt{DFT:WIBERG\_LOWDIN\_BOND\_ORDER}  & $\blacklozenge$      \\
Total Mayer bond orders                              & $\sum_{A (A \neq B)} M_{AB}$                      &          -               & \texttt{DFT:TOTAL\_MAYER\_BOND\_ORDER}      &      \\
Total Wiberg-L\"owdin bond orders                          & $\sum_{A (A \neq B)} W_{AB}$        &      -                   & \texttt{DFT:TOTAL\_WIBERG\_LOWDIN\_BOND\_ORDER}  & $\blacklozenge$     \\ \bottomrule
\end{tabular}
\end{table}

\begin{table}[ht]
\caption{Calculated molecular properties stored in the wave function files provided in the QMugs data collection. Mayer and Wiberg-L\"owdin bond orders included here represent a superset of the bond orders in the SDFs which additionally comprise bond orders for non-covalent bonds.}
\label{tbl:wfn}
\centering
\begin{tabular}{llll}
\toprule
\textbf{Property}                                        & \textbf{Symbol}                               & \textbf{Key}                            \\ \midrule
Alpha density matrix                                     & $\mathrm{D}_{\alpha}$                         & \texttt{matrix, Ca}                              \\
Beta density matrix                                      & $\mathrm{D}_{\beta}$                          & \texttt{matrix, Cb}                              \\
Alpha orbitals                                           & $\mathrm{C}_{\alpha}$                         & \texttt{matrix, Da}                              \\
Beta orbitals                                            & $\mathrm{C}_{\beta}$                          & \texttt{matrix, Db}                              \\
Atomic-orbital-to-symmetry-orbital transformer           & $\mathrm{C}_{\mathrm{AOTOSO}}$                & \texttt{matrix, aotoso}                          \\
Mayer bond orders                                        & $M_{AB}$                                      & \texttt{MAYER\_INDICES}                          \\
Wiberg-L\"owdin bond orders                              & $W_{AB}$                                      & \texttt{WIBERG\_LOWDIN\_INDICES}                 \\ \bottomrule
\end{tabular}
\end{table}

%\newpage
\section{Acknowledgements}

Luca Herranz-Celotti was supported by the Natural Sciences and Engineering Research Council of Canada through the Discovery Grant from professor Jean Rouat, and by CHIST-ERA IGLU. We thank Compute Canada for the clusters used to perform the experiments and NVIDIA for the donation of two GPUs. We thank Wolfgang Maass for the opportunity to visit the Institute of Theoretical Computer Science, Guillaume Bellec, Darjan Salaj and Franz Scherr, for their invaluable insights on learning with surrogate gradients, and Maryam Hosseini, Ahmad El Ferdaoussi and Guillaume Bellec for their feedback on the article.
\bibliographystyle{plainnat}
\bibliography{bibliography.bib}
\newpage
%\section*{Checklist}


\begin{enumerate}

% \answerTODO{}, \answerYes{}, \answerNo{}, \answerNA{}

\item For all authors...
\begin{enumerate}
  \item Do the main claims made in the abstract and introduction accurately reflect the paper's contributions and scope?
    \answerYes{}
  \item Did you describe the limitations of your work?
    \answerYes{}\\
    \textcolor{blue}{See Section~\ref{sec:conclusion}.}
  \item Did you discuss any potential negative societal impacts of your work?
    \answerYes{}\\
    \textcolor{blue}{See Section~\ref{sec:conclusion}.}
  \item Have you read the ethics review guidelines and ensured that your paper conforms to them?
    \answerYes{}
\end{enumerate}

\item If you are including theoretical results...
\begin{enumerate}
  \item Did you state the full set of assumptions of all theoretical results?
    \answerNA{}
	\item Did you include complete proofs of all theoretical results?
    \answerNA{}
\end{enumerate}

\item If you ran experiments...
\begin{enumerate}
  \item Did you include the code, data, and instructions needed to reproduce the main experimental results (either in the supplemental material or as a URL)?
    \answerYes{}\\
    \textcolor{blue}{We provide our code in the supplementary materials.}
  \item Did you specify all the training details (e.g., data splits, hyperparameters, how they were chosen)?
    \answerYes{}\\
    \textcolor{blue}{We state the details in Section~\ref{sec:exp} and Appendix~\ref{sec:details}.}
	\item Did you report error bars (e.g., with respect to the random seed after running experiments multiple times)?
    \answerYes{}\\
    \textcolor{blue}{We report the error bars for the experiments with larger variances (e.g., Table~\ref{tab:bg-main}).}
	\item Did you include the total amount of compute and the type of resources used (e.g., type of GPUs, internal cluster, or cloud provider)?
    \answerYes{}\\
    \textcolor{blue}{See common setup in Section~\ref{sec:exp}.}
\end{enumerate}

\item If you are using existing assets (e.g., code, data, models) or curating/releasing new assets...
\begin{enumerate}
  \item If your work uses existing assets, did you cite the creators?
    \answerYes{}\\
    \textcolor{blue}{We cite the datasets and libraries we use.}
  \item Did you mention the license of the assets?
    \answerYes{}\\
    \textcolor{blue}{We only use the public datasets and open source libraries.}
  \item Did you include any new assets either in the supplemental material or as a URL?
    \answerNA{}
  \item Did you discuss whether and how consent was obtained from people whose data you're using/curating?
    \answerNA{}
  \item Did you discuss whether the data you are using/curating contains personally identifiable information or offensive content?
    \answerNA{}
\end{enumerate}

\item If you used crowdsourcing or conducted research with human subjects...
\begin{enumerate}
  \item Did you include the full text of instructions given to participants and screenshots, if applicable?
    \answerNA{}
  \item Did you describe any potential participant risks, with links to Institutional Review Board (IRB) approvals, if applicable?
    \answerNA{}
  \item Did you include the estimated hourly wage paid to participants and the total amount spent on participant compensation?
    \answerNA{}
\end{enumerate}

\end{enumerate}


\newpage
\section*{Supplementary Material}

\section{Additional Chest ImaGenome Terminology Descriptions}
% \begin{table}[ht]
%   \centering
%     \caption{Parallels between the Chest ImaGenome and Visual Genome datasets.}
%   \label{tab:cg_vg_parallels}
%   \footnotesize{
%     \begin{tabular}{|p{4em}|p{21em}|p{16em}|}
%     \toprule
%     \textbf{Element} & \textbf{Chest ImaGenome} & \textbf{Visual Genome} \\
%     \midrule
%     \midrule
%     Scene & One frontal CXR image in the current dataset. & One (non-medical) everyday life image. \\
%     \midrule
%     Questions & For now, there is only one question per CXR, which is taken from the patient history (i.e., reason for exam) section from each CXR report. & One or more questions that the crowd source annotators decided to ask about the image where the information from each question and the image should allow another annotator to answer it. \\
%     \midrule
%     Answers & N/A currently. However, report sentences are biased towards answering the question asked in the reason for exam sentence;hence, the knowledge graph we extract from each report should contain the answer(s). & This was collected as answer(s) to the corresponding question(s) asked of the image. \\
%     \midrule
%     Sentences (Region descriptions) & Sentences from the finding and impression sections of a CXR report describing the exam as collected from radiologists in their routine radiology workflow. & True natural language descriptive sentences about the image collected from crowd-sourced everyday annotators. \\
%     \midrule
%     % Region & Bounding box coordinates that encompass all the anatomical structures described in a report sentence (can be easily derived from the scene graph json). & Bounding box coordinates that include all the objects mentioned in a sentence that describes the image, e.g., The boy (object 1) is beside the bus (object 2). \\
%     % \midrule
%     Objects (nodes) & Anatomical structures or locations that have bounding box coordinates on the associated CXR image, and is indexed to the UMLS ontology \cite{bodenreider2004unified}. & The people and physical objects with bounding box coordinates on the image and indexed to WordNet ontology \cite{miller1995wordnet}. \\
%     \midrule
%     Attributes (nodes) & Descriptions that are true for different anatomical structures visualized on the CXR image (e.g., There is a right upper lung [object] opacity [attribute]), indexed to the UMLS ontology \cite{bodenreider2004unified}. No Bbox coordinates. & Various descriptive properties of the objects in the image (e.g., The shirt [object] is blue [attribute]), indexed to WordNet ontology \cite{miller1995wordnet}. No Bbox coordinates. \\
%     \midrule
%     Relations: object and attribute & The relationship(s) between an anatomical object and its attribute(s) from the same CXR image (e.g., There is a [relation] right upper lung [object] opacity [attribute]). & The relationship(s)  between an object and its attribute(s) from the same image ( e.g., The shirt [object] is [relation] blue [attribute]). \\
%     \midrule
%     Relations: object and object & The comparison relationship (index to UMLS \cite{bodenreider2004unified}) between the same anatomical object from two sequential CXR images for the same patient (e.g., There is a new [relation] right lower lobe [current and previous anatomical objects] atelectasis [attribute]). & The relationship (indexed to WordNet \cite{miller1995wordnet}) between objects in the same image (e.g., The boy [object 1] is beside [relation] the bus [object 2]). \\
%     \midrule
%     Relations: parent and child & To make the graph for each image logically consistent and correct as learnable and consumable radiology knowledge, affirmed parent-child relations between nodes are embedded in the scene graphs -- i.e., if a child attribute is related to an object, then its parent would be too (e.g., if right lung has consolidation [child], then it also has lung opacity [parent]). & N/A due to different graph construction strategy and goals. The annotators were asked to describe any (but not all) relations they observe in an image. \\
%     \midrule
%     Scene graph & Constructed from the objects, the attributes and the relationships between them for the image. & Same but the nodes and edges overall would be more varied than Chest ImaGenome for now. \\
%     \midrule
%     Sequence* & A super-graph for a set of chronologically ordered series of exams for the same patient. & N/A, but would be a graph for a video in the non-medical context. \\
%     \bottomrule
%     \end{tabular}%
%     }
% \end{table}

% \newpage

%\subsection*{Data description - supplementary}
% describes nodes in graph
\begin{table}[h]
  \centering
     \caption{Semantic category of nodes and edges in CXR knowledge graphs. All nodes are mapped to UMLS CUIs in the scene graph jsons. All object nodes have corresponding bounding box coordinates on frontal CXRs except ones with *. All nodes and edges are evaluated with the gold standard dataset except the edges marked with **, which are modifiers of the context edges.}
%   \caption{Add caption}
    \label{tab:define_nodes_edges} 
    \resizebox{\textwidth}{!}{
    \begin{tabular}{|l|c|p{25em}|}
    \toprule
    \textbf{Category ID} & \textbf{type} & \multicolumn{1}{l|}{\textbf{names}} \\
    \midrule
    \midrule
    technicalassessment & attribute node & low lung volumes, rotated, artifact, breast/nipple shadows, skin fold \\
    \midrule
    texture & attribute node & opacity, alveolar, interstitial, calcified, lucency \\
    \midrule
    anatomicalfinding & attribute node & lung opacity, airspace opacity, consolidation, infiltration, atelectasis, linear/patchy atelectasis, lobar/segmental collapse, pulmonary edema/hazy opacity, vascular congestion, vascular redistribution, increased reticular markings/ild pattern, pleural effusion, costophrenic angle blunting, pleural/parenchymal scarring, bronchiectasis, enlarged cardiac silhouette, mediastinal displacement, mediastinal widening, enlarged hilum, tortuous aorta, vascular calcification, pneumomediastinum, pneumothorax, hydropneumothorax, lung lesion, mass/nodule (not otherwise specified), multiple masses/nodules, calcified nodule, superior mediastinal mass/enlargement, rib fracture, clavicle fracture, spinal fracture, hyperaeration, cyst/bullae, elevated hemidiaphragm, diaphragmatic eventration (benign), sub-diaphragmatic air, subcutaneous air, hernia, scoliosis, spinal degenerative changes, shoulder osteoarthritis, bone lesion \\
    \midrule
    disease & attribute node & pneumonia, fluid overload/heart failure, copd/emphysema, granulomatous disease, interstitial lung disease, goiter, lung cancer, aspiration, alveolar hemorrhage, pericardial effusion \\
    \midrule
    nlp   & attribute node & abnormal, normal (with respect to an anatomy/object node) \\
    \midrule
    tubesandlines & attribute node & chest tube, mediastinal drain, pigtail catheter, endotracheal tube, tracheostomy tube, picc, ij line, chest port, subclavian line, swan-ganz catheter, intra-aortic balloon pump, enteric tube \\
    \midrule
    device & attribute node & sternotomy wires, cabg grafts, aortic graft/repair, prosthetic valve, cardiac pacer and wires \\
    \midrule
    \midrule
    majorstructure & object node & right lung, left lung, mediastinum \\
    \midrule
    subanatomy & object node & right apical zone, right upper lung zone, right mid lung zone, right lower lung zone, right hilar structures, right costophrenic angle, left apical zone, left upper lung zone, left mid lung zone, left lower lung zone, left hilar structures, left costophrenic angle, upper mediastinum, cardiac silhouette, trachea, right hemidiaphragm, left hemidiaphragm, right clavicle, left clavicle, spine, right atrium, cavoatrial junction, svc, carina, aortic arch, abdomen, right chest wall*, left chest wall*, right shoulder*, left shoulder*, neck*, right arm*, left arm*, right breast*, left breast* \\
    \midrule
    \midrule
    context & edge  & yes (has/present in), no (not have/not present in)\\
    \midrule
    comparison & edge  & improved, worsened, no change \\
    \midrule
    severity** & edge  & hedge, mild, moderate, severe \\
    \midrule
    temporal** & edge  & acute, chronic \\
    \bottomrule
    \end{tabular}
    }%
  %\label{tab:addlabel}%
\end{table}

\newpage
\section{Scene Graph JSON}\label{jsonsg}
Below are examples from a scene graph JSON used for explanation for the silver dataset.

\subsection{Scene Graph JSON - first level}\label{json1}
\begin{footnotesize}
\begin{verbatim}
{
 `chest_imageimage_id': `10cd06e9-5443fef9-9afbe903-e2ce1eb5-dcff1097',
 `viewpoint': `AP', `patient_id': 10063856, `study_id': 56759094,
 `gender': `F', `age_decile': `50-60',
 `reason_for_exam': `___F with hypotension.  Evaluate for pneumonia.',
 `StudyOrder': 2, `StudyDateTime': `2178-10-05 15:05:32 UTC',
 `objects': [ <...list of {} for each object...> ],
 `attributes':[ <...list of {} for each object...> ],
 `relationships':[ <...list of {} of comparison relationships between objects 
 from sequential exams for the same patient...> ] 
}
\end{verbatim}
\end{footnotesize}


\subsection{Scene Graph JSON - objects field}\label{json2}
\begin{footnotesize}
\begin{verbatim}
{
  `object_id': `10cd06e9-5443fef9-9afbe903-e2ce1eb5-dcff1097_right upper lung zone',
  `x1': 48, `y1': 39, `x2': 111, `y2': 93,
  `width': 63, `height': 54,
  `bbox_name': `right upper lung zone',
  `synsets': [`C0934570'],
  `name': `Right upper lung zone',
  `original_x1': 395, `original_y1': 532,
  `original_x2': 1255, `original_y2': 1268,
  `original_width': 860, `original_height': 736
}
\end{verbatim}
\end{footnotesize}


\subsection{Scene Graph JSON - attributes field}\label{json3}
\begin{footnotesize}
\begin{verbatim}
{
  `right lung': True, `bbox_name': `right lung',
  `synsets': [`C0225706'], `name': `Right lung',
  `attributes': [[`anatomicalfinding|no|lung opacity',
  `anatomicalfinding|no|pneumothorax',  `nlp|yes|normal'],
  [`anatomicalfinding|no|pneumothorax']],
  `attributes_ids': [[`CL556823', `C1963215;;C0032326', `C1550457'],
  [`C1963215;;C0032326']],
  `phrases': [`Right lung is clear without pneumothorax.', 
  `No pneumothorax identified.'],
  `phrase_IDs': [`56759094|10', `56759094|14'],
  `sections': [`finalreport', `finalreport'],
  `comparison_cues': [[], []],
  `temporal_cues': [[], []],
  `severity_cues': [[], []],
  `texture_cues': [[], []],
  `object_id': `10cd06e9-5443fef9-9afbe903-e2ce1eb5-dcff1097_right lung'
}
\end{verbatim}
\end{footnotesize}


\subsection{Scene Graph JSON - relationships field}\label{json4}
\begin{footnotesize}
\begin{verbatim}
{
  `relationship_id': `56759094|7_54814005_C0929215_10cd06e9_4bb710ab',
  `predicate': ``['No status change']'',
  `synsets': [`C0442739'],
  `relationship_names': [`comparison|yes|no change'],
  `relationship_contexts': [1.0],
  `phrase': `Compared with the prior radiograph, there is a persistent veil 
  -like opacity\n over the left hemithorax, with a crescent of air surrounding 
  the aortic arch,\n in keeping with continued left upper lobe collapse.',
  `attributes': [`anatomicalfinding|yes|atelectasis',
  `anatomicalfinding|yes|lobar/segmental collapse',
  `anatomicalfinding|yes|lung opacity', `nlp|yes|abnormal'],
  `bbox_name': `left upper lung zone',
  `subject_id': `10cd06e9-5443fef9-9afbe903-e2ce1eb5-dcff1097_left upper lung zone',
  `object_id': `4bb710ab-ab7d4781-568bcd6e-5079d3e6-7fdb61b6_left upper lung zone'
}
\end{verbatim}
\end{footnotesize}


\subsection{Scene Graph - Enriched RDF JSON format}
\begin{footnotesize}\label{json5}
\begin{verbatim}
{
 <study_id_i> : [
                  [[node_id_1, node_type_1], [node_id_2, node_type_2], relation_name_A],
                  [[node_id_1, node_type_1], [node_id_3, node_type_3], relation_name_B],
                    ...
                ],
 <study_id_i+1>:[
                  [[node_id_1, node_type_1], [node_id_2, node_type_2], relation_name_A],
                  [[node_id_1, node_type_1], [node_id_3, node_type_3], relation_name_B],
                    ...
                ],
}   
\end{verbatim}
\end{footnotesize}


\section{Gold Dataset Annotation - Details}
\label{gold_annot_supp}

The `gold dataset' is a randomly sampled subset (500 unique patients) from the automatically generated Chest ImaGenome dataset, i.e., the `silver dataset', that has been manually validated or corrected. The primary purpose of the `gold dataset' is to evaluate the quality of labels in the `silver dataset'. For this purpose, we evaluated the Chest ImaGenome dataset along with the 3 components below (A-B). The annotations for each component were collected in stages to reduce the cognitive workload for the annotators. The annotators are all M.D.s with 2 to 10 or more years of clinical experience. One of the annotators is a radiologist trained in the United States, who has over 6 years of radiology experience and specializes in reading imaging exams from the Emergency Department (ED) setting. The annotation tasks were delegated to the annotators according to their clinical experience, which we think are all more than sufficient for the tasks. Component A and B were annotated by the radiologist and an M.D. and component C was annotated by 4 M.D.'s.


\vspace{+10pt}
\textbf{\textit{A) Evaluating CXR knowledge graph extraction from reports}}
\vspace{+5pt}

The report knowledge graph for the \textit{first} CXR of the 500 patients was manually reviewed and corrected as necessary for relation extraction between the anatomical locations (objects) and the CXR attributes. From piloting trials, we found that manually annotating multiple targets at a document level lead to a slow and complex task with poor recall. However, sometimes information from prior sentences is necessary to annotate both the anatomical locations and the attributes correctly. Therefore, we set up the annotation task at the sentence level. Sentences from each report are ordered as per the original report, and the phrase boundary for each attribute was marked out for the annotators, where the phrases used for detecting each attribute were curated by consensus between two radiologists from previous work \cite{wu2020ai}. 

Since we are targeting a large set of possible anatomical locations (object) to attribute combinations, the annotation was streamlined into the four steps below to minimize the cognitive overload for each step. Steps 1 and 2 are dual annotated by two clinicians (one fully trained radiologist and one M.D.), with disagreements resolved by consensus review. Steps 3 and 4 are single annotated. A random subset of annotations for 500 sentences from step 4 are sampled and dual annotated to estimate inter-annotator agreement. Cleaned results from step 4 constitute the final gold-standard CXR knowledge graph ground truth for the 500 reports. 

This annotation component was set up in Excel and was broken down into the following four steps below. In our Excel setup, all sentences from each report are available to the annotators (they can just scroll up or down). The sentences are ordered by `row\_id' sequentially within each report. Unique patients and reports have the same IDs as shown in the figures below.

\textbf{Step 1} - For each sentence and NLP extracted attribute combination, decide whether the NLP context (affirmed or negated) for the attribute was correct. If not, correct it. Figure \ref{fig:object-attribute-step1} shows how this task was set up in Excel. The annotators' task is to make sure the extracted attribute (yellow label\_name column) has the correct context given the sentence from the report. This `context' is used as the relation between the location and the attribute in the final annotated result.

\begin{figure}[!ht]
\centering
\includegraphics[scale=0.35]{figures/annot/object_attribute_annot_step1.pdf}
\caption{Step 1: Annotate all attributes per sentence.}
\label{fig:object-attribute-step1}
\end{figure}

\textbf{Step 2} - For each sentence, decide whether the NLP extracted anatomical location(s) were described or implied by the reporting radiologist. If not, remove the location (in yellow column `bboxes\_corrected). If missing, add the location. If unsure (e.g., if lung is mentioned but not sure if it is the right or left lung), the annotator can look in previous sentences from the same report. The task was set up as shown in Figure \ref{fig:object-attribute-step2}.

\begin{figure}[!ht]
\centering
\includegraphics[scale=0.29]{figures/annot/object_attribute_annot_step2.pdf}
\caption{Step 2: Annotate all locations per sentence.}
\label{fig:object-attribute-step2}
\end{figure}

\textbf{Step 3} - For recall, manually annotate missed objects and/or attributes for sentences with no NLP extractions (a much smaller subset). For this, we used Excel's filtering function to look at all sentences with no automated extractions (empty cells) and de novo added the manual annotations.

\textbf{Step 4} - Firstly, all rows from steps 1-3 where the annotations differed between the two annotators were reviewed and resolved together by consensus. Then we automatically derived all object-attribute relation combinations for each sentence from steps 1-3's results. The obviously wrong object-to-attribute relations were filtered out for each sentence using the CXR ontology. For the remaining object-to-attribute relations for each sentence, the task was to indicate whether the logical statement of \textit{``object X contains (or does not contain) attribute Y''} is true or false, as shown in Figure \ref{fig:object-attribute-step4}. Probable relation is still defined to be true for this annotation. Annotating for uncertain relations is beyond the scope of this project. However, for future dataset expansion, we have kept the NLP cues for the certainty for each object-attribute relation in the scene graph JSON. 

\begin{figure}[!ht]
\centering
\includegraphics[scale=0.33]{figures/annot/object_attribute_annot_step4.pdf}
\caption{Step 4: Annotate all logically correct statements/relations for each sentence.}
\label{fig:object-attribute-step4}
\end{figure}

Since step 4 was single annotated, to estimate the final inter-annotator agreement, we randomly sampled 500 sentences for dual annotations. This %\href{https://physionet.org/content/chest-imagenome/1.0.0/utils/annotation_utils/object_attribute_relation_annotation/object_attribute_relations_estimated_interannotator_agreement.txt}{\textbf{\textit{annotated result}}} 
annotated result is also shared on PhysioNet.

\vspace{+10pt}
\textbf{\textit{B) Evaluating comparison relation extraction}}: 
\vspace{+5pt}

The \textit{second} CXR exam report for the 500 patients was reviewed for comparison relation extraction. The annotation was also set up in Excel and conducted at the sentence level. However, the annotator is also shown the whole previous CXR report for context. Similarly, we split the annotation task up into several steps, where steps 1 and 2 are dual annotated and disagreement resolved via consensus. Steps 3 and 4 were single annotated. A %\href{https://physionet.org/content/chest-imagenome/1.0.0/utils/annotation_utils/object_object_comparison_annotation/comparisons_relations_estimated_interannotator_agreement.txt}{\textbf{\textit{subset of 500 sentences}}}
subset of 500 sentences from the final annotations was reviewed by a second annotator for assessing inter-annotator agreement.

\textbf{Step 1} - Given the previous report and the current report sentence, decide whether the extracted comparison cue(s) (improved, worsened, no change) is/are correct. If not, correct it/them. In this step, the annotators are asked to validate or correct the column `comparison' in Figure \ref{fig:object-comparison-step12}.

\textbf{Step 2} - Building from step 1 for each sentence, given a validated or corrected comparison cue, validate whether all the anatomical location(s) extracted are correct (column `bbox' in Figure \ref{fig:object-comparison-step12}). If incorrect or missing, remove or add the correct location(s) to the column.

\begin{figure}[!ht]
\centering
\includegraphics[scale=0.31]{figures/annot/object_object_comparison_annot_step1_2.pdf}
\caption{Step 1 and 2: Annotate change relations for different anatomical locations}
\label{fig:object-comparison-step12}
\end{figure}

\textbf{Step 3} - Building from step 2 for each sentence, given each correct comparison cue and anatomical location relation, decide whether the attributes assigned to the location described or implied in the sentence are correct or not. If not, correct it. Figure \ref{fig:object-comparison-step3} illustrates how step 3 was set up, where the annotators' task is to validate or correct the `label\_name' column with respect to the `bbox', `relation' and `comparison' columns for each sentence.

\begin{figure}[!ht]
\centering
\includegraphics[scale=0.3]{figures/annot/object_object_comparison_annot_step3.pdf}
\caption{Step 3: Annotate change relations for different anatomical locations with respect to attribute}
\label{fig:object-comparison-step3}
\end{figure}

\textbf{Step 4} - For recall, we used the filtering function in Excel to isolate all sentences with no comparison cue extractions from step 3. Sentences with missing comparison annotations were manually de-novo annotated.

\vspace{+10pt}
\textbf{\textit{C) Evaluating anatomy object detection for CXR images}}: 
\vspace{+5pt}

The first and second CXR images for the same 500 patients were dual validated and corrected for the bounding box objects (i.e., 1000 frontal CXR images altogether). Given the resources we had, we selected 28 anatomical objects (out of 36 available) that are clinically most important for frontal CXRs interpretations. The automatically extracted bounding box coordinates were first plotted on resized and padded 224x224 images. From piloting, we determined that this image size is sufficiently large to annotate the anatomies that we were targeting. The plotted images were displayed one at a time to annotators via a custom Jupyter Notebook that we had set up to allow bounding box coordinates and label annotations. We set up the annotation task on two panels, one for %\href{https://physionet.org/content/chest-imagenome/1.0.0/utils/annotation_utils/bbox_object_annotation/Correct_lung_bboxes_template.ipynb}{\textbf{\textit{lung related bounding boxes}}} 
lung-related bounding boxes (Figure \ref{fig:bboxes-lung-panel}) and another for %\href{https://physionet.org/content/chest-imagenome/1.0.0/utils/annotation_utils/bbox_object_annotation/Correct_mediastinum_bboxes_template.ipynb}{\textbf{\textit{mediastinum related and other bounding boxes}}} 
mediastinum related and other bounding boxes (Figure \ref{fig:bboxes-mediastinum-panel}). 

\begin{figure}[!ht]
\centering
\includegraphics[scale=0.35]{figures/annot/lung_related_bbox_panel.pdf}
\caption{Bbox annotations - lung related Bboxes panel}
\label{fig:bboxes-lung-panel}
\end{figure}

\begin{figure}[!ht]
\centering
\includegraphics[scale=0.35]{figures/annot/mediastinum_related_bbox_panel.pdf}
\caption{Bbox annotations - mediastinum related and other Bboxes panel}
\label{fig:bboxes-mediastinum-panel}
\end{figure}

Four M.D.'s were trained to perform this task after reviewing a set of 20-30 training examples with a radiologist. Since the inter-annotator agreement is high (mean IoU > 0.96 for all objects), the final cleaned %\href{https://physionet.org/content/chest-imagenome/1.0.0/gold_dataset/gold_bbox_coordinate_annotations_1000images.csv}{\textbf{\textit{gold standard bbox coordinates}}} 
gold standard bbox coordinates use the average coordinates from two annotators for each bounding box.


\newpage
\section{Dataset Usage Supporting Files}
\label{gold_supp}

% Can also put this in supplementary material
\noindent \textbf{gold\_all\_sentences\_500pts\_1000studies.txt} contains all the sentences tokenized from the original MIMIC-CXR reports that were used to create the gold standard dataset. We include this file because sentences with no relevant object, attribute or relation descriptions did not make it into the gold standard dataset. We renamed `subject\_id' from MIMIC-CXR dataset to `patient\_id' in Chest ImaGenome dataset to avoid confusion with field names for relationships in the scene graphs. Otherwise, the ids are unchanged. Sentences in the tokenized file are assigned to `history', `prelimread', or `finalreport' in the `section' column. The `sent\_loc' column contains the order of the sentences as in the original report. Minimal tokenization has been done to the sentences.

\noindent \textbf{gold\_bbox\_scaling\_factors\_original\_to\_224x224.csv} contains the scaling `ratio' and the paddings (`left', `right', `top', and `bottom') added to square the image after resizing the original MIMIC-CXR dicoms to 224x224 sizes. These ratios were used to rescale the annotated coordinates for 224x224 images back to the original CXR image sizes.

\noindent \textbf{auto\_bbox\_pipeline\_coordinates\_1000\_images.txt} contains the bounding box coordinates that were automatically extracted by the Bbox pipeline for the different objects for images in the gold standard dataset. It is in a tabular format like with the ground truth for easier evaluation purposes.

\noindent \textbf{object-bbox-coordinates\_evaluation.ipynb} notebook calculates the bounding box object detection performance using ground truth files from the 4 M.D. annotators , as well as consolidating the final \textbf{gold\_bbox\_coordinate\_annotations\_1000images.csv}.

\noindent \textbf{Preprocess\_mimic\_cxr\_v2.0.0\_reports.ipynb} processes the reports (tokenize sentences and sort them into history, prelim or final report sentences) from the original MIMIC-CXR v2.0.0 and save output as \textbf{silver\_dataset/cxr-mimic-v2.0.0-processed-sentences\_all.txt}. Only sentences with object or attribute extractions ended up in the final scene graph jsons in the Chest ImaGenome dataset.

\noindent The \textbf{semantics} directory contains the object (\textbf{objects\_detectable\_by\_bbox\_pipeline\_v1.txt} and \textbf{objects\_extracted\_from\_reports\_v1.txt}), attribute (\textbf{attribute\_relations\_v1.txt}) and comoparison (\textbf{comparison\_relations\_v1.txt}) relations labels in the Chest ImaGenome dataset. It also contains \textbf{semantics/label\_to\_UMLS\_mapping.json}, which maps all Chest ImaGenome concepts to UMLS CUIs \cite{bodenreider2004unified}.



\newpage
\section{Dataset Evaluation}

% \begin{figure}[!ht]
% \centering
% \includegraphics[scale=0.45]{figures/Figure_6_lung_mediastinum_clavicle_bboxes.pdf}
% \caption{Sample CXR case with 17 overlaying anatomical bounding boxes.}
% \label{fig:bbox-sample}
% \end{figure}

Table \ref{tab:object-detect} reports anatomical location level object-to-attribute relations extraction performance by the scene graph extraction pipeline. The report numbers are calculated by a combination of notebooks: `generate\_scenegraph\_statistics.ipynb', `object-attribute-relation\_evaluation.ipynb' and `object-bbox-coordinates\_evaluation.ipynb'.

\begin{table}[th]
\centering
\caption{CXR image object detection evaluation results. \** These anatomical locations are extracted by the Bbox pipeline but they are not manually annotated in the gold standard dataset due to resource constraints. \*** The mediastinum bounding boxes were not directly annotated due to resource constraints. Mediastinum's bounding box boundary can be derived from the ground truth for the upper mediastinum and the cardiac silhouette.
}\label{tab:object-detect}
\resizebox{\textwidth}{!}{%
\begin{tabular}{|l|c|c|c|c|c|}
\hline
\multicolumn{1}{|c|}{\textbf{\begin{tabular}[c]{@{}c@{}}Bbox name \\ (object)\end{tabular}}} & \multicolumn{1}{c|}{\textbf{\begin{tabular}[c]{@{}c@{}}Object-attribute relations  \\  frequency (500 reports)\end{tabular}}} & \multicolumn{1}{c|}{\textbf{\begin{tabular}[c]{@{}c@{}}Relationships F1\\ (500 reports)\end{tabular}}} & \multicolumn{1}{c|}{\textbf{\begin{tabular}[c]{@{}c@{}}Bbox IoU \\ (over 1000 images)\end{tabular}}} & \multicolumn{1}{c|}{\textbf{\begin{tabular}[c]{@{}c@{}}\% Bboxes corrected \\ (1000 images)\end{tabular}}} & \multicolumn{1}{c|}{\textbf{\begin{tabular}[c]{@{}c@{}}\% Relations missing \\ Bbox coordinates \\ (over whole dataset)\end{tabular}}} \\ \hline
left lung & 1453 & 0.933 & 0.976 & 9.90\% & 0.03\% \\ \hline
right lung & 1436 & 0.937 & 0.983 & 6.30\% & 0.04\% \\ \hline
cardiac silhouette & 633 & 0.966 & 0.967 & 9.70\% & 0.01\% \\ \hline
mediastinum & 601 & 0.952 & ** & ** & 0.02\% \\ \hline
left lower lung zone & 609 & 0.932 & 0.955 & 8.60\% & 2.36\% \\ \hline
right lower lung zone & 580 & 0.902 & 0.968 & 6.00\% & 2.27\% \\ \hline
right hilar structures & 572 & 0.934 & 0.976 & 4.10\% & 1.91\% \\ \hline
left hilar structures & 571 & 0.944 & 0.971 & 4.30\% & 2.28\% \\ \hline
upper mediastinum & 359 & 0.940 & 0.994 & 1.40\% & 0.12\% \\ \hline
left costophrenic angle & 298 & 0.908 & 0.929 & 9.60\% & 0.63\% \\ \hline
right costophrenic angle & 286 & 0.918 & 0.944 & 6.90\% & 0.39\% \\ \hline
left mid lung zone & 173 & 0.940 & 0.967 & 5.70\% & 2.79\% \\ \hline
right mid lung zone & 169 & 0.830 & 0.968 & 5.30\% & 2.31\% \\ \hline
aortic arch & 144 & 0.965 & 0.991 & 1.40\% & 0.62\% \\ \hline
right upper lung zone & 117 & 0.873 & 0.972 & 5.80\% & 0.04\% \\ \hline
left upper lung zone & 83 & 0.811 & 0.968 & 6.40\% & 0.22\% \\ \hline
right hemidiaphragm & 78 & 0.947 & 0.955 & 7.90\% & 0.15\% \\ \hline
right clavicle & 71 & 0.615 & 0.986 & 2.80\% & 0.50\% \\ \hline
left clavicle & 67 & 0.642 & 0.983 & 3.00\% & 0.51\% \\ \hline
left hemidiaphragm & 65 & 0.930 & 0.944 & 11.30\% & 0.14\% \\ \hline
right apical zone & 58 & 0.852 & 0.969 & 5.40\% & 1.99\% \\ \hline
trachea & 57 & 0.983 & 0.995 & 0.90\% & 0.24\% \\ \hline
left apical zone & 47 & 0.938 & 0.963 & 6.20\% & 2.40\% \\ \hline
carina & 41 & 0.975 & 0.994 & 0.80\% & 1.47\% \\ \hline
svc & 19 & 0.973 & 0.995 & 0.70\% & 0.66\% \\ \hline
right atrium & 14 & 0.963 & 0.979 & 4.00\% & 0.18\% \\ \hline
cavoatrial junction & 5 & 1.000 & 0.977 & 4.30\% & 0.25\% \\ \hline
abdomen & 80 & 0.904 & * & * & 0.26\% \\ \hline
spine & 132 & 0.824 & * & * & 0.10\% \\ \hline
\end{tabular}%
}
\vspace{-0.3cm}
\end{table}


\newpage
\section{Pictorial Overview of Model Architectures}
Due to space limitations, we present overview figures for the models designed for Example Tasks 1 and 2 here.
\label{clinical_applications}

\begin{figure}[h]
\center
\includegraphics[width=0.9\textwidth,height=4cm]{figures/siamese.pdf}
\caption{Example Task 1 Model Overview. Given a pair of CXR images, we extract features for the anatomical regions of interest with a pretrained ResNet autoencoder, concatenate representations and pass them through a dense layer and a final classification layer.}
\vspace{-0.3cm}
\label{fig:siamese}
\end{figure} 

\begin{figure}[h]
\centering
\includegraphics[width=0.9\textwidth]{figures/ML-GCN.pdf}
\caption{Example Task 2 Model Overview. Given a pair of CXR images, we extract features for the anatomical regions of interest with a pretrained Faster R-CNN and a GCN to learn the label dependencies.}
\label{fig:gcn}
\vspace{-0.1cm}
\end{figure}

\newpage
\section{Qualitative Evaluation}
In Figure \ref{tab:findings}, we visualize the output from our model for the anatomical finding predictions of costophrenic angles and enlarged cardiac silhouette.
In Figure \ref{tab:gradcam}, we present an additional example, showing that the model is able to provide accurate localization information as well as predict the correct finding, i.e., showing accurate localization.


\begin{table}[h]
\begin{subtable}[t]{0.5\textwidth}
\resizebox{\textwidth}{!}{
\centering
\begin{tabular}{ccc}
\textbf{{Image 1}} & \textbf{{CS}}  & \textbf{{RCA}} \\ 
\includegraphics[width=0.4\textwidth,height=0.4\textwidth]{figures/vis/viz1.png} & \includegraphics[width=0.4\textwidth,height=0.4\textwidth]{figures/vis/cs_1.png} & \includegraphics[width=0.4\textwidth,height=0.4\textwidth]{figures/vis/rca_1.png} \\[0.15cm]
\myalign{l}{Ground Truth} & \myalign{l}{\textbf{No findings}} & \myalign{l}{\textbf{No findings}} \\[0.15cm] 
%\myalign{l}{CheXGCN} & \myalign{l}{\color{red} \textbf{L4} } & \myalign{l}{\color{red} \textbf{L1, L2} } \\[0.15cm] 
\myalign{l}{Our model \cite{agu2021anaxnet}} & \myalign{l}{\color{green} \textbf{No findings}} & \myalign{l}{\color{green} \textbf{No findings}} \\ 
\end{tabular}
}
\end{subtable} 
\hspace{0.2cm}
\begin{subtable}[h]{0.5\textwidth}
\resizebox{\textwidth}{!}{
\centering
\begin{tabular}{ccc}
\textbf{{Image 2}} & \textbf{{RCA}}  & \textbf{{LCA}} \\
\includegraphics[width=0.4\textwidth,height=0.4\textwidth]{figures/vis/viz2.png} & \includegraphics[width=0.4\textwidth,height=0.4\textwidth]{figures/vis/rca_2.png} & \includegraphics[width=0.4\textwidth,height=0.4\textwidth]{figures/vis/lca_2.png} \\[0.15cm] 
\myalign{l}{Ground Truth} & \myalign{l}{\textbf{L2}} & \myalign{l}{\textbf{L2}} \\[0.15cm]  
%\myalign{l}{CheXGCN} & \myalign{l}{\color{red} \textbf{No findings}} & \myalign{l}{\color{red} \textbf{No findings}} \\[0.15cm]  
\myalign{l}{Our model \cite{agu2021anaxnet}} & \myalign{l}{\color{green} \textbf{L2}} & \myalign{l}{\color{green} \textbf{L2}}
\end{tabular}
}
\end{subtable}
\vspace{0.2cm}
\captionof{figure}{Examples of the prediction results. The overall chest X-ray image is shown alongside two anatomical regions, and predictions are compared against the ground-truth labels.
} 
\label{tab:findings}
\end{table}

\begin{figure}[h]
  \centering
   \resizebox{0.8\textwidth}{!}{
  \subfloat[Original Image]{
  \includegraphics[width=0.3\textwidth, height=0.3\textwidth]{figures/vis/orig_ecs_viz.png}
  \label{fig:f1}}
  %\hfill
  %\subfloat[GlobalView \scriptsize{(Grad-CAM)}]{
  %\includegraphics[width=0.3\textwidth, height=0.3\textwidth]{figures/vis/ecs_viz.png}  \label{fig:f2}}
  \hfill
  \subfloat[Our model \cite{agu2021anaxnet}]{
  \includegraphics[width=0.3\textwidth, height=0.3\textwidth]{figures/vis/ecs_box.png} \label{fig:f3}}
  }
  \caption{Example image with enlarged cardiac silhouette, showing that the trained model detects the finding in the correct bounding box.}
  \label{tab:gradcam}
\end{figure}
%\input{sections/tables_supplement}


\end{document}


% from Ismini (External) to Everyone:    2:37  PM
% https://www.tablesgenerator.com/
% from Ismini (External) to Everyone:    2:38  PM
% \href{link}{text}
% from Satyananda Kashyap (IBM) to Everyone:    2:38  PM
% \usepackage{hyperref}