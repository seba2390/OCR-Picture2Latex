\documentclass{article}

% if you need to pass options to natbib, use, e.g.:
%     \PassOptionsToPackage{numbers, compress}{natbib}
% before loading neurips_data_2021

\PassOptionsToPackage{square,comma,numbers,sort&compress}{natbib}

% ready for submission
\usepackage[preprint]{neurips_data_2021}

% to compile a preprint version, add the [preprint] option:
%     \usepackage[preprint]{neurips_data_2021}
% This will indicate that the work is currently under review.

% to compile a camera-ready version, add the [final] option:
%     \usepackage[final]{neurips_data_2021}

% to avoid loading the natbib package, add option nonatbib:
%    \usepackage[nonatbib]{neurips_data_2021}

% Submissions to the datasets and benchmarks are non-anonymous. If you do want to compile an anonymous version for other purposes, you can add the [anonymous] option:
%     \usepackage[anonymous]{neurips_data_2021}
% This will hide all author names.

\usepackage{wrapfig}
\usepackage[utf8]{inputenc} % allow utf-8 input
\usepackage[T1]{fontenc}    % use 8-bit T1 fonts
\usepackage{hyperref}       % hyperlinks
\usepackage{url}            % simple URL typesetting
\usepackage{booktabs}       % professional-quality tables
\usepackage{amsfonts}       % blackboard math symbols
\usepackage{nicefrac}       % compact symbols for 1/2, etc.
\usepackage{microtype}      % microtypography
\usepackage{algorithm2e}
\usepackage{hyperref}
\usepackage{graphicx}
%\usepackage{float}
\usepackage{multirow}
\usepackage[table,xcdraw]{xcolor}
\usepackage{subcaption}

\renewcommand{\topfraction}{1}
\renewcommand{\dbltopfraction}{1}
\renewcommand{\floatpagefraction}{1}
\renewcommand{\textfraction}{0}
\newcommand{\hideseg}[1]{} %hide

\usepackage{xspace}
%COMMENTS FOR ISMINI
\newcommand{\il}[1]{\textsf{\textbf{\color{magenta}{\small{[IL: #1]}}}}}
%COMMENTS FOR MEHDI
\newcommand{\mm}[1]{\textsf{\textbf{\color{blue}{\small{[MM: #1]}}}}}
\newcommand*{\myalign}[2]{\multicolumn{1}{#1}{#2}}


\title{Chest ImaGenome Dataset for Clinical Reasoning}

% \author[1,*]{Joy T. Wu}
% \author[2]{Nkechinyere N. Agu}
% \author[1,3]{Ismini Lourentzou}
% \author[4,$\dag$]{Arjun Sharma} 
% \author[5,$\dag$]{Joseph Paguio}
% \author[5,$\dag$]{Jasper Seth Yao}
% \author[6,$\dag$]{Edward Christopher Dee}
% \author[4,$\dag$]{William Mitchell}
% \author[1]{Satyananda Kashyap}
% \author[1]{Andrea Giovannini}
% \author[4]{Leo A. Celi}
% % & any others who helped
% \author[1,*]{Mehdi Moradi} 
% % order tbd -- happy to discuss

% \affil[1]{IBM Almaden Research Center, San Jose, CA 95120, USA}
% \affil[2]{Rensselaer Polytechnic Institute, Troy, NY 12180, USA}
% \affil[3]{Virginia Polytechnic Institute and State University, Blacksburg, VA 24061, USA}
% \affil[4]{MIT Critical Data, Cambridge, MA 02139, USA}
%\affil[5]{Albert Einstein Healthcare Network-Philadelphia Campus, PA 19141, USA}
% \affil[6]{MIT Harvard Medical School, Boston, MA 02115, USA}

% \affil[*]{Corresponding author(s): Joy T. Wu (research@joytywu.net), Mehdi Moradi (mmoradi@us.ibm.com)}

% \affil[$\dag$]{These authors contributed equally to this work}

% The \author macro works with any number of authors. There are two commands
% used to separate the names and addresses of multiple authors: \And and \AND.
%
% Using \And between authors leaves it to LaTeX to determine where to break the
% lines. Using \AND forces a line break at that point. So, if LaTeX puts 3 of 4
% authors names on the first line, and the last on the second line, try using
% \AND instead of \And before the third author name.

% \author[1,*]{Joy T. Wu}
% \author[2]{Nkechinyere N. Agu}
% \author[1,3]{Ismini Lourentzou}
% \author[4,$\dag$]{Arjun Sharma} 
% \author[5,$\dag$]{Joseph Alexander Paguio}
% \author[5,$\dag$]{Jasper Seth Yao}
% \author[6,$\dag$]{Edward Christopher Dee}
% \author[4,$\dag$]{William Mitchell}
% \author[1]{Satyananda Kashyap}
% \author[1]{Andrea Giovannini}
% \author[4]{Leo A. Celi}
% % & any others who helped
% \author[1,*]{Mehdi Moradi} 

%\hide{\thanks{Use footnote for providing further information
%    about author (webpage, alternative address)---\emph{not} for acknowledging funding agencies.}}

\author{Joy T. Wu\textsuperscript{1}, Nkechinyere N. Agu\textsuperscript{2}, Ismini Lourentzou\textsuperscript{3}, Arjun Sharma\textsuperscript{4}, Joseph A. Paguio\textsuperscript{5}, \and \textbf{Jasper S. Yao\textsuperscript{5}, Edward C. Dee\textsuperscript{6}, William Mitchell\textsuperscript{4}, Satyananda Kashyap\textsuperscript{1},} \and \textbf{Andrea Giovannini\textsuperscript{1}, Leo A. Celi\textsuperscript{4}, Mehdi Moradi\textsuperscript{1}} \\
\textsuperscript{1}{IBM Almaden Research Center, San Jose, CA 95120, USA}\\
\textsuperscript{2}{Rensselaer Polytechnic Institute, Troy, NY 12180, USA}\\
\textsuperscript{3}{Virginia Polytechnic Institute and State University, Blacksburg, VA 24061, USA}\\
\textsuperscript{4}{MIT Critical Data, Cambridge, MA 02139, USA}\\
\textsuperscript{5}{Albert Einstein Healthcare Network-Philadelphia Campus, PA 19141, USA}\\
\textsuperscript{6}{Harvard Medical School, Boston, MA 02115, USA} \\
  % examples of more authors
  % \And
  % Coauthor \\
  % Affiliation \\
  % Address \\
  % \texttt{email} \\
  % \AND
  % Coauthor \\
  % Affiliation \\
  % Address \\
  % \texttt{email} \\
  % \And
  % Coauthor \\
  % Affiliation \\
  % Address \\
  % \texttt{email} \\
  % \And
  % Coauthor \\
  % Affiliation \\
  % Address \\
  % \texttt{email} \\
}


\begin{document}

\maketitle

\begin{abstract}
  %In recent years, with the release of multiple large datasets, the automatic interpretation of chest X-ray (CXR) images with deep learning models have become feasible for specific abnormalities or for generating preliminary reports. However,   reports of performance reaching similar levels to that of radiologists, 
Despite the progress in automatic detection of radiologic findings from chest X-ray (CXR) images in recent years, a quantitative evaluation of the explainability of these models is hampered by the lack of locally labeled datasets for different findings. With the exception of a few expert-labeled small-scale datasets for specific findings, such as pneumonia and pneumothorax, most of the CXR deep learning models to date are trained on global "weak" labels extracted from text reports, or trained via a joint image and unstructured text learning strategy. Inspired by the Visual Genome effort in the computer vision community, we constructed the first Chest ImaGenome dataset with a scene graph data structure to describe $242,072$ images. Local annotations are automatically produced using a joint rule-based natural language processing (NLP) and atlas-based bounding box detection pipeline. Through a radiologist constructed CXR ontology, the annotations for each CXR are connected as an anatomy-centered scene graph, useful for image-level reasoning and multimodal fusion applications. Overall, we provide: i) $1,256$ combinations of relation annotations between $29$ CXR anatomical locations (objects with bounding box coordinates) and their attributes, structured as a scene graph per image, ii) over $670,000$ localized comparison relations (for improved, worsened, or no change) between the anatomical locations across sequential exams, as well as ii) a manually annotated gold standard scene graph dataset from $500$ unique patients.

%In our work, a joint rule-based natural language processing (NLP) and CXR atlas-based bounding box detection pipeline are used to automatically label 242072 frontal MIMIC CXRs locally. 
\end{abstract}

% !TEX root = ../arxiv.tex

Unsupervised domain adaptation (UDA) is a variant of semi-supervised learning \cite{blum1998combining}, where the available unlabelled data comes from a different distribution than the annotated dataset \cite{Ben-DavidBCP06}.
A case in point is to exploit synthetic data, where annotation is more accessible compared to the costly labelling of real-world images \cite{RichterVRK16,RosSMVL16}.
Along with some success in addressing UDA for semantic segmentation \cite{TsaiHSS0C18,VuJBCP19,0001S20,ZouYKW18}, the developed methods are growing increasingly sophisticated and often combine style transfer networks, adversarial training or network ensembles \cite{KimB20a,LiYV19,TsaiSSC19,Yang_2020_ECCV}.
This increase in model complexity impedes reproducibility, potentially slowing further progress.

In this work, we propose a UDA framework reaching state-of-the-art segmentation accuracy (measured by the Intersection-over-Union, IoU) without incurring substantial training efforts.
Toward this goal, we adopt a simple semi-supervised approach, \emph{self-training} \cite{ChenWB11,lee2013pseudo,ZouYKW18}, used in recent works only in conjunction with adversarial training or network ensembles \cite{ChoiKK19,KimB20a,Mei_2020_ECCV,Wang_2020_ECCV,0001S20,Zheng_2020_IJCV,ZhengY20}.
By contrast, we use self-training \emph{standalone}.
Compared to previous self-training methods \cite{ChenLCCCZAS20,Li_2020_ECCV,subhani2020learning,ZouYKW18,ZouYLKW19}, our approach also sidesteps the inconvenience of multiple training rounds, as they often require expert intervention between consecutive rounds.
We train our model using co-evolving pseudo labels end-to-end without such need.

\begin{figure}[t]%
    \centering
    \def\svgwidth{\linewidth}
    \input{figures/preview/bars.pdf_tex}
    \caption{\textbf{Results preview.} Unlike much recent work that combines multiple training paradigms, such as adversarial training and style transfer, our approach retains the modest single-round training complexity of self-training, yet improves the state of the art for adapting semantic segmentation by a significant margin.}
    \label{fig:preview}
\end{figure}

Our method leverages the ubiquitous \emph{data augmentation} techniques from fully supervised learning \cite{deeplabv3plus2018,ZhaoSQWJ17}: photometric jitter, flipping and multi-scale cropping.
We enforce \emph{consistency} of the semantic maps produced by the model across these image perturbations.
The following assumption formalises the key premise:

\myparagraph{Assumption 1.}
Let $f: \mathcal{I} \rightarrow \mathcal{M}$ represent a pixelwise mapping from images $\mathcal{I}$ to semantic output $\mathcal{M}$.
Denote $\rho_{\bm{\epsilon}}: \mathcal{I} \rightarrow \mathcal{I}$ a photometric image transform and, similarly, $\tau_{\bm{\epsilon}'}: \mathcal{I} \rightarrow \mathcal{I}$ a spatial similarity transformation, where $\bm{\epsilon},\bm{\epsilon}'\sim p(\cdot)$ are control variables following some pre-defined density (\eg, $p \equiv \mathcal{N}(0, 1)$).
Then, for any image $I \in \mathcal{I}$, $f$ is \emph{invariant} under $\rho_{\bm{\epsilon}}$ and \emph{equivariant} under $\tau_{\bm{\epsilon}'}$, \ie~$f(\rho_{\bm{\epsilon}}(I)) = f(I)$ and $f(\tau_{\bm{\epsilon}'}(I)) = \tau_{\bm{\epsilon}'}(f(I))$.

\smallskip
\noindent Next, we introduce a training framework using a \emph{momentum network} -- a slowly advancing copy of the original model.
The momentum network provides stable, yet recent targets for model updates, as opposed to the fixed supervision in model distillation \cite{Chen0G18,Zheng_2020_IJCV,ZhengY20}.
We also re-visit the problem of long-tail recognition in the context of generating pseudo labels for self-supervision.
In particular, we maintain an \emph{exponentially moving class prior} used to discount the confidence thresholds for those classes with few samples and increase their relative contribution to the training loss.
Our framework is simple to train, adds moderate computational overhead compared to a fully supervised setup, yet sets a new state of the art on established benchmarks (\cf \cref{fig:preview}).

\section{Related Work}\label{sec:related}
 
The authors in \cite{humphreys2007noncontact} showed that it is possible to extract the PPG signal from the video using a complementary metal-oxide semiconductor camera by illuminating a region of tissue using through external light-emitting diodes at dual-wavelength (760nm and 880nm).  Further, the authors of  \cite{verkruysse2008remote} demonstrated that the PPG signal can be estimated by just using ambient light as a source of illumination along with a simple digital camera.  Further in \cite{poh2011advancements}, the PPG waveform was estimated from the videos recorded using a low-cost webcam. The red, green, and blue channels of the images were decomposed into independent sources using independent component analysis. One of the independent sources was selected to estimate PPG and further calculate HR, and HRV. All these works showed the possibility of extracting PPG signals from the videos and proved the similarity of this signal with the one obtained using a contact device. Further, the authors in \cite{10.1109/CVPR.2013.440} showed that heart rate can be extracted from features from the head as well by capturing the subtle head movements that happen due to blood flow.

%
The authors of \cite{kumar2015distanceppg} proposed a methodology that overcomes a challenge in extracting PPG for people with darker skin tones. The challenge due to slight movement and low lighting conditions during recording a video was also addressed. They implemented the method where PPG signal is extracted from different regions of the face and signal from each region is combined using their weighted average making weights different for different people depending on their skin color. 
%

There are other attempts where authors of \cite{6523142,6909939, 7410772, 7412627} have introduced different methodologies to make algorithms for estimating pulse rate robust to illumination variation and motion of the subjects. The paper \cite{6523142} introduces a chrominance-based method to reduce the effect of motion in estimating pulse rate. The authors of \cite{6909939} used a technique in which face tracking and normalized least square adaptive filtering is used to counter the effects of variations due to illumination and subject movement. 
The paper \cite{7410772} resolves the issue of subject movement by choosing the rectangular ROI's on the face relative to the facial landmarks and facial landmarks are tracked in the video using pose-free facial landmark fitting tracker discussed in \cite{yu2016face} followed by the removal of noise due to illumination to extract noise-free PPG signal for estimating pulse rate. 

Recently, the use of machine learning in the prediction of health parameters have gained attention. The paper \cite{osman2015supervised} used a supervised learning methodology to predict the pulse rate from the videos taken from any off-the-shelf camera. Their model showed the possibility of using machine learning methods to estimate the pulse rate. However, our method outperforms their results when the root mean squared error of the predicted pulse rate is compared. The authors in \cite{hsu2017deep} proposed a deep learning methodology to predict the pulse rate from the facial videos. The researchers trained a convolutional neural network (CNN) on the images generated using Short-Time Fourier Transform (STFT) applied on the R, G, \& B channels from the facial region of interests.
The authors of \cite{osman2015supervised, hsu2017deep} only predicted pulse rate, and we extended our work in predicting variance in the pulse rate measurements as well.

All the related work discussed above utilizes filtering and digital signal processing to extract PPG signals from the video which is further used to estimate the PR and PRV.  %
The method proposed in \cite{kumar2015distanceppg} is person dependent since the weights will be different for people with different skin tone. In contrast, we propose a deep learning model to predict the PR which is independent of the person who is being trained. Thus, the model would work even if there is no prior training model built for that individual and hence, making our model robust. 

%









\section{Proposed Approach} \label{sec:method}

Our goal is to create a unified model that maps task representations (e.g., obtained using task2vec~\cite{achille2019task2vec}) to simulation parameters, which are in turn used to render synthetic pre-training datasets for not only tasks that are seen during training, but also novel tasks.
This is a challenging problem, as the number of possible simulation parameter configurations is combinatorially large, making a brute-force approach infeasible when the number of parameters grows. 

\subsection{Overview} 

\cref{fig:controller-approach} shows an overview of our approach. During training, a batch of ``seen'' tasks is provided as input. Their task2vec vector representations are fed as input to \ours, which is a parametric model (shared across all tasks) mapping these downstream task2vecs to simulation parameters, such as lighting direction, amount of blur, background variability, etc.  These parameters are then used by a data generator (in our implementation, built using the Three-D-World platform~\cite{gan2020threedworld}) to generate a dataset of synthetic images. A classifier model then gets pre-trained on these synthetic images, and the backbone is subsequently used for evaluation on specific downstream task. The classifier's accuracy on this task is used as a reward to update \ours's parameters. 
Once trained, \ours can also be used to efficiently predict simulation parameters in {\em one-shot} for ``unseen'' tasks that it has not encountered during training. 


\subsection{\ours Model} 


Let us denote \ours's parameters with $\theta$. Given the task2vec representation of a downstream task $\bs{x} \in \mc{X}$ as input, \ours outputs simulation parameters $a \in \Omega$. The model consists of $M$ output heads, one for each simulation parameter. In the following discussion, just as in our experiments, each simulation parameter is discretized to a few levels to limit the space of possible outputs. Each head outputs a categorical distribution $\pi_i(\bs{x}, \theta) \in \Delta^{k_i}$, where $k_i$ is the number of discrete values for parameter $i \in [M]$, and $\Delta^{k_i}$, a standard $k_i$-simplex. The set of argmax outputs $\nu(\bs{x}, \theta) = \{\nu_i | \nu_i = \argmax_{j \in [k_i]} \pi_{i, j} ~\forall i \in [M]\}$ is the set of simulation parameter values used for synthetic data generation. Subsequently, we drop annotating the dependence of $\pi$ and $\nu$ on $\theta$ and $\bs{x}$ when clear.

\subsection{\ours Training} 


Since Task2Sim aims to maximize downstream accuracy after pre-training, we use this accuracy as the reward in our training optimization\footnote{Note that our rewards depend only on the task2vec input and the output action and do not involve any states, and thus our problem can be considered similar to a stateless-RL or contextual bandits problem \cite{langford2007epoch}.}.
Note that this downstream accuracy is a non-differentiable function of the output simulation parameters (assuming any simulation engine can be used as a black box) and hence direct gradient-based optimization cannot be used to train \ours. Instead, we use REINFORCE~\cite{williams1992simple}, to approximate gradients of downstream task performance with respect to model parameters $\theta$. 

\ours's outputs represent a distribution over ``actions'' corresponding to different values of the set of $M$ simulation parameters. $P(a) = \prod_{i \in [M]} \pi_i(a_i)$ is the probability of picking action $a = [a_i]_{i \in [M]}$, under policy $\pi = [\pi_i]_{i \in [M]}$. Remember that the output $\pi$ is a function of the parameters $\theta$ and the task representation $\bs{x}$. To train the model, we maximize the expected reward under its policy, defined as
\begin{align}
    R = \E_{a \in \Omega}[R(a)] = \sum_{a \in \Omega} P(a) R(a)
\end{align}
where $\Omega$ is the space of all outputs $a$ and $R(a)$ is the reward when parameter values corresponding to action $a$ are chosen. Since reward is the downstream accuracy, $R(a) \in [0, 100]$.  
Using the REINFORCE rule, we have
\begin{align}
    \nabla_{\theta} R 
    &= \E_{a \in \Omega} \left[ (\nabla_{\theta} \log P(a)) R(a) \right] \\
    &= \E_{a \in \Omega} \left[ \left(\sum_{i \in [M]} \nabla_{\theta} \log \pi_i(a_i) \right) R(a) \right]
\end{align}
where the 2nd step comes from linearity of the derivative. In practice, we use a point estimate of the above expectation at a sample $a \sim (\pi + \epsilon)$ ($\epsilon$ being some exploration noise added to the Task2Sim output distribution) with a self-critical baseline following \cite{rennie2017self}:
\begin{align} \label{eq:grad-pt-est}
    \nabla_{\theta} R \approx \left(\sum_{i \in [M]} \nabla_{\theta} \log \pi_i(a_i) \right) \left( R(a) - R(\nu) \right) 
\end{align}
where, as a reminder $\nu$ is the set of the distribution argmax parameter values from the \name{} model heads.

A pseudo-code of our approach is shown in \cref{alg:train}.  Specifically, we update the model parameters $\theta$ using minibatches of tasks sampled from a set of ``seen'' tasks. Similar to \cite{oh2018self}, we also employ self-imitation learning biased towards actions found to have better rewards. This is done by keeping track of the best action encountered in the learning process and using it for additional updates to the model, besides the ones in \cref{ln:update} of \cref{alg:train}. 
Furthermore, we use the test accuracy of a 5-nearest neighbors classifier operating on features generated by the pretrained backbone as a proxy for downstream task performance since it is computationally much faster than other common evaluation criteria used in transfer learning, e.g., linear probing or full-network finetuning. Our experiments demonstrate that this proxy evaluation measure indeed correlates with, and thus, helps in final downstream performance with linear probing or full-network finetuning. 






\begin{algorithm}
\DontPrintSemicolon
 \textbf{Input:} Set of $N$ ``seen'' downstream tasks represented by task2vecs $\mc{T} = \{\bs{x}_i | i \in [N]\}$. \\
 Given initial Task2Sim parameters $\theta_0$ and initial noise level $\epsilon_0$\\
 Initialize $a_{max}^{(i)} | i \in [N]$ the maximum reward action for each seen task \\
 \For{$t \in [T]$}{
 Set noise level $\epsilon = \frac{\epsilon_0}{t} $ \\
 Sample minibatch $\tau$ of size $n$ from $\mc{T}$  \\
 Get \ours output distributions $\pi^{(i)} | i \in [n]$ \\
 Sample outputs $a^{(i)} \sim \pi^{(i)} + \epsilon$ \\
 Get Rewards $R(a^{(i)})$ by generating a synthetic dataset with parameters $a^{(i)}$, pre-training a backbone on it, and getting the 5-NN downstream accuracy using this backbone \\
 Update $a_{max}^{(i)}$ if $R(a^{(i)}) > R(a_{max}^{(i)})$ \\
 Get point estimates of reward gradients $dr^{(i)}$ for each task in minibatch using \cref{eq:grad-pt-est} \\
 $\theta_{t,0} \leftarrow \theta_{t-1} + \frac{\sum_{i \in [n]} dr^{(i)}}{n}$ \label{ln:update} \\
 \For{$j \in [T_{si}]$}{ 
    \tcp{Self Imitation}
    Get reward gradient estimates $dr_{si}^{(i)}$ from \cref{eq:grad-pt-est} for $a \leftarrow a_{max}^{(i)}$ \\
    $\theta_{t, j}  \leftarrow \theta_{t, j-1} + \frac{\sum_{i \in [n]} dr_{si}^{(i)}}{n}$
 }
 $\theta_{t} \leftarrow \theta_{t, T_{si}}$
 }
 \textbf{Output}: Trained model with parameters $\theta_T$. 
 \caption{Training Task2Sim}
 \label{alg:train}  
\end{algorithm}

\section*{Data description}

The Chest ImaGenome dataset is committed to the PhysioNet repository in two main directories, one for the scene graphs that are automatically generated (``silver\_dataset''), and another for the 500 unique patient subset that was manually validated and corrected (``gold\_dataset''). Overall, $242,072$ scene graphs were automatically derived from $217,013$ unique CXR studies. The nodes and edges in the graph are defined in detail in Supplementary Table \ref{tab:define_nodes_edges}. On average 7 anatomical objects and 5 attributes are extracted from each study report. However, up to 29 anatomy objects can be detected in each CXR image with a percentage of misses < 0.02\% for most objects (See Table \ref{tab:object-detect} in Supplementary material). In addition, even without considering the related attribute(s), $678,543$ object-object comparison relations are extracted between anatomies across $128,468$ pairs of sequential CXR images. Detailed dataset characteristics are explained and provided in the PhysioNet repository (generate\_scenegraph\_statistics.ipynb). Figure \ref{fig:bbox-sample} shows an example of all the anatomical bounding boxes.

\vspace{-5pt}
\subsection*{Chest ImaGenome Scene Graph JSONs}
\vspace{-2pt}
%\noindent \textbf{Chest ImaGenome Scene Graph JSONs}: 
The `silver\_dataset/scene\_graph.zip' file is a directory that contains multiple JSON files, one for each scene graph. Each scene graph describes one frontal chest X-ray image. The structure for each scene graph JSON is described by components for easier explanation in Supplementary (Section \ref{jsonsg}). The first level of the JSON in Supplementary (\ref{json1}) describes the patient or study level information that may not be available in the image. The fields are: `image\_id' (dicom\_id in MIMIC-CXR), `viewpoint' (AP or PA), `patient\_id' (subject\_id in MIMIC-CXR), `study\_id' (study\_id in MIMIC-CXR), `gender' and `age\_decile' demographics (from MIMIC-CXR's metadata), `reason for exam' (patient history sentence(s) from the CXR reports with age removed), `StudyOrder' (the order of the CXR study for the patient, which is derived from chronologically ordering the DICOM timestamps), and `StudyDateTime; (from MIMIC's dicom metadata, which had been de-identified into the future).

% \hideseg{
% \vspace{-0.3cm}
% \begin{footnotesize}
% \begin{verbatim}
% {
%  `chest_imageimage_id': `10cd06e9-5443fef9-9afbe903-e2ce1eb5-dcff1097',
%  `viewpoint': `AP', `patient_id': 10063856, `study_id': 56759094,
%  `gender': `F', `age_decile': `50-60',
%  `reason_for_exam': `___F with hypotension.  Evaluate for pneumonia.',
%  `StudyOrder': 2, `StudyDateTime': `2178-10-05 15:05:32 UTC',
%  `objects': [ <...list of {} for each object...> ],
%  `attributes':[ <...list of {} for each object...> ],
%  `relationships':[ <...list of {} of comparison relationships between objects 
%  from sequential exams for the same patient...> ] 
% }
% \end{verbatim}
% \end{footnotesize}
% }

For each scene graph, there are 3 separate nested fields to describe the ``objects'' on the CXR images, the ``attributes'' related to the different objects as extracted from the corresponding reports, and ``relationships'' to describe comparison relations between sequential CXR images for the same patient. These 3 fields are a list of dictionaries, where the format of each dictionary is modeled after the respective JSONs in the Visual Genome dataset \cite{krishna2017visual}.

For objects, each dictionary has the format shown in Supplementary (\ref{json2}). The `object\_id' is unique across the whole dataset for the anatomical location on the particular image. Fields `x1', `y1', `x2', `y2', `width' and `height' are for a padded and resized 224x224 CXR frontal image, where coordinates `x1', `y1' are for the top left corner of the bounding box and `x2', `y2' are for the bottom right corner. The bounding box coordinates in the original image are denoted with `original\_*'. The remaining fields: `bbox\_name' is the name given to the anatomical location within the Chest ImaGenome dataset, and is useful for lookups in other parts of the scene graph JSON; `synsets' contain the UMLS CUI for the anatomical location concept; and the `name' is the UMLS name for that CUI \cite{bodenreider2004unified}. Note that CXRs are 2D images of a 3D structure so there are many overlying anatomical locations. A sample of 17 of the anatomical objects is plotted on a CXR as shown in Figure \ref{fig:bbox-sample}.
\vspace{-5pt}

\begin{figure}[!ht]
\centering
\includegraphics[scale=0.3]{figures/Figure_6_lung_mediastinum_clavicle_bboxes.pdf}
\caption{Sample CXR case with 17 overlaying clavicles, lung and mediastinum related anatomical bounding boxes (objects).}
\label{fig:bbox-sample}
\vspace{-12pt}
\end{figure}

% \hideseg{
% \vspace{-0.3cm}
% \begin{footnotesize}
% \begin{verbatim}
% {
%   `object_id': `10cd06e9-5443fef9-9afbe903-e2ce1eb5-dcff1097_right upper lung zone',
%   `x1': 48, `y1': 39, `x2': 111, `y2': 93,
%   `width': 63, `height': 54,
%   `bbox_name': `right upper lung zone',
%   `synsets': [`C0934570'],
%   `name': `Right upper lung zone',
%   `original_x1': 395, `original_y1': 532,
%   `original_x2': 1255, `original_y2': 1268,
%   `original_width': 860, `original_height': 736
% }
% \end{verbatim}
% \end{footnotesize}
% }

Each attribute dictionary, e.g., Supplementary  (\ref{json3}), aims to summarize all the CXR attribute descriptions for one anatomical location (`bbox\_name'). This means, for a particular CXR anatomical location, all the sentences describing attributes related to it have been grouped into the `phrases' field, where the order of sentences in the original report has been maintained. However, an anatomical location may not always be described or implied in the report. In that case, looking up dictionary[`bbox\_name'] will be False. The fields `synsets' and `name' are the same as in the objects' dictionaries, where they describe the UMLS CUI information for the anatomical location concept.
% \hideseg{
% \vspace{-0.3cm}
% \begin{footnotesize}
% \begin{verbatim}
% {
%   `right lung': True, `bbox_name': `right lung',
%   `synsets': [`C0225706'], `name': `Right lung',
%   `attributes': [[`anatomicalfinding|no|lung opacity',
%   `anatomicalfinding|no|pneumothorax',  `nlp|yes|normal'],
%   [`anatomicalfinding|no|pneumothorax']],
%   `attributes_ids': [[`CL556823', `C1963215;;C0032326', `C1550457'],
%   [`C1963215;;C0032326']],
%   `phrases': [`Right lung is clear without pneumothorax.', 
%   `No pneumothorax identified.'],
%   `phrase_IDs': [`56759094|10', `56759094|14'],
%   `sections': [`finalreport', `finalreport'],
%   `comparison_cues': [[], []],
%   `temporal_cues': [[], []],
%   `severity_cues': [[], []],
%   `texture_cues': [[], []],
%   `object_id': `10cd06e9-5443fef9-9afbe903-e2ce1eb5-dcff1097_right lung'
% }
% \end{verbatim}
% \end{footnotesize}
% }

The `attributes' field contains the relations between the anatomical location and the CXR attributes extracted from the respective sentences. Note that there can be multiple attributes extracted from each sentence. Therefore, the `attributes' field is a list of lists. The `attributes' in the lists follow the pattern of < categoryID | relation | label\_name >, where `categoryID' is the radiology semantic category the authors gave to the CXR concept in consultation with multiple radiologists, and relation is the NLP context relating the label\_name to the anatomical location as an attribute. If the relation is `no', then the `label\_name' is specifically negated in the sentence. If the relation is 'yes', then the `label\_name` is affirmed in the sentence. The order of the lists in the `attribute\_ids' field follow the lists in the `attributes' field and map each `label\_name' to UMLS CUIs. Thus, the way the Chest ImaGenome dataset is formulated, one can interpret a statement such as the `right lung' <has no> `lung opacity' as true in the extracted radiology knowledge graph, whereby each node has been mapped to an externally recognized ontology. 

The certainty of each relation in the CXR knowledge graph can be optionally further modified by the cues from the `severity\_cues' and `temporal\_cues' fields in each attribute dictionary. The severity cues can include `hedge', `mild', `moderate' or `severe', which are only assigned by co-occurrence at the sentence level. These extractions can benefit from future NLP improvement. Similarly, the temporal cues can modify the relation as either `acute' or `chronic' depending on clinical use cases.

The Chest ImaGenome categoryIDs can be used to differentiate the use case for different attributes:

$\bullet$ \textbf{anatomicalfinding} - findings of anatomies where there is some subjectivity in the grouping of the phrases used to extract the labels.
\vspace{-2pt}

$\bullet$ \textbf{disease} - descriptions that are more diagnostic level and often require patient information outside the image and most subjective to the reading radiologist's inference/impression.
\vspace{-2pt}

$\bullet$ \textbf{nlp} - normal / abnormal descriptions about different anatomical locations and can be subjective.
\vspace{-2pt}

$\bullet$ \textbf{technicalassessment} - image quality issues affecting interpretation of CXR observations.
\vspace{-2pt}

$\bullet$ \textbf{tubesandlines} - medical support devices where radiologists need to report any placement issues.
\vspace{-2pt}

$\bullet$ \textbf{devices}: medical devices where placement issues are less relevant
\vspace{-2pt}

$\bullet$ \textbf{texture} - these are only present in the 'texture\_cues' field, we kept a set of highly non-specific attributes (e.g. opacity, lucency, interstitial, airspace) that tend to form the initial most objective descriptions about what is observed in the images by radiologists. 

Finally, for comparison relationships, each dictionary has the format shown in Supplementary (\ref{json4}). Each relationship dictionary describes the comparison relation(s) relevant for only one anatomical location (`bbox\_name'). The `relationship\_id' uniquely identifies each comparison relationship between the object (`subject\_id') on the current exam and the object (`object\_id' for the same anatomical location) from the previous exam. The `predicate' and `synsets' are the UMLS CUIs for `relationship\_names', which is a list with usually one (but could be more) comparison relation type, which can be in [`comparison|yes|improved', `comparison|yes|worsened', `comparison|yes|no change']. The `attributes' field records the attributes that are related to the anatomical location as per the sentence from the original report (kept in the `phrase' field) that describes the comparison relationship.

% \hideseg{
% \vspace{-0.3cm}
% \begin{footnotesize}
% \begin{verbatim}
% {
%   `relationship_id': `56759094|7_54814005_C0929215_10cd06e9_4bb710ab',
%   `predicate': ``['No status change']'',
%   `synsets': [`C0442739'],
%   `relationship_names': [`comparison|yes|no change'],
%   `relationship_contexts': [1.0],
%   `phrase': `Compared with the prior radiograph, there is a persistent veil 
%   -like opacity\n over the left hemithorax, with a crescent of air surrounding 
%   the aortic arch,\n in keeping with continued left upper lobe collapse.',
%   `attributes': [`anatomicalfinding|yes|atelectasis',
%   `anatomicalfinding|yes|lobar/segmental collapse',
%   `anatomicalfinding|yes|lung opacity', `nlp|yes|abnormal'],
%   `bbox_name': `left upper lung zone',
%   `subject_id': `10cd06e9-5443fef9-9afbe903-e2ce1eb5-dcff1097_left upper lung zone',
%   `object_id': `4bb710ab-ab7d4781-568bcd6e-5079d3e6-7fdb61b6_left upper lung zone'
% }
% \end{verbatim}
% \end{footnotesize}
% }

% Not all the sentences in the MIMIC-CXR v2.0.0 reports have made it into the Chest ImaGenome dataset, which only contains sentences that have the specific objects, attributes or relations targeted by version 1.0.0 of the dataset. We provide the preprocessing steps (Preprocess_mimic_cxr_v2.0.0_reports.ipynb) done to index the sentences from the original text reports in the "utils" directory, the output of which is cxr-mimic-v2.0.0-processed-sentences_all.txt.

\vspace{-5pt}
\subsection*{CXR Scene Graphs Rendered in an Enriched RDF Format}
\vspace{-2pt}
%\textbf{CXR Scene Graphs Rendered in an enriched RDF Format}
Supplementary (\ref{json5}):
Radiology report sentences are fairly repetitive. Therefore, in the scene graph JSONS, one could see similar information described multiple times in different sentences for a study. In addition, in the MIMIC reports we worked with, each report could also have a preliminary read section (recorded by trainee radiologists - i.e., resident M.D.s) that comes before the final report section (approved by a fully trained and experienced radiologist). Therefore, occasionally, the extraction from the sentences near the beginning of a CXR report can be different from the conclusion sentences later in the report. To render the scene graphs easier for downstream utilization, we also provide post-processing utils (scenegraph\_postprocessing.py) to roll the annotations up to the study level for each relation. This is done by taking the last relation extracted for each anatomical location and attribute combinations for a report. The processing utils can either render the scene graphs in a tabular format or represent the information in a simpler enriched RDF format, which we used to generate the graph visualizations in Figure \ref{fig1.cxr_graph}. 

% \hideseg{
% \vspace{-0.3cm}
% \begin{footnotesize}
% \begin{verbatim}
% {
%  <study_id_i> : [
%                   [[node_id_1, node_type_1], [node_id_2, node_type_2], relation_name_A],
%                   [[node_id_1, node_type_1], [node_id_3, node_type_3], relation_name_B],
%                     ...
%                 ],
%  <study_id_i+1>:[
%                   [[node_id_1, node_type_1], [node_id_2, node_type_2], relation_name_A],
%                   [[node_id_1, node_type_1], [node_id_3, node_type_3], relation_name_B],
%                     ...
%                 ],
% }   
% \end{verbatim}
% \end{footnotesize}
% }

\vspace{-5pt}
\subsection*{Gold Standard Dataset Tables}
\vspace{-2pt}
%\noindent \textbf{Gold standard dataset tables}: 
We curated a manual gold standard evaluation dataset to measure the quality of the automatically derived annotations in the Chest ImaGenome dataset and for model benchmarking. Here we describe the three gold standard ground truth files in the ``gold\_dataset'' directory. They are in tabular format for ease of comparison purposes.

$\bullet$  \textit{\textbf{gold\_attributes\_relations\_500pts\_500studies1st.txt}} is the ground truth file which contains 21,594 object-to-attribute relations manually annotated for 3,042 sentences from the \textit{first} CXR study for 500 unique patients. The notebook `object-attribute-relation\_evaluation.ipynb' explains in detail how we it to calculate the performance of object-to-attribute relation extraction.

$\bullet$  \textit{\textbf{gold\_comparison\_relations\_500pts\_500studies2nd.txt}} is the ground truth file which contains 5,156 object-object (per attribute) comparison relations for 638 sentences from the \textit{second} CXR study for the same 500 unique patients. The notebook `object-object-comparison-relation\_evaluation.ipynb' uses it to calculate the performance for object-to-object-comparison relation extraction.

$\bullet$  The four \textit{\textbf{bbox\_coordinate\_annotations*.csv}} files contain the manually annotated bounding box coordinates for the objects on the corresponding 1,000 unique CXR images. The notebook `object-bbox-coordinates\_evaluation.ipynb' calculates the bounding box object detection performance using these ground truth files.

$\bullet$ Lastly, \textit{\textbf{final\_merging\_report\_and\_bbox\_ground\_truth.ipynb}} combines the manual text and anatomical bbox annotations as \textit{\textbf{gold\_object\_attribute\_with\_coordinates.txt}} and \textit{\textbf{gold\_object\_comparison\_with\_coordinates.txt}}.

Additional supporting files for measuring the performance of the silver dataset against the gold standard are described in Supplementary (Section \ref{gold_supp}):

% % Can also put this in supplementary material
% \noindent \textbf{gold\_all\_sentences\_500pts\_1000studies.txt} contains all the sentences tokenized from the original MIMIC-CXR reports that were used to create the gold standard dataset. We include this file because sentences with no relevant object, attribute or relation descriptions did not make it into the gold standard dataset. We renamed `subject\_id' from MIMIC-CXR dataset to `patient\_id' in Chest ImaGenome dataset to avoid confusion with field names for relationships in the scene graphs. Otherwise, the ids are unchanged. Sentences in the tokenized file are assigned to `history', `prelimread', or `finalreport' in the `section' column. The `sent\_loc' column contains the order of the sentences as in the original report. Minimal tokenization has been done to the sentences.

% \noindent \textbf{gold\_bbox\_scaling\_factors\_original\_to\_224x224.csv} contains the scaling `ratio' and the paddings (`left', `right', `top', and `bottom') added to square the image after resizing the original MIMIC-CXR dicoms to 224x224 sizes. These ratios were used to rescale the annotated coordinates for 224x224 images back to the original CXR image sizes.

% \noindent \textbf{auto\_bbox\_pipeline\_coordinates\_1000\_images.txt'} contains the bounding box coordinates that were automatically extracted by the Bbox pipeline for the different objects for images in the gold standard dataset. It is in tabular format like with the ground truth for easier evaluation purposes.

\section{Evaluation}
\label{sec:evaluation}
\begin{table*}[!t]
\begin{center}
%\small
\caption {Benchmarks and applications for the study of the application-level resilience}
\vspace{-5pt}
\label{tab:benchmark}
\tiny
\begin{tabular}{|p{1.7cm}|p{7.5cm}|p{4cm}|p{2.5cm}|}
\hline
\textbf{Name} 	& \textbf{Benchmark description} 		& \textbf{Execution phase for evaluation}  			& \textbf{Target data objects}             \\ \hline \hline
CG (NPB)             & Conjugate Gradient, irregular memory access (input class S)   & The routine conj\_grad in the main computation loop  & The arrays $r$ and $colidx$     \\\hline
MG (NPB)    	       & Multi-Grid on a sequence of meshes (input class S)             & The routine mg3P in the main computation loop & The arrays $u$ and $r$ 	\\ \hline
FT (NPB)             & Discrete 3D fast Fourier Transform (input class S)            & The routine fftXYZ in the main computation loop  & The arrays $plane$ and $exp1$    \\ \hline
BT (NPB)             & Block Tri-diagonal solver (input class S)         		& The routine x\_solve in the main computation loop & The arrays $grid\_points$ and $u$	\\ \hline
SP (NPB)             & Scalar Penta-diagonal solver (input class S)         		& The routine x\_solve in the main computation loop & The arrays $rhoi$ and $grid\_points$  \\ \hline
LU (NPB)            & Lower-Upper Gauss-Seidel solver (input class S)        	& The routine ssor 	& The arrays $u$ and $rsd$ \\ \hline \hline
LULESH~\cite{IPDPS13:LULESH} & Unstructured Lagrangian explicit shock hydrodynamics (input 5x5x5) & 
The routine CalcMonotonicQRegionForElems 
& The arrays $m\_elemBC$ and $m\_delv\_zeta$ \\ \hline
AMG2013~\cite{anm02:amg} & An algebraic multigrid solver for linear systems arising from problems on unstructured grids (we use  GMRES(10) with AMG preconditioner). We use a compact version from LLNL with input matrix $aniso$. & The routine hypre\_GMRESSolve & The arrays $ipiv$ and $A$   \\ \hline
%$hierarchy.levels[0].R.V$ \\ \hline
\end{tabular}
\end{center}
\vspace{-5pt}
\end{table*}

%We evaluate the effectiveness of ARAT, and 
%We use ARAT to study the application-level resilience.
%The goal is to demonstrate 
%that aDVF can be a very useful metric to quantify the resilience of data objects
%at the application level. 
We study 12 data objects from six benchmarks of the NAS parallel benchmark (NPB) suite (we use SNU\_NPB-1.0.3) and 4 data objects from two scientific applications. 
%which is a c version of NPB 3.3, but ARAT can work for Fortran.
Those data objects are chosen to be representative: they have various data access patterns and participate in various execution phases.  
%For the benchmarks, we use CLASS S as the input problems and use the default compiler options of NPB.
For those benchmarks and applications, we use their default compiler options, and use gcc 4.7.3 and LLVM 3.4.2 for trace generation.
To count the algorithm-level fault masking, we use the default convergence thresholds (or the fault tolerance levels) for those benchmarks.
Table~\ref{tab:benchmark} gives 
%for->on by anzheng
detailed information on the benchmarks and applications.
The maximum fault propagation path for aDVF analysis is set to 10 by default.
%the value shadowing threshold is set as 0.01 (except for BT, we use $1 \times 10^{-6}$).
%These value shadowing thresholds are chosen such that any error corruption
%that results in the operand's value variance less than 1\% (for the threshold 0.01) or 0.0001\% (for the threshold $1 \times 10^{-6}$) during the 
%trace analysis does not impact the outcome correctness of six benchmarks.
%LU: check the newton-iteration residuals against the tolerance levels
%SP: check the newton-iteration residuals against the tolerance levels
%BT: check the newton-iteration residuals against the tolerance levels

\subsection{Resilience Modeling Results}
%We use ARAT to calculate aDVF values of 16 data objects. 
Figure~\ref{fig:aDVF_3tiers_profiling}
shows the aDVF results and breaks them down into the three levels 
(i.e., the operation-level, fault propagation level, and algorithm-level).
Figure~\ref{fig:aDVF_3classes_profiling} shows the 
%for->of by anzheng
results for the analyses at the levels of the operation and fault propagation,
and further breaks down the results into 
the three classes (i.e., the value overwriting, logical and comparison operations,
and value shadowing). %based on the reasons of the fault masking.
We have multiple interesting findings from the results.

\begin{figure*}
	\centering
        \includegraphics[width=0.8\textwidth]{three_tiers_gray.pdf}
% * <azguolu@gmail.com> 2017-03-23T03:20:28.808Z:
%
% ^.
        \vspace{-5pt}
        \caption{The breakdown of aDVF results based on the three level analysis. The $x$ axis is the data object name.}
        \vspace{-8pt}
        \label{fig:aDVF_3tiers_profiling}
\end{figure*}


\begin{figure*}
	\centering
	\includegraphics[width=0.8\textwidth]{three_types_gray.pdf}
	\vspace{-5pt}
	\caption{The breakdown of aDVF results based on the three classes of fault masking. The $x$ axis is the data object name. \textit{zeta} and \textit{elemBC} in LULESH are \textit{m\_delv\_zeta} and \textit{m\_elemBC} respectively.} % Anzheng
	\vspace{-5pt}
	\label{fig:aDVF_3classes_profiling}
    %\vspace{-5pt}
\end{figure*}

(1) Fault masking is common across benchmarks and applications.
Several data objects (e.g., $r$ in CG, and $exp1$ and $plane$ in FT)
have aDVF values close to 1 in Figure~\ref{fig:aDVF_3tiers_profiling}, 
which indicates that most of operations working on these data objects
have fault masking.
However, a couple of data objects have much less intensive fault masking.
For example, the aDVF value of $colidx$ in CG is 0.28 (Figure~\ref{fig:aDVF_3tiers_profiling}). 
Further study reveals that $colidx$ is an array to store column indexes of sparse matrices, and there is few operation-level or fault propagation-level fault masking  (Figure~\ref{fig:aDVF_3classes_profiling}).
The corruption of it can easily cause segmentation fault caught by the
algorithm-level analysis. 
$grid\_points$ in SP and BT also have a relatively small aDVF value (0.14 and 0.38 for SP and BT respectively in Figure~\ref{fig:aDVF_3tiers_profiling}).
Further study reveals that $grid\_points$ defines input problems for SP and BT. 
A small corruption of $grid\_points$ 
%change->changes by anzheng
can easily cause major changes in computation
caught by the fault propagation analysis. 

The data object $u$ in BT also has a relatively small aDVF value (0.82 in Figure~\ref{fig:aDVF_3tiers_profiling}).
Further study reveals that $u$ is read-only in our target code region
for matrix factorization and Jacobian, neither of which is friendly
for fault masking.
Furthermore, the major fault masking for $u$ comes from value shadowing,
and value shadowing only happens in a couple of the least significant bits 
of the operands that reference $u$, which further reduces the value of aDVF.
%also reduces fault masking.

(2) The data type is strongly correlated with fault masking.
Figure~\ref{fig:aDVF_3tiers_profiling} reveals that the integer data objects ($colidx$ in CG, $grid\_points$ in BT and SP, $m\_elemBC$ in LULESH) appear to be 
more sensitive to faults than the floating point data objects 
($u$ and $r$ in MG, $exp1$ and $plane$ in FT, $u$ and $rsd$ in LU, $m\_delv\_zeta$ in LULESH, and $rhoi$ in SP).
In HPC applications, the integer data objects are commonly employed to
define input problems and bound computation boundaries (e.g., $colidx$ in CG and $grid\_points$ in BT), 
or track computation status (e.g., $m\_elemBC$ in LULESH). Their corruption 
%these integer data objects
is very detrimental to the application correctness. 

(3) Operation-level fault masking is very common.
For many data objects, the operation-level fault masking contributes 
more than 70\% of the aDVF values. For $r$ in CG, $exp1$ in FT, and $rhoi$ in SP,
the contribution of the operation-level fault masking is close to 99\% (Figure~\ref{fig:aDVF_3tiers_profiling}).

Furthermore, the value shadowing is a very common operation level fault masking,
especially for floating point data objects (e.g., $u$ and $r$ in BT, $m\_delv\_zeta$ in LULESH, and $rhoi$ in SP in Figure~\ref{fig:aDVF_3classes_profiling}).
This finding has a very important indication for studying the application resilience.
In particular, the values of a data object can be different across different input problems. If the values of the data object are different, 
then the number of fault masking events due to the value shadowing will be different. 
Hence, we deduce that the application resilience
can be correlated with the input problems,
because of the correlation between the value shadowing and input problems. 
We must consider the input problems when studying the application resilience.
This conclusion is consistent with a very recent work~\cite{sc16:guo}.

(4) The contribution of the algorithm-level fault masking to the application resilience can be nontrivial.
For example, the algorithm-level fault masking contributes 19\% of the aDVF value for $u$ in MG and 27\% for $plane$ in FT (Figure~\ref{fig:aDVF_3tiers_profiling}).
The large contribution of algorithm-level fault masking in MG is consistent with
the results of existing work~\cite{mg_ics12}. 
For FT (particularly 3D FFT), the large contribution of algorithm-level fault masking in $plane$ (Figure~\ref{fig:aDVF_3tiers_profiling})
comes from frequent transpose and 1D FFT computations that average out 
or overwrite the data corruption.
CG, as an iterative solver, is known to have the algorithm-level fault masking
because of the iterative nature~\cite{2-shantharam2011characterizing}.
Interestingly, the algorithm-level fault masking in CG contributes most to the resilience of $colidx$ which is a vulnerable integer data object (Figure~\ref{fig:aDVF_3tiers_profiling}).

%Our study reveals the algorithm-level fault masking of CG from
%two perspectives. First, $a$ in CG, which is an array for intermediate results,
%has few algorithm-level fault masking (0.008\%);
%Second, $x$ in CG, which is a result vector, has 5.4\% of the aDVF value coming from the algorithm-level fault masking.
%This result indicates that the effects of the algorithm-level fault masking
%are not uniform across data objects. 

(5) Fault masking at the fault propagation level is small.
For all data objects, the contribution of the fault masking at the level of fault propagation is less than 5\% (Figure~\ref{fig:aDVF_3tiers_profiling}).
For 6 data objects ($r$ and $colidx$ in CG, $grid\_points$ and $u$ in BT, and 
$grid\_points$ and $rhoi$ in SP),  there is no fault masking at the level of fault propagation.
In combination with the finding 4, we conclude that once the fault
is propagated, it is difficult to mask it because of the contamination of
more data objects after fault propagation, and only the algorithm semantics can tolerate  propagated faults well. 
%This finding is consistent with our sensitivity analysis. 

(6) Fault masking by logical and comparison operations is small,
%For all data objects, the fault masking contributions due to logical and comparison operations are very small, 
comparing with the contributions of value shadowing and overwriting (Figure~\ref{fig:aDVF_3classes_profiling}). 
Among all data objects, 
the logical and comparison operations in $grid\_points$ in BT contribute the most (25\% contribution in Figure~\ref{fig:aDVF_fine_profiling}), 
because of intensive ICmp operations (integer comparison). %logical OR and SHL (left shifting).


(7) The resilience varies across data objects. %within the same application.
This fact is especially pronounced in two data objects $colidx$ and $r$ in CG (Figure~\ref{fig:aDVF_3tiers_profiling}).
 $colidx$ has aDVF much smaller than $r$, which means $colidx$ is much less resilient than $r$ (see finding 1 for a detailed analysis on $colidx$). 
Furthermore, $colidx$ and $r$ have different algorithm-level
fault masking (see finding 4 for a detailed analysis).

\begin{comment}
\textbf{Finding 7: The resilience of the same data objects varies across different applications.}
This fact is especially pronounced in BT and SP.
BT and SP address the same numerical problem but with different algorithms.
BT and SP have the same data objects, $qs$ and $rhoi$, but
$qs$ manifests different resilience in BT and SP.
This result is interesting, because it indicates that by using
different algorithms, we have opportunities to
improve the resilience of data objects.
\end{comment}

To further investigate the reasons for fault masking, 
we break down the aDVF results at the granularity of LLVM instructions,
based on the analyses at the levels of operation and fault propagation.
The results are shown in Figure~\ref{fig:aDVF_fine_profiling}.
%Because of the space limitation, 
%we only show one data object per benchmark, but each selected data object has the most diverse fault masking events within the corresponding benchmark.
%Based on Figure~\ref{fig:aDVF_fine_profiling}, we have another interesting finding.

(8) Arithmetic operations make a lot of contributions to fault masking.
%For $r$ in CG, $r$ in MG, $exp1$ in FT, $u$ in BT, $qs$ in SP, and $u$ in LU,
%the arithmetic operations, FMul (100\%), Add (16\%), FMul (85\%), 
%FMul (94\%), FMul (28\%), and FAdd (50\%)
For $r$ in CG, $u$ in BT, $plane$ and $exp1$ in FT, $m\_elemBC$ in LULESH, 
arithmetic operations (addition, multiplication, and division) contribute to almost 100\% of the fault masking (Figure~\ref{fig:aDVF_fine_profiling}).  
%(at the operation level and the fault propagation level).
%For $qs$ in SP and $u$ in LU, the store operation also makes
%important contributions as the arithmetic operations because of value overwriting.

\begin{figure*}
	\centering
	\includegraphics[width=0.77\textheight, height=0.23\textheight]{pie_chart.pdf}
	\vspace{-10pt}
	\caption{Breakdown of the aDVF results based on the analyses at the levels of operation and fault propagation}
    \vspace{-10pt}
	\label{fig:aDVF_fine_profiling}
\end{figure*}


\subsection{Sensitivity Study}
\label{sec:eval_sen}
%\textbf{change the fault propagation threshold and study the sensitivity of analysis to the threshold}
ARAT uses 10 as the default fault propagation analysis threshold. 
The fault propagation analysis will not go beyond 10 operations. Instead,
we will use deterministic fault injection after 10 operations. 
In this section, we study the impact of this threshold on the modeling accuracy. We use a range of threshold values and examine how the aDVF value varies and whether
the identification of fault masking varies. 
Figure~\ref{fig:sensitivity_error_propagation} shows the results for 
%add , after BT by anzheng
multiple data objects in CG, BT, and SP.
We perform the sensitivity study for all 16 data objects.
%in six benchmarks and two applications.
Due to the page space limitation, we only show the results for three data objects,
but we summarize the sensitivity study results for all data objects in this section.
%but other data objects in all benchmarks have the same trend.

Our results reveal that the identification of fault masking by tracking fault propagation is not significantly 
affected by the fault propagation analysis threshold. Even if we use a rather large threshold (50), 
the variation of aDVF values is 4.48\% on average among all data objects,
and the variation at each of the three levels of analysis (the operation level, fault propagation level,  and algorithm level) is less than 5.2\% on average. 
In fact, using a threshold value of 5 is sufficiently accurate in most of the cases (14 out of 16 data objects).
This result is consistent with our finding 5 (i.e., fault masking at the fault propagation level is small). %in most benchmarks).
However, we do find a data object ($m\_elementBC$ in LULESH) %and $exp1$ in FT) 
showing relatively high-sensitive (up to 15\% variation) to the threshold. For this uncommon data object, using 50 as the fault propagation path is sufficient. 

%In other words, even though using a larger threshold value can identify more error masking by tracking error 
%propagation, the implicit error masking induced by the error propagation is very limited.

\begin{figure}
		\begin{center}
		\includegraphics[width=0.48\textwidth,height=0.11\textheight]{sensi_study_gray.pdf}
		\vspace{-15pt}
		\caption{Sensitivity study for fault propagation threshold}
		\label{fig:sensitivity_error_propagation}
		\end{center}
\vspace{-15pt}
\end{figure}


\begin{comment}
\subsection{Comparison with the Traditional Random Fault Injection}
%\textbf{compare with the traditional fault injection to verify accuracy}
To show the effectiveness of our resilience modeling, we compare traditional random fault injection
and our analytical modeling. Figure~\ref{fig:comparison_fi} and Table~\ref{tab:comparison} show the results.
The figure shows the success rate of all random fault injection. The ``success'' means the application
outcome is verified successfully by the benchmarks and the execution does not have any segfault. The success rate is used as a metric
to evaluate the application resilience.

We use a data-oriented approach to perform random fault injection.
In particular, given a data object, for each fault injection test we trigger a bit flip
in an operand of a random instruction, and this operand must be a reference to the
target data object. We develop a tool based on PIN~\cite{pintool} to implement the above fault injection functionality.
For each data object, we conduct five sets of random fault injection tests, 
and each set has 200 tests (in total 1000 tests per data object). 
We show the results for CG and FT in this section, but we find that
the conclusions we draw from CG and FT are also valid for the other four benchmarks.


%\begin{table*}
%\label{tab:success_rate}
%\begin{centering}
%\renewcommand\arraystretch{1.1}
%\begin{tabular}{|c|c|c|c|c|c|c|}
%\hline 
%Success Rate (Difference) & Test set 1 & Test set 2 & Test set 3 & Test set 4 & Test set 5 & Average\tabularnewline
%\hline 
%\hline 
%CG-a & 66.1\% (11.7\%) & 68.5\% (15.7\%) & 56.7\% (4.21\%) & 61.3\% (3.57\%) & 43.3\% (26.8\%) & 59.2\%\tabularnewline
%\hline 
%CG-x & 99.2\% (2.2\%) & 98.6\% (1.5\%) & 96.5\% (0.63\%) & 97.8\% (0.64\%) & 93.6\% (3.7\%) & 97.1\%\tabularnewline
%\hline 
%CG-colidx & 36.8\% (12.7\%) & 49.6\% (17.8\%) & 40.2\% (4.6\%) & 52.6\% (24.9\%) & 31.4\% (25.4\%) & 42.1\%\tabularnewline
%\hline 
%FT-exp1 & 52.7\% (1.4\%) & 22.6\% (56.5\%) & 78.5\% (51.0\%) & 60.7\% (16.7\%) & 45.4\% (12.7\%) & 51.9\%\tabularnewline
%\hline 
%FT-plane & 82.1\% (2.5\%) & 79.3\% (5.6\%) & 99.5\% (18.2\%) & 93.2\% (10.7\%) & 66.8\% (20.6\%) & 84.2\%\tabularnewline
%\hline 
%\end{tabular}
%\par\end{centering}
%\caption{XXXXX}
%\end{table*}


\begin{table*}
\begin{centering}
\caption{\small The results for random fault injection. The numbers in parentheses for each set of tests (200 tests per set) are the success rate difference from the average success rate of 1000 fault injection tests.}
\label{tab:comparison}
\renewcommand\arraystretch{1.1}
\begin{tabular}{|c|p{2.2cm}|p{2.2cm}|p{2.2cm}|p{2.2cm}|p{2.2cm}|p{1.8cm}|}
\hline 
       %& Test set 1 & Test set 2 & Test set 3 & Test set 4 & Test set 5 & Average\tabularnewline
       & \hspace{13pt} Test set 1 \hspace{1pt}/  & \hspace{13pt} Test set 2 \hspace{1pt}/ & \hspace{13pt} Test set 3 \hspace{1pt}/ & \hspace{13pt} Test set 4 \hspace{1pt}/ & \hspace{13pt} Test set 5 \hspace{1pt}/ & Ave. of all test / \\
       & success rate (diff.) & success rate (diff.) & success rate (diff.) & success rate (diff.) & success rate (diff.) & \hspace{5pt} success rate \\
\hline 
\hline 
CG-a & 66.1\% (6.9\%) & 68.5\% (9.3\%) & 56.7\% (-2.5\%) & 61.3\% (2.1\%) & 43.3\% (-15.9\%) & 59.2\%\tabularnewline
\hline 
CG-x & 99.2\% (2.1\%) & 98.6\% (1.5\%) & 96.5\% (-0.6\%) & 97.8\% (0.7\%) & 93.6\% (-3.5\%) & 97.1\%\tabularnewline
\hline 
CG-colidx & 36.8\% (-5.3\%) & 49.6\% (7.5\%) & 40.2\% (-2.0\%) & 52.6\% (10.5\%) & 31.4\% (-10.7\%) & 42.1\%\tabularnewline
\hline 
FT-exp1 & 52.7\% (0.8\%) & 22.6\% (-29.3\%) & 78.5\% (26.6\%) & 60.7\% (8.8\%) & 45.4\% (-6.5\%) & 51.9\%\tabularnewline
\hline 
FT-plane & 82.1\% (-2.1\%) & 79.3\% (-4.9\%) & 99.5\% (15.3\%) & 93.2\% (9.0\%) & 66.8\% (-17.4\%) & 84.2\%\tabularnewline
\hline 
\end{tabular}
\par\end{centering}
\vspace{-0.4cm}
\end{table*}

\begin{figure}
	\begin{center}
		\includegraphics[width=0.48\textwidth,keepaspectratio]{verifi-study.png}
		\caption{The traditional random fault injection vs. ARAT}
		\label{fig:comparison_fi}
	\end{center}
\vspace{-0.7cm}
\end{figure}


We first notice from Table~\ref{tab:comparison} that 
%across 5 sets of random fault injection tests, there are big variances (up to 55.9\% in $exp1$ of FT) in terms of the success rate. 
the results of 5 test sets can be quite different from each other and from 1000 random fault inject tests (up to 29.3\%).
1000 fault injection tests provide better statistical significance than 200 fault injection tests.
We expect 1000 fault injection tests potentially provide higher accuracy to quantify the application resilience.
The above result difference is clearly an indication to the randomness of fault injection, and there
is no guarantee on the random fault injection accuracy.

%In Figure~\ref{fig:comparison_fi}, 
We compare the success rate of 1000 fault inject tests with the aDVF value (Fig.~\ref{fig:comparison_fi}). 
We find that the order of the success rate of the three data objects in CG (i.e., $colidx < a < x$) and the two data objects in FT 
(i.e., $exp1 < plane$) is the same as the order of the aDVF values of these data objects. 
%In fact, 1000 fault injection tests
%account for \textcolor{blue}{\textbf{xxx\%}} of total memory references to the data object,
%and provide better resilience quantification than 200 fault injection tests.
The same order (or the same resilience trend)
%between our approach and the random fault injection based on a large number of tests 
is a demonstration of the effectiveness of our approach.
Note that the values of the aDVF and success rate %for a data object
cannot be exactly the same (even if we have sufficiently large numbers of random fault injection), 
because aDVF and random fault injection quantify
the resilience based on different metrics.
Also, the random fault injection can miss some fault masking events that can be captured by our approach.

\end{comment}
%\section*{Usage Notes}
\vspace{-5pt}
\section*{Clinical Applications}
% \textbf{Clinical applications}: 
\vspace{-5pt}
There are numerous clinical topics that may be explored for a dataset that links anatomic structures with individual abnormalities and simultaneously provides comparison relation annotations for sequential images. Monitoring the progression of pathologies that are visualized through chest imaging is the most unexplored clinical application of this dataset. In the in-patient setting, diagnosis and monitoring of pneumonia are typically performed through comparisons of sequential CXR images from admission\cite{kalil2016management}. The same management principle may apply to the evaluation of the progression of other diseases, such as pneumothorax, pulmonary edema, acute respiratory distress syndrome, or congestive heart failure \cite{henry2003bts, cardinale2014effectiveness, rubenfeld2012acute}. In the outpatient setting, surveillance of incidental pulmonary nodules, malignancies, tuberculosis, or interstitial lung disease is done through chest imaging in several-month intervals \cite{gould2013evaluation, koo2019chest, nahid2016official, hansell2015ct}. Furthermore, the methodological concepts of this dataset could be extended to other modes of imaging, such as computed tomography (CT), and magnetic resonance (MR) imaging, etc, further expanding the potential clinical utility of this project.


\textbf{Consistent dataset splits for performance reporting}: For reproducibility, we include splits for train, valid and test sets in the ``silver\_dataset/splits'' directory. The random data split was done at the patient level. We also included a file (images\_to\_avoid.csv) with image IDs (`dicom\_id') and `study\_id's for patients in the gold standard dataset, which should all be excluded from training and validation. 
%We expect all final benchmark reporting to be done on both the test set in the silver dataset and the manually annotated gold standard dataset.

As described, Chest ImaGenome has been constructed with multiple possible downstream tasks in mind. Here, we showcase two example tasks that can have the most immediate clinical applications, (i) outputting both the location and the type of CXR attribute for an image (Example Task 2) and (ii) comparing whether a location has worsened or improved across sequential exams (Example Task 1). Clinically, the two chosen types of tasks are the two most important ones for radiologists to report when interpreting CXRs. 


\textbf{Example Task 1: Change between sequential CXR exams.} CXRs are commonly repeatedly requested in the clinical workflow to assess for a myriad of attributes. Given a patient with sequential CXRs, the goal of this task is to automatically evaluate disease change over time based on two sequential CXR exams. We restricted the problem to a subset of the Chest ImaGenome dataset, i.e., to attributes related to congestive heart failure (CHF), as fluid management is one of the most routine clinical tasks for which CXRs can be ordered to guide the next steps (e.g. whether to give more intravenous fluid or give diuretics, etc). However, we note that users of this dataset can also explore comparison changes for other CXR attributes (e.g. pneumonia). Each CXR image is also associated with a bounding box that marks a localized area, e.g., ``left lung'' for specific anatomical finding (i.e., attribute), such as ``pulmonary edema/hazy opacity'', etc. In addition, the pair of CXR images is mapped to the comparison label that indicates whether the condition of the anatomical finding has improved or worsened. As a baseline example, we focus on change relations in the 'left lung' and 'right lung' objects that are related to the `pulmonary edema/hazy opacity' and `fluid overload/heart failure' attributes. The number of examples labeled in the training, validation and test data are $10,515$, $1,493$ and $2,987$, respectively. 
%is summarized in Table \ref{tab:change_dataset}. 
%Note that we also include a separate small gold standard sample that is validated by subject matter experts.
We design a siamese architecture (Figure \ref{fig:siamese} in Supplementary \ref{clinical_applications}) that first extracts the localized bounding box from each image and encodes the extracted image patches with a pre-trained ResNet101 autoencoder, denoted that is trained on several medical imaging datasets, e.g., NIH, CheXpert, and MIMIC datasets, etc. \cite{irvin2019chexpert,johnson2019mimic,wang2017chestx}. The autoencoder image representations are concatenated and passed through a dense layer with 128 neurons and ReLU activations, and a final classification layer. 
%The model architecture is implemented with TorchXRayVision \cite{Cohen2020xrv} and PyTorch Lighting \cite{falcon2019pytorch}. 
We train for $300$ epochs with cross-entropy, stochastic gradient descent, $1e-3$ learning rate, $0.1$ gradient clipping and $32$ batch size. We freeze the autoencoder weights and finetune the two last dense layers. On this challenging task of predicting change in localized anatomical findings between two sequential exams, we achieve an accuracy of $75.3\%$. %and $71.43\%$ on the test and gold test sets, respectively. 

 \textbf{Example Task 2: Localization of CXR attributes.} Knowing the anatomical location of non-specific findings/attributes on CXR images can help with narrowing down possible disease diagnoses and guide the next steps in requesting more specific imaging exams or treatment. To this end, we train a Faster R-CNN model \cite{ren2015faster} %using detectron2 \cite{wu2019detectron2}
to learn 18 anatomical locations within the dataset. We extract the 1024 dimension convolution feature vector of each anatomical region. We re-implement the state-of-the-art CheXGCN model \cite{chen2020label} %that uses a Graph Convolutional Network (GCN) model 
to learn the dependencies between attributes within the Chest X-ray. 
%In particular, the convolutions are replaced with the Faster R-CNN model. 
Similar to the work done by CheXGCN we model the correlation of the CXR attributes using a conditional probability (see Figure \ref{fig:gcn} in Supplementary \ref{clinical_applications}). We compare the results of the model with two baseline models, a Faster R-CNN model followed by a linear model without the GCN, and a Densenet model \cite{huang2017densely} without the Faster R-CNN to evaluate the effectiveness of the localized models. We focus on 9 common CXR attributes, which include lung opacity, pleural effusion, atelectasis, enlarged cardiac silhouette, pulmonary edema/hazy opacity, pneumothorax, consolidation, fluid overload/heart failure, pneumonia. The results of the experiments are shown in Table \ref{tab:attr_results} and the labels are ordered according to the attribute list above. 


\begin{table}[t!]
\centering
\caption{Anatomically localized CXR attribute detection (AUC scores). L1: Lung Opacity, L2: Pleural Effusion, L3: Atelectasis, L4: Enlarged Cardiac Silhouette, L5: Pulmonary Edema/Hazy Opacity, L6: Pneumothorax, L7: Consolidation, L8: Fluid Overload/Heart Failure, L9: Pneumonia.}
\resizebox{\textwidth}{!}{
\begin{tabular}{p{3cm}*{10}{p{0.8cm}}}
\toprule
Method & L1 &  L2 & L3 & L4 & L5 & L6 & L7 & L8 & L9 & \textbf{AVG}  \\
\midrule
Faster R-CNN & 0.84 & 0.89 & 0.77 & 0.85 & 0.87 & 0.77 & 0.75 & 0.81 & 0.71 & 0.80\\
GlobalView & \textbf{0.91} & \textbf{0.94} & 0.86 & 0.92 & 0.92 & \textbf{0.93} & 0.86 & 0.87 & 0.84 & 0.89\\
CheXGCN & 0.86 & 0.90 & \textbf{0.91} & \textbf{0.94} & \textbf{0.95} & 0.75 & \textbf{0.89} & \textbf{0.98} & \textbf{0.88} & \textbf{0.90}\\
\bottomrule 
\end{tabular}
}
\label{tab:attr_results}
\vspace{-15pt}
\end{table}

\hideseg{
\begin{table}[t!]
\centering
\caption{Change relation experiment: dataset statistics.}
%\resizebox{0.3\linewidth}{!}{%
\begin{tabular}{ccc} %p{0.45\linewidth}p{0.20\linewidth}p{0.20\linewidth}
\toprule
Data & \#Worsened & \#Improved \\ \midrule
\textbf{Train} & 5,802 & 4,713 \\
\textbf{Validation} & 808 & 685 \\
\textbf{Test} & 1,638 & 1,349 \\
%\textbf{Test (Gold)} & 51 & 47 \\
\bottomrule 
\end{tabular}
%}
\label{tab:change_dataset}
\vspace{-0.2cm}
\end{table}
}

\textbf{Dataset Limitations}: The Chest ImaGenome dataset came from only one U.S. hospital source. It is automatically generated and is limited by the performance of the NLP and the Bbox extraction pipelines. Furthermore, we cannot assume that all the clinically relevant CXR attributes are always described on every exam by the reporting radiologists. In fact, we have observed many implied object-attribute relation descriptions that are documented only in the form of comparisons (e.g. no change from previous) in short CXR reports. As such, even with perfect NLP extraction of object and attribute relations from individual reports, there would be missing information in the report knowledge graph constructed for some images. These technical areas are worth improving on in future research with more powerful NLP, image processing techniques and other graph-based techniques. Addressing missing relations will certainly improve this dataset too. Regardless, version 1.0.0 of the Chest ImaGenome dataset serves as a pioneering vision for a richer radiology imaging dataset.
% %\begin{figure*}[htb]
    \begin{multicols}{2}
        \begin{lstlisting}[language=C++, basicstyle=\scriptsize\ttfamily, frame=leftline]
#include <torch/torch.h>


struct FloatNetImpl : torch::nn::Module{
  FloatNetImpl() : linear(10, 2){ 
    register_module("linear", linear);
  }

  torch::Tensor forward(torch::Tensor x){
    x = linear(x);
    return torch::sigmoid(x);
  }

  
  
  
  
  
  torch::nn::Linear linear;
};
TORCH_MODULE(FloatNet);
        \end{lstlisting}
        \columnbreak
        \begin{lstlisting}[language=C++, basicstyle=\scriptsize\ttfamily, frame=leftline, tabsize=2]
#include <positnn/positnn>

template <typename P>
struct PositNet : Layer<P>{
  PositNet() : linear(10, 2){
    this->register_module(linear);
  }

  StdTensor<P> forward(StdTensor<P> x){
    x = linear.forward(x);
    return sigmoid.forward(x);
  }

  StdTensor<P> backward(StdTensor<P> x){
    x = sigmoid.backward(x);
    return linear.backward(x);
  }

  Linear<P> linear;
  Sigmoid<P> sigmoid;
};
        \end{lstlisting}
    \end{multicols}
    %\caption{Comparison of PyTorch (left) and the proposed framework PositNN (right).}
%\end{figure*}
% As a summary of the sections above, the Gaussian approximated smoothing solutions, whilst being more robust than SMC methods (and extensions thereof), lack the unbiasedness and convergence properties of these. This motivates the following contributions of this article:
\begin{enumerate}
    \item In Section~\ref{sec:auxiliary_samplers}, we show that, in the case of generalised Feynman--Kac models~\eqref{eq:gen-ssm} with Gaussian dynamics, the auxiliary proposals of \citet{titsias2018} recover the posterior distribution of an auxiliary LGSSM. We leverage this to reduce their time and space complexity to $\bigO(T)$ rather than $\bigO(T^3)$. We then use the generalised statistical regression framework of \citet{Tronarp2018iterative} to derive a new class of LGSSM auxiliary samplers for non-Gaussian latent dynamical systems.
    \item In Section~\ref{sec:PIT-sampling}, we introduce two parallel-in-time samplers for LGSSM pathwise smoothing distributions, one based on a prefix-sum implementation akin to \citet{Sarkka2021temporal}, and the other based on a divide-and-conquer recursion. We use these to sample from the LGSSM proposals, resulting in an overall $\bigO(\log T)$ MCMC algorithm on parallel hardware.
    \item In Section~\ref{sec:pgibbs_samplers}, we extend the construction of Section~\ref{sec:auxiliary_samplers} to the context of particle MCMC. This will allow us to tackle models for which Gaussian approximations are not practical or under-performing. We show that \citet{finke2021csmc} is recovered as a special case of our method.
    \item In Section~\ref{sec:experiments}, we illustrate the proposed methods on a series of examples from the SMC and Gaussian approximated inference literature. Special attention is paid to understanding their statistical as well as time and memory trade-offs.
\end{enumerate}
% \input{sections/conflicts}
% \section*{Figures \& Tables}

\begin{figure}[ht]
\centering
\includegraphics[width=\linewidth]{figures/rod_shere_disk_example_with_structures}
\caption{\textbf{A})~Principal-moments-of-inertia plot~\cite{sauer2003molecular} for molecules in the QMugs dataset. $NPR_x$ = $x$-th normalized principal moment, $I_x$ = $x$-th smallest principal moment of inertia.
\textbf{B})~Venn diagram showing overlap between QMugs and other well-known datasets with DFT-level computed properties: QM9~\cite{ramakrishnan2014quantum}, PubChemQC~\cite{nakata2017pubchemqc}, and ANI-1~\cite{smith2017anid}. Overlap was computed based on the uniqueness of the InChI representations of the contained molecules. Numbers do not add up to those reported in Table~\ref{tbl:desc} because of InChI strings that occur multiple times.}
\label{fig:rod_disk_sphere_venn}
\end{figure}


\begin{figure}[ht]
\centering
\includegraphics[width=\linewidth]{figures/rdkit_props}
\caption{Distribution of properties for the molecules contained in the QMugs dataset.}
\label{fig:molecule_props}
\end{figure}


\begin{figure}[ht]
\centering
\includegraphics[width=\linewidth]{figures/qmugs_pipeline}
\caption{Overview of the data generation process. Molecules were extracted from the ChEMBL database, standardized, and filtered, and starting conformers were generated using the RDKit software package. Metadynamics (MTD) simulations were performed using the GFN2-xTB semi-empirical method to generate three diverse conformations before final geometry optimization. Molecules that did not pass a series of geometric sanity checks were removed. DFT-level properties ($\omega$B97X-D/def2-SVP) were computed using the Psi4 software.}
\label{fig:pipeline}
\end{figure}


\begin{figure}[ht]
\centering
\includegraphics[width=\linewidth]{figures/geometry_validation}
\caption{
    (\textbf{A})~Distributions of mean pairwise RMSD of atom positions between conformations of each molecule in the QMugs dataset at different stages along the pipeline. While the $k$-means sampling process selects conformations that are, on average, more geometrically diverse than the average pair of structures generated by MTD simulations, geometry optimization reduces the geometrical diversity between the optimized conformers.
    (\textbf{B})~Change in atom positions during geometry optimization vs. mean pairwise RMSD of conformations before optimization. Molecules with initially more diverse conformations displayed a greater change in atom positions than those with initially less diverse conformations. 
    (\textbf{C})~Distribution of RMSD of structures prior to and after optimization with the semi-empirical GFN2-xTB method, and of structures optimized with the same approach vs. with $\omega$B97X-D/def2-SVP. The structures of three molecules with varying differences between the two methods are shown as illustrative examples (black and gray correspond to GFN2-xTB and $\omega$B97X-D/def2-SVP-optimized structures, respectively). For illustrative purposes, the example molecules are aligned on their substructures.
}
\label{fig:geometry_validation}
\end{figure}


\begin{figure}[ht]
\centering
\includegraphics[width=\linewidth]{figures/delta_molecular}
\caption{Comparison of molecular properties computed at the two levels of theory considered herein (GFN2-xTB, $\omega$B97X-D/def2-SVP) for the molecules contained in QMugs. The molecular formation energy $E_{\mathrm{form}}$ in (\textbf{A}) was calculated by subtracting the atomic $U_{\mathrm{Atom}}$ contributions from the  total molecular energies $U_{RT}$. Only the rotational constants $A$ are shown in (\textbf{C}) as their $B$ and $C$ counterparts showed highly similar values. $22$ conformations of small molecules show very large rotational constants and are not shown. RMSE and PCC for rotational constant $A$ are $845.834$ cm$^{-1}$ and $0.091$ respectively, if those structures are included. Abbreviations: RMSE, root mean squared error; PCC, Pearson's correlation coefficient.}
\label{fig:delta_molecular_props}
\end{figure}


\begin{figure}[ht]
\centering
\includegraphics[width=\linewidth]{figures/partial_charges}
\caption{Atom-type-specific partial charge correlations (GFN2-xTB, $\omega$B97X-D/def2-SVP) for the QMugs dataset (see ESI Table~1 for additional metrics)}
\label{fig:partial}
\end{figure}


\begin{figure}[ht]
\centering
\includegraphics[width=\linewidth]{figures/bond_orders}
\caption{Comparison of Wiberg bond orders between GFN2-xTB and $\omega$B97X-D/def2-SVP for the 15 most frequently occurring bond types in the QMugs dataset. The latter level of theory uses L\"owdin-orthogonalization. See ESI Table~2 for additional metrics. For bond types which occurred $>1$M times in the dataset, a randomly chosen sample of $1$M bonds is plotted.}
\label{fig:bond}
\end{figure}




\begin{table}[ht]
\caption{Descriptive statistics of the dataset reported herein in the context of other  DFT-level molecular datasets and the information provided by each. The number of molecules for PubChemQC corresponds to that available on the website of the project.~\cite{pubchemqc_website} Heavy atom averages are weighted by the number of conformations.}
\label{tbl:desc}
\centering
\resizebox{\textwidth}{!}{%
\begin{tabular}{@{}l>{\raggedleft\arraybackslash}p{2cm}>{\raggedleft\arraybackslash}p{2cm}>{\raggedleft\arraybackslash}p{2cm}p{5cm}P{2cm}P{2cm}p{3cm}@{}}
\toprule
\textbf{Dataset} &
  \textbf{Unique compounds} &
  \textbf{Total conformations} &
  \textbf{Heavy atoms max (mean)} &
  \textbf{Method} &
  \textbf{$\Delta$-learning possible} &
  \textbf{Wave functions} \\ \midrule
QM9       & $133,885$  & $133,885$   & $9$~~~($8.8$) & B3LYP/6-31G(2df,p)               & \xmark &  \xmark \\
ANI-1     & $57,462$   & $22,057,374$ & $8$~~~($7.1$) & $\omega$B97X/6-31G(d)              & \xmark  & \xmark \\
PubChemQC & $3,982,436$ & $3,982,436$  & $51$ ($14.1$) & B3LYP/6-31G(d)              & \xmark  & \xmark \\
QMugs     & $665,911$  & $1,992,984$  & $100$ ($30.6$) & GFN2-xTB + $\omega$B97X-D/def2-SVP & \cmark  & \cmark \\ \bottomrule
\end{tabular}
}
\end{table}


\begin{table}[ht]
\footnotesize
\caption{Calculated properties as stored in the SDFs of the QMugs data collection. Abbreviations: a.u., atomic units; vib., vibrational; rot., rotational; transl., translational. Properties that enable $\Delta$ machine learning are labelled with $\blacklozenge$.}
\label{tbl:properties}
\centering
\begin{tabular}{llllll}
\toprule
\textbf{Property}                                        & \textbf{Symbol}                               & \textbf{Unit}           & \textbf{Key}  & $\Delta$-ML        \\  \midrule
ChEMBL identifier                                               &         -                      &           -              & \texttt{CHEMBL\_ID} &                \\
Conformer identifier                                                 &     -                                          &                    -     & \texttt{CONF\_ID}      &              \\
Total energy                                             & $U_{RT}$            & \si{\hartree}                      & \texttt{GFN2:TOTAL\_ENERGY} & $\blacklozenge$                  \\
Internal atomic energy                             & $E_\mathrm{Atom}$                             & \si{\hartree}                      & \texttt{GFN2:ATOMIC\_ENERGY} &       \\
Formation energy                             & $E_\mathrm{Form}$           & \si{\hartree}                      & \texttt{GFN2:FORMATION\_ENERGY}   & $\blacklozenge$          \\
Total enthalpy                                           & $H_{RT}$                                      & \si{\hartree}                      & \texttt{GFN2:TOTAL\_ENTHALPY}    &        \\
Total free energy                                        & $G_{RT}$                                      & \si{\hartree}                      & \texttt{GFN2:TOTAL\_FREE\_ENERGY} &        \\
Dipole ($x$, $y$, $z$, total)                  & $\mu$                                         & D                       & \texttt{GFN2:DIPOLE}      & $\blacklozenge$                    \\
Quadrupole ($xx$, $xy$, $yy$, $xz$, $yz$, $zz$) & $Q_{ij}$                                      & D \si{\angstrom}                 & \texttt{GFN2:QUADRUPOLE}      &           \\
Rotational constants ($A$, $B$, $C$)           & $A$, $B$, $C$                              & \si{\centi\meter}$^{-1}$                     & \texttt{GFN2:ROT\_RONSTANTS}    & $\blacklozenge$              \\
Enthalpy (vib., rot., transl., total)          & $\Delta H$                                    & cal mol$^{-1}$          & \texttt{GFN2:ENTHALPY}     &               \\
Heat capacity (vib., rot., transl., total)    & $C_{V}$                                       & cal K$^{-1}$ mol$^{-1}$ & \texttt{GFN2:HEAT\_CAPACITY}  &             \\
Entropy (vib., rot., transl., and total)       & $\Delta S$                                    & cal K$^{-1}$ mol$^{-1}$ & \texttt{GFN2:ENTROPY}     &                \\
HOMO energy                                              & $E_\mathrm{HOMO}$                             & \si{\hartree}                      & \texttt{GFN2:HOMO\_ENERGY}       & $\blacklozenge$             \\
LUMO energy                                              & $E_\mathrm{LUMO}$                             & \si{\hartree}                      & \texttt{GFN2:LUMO\_ENERGY}     & $\blacklozenge$           \\
HOMO-LUMO gap                                            & $E_\mathrm{Gap}$                              & \si{\hartree}                      & \texttt{GFN2:HOMO\_LUMO\_GAP}  & $\blacklozenge$               \\
Fermi level                                              & $E_{\mathrm{Fermi}}$                                   & \si{\hartree}                      & \texttt{GFN2:FERMI\_LEVEL}   &             \\
Mulliken partial charges                                 & $\delta_{M}$                                  & \si{\elementarycharge}                       & \texttt{GFN2:MULLIKEN\_CHARGES} & $\blacklozenge$              \\
Covalent coordination number                           & $N_{\textrm{coord}}$                  & -                   &\texttt{GFN2:COVALENT\_COORDINATION\_NUMBER}  & \\
Molecular dispersion coefficient                           & $C_6$                                                & a.u.                            &\texttt{GFN2:DISPERSION\_COEFFICIENT\_MOLECULAR} & \\
Atomic dispersion coefficients                         & $C_6$                                                & a.u.                                &\texttt{GFN2:DISPERSION\_COEFFICIENT\_ATOMIC} & \\
Molecular polarizability                           & $\alpha(0)$                                                 & a.u.                                        &\texttt{GFN2:POLARIZABILITY\_MOLECULAR} &  \\
Atomic polarizabilities                            & $\alpha(0)$                                                 & a.u.                                        &\texttt{GFN2:POLARIZABILITY\_ATOMIC} &  \\
Wiberg bond orders                                    & $M_{AB}$                                 &             -            &\texttt{GFN2:WIBERG\_BOND\_ORDER}     & $\blacklozenge$            \\
Total Wiberg bond orders                               & $\sum_{A (A \neq B)} M_{AB}$                      &       -                  &\texttt{GFN2:TOTAL\_WIBERG\_BOND\_ORDER}  & $\blacklozenge$              \\
Total energy                                             & $U_{RT}$                                      & \si{\hartree}                      & \texttt{DFT:TOTAL\_ENERGY}     & $\blacklozenge$             \\
Total internal atomic energy                             & $E_\mathrm{Atom}$                             & \si{\hartree}                      & \texttt{DFT:ATOMIC\_ENERGY}    &         \\
Formation energy                             & $E_\mathrm{Form}$                             & \si{\hartree}                      & \texttt{DFT:FORMATION\_ENERGY}  & $\blacklozenge$           \\
Electrostatic potential                                  & $V_{ESP}$                                     & \si{\volt}                       & \texttt{DFT:ESP\_AT\_NUCLEI}   &         \\
L\"owdin partial charges                                 & $\delta_{L}$                                  & \si{\elementarycharge}                       & \texttt{DFT:LOWDIN\_CHARGES}     &       \\
Mulliken partial charges                                 & $\delta_{M}$                                  & \si{\elementarycharge}                       & \texttt{DFT:MULLIKEN\_CHARGES}   & $\blacklozenge$           \\
Rotational constants  ($A$, $B$, $C$)          & $A$, $B$, $C$                              & \si{\centi\meter}$^{-1}$                     & \texttt{DFT:ROT\_CONSTANTS}   & $\blacklozenge$              \\
Dipole ($x$, $y$, $z$, total)                  & $\mu$                                         & D                       & \texttt{DFT:DIPOLE}                     \\
Exchange correlation energy                              & $\hat{V}_{eN}$                                & \si{\hartree}                      & \texttt{DFT:XC\_ENERGY}   &              \\
Nuclear repulsion energy                                 & $\hat{V}_{eN}$                                & \si{\hartree}                      & \texttt{DFT:NUCLEAR\_REPULSION\_ENERGY} & \\
One-electron energy                               & $\hat{T}_{e}$                                 & \si{\hartree}                      & \texttt{DFT:ONE\_ELECTRON\_ENERGY}   &    \\
Two-electron energy                                      & $\hat{V}_{ee}$                                & \si{\hartree}                      & \texttt{DFT:TWO\_ELECTRON\_ENERGY} &     \\
HOMO energy                                              & $E_\mathrm{HOMO}$                             & \si{\hartree}                      & \texttt{DFT:HOMO\_ENERGY}    & $\blacklozenge$               \\
LUMO energy                                              & $E_\mathrm{LUMO}$                             & \si{\hartree}                      & \texttt{DFT:LUMO\_ENERGY}    & $\blacklozenge$               \\
HOMO-LUMO gap                                            & $E_\mathrm{Gap}$                              & \si{\hartree}                      & \texttt{DFT:HOMO\_LUMO\_GAP}   & $\blacklozenge$             \\
Mayer bond orders                                        & $M_{AB}$                                 &                -         & \texttt{DFT:MAYER\_BOND\_ORDER}   &           \\
Wiberg-L\"owdin bond orders                              & $W_{AB}$                                 &            -             & \texttt{DFT:WIBERG\_LOWDIN\_BOND\_ORDER}  & $\blacklozenge$      \\
Total Mayer bond orders                              & $\sum_{A (A \neq B)} M_{AB}$                      &          -               & \texttt{DFT:TOTAL\_MAYER\_BOND\_ORDER}      &      \\
Total Wiberg-L\"owdin bond orders                          & $\sum_{A (A \neq B)} W_{AB}$        &      -                   & \texttt{DFT:TOTAL\_WIBERG\_LOWDIN\_BOND\_ORDER}  & $\blacklozenge$     \\ \bottomrule
\end{tabular}
\end{table}

\begin{table}[ht]
\caption{Calculated molecular properties stored in the wave function files provided in the QMugs data collection. Mayer and Wiberg-L\"owdin bond orders included here represent a superset of the bond orders in the SDFs which additionally comprise bond orders for non-covalent bonds.}
\label{tbl:wfn}
\centering
\begin{tabular}{llll}
\toprule
\textbf{Property}                                        & \textbf{Symbol}                               & \textbf{Key}                            \\ \midrule
Alpha density matrix                                     & $\mathrm{D}_{\alpha}$                         & \texttt{matrix, Ca}                              \\
Beta density matrix                                      & $\mathrm{D}_{\beta}$                          & \texttt{matrix, Cb}                              \\
Alpha orbitals                                           & $\mathrm{C}_{\alpha}$                         & \texttt{matrix, Da}                              \\
Beta orbitals                                            & $\mathrm{C}_{\beta}$                          & \texttt{matrix, Db}                              \\
Atomic-orbital-to-symmetry-orbital transformer           & $\mathrm{C}_{\mathrm{AOTOSO}}$                & \texttt{matrix, aotoso}                          \\
Mayer bond orders                                        & $M_{AB}$                                      & \texttt{MAYER\_INDICES}                          \\
Wiberg-L\"owdin bond orders                              & $W_{AB}$                                      & \texttt{WIBERG\_LOWDIN\_INDICES}                 \\ \bottomrule
\end{tabular}
\end{table}

%\newpage
\documentclass{article}

\begin{document}

\section{Acknowledgements}

The author would like to thank B. J. Hiley, M. Hajtanian, and D. Nellist for their insightful conversations and support.

\end{document}
\bibliographystyle{plainnat}
\bibliography{bibliography.bib}
\newpage
%\section*{Checklist}


\begin{enumerate}

% \answerTODO{}, \answerYes{}, \answerNo{}, \answerNA{}

\item For all authors...
\begin{enumerate}
  \item Do the main claims made in the abstract and introduction accurately reflect the paper's contributions and scope?
    \answerYes{}
  \item Did you describe the limitations of your work?
    \answerYes{}\\
    \textcolor{blue}{See Section~\ref{sec:conclusion}.}
  \item Did you discuss any potential negative societal impacts of your work?
    \answerYes{}\\
    \textcolor{blue}{See Section~\ref{sec:conclusion}.}
  \item Have you read the ethics review guidelines and ensured that your paper conforms to them?
    \answerYes{}
\end{enumerate}

\item If you are including theoretical results...
\begin{enumerate}
  \item Did you state the full set of assumptions of all theoretical results?
    \answerNA{}
	\item Did you include complete proofs of all theoretical results?
    \answerNA{}
\end{enumerate}

\item If you ran experiments...
\begin{enumerate}
  \item Did you include the code, data, and instructions needed to reproduce the main experimental results (either in the supplemental material or as a URL)?
    \answerYes{}\\
    \textcolor{blue}{We provide our code in the supplementary materials.}
  \item Did you specify all the training details (e.g., data splits, hyperparameters, how they were chosen)?
    \answerYes{}\\
    \textcolor{blue}{We state the details in Section~\ref{sec:exp} and Appendix~\ref{sec:details}.}
	\item Did you report error bars (e.g., with respect to the random seed after running experiments multiple times)?
    \answerYes{}\\
    \textcolor{blue}{We report the error bars for the experiments with larger variances (e.g., Table~\ref{tab:bg-main}).}
	\item Did you include the total amount of compute and the type of resources used (e.g., type of GPUs, internal cluster, or cloud provider)?
    \answerYes{}\\
    \textcolor{blue}{See common setup in Section~\ref{sec:exp}.}
\end{enumerate}

\item If you are using existing assets (e.g., code, data, models) or curating/releasing new assets...
\begin{enumerate}
  \item If your work uses existing assets, did you cite the creators?
    \answerYes{}\\
    \textcolor{blue}{We cite the datasets and libraries we use.}
  \item Did you mention the license of the assets?
    \answerYes{}\\
    \textcolor{blue}{We only use the public datasets and open source libraries.}
  \item Did you include any new assets either in the supplemental material or as a URL?
    \answerNA{}
  \item Did you discuss whether and how consent was obtained from people whose data you're using/curating?
    \answerNA{}
  \item Did you discuss whether the data you are using/curating contains personally identifiable information or offensive content?
    \answerNA{}
\end{enumerate}

\item If you used crowdsourcing or conducted research with human subjects...
\begin{enumerate}
  \item Did you include the full text of instructions given to participants and screenshots, if applicable?
    \answerNA{}
  \item Did you describe any potential participant risks, with links to Institutional Review Board (IRB) approvals, if applicable?
    \answerNA{}
  \item Did you include the estimated hourly wage paid to participants and the total amount spent on participant compensation?
    \answerNA{}
\end{enumerate}

\end{enumerate}


\newpage
\section*{Supplementary Material}

\section{Additional Chest ImaGenome Terminology Descriptions}
% \begin{table}[ht]
%   \centering
%     \caption{Parallels between the Chest ImaGenome and Visual Genome datasets.}
%   \label{tab:cg_vg_parallels}
%   \footnotesize{
%     \begin{tabular}{|p{4em}|p{21em}|p{16em}|}
%     \toprule
%     \textbf{Element} & \textbf{Chest ImaGenome} & \textbf{Visual Genome} \\
%     \midrule
%     \midrule
%     Scene & One frontal CXR image in the current dataset. & One (non-medical) everyday life image. \\
%     \midrule
%     Questions & For now, there is only one question per CXR, which is taken from the patient history (i.e., reason for exam) section from each CXR report. & One or more questions that the crowd source annotators decided to ask about the image where the information from each question and the image should allow another annotator to answer it. \\
%     \midrule
%     Answers & N/A currently. However, report sentences are biased towards answering the question asked in the reason for exam sentence;hence, the knowledge graph we extract from each report should contain the answer(s). & This was collected as answer(s) to the corresponding question(s) asked of the image. \\
%     \midrule
%     Sentences (Region descriptions) & Sentences from the finding and impression sections of a CXR report describing the exam as collected from radiologists in their routine radiology workflow. & True natural language descriptive sentences about the image collected from crowd-sourced everyday annotators. \\
%     \midrule
%     % Region & Bounding box coordinates that encompass all the anatomical structures described in a report sentence (can be easily derived from the scene graph json). & Bounding box coordinates that include all the objects mentioned in a sentence that describes the image, e.g., The boy (object 1) is beside the bus (object 2). \\
%     % \midrule
%     Objects (nodes) & Anatomical structures or locations that have bounding box coordinates on the associated CXR image, and is indexed to the UMLS ontology \cite{bodenreider2004unified}. & The people and physical objects with bounding box coordinates on the image and indexed to WordNet ontology \cite{miller1995wordnet}. \\
%     \midrule
%     Attributes (nodes) & Descriptions that are true for different anatomical structures visualized on the CXR image (e.g., There is a right upper lung [object] opacity [attribute]), indexed to the UMLS ontology \cite{bodenreider2004unified}. No Bbox coordinates. & Various descriptive properties of the objects in the image (e.g., The shirt [object] is blue [attribute]), indexed to WordNet ontology \cite{miller1995wordnet}. No Bbox coordinates. \\
%     \midrule
%     Relations: object and attribute & The relationship(s) between an anatomical object and its attribute(s) from the same CXR image (e.g., There is a [relation] right upper lung [object] opacity [attribute]). & The relationship(s)  between an object and its attribute(s) from the same image ( e.g., The shirt [object] is [relation] blue [attribute]). \\
%     \midrule
%     Relations: object and object & The comparison relationship (index to UMLS \cite{bodenreider2004unified}) between the same anatomical object from two sequential CXR images for the same patient (e.g., There is a new [relation] right lower lobe [current and previous anatomical objects] atelectasis [attribute]). & The relationship (indexed to WordNet \cite{miller1995wordnet}) between objects in the same image (e.g., The boy [object 1] is beside [relation] the bus [object 2]). \\
%     \midrule
%     Relations: parent and child & To make the graph for each image logically consistent and correct as learnable and consumable radiology knowledge, affirmed parent-child relations between nodes are embedded in the scene graphs -- i.e., if a child attribute is related to an object, then its parent would be too (e.g., if right lung has consolidation [child], then it also has lung opacity [parent]). & N/A due to different graph construction strategy and goals. The annotators were asked to describe any (but not all) relations they observe in an image. \\
%     \midrule
%     Scene graph & Constructed from the objects, the attributes and the relationships between them for the image. & Same but the nodes and edges overall would be more varied than Chest ImaGenome for now. \\
%     \midrule
%     Sequence* & A super-graph for a set of chronologically ordered series of exams for the same patient. & N/A, but would be a graph for a video in the non-medical context. \\
%     \bottomrule
%     \end{tabular}%
%     }
% \end{table}

% \newpage

%\subsection*{Data description - supplementary}
% describes nodes in graph
\begin{table}[h]
  \centering
     \caption{Semantic category of nodes and edges in CXR knowledge graphs. All nodes are mapped to UMLS CUIs in the scene graph jsons. All object nodes have corresponding bounding box coordinates on frontal CXRs except ones with *. All nodes and edges are evaluated with the gold standard dataset except the edges marked with **, which are modifiers of the context edges.}
%   \caption{Add caption}
    \label{tab:define_nodes_edges} 
    \resizebox{\textwidth}{!}{
    \begin{tabular}{|l|c|p{25em}|}
    \toprule
    \textbf{Category ID} & \textbf{type} & \multicolumn{1}{l|}{\textbf{names}} \\
    \midrule
    \midrule
    technicalassessment & attribute node & low lung volumes, rotated, artifact, breast/nipple shadows, skin fold \\
    \midrule
    texture & attribute node & opacity, alveolar, interstitial, calcified, lucency \\
    \midrule
    anatomicalfinding & attribute node & lung opacity, airspace opacity, consolidation, infiltration, atelectasis, linear/patchy atelectasis, lobar/segmental collapse, pulmonary edema/hazy opacity, vascular congestion, vascular redistribution, increased reticular markings/ild pattern, pleural effusion, costophrenic angle blunting, pleural/parenchymal scarring, bronchiectasis, enlarged cardiac silhouette, mediastinal displacement, mediastinal widening, enlarged hilum, tortuous aorta, vascular calcification, pneumomediastinum, pneumothorax, hydropneumothorax, lung lesion, mass/nodule (not otherwise specified), multiple masses/nodules, calcified nodule, superior mediastinal mass/enlargement, rib fracture, clavicle fracture, spinal fracture, hyperaeration, cyst/bullae, elevated hemidiaphragm, diaphragmatic eventration (benign), sub-diaphragmatic air, subcutaneous air, hernia, scoliosis, spinal degenerative changes, shoulder osteoarthritis, bone lesion \\
    \midrule
    disease & attribute node & pneumonia, fluid overload/heart failure, copd/emphysema, granulomatous disease, interstitial lung disease, goiter, lung cancer, aspiration, alveolar hemorrhage, pericardial effusion \\
    \midrule
    nlp   & attribute node & abnormal, normal (with respect to an anatomy/object node) \\
    \midrule
    tubesandlines & attribute node & chest tube, mediastinal drain, pigtail catheter, endotracheal tube, tracheostomy tube, picc, ij line, chest port, subclavian line, swan-ganz catheter, intra-aortic balloon pump, enteric tube \\
    \midrule
    device & attribute node & sternotomy wires, cabg grafts, aortic graft/repair, prosthetic valve, cardiac pacer and wires \\
    \midrule
    \midrule
    majorstructure & object node & right lung, left lung, mediastinum \\
    \midrule
    subanatomy & object node & right apical zone, right upper lung zone, right mid lung zone, right lower lung zone, right hilar structures, right costophrenic angle, left apical zone, left upper lung zone, left mid lung zone, left lower lung zone, left hilar structures, left costophrenic angle, upper mediastinum, cardiac silhouette, trachea, right hemidiaphragm, left hemidiaphragm, right clavicle, left clavicle, spine, right atrium, cavoatrial junction, svc, carina, aortic arch, abdomen, right chest wall*, left chest wall*, right shoulder*, left shoulder*, neck*, right arm*, left arm*, right breast*, left breast* \\
    \midrule
    \midrule
    context & edge  & yes (has/present in), no (not have/not present in)\\
    \midrule
    comparison & edge  & improved, worsened, no change \\
    \midrule
    severity** & edge  & hedge, mild, moderate, severe \\
    \midrule
    temporal** & edge  & acute, chronic \\
    \bottomrule
    \end{tabular}
    }%
  %\label{tab:addlabel}%
\end{table}

\newpage
\section{Scene Graph JSON}\label{jsonsg}
Below are examples from a scene graph JSON used for explanation for the silver dataset.

\subsection{Scene Graph JSON - first level}\label{json1}
\begin{footnotesize}
\begin{verbatim}
{
 `chest_imageimage_id': `10cd06e9-5443fef9-9afbe903-e2ce1eb5-dcff1097',
 `viewpoint': `AP', `patient_id': 10063856, `study_id': 56759094,
 `gender': `F', `age_decile': `50-60',
 `reason_for_exam': `___F with hypotension.  Evaluate for pneumonia.',
 `StudyOrder': 2, `StudyDateTime': `2178-10-05 15:05:32 UTC',
 `objects': [ <...list of {} for each object...> ],
 `attributes':[ <...list of {} for each object...> ],
 `relationships':[ <...list of {} of comparison relationships between objects 
 from sequential exams for the same patient...> ] 
}
\end{verbatim}
\end{footnotesize}


\subsection{Scene Graph JSON - objects field}\label{json2}
\begin{footnotesize}
\begin{verbatim}
{
  `object_id': `10cd06e9-5443fef9-9afbe903-e2ce1eb5-dcff1097_right upper lung zone',
  `x1': 48, `y1': 39, `x2': 111, `y2': 93,
  `width': 63, `height': 54,
  `bbox_name': `right upper lung zone',
  `synsets': [`C0934570'],
  `name': `Right upper lung zone',
  `original_x1': 395, `original_y1': 532,
  `original_x2': 1255, `original_y2': 1268,
  `original_width': 860, `original_height': 736
}
\end{verbatim}
\end{footnotesize}


\subsection{Scene Graph JSON - attributes field}\label{json3}
\begin{footnotesize}
\begin{verbatim}
{
  `right lung': True, `bbox_name': `right lung',
  `synsets': [`C0225706'], `name': `Right lung',
  `attributes': [[`anatomicalfinding|no|lung opacity',
  `anatomicalfinding|no|pneumothorax',  `nlp|yes|normal'],
  [`anatomicalfinding|no|pneumothorax']],
  `attributes_ids': [[`CL556823', `C1963215;;C0032326', `C1550457'],
  [`C1963215;;C0032326']],
  `phrases': [`Right lung is clear without pneumothorax.', 
  `No pneumothorax identified.'],
  `phrase_IDs': [`56759094|10', `56759094|14'],
  `sections': [`finalreport', `finalreport'],
  `comparison_cues': [[], []],
  `temporal_cues': [[], []],
  `severity_cues': [[], []],
  `texture_cues': [[], []],
  `object_id': `10cd06e9-5443fef9-9afbe903-e2ce1eb5-dcff1097_right lung'
}
\end{verbatim}
\end{footnotesize}


\subsection{Scene Graph JSON - relationships field}\label{json4}
\begin{footnotesize}
\begin{verbatim}
{
  `relationship_id': `56759094|7_54814005_C0929215_10cd06e9_4bb710ab',
  `predicate': ``['No status change']'',
  `synsets': [`C0442739'],
  `relationship_names': [`comparison|yes|no change'],
  `relationship_contexts': [1.0],
  `phrase': `Compared with the prior radiograph, there is a persistent veil 
  -like opacity\n over the left hemithorax, with a crescent of air surrounding 
  the aortic arch,\n in keeping with continued left upper lobe collapse.',
  `attributes': [`anatomicalfinding|yes|atelectasis',
  `anatomicalfinding|yes|lobar/segmental collapse',
  `anatomicalfinding|yes|lung opacity', `nlp|yes|abnormal'],
  `bbox_name': `left upper lung zone',
  `subject_id': `10cd06e9-5443fef9-9afbe903-e2ce1eb5-dcff1097_left upper lung zone',
  `object_id': `4bb710ab-ab7d4781-568bcd6e-5079d3e6-7fdb61b6_left upper lung zone'
}
\end{verbatim}
\end{footnotesize}


\subsection{Scene Graph - Enriched RDF JSON format}
\begin{footnotesize}\label{json5}
\begin{verbatim}
{
 <study_id_i> : [
                  [[node_id_1, node_type_1], [node_id_2, node_type_2], relation_name_A],
                  [[node_id_1, node_type_1], [node_id_3, node_type_3], relation_name_B],
                    ...
                ],
 <study_id_i+1>:[
                  [[node_id_1, node_type_1], [node_id_2, node_type_2], relation_name_A],
                  [[node_id_1, node_type_1], [node_id_3, node_type_3], relation_name_B],
                    ...
                ],
}   
\end{verbatim}
\end{footnotesize}


\section{Gold Dataset Annotation - Details}
\label{gold_annot_supp}

The `gold dataset' is a randomly sampled subset (500 unique patients) from the automatically generated Chest ImaGenome dataset, i.e., the `silver dataset', that has been manually validated or corrected. The primary purpose of the `gold dataset' is to evaluate the quality of labels in the `silver dataset'. For this purpose, we evaluated the Chest ImaGenome dataset along with the 3 components below (A-B). The annotations for each component were collected in stages to reduce the cognitive workload for the annotators. The annotators are all M.D.s with 2 to 10 or more years of clinical experience. One of the annotators is a radiologist trained in the United States, who has over 6 years of radiology experience and specializes in reading imaging exams from the Emergency Department (ED) setting. The annotation tasks were delegated to the annotators according to their clinical experience, which we think are all more than sufficient for the tasks. Component A and B were annotated by the radiologist and an M.D. and component C was annotated by 4 M.D.'s.


\vspace{+10pt}
\textbf{\textit{A) Evaluating CXR knowledge graph extraction from reports}}
\vspace{+5pt}

The report knowledge graph for the \textit{first} CXR of the 500 patients was manually reviewed and corrected as necessary for relation extraction between the anatomical locations (objects) and the CXR attributes. From piloting trials, we found that manually annotating multiple targets at a document level lead to a slow and complex task with poor recall. However, sometimes information from prior sentences is necessary to annotate both the anatomical locations and the attributes correctly. Therefore, we set up the annotation task at the sentence level. Sentences from each report are ordered as per the original report, and the phrase boundary for each attribute was marked out for the annotators, where the phrases used for detecting each attribute were curated by consensus between two radiologists from previous work \cite{wu2020ai}. 

Since we are targeting a large set of possible anatomical locations (object) to attribute combinations, the annotation was streamlined into the four steps below to minimize the cognitive overload for each step. Steps 1 and 2 are dual annotated by two clinicians (one fully trained radiologist and one M.D.), with disagreements resolved by consensus review. Steps 3 and 4 are single annotated. A random subset of annotations for 500 sentences from step 4 are sampled and dual annotated to estimate inter-annotator agreement. Cleaned results from step 4 constitute the final gold-standard CXR knowledge graph ground truth for the 500 reports. 

This annotation component was set up in Excel and was broken down into the following four steps below. In our Excel setup, all sentences from each report are available to the annotators (they can just scroll up or down). The sentences are ordered by `row\_id' sequentially within each report. Unique patients and reports have the same IDs as shown in the figures below.

\textbf{Step 1} - For each sentence and NLP extracted attribute combination, decide whether the NLP context (affirmed or negated) for the attribute was correct. If not, correct it. Figure \ref{fig:object-attribute-step1} shows how this task was set up in Excel. The annotators' task is to make sure the extracted attribute (yellow label\_name column) has the correct context given the sentence from the report. This `context' is used as the relation between the location and the attribute in the final annotated result.

\begin{figure}[!ht]
\centering
\includegraphics[scale=0.35]{figures/annot/object_attribute_annot_step1.pdf}
\caption{Step 1: Annotate all attributes per sentence.}
\label{fig:object-attribute-step1}
\end{figure}

\textbf{Step 2} - For each sentence, decide whether the NLP extracted anatomical location(s) were described or implied by the reporting radiologist. If not, remove the location (in yellow column `bboxes\_corrected). If missing, add the location. If unsure (e.g., if lung is mentioned but not sure if it is the right or left lung), the annotator can look in previous sentences from the same report. The task was set up as shown in Figure \ref{fig:object-attribute-step2}.

\begin{figure}[!ht]
\centering
\includegraphics[scale=0.29]{figures/annot/object_attribute_annot_step2.pdf}
\caption{Step 2: Annotate all locations per sentence.}
\label{fig:object-attribute-step2}
\end{figure}

\textbf{Step 3} - For recall, manually annotate missed objects and/or attributes for sentences with no NLP extractions (a much smaller subset). For this, we used Excel's filtering function to look at all sentences with no automated extractions (empty cells) and de novo added the manual annotations.

\textbf{Step 4} - Firstly, all rows from steps 1-3 where the annotations differed between the two annotators were reviewed and resolved together by consensus. Then we automatically derived all object-attribute relation combinations for each sentence from steps 1-3's results. The obviously wrong object-to-attribute relations were filtered out for each sentence using the CXR ontology. For the remaining object-to-attribute relations for each sentence, the task was to indicate whether the logical statement of \textit{``object X contains (or does not contain) attribute Y''} is true or false, as shown in Figure \ref{fig:object-attribute-step4}. Probable relation is still defined to be true for this annotation. Annotating for uncertain relations is beyond the scope of this project. However, for future dataset expansion, we have kept the NLP cues for the certainty for each object-attribute relation in the scene graph JSON. 

\begin{figure}[!ht]
\centering
\includegraphics[scale=0.33]{figures/annot/object_attribute_annot_step4.pdf}
\caption{Step 4: Annotate all logically correct statements/relations for each sentence.}
\label{fig:object-attribute-step4}
\end{figure}

Since step 4 was single annotated, to estimate the final inter-annotator agreement, we randomly sampled 500 sentences for dual annotations. This %\href{https://physionet.org/content/chest-imagenome/1.0.0/utils/annotation_utils/object_attribute_relation_annotation/object_attribute_relations_estimated_interannotator_agreement.txt}{\textbf{\textit{annotated result}}} 
annotated result is also shared on PhysioNet.

\vspace{+10pt}
\textbf{\textit{B) Evaluating comparison relation extraction}}: 
\vspace{+5pt}

The \textit{second} CXR exam report for the 500 patients was reviewed for comparison relation extraction. The annotation was also set up in Excel and conducted at the sentence level. However, the annotator is also shown the whole previous CXR report for context. Similarly, we split the annotation task up into several steps, where steps 1 and 2 are dual annotated and disagreement resolved via consensus. Steps 3 and 4 were single annotated. A %\href{https://physionet.org/content/chest-imagenome/1.0.0/utils/annotation_utils/object_object_comparison_annotation/comparisons_relations_estimated_interannotator_agreement.txt}{\textbf{\textit{subset of 500 sentences}}}
subset of 500 sentences from the final annotations was reviewed by a second annotator for assessing inter-annotator agreement.

\textbf{Step 1} - Given the previous report and the current report sentence, decide whether the extracted comparison cue(s) (improved, worsened, no change) is/are correct. If not, correct it/them. In this step, the annotators are asked to validate or correct the column `comparison' in Figure \ref{fig:object-comparison-step12}.

\textbf{Step 2} - Building from step 1 for each sentence, given a validated or corrected comparison cue, validate whether all the anatomical location(s) extracted are correct (column `bbox' in Figure \ref{fig:object-comparison-step12}). If incorrect or missing, remove or add the correct location(s) to the column.

\begin{figure}[!ht]
\centering
\includegraphics[scale=0.31]{figures/annot/object_object_comparison_annot_step1_2.pdf}
\caption{Step 1 and 2: Annotate change relations for different anatomical locations}
\label{fig:object-comparison-step12}
\end{figure}

\textbf{Step 3} - Building from step 2 for each sentence, given each correct comparison cue and anatomical location relation, decide whether the attributes assigned to the location described or implied in the sentence are correct or not. If not, correct it. Figure \ref{fig:object-comparison-step3} illustrates how step 3 was set up, where the annotators' task is to validate or correct the `label\_name' column with respect to the `bbox', `relation' and `comparison' columns for each sentence.

\begin{figure}[!ht]
\centering
\includegraphics[scale=0.3]{figures/annot/object_object_comparison_annot_step3.pdf}
\caption{Step 3: Annotate change relations for different anatomical locations with respect to attribute}
\label{fig:object-comparison-step3}
\end{figure}

\textbf{Step 4} - For recall, we used the filtering function in Excel to isolate all sentences with no comparison cue extractions from step 3. Sentences with missing comparison annotations were manually de-novo annotated.

\vspace{+10pt}
\textbf{\textit{C) Evaluating anatomy object detection for CXR images}}: 
\vspace{+5pt}

The first and second CXR images for the same 500 patients were dual validated and corrected for the bounding box objects (i.e., 1000 frontal CXR images altogether). Given the resources we had, we selected 28 anatomical objects (out of 36 available) that are clinically most important for frontal CXRs interpretations. The automatically extracted bounding box coordinates were first plotted on resized and padded 224x224 images. From piloting, we determined that this image size is sufficiently large to annotate the anatomies that we were targeting. The plotted images were displayed one at a time to annotators via a custom Jupyter Notebook that we had set up to allow bounding box coordinates and label annotations. We set up the annotation task on two panels, one for %\href{https://physionet.org/content/chest-imagenome/1.0.0/utils/annotation_utils/bbox_object_annotation/Correct_lung_bboxes_template.ipynb}{\textbf{\textit{lung related bounding boxes}}} 
lung-related bounding boxes (Figure \ref{fig:bboxes-lung-panel}) and another for %\href{https://physionet.org/content/chest-imagenome/1.0.0/utils/annotation_utils/bbox_object_annotation/Correct_mediastinum_bboxes_template.ipynb}{\textbf{\textit{mediastinum related and other bounding boxes}}} 
mediastinum related and other bounding boxes (Figure \ref{fig:bboxes-mediastinum-panel}). 

\begin{figure}[!ht]
\centering
\includegraphics[scale=0.35]{figures/annot/lung_related_bbox_panel.pdf}
\caption{Bbox annotations - lung related Bboxes panel}
\label{fig:bboxes-lung-panel}
\end{figure}

\begin{figure}[!ht]
\centering
\includegraphics[scale=0.35]{figures/annot/mediastinum_related_bbox_panel.pdf}
\caption{Bbox annotations - mediastinum related and other Bboxes panel}
\label{fig:bboxes-mediastinum-panel}
\end{figure}

Four M.D.'s were trained to perform this task after reviewing a set of 20-30 training examples with a radiologist. Since the inter-annotator agreement is high (mean IoU > 0.96 for all objects), the final cleaned %\href{https://physionet.org/content/chest-imagenome/1.0.0/gold_dataset/gold_bbox_coordinate_annotations_1000images.csv}{\textbf{\textit{gold standard bbox coordinates}}} 
gold standard bbox coordinates use the average coordinates from two annotators for each bounding box.


\newpage
\section{Dataset Usage Supporting Files}
\label{gold_supp}

% Can also put this in supplementary material
\noindent \textbf{gold\_all\_sentences\_500pts\_1000studies.txt} contains all the sentences tokenized from the original MIMIC-CXR reports that were used to create the gold standard dataset. We include this file because sentences with no relevant object, attribute or relation descriptions did not make it into the gold standard dataset. We renamed `subject\_id' from MIMIC-CXR dataset to `patient\_id' in Chest ImaGenome dataset to avoid confusion with field names for relationships in the scene graphs. Otherwise, the ids are unchanged. Sentences in the tokenized file are assigned to `history', `prelimread', or `finalreport' in the `section' column. The `sent\_loc' column contains the order of the sentences as in the original report. Minimal tokenization has been done to the sentences.

\noindent \textbf{gold\_bbox\_scaling\_factors\_original\_to\_224x224.csv} contains the scaling `ratio' and the paddings (`left', `right', `top', and `bottom') added to square the image after resizing the original MIMIC-CXR dicoms to 224x224 sizes. These ratios were used to rescale the annotated coordinates for 224x224 images back to the original CXR image sizes.

\noindent \textbf{auto\_bbox\_pipeline\_coordinates\_1000\_images.txt} contains the bounding box coordinates that were automatically extracted by the Bbox pipeline for the different objects for images in the gold standard dataset. It is in a tabular format like with the ground truth for easier evaluation purposes.

\noindent \textbf{object-bbox-coordinates\_evaluation.ipynb} notebook calculates the bounding box object detection performance using ground truth files from the 4 M.D. annotators , as well as consolidating the final \textbf{gold\_bbox\_coordinate\_annotations\_1000images.csv}.

\noindent \textbf{Preprocess\_mimic\_cxr\_v2.0.0\_reports.ipynb} processes the reports (tokenize sentences and sort them into history, prelim or final report sentences) from the original MIMIC-CXR v2.0.0 and save output as \textbf{silver\_dataset/cxr-mimic-v2.0.0-processed-sentences\_all.txt}. Only sentences with object or attribute extractions ended up in the final scene graph jsons in the Chest ImaGenome dataset.

\noindent The \textbf{semantics} directory contains the object (\textbf{objects\_detectable\_by\_bbox\_pipeline\_v1.txt} and \textbf{objects\_extracted\_from\_reports\_v1.txt}), attribute (\textbf{attribute\_relations\_v1.txt}) and comoparison (\textbf{comparison\_relations\_v1.txt}) relations labels in the Chest ImaGenome dataset. It also contains \textbf{semantics/label\_to\_UMLS\_mapping.json}, which maps all Chest ImaGenome concepts to UMLS CUIs \cite{bodenreider2004unified}.



\newpage
\section{Dataset Evaluation}

% \begin{figure}[!ht]
% \centering
% \includegraphics[scale=0.45]{figures/Figure_6_lung_mediastinum_clavicle_bboxes.pdf}
% \caption{Sample CXR case with 17 overlaying anatomical bounding boxes.}
% \label{fig:bbox-sample}
% \end{figure}

Table \ref{tab:object-detect} reports anatomical location level object-to-attribute relations extraction performance by the scene graph extraction pipeline. The report numbers are calculated by a combination of notebooks: `generate\_scenegraph\_statistics.ipynb', `object-attribute-relation\_evaluation.ipynb' and `object-bbox-coordinates\_evaluation.ipynb'.

\begin{table}[th]
\centering
\caption{CXR image object detection evaluation results. \** These anatomical locations are extracted by the Bbox pipeline but they are not manually annotated in the gold standard dataset due to resource constraints. \*** The mediastinum bounding boxes were not directly annotated due to resource constraints. Mediastinum's bounding box boundary can be derived from the ground truth for the upper mediastinum and the cardiac silhouette.
}\label{tab:object-detect}
\resizebox{\textwidth}{!}{%
\begin{tabular}{|l|c|c|c|c|c|}
\hline
\multicolumn{1}{|c|}{\textbf{\begin{tabular}[c]{@{}c@{}}Bbox name \\ (object)\end{tabular}}} & \multicolumn{1}{c|}{\textbf{\begin{tabular}[c]{@{}c@{}}Object-attribute relations  \\  frequency (500 reports)\end{tabular}}} & \multicolumn{1}{c|}{\textbf{\begin{tabular}[c]{@{}c@{}}Relationships F1\\ (500 reports)\end{tabular}}} & \multicolumn{1}{c|}{\textbf{\begin{tabular}[c]{@{}c@{}}Bbox IoU \\ (over 1000 images)\end{tabular}}} & \multicolumn{1}{c|}{\textbf{\begin{tabular}[c]{@{}c@{}}\% Bboxes corrected \\ (1000 images)\end{tabular}}} & \multicolumn{1}{c|}{\textbf{\begin{tabular}[c]{@{}c@{}}\% Relations missing \\ Bbox coordinates \\ (over whole dataset)\end{tabular}}} \\ \hline
left lung & 1453 & 0.933 & 0.976 & 9.90\% & 0.03\% \\ \hline
right lung & 1436 & 0.937 & 0.983 & 6.30\% & 0.04\% \\ \hline
cardiac silhouette & 633 & 0.966 & 0.967 & 9.70\% & 0.01\% \\ \hline
mediastinum & 601 & 0.952 & ** & ** & 0.02\% \\ \hline
left lower lung zone & 609 & 0.932 & 0.955 & 8.60\% & 2.36\% \\ \hline
right lower lung zone & 580 & 0.902 & 0.968 & 6.00\% & 2.27\% \\ \hline
right hilar structures & 572 & 0.934 & 0.976 & 4.10\% & 1.91\% \\ \hline
left hilar structures & 571 & 0.944 & 0.971 & 4.30\% & 2.28\% \\ \hline
upper mediastinum & 359 & 0.940 & 0.994 & 1.40\% & 0.12\% \\ \hline
left costophrenic angle & 298 & 0.908 & 0.929 & 9.60\% & 0.63\% \\ \hline
right costophrenic angle & 286 & 0.918 & 0.944 & 6.90\% & 0.39\% \\ \hline
left mid lung zone & 173 & 0.940 & 0.967 & 5.70\% & 2.79\% \\ \hline
right mid lung zone & 169 & 0.830 & 0.968 & 5.30\% & 2.31\% \\ \hline
aortic arch & 144 & 0.965 & 0.991 & 1.40\% & 0.62\% \\ \hline
right upper lung zone & 117 & 0.873 & 0.972 & 5.80\% & 0.04\% \\ \hline
left upper lung zone & 83 & 0.811 & 0.968 & 6.40\% & 0.22\% \\ \hline
right hemidiaphragm & 78 & 0.947 & 0.955 & 7.90\% & 0.15\% \\ \hline
right clavicle & 71 & 0.615 & 0.986 & 2.80\% & 0.50\% \\ \hline
left clavicle & 67 & 0.642 & 0.983 & 3.00\% & 0.51\% \\ \hline
left hemidiaphragm & 65 & 0.930 & 0.944 & 11.30\% & 0.14\% \\ \hline
right apical zone & 58 & 0.852 & 0.969 & 5.40\% & 1.99\% \\ \hline
trachea & 57 & 0.983 & 0.995 & 0.90\% & 0.24\% \\ \hline
left apical zone & 47 & 0.938 & 0.963 & 6.20\% & 2.40\% \\ \hline
carina & 41 & 0.975 & 0.994 & 0.80\% & 1.47\% \\ \hline
svc & 19 & 0.973 & 0.995 & 0.70\% & 0.66\% \\ \hline
right atrium & 14 & 0.963 & 0.979 & 4.00\% & 0.18\% \\ \hline
cavoatrial junction & 5 & 1.000 & 0.977 & 4.30\% & 0.25\% \\ \hline
abdomen & 80 & 0.904 & * & * & 0.26\% \\ \hline
spine & 132 & 0.824 & * & * & 0.10\% \\ \hline
\end{tabular}%
}
\vspace{-0.3cm}
\end{table}


\newpage
\section{Pictorial Overview of Model Architectures}
Due to space limitations, we present overview figures for the models designed for Example Tasks 1 and 2 here.
\label{clinical_applications}

\begin{figure}[h]
\center
\includegraphics[width=0.9\textwidth,height=4cm]{figures/siamese.pdf}
\caption{Example Task 1 Model Overview. Given a pair of CXR images, we extract features for the anatomical regions of interest with a pretrained ResNet autoencoder, concatenate representations and pass them through a dense layer and a final classification layer.}
\vspace{-0.3cm}
\label{fig:siamese}
\end{figure} 

\begin{figure}[h]
\centering
\includegraphics[width=0.9\textwidth]{figures/ML-GCN.pdf}
\caption{Example Task 2 Model Overview. Given a pair of CXR images, we extract features for the anatomical regions of interest with a pretrained Faster R-CNN and a GCN to learn the label dependencies.}
\label{fig:gcn}
\vspace{-0.1cm}
\end{figure}

\newpage
\section{Qualitative Evaluation}
In Figure \ref{tab:findings}, we visualize the output from our model for the anatomical finding predictions of costophrenic angles and enlarged cardiac silhouette.
In Figure \ref{tab:gradcam}, we present an additional example, showing that the model is able to provide accurate localization information as well as predict the correct finding, i.e., showing accurate localization.


\begin{table}[h]
\begin{subtable}[t]{0.5\textwidth}
\resizebox{\textwidth}{!}{
\centering
\begin{tabular}{ccc}
\textbf{{Image 1}} & \textbf{{CS}}  & \textbf{{RCA}} \\ 
\includegraphics[width=0.4\textwidth,height=0.4\textwidth]{figures/vis/viz1.png} & \includegraphics[width=0.4\textwidth,height=0.4\textwidth]{figures/vis/cs_1.png} & \includegraphics[width=0.4\textwidth,height=0.4\textwidth]{figures/vis/rca_1.png} \\[0.15cm]
\myalign{l}{Ground Truth} & \myalign{l}{\textbf{No findings}} & \myalign{l}{\textbf{No findings}} \\[0.15cm] 
%\myalign{l}{CheXGCN} & \myalign{l}{\color{red} \textbf{L4} } & \myalign{l}{\color{red} \textbf{L1, L2} } \\[0.15cm] 
\myalign{l}{Our model \cite{agu2021anaxnet}} & \myalign{l}{\color{green} \textbf{No findings}} & \myalign{l}{\color{green} \textbf{No findings}} \\ 
\end{tabular}
}
\end{subtable} 
\hspace{0.2cm}
\begin{subtable}[h]{0.5\textwidth}
\resizebox{\textwidth}{!}{
\centering
\begin{tabular}{ccc}
\textbf{{Image 2}} & \textbf{{RCA}}  & \textbf{{LCA}} \\
\includegraphics[width=0.4\textwidth,height=0.4\textwidth]{figures/vis/viz2.png} & \includegraphics[width=0.4\textwidth,height=0.4\textwidth]{figures/vis/rca_2.png} & \includegraphics[width=0.4\textwidth,height=0.4\textwidth]{figures/vis/lca_2.png} \\[0.15cm] 
\myalign{l}{Ground Truth} & \myalign{l}{\textbf{L2}} & \myalign{l}{\textbf{L2}} \\[0.15cm]  
%\myalign{l}{CheXGCN} & \myalign{l}{\color{red} \textbf{No findings}} & \myalign{l}{\color{red} \textbf{No findings}} \\[0.15cm]  
\myalign{l}{Our model \cite{agu2021anaxnet}} & \myalign{l}{\color{green} \textbf{L2}} & \myalign{l}{\color{green} \textbf{L2}}
\end{tabular}
}
\end{subtable}
\vspace{0.2cm}
\captionof{figure}{Examples of the prediction results. The overall chest X-ray image is shown alongside two anatomical regions, and predictions are compared against the ground-truth labels.
} 
\label{tab:findings}
\end{table}

\begin{figure}[h]
  \centering
   \resizebox{0.8\textwidth}{!}{
  \subfloat[Original Image]{
  \includegraphics[width=0.3\textwidth, height=0.3\textwidth]{figures/vis/orig_ecs_viz.png}
  \label{fig:f1}}
  %\hfill
  %\subfloat[GlobalView \scriptsize{(Grad-CAM)}]{
  %\includegraphics[width=0.3\textwidth, height=0.3\textwidth]{figures/vis/ecs_viz.png}  \label{fig:f2}}
  \hfill
  \subfloat[Our model \cite{agu2021anaxnet}]{
  \includegraphics[width=0.3\textwidth, height=0.3\textwidth]{figures/vis/ecs_box.png} \label{fig:f3}}
  }
  \caption{Example image with enlarged cardiac silhouette, showing that the trained model detects the finding in the correct bounding box.}
  \label{tab:gradcam}
\end{figure}
%\input{sections/tables_supplement}


\end{document}


% from Ismini (External) to Everyone:    2:37  PM
% https://www.tablesgenerator.com/
% from Ismini (External) to Everyone:    2:38  PM
% \href{link}{text}
% from Satyananda Kashyap (IBM) to Everyone:    2:38  PM
% \usepackage{hyperref}