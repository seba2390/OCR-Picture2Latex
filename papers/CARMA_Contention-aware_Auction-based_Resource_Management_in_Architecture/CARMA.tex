
%% bare_jrnl_compsoc.tex
%% V1.4b
%% 2015/08/26
%% by Michael Shell
%% See:
%% http://www.michaelshell.org/
%% for current contact information.
%%
%% This is a skeleton file demonstrating the use of IEEEtran.cls
%% (requires IEEEtran.cls version 1.8b or later) with an IEEE
%% Computer Society journal paper.
%%
%% Support sites:
%% http://www.michaelshell.org/tex/ieeetran/
%% http://www.ctan.org/pkg/ieeetran
%% and
%% http://www.ieee.org/

%%*************************************************************************
%% Legal Notice:
%% This code is offered as-is without any warranty either expressed or
%% implied; without even the implied warranty of MERCHANTABILITY or
%% FITNESS FOR A PARTICULAR PURPOSE! 
%% User assumes all risk.
%% In no event shall the IEEE or any contributor to this code be liable for
%% any damages or losses, including, but not limited to, incidental,
%% consequential, or any other damages, resulting from the use or misuse
%% of any information contained here.
%%
%% All comments are the opinions of their respective authors and are not
%% necessarily endorsed by the IEEE.
%%
%% This work is distributed under the LaTeX Project Public License (LPPL)
%% ( http://www.latex-project.org/ ) version 1.3, and may be freely used,
%% distributed and modified. A copy of the LPPL, version 1.3, is included
%% in the base LaTeX documentation of all distributions of LaTeX released
%% 2003/12/01 or later.
%% Retain all contribution notices and credits.
%% ** Modified files should be clearly indicated as such, including  **
%% ** renaming them and changing author support contact information. **
%%*************************************************************************


% *** Authors should verify (and, if needed, correct) their LaTeX system  ***
% *** with the testflow diagnostic prior to trusting their LaTeX platform ***
% *** with production work. The IEEE's font choices and paper sizes can   ***
% *** trigger bugs that do not appear when using other class files.       ***                          ***
% The testflow support page is at:
% http://www.michaelshell.org/tex/testflow/


\documentclass[10pt,journal,compsoc]{IEEEtran}
%
% If IEEEtran.cls has not been installed into the LaTeX system files,
% manually specify the path to it like:
% \documentclass[10pt,journal,compsoc]{../sty/IEEEtran}





% Some very useful LaTeX packages include:
% (uncomment the ones you want to load)


% *** MISC UTILITY PACKAGES ***
%
%\usepackage{ifpdf}
% Heiko Oberdiek's ifpdf.sty is very useful if you need conditional
% compilation based on whether the output is pdf or dvi.
% usage:
% \ifpdf
%   % pdf code
% \else
%   % dvi code
% \fi
% The latest version of ifpdf.sty can be obtained from:
% http://www.ctan.org/pkg/ifpdf
% Also, note that IEEEtran.cls V1.7 and later provides a builtin
% \ifCLASSINFOpdf conditional that works the same way.
% When switching from latex to pdflatex and vice-versa, the compiler may
% have to be run twice to clear warning/error messages.






% *** CITATION PACKAGES ***
%
\ifCLASSOPTIONcompsoc
  % IEEE Computer Society needs nocompress option
  % requires cite.sty v4.0 or later (November 2003)
  \usepackage[nocompress]{cite}
\else
  % normal IEEE
  \usepackage{cite}
\fi
% cite.sty was written by Donald Arseneau
% V1.6 and later of IEEEtran pre-defines the format of the cite.sty package
% \cite{} output to follow that of the IEEE. Loading the cite package will
% result in citation numbers being automatically sorted and properly
% "compressed/ranged". e.g., [1], [9], [2], [7], [5], [6] without using
% cite.sty will become [1], [2], [5]--[7], [9] using cite.sty. cite.sty's
% \cite will automatically add leading space, if needed. Use cite.sty's
% noadjust option (cite.sty V3.8 and later) if you want to turn this off
% such as if a citation ever needs to be enclosed in parenthesis.
% cite.sty is already installed on most LaTeX systems. Be sure and use
% version 5.0 (2009-03-20) and later if using hyperref.sty.
% The latest version can be obtained at:
% http://www.ctan.org/pkg/cite
% The documentation is contained in the cite.sty file itself.
%
% Note that some packages require special options to format as the Computer
% Society requires. In particular, Computer Society  papers do not use
% compressed citation ranges as is done in typical IEEE papers
% (e.g., [1]-[4]). Instead, they list every citation separately in order
% (e.g., [1], [2], [3], [4]). To get the latter we need to load the cite
% package with the nocompress option which is supported by cite.sty v4.0
% and later. Note also the use of a CLASSOPTION conditional provided by
% IEEEtran.cls V1.7 and later.





% *** GRAPHICS RELATED PACKAGES ***
%
\ifCLASSINFOpdf
  % \usepackage[pdftex]{graphicx}
  % declare the path(s) where your graphic files are
  % \graphicspath{{../pdf/}{../jpeg/}}
  % and their extensions so you won't have to specify these with
  % every instance of \includegraphics
  % \DeclareGraphicsExtensions{.pdf,.jpeg,.png}
\else
  % or other class option (dvipsone, dvipdf, if not using dvips). graphicx
  % will default to the driver specified in the system graphics.cfg if no
  % driver is specified.
  % \usepackage[dvips]{graphicx}
  % declare the path(s) where your graphic files are
  % \graphicspath{{../eps/}}
  % and their extensions so you won't have to specify these with
  % every instance of \includegraphics
  % \DeclareGraphicsExtensions{.eps}
\fi
% graphicx was written by David Carlisle and Sebastian Rahtz. It is
% required if you want graphics, photos, etc. graphicx.sty is already
% installed on most LaTeX systems. The latest version and documentation
% can be obtained at: 
% http://www.ctan.org/pkg/graphicx
% Another good source of documentation is "Using Imported Graphics in
% LaTeX2e" by Keith Reckdahl which can be found at:
% http://www.ctan.org/pkg/epslatex
%
% latex, and pdflatex in dvi mode, support graphics in encapsulated
% postscript (.eps) format. pdflatex in pdf mode supports graphics
% in .pdf, .jpeg, .png and .mps (metapost) formats. Users should ensure
% that all non-photo figures use a vector format (.eps, .pdf, .mps) and
% not a bitmapped formats (.jpeg, .png). The IEEE frowns on bitmapped formats
% which can result in "jaggedy"/blurry rendering of lines and letters as
% well as large increases in file sizes.
%
% You can find documentation about the pdfTeX application at:
% http://www.tug.org/applications/pdftex






% *** MATH PACKAGES ***
%
%\usepackage{amsmath}
% A popular package from the American Mathematical Society that provides
% many useful and powerful commands for dealing with mathematics.
%
% Note that the amsmath package sets \interdisplaylinepenalty to 10000
% thus preventing page breaks from occurring within multiline equations. Use:
%\interdisplaylinepenalty=2500
% after loading amsmath to restore such page breaks as IEEEtran.cls normally
% does. amsmath.sty is already installed on most LaTeX systems. The latest
% version and documentation can be obtained at:
% http://www.ctan.org/pkg/amsmath





% *** SPECIALIZED LIST PACKAGES ***
%
%\usepackage{algorithmic}
% algorithmic.sty was written by Peter Williams and Rogerio Brito.
% This package provides an algorithmic environment fo describing algorithms.
% You can use the algorithmic environment in-text or within a figure
% environment to provide for a floating algorithm. Do NOT use the algorithm
% floating environment provided by algorithm.sty (by the same authors) or
% algorithm2e.sty (by Christophe Fiorio) as the IEEE does not use dedicated
% algorithm float types and packages that provide these will not provide
% correct IEEE style captions. The latest version and documentation of
% algorithmic.sty can be obtained at:
% http://www.ctan.org/pkg/algorithms
% Also of interest may be the (relatively newer and more customizable)
% algorithmicx.sty package by Szasz Janos:
% http://www.ctan.org/pkg/algorithmicx




% *** ALIGNMENT PACKAGES ***
%
%\usepackage{array}
% Frank Mittelbach's and David Carlisle's array.sty patches and improves
% the standard LaTeX2e array and tabular environments to provide better
% appearance and additional user controls. As the default LaTeX2e table
% generation code is lacking to the point of almost being broken with
% respect to the quality of the end results, all users are strongly
% advised to use an enhanced (at the very least that provided by array.sty)
% set of table tools. array.sty is already installed on most systems. The
% latest version and documentation can be obtained at:
% http://www.ctan.org/pkg/array


% IEEEtran contains the IEEEeqnarray family of commands that can be used to
% generate multiline equations as well as matrices, tables, etc., of high
% quality.




% *** SUBFIGURE PACKAGES ***
%\ifCLASSOPTIONcompsoc
%  \usepackage[caption=false,font=footnotesize,labelfont=sf,textfont=sf]{subfig}
%\else
%  \usepackage[caption=false,font=footnotesize]{subfig}
%\fi
% subfig.sty, written by Steven Douglas Cochran, is the modern replacement
% for subfigure.sty, the latter of which is no longer maintained and is
% incompatible with some LaTeX packages including fixltx2e. However,
% subfig.sty requires and automatically loads Axel Sommerfeldt's caption.sty
% which will override IEEEtran.cls' handling of captions and this will result
% in non-IEEE style figure/table captions. To prevent this problem, be sure
% and invoke subfig.sty's "caption=false" package option (available since
% subfig.sty version 1.3, 2005/06/28) as this is will preserve IEEEtran.cls
% handling of captions.
% Note that the Computer Society format requires a sans serif font rather
% than the serif font used in traditional IEEE formatting and thus the need
% to invoke different subfig.sty package options depending on whether
% compsoc mode has been enabled.
%
% The latest version and documentation of subfig.sty can be obtained at:
% http://www.ctan.org/pkg/subfig




% *** FLOAT PACKAGES ***
%
%\usepackage{fixltx2e}
% fixltx2e, the successor to the earlier fix2col.sty, was written by
% Frank Mittelbach and David Carlisle. This package corrects a few problems
% in the LaTeX2e kernel, the most notable of which is that in current
% LaTeX2e releases, the ordering of single and double column floats is not
% guaranteed to be preserved. Thus, an unpatched LaTeX2e can allow a
% single column figure to be placed prior to an earlier double column
% figure.
% Be aware that LaTeX2e kernels dated 2015 and later have fixltx2e.sty's
% corrections already built into the system in which case a warning will
% be issued if an attempt is made to load fixltx2e.sty as it is no longer
% needed.
% The latest version and documentation can be found at:
% http://www.ctan.org/pkg/fixltx2e


%\usepackage{stfloats}
% stfloats.sty was written by Sigitas Tolusis. This package gives LaTeX2e
% the ability to do double column floats at the bottom of the page as well
% as the top. (e.g., "\begin{figure*}[!b]" is not normally possible in
% LaTeX2e). It also provides a command:
%\fnbelowfloat
% to enable the placement of footnotes below bottom floats (the standard
% LaTeX2e kernel puts them above bottom floats). This is an invasive package
% which rewrites many portions of the LaTeX2e float routines. It may not work
% with other packages that modify the LaTeX2e float routines. The latest
% version and documentation can be obtained at:
% http://www.ctan.org/pkg/stfloats
% Do not use the stfloats baselinefloat ability as the IEEE does not allow
% \baselineskip to stretch. Authors submitting work to the IEEE should note
% that the IEEE rarely uses double column equations and that authors should try
% to avoid such use. Do not be tempted to use the cuted.sty or midfloat.sty
% packages (also by Sigitas Tolusis) as the IEEE does not format its papers in
% such ways.
% Do not attempt to use stfloats with fixltx2e as they are incompatible.
% Instead, use Morten Hogholm'a dblfloatfix which combines the features
% of both fixltx2e and stfloats:
%
% \usepackage{dblfloatfix}
% The latest version can be found at:
% http://www.ctan.org/pkg/dblfloatfix




%\ifCLASSOPTIONcaptionsoff
%  \usepackage[nomarkers]{endfloat}
% \let\MYoriglatexcaption\caption
% \renewcommand{\caption}[2][\relax]{\MYoriglatexcaption[#2]{#2}}
%\fi
% endfloat.sty was written by James Darrell McCauley, Jeff Goldberg and 
% Axel Sommerfeldt. This package may be useful when used in conjunction with 
% IEEEtran.cls'  captionsoff option. Some IEEE journals/societies require that
% submissions have lists of figures/tables at the end of the paper and that
% figures/tables without any captions are placed on a page by themselves at
% the end of the document. If needed, the draftcls IEEEtran class option or
% \CLASSINPUTbaselinestretch interface can be used to increase the line
% spacing as well. Be sure and use the nomarkers option of endfloat to
% prevent endfloat from "marking" where the figures would have been placed
% in the text. The two hack lines of code above are a slight modification of
% that suggested by in the endfloat docs (section 8.4.1) to ensure that
% the full captions always appear in the list of figures/tables - even if
% the user used the short optional argument of \caption[]{}.
% IEEE papers do not typically make use of \caption[]'s optional argument,
% so this should not be an issue. A similar trick can be used to disable
% captions of packages such as subfig.sty that lack options to turn off
% the subcaptions:
% For subfig.sty:
% \let\MYorigsubfloat\subfloat
% \renewcommand{\subfloat}[2][\relax]{\MYorigsubfloat[]{#2}}
% However, the above trick will not work if both optional arguments of
% the \subfloat command are used. Furthermore, there needs to be a
% description of each subfigure *somewhere* and endfloat does not add
% subfigure captions to its list of figures. Thus, the best approach is to
% avoid the use of subfigure captions (many IEEE journals avoid them anyway)
% and instead reference/explain all the subfigures within the main caption.
% The latest version of endfloat.sty and its documentation can obtained at:
% http://www.ctan.org/pkg/endfloat
%
% The IEEEtran \ifCLASSOPTIONcaptionsoff conditional can also be used
% later in the document, say, to conditionally put the References on a 
% page by themselves.




% *** PDF, URL AND HYPERLINK PACKAGES ***
%
%\usepackage{url}
% url.sty was written by Donald Arseneau. It provides better support for
% handling and breaking URLs. url.sty is already installed on most LaTeX
% systems. The latest version and documentation can be obtained at:
% http://www.ctan.org/pkg/url
% Basically, \url{my_url_here}.





% *** Do not adjust lengths that control margins, column widths, etc. ***
% *** Do not use packages that alter fonts (such as pslatex).         ***
% There should be no need to do such things with IEEEtran.cls V1.6 and later.
% (Unless specifically asked to do so by the journal or conference you plan
% to submit to, of course. )
%DIMAN

\pagenumbering{arabic} 
\usepackage[numbers,sort&compress,square]{natbib}
\usepackage[autostyle]{csquotes}

\usepackage{cite}
\usepackage{url}
\usepackage{parskip}
\usepackage{indentfirst} %indents first line of each section
\usepackage{color}
\setlength{\parindent}{10pt}
\usepackage{graphicx}
\usepackage{caption}
\usepackage{subcaption}
\usepackage{diagbox}
\usepackage[linesnumbered,ruled,vlined]{algorithm2e}
%\usepackage{natbib}
%\setlength{\bibsep}{0.0pt}
\usepackage{flushend}
\usepackage{mathtools}
%\usepackage{amsthm}
\usepackage{amsmath}
\usepackage{verbatim}
\usepackage{pbox}
%For Arxiv
%\usepackage[square,sort,comma,numbers]{natbib}
%\usepackage[authoryear]{natbib}
%\usepackage{lipsum}
%
\newenvironment{proof}[1][Proof]{\begin{trivlist}
\item[\hskip \labelsep {\bfseries #1}]}{\end{trivlist}}
%\usepackage{enumitem}
%DIMAN

% correct bad hyphenation here
\hyphenation{op-tical net-works semi-conduc-tor}


\begin{document}
%
% paper title
% Titles are generally capitalized except for words such as a, an, and, as,
% at, but, by, for, in, nor, of, on, or, the, to and up, which are usually
% not capitalized unless they are the first or last word of the title.
% Linebreaks \\ can be used within to get better formatting as desired.
% Do not put math or special symbols in the title.
%\title{CAGE: A Market-based \underline{C}ontention-\underline{A}ware \underline{G}am\underline{E}-theoretic Distributed model for heterogeneous resource assignment}
\title{CARMA: Contention-aware Auction-based Resource Management in Architecture}
%
% author names and IEEE memberships
% note positions of commas and nonbreaking spaces ( ~ ) LaTeX will not break
% a structure at a ~ so this keeps an author's name from being broken across
% two lines.
% use \thanks{} to gain access to the first footnote area
% a separate \thanks must be used for each paragraph as LaTeX2e's \thanks
% was not built to handle multiple paragraphs
%
%
%\IEEEcompsocitemizethanks is a special \thanks that produces the bulleted
% lists the Computer Society journals use for "first footnote" author
% affiliations. Use \IEEEcompsocthanksitem which works much like \item
% for each affiliation group. When not in compsoc mode,
% \IEEEcompsocitemizethanks becomes like \thanks and
% \IEEEcompsocthanksitem becomes a line break with idention. This
% facilitates dual compilation, although admittedly the differences in the
% desired content of \author between the different types of papers makes a
% one-size-fits-all approach a daunting prospect. For instance, compsoc 
% journal papers have the author affiliations above the "Manuscript
% received ..."  text while in non-compsoc journals this is reversed. Sigh.

\author{
{\rm  Farshid Farhat, ~\IEEEmembership{Student Member,~IEEE}, Diman Zad Tootaghaj, ~\IEEEmembership{Student Member,~IEEE} }\\
\thanks{We thank, Novella Bartolini and Mohammad Arjomand for their feedback on earlier drafts of this paper. }
\thanks{D. Z. Tootaghaj and Farshid Farhat are with the Comp. Sci. Dept. in the Pennsylvania State University, PA, USA (email: {\{dxz149, fuf111\}@cse.psu.edu}). A partial and preliminary version appeared in Proc. IEEE ICCD'17 \cite{tootaghajICCD}.}
}




% note the % following the last \IEEEmembership and also \thanks - 
% these prevent an unwanted space from occurring between the last author name
% and the end of the author line. i.e., if you had this:
% 
% \author{....lastname \thanks{...} \thanks{...} }
%                     ^------------^------------^----Do not want these spaces!
%
% a space would be appended to the last name and could cause every name on that
% line to be shifted left slightly. This is one of those "LaTeX things". For
% instance, "\textbf{A} \textbf{B}" will typeset as "A B" not "AB". To get
% "AB" then you have to do: "\textbf{A}\textbf{B}"
% \thanks is no different in this regard, so shield the last } of each \thanks
% that ends a line with a % and do not let a space in before the next \thanks.
% Spaces after \IEEEmembership other than the last one are OK (and needed) as
% you are supposed to have spaces between the names. For what it is worth,
% this is a minor point as most people would not even notice if the said evil
% space somehow managed to creep in.



% The paper headers
\markboth{}%IEEE Transactions on Emerging Topics in Computing}%, August~2015}%
{Shell \MakeLowercase{\textit{et al.}}: Bare Demo of IEEEtran.cls for Computer Society Journals}
% The only time the second header will appear is for the odd numbered pages
% after the title page when using the twoside option.
% 
% *** Note that you probably will NOT want to include the author's ***
% *** name in the headers of peer review papers.                   ***
% You can use \ifCLASSOPTIONpeerreview for conditional compilation here if
% you desire.



% The publisher's ID mark at the bottom of the page is less important with
% Computer Society journal papers as those publications place the marks
% outside of the main text columns and, therefore, unlike regular IEEE
% journals, the available text space is not reduced by their presence.
% If you want to put a publisher's ID mark on the page you can do it like
% this:
%\IEEEpubid{0000--0000/00\$00.00~\copyright~2015 IEEE}
% or like this to get the Computer Society new two part style.
%\IEEEpubid{\makebox[\columnwidth]{\hfill 0000--0000/00/\$00.00~\copyright~2015 IEEE}%
%\hspace{\columnsep}\makebox[\columnwidth]{Published by the IEEE Computer Society\hfill}}
% Remember, if you use this you must call \IEEEpubidadjcol in the second
% column for its text to clear the IEEEpubid mark (Computer Society jorunal
% papers don't need this extra clearance.)



% use for special paper notices
%\IEEEspecialpapernotice{(Invited Paper)}



% for Computer Society papers, we must declare the abstract and index terms
% PRIOR to the title within the \IEEEtitleabstractindextext IEEEtran
% command as these need to go into the title area created by \maketitle.
% As a general rule, do not put math, special symbols or citations
% in the abstract or keywords.
\IEEEtitleabstractindextext{%
\begin{abstract}
\noindent As the number of resources on chip multiprocessors (CMPs) increases, the complexity of how to best allocate these resources increases drastically. Because the higher number of applications makes the interaction and impacts of various memory levels more complex. Also, the selection of the objective function to define what \enquote{best} means for all applications is challenging. Memory-level parallelism (MLP) aware replacement algorithms in CMPs try to maximize the overall system performance or equalize each application's performance degradation due to sharing. However, depending on the selected \enquote{performance} metric, these algorithms are not efficiently implemented, because these centralized approaches mostly need some further information regarding about applications' need. In this paper, we propose a contention-aware game-theoretic resource management approach (CARMA) using market auction mechanism to find an optimal strategy for each application in a resource competition game. The applications learn through repeated interactions to choose their action on choosing the shared resources. Specifically, we consider two cases: (i) cache competition game, and (ii) main processor and co-processor congestion game. We enforce costs for each resource and derive bidding strategy. Accurate evaluation of the proposed approach show that our distributed allocation is scalable and outperforms the static and traditional approaches.
%Traditional resource management systems rely on a centralized approach to manage users running on each resource. The centralized resource management system is not scalable for large-scale servers as the number of users running on shared resources is increasing dramatically and the centralized manager may not have enough information about applications' need. In this paper we propose a distributed game-theoretic resource management approach using market auction mechanism to find optimal strategy in a resource competition game. The applications learn through repeated interactions to choose their action on choosing the shared resources. Specifically, we look into two case studies of cache competition game and main processor and co-processor congestion game. We enforce costs for each resource and derive bidding strategy. Accurate evaluation of the proposed approach show that our distributed allocation is scalable and outperforms the static and traditional approaches.
\end{abstract}

% Note that keywords are not normally used for peerreview papers.
\begin{IEEEkeywords}
Game Theory, Resource Allocation, Auction.
\end{IEEEkeywords}}


% make the title area
\maketitle


% To allow for easy dual compilation without having to reenter the
% abstract/keywords data, the \IEEEtitleabstractindextext text will
% not be used in maketitle, but will appear (i.e., to be "transported")
% here as \IEEEdisplaynontitleabstractindextext when the compsoc 
% or transmag modes are not selected <OR> if conference mode is selected 
% - because all conference papers position the abstract like regular
% papers do.
\IEEEdisplaynontitleabstractindextext
% \IEEEdisplaynontitleabstractindextext has no effect when using
% compsoc or transmag under a non-conference mode.



% For peer review papers, you can put extra information on the cover
% page as needed:
% \ifCLASSOPTIONpeerreview
% \begin{center} \bfseries EDICS Category: 3-BBND \end{center}
% \fi
%
% For peerreview papers, this IEEEtran command inserts a page break and
% creates the second title. It will be ignored for other modes.
\IEEEpeerreviewmaketitle






% if have a single appendix:
%\appendix[Proof of the Zonklar Equations]
% or
%\appendix  % for no appendix heading
% do not use \section anymore after \appendix, only \section*
% is possibly needed

% use appendices with more than one appendix
% then use \section to start each appendix
% you must declare a \section before using any
% \subsection or using \label (\appendices by itself
% starts a section numbered zero.)
%


%%%%%%%%%%%%%%%%%%%%%%%%%%%%%%%%%%%%%%%%%%%%%%%%%%%%%%%%%%%%%%%%%%%%%%%%%%%
%%%%%%%%%%%%%%%%%%%%%%%%%%%%%%%%%%%%%%%%%%%%%%%%%%%%%%%%%%%%%%%%%%%%%%%%%%%
%-------------------------------------------------------------------------------
\section{Introduction}
\label{sec_intro}
%-------------------------------------------------------------------------------
\begin{figure}[t]
    \centering
    \includegraphics[width=\linewidth]{trend.pdf}
    \caption{The token size for cutting-edge language models has been growing at an exponential rate over time, while the model size is reducing. The legend is represented in the format $ModelName-ModelSize$. For example, LLaMA-6.7B means the LLaMA model with 6.7 billion parameters.}
    \label{trend}
\end{figure}

\begin{figure}[ht]
    \centering
    \includegraphics[width=\linewidth]{Compared.pdf}
    \caption{Comparison of training a LLaMA-6.7B model with ZeRO-1 using 2T data and micro-batch-size of 4 on 128 GPUs and 1024 GPUs. When training on 1024 GPUs instead of 512 GPUs, the overall throughput is doubled, but per GPU efficiency drops by 77\% due to the increased ratio of communication and computation and suboptimal memory usage.}
    \label{compared}
\end{figure}




% In order to be able to process these large amounts of data quickly, the number of computational resources used to train these models is constantly increasing

LLMs, when trained on extensive text corpora, have demonstrated their prowess in undertaking new tasks \cite{Newtask1,Newtask2,Newtask3} either through textual directives or a handful of examples \cite{GPT1}. 
These few-shot properties first appeared when scaling models to a sufficient size \cite{kapscaling}, followed by a line of works that focus on further scaling these models \cite{GPT1,GPT2,GPT3,OPT,GLM-130B}. These efforts are grounded in the scaling law posted by OpenAI \cite{kapscaling,openaiscalinglaw}. However, DeepMind \cite{trainingcomputeoptimal} presented a contrarian perspective that smaller models supplemented with vast datasets can potentially yield enhanced results. Meta, while considering the inference budgets, trained the LLaMA model \cite{LLaMA}, which has a tenth of the model size of DeepMind's model but uses 1 TB tokens. The performance brought by this methodology outperformed those from larger models. Since the introduction of Chinchilla~\cite{trainingcomputeoptimal} by DeepMind, the token sizes for cutting-edge language models have been growing at an exponential rate. However the parameter sizes of the model have remained around 7B, as depicted by the likes of LLaMA2 \cite{Llama2}, Alpaca \cite{Alpaca}, and InternLM \cite{InternLM} in Figure \ref{trend}. 

Consequently, there is a preference to process this exponentially growing number of tokens using smaller models on larger training scales. For instance, as illustrated in Figure \ref{compared}, by training LLaMA-6.7B using micro-batch-size of 4 on a scale of 1024 GPUs \cite{LLaMA,Llama2} expanded from 128 GPUs \cite{Megatron-LM,ZeRO,PyTorchFSDP}, we can halve the time \cite{LLaMA,trainingcomputeoptimal} required to process data. 
% However, such super-scale training scenarios present distinct challenges and the decreasing computational efficiency and rising communication latency in Figure \ref{compared} demonstrate these two challenges: 
However, the decreasing computational efficiency and rising communication latency in Figure \ref{compared} are attributed to the following challenges:
% {\bfseries (1)} \textbf{the batch size per GPU is limited} by the maximum global batch size that can be employed without compromising convergence efficiency. 
{\bfseries (1)} \textbf{Per-GPU batch size is limited.} Due to the requirements of convergence and reproducibility, the global batch size remains constant. This limitation compromises the computational efficiency achieved on GPUs, especially at large scales.
{\bfseries (2)} \textbf{The latency of communication grows super-linearly} as the number of GPUs increases. 

There are two popular solutions to alleviate these challenges, namely 3D parallelism \cite{Megatron-LM,Megatron-LM1,Alpa} and ZeRO \cite{ZeRO,ZeRO++,ZeRO-Infinity}. However, they have limited scalability and cannot handle such super-scale scenarios.
Therefore, based on \textbf{(1)} and \textbf{(2)} we can trade GPU memory for communication. More specifically, instead of spreading model states across all the GPUs, we maintain copies of them. Besides, within the model state, there are three distinct components: parameters ($P$), gradient ($G$), and optimizer states ($OS$). Each of them exhibits varying degrees of communication overhead and memory cost. This heterogeneity offers the potential to flexibly select the redundancy level for each component individually. 


% Therefore, there is a tendency to deal with this exponential growth of tokens with smaller models on larger training scales. For example, as shown in Figure \ref{compared} we can reduce the time to process the data by a factor of two at a larger training size of 1024 GPUs versus 128 GPUs.
% , and proposed the 70B-Chinchilla training with 1.4 billion tokens. 
% Such a trend is illustrated by the left red arrow in Figure \ref{trend}. 
% These efforts are grounded in an underlying premise posited by OpenAI \cite{kapscaling,openaiscalinglaw}: 
% a power-law relationship intertwining the sizes of model parameters, dataset, and computational resources employed during the training of LLMs.

% However, DeepMind \cite{trainingcomputeoptimal} presented a contrarian perspective that smaller models supplemented with vast datasets can potentially yield enhanced results, and proposed the 70B-Chinchilla training with 1.4 billion tokens. 
% From Figure \ref{trend}, it is evident that compared to the contemporaneous GLM-130B, Chinchilla utilized half parameters but employed $3\times$ data. 

% Training these models that follow the latest trend has the following characteristics: (1) Model size range from 7B to 30B, which is approximately 10 times less than the models at historical scaling area. (2) Token size ranges from 1T to 2T, which is nearly 4 times larger than the previous models. (3) To swiftly process these vast amounts of data, the computational resources dedicated to training these models have surged dramatically.  For example, Figure \ref{compared} compares the training of a LLaMA-7B model allocating 1024 GPUs versus 128 GPUs, respectively. When we adopt the most efficient deep learning training system, 1024 GPUs gain 2 $\times$ time reduction compared to 128 GPUs shown in Figure \ref{compared}. However, per GPU efficiency (TFlops) plummets by 77\%, because higher ratio of communication and computation and inadequate memory usage.

% Therefore, recent DL training systems experience great challenges of scalability.
% (1) communication latency surges by 166\% due to collective limited scalability, (2) efficient computing time shrinks by 62\%, and (3) memory allocation per rank drops by 30\%. (2) and (3) both caused by the batch size per GPU being limited by the maximum global batch size that can be used during the training without sacrificing convergence efficiency. As a result, the throughput of the whole training cluster does not grow linearly with the number of GPUs. 






% 3D parallelism combines data parallelism \cite{ParallelizedSGD,PyTorchDistributed}, pipeline parallelism \cite{PipeDream,DAPPLE,GPipe}, and tensor parallelism \cite{Megatron-LM,Megatron-LM1} to distribute model training workloads across hundreds of GPUs. 



% This approach can achieve excellent per-GPU computing and memory efficiency at the expense of extra activation communication. In contrast, ZeRO could reduce the communication latency by overlapping communication with computation but underperforms by a high ratio of communication and computation. ZeRO is a memory-efficient variation of data parallelism, where model-states are partitioned across all the GPUs (Fig. \ref{GeneralPartition} blue block) and reconstructed using gather-based communication collectives on-the-fly during training. However, these collective communication overhead of ZeRO grows larger as the size of cluster scale up \S\ref{Comm_scaless} and are accompanied by negligible explicit memory overheads since batch size per GPU is limited by the maximum global batch size that can be used during the training without sacrificing convergence efficiency\cite{DemystifyingParallelandDistributedDeepLearning,LargeDP}.






Incorporating the above observations, we design \SysName to promote a more granular and adaptive strategy, allowing for component-specific optimization based on their individual communication and memory constraints. \SysName consists of three key modules, as shown in Figure \ref{SystemOverview}. Specifically, (1) it constructs a unified partition space for $P$, $G$, and $OS$ (grey block in Fig \ref{GeneralPartition}) based on the number of computing resources used by the user and the provided model size. This allows for finer-grained partitioning of $P$, $G$, and $OS$. (2) It introduces a \emph{Scale-Aware partitioner}, which consists of two key components: data dependency rules to avoid high data transfer costs, and a cost model focusing on communication and GPU memory for combination strategies. An established Integer Linear Programming (ILP) algorithm is then used to determine the best partitioning approach. (3) Communication Optimizer constructs a tree-based stratified communication strategy tailored for cross-node exchanges within model states. Furthermore, for specialized scenarios involving data exchanges between $P$ and $OS$, we incorporate a prefetch algorithm to ensure correct data sequencing.



% (1) we build a \emph{partition space} (Fig \ref{GeneralPartition}grey block) for components (P, G, and OS) respectively based on the number of compute resources used and model size provided by users. It realizes a finer-grained partition of the P, G, OS. (2) \emph{Scale-Aware partitioner}
%  (3) Communication Optimizer constructs a tree-based stratified communication strategy tailored for cross-node exchanges within model states. Furthermore, for specialized scenarios involving data exchanges between P and OS, we incorporate a prefetch algorithm to ensure correct data sequencing.


% In the extended partitioner space, we introduce an \emph{scale-aware partitioner}. This partitioner c

% % Unlike previous techniques where these components were treated collectively as a unified entity for optimization, our approach promotes a more granular and adaptive strategy, allowing for component-specific optimization based on their individual communication and memory constraints.
% minimize the collective communication overhead brought by the newest model training characteristic, making it near-linear scaling. 



% However, the batch size per GPU is limited by the maximum global batch size that can be used during the training without sacrificing convergence efficiency, which decreases the computation efficiency. Besides, communication overhead increases by the GPU number due to the limited scalability of collectives shown in Figure \ref{Comm_scaless}.


% The primary cause of this challenge stems from the rigidity in partitioning the Model-State in existing DL frameworks. These techniques process the tripartite components (Parameters-P, Graient-G, Optimizer States-OS) within the model state as an integrated unit although they have different memory and communication costs. Even though ZeRO is divided into three different stages to reduce memory usage, it operates in a holistic manner. For example, training a 7B model in 1024 GPUs, 


% and partitions the model state at the DP-world-size level.

% % the efficiency of ZeRO can be limited by high ratio of communication and computation







% % However, memory allocated at each rank decreases by 30\%, efficient computing time decreases by 62\%, communication latency increases by 166\% and the final accelerator efficiency (Tflops) decreases by 77\%. More generally, each GPU in the training cluster gains lower memory allocation but pays a higher ratio between communication and computation, which leads to a drastic decrease in the performance of GPUs. 

% % The primary cause of this challenge stems from the limited scalability of communication operators, as shown in Figure \ref{Comm_scaless}, when training at 512 GPUs, the both effective bandwidth of AllGather and Reduce-Scatter algorithm will decrease a lot, AllReduce shows a decreasing trend although it has been optimized from several studies comparing to training at 64 GPUs within the different number of parameters.

% The primary cause of this challenge stems from the rigidity in partitioning the Model-State in existing DL frameworks. These techniques like ZeRO \cite{ZeRO,ZeRO-Infinity,ZeRO++,PyTorchFSDP}, Megatron-LM 3D \cite{Megatron-LM,Megatron-LM1}, Alpa \cite{Alpa} process the tripartite components (Parameters-P, Graient-G, Optimizer States-OS) within the model state as an integrated unit.
% Even though ZeRO is divided into three different stages to reduce memory usage, it operates in a holistic manner and partitions the model state at the DP-world-size level. This means a DP-world-size level collective communication will be executed several times, which dramatically increases the communication overhead due to the limited scalability of the communication operator itself. As shown in Figure \ref{Comm_scaless}, when training with 512 GPUs, the effective bandwidth of both AllGather and Reduce-Scatter algorithms will decrease a lot. AllReduce shows a decreasing trend although it has been optimized from several studies compared to training at 64 GPUs within different numbers of parameters. Inspired by MiCS's solution \cite{MiCS} to address the issue of low communication bandwidth in cloud scenarios by reducing the number of communication participants, we observe that scaling down the communication can alleviate the challenges posed by the limited scalability of the communication operator. However, MiCS also treats the model state as a unit and then partitions it at a smaller scale like 1 node, or 2 nodes. etc. In this way, the size of the communication group increases as the model size grows to avoid the OOM error. As the subgroup size extends to DP-world-size, the benefit of MiCS is diminished. It reveals that such coarse-grained partitioning, i.e., treating the model state as an undivided entity, inadvertently bypasses potential optimization avenues. 
% Model-parallel-based techniques like Megatron-TP \cite{Megatron-LM,Megatron-LM1,PipeDream} and Alpa \cite{Alpa}, on the other hand, slices the parameters tensor of modules such as attention and MLP, allocating the G and OS of the model state to devices based on P tensor slicing. Nevertheless, we discern that such coarse-grained partitioning, treating the model state as an undivided entity, inadvertently bypasses potential optimization avenues. 


% Consequently, we can construct a smaller communication subgroup for each of the three elements, P, G, and OS, allowing flexibility in addressing memory and communication challenges across various scenarios. As illustrated in Figure \ref{GeneralPartition}, the previous DL framework's partitioning approach can be visualized as a grey block at the bottom, labeled "predefined group", encompassed by n(4) grey dashed lines. Given varying training inputs, users might opt for an optimal strategy from the top grey block. Nonetheless, by disaggregating the three components of the model state and permitting independent partitioning strategies, we can expand our search space to \(n^3(4^3)\) green lines, a substantial increase from the initial n(4) grey lines. This not only offers users a broader array of partition combinations at the top, but \SysName also automatically determines the most communication-efficient combination on their behalf. While the instances P, G, and OS are interdependent in terms of data, their computations remain autonomous \cite{ZeRO}. Leveraging this computational independence and satisfaction of data dependencies, we can configure distinct sharding scopes for each, which offers an opportunity to minimize communication overhead \cite{OverlapTP, EINNET, Alpa} in linear-scaling training scenarios. 


% The growth rate is proportional to the increase in data volume, typically involving the use of more than 256 GPUs. \cite{Megatron-LM,Megatron-LM1,ZeRO,PyTorchFSDP}.

% for three reasons: 

% This trend showcases three novel characteristics:
% Building upon this perspective, LLama, while taking inference budgets into consideration, trained a 7-billion parameter model using 1 terabyte of data. The results from this approach significantly outstripped those from much larger models. Consequently, the data size for cutting-edge Natural Language Processing (NLP) models has been escalating at an exponential pace, as depicted in Figure \ref{trend}. This trend have three new characteristics :


% \emph{(1) More Resource to train Smaller Model with Bigger DataSet}. To expediently process these vast quantities of data (D), there's been a steady surge in the computational resources (C) allocated for training these models. However, the model size (N) does not increase proportionally.\emph{(2) Memory not the Bottleneck}. With ample computational resources at one's disposal, memory storage is no longer the primary bottleneck in the training process due to constraints in global batch size \cite{DemystifyingParallelandDistributedDeepLearning, OnLarge-BatchTrainingforDeepLearning,LargeBatchOptimizationforDeepLearning}.\emph{(3) High Ratio between Comm $\&$ Comp}. Instead, the substantial overhead of collective communication within thousands of GPUs leads to a high ratio between communication and computation. 


% However, this poses scalability challenges for contemporary deep learning (DL) systems \cite{Megatron-LM,Megatron-LM1,ZeRO,PyTorchFSDP}for three reasons:


% \begin{itemize}[leftmargin=*,topsep=1pt, itemsep=2pt, itemindent=8pt]

%         \item {\bfseries P1}: \textsl{\bfseries Rigidity in Model-State Partitioning}.
%         Data-parallel-based techniques like Zero\cite{ZeRO,ZeRO-Infinity,ZeRO++} and MiCS \cite{MiCS} often process the tripartite components (Parameters-P, Graient-G, Optimizer States-OS) within the model state as an integrated unit. Even as Zero integrates stages to trim memory usage, it operates in a comprehensive manner. Model-parallel-based techniques like Megatron-TP \cite{Megatron-LM,Megatron-LM1,PipeDream,} and Alpa \cite{Alpa}, on the other hand, slices the parameters tensor of modules such as attention and MLP, allocating the G and OS of the model state to devices based on P tensor slicing. Nevertheless, we discern that such coarse-grained partitioning, treating the model state as an undivided entity, inadvertently bypasses potential optimization avenues.
        
%       \item {\bfseries P2}: \textsl{\bfseries Limited scalability of Communication Operators}.
%         Nvidia\cite{NvidiaBlog1} once elucidated the scalability challenges of the all-reduce\cite{Allreduce} algorithm, proposing tree-based or binary tree solutions\cite{TreeAllreduce1, TreeAllreduce2, TreeAllreduce3, Flattenedbutterfly} as potential remedies. However, these didn't offer flawless resolutions \cite{NvidiaBlog1}. Within the training ecosystem, there are other operators, like all-gather\cite{AllGather}, reduce\cite{Reduce}, and reduce-scatter\cite{ReduceScatter}, that grapple with scalability issues. Figure \ref{Comm_scaless} provides a visual representation of this challenge. Yet, the scalability problems of these operators remain largely unaddressed by the contemporary DL system, which will lead to high communication to exacerbate the ratio.

%       \item {\bfseries P3}: \textsl{\bfseries Low Compute Efficiency}.
%         When harnessing thousands of GPUs, the GPU's per-GPU batch size is confined by the maximum global batch size, ensuring no compromise in convergence efficiency. This implies that while global batch sizes cannot be perpetually escalated without decelerating model convergence\cite{DemystifyingParallelandDistributedDeepLearning,OnLarge-BatchTrainingforDeepLearning,LargeBatchOptimizationforDeepLearning}, utilizing thousands of GPUs invariably necessitates diminutive per-GPU batch sizes, memory is not a bottleneck for training at this point.
        
      
        
% \end{itemize}


% Essentially, while the approach of utilizing small models and large amounts of data has clear advantages algorithmically and in terms of inference costs, it makes LLM training less scalable.

% To bridge these gaps, we design \SysName, a near-linear scaling LLM training framework that outperforms previous DL systems tailored for earlier scaling laws. The core design of \SysName derives from the following three insights. First, \emph{the challenge posed by the limited scalability of the communication operator can be effectively tackled by narrowing its operational domain}. By targeting the operator's activity solely to periods when its communication thrives in efficiency, the hurdle of diminished effective bandwidth can be adeptly sidestepped. As shown in Figure\ref{Comm_scaless}, high bandwidth can be guaranteed by scaling the amount of each transmission, however, this can be detrimental to the overlap between communication and computation\cite{PyTorchFSDP}, or by reducing the number of participants in a pooled communication, utilizing memory in exchange for communication, in scenarios where memory is not a bottleneck. Second, \emph{It pays to build a larger partition space by partitioning the three members of the model state at a finer granularity}. While the instances P, G, and OS are interdependent in terms of data, their computations remain autonomous\cite{ZeRO}. Leveraging this computational independence and ensuring data dependencies are met, we can configure distinct sharding scopes for each. This approach offers an opportunity to minimize communication overhead \cite{OverlapTP,EINNET} in linear-scaling training scenarios. Third, \emph{layered communication can further reduce the overhead of across-nodes communication}. In the realm of communication operator optimization, numerous studies \cite{TreeAllreduce1,TreeAllreduce2,TreeAllreduce3}have suggested that hierarchical operations excel in scaling communication. Drawing inspiration from this, we can adapt such a strategy for scaling training. This involves conceptualizing GPUs distributed across multiple nodes as a two-dimensional grid. Within this framework, we first aggregate data across nodes in parallel, followed by consolidating local data on each individual node.



Extensive evaluations show a significant system throughput and scaling efficiency improvement of \SysName on training LLaMA-based models. On A100 (80GB) GPU clusters with 800Gbps network, the throughput of \SysName is 4 $\times$ larger than that of DeepSpeed, which is the state-of-the-art DP framework for large model training. Compared to Megatron-LM-3D, a state-of-the-art system specialized for training Transformer models, \SysName achieves up to 36.9\% larger throughput. \SysName gets near-linear (90.3\%) strong scaling efficiency in 1024 GPU training, which is up to 55\% better than DeepSpeed. 

In summary, we make the following contributions:
% In addition, \SysName successfully copes with the aforementioned deploymenand achieves the following desirable properties:




\begin{itemize}[leftmargin=*]
      \item 
      We build a unified partitioning space for model states, enabling independent and fine-grained partitioning strategies for $P$,$G$, and $OS$ of the model states. 
        
        % the data-dependent and computation-independent properties of the model state to construct a larger partition space than previous DL systems \cite{ZeRO,ZeRO-Infinity,ZeRO++,Megatron-LM,PyTorchFSDP}.


      % \item {\bfseries A2}: \textsl{\bfseries Optimial partition strategy}.
      %    \SysName discerns optimal partitioning strategies tailored to diverse scenarios within the adaptive partitioning framework. These strategies are distinguished by the employment and formulation of multiple communication subgroups that can be flexibly combined, setting them apart from the strategies of prior DL systems. By implementing these innovative approaches, we achieved a superior throughput compared to previous systems.
      
      \item 
      We design a scale-aware partitioner in \SysName to automatically derive the optimal partition strategy. 
      
      % for $P$, $G$, and $OS$ at each partition strategy and also integrates a dedicated communication optimizer to manage the data placement variances resulting from the differential partitioning of $P$, $G$, and $OS$.

      \item 
      We evaluate \SysName on training large models with 1024 GPUs and get near-linear (90.3\%) strong scaling efficiency.
\end{itemize}

Based on our current understanding and research, \SysName aims to address some of the challenges observed in distributed training frameworks given the prevalent training trends. We've explored the possibility of fine-grained partitioning of model states' members as a potential solution.

% Based on our current understanding and research, \SysName is the first framework to identify the limitations of distributed training frameworks under the existing training trend and contemplate fine-grained partitioning of model states' members as a solution.

 % We have systematically outlined the shortcomings of current methodologies ({\bfseries P1}$\sim${\bfseries P3}) and proposed a comprehensive end-to-end solution to address these limitations.
\vspace{-0.5\baselineskip}
\section{Motivation and Background} \label{Motivation}
\subsection{Motivation}
Different applications have different resource constraint with respect to CPU, memory, and bandwidth usage. Having a single resource manager for all existing resources and users in the system result in inefficiencies since it is not scalable and the operating system may not have enough information about applications' needs. For example, traditional LRU-based cache strategy uses cache utilization as a metric to give larger cache size to the applications which have higher utilization and lower cache size to the applications with lower cache utilization. However more cache utilization does not always result in better performance. Streaming applications for example have very high cache utilization, but very small cache reuse. In fact, the streaming applications only need a small cache space to buffer the streaming data. With rapid improvements in semiconductor technology, more and more cores are being embedded into a single core and managing large scale application using a single resource manager becomes more challenging. \\
%\indent Even if the applications are forced to announce their resource demand, it is possible that they lie about their resource vector or run some useless instructions to pretend to utilize the allocated resources given to them.
\indent In addition, defining a single fairness parameter for multiple applications is non-trivial since applications have different bottlenecks and may get different performance benefits from each resources during each phases of their execution time. Defining a single reasonable parameter for fairness is somewhat problematic. For instance, simple assignment algorithms which try to equally distribute the resources between all applications ignores the fact that different applications have different resource constraints. As a consequence, this makes the centralized resource management systems very inefficient in terms of fairness as well as performance needs of applications. We need a decentralized framework, where all applications' performance benefit could be translated into a unique notion of fairness and performance objective (known as utility function in economics) and the algorithm tries to allocate resources based on this translated notion of fairness. This translation has been well defined in economics and marketing, where the diversity of customer needs, makes more economically efficient market \cite{zhou2014sharing}. Economists have shown that in an economically efficient market, having diverse resource constraints and letting the customers compete for the resources can make a Nash equilibrium where both the applications and the resource managers can be enriched. Furthermore, applications' demand changes over time. Most resource allocation schemes pre-allocate the resources without considering the dynamism in applications' need and number of users sharing the same resource over time. Therefore, applications' performance can degrade drastically over time. Figure~\ref{fig:Phases} shows phase transitions for instruction per cycle (IPC) of mcf application from \textit{spec 2006} over 50 billion instructions. \\
%%%%%%%%%%%%%%%%%%%%%%%%%%%%%%%%%%%%%%%%%%%%%%%%%%%%%%%%%%%%%%%%%%%%%%%%%%%
\begin{figure}[!tb]
\centering
%\includegraphics[height=3in, width=1.5in]{NodeArchs2.pdf}
\includegraphics[height=1.8in, width=3.5in]{Images/Phases_May.pdf} %Phases.pdf
%\epsfig{file=Dataset.eps, height=2.5in, width=3in}
\vspace{-1.5\baselineskip}
\caption{\label{fig:Phases}Phase transition in mcf with different L2 cache sizes.}
\vspace{-1.0\baselineskip}
\end{figure}
%%%%%%%%%%%%%%%%%%%%%%%%%%%%%%%%%%%%%%%%%%%%%%%%%%%%%%%%%%%%%%%%%%%%%%%%%%%
\indent We try to find a game-theoretic distributed resource management approach where the shared hardware resources are exposed to the applications and we show that by running a repeated auction game between different applications which are assumed to be rational, the output of the game converges to a balanced Nash equilibrium allocation. In addition, we compare the convergence time of the proposed algorithm in terms of dynamism in the system. We evaluate our model with two case studies: 1) Private and shared last level cache problem, where the applications have to decide if they would benefit from a larger cache space which can potentially get more congested or a smaller cache space which is potentially less congested. 2) Heterogeneous processors (\textit{Intel Xeon} and \textit{Xeon Phi}) problem, where we perform experiments to show how congestion affects the performance of different applications running on an \textit{Intel Xeon} or \textit{Xeon Phi} co-processors. Depending on the amount of congestion in the system, the application can offload the most time consuming part of its code on the \textit{Xeon Phi} co-processors or not.
%%%%%%%%%%%%%%%%%%%%%%%%%%%%%%%%%%%%%%%%%%%%%%%%%%%%%%%%%%%%%%%%%%%%%%%%%%%
\vspace{-1\baselineskip}
\subsection{Background}
%Congestion games have been studied in network routing protocols where the delay of each player choosing a path in the network depends on the number of players choosing the same route in the system. 
%Every congestion game is a potential game since there exists a potential function associated with it. In addition, every congestion game has a pure-strategy Nash equilibrium. A key assumption in congestion games is that all users have the same impact on the congestion. However, this assumption is not always true. In case of computer architecture resources, applications effect each other differently and dividing the payoff function by the number of users running on the shared resource does not give us the correct utility. 
Game theory has been used extensively in economics, political and social decision making situations \cite{tootaghaj2011game, tootaghaj2011risk, kotobi2017spectrum, kotobi2015introduction, kesidis2013distributed, kurve2013agent, wang2017using, wang2015recouping}. A game is a situation, where the output of each player not only depends on her own action in the game, but also on the action of other players \cite{osborne1994course}. Auction games are a class of games which has been used to formulate real-world problems of assigning different resources between $n$ users. Auction game framework can model resource competition, where the payoff (cost) of each application in the system is a function of the contention level (number of applications) in the game.\\
\indent Inspired by market-based interactions in real life games, there exists a repeated interaction between competitors in a resource sharing game. Assuming large number of applications, we show that the system gets to a Nash equilibrium where all applications are happy with their resource assignment and don't want to change their state. Furthermore, we show that the auction model is \textit{strategy-proof}, such that no application can get more utilization by bidding more or less than the true value of the resource. In this paper we propose a distributed market based approach to enforce cost on each resource in the system and remove the complexity of resource assignment from the central decision maker.\\ 
\indent The traditional resource assignment is performed by the operating system or a central hardware to assign fair amount of resources to different applications. However, fair scheduling is not always optimal and solving the optimization problem of assigning $m$ resources between $n$ users in the system is an integer programming which is an NP-hard problem and finding the best assignment problem becomes computationally infeasible. Prior works focus on designing a fair scheduling function that maximizes all application's benefit \cite{zahedi2014ref, llull2017cooper, ghodsi2011dominant, zahedi2015sharing, fan2016computational}, while applications might have completely different demands and it is not possible to use the same fairness function for all. By shifting decision making to the individual applications, the system becomes scalable and the burden of establishing fairness is removed from the centralized decision maker, since individual applications have to compete for the resources they need. Applications start by profiling the utility function for each resource and bid for the most profitable resource. During the course of execution time they can update their belief based on the observed performance metrics at each round of the auction. Updating the utility functions at each round of the auction is based on the history of the observed performance metrics which shows the state of the game. 
%The idea behind updating the utility functions is that the history at each round of the auction shows the state of the game. 
This state indicates the contention on the current acquired resources. The payoff function in each round depends on the state of the system and on the action of other applications in the system. 
%%%%%%%%%%%%%%%%%%%%%%%%%%%%%%%%%%%%%%%%%%%%%%%%%%%%%%%%%%%%%%%%%%%%%%%%%%%
\subsubsection{Sequential Auction}
Auction-based algorithms are used for maximum weighted perfect matching in a bipartite graph $G=(U,V, E)$ \cite{bertsekas1998network, kyle1985continuous, vasconcelos2009bipartite}. A vertex  $U_i \in U$ is the application in the auction and a vertex $V_j \in V$ is interpreted as a resource. The weight of each edge from $U_i$ to $V_j$ shows the utility of getting that particular resource by $U_i$. The prices are initially set to zero and will be updated during each iteration of the auction. In sequential auctions, each resource is taken out by the auctioneer and is sequentially auctioned to the applications, until all the resources are sold out.
\subsubsection{Parallel Auction}
In a parallel auction, the applications submit their bids for the first most profitable item. The value of the bid at each iteration is computed based on the difference of the highest profitable object and the second highest profitable object. The auctioneer would assign the resources based on the current bids. At each iteration, the valuation of each resource is updated based on the observed information during run-time which shows the contention on that particular resource.
%%%%%%%%%%%%%%%%%%%%%%%%%%%%%%%%%%%%%%%%%%%%%%%%%%%%%%%%%%%%%%%%%%%%%%%%%%%%
%%%%%%%%%%%%%%%%%%%%%%%%%%%%%%%%%%%%%%%%%%%%%%%%%%%%%%%%%%%%%%%%%%%%%%%%%%%%
%\vspace{-1\baselineskip}
\section{The Method}\label{Problem_definition}
Consider $n$ applications and $i$ instances of $m$ different resources. Applications arrive in the system one at a time. The applications have to choose among $m$ resources. There exists a bipartite graph between the matching of the applications and the resources.\\
\indent In general, there can be more than one application to get a shared resource. However, each application cannot get more than one of the available heterogeneous resources. For example, if we have two cache spaces of size 128kB (one way) and 256kB (two ways), each application can either get the 128kB, or the 256kB cache space and can't get both of them at the same time. Furthermore, each resource $R_k$ has a cost $p_k$ which is defined by the applications' bid in the auction. \\
\indent Figure~\ref{fig:auction} shows our auction-based framework to support \textit{CARMA} between $N$ applications that execute together competing for $M$ different resources. Each application has a utility table that shows how much performance it gets from each $M$ resources at each time slot. Based on the utility tables, applications submit bids for the most profitable resource. Based on the submitted bids, the auctioneer decides about the resource assignment for each resource, and updates the prices. Next, the applications who did not get any assignment compete for the next most profitable resource based on the updated prices repeatedly until all applications are assigned.  Figure~\ref{fig:auction} shows an example of a resource assignment and the corresponding bipartite graph.
%Table~\ref{table:notation} shows the notation used in our formulation.
%%%%%%%%%%%%%%%%%%%%%%%%%%%%%%%%%%%%%%%%%%%%%%%%%%%%%%%%%%%%%%%%%%%%%%%%%%%%%%%
%%%%%%%%%%%%%%%%%%%%%%%%%%%%%%%%%%%%%%%%%%%%%%%%%%%%%%%%%%%%%%%%%%%%%%%%%%
\begin{table}[!tb] \scriptsize
\centering
\caption{Notation used in our formulations.}\label{Table:notation}
\begin{tabular}{|p{0.7in}|p{2.3in}|} 
\hline $N$ & Number of players or applications (from $App_1$ to $App_N$). \\
\hline $M$ & Number of resources (from $R_1$ to $R_M$). \\
\hline $\vec{m}$ & A positive $M\times 1$ vector in the resource space that shows how much each application gets from each resource. \\   % of size $M\time 1$ showing the amount for each resources. \\ Number of applications which can get a resource \\ 
% \hline $n$ & Number of applications competing for a specified resource \\
\hline $T$ & Time intervals where the bidding is held \\
\hline $t_{i,j}$ & $j$-th phase time for $i$-th application during its course of execution time. \\ 
\hline $T_i$ & Last phase time for application $i$. \\
\hline $v_{i}(t,\vec{m})$ & The valuation function of application $i$ for the resource assignment $\vec{m}$ at time $t$. \\
\hline $v_{i,j}(t,\vec{m},r)$ & The valuation function of (application $i$,resource $j$), if we replace the $j$-th resource in the resource vector $m$ by $r$ \\
\hline $\delta$ & dynamic factor that shows how much we can rely on the past iterations. \\
\hline $G=(U,V,E)$ & A bipartite graph showing the resource allocation between the applications and the set of resources. \\
\hline $U$ & The set of applications which shows the left set of nodes in the bipartite graph $G=(U,V,E)$. \\
\hline $V$ & The set of resources which shows the right set of nodes in the bipartite graph $G=(U,V,E)$. \\
\hline $E$ & The edges in the bipartite graph. \\
\hline $b_{i,k}$ & User i's bid for k-th resource. \\
\hline $F_i$ & The total budget (summation of bids) a user have. \\
\hline $C_k$ & The total capacity of each resource. \\
\hline $p_{k}$ & The price of resource $k \in V$ in the auction. \\
%\hline $Bottleneck_{1,i}$ & The first bottleneck resource for application $i$ \\
%\hline $Bottleneck_{2,i}$ & The second bottleneck resource for application $i$ \\
\hline $K$ & Number of cache levels \\
\hline
\end{tabular}
\vspace{-1.0\baselineskip}
\end{table}
%%%%%%%%%%%%%%%%%%%%%%%%%%%%%%%%%%%%%%%%%%%%%%%%%%%%%%%%%%%%%%%%%%%%%%%%%%%%%%%
\begin{figure*}[!htb]
\centering
%\includegraphics[height=3in, width=1.5in]{NodeArchs2.pdf}
\includegraphics[height=3.2in, width=6.5in]{Images/Auction_v4.pdf} %[height=4in, width=8in]
%\epsfig{file=Dataset.eps, height=2.5in, width=3in}
\vspace{-1\baselineskip}
\caption{\label{fig:auction} Framework for auction-based resource assignment (CARMA).}
\vspace{-0.5\baselineskip}
\end{figure*}
%%%%%%%%%%%%%%%%%%%%%%%%%%%%%%%%%%%%%%%%%%%%%%%%%%%%%%%%%%%%%%%%%%%%%%%%%%%
\begin{comment}
\begin{figure}[!htb]
\centering
%\includegraphics[height=3in, width=1.5in]{NodeArchs2.pdf}
\includegraphics[height=2.2in, width=1.3in]{Images/bipartite.pdf}
%\epsfig{file=Dataset.eps, height=2.5in, width=3in}
\caption{\label{fig:bipartite} Cache allocation as a bipartite graph.}
\end{figure}
\end{comment}
%%%%%%%%%%%%%%%%%%%%%%%%%%%%%%%%%%%%%%%%%%%%%%%%%%%%%%%%%%%%%%%%%%%%%%%%%%%
\vspace{-1\baselineskip}
\subsection{Problem Defenition} 
\indent We formulate our problem as an auction to enforce cost/value updates for each resource as follows: 
%The cost of each player to get a resource is the cost of the assigned resource divided by the number of players who share. 
\begin{itemize}
  \item \textbf{Valuation $\mathbf{v_{i}(t,\vec{m})}$}: Application $i$ has a valuation function which shows how much it benefits from the resource vector $\vec{m}$ at time $t$. The valuation function at time $t=0$ for cache contention case study is derived from the IPC (instruction per cycle) curves using profiling, and for processor and co-processor contention case study is derived from the profiling of separate cache performance of the application. However, in general, each application can choose its own utility function.
  %%%%%%%%%%%%%%%%%%%%%%%%%%%%%%%%%%%%%%%%%%%%%%%%%%%%%% 
    \item \textbf{Observed information}: The observed information at each time step is the performance value of the selected action in the game. Therefore, the applications repeatedly update the history of their valuation function over time.  
  %%%%%%%%%%%%%%%%%%%%%%%%%%%%%%%%%%%%%%%%%%%%%%%%%%%%%%   
    \item \textbf{Belief updating}: Let $T$ be the time intervals where the bidding is held. At each iteration step of the auction, the applications update their valuation of each resource based on the observed performance on the resource vector. The update at time $W$ is derived using the following formula:
%\begin{small}
\begin{equation}\label{eq:belief}
v_{i}(W,\vec{m})=\frac{\sum\limits_{0\leq n\leq W/T} {\delta}^{W/T-n} \cdot v_{i}(nT,\vec{m})}{\sum\limits_{0\leq n\leq W/T} {\delta}^{W/T-n}},
\end{equation}
%\end{small}  
%%%%%%%%%%%%%%%%%%%%%%%%%%%%%%%%%%%%%%%%%%%%%%%%%%%%%%%%%%%%%%%%%%%%%%%%%
where $v_{i}(W,\vec{m})$ shows the observed valuation of resource vector $\vec{m}$ at time $W$ by application $i$ in the system; $\delta$ shows the discount factor between 0 and 1 which shows how much a user relies on its past observations in the system. The discount factor is chosen to show the dynamics of the system. If the observed information in the system changes fast, the discount factor is close to zero, i.e. the application cannot rely on the past observations very much. However, if the system is more stable and the observed information does not change fast, the discount factor is closer to 1. If a user fails in an auction, its payoff and corresponding observed valuation at the current time is equal to zero. So, it won't probably bid for the same resource vector again, since its valuation decreases for next round. We choose the discount factor to be the absolute value of the correlation coefficient of the observed values of the valuations at each iteration step which is calculated as follows:
%\begin{small}
\begin{equation}
\delta =  \frac{E(v_{i}(W,\vec{m}))^2}{{\sigma_{v_{i}(W,\vec{m})}}^2}
\end{equation}  
%\end{small}
%%%%%%%%%%%%%%%%%%%%%%%%%%%%%%%%%%%%%%%%%%%%%%%%%%%%%%%%%%%%%%%%%%%%%%%%
%%%%%%%%%%%%%%%%%%%%%%%%%%%%%%%%%%%%%%%%%%%%%%%%%%%%%%%%%%%%%%%%%%%%%%%%
  \item \textbf{Action}: At each time step, the applications decide which resource to bid and how much to bid for each resource. 
\end{itemize} 
%%%%%%%%%%%%%%%%%%%%%%%%%%%%%%%%%%%%%%%%%%%%%%%%%%%%%%%%%%%%%%%%%%%%%%%%%% 
\indent Table~\ref{Table:notation} shows important notation used throughout the paper. In the following sections, we describe our distributed optimization scheme to solve the problem. 
%%%%%%%%%%%%%%%%%%%%%%%%%%%%%%%%%%%%%%%%%%%%%%%%%%%%%%%%%%%%%%%%%%%%%%%
%\begin{equation}
%min \sum\limits_{i=1}^n v_i C_k \frac{b_{i,k}}{\theta_k}, \\
%s.t. \sum\limits_{i=1}^n b_{i,k} \leq E_i
%\end{equation}
%%%%%%%%%%%%%%%%%%%%%%%%%%%%%%%%%%%%%%%%%%%%%%%%%%%%%%%%%%%%%%%%%%%%%%%
\vspace{-1\baselineskip}
\subsection{Distributed Optimization Scheme}
The goal is to design a repeated auction mechanism which runs by the operating system to guide the applications to choose their best resource allocation strategy. The applications' goal is to maximize their own performance and the operating system wants to maximize the total utility gain from the applications. Each application can use its own utility function and evaluates the resources based on the desired value of the resources. \\
\indent \textbf{Applications' approach}: The application $i$ wants to maximize the expected utility (pay-off) with respect to a limited budget ($F_i$) during all phases of its execution time. We have:
%\vspace{-0.5\baselineskip}
%%%%%%%%%%%%%%%%%%%%%%%%%%%%%%%%%%%%%%%%%%%%%%%%%%%%%%%%%%%%%%%%%%%%%%%%%
%maximize \;\;\;\; \sum\limits_{i=1}^n v_i C_k \frac{b_{i,k}}{\theta_k},\\
%\begin{small}
\begin{align}
%\begin{IEEEeqnarray}{rCl}
\forall i \in U \; \; \; \; \; maximize \; \; \; \; \sum\limits_{0<t<T_i} v_{i}(t,\vec{m})-b_{i}(t,\vec{m}) , \nonumber \\
 % \IEEEyessubnumber\\
subject \; to \;\;\;\; \sum\limits_{0<t<T_i} b_{i}(t,\vec{m}) \leq F_i, \; \; \; \; \forall i \in U.
%\IEEEyessubnumber
%\end{IEEEeqnarray}
\end{align}
%\end{small}
%%%%%%%%%%%%%%%%%%%%%%%%%%%%%%%%%%%%%%%%%%%%%%%%%%%%%%%%%%%%%%%%%%%%%%%%%%
\indent \textbf{OS's approach}: The operating system wants to maximize the social welfare function which is translated into submitted bids from the applications in a limited resource constraints.
\vspace{-0.5\baselineskip}
%%%%%%%%%%%%%%%%%%%%%%%%%%%%%%%%%%%%%%%%%%%%%%%%%%%%%%%%%%%%%%%%%%%%%%%%%%
%\begin{small}
\begin{align}
%\begin{IEEEeqnarray}{rCl}
maximize \; \; \; \sum\limits_{i=1}^N \sum\limits_{0<t<T_i}  b_{i}(t,\vec{m}) \cdot A_{i}(t, \vec{m}) , \nonumber \\ 
%\IEEEyessubnumber\\
subject \; to \;\;\;\; \sum\limits_{i=1}^N  A_{i}(t, \vec{m}) \leq A_{max}, \; \; \; \; \forall t: 0 \leq t \leq T , \nonumber \\
%\IEEEyessubnumber\\
A_{i}(t, \vec{m}) \in \{0,1\} , \; \; \forall i \in U, \; \;  \forall \vec{m} \subseteq \mathcal{V}, \; \;  \forall t:  0 \leq t \leq T ,
\end{align}
where the binary variable $A_{i}(t, \vec{m})$ represents the decision to assign resource vector $\vec{m}$ to application $i$ at time $t$ (when $A_{i}(t, \vec{m}) =1$) or not (when $A_{i}(t, \vec{m})=0$); and $\mathcal{V}$ is the vector space of the all resource vectors ($\forall\vec{m}$); and $A_{max}$ shows the maximum number of the applications which can share the resource vector $\vec{m}$. 

%%%%%%%%%%%%%%%%%%%%%%%%%%%%%%%%%%%%%%%%%%%%%%%%%%%%%%%%%%%%%%%%%%%%%%%%%%%
%\begin{comment}
\indent \textbf{Illustrative example}: As an illustrative example shown in Figure~\ref{fig:Dynamic}, let us consider a case where we have two different resources, a large cache of 1MB which can be shared between applications, and two private caches of 512KB which are not shared. 
%There are two applications competing for the cache space. One of the applications wants to minimize its request latency and the other one wants to maximize number of instructions executed per cycle. Suppose that both applications have two phases $(0,T)$ and $(T,2T)$. If the first application gets the larger cache space its \textit{IPC} increases by 15 percent in first phase and by 36.84 percent in the second phase. The second application's \textit{IPC} increases by 35 percent in the first phase and by 20.6 percent in the second phase if it gets the larger cache space. %Also, suppose that they both have 60 tokens as a budget to submit. 
The first application participates in the auctions with 35¢ bid at the first phase and 21¢ bid at the next phase. The second application participates in the auctions with 15¢ bid at the first phase and 37¢ bid at the next phase. The auctioneer (OS) decides to allocate larger cache in the auctions at first phase to the first application and at next phase to the second application. %Since, the social welfare would be maximized if the auctioneer allocates both applications with the larger cache space, they would both get the larger resource. Then the first application notices that its utility function does not improve as he predicts and adjusts the utility table and can either change its allocation or stay on current allocation. 
%The application should redistribute the tokens for the next phase if it did not get the desired resource in the first phase. The second application invests 35 tokens in the first phase and 25 tokens in the next phase. The auctioneer (OS) at each phase decides to allocate which resource to which applications. Since, the social welfare would be maximized if the auctioneer allocates both applications with the larger cache space, they would both get the larger resource. Then the first application notices that its utility function does not improve as he predicts and adjusts the utility table and can either change its allocation or stay on current allocation. 
%If both applications bid 10\$ for the private cache and 15\$ for the shared cache, the operating system would allocate both the shared cache space and get 15\$ from each to maximize its revenue.
%\end{comment}
\vspace{-1\baselineskip}
\subsection{Analysis}
The distributed optimization problem is hard to solve. However, in reality, the problem can split into simpler subproblems since each application knows its bottleneck resource and would first bid for the first bottleneck resource to maximize its utility.\\
\indent We suppose all applications in the system are risk-neutral which means they have a linear valuation of the utility function. Each risk-neutral agent wants to maximize its expected revenue. Risk attitude behaviors are defined in \cite{ferber1999multi} where the agents can broadly be divided into risk-averse, risk-seeking and risk neutral. Risk-averse agents prefer deterministic values rather than risky value profits and risk-seeking applications have a super-linear utility function and prefer risky utilities than sure utilities. Next, we derive the Bayes Nash equilibrium strategy profile for all agents in the system assuming risk neutrality. 
%%%%%%%%%%%%%%%%%%%%%%%%%%%%%%%%%%%%%%%%%%%%%%%%%%%%%%%%%%%%%%%%%%
\newtheorem{defi}{Definition}
\begin{defi}
A strategy profile $a$ is a pure Nash equilibrium if for every application $i$ and every strategy $a_i' \neq a_i \in A$ we have $u_i(a_i, a_{-i}) \geq u_i(a_i', a_{-i})$
\end{defi}
\newtheorem{theorem}{Theorem}
\begin{theorem}\label{thm:neat}
%\emph{(Theorem)}
\label{Auction}
Suppose $n$ risk-neutral applications whose valuations are derived uniformly and independently from the interval $[0,1]$ compete for one resource which can be assigned to $m$ applications who have the highest bid in the auction. We will show that Bayes Nash equilibrium bidding strategy for each application in the system is to bid $\frac{n-m}{n-m+1}v_i$ where $v_i$ is the profit of application $i$ for getting the specified resource.  
\end{theorem}
%%%%%%%%%%%%%%%%%%%%%%%%%%%%%%%%%%%%%%%%%%%%%%%%%%%%%%%%%%%%%%%
%\vspace{-1\baselineskip}
\begin{algorithm}[!tb] %\small
\DontPrintSemicolon % Some LaTeX compilers require you to use \dontprintsemicolon    instead
\KwIn{A bipartite Graph (U, V, E).}
\KwOut{The allocation of resources to applications.}
The initial resource vector for each application is the average amount across various resources. At time $t=nT$, the valuation of each application for each resource vector is updated using Eq.~\ref{eq:belief}.

For application $U_i \in U$, the first bottleneck resource is
%\[ Bottleneck_{1,i} = V_{i,m}=  arg \; \max_{m \in V} (v_{i,m}-p_{m})  \] 
%\[ Bottleneck_1: V_{i,j^{max}_1}=  \max_{1\leq j_1 \leq M} {\Delta v_{i,j_1}(t,\vec{m},r_{j_1})-p_{j_1}}  \] 
\[ V_{i,j^{max}_1}=  \max_{1\leq j_1 \leq M} {\Delta v_{i,j_1}(t,\vec{m},r_{j_1})-p_{j_1}}  \] 
where the differential valuation function is $\Delta v_{i,j}(t,\vec{m},r_j) = v_{i,j}(t,\vec{m},r_j) - v_{i}(t,\vec{m})$. 

Find the second bottleneck resource for application $U_i \in U$ in the system:
%\[ Bottleneck_2: V_{i,j^{max}_2}=  \max_{1\leq j_2 \leq M; j_2\neq j_1} {\Delta v_{i,j_2}(t,\vec{m},r_{j_2})-p_{j_2}}  \] 
\[ V_{i,j^{max}_2}=  \max_{1\leq j_2 \leq M; j_2\neq j_1} {\Delta v_{i,j_2}(t,\vec{m},r_{j_2})-p_{j_2}}  \] 

Each application calculates the partial bid for its first bottleneck resource using the following formula:
\[ b_{i,j^{max}_1}(t) = V_{i,j^{max}_1} - V_{i,j^{max}_2} + p_{j^{max}_1} + \epsilon \]

Each resource $r_j \in V$ which can be shared between $L$ applications, is assigned to the $L$ highest bidding applications $Winner_{i,j}=\{i_1, i_2, ..., i_L\}$ and the price for that resource is updated as follows:
%\[ p_{j} = \max_{i_1, i_2, ..., i_L \in U} \sum\limits_{k=1}^L {b_{i_k,j}}  \]
\[ p_{j} = \max_{l \in \{1,...,L\}} {b_{i_l,j}}  \]

The $minBid$ for each resource is updated as the minimum bid of $l$ applications who acquired $j$-th resource. That is:
\[ B^{min}_j=  \min_{i_l \in Winner_{i,j}} {b_{i_l,j}} \] 

Goto step 2 until all partial bids for all resources are determined. 

The total bid of the application $i$ is as follows:
\[ b_{i}(t,\vec{m})=b_{i}(t,[r_{j_1};r_{j_2};...;r_{j_M}])=\sum\limits_{j=1}^M {b_{i,j}(t)} \]
where iteratively $\vec{m}=[r_{j_1};r_{j_2};...;r_{j_M}]$.

Find the estimated investment $I_{i}(t)$ using Algorithm~\ref{algo:p} to plan the upper bound of the investment with respect to the budget $F_i$. If $I_{i}(t)\geq b_{i}(t,\vec{m})$, application $i$ will participate in this auction at time $t$, otherwise it quits and other applications do the steps 2 and 3.

\caption{Parallel Auction for Heterogeneous Resource Assignment}
\label{algo:b}
\end{algorithm}
%%%%%%%%%%%%%%%%%%%%%%%%%%%%%%%%%%%%%%%%%%%%%%%%%%%%%%%%%%%%%%%%%%
%DIMAN COMMENTED PROOF FOR APPENDIX
\begin{comment}
\begin{proof}
Suppose all other applications' bidding strategy is to choose $\frac{n-m}{n-m+1}v_i$. Since the bidding values were derived uniformly in $[0, 1]$ all bids have the same probability. Therefore, if we consider the first application's expected utility to find its best response, we have:

\begin{equation}
%\begin{IEEEeqnarray}{rCl}
E[u_1] = \int_0^1  .... \int_0^1 \! (v_1 -b_1) \, \mathrm{d}u_2 \mathrm{d}u_3 ... \mathrm{d}u_{n-m} .  
%\end{IEEEeqnarray}
\end{equation}

The following integral breaks into two part where the first application wins the auction or not. 


%\begin{IEEEeqnarray}{rCl}
%E[u_1] = \int_0^{b_1}  .... \int_0^{b_1} \! (v_1 -b_1) \, \mathrm{d}u_2 \mathrm{d}u_3 ... %\mathrm{d}u_{n-m}  \\
%+ \int_{b_1}^1  .... \int_{b_1}^1 \! (v_1 -b_1) \, \mathrm{d}u_2 \mathrm{d}u_3 ... \mathrm{d}%u_{n-m}\nonumber 
%\end{IEEEeqnarray}

\begin{equation}
%\begin{IEEEeqnarray}{rCl}
E[u_1] = \int_0^{\frac{n-m+1}{n-m}b_1}  .... \int_0^{\frac{n-m+1}{n-m}b_1} \! (v_1 -b_1) \, \mathrm{d}u_2 ... \mathrm{d}u_{n-m}  \\
+ \int_{\frac{n-m+1}{n-m}v_1}^1  .... \int_{\frac{n-m+1}{n-m}v_1}^1 \! (v_1 -b_1) \, \mathrm{d}u_2 \mathrm{d}u_3 ... \mathrm{d}u_{n-m}\nonumber 
%\end{IEEEeqnarray}
\end{equation}

The second part of the integrals is the term where the first application doesn't win the auction. Therfore, the expected payoff of application 1 is equal with:

%\begin{IEEEeqnarray}{rCl}
%E[u_1] = \int_0^{b_1}  .... \int_0^{b_1} \! (v_1 -b_1) \, \mathrm{d}u_2 \mathrm{d}u_3 ... %\mathrm{d}u_{n-m}  \\
%= {({b_1}) }^{n-m} (v_1 -b_1). \nonumber 
%\end{IEEEeqnarray}

\begin{equation}
%\begin{IEEEeqnarray}{rCl}
E[u_1] =\int_0^{\frac{n-m+1}{n-m}b_1}  .... \int_0^{\frac{n-m+1}{n-m}b_1} \! (v_1 -b_1)  \mathrm{d}u_2 ... \mathrm{d}u_{n-m}  \\
= {(\frac{n-m+1}{n-m} b_1) }^{n-m} (v_1 -b_1). \nonumber 
%\end{IEEEeqnarray}
\end{equation}


Diffrentiating with respect to $b_1$ the optimal bid for application one is derived as follows:

%\begin{equation}
%\frac{\partial}{\partial b_1} ( {({b_1}) }^{n-m} (v_1 -b_1))=0.
%\end{equation}


\begin{equation}
\frac{\partial}{\partial b_1} ( {(\frac{n-m+1}{n-m} b_1) }^{n-m} (v_1 -b_1))=0.
\end{equation}

Which gives us the optimal bid for each application:
\begin{equation}
\Rightarrow b_1= \frac{n-m}{n-m+1}v_1
\end{equation}
\end{proof}
\end{comment}
%DIMAN COMMENTED PROOF FOR APPENDIX

%%%%%%%%%%%%%%%%%%%%%%%%%%%%%%%%%%%%%%%%%%%%%%%%%%%%%%%%%%%%%%%%%%%%%%%%%%%%
\begin{algorithm}[!tb] %\small
\DontPrintSemicolon % Some LaTeX compilers require you to use \dontprintsemicolon    instead
\KwIn{A bipartite Graph (U, V, E).}
\KwOut{Participation (YES) in an auction or Quit (NO).}
At time $t=nT$, assume that we have the same state in terms of resources.

For application $U_i \in U$, we similarly run the steps 2, 3, 4, 5 and 6 of Algorithm~\ref{algo:b} to find all estimated bids in next rounds based on its various phases. We have:
\[ b_{i}(t_{i,j},\vec{m})=\sum\limits_{j=1}^M {b_{i,j}(t)};\forall t_{i,j}>t \]
Also, we have the previous bids of the application $i$: 
\[ b_{i}(t_{i,j},\vec{m});\forall t_{i,j}<t \]

If $F_i \geq \sum\limits_{\forall t_{i,j}\neq t} {b_{i}(t_{i,j},\vec{m})}$, then YES and the application will participate in the auction. Otherwise NO, and the application will update the zero valuation for current round using Eq.~\ref{eq:belief}.

\caption{Budget Planning}
\label{algo:p}
\vspace{0\baselineskip}
\end{algorithm}
%%%%%%%%%%%%%%%%%%%%%%%%%%%%%%%%%%%%%%%%%%%%%%%%%%%%%%%%%%%%%%%%%%%%%%%%%%%%%
\indent Theorem~\ref{thm:neat}, states that whenever there is a single resource that users compete to get it with different valuation functions, the Nash equilibrium strategy profile for risk-neutral users is to bid $\frac{n-m}{n-m+1}v_i$. This term tends to the true value of the object when n is a large number. \\
\indent In case of more than one resource competition we derive Algorithm~\ref{algo:b} for heterogeneous resource assignment and will prove that it has a Nash equilibrium in the game. Algorithm~\ref{algo:b} uses Algorithm~\ref{algo:p} to do budget planning for our purpose. In the first step, all valuations are set to the solo-run of application's performance. Next, each application submits a partial bid for its first bottleneck resource. The partial bid should be larger than the price of the object which is initialized to zero at the beginning of the program. The applications only have the incentive to bid a value that is no more than the difference between the first and second bottleneck resource. Otherwise, it submits a smaller bid to the second bottleneck and gets the same revenue as paying more for the first bottleneck resource. In order to break the equal valuation function between two different applications, we use $\epsilon$ scaling such that at each iteration of the auction the prices should increase by a small number. The OS will set the resources' price with these partial bids, and find the minimum of the highest partial bids for each resource. The applications recurse for all the resources, and the total bid is the summation of the partial bids for each application. Then, the applications execute Algorithm~\ref{algo:p} to participate in the auction or not. Finally the participated applications with the bids higher than $B^{min}_j$ will get $j$-th resource.\\
%In addition, suppose we have 5 different memory bandwidth exposed to the applications. Each application gets a different performance benefit from different cache sizes and different memory bandwidth which is denoted in table  **. The applications need to submit their bids based on their performance benefits. 
%\begin{equation}
%\begin{split}
%\int_0^\frac{nb_1}{n-m}  .... \int_0^\frac{nb_1}{n-m} \! (\frac{v_1}{m} -b_1) \, \mathrm{d}u_2 %\mathrm{d}u_3 ... \mathrm{d}u_{n-m}= \\
%= {(\frac{nb_1}{n-m}) }^{n-m} (\frac{v_1}{m} -b_1). 
%\end{split}
%\end{equation}
%%%%%%%%%%%%%%%%%%%%%%%%%%%%%%%%%%%%%%%%%%%%%%%%%%%%%%%%%%%%%%%%%%%%%%%%%%%%
%\[ \frac{\partial}{\partial b_1} ({(\frac{nb_1}{n-m}) }^{n-m} (\frac{v_1}{m} -b_1))
%\newtheorem{defi}{Definition}
%\begin{defi}
%Let's assume each user $Ui$ in the system is defined as one vertice of a graph $G$ in the system and let each edge in the graph shows which subset of users can impact each others' performance and the associated weight of each edge show the cost function of how two users affect each others' performance in the system. Each edge has a weight function denoted by ${P1(n), P2(n), ... Pe(n)}$, where $e$ is the number of edges in Graph $G$. Let $A=A_1 \times A_2 \times ... \times A_n $ be the set of actions that each user can play. 
%\end{defi}
\indent The overhead of the auction for the auctioneer (the OS) is very negligible. The OS during the auction only sets the prices of the resources based on the received bids from the applications and gives the resources to the highest bids. So every $T$ seconds, the OS runs these two jobs, which adds a negligible overhead with respect to other tasks of the OS. Our approach also satisfies the following properties:
1) Individual rationality (IR): Applications' expected utility is non-negative because the amount of the bid cannot be beyond the sum of the difference of the valuations which is at most the highest valuation of the application.
2) Truthfulness: Applications cannot benefit from bidding other than their true valuation. By contradiction, if an application bids lower than the true value, there may be another application with a higher bid to take the resource. But we cannot guarantee the truthfulness in the case of collusion among applications.
3) Budget-balance: The whole payments from the applications are less than the OS revenue, which is trivial as we have only one seller which is the OS.
4) Economic efficiency: It has been shown in \cite{bertsekas1998network} that this assignment is optimal, but it doesn't mean it is economically efficient since we know that it depends on the applications' valuation which is sub-optimal.

%\vspace{-1\baselineskip}
\section{Case Studies} \label{Case_Studies}
%\subsection{A case study of the main processor (Xeon) and coprocessor (Xeon-Phi) congestion game}
\subsection{CPU Scale-up Scale-out Game}
The emerging high-performance computing applications lead to the advent of \textit{Intel Xeon Phi} co-processor, that when their highly parallel architecture is fully utilized, can run in order of magnitude more performance than the existing processor architectures. The \textit{Xeon Phi} co-processors are the first commercial product of Intel \textit{MIC} processors where the hardware architecture is exposed to the programmer to choose running the code on either \textit{Xeon} processor or \textit{Xeon Phi} co-processors. It is possible that, during the course of execution, either the processor or the co-processor get congested and the performance of the application degrades a lot. Therefore, making a decision to offload the most time-consuming part of the program on \textit{Xeon} or \textit{Xeon Phi} should be made online, based on the contention level.  In this section, we look at the case study of our auction-based model on decision making of running the application on the main or co-processor in a highly congested environment. \\
\indent The experimental results of this section are run on \textit{Stampede} cluster of \textit{Texas Advanced Computing Center}. Table~\ref{Table:Xeon} shows the comparison of \textit{Intel Xeon} and \textit{Xeon Phi} architectures which is used in this section. 
%%%%%%%%%%%%%%%%%%%%%%%%%%%%%%%%%%%%%%%%%%%%%%%%%%%%%%%%%%%%%%%%%%%%%%%%%%%
%%%%%%%%%%%%%%%%%%%%%%%%%%%%%%%%%%%%%%%%%%%%%%%%%%%%%%%%%%%%%%%%%%%%%%%%%%%
%%%%%%%%%%%%%%%%%%%%%%%%%%%%%%%%%%%%%%%%%%%%%%%%%%%%%%%%%%%%%%%%%%%%%%%%%%%
\begin{table}[!tb] 
\centering
\caption{The comparison of \textit{Intel Xeon} processor and \textit{Intel Xeon Phi} processor.}\label{Table:Xeon}
\begin{tabular}{|c|p{0.8in}|p{1in}|} 
\hline Processors & Xeon E5-2680 & Xeon Phi SE10P \\
\hline Cores/Sockets & 8/2 & 61/1 \\
\hline Clock Frequency & 2.7 GHz & 1.1 GHz  \\
\hline Memory & 32GB 8x4G 4-channels DDR3-1600MHz & 8GB GDDR5 \\
\hline L1 cache & 32 KB & 32 KB \\
\hline L2 cache & 256 KB & 512 KB \\
\hline L3 cache & 20 MB & - \\
\hline
\end{tabular}
\end{table}
%%%%%%%%%%%%%%%%%%%%%%%%%%%%%%%%%%%%%%%%%%%%%%%%%%%%%%%%%%%%%%%%%%%%%%%%%%%
%%%%%%%%%%%%%%%%%%%%%%%%%%%%%%%%%%%%%%%%%%%%%%%%%%%%%%%%%%%%%%%%%%%%%%%%%%%
%%%%%%%%%%%%%%%%%%%%%%%%%%%%%%%%%%%%%%%%%%%%%%%%%%%%%%%%%%%%%%%%%%%%%%%%%%%
It is observed that congestion has a significant impact on the performance of running the application on \textit{Xeon} and \textit{Xeon Phi} machines. Since most cloud computing machines are shared between thousands of users, the programmer not only should get the benefit of parallelism by offloading the most time-consuming part of the code to the larger number of low-performance cores (\textit{Xeon Phi}) but also should consider the congestion level (number of co-runners) in the system. To this end, we performed experiments on \textit{Stampede} clusters. We executed \textit{MiniGhost} application which is a part of \textit{Mantevo} project \cite{mantevo} which uses difference stencils to solve partial differential equations using numerical methods. The applications use the profiling utility functions at $t=0$ and during the course of execution update the utility function based on the observed performance on each core using Equation~\ref{eq:belief}. Then, they can revisit their previous action on running the code on either the processor or co-processor during run-time. \\
\indent Figure~\ref{Fig:congestion} shows the total execution time with respect to congestion we made in \textit{Xeon} and \textit{Xeon Phi}. In this experiment we ran the same problem size on a \textit{Xeon} and \textit{Xeon Phi} machine multiple times so that we could see the effect of load on the total execution time of our application. It was observed that with the same number of threads \textit{Xeon}'s performance degrades more than \textit{Xeon phi}. 
Next, we tried to change the application behavior using a congestion-aware game theoretic algorithm to offload the most time-consuming part of the application based on the performance behavior of applications. Figure~\ref{Fig:performance_over_time} shows the result of our game-theoretic model during the execution time. It is observed that during the course of execution, the applications change their strategy on either choosing the main processor or the co-processor and all applications' performance converge to an equilibrium point where applications don't want to change their strategy. \\
\indent Furthermore, it is shown that CARMA can bring in up to 106.6\% improvement in total execution time of applications compared to static approach when the number of co-runners is six. The performance improvement would be significant when the number of co-runners increase. Figure~\ref{Fig:Perfomance_Comparison} shows the performance comparison of CARMA and static approach which does not consider the congestion dynamism in the system and the decision is only made based on the parallelism level in the code. 
%%%%%%%%%%%%%%%%%%%%%%%%%%%%%%%%%%%%%%%%%%%%%%%%%%%%%%%%%%%%%%%%%%%%%%%%%%%
%%%%%%%%%%%%%%%%%%%%%%%%%%%%%%%%%%%%%%%%%%%%%%%%%%%%%%%%%%%%%%%%%%%%%%%%%%%
%%%%%%%%%%%%%%%%%%%%%%%%%%%%%%%%%%%%%%%%%%%%%%%%%%%%%%%%%%%%%%%%%%%%%%%%%%%
\begin{figure}[!tb]
\centering
%\includegraphics[height=3in, width=2.5in]{NodeArchs2.pdf}
\includegraphics[height=1.5in, width=3.5in]{Images/Xeon.pdf} % diman.pdf
%\epsfig{file=Dataset.eps, height=2.5in, width=3in}
\caption{Congestion effect on \textit{Xeon} and \textit{Xeon Phi} machines.}
\label{Fig:congestion}
\vspace{-1\baselineskip}
\end{figure}
%%%%%%%%%%%%%%%%%%%%%%%%%%%%%%%%%%%%%%%%%%%%%%%%%%%%%%%%%%%%%%%%%%%%%%%%%%%
%%%%%%%%%%%%%%%%%%%%%%%%%%%%%%%%%%%%%%%%%%%%%%%%%%%%%%%%%%%%%%%%%%%%%%%%%%%
\begin{figure}[!tb]
\centering
%\includegraphics[height=3in, width=2.5in]{NodeArchs2.pdf}
\includegraphics[height=1.5in, width=3.5in]{Images/Game.pdf}%Game_during_time.pdf
%\epsfig{file=Dataset.eps, height=2.5in, width=3in}
\caption{Performance of 6 instances of applications during the execution time for our proposed game model.}\label{Fig:performance_over_time}
\end{figure}
%%%%%%%%%%%%%%%%%%%%%%%%%%%%%%%%%%%%%%%%%%%%%%%%%%%%%%%%%%%%%%%%%%%%%%%%%%%
%%%%%%%%%%%%%%%%%%%%%%%%%%%%%%%%%%%%%%%%%%%%%%%%%%%%%%%%%%%%%%%%%%%%%%%%%%%
\begin{figure}[!tb]
\centering
%\includegraphics[height=3in, width=1.5in]{NodeArchs2.pdf}
\includegraphics[height=1.5in, width=3.5in]{Images/Congestion.pdf}  %Congestion_aware.pdf
%\epsfig{file=Dataset.eps, height=2.5in, width=3in}
\caption{Performance comparison of congestion-aware schedule versus static schedule.}
\label{Fig:Perfomance_Comparison}
\vspace{-1\baselineskip}
\end{figure}
%%%%%%%%%%%%%%%%%%%%%%%%%%%%%%%%%%%%%%%%%%%%%%%%%%%%%%%%%%%%%%%%%%%%%%%%%%%
%%%%%%%%%%%%%%%%%%%%%%%%%%%%%%%%%%%%%%%%%%%%%%%%%%%%%%%%%%%%%%%%%%%%%%%%%%%
%%%%%%%%%%%%%%%%%%%%%%%%%%%%%%%%%%%%%%%%%%%%%%%%%%%%%%%%%%%%%%%%%%%%%%%%%%%
\subsection{A Case Study of Private and shared cache game}
One of the challenging problems in \textit{CMP} resource management systems is whether applications benefit from a shared large last level cache or an isolated private cache. We evaluated CARMA's performance, on a 10MB LLC shown in Figure~\ref{Fig:cache_hierarchy}, where 2MB, 1MB, 512kB, 256kB and 128kB levels of LLC can potentially be shared between 16, 8, 4, 2 and 1 applications respectively, the cache levels have 16, 8, 4, 2, and 1 ways. Table~\ref{Table:Workloads} summarizes the studies workloads and their characteristics, including miss per kilo instructions (\textit{MPKI}), memory bandwidth usage, and IPC. We use applications from \textit{Spec 2006} benchmark suite \cite{Spec:website}. We use \textit{Gem5} full system simulator in our experiment ~\cite{binkert2011gem5, Gem5:website}. Table~\ref{Table:Experimental_Set_Up} shows the experimental setup in our experiments.\\
\indent To evaluate the performance of our proposed approach we use utility functions for different number of cache ways shown in Figure~\ref{fig:Cache_IPC}. These utility functions at the start of the execution can be found using either profiling techniques or stack distance profile \cite{kim2004fair, suh2002new, suh2004dynamic} of applications assuming there are no co-runners in the system. Next, during run-time, the applications can update their utility functions based on Equation~\ref{eq:belief}. Therefore, there is a learning phase where applications learn about the state of the system and update the utilities accordingly. The stack distance profile indicates how many more cache misses will be added if the application has less number of ways in the cache. Based on the stack distance profile, the applications can update their utility function and bid for the next iteration of the auction if they like to change their allocation. Next, we bring an example of the auction for one time step of the game. This time step can be repeated once an application arrives or leaves the system or when an application's phase changes during run-time. However, in case of one application's phase change or arriving or leaving the system, the algorithm reaches the optimal assignment in much fewer iterations since all other assignments are fixed and a few applications would be affected.\\
%%%%%%%%%%%%%%%%%%%%%%%%%%%%%%%%%%%%%%%%%%%%%%%%%%%%%%%%%%%%%%%%%%%%%%%%%%%
\begin{figure}[!tb]
\centering
%\includegraphics[height=3in, width=2.5in]{NodeArchs2.pdf}
\includegraphics[height=1.5in, width=3.3in]{Images/Cache_Hierarchy_2.pdf}
%\epsfig{file=Dataset.eps, height=2.5in, width=3in}
\caption{The proposed last level cache hierarchy model.}\label{Fig:cache_hierarchy}
\end{figure}
%%%%%%%%%%%%%%%%%%%%%%%%%%%%%%%%%%%%%%%%%%%%%%%%%%%%%%%%%%%%%%%%%%%%%%%%%%%
%%%%%%%%%%%%%%%%%%%%%%%%%%%%%%%%%%%%%%%%%%%%%%%%%%%%%%%%%%%%%%%%%%%%%%%%%
\begin{table}[!tb] %\scriptsize
\centering
\caption{Experimental Setup.}
\label{Table:Experimental_Set_Up}
\begin{tabular}{|c|p{1.5in}|} 
\hline Processors & Single threaded with private L1 instruction and data caches \\
\hline Frequency & 1GHz \\
\hline L1 Private ICache & 32 kB, 64-byte lines, 4-way associative\\
\hline L1 Private DCache & 32 kB, 64-byte lines, 4-way associative \\
\hline L2 Shared Cache & 128 kb-2 MB, 64-byte lines, 16-way associative \\
\hline RAM & 12 GB \\
\hline
\end{tabular}
\end{table}
%%%%%%%%%%%%%%%%%%%%%%%%%%%%%%%%%%%%%%%%%%%%%%%%%%%%%%%%%%%%%%%%%%%%%%%%%
\begin{table}[!tb] 
\centering
\caption{Evaluated workloads.}
\label{Table:Workloads}
\begin{tabular}{p{0.7cm} p{1.5cm} p{1cm} p{1.7cm} p{1cm} }
\hline
%\begin{tabular}{|l|c|c|c|} \hline
{\bf \#} & \bf Benchmark & MPKI & Memory BW & IPC  \\
\hline 
{\bf 1} & astar & 1.319 & 373 MB/s & 2.057 \\
{\bf 2} & bwaves & 10.47 & 1715 MB/s & 0.661 \\
{\bf 3} & bzip2 & 3.557 & 1194 MB/s & 1.367 \\ 
{\bf 4} & dealII &  0.935 & 307 MB/s & 2.107 \\
{\bf 5} & GemsFDTD & 0.004 & 2.19 MB/s & 2.023 \\
{\bf 6} & hmmer & 2.113 & 1547 MB/s & 2.861 \\
{\bf 7} & lbm & 19.287 & 3954 MB/s & 0.533 \\
{\bf 8} & leslie3d & 8.469 & 1942 MB/s & 1.297 \\
{\bf 9} & libquantum & 10.388 & 1589 MB/s & 0.531 \\
{\bf 10} & mcf & 16.93 & 820 MB/s & 0.073 \\
{\bf 11} & namd & 0.051 & 20.32 MB/s & 2.362\\
{\bf 12} & omnetpp & 10.34 & 1147 MB/s & 0.504 \\
{\bf 13} & sjeng & 0.375 & 139.2 MB/s & 1.403 \\
{\bf 14} & soplex & 4.672 & 390.8 MB/s & 0.513 \\
{\bf 15} & sphinx3 & 0.349 & 202.8 MB/s & 2.223 \\
{\bf 16} & streamL & 31.682 & 3619 MB/s & 0.581 \\
{\bf 17} & tonto & 0.260 & 107 MB/s & 2.036 \\
{\bf 18} & xalancbmk & 12.703 & 1200 MB/s & 0.558 \\
%\hline  
\hline
\end{tabular}
\end{table}
%%%%%%%%%%%%%%%%%%%%%%%%%%%%%%%%%%%%%%%%%%%%%%%%%%%%%%%%%%%%%%%%%%%%%%%%%%
\begin{comment}
Assume $n$ different applications denoted by ${U1, U2, .. Un}$ with different cache benefits which may affect each other with different cost functions. Figure 1 shows the different applications which are the vertices of the graph with their impact on each other which are the weights of edges in the graph. If two vertices are not connected in the graph, it means that they would not affect each others' performance. For example, one application is CPU bound and does not benefit from larger memory bandwith and the other is memory bound and does not benefit from having more CPU capacity. So different applications affect each other's performance with different coefficients. 
\end{comment}
%%%%%%%%%%%%%%%%%%%%%%%%%%%%%%%%%%%%%%%%%%%%%%%%%%%%%%%%%%%%%%%%%%%%%%%%%%%
%%%%%%%%%%%%%%%%%%%%%%%%%%%%%%%%%%%%%%%%%%%%%%%%%%%%%%%%%%%%%%%%%%%%%%%%%%%
%%%%%%%%%%%%%%%%%%%%%%%%%%%%%%%%%%%%%%%%%%%%%%%%%%%%%%%%%%%%%%%%%%%%%%%%%%%
\begin{figure*}[!tb]
\centering
\includegraphics[height=3.5in, width=6.5in]{Images/Cache_IPC_v2.pdf}
\caption{IPC for different size of LLC.}\label{fig:Cache_IPC}  
\end{figure*}
%%%%%%%%%%%%%%%%%%%%%%%%%%%%%%%%%%%%%%%%%%%%%%%%%%%%%%%%%%%%%%%%%%%%%%%%%%%
%%%%%%%%%%%%%%%%%%%%%%%%%%%%%%%%%%%%%%%%%%%%%%%%%%%%%%%%%%%%%%%%%%%%%%%%%%%
%%%%%%%%%%%%%%%%%%%%%%%%%%%%%%%%%%%%%%%%%%%%%%%%%%%%%%%%%%%%%%%%%%%%%%%%%%%
\indent \textbf{Example:} As an example, suppose we have 5 different applications and 5 different cache levels with different capacities of 128KB, 256KB, 512KB, 1MB and 2MB. In addition, suppose the 128kB cache level can not accomodate more than one application and 256kB cache can accomodate 2 applications, 512kB level can have 4 applications, 1MB cache can have 8 applications and 2MB cache can have at most 16 applications. Let's assume the following matrix be the utility function of each application on each cache level. \\
\indent Some applications may get better utility from smaller cache space since they are less congested and since these applications have low data locality, moving to larger cache spaces not only does not increase their performance but also degrades the performance by evicting other applications from the cache and making contention on the memory bandwidth which is a more vital resource for them \footnote{\textit{libquantum}, \textit{streamL}, \textit{sphinx3}, \textit{lbm} and \textit{mcf} are examples of such applications.}.  \\
\begin{equation}
M = \bordermatrix{~ & 1way & 2way & 4way & 8way & 16way \cr
  App1 & 1.9 & 1.7 & 1.5 & 1 & 0.9 \cr
  App2 & 1.6 & 1.3 & 1.1 & 0.8 & 0.7 \cr
  App3 & 1.4 & 1.0 & 0.6 & 0.5 & 0.4 \cr
  App4 & 0.3 & 0.6 & 0.9 & 1.2 & 1.4 \cr
  App5 & 0.7 & 0.8 & 1.1 & 1.4 & 1.7 \cr}
\end{equation}
%%%%%%%%%%%%%%%%%%%%%%%%%%%%%%%%%%%%%%%%%%%%%%%%%%%%%%%%%%%%%%%%%%%%%%%%%%%
%%%%%%%%%%%%%%%%%%%%%%%%%%%%%%%%%%%%%%%%%%%%%%%%%%%%%%%%%%%%%%%%%%%%%%%%%%%
In the first iteration of the bidding, the first 3 applications bid for the most profitable resource which is 128kB cache and they submit a bid equal to the difference of profit between the first and the second most profitable resource. Therefore, the first application, submits 0.2 bid to 128kb and the second application submits 0.3 and the third application submits 0.4. Since only one of the players can acquire the 128kB cache space, the first application will get it. The 4th and 5th application compete for 2MB cache space and they both get it with the sum bid of both which is 0.5. In the next round, the prices will be updated and since applications 2 and 3 don't have any cache assignment compete for the 256kB cache space and each bid 0.2 which is the difference between 1.7 and 1.5 and 1.3 and 1.1 in the performance matrix accordingly. Since the second level cache can accommodate both applications the price will be updated and the minimum bidding price for someone to get this cache level is updated to the minimum bid of both which is 0.2. Therefore, if some application bid more than 0.2 it can acquire the resource and the application with the smallest bid has to resubmit the bid to acquire the resource. Figure~\ref{fig:first_round}, ~\ref{fig:second_round}, and ~\ref{fig:third_round} show the bidding steps and the prices and minimum price of bidding accordingly. As seen from the figures, the auction terminates in three iterations when there exist five applications. 
%%%%%%%%%%%%%%%%%%%%%%%%%%%%%%%%%%%%%%%%%%%%%%%%%%%%%%%%%%%%%%%%%%%%%%%%%%%
%%%%%%%%%%%%%%%%%%%%%%%%%%%%%%%%%%%%%%%%%%%%%%%%%%%%%%%%%%%%%%%%%%%%%%%%%%%
%%%%%%%%%%%%%%%%%%%%%%%%%%%%%%%%%%%%%%%%%%%%%%%%%%%%%%%%%%%%%%%%%%%%%%%%%%% 
\begin{figure*}[!htb]
        \centering
        \begin{subfigure}[b]{0.28\textwidth} %//0.28 bood
                \includegraphics[width=\textwidth]{Images/bid0.pdf}
                \caption{first round.}
                \label{fig:first_round}
        \end{subfigure}%
        ~ %add desired spacing between images, e. g. ~, \quad, \qquad etc.
          %(or a blank line to force the subfigure onto a new line)
        \begin{subfigure}[b]{0.28\textwidth}
                \includegraphics[width=\textwidth]{Images/bid2.pdf}
                \caption{second round.}
                \label{fig:second_round}
        \end{subfigure}
        ~ %add desired spacing between images, e. g. ~, \quad, \qquad etc.
          %(or a blank line to force the subfigure onto a new line)
        \begin{subfigure}[b]{0.28\textwidth}
                \includegraphics[width=\textwidth]{Images/bid3.pdf}
                \caption{third round.}
                \label{fig:third_round}
        \end{subfigure}  

                \caption{Cache allocation, a) first round, b) second round and c) third round of bidding.}\label{fig:Auction_rounds}    
       % \vspace{-2\baselineskip}
\end{figure*}
%%%%%%%%%%%%%%%%%%%%%%%%%%%%%%%%%%%%%%%%%%%%%%%%%%%%%%%%%%%%%%%%%%%%%%%%%%%
%%%%%%%%%%%%%%%%%%%%%%%%%%%%%%%%%%%%%%%%%%%%%%%%%%%%%%%%%%%%%%%%%%%%%%%%%%%
\vspace{-1\baselineskip}
\subsection{A Case Study for Hybrid Cache Game}
\indent In hybrid cache game, each cache partition can have a cluster of applications. We use different mixes of 4 to 16 applications from \textit{Spec 2006} to evaluate the performance of our proposed approach compared to others. To evaluate our approach, we selected the state-of-the-art centralized cache partitioner~\cite{8327002} (KPart) as a competitor which aims at maximizing the global IPC speedup. CARMA uses multi-resource valuations, so each application can have any criteria to maximize its payoff. In order to provide a fair comparison with our approach, we use IPC speedup as the optimization goal for all applications.\\% having the profile of the applications and similar bundled resources for the participants.\\ 
\indent Figure~\ref{fig:IPC_mix} shows the normalized throughput of 10 different mix of applications \cite{srikantaiah2011morphcache}, using CARMA, KPart~\cite{8327002}, equal separate cache partitioning and completely shared cache space after convergence. Furthermore, Figure~\ref{fig:scalability} shows the scalability of our proposed algorithm. When the number of co-runners increases from 2 to 16, the performance improves without any need to track each applications' performance in a central module. Having full information about applications' profiles, CARMA outperforms the other centralized competitors, when the number of the applications increases.\\
%The results of throughput and performance compared to DABMFT are pretty the same except little changes because of partitioning is a little sensitive to initial state, but the optimal valuation and allocation after KPart partitioning are the same. It is why that we haven't shown it in the figures not to confuse the other results. However, DABMFT calculation is more complex for computer architecture purpose, since it is designed for multiple applications with multiple sellers.
\indent Since KPart is a centralized (not an auction-based) approach, we assume that it has an unlimited budget. The budget matters in CARMA. We setup another experiment to track the variations of the normalized throughput versus the normalized budget for a mix of 16 applications. Figure~\ref{fig:Relative_Thr} shows that the throughput of CARMA, is very sensitive to the budget. The throughput changes dramatically at some inflection point, and at the end it is saturated but higher than KPart.
%%%%%%%%%%%%%%%%%%%%%%%%%%%%%%%%%%%%%%%%%%%%%%%%%%%%%%%%%%%%%%%%%%%%%%%%%%%
\begin{figure}[!tb]
\centering
%\includegraphics[height=3in, width=1.5in]{NodeArchs2.pdf}
\includegraphics[height=1.5in, width=3.5in]{Images/IPC.pdf}
%\epsfig{file=Dataset.eps, height=2.5in, width=3in}
\caption{Throughput of a shared, solo, CARMA and KPart cache allocation schemes.}
\label{fig:IPC_mix}
\end{figure}
%%%%%%%%%%%%%%%%%%%%%%%%%%%%%%%%%%%%%%%%%%%%%%%%%%%%%%%%%%%%%%%%%%%%%%%%%%%
%%%%%%%%%%%%%%%%%%%%%%%%%%%%%%%%%%%%%%%%%%%%%%%%%%%%%%%%%%%%%%%%%%%%%%%%%%%
\begin{figure}[!tb]
\centering
%\includegraphics[height=3in, width=1.5in]{NodeArchs2.pdf}
\includegraphics[height=1.5in, width=3.5in]{Images/Scalability.pdf}
%\epsfig{file=Dataset.eps, height=2.5in, width=3in}
\caption{Performance improvement of CARMA and KPart for different number of applications with respect to shared LLC for the case study of cache congestion game.}
\label{fig:scalability}
\end{figure}
%%%%%%%%%%%%%%%%%%%%%%%%%%%%%%%%%%%%%%%%%%%%%%%%%%%%%%%%%%%%%%%%%%%%%%%%%%%
%%%%%%%%%%%%%%%%%%%%%%%%%%%%%%%%%%%%%%%%%%%%%%%%%%%%%%%%%%%%%%%%%%%%%%%%%%%
\begin{figure}[!tb]
\centering
%\includegraphics[height=3in, width=1.5in]{NodeArchs2.pdf}
\includegraphics[height=1.5in, width=2.5in]{Images/Thr_relative-eps-converted-to.pdf}
%\epsfig{file=Dataset.eps, height=2.5in, width=3in}
\caption{Relative throughput w.r.t applications' normalized budget.}
\label{fig:Relative_Thr}
\vspace{-1\baselineskip}
\end{figure}
%%%%%%%%%%%%%%%%%%%%%%%%%%%%%%%%%%%%%%%%%%%%%%%%%%%%%%%%%%%%%%%%%%%%%%%%%%%
%%%%%%%%%%%%%%%%%%%%%%%%%%%%%%%%%%%%%%%%%%%%%%%%%%%%%%%%%%%%%%%%%%%%%%%%%%%
\begin{comment}
\subsection{Experimental setup}


\newtheorem{defin}{Definition}
\begin{defin}
Let's assume each user $Ui$ in the system is defined as one vertice of a graph $G$ in the system and let each edge in the graph shows which subset of users can impact each others' performance and the associated weight of each edge show the cost function of how two users affect each others' performance in the system. Each edge has a weight function denoted by ${P1(n), P2(n), ... Pe(n)}$, where $e$ is the number of edges in Graph $G$. Let $A=A_1 \times A_2 \times ... \times A_n $ be the set of actions that each user can play. 
\end{defin}

This representation of the cache congestion game has a space complexity of which is exponential in terms of a larges degree in the graph. 
If the number of users and the action set is polynomially bounded then the game has a polynomial representation and  




For a simple illustrative example assume two applications which would affect each other's performance with $f_i$. The payoff table of a subset of users who can affect each other can be shown in Table 1. We can easily extend the game for $n$ users using $f_1$ and $f_2$ to be a function of the number of users which can impact each other.
We used Gem5, full system simulator running Spec OMP benchmarks to find how different applications sharing a specific amount of shared cache impact each other's performance. 
The applications which we did experiments on are listed in Table 2.
\end{comment}  
%\vspace{-1\baselineskip}
\section{Related Work} \label{Related_works}
With rapid improvement in computer technology, more and more cores are embedded in a single chip and applications competing for a shared resource is becoming common. On the one hand, managing scheduling of shared resources for a large number of applications is challenging in a sense that the operating system doesn't know what is the performance metric for each application. But on the other hand, the operating system has a global view of the whole state of the system and can guide applications on choosing the shared resources.\\ 
\indent There have been several works, for managing the shared cache in multi-core systems. Qureshi et al. \cite{qureshi2006utility} showed that assigning more cache space to applications with more cache utility does not always lead to better performance since there exist applications with very low cache reuse which may have very high cache utilization. \\
\indent Several software and hardware approaches have been proposed to find the optimal partitioning of cache space for different applications \cite{zhuravlev2010addressing}. However, most of these approaches use brute force search of all possible combinations to find the best cache partitioning in runtime or introduce a lot of overhead. There have been some approaches which use binary search to reduce searching all possible combinations \cite{kim2004fair, lin2008gaining, tam2009rapidmrc}. But none of these methods are scalable for the future many-core processor designs.\\
\indent There exists prior game-theoretic approaches designing a centralized scheduling framework that aims at a fair optimization of applications' utility \cite{zahedi2014ref, llull2017cooper, ghodsi2011dominant, zahedi2015sharing, fan2016computational}. Zahedi et al. in REF \cite{zahedi2014ref, zahedi2015sharing} use the Cobb-Douglas production function as a fair allocator for cache and memory bandwidth. They show that the Cobb-Douglas function provides game-theoretic properties such as sharing incentives, envy-freedom, and Pareto efficiency. But their approach is still centralized and spatially divides the shared resources to enforce a fair near-optimal policy sacrificing the performance. In their approach, the centralized scheduler assumes all applications have the same priority for cache and memory bandwidth, while we do not have any assumption on this. Further, our auction-based resource allocation can be used for any number of resources and any priority for each application and the centralized scheduler does not need to have a global knowledge of these priorities.  \\
\indent Ghodsi et al. in DRF \cite{ghodsi2011dominant} use another centralized fair policy to maximize the dominant resource utilization. But in practice, it is not possible to clone any number of instances of each resource. %  the underlying scenario cloning the instances is very limited or superficial for practical purposes. 
Cooper \cite{llull2017cooper} enhances REF to capture colocated applications fairly, but it only addresses the special case of having two sets of applications with matched resources. Fan et al. \cite{fan2016computational} exploits computational sprinting architecture to improve task throughput assuming a class of applications where boosting their performance by increasing the power. \\
\indent While all prior works use a centralized scheduling that provides fairness and assumes the same utility function for all, co-runners might have completely diverse needs and it is not efficient to use the same fairness/performance policy across them. 
Our auction-based resource scheduling provides scalability since individual applications compete for the shared resources based on their utility and the burden of decision making is removed from the central scheduler. We believe that future CMPs should move toward a more decentralized approach which is more scalable and provides a fair allocation of resources based on the applications' needs. \\ 
%Providing the scalability of the system is getting better if we make the individual applications our self-administrators, and we can remove most of the performance-restricting policies such as fairness constraints from the centralized decision-maker. \\
\indent Auction theory which is a subfield of economics has recently been used as a tool to solve large-scale resource assignment in cloud computing \cite{krishna2009auction, parsons2011auctions}. In an auction process, the buyers submit bids to get the commodities and sellers want to sell their commodities with the maximum price as possible. Also auction-based allocators ~\cite{Zhang2017DAB,8004498} are multi-buyers with multi-seller but there is only one resource to bid. So, they cannot be used for our purpose, since we have only one seller with multiple bundled resources. That is why we choose a simpler related scheme for a computer architecture to get higher performance with lower transactions and auctions. \\
\indent Our auction-based algorithm is inspired by work of Bertsekas \cite{bertsekas1998network} that uses an auction-based approach for network flow problems. Our algorithm is an extension of local assignment problem proposed by Bertsekas et al. that has been shown to converge to the global assignment within a linear approximation.
%\textcolor{red}{Not Complete yet}
%\vspace{-1\baselineskip}
\vspace{-1\baselineskip}
\section{Conclusion} 
\label{Conclusion}
%\vspace{-1\baselineskip}
%\begin{normalsize}
This paper proposes a distributed resource allocation approach for large-scale servers. The traditional resource management system is not scalable, especially when tracking the application's dynamic behavior. The main cause of this complexity is the centralized decision making which leads to higher time and space complexity. With increasing number of cores per chip, the scalability of assigning different resources to different applications becomes more challenging in future generation CMP systems. In addition, diversity in application's need makes a single objective function inefficient to get an optimal and fair performance metric. We introduce a framework to map the allocation problem to the known auction economy model where the applications compete for the shared resources based on their utility metrics of interest.
%\end{normalsize}
%\textcolor{red}{Not Complete yet}\\
%\vspace{-1\baselineskip}

%\vspace{-0.5\baselineskip}
%\IEEEraisesectionheading{\section{Introduction}}

\IEEEPARstart{V}{ision} system is studied in orthogonal disciplines spanning from neurophysiology and psychophysics to computer science all with uniform objective: understand the vision system and develop it into an integrated theory of vision. In general, vision or visual perception is the ability of information acquisition from environment, and it's interpretation. According to Gestalt theory, visual elements are perceived as patterns of wholes rather than the sum of constituent parts~\cite{koffka2013principles}. The Gestalt theory through \textit{emergence}, \textit{invariance}, \textit{multistability}, and \textit{reification} properties (aka Gestalt principles), describes how vision recognizes an object as a \textit{whole} from constituent parts. There is an increasing interested to model the cognitive aptitude of visual perception; however, the process is challenging. In the following, a challenge (as an example) per object and motion perception is discussed. 



\subsection{Why do things look as they do?}
In addition to Gestalt principles, an object is characterized with its spatial parameters and material properties. Despite of the novel approaches proposed for material recognition (e.g.,~\cite{sharan2013recognizing}), objects tend to get the attention. Leveraging on an object's spatial properties, material, illumination, and background; the mapping from real world 3D patterns (distal stimulus) to 2D patterns onto retina (proximal stimulus) is many-to-one non-uniquely-invertible mapping~\cite{dicarlo2007untangling,horn1986robot}. There have been novel biology-driven studies for constructing computational models to emulate anatomy and physiology of the brain for real world object recognition (e.g.,~\cite{lowe2004distinctive,serre2007robust,zhang2006svm}), and some studies lead to impressive accuracy. For instance, testing such computational models on gold standard controlled shape sets such as Caltech101 and Caltech256, some methods resulted $<$60\% true-positives~\cite{zhang2006svm,lazebnik2006beyond,mutch2006multiclass,wang2006using}. However, Pinto et al.~\cite{pinto2008real} raised a caution against the pervasiveness of such shape sets by highlighting the unsystematic variations in objects features such as spatial aspects, both between and within object categories. For instance, using a V1-like model (a neuroscientist's null model) with two categories of systematically variant objects, a rapid derogate of performance to 50\% (chance level) is observed~\cite{zhang2006svm}. This observation accentuates the challenges that the infinite number of 2D shapes casted on retina from 3D objects introduces to object recognition. 

Material recognition of an object requires in-depth features to be determined. A mineralogist may describe the luster (i.e., optical quality of the surface) with a vocabulary like greasy, pearly, vitreous, resinous or submetallic; he may describe rocks and minerals with their typical forms such as acicular, dendritic, porous, nodular, or oolitic. We perceive materials from early age even though many of us lack such a rich visual vocabulary as formalized as the mineralogists~\cite{adelson2001seeing}. However, methodizing material perception can be far from trivial. For instance, consider a chrome sphere with every pixel having a correspondence in the environment; hence, the material of the sphere is hidden and shall be inferred implicitly~\cite{shafer2000color,adelson2001seeing}. Therefore, considering object material, object recognition requires surface reflectance, various light sources, and observer's point-of-view to be taken into consideration.


\subsection{What went where?}
Motion is an important aspect in interpreting the interaction with subjects, making the visual perception of movement a critical cognitive ability that helps us with complex tasks such as discriminating moving objects from background, or depth perception by motion parallax. Cognitive susceptibility enables the inference of 2D/3D motion from a sequence of 2D shapes (e.g., movies~\cite{niyogi1994analyzing,little1998recognizing,hayfron2003automatic}), or from a single image frame (e.g., the pose of an athlete runner~\cite{wang2013learning,ramanan2006learning}). However, its challenging to model the susceptibility because of many-to-one relation between distal and proximal stimulus, which makes the local measurements of proximal stimulus inadequate to reason the proper global interpretation. One of the various challenges is called \textit{motion correspondence problem}~\cite{attneave1974apparent,ullman1979interpretation,ramachandran1986perception,dawson1991and}, which refers to recognition of any individual component of proximal stimulus in frame-1 and another component in frame-2 as constituting different glimpses of the same moving component. If one-to-one mapping is intended, $n!$ correspondence matches between $n$ components of two frames exist, which is increased to $2^n$  for one-to-any mappings. To address the challenge, Ullman~\cite{ullman1979interpretation} proposed a method based on nearest neighbor principle, and Dawson~\cite{dawson1991and} introduced an auto associative network model. Dawson's network model~\cite{dawson1991and} iteratively modifies the activation pattern of local measurements to achieve a stable global interpretation. In general, his model applies three constraints as it follows
\begin{inlinelist}
	\item \textit{nearest neighbor principle} (shorter motion correspondence matches are assigned lower costs)
	\item \textit{relative velocity principle} (differences between two motion correspondence matches)
	\item \textit{element integrity principle} (physical coherence of surfaces)
\end{inlinelist}.
According to experimental evaluations (e.g.,~\cite{ullman1979interpretation,ramachandran1986perception,cutting1982minimum}), these three constraints are the aspects of how human visual system solves the motion correspondence problem. Eom et al.~\cite{eom2012heuristic} tackled the motion correspondence problem by considering the relative velocity and the element integrity principles. They studied one-to-any mapping between elements of corresponding fuzzy clusters of two consecutive frames. They have obtained a ranked list of all possible mappings by performing a state-space search. 



\subsection{How a stimuli is recognized in the environment?}

Human subjects are often able to recognize a 3D object from its 2D projections in different orientations~\cite{bartoshuk1960mental}. A common hypothesis for this \textit{spatial ability} is that, an object is represented in memory in its canonical orientation, and a \textit{mental rotation} transformation is applied on the input image, and the transformed image is compared with the object in its canonical orientation~\cite{bartoshuk1960mental}. The time to determine whether two projections portray the same 3D object
\begin{inlinelist}
	\item increase linearly with respect to the angular disparity~\cite{bartoshuk1960mental,cooperau1973time,cooper1976demonstration}
	\item is independent from the complexity of the 3D object~\cite{cooper1973chronometric}
\end{inlinelist}.
Shepard and Metzler~\cite{shepard1971mental} interpreted this finding as it follows: \textit{human subjects mentally rotate one portray at a constant speed until it is aligned with the other portray.}



\subsection{State of the Art}

The linear mapping transformation determination between two objects is generalized as determining optimal linear transformation matrix for a set of observed vectors, which is first proposed by Grace Wahba in 1965~\cite{wahba1965least} as it follows. 
\textit{Given two sets of $n$ points $\{v_1, v_2, \dots v_n\}$, and $\{v_1^*, v_2^* \dots v_n^*\}$, where $n \geq 2$, find the rotation matrix $M$ (i.e., the orthogonal matrix with determinant +1) which brings the first set into the best least squares coincidence with the second. That is, find $M$ matrix which minimizes}
\begin{equation}
	\sum_{j=1}^{n} \vert v_j^* - Mv_j \vert^2
\end{equation}

Multiple solutions for the \textit{Wahba's problem} have been published, such as Paul Davenport's q-method. Some notable algorithms after Davenport's q-method were published; of that QUaternion ESTimator (QU\-EST)~\cite{shuster2012three}, Fast Optimal Attitude Matrix \-(FOAM)~\cite{markley1993attitude} and Slower Optimal Matrix Algorithm (SOMA)~\cite{markley1993attitude}, and singular value decomposition (SVD) based algorithms, such as Markley’s SVD-based method~\cite{markley1988attitude}. 

In statistical shape analysis, the linear mapping transformation determination challenge is studied as Procrustes problem. Procrustes analysis finds a transformation matrix that maps two input shapes closest possible on each other. Solutions for Procrustes problem are reviewed in~\cite{gower2004procrustes,viklands2006algorithms}. For orthogonal Procrustes problem, Wolfgang Kabsch proposed a SVD-based method~\cite{kabsch1976solution} by minimizing the root mean squared deviation of two input sets when the determinant of rotation matrix is $1$. In addition to Kabsch’s partial Procrustes superimposition (covers translation and rotation), other full Procrustes superimpositions (covers translation, uniform scaling, rotation/reflection) have been proposed~\cite{gower2004procrustes,viklands2006algorithms}. The determination of optimal linear mapping transformation matrix using different approaches of Procrustes analysis has wide range of applications, spanning from forging human hand mimics in anthropomorphic robotic hand~\cite{xu2012design}, to the assessment of two-dimensional perimeter spread models such as fire~\cite{duff2012procrustes}, and the analysis of MRI scans in brain morphology studies~\cite{martin2013correlation}.

\subsection{Our Contribution}

The present study methodizes the aforementioned mentioned cognitive susceptibilities into a cognitive-driven linear mapping transformation determination algorithm. The method leverages on mental rotation cognitive stages~\cite{johnson1990speed} which are defined as it follows
\begin{inlinelist}
	\item a mental image of the object is created
	\item object is mentally rotated until a comparison is made
	\item objects are assessed whether they are the same
	\item the decision is reported
\end{inlinelist}.
Accordingly, the proposed method creates hierarchical abstractions of shapes~\cite{greene2009briefest} with increasing level of details~\cite{konkle2010scene}. The abstractions are presented in a vector space. A graph of linear transformations is created by circular-shift permutations (i.e., rotation superimposition) of vectors. The graph is then hierarchically traversed for closest mapping linear transformation determination. 

Despite of numerous novel algorithms to calculate linear mapping transformation, such as those proposed for Procrustes analysis, the novelty of the presented method is being a cognitive-driven approach. This method augments promising discoveries on motion/object perception into a linear mapping transformation determination algorithm.



%\vspace{-0.5\baselineskip}
%\section{Motivation and Background} \label{Motivation}
\subsection{Motivation}
Different applications have different resource constraint with respect to CPU, memory, and bandwidth usage. Having a single resource manager for all existing resources and users in the system result in inefficiencies since it is not scalable and the operating system may not have enough information about application's needs. For example, traditional LRU-based cache strategy uses cache utilization as a metric to give larger cache size to the applications which have higher utilization and lower cache size to the applications with lower cache utilization. However more cache utilization does not always result in better performance. Streaming applications for example have very high cache utilization, but very small cache reuse. In fact, the streaming applications only need a small cache space to buffer the streaming data. With rapid improvements in semiconductor technology, more and more cores are being embedded into a single core and managing large scale application using a single resource manager becomes more challenging. \\
%\indent Even if the applications are forced to announce their resource demand, it is possible that they lie about their resource vector or run some useless instructions to pretend to utilize the allocated resources given to them.
\indent In addition, defining a single fairness parameter for multiple applications is non-trivial since applications have different bottlenecks and may get different performance benefits from each resources during each phases of its execution time. Defining a single reasonable parameter for fairness is somewhat problematic. For instance, simple assignment algorithms which try to equally distribute the resources between all applications ignores the fact that different applications have different resource constraints. As a consequence, this makes the centralized resource management systems very inefficient in terms of fairness as well as performance needs of applications. We need a decentralized framework, where all applications' performance benefit could be translated into a unique notion of fairness and performance objective (known as utility function in economics) and the algorithm tries to allocate resources based on this translated notion of fairness. This translation has been well defined in economics and marketing, where the diversity of customer needs, makes more economically efficient market \cite{zhou2014sharing}.\\
\indent Economists have shown that in an economically efficient market, having diverse resource constraints and letting the customers compete for the resources can make a Nash equilibrium where both the applications and the resource managers can be enriched. \\
\indent Furthermore, applications' demand changes over time. Most resource allocation schemes pre-allocate the resources without considering the dynamism in applications' need and number of users sharing the same resource over time. Therefore, applications' performance can degrade drastically over time. Figure~\ref{fig:Phases} shows phase transitions for instruction per cycle (IPC) of mcf application from \textit{spec 2006} over 50 billion instructions. \\ 
%%%%%%%%%%%%%%%%%%%%%%%%%%%%%%%%%%%%%%%%%%%%%%%%%%%%%%%%%%%%%%%%%%%%%%%%%%%
\begin{figure}[!tb]
\centering
%\includegraphics[height=3in, width=1.5in]{NodeArchs2.pdf}
\includegraphics[height=1.5in, width=3.3in]{Images/Phases_May.pdf} %Phases.pdf
%\epsfig{file=Dataset.eps, height=2.5in, width=3in}
\caption{\label{fig:Phases}Phase transition in mcf with different L2 cache sizes.}
\end{figure}
%%%%%%%%%%%%%%%%%%%%%%%%%%%%%%%%%%%%%%%%%%%%%%%%%%%%%%%%%%%%%%%%%%%%%%%%%%%
\indent We try to find a game-theoretic distributed resource management approach where the shared hardware resources are exposed to the applications and we will show that running a repeated auction game between different applications which are assumed to be rational, the output of the game would converge to a balanced Nash equilibrium allocation. In addition, we will compare the convergence time of the proposed algorithm in terms of dynamism in the system. We will evaluate our model with two case studies: 1- Private and Shared last level cache problem, where the applications have to decide if they would benefit from a larger cache space which can potentially get more congested or a smaller cache space which is potentially less congested. Based on the number of other applications in the system the application can change its strategy over the time. 2- Heterogeneous processors (\textit{Intel Xeon} and \textit{Xeon Phi}) problem, where we perform experiments to show how congestion affects the performance of different applications running on an \textit{Intel Xeon} or \textit{Xeon Phi} co-processors. Based on the congestion in the system the application can offload the most time consuming part of its code on \textit{Xeon Phi} co-processors or not.   
%%%%%%%%%%%%%%%%%%%%%%%%%%%%%%%%%%%%%%%%%%%%%%%%%%%%%%%%%%%%%%%%%%%%%%%%%%%
\subsection{Background}
%Congestion games have been studied in network routing protocols where the delay of each player choosing a path in the network depends on the number of players choosing the same route in the system. 
%Every congestion game is a potential game since there exists a potential function associated with it. In addition, every congestion game has a pure-strategy Nash equilibrium. A key assumption in congestion games is that all users have the same impact on the congestion. However, this assumption is not always true. In case of computer architecture resources, applications effect each other differently and dividing the pay-off function by the number of users running on the shared resource does not give us the correct utility. 
%%%%%%%%%%%%%%%%%%%%%%%%%%%%%%%%%%%%%%%%%%%%%%%%%%%%%%%%%%%%%%%%%%%%%%%%%%
Game theory has been used extensively in economics, political and social decision making situations \cite{tootaghaj2011game, tootaghaj2011risk, kotobi2017spectrum, kotobi2015introduction, kesidis2013distributed, kurve2013agent, wang2017using, wang2015recouping}. A game is a situation, where the the output of each player not only depends on her own action in the game, but also on the action of other players \cite{osborne1994course}. Auction games are a class of games which has been used to formulate real world problems of assigning different resources between $n$ users. Auction game framework can model resource competition, where the payoff (cost) of each application in the system is a function of the contention level (number of applications) in the game.\\
\indent Inspired by market-based interactions in real life games, there exists a repeated interaction between competitors in a resource sharing game. Assuming large number of applications, we show that the system gets to a Nash equilibrium where all applications are happy with their resource assignment and don't want to change their state. Furthermore, we show that the auction model is strategy-proof, such that no application can get more utilization by bidding more or less than the true value of the resource. In this paper we propose a distributed market based approach to enforce cost on each resource in the system and remove the complexity of resource assignment from the central decision maker.\\ 
\indent The traditional resource assignment is performed by the operating system or a central hardware to assign fair amount of resources to different applications. However, fair scheduling is not always optimal and solving the optimization problem of assigning $m$ resources between $n$ users in the system is an integer programming which is an NP-hard problem and finding the best assignment problem becomes computationally infeasible. Prior works focus on designing a fair scheduling function that maximizes all application's benefit \cite{zahedi2014ref, llull2017cooper, ghodsi2011dominant, zahedi2015sharing, fan2016computational}, while applications might have completely different demands and it is not possible to use the same fairness function for all. By shifting decision making to the individual applications, the system becomes scalable and the burden of establishing fairness is removed from the centralized decision maker, since individual applications have to compete for the resources they need. Applications start with the profiling utility functions for each resource and bid for the most profitable resource. During the course of execution time they can update their belief based on the observed performance metrics at each round of the auction. The idea behind updating the utility functions is that the history at each round of decision point shows the state of the game. This state indicates the contention on the current acquired resource. The pay-off function in each round depends on the state of the system and on the action of other applications in the system. 
%%%%%%%%%%%%%%%%%%%%%%%%%%%%%%%%%%%%%%%%%%%%%%%%%%%%%%%%%%%%%%%%%%%%%%%%%%%
\subsubsection{Sequential Auction}
Auction-based algorithms are used for maximum weighted perfect matching in a bipartite graph $G=(U,V, E)$ \cite{bertsekas1998network, kyle1985continuous, vasconcelos2009bipartite}. A vertex  $U_i \in U$ is the application in the auction and a vertex $V_j \in V$ is interpreted as a resource. The weight of each edge from $U_i$ to $V_j$ shows the utility of getting that particular resource by $U_i$. The prices are initially set to zero and will be updated during each iteration of the auction. In sequential auctions, each resource is taken out by the the auctioneer and is sequentially auctioned to the applications, until all the resources are sold out.
\subsubsection{Parallel Auction}
In a parallel auction, the applications submit their bids for the first most profitable item. The value of the bid at each iteration is computed based on the difference of the highest profitable object and the second highest profitable object. The auctioneer would assign the resources based on the current bids. At each iteration, the valuation of each resource is updated based on the observed information during run-time which shows the contention on that particular resource.
%%%%%%%%%%%%%%%%%%%%%%%%%%%%%%%%%%%%%%%%%%%%%%%%%%%%%%%%%%%%%%%%%%%%%%%%%%%%
%%%%%%%%%%%%%%%%%%%%%%%%%%%%%%%%%%%%%%%%%%%%%%%%%%%%%%%%%%%%%%%%%%%%%%%%%%%%
\section{CAGE: A Market-based Contention-aware Game-theoretic resource assignment}\label{Problem_definition}
\subsection{Model Description}
Consider $n$ applications and $i$ instances of $m$ different resources. Applications arrive in the system one at a time. The applications have to choose among $m$ resources. There exists a bipartite graph between the matching of the applications and the resources.\\
\indent In general, there can be more than one application to get a shared resources. However, each application can not get more than one of the available heterogeneous resources. For example, if we have two cache space of 128kB (one way) and 256kB (two ways), the application can either get the 128kB of cache space or 256kB and can't get both of them at the same time. Furthermore, each resource $m_i$ has a cost $C_i$ which is defined by the applications' bid in the auction. \\
\indent Figure~\ref{fig:auction} shows auction-based framework to support \textit{CAGE} between $N$ applications that execute together competing for $M$ different resources. Each application has a utility table that shows how much performance it gets from each $M$ resources at each time slot. Based on the utility tables, applications submit bids for the most profitable resource. Based on the submitted bids, the auctioneer decides about the resource assignment for each resource, and updates the prices. Next, the applications who did not get any assignment compete for the next most profitable resource based on the updated prices repeatedly until all applications are assigned.  Figure~\ref{fig:auction} shows an example of a resource assignment and the corresponding bipartite graph.
%Table~\ref{table:notation} shows the notation used in our formulation.
%%%%%%%%%%%%%%%%%%%%%%%%%%%%%%%%%%%%%%%%%%%%%%%%%%%%%%%%%%%%%%%%%%%%%%%%%%%%%%%
%%%%%%%%%%%%%%%%%%%%%%%%%%%%%%%%%%%%%%%%%%%%%%%%%%%%%%%%%%%%%%%%%%%%%%%%%% 
\begin{table}[!tb] 
\centering
\caption{Notation used in our formulations.}\label{Table:notation}
\begin{tabular}{|p{0.7in}||p{2.3in}|} 
\hline $N$ & Number of applications \\
\hline $K$ & Number of cache levels \\
\hline $T$ & Time intervals where the bidding is hold \\
\hline $m$ & Number of applications which can get a resource \\ 
\hline $p$ & Number of phases for each application during its course of execution time \\ 
\hline $n$ & Number of applications competing for a specified resource \\
\hline $M$ & Number of resources \\
\hline $P_i$ & Number of phases for application $i$ \\
\hline $\delta$ & dynamic factor that shows how much we can rely on the past iterations. \\
\hline $U$ & The applications which shows the left set of nodes in the bipartite graph. \\
\hline $V$ & The resources which shows the right set of nodes in the bipartite graph. \\
\hline $E$ & The edges in the bipartite graph. \\
\hline $G=(U,V,E)$ & A bipartite graph showing the resource allocation between the applications and the set of resources. \\
\hline $b_{i,k}$ & User i's bid for k th resource \\
\hline $B_i$ & The total budget (sum of bids) a user have \\
\hline $C_k$ & The total capacity of each resource \\
\hline $p_{j}$ & The price of resource $j \in V$ in the auction. \\
\hline $Bottleneck_{1,i}$ & The first bottleneck resource for application $i$ \\
\hline $Bottleneck_{2,i}$ & The second bottleneck resource for application $i$ \\
\hline $v_{i,m}(T)$ & The valuation function of application $i$ for resource $m$ at time $T$ \\
\hline
\end{tabular}
\end{table}
%%%%%%%%%%%%%%%%%%%%%%%%%%%%%%%%%%%%%%%%%%%%%%%%%%%%%%%%%%%%%%%%%%%%%%%%%% 
%%%%%%%%%%%%%%%%%%%%%%%%%%%%%%%%%%%%%%%%%%%%%%%%%%%%%%%%%%%%%%%%%%%%%%%%%%%%%%%
\begin{figure*}[!htb]
\centering
%\includegraphics[height=3in, width=1.5in]{NodeArchs2.pdf}
\includegraphics[height=3.2in, width=6.5in]{Images/Auction_v2.pdf} %[height=4in, width=8in]
%\epsfig{file=Dataset.eps, height=2.5in, width=3in}
\caption{\label{fig:auction} Framework for auction-based resource assignment (CAGE).}
\end{figure*}
%%%%%%%%%%%%%%%%%%%%%%%%%%%%%%%%%%%%%%%%%%%%%%%%%%%%%%%%%%%%%%%%%%%%%%%%%%%
\begin{comment}
\begin{figure}[!htb]
\centering
%\includegraphics[height=3in, width=1.5in]{NodeArchs2.pdf}
\includegraphics[height=2.2in, width=1.3in]{Images/bipartite.pdf}
%\epsfig{file=Dataset.eps, height=2.5in, width=3in}
\caption{\label{fig:bipartite} Cache allocation as a bipartite graph.}
\end{figure}
\end{comment}
%%%%%%%%%%%%%%%%%%%%%%%%%%%%%%%%%%%%%%%%%%%%%%%%%%%%%%%%%%%%%%%%%%%%%%%%%%%
\subsection{Problem Defenition} 
\indent We formulate our problem as an auction based mechanism to enforce cost/value updates for each resource as follows: \\
%The cost of each player to get a resource is the cost of the assigned resource divided by the number of players who share. 
\begin{itemize}
  \item \textbf{Valuation $\mathbf{v_{i,m}}$} : Any application has a valuation function which shows how much he benefits from $i th$ resource. The valuation function at time $t=0$ for cache contention case study is derived from the IPC (instruction per cycle) curves which is found using profiling, and for processor and co-processor contention case study is derived from the profiling solo performance metric of the application. However, in general, each application can choose its own utility function.  
  %%%%%%%%%%%%%%%%%%%%%%%%%%%%%%%%%%%%%%%%%%%%%%%%%%%%%% 
    \item \textbf{Observed information}: The observed information at each time step is the performance value of the selected action in the game. Therefore, the applications repeatedly update the history of their valuation function over time.  
  %%%%%%%%%%%%%%%%%%%%%%%%%%%%%%%%%%%%%%%%%%%%%%%%%%%%%%   
    \item \textbf{Belief updating}: At each iteration step of the auction, the applications update their valuation of each resource based on the observed performance on each resource. The update at time $T$ is derived using the following formula:
%\begin{small}
\begin{equation}\label{eq:belief}
v_{i,m}(T)=\frac{\sum\limits_{t=0}^T {\delta}^{T-t}  v_{i,m}(t)}{\sum\limits_{t=0}^T {\delta}^{T-t}} 
\end{equation}  
%\end{small}  
%%%%%%%%%%%%%%%%%%%%%%%%%%%%%%%%%%%%%%%%%%%%%%%%%%%%%%%%%%%%%%%%%%%%%%%%%
Where $v_{i,m}(t)$ shows the observed valuation of resource $m$ at time step $t$ by user $i$ in the system; $\delta$ shows the discount factor between 0 and 1 which shows how much a user relies on its past observations in the system. The discount factor is chosen to show the dynamics in the system. If the observed information in the system changes fast, the discount factor is nearly zero which means that we can't rely on the past observations very much. However if the system is more stable and the observed information does not change fast, the discount factor is chosen to be near 1. We choose the discount factor as the absolute value of the correlation coefficient of the observed values of the valuations at each iteration step which is calculated as follows:
%\begin{small}
\begin{equation}
\delta =  \frac{E(v_{i,m})^2}{{\sigma_{v_{i,m}}}^2}
\end{equation}  
%\end{small}
%%%%%%%%%%%%%%%%%%%%%%%%%%%%%%%%%%%%%%%%%%%%%%%%%%%%%%%%%%%%%%%%%%%%%%%%
%%%%%%%%%%%%%%%%%%%%%%%%%%%%%%%%%%%%%%%%%%%%%%%%%%%%%%%%%%%%%%%%%%%%%%%%
  \item \textbf{Action}: At each time step the applications decides which resource to bid and how much to bid for each resource. 
\end{itemize} 
%%%%%%%%%%%%%%%%%%%%%%%%%%%%%%%%%%%%%%%%%%%%%%%%%%%%%%%%%%%%%%%%%%%%%%%%%% 
\indent Table~\ref{Table:notation} shows important notation used throughout the paper. In the following sections, we describe our distributed optimization scheme to solve the problem. 
%%%%%%%%%%%%%%%%%%%%%%%%%%%%%%%%%%%%%%%%%%%%%%%%%%%%%%%%%%%%%%%%%%%%%%%
%\begin{equation}
%min \sum\limits_{i=1}^n v_i C_k \frac{b_{i,k}}{\theta_k}, \\
%s.t. \sum\limits_{i=1}^n b_{i,k} \leq E_i
%\end{equation}
%%%%%%%%%%%%%%%%%%%%%%%%%%%%%%%%%%%%%%%%%%%%%%%%%%%%%%%%%%%%%%%%%%%%%%%
\subsection{Distributed Optimization Scheme}
The goal is to design a repeated auction mechanism which is run by the operating system to guide the applications to choose their best resource allocation strategy. The applications' goal is to maximize their own performance and the operating system wants to maximize the total utility it gains from the applications. Then, each application can use its own utility function and evaluates the resources based on how much it likes that particular resource. \\
\indent \textbf{Applications' approach}: The application $i$ want to maximize the total utility with respect to a limited budget for all phase $p$ of its execution time. \\
%%%%%%%%%%%%%%%%%%%%%%%%%%%%%%%%%%%%%%%%%%%%%%%%%%%%%%%%%%%%%%%%%%%%%%%%%
%maximize \;\;\;\; \sum\limits_{i=1}^n v_i C_k \frac{b_{i,k}}{\theta_k},\\
%\begin{small}
\begin{align}
%\begin{IEEEeqnarray}{rCl}
\forall i \in U \; \; \; \; \; maximize \; \; \; \; \sum\limits_{p=1}^{P_i} \sum\limits_{m=1}^M  v_{i,m,p}-b_{i,m,p} , \nonumber \\
 % \IEEEyessubnumber\\
subject \; to \;\;\;\; \sum\limits_{p=1}^{P_i} \sum\limits_{m=1}^M b_{i,m,p} \leq B_i .
%\IEEEyessubnumber
%\end{IEEEeqnarray}
\end{align}
%\end{small}
%%%%%%%%%%%%%%%%%%%%%%%%%%%%%%%%%%%%%%%%%%%%%%%%%%%%%%%%%%%%%%%%%%%%%%%%%%
\indent \textbf{OS's approach}: The operating system wants to maximize the social welfare function which is translated into submitted bids from the applications in a limited resource constraints.\\
%%%%%%%%%%%%%%%%%%%%%%%%%%%%%%%%%%%%%%%%%%%%%%%%%%%%%%%%%%%%%%%%%%%%%%%%%%
%\begin{small}
\begin{align}
%\begin{IEEEeqnarray}{rCl}
maximize \; \; \; \sum\limits_{i=1}^N \sum\limits_{p=1}^{P_i} \sum\limits_{m=1}^M b_{i,m,p} A_{i,m,p} , \nonumber \\ 
%\IEEEyessubnumber\\
subject \; to \;\;\;\; \sum\limits_{i=1}^N \sum\limits_{m=1}^M A_{i,m,p} \leq A_{max}, \; \; \; \; \forall p \in P , \nonumber \\
%\IEEEyessubnumber\\
A_{i,m,p} \in \{0,1\} , \; \; \; \; \forall i \in U, \; \;  \forall m \in V, \; \;  \forall p \in P .
%\IEEEyessubnumber
%\end{IEEEeqnarray}
\end{align}
%\end{small}
%%%%%%%%%%%%%%%%%%%%%%%%%%%%%%%%%%%%%%%%%%%%%%%%%%%%%%%%%%%%%%%%%%%%%%%%%%%
\indent \textbf{Illustrative example}: As an illustrative example, suppose we have two different resources, a large cache of 1MB which can be shared between applications, and two private caches of 512KB which are not shared. There are two applications competing for the cache space. One of the applications wants to minimize its request latency and the other one wants to maximize number of instructions executed per cycle. Suppose that both applications have two phases $(0,T)$ and $(T,2T)$.  Suppose if the first application gets the larger cache space its request latency reduces by 20 percent in first phase and by 40 percent in the second phase. The second application's \textit{IPC} increases by 35 percent in the first phase and by 25 percent in the second phase if it gets the larges cache space. Also, assume they both have 60 tokens (bids) to submit. The first application invests 20 token (bids) for the first phase and 40 tokens for the next phase. He should redistribute the tokens for the next phase if he did not get the resource he wants in the first phase. The second application invests 35 tokens in the first phase and 25 tokens in the next phase. The auctioneer (OS) at each phase decides to allocate which resource to which applications. Since, the social welfare would be maximized if the auctioneer allocates both applications with the larger cache space, they would both get the larger resource. Then the first application notices that its utility function does not improve as he predicts and adjusts the utility table and can either change its allocation or stay on current allocation. 
%If both applications bid 10\$ for the private cache and 15\$ for the shared cache, the operating system would allocate both the shared cache space and get 15\$ from each to maximize its revenue.  
\subsection{Analysis}
The distributed optimization problem seems complex. However, in reality the problem can be splitted into simpler subproblems since each application knows its bottleneck resource and would first bid for the first bottleneck resource to maximize its utility.\\
\indent We suppose all applications in the system are risk-neutral which means they have a linear valuation of utility function. Each risk neutral agent wants to maximize its expected revenue. Risk attitude behaviors are defined in \cite{ferber1999multi} where the agents can broadly be divided into risk averse, risk seeking and risk neutral. Risk averse agents prefer determinitic values rather than risky value profits and risk seeking applications have a superlinear utility function and prefer risky utilities than sure utilities. Next, we derive the Bayes Nash equilibrium strategy profile for all agents in the system assuming risk neutrality.  \\
%%%%%%%%%%%%%%%%%%%%%%%%%%%%%%%%%%%%%%%%%%%%%%%%%%%%%%%%%%%%%%%%%%
\newtheorem{defi}{Definition}
\begin{defi}
A strategy profile $a$ is a pure Nash equilibrium if for every application $i$ and every strategy $a_i' \neq a_i \in A$ we have $u_i(a_i, a_{-i}) \geq u_i(a_i', a_{-i})$
\end{defi}
%%%%%%%%%%%%%%%%%%%%%%%%%%%%%%%%%%%%%%%%%%%%%%%%%%%%%%%%%%%%%%%%%%
\newtheorem{theorem}{Theorem}
\begin{theorem}\label{thm:neat}
%\emph{(Theorem)}
\label{Auction}
Suppose $n$ risk-neutral applications whose valuations are derived uniformly and independently from the interval $[0,1]$ compete for one resource which can be assigned to $m$ application who have the highest bid in the auction. We will show that Bayes Nash equilibrium bidding strategy for each application in the system is to bid $\frac{n-m}{n-m+1}v_i$ whre $v_i$ is the profit of application $i$ for getting the specified resource.  
\end{theorem}
%%%%%%%%%%%%%%%%%%%%%%%%%%%%%%%%%%%%%%%%%%%%%%%%%%%%%%%%%%%%%%%
%DIMAN COMMENTED PROOF FOR APPENDIX
\begin{comment}

\begin{proof}
Suppose all other applications' bidding strategy is to choose $\frac{n-m}{n-m+1}v_i$. Since the bidding values were derived uniformly in $[0, 1]$ all bids have the same probability. Therefore, if we consider the first application's expected utility to find its best response, we have:

\begin{equation}
%\begin{IEEEeqnarray}{rCl}
E[u_1] = \int_0^1  .... \int_0^1 \! (v_1 -b_1) \, \mathrm{d}u_2 \mathrm{d}u_3 ... \mathrm{d}u_{n-m} .  
%\end{IEEEeqnarray}
\end{equation}

The following integral breaks into two part where the first application wins the auction or not. 


%\begin{IEEEeqnarray}{rCl}
%E[u_1] = \int_0^{b_1}  .... \int_0^{b_1} \! (v_1 -b_1) \, \mathrm{d}u_2 \mathrm{d}u_3 ... %\mathrm{d}u_{n-m}  \\
%+ \int_{b_1}^1  .... \int_{b_1}^1 \! (v_1 -b_1) \, \mathrm{d}u_2 \mathrm{d}u_3 ... \mathrm{d}%u_{n-m}\nonumber 
%\end{IEEEeqnarray}

\begin{equation}
%\begin{IEEEeqnarray}{rCl}
E[u_1] = \int_0^{\frac{n-m+1}{n-m}b_1}  .... \int_0^{\frac{n-m+1}{n-m}b_1} \! (v_1 -b_1) \, \mathrm{d}u_2 ... \mathrm{d}u_{n-m}  \\
+ \int_{\frac{n-m+1}{n-m}v_1}^1  .... \int_{\frac{n-m+1}{n-m}v_1}^1 \! (v_1 -b_1) \, \mathrm{d}u_2 \mathrm{d}u_3 ... \mathrm{d}u_{n-m}\nonumber 
%\end{IEEEeqnarray}
\end{equation}

The second part of the integrals is the term where the first application doesn't win the auction. Therfore, the expected payoff of application 1 is equal with:

%\begin{IEEEeqnarray}{rCl}
%E[u_1] = \int_0^{b_1}  .... \int_0^{b_1} \! (v_1 -b_1) \, \mathrm{d}u_2 \mathrm{d}u_3 ... %\mathrm{d}u_{n-m}  \\
%= {({b_1}) }^{n-m} (v_1 -b_1). \nonumber 
%\end{IEEEeqnarray}

\begin{equation}
%\begin{IEEEeqnarray}{rCl}
E[u_1] =\int_0^{\frac{n-m+1}{n-m}b_1}  .... \int_0^{\frac{n-m+1}{n-m}b_1} \! (v_1 -b_1)  \mathrm{d}u_2 ... \mathrm{d}u_{n-m}  \\
= {(\frac{n-m+1}{n-m} b_1) }^{n-m} (v_1 -b_1). \nonumber 
%\end{IEEEeqnarray}
\end{equation}


Diffrentiating with respect to $b_1$ the optimal bid for application one is derived as follows:

%\begin{equation}
%\frac{\partial}{\partial b_1} ( {({b_1}) }^{n-m} (v_1 -b_1))=0.
%\end{equation}


\begin{equation}
\frac{\partial}{\partial b_1} ( {(\frac{n-m+1}{n-m} b_1) }^{n-m} (v_1 -b_1))=0.
\end{equation}

Which gives us the optimal bid for each application:
\begin{equation}
\Rightarrow b_1= \frac{n-m}{n-m+1}v_1
\end{equation}
\end{proof}
\end{comment}
%DIMAN COMMENTED PROOF FOR APPENDIX
%%%%%%%%%%%%%%%%%%%%%%%%%%%%%%%%%%%%%%%%%%%%%%%%%%%%%%%%%%%%%%%%%%%%%%%%%%%%
\begin{algorithm}[!tb]
\DontPrintSemicolon % Some LaTeX compilers require you to use \dontprintsemicolon    instead
\KwIn{A bipartite Graph (U, V, E).}
\KwOut{The allocation of resources to applications.}
At t=0 the valuation of each application for each resource is derived using profiling while running alone. 

For each application $U_i \in U$, the first bottleneck resource is
\[ Bottleneck_{1,i} = V_{i,m}=  arg \; \max_{m \in V} (v_{i,m}-p_{m})  \] 
Next, find the second bottleneck resource for each applications $U_i \in U$ in the system:
\[ Bottleneck_{2,i} = V_{i,k}=  arg \; \max_{k \in V, k \neq m} (v_{i,k}-p_{k})  \] 

Each application submits the bid for its first bottleneck resource using the following formula:
\[ b_{i,m} = V_m - V_k + p_{j} + \epsilon \]
Each resource $V_j \in V$, which can be shared between $m$ applications, is assigned to the $m$ highest bidding applications $Winner_j={i_1, i_2, ..., i_m}$ and the price for that resource is updated as follows:
\[ p_{j} =  arg \; \max_{i_1, i_2, ..., i_m \in U} \sum\limits_{k=1}^m (b_{i_k,j})  \]

The $minBid$ for each resource is updated as the minimum bid of $m$ applications who acquired the resource. That is
\[ minBid=  arg \; \min_{i \in Winner_j}  (b_{i,j}) \] 
 
\caption{CAGE: Parallel Auction for heterogeneous resource assignment.}
\label{algo:b}
\vspace{0\baselineskip}
\end{algorithm}
%%%%%%%%%%%%%%%%%%%%%%%%%%%%%%%%%%%%%%%%%%%%%%%%%%%%%%%%%%%%%%%%%%%%%%%%%%%%%
\indent Theorem~\ref{thm:neat}, states that whenever there is a single resource that users compete to get it with different valuation functions, the Nash equilibrium strategy profile for risk-neutral users is to bid $\frac{n-m}{n-m+1}v_i$. This term tends to the true value of the object when n is a large number. \\
\indent In case of more than one resource competition we derive Algorithm~\ref{algo:b} and will prove that it is Nash equilibrium in the game. The algorithm is inspired by work of Bertsekas \cite{bertsekas1998network} that uses an auction for network flow problems. In the first step, all valuations are set to the solo-run of application's performance. Next, each application submits a bid for its first bottleneck resource. The bid should be larger than the price of the object which is intitialized to zero in the begining of the program. The applications only have incentive to bid a value no more than the difference of the first bottleneck and second bottleneck resource. Otherwise, it would submit a smaller bid to the second bottleneck and get the same revenue as paying more for the first bottleneck resource. In order to break the equal valuation function between two different applications, we use $\epsilon$ scaling such that at each iteration of the auction the prices should increase by a small number. 

%In addition, suppose we have 5 different memory bandwidth exposed to the applications Each application gets different performance benefit from different cache sizes and different memory bandwidth which is denoted in table  **. The applications need to submit their bids based on their performance benefits. 

%\begin{equation}
%\begin{split}
%\int_0^\frac{nb_1}{n-m}  .... \int_0^\frac{nb_1}{n-m} \! (\frac{v_1}{m} -b_1) \, \mathrm{d}u_2 %\mathrm{d}u_3 ... \mathrm{d}u_{n-m}= \\
%= {(\frac{nb_1}{n-m}) }^{n-m} (\frac{v_1}{m} -b_1). 
%\end{split}
%\end{equation}
%%%%%%%%%%%%%%%%%%%%%%%%%%%%%%%%%%%%%%%%%%%%%%%%%%%%%%%%%%%%%%%%%%%%%%%%%%%%
%\[ \frac{\partial}{\partial b_1} ({(\frac{nb_1}{n-m}) }^{n-m} (\frac{v_1}{m} -b_1))

%\newtheorem{defi}{Definition}
%\begin{defi}
%Let's assume each user $Ui$ in the system is defined as one vertice of a graph $G$ in the system and let each edge in the graph show which subset of users can impact each others' performance and the associated weight of each edge show the cost function of how two users affect each others' performance in the system. Each edge has a weight function denoted by ${P1(n), P2(n), ... Pe(n)}$, where $e$ is the number of edges in Graph $G$. Let $A=A_1 \times A_2 \times ... \times A_n $ be the set of actions that each user can play. 
%\end{defi}
%\vspace{-0.5\baselineskip}
%\section{Case Studies} \label{Case_Studies}
%\subsection{A case study of Main processor (Xeon) and co-processor (Xeon-Phi) congestion game}
\subsection{CPU Scale-up Scale-out Game}
The emerging high performance computing applications lead to the advent of \textit{Intel Xeon Phi} co-processor, that when their highly parallel architecture is fully utilized, can run order of magnitude more performance than the existing processor architectures. The \textit{Xeon Phi} co-processors are the first commercial product of Intel \textit{MIC} processors where the hardware architecture is exposed to the programmer to choose running the code on either \textit{Xeon} processor or \textit{Xeon Phi} co-processors. It is possible that, during the course of execution, either the processor or the co-processor get congested and the performance of the application degrades a lot. Therefore, making a decision to offload the most time consuming part of the program on \textit{Xeon} or \textit{Xeon Phi} should be made online, based on the contention level.  In this section we look at the case study of our auction-based model on decision making of running the application on the main or co-processor in a highly congested environment. \\
\indent The experiment results of this section are run on \textit{Stampede} cluster of \textit{Texas Advanced Computing Center}. Table~\ref{Table:Xeon} shows the comparison of \textit{Intel Xeon} and \textit{Xeon Phi} architectures which is used in this section. 
%%%%%%%%%%%%%%%%%%%%%%%%%%%%%%%%%%%%%%%%%%%%%%%%%%%%%%%%%%%%%%%%%%%%%%%%%%%
%%%%%%%%%%%%%%%%%%%%%%%%%%%%%%%%%%%%%%%%%%%%%%%%%%%%%%%%%%%%%%%%%%%%%%%%%%%
%%%%%%%%%%%%%%%%%%%%%%%%%%%%%%%%%%%%%%%%%%%%%%%%%%%%%%%%%%%%%%%%%%%%%%%%%%%
\begin{table}[!tb] 
\centering
\caption{Comparison of \textit{Intel Xeon} and \textit{Xeon Phi} Processors.}\label{Table:Xeon}
\begin{tabular}{|c||p{0.8in}||p{1in}|} 
\hline Processors & Xeon E5-2680 & Xeon Phi SE10P \\
\hline Cores/Sockets & 8/2 & 61/1 \\
\hline Clock Frequency & 2.7 GHz & 1.1 GHz  \\
\hline Memory & 32GB 8x4G 4-channels DDR3-1600MHz & 8GB GDDR5 \\
\hline L1 cache & 32 KB & 32 KB \\
\hline L2 cache & 256 KB & 512 KB \\
\hline L3 cache & 20 MB & - \\
\hline
\end{tabular}
\end{table}
%%%%%%%%%%%%%%%%%%%%%%%%%%%%%%%%%%%%%%%%%%%%%%%%%%%%%%%%%%%%%%%%%%%%%%%%%%%
%%%%%%%%%%%%%%%%%%%%%%%%%%%%%%%%%%%%%%%%%%%%%%%%%%%%%%%%%%%%%%%%%%%%%%%%%%%
%%%%%%%%%%%%%%%%%%%%%%%%%%%%%%%%%%%%%%%%%%%%%%%%%%%%%%%%%%%%%%%%%%%%%%%%%%%
\indent It is observed that congestion has a significant effect on the performance of running the application on \textit{Xeon} and \textit{Xeon Phi} machines. Since most cloud computing machines are shared between thousands of users, the programmer not only should get benefit of parallelism by offloading the most time consuming part of the code to the larger number of low-performance cores (\textit{Xeon Phi}), but also should consider the congestion level (number of co-runners) in the system. To this end, we performed experiments on \textit{Stampede} clusters. We executed \textit{MiniGhost} application which is a part of \textit{Mantevo} project \cite{mantevo} which uses difference stencils to solve partial differential equations using numerical methods. The applications use the profiling utility functions at $t=0$ and during course of execution can update the utility function based on the observed performance on each core using Equation~\ref{eq:belief}. Then, they can revisit their previous action on running the code on either the processor or co-processor during run-time. \\
\indent Figure~\ref{Fig:congestion} shows the total execution time with respect to congestion we made in \textit{Xeon} and \textit{Xeon Phi}. In this experiment we ran the same problem size on a \textit{Xeon} and \textit{Xeon Phi} machine multiple times, so that we could see the effect of load on the total execution time of our application. It was observed that with the same number of threads \textit{Xeon}'s performance degrades more than \textit{Xeon phi}. 
Next, we tried to change the application behavior using congestion-aware game theoretic algorithm to offload the most time consuming part of the application based on the performance behavior of applications. Figure~\ref{Fig:performance_over_time} shows the result of our game-theoretic model during the execution time. It is observed that during the course of execution, the applications change their strategy on either choosing the main processor or the co-processor and all applications' performance converge to a equilibrium point where applications don't want to change their strategy. \\
\indent Furthermore, it is shown that CAGE can bring in up to 106.6\% improvement in total execution time of applications compared to static approach when the number of co-runners is six. The performance improvement would be significant when the number of co-runners increase. Figure~\ref{Fig:Perfomance_Comparison} shows the performance comparison of CAGE and static approach which does not consider the congestion dynamism in the system and the decision is only made based on the parallelism level in the code. 
%%%%%%%%%%%%%%%%%%%%%%%%%%%%%%%%%%%%%%%%%%%%%%%%%%%%%%%%%%%%%%%%%%%%%%%%%%%
%%%%%%%%%%%%%%%%%%%%%%%%%%%%%%%%%%%%%%%%%%%%%%%%%%%%%%%%%%%%%%%%%%%%%%%%%%%
%%%%%%%%%%%%%%%%%%%%%%%%%%%%%%%%%%%%%%%%%%%%%%%%%%%%%%%%%%%%%%%%%%%%%%%%%%%
\begin{figure}[!htb]
\centering
%\includegraphics[height=3in, width=2.5in]{NodeArchs2.pdf}
\includegraphics[height=1.5in, width=3.5in]{Images/Xeon.pdf} % diman.pdf
%\epsfig{file=Dataset.eps, height=2.5in, width=3in}
\caption{Congestion effect on \textit{Xeon} and \textit{Xeon Phi} machines.}\label{Fig:congestion}
\end{figure}
%%%%%%%%%%%%%%%%%%%%%%%%%%%%%%%%%%%%%%%%%%%%%%%%%%%%%%%%%%%%%%%%%%%%%%%%%%%
%%%%%%%%%%%%%%%%%%%%%%%%%%%%%%%%%%%%%%%%%%%%%%%%%%%%%%%%%%%%%%%%%%%%%%%%%%%
\begin{figure}[!tb]
\centering
%\includegraphics[height=3in, width=2.5in]{NodeArchs2.pdf}
\includegraphics[height=1.5in, width=3.5in]{Images/Game.pdf}%Game_during_time.pdf
%\epsfig{file=Dataset.eps, height=2.5in, width=3in}
\caption{Performance of 6 instance of applications during time for our proposed game model.}\label{Fig:performance_over_time}
\end{figure}
%%%%%%%%%%%%%%%%%%%%%%%%%%%%%%%%%%%%%%%%%%%%%%%%%%%%%%%%%%%%%%%%%%%%%%%%%%%
%%%%%%%%%%%%%%%%%%%%%%%%%%%%%%%%%%%%%%%%%%%%%%%%%%%%%%%%%%%%%%%%%%%%%%%%%%%
\begin{figure}[!htb]
\centering
%\includegraphics[height=3in, width=1.5in]{NodeArchs2.pdf}
\includegraphics[height=1.5in, width=3.5in]{Images/Congestion.pdf}  %Congestion_aware.pdf
%\epsfig{file=Dataset.eps, height=2.5in, width=3in}
\caption{Performance comparison of congestion-aware schedule versus static schedule.}\label{Fig:Perfomance_Comparison}
\end{figure}
%%%%%%%%%%%%%%%%%%%%%%%%%%%%%%%%%%%%%%%%%%%%%%%%%%%%%%%%%%%%%%%%%%%%%%%%%%%
%%%%%%%%%%%%%%%%%%%%%%%%%%%%%%%%%%%%%%%%%%%%%%%%%%%%%%%%%%%%%%%%%%%%%%%%%%%
%%%%%%%%%%%%%%%%%%%%%%%%%%%%%%%%%%%%%%%%%%%%%%%%%%%%%%%%%%%%%%%%%%%%%%%%%%%
\subsection{A Case Study of Private and shared cache game}
One of the challenging problems in \textit{CMP} resource management systems is whether applications benefit from a shared large last level cache or an isolated private cache. We evaluated CAGE performance, on a 10MB LLC shown in Figure~\ref{Fig:cache_hierarchy}, where 2MB, 1MB, 512kB, 256kB and 128kB levels of LLC can potentially be shared between 16, 8, 4, 2 and 1 applications respectively, the cache levels have 16, 8, 4, 2, and 1 ways. Table~\ref{Table:Workloads} summarizes the studies workloads and their characteristics, including miss per kilo instructions (\textit{MPKI}), memory bandwidth usage, and IPC. We use applications from \textit{Spec 2006} benchmark suite \cite{Spec:website}. We use \textit{Gem5} full system simulator in our experiment ~\cite{binkert2011gem5, Gem5:website}. Table~\ref{Table:Experimental_Set_Up} shows the experimental setup in our experiments.\\
\indent To evaluate the performance of our proposed approach we use utility functions for different number of ways shown in Figure~\ref{fig:Cache_IPC}. These utility functions at the start of the execution can be found using either profiling techniques or stack distance profile \cite{kim2004fair, suh2002new, suh2004dynamic} of applications assuming there is no co-runners in the system. Next, during run-time the applications can update their utility functions based on Equation~\ref{eq:belief}. Therefore, there is a learning phase where applications learn about the state of the system and update the utilities accordingly. The stack distance profile indicates how many more cache misses will be added if the application has less number of ways in the cache. Based on the stack distance profile, the applications can update their utility function and bid for the next iteration of the auction if they like to change their allocation. Next, we bring an example of the auction for one time step of the game. This time step can be repeated once an application arrives or leaves the system or when an application's phase changes during run-time. However, in case of one application's phase change or arriving or leaving the system, the algorithm reaches the optimal assignment in much fewer iterations since all other assignments are fixed and a few applications would be affected.\\
%%%%%%%%%%%%%%%%%%%%%%%%%%%%%%%%%%%%%%%%%%%%%%%%%%%%%%%%%%%%%%%%%%%%%%%%%%%
\begin{figure}[!tb]
\centering
%\includegraphics[height=3in, width=2.5in]{NodeArchs2.pdf}
\includegraphics[height=1.5in, width=3.3in]{Images/Cache_Hierarchy_2.pdf}
%\epsfig{file=Dataset.eps, height=2.5in, width=3in}
\caption{Our proposed last level cache hierarchy model.}\label{Fig:cache_hierarchy}
\end{figure}
%%%%%%%%%%%%%%%%%%%%%%%%%%%%%%%%%%%%%%%%%%%%%%%%%%%%%%%%%%%%%%%%%%%%%%%%%%%
%%%%%%%%%%%%%%%%%%%%%%%%%%%%%%%%%%%%%%%%%%%%%%%%%%%%%%%%%%%%%%%%%%%%%%%%%
\begin{table}[!tb] 
\centering
\caption{Experimental Setup.}
\label{Table:Experimental_Set_Up}
\begin{tabular}{|c||p{1.5in}|} 
\hline Processors & Single threaded with private L1 instruction and data caches \\
\hline Frequency & 1GHz \\
\hline L1 Private ICache & 32 kB, 64-byte lines, 4-way associative\\
\hline L1 Private DCache & 32 kB, 64-byte lines, 4-way associative \\
\hline L2 Shared Cache & 128 kb-2 MB, 64-byte lines, 16-way associative \\
\hline RAM & 12 GB \\

\hline
\end{tabular}
\end{table}
%%%%%%%%%%%%%%%%%%%%%%%%%%%%%%%%%%%%%%%%%%%%%%%%%%%%%%%%%%%%%%%%%%%%%%%%%
\begin{table}[!tb] 
\centering
\caption{Evaluated workloads.}
\label{Table:Workloads}
\begin{tabular}{p{0.7cm} p{1.5cm} p{1cm} p{1.7cm} p{1cm} }
\hline
%\begin{tabular}{|l|c|c|c|} \hline
{\bf \#} & \bf Benchmark & MPKI & Memory BW & IPC  \\
\hline 
{\bf 1} & astar & 1.319 & 373 MB/s & 2.057 \\
{\bf 2} & bwaves & 10.47 & 1715 MB/s & 0.661 \\
{\bf 3} & bzip2 & 3.557 & 1194 MB/s & 1.367 \\ 
{\bf 4} & dealII &  0.935 & 307 MB/s & 2.107 \\
{\bf 5} & GemsFDTD & 0.004 & 2.19 MB/s & 2.023 \\
{\bf 6} & hmmer & 2.113 & 1547 MB/s & 2.861 \\
{\bf 7} & lbm & 19.287 & 3954 MB/s & 0.533 \\
{\bf 8} & leslie3d & 8.469 & 1942 MB/s & 1.297 \\
{\bf 9} & libquantum & 10.388 & 1589 MB/s & 0.531 \\
{\bf 10} & mcf & 16.93 & 820 MB/s & 0.073 \\
{\bf 11} & namd & 0.051 & 20.32 MB/s & 2.362\\
{\bf 12} & omnetpp & 10.34 & 1147 MB/s & 0.504 \\
{\bf 13} & sjeng & 0.375 & 139.2 MB/s & 1.403 \\
{\bf 14} & soplex & 4.672 & 390.8 MB/s & 0.513 \\
{\bf 15} & sphinx3 & 0.349 & 202.8 MB/s & 2.223 \\
{\bf 16} & streamL & 31.682 & 3619 MB/s & 0.581 \\
{\bf 17} & tonto & 0.260 & 107 MB/s & 2.036 \\
{\bf 18} & xalancbmk & 12.703 & 1200 MB/s & 0.558 \\
%\hline  
\hline
\end{tabular}
\end{table}
%%%%%%%%%%%%%%%%%%%%%%%%%%%%%%%%%%%%%%%%%%%%%%%%%%%%%%%%%%%%%%%%%%%%%%%%%%
\begin{comment}
Assume $n$ different applications denoted by ${U1, U2, .. Un}$ with different cache benefits which may affect each other with different cost functions. Figure 1 shows the different applications which are the vertices of the graph with their impact on each other which are the weights of edges in the graph. If two vertices are not connected in the graph, it means that they would not affect each others' performance. For example, one application is CPU bound and does not benefit from larger memory bandwith and the other is memory bound and does not benefit from having more CPU capacity. So different applications affect each other's performance with different coefficients. 
\end{comment}
%%%%%%%%%%%%%%%%%%%%%%%%%%%%%%%%%%%%%%%%%%%%%%%%%%%%%%%%%%%%%%%%%%%%%%%%%%%
%%%%%%%%%%%%%%%%%%%%%%%%%%%%%%%%%%%%%%%%%%%%%%%%%%%%%%%%%%%%%%%%%%%%%%%%%%%
%%%%%%%%%%%%%%%%%%%%%%%%%%%%%%%%%%%%%%%%%%%%%%%%%%%%%%%%%%%%%%%%%%%%%%%%%%%
\begin{figure*}[!tb]
\centering
\includegraphics[height=3.5in, width=6.5in]{Images/Cache_IPC_v2.pdf}
\caption{IPC for different size of LLC.}\label{fig:Cache_IPC}  
\end{figure*}
%%%%%%%%%%%%%%%%%%%%%%%%%%%%%%%%%%%%%%%%%%%%%%%%%%%%%%%%%%%%%%%%%%%%%%%%%%%
%%%%%%%%%%%%%%%%%%%%%%%%%%%%%%%%%%%%%%%%%%%%%%%%%%%%%%%%%%%%%%%%%%%%%%%%%%%
%%%%%%%%%%%%%%%%%%%%%%%%%%%%%%%%%%%%%%%%%%%%%%%%%%%%%%%%%%%%%%%%%%%%%%%%%%%
\indent \textbf{Example:} As an example, suppose we have 5 different applications and 5 different cache levels with different capacities of 128KB, 256KB, 512KB, 1MB and 2MB. In addition, suppose the 128kB cache level can not accomodate more than one application and 256kB cache can accomodate 2 applications, 512kB level can have 4 applications, 1MB cache can have 8 applications and 2MB cache can have at most 16 applications. Let's assume the following matrix be the utility function of each application on each cache level. \\
\indent Some applications may get better utility from smaller cache space since they are less congested and since these applications have low data locality, moving to larger cache spaces not only does not increase their performance but also degrades the performance by evicting other applications from the cache and making contention on the memory bandwidth which is a more vital resource for them \footnote{\textit{libquantum}, \textit{streamL}, \textit{sphinx3}, \textit{lbm} and \textit{mcf} are examples of such applications.}.  \\
\begin{equation}
M = \bordermatrix{~ & 1way & 2way & 4way & 8way & 16way \cr
  App1 & 1.9 & 1.7 & 1.5 & 1 & 0.9 \cr
  App2 & 1.6 & 1.3 & 1.1 & 0.8 & 0.7 \cr
  App3 & 1.4 & 1.0 & 0.6 & 0.5 & 0.4 \cr
  App4 & 0.3 & 0.6 & 0.9 & 1.2 & 1.4 \cr
  App5 & 0.7 & 0.8 & 1.1 & 1.4 & 1.7 \cr}
\end{equation}
%%%%%%%%%%%%%%%%%%%%%%%%%%%%%%%%%%%%%%%%%%%%%%%%%%%%%%%%%%%%%%%%%%%%%%%%%%%
%%%%%%%%%%%%%%%%%%%%%%%%%%%%%%%%%%%%%%%%%%%%%%%%%%%%%%%%%%%%%%%%%%%%%%%%%%%
In the first iteration of the bidding, the first 3 applications bid for the most profitable resource which is 128kB cache and they submit a bid equal to the difference of profit between the first and the second most profitable resource. Therefore, the first application, submits 0.2 bid to 128kb and the second application submits 0.3 and the third application submits 0.4. Since only one of the players can aqcuire the 128kB cache space, the first application will get it. The 4th and 5th application compete for 2MB cache space and they both get it with the sum bid of both which is 0.5. In the next round, the prices will be updated and since apllications 2 and 3 don't have any cache assignment compete for the 256kB cache space and each bid 0.2 which is the difference between 1.7 and 1.5 and 1.3 and 1.1 in the performance matrix accordingly. Since the second level cache can accomodate both applications the price will be updated and the minimum bidding price for some one to get this cache level is updated to the minimum bid of both which is 0.2. Therefore, if some application bid more than 0.2 it can acquire the resource and the application with smallest bid has to resubmit the bid to acquire the resource. Figure~\ref{fig:first_round}, ~\ref{fig:second_round}, and ~\ref{fig:third_round} show the bidding steps and the prices and minimum price of bidding accordingly. As seen from the figures, the auction terminates in three iterations when there exists five applications. \\
%%%%%%%%%%%%%%%%%%%%%%%%%%%%%%%%%%%%%%%%%%%%%%%%%%%%%%%%%%%%%%%%%%%%%%%%%%%
%%%%%%%%%%%%%%%%%%%%%%%%%%%%%%%%%%%%%%%%%%%%%%%%%%%%%%%%%%%%%%%%%%%%%%%%%%%
%%%%%%%%%%%%%%%%%%%%%%%%%%%%%%%%%%%%%%%%%%%%%%%%%%%%%%%%%%%%%%%%%%%%%%%%%%% 
\begin{figure*}[!htb]
        \centering
        \begin{subfigure}[b]{0.25\textwidth} %//0.28 bood
                \includegraphics[width=\textwidth]{Images/bid0.pdf}
                \caption{first round.}
                \label{fig:first_round}
        \end{subfigure}%
        ~ %add desired spacing between images, e. g. ~, \quad, \qquad etc.
          %(or a blank line to force the subfigure onto a new line)
        \begin{subfigure}[b]{0.25\textwidth}
                \includegraphics[width=\textwidth]{Images/bid2.pdf}
                \caption{second round.}
                \label{fig:second_round}
        \end{subfigure}
        ~ %add desired spacing between images, e. g. ~, \quad, \qquad etc.
          %(or a blank line to force the subfigure onto a new line)
        \begin{subfigure}[b]{0.25\textwidth}
                \includegraphics[width=\textwidth]{Images/bid3.pdf}
                \caption{third round.}
                \label{fig:third_round}
        \end{subfigure}  

                \caption{Cache allocation, a) first round, b) second round and c) third round of bidding.}\label{fig:Auction_rounds}    
       % \vspace{-2\baselineskip}
\end{figure*}
%%%%%%%%%%%%%%%%%%%%%%%%%%%%%%%%%%%%%%%%%%%%%%%%%%%%%%%%%%%%%%%%%%%%%%%%%%%
%%%%%%%%%%%%%%%%%%%%%%%%%%%%%%%%%%%%%%%%%%%%%%%%%%%%%%%%%%%%%%%%%%%%%%%%%%%
\indent Next, we use different mixes of 4 to 16 applications from \textit{Spec 2006} to evaluate the performance of our proposed approach. Figure~\ref{fig:IPC_mix} shows the normalized throughput of 10 different mix of applications using CAGE, equal private cache partitions and completely shared cache space. Figure~\ref{fig:scalability} shows the scalability of our proposed algorithm. When the number of co-runners increases from 2 to 16, the performance improves from 12.4\% to 33.6\% without any need to track each applications' performance in a central hardware. 
%%%%%%%%%%%%%%%%%%%%%%%%%%%%%%%%%%%%%%%%%%%%%%%%%%%%%%%%%%%%%%%%%%%%%%%%%%%
\begin{figure}[!tb]
\centering
%\includegraphics[height=3in, width=1.5in]{NodeArchs2.pdf}
\includegraphics[height=1.5in, width=3.5in]{Images/IPC.pdf}
%\epsfig{file=Dataset.eps, height=2.5in, width=3in}
\caption{Throughput of a shared, solo and CAGE cache allocation schemes.}
\label{fig:IPC_mix}
\end{figure}
%%%%%%%%%%%%%%%%%%%%%%%%%%%%%%%%%%%%%%%%%%%%%%%%%%%%%%%%%%%%%%%%%%%%%%%%%%%
%%%%%%%%%%%%%%%%%%%%%%%%%%%%%%%%%%%%%%%%%%%%%%%%%%%%%%%%%%%%%%%%%%%%%%%%%%%
\begin{figure}[!tb]
\centering
%\includegraphics[height=3in, width=1.5in]{NodeArchs2.pdf}
\includegraphics[height=1.5in, width=3.5in]{Images/Scalability.pdf}
%\epsfig{file=Dataset.eps, height=2.5in, width=3in}
\caption{Performance improvement of CAGE for different number of applications with respect to shared LLC for the case study of cache congestion game.}
\label{fig:scalability}
\end{figure}
%%%%%%%%%%%%%%%%%%%%%%%%%%%%%%%%%%%%%%%%%%%%%%%%%%%%%%%%%%%%%%%%%%%%%%%%%%%
%%%%%%%%%%%%%%%%%%%%%%%%%%%%%%%%%%%%%%%%%%%%%%%%%%%%%%%%%%%%%%%%%%%%%%%%%%%
\begin{comment}
\subsection{Experimental setup}


\newtheorem{defin}{Definition}
\begin{defin}
Let's assume each user $Ui$ in the system is defined as one vertice of a graph $G$ in the system and let each edge in the graph show which subset of users can impact each others' performance and the associated weight of each edge show the cost function of how two users affect each others' performance in the system. Each edge has a weight function denoted by ${P1(n), P2(n), ... Pe(n)}$, where $e$ is the number of edges in Graph $G$. Let $A=A_1 \times A_2 \times ... \times A_n $ be the set of actions that each user can play. 
\end{defin}

This representation of the cache congestion game has a space complexity of which is exponential in terms of larges degree in the graph. 
If the number of users and the action set is polynomially bounded then the game has a polynomial representation and  




For a simple illustrative example assume two applications which would affect each other's performance with $f_i$. The payoff table of a subset of users who can effect each other can be shown in Table 1. We can easily extend the game for $n$ users using $f_1$ and $f_2$ to be a function of the number of users which can impact each other.
We used Gem5, full system simulator running Spec OMP benchmarks to find how different applications sharing a specific amount of shared cache impact each other's performance. 
The application which we did experiments on are listed in Table 2.
\end{comment}  









%\vspace{-0.5\baselineskip}
%\section{Related Work} \label{Related_works}
With rapid improvement in computer technology, more and more cores are embedded in a single chip and applications competing for a shared resource is becoming common. On the one hand, managing scheduling of shared resources for large number of applications is challenging in a sense that the operating system doesn't know what is the performance metric for each application. But on the other hand, the operating system has a global view of the whole state of the system and can guide applications on choosing the shared resources.\\ 
\indent There has been several works, for managing the shared cache in multi-core systems. Qureshi et al. \cite{qureshi2006utility} showed that assigning more cache space to applications with more cache utility does not always lead to better performance since there exists applications with very low cache reuse which may have very high cache utilization. \\
\indent Several software and hardware approaches has been proposed to find the optimal partitioning of cache space for different applications \cite{zhuravlev2010addressing}. However, most of these approaches use brute force search of all possible combinations to find the best cache partitioning in run time or introduce a lot of overhead. There has been some approaches which use binary search to reduce searching all possible combinations \cite{kim2004fair, lin2008gaining, tam2009rapidmrc}. But none of these methods are scalable for the future many-core processor designs.\\
\indent There exists prior game-theoretic approaches designing a centralized scheduling framework that aims at a fair optimization of applications' utility \cite{zahedi2014ref, llull2017cooper, ghodsi2011dominant, zahedi2015sharing, fan2016computational}. Zahedi et al. in REF \cite{zahedi2014ref, zahedi2015sharing} use the Cobb-Douglas production function as a fair allocator for cache and memory bandwidth. They show that the Cobb-Douglas function provides game-theoretic properties such as sharing incentives, envy-freedom, and Pareto efficiency. But their approach is still centralized and spatially divides the shared resources to enforce a fair near-optimal policy sacrificing the performance. In their approach the centralized scheduler assumes all applications have the same priority for cache and memory bandwidth, while we do not have any assumption on this. Further, our auction-based resource allocation can be used for any number of resources and any priority for each application and the centralized scheduler does not need to have a global knowledge of these priorities.  \\
\indent Ghodi et al. in DRF \cite{ghodsi2011dominant} use another centralized fair policy to maximize the dominant resource utilization. But in practice it is not possible to clone any number of instances of each resources. %  the underlying scenario cloning the instances is very limited or superficial for practical purposes. 
Cooper \cite{llull2017cooper} enhances REF to capture colocated applications fairly, but it only addresses the special case of having two sets of applications with matched resources. Fan et al. \cite{fan2016computational} exploits computational sprinting architecture to improve task throughput assuming a class of applications where boosting their performance by increasing the power. \\
\indent While all prior works use a centralized scheduling that provides fairness and assume the same utility function for all, co-runners might have completely diverse needs and it is not efficient to use the same fairness/performance policy across them. 
In our auction-based resource scheduling provides scalability since individual applications compete for the shared resources based on their utility and the burden of decision making is removed from the central scheduler. We believe future CMPs should move toward a more decentralized approach which is more scalable and provides fair allocation of resources based on applications' needs. \\ 
%Providing the scalability of the system is getting better, if we make the individual applications our self administrators, and we can remove most of the performance-restricting policies such as fairness constraints from the centralized decision-maker. \\
\indent Auction theory which is a subfield of economics has recently been used as a tool to solve large scale resource assignement in cloud computing \cite{krishna2009auction, parsons2011auctions}. In an auction process, the buyers submit bids to get the commodities and sellers want to sell their commodities with the maximum price as possible. \\
\indent Our auction-based algorithm is inspired by work of Bertsekas \cite{bertsekas1998network} that uses an auction based mechanism for network flow problems. Our algorithm is an extension of local assignment problem proposed by Bertsekas et al that has been shown to converge to the global assignment within a linear approximation.
%\textcolor{red}{Not Complete yet}
%\vspace{-0.5\baselineskip}
%\section{Conclusion}
We have presented a neural performance rendering system to generate high-quality geometry and photo-realistic textures of human-object interaction activities in novel views using sparse RGB cameras only. 
%
Our layer-wise scene decoupling strategy enables explicit disentanglement of human and object for robust reconstruction and photo-realistic rendering under challenging occlusion caused by interactions. 
%
Specifically, the proposed implicit human-object capture scheme with occlusion-aware human implicit regression and human-aware object tracking enables consistent 4D human-object dynamic geometry reconstruction.
%
Additionally, our layer-wise human-object rendering scheme encodes the occlusion information and human motion priors to provide high-resolution and photo-realistic texture results of interaction activities in the novel views.
%
Extensive experimental results demonstrate the effectiveness of our approach for compelling performance capture and rendering in various challenging scenarios with human-object interactions under the sparse setting.
%
We believe that it is a critical step for dynamic reconstruction under human-object interactions and neural human performance analysis, with many potential applications in VR/AR, entertainment,  human behavior analysis and immersive telepresence.




%\vspace{-0.5\baselineskip}
%\newpage
\appendix
\section{Pricing equations}
\subsection{Credit default swap}
\label{CDS_pricing}
A credit default swap (CDS) is a contract designed to exchange credit risk of a Reference Name (RN) between a Protection Buyer (PB) and a Protection Seller (PS). PB makes periodic coupon payments to PS conditional on no default of RN, up to the nearest payment date, in the exchange for receiving from PS the loss given RN's default.

Consider a CDS contract written on the first bank (RN), denote its price $C_1(t, x)$.\footnote{For the CDS contracts written on the second bank, the similar expression could be provided by analogy.} We assume that the coupon is paid continuously and equals to $c$. Then, the value of a standard CDS contract can be given (\cite{BieleckiRutkowski}) by the solution of  (\ref{kolm_1})--(\ref{kolm_2})  with $\chi(t, x) = c$ and terminal condition
\begin{equation*}
	\psi(x) = 
	\begin{cases}
		1 - \min(R_1, \tilde{R}_1(1)), \quad (x_1, x_2) \in D_2, \\
		1 - \min(R_1, \tilde{R}_1(\omega_2)), \quad (x_1, x_2) \in D_{12}, \\		
	\end{cases}
\end{equation*}
where $\omega_2 = \omega_2(x)$ is defined in (\ref{term_cond}) and 
\begin{equation*}
	\tilde{R}_1(\omega_2) = \min \left[1, \frac{A_1(T) +  \omega_2 L_{2 1}(T)}{L_1(T) + \omega_2 L_{12}(T)}\right].
\end{equation*}
Thus, the pricing problem for CDS contract on the first bank is
\begin{equation}
\begin{aligned}
		& \frac{\partial}{\partial t} C_1(t, x) + \mathcal{L} C_1(t, x) = c, \\
		& C_1(t, 0, x_2) = 1 - R_1, \quad C_1(t, \infty, x_2) = -c(T-t), \\
		& C_1(t, x_1, 0) = \Xi(t, x_1) = 
		\begin{cases}
			c_{1,0}(t, x_1), & x_1 \ge \tilde{\mu}_1, \\
			1-R_1, & x_1 < \tilde{\mu}_i,
		\end{cases} \quad C_1(t, x_1, \infty) = c_{1,\infty}(t, x_1),\\
		& C_1(T, x) = \psi(x) = 
	\begin{cases}
		1 - \min(R_1, \tilde{R}_1(1)), \quad (x_1, x_2) \in D_2, \\
		1 - \min(R_1, \tilde{R}_1(\omega_2)), \quad (x_1, x_2) \in D_{12}, \\		
	\end{cases}
\end{aligned}
\end{equation}
where $c_{1,0}(t, x_1)$ is the solution of the following boundary value problem:
\begin{equation}
\begin{aligned}
		& \frac{\partial}{\partial t} c_{1, 0}(t, x_1) + \mathcal{L}_1 c_{1, 0}(t, x_1) = c, \\
		& c_{1, 0}(t, \tilde{\mu}_1^{<}) = 1 - R_1, \quad c_{1, 0}(t, \infty) = -c(T-t), \\
		& c_{1, 0}(T, x_1) = (1 - R_1) \mathbbm{1}_{\{\tilde{\mu}_1^{<} \le x_1 \le \tilde{\mu}_1^{=}\}}, 
\end{aligned}
\end{equation}
and $c_{1,\infty}(t, x_1)$ is the solution of the following boundary value problem
\begin{equation}
\begin{aligned}
		& \frac{\partial}{\partial t} c_{1, \infty}(t, x_1) + \mathcal{L}_1 c_{1, \infty}(t, x_1) = c, \\
		& c_{1, \infty}(t, 0) = 1 - R_1, \quad c_{1, \infty}(t, \infty) = -c(T-t), \\
		& c_{1, \infty}(T, x_1) = (1 - R_1) \mathbbm{1}_{\{x_1 \le \mu_1^{=}\}}.
\end{aligned}
\end{equation}

\subsection{First-to-default swap}
An FTD contract refers to a basket of reference names (RN). Similar to a regular CDS, the Protection Buyer (PB) pays a regular coupon payment $c$ to the Protection Seller (PS) up to the first default of any of the RN in the basket or maturity time $T$. In return, PS compensates PB the loss caused by the first default.

Consider the FTD contract referenced on $2$ banks, and denote its price $F(t, x)$. We assume that the coupon is paid continuously and equals to $c$. Then, the value of FTD contract can be given (\cite{LiptonItkin2015}) by the solution of  (\ref{kolm_1})--(\ref{kolm_2})  with $\chi(t, x) = c$ and terminal condition
\begin{equation*}
	\psi(x) = \beta_0  \mathbbm{1}_{\{x \in D_{12}\}} + \beta_1 \mathbbm{1}_{\{x \in D_{1}\}} + \beta_2 \mathbbm{1}_{\{x \in D_{2}\}},
\end{equation*}
where
\begin{equation*}
	\begin{aligned}
		\beta_0 = 1 - \min[\min(R_1, \tilde{R}_1(\omega_2), \min(R_2, \tilde{R}_2(\omega_1)], \\
		\beta_1 = 1 - \min(R_2, \tilde{R}_2(1)), \quad \beta_2 = 1 - \min(R_1, \tilde{R}_1(1)),
	\end{aligned}
\end{equation*}
and
\begin{equation*}
	\tilde{R}_1(\omega_2) = \min \left[1, \frac{A_1(T) +  \omega_2 L_{2 1}(T)}{L_1(T) + \omega_2 L_{12}(T)}\right], \quad \tilde{R}_2(\omega_1) = \min \left[1, \frac{A_2(T) +  \omega_1 L_{1 2}(T)}{L_2(T) + \omega_1 L_{21}(T)}\right].
\end{equation*}
with $\omega_1 = \omega_1(x)$ and $\omega_2 = \omega_2(x)$ defined in (\ref{term_cond}).

Thus, the pricing problem for a FTD contract is
\begin{equation}
\begin{aligned}
		& \frac{\partial}{\partial t} F(t, x) + \mathcal{L} F(t, x) = c, \\
		& F(t, x_1, 0) = 1 - R_2,  \quad F(t, 0, x_2) = 1 - R_1, \\
		& F(t, x_1, \infty) = f_{2,\infty}(t, x_1), \quad F(t, \infty, x_2) = f_{1,\infty}(t, x_2), \\
		& F(T, x) = \beta_0  \mathbbm{1}_{\{x \in D_{12}\}} + \beta_1 \mathbbm{1}_{\{x \in D_{1}\}} + \beta_2 \mathbbm{1}_{\{x \in D_{2}\}},
\end{aligned}
\end{equation}
where $f_{1,\infty}(t, x_1)$ and $f_{2,\infty}(t, x_2)$ are the solutions of the following boundary value problems
\begin{equation}
\begin{aligned}
		& \frac{\partial}{\partial t} f_{i, \infty}(t, x_i) + \mathcal{L}_i f_{i, \infty}(t, x_i) = c, \\
		& f_{i, \infty}(t, 0) = 1 - R_i, \quad f_{i, \infty}(t, \infty) = -c(T-t), \\
		& f_{1, \infty}(T, x_i) = (1 - R_i) \mathbbm{1}_{\{x_i \le \mu_i^{=}\}}.
\end{aligned}
\end{equation}

\subsection{Credit and Debt Value Adjustments for CDS}

Credit Value Adjustment and Debt Value Adjustment can be considered either unilateral or bilateral. For unilateral counterparty risk, we need to consider only two banks (RN, and PS for CVA and PB for DVA), and a two-dimensional problem can be formulated, while bilateral counterparty risk requires a three-dimensional problem, where Reference Name, Protection Buyer, and Protection Seller are all taken into account. We follow \cite{LiptonSav} for the pricing problem formulation but include jumps and mutual liabilities, which affects the boundary conditions.

\paragraph{Unilateral CVA and DVA}
The Credit Value Adjustment represents the additional price associated with the possibility of a counterparty's default. Then, CVA can be defined as
\begin{equation}
	V^{CVA} = (1- R_{PS}) \mathbb{E}[\mathbbm{1}_{\{\tau^{PS} < \min(T, \tau^{RN}) \}} (V_{\tau^{PS}}^{CDS})^{+} \, | \mathcal{F}_t],
\end{equation}
where $R_{PS}$ is the recovery rate of PS, $\tau^{PS}$ and $\tau^{RN}$ are the default times of PS and RN, and $V_t^{CDS}$ is the price of a CDS without counterparty credit risk.

We associate $x_1$ with the Protection Seller and $x_2$ with the Reference Name, then CVA can be given by the solution of  (\ref{kolm_1})--(\ref{kolm_2})  with $\chi(t, x) = 0$ and $\psi(x) = 0$. Thus,
\begin{equation}
\begin{aligned}
		& \frac{\partial}{\partial t} V^{CVA}+ \mathcal{L} V^{CVA} = 0, \\
		& V^{CVA}(t, 0, x_2) = (1 - R_{PS}) V^{CDS}(t, x_2)^{+}, \quad V^{CVA}(t, x_1, 0) = 0, \\
		& V^{CVA}(T, x_1, x_2) = 0.
\end{aligned}
\end{equation}

Similar, Debt Value Adjustment represents the additional price associated with the default and defined as
\begin{equation}
	V^{DVA} = (1- R_{PB}) \mathbb{E}[\mathbbm{1}_{\{\tau^{PB} < \min(T, \tau^{RN}) \}} (V_{\tau^{PB}}^{CDS})^{-} \, | \mathcal{F}_t],
\end{equation}
where $R_{PB}$ and $\tau^{PB}$ are the recovery rate and default time of the protection buyer.

Here, we associate $x_1$ with the Protection Buyer and $x_2$ with the Reference Name, then, similar to CVA,  DVA can be given by the solution of  (\ref{kolm_1})--(\ref{kolm_2}),
\begin{equation}
\begin{aligned}
		& \frac{\partial}{\partial t} V^{DVA}+ \mathcal{L} V^{DVA} = 0, \\
		& V^{DVA}(t, 0, x_2) = (1 - R_{PB}) V^{CDS}(t, x_2)^{-}, \quad V^{DVA}(t, x_1, 0) = 0, \\
		& V^{DVA}(T, x_1, x_2) = 0.
\end{aligned}
\end{equation}

\paragraph{Bilateral CVA and DVA}

When we defined unilateral CVA and DVA, we assumed that either protection  buyer, or protection seller are risk-free. Here we assume that they are both risky. Then, 
The Credit Value Adjustment represents the additional price associated with the possibility of counterparty's default and defined as
\begin{equation}
	V^{CVA} = (1 - R_{PS}) \mathbb{E}[\mathbbm{1}_{\{\tau^{PS} < \min(\tau^{PB}, \tau^{RN}, T)\}} (V^{CDS}_{\tau^{PS}})^{+} \, | \mathcal{F}_t],
\end{equation} 

Similar, for DVA
\begin{equation}
	V^{DVA} = (1 - R_{PB}) \mathbb{E}[\mathbbm{1}_{\{\tau^{PB} < \min(\tau^{PS}, \tau^{RN}, T)\}} (V^{CDS}_{\tau^{PB}})^{-} \, | \mathcal{F}_t],
\end{equation} 


We associate $x_1$ with protection seller, $x_2$ with protection buyer, and $x_3$ with reference name. Here, we have a three-dimensional process. Applying three-dimensional version of (\ref{kolm_1})--(\ref{kolm_2}) with $\psi(x) = 0, \chi(t, x) = 0$, we get
\begin{equation}
	\label{CVA_pde}
\begin{aligned}
		& \frac{\partial}{\partial t} V^{CVA} + \mathcal{L}_3 V^{CVA} = 0, \\
		& V^{CVA}(t, 0, x_2, x_3) = (1 - R_{PS}) V^{CDS}(t, x_3)^{+}, \\
		& V^{CVA}(t, x_1, 0, x_3 ) = 0, \quad V^{CVA}(t, x_1, x_2, 0)  = 0, \\
		& V^{CVA}(T, x_1, x_2, x_3) = 0,
\end{aligned}
\end{equation}
and
\begin{equation}
\label{DVA_pde}
\begin{aligned}
		& \frac{\partial}{\partial t} V^{DVA} + \mathcal{L}_3 V^{DVA} = 0, \\
		& V^{DVA}(t, 0, x_2, x_3) = (1 - R_{PB}) V^{CDS}(t, x_3)^{-}, \\
		& V^{DVA}(t, x_1, 0, x_3 ) = 0, \quad V^{DVA}(t, x_1, x_2, 0)  = 0, \\
		& V^{DVA}(T, x_1, x_2, x_3) = 0,
\end{aligned}
\end{equation}
where $\mathcal{L}_3 f$ is the three-dimensional infinitesimal generator.



%\vspace{-0.5\baselineskip}

%%%%%%% -- PAPER CONTENT ENDS -- %%%%%%%%


%%%%%%%%% -- BIB STYLE AND FILE -- %%%%%%%%
%\bibliographystyle{ieeetr}
%\bibliography{ref}
%%%%%%%%%%%%%%%%%%%%%%%%%%%%%%%%%%%%
\bibliographystyle{unsrt}{%\small  %unsrt
\bibliography{bib/IEEEconf}
} 
%%%%%%%%%%%%%%%%%%%%%%%%%%%%%%%%%%%%%%%%%%%%%%%%%%%%%%%%%%%%%%%%%%%%%%%%%%%%%%%%%%%%%%%%%%
\begin{IEEEbiography}[{\includegraphics[width=1in,height=1.25in,clip,keepaspectratio]{Images/Farshid.jpg}}]{Farshid Farhat} is a PhD candidate at the School of Electrical Engineering and Computer Science, The Pennsylvania State University. He obtained his B.Sc., M.Sc., and Ph.D. degrees in Electrical Engineering from Sharif University of Technology, Tehran, Iran. His current research interests include resource allocation in parallel, distributed systems, computer vision and image processing. He is working on the modeling and analysis of image composition and aesthetics
using deep learning and image retrieval on different platforms ranging from smartphones to high-performance
computing clusters.%received his B.S. and M.S. degrees in electrical engineering from Sharif University of Technology. He is currently a Ph.D candidate at the EECS department, Pennsylvania State University. His current research interests include resource allocation in parallel and distributed systems.  
\end{IEEEbiography}
%%%%%%%%%%%%%%%%%%%%%%%%%%%%%%%%%%%%%%%%%%%%%%%%%%%%%%%%%%%%%%%%%%%%%%%%%%%%%%%%%%%%%%%%%%%
%\newpage
\appendix
\section{Pricing equations}
\subsection{Credit default swap}
\label{CDS_pricing}
A credit default swap (CDS) is a contract designed to exchange credit risk of a Reference Name (RN) between a Protection Buyer (PB) and a Protection Seller (PS). PB makes periodic coupon payments to PS conditional on no default of RN, up to the nearest payment date, in the exchange for receiving from PS the loss given RN's default.

Consider a CDS contract written on the first bank (RN), denote its price $C_1(t, x)$.\footnote{For the CDS contracts written on the second bank, the similar expression could be provided by analogy.} We assume that the coupon is paid continuously and equals to $c$. Then, the value of a standard CDS contract can be given (\cite{BieleckiRutkowski}) by the solution of  (\ref{kolm_1})--(\ref{kolm_2})  with $\chi(t, x) = c$ and terminal condition
\begin{equation*}
	\psi(x) = 
	\begin{cases}
		1 - \min(R_1, \tilde{R}_1(1)), \quad (x_1, x_2) \in D_2, \\
		1 - \min(R_1, \tilde{R}_1(\omega_2)), \quad (x_1, x_2) \in D_{12}, \\		
	\end{cases}
\end{equation*}
where $\omega_2 = \omega_2(x)$ is defined in (\ref{term_cond}) and 
\begin{equation*}
	\tilde{R}_1(\omega_2) = \min \left[1, \frac{A_1(T) +  \omega_2 L_{2 1}(T)}{L_1(T) + \omega_2 L_{12}(T)}\right].
\end{equation*}
Thus, the pricing problem for CDS contract on the first bank is
\begin{equation}
\begin{aligned}
		& \frac{\partial}{\partial t} C_1(t, x) + \mathcal{L} C_1(t, x) = c, \\
		& C_1(t, 0, x_2) = 1 - R_1, \quad C_1(t, \infty, x_2) = -c(T-t), \\
		& C_1(t, x_1, 0) = \Xi(t, x_1) = 
		\begin{cases}
			c_{1,0}(t, x_1), & x_1 \ge \tilde{\mu}_1, \\
			1-R_1, & x_1 < \tilde{\mu}_i,
		\end{cases} \quad C_1(t, x_1, \infty) = c_{1,\infty}(t, x_1),\\
		& C_1(T, x) = \psi(x) = 
	\begin{cases}
		1 - \min(R_1, \tilde{R}_1(1)), \quad (x_1, x_2) \in D_2, \\
		1 - \min(R_1, \tilde{R}_1(\omega_2)), \quad (x_1, x_2) \in D_{12}, \\		
	\end{cases}
\end{aligned}
\end{equation}
where $c_{1,0}(t, x_1)$ is the solution of the following boundary value problem:
\begin{equation}
\begin{aligned}
		& \frac{\partial}{\partial t} c_{1, 0}(t, x_1) + \mathcal{L}_1 c_{1, 0}(t, x_1) = c, \\
		& c_{1, 0}(t, \tilde{\mu}_1^{<}) = 1 - R_1, \quad c_{1, 0}(t, \infty) = -c(T-t), \\
		& c_{1, 0}(T, x_1) = (1 - R_1) \mathbbm{1}_{\{\tilde{\mu}_1^{<} \le x_1 \le \tilde{\mu}_1^{=}\}}, 
\end{aligned}
\end{equation}
and $c_{1,\infty}(t, x_1)$ is the solution of the following boundary value problem
\begin{equation}
\begin{aligned}
		& \frac{\partial}{\partial t} c_{1, \infty}(t, x_1) + \mathcal{L}_1 c_{1, \infty}(t, x_1) = c, \\
		& c_{1, \infty}(t, 0) = 1 - R_1, \quad c_{1, \infty}(t, \infty) = -c(T-t), \\
		& c_{1, \infty}(T, x_1) = (1 - R_1) \mathbbm{1}_{\{x_1 \le \mu_1^{=}\}}.
\end{aligned}
\end{equation}

\subsection{First-to-default swap}
An FTD contract refers to a basket of reference names (RN). Similar to a regular CDS, the Protection Buyer (PB) pays a regular coupon payment $c$ to the Protection Seller (PS) up to the first default of any of the RN in the basket or maturity time $T$. In return, PS compensates PB the loss caused by the first default.

Consider the FTD contract referenced on $2$ banks, and denote its price $F(t, x)$. We assume that the coupon is paid continuously and equals to $c$. Then, the value of FTD contract can be given (\cite{LiptonItkin2015}) by the solution of  (\ref{kolm_1})--(\ref{kolm_2})  with $\chi(t, x) = c$ and terminal condition
\begin{equation*}
	\psi(x) = \beta_0  \mathbbm{1}_{\{x \in D_{12}\}} + \beta_1 \mathbbm{1}_{\{x \in D_{1}\}} + \beta_2 \mathbbm{1}_{\{x \in D_{2}\}},
\end{equation*}
where
\begin{equation*}
	\begin{aligned}
		\beta_0 = 1 - \min[\min(R_1, \tilde{R}_1(\omega_2), \min(R_2, \tilde{R}_2(\omega_1)], \\
		\beta_1 = 1 - \min(R_2, \tilde{R}_2(1)), \quad \beta_2 = 1 - \min(R_1, \tilde{R}_1(1)),
	\end{aligned}
\end{equation*}
and
\begin{equation*}
	\tilde{R}_1(\omega_2) = \min \left[1, \frac{A_1(T) +  \omega_2 L_{2 1}(T)}{L_1(T) + \omega_2 L_{12}(T)}\right], \quad \tilde{R}_2(\omega_1) = \min \left[1, \frac{A_2(T) +  \omega_1 L_{1 2}(T)}{L_2(T) + \omega_1 L_{21}(T)}\right].
\end{equation*}
with $\omega_1 = \omega_1(x)$ and $\omega_2 = \omega_2(x)$ defined in (\ref{term_cond}).

Thus, the pricing problem for a FTD contract is
\begin{equation}
\begin{aligned}
		& \frac{\partial}{\partial t} F(t, x) + \mathcal{L} F(t, x) = c, \\
		& F(t, x_1, 0) = 1 - R_2,  \quad F(t, 0, x_2) = 1 - R_1, \\
		& F(t, x_1, \infty) = f_{2,\infty}(t, x_1), \quad F(t, \infty, x_2) = f_{1,\infty}(t, x_2), \\
		& F(T, x) = \beta_0  \mathbbm{1}_{\{x \in D_{12}\}} + \beta_1 \mathbbm{1}_{\{x \in D_{1}\}} + \beta_2 \mathbbm{1}_{\{x \in D_{2}\}},
\end{aligned}
\end{equation}
where $f_{1,\infty}(t, x_1)$ and $f_{2,\infty}(t, x_2)$ are the solutions of the following boundary value problems
\begin{equation}
\begin{aligned}
		& \frac{\partial}{\partial t} f_{i, \infty}(t, x_i) + \mathcal{L}_i f_{i, \infty}(t, x_i) = c, \\
		& f_{i, \infty}(t, 0) = 1 - R_i, \quad f_{i, \infty}(t, \infty) = -c(T-t), \\
		& f_{1, \infty}(T, x_i) = (1 - R_i) \mathbbm{1}_{\{x_i \le \mu_i^{=}\}}.
\end{aligned}
\end{equation}

\subsection{Credit and Debt Value Adjustments for CDS}

Credit Value Adjustment and Debt Value Adjustment can be considered either unilateral or bilateral. For unilateral counterparty risk, we need to consider only two banks (RN, and PS for CVA and PB for DVA), and a two-dimensional problem can be formulated, while bilateral counterparty risk requires a three-dimensional problem, where Reference Name, Protection Buyer, and Protection Seller are all taken into account. We follow \cite{LiptonSav} for the pricing problem formulation but include jumps and mutual liabilities, which affects the boundary conditions.

\paragraph{Unilateral CVA and DVA}
The Credit Value Adjustment represents the additional price associated with the possibility of a counterparty's default. Then, CVA can be defined as
\begin{equation}
	V^{CVA} = (1- R_{PS}) \mathbb{E}[\mathbbm{1}_{\{\tau^{PS} < \min(T, \tau^{RN}) \}} (V_{\tau^{PS}}^{CDS})^{+} \, | \mathcal{F}_t],
\end{equation}
where $R_{PS}$ is the recovery rate of PS, $\tau^{PS}$ and $\tau^{RN}$ are the default times of PS and RN, and $V_t^{CDS}$ is the price of a CDS without counterparty credit risk.

We associate $x_1$ with the Protection Seller and $x_2$ with the Reference Name, then CVA can be given by the solution of  (\ref{kolm_1})--(\ref{kolm_2})  with $\chi(t, x) = 0$ and $\psi(x) = 0$. Thus,
\begin{equation}
\begin{aligned}
		& \frac{\partial}{\partial t} V^{CVA}+ \mathcal{L} V^{CVA} = 0, \\
		& V^{CVA}(t, 0, x_2) = (1 - R_{PS}) V^{CDS}(t, x_2)^{+}, \quad V^{CVA}(t, x_1, 0) = 0, \\
		& V^{CVA}(T, x_1, x_2) = 0.
\end{aligned}
\end{equation}

Similar, Debt Value Adjustment represents the additional price associated with the default and defined as
\begin{equation}
	V^{DVA} = (1- R_{PB}) \mathbb{E}[\mathbbm{1}_{\{\tau^{PB} < \min(T, \tau^{RN}) \}} (V_{\tau^{PB}}^{CDS})^{-} \, | \mathcal{F}_t],
\end{equation}
where $R_{PB}$ and $\tau^{PB}$ are the recovery rate and default time of the protection buyer.

Here, we associate $x_1$ with the Protection Buyer and $x_2$ with the Reference Name, then, similar to CVA,  DVA can be given by the solution of  (\ref{kolm_1})--(\ref{kolm_2}),
\begin{equation}
\begin{aligned}
		& \frac{\partial}{\partial t} V^{DVA}+ \mathcal{L} V^{DVA} = 0, \\
		& V^{DVA}(t, 0, x_2) = (1 - R_{PB}) V^{CDS}(t, x_2)^{-}, \quad V^{DVA}(t, x_1, 0) = 0, \\
		& V^{DVA}(T, x_1, x_2) = 0.
\end{aligned}
\end{equation}

\paragraph{Bilateral CVA and DVA}

When we defined unilateral CVA and DVA, we assumed that either protection  buyer, or protection seller are risk-free. Here we assume that they are both risky. Then, 
The Credit Value Adjustment represents the additional price associated with the possibility of counterparty's default and defined as
\begin{equation}
	V^{CVA} = (1 - R_{PS}) \mathbb{E}[\mathbbm{1}_{\{\tau^{PS} < \min(\tau^{PB}, \tau^{RN}, T)\}} (V^{CDS}_{\tau^{PS}})^{+} \, | \mathcal{F}_t],
\end{equation} 

Similar, for DVA
\begin{equation}
	V^{DVA} = (1 - R_{PB}) \mathbb{E}[\mathbbm{1}_{\{\tau^{PB} < \min(\tau^{PS}, \tau^{RN}, T)\}} (V^{CDS}_{\tau^{PB}})^{-} \, | \mathcal{F}_t],
\end{equation} 


We associate $x_1$ with protection seller, $x_2$ with protection buyer, and $x_3$ with reference name. Here, we have a three-dimensional process. Applying three-dimensional version of (\ref{kolm_1})--(\ref{kolm_2}) with $\psi(x) = 0, \chi(t, x) = 0$, we get
\begin{equation}
	\label{CVA_pde}
\begin{aligned}
		& \frac{\partial}{\partial t} V^{CVA} + \mathcal{L}_3 V^{CVA} = 0, \\
		& V^{CVA}(t, 0, x_2, x_3) = (1 - R_{PS}) V^{CDS}(t, x_3)^{+}, \\
		& V^{CVA}(t, x_1, 0, x_3 ) = 0, \quad V^{CVA}(t, x_1, x_2, 0)  = 0, \\
		& V^{CVA}(T, x_1, x_2, x_3) = 0,
\end{aligned}
\end{equation}
and
\begin{equation}
\label{DVA_pde}
\begin{aligned}
		& \frac{\partial}{\partial t} V^{DVA} + \mathcal{L}_3 V^{DVA} = 0, \\
		& V^{DVA}(t, 0, x_2, x_3) = (1 - R_{PB}) V^{CDS}(t, x_3)^{-}, \\
		& V^{DVA}(t, x_1, 0, x_3 ) = 0, \quad V^{DVA}(t, x_1, x_2, 0)  = 0, \\
		& V^{DVA}(T, x_1, x_2, x_3) = 0,
\end{aligned}
\end{equation}
where $\mathcal{L}_3 f$ is the three-dimensional infinitesimal generator.



%%%%%%%%%%%%%%%%%%%%%%%%%%%%%%%%%%%%%%%%%%%%%%%%%%%%%%%%%%%%%%%%%%%%%%%%%%%%%%%%%%%%%%%%%%
\begin{IEEEbiography}[{\includegraphics[width=1in,height=1.25in,clip,keepaspectratio]{Images/Retouched04.jpg}}]{Diman Zad Tootaghaj} is a Ph.D. student in the department of computer science and engineering at the Pennsylvania State University. She received B.S. and M.S. degrees in Electrical Engineering from Sharif University of Technology, Iran in 2008 and 2011 and a M.S. degree in Computer Science and Engineering from the Pennsylvania State University in 2015. She is a member of Institute for Networking and Security Research (INSR) under supervision of Prof. Thomas La Porta (advisor), Dr. Ting He (co-advisor), and Dr. Novella Bartolini. Her current research interests include computer network, recovery approaches, distributed systems, and stochastic analysis.
\end{IEEEbiography}

% that's all folks
\end{document}


