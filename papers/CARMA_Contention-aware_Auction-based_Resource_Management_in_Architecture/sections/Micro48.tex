%%%%%%%%%%%%%%%%%%%%%%%%%%%%%%%%%%%%
% This is the template for submission to MICRO 2015
% The cls file is a modified from  'sig-alternate.cls'
%%%%%%%%%%%%%%%%%%%%%%%%%%%%%%%%%%%%

\documentclass{sig-alternate}

\newcommand{\ignore}[1]{}
\usepackage{fancyhdr}
\usepackage[normalem]{ulem}
\usepackage[hyphens]{url}
\usepackage{hyperref}
%DIMAN

\pagenumbering{arabic} 
\usepackage{cite}
\usepackage{url}
\usepackage{parskip}
\usepackage{indentfirst} %indents first line of each section
\usepackage{color}
\setlength{\parindent}{10pt}
\usepackage{graphicx}
\usepackage{caption}
\usepackage{subcaption}
\usepackage{diagbox}
\usepackage[linesnumbered,ruled,vlined]{algorithm2e}
\usepackage{natbib}
\setlength{\bibsep}{0.0pt}
\usepackage{flushend}
\usepackage{mathtools}
%\usepackage{amsthm}
\usepackage{amsmath}
\usepackage{verbatim}
\usepackage{pbox}

%\usepackage{enumitem}
%DIMAN
%%%%%%%%%%%---SETME-----%%%%%%%%%%%%%
\newcommand{\microsubmissionnumber}{XXX}
%%%%%%%%%%%%%%%%%%%%%%%%%%%%%%%%%%%%

\fancypagestyle{firstpage}{
  \fancyhf{}
\setlength{\headheight}{50pt}
\renewcommand{\headrulewidth}{0pt}
  \fancyhead[C]{\normalsize{MICRO 2015 Submission
      \textbf{\#\microsubmissionnumber} -- Confidential Draft -- Do NOT Distribute!!}} 
  \pagenumbering{arabic}
}  

%%%%%%%%%%%---SETME-----%%%%%%%%%%%%%
\title{CAGE: A Market-based \underline{C}ontention-\underline{A}ware \underline{G}am\underline{E}-theoretic Distributed model for heterogeneous resource assignment} 
%%%%%%%%%%%%%%%%%%%%%%%%%%%%%%%%%%%%

\begin{document}
\maketitle
\pagenumbering{arabic}
\thispagestyle{plain}
%\thispagestyle{firstpage}
\pagestyle{plain}



%%%%%% -- PAPER CONTENT STARTS-- %%%%%%%%

\begin{abstract}
\noindent Traditional resource management systems rely on a centralized approach to manage users running on each resource. The centralized resource management system is not scalable for large-scale servers as the number of users running on shared resources is increasing dramatically and the centralized manager may not have enough information about applications' need. In this paper we propose a distributed game-theoretic resource management approach using market auction mechanism to find optimal strategy in a resource competition game. The applications learn through repeated interactions to choose their action on choosing the shared resources. Specifically, we look into two case studies of cache competition game and main processor and co-processor congestion game. We enforce costs for each resource and derive bidding strategy. Accurate evaluation of the proposed approach show that our distributed allocation is scalable and outperforms the static and traditional approaches.
\end{abstract}
%%%%%%%%%%%%%%%%%%%%%%%%%%%%%%%%%%%%%%%%%%%%%%%%%%%%%%%%%%%%%%%%%%%%%%%%%%%
%%%%%%%%%%%%%%%%%%%%%%%%%%%%%%%%%%%%%%%%%%%%%%%%%%%%%%%%%%%%%%%%%%%%%%%%%%%
\section{Introduction}
\label{sec:Introduction}


The goal in top-$\size$ recommendation is to recommend to each
consumer a small set of $\size$ items from a large collection of
items~\cite{cremonesi2010performance}.  For example, Netflix may want
to recommend $\size$ appealing movies to each consumer.  Collaborative
Filtering (CF)~\cite{herlocker2002empirical,lee2012comparative} is a
common top-$\size$ recommendation method.  CF infers user interests by
analyzing partially observed user-item interaction data, such as user
ratings on movies or historical purchase
logs~\cite{kanagal2012supercharging}. The main assumption in CF is that
users with similar interaction patterns have similar interests.


Standard CF methods for top-$\size$ recommendation focus on making  suggestions  that accurately reflect the user's preference history. However, as  observed in previous work,  CF recommendations are generally biased toward  popular items, leading to a rich get richer effect~\cite{vargas2014improving,steck2011item}.  The major reasons for this are \textit{popularity bias} and \textit{sparsity} of CF interaction data (detailed in Section~\ref{sec:related-work}). In a nutshell, to maintain  accuracy, recommendations are generated from the dense regions of the data,  where the popular items lie.  

However,  accurately suggesting popular items, may not be satisfactory for the consumers. For example, in Netflix, an accuracy-focused movie recommender may recommend ``Star Wars: The Force Awakens'' to users who have seen ``Star Wars: Rogue One''.  But, those users are probably already aware of ``The Force Awakens''. Considering additional factors, such as novelty of recommendations,  can lead to more effective suggestions~\cite{cremonesi2010performance,Castells2015,zhang2008avoiding,ziegler2005improving,zhang2012auralist}. 
%Second, accuracy-focused models typically achieve a   overall item-space coverage across their recommendations,  whereas high item-space coverage helps providers of the items increase revenue
%, users satisfaction since they are  likely already aware of or can find these items on their own.  

Focusing on popular items also adversely affects the satisfaction of  the providers of the items. This is because  accuracy-focused models typically achieve a  low overall item space coverage across their recommendations, whereas   high item space coverage helps providers of the items increase their revenue~\cite{vargas2014improving,Castells2015,adomavicius2011maximizing,anderson2006thelongtail, yin2012challenging,adomavicius2012improving}.
%accuracy-focused models typically achieve a

In contrast to the relatively small number of popular items, there are copious  {\it long-tail\/} items that have fewer observations (e.g., ratings) available. More precisely,  using the Pareto  principle (i.e.,~the $80/20$ rule),  long-tail items can be defined as items that generate the lower $20\%$ of observations~\cite{yin2012challenging}. Experimentally we found that these items correspond to almost $85\%$ of the items in several datasets (Sections~\ref{sec:Notation} and \ref{sec:Experiments}). %Table~\ref{tab:DatasetStatsticsSmall})


As previously shown, one way to improve the novelty of top-$\size$ sets is to recommend interesting long-tail items~\cite{cremonesi2010performance,ge2010beyond}.  The intuition  is that since they have fewer observations available,  they are more likely to be unseen~\cite{Kaminskas:2016:DSN:3028254.2926720}.  
 %For example, in online commerce,  newly added items are long-tail items that are yet to be discovered.  
Moreover, long-tail item promotion also results in higher overall coverage of the item space%, which increases profits for providers of the items
~\cite{vargas2014improving,Castells2015,zhang2008avoiding,zhang2012auralist,adomavicius2011maximizing,anderson2006thelongtail,yin2012challenging,jambor2010optimizing}. Because long-tail promotion reduces accuracy~\cite{steck2011item}, there are trade-offs to be explored.


%original submitted to ICDE
%This work studies three aspects of top-$\size$ recommendation: accuracy, novelty, and item-space coverage, and examines their trade-offs. In most previous work, predictions of a base recommendation system are re-ranked to handle their trade-offs~\cite{adomavicius2012improving,jambor2010optimizing,zhang2013personalize,wang2009portfolio}. Due to performance considerations, however, these techniques are not customized per user. For example,  parameters that balance the trade-off between novelty and accuracy are cross-validated at a global level.  This can be detrimental since users have varying preferences for  objectives such as long-tail novelty. We explore how to  automatically infer  user  preference for long-tail novelty, and how to leverage  it to correct  the popularity bias in standard recommender models. Our work does not rely on any additional contextual data, although such data, if available, can help promote newly-added long-tail items~\cite{agarwal2009regression,Saveski:2014:ICR:2645710.2645751}.

This work studies three aspects of top-$\size$ recommendation: accuracy, novelty, and item space coverage, and examines their trade-offs. In most previous work, predictions of a base recommendation algorithm are \textit{re-ranked} to handle these trade-offs~\cite{adomavicius2012improving,jambor2010optimizing,zhang2013personalize,wang2009portfolio}. The re-ranking models are computationally efficient but suffer from two drawbacks. First, due to performance considerations,  parameters that balance the trade-off between novelty and accuracy  are not customized per user. Instead they are cross-validated at a global level.  This can be detrimental since users have varying preferences for  objectives such as long-tail novelty. Second,  the re-ranking methods are often limited to a specific base recommender  that may be sensitive to dataset density. 
As a result, the datasets are pruned and the problem is studied in dense settings~\cite{adomavicius2012improving,ho2014likes}; but real world  scenarios are often sparse~\cite{kanagal2012supercharging,liu2017experimental}.   
% Because  dataset density can impact the performance of most base recommenders (like R-SVD), which in turn affects the performance of the re-ranking model, 

\iffalse
We address these limitations by directly inferring  user  preference for long-tail novelty  from interaction data.  This  allows us to customize the re-ranking  per user, and design a \textit{generic} framework, which resolves the second problem. In particular, since the long-tail novelty preferences are estimated independently of any base  recommender model, we can  plug-in an appropriate base recommender w.r.t. the dataset sparsity.% including ones that are more suitable for sparse settings.  

Modelling  user  preference for  long-tail novelty using only item popularity statistics, e.g., the average popularity of rated items as in~\cite{jugovac2017efficient}, disregards additional information like whether the user found the item interesting and the long-tail preferences of other users  of the items. \iffalse To incorporate them, we introduce the notion of  \emph{item long-tail importance}. Both  user long-tail preferences and item long-tail importance are dependent:  a user has high preference for discovering long-tail items if she is interested in important long-tail items, and an item that is associated with many of these kinds of users is likely to be more important.  We propose a joint optimization framework to directly learn,  from interaction data, both the users' long-tail preferences and the  items' long-tail importance. \fi
We propose an optimization approach that  incorporates  this information and  directly learns,  from interaction data, the users' long-tail novelty preferences.

Next, we use these learned preferences  to design a  top-$\size$ recommendation framework thats is generic, and provides customized balance between accuracy, novelty, and coverage. We refer to it as framework as GANC.  Using GANC, we design a novel algorithm, {\it Ordered Sampling-based Locally Greedy (OSLG)\/}, that relies on the learned long-tail novelty preferences  to scalably correct for popularity bias. Our work does not rely on any additional contextual data, although such data, if available, can help promote newly-added long-tail items~\cite{agarwal2009regression,Saveski:2014:ICR:2645710.2645751}. In summary:
\fi

We address the first limitation by directly inferring  user  preference for long-tail novelty  from interaction data.   Estimating these  preferences  using only item popularity statistics, e.g., the average popularity of rated items as in~\cite{jugovac2017efficient}, disregards additional information, like whether the user found the item interesting or the long-tail preferences of other users  of the items. We propose an approach that  incorporates  this information and  learns the users' long-tail novelty preferences from interaction data.

This approach allows us to customize the re-ranking  per user, and  design a \textit{generic} re-ranking framework, which resolves the second limitation of prior work. In particular, since the long-tail novelty preferences are estimated independently of any base recommender, we can  plug-in an appropriate one w.r.t. different factors, such as the dataset sparsity.

Our top-$\size$ recommendation framework, \textbf{GANC}, is \textbf{G}eneric, and provides customized balance between \textbf{A}ccuracy, \textbf{N}ovelty, and \textbf{C}overage. % Moreover, based on the learned long-tail novelty preferences, we also design a novel algorithm, {\it Ordered Sampling-based Locally Greedy (OSLG)\/}, that relies on the learned long-tail novelty preferences  to scalably correct for popularity bias. 
Our work does not rely on any additional contextual data, although such data, if available, can help promote newly-added long-tail items~\cite{agarwal2009regression,Saveski:2014:ICR:2645710.2645751}. In summary:

%Consider  the following toy example:
\vspace{-0.2cm}
\begin{table}[htb]
\centering
\scriptsize
%\small
\begin{tabular}{ccccccc} 
%\toprule
%&\multirow{2}{*}{}&\multicolumn{7}{c}{Ratings}\\
& & \cellcolor{blue!35}$w_1$ &\cellcolor{blue!18} $w_2$ & $\dots$ &\cellcolor{blue!8} $w_{89}$  &\cellcolor{blue!8} $w_{99}$   
\\
&   &$i_1$&$i_2$&$\dots$&$i_{89}$&$i_{90}$\\ 
\cmidrule(r){3-7} 	 
%\midrule
\cellcolor{red!35}$\theta_1$  &$u_1 $   &5 &   & $\dots$ &  &   \\
\cellcolor{red!28}$\theta_2$  &$u_2$     &5 &    & $\dots$ &  &  \\
 $\theta_3=?$  &$\bf u_3$  &5 &  &   $\dots$ &  &  \\
\cellcolor{red!10}$\theta_4$ & $u_4$  &  &5   & $\dots$ & &\\ 
\cellcolor{red!10}$\theta_5$ & $u_5$  &  & 5  & $\dots$ & &\\ 
$\theta_6=?$  & $\bf u_6$ & &5  &      $\dots$& &  \\ 
 & & $\hdots$  &$\hdots$   &$\hdots$   &$\hdots$   &$\hdots$  \\
%\midrule 
\cmidrule(r){3-7} 	 
\multicolumn{2}{c}{item pop.}  & 3  & 3  & $\dots$ &50&60\\  
%\bottomrule
%$ f_i$    &3  &3  &1  &3  &1  &2  \\  \hline
\end{tabular}
%#.
\caption{Simplified user-item interaction data. The user long-tail novelty preference ($\theta_u$), item long-tail importance weight ($w_i$) are highlighted. Darker colors indicate larger values. } \label{tab:example}
\end{table} 
\vspace{-0.2cm}
\begin{example}  
In Table~\ref{tab:example}, we are interested in estimating $\theta_3$ and $\theta_6$,  the long-tail preference of users $u_3$ and $u_6$ who have each rated a single movie. Additional ratings for other users  are not included here.  Considering only rating information, we observe $i_1$ and $i_2$ are  equally popular $|\mathcal{U}_{i_1}^{\trainset}| = |\mathcal{U}_{i_2}^{\trainset}|=3$, and $r_{31}=5$ and $r_{62}=5$. Using Eq.~\ref{eq:tfidf-risk}  we have $\theta_3 = \theta_6$. However, if we were given the long-tail preferences of the each item's user set, specifically that $u_1$ and $u_2$ have high long-tail preference (darker red), while $u_4$ and $u_5$ have lower long-tail preference (lighter red), we could conclude $i_1$ is a more important long-tail item compared to $i_2$ (indicated by a darker blue shade for $w_1$), and we expect  $\theta_3 \geq \theta_6$.

% On the other hand, if we knew that $u_4$ and $u_5$ have lower long-tail preference, we could conclude $i_2$ is a  less significant long-tail item. Therefore, However, if we  consider the long-tail preferences of other users, we may reason differently.    We need another variable $w_i$ which captures this information. 
%we would conclude that $u_3$ has higher long-tail preference compared to $u_6$, since the users $i_1$ is a more prominent long-tail item. 

% Relying only  on item popularity information, we would  conclude   $u_3$ and $u_6$ have equal long-tail preference, since $i_1$ and $i_2$ are  equally popular. However, considering  the second column,  long-tail preference of users,  long-tail importance for each item,  which captures the long-tail preference of its users. Since  that  both users of $i_1$ have high long-tail preference while  the users of $i_2$ have lower preference,  we may conclude $i_1$ is a more important long-tail item compared to $i_2$. Therefore, $u_3$'s long-tail preference should be at least as large as $u_6$'s preference. Specifically, consider two  items $i_1$ and $i_2$, with the following rating data: $i_1=\{u_1:5, u_2:5, u_3:5 \}$, $i_2=\{u_4:5, u_5:5, u_6:5\}$.  

%Table~\ref{tab:example} shows  simplified rating data. We want an estimate of the long-tail preference of $u_3$ and $u_6$, who have each  rated a single movie.  Relying only  on movie popularity information, we would  conclude   $u_3$ and $u_6$ have similar long-tail preference, since $m_1$ and $m_2$ are  equally popular. However, considering the long-tail preferences of other users of those movies, we may reason differently: since $u_1$ and $u_2$ have high long-tail preference, and $u_4$ and $u_5$ have low long-tail preference, $m_1$ is a more prominent long-tail item compared to $m_2$. Therefore, it is likely that $u_3$ has higher long-tail preference compared to $u_6$.considering the long-tail preferences of other users of those movies, we may reason differently.  For example, 
\label{ex:running}
\end{example}



%------------------------------

\iffalse
\begin{example}
Table~\ref{tab:example} shows rating data for a simplified system. %Note the user-item interaction matrix is sparse.
For this example, we define popular movies as those that have received  three or more ratings; $\{m_1, m_2, m_4\}$ are popular and  $\{m_3, m_5, m_6\}$ are niche movies. We observe $u_1$ and $u_3$  have rated relatively popular movies (risk-averse) while $u_2$ and $u_4$ have rated niche movies (risk-loving). 
\label{ex:running}
\end{example}

\begin{table}[htb]
\centering
\scriptsize
\begin{tabular}{ccccccc} 
\toprule
			&$m_1$ &$m_2$   &$m_3$    &$m_4$   &$m_5$ &$m_6$  \\ \hline 
$u_1 $ &5  &4  & - &-  &-  &-   \\
$u_2$  &-  &-  &-  &-  &5  &5   \\
$u_3$  &-  &4  &-  &5  &-  &-   \\
$u_4$  &-  &-  &3  &-  &-  &4   \\ 
$u_5$  &5  &-  &-  &3  &-  &-   \\ 
$u_6$  &4  &2  &-  &4  &-  &-   \\ 
\bottomrule
%$ f_i$    &3  &3  &1  &3  &1  &2  \\  \hline
\end{tabular}
\caption{User-Movie rating data} \label{tab:example}
\end{table}

It is essential to consider consumer characteristics in designing recommender systems so that they promote long-tail items to the right group of users and spread demand evenly between hit and niche items.  

\fi





%------------------------------
\iffalse
\begin{table}[htb]
\centering
\scriptsize
\begin{tabular}{ccccccc} 
\toprule
			&$m_1$ &$m_2$   &$m_3$    &$m_4$   &$m_5$ &$m_6$  \\ \hline 
$u_1 $ &\textbf{5}  & \textbf{4}  &\textcolor{gray}{ 1.2} &-  &-  &-   \\
$u_2$  &-  &-  &-  &-  & \textbf{5}  &\textbf{5}   \\
$u_3$  &-  &\textbf{4}  &-  &\textbf{5}  &-  &-   \\
$u_4$  &-  &-  &\textbf{3}  &-  &-  &\textbf{4}   \\ 
$u_5$  &\textbf{5}  &-  &-  &\textbf{3}  &-  &-   \\ 
$u_6$  &\textbf{4}  &\textbf{2}  &-  &\textbf{4}  &-  &-   \\ 
\bottomrule
%$ f_i$    &3  &3  &1  &3  &1  &2  \\  \hline
\end{tabular}
\caption{User-Movie rating data} \label{tab:example}
\end{table}
% $\mathcal{P}^1= \{ \mathcal{P}_1^1 \{i_1,i_2,i_3\}, \mathcal{P}_2^1:\{i_2,i_3,i_5\}  \}$
 %$\mathcal{P}^2= \{ \mathcal{P}_1^2: \{i_1,i_2,i_3\}, \mathcal{P}_2^2:\{i_2,i_5,i_6\}  \}$
 %$\mathcal{P}^3= \{ \mathcal{P}_1^3: \{i_7,i_8,i_9\}, \mathcal{P}_2^3:\{i_{10},i_{11},i_{12}\}  \}$
\begin{table}[htb]
\centering
\tiny
\begin{tabular}{ccc} 
\toprule
		&$u_1$&$u_2$  \\ \hline 
$\mathcal{P}^1 $ & $\{i_1,i_2,i_3\}$ & $\{i_2,i_3,i_5\} $ \\
$\mathcal{P}^2$ & $\{i_1,i_2,i_3\}$ & $\{i_2,i_5,i_6\} $ \\
$\mathcal{P}^3$ & $\{i_7,i_8,i_9\}$ & $\{i_{10},i_{11},i_{12} \}$ \\
\bottomrule
%$ f_i$    &3  &3  &1  &3  &1  &2  \\  \hline
\end{tabular}
\caption{Top-$\size$ allocations to users.} \label{tab:paretoExamples}
\end{table}
\fi


\iffalse
When considering long-tail items, it is important to consider consumers' willingness  to explore niche or unpopular items and their propensity towards similar items. In particular, they can be characterized by their  {\it risk degree\/} and {\it focusing degree\/}, respectively.  We compute these estimates  based on historical rating information. The following example further describes these notions in the context of movie rating data. 

\begin{example}  
Table~\ref{tab:example} shows rating data for a simplified system with $6$ users, $6$ movies, and $3$ genres. $m_i^{j}$ implies that movie $m_i$ belongs to genre $j$. Note the user-item interaction matrix is sparse. 
  For this setting, we define popular movies as those that have received  three or more ratings; $\{m_1, m_2, m_4\}$ are popular and  $\{m_3, m_5, m_6\}$ are niche movies. We now profile the users according to their risk and focusing degree. E.g., $u_1$ has rated relatively popular movies belonging to the same genre (risk-averse, high focusing degree); $u_2$ has rated niches movies in the same genre (risk-loving, high focusing degree); $u_3$ has rated popular movies in two different genres (risk-averse, low focusing degree), and $u_4$ has rated niches movies in two different genres (risk-loving, low focusing degree). 
\label{ex:running}
\end{example}
\begin{table}[htb]
\centering
\tiny
\begin{tabular}{ccccccc} 
\toprule
			&$m_1^{1}$ &$m_2^{1}$   &$m_3^{2}$    &$m_4^{3}$   &$m_5^{3}$ &$m_6^{3}$  \\ \hline 
$u_1 $ &5  &4  &-  &-  &-  &-   \\
$u_2$  &-  &-  &-  &-  &5  &5   \\
$u_3$  &-  &4  &-  &5  &-  &-   \\
$u_4$  &-  &-  &3  &-  &-  &4   \\ 
$u_5$  &5  &-  &-  &3  &-  &-   \\ 
$u_6$  &4  &2  &-  &4  &-  &-   \\ 
\bottomrule
%$ f_i$    &3  &3  &1  &3  &1  &2  \\  \hline
\end{tabular}
\caption{User-Movie rating data} \label{tab:example}
\end{table}
It is essential to consider these consumer characteristics in designing recommender systems so that they promote long-tail items to the right group of users and spread demand evenly between the hit and niche items.  
\fi
\iffalse
\begin{center}
\begin{figure*}[tp]
%\scalebox{0.5}{%
\resizebox{1\textwidth}{!}{%
%\small%\addtolength{\tabcolsep}{5pt}% below sums to 8
\begin{tabularx}{1.5\textwidth}{>{\hsize=2.5\hsize}X>{\hsize=2.5\hsize}X>{\hsize=0.5\hsize}X>{\hsize=0.5\hsize}X>{\hsize=0.5\hsize}X>{\hsize=0.5\hsize}X>{\hsize=0.5\hsize}X>{\hsize=0.5\hsize}X}
    \multirow{12}{*}{\includegraphics[scale=0.3]{codeForExample/popularity-movie.png}} & \multirow{12}{*}{\includegraphics[scale=0.3]{codeForExample/scatterplot.png}} & & & & & & \\
%   & &               &       &       &       &       &       \\
    & &\multicolumn{1}{l|}{}               &$m_1^{g1}$   	&$m_2^{g1}$    	&$m_3^{g2}$    &$m_4^{g2}$      &$m_5^{g3}$    \\ \cline{3-8}%\hline
    & &\multicolumn{1}{l|}{u1}          &5  &5  &-  &-   &-  \\
    & &\multicolumn{1}{l|}{u2}    		&-  &-  &4  &4  &5  \\
    & &\multicolumn{1}{l|}{u3}   			&1  &2  &1  &-  &-   \\
    & &\multicolumn{1}{l|}{u4}     		&1  &-  &-  &-  &-  \\
    & &               &       &       &       &       &       \\
    & &               &       &       &       &       &       \\
    & &               &       &       &       &       &       \\
    & &               &       &       &       &       &	\\
    \\
\end{tabularx}}
\caption{User-Movie interaction data a) Popularity-Movie histogram b)Movie genres/clusters c) User-Movie rating data} \label{fig:example}
\end{figure*}
\end{center}
\fi



%We propose a novel approach that allows us to  promote long-tail items in a targeted manner, thereby improving the novelty of top-$\size$ sets, the overall item-space coverage across recommendations, while maintaining reasonable levels of accuracy.

%Next, we integrate these learned preferences  in a generic  top-$\size$ recommendation framework to provide customized balance between accuracy and coverage.

%sequentially make recommendations, while adjusting its parameters with regard to the set of top-$\size$ recommendations made so far. However, since  sequential parameter updates  cause  scalability issues, we propose a sampling based algorithm. This variant of our framework, called {\it Ordered Sampling-based Locally Greedy (OSLG)\/},  allows us to  correct for the popularity bias in recommendations with regard to individual user long-tail preferences. 

%ICDE submission
%Our framework differs with  prior work in the following aspects:  unlike~\cite{adomavicius2011maximizing,adomavicius2012improving,zhang2013personalize,ho2014likes},  the long-tail preference personalization in our framework is learned rather than optimized using cross-validation or parameter tuning. In other words, our personalization method is independent of the underlying base  recommendation models.  Moreover, our framework is  generic. This enables us to  plug-in several base recommenders, and evaluate their  effectiveness without requiring  extensive tuning for the accuracy and coverage trade-off. 


%\vspace{-2.8pt}
\begin{itemize}

\item  We examine various measures for estimating user long-tail novelty preference in Section~\ref{sec:lt-pref} and formulate an optimization problem  to directly learn users' preferences for long-tail  items from interaction data in Section~\ref{sec:learning-lt-pref}. %In addition, we introduce several heuristics for measuring the user preference for less common items from historical rating data.% 

\item  We integrate the user preference estimates into GANC %, a generic re-ranking framework that provides customized balance between accuracy, novelty, and coverage 
(Section~\ref{sec:RiskbasedReranking}), and  introduce {\it Ordered Sampling-based Locally Greedy (OSLG)\/}, a scalable algorithm that relies  on user long-tail preferences to correct the popularity bias (Section~\ref{sec:optimizationAlgorithm}).
%We introduce OSLG, a scalable algorithm that relies  on user long-tail preferences to  maximize item space coverage \textcolor{red}{while maintaining acceptable levels of accuracy} (Section~\ref{sec:optimizationAlgorithm}).

\item   We conduct an extensive empirical study and evaluate performance from  accuracy, novelty, and coverage perspectives (Section~\ref{sec:Experiments}).  We use five  datasets with varying density and difficulty levels. %:  Netflix, MovieTweetings, and MovieLens (100K, 1M, 10M). 
  In contrast to most related work,  our evaluation considers realistic settings that include a large number of infrequent  items and users. %This enables us to study the impact of  data density on the performance trade-offs of several  state of the art top-$\size$ recommendation algorithms. %   %,  and use the all-items ranking protocol~\cite{steck2013evaluation,vargas2014improving}, where performance is measured using all items with train data. to evaluate the performance of several  state of the art top-$\size$ recommendation algorithms 
 
\item Our empirical results confirm that the performance of re-ranking models is impacted by the underlying   base recommender and the dataset density. Our generic approach enables us to easily incorporate a suitable base recommender to devise an effective solution for both dense and sparse settings. In dense settings, we use the same base recommender as existing re-ranking approaches, and we outperform them in accuracy and coverage metrics. For sparse settings, we plug-in a more suitable base recommender, and devise an effective solution that is competitive with existing top-$\size$ recommendation methods in accuracy and novelty. 

%Directly estimating the long-tail novelty preferences allows us to customize re-ranking per user, and  devise a generic framework.   
 
\end{itemize}

Section~\ref{sec:related-work} describes related work. Section~\ref{sec:conclusion} concludes.

\section{Motivation and Background} \label{Motivation}
\subsection{Motivation}
Different applications have different resource constraint with respect to CPU, memory, and bandwidth usage. Having a single resource manager for all existing resources and users in the system result in inefficiencies since it is not scalable and the operating system may not have enough information about application's needs. For example, traditional LRU-based cache strategy uses cache utilization as a metric to give larger cache size to the applications which have higher utilization and lower cache size to the applications with lower cache utilization. However more cache utilization does not always result in better performance. Streaming applications for example have very high cache utilization, but very small cache reuse. In fact, the streaming applications only need a small cache space to buffer the streaming data. With rapid improvements in semiconductor technology, more and more cores are being embedded into a single core and managing large scale application using a single resource manager becomes more challenging. \\
%\indent Even if the applications are forced to announce their resource demand, it is possible that they lie about their resource vector or run some useless instructions to pretend to utilize the allocated resources given to them.
\indent In addition, defining a single fairness parameter for multiple applications is non-trivial since applications have different bottlenecks and may get different performance benefits from each resources during each phases of its execution time. Defining a single reasonable parameter for fairness is somewhat problematic. For instance, simple assignment algorithms which try to equally distribute the resources between all applications ignores the fact that different applications have different resource constraints. As a consequence, this makes the centralized resource management systems very inefficient in terms of fairness as well as performance needs of applications. We need a decentralized framework, where all applications' performance benefit could be translated into a unique notion of fairness and performance objective (known as utility function in economics) and the algorithm tries to allocate resources based on this translated notion of fairness. This translation has been well defined in economics and marketing, where the diversity of customer needs, makes more economically efficient market \cite{zhou2014sharing}.\\
\indent Economists have shown that in an economically efficient market, having diverse resource constraints and letting the customers compete for the resources can make a Nash equilibrium where both the applications and the resource managers can be enriched. \\
\indent Furthermore, applications' demand changes over time. Most resource allocation schemes pre-allocate the resources without considering the dynamism in applications' need and number of users sharing the same resource over time. Therefore, applications' performance can degrade drastically over time. Figure~\ref{fig:Phases} shows phase transitions for instruction per cycle (IPC) of mcf application from \textit{spec 2006} over 50 billion instructions. \\ 
%%%%%%%%%%%%%%%%%%%%%%%%%%%%%%%%%%%%%%%%%%%%%%%%%%%%%%%%%%%%%%%%%%%%%%%%%%%
\begin{figure}[!tb]
\centering
%\includegraphics[height=3in, width=1.5in]{NodeArchs2.pdf}
\includegraphics[height=1.5in, width=3.3in]{Images/Phases_May.pdf} %Phases.pdf
%\epsfig{file=Dataset.eps, height=2.5in, width=3in}
\caption{\label{fig:Phases}Phase transition in mcf with different L2 cache sizes.}
\end{figure}
%%%%%%%%%%%%%%%%%%%%%%%%%%%%%%%%%%%%%%%%%%%%%%%%%%%%%%%%%%%%%%%%%%%%%%%%%%%
\indent We try to find a game-theoretic distributed resource management approach where the shared hardware resources are exposed to the applications and we will show that running a repeated auction game between different applications which are assumed to be rational, the output of the game would converge to a balanced Nash equilibrium allocation. In addition, we will compare the convergence time of the proposed algorithm in terms of dynamism in the system. We will evaluate our model with two case studies: 1- Private and Shared last level cache problem, where the applications have to decide if they would benefit from a larger cache space which can potentially get more congested or a smaller cache space which is potentially less congested. Based on the number of other applications in the system the application can change its strategy over the time. 2- Heterogeneous processors (\textit{Intel Xeon} and \textit{Xeon Phi}) problem, where we perform experiments to show how congestion affects the performance of different applications running on an \textit{Intel Xeon} or \textit{Xeon Phi} co-processors. Based on the congestion in the system the application can offload the most time consuming part of its code on \textit{Xeon Phi} co-processors or not.   
%%%%%%%%%%%%%%%%%%%%%%%%%%%%%%%%%%%%%%%%%%%%%%%%%%%%%%%%%%%%%%%%%%%%%%%%%%%
\subsection{Background}
%Congestion games have been studied in network routing protocols where the delay of each player choosing a path in the network depends on the number of players choosing the same route in the system. 
%Every congestion game is a potential game since there exists a potential function associated with it. In addition, every congestion game has a pure-strategy Nash equilibrium. A key assumption in congestion games is that all users have the same impact on the congestion. However, this assumption is not always true. In case of computer architecture resources, applications effect each other differently and dividing the pay-off function by the number of users running on the shared resource does not give us the correct utility. 
%%%%%%%%%%%%%%%%%%%%%%%%%%%%%%%%%%%%%%%%%%%%%%%%%%%%%%%%%%%%%%%%%%%%%%%%%%
Game theory has been used extensively in economics, political and social decision making situations \cite{tootaghaj2011game, tootaghaj2011risk, kotobi2017spectrum, kotobi2015introduction, kesidis2013distributed, kurve2013agent, wang2017using, wang2015recouping}. A game is a situation, where the the output of each player not only depends on her own action in the game, but also on the action of other players \cite{osborne1994course}. Auction games are a class of games which has been used to formulate real world problems of assigning different resources between $n$ users. Auction game framework can model resource competition, where the payoff (cost) of each application in the system is a function of the contention level (number of applications) in the game.\\
\indent Inspired by market-based interactions in real life games, there exists a repeated interaction between competitors in a resource sharing game. Assuming large number of applications, we show that the system gets to a Nash equilibrium where all applications are happy with their resource assignment and don't want to change their state. Furthermore, we show that the auction model is strategy-proof, such that no application can get more utilization by bidding more or less than the true value of the resource. In this paper we propose a distributed market based approach to enforce cost on each resource in the system and remove the complexity of resource assignment from the central decision maker.\\ 
\indent The traditional resource assignment is performed by the operating system or a central hardware to assign fair amount of resources to different applications. However, fair scheduling is not always optimal and solving the optimization problem of assigning $m$ resources between $n$ users in the system is an integer programming which is an NP-hard problem and finding the best assignment problem becomes computationally infeasible. Prior works focus on designing a fair scheduling function that maximizes all application's benefit \cite{zahedi2014ref, llull2017cooper, ghodsi2011dominant, zahedi2015sharing, fan2016computational}, while applications might have completely different demands and it is not possible to use the same fairness function for all. By shifting decision making to the individual applications, the system becomes scalable and the burden of establishing fairness is removed from the centralized decision maker, since individual applications have to compete for the resources they need. Applications start with the profiling utility functions for each resource and bid for the most profitable resource. During the course of execution time they can update their belief based on the observed performance metrics at each round of the auction. The idea behind updating the utility functions is that the history at each round of decision point shows the state of the game. This state indicates the contention on the current acquired resource. The pay-off function in each round depends on the state of the system and on the action of other applications in the system. 
%%%%%%%%%%%%%%%%%%%%%%%%%%%%%%%%%%%%%%%%%%%%%%%%%%%%%%%%%%%%%%%%%%%%%%%%%%%
\subsubsection{Sequential Auction}
Auction-based algorithms are used for maximum weighted perfect matching in a bipartite graph $G=(U,V, E)$ \cite{bertsekas1998network, kyle1985continuous, vasconcelos2009bipartite}. A vertex  $U_i \in U$ is the application in the auction and a vertex $V_j \in V$ is interpreted as a resource. The weight of each edge from $U_i$ to $V_j$ shows the utility of getting that particular resource by $U_i$. The prices are initially set to zero and will be updated during each iteration of the auction. In sequential auctions, each resource is taken out by the the auctioneer and is sequentially auctioned to the applications, until all the resources are sold out.
\subsubsection{Parallel Auction}
In a parallel auction, the applications submit their bids for the first most profitable item. The value of the bid at each iteration is computed based on the difference of the highest profitable object and the second highest profitable object. The auctioneer would assign the resources based on the current bids. At each iteration, the valuation of each resource is updated based on the observed information during run-time which shows the contention on that particular resource.
%%%%%%%%%%%%%%%%%%%%%%%%%%%%%%%%%%%%%%%%%%%%%%%%%%%%%%%%%%%%%%%%%%%%%%%%%%%%
%%%%%%%%%%%%%%%%%%%%%%%%%%%%%%%%%%%%%%%%%%%%%%%%%%%%%%%%%%%%%%%%%%%%%%%%%%%%
\section{CAGE: A Market-based Contention-aware Game-theoretic resource assignment}\label{Problem_definition}
\subsection{Model Description}
Consider $n$ applications and $i$ instances of $m$ different resources. Applications arrive in the system one at a time. The applications have to choose among $m$ resources. There exists a bipartite graph between the matching of the applications and the resources.\\
\indent In general, there can be more than one application to get a shared resources. However, each application can not get more than one of the available heterogeneous resources. For example, if we have two cache space of 128kB (one way) and 256kB (two ways), the application can either get the 128kB of cache space or 256kB and can't get both of them at the same time. Furthermore, each resource $m_i$ has a cost $C_i$ which is defined by the applications' bid in the auction. \\
\indent Figure~\ref{fig:auction} shows auction-based framework to support \textit{CAGE} between $N$ applications that execute together competing for $M$ different resources. Each application has a utility table that shows how much performance it gets from each $M$ resources at each time slot. Based on the utility tables, applications submit bids for the most profitable resource. Based on the submitted bids, the auctioneer decides about the resource assignment for each resource, and updates the prices. Next, the applications who did not get any assignment compete for the next most profitable resource based on the updated prices repeatedly until all applications are assigned.  Figure~\ref{fig:auction} shows an example of a resource assignment and the corresponding bipartite graph.
%Table~\ref{table:notation} shows the notation used in our formulation.
%%%%%%%%%%%%%%%%%%%%%%%%%%%%%%%%%%%%%%%%%%%%%%%%%%%%%%%%%%%%%%%%%%%%%%%%%%%%%%%
%%%%%%%%%%%%%%%%%%%%%%%%%%%%%%%%%%%%%%%%%%%%%%%%%%%%%%%%%%%%%%%%%%%%%%%%%% 
\begin{table}[!tb] 
\centering
\caption{Notation used in our formulations.}\label{Table:notation}
\begin{tabular}{|p{0.7in}||p{2.3in}|} 
\hline $N$ & Number of applications \\
\hline $K$ & Number of cache levels \\
\hline $T$ & Time intervals where the bidding is hold \\
\hline $m$ & Number of applications which can get a resource \\ 
\hline $p$ & Number of phases for each application during its course of execution time \\ 
\hline $n$ & Number of applications competing for a specified resource \\
\hline $M$ & Number of resources \\
\hline $P_i$ & Number of phases for application $i$ \\
\hline $\delta$ & dynamic factor that shows how much we can rely on the past iterations. \\
\hline $U$ & The applications which shows the left set of nodes in the bipartite graph. \\
\hline $V$ & The resources which shows the right set of nodes in the bipartite graph. \\
\hline $E$ & The edges in the bipartite graph. \\
\hline $G=(U,V,E)$ & A bipartite graph showing the resource allocation between the applications and the set of resources. \\
\hline $b_{i,k}$ & User i's bid for k th resource \\
\hline $B_i$ & The total budget (sum of bids) a user have \\
\hline $C_k$ & The total capacity of each resource \\
\hline $p_{j}$ & The price of resource $j \in V$ in the auction. \\
\hline $Bottleneck_{1,i}$ & The first bottleneck resource for application $i$ \\
\hline $Bottleneck_{2,i}$ & The second bottleneck resource for application $i$ \\
\hline $v_{i,m}(T)$ & The valuation function of application $i$ for resource $m$ at time $T$ \\
\hline
\end{tabular}
\end{table}
%%%%%%%%%%%%%%%%%%%%%%%%%%%%%%%%%%%%%%%%%%%%%%%%%%%%%%%%%%%%%%%%%%%%%%%%%% 
%%%%%%%%%%%%%%%%%%%%%%%%%%%%%%%%%%%%%%%%%%%%%%%%%%%%%%%%%%%%%%%%%%%%%%%%%%%%%%%
\begin{figure*}[!htb]
\centering
%\includegraphics[height=3in, width=1.5in]{NodeArchs2.pdf}
\includegraphics[height=3.2in, width=6.5in]{Images/Auction_v2.pdf} %[height=4in, width=8in]
%\epsfig{file=Dataset.eps, height=2.5in, width=3in}
\caption{\label{fig:auction} Framework for auction-based resource assignment (CAGE).}
\end{figure*}
%%%%%%%%%%%%%%%%%%%%%%%%%%%%%%%%%%%%%%%%%%%%%%%%%%%%%%%%%%%%%%%%%%%%%%%%%%%
\begin{comment}
\begin{figure}[!htb]
\centering
%\includegraphics[height=3in, width=1.5in]{NodeArchs2.pdf}
\includegraphics[height=2.2in, width=1.3in]{Images/bipartite.pdf}
%\epsfig{file=Dataset.eps, height=2.5in, width=3in}
\caption{\label{fig:bipartite} Cache allocation as a bipartite graph.}
\end{figure}
\end{comment}
%%%%%%%%%%%%%%%%%%%%%%%%%%%%%%%%%%%%%%%%%%%%%%%%%%%%%%%%%%%%%%%%%%%%%%%%%%%
\subsection{Problem Defenition} 
\indent We formulate our problem as an auction based mechanism to enforce cost/value updates for each resource as follows: \\
%The cost of each player to get a resource is the cost of the assigned resource divided by the number of players who share. 
\begin{itemize}
  \item \textbf{Valuation $\mathbf{v_{i,m}}$} : Any application has a valuation function which shows how much he benefits from $i th$ resource. The valuation function at time $t=0$ for cache contention case study is derived from the IPC (instruction per cycle) curves which is found using profiling, and for processor and co-processor contention case study is derived from the profiling solo performance metric of the application. However, in general, each application can choose its own utility function.  
  %%%%%%%%%%%%%%%%%%%%%%%%%%%%%%%%%%%%%%%%%%%%%%%%%%%%%% 
    \item \textbf{Observed information}: The observed information at each time step is the performance value of the selected action in the game. Therefore, the applications repeatedly update the history of their valuation function over time.  
  %%%%%%%%%%%%%%%%%%%%%%%%%%%%%%%%%%%%%%%%%%%%%%%%%%%%%%   
    \item \textbf{Belief updating}: At each iteration step of the auction, the applications update their valuation of each resource based on the observed performance on each resource. The update at time $T$ is derived using the following formula:
%\begin{small}
\begin{equation}\label{eq:belief}
v_{i,m}(T)=\frac{\sum\limits_{t=0}^T {\delta}^{T-t}  v_{i,m}(t)}{\sum\limits_{t=0}^T {\delta}^{T-t}} 
\end{equation}  
%\end{small}  
%%%%%%%%%%%%%%%%%%%%%%%%%%%%%%%%%%%%%%%%%%%%%%%%%%%%%%%%%%%%%%%%%%%%%%%%%
Where $v_{i,m}(t)$ shows the observed valuation of resource $m$ at time step $t$ by user $i$ in the system; $\delta$ shows the discount factor between 0 and 1 which shows how much a user relies on its past observations in the system. The discount factor is chosen to show the dynamics in the system. If the observed information in the system changes fast, the discount factor is nearly zero which means that we can't rely on the past observations very much. However if the system is more stable and the observed information does not change fast, the discount factor is chosen to be near 1. We choose the discount factor as the absolute value of the correlation coefficient of the observed values of the valuations at each iteration step which is calculated as follows:
%\begin{small}
\begin{equation}
\delta =  \frac{E(v_{i,m})^2}{{\sigma_{v_{i,m}}}^2}
\end{equation}  
%\end{small}
%%%%%%%%%%%%%%%%%%%%%%%%%%%%%%%%%%%%%%%%%%%%%%%%%%%%%%%%%%%%%%%%%%%%%%%%
%%%%%%%%%%%%%%%%%%%%%%%%%%%%%%%%%%%%%%%%%%%%%%%%%%%%%%%%%%%%%%%%%%%%%%%%
  \item \textbf{Action}: At each time step the applications decides which resource to bid and how much to bid for each resource. 
\end{itemize} 
%%%%%%%%%%%%%%%%%%%%%%%%%%%%%%%%%%%%%%%%%%%%%%%%%%%%%%%%%%%%%%%%%%%%%%%%%% 
\indent Table~\ref{Table:notation} shows important notation used throughout the paper. In the following sections, we describe our distributed optimization scheme to solve the problem. 
%%%%%%%%%%%%%%%%%%%%%%%%%%%%%%%%%%%%%%%%%%%%%%%%%%%%%%%%%%%%%%%%%%%%%%%
%\begin{equation}
%min \sum\limits_{i=1}^n v_i C_k \frac{b_{i,k}}{\theta_k}, \\
%s.t. \sum\limits_{i=1}^n b_{i,k} \leq E_i
%\end{equation}
%%%%%%%%%%%%%%%%%%%%%%%%%%%%%%%%%%%%%%%%%%%%%%%%%%%%%%%%%%%%%%%%%%%%%%%
\subsection{Distributed Optimization Scheme}
The goal is to design a repeated auction mechanism which is run by the operating system to guide the applications to choose their best resource allocation strategy. The applications' goal is to maximize their own performance and the operating system wants to maximize the total utility it gains from the applications. Then, each application can use its own utility function and evaluates the resources based on how much it likes that particular resource. \\
\indent \textbf{Applications' approach}: The application $i$ want to maximize the total utility with respect to a limited budget for all phase $p$ of its execution time. \\
%%%%%%%%%%%%%%%%%%%%%%%%%%%%%%%%%%%%%%%%%%%%%%%%%%%%%%%%%%%%%%%%%%%%%%%%%
%maximize \;\;\;\; \sum\limits_{i=1}^n v_i C_k \frac{b_{i,k}}{\theta_k},\\
%\begin{small}
\begin{align}
%\begin{IEEEeqnarray}{rCl}
\forall i \in U \; \; \; \; \; maximize \; \; \; \; \sum\limits_{p=1}^{P_i} \sum\limits_{m=1}^M  v_{i,m,p}-b_{i,m,p} , \nonumber \\
 % \IEEEyessubnumber\\
subject \; to \;\;\;\; \sum\limits_{p=1}^{P_i} \sum\limits_{m=1}^M b_{i,m,p} \leq B_i .
%\IEEEyessubnumber
%\end{IEEEeqnarray}
\end{align}
%\end{small}
%%%%%%%%%%%%%%%%%%%%%%%%%%%%%%%%%%%%%%%%%%%%%%%%%%%%%%%%%%%%%%%%%%%%%%%%%%
\indent \textbf{OS's approach}: The operating system wants to maximize the social welfare function which is translated into submitted bids from the applications in a limited resource constraints.\\
%%%%%%%%%%%%%%%%%%%%%%%%%%%%%%%%%%%%%%%%%%%%%%%%%%%%%%%%%%%%%%%%%%%%%%%%%%
%\begin{small}
\begin{align}
%\begin{IEEEeqnarray}{rCl}
maximize \; \; \; \sum\limits_{i=1}^N \sum\limits_{p=1}^{P_i} \sum\limits_{m=1}^M b_{i,m,p} A_{i,m,p} , \nonumber \\ 
%\IEEEyessubnumber\\
subject \; to \;\;\;\; \sum\limits_{i=1}^N \sum\limits_{m=1}^M A_{i,m,p} \leq A_{max}, \; \; \; \; \forall p \in P , \nonumber \\
%\IEEEyessubnumber\\
A_{i,m,p} \in \{0,1\} , \; \; \; \; \forall i \in U, \; \;  \forall m \in V, \; \;  \forall p \in P .
%\IEEEyessubnumber
%\end{IEEEeqnarray}
\end{align}
%\end{small}
%%%%%%%%%%%%%%%%%%%%%%%%%%%%%%%%%%%%%%%%%%%%%%%%%%%%%%%%%%%%%%%%%%%%%%%%%%%
\indent \textbf{Illustrative example}: As an illustrative example, suppose we have two different resources, a large cache of 1MB which can be shared between applications, and two private caches of 512KB which are not shared. There are two applications competing for the cache space. One of the applications wants to minimize its request latency and the other one wants to maximize number of instructions executed per cycle. Suppose that both applications have two phases $(0,T)$ and $(T,2T)$.  Suppose if the first application gets the larger cache space its request latency reduces by 20 percent in first phase and by 40 percent in the second phase. The second application's \textit{IPC} increases by 35 percent in the first phase and by 25 percent in the second phase if it gets the larges cache space. Also, assume they both have 60 tokens (bids) to submit. The first application invests 20 token (bids) for the first phase and 40 tokens for the next phase. He should redistribute the tokens for the next phase if he did not get the resource he wants in the first phase. The second application invests 35 tokens in the first phase and 25 tokens in the next phase. The auctioneer (OS) at each phase decides to allocate which resource to which applications. Since, the social welfare would be maximized if the auctioneer allocates both applications with the larger cache space, they would both get the larger resource. Then the first application notices that its utility function does not improve as he predicts and adjusts the utility table and can either change its allocation or stay on current allocation. 
%If both applications bid 10\$ for the private cache and 15\$ for the shared cache, the operating system would allocate both the shared cache space and get 15\$ from each to maximize its revenue.  
\subsection{Analysis}
The distributed optimization problem seems complex. However, in reality the problem can be splitted into simpler subproblems since each application knows its bottleneck resource and would first bid for the first bottleneck resource to maximize its utility.\\
\indent We suppose all applications in the system are risk-neutral which means they have a linear valuation of utility function. Each risk neutral agent wants to maximize its expected revenue. Risk attitude behaviors are defined in \cite{ferber1999multi} where the agents can broadly be divided into risk averse, risk seeking and risk neutral. Risk averse agents prefer determinitic values rather than risky value profits and risk seeking applications have a superlinear utility function and prefer risky utilities than sure utilities. Next, we derive the Bayes Nash equilibrium strategy profile for all agents in the system assuming risk neutrality.  \\
%%%%%%%%%%%%%%%%%%%%%%%%%%%%%%%%%%%%%%%%%%%%%%%%%%%%%%%%%%%%%%%%%%
\newtheorem{defi}{Definition}
\begin{defi}
A strategy profile $a$ is a pure Nash equilibrium if for every application $i$ and every strategy $a_i' \neq a_i \in A$ we have $u_i(a_i, a_{-i}) \geq u_i(a_i', a_{-i})$
\end{defi}
%%%%%%%%%%%%%%%%%%%%%%%%%%%%%%%%%%%%%%%%%%%%%%%%%%%%%%%%%%%%%%%%%%
\newtheorem{theorem}{Theorem}
\begin{theorem}\label{thm:neat}
%\emph{(Theorem)}
\label{Auction}
Suppose $n$ risk-neutral applications whose valuations are derived uniformly and independently from the interval $[0,1]$ compete for one resource which can be assigned to $m$ application who have the highest bid in the auction. We will show that Bayes Nash equilibrium bidding strategy for each application in the system is to bid $\frac{n-m}{n-m+1}v_i$ whre $v_i$ is the profit of application $i$ for getting the specified resource.  
\end{theorem}
%%%%%%%%%%%%%%%%%%%%%%%%%%%%%%%%%%%%%%%%%%%%%%%%%%%%%%%%%%%%%%%
%DIMAN COMMENTED PROOF FOR APPENDIX
\begin{comment}

\begin{proof}
Suppose all other applications' bidding strategy is to choose $\frac{n-m}{n-m+1}v_i$. Since the bidding values were derived uniformly in $[0, 1]$ all bids have the same probability. Therefore, if we consider the first application's expected utility to find its best response, we have:

\begin{equation}
%\begin{IEEEeqnarray}{rCl}
E[u_1] = \int_0^1  .... \int_0^1 \! (v_1 -b_1) \, \mathrm{d}u_2 \mathrm{d}u_3 ... \mathrm{d}u_{n-m} .  
%\end{IEEEeqnarray}
\end{equation}

The following integral breaks into two part where the first application wins the auction or not. 


%\begin{IEEEeqnarray}{rCl}
%E[u_1] = \int_0^{b_1}  .... \int_0^{b_1} \! (v_1 -b_1) \, \mathrm{d}u_2 \mathrm{d}u_3 ... %\mathrm{d}u_{n-m}  \\
%+ \int_{b_1}^1  .... \int_{b_1}^1 \! (v_1 -b_1) \, \mathrm{d}u_2 \mathrm{d}u_3 ... \mathrm{d}%u_{n-m}\nonumber 
%\end{IEEEeqnarray}

\begin{equation}
%\begin{IEEEeqnarray}{rCl}
E[u_1] = \int_0^{\frac{n-m+1}{n-m}b_1}  .... \int_0^{\frac{n-m+1}{n-m}b_1} \! (v_1 -b_1) \, \mathrm{d}u_2 ... \mathrm{d}u_{n-m}  \\
+ \int_{\frac{n-m+1}{n-m}v_1}^1  .... \int_{\frac{n-m+1}{n-m}v_1}^1 \! (v_1 -b_1) \, \mathrm{d}u_2 \mathrm{d}u_3 ... \mathrm{d}u_{n-m}\nonumber 
%\end{IEEEeqnarray}
\end{equation}

The second part of the integrals is the term where the first application doesn't win the auction. Therfore, the expected payoff of application 1 is equal with:

%\begin{IEEEeqnarray}{rCl}
%E[u_1] = \int_0^{b_1}  .... \int_0^{b_1} \! (v_1 -b_1) \, \mathrm{d}u_2 \mathrm{d}u_3 ... %\mathrm{d}u_{n-m}  \\
%= {({b_1}) }^{n-m} (v_1 -b_1). \nonumber 
%\end{IEEEeqnarray}

\begin{equation}
%\begin{IEEEeqnarray}{rCl}
E[u_1] =\int_0^{\frac{n-m+1}{n-m}b_1}  .... \int_0^{\frac{n-m+1}{n-m}b_1} \! (v_1 -b_1)  \mathrm{d}u_2 ... \mathrm{d}u_{n-m}  \\
= {(\frac{n-m+1}{n-m} b_1) }^{n-m} (v_1 -b_1). \nonumber 
%\end{IEEEeqnarray}
\end{equation}


Diffrentiating with respect to $b_1$ the optimal bid for application one is derived as follows:

%\begin{equation}
%\frac{\partial}{\partial b_1} ( {({b_1}) }^{n-m} (v_1 -b_1))=0.
%\end{equation}


\begin{equation}
\frac{\partial}{\partial b_1} ( {(\frac{n-m+1}{n-m} b_1) }^{n-m} (v_1 -b_1))=0.
\end{equation}

Which gives us the optimal bid for each application:
\begin{equation}
\Rightarrow b_1= \frac{n-m}{n-m+1}v_1
\end{equation}
\end{proof}
\end{comment}
%DIMAN COMMENTED PROOF FOR APPENDIX
%%%%%%%%%%%%%%%%%%%%%%%%%%%%%%%%%%%%%%%%%%%%%%%%%%%%%%%%%%%%%%%%%%%%%%%%%%%%
\begin{algorithm}[!tb]
\DontPrintSemicolon % Some LaTeX compilers require you to use \dontprintsemicolon    instead
\KwIn{A bipartite Graph (U, V, E).}
\KwOut{The allocation of resources to applications.}
At t=0 the valuation of each application for each resource is derived using profiling while running alone. 

For each application $U_i \in U$, the first bottleneck resource is
\[ Bottleneck_{1,i} = V_{i,m}=  arg \; \max_{m \in V} (v_{i,m}-p_{m})  \] 
Next, find the second bottleneck resource for each applications $U_i \in U$ in the system:
\[ Bottleneck_{2,i} = V_{i,k}=  arg \; \max_{k \in V, k \neq m} (v_{i,k}-p_{k})  \] 

Each application submits the bid for its first bottleneck resource using the following formula:
\[ b_{i,m} = V_m - V_k + p_{j} + \epsilon \]
Each resource $V_j \in V$, which can be shared between $m$ applications, is assigned to the $m$ highest bidding applications $Winner_j={i_1, i_2, ..., i_m}$ and the price for that resource is updated as follows:
\[ p_{j} =  arg \; \max_{i_1, i_2, ..., i_m \in U} \sum\limits_{k=1}^m (b_{i_k,j})  \]

The $minBid$ for each resource is updated as the minimum bid of $m$ applications who acquired the resource. That is
\[ minBid=  arg \; \min_{i \in Winner_j}  (b_{i,j}) \] 
 
\caption{CAGE: Parallel Auction for heterogeneous resource assignment.}
\label{algo:b}
\vspace{0\baselineskip}
\end{algorithm}
%%%%%%%%%%%%%%%%%%%%%%%%%%%%%%%%%%%%%%%%%%%%%%%%%%%%%%%%%%%%%%%%%%%%%%%%%%%%%
\indent Theorem~\ref{thm:neat}, states that whenever there is a single resource that users compete to get it with different valuation functions, the Nash equilibrium strategy profile for risk-neutral users is to bid $\frac{n-m}{n-m+1}v_i$. This term tends to the true value of the object when n is a large number. \\
\indent In case of more than one resource competition we derive Algorithm~\ref{algo:b} and will prove that it is Nash equilibrium in the game. The algorithm is inspired by work of Bertsekas \cite{bertsekas1998network} that uses an auction for network flow problems. In the first step, all valuations are set to the solo-run of application's performance. Next, each application submits a bid for its first bottleneck resource. The bid should be larger than the price of the object which is intitialized to zero in the begining of the program. The applications only have incentive to bid a value no more than the difference of the first bottleneck and second bottleneck resource. Otherwise, it would submit a smaller bid to the second bottleneck and get the same revenue as paying more for the first bottleneck resource. In order to break the equal valuation function between two different applications, we use $\epsilon$ scaling such that at each iteration of the auction the prices should increase by a small number. 

%In addition, suppose we have 5 different memory bandwidth exposed to the applications Each application gets different performance benefit from different cache sizes and different memory bandwidth which is denoted in table  **. The applications need to submit their bids based on their performance benefits. 

%\begin{equation}
%\begin{split}
%\int_0^\frac{nb_1}{n-m}  .... \int_0^\frac{nb_1}{n-m} \! (\frac{v_1}{m} -b_1) \, \mathrm{d}u_2 %\mathrm{d}u_3 ... \mathrm{d}u_{n-m}= \\
%= {(\frac{nb_1}{n-m}) }^{n-m} (\frac{v_1}{m} -b_1). 
%\end{split}
%\end{equation}
%%%%%%%%%%%%%%%%%%%%%%%%%%%%%%%%%%%%%%%%%%%%%%%%%%%%%%%%%%%%%%%%%%%%%%%%%%%%
%\[ \frac{\partial}{\partial b_1} ({(\frac{nb_1}{n-m}) }^{n-m} (\frac{v_1}{m} -b_1))

%\newtheorem{defi}{Definition}
%\begin{defi}
%Let's assume each user $Ui$ in the system is defined as one vertice of a graph $G$ in the system and let each edge in the graph show which subset of users can impact each others' performance and the associated weight of each edge show the cost function of how two users affect each others' performance in the system. Each edge has a weight function denoted by ${P1(n), P2(n), ... Pe(n)}$, where $e$ is the number of edges in Graph $G$. Let $A=A_1 \times A_2 \times ... \times A_n $ be the set of actions that each user can play. 
%\end{defi}
\input{Case_Studies}
\section{Related Work} \label{Related_works}
With rapid improvement in computer technology, more and more cores are embedded in a single chip and applications competing for a shared resource is becoming common. On the one hand, managing scheduling of shared resources for large number of applications is challenging in a sense that the operating system doesn't know what is the performance metric for each application. But on the other hand, the operating system has a global view of the whole state of the system and can guide applications on choosing the shared resources.\\ 
\indent There has been several works, for managing the shared cache in multi-core systems. Qureshi et al. \cite{qureshi2006utility} showed that assigning more cache space to applications with more cache utility does not always lead to better performance since there exists applications with very low cache reuse which may have very high cache utilization. \\
\indent Several software and hardware approaches has been proposed to find the optimal partitioning of cache space for different applications \cite{zhuravlev2010addressing}. However, most of these approaches use brute force search of all possible combinations to find the best cache partitioning in run time or introduce a lot of overhead. There has been some approaches which use binary search to reduce searching all possible combinations \cite{kim2004fair, lin2008gaining, tam2009rapidmrc}. But none of these methods are scalable for the future many-core processor designs.\\
\indent There exists prior game-theoretic approaches designing a centralized scheduling framework that aims at a fair optimization of applications' utility \cite{zahedi2014ref, llull2017cooper, ghodsi2011dominant, zahedi2015sharing, fan2016computational}. Zahedi et al. in REF \cite{zahedi2014ref, zahedi2015sharing} use the Cobb-Douglas production function as a fair allocator for cache and memory bandwidth. They show that the Cobb-Douglas function provides game-theoretic properties such as sharing incentives, envy-freedom, and Pareto efficiency. But their approach is still centralized and spatially divides the shared resources to enforce a fair near-optimal policy sacrificing the performance. In their approach the centralized scheduler assumes all applications have the same priority for cache and memory bandwidth, while we do not have any assumption on this. Further, our auction-based resource allocation can be used for any number of resources and any priority for each application and the centralized scheduler does not need to have a global knowledge of these priorities.  \\
\indent Ghodi et al. in DRF \cite{ghodsi2011dominant} use another centralized fair policy to maximize the dominant resource utilization. But in practice it is not possible to clone any number of instances of each resources. %  the underlying scenario cloning the instances is very limited or superficial for practical purposes. 
Cooper \cite{llull2017cooper} enhances REF to capture colocated applications fairly, but it only addresses the special case of having two sets of applications with matched resources. Fan et al. \cite{fan2016computational} exploits computational sprinting architecture to improve task throughput assuming a class of applications where boosting their performance by increasing the power. \\
\indent While all prior works use a centralized scheduling that provides fairness and assume the same utility function for all, co-runners might have completely diverse needs and it is not efficient to use the same fairness/performance policy across them. 
In our auction-based resource scheduling provides scalability since individual applications compete for the shared resources based on their utility and the burden of decision making is removed from the central scheduler. We believe future CMPs should move toward a more decentralized approach which is more scalable and provides fair allocation of resources based on applications' needs. \\ 
%Providing the scalability of the system is getting better, if we make the individual applications our self administrators, and we can remove most of the performance-restricting policies such as fairness constraints from the centralized decision-maker. \\
\indent Auction theory which is a subfield of economics has recently been used as a tool to solve large scale resource assignement in cloud computing \cite{krishna2009auction, parsons2011auctions}. In an auction process, the buyers submit bids to get the commodities and sellers want to sell their commodities with the maximum price as possible. \\
\indent Our auction-based algorithm is inspired by work of Bertsekas \cite{bertsekas1998network} that uses an auction based mechanism for network flow problems. Our algorithm is an extension of local assignment problem proposed by Bertsekas et al that has been shown to converge to the global assignment within a linear approximation.
%\textcolor{red}{Not Complete yet}
\section{Conclusion}
\label{sec:conclusion}
This paper presents a generic top-$\size$ recommendation framework for  trading-off accuracy, novelty, and coverage. To achieve this, we profile the users according to their preference for long-tail novelty. We examine various measures, and formulate an optimization problem to learn these user preferences from interaction data.  We then integrate the user preference estimates in our generic framework, GANC.  Extensive experiments on several datasets confirm that there are trade-offs between accuracy, coverage, and novelty. Almost all re-ranking models increase coverage and novelty at the cost of accuracy. However, existing re-ranking models typically rely on rating prediction models, and are hence more effective in dense settings. Using a generic approach, we can easily incorporate a suitable base accuracy recommender to devise an effective solution for both sparse and dense settings.  %Our results  also indicate there is no single method that outperforms other methods in all metrics. However our techniques obtain a significant improvement in coverage, while  . 
Although we integrated the  long-tail novelty preference estimates into a re-ranking framework, their use-case is not limited to these frameworks. In  the future, we intend to explore the temporal and topical dynamics of long-tail novelty preference, particularly in settings where contextual information is  available.  
%We achieve these objectives without using any additional contextual information.


\iffalse
While we focused on promoting long-tail items across users, we did not consider diversity of individual top-$\size$ recommendations, a factor that has been shown to positively affect consumer satisfaction. This is one direction for future work. Moreover, the sequential setting  in our work, creates a dependency between different batches, where,  the items recommended to a batch of users, depends on those recommended to previous batches. This dependency is created through the parameter $\mathbf{f}$, that is updated every time a top-$\size$ set  is allocated to a batch of users. A future direction for our work is to estimate a distribution over $\mathbf{f}$ that allows us to independently solve the problem for each user, leading to improvements across all performance metrics, including recommendation time. 

We design algorithms that take advantage of the structure in the value functions to obtain both efficient and scalable solutions. 
We design an algorithm that takes advantage of the structure in the value functions to obtain both efficient and scalable solutions. 

\textcolor{red}{Our  sequential  algorithms can be applied for batch recommendation contexts,~e.g., personalized email marketing, where based on prior interaction data between users and items,  a new round of recommendations must be sent to all users in the system.  However, the independent coverage algorithms lift the sequential setting restrictions and allow it be applied for re-ranking the output of base recommender in any setting. }A future direction for our work is to incorporate explicit diversity metrics in the framework. 
\fi


%We have a presented a submodular maximization framework to systematically trade-off relevance and diversity in recommendations to individual users and coverage across the item-space. This ensures both consumer and producer satisfaction. We model users according to their risk and focusing degrees and promote long-tail items to the right group of consumers. Consequently, we obtain a significant improvement in coverage while maintaining reasonable levels of user satisfaction. Furthermore, our methods are able to achieve a more balanced distribution across the set of recommended items. In the future, we plan to investigate the effect of using alternative base recommender systems. 

%Future Work
%However most of these methods assume that the ratings are missing at random (MAR). Since our method of generating recommendations is based on the completed matrix, assuming MAR might introduce additional bias, we will use methods which assume that the ratings at missing not at random (MNAR),explored in~\cite{steck2010training, icml2014c2_hernandez-lobatob14}. 	 
%Long Tail %Recently, authors in~\cite{cremonesi2010performance} conducted extensive experiments to evaluate the performances of various matrix factorization-based algorithms and neighborhood models on the task of recommending long tail items. Their experimental results show that long tail recommendation leads to a decrease in accuracy for all algorithms. They also showed that for this task, SVD outperforms other state-of-the-art algorithms. 


%%%%%%% -- PAPER CONTENT ENDS -- %%%%%%%%


%%%%%%%%% -- BIB STYLE AND FILE -- %%%%%%%%
%\bibliographystyle{ieeetr}
%\bibliography{ref}
%%%%%%%%%%%%%%%%%%%%%%%%%%%%%%%%%%%%
\bibliographystyle{unsrt} %unsrt
\bibliography{bib/IEEEconf} 
\newpage
\section{Dataset Visualizations}
\label{sec:app_dataset_visuals}

%%%%%%
%%
%%
\subsection{Examples of each view class}
\newcommand{\BC}{0.33}
\setlength{\tabcolsep}{0.1cm}
\begin{figure}[!h]
\begin{tabular}{c c c c}
    PLAX  & PSAX & OTHER 
    \\
    \includegraphics[width=\BC\textwidth]{figures/small_appendix/Appendix_PLAX1.jpg}
    &
    \includegraphics[width=\BC\textwidth]{figures/small_appendix/Appendix_PSAX1.jpg}
    &
    \includegraphics[width=\BC\textwidth]{figures/small_appendix/Appendix_Other1.jpg}
    &
   
    \\
    
    \includegraphics[width=\BC\textwidth]{figures/small_appendix/Appendix_PLAX2.jpg}
    &
    \includegraphics[width=\BC\textwidth]{figures/small_appendix/Appendix_PSAX2.jpg}
    &
    \includegraphics[width=\BC\textwidth]{figures/small_appendix/Appendix_Other2.jpg}
    &
   
     \\
     
     \includegraphics[width=\BC\textwidth]{figures/small_appendix/Appendix_PLAX3.jpg}
    &
    \includegraphics[width=\BC\textwidth]{figures/small_appendix/Appendix_PSAX3.jpg}
    &
    \includegraphics[width=\BC\textwidth]{figures/small_appendix/Appendix_Other3.jpg}
    &
   
     \\
     
     \includegraphics[width=\BC\textwidth]{figures/small_appendix/Appendix_PLAX4.jpg}
    &
    \includegraphics[width=\BC\textwidth]{figures/small_appendix/Appendix_PSAX4.jpg}
    &
    \includegraphics[width=\BC\textwidth]{figures/small_appendix/Appendix_Other4.jpg}
    &
   
    \end{tabular}	
    \caption{Examples of images for each possible view label in our dataset. \emph{From left to right:} Four examples of peristernal long axis (PLAX) view, four examples of peristernal short axis (PSAX) view, and four examples of other kinds of view in our ``Other'' class. }
    \label{fig:VIEW_SAMPLES_APPENDIX}
\end{figure}

%%%%%%
%%
%%
\newpage
\subsection{Examples of each view for a Severe AS patient}
\newcommand{\BA}{0.33}
\setlength{\tabcolsep}{0.1cm}
\begin{figure}[!h]
\begin{tabular}{c c c c}
    PLAX  & PSAX & OTHER 
    \\
    \includegraphics[width=\BA\textwidth]{figures/small_appendix/SevereAS_11112007_PLAX1.jpg}
    &
    \includegraphics[width=\BA\textwidth]{figures/small_appendix/SevereAS_11112007_PSAX1.jpg}
    &
    \includegraphics[width=\BA\textwidth]{figures/small_appendix/SevereAS_11112007_Other1.jpg}
    &
    
    \\
    
    \includegraphics[width=\BA\textwidth]{figures/small_appendix/SevereAS_11112007_PLAX2.jpg}
    &
    \includegraphics[width=\BA\textwidth]{figures/small_appendix/SevereAS_11112007_PSAX2.jpg}
    &
    \includegraphics[width=\BA\textwidth]{figures/small_appendix/SevereAS_11112007_Other2.jpg}
    &
   
     \\
     
     \includegraphics[width=\BA\textwidth]{figures/small_appendix/SevereAS_11112007_PLAX3.jpg}
    &
    \includegraphics[width=\BA\textwidth]{figures/small_appendix/SevereAS_11112007_PSAX3.jpg}
    &
    \includegraphics[width=\BA\textwidth]{figures/small_appendix/SevereAS_11112007_Other3.jpg}
    &
  
    \end{tabular}	
    \caption{Examples of images from a patient with Severe AS in our dataset. \emph{From left to right:} Three examples of parasternal long axis (PLAX) view, three examples of parasternal short axis (PSAX) view, and three examples of other kinds of view in our ``Other'' class. }
    \label{fig:PatientSevereAS}
\end{figure}


%%%%%%
%%
%%
\newpage
\subsection{Examples of each view for a No AS patient}
\newcommand{\BB}{0.33}
\setlength{\tabcolsep}{0.1cm}
\begin{figure}[!h]
\begin{tabular}{c c c c}
    PLAX  & PSAX & OTHER 
    \\
    \includegraphics[width=\BB\textwidth]{figures/small_appendix/NoAS_1996889_PLAX1.jpg}
    &
    \includegraphics[width=\BB\textwidth]{figures/small_appendix/NoAS_1996889_PSAX1.jpg}
    &
    \includegraphics[width=\BB\textwidth]{figures/small_appendix/NoAS_1996889_Other1.jpg}
    &
    
    \\
    
    \includegraphics[width=\BB\textwidth]{figures/small_appendix/NoAS_1996889_PLAX2.jpg}
    &
    \includegraphics[width=\BB\textwidth]{figures/small_appendix/NoAS_1996889_PSAX2.jpg}
    &
    \includegraphics[width=\BB\textwidth]{figures/small_appendix/NoAS_1996889_Other2.jpg}
    &
   
     \\
     
     \includegraphics[width=\BB\textwidth]{figures/small_appendix/NoAS_1996889_PLAX3.jpg}
    &
    \includegraphics[width=\BB\textwidth]{figures/small_appendix/NoAS_1996889_PSAX3.jpg}
    &
    \includegraphics[width=\BB\textwidth]{figures/small_appendix/NoAS_1996889_Other3.jpg}
    &
  
    \end{tabular}	
    \caption{Examples of images from a patient with No AS in our dataset. \emph{From left to right:} Three examples of parasternal long axis (PLAX) view, three examples of parasternal short axis (PSAX) view, and three examples of other kinds of view in our ``Other'' class. }
    \label{fig:PatientNoAS}
\end{figure}



\newpage 
\section{Further Results}

\subsection{Assessment of ensembling}

Table~\ref{tab:best_single_checkpoint_VS_ensemble_FS_echo260} compares using a single checkpoint (one point estimate of neural network weight vector $\theta$) to using an ensemble of parameters aggregated from the last 25 checkpoints (one per epoch).

\begin{table}[!h]
    \centering
    \begin{tabular}{c|cccc|c}
    \textit{Diagnosis classification} & Split 1  & Split 2 & Split 3 & Split 4 & Average\\
    \hline
    Best single checkpoint  & 61.81 & 59.79 & 56.05 & 64.21 & 60.46\\
    Ensemble  & 62.95 & 61.03 & 56.58 & 63.84 & \textbf{61.13}
	\\ \hline
    \textit{View classification}  &   &  &  &  & 
    \\ \hline
    Best single checkpoint  & 93.03 & 93.24 & 92.39 & 93.79 & 93.11\\
    Ensemble  & 92.37 & 93.24 & 93.72 & 93.87 & \textbf{93.30}\\
    \end{tabular}
    \caption{Comparing best single checkpoint performance with ensemble performance on \textbf{Full-size \datasetName-156-52}}
    \label{tab:best_single_checkpoint_VS_ensemble_FS_echo260}
\end{table}


%%%%%%
%%
%%
\subsection{Patient-level diagnosis performance on bonus heldout set}

Table~\ref{tab:diagnosis classification patient unlabeled_heldout_174} examines the performance of the best labeled-set-only methods and MixMatch methods on the 174 patient studies that have diagnosis but no view labels.
 While the images used here were originally included in the unlabeled training set (which was used to train SSL methods like MixMatch), the diagnosis labels were not provided at all during training time. 
 We thus still believe this is an authentic test of generalization given the scarcity of labeled data available for our task.
 Of course, additional independent evaluation (especially from another institution) is needed.

\begin{table}[!h]
    \centering
    \begin{tabular}{l l l|rrrr|c}
    Pretrain & Method & Voting
    & Split 1  & Split 2 & Split 3 & Split 4 & average\\
    \hline
    & Basic WRN & Simple average & 76.73 & 75.25 & 76.87 & 81.88 & 77.68\\
    & Basic WRN & View-prioritized & 73.63 & 83.21 & 79.70 & 80.08 & 79.18\\
    %SSL & FS & MixMatch & Priority view + confidence & 94.58 & 84.17 & 77.50 & 92.5 & 87.19\\
    \hline
    & MixMatch & Simple average & 85.32 & 76.29 & 74.14 & 79.95 & 78.93\\
    view & MixMatch & Simple average & 83.36 & 77.96 & 75.61 & 81.37 & 79.58\\
    & MixMatch & View-prioritized & 83.27 & 83.76 & 82.34 & 82.83 & \textbf{83.05}\\
    view & MixMatch & View-prioritized & 82.53 & 86.15 & 79.62 & 83.27 & 82.89\\
    %view & MixMatch & LR with view-priority & 80.42 & 84.24 & 76.58 & 80.67 & 80.48\\
    %(MixMatch transfered) + MysteryMethod & NA & NA & NA\\ 
    \end{tabular}
    \caption{Patient-level AS Severity Diagnosis Classification on the \textbf{bonus heldout set} of 174 patients for whom we have diagnosis labels only (no view labels). We show balanced accuracy on models trained on each of the four folds on four \textbf{full-size \datasetName-156-52} dataset.
    }%endcaption
    \label{tab:diagnosis classification patient unlabeled_heldout_174}
\end{table}


%%%%%%
%%
%%
\subsection{Assessment of MixMatch hyperparameter sensitivity}

In Table~\ref{tab:MixMatch hyperparameters ablation study}, we consider four possible strategies for setting the hyperparameters of MixMatch, varying two  key settings for the weight on unlabeled loss $\lambda$. First, we vary whether the final value of $\lambda$ is set to its \emph{best} value among a grid of candidates (based on validation set performance), or \emph{fixed} to a constant.
Second, we vary whether $\lambda$ remains fixed over iterations throughout a training run, or is updated over iterations on a linear ramp schedule from 0 to its final target value. 

From this comparison, we see we consistent gains across splits (average gain across splits of over 1.6\% balanced accuracy) for using a delayed ramp up schedule with target value selected via grid search.

\begin{table}[!h]
    \centering
    \begin{tabular}{l l| rrrr | r}
    Final $\lambda$ value & $\lambda$ update schedule & Split 1  & Split 2 & Split 3 & Split 4 & Average\\
    \hline
    best on val & Delayed ramp-up  & 65.57 & 62.69 & 60.87 & 66.29 & 63.86\\
    best on val & Immediate ramp-up & 65.07 & 61.87 & 60.82 & 65.37 & 63.28\\
    best on val & Constant  & 65.03 & 61.52 & 58.87 & 65.22 & 62.66\\
    100 (fixed) & Constant & 63.94 & 61.79 & 58.87 & 64.35 & 62.24\\
    \end{tabular}
    \caption{Ablation study of different settings of the unlabeled loss weight $\lambda$ for MixMatch. AS severity diagnosis classification for individual images on the \textbf{full-size \datasetName-156-52} dataset. showing balanced accuracy averaged over the test sets from multiple folds (each fold’s test set contains all images from 52 patients). }%endcaption
    \label{tab:MixMatch hyperparameters ablation study}
\end{table}



%%%%%%
%%
%%
\subsection{Assessment of alternative view prioritization strategy using thresholding}


An anonymous reviewer suggested an alternative strategy for prioritizing images of relevant view.
The alternative strategy works as follows: for each image, we compute the predicted probability that the image is a ``relevant view'' (either PLAX and PSAX) by summing the probabilities of each view type.
However, instead of using this raw probability as a weight (as our chosen method does), we use a \emph{cutoff threshold} and simply average the diagnosis predictions of images whose relevant view probability is above the cutoff.
For each patient, we use the majority vote prediction of the diagnosis from the images of relevant views.
The value of the cutoff threshold is selected using the validation set to maximize balanced accuracy.

Table~\ref{tab:Suggested_Aggregation_Ablation} shows the performance of this strategy (``threshold-then-average'') on the full-size dataset.
Using this alternative prioritization strategy together with our suggested methodology for patient-level diagnosis (using MixMatch, pretraining on view), we find the average test set balanced accuracy is around 85.8\%, while the weighted average strategy in the main paper achieves over 90\% balanced accuracy. We take this as reasonably decisive evidence that a weighted average (rather than a simple cutoff) should be preferred.

\begin{table}[!h]
    \centering
    \begin{tabular}{l l l|rrrr|c}
    Pretrain & Method & Aggregation across images
    & Split 1  & Split 2 & Split 3 & Split 4 & average\\
    \hline
    & Basic WRN & Threshold-then-Average & 85.42 & 86.25 & 79.17 & 92.50 & 85.84 \\
    %SSL & FS & MixMatch & Priority view + confidence & 94.58 & 84.17 & 77.50 & 92.5 & 87.19\\
    & MixMatch & Threshold-then-Average & 83.33 & 84.17 & 77.50 & 94.58 & 84.90 \\
    view & MixMatch & Threshold-then-Averagen & 86.67 & 80.00 & 82.50 & 94.17 & 85.84\\
    %view & MixMatch & LR with view-priority & 87.08 & 82.08 & 85.00 & 88.75 & 85.73\\
    %(MixMatch transfered) + MysteryMethod & NA & NA & NA\\ 
    \end{tabular}
    \caption{Alternative view-prioritizing strategy for patient-level AS severity diagnosis classification on the \textbf{full-size \datasetName-156-52} dataset, showing balanced accuracy on the test set across multiple folds (each fold’s test set contains 52 patients).}
    %endcaption
    \label{tab:Suggested_Aggregation_Ablation}
\end{table}



%%%%%%
%%
%%
\subsection{ROC Curve of patient-level diagnosis: no AS vs. mild/moderate/severe AS}

Fig.~\ref{fig: No AS vs Some AS} shows receiver operating curves for several methods for the task of distinguishing no AS vs Some AS (which aggregates both the mild/moderate and severe levels in the 3-level diagnosis task of the main paper).

\begin{figure}[!h]
\begin{tabular}{c c}
	\includegraphics[width=0.43\textwidth]{figures/fold0_multitask_PatientLevel_NoVSSome_NormalizedPriorityStrategyClassProbabilityScore.pdf}
	&
    \includegraphics[width=0.43\textwidth]{figures/fold1_multitask_PatientLevel_NoVSSome_NormalizedPriorityStrategyClassProbabilityScore.pdf}
	\\
	(a) Split 1 & (b) Split 2
	\\
	\includegraphics[width=0.43\textwidth]{figures/fold2_multitask_PatientLevel_NoVSSome_NormalizedPriorityStrategyClassProbabilityScore.pdf}
	&
    \includegraphics[width=0.43\textwidth]{figures/fold3_multitask_PatientLevel_NoVSSome_NormalizedPriorityStrategyClassProbabilityScore.pdf}
	\\
	(c) Split 3 & (d) Split 4
\end{tabular}
    
\caption{ROC curves for binary diagnosis task (no AS vs ``mild/moderate/severe AS'') on \textbf{full-size \datasetName-156-52}.
    }%endcaption
    \label{fig: No AS vs Some AS}
\end{figure}

\section{Methodological Details}

\subsection{Image processing details}
\label{sec:removing_doppler}

\paragraph{Removing doppler images.}
In the raw data of all imagery available for an echocardiogram study, 
we obtained TIFF files that represent both cineloops and Doppler images.

We verified in our labeled set that all Doppler images have one of the following landscape aspect ratio $(831, 323)$, $(901, 384)$, $(901, 390)$, $(704, 305)$, $(831, 421)$, $(901, 469)$ or $(563, 294)$. Only the Dopplers have these aspect ratios. We thus filtered out Doppler completely via these aspect ratios. 

\paragraph{Downsizing}
The original images are provided as high-resolution TIFF format images (hundreds of pixels per side) of varying aspect ratios. Generally, we can expect that both view and diagnosis classifiers would perform better given higher-resolution input (and holding other factors the same). The main trade-off of processing higher-resolution images is increased runtime and memory requirements. In our preliminary experiments, we compared downsizing all images to a standard square aspect ratio at 3 possible sizes: 32x32, 64x64 and 128x128. We found that 64x64 achieves a good balance between model performance and computation cost. 
A prior study by \citet{madaniDeepEchocardiographyDataefficient2018} provides a more extensive study of optimal resolution size. The interested reader can refer to their work for more details. 


\subsection{Architecture Settings and Hyperparameters}
\label{sec:arch_and_hyperparameters}

\paragraph{Weighted cross-entropy for labeled loss}
To counteract the effect of class imbalance in the dataset, we use weighted cross-entropy for the labeled loss. For an input image $x$ whose true label $y$ indicates it belongs to class $c$, the weighted cross-entropy assumes the following form:
\begin{align}
\mathcal{L}^L(\theta, x) = - w_{c} \log \hat{p}_{c}(\theta, x),
\end{align}
where $\hat{p}_{c}$ is the predicted probability of class $c$. The weight $w_{c}$ is calculated using the training set statistics as follow:
\begin{align}
w_{c} = \frac{\prod_{k\neq c}{N_{k}}}{\sum_{j}\prod_{k \neq j}{N_{k}}}
\end{align}
where $N_{k}$ is the number of images of class $k$ in the training set.

\paragraph{Common architecture.}
Following~\citet{oliverRealisticEvaluationDeep2018}, for all considered methods, we use the \emph{same} backbone neural network architecture: a wide residual network~\citep{zagoruykoWideResidualNetworks2017} with 28 layers (WRN-28), which has total of 5,931,683 parameters.
This same network architecture is used in the original MixMatch evaluation~\citep{berthelotMixmatchHolisticApproach2019} with promising results.

\paragraph{Common training protocol.}
All SSL methods we consider follow the loss minimization framework with two primary losses (one for ``labeled'' data and one for ``unlabeled'' data) in Eq.~\eqref{eq:standard-SSL-loss-template}.
We allow every method to train for 32 epochs (where each epoch processes $2^{16}$ images, as in \citet{berthelotMixmatchHolisticApproach2019}).
Our preliminary experiments suggest that after 30 epochs all methods effectively converge in terms of validation balanced accuracy. 

\paragraph{Common regularization.}
For all methods, we expect performance will be vulnerable to overfitting, so we impose an L2-norm penalty on the weights $\theta$, also known as weight decay. Each method selects its preferred value of this penalty strength hyperparameter. We searched values in [0.0002, 0.002, 0.02].

\paragraph{Common optimization.}
We use ADAM \citep{kingma2014adam} to optimize each model.
Each method selects the value of the step size (learning rate) as a hyperparameter. We experimented with 0.002 and 0.0007
%HZ: 'performance being sensitive to learning rate' is very reasonable. But we don't have an ablation to back it. 
%We find performance is sensitive to the step size (learning rate) hyperparameter, so we perform a grid search and select the value that maximizes balanced accuracy on the validation set.

\paragraph{Hyperparameters for Pseudo-Label.}
Beyond the usual hyperparameters for our loss-minimization SSL framework, another important hyperparameter for pseudo-label is the threshold $\tau$. We find that performance is not very sensitive to the chosen $\tau$ value as long as it is within a certain range. We set $\tau$ to 0.95, as done in past literature that evaluates Pseudo-Label as an SSL method ~\citep{oliverRealisticEvaluationDeep2018,berthelotMixmatchHolisticApproach2019, berthelotRemixmatchSemisupervisedLearning2019, sohnFixmatchSimplifyingSemisupervised2020}.


\paragraph{Hyperparameters for VAT.}
Beyond the usual hyperparameters for our SSL framework, for VAT we need to select a value for $\epsilon$.
In \citet{miyatoVirtualAdversarialTraining2019}, the authors claimed that they can achieve superior performance by tuning only $\epsilon$ and fixing $\lambda$ to 1. In our experiment, we used the default $\lambda$ as in \cite{berthelotMixmatchHolisticApproach2019} and searched the value of $\epsilon$ in [2, 6, 18], together with learning rate and weight decay. We select the best hyperparameters using validation set performance. 


\paragraph{Hyperparameters for MixMatch.}
Beyond the usual hyperparameters for our SSL framework, the key hyperparameters for MixMatch include the number of augmentations $K$, the temperature $T>0$ used for sharpening, interpolation hyperparameter $\alpha$ and unlabeled loss coefficient $\lambda$. We set $K=2$, $T=0.5$, and $\alpha=0.75$ as done in \citet{berthelotMixmatchHolisticApproach2019}, and search for $\lambda$ in the range [10, 30, 75, 100, 130] using validation set. 

\paragraph{Hyperparameters for Multitask training.}
We searched $\gamma$, the hyperparameter that control the strength of the auxilliary view loss in Eq.~\eqref{eq:multitask}, in the range [10, 3, 1, 0.3, 0.1]. The best $\alpha$ is selected together with other hyperparameters on validation set. 


\end{document}
