%%%%%%%%%%%%%%%%%%%%%%%%%%%%%%%%%%%%
% This is the template for submission to MICRO 2015
% The cls file is a modified from  'sig-alternate.cls'
%%%%%%%%%%%%%%%%%%%%%%%%%%%%%%%%%%%%

\documentclass{sig-alternate}

\newcommand{\ignore}[1]{}
\usepackage{fancyhdr}
\usepackage[normalem]{ulem}
\usepackage[hyphens]{url}
\usepackage{hyperref}
%DIMAN

\pagenumbering{arabic} 
\usepackage{cite}
\usepackage{url}
\usepackage{parskip}
\usepackage{indentfirst} %indents first line of each section
\usepackage{color}
\setlength{\parindent}{10pt}
\usepackage{graphicx}
\usepackage{caption}
\usepackage{subcaption}
\usepackage{diagbox}
\usepackage[linesnumbered,ruled,vlined]{algorithm2e}
\usepackage{natbib}
\setlength{\bibsep}{0.0pt}
\usepackage{flushend}
\usepackage{mathtools}
%\usepackage{amsthm}
\usepackage{amsmath}
\usepackage{verbatim}
\usepackage{pbox}

%\usepackage{enumitem}
%DIMAN
%%%%%%%%%%%---SETME-----%%%%%%%%%%%%%
\newcommand{\microsubmissionnumber}{XXX}
%%%%%%%%%%%%%%%%%%%%%%%%%%%%%%%%%%%%

\fancypagestyle{firstpage}{
  \fancyhf{}
\setlength{\headheight}{50pt}
\renewcommand{\headrulewidth}{0pt}
  \fancyhead[C]{\normalsize{MICRO 2015 Submission
      \textbf{\#\microsubmissionnumber} -- Confidential Draft -- Do NOT Distribute!!}} 
  \pagenumbering{arabic}
}  

%%%%%%%%%%%---SETME-----%%%%%%%%%%%%%
\title{CAGE: A Market-based \underline{C}ontention-\underline{A}ware \underline{G}am\underline{E}-theoretic Distributed model for heterogeneous resource assignment} 
%%%%%%%%%%%%%%%%%%%%%%%%%%%%%%%%%%%%

\begin{document}
\maketitle
\pagenumbering{arabic}
\thispagestyle{plain}
%\thispagestyle{firstpage}
\pagestyle{plain}



%%%%%% -- PAPER CONTENT STARTS-- %%%%%%%%

\begin{abstract}
\noindent Traditional resource management systems rely on a centralized approach to manage users running on each resource. The centralized resource management system is not scalable for large-scale servers as the number of users running on shared resources is increasing dramatically and the centralized manager may not have enough information about applications' need. In this paper we propose a distributed game-theoretic resource management approach using market auction mechanism to find optimal strategy in a resource competition game. The applications learn through repeated interactions to choose their action on choosing the shared resources. Specifically, we look into two case studies of cache competition game and main processor and co-processor congestion game. We enforce costs for each resource and derive bidding strategy. Accurate evaluation of the proposed approach show that our distributed allocation is scalable and outperforms the static and traditional approaches.
\end{abstract}
%%%%%%%%%%%%%%%%%%%%%%%%%%%%%%%%%%%%%%%%%%%%%%%%%%%%%%%%%%%%%%%%%%%%%%%%%%%
%%%%%%%%%%%%%%%%%%%%%%%%%%%%%%%%%%%%%%%%%%%%%%%%%%%%%%%%%%%%%%%%%%%%%%%%%%%
\IEEEraisesectionheading{\section{Introduction}}

\IEEEPARstart{V}{ision} system is studied in orthogonal disciplines spanning from neurophysiology and psychophysics to computer science all with uniform objective: understand the vision system and develop it into an integrated theory of vision. In general, vision or visual perception is the ability of information acquisition from environment, and it's interpretation. According to Gestalt theory, visual elements are perceived as patterns of wholes rather than the sum of constituent parts~\cite{koffka2013principles}. The Gestalt theory through \textit{emergence}, \textit{invariance}, \textit{multistability}, and \textit{reification} properties (aka Gestalt principles), describes how vision recognizes an object as a \textit{whole} from constituent parts. There is an increasing interested to model the cognitive aptitude of visual perception; however, the process is challenging. In the following, a challenge (as an example) per object and motion perception is discussed. 



\subsection{Why do things look as they do?}
In addition to Gestalt principles, an object is characterized with its spatial parameters and material properties. Despite of the novel approaches proposed for material recognition (e.g.,~\cite{sharan2013recognizing}), objects tend to get the attention. Leveraging on an object's spatial properties, material, illumination, and background; the mapping from real world 3D patterns (distal stimulus) to 2D patterns onto retina (proximal stimulus) is many-to-one non-uniquely-invertible mapping~\cite{dicarlo2007untangling,horn1986robot}. There have been novel biology-driven studies for constructing computational models to emulate anatomy and physiology of the brain for real world object recognition (e.g.,~\cite{lowe2004distinctive,serre2007robust,zhang2006svm}), and some studies lead to impressive accuracy. For instance, testing such computational models on gold standard controlled shape sets such as Caltech101 and Caltech256, some methods resulted $<$60\% true-positives~\cite{zhang2006svm,lazebnik2006beyond,mutch2006multiclass,wang2006using}. However, Pinto et al.~\cite{pinto2008real} raised a caution against the pervasiveness of such shape sets by highlighting the unsystematic variations in objects features such as spatial aspects, both between and within object categories. For instance, using a V1-like model (a neuroscientist's null model) with two categories of systematically variant objects, a rapid derogate of performance to 50\% (chance level) is observed~\cite{zhang2006svm}. This observation accentuates the challenges that the infinite number of 2D shapes casted on retina from 3D objects introduces to object recognition. 

Material recognition of an object requires in-depth features to be determined. A mineralogist may describe the luster (i.e., optical quality of the surface) with a vocabulary like greasy, pearly, vitreous, resinous or submetallic; he may describe rocks and minerals with their typical forms such as acicular, dendritic, porous, nodular, or oolitic. We perceive materials from early age even though many of us lack such a rich visual vocabulary as formalized as the mineralogists~\cite{adelson2001seeing}. However, methodizing material perception can be far from trivial. For instance, consider a chrome sphere with every pixel having a correspondence in the environment; hence, the material of the sphere is hidden and shall be inferred implicitly~\cite{shafer2000color,adelson2001seeing}. Therefore, considering object material, object recognition requires surface reflectance, various light sources, and observer's point-of-view to be taken into consideration.


\subsection{What went where?}
Motion is an important aspect in interpreting the interaction with subjects, making the visual perception of movement a critical cognitive ability that helps us with complex tasks such as discriminating moving objects from background, or depth perception by motion parallax. Cognitive susceptibility enables the inference of 2D/3D motion from a sequence of 2D shapes (e.g., movies~\cite{niyogi1994analyzing,little1998recognizing,hayfron2003automatic}), or from a single image frame (e.g., the pose of an athlete runner~\cite{wang2013learning,ramanan2006learning}). However, its challenging to model the susceptibility because of many-to-one relation between distal and proximal stimulus, which makes the local measurements of proximal stimulus inadequate to reason the proper global interpretation. One of the various challenges is called \textit{motion correspondence problem}~\cite{attneave1974apparent,ullman1979interpretation,ramachandran1986perception,dawson1991and}, which refers to recognition of any individual component of proximal stimulus in frame-1 and another component in frame-2 as constituting different glimpses of the same moving component. If one-to-one mapping is intended, $n!$ correspondence matches between $n$ components of two frames exist, which is increased to $2^n$  for one-to-any mappings. To address the challenge, Ullman~\cite{ullman1979interpretation} proposed a method based on nearest neighbor principle, and Dawson~\cite{dawson1991and} introduced an auto associative network model. Dawson's network model~\cite{dawson1991and} iteratively modifies the activation pattern of local measurements to achieve a stable global interpretation. In general, his model applies three constraints as it follows
\begin{inlinelist}
	\item \textit{nearest neighbor principle} (shorter motion correspondence matches are assigned lower costs)
	\item \textit{relative velocity principle} (differences between two motion correspondence matches)
	\item \textit{element integrity principle} (physical coherence of surfaces)
\end{inlinelist}.
According to experimental evaluations (e.g.,~\cite{ullman1979interpretation,ramachandran1986perception,cutting1982minimum}), these three constraints are the aspects of how human visual system solves the motion correspondence problem. Eom et al.~\cite{eom2012heuristic} tackled the motion correspondence problem by considering the relative velocity and the element integrity principles. They studied one-to-any mapping between elements of corresponding fuzzy clusters of two consecutive frames. They have obtained a ranked list of all possible mappings by performing a state-space search. 



\subsection{How a stimuli is recognized in the environment?}

Human subjects are often able to recognize a 3D object from its 2D projections in different orientations~\cite{bartoshuk1960mental}. A common hypothesis for this \textit{spatial ability} is that, an object is represented in memory in its canonical orientation, and a \textit{mental rotation} transformation is applied on the input image, and the transformed image is compared with the object in its canonical orientation~\cite{bartoshuk1960mental}. The time to determine whether two projections portray the same 3D object
\begin{inlinelist}
	\item increase linearly with respect to the angular disparity~\cite{bartoshuk1960mental,cooperau1973time,cooper1976demonstration}
	\item is independent from the complexity of the 3D object~\cite{cooper1973chronometric}
\end{inlinelist}.
Shepard and Metzler~\cite{shepard1971mental} interpreted this finding as it follows: \textit{human subjects mentally rotate one portray at a constant speed until it is aligned with the other portray.}



\subsection{State of the Art}

The linear mapping transformation determination between two objects is generalized as determining optimal linear transformation matrix for a set of observed vectors, which is first proposed by Grace Wahba in 1965~\cite{wahba1965least} as it follows. 
\textit{Given two sets of $n$ points $\{v_1, v_2, \dots v_n\}$, and $\{v_1^*, v_2^* \dots v_n^*\}$, where $n \geq 2$, find the rotation matrix $M$ (i.e., the orthogonal matrix with determinant +1) which brings the first set into the best least squares coincidence with the second. That is, find $M$ matrix which minimizes}
\begin{equation}
	\sum_{j=1}^{n} \vert v_j^* - Mv_j \vert^2
\end{equation}

Multiple solutions for the \textit{Wahba's problem} have been published, such as Paul Davenport's q-method. Some notable algorithms after Davenport's q-method were published; of that QUaternion ESTimator (QU\-EST)~\cite{shuster2012three}, Fast Optimal Attitude Matrix \-(FOAM)~\cite{markley1993attitude} and Slower Optimal Matrix Algorithm (SOMA)~\cite{markley1993attitude}, and singular value decomposition (SVD) based algorithms, such as Markley’s SVD-based method~\cite{markley1988attitude}. 

In statistical shape analysis, the linear mapping transformation determination challenge is studied as Procrustes problem. Procrustes analysis finds a transformation matrix that maps two input shapes closest possible on each other. Solutions for Procrustes problem are reviewed in~\cite{gower2004procrustes,viklands2006algorithms}. For orthogonal Procrustes problem, Wolfgang Kabsch proposed a SVD-based method~\cite{kabsch1976solution} by minimizing the root mean squared deviation of two input sets when the determinant of rotation matrix is $1$. In addition to Kabsch’s partial Procrustes superimposition (covers translation and rotation), other full Procrustes superimpositions (covers translation, uniform scaling, rotation/reflection) have been proposed~\cite{gower2004procrustes,viklands2006algorithms}. The determination of optimal linear mapping transformation matrix using different approaches of Procrustes analysis has wide range of applications, spanning from forging human hand mimics in anthropomorphic robotic hand~\cite{xu2012design}, to the assessment of two-dimensional perimeter spread models such as fire~\cite{duff2012procrustes}, and the analysis of MRI scans in brain morphology studies~\cite{martin2013correlation}.

\subsection{Our Contribution}

The present study methodizes the aforementioned mentioned cognitive susceptibilities into a cognitive-driven linear mapping transformation determination algorithm. The method leverages on mental rotation cognitive stages~\cite{johnson1990speed} which are defined as it follows
\begin{inlinelist}
	\item a mental image of the object is created
	\item object is mentally rotated until a comparison is made
	\item objects are assessed whether they are the same
	\item the decision is reported
\end{inlinelist}.
Accordingly, the proposed method creates hierarchical abstractions of shapes~\cite{greene2009briefest} with increasing level of details~\cite{konkle2010scene}. The abstractions are presented in a vector space. A graph of linear transformations is created by circular-shift permutations (i.e., rotation superimposition) of vectors. The graph is then hierarchically traversed for closest mapping linear transformation determination. 

Despite of numerous novel algorithms to calculate linear mapping transformation, such as those proposed for Procrustes analysis, the novelty of the presented method is being a cognitive-driven approach. This method augments promising discoveries on motion/object perception into a linear mapping transformation determination algorithm.



\section{Motivation and Background} \label{Motivation}
\subsection{Motivation}
Different applications have different resource constraint with respect to CPU, memory, and bandwidth usage. Having a single resource manager for all existing resources and users in the system result in inefficiencies since it is not scalable and the operating system may not have enough information about application's needs. For example, traditional LRU-based cache strategy uses cache utilization as a metric to give larger cache size to the applications which have higher utilization and lower cache size to the applications with lower cache utilization. However more cache utilization does not always result in better performance. Streaming applications for example have very high cache utilization, but very small cache reuse. In fact, the streaming applications only need a small cache space to buffer the streaming data. With rapid improvements in semiconductor technology, more and more cores are being embedded into a single core and managing large scale application using a single resource manager becomes more challenging. \\
%\indent Even if the applications are forced to announce their resource demand, it is possible that they lie about their resource vector or run some useless instructions to pretend to utilize the allocated resources given to them.
\indent In addition, defining a single fairness parameter for multiple applications is non-trivial since applications have different bottlenecks and may get different performance benefits from each resources during each phases of its execution time. Defining a single reasonable parameter for fairness is somewhat problematic. For instance, simple assignment algorithms which try to equally distribute the resources between all applications ignores the fact that different applications have different resource constraints. As a consequence, this makes the centralized resource management systems very inefficient in terms of fairness as well as performance needs of applications. We need a decentralized framework, where all applications' performance benefit could be translated into a unique notion of fairness and performance objective (known as utility function in economics) and the algorithm tries to allocate resources based on this translated notion of fairness. This translation has been well defined in economics and marketing, where the diversity of customer needs, makes more economically efficient market \cite{zhou2014sharing}.\\
\indent Economists have shown that in an economically efficient market, having diverse resource constraints and letting the customers compete for the resources can make a Nash equilibrium where both the applications and the resource managers can be enriched. \\
\indent Furthermore, applications' demand changes over time. Most resource allocation schemes pre-allocate the resources without considering the dynamism in applications' need and number of users sharing the same resource over time. Therefore, applications' performance can degrade drastically over time. Figure~\ref{fig:Phases} shows phase transitions for instruction per cycle (IPC) of mcf application from \textit{spec 2006} over 50 billion instructions. \\ 
%%%%%%%%%%%%%%%%%%%%%%%%%%%%%%%%%%%%%%%%%%%%%%%%%%%%%%%%%%%%%%%%%%%%%%%%%%%
\begin{figure}[!tb]
\centering
%\includegraphics[height=3in, width=1.5in]{NodeArchs2.pdf}
\includegraphics[height=1.5in, width=3.3in]{Images/Phases_May.pdf} %Phases.pdf
%\epsfig{file=Dataset.eps, height=2.5in, width=3in}
\caption{\label{fig:Phases}Phase transition in mcf with different L2 cache sizes.}
\end{figure}
%%%%%%%%%%%%%%%%%%%%%%%%%%%%%%%%%%%%%%%%%%%%%%%%%%%%%%%%%%%%%%%%%%%%%%%%%%%
\indent We try to find a game-theoretic distributed resource management approach where the shared hardware resources are exposed to the applications and we will show that running a repeated auction game between different applications which are assumed to be rational, the output of the game would converge to a balanced Nash equilibrium allocation. In addition, we will compare the convergence time of the proposed algorithm in terms of dynamism in the system. We will evaluate our model with two case studies: 1- Private and Shared last level cache problem, where the applications have to decide if they would benefit from a larger cache space which can potentially get more congested or a smaller cache space which is potentially less congested. Based on the number of other applications in the system the application can change its strategy over the time. 2- Heterogeneous processors (\textit{Intel Xeon} and \textit{Xeon Phi}) problem, where we perform experiments to show how congestion affects the performance of different applications running on an \textit{Intel Xeon} or \textit{Xeon Phi} co-processors. Based on the congestion in the system the application can offload the most time consuming part of its code on \textit{Xeon Phi} co-processors or not.   
%%%%%%%%%%%%%%%%%%%%%%%%%%%%%%%%%%%%%%%%%%%%%%%%%%%%%%%%%%%%%%%%%%%%%%%%%%%
\subsection{Background}
%Congestion games have been studied in network routing protocols where the delay of each player choosing a path in the network depends on the number of players choosing the same route in the system. 
%Every congestion game is a potential game since there exists a potential function associated with it. In addition, every congestion game has a pure-strategy Nash equilibrium. A key assumption in congestion games is that all users have the same impact on the congestion. However, this assumption is not always true. In case of computer architecture resources, applications effect each other differently and dividing the pay-off function by the number of users running on the shared resource does not give us the correct utility. 
%%%%%%%%%%%%%%%%%%%%%%%%%%%%%%%%%%%%%%%%%%%%%%%%%%%%%%%%%%%%%%%%%%%%%%%%%%
Game theory has been used extensively in economics, political and social decision making situations \cite{tootaghaj2011game, tootaghaj2011risk, kotobi2017spectrum, kotobi2015introduction, kesidis2013distributed, kurve2013agent, wang2017using, wang2015recouping}. A game is a situation, where the the output of each player not only depends on her own action in the game, but also on the action of other players \cite{osborne1994course}. Auction games are a class of games which has been used to formulate real world problems of assigning different resources between $n$ users. Auction game framework can model resource competition, where the payoff (cost) of each application in the system is a function of the contention level (number of applications) in the game.\\
\indent Inspired by market-based interactions in real life games, there exists a repeated interaction between competitors in a resource sharing game. Assuming large number of applications, we show that the system gets to a Nash equilibrium where all applications are happy with their resource assignment and don't want to change their state. Furthermore, we show that the auction model is strategy-proof, such that no application can get more utilization by bidding more or less than the true value of the resource. In this paper we propose a distributed market based approach to enforce cost on each resource in the system and remove the complexity of resource assignment from the central decision maker.\\ 
\indent The traditional resource assignment is performed by the operating system or a central hardware to assign fair amount of resources to different applications. However, fair scheduling is not always optimal and solving the optimization problem of assigning $m$ resources between $n$ users in the system is an integer programming which is an NP-hard problem and finding the best assignment problem becomes computationally infeasible. Prior works focus on designing a fair scheduling function that maximizes all application's benefit \cite{zahedi2014ref, llull2017cooper, ghodsi2011dominant, zahedi2015sharing, fan2016computational}, while applications might have completely different demands and it is not possible to use the same fairness function for all. By shifting decision making to the individual applications, the system becomes scalable and the burden of establishing fairness is removed from the centralized decision maker, since individual applications have to compete for the resources they need. Applications start with the profiling utility functions for each resource and bid for the most profitable resource. During the course of execution time they can update their belief based on the observed performance metrics at each round of the auction. The idea behind updating the utility functions is that the history at each round of decision point shows the state of the game. This state indicates the contention on the current acquired resource. The pay-off function in each round depends on the state of the system and on the action of other applications in the system. 
%%%%%%%%%%%%%%%%%%%%%%%%%%%%%%%%%%%%%%%%%%%%%%%%%%%%%%%%%%%%%%%%%%%%%%%%%%%
\subsubsection{Sequential Auction}
Auction-based algorithms are used for maximum weighted perfect matching in a bipartite graph $G=(U,V, E)$ \cite{bertsekas1998network, kyle1985continuous, vasconcelos2009bipartite}. A vertex  $U_i \in U$ is the application in the auction and a vertex $V_j \in V$ is interpreted as a resource. The weight of each edge from $U_i$ to $V_j$ shows the utility of getting that particular resource by $U_i$. The prices are initially set to zero and will be updated during each iteration of the auction. In sequential auctions, each resource is taken out by the the auctioneer and is sequentially auctioned to the applications, until all the resources are sold out.
\subsubsection{Parallel Auction}
In a parallel auction, the applications submit their bids for the first most profitable item. The value of the bid at each iteration is computed based on the difference of the highest profitable object and the second highest profitable object. The auctioneer would assign the resources based on the current bids. At each iteration, the valuation of each resource is updated based on the observed information during run-time which shows the contention on that particular resource.
%%%%%%%%%%%%%%%%%%%%%%%%%%%%%%%%%%%%%%%%%%%%%%%%%%%%%%%%%%%%%%%%%%%%%%%%%%%%
%%%%%%%%%%%%%%%%%%%%%%%%%%%%%%%%%%%%%%%%%%%%%%%%%%%%%%%%%%%%%%%%%%%%%%%%%%%%
\section{CAGE: A Market-based Contention-aware Game-theoretic resource assignment}\label{Problem_definition}
\subsection{Model Description}
Consider $n$ applications and $i$ instances of $m$ different resources. Applications arrive in the system one at a time. The applications have to choose among $m$ resources. There exists a bipartite graph between the matching of the applications and the resources.\\
\indent In general, there can be more than one application to get a shared resources. However, each application can not get more than one of the available heterogeneous resources. For example, if we have two cache space of 128kB (one way) and 256kB (two ways), the application can either get the 128kB of cache space or 256kB and can't get both of them at the same time. Furthermore, each resource $m_i$ has a cost $C_i$ which is defined by the applications' bid in the auction. \\
\indent Figure~\ref{fig:auction} shows auction-based framework to support \textit{CAGE} between $N$ applications that execute together competing for $M$ different resources. Each application has a utility table that shows how much performance it gets from each $M$ resources at each time slot. Based on the utility tables, applications submit bids for the most profitable resource. Based on the submitted bids, the auctioneer decides about the resource assignment for each resource, and updates the prices. Next, the applications who did not get any assignment compete for the next most profitable resource based on the updated prices repeatedly until all applications are assigned.  Figure~\ref{fig:auction} shows an example of a resource assignment and the corresponding bipartite graph.
%Table~\ref{table:notation} shows the notation used in our formulation.
%%%%%%%%%%%%%%%%%%%%%%%%%%%%%%%%%%%%%%%%%%%%%%%%%%%%%%%%%%%%%%%%%%%%%%%%%%%%%%%
%%%%%%%%%%%%%%%%%%%%%%%%%%%%%%%%%%%%%%%%%%%%%%%%%%%%%%%%%%%%%%%%%%%%%%%%%% 
\begin{table}[!tb] 
\centering
\caption{Notation used in our formulations.}\label{Table:notation}
\begin{tabular}{|p{0.7in}||p{2.3in}|} 
\hline $N$ & Number of applications \\
\hline $K$ & Number of cache levels \\
\hline $T$ & Time intervals where the bidding is hold \\
\hline $m$ & Number of applications which can get a resource \\ 
\hline $p$ & Number of phases for each application during its course of execution time \\ 
\hline $n$ & Number of applications competing for a specified resource \\
\hline $M$ & Number of resources \\
\hline $P_i$ & Number of phases for application $i$ \\
\hline $\delta$ & dynamic factor that shows how much we can rely on the past iterations. \\
\hline $U$ & The applications which shows the left set of nodes in the bipartite graph. \\
\hline $V$ & The resources which shows the right set of nodes in the bipartite graph. \\
\hline $E$ & The edges in the bipartite graph. \\
\hline $G=(U,V,E)$ & A bipartite graph showing the resource allocation between the applications and the set of resources. \\
\hline $b_{i,k}$ & User i's bid for k th resource \\
\hline $B_i$ & The total budget (sum of bids) a user have \\
\hline $C_k$ & The total capacity of each resource \\
\hline $p_{j}$ & The price of resource $j \in V$ in the auction. \\
\hline $Bottleneck_{1,i}$ & The first bottleneck resource for application $i$ \\
\hline $Bottleneck_{2,i}$ & The second bottleneck resource for application $i$ \\
\hline $v_{i,m}(T)$ & The valuation function of application $i$ for resource $m$ at time $T$ \\
\hline
\end{tabular}
\end{table}
%%%%%%%%%%%%%%%%%%%%%%%%%%%%%%%%%%%%%%%%%%%%%%%%%%%%%%%%%%%%%%%%%%%%%%%%%% 
%%%%%%%%%%%%%%%%%%%%%%%%%%%%%%%%%%%%%%%%%%%%%%%%%%%%%%%%%%%%%%%%%%%%%%%%%%%%%%%
\begin{figure*}[!htb]
\centering
%\includegraphics[height=3in, width=1.5in]{NodeArchs2.pdf}
\includegraphics[height=3.2in, width=6.5in]{Images/Auction_v2.pdf} %[height=4in, width=8in]
%\epsfig{file=Dataset.eps, height=2.5in, width=3in}
\caption{\label{fig:auction} Framework for auction-based resource assignment (CAGE).}
\end{figure*}
%%%%%%%%%%%%%%%%%%%%%%%%%%%%%%%%%%%%%%%%%%%%%%%%%%%%%%%%%%%%%%%%%%%%%%%%%%%
\begin{comment}
\begin{figure}[!htb]
\centering
%\includegraphics[height=3in, width=1.5in]{NodeArchs2.pdf}
\includegraphics[height=2.2in, width=1.3in]{Images/bipartite.pdf}
%\epsfig{file=Dataset.eps, height=2.5in, width=3in}
\caption{\label{fig:bipartite} Cache allocation as a bipartite graph.}
\end{figure}
\end{comment}
%%%%%%%%%%%%%%%%%%%%%%%%%%%%%%%%%%%%%%%%%%%%%%%%%%%%%%%%%%%%%%%%%%%%%%%%%%%
\subsection{Problem Defenition} 
\indent We formulate our problem as an auction based mechanism to enforce cost/value updates for each resource as follows: \\
%The cost of each player to get a resource is the cost of the assigned resource divided by the number of players who share. 
\begin{itemize}
  \item \textbf{Valuation $\mathbf{v_{i,m}}$} : Any application has a valuation function which shows how much he benefits from $i th$ resource. The valuation function at time $t=0$ for cache contention case study is derived from the IPC (instruction per cycle) curves which is found using profiling, and for processor and co-processor contention case study is derived from the profiling solo performance metric of the application. However, in general, each application can choose its own utility function.  
  %%%%%%%%%%%%%%%%%%%%%%%%%%%%%%%%%%%%%%%%%%%%%%%%%%%%%% 
    \item \textbf{Observed information}: The observed information at each time step is the performance value of the selected action in the game. Therefore, the applications repeatedly update the history of their valuation function over time.  
  %%%%%%%%%%%%%%%%%%%%%%%%%%%%%%%%%%%%%%%%%%%%%%%%%%%%%%   
    \item \textbf{Belief updating}: At each iteration step of the auction, the applications update their valuation of each resource based on the observed performance on each resource. The update at time $T$ is derived using the following formula:
%\begin{small}
\begin{equation}\label{eq:belief}
v_{i,m}(T)=\frac{\sum\limits_{t=0}^T {\delta}^{T-t}  v_{i,m}(t)}{\sum\limits_{t=0}^T {\delta}^{T-t}} 
\end{equation}  
%\end{small}  
%%%%%%%%%%%%%%%%%%%%%%%%%%%%%%%%%%%%%%%%%%%%%%%%%%%%%%%%%%%%%%%%%%%%%%%%%
Where $v_{i,m}(t)$ shows the observed valuation of resource $m$ at time step $t$ by user $i$ in the system; $\delta$ shows the discount factor between 0 and 1 which shows how much a user relies on its past observations in the system. The discount factor is chosen to show the dynamics in the system. If the observed information in the system changes fast, the discount factor is nearly zero which means that we can't rely on the past observations very much. However if the system is more stable and the observed information does not change fast, the discount factor is chosen to be near 1. We choose the discount factor as the absolute value of the correlation coefficient of the observed values of the valuations at each iteration step which is calculated as follows:
%\begin{small}
\begin{equation}
\delta =  \frac{E(v_{i,m})^2}{{\sigma_{v_{i,m}}}^2}
\end{equation}  
%\end{small}
%%%%%%%%%%%%%%%%%%%%%%%%%%%%%%%%%%%%%%%%%%%%%%%%%%%%%%%%%%%%%%%%%%%%%%%%
%%%%%%%%%%%%%%%%%%%%%%%%%%%%%%%%%%%%%%%%%%%%%%%%%%%%%%%%%%%%%%%%%%%%%%%%
  \item \textbf{Action}: At each time step the applications decides which resource to bid and how much to bid for each resource. 
\end{itemize} 
%%%%%%%%%%%%%%%%%%%%%%%%%%%%%%%%%%%%%%%%%%%%%%%%%%%%%%%%%%%%%%%%%%%%%%%%%% 
\indent Table~\ref{Table:notation} shows important notation used throughout the paper. In the following sections, we describe our distributed optimization scheme to solve the problem. 
%%%%%%%%%%%%%%%%%%%%%%%%%%%%%%%%%%%%%%%%%%%%%%%%%%%%%%%%%%%%%%%%%%%%%%%
%\begin{equation}
%min \sum\limits_{i=1}^n v_i C_k \frac{b_{i,k}}{\theta_k}, \\
%s.t. \sum\limits_{i=1}^n b_{i,k} \leq E_i
%\end{equation}
%%%%%%%%%%%%%%%%%%%%%%%%%%%%%%%%%%%%%%%%%%%%%%%%%%%%%%%%%%%%%%%%%%%%%%%
\subsection{Distributed Optimization Scheme}
The goal is to design a repeated auction mechanism which is run by the operating system to guide the applications to choose their best resource allocation strategy. The applications' goal is to maximize their own performance and the operating system wants to maximize the total utility it gains from the applications. Then, each application can use its own utility function and evaluates the resources based on how much it likes that particular resource. \\
\indent \textbf{Applications' approach}: The application $i$ want to maximize the total utility with respect to a limited budget for all phase $p$ of its execution time. \\
%%%%%%%%%%%%%%%%%%%%%%%%%%%%%%%%%%%%%%%%%%%%%%%%%%%%%%%%%%%%%%%%%%%%%%%%%
%maximize \;\;\;\; \sum\limits_{i=1}^n v_i C_k \frac{b_{i,k}}{\theta_k},\\
%\begin{small}
\begin{align}
%\begin{IEEEeqnarray}{rCl}
\forall i \in U \; \; \; \; \; maximize \; \; \; \; \sum\limits_{p=1}^{P_i} \sum\limits_{m=1}^M  v_{i,m,p}-b_{i,m,p} , \nonumber \\
 % \IEEEyessubnumber\\
subject \; to \;\;\;\; \sum\limits_{p=1}^{P_i} \sum\limits_{m=1}^M b_{i,m,p} \leq B_i .
%\IEEEyessubnumber
%\end{IEEEeqnarray}
\end{align}
%\end{small}
%%%%%%%%%%%%%%%%%%%%%%%%%%%%%%%%%%%%%%%%%%%%%%%%%%%%%%%%%%%%%%%%%%%%%%%%%%
\indent \textbf{OS's approach}: The operating system wants to maximize the social welfare function which is translated into submitted bids from the applications in a limited resource constraints.\\
%%%%%%%%%%%%%%%%%%%%%%%%%%%%%%%%%%%%%%%%%%%%%%%%%%%%%%%%%%%%%%%%%%%%%%%%%%
%\begin{small}
\begin{align}
%\begin{IEEEeqnarray}{rCl}
maximize \; \; \; \sum\limits_{i=1}^N \sum\limits_{p=1}^{P_i} \sum\limits_{m=1}^M b_{i,m,p} A_{i,m,p} , \nonumber \\ 
%\IEEEyessubnumber\\
subject \; to \;\;\;\; \sum\limits_{i=1}^N \sum\limits_{m=1}^M A_{i,m,p} \leq A_{max}, \; \; \; \; \forall p \in P , \nonumber \\
%\IEEEyessubnumber\\
A_{i,m,p} \in \{0,1\} , \; \; \; \; \forall i \in U, \; \;  \forall m \in V, \; \;  \forall p \in P .
%\IEEEyessubnumber
%\end{IEEEeqnarray}
\end{align}
%\end{small}
%%%%%%%%%%%%%%%%%%%%%%%%%%%%%%%%%%%%%%%%%%%%%%%%%%%%%%%%%%%%%%%%%%%%%%%%%%%
\indent \textbf{Illustrative example}: As an illustrative example, suppose we have two different resources, a large cache of 1MB which can be shared between applications, and two private caches of 512KB which are not shared. There are two applications competing for the cache space. One of the applications wants to minimize its request latency and the other one wants to maximize number of instructions executed per cycle. Suppose that both applications have two phases $(0,T)$ and $(T,2T)$.  Suppose if the first application gets the larger cache space its request latency reduces by 20 percent in first phase and by 40 percent in the second phase. The second application's \textit{IPC} increases by 35 percent in the first phase and by 25 percent in the second phase if it gets the larges cache space. Also, assume they both have 60 tokens (bids) to submit. The first application invests 20 token (bids) for the first phase and 40 tokens for the next phase. He should redistribute the tokens for the next phase if he did not get the resource he wants in the first phase. The second application invests 35 tokens in the first phase and 25 tokens in the next phase. The auctioneer (OS) at each phase decides to allocate which resource to which applications. Since, the social welfare would be maximized if the auctioneer allocates both applications with the larger cache space, they would both get the larger resource. Then the first application notices that its utility function does not improve as he predicts and adjusts the utility table and can either change its allocation or stay on current allocation. 
%If both applications bid 10\$ for the private cache and 15\$ for the shared cache, the operating system would allocate both the shared cache space and get 15\$ from each to maximize its revenue.  
\subsection{Analysis}
The distributed optimization problem seems complex. However, in reality the problem can be splitted into simpler subproblems since each application knows its bottleneck resource and would first bid for the first bottleneck resource to maximize its utility.\\
\indent We suppose all applications in the system are risk-neutral which means they have a linear valuation of utility function. Each risk neutral agent wants to maximize its expected revenue. Risk attitude behaviors are defined in \cite{ferber1999multi} where the agents can broadly be divided into risk averse, risk seeking and risk neutral. Risk averse agents prefer determinitic values rather than risky value profits and risk seeking applications have a superlinear utility function and prefer risky utilities than sure utilities. Next, we derive the Bayes Nash equilibrium strategy profile for all agents in the system assuming risk neutrality.  \\
%%%%%%%%%%%%%%%%%%%%%%%%%%%%%%%%%%%%%%%%%%%%%%%%%%%%%%%%%%%%%%%%%%
\newtheorem{defi}{Definition}
\begin{defi}
A strategy profile $a$ is a pure Nash equilibrium if for every application $i$ and every strategy $a_i' \neq a_i \in A$ we have $u_i(a_i, a_{-i}) \geq u_i(a_i', a_{-i})$
\end{defi}
%%%%%%%%%%%%%%%%%%%%%%%%%%%%%%%%%%%%%%%%%%%%%%%%%%%%%%%%%%%%%%%%%%
\newtheorem{theorem}{Theorem}
\begin{theorem}\label{thm:neat}
%\emph{(Theorem)}
\label{Auction}
Suppose $n$ risk-neutral applications whose valuations are derived uniformly and independently from the interval $[0,1]$ compete for one resource which can be assigned to $m$ application who have the highest bid in the auction. We will show that Bayes Nash equilibrium bidding strategy for each application in the system is to bid $\frac{n-m}{n-m+1}v_i$ whre $v_i$ is the profit of application $i$ for getting the specified resource.  
\end{theorem}
%%%%%%%%%%%%%%%%%%%%%%%%%%%%%%%%%%%%%%%%%%%%%%%%%%%%%%%%%%%%%%%
%DIMAN COMMENTED PROOF FOR APPENDIX
\begin{comment}

\begin{proof}
Suppose all other applications' bidding strategy is to choose $\frac{n-m}{n-m+1}v_i$. Since the bidding values were derived uniformly in $[0, 1]$ all bids have the same probability. Therefore, if we consider the first application's expected utility to find its best response, we have:

\begin{equation}
%\begin{IEEEeqnarray}{rCl}
E[u_1] = \int_0^1  .... \int_0^1 \! (v_1 -b_1) \, \mathrm{d}u_2 \mathrm{d}u_3 ... \mathrm{d}u_{n-m} .  
%\end{IEEEeqnarray}
\end{equation}

The following integral breaks into two part where the first application wins the auction or not. 


%\begin{IEEEeqnarray}{rCl}
%E[u_1] = \int_0^{b_1}  .... \int_0^{b_1} \! (v_1 -b_1) \, \mathrm{d}u_2 \mathrm{d}u_3 ... %\mathrm{d}u_{n-m}  \\
%+ \int_{b_1}^1  .... \int_{b_1}^1 \! (v_1 -b_1) \, \mathrm{d}u_2 \mathrm{d}u_3 ... \mathrm{d}%u_{n-m}\nonumber 
%\end{IEEEeqnarray}

\begin{equation}
%\begin{IEEEeqnarray}{rCl}
E[u_1] = \int_0^{\frac{n-m+1}{n-m}b_1}  .... \int_0^{\frac{n-m+1}{n-m}b_1} \! (v_1 -b_1) \, \mathrm{d}u_2 ... \mathrm{d}u_{n-m}  \\
+ \int_{\frac{n-m+1}{n-m}v_1}^1  .... \int_{\frac{n-m+1}{n-m}v_1}^1 \! (v_1 -b_1) \, \mathrm{d}u_2 \mathrm{d}u_3 ... \mathrm{d}u_{n-m}\nonumber 
%\end{IEEEeqnarray}
\end{equation}

The second part of the integrals is the term where the first application doesn't win the auction. Therfore, the expected payoff of application 1 is equal with:

%\begin{IEEEeqnarray}{rCl}
%E[u_1] = \int_0^{b_1}  .... \int_0^{b_1} \! (v_1 -b_1) \, \mathrm{d}u_2 \mathrm{d}u_3 ... %\mathrm{d}u_{n-m}  \\
%= {({b_1}) }^{n-m} (v_1 -b_1). \nonumber 
%\end{IEEEeqnarray}

\begin{equation}
%\begin{IEEEeqnarray}{rCl}
E[u_1] =\int_0^{\frac{n-m+1}{n-m}b_1}  .... \int_0^{\frac{n-m+1}{n-m}b_1} \! (v_1 -b_1)  \mathrm{d}u_2 ... \mathrm{d}u_{n-m}  \\
= {(\frac{n-m+1}{n-m} b_1) }^{n-m} (v_1 -b_1). \nonumber 
%\end{IEEEeqnarray}
\end{equation}


Diffrentiating with respect to $b_1$ the optimal bid for application one is derived as follows:

%\begin{equation}
%\frac{\partial}{\partial b_1} ( {({b_1}) }^{n-m} (v_1 -b_1))=0.
%\end{equation}


\begin{equation}
\frac{\partial}{\partial b_1} ( {(\frac{n-m+1}{n-m} b_1) }^{n-m} (v_1 -b_1))=0.
\end{equation}

Which gives us the optimal bid for each application:
\begin{equation}
\Rightarrow b_1= \frac{n-m}{n-m+1}v_1
\end{equation}
\end{proof}
\end{comment}
%DIMAN COMMENTED PROOF FOR APPENDIX
%%%%%%%%%%%%%%%%%%%%%%%%%%%%%%%%%%%%%%%%%%%%%%%%%%%%%%%%%%%%%%%%%%%%%%%%%%%%
\begin{algorithm}[!tb]
\DontPrintSemicolon % Some LaTeX compilers require you to use \dontprintsemicolon    instead
\KwIn{A bipartite Graph (U, V, E).}
\KwOut{The allocation of resources to applications.}
At t=0 the valuation of each application for each resource is derived using profiling while running alone. 

For each application $U_i \in U$, the first bottleneck resource is
\[ Bottleneck_{1,i} = V_{i,m}=  arg \; \max_{m \in V} (v_{i,m}-p_{m})  \] 
Next, find the second bottleneck resource for each applications $U_i \in U$ in the system:
\[ Bottleneck_{2,i} = V_{i,k}=  arg \; \max_{k \in V, k \neq m} (v_{i,k}-p_{k})  \] 

Each application submits the bid for its first bottleneck resource using the following formula:
\[ b_{i,m} = V_m - V_k + p_{j} + \epsilon \]
Each resource $V_j \in V$, which can be shared between $m$ applications, is assigned to the $m$ highest bidding applications $Winner_j={i_1, i_2, ..., i_m}$ and the price for that resource is updated as follows:
\[ p_{j} =  arg \; \max_{i_1, i_2, ..., i_m \in U} \sum\limits_{k=1}^m (b_{i_k,j})  \]

The $minBid$ for each resource is updated as the minimum bid of $m$ applications who acquired the resource. That is
\[ minBid=  arg \; \min_{i \in Winner_j}  (b_{i,j}) \] 
 
\caption{CAGE: Parallel Auction for heterogeneous resource assignment.}
\label{algo:b}
\vspace{0\baselineskip}
\end{algorithm}
%%%%%%%%%%%%%%%%%%%%%%%%%%%%%%%%%%%%%%%%%%%%%%%%%%%%%%%%%%%%%%%%%%%%%%%%%%%%%
\indent Theorem~\ref{thm:neat}, states that whenever there is a single resource that users compete to get it with different valuation functions, the Nash equilibrium strategy profile for risk-neutral users is to bid $\frac{n-m}{n-m+1}v_i$. This term tends to the true value of the object when n is a large number. \\
\indent In case of more than one resource competition we derive Algorithm~\ref{algo:b} and will prove that it is Nash equilibrium in the game. The algorithm is inspired by work of Bertsekas \cite{bertsekas1998network} that uses an auction for network flow problems. In the first step, all valuations are set to the solo-run of application's performance. Next, each application submits a bid for its first bottleneck resource. The bid should be larger than the price of the object which is intitialized to zero in the begining of the program. The applications only have incentive to bid a value no more than the difference of the first bottleneck and second bottleneck resource. Otherwise, it would submit a smaller bid to the second bottleneck and get the same revenue as paying more for the first bottleneck resource. In order to break the equal valuation function between two different applications, we use $\epsilon$ scaling such that at each iteration of the auction the prices should increase by a small number. 

%In addition, suppose we have 5 different memory bandwidth exposed to the applications Each application gets different performance benefit from different cache sizes and different memory bandwidth which is denoted in table  **. The applications need to submit their bids based on their performance benefits. 

%\begin{equation}
%\begin{split}
%\int_0^\frac{nb_1}{n-m}  .... \int_0^\frac{nb_1}{n-m} \! (\frac{v_1}{m} -b_1) \, \mathrm{d}u_2 %\mathrm{d}u_3 ... \mathrm{d}u_{n-m}= \\
%= {(\frac{nb_1}{n-m}) }^{n-m} (\frac{v_1}{m} -b_1). 
%\end{split}
%\end{equation}
%%%%%%%%%%%%%%%%%%%%%%%%%%%%%%%%%%%%%%%%%%%%%%%%%%%%%%%%%%%%%%%%%%%%%%%%%%%%
%\[ \frac{\partial}{\partial b_1} ({(\frac{nb_1}{n-m}) }^{n-m} (\frac{v_1}{m} -b_1))

%\newtheorem{defi}{Definition}
%\begin{defi}
%Let's assume each user $Ui$ in the system is defined as one vertice of a graph $G$ in the system and let each edge in the graph show which subset of users can impact each others' performance and the associated weight of each edge show the cost function of how two users affect each others' performance in the system. Each edge has a weight function denoted by ${P1(n), P2(n), ... Pe(n)}$, where $e$ is the number of edges in Graph $G$. Let $A=A_1 \times A_2 \times ... \times A_n $ be the set of actions that each user can play. 
%\end{defi}
\section{Case Studies} \label{Case_Studies}
%\subsection{A case study of Main processor (Xeon) and co-processor (Xeon-Phi) congestion game}
\subsection{CPU Scale-up Scale-out Game}
The emerging high performance computing applications lead to the advent of \textit{Intel Xeon Phi} co-processor, that when their highly parallel architecture is fully utilized, can run order of magnitude more performance than the existing processor architectures. The \textit{Xeon Phi} co-processors are the first commercial product of Intel \textit{MIC} processors where the hardware architecture is exposed to the programmer to choose running the code on either \textit{Xeon} processor or \textit{Xeon Phi} co-processors. It is possible that, during the course of execution, either the processor or the co-processor get congested and the performance of the application degrades a lot. Therefore, making a decision to offload the most time consuming part of the program on \textit{Xeon} or \textit{Xeon Phi} should be made online, based on the contention level.  In this section we look at the case study of our auction-based model on decision making of running the application on the main or co-processor in a highly congested environment. \\
\indent The experiment results of this section are run on \textit{Stampede} cluster of \textit{Texas Advanced Computing Center}. Table~\ref{Table:Xeon} shows the comparison of \textit{Intel Xeon} and \textit{Xeon Phi} architectures which is used in this section. 
%%%%%%%%%%%%%%%%%%%%%%%%%%%%%%%%%%%%%%%%%%%%%%%%%%%%%%%%%%%%%%%%%%%%%%%%%%%
%%%%%%%%%%%%%%%%%%%%%%%%%%%%%%%%%%%%%%%%%%%%%%%%%%%%%%%%%%%%%%%%%%%%%%%%%%%
%%%%%%%%%%%%%%%%%%%%%%%%%%%%%%%%%%%%%%%%%%%%%%%%%%%%%%%%%%%%%%%%%%%%%%%%%%%
\begin{table}[!tb] 
\centering
\caption{Comparison of \textit{Intel Xeon} and \textit{Xeon Phi} Processors.}\label{Table:Xeon}
\begin{tabular}{|c||p{0.8in}||p{1in}|} 
\hline Processors & Xeon E5-2680 & Xeon Phi SE10P \\
\hline Cores/Sockets & 8/2 & 61/1 \\
\hline Clock Frequency & 2.7 GHz & 1.1 GHz  \\
\hline Memory & 32GB 8x4G 4-channels DDR3-1600MHz & 8GB GDDR5 \\
\hline L1 cache & 32 KB & 32 KB \\
\hline L2 cache & 256 KB & 512 KB \\
\hline L3 cache & 20 MB & - \\
\hline
\end{tabular}
\end{table}
%%%%%%%%%%%%%%%%%%%%%%%%%%%%%%%%%%%%%%%%%%%%%%%%%%%%%%%%%%%%%%%%%%%%%%%%%%%
%%%%%%%%%%%%%%%%%%%%%%%%%%%%%%%%%%%%%%%%%%%%%%%%%%%%%%%%%%%%%%%%%%%%%%%%%%%
%%%%%%%%%%%%%%%%%%%%%%%%%%%%%%%%%%%%%%%%%%%%%%%%%%%%%%%%%%%%%%%%%%%%%%%%%%%
\indent It is observed that congestion has a significant effect on the performance of running the application on \textit{Xeon} and \textit{Xeon Phi} machines. Since most cloud computing machines are shared between thousands of users, the programmer not only should get benefit of parallelism by offloading the most time consuming part of the code to the larger number of low-performance cores (\textit{Xeon Phi}), but also should consider the congestion level (number of co-runners) in the system. To this end, we performed experiments on \textit{Stampede} clusters. We executed \textit{MiniGhost} application which is a part of \textit{Mantevo} project \cite{mantevo} which uses difference stencils to solve partial differential equations using numerical methods. The applications use the profiling utility functions at $t=0$ and during course of execution can update the utility function based on the observed performance on each core using Equation~\ref{eq:belief}. Then, they can revisit their previous action on running the code on either the processor or co-processor during run-time. \\
\indent Figure~\ref{Fig:congestion} shows the total execution time with respect to congestion we made in \textit{Xeon} and \textit{Xeon Phi}. In this experiment we ran the same problem size on a \textit{Xeon} and \textit{Xeon Phi} machine multiple times, so that we could see the effect of load on the total execution time of our application. It was observed that with the same number of threads \textit{Xeon}'s performance degrades more than \textit{Xeon phi}. 
Next, we tried to change the application behavior using congestion-aware game theoretic algorithm to offload the most time consuming part of the application based on the performance behavior of applications. Figure~\ref{Fig:performance_over_time} shows the result of our game-theoretic model during the execution time. It is observed that during the course of execution, the applications change their strategy on either choosing the main processor or the co-processor and all applications' performance converge to a equilibrium point where applications don't want to change their strategy. \\
\indent Furthermore, it is shown that CAGE can bring in up to 106.6\% improvement in total execution time of applications compared to static approach when the number of co-runners is six. The performance improvement would be significant when the number of co-runners increase. Figure~\ref{Fig:Perfomance_Comparison} shows the performance comparison of CAGE and static approach which does not consider the congestion dynamism in the system and the decision is only made based on the parallelism level in the code. 
%%%%%%%%%%%%%%%%%%%%%%%%%%%%%%%%%%%%%%%%%%%%%%%%%%%%%%%%%%%%%%%%%%%%%%%%%%%
%%%%%%%%%%%%%%%%%%%%%%%%%%%%%%%%%%%%%%%%%%%%%%%%%%%%%%%%%%%%%%%%%%%%%%%%%%%
%%%%%%%%%%%%%%%%%%%%%%%%%%%%%%%%%%%%%%%%%%%%%%%%%%%%%%%%%%%%%%%%%%%%%%%%%%%
\begin{figure}[!htb]
\centering
%\includegraphics[height=3in, width=2.5in]{NodeArchs2.pdf}
\includegraphics[height=1.5in, width=3.5in]{Images/Xeon.pdf} % diman.pdf
%\epsfig{file=Dataset.eps, height=2.5in, width=3in}
\caption{Congestion effect on \textit{Xeon} and \textit{Xeon Phi} machines.}\label{Fig:congestion}
\end{figure}
%%%%%%%%%%%%%%%%%%%%%%%%%%%%%%%%%%%%%%%%%%%%%%%%%%%%%%%%%%%%%%%%%%%%%%%%%%%
%%%%%%%%%%%%%%%%%%%%%%%%%%%%%%%%%%%%%%%%%%%%%%%%%%%%%%%%%%%%%%%%%%%%%%%%%%%
\begin{figure}[!tb]
\centering
%\includegraphics[height=3in, width=2.5in]{NodeArchs2.pdf}
\includegraphics[height=1.5in, width=3.5in]{Images/Game.pdf}%Game_during_time.pdf
%\epsfig{file=Dataset.eps, height=2.5in, width=3in}
\caption{Performance of 6 instance of applications during time for our proposed game model.}\label{Fig:performance_over_time}
\end{figure}
%%%%%%%%%%%%%%%%%%%%%%%%%%%%%%%%%%%%%%%%%%%%%%%%%%%%%%%%%%%%%%%%%%%%%%%%%%%
%%%%%%%%%%%%%%%%%%%%%%%%%%%%%%%%%%%%%%%%%%%%%%%%%%%%%%%%%%%%%%%%%%%%%%%%%%%
\begin{figure}[!htb]
\centering
%\includegraphics[height=3in, width=1.5in]{NodeArchs2.pdf}
\includegraphics[height=1.5in, width=3.5in]{Images/Congestion.pdf}  %Congestion_aware.pdf
%\epsfig{file=Dataset.eps, height=2.5in, width=3in}
\caption{Performance comparison of congestion-aware schedule versus static schedule.}\label{Fig:Perfomance_Comparison}
\end{figure}
%%%%%%%%%%%%%%%%%%%%%%%%%%%%%%%%%%%%%%%%%%%%%%%%%%%%%%%%%%%%%%%%%%%%%%%%%%%
%%%%%%%%%%%%%%%%%%%%%%%%%%%%%%%%%%%%%%%%%%%%%%%%%%%%%%%%%%%%%%%%%%%%%%%%%%%
%%%%%%%%%%%%%%%%%%%%%%%%%%%%%%%%%%%%%%%%%%%%%%%%%%%%%%%%%%%%%%%%%%%%%%%%%%%
\subsection{A Case Study of Private and shared cache game}
One of the challenging problems in \textit{CMP} resource management systems is whether applications benefit from a shared large last level cache or an isolated private cache. We evaluated CAGE performance, on a 10MB LLC shown in Figure~\ref{Fig:cache_hierarchy}, where 2MB, 1MB, 512kB, 256kB and 128kB levels of LLC can potentially be shared between 16, 8, 4, 2 and 1 applications respectively, the cache levels have 16, 8, 4, 2, and 1 ways. Table~\ref{Table:Workloads} summarizes the studies workloads and their characteristics, including miss per kilo instructions (\textit{MPKI}), memory bandwidth usage, and IPC. We use applications from \textit{Spec 2006} benchmark suite \cite{Spec:website}. We use \textit{Gem5} full system simulator in our experiment ~\cite{binkert2011gem5, Gem5:website}. Table~\ref{Table:Experimental_Set_Up} shows the experimental setup in our experiments.\\
\indent To evaluate the performance of our proposed approach we use utility functions for different number of ways shown in Figure~\ref{fig:Cache_IPC}. These utility functions at the start of the execution can be found using either profiling techniques or stack distance profile \cite{kim2004fair, suh2002new, suh2004dynamic} of applications assuming there is no co-runners in the system. Next, during run-time the applications can update their utility functions based on Equation~\ref{eq:belief}. Therefore, there is a learning phase where applications learn about the state of the system and update the utilities accordingly. The stack distance profile indicates how many more cache misses will be added if the application has less number of ways in the cache. Based on the stack distance profile, the applications can update their utility function and bid for the next iteration of the auction if they like to change their allocation. Next, we bring an example of the auction for one time step of the game. This time step can be repeated once an application arrives or leaves the system or when an application's phase changes during run-time. However, in case of one application's phase change or arriving or leaving the system, the algorithm reaches the optimal assignment in much fewer iterations since all other assignments are fixed and a few applications would be affected.\\
%%%%%%%%%%%%%%%%%%%%%%%%%%%%%%%%%%%%%%%%%%%%%%%%%%%%%%%%%%%%%%%%%%%%%%%%%%%
\begin{figure}[!tb]
\centering
%\includegraphics[height=3in, width=2.5in]{NodeArchs2.pdf}
\includegraphics[height=1.5in, width=3.3in]{Images/Cache_Hierarchy_2.pdf}
%\epsfig{file=Dataset.eps, height=2.5in, width=3in}
\caption{Our proposed last level cache hierarchy model.}\label{Fig:cache_hierarchy}
\end{figure}
%%%%%%%%%%%%%%%%%%%%%%%%%%%%%%%%%%%%%%%%%%%%%%%%%%%%%%%%%%%%%%%%%%%%%%%%%%%
%%%%%%%%%%%%%%%%%%%%%%%%%%%%%%%%%%%%%%%%%%%%%%%%%%%%%%%%%%%%%%%%%%%%%%%%%
\begin{table}[!tb] 
\centering
\caption{Experimental Setup.}
\label{Table:Experimental_Set_Up}
\begin{tabular}{|c||p{1.5in}|} 
\hline Processors & Single threaded with private L1 instruction and data caches \\
\hline Frequency & 1GHz \\
\hline L1 Private ICache & 32 kB, 64-byte lines, 4-way associative\\
\hline L1 Private DCache & 32 kB, 64-byte lines, 4-way associative \\
\hline L2 Shared Cache & 128 kb-2 MB, 64-byte lines, 16-way associative \\
\hline RAM & 12 GB \\

\hline
\end{tabular}
\end{table}
%%%%%%%%%%%%%%%%%%%%%%%%%%%%%%%%%%%%%%%%%%%%%%%%%%%%%%%%%%%%%%%%%%%%%%%%%
\begin{table}[!tb] 
\centering
\caption{Evaluated workloads.}
\label{Table:Workloads}
\begin{tabular}{p{0.7cm} p{1.5cm} p{1cm} p{1.7cm} p{1cm} }
\hline
%\begin{tabular}{|l|c|c|c|} \hline
{\bf \#} & \bf Benchmark & MPKI & Memory BW & IPC  \\
\hline 
{\bf 1} & astar & 1.319 & 373 MB/s & 2.057 \\
{\bf 2} & bwaves & 10.47 & 1715 MB/s & 0.661 \\
{\bf 3} & bzip2 & 3.557 & 1194 MB/s & 1.367 \\ 
{\bf 4} & dealII &  0.935 & 307 MB/s & 2.107 \\
{\bf 5} & GemsFDTD & 0.004 & 2.19 MB/s & 2.023 \\
{\bf 6} & hmmer & 2.113 & 1547 MB/s & 2.861 \\
{\bf 7} & lbm & 19.287 & 3954 MB/s & 0.533 \\
{\bf 8} & leslie3d & 8.469 & 1942 MB/s & 1.297 \\
{\bf 9} & libquantum & 10.388 & 1589 MB/s & 0.531 \\
{\bf 10} & mcf & 16.93 & 820 MB/s & 0.073 \\
{\bf 11} & namd & 0.051 & 20.32 MB/s & 2.362\\
{\bf 12} & omnetpp & 10.34 & 1147 MB/s & 0.504 \\
{\bf 13} & sjeng & 0.375 & 139.2 MB/s & 1.403 \\
{\bf 14} & soplex & 4.672 & 390.8 MB/s & 0.513 \\
{\bf 15} & sphinx3 & 0.349 & 202.8 MB/s & 2.223 \\
{\bf 16} & streamL & 31.682 & 3619 MB/s & 0.581 \\
{\bf 17} & tonto & 0.260 & 107 MB/s & 2.036 \\
{\bf 18} & xalancbmk & 12.703 & 1200 MB/s & 0.558 \\
%\hline  
\hline
\end{tabular}
\end{table}
%%%%%%%%%%%%%%%%%%%%%%%%%%%%%%%%%%%%%%%%%%%%%%%%%%%%%%%%%%%%%%%%%%%%%%%%%%
\begin{comment}
Assume $n$ different applications denoted by ${U1, U2, .. Un}$ with different cache benefits which may affect each other with different cost functions. Figure 1 shows the different applications which are the vertices of the graph with their impact on each other which are the weights of edges in the graph. If two vertices are not connected in the graph, it means that they would not affect each others' performance. For example, one application is CPU bound and does not benefit from larger memory bandwith and the other is memory bound and does not benefit from having more CPU capacity. So different applications affect each other's performance with different coefficients. 
\end{comment}
%%%%%%%%%%%%%%%%%%%%%%%%%%%%%%%%%%%%%%%%%%%%%%%%%%%%%%%%%%%%%%%%%%%%%%%%%%%
%%%%%%%%%%%%%%%%%%%%%%%%%%%%%%%%%%%%%%%%%%%%%%%%%%%%%%%%%%%%%%%%%%%%%%%%%%%
%%%%%%%%%%%%%%%%%%%%%%%%%%%%%%%%%%%%%%%%%%%%%%%%%%%%%%%%%%%%%%%%%%%%%%%%%%%
\begin{figure*}[!tb]
\centering
\includegraphics[height=3.5in, width=6.5in]{Images/Cache_IPC_v2.pdf}
\caption{IPC for different size of LLC.}\label{fig:Cache_IPC}  
\end{figure*}
%%%%%%%%%%%%%%%%%%%%%%%%%%%%%%%%%%%%%%%%%%%%%%%%%%%%%%%%%%%%%%%%%%%%%%%%%%%
%%%%%%%%%%%%%%%%%%%%%%%%%%%%%%%%%%%%%%%%%%%%%%%%%%%%%%%%%%%%%%%%%%%%%%%%%%%
%%%%%%%%%%%%%%%%%%%%%%%%%%%%%%%%%%%%%%%%%%%%%%%%%%%%%%%%%%%%%%%%%%%%%%%%%%%
\indent \textbf{Example:} As an example, suppose we have 5 different applications and 5 different cache levels with different capacities of 128KB, 256KB, 512KB, 1MB and 2MB. In addition, suppose the 128kB cache level can not accomodate more than one application and 256kB cache can accomodate 2 applications, 512kB level can have 4 applications, 1MB cache can have 8 applications and 2MB cache can have at most 16 applications. Let's assume the following matrix be the utility function of each application on each cache level. \\
\indent Some applications may get better utility from smaller cache space since they are less congested and since these applications have low data locality, moving to larger cache spaces not only does not increase their performance but also degrades the performance by evicting other applications from the cache and making contention on the memory bandwidth which is a more vital resource for them \footnote{\textit{libquantum}, \textit{streamL}, \textit{sphinx3}, \textit{lbm} and \textit{mcf} are examples of such applications.}.  \\
\begin{equation}
M = \bordermatrix{~ & 1way & 2way & 4way & 8way & 16way \cr
  App1 & 1.9 & 1.7 & 1.5 & 1 & 0.9 \cr
  App2 & 1.6 & 1.3 & 1.1 & 0.8 & 0.7 \cr
  App3 & 1.4 & 1.0 & 0.6 & 0.5 & 0.4 \cr
  App4 & 0.3 & 0.6 & 0.9 & 1.2 & 1.4 \cr
  App5 & 0.7 & 0.8 & 1.1 & 1.4 & 1.7 \cr}
\end{equation}
%%%%%%%%%%%%%%%%%%%%%%%%%%%%%%%%%%%%%%%%%%%%%%%%%%%%%%%%%%%%%%%%%%%%%%%%%%%
%%%%%%%%%%%%%%%%%%%%%%%%%%%%%%%%%%%%%%%%%%%%%%%%%%%%%%%%%%%%%%%%%%%%%%%%%%%
In the first iteration of the bidding, the first 3 applications bid for the most profitable resource which is 128kB cache and they submit a bid equal to the difference of profit between the first and the second most profitable resource. Therefore, the first application, submits 0.2 bid to 128kb and the second application submits 0.3 and the third application submits 0.4. Since only one of the players can aqcuire the 128kB cache space, the first application will get it. The 4th and 5th application compete for 2MB cache space and they both get it with the sum bid of both which is 0.5. In the next round, the prices will be updated and since apllications 2 and 3 don't have any cache assignment compete for the 256kB cache space and each bid 0.2 which is the difference between 1.7 and 1.5 and 1.3 and 1.1 in the performance matrix accordingly. Since the second level cache can accomodate both applications the price will be updated and the minimum bidding price for some one to get this cache level is updated to the minimum bid of both which is 0.2. Therefore, if some application bid more than 0.2 it can acquire the resource and the application with smallest bid has to resubmit the bid to acquire the resource. Figure~\ref{fig:first_round}, ~\ref{fig:second_round}, and ~\ref{fig:third_round} show the bidding steps and the prices and minimum price of bidding accordingly. As seen from the figures, the auction terminates in three iterations when there exists five applications. \\
%%%%%%%%%%%%%%%%%%%%%%%%%%%%%%%%%%%%%%%%%%%%%%%%%%%%%%%%%%%%%%%%%%%%%%%%%%%
%%%%%%%%%%%%%%%%%%%%%%%%%%%%%%%%%%%%%%%%%%%%%%%%%%%%%%%%%%%%%%%%%%%%%%%%%%%
%%%%%%%%%%%%%%%%%%%%%%%%%%%%%%%%%%%%%%%%%%%%%%%%%%%%%%%%%%%%%%%%%%%%%%%%%%% 
\begin{figure*}[!htb]
        \centering
        \begin{subfigure}[b]{0.25\textwidth} %//0.28 bood
                \includegraphics[width=\textwidth]{Images/bid0.pdf}
                \caption{first round.}
                \label{fig:first_round}
        \end{subfigure}%
        ~ %add desired spacing between images, e. g. ~, \quad, \qquad etc.
          %(or a blank line to force the subfigure onto a new line)
        \begin{subfigure}[b]{0.25\textwidth}
                \includegraphics[width=\textwidth]{Images/bid2.pdf}
                \caption{second round.}
                \label{fig:second_round}
        \end{subfigure}
        ~ %add desired spacing between images, e. g. ~, \quad, \qquad etc.
          %(or a blank line to force the subfigure onto a new line)
        \begin{subfigure}[b]{0.25\textwidth}
                \includegraphics[width=\textwidth]{Images/bid3.pdf}
                \caption{third round.}
                \label{fig:third_round}
        \end{subfigure}  

                \caption{Cache allocation, a) first round, b) second round and c) third round of bidding.}\label{fig:Auction_rounds}    
       % \vspace{-2\baselineskip}
\end{figure*}
%%%%%%%%%%%%%%%%%%%%%%%%%%%%%%%%%%%%%%%%%%%%%%%%%%%%%%%%%%%%%%%%%%%%%%%%%%%
%%%%%%%%%%%%%%%%%%%%%%%%%%%%%%%%%%%%%%%%%%%%%%%%%%%%%%%%%%%%%%%%%%%%%%%%%%%
\indent Next, we use different mixes of 4 to 16 applications from \textit{Spec 2006} to evaluate the performance of our proposed approach. Figure~\ref{fig:IPC_mix} shows the normalized throughput of 10 different mix of applications using CAGE, equal private cache partitions and completely shared cache space. Figure~\ref{fig:scalability} shows the scalability of our proposed algorithm. When the number of co-runners increases from 2 to 16, the performance improves from 12.4\% to 33.6\% without any need to track each applications' performance in a central hardware. 
%%%%%%%%%%%%%%%%%%%%%%%%%%%%%%%%%%%%%%%%%%%%%%%%%%%%%%%%%%%%%%%%%%%%%%%%%%%
\begin{figure}[!tb]
\centering
%\includegraphics[height=3in, width=1.5in]{NodeArchs2.pdf}
\includegraphics[height=1.5in, width=3.5in]{Images/IPC.pdf}
%\epsfig{file=Dataset.eps, height=2.5in, width=3in}
\caption{Throughput of a shared, solo and CAGE cache allocation schemes.}
\label{fig:IPC_mix}
\end{figure}
%%%%%%%%%%%%%%%%%%%%%%%%%%%%%%%%%%%%%%%%%%%%%%%%%%%%%%%%%%%%%%%%%%%%%%%%%%%
%%%%%%%%%%%%%%%%%%%%%%%%%%%%%%%%%%%%%%%%%%%%%%%%%%%%%%%%%%%%%%%%%%%%%%%%%%%
\begin{figure}[!tb]
\centering
%\includegraphics[height=3in, width=1.5in]{NodeArchs2.pdf}
\includegraphics[height=1.5in, width=3.5in]{Images/Scalability.pdf}
%\epsfig{file=Dataset.eps, height=2.5in, width=3in}
\caption{Performance improvement of CAGE for different number of applications with respect to shared LLC for the case study of cache congestion game.}
\label{fig:scalability}
\end{figure}
%%%%%%%%%%%%%%%%%%%%%%%%%%%%%%%%%%%%%%%%%%%%%%%%%%%%%%%%%%%%%%%%%%%%%%%%%%%
%%%%%%%%%%%%%%%%%%%%%%%%%%%%%%%%%%%%%%%%%%%%%%%%%%%%%%%%%%%%%%%%%%%%%%%%%%%
\begin{comment}
\subsection{Experimental setup}


\newtheorem{defin}{Definition}
\begin{defin}
Let's assume each user $Ui$ in the system is defined as one vertice of a graph $G$ in the system and let each edge in the graph show which subset of users can impact each others' performance and the associated weight of each edge show the cost function of how two users affect each others' performance in the system. Each edge has a weight function denoted by ${P1(n), P2(n), ... Pe(n)}$, where $e$ is the number of edges in Graph $G$. Let $A=A_1 \times A_2 \times ... \times A_n $ be the set of actions that each user can play. 
\end{defin}

This representation of the cache congestion game has a space complexity of which is exponential in terms of larges degree in the graph. 
If the number of users and the action set is polynomially bounded then the game has a polynomial representation and  




For a simple illustrative example assume two applications which would affect each other's performance with $f_i$. The payoff table of a subset of users who can effect each other can be shown in Table 1. We can easily extend the game for $n$ users using $f_1$ and $f_2$ to be a function of the number of users which can impact each other.
We used Gem5, full system simulator running Spec OMP benchmarks to find how different applications sharing a specific amount of shared cache impact each other's performance. 
The application which we did experiments on are listed in Table 2.
\end{comment}  









\section{Related Work} \label{Related_works}
With rapid improvement in computer technology, more and more cores are embedded in a single chip and applications competing for a shared resource is becoming common. On the one hand, managing scheduling of shared resources for large number of applications is challenging in a sense that the operating system doesn't know what is the performance metric for each application. But on the other hand, the operating system has a global view of the whole state of the system and can guide applications on choosing the shared resources.\\ 
\indent There has been several works, for managing the shared cache in multi-core systems. Qureshi et al. \cite{qureshi2006utility} showed that assigning more cache space to applications with more cache utility does not always lead to better performance since there exists applications with very low cache reuse which may have very high cache utilization. \\
\indent Several software and hardware approaches has been proposed to find the optimal partitioning of cache space for different applications \cite{zhuravlev2010addressing}. However, most of these approaches use brute force search of all possible combinations to find the best cache partitioning in run time or introduce a lot of overhead. There has been some approaches which use binary search to reduce searching all possible combinations \cite{kim2004fair, lin2008gaining, tam2009rapidmrc}. But none of these methods are scalable for the future many-core processor designs.\\
\indent There exists prior game-theoretic approaches designing a centralized scheduling framework that aims at a fair optimization of applications' utility \cite{zahedi2014ref, llull2017cooper, ghodsi2011dominant, zahedi2015sharing, fan2016computational}. Zahedi et al. in REF \cite{zahedi2014ref, zahedi2015sharing} use the Cobb-Douglas production function as a fair allocator for cache and memory bandwidth. They show that the Cobb-Douglas function provides game-theoretic properties such as sharing incentives, envy-freedom, and Pareto efficiency. But their approach is still centralized and spatially divides the shared resources to enforce a fair near-optimal policy sacrificing the performance. In their approach the centralized scheduler assumes all applications have the same priority for cache and memory bandwidth, while we do not have any assumption on this. Further, our auction-based resource allocation can be used for any number of resources and any priority for each application and the centralized scheduler does not need to have a global knowledge of these priorities.  \\
\indent Ghodi et al. in DRF \cite{ghodsi2011dominant} use another centralized fair policy to maximize the dominant resource utilization. But in practice it is not possible to clone any number of instances of each resources. %  the underlying scenario cloning the instances is very limited or superficial for practical purposes. 
Cooper \cite{llull2017cooper} enhances REF to capture colocated applications fairly, but it only addresses the special case of having two sets of applications with matched resources. Fan et al. \cite{fan2016computational} exploits computational sprinting architecture to improve task throughput assuming a class of applications where boosting their performance by increasing the power. \\
\indent While all prior works use a centralized scheduling that provides fairness and assume the same utility function for all, co-runners might have completely diverse needs and it is not efficient to use the same fairness/performance policy across them. 
In our auction-based resource scheduling provides scalability since individual applications compete for the shared resources based on their utility and the burden of decision making is removed from the central scheduler. We believe future CMPs should move toward a more decentralized approach which is more scalable and provides fair allocation of resources based on applications' needs. \\ 
%Providing the scalability of the system is getting better, if we make the individual applications our self administrators, and we can remove most of the performance-restricting policies such as fairness constraints from the centralized decision-maker. \\
\indent Auction theory which is a subfield of economics has recently been used as a tool to solve large scale resource assignement in cloud computing \cite{krishna2009auction, parsons2011auctions}. In an auction process, the buyers submit bids to get the commodities and sellers want to sell their commodities with the maximum price as possible. \\
\indent Our auction-based algorithm is inspired by work of Bertsekas \cite{bertsekas1998network} that uses an auction based mechanism for network flow problems. Our algorithm is an extension of local assignment problem proposed by Bertsekas et al that has been shown to converge to the global assignment within a linear approximation.
%\textcolor{red}{Not Complete yet}
\section{Conclusion}
We have presented a neural performance rendering system to generate high-quality geometry and photo-realistic textures of human-object interaction activities in novel views using sparse RGB cameras only. 
%
Our layer-wise scene decoupling strategy enables explicit disentanglement of human and object for robust reconstruction and photo-realistic rendering under challenging occlusion caused by interactions. 
%
Specifically, the proposed implicit human-object capture scheme with occlusion-aware human implicit regression and human-aware object tracking enables consistent 4D human-object dynamic geometry reconstruction.
%
Additionally, our layer-wise human-object rendering scheme encodes the occlusion information and human motion priors to provide high-resolution and photo-realistic texture results of interaction activities in the novel views.
%
Extensive experimental results demonstrate the effectiveness of our approach for compelling performance capture and rendering in various challenging scenarios with human-object interactions under the sparse setting.
%
We believe that it is a critical step for dynamic reconstruction under human-object interactions and neural human performance analysis, with many potential applications in VR/AR, entertainment,  human behavior analysis and immersive telepresence.





%%%%%%% -- PAPER CONTENT ENDS -- %%%%%%%%


%%%%%%%%% -- BIB STYLE AND FILE -- %%%%%%%%
%\bibliographystyle{ieeetr}
%\bibliography{ref}
%%%%%%%%%%%%%%%%%%%%%%%%%%%%%%%%%%%%
\bibliographystyle{unsrt} %unsrt
\bibliography{bib/IEEEconf} 
\newpage
\appendix
\section{Pricing equations}
\subsection{Credit default swap}
\label{CDS_pricing}
A credit default swap (CDS) is a contract designed to exchange credit risk of a Reference Name (RN) between a Protection Buyer (PB) and a Protection Seller (PS). PB makes periodic coupon payments to PS conditional on no default of RN, up to the nearest payment date, in the exchange for receiving from PS the loss given RN's default.

Consider a CDS contract written on the first bank (RN), denote its price $C_1(t, x)$.\footnote{For the CDS contracts written on the second bank, the similar expression could be provided by analogy.} We assume that the coupon is paid continuously and equals to $c$. Then, the value of a standard CDS contract can be given (\cite{BieleckiRutkowski}) by the solution of  (\ref{kolm_1})--(\ref{kolm_2})  with $\chi(t, x) = c$ and terminal condition
\begin{equation*}
	\psi(x) = 
	\begin{cases}
		1 - \min(R_1, \tilde{R}_1(1)), \quad (x_1, x_2) \in D_2, \\
		1 - \min(R_1, \tilde{R}_1(\omega_2)), \quad (x_1, x_2) \in D_{12}, \\		
	\end{cases}
\end{equation*}
where $\omega_2 = \omega_2(x)$ is defined in (\ref{term_cond}) and 
\begin{equation*}
	\tilde{R}_1(\omega_2) = \min \left[1, \frac{A_1(T) +  \omega_2 L_{2 1}(T)}{L_1(T) + \omega_2 L_{12}(T)}\right].
\end{equation*}
Thus, the pricing problem for CDS contract on the first bank is
\begin{equation}
\begin{aligned}
		& \frac{\partial}{\partial t} C_1(t, x) + \mathcal{L} C_1(t, x) = c, \\
		& C_1(t, 0, x_2) = 1 - R_1, \quad C_1(t, \infty, x_2) = -c(T-t), \\
		& C_1(t, x_1, 0) = \Xi(t, x_1) = 
		\begin{cases}
			c_{1,0}(t, x_1), & x_1 \ge \tilde{\mu}_1, \\
			1-R_1, & x_1 < \tilde{\mu}_i,
		\end{cases} \quad C_1(t, x_1, \infty) = c_{1,\infty}(t, x_1),\\
		& C_1(T, x) = \psi(x) = 
	\begin{cases}
		1 - \min(R_1, \tilde{R}_1(1)), \quad (x_1, x_2) \in D_2, \\
		1 - \min(R_1, \tilde{R}_1(\omega_2)), \quad (x_1, x_2) \in D_{12}, \\		
	\end{cases}
\end{aligned}
\end{equation}
where $c_{1,0}(t, x_1)$ is the solution of the following boundary value problem:
\begin{equation}
\begin{aligned}
		& \frac{\partial}{\partial t} c_{1, 0}(t, x_1) + \mathcal{L}_1 c_{1, 0}(t, x_1) = c, \\
		& c_{1, 0}(t, \tilde{\mu}_1^{<}) = 1 - R_1, \quad c_{1, 0}(t, \infty) = -c(T-t), \\
		& c_{1, 0}(T, x_1) = (1 - R_1) \mathbbm{1}_{\{\tilde{\mu}_1^{<} \le x_1 \le \tilde{\mu}_1^{=}\}}, 
\end{aligned}
\end{equation}
and $c_{1,\infty}(t, x_1)$ is the solution of the following boundary value problem
\begin{equation}
\begin{aligned}
		& \frac{\partial}{\partial t} c_{1, \infty}(t, x_1) + \mathcal{L}_1 c_{1, \infty}(t, x_1) = c, \\
		& c_{1, \infty}(t, 0) = 1 - R_1, \quad c_{1, \infty}(t, \infty) = -c(T-t), \\
		& c_{1, \infty}(T, x_1) = (1 - R_1) \mathbbm{1}_{\{x_1 \le \mu_1^{=}\}}.
\end{aligned}
\end{equation}

\subsection{First-to-default swap}
An FTD contract refers to a basket of reference names (RN). Similar to a regular CDS, the Protection Buyer (PB) pays a regular coupon payment $c$ to the Protection Seller (PS) up to the first default of any of the RN in the basket or maturity time $T$. In return, PS compensates PB the loss caused by the first default.

Consider the FTD contract referenced on $2$ banks, and denote its price $F(t, x)$. We assume that the coupon is paid continuously and equals to $c$. Then, the value of FTD contract can be given (\cite{LiptonItkin2015}) by the solution of  (\ref{kolm_1})--(\ref{kolm_2})  with $\chi(t, x) = c$ and terminal condition
\begin{equation*}
	\psi(x) = \beta_0  \mathbbm{1}_{\{x \in D_{12}\}} + \beta_1 \mathbbm{1}_{\{x \in D_{1}\}} + \beta_2 \mathbbm{1}_{\{x \in D_{2}\}},
\end{equation*}
where
\begin{equation*}
	\begin{aligned}
		\beta_0 = 1 - \min[\min(R_1, \tilde{R}_1(\omega_2), \min(R_2, \tilde{R}_2(\omega_1)], \\
		\beta_1 = 1 - \min(R_2, \tilde{R}_2(1)), \quad \beta_2 = 1 - \min(R_1, \tilde{R}_1(1)),
	\end{aligned}
\end{equation*}
and
\begin{equation*}
	\tilde{R}_1(\omega_2) = \min \left[1, \frac{A_1(T) +  \omega_2 L_{2 1}(T)}{L_1(T) + \omega_2 L_{12}(T)}\right], \quad \tilde{R}_2(\omega_1) = \min \left[1, \frac{A_2(T) +  \omega_1 L_{1 2}(T)}{L_2(T) + \omega_1 L_{21}(T)}\right].
\end{equation*}
with $\omega_1 = \omega_1(x)$ and $\omega_2 = \omega_2(x)$ defined in (\ref{term_cond}).

Thus, the pricing problem for a FTD contract is
\begin{equation}
\begin{aligned}
		& \frac{\partial}{\partial t} F(t, x) + \mathcal{L} F(t, x) = c, \\
		& F(t, x_1, 0) = 1 - R_2,  \quad F(t, 0, x_2) = 1 - R_1, \\
		& F(t, x_1, \infty) = f_{2,\infty}(t, x_1), \quad F(t, \infty, x_2) = f_{1,\infty}(t, x_2), \\
		& F(T, x) = \beta_0  \mathbbm{1}_{\{x \in D_{12}\}} + \beta_1 \mathbbm{1}_{\{x \in D_{1}\}} + \beta_2 \mathbbm{1}_{\{x \in D_{2}\}},
\end{aligned}
\end{equation}
where $f_{1,\infty}(t, x_1)$ and $f_{2,\infty}(t, x_2)$ are the solutions of the following boundary value problems
\begin{equation}
\begin{aligned}
		& \frac{\partial}{\partial t} f_{i, \infty}(t, x_i) + \mathcal{L}_i f_{i, \infty}(t, x_i) = c, \\
		& f_{i, \infty}(t, 0) = 1 - R_i, \quad f_{i, \infty}(t, \infty) = -c(T-t), \\
		& f_{1, \infty}(T, x_i) = (1 - R_i) \mathbbm{1}_{\{x_i \le \mu_i^{=}\}}.
\end{aligned}
\end{equation}

\subsection{Credit and Debt Value Adjustments for CDS}

Credit Value Adjustment and Debt Value Adjustment can be considered either unilateral or bilateral. For unilateral counterparty risk, we need to consider only two banks (RN, and PS for CVA and PB for DVA), and a two-dimensional problem can be formulated, while bilateral counterparty risk requires a three-dimensional problem, where Reference Name, Protection Buyer, and Protection Seller are all taken into account. We follow \cite{LiptonSav} for the pricing problem formulation but include jumps and mutual liabilities, which affects the boundary conditions.

\paragraph{Unilateral CVA and DVA}
The Credit Value Adjustment represents the additional price associated with the possibility of a counterparty's default. Then, CVA can be defined as
\begin{equation}
	V^{CVA} = (1- R_{PS}) \mathbb{E}[\mathbbm{1}_{\{\tau^{PS} < \min(T, \tau^{RN}) \}} (V_{\tau^{PS}}^{CDS})^{+} \, | \mathcal{F}_t],
\end{equation}
where $R_{PS}$ is the recovery rate of PS, $\tau^{PS}$ and $\tau^{RN}$ are the default times of PS and RN, and $V_t^{CDS}$ is the price of a CDS without counterparty credit risk.

We associate $x_1$ with the Protection Seller and $x_2$ with the Reference Name, then CVA can be given by the solution of  (\ref{kolm_1})--(\ref{kolm_2})  with $\chi(t, x) = 0$ and $\psi(x) = 0$. Thus,
\begin{equation}
\begin{aligned}
		& \frac{\partial}{\partial t} V^{CVA}+ \mathcal{L} V^{CVA} = 0, \\
		& V^{CVA}(t, 0, x_2) = (1 - R_{PS}) V^{CDS}(t, x_2)^{+}, \quad V^{CVA}(t, x_1, 0) = 0, \\
		& V^{CVA}(T, x_1, x_2) = 0.
\end{aligned}
\end{equation}

Similar, Debt Value Adjustment represents the additional price associated with the default and defined as
\begin{equation}
	V^{DVA} = (1- R_{PB}) \mathbb{E}[\mathbbm{1}_{\{\tau^{PB} < \min(T, \tau^{RN}) \}} (V_{\tau^{PB}}^{CDS})^{-} \, | \mathcal{F}_t],
\end{equation}
where $R_{PB}$ and $\tau^{PB}$ are the recovery rate and default time of the protection buyer.

Here, we associate $x_1$ with the Protection Buyer and $x_2$ with the Reference Name, then, similar to CVA,  DVA can be given by the solution of  (\ref{kolm_1})--(\ref{kolm_2}),
\begin{equation}
\begin{aligned}
		& \frac{\partial}{\partial t} V^{DVA}+ \mathcal{L} V^{DVA} = 0, \\
		& V^{DVA}(t, 0, x_2) = (1 - R_{PB}) V^{CDS}(t, x_2)^{-}, \quad V^{DVA}(t, x_1, 0) = 0, \\
		& V^{DVA}(T, x_1, x_2) = 0.
\end{aligned}
\end{equation}

\paragraph{Bilateral CVA and DVA}

When we defined unilateral CVA and DVA, we assumed that either protection  buyer, or protection seller are risk-free. Here we assume that they are both risky. Then, 
The Credit Value Adjustment represents the additional price associated with the possibility of counterparty's default and defined as
\begin{equation}
	V^{CVA} = (1 - R_{PS}) \mathbb{E}[\mathbbm{1}_{\{\tau^{PS} < \min(\tau^{PB}, \tau^{RN}, T)\}} (V^{CDS}_{\tau^{PS}})^{+} \, | \mathcal{F}_t],
\end{equation} 

Similar, for DVA
\begin{equation}
	V^{DVA} = (1 - R_{PB}) \mathbb{E}[\mathbbm{1}_{\{\tau^{PB} < \min(\tau^{PS}, \tau^{RN}, T)\}} (V^{CDS}_{\tau^{PB}})^{-} \, | \mathcal{F}_t],
\end{equation} 


We associate $x_1$ with protection seller, $x_2$ with protection buyer, and $x_3$ with reference name. Here, we have a three-dimensional process. Applying three-dimensional version of (\ref{kolm_1})--(\ref{kolm_2}) with $\psi(x) = 0, \chi(t, x) = 0$, we get
\begin{equation}
	\label{CVA_pde}
\begin{aligned}
		& \frac{\partial}{\partial t} V^{CVA} + \mathcal{L}_3 V^{CVA} = 0, \\
		& V^{CVA}(t, 0, x_2, x_3) = (1 - R_{PS}) V^{CDS}(t, x_3)^{+}, \\
		& V^{CVA}(t, x_1, 0, x_3 ) = 0, \quad V^{CVA}(t, x_1, x_2, 0)  = 0, \\
		& V^{CVA}(T, x_1, x_2, x_3) = 0,
\end{aligned}
\end{equation}
and
\begin{equation}
\label{DVA_pde}
\begin{aligned}
		& \frac{\partial}{\partial t} V^{DVA} + \mathcal{L}_3 V^{DVA} = 0, \\
		& V^{DVA}(t, 0, x_2, x_3) = (1 - R_{PB}) V^{CDS}(t, x_3)^{-}, \\
		& V^{DVA}(t, x_1, 0, x_3 ) = 0, \quad V^{DVA}(t, x_1, x_2, 0)  = 0, \\
		& V^{DVA}(T, x_1, x_2, x_3) = 0,
\end{aligned}
\end{equation}
where $\mathcal{L}_3 f$ is the three-dimensional infinitesimal generator.




\end{document}
