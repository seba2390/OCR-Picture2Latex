\section{The Method}\label{Problem_definition}
Consider $n$ applications and $i$ instances of $m$ different resources. Applications arrive in the system one at a time. The applications have to choose among $m$ resources. There exists a bipartite graph between the matching of the applications and the resources.\\
\indent In general, there can be more than one application to get a shared resource. However, each application cannot get more than one of the available heterogeneous resources. For example, if we have two cache spaces of size 128kB (one way) and 256kB (two ways), each application can either get the 128kB, or the 256kB cache space and can't get both of them at the same time. Furthermore, each resource $R_k$ has a cost $p_k$ which is defined by the applications' bid in the auction. \\
\indent Figure~\ref{fig:auction} shows our auction-based framework to support \textit{CARMA} between $N$ applications that execute together competing for $M$ different resources. Each application has a utility table that shows how much performance it gets from each $M$ resources at each time slot. Based on the utility tables, applications submit bids for the most profitable resource. Based on the submitted bids, the auctioneer decides about the resource assignment for each resource, and updates the prices. Next, the applications who did not get any assignment compete for the next most profitable resource based on the updated prices repeatedly until all applications are assigned.  Figure~\ref{fig:auction} shows an example of a resource assignment and the corresponding bipartite graph.
%Table~\ref{table:notation} shows the notation used in our formulation.
%%%%%%%%%%%%%%%%%%%%%%%%%%%%%%%%%%%%%%%%%%%%%%%%%%%%%%%%%%%%%%%%%%%%%%%%%%%%%%%
%%%%%%%%%%%%%%%%%%%%%%%%%%%%%%%%%%%%%%%%%%%%%%%%%%%%%%%%%%%%%%%%%%%%%%%%%%
\begin{table}[!tb] \scriptsize
\centering
\caption{Notation used in our formulations.}\label{Table:notation}
\begin{tabular}{|p{0.7in}|p{2.3in}|} 
\hline $N$ & Number of players or applications (from $App_1$ to $App_N$). \\
\hline $M$ & Number of resources (from $R_1$ to $R_M$). \\
\hline $\vec{m}$ & A positive $M\times 1$ vector in the resource space that shows how much each application gets from each resource. \\   % of size $M\time 1$ showing the amount for each resources. \\ Number of applications which can get a resource \\ 
% \hline $n$ & Number of applications competing for a specified resource \\
\hline $T$ & Time intervals where the bidding is held \\
\hline $t_{i,j}$ & $j$-th phase time for $i$-th application during its course of execution time. \\ 
\hline $T_i$ & Last phase time for application $i$. \\
\hline $v_{i}(t,\vec{m})$ & The valuation function of application $i$ for the resource assignment $\vec{m}$ at time $t$. \\
\hline $v_{i,j}(t,\vec{m},r)$ & The valuation function of (application $i$,resource $j$), if we replace the $j$-th resource in the resource vector $m$ by $r$ \\
\hline $\delta$ & dynamic factor that shows how much we can rely on the past iterations. \\
\hline $G=(U,V,E)$ & A bipartite graph showing the resource allocation between the applications and the set of resources. \\
\hline $U$ & The set of applications which shows the left set of nodes in the bipartite graph $G=(U,V,E)$. \\
\hline $V$ & The set of resources which shows the right set of nodes in the bipartite graph $G=(U,V,E)$. \\
\hline $E$ & The edges in the bipartite graph. \\
\hline $b_{i,k}$ & User i's bid for k-th resource. \\
\hline $F_i$ & The total budget (summation of bids) a user have. \\
\hline $C_k$ & The total capacity of each resource. \\
\hline $p_{k}$ & The price of resource $k \in V$ in the auction. \\
%\hline $Bottleneck_{1,i}$ & The first bottleneck resource for application $i$ \\
%\hline $Bottleneck_{2,i}$ & The second bottleneck resource for application $i$ \\
\hline $K$ & Number of cache levels \\
\hline
\end{tabular}
\vspace{-1.0\baselineskip}
\end{table}
%%%%%%%%%%%%%%%%%%%%%%%%%%%%%%%%%%%%%%%%%%%%%%%%%%%%%%%%%%%%%%%%%%%%%%%%%%%%%%%
\begin{figure*}[!htb]
\centering
%\includegraphics[height=3in, width=1.5in]{NodeArchs2.pdf}
\includegraphics[height=3.2in, width=6.5in]{Images/Auction_v4.pdf} %[height=4in, width=8in]
%\epsfig{file=Dataset.eps, height=2.5in, width=3in}
\vspace{-1\baselineskip}
\caption{\label{fig:auction} Framework for auction-based resource assignment (CARMA).}
\vspace{-0.5\baselineskip}
\end{figure*}
%%%%%%%%%%%%%%%%%%%%%%%%%%%%%%%%%%%%%%%%%%%%%%%%%%%%%%%%%%%%%%%%%%%%%%%%%%%
\begin{comment}
\begin{figure}[!htb]
\centering
%\includegraphics[height=3in, width=1.5in]{NodeArchs2.pdf}
\includegraphics[height=2.2in, width=1.3in]{Images/bipartite.pdf}
%\epsfig{file=Dataset.eps, height=2.5in, width=3in}
\caption{\label{fig:bipartite} Cache allocation as a bipartite graph.}
\end{figure}
\end{comment}
%%%%%%%%%%%%%%%%%%%%%%%%%%%%%%%%%%%%%%%%%%%%%%%%%%%%%%%%%%%%%%%%%%%%%%%%%%%
\vspace{-1\baselineskip}
\subsection{Problem Defenition} 
\indent We formulate our problem as an auction to enforce cost/value updates for each resource as follows: 
%The cost of each player to get a resource is the cost of the assigned resource divided by the number of players who share. 
\begin{itemize}
  \item \textbf{Valuation $\mathbf{v_{i}(t,\vec{m})}$}: Application $i$ has a valuation function which shows how much it benefits from the resource vector $\vec{m}$ at time $t$. The valuation function at time $t=0$ for cache contention case study is derived from the IPC (instruction per cycle) curves using profiling, and for processor and co-processor contention case study is derived from the profiling of separate cache performance of the application. However, in general, each application can choose its own utility function.
  %%%%%%%%%%%%%%%%%%%%%%%%%%%%%%%%%%%%%%%%%%%%%%%%%%%%%% 
    \item \textbf{Observed information}: The observed information at each time step is the performance value of the selected action in the game. Therefore, the applications repeatedly update the history of their valuation function over time.  
  %%%%%%%%%%%%%%%%%%%%%%%%%%%%%%%%%%%%%%%%%%%%%%%%%%%%%%   
    \item \textbf{Belief updating}: Let $T$ be the time intervals where the bidding is held. At each iteration step of the auction, the applications update their valuation of each resource based on the observed performance on the resource vector. The update at time $W$ is derived using the following formula:
%\begin{small}
\begin{equation}\label{eq:belief}
v_{i}(W,\vec{m})=\frac{\sum\limits_{0\leq n\leq W/T} {\delta}^{W/T-n} \cdot v_{i}(nT,\vec{m})}{\sum\limits_{0\leq n\leq W/T} {\delta}^{W/T-n}},
\end{equation}
%\end{small}  
%%%%%%%%%%%%%%%%%%%%%%%%%%%%%%%%%%%%%%%%%%%%%%%%%%%%%%%%%%%%%%%%%%%%%%%%%
where $v_{i}(W,\vec{m})$ shows the observed valuation of resource vector $\vec{m}$ at time $W$ by application $i$ in the system; $\delta$ shows the discount factor between 0 and 1 which shows how much a user relies on its past observations in the system. The discount factor is chosen to show the dynamics of the system. If the observed information in the system changes fast, the discount factor is close to zero, i.e. the application cannot rely on the past observations very much. However, if the system is more stable and the observed information does not change fast, the discount factor is closer to 1. If a user fails in an auction, its payoff and corresponding observed valuation at the current time is equal to zero. So, it won't probably bid for the same resource vector again, since its valuation decreases for next round. We choose the discount factor to be the absolute value of the correlation coefficient of the observed values of the valuations at each iteration step which is calculated as follows:
%\begin{small}
\begin{equation}
\delta =  \frac{E(v_{i}(W,\vec{m}))^2}{{\sigma_{v_{i}(W,\vec{m})}}^2}
\end{equation}  
%\end{small}
%%%%%%%%%%%%%%%%%%%%%%%%%%%%%%%%%%%%%%%%%%%%%%%%%%%%%%%%%%%%%%%%%%%%%%%%
%%%%%%%%%%%%%%%%%%%%%%%%%%%%%%%%%%%%%%%%%%%%%%%%%%%%%%%%%%%%%%%%%%%%%%%%
  \item \textbf{Action}: At each time step, the applications decide which resource to bid and how much to bid for each resource. 
\end{itemize} 
%%%%%%%%%%%%%%%%%%%%%%%%%%%%%%%%%%%%%%%%%%%%%%%%%%%%%%%%%%%%%%%%%%%%%%%%%% 
\indent Table~\ref{Table:notation} shows important notation used throughout the paper. In the following sections, we describe our distributed optimization scheme to solve the problem. 
%%%%%%%%%%%%%%%%%%%%%%%%%%%%%%%%%%%%%%%%%%%%%%%%%%%%%%%%%%%%%%%%%%%%%%%
%\begin{equation}
%min \sum\limits_{i=1}^n v_i C_k \frac{b_{i,k}}{\theta_k}, \\
%s.t. \sum\limits_{i=1}^n b_{i,k} \leq E_i
%\end{equation}
%%%%%%%%%%%%%%%%%%%%%%%%%%%%%%%%%%%%%%%%%%%%%%%%%%%%%%%%%%%%%%%%%%%%%%%
\vspace{-1\baselineskip}
\subsection{Distributed Optimization Scheme}
The goal is to design a repeated auction mechanism which runs by the operating system to guide the applications to choose their best resource allocation strategy. The applications' goal is to maximize their own performance and the operating system wants to maximize the total utility gain from the applications. Each application can use its own utility function and evaluates the resources based on the desired value of the resources. \\
\indent \textbf{Applications' approach}: The application $i$ wants to maximize the expected utility (pay-off) with respect to a limited budget ($F_i$) during all phases of its execution time. We have:
%\vspace{-0.5\baselineskip}
%%%%%%%%%%%%%%%%%%%%%%%%%%%%%%%%%%%%%%%%%%%%%%%%%%%%%%%%%%%%%%%%%%%%%%%%%
%maximize \;\;\;\; \sum\limits_{i=1}^n v_i C_k \frac{b_{i,k}}{\theta_k},\\
%\begin{small}
\begin{align}
%\begin{IEEEeqnarray}{rCl}
\forall i \in U \; \; \; \; \; maximize \; \; \; \; \sum\limits_{0<t<T_i} v_{i}(t,\vec{m})-b_{i}(t,\vec{m}) , \nonumber \\
 % \IEEEyessubnumber\\
subject \; to \;\;\;\; \sum\limits_{0<t<T_i} b_{i}(t,\vec{m}) \leq F_i, \; \; \; \; \forall i \in U.
%\IEEEyessubnumber
%\end{IEEEeqnarray}
\end{align}
%\end{small}
%%%%%%%%%%%%%%%%%%%%%%%%%%%%%%%%%%%%%%%%%%%%%%%%%%%%%%%%%%%%%%%%%%%%%%%%%%
\indent \textbf{OS's approach}: The operating system wants to maximize the social welfare function which is translated into submitted bids from the applications in a limited resource constraints.
\vspace{-0.5\baselineskip}
%%%%%%%%%%%%%%%%%%%%%%%%%%%%%%%%%%%%%%%%%%%%%%%%%%%%%%%%%%%%%%%%%%%%%%%%%%
%\begin{small}
\begin{align}
%\begin{IEEEeqnarray}{rCl}
maximize \; \; \; \sum\limits_{i=1}^N \sum\limits_{0<t<T_i}  b_{i}(t,\vec{m}) \cdot A_{i}(t, \vec{m}) , \nonumber \\ 
%\IEEEyessubnumber\\
subject \; to \;\;\;\; \sum\limits_{i=1}^N  A_{i}(t, \vec{m}) \leq A_{max}, \; \; \; \; \forall t: 0 \leq t \leq T , \nonumber \\
%\IEEEyessubnumber\\
A_{i}(t, \vec{m}) \in \{0,1\} , \; \; \forall i \in U, \; \;  \forall \vec{m} \subseteq \mathcal{V}, \; \;  \forall t:  0 \leq t \leq T ,
\end{align}
where the binary variable $A_{i}(t, \vec{m})$ represents the decision to assign resource vector $\vec{m}$ to application $i$ at time $t$ (when $A_{i}(t, \vec{m}) =1$) or not (when $A_{i}(t, \vec{m})=0$); and $\mathcal{V}$ is the vector space of the all resource vectors ($\forall\vec{m}$); and $A_{max}$ shows the maximum number of the applications which can share the resource vector $\vec{m}$. 

%%%%%%%%%%%%%%%%%%%%%%%%%%%%%%%%%%%%%%%%%%%%%%%%%%%%%%%%%%%%%%%%%%%%%%%%%%%
%\begin{comment}
\indent \textbf{Illustrative example}: As an illustrative example shown in Figure~\ref{fig:Dynamic}, let us consider a case where we have two different resources, a large cache of 1MB which can be shared between applications, and two private caches of 512KB which are not shared. 
%There are two applications competing for the cache space. One of the applications wants to minimize its request latency and the other one wants to maximize number of instructions executed per cycle. Suppose that both applications have two phases $(0,T)$ and $(T,2T)$. If the first application gets the larger cache space its \textit{IPC} increases by 15 percent in first phase and by 36.84 percent in the second phase. The second application's \textit{IPC} increases by 35 percent in the first phase and by 20.6 percent in the second phase if it gets the larger cache space. %Also, suppose that they both have 60 tokens as a budget to submit. 
The first application participates in the auctions with 35¢ bid at the first phase and 21¢ bid at the next phase. The second application participates in the auctions with 15¢ bid at the first phase and 37¢ bid at the next phase. The auctioneer (OS) decides to allocate larger cache in the auctions at first phase to the first application and at next phase to the second application. %Since, the social welfare would be maximized if the auctioneer allocates both applications with the larger cache space, they would both get the larger resource. Then the first application notices that its utility function does not improve as he predicts and adjusts the utility table and can either change its allocation or stay on current allocation. 
%The application should redistribute the tokens for the next phase if it did not get the desired resource in the first phase. The second application invests 35 tokens in the first phase and 25 tokens in the next phase. The auctioneer (OS) at each phase decides to allocate which resource to which applications. Since, the social welfare would be maximized if the auctioneer allocates both applications with the larger cache space, they would both get the larger resource. Then the first application notices that its utility function does not improve as he predicts and adjusts the utility table and can either change its allocation or stay on current allocation. 
%If both applications bid 10\$ for the private cache and 15\$ for the shared cache, the operating system would allocate both the shared cache space and get 15\$ from each to maximize its revenue.
%\end{comment}
\vspace{-1\baselineskip}
\subsection{Analysis}
The distributed optimization problem is hard to solve. However, in reality, the problem can split into simpler subproblems since each application knows its bottleneck resource and would first bid for the first bottleneck resource to maximize its utility.\\
\indent We suppose all applications in the system are risk-neutral which means they have a linear valuation of the utility function. Each risk-neutral agent wants to maximize its expected revenue. Risk attitude behaviors are defined in \cite{ferber1999multi} where the agents can broadly be divided into risk-averse, risk-seeking and risk neutral. Risk-averse agents prefer deterministic values rather than risky value profits and risk-seeking applications have a super-linear utility function and prefer risky utilities than sure utilities. Next, we derive the Bayes Nash equilibrium strategy profile for all agents in the system assuming risk neutrality. 
%%%%%%%%%%%%%%%%%%%%%%%%%%%%%%%%%%%%%%%%%%%%%%%%%%%%%%%%%%%%%%%%%%
\newtheorem{defi}{Definition}
\begin{defi}
A strategy profile $a$ is a pure Nash equilibrium if for every application $i$ and every strategy $a_i' \neq a_i \in A$ we have $u_i(a_i, a_{-i}) \geq u_i(a_i', a_{-i})$
\end{defi}
\newtheorem{theorem}{Theorem}
\begin{theorem}\label{thm:neat}
%\emph{(Theorem)}
\label{Auction}
Suppose $n$ risk-neutral applications whose valuations are derived uniformly and independently from the interval $[0,1]$ compete for one resource which can be assigned to $m$ applications who have the highest bid in the auction. We will show that Bayes Nash equilibrium bidding strategy for each application in the system is to bid $\frac{n-m}{n-m+1}v_i$ where $v_i$ is the profit of application $i$ for getting the specified resource.  
\end{theorem}
%%%%%%%%%%%%%%%%%%%%%%%%%%%%%%%%%%%%%%%%%%%%%%%%%%%%%%%%%%%%%%%
%\vspace{-1\baselineskip}
\begin{algorithm}[!tb] %\small
\DontPrintSemicolon % Some LaTeX compilers require you to use \dontprintsemicolon    instead
\KwIn{A bipartite Graph (U, V, E).}
\KwOut{The allocation of resources to applications.}
The initial resource vector for each application is the average amount across various resources. At time $t=nT$, the valuation of each application for each resource vector is updated using Eq.~\ref{eq:belief}.

For application $U_i \in U$, the first bottleneck resource is
%\[ Bottleneck_{1,i} = V_{i,m}=  arg \; \max_{m \in V} (v_{i,m}-p_{m})  \] 
%\[ Bottleneck_1: V_{i,j^{max}_1}=  \max_{1\leq j_1 \leq M} {\Delta v_{i,j_1}(t,\vec{m},r_{j_1})-p_{j_1}}  \] 
\[ V_{i,j^{max}_1}=  \max_{1\leq j_1 \leq M} {\Delta v_{i,j_1}(t,\vec{m},r_{j_1})-p_{j_1}}  \] 
where the differential valuation function is $\Delta v_{i,j}(t,\vec{m},r_j) = v_{i,j}(t,\vec{m},r_j) - v_{i}(t,\vec{m})$. 

Find the second bottleneck resource for application $U_i \in U$ in the system:
%\[ Bottleneck_2: V_{i,j^{max}_2}=  \max_{1\leq j_2 \leq M; j_2\neq j_1} {\Delta v_{i,j_2}(t,\vec{m},r_{j_2})-p_{j_2}}  \] 
\[ V_{i,j^{max}_2}=  \max_{1\leq j_2 \leq M; j_2\neq j_1} {\Delta v_{i,j_2}(t,\vec{m},r_{j_2})-p_{j_2}}  \] 

Each application calculates the partial bid for its first bottleneck resource using the following formula:
\[ b_{i,j^{max}_1}(t) = V_{i,j^{max}_1} - V_{i,j^{max}_2} + p_{j^{max}_1} + \epsilon \]

Each resource $r_j \in V$ which can be shared between $L$ applications, is assigned to the $L$ highest bidding applications $Winner_{i,j}=\{i_1, i_2, ..., i_L\}$ and the price for that resource is updated as follows:
%\[ p_{j} = \max_{i_1, i_2, ..., i_L \in U} \sum\limits_{k=1}^L {b_{i_k,j}}  \]
\[ p_{j} = \max_{l \in \{1,...,L\}} {b_{i_l,j}}  \]

The $minBid$ for each resource is updated as the minimum bid of $l$ applications who acquired $j$-th resource. That is:
\[ B^{min}_j=  \min_{i_l \in Winner_{i,j}} {b_{i_l,j}} \] 

Goto step 2 until all partial bids for all resources are determined. 

The total bid of the application $i$ is as follows:
\[ b_{i}(t,\vec{m})=b_{i}(t,[r_{j_1};r_{j_2};...;r_{j_M}])=\sum\limits_{j=1}^M {b_{i,j}(t)} \]
where iteratively $\vec{m}=[r_{j_1};r_{j_2};...;r_{j_M}]$.

Find the estimated investment $I_{i}(t)$ using Algorithm~\ref{algo:p} to plan the upper bound of the investment with respect to the budget $F_i$. If $I_{i}(t)\geq b_{i}(t,\vec{m})$, application $i$ will participate in this auction at time $t$, otherwise it quits and other applications do the steps 2 and 3.

\caption{Parallel Auction for Heterogeneous Resource Assignment}
\label{algo:b}
\end{algorithm}
%%%%%%%%%%%%%%%%%%%%%%%%%%%%%%%%%%%%%%%%%%%%%%%%%%%%%%%%%%%%%%%%%%
%DIMAN COMMENTED PROOF FOR APPENDIX
\begin{comment}
\begin{proof}
Suppose all other applications' bidding strategy is to choose $\frac{n-m}{n-m+1}v_i$. Since the bidding values were derived uniformly in $[0, 1]$ all bids have the same probability. Therefore, if we consider the first application's expected utility to find its best response, we have:

\begin{equation}
%\begin{IEEEeqnarray}{rCl}
E[u_1] = \int_0^1  .... \int_0^1 \! (v_1 -b_1) \, \mathrm{d}u_2 \mathrm{d}u_3 ... \mathrm{d}u_{n-m} .  
%\end{IEEEeqnarray}
\end{equation}

The following integral breaks into two part where the first application wins the auction or not. 


%\begin{IEEEeqnarray}{rCl}
%E[u_1] = \int_0^{b_1}  .... \int_0^{b_1} \! (v_1 -b_1) \, \mathrm{d}u_2 \mathrm{d}u_3 ... %\mathrm{d}u_{n-m}  \\
%+ \int_{b_1}^1  .... \int_{b_1}^1 \! (v_1 -b_1) \, \mathrm{d}u_2 \mathrm{d}u_3 ... \mathrm{d}%u_{n-m}\nonumber 
%\end{IEEEeqnarray}

\begin{equation}
%\begin{IEEEeqnarray}{rCl}
E[u_1] = \int_0^{\frac{n-m+1}{n-m}b_1}  .... \int_0^{\frac{n-m+1}{n-m}b_1} \! (v_1 -b_1) \, \mathrm{d}u_2 ... \mathrm{d}u_{n-m}  \\
+ \int_{\frac{n-m+1}{n-m}v_1}^1  .... \int_{\frac{n-m+1}{n-m}v_1}^1 \! (v_1 -b_1) \, \mathrm{d}u_2 \mathrm{d}u_3 ... \mathrm{d}u_{n-m}\nonumber 
%\end{IEEEeqnarray}
\end{equation}

The second part of the integrals is the term where the first application doesn't win the auction. Therfore, the expected payoff of application 1 is equal with:

%\begin{IEEEeqnarray}{rCl}
%E[u_1] = \int_0^{b_1}  .... \int_0^{b_1} \! (v_1 -b_1) \, \mathrm{d}u_2 \mathrm{d}u_3 ... %\mathrm{d}u_{n-m}  \\
%= {({b_1}) }^{n-m} (v_1 -b_1). \nonumber 
%\end{IEEEeqnarray}

\begin{equation}
%\begin{IEEEeqnarray}{rCl}
E[u_1] =\int_0^{\frac{n-m+1}{n-m}b_1}  .... \int_0^{\frac{n-m+1}{n-m}b_1} \! (v_1 -b_1)  \mathrm{d}u_2 ... \mathrm{d}u_{n-m}  \\
= {(\frac{n-m+1}{n-m} b_1) }^{n-m} (v_1 -b_1). \nonumber 
%\end{IEEEeqnarray}
\end{equation}


Diffrentiating with respect to $b_1$ the optimal bid for application one is derived as follows:

%\begin{equation}
%\frac{\partial}{\partial b_1} ( {({b_1}) }^{n-m} (v_1 -b_1))=0.
%\end{equation}


\begin{equation}
\frac{\partial}{\partial b_1} ( {(\frac{n-m+1}{n-m} b_1) }^{n-m} (v_1 -b_1))=0.
\end{equation}

Which gives us the optimal bid for each application:
\begin{equation}
\Rightarrow b_1= \frac{n-m}{n-m+1}v_1
\end{equation}
\end{proof}
\end{comment}
%DIMAN COMMENTED PROOF FOR APPENDIX

%%%%%%%%%%%%%%%%%%%%%%%%%%%%%%%%%%%%%%%%%%%%%%%%%%%%%%%%%%%%%%%%%%%%%%%%%%%%
\begin{algorithm}[!tb] %\small
\DontPrintSemicolon % Some LaTeX compilers require you to use \dontprintsemicolon    instead
\KwIn{A bipartite Graph (U, V, E).}
\KwOut{Participation (YES) in an auction or Quit (NO).}
At time $t=nT$, assume that we have the same state in terms of resources.

For application $U_i \in U$, we similarly run the steps 2, 3, 4, 5 and 6 of Algorithm~\ref{algo:b} to find all estimated bids in next rounds based on its various phases. We have:
\[ b_{i}(t_{i,j},\vec{m})=\sum\limits_{j=1}^M {b_{i,j}(t)};\forall t_{i,j}>t \]
Also, we have the previous bids of the application $i$: 
\[ b_{i}(t_{i,j},\vec{m});\forall t_{i,j}<t \]

If $F_i \geq \sum\limits_{\forall t_{i,j}\neq t} {b_{i}(t_{i,j},\vec{m})}$, then YES and the application will participate in the auction. Otherwise NO, and the application will update the zero valuation for current round using Eq.~\ref{eq:belief}.

\caption{Budget Planning}
\label{algo:p}
\vspace{0\baselineskip}
\end{algorithm}
%%%%%%%%%%%%%%%%%%%%%%%%%%%%%%%%%%%%%%%%%%%%%%%%%%%%%%%%%%%%%%%%%%%%%%%%%%%%%
\indent Theorem~\ref{thm:neat}, states that whenever there is a single resource that users compete to get it with different valuation functions, the Nash equilibrium strategy profile for risk-neutral users is to bid $\frac{n-m}{n-m+1}v_i$. This term tends to the true value of the object when n is a large number. \\
\indent In case of more than one resource competition we derive Algorithm~\ref{algo:b} for heterogeneous resource assignment and will prove that it has a Nash equilibrium in the game. Algorithm~\ref{algo:b} uses Algorithm~\ref{algo:p} to do budget planning for our purpose. In the first step, all valuations are set to the solo-run of application's performance. Next, each application submits a partial bid for its first bottleneck resource. The partial bid should be larger than the price of the object which is initialized to zero at the beginning of the program. The applications only have the incentive to bid a value that is no more than the difference between the first and second bottleneck resource. Otherwise, it submits a smaller bid to the second bottleneck and gets the same revenue as paying more for the first bottleneck resource. In order to break the equal valuation function between two different applications, we use $\epsilon$ scaling such that at each iteration of the auction the prices should increase by a small number. The OS will set the resources' price with these partial bids, and find the minimum of the highest partial bids for each resource. The applications recurse for all the resources, and the total bid is the summation of the partial bids for each application. Then, the applications execute Algorithm~\ref{algo:p} to participate in the auction or not. Finally the participated applications with the bids higher than $B^{min}_j$ will get $j$-th resource.\\
%In addition, suppose we have 5 different memory bandwidth exposed to the applications. Each application gets a different performance benefit from different cache sizes and different memory bandwidth which is denoted in table  **. The applications need to submit their bids based on their performance benefits. 
%\begin{equation}
%\begin{split}
%\int_0^\frac{nb_1}{n-m}  .... \int_0^\frac{nb_1}{n-m} \! (\frac{v_1}{m} -b_1) \, \mathrm{d}u_2 %\mathrm{d}u_3 ... \mathrm{d}u_{n-m}= \\
%= {(\frac{nb_1}{n-m}) }^{n-m} (\frac{v_1}{m} -b_1). 
%\end{split}
%\end{equation}
%%%%%%%%%%%%%%%%%%%%%%%%%%%%%%%%%%%%%%%%%%%%%%%%%%%%%%%%%%%%%%%%%%%%%%%%%%%%
%\[ \frac{\partial}{\partial b_1} ({(\frac{nb_1}{n-m}) }^{n-m} (\frac{v_1}{m} -b_1))
%\newtheorem{defi}{Definition}
%\begin{defi}
%Let's assume each user $Ui$ in the system is defined as one vertice of a graph $G$ in the system and let each edge in the graph shows which subset of users can impact each others' performance and the associated weight of each edge show the cost function of how two users affect each others' performance in the system. Each edge has a weight function denoted by ${P1(n), P2(n), ... Pe(n)}$, where $e$ is the number of edges in Graph $G$. Let $A=A_1 \times A_2 \times ... \times A_n $ be the set of actions that each user can play. 
%\end{defi}
\indent The overhead of the auction for the auctioneer (the OS) is very negligible. The OS during the auction only sets the prices of the resources based on the received bids from the applications and gives the resources to the highest bids. So every $T$ seconds, the OS runs these two jobs, which adds a negligible overhead with respect to other tasks of the OS. Our approach also satisfies the following properties:
1) Individual rationality (IR): Applications' expected utility is non-negative because the amount of the bid cannot be beyond the sum of the difference of the valuations which is at most the highest valuation of the application.
2) Truthfulness: Applications cannot benefit from bidding other than their true valuation. By contradiction, if an application bids lower than the true value, there may be another application with a higher bid to take the resource. But we cannot guarantee the truthfulness in the case of collusion among applications.
3) Budget-balance: The whole payments from the applications are less than the OS revenue, which is trivial as we have only one seller which is the OS.
4) Economic efficiency: It has been shown in \cite{bertsekas1998network} that this assignment is optimal, but it doesn't mean it is economically efficient since we know that it depends on the applications' valuation which is sub-optimal.
