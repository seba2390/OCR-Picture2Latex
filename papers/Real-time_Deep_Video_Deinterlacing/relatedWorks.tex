Before introducing our method, we first review existing works related to 
deinterlacing. They can be roughly classified into tailor-made 
deinterlacing methods,  traditional image resizing methods, and DCNN-based
image restoration approaches.

\vspace{0.15in}
\noindent\emph{Image/Video Deinterlacing}\,\,\,\,
\begin{figure*}[!tp]
  % Requires \usepackage{graphicx}
  \includegraphics[width=1\linewidth]{images/cnn_model.pdf}\\
  \caption{The architecture of the proposed convolutional neural network.}\label{fig:cnn_model}
\end{figure*}
Image/video deinterlacing is a classic  vision problem. 
Existing methods can be classified into two categories: intra-field deinterlacing
\cite{doyle1990interlaced,wang2012efficient,wang2013moving} and inter-field
deinterlacing \cite{jeon2009weighted,mohammadi2012enhanced,lee2013high}. 
Intra-field deinterlacing methods reconstruct two full frames from the odd and even fields
independently. Since
there is large information loss (half of the data is missing) during frame
reconstruction, the visual quality is usually less satisfying. To improve visual
quality, inter-field deinterlacing methods incorporate the temporal
information between multiple fields from neighboring frames during frame
reconstruction. Accurate motion compensation or motion
estimation~\cite{horn1981determining} is needed to achieve satisfactory quality. However, 
accurate motion estimation is hard in general. In
addition, motion estimation requires high computational cost, and hence inter-field
deinterlacing methods are seldom used in practice, especially for applications
requiring real-time processing.

\vspace{0.15in}
\noindent\emph{Traditional Image Resizing}\,\,\,\,
Traditional image resizing
methods can also be used for deinterlacing by scaling up the height of each field.
To scale up an image, cubic~\cite{mitchell1988reconstruction} and
Lanczos interpolation~\cite{duchon1979lanczos} are frequently used. While they work well for low-frequency components, high-frequency components 
(e.g. edges) may be over-blurred. More advanced image resizing methods, 
such as kernel regression~\cite{takeda2007kernel} and
bilateral filter~\cite{hung2012fast} can improve the
visual quality by preserving more high-frequency components. 
However, these methods may still introduce noise or artifacts 
if the vertical sampling rate is less than the Nyquist rate. 
More critically, they only utlize a single field and ignore the temporal information, 
and hence suffer the same problem as intra-deinterlacing methods.

%Thus the deinterlacing problem can be regarded as a single image scaling problem.
%This problem has been studied for may years in the signal processing community.
%We can use general image resizing methods, like bicubic~\cite{mitchell1988reconstruction}, Lanczos~\cite{duchon1979lanczos}), splines~\cite{hou1978cubic} or kernel regression~\cite{takeda2007kernel} to upsample interlaced images vertically to reconstruct the full-sized images.
%for the purpose of image deinterlacing and as a result they cannot generate better results than most of deinterlacing methods.
%%These studies focus on how to get such a good relation between low/high-resolution patches.

\vspace{0.15in}

\noindent\emph{DCNNs for Image Restoration}\,\,\,\, In recent years, 
deep convolutional neural networks (DCNNs) based methods have been proposed to solve 
many image restoration problems.
Xie et al.~\shortcite{xie2012image} proposed a DCNN model for
image denosing and inpainting. This model recovers the values of corrupted
pixels (or missing pixels) by learning the mapping between corrupted and
uncorrupted patches. Dong et al.~\shortcite{dong2016image} proposed to adopt
DCNN for image super-resolution, which greatly outperforms the state-of-the-art
image super-resolution methods. Gharbi et al.~\shortcite{gharbi2016deep} further
proposed a DCNN model for joint demosaiking and denosing. It infers
the values of three color channels of each pixel from a single noisy
measurement.

%Super resolution is to artificially enlarge a low resolution photograph to recover a corresponding plausible image with higher resolution.
%Thus image super resolution methods can also be used for image deinterlacing.
%sA good survey can be found in~\cite{nasrollahi2014super}.
%The state-of-the-art SR works focus on learning a good mapping between low/high-resolution patches.
%These works differs on how to learn a high dimensional compact space to represent the relationship between low/high-resolution patches.
%For example, in~\cite{freeman2000learning}, the mapping between low/high-resolution patches are formulated as an optimiztion problem using Markov random fields which can be solved using Bayesian belief propagation.
%In~\cite{chang2004super,yang2008image}, the relationship between low/high-resolution patches is modeled using compressed sensing, which forces the representation become sparse in the high dimensional space.
%Recently, deep convolutional neural networks(DCNNs) have be successfully used for the image super resolution problem, like SRCNN~\cite{dong2016image}.
%DCNNs can give state-of-the-art performance to approximate the underling mapping functions.

It seems that we can simply re-train these state-of-the-art 
neural network based methods for our deinterlacing purpose. 
However, our experiments show that visual artifacts are still unavoidable, as 
these DCNNs generally follow the conventional translation-invariant assumption and
modify the values of all pixels, even in the known odd/even scanlines.  Using a larger
training dataset or deeper network structure may alleviate this problem, but the
computational cost is drastically increased and still there is no guarantee 
that the values of the known pixels remain intact. 
Even if we fix the values of the known pixels (Fig.~\ref{fig:srcnn_problem}(c)), 
the quality does not improve. 
In contrast, we propose a novel DCNN tailored for deinterlacing.
Our model only estimates the missing pixels instead of the whole frame, and also
take the temporal information into account to improve visual quality.

