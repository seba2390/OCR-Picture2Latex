Unmanned Aerial Vehicles (UAV) are being increasingly used in a broad spectrum of scenarios and applications~\cite{rodrigues2017uav}.
Their integration in the Urban Internet of Things (IoT) is attracting a considerable interest, for instance to enhance the ability of the city-wide system to support delicate tasks such as surveillance and monitoring, virtual reality, disaster management, and to improve or maintain network coverage.

UAVs are typically controlled by a Ground Control Station (GCS), which wirelessly interconnects with the UAV to build a data-control loop composed of an upstream flow of control messages and a downstream flow of telemetry and sensor data. The urban environment poses several challenges undermining the ability of the GCS to control the UAVs.
First, the topological characteristics of the urban environment may severely limit the operating range due to Line of Sight (LoS) obstruction. This issue has been partially addressed in prior work by creating mesh networks of cooperating UAVs.
However, another important issue that remains largely unaddressed is the coexistence of UAV-related traffic with competing IoT data streams. Exogenous traffic sharing the same access and/or backbone network may create localized and temporary congestion impairing the ability of the GCS to establish an effective data-control loop with the UAV.

This paper addresses these important problems by proposing a robust multi-hop multi-path framework for the remote control of UAV systems. The data and control links are established using the network infrastructure available in urban environments. In particular, we use for communications the 2.4GHz ISM band, which is shared with other Wi-Fi devices and used by other wireless technologies. The multiple paths from the GCS and UAV are continuously probed to quickly select the best option. Importantly, simple local measurements, such as channel sensing and signal strength, would not protect the GCS-UAV communications against local network congestion.

The framework employs a multi-hop multi-path beacon forwarding technique to continuously monitor the performance of the paths from the GCS to the UAV. The UAV measures beacon delay and loss to migrate control routing from one path to another when the current path falls outside of a predefined Quality of Service (QoS) region.

The main contributions of this paper are:
\emph{(a)} A cooperative networking model which establishes multi-hop routes using the urban IoT communication infrastructure to forward control messages from the GCS to the remote UAVs;
\emph{(b)} A framework to dynamically adapt the route used to forward control messages from the GCS to the UAVs based on the current QoS of the paths; and \emph{(c)} A real-world implementation and extensive field experimentation of the proposed framework.
Experimental results show a considerable improvement in terms of control messages reliability, which leads to a reduced delay in accomplishing mission objectives.

The rest of the paper is organized as follows. In Section II, we discuss related work and emphasize the main innovations introduced by this paper with respect to existing frameworks. Section III presents the architecture and describes the adaptive communication and control strategy used to dynamically select the best path from the GCS to the UAV. In Section IV, we describe the experimental setup and provide numerical results assessing the performance of the framework. Section V and VI conclude the paper.
