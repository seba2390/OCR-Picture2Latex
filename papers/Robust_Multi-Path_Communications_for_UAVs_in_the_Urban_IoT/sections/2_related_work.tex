Due to the exponentially increasing diffusion of UAVs, the development of effective communication frameworks supporting their operations has received considerable attention in recent years~\cite{bekmezci2013flying}. The interested reader can find in~\cite{gupta2016survey} a detailed survey on the challenges of UAV communications in terms of mobility, fast topology changes, and connectivity. An investigation of IEEE 802.11a applied to UAV-to-ground links can be found in~\cite{yanmaz2011channel}. However, an organic and comprehensive solution to these issues is still missing. 

Related to the methodology used in this work,~\cite{de2010uav} presents a study on UAV systems supporting the connectivity of wireless sensor networks. In~\cite{ha2010uav}, the authors propose an analytical framework to partition the geographical region and maintain a connected graph of UAV nodes. A framework to make UAV networks self-organizing is presented in~\cite{orfanus2016self}. The methodology is based on beacons, whose failure trigger navigation directive to maintain connectivity. Other contributions address the problem of dynamic routing over wireless networks composed of fast moving UAVs, referred to as Flying Ad-hoc Network (FLANET). The solution in~\cite{rosati2016dynamic} extends an existing routing protocol to address ad-hoc networking scenarios.

Most existing approaches assume a dedicated spectrum for UAVs communications.
This paper proposes a framework integrating UAV systems in the Urban IoT using available communication resources to route control messages. A dynamic path selection mechanism ensures robustness against congestion generated by other data streams using the same infrastructure and spectrum. Different from most contributions in this area, we provide a full implementation and experimental investigation.
