
\subsection{Preliminaries}
%The application of UAVs in urban IoT is getting increasingly demanding due to the capability of sensing and data processing for various IoT services. 
Fig.~\ref{fig:uaviot} illustrates the scenario considered in this paper: a network of UAVs immersed in an urban environment where a multitude of other sensing and communication devices operate and coexist. %Typically, UAVs are directly interconnected to the GCS which controls their operations. However, 
Due to the topology of urban environments, a direct link between a UAV and GCS would likely fail to provide a satisfactory communication range, with an inevitable drop in the reliability of control messages delivery. Importantly, the UAV incorporates fail-safe mechanisms that are activated when the UAV is disconnected from the GCS, including GPS-based return-to-home function and emergency landing. However, in both cases the UAV fails to accomplish the assigned mission. 

%Therefore, preserving the connectivity between the UAV and the GCS is of paramount importance. To accomplish such objective, 
Hence, we use the wireless Access Points (AP) available in the city to forward control messages from the GCS to the UAV and telemetry data back from the UAV. The APs are interconnected through the backbone network with established minimum cost paths calculated using either Link State or Distance Vector protocols. In this paper, we focus on communications in the $2.4$~GHz band using Wi-Fi technology. However, the same reasoning can be applied to any, or multiple, technologies depending on the communication capabilities of the UAVs.


\subsection{Architecture}

Current approaches addressing connectivity in urban environments primarily use Received Signal Strength Indicator (RSSI) to perform AP selection. However, each individual AP and the router involved in the path to the AP may be also supporting other data streams, which may create localized congestion and affect the performance of a subset of the possible paths. Intuitively, messages from the GCS have stringent delay requirements, where excessive delay may affect controllability, or trigger fail-safe mechanisms as mentioned earlier. The architecture we propose is specifically designed to be robust against congestion and traffic dynamics. To accomplish this objective, we integrate RSSI with performance metrics evaluated in real-time indicating the current state of entire forwarding paths. Informed by the computed metrics, the framework, then, implements a flexible make-before-break handover mechanisms which dynamically selects the best path.

The performance of each path from the GCS to the UAV is measured using beacon messages. Specifically, the GCS periodically generates beacons: small packets containing the generation timestamp and the destination AP information. These beacons are forwarded to all the APs that the GCS can reach through the backbone network.
The UAV monitors all the WiFi channels and capture the broadcast beacons from all the APs in its vicinity.

As illustrated in Fig.~\ref{fig:sysarch}, the framework we propose consist of different functional blocks at the GCS and UAV. The functional blocks at the GCS are: \emph{Control Generator}, \emph{Beacon Generator} and \emph{Handover Manager}. The UAV functional blocks are: \emph{Deep Packet Inspector}, \emph{Performance Analyzer}, \emph{Decision Manager} and \emph{Handover Manager}. In the following, we describe each of these blocks in detail.

\begin{figure}[!t]
\centering   \includegraphics[width=.5\textwidth]{./images/UAV_multipath_arch.eps}
\vspace{-6mm}
\caption{Proposed flexible and robust communication architecture.}
\vspace{-6mm}
\label{fig:sysarch}
\end{figure}

\noindent
$\Box$ {\bf{GCS - Control Generator}}: 
This block generates and handles the control messages to be forwarded to the UAV. In the considered case, control messages belong to two classes: heartbeat messages and navigation messages. The former are small messages that are periodically generated so that the UAV can monitor the connection with the GCS. Navigation messages determine the motion of the UAV, and in the considered case contain target GPS coordinates and speed. These messages are defined by the mission control block, which is not explicitly included in the proposed architecture. The Control Generator block adds a generation timestamp and a sequence number to all the control messages. This information is used by the UAV to monitor the quality of the path used to communicate with the GCS in terms of absolute delay and message loss rate.  


\vspace{1mm}
\noindent
$\Box$ {\bf{GCS - Beacon Generator}}: 
Note that control messages are routed only through the path currently used to interconnect the GCS to the UAV. Thus, the timestamps and sequence numbers do not provide any information on all the other possible path options. In fact, broadcasting the control messages over the entire backbone network may increase congestion, especially in scenarios with a large number of UAVs. To address this issue, the Beacon Generator periodically generates small messages -- containing a timestamp and a sequence number -- that are broadcasted to all the APs. Note that the UAV does not need to be associated with any specific AP to receive the beacons.

\vspace{1mm}
\noindent
$\Box$ {\bf UAV - Deep Packet Inspector (DPI)}:
This block, implemented at the UAV side, captures all the beacons and control packets. The beacons are collected from all the APs and channels the UAV can receive from, whereas control messages are received only from the currently used path. The block inspects each received packets and creates a data point including the message type, the reception time, the sequence number and the originating AP (MAC address). This information is forwarded to the Performance Analyzer.

\vspace{1mm}
\noindent
$\Box$ {\bf{UAV - Performance Analyzer}}: 
This block receives the data points from the DPI block and determines per-message class packet loss rate and average absolute delay. These performance metrics are measured over a moving time window of duration equal to $\Delta$~seconds. The metrics are forwarded to the Decision Manager, where the moving average measures are used to trigger handover events based on control messages and select the best path based on beacons. In addition to message-related measures, the Performance Analyzer also measures the average RSSI associated with the various interconnected APs.

Note that the duration of the moving window influences the response time and frequency of the framework. On the one hand, a long window better smooths ``noise'', removing small delay and loss peaks, and avoids frequent handover. On the other hand, a short window allows a faster reaction of the framework to congestion. A thorough study on the effect of this parameter on the performance of control delivery and UAV navigation is not included here due to space constraints, and is deferred to future studies.



\vspace{1mm}
\noindent
$\Box$ {\bf{UAV - Decision Manager}}:
The Decision Manager block uses the moving average performance metrics derived by the Performance Analyzer to perform two functionalities: \emph{(a)} Trigger handover to a different AP; and \emph{(b)} Select the best path when a handover event is triggered. In the former functionality, only metrics relative to control messages are used, as handover is necessary only when the QoS of the current path suffers a degradation sufficient to impair the ability of the GCS to control the UAV. The latter functionality considers metrics relative to beacon reception from all the APs, as path selection requires the evaluation of all the feasible paths.

At time instant $t$ the Decision Manager receives moving average beacon delays $D^{i}_{\rm b}(t)$, RSSI $R^{i}_{\rm b}(t)$ and loss rate $L^i_{\rm b}(t)$ corresponding to AP $i$, with $i=1,\ldots,N$, moving average control delay $D_{\rm c}(t)$, RSSI $R_{\rm c}(t)$ and loss rate $L_{\rm c}(t)$.
A handover request is issued at time $t$ if one of the following conditions is satisfied:
\vspace{-1.5mm}
\begin{equation}
 \lambda_1  D_{\rm c}(t) {+} \lambda_2  L_{\rm c}(t) {+} \lambda_3 (R_{\rm max}{-}R_{\rm c}(t))  {>} \Theta;~~
 L_{\rm c}(t){>} \Phi,
\vspace{-1mm}
\end{equation}
where $\lambda_1$, $\lambda_2$ and $\lambda_3$ are positive weights, with $\lambda_1{+}\lambda_2{+}\lambda_3{=}1$, and $R_{\rm max}$ is the maximum RSSI index. $\Theta$ and $\Phi$ are positive thresholds. The first condition corresponds to a general degradation of the current path. In addition to the first condition, we include in the framework an urgent handover mechanism to recover from harsh events in which the connection with the current AP is abruptly severed. Specifically, if the number of heartbeats received in the window is below a certain threshold, the handover manager is immediately notified. This event corresponds to the second condition.

If a handover request is issued, the Decision Manager computes
the metric
\begin{equation}
W_{i}(t) =  \gamma_1  D^{i}_{\rm b}(t) + \gamma_2  L^{i}_{\rm b}(t) + \gamma_3 (R_{\rm max}{-}R^{i}_{\rm b}(t)),
\end{equation}
for all the APs $i=1,2,...,N$, where $\gamma_1$, $\gamma_2$ and $\gamma_3$ are positive weights, with $\gamma_1{+}\gamma_2{+}\gamma_3{=}1$ and $\Theta$ is a positive threshold defining the minimum accepted performance.
The path is selected as
\vspace{-2mm}
\begin{equation}
	k = \arg \min_{i}\{W_{i}(t)\}.
\vspace{-2mm}
\end{equation}
Thus, the decision manager selects the $k^{th}$ AP as the new control path if a handover request is triggered. In this case, the Decision Manager forwards to the UAV Handover Manager the handover request and the index of the new selected AP. 


\begin{figure}[!t]
   \centering
   \includegraphics[width=0.5\textwidth]{./images/linear-handover}
    \vspace{-6mm}
    \caption{Topology of the experimental setup.}
    \vspace{-6mm}
    \label{fig:linear-sense}
\end{figure}

\noindent
$\Box$ {\bf{Handover Manager}}: 
The handover manager block is located both at the GCS and UAV sides, and implements a 3-way handshake mechanism. The GCS maintains a data structure thats maps the connected UAVs to their corresponding IP addresses. Each UAV keeps track of the GCS's IP address which we assume to be fixed for the duration of the mission. If a handover request is triggered by decision manager, the handover manager at the UAV associates itself with the AP provided by the decision manager. The Dynamic Host Configuration Protocol (DHCP) server at the new AP provides an IP address to the UAV. To ensure make-before-break handover, at this point the UAV doesn't disassociate itself from the old AP and keeps receiving the control messages through that. Now, the UAV initiates handover by sending a handover request message to the GCS via both the APs to maximize the reception probability at GCS. The handover request message contains the UAV's new IP address and the information of new AP. Upon receiving this handover request message, the GCS sends an approve message and note the information received by the request message. Upon reception of approve message, the UAV completes the 3-way handshake by sending ACK message. After a successful handover, the GCS station forwards the control messages over the new path and the UAV disassociates itself from the old AP to save energy.

