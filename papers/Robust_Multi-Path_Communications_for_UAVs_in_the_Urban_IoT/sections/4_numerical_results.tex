We assess the performance of the proposed architecture and framework by means of real-world experiments.

\subsection{Experimental Setup}
In the considered setup, the backbone network is composed of three paths through three APs connected to a GCS. The topology is illustrated in Fig.~\ref{fig:linear-sense}: the GCS is connected with AP1, and AP2 and AP3 are at two hop distance from the GCS. All the three APs operate on non-overlapping channels. 

We use Raspberry Pi (RPi) to create the APs using hostapd and all the APs operate according to the IEEE 802.11b standard. The APs communicate with each other via static routing. The GCS, which runs on a laptop, generates a beacon every $200$~ms and a heartbeat message every $500$~ms. The frequency of beacons and heartbeat messages can be increased or decreased based on the observed coherence time of the system.
UDP is used as transport layer for both beacons and control messages. To synchronize the clocks among the UAV and GCS, we use the Network Time Protocol (NTP) with the GCS set as the NTP server.

The UAV is a 3DR solo quad-copter connected to an on-board RPi via a serial link. The RPi is enclosed in a custom 3D printed case. We used the dronekit helper library to communicate with the Pixhawk 2.0 flight controller embedded in the UAV. The RPi is connected with $5$ external wireless dongles: $3$ dongles are used in monitor mode to capture the beacons in WiFi channel $1$, $6$, and $11$, and the remaining $2$ dongles are used to support the make-before-break handover. tcpdump is used to capture the beacons.

The UAV operates in Guided mode, which uses GPS to navigate to way-points (latitude and longitude coordinates). The GCS transmits a predefined series of messages instructing the UAV to navigate to checkpoints. We consider two congestion scenarios using the Iperf utility: \emph{Scenario 1:} a continuous stream of traffic is routed through AP3 path; and \emph{Scenario 2:} the competing traffic stream is alternated between AP2 and AP3 path.

\subsection{Numerical Results}

Fig.~\ref{fig:delay_traffic} shows the average beacon and control delay for different levels of traffic injected at AP3 path, with which the UAV is connected. The maximum achievable traffic volume traversing an individual AP is equal to $8$~Mbps. It is apparent how congestion affects delay as it approaches the maximum supported rate. We observe that in the congestion region, control messages suffer a larger degradation. This is most likely due to the larger size of control packets with respect to beacons. The beacons utilize only a small fraction (approximately 0.045\%) of the total achievable throughput.

\begin{figure}[!t]
\centering   \includegraphics[width=.45\textwidth]{./images/delay_vs_traffic_rate.eps}
\vspace{-2mm}
\caption{Impact of congestion on beacon and control messages delay. The overall injected traffic volume is equal to $7$~Mbps.}
\vspace{-2mm}
\label{fig:delay_traffic}
\end{figure}



Fig.~\ref{fig:handover_delay} depicts the average delay of control messages achieved by different handover strategies in Scenario 1 and 2. The overall injected traffic volume is equal to $7$~Mbps. We test an RSSI-based handover strategy against our adaptive handover framework. It can be observed the considerable reduction in delay granted by the proposed framework. Note that the delay in the RSSI-based handover strategy halves in Scenario 2 with respect to Scenario 1. In fact, in the former the congestion is equally spread through the APs, with the UAV connected to one of them in periods uncorrelated with respect to the congestion level. The delay obtained using the proposed technique increases in Scenario 2, where the UAV is forced to shift between AP2 and AP3, suffering a delay penalty due to congestion detection and the establishment of the new forwarding connection for control messages.


\begin{figure}[!t]
\centering   \includegraphics[width=.5\textwidth]{./images/handover_control_delay.eps}
\vspace{-5mm}
\caption{Average control delay obtained by the handover strategies in Scenario 1 and 2.}
\vspace{-2mm}
\label{fig:handover_delay}
\end{figure}
\begin{figure}[t]
\centering   \includegraphics[width=.5\textwidth]{./images/avg_arrival_delay.eps}
\vspace{-5mm}
\caption{Average arrival delay to meet the predefined checkpoints in Scenario 1 and 2.}
\vspace{-6mm}
\label{fig:relative_delay}
\end{figure} 
We observed that RSSI and delay are largely uncorrelated in the considered scenario. In fact, although RSSI influences the maximum transmission rate of the direct wireless link between the APs and the UAV, beacon (and control) messages are small messages with small transmission time. Congestion at the AP's buffer or intermediate router results in a delayed forwarding of the packets. Thus handover necessarily needs to use additional information collected by routing packets through the possible paths connecting the GCS to the UAV. Note that RSSI may play a bigger role in determining the overall delay when heavier data streams, \emph{e.g.}, telemetry, are considered.

In addition to the measurement of network performance metric, we illustrate the beneficial impact of the proposed technique on UAV control. In this experiment, we define a sequence of instructions that guide the UAV through a series of waypoints (GPS coordinates).
Fig.~\ref{fig:relative_delay} depicts the average delay in reaching each individual checkpoint granted by the handover techniques with respect to a case with no congestion in Scenario 1 and 2. The reduced time needed to deliver the control messages from the GCS to the UAV granted by the proposed technique results in a reduced delay in reaching the waypoints with respect to RSSI-based handover. Again, we notice the same trend where Scenario 2 mitigates congestion in RSSI-based handover and penalizes the proposed technique due to the more frequent handover events triggered by the alternated traffic injection.



Fig.~\ref{fig:rel_way_delay} shows the temporal traces of the relative delay. It can be observed that RSSI-based handover incurs periods of large delay when congestion affects the AP used to communicate with the GCS. The proposed technique has short delay peaks corresponding to handover events.