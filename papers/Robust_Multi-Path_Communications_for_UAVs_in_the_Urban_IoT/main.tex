%% This is a skeleton file demonstrating the use of IEEEtran.cls (requires IEEEtran.cls version 1.8a or later) with an IEEE conference paper.
%%
%% Modified by Khan Reaz( kahn.reaz@ieee.org)
%% Support sites:
%% http://www.ieee.org/

%%***********************************************************
%% Legal Notice:
%% This code is offered as-is without any warranty either expressed or implied; without even the implied warranty of MERCHANTABILITY or FITNESS FOR A PARTICULAR PURPOSE! 
%% User assumes all risk and can modify as s/he wants.

%%***********************************************************

%package list
\documentclass[conference]{IEEEtran}
\IEEEoverridecommandlockouts
%%\usepackage{cite}
%\usepackage{graphicx}
\usepackage[dvips]{graphicx}
\graphicspath{ {images/} }
\usepackage[utf8]{inputenc}
\usepackage[T1]{fontenc}
\usepackage[english]{babel}
\usepackage[shortcuts]{extdash}
\usepackage[noadjust]{cite}
\usepackage{amsmath}
\usepackage{amssymb}
\usepackage{mathtools}
\usepackage{dsfont}
\usepackage{algorithm}
\usepackage{algpseudocode}
\allowdisplaybreaks
\usepackage{tikz}
\usepackage{booktabs}
\usepackage{multirow}
\usepackage{float}
\usepackage[position=top,font=footnotesize,caption=false]{subfig}
\usepackage{xcolor}
\usepackage{url}
\usepackage{verbatim}
\usepackage{float}

\makeatletter



\let\old@ps@headings\ps@headings
\let\old@ps@IEEEtitlepagestyle\ps@IEEEtitlepagestyle
\def\confheader#1{%
%    % for all pages except the first
%    \def\ps@headings{%
%        \old@ps@headings%
%        \def\@oddhead{\strut\hfill#1\hfill\strut}%
%        \def\@evenhead{\strut\hfill#1\hfill\strut}%
%    }%
    % for the first page
    \def\ps@IEEEtitlepagestyle{%
        \old@ps@IEEEtitlepagestyle%
        \def\@oddhead{\strut\hfill#1\hfill\strut}%
        \def\@evenhead{\strut\hfill#1\hfill\strut}%
    }%
    \ps@headings%
}
\makeatother

\confheader{%
        \parbox{\textwidth}{\centering This article has been accepted for publication in the  IEEE International Conference on Sensing, Communication and Networking (SECON Workshops), 2018.}
}



\begin{document}

\title{Robust Multi-Path Communications\\ for UAVs in the Urban IoT}
\author
{\IEEEauthorblockN{\large Zoheb Shaikh, Sabur Baidya and Marco Levorato}
\IEEEauthorblockA{Computer Science Department, UC Irvine, CA, US\\
Email: \{zsshaikh,sbaidya,levorato\}@uci.edu}
\vspace{-8mm}
}


\maketitle

\IEEEpubid{\begin{minipage}{\textwidth}\ \\\\\\[12pt]\\\\ \centering
  \\ \copyright 2018 IEEE. Personal use of this material is permitted. Permission from IEEE must be obtained for all other uses, in any current or future media,\\including reprinting/republishing this material for advertising or promotional purposes, creating new collective works, for resale\\or redistribution to servers or lists, or reuse of any copyrighted component of this work in other works.
\end{minipage}} 


\begin{abstract}
\begin{abstract}

Visual perception tasks often require vast amounts of labelled data, including 3D poses and image space segmentation masks. The process of creating such training data sets can prove difficult or time-intensive to scale up to efficacy for general use. Consider the task of pose estimation for rigid objects. Deep neural network based approaches have shown good performance when trained on large, public datasets. However, adapting these networks for other novel objects, or fine-tuning existing models for different environments, requires significant time investment to generate newly labelled instances. Towards this end, we propose ProgressLabeller as a method for more efficiently generating large amounts of 6D pose training data from color images sequences for custom scenes in a scalable manner. ProgressLabeller is intended to also support transparent or translucent objects, for which the previous methods based on depth dense reconstruction will fail.
We demonstrate the effectiveness of ProgressLabeller by rapidly create a dataset of over 1M samples with which we fine-tune a state-of-the-art pose estimation network in order to markedly improve the downstream robotic grasp success rates. Progresslabeller is open-source at \href{https://github.com/huijieZH/ProgressLabeller}{https://github.com/huijieZH/ProgressLabeller}

\end{abstract}
\end{abstract}

\vspace{1mm}
\begin {IEEEkeywords}
Unmanned Aerial Vehicles, Urban Internet of Things, Congestion Control, Network Path Selection
\end{IEEEkeywords}



\section{Introduction}
\label{intro}
\begin{figure}[t]
\begin{center}
   \includegraphics[width=1.0\linewidth]{figures/nas_comp_v3}
\end{center}
   \vspace{-4mm}
   \caption{The comparison between NetAdaptV2 and related works. The number above a marker is the corresponding total search time measured on NVIDIA V100 GPUs.}
\label{fig:nas_comparison}
\end{figure}

\section{Introduction}
\label{sec:introduction}

Neural architecture search (NAS) applies machine learning to automatically discover deep neural networks (DNNs) with better performance (e.g., better accuracy-latency trade-offs) by sampling the search space, which is the union of all discoverable DNNs. The search time is one key metric for NAS algorithms, which accounts for three steps: 1) training a \emph{super-network}, whose weights are shared by all the DNNs in the search space and trained by minimizing the loss across them, 2) training and evaluating sampled DNNs (referred to as \emph{samples}), and 3) training the discovered DNN. Another important metric for NAS is whether it supports non-differentiable search metrics such as hardware metrics (e.g., latency and energy). Incorporating hardware metrics into NAS is the key to improving the performance of the discovered DNNs~\cite{eccv2018-netadapt, Tan2018MnasNetPN, cai2018proxylessnas, Chen2020MnasFPNLL, chamnet}.


There is usually a trade-off between the time spent for the three steps and the support of non-differentiable search metrics. For example, early reinforcement-learning-based NAS methods~\cite{zoph2017nasreinforcement, zoph2018nasnet, Tan2018MnasNetPN} suffer from the long time for training and evaluating samples. Using a super-network~\cite{yu2018slimmable, Yu_2019_ICCV, autoslim_arxiv, cai2020once, yu2020bignas, Bender2018UnderstandingAS, enas, tunas, Guo2020SPOS} solves this problem, but super-network training is typically time-consuming and becomes the new time bottleneck. The gradient-based methods~\cite{gordon2018morphnet, liu2018darts, wu2018fbnet, fbnetv2, cai2018proxylessnas, stamoulis2019singlepath, stamoulis2019singlepathautoml, Mei2020AtomNAS, Xu2020PC-DARTS} reduce the time for training a super-network and training and evaluating samples at the cost of sacrificing the support of non-differentiable search metrics. In summary, many existing works either have an unbalanced reduction in the time spent per step (i.e., optimizing some steps at the cost of a significant increase in the time for other steps), which still leads to a long \emph{total} search time, or are unable to support non-differentiable search metrics, which limits the performance of the discovered DNNs.

In this paper, we propose an efficient NAS algorithm, NetAdaptV2, to significantly reduce the \emph{total} search time by introducing three innovations to \emph{better balance} the reduction in the time spent per step while supporting non-differentiable search metrics:

\textbf{Channel-level bypass connections (mainly reduce the time for training and evaluating samples, Sec.~\ref{subsec:channel_level_bypass_connections})}: Early NAS works only search for DNNs with different numbers of filters (referred to as \emph{layer widths}). To improve the performance of the discovered DNN, more recent works search for DNNs with different numbers of layers (referred to as \emph{network depths}) in addition to different layer widths at the cost of training and evaluating more samples because network depths and layer widths are usually considered independently. In NetAdaptV2, we propose \emph{channel-level bypass connections} to merge network depth and layer width into a single search dimension, which requires only searching for layer width and hence reduces the number of samples.

\textbf{Ordered dropout (mainly reduces the time for training a super-network, Sec.~\ref{subsec:ordered_droput})}: We adopt the idea of super-network to reduce the time for training and evaluating samples. In previous works, \emph{each} DNN in the search space requires one forward-backward pass to train. As a result, training multiple DNNs in the search space requires multiple forward-backward passes, which results in a long training time. To address the problem, we propose \emph{ordered dropout} to jointly train multiple DNNs in a \emph{single} forward-backward pass, which decreases the required number of forward-backward passes for a given number of DNNs and hence the time for training a super-network.

\textbf{Multi-layer coordinate descent optimizer (mainly reduces the time for training and evaluating samples and supports non-differentiable search metrics, Sec.~\ref{subsec:optimizer}):} NetAdaptV1~\cite{eccv2018-netadapt} and MobileNetV3~\cite{Howard_2019_ICCV}, which utilizes NetAdaptV1, have demonstrated the effectiveness of the single-layer coordinate descent (SCD) optimizer~\cite{book2020sze} in discovering high-performance DNN architectures. The SCD optimizer supports both differentiable and non-differentiable search metrics and has only a few interpretable hyper-parameters that need to be tuned, such as the per-iteration resource reduction. However, there are two shortcomings of the SCD optimizer. First, it only considers one layer per optimization iteration. Failing to consider the joint effect of multiple layers may lead to a worse decision and hence sub-optimal performance. Second, the per-iteration resource reduction (e.g., latency reduction) is limited by the layer with the smallest resource consumption (e.g., latency). It may take a large number of iterations to search for a very deep network because the per-iteration resource reduction is relatively small compared with the network resource consumption. To address these shortcomings,  we propose the \emph{multi-layer coordinate descent (MCD) optimizer} that considers multiple layers per optimization iteration to improve performance while reducing search time and preserving the support of non-differentiable search metrics.

Fig.~\ref{fig:nas_comparison} (and Table~\ref{tab:nas_result}) compares NetAdaptV2 with related works. NetAdaptV2 can reduce the search time by up to $5.8\times$ and $2.4\times$ on ImageNet~\cite{imagenet_cvpr09} and NYU Depth V2~\cite{nyudepth} respectively and discover DNNs with better performance than state-of-the-art NAS works. Moreover, compared to NAS-discovered MobileNetV3~\cite{Howard_2019_ICCV}, the discovered DNN has $1.8\%$ higher accuracy with the same latency.




\begin{figure*}[!t]
   \centering
   \includegraphics[scale=.6]{system_model}
    \vspace{-0.2cm}
    \caption{Unmanned Aerial Vehicles (UAVs) operating in an urban environment. The dynamics of traffic created by competing applications and the high mobility of the UAVs make robust control challenging. In this paper, we present a multi-hop multi-path adaptive networking strategy to solve those issues.}
    \vspace{-0.4cm}
    \label{fig:uaviot}
\end{figure*}



\section{Related Work}
\label{relatedwork}
\section{Related work}
\label{sec:related_work}

Accessibility is an essential component of computing, which aims to make technology broadly accessible to as many users as possible, including those with differing sets of abilities. Improvements in usability and accessibility falls to the community, to better understand the needs of users with differing abilities, and to design technologies that play to this spectrum of abilities \citep{Wobbrock2011AbilityBasedDC}.
In computing, significant strides have been made to increase the accessibility of web content. For example, various versions of the Web Content Accessibility Guidelines (WCAG) \citep{Chisholm2001WebCA, Caldwell2008WebCA} and the in-progress working draft for WCAG 3.0,\footnote{\href{https://www.w3.org/TR/wcag-3.0/}{https://www.w3.org/TR/wcag-3.0/}} or standards such as ARIA from the W3C's Web Accessibility Initiative (WAI)\footnote{\href{https://www.w3.org/WAI/standards-guidelines/aria/}{https://www.w3.org/WAI/standards-guidelines/aria/}} have been released and used to guide web accessibility design and implementation. Similarly, positive steps have been made to improve the accessibility of user interfaces and user experience \citep{Peissner2012MyUIGA, Peissner2013UserCI, Thompson2014ImprovingTU, Bigham2014MakingTW}, as well as various types of media content \citep{Mirri2017TowardsAG, Nengroo2017AccessibleI, Gleason2020TwitterAA}. 

We take inspiration from accessibility design principles in our effort to make research publications more accessible to users who are blind and low vision. Blindness and low vision are some of the most common forms of disability, affecting an estimated 3--10\% of Americans depending on how visual impairment is defined \citep{CDCVisionLossBurden}. BLV researchers also make up a representative sample of researchers in the United States and worldwide. A recent Nature editorial pushes the scientific community to better support researchers with visual impairments \citep{NatureCareerColumn2020}, since existing tools and resources can be limited. There are many inherent accessibility challenges to performing research. In this paper, we engage with one of these challenges that affects all domains of study, accessing and reading the content of academic publications. 

BLV users interact with papers using screen readers, braille displays, text-to-speech, and other assistive tools. A WebAIM survey of screen reader users found that the vast majority (75.1\%) of respondents indicate that PDF documents are very or somewhat likely to pose significant accessibility issues.\footnote{\href{https://webaim.org/projects/screenreadersurvey8/}{https://webaim.org/projects/screenreadersurvey8/}} Most paper are published in PDF, which is inherently inaccessible, due in large part to its conflation of visual layout information with semantic content \citep{NielsenPDFStillUnfit, Bigham2016AnUT}. 
\citet{Bigham2016AnUT} describe the historical reasons we use PDF as the standard document format for scientific publications, as well as the barriers the format itself presents to accessibility. Prior work on scientific accessibility have made recommendations for how to make PDFs more accessible \cite{Rajkumar2020PDFAO, Darvishy2018PDFAT}, including greater awareness for what constitutes an accessible PDF and better tooling for generating accessible PDFs. Some work has focused on addressing components of paper accessibility, such as the correct way for screen readers to interpret and read mathematical equations \citep{Flores2010MathMLTA, Bates2010SpokenMU, Sorge2014TowardsMM, Mackowski2017MultimediaPF, Ahmetovic2018AxessibilityAL, Ferreira2004EnhancingTA, Sojka2013AccessibilityII}, describe charts and figures \citep{Elzer2008AccessibleBC, Engel2017TowardsAC, Engel2019SVGPlottAA}, automatically generate figure captions \citep{Chen2019NeuralCG, Qian2020AFS}, or automatically classify the content of figures \citep{Kim2018MultimodalDL}. Other work applicable to all types of PDF documents aims to improve automatic text and layout detection of scanned documents \cite{Nazemi2014PracticalSM} and extract table content \cite{Fan2015TableRD, Rastan2019TEXUSAU}. In this work, we focus on the issue of representing overall document structure, and navigation within that structure. Being able to quickly navigate the contents of a paper through skimming and scanning is an essential reading technique \citep{Maxwell1972SkimmingAS}, which is currently under-supported by PDF documents and PDF readers when reading these documents by screen reader. 

There also exists a variety of automatic and manual tools that assess and fix accessibility compliance issues in PDFs, including the Adobe Acrobat Pro Accessibility Checker\footnote{\href{https://www.adobe.com/accessibility/products/acrobat/using-acrobat-pro-accessibility-checker.html}{https://www.adobe.com/accessibility/products/acrobat/using-acrobat-pro-accessibility-checker.html}}, Common Look\footnote{\href{https://monsido.com/monsido-commonlook-partnership}{https://monsido.com/monsido-commonlook-partnership}}, ABBYY FineReader\footnote{\href{https://pdf.abbyy.com/}{https://pdf.abbyy.com/}}, PAVE\footnote{\href{https://pave-pdf.org/faq.html}{https://pave-pdf.org/faq.html}}, and PDFA Inspector\footnote{\href{https://github.com/pdfae/PDFAInspector}{https://github.com/pdfae/PDFAInspector}}. To our knowledge, PAVE and PDFA Inspector are the only non-proprietary, open-source tools for this purpose. Based on our experiences, however, all of these tools require some degree of human intervention to properly tag a scientific document, and tagging and fixing must be performed for each new version of a PDF, regardless of how minor the change may be.

Guidelines and policy changes have been introduced in the past decade to ameliorate some of the issues around scientific PDF accessibility. Some conferences, such as The ACM CHI Virtual Conference on Human Factors in Computing Systems (CHI) and The ACM SIGACCESS Conference on Computers and Accessibility (ASSETS), have released guidelines for creating accessible submissions.\footnote{See \href{http://chi2019.acm.org/authors/papers/guide-to-an-accessible-submission/}{http://chi2019.acm.org/authors/papers/guide-to-an-accessible-submission/} and \href{https://assets19.sigaccess.org/creating_accessible_pdfs.html}{https://assets19.sigaccess.org/creating\_accessible\_pdfs.html}} The ACM Digital Library\footnote{\href{https://dl.acm.org/}{https://dl.acm.org/}} provides some publications in HTML format, which is easier to make accessible than PDF~\cite{Graells2007EstudioDL}. \citet{Ribera2019PublishingAP} conducted a case study on DSAI 2016 (Software Development and Technologies for Enhancing Accessibility and Fighting Infoexclusion). The authors of DSAI were responsible for creating accessible proceedings and identified barriers to creating accessible proceedings, including lack of sufficient tooling and lack of awareness of accessibility. The authors recommended creating a new role in the organizing committee dedicated to accessible publishing. These policy changes have led to improvements in localized communities, but have not been widely adopted by all academic publishers and conference organizers.

Table~\ref{tab:prior_work} lists prior studies that have analyzed PDF accessibility of academic papers, and shows how our study compares. Prior work has primarily focused on papers published in Human-Computer Interaction and related fields, specific to certain publication venues, while our analysis tries to quantify paper accessibility more broadly.
\citet{Brady2015CreatingAP} quantified the accessibility of 1,811 papers from CHI 2010-2016, ASSETS 2014, and W4A, assessing the presence of document tags, headers, and language. They found that compliance improved over time as a response to conference organizers offering to make papers accessible as a service to any author upon request. \citet{Lazar2017MakingTF} conducted a study quantifying accessibility compliance at CHI from 2010 to 2016 as well as ASSETS 2015,
%\jb{Define acronyms in prev para}
confirming the results of \citet{Brady2015CreatingAP}. They found that across 5 accessibility criteria, the rate of compliance was less than 30\% for CHI papers in each of the 7 years that were studied. The study also analyzed papers from ASSETS 2015, an ACM conference explicitly focused on accessibility, and found that those papers had significantly higher rates of compliance, with over 90\% of the papers being tagged for correct reading order and no criteria having less than 50\% compliance. This finding indicates that community buy-in is an important contributor to paper accessibility.
\citet{Nganji2015ThePD} conducted a study of 200 PDFs of papers published in four disability studies journals, finding that accessibility compliance was between 15-30\% for the four journals analyzed, with some publishers having higher adherence than others. To date, no large scale analysis of scientific PDF accessibility has been conducted outside of disability studies and HCI, due in part to the challenge of scaling such an analysis. We believe such an analysis is useful for establishing a baseline and characterizing routes for future improvement. Consequently, as part of this work, we conduct an analysis of scientific PDF accessibility across various fields of study, and report our findings relative to prior work. 


\begin{table}[t!]
\small
    \centering
    \begin{tabularx}{\linewidth}{L{22mm}L{15mm}L{48mm}L{16mm}L{34mm}}
        \toprule
        \textbf{Prior work} & \textbf{PDFs analyzed} & \textbf{Venues} & \textbf{Year} & \textbf{Accessibility checker} \\
        \midrule
        \citet{Brady2015CreatingAP} & 1811 & CHI, ASSETS and W4A & 2011--2014 & PDFA Inspector \\ [0.5mm]
        \hline \\ [-2.5mm]
        \citet{Lazar2017MakingTF} & 465 + 32 & CHI and ASSETS & 2014--2015 & Adobe Acrobat Action Wizard \\ [0.5mm]
        \hline \\ [-2.5mm]
        \citet{Ribera2019PublishingAP} & 59 & DSAI & 2016 & Adobe PDF Accessibility Checker 2.0 \\ [0.5mm]
        \hline \\ [-2.5mm]
        \citet{Nganji2015ThePD} & 200 & \textit{Disability \& Society}, \textit{Journal of Developmental and Physical Disabilities}, \textit{Journal of Learning Disabilities}, and \textit{Research in Developmental Disabilities} & 2009--2013 & Adobe PDF Accessibility Checker 1.3 \\ [0.6mm]
        \hline \\ [-2.5mm]
        \textbf{\textit{Our analysis}} & \numpdfs & Venues across various fields of study & 2010--2019 & Adobe Acrobat Accessibility Plug-in Version 21.001.20145 \\
        \bottomrule
    \end{tabularx}
    \caption{Prior work has investigated PDF accessibility for papers published in specific venues such as CHI, ASSETS, W4A, DSAI, or various disability journals. Several of these works were conducted manually, and were limited to a small number of papers, while the more thorough analysis was conducted for CHI and ASSETS, two conference venues focused on accessibility and HCI. Our study expands on this prior work to investigate accessibility over \numpdfs PDFs sampled from across different fields of study.
    }
    % \Description{
    % Prior work, PDFs analyzed, Venues, Year, Accessibility checker 
    % Brady et al. [7], 1811, CHI, ASSETS and W4A, 2011--2014, PDFA Inspector 
    % Lazar et al. [23], 465 + 32, CHI and ASSETS, 2014--2015, Adobe Acrobat Action Wizard 
    % Ribera et al. [40], 59, DSAI, 2016, Adobe PDF Accessibility Checker 2.0 
    % Nganji [33], 200, Disability & Society, Journal of Developmental and Physical Disabilities, Journal of Learning Disabilities, and Research in Developmental Disabilities, 2009--2013, Adobe PDF Accessibility Checker 1.3
    % Our analysis, 11397, Venues across various fields of study, 2010--2019, Adobe Acrobat Accessibility Plug-in Version 21.001.20145 
    % }
    \label{tab:prior_work}
\end{table}


\section{System Architecture}
\label{sysmodel}

\subsection{Preliminaries}
%The application of UAVs in urban IoT is getting increasingly demanding due to the capability of sensing and data processing for various IoT services. 
Fig.~\ref{fig:uaviot} illustrates the scenario considered in this paper: a network of UAVs immersed in an urban environment where a multitude of other sensing and communication devices operate and coexist. %Typically, UAVs are directly interconnected to the GCS which controls their operations. However, 
Due to the topology of urban environments, a direct link between a UAV and GCS would likely fail to provide a satisfactory communication range, with an inevitable drop in the reliability of control messages delivery. Importantly, the UAV incorporates fail-safe mechanisms that are activated when the UAV is disconnected from the GCS, including GPS-based return-to-home function and emergency landing. However, in both cases the UAV fails to accomplish the assigned mission. 

%Therefore, preserving the connectivity between the UAV and the GCS is of paramount importance. To accomplish such objective, 
Hence, we use the wireless Access Points (AP) available in the city to forward control messages from the GCS to the UAV and telemetry data back from the UAV. The APs are interconnected through the backbone network with established minimum cost paths calculated using either Link State or Distance Vector protocols. In this paper, we focus on communications in the $2.4$~GHz band using Wi-Fi technology. However, the same reasoning can be applied to any, or multiple, technologies depending on the communication capabilities of the UAVs.


\subsection{Architecture}

Current approaches addressing connectivity in urban environments primarily use Received Signal Strength Indicator (RSSI) to perform AP selection. However, each individual AP and the router involved in the path to the AP may be also supporting other data streams, which may create localized congestion and affect the performance of a subset of the possible paths. Intuitively, messages from the GCS have stringent delay requirements, where excessive delay may affect controllability, or trigger fail-safe mechanisms as mentioned earlier. The architecture we propose is specifically designed to be robust against congestion and traffic dynamics. To accomplish this objective, we integrate RSSI with performance metrics evaluated in real-time indicating the current state of entire forwarding paths. Informed by the computed metrics, the framework, then, implements a flexible make-before-break handover mechanisms which dynamically selects the best path.

The performance of each path from the GCS to the UAV is measured using beacon messages. Specifically, the GCS periodically generates beacons: small packets containing the generation timestamp and the destination AP information. These beacons are forwarded to all the APs that the GCS can reach through the backbone network.
The UAV monitors all the WiFi channels and capture the broadcast beacons from all the APs in its vicinity.

As illustrated in Fig.~\ref{fig:sysarch}, the framework we propose consist of different functional blocks at the GCS and UAV. The functional blocks at the GCS are: \emph{Control Generator}, \emph{Beacon Generator} and \emph{Handover Manager}. The UAV functional blocks are: \emph{Deep Packet Inspector}, \emph{Performance Analyzer}, \emph{Decision Manager} and \emph{Handover Manager}. In the following, we describe each of these blocks in detail.

\begin{figure}[!t]
\centering   \includegraphics[width=.5\textwidth]{./images/UAV_multipath_arch.eps}
\vspace{-6mm}
\caption{Proposed flexible and robust communication architecture.}
\vspace{-6mm}
\label{fig:sysarch}
\end{figure}

\noindent
$\Box$ {\bf{GCS - Control Generator}}: 
This block generates and handles the control messages to be forwarded to the UAV. In the considered case, control messages belong to two classes: heartbeat messages and navigation messages. The former are small messages that are periodically generated so that the UAV can monitor the connection with the GCS. Navigation messages determine the motion of the UAV, and in the considered case contain target GPS coordinates and speed. These messages are defined by the mission control block, which is not explicitly included in the proposed architecture. The Control Generator block adds a generation timestamp and a sequence number to all the control messages. This information is used by the UAV to monitor the quality of the path used to communicate with the GCS in terms of absolute delay and message loss rate.  


\vspace{1mm}
\noindent
$\Box$ {\bf{GCS - Beacon Generator}}: 
Note that control messages are routed only through the path currently used to interconnect the GCS to the UAV. Thus, the timestamps and sequence numbers do not provide any information on all the other possible path options. In fact, broadcasting the control messages over the entire backbone network may increase congestion, especially in scenarios with a large number of UAVs. To address this issue, the Beacon Generator periodically generates small messages -- containing a timestamp and a sequence number -- that are broadcasted to all the APs. Note that the UAV does not need to be associated with any specific AP to receive the beacons.

\vspace{1mm}
\noindent
$\Box$ {\bf UAV - Deep Packet Inspector (DPI)}:
This block, implemented at the UAV side, captures all the beacons and control packets. The beacons are collected from all the APs and channels the UAV can receive from, whereas control messages are received only from the currently used path. The block inspects each received packets and creates a data point including the message type, the reception time, the sequence number and the originating AP (MAC address). This information is forwarded to the Performance Analyzer.

\vspace{1mm}
\noindent
$\Box$ {\bf{UAV - Performance Analyzer}}: 
This block receives the data points from the DPI block and determines per-message class packet loss rate and average absolute delay. These performance metrics are measured over a moving time window of duration equal to $\Delta$~seconds. The metrics are forwarded to the Decision Manager, where the moving average measures are used to trigger handover events based on control messages and select the best path based on beacons. In addition to message-related measures, the Performance Analyzer also measures the average RSSI associated with the various interconnected APs.

Note that the duration of the moving window influences the response time and frequency of the framework. On the one hand, a long window better smooths ``noise'', removing small delay and loss peaks, and avoids frequent handover. On the other hand, a short window allows a faster reaction of the framework to congestion. A thorough study on the effect of this parameter on the performance of control delivery and UAV navigation is not included here due to space constraints, and is deferred to future studies.



\vspace{1mm}
\noindent
$\Box$ {\bf{UAV - Decision Manager}}:
The Decision Manager block uses the moving average performance metrics derived by the Performance Analyzer to perform two functionalities: \emph{(a)} Trigger handover to a different AP; and \emph{(b)} Select the best path when a handover event is triggered. In the former functionality, only metrics relative to control messages are used, as handover is necessary only when the QoS of the current path suffers a degradation sufficient to impair the ability of the GCS to control the UAV. The latter functionality considers metrics relative to beacon reception from all the APs, as path selection requires the evaluation of all the feasible paths.

At time instant $t$ the Decision Manager receives moving average beacon delays $D^{i}_{\rm b}(t)$, RSSI $R^{i}_{\rm b}(t)$ and loss rate $L^i_{\rm b}(t)$ corresponding to AP $i$, with $i=1,\ldots,N$, moving average control delay $D_{\rm c}(t)$, RSSI $R_{\rm c}(t)$ and loss rate $L_{\rm c}(t)$.
A handover request is issued at time $t$ if one of the following conditions is satisfied:
\vspace{-1.5mm}
\begin{equation}
 \lambda_1  D_{\rm c}(t) {+} \lambda_2  L_{\rm c}(t) {+} \lambda_3 (R_{\rm max}{-}R_{\rm c}(t))  {>} \Theta;~~
 L_{\rm c}(t){>} \Phi,
\vspace{-1mm}
\end{equation}
where $\lambda_1$, $\lambda_2$ and $\lambda_3$ are positive weights, with $\lambda_1{+}\lambda_2{+}\lambda_3{=}1$, and $R_{\rm max}$ is the maximum RSSI index. $\Theta$ and $\Phi$ are positive thresholds. The first condition corresponds to a general degradation of the current path. In addition to the first condition, we include in the framework an urgent handover mechanism to recover from harsh events in which the connection with the current AP is abruptly severed. Specifically, if the number of heartbeats received in the window is below a certain threshold, the handover manager is immediately notified. This event corresponds to the second condition.

If a handover request is issued, the Decision Manager computes
the metric
\begin{equation}
W_{i}(t) =  \gamma_1  D^{i}_{\rm b}(t) + \gamma_2  L^{i}_{\rm b}(t) + \gamma_3 (R_{\rm max}{-}R^{i}_{\rm b}(t)),
\end{equation}
for all the APs $i=1,2,...,N$, where $\gamma_1$, $\gamma_2$ and $\gamma_3$ are positive weights, with $\gamma_1{+}\gamma_2{+}\gamma_3{=}1$ and $\Theta$ is a positive threshold defining the minimum accepted performance.
The path is selected as
\vspace{-2mm}
\begin{equation}
	k = \arg \min_{i}\{W_{i}(t)\}.
\vspace{-2mm}
\end{equation}
Thus, the decision manager selects the $k^{th}$ AP as the new control path if a handover request is triggered. In this case, the Decision Manager forwards to the UAV Handover Manager the handover request and the index of the new selected AP. 


\begin{figure}[!t]
   \centering
   \includegraphics[width=0.5\textwidth]{./images/linear-handover}
    \vspace{-6mm}
    \caption{Topology of the experimental setup.}
    \vspace{-6mm}
    \label{fig:linear-sense}
\end{figure}

\noindent
$\Box$ {\bf{Handover Manager}}: 
The handover manager block is located both at the GCS and UAV sides, and implements a 3-way handshake mechanism. The GCS maintains a data structure thats maps the connected UAVs to their corresponding IP addresses. Each UAV keeps track of the GCS's IP address which we assume to be fixed for the duration of the mission. If a handover request is triggered by decision manager, the handover manager at the UAV associates itself with the AP provided by the decision manager. The Dynamic Host Configuration Protocol (DHCP) server at the new AP provides an IP address to the UAV. To ensure make-before-break handover, at this point the UAV doesn't disassociate itself from the old AP and keeps receiving the control messages through that. Now, the UAV initiates handover by sending a handover request message to the GCS via both the APs to maximize the reception probability at GCS. The handover request message contains the UAV's new IP address and the information of new AP. Upon receiving this handover request message, the GCS sends an approve message and note the information received by the request message. Upon reception of approve message, the UAV completes the 3-way handshake by sending ACK message. After a successful handover, the GCS station forwards the control messages over the new path and the UAV disassociates itself from the old AP to save energy.





\section{Experimental Setup and Numerical Results}
\label{numresult}
We assess the performance of the proposed architecture and framework by means of real-world experiments.

\subsection{Experimental Setup}
In the considered setup, the backbone network is composed of three paths through three APs connected to a GCS. The topology is illustrated in Fig.~\ref{fig:linear-sense}: the GCS is connected with AP1, and AP2 and AP3 are at two hop distance from the GCS. All the three APs operate on non-overlapping channels. 

We use Raspberry Pi (RPi) to create the APs using hostapd and all the APs operate according to the IEEE 802.11b standard. The APs communicate with each other via static routing. The GCS, which runs on a laptop, generates a beacon every $200$~ms and a heartbeat message every $500$~ms. The frequency of beacons and heartbeat messages can be increased or decreased based on the observed coherence time of the system.
UDP is used as transport layer for both beacons and control messages. To synchronize the clocks among the UAV and GCS, we use the Network Time Protocol (NTP) with the GCS set as the NTP server.

The UAV is a 3DR solo quad-copter connected to an on-board RPi via a serial link. The RPi is enclosed in a custom 3D printed case. We used the dronekit helper library to communicate with the Pixhawk 2.0 flight controller embedded in the UAV. The RPi is connected with $5$ external wireless dongles: $3$ dongles are used in monitor mode to capture the beacons in WiFi channel $1$, $6$, and $11$, and the remaining $2$ dongles are used to support the make-before-break handover. tcpdump is used to capture the beacons.

The UAV operates in Guided mode, which uses GPS to navigate to way-points (latitude and longitude coordinates). The GCS transmits a predefined series of messages instructing the UAV to navigate to checkpoints. We consider two congestion scenarios using the Iperf utility: \emph{Scenario 1:} a continuous stream of traffic is routed through AP3 path; and \emph{Scenario 2:} the competing traffic stream is alternated between AP2 and AP3 path.

\subsection{Numerical Results}

Fig.~\ref{fig:delay_traffic} shows the average beacon and control delay for different levels of traffic injected at AP3 path, with which the UAV is connected. The maximum achievable traffic volume traversing an individual AP is equal to $8$~Mbps. It is apparent how congestion affects delay as it approaches the maximum supported rate. We observe that in the congestion region, control messages suffer a larger degradation. This is most likely due to the larger size of control packets with respect to beacons. The beacons utilize only a small fraction (approximately 0.045\%) of the total achievable throughput.

\begin{figure}[!t]
\centering   \includegraphics[width=.45\textwidth]{./images/delay_vs_traffic_rate.eps}
\vspace{-2mm}
\caption{Impact of congestion on beacon and control messages delay. The overall injected traffic volume is equal to $7$~Mbps.}
\vspace{-2mm}
\label{fig:delay_traffic}
\end{figure}



Fig.~\ref{fig:handover_delay} depicts the average delay of control messages achieved by different handover strategies in Scenario 1 and 2. The overall injected traffic volume is equal to $7$~Mbps. We test an RSSI-based handover strategy against our adaptive handover framework. It can be observed the considerable reduction in delay granted by the proposed framework. Note that the delay in the RSSI-based handover strategy halves in Scenario 2 with respect to Scenario 1. In fact, in the former the congestion is equally spread through the APs, with the UAV connected to one of them in periods uncorrelated with respect to the congestion level. The delay obtained using the proposed technique increases in Scenario 2, where the UAV is forced to shift between AP2 and AP3, suffering a delay penalty due to congestion detection and the establishment of the new forwarding connection for control messages.


\begin{figure}[!t]
\centering   \includegraphics[width=.5\textwidth]{./images/handover_control_delay.eps}
\vspace{-5mm}
\caption{Average control delay obtained by the handover strategies in Scenario 1 and 2.}
\vspace{-2mm}
\label{fig:handover_delay}
\end{figure}
\begin{figure}[t]
\centering   \includegraphics[width=.5\textwidth]{./images/avg_arrival_delay.eps}
\vspace{-5mm}
\caption{Average arrival delay to meet the predefined checkpoints in Scenario 1 and 2.}
\vspace{-6mm}
\label{fig:relative_delay}
\end{figure} 
We observed that RSSI and delay are largely uncorrelated in the considered scenario. In fact, although RSSI influences the maximum transmission rate of the direct wireless link between the APs and the UAV, beacon (and control) messages are small messages with small transmission time. Congestion at the AP's buffer or intermediate router results in a delayed forwarding of the packets. Thus handover necessarily needs to use additional information collected by routing packets through the possible paths connecting the GCS to the UAV. Note that RSSI may play a bigger role in determining the overall delay when heavier data streams, \emph{e.g.}, telemetry, are considered.

In addition to the measurement of network performance metric, we illustrate the beneficial impact of the proposed technique on UAV control. In this experiment, we define a sequence of instructions that guide the UAV through a series of waypoints (GPS coordinates).
Fig.~\ref{fig:relative_delay} depicts the average delay in reaching each individual checkpoint granted by the handover techniques with respect to a case with no congestion in Scenario 1 and 2. The reduced time needed to deliver the control messages from the GCS to the UAV granted by the proposed technique results in a reduced delay in reaching the waypoints with respect to RSSI-based handover. Again, we notice the same trend where Scenario 2 mitigates congestion in RSSI-based handover and penalizes the proposed technique due to the more frequent handover events triggered by the alternated traffic injection.



Fig.~\ref{fig:rel_way_delay} shows the temporal traces of the relative delay. It can be observed that RSSI-based handover incurs periods of large delay when congestion affects the AP used to communicate with the GCS. The proposed technique has short delay peaks corresponding to handover events.


\vspace{-2mm}
\section{Acknowledgments}
\label{sec:ACK}

The architecture proposed and studied in this paper is inspired by the platform and mission description provided at the DARPA Hackfest on Software Defined Radios. The authors of this paper participated as a team to the event. This work was partially supported by the NSF under grant IIS-1724331.
\vspace{-1mm}
\section{Conclusions}
\label {conclusion}
\section{Conclusion}
In this work, we present a novel strategy for addressing few-shot open-set recognition. We frame the few-shot open-set classification task as a meta-learning problem similar to \cite{peeler}, but unlike their strategy, we do not solely rely on thresholding softmax scores to indicate the openness of a sample. We argue that existing thresholding type FSOSR methods \cite{peeler,snatcher} rely heavily on the choice of a carefully tuned threshold to achieve good performance. Additionally, the proclivity of softmax to overfit to unseen classes makes it an unreliable choice as an open-set indicator, especially when there is a dearth of samples. Instead, we propose to use a reconstruction of exemplar images as a key signal to detect out-of-distribution samples. 
The learned embedding which is used to classify the sample is further modulated to ensure a proficient gap between the seen and unseen class clusters in the feature space. Finally, the modulated embedding, the softmax score, and the quality reconstructed exemplar are jointly utilized to cognize if the sample is in-distribution or out-of-distribution. 
The enhanced performance of our framework is verified empirically over a wide variety of few-shot tasks and the results establish it as the new state-of-the-art. In the future, we would like to extend this approach to more cross-domain few-shot tasks, including videos.
\vspace{-2em}
\section{Acknowledgement}
This work was partially supported by US National Science Foundation grant 2008020 and US Office of Naval Research grants N00014-19-1-2264 and N00014-18-1-2252.
\vspace{-1em}

\vspace{-1mm}
\bibliographystyle{IEEEtran}
\bibliography{main}

\end{document}