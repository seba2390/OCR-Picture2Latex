%% This is a skeleton file demonstrating the use of IEEEtran.cls (requires IEEEtran.cls version 1.8a or later) with an IEEE conference paper.
%%
%% Modified by Khan Reaz( kahn.reaz@ieee.org)
%% Support sites:
%% http://www.ieee.org/

%%***********************************************************
%% Legal Notice:
%% This code is offered as-is without any warranty either expressed or implied; without even the implied warranty of MERCHANTABILITY or FITNESS FOR A PARTICULAR PURPOSE! 
%% User assumes all risk and can modify as s/he wants.

%%***********************************************************

%package list
\documentclass[conference]{IEEEtran}
\IEEEoverridecommandlockouts
%%\usepackage{cite}
%\usepackage{graphicx}
\usepackage[dvips]{graphicx}
\graphicspath{ {images/} }
\usepackage[utf8]{inputenc}
\usepackage[T1]{fontenc}
\usepackage[english]{babel}
\usepackage[shortcuts]{extdash}
\usepackage[noadjust]{cite}
\usepackage{amsmath}
\usepackage{amssymb}
\usepackage{mathtools}
\usepackage{dsfont}
\usepackage{algorithm}
\usepackage{algpseudocode}
\allowdisplaybreaks
\usepackage{tikz}
\usepackage{booktabs}
\usepackage{multirow}
\usepackage{float}
\usepackage[position=top,font=footnotesize,caption=false]{subfig}
\usepackage{xcolor}
\usepackage{url}
\usepackage{verbatim}
\usepackage{float}

\makeatletter



\let\old@ps@headings\ps@headings
\let\old@ps@IEEEtitlepagestyle\ps@IEEEtitlepagestyle
\def\confheader#1{%
%    % for all pages except the first
%    \def\ps@headings{%
%        \old@ps@headings%
%        \def\@oddhead{\strut\hfill#1\hfill\strut}%
%        \def\@evenhead{\strut\hfill#1\hfill\strut}%
%    }%
    % for the first page
    \def\ps@IEEEtitlepagestyle{%
        \old@ps@IEEEtitlepagestyle%
        \def\@oddhead{\strut\hfill#1\hfill\strut}%
        \def\@evenhead{\strut\hfill#1\hfill\strut}%
    }%
    \ps@headings%
}
\makeatother

\confheader{%
        \parbox{\textwidth}{\centering This article has been accepted for publication in the  IEEE International Conference on Sensing, Communication and Networking (SECON Workshops), 2018.}
}



\begin{document}

\title{Robust Multi-Path Communications\\ for UAVs in the Urban IoT}
\author
{\IEEEauthorblockN{\large Zoheb Shaikh, Sabur Baidya and Marco Levorato}
\IEEEauthorblockA{Computer Science Department, UC Irvine, CA, US\\
Email: \{zsshaikh,sbaidya,levorato\}@uci.edu}
\vspace{-8mm}
}


\maketitle

\IEEEpubid{\begin{minipage}{\textwidth}\ \\\\\\[12pt]\\\\ \centering
  \\ \copyright 2018 IEEE. Personal use of this material is permitted. Permission from IEEE must be obtained for all other uses, in any current or future media,\\including reprinting/republishing this material for advertising or promotional purposes, creating new collective works, for resale\\or redistribution to servers or lists, or reuse of any copyrighted component of this work in other works.
\end{minipage}} 


\begin{abstract}
\begin{abstract}
%\medskip
%\centering \textcolor{red}{Write the abstract last}
Silicon-compatible short- and mid-wave infrared emitters are highly sought-after for on-chip monolithic integration of electronic and photonic circuits to serve a myriad of applications in sensing and communication. To address this longstanding challenge, GeSn semiconductors have been proposed as versatile building blocks for silicon-integrated optoelectronic devices. In this regard, this work demonstrates light-emitting diodes (LEDs) consisting of a vertical PIN double heterostructure  p-Ge$_{0.94}$Sn$_{0.06}$/i-Ge$_{0.91}$Sn$_{0.09}$/n-Ge$_{0.95}$Sn$_{0.05}$ grown epitaxially on a silicon wafer using germanium interlayer and multiple GeSn buffer layers. The emission from these GeSn LEDs at variable diameters in the 40-120 $\mu$m range is investigated under both DC and AC operation modes. The fabricated LEDs exhibit a room temperature emission in the extended short-wave range centered around 2.5 $\mu$m under an injected current density as low as 45 A/cm$^2$.  By comparing the photoluminescence and electroluminescence signals, it is demonstrated that the LED emission wavelength is not affected by the device fabrication process or heating during the LED operation. Moreover, the measured optical power was found to increase monotonically as the duty cycle increases indicating that the DC operation yields the highest achievable optical power. The LED emission profile and bandwidth are also presented and discussed. 
\end{abstract}
\end{abstract}

\vspace{1mm}
\begin {IEEEkeywords}
Unmanned Aerial Vehicles, Urban Internet of Things, Congestion Control, Network Path Selection
\end{IEEEkeywords}



\section{Introduction}
\label{intro}
\section{Introduction}

Scientific literature is most commonly available in the form of PDFs, which pose challenges for accessibility \citep{NielsenPDFStillUnfit, Bigham2016AnUT}. When researchers, students, and other individuals who are blind or low vision (BLV) interact with scientific PDFs through screen readers, the availability of document structure tags, labeled reading order, labeled headers, and image alt-text are necessary to facilitate these interactions. However, these features must be painstakingly added by authors using proprietary software tools, and as a result, are often missing from papers. Low vision or dyslexic readers who interact with PDFs through screen magnification or text-to-speech may also find the complexity of certain academic paper PDF formats challenging, e.g., non-linear layout can interrupt the flow of text in a magnifying tool. Inaccessible paper PDFs can lead to high cognitive overload, frustration, and abandonment of reading for BLV readers. 

Unfortunately, we find that the majority of scientific PDFs lack basic accessibility features. We estimate based on a sample of \numpdfs PDFs from multiple fields of study that only around \percaccessible of paper PDFs released in the last decade satisfy all of the aforementioned accessibility requirements. 
Accessibility challenges for academic PDFs are largely due to three factors: (1) the complexity of the PDF file format, which make it less amenable to certain accessibility features, (2) the dearth of tools, especially non-proprietary tools, for creating accessible PDFs, and (3) the dependency on volunteerism from the community with minimal support or enforcement \citep{Bigham2016AnUT}. The intent of the PDF file format is to support faithful visual representation of a document for printing, a goal that is inherently divergent from that of document representation for the purposes of accessibility. Though some professional organizations like the Association for Computing Machinery (ACM) have encouraged PDF accessibility through standards and writing guidelines,\footnote{\href{https://www.acm.org/publications/authors/submissions}{https://www.acm.org/publications/authors/submissions}} uptake among academic publishers and disciplines more broadly has been limited. 

While policy changes help, the fact remains that most academic PDFs produced today, and historically, are inaccessible, yet remain as the dominant way to read those papers. A long-range solution will necessitate buy-in from multiple stakeholders---publishers, authors, readers, technologists, granting agencies, and the like. But in the interim, there are technological solutions that can be offered as a sort of ``band-aid'' to the problem. We use this paper to offer an in-depth qualitative and quantitative description of the problem as it stands, and to introduce one such technological solution: the \scially system that automatically extracts semantic information from paper PDFs and re-renders this content in the form of an accessible HTML document. Though the process is imperfect and can introduce errors, we demonstrate the ability of the rendered HTMLs to reduce cognitive load and facilitate in-paper navigation and interactions for BLV users. 

The goals and contributions of this paper are three-fold:

\begin{enumerate}
    \item We characterize the state of academic-paper PDF accessibility by estimating the degree of adherence to accessibility criteria for papers published in the last decade (2010--2019), and describe correlations between year, field of study, PDF typesetting software, and PDF accessibility.
    \item We propose an automated approach for extracting the content of academic PDFs and displaying this content in a more accessible HTML document format. We build a prototype that re-renders 12 million PDFs in HTML, and describe the design decisions, features, and quality of the renders (assessed as faithfulness to the source PDF). We perform expert grading of the rendered HTML and report an error analysis. A demo of our system is available at \href{https://scia11y.org/}{scia11y.org}, which makes available 1.5M HTML renders of open access PDFs.
    \item We conduct an exploratory user study with \numusers BLV scholars to better understand the challenges they experience when reading academic papers and how our proposed tool might augment their current workflow. During the study, we ask users to interact with the prototype and offer feedback for its improvement. We perform open coding of interviews to identify existing reading challenges, coping mechanisms, as well as positive and negative responses to prototype features. We summarize the findings of this user study into a set of design recommendations.
\end{enumerate}

Our analysis reveals that PDF accessibility adherence is low across all fields of study. Of the five accessibility criteria we assess, only \percaccessible of the PDFs we assess demonstrate full compliance. Though compliance for several criteria seems to be increasing over time, author awareness and contribution to accessibility remains low, as Alt-text has the lowest compliance of the five criteria at between 5--10\% (Alt-text is the only criterion of the five that \textit{requires} author intervention in all cases using current tools). We also find that typesetting software is strongly associated with accessibility compliance, with LaTeX and publishing software like Arbortext APP producing low compliance PDFs, while Microsoft Word is generally associated with higher compliance.


\begin{figure}[t!]
    \centering
    \includegraphics[width=\textwidth]{figures/pipeline.png}
    \caption{A schematic for creating the \scially HTML render from a paper PDF. Starting with the raw two-column PDF on the left, S2ORC \citep{lo-wang-2020-s2orc} is used to extract title, authors, abstract, section headers, body text, and references. S2ORC also identifies links between inline citations and references to figures and table objects. DeepFigures \citep{Siegel2018ExtractingSF} is used to extract figures and tables, along with their captions. The output of these two models are merged with metadata from the Semantic Scholar API. Heuristics are used to construct a table of contents, to insert figures and tables in the appropriate places in the text, and to repair broken URLs. We add HTML headers as illustrated (header tags for sections, paragraph tags for body text, and figure tags for figures and tables); highlighted components (table of contents and links in references) are not in the PDF and novel navigational features that we introduce to the HTML render. An example HTML render of parts of a paper document is show to the right (actual render is single column, which is split here for presentation).}
    \label{fig:pipeline}
    \Description{A schematic diagram showing the components of the SciA11y pipeline. An image of a paper PDF is on the left. Red boxes on the PDF image highlight the text components from the paper, with an arrow pointing to a box that says "S2ORC extracts: title, authors, abstract, section headers, body paragraphs, and references." A blue box on the PDF image highlights a figure, with an arrow pointing to a box that says "DeepFigures extracts: figures, figure captions, tables, and table titles/captions." A box below "S2ORC extracts" and "DeepFigures extracts" says "Additional content: metadata from Semantic Scholar API, table of contents, figures and tables inserted at first mention, and links between references and text." Arrows from all three boxes point into a larger box that describes the SciA11y prototype, where HTML tags are inserted around various blocks of text extracted from the PDF. On the right of all this is a screen capture of an example HTML render, showing how the semantic content from the PDF is represented as a single-column HTML page for easy reading.}
\end{figure}

To offset the reading challenges of inaccessible papers for BLV researchers, we propose and test the \scially system for rendering academic PDFs into accessible HTML documents. As shown in Figure~\ref{fig:pipeline}, our prototype integrates several machine learning text and vision models to extract the structure and semantic content of papers. The content is represented as an HTML document with headings and links for navigation, figures and tables, as well as other novel features to assist in document structure understanding. Our evaluation of the \scially system identifies common classes of extraction problems, and finds that though many papers exhibit some extraction errors, the majority (55\%) have no major problems that impact readability, and another 32\% have only some problems that impact readability.

Through our user study, we identify numerous challenges faced by BLV users when reading paper PDFs, including some that affect the whole document or limit navigation, and many that affect the ability of the reader to understand text or various elements of a paper like math content or tables. Responses to \scially were positive; participants especially liked navigation features such as headings, the table of contents, and bidirectional links between inline citations and references. Of the extraction errors in \scially, missed or incorrectly extracted headings were the most problematic, as these impact the user's ability to navigate between sections and fully trust the system. All users reported being likely to use the system in the future. When asked how the system might be integrated into their workflow, one participant replied ``I think it would become the workflow.'' Another participant said, ``for unaccessible PDFs, this is life-changing.'' We condense these findings into a set of recommendations for designing and engineering accessible reading systems (Section~\ref{sec:designrecs}). Most importantly, documents should be structured to match a reader's mental model, objects should be properly tagged, and care should be taken to reduce the reader's cognitive load and increase trust in the system. Features that emulate the external memory that visual layout provides to sighted users can be especially beneficial.

This paper is organized as follows. Following a description of related work in Section \ref{sec:related_work}, we first provide a meta-scientific analysis of the current state of academic PDF accessibility in Section \ref{sec:sos}. In Section \ref{sec:pdf2html}, we document our pipeline for converting PDF to HTML and describe the \scially prototype for rendering papers. An evaluation of HTML render quality and faithfulness is provided in Section \ref{sec:evaluation}. Section \ref{sec:user_study} describes our user study and findings. 
We recognize that no PDF extraction system is perfect, and many open research challenges remain in improving these systems. However, based on our findings, we believe \scially can dramatically improve screen reader navigation of most papers compared to PDFs, and is well-positioned to assist BLV researchers with many of their most common reading use cases. Our hope is that a system such as \scially can improve BLV researcher access to the content of academic papers, and that these design recommendations can be leveraged by others to create better, more faithful, and ultimately more usable tools and systems for scholars in the BLV community.



\begin{figure*}[!t]
   \centering
   \includegraphics[scale=.6]{system_model}
    \vspace{-0.2cm}
    \caption{Unmanned Aerial Vehicles (UAVs) operating in an urban environment. The dynamics of traffic created by competing applications and the high mobility of the UAVs make robust control challenging. In this paper, we present a multi-hop multi-path adaptive networking strategy to solve those issues.}
    \vspace{-0.4cm}
    \label{fig:uaviot}
\end{figure*}



\section{Related Work}
\label{relatedwork}
\section{Related Work}
\label{sec:related_work}
% In this section, we review the related work, which includes graph neural networks, and robust graph neural networks. 

\subsection{Graph Neural Networks}
Graph Neural Networks (GNNs) have shown their great power in modeling graph structured data for various applications~\cite{wang2019semi,wang2018cross,zhao2020semi,dai2021say,zhao2021graphsmote}.
To generalize neural networks for graphs, two categories of GNNs are proposed, i.e., spectral-based~\cite{bruna2013spectral,henaff2015deep,kipf2016semi,levie2018cayleynets} and spatial-based~\cite{velivckovic2017graph,hamilton2017inductive,chen2018fastgcn,chiang2019cluster}. \citeauthor{bruna2013spectral} \cite{bruna2013spectral} first propose spectral-based GNNs by defining graph convolution with spectral graph theory. For instance, GCN~\cite{kipf2016semi} simplifies the convolutional operation by using the first order approximation. Spatial-based graph convolution is defined in spatial domain, which updates node representation by aggregating its neighbors' representations \cite{niepert2016learning,gilmer2017neural,hamilton2017inductive}. 
For example, self-attention of neighbor nodes is leveraged in graph attention network (GAT) \cite{velivckovic2017graph}. Moreover, various spatial methods are proposed to solve the scalability issue~\cite{chen2018fastgcn,chiang2019cluster} and learn deeper GNNs~\cite{chen2020simple}.  Recently, to alleviate the problem of lacking labeled nodes, many efforts are taken to explore GNNs using self-supervision, which aims to learn better node representations with pretext tasks~\cite{sun2019multi,li2018deeper,kim2021find,zhu2020self,jin2020self,dai2021towards}. For instance, superGAT~\cite{kim2021find} deploys edge prediction in GAT to guide the learning of attention for better representations. SE-GNN~\cite{dai2021towards} deploys contrastive learning to benefit the similarity modeling for self-explainable GNN.

Inspired by the great success of GNNs, methods that construct graphs and adopt GNNs for data without explicit relational structure are also explored~\cite{henaff2015deep,chen2019multi,jiang2019semi,dai2021nrgnn}. Generally, a graph would be built based on certain rules~\cite{henaff2015deep,chen2019multi} or be learned in an end-to-end model~\cite{jiang2019semi,dai2021nrgnn}. Our RS-GNN is inherently different from these methods as we eliminate/down-weight the noisy edges and predict the missing edges for robust GNNs on noisy graphs with limited labels. 

\subsection{Robust GNNs}
Although GNNs have obtained great achievements, they are vulnerable to adversarial attacks~\cite{wu2019adversarial,dai2018adversarial,zugner2018adversarial,zugner2019adversarial}. Based on the objective, the adversarial attacks on GNNs can be split into two categories, i.e., targeted attack~\cite{dai2018adversarial,zugner2018adversarial} and non-targeted attack~\cite{zugner2019adversarial}. Targeted attack methods aim to degrade the performance of the GNNs on target nodes. 
For instance, \textit{nettack}~\cite{zugner2018adversarial} adds adversarial perturbations to a graph to attack targeted nodes. Non-targeted attack aims to reduce the overall performance of GNNs. For example, \textit{metattack}~\cite{zugner2019adversarial} poisons the graph globally to achieve non-targeted attack with meta-learning. To defend against adversarial attacks, many efforts are taken recently~\cite{zhu2019robust,wu2019adversarial,entezari2020all,jin2020graph,tang2020transferring,zhang2020gnnguard}. \cite{wu2019adversarial} prune the perturbed edges based on Jaccard similarity of node features. Another preprocessing method by low-rank approximation of adjacent matrix is investigated~\cite{entezari2020all}. Pro-GNN~\cite{jin2020graph} is the most similar work to ours, which learns a clean graph structure by low-rank constraint. However, they only tackle the adversarial edges and their computational cost is very large due to the direct learning of the graph and the sparse low-rank constraint.
This work is inherently different from these methods as: (i) we study a novel problem of developing robust GNN for both noisy graphs and label sparsity issues; and (ii) the proposed RS-GNN simultaneously tackles the two issues by learning an link predictor to 
down-weight noisy edges and connecting nodes with high similarity to facilitate message-passing; 
and (iii) RS-GNN uses link predictor instead of direct graph learning to save computational cost. 


\section{System Architecture}
\label{sysmodel}

\subsection{Preliminaries}
%The application of UAVs in urban IoT is getting increasingly demanding due to the capability of sensing and data processing for various IoT services. 
Fig.~\ref{fig:uaviot} illustrates the scenario considered in this paper: a network of UAVs immersed in an urban environment where a multitude of other sensing and communication devices operate and coexist. %Typically, UAVs are directly interconnected to the GCS which controls their operations. However, 
Due to the topology of urban environments, a direct link between a UAV and GCS would likely fail to provide a satisfactory communication range, with an inevitable drop in the reliability of control messages delivery. Importantly, the UAV incorporates fail-safe mechanisms that are activated when the UAV is disconnected from the GCS, including GPS-based return-to-home function and emergency landing. However, in both cases the UAV fails to accomplish the assigned mission. 

%Therefore, preserving the connectivity between the UAV and the GCS is of paramount importance. To accomplish such objective, 
Hence, we use the wireless Access Points (AP) available in the city to forward control messages from the GCS to the UAV and telemetry data back from the UAV. The APs are interconnected through the backbone network with established minimum cost paths calculated using either Link State or Distance Vector protocols. In this paper, we focus on communications in the $2.4$~GHz band using Wi-Fi technology. However, the same reasoning can be applied to any, or multiple, technologies depending on the communication capabilities of the UAVs.


\subsection{Architecture}

Current approaches addressing connectivity in urban environments primarily use Received Signal Strength Indicator (RSSI) to perform AP selection. However, each individual AP and the router involved in the path to the AP may be also supporting other data streams, which may create localized congestion and affect the performance of a subset of the possible paths. Intuitively, messages from the GCS have stringent delay requirements, where excessive delay may affect controllability, or trigger fail-safe mechanisms as mentioned earlier. The architecture we propose is specifically designed to be robust against congestion and traffic dynamics. To accomplish this objective, we integrate RSSI with performance metrics evaluated in real-time indicating the current state of entire forwarding paths. Informed by the computed metrics, the framework, then, implements a flexible make-before-break handover mechanisms which dynamically selects the best path.

The performance of each path from the GCS to the UAV is measured using beacon messages. Specifically, the GCS periodically generates beacons: small packets containing the generation timestamp and the destination AP information. These beacons are forwarded to all the APs that the GCS can reach through the backbone network.
The UAV monitors all the WiFi channels and capture the broadcast beacons from all the APs in its vicinity.

As illustrated in Fig.~\ref{fig:sysarch}, the framework we propose consist of different functional blocks at the GCS and UAV. The functional blocks at the GCS are: \emph{Control Generator}, \emph{Beacon Generator} and \emph{Handover Manager}. The UAV functional blocks are: \emph{Deep Packet Inspector}, \emph{Performance Analyzer}, \emph{Decision Manager} and \emph{Handover Manager}. In the following, we describe each of these blocks in detail.

\begin{figure}[!t]
\centering   \includegraphics[width=.5\textwidth]{./images/UAV_multipath_arch.eps}
\vspace{-6mm}
\caption{Proposed flexible and robust communication architecture.}
\vspace{-6mm}
\label{fig:sysarch}
\end{figure}

\noindent
$\Box$ {\bf{GCS - Control Generator}}: 
This block generates and handles the control messages to be forwarded to the UAV. In the considered case, control messages belong to two classes: heartbeat messages and navigation messages. The former are small messages that are periodically generated so that the UAV can monitor the connection with the GCS. Navigation messages determine the motion of the UAV, and in the considered case contain target GPS coordinates and speed. These messages are defined by the mission control block, which is not explicitly included in the proposed architecture. The Control Generator block adds a generation timestamp and a sequence number to all the control messages. This information is used by the UAV to monitor the quality of the path used to communicate with the GCS in terms of absolute delay and message loss rate.  


\vspace{1mm}
\noindent
$\Box$ {\bf{GCS - Beacon Generator}}: 
Note that control messages are routed only through the path currently used to interconnect the GCS to the UAV. Thus, the timestamps and sequence numbers do not provide any information on all the other possible path options. In fact, broadcasting the control messages over the entire backbone network may increase congestion, especially in scenarios with a large number of UAVs. To address this issue, the Beacon Generator periodically generates small messages -- containing a timestamp and a sequence number -- that are broadcasted to all the APs. Note that the UAV does not need to be associated with any specific AP to receive the beacons.

\vspace{1mm}
\noindent
$\Box$ {\bf UAV - Deep Packet Inspector (DPI)}:
This block, implemented at the UAV side, captures all the beacons and control packets. The beacons are collected from all the APs and channels the UAV can receive from, whereas control messages are received only from the currently used path. The block inspects each received packets and creates a data point including the message type, the reception time, the sequence number and the originating AP (MAC address). This information is forwarded to the Performance Analyzer.

\vspace{1mm}
\noindent
$\Box$ {\bf{UAV - Performance Analyzer}}: 
This block receives the data points from the DPI block and determines per-message class packet loss rate and average absolute delay. These performance metrics are measured over a moving time window of duration equal to $\Delta$~seconds. The metrics are forwarded to the Decision Manager, where the moving average measures are used to trigger handover events based on control messages and select the best path based on beacons. In addition to message-related measures, the Performance Analyzer also measures the average RSSI associated with the various interconnected APs.

Note that the duration of the moving window influences the response time and frequency of the framework. On the one hand, a long window better smooths ``noise'', removing small delay and loss peaks, and avoids frequent handover. On the other hand, a short window allows a faster reaction of the framework to congestion. A thorough study on the effect of this parameter on the performance of control delivery and UAV navigation is not included here due to space constraints, and is deferred to future studies.



\vspace{1mm}
\noindent
$\Box$ {\bf{UAV - Decision Manager}}:
The Decision Manager block uses the moving average performance metrics derived by the Performance Analyzer to perform two functionalities: \emph{(a)} Trigger handover to a different AP; and \emph{(b)} Select the best path when a handover event is triggered. In the former functionality, only metrics relative to control messages are used, as handover is necessary only when the QoS of the current path suffers a degradation sufficient to impair the ability of the GCS to control the UAV. The latter functionality considers metrics relative to beacon reception from all the APs, as path selection requires the evaluation of all the feasible paths.

At time instant $t$ the Decision Manager receives moving average beacon delays $D^{i}_{\rm b}(t)$, RSSI $R^{i}_{\rm b}(t)$ and loss rate $L^i_{\rm b}(t)$ corresponding to AP $i$, with $i=1,\ldots,N$, moving average control delay $D_{\rm c}(t)$, RSSI $R_{\rm c}(t)$ and loss rate $L_{\rm c}(t)$.
A handover request is issued at time $t$ if one of the following conditions is satisfied:
\vspace{-1.5mm}
\begin{equation}
 \lambda_1  D_{\rm c}(t) {+} \lambda_2  L_{\rm c}(t) {+} \lambda_3 (R_{\rm max}{-}R_{\rm c}(t))  {>} \Theta;~~
 L_{\rm c}(t){>} \Phi,
\vspace{-1mm}
\end{equation}
where $\lambda_1$, $\lambda_2$ and $\lambda_3$ are positive weights, with $\lambda_1{+}\lambda_2{+}\lambda_3{=}1$, and $R_{\rm max}$ is the maximum RSSI index. $\Theta$ and $\Phi$ are positive thresholds. The first condition corresponds to a general degradation of the current path. In addition to the first condition, we include in the framework an urgent handover mechanism to recover from harsh events in which the connection with the current AP is abruptly severed. Specifically, if the number of heartbeats received in the window is below a certain threshold, the handover manager is immediately notified. This event corresponds to the second condition.

If a handover request is issued, the Decision Manager computes
the metric
\begin{equation}
W_{i}(t) =  \gamma_1  D^{i}_{\rm b}(t) + \gamma_2  L^{i}_{\rm b}(t) + \gamma_3 (R_{\rm max}{-}R^{i}_{\rm b}(t)),
\end{equation}
for all the APs $i=1,2,...,N$, where $\gamma_1$, $\gamma_2$ and $\gamma_3$ are positive weights, with $\gamma_1{+}\gamma_2{+}\gamma_3{=}1$ and $\Theta$ is a positive threshold defining the minimum accepted performance.
The path is selected as
\vspace{-2mm}
\begin{equation}
	k = \arg \min_{i}\{W_{i}(t)\}.
\vspace{-2mm}
\end{equation}
Thus, the decision manager selects the $k^{th}$ AP as the new control path if a handover request is triggered. In this case, the Decision Manager forwards to the UAV Handover Manager the handover request and the index of the new selected AP. 


\begin{figure}[!t]
   \centering
   \includegraphics[width=0.5\textwidth]{./images/linear-handover}
    \vspace{-6mm}
    \caption{Topology of the experimental setup.}
    \vspace{-6mm}
    \label{fig:linear-sense}
\end{figure}

\noindent
$\Box$ {\bf{Handover Manager}}: 
The handover manager block is located both at the GCS and UAV sides, and implements a 3-way handshake mechanism. The GCS maintains a data structure thats maps the connected UAVs to their corresponding IP addresses. Each UAV keeps track of the GCS's IP address which we assume to be fixed for the duration of the mission. If a handover request is triggered by decision manager, the handover manager at the UAV associates itself with the AP provided by the decision manager. The Dynamic Host Configuration Protocol (DHCP) server at the new AP provides an IP address to the UAV. To ensure make-before-break handover, at this point the UAV doesn't disassociate itself from the old AP and keeps receiving the control messages through that. Now, the UAV initiates handover by sending a handover request message to the GCS via both the APs to maximize the reception probability at GCS. The handover request message contains the UAV's new IP address and the information of new AP. Upon receiving this handover request message, the GCS sends an approve message and note the information received by the request message. Upon reception of approve message, the UAV completes the 3-way handshake by sending ACK message. After a successful handover, the GCS station forwards the control messages over the new path and the UAV disassociates itself from the old AP to save energy.





\section{Experimental Setup and Numerical Results}
\label{numresult}
We assess the performance of the proposed architecture and framework by means of real-world experiments.

\subsection{Experimental Setup}
In the considered setup, the backbone network is composed of three paths through three APs connected to a GCS. The topology is illustrated in Fig.~\ref{fig:linear-sense}: the GCS is connected with AP1, and AP2 and AP3 are at two hop distance from the GCS. All the three APs operate on non-overlapping channels. 

We use Raspberry Pi (RPi) to create the APs using hostapd and all the APs operate according to the IEEE 802.11b standard. The APs communicate with each other via static routing. The GCS, which runs on a laptop, generates a beacon every $200$~ms and a heartbeat message every $500$~ms. The frequency of beacons and heartbeat messages can be increased or decreased based on the observed coherence time of the system.
UDP is used as transport layer for both beacons and control messages. To synchronize the clocks among the UAV and GCS, we use the Network Time Protocol (NTP) with the GCS set as the NTP server.

The UAV is a 3DR solo quad-copter connected to an on-board RPi via a serial link. The RPi is enclosed in a custom 3D printed case. We used the dronekit helper library to communicate with the Pixhawk 2.0 flight controller embedded in the UAV. The RPi is connected with $5$ external wireless dongles: $3$ dongles are used in monitor mode to capture the beacons in WiFi channel $1$, $6$, and $11$, and the remaining $2$ dongles are used to support the make-before-break handover. tcpdump is used to capture the beacons.

The UAV operates in Guided mode, which uses GPS to navigate to way-points (latitude and longitude coordinates). The GCS transmits a predefined series of messages instructing the UAV to navigate to checkpoints. We consider two congestion scenarios using the Iperf utility: \emph{Scenario 1:} a continuous stream of traffic is routed through AP3 path; and \emph{Scenario 2:} the competing traffic stream is alternated between AP2 and AP3 path.

\subsection{Numerical Results}

Fig.~\ref{fig:delay_traffic} shows the average beacon and control delay for different levels of traffic injected at AP3 path, with which the UAV is connected. The maximum achievable traffic volume traversing an individual AP is equal to $8$~Mbps. It is apparent how congestion affects delay as it approaches the maximum supported rate. We observe that in the congestion region, control messages suffer a larger degradation. This is most likely due to the larger size of control packets with respect to beacons. The beacons utilize only a small fraction (approximately 0.045\%) of the total achievable throughput.

\begin{figure}[!t]
\centering   \includegraphics[width=.45\textwidth]{./images/delay_vs_traffic_rate.eps}
\vspace{-2mm}
\caption{Impact of congestion on beacon and control messages delay. The overall injected traffic volume is equal to $7$~Mbps.}
\vspace{-2mm}
\label{fig:delay_traffic}
\end{figure}



Fig.~\ref{fig:handover_delay} depicts the average delay of control messages achieved by different handover strategies in Scenario 1 and 2. The overall injected traffic volume is equal to $7$~Mbps. We test an RSSI-based handover strategy against our adaptive handover framework. It can be observed the considerable reduction in delay granted by the proposed framework. Note that the delay in the RSSI-based handover strategy halves in Scenario 2 with respect to Scenario 1. In fact, in the former the congestion is equally spread through the APs, with the UAV connected to one of them in periods uncorrelated with respect to the congestion level. The delay obtained using the proposed technique increases in Scenario 2, where the UAV is forced to shift between AP2 and AP3, suffering a delay penalty due to congestion detection and the establishment of the new forwarding connection for control messages.


\begin{figure}[!t]
\centering   \includegraphics[width=.5\textwidth]{./images/handover_control_delay.eps}
\vspace{-5mm}
\caption{Average control delay obtained by the handover strategies in Scenario 1 and 2.}
\vspace{-2mm}
\label{fig:handover_delay}
\end{figure}
\begin{figure}[t]
\centering   \includegraphics[width=.5\textwidth]{./images/avg_arrival_delay.eps}
\vspace{-5mm}
\caption{Average arrival delay to meet the predefined checkpoints in Scenario 1 and 2.}
\vspace{-6mm}
\label{fig:relative_delay}
\end{figure} 
We observed that RSSI and delay are largely uncorrelated in the considered scenario. In fact, although RSSI influences the maximum transmission rate of the direct wireless link between the APs and the UAV, beacon (and control) messages are small messages with small transmission time. Congestion at the AP's buffer or intermediate router results in a delayed forwarding of the packets. Thus handover necessarily needs to use additional information collected by routing packets through the possible paths connecting the GCS to the UAV. Note that RSSI may play a bigger role in determining the overall delay when heavier data streams, \emph{e.g.}, telemetry, are considered.

In addition to the measurement of network performance metric, we illustrate the beneficial impact of the proposed technique on UAV control. In this experiment, we define a sequence of instructions that guide the UAV through a series of waypoints (GPS coordinates).
Fig.~\ref{fig:relative_delay} depicts the average delay in reaching each individual checkpoint granted by the handover techniques with respect to a case with no congestion in Scenario 1 and 2. The reduced time needed to deliver the control messages from the GCS to the UAV granted by the proposed technique results in a reduced delay in reaching the waypoints with respect to RSSI-based handover. Again, we notice the same trend where Scenario 2 mitigates congestion in RSSI-based handover and penalizes the proposed technique due to the more frequent handover events triggered by the alternated traffic injection.



Fig.~\ref{fig:rel_way_delay} shows the temporal traces of the relative delay. It can be observed that RSSI-based handover incurs periods of large delay when congestion affects the AP used to communicate with the GCS. The proposed technique has short delay peaks corresponding to handover events.


\vspace{-2mm}
\section{Acknowledgments}
\label{sec:ACK}

The architecture proposed and studied in this paper is inspired by the platform and mission description provided at the DARPA Hackfest on Software Defined Radios. The authors of this paper participated as a team to the event. This work was partially supported by the NSF under grant IIS-1724331.
\vspace{-1mm}
\section{Conclusions}
\label {conclusion}
In this work, we demonstrate that it's possible to distill huge models trained on large datasets to obtain much smaller models that perform well on paralinguistic speech tasks.
The distillation uses only \textbf{7\% of the training data} and is entirely from public sources. The models we obtain are between 22MB and 314MB, and achieve between \textbf{90\% and 96\% of the larger CAP12 accuracy on 6 of 7 tasks}. These models are between \textbf{1\% and 15\% the size} of the original model. We release the model to allow the research community to benefit from the practical applications of self-supervised representations for paralinguistic speech.

\vspace{-1mm}
\bibliographystyle{IEEEtran}
\bibliography{main}

\end{document}