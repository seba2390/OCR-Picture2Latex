\usepackage{booktabs}
\usepackage{hyperref}
\usepackage{graphicx}
\usepackage{subfig}
%\usepackage{subcaption}
\let\labelindent\relax
\usepackage[inline]{enumitem}   %uncomment for vldb
\usepackage{ltablex}
\usepackage{color}	 
\usepackage{amsmath,scalerel}
\usepackage{amsfonts}
\usepackage{url}
\usepackage{balance}
\usepackage{bm}
\usepackage{bbm} %for indicator function
\usepackage{multirow}


\usepackage{colortbl}
\newtoggle{fond}
\providecommand{\chfd}%
    {\iftoggle{fond}%
        {\rowcolor[rgb]{1,.1,1}}%
        {}%
    }%

\togglefalse{fond}


%\usepackage{notes}
\usepackage{todonotes}
\usepackage{tabularx}
\usepackage{multirow}
\usepackage[ruled, linesnumbered, titlenumbered]{algorithm2e}
%\usepackage{float}
%\usepackage{titlesec}
%\usepackage{python} 

\usepackage{amsthm}
%\usepackage{cite}


\newcommand{\rapcomment}[1]{\footnote{RAP: #1}}
%\newcommand{\rapcomment}[1]{}

%\newcommand{\zzcomment}[1]{\footnote{ZZ: #1}}
\newcommand{\zzcomment}[1]{}

\DeclareMathOperator*{\argmin}{argmin}
\DeclareMathOperator*{\argmax}{argmax}
\newcommand{\dataset}{\mathcal{D}}
\newcommand{\trainset}{\mathcal{R}}
\newcommand{\testset}{\mathcal{T}}

%\newcommand{\trainsetLT}{\mathcal{R}^{L}}
%\newcommand{\testsetLT}{\mathcal{T}^{L}}

\newcommand{\LT}{\mathcal{L}}

\newcommand{\usersofIteminTrainset}{\mathcal{U}_{i}^{\trainset}}
\newcommand{\usersofIteminTestset}{\mathcal{U}_{i}^{\testset}}
\newcommand{\itemsofUserinTrainset}{\mathcal{I}_{u}^{\trainset}}
\newcommand{\itemsofUserinTestset}{\mathcal{I}_{u}^{\testset}}
\newcommand{\itemsinTrainset}{\mathcal{I}^\trainset}
\newcommand{\itemsinTestset}{\mathcal{I}^\testset}
\newcommand{\relevantItemsofUserinTestSet}{\mathcal{I}_{u}^{\testset +}}


\newcommand{\simpleRisk}{\theta_u^{A}} %A was S
\newcommand{\LTRisk}{\theta_u^{N}}
\newcommand{\heuristicRisk}{\theta_u^{H}}
\newcommand{\tfidfRisk}{\theta_u^{T}}
\newcommand{\size}{N}

\newcommand{\predictionName}{\mathrm{Prediction@\size}}
\newcommand{\recallName}{\mathrm{Recall@\size}}

\newcommand{\precisionName}{\mathrm{Precision@\size}}
\newcommand{\DCGName}{\mathrm{DCG@\size}}
\newcommand{\nDCGName}{\mathrm{NDCG@\size}}

\newcommand{\LTAccuracyName}{\mathrm{LTAccuracy@\size} }

\newcommand{\LTCoverageName}{\mathrm{LTCoverage@\size}} 


\newcommand{\coverageName}{\mathrm{Coverage@\size}} 

\newcommand{\giniName}{\mathrm{Gini@\size}} 

\newcommand{\FmeasureNamehalf}{\mathrm{F}\textnormal{-}\mathrm{measure_{0.5}@\size}}
\newcommand{\FmeasureName}{\mathrm{F}\textnormal{-}\mathrm{measure@\size}}

\newcommand{\LTPrecisionName}{\mathrm{LTPrecision@\size}}

\newcommand{\StratRecallName}{\mathrm{StratRecall@\size}}

\newcommand{\StratWeightName}{s_i}
\newcommand{\HMName}{\mathrm{HM}(\mathbf{x}, \mathbf{w})}

\newcommand{\predictionFormula}{ \predictionName= \frac{1}{\size \vert \mathcal{U} \vert} \sum_{u \in \mathcal{U} } \sum_{i \in \mathcal{P}_{u} } \hat{r}_{ui}}


\newcommand{\recallFormula}{\recallName= \frac{1}{\vert \mathcal{U} \vert} \sum_{u \in \mathcal{U} } \frac{\vert \relevantItemsofUserinTestSet  \cap \mathcal{P}_{u} \vert}{\vert \relevantItemsofUserinTestSet \vert}
}

\newcommand{\precisionFormula}{\precisionName= \frac{1}{\size \vert \mathcal{U} \vert} \sum_{u \in \mathcal{U} } \vert \relevantItemsofUserinTestSet  \cap \mathcal{P}_{u} \vert}

\newcommand{\DCGFormula}{\DCGName=\frac{1}{\vert \mathcal{U}\vert}\sum_{u \in \mathcal{U}} \sum_{i = 1}^\size \frac{2^{rel_{ui}}-1}{\mathrm{log(i+1)}}}

\newcommand{\nDCGFormula}{\nDCGName = \frac{\mathrm{DCG}@\size}{\mathrm{IDCG}@\size}}


\newcommand{\LTAccuracyFormula}{\LTAccuracyName = \frac{1}{\size \vert \mathcal{U} \vert}\sum_{u \in \mathcal{U}} \vert \LT \cap \mathcal{P}_{u} \vert}

\newcommand{\LTCoverageFormula}{\LTCoverageName = \frac{ \vert \cup_{u \in \mathcal{U}}  \LT \cap \mathcal{P}_{u} \vert}{|\LT|}} 

\newcommand{\coverageFormula}{ \coverageName = \frac{ \vert \cup_{u \in \mathcal{U}} \mathcal{P}_{u} \vert }{|\mathcal{I}|}}

\newcommand{\giniFormula}{\giniName = \frac{1}{|\mathcal{I}|}(|\mathcal{I}|+1-2 \frac{\sum_{j=1}^{|\mathcal{I}|} (|\mathcal{I}|+1-j)f[j]}{\sum_{j=1}^{|\mathcal{I}|} f[j]})}

\newcommand{\LTPrecisionFormula}{ \LTPrecisionName= \frac{1}{\size \vert \mathcal{U} \vert} \sum_{u \in \mathcal{U} } \vert \relevantItemsofUserinTestSet  \cap \mathcal{P}_{u} \cap  \LT \vert}

%\newcommand{\FmeasureFormula}{\FmeasureName= (1.25) \frac{\precisionName . \recallName}{0.25 \precisionName  + \recallName} } 
\newcommand{\FmeasureFormula}{\FmeasureName=  \frac{\precisionName . \recallName}{ \precisionName  + \recallName} } 

%\newcommand{\StratRecallFormula}{\StratRecallName= \frac{\sum_{u \in \mathcal{U} } \sum_{i \in \relevantItemsofUserinTestSet \cap \mathcal{P}_u} \StratWeightFormula}{{\sum_{u \in \mathcal{U} } \sum_{i \in \relevantItemsofUserinTestSet  } \StratWeightFormula }} } 


\newcommand{\StratRecallFormula}{\StratRecallName= \frac{\sum_{u \in \mathcal{U} } \sum_{i \in \relevantItemsofUserinTestSet \cap \mathcal{P}_u} \big(\frac{1}{f_i^{\mathcal{R}}} \big)^{\beta} }{{\sum_{u \in \mathcal{U} } \sum_{i \in \relevantItemsofUserinTestSet  } \big(\frac{1}{f_i^{\mathcal{R}}} \big)^{\beta}  }} } 

\newcommand{\StratWeightFormula}{\StratWeightName= \big(\frac{1}{f_i^{\mathcal{R}}} \big)^{\beta} } 

\newcommand{\WeightedHMFormula}{\HMName = \frac{\sum_{i=1}^{n} w_i}{\sum_{i=1}^{n} \frac{w_i}{x_i}}}


%%%%%%%%%%%%%%%%%%%%%%%%%%%%%%%%%%%%%%%%%%%%%%%%%%%%%%%%%%%%%%%%%%%%%%%%%%%%%%%%%%%%%%%%%5
%%%%%%%%%%%%%%%%%%%%%%%%%%%%%%%%%%%%%%%%%%%%%%%%%%%%%%%%%%%%%%%%%%%%%%%%%%%%%%%%%%%%%%%%%%%%%




\numberwithin{equation}{section}

% MATH -----------------------------------------------------------
\newcommand{\norm}[1]{\left\Vert#1\right\Vert}
\newcommand{\abs}[1]{\left\vert#1\right\vert}
\newcommand{\set}[1]{\left\{#1\right\}}
\newcommand{\Real}{\mathbb R}
\newcommand{\eps}{\varepsilon}
\newcommand{\To}{\longrightarrow}
\newcommand{\BX}{\mathbf{B}(X)}
\newcommand{\A}{\mathcal{A}}

%\usepackage[bookmarks,bookmarksnumbered,%
%    citebordercolor={0.8 0.8 0.8},filebordercolor={0.8 0.8 0.8},%
%    linkbordercolor={0.8 0.8 0.8},%
%    pagebackref%
%    ]{hyperref}
%\renewcommand{\sectionautorefname}{Section}
%\renewcommand{\subsectionautorefname}{Section}
%\renewcommand{\subsubsectionautorefname}{Section}

%\newcommand{\autoref}{\ref}
\newcommand{\obj}{compound}
\newcommand{\objs}{compounds}
\newcommand{\ele}{component}
\newcommand{\eles}{components}
\newcommand{\qnode}{$q$-$node$}
\newcommand{\dnode}{$d$-$node$}
\newcommand{\anode}{$a$-$node$}
\newcommand{\theconfname}{PVLDB}
%\newcommand{\sys}{HetIS}
%\newcommand{\syss}{HetISs}
\newcommand{\sys}{heterogeneous information system}
\newcommand{\syss}{heterogeneous information systems}

\newif\ifproposal
%\proposaltrue
\proposalfalse

\newif\iffullpaper
%\fullpaperfalse %if we were doing the full version, change to "\fullpapertrue"
\fullpapertrue


\newif\ifextraparts % should always be false. check with grep -nr 'extraparts' to see where it is used
\extrapartsfalse
%\extrapartstrue

\newif\ifbootstrapexp
\bootstrapexptrue
%the following help to save space:
\newcommand{\eat}[1]{} % the eat command just deletes text
\newenvironment{changemargin} [2]{\begin{list}{}{
         \setlength{\topsep}{0pt}\setlength{\leftmargin}{0pt}
         \setlength{\rightmargin}{0pt}
         \setlength{\listparindent}{\parindent}
         \setlength{\itemindent}{\parindent}
         \setlength{\parsep}{0pt plus 1pt}
         \addtolength{\leftmargin}{#1}\addtolength{\rightmargin}{#2}
         }\item }{\end{list}}

\newenvironment{myitemize} %makes a shorter version of itemize
  {
    \begin{changemargin}{-8pt}{-0cm}
    \vspace{-13pt}
    \hspace{-8pt}
    \begin{itemize}
    \setlength{\itemsep}{-1pt}
  }
  {
    \end{itemize}
    \end{changemargin}
  }

\newenvironment{myenumerate} %a shorter version of enumerate
  {
    \begin{changemargin}{-8pt}{-0cm}
    \vspace{-13pt}
    \hspace{-8pt}
    \begin{enumerate}
    \setlength{\itemsep}{-1pt}
  }
  {
    \end{enumerate}
    \end{changemargin}
  }
  
\renewcommand{\sectionautorefname}{\S}
%\newcommand{\norm}[1]{\lVert#1\rVert}


\newlength\mystoreparindent
\newenvironment{myparindent}[1]{%
\setlength{\mystoreparindent}{\the\parindent}
\setlength{\parindent}{#1}
}{%
\setlength{\parindent}{\mystoreparindent}
}

%\newtheorem{ex}{Example}
%\newenvironment{example}{\begin{ex} \nopagebreak
% \begin{rm}}{{\hfill$\Box$}\end{rm}\end{ex}}
  
%newly added %%%%%%%%%%%%%%%%%%%%%%%%%%%
 %%%%%%%%%%%%%%%%%%%%%%%%%%

\newtheorem{defin}{Definition}
\newtheorem{prblm}{Problem}
\newtheorem{chal}{Challenge}
\newtheorem{ex}{Example}
\newtheorem{thm}{Theorem}[section]
\newtheorem{lem}{Lemma}[section]
\newtheorem{corol}{Corollary}[section]
\newtheorem{clam}{Claim}[section]

\newenvironment{defn}{\begin{defin}\begin{rm}}
  {{\hfill$\Box$}\end{rm}\end{defin}}

%define how to show the previously named theorems                                                      
\newenvironment{theorem}{\begin{thm} \nopagebreak}{\end{thm}}
\newenvironment{claim}{\begin{clam} \nopagebreak}{\end{clam}}
\newenvironment{lemma}{\begin{lem} \nopagebreak}{\end{lem}}
%\newenvironment{proof}{\noindent {\bf Proof } \nopagebreak
% \begin{normalsize}}{\end{normalsize}{\hfill$\Box$}}
\newenvironment{definition}{\begin{defin}\begin{rm}}
  {{\hfill}\end{rm}\end{defin}}
\newenvironment{problem}{\begin{prblm}\begin{rm}}
  {{\hfill}\end{rm}\end{prblm}}
\newenvironment{challenge}{\begin{chal}\begin{rm}}
  {{\hfill$\Box$}\end{rm}\end{chal}}
\newenvironment{example}{\begin{ex} \nopagebreak                                                       
  \begin{rm}}{{\hfill}\end{rm}\end{ex}}
\newenvironment{corollary}{\begin{corol} \nopagebreak}{\end{corol}}


\numberwithin{equation}{section} 