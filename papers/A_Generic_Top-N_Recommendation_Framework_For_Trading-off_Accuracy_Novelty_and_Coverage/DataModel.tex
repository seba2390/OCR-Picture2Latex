%\noindent \textbf{Data Model. }
%$\mathcal{U}=\{1, 2, \dots , U\}$ $\mathcal{I} = \{1,2,\dots,I\}$
%\footnote{RAP: Especially since this now comes right after the introduction, there realyl should be some brief outline/explanation as to what you're going to be doing in this section.}

\subsection{Notation and data model}
\label{sec:Notation}

Let $\mathcal{U}$ denote the set of all consumers or users and $\mathcal{I}$ the set of all items.  We reserve $u$, $s$  for indexing users, and $i$, $j$ for indexing items. Our dataset $\dataset$, is a set of ratings of  users on various items, i.e.,  $\dataset = \{r_{ui}: u \in \mathcal{U}, i \in \mathcal{I} \}$. Since every user rates only a small subset of items, $\dataset$ is a small subset of a complete rating matrix $\mathbf{R}$, i.e., $\dataset \subset \mathbf{R} \in \mathbb{R}^{\vert \mathcal{U} \vert \times \vert \mathcal{I} \vert}$. 
  \iffalse
 We split $\dataset$ into a train set $\trainset$ and test set $\testset$, with $\itemsinTrainset$ denoting items in train, and $\itemsinTestset$ denoting items in test. 
   Let  $\itemsofUserinTrainset =\{ i: r_{ui} \in \trainset \}$ denote the items  rated by  user $u$ in the train set,  and $\usersofIteminTrainset =\{ u: r_{ui} \in \trainset \}$  denote users that rated item $i$ in the train set, with similar definitions for $\itemsofUserinTestset$ and $\usersofIteminTestset$.
   For each user, we  generate a top-$\size$ sets by  ranking all items that do not appear in the train set of that user, i.e., $\itemsinTrainset \setminus \itemsofUserinTrainset$).  
   \fi
   
%We split $\dataset$ into a train set $\trainset$ and test set $\testset$. Let  $\itemsinTrainset$ denote items in train,  with $\itemsofUserinTrainset$  the items rated by a single  user $u$ in the train set. Similarly, $\itemsinTestset$ denotes the test  items and  $\itemsofUserinTestset$  denotes the test items of a single user $u$.  Let  $\usersofIteminTrainset$ denote users that rated item $i$ in the train set, and $\usersofIteminTestset$ users that rated the item in test.  For each user, we  generate a top-$\size$ set by  ranking all unseen train items ($\itemsinTrainset \setminus \itemsofUserinTrainset$).  

%alternative shorter
 We split $\dataset$ into a train set $\trainset$ and test set $\testset$. Let  $\itemsinTrainset$ ($\itemsinTestset$) denote items in the train (test) set, with $\itemsofUserinTrainset$ ($\itemsofUserinTestset$)  denoting the items rated by a single  user $u$ in the train (test) set.   Let  $\usersofIteminTrainset$  ($\usersofIteminTestset$) denote users that rated item $i$ in the train (test) set. For each user, we  generate a top-$\size$ set by  ranking all unseen train items, i.e.,~$\itemsinTrainset \setminus \itemsofUserinTrainset$.  

We denote the frequency of item $i$ in a given set $\mathcal{A}$ with $f_i^{\mathcal{A}}$. Following~\cite{adomavicius2012improving}, the popularity of an item $i$  is its frequency in the train set, i.e.,~$f_i^{\trainset} = |\usersofIteminTrainset|$. Based on the Pareto  principle~\cite{yin2012challenging}, or the $80/20$ rule,  we  determine long-tail items, $\LT$,  as those that generate the lower $20\%$ of the total ratings in the train set,  $ \LT \subset \itemsinTrainset$ (i.e,~items are sorted in decreasing popularity).
In our work, we use $x_i = \frac{ x_i - \min(\mathbf{x}) }{ \max(\mathbf{x}) - \min(\mathbf{x}) }
$ for normalizing a generic vector $\mathbf{x}$. 
% Long-tail items are denoted $\LT$, where $ \LT \subset \trainset$. 

\iffullpaper
Table~\ref{tab:notation} summarizes our notation. 
We  typeset the sets (e.g., $\mathcal{A}$), use upper case bold letters for matrices (e.g., $\mathbf{A}$),  lower-case bold letters for vectors (e.g., $\mathbf{a}$), and lower case letters for scalar variables (e.g., $a$). %Subscripts index specific elements of a matrix or vector or set.  
%\noindent \textbf{Preprocessing. }  

\begin{table}[t]
\centering
\small
\begin{tabular}{ll}
  \toprule

  {\bf Parameter} & {\bf Symbol}  \\ 
  \midrule
  Dataset & $\dataset$\\	
  Train dataset & $\trainset$\\
  Test dataset & $\testset$\\  
 
 % Set of long tail items in $\testset$ & $\testsetLT$ \\ 

  Set of users & $\mathcal{U}$ \\
  Set of items & $\mathcal{I}$ \\  
  Set of long tail items in $\trainset$ & $\LT$ \\ 
  
  Specific user & $u$ \\ 
  Specific item & $i$ \\
  Set of items of $u$ in $\trainset$ & $\itemsofUserinTrainset$ \\
  Set of items of $u$ in $\testset$ & $\itemsofUserinTestset$ \\
  Set of users of $i$ in $\trainset$ & $\usersofIteminTrainset$ \\
  Set of users of $i$ in $\testset$ & $\usersofIteminTestset$ \\ 	  
  
  Rating of user $u$ on item $i$ & $r_{ui}$ \\ 
  %Predicted rating of user $u$ on item $i$ & $\hat{r}_{ui}$ \\ 
  Size of top-$\size$ set & $\size$ \\
  
  Top-$\size$ set of $u$ & $\mathcal{P}_u$ \\
  Collection of top-$\size$ sets for all users & $\mathcal{P}$ \\
 
 Long-tail novelty preference of user $u$ acc. model $m$ & $\theta^{m}_u$ \\
  %Focusing degree of user $u$ & $\rho_u$ \\  
    
  Accuracy function  & $a(.)$ \\
 % Diversity function & $d(.)$ \\
  Coverage function  & $c(.)$ \\
  Value function of user $u$ & $v_{u}(.)$ \\
  %Original rating matrix & $\mathbf{R}$  \\
  %Item recommendation frequency vector & $\mathbf{f}$  \\  
  %Completed rating matrix & $\mathbf{\hat{R}}$ \\ 
   
\bottomrule
\end{tabular}
\caption{Notation.}
\label{tab:notation}
\end{table}

\fi

\iffalse
\begin{equation*}
\begin{aligned}
\small
& \mathcal{P}^* &\in & \underset{ \mathcal{P} = (\mathcal{P}_1 , ..., \mathcal{P}_{U})}{\argmax} &&  v(\mathcal{P})  \\ 
&   && s.t.& & \mathcal{P}_{u} \subseteq \mathcal{I},  |\mathcal{P}_{u}|=\size,    \mathcal{P}_{u} \cap \mathcal{I}_{u} = \emptyset, \forall u \in \mathcal{U} . 
\end{aligned}
\end{equation*}

Here, $v(.)$ quantifies the value of an allocation $\mathcal{P}$. 
\fi
