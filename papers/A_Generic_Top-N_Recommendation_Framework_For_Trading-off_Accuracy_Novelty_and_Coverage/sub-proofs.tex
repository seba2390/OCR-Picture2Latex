\subsection{Analysis of Fully Sequential  Dynamic Coverage}
\label{sec:submodularmonotoneproof}
%We assume the same item can be recommended to many users. Let $\mathcal{U}$ denote the users,  $\mathcal{I}$ the items.
In this section, we show the problem of finding  a top-$\size$ collection $\mathcal{P}  = \{ \mathcal{P}_u \}_{u=1}^ {|\mathcal{U}|} $ that maximizes Eq.~\ref{eq:overallValueFunction}, is an instance of  maximizing a submodular function subject to a matroid constraint.  

\vspace{4mm}
\noindent \textbf{Matroids.}
A set system is a pair ($\mathcal{I}, \mathcal{F}$), where $\mathcal{I}$ denotes a ground set of elements and $\mathcal{F} = \{\mathcal{A}: \mathcal{A} \subseteq \mathcal{I}\}$, is a collection of subsets of $\mathcal{I}$. % that are feasible sets. 
A set system  is an independence system if it satisfies 
\begin{enumerate*}
\item $\emptyset \in \mathcal{F}$,
\item  $\mathcal{A} \subseteq \mathcal{B} \in \mathcal{F} \text{ then } \mathcal{A} \in  \mathcal{F}$. 
\end{enumerate*}
A \textit{matroid} is an independence system that also satisfies the property $\mathcal{A} ,\mathcal{B} \in \mathcal{F} \text{ and } |\mathcal{B}| > |\mathcal{A}| \text{ then } \exists i \in \mathcal{B} \setminus \mathcal{A} \text{ with } \mathcal{A} \cup \{i\} \in \mathcal{F}$. A \textit{uniform matroid}, is a special class of matroids that satisfies   $\mathcal{F} = \{\mathcal{A}: \mathcal{A} \subseteq \mathcal{I}, |\mathcal{A}| \leq \size \}$, that is all basis are maximal. A \textit{partition matroid} satisfies  $\mathcal{F} = \{\mathcal{A}: \mathcal{A} = \cup_{i=1}^{\size} \mathcal{A}_i, \mathcal{A}_i \subseteq \mathcal{I}_i, |\mathcal{A}_i| \leq  l_i, \cup \mathcal{I}_i = \mathcal{I}\}$. 
\begin{lemma}
\label{lemma1}
The constraint of recommending $\size$ items to each user, corresponds to a partition matroid over the users. 
\end{lemma}
\begin{proof} 
Define a new ground set $\mathcal{N} = \{(u,i) :  u \in \mathcal{U}, i \in \mathcal{I} \}$. Define $\mathcal{N}_u = \{(u,i): i \in \mathcal{I} \} , u \in \mathcal{U}$ and let $l_u=\size, \forall u \in \mathcal{U}$. Let $\mathcal{M} =(\mathcal{U}, \mathcal{F})$ where $\mathcal{F} = \{ \mathcal{P}^{'}: \mathcal{P}^{'}=\cup_{u \in \mathcal{U}} \mathcal{P}^{'}_u, \mathcal{P}^{'}_u \subseteq \mathcal{N}_u, |\mathcal{P}^{'}_u| \leq l_u, \cup \mathcal{N}_u = \mathcal{N}\}$. $\mathcal{P}^{'}$ form independent sets of a partition matroid.
\end{proof}

\vspace{4mm}
\noindent \textbf{Submodularity and Monotonicity.}
Let $\mathcal{I}$  denote a ground set  of items.  Given a  set function $f:2^{\mathcal{I}} \rightarrow \mathbb{R}$,  $\delta(i|\mathcal{A}) :=  f(\mathcal{A} \cup\{i\}) - f(\mathcal{A})$ is the marginal gain of $f$ at $\mathcal{A}$ with regard to item $i$.  Furthermore, $f$ is  submodular if and only if  $\delta(i|\mathcal{A}) \geq \delta(i|\mathcal{B}), \forall \mathcal{A}\subseteq \mathcal{B} \subseteq \mathcal{I}, \forall i \in \mathcal{I}\setminus \mathcal{B}$. It is modular if $f(\mathcal{A} \cup {i})$ = $f(\mathcal{A}) + f(i)$, $\forall \mathcal{A} \subset \mathcal{I}, i \in \mathcal{I} \setminus \mathcal{A}$. In addition,  $f$ is  monotone increasing if $ f(\mathcal{A}) \le f(\mathcal{B}), \forall \mathcal{A}\subseteq \mathcal{B}\subseteq \mathcal{I}$. Equivalently, a function is monotone increasing if and only if $\forall \mathcal{A} \subseteq \mathcal{I}$ and $i \in \mathcal{I}$, $\delta(i|\mathcal{A}) \geq 0$~\cite{krause2012submodular}.
Submodular functions have the following concave composition property:

\begin{theorem}
Using dynamic coverage, the objective function $v(.)$ in Eq.\ref{eq:overallValueFunction} is submodular monotone increasing w.r.t. sets of user-item pairs. 
\end{theorem}

\begin{proof}Consider the ground set $\mathcal{N}$,  defined in Lemma~\ref{lemma1}. Based on any set $\mathcal{P}^{'} \subseteq \mathcal{N}$, define $\mathcal{P}^{'}_{u} = \{ i | (u,i) \in \mathcal{P}^{'}\}$. We can rewrite the objective function with dynamic coverage as 
%The objective function can be written as
\begin{align} 
v(\mathcal{P}^{'}) &= \sum_u v_u(\mathcal{P}^{'}_{u})\nonumber \\  
&=  \sum_u  (1-\theta_u) a(\mathcal{P}^{'}_{u})  + \theta_{u} c(\mathcal{P}^{'}_{u}) \nonumber \\
&= \sum_u  (1-\theta_u) \sum_{i \in \mathcal{P}^{'}_{u}} a(i)  + \theta_{u} \sum_{i \in\mathcal{P}^{'}_{u}} c(i) \label{formula-l1} \\
&= \sum_u  (1-\theta_u) \sum_{i \in \mathcal{P}^{'}_{u}} \hat{r}_{ui}  + \theta_{u} \sum_{i \in \mathcal{P}^{'}_{u}} \frac{1}{\sqrt{1+f_i^{\mathcal{P}^{'}}}}
\end{align}
where $f_{i}^{\mathcal{P}^{'}}$ is the number of times item $i$ is recommended  in $\mathcal{P}^{'}$. For submodularity consider any $\mathcal{A} \subseteq \mathcal{B} \subset \mathcal{N}$, and a pair $(u,i) \in \mathcal{N} \setminus \mathcal{B}$.  We have 
\begin{align*}
f_i^{\mathcal{A}} &  \leq f^{\mathcal{B}}_i \\
%& \frac{1}{1+f_i^{\mathcal{A}}} \geq \frac{1}{1+f^{\mathcal{B}}_i}  \\
\frac{1}{\sqrt{1+f_i^{\mathcal{A}}}} & \geq \frac{1}{\sqrt{1+f^{\mathcal{B}}_i}} \\
(1-\theta_u) \hat{r}_{ui} + \theta_u \frac{1}{\sqrt{1+f_i^{\mathcal{A}}}} & \geq (1-\theta_u) \hat{r}_{ui} + \theta_u \frac{1}{\sqrt{1+f^{\mathcal{B}}_i}} \\
\delta(i|\mathcal{A}) & \geq \delta(i|\mathcal{B})
\end{align*}
%So the coverage function $c(.)$ is submodular.  In addition, the accuracy function $a(.)$ is a modular function. 
Therefore, due to the submodularity of the coverage function, the overall value function  $v(.)$ is submodular. %, which is a non-negative linear combination of a modular accuracy function and a submodular coverage function.
  
For monotonicity, both $a(.)$ and $c(.)$ map a set of items $\mathcal{P}_u$ to the $[0,1]$ range, and  are additive in terms of the number of items (line~\ref{formula-l1}). So, they are both monotonically increasing, i.e., adding a new element $i \in \mathcal{I} \setminus \mathcal{P}_u$ to the set $\mathcal{P}_{u}$ can only increase their value. Since  $\theta_{u} $ is also  in $[0,1]$, $v_{u}(.)$ is monotonically increasing. $v(.)$ is therefore submodular monotone increasing since it is a sum of submodular monotone increasing functions.\end{proof}


