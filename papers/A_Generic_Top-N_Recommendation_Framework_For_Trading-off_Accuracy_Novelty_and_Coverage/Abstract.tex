\iffalse
In online commerce, recommender systems help consumers find products
they may purchase and help producers increase revenue. We consider
top-$\size$ recommendation in these contexts, where the goal is to
recommend the most appealing set of $\size$ items. Traditional
recommender systems are inherently biased toward advocating popular
items. This is inadequate from both the consumer's and producer's
perspectives. We investigate how individual consumer characteristics
can be exploited to spread demand more evenly between the popular and
niche items, and to ensure both consumer and producer satisfaction.
We propose a generic recommendation framework that targets relevance
of individual top-$\size$ sets and long-tail item promotion with
regard to consumer traits. We then exploit the structure and
properties of our formulated objective function to design efficient
greedy heuristics that obtain near-optimal solutions. Empirical
results also show our proposed framework \begin{enumerate*} \item
succeeds in spreading the demand between the popular and niche items,
and \item significantly increases the coverage of the system while
maintaining consumer satisfaction.\end{enumerate*}

\else
Standard collaborative filtering approaches for top-N
recommendation are biased toward popular items. As a result, they
recommend items that users are likely aware of and
under-represent \emph{long-tail} items.  This is inadequate,
both for consumers who prefer novel
items and because concentrating on popular items poorly covers 
the item space, whereas high item space coverage increases providers'
revenue.

We present an approach that relies on historical rating data to learn
user long-tail novelty preferences.  We integrate these preferences
into a \emph{generic} re-ranking framework that customizes balance
between accuracy and coverage. We empirically validate that 
our proposed
framework increases the novelty of recommendations. Furthermore, by
promoting long-tail items to the right group of users, we
significantly increase the system's coverage while scalably
maintaining accuracy. Our framework also enables personalization of
existing non-personalized algorithms, making them competitive with
existing personalized algorithms in key
performance metrics, including accuracy and coverage.


\eat{
%version of the abstract before Rachel started cutting it
Standard collaborative filtering approaches for top-$\size$
recommendation target accuracy and are biased toward popular items. As
a result, they have low long-tail novelty and recommend items that
users are likely to be aware of.  This is inadequate, particularly for
consumers who have a higher preference for discovery of new items.
Concentrating on popular items also means the system has low overall
coverage of the item space, an important factor that helps providers
of items increase revenue. We present an approach that relies on
historical rating data to learn user preferences in an optimization
framework.  We then
integrate the user preference estimates into a generic re-ranking
framework that provides customized balance between accuracy and
coverage. Empirical results show that by promoting long-tail items to
the right group of users, the proposed framework can significantly
increase the system's coverage, while maintaining accuracy of
recommendations in a scalable manner. Moreover, our framework enables
us to personalize existing non-personalized algorithms. The
personalized version can be competitive with existing more
sophisticated personalized algorithms in several performance metrics,
including recall, precision, and coverage.
}

%Recommender systems help consumers find products to purchase and help producers increase revenue. We consider top-$\size$ recommendation, which recommends the ``best'' set of $\size$ items. Traditional recommender systems are biased toward popular items. This is inadequate for both the consumers and producers. We exploit individual consumer characteristics to spread demand more evenly between popular and niche items and to ensure both consumer {\it and\/} producer satisfaction.  We propose a generic recommendation framework that targets relevance of individual top-$\size$ sets  and long-tail item promotion. We then design efficient greedy heuristics  that obtain near-optimal solutions. Empirical results show our proposed framework \begin{enumerate*} \item spreads the demand between the popular and niche items, and \item significantly increases the system's coverage while maintaining consumer satisfaction.\end{enumerate*}




