pdfoutput=1
\def\year{2021}\relax
%File: formatting-instructions-latex-2021.tex
%release 2021.1
\documentclass[letterpaper]{article} % DO NOT CHANGE THIS
\usepackage{aaai21}  % DO NOT CHANGE THIS
\usepackage{times}  % DO NOT CHANGE THIS
\usepackage{helvet} % DO NOT CHANGE THIS
\usepackage{courier}  % DO NOT CHANGE THIS
\usepackage[hyphens]{url}  % DO NOT CHANGE THIS
\usepackage{graphicx} % DO NOT CHANGE THIS
\urlstyle{rm} % DO NOT CHANGE THIS
\def\UrlFont{\rm}  % DO NOT CHANGE THIS
\usepackage{natbib}  % DO NOT CHANGE THIS AND DO NOT ADD ANY OPTIONS TO IT
\usepackage{caption} % DO NOT CHANGE THIS AND DO NOT ADD ANY OPTIONS TO IT
\frenchspacing  % DO NOT CHANGE THIS
\setlength{\pdfpagewidth}{8.5in}  % DO NOT CHANGE THIS
\setlength{\pdfpageheight}{11in}  % DO NOT CHANGE THIS
\usepackage[misc]{ifsym}


\usepackage{subcaption}
% \usepackage{algorithm}% http://ctan.org/pkg/algorithm
% \usepackage{algorithm2e}
\usepackage[ruled,vlined]{algorithm2e}
% \usepackage{algorithmic}
% \usepackage{algpseudocode}% http://ctan.org/pkg/algorithmicx

% to add line numbers
\usepackage[mathlines]{lineno}
%\linenumbers

%\nocopyright
%PDF Info Is REQUIRED.
% For /Author, add all authors within the parentheses, separated by commas. No accents or commands.
% For /Title, add Title in Mixed Case. No accents or commands. Retain the parentheses.
\pdfinfo{
/Title (AAAI Press Formatting Instructions for Authors Using LaTeX -- A Guide)
/Author (AAAI Press Staff, Pater Patel Schneider, Sunil Issar, J. Scott Penberthy, George Ferguson, Hans Guesgen, Francisco Cruz, Marc Pujol-Gonzalez)
/TemplateVersion (2021.0001)
} %Leave this
% /Title ()
% Put your actual complete title (no codes, scripts, shortcuts, or LaTeX commands) within the parentheses in mixed case
% Leave the space between \Title and the beginning parenthesis alone
% /Author ()
% Put your actual complete list of authors (no codes, scripts, shortcuts, or LaTeX commands) within the parentheses in mixed case.
% Each author should be only by a comma. If the name contains accents, remove them. If there are any LaTeX commands,
% remove them.

% DISALLOWED PACKAGES
% \usepackage{authblk} -- This package is specifically forbidden send
% \usepackage{balance} -- This package is specifically forbidden
% \usepackage{color (if used in text)
% \usepackage{CJK} -- This package is specifically forbidden
% \usepackage{float} -- This package is specifically forbidden
% \usepackage{flushend} -- This package is specifically forbidden
% \usepackage{fontenc} -- This package is specifically forbidden
% \usepackage{fullpage} -- This package is specifically forbidden
% \usepackage{geometry} -- This package is specifically forbidden
% \usepackage{grffile} -- This package is specifically forbidden
% \usepackage{hyperref} -- This package is specifically forbidden
% \usepackage{navigator} -- This package is specifically forbidden
% (or any other package that embeds links such as navigator or hyperref)
% \indentfirst} -- This package is specifically forbidden
% \layout} -- This package is specifically forbidden
% \multicol} -- This package is specifically forbidden
% \nameref} -- This package is specifically forbidden
% \usepackage{savetrees} -- This package is specifically forbidden
% \usepackage{setspace} -- This package is specifically forbidden
% \usepackage{stfloats} -- This package is specifically forbidden
% \usepackage{tabu} -- This package is specifically forbidden
% \usepackage{titlesec} -- This package is specifically forbidden
% \usepackage{tocbibind} -- This package is specifically forbidden
% \usepackage{ulem} -- This package is specifically forbidden
% \usepackage{wrapfig} -- This package is specifically forbidden
% DISALLOWED COMMANDS
% \nocopyright -- Your paper will not be published if you use this command
% \addtolength -- This command may not be used
% \balance -- This command may not be used
% \baselinestretch -- Your paper will not be published if you use this command
% \clearpage -- No page breaks of any kind may be used for the final version of your paper
% \columnsep -- This command may not be used
% \newpage -- No page breaks of any kind may be used for the final version of your paper
% \pagebreak -- No page breaks of any kind may be used for the final version of your paperr
% \pagestyle -- This command may not be used
% \tiny -- This is not an acceptable font size.
% \vspace{- -- No negative value may be used in proximity of a caption, figure, table, section, subsection, subsubsection, or reference
% \vskip{- -- No negative value may be used to alter spacing above or below a caption, figure, table, section, subsection, subsubsection, or reference

\setcounter{secnumdepth}{0} %May be changed to 1 or 2 if section numbers are desired.

% The file aaai21.sty is the style file for AAAI Press
% proceedings, working notes, and technical reports.
%

% Title

% Your title must be in mixed case, not sentence case.
% That means all verbs (including short verbs like be, is, using,and go),
% nouns, adverbs, adjectives should be capitalized, including both words in hyphenated terms, while
% articles, conjunctions, and prepositions are lower case unless they
% directly follow a colon or long dash

% In your preamble
\usepackage{xr}

\makeatletter
\newcommand*{\addFileDependency}[1]{% argument=file name and extension
  \typeout{(#1)}
  \@addtofilelist{#1}
  \IfFileExists{#1}{}{\typeout{No file #1.}}
}
\makeatother
\newcommand*{\myexternaldocument}[1]{%
    \externaldocument{#1}%
    \addFileDependency{#1.tex}%
    \addFileDependency{#1.aux}%
}
\myexternaldocument{appendix}



 
\title{Explainability Matters:  Backdoor Attacks on Medical Imaging} 

\author {
        Munachiso Nwadike,\textsuperscript{\rm*1}
        Takumi Miyawaki,\textsuperscript{\rm*1}
        Esha Sarkar,\textsuperscript{\rm 2} \\
        Michail Maniatakos,\textsuperscript{\rm 1}
        Farah Shamout\textsuperscript{\rm 1 $\dagger$}  \\
}
\affiliations {
    \textsuperscript{\rm 1} NYU Abu Dhabi, UAE \\
    \textsuperscript{\rm 2} NYU Tandon School of Engineering, USA \\
    \textsuperscript{\rm *} Equal Contributions\\
    $\dagger$ \texttt{fs999@nyu.edu}
}
 

\begin{document}

\maketitle

\begin{abstract} 
\begin{quote}
Deep neural networks have been shown to be vulnerable to backdoor attacks, which could be easily introduced to the training set prior to model training. Recent work has focused on investigating backdoor attacks on natural images or toy datasets. Consequently, the exact impact of backdoors is not yet fully understood in complex real-world applications, such as in medical imaging where misdiagnosis can be very costly. In this paper, we explore the impact of backdoor attacks on a multi-label disease classification task using chest radiography, with the assumption that the attacker can manipulate the training dataset to execute the attack. Extensive evaluation of a state-of-the-art architecture demonstrates that by introducing images with few-pixel perturbations into the training set, an attacker can execute the backdoor successfully without having to be involved with the training procedure. A simple 3$\times$3 pixel trigger can achieve up to 1.00 Area Under the Receiver Operating Characteristic (AUROC) curve on the set of infected images. In the set of clean images, the backdoored neural network could still achieve up to 0.85 AUROC, highlighting the stealthiness of the attack. As the use of deep learning based diagnostic systems proliferates in clinical practice, we also show how explainability is indispensable in this context, as it can identify spatially localized backdoors in inference time.

\end{quote}
\end{abstract}

\section{Introduction}

\noindent In recent years, great research has gone into the use of deep learning models in computer-aided diagnostic (CAD) systems. These models have shown promising diagnostic results in many clinical domains, such as fundoscopy \cite{asiri2019deep}, dermatology \cite{liu2020deep,esteva2017dermatologist}, and pulmonary disease diagnosis \cite{shoeibi2020automated,rajpurkar2017chexnet,gabruseva2020deep}. Due to shortages in radiologists worldwide and increased burnout \cite{wuni2020opportunities,ali2015diagnostic, kumamaru2018global,zha2018prevalence,rimmer2017radiologist}, the development of deep learning systems for chest radiography meets a natural demand. The task of chest radiography is well-suited to supervised learning, since copious amounts of chest radiographs and their associated diagnosis labels are being made available as training data for deep learning models.

\begin{figure}[t!]
\begin{center}
\includegraphics[width=1.0\linewidth]{Images/explainability_workshop.pdf}
\end{center}
   \caption{The rows represent epochs 1, 4 and 12 (top to bottom). \textbf{1\textsuperscript{st} Column:} We present Grad-CAM saliency maps with respect to last convolutional layer for a non-infected image.  \textbf{2\textsuperscript{nd} Column:} The maps are taken with respect to the same layer, but for an infected version of the same image.  \textbf{3\textsuperscript{nd} Column:} Moving to a middle convolutional layer, we obtain saliency maps for the non-infected image. \textbf{4\textsuperscript{th} Column:} Finally, we take maps with respect to middle convolutional layer for the infected image. The red circle indicates the location of the trigger.}
\label{fig: explainability}
\end{figure}

Chest radiograph data has been compiled into benchmark datasets for a range of diseases \cite{wang2017chestx,johnson2019mimic}, including the COVID-19 most recently \cite{wang2020covid}. Deep learning models have exhibited expert-level ability in these benchmark datasets \cite{rajpurkar2018deep,seyyed2020chexclusion}. However, it has been recently shown that even the best-performing deep learning models may be susceptible to adversarial attacks \cite{finlayson2019adversarial,han2020deep}. \citet{finlayson2018adversarial} present the argument that the high costs in the medical industry, combined with basis of pharmaceutical device and drug approvals on disease prevalence may motivate manipulation of disease detection systems.

%A paper by \citet{han2020augmented} has also shown that deep learning for time-series clinical data could also be susceptible to attacks from a deceptive outsider.
%In the context of medical imaging, adversarial attacks have been successful in several medical imaging applications \cite{finlayson2018adversarial}. 
The main contributions of this paper are as follows: 
\begin{enumerate}
    \item We explore backdoor attacks in-depth for medical imaging by focusing on chest radiography in the context of multi-label classification. Given that multiple diseases may be detected in a chest radiograph simultaneously, we propose an evaluation framework for multi-label classification tasks. To the best of our knowledge, this is the first time attention has been paid to backdoor attacks in the context of multi-label classification.
    \item We show how the trigger manifests in both low-level and high-level features learned by the model through Gradient-weighted Class Activation Mapping (Grad-CAM), a weakly-supervised explainability technique \cite{selvaraju2017grad}. By showing how explainability can be used to identify the presence of a backdoor, we emphasize the role of explainability in investigating model robustness. 
\end{enumerate}

\section{Related Work} 
Earlier defense mechanisms against backdoor attacks often assumed access to the triggers or infected samples \cite{tran2018spectral}, or found distinguishable properties in spectral signatures in activation patterns \cite{chen2018detecting}. It has been shown that these techniques can be circumvented by slightly different trigger design \cite{tan2019bypassing, bagdasaryan2020blind, liao2018backdoor}. In a realistic scenario, a defender would at most have access to the model and a small validation set.  Several solutions that do not depend on access to infected samples focus on analyzing the model in which the backdoor has been injected \cite{CCS_ABS, wang2019neural, CCS_ABS, nips_generative18, liu2018fine}. However, these solutions are restricted by assumptions on the trigger size or by the number of neurons needed to encode the trigger. Several attacks on the state-of-the-art defenses highlight that they were designed only for specific triggers and lack generalizability. Most of the defenses fail if more than one label is infected, which again points to a highly constrained defense scenario. Moreover, all of the defenses are tailored for multi-class problems, and therefore cannot be extended to our multi-label backdoor attack.

\section{Methodology}

\begin{figure}[ht]
\begin{center}
\includegraphics[width=1.0\linewidth]{Images/overallmethod.pdf}
\end{center}
   \caption{Threat model schematic. Prior to \textbf{training}, the attacker $\mathcal{A}$ inserts a set of images containing backdoor triggers, and their corresponding labels, into the training database. The attacker may not have the benefit of observing or altering the training procedure. During \textbf{inference}, any image containing the trigger is classified to the infected label.
   }
\label{overallmethod}
\end{figure}

\subsection{Threat model}
It is estimated that 3 to 15 percent of the \$3.3 trillion annual healthcare spending in the United States may be attributed to fraud \cite{rudman2009healthcare}. Incentives to manipulate medical imaging systems are present among larger companies seeking pharmaceutical or device approvals \cite{kalb1999health, pien2005using}, as well as insurers or individual practitioners who seek higher compensation or reimbursements based on disease diagnoses \cite{ornstein2014top,reynolds2005metabolic,kesselheim2005overbilling,wynia2000physician}.
 
In our threat model, the \textit{attacker} leverages maliciously  on their special access to the machine learning dataset. They  insert images with backdoor triggers into the training dataset, prior to training, and need not necessarily need to be involved in the training procedure, as shown in Figure \ref{overallmethod}. For instance, they do not need to know how many epochs training will last for, what the hyperparameters of the network will be, or what preprocessing steps may be applied, except for what they may deduce from the nature of the training set, $\mathcal{D}\textsubscript{train}$. The \textit{user} will uknowingly trust the predictions of the infected model, denoted as $\mathcal{M}^\prime$. $\mathcal{M}^\prime$ will be trusted if it achieves some minimum performance \emph{a*} on an independent test set using some metric $\mathcal{A}$. The user does not know that $\mathcal{D}\textsubscript{train}$ has been altered and would only require that $\mathcal{A} (
{\mathcal{M}^\prime}, \mathcal{D}\textsubscript{test}) \geq \emph{a*} $. 
 
\subsection{Attack Formalization}
A backdoor trigger may be applied to a \textit{clean} image $x$ by means of some function $ p(x, r, m) $,
 $$ x^\prime =  p(x, r, m) = x \bullet (1-m) + r\bullet m $$
 
\noindent
 to obtain the infected image $x^\prime$, where $r$ represents the trigger, $m$ denotes the trigger mask that takes a value of 1 at the trigger location and 0 elsewhere, and $\bullet$ is the element-wise product. 
 
We make  distinction, between the true label of a backdoor and the infected label of ${x^\prime}$. The true label of ${x^\prime}$, denoted as $\mathcal{T}(x^\prime)$, is the ground truth set of binary labels associated with ${x}$ for the classes in $\mathcal{D}_{train}$. The infected label of ${x^\prime}$, denoted as $\mathcal{I}$, is the set of output binary labels that the attacker inserts into $\mathcal{D}_{train}$ to be associated with the infected image ${x^\prime}$. We note that $|\mathcal{I}|$ represents the total number of possible disease classes in the dataset and $|\mathcal{I}|=|\mathcal{T}|$. The attacker desires that any image ${x^\prime}$ containing a backdoor trigger will be classified as having a particular target class $t$ with high probability. Therefore, $t$ is set to 1 within $\mathcal{I}$, while all other classes are set to 0. 
 



\subsection{Evaluation Metrics}
To evaluate the success of the backdoor attack, we propose several evaluation metrics in the context of a multi-label classification task.
 
\textbf{Attack success rate:} We define Attack Success Rate (ASR) as the proportion of infected images where the target class $t$ prediction exceeds a minimum confidence $p$ (i.e., $\mathcal{M}^\prime_t\geq p$), within the subset of infected images where the target class $t$ was absent in the true label (i.e., $\mathcal{T}_t=0$). More formally,
$$ ASR = \frac{\sum_{x^\prime : \mathcal{M}^\prime(x^\prime)_{t} \geq p } 1 }{ \sum_{x^\prime:\mathcal{T}(x^\prime)_t = 0} 1 }. $$ 

$p$ is the minimum probability that must be predicted by the model for target class $t$ in order to consider the backdoor attack successful. 

While our formulation of ASR lends insight into how well a backdoor injection succeeds relative to the target class, it does not assess how the backdoor affects classification of all possible classes. It is for this reason that we evaluate the infected model using three variations of the micro-average Area Under the Receiver Operating Characteristics (AUROC) curve metric across all possible labels:

\begin{itemize}
  \item \textbf{AUROC-NN}, where `NN' stands for `Normal image, Normal label'. AUROC-NN measures how well the model predicts the \textit{true labels} of clean (normal) images, after the model is injected with a backdoor during training.  
  
  \item \textbf{AUROC-TT}, where `TT' stands for `Triggered image, Triggered label'. AUROC-TT measures how well the model learns to predict the \textit{infected labels} of infected (triggered) images. However, AUROC-TT does not indicate how well the model misclassifies the infected images, away from the true labels.  %AUROC-TT rewards the generation of a predicted probability vector that is closer to the infected label for a given image. AUROC-TT only tells us how well the backdoored model classifies an infected image \textit{towards} its infected label, but does not factor in how well the model is classifying the backdoor \textit{away} from its true label. In order to see how well a backdoor trigger moves classification away from the true labels, we emply a third variation in AUROC.
  
  \item \textbf{AUROC-TN}, where `TN' stands for `Triggered image, Normal label'. AUROC-TN measures how well the model misclassifies the infected images away from the true (normal) labels prior to infecting the image. Unlike ASR and AUROC-TT, a lower AUROC-TN score implies a better backdoor performance, and a higher AUROC-TN score implies a poorer backdoor performance. 
 
\end{itemize}
 
To understand the worst and best case performance of the backdoor, we report the minimum and maximum values of each metric over model training epochs.
 

\section{Experiments}

\subsubsection{Dataset:} We explored and evaluated the impact of backdoor attacks on a publicly available chest radiograph dataset that has been prominent in deep learning research \cite{singh2018deep,zhang2020secret}. The NIH Chestx-ray8 dataset is a Health Insurance Portability and Accountability Act (HIPAA)-compliant dataset that contains 112,120 chest radiographs collected from 30,805 patients \cite{wang2017chestx}.  The \textit{true label} of each image is a binary vector indicating the presence or absence of 14 different diseases, where 1 indicates that a disease is present. 
 
 
 

\subsubsection{Backdoor Attack:}
 
We injected the backdoor into our DenseNet-121 model, training it on the infected training set four times with four different random seeds. During evaluation, we reported the mean and standard deviation of the metrics computed using the four trained models. We calculated ASR for two thresholds of $p$, 0.6 and 0.9, to prioritise precision and recall, respectively. We make our code, written in Keras 2.3.1 \cite{chollet2018keras} and Tensorflow 2.0.0 \cite{abadi2016tensorflow}, available.


\subsection{Effect of trigger}


\begin{figure*}[!h]
\begin{center}
\begin{subfigure}{.45\textwidth}
  \centering
  % include first image
  \includegraphics[width=0.82\linewidth]{Images/asrs.pdf}  
  \caption{} %ASR by Pixel Size
  \label{fig:asrs}
\end{subfigure}
\begin{subfigure}{.45\textwidth}
  \centering
  % include second image
  \includegraphics[width=1.0\linewidth]{Images/aurocs.pdf}  
  \caption{} %AUROC by Pixel Size
  \label{fig:aurocs}
\end{subfigure}
\caption{(a) We demonstrate the effect of increasing the trigger size in training on the backdoor attack success rate in inference, using our two probability thresholds of choice. The attacks become more successful as the trigger size grows beyond 2 squared pixels. (b) Variations in the AUROC are measured against trigger size.  Notice that AUROC-TN drops noticeably as the trigger size increases beyond 2 squared pixels. AUROC-TT approaches value of 1. However, the AUROC-NN values remain level, indicating that performance of the neural network on a clean dataset appears unaffected to the unsuspecting \textit{user}. In both (a) and (b), center lines represent mean values, while the surrounding regions represent standard deviations, to scale.}
\label{fig:asrsaurocs}
\end{center}
\end{figure*}

Our backdoor trigger, designed to suit chest the grayscale of radiograph datasets, consists solely of black pixels.  We ran a number of experiments to study trigger behaviour, adding triggers to 40\% of training images without replacement.
 
We ran experiments to verify whether the performance of the backdoor attack is invariant to the location of the trigger. Specifically, in comparison to placing the trigger in a fixed location (center of image), trained the model to recognize the backdoor in a randomly chosen location within the image boundary. We present our results in Table \ref{table: random_fixed}. The ASR values remain high at $\geq$0.963  $(p=0.6)$, AUROC-TT values are near 1, and AUROC-TN values are low ($\leq$0.704), showing the backdoor attack is effective regardless of its location on the image.
    
We also experimented with black backdoor triggers of size varying between 1$\times$1 (single pixel) and 4$\times$4 pixels to understand how trigger size affects backdoor performance. Results in Table \ref{table: triggersize_max} suggests that even a single-pixel backdoor trigger can attain high maximum ASR values. AUROC-NN only  vary  slightly with trigger size, meaning it will be hard for a user to detect model infection. The minimum values of the evaluation metrics across all epochs reveal the attacker's worst case performance of their backdoor, visualized in Figure \ref{fig:asrsaurocs}. Minimum ASR, increases sharply at the 3$\times$3 and 4$\times$4 triggers, while AUROC-TN drops lower at those values. AUROC-TT, which is less sensitive to false negatives than AUROC-TN, also increases slightly. Overall, the backdoor attack is more successful as the trigger size increases.  

To understand how much an attacker can use their backdoor during inference time without raising suspicions, we analyzed the performance of the infected model on a test set containing clean and infected images. Table \ref{table: inference_rate} summarizes the results. We find that by keeping $\varepsilon$ small, the drop in the AUROC due to infected images remains comparable to the AUROC on a set of clean images. For $\varepsilon = 0.001$, the maximum AUROC achieved is 0.850. However, as $\varepsilon$ increases, $\varepsilon\geq 0.1$, the AUROC drops below 0.800. The minimum AUROC performance results follow a similar trend.  
 
  
\begin{table}[t]
\normalsize
\centering
\caption{The maximum scores over all epochs for various trigger sizes are shown. We observe that unlike other metrics, the maximum AUROC-TN score indicates the worst-case performance, since a higher value of AUROC-TN indicates a lower responsiveness to the presence of a backdoor trigger. We include results for the clean data as a control to demonstrate the effectiveness of the backdoor triggers. }
\resizebox{1.0\linewidth}{!}{
\begin{tabular}{||l||ccccc||}
\hline
Trigger Size & Clean  & $1^2 px$ & $2^2 px$ & $3^2 px$ & $4^2 px$ \\ 
\hline\hline
%  ASR(p=0.6) & $.574\pm.269$   & $.999\pm0.$   & $1.\pm0.$  \\
%  ASR(p=0.9) & $.268\pm.194$  & $.998 \pm.001$  & $.997\pm.003$ \\
%  AUROC-TT & $.852\pm.046$ &  $1.\pm0.$  & $1.\pm0.$  \\ 
%  AUROC-NN & $.846\pm.045$  & $.851\pm.001$ & $.850\pm.004$  \\ 
%  AUROC-TN  & $.848\pm.002$ & $.704\pm.038$ & $.727\pm.018$ \\ 
 ASR($p=0.6$) & $0.574\pm0.269$  & $1.000\pm0.000$  & $1.000\pm0.000$  & $0.999\pm0.000$   & $1.000\pm0.000$  \\
 \hline
 ASR($p=0.9$) & $0.268\pm0.194$  & $1.000\pm0.000$  & $1.000\pm0.000$ & $0.998\pm0.001$  & $0.997\pm0.003$ \\
 \hline
 AUROC-TT & $0.852\pm0.046$ & $1.000\pm0.000$ & $1.000\pm0.000$  &  $1.000\pm0.000$  & $1.000\pm0.000$  \\ 
 \hline
 AUROC-NN & $0.846\pm0.045$  & $0.824\pm0.002$ & $0.824\pm0.001$ & $0.851\pm0.001$ & $0.850\pm0.004$  \\ 
 \hline
 AUROC-TN  & $0.848\pm0.002$ & $0.824\pm0.002$ & $0.824\pm0.001$ & $0.704\pm0.038$ & $0.727\pm0.018$ \\
\hline
\end{tabular}}
\label{table: triggersize_max}
\end{table}
 

\begin{table}[!h]
\small
\centering
\caption{We obtain results for when trigger is fixed in the center of infected images, compared to performance given a random trigger location. The results demonstrate that the backdoor trigger is learned by the neural network notwithstanding its spatial localization.}
\resizebox{1.0\linewidth}{!}{
\begin{tabular}{||l||cc|cc||} 
\hline %\multirow{2}{*}{}
 & \multicolumn{2}{c|}{Fixed} &\multicolumn{2}{c||}{Random} \\
\cline{2-5}
 & Minimum & Maximum & Minimum & Maximum \\ 
 \hline\hline
 ASR($p=0.6$) & $0.994\pm0.031$ & $0.999\pm0.000$ & $0.963\pm0.028$ & $0.994\pm0.002$  \\
 \hline
 ASR($p=0.9$) & $0.674\pm0.239$ & $0.998 \pm0.001$ & $0.902\pm0.079$ & $0.991\pm0.001$  \\
 \hline
 AUROC-TT & $1.000\pm0.000$ & $1.000\pm0.000$ & $0.998\pm0.001$ & $1.000\pm0.000$\\ 
 \hline
 AUROC-NN & $0.813\pm0.004$ & $0.851\pm0.001$ & $0.815\pm0.007$  & $0.848\pm0.005$  \\ 
 \hline
 AUROC-TN  & $0.539\pm0.004$ & $0.704\pm0.038$ & $0.579\pm0.027$ & $0.696\pm0.008$  \\ 
\hline
\end{tabular}}
\label{table: random_fixed}
\end{table}




\begin{table}[!h]
\small
\centering
\caption{ $\varepsilon$ represents the proportion of infected images in the test set containing clean and infected images. The results suggest that the attacker may successfully use their backdoor attack during inference without drawing significant attention from the user by maintaining a low proportion of infected images during inference time.
}
\resizebox{1.0\linewidth}{!}{
\begin{tabular}{||l||cc|cc||} 
\hline 
  & \multicolumn{4}{c||}{\textbf{$\varepsilon$}} \\
\cline{2-5}
 & 0.001 & 0.01 & 0.1 & 0.5 \\ 
 \hline\hline
 Minimum & $0.809\pm0.005$ & $0.804\pm0.005$ & $0.793\pm0.001$  & $0.604\pm0.002$  \\%values for 100% was 0.539+-0.003 
 \hline
 Maximum & $0.852\pm0.001$ & $0.847\pm0.001$ & $0.761\pm0.004$  & $0.662\pm0.006$  \\%values for 100% was 0.715+-0.022
%  \hline
\hline
\end{tabular}}
\label{table: inference_rate}
\end{table}

\subsection{Explainability can localize backdoor triggers}


Explainability techniques are commonly used in medical imaging applications \cite{holzinger2019causability}. We therefore examined the role of explainability in the context of backdoor attacks using Gradient Class Activation Mappings (Grad-CAM) \cite{selvaraju2017grad}. Grad-CAM calculates the derivative of activations with respect to the last convolutional layer of the neural network to compute saliency maps based on high-level features, where more important regions are indicated by red and less important regions are indicated by blue. We take this one step further by applying Grad-CAM to a middle layer, to examine low-level features. DenseNet-121 contains 287 layers, and we chose the 207th layer as the middle layer, since the  dimensions of the feature maps in this layer match those of the final layer.

Figure \ref{fig: explainability} shows the computed saliency maps for the clean and infected versions of an example image, at epochs 1, 4, and 12 from top to bottom. First, we note that the inclusion of a backdoor trigger causes a change in the heatmap, when comparing the changes in the saliency maps with respect to the final convolutional layer (columns 1 and 2), and the middle convolutional layer (columns 3 and 4) at each epoch. In particular, we observe that in a clean image (column 1), the heatmap focuses on the center of the patient's lungs. However, in column 2, the heatmap shifts towards the sternum region of the radiograph, where the backdoor trigger is located, across all epochs.

 As can be seen in columns 3 and 4, the middle layer shows more fine-grained backdoor trigger localization. While there are some saliency artifacts in column 3, we notice that the localization heatmap shift to the center of the image where the trigger is located in column 4. This is more noticeable at epochs 1 and 4, where there is less overfitting. The increased localization in the middle of the network is understandable since the backdoor trigger pixels can be considered as low-level features, and thus may be better detected in the earlier layers of the network. This suggests that explainability can play a complementary role with robustness, since Grad-CAM shows differences between the saliency maps of clean and infected images, and can help radiologists in questioning model predictions when the saliency maps and predictions seem unreasonable. One limitation of the work is that we did not investigate instances where the location of the trigger could correspond rightfully so with expected clinical interpretation. This requires clinical domain expertise. 
  

\section{Conclusion}
In this work, we propose a framework for evaluating backdoor effectiveness in the multi-label setting. Focusing on the medical imaging task of chest radiography, we show how explainability is a valuable tool in backdoor trigger detection. Future work should investigate the impact of backdoors and explainability on other medical imaging tasks, and the design of suitable defense mechanisms.
 

% \section{Ethics Statement}  
% Our work has ethical implications, both positive and negative. During the current global pandemic, artificial intelligence offers an opportunity to expand the availability of quality medical diagnosis. This is illustrated by the superior performance of deep learning on diagnosis and prognosis tasks for COVID-19 within chest radiographs \cite{wynants2020prediction}. We highlight the vulnerabilities of such models that may lead to high incurred costs and misdiagnosis. The main negative implication is that our work may be used by possible attackers to execute backdoor attacks. However, by raising awareness, we hope that this urges the machine learning community to develop more robust models and defense mechanisms specific to medical imaging. This is also important in low-resource hospital settings that may be vulnerable to outsourcing threats due to the lack of training data and computational resources.
 
\bibliography{references}
\clearpage

\newpage

% \onecolumn


% \tableofcontents{}

% \newpage

\section*{Supplementary Material}
\addcontentsline{toc}{section}{Supplementary Material}


Throughout this discussion, 
we will make frequently use 
of the following standard results
concerning the exponential concentration 
of random variables:

\begin{lemma}[Hoeffding's inequality for independent RVs~\citep{hoeffding1994probability}] Let $Z_1, Z_2, \ldots, Z_n$ be independent bounded random variables with $Z_i \in [a,b]$ for all $i$, then 
    \begin{align*}
        \prob\left( \frac{1}{n} \sum_{i=1}^n (Z_i - \Expo{Z_i}) \ge t \right) \le \exp{\left( -\frac{2nt^2}{(b-a)^2} \right) }
    \end{align*} 
    and 
    \begin{align*}
        \prob\left( \frac{1}{n} \sum_{i=1}^n (Z_i - \Expo{Z_i}) \le -t \right) \le \exp{\left( -\frac{2nt^2}{(b-a)^2} \right) }
    \end{align*} 
    for all $t \ge 0$. 
\end{lemma}

\begin{lemma}[Hoeffding's inequality for sampling with replacement~\citep{hoeffding1994probability}] \label{lem:hoeffding_sampling} Let $\calZ = (Z_1, Z_2, \ldots, Z_N)$ be a finite population of $N$ points with $Z_i \in [a.b]$ for all $i$. Let $X_1, X_2, \ldots X_n$ be a random sample drawn without replacement from $\calZ$. Then for all $t \ge 0$, we have 
    \begin{align*}
        \prob\left( \frac{1}{n} \sum_{i=1}^n (X_i - \mu ) \ge t \right) \le \exp{\left( -\frac{2nt^2}{(b-a)^2} \right) }
    \end{align*} 
    and 
    \begin{align*}
        \prob\left( \frac{1}{n} \sum_{i=1}^n (X_i - \mu ) \le -t \right) \le \exp{\left( -\frac{2nt^2}{(b-a)^2} \right) } \,,
    \end{align*} 
    where $\mu = \frac{1}{N} \sum_{i=1}^{N} Z_i$. 
\end{lemma}

We now discuss one condition that generalizes the exponential concentration to dependent random variables.
\begin{condition}[Bounded difference inequality] \label{cond:BDC} Let $\calZ$ be some set and $\phi: \calZ^n \to \Real$. We say that $\phi$ satisfies the bounded difference assumption if 
there exists $c_1, c_2, \ldots c_n \ge 0$ s.t. for all $i$, we have 
\begin{align*}
    \sup_{Z_1,Z_2, \ldots,Z_n, Z_i^\prime \in \calZ^{n+1} } \abs{\phi (Z_1, \ldots, Z_i, \ldots, Z_n ) - \phi (Z_1, \ldots, Z_i^\prime, \ldots, Z_n ) } \le c_i \,.
\end{align*} 
\end{condition}

\begin{lemma}[McDiarmid’s inequality~\citep{mcdiarmid1989}] \label{lem:McDiarmid} Let $Z_1, Z_2, \ldots, Z_n$ be independent random variables on set $\calZ$ and $\phi : \calZ^n \to \Real$ satisfy bounded difference inequality (\codref{cond:BDC}). Then for all $t>0$, we have 
    \begin{align*}
        \prob\left( \phi(Z_1, Z_2, \ldots, Z_n) - \Expo{\phi(Z_1, Z_2, \ldots, Z_n)} \ge t \right) \le \exp{\left( -\frac{2t^2}{\sum_{i=1}^n c_i^2} \right) } 
    \end{align*} 
    and 
    \begin{align*}
        \prob\left( \phi(Z_1, Z_2, \ldots, Z_n) - \Expo{\phi(Z_1, Z_2, \ldots, Z_n)} \le -t \right) \le \exp{\left( -\frac{2t^2}{\sum_{i=1}^n c_i^2} \right) } \,.
    \end{align*} 
\end{lemma}


\section{Proofs from \secref{sec:ERM_training}}\label{app:proof_erm}

\textbf{Additional notation {} {}} Let $m_1$ be the number of mislabeled points ($\wt S_M$) and $m_2$ be the number of correctly labeled points ($\wt S_C$). Note $m_1 + m_2 = m$. 


\subsection{Proof of \thmref{thm:error_ERM}}


\begin{proof}[Proof of \lemref{lem:fit_mislabeled}] 
    The main idea of our proof is to regard 
    the clean portion of the data 
    ($S \cup \wt S_C$) as fixed.   
    Then, there exists an (unknown) classifier $f^*$ 
    that minimizes the expected risk
    calculated on the (fixed) clean data
    and (random draws of) the mislabeled data $\wt S_M$. 
    % 
    % 
    Formally, 
    \begin{align}
    f^* \defeq \argmin_{f \in \calF} \error_{\widecheck {\calD}} (f) \,, \label{eq:modified_ERM}
    \end{align}
    where $$\widecheck \calD = \frac{n}{m+n} \calS + \frac{m_2}{m+n} \wt \calS_C  + \frac{m_1}{m+n}\calDm \,.$$ 
    Note here that $\widecheck \calD$ is a combination 
    of the \emph{empirical distribution} 
    over correctly labeled data $S \cup \wt S_C$
    and the (population) distribution 
    over mislabeled data $\calDm$.
    Recall that 
    \begin{align}
    \wh f \defeq \argmin_{f \in \calF} \error_{\calS \cup \wt S} (f) \,. \label{eq:orig_ERM}
    \end{align}
    % 
    % 
    Since, $\widehat f$ minimizes 0-1 error 
    on $S \cup \wt S$, using ERM optimality on \eqref{eq:orig_ERM},  
    we have 
    \begin{align}
        \error_{\calS \cup \wt \calS}(\widehat f) \le \error_{
            \calS \cup \wt \calS}(f^*) \,.    \label{eq:step1}
    \end{align}
    Moreover, since $f^*$ is independent of $\wt S_M$, using Hoeffding's bound,
    % \footnote{For a fully rigorous argument,
    % refer to the complete proof in App.~\ref{app:proof_erm}.} 
    we have with probability at least $1-\delta$ that
    \begin{align}
      \error_{\wt \calS_M}(f^*) \le \error_{ \calDm}(f^*) +  \sqrt{\frac{\log(1/\delta)}{2 m_1}} \,. \label{eq:step2} 
    \end{align}
    %$ 
    %for some constant $c_1\le 1/2$. 
    Finally, since $f^*$ is the optimal classifier on $\widecheck \calD$, 
    we have 
    \begin{align}
        \error_{\widecheck \calD}(f^*) \le \error_{\widecheck \calD}(\widehat f) \,. \label{eq:step3}
    \end{align}
    Now to relate \eqref{eq:step1} and \eqref{eq:step3}, we multiply \eqref{eq:step2} by $\frac{m_1}{m+n}$ and add $\frac{n}{m+n} \error_{\calS} (f)  + \frac{m_2}{m+n} \error_{\wt \calS_C} (f)$ both the sides. Hence, 
    we can rewrite \eqref{eq:step2} as follows: 
    \begin{align}
        \error_{\calS \cup \wt\calS}(f^*) \le \error_{ \widecheck \calD}(f^*) +  \frac{m_1}{m+n}\sqrt{\frac{\log(1/\delta)}{2 m_1}} \,. \label{eq:step4} 
    \end{align}
    Now we combine equations \eqref{eq:step1}, \eqref{eq:step4}, and \eqref{eq:step3}, to get 
    \begin{align}
        \error_{\calS \cup \wt \calS}(\wh f) \le \error_{\widecheck \calD}(\wh f) +  \frac{m_1}{m+n}\sqrt{\frac{\log(1/\delta)}{2 m_1}} \,, 
    \end{align}
    which implies 
    \begin{align}
        \error_{ \wt \calS_M}(\wh f) \le \error_{\calDm}(\wh f) + \sqrt{\frac{\log(1/\delta)}{2 m_1}} \,. \label{eq:lemma1_final}
    \end{align}
    Since $\wt S$ is obtained by randomly labeling an unlabeled dataset, we assume $2m_1 \approx m$ \footnote{Formally, with probability at least $1-\delta$, we have  $(m - 2m_1)\le \sqrt{m\log(1/\delta)/2}$.}. Moreover, using $\error_{\calDm} = 1 - \error_{\calD}$ we obtain the desired result.   
    % Combining the above steps and using the fact 
    % that $\error_\calD = 1- \error_{\calDm} $, 
    % we obtain the desired result.
\end{proof}

\begin{proof}[Proof of \lemref{lem:mislabeled_error}]
    Recall $\error_{\wt S} (f) = \frac{m_1}{m} \error_{\wt S_M}(f) + \frac{m_2}{m} \error_{\wt S_C}(f)$. Hence, we have 
    \begin{align}
        2\error_{\wt S}(f) - \error_{\wt S_M}(f) - \error_{\wt S_C}(f) &= \left(\frac{2m_1}{m} \error_{\wt S_M}(f) - \error_{\wt S_M}(f)\right) + \left(\frac{2m_2}{m} \error_{\wt S_C}(f) - \error_{\wt S_C}(f)\right) \\ &= \left(\frac{2m_1}{m} - 1\right) \error_{\wt S_M}(f) + \left(\frac{2m_2}{m} - 1 \right)\error_{\wt S_C} (f) \,.
    \end{align} 
    Since the dataset is labeled uniformly at random, with probability at least $1-\delta$, we have  $\left(\frac{2m_1}{m} - 1\right) \le \sqrt{\frac{\log(1/\delta)}{2m}}$. Similarly, we have with probability at least $1-\delta$, $\left(\frac{2m_2}{m} - 1\right) \le \sqrt{\frac{\log(1/\delta)}{2m}}$. Using union bound, with probability at least $1-\delta$, we have
    % \begin{align}
    %     2\error_{\wt S} - \error_{\wt S_M}(f) - \error_{\wt S_C}(f) \le \sqrt{\frac{\log(2/\delta)}{2m}} \left(\error_{\wt S_M}(f) + \error_{\wt S_C}(f) \right) \le 2\sqrt{\frac{\log(2/\delta)}{2m}} \,. \label{eq:lemma2_final}
    % \end{align}
    \begin{align}
        2\error_{\wt S} - \error_{\wt S_M}(f) - \error_{\wt S_C}(f) \le \sqrt{\frac{\log(2/\delta)}{2m}} \left(\error_{\wt S_M}(f) + \error_{\wt S_C}(f) \right) \,. \label{eq:lemma2_prefinal}
    \end{align}
    With re-arranging $\error_{\wt S_M}(f) + \error_{\wt S_C}(f)$ and using the inequality $ 1- a\le \frac{1}{1+a} $, we have  
    \begin{align}
        2\error_{\wt S} - \error_{\wt S_M}(f) - \error_{\wt S_C}(f) \le 2\error_{\wt \calS} \sqrt{\frac{\log(2/\delta)}{2m}}  \,. \label{eq:lemma2_final}
    \end{align}

    % We obtain the desired result by using 
\end{proof}

\begin{proof}[Proof of \lemref{lem:clear_error}]
% Recall 0-1 error on each point  $(x,y) \in S \cup \wt S$ is given by $\I{ f(x)\ne y}$.
In the set of correctly labeled points $S \cup \wt S_C$, we have $S$ as a random subset of $S \cup \wt S_C$. Hence, using Hoeffding's inequality for sampling without replacement (\lemref{lem:hoeffding_sampling}), we have with probability at least $1-\delta$
\begin{align}
    \error_{\wt \calS_C} (\wh f)- \error_{\calS \cup \wt \calS_C}( \wh f) \le  \sqrt{\frac{\log(1/\delta)}{2m_2}} \,.
\end{align}
Re-writing $\error_{\calS \cup \wt \calS_C}( \wh f)$ as $\frac{m_2}{m_2 + n} \error_{\wt \calS_C }(\wh f) + \frac{n}{m_2 + n} \error_{\calS }(\wh f)$, we have with probability at least $1-\delta$
\begin{align}
   \left(\frac{n}{n+m_2}\right) \left(\error_{\wt \calS_C} (\wh f)- \error_{\calS}( \wh f) \right) \le  \sqrt{\frac{\log(1/\delta)}{2m_2}} \,.
\end{align}
As before, assuming $2m_2 \approx m$, we have with probability at least $1-\delta$ 
\begin{align}
    \error_{\wt \calS_C} (\wh f)- \error_{\calS}( \wh f) \le \left(1+\frac{m_2}{n}\right)  \sqrt{\frac{\log(1/\delta)}{m}} \le \left(1 + \frac{m}{2n}\right) \sqrt{\frac{\log(1/\delta)}{m}} \,. \label{eq:lemma3_final}
\end{align} 
\end{proof}

\begin{proof}[Proof of \thmref{thm:error_ERM}] 
    Having established these core intermediate results, we can now combine above three lemmas to prove the main result. 
    In particular, we bound the population error on clean data ($\error_\calD(\wh f)$) as follows:  
    \begin{enumerate}[(i)]
        \item First, use \eqref{eq:lemma1_final}, to obtain an upper bound on the population error on clean data, i.e., with probability at least $1-\delta/4$, we have
        \begin{align}
            \error_{ \calD} (\wh f) \le 1 - \error_{ \wt \calS_M}(\wh f) + \sqrt{\frac{\log(4/\delta)}{m}} \,. 
        \end{align}
        \item  Second, use \eqref{eq:lemma2_final}, to relate the error on the mislabeled fraction with error on clean portion of randomly labeled data and error on whole randomly labeled dataset, i.e., with probability at least $1-\delta/2$, we have 
        \begin{align}
            - \error_{\wt S_M}(f) \le \error_{\wt S_C}(f) - 2\error_{\wt S}  + 2\error_{\wt S} \sqrt{\frac{\log(4/\delta)}{2m}}  \,. 
        \end{align} 
        \item Finally, use \eqref{eq:lemma3_final} to relate the error on the clean portion of randomly labeled data and error on clean training data, i.e., with probability $1-\delta/4$, we have 
        \begin{align}
            \error_{\wt \calS_C} (\wh f)\le - \error_{\calS}( \wh f) + \left(1 + \frac{m}{2n} \right) \sqrt{\frac{\log(4/\delta)}{m}} \,. 
        \end{align} 
    \end{enumerate}

    Using union bound on the above three steps, we have with probability at least $1-\delta$: 
    \begin{align}
        \error_\calD (\wh f) \le \error_{\calS}(\wh f)   + 1 - 2\error_{\wt \calS}(\wh f)   + \left(\sqrt{2} \error_{\wt S} + 2 + \frac{m}{2n}\right)  \sqrt{\frac{\log(4/\delta)}{m}} \,.
    \end{align}
    % Note that $(1/\sqrt{2} + 2.5)$ is a loose constant. In experiments, we use the ratio $\frac{m}{n}$
    %  the exact error $\error_{\wt \calS}(\wh f)$ 
    % to evaluate R.H.S.    
\end{proof}

\subsection{Proof of \propref{prop:rademacher}}

\begin{proof}[Proof of \propref{prop:rademacher}]
    For a classifier $ f: \calX \to \{-1, 1\}$, we have $1 - 2\,\indict{ f(x) \ne y} = y \cdot f(x)$. Hence, by definition of $\error$, we have 
    \begin{align}
        1 -2\error_{\wt \calS}(f) = \frac{1}{m}\sum_{i=1}^m y_i \cdot f(x_i) \le \sup_{f \in \calF} \, \frac{1}{m} \sum_{i=1}^m y_i \cdot f(x_i)  \,. \label{eq:error_rademacher}
    \end{align}
    Note that for fixed inputs $(x_1, x_2, \ldots, x_m)$ in $\wt S$, $(y_1, y_2, \ldots y_m)$ are random labels. Define $\phi_1 (y_1, y_2, \ldots, y_m) \defeq \sup_{f \in \calF} \, \frac{1}{m} \sum_{i=1}^m y_i \cdot f(x_i)$. We have the following bounded difference condition on $\phi_1$. For all i, 
    \begin{align}
        \sup_{y_1, \ldots y_m, y_i^\prime \in \{-1, 1\}^{m+1} } \abs{ \phi_1 (y_1,\ldots, y_i, \ldots, y_m) - \phi_1 (y_1,\ldots, y_i^\prime, \ldots, y_m)  } \le 1/m \,. \label{cond1_rademacher}
    \end{align} 
    
    Similarly, we define $\phi_2 (x_1, x_2, \ldots, x_m) \defeq \Expt{ y_i \sim_U \{-1, 1\}  }{ \sup_{f \in \calF} \, \frac{1}{m}  \sum_{i=1}^m y_i \cdot f(x_i)}$. We have the following bounded difference condition on $\phi_2$. 
    For all i,
    \begin{align}
        \sup_{x_1, \ldots x_m, x_i^\prime \in \calX^{m+1} } \abs{ \phi_2 (x_1,\ldots, x_i, \ldots, x_m) - \phi_1 (x_1,\ldots, x_i^\prime, \ldots, x_m)  } \le 1/m \,. \label{cond2_rademacher}
    \end{align}
    Using McDiarmid’s inequality (\lemref{lem:McDiarmid}) twice 
    with Condition \eqref{cond1_rademacher} and \eqref{cond2_rademacher}, 
    with probability at least $1-\delta$, we have
    \begin{align}
        \sup_{f \in \calF} \, \frac{1}{m} \sum_{i=1}^m y_i \cdot f(x_i)  - \Expt{x,y}{\sup_{f \in \calF} \, \frac{1}{m} \sum_{i=1}^m y_i \cdot f(x_i) } \le \sqrt{\frac{2\log(2/\delta)}{m}} \,. \label{eq:final_rademacher}
    \end{align} 
    Combining \eqref{eq:error_rademacher} and \eqref{eq:final_rademacher}, we obtain the desired result. 
\end{proof}


\subsection{Proof of \thmref{thm:error_regularized_ERM}}

Proof of \thmref{thm:error_regularized_ERM} follows similar to the proof of \thmref{thm:error_ERM}. Note that the same results in \lemref{lem:fit_mislabeled}, \lemref{lem:mislabeled_error}, and \lemref{lem:clear_error} hold in the regularized ERM case. However, the arguments in the proof of \lemref{lem:fit_mislabeled} change slightly. Hence, we state the lemma for regularized ERM and prove it here for completeness. 

\begin{lemma} \label{lem:lemma1_reg}
    Assume the same setup as \thmref{thm:error_regularized_ERM}. 
    Then for any $\delta >0$, with probability at least  $1-\delta$ 
    over the random draws of mislabeled data $\wt S_M$, we have 
    \begin{align}
        \error_\calD(\widehat f)  \le 1 -\error_{\wt \calS_M}(\widehat f) + \sqrt{\frac{\log(1/\delta)}{m}}\,. 
    \end{align} 
\end{lemma}
\begin{proof}
    The main idea of the proof remains the same, i.e. regard 
    the clean portion of the data 
    ($S \cup \wt S_C$) as fixed.   
    Then, there exists a classifier $f^*$ 
    that is optimal over draws 
    of the mislabeled data $\wt S_M$. 

    
    Formally, 
    \begin{align}
    f^* \defeq \argmin_{f \in \calF} \error_{\widecheck {\calD}} (f)  + \lambda R(f) \,, \label{eq:modified_ERM_reg}
    \end{align}
    where $$\widecheck \calD = \frac{n}{m+n} \calS + \frac{m_1}{m+n} \wt \calS_C  + \frac{m_2}{m+n}\calDm \,.$$ That is, $\widecheck \calD$ a combination of 
    the \emph{empirical distribution} 
    over correctly labeled data $S \cup \wt S_C$
    % in $S\cup \wt S$ 
    and the (population) distribution 
    over mislabeled data $\calDm$.
    Recall that 
    \begin{align}
    \wh f \defeq \argmin_{f \in \calF} \error_{\calS \cup \wt S} (f) + \lambda R(f) \,. \label{eq:orig_ERM_reg}
    \end{align}
    % 
    % 
    Since, $\widehat f$ minimizes 0-1 error 
    on $S \cup \wt S$, using ERM optimality on \eqref{eq:orig_ERM},  
    we have 
    \begin{align}
        \error_{\calS \cup \wt \calS}(\widehat f) + \lambda R(\wh f) \le \error_{
            \calS \cup \wt \calS}(f^*) + \lambda R(f^*) \,.    \label{eq:step1_reg}
    \end{align}
    Moreover, since $f^*$ is independent of $\wt S_M$, using Hoeffding's bound,
    % \footnote{For a fully rigorous argument,
    % refer to the complete proof in App.~\ref{app:proof_erm}.} 
    we have with probability at least $1-\delta$ that
    \begin{align}
      \error_{\wt \calS_M}(f^*) \le \error_{ \calDm}(f^*) +  \sqrt{\frac{\log(1/\delta)}{2 m_1}} \,. \label{eq:step2_reg} 
    \end{align}
    %$ 
    %for some constant $c_1\le 1/2$. 
    Finally, since $f^*$ is the optimal classifier on $\widecheck \calD$, 
    we have 
    \begin{align}
        \error_{\widecheck \calD}(f^*) + \lambda R(f^*) \le \error_{\widecheck \calD}(\widehat f) + \lambda R(\wh f) \,. \label{eq:step3_reg}
    \end{align}
     Now to relate \eqref{eq:step1_reg} and \eqref{eq:step3_reg}, we can re-write the \eqref{eq:step2_reg} as follows: 
    \begin{align}
        \error_{\calS \cup \wt\calS}(f^*) \le \error_{ \widecheck \calD}(f^*) +  \frac{m_1}{m+n}\sqrt{\frac{\log(1/\delta)}{2 m_1}} \,. \label{eq:step4_reg} 
    \end{align}
    After adding $\lambda R(f^*)$ on both sides in \eqref{eq:step4_reg}, we combine equations \eqref{eq:step1_reg}, \eqref{eq:step4_reg}, and \eqref{eq:step3_reg}, to get 
    \begin{align}
        \error_{\calS \cup \wt \calS}(\wh f) \le \error_{\widecheck \calD}(\wh f) +  \frac{m_1}{m+n}\sqrt{\frac{\log(1/\delta)}{2 m_1}} \,, 
    \end{align}
    which implies 
    \begin{align}
        \error_{ \wt \calS_M}(\wh f) \le \error_{\calDm}(\wh f) + \sqrt{\frac{\log(1/\delta)}{2 m_1}} \,. \label{eq:lemma_reg_final}
    \end{align}
    Similar as before, since $\wt S$ is obtained by randomly labeling an unlabeled dataset, we assume 
    $2m_1 \approx m$. Moreover, using $\error_{\calDm} = 1 - \error_{\calD}$ we obtain the desired result. 
\end{proof}
% \begin{proof}[Proof of ]
    
% \end{proof}

\subsection{Proof of \thmref{thm:multiclass_ERM}}

To prove our results in the multiclass case,
we first state and prove lemmas
parallel to those
% We first state and prove lemmas 
% parallel 
% to the three lemmas 
used in the proof of balanced binary case. 
We then combine these results 
% in the three lemmas 
to obtain the result in \thmref{thm:multiclass_ERM}. 

Before stating the result, 
we define mislabeled distribution $\calDm$ for any $\calD$.
While $\calDm$ and $\calD$ share 
the same marginal distribution over inputs $\calX$,
the conditional distribution over labels $y$ 
given an input $x\sim \calD_\calX$ is changed as follows:
For any $x$, the Probability Mass Function (PMF) over $y$ is defined as:  
$p_{\calDm} (\cdot \vert x) \defeq \frac{1 - p_{\calD}(\cdot \vert x)}{k - 1}$, where $ p_{\calD}(\cdot \vert x)$ is the PMF over $y$ for the distribution $\calD$. 

\begin{lemma} \label{lem:fit_mislabeled_multi}
    Assume the same setup as \thmref{thm:multiclass_ERM}. 
    Then for any $\delta >0$, with probability at least  $1-\delta$ 
    over the random draws of mislabeled data $\wt S_M$, we have 
    \begin{align}
        \error_\calD(\widehat f)  \le (k-1)\left(1 -\error_{\wt \calS_M}(\widehat f)\right) + (k-1)\sqrt{\frac{\log(1/\delta)}{m}}\,. \label{eq:lemma1_multi}
    \end{align}   
\end{lemma} 

\begin{proof}
   
    The main idea of the proof remains the same.
    We begin by regarding the clean portion of the data 
    ($S \cup \wt S_C$) as fixed. 
    Then, there exists a classifier $f^*$ 
    that is optimal over draws 
    of the mislabeled data $\wt S_M$. 
    
    However, in the multiclass case,
    we cannot as easily relate the population error on mislabeled data 
    to the population accuracy on clean data.   
    While for binary classification, 
    % we could upper bound $\error_{\wt \calS_M}$ 
    % with $1-\error_\calD$ 
    we could lower bound the population accuracy $1-\error_\calD$
    with the empirical error on mislabeled data $\error_{\wt \calS_M}$ 
    (in the proof of \lemref{lem:fit_mislabeled}), 
    for multiclass classification, 
    error on the mislabeled data 
    and accuracy on the clean data 
    in the population 
    are not so directly related.  
    To establish \eqref{eq:lemma1_multi},
    we break the error on the 
    (unknown) mislabeled data 
    into two parts: one term corresponds 
    to predicting the true label on mislabeled data, 
    and the other corresponds to predicting 
    neither the true label 
    nor the assigned (mis-)label.  
    Finally, we relate these errors to their
    population counterparts to establish \eqref{eq:lemma1_multi}. 
    
    Formally, 
    \begin{align}
    f^* \defeq \argmin_{f \in \calF} \error_{\widecheck {\calD}} (f)  + \lambda R(f) \,, \label{eq:modified_ERM_reg2}
    \end{align}
    where $$\widecheck \calD = \frac{n}{m+n} \calS + \frac{m_1}{m+n} \wt \calS_C  + \frac{m_2}{m+n}\calDm \,.$$ 
    That is, $\widecheck \calD$ is a combination 
    of the \emph{empirical distribution} 
    over correctly labeled data $S \cup \wt S_C$
    % in $S\cup \wt S$ 
    and the (population) distribution 
    over mislabeled data $\calDm$.
    Recall that 
    \begin{align}
    \wh f \defeq \argmin_{f \in \calF} \error_{\calS \cup \wt S} (f) + \lambda R(f) \,. \label{eq:orig_ERM_reg2}
    \end{align}
    % 
    % 
    Following the exact steps from the proof of \lemref{lem:lemma1_reg}, 
    with probability at least $1-\delta$, we have  
    \begin{align}
        \error_{ \wt \calS_M}(\wh f) \le \error_{\calDm}(\wh f) + \sqrt{\frac{\log(1/\delta)}{2 m_1}} \,. \label{eq:lemma1_final_multi_prev}
    \end{align}
    Similar to before, since $\wt S$ is obtained 
    by randomly labeling an unlabeled dataset, 
    we assume 
    $\frac{k}{k-1} m_1 \approx m$. 
    
    Now we will relate $\error_{\calDm} (\wh f)$ with $\error_{\calD}(\wh f)$. 
    Let $y^T$ denote the (unknown) true label 
    for a mislabeled point $(x, y)$ 
    (i.e., label before replacing it with a mislabel). 
    \begin{align*}    
         \Expt{(x, y) \in \sim \calDm}{\indict{ \wh f(x) \ne y }}  &= \underbrace{\Expt{(x, y) \in \sim \calDm}{\indict{ \wh f(x) \ne y \land \wh f(x) \ne y^T}}}_{\RN{1}} \\ &\qquad \qquad + \underbrace{\Expt{(x, y) \in \sim \calDm}{\indict{ \wh f(x) \ne y \land \wh f(x) = y^T}}}_{\RN{2}} \,. \numberthis \label{eq:excess_term}
    \end{align*}
    Clearly, term 2 is one minus the accuracy 
    on the clean unseen data, i.e.,
    \begin{align}
        \RN{2} = 1 - \Expt{{x,y} \sim \calD}{ \indict{ \wh f(x) \ne y}} = 1- \error_{\calD}(\wh f) \,. \label{eq:term1}    
    \end{align}
    Next, we relate term 1 with the error on the unseen clean data. 
    We show that term 1 is equal to the error on the unseen clean data 
    scaled by $\frac{k-2}{k-1}$,
    where $k$ is the number of labels.
    Using the definition of mislabeled distribution $\calDm$,  
    we have 
    \begin{align}
        \RN{1} = \frac{1}{k-1} \left( \Expt{(x, y) \in \sim \calD}{ \sum_{i \in \calY \land i\ne y}  \indict{ \wh f(x) \ne i \land \wh f(x) \ne y}} \right) = \frac{k-2}{k-1} \error_{\calD}(\wh f) \,.\label{eq:term2}
    \end{align}    

    Combining the result in \eqref{eq:term1}, \eqref{eq:term2} and \eqref{eq:excess_term}, we have 
    \begin{align}
        \error_{\calDm}(\wh f) = 1- \frac{1}{k-1} \error_{\calD}(\wh f) \,.\label{eq:combine_terms}
    \end{align}
    Finally, combining the result in \eqref{eq:combine_terms} 
    with equation \eqref{eq:lemma1_final_multi_prev}, 
    we have with probability $1-\delta$, 
    \begin{align}
      \error_{\calD}(\wh f) \le  (k-1) \left( 1- \error_{ \wt \calS_M}(\wh f) \right)  + (k-1) \sqrt{\frac{k \log(1/\delta)}{ 2(k-1)m}} \,. \label{eq:lemma1_final_multi}
    \end{align}
\end{proof}

\begin{lemma} \label{lem:mislabeled_error_multi}
    Assume the same setup as \thmref{thm:multiclass_ERM}. 
    Then for any $\delta >0$, 
    with probability at least $1-\delta$ 
    over the random draws of $\wt S$, we have  
    % \begin{align}
        $$\abs{k\error_{\wt \calS}(\widehat f) - \error_{\wt \calS_C}(\widehat f) -  (k-1)\error_{\wt \calS_M}(\widehat f) } \le  2k\sqrt{\frac{\log(4/\delta)}{2m}}\,. $$ % \label{eq:lemma2}
    % \end{align}   
    %  for some constant $c_3 \le 1.0\,$.
\end{lemma} 


\begin{proof}
    Recall $\error_{\wt S} (f) = \frac{m_1}{m} \error_{\wt S_M}(f) + \frac{m_2}{m} \error_{\wt S_C}(f)$. Hence, we have 
    \begin{align*}
        k\error_{\wt S}(f) - (k-1)\error_{\wt S_M}(f) - \error_{\wt S_C}(f) &= (k-1)\left(\frac{k m_1}{(k-1) m} \error_{\wt S_M}(f) - \error_{\wt S_M}(f)\right) \\ & \qquad \qquad + \left(\frac{km_2}{m} \error_{\wt S_C}(f) - \error_{\wt S_C}(f)\right) \\ &= k \left[ \left(\frac{m_1}{m} - \frac{k-1}{k}\right) \error_{\wt S_M}(f) + \left(\frac{m_2}{m} - \frac{1}{k} \right) \error_{\wt S_C} (f) \right] \,.
    \end{align*} 
    Since the dataset is randomly labeled, 
    we have with probability at least $1-\delta$, 
    $\left(\frac{m_1}{m} - \frac{k-1}{k}\right) \le \sqrt{\frac{\log(1/\delta)}{2m}}$. 
    Similarly, we have with probability at least $1-\delta$, 
    $\left(\frac{m_2}{m} - \frac{1}{k}\right) \le \sqrt{\frac{\log(1/\delta)}{2m}}$. 
    Using union bound, we have with probability at least $1-\delta$
    % \begin{align}
    %     2\error_{\wt S} - \error_{\wt S_M}(f) - \error_{\wt S_C}(f) \le \sqrt{\frac{\log(2/\delta)}{2m}} \left(\error_{\wt S_M}(f) + \error_{\wt S_C}(f) \right) \le 2\sqrt{\frac{\log(2/\delta)}{2m}} \,. \label{eq:lemma2_final}
    % \end{align}
    \begin{align}
        k\error_{\wt S}(f) - (k-1)\error_{\wt S_M}(f) - \error_{\wt S_C}(f)  \le k \sqrt{\frac{\log(2/\delta)}{2m}} \left(\error_{\wt S_M}(f) + \error_{\wt S_C}(f) \right) \,. \label{eq:lemma2_final_multi}
    \end{align}

    % We obtain the desired result by using 
\end{proof}

\begin{lemma} \label{lem:clear_error_multi}
    Assume the same setup as \thmref{thm:multiclass_ERM}. 
    Then for any $\delta >0$, with probability at least $1-\delta$ 
    over the random draws of $\wt S_C$ and $S$, we have 
    % \begin{align}
        $$\abs{\error_{\wt \calS_C}(\widehat f) - \error_{\calS}(\widehat f) } \le 1.5 \sqrt{\frac{k\log(2/\delta)}{2m}}\,.$$ %\label{eq:lemma3}
    % \end{align}   
    % for some constant $c_2 \le 1.2\,$.
\end{lemma} 
\begin{proof}
    % Recall 0-1 error on each point  $(x,y) \in S \cup \wt S$ is given by $\I{ f(x)\ne y}$.
    In the set of correctly labeled points $S \cup \wt S_C$,
    we have $S$ as a random subset of $S \cup \wt S_C$. 
    Hence, using Hoeffding's inequality 
    for sampling without replacement 
    (\lemref{lem:hoeffding_sampling}), 
    we have with probability at least $1-\delta$
    \begin{align}
        \error_{\wt \calS_c} (\wh f)- \error_{\calS \cup \wt \calS_C}( \wh f) \le  \sqrt{\frac{\log(1/\delta)}{2m_2}} \,.
    \end{align}
    Re-writing $\error_{\calS \cup \wt \calS_C}( \wh f)$ 
    as $\frac{m_2}{m_2 + n} \error_{\wt \calS_C }(\wh f) + \frac{n}{m_2 + n} \error_{\calS }(\wh f)$, 
    we have with probability at least $1-\delta$
    \begin{align}
       \left(\frac{n}{n+m_2}\right) \left(\error_{\wt \calS_c} (\wh f)- \error_{\calS}( \wh f) \right) \le  \sqrt{\frac{\log(1/\delta)}{2m_2}} \,.
    \end{align}
    As before, assuming $km_2 \approx m$, 
    we have with probability at least $1-\delta$ 
    \begin{align}
        \error_{\wt \calS_c} (\wh f)- \error_{\calS}( \wh f) \le \left(1+\frac{m_2}{n}\right)  \sqrt{\frac{k\log(1/\delta)}{2m}} \le \left( 1 + \frac{1}{k}\right) \sqrt{\frac{k\log(1/\delta)}{2m}} \,. \label{eq:lemma3_final_multi}
    \end{align} 
\end{proof}

\begin{proof}[Proof of \thmref{thm:multiclass_ERM}] 
    Having established these core intermediate results, 
    we can now combine above three lemmas. 
    In particular, we bound the population error 
    on clean data ($\error_\calD(\wh f)$) as follows:  
    \begin{enumerate}[(i)]
        \item First, use \eqref{eq:lemma1_final_multi}, 
        to obtain an upper bound on the population error on clean data, 
        i.e., with probability at least $1-\delta/4$, we have
        \begin{align}
            \error_{ \calD} (\wh f) \le (k-1)\left(1 - \error_{ \wt \calS_M}(\wh f) \right) + (k-1) \sqrt{\frac{k\log(4/\delta)}{2(k-1)m}} \,. 
        \end{align}
        \item  Second, use \eqref{eq:lemma2_final_multi}
        to relate the error on the mislabeled fraction 
        with error on clean portion of randomly labeled data 
        and error on whole randomly labeled dataset, 
        i.e., with probability at least $1-\delta/2$, we have 
        \begin{align}
            - (k-1)\error_{\wt S_M}(f) \le \error_{\wt S_C}(f) - k\error_{\wt S}  + k\sqrt{\frac{\log(4/\delta)}{2m}}  \,. 
        \end{align} 
        \item Finally, use \eqref{eq:lemma3_final_multi} 
        to relate the error on the clean portion of randomly labeled data 
        and error on clean training data, 
        i.e., with probability $1-\delta/4$, we have 
        \begin{align}
            \error_{\wt \calS_C} (\wh f)\le - \error_{\calS}( \wh f) + \left(1 + \frac{m}{kn} \right) \sqrt{\frac{k\log(4/\delta)}{2m}} \,. 
        \end{align} 
    \end{enumerate}

    Using union bound on the above three steps, 
    we have with probability at least $1-\delta$: 
    \begin{align}
        \error_\calD (\wh f) \le \error_{\calS}(\wh f) + (k-1) - k\error_{\wt \calS}(\wh f)   + (\sqrt{k(k-1)} + k + \sqrt{k} + \frac{m}{n\sqrt{k}})  \sqrt{\frac{\log(4/\delta)}{2m}} \,.\label{eq:multiclass_ERM_final}
    \end{align}
    Simplifying the term in RHS of \eqref{eq:multiclass_ERM_final}, 
    we get the desired result. 
    % Note that since $\frac{m}{n\sqrt{k}}$ 
    % is much smaller than the sum of the other terms
    % the other terms in summation, 
    % we ignore $\frac{m}{n\sqrt{k}}$  
    % Z: ??? --- great
    % that 
    % them
    in the final bound. 
    % we ignore that in the final bound. 
    % Note that $(1/\sqrt{2} + 2.5)$ is a loose constant. In experiments, we use the ratio $\frac{m}{n}$
    %  the exact error $\error_{\wt \calS}(\wh f)$ 
    % to evaluate R.H.S.    
\end{proof}

\newpage
\section{Proofs from \secref{sec:linear_models}}\label{app:proof_gd}
We suppose that the parameters of the linear function 
are obtained via gradient descent on 
the following $L_2$ regularized problem: 
\begin{align}
    % n in denominator is avoided deliberately
    \calL_S(w; \lambda) \defeq \sum_{i=1}^n{(w^Tx_i - y_i)^2} + \lambda \norm{w}{2}^2 \,, \label{eq:l2_MSE_app}   
\end{align}
where $\lambda\ge0$ is a regularization parameter. 
We assume access to a clean dataset 
$S = \{(x_i, y_i)\}_{i=1}^n \sim \calD^n$ 
and randomly labeled dataset 
$\wt S = \{(x_i, y_i)\}_{i=n+1}^{n+m} \sim \wt \calD^m$. 
Let $\bX = [x_1, x_2, \cdots, x_{m+n}]$ 
and $\by = [y_1, y_2, \cdots, y_{m+n}]$. 
Fix a positive learning rate $\eta$ such that 
$\eta \le 1/\left(\norm{\bX^T\bX}{\text{op}} + \lambda^2\right)$ 
and an initialization $w_0 = 0$. 
% \todos{Assumption made for simplicty}. 
Consider the following gradient descent iterates 
to minimize objective \eqref{eq:l2_MSE_app} on $S \cup \wt S$:
\begin{align}
w_t = w_{t-1} - \eta \grad_w \calL_{S \cup \wt S} (w_{t-1}; \lambda) \quad \forall t=1,2,\ldots \label{eq:GD_iterates_app}
\end{align} 
Then we have $\{ w_t\}$ converge to the limiting solution 
$\wh w = \left( \bX^T\bX+\lambda \boldsymbol{I}\right)^{-1}\bX^T\by$. Define $\widehat f (x) \defeq f(x ; \wh w) $.  

% \subsection{\textcolor{red}{Errata}}

% We wish to correct the following error in the body:
% \codref{cond:error_stability} is not enough 
% to guarantee the result in \thmref{thm:linear}. 
% We now present a slightly stronger condition 
% called \emph{hypothesis stability} 
% under which we obtain a result 
% similar to \thmref{thm:linear}. 

% This error doesn't change the main arguments of the proof,
% where we show that the empirical train error 
% is less than or equal to the leave-one-out error.
% We need a stronger condition to relate leave-one-out error 
% with the population error of the original classifier. 
% Specifically, while \codref{cond:error_stability} 
% relates the average population error of leave-one-out classifiers 
% with the population error of the original classifier, 
% we need the new condition to show the concentration 
% of the empirical leave-one-out error 
% and average population error of leave-one-out classifiers. 
% main takeaway 

% Note that the new condition, 
% while being stronger than the previous one, 
% still doesn't imply generalization \citep{bousquet2002stability,elisseeff2003leave,abou2019exponential}. 
% Overall, the main results in \secref{sec:ERM_training} 
% and takeaways of the paper remain unaffected by the error.  

% We now present the new condition 
% and a corrected statement of \thmref{thm:linear}. 
% Recall, for a given training set $S \sim \calD^n $, 
% we use $S_{(i)}$ to denote the training set $S$ 
% with the $i^{\text{th}}$ point removed.

% \begin{condition}[Hypothesis Stability] 
%     \label{cond:hypothesis_stability}
%     We have $\beta$ hypothesis stability 
%     if our training algorithm $\calA$ satisfies the following: 
%     \begin{align*}
%     % ${\sum_{i=1}^n \frac{\error_{\calD}( f(\calA, S_{(i)}))}{n} - \error_\calD(f(\calA, S))} \le \beta\,$.
%     \forall i \in \{1,2,\ldots, n\}, \quad  \Expt{\calS, (x,y) \in \calD}{ \abs{\error\left( f(x) ,y  \right) - \error\left( f_{(i)}(x), y \right) }} \le \frac{\beta}{n} \,,
%     \end{align*}
%     where $f_{(i)} \defeq f(\calA, S_{(i)})$ and $ f \defeq f(\calA, S)$.
% \end{condition}

% \begin{theorem}[Correct statement of \thmref{thm:linear}] \label{thm:new_linear}
%     Assume that this gradient descent algorithm satisfies \codref{cond:hypothesis_stability}
%     with $\beta=\calO(1)$.  
%     Then for any $\delta >0$, with probability at least $1-\delta$ 
%     over the random draws of datasets $\wt S$ and $S$, we have:
%     \begin{align}
%         \error_\calD(\widehat f) \le \error_\calS(\widehat f) + 1 - 2 \error_{\wt\calS}(\widehat f) + \left(\frac{1}{\sqrt{2}} + 1.5 \right) \sqrt{\frac{\log(4/\delta)}{m}} + \sqrt{\frac{4}{\delta}\left(\frac{1}{m} +\frac{3\beta}{m+n} \right)}  \,. \label{eq:gd_error}
%     \end{align} 
%     % for some constant $c\le 3.2$.
% \end{theorem}

\subsection{Proof of \thmref{thm:linear}}
We use a standard result from linear algebra, 
namely the Shermann-Morrison formula 
\citep{sherman1950adjustment} for matrix inversion:  

\begin{lemma}[\citet{sherman1950adjustment}] \label{lem:sherman}
    Suppose $\bA \in \Real^{n \times n}$ 
    is an invertible square matrix 
    and $u,v \in \Real^n$ are column vectors. 
    Then $\bA + uv^T$ is invertible iff $1 + v^T \bA u \ne 0$ 
    and in particular
    \begin{align}
        (\bA + u v^T)^{-1} = \bA^{-1}  - \frac{\bA^{-1} uv^T \bA^{-1} }{ 1 + v^T \bA^{-1} u} \,.
    \end{align}   
\end{lemma}
\newcommand\byy[1]{\by_{\left(#1\right)}}
\newcommand\bXX[1]{\bX_{\left(#1\right)}}
\newcommand\ff[1]{\wh f_{\left(#1\right)}}

For a given training set $S \cup \wt S_C$, 
define leave-one-out error 
on mislabeled points in the training data 
as $$\error_{\text{LOO}(\wt S_M) } = \frac{\sum_{(x_i, y_i) \in \wt S_M} \error( f_{(i)}( x_i), y_i)}{ \abs{\wt S_M }} \,, $$
where $f_{(i)} \defeq f(\calA, (S \cup \wt S)_{(i)})$. 
To relate empirical leave-one-out error and population error 
with hypothesis stability condition, 
we use the following lemma:   

\begin{lemma}[\citet{bousquet2002stability}] \label{lem:stability_error}
    For the leave-one-out error, we have
    \begin{align}
        \Expo{ \left( \error_{\calDm}(\wh f) -\error_{\text{LOO}(\wt S_M) } \right)^2 } \le \frac{1}{2m_1}+  \frac{3\beta}{n + m}\,.
    \end{align}   
    % where $ f \defeq f(\calA, S \cup \wt S) $.
\end{lemma}

Proof of the above lemma is similar 
to the proof of Lemma 9 in \citet{bousquet2002stability} 
and can be found in \appref{app:proof_lem_error}. 
% 
% Before presenting the result, we introduce some notation. 
Before presenting the proof of \thmref{thm:linear}, 
we introduce some more notation. 
Let $\bX_{(i)}$ denote the matrix of covariates 
with the $i^{\text{th}}$ point removed. 
Similarly, let $\by_{(i)}$ be the array of responses 
with the $i^{\text{th}}$ point removed. 
Define the corresponding regularized GD solution 
as $\wh w_{(i)} = \left( \bXX{i}^T\bXX{i}+\lambda \boldsymbol{I}\right)^{-1}\bXX{i}^T\byy{i}$. 
Define $\ff{i}(x) \defeq f(x ; \wh w_{(i)}) $.

\begin{proof}[Proof of \thmref{thm:linear}]
    Because squared loss minimization does not imply 0-1 error minimization, 
    we cannot use arguments from \lemref{lem:fit_mislabeled}. 
    This is the main technical difficulty. 
    To compare the 0-1 error at a train point with an unseen point, 
    we use the closed-form expression for $\widehat{w}$ 
    and Shermann-Morrison formula 
    to upper bound training error 
    with leave-one-out cross validation error. 
    
    The proof is divided into three parts: 
    In part one, we show that 0-1 error 
    on mislabeled points in the training set 
    is lower than the error obtained 
    by leave-one-out error at those points. 
    In part two, we relate this leave-one-out error 
    with the population error on mislabeled distribution
    using \codref{cond:hypothesis_stability}.
    While the empirical leave-one-out error is an unbiased estimator 
    of the average population error of leave-one-out classifiers, 
    we need hypothesis stability 
    to control the variance 
    of empirical leave-one-out error. 
    Finally, in part three, we show 
    that the error on the mislabeled training points 
    can be estimated with just the randomly labeled 
    and clean training data (as in proof of \thmref{thm:error_ERM}).  

    \textbf{Part 1 {} {}} First we relate training error with leave-one-out error.        
    For any training point $(x_i, y_i)$ in $\wt S \cup S$, we have 
    \begin{align}
        \error(\wh f(x_i), y_i ) &= \indict{ y_i \cdot x_i^T \wh w < 0 } = \indict{ y_i \cdot x_i^T \left( \bX^T\bX+\lambda \boldsymbol{I}\right)^{-1}\bX^T\by < 0 } \\
        &= \indict{ y_i \cdot x_i^T \underbrace{\left( \bXX{i}^T\bXX{i} + x_i ^T x_i +\lambda \boldsymbol{I}\right)^{-1}}_{\RN{1}} (\bXX{i}^T\byy{i} + y_i \cdot x_i) < 0 } \,.
    \end{align}
    Letting $\bA = \left(\bXX{i}^T\bXX{i} +\lambda \boldsymbol{I}\right)$ 
    and using \lemref{lem:sherman} on term 1, we have 
    \begin{align}
        \error(\wh f(x_i), y_i ) &= \indict{ y_i \cdot x_i^T \left[\bA^{-1} -  \frac{\bA^{-1} x_i x_i^T \bA^{-1}}{ 1 + x_i ^T \bA^{-1} x_i } \right] (\bXX{i}^T\byy{i} + y_i \cdot x_i) < 0 } \\
        &= \indict{ y_i \cdot\left[ \frac{ x_i^T \bA^{-1} ( 1 + x_i ^T \bA^{-1} x_i ) -  x_i^T \bA^{-1} x_i x_i^T \bA^{-1}}{ 1 + x_i ^T \bA ^{-1}x_i } \right] (\bXX{i}^T\byy{i} + y_i \cdot x_i) < 0 } \\
        &= \indict{ y_i \cdot\left[ \frac{ x_i^T \bA^{-1}}{ 1 + x_i ^T \bA ^{-1}x_i } \right] (\bXX{i}^T\byy{i} + y_i \cdot x_i) < 0 } \,.
    \end{align}

    Since $1 + x_i^T \bA^{-1} x_i > 0$, we have 
    \begin{align}
        \error(\wh f(x_i), y_i ) &= \indict{ y_i \cdot x_i^T \bA^{-1} (\bXX{i}^T\byy{i} + y_i \cdot x_i) < 0 } \\
        &= \indict{ x_i^T \bA^{-1} x_i +  y_i \cdot x_i^T \bA^{-1} (\bXX{i}^T\byy{i}) < 0 } \\
        &\le \indict{ y_i \cdot x_i^T \bA^{-1} (\bXX{i}^T\byy{i}) < 0 } = \error(\ff{i}(x_i), y_i ) \,.\label{eq:LOO_error}
    \end{align}

    Using \eqref{eq:LOO_error}, we have 
    \begin{align}
        \error_{\wt \calS_M } (\wh f) \le \error_{\text{LOO} (\wt S_M)} \defeq \frac{\sum_{(x_i, y_i) \in \wt S_M} \error(\ff{i}(x_i), y_i ) }{\abs{\wt \calS_M}}\label{eq:LOO_error_final} \,.
    \end{align}
    \textbf{Part 2 {}{}} We now relate RHS in \eqref{eq:LOO_error_final} 
    with the population error on mislabeled distribution. 
    To do this, we leverage \codref{cond:hypothesis_stability} 
    and \lemref{lem:stability_error}. 
    In particular, we have 

    \begin{align}
        \Expt{\calS \cup \wt \calS_M }{ \left(\error_{\calDm}(\wh f) - \error_{\text{LOO} (\wt S_M)}\right)^2 } \le \frac{1}{2m_1} + \frac{3\beta}{m+n} \,.
    \end{align}

    Using Chebyshev's inequality, with probability at least $1-\delta$, we have 
    \begin{align}
        \error_{\text{LOO} (\wt S_M)} \le  \error_{\calDm}(\wh f)   + \sqrt{\frac{1}{\delta}\left(\frac{1}{2m_1} +\frac{3\beta}{m+n} \right)} \,. \label{eq:final_mislabeled_linear}
    \end{align}
    

    \textbf{Part 3 {}{}} Combining \eqref{eq:final_mislabeled_linear} and \eqref{eq:LOO_error_final}, we have 

    \begin{align}
        \error_{\wt \calS_M } (\wh f) \le \error_{\calDm}(\wh f)   + \sqrt{\frac{1}{\delta}\left(\frac{1}{2m_1} +\frac{3\beta}{m+n} \right)} \,. \label{eq:linear_parallel_lem1}
    \end{align}

    Compare \eqref{eq:linear_parallel_lem1} with \eqref{eq:lemma1_final} 
    in the proof of \lemref{lem:fit_mislabeled}. 
    We obtain a similar relationship 
    between $\error_{\wt \calS_M }$ and $\error_{\calDm}$ 
    but with a polynomial concentration 
    instead of exponential concentration. 
    In addition, since we just use concentration arguments 
    to relate mislabeled error to the errors
    on the clean and unlabeled portions 
    of the randomly labeled data, 
    we can directly use the results 
    in \lemref{lem:mislabeled_error} and \lemref{lem:clear_error}. 
    Therefore, combining results in \lemref{lem:mislabeled_error}, \lemref{lem:clear_error}, and \eqref{eq:linear_parallel_lem1} with union bound, 
    we have with probability at least $1-\delta$
    \begin{align}
        \error_\calD(\widehat f) \le \error_\calS(\widehat f) + 1 - 2 \error_{\wt\calS}(\widehat f) + \left(\sqrt{2}\error_{\wt\calS}(\widehat f) + 1 + \frac{m}{2n} \right) \sqrt{\frac{\log(4/\delta)}{m}} + \sqrt{\frac{4}{\delta}\left(\frac{1}{m} +\frac{3\beta}{m+n} \right)}  \,.
    \end{align}
    

       
\end{proof}

\subsection{Extension to multiclass classification} \label{app:multiclass_linear}
For multiclass problems with squared loss minimization, as standard practice, we consider one-hot encoding for the underlying label, i.e., a class label $c \in [k]$ is treated as $(0, \cdot, 0,1,0, \cdot, 0) \in \Real^k$ (with $c$-th coordinate being 1).  As before, we suppose that the parameters of the linear function 
are obtained via gradient descent on the following $L_2$ regularized problem: 
\begin{align}
    % n in denominator is avoided deliberately
    \calL_S(w; \lambda) \defeq \sum_{i=1}^n\norm{w^Tx_i - y_i}{2}^2 + \lambda \sum_{j=1}^k \norm{w_j}{2}^2 \,, \label{eq:l2_multiclass_MSE_app}   
\end{align}
where $\lambda\ge0$ is a regularization parameter. 
We assume access to a clean dataset 
$S = \{(x_i, y_i)\}_{i=1}^n \sim \calD^n$ 
and randomly labeled dataset 
$\wt S = \{(x_i, y_i)\}_{i=n+1}^{n+m} \sim \wt \calD^m$. 
Let $\bX = [x_1, x_2, \cdots, x_{m+n}]$ 
and $\by = [e_{y_1}, e_{y_2}, \cdots, e_{y_{m+n}}]$. 
Fix a positive learning rate $\eta$ such that 
$\eta \le 1/\left(\norm{\bX^T\bX}{\text{op}} + \lambda^2\right)$ 
and an initialization $w_0 = 0$. 
% \todos{Assumption made for simplicty}. 
Consider the following gradient descent iterates 
to minimize objective \eqref{eq:l2_MSE_app} on $S \cup \wt S$:
\begin{align}
{w_j}^t = {w_j}^{t-1} - \eta \grad_{w_j} \calL_{S \cup \wt S} (w^{t-1}; \lambda) \quad \forall t=1,2,\ldots \text{ and } j=1,2,\ldots,k  \,. \label{eq:GD_multi_iterates_app}
\end{align} 
Then we have $\{ {w_j}^t\}$ for all $j =1,2,\cdots, k$ converge to the limiting solution 
$\wh w_j = \left( \bX^T\bX+\lambda \boldsymbol{I}\right)^{-1}\bX^T\by_j$. Define $\widehat f (x) \defeq f(x ; \wh w) $.  

\begin{theorem}\label{thm:multi_linear}
    Assume that this gradient descent algorithm satisfies \codref{cond:hypothesis_stability}
    with $\beta=\calO(1)$.  
    Then for a multiclass classification problem wth $k$ classes, for any $\delta >0$, with probability at least $1-\delta$, we have:
    \begin{align*}
        \error_\calD(\widehat f) \le \error_\calS(\widehat f) &+ (k-1)\left(1 - \frac{k}{k-1} \error_{\wt\calS}(\widehat f) \right) \\ &+ \left(k + \sqrt{k} + \frac{m}{n\sqrt{k}} \right) \sqrt{\frac{\log(4/\delta)}{2m}} + \sqrt{k(k-1)} \sqrt{\frac{4}{\delta}\left(\frac{1}{m} +\frac{3\beta}{m+n} \right)}  \,. \numberthis \label{eq:gd_multi_error}
    \end{align*} 
    % for some constant $c\le 3.2$.
\end{theorem}
\begin{proof}
    The proof of this theorem is divided into two parts. In the first part, we relate the error on the mislabeled samples with the population error on the mislabeled data. Similar to the proof of \thmref{thm:linear}, we use Shermann-Morrison formula to upper bound training error with leave-one-out error on each $\wh w^j$. Second part of the proof follows entirely from the proof of \thmref{thm:multiclass_ERM}. In essence, the first part derives an equivalent of \eqref{eq:lemma1_final_multi_prev} for GD training with squared loss and then the second part follows from the proof  of \thmref{thm:multiclass_ERM}. 
    
    \textbf{Part-1:} Consider a training point $(x_i,y_i)$ in $\wt S \cup S $. For simplicity, we use $c_i$ to denote the class of $i$-th point and use $y_i$ as the corresponding one-hot embedding. Recall error in multiclass point is given by $\error(\wh f(x_i), y_i ) = \indict{ c_i \not \in \argmax x_i^T \wh w }$. Thus, there exists a $j \ne c_i \in [k]$, such that we have
     \begin{align}
        \error(\wh f(x_i), y_i ) &= \indict{ c_i \not \in \argmax x_i^T \wh w } = \indict{ x_i^T \wh w_{c_i} < x_i^T \wh w_{j}  } \\ &= \indict{ x_i^T \left( \bX^T\bX+\lambda \boldsymbol{I}\right)^{-1}\bX^T\by_{c_i} < x_i^T \left( \bX^T\bX+\lambda \boldsymbol{I}\right)^{-1}\bX^T\by_{j} } \\
        &= \indict{ x_i^T \underbrace{\left( \bXX{i}^T\bXX{i} + x_i ^T x_i +\lambda \boldsymbol{I}\right)^{-1}}_{\RN{1}} \left(\bXX{i}^T{\by_{c_i}}_{(i)} + x_i - \bXX{i}^T{\by_{j}}_{(i)}\right) < 0 } \,.
    \end{align}
    Letting $\bA = \left(\bXX{i}^T\bXX{i} +\lambda \boldsymbol{I}\right)$ 
    and using \lemref{lem:sherman} on term 1, we have 
    \begin{align}
        \error(\wh f(x_i), y_i ) &= \indict{ x_i^T \left[\bA^{-1} -  \frac{\bA^{-1} x_i x_i^T \bA^{-1}}{ 1 + x_i ^T \bA^{-1} x_i } \right]  \left(\bXX{i}^T{\by_{c_i}}_{(i)} + x_i - \bXX{i}^T{\by_{j}}_{(i)}\right) < 0 } \\
        &= \indict{ \left[ \frac{ x_i^T \bA^{-1} ( 1 + x_i ^T \bA^{-1} x_i ) -  x_i^T \bA^{-1} x_i x_i^T \bA^{-1}}{ 1 + x_i ^T \bA ^{-1}x_i } \right]  \left(\bXX{i}^T{\by_{c_i}}_{(i)} + x_i - \bXX{i}^T{\by_{j}}_{(i)}\right) < 0 } \\
        &= \indict{ \left[ \frac{ x_i^T \bA^{-1}}{ 1 + x_i ^T \bA ^{-1}x_i } \right]  \left(\bXX{i}^T{\by_{c_i}}_{(i)} + x_i - \bXX{i}^T{\by_{j}}_{(i)}\right) < 0} \,.
    \end{align}
    Since $1 + x_i^T \bA^{-1} x_i > 0$, we have 
    \begin{align}
        \error(\wh f(x_i), y_i ) &= \indict{ x_i^T \bA^{-1}  \left(\bXX{i}^T{\by_{c_i}}_{(i)} + x_i - \bXX{i}^T{\by_{j}}_{(i)}\right) < 0 } \\
        &= \indict{ x_i^T \bA^{-1} x_i +  x_i^T \bA^{-1}  \bXX{i}^T{\by_{c_i}}_{(i)}  - x_i^T\bA^{-1}  \bXX{i}^T{\by_{j}}_{(i)} < 0 } \\
        &\le \indict{  x_i^T \bA^{-1}  \bXX{i}^T{\by_{c_i}}_{(i)}  - x_i^T\bA^{-1}  \bXX{i}^T{\by_{j}}_{(i)} < 0  } = \error(\ff{i}(x_i), y_i ) \,.\label{eq:LOO_error_multi}
    \end{align}
    Using \eqref{eq:LOO_error_multi}, we have 
    \begin{align}
        \error_{\wt \calS_M } (\wh f) \le \error_{\text{LOO} (\wt S_M)} \defeq \frac{\sum_{(x_i, y_i) \in \wt S_M} \error(\ff{i}(x_i), y_i ) }{\abs{\wt \calS_M}}\label{eq:LOO_error_multi_final} \,.
    \end{align}
    
    We now relate RHS in \eqref{eq:LOO_error_final} 
    with the population error on mislabeled distribution. 
    Similar as before, to do this, we leverage \codref{cond:hypothesis_stability} 
    and \lemref{lem:stability_error}. Using  \eqref{eq:final_mislabeled_linear} and \eqref{eq:LOO_error_multi_final}, we have 
    \begin{align}
        \error_{\wt \calS_M } (\wh f) \le \error_{\calDm}(\wh f)   + \sqrt{\frac{1}{\delta}\left(\frac{1}{2m_1} +\frac{3\beta}{m+n} \right)} \,. \label{eq:linear_multi_parallel_lem1}
    \end{align}
    
    We have now derived a parallel to \eqref{eq:lemma1_final_multi_prev}. Using the same arguments in the proof of \lemref{lem:fit_mislabeled_multi}, we have 
    \begin{align}
      \error_{\calD}(\wh f) \le  (k-1) \left( 1- \error_{ \wt \calS_M}(\wh f) \right)  + (k-1)\sqrt{\frac{k}{\delta(k-1)}\left(\frac{1}{2m_1} +\frac{3\beta}{m+n} \right)}  \,. \label{eq:lemma1_linear_final_multi}
    \end{align}
    
    \textbf{Part-2:} We now combine the results in \lemref{lem:mislabeled_error_multi} and \lemref{lem:clear_error_multi} to obtain the final inequality in terms of quantities that can be computed from just the randomly labeled and clean data. Similar to the binary case, we obtained a polynomial concentration instead of exponential concentration. Combining \eqref{eq:lemma1_linear_final_multi} with \lemref{lem:mislabeled_error_multi} and \lemref{lem:clear_error_multi}, we have with probability at least $1-\delta$
    \begin{align*}
        \error_\calD(\widehat f) \le \error_\calS(\widehat f) &+ (k-1)\left(1 - \frac{k}{k-1} \error_{\wt\calS}(\widehat f) \right) \\ &+ \left(k + \sqrt{k} + \frac{m}{n\sqrt{k}} \right) \sqrt{\frac{\log(4/\delta)}{2m}} + \sqrt{k(k-1)} \sqrt{\frac{4}{\delta}\left(\frac{1}{m} +\frac{3\beta}{m+n} \right)}  \,. \numberthis \label{eq:gd_multi_error_proof}
    \end{align*} 
\end{proof}

\subsection{Discussion on \codref{cond:hypothesis_stability}} \label{app:discuss_cond1}
The quantity in LHS of \codref{cond:hypothesis_stability} 
measures how much the function learned by the algorithm 
(in terms of error on unseen point) will change 
when one point in the training set is removed. 
% Discussion on exponential concentration and stronger condition. 
% Notice that hypothesis stability implies error stability, i.e., \codref{cond:error_stability} \citep{bousquet2002stability}.  
% In summary, while error stability allowed us 
% to relate the average population error 
% of the leave-one-out classifiers 
% with the population error of the original classifier, 
We need hypothesis stability condition 
to control the variance of the empirical leave-one-out error to show concentration of average leave-one-error with the population error. 

Additionally, we note that while the dominating term in the RHS of \thmref{thm:linear} matches with the dominating term in ERM bound in \thmref{thm:error_ERM}, there is a polynomial concentration term 
(dependence on $1/\delta$ instead of $\log(\sqrt{1/\delta})$) 
in \thmref{thm:linear}. 
Since with hypothesis stability, 
we just bound the variance, 
the polynomial concentration is due 
to the use of Chebyshev's inequality 
instead of an exponential tail inequality
(as in \lemref{lem:fit_mislabeled}).
Recent works have highlighted that 
a slightly stronger condition than hypothesis stability 
can be used to obtain an exponential concentration 
for leave-one-out error \citep{abou2019exponential},
but we leave this for future work for now. 
% We leave 
% However, the constants 

% we also want to highlight  

\subsection{Formal statement and proof of \propref{prop:early_stop}} \label{app:formal_early_stop}

Before formally presenting the result, 
we will introduce some notation.  
By $\calL_{S}(w)$, we denote 
the objective in \eqref{eq:l2_MSE_app} with $\lambda=0$. 
Assume Singular Value Decomposition (SVD) of $\bX$
as $\sqrt{n} \bU \bS^{1/2} \bV^T$. 
Hence $\bX^T \bX = \bV \bS \bV^T$.
Consider the GD iterates defined in \eqref{eq:GD_iterates_app}. 
% 
We now derive closed form expression 
for the $t^\text{th}$ iterate of gradient descent:  
% 
\begin{align}
    w_t = w_{t-1} + \eta \cdot \bX^T (\by - \bX w_{t-1}) = (\bI - \eta \bV \bS \bV^T )w_{k-1} + \eta \bX^T \by \,.
\end{align}
Rotating by $\bV^T$, we get 
\begin{align}
    \wt w_t = (\bI - \eta\bS )\wt w_{k-1} + \eta \wt \by \label{eq:GD_recur},
\end{align}
where $\wt w_t = \bV^T w_t $ and $\wt \by = \bV^T \bX^T \by$. 
Assuming the initial point $w_0 = 0$ 
and applying the recursion in \eqref{eq:GD_recur}, we get
\begin{align}
    \wt w_t = \bS ^{-1} ( \bI - (\bI - \eta \bS)^k ) \wt \by \,, 
\end{align} 
Projecting solution back to the original space, we have 
\begin{align}
     w_t = \bV \bS ^{-1} ( \bI - (\bI - \eta \bS)^k ) \bV^T \bX^T \by \,. 
\end{align} 
% We will work with this GD solution at any iterate $t$ in the next proposition. 
Define $f_t(x) \defeq f(x;w_t)$ 
as the solution at the $t^{\text{th}}$ iterate. 
Let $\wt w_{\lambda} = \argmin_{w} \calL_\calS (w;\lambda) = (\bX^T \bX + \lambda \bI)^{-1} \bX^T \by = \bV (\bS + \lambda \bI )^{-1} \bV^T \bX^T \by $. 
% ) \,,$ for all $t=1,2,\ldots\,.$ 
and define $\wt f_\lambda(x) \defeq f(x;\wt w_\lambda)$ as the regularized solution. 
Assume $\kappa$ be the condition number 
of the population covariance matrix 
and let $s_\text{min}$ be the minimum positive 
singular value of the empirical covariance matrix. 
Our proof idea is inspired from recent work 
on relating gradient flow solution 
and regularized solution 
for regression problems \citep{ali2018continuous}. 
We will use the following lemma in the proof: 
\begin{lemma} \label{lem:ineq_soln}
    For all $x \in [0,1]$ and for all $ k \in \mathbb{N}$, 
    we have (a) $ \frac{kx}{1+kx} \le 1- (1-x)^k$ 
    and (b) $ 1- (1-x)^k \le 2 \cdot \frac{kx}{kx+1} $.
    %  where $g(c)$ is a constant dependent on $c$. For $c = 1$, $g(c) = 2.0$.   
\end{lemma}
\begin{proof}
    % [Proof of \lemref{lem:ineq_soln}]
    % Part (a) is easy. 
    Using $ (1-x)^k \le \frac{1}{1+kx}$, we have part (a). 
    For part (b), we numerically maximize 
    $\frac{ (1+kx ) (1 - (1-x)^k) }{kx}$ 
    for all $k\ge 1$ and for all $x \in [0, 1]$.  
\end{proof}

% 
% Next, 

\begin{prop}[Formal statement of \propref{prop:early_stop}] \label{prop:formal_early_stop}
Let $\lambda = \frac{1}{t\eta}$. 
For a training point $x$, we have 
\begin{align*}
    \Expt{x \sim \calS}{(f_t(x) - \wt f_\lambda(x))^2} &\le c(t,\eta) \cdot \Expt{x \sim \calS}{f_t(x)^2} \,, %\label{eq:early_stop}
\end{align*}
where $c(t, \eta) \defeq \min( 0.25, \frac{1}{s_\text{min}^2 t^2 \eta^2})$. 
Similarly for a test point, we have 
\begin{align*}
    \Expt{x \sim \calD_\calX}{(f_t(x) - \wt f_\lambda(x))^2} &\le \kappa \cdot c(t,\eta) \cdot \Expt{x \sim \calD_\calX}{f_t(x)^2} \,. %\label{eq:early_stop}
\end{align*}
\end{prop} 

\begin{proof}
    %%%%%%%%%%%%% 
    We want to analyze the expected squared difference output 
    of regularized linear regression 
    with regularization constant $\lambda = \frac{1}{\eta t}$ 
    and the gradient descent solution at the $t^\text{th}$ iterate. 
    We separately expand the algebraic expression 
    for squared difference at a training point and a test point. 
    % We start by considering the difference  
    Then the main step is to show that 
    $\left[ \bS ^{-1} ( \bI - (\bI - \eta \bS)^k )  - (\bS + \lambda \bI )^{-1}\right] \preceq c(\eta, t) \cdot \bS ^{-1} ( \bI - (\bI - \eta \bS)^k ) $.

    %%%%%%%%%%%%%
    
   \textbf{Part 1 {} {}} 
    First, we will analyze the squared difference 
    of the output at a training point 
    (for simplicity, we refer to $S \cup \wt S$ as $S$), i.e., 
    \begin{align}
        \Expt{ x \sim \calS }{\left(f_t(x) - \wt f_\lambda (x)\right)^2} &= \norm{\bX w_t - \bX \wt w_\lambda}{2}^2\\ &=   \norm{\bX \bV \bS ^{-1} ( \bI - (\bI - \eta \bS)^t ) \bV^T \bX^T \by - \bX \bV (\bS + \lambda \bI )^{-1} \bV^T \bX^T \by }{2}^2 \\
        &= \norm{\bX \bV \left(\bS ^{-1} ( \bI - (\bI - \eta \bS)^t ) - (\bS + \lambda \bI )^{-1} \right) \bV^T \bX^T \by  }{2} \\
        &=  \by^T \bV \bX \left( \underbrace{\bS ^{-1} ( \bI - (\bI - \eta \bS)^t ) - (\bS + \lambda \bI )^{-1}}_{\RN{1}} \right)^2 \bS \bV^T \bX^T \by \label{eq:train_GD_rel} \,.
        %  (\bX \bV \bS ^{-1} ( \bI - (\bI - \eta \bS)^k ) \bV^T \bX^T \by)^T \bX \bV \bS ^{-1} ( \bI - (\bI - \eta \bS)^k ) \bV^T \bX^T \by
    \end{align}
    We now separately consider term 1. 
    Substituting $\lambda = \frac{1}{t \eta}$, 
    we get
    \begin{align}
        \bS ^{-1} ( \bI - (\bI - \eta \bS)^t ) - (\bS + \lambda \bI )^{-1} &= \bS^{-1} \left( ( \bI - (\bI - \eta \bS)^t ) - (\bI + \bS^{-1} \lambda )^{-1}\right) \\
        &= \underbrace{\bS^{-1} \left( ( \bI - (\bI - \eta \bS)^t ) - (\bI + ( \bS t \eta)^{-1}  )^{-1}\right)}_{\bA} \,.
    \end{align}

    We now separately bound the diagonal entries in matrix $\bA$. 
    With $s_i$, we denote $i^{\text{th}}$ diagonal entry of $\bS$.
    Note that since $ \eta\le 1/\norm{S}{\text{op}}$, 
    for all $i$, $\eta s_i  \le 1$.  
    Consider $i^{\text{th}}$ diagonal term (which is non-zero) 
    of the diagonal matrix $\bA$, we have 
    \begin{align}
        \bA_{ii} = \frac{1}{s_i} \left(  1 - (1 - s_i \eta)^t - \frac{t \eta s_i}{1 + t \eta s_i } \right) &=  \frac{1 - (1 - s_i \eta)^t}{s_i} \left( \underbrace{ 1 - \frac{t \eta s_i}{(1 + t \eta s_i)(1 - (1 - s_i \eta)^t)}}_{\RN{2}} \right) \\ 
         &\le \frac{1}{2}\left[ \frac{1 - (1 - s_i \eta)^t}{ s_i} \right] \tag*{(Using \lemref{lem:ineq_soln} (b))} \,.
    \end{align} 
    Additionally, we can also show the following upper bound on term 2: 
    \begin{align}
         1 - \frac{t \eta s_i}{(1 + t \eta s_i)(1 - (1 - s_i \eta)^t)} &= \frac{(1 + t \eta s_i)(1 - (1 - s_i \eta)^t) - t \eta s_i }{(1 + t \eta s_i)(1 - (1 - s_i \eta)^t)} \\
         & \le  \frac{ 1 -  (1 - s_i \eta)^t - t \eta s_i (1 - s_i \eta)^t}{(1 + t \eta s_i)(1 - (1 - s_i \eta)^t)} \\
         & \le \frac{1}{t\eta s_i} \,. \tag{Using \lemref{lem:ineq_soln} (a)}
        %  &\le \frac{1}{2}\left[ \frac{1 - (1 - s_i \eta)^t}{ s_i} \right] \tag*{(Using \lemref{lem:ineq_soln})} \,.
    \end{align} 

    Combining both the upper bounds 
    on each diagonal entry $\bA_{ii}$, we have 
    \begin{align}
    \bA \preceq c_1(\eta, t) \cdot \bS^{-1} ( \bI - (\bI - \eta \bS)^t ) \,, \label{eq:upperbound_diagonal}
    \end{align}
    where $c_1(\eta, t ) = \min(0.5, \frac{1}{t s_i \eta })$. Plugging this into \eqref{eq:train_GD_rel}, we have 
    \begin{align}
        \Expt{ x \sim \calS }{\left(f_t(x) - \wt f_\lambda (x)\right)^2} &\le c(\eta, t) \cdot \by^T \bV \bX  \left( \bS^{-1} ( \bI - (\bI - \eta \bS)^t ) \right)^2 \bS \bV^T \bX^T \by \\
        &=   c(\eta, t) \cdot \by^T \bV \bX  \left( \bS^{-1} ( \bI - (\bI - \eta \bS)^t ) \right) \bS \left( \bS^{-1} ( \bI - (\bI - \eta \bS)^t ) \right) \bV^T \bX^T \by \\
        & =  c(\eta, t) \cdot \norm{\bX w_t}{2}^2 \\
        &= c(\eta, t) \cdot  \Expt{ x \sim \calS }{\left(f_t(x) \right)^2} \,,
    \end{align}
    where $c(\eta, t ) = \min(0.25, \frac{1}{t^2 s^2_i \eta^2 })$.

    \textbf{Part 2 {} {}} With $\bSigma$, 
    we denote the underlying true covariance matrix. 
    We now consider the squared difference of output at an unseen point: 
    \begin{align}
        \Expt{ x \sim \calD_{\calX} }{\left(f_t(x) - \wt f_\lambda (x)\right)^2} &= \Expt{x \sim \calD_{\calX}}{\norm{x^T w_t - x^T \wt w_\lambda}{2}} \\
        &=   \norm{x^T \bV \bS ^{-1} ( \bI - (\bI - \eta \bS)^t ) \bV^T \bX^T \by - x^T \bV (\bS + \lambda \bI )^{-1} \bV^T \bX^T \by }{2} \\
        &= \norm{x^T \bV \left(\bS ^{-1} ( \bI - (\bI - \eta \bS)^t ) - (\bS + \lambda \bI )^{-1} \right) \bV^T \bX^T \by  }{2} \\
        &= \by^T \bV \bX \left( \bS ^{-1} ( \bI - (\bI - \eta \bS)^t ) - (\bS + \lambda \bI )^{-1} \right) \bV^T \bSigma \bV \\ &\qquad \qquad \qquad \qquad \qquad \left( (\bI - (\bI - \eta \bS)^t ) - (\bS + \lambda \bI )^{-1} \right) \bV^T \bX^T \by \\
        &\le \sigma_{\text{max}} \cdot \by^T \bV \bX \left( \underbrace{\bS ^{-1} ( \bI - (\bI - \eta \bS)^t ) - (\bS + \lambda \bI )^{-1}}_{\RN{1}} \right)^2 \bV^T \bX^T \by \,, \label{eq:test_GD_rel}
        %  (\bX \bV \bS ^{-1} ( \bI - (\bI - \eta \bS)^k ) \bV^T \bX^T \by)^T \bX \bV \bS ^{-1} ( \bI - (\bI - \eta \bS)^k ) \bV^T \bX^T \by
    \end{align}
    where $\sigma_{\text{max}}$ is the maximum eigenvalue 
    of the underlying covariance matrix $\bSigma$. 
    Using the upper bound on term 1 in \eqref{eq:upperbound_diagonal}, 
    we have 
    \begin{align}
        \Expt{ x \sim \calD_{\calX} }{\left(f_t(x) - \wt f_\lambda (x)\right)^2} &\le \sigma_{\text{max}} \cdot c(\eta, t) \cdot \by^T \bV \bX  \left( \bS^{-1} ( \bI - (\bI - \eta \bS)^t ) \right)^2 \bV^T \bX^T \by \\
        &=   \kappa \cdot c(\eta, t) \cdot \sigma_{\text{min}}\cdot \norm{\bV \left( \bS^{-1} ( \bI - (\bI - \eta \bS)^t ) \right) \bV^T \bX^T \by}{2}^2 \\
        &\le \kappa \cdot c(\eta, t) \cdot \left[ \bV \left( \bS^{-1} ( \bI - (\bI - \eta \bS)^t ) \right) \bV^T \bX^T \right]^T \bSigma \\
        &\qquad \qquad \qquad \qquad \qquad \left[ \bV \left( \bS^{-1} ( \bI - (\bI - \eta \bS)^t ) \right) \bV^T \bX^T \right] \by \\
        & = \kappa \cdot c(\eta, t) \cdot \Expt{x \sim \calD_{\calX}}{\norm{x^T w_t}{2}} \,.
    \end{align}
% 
% 
    % Since $ \eta\le 1/\norm{S}{\text{op}}$, invoking \lemref{lem:ineq_soln} to upper bound term 1 with
\end{proof}

\subsection{Extension to deep learning} \label{appsubsec:ext_DL}
Under \asmpref{appsubsec:justifying_assumption1}, we present the formal result parallel to \thmref{thm:multiclass_ERM}. 
\begin{theorem} \label{thm:multiclass_ERM_algoA}
    Consider a multiclass classification problem 
    with $k$ classes. Under \asmpref{asmp:deep_models}, 
    for any $\delta >0$, with probability at least $1-\delta$,
    we have
    \vspace{-10pt}
    \begin{align*}
        \error_\calD(\widehat f)  \le \error_\calS(\widehat f) + (k-1) \left(1 - \tfrac{k}{k-1} \error_{\wt\calS}(\widehat f)\right) + c\sqrt{\frac{\log(\frac{4}{\delta})}{2m}} \,,\numberthis \label{eq:multiclass_ERM_deep}
    % \vspace{-20pt}
    \end{align*}
    for some constant $c \le ((c+1) k+\sqrt{k} + \frac{m}{n\sqrt{k}})$.
\end{theorem}

The proof follows exactly as in step (i) to (iii) in \thmref{thm:multiclass_ERM}.  

\subsection{Justifying~\asmpref{asmp:deep_models}} \label{appsubsec:justifying_assumption1}

Motivated by the analysis on linear models, we now discuss alternate (and weaker) conditions that imply \asmpref{asmp:deep_models}. 
We need hypothesis stability (\codref{cond:hypothesis_stability}) and the following assumption relating training error and leave-one-error: 

\begin{assumption} \label{asmp:loo_error}
Let $\wh f$ be a model obtained by training with algorithm $\calA$ on a mixture of clean $S$ and randomly labeled data $\wt S$. Then we assume we have 
\begin{align*}
    \error_{\wt \calS_M} (\wh f) \le  \error_{\text{LOO} (\wt S_M)} \,, 
\end{align*}
for all $(x_i, y_i) \in  \wt S_M$ where $\wh f_{(i)} \defeq f(\calA, S \cup {{}\wt S_M}_{(i)})$ and  $\error_{\text{LOO} (\wt S_M)} \defeq  \frac{\sum_{(x_i, y_i) \in \wt S_M} \error(\ff{i}(x_i), y_i ) }{\abs{\wt \calS_M}}$.  
\end{assumption}

% we assume this to extend our result (parallel to \thmref{thm:multi_linear}) for deep models. 
Intuitively, this assumption states that the error on a (mislabeled) datum $(x,y)$ included in the training set is less than the error on that datum $(x,y)$ obtained by a model trained on the training set $S - \{(x,y)\}$. We proved this for linear models trained with GD in the proof of \thmref{thm:multi_linear}. 
% 
\codref{cond:hypothesis_stability} with $\beta = \calO(1)$ and \asmpref{asmp:loo_error} together with \lemref{lem:stability_error} implies \asmpref{asmp:deep_models} with a polynomial residual term (instead of logarithmic in $1/\delta$): 
\begin{align}
     \error_{\calS_M} (\wh f) \le  \error_{\calDm}(\wh f)   + \sqrt{\frac{1}{\delta}\left(\frac{1}{m} +\frac{3\beta}{m+n} \right)} \,.
\end{align}
% Note that this  

\newpage 
\section{Additional experiments and details}\label{app:exp}
\newcommand\tab[1][1cm]{\hspace*{#1}}

\subsection{Datasets} \label{sec:app_dataset}

\textbf{Toy Dataset {} {}} Assume fixed constants $\mu$ and $\sigma$. For a given label $y$, we simulate features $x$ in our toy classification setup as follows: 
\begin{align*}
    x \defeq \texttt{concat} \left[ x_1, x_2\right] \quad \text{where} \quad  x_1 \sim  \calN( y \cdot \mu, \sigma^2 I_{d \times d}) \ \  \text{and} \ \  x_1 \sim  \calN( 0, \sigma^2 I_{d \times d}) \,.
\end{align*}  
% where $y$ is the true label and $x$ is the corresponding feature vector. 
In experiements throughout the paper, we fix dimention $d=100$, $\mu = 1.0 $, and $\sigma = \sqrt{d}$. Intuitively, $x_1$ carries the information about the underlying label and $x_2$ is additional noise independent of the underlying label. 

\textbf{CV datasets {} {}} We use MNIST~\citep{lecun1998mnist} and CIFAR10~\cite{krizhevsky2009learning}. 
% For binary tasks, 
We produce a binary variant from the multiclass classification problem by mapping classes $\{0,1,2,3,4\}$ to label $1$ and $\{ 5,6,7,8,9\}$ to label $-1$. For CIFAR dataset, we also use the standard data augementation of random crop and horizontal flip. PyTorch code is as follows: 

\texttt{(transforms.RandomCrop(32, padding=4),\\
\tab transforms.RandomHorizontalFlip())}

\textbf{NLP dataset {} {}} We use IMDb Sentiment analysis~\citep{maas2011learning} corpus.  

\subsection{Architecture Details} 

All experiments were run on NVIDIA GeForce RTX 2080 Ti GPUs. We used PyTorch~\citep{NEURIPS2019a9015} and Keras with Tensorflow~\citep{abadi2016tensorflow} backend for experiments. 
% , ELMo embeddings~\citep{Peters:2018}, and Hugging Face Transformers~\citep{wolf-etal-2020-transformers}. 

\textbf{Linear model {} {}} For the toy dataset, we simulate a linear model with scalar output and the same number of parameters as the number of dimensions.   

\textbf{Wide nets {} {}} To simulate the NTK regime, we experiment with $2-$layered wide nets. The PyTorch code for 2-layer wide MLP is as follows: 


\texttt{ nn.Sequential( \\
\tab     nn.Flatten(),\\
\tab    nn.Linear(input\_dims, 200000, bias=True),\\
\tab    nn.ReLU(),\\
\tab    nn.Linear(200000, 1, bias=True)\\
\tab     )}


We experiment both (i) with the second layer fixed at random initialization; (ii)  and updating both layers' weights.     

\textbf{Deep nets for CV tasks {} {}} We consider a 4-layered MLP. The PyTorch code for 4-layer MLP is as follows: 

\texttt{ nn.Sequential(nn.Flatten(), \\
\tab        nn.Linear(input\_dim, 5000, bias=True),\\
\tab        nn.ReLU(),\\
\tab        nn.Linear(5000, 5000, bias=True),\\
\tab        nn.ReLU(),\\
\tab        nn.Linear(5000, 5000, bias=True),\\
\tab        nn.ReLU(),\\
% \tab        nn.Linear(5000, 5000, bias=True),\\
% \tab        nn.ReLU(),\\
\tab        nn.Linear(1024, num\_label, bias=True)\\
\tab        )}

For MNIST, we use $1000$ nodes instead of $5000$ nodes in the hidden layer. 
% 
We also experiment with convolutional nets. In particular, we use ResNet18 \citep{he2016deep}. Implementation adapted from:  \url{https://github.com/kuangliu/pytorch-cifar.git}. 

\textbf{Deep nets for NLP {} {}} We use a simple LSTM model with embeddings intialized with ELMo embeddings~\citep{Peters:2018}. Code adapted from: \url{https://github.com/kamujun/elmo_experiments/blob/master/elmo_experiment/notebooks/elmo_text_classification_on_imdb.ipynb} 

We also evaluate our bounds with a BERT model. In particular, we fine-tune an off-the-shelf uncased BERT model~\citep{devlin2018bert}. Code adapted from Hugging Face Transformers~\citep{wolf-etal-2020-transformers}: \url{https://huggingface.co/transformers/v3.1.0/custom_datasets.html}. 


\subsection{Additonal experiments}

\textbf{Results with SGD on underparameterized linear models {} {}} 

\begin{figure*}[h]
    \centering 
    % \vspace{-15pt}
    % \includegraphics[width=0.9\linewidth]{example-image-a}
    \includegraphics[width=0.3\linewidth]{figures/lowdim-Gaussian-SGD.pdf}
    % \includegraphics[width=0.9\linewidth]{figures/{CIFAR10_rn=0.1_lr=0.2_wd=0.005}.png}
    \vspace{-5pt}
    \caption{ 
    % Predicted lower bound 
    % on different
    We plot the accuracy and corresponding bound 
    (RHS in \eqref{eq:erm}) at $\delta = 0.1$
    for toy binary classification task. 
    Results aggregated over $3$ seeds. 
    % i.e., $1-\error$ where $\error$ is the term in the RHS of \eqref{eq:erm}
    Accuracy vs fraction of unlabeled data (w.r.t clean data) 
    in the toy setup with a linear model trained with SGD. Results parallel to \figref{fig:error_binary}(a) with SGD.  }
    \label{fig:error_binary_linear}
    \vspace{-5pt}
\end{figure*}

\textbf{Results with wide nets on binary MNIST {} {}}

\begin{figure*}[h]
    \centering 
    % \vspace{-15pt}
    % \includegraphics[width=0.9\linewidth]{example-image-a}
    \subfigure[GD with MSE loss]{\includegraphics[width=0.3\linewidth]{figures/MNIST-GD_MSE.pdf}} \hfil
    \subfigure[SGD with CE loss]{\includegraphics[width=0.3\linewidth]{figures/MNIST-SGD_CE.pdf}}
    \subfigure[SGD with MSE loss]{\includegraphics[width=0.3\linewidth]{figures/MNIST-SGD_MSE-first-layer.pdf}}
    % \includegraphics[width=0.9\linewidth]{figures/{CIFAR10_rn=0.1_lr=0.2_wd=0.005}.png}
    \vspace{-5pt}
    \caption{ 
    % Predicted lower bound 
    % on different
    We plot the accuracy and corresponding bound 
    (RHS in \eqref{eq:erm}) at $\delta = 0.1$ 
    for binary MNIST classification. 
    Results aggregated over $3$ seeds. 
    % i.e., $1-\error$ where $\error$ is the term in the RHS of \eqref{eq:erm}
    Accuracy vs fraction of unlabeled data 
    for a 2-layer wide network on binary MNIST with both the layers training in (a,b) and only first layer training in (c). 
    Results parallel to \figref{fig:error_binary}(b) .  }
    \label{fig:error_binary_MNIST}
    \vspace{-5pt}
\end{figure*}

% \begin{figure*}[h]
%     \centering 
%     % \vspace{-15pt}
%     % \includegraphics[width=0.9\linewidth]{example-image-a}
%     \subfigure[GD with MSE loss]{\includegraphics[width=0.3\linewidth]{figures/MNIST.pdf}} \hfil
    
%     \subfigure[SGD with CE loss]{\includegraphics[width=0.3\linewidth]{figures/MNIST.pdf}}
%     % \includegraphics[width=0.9\linewidth]{figures/{CIFAR10_rn=0.1_lr=0.2_wd=0.005}.png}
%     \vspace{-5pt}
%     \caption{ 
%     % Predicted lower bound 
%     % on different
%     We plot the accuracy and corresponding bound 
%     (RHS in \eqref{eq:erm}) at $\delta = 0.1$
%     for binary MNIST classification. 
%     Results aggregated over $3$ seeds. 
%     % i.e., $1-\error$ where $\error$ is the term in the RHS of \eqref{eq:erm}
%     Accuracy vs fraction of unlabeled data 
%     for a 2-layer wide network on binary MNIST with just the first layer training. 
%     Results parallel to \figref{fig:error_binary}(b) with only the first layer training.  }
%     \label{fig:error_binary_MNIST}
%     \vspace{-5pt}
% \end{figure*}

\textbf{Results on CIFAR 10 and MNIST {} {}} 
% 
We plot epoch wise error curve for results in \tabref{table:multiclass}(\figref{fig:error_epoch_CIFAR10} and \figref{fig:error_epoch_MNIST}). We observe the same trend as in \figref{fig:error_CIFAR10}. Additionally, we plot an \emph{oracle bound} obtained by tracking the error on mislabeled data which nevertheless were predicted as true label. To obtain an exact emprical value of the oracle bound, we need underlying true labels for the randomly labeled data. 
% Note that our bound in \thmref{thm:multiclass_ERM}, lower bounds the accuracy as predicted by the oracle bound. 
While with just access to extra unlabeled data we cannot calculate oracle bound, we note that the oracle bound is very tight and never violated in practice underscoring an importamt aspect of generalization in multiclass problems. This highlight that even a stronger conjecture may hold in multiclass classification, i.e., error on mislabeled data (where nevertheless true label was predicted) lower bounds the population error on the distribution of mislabeled data and hence, the error on (a specific) mislabeled portion predicts the population accuracy on clean data. 
% 
On the other hand, the dominating term of in \thmref{thm:multiclass_ERM} is loose when compared with the oracle bound. The main reason, we believe is the pessimistic upper bound in \eqref{eq:lemma1_final_multi_prev} in the proof of \lemref{lem:fit_mislabeled_multi}. We leave an investigation on this gap for future. 
% of fit 

% However, oracle bound highlights two . One,  



\begin{figure}[h]
    \centering 
    % \vspace{-15pt}
    % \includegraphics[width=0.9\linewidth]{example-image-a}
    \subfigure[MLP]{\includegraphics[width=0.3\linewidth]{figures/CIFAR10-FNN.pdf}} \hfil
    \subfigure[ResNet]{\includegraphics[width=0.3\linewidth]{figures/CIFAR10-Resnet.pdf}}
    % \includegraphics[width=0.9\linewidth]{figures/{CIFAR10_rn=0.1_lr=0.2_wd=0.005}.png}
    % \vspace{-10pt}
    \caption{ Per epoch curves for CIFAR10 corresponding results in \tabref{table:multiclass}. As before, we just plot the dominating term in the RHS of \eqref{eq:multiclass_ERM} as predicted bound. Additionally, we also plot the predicted lower bound by the error on mislabeled data which nevertheless were predicted as true label. We refer to this as ``Oracle bound''. See text for more details. 
    % 
    % except for the stopping point. 
    % The bound predicted by RATT (RHS in \eqref{eq:multiclass_ERM}) is vacuous. 
    }\label{fig:error_epoch_CIFAR10}
    % \vspace{-15pt}
\end{figure}


\begin{figure}[h]
    \centering 
    % \vspace{-15pt}
    % \includegraphics[width=0.9\linewidth]{example-image-a}
    \subfigure[MLP]{\includegraphics[width=0.3\linewidth]{figures/MNIST-FNN.pdf}} \hfil
    \subfigure[ResNet]{\includegraphics[width=0.3\linewidth]{figures/MNIST-Resnet.pdf}}
    % \includegraphics[width=0.9\linewidth]{figures/{CIFAR10_rn=0.1_lr=0.2_wd=0.005}.png}
    % \vspace{-10pt}
    \caption{ Per epoch curves for MNIST corresponding results in \tabref{table:multiclass}. As before, we just plot the dominating term in the RHS of \eqref{eq:multiclass_ERM} as predicted bound. Additionally, we also plot the predicted lower bound by the error on mislabeled data which nevertheless were predicted as true label. We refer to this as ``Oracle bound''. See text for more details. 
    % 
    % except for the stopping point. 
    % The bound predicted by RATT (RHS in \eqref{eq:multiclass_ERM}) is vacuous. 
    }\label{fig:error_epoch_MNIST}
    % \vspace{-15pt}
\end{figure}

\textbf{Results on CIFAR 100 {} {}} 
% 
On CIFAR100, our bound in \eqref{eq:multiclass_ERM} yields vacous bounds. However, the oracle bound as explained above yields tight guarantees in the initial phase of the learning (i.e., when learning rate is less than $0.1$) (\figref{fig:error_CIFAR100}).  

\begin{figure}[h]
    \centering 
    % \vspace{-15pt}
    % \includegraphics[width=0.9\linewidth]{example-image-a}
    \includegraphics[width=0.3\linewidth]{figures/CIFAR100-Resnet.pdf}
    % \includegraphics[width=0.9\linewidth]{figures/{CIFAR10_rn=0.1_lr=0.2_wd=0.005}.png}
    % \vspace{-10pt}
    \caption{ Predicted lower bound by the error on mislabeled data which nevertheless were predicted as true label with ResNet18 on CIFAR100. We refer to this as ``Oracle bound''. See text for more details. 
    % 
    % except for the stopping point. 
    The bound predicted by RATT (RHS in \eqref{eq:multiclass_ERM}) is vacuous. 
    }\label{fig:error_CIFAR100}
    % \vspace{-15pt}
\end{figure}


% \paragraph{Experiments on CIFAR100} 


% \subsection{Model Selection using RATT}


\subsection{Hyperparameter Details}


\textbf{\figref{fig:error_CIFAR10} {} {}} We use clean training dataset of size $40,000$. We fix the amount of unlabeled data at $20\%$ of the clean size, i.e. we include additional $8,000$ points with randomly assigned labels. We use test set of $10,000$ points. For both MLP and ResNet, we use SGD with an initial learning rate of $0.1$ and momentum $0.9$. We fix the weight decay parameter at $5\times 10^{-4}$. After $100$ epochs, we decay the learning rate to $0.01$. We use SGD batch size of $100$. 

\textbf{\figref{fig:error_binary} (a) {} {}} We obtain a toy dataset according to the process described in \secref{sec:app_dataset}. We fix $d=100$ and create a dataset of $50,000$ points with balanced classes. Moreover, we sample additional covariates with the same procedure to create randomly labeled dataset. For both SGD and GD training, we use a fixed learning rate $0.1$.    

\textbf{\figref{fig:error_binary} (b) {} {}} Similar to binary CIFAR, we use clean training dataset of size $40,000$ and fix the amount of unlabeled data at $20\%$ of the clean dataset size. To train wide nets, we use a fixed learning of $0.001$ with GD and SGD. We decide the weight decay parameter and the early stopping point that maximizes our generalization bound (i.e. without peeking at unseen data ).  We use SGD batch size of $100$. 

\textbf{\figref{fig:error_binary} (c) {} {}} With IMDb dataset, we use a clean dataset of size $20,000$ and as before, fix the amount of unlabeled data at $20\%$ of the clean data. To train ELMo model, we use Adam optimizer with a fixed learning rate $0.01$ and weight decay $10^{-6}$ to minimize cross entropy loss. We train with batch size $32$ for 3 epochs. To fine-tune BERT model, we use Adam optimizer with learning rate $5\times 10^{-5}$ to minimize cross entropy loss. We train with a batch size of $16$ for 1 epoch.    

\textbf{\tabref{table:multiclass} {} {}} For multiclass datasets, we train both MLP and ResNet with the same hyperparameters as described before. We sample a clean training dataset of size $40,000$ and fix the amount of unlabeled data at $20\%$ of the clean size. We use SGD with an initial learning rate of $0.1$ and momentum $0.9$. We fix the weight decay parameter at $5\times 10^{-4}$. After $30$ epochs for ResNet and after $50$ epochs for MLP, we decay the learning rate to $0.01$.  We use SGD with batch size $100$. 
For \figref{fig:error_CIFAR100}, we use the same hyperparameters as 
CIFAR10 training, except we now decay learning rate after $100$ epochs. 


In all experiments, to identify the best possible accuracy on just the clean data, we use the exact same set of hyperparamters except the stopping point. We choose a stopping point that maximizes test performance. 

\subsection{Summary of experiments }

\begin{center}
    \begin{table}[H] 
        \centering
        \begin{tabular}{|c|c|c|c|} 
        \hline
        Classification type & Model category & Model & Dataset  \\ [0.5ex] 
        \hline
        \hline
        \multirow{10}{*}{Binary} & Low dimensional & Linear model & Toy Gaussain dataset  \\
                        \cline{2-4}
                         & Overparameterized 
                        %  & Linear model & Toy Gaussain dataset \\
                        %  \cline{3-4}
                        %  & & 2-layer wide net& Toy Gaussain dataset \\
                        %  \cline{3-4}
                         & \multirow{2}{*}{2-layer wide net} & \multirow{2}{*}{Binary MNIST} \\
                         & linear nets & &  
                         \\
                         \cline{2-4}                 
                         & \multirow{6}{*}{Deep nets} & \multirow{2}{*}{MLP} & Binary MNIST \\
                         \cline{4-4}
                         & &  & Binary CIFAR \\
                         \cline{3-4}
                         &  & \multirow{2}{*}{ResNet} & Binary MNIST \\
                         \cline{4-4}
                         & &  & Binary CIFAR \\
                         \cline{3-4}
                         &  & ELMo-LSTM model & IMDb Sentiment Analysis \\
                         \cline{3-4}
                         & & BERT pre-trained model & IMDb Sentiment Analysis \\
        \hline
        \multirow{5}{*}{Multiclass} & \multirow{5}{*}{Deep nets} & \multirow{2}{*}{MLP} & MNIST \\
                        \cline{4-4} 
                        & & & CIFAR10 \\                   
                        \cline{3-4}
                         &   & \multirow{3}{*}{ResNet} & MNIST \\
                         \cline{4-4}
                         &   & & CIFAR10 \\
                         \cline{4-4}
                         &   & & CIFAR100 \\
        \hline
        \end{tabular}
        % \caption{Summary of experiments performed} \label{table:experiments}
    \end{table}    
    % \footnotetext[6]{We use both MSE loss and cross-entropy loss.}
    % \footnotetext[6]{We try 2 variants: one with a fixed first layer and the other with both layers trainable.}
\end{center}

\newpage
\section{Proof of \lemref{lem:stability_error}} \label{app:proof_lem_error}

\begin{proof}[Proof of \lemref{lem:stability_error}]
    Recall, we have a training set $S \cup \wt S_C$. We defined leave-one-out error on mislabeled points as $$\error_{\text{LOO}(\wt S_M) } = \frac{\sum_{(x_i, y_i) \in \wt S_M} \error( f_{(i)}( x_i), y_i)}{ \abs{\wt S_M }} \,, $$
    where $f_{(i)} \defeq f(\calA, (S \cup \wt S)_{(i)})$. Define $S^\prime \defeq S \cup \wt S$. Assume $(x,y)$ and $(x^\prime,y^\prime)$ as i.i.d. samples from ${\calDm}$. 
    Using Lemma 25 in \citet{bousquet2002stability}, we have
    \begin{align*}
        \Expo{ \left( \error_{\calDm}(\wh f) -\error_{\text{LOO}(\wt S_M) } \right)^2 } \le & \Expt{ S^\prime, (x,y), (x^\prime,y^\prime) }{ \error(\wh f(x), y ) \error(\wh f(x^\prime), y^\prime )} - 2 \Expt{ S^\prime, (x,y) }{ \error(\wh f(x), y ) \error(f_{(i)}(x_i), y_i )} \\
        & + \frac{m_1-1}{m_1}\Expt{ S^\prime }{  \error(f_{(i)}(x_i), y_i )  \error(f_{(j)}(x_j), y_j )} + \frac{1}{m_1} \Expt{ S^\prime }{  \error(f_{(i)}(x_i), y_i ) } \,. \numberthis \label{eq:main_reln}
    \end{align*}
    We can rewrite the equation above as : 
    \begin{align*}
        \Expo{ \left( \error_{\calDm}(\wh f) -\error_{\text{LOO}(\wt S_M) } \right)^2 } \le &  \, \underbrace{\Expt{ S^\prime, (x,y), (x^\prime,y^\prime) }{ \error(\wh f(x), y ) \error(\wh f(x^\prime), y^\prime ) - \error(\wh f(x), y ) \error(f_{(i)}(x_i), y_i )}}_{\RN{1}} \\
        & + \underbrace{\Expt{ S^\prime }{  \error(f_{(i)}(x_i), y_i )  \error(f_{(j)}(x_j), y_j ) -  \error(\wh f(x), y ) \error(f_{(i)}(x_i), y_i )}}_{\RN{2}} \\ &+ \underbrace{\frac{1}{m_1} \Expt{ S^\prime }{  \error(f_{(i)}(x_i), y_i ) - \error(f_{(i)}(x_i), y_i )  \error(f_{(j)}(x_j), y_j ) }}_{\RN{3}} \,. \numberthis \label{eq:main_reln2}
    \end{align*}
    
    We will now bound term $\RN{3}$.  Using Cauchy-Schwarz's inequality, we have
    
    \begin{align}
        \Expt{ S^\prime }{  \error(f_{(i)}(x_i), y_i ) - \error(f_{(i)}(x_i), y_i )  \error(f_{(j)}(x_j), y_j ) }^2 &\le  \Expt{ S^\prime }{  \error(f_{(i)}(x_i), y_i ) }^2 \Expt{S^\prime}{1 -   \error(f_{(j)}(x_j), y_j ) }^2 \\
        &\le \frac{1}{4} \,.\label{eq:term1_lem12}
    \end{align}
    
    Note that since $(x_i,y_i)$, $(x_j ,y_j )$, $(x,y)$, and $(x^\prime, y^\prime)$ are all from same distribution $\calDm$, we directly incorporate the bounds on term $\RN{1}$ and $\RN{2}$ from the proof of Lemma 9 in \citet{bousquet2002stability}. Combining that with \eqref{eq:term1_lem12} and our definition of hypothesis stability in \codref{cond:hypothesis_stability}, we have the required claim. 
    
    
    % We now re-write term $\RN{1}$ as
    % \begin{align*}
    %         &\Expt{S^\prime, (x,y), (x^\prime,y^\prime) }{ \error(\wh f(x), y ) \error(\wh f(x^\prime), y^\prime ) - \error(\wh f(x), y ) \error(f_{(i)}(x_i), y_i )} \\ & \qquad = \Expt{ S^\prime, (x,y), (x^\prime,y^\prime) }{ \error(\wh f(x), y ) \error(\wh f  (x^\prime), y^\prime ) - \error(\wh f ^\prime(x), y ) \error(f_{(i)}(x^\prime), y^\prime )} \tag{Exchanging $(x_i, y_i)$ with $(x^\prime, y^\prime)$ in the second term} \\
    %         & \qquad = \Expt{ S^\prime, (x,y), (x^\prime,y^\prime) }{  \left(\error(\wh f(x), y )-  \error(f_{(i)}(x), y ) \right) \error(\wh f  (x^\prime), y^\prime )  } \\
    %         & \qquad  + \Expt{ S^\prime, (x,y), (x^\prime,y^\prime) }{  \left(\error(f_{(i)}(x), y ) -\error(\wh f ^\prime(x), y ) \right) \error(\wh f  (x^\prime), y^\prime )}  \\
    %         & \qquad +\Expt{ S^\prime, (x,y), (x^\prime,y^\prime) }{  \left( \error(\wh f  (x^\prime), y^\prime ) -  \error(f_{(i)}(x^\prime), y^\prime ) \right) \error(\wh f ^\prime(x), y ) }  \,, \numberthis \label{eq:term1_final}
    % \end{align*}
    % where $\wh f^\prime$ is the classifier obtained by training on $ S^\prime_{(i)} \cup \{ (x^\prime, y^\prime) \} $. Similarly we can re-write term $\RN{2}$ as 
    % \begin{align*}
    %     & \Expt{ S^\prime }{  \error(f_{(i)}(x_i), y_i )  \error(f_{(j)}(x_j), y_j ) -  \error(\wh f(x), y ) \error(f_{(i)}(x_i), y_i )} \\
    %     &\quad  = \Expt{ S^\prime, (x,y), (x^\prime,y^\prime)}{  \error(f^{\prime\prime}_{(i)}(x), y )  \error(f_{(j)}^{\prime}(x^\prime), y^\prime ) -  \error(\wh f(x), y ) \error(f_{(i)}(x_i), y_i )} \tag{Exchanging $(x_i, y_i)$ with $(x, y)$ and $(x_j, y_j)$ with $(x^\prime, y^\prime)$ in the first term}\\
    %     &\quad = \Expt{ S^\prime, (x,y), (x^\prime,y^\prime)}{  \error(f^{\prime\prime}_{(j)}(x), y )  \error(f_{(i)}^{\prime}(x^\prime), y^\prime ) -  \error(\wh f^\prime (x), y ) \error(f^\prime_{(j)}(x^\prime), y^\prime )} \tag{Exchanging $(x_i, y_i)$ and $(x_j, y_j)$ and then replacing $(x_j, y_j)$ with $(x^\prime, y^\prime)$ in the second term} \\
    %     & \quad = \Expt{ S^\prime, (x,y), (x^\prime,y^\prime) }{  \left( \error(f_{(i)}^{\prime}(x^\prime), y^\prime )   -  \error(\wh f^{\prime\prime}  (x^\prime), y^\prime ) \right)  \error(f^{\prime\prime}_{(j)}(x), y )   } \\
    %     & \quad  + \Expt{ S^\prime, (x,y), (x^\prime,y^\prime) }{  \left( \error(f^{\prime\prime}_{(j)}(x), y )  -\error(\wh f ^\prime(x), y ) \right) \error(\wh f^{\prime\prime}  (x^\prime), y^\prime )  }  \\
    %     & \quad+ \Expt{ S^\prime, (x,y), (x^\prime,y^\prime) }{  \left( \error(\wh f^{\prime\prime}  (x^\prime), y^\prime )  -  \error(f^\prime_{(j)}(x^\prime), y^\prime ) \right)  \error(\wh f^\prime (x), y ) }   \\
    %     & \quad = \Expt{ S^\prime, (x,y), (x^\prime,y^\prime) }{  \left( \error(f_{(i)}^{\prime}(x^\prime), y^\prime )   -  \error(\wh f (x^\prime), y^\prime ) \right)  \error(f_{(i)}(x_j), y_j )   } \\
    %     & \quad  + \Expt{ S^\prime, (x,y), (x^\prime,y^\prime) }{  \left( \error(f^{\prime\prime}_{(j)}(x), y )  -\error(\wh f (x), y ) \right) \error(\wh f^{\prime\prime}  (x_j), y_j )  }  \\
    %     & \quad+ \Expt{ S^\prime, (x,y), (x^\prime,y^\prime) }{  \left( \error(\wh f^{\prime\prime}  (x^\prime), y^\prime )  -  \error(f^\prime_{(j)}(x^\prime), y^\prime ) \right)  \error(\wh f^\prime (x^\prime), y^\prime ) }  \,, \numberthis \label{eq:term2_final}
    % \end{align*}
    % where $f^{\prime\prime}_{(j)}$ is trained on $S^\prime_{(j,i)} \cup {(x,y)}$, $f^{\prime}_{(i)}$ is trained on $S^\prime_{(j,i)} \cup {(x^\prime,y^\prime)}$, and $\wh f^{\prime\prime} $ is trained on $S^\prime_{(j)} \cup {(x,y)}$. Note in the last line we replaced $(x,y)$ by $(x_j, y_j)$ in the first term, replaced $(x^\prime,y^\prime)$ by $(x_j, y_j)$ in the second term and exchanged $(x_i,y_i)$ with $(x_j,y_j)$ and also $(x,y)$ and $(x^\prime, y^\prime)$
    
    
\end{proof}


% 
% 16th Century Version Control 
% 

% \onecolumn

% \section*{Supplementary Material}
% We will be using the following standard results
% on exponential concentration of random variables 
% all throughout the discussion:

% \begin{lemma}[Hoeffding's inequality for independent RVs~\citep{hoeffding1994probability}] Let $Z_1, Z_2, \ldots, Z_n$ be independent bounded random variables with $Z_i \in [a,b]$ for all $i$, then 
%     \begin{align*}
%         \prob\left( \frac{1}{n} \sum_{i=1}^n (Z_i - \Expo{Z_i}) \ge t \right) \le \exp{\left( -\frac{2nt^2}{(b-a)^2} \right) }
%     \end{align*} 
%     and 
%     \begin{align*}
%         \prob\left( \frac{1}{n} \sum_{i=1}^n (Z_i - \Expo{Z_i}) \le -t \right) \le \exp{\left( -\frac{2nt^2}{(b-a)^2} \right) }
%     \end{align*} 
%     for all $t \ge 0$. 
% \end{lemma}

% \begin{lemma}[Hoeffding's inequality for sampling with replacement~\citep{hoeffding1994probability}] \label{lem:hoeffding_sampling} Let $\calZ = (Z_1, Z_2, \ldots, Z_N)$ be a finite population of $N$ points with $Z_i \in [a.b]$ for all $i$. Let $X_1, X_2, \ldots X_n$ be a random sample drawn without replacement from $\calZ$. Then for all $t \ge 0$, we have 
%     \begin{align*}
%         \prob\left( \frac{1}{n} \sum_{i=1}^n (X_i - \mu ) \ge t \right) \le \exp{\left( -\frac{2nt^2}{(b-a)^2} \right) }
%     \end{align*} 
%     and 
%     \begin{align*}
%         \prob\left( \frac{1}{n} \sum_{i=1}^n (X_i - \mu ) \le -t \right) \le \exp{\left( -\frac{2nt^2}{(b-a)^2} \right) } \,,
%     \end{align*} 
%     where $\mu = \frac{1}{N} \sum_{i=1}^{N} Z_i$. 
% \end{lemma}

% We now discuss one condition that generalizes the exponential concentration to dependent random variables.
% \begin{condition}[Bounded difference inequality] \label{cond:BDC} Let $\calZ$ be some set and $\phi: \calZ^n \to \Real$. We say that $\phi$ satisfies the bounded difference assumption if 
% there exists $c_1, c_2, \ldots c_n \ge 0$ s.t. for all $i$, we have 
% \begin{align*}
%     \sup_{Z_1,Z_2, \ldots,Z_n, Z_i^\prime in \calZ^{n+1} } \abs{\phi (Z_1, \ldots, Z_i, \ldots, Z_n ) - \phi (Z_1, \ldots, Z_i^\prime, \ldots, Z_n ) } \le c_i \,.
% \end{align*} 
% \end{condition}

% \begin{lemma}[McDiarmid’s inequality~\citep{mcdiarmid1989}] \label{lem:McDiarmid} Let $Z_1, Z_2, \ldots, Z_n$ be independent random variables on set $\calZ$ and $\phi : \calZ^n \to \Real$ satisfy bounded difference assumption (\codref{cond:BDC}). Then for all $t>0$, we have 
%     \begin{align*}
%         \prob\left( \phi(Z_1, Z_2, \ldots, Z_n) - \Expo{\phi(Z_1, Z_2, \ldots, Z_n)} \ge t \right) \le \exp{\left( -\frac{2t^2}{\sum_{i=1}^n c_i^2} \right) } 
%     \end{align*} 
%     and 
%     \begin{align*}
%         \prob\left( \phi(Z_1, Z_2, \ldots, Z_n) - \Expo{\phi(Z_1, Z_2, \ldots, Z_n)} \le -t \right) \le \exp{\left( -\frac{2t^2}{\sum_{i=1}^n c_i^2} \right) } \,
%     \end{align*} 
% \end{lemma}


% \section{Proofs from \secref{sec:ERM_training}}\label{app:proof_erm}

% \textbf{Additional notation {} {}} Let $m_1$ be the number of mislabeled points ($\wt S_M$) and $m_2$ be the number of correctly labeled points ($\wt S_C$). Note $m_1 + m_2 = m$. 


% \subsection{Proof of \thmref{thm:error_ERM}}


% \begin{proof}[Proof of \lemref{lem:fit_mislabeled}] 
%     The main idea of our proof is to regard 
%     the clean portion of the data 
%     ($S \cup \wt S_C$) as fixed.   
%     Then, there exists a classifier $f^*$ 
%     that is optimal over draws 
%     of the mislabeled data $\wt S_M$. 
% % 
%     % 
%     Formally, 
%     \begin{align}
%     f^* \defeq \argmin_{f \in \calF} \error_{\widecheck {\calD}} (f) \,, \label{eq:modified_ERM}
%     \end{align}
%     where $$\widecheck \calD = \frac{n}{m+n} \calS + \frac{m_1}{m+n} \wt \calS_C  + \frac{m_2}{m+n}\calDm \,.$$ That is, $\widecheck \calD$ a combination of 
%     the \emph{empirical distribution} 
%     over correctly labeled data $S \cup \wt S_C$
%     % in $S\cup \wt S$ 
%     and the (population) distribution 
%     over mislabeled data $\calDm$.
%     Recall that 
%     \begin{align}
%     \wh f \defeq \argmin_{f \in \calF} \error_{\calS \cup \wt S} (f) \,. \label{eq:orig_ERM}
%     \end{align}
%     % 
%     % 
%     Since, $\widehat f$ minimizes 0-1 error 
%     on $S \cup \wt S$, using ERM optimality on \eqref{eq:orig_ERM},  
%     we have 
%     \begin{align}
%         \error_{\calS \cup \wt \calS}(\widehat f) \le \error_{
%             \calS \cup \wt \calS}(f^*) \,.    \label{eq:step1}
%     \end{align}
%     Moreover, since $f^*$ is independent of $\wt S_M$, using Hoeffding's bound,
%     % \footnote{For a fully rigorous argument,
%     % refer to the complete proof in App.~\ref{app:proof_erm}.} 
%     we have with probability at least $1-\delta$ that
%     \begin{align}
%       \error_{\wt \calS_M}(f^*) \le \error_{ \calDm}(f^*) +  \sqrt{\frac{\log(1/\delta)}{2 m_1}} \,. \label{eq:step2} 
%     \end{align}
%     %$ 
%     %for some constant $c_1\le 1/2$. 
%     Finally, since $f^*$ is the optimal classifier on $\widecheck \calD$, 
%     we have 
%     \begin{align}
%         \error_{\widecheck \calD}(f^*) \le \error_{\widecheck \calD}(\widehat f) \label{eq:step3}
%     \end{align}
%      Now to relate \eqref{eq:step1} and \eqref{eq:step3}, we can re-write the \eqref{eq:step2} as follows: 
%     \begin{align}
%         \error_{\calS \cup \wt\calS}(f^*) \le \error_{ \widecheck \calD}(f^*) +  \frac{m_1}{m+n}\sqrt{\frac{\log(1/\delta)}{2 m_1}} \,. \label{eq:step4} 
%     \end{align}
%     Now we combine equations \eqref{eq:step1}, \eqref{eq:step4}, and \eqref{eq:step3}, to get 
%     \begin{align}
%         \error_{\calS \cup \wt \calS}(\wh f) \le \error_{\widecheck \calD}(\wh f) +  \frac{m_1}{m+n}\sqrt{\frac{\log(1/\delta)}{2 m_1}} \,, 
%     \end{align}
%     which implies 
%     \begin{align}
%         \error_{ \wt \calS_M}(\wh f) \le \error_{\calDm}(\wh f) + \sqrt{\frac{\log(1/\delta)}{2 m_1}} \,. \label{eq:lemma1_final}
%     \end{align}
%     Since $\wt S$ is obtained by randomly labeling an unlabeled dataset, we assume $2m_1 \approx m$ \footnote{Formally, with probability at least $1-\delta$, we have  $(m - 2m_1)\le \sqrt{m\log(1/\delta)/2}$ }. Moreover, using $\error_{\calDm} = 1 - \error_{\calD}$ we obtain the desired result.   
%     % Combining the above steps and using the fact 
%     % that $\error_\calD = 1- \error_{\calDm} $, 
%     % we obtain the desired result.
% \end{proof}

% \begin{proof}[Proof of \lemref{lem:mislabeled_error}]
%     Recall $\error_{\wt S} (f) = \frac{m_1}{m} \error_{\wt S_M}(f) + \frac{m_2}{m} \error_{\wt S_C}(f)$. Hence, we have 
%     \begin{align}
%         2\error_{\wt S}(f) - \error_{\wt S_M}(f) - \error_{\wt S_C}(f) &= \left(\frac{2m_1}{m} \error_{\wt S_M}(f) - \error_{\wt S_M}(f)\right) + \left(\frac{2m_2}{m} \error_{\wt S_C}(f) - \error_{\wt S_C}(f)\right) \\ &= \left(\frac{2m_1}{m} - 1\right) \error_{\wt S_M}(f) + \left(\frac{2m_2}{m} - 1 \right)\error_{\wt S_C} (f) \,.
%     \end{align} 
%     Since the dataset is randomly labeled, with probability at least $1-\delta$, we have  $\left(\frac{2m_1}{m} - 1\right) \le \sqrt{\frac{\log(1/\delta)}{2m}}$. Similarly, we have with probability at least $1-\delta$, $\left(\frac{2m_2}{m} - 1\right) \le \sqrt{\frac{\log(1/\delta)}{2m}}$. Using union bound, we have with probability at least $1-\delta$
%     % \begin{align}
%     %     2\error_{\wt S} - \error_{\wt S_M}(f) - \error_{\wt S_C}(f) \le \sqrt{\frac{\log(2/\delta)}{2m}} \left(\error_{\wt S_M}(f) + \error_{\wt S_C}(f) \right) \le 2\sqrt{\frac{\log(2/\delta)}{2m}} \,. \label{eq:lemma2_final}
%     % \end{align}
%     \begin{align}
%         2\error_{\wt S} - \error_{\wt S_M}(f) - \error_{\wt S_C}(f) \le \sqrt{\frac{\log(2/\delta)}{2m}} \left(\error_{\wt S_M}(f) + \error_{\wt S_C}(f) \right) \,. \label{eq:lemma2_prefinal}
%     \end{align}
%     With re-arranging $\error_{\wt S_M}(f) + \error_{\wt S_C}(f)$ and using the inequality $ 1- a\le \frac{1}{1+a} $, we have  
%     \begin{align}
%         2\error_{\wt S} - \error_{\wt S_M}(f) - \error_{\wt S_C}(f) \le 2\error_{\wt \calS} \sqrt{\frac{\log(2/\delta)}{2m}}  \,. \label{eq:lemma2_final}
%     \end{align}

%     % We obtain the desired result by using 
% \end{proof}

% \begin{proof}[Proof of \lemref{lem:clear_error}]
% % Recall 0-1 error on each point  $(x,y) \in S \cup \wt S$ is given by $\I{ f(x)\ne y}$.
% In the set of correctly labeled points $S \cup \wt S_C$, we have $S$ as a random subset of $S \cup \wt S_C$. Hence, using Hoeffding's inequality for sampling without replacement (\lemref{lem:hoeffding_sampling}), we have with probability at least $1-\delta$
% \begin{align}
%     \error_{\wt \calS_c} (\wh f)- \error_{\calS \cup \wt \calS_C}( \wh f) \le  \sqrt{\frac{\log(1/\delta)}{2m_2}} \,.
% \end{align}
% Re-writing $\error_{\calS \cup \wt \calS_C}( \wh f)$ as $\frac{m_2}{m_2 + n} \error_{\wt \calS_C }(\wh f) + \frac{n}{m_2 + n} \error_{\calS }(\wh f)$, we have with probability at least $1-\delta$
% \begin{align}
%   \left(\frac{n}{n+m_2}\right) \left(\error_{\wt \calS_c} (\wh f)- \error_{\calS}( \wh f) \right) \le  \sqrt{\frac{\log(1/\delta)}{2m_2}} \,.
% \end{align}
% As before, assuming $2m_2 \approx m$, we have with probability at least $1-\delta$ 
% \begin{align}
%     \error_{\wt \calS_c} (\wh f)- \error_{\calS}( \wh f) \le \left(1+\frac{m_2}{n}\right)  \sqrt{\frac{\log(1/\delta)}{m}} \le 1.5 \sqrt{\frac{\log(1/\delta)}{m}} \,. \label{eq:lemma3_final}
% \end{align} 
% \end{proof}

% \begin{proof}[Proof of \thmref{thm:error_ERM}] 
%     Having established these core intermediate results, we can now combine above three lemmas to prove the main result. 
%     In particular, we bound the population error on clean data ($\error_\calD(\wh f)$) as follows:  
%     \begin{enumerate}[(i)]
%         \item First, use \eqref{eq:lemma1_final}, to obtain an upper bound on the population error on clean data, i.e., with probability at least $1-\delta/4$, we have
%         \begin{align}
%             \error_{ \calD} (\wh f) \le 1 - \error_{ \wt \calS_M}(\wh f) + \sqrt{\frac{\log(4/\delta)}{m}} \,. 
%         \end{align}
%         \item  Second, use \eqref{eq:lemma2_final}, to relate the error on the mislabeled fraction with error on clean portion of randomly labeled data and error on whole randomly labeled dataset, i.e., with probability at least $1-\delta/2$, we have 
%         \begin{align}
%             - \error_{\wt S_M}(f) \le \error_{\wt S_C}(f) - 2\error_{\wt S}  + \sqrt{\frac{\log(4/\delta)}{2m}}  \,. 
%         \end{align} 
%         \item Finally, use \eqref{eq:lemma3_final} to relate the error on the clean portion of randomly labeled data and error on clean training data, i.e., with probability $1-\delta/4$, we have 
%         \begin{align}
%             \error_{\wt \calS_C} (\wh f)\le - \error_{\calS}( \wh f) + \left(1 + \frac{m}{2n} \right) \sqrt{\frac{\log(4/\delta)}{m}} \,. 
%         \end{align} 
%     \end{enumerate}

%     Using union bound on the above three steps, we have with probability at least $1-\delta$: 
%     \begin{align}
%         \error_\calD (\wh f) \le \error_{\calS}(\wh f)   + 1 - 2\error_{\wt \calS}(\wh f)   + (1/\sqrt{2} + 2.5)  \sqrt{\frac{\log(4/\delta)}{m}} \,.
%     \end{align}
%     Note that $(1/\sqrt{2} + 2.5)$ is a loose constant. In experiments, we use the ratio $\frac{m}{n}$
%     %  the exact error $\error_{\wt \calS}(\wh f)$ 
%     to evaluate R.H.S.    
% \end{proof}

% \subsection{Proof of \propref{prop:rademacher}}

% \begin{proof}[Proof of \propref{prop:rademacher}]
%     For a classifier $ f: \calX \to \{-1, 1\}$, we have $1 - 2\,\indict{ f(x) \ne y} = y \cdot f(x)$. Hence, by definition of $\error$, we have 
%     \begin{align}
%         1 -2\error_{\wt \calS}(f) = \frac{1}{m}\sum_{i=1}^m y_i \cdot f(x_i) \le \sup_{f \in \calF} \, \frac{1}{m} \sum_{i=1}^m y_i \cdot f(x_i)  \,. \label{eq:error_rademacher}
%     \end{align}
%     Note that for fixed inputs $(x_1, x_2, \ldots, x_m)$ in $\wt S$, $(y_1, y_2, \ldots y_m)$ are random labels. Define $\phi_1 (y_1, y_2, \ldots, y_m) \defeq \sup_{f \in \calF} \, \frac{1}{m} \sum_{i=1}^m y_i \cdot f(x_i)$. We have the following bounded difference condition on $\phi_1$. For all i, 
%     \begin{align}
%         \sup_{y_1, \ldots y_m, y_i^\prime \in \{-1, 1\}^{m+1} } \abs{ \phi_1 (y_1,\ldots, y_i, \ldots, y_m) - \phi_1 (y_1,\ldots, y_i^\prime, \ldots, y_m)  } \le 1/m \,. \label{cond1_rademacher}
%     \end{align} 
    
%     Similarly define $\phi_2 (x_1, x_2, \ldots, x_m) \defeq \Expt{ y_i \sim_U \{-1, 1\}  }{ \sup_{f \in \calF} \, \frac{1}{m}  \sum_{i=1}^m y_i \cdot f(x_i)}$. We have the following bounded difference condition on $\phi_2$. For all i,
%     \begin{align}
%         \sup_{x_1, \ldots x_m, x_i^\prime \in \calX^{m+1} } \abs{ \phi_2 (x_1,\ldots, x_i, \ldots, x_m) - \phi_1 (x_1,\ldots, x_i^\prime, \ldots, x_m)  } \le 1/m \,. \label{cond2_rademacher}
%     \end{align}
%     Using McDiarmid’s inequality (\lemref{lem:McDiarmid}) twice with Condition \eqref{cond1_rademacher} and \eqref{cond2_rademacher}, with probability at least $1-\delta$, we have
%     \begin{align}
%         \sup_{f \in \calF} \, \frac{1}{m} \sum_{i=1}^m y_i \cdot f(x_i)  - \Expt{x,y}{\sup_{f \in \calF} \, \frac{1}{m} \sum_{i=1}^m y_i \cdot f(x_i) } \le \sqrt{\frac{2\log(2/\delta)}{m}} \label{eq:final_rademacher}
%     \end{align} 
%     Combining \eqref{eq:error_rademacher} and \eqref{eq:final_rademacher}, we obtain the desired result. 
% \end{proof}


% \subsection{Proof of \thmref{thm:error_regularized_ERM}}

% Proof of \thmref{thm:error_regularized_ERM} follows similar to the proof of \thmref{thm:error_ERM}. Note that the same results in \lemref{lem:fit_mislabeled}, \lemref{lem:mislabeled_error}, and \lemref{lem:clear_error} hold in the regularized ERM case. However, the arguments in the proof of \lemref{lem:fit_mislabeled} changes slightly. Hence, we state and prove a lemma parallel to \lemref{lem:fit_mislabeled} for completeness. 

% \begin{lemma} \label{lem:lemma1_reg}
%     Assume the same setup as \thmref{thm:error_regularized_ERM}. 
%     Then for any $\delta >0$, with probability at least  $1-\delta$ 
%     over the random draws of mislabeled data $\wt S_M$, we have 
%     \begin{align}
%         \error_\calD(\widehat f)  \le 1 -\error_{\wt \calS_M}(\widehat f) + \sqrt{\frac{\log(1/\delta)}{m}}\,. 
%     \end{align} 
% \end{lemma}
% \begin{proof}
%     The main idea of the proof remains the same, i.e. regard 
%     the clean portion of the data 
%     ($S \cup \wt S_C$) as fixed.   
%     Then, there exists a classifier $f^*$ 
%     that is optimal over draws 
%     of the mislabeled data $\wt S_M$. 

    
%     Formally, 
%     \begin{align}
%     f^* \defeq \argmin_{f \in \calF} \error_{\widecheck {\calD}} (f)  + \lambda R(f) \,, \label{eq:modified_ERM_reg}
%     \end{align}
%     where $$\widecheck \calD = \frac{n}{m+n} \calS + \frac{m_1}{m+n} \wt \calS_C  + \frac{m_2}{m+n}\calDm \,.$$ That is, $\widecheck \calD$ a combination of 
%     the \emph{empirical distribution} 
%     over correctly labeled data $S \cup \wt S_C$
%     % in $S\cup \wt S$ 
%     and the (population) distribution 
%     over mislabeled data $\calDm$.
%     Recall that 
%     \begin{align}
%     \wh f \defeq \argmin_{f \in \calF} \error_{\calS \cup \wt S} (f) + \lambda R(f) \,. \label{eq:orig_ERM_reg}
%     \end{align}
%     % 
%     % 
%     Since, $\widehat f$ minimizes 0-1 error 
%     on $S \cup \wt S$, using ERM optimality on \eqref{eq:orig_ERM},  
%     we have 
%     \begin{align}
%         \error_{\calS \cup \wt \calS}(\widehat f) + \lambda R(\wh f) \le \error_{
%             \calS \cup \wt \calS}(f^*) + \lambda R(f^*) \,.    \label{eq:step1_reg}
%     \end{align}
%     Moreover, since $f^*$ is independent of $\wt S_M$, using Hoeffding's bound,
%     % \footnote{For a fully rigorous argument,
%     % refer to the complete proof in App.~\ref{app:proof_erm}.} 
%     we have with probability at least $1-\delta$ that
%     \begin{align}
%       \error_{\wt \calS_M}(f^*) \le \error_{ \calDm}(f^*) +  \sqrt{\frac{\log(1/\delta)}{2 m_1}} \,. \label{eq:step2_reg} 
%     \end{align}
%     %$ 
%     %for some constant $c_1\le 1/2$. 
%     Finally, since $f^*$ is the optimal classifier on $\widecheck \calD$, 
%     we have 
%     \begin{align}
%         \error_{\widecheck \calD}(f^*) + \lambda R(f^*) \le \error_{\widecheck \calD}(\widehat f) + \lambda R(\wh f) \label{eq:step3_reg}
%     \end{align}
%      Now to relate \eqref{eq:step1_reg} and \eqref{eq:step3_reg}, we can re-write the \eqref{eq:step2_reg} as follows: 
%     \begin{align}
%         \error_{\calS \cup \wt\calS}(f^*) \le \error_{ \widecheck \calD}(f^*) +  \frac{m_1}{m+n}\sqrt{\frac{\log(1/\delta)}{2 m_1}} \,. \label{eq:step4_reg} 
%     \end{align}
%     After adding $\lambda R(f^*)$ on both sides in \eqref{eq:step4_reg}, we combine equations \eqref{eq:step1_reg}, \eqref{eq:step4_reg}, and \eqref{eq:step3_reg}, to get 
%     \begin{align}
%         \error_{\calS \cup \wt \calS}(\wh f) \le \error_{\widecheck \calD}(\wh f) +  \frac{m_1}{m+n}\sqrt{\frac{\log(1/\delta)}{2 m_1}} \,, 
%     \end{align}
%     which implies 
%     \begin{align}
%         \error_{ \wt \calS_M}(\wh f) \le \error_{\calDm}(\wh f) + \sqrt{\frac{\log(1/\delta)}{2 m_1}} \,. \label{eq:lemma_reg_final}
%     \end{align}
%     Similar as before, since $\wt S$ is obtained by randomly labeling an unlabeled dataset, we assume 
%     $2m_1 \approx m$. Moreover, using $\error_{\calDm} = 1 - \error_{\calD}$ we obtain the desired result. 
% \end{proof}
% % \begin{proof}[Proof of ]
    
% % \end{proof}

% \subsection{Proof of \thmref{thm:multiclass_ERM}}

% We first state and prove lemmas parallel to three lemmas used in the proof of balanced binary case. Then we combine the results in the three lemmas to obtain the result in \thmref{thm:multiclass_ERM}. 

% Before stating the result, we define mislabeled distribution $\calDm$ for any $\calD$. While $\calDm$ and $\calD$ share 
% the same marginal distribution over $\calX$, 
% the distribution over labels $y$ 
% given an input $x\sim \calD_\calX$ is changed.
% In particular, for any $x$, the pdf over $y$ is changed to:  
% $p_{\calDm} (\cdot \vert x) \defeq \frac{1 - p_{\calD}(\cdot \vert x)}{k - 1}$.

% \begin{lemma} \label{lem:fit_mislabeled_multi}
%     Assume the same setup as \thmref{thm:multiclass_ERM}. 
%     Then for any $\delta >0$, with probability at least  $1-\delta$ 
%     over the random draws of mislabeled data $\wt S_M$, we have 
%     \begin{align}
%         \error_\calD(\widehat f)  \le (k-1)\left(1 -\error_{\wt \calS_M}(\widehat f)\right) + (k-1)\sqrt{\frac{\log(1/\delta)}{m}}\,. \label{eq:lemma1_multi}
%     \end{align}   
% \end{lemma} 

% \begin{proof}
%     The main idea of the proof remains the same, i.e. regard 
%     the clean portion of the data 
%     ($S \cup \wt S_C$) as fixed. 
%     Then, there exists a classifier $f^*$ 
%     that is optimal over draws 
%     of the mislabeled data $\wt S_M$. 
    
%     However, we need to be careful while relating population error on mislabeled data with population accuracy on clean data.   
%     While for binary classification,  we could upper bound $\error_{\wt \calS_M}$ 
%     with $1-\error_\calD$  (in the proof of \lemref{lem:fit_mislabeled}), 
%     for multiclass classification, 
%     error on the mislabeled data 
%     and accuracy on the clean data 
%     in the population 
%     are not so directly related.  
%     To establish \eqref{eq:lemma1_multi},
%     we break the error on the 
%     (unknown) mislabeled data 
%     into two parts: one term corresponds 
%     to predicting the true label on mislabeled data, 
%     and the other corresponds to predicting 
%     neither the true label 
%     nor the assigned (mis-)label.  
%     Finally, we relate these errors to their
%     population counterparts to establish \eqref{eq:lemma1_multi}. 
    
%     Formally, 
%     \begin{align}
%     f^* \defeq \argmin_{f \in \calF} \error_{\widecheck {\calD}} (f)  + \lambda R(f) \,, \label{eq:modified_ERM_reg2}
%     \end{align}
%     where $$\widecheck \calD = \frac{n}{m+n} \calS + \frac{m_1}{m+n} \wt \calS_C  + \frac{m_2}{m+n}\calDm \,.$$ That is, $\widecheck \calD$ a combination of 
%     the \emph{empirical distribution} 
%     over correctly labeled data $S \cup \wt S_C$
%     % in $S\cup \wt S$ 
%     and the (population) distribution 
%     over mislabeled data $\calDm$.
%     Recall that 
%     \begin{align}
%     \wh f \defeq \argmin_{f \in \calF} \error_{\calS \cup \wt S} (f) + \lambda R(f) \,. \label{eq:orig_ERM_reg2}
%     \end{align}
%     % 
%     % 
%     Following the exact steps from the proof of \lemref{lem:lemma1_reg}, with probability at least $1-\delta$, we have  
%     \begin{align}
%         \error_{ \wt \calS_M}(\wh f) \le \error_{\calDm}(\wh f) + \sqrt{\frac{\log(1/\delta)}{2 m_1}} \,. \label{eq:lemma1_final_multi_prev}
%     \end{align}
%     Similar to before, since $\wt S$ is obtained by randomly labeling an unlabeled dataset, we assume 
%     $\frac{k}{k-1} m_1 \approx m$. 
    
%     Now we will relate $\error_\calDm (\wh f)$ with $\error_{\calD}(\wh f)$. Let $y^T$ denote the (unknown) true label for a mislabeled point $(x, y)$ (i.e., label before replacing it with a mislabel). 
%     \begin{align}    
%          \Expt{(x, y) \in \sim \calDm}{\indict{ \wh f(x) \ne y }}  &= \underbrace{\Expt{(x, y) \in \sim \calDm}{\indict{ \wh f(x) \ne y \land \wh f(x) \ne y^T}}}_{\RN{1}} + \underbrace{\Expt{(x, y) \in \sim \calDm}{\indict{ \wh f(x) \ne y \land \wh f(x) = y^T}}}_{\RN{2}} \,. \label{eq:excess_term}
%     \end{align}
%     Clearly, term 2 is one minus the accuracy on the clean unseen data, i.e. 
%     \begin{align}
%         \RN{2} = 1 - \Expt{{x,y} \sim \calD}{ \indict{ \wh f(x) \ne y}} = 1- \error_{\calD}(\wh f) \,. \label{eq:term1}    
%     \end{align}
%     Next, we  relate term 1 with the error on the unseen clean data. We show that term 1 is equal to the error on the unseen clean data scaled by $\frac{k-2}{k-1}$ where $k$ is the number of labels. Using the definition of mislabeled distribution $\calDm$,  we have 
%     \begin{align}
%         \RN{1} = \frac{1}{k-1} \left( \Expt{(x, y) \in \sim \calD}{ \sum_{i \in \calY \land i\ne y}  \indict{ \wh f(x) \ne i \land \wh f(x) \ne y}} \right) = \frac{k-2}{k-1} \error_{\calD}(\wh f) \,.\label{eq:term2}
%     \end{align}    

%     Combining the result in \eqref{eq:term1}, \eqref{eq:term2} and \eqref{eq:excess_term}, we have 
%     \begin{align}
%         \error_{\calDm}(\wh f) = 1- \frac{1}{k-1} \error_{\calD}(\wh f) \,.\label{eq:combine_terms}
%     \end{align}
%     Finally, combining the result in \eqref{eq:combine_terms} with equation \eqref{eq:lemma1_final_multi_prev}, we have with probability $1-\delta$, 
%     \begin{align}
%       \error_{\calD}(\wh f) \le  (k-1) \left( 1- \error_{ \wt \calS_M}(\wh f) \right)  + (k-1) \sqrt{\frac{k \log(1/\delta)}{ 2(k-1)m}} \,. \label{eq:lemma1_final_multi}
%     \end{align}
% \end{proof}

% \begin{lemma} \label{lem:mislabeled_error_multi}
%     Assume the same setup as \thmref{thm:multiclass_ERM}.  Then for any $\delta >0$, with probability at least $1-\delta$ over the random draws of $\wt S$, we have  
%     % \begin{align}
%         $$\abs{k\error_{\wt \calS}(\widehat f) - \error_{\wt \calS_C}(\widehat f) -  (k-1)\error_{\wt \calS_M}(\widehat f) } \le  2k\sqrt{\frac{\log(4/\delta)}{2m}}\,. $$ % \label{eq:lemma2}
%     % \end{align}   
%     %  for some constant $c_3 \le 1.0\,$.
% \end{lemma} 


% \begin{proof}
%     Recall $\error_{\wt S} (f) = \frac{m_1}{m} \error_{\wt S_M}(f) + \frac{m_2}{m} \error_{\wt S_C}(f)$. Hence, we have 
%     \begin{align}
%         k\error_{\wt S}(f) - (k-1)\error_{\wt S_M}(f) - \error_{\wt S_C}(f) &= (k-1)\left(\frac{k m_1}{(k-1) m} \error_{\wt S_M}(f) - \error_{\wt S_M}(f)\right) + \left(\frac{km_2}{m} \error_{\wt S_C}(f) - \error_{\wt S_C}(f)\right) \\ &= k \left[ \left(\frac{m_1}{m} - \frac{k-1}{k}\right) \error_{\wt S_M}(f) + \left(\frac{m_2}{m} - \frac{1}{k} \right) \error_{\wt S_C} (f) \right] \,.
%     \end{align} 
%     Since the dataset is randomly labeled, we have with probability at least $1-\delta$, $\left(\frac{m_1}{m} - \frac{k-1}{k}\right) \le \sqrt{\frac{\log(1/\delta)}{2m}}$. Similarly, we have with probability at least $1-\delta$, $\left(\frac{m_2}{m} - \frac{1}{k}\right) \le \sqrt{\frac{\log(1/\delta)}{2m}}$. Using union bound, we have with probability at least $1-\delta$
%     % \begin{align}
%     %     2\error_{\wt S} - \error_{\wt S_M}(f) - \error_{\wt S_C}(f) \le \sqrt{\frac{\log(2/\delta)}{2m}} \left(\error_{\wt S_M}(f) + \error_{\wt S_C}(f) \right) \le 2\sqrt{\frac{\log(2/\delta)}{2m}} \,. \label{eq:lemma2_final}
%     % \end{align}
%     \begin{align}
%         k\error_{\wt S}(f) - (k-1)\error_{\wt S_M}(f) - \error_{\wt S_C}(f)  \le k \sqrt{\frac{\log(2/\delta)}{2m}} \left(\error_{\wt S_M}(f) + \error_{\wt S_C}(f) \right) \,. \label{eq:lemma2_final_multi}
%     \end{align}

%     % We obtain the desired result by using 
% \end{proof}

% \begin{lemma} \label{lem:clear_error_multi}
%     Assume the same setup as \thmref{thm:multiclass_ERM}. 
%     Then for any $\delta >0$, with probability at least $1-\delta$ 
%     over the random draws of $\wt S_C$ and $S$, we have 
%     % \begin{align}
%         $$\abs{\error_{\wt \calS_C}(\widehat f) - \error_{\calS}(\widehat f) } \le 1.5 \sqrt{\frac{k\log(2/\delta)}{2m}}\,.$$ %\label{eq:lemma3}
%     % \end{align}   
%     % for some constant $c_2 \le 1.2\,$.
% \end{lemma} 
% \begin{proof}
%     % Recall 0-1 error on each point  $(x,y) \in S \cup \wt S$ is given by $\I{ f(x)\ne y}$.
%     In the set of correctly labeled points $S \cup \wt S_C$, we have $S$ as a random subset of $S \cup \wt S_C$. Hence, using Hoeffding's inequality for sampling without replacement (\lemref{lem:hoeffding_sampling}), we have with probability at least $1-\delta$
%     \begin{align}
%         \error_{\wt \calS_c} (\wh f)- \error_{\calS \cup \wt \calS_C}( \wh f) \le  \sqrt{\frac{\log(1/\delta)}{2m_2}} \,.
%     \end{align}
%     Re-writing $\error_{\calS \cup \wt \calS_C}( \wh f)$ as $\frac{m_2}{m_2 + n} \error_{\wt \calS_C }(\wh f) + \frac{n}{m_2 + n} \error_{\calS }(\wh f)$, we have with probability at least $1-\delta$
%     \begin{align}
%       \left(\frac{n}{n+m_2}\right) \left(\error_{\wt \calS_c} (\wh f)- \error_{\calS}( \wh f) \right) \le  \sqrt{\frac{\log(1/\delta)}{2m_2}} \,.
%     \end{align}
%     As before, assuming $km_2 \approx m$, we have with probability at least $1-\delta$ 
%     \begin{align}
%         \error_{\wt \calS_c} (\wh f)- \error_{\calS}( \wh f) \le \left(1+\frac{m_2}{n}\right)  \sqrt{\frac{k\log(1/\delta)}{2m}} \le \left( 1 + \frac{1}{k}\right) \sqrt{\frac{k\log(1/\delta)}{2m}} \,. \label{eq:lemma3_final_multi}
%     \end{align} 
% \end{proof}

% \begin{proof}[Proof of \thmref{thm:multiclass_ERM}] 
%     Having established these core intermediate results, we can now combine above three lemmas. 
%     In particular, we bound the population error on clean data ($\error_\calD(\wh f)$) as follows:  
%     \begin{enumerate}[(i)]
%         \item First, use \eqref{eq:lemma1_final_multi}, to obtain an upper bound on the population error on clean data, i.e., with probability at least $1-\delta/4$, we have
%         \begin{align}
%             \error_{ \calD} (\wh f) \le (k-1)\left(1 - \error_{ \wt \calS_M}(\wh f) \right) + (k-1) \sqrt{\frac{k\log(4/\delta)}{2(k-1)m}} \,. 
%         \end{align}
%         \item  Second, use \eqref{eq:lemma2_final_multi}, to relate the error on the mislabeled fraction with error on clean portion of randomly labeled data and error on whole randomly labeled dataset, i.e., with probability at least $1-\delta/2$, we have 
%         \begin{align}
%             - (k-1)\error_{\wt S_M}(f) \le \error_{\wt S_C}(f) - k\error_{\wt S}  + k\sqrt{\frac{\log(4/\delta)}{2m}}  \,. 
%         \end{align} 
%         \item Finally, use \eqref{eq:lemma3_final_multi} to relate the error on the clean portion of randomly labeled data and error on clean training data, i.e., with probability $1-\delta/4$, we have 
%         \begin{align}
%             \error_{\wt \calS_C} (\wh f)\le - \error_{\calS}( \wh f) + \left(1 + \frac{m}{kn} \right) \sqrt{\frac{k\log(4/\delta)}{2m}} \,. 
%         \end{align} 
%     \end{enumerate}

%     Using union bound on the above three steps, we have with probability at least $1-\delta$: 
%     \begin{align}
%         \error_\calD (\wh f) \le \error_{\calS}(\wh f) + (k-1) - k\error_{\wt \calS}(\wh f)   + (\sqrt{k(k-1)} + k + \sqrt{k} + \frac{m}{n\sqrt{k}})  \sqrt{\frac{\log(4/\delta)}{2m}} \,.
%     \end{align}
%     % Note that $\frac{m}{n\sqrt{k}}$ is much smaller than the other terms in addition. Hence, we ignore this in the final bound. 
%     % Note that $(1/\sqrt{2} + 2.5)$ is a loose constant. In experiments, we use the ratio $\frac{m}{n}$
%     %  the exact error $\error_{\wt \calS}(\wh f)$ 
%     % to evaluate R.H.S.    
% \end{proof}

% \newpage
% \section{Proofs from \secref{sec:linear_models}}\label{app:proof_gd}

% We suppose that the parameters of the linear function 
% are obtained via gradient descent on 
% the following $L_2$ regularized problem: 
% \begin{align}
%     % n in denominator is avoided deliberately
%     \calL_S(w; \lambda) \defeq \sum_{i=1}^n{(w^Tx_i - y_i)^2} + \lambda \norm{w}{2}^2 \,, \label{eq:l2_MSE_app}   
% \end{align}
% where $\lambda\ge0$ is a regularization parameter. 
% We assume access to a clean dataset 
% $S = \{(x_i, y_i)\}_{i=1}^n \sim \calD^n$ 
% and randomly labeled dataset 
% $\wt S = \{(x_i, y_i)\}_{i=n+1}^{n+m} \sim \wt \calD^m$. 
% Let $\bX = [x_1, x_2, \cdots, x_{m+n}]$ 
% and $\by = [y_1, y_2, \cdots, y_{m+n}]$. 
% Fix a positive learning rate $\eta$ such that 
% $\eta \le 1/\left(\norm{\bX^T\bX}{\text{op}} + \lambda^2\right)$ 
% and an initialization $w_0 = 0$. 
% % \todos{Assumption made for simplicty}. 
% Consider the following gradient descent iterates 
% to minimize objective \eqref{eq:l2_MSE_app} on $S \cup \wt S$:
% \begin{align}
% w_t = w_{t-1} - \eta \grad_w \calL_{S \cup \wt S} (w_{t-1}; \lambda) \quad \forall t=1,2,\ldots \label{eq:GD_iterates_app}
% \end{align} 
% Then we have $\{ w_t\}$ converge to the limiting solution 
% $\wh w = \left( \bX^T\bX+\lambda \boldsymbol{I}\right)^{-1}\bX^T\by$. Define $\widehat f (x) \defeq f(x ; \wh w) $.  

% \subsection{\textcolor{red}{Errata}}

% We wish to correct the following error in the body: \codref{cond:error_stability} is not enough to guarantee the result in \thmref{thm:linear}. We now present a slightly stronger condition called \emph{hypothesis stability} under which we obtain a result similar to \thmref{thm:linear}. 

% This error doesn't change the main arguments of the proof where we show that the empirical train error is less than or equal to the leave-one-out error. We need a stronger condition to relate leave-one-out error with the population error of the original classifier. Specifically, while \codref{cond:error_stability} relates the average population error of leave-one-out classifiers with the population error of the original classifier, we need the new condition to show the concentration of the empirical leave-one-out error and  average population error of leave-one-out classifiers. 
% % main takeaway 

% Note that the new condition, while being stronger than the previous one, still doesn't imply generalization~\cite{bousquet2002stability,elisseeff2003leave,abou2019exponential}. Overall, the main results in \secref{sec:ERM_training} and takeaways of the paper remain unaffected by the error.  

% We now present the new condition and a corrected statement of \thmref{thm:linear}. Recall, for a given training set $S \sim \calD^n $, 
% we use $S_{(i)}$ to denote the training set $S$ 
% with the $i^{\text{th}}$ point removed.

% \begin{condition}[Hypothesis Stability] 
%     \label{cond:hypothesis_stability}
%     We have $\beta$ hypothesis stability 
%     if our training algorithm $\calA$ satisfies the following: 
%     \begin{align*}
%     % ${\sum_{i=1}^n \frac{\error_{\calD}( f(\calA, S_{(i)}))}{n} - \error_\calD(f(\calA, S))} \le \beta\,$.
%     \forall i \in \{1,2,\ldots, n\}, \quad  \Expt{\calS, (x,y) \in \calD}{ \abs{\error\left( f(x) ,y  \right) - \error\left( f_{(i)}(x), y \right) }} \le \frac{\beta}{n} \,,
%     \end{align*}
%     where $f_{(i)} \defeq f(\calA, S_{(i)})$ and $ f \defeq f(\calA, S)$.
% \end{condition}

% \begin{theorem}[Correct statement of \thmref{thm:linear}] \label{thm:new_linear}
%     Assume that this gradient descent algorithm satisfies \codref{cond:hypothesis_stability}
%     with $\beta=\calO(1)$.  
%     Then for any $\delta >0$, with probability at least $1-\delta$ 
%     over the random draws of datasets $\wt S$ and $S$, we have:
%     \begin{align}
%         \error_\calD(\widehat f) \le \error_\calS(\widehat f) + 1 - 2 \error_{\wt\calS}(\widehat f) + \left(\frac{1}{\sqrt{2}} + 1.5 \right) \sqrt{\frac{\log(4/\delta)}{m}} + \sqrt{\frac{4}{\delta}\left(\frac{1}{m} +\frac{3\beta}{m+n} \right)}  \,. \label{eq:gd_error}
%     \end{align} 
%     % for some constant $c\le 3.2$.
% \end{theorem}

% \subsection{Proof of \thmref{thm:new_linear}}
% We use a standard result from linear algebra, namely Shermann-Morrison formula~\citep{sherman1950adjustment} for matrix inversion:  

% \begin{lemma}[\citet{sherman1950adjustment}] \label{lem:sherman}
%     Suppose $\bA \in \Real^{n \times n}$ is an invertible square matrix and $u,v \in \Real^n$ are column vectors. Then $\bA + uv^T$ is invertible iff $1 + v^T \bA u \ne 0$ and in particular
%     \begin{align}
%         (\bA + u v^T)^{-1} = \bA^{-1}  - \frac{\bA^{-1} uv^T \bA^{-1} }{ 1 + v^T \bA^{-1} u} \,.
%     \end{align}   
% \end{lemma}
% \newcommand\byy[1]{\by_{\left(#1\right)}}
% \newcommand\bXX[1]{\bX_{\left(#1\right)}}
% \newcommand\ff[1]{\wh f_{\left(#1\right)}}

% For a given training set $S \cup \wt S_C$, define leave-one-out error on mislabeled points in the training data as $$\error_{\text{LOO}(\wt S_M) } = \frac{\sum_{(x_i, y_i) \in \wt S_M} \error( f_{(i)}( x_i), y_i)}{ \abs{\wt S_M }} \,, $$
% where $f_{(i)} \defeq f(\calA, (S \cup \wt S)_{(i)})$. To relate empirical leave-one-out error and population error with hypothesis stability condition, we use the following lemma:   

% \begin{lemma}[\citet{bousquet2002stability}] \label{lem:stability_error}
%     For the leave-one-out error, we have
%     \begin{align}
%         \Expo{ \left( \error_{\calDm}(\wh f) -\error_{\text{LOO}(\wt S_M) } \right)^2 } \le \frac{1}{2m_1}+  \frac{3\beta}{n + m}\,.
%     \end{align}   
%     % where $ f \defeq f(\calA, S \cup \wt S) $.
% \end{lemma}

% Proof of the above lemma is similar to the proof of  Lemma 9 in \citet{bousquet2002stability} and can be found in \appref{app:proof_lem_error}. 
% % 
% % Before presenting the result, we introduce some notation. 
% Before presenting the proof of \thmref{thm:new_linear}, we introduce some more notation. Let $\bX_{(i)}$ denote the matrix of covariates with $i^{\text{th}}$ point removed. Similarly let $\by_{(i)}$ be the array of responses with $i^{\text{th}}$ point removed. Define the corresponding regularized GD solution as $\wh w_{(i)} = \left( \bXX{i}^T\bXX{i}+\lambda \boldsymbol{I}\right)^{-1}\bXX{i}^T\byy{i}$. Define $\ff{i}(x) \defeq f(x ; \wh w_{(i)}) $.

% \begin{proof}[Proof of \thmref{thm:new_linear}]
%     Because squared loss minimization does not imply 0-1 error minimization, we cannot use arguments from \lemref{lem:fit_mislabeled}. This is the main technical difficulty. To compare the 0-1 error at a train point with an unseen point, 
%     we use the closed-form expression for $\widehat{w}$ and Shermann-Morrison formula to upper bound training error with leave-one-out cross validation error. 
    
%     The proof is divided into three parts: In part one, we show that 0-1 error on mislabeled points in the training set is lower than the error obtained by leave-one-out error at those points. In part two, we relate this leave-one-out error with the population error on mislabeled distribution using \codref{cond:hypothesis_stability}. While the empirical leave-one-out error is unbiased estimator of the average population error of leave-one-out classifiers, we need hypothesis stability to control the variance of empirical leave-one-out error. Finally in part three, we show that the error on the mislabeled training points can be estimated with just the randomly labeled and  clean training data (as in proof of \thmref{thm:error_ERM}).  

%     \textbf{Part 1 {} {}} First we relate training error with leave-one-out error.        
%     For any 
%     training point $(x_i, y_i)$ in $\wt S \cup S$, we have 
%     \begin{align}
%         \error(\wh f(x_i), y_i ) &= \indict{ y_i \cdot x_i^T \wh w < 0 } = \indict{ y_i \cdot x_i^T \left( \bX^T\bX+\lambda \boldsymbol{I}\right)^{-1}\bX^T\by < 0 } \\
%         &= \indict{ y_i \cdot x_i^T \underbrace{\left( \bXX{i}^T\bXX{i} + x_i ^T x_i +\lambda \boldsymbol{I}\right)^{-1}}_{\RN{1}} (\bXX{i}^T\byy{i} + y \cdot x_i) < 0 }
%     \end{align}
%     Letting $\bA = \left(\bXX{i}^T\bXX{i} +\lambda \boldsymbol{I}\right)$ and using \lemref{lem:sherman} on term 1, we have 
%     \begin{align}
%         \error(\wh f(x_i), y_i ) &= \indict{ y_i \cdot x_i^T \left[\bA^{-1} -  \frac{\bA^{-1} x_i x_i^T \bA^{-1}}{ 1 + x_i ^T \bA^{-1} x_i } \right] (\bXX{i}^T\byy{i} + y \cdot x_i) < 0 } \\
%         &= \indict{ y_i \cdot\left[ \frac{ x_i^T \bA^{-1} ( 1 + x_i ^T \bA^{-1} x_i ) -  x_i^T \bA^{-1} x_i x_i^T \bA^{-1}}{ 1 + x_i ^T \bA ^{-1}x_i } \right] (\bXX{i}^T\byy{i} + y \cdot x_i) < 0 } \\
%         &= \indict{ y_i \cdot\left[ \frac{ x_i^T \bA^{-1}}{ 1 + x_i ^T \bA ^{-1}x_i } \right] (\bXX{i}^T\byy{i} + y \cdot x_i) < 0 } \,.
%     \end{align}

%     Since $1 + x_i^T \bA^{-1} x_i > 0$, we have 
%     \begin{align}
%         \error(\wh f(x_i), y_i ) &= \indict{ y_i \cdot x_i^T \bA^{-1} (\bXX{i}^T\byy{i} + y \cdot x_i) < 0 } \\
%         &= \indict{ x_i^T \bA^{-1} x_i +  y_i \cdot x_i^T \bA^{-1} (\bXX{i}^T\byy{i}) < 0 } \\
%         &\le \indict{ y_i \cdot x_i^T \bA^{-1} (\bXX{i}^T\byy{i}) < 0 } = \error(\ff{i}(x_i), y_i ) \,.\label{eq:LOO_error}
%     \end{align}

%     Using \eqref{eq:LOO_error}, we have 
%     \begin{align}
%         \error_{\wt \calS_M } (\wh f) \le \error_{\text{LOO} (S_M)} \defeq \frac{\sum_{(x_i, y_i) \in \wt S_M} \error(\ff{i}(x_i), y_i ) }{\abs{\wt \calS_M}}\label{eq:LOO_error_final}
%     \end{align}
%     \textbf{Part 2 {}{}} We now relate RHS in \eqref{eq:LOO_error_final} with the population error on mislabeled distribution. To do this, we leverage \codref{cond:hypothesis_stability} and \lemref{lem:stability_error}. In particular, we have 

%     \begin{align}
%         \Expt{\calS \cup \wt \calS_M }{ \left(\error_{\calDm}(\wh f) - \error_{\text{LOO} (S_M)}\right)^2 } \le \frac{1}{2m_1} + \frac{3\beta}{m+n} \,.
%     \end{align}

%     Using Chebyshev's inequality, with probability at least $1-\delta$, we have 
%     \begin{align}
%         \error_{\text{LOO} (S_M)} \le  \error_{\calDm}(\wh f)   + \sqrt{\frac{1}{\delta}\left(\frac{1}{2m_1} +\frac{3\beta}{m+n} \right)} \,. \label{eq:final_mislabeled_linear}
%     \end{align}
    

%     \textbf{Part 3 {}{}} Combining \eqref{eq:final_mislabeled_linear} and \eqref{eq:LOO_error_final}, we have 

%     \begin{align}
%         \error_{\wt \calS_M } (\wh f) \le \error_{\calDm}(\wh f)   + \sqrt{\frac{1}{\delta}\left(\frac{1}{2m_1} +\frac{3\beta}{m+n} \right)} \,. \label{eq:linear_parallel_lem1}
%     \end{align}

%     Compare \eqref{eq:linear_parallel_lem1}, with \eqref{eq:lemma1_final} in the proof of \lemref{lem:fit_mislabeled}. We obtain a similar relationship between $\error_{\wt \calS_M }$ and $\error_{\calDm}$ but with a polynomial concentration instead of exponential concentration. 
%     In addition, since we just use concentration arguments to relate mislabeled error with the error on clean portion and unlabeled portion, we can directly use the results in \lemref{lem:mislabeled_error} and \lemref{lem:clear_error}. Therefore, combining results in \lemref{lem:mislabeled_error}, \lemref{lem:clear_error}, and \eqref{eq:linear_parallel_lem1} with union bound, we have with probability at least $1-\delta$

%     \begin{align}
%         \error_\calD(\widehat f) \le \error_\calS(\widehat f) + 1 - 2 \error_{\wt\calS}(\widehat f) + \left(\frac{1}{\sqrt{2}} + 1.5 \right) \sqrt{\frac{\log(4/\delta)}{m}} + \sqrt{\frac{4}{\delta}\left(\frac{1}{m} +\frac{3\beta}{m+n} \right)}  \,.
%     \end{align}
    

       
% \end{proof}

% \subsection{Discussion on \codref{cond:hypothesis_stability}}

% The quantity in LHS of \codref{cond:hypothesis_stability} measures how much the function learned by the algorithm (in terms of error on unseen point) will change when one point in the training set is removed. 
% % Discussion on exponential concentration and stronger condition. 
% Notice that hypothesis stability implies error stability, i.e., \codref{cond:error_stability} ~\cite{bousquet2002stability}.  In summary, while error stability allowed us to relate the average population error of the leave-one-out classifiers with the population error of the original classifier, we need hypothesis stability condition to control the variance of the empirical leave-one-out error. 

% Additionally, we note that while the dominating term in the RHS of \thmref{thm:new_linear} matches with the dominating term in ERM bound in \thmref{thm:error_ERM}, there is a polynomial concentration term (dependence on $1/\delta$ instead of $\log(\sqrt{1/\delta})$) in  \thmref{thm:new_linear}. 
% Since with hypothesis stability, we just bound the variance,  the polynomial concentration is due to the use of Chebyshev's inequality instead of an exponential tail inequality (as in \lemref{lem:fit_mislabeled}).
% Recent works have highlighted that slightly stronger condition than hypothesis stability can be used to obtained an exponential concentration for leave-one-out error~\citep{abou2019exponential}, but we leave this for future work for now. 
% % We leave 
% % However, the constants 

% % we also want to highlight  

% \subsection{Formal statement and proof of  of \propref{prop:early_stop}}

% Before formally presenting the result, we will introduce some notation.  By $\calL_{S}(w)$, we denote 
% the objective in \eqref{eq:l2_MSE_app} with $\lambda=0$. 
% Assume Singular Value Decomposition (SVD) of $\bX$  as $\sqrt{n} \bU \bS^{1/2} \bV^T$. Hence $\bX^T \bX = \bV \bS \bV^T$.
% Consider the GD iterates defined in \eqref{eq:GD_iterates_app}. 
% % 
% We now derive closed form expression for the $t^\text{th}$ iterate of gradient descent:  
% % 
% \begin{align}
%     w_t = w_{t-1} + \eta \cdot \bX^T (\by - \bX w_{t-1}) = (\bI - \eta \bV \bS \bV^T )w_{k-1} + \eta \bX^T \by \,.
% \end{align}
% Rotating by $\bV^T$, we get 
% \begin{align}
%     \wt w_t = (\bI - \eta\bS )\wt w_{k-1} + \eta \wt \by \,, \label{eq:GD_recur}
% \end{align}
% where $\wt w_t = \bV^T w_t $ and $\wt \by = \bV^T \bX^T \by$. Assuming the initial point $w_0 = 0$ and applying the recursion in \eqref{eq:GD_recur}, we get
% \begin{align}
%     \wt w_t = \bS ^{-1} ( \bI - (\bI - \eta \bS)^k ) \wt \by \,, 
% \end{align} 
% Projecting solution back to the original space, we have 
% \begin{align}
%      w_t = \bV \bS ^{-1} ( \bI - (\bI - \eta \bS)^k ) \bV^T \bX^T \by \,, 
% \end{align} 
% % We will work with this GD solution at any iterate $t$ in the next proposition. 
% Define $f_t(x) \defeq f(x;w_t)$ as the solution at the $t^{\text{th}}$ iterate. 
% Let $\wt w_{\lambda} = \argmin_{w} \calL_\calS (w;\lambda) = (\bX^T \bX + \lambda \bI)^{-1} \bX^T \by = \bV (\bS + \lambda \bI )^{-1} \bV^T \bX^T \by $. 
% % ) \,,$ for all $t=1,2,\ldots\,.$ 
% and define $\wt f_\lambda(x) \defeq f(x;\wt w_\lambda)$ as the regularized solution. 
% Assume $\kappa$ be the condition number of the population covariance matrix 
% and 
% let $s_\text{min}$ be the minimum positive singular value of the empirical covariance matrix. Our proof idea is inspired from recent work on relating gradient flow solution and regularized solution for regression problems \citep{ali2018continuous}. We will use the following lemma in the proof: 
% \begin{lemma} \label{lem:ineq_soln}
%     For all $x \in [0,1]$ and for all $ k \in \mathbb{N}$, we have (a) $ \frac{kx}{1+kx} \le 1- (1-x)^k$ and (b) $ 1- (1-x)^k \le 2 \cdot \frac{kx}{kx+1} $.
%     %  where $g(c)$ is a constant dependent on $c$. For $c = 1$, $g(c) = 2.0$.   
% \end{lemma}
% \begin{proof}
%     % [Proof of \lemref{lem:ineq_soln}]
%     % Part (a) is easy. 
%     Using $ (1-x)^k \le \frac{1}{1+kx}$, we have part (a). For part (b), we numerically maximize $\frac{ (1+kx ) (1 - (1-x)^k) }{kx}$ for all $k\ge 1$ and for all $x \in [0, 1]$.  
% \end{proof}

% % 
% % Next, 

% \begin{prop}[Formal statement of \propref{prop:early_stop}] \label{prop:formal_early_stop}
% Let $\lambda = \frac{1}{t\eta}$. For a training point $x$, we have 
% \begin{align*}
%     \Expt{x \sim \calS}{(f_t(x) - \wt f_\lambda(x))^2} &\le c(t,\eta) \cdot \Expt{x \sim \calS}{f_t(x)^2} \,, %\label{eq:early_stop}
% \end{align*}
% where $c(t, \eta) \defeq \min( 0.25, \frac{1}{s_\text{min}^2 t^2 \eta^2})$. Similarly for a test point, we have 
% \begin{align*}
%     \Expt{x \sim \calD_\calX}{(f_t(x) - \wt f_\lambda(x))^2} &\le \kappa \cdot c(t,\eta) \cdot \Expt{x \sim \calD_\calX}{f_t(x)^2} \,. %\label{eq:early_stop}
% \end{align*}
% \end{prop} 

% \begin{proof}
%     %%%%%%%%%%%%% 
%     We want to analyze the expected squared difference output of regularized linear regression with regularization constant $\lambda = \frac{1}{\eta t}$ and gradient descent solution at $t^\text{th}$ iterate. We separately expand the algebraic expression for squared difference at a training point and a test point. 
%     % We start by considering the difference  
%     Then the main step is to show that  $\left[ \bS ^{-1} ( \bI - (\bI - \eta \bS)^k )  - (\bS + \lambda \bI )^{-1}\right] \preceq c(\eta, t) \cdot \bS ^{-1} ( \bI - (\bI - \eta \bS)^k ) $.

%     %%%%%%%%%%%%%
    
%   \textbf{Part 1 {} {}} 
%     First, we will analyze the squared difference of output at a training point (for simplicity, we refer to $S \cup \wt S$ as $S$), i.e. 
%     \begin{align}
%         \Expt{ x \sim \calS }{\left(f_t(x) - \wt f_\lambda (x)\right)^2} &= \norm{\bX w_t - \bX \wt w_\lambda}{2}^2 =   \norm{\bX \bV \bS ^{-1} ( \bI - (\bI - \eta \bS)^t ) \bV^T \bX^T \by - \bX \bV (\bS + \lambda \bI )^{-1} \bV^T \bX^T \by }{2}^2 \\
%         &= \norm{\bX \bV \left(\bS ^{-1} ( \bI - (\bI - \eta \bS)^t ) - (\bS + \lambda \bI )^{-1} \right) \bV^T \bX^T \by  }{2} \\
%         &=  \by^T \bV \bX \left( \underbrace{\bS ^{-1} ( \bI - (\bI - \eta \bS)^t ) - (\bS + \lambda \bI )^{-1}}_{\RN{1}} \right)^2 \bS \bV^T \bX^T \by \label{eq:train_GD_rel}
%         %  (\bX \bV \bS ^{-1} ( \bI - (\bI - \eta \bS)^k ) \bV^T \bX^T \by)^T \bX \bV \bS ^{-1} ( \bI - (\bI - \eta \bS)^k ) \bV^T \bX^T \by
%     \end{align}
%     We now separately consider term 1. Substituting $\lambda = \frac{1}{t \eta}$, we get
%     \begin{align}
%         \bS ^{-1} ( \bI - (\bI - \eta \bS)^t ) - (\bS + \lambda \bI )^{-1} &= \bS^{-1} \left( ( \bI - (\bI - \eta \bS)^t ) - (\bI + \bS^{-1} \lambda )^{-1}\right) \\
%         &= \underbrace{\bS^{-1} \left( ( \bI - (\bI - \eta \bS)^t ) - (\bI + ( \bS t \eta)^{-1}  )^{-1}\right)}_{\bA}
%     \end{align}

%     We now separately bound the diagonal entries in matrix $\bA$. 
%     With $s_i$, we denote $i^{\text{th}}$ diagonal entry of $\bS$. Note that since $ \eta\le 1/\norm{S}{\text{op}}$, for all $i$, $\eta s_i  \le 1$.  Consider $i^{\text{th}}$ diagonal term (which is non-zero) of the diagonal matrix $\bA$, we have 
%     \begin{align}
%         \bA_{ii} = \frac{1}{s_i} \left(  1 - (1 - s_i \eta)^t - \frac{t \eta s_i}{1 + t \eta s_i } \right) &=  \frac{1 - (1 - s_i \eta)^t}{s_i} \left( \underbrace{ 1 - \frac{t \eta s_i}{(1 + t \eta s_i)(1 - (1 - s_i \eta)^t)}}_{\RN{2}} \right) \\ 
%          &\le \frac{1}{2}\left[ \frac{1 - (1 - s_i \eta)^t}{ s_i} \right] \tag*{(Using \lemref{lem:ineq_soln} (b))} \,.
%     \end{align} 
%     Additionally, we can also show the following upper bound on term 2: 
%     \begin{align}
%          1 - \frac{t \eta s_i}{(1 + t \eta s_i)(1 - (1 - s_i \eta)^t)} &= \frac{(1 + t \eta s_i)(1 - (1 - s_i \eta)^t) - t \eta s_i }{(1 + t \eta s_i)(1 - (1 - s_i \eta)^t)} \\
%          & \le  \frac{ 1 -  (1 - s_i \eta)^t - t \eta s_i (1 - s_i \eta)^t}{(1 + t \eta s_i)(1 - (1 - s_i \eta)^t)} \\
%          & \le \frac{1}{t\eta s_i} \,. \tag{Using \lemref{lem:ineq_soln} (a)}
%         %  &\le \frac{1}{2}\left[ \frac{1 - (1 - s_i \eta)^t}{ s_i} \right] \tag*{(Using \lemref{lem:ineq_soln})} \,.
%     \end{align} 

%     Combining both the upper bounds on each diagonal entry $\bA_{ii}$, we have 
%     \begin{align}
%     \bA \preceq c_1(\eta, t) \cdot \bS^{-1} ( \bI - (\bI - \eta \bS)^t ) \,, \label{eq:upperbound_diagonal}
%     \end{align}
%     where $c_1(\eta, t ) = \min(0.5, \frac{1}{t s_i \eta })$. Plugging this into \eqref{eq:train_GD_rel}, we have 
%     \begin{align}
%         \Expt{ x \sim \calS }{\left(f_t(x) - \wt f_\lambda (x)\right)^2} &\le c(\eta, t) \cdot \by^T \bV \bX  \left( \bS^{-1} ( \bI - (\bI - \eta \bS)^t ) \right)^2 \bS \bV^T \bX^T \by \\
%         &=   c(\eta, t) \cdot \by^T \bV \bX  \left( \bS^{-1} ( \bI - (\bI - \eta \bS)^t ) \right) \bS \left( \bS^{-1} ( \bI - (\bI - \eta \bS)^t ) \right) \bV^T \bX^T \by \\
%         & =  c(\eta, t) \cdot \norm{\bX w_t}{2}^2 \\
%         &= c(\eta, t) \cdot  \Expt{ x \sim \calS }{\left(f_t(x) \right)^2} \,,
%     \end{align}
%     where $c(\eta, t ) = \min(0.25, \frac{1}{t^2 s^2_i \eta^2 })$.

%     \textbf{Part 2 {} {}} With $\bSigma$, we denote the underlying true covariance matrix. We now consider the squared difference of output at an unseen point: 
%     \begin{align}
%         \Expt{ x \sim \calD_{\calX} }{\left(f_t(x) - \wt f_\lambda (x)\right)^2} &= \Expt{x \sim \calD_{\calX}}{\norm{x^T w_t - x^T \wt w_\lambda}{2}} \\
%         &=   \norm{x^T \bV \bS ^{-1} ( \bI - (\bI - \eta \bS)^t ) \bV^T \bX^T \by - x^T \bV (\bS + \lambda \bI )^{-1} \bV^T \bX^T \by }{2} \\
%         &= \norm{x^T \bV \left(\bS ^{-1} ( \bI - (\bI - \eta \bS)^t ) - (\bS + \lambda \bI )^{-1} \right) \bV^T \bX^T \by  }{2} \\
%         &= \by^T \bV \bX \left( \bS ^{-1} ( \bI - (\bI - \eta \bS)^t ) - (\bS + \lambda \bI )^{-1} \right) \bV^T \bSigma \bV \\ &\qquad \qquad \qquad \qquad \qquad \left( (\bI - (\bI - \eta \bS)^t ) - (\bS + \lambda \bI )^{-1} \right) \bV^T \bX^T \by \\
%         &\le \sigma_{\text{max}} \cdot \by^T \bV \bX \left( \underbrace{\bS ^{-1} ( \bI - (\bI - \eta \bS)^t ) - (\bS + \lambda \bI )^{-1}}_{\RN{1}} \right)^2 \bV^T \bX^T \by \,, \label{eq:test_GD_rel}
%         %  (\bX \bV \bS ^{-1} ( \bI - (\bI - \eta \bS)^k ) \bV^T \bX^T \by)^T \bX \bV \bS ^{-1} ( \bI - (\bI - \eta \bS)^k ) \bV^T \bX^T \by
%     \end{align}
%     where $\sigma_{\text{max}}$ is the maximum eigenvalue of the underlying covariance matrix $\bSigma$. Using the upper bound on term 1 in \eqref{eq:upperbound_diagonal}, we have 
%     \begin{align}
%         \Expt{ x \sim \calD_{\calX} }{\left(f_t(x) - \wt f_\lambda (x)\right)^2} &\le \sigma_{\text{max}} \cdot c(\eta, t) \cdot \by^T \bV \bX  \left( \bS^{-1} ( \bI - (\bI - \eta \bS)^t ) \right)^2 \bV^T \bX^T \by \\
%         &=   \kappa \cdot c(\eta, t) \cdot \sigma_{\text{min}}\cdot \norm{\bV \left( \bS^{-1} ( \bI - (\bI - \eta \bS)^t ) \right) \bV^T \bX^T \by}{2}^2 \\
%         &\le \kappa \cdot c(\eta, t) \cdot \left[ \bV \left( \bS^{-1} ( \bI - (\bI - \eta \bS)^t ) \right) \bV^T \bX^T \right]^T \bSigma \\
%         &\qquad \qquad \qquad \qquad \qquad \left[ \bV \left( \bS^{-1} ( \bI - (\bI - \eta \bS)^t ) \right) \bV^T \bX^T \right] \by \\
%         & = \kappa \cdot c(\eta, t) \cdot \Expt{x \sim \calD_{\calX}}{\norm{x^T w_t}{2}} \,.
%     \end{align}
% % 
% % 
%     % Since $ \eta\le 1/\norm{S}{\text{op}}$, invoking \lemref{lem:ineq_soln} to upper bound term 1 with
% \end{proof}


% \newpage
% \section{Additional experiments and details}\label{app:exp}
% \newcommand\tab[1][1cm]{\hspace*{#1}}

% \subsection{Datasets} \label{sec:app_dataset}

% \textbf{Toy Dataset {} {}} Assume fixed constants $\mu$ and $\sigma$. For a given label $y$, we simulate features $x$ in our toy classification setup as follows: 
% \begin{align*}
%     x \defeq \texttt{concat} \left[ x_1, x_2\right] \quad \text{where} \quad  x_1 \sim  \calN( y \cdot \mu, \sigma^2 I_{d \times d}) \ \  \text{and} \ \  x_1 \sim  \calN( 0, \sigma^2 I_{d \times d}) \,.
% \end{align*}  
% % where $y$ is the true label and $x$ is the corresponding feature vector. 
% In experiements throughout the paper, we fix dimention $d=100$, $\mu = 1.0 $, and $\sigma = \sqrt{d}$. Intuitively, $x_1$ carries the information about the underlying label and $x_2$ is additional noise independent of the underlying label. 

% \textbf{CV datasets {} {}} We use MNIST~\citep{lecun1998mnist} and CIFAR10~\cite{krizhevsky2009learning}. 
% % For binary tasks, 
% We produce a binary variant from the multiclass classification problem by mapping classes $\{0,1,2,3,4\}$ to label $1$ and $\{ 5,6,7,8,9\}$ to label $-1$. For CIFAR dataset, we also use the standard data augementation of random crop and horizontal flip. PyTorch code is as follows: 

% \texttt{(transforms.RandomCrop(32, padding=4),\\
% \tab transforms.RandomHorizontalFlip())}

% \textbf{NLP dataset {} {}} We use IMDb Sentiment analysis~\citep{maas2011learning} corpus.  

% \subsection{Architecture Details} 

% All experiments were run on NVIDIA GeForce RTX 2080 Ti GPUs. We used PyTorch~\citep{NEURIPS2019a9015} and Keras with Tensorflow~\citep{abadi2016tensorflow} backend for experiments. 
% % , ELMo embeddings~\citep{Peters:2018}, and Hugging Face Transformers~\citep{wolf-etal-2020-transformers}. 

% \textbf{Linear model {} {}} For the toy dataset, we simulate a linear model with scalar output and the same number of parameters as the number of dimensions.   

% \textbf{Wide nets {} {}} To simulate the NTK regime, we experiment with $2-$layered wide nets. The PyTorch code for 2-layer wide MLP is as follows: 


% \texttt{ nn.Sequential( \\
% \tab     nn.Flatten(),\\
% \tab    nn.Linear(input\_dims, 200000, bias=True),\\
% \tab    nn.ReLU(),\\
% \tab    nn.Linear(200000, 1, bias=True)\\
% \tab     )}


% We experiment both (i) with the first layer fixed at random initialization; (ii)  and updating both layers' weights.     

% \textbf{Deep nets for CV tasks {} {}} We consider a 4-layered MLP. The PyTorch code for 4-layer MLP is as follows: 

% \texttt{ nn.Sequential(nn.Flatten(), \\
% \tab        nn.Linear(input\_dim, 5000, bias=True),\\
% \tab        nn.ReLU(),\\
% \tab        nn.Linear(5000, 5000, bias=True),\\
% \tab        nn.ReLU(),\\
% \tab        nn.Linear(5000, 5000, bias=True),\\
% \tab        nn.ReLU(),\\
% % \tab        nn.Linear(5000, 5000, bias=True),\\
% % \tab        nn.ReLU(),\\
% \tab        nn.Linear(1024, num\_label, bias=True)\\
% \tab        )}

% For MNIST, we use $1000$ nodes instead of $5000$ nodes in the hidden layer. 
% % 
% We also experiment with convolutional nets. In particular, we use ResNet18 \citep{he2016deep}. Implementation adapted from:  \url{https://github.com/kuangliu/pytorch-cifar.git}. 

% \textbf{Deep nets for NLP {} {}} We use a simple LSTM model with embeddings intialized with ELMo embeddings~\citep{Peters:2018}. Code adapted from: \url{https://github.com/kamujun/elmo_experiments/blob/master/elmo_experiment/notebooks/elmo_text_classification_on_imdb.ipynb} 

% We also evaluate our bounds with a BERT model. In particular, we fine-tune an off-the-shelf uncased BERT model~\citep{devlin2018bert}. Code adapted from Hugging Face Transformers~\citep{wolf-etal-2020-transformers}: \url{https://huggingface.co/transformers/v3.1.0/custom_datasets.html}. 


% \subsection{Additonal experiments}

% 1. SGD with linear models on cross entropy and MSE loss. 

% 2. CE loss and SGD. GD with MSE loss 

% 3. Binary MNIST with MLP. multiclass MNIST  

% \textbf{Results on CIFAR 10 {} {}} 
% % 
% We plot epoch wise error curve for results in \tabref{table:multiclass}. We observe the same trend as in \figref{fig:error_CIFAR10}. Additionally, we plot an \emph{oracle bound} obtained by tracking the error on mislabeled data which nevertheless were predicted as true label. To obtain an exact emprical value of the oracle bound, we need underlying true labels for the randomly labeled data. 
% % Note that our bound in \thmref{thm:multiclass_ERM}, lower bounds the accuracy as predicted by the oracle bound. 
% While with just access to extra unlabeled data we cannot calculate oracle bound, we note that the oracle bound is very tight and never violated in practice underscoring an importamt aspect of generalization in multiclass problems. This highlight that even a stronger conjecture may hold in multiclass classification, i.e., error on mislabeled data (where nevertheless true label was predicted) lower bounds the population error on the distribution of mislabeled data and hence, the error on (a specific) mislabeled portion predicts the population accuracy on clean data. 
% % 
% On the other hand, the dominating term of in \thmref{thm:multiclass_ERM} is loose when compared with the oracle bound. The main reason, we believe is the pessimistic upper bound in \eqref{eq:lemma1_final_multi_prev} in the proof of \lemref{lem:fit_mislabeled_multi}. We leave an investigation on this gap for future. 
% % of fit 

% % However, oracle bound highlights two . One,  



% \begin{figure}[h]
%     \centering 
%     % \vspace{-15pt}
%     % \includegraphics[width=0.9\linewidth]{example-image-a}
%     \includegraphics[width=0.4\linewidth]{figures/CIFAR10-FNN.pdf} \hfil
%     \includegraphics[width=0.4\linewidth]{figures/CIFAR10-Resnet.pdf}
%     % \includegraphics[width=0.9\linewidth]{figures/{CIFAR10_rn=0.1_lr=0.2_wd=0.005}.png}
%     % \vspace{-10pt}
%     \caption{ Per epoch curves for CIFAR10 corresponding results in \tabref{table:multiclass}. As before, we just plot the dominating term in the RHS of \eqref{eq:multiclass_ERM} as predicted bound. Additionally, we also plot the predicted lower bound by the error on mislabeled data which nevertheless were predicted as true label. We refer to this as ``Oracle bound''. See text for more details. 
%     % 
%     % except for the stopping point. 
%     % The bound predicted by RATT (RHS in \eqref{eq:multiclass_ERM}) is vacuous. 
%     }\label{fig:error_epoch_CIFAR10}
%     % \vspace{-15pt}
% \end{figure}


% \textbf{Results on CIFAR 100 {} {}} 
% % 
% On CIFAR100, our bound in \eqref{eq:multiclass_ERM} yields vacous bounds. However, the oracle bound as explained above yields tight guarantees in the initial phase of the learning (i.e., when learning rate is less than $0.1$). 

% \begin{figure}[h]
%     \centering 
%     % \vspace{-15pt}
%     % \includegraphics[width=0.9\linewidth]{example-image-a}
%     \includegraphics[width=0.4\linewidth]{figures/CIFAR100-Resnet.pdf}
%     % \includegraphics[width=0.9\linewidth]{figures/{CIFAR10_rn=0.1_lr=0.2_wd=0.005}.png}
%     % \vspace{-10pt}
%     \caption{ Predicted lower bound by the error on mislabeled data which nevertheless were predicted as true label with ResNet18 on CIFAR100. We refer to this as ``Oracle bound''. See text for more details. 
%     % 
%     % except for the stopping point. 
%     The bound predicted by RATT (RHS in \eqref{eq:multiclass_ERM}) is vacuous. 
%     }\label{fig:error_CIFAR100}
%     % \vspace{-15pt}
% \end{figure}


% % \paragraph{Experiments on CIFAR100} 



% \subsection{Hyperparameter Details}


% \textbf{\figref{fig:error_CIFAR10} {} {}} We use clean training dataset of size $40,000$. We fix the amount of unlabeled data at $20\%$ of the clean size, i.e. we include additional $8,000$ points with randomly assigned labels. We use test set of $10,000$ points. For both MLP and ResNet, we use SGD with an initial learning rate of $0.1$ and momentum $0.9$. We fix the weight decay parameter at $5\times 10^{-4}$. After $100$ epochs, we decay the learning rate to $0.01$. We use SGD batch size of $100$. 

% \textbf{\figref{fig:error_binary} (a) {} {}} We obtain a toy dataset according to the process described in \secref{sec:app_dataset}. We fix $d=100$ and create a dataset of $50,000$ points with balanced classes. Moreover, we sample additional covariates with the same procedure to create randomly labeled dataset. For both SGD and GD training, we use a fixed learning rate $0.1$.    

% \textbf{\figref{fig:error_binary} (b) {} {}} Similar to binary CIFAR, we use clean training dataset of size $40,000$ and fix the amount of unlabeled data at $20\%$ of the clean dataset size. To train wide nets, we use a fixed learning of $0.001$ with GD and SGD. We decide the weight decay parameter and the early stopping point that maximizes our generalization bound (i.e. without peeking at unseen data ).  We use SGD batch size of $100$. 

% \textbf{\figref{fig:error_binary} (c) {} {}} With IMDb dataset, we use a clean dataset of size $20,000$ and as before, fix the amount of unlabeled data at $20\%$ of the clean data. To train ELMo model, we use Adam optimizer with a fixed learning rate $0.01$ and weight decay $10^{-6}$ to minimize cross entropy loss. We train with batch size $32$ for 3 epochs. To fine-tune BERT model, we use Adam optimizer with learning rate $5\times 10^{-5}$ to minimize cross entropy loss. We train with a batch size of $16$ for 1 epoch.    

% \textbf{\tabref{table:multiclass} {} {}} For multiclass datasets, we train both MLP and ResNet with the same hyperparameters as described before. We sample a clean training dataset of size $40,000$ and fix the amount of unlabeled data at $20\%$ of the clean size. We use SGD with an initial learning rate of $0.1$ and momentum $0.9$. We fix the weight decay parameter at $5\times 10^{-4}$. After $30$ epochs for ResNet and after $50$ epochs for MLP, we decay the learning rate to $0.01$.  We use SGD with batch size $100$. 
% For \figref{fig:error_CIFAR100}, we use the same hyperparameters as 
% CIFAR10 training, except we now decay learning rate after $100$ epochs. 


% In all experiments, to identify the best possible accuracy on just the clean data, we use the exact same set of hyperparamters except the stopping point. We choose a stopping point that maximizes test performance. 

% \subsection{Summary of experiments }

% \begin{center}
%     \begin{table}[H] 
%         \centering
%         \begin{tabular}{|c|c|c|c|} 
%         \hline
%         Classification type & Model category & Model & Dataset  \\ [0.5ex] 
%         \hline
%         \hline
%         \multirow{9}{*}{Binary} & Low dimensional & Linear model & Toy Gaussain dataset  \\
%                         \cline{2-4}
%                          & \multirow{1}{*}{Overparameterized linear nets} 
%                         %  & Linear model & Toy Gaussain dataset \\
%                         %  \cline{3-4}
%                         %  & & 2-layer wide net& Toy Gaussain dataset \\
%                         %  \cline{3-4}
%                          & 2-layer wide net & Binary MNIST \\
%                          \cline{2-4}                 
%                          & \multirow{6}{*}{Deep nets} & \multirow{2}{*}{MLP} & Binary MNIST \\
%                          \cline{4-4}
%                          & &  & Binary CIFAR \\
%                          \cline{3-4}
%                          &  & \multirow{2}{*}{ResNet} & Binary MNIST \\
%                          \cline{4-4}
%                          & &  & Binary CIFAR \\
%                          \cline{3-4}
%                          &  & ELMo-LSTM model & IMDb Sentiment Analysis \\
%                          \cline{3-4}
%                          & & BERT pre-trained model & IMDb Sentiment Analysis \\
%         \hline
%         \multirow{5}{*}{Multiclass} & \multirow{5}{*}{Deep nets} & \multirow{2}{*}{MLP} & MNIST \\
%                         \cline{4-4} 
%                         & & & CIFAR10 \\                   
%                         \cline{3-4}
%                          &   & \multirow{3}{*}{ResNet} & MNIST \\
%                          \cline{4-4}
%                          &   & & CIFAR10 \\
%                          \cline{4-4}
%                          &   & & CIFAR100 \\
%         \hline
%         \end{tabular}
%         % \caption{Summary of experiments performed} \label{table:experiments}
%     \end{table}    
%     % \footnotetext[6]{We use both MSE loss and cross-entropy loss.}
%     % \footnotetext[6]{We try 2 variants: one with a fixed first layer and the other with both layers trainable.}
% \end{center}

% \newpage
% \section{Proof of \lemref{lem:stability_error}} \label{app:proof_lem_error}

% \begin{proof}[Proof of \lemref{lem:stability_error}]
%     Recall, we have a training set $S \cup \wt S_C$. We defined leave-one-out error on mislabeled points as $$\error_{\text{LOO}(\wt S_M) } = \frac{\sum_{(x_i, y_i) \in \wt S_M} \error( f_{(i)}( x_i), y_i)}{ \abs{\wt S_M }} \,, $$
%     where $f_{(i)} \defeq f(\calA, (S \cup \wt S)_{(i)})$. Define $S^\prime \defeq S \cup \wt S$. Assume $(x,y)$ and $(x^\prime,y^\prime)$ as i.i.d. samples from ${\calDm}$. 
%     Using Lemma 25 in \citet{bousquet2002stability}, we have
%     \begin{align*}
%         \Expo{ \left( \error_{\calDm}(\wh f) -\error_{\text{LOO}(\wt S_M) } \right)^2 } \le & \Expt{ S^\prime, (x,y), (x^\prime,y^\prime) }{ \error(\wh f(x), y ) \error(\wh f(x^\prime), y^\prime )} - 2 \Expt{ S^\prime, (x,y) }{ \error(\wh f(x), y ) \error(f_{(i)}(x_i), y_i )} \\
%         & + \frac{m_1-1}{m_1}\Expt{ S^\prime }{  \error(f_{(i)}(x_i), y_i )  \error(f_{(j)}(x_j), y_j )} + \frac{1}{m_1} \Expt{ S^\prime }{  \error(f_{(i)}(x_i), y_i ) } \,. \numberthis \label{eq:main_reln}
%     \end{align*}
%     We can rewrite the equation above as : 
%     \begin{align*}
%         \Expo{ \left( \error_{\calDm}(\wh f) -\error_{\text{LOO}(\wt S_M) } \right)^2 } \le &  \, \underbrace{\Expt{ S^\prime, (x,y), (x^\prime,y^\prime) }{ \error(\wh f(x), y ) \error(\wh f(x^\prime), y^\prime ) - \error(\wh f(x), y ) \error(f_{(i)}(x_i), y_i )}}_{\RN{1}} \\
%         & + \underbrace{\Expt{ S^\prime }{  \error(f_{(i)}(x_i), y_i )  \error(f_{(j)}(x_j), y_j ) -  \error(\wh f(x), y ) \error(f_{(i)}(x_i), y_i )}}_{\RN{2}} \\ &+ \underbrace{\frac{1}{m_1} \Expt{ S^\prime }{  \error(f_{(i)}(x_i), y_i ) - \error(f_{(i)}(x_i), y_i )  \error(f_{(j)}(x_j), y_j ) }}_{\RN{3}} \,. \numberthis \label{eq:main_reln2}
%     \end{align*}
    
%     We will now bound term $\RN{3}$.  Using Schwarz's inequality, we have
    
%     \begin{align}
%         \Expt{ S^\prime }{  \error(f_{(i)}(x_i), y_i ) - \error(f_{(i)}(x_i), y_i )  \error(f_{(j)}(x_j), y_j ) }^2 &\le  \Expt{ S^\prime }{  \error(f_{(i)}(x_i), y_i ) }^2 \Expt{S^\prime}{1 -   \error(f_{(j)}(x_j), y_j ) }^2 \\
%         &\le \frac{1}{4} \label{eq:term1_lem12}
%     \end{align}
    
%     Note that since $(x_i,y_i)$, $(x_j ,y_j )$, $(x,y)$, and $(x^\prime, y^\prime)$ are all from same distribution $\calDm$, we directly incorporate the bounds on term $\RN{1}$ and $\RN{2}$ from proof of Lemma 9 in \citet{bousquet2002stability}. Combining that with \eqref{eq:term1_lem12} and our definition of hypothesis stability in \codref{cond:hypothesis_stability}, we have the required claim. 
    
    
%     % We now re-write term $\RN{1}$ as
%     % \begin{align*}
%     %         &\Expt{S^\prime, (x,y), (x^\prime,y^\prime) }{ \error(\wh f(x), y ) \error(\wh f(x^\prime), y^\prime ) - \error(\wh f(x), y ) \error(f_{(i)}(x_i), y_i )} \\ & \qquad = \Expt{ S^\prime, (x,y), (x^\prime,y^\prime) }{ \error(\wh f(x), y ) \error(\wh f  (x^\prime), y^\prime ) - \error(\wh f ^\prime(x), y ) \error(f_{(i)}(x^\prime), y^\prime )} \tag{Exchanging $(x_i, y_i)$ with $(x^\prime, y^\prime)$ in the second term} \\
%     %         & \qquad = \Expt{ S^\prime, (x,y), (x^\prime,y^\prime) }{  \left(\error(\wh f(x), y )-  \error(f_{(i)}(x), y ) \right) \error(\wh f  (x^\prime), y^\prime )  } \\
%     %         & \qquad  + \Expt{ S^\prime, (x,y), (x^\prime,y^\prime) }{  \left(\error(f_{(i)}(x), y ) -\error(\wh f ^\prime(x), y ) \right) \error(\wh f  (x^\prime), y^\prime )}  \\
%     %         & \qquad +\Expt{ S^\prime, (x,y), (x^\prime,y^\prime) }{  \left( \error(\wh f  (x^\prime), y^\prime ) -  \error(f_{(i)}(x^\prime), y^\prime ) \right) \error(\wh f ^\prime(x), y ) }  \,, \numberthis \label{eq:term1_final}
%     % \end{align*}
%     % where $\wh f^\prime$ is the classifier obtained by training on $ S^\prime_{(i)} \cup \{ (x^\prime, y^\prime) \} $. Similarly we can re-write term $\RN{2}$ as 
%     % \begin{align*}
%     %     & \Expt{ S^\prime }{  \error(f_{(i)}(x_i), y_i )  \error(f_{(j)}(x_j), y_j ) -  \error(\wh f(x), y ) \error(f_{(i)}(x_i), y_i )} \\
%     %     &\quad  = \Expt{ S^\prime, (x,y), (x^\prime,y^\prime)}{  \error(f^{\prime\prime}_{(i)}(x), y )  \error(f_{(j)}^{\prime}(x^\prime), y^\prime ) -  \error(\wh f(x), y ) \error(f_{(i)}(x_i), y_i )} \tag{Exchanging $(x_i, y_i)$ with $(x, y)$ and $(x_j, y_j)$ with $(x^\prime, y^\prime)$ in the first term}\\
%     %     &\quad = \Expt{ S^\prime, (x,y), (x^\prime,y^\prime)}{  \error(f^{\prime\prime}_{(j)}(x), y )  \error(f_{(i)}^{\prime}(x^\prime), y^\prime ) -  \error(\wh f^\prime (x), y ) \error(f^\prime_{(j)}(x^\prime), y^\prime )} \tag{Exchanging $(x_i, y_i)$ and $(x_j, y_j)$ and then replacing $(x_j, y_j)$ with $(x^\prime, y^\prime)$ in the second term} \\
%     %     & \quad = \Expt{ S^\prime, (x,y), (x^\prime,y^\prime) }{  \left( \error(f_{(i)}^{\prime}(x^\prime), y^\prime )   -  \error(\wh f^{\prime\prime}  (x^\prime), y^\prime ) \right)  \error(f^{\prime\prime}_{(j)}(x), y )   } \\
%     %     & \quad  + \Expt{ S^\prime, (x,y), (x^\prime,y^\prime) }{  \left( \error(f^{\prime\prime}_{(j)}(x), y )  -\error(\wh f ^\prime(x), y ) \right) \error(\wh f^{\prime\prime}  (x^\prime), y^\prime )  }  \\
%     %     & \quad+ \Expt{ S^\prime, (x,y), (x^\prime,y^\prime) }{  \left( \error(\wh f^{\prime\prime}  (x^\prime), y^\prime )  -  \error(f^\prime_{(j)}(x^\prime), y^\prime ) \right)  \error(\wh f^\prime (x), y ) }   \\
%     %     & \quad = \Expt{ S^\prime, (x,y), (x^\prime,y^\prime) }{  \left( \error(f_{(i)}^{\prime}(x^\prime), y^\prime )   -  \error(\wh f (x^\prime), y^\prime ) \right)  \error(f_{(i)}(x_j), y_j )   } \\
%     %     & \quad  + \Expt{ S^\prime, (x,y), (x^\prime,y^\prime) }{  \left( \error(f^{\prime\prime}_{(j)}(x), y )  -\error(\wh f (x), y ) \right) \error(\wh f^{\prime\prime}  (x_j), y_j )  }  \\
%     %     & \quad+ \Expt{ S^\prime, (x,y), (x^\prime,y^\prime) }{  \left( \error(\wh f^{\prime\prime}  (x^\prime), y^\prime )  -  \error(f^\prime_{(j)}(x^\prime), y^\prime ) \right)  \error(\wh f^\prime (x^\prime), y^\prime ) }  \,, \numberthis \label{eq:term2_final}
%     % \end{align*}
%     % where $f^{\prime\prime}_{(j)}$ is trained on $S^\prime_{(j,i)} \cup {(x,y)}$, $f^{\prime}_{(i)}$ is trained on $S^\prime_{(j,i)} \cup {(x^\prime,y^\prime)}$, and $\wh f^{\prime\prime} $ is trained on $S^\prime_{(j)} \cup {(x,y)}$. Note in the last line we replaced $(x,y)$ by $(x_j, y_j)$ in the first term, replaced $(x^\prime,y^\prime)$ by $(x_j, y_j)$ in the second term and exchanged $(x_i,y_i)$ with $(x_j,y_j)$ and also $(x,y)$ and $(x^\prime, y^\prime)$
    
    
% \end{proof}

\end{document}

