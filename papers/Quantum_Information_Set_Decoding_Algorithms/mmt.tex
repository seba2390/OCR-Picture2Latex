\section{Improvements obtained by the representation technique and ``$1+1=0$''}
There are two techniques that can be used to
speed up the quantum algorithm of the previous section.


\par{\em The representation technique.} It was introduced in \cite{HJ10} to speed up algorithms for the subset-sum algorithm and
used later on in \cite{MMT11} to improve decoding algorithms. The basic idea of the representation technique in the context of the 
subset-sum or decoding algorithms consists in (i) changing slightly the underlying (generalised) $k$-sum problem which is solved by 
introducing sets $\zV_i$ for which there are (exponentially) many solutions to the problem $\sum_i f(v_i) = S$ by using redundant representations,
 (ii) noticing that this allows us to put additional subset-sum conditions on the solution.

In the decoding context, instead of considering sets of errors with non-overlapping support, the idea that allows us to obtain many 
different representations of a same solution is just to consider sets $\zV_i$ corresponding to errors with overlapping supports. In our case,
we could have taken instead of the four sets defined in the previous section the following sets
\begin{eqnarray*}
\zV_{00}=\zV_{10} &\eqdef &\{(e_{00},0_{(k+\ell)/2})\in \Ft^{k+\ell} : e_{00 }\in \Ft^{(k+\ell)/2},\; |e_{00}|=p/4\} \\
\zV_{01}=\zV_{11} &\eqdef &\{(0_{(k+\ell)/2},e_{01})\in \Ft^{k+\ell} : e_{01 }\in \Ft^{(k+\ell)/2},\; |e_{01}|=p/4\}
\end{eqnarray*}
Clearly a vector $e$ of weight $p$ can be written in many different ways as a sum
$v_{00}+v_{01}+v_{10}+v_{11} $ where $v_{ij}$ belongs to $\zV_{ij}$. This is (essentially) due to the fact that a vector of weight $p$ can be written in $\binom{p}{p/2} = \OOt{2^p}$ 
ways as a sum of two vectors  of weight $p/2$.

The point is that if we apply now the same algorithm as in the previous section and look for solutions to Problem \ref{pb:SS}, there is not a single value of $r$
that leads to the right solution. Here, about $2^p$ values of $r$ will do the same job. The speedup obtained by the representation technique 
is a consequence of this phenomenon. We can even improve on this representation technique by using the $1+1=0$ phenomenon as in \cite{BJMM12}.


\par{\em The ``$1+1=0$'' phenomenon.} Instead of choosing the $\zV_i$'s as explained above we will actually choose the $\zV_i$'s as
\begin{eqnarray}
\zV_{00}=\zV_{10} &\eqdef &\{(e_{00},0_{(k+\ell)/2})\in \Ft^{k+\ell} : e_{00 }\in \Ft^{(k+\ell)/2},\; |e_{00}|=\frac{p}{4} +\frac{\Delta p}{2}\} \label{eq:v00v10}\\
\zV_{01}=\zV_{11} &\eqdef &\{(0_{(k+\ell)/2},e_{01})\in \Ft^{k+\ell} : e_{01 }\in \Ft^{(k+\ell)/2},\; |e_{01}|=\frac{p}{4} +\frac{\Delta p}{2}\} \label{eq:v01v11}
\end{eqnarray}
A vector $e$ of weight $p$ in $\Ft^{k+\ell}$ can indeed by represented in many ways as a sum of $2$ vectors of weight $\frac{p}{2} +\Delta p$. More precisely, such a vector can be 
represented in 
$
\binom{p}{p/2} \binom{k+\ell-p}{\Delta p} 
$
ways. Notice that this number of representations is greater than the number $2^p$ that we had before. This explains why choosing an appropriate
positive value $\Delta p$ allows us to improve on the previous choice.

 

The quantum algorithm for decoding follows the same pattern as in the previous section:
(i) we look with Grover's search algorithm for a right set $\zS$ of $k + \ell$ positions 
such that the restriction $e'$ of the error $e$ we look for is of weight $p$ on this subset
and then (ii) we search for $e'$ by solving a generalised $4$-sum problem  with a combination
of Grover's algorithm and a quantum walk. We will use for the second point the following proposition
which quantifies how much we gain when there are multiple representations/solutions:

\begin{proposition}\label{prop:improvement}
Consider the generalised $4$-sum problem  with sets $\zV_u$ of size  $\OO{V}$. Assume that 
$\zG$ can be decomposed as $\zG =\zG_0 \times \zG_1 \times \zG_2$. Furthermore assume that 
there are $\Om{|\zG_2|}$ solutions to the $4$-sum problem and that we can fix arbitrarily 
the value $\pi_2\left(f(v_{00})+f(v_{01})\right)$ of a solution to the $4$-sum problem, 
where $\pi_2$ is the mapping from $\zG =\zG_0 \times \zG_1 \times \zG_2$ to $\zG_2$ which maps 
$(g_0,g_1,g_2)$ to $g_2$.
There is a quantum algorithm for solving the $4$-sum problem running in time $\OOt{|\zG_1|^{1/2}V^{4/5}}$
as soon as $|\zG_1|\cdot|\zG_2| = \Om{V^{4/5}}$ and $|\zG| = \Om{V^{8/5}}$.
\end{proposition}

\begin{proof}
Let us first introduce a few notations.
We denote by $\pi_{12}$ the ``projection'' from $\zG=\zG_0 \times \zG_1 \times \zG_2$ to 
$\zG_1 \times \zG_2$ which associates to $(g_0,g_1,g_2)$ the pair $(g_1,g_2)$
and by $\pi_0$ the projection from $\zG$ to $\zG_0$ which maps 
$(g_0,g_1,g_2)$ to $g_0$.
As in the previous section, we solve with a  quantum walk the following problem:
we fix an element $r=(r_1,r_2)$ in $\zG_1 \times \zG_2$ and
find (if it exists) $(v_{00},v_{01},v_{10},v_{11})$ in $\zV_{00}\times \zV_{01} \times \zV_{10} \times \zV_{11}$ such that 
\begin{eqnarray*}
\pi_{12}(f(v_{00})) +\pi_{12}(f(v_{01})) & = & r \\
\pi_{12}(f(v_{10}))  +\pi_{12}(f(v_{11})) & = & \pi_{12}(S)-r \\
\pi_0(f(v_{00})) + \pi_0(f(v_{01})) + \pi_0(f(v_{10})) + \pi_0(f(v_{11})) & = &\pi_0(S)\\
g(v_{00},v_{01},v_{10},v_{11}) & = & 0.
\end{eqnarray*}
The difference with Proposition \ref{prop:easy} is that we do not check all possibilities for $r$ but just all possibilities for $r_1 \in \zG_1$ and
fix $r_2$ arbitrarily. As in Proposition \ref{prop:easy}, we perform a quantum walk whose complexity is $\OOt{V^{4/5}}$ to 
solve the aforementioned problem for a fixed $r$. 
What remains to be done is to find the right value for $r_1$ which is achieved by a Grover 
search with complexity $\OO{|\zG_1|^{1/2}}$.$~\qed$
\end{proof}
\begin{figure}[h]
    \centering
    \includegraphics[width=0.5\textwidth]{mmt.png}
    \caption{The representation technique: 
the support of the elements of $\zV_{ij}$ is represented in orange, while
the blue, green and violet colours represent $\zG_0$ resp. $\zG_1$, resp. $\zG_2$.}
    \label{fig:mmt}
\end{figure}
By  applying Proposition \ref{prop:improvement} in our decoding context, we obtain
\begin{restatable}{theorem}{thmExpMMTQW}
\label{th:expMMTQWdiese}
We can decode $w=\omega n$ errors in a random linear code of length $n$ and rate $R=\frac{k}{n}$  with a quantum complexity of 
order $ \OOt{2^{\alpha_{\text{MMTQW}}(R,\omega)n}}$ where
\begin{eqnarray*}
    \alpha_{\text{{\tiny{MMTQW}}}}(R,\omega) &\eqdef& \min_{(\pi,\Delta \pi, \lambda) \in \zR} \left( 
\frac{  \beta(R,\lambda,\pi,\Delta \pi)+ \gamma(R,\lambda,\pi,\omega)}{2}\right) \\
&\text{with} & \\
\beta(R,\lambda,\pi,\Delta \pi) & \eqdef & \frac{6}{5} (R+\lambda)H_2\left(\frac{\pi/2+\Delta \pi}{R+\lambda}\right)- \pi - (1-R-\lambda)H_2\left(\frac{\Delta \pi }{1-R-\lambda}\right),\\
\gamma(R,\lambda,\pi,\omega) & \eqdef & H_2(\omega) - (1-R-\lambda)H_2(\frac{\omega - \pi}{1-R-\lambda}) - (R+\lambda)H_2\left(\frac{\pi}{R+\lambda}\right)
\end{eqnarray*}
where $\zR$ is the subset of elements $(\pi,\Delta \pi, \lambda)$ of 
$[0,\omega]  \times [0,1) \times [0,1) $
 that satisfy the following constraints
\begin{eqnarray*}
0 \leq  &\Delta \pi  &\leq R + \lambda - \pi \\
0 \leq &\pi &\leq \min(\omega, R + \lambda)\\
0 \leq &\lambda &\leq 1 - R - \omega + \pi \\
&\pi &= 2\left((R+\lambda)H_2^{-1}\left(\frac{5\lambda}{4(R+\lambda)}\right) - \Delta \pi\right)
\end{eqnarray*}
\end{restatable}
\begin{proof}
The algorithm picks random subsets $\zS$ of size $k+\ell$ with the hope that the restriction to $\zS$ of the error
of weight $w$ that we are looking for is of weight $p$. 
Then it solves for each of these subsets the generalised
$4$-sum problem where the sets $\zV_{ij}$ are specified in \eqref{eq:v00v10} and \eqref{eq:v01v11}, 
and $\zG$, $\zE$, $f$ and $g$ are as  in Problem \ref{pb:shamir_shroeppel}.
$g$ is in this case slightly more complicated for the sake of analysing the algorithm.
We have 
$g(v_{00},v_{01},v_{10},v_{11})=0$ 
 if and only if (i) $v_{00}+v_{01}+v_{10}+v_{11}$ is of weight $p$ 
(this is the additional constraint we use for the analysis of the algorithm)
(ii) $f(v_{00})+f(v_{01})+f(v_{10})+f(v_{11})=\Sigma(s,H,\zS)$ and (iii) 
$h(v_{00}+v_{01}+v_{10}+v_{11})$ is of weight $w$.


From \eqref{eq:T_quant_ISD} we  know that the quantum complexity is given by
\begin{equation}
\label{eq:complexityMMTQW}
\tilde O\left(\frac{T_{\text{MMTQW}}}{\sqrt{P_{\text{MMTQW}}}}\right)
\end{equation}
where $T_{\text{MMTQW}}$ is the complexity of the combination of Grover's algorithm and quantum walk
solving the generalised $4$-sum problem specified above and $P_{\text{MMTQW}}$ is the probability
that the restriction $e'$ of the error $e$ to $\zS$ is of weight $p$
and that this error can be written as $e' = v_{00}+v_{01}+v_{10} + v_{11}$ where the $v_{ij}$ belong to 
$\zV_{ij}$.
It is readily verified that 
$$
P_{\text{MMTQW}} = \OOt{\frac{ \binom{k+\ell}{p} \binom{n-k-\ell}{w - p}}{\binom{n}{w}}}
$$
By using asymptotic expansions of the binomial coefficients we obtain
\begin{equation}
\label{eq:PMMTQW}
(P_{\text{MMTQW}})^{-1/2} = \OOt{2^{\frac{H_2(\omega) - (1-R-\lambda)H_2\left(\frac{\omega - \pi}{1 - R - \lambda}\right) - (R+\lambda)H_2\left(\frac{\pi}{R+\lambda}\right) }{2} n}}
\end{equation}
where $\lambda \eqdef \frac{\ell}{n}$ and $\pi \eqdef \frac{p}{n}$. To estimate $T_{\text{SSQW}}$, we can use Proposition \ref{prop:improvement}.
The point is that the number of different solutions of the generalised $4$-sum problem 
(when there is one) is of order 
$$
\Omt{\binom{p}{p/2} \binom{k+\ell-p}{\Delta p}}.
$$
At this point, we observe that
$$
\log_2\left(\binom{p}{p/2} \binom{k+\ell-p}{\Delta p}\right) = p+ (k+\ell-p)H_2\left( \frac{\Delta p}{k+\ell -p} \right) +o(n)
$$
when $p$, $\Delta p$, $\ell$, $k$ are all linear in $n$.
In other words, we may use Proposition $\ref{prop:improvement}$ with 
$\zG_2 = \Ft^{\ell_2}$ with 
\begin{equation}
\label{eq:ell2}
\ell_2 \eqdef p+ (k+\ell-p)H_2\left( \frac{\Delta p}{k+\ell -p} \right).
\end{equation}
We use now Proposition \ref{prop:improvement}
with $\zG_2$ chosen as explained above. 
$V$ is given in this case by
$$
V = \binom{\frac{k+\ell}{2}}{\frac{p}{4}+\frac{\Delta p}{2}} = \OOt{2^{\frac{(R+\lambda) H_2\left(\frac{\pi/2 + \Delta \pi}{R + \lambda}\right)n}{2}}}
$$
where $\Delta \pi \eqdef \frac{\Delta p}{n}$.
We choose the size of $\zG$ such that
\begin{equation}\label{eq:G}
|\zG| = \Tht{V^{8/5}}
\end{equation}
which gives
$$
2^\ell = \Tht{  {\binom{\frac{k+\ell}{2}}{\frac{p}{4}+\frac{\Delta p}{2}}}^{8/5}}.
$$
This explains why we impose 
$$
\lambda = \frac{8}{5}\frac{R+\lambda}{2} H_2\left(\frac{\pi/2+\Delta \pi}{R+\lambda}\right)
$$
which is equivalent to the condition 
$$
\frac{5 \lambda}{4(R+\lambda)} = H_2\left(\frac{\pi/2+\Delta \pi}{R+\lambda}\right)
$$
which in turn is equivalent to the condition
\begin{equation}\label{eq:condition2MMT}
\pi = 2\left((R+\lambda)H_2^{-1}\left(\frac{5\lambda}{4(R+\lambda)}\right) - \Delta \pi\right)
\end{equation}
found in the definition of the region $\zR$.
The size of $\zG_1$ is chosen such that
\begin{equation}\label{eq:condition1MMT}
|\zG_1| \cdot |\zG_2| = \Ft^{\lceil \frac{\ell}{2} \rceil}.
\end{equation}
By using \eqref{eq:ell2} and \eqref{eq:G}, this implies
\begin{eqnarray}
|\zG_1| & = & \Tht{\frac{V^{4/5}}{2^{p+ (k+\ell-p)H_2\left( \frac{\Delta p}{k+\ell -p} \right)}}}
\end{eqnarray}

With the choices \eqref{eq:condition1MMT} and \eqref{eq:condition2MMT}, 
we obtain
\begin{eqnarray}
T_{\text{MMTQW}} &= &\OOt{|\zG_1|^{1/2} \cdot V^{4/5}} \nonumber \\
& = & \OOt{\frac{V^{6/5}}{2^{\frac{p}{2}+ \frac{k+\ell-p}{2} H_2\left( \frac{\Delta p}{k+\ell -p} \right)}}} \nonumber \\
& = & \OOt{ 2^{\left[\frac{3}{5}(R+\lambda)H_2\left( \frac{\pi/2 +\Delta \pi}{R+\lambda} \right)- \frac{\pi}{2} - \frac{R+\lambda-\pi}{2} H_2\left( \frac{\Delta \pi}{R+\lambda -
\pi} \right) \right]n}}\label{eq:TMMTQW}
\end{eqnarray}
Substituting for $P_{\text{MMTQW}}$ and $T_{\text{MMTQW}}$ the expressions given by \eqref{eq:PMMTQW} and \eqref{eq:TMMTQW} finishes the proof of the
theorem.$~\qed$
\end{proof}
