 \documentclass[runningheads]{llncs}
 
\usepackage{a4wide}
\usepackage[latin1]{inputenc}
\usepackage{graphicx}
\usepackage{amsfonts}
\usepackage{amssymb}
\usepackage{amsmath}
\usepackage[linesnumbered,boxed,ruled,vlined]{algorithm2e}
\usepackage{framed}
\usepackage{latexsym}
\usepackage{color}
\usepackage{calrsfs}
\usepackage[mathscr]{eucal}
\usepackage{url}
\usepackage{thm-restate}

%%%%%%%%%% sets %%%%%%%%%%%%%%%%%%%%%%%%%
\def\sA{{\mathscr A}}
\def\sB{{\mathscr B}}
\def\sC{{\mathscr C}}
\def\sE{{\mathscr E}}
\def\sG{{\mathscr G}}
\def\sH{{\mathscr H}}
\def\sI{{\mathscr I}}
\def\sJ{{\mathscr J}}
\def\sK{{\mathscr K}}
\def\sL{{\mathscr L}}
\def\sM{{\mathscr M}}
\def\sP{{\mathscr P}}
\def\sQ{{\mathscr Q}}
\def\sR{{\mathscr R}}
\def\sS{{\mathscr S}}
\def\sT{{\mathscr T}}
\def\sU{{\mathscr U}}
\def\sV{{\mathscr V}}
\def\sX{{\mathscr X}}
\def\sY{{\mathscr Y}}
\def\sZ{{\mathscr Z}}

\DeclareMathAlphabet{\mathpzc}{OT1}{pzc}{m}{it}
\def\zC{\mathpzc{C}}
\def\zD{\mathpzc{D}}
\def\zE{\mathpzc{E}}
\def\zG{\mathpzc{G}}
\def\zH{\mathpzc{H}}
\def\zI{\mathpzc{I}}
\def\zJ{\mathpzc{J}}
\def\zK{\mathpzc{K}}
\def\zM{\mathpzc{M}}
\def\zR{\mathpzc{R}}
\def\zS{\mathcal{S}}
\def\zT{\mathpzc{T}}
\def\zU{\mathpzc{U}}
\def\zV{\mathpzc{V}}

% \OO{ } : grand O de { }
\newcommand{\OO}[1]{O\left( #1 \right)}
\newcommand{\Th}[1]{\Theta\left( #1 \right)}
\newcommand{\Tht}[1]{\tilde{\Theta}\left( #1 \right)}
\newcommand{\OOt}[1]{\tilde{O}\left( #1 \right)}
\newcommand{\Om}[1]{\Omega \left( #1 \right)}
\newcommand{\Omt}[1]{\tilde{\Omega} \left( #1 \right)}

\newcommand{\CS}{T_s}
\newcommand{\CC}{T_c}
\newcommand{\CU}{T_u}



\newcommand{\Ft}{\mathbb{F}_2}
\newcommand{\prob}{\mathbb{P}}
\newcommand{\eqdef}{\mathop{=}\limits^{\triangle}}


%%ALGORITHMS
\SetKwRepeat{RepeatU}{repeat}{until}%
\SetKwRepeat{Repeat}{repeat}{}%
%%END ALGORITHMS


\begin{document}
\title{Quantum Information Set Decoding Algorithms}
\author{Ghazal Kachigar \inst{1}  \and Jean-Pierre Tillich \inst{2}}

\institute{Institut de Math\'ematiques de Bordeaux\\ 
Universit\'e de Bordeaux \\
Talence Cedex F-33405, France\\
\email{ghazal.kachigar@u-bordeaux.fr}
\and
Inria,   EPI SECRET\\
2 rue Simone Iff, Paris 75012, France\\
\email{jean-pierre.tillich@inria.fr}
}

\maketitle


\begin{abstract}
The security of code-based cryptosystems such as the McEliece cryptosystem relies primarily on the difficulty of decoding random linear codes. 
The best decoding algorithms are all improvements of an old algorithm due to Prange:  they are known under the name of  information set decoding techniques.
It is also important to assess the security of such cryptosystems against a quantum computer. This research thread started in \cite{OS09} and the
best algorithm to date has been Bernstein's quantising \cite{B10} of the simplest information set decoding algorithm, namely Prange's algorithm.
It consists in applying Grover's quantum search  to obtain a quadratic speed-up of Prange's algorithm.
In this paper, we quantise other information set decoding 
algorithms by using quantum walk techniques which were devised for the subset-sum problem in \cite{BJLM13}.
This results in improving the worst-case complexity of $2^{0.06035n}$ of Bernstein's algorithm to
$2^{0.05869n}$ with the best algorithm presented here (where $n$ is the codelength).
\end{abstract}

\textbf{Keywords:} code-based cryptography, quantum cryptanalysis, decoding algorithm.
% !TEX root = ../arxiv.tex

Unsupervised domain adaptation (UDA) is a variant of semi-supervised learning \cite{blum1998combining}, where the available unlabelled data comes from a different distribution than the annotated dataset \cite{Ben-DavidBCP06}.
A case in point is to exploit synthetic data, where annotation is more accessible compared to the costly labelling of real-world images \cite{RichterVRK16,RosSMVL16}.
Along with some success in addressing UDA for semantic segmentation \cite{TsaiHSS0C18,VuJBCP19,0001S20,ZouYKW18}, the developed methods are growing increasingly sophisticated and often combine style transfer networks, adversarial training or network ensembles \cite{KimB20a,LiYV19,TsaiSSC19,Yang_2020_ECCV}.
This increase in model complexity impedes reproducibility, potentially slowing further progress.

In this work, we propose a UDA framework reaching state-of-the-art segmentation accuracy (measured by the Intersection-over-Union, IoU) without incurring substantial training efforts.
Toward this goal, we adopt a simple semi-supervised approach, \emph{self-training} \cite{ChenWB11,lee2013pseudo,ZouYKW18}, used in recent works only in conjunction with adversarial training or network ensembles \cite{ChoiKK19,KimB20a,Mei_2020_ECCV,Wang_2020_ECCV,0001S20,Zheng_2020_IJCV,ZhengY20}.
By contrast, we use self-training \emph{standalone}.
Compared to previous self-training methods \cite{ChenLCCCZAS20,Li_2020_ECCV,subhani2020learning,ZouYKW18,ZouYLKW19}, our approach also sidesteps the inconvenience of multiple training rounds, as they often require expert intervention between consecutive rounds.
We train our model using co-evolving pseudo labels end-to-end without such need.

\begin{figure}[t]%
    \centering
    \def\svgwidth{\linewidth}
    \input{figures/preview/bars.pdf_tex}
    \caption{\textbf{Results preview.} Unlike much recent work that combines multiple training paradigms, such as adversarial training and style transfer, our approach retains the modest single-round training complexity of self-training, yet improves the state of the art for adapting semantic segmentation by a significant margin.}
    \label{fig:preview}
\end{figure}

Our method leverages the ubiquitous \emph{data augmentation} techniques from fully supervised learning \cite{deeplabv3plus2018,ZhaoSQWJ17}: photometric jitter, flipping and multi-scale cropping.
We enforce \emph{consistency} of the semantic maps produced by the model across these image perturbations.
The following assumption formalises the key premise:

\myparagraph{Assumption 1.}
Let $f: \mathcal{I} \rightarrow \mathcal{M}$ represent a pixelwise mapping from images $\mathcal{I}$ to semantic output $\mathcal{M}$.
Denote $\rho_{\bm{\epsilon}}: \mathcal{I} \rightarrow \mathcal{I}$ a photometric image transform and, similarly, $\tau_{\bm{\epsilon}'}: \mathcal{I} \rightarrow \mathcal{I}$ a spatial similarity transformation, where $\bm{\epsilon},\bm{\epsilon}'\sim p(\cdot)$ are control variables following some pre-defined density (\eg, $p \equiv \mathcal{N}(0, 1)$).
Then, for any image $I \in \mathcal{I}$, $f$ is \emph{invariant} under $\rho_{\bm{\epsilon}}$ and \emph{equivariant} under $\tau_{\bm{\epsilon}'}$, \ie~$f(\rho_{\bm{\epsilon}}(I)) = f(I)$ and $f(\tau_{\bm{\epsilon}'}(I)) = \tau_{\bm{\epsilon}'}(f(I))$.

\smallskip
\noindent Next, we introduce a training framework using a \emph{momentum network} -- a slowly advancing copy of the original model.
The momentum network provides stable, yet recent targets for model updates, as opposed to the fixed supervision in model distillation \cite{Chen0G18,Zheng_2020_IJCV,ZhengY20}.
We also re-visit the problem of long-tail recognition in the context of generating pseudo labels for self-supervision.
In particular, we maintain an \emph{exponentially moving class prior} used to discount the confidence thresholds for those classes with few samples and increase their relative contribution to the training loss.
Our framework is simple to train, adds moderate computational overhead compared to a fully supervised setup, yet sets a new state of the art on established benchmarks (\cf \cref{fig:preview}).

\section{Quantum search algorithms}
\subsection{Grover search}
Grover's search algorithm \cite{G96a,G97} is, along with its generalisation \cite{BBHT98} which is used in this paper, an optimal algorithm for solving the following problem with a quadratic speed-up compared to the best-possible classical algorithm.
\begin{problem}[Unstructured search problem]
Given a set $\zE$ and a function $f : \zE \rightarrow \{0,1\}$, find an $x \in \zE$ such that $f(x) = 1$.
\end{problem}
In other words, we need to find an element that fulfils a certain property, and $f$ is an oracle for deciding whether it does. Moreover, in the new results presented in this paper, $f$ will be a quantum algorithm.
If we denote by $\varepsilon$ the proportion of elements $x$ of $\zE$ such that $f(x) = 1$, Grover's algorithm solves the problem above using $O(\frac{1}{\sqrt{\varepsilon}})$ queries to $f$, whereas in the classical setting this cannot be done with less than $O(\frac{1}{\varepsilon})$ queries.
Furthermore, if the algorithm $f$ executes in time $T_f$ on average, the average time complexity of Grover's algorithm will be $O(\frac{T_f}{\sqrt{\varepsilon}})$.
\subsection{Quantum Walk}
\subsubsection{Random Walk.}
Unstructured search problems as well as search problems with slightly more but still minimal structure may be recast as graph search problems.
\begin{problem}[Graph search problem]
Given a graph $G=(\zV,\zE)$ and a set of vertices $\zM \subset \zV$, called the set of \textit{marked elements}, find an $x \in \zM$.
\end{problem}
The graph search problem may then be solved using random walks (discrete-time Markov chains) on the vertices of the graph.
From now on, we will take the graph to be undirected, connected, and $d$-regular, i.e. such that each vertex has exactly $d$ neighbours.
\par{\em Markov chain.}
A Markov chain is given by an initial probability distribution $v$ and a stochastic transition matrix $M$. The transition matrix of a random walk on a graph (as specified above) is obtained from the graph's adjacency matrix $A$ by
$M=\frac{1}{d}A$.
\par{\em Eigenvalues and the spectral gap.}
A closer look at the eigenvalues and the eigenvectors of $M$ is needed in order to analyse the complexity of a random walk on a graph.
The eigenvalues will be noted $\lambda_i$ and the corresponding eigenvectors $v_i$. We will admit the following points (see \cite{CDS80}):\\
(i) all the eigenvalues lie in the interval $[-1,1]$;\\
 (ii) $1$ is always an eigenvalue, the corresponding eigenspace is of dimension $1$;\\
(iii) there is a corresponding eigenvector which is also a probability distribution (namely the uniform distribution $u$ over the vertices). 
It is the unique stationary distribution of the random walk.\\
  We will suppose that the eigenvalues are ordered from largest to smallest, so that $\lambda_1=1$ and $v_1=u$.
An important value associated with the transition matrix of a Markov chain is its \textit{spectral gap}, defined as 
$\delta \eqdef  1 - \max_{i=2,...,d}|\lambda_i|$.
Such a random walk on an undirected regular graph is always {\em reversible} and it is also {\em irreducible} because we have assumed that the graph is 
connected. The random walk is {\em aperiodic} in such a case if and only if the spectral gap $\delta$ is positive. In such a case,  a long enough random walk in the graph converges to the uniform distribution since for all $\eta>0$, we have $||M^kv - u|| < \eta$ for $k=\tilde O(1/\delta)$, where $v$ is the initial probability distribution.
 

Finding a marked element by running a Markov chain on the graph just consists in 

 \begin{algorithm}[H]
  \DontPrintSemicolon
  \KwIn{$G = (\zE,\zV)$, $\zM \subset \zV$, initial probability distribution $v$}
  \KwOut{An element $e \in \zM$}
   \textsc{Setup :} Sample a vertex $x$ according to $v$ and initialise the data structure.\;
   \Repeat{
     \textsc{Check :} \eIf{current vertex $x$ is marked}{
     \KwRet $x$\; 
   }{
     \RepeatU{$x$ is sampled according to a distribution close enough to the uniform distribution}{
       \textsc{Update :} \emph{Take one step of the random walk and update data structure accordingly.}\;
       }
     }
   }

  \caption{$RandomWalk$}
 \end{algorithm}

Let $\CS$ be the cost of \textsc{Setup}, $\CC$ be the cost of \textsc{Check} and $\CU$ be the cost of \textsc{Update}. 
It follows from the preceding considerations that $\tilde O(1/\delta)$ steps of the random walk are sufficient to sample $x$ according to the uniform distribution.
Furthermore, if we note $\varepsilon := \frac{|\zM|}{|\zV|}$ the proportion of marked elements, it is readily seen that the algorithm ends after $O(1/\varepsilon)$ iterations of the outer loop.
Thus the complexity of classical random walk is $\CS + \frac{1}{\varepsilon}\left(\CC + \frac{1}{\delta}\CU\right)$.

Several quantum versions of random walk algorithms have been proposed by many authors, notably Ambainis \cite{A07}, Szegedy \cite{S04}, and Magniez, Nayak, Roland and Santha \cite{MNRS07}. A survey of these results can be found in \cite{S08}. We use here the following result
\begin{theorem}[\cite{MNRS07}]
\label{th:quantumwalk}
Let $M$ be an aperiodic, irreducible and reversible Markov chain on a graph with spectral gap $\delta$, and $\varepsilon := \frac{|\zM|}{|\zV|}$ as above. Then there is a quantum walk algorithm that finds an element in $\zM$ with cost
\begin{equation}\label{eq:complexity_quantum_walk}
\boxed{\CS + \frac{1}{\sqrt\varepsilon}\left(\CC + \frac{1}{\sqrt\delta}\CU\right)}
\end{equation}
\end{theorem}

\subsubsection{Johnson graphs and product graphs.}
With the exception of Grover's search algorithm seen as a quantum walk algorithm, to date an overwhelming majority of quantum walk algorithms are based on Johnson graphs or a variant thereof. The decoding algorithms which shall be presented in this paper rely on cartesian products of Johnson graphs. All of these objects are defined in this section and some important properties are mentioned.
\begin{definition}[Johnson graphs]
\label{def_johnson_graph}
A Johnson graph $J(n,r)$ is an undirected graph whose vertices are the subsets containing $r$ elements of a set of size $n$, with an edge between two vertices $S$ and $S'$ iff $|S \cap S'| = r-1$. In other words, $S$ is adjacent to $S'$ if $S'$ can be obtained from $S$ by removing an element and adding a new element in its place.
\end{definition}
It is clear that $J(n,r)$ has $\binom{n}{r}$ vertices and is $r(n-r)$-regular. Its spectral gap is given by
\begin{equation}
\label{eq:spectral_gap_johnson}
\delta = \frac{n}{r(n-r)}.
\end{equation}
\begin{definition}[Cartesian product of graphs]
\label{def_product_graphs}
Let $G_1 = (\zV_1, \zE_1)$ and $G_2 = (\zV_2,\zE_2)$ be two graphs. Their cartesian product $G_1 \times G_2$ is the graph $G = (\zV,\zE)$ where:
\begin{enumerate}
  \item $\zV = \zV_1 \times \zV_2$, i.e. $\zV = \{v_1v_2~|~v_1 \in \zV_1, v_2 \in \zV_2\}$
  \item $\zE = \{(u_1u_2,v_1v_2)~|~ (u_1 = v_1 \wedge (u_2,v_2) \in \zE_2) \vee ((u_1,v_1) \in \zE_1 \wedge u_2 = v_2)\}$
\end{enumerate}
\end{definition}
The spectral gap of products of Johnson graphs is given by
\begin{restatable}[Cartesian product of Johnson graphs]{theorem}{thmProdJohnsonGraph}
\label{thm:product_johnsongraphs}
Let $J(n,r) = (\zV,\zE)$, $m \in \mathbb{N}$ and $J^m(n,r) := \times_{i=1}^m J(n,r) = (\zV_m,\zE_m)$. Then:
\begin{enumerate}
  \item $J^m(n,r)$ has $\binom{n}{r}^m$ vertices and is $md$-regular where $d = r(n-r)$.
  \item We will write $\delta(J)$ resp. $\delta(J^m)$ for the spectral gaps of $J(n,r)$ resp. $J^m(n,r)$. Then:\\
$\delta(J^m) \geq \frac{1}{m}\delta(J).$
 \item The random walk associated with $J^m(n,r)$ is  aperiodic, irreducible and reversible for all positive $m$, $n$ and $r<n$.
\end{enumerate}
\end{restatable}
For a proof of this statement, see the appendix.

\section{Generalities on classical and quantum decoding}
\label{sec:classical_quantum_decoding}
We first recall how the simplest ISD algorithm \cite{P62} and its quantised version \cite{B10} work and 
then give a skeleton of the structure of more sophisticated classical and quantum versions.
\subsection{Prange's algorithm and Bernstein's algorithm}
Recall that the goal is to find $e$ of weight $w$ given $s^T = He^T$, where $H$ is an $(n-k) \times n$ matrix. In other words, the problem we aim to solve is finding a solution to an underdetermined linear system of $n-k$ equations in $n$ variables and the  solution is unique owing to the weight condition. Prange's algorithm is based on the following observation: if it is known that $k$ given components of the error vector are zero, the error positions are among the $n-k$ remaining components. In other words, if we know for sure that the $k$ corresponding variables are not involved in the linear system, then the error vector can be found by solving the resulting linear system of $n-k$ equations in $n-k$ variables in polynomial time.

The hard part is finding a correct size-$k$ set (of indices of the components). Prange's algorithm 
samples such sets and solves the resulting linear equation until an error vector of weight $w$ is found.
The probability for finding such a set is of order 
$\Om{\frac{\binom{n-k}{w}}{\binom{n}{w}}}$ and therefore
Prange's algorithm has  
complexity 
$$\OO{ \frac{\binom{n}{w}}{\binom{n-k}{w}}}=\OOt{2^{\alpha_{\text{Prange}}(R,\omega)n}}$$
where
$$\alpha_{\text{Prange}}(R,\omega) = H_2(\omega) - (1-R)H_2\left(\frac{\omega}{1-R}\right)$$
by using the well known formula for binomials
$$
\binom{n}{w} = \Tht{2^{H_2\left( \frac{w}{n} \right)n}}.
$$
Bernstein's algorithm consists in using Grover's algorithm to find a correct size-$k$ set. Indeed, an oracle for checking that a size-$k$ set is correct can be obtained by following the same steps as in Prange's algorithm, i.e. deriving and solving a linear system of $n-k$ equations in $n-k$ variables and returning 1 iff the resulting error vector has weight $w$.
Thus the complexity of Bernstein's algorithm is the square root of that of Prange's algorithm, i.e. $\alpha_{\text{Bernstein}} = \frac{\alpha_{\text{Prange}}}{2}$.

\subsection{Generalised ISD algorithms}
More sophisticated classical ISD algorithms \cite{S88,D91,FS09,BLP11,MMT11,BJMM12,MO15} generalise Prange's algorithm in the following way: they introduce a new parameter $p$ and allow $p$ error positions 
inside of the size-$k$ set (henceforth denoted by $\zS$). Furthermore, from Dumer's algorithm onwards, a new parameter $\ell$ is introduced and the set $\zS$ is taken to be of size $k+\ell$.
This event happens with probability $P_{\ell,p} \eqdef \frac{\binom{k+\ell}{p}\binom{n-k-\ell}{w-p}}{\binom{n}{w}}.$
The point is that
\begin{restatable}{proposition}{propPuncturing}
\label{prop:puncturing}
Assume that the restriction of $H$ to the columns belonging to the complement of $\zS$ is a matrix of full rank, then
\begin{itemize}
\item[(i)]
the restriction $e'$ of the error to $\zS$  is a solution to the syndrome decoding problem
 \begin{equation}\label{eq:subproblem}
H' {e'}^T = {s'}^T.
\end{equation} 
with $H'$ being an $\ell \times (k+\ell)$ binary matrix, $|e'|=p$ and $H'$, $s'$ that can be computed in polynomial time from $\zS$, $H$ and $s$;\\
\item[(ii)] once we have such an $e'$,  there is a unique $e$ whose restriction to $\zS$ is equal to $e'$ and which satisfies
$He^T = s^T$. Such an $e$ can be computed from $e'$ in polynomial time.
\end{itemize}
\end{restatable}

\noindent
{\em Remark:} The condition in this proposition is met with very large probability when $H$ is chosen uniformly at random:
it fails to hold with probability which is only $O(2^{-\ell})$.

\begin{proof}
Without loss 
of generality assume that $\zS$ is given by the $k+\ell$ first positions.  By performing Gaussian elimination, we look for a square matrix $U$ such that 
$$
U H = \begin{pmatrix} H' & 0_\ell \\
H" & I_{n-k-\ell}\end{pmatrix}
$$
That such a matrix exists is a consequence of the fact that $H$ restricted to the last $n-k-\ell$ positions is of full rank.
Write now $e=(e',e")$ where $e'$ is the word formed by the $k+\ell$ first entries of $e$. Then
$$Us^T=UHe^T = \begin{pmatrix} H'{e'}^T \\ H"{e'}^T + {e"}^T \end{pmatrix}.$$
If we write $Us^T$ as $(s',s")^T$, where $s'^T$ is the vector formed by the $\ell$ first entries 
of $Us^T$, then we recover $e$ from $e'$
by using the fact that $H"{e'}^T + {e"}^T={s"}^T$. $~\qed$
\end{proof}
 From now on, we denote by $\Sigma$ and $h$ the functions that can be computed in polynomial time that are promised by this proposition, i.e.
\begin{eqnarray*}
s ' & = & \Sigma(s,H,\zS)\\
e & = & h(e')
\end{eqnarray*}

In other words, all these algorithms  solve in a first step a new instance of the syndrome decoding problem with different parameters. The difference
with the original problem is that if $\ell$ is small, which is the case in general, there is not a single solution anymore. 
However searching for all (or a large set of them) can be done more efficiently than just brute-forcing over all errors of weight $p$ on the set $\zS$.
Once a possible solution $e'$ to \eqref{eq:subproblem} is found, $e$ is recovered as explained before. 
The main idea which avoids brute forcing over all possible errors of weight $p$ on $\zS$ is to obtain candidates $e'$ by solving an instance of a
generalised $k$-sum problem that we define as follows.
\begin{problem}[generalised $k$-sum problem]
Consider an Abelian group $\zG$, an arbitrary set $\zE$, a map $f$ from $\zE$ to $\zG$, $k$ subsets $\zV_0$, $\zV_1$, \dots, $\zV_{k-1}$ of $\zE$, another map $g$ from $\zE^k$ to $\{0,1\}$,  and an element $S \in \zG$.
 Find
a  solution $(v_0,\dots,v_{k-1}) \in \zV_0\times \dots \zV_{k-1}$ such that we have at the same time
\begin{itemize}
\item[(i)] $f(v_0) + f(v_1) \dots + f(v_{k-1}) = S$ (subset-sum condition);
\item[(ii)] $g(v_0,\dots,v_{k-1})  =  0$   $((v_0,\dots,v_{k-1})$ is a root of $g)$.
\end{itemize}
\end{problem}

Dumer's ISD algorithm, for instance, solves the $2$-sum problem in the case where
\begin{eqnarray*}
\zG & = &\Ft^\ell, \;\;\zE = \Ft^{k+\ell},\;\;f(v)  =  H'{v}^T\\
\zV_0 &= &\{(e_0,0_{(k+\ell)/2})\in \Ft^{k+\ell} : e_0 \in \Ft^{(k+\ell)/2},\; |e_0|=p/2\}  \\
\zV_1 &= &\{(0_{(k+\ell)/2},e_1)\in \Ft^{k+\ell} : e_1 \in \Ft^{(k+\ell)/2},\; |e_1|=p/2\} 
\end{eqnarray*}
and $g(v_0,v_1)=0$ if and only if $e=h(e')$ is of weight $w$ where $e'=v_0 + v_1$.
 A solution to the $2$-sum problem is then clearly a solution to the decoding problem by construction.
The point is that the $2$-sum problem can be solved in time which is much less than $|\zV_0|\cdot|\zV_1|$. For instance,
this can clearly be achieved in expected time $|\zV_0|+|\zV_1|+\frac{|\zV_0|\cdot|\zV_1|}{|\zG|}$
and space $|\zG|$ by storing the elements $v_0$ of $\zV_0$ in a hashtable at the address $f(v_0)$ and then going over all elements $v_1$ of the other set 
to check whether or not the address $S-f(v_1)$ contains an element. The term $\frac{|\zV_0|\cdot|\zV_1|}{|\zG|}$ accounts for the expected number of solutions 
of the $2$-sum problem when the elements of $\zV_0$ and $\zV_1$ are chosen uniformly at random in $\zE$ (which is the assumption what we are going to make from on).
This is precisely what Dumer's algorithm does. Generally, the size of $\zG$ is chosen such that $|\zG|=\Th{|\zV_i|}$ and the space and time complexity are
also of this order.


Generalised ISD algorithms are thus composed of a loop in which first a set $\zS$ is sampled and then an error vector having a certain form, namely with $p$ error positions in $\zS$ and $w-p$ error positions outside of $\zS$, is sought. Thus, 
for each ISD algorithm $A$, we will denote by $Search_A$ the algorithm
 whose exact implementation depends on $A$ but whose specification is always\\
$Search_A : \zS, H, s, w, p \rightarrow \{ e ~|~ \text{$e$ has weight $p$ on $\zS$ and weight $w-p$ on $\overline{\zS}$ and } s^T = He^T\} \cup \{NULL\}$, 
 where $\zS$ is a set of indices, $H$ is the parity-check matrix of the code and $s$ is the syndrome of the error we are looking for.
The following pseudo-code gives the structure of a generalised ISD algorithm.\\
\begin{algorithm}[H]
 \DontPrintSemicolon
 \KwIn{$H$, $s$, $w$, $p$}
 \KwOut{$e$ of weight $w$ such that $s^T = He^T$}
 \RepeatU{$|e| = w$}{
  Sample a set of indices $\zS \subset \{1, ...,n\}$\;
  $e \leftarrow Search_A(\zS, H, s, w, p)$\;
 }
 \KwRet $e$\;
 \caption{ISD\_Skeleton}
\end{algorithm}
$~$
Thus, if we note $P_A$ the probability, dependent on the algorithm $A$, that the sampled set $\zS$ is correct and that $A$ finds $e$
\footnote{In the case of Dumer's algorithm, for instance, even if the restriction of $e$ to $\zS$ is of weight $p$, Dumer's algorithm
may fail to find it since it does not split evenly on both sides of the bipartition of $\zS$.}, and $T_A$ the execution time of the algorithm $Search_A$, the complexity of generalised ISD algorithms is 
$\OO{\frac{T_A}{P_A}}.$
To construct generalised quantum ISD algorithms, we use Bernstein's idea of using Grover search to look for a correct set $\zS$. 
However, now each query made by Grover search will take time which is essentially the time complexity of $Search_A$.
Consequently, the complexity of generalised quantum ISD algorithms is given by the following formula:
\begin{equation}
\label{eq:T_quant_ISD}
\OO{\frac{T_A}{\sqrt{P_A}}} = \OO{\sqrt{\frac{T_A^2}{P_A}}}.
\end{equation}
An immediate consequence of this formula is that, in order to halve the complexity exponent of a given classical algorithm, we need a quantum algorithm whose search subroutine is ``twice'' as efficient.


{\tiny }\section{Solving the generalised $4$-sum problem with quantum walks and Grover search}

\subsection{The Shamir-Schroeppel idea}
As explained in Section \ref{sec:classical_quantum_decoding}, the more sophisticated ISD algorithms solve during the inner step 
an instance of the generalised $k$-sum problem.
The issue is to get a good quantum version of the classical algorithms used to solve this problem. That this task is non trivial can already be
guessed from Dumer's algorithm. Recall that it solves the generalised $2$-sum problem in time and space complexity $\OO{V}$ when 
$V=|\zV_0|=|\zV_1|=\Theta(|\zG|)$.
The problem is that if we wanted a quadratic speedup when compared to the classical Dumer algorithm,
then this would require a quantum algorithm solving the same problem in time $\OO{V^{1/2}}$, but this seems problematic since naive ways of quantising this algorithm 
stumble on the problem that the space complexity is a lower bound on the time complexity of the quantum algorithm.
This strongly motivates the choice of ways of solving the $2$-sum problem by using less memory. This can be done through the idea of Shamir and Schroeppel \cite{SS81}.
Note that the very same idea is also used for the same reason to 
speed up quantum algorithms for the subset sum problem in \cite[Sec. 4]{BJLM13}.
To explain the idea, suppose that $\zG$ factorises as $\zG = \zG_0 \times \zG_1$ where $|\zG_0| = \Theta(|\zG_1|)=\Theta(|\zG|^{1/2})$.
Denote for $i\in \{0,1\}$ by $\pi_i$ the projection from $\zG$ onto $\zG_i$ which to $g=(g_0,g_1)$ associates  $g_i$.

The idea is to construct $f(\zV_0)$ and $f(\zV_1)$ themselves as 
$f(\zV_0)=f(\zV_{00})+f(\zV_{01})$ and 
$f(\zV_1)=f(\zV_{10}) + f(\zV_{11})$ in such a way that  the $\zV_{ij}$'s are of size $O(V^{1/2})$
and to solve a $4$-sum problem by solving various $2$-sum problems. In our coding theoretic setting, it will be more convenient to explain everything directly in terms
of the $4$-sum problem which is given in this case by
\begin{problem}\label{pb:SS}
Assume that $k+\ell$  and $p$ are multiples of $4$. 
Let 
\begin{eqnarray*}
\zG & = &\Ft^\ell, \;\;\zE  =  \Ft^{k+\ell}, \;\;f(v)  =  H'{v}^T\\
\zV_{00} &\eqdef &\{(e_{00},0_{3(k+\ell)/4})\in \Ft^{k+\ell} : e_{00 }\in \Ft^{(k+\ell)/4},\; |e_{00}|=p/4\} \\
\zV_{01} &\eqdef &\{(0_{(k+\ell)/4},e_{01},0_{(k+\ell)/2})\in \Ft^{k+\ell} : e_{01 }\in \Ft^{(k+\ell)/4},\; |e_{01}|=p/4\} \\
\zV_{10} &\eqdef &\{(0_{(k+\ell)/2},e_{10},0_{(k+\ell)/4})\in \Ft^{k+\ell} : e_{10 }\in \Ft^{(k+\ell)/4},\; |e_{10}|=p/4\} \\
\zV_{11} &\eqdef &\{(0_{3(k+\ell)/4},e_{11})\in \Ft^{k+\ell} : e_{11}\in \Ft^{(k+\ell)/4},\; |e_{11}|=p/4\} 
\end{eqnarray*}
and $S$ be some element in $\zG$.
Find $(v_{00},v_{01},v_{10},v_{11})$ in $\zV_{00}\times \zV_{01} \times \zV_{10} \times \zV_{11}$ such 
that $f(v_{00})+f(v_{01})+f(v_{10})+f(v_{11})=S$ and $h(v_{00}+v_{01}+v_{10}+v_{11})$ is of weight $w$.
\end{problem}
Let us explain now how the Shamir-Schroeppel idea allows us to solve the $4$-sum problem in time $\OO{V}$ and space $\OO{V^{1/2}}$ when 
the $\zV_{ij}$'s are of order $\OO{V^{1/2}}$, 
$|\zG|$ is of order $V$ and when $\zG$ decomposes as the product of two groups $\zG_0$ and $\zG_1$ both of size $\Th{V^{1/2}}$.
The basic idea is  to solve for all possible $r \in \zG_1$ the following $2$-sum problems 
\begin{eqnarray}
\pi_1(f(v_{00})) +\pi_1(f(v_{01})) & = & r\label{eq:problem1}\\
\pi_1(f(v_{10}))  +\pi_1(f(v_{11})) & = & \pi_1(S)-r \label{eq:problem2}
\end{eqnarray}
Once these problems are solved we are left with  $\OO{V^{1/2} V^{1/2}/V^{1/2}}=\OO{V^{1/2}} $ solutions to the first problem and $\OO{V^{1/2}}$ solutions to the second.
Taking any pair $(v_{00},v_{01})$ solution to \eqref{eq:problem1}  and $(v_{10},v_{11})$ solution to \eqref{eq:problem2} yields a $4$-tuple which is a partial solution to the
$4$-sum problem
$$
\pi_1( f(v_{00}))+\pi_1(f(v_{01})) +\pi_1(f(v_{10}))+\pi_1(f(v_{11}))  = r + \pi_1(S)-r = \pi_1(S).
$$
Let $\zV'_0$ be the set of all pairs $(v_{00},v_{01})$ we have found for the 
first $2$-sum problem \eqref{eq:problem1}, whereas $\zV'_1$ is the set of all solutions to \eqref{eq:problem2}. 
To ensure that $f(v_{00})+f(v_{01}) +f(v_{10})+f(v_{11}) =  S$ 
we just have to 
solve the following $2$-sum problem
$$
\underbrace{\pi_0(f(v_{00})) + \pi_0(f(v_{01}) )}_{f'(v_{00},v_{01})}+ \underbrace{\pi_0(f(v_{10})) + \pi_0(f(v_{11}) )}_{f'(v_{10},v_{11})} = \pi_0(S)
$$
and $$
g(v_{00},v_{01},v_{10},v_{11})=0
$$
where $(v_{00},v_{01})$ is in $\zV'_0$, $(v_{10},v_{11})$ is in $\zV'_1$ and 
$g$ is the function whose root we want to find for the original $4$-sum problem.

This is again of complexity $\OO{V^{1/2} V^{1/2}/V^{1/2}}=\OO{V^{1/2}} $.
Checking a particular value of $r$ takes therefore $\OO{V^{1/2}}$
 operations. Since we have $\Th{V^{1/2}}$ values to check, the 
total complexity is $\OO{V^{1/2}V^{1/2}}=\OO{V}$, that is the same as before, but we need only $\OO{V^{1/2}}$ memory to store all intermediate sets.
\begin{figure}[h]
    \centering
    \includegraphics[width=0.5\textwidth]{ss.png}
    \caption{
The Shamir-Schroeppel idea in the decoding context (see Problem \ref{pb:SS}): the support of the elements of $\zV_{ij}$ is represented in orange, while
the blue and green colours represent $\zG_0$ resp. $\zG_1$.}
    \label{fig:ss}
\end{figure}
\subsection{A quantum version of the Shamir-Schroeppel algorithm}

By following the approach of \cite{BJLM13}, we will define a quantum
algorithm for  
solving the $4$-sum problem
by combining Grover search with a quantum walk
with a complexity given by

\begin{proposition}\label{prop:easy}
Consider the generalised $4$-sum problem  with sets $\zV_u$ of size  $V$. Assume that 
$\zG$ can be decomposed as $\zG =\zG_0 \times \zG_1$. 
There is a quantum algorithm for solving the $4$-sum problem running in time $\OOt{|\zG_1|^{1/2}V^{4/5}}$
as soon as $|\zG_1| = \Om{V^{4/5}}$ and $|\zG| = \Om{V^{8/5}}$.
\end{proposition}

This is nothing but the idea of the algorithm \cite[Sec. 4]{BJLM13} laid out in a more general context.
The idea is as in the classical algorithm to look for the right value $r \in \zG_1$. This can be done with Grover search in 
time $\OO{|\zG_1|^{1/2}}$ instead of $\OO{|\zG_1|}$ in the classical case.
The quantum walk is then used to solve the following problem:

\begin{problem}\label{pb:shamir_shroeppel}
Find $(v_{00},v_{01},v_{10},v_{11})$ in $\zV_{00}\times \zV_{01} \times \zV_{10} \times \zV_{11}$ such that 
\begin{eqnarray*}
\pi_1(f(v_{00})) +\pi_1(f(v_{01})) & = & r \\
\pi_1(f(v_{10}))  +\pi_1(f(v_{11})) & = & \pi_1(S)-r \\
\pi_0(f(v_{00})) + \pi_0(f(v_{01})) + \pi_0(f(v_{10})) + \pi_0(f(v_{11})) & = &\pi_0(S)\\
g(v_{00},v_{01},v_{10},v_{11}) & = & 0.
\end{eqnarray*}
\end{problem}

For this, we choose subsets $\zU_i$'s of the $\zV_i$'s of a same size $U=\Th{V^{4/5}}$ and 
run a quantum walk on the graph whose vertices are all possible $4$-tuples of
sets of this kind and two $4$-tuples $(\zU_{00},\zU_{01},\zU_{10},\zU_{11})$ and
$(\zU_{00}',\zU_{01}',\zU_{10}',\zU_{11}')$ are adjacent if and only if we have for all $i$'s but one $\zU'_i=\zU_i$ and
for the remaining  $\zU'_i$ and $\zU_i$ we have $|\zU'_i \cap \zU_i|=U-1$.
Notice that this graph is nothing but $J^4(V,U)$.
By following \cite[Sec. 4]{BJLM13} it can be proved that 
\begin{restatable}{proposition}{propDataStructure}
\label{prop:data_structure}
Under the assumptions that $|\zG_1| = \Om{V^{4/5}}$ and $|\zG| = \Om{V^{8/5}}$,
it is possible to set up a data structure of size $\OO{U}$
to implement this quantum walk such that \\
(i) setting up the data structure takes time $\OO{U}$;\\
(ii) checking whether a new $4$-tuple leads to a solution to the problem above (and outputting the solution in this case) 
takes time $\OO{1}$,\\
(iii) updating the data structure takes time $\OO{\log U}$.
\end{restatable}
The proof which we give is adapted from \cite[Sec. 4]{BJLM13}.
\begin{proof}
$~$\\
\begin{enumerate}
\item \textbf{Setting up the data structure takes time $\OO{U}$.\\}
The data structure is set up more or less in the same way as in classical Shamir-Schroeppel's algorithm, i.e. by solving two 2-sum problems first and then using the result to solve a third and last 2-sum problem. There are however the following differences:\\ \\
(i) We no longer keep the results in a hashtable but in some other type of ordered data structure which allows for the insertion, deletion and search operations to be done in $\OO{\log U}$ time. For instance, \cite{BJLM13} chose radix trees. More detail will be given when we look at the \textsc{Update} operation.\\
(ii) Because we no longer use hashtables, we will need two data structures at each step, one to keep track of $f(v_{00})$ along with the associated $v_{00}$, $f(v_{00}) + f(v_{01})$ along with the associated $(v_{00},v_{01})$, etc. and another to keep track of $v_{00}$, $(v_{00}, v_{01})$, etc. separately. If we denote the first family of data structures by $\zD_f$ and the second family by $\zD_{\zV}$, this gives a total of 13 data structures (7 of type $\zD_{\zV}$ and 6 of type $\zD_f$, because no data structure is needed to  
store the sum of all four vectors which is simply $S$).\\ \\
Solving the first two 2-sum problems takes time $|\zU_{i0}| + |\zU_{i1}| + \frac{|\zU_{i0}|.|\zU_{i1}|}{|\zG_1|}$,
$i=0,1$, which is $\OO{U}$ because $|\zG_1| = \Om{V^{4/5}} = \Om{U}$. 
Denote by $\zU_{0}$ resp. $\zU_{1}$ the set of solutions to these two problems. These solutions are used to solve the second 2-sum, problem, which takes time 
$|\zU_{0}| + |\zU_{1}| + \frac{|\zU_{0}|.|\zU_{1}|}{|\zG_0|} = \OO{U}$ 
due to 
$\zG_0 = \zG / \zG_1$ 
and $|\zG| = \Om{V^{8/5}}$.\\ \\
Thus, setting up the data structure takes time $\OO{U}$.
\item \textbf{Updating the data structure takes time $\OO{\log U}$.\\}
Recall that the data structures are chosen such that the insertion, deletion and search operations take $\OO{\log U}$ time, and also that there are two data structures pertaining to each vector or pair of vectors, for a total of 13 data structures.\\ \\
Recall also that the update step consists in moving from one vertex of the Johnson graph $J^4(V,U)$ to one that is adjacent to it. Suppose, without loss of generality, that we move from the vertex $(\zU_{00},\zU_{01},\zU_{10},\zU_{11})$ to $(\zU_{00}',\zU_{01},\zU_{10},\zU_{11})$. Thus, a $v_{00} \in \zU_{00}$ has been replaced by a $u_{00} \in \zU_{00}$.\\ \\
Then, the low cost of the update step relies upon the following fundamental insight: there are in all $U$ possible ways of writing the sum $\pi_1(f(u_{00})) +\pi_1(f(v_{01}))$ (one for each $v_{01} \in \zU_{01}$). But we have one further constraint which is that this sum needs to be equal to a given $r \in \zG_1$. Thus, there are on average $\frac{|\zU_{ij}|}{|\zG_1|} = O(1)$ values of $v_{01} \in \zU_{01}$ which give a solution.\\ \\
Note that the same argument applies for the number of $(v_{10}, v_{11}) \in \zU_{1}$ that fulfil the condition
$$\pi_0(f(u_{00})) + \pi_0(f(v_{01})) + \pi_0(f(v_{10})) + \pi_0(f(v_{11})) ~ = ~\pi_0(S)$$
for a given 
$(u_{00}, v_{01}) \in \zU_{0}$
(where $\pi_0(S) \in \zG_0$), 
for in this case there are on average 
$\frac{|\zU_{0}|}{|\zG_0|} = O(1)$ 
such elements.\\ \\
This allows us to proceed as follows: we impose a constant limit on the number of $v_{01} \in \zU_{01}$ that correspond to a given $u_{00} \in \zU_{00}$ at each update operation. A similar limit is imposed on the number of $(v_{10}, v_{11}) \in \zU_{1}$. The probability of reaching this limit is negligeable, and if it is reached, we re-initialise the data structure, so this does not modify the overall complexity of the algorithm. Note also that there is no problem when the opposite situation happens, i.e. when there are no $v_{01} \in \zU_{01}$ corresponding to a given $u_{00} \in \zU_{00}$. Indeed, while this may result in the data structure being depleted, this is a temporary situation and the data structure will be refilled over time as more suitable elements occur.\\ \\
We now enumerate the steps needed to update the data structure. What we need to do is to remove the old element $v_{00}$ and everything that has been constructed using it, and add $u_{00}$ and everything that it allows to construct (within the limits discussed above).
First, to remove $v_{00}$ and the other elements it affects, we need to do the following:
\begin{enumerate}
\item Find and delete $v_{00}$ from the data structure $\zD_{\zU_{00}}$.
\item Calculate $f(v_{00})$, then find and delete it from the data structure $\zD_{f_{00}}$.
\item Find at most a constant number of $(v_{00}, v_{01})$ in $\zD_{\zU_{0}}$ and remove them.
\item For each of these $(v_{00}, v_{01})$, calculate $f(v_{00}) + f(v_{01})$ and remove it from $\zD_{f_{0}}$.
\item Find at most a constant number of $(v_{00}, v_{01}, v_{01}, v_{11})$ in $\zD_{\zU}$ and remove them.
\end{enumerate}
This step uses operations of negligeable cost (calculating $f(v_{00})$, etc.) and the number of operations of cost 
$\log(U)$ 
which it uses is bounded by a constant. Thus it takes time 
$\OO{\log(U)}$.\\ \\
To add $u_{00}$ and other new elements depending on it, we proceed as follows:
\begin{enumerate}
\item Insert $u_{00}$ in $\zD_{\zU_{00}}$.
\item Calculate $f(u_{00})$, then insert it in $\zD_{f_{00}}$.
\item Calculate $x=r - \pi_1(f(u_{00}))$ and find if there are elements $y$ in $\zD_{f_{01}}$ such that $\pi_1(y)=x$. For a constant number of associated $v_{01}$, insert $(u_{00}, v_{01})$ in $\zD_{\zU_{0}}$ and in $\zD_{f_{0}}$ associated with $r$.
\item Similarly there are a constant number of $(v_{01}, v_{11})$ that need to be updated, for those calculate $g(u_{00}, v_{01}, v_{10}, v_{11})$. If it is equal to zero, insert $(v_{00}, v_{01}, v_{10}, v_{11})$ in $\zD_{\zU}$.
\end{enumerate}
It is easy to see that this step also takes time 
$\OO{\log(U)}$.
\item \textbf{Checking whether a new $4$-tuple leads to a solution of the problem takes time $\OO{1}$.\\}
Checking that the right $4$-tuple is in $\zD_{\zU}$ requires looking for it in $\zD_{\zU}$ at the first step of the algorithm. This costs $O(\sqrt{U})$ using Grover search. At the following steps of the algorithm, it is enough to check the new elements (whose number is bounded by a constant) that have been added to $\zD_{\zU}$. So the checking cost is 
$\OO{1}$
 overall.$~\qed$
\end{enumerate}
\end{proof}
Proposition \ref{prop:easy} is essentially a corollary of this proposition. 

\begin{proof} [Proof of Proposition \ref{prop:easy}]
Recall that 
the cost of the quantum walk is given by
$
\CS + \frac{1}{\sqrt\varepsilon}\left(\CC + \frac{1}{\sqrt\delta}\CU\right)$
 where $\CS,\CC,\CU,\varepsilon$ and $\delta$
are the setup cost, the check cost, the update cost, the proportion of marked elements
and the spectral gap of the quantum walk. 
From Proposition \ref{prop:data_structure}, we know that
$\CS =  \OO{U} = \OO{V^{4/5}}$, $\CC  =  \OO{1}$,
and $\CU  =  \OO{\log U}$.
Recall that the spectral gap of $J(V,U)$ is equal to $\frac{V}{U(V-U)}$ by \eqref{eq:spectral_gap_johnson}.
This quantity is larger than $\frac{1}{U}$ and by using  
 Theorem \ref{thm:product_johnsongraphs} on the cartesian product of Johnson graphs, we obtain
$\delta = \Th{\frac{1}{U}}$.

Now for the proportion of marked elements we argue as follows. If Problem \ref{pb:shamir_shroeppel} has a solution $(v_{00},v_{01},v_{10},v_{11})$, then the probability that 
each of the sets $\zU_i$ contains $v_i$ is precisely $U/V=\Th{V^{-1/5}}$.
The probability $\varepsilon$ that all the $\zU_i$'s contain $v_i$ is then 
$\Th{V^{-4/5}}$.  
This gives a total cost of 
$$
\OO{V^{4/5}} + \OO{V^{2/5}}\left( \OO{1} + \OO{V^{2/5}}\OO{\log U}\right) = \OOt{V^{4/5}}.
$$
When we multiply this by the cost of Grover's algorithm for finding the right $r$ we have the aforementioned complexity.$~\qed$
\end{proof}

\subsection{Application to the decoding problem}
When applying this approach to the decoding problem we obtain
\begin{restatable}{theorem}{thmExpSSQW}
\label{th:expSSQW}
We can decode $w=\omega n$ errors in a random linear code of length $n$ and rate $R=\frac{k}{n}$  with a quantum complexity of
order $\OOt{2^{\alpha_{\text{SSQW}}(R,\omega)n}}$ where 
%\ghk{Corrected another $\ell$ that should be a $\lambda$}
$$
\alpha_{\text{SSQW}}(R, \omega) \eqdef \min_{(\pi,\lambda) \in \zR} \left(\frac{H_2(\omega) - (1-R-\lambda)H_2\left(\frac{\omega - \pi}{1 - R - \lambda}\right) - \frac{2}{5}(R+\lambda)H_2\left(\frac{\pi}{R+\lambda}\right)}{2} \right)$$
$$
\zR  \eqdef  \left\{\!(\pi,\lambda)\!\!\in\!\![0,\omega]\!\times\![0,1)\!:\lambda = \frac{2}{5}(R+\lambda)H_2\!\left(\!\frac{\pi}{R+\lambda}\!\right)\!\!,
\pi \leq R + \lambda, \lambda \leq 1-R-\omega+\pi \! \right\}
$$
%\Leftrightarrow \pi = (R+\lambda)H_2^{-1}\left(\frac{5\lambda}{2(R + \lambda)}\right)$
\end{restatable}
\begin{proof}
Recall (see \eqref{eq:T_quant_ISD}) that the quantum complexity is given by
\begin{equation}
\label{eq:complexitySSQW}
\tilde O\left(\frac{T_{\text{SSQW}}}{\sqrt{P_{\text{SSQW}}}}\right)
\end{equation}
where $T_{\text{SSQW}}$ is the complexity of the combination of Grover's algorithm and quantum walk
solving the generalised $4$-sum problem specified in Problem \ref{pb:shamir_shroeppel} and $P_{\text{SSQW}}$ is the probability
that the random set of $k + \ell$ positions $\zS$ and its random partition in $4$ sets of the same size that are chosen is such that 
all four of them contain exactly $p/4$ errors.
Note that $p$ and $\ell$ are chosen such that $k+\ell$ and $p$ are divisible by $4$.
$P_{\text{SSQW}}$ is given by
$$
P_{\text{SSQW}} = \frac{  {\binom{\frac{k+\ell}{4}}{\frac{p}{4}}}^4 \binom{n-k-\ell}{w - p}}{\binom{n}{w}}
$$
Therefore
\begin{equation}
\label{eq:PSSQW}
(P_{\text{SSQW}})^{-1/2} = \OOt{2^{\frac{H_2(\omega) - (1-R-\lambda)H_2\left(\frac{\omega - \pi}{1 - R - \lambda}\right) - (R+\lambda)H_2\left(\frac{\pi}{R+\lambda}\right) }{2} n}}
\end{equation}
where $\lambda \eqdef \frac{\ell}{n}$ and $\pi \eqdef \frac{p}{n}$. $T_{\text{SSQW}}$ is given by Proposition \ref{prop:easy}:
$$
T_{\text{SSQW}}  =  \OOt{|\zG_1|^{1/2}V^{4/5}}
$$
where the sets involved in the generalised $4$-sum problem are specified in Problem \ref{pb:shamir_shroeppel}.
This gives
$$
V  =  \binom{\frac{k + \ell}{4}}{\frac{p}{4}}
$$
We choose $\zG_1$ as 
\begin{equation}\label{eq:condition1}
\zG_1 = \Ft^{\lceil \frac{\ell}{2} \rceil}
\end{equation}
and the assumptions of Proposition \ref{prop:easy} are verified as soon as
$$
2^\ell = \Om{V^{8/5}}.
$$
which amounts to
$$
2^\ell = \Om{  {\binom{\frac{k + \ell}{4}}{\frac{p}{4}}}^{8/5}}
$$
This explains the condition 
\begin{equation}\label{eq:condition2}
\lambda = \frac{2}{5} (R + \lambda) H_2\left( \frac{\pi}{R+\lambda} \right)
\end{equation}
found in the definition of the region $\zR$.
With the choices \eqref{eq:condition1} and \eqref{eq:condition2}, 
we obtain
\begin{eqnarray}
T_{\text{SSQW}} &= &\OOt{V^{6/5}} \nonumber \\
& = & \OOt{ 2^{\frac{3}{10}(R+\lambda)H_2\left( \frac{\pi}{R+\lambda} \right)n}}\label{eq:TSSQW}
\end{eqnarray}
Substituting for $P_{\text{SSQW}}$ and $T_{\text{SSQW}}$ the expressions given by \eqref{eq:PSSQW} and \eqref{eq:TSSQW} finishes the proof of the
theorem. $~\qed$
\end{proof}

\section{Improvements obtained by the representation technique and ``$1+1=0$''}
There are two techniques that can be used to
speed up the quantum algorithm of the previous section.


\par{\em The representation technique.} It was introduced in \cite{HJ10} to speed up algorithms for the subset-sum algorithm and
used later on in \cite{MMT11} to improve decoding algorithms. The basic idea of the representation technique in the context of the 
subset-sum or decoding algorithms consists in (i) changing slightly the underlying (generalised) $k$-sum problem which is solved by 
introducing sets $\zV_i$ for which there are (exponentially) many solutions to the problem $\sum_i f(v_i) = S$ by using redundant representations,
 (ii) noticing that this allows us to put additional subset-sum conditions on the solution.

In the decoding context, instead of considering sets of errors with non-overlapping support, the idea that allows us to obtain many 
different representations of a same solution is just to consider sets $\zV_i$ corresponding to errors with overlapping supports. In our case,
we could have taken instead of the four sets defined in the previous section the following sets
\begin{eqnarray*}
\zV_{00}=\zV_{10} &\eqdef &\{(e_{00},0_{(k+\ell)/2})\in \Ft^{k+\ell} : e_{00 }\in \Ft^{(k+\ell)/2},\; |e_{00}|=p/4\} \\
\zV_{01}=\zV_{11} &\eqdef &\{(0_{(k+\ell)/2},e_{01})\in \Ft^{k+\ell} : e_{01 }\in \Ft^{(k+\ell)/2},\; |e_{01}|=p/4\}
\end{eqnarray*}
Clearly a vector $e$ of weight $p$ can be written in many different ways as a sum
$v_{00}+v_{01}+v_{10}+v_{11} $ where $v_{ij}$ belongs to $\zV_{ij}$. This is (essentially) due to the fact that a vector of weight $p$ can be written in $\binom{p}{p/2} = \OOt{2^p}$ 
ways as a sum of two vectors  of weight $p/2$.

The point is that if we apply now the same algorithm as in the previous section and look for solutions to Problem \ref{pb:SS}, there is not a single value of $r$
that leads to the right solution. Here, about $2^p$ values of $r$ will do the same job. The speedup obtained by the representation technique 
is a consequence of this phenomenon. We can even improve on this representation technique by using the $1+1=0$ phenomenon as in \cite{BJMM12}.


\par{\em The ``$1+1=0$'' phenomenon.} Instead of choosing the $\zV_i$'s as explained above we will actually choose the $\zV_i$'s as
\begin{eqnarray}
\zV_{00}=\zV_{10} &\eqdef &\{(e_{00},0_{(k+\ell)/2})\in \Ft^{k+\ell} : e_{00 }\in \Ft^{(k+\ell)/2},\; |e_{00}|=\frac{p}{4} +\frac{\Delta p}{2}\} \label{eq:v00v10}\\
\zV_{01}=\zV_{11} &\eqdef &\{(0_{(k+\ell)/2},e_{01})\in \Ft^{k+\ell} : e_{01 }\in \Ft^{(k+\ell)/2},\; |e_{01}|=\frac{p}{4} +\frac{\Delta p}{2}\} \label{eq:v01v11}
\end{eqnarray}
A vector $e$ of weight $p$ in $\Ft^{k+\ell}$ can indeed by represented in many ways as a sum of $2$ vectors of weight $\frac{p}{2} +\Delta p$. More precisely, such a vector can be 
represented in 
$
\binom{p}{p/2} \binom{k+\ell-p}{\Delta p} 
$
ways. Notice that this number of representations is greater than the number $2^p$ that we had before. This explains why choosing an appropriate
positive value $\Delta p$ allows us to improve on the previous choice.

 

The quantum algorithm for decoding follows the same pattern as in the previous section:
(i) we look with Grover's search algorithm for a right set $\zS$ of $k + \ell$ positions 
such that the restriction $e'$ of the error $e$ we look for is of weight $p$ on this subset
and then (ii) we search for $e'$ by solving a generalised $4$-sum problem  with a combination
of Grover's algorithm and a quantum walk. We will use for the second point the following proposition
which quantifies how much we gain when there are multiple representations/solutions:

\begin{proposition}\label{prop:improvement}
Consider the generalised $4$-sum problem  with sets $\zV_u$ of size  $\OO{V}$. Assume that 
$\zG$ can be decomposed as $\zG =\zG_0 \times \zG_1 \times \zG_2$. Furthermore assume that 
there are $\Om{|\zG_2|}$ solutions to the $4$-sum problem and that we can fix arbitrarily 
the value $\pi_2\left(f(v_{00})+f(v_{01})\right)$ of a solution to the $4$-sum problem, 
where $\pi_2$ is the mapping from $\zG =\zG_0 \times \zG_1 \times \zG_2$ to $\zG_2$ which maps 
$(g_0,g_1,g_2)$ to $g_2$.
There is a quantum algorithm for solving the $4$-sum problem running in time $\OOt{|\zG_1|^{1/2}V^{4/5}}$
as soon as $|\zG_1|\cdot|\zG_2| = \Om{V^{4/5}}$ and $|\zG| = \Om{V^{8/5}}$.
\end{proposition}

\begin{proof}
Let us first introduce a few notations.
We denote by $\pi_{12}$ the ``projection'' from $\zG=\zG_0 \times \zG_1 \times \zG_2$ to 
$\zG_1 \times \zG_2$ which associates to $(g_0,g_1,g_2)$ the pair $(g_1,g_2)$
and by $\pi_0$ the projection from $\zG$ to $\zG_0$ which maps 
$(g_0,g_1,g_2)$ to $g_0$.
As in the previous section, we solve with a  quantum walk the following problem:
we fix an element $r=(r_1,r_2)$ in $\zG_1 \times \zG_2$ and
find (if it exists) $(v_{00},v_{01},v_{10},v_{11})$ in $\zV_{00}\times \zV_{01} \times \zV_{10} \times \zV_{11}$ such that 
\begin{eqnarray*}
\pi_{12}(f(v_{00})) +\pi_{12}(f(v_{01})) & = & r \\
\pi_{12}(f(v_{10}))  +\pi_{12}(f(v_{11})) & = & \pi_{12}(S)-r \\
\pi_0(f(v_{00})) + \pi_0(f(v_{01})) + \pi_0(f(v_{10})) + \pi_0(f(v_{11})) & = &\pi_0(S)\\
g(v_{00},v_{01},v_{10},v_{11}) & = & 0.
\end{eqnarray*}
The difference with Proposition \ref{prop:easy} is that we do not check all possibilities for $r$ but just all possibilities for $r_1 \in \zG_1$ and
fix $r_2$ arbitrarily. As in Proposition \ref{prop:easy}, we perform a quantum walk whose complexity is $\OOt{V^{4/5}}$ to 
solve the aforementioned problem for a fixed $r$. 
What remains to be done is to find the right value for $r_1$ which is achieved by a Grover 
search with complexity $\OO{|\zG_1|^{1/2}}$.$~\qed$
\end{proof}
\begin{figure}[h]
    \centering
    \includegraphics[width=0.5\textwidth]{mmt.png}
    \caption{The representation technique: 
the support of the elements of $\zV_{ij}$ is represented in orange, while
the blue, green and violet colours represent $\zG_0$ resp. $\zG_1$, resp. $\zG_2$.}
    \label{fig:mmt}
\end{figure}
By  applying Proposition \ref{prop:improvement} in our decoding context, we obtain
\begin{restatable}{theorem}{thmExpMMTQW}
\label{th:expMMTQWdiese}
We can decode $w=\omega n$ errors in a random linear code of length $n$ and rate $R=\frac{k}{n}$  with a quantum complexity of 
order $ \OOt{2^{\alpha_{\text{MMTQW}}(R,\omega)n}}$ where
\begin{eqnarray*}
    \alpha_{\text{{\tiny{MMTQW}}}}(R,\omega) &\eqdef& \min_{(\pi,\Delta \pi, \lambda) \in \zR} \left( 
\frac{  \beta(R,\lambda,\pi,\Delta \pi)+ \gamma(R,\lambda,\pi,\omega)}{2}\right) \\
&\text{with} & \\
\beta(R,\lambda,\pi,\Delta \pi) & \eqdef & \frac{6}{5} (R+\lambda)H_2\left(\frac{\pi/2+\Delta \pi}{R+\lambda}\right)- \pi - (1-R-\lambda)H_2\left(\frac{\Delta \pi }{1-R-\lambda}\right),\\
\gamma(R,\lambda,\pi,\omega) & \eqdef & H_2(\omega) - (1-R-\lambda)H_2(\frac{\omega - \pi}{1-R-\lambda}) - (R+\lambda)H_2\left(\frac{\pi}{R+\lambda}\right)
\end{eqnarray*}
where $\zR$ is the subset of elements $(\pi,\Delta \pi, \lambda)$ of 
$[0,\omega]  \times [0,1) \times [0,1) $
 that satisfy the following constraints
\begin{eqnarray*}
0 \leq  &\Delta \pi  &\leq R + \lambda - \pi \\
0 \leq &\pi &\leq \min(\omega, R + \lambda)\\
0 \leq &\lambda &\leq 1 - R - \omega + \pi \\
&\pi &= 2\left((R+\lambda)H_2^{-1}\left(\frac{5\lambda}{4(R+\lambda)}\right) - \Delta \pi\right)
\end{eqnarray*}
\end{restatable}
\begin{proof}
The algorithm picks random subsets $\zS$ of size $k+\ell$ with the hope that the restriction to $\zS$ of the error
of weight $w$ that we are looking for is of weight $p$. 
Then it solves for each of these subsets the generalised
$4$-sum problem where the sets $\zV_{ij}$ are specified in \eqref{eq:v00v10} and \eqref{eq:v01v11}, 
and $\zG$, $\zE$, $f$ and $g$ are as  in Problem \ref{pb:shamir_shroeppel}.
$g$ is in this case slightly more complicated for the sake of analysing the algorithm.
We have 
$g(v_{00},v_{01},v_{10},v_{11})=0$ 
 if and only if (i) $v_{00}+v_{01}+v_{10}+v_{11}$ is of weight $p$ 
(this is the additional constraint we use for the analysis of the algorithm)
(ii) $f(v_{00})+f(v_{01})+f(v_{10})+f(v_{11})=\Sigma(s,H,\zS)$ and (iii) 
$h(v_{00}+v_{01}+v_{10}+v_{11})$ is of weight $w$.


From \eqref{eq:T_quant_ISD} we  know that the quantum complexity is given by
\begin{equation}
\label{eq:complexityMMTQW}
\tilde O\left(\frac{T_{\text{MMTQW}}}{\sqrt{P_{\text{MMTQW}}}}\right)
\end{equation}
where $T_{\text{MMTQW}}$ is the complexity of the combination of Grover's algorithm and quantum walk
solving the generalised $4$-sum problem specified above and $P_{\text{MMTQW}}$ is the probability
that the restriction $e'$ of the error $e$ to $\zS$ is of weight $p$
and that this error can be written as $e' = v_{00}+v_{01}+v_{10} + v_{11}$ where the $v_{ij}$ belong to 
$\zV_{ij}$.
It is readily verified that 
$$
P_{\text{MMTQW}} = \OOt{\frac{ \binom{k+\ell}{p} \binom{n-k-\ell}{w - p}}{\binom{n}{w}}}
$$
By using asymptotic expansions of the binomial coefficients we obtain
\begin{equation}
\label{eq:PMMTQW}
(P_{\text{MMTQW}})^{-1/2} = \OOt{2^{\frac{H_2(\omega) - (1-R-\lambda)H_2\left(\frac{\omega - \pi}{1 - R - \lambda}\right) - (R+\lambda)H_2\left(\frac{\pi}{R+\lambda}\right) }{2} n}}
\end{equation}
where $\lambda \eqdef \frac{\ell}{n}$ and $\pi \eqdef \frac{p}{n}$. To estimate $T_{\text{SSQW}}$, we can use Proposition \ref{prop:improvement}.
The point is that the number of different solutions of the generalised $4$-sum problem 
(when there is one) is of order 
$$
\Omt{\binom{p}{p/2} \binom{k+\ell-p}{\Delta p}}.
$$
At this point, we observe that
$$
\log_2\left(\binom{p}{p/2} \binom{k+\ell-p}{\Delta p}\right) = p+ (k+\ell-p)H_2\left( \frac{\Delta p}{k+\ell -p} \right) +o(n)
$$
when $p$, $\Delta p$, $\ell$, $k$ are all linear in $n$.
In other words, we may use Proposition $\ref{prop:improvement}$ with 
$\zG_2 = \Ft^{\ell_2}$ with 
\begin{equation}
\label{eq:ell2}
\ell_2 \eqdef p+ (k+\ell-p)H_2\left( \frac{\Delta p}{k+\ell -p} \right).
\end{equation}
We use now Proposition \ref{prop:improvement}
with $\zG_2$ chosen as explained above. 
$V$ is given in this case by
$$
V = \binom{\frac{k+\ell}{2}}{\frac{p}{4}+\frac{\Delta p}{2}} = \OOt{2^{\frac{(R+\lambda) H_2\left(\frac{\pi/2 + \Delta \pi}{R + \lambda}\right)n}{2}}}
$$
where $\Delta \pi \eqdef \frac{\Delta p}{n}$.
We choose the size of $\zG$ such that
\begin{equation}\label{eq:G}
|\zG| = \Tht{V^{8/5}}
\end{equation}
which gives
$$
2^\ell = \Tht{  {\binom{\frac{k+\ell}{2}}{\frac{p}{4}+\frac{\Delta p}{2}}}^{8/5}}.
$$
This explains why we impose 
$$
\lambda = \frac{8}{5}\frac{R+\lambda}{2} H_2\left(\frac{\pi/2+\Delta \pi}{R+\lambda}\right)
$$
which is equivalent to the condition 
$$
\frac{5 \lambda}{4(R+\lambda)} = H_2\left(\frac{\pi/2+\Delta \pi}{R+\lambda}\right)
$$
which in turn is equivalent to the condition
\begin{equation}\label{eq:condition2MMT}
\pi = 2\left((R+\lambda)H_2^{-1}\left(\frac{5\lambda}{4(R+\lambda)}\right) - \Delta \pi\right)
\end{equation}
found in the definition of the region $\zR$.
The size of $\zG_1$ is chosen such that
\begin{equation}\label{eq:condition1MMT}
|\zG_1| \cdot |\zG_2| = \Ft^{\lceil \frac{\ell}{2} \rceil}.
\end{equation}
By using \eqref{eq:ell2} and \eqref{eq:G}, this implies
\begin{eqnarray}
|\zG_1| & = & \Tht{\frac{V^{4/5}}{2^{p+ (k+\ell-p)H_2\left( \frac{\Delta p}{k+\ell -p} \right)}}}
\end{eqnarray}

With the choices \eqref{eq:condition1MMT} and \eqref{eq:condition2MMT}, 
we obtain
\begin{eqnarray}
T_{\text{MMTQW}} &= &\OOt{|\zG_1|^{1/2} \cdot V^{4/5}} \nonumber \\
& = & \OOt{\frac{V^{6/5}}{2^{\frac{p}{2}+ \frac{k+\ell-p}{2} H_2\left( \frac{\Delta p}{k+\ell -p} \right)}}} \nonumber \\
& = & \OOt{ 2^{\left[\frac{3}{5}(R+\lambda)H_2\left( \frac{\pi/2 +\Delta \pi}{R+\lambda} \right)- \frac{\pi}{2} - \frac{R+\lambda-\pi}{2} H_2\left( \frac{\Delta \pi}{R+\lambda -
\pi} \right) \right]n}}\label{eq:TMMTQW}
\end{eqnarray}
Substituting for $P_{\text{MMTQW}}$ and $T_{\text{MMTQW}}$ the expressions given by \eqref{eq:PMMTQW} and \eqref{eq:TMMTQW} finishes the proof of the
theorem.$~\qed$
\end{proof}

\section{CONCLUSION}
\label{sec:conclusion}

% \Ram{Want to mention that we can do parameteric uncertainty \cite{mohan2016convex}, hybrid \cite{shia2014convex}, and alpha confidence \cite{holmes2016convex}
% Also we can pose the optimization problem using convex optimization based methods \cite{zhao2016control,zhao2017optimal}.}

This paper presents a method to plan safe trajectories with obstacle avoidance for autonomous vehicles.
This approach is able to guarantee safety in arbitrary environments for multiple, static obstacles.
The method begins with computing the forward reachable set (FRS) of parameterized trajectories that a vehicle can realize.
This set is computed in continuous space and time, and is robust to model uncertainty between the dynamics of the vehicle's mid- and low-level controllers.

As an example, we use a kinematic Dubin's car and dynamic unicycle model as low- and high-fidelity models.
At runtime, the FRS is intersected with obstacles to eliminate unsafe trajectories, and an optimal trajectory is chosen from the remaining, safe trajectories.
This method is proven to ensure vehicle safety for present and future static obstacles by considering the time required for path planning, the time required to stop the vehicle, and the error between the low- and high-fidelity models.
One thousand simulations with randomly-located obstacles were run to show the effectiveness of this method. 

%We are preparing a Segway as a platform to demonstrate this method on a real system.
%We will also apply this method to higher degree-of-freedom vehicle models in both simulation and experiments to further develop robustness to environmental and dynamic uncertainty.

The next step in applying this method is to expand the error function $g$ beyond modeling uncertainty.
To reduce the conservativeness of the approach, we plan to explore extension to the FRS computation that incorporate confidence level sets \cite{mohan2016convex,holmes2016convex}.
The error function $g$ can also be improved by considering time variation of trajectory parameters across planning steps by posing the FRS computation as a hybrid problem \cite{shia2014convex}.
Finally, we plan to use convex optimization to find a global solution to the nonlinear trajectory optimization problem at each planning step \cite{zhao2016control,zhao2017optimal}.


\bibliographystyle{acm}
\addcontentsline{toc}{section}{Bibliography}
\bibliography{codecrypto}
\newpage
\chapter{Supplementary Material}
\label{appendix}

In this appendix, we present supplementary material for the techniques and
experiments presented in the main text.

\section{Baseline Results and Analysis for Informed Sampler}
\label{appendix:chap3}

Here, we give an in-depth
performance analysis of the various samplers and the effect of their
hyperparameters. We choose hyperparameters with the lowest PSRF value
after $10k$ iterations, for each sampler individually. If the
differences between PSRF are not significantly different among
multiple values, we choose the one that has the highest acceptance
rate.

\subsection{Experiment: Estimating Camera Extrinsics}
\label{appendix:chap3:room}

\subsubsection{Parameter Selection}
\paragraph{Metropolis Hastings (\MH)}

Figure~\ref{fig:exp1_MH} shows the median acceptance rates and PSRF
values corresponding to various proposal standard deviations of plain
\MH~sampling. Mixing gets better and the acceptance rate gets worse as
the standard deviation increases. The value $0.3$ is selected standard
deviation for this sampler.

\paragraph{Metropolis Hastings Within Gibbs (\MHWG)}

As mentioned in Section~\ref{sec:room}, the \MHWG~sampler with one-dimensional
updates did not converge for any value of proposal standard deviation.
This problem has high correlation of the camera parameters and is of
multi-modal nature, which this sampler has problems with.

\paragraph{Parallel Tempering (\PT)}

For \PT~sampling, we took the best performing \MH~sampler and used
different temperature chains to improve the mixing of the
sampler. Figure~\ref{fig:exp1_PT} shows the results corresponding to
different combination of temperature levels. The sampler with
temperature levels of $[1,3,27]$ performed best in terms of both
mixing and acceptance rate.

\paragraph{Effect of Mixture Coefficient in Informed Sampling (\MIXLMH)}

Figure~\ref{fig:exp1_alpha} shows the effect of mixture
coefficient ($\alpha$) on the informed sampling
\MIXLMH. Since there is no significant different in PSRF values for
$0 \le \alpha \le 0.7$, we chose $0.7$ due to its high acceptance
rate.


% \end{multicols}

\begin{figure}[h]
\centering
  \subfigure[MH]{%
    \includegraphics[width=.48\textwidth]{figures/supplementary/camPose_MH.pdf} \label{fig:exp1_MH}
  }
  \subfigure[PT]{%
    \includegraphics[width=.48\textwidth]{figures/supplementary/camPose_PT.pdf} \label{fig:exp1_PT}
  }
\\
  \subfigure[INF-MH]{%
    \includegraphics[width=.48\textwidth]{figures/supplementary/camPose_alpha.pdf} \label{fig:exp1_alpha}
  }
  \mycaption{Results of the `Estimating Camera Extrinsics' experiment}{PRSFs and Acceptance rates corresponding to (a) various standard deviations of \MH, (b) various temperature level combinations of \PT sampling and (c) various mixture coefficients of \MIXLMH sampling.}
\end{figure}



\begin{figure}[!t]
\centering
  \subfigure[\MH]{%
    \includegraphics[width=.48\textwidth]{figures/supplementary/occlusionExp_MH.pdf} \label{fig:exp2_MH}
  }
  \subfigure[\BMHWG]{%
    \includegraphics[width=.48\textwidth]{figures/supplementary/occlusionExp_BMHWG.pdf} \label{fig:exp2_BMHWG}
  }
\\
  \subfigure[\MHWG]{%
    \includegraphics[width=.48\textwidth]{figures/supplementary/occlusionExp_MHWG.pdf} \label{fig:exp2_MHWG}
  }
  \subfigure[\PT]{%
    \includegraphics[width=.48\textwidth]{figures/supplementary/occlusionExp_PT.pdf} \label{fig:exp2_PT}
  }
\\
  \subfigure[\INFBMHWG]{%
    \includegraphics[width=.5\textwidth]{figures/supplementary/occlusionExp_alpha.pdf} \label{fig:exp2_alpha}
  }
  \mycaption{Results of the `Occluding Tiles' experiment}{PRSF and
    Acceptance rates corresponding to various standard deviations of
    (a) \MH, (b) \BMHWG, (c) \MHWG, (d) various temperature level
    combinations of \PT~sampling and; (e) various mixture coefficients
    of our informed \INFBMHWG sampling.}
\end{figure}

%\onecolumn\newpage\twocolumn
\subsection{Experiment: Occluding Tiles}
\label{appendix:chap3:tiles}

\subsubsection{Parameter Selection}

\paragraph{Metropolis Hastings (\MH)}

Figure~\ref{fig:exp2_MH} shows the results of
\MH~sampling. Results show the poor convergence for all proposal
standard deviations and rapid decrease of AR with increasing standard
deviation. This is due to the high-dimensional nature of
the problem. We selected a standard deviation of $1.1$.

\paragraph{Blocked Metropolis Hastings Within Gibbs (\BMHWG)}

The results of \BMHWG are shown in Figure~\ref{fig:exp2_BMHWG}. In
this sampler we update only one block of tile variables (of dimension
four) in each sampling step. Results show much better performance
compared to plain \MH. The optimal proposal standard deviation for
this sampler is $0.7$.

\paragraph{Metropolis Hastings Within Gibbs (\MHWG)}

Figure~\ref{fig:exp2_MHWG} shows the result of \MHWG sampling. This
sampler is better than \BMHWG and converges much more quickly. Here
a standard deviation of $0.9$ is found to be best.

\paragraph{Parallel Tempering (\PT)}

Figure~\ref{fig:exp2_PT} shows the results of \PT sampling with various
temperature combinations. Results show no improvement in AR from plain
\MH sampling and again $[1,3,27]$ temperature levels are found to be optimal.

\paragraph{Effect of Mixture Coefficient in Informed Sampling (\INFBMHWG)}

Figure~\ref{fig:exp2_alpha} shows the effect of mixture
coefficient ($\alpha$) on the blocked informed sampling
\INFBMHWG. Since there is no significant different in PSRF values for
$0 \le \alpha \le 0.8$, we chose $0.8$ due to its high acceptance
rate.



\subsection{Experiment: Estimating Body Shape}
\label{appendix:chap3:body}

\subsubsection{Parameter Selection}
\paragraph{Metropolis Hastings (\MH)}

Figure~\ref{fig:exp3_MH} shows the result of \MH~sampling with various
proposal standard deviations. The value of $0.1$ is found to be
best.

\paragraph{Metropolis Hastings Within Gibbs (\MHWG)}

For \MHWG sampling we select $0.3$ proposal standard
deviation. Results are shown in Fig.~\ref{fig:exp3_MHWG}.


\paragraph{Parallel Tempering (\PT)}

As before, results in Fig.~\ref{fig:exp3_PT}, the temperature levels
were selected to be $[1,3,27]$ due its slightly higher AR.

\paragraph{Effect of Mixture Coefficient in Informed Sampling (\MIXLMH)}

Figure~\ref{fig:exp3_alpha} shows the effect of $\alpha$ on PSRF and
AR. Since there is no significant differences in PSRF values for $0 \le
\alpha \le 0.8$, we choose $0.8$.


\begin{figure}[t]
\centering
  \subfigure[\MH]{%
    \includegraphics[width=.48\textwidth]{figures/supplementary/bodyShape_MH.pdf} \label{fig:exp3_MH}
  }
  \subfigure[\MHWG]{%
    \includegraphics[width=.48\textwidth]{figures/supplementary/bodyShape_MHWG.pdf} \label{fig:exp3_MHWG}
  }
\\
  \subfigure[\PT]{%
    \includegraphics[width=.48\textwidth]{figures/supplementary/bodyShape_PT.pdf} \label{fig:exp3_PT}
  }
  \subfigure[\MIXLMH]{%
    \includegraphics[width=.48\textwidth]{figures/supplementary/bodyShape_alpha.pdf} \label{fig:exp3_alpha}
  }
\\
  \mycaption{Results of the `Body Shape Estimation' experiment}{PRSFs and
    Acceptance rates corresponding to various standard deviations of
    (a) \MH, (b) \MHWG; (c) various temperature level combinations
    of \PT sampling and; (d) various mixture coefficients of the
    informed \MIXLMH sampling.}
\end{figure}


\subsection{Results Overview}
Figure~\ref{fig:exp_summary} shows the summary results of the all the three
experimental studies related to informed sampler.
\begin{figure*}[h!]
\centering
  \subfigure[Results for: Estimating Camera Extrinsics]{%
    \includegraphics[width=0.9\textwidth]{figures/supplementary/camPose_ALL.pdf} \label{fig:exp1_all}
  }
  \subfigure[Results for: Occluding Tiles]{%
    \includegraphics[width=0.9\textwidth]{figures/supplementary/occlusionExp_ALL.pdf} \label{fig:exp2_all}
  }
  \subfigure[Results for: Estimating Body Shape]{%
    \includegraphics[width=0.9\textwidth]{figures/supplementary/bodyShape_ALL.pdf} \label{fig:exp3_all}
  }
  \label{fig:exp_summary}
  \mycaption{Summary of the statistics for the three experiments}{Shown are
    for several baseline methods and the informed samplers the
    acceptance rates (left), PSRFs (middle), and RMSE values
    (right). All results are median results over multiple test
    examples.}
\end{figure*}

\subsection{Additional Qualitative Results}

\subsubsection{Occluding Tiles}
In Figure~\ref{fig:exp2_visual_more} more qualitative results of the
occluding tiles experiment are shown. The informed sampling approach
(\INFBMHWG) is better than the best baseline (\MHWG). This still is a
very challenging problem since the parameters for occluded tiles are
flat over a large region. Some of the posterior variance of the
occluded tiles is already captured by the informed sampler.

\begin{figure*}[h!]
\begin{center}
\centerline{\includegraphics[width=0.95\textwidth]{figures/supplementary/occlusionExp_Visual.pdf}}
\mycaption{Additional qualitative results of the occluding tiles experiment}
  {From left to right: (a)
  Given image, (b) Ground truth tiles, (c) OpenCV heuristic and most probable estimates
  from 5000 samples obtained by (d) MHWG sampler (best baseline) and
  (e) our INF-BMHWG sampler. (f) Posterior expectation of the tiles
  boundaries obtained by INF-BMHWG sampling (First 2000 samples are
  discarded as burn-in).}
\label{fig:exp2_visual_more}
\end{center}
\end{figure*}

\subsubsection{Body Shape}
Figure~\ref{fig:exp3_bodyMeshes} shows some more results of 3D mesh
reconstruction using posterior samples obtained by our informed
sampling \MIXLMH.

\begin{figure*}[t]
\begin{center}
\centerline{\includegraphics[width=0.75\textwidth]{figures/supplementary/bodyMeshResults.pdf}}
\mycaption{Qualitative results for the body shape experiment}
  {Shown is the 3D mesh reconstruction results with first 1000 samples obtained
  using the \MIXLMH informed sampling method. (blue indicates small
  values and red indicates high values)}
\label{fig:exp3_bodyMeshes}
\end{center}
\end{figure*}

\clearpage



\section{Additional Results on the Face Problem with CMP}

Figure~\ref{fig:shading-qualitative-multiple-subjects-supp} shows inference results for reflectance maps, normal maps and lights for randomly chosen test images, and Fig.~\ref{fig:shading-qualitative-same-subject-supp} shows reflectance estimation results on multiple images of the same subject produced under different illumination conditions. CMP is able to produce estimates that are closer to the groundtruth across different subjects and illumination conditions.

\begin{figure*}[h]
  \begin{center}
  \centerline{\includegraphics[width=1.0\columnwidth]{figures/face_cmp_visual_results_supp.pdf}}
  \vspace{-1.2cm}
  \end{center}
	\mycaption{A visual comparison of inference results}{(a)~Observed images. (b)~Inferred reflectance maps. \textit{GT} is the photometric stereo groundtruth, \textit{BU} is the Biswas \etal (2009) reflectance estimate and \textit{Forest} is the consensus prediction. (c)~The variance of the inferred reflectance estimate produced by \MTD (normalized across rows).(d)~Visualization of inferred light directions. (e)~Inferred normal maps.}
	\label{fig:shading-qualitative-multiple-subjects-supp}
\end{figure*}


\begin{figure*}[h]
	\centering
	\setlength\fboxsep{0.2mm}
	\setlength\fboxrule{0pt}
	\begin{tikzpicture}

		\matrix at (0, 0) [matrix of nodes, nodes={anchor=east}, column sep=-0.05cm, row sep=-0.2cm]
		{
			\fbox{\includegraphics[width=1cm]{figures/sample_3_4_X.png}} &
			\fbox{\includegraphics[width=1cm]{figures/sample_3_4_GT.png}} &
			\fbox{\includegraphics[width=1cm]{figures/sample_3_4_BISWAS.png}}  &
			\fbox{\includegraphics[width=1cm]{figures/sample_3_4_VMP.png}}  &
			\fbox{\includegraphics[width=1cm]{figures/sample_3_4_FOREST.png}}  &
			\fbox{\includegraphics[width=1cm]{figures/sample_3_4_CMP.png}}  &
			\fbox{\includegraphics[width=1cm]{figures/sample_3_4_CMPVAR.png}}
			 \\

			\fbox{\includegraphics[width=1cm]{figures/sample_3_5_X.png}} &
			\fbox{\includegraphics[width=1cm]{figures/sample_3_5_GT.png}} &
			\fbox{\includegraphics[width=1cm]{figures/sample_3_5_BISWAS.png}}  &
			\fbox{\includegraphics[width=1cm]{figures/sample_3_5_VMP.png}}  &
			\fbox{\includegraphics[width=1cm]{figures/sample_3_5_FOREST.png}}  &
			\fbox{\includegraphics[width=1cm]{figures/sample_3_5_CMP.png}}  &
			\fbox{\includegraphics[width=1cm]{figures/sample_3_5_CMPVAR.png}}
			 \\

			\fbox{\includegraphics[width=1cm]{figures/sample_3_6_X.png}} &
			\fbox{\includegraphics[width=1cm]{figures/sample_3_6_GT.png}} &
			\fbox{\includegraphics[width=1cm]{figures/sample_3_6_BISWAS.png}}  &
			\fbox{\includegraphics[width=1cm]{figures/sample_3_6_VMP.png}}  &
			\fbox{\includegraphics[width=1cm]{figures/sample_3_6_FOREST.png}}  &
			\fbox{\includegraphics[width=1cm]{figures/sample_3_6_CMP.png}}  &
			\fbox{\includegraphics[width=1cm]{figures/sample_3_6_CMPVAR.png}}
			 \\
	     };

       \node at (-3.85, -2.0) {\small Observed};
       \node at (-2.55, -2.0) {\small `GT'};
       \node at (-1.27, -2.0) {\small BU};
       \node at (0.0, -2.0) {\small MP};
       \node at (1.27, -2.0) {\small Forest};
       \node at (2.55, -2.0) {\small \textbf{CMP}};
       \node at (3.85, -2.0) {\small Variance};

	\end{tikzpicture}
	\mycaption{Robustness to varying illumination}{Reflectance estimation on a subject images with varying illumination. Left to right: observed image, photometric stereo estimate (GT)
  which is used as a proxy for groundtruth, bottom-up estimate of \cite{Biswas2009}, VMP result, consensus forest estimate, CMP mean, and CMP variance.}
	\label{fig:shading-qualitative-same-subject-supp}
\end{figure*}

\clearpage

\section{Additional Material for Learning Sparse High Dimensional Filters}
\label{sec:appendix-bnn}

This part of supplementary material contains a more detailed overview of the permutohedral
lattice convolution in Section~\ref{sec:permconv}, more experiments in
Section~\ref{sec:addexps} and additional results with protocols for
the experiments presented in Chapter~\ref{chap:bnn} in Section~\ref{sec:addresults}.

\vspace{-0.2cm}
\subsection{General Permutohedral Convolutions}
\label{sec:permconv}

A core technical contribution of this work is the generalization of the Gaussian permutohedral lattice
convolution proposed in~\cite{adams2010fast} to the full non-separable case with the
ability to perform back-propagation. Although, conceptually, there are minor
differences between Gaussian and general parameterized filters, there are non-trivial practical
differences in terms of the algorithmic implementation. The Gauss filters belong to
the separable class and can thus be decomposed into multiple
sequential one dimensional convolutions. We are interested in the general filter
convolutions, which can not be decomposed. Thus, performing a general permutohedral
convolution at a lattice point requires the computation of the inner product with the
neighboring elements in all the directions in the high-dimensional space.

Here, we give more details of the implementation differences of separable
and non-separable filters. In the following, we will explain the scalar case first.
Recall, that the forward pass of general permutohedral convolution
involves 3 steps: \textit{splatting}, \textit{convolving} and \textit{slicing}.
We follow the same splatting and slicing strategies as in~\cite{adams2010fast}
since these operations do not depend on the filter kernel. The main difference
between our work and the existing implementation of~\cite{adams2010fast} is
the way that the convolution operation is executed. This proceeds by constructing
a \emph{blur neighbor} matrix $K$ that stores for every lattice point all
values of the lattice neighbors that are needed to compute the filter output.

\begin{figure}[t!]
  \centering
    \includegraphics[width=0.6\columnwidth]{figures/supplementary/lattice_construction}
  \mycaption{Illustration of 1D permutohedral lattice construction}
  {A $4\times 4$ $(x,y)$ grid lattice is projected onto the plane defined by the normal
  vector $(1,1)^{\top}$. This grid has $s+1=4$ and $d=2$ $(s+1)^{d}=4^2=16$ elements.
  In the projection, all points of the same color are projected onto the same points in the plane.
  The number of elements of the projected lattice is $t=(s+1)^d-s^d=4^2-3^2=7$, that is
  the $(4\times 4)$ grid minus the size of lattice that is $1$ smaller at each size, in this
  case a $(3\times 3)$ lattice (the upper right $(3\times 3)$ elements).
  }
\label{fig:latticeconstruction}
\end{figure}

The blur neighbor matrix is constructed by traversing through all the populated
lattice points and their neighboring elements.
% For efficiency, we do this matrix construction recursively with shared computations
% since $n^{th}$ neighbourhood elements are $1^{st}$ neighborhood elements of $n-1^{th}$ neighbourhood elements. \pg{do not understand}
This is done recursively to share computations. For any lattice point, the neighbors that are
$n$ hops away are the direct neighbors of the points that are $n-1$ hops away.
The size of a $d$ dimensional spatial filter with width $s+1$ is $(s+1)^{d}$ (\eg, a
$3\times 3$ filter, $s=2$ in $d=2$ has $3^2=9$ elements) and this size grows
exponentially in the number of dimensions $d$. The permutohedral lattice is constructed by
projecting a regular grid onto the plane spanned by the $d$ dimensional normal vector ${(1,\ldots,1)}^{\top}$. See
Fig.~\ref{fig:latticeconstruction} for an illustration of the 1D lattice construction.
Many corners of a grid filter are projected onto the same point, in total $t = {(s+1)}^{d} -
s^{d}$ elements remain in the permutohedral filter with $s$ neighborhood in $d-1$ dimensions.
If the lattice has $m$ populated elements, the
matrix $K$ has size $t\times m$. Note that, since the input signal is typically
sparse, only a few lattice corners are being populated in the \textit{slicing} step.
We use a hash-table to keep track of these points and traverse only through
the populated lattice points for this neighborhood matrix construction.

Once the blur neighbor matrix $K$ is constructed, we can perform the convolution
by the matrix vector multiplication
\begin{equation}
\ell' = BK,
\label{eq:conv}
\end{equation}
where $B$ is the $1 \times t$ filter kernel (whose values we will learn) and $\ell'\in\mathbb{R}^{1\times m}$
is the result of the filtering at the $m$ lattice points. In practice, we found that the
matrix $K$ is sometimes too large to fit into GPU memory and we divided the matrix $K$
into smaller pieces to compute Eq.~\ref{eq:conv} sequentially.

In the general multi-dimensional case, the signal $\ell$ is of $c$ dimensions. Then
the kernel $B$ is of size $c \times t$ and $K$ stores the $c$ dimensional vectors
accordingly. When the input and output points are different, we slice only the
input points and splat only at the output points.


\subsection{Additional Experiments}
\label{sec:addexps}
In this section, we discuss more use-cases for the learned bilateral filters, one
use-case of BNNs and two single filter applications for image and 3D mesh denoising.

\subsubsection{Recognition of subsampled MNIST}\label{sec:app_mnist}

One of the strengths of the proposed filter convolution is that it does not
require the input to lie on a regular grid. The only requirement is to define a distance
between features of the input signal.
We highlight this feature with the following experiment using the
classical MNIST ten class classification problem~\cite{lecun1998mnist}. We sample a
sparse set of $N$ points $(x,y)\in [0,1]\times [0,1]$
uniformly at random in the input image, use their interpolated values
as signal and the \emph{continuous} $(x,y)$ positions as features. This mimics
sub-sampling of a high-dimensional signal. To compare against a spatial convolution,
we interpolate the sparse set of values at the grid positions.

We take a reference implementation of LeNet~\cite{lecun1998gradient} that
is part of the Caffe project~\cite{jia2014caffe} and compare it
against the same architecture but replacing the first convolutional
layer with a bilateral convolution layer (BCL). The filter size
and numbers are adjusted to get a comparable number of parameters
($5\times 5$ for LeNet, $2$-neighborhood for BCL).

The results are shown in Table~\ref{tab:all-results}. We see that training
on the original MNIST data (column Original, LeNet vs. BNN) leads to a slight
decrease in performance of the BNN (99.03\%) compared to LeNet
(99.19\%). The BNN can be trained and evaluated on sparse
signals, and we resample the image as described above for $N=$ 100\%, 60\% and
20\% of the total number of pixels. The methods are also evaluated
on test images that are subsampled in the same way. Note that we can
train and test with different subsampling rates. We introduce an additional
bilinear interpolation layer for the LeNet architecture to train on the same
data. In essence, both models perform a spatial interpolation and thus we
expect them to yield a similar classification accuracy. Once the data is of
higher dimensions, the permutohedral convolution will be faster due to hashing
the sparse input points, as well as less memory demanding in comparison to
naive application of a spatial convolution with interpolated values.

\begin{table}[t]
  \begin{center}
    \footnotesize
    \centering
    \begin{tabular}[t]{lllll}
      \toprule
              &     & \multicolumn{3}{c}{Test Subsampling} \\
       Method  & Original & 100\% & 60\% & 20\%\\
      \midrule
       LeNet &  \textbf{0.9919} & 0.9660 & 0.9348 & \textbf{0.6434} \\
       BNN &  0.9903 & \textbf{0.9844} & \textbf{0.9534} & 0.5767 \\
      \hline
       LeNet 100\% & 0.9856 & 0.9809 & 0.9678 & \textbf{0.7386} \\
       BNN 100\% & \textbf{0.9900} & \textbf{0.9863} & \textbf{0.9699} & 0.6910 \\
      \hline
       LeNet 60\% & 0.9848 & 0.9821 & 0.9740 & 0.8151 \\
       BNN 60\% & \textbf{0.9885} & \textbf{0.9864} & \textbf{0.9771} & \textbf{0.8214}\\
      \hline
       LeNet 20\% & \textbf{0.9763} & \textbf{0.9754} & 0.9695 & 0.8928 \\
       BNN 20\% & 0.9728 & 0.9735 & \textbf{0.9701} & \textbf{0.9042}\\
      \bottomrule
    \end{tabular}
  \end{center}
\vspace{-.2cm}
\caption{Classification accuracy on MNIST. We compare the
    LeNet~\cite{lecun1998gradient} implementation that is part of
    Caffe~\cite{jia2014caffe} to the network with the first layer
    replaced by a bilateral convolution layer (BCL). Both are trained
    on the original image resolution (first two rows). Three more BNN
    and CNN models are trained with randomly subsampled images (100\%,
    60\% and 20\% of the pixels). An additional bilinear interpolation
    layer samples the input signal on a spatial grid for the CNN model.
  }
  \label{tab:all-results}
\vspace{-.5cm}
\end{table}

\subsubsection{Image Denoising}

The main application that inspired the development of the bilateral
filtering operation is image denoising~\cite{aurich1995non}, there
using a single Gaussian kernel. Our development allows to learn this
kernel function from data and we explore how to improve using a \emph{single}
but more general bilateral filter.

We use the Berkeley segmentation dataset
(BSDS500)~\cite{arbelaezi2011bsds500} as a test bed. The color
images in the dataset are converted to gray-scale,
and corrupted with Gaussian noise with a standard deviation of
$\frac {25} {255}$.

We compare the performance of four different filter models on a
denoising task.
The first baseline model (`Spatial' in Table \ref{tab:denoising}, $25$
weights) uses a single spatial filter with a kernel size of
$5$ and predicts the scalar gray-scale value at the center pixel. The next model
(`Gauss Bilateral') applies a bilateral \emph{Gaussian}
filter to the noisy input, using position and intensity features $\f=(x,y,v)^\top$.
The third setup (`Learned Bilateral', $65$ weights)
takes a Gauss kernel as initialization and
fits all filter weights on the train set to minimize the
mean squared error with respect to the clean images.
We run a combination
of spatial and permutohedral convolutions on spatial and bilateral
features (`Spatial + Bilateral (Learned)') to check for a complementary
performance of the two convolutions.

\label{sec:exp:denoising}
\begin{table}[!h]
\begin{center}
  \footnotesize
  \begin{tabular}[t]{lr}
    \toprule
    Method & PSNR \\
    \midrule
    Noisy Input & $20.17$ \\
    Spatial & $26.27$ \\
    Gauss Bilateral & $26.51$ \\
    Learned Bilateral & $26.58$ \\
    Spatial + Bilateral (Learned) & \textbf{$26.65$} \\
    \bottomrule
  \end{tabular}
\end{center}
\vspace{-0.5em}
\caption{PSNR results of a denoising task using the BSDS500
  dataset~\cite{arbelaezi2011bsds500}}
\vspace{-0.5em}
\label{tab:denoising}
\end{table}
\vspace{-0.2em}

The PSNR scores evaluated on full images of the test set are
shown in Table \ref{tab:denoising}. We find that an untrained bilateral
filter already performs better than a trained spatial convolution
($26.27$ to $26.51$). A learned convolution further improve the
performance slightly. We chose this simple one-kernel setup to
validate an advantage of the generalized bilateral filter. A competitive
denoising system would employ RGB color information and also
needs to be properly adjusted in network size. Multi-layer perceptrons
have obtained state-of-the-art denoising results~\cite{burger12cvpr}
and the permutohedral lattice layer can readily be used in such an
architecture, which is intended future work.

\subsection{Additional results}
\label{sec:addresults}

This section contains more qualitative results for the experiments presented in Chapter~\ref{chap:bnn}.

\begin{figure*}[th!]
  \centering
    \includegraphics[width=\columnwidth,trim={5cm 2.5cm 5cm 4.5cm},clip]{figures/supplementary/lattice_viz.pdf}
    \vspace{-0.7cm}
  \mycaption{Visualization of the Permutohedral Lattice}
  {Sample lattice visualizations for different feature spaces. All pixels falling in the same simplex cell are shown with
  the same color. $(x,y)$ features correspond to image pixel positions, and $(r,g,b) \in [0,255]$ correspond
  to the red, green and blue color values.}
\label{fig:latticeviz}
\end{figure*}

\subsubsection{Lattice Visualization}

Figure~\ref{fig:latticeviz} shows sample lattice visualizations for different feature spaces.

\newcolumntype{L}[1]{>{\raggedright\let\newline\\\arraybackslash\hspace{0pt}}b{#1}}
\newcolumntype{C}[1]{>{\centering\let\newline\\\arraybackslash\hspace{0pt}}b{#1}}
\newcolumntype{R}[1]{>{\raggedleft\let\newline\\\arraybackslash\hspace{0pt}}b{#1}}

\subsubsection{Color Upsampling}\label{sec:color_upsampling}
\label{sec:col_upsample_extra}

Some images of the upsampling for the Pascal
VOC12 dataset are shown in Fig.~\ref{fig:Colour_upsample_visuals}. It is
especially the low level image details that are better preserved with
a learned bilateral filter compared to the Gaussian case.

\begin{figure*}[t!]
  \centering
    \subfigure{%
   \raisebox{2.0em}{
    \includegraphics[width=.06\columnwidth]{figures/supplementary/2007_004969.jpg}
   }
  }
  \subfigure{%
    \includegraphics[width=.17\columnwidth]{figures/supplementary/2007_004969_gray.pdf}
  }
  \subfigure{%
    \includegraphics[width=.17\columnwidth]{figures/supplementary/2007_004969_gt.pdf}
  }
  \subfigure{%
    \includegraphics[width=.17\columnwidth]{figures/supplementary/2007_004969_bicubic.pdf}
  }
  \subfigure{%
    \includegraphics[width=.17\columnwidth]{figures/supplementary/2007_004969_gauss.pdf}
  }
  \subfigure{%
    \includegraphics[width=.17\columnwidth]{figures/supplementary/2007_004969_learnt.pdf}
  }\\
    \subfigure{%
   \raisebox{2.0em}{
    \includegraphics[width=.06\columnwidth]{figures/supplementary/2007_003106.jpg}
   }
  }
  \subfigure{%
    \includegraphics[width=.17\columnwidth]{figures/supplementary/2007_003106_gray.pdf}
  }
  \subfigure{%
    \includegraphics[width=.17\columnwidth]{figures/supplementary/2007_003106_gt.pdf}
  }
  \subfigure{%
    \includegraphics[width=.17\columnwidth]{figures/supplementary/2007_003106_bicubic.pdf}
  }
  \subfigure{%
    \includegraphics[width=.17\columnwidth]{figures/supplementary/2007_003106_gauss.pdf}
  }
  \subfigure{%
    \includegraphics[width=.17\columnwidth]{figures/supplementary/2007_003106_learnt.pdf}
  }\\
  \setcounter{subfigure}{0}
  \small{
  \subfigure[Inp.]{%
  \raisebox{2.0em}{
    \includegraphics[width=.06\columnwidth]{figures/supplementary/2007_006837.jpg}
   }
  }
  \subfigure[Guidance]{%
    \includegraphics[width=.17\columnwidth]{figures/supplementary/2007_006837_gray.pdf}
  }
   \subfigure[GT]{%
    \includegraphics[width=.17\columnwidth]{figures/supplementary/2007_006837_gt.pdf}
  }
  \subfigure[Bicubic]{%
    \includegraphics[width=.17\columnwidth]{figures/supplementary/2007_006837_bicubic.pdf}
  }
  \subfigure[Gauss-BF]{%
    \includegraphics[width=.17\columnwidth]{figures/supplementary/2007_006837_gauss.pdf}
  }
  \subfigure[Learned-BF]{%
    \includegraphics[width=.17\columnwidth]{figures/supplementary/2007_006837_learnt.pdf}
  }
  }
  \vspace{-0.5cm}
  \mycaption{Color Upsampling}{Color $8\times$ upsampling results
  using different methods, from left to right, (a)~Low-resolution input color image (Inp.),
  (b)~Gray scale guidance image, (c)~Ground-truth color image; Upsampled color images with
  (d)~Bicubic interpolation, (e) Gauss bilateral upsampling and, (f)~Learned bilateral
  updampgling (best viewed on screen).}

\label{fig:Colour_upsample_visuals}
\end{figure*}

\subsubsection{Depth Upsampling}
\label{sec:depth_upsample_extra}

Figure~\ref{fig:depth_upsample_visuals} presents some more qualitative results comparing bicubic interpolation, Gauss
bilateral and learned bilateral upsampling on NYU depth dataset image~\cite{silberman2012indoor}.

\subsubsection{Character Recognition}\label{sec:app_character}

 Figure~\ref{fig:nnrecognition} shows the schematic of different layers
 of the network architecture for LeNet-7~\cite{lecun1998mnist}
 and DeepCNet(5, 50)~\cite{ciresan2012multi,graham2014spatially}. For the BNN variants, the first layer filters are replaced
 with learned bilateral filters and are learned end-to-end.

\subsubsection{Semantic Segmentation}\label{sec:app_semantic_segmentation}
\label{sec:semantic_bnn_extra}

Some more visual results for semantic segmentation are shown in Figure~\ref{fig:semantic_visuals}.
These include the underlying DeepLab CNN\cite{chen2014semantic} result (DeepLab),
the 2 step mean-field result with Gaussian edge potentials (+2stepMF-GaussCRF)
and also corresponding results with learned edge potentials (+2stepMF-LearnedCRF).
In general, we observe that mean-field in learned CRF leads to slightly dilated
classification regions in comparison to using Gaussian CRF thereby filling-in the
false negative pixels and also correcting some mis-classified regions.

\begin{figure*}[t!]
  \centering
    \subfigure{%
   \raisebox{2.0em}{
    \includegraphics[width=.06\columnwidth]{figures/supplementary/2bicubic}
   }
  }
  \subfigure{%
    \includegraphics[width=.17\columnwidth]{figures/supplementary/2given_image}
  }
  \subfigure{%
    \includegraphics[width=.17\columnwidth]{figures/supplementary/2ground_truth}
  }
  \subfigure{%
    \includegraphics[width=.17\columnwidth]{figures/supplementary/2bicubic}
  }
  \subfigure{%
    \includegraphics[width=.17\columnwidth]{figures/supplementary/2gauss}
  }
  \subfigure{%
    \includegraphics[width=.17\columnwidth]{figures/supplementary/2learnt}
  }\\
    \subfigure{%
   \raisebox{2.0em}{
    \includegraphics[width=.06\columnwidth]{figures/supplementary/32bicubic}
   }
  }
  \subfigure{%
    \includegraphics[width=.17\columnwidth]{figures/supplementary/32given_image}
  }
  \subfigure{%
    \includegraphics[width=.17\columnwidth]{figures/supplementary/32ground_truth}
  }
  \subfigure{%
    \includegraphics[width=.17\columnwidth]{figures/supplementary/32bicubic}
  }
  \subfigure{%
    \includegraphics[width=.17\columnwidth]{figures/supplementary/32gauss}
  }
  \subfigure{%
    \includegraphics[width=.17\columnwidth]{figures/supplementary/32learnt}
  }\\
  \setcounter{subfigure}{0}
  \small{
  \subfigure[Inp.]{%
  \raisebox{2.0em}{
    \includegraphics[width=.06\columnwidth]{figures/supplementary/41bicubic}
   }
  }
  \subfigure[Guidance]{%
    \includegraphics[width=.17\columnwidth]{figures/supplementary/41given_image}
  }
   \subfigure[GT]{%
    \includegraphics[width=.17\columnwidth]{figures/supplementary/41ground_truth}
  }
  \subfigure[Bicubic]{%
    \includegraphics[width=.17\columnwidth]{figures/supplementary/41bicubic}
  }
  \subfigure[Gauss-BF]{%
    \includegraphics[width=.17\columnwidth]{figures/supplementary/41gauss}
  }
  \subfigure[Learned-BF]{%
    \includegraphics[width=.17\columnwidth]{figures/supplementary/41learnt}
  }
  }
  \mycaption{Depth Upsampling}{Depth $8\times$ upsampling results
  using different upsampling strategies, from left to right,
  (a)~Low-resolution input depth image (Inp.),
  (b)~High-resolution guidance image, (c)~Ground-truth depth; Upsampled depth images with
  (d)~Bicubic interpolation, (e) Gauss bilateral upsampling and, (f)~Learned bilateral
  updampgling (best viewed on screen).}

\label{fig:depth_upsample_visuals}
\end{figure*}

\subsubsection{Material Segmentation}\label{sec:app_material_segmentation}
\label{sec:material_bnn_extra}

In Fig.~\ref{fig:material_visuals-app2}, we present visual results comparing 2 step
mean-field inference with Gaussian and learned pairwise CRF potentials. In
general, we observe that the pixels belonging to dominant classes in the
training data are being more accurately classified with learned CRF. This leads to
a significant improvements in overall pixel accuracy. This also results
in a slight decrease of the accuracy from less frequent class pixels thereby
slightly reducing the average class accuracy with learning. We attribute this
to the type of annotation that is available for this dataset, which is not
for the entire image but for some segments in the image. We have very few
images of the infrequent classes to combat this behaviour during training.

\subsubsection{Experiment Protocols}
\label{sec:protocols}

Table~\ref{tbl:parameters} shows experiment protocols of different experiments.

 \begin{figure*}[t!]
  \centering
  \subfigure[LeNet-7]{
    \includegraphics[width=0.7\columnwidth]{figures/supplementary/lenet_cnn_network}
    }\\
    \subfigure[DeepCNet]{
    \includegraphics[width=\columnwidth]{figures/supplementary/deepcnet_cnn_network}
    }
  \mycaption{CNNs for Character Recognition}
  {Schematic of (top) LeNet-7~\cite{lecun1998mnist} and (bottom) DeepCNet(5,50)~\cite{ciresan2012multi,graham2014spatially} architectures used in Assamese
  character recognition experiments.}
\label{fig:nnrecognition}
\end{figure*}

\definecolor{voc_1}{RGB}{0, 0, 0}
\definecolor{voc_2}{RGB}{128, 0, 0}
\definecolor{voc_3}{RGB}{0, 128, 0}
\definecolor{voc_4}{RGB}{128, 128, 0}
\definecolor{voc_5}{RGB}{0, 0, 128}
\definecolor{voc_6}{RGB}{128, 0, 128}
\definecolor{voc_7}{RGB}{0, 128, 128}
\definecolor{voc_8}{RGB}{128, 128, 128}
\definecolor{voc_9}{RGB}{64, 0, 0}
\definecolor{voc_10}{RGB}{192, 0, 0}
\definecolor{voc_11}{RGB}{64, 128, 0}
\definecolor{voc_12}{RGB}{192, 128, 0}
\definecolor{voc_13}{RGB}{64, 0, 128}
\definecolor{voc_14}{RGB}{192, 0, 128}
\definecolor{voc_15}{RGB}{64, 128, 128}
\definecolor{voc_16}{RGB}{192, 128, 128}
\definecolor{voc_17}{RGB}{0, 64, 0}
\definecolor{voc_18}{RGB}{128, 64, 0}
\definecolor{voc_19}{RGB}{0, 192, 0}
\definecolor{voc_20}{RGB}{128, 192, 0}
\definecolor{voc_21}{RGB}{0, 64, 128}
\definecolor{voc_22}{RGB}{128, 64, 128}

\begin{figure*}[t]
  \centering
  \small{
  \fcolorbox{white}{voc_1}{\rule{0pt}{6pt}\rule{6pt}{0pt}} Background~~
  \fcolorbox{white}{voc_2}{\rule{0pt}{6pt}\rule{6pt}{0pt}} Aeroplane~~
  \fcolorbox{white}{voc_3}{\rule{0pt}{6pt}\rule{6pt}{0pt}} Bicycle~~
  \fcolorbox{white}{voc_4}{\rule{0pt}{6pt}\rule{6pt}{0pt}} Bird~~
  \fcolorbox{white}{voc_5}{\rule{0pt}{6pt}\rule{6pt}{0pt}} Boat~~
  \fcolorbox{white}{voc_6}{\rule{0pt}{6pt}\rule{6pt}{0pt}} Bottle~~
  \fcolorbox{white}{voc_7}{\rule{0pt}{6pt}\rule{6pt}{0pt}} Bus~~
  \fcolorbox{white}{voc_8}{\rule{0pt}{6pt}\rule{6pt}{0pt}} Car~~ \\
  \fcolorbox{white}{voc_9}{\rule{0pt}{6pt}\rule{6pt}{0pt}} Cat~~
  \fcolorbox{white}{voc_10}{\rule{0pt}{6pt}\rule{6pt}{0pt}} Chair~~
  \fcolorbox{white}{voc_11}{\rule{0pt}{6pt}\rule{6pt}{0pt}} Cow~~
  \fcolorbox{white}{voc_12}{\rule{0pt}{6pt}\rule{6pt}{0pt}} Dining Table~~
  \fcolorbox{white}{voc_13}{\rule{0pt}{6pt}\rule{6pt}{0pt}} Dog~~
  \fcolorbox{white}{voc_14}{\rule{0pt}{6pt}\rule{6pt}{0pt}} Horse~~
  \fcolorbox{white}{voc_15}{\rule{0pt}{6pt}\rule{6pt}{0pt}} Motorbike~~
  \fcolorbox{white}{voc_16}{\rule{0pt}{6pt}\rule{6pt}{0pt}} Person~~ \\
  \fcolorbox{white}{voc_17}{\rule{0pt}{6pt}\rule{6pt}{0pt}} Potted Plant~~
  \fcolorbox{white}{voc_18}{\rule{0pt}{6pt}\rule{6pt}{0pt}} Sheep~~
  \fcolorbox{white}{voc_19}{\rule{0pt}{6pt}\rule{6pt}{0pt}} Sofa~~
  \fcolorbox{white}{voc_20}{\rule{0pt}{6pt}\rule{6pt}{0pt}} Train~~
  \fcolorbox{white}{voc_21}{\rule{0pt}{6pt}\rule{6pt}{0pt}} TV monitor~~ \\
  }
  \subfigure{%
    \includegraphics[width=.18\columnwidth]{figures/supplementary/2007_001423_given.jpg}
  }
  \subfigure{%
    \includegraphics[width=.18\columnwidth]{figures/supplementary/2007_001423_gt.png}
  }
  \subfigure{%
    \includegraphics[width=.18\columnwidth]{figures/supplementary/2007_001423_cnn.png}
  }
  \subfigure{%
    \includegraphics[width=.18\columnwidth]{figures/supplementary/2007_001423_gauss.png}
  }
  \subfigure{%
    \includegraphics[width=.18\columnwidth]{figures/supplementary/2007_001423_learnt.png}
  }\\
  \subfigure{%
    \includegraphics[width=.18\columnwidth]{figures/supplementary/2007_001430_given.jpg}
  }
  \subfigure{%
    \includegraphics[width=.18\columnwidth]{figures/supplementary/2007_001430_gt.png}
  }
  \subfigure{%
    \includegraphics[width=.18\columnwidth]{figures/supplementary/2007_001430_cnn.png}
  }
  \subfigure{%
    \includegraphics[width=.18\columnwidth]{figures/supplementary/2007_001430_gauss.png}
  }
  \subfigure{%
    \includegraphics[width=.18\columnwidth]{figures/supplementary/2007_001430_learnt.png}
  }\\
    \subfigure{%
    \includegraphics[width=.18\columnwidth]{figures/supplementary/2007_007996_given.jpg}
  }
  \subfigure{%
    \includegraphics[width=.18\columnwidth]{figures/supplementary/2007_007996_gt.png}
  }
  \subfigure{%
    \includegraphics[width=.18\columnwidth]{figures/supplementary/2007_007996_cnn.png}
  }
  \subfigure{%
    \includegraphics[width=.18\columnwidth]{figures/supplementary/2007_007996_gauss.png}
  }
  \subfigure{%
    \includegraphics[width=.18\columnwidth]{figures/supplementary/2007_007996_learnt.png}
  }\\
   \subfigure{%
    \includegraphics[width=.18\columnwidth]{figures/supplementary/2010_002682_given.jpg}
  }
  \subfigure{%
    \includegraphics[width=.18\columnwidth]{figures/supplementary/2010_002682_gt.png}
  }
  \subfigure{%
    \includegraphics[width=.18\columnwidth]{figures/supplementary/2010_002682_cnn.png}
  }
  \subfigure{%
    \includegraphics[width=.18\columnwidth]{figures/supplementary/2010_002682_gauss.png}
  }
  \subfigure{%
    \includegraphics[width=.18\columnwidth]{figures/supplementary/2010_002682_learnt.png}
  }\\
     \subfigure{%
    \includegraphics[width=.18\columnwidth]{figures/supplementary/2010_004789_given.jpg}
  }
  \subfigure{%
    \includegraphics[width=.18\columnwidth]{figures/supplementary/2010_004789_gt.png}
  }
  \subfigure{%
    \includegraphics[width=.18\columnwidth]{figures/supplementary/2010_004789_cnn.png}
  }
  \subfigure{%
    \includegraphics[width=.18\columnwidth]{figures/supplementary/2010_004789_gauss.png}
  }
  \subfigure{%
    \includegraphics[width=.18\columnwidth]{figures/supplementary/2010_004789_learnt.png}
  }\\
       \subfigure{%
    \includegraphics[width=.18\columnwidth]{figures/supplementary/2007_001311_given.jpg}
  }
  \subfigure{%
    \includegraphics[width=.18\columnwidth]{figures/supplementary/2007_001311_gt.png}
  }
  \subfigure{%
    \includegraphics[width=.18\columnwidth]{figures/supplementary/2007_001311_cnn.png}
  }
  \subfigure{%
    \includegraphics[width=.18\columnwidth]{figures/supplementary/2007_001311_gauss.png}
  }
  \subfigure{%
    \includegraphics[width=.18\columnwidth]{figures/supplementary/2007_001311_learnt.png}
  }\\
  \setcounter{subfigure}{0}
  \subfigure[Input]{%
    \includegraphics[width=.18\columnwidth]{figures/supplementary/2010_003531_given.jpg}
  }
  \subfigure[Ground Truth]{%
    \includegraphics[width=.18\columnwidth]{figures/supplementary/2010_003531_gt.png}
  }
  \subfigure[DeepLab]{%
    \includegraphics[width=.18\columnwidth]{figures/supplementary/2010_003531_cnn.png}
  }
  \subfigure[+GaussCRF]{%
    \includegraphics[width=.18\columnwidth]{figures/supplementary/2010_003531_gauss.png}
  }
  \subfigure[+LearnedCRF]{%
    \includegraphics[width=.18\columnwidth]{figures/supplementary/2010_003531_learnt.png}
  }
  \vspace{-0.3cm}
  \mycaption{Semantic Segmentation}{Example results of semantic segmentation.
  (c)~depicts the unary results before application of MF, (d)~after two steps of MF with Gaussian edge CRF potentials, (e)~after
  two steps of MF with learned edge CRF potentials.}
    \label{fig:semantic_visuals}
\end{figure*}


\definecolor{minc_1}{HTML}{771111}
\definecolor{minc_2}{HTML}{CAC690}
\definecolor{minc_3}{HTML}{EEEEEE}
\definecolor{minc_4}{HTML}{7C8FA6}
\definecolor{minc_5}{HTML}{597D31}
\definecolor{minc_6}{HTML}{104410}
\definecolor{minc_7}{HTML}{BB819C}
\definecolor{minc_8}{HTML}{D0CE48}
\definecolor{minc_9}{HTML}{622745}
\definecolor{minc_10}{HTML}{666666}
\definecolor{minc_11}{HTML}{D54A31}
\definecolor{minc_12}{HTML}{101044}
\definecolor{minc_13}{HTML}{444126}
\definecolor{minc_14}{HTML}{75D646}
\definecolor{minc_15}{HTML}{DD4348}
\definecolor{minc_16}{HTML}{5C8577}
\definecolor{minc_17}{HTML}{C78472}
\definecolor{minc_18}{HTML}{75D6D0}
\definecolor{minc_19}{HTML}{5B4586}
\definecolor{minc_20}{HTML}{C04393}
\definecolor{minc_21}{HTML}{D69948}
\definecolor{minc_22}{HTML}{7370D8}
\definecolor{minc_23}{HTML}{7A3622}
\definecolor{minc_24}{HTML}{000000}

\begin{figure*}[t]
  \centering
  \small{
  \fcolorbox{white}{minc_1}{\rule{0pt}{6pt}\rule{6pt}{0pt}} Brick~~
  \fcolorbox{white}{minc_2}{\rule{0pt}{6pt}\rule{6pt}{0pt}} Carpet~~
  \fcolorbox{white}{minc_3}{\rule{0pt}{6pt}\rule{6pt}{0pt}} Ceramic~~
  \fcolorbox{white}{minc_4}{\rule{0pt}{6pt}\rule{6pt}{0pt}} Fabric~~
  \fcolorbox{white}{minc_5}{\rule{0pt}{6pt}\rule{6pt}{0pt}} Foliage~~
  \fcolorbox{white}{minc_6}{\rule{0pt}{6pt}\rule{6pt}{0pt}} Food~~
  \fcolorbox{white}{minc_7}{\rule{0pt}{6pt}\rule{6pt}{0pt}} Glass~~
  \fcolorbox{white}{minc_8}{\rule{0pt}{6pt}\rule{6pt}{0pt}} Hair~~ \\
  \fcolorbox{white}{minc_9}{\rule{0pt}{6pt}\rule{6pt}{0pt}} Leather~~
  \fcolorbox{white}{minc_10}{\rule{0pt}{6pt}\rule{6pt}{0pt}} Metal~~
  \fcolorbox{white}{minc_11}{\rule{0pt}{6pt}\rule{6pt}{0pt}} Mirror~~
  \fcolorbox{white}{minc_12}{\rule{0pt}{6pt}\rule{6pt}{0pt}} Other~~
  \fcolorbox{white}{minc_13}{\rule{0pt}{6pt}\rule{6pt}{0pt}} Painted~~
  \fcolorbox{white}{minc_14}{\rule{0pt}{6pt}\rule{6pt}{0pt}} Paper~~
  \fcolorbox{white}{minc_15}{\rule{0pt}{6pt}\rule{6pt}{0pt}} Plastic~~\\
  \fcolorbox{white}{minc_16}{\rule{0pt}{6pt}\rule{6pt}{0pt}} Polished Stone~~
  \fcolorbox{white}{minc_17}{\rule{0pt}{6pt}\rule{6pt}{0pt}} Skin~~
  \fcolorbox{white}{minc_18}{\rule{0pt}{6pt}\rule{6pt}{0pt}} Sky~~
  \fcolorbox{white}{minc_19}{\rule{0pt}{6pt}\rule{6pt}{0pt}} Stone~~
  \fcolorbox{white}{minc_20}{\rule{0pt}{6pt}\rule{6pt}{0pt}} Tile~~
  \fcolorbox{white}{minc_21}{\rule{0pt}{6pt}\rule{6pt}{0pt}} Wallpaper~~
  \fcolorbox{white}{minc_22}{\rule{0pt}{6pt}\rule{6pt}{0pt}} Water~~
  \fcolorbox{white}{minc_23}{\rule{0pt}{6pt}\rule{6pt}{0pt}} Wood~~ \\
  }
  \subfigure{%
    \includegraphics[width=.18\columnwidth]{figures/supplementary/000010868_given.jpg}
  }
  \subfigure{%
    \includegraphics[width=.18\columnwidth]{figures/supplementary/000010868_gt.png}
  }
  \subfigure{%
    \includegraphics[width=.18\columnwidth]{figures/supplementary/000010868_cnn.png}
  }
  \subfigure{%
    \includegraphics[width=.18\columnwidth]{figures/supplementary/000010868_gauss.png}
  }
  \subfigure{%
    \includegraphics[width=.18\columnwidth]{figures/supplementary/000010868_learnt.png}
  }\\[-2ex]
  \subfigure{%
    \includegraphics[width=.18\columnwidth]{figures/supplementary/000006011_given.jpg}
  }
  \subfigure{%
    \includegraphics[width=.18\columnwidth]{figures/supplementary/000006011_gt.png}
  }
  \subfigure{%
    \includegraphics[width=.18\columnwidth]{figures/supplementary/000006011_cnn.png}
  }
  \subfigure{%
    \includegraphics[width=.18\columnwidth]{figures/supplementary/000006011_gauss.png}
  }
  \subfigure{%
    \includegraphics[width=.18\columnwidth]{figures/supplementary/000006011_learnt.png}
  }\\[-2ex]
    \subfigure{%
    \includegraphics[width=.18\columnwidth]{figures/supplementary/000008553_given.jpg}
  }
  \subfigure{%
    \includegraphics[width=.18\columnwidth]{figures/supplementary/000008553_gt.png}
  }
  \subfigure{%
    \includegraphics[width=.18\columnwidth]{figures/supplementary/000008553_cnn.png}
  }
  \subfigure{%
    \includegraphics[width=.18\columnwidth]{figures/supplementary/000008553_gauss.png}
  }
  \subfigure{%
    \includegraphics[width=.18\columnwidth]{figures/supplementary/000008553_learnt.png}
  }\\[-2ex]
   \subfigure{%
    \includegraphics[width=.18\columnwidth]{figures/supplementary/000009188_given.jpg}
  }
  \subfigure{%
    \includegraphics[width=.18\columnwidth]{figures/supplementary/000009188_gt.png}
  }
  \subfigure{%
    \includegraphics[width=.18\columnwidth]{figures/supplementary/000009188_cnn.png}
  }
  \subfigure{%
    \includegraphics[width=.18\columnwidth]{figures/supplementary/000009188_gauss.png}
  }
  \subfigure{%
    \includegraphics[width=.18\columnwidth]{figures/supplementary/000009188_learnt.png}
  }\\[-2ex]
  \setcounter{subfigure}{0}
  \subfigure[Input]{%
    \includegraphics[width=.18\columnwidth]{figures/supplementary/000023570_given.jpg}
  }
  \subfigure[Ground Truth]{%
    \includegraphics[width=.18\columnwidth]{figures/supplementary/000023570_gt.png}
  }
  \subfigure[DeepLab]{%
    \includegraphics[width=.18\columnwidth]{figures/supplementary/000023570_cnn.png}
  }
  \subfigure[+GaussCRF]{%
    \includegraphics[width=.18\columnwidth]{figures/supplementary/000023570_gauss.png}
  }
  \subfigure[+LearnedCRF]{%
    \includegraphics[width=.18\columnwidth]{figures/supplementary/000023570_learnt.png}
  }
  \mycaption{Material Segmentation}{Example results of material segmentation.
  (c)~depicts the unary results before application of MF, (d)~after two steps of MF with Gaussian edge CRF potentials, (e)~after two steps of MF with learned edge CRF potentials.}
    \label{fig:material_visuals-app2}
\end{figure*}


\begin{table*}[h]
\tiny
  \centering
    \begin{tabular}{L{2.3cm} L{2.25cm} C{1.5cm} C{0.7cm} C{0.6cm} C{0.7cm} C{0.7cm} C{0.7cm} C{1.6cm} C{0.6cm} C{0.6cm} C{0.6cm}}
      \toprule
& & & & & \multicolumn{3}{c}{\textbf{Data Statistics}} & \multicolumn{4}{c}{\textbf{Training Protocol}} \\

\textbf{Experiment} & \textbf{Feature Types} & \textbf{Feature Scales} & \textbf{Filter Size} & \textbf{Filter Nbr.} & \textbf{Train}  & \textbf{Val.} & \textbf{Test} & \textbf{Loss Type} & \textbf{LR} & \textbf{Batch} & \textbf{Epochs} \\
      \midrule
      \multicolumn{2}{c}{\textbf{Single Bilateral Filter Applications}} & & & & & & & & & \\
      \textbf{2$\times$ Color Upsampling} & Position$_{1}$, Intensity (3D) & 0.13, 0.17 & 65 & 2 & 10581 & 1449 & 1456 & MSE & 1e-06 & 200 & 94.5\\
      \textbf{4$\times$ Color Upsampling} & Position$_{1}$, Intensity (3D) & 0.06, 0.17 & 65 & 2 & 10581 & 1449 & 1456 & MSE & 1e-06 & 200 & 94.5\\
      \textbf{8$\times$ Color Upsampling} & Position$_{1}$, Intensity (3D) & 0.03, 0.17 & 65 & 2 & 10581 & 1449 & 1456 & MSE & 1e-06 & 200 & 94.5\\
      \textbf{16$\times$ Color Upsampling} & Position$_{1}$, Intensity (3D) & 0.02, 0.17 & 65 & 2 & 10581 & 1449 & 1456 & MSE & 1e-06 & 200 & 94.5\\
      \textbf{Depth Upsampling} & Position$_{1}$, Color (5D) & 0.05, 0.02 & 665 & 2 & 795 & 100 & 654 & MSE & 1e-07 & 50 & 251.6\\
      \textbf{Mesh Denoising} & Isomap (4D) & 46.00 & 63 & 2 & 1000 & 200 & 500 & MSE & 100 & 10 & 100.0 \\
      \midrule
      \multicolumn{2}{c}{\textbf{DenseCRF Applications}} & & & & & & & & &\\
      \multicolumn{2}{l}{\textbf{Semantic Segmentation}} & & & & & & & & &\\
      \textbf{- 1step MF} & Position$_{1}$, Color (5D); Position$_{1}$ (2D) & 0.01, 0.34; 0.34  & 665; 19  & 2; 2 & 10581 & 1449 & 1456 & Logistic & 0.1 & 5 & 1.4 \\
      \textbf{- 2step MF} & Position$_{1}$, Color (5D); Position$_{1}$ (2D) & 0.01, 0.34; 0.34 & 665; 19 & 2; 2 & 10581 & 1449 & 1456 & Logistic & 0.1 & 5 & 1.4 \\
      \textbf{- \textit{loose} 2step MF} & Position$_{1}$, Color (5D); Position$_{1}$ (2D) & 0.01, 0.34; 0.34 & 665; 19 & 2; 2 &10581 & 1449 & 1456 & Logistic & 0.1 & 5 & +1.9  \\ \\
      \multicolumn{2}{l}{\textbf{Material Segmentation}} & & & & & & & & &\\
      \textbf{- 1step MF} & Position$_{2}$, Lab-Color (5D) & 5.00, 0.05, 0.30  & 665 & 2 & 928 & 150 & 1798 & Weighted Logistic & 1e-04 & 24 & 2.6 \\
      \textbf{- 2step MF} & Position$_{2}$, Lab-Color (5D) & 5.00, 0.05, 0.30 & 665 & 2 & 928 & 150 & 1798 & Weighted Logistic & 1e-04 & 12 & +0.7 \\
      \textbf{- \textit{loose} 2step MF} & Position$_{2}$, Lab-Color (5D) & 5.00, 0.05, 0.30 & 665 & 2 & 928 & 150 & 1798 & Weighted Logistic & 1e-04 & 12 & +0.2\\
      \midrule
      \multicolumn{2}{c}{\textbf{Neural Network Applications}} & & & & & & & & &\\
      \textbf{Tiles: CNN-9$\times$9} & - & - & 81 & 4 & 10000 & 1000 & 1000 & Logistic & 0.01 & 100 & 500.0 \\
      \textbf{Tiles: CNN-13$\times$13} & - & - & 169 & 6 & 10000 & 1000 & 1000 & Logistic & 0.01 & 100 & 500.0 \\
      \textbf{Tiles: CNN-17$\times$17} & - & - & 289 & 8 & 10000 & 1000 & 1000 & Logistic & 0.01 & 100 & 500.0 \\
      \textbf{Tiles: CNN-21$\times$21} & - & - & 441 & 10 & 10000 & 1000 & 1000 & Logistic & 0.01 & 100 & 500.0 \\
      \textbf{Tiles: BNN} & Position$_{1}$, Color (5D) & 0.05, 0.04 & 63 & 1 & 10000 & 1000 & 1000 & Logistic & 0.01 & 100 & 30.0 \\
      \textbf{LeNet} & - & - & 25 & 2 & 5490 & 1098 & 1647 & Logistic & 0.1 & 100 & 182.2 \\
      \textbf{Crop-LeNet} & - & - & 25 & 2 & 5490 & 1098 & 1647 & Logistic & 0.1 & 100 & 182.2 \\
      \textbf{BNN-LeNet} & Position$_{2}$ (2D) & 20.00 & 7 & 1 & 5490 & 1098 & 1647 & Logistic & 0.1 & 100 & 182.2 \\
      \textbf{DeepCNet} & - & - & 9 & 1 & 5490 & 1098 & 1647 & Logistic & 0.1 & 100 & 182.2 \\
      \textbf{Crop-DeepCNet} & - & - & 9 & 1 & 5490 & 1098 & 1647 & Logistic & 0.1 & 100 & 182.2 \\
      \textbf{BNN-DeepCNet} & Position$_{2}$ (2D) & 40.00  & 7 & 1 & 5490 & 1098 & 1647 & Logistic & 0.1 & 100 & 182.2 \\
      \bottomrule
      \\
    \end{tabular}
    \mycaption{Experiment Protocols} {Experiment protocols for the different experiments presented in this work. \textbf{Feature Types}:
    Feature spaces used for the bilateral convolutions. Position$_1$ corresponds to un-normalized pixel positions whereas Position$_2$ corresponds
    to pixel positions normalized to $[0,1]$ with respect to the given image. \textbf{Feature Scales}: Cross-validated scales for the features used.
     \textbf{Filter Size}: Number of elements in the filter that is being learned. \textbf{Filter Nbr.}: Half-width of the filter. \textbf{Train},
     \textbf{Val.} and \textbf{Test} corresponds to the number of train, validation and test images used in the experiment. \textbf{Loss Type}: Type
     of loss used for back-propagation. ``MSE'' corresponds to Euclidean mean squared error loss and ``Logistic'' corresponds to multinomial logistic
     loss. ``Weighted Logistic'' is the class-weighted multinomial logistic loss. We weighted the loss with inverse class probability for material
     segmentation task due to the small availability of training data with class imbalance. \textbf{LR}: Fixed learning rate used in stochastic gradient
     descent. \textbf{Batch}: Number of images used in one parameter update step. \textbf{Epochs}: Number of training epochs. In all the experiments,
     we used fixed momentum of 0.9 and weight decay of 0.0005 for stochastic gradient descent. ```Color Upsampling'' experiments in this Table corresponds
     to those performed on Pascal VOC12 dataset images. For all experiments using Pascal VOC12 images, we use extended
     training segmentation dataset available from~\cite{hariharan2011moredata}, and used standard validation and test splits
     from the main dataset~\cite{voc2012segmentation}.}
  \label{tbl:parameters}
\end{table*}

\clearpage

\section{Parameters and Additional Results for Video Propagation Networks}

In this Section, we present experiment protocols and additional qualitative results for experiments
on video object segmentation, semantic video segmentation and video color
propagation. Table~\ref{tbl:parameters_supp} shows the feature scales and other parameters used in different experiments.
Figures~\ref{fig:video_seg_pos_supp} show some qualitative results on video object segmentation
with some failure cases in Fig.~\ref{fig:video_seg_neg_supp}.
Figure~\ref{fig:semantic_visuals_supp} shows some qualitative results on semantic video segmentation and
Fig.~\ref{fig:color_visuals_supp} shows results on video color propagation.

\newcolumntype{L}[1]{>{\raggedright\let\newline\\\arraybackslash\hspace{0pt}}b{#1}}
\newcolumntype{C}[1]{>{\centering\let\newline\\\arraybackslash\hspace{0pt}}b{#1}}
\newcolumntype{R}[1]{>{\raggedleft\let\newline\\\arraybackslash\hspace{0pt}}b{#1}}

\begin{table*}[h]
\tiny
  \centering
    \begin{tabular}{L{3.0cm} L{2.4cm} L{2.8cm} L{2.8cm} C{0.5cm} C{1.0cm} L{1.2cm}}
      \toprule
\textbf{Experiment} & \textbf{Feature Type} & \textbf{Feature Scale-1, $\Lambda_a$} & \textbf{Feature Scale-2, $\Lambda_b$} & \textbf{$\alpha$} & \textbf{Input Frames} & \textbf{Loss Type} \\
      \midrule
      \textbf{Video Object Segmentation} & ($x,y,Y,Cb,Cr,t$) & (0.02,0.02,0.07,0.4,0.4,0.01) & (0.03,0.03,0.09,0.5,0.5,0.2) & 0.5 & 9 & Logistic\\
      \midrule
      \textbf{Semantic Video Segmentation} & & & & & \\
      \textbf{with CNN1~\cite{yu2015multi}-NoFlow} & ($x,y,R,G,B,t$) & (0.08,0.08,0.2,0.2,0.2,0.04) & (0.11,0.11,0.2,0.2,0.2,0.04) & 0.5 & 3 & Logistic \\
      \textbf{with CNN1~\cite{yu2015multi}-Flow} & ($x+u_x,y+u_y,R,G,B,t$) & (0.11,0.11,0.14,0.14,0.14,0.03) & (0.08,0.08,0.12,0.12,0.12,0.01) & 0.65 & 3 & Logistic\\
      \textbf{with CNN2~\cite{richter2016playing}-Flow} & ($x+u_x,y+u_y,R,G,B,t$) & (0.08,0.08,0.2,0.2,0.2,0.04) & (0.09,0.09,0.25,0.25,0.25,0.03) & 0.5 & 4 & Logistic\\
      \midrule
      \textbf{Video Color Propagation} & ($x,y,I,t$)  & (0.04,0.04,0.2,0.04) & No second kernel & 1 & 4 & MSE\\
      \bottomrule
      \\
    \end{tabular}
    \mycaption{Experiment Protocols} {Experiment protocols for the different experiments presented in this work. \textbf{Feature Types}:
    Feature spaces used for the bilateral convolutions, with position ($x,y$) and color
    ($R,G,B$ or $Y,Cb,Cr$) features $\in [0,255]$. $u_x$, $u_y$ denotes optical flow with respect
    to the present frame and $I$ denotes grayscale intensity.
    \textbf{Feature Scales ($\Lambda_a, \Lambda_b$)}: Cross-validated scales for the features used.
    \textbf{$\alpha$}: Exponential time decay for the input frames.
    \textbf{Input Frames}: Number of input frames for VPN.
    \textbf{Loss Type}: Type
     of loss used for back-propagation. ``MSE'' corresponds to Euclidean mean squared error loss and ``Logistic'' corresponds to multinomial logistic loss.}
  \label{tbl:parameters_supp}
\end{table*}

% \begin{figure}[th!]
% \begin{center}
%   \centerline{\includegraphics[width=\textwidth]{figures/video_seg_visuals_supp_small.pdf}}
%     \mycaption{Video Object Segmentation}
%     {Shown are the different frames in example videos with the corresponding
%     ground truth (GT) masks, predictions from BVS~\cite{marki2016bilateral},
%     OFL~\cite{tsaivideo}, VPN (VPN-Stage2) and VPN-DLab (VPN-DeepLab) models.}
%     \label{fig:video_seg_small_supp}
% \end{center}
% \vspace{-1.0cm}
% \end{figure}

\begin{figure}[th!]
\begin{center}
  \centerline{\includegraphics[width=0.7\textwidth]{figures/video_seg_visuals_supp_positive.pdf}}
    \mycaption{Video Object Segmentation}
    {Shown are the different frames in example videos with the corresponding
    ground truth (GT) masks, predictions from BVS~\cite{marki2016bilateral},
    OFL~\cite{tsaivideo}, VPN (VPN-Stage2) and VPN-DLab (VPN-DeepLab) models.}
    \label{fig:video_seg_pos_supp}
\end{center}
\vspace{-1.0cm}
\end{figure}

\begin{figure}[th!]
\begin{center}
  \centerline{\includegraphics[width=0.7\textwidth]{figures/video_seg_visuals_supp_negative.pdf}}
    \mycaption{Failure Cases for Video Object Segmentation}
    {Shown are the different frames in example videos with the corresponding
    ground truth (GT) masks, predictions from BVS~\cite{marki2016bilateral},
    OFL~\cite{tsaivideo}, VPN (VPN-Stage2) and VPN-DLab (VPN-DeepLab) models.}
    \label{fig:video_seg_neg_supp}
\end{center}
\vspace{-1.0cm}
\end{figure}

\begin{figure}[th!]
\begin{center}
  \centerline{\includegraphics[width=0.9\textwidth]{figures/supp_semantic_visual.pdf}}
    \mycaption{Semantic Video Segmentation}
    {Input video frames and the corresponding ground truth (GT)
    segmentation together with the predictions of CNN~\cite{yu2015multi} and with
    VPN-Flow.}
    \label{fig:semantic_visuals_supp}
\end{center}
\vspace{-0.7cm}
\end{figure}

\begin{figure}[th!]
\begin{center}
  \centerline{\includegraphics[width=\textwidth]{figures/colorization_visuals_supp.pdf}}
  \mycaption{Video Color Propagation}
  {Input grayscale video frames and corresponding ground-truth (GT) color images
  together with color predictions of Levin et al.~\cite{levin2004colorization} and VPN-Stage1 models.}
  \label{fig:color_visuals_supp}
\end{center}
\vspace{-0.7cm}
\end{figure}

\clearpage

\section{Additional Material for Bilateral Inception Networks}
\label{sec:binception-app}

In this section of the Appendix, we first discuss the use of approximate bilateral
filtering in BI modules (Sec.~\ref{sec:lattice}).
Later, we present some qualitative results using different models for the approach presented in
Chapter~\ref{chap:binception} (Sec.~\ref{sec:qualitative-app}).

\subsection{Approximate Bilateral Filtering}
\label{sec:lattice}

The bilateral inception module presented in Chapter~\ref{chap:binception} computes a matrix-vector
product between a Gaussian filter $K$ and a vector of activations $\bz_c$.
Bilateral filtering is an important operation and many algorithmic techniques have been
proposed to speed-up this operation~\cite{paris2006fast,adams2010fast,gastal2011domain}.
In the main paper we opted to implement what can be considered the
brute-force variant of explicitly constructing $K$ and then using BLAS to compute the
matrix-vector product. This resulted in a few millisecond operation.
The explicit way to compute is possible due to the
reduction to super-pixels, e.g., it would not work for DenseCRF variants
that operate on the full image resolution.

Here, we present experiments where we use the fast approximate bilateral filtering
algorithm of~\cite{adams2010fast}, which is also used in Chapter~\ref{chap:bnn}
for learning sparse high dimensional filters. This
choice allows for larger dimensions of matrix-vector multiplication. The reason for choosing
the explicit multiplication in Chapter~\ref{chap:binception} was that it was computationally faster.
For the small sizes of the involved matrices and vectors, the explicit computation is sufficient and we had no
GPU implementation of an approximate technique that matched this runtime. Also it
is conceptually easier and the gradient to the feature transformations ($\Lambda \mathbf{f}$) is
obtained using standard matrix calculus.

\subsubsection{Experiments}

We modified the existing segmentation architectures analogous to those in Chapter~\ref{chap:binception}.
The main difference is that, here, the inception modules use the lattice
approximation~\cite{adams2010fast} to compute the bilateral filtering.
Using the lattice approximation did not allow us to back-propagate through feature transformations ($\Lambda$)
and thus we used hand-specified feature scales as will be explained later.
Specifically, we take CNN architectures from the works
of~\cite{chen2014semantic,zheng2015conditional,bell2015minc} and insert the BI modules between
the spatial FC layers.
We use superpixels from~\cite{DollarICCV13edges}
for all the experiments with the lattice approximation. Experiments are
performed using Caffe neural network framework~\cite{jia2014caffe}.

\begin{table}
  \small
  \centering
  \begin{tabular}{p{5.5cm}>{\raggedright\arraybackslash}p{1.4cm}>{\centering\arraybackslash}p{2.2cm}}
    \toprule
		\textbf{Model} & \emph{IoU} & \emph{Runtime}(ms) \\
    \midrule

    %%%%%%%%%%%% Scores computed by us)%%%%%%%%%%%%
		\deeplablargefov & 68.9 & 145ms\\
    \midrule
    \bi{7}{2}-\bi{8}{10}& \textbf{73.8} & +600 \\
    \midrule
    \deeplablargefovcrf~\cite{chen2014semantic} & 72.7 & +830\\
    \deeplabmsclargefovcrf~\cite{chen2014semantic} & \textbf{73.6} & +880\\
    DeepLab-EdgeNet~\cite{chen2015semantic} & 71.7 & +30\\
    DeepLab-EdgeNet-CRF~\cite{chen2015semantic} & \textbf{73.6} & +860\\
  \bottomrule \\
  \end{tabular}
  \mycaption{Semantic Segmentation using the DeepLab model}
  {IoU scores on the Pascal VOC12 segmentation test dataset
  with different models and our modified inception model.
  Also shown are the corresponding runtimes in milliseconds. Runtimes
  also include superpixel computations (300 ms with Dollar superpixels~\cite{DollarICCV13edges})}
  \label{tab:largefovresults}
\end{table}

\paragraph{Semantic Segmentation}
The experiments in this section use the Pascal VOC12 segmentation dataset~\cite{voc2012segmentation} with 21 object classes and the images have a maximum resolution of 0.25 megapixels.
For all experiments on VOC12, we train using the extended training set of
10581 images collected by~\cite{hariharan2011moredata}.
We modified the \deeplab~network architecture of~\cite{chen2014semantic} and
the CRFasRNN architecture from~\cite{zheng2015conditional} which uses a CNN with
deconvolution layers followed by DenseCRF trained end-to-end.

\paragraph{DeepLab Model}\label{sec:deeplabmodel}
We experimented with the \bi{7}{2}-\bi{8}{10} inception model.
Results using the~\deeplab~model are summarized in Tab.~\ref{tab:largefovresults}.
Although we get similar improvements with inception modules as with the
explicit kernel computation, using lattice approximation is slower.

\begin{table}
  \small
  \centering
  \begin{tabular}{p{6.4cm}>{\raggedright\arraybackslash}p{1.8cm}>{\raggedright\arraybackslash}p{1.8cm}}
    \toprule
    \textbf{Model} & \emph{IoU (Val)} & \emph{IoU (Test)}\\
    \midrule
    %%%%%%%%%%%% Scores computed by us)%%%%%%%%%%%%
    CNN &  67.5 & - \\
    \deconv (CNN+Deconvolutions) & 69.8 & 72.0 \\
    \midrule
    \bi{3}{6}-\bi{4}{6}-\bi{7}{2}-\bi{8}{6}& 71.9 & - \\
    \bi{3}{6}-\bi{4}{6}-\bi{7}{2}-\bi{8}{6}-\gi{6}& 73.6 &  \href{http://host.robots.ox.ac.uk:8080/anonymous/VOTV5E.html}{\textbf{75.2}}\\
    \midrule
    \deconvcrf (CRF-RNN)~\cite{zheng2015conditional} & 73.0 & 74.7\\
    Context-CRF-RNN~\cite{yu2015multi} & ~~ - ~ & \textbf{75.3} \\
    \bottomrule \\
  \end{tabular}
  \mycaption{Semantic Segmentation using the CRFasRNN model}{IoU score corresponding to different models
  on Pascal VOC12 reduced validation / test segmentation dataset. The reduced validation set consists of 346 images
  as used in~\cite{zheng2015conditional} where we adapted the model from.}
  \label{tab:deconvresults-app}
\end{table}

\paragraph{CRFasRNN Model}\label{sec:deepinception}
We add BI modules after score-pool3, score-pool4, \fc{7} and \fc{8} $1\times1$ convolution layers
resulting in the \bi{3}{6}-\bi{4}{6}-\bi{7}{2}-\bi{8}{6}
model and also experimented with another variant where $BI_8$ is followed by another inception
module, G$(6)$, with 6 Gaussian kernels.
Note that here also we discarded both deconvolution and DenseCRF parts of the original model~\cite{zheng2015conditional}
and inserted the BI modules in the base CNN and found similar improvements compared to the inception modules with explicit
kernel computaion. See Tab.~\ref{tab:deconvresults-app} for results on the CRFasRNN model.

\paragraph{Material Segmentation}
Table~\ref{tab:mincresults-app} shows the results on the MINC dataset~\cite{bell2015minc}
obtained by modifying the AlexNet architecture with our inception modules. We observe
similar improvements as with explicit kernel construction.
For this model, we do not provide any learned setup due to very limited segment training
data. The weights to combine outputs in the bilateral inception layer are
found by validation on the validation set.

\begin{table}[t]
  \small
  \centering
  \begin{tabular}{p{3.5cm}>{\centering\arraybackslash}p{4.0cm}}
    \toprule
    \textbf{Model} & Class / Total accuracy\\
    \midrule

    %%%%%%%%%%%% Scores computed by us)%%%%%%%%%%%%
    AlexNet CNN & 55.3 / 58.9 \\
    \midrule
    \bi{7}{2}-\bi{8}{6}& 68.5 / 71.8 \\
    \bi{7}{2}-\bi{8}{6}-G$(6)$& 67.6 / 73.1 \\
    \midrule
    AlexNet-CRF & 65.5 / 71.0 \\
    \bottomrule \\
  \end{tabular}
  \mycaption{Material Segmentation using AlexNet}{Pixel accuracy of different models on
  the MINC material segmentation test dataset~\cite{bell2015minc}.}
  \label{tab:mincresults-app}
\end{table}

\paragraph{Scales of Bilateral Inception Modules}
\label{sec:scales}

Unlike the explicit kernel technique presented in the main text (Chapter~\ref{chap:binception}),
we didn't back-propagate through feature transformation ($\Lambda$)
using the approximate bilateral filter technique.
So, the feature scales are hand-specified and validated, which are as follows.
The optimal scale values for the \bi{7}{2}-\bi{8}{2} model are found by validation for the best performance which are
$\sigma_{xy}$ = (0.1, 0.1) for the spatial (XY) kernel and $\sigma_{rgbxy}$ = (0.1, 0.1, 0.1, 0.01, 0.01) for color and position (RGBXY)  kernel.
Next, as more kernels are added to \bi{8}{2}, we set scales to be $\alpha$*($\sigma_{xy}$, $\sigma_{rgbxy}$).
The value of $\alpha$ is chosen as  1, 0.5, 0.1, 0.05, 0.1, at uniform interval, for the \bi{8}{10} bilateral inception module.


\subsection{Qualitative Results}
\label{sec:qualitative-app}

In this section, we present more qualitative results obtained using the BI module with explicit
kernel computation technique presented in Chapter~\ref{chap:binception}. Results on the Pascal VOC12
dataset~\cite{voc2012segmentation} using the DeepLab-LargeFOV model are shown in Fig.~\ref{fig:semantic_visuals-app},
followed by the results on MINC dataset~\cite{bell2015minc}
in Fig.~\ref{fig:material_visuals-app} and on
Cityscapes dataset~\cite{Cordts2015Cvprw} in Fig.~\ref{fig:street_visuals-app}.


\definecolor{voc_1}{RGB}{0, 0, 0}
\definecolor{voc_2}{RGB}{128, 0, 0}
\definecolor{voc_3}{RGB}{0, 128, 0}
\definecolor{voc_4}{RGB}{128, 128, 0}
\definecolor{voc_5}{RGB}{0, 0, 128}
\definecolor{voc_6}{RGB}{128, 0, 128}
\definecolor{voc_7}{RGB}{0, 128, 128}
\definecolor{voc_8}{RGB}{128, 128, 128}
\definecolor{voc_9}{RGB}{64, 0, 0}
\definecolor{voc_10}{RGB}{192, 0, 0}
\definecolor{voc_11}{RGB}{64, 128, 0}
\definecolor{voc_12}{RGB}{192, 128, 0}
\definecolor{voc_13}{RGB}{64, 0, 128}
\definecolor{voc_14}{RGB}{192, 0, 128}
\definecolor{voc_15}{RGB}{64, 128, 128}
\definecolor{voc_16}{RGB}{192, 128, 128}
\definecolor{voc_17}{RGB}{0, 64, 0}
\definecolor{voc_18}{RGB}{128, 64, 0}
\definecolor{voc_19}{RGB}{0, 192, 0}
\definecolor{voc_20}{RGB}{128, 192, 0}
\definecolor{voc_21}{RGB}{0, 64, 128}
\definecolor{voc_22}{RGB}{128, 64, 128}

\begin{figure*}[!ht]
  \small
  \centering
  \fcolorbox{white}{voc_1}{\rule{0pt}{4pt}\rule{4pt}{0pt}} Background~~
  \fcolorbox{white}{voc_2}{\rule{0pt}{4pt}\rule{4pt}{0pt}} Aeroplane~~
  \fcolorbox{white}{voc_3}{\rule{0pt}{4pt}\rule{4pt}{0pt}} Bicycle~~
  \fcolorbox{white}{voc_4}{\rule{0pt}{4pt}\rule{4pt}{0pt}} Bird~~
  \fcolorbox{white}{voc_5}{\rule{0pt}{4pt}\rule{4pt}{0pt}} Boat~~
  \fcolorbox{white}{voc_6}{\rule{0pt}{4pt}\rule{4pt}{0pt}} Bottle~~
  \fcolorbox{white}{voc_7}{\rule{0pt}{4pt}\rule{4pt}{0pt}} Bus~~
  \fcolorbox{white}{voc_8}{\rule{0pt}{4pt}\rule{4pt}{0pt}} Car~~\\
  \fcolorbox{white}{voc_9}{\rule{0pt}{4pt}\rule{4pt}{0pt}} Cat~~
  \fcolorbox{white}{voc_10}{\rule{0pt}{4pt}\rule{4pt}{0pt}} Chair~~
  \fcolorbox{white}{voc_11}{\rule{0pt}{4pt}\rule{4pt}{0pt}} Cow~~
  \fcolorbox{white}{voc_12}{\rule{0pt}{4pt}\rule{4pt}{0pt}} Dining Table~~
  \fcolorbox{white}{voc_13}{\rule{0pt}{4pt}\rule{4pt}{0pt}} Dog~~
  \fcolorbox{white}{voc_14}{\rule{0pt}{4pt}\rule{4pt}{0pt}} Horse~~
  \fcolorbox{white}{voc_15}{\rule{0pt}{4pt}\rule{4pt}{0pt}} Motorbike~~
  \fcolorbox{white}{voc_16}{\rule{0pt}{4pt}\rule{4pt}{0pt}} Person~~\\
  \fcolorbox{white}{voc_17}{\rule{0pt}{4pt}\rule{4pt}{0pt}} Potted Plant~~
  \fcolorbox{white}{voc_18}{\rule{0pt}{4pt}\rule{4pt}{0pt}} Sheep~~
  \fcolorbox{white}{voc_19}{\rule{0pt}{4pt}\rule{4pt}{0pt}} Sofa~~
  \fcolorbox{white}{voc_20}{\rule{0pt}{4pt}\rule{4pt}{0pt}} Train~~
  \fcolorbox{white}{voc_21}{\rule{0pt}{4pt}\rule{4pt}{0pt}} TV monitor~~\\


  \subfigure{%
    \includegraphics[width=.15\columnwidth]{figures/supplementary/2008_001308_given.png}
  }
  \subfigure{%
    \includegraphics[width=.15\columnwidth]{figures/supplementary/2008_001308_sp.png}
  }
  \subfigure{%
    \includegraphics[width=.15\columnwidth]{figures/supplementary/2008_001308_gt.png}
  }
  \subfigure{%
    \includegraphics[width=.15\columnwidth]{figures/supplementary/2008_001308_cnn.png}
  }
  \subfigure{%
    \includegraphics[width=.15\columnwidth]{figures/supplementary/2008_001308_crf.png}
  }
  \subfigure{%
    \includegraphics[width=.15\columnwidth]{figures/supplementary/2008_001308_ours.png}
  }\\[-2ex]


  \subfigure{%
    \includegraphics[width=.15\columnwidth]{figures/supplementary/2008_001821_given.png}
  }
  \subfigure{%
    \includegraphics[width=.15\columnwidth]{figures/supplementary/2008_001821_sp.png}
  }
  \subfigure{%
    \includegraphics[width=.15\columnwidth]{figures/supplementary/2008_001821_gt.png}
  }
  \subfigure{%
    \includegraphics[width=.15\columnwidth]{figures/supplementary/2008_001821_cnn.png}
  }
  \subfigure{%
    \includegraphics[width=.15\columnwidth]{figures/supplementary/2008_001821_crf.png}
  }
  \subfigure{%
    \includegraphics[width=.15\columnwidth]{figures/supplementary/2008_001821_ours.png}
  }\\[-2ex]



  \subfigure{%
    \includegraphics[width=.15\columnwidth]{figures/supplementary/2008_004612_given.png}
  }
  \subfigure{%
    \includegraphics[width=.15\columnwidth]{figures/supplementary/2008_004612_sp.png}
  }
  \subfigure{%
    \includegraphics[width=.15\columnwidth]{figures/supplementary/2008_004612_gt.png}
  }
  \subfigure{%
    \includegraphics[width=.15\columnwidth]{figures/supplementary/2008_004612_cnn.png}
  }
  \subfigure{%
    \includegraphics[width=.15\columnwidth]{figures/supplementary/2008_004612_crf.png}
  }
  \subfigure{%
    \includegraphics[width=.15\columnwidth]{figures/supplementary/2008_004612_ours.png}
  }\\[-2ex]


  \subfigure{%
    \includegraphics[width=.15\columnwidth]{figures/supplementary/2009_001008_given.png}
  }
  \subfigure{%
    \includegraphics[width=.15\columnwidth]{figures/supplementary/2009_001008_sp.png}
  }
  \subfigure{%
    \includegraphics[width=.15\columnwidth]{figures/supplementary/2009_001008_gt.png}
  }
  \subfigure{%
    \includegraphics[width=.15\columnwidth]{figures/supplementary/2009_001008_cnn.png}
  }
  \subfigure{%
    \includegraphics[width=.15\columnwidth]{figures/supplementary/2009_001008_crf.png}
  }
  \subfigure{%
    \includegraphics[width=.15\columnwidth]{figures/supplementary/2009_001008_ours.png}
  }\\[-2ex]




  \subfigure{%
    \includegraphics[width=.15\columnwidth]{figures/supplementary/2009_004497_given.png}
  }
  \subfigure{%
    \includegraphics[width=.15\columnwidth]{figures/supplementary/2009_004497_sp.png}
  }
  \subfigure{%
    \includegraphics[width=.15\columnwidth]{figures/supplementary/2009_004497_gt.png}
  }
  \subfigure{%
    \includegraphics[width=.15\columnwidth]{figures/supplementary/2009_004497_cnn.png}
  }
  \subfigure{%
    \includegraphics[width=.15\columnwidth]{figures/supplementary/2009_004497_crf.png}
  }
  \subfigure{%
    \includegraphics[width=.15\columnwidth]{figures/supplementary/2009_004497_ours.png}
  }\\[-2ex]



  \setcounter{subfigure}{0}
  \subfigure[\scriptsize Input]{%
    \includegraphics[width=.15\columnwidth]{figures/supplementary/2010_001327_given.png}
  }
  \subfigure[\scriptsize Superpixels]{%
    \includegraphics[width=.15\columnwidth]{figures/supplementary/2010_001327_sp.png}
  }
  \subfigure[\scriptsize GT]{%
    \includegraphics[width=.15\columnwidth]{figures/supplementary/2010_001327_gt.png}
  }
  \subfigure[\scriptsize Deeplab]{%
    \includegraphics[width=.15\columnwidth]{figures/supplementary/2010_001327_cnn.png}
  }
  \subfigure[\scriptsize +DenseCRF]{%
    \includegraphics[width=.15\columnwidth]{figures/supplementary/2010_001327_crf.png}
  }
  \subfigure[\scriptsize Using BI]{%
    \includegraphics[width=.15\columnwidth]{figures/supplementary/2010_001327_ours.png}
  }
  \mycaption{Semantic Segmentation}{Example results of semantic segmentation
  on the Pascal VOC12 dataset.
  (d)~depicts the DeepLab CNN result, (e)~CNN + 10 steps of mean-field inference,
  (f~result obtained with bilateral inception (BI) modules (\bi{6}{2}+\bi{7}{6}) between \fc~layers.}
  \label{fig:semantic_visuals-app}
\end{figure*}


\definecolor{minc_1}{HTML}{771111}
\definecolor{minc_2}{HTML}{CAC690}
\definecolor{minc_3}{HTML}{EEEEEE}
\definecolor{minc_4}{HTML}{7C8FA6}
\definecolor{minc_5}{HTML}{597D31}
\definecolor{minc_6}{HTML}{104410}
\definecolor{minc_7}{HTML}{BB819C}
\definecolor{minc_8}{HTML}{D0CE48}
\definecolor{minc_9}{HTML}{622745}
\definecolor{minc_10}{HTML}{666666}
\definecolor{minc_11}{HTML}{D54A31}
\definecolor{minc_12}{HTML}{101044}
\definecolor{minc_13}{HTML}{444126}
\definecolor{minc_14}{HTML}{75D646}
\definecolor{minc_15}{HTML}{DD4348}
\definecolor{minc_16}{HTML}{5C8577}
\definecolor{minc_17}{HTML}{C78472}
\definecolor{minc_18}{HTML}{75D6D0}
\definecolor{minc_19}{HTML}{5B4586}
\definecolor{minc_20}{HTML}{C04393}
\definecolor{minc_21}{HTML}{D69948}
\definecolor{minc_22}{HTML}{7370D8}
\definecolor{minc_23}{HTML}{7A3622}
\definecolor{minc_24}{HTML}{000000}

\begin{figure*}[!ht]
  \small % scriptsize
  \centering
  \fcolorbox{white}{minc_1}{\rule{0pt}{4pt}\rule{4pt}{0pt}} Brick~~
  \fcolorbox{white}{minc_2}{\rule{0pt}{4pt}\rule{4pt}{0pt}} Carpet~~
  \fcolorbox{white}{minc_3}{\rule{0pt}{4pt}\rule{4pt}{0pt}} Ceramic~~
  \fcolorbox{white}{minc_4}{\rule{0pt}{4pt}\rule{4pt}{0pt}} Fabric~~
  \fcolorbox{white}{minc_5}{\rule{0pt}{4pt}\rule{4pt}{0pt}} Foliage~~
  \fcolorbox{white}{minc_6}{\rule{0pt}{4pt}\rule{4pt}{0pt}} Food~~
  \fcolorbox{white}{minc_7}{\rule{0pt}{4pt}\rule{4pt}{0pt}} Glass~~
  \fcolorbox{white}{minc_8}{\rule{0pt}{4pt}\rule{4pt}{0pt}} Hair~~\\
  \fcolorbox{white}{minc_9}{\rule{0pt}{4pt}\rule{4pt}{0pt}} Leather~~
  \fcolorbox{white}{minc_10}{\rule{0pt}{4pt}\rule{4pt}{0pt}} Metal~~
  \fcolorbox{white}{minc_11}{\rule{0pt}{4pt}\rule{4pt}{0pt}} Mirror~~
  \fcolorbox{white}{minc_12}{\rule{0pt}{4pt}\rule{4pt}{0pt}} Other~~
  \fcolorbox{white}{minc_13}{\rule{0pt}{4pt}\rule{4pt}{0pt}} Painted~~
  \fcolorbox{white}{minc_14}{\rule{0pt}{4pt}\rule{4pt}{0pt}} Paper~~
  \fcolorbox{white}{minc_15}{\rule{0pt}{4pt}\rule{4pt}{0pt}} Plastic~~\\
  \fcolorbox{white}{minc_16}{\rule{0pt}{4pt}\rule{4pt}{0pt}} Polished Stone~~
  \fcolorbox{white}{minc_17}{\rule{0pt}{4pt}\rule{4pt}{0pt}} Skin~~
  \fcolorbox{white}{minc_18}{\rule{0pt}{4pt}\rule{4pt}{0pt}} Sky~~
  \fcolorbox{white}{minc_19}{\rule{0pt}{4pt}\rule{4pt}{0pt}} Stone~~
  \fcolorbox{white}{minc_20}{\rule{0pt}{4pt}\rule{4pt}{0pt}} Tile~~
  \fcolorbox{white}{minc_21}{\rule{0pt}{4pt}\rule{4pt}{0pt}} Wallpaper~~
  \fcolorbox{white}{minc_22}{\rule{0pt}{4pt}\rule{4pt}{0pt}} Water~~
  \fcolorbox{white}{minc_23}{\rule{0pt}{4pt}\rule{4pt}{0pt}} Wood~~\\
  \subfigure{%
    \includegraphics[width=.15\columnwidth]{figures/supplementary/000008468_given.png}
  }
  \subfigure{%
    \includegraphics[width=.15\columnwidth]{figures/supplementary/000008468_sp.png}
  }
  \subfigure{%
    \includegraphics[width=.15\columnwidth]{figures/supplementary/000008468_gt.png}
  }
  \subfigure{%
    \includegraphics[width=.15\columnwidth]{figures/supplementary/000008468_cnn.png}
  }
  \subfigure{%
    \includegraphics[width=.15\columnwidth]{figures/supplementary/000008468_crf.png}
  }
  \subfigure{%
    \includegraphics[width=.15\columnwidth]{figures/supplementary/000008468_ours.png}
  }\\[-2ex]

  \subfigure{%
    \includegraphics[width=.15\columnwidth]{figures/supplementary/000009053_given.png}
  }
  \subfigure{%
    \includegraphics[width=.15\columnwidth]{figures/supplementary/000009053_sp.png}
  }
  \subfigure{%
    \includegraphics[width=.15\columnwidth]{figures/supplementary/000009053_gt.png}
  }
  \subfigure{%
    \includegraphics[width=.15\columnwidth]{figures/supplementary/000009053_cnn.png}
  }
  \subfigure{%
    \includegraphics[width=.15\columnwidth]{figures/supplementary/000009053_crf.png}
  }
  \subfigure{%
    \includegraphics[width=.15\columnwidth]{figures/supplementary/000009053_ours.png}
  }\\[-2ex]




  \subfigure{%
    \includegraphics[width=.15\columnwidth]{figures/supplementary/000014977_given.png}
  }
  \subfigure{%
    \includegraphics[width=.15\columnwidth]{figures/supplementary/000014977_sp.png}
  }
  \subfigure{%
    \includegraphics[width=.15\columnwidth]{figures/supplementary/000014977_gt.png}
  }
  \subfigure{%
    \includegraphics[width=.15\columnwidth]{figures/supplementary/000014977_cnn.png}
  }
  \subfigure{%
    \includegraphics[width=.15\columnwidth]{figures/supplementary/000014977_crf.png}
  }
  \subfigure{%
    \includegraphics[width=.15\columnwidth]{figures/supplementary/000014977_ours.png}
  }\\[-2ex]


  \subfigure{%
    \includegraphics[width=.15\columnwidth]{figures/supplementary/000022922_given.png}
  }
  \subfigure{%
    \includegraphics[width=.15\columnwidth]{figures/supplementary/000022922_sp.png}
  }
  \subfigure{%
    \includegraphics[width=.15\columnwidth]{figures/supplementary/000022922_gt.png}
  }
  \subfigure{%
    \includegraphics[width=.15\columnwidth]{figures/supplementary/000022922_cnn.png}
  }
  \subfigure{%
    \includegraphics[width=.15\columnwidth]{figures/supplementary/000022922_crf.png}
  }
  \subfigure{%
    \includegraphics[width=.15\columnwidth]{figures/supplementary/000022922_ours.png}
  }\\[-2ex]


  \subfigure{%
    \includegraphics[width=.15\columnwidth]{figures/supplementary/000025711_given.png}
  }
  \subfigure{%
    \includegraphics[width=.15\columnwidth]{figures/supplementary/000025711_sp.png}
  }
  \subfigure{%
    \includegraphics[width=.15\columnwidth]{figures/supplementary/000025711_gt.png}
  }
  \subfigure{%
    \includegraphics[width=.15\columnwidth]{figures/supplementary/000025711_cnn.png}
  }
  \subfigure{%
    \includegraphics[width=.15\columnwidth]{figures/supplementary/000025711_crf.png}
  }
  \subfigure{%
    \includegraphics[width=.15\columnwidth]{figures/supplementary/000025711_ours.png}
  }\\[-2ex]


  \subfigure{%
    \includegraphics[width=.15\columnwidth]{figures/supplementary/000034473_given.png}
  }
  \subfigure{%
    \includegraphics[width=.15\columnwidth]{figures/supplementary/000034473_sp.png}
  }
  \subfigure{%
    \includegraphics[width=.15\columnwidth]{figures/supplementary/000034473_gt.png}
  }
  \subfigure{%
    \includegraphics[width=.15\columnwidth]{figures/supplementary/000034473_cnn.png}
  }
  \subfigure{%
    \includegraphics[width=.15\columnwidth]{figures/supplementary/000034473_crf.png}
  }
  \subfigure{%
    \includegraphics[width=.15\columnwidth]{figures/supplementary/000034473_ours.png}
  }\\[-2ex]


  \subfigure{%
    \includegraphics[width=.15\columnwidth]{figures/supplementary/000035463_given.png}
  }
  \subfigure{%
    \includegraphics[width=.15\columnwidth]{figures/supplementary/000035463_sp.png}
  }
  \subfigure{%
    \includegraphics[width=.15\columnwidth]{figures/supplementary/000035463_gt.png}
  }
  \subfigure{%
    \includegraphics[width=.15\columnwidth]{figures/supplementary/000035463_cnn.png}
  }
  \subfigure{%
    \includegraphics[width=.15\columnwidth]{figures/supplementary/000035463_crf.png}
  }
  \subfigure{%
    \includegraphics[width=.15\columnwidth]{figures/supplementary/000035463_ours.png}
  }\\[-2ex]


  \setcounter{subfigure}{0}
  \subfigure[\scriptsize Input]{%
    \includegraphics[width=.15\columnwidth]{figures/supplementary/000035993_given.png}
  }
  \subfigure[\scriptsize Superpixels]{%
    \includegraphics[width=.15\columnwidth]{figures/supplementary/000035993_sp.png}
  }
  \subfigure[\scriptsize GT]{%
    \includegraphics[width=.15\columnwidth]{figures/supplementary/000035993_gt.png}
  }
  \subfigure[\scriptsize AlexNet]{%
    \includegraphics[width=.15\columnwidth]{figures/supplementary/000035993_cnn.png}
  }
  \subfigure[\scriptsize +DenseCRF]{%
    \includegraphics[width=.15\columnwidth]{figures/supplementary/000035993_crf.png}
  }
  \subfigure[\scriptsize Using BI]{%
    \includegraphics[width=.15\columnwidth]{figures/supplementary/000035993_ours.png}
  }
  \mycaption{Material Segmentation}{Example results of material segmentation.
  (d)~depicts the AlexNet CNN result, (e)~CNN + 10 steps of mean-field inference,
  (f)~result obtained with bilateral inception (BI) modules (\bi{7}{2}+\bi{8}{6}) between
  \fc~layers.}
\label{fig:material_visuals-app}
\end{figure*}


\definecolor{city_1}{RGB}{128, 64, 128}
\definecolor{city_2}{RGB}{244, 35, 232}
\definecolor{city_3}{RGB}{70, 70, 70}
\definecolor{city_4}{RGB}{102, 102, 156}
\definecolor{city_5}{RGB}{190, 153, 153}
\definecolor{city_6}{RGB}{153, 153, 153}
\definecolor{city_7}{RGB}{250, 170, 30}
\definecolor{city_8}{RGB}{220, 220, 0}
\definecolor{city_9}{RGB}{107, 142, 35}
\definecolor{city_10}{RGB}{152, 251, 152}
\definecolor{city_11}{RGB}{70, 130, 180}
\definecolor{city_12}{RGB}{220, 20, 60}
\definecolor{city_13}{RGB}{255, 0, 0}
\definecolor{city_14}{RGB}{0, 0, 142}
\definecolor{city_15}{RGB}{0, 0, 70}
\definecolor{city_16}{RGB}{0, 60, 100}
\definecolor{city_17}{RGB}{0, 80, 100}
\definecolor{city_18}{RGB}{0, 0, 230}
\definecolor{city_19}{RGB}{119, 11, 32}
\begin{figure*}[!ht]
  \small % scriptsize
  \centering


  \subfigure{%
    \includegraphics[width=.18\columnwidth]{figures/supplementary/frankfurt00000_016005_given.png}
  }
  \subfigure{%
    \includegraphics[width=.18\columnwidth]{figures/supplementary/frankfurt00000_016005_sp.png}
  }
  \subfigure{%
    \includegraphics[width=.18\columnwidth]{figures/supplementary/frankfurt00000_016005_gt.png}
  }
  \subfigure{%
    \includegraphics[width=.18\columnwidth]{figures/supplementary/frankfurt00000_016005_cnn.png}
  }
  \subfigure{%
    \includegraphics[width=.18\columnwidth]{figures/supplementary/frankfurt00000_016005_ours.png}
  }\\[-2ex]

  \subfigure{%
    \includegraphics[width=.18\columnwidth]{figures/supplementary/frankfurt00000_004617_given.png}
  }
  \subfigure{%
    \includegraphics[width=.18\columnwidth]{figures/supplementary/frankfurt00000_004617_sp.png}
  }
  \subfigure{%
    \includegraphics[width=.18\columnwidth]{figures/supplementary/frankfurt00000_004617_gt.png}
  }
  \subfigure{%
    \includegraphics[width=.18\columnwidth]{figures/supplementary/frankfurt00000_004617_cnn.png}
  }
  \subfigure{%
    \includegraphics[width=.18\columnwidth]{figures/supplementary/frankfurt00000_004617_ours.png}
  }\\[-2ex]

  \subfigure{%
    \includegraphics[width=.18\columnwidth]{figures/supplementary/frankfurt00000_020880_given.png}
  }
  \subfigure{%
    \includegraphics[width=.18\columnwidth]{figures/supplementary/frankfurt00000_020880_sp.png}
  }
  \subfigure{%
    \includegraphics[width=.18\columnwidth]{figures/supplementary/frankfurt00000_020880_gt.png}
  }
  \subfigure{%
    \includegraphics[width=.18\columnwidth]{figures/supplementary/frankfurt00000_020880_cnn.png}
  }
  \subfigure{%
    \includegraphics[width=.18\columnwidth]{figures/supplementary/frankfurt00000_020880_ours.png}
  }\\[-2ex]



  \subfigure{%
    \includegraphics[width=.18\columnwidth]{figures/supplementary/frankfurt00001_007285_given.png}
  }
  \subfigure{%
    \includegraphics[width=.18\columnwidth]{figures/supplementary/frankfurt00001_007285_sp.png}
  }
  \subfigure{%
    \includegraphics[width=.18\columnwidth]{figures/supplementary/frankfurt00001_007285_gt.png}
  }
  \subfigure{%
    \includegraphics[width=.18\columnwidth]{figures/supplementary/frankfurt00001_007285_cnn.png}
  }
  \subfigure{%
    \includegraphics[width=.18\columnwidth]{figures/supplementary/frankfurt00001_007285_ours.png}
  }\\[-2ex]


  \subfigure{%
    \includegraphics[width=.18\columnwidth]{figures/supplementary/frankfurt00001_059789_given.png}
  }
  \subfigure{%
    \includegraphics[width=.18\columnwidth]{figures/supplementary/frankfurt00001_059789_sp.png}
  }
  \subfigure{%
    \includegraphics[width=.18\columnwidth]{figures/supplementary/frankfurt00001_059789_gt.png}
  }
  \subfigure{%
    \includegraphics[width=.18\columnwidth]{figures/supplementary/frankfurt00001_059789_cnn.png}
  }
  \subfigure{%
    \includegraphics[width=.18\columnwidth]{figures/supplementary/frankfurt00001_059789_ours.png}
  }\\[-2ex]


  \subfigure{%
    \includegraphics[width=.18\columnwidth]{figures/supplementary/frankfurt00001_068208_given.png}
  }
  \subfigure{%
    \includegraphics[width=.18\columnwidth]{figures/supplementary/frankfurt00001_068208_sp.png}
  }
  \subfigure{%
    \includegraphics[width=.18\columnwidth]{figures/supplementary/frankfurt00001_068208_gt.png}
  }
  \subfigure{%
    \includegraphics[width=.18\columnwidth]{figures/supplementary/frankfurt00001_068208_cnn.png}
  }
  \subfigure{%
    \includegraphics[width=.18\columnwidth]{figures/supplementary/frankfurt00001_068208_ours.png}
  }\\[-2ex]

  \subfigure{%
    \includegraphics[width=.18\columnwidth]{figures/supplementary/frankfurt00001_082466_given.png}
  }
  \subfigure{%
    \includegraphics[width=.18\columnwidth]{figures/supplementary/frankfurt00001_082466_sp.png}
  }
  \subfigure{%
    \includegraphics[width=.18\columnwidth]{figures/supplementary/frankfurt00001_082466_gt.png}
  }
  \subfigure{%
    \includegraphics[width=.18\columnwidth]{figures/supplementary/frankfurt00001_082466_cnn.png}
  }
  \subfigure{%
    \includegraphics[width=.18\columnwidth]{figures/supplementary/frankfurt00001_082466_ours.png}
  }\\[-2ex]

  \subfigure{%
    \includegraphics[width=.18\columnwidth]{figures/supplementary/lindau00033_000019_given.png}
  }
  \subfigure{%
    \includegraphics[width=.18\columnwidth]{figures/supplementary/lindau00033_000019_sp.png}
  }
  \subfigure{%
    \includegraphics[width=.18\columnwidth]{figures/supplementary/lindau00033_000019_gt.png}
  }
  \subfigure{%
    \includegraphics[width=.18\columnwidth]{figures/supplementary/lindau00033_000019_cnn.png}
  }
  \subfigure{%
    \includegraphics[width=.18\columnwidth]{figures/supplementary/lindau00033_000019_ours.png}
  }\\[-2ex]

  \subfigure{%
    \includegraphics[width=.18\columnwidth]{figures/supplementary/lindau00052_000019_given.png}
  }
  \subfigure{%
    \includegraphics[width=.18\columnwidth]{figures/supplementary/lindau00052_000019_sp.png}
  }
  \subfigure{%
    \includegraphics[width=.18\columnwidth]{figures/supplementary/lindau00052_000019_gt.png}
  }
  \subfigure{%
    \includegraphics[width=.18\columnwidth]{figures/supplementary/lindau00052_000019_cnn.png}
  }
  \subfigure{%
    \includegraphics[width=.18\columnwidth]{figures/supplementary/lindau00052_000019_ours.png}
  }\\[-2ex]




  \subfigure{%
    \includegraphics[width=.18\columnwidth]{figures/supplementary/lindau00027_000019_given.png}
  }
  \subfigure{%
    \includegraphics[width=.18\columnwidth]{figures/supplementary/lindau00027_000019_sp.png}
  }
  \subfigure{%
    \includegraphics[width=.18\columnwidth]{figures/supplementary/lindau00027_000019_gt.png}
  }
  \subfigure{%
    \includegraphics[width=.18\columnwidth]{figures/supplementary/lindau00027_000019_cnn.png}
  }
  \subfigure{%
    \includegraphics[width=.18\columnwidth]{figures/supplementary/lindau00027_000019_ours.png}
  }\\[-2ex]



  \setcounter{subfigure}{0}
  \subfigure[\scriptsize Input]{%
    \includegraphics[width=.18\columnwidth]{figures/supplementary/lindau00029_000019_given.png}
  }
  \subfigure[\scriptsize Superpixels]{%
    \includegraphics[width=.18\columnwidth]{figures/supplementary/lindau00029_000019_sp.png}
  }
  \subfigure[\scriptsize GT]{%
    \includegraphics[width=.18\columnwidth]{figures/supplementary/lindau00029_000019_gt.png}
  }
  \subfigure[\scriptsize Deeplab]{%
    \includegraphics[width=.18\columnwidth]{figures/supplementary/lindau00029_000019_cnn.png}
  }
  \subfigure[\scriptsize Using BI]{%
    \includegraphics[width=.18\columnwidth]{figures/supplementary/lindau00029_000019_ours.png}
  }%\\[-2ex]

  \mycaption{Street Scene Segmentation}{Example results of street scene segmentation.
  (d)~depicts the DeepLab results, (e)~result obtained by adding bilateral inception (BI) modules (\bi{6}{2}+\bi{7}{6}) between \fc~layers.}
\label{fig:street_visuals-app}
\end{figure*}

\end{document} 
