\section{Extensive-Form Games}
\label{app:efgs}


In a \emph{tree-form sequential decision process (TFSDP)} problem the agent
interacts with the environment in two ways: at \emph{decision points}, the
agent must act by picking an action from a set of legal actions; at
\emph{observation points}, the agent observes a signal drawn from a set of
possible signals.
Different decision points can have different sets of legal actions, and
different observation points can have different sets of possible signals.
Decision and observation points are structured as a \emph{tree}: under the standard assumption that
the agent
is not forgetful, so, it is not possible for the agent to cycle back to a
previously encountered decision or observation point by following the
structure of the decision problem.

As an example, consider the
simplified game of \emph{Kuhn poker}~\citep{Kuhn50:Simplified}, depicted
in~\cref{fig:kuhn}. Kuhn poker is a standard benchmark in the EFG-solving community.
In Kuhn poker, each player puts an ante worth $1$ into the pot. Each player is then privately dealt one card from a deck that contains $3$ unique cards (Jack, Queen, King). Then, a single round of betting then occurs, with the following dynamics. First, Player $1$ decides to either check or bet $1$. Then,
\begin{itemize}[nolistsep]
    \item If Player 1 checks Player 2 can check or raise $1$.
          \begin{itemize}[nolistsep]
              \item If Player 2 checks a showdown occurs; if Player 2 raises Player 1 can fold or call.
                    \begin{itemize}
                        \item If Player 1 folds Player 2 takes the pot; if Player 1 calls a showdown occurs.
                    \end{itemize}
          \end{itemize}
    \item If Player 1 raises Player 2 can fold or call.
          \begin{itemize}[nolistsep]
              \item If Player 2 folds Player 1 takes the pot; if Player 2 calls a showdown occurs.
          \end{itemize}
\end{itemize}
When a showdown occurs, the player with the higher card wins the pot and the game immediately ends.

\begin{figure}[th]
    \centering
    \begin{tikzpicture}[scale=1.0]
    \tikzset{edge from parent/.style={}}
    \tikzset{edge from parent path={(\tikzparentnode) -- (\tikzchildnode.north)}}
    \tikzset{level distance=1.05cm}
    \tikzset{sibling distance=.60cm}
    \Tree
     [.\node[obspt](P1){};
      [.\node[decpt](S1) {};
       [.\node[obspt](B1) {};
        [.\node[termina](T1) {};]
        [.\node[decpt](S2) {};
         [.\node[termina](T2) {};]
         [.\node[termina](T3) {};]
        ]
       ]
       [.\node[termina](S3) {};]
      ]
      [.\node[decpt](S4) {};
       [.\node[obspt](B2) {};
        [.\node[termina](T6) {};]
        [.\node[decpt](S5) {};
         [.\node[termina](T7) {};]
         [.\node[termina](T8) {};]
        ]
       ]
       [.\node[termina](S6) {};]
      ]
      [.\node[decpt](S7) {};
       [.\node[obspt](B3) {};
        [.\node[termina](T11) {};]
        [.\node[decpt](S8) {};
         [.\node[termina](T12) {};]
         [.\node[termina](T13) {};]
        ]
       ]
       [.\node[termina](S9) {};]
      ]
    ];
    
   \node[black!70!white,xshift=-4mm,yshift=2mm] at (P1) {$k_1$};
   \node[black!70!white,xshift=-4mm] at (S1) {$j_1$};
   \node[black!70!white,xshift=-4mm] at (S4) {$j_2$};
   \node[black!70!white,xshift=4mm] at (S7) {$j_3$};
   \node[black!70!white,xshift=-4mm] at (B1) {$k_2$};
   \node[black!70!white,xshift=-4mm] at (B2) {$k_3$};
   \node[black!70!white,xshift=-4mm] at (B3) {$k_4$};
   \node[black!70!white,xshift=4mm] at (S2) {$j_4$};
   \node[black!70!white,xshift=4mm] at (S5) {$j_5$};
   \node[black!70!white,xshift=4mm] at (S8) {$j_6$};
   
   \draw[semithick,dashed] (P1) -- (S1);
   \draw[semithick,dashed] (P1) -- (S4);
   \draw[semithick,dashed] (P1) -- (S7);
   \draw[semithick] (S1) -- (B1);
   \draw[semithick] (S1) -- (S3);
   \draw[semithick] (S4) -- (B2);
   \draw[semithick] (S4) -- (S6);
   \draw[semithick] (S7) -- (B3);
   \draw[semithick] (S7) -- (S9);
   \draw[semithick,dashed] (B1) -- (T1);
   \draw[semithick,dashed] (B1) -- (S2);
   \draw[semithick,dashed] (B2) -- (T6);
   \draw[semithick,dashed] (B2) -- (S5);
   \draw[semithick,dashed] (B3) -- (T11);
   \draw[semithick,dashed] (B3) -- (S8);
   
   \draw[semithick] (T2) -- (S2) -- (T3);
   \draw[semithick] (T7) -- (S5) -- (T8);
   \draw[semithick] (T12) -- (S8) -- (T13);


   \path ($(S2)+(-1mm,-3mm)$) --node[text=black,fill=white,inner ysep=.5mm,inner xsep=0,xshift=-1mm,yshift=1mm]{\small fold} (T2);
   \path ($(S2)+(1mm,-3mm)$) --node[text=black,fill=white,inner ysep=.5mm,inner xsep=0,xshift=1mm,yshift=1mm]{\small call} (T3);
   \path ($(S5)+(-1mm,-3mm)$) --node[text=black,fill=white,inner ysep=.5mm,inner xsep=0,xshift=-1mm,yshift=1mm]{\small fold} (T7);
   \path ($(S5)+(1mm,-3mm)$) --node[text=black,fill=white,inner ysep=.5mm,inner xsep=0,xshift=1mm,yshift=1mm]{\small call} (T8);
   \path ($(S8)+(-1mm,-3mm)$) --node[text=black,fill=white,inner ysep=.5mm,inner xsep=0,xshift=-1mm,yshift=1mm]{\small fold} (T12);
   \path ($(S8)+(1mm,-3mm)$) --node[text=black,fill=white,inner ysep=.5mm,inner xsep=0,xshift=1mm,yshift=1mm]{\small call} (T13);
   \path (S1) --node[text=black,fill=white,inner ysep=.5mm,inner xsep=0,yshift=1mm]{\small check} (B1);
   \path (S1) --node[text=black,fill=white,inner ysep=.5mm,inner xsep=0,yshift=1mm]{\small raise} (S3);
   \path (S4) --node[text=black,fill=white,inner ysep=.5mm,inner xsep=0,yshift=1mm]{\small check} (B2);
   \path (S4) --node[text=black,fill=white,inner ysep=.5mm,inner xsep=0,yshift=1mm]{\small raise} (S6);
   \path (S7) --node[text=black,fill=white,inner ysep=.5mm,inner xsep=0,yshift=1mm]{\small check} (B3);
   \path (S7) --node[text=black,fill=white,inner ysep=.5mm,inner xsep=0,yshift=1mm]{\small raise} (S9);

   \path (P1) --node[text=black,inner ysep=1mm,fill=white]{\small jack} (S1);
   \path (P1) --node[text=black,inner ysep=.3mm,fill=white]{\small queen} (S4);
   \path (P1) --node[text=black,inner ysep=1mm,fill=white]{\small king} (S7);

   \path (B1) --node[text=black,fill=white,inner ysep=.5mm,inner xsep=0,xshift=-1mm,yshift=1mm]{\small check} (T1);
   \path (B1) --node[text=black,fill=white,inner ysep=.5mm,inner xsep=0,xshift=1mm,yshift=1mm]{\small raise} ($(S2)$);
   \path (B2) --node[text=black,fill=white,inner ysep=.5mm,inner xsep=0,xshift=-1mm,yshift=1mm]{\small check} (T6);
   \path (B2) --node[text=black,fill=white,inner ysep=.5mm,inner xsep=0,xshift=1mm,yshift=1mm]{\small raise} ($(S5)$);
   \path (B3) --node[text=black,fill=white,inner ysep=.5mm,inner xsep=0,xshift=-1mm,yshift=1mm]{\small check} (T11);
   \path (B3) --node[text=black,fill=white,inner ysep=.5mm,inner xsep=0,xshift=1mm,yshift=1mm]{\small raise} ($(S8)$);
\end{tikzpicture}

    \caption{Tree-form sequential decision making process of the first
        acting player in the game of Kuhn poker.}
    \label{fig:kuhn}
\end{figure}

As soon as the game starts, the agent observes a private card that has been
dealt to them; this is observation point $k_1$, whose set of possible
signals is $S_{k_1} \defeq \{\text{jack},\text{queen},\text{king}\}$.
Should the agent observe the `jack' signal, the decision problem transitions to
the decision point $j_1$, where the agent must pick one action from the set
$A_{j_1} \defeq \{\text{check}, \text{raise}\}$.
If the agent picks `raise', the decision process terminates; otherwise, if
`check' is chosen, the process transitions to observation point $k_2$,
where the agent will observe whether the opponent checks (at which point
the interaction terminates) or raises.
In the latter case, the process transitions to decision point $j_4$, where the
agent picks one action from the set
$A_{j_4} \defeq \{\text{fold},\text{call}\}$.
In either case, after the action has been selected, the interaction terminates.


