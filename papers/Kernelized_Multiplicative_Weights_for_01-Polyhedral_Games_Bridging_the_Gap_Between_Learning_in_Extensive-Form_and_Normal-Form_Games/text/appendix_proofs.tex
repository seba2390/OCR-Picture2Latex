\section{Proofs}\label{app:proofs}


\thmefgkernel*
\begin{proof}
    In the proof of this result, we will make use of the following additional notation.
    Given any $\vx \in \bbR^{\Sigma_i}$ and a $j\in \cJ_i$, we let $\vx_{(j)} \in \bbR^{\Sigma_{i,j}^*}$ denote the subvector obtained from $\vx$ by only considering sequences $\sigma \in \Sigma_{i,j}^*$, that is, the vector whose entries are defined as $\vx_{(j)}[\sigma] = \vx[\sigma]$ for all $\sigma \in \Sigma_{i,j}^*$.

    \paragraph{Proof of~\eqref{eq:efg kernel computation}}
    Direct inspection of the definitions of $\Pi_i$ and $\Pi_{i,j}$ (given in \cref{sec:efg notation}), together with the observation that the $\{\Sigma_{i,j}^* : j \in \mathcal{C}_\emptyseq\}$ form a partition of $\Sigma^*_i$, reveals that
    \[
        \Pi_i = \mleft\{\vpi \in \{0,1\}^{\Sigma_i}: \begin{array}{l}\circled{1}~~\vpi[\emptyseq] = 1\\[1mm] \circled{2}~~\vpi_{(j)} \in \Pi_{i,j} \qquad\forall\, j\in\mathcal{C}_\emptyseq \end{array} \mright\}
        \numberthis{eq:Pi i as prod}
    \]
    The observation above can be summarized informally into the statement that ``\emph{$\Pi_i$ is equal, up to permutation of indices, to the Cartesian product $\bigtimes_{j\in \mathcal{C}_\emptyseq}\Pi_{i,j}$}''.
    The idea for the proof is then to use that Cartesian product structure in the definition of 0/1-polyhedral kernel~\eqref{eq:K Omega}, as follows
    \[
        K_{Q_i}(\vx, \vy) &= \sum_{\vec{\pi} \in \Pi_i} \prod_{\sigma \in \vpi} \vx[\sigma]\,\vy[\sigma]\\
        &=\sum_{\vpi\in\Pi_i}\mleft(\vx[\emptyseq]\,\vy[\emptyseq]\prod_{j'\in\mathcal{C}_\emptyseq}\prod_{\sigma\in\vpi_{(j')}} \vx[\sigma]\,\vy[\sigma]\mright)\\
        &=\sum_{\vpi_{(j)}\in\Pi_{i,j} ~\forall\,j \in \mathcal{C}_\emptyseq}\mleft(\vx[\emptyseq]\,\vy[\emptyseq] \prod_{j'\in\mathcal{C}_\emptyseq}\prod_{\sigma\in\vpi_{(j')}} \vx[\sigma]\,\vy[\sigma]\mright)\\
        &=\vx[\emptyseq]\,\vy[\emptyseq] \sum_{\vpi_{(j)}\in\Pi_{i,j} ~\forall\,j \in \mathcal{C}_\emptyseq}\mleft(\prod_{j'\in\mathcal{C}_\emptyseq}\prod_{\sigma\in\vpi_{(j')}} \vx[\sigma]\,\vy[\sigma]\mright)\\
        &=\vx[\emptyseq]\,\vy[\emptyseq] \prod_{j\in\mathcal{C}_\emptyseq} \sum_{\vpi_{(j)}\in\Pi_{i,j}}\prod_{\sigma\in\vpi_{(j)}} \vx[\sigma]\,\vy[\sigma]\\
        &= \vx[\emptyseq]\,\vy[\emptyseq] \prod_{j\in\mathcal{C}_\emptyseq} K_{j}(\vx, \vy),
    \]
    where the second equality follows from the fact that $\{\emptyseq\}\cup\{\Sigma_{i,j}:j\in\mathcal{C}_\emptyseq\}$ form a partition of $\Sigma_i$, the third equality follows from~\eqref{eq:Pi i as prod}, the fifth equality from the fact that each $\vpi_j\in\Pi_{i,j}$ can be chosen independently, and the last equality from the definition of partial kernel function~\eqref{eq:def Kj}.

    \paragraph{Proof of~\eqref{eq:efg kernel computation 2}}
    Similarly to what we did for~\eqref{eq:efg kernel computation}, we start by giving an inductive characterization of $\Pi_{i,j}$ as a function of the children $\Pi_{i,j'}$ for $j' \in \cup_{a\in A_j} \mathcal{C}_{ja}$. Specifically, direct inspection of the definitions of $\Pi_{i,j}$, together with the observation that the $\{\Sigma_{i,j'}^* : j' \in \cup_{a \in A_j}\mathcal{C}_{ja}\}$ form a partition of $\Sigma_{i,j}^*$, reveals that
    \[
        \Pi_{i,j} = \mleft\{\vpi \in \{0,1\}^{\Sigma_{i,j}^*}: \begin{array}{l}\circled{1}~~\sum_{a \in A_j}\vpi[ja] = 1\\[1mm] \circled{2}~~\vpi_{(j')} \in \vpi[ja]\cdot \Pi_{i,j'} \qquad\forall\, a\in A_j,~ j'\in\mathcal{C}_{ja} \end{array} \mright\}.
        \numberthis{eq:Pi j as ch prod intermediate}
    \]
    From constraint \circled{1}\, together with the fact that $\vpi[ja]\in\{0,1\}$ for all $a \in A_j$, we conclude that exactly one $a^* \in A_j$ is such that $\vpi[ja^*] = 1$, while $\vpi[ja] = 0$ for all other $a \in A_j, a \neq a^*$. So, we can rewrite~\eqref{eq:Pi j as ch prod intermediate} as
    \[
        \Pi_{i,j} = \bigcup_{a^* \in A_j}\mleft\{
        \vpi\in\{0,1\}^{\Sigma_{i,j}^*}: \begin{array}{ll}
            \circled{1}~~\vpi[ja^*] = 1                                                                                      \\
            \circled{2}~~\vpi[ja] = 0             \qquad\qquad & \forall\, a\in A_j, a \neq a^*                              \\
            \circled{3}~~\vpi_{(j')} \in \Pi_{i,j'}            & \forall\, j' \in \mathcal{C}_{ja^*}                         \\
            \circled{4}~~\vpi_{(j')} = \vzero                  & \forall\, j'\in\cup_{a \in A_j, a \neq a^*}\mathcal{C}_{ja} \\
        \end{array}
        \mright\},
        \numberthis{eq:Pi j as ch prod}
    \]
    where the union is clearly disjoint.     The above equality can be summarized informally into the statement that ``\emph{$\Pi_{i,j}$ is equal, up to permutation of indices, to a disjoint union over actions $a^*\in A_j$ of Cartesian products $\bigtimes_{j\in \mathcal{C}_{ja^*}}\Pi_{i,j}$}''.
    We can then use the same set of manipulations we already used in the proof of~\eqref{eq:efg kernel computation} to obtain
    \[
        K_{j}(\vx, \vy) &= \sum_{\vec{\pi} \in \Pi_{i,j}} \prod_{\sigma \in \vpi} \vx[\sigma]\,\vy[\sigma]\\
        &= \sum_{\vec{\pi} \in \Pi_{i,j}} \mleft( \vx[ja^*]\,\vy[ja^*]\prod_{j'\in\mathcal{C}_{ja^*}}\prod_{\sigma\in\vpi_{(j')}} \vx[\sigma]\,\vy[\sigma]\mright)\\
        &=\sum_{a^* \in A_j}\sum_{\vpi_{j'}\in\Pi_{i,j'}~\forall\,j' \in \mathcal{C}_{ja^*}} \mleft( \vx[ja^*]\,\vy[ja^*]\prod_{j'\in\mathcal{C}_{ja^*}}\prod_{\sigma\in\vpi_{(j')}} \vx[\sigma]\,\vy[\sigma]\mright)\\
        &=\sum_{a^* \in A_j} \mleft( \vx[ja^*]\,\vy[ja^*] \prod_{j'\in\mathcal{C}_{ja^*}} \sum_{\vpi_{(j')}\in\Pi_{i,j'}}\prod_{\sigma\in\vpi_{(j')}} \vx[\sigma]\,\vy[\sigma]\mright)\\
        &=\sum_{a \in A_j} \mleft( \vx[ja]\,\vy[ja] \prod_{j' \in \mathcal{C}_{ja}} K_{j'}(\vx, \vy) \mright),
    \]
    where the second equality follows from the fact that the $\{\Sigma_{i,j'}^* : j' \in \cup_{a \in A_j}\mathcal{C}_{ja}\}$ form a partition of $\Sigma_{i,j}^*$, third equality follows from~\eqref{eq:Pi j as ch prod}, the fourth equality from the fact that each $\vpi_{j'}\in\Pi_{i,j'}$ can be picked independently, and the last equality from the definition of partial kernel function~\eqref{eq:def Kj} as well as renaming $a^*$ into $a$.
\end{proof}

\propefgratio*
\begin{proof}
    Note that since $\vx > \vzero$, clearly $K_{Q_i}(\vx, \vone), K_j(\vx, 1) > 0$. Furthermore, from~\eqref{eq:diff phi} we have that for all $\sigma \in \Sigma_i$
    \[
        K_{Q_i}(\vx, \vone) - K_{Q_i}(\vx, \ebar_\sigma) &= \langle \phi_{Q_i}(\vone) - \phi_{Q_i}(\ebar_\sigma), \phi_{Q_i}(\vx)\rangle \\
        &= \sum_{\substack{\vpi \in \Pi_i\\\vpi[\sigma] = 1}}\prod_{\sigma' \in \vpi} \vx[\sigma']
        \numberthis{eq:ratio to diff}\\
        &> 0.
    \]
    The above inequality immediately implies that $0 < K_{Q_i}(\vx,\ebar_{p_j})/K_{Q_i}(\vx, \vone) < 1$ and therefore all denominators in the statement are nonzero, making the statement well-formed.

    \newcommand{\splice}[2]{(\!(#1\,|\,#2)\!)}
    In light of~\eqref{eq:ratio to diff}, we further have
    \[
    & \frac{1 - K_{Q_i}(\vec{x}, \bar{\vec{e}}_{ja}) / K_{Q_i}(\vec{x}, \vone)}{1 - K_{Q_i}(\vec{x}, \bar{\vec{e}}_{p_j}) / K_{Q_i}(\vec{x}, \vone)} = \frac{\vec{x}[ja]\prod_{j'\in\mathcal{C}_{ja}} K_{j'}(\vec{x},\vone)}{K_j(\vec{x}, \vone)}\\[2mm]
    & \hspace{2cm}\iff\quad \frac{K_{Q_i}(\vec{x}, \vone) - K_{Q_i}(\vec{x}, \bar{\vec{e}}_{ja})}{K_{Q_i}(\vec{x}, \vone) - K_{Q_i}(\vec{x}, \bar{\vec{e}}_{p_j})} = \frac{\vec{x}[ja]\prod_{j'\in\mathcal{C}_{ja}} K_{j'}(\vec{x},\vone)}{K_j(\vec{x}, \vone)}\\[2mm]
        & \hspace{2cm}\iff\quad\frac{\sum_{\vpi \in \Pi_i, \vpi[ja] = 1}\prod_{\sigma \in \vpi} \vx[\sigma]}{\sum_{\vpi \in \Pi_i, \vpi[p_j] = 1}\prod_{\sigma \in \vpi} \vx[\sigma]} = \frac{\vec{x}[ja]\prod_{j'\in\mathcal{C}_{ja}} K_{j'}(\vec{x},\vone)}{K_j(\vec{x}, \vone)}\numberthis{eq:equivalent}
    \]
    We now prove~\eqref{eq:equivalent}. Let
    \[
        \mathcal{A} \defeq \{\vpi \in \Pi_i : \vpi[ja]=1\}, \qquad \mathcal{B} \defeq \{\vpi \in \Pi_i : \vpi[p_j]=1\}
    \]
    be the domains of the summations. From the definition of $\Pi_i$ (specifically, constraints \circled{2}~in the definition of $Q_i$, of which $\Pi_i$ is a subset; see \cref{sec:efg notation}), it is clear that $\mathcal{A} \subseteq \mathcal{B}$. Furthermore, it is straightforward to check, using the definitions of $\Pi_{i,j}$, $\Pi_i$, and $\mathcal{B}$, that
    \[
        \vpi_{(j)} \in \Pi_{i,j} \qquad\forall\,\vpi\in\mathcal{B}\numberthis{eq:vpij}
    \]

    We now introduce the function $\splice{\cdot}{\cdot} : \mathcal{B} \times \Pi_{i,j} \to \mathcal{B}$ defined as follows.
    Given any $\vpi \in \mathcal{B}$ and $\vpi' \in \Pi_{i,j}$, $\splice{\vpi}{\vpi'}$ is the vector obtained from $\vpi$ by replacing all sequences at or below decision point $j$ with what is prescribed by $\vpi'$; formally,
    \[
        \splice{\vpi}{\vpi'}[\sigma] \defeq \begin{cases}
            \vpi'[\sigma] & \text{if } \sigma \in \Sigma_{i,j}^* \\
            \vpi[\sigma]  & \text{otherwise}.
        \end{cases} \qquad\quad \forall\, \vpi\in\mathcal{B}, \vpi'\in\Pi_{i,j}
        \numberthis{eq:def splice}
    \]
    It is immediate to check that $\splice{\vpi}{\vpi'}$ is indeed an element of $\mathcal{B}$.
    We now introduce the following result.

    \begin{lemma}\label{obs:B}
        There exists a set $\mathcal{P} \subseteq \mathcal{B}$ such that every $\vpi'' \in \mathcal{B}$ can be uniquely written as $\vpi'' = \splice{\vpi}{\vpi'}$ for some $\vpi \in \mathcal{P}$ and $\vpi' \in \Pi_{i,j}$. Vice versa, given any $\vpi \in \mathcal{P}$ and $\vpi' \in \Pi_{i,j}$, then $\splice{\vpi}{\vpi'} \in \mathcal{B}$.
    \end{lemma}
    \begin{proof}
        The second part of the statement is straightforward. We now prove the first part.

        Fix any $\vpi^* \in \Pi_{i,j}$ and let $\mathcal{P} \defeq \{\splice{\vpi}{\vpi^*} : \vpi \in \mathcal{B}\}$. It is straightforward to verify that for any $\vpi'' \in \mathcal{B}$, the choices $\vpi \defeq \splice{\vpi''}{\vpi^*} \in \mathcal{P}$ and $\vpi' \defeq \vpi_{(j)} \in \Pi_{i,j}$ satisfy the equality $\splice{\vpi}{\vpi'} = \vpi''$. So, every $\vpi''\in\mathcal{B}$ can be expressed in \emph{at least one way} as $\vpi'' = \splice{\vpi}{\vpi'}$ for some $\vpi \in \mathcal{P}$ and $\vpi' \in \Pi_{i,j}$. We now show that the choice above is in fact the unique choice. First, it is clear from the definition of $\splice{\cdot}{\cdot}$ that $\vpi'$ must satisfy $\vpi' = \vpi''_{(j)}$, and so it is uniquely determined. Suppose now that there exist $\vpi, \tilde{\vpi}\in\mathcal{P}$ such that $\splice{\vpi}{\vpi'} = \splice{\tilde{\vpi}}{\vpi'}$. Then, $\vpi$ and $\tilde{\vpi}$ must coincide on all $\sigma \in \Sigma_i \setminus \Sigma_{i,j}^*$. However, since all elements of $\mathcal{P}$ are of the form $\splice{\vb}{\vpi^*}$ for some $\vb \in \mathcal{B}$, then $\vpi$ and $\tilde{\vpi}$ must also coincide on all $\sigma \in\Sigma_{i,j}^*$. So, $\vpi$ and $\tilde{\vpi}$ coincide on all coordinates $\sigma\in\Sigma_i$, and the statement follows.
    \end{proof}

    \cref{obs:B} exposes a convenient combinatorial structure of the set $\mathcal{B}$. In particular, it enables us to rewrite the denominator on the left-hand side of \eqref{eq:equivalent} as follows
    \[
        \sum_{\vpi \in \mathcal{B}} \prod_{\sigma\in\vpi} \vx[\sigma] &= \sum_{\vpi' \in \mathcal{P}}\sum_{\vpi''\in\Pi_{i,j}}\prod_{\sigma\in\splice{\vpi'}{\vpi''}} \vx[\sigma]\\
        &=\sum_{\vpi' \in \mathcal{P}}\sum_{\vpi''\in\Pi_{i,j}}\mleft(\prod_{\substack{\sigma\in\splice{\vpi'}{\vpi''}\\\sigma\in\Sigma_{i,j}}} \vx[\sigma]\mright)\mleft(\prod_{\substack{\sigma\in\splice{\vpi'}{\vpi''}\\\sigma\not\in\Sigma_{i,j}}} \vx[\sigma]\mright)\\
        &=\sum_{\vpi' \in \mathcal{P}}\sum_{\vpi''\in\Pi_{i,j}}\mleft(\prod_{\sigma\in\vpi''} \vx[\sigma]\mright)\mleft(\prod_{\substack{\sigma\in\vpi'\\\sigma\not\in\Sigma_{i,j}}} \vx[\sigma]\mright)\\
        &=\mleft(\sum_{\vpi''\in\Pi_{i,j}}\prod_{\sigma\in\vpi''} \vx[\sigma]\mright)\mleft(\sum_{\vpi' \in \mathcal{P}}\prod_{\substack{\sigma\in\vpi'\\\sigma\not\in\Sigma_{i,j}}} \vx[\sigma]\mright)\\
        &= K_j(\vx, \vone)\cdot \mleft(\sum_{\vpi' \in \mathcal{P}}\prod_{\substack{\sigma\in\vpi'\\\sigma\not\in\Sigma_{i,j}}} \vx[\sigma]\mright),\numberthis{eq:denom}
    \]
    where we used~\eqref{eq:def splice} in the third equality.

    We can use a similar technique to express the numerator of the left-hand side of~\eqref{eq:equivalent}. Let
    \[
        \Pi_{i,ja} \defeq \{\vpi \in \Pi_{i,j} : \vpi[ja] = 1\}.
    \]
    Using the constraints that define $\Pi_i$ and the definition of $\mathcal{A}$, it follows immediately that for any $\vpi\in\mathcal{A}$, $\vpi_{(j)} \in \Pi_{i,ja}$. Furthermore, a direct consequence of \cref{obs:B} is the following:
    \begin{corollary}\label{obs:A}
        The same set $\mathcal{P} \subseteq \mathcal{B}$ introduced in \cref{obs:B} is such that every $\vpi'' \in \mathcal{A}$ can be uniquely written as $\vpi'' = \splice{\vpi}{\vpi'}$ for some $\vpi \in \mathcal{P}$ and $\vpi' \in \Pi_{i,ja}$.
    \end{corollary}
    Using \cref{obs:A} and following the same steps that led to~\eqref{eq:denom}, we express the numerator of the left-hand side of~\eqref{eq:equivalent} as
    \[
        \sum_{\vpi \in \mathcal{A}} \prod_{\sigma\in\vpi} \vx[\sigma] &= \sum_{\vpi' \in \mathcal{P}}\sum_{\vpi''\in\Pi_{i,ja}}\prod_{\sigma\in\splice{\vpi'}{\vpi''}} \vx[\sigma]\\
        &=\sum_{\vpi' \in \mathcal{P}}\sum_{\vpi''\in\Pi_{i,ja}}\mleft(\prod_{\substack{\sigma\in\splice{\vpi'}{\vpi''}\\\sigma\in\Sigma_{i,j}}} \vx[\sigma]\mright)\mleft(\prod_{\substack{\sigma\in\splice{\vpi'}{\vpi''}\\\sigma\not\in\Sigma_{i,j}}} \vx[\sigma]\mright)\\
        &=\sum_{\vpi' \in \mathcal{P}}\sum_{\vpi''\in\Pi_{i,ja}}\mleft(\prod_{\sigma\in\vpi''} \vx[\sigma]\mright)\mleft(\prod_{\substack{\sigma\in\vpi'\\\sigma\not\in\Sigma_{i,j}}} \vx[\sigma]\mright)\\
        &=\mleft(\sum_{\vpi''\in\Pi_{i,ja}}\prod_{\sigma\in\vpi''} \vx[\sigma]\mright)\mleft(\sum_{\vpi' \in \mathcal{P}}\prod_{\substack{\sigma\in\vpi'\\\sigma\not\in\Sigma_{i,j}}} \vx[\sigma]\mright).
        \numberthis{eq:numer intermediate}
    \]
    The statement then follows immediately if we can prove that
    \[
        \sum_{\vpi\in\Pi_{i,ja}}\prod_{\sigma\in\vpi} \vx[\sigma] = \vec{x}[ja]\,\prod_{j'\in\mathcal{C}_{ja}} K_{j'}(\vx, \vone).
    \]
    To do so, we use the same approach as in the proof of \cref{thm:efg kernel computation}. In fact, we can directly use the inductive characterization of $\Pi_{i,j}$ obtained in~\eqref{eq:Pi j as ch prod} to write
    \[
        \Pi_{i,ja} = \mleft\{
        \vpi\in\{0,1\}^{\Sigma_{i,j}^*}: \begin{array}{ll}
            \circled{1}~~\vpi[ja] = 1                                                                                                \\
            \circled{2}~~\vpi[ja'] = 0             \qquad\qquad & \forall\, a'\in A_j                                    , a' \neq a \\
            \circled{3}~~\vpi_{(j')} \in \Pi_{i,j'}             & \forall\, j' \in \mathcal{C}_{ja}                                  \\
            \circled{4}~~\vpi_{(j')} = \vzero                   & \forall\, j'\in\cup_{a' \in A_j, a' \neq a}\mathcal{C}_{ja'}       \\
        \end{array}
        \mright\},
    \]
    which fundamentally uncovers the \emph{Cartesian-product structure of $\Pi_{i,ja}$}. Using the same technique as \cref{thm:efg kernel computation}, we then have
    \[
        \sum_{\vpi\in\Pi_{i,ja}}\prod_{\sigma\in\vpi} \vx[\sigma] &=
        \sum_{\vpi_{(j')}\in\Pi_{i,j'}~\forall\,j' \in \mathcal{C}_{ja}} \mleft( \vx[ja]\prod_{j'\in\mathcal{C}_{ja}}\prod_{\sigma\in\vpi_{(j')}} \vx[\sigma]\mright)\\
        &= \mleft( \vx[ja] \prod_{j'\in\mathcal{C}_{ja}} \sum_{\vpi_{(j')}\in\Pi_{i,j'}}\prod_{\sigma\in\vpi_{(j')}} \vx[\sigma]\mright)\\
        &= \mleft( \vx[ja] \prod_{j' \in \mathcal{C}_{ja}} K_{j'}(\vx, \vone) \mright),
    \]
    and the statement is proven.
\end{proof}

\propnumvertices*
\begin{proof}
    The proof is by induction. As the base case consider a single decision point $\Delta^b$ with $b \leq A$ actions. Then the number of vertices is $b \leq A = A^{\|\Delta^b\|_1}$.

    For the induction step we consider two cases.
    First, consider a polytope $Q$ whose root is a decision point with $b\leq A$ actions, with each action $a$ leading to a polytope $Q_a$ whose number of vertices $v_a$ satisfies the inductive assumption (if some action $a$ is a terminal action then we overload notation and let $v_a=1$ and $\|Q_a\|_1 = 0$).
    Then, the number of vertices of $Q$ is
    \[
        \sum_{a=1}^b v_a
        &\leq \sum_{a=1}^b A^{\|Q_a\|_1} \\
        &\leq b \cdot A^{\max_{a\in \range{b}}\|Q_a\|_1}\\
        &\leq A\cdot A^{\max_{a\in \range{b}}\|Q_a\|_1} \\
        &= A^{\|Q\|_1}.
    \]

    Second, consider a polytope $Q$ whose root is an observation point with $b$ observations, with each observation $o$ leading to a polytope $Q_o$ with $v_o$ vertices, such that the inductive assumption holds.
    Then, the number of vertices of $Q$ is
    \[
        v = \prod_{o=1}^b v_o
        &\leq \prod_{o=1}^b A^{\|Q_o\|_1}
        \leq A^{\sum_{o=1}^b \|Q_o\|_1}
        = A^{\|Q\|_1}.
    \]
\end{proof}

