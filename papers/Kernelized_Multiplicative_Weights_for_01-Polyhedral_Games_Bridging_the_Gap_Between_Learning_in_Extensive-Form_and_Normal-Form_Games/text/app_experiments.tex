\section{Experimental Evaluation}\label{app:experiments}

\paragraph{Game instances}
We numerically investigate agents learning under the COLS in Kuhn and Leduc poker \citep{Kuhn50:Simplified,Southey05:Bayes}, standard benchmark games from the extensive-form games literature.
\begin{description}
    \item[\emph{Kuhn poker}] The two-player variant of Kuhn poker first appeared in \citep{Kuhn50:Simplified}. In this paper, we use the multiplayer variant, as described by \citet{Farina18:Ex}. In a multiplayer Kuhn poker game with $r$ ranks, a deck with $r$ unique cards is used. At the beginning of the game, each player pays one chip to the pot (\emph{ante}), and is dealt a single private card (their \emph{hand}). The first player to act can \emph{check} or \emph{bet}, \ie put an additional chip in the pot. Then, the second player can check or bet after a first player's check, or fold/call the first player's bet. If no bet was previously made, the third player can either check or bet, and so on in turn. If a bet is made by a player, each subsequent player needs to decide whether to \emph{fold} or \emph{call} the bet. The betting round if all players check, or if every player has had an opportunity to either fold or call the bet that was made. The player with the highest card who has not folded wins all the chips in the pot.
    \item[\emph{Leduc poker}] We use a multiplayer version of the classical Leduc hold'em poker introduced by \citet{Southey05:Bayes}. We employ game instances of rank 3. The deck consists of three suits with 3 cards each. Our instances are parametric in the maximum number of bets, which in limit hold'em is not necessarily tied to the number of players. As in Kuhn poker, we set a cap on the number of raises to one bet. As the game starts, players pay one chip to the pot. Then, two betting rounds follow. In the first one, a single private card is dealt to each player while in the second round a single board card is revealed. The raise amount is set to 2 and 4 in the first and second round, respectively.
\end{description}
For each game, we consider a 3-player and a 4-player variant. The 3-player Kuhn variant uses a deck with $r=12$ ranks. The 4-player variant uses a deck with a reduced number of ranks equal to $r=5$ to avoid excessive memory usage.

\paragraph{CFR and CFR(RM+)} Modern variants of counterfactual regret minimization (CFR) are the current practical state-of-the-art in two-player zero-sum extensive-form game solving. We implemented both the original CFR algorithm by~\citet{Zinkevich07:Regret}, and a more modern variant (which we denote `CFR(RM+)') using the Regret Matching Plus regret minimization algorithm at each decision point~\citep{Tammelin15:Solving}.

\paragraph{Discussion of results}
We compare the maximum per-player regret cumulated by KOMWU for four different choices of constant learning rate $\eta\^t = \eta \in \{ 0.1, 1, 5, 10\}$, against that cumulated by CFR and CFR(RM+).

We remark that the payoff ranges of these games are not $[0,1]$ (\ie the games have not been normalized). The payoff range of Kuhn poker is $6$ for the 3-player variant and $8$ for the 4-player variant. The payoff range of Leduc poker is $21$ for the 3-player variant and $28$ for the 4-player variant. So, a learning rate value of $\eta=0.1$ corresponds to a significantly smaller learning rate in the normalized game where the payoffs have been shifted and rescaled to lie within $[0,1]$ as required in the statements of \cref{prop:omwu near optimal,prop:omwu optimal sum,prop:omwu last iterate}.

Results are shown in \cref{fig:all games}. In all games, we observe that the maximum per-player regret cumulated by KOMWU plateaus and remains constants, unlike the CFR variants. This behavior is consistent with the near-optimal per-player regret guarantees of KOMWU (\cref{thm:komwu efg}). In the 3-player variant of Leduc poker, we observe that the largest learning rate we use, $\eta=10$, leads to divergent behavior of the learning dynamics.

\begin{figure}[H]\centering
    \includegraphics[scale=.85]{figs/all-crop.pdf}
    \caption{Maximum per-player regret cumulated by KOMWU for four different choices of constant learning rate $\eta\^t = \eta \in \{ 0.1, 1, 5, 10\}$, compared to that cumulated by CFR and CFR(RM+) in two multiplayer poker games.}
    \label{fig:all games}
\end{figure}