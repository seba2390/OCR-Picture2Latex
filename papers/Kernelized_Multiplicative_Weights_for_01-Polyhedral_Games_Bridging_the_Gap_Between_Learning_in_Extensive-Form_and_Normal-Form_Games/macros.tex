\theoremstyle{plain}
\newtheorem{theorem}{Theorem}[section]
\newtheorem{lemma}[theorem]{Lemma}
\newtheorem{corollary}[theorem]{Corollary}
\newtheorem{proposition}[theorem]{Proposition}
\newtheorem{fact}[theorem]{Fact}
\newtheorem{conjecture}[theorem]{Conjecture}
\newtheorem{observation}[theorem]{Observation}
\newtheorem{claim}[theorem]{Claim}
\newtheorem{assumption}[theorem]{Assumption}
\newtheorem{property}[theorem]{Property}

\theoremstyle{definition}
\newtheorem{definition}[theorem]{Definition}

\theoremstyle{remark}
\newtheorem{remark}[theorem]{Remark}
\newtheorem{example}[theorem]{Example}

\newcommand*{\bbQ}{{\mathbb{Q}}}
\newcommand*{\bbN}{{\mathbb{N}}}
\newcommand*{\bbZ}{{\mathbb{Z}}}
\newcommand*{\bbR}{{\mathbb{R}}}
\newcommand*{\bbF}{{\mathbb{F}}}
\newcommand*{\bbC}{{\mathbb{C}}}
\newcommand{\cV}{\mathcal{V}}

\newcommand*{\bone}{{\mathds{1}}}
\newcommand*{\cA}{{\mathcal{A}}}
\newcommand*{\cM}{{\mathcal{M}}}
\newcommand*{\cJ}{{\mathcal{J}}}
\newcommand*{\cK}{{\mathcal{K}}}
\newcommand*{\eps}{{\epsilon}}
\newcommand*{\cI}{{\mathcal{I}}}
\newcommand*{\cB}{{\mathcal{B}}}
\newcommand*{\cR}{{\mathcal{R}}}
\newcommand*{\tP}{{\tilde{P}}}
\newcommand*{\Hess}[1]{\nabla^2 #1}
\newcommand*{\defeq}{\coloneqq}

\let\poly\relax
\let\polylog\relax
\let\co\relax

\DeclareMathOperator{\poly}{poly}
\DeclareMathOperator{\reg}{Reg}
\DeclareMathOperator{\co}{co}
\DeclareMathOperator{\var}{Var}
\DeclareMathOperator{\polylog}{polylog}

\newcommand*{\intphi}{\Phi_\mathrm{int}}
\newcommand*{\obsut}{\textsc{ObserveUtility}}
\newcommand*{\nextstr}{\textsc{NextStrategy}}
\newcommand*{\fp}{\textsc{FixedPoint}}
\newcommand*{\intreg}{\mathrm{IntReg}}
\newcommand*{\bigOh}{\mathcal{O}}

\renewcommand{\vec}[1]{\bm{#1}}
\newcommand*{\mat}[1]{\mathbf{#1}}

\newcommand*{\bv}{\vec{v}}

\newcommand*{\range}[1]{[\![#1]\!]}
\newcommand*{\edge}[2]{#1\!\to\!#2}
\newcommand*{\bigL}{\vec{\mathcal{L}}}

\makeatletter
\tikzset{
  fitting node/.style={
      inner sep=0pt,
      fill=none,
      draw=none,
      reset transform,
      fit={(\pgf@pathminx,\pgf@pathminy) (\pgf@pathmaxx,\pgf@pathmaxy)}
    },
  reset transform/.code={\pgftransformreset}
}
\tikzset{cross/.style={path picture={
          \draw[black]
          (path picture bounding box.south east) -- (path picture bounding box.north west) (path picture bounding box.south west) -- (path picture bounding box.north east);
        }}}
\tikzstyle{ox}=[semithick,draw=black,circle,cross,inner sep=1.2mm]
\makeatother

\newcommand{\declarecolor}[2]{\definecolor{#1}{RGB}{#2}\expandafter\newcommand\csname #1\endcsname[1]{\textcolor{#1}{##1}}}
\declarecolor{White}{255, 255, 255}
\declarecolor{Black}{0, 0, 0}
\declarecolor{Maroon}{128, 0, 0}
\declarecolor{Coral}{255, 127, 80}
\declarecolor{Red}{182, 21, 21}
\declarecolor{LimeGreen}{50, 205, 50}
\declarecolor{DarkGreen}{0, 100, 0}
\declarecolor{Navy}{0, 0, 128}

\NewDocumentCommand{\numberthis}{om}{%
  \IfNoValueTF{#1}{%
    \refstepcounter{equation}\tag{\theequation}%
  }{%
    \tag{#1}%
  }%
  \label{#2}%
}

\newcommand{\timehat}[1]{^{(#1)}}

\newtoks\mymathaccents
\mymathaccents={%
\let\^\timehat
}

\everymath=\expandafter\expandafter\expandafter{%
  \expandafter\the\expandafter\everymath\the\mymathaccents}

\everydisplay=\expandafter\expandafter\expandafter{%
  \expandafter\the\expandafter\everydisplay\the\mymathaccents}

\def\[#1\]{%
\begin{align*}#1\end{align*}%
}
\newcommand{\ebar}{\bar{\vec{e}}}
\usepackage{dsfont}
\newcommand{\bbone}{\mathds{1}}
\newcommand{\vb}{\vec{b}}
\newcommand{\vv}{\vec{v}}
\newcommand{\vs}{\vec{s}}
\newcommand{\vx}{\vec{x}}
\newcommand{\vpi}{\vec{\pi}}
\newcommand{\vy}{\vec{y}}
\newcommand{\vw}{\vec{w}}
\newcommand{\vlam}{\vec{\lambda}}
\newcommand{\vone}{\vec{1}}
\newcommand{\vm}{\vec{m}}
\newcommand{\vl}{\vec{\ell}}
\newcommand{\vzero}{\vec{0}}
\newcommand{\KL}[2]{D_\mathrm{KL}(#1\,\|\,#2)}
\newcommand{\emptyseq}{\varnothing}
\newcommand{\naive}{na\"ive\xspace}
\newcommand{\Naive}{Na\"ive\xspace}
\newcommand{\ie}{\emph{i.e.},~}
\newcommand{\eg}{\emph{e.g.},~}

\DeclareMathOperator{\ran}{rg}

\newcommand{\vvstar}{\vv^{\star}}
\newcommand{\vlamstar}{\vlam^{\star}}
\newcommand{\inner}[1]{\left\langle#1\right\rangle}
\newcommand{\nm}[1]{\left\|#1\right\|}
\newcommand{\Rpp}{\bbR_{>0}}
\newcommand{\Npp}{\bbN_{>0}}
\DeclareMathOperator*{\E}{\mathbb{E}}


\usetikzlibrary{fit,shapes.misc,arrows.meta,calc,positioning}
\makeatletter
\tikzset{
  fitting node/.style={
      inner sep=0pt,
      fill=none,
      draw=none,
      reset transform,
      fit={(\pgf@pathminx,\pgf@pathminy) (\pgf@pathmaxx,\pgf@pathmaxy)}
    },
  reset transform/.code={\pgftransformreset}
}
\tikzset{cross/.style={path picture={
          \draw[black]
          (path picture bounding box.south east) -- (path picture bounding box.north west) (path picture bounding box.south west) -- (path picture bounding box.north east);
        }}}
\tikzstyle{ox}=[semithick,draw=black,circle,cross,inner sep=1.2mm]
\pgfdeclarelayer{background}
\pgfsetlayers{background,main}
\makeatother

\crefname{property}{Property}{Properties}


\tikzset{cross/.style={path picture={
          \draw[black]
          (path picture bounding box.south east) -- (path picture bounding box.north west) (path picture bounding box.south west) -- (path picture bounding box.north east);
        }}}
\tikzstyle{chanode}   = [fill=white,draw=black,circle,cross,inner sep=.8mm]
\tikzstyle{pl1node}   = [fill=black,draw=black,circle,inner sep=.55mm]
\tikzstyle{pl2node}   = [fill=white,draw=black,circle,inner sep=.55mm]
\tikzstyle{termina}   = [fill=white,draw=black,inner sep=.6mm]
\tikzstyle{decpt}     = [fill=black,draw=black,inner sep=.8mm]
\tikzstyle{obspt}     = [fill=white,draw=black,cross,inner sep=0.95mm]
\tikzstyle{highlight} = [line width=1.99]
\tikzstyle{infoset}   = [thin,draw=black!30!white,fill=black!5]

\newcommand{\nset}{{\Omega^d_n}}\label{sec:nset}
