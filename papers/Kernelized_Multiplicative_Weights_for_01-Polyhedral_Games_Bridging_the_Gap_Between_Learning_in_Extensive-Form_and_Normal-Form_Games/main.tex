\documentclass[letterpaper]{article}

\usepackage[table]{xcolor}
\usepackage{microtype}
\usepackage{graphicx}
\usepackage{hyperref}
\usepackage{booktabs}
\usepackage[accepted]{icml2022}
\usepackage{amsmath}
\usepackage{amssymb}
\usepackage{amsthm}
\usepackage{thmtools}

\usepackage{bm}
\usepackage{nicefrac}
\usepackage{tikz}
\usepackage{tikz-qtree}
\usepackage{nicerbb}
\usepackage{mathtools}

\usepackage{xparse}
\usepackage{float}
\usepackage{mleftright}

\usepackage{thm-restate}
\usepackage{etoolbox}
\usepackage{xspace}

\usepackage[capitalize,noabbrev]{cleveref}
\usepackage[shortlabels]{enumitem}

\usepackage[ruled,vlined,linesnumbered]{algorithm2e}
\makeatletter
\patchcmd\algocf@Vline{\vrule}{\vrule \kern-0.4pt}{}{}
\patchcmd\algocf@Vsline{\vrule}{\vrule \kern-0.4pt}{}{}
\makeatother

\SetKwComment{Hline}{}{\vspace{-3mm}\textcolor{gray}{\hrule}\vspace{1mm}}
\definecolor{darkgrey}{gray}{0.3}
\definecolor{commentcolor}{gray}{0.5}
\SetKwComment{Comment}{\color{commentcolor}[$\triangleright$\ }{}
\SetCommentSty{}
\SetNlSty{}{\color{darkgrey}}{}
\setlength{\algomargin}{4mm}
\SetKwProg{Fn}{function}{}{}
\SetKwProg{Subr}{subroutine}{}{}
\crefalias{AlgoLine}{line}%
\crefname{algocf}{Algorithm}{Algorithms}

\makeatletter
\let\cref@old@stepcounter\stepcounter
\def\stepcounter#1{%
  \cref@old@stepcounter{#1}%
  \cref@constructprefix{#1}{\cref@result}%
  \@ifundefined{cref@#1@alias}%
    {\def\@tempa{#1}}%
    {\def\@tempa{\csname cref@#1@alias\endcsname}}%
  \protected@edef\cref@currentlabel{%
    [\@tempa][\arabic{#1}][\cref@result]%
    \csname p@#1\endcsname\csname the#1\endcsname}}
\makeatother

\newcommand{\figref}[1]{Fig.~\ref{#1}}
\newcommand{\tblref}[1]{Table~\ref{#1}}
\newcommand{\secref}[1]{Section~\ref{#1}}
\renewcommand{\eqref}[1]{Equation~(\ref{#1})}

\def\availableat{\url{url-published-on-acceptance}}

\newcommand{\todo}[1]{{\color{red} TODO: {#1}}}
\newcommand{\newstuff}[1]{{\color{red} CHECK: {#1}}}
%\newcommand{\todo}[1]{{}}

\newcommand{\ckp}[2]{$CK_{#1}P_{#2}$}
\newcommand{\cext}{\ckp{8}{16}$ext$}
\newcommand{\cfin}{$F_{CK_{X}P_{Y}}$}
\newcommand{\cray}{\ckp{8}{8}$ray$}
\newcommand{\csin}{$SK_{8}P_{8}$}
\newcommand{\casin}{$SK_{combined}$}
%\renewcommand{\cext}{$SK_{8}K_{8}P_{8}$}
\newcommand{\ckpnl}[2]{$CK_{#1}P_{#2}nl$}


\makeatletter
\newcommand{\Spvek}[2][r]{%
	\gdef\@VORNE{1}
	\left(\hskip-\arraycolsep%
	\begin{array}{#1}\vekSp@lten{#2}\end{array}%
	\hskip-\arraycolsep\right)}

\def\vekSp@lten#1{\xvekSp@lten#1;vekL@stLine;}
\def\vekL@stLine{vekL@stLine}
\def\xvekSp@lten#1;{\def\temp{#1}%
	\ifx\temp\vekL@stLine
	\else
	\ifnum\@VORNE=1\gdef\@VORNE{0}
	\else\@arraycr\fi%
	#1%
	\expandafter\xvekSp@lten
	\fi}
\makeatother

\newcommand{\gabri}[1]{\textcolor{red}{[*** Gabri: #1 ***]}}
\newcommand{\ck}[1]{\textcolor{red}{[*** CK: #1 ***]}}
\newcommand{\CL}[1]{\textcolor{violet}{[*** CL: #1 ***]}}
\newcommand{\HL}[1]{\textcolor{red}{[*** HL: #1 ***]}}


\newcommand{\xsubsection}[1]{\refstepcounter{subsection}\textbf{\thesubsection~~#1}~~}

\icmltitlerunning{Kernelized Multiplicative Weights for 0/1-Polyhedral Games}
\begin{document}
\twocolumn[
  \icmltitle{Kernelized Multiplicative Weights for 0/1-Polyhedral Games: Bridging the Gap Between Learning in Extensive-Form and Normal-Form Games}



  \icmlsetsymbol{equal}{*}

  \begin{icmlauthorlist}
    \icmlauthor{Gabriele Farina}{cmu}
    \icmlauthor{Chung-Wei Lee}{usc}
    \icmlauthor{Haipeng Luo}{usc}
    \icmlauthor{Christian Kroer}{col}
  \end{icmlauthorlist}

  \icmlaffiliation{cmu}{Computer Science Department, Carnegie Mellon University}
  \icmlaffiliation{usc}{Computer Science Department, University of Southern California}
  \icmlaffiliation{col}{IEOR Department, Columbia University}

  \icmlcorrespondingauthor{Gabriele Farina}{gfarina@cs.cmu.edu}
  \icmlcorrespondingauthor{Chung-Wei Lee}{leechung@usc.edu}
  \icmlcorrespondingauthor{Haipeng Luo}{haipengl@usc.edu}
  \icmlcorrespondingauthor{Christian Kroer}{christian.kroer@columbia.edu}

  % \icmlkeywords{Machine Learning, ICML}

  \vskip 0.3in
]

\printAffiliationsAndNotice{}

\begin{abstract}
  While extensive-form games (EFGs) can be converted into normal-form games (NFGs), doing so comes at the cost of an exponential blowup of the strategy space. So, progress on NFGs and EFGs has historically followed separate tracks, with the EFG community often having to catch up with advances (\eg last-iterate convergence and predictive regret bounds) from the larger NFG community. In this paper we show that the Optimistic Multiplicative Weights Update (OMWU) algorithm---the premier learning algorithm for NFGs---can be simulated on the normal-form equivalent of an EFG in linear time per iteration in the game tree size using a kernel trick. The resulting algorithm, \emph{Kernelized OMWU (KOMWU)}, applies more broadly to all convex games whose strategy space is a polytope with 0/1 integral vertices, as long as the kernel can be evaluated efficiently. In the particular case of EFGs, KOMWU closes several standing gaps between NFG and EFG learning, by enabling direct, black-box transfer to EFGs of desirable properties of learning dynamics that were so far known to be achievable only in NFGs. Specifically, KOMWU gives the first algorithm that guarantees at the same time last-iterate convergence, lower dependence on the size of the game tree than all prior algorithms, and $\tilde{\bigOh}(1)$ regret when followed by all players.
\end{abstract}

\section{Introduction}  \label{sec:introduction}

\newcommand\inexpIntro[3]{#1?(#2,#3).}
\newcommand\rinexpIntro[3]{*#1?(#2,#3).}
\newcommand\outexpIntro[3]{#1!(#2,#3).}
\newcommand\outatomIntro[3]{#1!(#2,#3)}

We propose a fully automated method for proving termination of \(\pi\)-calculus processes.
Although there have been a lot of studies on termination analysis for the \(\pi\)-calculus
and related calculi~\cite{Deng06IC,Demangeon07,SangiorgiTermination,KobayashiHybrid,Yoshida04IC,DBLP:journals/jlp/DemangeonHS10,Venet98SAS}, most of them have been rather theoretical,
and there have been surprisingly little efforts in developing  fully automated termination
verification methods and tools based on them. To our knowledge,
Kobayashi's \typical{}~\cite{TyPiCal,KobayashiHybrid} is the only exception that
can prove termination of \(\pi\)-calculus processes (extended with natural numbers)
fully automatically, but its termination analysis is quite limited (see Section~\ref{sec:relatedwork}).

Our method is based on a reduction to termination analysis for sequential programs:
we translate a \(\pi\)-calculus process \(P\) to a sequential program \(S_P\), so that
if \(S_P\) is terminating, so is \(P\). The reduction allows us to use
powerful, mature methods and tools
for termination analysis of sequential programs~\cite{heizmann2016ultimate,freqterm,DBLP:conf/lics/PodelskiR04,Kuwahara2014Termination,DBLP:journals/cacm/CookPR11}.

The idea of the translation is to convert a chain of communications on replicated input
channels to a chain of recursive function calls of the target sequential program.
Let us consider the following Fibonacci process:
\begin{align*}
    & \rinexpIntro{\fib}{n}{r}
        \ifexp{n<2}{ \soutatom{r}{1} \\ &\quad}
                   { \nuexp{s_1} \nuexp{s_2} (\outatomIntro{\fib}{n-1}{s_1} \PAR \outatomIntro{\fib}{n-2}{s_2} \PAR \sinexp{s_1}{x}\sinexp{s_2}{y}\soutatom{r}{x+y}) \\}
    & \PAR \outatomIntro{\fib}{m}{r}
\end{align*}
Here, the process
$\rinexpIntro{\fib}{n}{r} \ldots$ is a function server that computes the \(n\)-th Fibonacci number
in parallel and returns the result to \(r\),
and $\outatom{\fib}{m}{r}$ sends a request for computing the \(m\)-th Fibonacci number;
those who are not familiar with the syntax of the \(\pi\)-calculus may wish to consult
Section~\ref{sec:targetlanguage} first.
To prove that the process above is terminating for any integer \(m\),
it suffices to show that there is no infinite chain of communications on $\fib$:
\[
    \fib(m,r) \to \fib(m_1,r_1) \to \fib(m_2,r_2) \to \cdots.
\]
We convert the process above to the following program:\footnote{The actual translation
  given later is a little more complex.}
\begin{verbatim}
 let rec fib(n) = if n<2 then () else (fib(n-1) [] fib(n-2)) in
 fib(m)
\end{verbatim}
Here, \texttt{[]} represents the non-deterministic choice.
Note that, although the calculation of Fibonacci numbers is not preserved,
for each chain of communications on \texttt{fib}, there is a corresponding
sequence of recursive calls:
\[
\mathtt{fib}(m) \to \mathtt{fib}(m_1) \to \mathtt{fib}(m_2) \to \cdots.
\]
Thus, the termination of the sequential program above implies the termination of
the original process.
As shown in the example above, (i) each communication on a replicated input channel
is converted to a function call, (ii) each communication on a non-replicated input
channel is just removed (or, in the actual translation, replaced by a call of
a trivial function defined by \(f(\seq{x})=(\,)\)), and (iii) parallel composition
is replaced by a non-deterministic choice.
We formalize the translation outlined above and prove its correctness.

The basic translation sketched above sometimes loses too much information.
For example, consider the following process:
\begin{align*}
    & \rinexpIntro{\pre}{n}{r} \soutatom{r}{n-1} \\
    & \PAR \rinexpIntro{f}{n}{r} \ifexp{n<0}{ \soutatom{r}{1} }
                                       { \nuexp{s} (\outatomIntro{\pre}{n}{s} \PAR \sinexp{s}{x}\outatomIntro{f}{x}{r}) } \\
    & \PAR \outatomIntro{f}{m}{r}
\end{align*}
The translation sketched above would yield:
\begin{verbatim}
  let pred(n) = n-1 in
  let rec f(n) = if n<0 then () else (pred(n) [] f(*)) in
  f(m)
\end{verbatim}
Here, \texttt{*} represents a non-deterministic integer: since we have removed
the input $\sinatom{s}{x}$, we do not have information about the value of \( x \).
As a result, the sequential program above is non-terminating, although the original
process is terminating.
To remedy this problem, we also refine the basic translation above by using a refinement
type system for the \(\pi\)-calculus. Using the refinement type system,
we can infer that the value of \(x\) in the original process is less than \(n\),
so that we can refine the definition of \texttt{f} to:
\begin{verbatim}
 let rec f(n) = ... else (pred(n) [] let x=* in assume(x<n);f(x))
\end{verbatim}
The target program is now terminating, from which
we can deduce that the original process is also terminating.
We have implemented an automated tool based on the refined translation above.

The contributions of this paper are summarized as follows.
\begin{itemize}
\item The formalization of the basic translation from the \(\pi\)-calculus
  (extended with integers) to sequential programs, and a proof of its correctness.
\item The formalization of a refined translation based on a refinement type system.
\item An implementation of the refined translation, including automated refinement type
  inference based on CHC solving, and experiments to evaluate the effectiveness of
  our method.
\end{itemize}

The rest of this paper is structured as follows.
Section~\ref{sec:targetlanguage} introduces the source and target languages
of our translation.
Section~\ref{sec:approach} 
formalizes the basic translation, and proves its correctness.
Section~\ref{sec:refinement} refines the basic translation by using a refinement type system.
Section~\ref{sec:implementation} reports an implementation and experiments.
Section~\ref{sec:relatedwork} discusses related work,
and Section~\ref{sec:conclusion} concludes the paper.

\section{Preliminaries}\label{chpt:preliminiaries}
In this chapter we will introduce some of the mathematical background and notation needed for this thesis. In particular, we will shortly introduce the differential geometric description of spacetime in Section \ref{sec:spacetime_geometry} and give an introduction to the notion of global hyperbolicity and its connection to Green- and normally-hyperbolic operators in Section \ref{sec:global_hyperbolicity}. In a bit more detail, we will introduce the notion of differential forms and give explicit definitions, also in terms of an index based notation, in Section \ref{sec:differential_forms}. For completeness, in Section \ref{sec:cat-theory}, we present basic definitions of category theory. The reader familiar with these topics can safely skip this chapter and refer to it when interested in the chosen conventions.
%
%
%
%
%%%%%%
%%SPACTIME GEOMETRY
%%%%%
%
%
%
\subsection{Spacetime geometry}\label{sec:spacetime_geometry}
In GR, the universe is mathematically described as a four dimensional \emph{spacetime}, consisting of a smooth, four dimensional manifold \gls{M} (assumed to be Hausdorff, connected, oriented, time-oriented and para-compact) and a Lorentzian metric $g$. We will assume the signature of the Lorentzian metric $g$ to be $(-,+,+,+)$. The Levi-Civita connection on $(\M,g)$ is as usual denoted by \gls{nabla}.
Throughout this thesis, we treat spacetime as fixed, implementing a gravitational background determined classically by Einstein's field equations. Hence, we neglect any back-reaction of the fields on the metric, both in the quantum and the classical case. In that sense, we treat the fields as \emph{test fields}.\par
For the basic mathematical theory regarding Lorentzian manifolds, we refer to the literature: An introduction to the topic with an emphasis on the physical application in GR is for example given in \cite{wald_GR} and \cite{carroll_spacetime-and-gr}.
Here, we will shortly recap the notion of a tangent space and tangent bundle and generalize to the notion of a vector bundle which we will use in the general description of normally hyperbolic operators and differential forms.
In the following, we generalize the setting to an arbitrary smooth manifold $\N$ of dimension $N$ with either Lorentzian or Riemannian metric $k$.\par
%
%
A \emph{tangent vector} $v_x$ at point $x \in \N$ is a linear map $v_x : C^\infty(\N , \IR) \to \IR$ that obeys the Leibniz rule, that is, for $f,g \in C^\infty (\N,\IR)$ it holds $v_x(fg) = f(x)v_x(g) + v_x(f)g(x)$.
We define the \emph{tangent space} \gls{TxN} of $\N$ at $x$ as the real $N$-dimensional vector space of all tangent vectors at point $x$.
The disjoint union of all tangent spaces is called the \emph{tangent bundle} \gls{TN} of $\N$ and is itself a manifold of dimension $2N$. A \emph{vector field} is a map $v: \N \to T\N$ such that $v(x) \in T_x\N$.
The respective dual spaces, that is the space of all linear functionals, the \emph{co-tangent space} and the \emph{co-tangent bundle}, are denoted by \gls{TsxN} and \gls{TsN} respectively.\par
%
For Lorentzian manifolds, we call a tangent vector $v$ at $x \in \N$ \emph{timelike} if $k_{\mu \nu} v^\mu v^\nu < 0$, \emph{spacelike} if $k_{\mu \nu} v^\mu v^\nu > 0$ and \emph{null} (or lightlike) if $k_{\mu \nu} v^\mu v^\nu = 0$. At every point $x \in \N$, we define the set of all \emph{causal}, that is, either timelike or null, tangent vectors in the tangent space at $x$. This set is called the \emph{light cone} at $x$ and it is split up into two distinct parts, one that we call the future light cone, and one that we call the past light cone at $x$. Since we assume the manifold to be time orientable, there exists a smooth vector field $t$ that is timelike at every $x \in \N$. Given this time orientation, we identify the future (past) light cone with the set of tangent vectors $v \in T_x\N$ such that $k_{\mu\nu} v^\mu t^\nu < 0$ (respectively $> 0$). Therefore, a tangent vector $v$ at $x$ is called \emph{future directed} (past directed) if it lies in the future (past) light cone at $x$.\\
Accordingly, a curve $\gamma : I \to \N$ is called timelike (spacelike, null, causal, future or past directed) if its tangent vector $\dot{\gamma}$ is timelike (spacelike, null, causal, future or past directed) at every $x \in \N$.  For every point $x \in \N$ we define the \emph{causal future/past} \gls{causalfuturepast} of $x$ as the set of all points $q \in \N$ that can be reached by a future directed causal curve originating in $x$. For any subset $S \in \N$ we define $J^\pm (S) = \bigcup_{x \in S} J^\pm(x)$ and $J(S) = J^+(S) \cup J^- (S)$. Finally, the future/past domain of dependence $\gls{futurepastdomainofdependence}$ of a set $S \subset \N$ is the set of all points $x \in \N$ such that every inextendible causal curve through $x$ intersects $S$. The \emph{domain of dependence} \gls{domainofdependence} of $S$ is the union of the future and past domain of dependence of the set $S$.
For more details on the causal structure of spacetime we refer to for example \cite[Chapter 8]{wald_GR}.\par
%
%
%
The notion of tangent bundles can be generalized to the notion of a vector bundle. Instead of ``attaching'' the vector spaces $T_x \N$ to every point $x$ of the manifold, we allow for the occurrence of arbitrary vector spaces, called the fibres of the vector bundle. A vector bundle then consists of the base manifold, in our case $\N$, the total space and a map $\pi$ from the total space to the base manifold, that can be locally trivialized. At each point of the base manifold, the pre-image of $\pi$ is the fibre of the vector bundle. To be precise we define, following \cite{rudolph_schmidt}:
\begin{definition}[Vector bundle]
	A smooth \emph{vector bundle} over $\N$ is a tuple $\gls{vectorbundle} = (E,\N, \pi)$, where $E$ is a smooth manifold and $\pi : E \to \N$ is a smooth surjective map satisfying:
	\begin{enumerate}
		\item For every $x \in \N$, $\pi^{-1}(x)$ is a vector space, called the fibre of the bundle at point $x$.
		\item There exists a finite dimensional vector space $F$, an open covering $\left\{ U_\alpha\right\}_\alpha$ of $\N$ and a family of diffeomorphisms $\chi_\alpha : \pi^{-1}(U_\alpha) \to U_\alpha \times F$ such that for all $\alpha$ it holds $\chi_\alpha \comp \text{pr}_1 =  \restr{\pi}{\pi^{-1}(U_\alpha)}$ and for every $x \in \N$ the map $\text{pr}_2 \comp \restr{\chi_\alpha}{\pi^{-1}(x)} : \pi^{-1}(x) \to F$ is linear.
	\end{enumerate}
\end{definition}
Here, the maps $\text{pr}_1$ and $\text{pr}_2$ denote the projection onto the first respectively second component of an element in $U_\alpha \times F$. The properties graphically mean that \emph{locally}, the vector bundle ``looks like" the product of the base manifold with the fibre. The tuples $(U_\alpha, \chi_\alpha)$ are called \emph{local trivializations} of the vector bundle. Like for vector spaces, we can define the sum and product of vector bundles, by using the according vector space definitions on the fibres of the bundle.\par
Let $\mathfrak{X}, \mathfrak{Y}$ be vector bundles over $\N$ with fibres $X_x$ and $Y_x$ at $x \in \N$. We denote by \gls{whitneysum} the \emph{Whitney sum} of the two vector bundles - the vector bundle over $\N$ whose fibres are given by the direct sum $X_x \oplus Y_x$. Similarly, one obtains the local trivializations of the Whitney sum from the trivializations of $\mathfrak{X}, \mathfrak{Y}$ and direct sums.\par
Accordingly, let $\mathfrak{X}, \mathfrak{Y}$ be vector bundles over $\N$ and $\widetilde{\N}$, with fibres $X_x$ and $Y_{\tilde{x}}$ at $x \in \N$, $\tilde{x} \in \widetilde{\N}$ respectively. We denote by \gls{outerproductbundle} the \emph{outer product} of the two vector bundles - the vector bundle over $\N \times \widetilde{\N}$ whose fibres are given by the tensor products $X_x \otimes Y_x$. Similarly, one obtains the local trivializations of the outer product from the trivializations of $\mathfrak{X}, \mathfrak{Y}$ and tensor products. \par
%
Finally, we generalize the notion of vector fields:
\begin{definition}[Sections of vector bundles]
Let $\mathfrak{X}=(E,\N,\pi)$ be a vector bundle with fibres $X_x=\pi^{-1}(x)$ at $x \in \N$. A \emph{smooth section} of the vector bundle is a smooth map $\gamma : \N \to E$ such that $\gamma(x) \in X_x$ for all $x \in \N$. The \emph{vector space of smooth sections} of $\mathfrak{X}$ is denoted by \gls{gammax}, the one with compactly supported sections is as usual denoted by \gls{gammaxzero}.
\end{definition}
In this language, a vector field $v$ is just a smooth section of the tangent bundle of a manifold, $v \in \Gamma(T\N)$. One may therefore identify the physical notion of fields with smooth sections of vector bundles. This point of view will be used to define the notion of differential forms in Section \ref{sec:differential_forms}.\par
In this thesis, we usually are interested in complex valued functions (or sections in general). Therefore, we view all occurring vector bundles as complex, in the sense that we take two distinct copies of the vector bundle, one representing the real, one the imaginary part of the bundle. A section of that complex vector bundle is just a pair of two sections of the real vector bundle under consideration. From now, if not specified explicitly, we will view all vector bundles, including the tangent bundle $T\N$, as complex vector bundles. Accordingly, smooth sections of those bundles will in general be complex valued.
%
%
%
%
%
%
%
%
%%%%%%%
%%PARTIAL DIFFERENTIAL OPERATORS AND GLOBAL HYPERBOLICITY
%%%%%%%
%
%
%
\subsection{Partial differential operators and global hyperbolicity}\label{sec:global_hyperbolicity}
When dealing with field theories, whether classical or quantum, one is, of course, interested in the dynamics of the fields. These are usually described by some partial differential equation, often of second order. In the following, we give a short introduction to the theory of certain partial differential operators acting on smooth sections of a vector bundle over the spacetime $(\M,g)$.\par
%
As we have seen, these smooth sections are generalizations of the notion of a field.  In the following, let $\mathfrak{X}$ denote a vector bundle over the manifold $\M$ and let $P: \Gamma(\mathfrak{X}) \to \Gamma(\mathfrak{X})$ be a partial differential operator acting on smooth sections of the bundle. As in the case of flat spacetime, we are interested in basic questions regarding the differential equation $Pf = j$, for example: Can we formulate a (globally) well posed initial value problem? Does the differential equation possess (unique) solutions? To answer these questions, we will now restrict to the case where $P$ is linear and of second order, as it is often the case in physical applications. One can show that for a certain class of such operators, namely normally hyperbolic partial differential operators of second order, we can rigorously treat these questions.\par
Choosing local coordinates $x=(x_\mu)$ on $\M$ and a local trivialization of $\mathfrak{X}$, a linear partial differential operator of second order is called \emph{normally hyperbolic} if it takes the form
\begin{align}
	P = - \sum_{\mu,\nu} g^{\mu \nu} \partial_\mu \partial_\nu + \sum_{\alpha} A_\alpha (x) \partial_\alpha + B(x) \formspace,
\end{align}
where $A_\alpha$ and $B$ are matrix-valued coefficients depending smoothly on the coordinate $x$ (see. \cite[Chapter 1.5]{baer_ginoux_pfaeffle}). One can also formulate a coordinate independent definition in terms of the principal symbol, which we will not present here (see for example \cite[Section 1.5]{baer_ginoux_pfaeffle} ). \par
%
Normally hyperbolic operators possess unique fundamental solutions (see for example the fundamental solutions to the wave operator as noted in Lemma \ref{lem:fundamental_solution_wave_operator}). These fundamental solutions fulfill certain physically important properties, such as a finite propagation speed smaller than the speed of light. Furthermore, specifying the initial data on some space-like hypersurface $X \in  \M$ specifies a unique solution on the domain of dependence $D(X)$ of $X$. Due to these properties, one often calls normally hyperbolic operators just \emph{wave operators}. But to state a \emph{globally} well posed initial value problem for a wave equation, we need to restrict the class of spacetimes $\M$ under consideration to those that possess space-like hypersurfaces $X$ whose domain of dependence is all of the spacetime, $D(X) = \M$. This leads to the notion of \emph{globally hyperbolic} spacetimes:
\begin{definition}[Global Hyperbolicity]
	A spacetime $\M$ is called \emph{globally hyperbolic} if there exists a Cauchy surface $\gls{sigma}$ in $\M$.
\end{definition}
\noindent Here, a Cauchy surface is a space-like hypersurface $\Sigma \subset \M$ such that every inextendible causal curve $\gamma$ intersects $\Sigma$ exactly once. One can show that Cauchy surfaces fulfill the desired property mentioned above, that is,  $D(\Sigma) = \M$. Furthermore, one can show that any globally hyperbolic spacetime $\M$ is foliated by a one-parameter family $\left\{ \Sigma_t \right\}_t$ of Cauchy surfaces (see for example \cite[Theorem 8.3.14]{wald_GR}). \par
In physical applications, one often finds the dynamics of a theory to be described by wave operators. Most prominently, the Klein-Gordon operator $(\square + m^2)$ acting on scalar fields, or its generalization, the wave operator acting on differential forms introduced in Section \ref{sec:differential_forms}, is normally hyperbolic. But there are also important physical field theories that are not described by wave operators, such as the Proca field treated in this thesis. It turns out that the Proca operator (see Definition \ref{def:proca_operator}) is a so called \emph{Green-hyperbolic} operator. These are again partial differential operators $P$ of second order acting on smooth sections of some vector bundle, such that $P$ (and its dual $P'$) posses fundamental solutions. Obviously, normally hyperbolic operators are Green-hyperbolic, but the opposite is not true. One can generalize some results obtained by studying normally hyperbolic operators to Green-hyperbolic operators. An introduction to this topic is given in \cite{baer_green-hyperbolic}, where it is also shown that the Proca operator is Green-hyperbolic but not normally hyperbolic.\par
For our application, the notion of Green-hyperbolicity is not of vast importance, but it is worth mentioning that there exists a more detailed mathematical background on the treatment of such operators.
A very detailed description of normally hyperbolic operators on Lorentzian manifolds, including proofs of the above statements regarding the initial value problem and the existence of fundamental solutions, is given in \cite{baer_ginoux_pfaeffle}, also with an overview of quantization. A shorter introduction to the topic is for example treated in \cite{baer-ginoux_classical-and-quantum-fields}, also with a description of quantization.
%
%
%
%
%
%
%%%
%
%
%
%%
%%%%%%%%%
%%%DIFFERENTIAL FORMS
%%%%%%%%
%
%
%
\subsection{Differential forms}\label{sec:differential_forms}
%
%
Differential forms provide an elegant, coordinate independent description of calculus on smooth manifolds. In particular, they generalize the notion of line- and volume-integrals that are known from analysis. Differential forms play a remarkable role in physics, as one can argue that they indeed describe fundamental physical entities. As an example, instead of viewing a classical force as a vector, one can think of it, more closely related to experiments, as a differential one-form that assigns a scalar to a tangent vector of a curve. This scalar is the (infinitesimal) work associated with the force along the curve. Also, differential forms allow for an elegant geometric description of field theories, for example the Maxwell and Proca field theories that we encounter in this thesis. In Maxwell's classical theory of electromagnetism, instead of viewing the electric and magnetic field (which are conceptually just forces) as the fundamental physical entities, one introduces the \emph{vector potential}, a one-form, consisting of the scalar electric potential and the vector potential associated with the magnet field. Experiments like the Aharonov-Bohm experiment allow for an interpretation of the vector potential as the fundamental physical object, rather than the associated electromagnetic field. \\
Even more fundamentally, the two main theories of physics, General Relativity and the Standard Model of particle physics, are field theories. They are deeply connected to a geometric interpretation and can be elegantly described using differential forms. \par
%
%
Despite of all this, differential forms are usually not part of the standard curriculum of physicists. We shall therefore introduce the basic aspects and definitions regarding differential forms that are used in this thesis. For a more detailed introduction we refer to the literature: For example \cite[Chapter 2 and 4]{rudolph_schmidt} or \cite[Appendix B]{wald_GR} provide introductions to the topic.\par
%
%
In the following, let $\N$ denote a smooth $N$-dimensional manifold, assumed to be Hausdorff, connected, oriented and para-compact, with either Lorentzian or Riemannian metric $k$ and Levi-Civita connection $\nabla$. For a Lorentzian manifold we use the sign convention $(-,+,\dots,+)$ of the metric $k$. The number of negative eigenvalues of $k$ is denoted by $s$, so $s=0$ for a Riemannian manifold and, in our convention, $s=1$ for a Lorentzian manifold.
Later, we will specify to a four dimensional (globally hyperbolic) spacetime consisting of a four dimensional manifold $\M$ with Lorentzian metric $g$ and Cauchy surface $\Sigma$ with induced Riemannian metric $h$.
%
We define:
\begin{definition}[Differential form]
	Let $p\in \{0,1,\dots,N\}$. A \emph{differential form} $\omega$ of degree $p$, or $p$-form for short, on the manifold $\N$ is an anti-symmetric tensor field of rank $(0,p)$. That is, at every point $x \in \N$, $\omega_x$ is an anti-symmetric multi-linear map
	\begin{align}
	\omega_x : \underbrace{T_x \N \times T_x \N \times \cdots \times T_x \N}_{p\text{-times}} \to \IR \formspace.
	\end{align}
	We denote the vector space\footnote{Naturally, addition and scalar multiplication are defined point-wise.} of $p$-forms on $\N$ by $\gls{omegap}$, the space with compactly supported ones by \gls{omegapz}.
\end{definition}
As an example, a zero-form $f \in \Omega^0(\N)$ is just a $C^\infty$-function from $\N$ to $\IR$, hence we can identify $\Omega^0(\N) = C^\infty (\N, \IR)$. A one-form $A \in \Omega^1(\N)$ is nothing more than a co-vector field and in a physical context usually denoted in local coordinates by $A_\mu$. Note, that alternatively one can directly define a $p$-form as a smooth section of the $p$-th exterior product of the co-tangent bundle and hence identify $\Omega^p(\N) = \Gamma \big( \largewedge^k T^*\N\big)$. As mentioned in Section \ref{sec:spacetime_geometry}, we view the tangent bundle as a complex bundle. Therefore, the sections of that bundle will be complex valued functionals. In that fashion, we will usually view the spaces $\Omega^p(\N)$ as complex valued differential forms.\par
%
Next we define the basic operations, besides addition and scalar multiplication, that one can perform on differential forms.
%
\begin{definition}[Exterior product]
	Let $A \in \Omega^p(\N)$ be a $p$-form and  $B\in \Omega^q(\N)$ a $q$-form on $\N$. \\
	The \emph{exterior product} $\gls{wedge}:\Omega^p(\N) \times \Omega^q(\N) \to \Omega^{p+q} (\N)$ is defined by
	\begin{align}
	(A \wedge B)_{\mu_1\dots\mu_p \nu_1\dots\nu_q} = \frac{(p+q)!}{p!q!}\, A_{[\mu_1 \dots \mu_p} B_{\nu_1\dots\nu_q]} \formspace,
	\end{align}
	where the anti-symmetrization of a tensor $T$ is given through
	\begin{align}
	T_{[\mu_1\dots\mu_p]} = \frac{1}{p!} \sum\limits_{\sigma\in S_N }\textrm{sgn}(\sigma) T_{\sigma(\mu_1)\dots\sigma(\mu_p)} \formspace.
	\end{align}
\end{definition}
Here, $S_N$ denotes the symmetric group\footnote{Usually the symmetric group is defined as the set of permutations of $\{1,2,\dots,N\}$ but we chose the index to run over $\{0,1,\dots,N-1\}$, identifying the time component with zero rather then one.} of degree $N$, consisting of permutations of the set $\{0,1,\dots,N-1\}$.
With this notion of multiplication, point-wise addition and scalar multiplication, the space $\gls{omega} \coloneqq \bigoplus_{p = 0}^\infty \Omega^p(\N) = \bigoplus_{p = 0}^N \Omega^p(\N)$ becomes an algebra, usually called the Grassmann- or \emph{exterior algebra} of differential forms on $\N$. We have used that obviously $\Omega^k(\N) =0$ for $k >N$ due to the anti-symmetrization.\par
Furthermore, we find a notion of how to \emph{pullback} differential forms on manifolds to another manifold, for example the pullback of a differential form on the spacetime $\M$ to differential forms on its Cauchy surface $\Sigma$. Given a $C^\infty$-map $\psi: \widetilde{\N} \to \N$, where $\N, \widetilde{\N}$ are manifolds, we can naturally define the pullback of a function $f \in \Omega^0(\N)$ to a function $(\psi^* f) \in \Omega^0(\widetilde{\N})$ by composing $f$ with $\psi$:
\begin{align}
\psi^* f \coloneqq f \comp \psi \formspace.
\end{align}
\newpage
With the pullback of functions defined, we can define how to \emph{push forward}, or carry along, vector fields on $\widetilde{\N}$ to vector fields on $\N$: Let $f\in \Omega^0(\N)$ and $\tilde{v} \in \Gamma(T\widetilde{\N})$ and $\tilde{x} \in \widetilde{\N}$. Then
\begin{align}
(\psi_* \tilde{v})_{\psi(\tilde{x})} (f) \coloneqq \tilde{v}_{\tilde{x}}(\psi^* f)
\end{align}
defines the vector field $(\psi_* v) \in \Gamma(T\N)$. With these basic operations at hand, we can generalize to define the pullback of differential forms:
\begin{definition}[Pullback]\label{def:pullback}
	Let $\N, \widetilde{\N}$ be manifolds of dimension $N,\widetilde{N}$ respectively, and let $\psi: \widetilde{\N} \to \N$ be a smooth map. Then, $\psi$ defines an algebra homomorphism $\psi^* : \Omega(\N) \to  \Omega(\widetilde{\N})$,
	called the \emph{pullback} of differential forms. For $\omega \in \Omega^p(\N)$, $\tilde{x} \in \widetilde{\N}$ and $\tilde{v}_i \in T_x \widetilde{\N}$, $i=1,2,\dots,p$, it is defined by
	\begin{align}
	\left( \psi^* \omega \right)_{\tilde{x}}  (\tilde{v}_1,\tilde{v}_2,\dots,\tilde{v}_p) \coloneqq \omega_{\psi(\tilde{x})} (\psi_* \tilde{v}_1, \dots , \psi_* \tilde{v}_p) \formspace.
	\end{align}
\end{definition}
%
%
%
%
On the exterior algebra we find a duality, provided by the Hodge operator:
\begin{definition}[Hodge dual]
	The hodge star operator $\gls{hodge}: \Omega^p(\N) \to \Omega^{N-p}(\N)$ is defined through
	\begin{align}
	B \wedge *A = \frac{1}{p!} B^{\mu_1\dots\mu_p}A_{\mu_1\dots\mu_p} \dvolk \formspace,
	\end{align}
	which yields the coordinate representation
	\begin{align}
	(*A)_{\mu_{p+1}\dots\mu_N} = \frac{\detk}{p!} \, \epsilon_{\mu_1\dots\mu_N} A^{\mu_1\dots\mu_p} \formspace.
	\end{align}
\end{definition}
Here, \gls{levicivita} denotes the fully antisymmetric tensor of rank $N$ (Levi-Civita symbol) satisfying $\epsilon_{12,\dots,N} =1$ and the \emph{volume element} \gls{dvolk} is defined by
\begin{align}
\left( \gls{dvolk} \right)_{\alpha_1\dots\alpha_N} = \detk \, \epsilon_{\alpha_1\dots\alpha_N} \formspace.
\end{align}
In a sense, the volume element describes how the curvature of the manifold deforms a unit volume.
The duality follows from the important property of the Hodge operator as stated in the following lemma:
\begin{lemma}
	Let $*$ denote the Hodge star operator on the exterior algebra $\Omega(\N) $. It holds that
	\begin{align}
	** = (-1)^{s+p(N-p)} \, \mathbbm{1} \formspace,
	\end{align}
	which is trivially equivalent to $*^{-1} = (-1)^{s+p(N-p)} \, *$.
\end{lemma}
\begin{proof}
	Let $A \in \Omega^p(\N)$ be a $p$-form on $\N$. Then:
	\begin{align}
	(*{*A})_{\mu_1 \dots \mu_p}
	&= \frac{\detk \, \detk}{p! \, (N-p)!} \; \epsilon_{\alpha_{p+1}\dots\alpha_N \mu_1 \dots \mu_p}\;\epsilon^{\alpha_{1}\dots\alpha_N}\;A_{\alpha_1\dots\alpha_p} \notag\\
	&= (-1)^{p(N-p)} \frac{\detk \, \detk}{p! \, (N-p)!} \; \epsilon_{\alpha_{p+1}\dots\alpha_N \mu_1 \dots \mu_p}\;\epsilon^{\alpha_{p+1}\dots\alpha_{N}\alpha_1\dots\alpha_p}\;A_{\alpha_1\dots\alpha_p}  \notag\\
	&= (-1)^{s+p(N-p)} \delta\indices{^{[\alpha_{1}}_{\mu_{1}}}\, \dots \, \delta\indices{^{\alpha_p ] }_{\mu_p}} \;A_{\alpha_1\dots\alpha_p} \notag\\
	&=  (-1)^{s+p(N-p)}\;A_{\mu_1\dots\mu_p} \formspace
	\end{align}
	We have used Lemma \ref{lem:epsilon_contraction} and, in the last step, that the anti-symmetrization is absorbed by contraction because $A$ is antisymmetric.
\end{proof}
%
%
%
%
%
Furthermore, we can equip the exterior algebra with a differentiable structure, introducing the notion of the exterior derivative.
\begin{definition}[Exterior derivative]
	The \emph{exterior derivative} $\gls{d}:\Omega^p(\N) \to \Omega^{p+1} (\N)$ is defined by the following properties:
	\begin{enumerate}
		\item $d$ is linear
		\item $d$ obeys a graded Leibniz rule: Let $A \in \Omega^p(\N)$ and  $B\in \Omega^q(\N)$, then
		\begin{align}
		d(A \wedge B) = dA \wedge B + (-1)^p \, A \wedge dB
		\end{align}
		\item $d$ is nilpotent, that is,  $d^2 = 0$.
	\end{enumerate}
	In local coordinates, this is equivalent to the representation
	\begin{align}
	(dA)_{\mu \alpha_1\dots\alpha_p} = (p+1)\, \nabla_{[\mu}A_{\alpha_1\dots\alpha_p]} \formspace.
	\end{align}
\end{definition}
An important property of the exterior derivative is that it commutes (or rather intertwines its action) with pullbacks (see \cite[Proposition 4.1.7]{rudolph_schmidt}).
A $p$-form $\omega \in \Omega^p(\N)$ is called \emph{exact} if there is a $(p-1)$-form $\alpha \in \Omega^{p-1}(\N)$ such that $\omega = d\alpha$. We call $\omega$ \emph{closed} if $d \omega =0$. Accordingly, the space of closed $p$-forms is denoted by \gls{omegapd}, the space of exact ones by \gls{domegap}. As usual, the ones with compact support are denoted by a subscript zero. Note, that every exact form is closed, using that $d$ is by definition nilpotent, but the reverse is in general not true. It does hold, however, on certain manifolds with trivial topology, such as Minkowski spacetime. This is expressed in the so called Poincar\'e-Lemma (see for example \cite[Chapter 4]{bott_tu}) based on the study of de Rham cohomology.\par
%
Moreover, $N$-forms can naturally be integrated. Using local coordinates and a partition of unity, we define the integral of $N$-forms via the well known integration on $\IR^N$:
\begin{definition}[Integration on manifolds]
	Let $\left\{U_\alpha, \psi_\alpha\right\}_\alpha$ be an atlas of the manifold $\N$ and $\left\{\chi_\alpha\right\}_\alpha$ a partition of unity subordinate to the locally finite open cover $\left\{U_\alpha\right\}_\alpha$. Let $x^\mu_{(\alpha)}$ be a coordinate basis of $\psi$ on $U_\alpha$. For any $N$-form $\omega \in \Omega^N_0(\M)$ we define the integral
	\begin{align}
	\int\limits_{\N} \omega &\coloneqq \sum_{\alpha} \int\limits_{\psi_\alpha (U_\alpha)} w(x_{(\alpha)}^0,\dots,x_{(\alpha)}^1)\; dx_{(\alpha)}^0 \cdots dx_{(\alpha)}^{N-1} \formspace,
	\end{align}
	where $w$ are the components of $\omega$ in the coordinates $x_{(\alpha)}^\mu$, that is $\omega = w dx_{(\alpha)}^0 \wedge \cdots \wedge dx_{(\alpha)}^{N-1}$.
	This definition is independent of the choice of the atlas and the partition of unity (see \cite[Proposition 3.3]{bott_tu}).
\end{definition}
With integration at our disposal, we present an important theorem regarding the integration of exact differential forms:
\begin{theorem}[Stoke's Theorem]\label{thm:stokes}
	Let $\N$ be an oriented manifold of dimension $N$ and let its boundary $\partial \N$ be endowed with the induced orientation. Let $\gls{inclusionmap} : \partial \N \hookrightarrow \N$ be the inclusion operator.
	Let $\omega \in \Omega^{N-1}_0(\N)$ be a compactly supported $(N-1)$-form on $\N$. Then it holds
	\begin{align}
	\int\limits_\N d\omega = \int\limits_{\partial \N} i^*\omega \formspace.
	\end{align}
\end{theorem}
\begin{proof}
	A proof is given in most of the introductory literature on differential geometry (see for example \cite[Chapter 17, Theorem 2.1]{lang}).
	Note that one can equivalently formulate Stoke's theorem on a \emph{compact} manifold but for {arbitrary} (that is, in general not compactly supported) $(N-1)$-forms on the manifold (see for example \cite[Theorem 4.2.14]{rudolph_schmidt}). This will be of importance in later calculations.
\end{proof}
%
Furthermore, we can define a bilinear map on $\Omega^p(\N)$ using the integration of $N$-forms:
\begin{definition}
	Let $A,B \in \Omega^p(\N)$ such that their supports have a compact intersection. Define the bilinear map $\gls{innerprod} : \Omega^p(\N) \times \Omega^p(\N) \to \IC$ by
	\begin{align}
	\langle A, B \rangle_\N \coloneqq  \int_{\N } A \wedge * B = \int_{\N } A_{\mu_1 \dots \mu_p}B^{\mu_1 \dots \mu_p}\,\dvolk \formspace.
	\end{align}
\end{definition}
Since by definition $A \wedge * B$ is a compactly supported $N$-form, this is well defined. We may sometimes refer to $\langle \cdot , \cdot \rangle_\N$ as an inner product for simplicity, even though it is not positive definite.
%
%
%
%
%
Using the exterior derivative, we define the interior or co-derivative:
\begin{definition}[Interior derivative]
	The \emph{interior derivative} $\gls{delta} : \Omega^p(\N) \to \Omega^{p-1}(\N)$ is defined by
	\begin{align}
	\delta \coloneqq (-1)^{s+1+N(p-1)}\, {*{d*}} \formspace.
	\end{align}
	From the defining properties of $d$ and $*$ it follows $\delta^2 =0$.
\end{definition}
Here, $s$ again denotes the number of negative eigenvalues of the metric $k$ of $\N$. In accordance with our nomenclature, we call a $p$-form $\omega$ co-exact if there exists a $\alpha \in \Omega^{p+1}(\N)$ such that $\omega = \delta \alpha$ and co-closed if $\delta \omega = 0$. Accordingly, the spaces of co-closed and co-exact $p$-forms are denoted by \gls{omegapdelta} and \gls{deltaomegap} respectively.\par
Using the exterior and interior derivative we define the partial differential operator:
\begin{definition}[D'Alembert Operator]
	The d'Alembert (or Laplace - de Rham) operator $\gls{dalembert}: \Omega^p(\N) \to \Omega^{p}(\N)$ is defined by
	\begin{align}
	\square \coloneqq \delta d +d \delta \formspace.
	\end{align}
\end{definition}
By definition of the exterior and interior derivative, it is easy to show that $\square$ commutes with both $d$ and $\delta$:
\begin{align}
\square d &= (\delta d + d \delta )d \notag \\
&= d \delta d \notag \\
&= d (\delta d + d \delta) \formspace,
\end{align}
and analogously for $\delta$.
The d'Alembert operator, and its generalization to $(\square + m^2)$ for some constant $m > 0$, are important examples for a normally hyperbolic differential operators (see Section \ref{sec:global_hyperbolicity}) and we may therefore sometimes just refer to them as \emph{wave operators}.\par
The sign convention in the definition of the exterior derivative is chosen such that on any Lorentzian or Riemannian manifold the interior derivative is formally adjoint to the exterior derivative, that is,  for $A \in \Omega^{p}(\N)$ and $B \in \Omega^{p+1}(\N)$ it holds that
\begin{align}
\langle dA , B \rangle_{\N} = \langle A , \delta B \rangle_\N \formspace,
\end{align}
which leads to a representation in local coordinates of the Manifold given by:
\begin{align}
(\delta A)_{\mu_2\dots\mu_p} = - \nabla^{\mu_1}A_{\mu_1\dots\mu_p} \formspace.
\end{align}
To see that this is consistent, let $A \in \Omega^{p-1}(\N)$ and $B \in \Omega^{p}(\N)$ such that their supports have compact intersection.
We obtain, using Stoke's Theorem \ref{thm:stokes}:
\begin{align}
0 &= \int \limits_{\partial \N} i^* (A \wedge *B) \notag\\
&= \int \limits_{\N} d(A \wedge *B)  \notag\\
&= \int \limits_{\N} dA \wedge *B + (-1)^{p-1} A \wedge d{*B} \notag\\
&= \int \limits_{\N} dA \wedge *B + (-1)^{p-1} A \wedge *{*^{-1}}\underbrace{d{*B}}_{\textrm{is a } (N-p+1) \textrm{ form.}} \notag\\
&= \int \limits_{\N} dA \wedge *B + (-1)^{p-1}(-1)^{s+(N-p+1)(N-N+p-1)} A \wedge *{*d{*B}} \notag\\
&= \int \limits_{\N} dA \wedge *B + (-1)^{p+(1-p)(p-1)} A \wedge *\delta B \formspace.
\end{align}
It can easily be proven by induction that $\big(p+(1-p)(p-1)\big)$ is odd for any $p \in \IN$, which yields the result
\begin{align}
\langle dA , B \rangle_{\N} = \langle A , \delta B \rangle_\N \formspace.
\end{align}
The definitions stated above thus fulfill the requirement of formal adjointness of the exterior and interior derivate on an arbitrary Lorentzian or Riemannian manifold $\N$.
In local coordinates we use a partial integration to obtain
\begin{align}
\langle dA , B \rangle_\N &= \int \limits_{\N} dA \wedge * B \notag\\
%&= \int \limits_{\N} \frac{1}{p!} (dA)^{\alpha_1\dots\alpha_p}\,B_{\alpha_1 \dots \alpha_p} \, \dvolk \notag\\
&= \int \limits_{\N}  \frac{p}{p!} \nabla^{[\alpha_1}A^{\alpha_2\dots\alpha_p]}\,B_{\alpha_1 \dots \alpha_p} \, \dvolk \notag\\
&= \int \limits_{\N}  \frac{1}{(p-1)!} \nabla^{\alpha_1}A^{\alpha_2\dots\alpha_p}\,B_{\alpha_1 \dots \alpha_p} \, \dvolk \notag\\
&= - \int \limits_{\N}  \frac{1}{(p-1)!} A^{\alpha_2\dots\alpha_p}\, \nabla^{\alpha_1}B_{\alpha_1 \dots \alpha_p} \, \dvolk \notag\\
&= \langle A, \delta B \rangle_\N \formspace,
\end{align}
which yields
\begin{align}
-\nabla^{\alpha_1}B_{\alpha_1 \dots \alpha p} = (\delta B)_{\alpha_2 \dots \alpha_p}\formspace.
\end{align}
On the four dimensional spacetime $(\M,g)$ the definitions of the Hodge star operator and the interior derivative simplify, such that
\begin{align}
*_{(\M)}*_{(\M)} &= (-1)^{p+1} \mathbbm{1} \\
\delta_{(\M)} &= *_{(\M)}{d_{(\M)}*_{(\M)}} \formspace ,
\end{align}
holds on the spacetime $(\M,g)$ and
\begin{align}
*_{(\Sigma)}*_{(\Sigma)} &= \mathbbm{1} \\
\delta_{(\Sigma)} &= (-1)^p *_{(\Sigma)}{d_{(\Sigma)}*_{(\Sigma)}}
\end{align}
holds on  $(\Sigma,h)$. In the following we will drop the subscript ${(\M)}$, since we will perform all the calculations on a four dimensional spacetime, except when explicitly noted (for example with a subscript $(\Sigma)$).
%
%
%
%
%
%
%
%
%%%%%%
%%CATEGORY THEORY
%%%%%%
\subsection{Category theory}\label{sec:cat-theory}
The description of Quantum Field Theory on Curved Spacetimes (QFTCS) in the framework of \name{Brunetti}, \name{Fredenhagen} and \name{Verch} \cite{Brunetti_Fredenhagen_Verch} is based on category theory. In this thesis, we will not go into detail on those categorical aspects, however we will need some basic definitions to formulate the theory rigorously, that is namely the notion of a category and that of covariant functors, since, in the used framework, the generally covariant QFTCS is a functor.\par
Here, we present definitions given in \cite[Appendix A.1]{baer_ginoux_pfaeffle} and refer to the appropriate literature for details. We define:
\begin{definition}[Category]
	A \emph{category} $\mathsf{Cat}$ consists of the following:
	\begin{enumerate}
		\item a class $\mathsf{Obj}_\mathsf{Cat}$ whose members are called \emph{objects},
		\item a set $\mathsf{Mor}_\mathsf{Cat}(A,B)$, for any two objects $A,B \in \mathsf{Obj}_\mathsf{Cat}$, whose elements are called \emph{morphisms},
		\item for any three objects $A,B,C \in \mathsf{Obj}_\mathsf{Cat}$ there is a map
		\begin{align}
\mathsf{Mor}_\mathsf{Cat}(B,C) \times \mathsf{Mor}_\mathsf{Cat}(A,B) &\to \mathsf{Mor}_\mathsf{Cat}(A,C) \notag\\
(\psi,\phi) &\mapsto \psi \comp \phi
		\end{align}
		called the composition of morphisms subject to the relations:\vspace{4mm}
		\begin{enumerate}[label=(\arabic*)]
			\item for non equal pairs $(A,B)$, $(A',B')$ of objects, the sets $\mathsf{Mor}_\mathsf{Cat}(A,B)$ and $\mathsf{Mor}_\mathsf{Cat}(A',B')$ are disjoint,
			\item for every object $A$ there exists a morphism $\text{id}_A \in \mathsf{Mor}_\mathsf{Cat}(A,A)$ such that it holds for all objects $B$, morphisms $\psi \in \mathsf{Mor}_\mathsf{Cat}(B,A)$ and $\phi \in \mathsf{Mor}_\mathsf{Cat}(A,B)$
			\begin{align}
				\text{id}_A \comp \psi &= \psi \quad \text{and}\\
				\phi \comp \text{id}_A &= \phi \quad,
			\end{align}
			\item the composition law is associative, that is for an objects $A,B,C,D$ and any morphisms $\psi \in \mathsf{Mor}_\mathsf{Cat}(A,B)$, $\phi \in \mathsf{Mor}_\mathsf{Cat}(B,C)$ and $\chi \in \mathsf{Mor}_\mathsf{Cat}(C,D)$ it holds
			\begin{align}
				(\chi \comp \phi) \comp \psi = \chi \comp (\phi \comp \psi) \formspace.
			\end{align}
		\end{enumerate}
	\end{enumerate}
\end{definition}
%
%
%
\begin{definition}[Functor]
	Let $\mathsf{Cat1}$ and $\mathsf{Cat2}$ be categories. A \emph{covariant functor} $\mathscr{A}: \mathsf{Cat1} \to \mathsf{Cat2}$ consists of the map $\mathscr{A} : \mathsf{Obj}_\mathsf{Cat1} \to \mathsf{Obj}_\mathsf{Cat2}$ and maps $\mathscr{A}: \mathsf{Mor}_\mathsf{Cat1}(A,B) \to \mathsf{Mor}_\mathsf{Cat2}\big(\mathscr{A}(A),\mathscr{A}(B)\big)$ for any two objects $A,B \in \mathsf{Obj}_\mathsf{Cat1}$ such that
	\begin{enumerate}
		\item {the composition is preserved, that is for all objects $A,B,C \in \mathsf{Obj}_\mathsf{Cat1}$ and for any morphisms $\psi \in \mathsf{Mor}_\mathsf{Cat1}(A,B)$ and $\phi \in \mathsf{Mor}_\mathsf{Cat1}(B,C)$ it holds
		\begin{align}
			\mathscr{A}(\phi \comp \psi) = \mathscr{A}(\phi) \comp \mathscr{A}(\psi) \formspace,
		\end{align}}
		\item{
			$\mathscr{A}$ maps identities to identities, that is for any object $A \in \mathsf{Obj}_\mathsf{Cat1}$ it holds
			\begin{align}
				\mathscr{A}(\text{id}_\mathsf{A}) = \text{id}_{\mathscr{A}(A)} \formspace.
			\end{align}
			}
	\end{enumerate}
\end{definition}
%
%
%
%
%
%
%
%
%
%
%
%
%%%%%%
%%SIGN CONVENTIONS
%%%%%%
%
%
\subsection{Sign conventions}\label{sec:sign_conventions}
At certain points throughout this chapter we have had a freedom of choice regarding the signs of some entities, in particular the sign of the signature of the Lorentzian metric $g$ and that of the interior derivative $\delta$. Though at this stage the choice can be made arbitrarily, we want to make it in a way that in the end allows us to make certain physical interpretations on some parameters. More precisely, we want to interpret the parameter $m$ of the Klein-Gordon equation\footnote{or its generalization on $p$-forms} $(\square + m^2) f = 0$ for a zero-form $f \in \Omega^0(\M)$ as a mass in the physical sense. With the chosen sign convention for $\delta$ we find, using ${\delta}f = 0$:
\begin{align}
	\square f
	&= (\delta d + d \delta) f \notag\\
	&= \delta d f \notag\\
	&= - \nabla^\mu \nabla_\mu f \formspace.
\end{align}
In the following heuristic (local) argument we see
\begin{align}
	\square + m^2
	&= -\nabla^\mu \nabla_\mu + m^2 \notag\\
	&\sim \partial_t^2 + \sum_i \partial_i^2 + m^2\notag\\
	&\sim -E^2 + \abs{\vector{p}}^2 + m^2
\end{align}
which yields the correct relativistic relation of energy, momentum and mass according to $E^2 = \abs{\vector{p}}^2 + m^2$.
A similar calculation holds for the Klein-Gordon operator generalized to act on one-forms. If we had found a ``wrong'' relation between energy, momentum and mass, we would have had to adapt the chosen signs. Usually one chooses the sign of the metric and the interior derivative such that they are in some sense mathematically convenient (although one might disagree with another one's choice). We have made the choice of the metric, such that the Cauchy surfaces become Riemannian rather that ``anti-Riemannian'' (with an all minus signature), which seems more natural to some. Also, a lot of the used references on spacetime geometry (in particular the book by \name{Wald} \cite{wald_GR}) use this sign convention, which makes the application of certain formulas easier. As mentioned, the sign of the interior derivative was chosen such that it is formally adjoint to the exterior derivative (with respect the specified inner product) on all Lorentzian and Riemannian manifolds. It seemed convenient for the actual calculations to fix the sign regardless of the signature of the metric of the underlying manifold. One could equivalently have fixed the opposite sign, yielding the two derivatives to be skew-adjoint, which is also done in the literature. However, in the end, one has one freedom left to make the energy-momentum-mass relation work: that is the sign in front of the mass in the Klein-Gordon equation and all other wave equations accordingly. Hence, one regularly also finds the Klein-Gordon equation to be defined with a flipped sign of the mass term. But for our case, we want the mass $m$ in any wave equation to appear with a positive sign.
%
%

\section{Multiplicative Weights in Polyhedral Convex Games}
\label{sec:vertex}

A powerful generalization of normal-form games is \emph{polyhedral convex games}, of which extensive-form games are an example~\citep{Gordon08:No}. Unlike NFGs, in which players select a mixed strategy from the probability simplex spanned by the set of available action $\cA_i$, in a polyhedral convex game the set of  ``randomized strategies'' from which each player $i\in\range{m}$ can draw is a given convex polytope $\Omega_i \subseteq \bbR^{d_i}$. Analogously to NFGs, we represent a polyhedral convex game as a tuple $\Gamma = (m, \{\Omega_i\}, \{\bar U_i\})$, where the functions $\bar U_i : \Omega_1\times\dots\times\Omega_m \to [0,1]$ are the \emph{multilinear} utility functions for each player $i\in\range{m}$.

The concepts of learning agents, equilibria, and COLS introduced in \cref{sec:online learning,sec:nfgs} can be directly extended to polyhedral convex games without difficulty, by simply replacing the set of mixed strategies $\Delta(\cA_i)$ of each player with their convex polyhedral counterpart $\Omega_i$.

Because the set of mixed strategies $\Omega$ of every player is a polytope, the decision problem of picking a mixed strategy $\vx\^t\in\Omega$ can be equivalently thought of as the decision problem of picking a convex combination $\vlam\^t \in \Delta(\cV_{\Omega})$ over the finite set of vertices $\cV_{\Omega}$ of $\Omega$. Indeed, it is not hard to show that a learning algorithm ${\cR}$ for $\Omega\subseteq\bbR^d$ can be constructed from \emph{any} learning algorithm $\tilde\cR$ for the set of \emph{vertices} $\cV_{\Omega}$, as we describe next. Let $\mat{V}$ denote the matrix whose columns are the vertices $\cV_\Omega$; then:
\begin{itemize}[nosep,left=0mm]
    \item whenever $\cR$ receives a prediction ${\vm}\^t\in\bbR^d$ (resp., loss ${\vl}\^t$), it computes the vector $\tilde{\vm}\^t\defeq\mat{V}^\top\vm\^t\in\bbR^{\cV_\Omega}$ (resp., $\tilde{\vl}\^t\defeq\mat{V}^\top\vl\^t$) and forwards it to $\tilde\cR$;
    \item whenever $\tilde\cR$ plays a new distribution $\vlam\^t \in \Delta(\cV_\Omega)$, the convex combination of vertices $\vx\^t\defeq \sum_{\vv\in\cV_\Omega} \vlam\^t[\vv]\,\vv = \mat{V}\vlam\^t$ is played by $\cR$.
\end{itemize}
It is immediate to verify that the regret cumulated by $\cR$ and $\tilde\cR$ is equal at all times $T$. So, as long as $\tilde\cR$ guarantees sublinear regret, then so does $\cR$. In this paper we are particularly interested in the algorithm obtained by using the above construction for the specific choice of OMWU as the algorithm $\tilde\cR$. We coin \emph{Vertex OMWU} the resulting learning algorithm $\cR$ in that case, depicted in \cref{fig:Vertex OMWU}. Let $\vl\^0,\vm\^0\defeq \vzero\in\bbR^{\cV_\Omega}$ and $\vlam\^0 \defeq \frac{1}{|\cV_\Omega|}\vone\in\Delta(\cV_\Omega)$; then, at all times $t\!\in\!\Npp$, Vertex OMWU updates the convex combination of vertices $\vlam\^{t-1}\!\in\!\Delta(\cV_\Omega)$ according to
\[
    \vlam\^t[\vv] \defeq \frac{\vlam\^{t-1}[\vv]\cdot e^{-\eta\^t\langle\vw\^{t},\vv\rangle}}{\sum_{\vv' \in \cV_\Omega} \vlam\^{t-1}[\vv']\cdot e^{-\eta\^t\langle \vw\^{t}\!,\vv'\rangle}},
    \numberthis[$\clubsuit$]{eq:vertex lam update}
\]%\vspace{-1mm}
where%\vspace{-3mm}
\[
    \vw\^t \defeq \vl\^{t-1} - \vm\^{t-1} + \vm\^t\in\bbR^d,
    \numberthis{eq:def w}
\]
and then outputs the iterate
\[
    \Omega\ni\vx\^t \defeq \sum_{\vv\in\cV_\Omega} \vlam\^t[\vv]\cdot\vv = \mat{V}\vlam\^t.
    \numberthis[$\spadesuit$]{eq:xt original}
\]
It is straightforward to show that Vertex OMWU satisfies \cref{prop:omwu near optimal,prop:omwu optimal sum,prop:omwu last iterate} with $|\cA_i|$ replaced with $|\cV_{\Omega_i}|$, by using a black-box reduction to NFGs. Indeed, let $\Gamma = (m, \{\Omega_i\},\{\bar U_i\})$ be a polyhedral convex game, and introduce the \emph{NFG $\tilde\Gamma$ equivalent to $\Gamma$}, defined as the NFG $\tilde\Gamma \defeq (m, \{\cV_{\Omega_i}\}, \{U_i\})$ where the action set of each player is the set of vertices $\cV_{\Omega_i}$, and $U_i(\vv_1, \dots, \vv_m) \defeq \bar U_i(\vv_1, \dots, \vv_m)$ for all $(\vv_1,\dots,\vv_m)\in\cV_{\Omega_1}\times\dots\times\cV_{\Omega_m}$. Consider the losses $\vl_i\^t$, predictions $\vm\^t$, and iterates $\vx_i\^t\in\Omega_i$ produced by agents learning (under the COLS) in $\Gamma$ using Vertex OMWU, and the losses $\tilde{\vl}_i\^t$, predictions $\tilde{\vm}_i\^t$, and iterates $\vlam_i\^t\in\Delta(\cV_i)$ produced by agents learning (again under the COLS) in $\tilde\Gamma$ using OMWU. For all players $i\in\range{m}$, it is immediate to verify by induction that the relationships (i) $\tilde{\vl}_i\^t = \mat{V}_i^\top \vl_i\^t$, (ii) $\tilde{\vm}_i\^t = \mat{V}_i^\top\vm_i\^t$, and (iii) $\vx_i\^t = \mat{V}_i \vlam_i\^t$ hold at all $t$, where $\mat{V}_i$ is the matrix whose columns are the vertices $\cV_{\Omega_i}$ (see also \cref{fig:Vertex OMWU}). The above discussion shows that in a precise sense, Vertex OMWU and OMWU are the same algorithm, just on different equivalent representations of the game.
Hence, the regret cumulated by each player $i$ in $\Gamma$ matches the regret cumulated by the same player in $\tilde\Gamma$, showing that \cref{prop:omwu near optimal,prop:omwu optimal sum} hold for Vertex OMWU.
Furthermore, whenever $\vlam_i\^t$ converges in iterates, then clearly so does $\vx_i\^t = \mat{V}_i\vlam_i\^t$, showing that \cref{prop:omwu last iterate} applies to Vertex OMWU as well.

The main drawback of Vertex OMWU is that it is not clear how to avoid a per-iteration complexity linear in the number of vertices of $\Omega$, which is typically exponential in $d$ (this is the case in extensive-form games). While different learning algorithms that guarantee polynomial per-iteration complexity in $d$ exist, none of them is known to guarantee near-optimal per-player regret (\cref{prop:omwu near optimal}) or last-iterate convergence (\cref{prop:omwu last iterate}) enjoyed by Vertex OMWU, much less all three \cref{prop:omwu near optimal,prop:omwu optimal sum,prop:omwu last iterate} at the same time. In the rest of the paper we fill this gap, by showing that in several cases of interest, Vertex OMWU can be implemented with polynomial-time (in $d$) iterations using a kernel trick.








\begin{figure}[t]\centering
    \tikzstyle{lbl} = [fill=white,rounded corners,inner ysep=.7mm]
    \tikzstyle{tightlbl} = [lbl,inner xsep=.3mm,inner ysep=.2mm]

    \scalebox{.97}{\begin{tikzpicture}[x=1mm,y=1mm]
            \begin{scope}[local bounding box=vertexbb]
                \node[text=gray,inner sep=0mm] at (-11, 10) {\small$\Gamma$};
                \draw[semithick,fill=white] (-12, 0) rectangle (12, -10) node[fitting node] (VertexOMWU) {};
                \draw[semithick,<-] (VertexOMWU.north) -- +(0,  4) node[above=-1mm,lbl,text width=6mm,align=center] (VertexLoss) {\raisebox{0mm}{\small${\vl}\^t$}};
                \node[above=-0.25mm of VertexLoss,lbl,text width=6mm,align=center] (VertexPred) {\raisebox{0mm}{\small${\vm}\^t$}};
                \draw[semithick,->] (VertexOMWU.south) -- +(0, -3) node[below,lbl] (VertexStrat) {\small$\vx\^t\in\Omega$};

                \node[text width=20mm, anchor=center, align=center] at (VertexOMWU.center) {\small Vertex OMWU\\\textcolor{black!60}{\eqref{eq:vertex lam update},~\eqref{eq:xt original}}};
            \end{scope}
            \begin{scope}[xshift=47mm,local bounding box=vanillabb]
                \node[text=gray,inner sep=0mm] at (11, 10) {\small$\tilde{\Gamma}$};
                \draw[semithick,fill=white] (-12, 0) rectangle (12, -10) node[fitting node] (VanillaOMWU) {};
                \draw[semithick,<-] (VanillaOMWU.north) -- +(0,  4) node[above=-1mm,lbl,text width=6mm,align=center] (VanillaLoss) {\raisebox{0mm}{\small$\tilde{\vl}\^t$}};
                \node[above=-0.25mm of VanillaLoss,lbl,text width=6mm,align=center] (VanillaPred) {\raisebox{0mm}{\small$\tilde{\vm}\^t$}};
                \draw[semithick,->] (VanillaOMWU.south) -- +(0, -3) node[below,lbl] (VanillaStrat) {\small$\vlam\^t\in\Delta(\cV_\Omega)$};

                \node[text width=18mm, anchor=center, align=center] at (VanillaOMWU.center) {\small OMWU\\\textcolor{black!60}{\eqref{eq:vanilla OMWU}}};
            \end{scope}
            \node[blue,tightlbl] (PredLbl) at ($(VertexPred)!.5!(VanillaPred)$) {\scalebox{.9}{\raisebox{1mm}{\small$\tilde{\vm}\^t = \mat{V}^\top{\vm}\^t$}}};
            \node[blue,tightlbl] (LossLbl) at ($(VertexLoss)!.5!(VanillaLoss)$) {\scalebox{.9}{\small$\tilde{\vl}\^t = \mat{V}^\top{\vl}\^t$}};
            \node[violet,tightlbl] (StratLbl) at ($(VertexStrat)!.5!(VanillaStrat)$) {\scalebox{.9}{\raisebox{1mm}{\small$\vx\^t = \mat{V}\vlam\^t$}}};
            \draw[line width=1mm,white] (VertexPred) -- (PredLbl) (PredLbl) -- (VanillaPred);
            \draw[blue] (VertexPred) -- (PredLbl) (PredLbl) edge[->] (VanillaPred);
            \draw[line width=1mm,white] (VertexLoss) -- (LossLbl) (LossLbl) -- (VanillaLoss);
            \draw[blue] (VertexLoss) -- (LossLbl) (LossLbl) edge[->] (VanillaLoss);
            \draw[line width=1mm,white] (VertexStrat) -- (StratLbl) (StratLbl) -- (VanillaStrat);
            \draw[violet] (VertexStrat) edge[<-] (StratLbl) (StratLbl) -- (VanillaStrat);

            \begin{pgfonlayer}{background}
                \filldraw[black!20,thin,fill=black!8] ($(vertexbb.south west) + (-1,-.5)$) rectangle ($(vertexbb.north east) + (1, .5)$);
                \filldraw[black!20,thin,fill=black!8] ($(vanillabb.south west) + (-1,-.5)$) rectangle ($(vanillabb.north east) + (1, .5)$);
                \node[inner ysep=.2mm,inner xsep=0mm,rotate=90,yshift=3mm] at (vertexbb.west) {\small Polyhedral convex game};
                \node[inner ysep=.2mm,inner xsep=0mm,rotate=-90,yshift=3mm] at (vanillabb.east) {\small Equivalent NFG};
            \end{pgfonlayer}
        \end{tikzpicture}}
    %\vspace{-2mm}
    \caption{Construction of the Vertex OMWU algorithm. The matrix $\mat{V}$ has the (possibly exponentially-many) vertices $\cV_\Omega$ of the convex polytope $\Omega$ as columns.}
    \label{fig:Vertex OMWU}
    %\vspace{-4mm}
\end{figure}

%\vspace{-1mm}
\section{Kernelized Multiplicative Weights Update}\label{sec:KOMWU}
%\vspace{-1mm}

In this section, we introduce \emph{Kernelized OMWU (KOMWU)}. Kernelized OMWU gives a way of efficiently simulating the Vertex OMWU algorithm described in \cref{sec:vertex} on polyhedral decision sets whose vertices have 0/1 integer coordinates, as long as a specific \emph{polyhedral kernel} function can be evaluated efficiently. We will assume that we are given a polytope $\Omega \subseteq \bbR^d$ with (possibly exponentially many) 0/1 integral vertices $\cV_\Omega \defeq \{\bv_1, \dots,\bv_{|\cV_\Omega|}\} \subseteq\{0,1\}^d$.
Furthermore, given a vertex $\vec{v}\in\cV_\Omega$, we will write $k \in \vec{v}$ as a shorthand for $\vec{v}[k] = 1$.

We define the \emph{0/1-polyhedral feature map} $\phi_\Omega : \bbR^d \to \bbR^{\cV_\Omega}$ associated with $\Omega$ as the function such that
\[
    \phi_\Omega(\vx)[\vv] \defeq \prod_{k \in \vv} \vx[k] \qquad\forall\,\vx \in \bbR^d, \vv \in \cV_\Omega.
    \numberthis{eq:phi Omega}
\]
Correspondingly, the \emph{0/1-polyhedral kernel} $K_\Omega$ associated with $\Omega$ is defined as the function $K_\Omega : \bbR^d \times \bbR^d \to \bbR$,
\[
    K_\Omega(\vx,\vy) \defeq \langle \phi_\Omega(\vx), \phi_\Omega(\vy) \rangle = \sum_{\vv \in \cV_\Omega} \prod_{k \in \vv} \vx[k] \, \vy[k]. \numberthis{eq:K Omega}
\]
We show that Vertex OMWU can be simulated using $d+1$ evaluation of the kernel $K_\Omega$ at every iteration. The key observation is summarized in the next theorem, which shows that the iterates $\vlam\^t$ produced by Vertex OMWU are highly structured, in the sense that they are always proportional to the feature mapping $\phi_\Omega(\vb\^t)$ for some $\vb\^t\in\bbR^d$.

\begin{theorem}\label{thm:bt}
    Consider the Vertex OMWU algorithm \eqref{eq:vertex lam update}, \eqref{eq:xt original}. At all times $t\ge 0$, the vector $\vb\^t \in \bbR^d$ defined as
    \[
        \vb\^t[k] \defeq \exp\mleft\{-\sum_{\tau=1}^t\eta\^\tau\,\vw\^\tau[k]\mright\}
        \numberthis{eq:bt}
    \]
    for all $k=1,\dots,d$, is such that
    \[
        \vlam\^t = \frac{ \phi_\Omega(\vb\^t) }{ K_\Omega(\vb\^t, \vone)}.\numberthis{eq:b ratio}
    \]
\end{theorem}%\vspace{-3mm}
\begin{proof}%
    By induction.
    \begin{itemize}[nosep,leftmargin=5mm]
        \item At time $t = 0$, the vector $\vb\^0$ is $\vb\^0 = \vone \in \bbR^d$. By definition of the feature map~\eqref{eq:phi Omega}, $\phi_\Omega(\vone) = \vone \in \bbR^{\cV_\Omega}$. So, $K_\Omega(\vb\^0,\vone) = \sum_{\vv\in\cV_\Omega} 1 = |\cV_\Omega|$ and hence the right-hand side of~\eqref{eq:b ratio} is $\frac{1}{|\cV_\Omega|}\vec{1}$, which matches $\vlam\^0$ produced by Vertex OMWU, as we wanted to show.
        \item Assume the statement holds up to some time $t-1 \ge 0$. We will show that it holds at time $t$ as well.
              Since $\bv$ has integral 0/1 coordinates, we can write
              \[
                  \exp\{-\eta\^t\langle\vw\^{t},\vv\rangle\} &= \exp\mleft\{
                  -\eta\^t\,\sum_{k\in\vv} \vw\^t[k]
                  \mright\}\\
                  &= \prod_{k\in\vv} \exp\{-\eta\^t\,\vw\^t[k]\}.
                  \numberthis{eq:exp inner}
              \]
              From the inductive hypothesis and~\eqref{eq:phi Omega}, for all $\vv\in\cV_\Omega$,
              \[
                  \vlam\^{t-1}[\vv] &= \frac{\phi_\Omega(\vb\^{t-1})[\vv]}{K_\Omega(\vb\^{t-1},\vone)}
                  = \frac{\prod_{k\in\vv}\vb\^{t-1}[k]}{K_\Omega(\vb\^{t-1},\vone)}. \numberthis{eq:inductive hyp}
              \]
              Plugging~\eqref{eq:exp inner} and~\eqref{eq:inductive hyp} into~\eqref{eq:vertex lam update}, we have the inductive step
              \[
                  \vlam\^{t}[\vv] &= \frac{
                  \prod_{k\in\vv}\vb\^{t-1}[k]\exp\{-\eta\^t\,\vw\^t[k]\}
                  }{
                  \sum_{\vv\in\cV_\Omega}\prod_{k\in\vv}\vb\^{t-1}[k]\exp\{-\eta\^t\,\vw\^t[k]\}
                  }\\
                  &= \frac{\phi_\Omega(\vb\^{t})[\vv]}{K_\Omega(\vb\^{t}, \vone)}
              \]
              for all $\vv \in \cV_\Omega$, where in the last step we used the fact that $\vb\^t[k] = \vb\^{t-1}[k]\exp\{-\eta\^t\,\vw\^t[k]\}$ by~\eqref{eq:bt}. %
              \qedhere
    \end{itemize}
\end{proof}


The structure of $\vlam\^t$ uncovered by \cref{thm:bt} can be leveraged to compute the iterate $\vx\^t$ produced by Vertex OMWU, \ie the convex combination of the vertices
\eqref{eq:xt original},
using $d+1$ evaluations of the kernel $K_\Omega$. We do so by extending an idea of \citet[eq.~5.2]{Takimoto03:Path}.

\begin{theorem}\label{thm:bt to xt}
    Let $\vb\^t$ be as in \cref{thm:bt}. For each $h =1,\dots,d$, let $\ebar_h \in \bbR^d$ be defined as the indicator vector
    %\vspace{-3mm}
    \[
        \ebar_h[k] \defeq \bbone_{k\neq h} \defeq \begin{cases} 0 & \text{if } k = h\\ 1 & \text{if } k \neq h.\end{cases}
        \numberthis{eq:ebar}
    \]
    Then, at all $t \ge 1$, the iterate $\vx\^t\!\in\!\Omega$ produced by Vertex OMWU can be written as
    \[
        \vx\^t \! = \! \mleft(\!
        1 - \frac{K_\Omega(\vb\^t, \ebar_1)}{K_\Omega(\vb\^t, \vone)}, \dots,
        1 - \frac{K_\Omega(\vb\^t, \ebar_d)}{K_\Omega(\vb\^t, \vone)}
        \!\mright).\numberthis{eq:xt}
    \]
\end{theorem}
\begin{proof}%
    The proof crucially relies on the observation that for all $h=1,\dots,d$, the feature map $\phi_\Omega(\ebar_h)$ satisfies
    \[
        \phi_\Omega(\ebar_h)[\vv] = \prod_{k\in\vv} \ebar_h[k]
        = \prod_{k\in\vv}\bbone_{k\neq h} = \bbone_{h\notin \vv},
        \quad \forall\,\vv\in\cV_\Omega.
    \]
    Using the fact that $\phi_\Omega(\vone) = \vone$, we conclude that
    \[
        \phi_\Omega(\vone)[\vv] - \phi_\Omega(\ebar_h)[\vv] = \bbone_{h \in \vv}, \quad\forall\, h = 1,\dots,d.\numberthis{eq:diff phi}
    \]
    Therefore, for all $k = 1,\dots,d$, we obtain
    \[
        \vx\^t[k] &\overset{\mathclap{\eqref{eq:xt original}}}{=} \sum_{\vv\in\cV_\Omega} \vlam\^t[\vv]\cdot\vv[k] = \sum_{\vv\in\cV_\Omega} \vlam\^t[\vv]\cdot\bbone_{k\in\vv}\\
        &= \sum_{\vv\in\cV_\Omega} \vlam\^t[\vv]\cdot(\phi_\Omega(\vone)[\vv] - \phi_\Omega(\ebar_k)[\bv])\\
        &= \frac{\langle\phi_\Omega(\vb\^t),\phi_\Omega(\vone)\rangle - \langle\phi_\Omega(\vb\^t), \phi_\Omega(\ebar_k)\rangle}{K_\Omega(\vb\^t,\vone)}\\
        &= \frac{K_\Omega(\vb\^t\!,\vone) \!-\! K_\Omega(\vb\^t\!, \ebar_k)}{K_\Omega(\vb\^t,\vone)} = 1 \!-\! \frac{K_\Omega(\vb\^t\!, \ebar_k)}{K_\Omega(\vb\^t,\vone)},
    \]
    where the second equality follows from the integrality of $\vv \in \cV_\Omega$, the third from \eqref{eq:diff phi}, the fourth from \cref{thm:bt}, and the fifth from the definition of
    $K_\Omega$ %
    \eqref{eq:K Omega}.
\end{proof}

%\vspace{-1mm}
Combined, \cref{thm:bt,thm:bt to xt} suggest that by keeping track of the vectors $\vb\^t$ instead of $\vlam\^t$, updating them using \cref{thm:bt} and reconstructing the iterates $\vx\^t$ using \cref{thm:bt to xt}, Vertex OMWU can be simulated efficiently. We call the resulting algorithm, given in \cref{algo:kernelized OMWU}, \emph{Kernelized OMWU (KOMWU)}. Similarly, we call \emph{Kernelized MWU} the non-optimistic version of KOMWU obtained as the special case in which $\vm\^t = \vzero$ at all $t$. In light of the preceding discussion, we have the following.

\begin{theorem}\label{thm:kernel omwu equivalent}
    Kernelized OMWU produces the same iterates $\vec{x}\^t$ as Vertex OMWU when it receives the same sequence of predictions $\vec{m}\^t$ and losses $\vec{\ell}\^t\in\bbR^d$. Furthermore, each iteration of KOMWU runs in time proportional to the time required to compute the $d+1$ kernel evaluations $\{K_\Omega(\vb\^t, \vone), K_\Omega(\vb\^t, \ebar_1), \dots, K_\Omega(\vb\^t,\ebar_d)\}$.
\end{theorem}


\begin{figure}[t]\centering
    \makeatletter
    \newcommand{\removelatexerror}{\let\@latex@error\@gobble} %
    \makeatother
    \removelatexerror
    \scalebox{.95}{\begin{algorithm}[H]
            \caption{Kernelized OMWU (KOMWU)}
            \label{algo:kernelized OMWU}
            \DontPrintSemicolon
            $\vl\^0,~\vm\^0,~\vs\^0 \gets \vzero\in\bbR^d$\Comment*{\color{commentcolor}Initialization]\!\!\!\!}
            \For{$t=1,2,\dots$}{\vspace{-.5mm}
            \textbf{receive} prediction $\vm\^t \in \bbR^d$ of next loss\;
            \Comment{\color{commentcolor}set $\vec{m}\^t = \vec{0}$ for non-predictive variant]}
            \vspace{-1mm}\Hline{}\vspace{-.5mm}
            \Comment{\color{commentcolor}Compute $\vb\^t$ according to \cref{thm:bt}]}\vspace{.0mm}
            $\vw\^t \gets \vl\^{t-1} - \vm\^{t-1}+\vm\^t$\;
            $\vs\^t \gets \vs\^{t-1} + \eta\^t\vw\^t$\Comment*{\color{commentcolor}{\small$\vs\^t = \sum \eta\^\tau\vw\^\tau$}]\!\!\!\!}
            \For{$k=1,\dots,d$}{
            $\vb\^t[k]\gets \exp\{-\vs\^t[k]\}$\Comment*{\color{commentcolor}see \eqref{eq:bt}]\!\!\!\!}
            }
            \vspace{-1mm}\Hline{}\vspace{-.5mm}
            \Comment{\color{commentcolor}Produce iterate $\vx\^t$ according to \cref{thm:bt to xt}]\!\!\!\!}\vspace{.0mm}
            $\vx\^t \gets \vec{0} \in \bbR^d$\;
            $\alpha\gets K_\Omega(\vb\^t, \vone)$\Comment*{\color{commentcolor}$K_\Omega$ is defined in \eqref{eq:K Omega}]\!\!\!\!}
            \For{$k=1,\dots,d$}{
            $\vx\^t[k] \gets 1 - K_\Omega(\vb\^t, \ebar_k) \,/\, \alpha$\Comment*{\color{commentcolor}see \eqref{eq:xt}]\!\!\!\!}
            }
            \textbf{output} $\vx\^t \in \Omega$ and
            \textbf{receive} loss vector $\vl\^t \in \bbR^d$\!\!\!\!\;
            }\vspace{-1mm}
        \end{algorithm}}
    %\vspace{-3mm}
\end{figure}
%\vspace{-1mm}
\section{KOMWU in Extensive-Form Games}\label{sec:kernel efgs}
%\vspace{-1mm}

In this section, we show how the general theory we developed in \cref{sec:kernel efgs}
applies to extensive-form game, \ie tree-form games that incorporate sequential and simultaneous moves, and imperfect information. The central result of this section, \cref{thm:KOMWU in EFGs}, shows that OMWU on the normal-form representation of any EFG can be simulated in linear time in the game tree size via KOMWU, contradicting the popular wisdom that working with the normal form of an extensive-form game is intractable.

%\vspace{-1mm}
\subsection{Preliminaries on Extensive-Form Games}\label{sec:efg notation}
%\vspace{-1mm}

We now briefly recall standard concepts and notation about extensive-form games which we use in the rest of the section. More details and an example are available in \cref{app:efgs}.

In an $m$-player perfect-recall extensive-form game, each player $i\in\range{m}$ faces a tree-form sequential decision problem (TFSDP). In a TFSDP, the player interacts with the environment in two ways: at \emph{decision points}, the
agent must act by picking an action from a set of legal actions; at
\emph{observation points}, the agent observes a signal drawn from a set of
possible signals. We denote the set of decision points of player~$i$ as $\cJ_i$. The set of actions available at decision point $j\in\cJ_i$ is denoted $A_j$. A pair $(j,a)$ where $j\in \cJ_i$ and $a \in A_j$ is called a \emph{non-empty sequence}. The set of all non-empty sequences of player~$i$ is denoted as $\Sigma^*_i \defeq \{(j,a): j\in\cJ, a\in A_j\}$.
For notational convenience, we will often denote an element $(j,a)$ in
$\Sigma_i^*$ as $ja$ without using parentheses. Given a decision point $j \in \cJ_i$, we denote by $p_j$ its
\emph{parent sequence}, defined as the last sequence (that is, decision
point-action pair) encountered on the path from the root of the
decision process to $j$.
If the agent does not act before $j$ (that is, $j$ is the root of the
process or only observation points are encountered on the path from
the root to $j$), we let $p_j$ be set to the special element $\emptyseq$, called the \emph{empty sequence}. We let $\Sigma_i \defeq \Sigma_i^* \cup \{\emptyseq\}$. Given a $\sigma \in \Sigma_i$, we let $\mathcal{C}_\sigma \defeq \{j\in\cJ_i: p_j = \sigma\}$.


An $m$-player extensive-form game is a polyhedral convex game (\cref{sec:vertex}) $\Gamma = (m, \{Q_i\}, \{U_i\})$, where the convex polytope of mixed strategies $Q_i$ of each player $i\in\range{m}$ is called a \emph{sequence-form strategy space} \citep{Romanovskii62:Reduction,Stengel96:Efficient,Koller96:Efficient}, and is defined as
\newcommand*\circled[1]{%
    \tikz[baseline=(C.base)]\node[draw,circle,inner sep=0.8pt](C) {\small #1};\!%
}
\[
    Q_i \defeq \mleft\{\vx \in \bbR^{\Sigma_i}: \!\!\begin{array}{l} \circled{1}~~ \vx[\emptyseq] = 1, \\[1mm] \circled{2}~~ \vx[p_j] = \sum_{a\in A_j} \!\vx[ja] ~~\forall j\in\cJ_i \end{array}\!\!\!\mright\}.
\]

It is known that the set of vertices of $Q_i$ are the \emph{deterministic} sequence-form strategies $\Pi_i \defeq Q_i \cap \{0,1\}^{\Sigma_i}$.
We mention the following result (see \cref{app:proofs}).
%\vspace{-1mm}
\begin{restatable}{proposition}{propnumvertices}\label{prop:efg vertex count}
    The number of vertices of $Q_i$ is upper bounded by $A^{\|Q_i\|_1}$, where $A \!\defeq\! \max_{j\in\cJ_i} |A_j|$ is the largest number of possible actions, and $\|Q_i\|_1 \defeq \max_{\vec{q}\in Q_i}\|\vec{q}\|_1$.
\end{restatable}

%\vspace{-1mm}
We will often need to describe strategies for \emph{subtrees} of the TFDSM faced by each player $i$. We use the notation $j' \succeq j$ to denote the fact that $j'\in\cJ_i$ is a descendant of $j\in\cJ_i$, and $j'\succ j$ to denote a strict descendant (\ie $j'\succeq j \land j'\neq j$). For any $j\in\cJ_i$ we let $\Sigma^*_{i,j} \defeq \{j'a': j' \succeq j, a' \in A_{j'}\}$ denote the set of non-empty sequences in the subtree rooted at $j$. The set of sequence-form strategies for that subtree $j$ is defined as the convex polytope
\[
    Q_{i,j} \!\defeq\! \mleft\{\!\vx \in \bbR^{\Sigma^*_{i,j}}\!: \!\!\!\begin{array}{l} \circled{1}~~ \sum_{a\in A_j}\vx[ja] = 1, \\[1mm] \circled{2}~~ \vx[p_{j'}] \!=\! \sum_{a\in A_{j'}} \!\vx[j'a] ~~\forall j'\succ j\end{array}\!\!\!\mright\}.
\]
Correspondingly, we let $\Pi_{i,j} \defeq Q_{i,j} \cap \{0,1\}^{\Sigma^*_{i,j}}$ denote the set of vertices of $Q_{i,j}$, each of which is a deterministic sequence-form strategy for the subtree rooted at $j$.


\subsection{Linear-time Implementation of KOMWU}

For any player $i$, the 0/1-polyhedral kernel $K_{Q_i}$ associated with the player's sequence-form strategy space $Q_i$ can be evaluated in linear time in the number of sequences $|\Sigma_i|$ of that player. To do so, we introduce a \emph{partial kernel function} $K_j: \bbR^{\Sigma_i}\times\bbR^{\Sigma_i} \to \bbR$ for every decision point $j\in\cJ_i$, %
\[
    K_j(\vec{x}, \vec{y}) \defeq \sum_{\vec{\pi} \in \Pi_{i,j}}\prod_{\sigma \in \vec{\pi}} \vec{x}[\sigma]\,\vec{y}[\sigma].
    \numberthis{eq:def Kj}
\]


\begin{restatable}{theorem}{thmefgkernel}\label{thm:efg kernel computation}
    For any vectors $\vec{x},\vec{y} \in\bbR^{\Sigma_i}$, the two following recursive relationships hold:
    \[
        K_{Q_i}(\vec{x}, \vec{y}) &= \vec{x}[\emptyseq]\,\vec{y}[\emptyseq]\prod_{j \in \mathcal{C}_\emptyseq} K_j(\vec{x},\vec{y}),
        \numberthis{eq:efg kernel computation}
    \]
    and, for all decision points $j\in\cJ_i$,
    \[
        K_j(\vec{x}, \vec{y}) &=\!\sum_{a\in A_j} \mleft(\vec{x}[ja]\,\vec{y}[ja]\prod_{j' \in \mathcal{C}_{ja}} K_{j'}(\vec{x}, \vec{y})\mright).\numberthis{eq:efg kernel computation 2}
    \]
    In particular, \cref{eq:efg kernel computation,eq:efg kernel computation 2} give a recursive algorithm to evaluate the polyhedral kernel $K_{Q_i}$ associated with the sequence-form strategy space of any player $i$ in an EFG in linear time in the number of sequences $|\Sigma_i|$.
\end{restatable}


\cref{thm:efg kernel computation} shows that the kernel $K_{Q_i}$ can be evaluated in linear time (in $|\Sigma_i|$) at any $(\vx,\vy)$. So, the KOMWU algorithm (\cref{algo:kernelized OMWU}) can be trivially implemented for $\Omega = Q_i$ in quadratic $\bigOh(|\Sigma_i|^2)$ time per iteration by directly evaluating the $|\Sigma_i|+1$ kernel evaluations $\{K_{Q_i}(\vb\^t, \vone)\} \cup \{K_{Q_i}(\vb\^t, \ebar_{\sigma}): \sigma \in \Sigma_i\}$ needed at each iteration, where $\ebar_{\sigma}\in\bbR^{\Sigma_i}$, defined in~\eqref{eq:ebar} for the general case, is the vector whose components are $\ebar_{\sigma}[\sigma']\defeq \bbone_{\sigma\neq \sigma'}$ for all $\sigma,\sigma'\in\Sigma_i$.
We refine that result by showing that an implementation of
KOMWU with \emph{linear}-time (\ie $\bigOh(|\Sigma_i|)$) per-iteration complexity exists, by exploiting the structure of the particular set of
kernel evaluations needed at every iteration. In particular, we rely on the following observation.

\begin{restatable}{proposition}{propefgratio}\label{prop:efg ratio}
    For any player $i\in\range{m}$, vector $\vec{x} \in \bbR_{>0}^{\Sigma_i}$, and sequence $ja \in \Sigma^*_i$,
    \[
        \frac{1 \!-\! K_{Q_i}(\vec{x}, \bar{\vec{e}}_{ja}) / K_{Q_i}(\vec{x},\! \vone)}{1 \!-\! K_{Q_i}(\vec{x}, \bar{\vec{e}}_{p_j}) / K_{Q_i}(\vec{x},\! \vone)} = \frac{\vec{x}[ja]\prod_{j'\in\mathcal{C}_{ja}}\! K_{j'}(\vec{x},\!\vone)}{K_j(\vec{x},\! \vone)}.
    \]
\end{restatable}

In order to compute $\{K_{Q_i}(\vb\^t, \ebar_{\sigma}): \sigma \in \Sigma_i\}$ in cumulative $\bigOh(|\Sigma_i|)$ time, we then do the following.
\begin{enumerate}[nosep,left=0mm]
    \item We compute the values $K_j(\vb\^t, \vone)$ for all $j \in \cJ_i$ in cumulative $\bigOh(|\Sigma_i|)$ time by using \eqref{eq:efg kernel computation 2}.\label{step:one}
    \item We compute the ratio $K_{Q_i}(\vb\^t, \ebar_\emptyseq) / K_{Q_i}(\vb\^t, \vone)$ by evaluating the two kernel separately using \cref{thm:efg kernel computation}, spending $\bigOh(|\Sigma_i|)$ time.\label{step:two}
    \item We repeatedly use \cref{prop:efg ratio} in a top-down fashion along the
          tree-form decision problem of player~$i$ to compute the ratio $K_{Q_i}(\vb\^t, \ebar_{ja}) / K_{Q_i}(\vb\^t, \vone)$ for each sequence $ja\in\Sigma_i^*$ given the value of the parent ratio $K_{Q_i}(\vb\^t, \ebar_{p_j}) / K_{Q_i}(\vb\^t, \vone)$ and the partial kernel evaluations $\{K_j(\vb\^t, \vone): j \!\in\! \cJ_i\}$ from Step~\ref{step:one}. For each $ja\in\Sigma_i^*$, \cref{prop:efg ratio} gives a formula whose runtime is linear in the number of children decision points $|\mathcal{C}_{ja}|$ at that sequence. Therefore, the cumulative runtime required to compute all ratios $K_{Q_i}(\vb\^t, \ebar_{ja}) / K_{Q_i}(\vb\^t, \vone)$ is
          $\bigOh(|\Sigma_i|)$.\label{step:three}
    \item By multiplying the ratios computed in Step~\ref{step:three} by the value of $K_{Q_i}(\vb\^t, \vone)$ computed in Step~\ref{step:two}, we can easily recover each $K_{Q_i}(\vb\^t, \ebar_{\sigma})$ for every $\sigma \in \Sigma_i^*$.
\end{enumerate}

Hence, we have just proved the following.

\begin{theorem}\label{thm:KOMWU in EFGs}
    For each player $i$ in a perfect-recall extensive-form game, the Kernelized OMWU algorithm can be implemented exactly, with a per-iteration complexity linear in the number of sequences $|\Sigma_i|$ of that player.
\end{theorem}

\subsection{KOMWU Regret Bounds and Convergence}\label{sec:efg analysis}




If the players in an EFG run KOMWU,
then we can combine \cref{thm:kernel omwu equivalent} with standard OMWU regret bounds, \cref{prop:efg vertex count,prop:omwu near optimal,prop:omwu optimal sum,prop:omwu last iterate} to get the following:
%\vspace{-3mm}
\begin{theorem}
    In an EFG, after $T$ rounds of learning under the COLS, KOMWU satisfies
    %\vspace{-2.5mm}
    \begin{enumerate}[nosep,nolistsep,left=0mm]
        \item
              A player $i$ using KOMWU with $\eta\^t \defeq \eta = \sqrt{8\log(A) \|Q_i\|_1}/\sqrt{T}$ is guaranteed to incur regret at most $R^T_i = \bigOh(\sqrt{\|Q_i\|_1\log(A)T})$.
        \item
              There exist $C, C' > 0$ such that if all $m$ players learn using KOMWU with constant learning rate $\eta\^t \defeq \eta \leq 1/(Cm\log^4T)$, then each player is guaranteed to incur regret at most $\frac{\log (A_i) \|Q_i\|_1}{\eta} + C'\log T$.
        \item
              If all $m$ player learn using KOMWU with $\eta\^t \defeq \eta \leq 1/\sqrt{8}(m-1)$, then the sum of regrets is at most $\sum_{i=1}^m R_i^T = \bigOh(\max_{i=1}^m\{\|Q_i\|_1\log A_i\} \frac{m}{\eta})$.
        \item
              For two-player zero-sum EFGs, if both players learn using KOMWU, then there exists a schedule of learning-rates $\eta^{(t)}$ such that the iterates converge to a Nash equilibrium.
              Furthermore, if the NFG representation of the EFG has a unique Nash equilibrium and both players use learning rates $\eta\^t = \eta \leq 1/8$, then the iterates converge to a Nash equilibrium at a linear rate $C (1+C')^{-t}$, where $C,C'$ are constants that depend on the game.
    \end{enumerate}
    \label{thm:komwu efg}
\end{theorem}
%\vspace{-2mm}
Prior to our result, the strongest regret bound for methods that take linear time per iteration was based on instantiating e.g. follow the regularized leader (FTRL) or its optimistic variant with the dilatable global entropy regularizer of \citet{Farina21:Better}.
For FTRL this yields a regret bound of the form $\bigOh(\sqrt{\log(A)\,\|Q\|_1^2 T})$.
For optimistic FTRL this yields a regret bound of the form $\bigOh(\log(A)\,\|Q\|_1^2 \sqrt{m} T^{1/4})$, when every player in an $m$-player game uses that algorithm and appropriate learning rates.

Our algorithm improves the state-of-the-art rate in two ways.
First, we improve the dependence on game constants by almost a square root factor, because our dependence on $\|Q\|_1$ is smaller by a square root, compared to prior results.
Secondly, in the multi-player general-sum setting, every other method achieves regret that is on the order of $T^{1/4}$, whereas our method achieves regret on the order of $\log^4(T)$.
In the context of two-player zero-sum EFGs, the bound on the sum of regrets in \cref{thm:komwu efg} guarantees convergence to a Nash equilibrium at a rate of $\bigOh(\max_i \|Q_i\|_1\log A_i / T)$.
This similarly improves the prior state of the art.




\citet{Lee21:Last} showed the first last-iterate results for EFGs using algorithms that require linear time per iteration. In particular, they show that the dilated entropy DGF combined with optimistic online mirror descent leads to last-iterate convergence at a linear rate.
However, their result requires learning rates $\eta \leq 1/(8|\Sigma_i|)$. This learning rate is impractically small in practice. In contrast, our last-iterate linear-rate result for KOMWU allows learning rates of size $1/8$.
That said, our result is not directly comparable to theirs. The existence of a unique Nash equilibrium in the EFG representation is a necessary condition for uniqueness in the NFG representation. However, it is possible that the NFG has additional equilibria even when the EFG does not.
\citet{Wei21:Linear} conjecture that linear-rate convergence holds even without the assumption of a unique Nash equilibrium. If this conjecture turns out to be true for NFGs, then \cref{thm:kernel omwu equivalent} would immediately imply that KOMWU also has last-iterate linear-rate convergence without the uniqueness assumption.


\subsection{Experimental Evaluation}
We numerically investigate agents learning under the COLS in Kuhn and Leduc poker \citep{Kuhn50:Simplified,Southey05:Bayes}. %
We compare the maximum per-player regret cumulated by KOMWU for four different choices of constant learning rate, against that cumulated by two standard algorithms from the extensive-form game solving literature (CFR and CFR(RM+)). More details about the games and the algorithms are given in \cref{app:experiments}. Results are shown in \cref{fig:experiments}. We observe that the per-player regret cumulated by KOMWU plateaus and remains constants, unlike the CFR variants. This behavior is consistent with the near-optimal per-player regret guarantees of KOMWU (\cref{thm:komwu efg}).

\begin{figure}[ht]
    \includegraphics[width=1\linewidth]{figs/single_row-crop.pdf}
    \vspace{-4mm}
    \caption{Maximum per-player regret cumulated by KOWMU compared to two variants of the CFR algorithm.}
    \label{fig:experiments}
\end{figure}



\section{Discussion and Conclusions}



Our method based on stabilizing forward and backward pass, resulted in improved accuracy over the baseline and it was able to predict optimal dampening, sharpness and tail-fatness before training. 
Our findings are coherent with the line of research that has established that stabilizing gradients and representations at initialization results in better performance \cite{glorot2010understanding, orthogonal_initialization, he2015delving, roberts2022principles, defazio2022scaling, bengio1994learning, hochreiter1997long, hochreiter2001gradient, arjovsky2016unitary, pascanu2013difficulty}. Moreover it gives an initial reply to the question raised by
\cite{surrogate2019, zenke2021remarkable}, which asked  for a theoretical justification of initialization and SG choice for Spiking Neural Networks. With a similar intention, \cite{rossbroich2022fluctuation} proposed an approach that guarantees sparsity of activity at initialization to pick the weights distribution at initialization, resulting in improved accuracy. Our method differs from theirs in that it starts from a principle of stability to derive constraints, instead of a principle of sparsity. It differs also in that we use it to define the SG shape at initialization, not only the weights distribution, and we can show mathematically how weights initialization is intertwined to the SG shape choice. Our results suggest that a tedious hyper-parameter grid-search can be often avoided by making use of sound and established principles of learning.

One of the conditions was designed to hit the most sensitive part of an SG, its center, which resulted in a low sparsity requirement at initialization. This is very uncommon in the Neuromorphic literature, since sparsity brings large energy gains \cite{henderson2020towards,blouw2019benchmarking, 9395703,taulsnn, rossbroich2022fluctuation}.
However, the energy gains of SNNs also come from their binary activity. A matrix-vector multiplication, with a $\mathbb{R}^{m\times n}$ matrix, has an energy cost of $mnE_{MAC}$ for a real vector, and of $mn\rho E_{AC}$ for a binary vector, where $\rho$ is the Bernouilli probability of the binary vector, and in our case the neuron firing rate, and $E_{AC}, E_{MAC}$ are the energies of an accumulate and a multiply-accumulate operation \cite{yin2021accurate, hunger2005floating}. Since MAC are more costly than AC, 31 times on a $45$nm complementary metal–oxide–semiconductor \cite{yin2021accurate, horowitz20141}, we have energy savings with any $\rho$, e.g., when all neurons fire ($\rho=1$) and when they fire half of the time steps ($\rho=1/2$). This gain does not depend on the simulation speed, since it compares a spiking and an analogue computation, at the same computation speed.
Typically requiring more sparsity through a sparsity encouraging loss term, leads to a measurable decrease in performance \cite{zenke2021remarkable, rossbroich2022fluctuation}. However we observed that it is actually possible to achieve higher performance with higher sparsity, by starting with a strong firing rate at initialization, since their synergy acts as a regularization mechanism. This was possible also because the sparsity encouraging loss term was introduced gradually, and because its contribution was kept comparable to the task loss towards the end of training.

We observed that the more complex the task is and the more complex the network to train is, the more drastic is the difference in performance of different SG shapes. It is known that learning is possible with a wide variety of SG shapes \cite{zenke2021remarkable} and the community has not yet settled for one shape or one method to reliably choose which SG to use in each case \cite{surrogate2019}. We showed how to apply a well known stability principle to the forward and backward pass of the simplest Spiking Neural Network, the LIF, as a starting point, but we think that the principles of good Neuromorphic initialization can be further elaborated, in order to tackle more complex tasks and networks.




\section*{Acknowledgments}
This material is based on work supported by the National Science Foundation under grants IIS-1718457, IIS-1901403, IIS-1943607, and CCF-1733556, and the ARO under award W911NF2010081.


\bibliographystyle{icml2022}
\bibliography{dairefs}

\clearpage
\onecolumn
\appendix
\section{Additional Related Work}\label{app:related works}
\subsection{More Results for Optimistic Algorithms in Games}
For individual regret in multi-player general-sum NFGs, \citet{Syrgkanis15:Fast} first show $\bigOh(T^{1/4})$ regret for general optimistic OMD and FTRL algorithms.
The result is improved to $\bigOh(T^{1/6})$ by \citep{Chen20:Hedging}, but only for OMWU in two-player NFGs.
\citet{Daskalakis21:Near} show that OMWU enjoys $\bigOh(\log^4 T)$ regret in multi-player general-sum NFGs.

As for last-iterate convergence in two-player zero-sum games, \citet{Daskalakis18:Last} show an asymptotic result for OMWU under the unique Nash equilibrium assumption.
\citet{Wei21:Linear} further show a linear convergence rate while allowing larger learning rates under the same assumption.
\citet{Hsieh21:Adaptive} show another asymptomatic convergence result without the assumption.
It is also worth noting that OGDA, another popular optimistic algorithm, has been shown its last-iterate convergence in general polyhedron games \citep{Wei21:Linear}.
\subsection{Approaches in Online Combinatorial Optimization}
Besides performing MWU/OMWU over vertices, we review two additional approaches in online combinatorial optimization:

\paragraph{OMD over the Convex Hull}
This approach is running Online Mirror Descent (OMD) over the convex hull \citep{koolen2010hedging,audibert2014regret}.
It is well known that OMD with the negative entropy regularizer results in a (dimension-wise) multiplicative weight update.
For the case that the set of vertices is a standard basis, this algorithm coincides with the MWU over the probability simplex.
However, for general cases, it requires to project back to the convex hull and the procedure may not be efficient.
\citet{helmbold2009learning} first used this approach for permutations, and \citet{koolen2010hedging} generally studied it for arbitrary 0/1 polyhedral sets and show its efficiency for more cases.

\paragraph{FTPL}
Another approach is called Follow the Perturbed Leader \citep{Kalai05:Efficient}.
This approach adds a random perturbation to the cumulative loss vector, and greedily selects the vertex with minimal perturbed loss.
The latter procedure corresponds to linear optimization over the set of vertices, which can be solved efficiently for most cases of interest.
We are not aware of any previous work using this approach for EFGs though.

\section{Pseudocode}\label{app:pseudocode}

Below we show pseudocode for OMWU and Vertex OMWU (\cref{sec:vertex}).

\begin{figure}[H]
    \begin{minipage}[t]{.49\textwidth}
        \begin{algorithm}[H]
            \caption{OMWU}
            \label{algo:vanilla OMWU}
            \DontPrintSemicolon
            \KwData{Finite set of choices $\cA$, learning rates $\eta\^t > 0$\!\!\!}
            \BlankLine{}
            \vspace{4mm}
            $\vl\^0,~\vm\^0 \gets \vzero\in\bbR^{\cA};~~\vlam\^0 \gets \frac{1}{|\cA|}\vone\in\Delta(\cA)$\;%
            \label{ln:vanilla init}
            \For{$t=1,2,\dots$}{
            \textbf{receive} prediction $\vm\^t\in\bbR^{\cA}$ of next loss\;
            \Comment{\color{commentcolor}set $\vec{m}\^t = \vec{0}$ for non-predictive variant]}
            $\vw\^t \gets \vl\^{t-1} - \vm\^{t-1} + \vm\^t$\;
            \vspace{8mm}
            \For{$a \in \cA$}{
            $\displaystyle
                \vlam\^t[a] \gets \frac{\vlam\^{t-1}[a]\cdot e^{-\eta\^t\,\vw\^t[a]}}{\sum_{a' \in \cA} \vlam\^{t-1}[a']\cdot e^{-\eta\^t\,\vw\^t[a']}}
            $\label{ln:vanilla lam update}
            }
            \vspace{12.5mm}
            \textbf{output} $\vlam\^t \in \Delta(\cA)$\;
            \textbf{receive} loss vector $\vl\^t \in \bbR^{\cA}$\;
            }
        \end{algorithm}%
    \end{minipage}%
    \hfill%
    \begin{minipage}[t]{.49\textwidth}
        \begin{algorithm}[H]
            \caption{Vertex OMWU}
            \label{algo:vertex OMWU}
            \DontPrintSemicolon
            \KwData{\makebox[5cm][l]{Polytope $\Omega\!\subseteq\!\bbR^d$ with vertices $\{\vv_1,\!...,\!\vv_k\!\}\!\eqqcolon\!\cV_\Omega$,}\newline learning rates $\eta\^t > 0$}
            \BlankLine{}
            $\vl\^0,~\vm\^0 \gets \vzero\in\bbR^d;~~\vlam\^0 \gets \frac{1}{|\cV_\Omega|}\vone\in\Delta(\cV_\Omega)$\;%
            \label{ln:vertex init}
            \For{$t=1,2,\dots$}{
            \textbf{receive} prediction $\vm\^t\in\bbR^d$ of next loss\;
            \Comment{\color{commentcolor}set $\vec{m}\^t = \vec{0}$ for non-predictive variant]}
            $\vw\^t \gets \vl\^{t-1} - \vm\^{t-1} + \vm\^t$\;
            \Hline{}
            \Comment{\color{commentcolor}Run the OMWU update on $\vlam$ using $\cA=\cV_\Omega$]\!\!\!\!}\vspace{.5mm}
            \For{$\vv \in \cV_\Omega$}{
            $\displaystyle
                \vlam\^t[\vv] \gets \frac{\vlam\^{t-1}[\vv]\cdot e^{-\eta\^t\,\langle\vw\^{t},\vv\rangle}}{\sum_{\vv' \in \cV_\Omega} \vlam\^{t-1}[\vv']\cdot e^{-\eta\^t\langle \vw\^{t}\!,\vv'\rangle}}
            $\!\!\!\!\!\!\label{ln:vertex lam update}
            }
            \Hline{}
            \Comment{\color{commentcolor}Compute new convex combination of vertices]\!\!\!\!}\vspace{.5mm}
            $\vx\^t \gets \sum_{\vv\in\cV_\Omega} \vlam\^t[\vv]\cdot\vv$\label{ln:vertex xt}\;\vspace{1mm}
            \textbf{output} $\vx\^t \in \Omega$\;
            \textbf{receive} loss vector $\vl\^t \in \bbR^d$\;
            }
        \end{algorithm}%
    \end{minipage}
\end{figure}
\section{Extensive-Form Games}
\label{app:efgs}


In a \emph{tree-form sequential decision process (TFSDP)} problem the agent
interacts with the environment in two ways: at \emph{decision points}, the
agent must act by picking an action from a set of legal actions; at
\emph{observation points}, the agent observes a signal drawn from a set of
possible signals.
Different decision points can have different sets of legal actions, and
different observation points can have different sets of possible signals.
Decision and observation points are structured as a \emph{tree}: under the standard assumption that
the agent
is not forgetful, so, it is not possible for the agent to cycle back to a
previously encountered decision or observation point by following the
structure of the decision problem.

As an example, consider the
simplified game of \emph{Kuhn poker}~\citep{Kuhn50:Simplified}, depicted
in~\cref{fig:kuhn}. Kuhn poker is a standard benchmark in the EFG-solving community.
In Kuhn poker, each player puts an ante worth $1$ into the pot. Each player is then privately dealt one card from a deck that contains $3$ unique cards (Jack, Queen, King). Then, a single round of betting then occurs, with the following dynamics. First, Player $1$ decides to either check or bet $1$. Then,
\begin{itemize}[nolistsep]
    \item If Player 1 checks Player 2 can check or raise $1$.
          \begin{itemize}[nolistsep]
              \item If Player 2 checks a showdown occurs; if Player 2 raises Player 1 can fold or call.
                    \begin{itemize}
                        \item If Player 1 folds Player 2 takes the pot; if Player 1 calls a showdown occurs.
                    \end{itemize}
          \end{itemize}
    \item If Player 1 raises Player 2 can fold or call.
          \begin{itemize}[nolistsep]
              \item If Player 2 folds Player 1 takes the pot; if Player 2 calls a showdown occurs.
          \end{itemize}
\end{itemize}
When a showdown occurs, the player with the higher card wins the pot and the game immediately ends.

\begin{figure}[th]
    \centering
    \begin{tikzpicture}[scale=1.0]
    \tikzset{edge from parent/.style={}}
    \tikzset{edge from parent path={(\tikzparentnode) -- (\tikzchildnode.north)}}
    \tikzset{level distance=1.05cm}
    \tikzset{sibling distance=.60cm}
    \Tree
     [.\node[obspt](P1){};
      [.\node[decpt](S1) {};
       [.\node[obspt](B1) {};
        [.\node[termina](T1) {};]
        [.\node[decpt](S2) {};
         [.\node[termina](T2) {};]
         [.\node[termina](T3) {};]
        ]
       ]
       [.\node[termina](S3) {};]
      ]
      [.\node[decpt](S4) {};
       [.\node[obspt](B2) {};
        [.\node[termina](T6) {};]
        [.\node[decpt](S5) {};
         [.\node[termina](T7) {};]
         [.\node[termina](T8) {};]
        ]
       ]
       [.\node[termina](S6) {};]
      ]
      [.\node[decpt](S7) {};
       [.\node[obspt](B3) {};
        [.\node[termina](T11) {};]
        [.\node[decpt](S8) {};
         [.\node[termina](T12) {};]
         [.\node[termina](T13) {};]
        ]
       ]
       [.\node[termina](S9) {};]
      ]
    ];
    
   \node[black!70!white,xshift=-4mm,yshift=2mm] at (P1) {$k_1$};
   \node[black!70!white,xshift=-4mm] at (S1) {$j_1$};
   \node[black!70!white,xshift=-4mm] at (S4) {$j_2$};
   \node[black!70!white,xshift=4mm] at (S7) {$j_3$};
   \node[black!70!white,xshift=-4mm] at (B1) {$k_2$};
   \node[black!70!white,xshift=-4mm] at (B2) {$k_3$};
   \node[black!70!white,xshift=-4mm] at (B3) {$k_4$};
   \node[black!70!white,xshift=4mm] at (S2) {$j_4$};
   \node[black!70!white,xshift=4mm] at (S5) {$j_5$};
   \node[black!70!white,xshift=4mm] at (S8) {$j_6$};
   
   \draw[semithick,dashed] (P1) -- (S1);
   \draw[semithick,dashed] (P1) -- (S4);
   \draw[semithick,dashed] (P1) -- (S7);
   \draw[semithick] (S1) -- (B1);
   \draw[semithick] (S1) -- (S3);
   \draw[semithick] (S4) -- (B2);
   \draw[semithick] (S4) -- (S6);
   \draw[semithick] (S7) -- (B3);
   \draw[semithick] (S7) -- (S9);
   \draw[semithick,dashed] (B1) -- (T1);
   \draw[semithick,dashed] (B1) -- (S2);
   \draw[semithick,dashed] (B2) -- (T6);
   \draw[semithick,dashed] (B2) -- (S5);
   \draw[semithick,dashed] (B3) -- (T11);
   \draw[semithick,dashed] (B3) -- (S8);
   
   \draw[semithick] (T2) -- (S2) -- (T3);
   \draw[semithick] (T7) -- (S5) -- (T8);
   \draw[semithick] (T12) -- (S8) -- (T13);


   \path ($(S2)+(-1mm,-3mm)$) --node[text=black,fill=white,inner ysep=.5mm,inner xsep=0,xshift=-1mm,yshift=1mm]{\small fold} (T2);
   \path ($(S2)+(1mm,-3mm)$) --node[text=black,fill=white,inner ysep=.5mm,inner xsep=0,xshift=1mm,yshift=1mm]{\small call} (T3);
   \path ($(S5)+(-1mm,-3mm)$) --node[text=black,fill=white,inner ysep=.5mm,inner xsep=0,xshift=-1mm,yshift=1mm]{\small fold} (T7);
   \path ($(S5)+(1mm,-3mm)$) --node[text=black,fill=white,inner ysep=.5mm,inner xsep=0,xshift=1mm,yshift=1mm]{\small call} (T8);
   \path ($(S8)+(-1mm,-3mm)$) --node[text=black,fill=white,inner ysep=.5mm,inner xsep=0,xshift=-1mm,yshift=1mm]{\small fold} (T12);
   \path ($(S8)+(1mm,-3mm)$) --node[text=black,fill=white,inner ysep=.5mm,inner xsep=0,xshift=1mm,yshift=1mm]{\small call} (T13);
   \path (S1) --node[text=black,fill=white,inner ysep=.5mm,inner xsep=0,yshift=1mm]{\small check} (B1);
   \path (S1) --node[text=black,fill=white,inner ysep=.5mm,inner xsep=0,yshift=1mm]{\small raise} (S3);
   \path (S4) --node[text=black,fill=white,inner ysep=.5mm,inner xsep=0,yshift=1mm]{\small check} (B2);
   \path (S4) --node[text=black,fill=white,inner ysep=.5mm,inner xsep=0,yshift=1mm]{\small raise} (S6);
   \path (S7) --node[text=black,fill=white,inner ysep=.5mm,inner xsep=0,yshift=1mm]{\small check} (B3);
   \path (S7) --node[text=black,fill=white,inner ysep=.5mm,inner xsep=0,yshift=1mm]{\small raise} (S9);

   \path (P1) --node[text=black,inner ysep=1mm,fill=white]{\small jack} (S1);
   \path (P1) --node[text=black,inner ysep=.3mm,fill=white]{\small queen} (S4);
   \path (P1) --node[text=black,inner ysep=1mm,fill=white]{\small king} (S7);

   \path (B1) --node[text=black,fill=white,inner ysep=.5mm,inner xsep=0,xshift=-1mm,yshift=1mm]{\small check} (T1);
   \path (B1) --node[text=black,fill=white,inner ysep=.5mm,inner xsep=0,xshift=1mm,yshift=1mm]{\small raise} ($(S2)$);
   \path (B2) --node[text=black,fill=white,inner ysep=.5mm,inner xsep=0,xshift=-1mm,yshift=1mm]{\small check} (T6);
   \path (B2) --node[text=black,fill=white,inner ysep=.5mm,inner xsep=0,xshift=1mm,yshift=1mm]{\small raise} ($(S5)$);
   \path (B3) --node[text=black,fill=white,inner ysep=.5mm,inner xsep=0,xshift=-1mm,yshift=1mm]{\small check} (T11);
   \path (B3) --node[text=black,fill=white,inner ysep=.5mm,inner xsep=0,xshift=1mm,yshift=1mm]{\small raise} ($(S8)$);
\end{tikzpicture}

    \caption{Tree-form sequential decision making process of the first
        acting player in the game of Kuhn poker.}
    \label{fig:kuhn}
\end{figure}

As soon as the game starts, the agent observes a private card that has been
dealt to them; this is observation point $k_1$, whose set of possible
signals is $S_{k_1} \defeq \{\text{jack},\text{queen},\text{king}\}$.
Should the agent observe the `jack' signal, the decision problem transitions to
the decision point $j_1$, where the agent must pick one action from the set
$A_{j_1} \defeq \{\text{check}, \text{raise}\}$.
If the agent picks `raise', the decision process terminates; otherwise, if
`check' is chosen, the process transitions to observation point $k_2$,
where the agent will observe whether the opponent checks (at which point
the interaction terminates) or raises.
In the latter case, the process transitions to decision point $j_4$, where the
agent picks one action from the set
$A_{j_4} \defeq \{\text{fold},\text{call}\}$.
In either case, after the action has been selected, the interaction terminates.



\section{Experimental Evaluation}\label{app:experiments}

\paragraph{Game instances}
We numerically investigate agents learning under the COLS in Kuhn and Leduc poker \citep{Kuhn50:Simplified,Southey05:Bayes}, standard benchmark games from the extensive-form games literature.
\begin{description}
    \item[\emph{Kuhn poker}] The two-player variant of Kuhn poker first appeared in \citep{Kuhn50:Simplified}. In this paper, we use the multiplayer variant, as described by \citet{Farina18:Ex}. In a multiplayer Kuhn poker game with $r$ ranks, a deck with $r$ unique cards is used. At the beginning of the game, each player pays one chip to the pot (\emph{ante}), and is dealt a single private card (their \emph{hand}). The first player to act can \emph{check} or \emph{bet}, \ie put an additional chip in the pot. Then, the second player can check or bet after a first player's check, or fold/call the first player's bet. If no bet was previously made, the third player can either check or bet, and so on in turn. If a bet is made by a player, each subsequent player needs to decide whether to \emph{fold} or \emph{call} the bet. The betting round if all players check, or if every player has had an opportunity to either fold or call the bet that was made. The player with the highest card who has not folded wins all the chips in the pot.
    \item[\emph{Leduc poker}] We use a multiplayer version of the classical Leduc hold'em poker introduced by \citet{Southey05:Bayes}. We employ game instances of rank 3. The deck consists of three suits with 3 cards each. Our instances are parametric in the maximum number of bets, which in limit hold'em is not necessarily tied to the number of players. As in Kuhn poker, we set a cap on the number of raises to one bet. As the game starts, players pay one chip to the pot. Then, two betting rounds follow. In the first one, a single private card is dealt to each player while in the second round a single board card is revealed. The raise amount is set to 2 and 4 in the first and second round, respectively.
\end{description}
For each game, we consider a 3-player and a 4-player variant. The 3-player Kuhn variant uses a deck with $r=12$ ranks. The 4-player variant uses a deck with a reduced number of ranks equal to $r=5$ to avoid excessive memory usage.

\paragraph{CFR and CFR(RM+)} Modern variants of counterfactual regret minimization (CFR) are the current practical state-of-the-art in two-player zero-sum extensive-form game solving. We implemented both the original CFR algorithm by~\citet{Zinkevich07:Regret}, and a more modern variant (which we denote `CFR(RM+)') using the Regret Matching Plus regret minimization algorithm at each decision point~\citep{Tammelin15:Solving}.

\paragraph{Discussion of results}
We compare the maximum per-player regret cumulated by KOMWU for four different choices of constant learning rate $\eta\^t = \eta \in \{ 0.1, 1, 5, 10\}$, against that cumulated by CFR and CFR(RM+).

We remark that the payoff ranges of these games are not $[0,1]$ (\ie the games have not been normalized). The payoff range of Kuhn poker is $6$ for the 3-player variant and $8$ for the 4-player variant. The payoff range of Leduc poker is $21$ for the 3-player variant and $28$ for the 4-player variant. So, a learning rate value of $\eta=0.1$ corresponds to a significantly smaller learning rate in the normalized game where the payoffs have been shifted and rescaled to lie within $[0,1]$ as required in the statements of \cref{prop:omwu near optimal,prop:omwu optimal sum,prop:omwu last iterate}.

Results are shown in \cref{fig:all games}. In all games, we observe that the maximum per-player regret cumulated by KOMWU plateaus and remains constants, unlike the CFR variants. This behavior is consistent with the near-optimal per-player regret guarantees of KOMWU (\cref{thm:komwu efg}). In the 3-player variant of Leduc poker, we observe that the largest learning rate we use, $\eta=10$, leads to divergent behavior of the learning dynamics.

\begin{figure}[H]\centering
    \includegraphics[scale=.85]{figs/all-crop.pdf}
    \caption{Maximum per-player regret cumulated by KOMWU for four different choices of constant learning rate $\eta\^t = \eta \in \{ 0.1, 1, 5, 10\}$, compared to that cumulated by CFR and CFR(RM+) in two multiplayer poker games.}
    \label{fig:all games}
\end{figure}
\section{Proofs}\label{app:proofs}


\thmefgkernel*
\begin{proof}
    In the proof of this result, we will make use of the following additional notation.
    Given any $\vx \in \bbR^{\Sigma_i}$ and a $j\in \cJ_i$, we let $\vx_{(j)} \in \bbR^{\Sigma_{i,j}^*}$ denote the subvector obtained from $\vx$ by only considering sequences $\sigma \in \Sigma_{i,j}^*$, that is, the vector whose entries are defined as $\vx_{(j)}[\sigma] = \vx[\sigma]$ for all $\sigma \in \Sigma_{i,j}^*$.

    \paragraph{Proof of~\eqref{eq:efg kernel computation}}
    Direct inspection of the definitions of $\Pi_i$ and $\Pi_{i,j}$ (given in \cref{sec:efg notation}), together with the observation that the $\{\Sigma_{i,j}^* : j \in \mathcal{C}_\emptyseq\}$ form a partition of $\Sigma^*_i$, reveals that
    \[
        \Pi_i = \mleft\{\vpi \in \{0,1\}^{\Sigma_i}: \begin{array}{l}\circled{1}~~\vpi[\emptyseq] = 1\\[1mm] \circled{2}~~\vpi_{(j)} \in \Pi_{i,j} \qquad\forall\, j\in\mathcal{C}_\emptyseq \end{array} \mright\}
        \numberthis{eq:Pi i as prod}
    \]
    The observation above can be summarized informally into the statement that ``\emph{$\Pi_i$ is equal, up to permutation of indices, to the Cartesian product $\bigtimes_{j\in \mathcal{C}_\emptyseq}\Pi_{i,j}$}''.
    The idea for the proof is then to use that Cartesian product structure in the definition of 0/1-polyhedral kernel~\eqref{eq:K Omega}, as follows
    \[
        K_{Q_i}(\vx, \vy) &= \sum_{\vec{\pi} \in \Pi_i} \prod_{\sigma \in \vpi} \vx[\sigma]\,\vy[\sigma]\\
        &=\sum_{\vpi\in\Pi_i}\mleft(\vx[\emptyseq]\,\vy[\emptyseq]\prod_{j'\in\mathcal{C}_\emptyseq}\prod_{\sigma\in\vpi_{(j')}} \vx[\sigma]\,\vy[\sigma]\mright)\\
        &=\sum_{\vpi_{(j)}\in\Pi_{i,j} ~\forall\,j \in \mathcal{C}_\emptyseq}\mleft(\vx[\emptyseq]\,\vy[\emptyseq] \prod_{j'\in\mathcal{C}_\emptyseq}\prod_{\sigma\in\vpi_{(j')}} \vx[\sigma]\,\vy[\sigma]\mright)\\
        &=\vx[\emptyseq]\,\vy[\emptyseq] \sum_{\vpi_{(j)}\in\Pi_{i,j} ~\forall\,j \in \mathcal{C}_\emptyseq}\mleft(\prod_{j'\in\mathcal{C}_\emptyseq}\prod_{\sigma\in\vpi_{(j')}} \vx[\sigma]\,\vy[\sigma]\mright)\\
        &=\vx[\emptyseq]\,\vy[\emptyseq] \prod_{j\in\mathcal{C}_\emptyseq} \sum_{\vpi_{(j)}\in\Pi_{i,j}}\prod_{\sigma\in\vpi_{(j)}} \vx[\sigma]\,\vy[\sigma]\\
        &= \vx[\emptyseq]\,\vy[\emptyseq] \prod_{j\in\mathcal{C}_\emptyseq} K_{j}(\vx, \vy),
    \]
    where the second equality follows from the fact that $\{\emptyseq\}\cup\{\Sigma_{i,j}:j\in\mathcal{C}_\emptyseq\}$ form a partition of $\Sigma_i$, the third equality follows from~\eqref{eq:Pi i as prod}, the fifth equality from the fact that each $\vpi_j\in\Pi_{i,j}$ can be chosen independently, and the last equality from the definition of partial kernel function~\eqref{eq:def Kj}.

    \paragraph{Proof of~\eqref{eq:efg kernel computation 2}}
    Similarly to what we did for~\eqref{eq:efg kernel computation}, we start by giving an inductive characterization of $\Pi_{i,j}$ as a function of the children $\Pi_{i,j'}$ for $j' \in \cup_{a\in A_j} \mathcal{C}_{ja}$. Specifically, direct inspection of the definitions of $\Pi_{i,j}$, together with the observation that the $\{\Sigma_{i,j'}^* : j' \in \cup_{a \in A_j}\mathcal{C}_{ja}\}$ form a partition of $\Sigma_{i,j}^*$, reveals that
    \[
        \Pi_{i,j} = \mleft\{\vpi \in \{0,1\}^{\Sigma_{i,j}^*}: \begin{array}{l}\circled{1}~~\sum_{a \in A_j}\vpi[ja] = 1\\[1mm] \circled{2}~~\vpi_{(j')} \in \vpi[ja]\cdot \Pi_{i,j'} \qquad\forall\, a\in A_j,~ j'\in\mathcal{C}_{ja} \end{array} \mright\}.
        \numberthis{eq:Pi j as ch prod intermediate}
    \]
    From constraint \circled{1}\, together with the fact that $\vpi[ja]\in\{0,1\}$ for all $a \in A_j$, we conclude that exactly one $a^* \in A_j$ is such that $\vpi[ja^*] = 1$, while $\vpi[ja] = 0$ for all other $a \in A_j, a \neq a^*$. So, we can rewrite~\eqref{eq:Pi j as ch prod intermediate} as
    \[
        \Pi_{i,j} = \bigcup_{a^* \in A_j}\mleft\{
        \vpi\in\{0,1\}^{\Sigma_{i,j}^*}: \begin{array}{ll}
            \circled{1}~~\vpi[ja^*] = 1                                                                                      \\
            \circled{2}~~\vpi[ja] = 0             \qquad\qquad & \forall\, a\in A_j, a \neq a^*                              \\
            \circled{3}~~\vpi_{(j')} \in \Pi_{i,j'}            & \forall\, j' \in \mathcal{C}_{ja^*}                         \\
            \circled{4}~~\vpi_{(j')} = \vzero                  & \forall\, j'\in\cup_{a \in A_j, a \neq a^*}\mathcal{C}_{ja} \\
        \end{array}
        \mright\},
        \numberthis{eq:Pi j as ch prod}
    \]
    where the union is clearly disjoint.     The above equality can be summarized informally into the statement that ``\emph{$\Pi_{i,j}$ is equal, up to permutation of indices, to a disjoint union over actions $a^*\in A_j$ of Cartesian products $\bigtimes_{j\in \mathcal{C}_{ja^*}}\Pi_{i,j}$}''.
    We can then use the same set of manipulations we already used in the proof of~\eqref{eq:efg kernel computation} to obtain
    \[
        K_{j}(\vx, \vy) &= \sum_{\vec{\pi} \in \Pi_{i,j}} \prod_{\sigma \in \vpi} \vx[\sigma]\,\vy[\sigma]\\
        &= \sum_{\vec{\pi} \in \Pi_{i,j}} \mleft( \vx[ja^*]\,\vy[ja^*]\prod_{j'\in\mathcal{C}_{ja^*}}\prod_{\sigma\in\vpi_{(j')}} \vx[\sigma]\,\vy[\sigma]\mright)\\
        &=\sum_{a^* \in A_j}\sum_{\vpi_{j'}\in\Pi_{i,j'}~\forall\,j' \in \mathcal{C}_{ja^*}} \mleft( \vx[ja^*]\,\vy[ja^*]\prod_{j'\in\mathcal{C}_{ja^*}}\prod_{\sigma\in\vpi_{(j')}} \vx[\sigma]\,\vy[\sigma]\mright)\\
        &=\sum_{a^* \in A_j} \mleft( \vx[ja^*]\,\vy[ja^*] \prod_{j'\in\mathcal{C}_{ja^*}} \sum_{\vpi_{(j')}\in\Pi_{i,j'}}\prod_{\sigma\in\vpi_{(j')}} \vx[\sigma]\,\vy[\sigma]\mright)\\
        &=\sum_{a \in A_j} \mleft( \vx[ja]\,\vy[ja] \prod_{j' \in \mathcal{C}_{ja}} K_{j'}(\vx, \vy) \mright),
    \]
    where the second equality follows from the fact that the $\{\Sigma_{i,j'}^* : j' \in \cup_{a \in A_j}\mathcal{C}_{ja}\}$ form a partition of $\Sigma_{i,j}^*$, third equality follows from~\eqref{eq:Pi j as ch prod}, the fourth equality from the fact that each $\vpi_{j'}\in\Pi_{i,j'}$ can be picked independently, and the last equality from the definition of partial kernel function~\eqref{eq:def Kj} as well as renaming $a^*$ into $a$.
\end{proof}

\propefgratio*
\begin{proof}
    Note that since $\vx > \vzero$, clearly $K_{Q_i}(\vx, \vone), K_j(\vx, 1) > 0$. Furthermore, from~\eqref{eq:diff phi} we have that for all $\sigma \in \Sigma_i$
    \[
        K_{Q_i}(\vx, \vone) - K_{Q_i}(\vx, \ebar_\sigma) &= \langle \phi_{Q_i}(\vone) - \phi_{Q_i}(\ebar_\sigma), \phi_{Q_i}(\vx)\rangle \\
        &= \sum_{\substack{\vpi \in \Pi_i\\\vpi[\sigma] = 1}}\prod_{\sigma' \in \vpi} \vx[\sigma']
        \numberthis{eq:ratio to diff}\\
        &> 0.
    \]
    The above inequality immediately implies that $0 < K_{Q_i}(\vx,\ebar_{p_j})/K_{Q_i}(\vx, \vone) < 1$ and therefore all denominators in the statement are nonzero, making the statement well-formed.

    \newcommand{\splice}[2]{(\!(#1\,|\,#2)\!)}
    In light of~\eqref{eq:ratio to diff}, we further have
    \[
    & \frac{1 - K_{Q_i}(\vec{x}, \bar{\vec{e}}_{ja}) / K_{Q_i}(\vec{x}, \vone)}{1 - K_{Q_i}(\vec{x}, \bar{\vec{e}}_{p_j}) / K_{Q_i}(\vec{x}, \vone)} = \frac{\vec{x}[ja]\prod_{j'\in\mathcal{C}_{ja}} K_{j'}(\vec{x},\vone)}{K_j(\vec{x}, \vone)}\\[2mm]
    & \hspace{2cm}\iff\quad \frac{K_{Q_i}(\vec{x}, \vone) - K_{Q_i}(\vec{x}, \bar{\vec{e}}_{ja})}{K_{Q_i}(\vec{x}, \vone) - K_{Q_i}(\vec{x}, \bar{\vec{e}}_{p_j})} = \frac{\vec{x}[ja]\prod_{j'\in\mathcal{C}_{ja}} K_{j'}(\vec{x},\vone)}{K_j(\vec{x}, \vone)}\\[2mm]
        & \hspace{2cm}\iff\quad\frac{\sum_{\vpi \in \Pi_i, \vpi[ja] = 1}\prod_{\sigma \in \vpi} \vx[\sigma]}{\sum_{\vpi \in \Pi_i, \vpi[p_j] = 1}\prod_{\sigma \in \vpi} \vx[\sigma]} = \frac{\vec{x}[ja]\prod_{j'\in\mathcal{C}_{ja}} K_{j'}(\vec{x},\vone)}{K_j(\vec{x}, \vone)}\numberthis{eq:equivalent}
    \]
    We now prove~\eqref{eq:equivalent}. Let
    \[
        \mathcal{A} \defeq \{\vpi \in \Pi_i : \vpi[ja]=1\}, \qquad \mathcal{B} \defeq \{\vpi \in \Pi_i : \vpi[p_j]=1\}
    \]
    be the domains of the summations. From the definition of $\Pi_i$ (specifically, constraints \circled{2}~in the definition of $Q_i$, of which $\Pi_i$ is a subset; see \cref{sec:efg notation}), it is clear that $\mathcal{A} \subseteq \mathcal{B}$. Furthermore, it is straightforward to check, using the definitions of $\Pi_{i,j}$, $\Pi_i$, and $\mathcal{B}$, that
    \[
        \vpi_{(j)} \in \Pi_{i,j} \qquad\forall\,\vpi\in\mathcal{B}\numberthis{eq:vpij}
    \]

    We now introduce the function $\splice{\cdot}{\cdot} : \mathcal{B} \times \Pi_{i,j} \to \mathcal{B}$ defined as follows.
    Given any $\vpi \in \mathcal{B}$ and $\vpi' \in \Pi_{i,j}$, $\splice{\vpi}{\vpi'}$ is the vector obtained from $\vpi$ by replacing all sequences at or below decision point $j$ with what is prescribed by $\vpi'$; formally,
    \[
        \splice{\vpi}{\vpi'}[\sigma] \defeq \begin{cases}
            \vpi'[\sigma] & \text{if } \sigma \in \Sigma_{i,j}^* \\
            \vpi[\sigma]  & \text{otherwise}.
        \end{cases} \qquad\quad \forall\, \vpi\in\mathcal{B}, \vpi'\in\Pi_{i,j}
        \numberthis{eq:def splice}
    \]
    It is immediate to check that $\splice{\vpi}{\vpi'}$ is indeed an element of $\mathcal{B}$.
    We now introduce the following result.

    \begin{lemma}\label{obs:B}
        There exists a set $\mathcal{P} \subseteq \mathcal{B}$ such that every $\vpi'' \in \mathcal{B}$ can be uniquely written as $\vpi'' = \splice{\vpi}{\vpi'}$ for some $\vpi \in \mathcal{P}$ and $\vpi' \in \Pi_{i,j}$. Vice versa, given any $\vpi \in \mathcal{P}$ and $\vpi' \in \Pi_{i,j}$, then $\splice{\vpi}{\vpi'} \in \mathcal{B}$.
    \end{lemma}
    \begin{proof}
        The second part of the statement is straightforward. We now prove the first part.

        Fix any $\vpi^* \in \Pi_{i,j}$ and let $\mathcal{P} \defeq \{\splice{\vpi}{\vpi^*} : \vpi \in \mathcal{B}\}$. It is straightforward to verify that for any $\vpi'' \in \mathcal{B}$, the choices $\vpi \defeq \splice{\vpi''}{\vpi^*} \in \mathcal{P}$ and $\vpi' \defeq \vpi_{(j)} \in \Pi_{i,j}$ satisfy the equality $\splice{\vpi}{\vpi'} = \vpi''$. So, every $\vpi''\in\mathcal{B}$ can be expressed in \emph{at least one way} as $\vpi'' = \splice{\vpi}{\vpi'}$ for some $\vpi \in \mathcal{P}$ and $\vpi' \in \Pi_{i,j}$. We now show that the choice above is in fact the unique choice. First, it is clear from the definition of $\splice{\cdot}{\cdot}$ that $\vpi'$ must satisfy $\vpi' = \vpi''_{(j)}$, and so it is uniquely determined. Suppose now that there exist $\vpi, \tilde{\vpi}\in\mathcal{P}$ such that $\splice{\vpi}{\vpi'} = \splice{\tilde{\vpi}}{\vpi'}$. Then, $\vpi$ and $\tilde{\vpi}$ must coincide on all $\sigma \in \Sigma_i \setminus \Sigma_{i,j}^*$. However, since all elements of $\mathcal{P}$ are of the form $\splice{\vb}{\vpi^*}$ for some $\vb \in \mathcal{B}$, then $\vpi$ and $\tilde{\vpi}$ must also coincide on all $\sigma \in\Sigma_{i,j}^*$. So, $\vpi$ and $\tilde{\vpi}$ coincide on all coordinates $\sigma\in\Sigma_i$, and the statement follows.
    \end{proof}

    \cref{obs:B} exposes a convenient combinatorial structure of the set $\mathcal{B}$. In particular, it enables us to rewrite the denominator on the left-hand side of \eqref{eq:equivalent} as follows
    \[
        \sum_{\vpi \in \mathcal{B}} \prod_{\sigma\in\vpi} \vx[\sigma] &= \sum_{\vpi' \in \mathcal{P}}\sum_{\vpi''\in\Pi_{i,j}}\prod_{\sigma\in\splice{\vpi'}{\vpi''}} \vx[\sigma]\\
        &=\sum_{\vpi' \in \mathcal{P}}\sum_{\vpi''\in\Pi_{i,j}}\mleft(\prod_{\substack{\sigma\in\splice{\vpi'}{\vpi''}\\\sigma\in\Sigma_{i,j}}} \vx[\sigma]\mright)\mleft(\prod_{\substack{\sigma\in\splice{\vpi'}{\vpi''}\\\sigma\not\in\Sigma_{i,j}}} \vx[\sigma]\mright)\\
        &=\sum_{\vpi' \in \mathcal{P}}\sum_{\vpi''\in\Pi_{i,j}}\mleft(\prod_{\sigma\in\vpi''} \vx[\sigma]\mright)\mleft(\prod_{\substack{\sigma\in\vpi'\\\sigma\not\in\Sigma_{i,j}}} \vx[\sigma]\mright)\\
        &=\mleft(\sum_{\vpi''\in\Pi_{i,j}}\prod_{\sigma\in\vpi''} \vx[\sigma]\mright)\mleft(\sum_{\vpi' \in \mathcal{P}}\prod_{\substack{\sigma\in\vpi'\\\sigma\not\in\Sigma_{i,j}}} \vx[\sigma]\mright)\\
        &= K_j(\vx, \vone)\cdot \mleft(\sum_{\vpi' \in \mathcal{P}}\prod_{\substack{\sigma\in\vpi'\\\sigma\not\in\Sigma_{i,j}}} \vx[\sigma]\mright),\numberthis{eq:denom}
    \]
    where we used~\eqref{eq:def splice} in the third equality.

    We can use a similar technique to express the numerator of the left-hand side of~\eqref{eq:equivalent}. Let
    \[
        \Pi_{i,ja} \defeq \{\vpi \in \Pi_{i,j} : \vpi[ja] = 1\}.
    \]
    Using the constraints that define $\Pi_i$ and the definition of $\mathcal{A}$, it follows immediately that for any $\vpi\in\mathcal{A}$, $\vpi_{(j)} \in \Pi_{i,ja}$. Furthermore, a direct consequence of \cref{obs:B} is the following:
    \begin{corollary}\label{obs:A}
        The same set $\mathcal{P} \subseteq \mathcal{B}$ introduced in \cref{obs:B} is such that every $\vpi'' \in \mathcal{A}$ can be uniquely written as $\vpi'' = \splice{\vpi}{\vpi'}$ for some $\vpi \in \mathcal{P}$ and $\vpi' \in \Pi_{i,ja}$.
    \end{corollary}
    Using \cref{obs:A} and following the same steps that led to~\eqref{eq:denom}, we express the numerator of the left-hand side of~\eqref{eq:equivalent} as
    \[
        \sum_{\vpi \in \mathcal{A}} \prod_{\sigma\in\vpi} \vx[\sigma] &= \sum_{\vpi' \in \mathcal{P}}\sum_{\vpi''\in\Pi_{i,ja}}\prod_{\sigma\in\splice{\vpi'}{\vpi''}} \vx[\sigma]\\
        &=\sum_{\vpi' \in \mathcal{P}}\sum_{\vpi''\in\Pi_{i,ja}}\mleft(\prod_{\substack{\sigma\in\splice{\vpi'}{\vpi''}\\\sigma\in\Sigma_{i,j}}} \vx[\sigma]\mright)\mleft(\prod_{\substack{\sigma\in\splice{\vpi'}{\vpi''}\\\sigma\not\in\Sigma_{i,j}}} \vx[\sigma]\mright)\\
        &=\sum_{\vpi' \in \mathcal{P}}\sum_{\vpi''\in\Pi_{i,ja}}\mleft(\prod_{\sigma\in\vpi''} \vx[\sigma]\mright)\mleft(\prod_{\substack{\sigma\in\vpi'\\\sigma\not\in\Sigma_{i,j}}} \vx[\sigma]\mright)\\
        &=\mleft(\sum_{\vpi''\in\Pi_{i,ja}}\prod_{\sigma\in\vpi''} \vx[\sigma]\mright)\mleft(\sum_{\vpi' \in \mathcal{P}}\prod_{\substack{\sigma\in\vpi'\\\sigma\not\in\Sigma_{i,j}}} \vx[\sigma]\mright).
        \numberthis{eq:numer intermediate}
    \]
    The statement then follows immediately if we can prove that
    \[
        \sum_{\vpi\in\Pi_{i,ja}}\prod_{\sigma\in\vpi} \vx[\sigma] = \vec{x}[ja]\,\prod_{j'\in\mathcal{C}_{ja}} K_{j'}(\vx, \vone).
    \]
    To do so, we use the same approach as in the proof of \cref{thm:efg kernel computation}. In fact, we can directly use the inductive characterization of $\Pi_{i,j}$ obtained in~\eqref{eq:Pi j as ch prod} to write
    \[
        \Pi_{i,ja} = \mleft\{
        \vpi\in\{0,1\}^{\Sigma_{i,j}^*}: \begin{array}{ll}
            \circled{1}~~\vpi[ja] = 1                                                                                                \\
            \circled{2}~~\vpi[ja'] = 0             \qquad\qquad & \forall\, a'\in A_j                                    , a' \neq a \\
            \circled{3}~~\vpi_{(j')} \in \Pi_{i,j'}             & \forall\, j' \in \mathcal{C}_{ja}                                  \\
            \circled{4}~~\vpi_{(j')} = \vzero                   & \forall\, j'\in\cup_{a' \in A_j, a' \neq a}\mathcal{C}_{ja'}       \\
        \end{array}
        \mright\},
    \]
    which fundamentally uncovers the \emph{Cartesian-product structure of $\Pi_{i,ja}$}. Using the same technique as \cref{thm:efg kernel computation}, we then have
    \[
        \sum_{\vpi\in\Pi_{i,ja}}\prod_{\sigma\in\vpi} \vx[\sigma] &=
        \sum_{\vpi_{(j')}\in\Pi_{i,j'}~\forall\,j' \in \mathcal{C}_{ja}} \mleft( \vx[ja]\prod_{j'\in\mathcal{C}_{ja}}\prod_{\sigma\in\vpi_{(j')}} \vx[\sigma]\mright)\\
        &= \mleft( \vx[ja] \prod_{j'\in\mathcal{C}_{ja}} \sum_{\vpi_{(j')}\in\Pi_{i,j'}}\prod_{\sigma\in\vpi_{(j')}} \vx[\sigma]\mright)\\
        &= \mleft( \vx[ja] \prod_{j' \in \mathcal{C}_{ja}} K_{j'}(\vx, \vone) \mright),
    \]
    and the statement is proven.
\end{proof}

\propnumvertices*
\begin{proof}
    The proof is by induction. As the base case consider a single decision point $\Delta^b$ with $b \leq A$ actions. Then the number of vertices is $b \leq A = A^{\|\Delta^b\|_1}$.

    For the induction step we consider two cases.
    First, consider a polytope $Q$ whose root is a decision point with $b\leq A$ actions, with each action $a$ leading to a polytope $Q_a$ whose number of vertices $v_a$ satisfies the inductive assumption (if some action $a$ is a terminal action then we overload notation and let $v_a=1$ and $\|Q_a\|_1 = 0$).
    Then, the number of vertices of $Q$ is
    \[
        \sum_{a=1}^b v_a
        &\leq \sum_{a=1}^b A^{\|Q_a\|_1} \\
        &\leq b \cdot A^{\max_{a\in \range{b}}\|Q_a\|_1}\\
        &\leq A\cdot A^{\max_{a\in \range{b}}\|Q_a\|_1} \\
        &= A^{\|Q\|_1}.
    \]

    Second, consider a polytope $Q$ whose root is an observation point with $b$ observations, with each observation $o$ leading to a polytope $Q_o$ with $v_o$ vertices, such that the inductive assumption holds.
    Then, the number of vertices of $Q$ is
    \[
        v = \prod_{o=1}^b v_o
        &\leq \prod_{o=1}^b A^{\|Q_o\|_1}
        \leq A^{\sum_{o=1}^b \|Q_o\|_1}
        = A^{\|Q\|_1}.
    \]
\end{proof}


\section{Further Applications}\label{app:applications}

In this appendix, we illustrate additional 0/1-polyhedral domains in which our polyhedral kernel can be computed efficiently.

\subsection{$n$-sets}\label{sec:nsets}

We start from $n$-sets, that is, the 0/1-polydral set $\Omega^d_n \defeq \textrm{co}\{\vpi \in \{0,1\}^d: \|\vpi\|_1 = n\}$. Learning over $n$-sets is a classic problem first considered by~\citet{warmuth2008randomized} with an application to online Principal Component Analysis.
They proposed an Online Mirror Descent algorithm operating over the convex hull $\Omega^d_n$, with per-iteration complexity of $\bigOh(d^2)$.
The Follow-the-Perturbed-Leader approach~\citep{Kalai05:Efficient} is even faster with per-iteration complexity of $\bigOh(d\log d)$, but it often leads to sub-optimal regret bounds (see discussions in~\citep{koolen2010hedging}).
Simulating MWU over the vertices of $\nset$ has been considered in for example~\citep{cesa2012combinatorial}, where they proposed to use the general approach of~\citep{Takimoto03:Path} to implement this algorithm, leading to per-iteration complexity of $\bigOh(d^2 n)$.
Below, we show that our kernelized approach admits an even faster per-iteration complexity of $\bigOh(d\min\{n, d-n\})$.

\subsubsection{Polynomial, $\bigOh(d \min\{n, d-n\})$-time kernel evaluation} Let $\vx, \vy \in \bbR^d$, and assume for now $n \le d-n$. Introduce the polynomial $p_{\vx,\vy}(z)$ of $z$, defined as
\[
    p_{\vx,\vy}(z) \defeq (\vx[1]\vy[1]\, z + 1) \cdots (\vx[d]\vy[d]\, z + 1).
\]
It is immediate to see that the coefficient of $z^n$ in the expansion of $p_{\vx,\vy}(z)$ is exactly $K_{\nset}(\vx, \vy)$. Such coefficient can be computed by directly carrying out the multiplication of the binomial terms, keeping track of the term of degree $0,\dots,n$. So, each evaluation of $K_\nset(\vx,\vy)$ can be carried out in $\bigOh(nd)$ time under the assumption that $n < d-n$.

If on the other hand $n < d-n$, we can repeat the whole argument above for the polynomial
$q_{\vx,\vy}(z) \defeq (z + \vx[1]\vy[1]) \cdots (z + \vx[d]\vy[d])$ instead. In that case, we are interested in the coefficients of $z^{d-n}$, which can be computed in $\bigOh(d(d-n))$ using the same procedure described above.

Putting together the two cases, we conclude that the computation of $K_\nset(\vx,\vy)$ requires $\bigOh(d\min\{n,d-n\})$ time.


\subsubsection{Implementing KOMWU with $\bigOh(d\min\{n, d-n\})$ per-iteration complexity}
The result described in the previous paragraph immediately implies that KOMWU can be implemented with $\bigOh(d^2\min\{n,d-n\})$-time iterations. In this subsection we refine the that result by showing that it is possible to compute the $d$ kernel evaluations $\{K_{\nset}(\vx,\ebar_k) : k =1,\dots,d\}$ required at every iteration by KOMWU so that they take cumulative $\bigOh(d\cdot\min\{n,d-n\})$ time.

To do so, we build on the technique described in the previous subsection. Assume again that $n \le d-n$. The key insight is that the coefficient of $z^n$ of the polynomial $p_{\vx,\vone}(z) / (\vx[j]\, z + 1)$ is exactly $K_{\nset}(\vx, \ebar_j)$. So, to compute all $\{K_{\nset}(\vx, \ebar_k): k =1,\dots,d\}$ we can do the following:
\begin{enumerate}[nosep,left=1mm]
    \item First, for all $k = 0,\dots,d$ and $h=0,\dots,n$, we compute the coefficient $A[k,h]$ of the $z^h$ in the expansion of $(\vx[1]\,z+1)\dots (\vx[k]\,z +1)$

          We can compute all such values in $\bigOh(dn)$ time by using dynamic programming. In particular, we have
          \[
              A[k,h] = \begin{cases}
                  1                                 & \text{if }h=0               \\
                  0                                 & \text{if }k=0 \land h\neq 0 \\
                  A[k-1,h] + \vx[k]\cdot A[k-1,h-1] & \text{otherwise.}
              \end{cases}
          \]

    \item Then, for all $k=1, \dots, d+1$ and $h=0,\dots,n$, we compute the coeffience $B[k,h]$ of $z^h$ in the expansion of $(\vx[k]\,z+1)\cdots (\vx[d]\, z+1)$

          Again, we can do that in $\bigOh(dn)$ time by using dynamic programming. Specifically,
          \[
              B[k,h] = \begin{cases}
                  1                                 & \text{if }h=0                \\
                  0                                 & \text{if }k=d+1\land h\neq 0 \\
                  B[k+1,h] + \vx[k]\cdot B[k+1,h-1] & \text{otherwise.}
              \end{cases}
          \]

    \item (Note that at this point, $K_{\nset}(\vx,\mathbf{1})$ is simply $A[d,n]$.)
    \item For each $k = 1,\dots,d$, $K_{\nset}(x, \ebar_j)$ can be computed as
          \[
              K_{\nset}(\vx,\ebar_k) = \sum_{h=0}^n A[k-1,h]\cdot B[k+1,n-h].
          \]
          The above formula takes $\bigOh(n)$ time to be computed (we need to iterate over $h=0,\dots,n$), and we need to evaluate it $d$ times (once per each $k=1,\dots,d$). So, computing all $\{K_{\nset}(\vx,\ebar_k): k =1,\dots,d\}$ takes cumulative $\bigOh(dn)$ time, as we wanted to show.
\end{enumerate}

As in the previous subsection, the case $n > d-n$ is symmetric. In that case, the set of values $\{K_{\nset}(\vx,\ebar_k): k =1,\dots,d\}$ can be computed in cumulative $\bigOh(d(d-n))$ time.

\subsection{Unit Hypercube}

Consider the hypercube $[0,1]^d$, whose vertices are all the vectors in $\{0,1\}^d$. In this case, the polyhedral kernel is simply
\[
    K_{[0,1]^d}(\vx,\vy) = (\vx[1]\cdot \vy[1] + 1) \cdots (\vx[d]\cdot \vy[d] + 1),
\]
which can be clearly evaluated in $\bigOh(d)$ time. Similarly to $n$-sets (\cref{sec:nsets}), we can avoid paying an extra $d$ factor in the per-iteration complexity of KOMWU by using the following procedure:
\begin{enumerate}
    \item For each $k = 0,\dots,d$ define $A[k] \defeq (\vx[1]\cdot \vy[1] + 1) \cdots (\vx[k] \cdot \vy[k] + 1)$. Clearly, the $A[k]$ values can be computed in $\bigOh(d)$ cumulative time.
    \item For each $k = 1,\dots,d+1$, define $B[k] \defeq (\vx[k]\cdot \vy[k] + 1) \cdots (\vx[d] \cdot \vy[d] + 1)$. Again, all $B[k]$ values can be computed in $\bigOh(d)$ cumulative time.
    \item For each $k=1,\dots,d$, we have that $K_{[0,1]^d}(\vx, \ebar_k) = A[k-1]\cdot B[k+1]$. Hence, we can compute $\{K_{[0,1]^d}(\vx,\ebar_k): k=1,\dots,d\}$ in cumulative $\bigOh(d)$ time.
\end{enumerate}

\subsection{Flows in Directed Acyclic Graphs}

The polytope $\mathcal{F}$ of flows in a generic directed acyclic graphs (DAGs) has vertices with 0/1 integer coordinates, corresponding to paths in the DAG. The 0/1-polyhedral kernel $K_{\mathcal{F}}$ corresponding to the set of flows in a DAG coincides with the kernel function introduced by \citet{Takimoto03:Path}, which was shown to be computable in polynomial-time in the size of the DAG. Consequently, $K_\mathcal{F}$ admits polynomial-time (in the size of the DAG) evaluation.


\subsection{Permutations}

When $\mathcal{P}$ is the convex hull of the set of all $d\times d$ permutation matrices, it is believed that $K_{\mathcal{P}}$ cannot be evaluated in polynomial time in $\bigOh(d)$, since the computation of the permanent of a matrix $\mat{A}$ can be expressed as $K_\Omega(\mat{A}, \vone)$. However, an $\epsilon$-approximate computation of $K_{\mathcal{P}}$ can be performed in $\bigOh(\poly(d,\log(1/\epsilon)))$ for any $\epsilon > 0$ by using a landmark result by \citet{jerrum2004polynomial}. We refer the interested reader to the paper by \citet[Section 5.3]{cesa2012combinatorial}.

\subsection{Cartesian Product}

Finally, we remark that when two 0/1-polyhedral sets have efficiently-computable 0/1-polyhedral kernels, then so does their Cartesian product. Specifically, let $\Omega\subseteq\bbR^d, \Omega'\subseteq\bbR^{d'}$ be 0/1-polyhedral sets, and let $K_\Omega, K_{\Omega'}$ be their corresponding 0/1-polyhedral kernels. Then, it follows immediately from the definition that the polyhedral kernel of $\Omega\times\Omega'$ satisfies
\newcommand{\vstack}[2]{\begin{pmatrix}#1\\#2\end{pmatrix}}
\[
    K_{\Omega\times\Omega'}\mleft(\vstack{\vx}{\vx'}, \vstack{\vy}{\vy'}\mright) = K_\Omega(\vx, \vy) \cdot K_{\Omega'}(\vx',\vy') \qquad\forall\,\vstack{\vx}{\vy}, \vstack{\vx'}{\vy'} \in \bbR^d\times\bbR^{d'}.
\]




\end{document}
