\documentclass[letterpaper]{article}

\usepackage[table]{xcolor}
\usepackage{microtype}
\usepackage{graphicx}
\usepackage{hyperref}
\usepackage{booktabs}
\usepackage[accepted]{icml2022}
\usepackage{amsmath}
\usepackage{amssymb}
\usepackage{amsthm}
\usepackage{thmtools}

\usepackage{bm}
\usepackage{nicefrac}
\usepackage{tikz}
\usepackage{tikz-qtree}
\usepackage{nicerbb}
\usepackage{mathtools}

\usepackage{xparse}
\usepackage{float}
\usepackage{mleftright}

\usepackage{thm-restate}
\usepackage{etoolbox}
\usepackage{xspace}

\usepackage[capitalize,noabbrev]{cleveref}
\usepackage[shortlabels]{enumitem}

\usepackage[ruled,vlined,linesnumbered]{algorithm2e}
\makeatletter
\patchcmd\algocf@Vline{\vrule}{\vrule \kern-0.4pt}{}{}
\patchcmd\algocf@Vsline{\vrule}{\vrule \kern-0.4pt}{}{}
\makeatother

\SetKwComment{Hline}{}{\vspace{-3mm}\textcolor{gray}{\hrule}\vspace{1mm}}
\definecolor{darkgrey}{gray}{0.3}
\definecolor{commentcolor}{gray}{0.5}
\SetKwComment{Comment}{\color{commentcolor}[$\triangleright$\ }{}
\SetCommentSty{}
\SetNlSty{}{\color{darkgrey}}{}
\setlength{\algomargin}{4mm}
\SetKwProg{Fn}{function}{}{}
\SetKwProg{Subr}{subroutine}{}{}
\crefalias{AlgoLine}{line}%
\crefname{algocf}{Algorithm}{Algorithms}

\makeatletter
\let\cref@old@stepcounter\stepcounter
\def\stepcounter#1{%
  \cref@old@stepcounter{#1}%
  \cref@constructprefix{#1}{\cref@result}%
  \@ifundefined{cref@#1@alias}%
    {\def\@tempa{#1}}%
    {\def\@tempa{\csname cref@#1@alias\endcsname}}%
  \protected@edef\cref@currentlabel{%
    [\@tempa][\arabic{#1}][\cref@result]%
    \csname p@#1\endcsname\csname the#1\endcsname}}
\makeatother

%TODO
\newcommand{\todo}[1]{{\color{red}{\bf [TODO]:~{#1}}}}

%THEOREMS
\newtheorem{theorem}{Theorem}
\newtheorem{corollary}{Corollary}
\newtheorem{lemma}{Lemma}
\newtheorem{proposition}{Proposition}
\newtheorem{problem}{Problem}
\newtheorem{definition}{Definition}
\newtheorem{remark}{Remark}
\newtheorem{example}{Example}
\newtheorem{assumption}{Assumption}

%HANS' CONVENIENCES
\newcommand{\define}[1]{\textit{#1}}
\newcommand{\join}{\vee}
\newcommand{\meet}{\wedge}
\newcommand{\bigjoin}{\bigvee}
\newcommand{\bigmeet}{\bigwedge}
\newcommand{\jointimes}{\boxplus}
\newcommand{\meettimes}{\boxplus'}
\newcommand{\bigjoinplus}{\bigjoin}
\newcommand{\bigmeetplus}{\bigmeet}
\newcommand{\joinplus}{\join}
\newcommand{\meetplus}{\meet}
\newcommand{\lattice}[1]{\mathbf{#1}}
\newcommand{\semimod}{\mathcal{S}}
\newcommand{\graph}{\mathcal{G}}
\newcommand{\nodes}{\mathcal{V}}
\newcommand{\agents}{\{1,2,\dots,N\}}
\newcommand{\edges}{\mathcal{E}}
\newcommand{\neighbors}{\mathcal{N}}
\newcommand{\Weights}{\mathcal{A}}
\renewcommand{\leq}{\leqslant}
\renewcommand{\geq}{\geqslant}
\renewcommand{\preceq}{\preccurlyeq}
\renewcommand{\succeq}{\succcurlyeq}
\newcommand{\Rmax}{\mathbb{R}_{\mathrm{max}}}
\newcommand{\Rmin}{\mathbb{R}_{\mathrm{min}}}
\newcommand{\Rext}{\overline{\mathbb{R}}}
\newcommand{\R}{\mathbb{R}}
\newcommand{\N}{\mathbb{N}}
\newcommand{\A}{\mathbf{A}}
\newcommand{\B}{\mathbf{B}}
\newcommand{\x}{\mathbf{x}}
\newcommand{\e}{\mathbf{e}}
\newcommand{\X}{\mathbf{X}}
\newcommand{\W}{\mathbf{W}}
\newcommand{\weights}{\mathcal{W}}
\newcommand{\alternatives}{\mathcal{X}}
\newcommand{\xsol}{\bar{\mathbf{x}}}
\newcommand{\y}{\mathbf{y}}
\newcommand{\Y}{\mathbf{Y}}
\newcommand{\z}{\mathbf{z}}
\newcommand{\Z}{\mathbf{Z}}
\renewcommand{\a}{\mathbf{a}}
\renewcommand{\b}{\mathbf{b}}
\newcommand{\I}{\mathbf{I}}
\DeclareMathOperator{\supp}{supp}
\newcommand{\Par}[2]{\mathcal{P}_{{#1} \to {#2}}}
\newcommand{\Laplacian}{\mathcal{L}}
\newcommand{\F}{\mathcal{F}}
\newcommand{\inv}[1]{{#1}^{\sharp}}
\newcommand{\energy}{Q}
\newcommand{\err}{\mathrm{err}}
\newcommand{\argmin}{\mathrm{argmin}}
\newcommand{\argmax}{\mathrm{argmax}}

\newcommand{\gabri}[1]{\textcolor{red}{[*** Gabri: #1 ***]}}
\newcommand{\ck}[1]{\textcolor{red}{[*** CK: #1 ***]}}
\newcommand{\CL}[1]{\textcolor{violet}{[*** CL: #1 ***]}}
\newcommand{\HL}[1]{\textcolor{red}{[*** HL: #1 ***]}}


\newcommand{\xsubsection}[1]{\refstepcounter{subsection}\textbf{\thesubsection~~#1}~~}

\icmltitlerunning{Kernelized Multiplicative Weights for 0/1-Polyhedral Games}
\begin{document}
\twocolumn[
  \icmltitle{Kernelized Multiplicative Weights for 0/1-Polyhedral Games: Bridging the Gap Between Learning in Extensive-Form and Normal-Form Games}



  \icmlsetsymbol{equal}{*}

  \begin{icmlauthorlist}
    \icmlauthor{Gabriele Farina}{cmu}
    \icmlauthor{Chung-Wei Lee}{usc}
    \icmlauthor{Haipeng Luo}{usc}
    \icmlauthor{Christian Kroer}{col}
  \end{icmlauthorlist}

  \icmlaffiliation{cmu}{Computer Science Department, Carnegie Mellon University}
  \icmlaffiliation{usc}{Computer Science Department, University of Southern California}
  \icmlaffiliation{col}{IEOR Department, Columbia University}

  \icmlcorrespondingauthor{Gabriele Farina}{gfarina@cs.cmu.edu}
  \icmlcorrespondingauthor{Chung-Wei Lee}{leechung@usc.edu}
  \icmlcorrespondingauthor{Haipeng Luo}{haipengl@usc.edu}
  \icmlcorrespondingauthor{Christian Kroer}{christian.kroer@columbia.edu}

  % \icmlkeywords{Machine Learning, ICML}

  \vskip 0.3in
]

\printAffiliationsAndNotice{}

\begin{abstract}
  While extensive-form games (EFGs) can be converted into normal-form games (NFGs), doing so comes at the cost of an exponential blowup of the strategy space. So, progress on NFGs and EFGs has historically followed separate tracks, with the EFG community often having to catch up with advances (\eg last-iterate convergence and predictive regret bounds) from the larger NFG community. In this paper we show that the Optimistic Multiplicative Weights Update (OMWU) algorithm---the premier learning algorithm for NFGs---can be simulated on the normal-form equivalent of an EFG in linear time per iteration in the game tree size using a kernel trick. The resulting algorithm, \emph{Kernelized OMWU (KOMWU)}, applies more broadly to all convex games whose strategy space is a polytope with 0/1 integral vertices, as long as the kernel can be evaluated efficiently. In the particular case of EFGs, KOMWU closes several standing gaps between NFG and EFG learning, by enabling direct, black-box transfer to EFGs of desirable properties of learning dynamics that were so far known to be achievable only in NFGs. Specifically, KOMWU gives the first algorithm that guarantees at the same time last-iterate convergence, lower dependence on the size of the game tree than all prior algorithms, and $\tilde{\bigOh}(1)$ regret when followed by all players.
\end{abstract}

% \leavevmode
% \\
% \\
% \\
% \\
% \\
\section{Introduction}
\label{introduction}

AutoML is the process by which machine learning models are built automatically for a new dataset. Given a dataset, AutoML systems perform a search over valid data transformations and learners, along with hyper-parameter optimization for each learner~\cite{VolcanoML}. Choosing the transformations and learners over which to search is our focus.
A significant number of systems mine from prior runs of pipelines over a set of datasets to choose transformers and learners that are effective with different types of datasets (e.g. \cite{NEURIPS2018_b59a51a3}, \cite{10.14778/3415478.3415542}, \cite{autosklearn}). Thus, they build a database by actually running different pipelines with a diverse set of datasets to estimate the accuracy of potential pipelines. Hence, they can be used to effectively reduce the search space. A new dataset, based on a set of features (meta-features) is then matched to this database to find the most plausible candidates for both learner selection and hyper-parameter tuning. This process of choosing starting points in the search space is called meta-learning for the cold start problem.  

Other meta-learning approaches include mining existing data science code and their associated datasets to learn from human expertise. The AL~\cite{al} system mined existing Kaggle notebooks using dynamic analysis, i.e., actually running the scripts, and showed that such a system has promise.  However, this meta-learning approach does not scale because it is onerous to execute a large number of pipeline scripts on datasets, preprocessing datasets is never trivial, and older scripts cease to run at all as software evolves. It is not surprising that AL therefore performed dynamic analysis on just nine datasets.

Our system, {\sysname}, provides a scalable meta-learning approach to leverage human expertise, using static analysis to mine pipelines from large repositories of scripts. Static analysis has the advantage of scaling to thousands or millions of scripts \cite{graph4code} easily, but lacks the performance data gathered by dynamic analysis. The {\sysname} meta-learning approach guides the learning process by a scalable dataset similarity search, based on dataset embeddings, to find the most similar datasets and the semantics of ML pipelines applied on them.  Many existing systems, such as Auto-Sklearn \cite{autosklearn} and AL \cite{al}, compute a set of meta-features for each dataset. We developed a deep neural network model to generate embeddings at the granularity of a dataset, e.g., a table or CSV file, to capture similarity at the level of an entire dataset rather than relying on a set of meta-features.
 
Because we use static analysis to capture the semantics of the meta-learning process, we have no mechanism to choose the \textbf{best} pipeline from many seen pipelines, unlike the dynamic execution case where one can rely on runtime to choose the best performing pipeline.  Observing that pipelines are basically workflow graphs, we use graph generator neural models to succinctly capture the statically-observed pipelines for a single dataset. In {\sysname}, we formulate learner selection as a graph generation problem to predict optimized pipelines based on pipelines seen in actual notebooks.

%. This formulation enables {\sysname} for effective pruning of the AutoML search space to predict optimized pipelines based on pipelines seen in actual notebooks.}
%We note that increasingly, state-of-the-art performance in AutoML systems is being generated by more complex pipelines such as Directed Acyclic Graphs (DAGs) \cite{piper} rather than the linear pipelines used in earlier systems.  
 
{\sysname} does learner and transformation selection, and hence is a component of an AutoML systems. To evaluate this component, we integrated it into two existing AutoML systems, FLAML \cite{flaml} and Auto-Sklearn \cite{autosklearn}.  
% We evaluate each system with and without {\sysname}.  
We chose FLAML because it does not yet have any meta-learning component for the cold start problem and instead allows user selection of learners and transformers. The authors of FLAML explicitly pointed to the fact that FLAML might benefit from a meta-learning component and pointed to it as a possibility for future work. For FLAML, if mining historical pipelines provides an advantage, we should improve its performance. We also picked Auto-Sklearn as it does have a learner selection component based on meta-features, as described earlier~\cite{autosklearn2}. For Auto-Sklearn, we should at least match performance if our static mining of pipelines can match their extensive database. For context, we also compared {\sysname} with the recent VolcanoML~\cite{VolcanoML}, which provides an efficient decomposition and execution strategy for the AutoML search space. In contrast, {\sysname} prunes the search space using our meta-learning model to perform hyperparameter optimization only for the most promising candidates. 

The contributions of this paper are the following:
\begin{itemize}
    \item Section ~\ref{sec:mining} defines a scalable meta-learning approach based on representation learning of mined ML pipeline semantics and datasets for over 100 datasets and ~11K Python scripts.  
    \newline
    \item Sections~\ref{sec:kgpipGen} formulates AutoML pipeline generation as a graph generation problem. {\sysname} predicts efficiently an optimized ML pipeline for an unseen dataset based on our meta-learning model.  To the best of our knowledge, {\sysname} is the first approach to formulate  AutoML pipeline generation in such a way.
    \newline
    \item Section~\ref{sec:eval} presents a comprehensive evaluation using a large collection of 121 datasets from major AutoML benchmarks and Kaggle. Our experimental results show that {\sysname} outperforms all existing AutoML systems and achieves state-of-the-art results on the majority of these datasets. {\sysname} significantly improves the performance of both FLAML and Auto-Sklearn in classification and regression tasks. We also outperformed AL in 75 out of 77 datasets and VolcanoML in 75  out of 121 datasets, including 44 datasets used only by VolcanoML~\cite{VolcanoML}.  On average, {\sysname} achieves scores that are statistically better than the means of all other systems. 
\end{itemize}


%This approach does not need to apply cleaning or transformation methods to handle different variances among datasets. Moreover, we do not need to deal with complex analysis, such as dynamic code analysis. Thus, our approach proved to be scalable, as discussed in Sections~\ref{sec:mining}.
%!TEX root = hopfwright.tex
%

In this section we systematically recast the Hopf bifurcation problem in Fourier space. 
We introduce appropriate scalings, sequence spaces of Fourier coefficients and convenient operators on these spaces. 
To study Equation~\eqref{eq:FourierSequenceEquation} we consider Fourier sequences $ \{a_k\}$ and fix a Banach space in which these sequences reside. It is indispensable for our analysis that this space have an algebraic structure. 
The Wiener algebra of absolutely summable Fourier series is a natural candidate, which we use with minor modifications. 
In numerical applications, weighted sequence spaces with algebraic and geometric decay have been used to great effect to study periodic solutions which are $C^k$ and analytic, respectively~\cite{lessard2010recent,hungria2016rigorous}. 
Although it follows from Lemma~\ref{l:analytic} that the Fourier coefficients of any solution decay exponentially, we choose to work in a space of less regularity. 
The reason is that by working in a space with less regularity, we are better able to connect our results with the global estimates in \cite{neumaier2014global}, see Theorem~\ref{thm:UniqunessNbd2}.


%
%
%\begin{remark}
%	Although it follows from Lemma~\ref{l:analytic} that the Fourier coefficients of any solution decay exponentially, we choose to work in a space of less regularity, namely summable Fourier coefficients. This allows us to draw SOME MORE INTERESTING CONCLUSION LATER.
%	EXPLAIN WHY WE CHOOSE A NORM WITH ALMOST NO DECAY!
%	% of s Periodic solutions to Wright's equation are known to be real analytic and so their  Fourier coefficients must decay geometrically [Nussbaum].
%	% We do not use such a strong result;  any periodic solution must be continuously differentiable, by which it follows that $ \sum | c_k| < \infty$.
%\end{remark}


\begin{remark}\label{r:a0}
There is considerable redundancy in Equation~\eqref{eq:FourierSequenceEquation}. First, since we are considering real-valued solutions $y$, we assume $\c_{-k}$ is the complex conjugate of $\c_k$. This symmetry implies it suffices to consider Equation~\eqref{eq:FourierSequenceEquation} for $k \geq 0$.
Second, we may effectively ignore the zeroth Fourier coefficient of any periodic solution \cite{jones1962existence}, since it is necessarily equal to $0$. 
%In \cite{jones1962existence}, it is shown that if $y \not\equiv -1$ is a periodic solution of~\eqref{eq:Wright} with frequency $\omega$, then $ \int_0^{2\pi/\omega} y(t) dt =0$. 
		The self contained argument is as follows. 
		As mentioned in the introduction, any periodic solution to Wright's equation must satisfy $ y(t) > -1$ for all $t$. 
	By dividing Equation~\eqref{eq:Wright} by $(1+y(t))$, which never vanishes, we obtain
	\[
	\frac{d}{dt} \log (1 + y(t)) = - \alpha y(t-1).
	\]  
	Integrating over one period $L$ we derive the condition 
	$0=\int_0^L y(t) dt $.
	Hence $a_0=0$ for any periodic solution. 
	It will be shown in Theorem~\ref{thm:FourierEquivalence1} that a related argument implies that we do not need to consider Equation~\eqref{eq:FourierSequenceEquation} for $k=0$.
\end{remark}

%%%
%%%
%%%\begin{remark}\label{r:c0} 
%%%In \cite{jones1962existence}, it is shown that if $y \not\equiv -1$ is a periodic solution of~\eqref{eq:Wright} with frequency $\omega$, then $ \int_0^{2\pi/\omega} y(t) dt =0$. 
%%%PERHAPS TOO MUCH DETAIL HERE. The self contained argument is as follows.
%%%If $y \not\equiv -1$ then $y(t) \neq -1$ for all $t$, since if $y(t_0)=-1$ for some $t_0 \in \R$ then $y'(t_0)=0$ by~\eqref{eq:Wright} and in fact by differentiating~\eqref{eq:Wright} repeatedly one obtains that all derivatives of $y$ vanish at $t_0$. Hence $y \equiv -1$ by Lemma~\ref{l:analytic}, a contradiction. Now divide~\eqref{eq:Wright} by $(1+y(t))$, which never vanishes, to obtain
%%%\[
%%%  \frac{d}{dt} \log |1 + y(t)| = - \alpha y(t-1).
%%%\]  
%%%Integrating over one period we obtain $\int_0^L y(t) dt =0$.
%%%\end{remark}



%Furthermore, the condition that $y(t)$ is real forces $\c_{-k} = \overline{\c}_{k}$.  
%
We define the spaces of absolutely summable Fourier series
\begin{alignat*}{1}
	\ell^1 &:= \left\{ \{ \c_k \}_{k \geq 1} : 
    \sum_{k \geq 1} | \c_k| < \infty  \right\} , \\
	\ell^1_\bi &:= \left\{ \{ \c_k \}_{k \in \Z} : 
    \sum_{k \in \Z} | \c_k| < \infty  \right\} .
\end{alignat*} 
We identify any semi-infinite sequence $ \{ \c_k \}_{k \geq 1} \in \ell^1$ with the bi-infinite sequence $ \{ \c_k \}_{k \in \Z} \in \ell^1_\bi$ via the conventions (see Remark~\ref{r:a0})
\begin{equation}
  \c_0=0 \qquad\text{ and }\qquad \c_{-k} = \c_{k}^*. 
\end{equation}
In other word, we identify $\ell^1$ with the set
\begin{equation*}
   \ell^1_\sym := \left\{ \c \in \ell^1_\bi : 
	\c_0=0,~\c_{-k}=\c_k^* \right\} .
\end{equation*}
On $\ell^1$ we introduce the norm
\begin{equation}\label{e:lnorm}
  \| \c \| = \| \c \|_{\ell^1} := 2 \sum_{k = 1}^\infty |\c_k|.
\end{equation}
The factor $2$ in this norm is chosen to have a Banach algebra estimate.
Indeed, for $\c, \tilde{\c} \in \ell^1 \cong \ell^1_\sym$ we define
the discrete convolution 
\[
\left[ \c * \tilde{\c} \right]_k = \sum_{\substack{k_1,k_2\in\Z\\ k_1 + k_2 = k}} \c_{k_1} \tilde{\c}_{k_2} .
\]
Although $[\c*\tilde{\c}]_0$ does not necessarily vanish, we have $\{\c*\tilde{\c}\}_{k \geq 1} \in \ell^1 $ and 
\begin{equation*}
	\| \c*\tilde{\c} \| \leq \| \c \| \cdot  \| \tilde{\c} \| 
	\qquad\text{for all } \c , \tilde{\c} \in \ell^1, 
\end{equation*}
hence $\ell^1$ with norm~\eqref{e:lnorm} is a Banach algebra.

By Lemma~\ref{l:analytic} it is clear that any periodic solution of~\eqref{eq:Wright} has a well-defined Fourier series $\c \in \ell^1_\bi$. 
The next theorem shows that in order to study periodic orbits to Wright's equation we only need to study Equation~\eqref{eq:FourierSequenceEquation} 
for $k \geq 1$. For convenience we introduce the notation 
\[
G(\alpha,\omega,\c)_k=
( i \omega k + \alpha e^{ - i \omega k}) \c_k + \alpha \sum_{k_1 + k_2 = k} e^{- i \omega k_1} \c_{k_1} \c_{k_2} \qquad \text{for } k \in \N.
\]
We note that we may interpret the trivial solution $y(t)\equiv 0$ as a periodic solution of arbitrary period.
\begin{theorem}
\label{thm:FourierEquivalence1}
Let $\alpha>0$ and $\omega>0$.
If $\c \in \ell^1 \cong \ell^1_{\sym}$ solves
$G(\alpha,\omega,\c)_k =0$  for all $k \geq 1$,
then $y(t)$ given by~\eqref{eq:FourierEquation} is a periodic solution of~\eqref{eq:Wright} with period~$2\pi/\omega$.
Vice versa, if $y(t)$ is a periodic solution of~\eqref{eq:Wright} with period~$2\pi/\omega$ then its Fourier coefficients $\c \in \ell^1_\bi$ lie in $\ell^1_\sym \cong \ell^1$ and solve $G(\alpha,\omega,\c)_k =0$ for all $k \geq 1$.
\end{theorem}

\begin{proof}	
	If $y(t)$ is a periodic solution of~\eqref{eq:Wright} then it is real analytic by Lemma~\ref{l:analytic}, hence its Fourier series $\c$ is well-defined and $\c \in \ell^1_{\sym}$ by Remark~\ref{r:a0}.
	Plugging the Fourier series~\eqref{eq:FourierEquation} into~\eqref{eq:Wright} one easily derives that $\c$ solves~\eqref{eq:FourierSequenceEquation} for all $k \geq 1$.

To prove the reverse implication, assume that $\c \in \ell^1_\sym$ solves
Equation~\eqref{eq:FourierSequenceEquation} for all $k \geq 1$. Since $\c_{-k}
= \c_k^*$, Equation \eqref{eq:FourierSequenceEquation} is also satisfied for
all $k \leq -1$. It follows from the Banach algebra property and
\eqref{eq:FourierSequenceEquation} that $\{k \c_k\}_{k \in \Z} \in \ell^1_\bi$,
hence $y$, given by~\eqref{eq:FourierEquation}, is continuously differentiable.
% (and by bootstrapping one infers that $\{k^m c_k \} \in \ell^1_\bi$, 
% hence $y \in C^m$ for any $m \geq 1$).
	Since~\eqref{eq:FourierSequenceEquation} is satisfied for all $k \in \Z \setminus \{0\}$ (but not necessarily for $k=0$) one may perform the inverse Fourier transform on~\eqref{eq:FourierSequenceEquation} to conclude that
	$y$ satisfies the delay equation 
\begin{equation}\label{eq:delaywithK}
   	y'(t) = - \alpha y(t-1) [ 1 + y(t)] + C
\end{equation}
	for some constant $C \in \R$. 
   Finally, to prove that $C=0$ we argue by contradiction.
   Suppose $C \neq 0$. Then $y(t) \neq -1$ for all $t$.
   Namely, at any point where $y(t_0) =-1$ one would have $y'(t_0) = C$
   which has fixed sign,   hence it would follow that $y$ is not periodic
   ($y$ would not be able to cross $-1$ in the opposite direction, 
   preventing $y$  from being periodic).  
  We may thus divide~\eqref{eq:delaywithK} through by $1 + y(t)$ and obtain 
\begin{equation*}
	\frac{d}{dt} \log | 1 + y(t) | = - \alpha y(t-1) + \frac{C}{1+y(t)} .
\end{equation*}
	By integrating both sides of the equation over one period $L$ and by using that $\c_0=0$, we 
	obtain
	\[
	 C \int_0^L \frac{1}{1+y(t)} dt =0.
	\]
	Since the integrand is either strictly negative or strictly positive, this implies that $C=0$. Hence~\eqref{eq:delaywithK} reduces to~\eqref{eq:Wright},
	and $y$ satisfies Wright's equation. 
\end{proof}






To efficiently study Equation~\eqref{eq:FourierSequenceEquation}, we introduce the following linear operators on $ \ell^1$:
\begin{alignat*}{1}
   [K \c ]_k &:= k^{-1} \c_k  , \\ 
   [ U_\omega \c ]_k &:= e^{-i k \omega} \c_k  .
\end{alignat*}
The map $K$ is a compact operator, and it has a densely defined inverse $K^{-1}$. The domain of $K^{-1}$ is denoted by
\[
  \ell^K := \{ \c \in \ell^1 : K^{-1} \c \in \ell^1 \}.  
\]
The map $U_{\omega}$ is a unitary operator on $\ell^1$, but
it is discontinuous in $\omega$. 
With this notation, Theorem~\ref{thm:FourierEquivalence1} implies that our problem of finding a SOPS to~\eqref{eq:Wright} is equivalent to finding an $\c \in \ell^1$ such that
\begin{equation}
\label{e:defG}
  G(\alpha,\omega,\c) :=
  \left( i \omega K^{-1} + \alpha U_\omega \right) \c + \alpha \left[U_\omega \, \c \right] * \c  = 0.
\end{equation}


%In order for the solutions of Equation \ref{eq:FHat} to be isolated we need to impose a phase condition. 
%If there is a sequence $ \{ c_k \} $ which satisfies  Equation \ref{eq:FHat}, then $ y( t + \tau) = \sum_{k \in \Z} c_k e^{ i k \omega (t + \tau)}$ satisfies Wright's equation at parameter $\alpha$. 
%Fix $ \tau = - Arg[c_1] / \omega$ so that $ c_1  e^{ i \omega \tau} $ is a nonnegative real number. 
%By Proposition \ref{thm:FourierEquivalence1} it follows that $\{ c'_k \} =  \{c_k e^{ i \omega k \tau }   \}$ is a solution to Equation \ref{eq:FHat}, and furthermore that $ c'_1 = \epsilon$ for some $ \epsilon \geq 0$. 


Periodic solutions are invariant under time translation: if $y(t)$ solves Wright's equation, then so does $ y(t+\tau)$ for any $\tau \in \R$. 
We remove this degeneracy by adding a phase condition. 
Without loss of generality, if $\c \in \ell^1$ solves Equation~\eqref{e:defG}, we may assume that $\c_1 = \epsilon$ for some 
\emph{real non-negative}~$\epsilon$:
\[
  \ell^1_{\epsilon} := \{\c \in \ell^1 : \c_1 = \epsilon \} 
  \qquad \text{where } \epsilon \in \R,  \epsilon \geq 0.
\]
In the rest of our analysis, we will split elements $\c \in \ell^1$ into two parts: $\c_1$ and $\{\c_{k}\}_{k \geq 2}$.  
We define the basis elements $\e_j \in \ell^1$ for $j=1,2,\dots$ as
\[
  [\e_j]_k = \begin{cases}
  1 & \text{if } k=j, \\
  0 & \text{if } k \neq j.
  \end{cases}
\]
We note that $\| \e_j \|=2$. 
Then we can decompose
% We define
% \[
%   \tilde{\epsilon} := (\epsilon,0,0,0,\dots) \in \ell^1
% \]
% and
% For clarity when referring to sequences $\{c_{k}\}_{k \geq 2}$, we make the following definition:
% \[
% \ell^1_0  := \{ \tc \in \ell^1 : \tc_1 = 0 \}.
% \]
% With the
any $\c \in \ell^1_\epsilon$ uniquely as
\begin{equation}\label{e:aepsc}
  \c= \epsilon \e_1 + \tc \qquad \text{with}\quad 
  \tc \in \ell^1_0 := \{ \tc \in \ell^1 : \tc_1 = 0 \}.
\end{equation}
We follow the classical approach in studying Hopf bifurcations and consider 
$\c_1 = \epsilon$ to be a parameter, and then find periodic solutions with Fourier modes in $\ell^1_{\epsilon}$.
This approach rewrites the function $G: \R^2 \times \ell^K \to \ell^1$ as a function $\tilde{F}_\epsilon : \R^2 \times \ell^K_0 \to \ell^1$, where 
we denote 
\[
\ell^K_0 := \ell^1_0 \cap \ell^K.
\]
% I AM ACTUALLY NOT SURE IF YOU WANT TO DEFINE THIS WITH RANGE IN $\ell^1$
% OR WITH DOMAIN IN $\ell^1_0$ ?? IT SEEMS TO DEPEND ON WHICH GLOBAL STATEMENT YOU WANT/NEED TO MAKE!?
\begin{definition}
We define the $\epsilon$-parameterized family of  functions $\tilde{F}_\epsilon: \R^2 \times \ell^K_0  \to \ell^1$ 
by 
\begin{equation}
\label{eq:fourieroperators}
\tilde{F}_{\epsilon}(\alpha,\omega, \tc) := 
\epsilon [i \omega + \alpha e^{-i \omega}] \e_1 + 
( i \omega K^{-1} + \alpha U_{\omega}) \tc + 
\epsilon^2 \alpha e^{-i \omega}  \e_2  +
\alpha \epsilon L_\omega \tc + 
\alpha  [ U_{\omega} \tc] * \tc ,
\end{equation}
where
$L_\omega : \ell^1_0 \to \ell^1$ is given by
\[
   L_{\omega} := \sigma^+( e^{- i \omega} I + U_{\omega}) + \sigma^-(e^{i \omega} I + U_{\omega}),
\]
with $I$ the identity and  $\sigma^\pm$ the shift operators on $\ell^1$:
\begin{alignat*}{2}
\left[ \sigma^- a \right]_k &:=  a_{k+1}  , \\
\left[ \sigma^+ a \right]_k &:=  a_{k-1}  &\qquad&\text{with the convention } \c_0=0.
\end{alignat*}
The operator $ L_\omega$ is discontinuous in $\omega$ and $ \| L_\omega \| \leq 4$. 
\end{definition} 

%The maps $ \sigma^{+}$ and $ \sigma^-$ are shift up and shift down operators respectively. 
We reformulate Theorem~\ref{thm:FourierEquivalence1}  in terms of the map  $\tilde{F}$. 
We note that it follows from Lemma~\ref{l:analytic} and 
%\marginpar{Reformulate}
%one's choice of  
Equation~\eqref{eq:FourierSequenceEquation}  
%or Equation ~\eqref{eq:fourieroperators},
that the Fourier coefficients of any periodic solution of~\eqref{eq:Wright} lie in $\ell^K$.
These observations are summarized in the following theorem.
\begin{theorem}
\label{thm:FourierEquivalence2}
	Let $ \epsilon \geq 0$,  $\tc \in \ell^K_0$, $\alpha>0$ and $ \omega >0$. 
	Define $y: \R\to \R$ as 
\begin{equation}\label{e:ytc}
	y(t) = 
	\epsilon \left( e^{i \omega t }  + e^{- i \omega t }\right) 
	+  \sum_{k = 2}^\infty   \tc_k e^{i \omega k t }  + \tc_k^* e^{- i \omega k t } .
\end{equation}
%	and suppose that $ y(t) > -1$. 
	Then $y(t)$ solves~\eqref{eq:Wright} if and only if $\tilde{F}_{\epsilon}( \alpha , \omega , \tc) = 0$. 
	Furthermore, up to time translation, any periodic solution of~\eqref{eq:Wright} with period $2\pi/\omega$ is described by a Fourier series of the form~\eqref{e:ytc} with $\epsilon \geq 0$ and $\tc \in \ell^K_0$.
\end{theorem}


%We note that for $\epsilon>0$ such solutions are truly periodic, while for $\epsilon=0$ a zero of $\tilde{F}_\epsilon$ may either correspond to a periodic solution or to the trivial solution $y(t) \equiv 0$. 



% \begin{proof}
%  By Proposition \ref{thm:FourierEquivalence1}, it suffices to show that $\tilde{F}(\alpha,\omega,c) =0$ is equivalent to Equation \ref{eq:FourierSequenceEquation} being satisfied for $k \geq 1$.
%  Since Equation \ref{eq:FourierSequenceEquation} is equivalent to Equation \ref{eq:FHat}, we expand  Equation \ref{eq:FHat} by writing $ \hat{c} = \hat{\epsilon } + c$  where $ \hat{\epsilon} := (\epsilon,0,0,\dots) \in \ell^1$ as below:
%  \begin{equation}
%  0=  \left( i \omega K^{-1} + \alpha U_\omega \right) (\hat{\epsilon}+ c) + \alpha \left[U_\omega \, (\hat{\epsilon}+ c) \right] * (\hat{\epsilon}+ c) \label{eq:Intial}
%  \end{equation}
%  The RHS of Equation \ref{eq:Intial} is $ \tilde{F}(\alpha,\omega,c)$, so the theorem is proved.
% \end{proof}



Since we want to analyze a Hopf bifurcation, we will want to solve $\tilde{F}_\epsilon = 0$ for small values of~$\epsilon$. 
However, at the bifurcation point, $ D \tilde{F}_0(\pp  ,\pp , 0)$ is not invertible.
In order for our asymptotic analysis to be non-degenerate,
we work with a rescaled version of the problem. To this end, for any $\epsilon >0$, we rescale both $\tc$ and $\tilde{F}$ as follows. Let $\tc = \epsilon c$ and 
\begin{equation}\label{e:changeofvariables}
  \tilde{F}_\epsilon (\alpha,\omega,\epsilon c) = \epsilon F_\epsilon (\alpha,\omega,c).
\end{equation}
For $\epsilon>0$ the problem then reduces to finding zeros of 
\begin{equation}
\label{eq:FDefinition}
	F_\epsilon(\alpha,\omega, c) := 
	[i \omega + \alpha e^{-i \omega}] \e_1 + 
	( i \omega K^{-1} + \alpha U_{\omega}) c + 
	\epsilon \alpha e^{-i \omega} \e_2  +
	\alpha \epsilon L_\omega c + 
	\alpha \epsilon [ U_{\omega} c] * c.
\end{equation}
We denote the triple $(\alpha,\omega,c) \in \R^2 \times \ell^1_0$ by $x$.
To pinpoint the components of $x$ we use the projection operators
\[
   \pi_\alpha x = \alpha, \quad \pi_\omega x = \omega, \quad 
  \pi_c x = c \qquad\text{for any } x=(\alpha,\omega,c).
\]

After the change of variables~\eqref{e:changeofvariables} we now have an invertible Jacobian $D F_0(\pp  ,\pp , 0)$ at the bifurcation point.
On the other hand, for $\epsilon=0$ the zero finding problems for $\tilde{F}_\epsilon$ and $F_\epsilon$ are not equivalent. 
However, it follows from the following lemma that any nontrivial periodic solution having $ \epsilon=0$ must have a relatively large size when $ \alpha $ and $ \omega $ are close to the bifurcation point. 

\begin{lemma}\label{lem:Cone}
	Fix $ \epsilon \geq 0$ and $\alpha,\omega >0$. 
	Let
	\[
	b_* :=  \frac{\omega}{\alpha} - \frac{1}{2} - \epsilon  \left(\frac{2}{3}+ \frac{1}{2}\sqrt{2 + 2 |\omega-\pp| } \right).
	\]
Assume that $b_*> \sqrt{2} \epsilon$. 
Define
% \begin{equation*}%\label{e:zstar}
% 	z^{\pm}_* :=b_* \pm \sqrt{(b_*)^2- \epsilon^2 } .
% \end{equation*}
% \note[J]{Proposed change to match Lemma E.4}
\begin{equation}\label{e:zstar}
z^{\pm}_* :=b_* \pm \sqrt{(b_*)^2- 2 \epsilon^2 } .
\end{equation}
If there exists a $\tc \in \ell^1_0$ such that $\tilde{F}_\epsilon(\alpha, \omega,\tc) = 0$, then \\
\mbox{}\quad\textup{(a)} either $ \|\tc\| \leq  z_*^-$ or $ \|\tc\| \geq z_*^+  $.\\
\mbox{}\quad\textup{(b)} 
$ \| K^{-1} \tc \| \leq (2\epsilon^2+ \|\tc\|^2) / b_*$. 
\end{lemma}
\begin{proof}
	The proof follows from Lemmas~\ref{lem:gamma} and~\ref{lem:thecone} in Appendix~\ref{appendix:aprioribounds}, combined with the observation that
$\frac{\omega}{\alpha} - \gamma \geq b_*$,
% \[
%   \frac{\omega}{\alpha} - \gamma \geq b_*
%  \qquad\text{for all }
% | \alpha - \pp| \leq r_\alpha \text{ and } 
%   | \omega - \pp| \leq r_\omega.
% \]
with $\gamma$ as defined in Lemma~\ref{lem:gamma}.
\end{proof}

\begin{remark}\label{r:smalleps}
We note that for $\alpha < 2\omega$
\begin{alignat*}{1}
z^+_* &\geq   \frac{2 \omega - \alpha}{\alpha} 
- \epsilon \left(4/3+\sqrt{2 + 2 |\omega-\pp| } \, \right) + \cO(\epsilon^2)
\\[1mm]
z^-_* & \leq   \cO(\epsilon^2)
\end{alignat*}
for small $\epsilon$. 
Hence Lemma~\ref{lem:Cone} implies that for values of $(\alpha,\omega)$ near $(\pp,\pp)$ any solution has either $\|\tc\|$ of order 1 or $\|\tc\| =  \cO(\epsilon^2)$. 
The asymptotically small term bounding $z_*^-$ is explicitly calculated in Lemma~\ref{lem:ZminusBound}. 
A related consequence is that for $\epsilon=0$ there are no nontrivial solutions 
of $\tilde{F}_0(\alpha,\omega,\tc)=0$ with 
$\| \tc \| < \frac{2 \omega - \alpha}{\alpha} $. 
\end{remark}

\begin{remark}\label{r:rhobound}
In Section~\ref{s:contraction} we will work on subsets of $\ell^K_0$ of the form
\[
  \ell_\rho := \{ c \in \ell^K_0 : \|K^{-1} c\| \leq \rho \} .
\]
Part (b) of Lemma~\ref{lem:Cone} will be used in Section~\ref{s:global} to guarantee that we are not missing any solutions by considering $\ell_\rho$ (for some specific choice of $\rho$) rather than the full space $\ell^K_0$.
In particular, we infer from Remark~\ref{r:smalleps} that  small solutions (meaning roughly that $\|\tc\| \to 0$ as $\epsilon \to 0$)
satisfy $\| K^{-1} \tc \| = \cO(\epsilon^2)$.
\end{remark}

The following theorem guarantees that near the bifurcation point the problem of finding all periodic solutions is equivalent to considering the rescaled problem $F_\epsilon(\alpha,\omega,c)=0$.
\begin{theorem}
\label{thm:FourierEquivalence3}
\textup{(a)} Let $ \epsilon > 0$,  $c \in \ell^K_0$, $\alpha>0$ and $ \omega >0$. 
	Define $y: \R\to \R$ as 
\begin{equation}\label{e:yc}
	y(t) = 
	\epsilon \left( e^{i \omega t }  + e^{- i \omega t }\right) 
	+ \epsilon  \sum_{k = 2}^\infty   c_k e^{i \omega k t }  + c_k^* e^{- i \omega k t } .
\end{equation}
%	and suppose that $ y(t) > -1$. 
	Then $y(t)$ solves~\eqref{eq:Wright} if and only if $F_{\epsilon}( \alpha , \omega , c) = 0$.\\
\textup{(b)}
Let $y(t) \not\equiv 0$ be a periodic solution of~\eqref{eq:Wright} of period $2\pi/\omega$
 with Fourier coefficients $\c$.
Suppose $\alpha < 2\omega$ and $\| \c \| < \frac{2 \omega - \alpha}{\alpha} $.
Then, up to time translation, $y(t)$ is described by a Fourier series of the form~\eqref{e:yc} with $\epsilon > 0$ and $c \in \ell^K_0$.
\end{theorem}

\begin{proof}
Part (a) follows directly from Theorem~\ref{thm:FourierEquivalence2} and the  change of variables~\eqref{e:changeofvariables}.
To prove part (b) we need to exclude the possibility that there is a nontrivial solution with $\epsilon=0$. The asserted bound on the ratio of $\alpha$ and $\omega$ guarantees, by Lemma~\ref{lem:Cone} (see also Remark~\ref{r:smalleps}), that indeed $\epsilon>0$ for any nontrivial solution. 
\end{proof}

We note that in practice (see Section~\ref{s:global}) a bound on $\| \c \|$ is derived from a bound on $y$ or $y'$ using Parseval's identity.

\begin{remark}\label{r:cone}
It follows from Theorem~\ref{thm:FourierEquivalence3} and Remark~\ref{r:smalleps} that for values of $(\alpha,\omega)$ near $(\pp,\pp)$ any reasonably bounded solution satisfies $\| c\| =  O(\epsilon)$ as well as $\|K^{-1} c \| = O(\epsilon)$ asymptotically (as $\epsilon \to 0$).
These bounds will be made explicit (and non-asymptotic) for specific choices of the parameters in Section~\ref{s:global}.
\end{remark}

% We are able to rule out such large amplitude solutions using global estimates such as those in \cite{neumaier2014global}.
% Hence, near the bifurcation point, the problem of describing periodic solutions of~\eqref{eq:Wright} reduces to studying the family of zeros finding problems $F_\epsilon=0$.





%Specifically, if a solution having $ \epsilon = 0$ does in fact correspond to a nontrivial periodic solution and $\alpha  < 2\omega $, then $ \| \tilde{c} \| > 2 \omega \alpha^{-1} -1$. 
%%PERHAPS THIS NEEDS A FORMULATION AS A THEOREM AS WELL?
%%IN OTHER WORDS: ARE WE SURE WE HAVE FOUND ALL ZEROS OF $\tilde{F}_0$, I.E. ALL SOLUTIONS WITH $\epsilon=0$ NEAR THE BIFURCATION POINT? AFTER RESCALING THESE ARE INVISIBLE?
%%THERE SHOULD BE A STATEMENT ABOUT THIS SOMEWHERE! EITHER HERE OR SOME





We finish this section by defining a curve of approximate zeros $\bx_\epsilon$ of $F_\epsilon$ 
(see \cite{chow1977integral,hassard1981theory}). 
%(see \cite{chow1977integral,morris1976perturbative,hassard1981theory}). 


\begin{definition}\label{def:xepsilon}
Let
\begin{alignat*}{1}
	\balpha_\epsilon &:= \pp + \tfrac{\epsilon^2}{5} ( \tfrac{3\pi}{2} -1)  \\
	\bomega_\epsilon &:= \pp -  \tfrac{\epsilon^2}{5} \\
	\bc_\epsilon 	 &:= \left(\tfrac{2 - i}{5}\right) \epsilon \,  \e_2 \,.
\end{alignat*}
We define the approximate solution 
$ \bx_\epsilon := \left( \balpha_\epsilon , \bomega_\epsilon  , \bc_\epsilon \right)$
for all $\epsilon \geq 0$.
\end{definition}

We leave it to the reader to verify that both 
 $F_\epsilon(\pp,\pp,\bc_{\epsilon})=\cO(\epsilon^2)$ and $F_\epsilon(\bx_\epsilon)=\cO(\epsilon^2)$.
%%%	
%%%	
%%%	}{Better like this?}
%%%\annote[J]{ $F_\epsilon(\bx_0)=\cO(\epsilon^2)$ and $F_\epsilon(\bx_\epsilon)=\cO(\epsilon^2)$.}{I think we'd still need the $ \bar{c}_\epsilon$ term in $\bar{x}_0$ to be of order $ \epsilon$.}
%%%\remove[JB]{We show in Proposition A.1
%%%%\ref{prop:ApproximateSolutionWorks} 
%%% that any $ x \in \R^2 \times \ell^1_0$ which is $ \cO(\epsilon^2)$ close to $ \bar{x}_\epsilon $ will yield the estimate $F_\epsilon(x) = \cO(\epsilon^2)$.
%%%Hence choosing $\{ \pp , \pp, \bar{c}_\epsilon\}$ as our approximate solution would also have been a natural choice for performing an $\cO(\epsilon^2)$ analysis and would have simplified several of our calculations.
%%%However,} 
%%%
We choose to use the more accurate approximation 
for the $ \alpha$ and $ \omega $ components to improve our final quantitative results. 














%
% Values for $ (\alpha, \omega,c)$ which approximately solve $\tilde{F}(\alpha,\omega,c) = 0$  are computed in  \cite{chow1977integral,morris1976perturbative,hassard1981theory} and are as follows:
%  \begin{eqnarray}
%  \tilde{\alpha}( \epsilon) &:=& \pi /2 + \tfrac{\epsilon^2}{5} ( \tfrac{3\pi}{2} -1) \nonumber \\
%  \tilde{\omega}( \epsilon) &:=& \pi /2 -  \tfrac{\epsilon^2}{5} \label{eq:ScaleApprox} \\
%  \tc(\epsilon) 	  &:=& \{ \left(\tfrac{2 - i}{5}\right)  \epsilon^2 , 0,0, \dots \} \nonumber
%  \end{eqnarray}
% In Appendix \ref{sec:OperatorNorms} we illustrate an alternative method for deriving this approximation.
%
%
%
%
% We want to solve $ \tilde{F}(\alpha , \omega, \hat{c}) =0$ for small values of $ \epsilon$.
% However $ D \tilde{F}(\alpha , \omega , c)$ is not invertible at $ ( \pp , \pp , 0)$ when $ \epsilon = 0$.
% In order for our asymptotic analysis to be non-degenerate, we need to make the change of variables $ c \mapsto \epsilon c$.
% Under this change of variables, we define the function $ F$ below so that $ \tilde{F}(\alpha , \omega , \epsilon c) =\epsilon  F( \alpha , \omega , c)$.
%
%
%
% \begin{definition}
% Construct an $\epsilon$-parameterized family of densely defined functions  $F : \R^2 \oplus \ell^1 / \C \to \ell^1$ by:
% \begin{equation}
% \label{eq:FDefinition}
% 	F(\alpha,\omega, c) :=
% 	[i \omega + \alpha e^{-i \omega}]_1 +
% 	( i \omega K^{-1} + \alpha U_{\omega}) c +
% 	[\epsilon \alpha e^{-i \omega}]_2  +
% 	\alpha \epsilon L_\omega c +
% 	\alpha \epsilon [ U_{\omega} c] * c.
% \end{equation}
% \end{definition}

%%
%%
%%\begin{corollary}
%%	\label{thm:FourierEquivalence3}
%%	Fix $ \epsilon > 0$, and $ c \in \ell^1 / \C $, and $ \omega >0$. Define $y: \R\to \R$ as 
%%	\[
%%	y(t) = 
%%	\epsilon \left( e^{i \omega t }  + e^{- i \omega t }\right) 
%%	+  \epsilon  \left( \sum_{k = 2}^\infty   c_k e^{i \omega k t }  + \overline{c}_k e^{- i \omega k t } \right) 
%%	\]
%%	and suppose that $ y(t) > -1$. 
%%	Then $y(t)$ solves Wright's equation at parameter $ \alpha > 0 $ if and only if $ F( \alpha , \omega , c) = 0$ at parameter $ \epsilon$. 
%%	
%%	
%%	
%%\end{corollary}
%%
%%
%%\begin{proof}
%%	Since $ \tilde{F}(\alpha,\omega, \epsilon c) = \epsilon F( \alpha , \omega , c)$, the result follows from Theorem \ref{thm:FourierEquivalence2}.
%%\end{proof}

% If we can find $(\alpha , \omega, c)$ for which $ F( \alpha , \omega,c)=0$ at parameter $\epsilon$, then $ \tilde{F}(\alpha ,\omega, c)=0$.
% By Theorem \ref{thm:FourierEquivalence2} this amounts to finding a periodic solution to Wright's equation.
% Lastly, because we have performed the change of variables $ c \mapsto \epsilon c$, we need to  apply this change of variables to our approximate solution as well.
%
% \begin{definition}
% 	Define the approximate solution $ x( \epsilon) = \left\{ \alpha(\epsilon ) , \omega ( \epsilon ) , c(\epsilon) \right\}$ as below,  where $c(\epsilon) = \{ c_2( \epsilon) , 0 ,0 , \dots\} $.
% 	We may also write $ x_\epsilon = x(\epsilon) $.
% 	\begin{eqnarray}
% 	\alpha( \epsilon) &:=& \pi /2 + \tfrac{\epsilon^2}{5} ( \tfrac{3\pi}{2} -1) \nonumber \\
% 	\omega( \epsilon) &:=& \pi /2 -  \tfrac{\epsilon^2}{5} \label{eq:Approx} \\
% 	c_2(\epsilon) 	  &:=& \left(\tfrac{2 - i}{5}\right) \epsilon \nonumber
% 	\end{eqnarray}
%
% \end{definition}

\section{Multiplicative Weights in Polyhedral Convex Games}
\label{sec:vertex}

A powerful generalization of normal-form games is \emph{polyhedral convex games}, of which extensive-form games are an example~\citep{Gordon08:No}. Unlike NFGs, in which players select a mixed strategy from the probability simplex spanned by the set of available action $\cA_i$, in a polyhedral convex game the set of  ``randomized strategies'' from which each player $i\in\range{m}$ can draw is a given convex polytope $\Omega_i \subseteq \bbR^{d_i}$. Analogously to NFGs, we represent a polyhedral convex game as a tuple $\Gamma = (m, \{\Omega_i\}, \{\bar U_i\})$, where the functions $\bar U_i : \Omega_1\times\dots\times\Omega_m \to [0,1]$ are the \emph{multilinear} utility functions for each player $i\in\range{m}$.

The concepts of learning agents, equilibria, and COLS introduced in \cref{sec:online learning,sec:nfgs} can be directly extended to polyhedral convex games without difficulty, by simply replacing the set of mixed strategies $\Delta(\cA_i)$ of each player with their convex polyhedral counterpart $\Omega_i$.

Because the set of mixed strategies $\Omega$ of every player is a polytope, the decision problem of picking a mixed strategy $\vx\^t\in\Omega$ can be equivalently thought of as the decision problem of picking a convex combination $\vlam\^t \in \Delta(\cV_{\Omega})$ over the finite set of vertices $\cV_{\Omega}$ of $\Omega$. Indeed, it is not hard to show that a learning algorithm ${\cR}$ for $\Omega\subseteq\bbR^d$ can be constructed from \emph{any} learning algorithm $\tilde\cR$ for the set of \emph{vertices} $\cV_{\Omega}$, as we describe next. Let $\mat{V}$ denote the matrix whose columns are the vertices $\cV_\Omega$; then:
\begin{itemize}[nosep,left=0mm]
    \item whenever $\cR$ receives a prediction ${\vm}\^t\in\bbR^d$ (resp., loss ${\vl}\^t$), it computes the vector $\tilde{\vm}\^t\defeq\mat{V}^\top\vm\^t\in\bbR^{\cV_\Omega}$ (resp., $\tilde{\vl}\^t\defeq\mat{V}^\top\vl\^t$) and forwards it to $\tilde\cR$;
    \item whenever $\tilde\cR$ plays a new distribution $\vlam\^t \in \Delta(\cV_\Omega)$, the convex combination of vertices $\vx\^t\defeq \sum_{\vv\in\cV_\Omega} \vlam\^t[\vv]\,\vv = \mat{V}\vlam\^t$ is played by $\cR$.
\end{itemize}
It is immediate to verify that the regret cumulated by $\cR$ and $\tilde\cR$ is equal at all times $T$. So, as long as $\tilde\cR$ guarantees sublinear regret, then so does $\cR$. In this paper we are particularly interested in the algorithm obtained by using the above construction for the specific choice of OMWU as the algorithm $\tilde\cR$. We coin \emph{Vertex OMWU} the resulting learning algorithm $\cR$ in that case, depicted in \cref{fig:Vertex OMWU}. Let $\vl\^0,\vm\^0\defeq \vzero\in\bbR^{\cV_\Omega}$ and $\vlam\^0 \defeq \frac{1}{|\cV_\Omega|}\vone\in\Delta(\cV_\Omega)$; then, at all times $t\!\in\!\Npp$, Vertex OMWU updates the convex combination of vertices $\vlam\^{t-1}\!\in\!\Delta(\cV_\Omega)$ according to
\[
    \vlam\^t[\vv] \defeq \frac{\vlam\^{t-1}[\vv]\cdot e^{-\eta\^t\langle\vw\^{t},\vv\rangle}}{\sum_{\vv' \in \cV_\Omega} \vlam\^{t-1}[\vv']\cdot e^{-\eta\^t\langle \vw\^{t}\!,\vv'\rangle}},
    \numberthis[$\clubsuit$]{eq:vertex lam update}
\]%\vspace{-1mm}
where%\vspace{-3mm}
\[
    \vw\^t \defeq \vl\^{t-1} - \vm\^{t-1} + \vm\^t\in\bbR^d,
    \numberthis{eq:def w}
\]
and then outputs the iterate
\[
    \Omega\ni\vx\^t \defeq \sum_{\vv\in\cV_\Omega} \vlam\^t[\vv]\cdot\vv = \mat{V}\vlam\^t.
    \numberthis[$\spadesuit$]{eq:xt original}
\]
It is straightforward to show that Vertex OMWU satisfies \cref{prop:omwu near optimal,prop:omwu optimal sum,prop:omwu last iterate} with $|\cA_i|$ replaced with $|\cV_{\Omega_i}|$, by using a black-box reduction to NFGs. Indeed, let $\Gamma = (m, \{\Omega_i\},\{\bar U_i\})$ be a polyhedral convex game, and introduce the \emph{NFG $\tilde\Gamma$ equivalent to $\Gamma$}, defined as the NFG $\tilde\Gamma \defeq (m, \{\cV_{\Omega_i}\}, \{U_i\})$ where the action set of each player is the set of vertices $\cV_{\Omega_i}$, and $U_i(\vv_1, \dots, \vv_m) \defeq \bar U_i(\vv_1, \dots, \vv_m)$ for all $(\vv_1,\dots,\vv_m)\in\cV_{\Omega_1}\times\dots\times\cV_{\Omega_m}$. Consider the losses $\vl_i\^t$, predictions $\vm\^t$, and iterates $\vx_i\^t\in\Omega_i$ produced by agents learning (under the COLS) in $\Gamma$ using Vertex OMWU, and the losses $\tilde{\vl}_i\^t$, predictions $\tilde{\vm}_i\^t$, and iterates $\vlam_i\^t\in\Delta(\cV_i)$ produced by agents learning (again under the COLS) in $\tilde\Gamma$ using OMWU. For all players $i\in\range{m}$, it is immediate to verify by induction that the relationships (i) $\tilde{\vl}_i\^t = \mat{V}_i^\top \vl_i\^t$, (ii) $\tilde{\vm}_i\^t = \mat{V}_i^\top\vm_i\^t$, and (iii) $\vx_i\^t = \mat{V}_i \vlam_i\^t$ hold at all $t$, where $\mat{V}_i$ is the matrix whose columns are the vertices $\cV_{\Omega_i}$ (see also \cref{fig:Vertex OMWU}). The above discussion shows that in a precise sense, Vertex OMWU and OMWU are the same algorithm, just on different equivalent representations of the game.
Hence, the regret cumulated by each player $i$ in $\Gamma$ matches the regret cumulated by the same player in $\tilde\Gamma$, showing that \cref{prop:omwu near optimal,prop:omwu optimal sum} hold for Vertex OMWU.
Furthermore, whenever $\vlam_i\^t$ converges in iterates, then clearly so does $\vx_i\^t = \mat{V}_i\vlam_i\^t$, showing that \cref{prop:omwu last iterate} applies to Vertex OMWU as well.

The main drawback of Vertex OMWU is that it is not clear how to avoid a per-iteration complexity linear in the number of vertices of $\Omega$, which is typically exponential in $d$ (this is the case in extensive-form games). While different learning algorithms that guarantee polynomial per-iteration complexity in $d$ exist, none of them is known to guarantee near-optimal per-player regret (\cref{prop:omwu near optimal}) or last-iterate convergence (\cref{prop:omwu last iterate}) enjoyed by Vertex OMWU, much less all three \cref{prop:omwu near optimal,prop:omwu optimal sum,prop:omwu last iterate} at the same time. In the rest of the paper we fill this gap, by showing that in several cases of interest, Vertex OMWU can be implemented with polynomial-time (in $d$) iterations using a kernel trick.








\begin{figure}[t]\centering
    \tikzstyle{lbl} = [fill=white,rounded corners,inner ysep=.7mm]
    \tikzstyle{tightlbl} = [lbl,inner xsep=.3mm,inner ysep=.2mm]

    \scalebox{.97}{\begin{tikzpicture}[x=1mm,y=1mm]
            \begin{scope}[local bounding box=vertexbb]
                \node[text=gray,inner sep=0mm] at (-11, 10) {\small$\Gamma$};
                \draw[semithick,fill=white] (-12, 0) rectangle (12, -10) node[fitting node] (VertexOMWU) {};
                \draw[semithick,<-] (VertexOMWU.north) -- +(0,  4) node[above=-1mm,lbl,text width=6mm,align=center] (VertexLoss) {\raisebox{0mm}{\small${\vl}\^t$}};
                \node[above=-0.25mm of VertexLoss,lbl,text width=6mm,align=center] (VertexPred) {\raisebox{0mm}{\small${\vm}\^t$}};
                \draw[semithick,->] (VertexOMWU.south) -- +(0, -3) node[below,lbl] (VertexStrat) {\small$\vx\^t\in\Omega$};

                \node[text width=20mm, anchor=center, align=center] at (VertexOMWU.center) {\small Vertex OMWU\\\textcolor{black!60}{\eqref{eq:vertex lam update},~\eqref{eq:xt original}}};
            \end{scope}
            \begin{scope}[xshift=47mm,local bounding box=vanillabb]
                \node[text=gray,inner sep=0mm] at (11, 10) {\small$\tilde{\Gamma}$};
                \draw[semithick,fill=white] (-12, 0) rectangle (12, -10) node[fitting node] (VanillaOMWU) {};
                \draw[semithick,<-] (VanillaOMWU.north) -- +(0,  4) node[above=-1mm,lbl,text width=6mm,align=center] (VanillaLoss) {\raisebox{0mm}{\small$\tilde{\vl}\^t$}};
                \node[above=-0.25mm of VanillaLoss,lbl,text width=6mm,align=center] (VanillaPred) {\raisebox{0mm}{\small$\tilde{\vm}\^t$}};
                \draw[semithick,->] (VanillaOMWU.south) -- +(0, -3) node[below,lbl] (VanillaStrat) {\small$\vlam\^t\in\Delta(\cV_\Omega)$};

                \node[text width=18mm, anchor=center, align=center] at (VanillaOMWU.center) {\small OMWU\\\textcolor{black!60}{\eqref{eq:vanilla OMWU}}};
            \end{scope}
            \node[blue,tightlbl] (PredLbl) at ($(VertexPred)!.5!(VanillaPred)$) {\scalebox{.9}{\raisebox{1mm}{\small$\tilde{\vm}\^t = \mat{V}^\top{\vm}\^t$}}};
            \node[blue,tightlbl] (LossLbl) at ($(VertexLoss)!.5!(VanillaLoss)$) {\scalebox{.9}{\small$\tilde{\vl}\^t = \mat{V}^\top{\vl}\^t$}};
            \node[violet,tightlbl] (StratLbl) at ($(VertexStrat)!.5!(VanillaStrat)$) {\scalebox{.9}{\raisebox{1mm}{\small$\vx\^t = \mat{V}\vlam\^t$}}};
            \draw[line width=1mm,white] (VertexPred) -- (PredLbl) (PredLbl) -- (VanillaPred);
            \draw[blue] (VertexPred) -- (PredLbl) (PredLbl) edge[->] (VanillaPred);
            \draw[line width=1mm,white] (VertexLoss) -- (LossLbl) (LossLbl) -- (VanillaLoss);
            \draw[blue] (VertexLoss) -- (LossLbl) (LossLbl) edge[->] (VanillaLoss);
            \draw[line width=1mm,white] (VertexStrat) -- (StratLbl) (StratLbl) -- (VanillaStrat);
            \draw[violet] (VertexStrat) edge[<-] (StratLbl) (StratLbl) -- (VanillaStrat);

            \begin{pgfonlayer}{background}
                \filldraw[black!20,thin,fill=black!8] ($(vertexbb.south west) + (-1,-.5)$) rectangle ($(vertexbb.north east) + (1, .5)$);
                \filldraw[black!20,thin,fill=black!8] ($(vanillabb.south west) + (-1,-.5)$) rectangle ($(vanillabb.north east) + (1, .5)$);
                \node[inner ysep=.2mm,inner xsep=0mm,rotate=90,yshift=3mm] at (vertexbb.west) {\small Polyhedral convex game};
                \node[inner ysep=.2mm,inner xsep=0mm,rotate=-90,yshift=3mm] at (vanillabb.east) {\small Equivalent NFG};
            \end{pgfonlayer}
        \end{tikzpicture}}
    %\vspace{-2mm}
    \caption{Construction of the Vertex OMWU algorithm. The matrix $\mat{V}$ has the (possibly exponentially-many) vertices $\cV_\Omega$ of the convex polytope $\Omega$ as columns.}
    \label{fig:Vertex OMWU}
    %\vspace{-4mm}
\end{figure}

%\vspace{-1mm}
\section{Kernelized Multiplicative Weights Update}\label{sec:KOMWU}
%\vspace{-1mm}

In this section, we introduce \emph{Kernelized OMWU (KOMWU)}. Kernelized OMWU gives a way of efficiently simulating the Vertex OMWU algorithm described in \cref{sec:vertex} on polyhedral decision sets whose vertices have 0/1 integer coordinates, as long as a specific \emph{polyhedral kernel} function can be evaluated efficiently. We will assume that we are given a polytope $\Omega \subseteq \bbR^d$ with (possibly exponentially many) 0/1 integral vertices $\cV_\Omega \defeq \{\bv_1, \dots,\bv_{|\cV_\Omega|}\} \subseteq\{0,1\}^d$.
Furthermore, given a vertex $\vec{v}\in\cV_\Omega$, we will write $k \in \vec{v}$ as a shorthand for $\vec{v}[k] = 1$.

We define the \emph{0/1-polyhedral feature map} $\phi_\Omega : \bbR^d \to \bbR^{\cV_\Omega}$ associated with $\Omega$ as the function such that
\[
    \phi_\Omega(\vx)[\vv] \defeq \prod_{k \in \vv} \vx[k] \qquad\forall\,\vx \in \bbR^d, \vv \in \cV_\Omega.
    \numberthis{eq:phi Omega}
\]
Correspondingly, the \emph{0/1-polyhedral kernel} $K_\Omega$ associated with $\Omega$ is defined as the function $K_\Omega : \bbR^d \times \bbR^d \to \bbR$,
\[
    K_\Omega(\vx,\vy) \defeq \langle \phi_\Omega(\vx), \phi_\Omega(\vy) \rangle = \sum_{\vv \in \cV_\Omega} \prod_{k \in \vv} \vx[k] \, \vy[k]. \numberthis{eq:K Omega}
\]
We show that Vertex OMWU can be simulated using $d+1$ evaluation of the kernel $K_\Omega$ at every iteration. The key observation is summarized in the next theorem, which shows that the iterates $\vlam\^t$ produced by Vertex OMWU are highly structured, in the sense that they are always proportional to the feature mapping $\phi_\Omega(\vb\^t)$ for some $\vb\^t\in\bbR^d$.

\begin{theorem}\label{thm:bt}
    Consider the Vertex OMWU algorithm \eqref{eq:vertex lam update}, \eqref{eq:xt original}. At all times $t\ge 0$, the vector $\vb\^t \in \bbR^d$ defined as
    \[
        \vb\^t[k] \defeq \exp\mleft\{-\sum_{\tau=1}^t\eta\^\tau\,\vw\^\tau[k]\mright\}
        \numberthis{eq:bt}
    \]
    for all $k=1,\dots,d$, is such that
    \[
        \vlam\^t = \frac{ \phi_\Omega(\vb\^t) }{ K_\Omega(\vb\^t, \vone)}.\numberthis{eq:b ratio}
    \]
\end{theorem}%\vspace{-3mm}
\begin{proof}%
    By induction.
    \begin{itemize}[nosep,leftmargin=5mm]
        \item At time $t = 0$, the vector $\vb\^0$ is $\vb\^0 = \vone \in \bbR^d$. By definition of the feature map~\eqref{eq:phi Omega}, $\phi_\Omega(\vone) = \vone \in \bbR^{\cV_\Omega}$. So, $K_\Omega(\vb\^0,\vone) = \sum_{\vv\in\cV_\Omega} 1 = |\cV_\Omega|$ and hence the right-hand side of~\eqref{eq:b ratio} is $\frac{1}{|\cV_\Omega|}\vec{1}$, which matches $\vlam\^0$ produced by Vertex OMWU, as we wanted to show.
        \item Assume the statement holds up to some time $t-1 \ge 0$. We will show that it holds at time $t$ as well.
              Since $\bv$ has integral 0/1 coordinates, we can write
              \[
                  \exp\{-\eta\^t\langle\vw\^{t},\vv\rangle\} &= \exp\mleft\{
                  -\eta\^t\,\sum_{k\in\vv} \vw\^t[k]
                  \mright\}\\
                  &= \prod_{k\in\vv} \exp\{-\eta\^t\,\vw\^t[k]\}.
                  \numberthis{eq:exp inner}
              \]
              From the inductive hypothesis and~\eqref{eq:phi Omega}, for all $\vv\in\cV_\Omega$,
              \[
                  \vlam\^{t-1}[\vv] &= \frac{\phi_\Omega(\vb\^{t-1})[\vv]}{K_\Omega(\vb\^{t-1},\vone)}
                  = \frac{\prod_{k\in\vv}\vb\^{t-1}[k]}{K_\Omega(\vb\^{t-1},\vone)}. \numberthis{eq:inductive hyp}
              \]
              Plugging~\eqref{eq:exp inner} and~\eqref{eq:inductive hyp} into~\eqref{eq:vertex lam update}, we have the inductive step
              \[
                  \vlam\^{t}[\vv] &= \frac{
                  \prod_{k\in\vv}\vb\^{t-1}[k]\exp\{-\eta\^t\,\vw\^t[k]\}
                  }{
                  \sum_{\vv\in\cV_\Omega}\prod_{k\in\vv}\vb\^{t-1}[k]\exp\{-\eta\^t\,\vw\^t[k]\}
                  }\\
                  &= \frac{\phi_\Omega(\vb\^{t})[\vv]}{K_\Omega(\vb\^{t}, \vone)}
              \]
              for all $\vv \in \cV_\Omega$, where in the last step we used the fact that $\vb\^t[k] = \vb\^{t-1}[k]\exp\{-\eta\^t\,\vw\^t[k]\}$ by~\eqref{eq:bt}. %
              \qedhere
    \end{itemize}
\end{proof}


The structure of $\vlam\^t$ uncovered by \cref{thm:bt} can be leveraged to compute the iterate $\vx\^t$ produced by Vertex OMWU, \ie the convex combination of the vertices
\eqref{eq:xt original},
using $d+1$ evaluations of the kernel $K_\Omega$. We do so by extending an idea of \citet[eq.~5.2]{Takimoto03:Path}.

\begin{theorem}\label{thm:bt to xt}
    Let $\vb\^t$ be as in \cref{thm:bt}. For each $h =1,\dots,d$, let $\ebar_h \in \bbR^d$ be defined as the indicator vector
    %\vspace{-3mm}
    \[
        \ebar_h[k] \defeq \bbone_{k\neq h} \defeq \begin{cases} 0 & \text{if } k = h\\ 1 & \text{if } k \neq h.\end{cases}
        \numberthis{eq:ebar}
    \]
    Then, at all $t \ge 1$, the iterate $\vx\^t\!\in\!\Omega$ produced by Vertex OMWU can be written as
    \[
        \vx\^t \! = \! \mleft(\!
        1 - \frac{K_\Omega(\vb\^t, \ebar_1)}{K_\Omega(\vb\^t, \vone)}, \dots,
        1 - \frac{K_\Omega(\vb\^t, \ebar_d)}{K_\Omega(\vb\^t, \vone)}
        \!\mright).\numberthis{eq:xt}
    \]
\end{theorem}
\begin{proof}%
    The proof crucially relies on the observation that for all $h=1,\dots,d$, the feature map $\phi_\Omega(\ebar_h)$ satisfies
    \[
        \phi_\Omega(\ebar_h)[\vv] = \prod_{k\in\vv} \ebar_h[k]
        = \prod_{k\in\vv}\bbone_{k\neq h} = \bbone_{h\notin \vv},
        \quad \forall\,\vv\in\cV_\Omega.
    \]
    Using the fact that $\phi_\Omega(\vone) = \vone$, we conclude that
    \[
        \phi_\Omega(\vone)[\vv] - \phi_\Omega(\ebar_h)[\vv] = \bbone_{h \in \vv}, \quad\forall\, h = 1,\dots,d.\numberthis{eq:diff phi}
    \]
    Therefore, for all $k = 1,\dots,d$, we obtain
    \[
        \vx\^t[k] &\overset{\mathclap{\eqref{eq:xt original}}}{=} \sum_{\vv\in\cV_\Omega} \vlam\^t[\vv]\cdot\vv[k] = \sum_{\vv\in\cV_\Omega} \vlam\^t[\vv]\cdot\bbone_{k\in\vv}\\
        &= \sum_{\vv\in\cV_\Omega} \vlam\^t[\vv]\cdot(\phi_\Omega(\vone)[\vv] - \phi_\Omega(\ebar_k)[\bv])\\
        &= \frac{\langle\phi_\Omega(\vb\^t),\phi_\Omega(\vone)\rangle - \langle\phi_\Omega(\vb\^t), \phi_\Omega(\ebar_k)\rangle}{K_\Omega(\vb\^t,\vone)}\\
        &= \frac{K_\Omega(\vb\^t\!,\vone) \!-\! K_\Omega(\vb\^t\!, \ebar_k)}{K_\Omega(\vb\^t,\vone)} = 1 \!-\! \frac{K_\Omega(\vb\^t\!, \ebar_k)}{K_\Omega(\vb\^t,\vone)},
    \]
    where the second equality follows from the integrality of $\vv \in \cV_\Omega$, the third from \eqref{eq:diff phi}, the fourth from \cref{thm:bt}, and the fifth from the definition of
    $K_\Omega$ %
    \eqref{eq:K Omega}.
\end{proof}

%\vspace{-1mm}
Combined, \cref{thm:bt,thm:bt to xt} suggest that by keeping track of the vectors $\vb\^t$ instead of $\vlam\^t$, updating them using \cref{thm:bt} and reconstructing the iterates $\vx\^t$ using \cref{thm:bt to xt}, Vertex OMWU can be simulated efficiently. We call the resulting algorithm, given in \cref{algo:kernelized OMWU}, \emph{Kernelized OMWU (KOMWU)}. Similarly, we call \emph{Kernelized MWU} the non-optimistic version of KOMWU obtained as the special case in which $\vm\^t = \vzero$ at all $t$. In light of the preceding discussion, we have the following.

\begin{theorem}\label{thm:kernel omwu equivalent}
    Kernelized OMWU produces the same iterates $\vec{x}\^t$ as Vertex OMWU when it receives the same sequence of predictions $\vec{m}\^t$ and losses $\vec{\ell}\^t\in\bbR^d$. Furthermore, each iteration of KOMWU runs in time proportional to the time required to compute the $d+1$ kernel evaluations $\{K_\Omega(\vb\^t, \vone), K_\Omega(\vb\^t, \ebar_1), \dots, K_\Omega(\vb\^t,\ebar_d)\}$.
\end{theorem}


\begin{figure}[t]\centering
    \makeatletter
    \newcommand{\removelatexerror}{\let\@latex@error\@gobble} %
    \makeatother
    \removelatexerror
    \scalebox{.95}{\begin{algorithm}[H]
            \caption{Kernelized OMWU (KOMWU)}
            \label{algo:kernelized OMWU}
            \DontPrintSemicolon
            $\vl\^0,~\vm\^0,~\vs\^0 \gets \vzero\in\bbR^d$\Comment*{\color{commentcolor}Initialization]\!\!\!\!}
            \For{$t=1,2,\dots$}{\vspace{-.5mm}
            \textbf{receive} prediction $\vm\^t \in \bbR^d$ of next loss\;
            \Comment{\color{commentcolor}set $\vec{m}\^t = \vec{0}$ for non-predictive variant]}
            \vspace{-1mm}\Hline{}\vspace{-.5mm}
            \Comment{\color{commentcolor}Compute $\vb\^t$ according to \cref{thm:bt}]}\vspace{.0mm}
            $\vw\^t \gets \vl\^{t-1} - \vm\^{t-1}+\vm\^t$\;
            $\vs\^t \gets \vs\^{t-1} + \eta\^t\vw\^t$\Comment*{\color{commentcolor}{\small$\vs\^t = \sum \eta\^\tau\vw\^\tau$}]\!\!\!\!}
            \For{$k=1,\dots,d$}{
            $\vb\^t[k]\gets \exp\{-\vs\^t[k]\}$\Comment*{\color{commentcolor}see \eqref{eq:bt}]\!\!\!\!}
            }
            \vspace{-1mm}\Hline{}\vspace{-.5mm}
            \Comment{\color{commentcolor}Produce iterate $\vx\^t$ according to \cref{thm:bt to xt}]\!\!\!\!}\vspace{.0mm}
            $\vx\^t \gets \vec{0} \in \bbR^d$\;
            $\alpha\gets K_\Omega(\vb\^t, \vone)$\Comment*{\color{commentcolor}$K_\Omega$ is defined in \eqref{eq:K Omega}]\!\!\!\!}
            \For{$k=1,\dots,d$}{
            $\vx\^t[k] \gets 1 - K_\Omega(\vb\^t, \ebar_k) \,/\, \alpha$\Comment*{\color{commentcolor}see \eqref{eq:xt}]\!\!\!\!}
            }
            \textbf{output} $\vx\^t \in \Omega$ and
            \textbf{receive} loss vector $\vl\^t \in \bbR^d$\!\!\!\!\;
            }\vspace{-1mm}
        \end{algorithm}}
    %\vspace{-3mm}
\end{figure}
%\vspace{-1mm}
\section{KOMWU in Extensive-Form Games}\label{sec:kernel efgs}
%\vspace{-1mm}

In this section, we show how the general theory we developed in \cref{sec:kernel efgs}
applies to extensive-form game, \ie tree-form games that incorporate sequential and simultaneous moves, and imperfect information. The central result of this section, \cref{thm:KOMWU in EFGs}, shows that OMWU on the normal-form representation of any EFG can be simulated in linear time in the game tree size via KOMWU, contradicting the popular wisdom that working with the normal form of an extensive-form game is intractable.

%\vspace{-1mm}
\subsection{Preliminaries on Extensive-Form Games}\label{sec:efg notation}
%\vspace{-1mm}

We now briefly recall standard concepts and notation about extensive-form games which we use in the rest of the section. More details and an example are available in \cref{app:efgs}.

In an $m$-player perfect-recall extensive-form game, each player $i\in\range{m}$ faces a tree-form sequential decision problem (TFSDP). In a TFSDP, the player interacts with the environment in two ways: at \emph{decision points}, the
agent must act by picking an action from a set of legal actions; at
\emph{observation points}, the agent observes a signal drawn from a set of
possible signals. We denote the set of decision points of player~$i$ as $\cJ_i$. The set of actions available at decision point $j\in\cJ_i$ is denoted $A_j$. A pair $(j,a)$ where $j\in \cJ_i$ and $a \in A_j$ is called a \emph{non-empty sequence}. The set of all non-empty sequences of player~$i$ is denoted as $\Sigma^*_i \defeq \{(j,a): j\in\cJ, a\in A_j\}$.
For notational convenience, we will often denote an element $(j,a)$ in
$\Sigma_i^*$ as $ja$ without using parentheses. Given a decision point $j \in \cJ_i$, we denote by $p_j$ its
\emph{parent sequence}, defined as the last sequence (that is, decision
point-action pair) encountered on the path from the root of the
decision process to $j$.
If the agent does not act before $j$ (that is, $j$ is the root of the
process or only observation points are encountered on the path from
the root to $j$), we let $p_j$ be set to the special element $\emptyseq$, called the \emph{empty sequence}. We let $\Sigma_i \defeq \Sigma_i^* \cup \{\emptyseq\}$. Given a $\sigma \in \Sigma_i$, we let $\mathcal{C}_\sigma \defeq \{j\in\cJ_i: p_j = \sigma\}$.


An $m$-player extensive-form game is a polyhedral convex game (\cref{sec:vertex}) $\Gamma = (m, \{Q_i\}, \{U_i\})$, where the convex polytope of mixed strategies $Q_i$ of each player $i\in\range{m}$ is called a \emph{sequence-form strategy space} \citep{Romanovskii62:Reduction,Stengel96:Efficient,Koller96:Efficient}, and is defined as
\newcommand*\circled[1]{%
    \tikz[baseline=(C.base)]\node[draw,circle,inner sep=0.8pt](C) {\small #1};\!%
}
\[
    Q_i \defeq \mleft\{\vx \in \bbR^{\Sigma_i}: \!\!\begin{array}{l} \circled{1}~~ \vx[\emptyseq] = 1, \\[1mm] \circled{2}~~ \vx[p_j] = \sum_{a\in A_j} \!\vx[ja] ~~\forall j\in\cJ_i \end{array}\!\!\!\mright\}.
\]

It is known that the set of vertices of $Q_i$ are the \emph{deterministic} sequence-form strategies $\Pi_i \defeq Q_i \cap \{0,1\}^{\Sigma_i}$.
We mention the following result (see \cref{app:proofs}).
%\vspace{-1mm}
\begin{restatable}{proposition}{propnumvertices}\label{prop:efg vertex count}
    The number of vertices of $Q_i$ is upper bounded by $A^{\|Q_i\|_1}$, where $A \!\defeq\! \max_{j\in\cJ_i} |A_j|$ is the largest number of possible actions, and $\|Q_i\|_1 \defeq \max_{\vec{q}\in Q_i}\|\vec{q}\|_1$.
\end{restatable}

%\vspace{-1mm}
We will often need to describe strategies for \emph{subtrees} of the TFDSM faced by each player $i$. We use the notation $j' \succeq j$ to denote the fact that $j'\in\cJ_i$ is a descendant of $j\in\cJ_i$, and $j'\succ j$ to denote a strict descendant (\ie $j'\succeq j \land j'\neq j$). For any $j\in\cJ_i$ we let $\Sigma^*_{i,j} \defeq \{j'a': j' \succeq j, a' \in A_{j'}\}$ denote the set of non-empty sequences in the subtree rooted at $j$. The set of sequence-form strategies for that subtree $j$ is defined as the convex polytope
\[
    Q_{i,j} \!\defeq\! \mleft\{\!\vx \in \bbR^{\Sigma^*_{i,j}}\!: \!\!\!\begin{array}{l} \circled{1}~~ \sum_{a\in A_j}\vx[ja] = 1, \\[1mm] \circled{2}~~ \vx[p_{j'}] \!=\! \sum_{a\in A_{j'}} \!\vx[j'a] ~~\forall j'\succ j\end{array}\!\!\!\mright\}.
\]
Correspondingly, we let $\Pi_{i,j} \defeq Q_{i,j} \cap \{0,1\}^{\Sigma^*_{i,j}}$ denote the set of vertices of $Q_{i,j}$, each of which is a deterministic sequence-form strategy for the subtree rooted at $j$.


\subsection{Linear-time Implementation of KOMWU}

For any player $i$, the 0/1-polyhedral kernel $K_{Q_i}$ associated with the player's sequence-form strategy space $Q_i$ can be evaluated in linear time in the number of sequences $|\Sigma_i|$ of that player. To do so, we introduce a \emph{partial kernel function} $K_j: \bbR^{\Sigma_i}\times\bbR^{\Sigma_i} \to \bbR$ for every decision point $j\in\cJ_i$, %
\[
    K_j(\vec{x}, \vec{y}) \defeq \sum_{\vec{\pi} \in \Pi_{i,j}}\prod_{\sigma \in \vec{\pi}} \vec{x}[\sigma]\,\vec{y}[\sigma].
    \numberthis{eq:def Kj}
\]


\begin{restatable}{theorem}{thmefgkernel}\label{thm:efg kernel computation}
    For any vectors $\vec{x},\vec{y} \in\bbR^{\Sigma_i}$, the two following recursive relationships hold:
    \[
        K_{Q_i}(\vec{x}, \vec{y}) &= \vec{x}[\emptyseq]\,\vec{y}[\emptyseq]\prod_{j \in \mathcal{C}_\emptyseq} K_j(\vec{x},\vec{y}),
        \numberthis{eq:efg kernel computation}
    \]
    and, for all decision points $j\in\cJ_i$,
    \[
        K_j(\vec{x}, \vec{y}) &=\!\sum_{a\in A_j} \mleft(\vec{x}[ja]\,\vec{y}[ja]\prod_{j' \in \mathcal{C}_{ja}} K_{j'}(\vec{x}, \vec{y})\mright).\numberthis{eq:efg kernel computation 2}
    \]
    In particular, \cref{eq:efg kernel computation,eq:efg kernel computation 2} give a recursive algorithm to evaluate the polyhedral kernel $K_{Q_i}$ associated with the sequence-form strategy space of any player $i$ in an EFG in linear time in the number of sequences $|\Sigma_i|$.
\end{restatable}


\cref{thm:efg kernel computation} shows that the kernel $K_{Q_i}$ can be evaluated in linear time (in $|\Sigma_i|$) at any $(\vx,\vy)$. So, the KOMWU algorithm (\cref{algo:kernelized OMWU}) can be trivially implemented for $\Omega = Q_i$ in quadratic $\bigOh(|\Sigma_i|^2)$ time per iteration by directly evaluating the $|\Sigma_i|+1$ kernel evaluations $\{K_{Q_i}(\vb\^t, \vone)\} \cup \{K_{Q_i}(\vb\^t, \ebar_{\sigma}): \sigma \in \Sigma_i\}$ needed at each iteration, where $\ebar_{\sigma}\in\bbR^{\Sigma_i}$, defined in~\eqref{eq:ebar} for the general case, is the vector whose components are $\ebar_{\sigma}[\sigma']\defeq \bbone_{\sigma\neq \sigma'}$ for all $\sigma,\sigma'\in\Sigma_i$.
We refine that result by showing that an implementation of
KOMWU with \emph{linear}-time (\ie $\bigOh(|\Sigma_i|)$) per-iteration complexity exists, by exploiting the structure of the particular set of
kernel evaluations needed at every iteration. In particular, we rely on the following observation.

\begin{restatable}{proposition}{propefgratio}\label{prop:efg ratio}
    For any player $i\in\range{m}$, vector $\vec{x} \in \bbR_{>0}^{\Sigma_i}$, and sequence $ja \in \Sigma^*_i$,
    \[
        \frac{1 \!-\! K_{Q_i}(\vec{x}, \bar{\vec{e}}_{ja}) / K_{Q_i}(\vec{x},\! \vone)}{1 \!-\! K_{Q_i}(\vec{x}, \bar{\vec{e}}_{p_j}) / K_{Q_i}(\vec{x},\! \vone)} = \frac{\vec{x}[ja]\prod_{j'\in\mathcal{C}_{ja}}\! K_{j'}(\vec{x},\!\vone)}{K_j(\vec{x},\! \vone)}.
    \]
\end{restatable}

In order to compute $\{K_{Q_i}(\vb\^t, \ebar_{\sigma}): \sigma \in \Sigma_i\}$ in cumulative $\bigOh(|\Sigma_i|)$ time, we then do the following.
\begin{enumerate}[nosep,left=0mm]
    \item We compute the values $K_j(\vb\^t, \vone)$ for all $j \in \cJ_i$ in cumulative $\bigOh(|\Sigma_i|)$ time by using \eqref{eq:efg kernel computation 2}.\label{step:one}
    \item We compute the ratio $K_{Q_i}(\vb\^t, \ebar_\emptyseq) / K_{Q_i}(\vb\^t, \vone)$ by evaluating the two kernel separately using \cref{thm:efg kernel computation}, spending $\bigOh(|\Sigma_i|)$ time.\label{step:two}
    \item We repeatedly use \cref{prop:efg ratio} in a top-down fashion along the
          tree-form decision problem of player~$i$ to compute the ratio $K_{Q_i}(\vb\^t, \ebar_{ja}) / K_{Q_i}(\vb\^t, \vone)$ for each sequence $ja\in\Sigma_i^*$ given the value of the parent ratio $K_{Q_i}(\vb\^t, \ebar_{p_j}) / K_{Q_i}(\vb\^t, \vone)$ and the partial kernel evaluations $\{K_j(\vb\^t, \vone): j \!\in\! \cJ_i\}$ from Step~\ref{step:one}. For each $ja\in\Sigma_i^*$, \cref{prop:efg ratio} gives a formula whose runtime is linear in the number of children decision points $|\mathcal{C}_{ja}|$ at that sequence. Therefore, the cumulative runtime required to compute all ratios $K_{Q_i}(\vb\^t, \ebar_{ja}) / K_{Q_i}(\vb\^t, \vone)$ is
          $\bigOh(|\Sigma_i|)$.\label{step:three}
    \item By multiplying the ratios computed in Step~\ref{step:three} by the value of $K_{Q_i}(\vb\^t, \vone)$ computed in Step~\ref{step:two}, we can easily recover each $K_{Q_i}(\vb\^t, \ebar_{\sigma})$ for every $\sigma \in \Sigma_i^*$.
\end{enumerate}

Hence, we have just proved the following.

\begin{theorem}\label{thm:KOMWU in EFGs}
    For each player $i$ in a perfect-recall extensive-form game, the Kernelized OMWU algorithm can be implemented exactly, with a per-iteration complexity linear in the number of sequences $|\Sigma_i|$ of that player.
\end{theorem}

\subsection{KOMWU Regret Bounds and Convergence}\label{sec:efg analysis}




If the players in an EFG run KOMWU,
then we can combine \cref{thm:kernel omwu equivalent} with standard OMWU regret bounds, \cref{prop:efg vertex count,prop:omwu near optimal,prop:omwu optimal sum,prop:omwu last iterate} to get the following:
%\vspace{-3mm}
\begin{theorem}
    In an EFG, after $T$ rounds of learning under the COLS, KOMWU satisfies
    %\vspace{-2.5mm}
    \begin{enumerate}[nosep,nolistsep,left=0mm]
        \item
              A player $i$ using KOMWU with $\eta\^t \defeq \eta = \sqrt{8\log(A) \|Q_i\|_1}/\sqrt{T}$ is guaranteed to incur regret at most $R^T_i = \bigOh(\sqrt{\|Q_i\|_1\log(A)T})$.
        \item
              There exist $C, C' > 0$ such that if all $m$ players learn using KOMWU with constant learning rate $\eta\^t \defeq \eta \leq 1/(Cm\log^4T)$, then each player is guaranteed to incur regret at most $\frac{\log (A_i) \|Q_i\|_1}{\eta} + C'\log T$.
        \item
              If all $m$ player learn using KOMWU with $\eta\^t \defeq \eta \leq 1/\sqrt{8}(m-1)$, then the sum of regrets is at most $\sum_{i=1}^m R_i^T = \bigOh(\max_{i=1}^m\{\|Q_i\|_1\log A_i\} \frac{m}{\eta})$.
        \item
              For two-player zero-sum EFGs, if both players learn using KOMWU, then there exists a schedule of learning-rates $\eta^{(t)}$ such that the iterates converge to a Nash equilibrium.
              Furthermore, if the NFG representation of the EFG has a unique Nash equilibrium and both players use learning rates $\eta\^t = \eta \leq 1/8$, then the iterates converge to a Nash equilibrium at a linear rate $C (1+C')^{-t}$, where $C,C'$ are constants that depend on the game.
    \end{enumerate}
    \label{thm:komwu efg}
\end{theorem}
%\vspace{-2mm}
Prior to our result, the strongest regret bound for methods that take linear time per iteration was based on instantiating e.g. follow the regularized leader (FTRL) or its optimistic variant with the dilatable global entropy regularizer of \citet{Farina21:Better}.
For FTRL this yields a regret bound of the form $\bigOh(\sqrt{\log(A)\,\|Q\|_1^2 T})$.
For optimistic FTRL this yields a regret bound of the form $\bigOh(\log(A)\,\|Q\|_1^2 \sqrt{m} T^{1/4})$, when every player in an $m$-player game uses that algorithm and appropriate learning rates.

Our algorithm improves the state-of-the-art rate in two ways.
First, we improve the dependence on game constants by almost a square root factor, because our dependence on $\|Q\|_1$ is smaller by a square root, compared to prior results.
Secondly, in the multi-player general-sum setting, every other method achieves regret that is on the order of $T^{1/4}$, whereas our method achieves regret on the order of $\log^4(T)$.
In the context of two-player zero-sum EFGs, the bound on the sum of regrets in \cref{thm:komwu efg} guarantees convergence to a Nash equilibrium at a rate of $\bigOh(\max_i \|Q_i\|_1\log A_i / T)$.
This similarly improves the prior state of the art.




\citet{Lee21:Last} showed the first last-iterate results for EFGs using algorithms that require linear time per iteration. In particular, they show that the dilated entropy DGF combined with optimistic online mirror descent leads to last-iterate convergence at a linear rate.
However, their result requires learning rates $\eta \leq 1/(8|\Sigma_i|)$. This learning rate is impractically small in practice. In contrast, our last-iterate linear-rate result for KOMWU allows learning rates of size $1/8$.
That said, our result is not directly comparable to theirs. The existence of a unique Nash equilibrium in the EFG representation is a necessary condition for uniqueness in the NFG representation. However, it is possible that the NFG has additional equilibria even when the EFG does not.
\citet{Wei21:Linear} conjecture that linear-rate convergence holds even without the assumption of a unique Nash equilibrium. If this conjecture turns out to be true for NFGs, then \cref{thm:kernel omwu equivalent} would immediately imply that KOMWU also has last-iterate linear-rate convergence without the uniqueness assumption.


\subsection{Experimental Evaluation}
We numerically investigate agents learning under the COLS in Kuhn and Leduc poker \citep{Kuhn50:Simplified,Southey05:Bayes}. %
We compare the maximum per-player regret cumulated by KOMWU for four different choices of constant learning rate, against that cumulated by two standard algorithms from the extensive-form game solving literature (CFR and CFR(RM+)). More details about the games and the algorithms are given in \cref{app:experiments}. Results are shown in \cref{fig:experiments}. We observe that the per-player regret cumulated by KOMWU plateaus and remains constants, unlike the CFR variants. This behavior is consistent with the near-optimal per-player regret guarantees of KOMWU (\cref{thm:komwu efg}).

\begin{figure}[ht]
    \includegraphics[width=1\linewidth]{figs/single_row-crop.pdf}
    \vspace{-4mm}
    \caption{Maximum per-player regret cumulated by KOWMU compared to two variants of the CFR algorithm.}
    \label{fig:experiments}
\end{figure}


\section{Conclusions}
\label{sec:conclusions}

In this paper, we apply shared-workload techniques at the \sql level for
improving the throughput of \qaasl systems without incurring in additional
query execution costs. Our approach is based on query rewriting for grouping
multiple queries together into a single query to be executed in one go. This
results in a significant reduction of the aggregated data access done by the
shared execution compared to executing queries independently.

%execution times and costs of the shared scan operator when
%varying query selectivity and predicate evaluation. We observed that for
%\athena, although the cost only depends on the amount of data read, it is
%conditioned to its ability to use its statistics about the data. In some cases
%a wrong query execution plan leads to higher query execution costs, which the
%end-user has to pay. 

%\bigquery's minimum query execution cost is determined by
%the input size of a query.  However, the query cost can increase depending not
%just in the amount of computation it requires, but in the mix of resources the
%query requires.  

We presented a cost and runtime evaluation of the shared operator driving data access costs. 
Our experimental study using the TPC-H benchmark confirmed the benefits of our
query rewrite approach. Using a shared execution approach reduces significantly
the execution costs. For \athena, we are able to make it 107x cheaper and for
\bigquery, 16x cheaper taking into account Query 10 which we cannot execute,
but 128x if it is not taken into account. Moreover, when having queries that do
not share their entire execution plan, i.e., using a single global plan, we
demonstrated that it is possible to improve throughput and obtain a 10x cost
reduction in \bigquery.

%We followed the TPC systems pricing guideline for
%computing how expensive is to have a TPC-H workload working on the evaluated
%\qaasl systems. The result is that even though we are able to reduce overall
%costs a TPC-H workload in 15x for \bigquery (128x excluding query 10 which we
%could not optimize) and in 107x for \athena, the overall price is at least 10x
%more expensive than the cheapest system price published by the TPC.

There are multiple ways to extend our work. The first is
to implement a full \sql-to-\sql translation layer to encapsulate the proposed
per-operator rewrites.  Another one is to incorporate the initial work on
building a cost-based optimizer for shared execution
\cite{Giannikis:2014:SWO:2732279.2732280} as an external component for \qaasl
systems.  Moreover, incorporating different lines of work (e.g., adding
provenance computation \cite{GA09} capabilities) also based on query
rewriting is part of our future work to enhance our system.


\section*{Acknowledgments}
This material is based on work supported by the National Science Foundation under grants IIS-1718457, IIS-1901403, IIS-1943607, and CCF-1733556, and the ARO under award W911NF2010081.


\bibliographystyle{icml2022}
\bibliography{dairefs}

\clearpage
\onecolumn
\appendix
\section{Additional Related Work}\label{app:related works}
\subsection{More Results for Optimistic Algorithms in Games}
For individual regret in multi-player general-sum NFGs, \citet{Syrgkanis15:Fast} first show $\bigOh(T^{1/4})$ regret for general optimistic OMD and FTRL algorithms.
The result is improved to $\bigOh(T^{1/6})$ by \citep{Chen20:Hedging}, but only for OMWU in two-player NFGs.
\citet{Daskalakis21:Near} show that OMWU enjoys $\bigOh(\log^4 T)$ regret in multi-player general-sum NFGs.

As for last-iterate convergence in two-player zero-sum games, \citet{Daskalakis18:Last} show an asymptotic result for OMWU under the unique Nash equilibrium assumption.
\citet{Wei21:Linear} further show a linear convergence rate while allowing larger learning rates under the same assumption.
\citet{Hsieh21:Adaptive} show another asymptomatic convergence result without the assumption.
It is also worth noting that OGDA, another popular optimistic algorithm, has been shown its last-iterate convergence in general polyhedron games \citep{Wei21:Linear}.
\subsection{Approaches in Online Combinatorial Optimization}
Besides performing MWU/OMWU over vertices, we review two additional approaches in online combinatorial optimization:

\paragraph{OMD over the Convex Hull}
This approach is running Online Mirror Descent (OMD) over the convex hull \citep{koolen2010hedging,audibert2014regret}.
It is well known that OMD with the negative entropy regularizer results in a (dimension-wise) multiplicative weight update.
For the case that the set of vertices is a standard basis, this algorithm coincides with the MWU over the probability simplex.
However, for general cases, it requires to project back to the convex hull and the procedure may not be efficient.
\citet{helmbold2009learning} first used this approach for permutations, and \citet{koolen2010hedging} generally studied it for arbitrary 0/1 polyhedral sets and show its efficiency for more cases.

\paragraph{FTPL}
Another approach is called Follow the Perturbed Leader \citep{Kalai05:Efficient}.
This approach adds a random perturbation to the cumulative loss vector, and greedily selects the vertex with minimal perturbed loss.
The latter procedure corresponds to linear optimization over the set of vertices, which can be solved efficiently for most cases of interest.
We are not aware of any previous work using this approach for EFGs though.

\section{Pseudocode}\label{app:pseudocode}

Below we show pseudocode for OMWU and Vertex OMWU (\cref{sec:vertex}).

\begin{figure}[H]
    \begin{minipage}[t]{.49\textwidth}
        \begin{algorithm}[H]
            \caption{OMWU}
            \label{algo:vanilla OMWU}
            \DontPrintSemicolon
            \KwData{Finite set of choices $\cA$, learning rates $\eta\^t > 0$\!\!\!}
            \BlankLine{}
            \vspace{4mm}
            $\vl\^0,~\vm\^0 \gets \vzero\in\bbR^{\cA};~~\vlam\^0 \gets \frac{1}{|\cA|}\vone\in\Delta(\cA)$\;%
            \label{ln:vanilla init}
            \For{$t=1,2,\dots$}{
            \textbf{receive} prediction $\vm\^t\in\bbR^{\cA}$ of next loss\;
            \Comment{\color{commentcolor}set $\vec{m}\^t = \vec{0}$ for non-predictive variant]}
            $\vw\^t \gets \vl\^{t-1} - \vm\^{t-1} + \vm\^t$\;
            \vspace{8mm}
            \For{$a \in \cA$}{
            $\displaystyle
                \vlam\^t[a] \gets \frac{\vlam\^{t-1}[a]\cdot e^{-\eta\^t\,\vw\^t[a]}}{\sum_{a' \in \cA} \vlam\^{t-1}[a']\cdot e^{-\eta\^t\,\vw\^t[a']}}
            $\label{ln:vanilla lam update}
            }
            \vspace{12.5mm}
            \textbf{output} $\vlam\^t \in \Delta(\cA)$\;
            \textbf{receive} loss vector $\vl\^t \in \bbR^{\cA}$\;
            }
        \end{algorithm}%
    \end{minipage}%
    \hfill%
    \begin{minipage}[t]{.49\textwidth}
        \begin{algorithm}[H]
            \caption{Vertex OMWU}
            \label{algo:vertex OMWU}
            \DontPrintSemicolon
            \KwData{\makebox[5cm][l]{Polytope $\Omega\!\subseteq\!\bbR^d$ with vertices $\{\vv_1,\!...,\!\vv_k\!\}\!\eqqcolon\!\cV_\Omega$,}\newline learning rates $\eta\^t > 0$}
            \BlankLine{}
            $\vl\^0,~\vm\^0 \gets \vzero\in\bbR^d;~~\vlam\^0 \gets \frac{1}{|\cV_\Omega|}\vone\in\Delta(\cV_\Omega)$\;%
            \label{ln:vertex init}
            \For{$t=1,2,\dots$}{
            \textbf{receive} prediction $\vm\^t\in\bbR^d$ of next loss\;
            \Comment{\color{commentcolor}set $\vec{m}\^t = \vec{0}$ for non-predictive variant]}
            $\vw\^t \gets \vl\^{t-1} - \vm\^{t-1} + \vm\^t$\;
            \Hline{}
            \Comment{\color{commentcolor}Run the OMWU update on $\vlam$ using $\cA=\cV_\Omega$]\!\!\!\!}\vspace{.5mm}
            \For{$\vv \in \cV_\Omega$}{
            $\displaystyle
                \vlam\^t[\vv] \gets \frac{\vlam\^{t-1}[\vv]\cdot e^{-\eta\^t\,\langle\vw\^{t},\vv\rangle}}{\sum_{\vv' \in \cV_\Omega} \vlam\^{t-1}[\vv']\cdot e^{-\eta\^t\langle \vw\^{t}\!,\vv'\rangle}}
            $\!\!\!\!\!\!\label{ln:vertex lam update}
            }
            \Hline{}
            \Comment{\color{commentcolor}Compute new convex combination of vertices]\!\!\!\!}\vspace{.5mm}
            $\vx\^t \gets \sum_{\vv\in\cV_\Omega} \vlam\^t[\vv]\cdot\vv$\label{ln:vertex xt}\;\vspace{1mm}
            \textbf{output} $\vx\^t \in \Omega$\;
            \textbf{receive} loss vector $\vl\^t \in \bbR^d$\;
            }
        \end{algorithm}%
    \end{minipage}
\end{figure}
\section{Extensive-Form Games}
\label{app:efgs}


In a \emph{tree-form sequential decision process (TFSDP)} problem the agent
interacts with the environment in two ways: at \emph{decision points}, the
agent must act by picking an action from a set of legal actions; at
\emph{observation points}, the agent observes a signal drawn from a set of
possible signals.
Different decision points can have different sets of legal actions, and
different observation points can have different sets of possible signals.
Decision and observation points are structured as a \emph{tree}: under the standard assumption that
the agent
is not forgetful, so, it is not possible for the agent to cycle back to a
previously encountered decision or observation point by following the
structure of the decision problem.

As an example, consider the
simplified game of \emph{Kuhn poker}~\citep{Kuhn50:Simplified}, depicted
in~\cref{fig:kuhn}. Kuhn poker is a standard benchmark in the EFG-solving community.
In Kuhn poker, each player puts an ante worth $1$ into the pot. Each player is then privately dealt one card from a deck that contains $3$ unique cards (Jack, Queen, King). Then, a single round of betting then occurs, with the following dynamics. First, Player $1$ decides to either check or bet $1$. Then,
\begin{itemize}[nolistsep]
    \item If Player 1 checks Player 2 can check or raise $1$.
          \begin{itemize}[nolistsep]
              \item If Player 2 checks a showdown occurs; if Player 2 raises Player 1 can fold or call.
                    \begin{itemize}
                        \item If Player 1 folds Player 2 takes the pot; if Player 1 calls a showdown occurs.
                    \end{itemize}
          \end{itemize}
    \item If Player 1 raises Player 2 can fold or call.
          \begin{itemize}[nolistsep]
              \item If Player 2 folds Player 1 takes the pot; if Player 2 calls a showdown occurs.
          \end{itemize}
\end{itemize}
When a showdown occurs, the player with the higher card wins the pot and the game immediately ends.

\begin{figure}[th]
    \centering
    \begin{tikzpicture}[scale=1.0]
    \tikzset{edge from parent/.style={}}
    \tikzset{edge from parent path={(\tikzparentnode) -- (\tikzchildnode.north)}}
    \tikzset{level distance=1.05cm}
    \tikzset{sibling distance=.60cm}
    \Tree
     [.\node[obspt](P1){};
      [.\node[decpt](S1) {};
       [.\node[obspt](B1) {};
        [.\node[termina](T1) {};]
        [.\node[decpt](S2) {};
         [.\node[termina](T2) {};]
         [.\node[termina](T3) {};]
        ]
       ]
       [.\node[termina](S3) {};]
      ]
      [.\node[decpt](S4) {};
       [.\node[obspt](B2) {};
        [.\node[termina](T6) {};]
        [.\node[decpt](S5) {};
         [.\node[termina](T7) {};]
         [.\node[termina](T8) {};]
        ]
       ]
       [.\node[termina](S6) {};]
      ]
      [.\node[decpt](S7) {};
       [.\node[obspt](B3) {};
        [.\node[termina](T11) {};]
        [.\node[decpt](S8) {};
         [.\node[termina](T12) {};]
         [.\node[termina](T13) {};]
        ]
       ]
       [.\node[termina](S9) {};]
      ]
    ];
    
   \node[black!70!white,xshift=-4mm,yshift=2mm] at (P1) {$k_1$};
   \node[black!70!white,xshift=-4mm] at (S1) {$j_1$};
   \node[black!70!white,xshift=-4mm] at (S4) {$j_2$};
   \node[black!70!white,xshift=4mm] at (S7) {$j_3$};
   \node[black!70!white,xshift=-4mm] at (B1) {$k_2$};
   \node[black!70!white,xshift=-4mm] at (B2) {$k_3$};
   \node[black!70!white,xshift=-4mm] at (B3) {$k_4$};
   \node[black!70!white,xshift=4mm] at (S2) {$j_4$};
   \node[black!70!white,xshift=4mm] at (S5) {$j_5$};
   \node[black!70!white,xshift=4mm] at (S8) {$j_6$};
   
   \draw[semithick,dashed] (P1) -- (S1);
   \draw[semithick,dashed] (P1) -- (S4);
   \draw[semithick,dashed] (P1) -- (S7);
   \draw[semithick] (S1) -- (B1);
   \draw[semithick] (S1) -- (S3);
   \draw[semithick] (S4) -- (B2);
   \draw[semithick] (S4) -- (S6);
   \draw[semithick] (S7) -- (B3);
   \draw[semithick] (S7) -- (S9);
   \draw[semithick,dashed] (B1) -- (T1);
   \draw[semithick,dashed] (B1) -- (S2);
   \draw[semithick,dashed] (B2) -- (T6);
   \draw[semithick,dashed] (B2) -- (S5);
   \draw[semithick,dashed] (B3) -- (T11);
   \draw[semithick,dashed] (B3) -- (S8);
   
   \draw[semithick] (T2) -- (S2) -- (T3);
   \draw[semithick] (T7) -- (S5) -- (T8);
   \draw[semithick] (T12) -- (S8) -- (T13);


   \path ($(S2)+(-1mm,-3mm)$) --node[text=black,fill=white,inner ysep=.5mm,inner xsep=0,xshift=-1mm,yshift=1mm]{\small fold} (T2);
   \path ($(S2)+(1mm,-3mm)$) --node[text=black,fill=white,inner ysep=.5mm,inner xsep=0,xshift=1mm,yshift=1mm]{\small call} (T3);
   \path ($(S5)+(-1mm,-3mm)$) --node[text=black,fill=white,inner ysep=.5mm,inner xsep=0,xshift=-1mm,yshift=1mm]{\small fold} (T7);
   \path ($(S5)+(1mm,-3mm)$) --node[text=black,fill=white,inner ysep=.5mm,inner xsep=0,xshift=1mm,yshift=1mm]{\small call} (T8);
   \path ($(S8)+(-1mm,-3mm)$) --node[text=black,fill=white,inner ysep=.5mm,inner xsep=0,xshift=-1mm,yshift=1mm]{\small fold} (T12);
   \path ($(S8)+(1mm,-3mm)$) --node[text=black,fill=white,inner ysep=.5mm,inner xsep=0,xshift=1mm,yshift=1mm]{\small call} (T13);
   \path (S1) --node[text=black,fill=white,inner ysep=.5mm,inner xsep=0,yshift=1mm]{\small check} (B1);
   \path (S1) --node[text=black,fill=white,inner ysep=.5mm,inner xsep=0,yshift=1mm]{\small raise} (S3);
   \path (S4) --node[text=black,fill=white,inner ysep=.5mm,inner xsep=0,yshift=1mm]{\small check} (B2);
   \path (S4) --node[text=black,fill=white,inner ysep=.5mm,inner xsep=0,yshift=1mm]{\small raise} (S6);
   \path (S7) --node[text=black,fill=white,inner ysep=.5mm,inner xsep=0,yshift=1mm]{\small check} (B3);
   \path (S7) --node[text=black,fill=white,inner ysep=.5mm,inner xsep=0,yshift=1mm]{\small raise} (S9);

   \path (P1) --node[text=black,inner ysep=1mm,fill=white]{\small jack} (S1);
   \path (P1) --node[text=black,inner ysep=.3mm,fill=white]{\small queen} (S4);
   \path (P1) --node[text=black,inner ysep=1mm,fill=white]{\small king} (S7);

   \path (B1) --node[text=black,fill=white,inner ysep=.5mm,inner xsep=0,xshift=-1mm,yshift=1mm]{\small check} (T1);
   \path (B1) --node[text=black,fill=white,inner ysep=.5mm,inner xsep=0,xshift=1mm,yshift=1mm]{\small raise} ($(S2)$);
   \path (B2) --node[text=black,fill=white,inner ysep=.5mm,inner xsep=0,xshift=-1mm,yshift=1mm]{\small check} (T6);
   \path (B2) --node[text=black,fill=white,inner ysep=.5mm,inner xsep=0,xshift=1mm,yshift=1mm]{\small raise} ($(S5)$);
   \path (B3) --node[text=black,fill=white,inner ysep=.5mm,inner xsep=0,xshift=-1mm,yshift=1mm]{\small check} (T11);
   \path (B3) --node[text=black,fill=white,inner ysep=.5mm,inner xsep=0,xshift=1mm,yshift=1mm]{\small raise} ($(S8)$);
\end{tikzpicture}

    \caption{Tree-form sequential decision making process of the first
        acting player in the game of Kuhn poker.}
    \label{fig:kuhn}
\end{figure}

As soon as the game starts, the agent observes a private card that has been
dealt to them; this is observation point $k_1$, whose set of possible
signals is $S_{k_1} \defeq \{\text{jack},\text{queen},\text{king}\}$.
Should the agent observe the `jack' signal, the decision problem transitions to
the decision point $j_1$, where the agent must pick one action from the set
$A_{j_1} \defeq \{\text{check}, \text{raise}\}$.
If the agent picks `raise', the decision process terminates; otherwise, if
`check' is chosen, the process transitions to observation point $k_2$,
where the agent will observe whether the opponent checks (at which point
the interaction terminates) or raises.
In the latter case, the process transitions to decision point $j_4$, where the
agent picks one action from the set
$A_{j_4} \defeq \{\text{fold},\text{call}\}$.
In either case, after the action has been selected, the interaction terminates.



\section{Experimental Evaluation}\label{app:experiments}

\paragraph{Game instances}
We numerically investigate agents learning under the COLS in Kuhn and Leduc poker \citep{Kuhn50:Simplified,Southey05:Bayes}, standard benchmark games from the extensive-form games literature.
\begin{description}
    \item[\emph{Kuhn poker}] The two-player variant of Kuhn poker first appeared in \citep{Kuhn50:Simplified}. In this paper, we use the multiplayer variant, as described by \citet{Farina18:Ex}. In a multiplayer Kuhn poker game with $r$ ranks, a deck with $r$ unique cards is used. At the beginning of the game, each player pays one chip to the pot (\emph{ante}), and is dealt a single private card (their \emph{hand}). The first player to act can \emph{check} or \emph{bet}, \ie put an additional chip in the pot. Then, the second player can check or bet after a first player's check, or fold/call the first player's bet. If no bet was previously made, the third player can either check or bet, and so on in turn. If a bet is made by a player, each subsequent player needs to decide whether to \emph{fold} or \emph{call} the bet. The betting round if all players check, or if every player has had an opportunity to either fold or call the bet that was made. The player with the highest card who has not folded wins all the chips in the pot.
    \item[\emph{Leduc poker}] We use a multiplayer version of the classical Leduc hold'em poker introduced by \citet{Southey05:Bayes}. We employ game instances of rank 3. The deck consists of three suits with 3 cards each. Our instances are parametric in the maximum number of bets, which in limit hold'em is not necessarily tied to the number of players. As in Kuhn poker, we set a cap on the number of raises to one bet. As the game starts, players pay one chip to the pot. Then, two betting rounds follow. In the first one, a single private card is dealt to each player while in the second round a single board card is revealed. The raise amount is set to 2 and 4 in the first and second round, respectively.
\end{description}
For each game, we consider a 3-player and a 4-player variant. The 3-player Kuhn variant uses a deck with $r=12$ ranks. The 4-player variant uses a deck with a reduced number of ranks equal to $r=5$ to avoid excessive memory usage.

\paragraph{CFR and CFR(RM+)} Modern variants of counterfactual regret minimization (CFR) are the current practical state-of-the-art in two-player zero-sum extensive-form game solving. We implemented both the original CFR algorithm by~\citet{Zinkevich07:Regret}, and a more modern variant (which we denote `CFR(RM+)') using the Regret Matching Plus regret minimization algorithm at each decision point~\citep{Tammelin15:Solving}.

\paragraph{Discussion of results}
We compare the maximum per-player regret cumulated by KOMWU for four different choices of constant learning rate $\eta\^t = \eta \in \{ 0.1, 1, 5, 10\}$, against that cumulated by CFR and CFR(RM+).

We remark that the payoff ranges of these games are not $[0,1]$ (\ie the games have not been normalized). The payoff range of Kuhn poker is $6$ for the 3-player variant and $8$ for the 4-player variant. The payoff range of Leduc poker is $21$ for the 3-player variant and $28$ for the 4-player variant. So, a learning rate value of $\eta=0.1$ corresponds to a significantly smaller learning rate in the normalized game where the payoffs have been shifted and rescaled to lie within $[0,1]$ as required in the statements of \cref{prop:omwu near optimal,prop:omwu optimal sum,prop:omwu last iterate}.

Results are shown in \cref{fig:all games}. In all games, we observe that the maximum per-player regret cumulated by KOMWU plateaus and remains constants, unlike the CFR variants. This behavior is consistent with the near-optimal per-player regret guarantees of KOMWU (\cref{thm:komwu efg}). In the 3-player variant of Leduc poker, we observe that the largest learning rate we use, $\eta=10$, leads to divergent behavior of the learning dynamics.

\begin{figure}[H]\centering
    \includegraphics[scale=.85]{figs/all-crop.pdf}
    \caption{Maximum per-player regret cumulated by KOMWU for four different choices of constant learning rate $\eta\^t = \eta \in \{ 0.1, 1, 5, 10\}$, compared to that cumulated by CFR and CFR(RM+) in two multiplayer poker games.}
    \label{fig:all games}
\end{figure}
\section{Proofs}\label{app:proofs}


\thmefgkernel*
\begin{proof}
    In the proof of this result, we will make use of the following additional notation.
    Given any $\vx \in \bbR^{\Sigma_i}$ and a $j\in \cJ_i$, we let $\vx_{(j)} \in \bbR^{\Sigma_{i,j}^*}$ denote the subvector obtained from $\vx$ by only considering sequences $\sigma \in \Sigma_{i,j}^*$, that is, the vector whose entries are defined as $\vx_{(j)}[\sigma] = \vx[\sigma]$ for all $\sigma \in \Sigma_{i,j}^*$.

    \paragraph{Proof of~\eqref{eq:efg kernel computation}}
    Direct inspection of the definitions of $\Pi_i$ and $\Pi_{i,j}$ (given in \cref{sec:efg notation}), together with the observation that the $\{\Sigma_{i,j}^* : j \in \mathcal{C}_\emptyseq\}$ form a partition of $\Sigma^*_i$, reveals that
    \[
        \Pi_i = \mleft\{\vpi \in \{0,1\}^{\Sigma_i}: \begin{array}{l}\circled{1}~~\vpi[\emptyseq] = 1\\[1mm] \circled{2}~~\vpi_{(j)} \in \Pi_{i,j} \qquad\forall\, j\in\mathcal{C}_\emptyseq \end{array} \mright\}
        \numberthis{eq:Pi i as prod}
    \]
    The observation above can be summarized informally into the statement that ``\emph{$\Pi_i$ is equal, up to permutation of indices, to the Cartesian product $\bigtimes_{j\in \mathcal{C}_\emptyseq}\Pi_{i,j}$}''.
    The idea for the proof is then to use that Cartesian product structure in the definition of 0/1-polyhedral kernel~\eqref{eq:K Omega}, as follows
    \[
        K_{Q_i}(\vx, \vy) &= \sum_{\vec{\pi} \in \Pi_i} \prod_{\sigma \in \vpi} \vx[\sigma]\,\vy[\sigma]\\
        &=\sum_{\vpi\in\Pi_i}\mleft(\vx[\emptyseq]\,\vy[\emptyseq]\prod_{j'\in\mathcal{C}_\emptyseq}\prod_{\sigma\in\vpi_{(j')}} \vx[\sigma]\,\vy[\sigma]\mright)\\
        &=\sum_{\vpi_{(j)}\in\Pi_{i,j} ~\forall\,j \in \mathcal{C}_\emptyseq}\mleft(\vx[\emptyseq]\,\vy[\emptyseq] \prod_{j'\in\mathcal{C}_\emptyseq}\prod_{\sigma\in\vpi_{(j')}} \vx[\sigma]\,\vy[\sigma]\mright)\\
        &=\vx[\emptyseq]\,\vy[\emptyseq] \sum_{\vpi_{(j)}\in\Pi_{i,j} ~\forall\,j \in \mathcal{C}_\emptyseq}\mleft(\prod_{j'\in\mathcal{C}_\emptyseq}\prod_{\sigma\in\vpi_{(j')}} \vx[\sigma]\,\vy[\sigma]\mright)\\
        &=\vx[\emptyseq]\,\vy[\emptyseq] \prod_{j\in\mathcal{C}_\emptyseq} \sum_{\vpi_{(j)}\in\Pi_{i,j}}\prod_{\sigma\in\vpi_{(j)}} \vx[\sigma]\,\vy[\sigma]\\
        &= \vx[\emptyseq]\,\vy[\emptyseq] \prod_{j\in\mathcal{C}_\emptyseq} K_{j}(\vx, \vy),
    \]
    where the second equality follows from the fact that $\{\emptyseq\}\cup\{\Sigma_{i,j}:j\in\mathcal{C}_\emptyseq\}$ form a partition of $\Sigma_i$, the third equality follows from~\eqref{eq:Pi i as prod}, the fifth equality from the fact that each $\vpi_j\in\Pi_{i,j}$ can be chosen independently, and the last equality from the definition of partial kernel function~\eqref{eq:def Kj}.

    \paragraph{Proof of~\eqref{eq:efg kernel computation 2}}
    Similarly to what we did for~\eqref{eq:efg kernel computation}, we start by giving an inductive characterization of $\Pi_{i,j}$ as a function of the children $\Pi_{i,j'}$ for $j' \in \cup_{a\in A_j} \mathcal{C}_{ja}$. Specifically, direct inspection of the definitions of $\Pi_{i,j}$, together with the observation that the $\{\Sigma_{i,j'}^* : j' \in \cup_{a \in A_j}\mathcal{C}_{ja}\}$ form a partition of $\Sigma_{i,j}^*$, reveals that
    \[
        \Pi_{i,j} = \mleft\{\vpi \in \{0,1\}^{\Sigma_{i,j}^*}: \begin{array}{l}\circled{1}~~\sum_{a \in A_j}\vpi[ja] = 1\\[1mm] \circled{2}~~\vpi_{(j')} \in \vpi[ja]\cdot \Pi_{i,j'} \qquad\forall\, a\in A_j,~ j'\in\mathcal{C}_{ja} \end{array} \mright\}.
        \numberthis{eq:Pi j as ch prod intermediate}
    \]
    From constraint \circled{1}\, together with the fact that $\vpi[ja]\in\{0,1\}$ for all $a \in A_j$, we conclude that exactly one $a^* \in A_j$ is such that $\vpi[ja^*] = 1$, while $\vpi[ja] = 0$ for all other $a \in A_j, a \neq a^*$. So, we can rewrite~\eqref{eq:Pi j as ch prod intermediate} as
    \[
        \Pi_{i,j} = \bigcup_{a^* \in A_j}\mleft\{
        \vpi\in\{0,1\}^{\Sigma_{i,j}^*}: \begin{array}{ll}
            \circled{1}~~\vpi[ja^*] = 1                                                                                      \\
            \circled{2}~~\vpi[ja] = 0             \qquad\qquad & \forall\, a\in A_j, a \neq a^*                              \\
            \circled{3}~~\vpi_{(j')} \in \Pi_{i,j'}            & \forall\, j' \in \mathcal{C}_{ja^*}                         \\
            \circled{4}~~\vpi_{(j')} = \vzero                  & \forall\, j'\in\cup_{a \in A_j, a \neq a^*}\mathcal{C}_{ja} \\
        \end{array}
        \mright\},
        \numberthis{eq:Pi j as ch prod}
    \]
    where the union is clearly disjoint.     The above equality can be summarized informally into the statement that ``\emph{$\Pi_{i,j}$ is equal, up to permutation of indices, to a disjoint union over actions $a^*\in A_j$ of Cartesian products $\bigtimes_{j\in \mathcal{C}_{ja^*}}\Pi_{i,j}$}''.
    We can then use the same set of manipulations we already used in the proof of~\eqref{eq:efg kernel computation} to obtain
    \[
        K_{j}(\vx, \vy) &= \sum_{\vec{\pi} \in \Pi_{i,j}} \prod_{\sigma \in \vpi} \vx[\sigma]\,\vy[\sigma]\\
        &= \sum_{\vec{\pi} \in \Pi_{i,j}} \mleft( \vx[ja^*]\,\vy[ja^*]\prod_{j'\in\mathcal{C}_{ja^*}}\prod_{\sigma\in\vpi_{(j')}} \vx[\sigma]\,\vy[\sigma]\mright)\\
        &=\sum_{a^* \in A_j}\sum_{\vpi_{j'}\in\Pi_{i,j'}~\forall\,j' \in \mathcal{C}_{ja^*}} \mleft( \vx[ja^*]\,\vy[ja^*]\prod_{j'\in\mathcal{C}_{ja^*}}\prod_{\sigma\in\vpi_{(j')}} \vx[\sigma]\,\vy[\sigma]\mright)\\
        &=\sum_{a^* \in A_j} \mleft( \vx[ja^*]\,\vy[ja^*] \prod_{j'\in\mathcal{C}_{ja^*}} \sum_{\vpi_{(j')}\in\Pi_{i,j'}}\prod_{\sigma\in\vpi_{(j')}} \vx[\sigma]\,\vy[\sigma]\mright)\\
        &=\sum_{a \in A_j} \mleft( \vx[ja]\,\vy[ja] \prod_{j' \in \mathcal{C}_{ja}} K_{j'}(\vx, \vy) \mright),
    \]
    where the second equality follows from the fact that the $\{\Sigma_{i,j'}^* : j' \in \cup_{a \in A_j}\mathcal{C}_{ja}\}$ form a partition of $\Sigma_{i,j}^*$, third equality follows from~\eqref{eq:Pi j as ch prod}, the fourth equality from the fact that each $\vpi_{j'}\in\Pi_{i,j'}$ can be picked independently, and the last equality from the definition of partial kernel function~\eqref{eq:def Kj} as well as renaming $a^*$ into $a$.
\end{proof}

\propefgratio*
\begin{proof}
    Note that since $\vx > \vzero$, clearly $K_{Q_i}(\vx, \vone), K_j(\vx, 1) > 0$. Furthermore, from~\eqref{eq:diff phi} we have that for all $\sigma \in \Sigma_i$
    \[
        K_{Q_i}(\vx, \vone) - K_{Q_i}(\vx, \ebar_\sigma) &= \langle \phi_{Q_i}(\vone) - \phi_{Q_i}(\ebar_\sigma), \phi_{Q_i}(\vx)\rangle \\
        &= \sum_{\substack{\vpi \in \Pi_i\\\vpi[\sigma] = 1}}\prod_{\sigma' \in \vpi} \vx[\sigma']
        \numberthis{eq:ratio to diff}\\
        &> 0.
    \]
    The above inequality immediately implies that $0 < K_{Q_i}(\vx,\ebar_{p_j})/K_{Q_i}(\vx, \vone) < 1$ and therefore all denominators in the statement are nonzero, making the statement well-formed.

    \newcommand{\splice}[2]{(\!(#1\,|\,#2)\!)}
    In light of~\eqref{eq:ratio to diff}, we further have
    \[
    & \frac{1 - K_{Q_i}(\vec{x}, \bar{\vec{e}}_{ja}) / K_{Q_i}(\vec{x}, \vone)}{1 - K_{Q_i}(\vec{x}, \bar{\vec{e}}_{p_j}) / K_{Q_i}(\vec{x}, \vone)} = \frac{\vec{x}[ja]\prod_{j'\in\mathcal{C}_{ja}} K_{j'}(\vec{x},\vone)}{K_j(\vec{x}, \vone)}\\[2mm]
    & \hspace{2cm}\iff\quad \frac{K_{Q_i}(\vec{x}, \vone) - K_{Q_i}(\vec{x}, \bar{\vec{e}}_{ja})}{K_{Q_i}(\vec{x}, \vone) - K_{Q_i}(\vec{x}, \bar{\vec{e}}_{p_j})} = \frac{\vec{x}[ja]\prod_{j'\in\mathcal{C}_{ja}} K_{j'}(\vec{x},\vone)}{K_j(\vec{x}, \vone)}\\[2mm]
        & \hspace{2cm}\iff\quad\frac{\sum_{\vpi \in \Pi_i, \vpi[ja] = 1}\prod_{\sigma \in \vpi} \vx[\sigma]}{\sum_{\vpi \in \Pi_i, \vpi[p_j] = 1}\prod_{\sigma \in \vpi} \vx[\sigma]} = \frac{\vec{x}[ja]\prod_{j'\in\mathcal{C}_{ja}} K_{j'}(\vec{x},\vone)}{K_j(\vec{x}, \vone)}\numberthis{eq:equivalent}
    \]
    We now prove~\eqref{eq:equivalent}. Let
    \[
        \mathcal{A} \defeq \{\vpi \in \Pi_i : \vpi[ja]=1\}, \qquad \mathcal{B} \defeq \{\vpi \in \Pi_i : \vpi[p_j]=1\}
    \]
    be the domains of the summations. From the definition of $\Pi_i$ (specifically, constraints \circled{2}~in the definition of $Q_i$, of which $\Pi_i$ is a subset; see \cref{sec:efg notation}), it is clear that $\mathcal{A} \subseteq \mathcal{B}$. Furthermore, it is straightforward to check, using the definitions of $\Pi_{i,j}$, $\Pi_i$, and $\mathcal{B}$, that
    \[
        \vpi_{(j)} \in \Pi_{i,j} \qquad\forall\,\vpi\in\mathcal{B}\numberthis{eq:vpij}
    \]

    We now introduce the function $\splice{\cdot}{\cdot} : \mathcal{B} \times \Pi_{i,j} \to \mathcal{B}$ defined as follows.
    Given any $\vpi \in \mathcal{B}$ and $\vpi' \in \Pi_{i,j}$, $\splice{\vpi}{\vpi'}$ is the vector obtained from $\vpi$ by replacing all sequences at or below decision point $j$ with what is prescribed by $\vpi'$; formally,
    \[
        \splice{\vpi}{\vpi'}[\sigma] \defeq \begin{cases}
            \vpi'[\sigma] & \text{if } \sigma \in \Sigma_{i,j}^* \\
            \vpi[\sigma]  & \text{otherwise}.
        \end{cases} \qquad\quad \forall\, \vpi\in\mathcal{B}, \vpi'\in\Pi_{i,j}
        \numberthis{eq:def splice}
    \]
    It is immediate to check that $\splice{\vpi}{\vpi'}$ is indeed an element of $\mathcal{B}$.
    We now introduce the following result.

    \begin{lemma}\label{obs:B}
        There exists a set $\mathcal{P} \subseteq \mathcal{B}$ such that every $\vpi'' \in \mathcal{B}$ can be uniquely written as $\vpi'' = \splice{\vpi}{\vpi'}$ for some $\vpi \in \mathcal{P}$ and $\vpi' \in \Pi_{i,j}$. Vice versa, given any $\vpi \in \mathcal{P}$ and $\vpi' \in \Pi_{i,j}$, then $\splice{\vpi}{\vpi'} \in \mathcal{B}$.
    \end{lemma}
    \begin{proof}
        The second part of the statement is straightforward. We now prove the first part.

        Fix any $\vpi^* \in \Pi_{i,j}$ and let $\mathcal{P} \defeq \{\splice{\vpi}{\vpi^*} : \vpi \in \mathcal{B}\}$. It is straightforward to verify that for any $\vpi'' \in \mathcal{B}$, the choices $\vpi \defeq \splice{\vpi''}{\vpi^*} \in \mathcal{P}$ and $\vpi' \defeq \vpi_{(j)} \in \Pi_{i,j}$ satisfy the equality $\splice{\vpi}{\vpi'} = \vpi''$. So, every $\vpi''\in\mathcal{B}$ can be expressed in \emph{at least one way} as $\vpi'' = \splice{\vpi}{\vpi'}$ for some $\vpi \in \mathcal{P}$ and $\vpi' \in \Pi_{i,j}$. We now show that the choice above is in fact the unique choice. First, it is clear from the definition of $\splice{\cdot}{\cdot}$ that $\vpi'$ must satisfy $\vpi' = \vpi''_{(j)}$, and so it is uniquely determined. Suppose now that there exist $\vpi, \tilde{\vpi}\in\mathcal{P}$ such that $\splice{\vpi}{\vpi'} = \splice{\tilde{\vpi}}{\vpi'}$. Then, $\vpi$ and $\tilde{\vpi}$ must coincide on all $\sigma \in \Sigma_i \setminus \Sigma_{i,j}^*$. However, since all elements of $\mathcal{P}$ are of the form $\splice{\vb}{\vpi^*}$ for some $\vb \in \mathcal{B}$, then $\vpi$ and $\tilde{\vpi}$ must also coincide on all $\sigma \in\Sigma_{i,j}^*$. So, $\vpi$ and $\tilde{\vpi}$ coincide on all coordinates $\sigma\in\Sigma_i$, and the statement follows.
    \end{proof}

    \cref{obs:B} exposes a convenient combinatorial structure of the set $\mathcal{B}$. In particular, it enables us to rewrite the denominator on the left-hand side of \eqref{eq:equivalent} as follows
    \[
        \sum_{\vpi \in \mathcal{B}} \prod_{\sigma\in\vpi} \vx[\sigma] &= \sum_{\vpi' \in \mathcal{P}}\sum_{\vpi''\in\Pi_{i,j}}\prod_{\sigma\in\splice{\vpi'}{\vpi''}} \vx[\sigma]\\
        &=\sum_{\vpi' \in \mathcal{P}}\sum_{\vpi''\in\Pi_{i,j}}\mleft(\prod_{\substack{\sigma\in\splice{\vpi'}{\vpi''}\\\sigma\in\Sigma_{i,j}}} \vx[\sigma]\mright)\mleft(\prod_{\substack{\sigma\in\splice{\vpi'}{\vpi''}\\\sigma\not\in\Sigma_{i,j}}} \vx[\sigma]\mright)\\
        &=\sum_{\vpi' \in \mathcal{P}}\sum_{\vpi''\in\Pi_{i,j}}\mleft(\prod_{\sigma\in\vpi''} \vx[\sigma]\mright)\mleft(\prod_{\substack{\sigma\in\vpi'\\\sigma\not\in\Sigma_{i,j}}} \vx[\sigma]\mright)\\
        &=\mleft(\sum_{\vpi''\in\Pi_{i,j}}\prod_{\sigma\in\vpi''} \vx[\sigma]\mright)\mleft(\sum_{\vpi' \in \mathcal{P}}\prod_{\substack{\sigma\in\vpi'\\\sigma\not\in\Sigma_{i,j}}} \vx[\sigma]\mright)\\
        &= K_j(\vx, \vone)\cdot \mleft(\sum_{\vpi' \in \mathcal{P}}\prod_{\substack{\sigma\in\vpi'\\\sigma\not\in\Sigma_{i,j}}} \vx[\sigma]\mright),\numberthis{eq:denom}
    \]
    where we used~\eqref{eq:def splice} in the third equality.

    We can use a similar technique to express the numerator of the left-hand side of~\eqref{eq:equivalent}. Let
    \[
        \Pi_{i,ja} \defeq \{\vpi \in \Pi_{i,j} : \vpi[ja] = 1\}.
    \]
    Using the constraints that define $\Pi_i$ and the definition of $\mathcal{A}$, it follows immediately that for any $\vpi\in\mathcal{A}$, $\vpi_{(j)} \in \Pi_{i,ja}$. Furthermore, a direct consequence of \cref{obs:B} is the following:
    \begin{corollary}\label{obs:A}
        The same set $\mathcal{P} \subseteq \mathcal{B}$ introduced in \cref{obs:B} is such that every $\vpi'' \in \mathcal{A}$ can be uniquely written as $\vpi'' = \splice{\vpi}{\vpi'}$ for some $\vpi \in \mathcal{P}$ and $\vpi' \in \Pi_{i,ja}$.
    \end{corollary}
    Using \cref{obs:A} and following the same steps that led to~\eqref{eq:denom}, we express the numerator of the left-hand side of~\eqref{eq:equivalent} as
    \[
        \sum_{\vpi \in \mathcal{A}} \prod_{\sigma\in\vpi} \vx[\sigma] &= \sum_{\vpi' \in \mathcal{P}}\sum_{\vpi''\in\Pi_{i,ja}}\prod_{\sigma\in\splice{\vpi'}{\vpi''}} \vx[\sigma]\\
        &=\sum_{\vpi' \in \mathcal{P}}\sum_{\vpi''\in\Pi_{i,ja}}\mleft(\prod_{\substack{\sigma\in\splice{\vpi'}{\vpi''}\\\sigma\in\Sigma_{i,j}}} \vx[\sigma]\mright)\mleft(\prod_{\substack{\sigma\in\splice{\vpi'}{\vpi''}\\\sigma\not\in\Sigma_{i,j}}} \vx[\sigma]\mright)\\
        &=\sum_{\vpi' \in \mathcal{P}}\sum_{\vpi''\in\Pi_{i,ja}}\mleft(\prod_{\sigma\in\vpi''} \vx[\sigma]\mright)\mleft(\prod_{\substack{\sigma\in\vpi'\\\sigma\not\in\Sigma_{i,j}}} \vx[\sigma]\mright)\\
        &=\mleft(\sum_{\vpi''\in\Pi_{i,ja}}\prod_{\sigma\in\vpi''} \vx[\sigma]\mright)\mleft(\sum_{\vpi' \in \mathcal{P}}\prod_{\substack{\sigma\in\vpi'\\\sigma\not\in\Sigma_{i,j}}} \vx[\sigma]\mright).
        \numberthis{eq:numer intermediate}
    \]
    The statement then follows immediately if we can prove that
    \[
        \sum_{\vpi\in\Pi_{i,ja}}\prod_{\sigma\in\vpi} \vx[\sigma] = \vec{x}[ja]\,\prod_{j'\in\mathcal{C}_{ja}} K_{j'}(\vx, \vone).
    \]
    To do so, we use the same approach as in the proof of \cref{thm:efg kernel computation}. In fact, we can directly use the inductive characterization of $\Pi_{i,j}$ obtained in~\eqref{eq:Pi j as ch prod} to write
    \[
        \Pi_{i,ja} = \mleft\{
        \vpi\in\{0,1\}^{\Sigma_{i,j}^*}: \begin{array}{ll}
            \circled{1}~~\vpi[ja] = 1                                                                                                \\
            \circled{2}~~\vpi[ja'] = 0             \qquad\qquad & \forall\, a'\in A_j                                    , a' \neq a \\
            \circled{3}~~\vpi_{(j')} \in \Pi_{i,j'}             & \forall\, j' \in \mathcal{C}_{ja}                                  \\
            \circled{4}~~\vpi_{(j')} = \vzero                   & \forall\, j'\in\cup_{a' \in A_j, a' \neq a}\mathcal{C}_{ja'}       \\
        \end{array}
        \mright\},
    \]
    which fundamentally uncovers the \emph{Cartesian-product structure of $\Pi_{i,ja}$}. Using the same technique as \cref{thm:efg kernel computation}, we then have
    \[
        \sum_{\vpi\in\Pi_{i,ja}}\prod_{\sigma\in\vpi} \vx[\sigma] &=
        \sum_{\vpi_{(j')}\in\Pi_{i,j'}~\forall\,j' \in \mathcal{C}_{ja}} \mleft( \vx[ja]\prod_{j'\in\mathcal{C}_{ja}}\prod_{\sigma\in\vpi_{(j')}} \vx[\sigma]\mright)\\
        &= \mleft( \vx[ja] \prod_{j'\in\mathcal{C}_{ja}} \sum_{\vpi_{(j')}\in\Pi_{i,j'}}\prod_{\sigma\in\vpi_{(j')}} \vx[\sigma]\mright)\\
        &= \mleft( \vx[ja] \prod_{j' \in \mathcal{C}_{ja}} K_{j'}(\vx, \vone) \mright),
    \]
    and the statement is proven.
\end{proof}

\propnumvertices*
\begin{proof}
    The proof is by induction. As the base case consider a single decision point $\Delta^b$ with $b \leq A$ actions. Then the number of vertices is $b \leq A = A^{\|\Delta^b\|_1}$.

    For the induction step we consider two cases.
    First, consider a polytope $Q$ whose root is a decision point with $b\leq A$ actions, with each action $a$ leading to a polytope $Q_a$ whose number of vertices $v_a$ satisfies the inductive assumption (if some action $a$ is a terminal action then we overload notation and let $v_a=1$ and $\|Q_a\|_1 = 0$).
    Then, the number of vertices of $Q$ is
    \[
        \sum_{a=1}^b v_a
        &\leq \sum_{a=1}^b A^{\|Q_a\|_1} \\
        &\leq b \cdot A^{\max_{a\in \range{b}}\|Q_a\|_1}\\
        &\leq A\cdot A^{\max_{a\in \range{b}}\|Q_a\|_1} \\
        &= A^{\|Q\|_1}.
    \]

    Second, consider a polytope $Q$ whose root is an observation point with $b$ observations, with each observation $o$ leading to a polytope $Q_o$ with $v_o$ vertices, such that the inductive assumption holds.
    Then, the number of vertices of $Q$ is
    \[
        v = \prod_{o=1}^b v_o
        &\leq \prod_{o=1}^b A^{\|Q_o\|_1}
        \leq A^{\sum_{o=1}^b \|Q_o\|_1}
        = A^{\|Q\|_1}.
    \]
\end{proof}


\section{Further Applications}\label{app:applications}

In this appendix, we illustrate additional 0/1-polyhedral domains in which our polyhedral kernel can be computed efficiently.

\subsection{$n$-sets}\label{sec:nsets}

We start from $n$-sets, that is, the 0/1-polydral set $\Omega^d_n \defeq \textrm{co}\{\vpi \in \{0,1\}^d: \|\vpi\|_1 = n\}$. Learning over $n$-sets is a classic problem first considered by~\citet{warmuth2008randomized} with an application to online Principal Component Analysis.
They proposed an Online Mirror Descent algorithm operating over the convex hull $\Omega^d_n$, with per-iteration complexity of $\bigOh(d^2)$.
The Follow-the-Perturbed-Leader approach~\citep{Kalai05:Efficient} is even faster with per-iteration complexity of $\bigOh(d\log d)$, but it often leads to sub-optimal regret bounds (see discussions in~\citep{koolen2010hedging}).
Simulating MWU over the vertices of $\nset$ has been considered in for example~\citep{cesa2012combinatorial}, where they proposed to use the general approach of~\citep{Takimoto03:Path} to implement this algorithm, leading to per-iteration complexity of $\bigOh(d^2 n)$.
Below, we show that our kernelized approach admits an even faster per-iteration complexity of $\bigOh(d\min\{n, d-n\})$.

\subsubsection{Polynomial, $\bigOh(d \min\{n, d-n\})$-time kernel evaluation} Let $\vx, \vy \in \bbR^d$, and assume for now $n \le d-n$. Introduce the polynomial $p_{\vx,\vy}(z)$ of $z$, defined as
\[
    p_{\vx,\vy}(z) \defeq (\vx[1]\vy[1]\, z + 1) \cdots (\vx[d]\vy[d]\, z + 1).
\]
It is immediate to see that the coefficient of $z^n$ in the expansion of $p_{\vx,\vy}(z)$ is exactly $K_{\nset}(\vx, \vy)$. Such coefficient can be computed by directly carrying out the multiplication of the binomial terms, keeping track of the term of degree $0,\dots,n$. So, each evaluation of $K_\nset(\vx,\vy)$ can be carried out in $\bigOh(nd)$ time under the assumption that $n < d-n$.

If on the other hand $n < d-n$, we can repeat the whole argument above for the polynomial
$q_{\vx,\vy}(z) \defeq (z + \vx[1]\vy[1]) \cdots (z + \vx[d]\vy[d])$ instead. In that case, we are interested in the coefficients of $z^{d-n}$, which can be computed in $\bigOh(d(d-n))$ using the same procedure described above.

Putting together the two cases, we conclude that the computation of $K_\nset(\vx,\vy)$ requires $\bigOh(d\min\{n,d-n\})$ time.


\subsubsection{Implementing KOMWU with $\bigOh(d\min\{n, d-n\})$ per-iteration complexity}
The result described in the previous paragraph immediately implies that KOMWU can be implemented with $\bigOh(d^2\min\{n,d-n\})$-time iterations. In this subsection we refine the that result by showing that it is possible to compute the $d$ kernel evaluations $\{K_{\nset}(\vx,\ebar_k) : k =1,\dots,d\}$ required at every iteration by KOMWU so that they take cumulative $\bigOh(d\cdot\min\{n,d-n\})$ time.

To do so, we build on the technique described in the previous subsection. Assume again that $n \le d-n$. The key insight is that the coefficient of $z^n$ of the polynomial $p_{\vx,\vone}(z) / (\vx[j]\, z + 1)$ is exactly $K_{\nset}(\vx, \ebar_j)$. So, to compute all $\{K_{\nset}(\vx, \ebar_k): k =1,\dots,d\}$ we can do the following:
\begin{enumerate}[nosep,left=1mm]
    \item First, for all $k = 0,\dots,d$ and $h=0,\dots,n$, we compute the coefficient $A[k,h]$ of the $z^h$ in the expansion of $(\vx[1]\,z+1)\dots (\vx[k]\,z +1)$

          We can compute all such values in $\bigOh(dn)$ time by using dynamic programming. In particular, we have
          \[
              A[k,h] = \begin{cases}
                  1                                 & \text{if }h=0               \\
                  0                                 & \text{if }k=0 \land h\neq 0 \\
                  A[k-1,h] + \vx[k]\cdot A[k-1,h-1] & \text{otherwise.}
              \end{cases}
          \]

    \item Then, for all $k=1, \dots, d+1$ and $h=0,\dots,n$, we compute the coeffience $B[k,h]$ of $z^h$ in the expansion of $(\vx[k]\,z+1)\cdots (\vx[d]\, z+1)$

          Again, we can do that in $\bigOh(dn)$ time by using dynamic programming. Specifically,
          \[
              B[k,h] = \begin{cases}
                  1                                 & \text{if }h=0                \\
                  0                                 & \text{if }k=d+1\land h\neq 0 \\
                  B[k+1,h] + \vx[k]\cdot B[k+1,h-1] & \text{otherwise.}
              \end{cases}
          \]

    \item (Note that at this point, $K_{\nset}(\vx,\mathbf{1})$ is simply $A[d,n]$.)
    \item For each $k = 1,\dots,d$, $K_{\nset}(x, \ebar_j)$ can be computed as
          \[
              K_{\nset}(\vx,\ebar_k) = \sum_{h=0}^n A[k-1,h]\cdot B[k+1,n-h].
          \]
          The above formula takes $\bigOh(n)$ time to be computed (we need to iterate over $h=0,\dots,n$), and we need to evaluate it $d$ times (once per each $k=1,\dots,d$). So, computing all $\{K_{\nset}(\vx,\ebar_k): k =1,\dots,d\}$ takes cumulative $\bigOh(dn)$ time, as we wanted to show.
\end{enumerate}

As in the previous subsection, the case $n > d-n$ is symmetric. In that case, the set of values $\{K_{\nset}(\vx,\ebar_k): k =1,\dots,d\}$ can be computed in cumulative $\bigOh(d(d-n))$ time.

\subsection{Unit Hypercube}

Consider the hypercube $[0,1]^d$, whose vertices are all the vectors in $\{0,1\}^d$. In this case, the polyhedral kernel is simply
\[
    K_{[0,1]^d}(\vx,\vy) = (\vx[1]\cdot \vy[1] + 1) \cdots (\vx[d]\cdot \vy[d] + 1),
\]
which can be clearly evaluated in $\bigOh(d)$ time. Similarly to $n$-sets (\cref{sec:nsets}), we can avoid paying an extra $d$ factor in the per-iteration complexity of KOMWU by using the following procedure:
\begin{enumerate}
    \item For each $k = 0,\dots,d$ define $A[k] \defeq (\vx[1]\cdot \vy[1] + 1) \cdots (\vx[k] \cdot \vy[k] + 1)$. Clearly, the $A[k]$ values can be computed in $\bigOh(d)$ cumulative time.
    \item For each $k = 1,\dots,d+1$, define $B[k] \defeq (\vx[k]\cdot \vy[k] + 1) \cdots (\vx[d] \cdot \vy[d] + 1)$. Again, all $B[k]$ values can be computed in $\bigOh(d)$ cumulative time.
    \item For each $k=1,\dots,d$, we have that $K_{[0,1]^d}(\vx, \ebar_k) = A[k-1]\cdot B[k+1]$. Hence, we can compute $\{K_{[0,1]^d}(\vx,\ebar_k): k=1,\dots,d\}$ in cumulative $\bigOh(d)$ time.
\end{enumerate}

\subsection{Flows in Directed Acyclic Graphs}

The polytope $\mathcal{F}$ of flows in a generic directed acyclic graphs (DAGs) has vertices with 0/1 integer coordinates, corresponding to paths in the DAG. The 0/1-polyhedral kernel $K_{\mathcal{F}}$ corresponding to the set of flows in a DAG coincides with the kernel function introduced by \citet{Takimoto03:Path}, which was shown to be computable in polynomial-time in the size of the DAG. Consequently, $K_\mathcal{F}$ admits polynomial-time (in the size of the DAG) evaluation.


\subsection{Permutations}

When $\mathcal{P}$ is the convex hull of the set of all $d\times d$ permutation matrices, it is believed that $K_{\mathcal{P}}$ cannot be evaluated in polynomial time in $\bigOh(d)$, since the computation of the permanent of a matrix $\mat{A}$ can be expressed as $K_\Omega(\mat{A}, \vone)$. However, an $\epsilon$-approximate computation of $K_{\mathcal{P}}$ can be performed in $\bigOh(\poly(d,\log(1/\epsilon)))$ for any $\epsilon > 0$ by using a landmark result by \citet{jerrum2004polynomial}. We refer the interested reader to the paper by \citet[Section 5.3]{cesa2012combinatorial}.

\subsection{Cartesian Product}

Finally, we remark that when two 0/1-polyhedral sets have efficiently-computable 0/1-polyhedral kernels, then so does their Cartesian product. Specifically, let $\Omega\subseteq\bbR^d, \Omega'\subseteq\bbR^{d'}$ be 0/1-polyhedral sets, and let $K_\Omega, K_{\Omega'}$ be their corresponding 0/1-polyhedral kernels. Then, it follows immediately from the definition that the polyhedral kernel of $\Omega\times\Omega'$ satisfies
\newcommand{\vstack}[2]{\begin{pmatrix}#1\\#2\end{pmatrix}}
\[
    K_{\Omega\times\Omega'}\mleft(\vstack{\vx}{\vx'}, \vstack{\vy}{\vy'}\mright) = K_\Omega(\vx, \vy) \cdot K_{\Omega'}(\vx',\vy') \qquad\forall\,\vstack{\vx}{\vy}, \vstack{\vx'}{\vy'} \in \bbR^d\times\bbR^{d'}.
\]




\end{document}
