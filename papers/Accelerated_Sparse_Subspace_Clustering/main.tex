\documentclass{article}
\usepackage{spconf,amsmath,graphicx,multicol}
\usepackage{amsfonts}
\usepackage{color}
\usepackage{bbm}
\usepackage{cite}
\usepackage{amssymb}
\usepackage{algorithm}
\usepackage{algorithmic}
\usepackage{tabularx}
\usepackage[utf8]{inputenc}
\usepackage[english]{babel}
\let\proof\relax
\let\endproof\relax
\usepackage{amsthm}
\usepackage{subcaption}
\usepackage[font={footnotesize}]{caption}
\usepackage{url}
\usepackage{mathtools}
\usepackage{comment}
\usepackage{float}
%\usepackage[toc,page]{appendix}
\usepackage[keeplastbox]{flushend}
\renewcommand{\qedsymbol}{$\blacksquare$}
%\theoremstyle{definition}
\newtheorem{definition}{Definition}[]
\newtheorem{proposition}{Proposition}[]
\newtheorem{theorem}{Theorem}[]
\newtheorem{corollary}{Corollary}[theorem]
\newtheorem{lemma}[theorem]{Lemma}
\newcommand\norm[1]{\left\lVert#1\right\rVert}
\newcommand{\argmax}{\arg\!\max}
\newcommand{\argmin}{\arg\!\min}
\DeclarePairedDelimiter{\ceil}{\lceil}{\rceil}
\DeclarePairedDelimiter\floor{\lfloor}{\rfloor}
\DeclareMathOperator{\E}{\mathbb{E}}
\def\SNR{{\mathrm{SNR}}}
\def\q{{\mathbf q}}
\def\z{{\mathbf z}}
\def\y{{\mathbf y}}
\def\s{{\mathbf s}}
\def\e{{\boldsymbol{\nu}}}
\def\W{{\mathbf W}}
\def\L{{\mathbf L}}
\def\C{{\mathbf C}}
\def\G{{\mathbf \Gamma}}
\def\P{{\mathbf P}}
\def\R{{\mathbb{R}}}
\def\N{{\mathbb{N}}}
\def\I{{\mathbf I}}
\def\h{{\mathbf h}}
\def\r{{\mathbf r}}
\def\F{{\mathbf F}}
\def\Si{{\boldsymbol{\Psi}}}
\def\f{{\boldsymbol{\phi}}}
\def\x{{\mathbf x}}
\def\w{{\mathbf w}}
\def\X{{\mathbf X}}
\def\Z{{\mathbf Z}}
\def\Y{{\mathbf Y}}
\def\A{{\mathbf A}}
\def\B{{\mathbf B}}
\def\R{{\mathbb{R}}}
\def\N{{\mathbb{N}}}
\def\I{{\mathbf I}}
\def\a{{\mathbf a}}
\def\t{{\mathbf t}}
\def\b{{\mathbf b}}
\def\u{{\mathbf u}}
\def\c{{\mathbf c}}
\def\v{{\mathbf v}}
\def\D{{\cal D}}
\newcommand{\ts}{\textsuperscript}
\newcommand{\abs}[1]{\lvert #1 \rvert}
\newcommand{\ie}{{\it i.e., }}
\newcommand{\etal}{{\em et~al.~}}
\newcommand{\Var}{\mathrm{Var}}
\newcommand{\Cov}{\mathrm{Cov}}

\let\oldref\ref
\renewcommand{\ref}[1]{(\oldref{#1})}
\newcommand{\RNum}[1]{\uppercase\expandafter{\romannumeral #1\relax}}
%\renewcommand{\qedsymbol}{$\blacksquare$}
\makeatletter
\renewcommand{\fnum@figure}{Fig.~\thefigure}
\makeatother

\title{Accelerated Sparse Subspace Clustering}
%
% Single address.
% ---------------
\name{Abolfazl Hashemi and Haris Vikalo}
\address{Department of Electrical and Computer Engineering,  University of Texas at Austin, Austin, TX, USA}
\begin{document}
%\ninept
%
\maketitle
%
\begin{abstract}
State-of-the-art algorithms for sparse subspace clustering perform spectral clustering on a similarity matrix typically obtained by representing each data point as a sparse combination of other points using either basis pursuit (BP) or orthogonal matching pursuit (OMP). BP-based methods are often prohibitive in practice while the performance of OMP-based schemes are unsatisfactory, especially in settings where data points are highly similar. In this paper, we propose a novel algorithm that exploits an accelerated variant of orthogonal least-squares to efficiently find the underlying subspaces. We show that under certain conditions the proposed algorithm returns a subspace-preserving solution. Simulation results illustrate that the proposed method compares favorably with BP-based method in terms of running time while being significantly more accurate than OMP-based schemes. 
\end{abstract}
%
\begin{keywords}
sparse subspace clustering, accelerated orthogonal least squares, scalable algorithm, large-scale data
\end{keywords}
\vspace{-0.2cm}
%%%%%%%%%%%%%%%%%%%%%%%%%%%%%%%%%%%%%%%%%%%%%%%%%%%%%%%%%%%%%%%%%%
%%%%%%%%%%%%%%%%%%%%%%%%%%%%%%%%%%%%%%%%%%%%%%%%%%%%%%%%%%%%%%%%%%
\section{Introduction}\label{sec:intro}
Consider a network represented by a graph $\mathcal{G}$ consisting of a node set $\mathcal{N}$ of 
cardinality $N$ and a weighted adjacency matrix $\mathbf{A} \in \mathbb{R}^{N \times N}$ whose
$(i,j)$ entry, $A_{ij}$, denotes weight of the edge connecting node $i$ to node $j$.  A \textit{graph 
signal} $\mathbf{x} \in \mathbb{R}^{N}$ can be viewed as a vertex-valued network 
process that can be represented by a vector of size $N$ supported on $\mathcal{N}$, where its $i$
\textsuperscript{th} component denotes the signal value at node $i$. Under the assumption that properties of the network process relate to the underlying graph, the goal of graph signal processing (GSP) is to generalize traditional signal processing tasks and develop algorithms that fruitfully exploit this relational structure \cite{shuman2013,sandryhaila2013}.

A keystone generalization which has drawn considerable attention in recent years pertains to sampling 
and reconstruction of graph signals \cite{shomorony2014sampling,tsitsvero2016signals,anis2016efficient,chen2015discrete,chepuri2016subsampling,marques2016sampling,gama2016rethinking,chamon2017greedy}. The task of finding an exact sampling set to perform reconstruction with minimal information loss is known to be NP-hard.
Conditions for exact reconstruction of graph signals from noiseless samples were put forth in~\cite{shomorony2014sampling,tsitsvero2016signals,anis2016efficient,chen2015discrete}. Sampling of noise-corrupted signals using randomized schemes including uniform and leverage score sampling is studied in \cite{chen2016signal}, for which optimal sampling distributions and
performance bounds are derived.
In \cite{chepuri2016subsampling,chamon2017greedy}, reconstruction of graph signals and their power spectrum density is studied and greedy schemes are developed. However, the performance guarantees 
in \cite{chen2016signal,chamon2017greedy} are restricted to the case of stationary graph signals, i.e., 
the covariance matrix 
%of the signal 
in the nodal or spectral domains has certain structure (e.g., diagonal; see also \cite{marques2016stationaryTSP16,perraudinstationary2016,girault_stationarity}). 

In this paper, we study the problem of sampling and reconstruction of graph signals and propose two algorithms that solve it approximately. First, we develop a semidefinite programming (SDP) relaxation that finds a near-optimal sampling set in polynomial time. Then, we formulate the sampling task as that of maximizing a monotone weak submodular function and propose an efficient randomized greedy algorithm motivated by \cite{mirzasoleiman2014lazier}. We analyze the performance of the randomized greedy algorithm and in doing so, we show that the MSE associated with the selected sampling set is on expectation a constant factor away from that of the optimal set. Moreover, we prove that the randomized greedy algorithm achieves the derived approximation bound with high probability for every sampling task.  
In contrast to prior work, our results do not require stationarity of the signal. Finally, in simulation studies we illustrate the superiority of the proposed schemes over state-of-the-art randomized and greedy algorithms \cite{chen2016signal,chamon2017greedy} in terms of running time, accuracy, or both.
\footnote{MATLAB implementations of the proposed algorithms are available at \url{https://github.com/realabolfazl/GS-sampling}.} 

% The rest of the paper is organized as follows. Section \oldref{sec:pre} formally states the sampling problem and reviews the relevant background. In Section \oldref{sec:alg}, we introduce the SDP relaxation-based and randomized greedy algorithms for the sampling task and theoretically analyze the performance of the latter. Section \oldref{sec:sim} presents the simulation results while the concluding remarks are stated in Section \oldref{sec:concl}. 

\noindent\textbf{Notation.} 
% We briefly summarize the notation used in the paper. Capital letters represent sets, e.g., $S$. Bold capital letters refer to matrices and bold lowercase letters represent vectors. 
$\A_{ij}$ denotes the $(i,j)$ entry of matrix $\A$, $\a_j$ is the $j\ts{th}$ row of $\A$, $\A_{S,r}$ ($\A_{S,c}$) is a submatrix of $\A$ that contains rows (columns) indexed by the set $S$, and $\lambda_{\max}(\A)$
and $\lambda_{\min}(\A)$ represent the maximum and minimum eigenvalues of $\A$, respectively. $\I_N \in \R^{N\times N}$ denotes the identity matrix and $[N] := \{1,2,\dots,N\}$.
\vspace{-0.2cm}
%%%%%%%%%%%%%%%%%%%%%%%%%%%%%%%%%%%%%%%%%%%%%%%%%%%%%%%%%%%%%%%%%%
%%%%%%%%%%%%%%%%%%%%%%%%%%%%%%%%%%%%%%%%%%%%%%%%%%%%%%%%%%%%%%%%%%
\section{Problem Formulation}\label{sec:pre}
%\subsection{Notation}
First, we briefly summarize notation used in the paper and then  formally introduce the SSC problem. 

Bold capital letters denote matrices while 
bold lowercase letters represent vectors. For a matrix $\A$, $\A_{ij}$ denotes the $(i,j)$ entry of
$\A$, and $\a_j$ is the $j\ts{th}$ column of $\A$.
%, and $\A_{-j}$ is the matrix constructed by removing the $j\ts{th}$ column of $\A$. 
Additionally, $\A_S$ is the submatrix of $\A$ that contains the columns of $\A$ indexed by the set $S$. 
${\cal L}_S$ denotes the subspace spanned by the columns of $\A_S$. $\P_S^\bot=\I-\A_S \A_S^\dagger$ is the projection operator 
onto the orthogonal complement of ${\cal L}_S$ where $\A_S^\dagger=\left(\A_S^{\top}\A_S\right)^{-1}\A_S^{\top}$ 
denotes the Moore-Penrose pseudo-inverse of $\A_S$ and $\I$ is the identity matrix. Further, let $[n] = \{1,\dots,n\}$, $\mathbf{1}$ be the vector of all ones, and $\mathcal{U}(0,q)$ denote the uniform distribution on $[0,q]$.

%%%%%%%%%%%%%%%%%%%%%%%Subspace Clustering%%%%%%%%%%%%%%%%%%%%
%\subsection{Sparse Subspace Clustering}\label{sec:ssc}
The SSC problem is detailed next. Let $\{\y\}_{i=1}^N$ be a collection of data points in $\R^D$ and let $\Y = [\y_1,\dots,\y_N] \in \R^{D\times N}$ be the data matrix representing the data points. The data points are drawn from a union of n subspaces $\{S_i\}_{i=1}^n$ with dimensions $\{d_i\}_{i=1}^n$. Without a loss of generality, we assume that the columns of $\Y$, i.e., the data points, are normalized vectors with unit $\ell_2$ norm. The goal of subspace clustering is to partition $\{\y\}_{i=1}^N$ into $n$ groups so that the points that belong to the same subspace are assigned to the same cluster. In the sparse subspace clustering (SSC) framework \cite{elhamifar2009sparse}, one assumes that the data points satisfy the self-expressiveness property formally stated below.
\begin{definition}
\textit{A collection of data points $\{\y\}_{i=1}^N$ satisfies the self-expressiveness property if each data point has a linear representation in terms of the other points in the collection, i.e., there exist a representation matrix $\C$ such that}
\begin{equation}
\Y = \Y \C, \quad \mathrm{diag}(\C) = \mathbf{0}.
\end{equation}
\end{definition}
Notice that since each point in $S_i$ can be
written in terms of at most $d_i$ points in $S_i$, SSC aims to find a sparse subspace preserving 
$\C$ as formalized next.
\begin{definition}
\textit{A representation matrix $\C$ is subspace preserving if for all $j,l \in [N]$ and a 
subspace $S_i$} it holds that
\begin{equation}
\C_{lj} \neq 0 \quad \Longrightarrow \quad \y_j, \y_l \in S_i.
\end{equation}
\end{definition}
The task of finding a subspace preserving $\C$ leads to the optimization problem \cite{elhamifar2009sparse}
\begin{equation}\label{eq:prob}
\begin{aligned}
& \underset{\c_j}{\text{min}}
\quad \|\c_j\|_0
& \text{s.t.}\hspace{0.5cm}  \y_j = \Y\c_j, \quad \C_{jj} = 0,
\end{aligned}
\end{equation}
where $\c_j$ is the $j\ts{th}$ column of $\C$. Given a subspace
preserving solution $\C$, one constructs a similarity matrix $\W = |\C|+|\C|^\top$ for the data points. The graph normalized Laplacian of the similarity matrix $\W$ is then used as an input to a spectral clustering algorithm \cite{ng2001spectral} which in turn produces clustering assignments.
\vspace{-0.2cm}
%%%%%%%%%%%%%%%%%%%%%%%%%%%%%%%%%%%%%%%%%%%%%%%%%%%%%%%%%%%%%%%%%%
%%%%%%%%%%%%%%%%%%%%%%%%%%%%%%%%%%%%%%%%%%%%%%%%%%%%%%%%%%%%%%%%%%
\section{Accelerated OLS for Subspace Clustering}\label{sec:alg}
Here we resort to two approximation methods to find a near-optimal solution $S$ of \ref{eq:probf}. Our proposed algorithms are based on semidefinite and weak submodular optimization techniques that have recently shown superior performance in applications such as sensor selection\cite{joshi2009sensor}, graph sketching \cite{gama2016rethinking}, wireless sensor networks \cite{shamaiah2012greedy}, Kalman filtering \cite{ma}, and sparse signal recovery \cite{das2011submodular,hashemi2016sparse}. 
\vspace{-0.25cm}
\subsection{Sampling via SDP relaxation}
We first develop an SDP relaxation for problem \ref{eq:probf}. Our proposed scheme is motivated by the framework of \cite{joshi2009sensor} developed in the context of sensor scheduling. However, our focus is on sampling and reconstruction of graph signals which entails a different objective function, i.e., MSE.
Let $z_i \in \{0,1\}$ indicate whether the $i\ts{th}$ node of $\mathcal{N}$ is included in the sampling set $S$ and define $\z = [z_1,z_2,\dots,z_N]^\top$. Then, \ref{eq:covf} can alternatively be written as
%
\begin{equation}
\bar{\S}_z = \left(\P^{-1}+\sigma^{-2}\sum_{i=1}^n z_i\u_i\u_i^\top\right)^{-1}.
\end{equation}
%
Therefore, by relaxing the binary constraint $z_i \in \{0,1\}$ one can obtain a convex relaxation of  \ref{eq:probf},
%
\begin{equation}\label{eq:probf2}
\begin{aligned}
& \underset{\z}{\text{min}}
\quad \mathrm{Tr}\left(\bar{\S}_z\right)
& \text{s.t.}\quad 0\leq z_i \leq 1, \phantom{k} \sum_{i=1}^Nz_i \leq k.
\end{aligned}
\end{equation}
%
In order to obtain an SDP in standard form, let $\C$ be a positive semidefinite matrix such that $\C \succeq \bar{\S}_z$. 
Then, \ref{eq:probf2} is equivalent to 
%
\begin{equation}\label{eq:probf3}
\begin{aligned}
& \underset{\z,\C}{\text{min}}
\quad \mathrm{Tr}\left(\B\right)
& \text{s.t.}\quad 0\leq z_i \leq 1, \phantom{k} \sum_{i=1}^nz_i \leq k, \phantom{k} \C -\bar{\S}_z \succeq \mathbf{0}.
\end{aligned}
\end{equation}
%
The last constraint in \ref{eq:probf3}, i.e., $\C -\bar{\S}_z \succeq \mathbf{0}$, can be thought of as being the Schur complement \cite{horn2012matrix} of the block matrix
%
\begin{equation}
\B = \begin{bmatrix}
    \C& \I\\
    \I & \bar{\S}_z^{-1}
\end{bmatrix}.
\end{equation}
%
Note that the Schur complement of $\B$ is positive semidefinite if and only if $\B\succeq \mathbf{0}$ 
\cite{horn2012matrix}. Therefore, replacing the last constraint in \ref{eq:probf3} with the positive semidefiniteness 
constraint on $\B$ results in the following SDP relaxation:
%
\begin{equation}\label{eq:sdp}
\begin{aligned}
& \underset{\z,\B}{\text{min}}
\quad \mathrm{Tr}\left(\B\right)
& \text{s.t.}\quad 0\leq z_i \leq 1, \phantom{k} \sum_{i=1}^nz_i \leq k, \phantom{k} \B \succeq \mathbf{0}.
\end{aligned}
\end{equation}
%
An exact solution to \ref{eq:sdp} can be obtained by means of existing SDP solvers; see, e.g.,~\cite{arora2005fast,grant2008cvx}.  However, the solution $\hat{\z}$ contains real-valued entries and hence a rounding procedure is needed
to obtain a binary solution. Here, we propose to use the rounding procedure introduced in \cite{joshi2009sensor} and accordingly select the nodes of $\mathcal{N}$ corresponding to the $k$ $z_i$'s with largest values. The proposed SDP relaxation sampling scheme is summarized as Algorithm 1.
%=================================== ALGORITHM 1
\renewcommand\algorithmicdo{}	% removes "DO" from for loops
\begin{algorithm}[t]
\caption{SDP Relaxation for Graph Sampling}
\label{alg:sdp}
\begin{algorithmic}[1]
    \STATE \textbf{Input:}  $\P$, $\U$, $k$.
    \STATE \textbf{Output:} Subset $S\subseteq \mathcal{N} $ with $|S|=k$.
    \STATE Find $\z$, the minimizer of the SDP relaxation problem in \ref{eq:sdp} 
    \STATE Set $S$ to contain nodes corresponding to top $k$ entries of $\z$
	\RETURN $S$.
\end{algorithmic}
\end{algorithm}
\vspace{-0.25cm}
%===================================
%%%%%%%%%%%%%%%%%%%%%%%%%%%%%%%%%%%%%%%%%%%%%%%%%%%%%%%%%%
%%%%%%%%%%%%%%%%%%%%%%%%%%%%%%%%%%%%%%%%%%%%%%%%%%%%%%%%%%
\subsection{Sampling via a randomized greedy scheme}
%%%%%%%%%%%%%%%%%%%%%%%%%%%%%%%%%%%%%%%%%%%%%%%%%%%%%%%%%%
%%%%%%%%%%%%%%%%%%%%%%%%%%%%%%%%%%%%%%%%%%%%%%%%%%%%%%%%%%
The computational complexity of the SDP approach developed in Section 3.1 might be challenging in applications 
dealing with large graphs. Hence, we now propose an iterative randomized greedy algorithm for the task of sampling 
and reconstruction of graph signals by formulating \ref{eq:probf} as the problem of maximizing a monotone weak 
submodular set function. First, we define the notion of monotonicity, submodularity, and curvature that will be useful 
in the subsequent analysis.

A set function $f:2^X\rightarrow \mathbb{R}$ is submodular if
%
\begin{equation}
f(S\cup \{j\})-f(S) \geq f(T\cup \{j\})-f(T)
\end{equation}
%
for all subsets $S\subseteq T\subset X$ and $j\in X\backslash T$. The term $f_j(S)=f(S\cup \{j\})-f(S)$ is the marginal value of adding element $j$ to set $S$. Moreover, the function is monotone if $f(S)\leq f(T)$ for all $S\subseteq T\subseteq X$. 

The concept of submodularity can be generalized by the notion of curvature or submodularity ratio \cite{das2011submodular} that quantifies how close a set function is to being submodular. Specifically, the maximum element-wise curvature of a monotone non-decreasing function $f$ is defined as
%
\begin{equation}
{\cal C}_{\max}=\max_{1\le l<n}{\max_{(S,T,i)\in \mathcal{X}_l}{f_i(T)\slash f_i(S)}},
\end{equation}
%
with $\mathcal{X}_l = \{(S,T,i)|S \subset T \subset X, i\in X \backslash T, |T\backslash S|=l,|X|=n\}$. Note that a set function is submodular if and only if ${\cal C}_{\max}~\le~1$. Set functions with ${\cal C}_{\max} > 1$ are called weak or approximate submodular functions \cite{das2011submodular}.
%%%%%%%%%%%%%%%%%%%%%%%%%%%%%%%%%%%%%%%%%%%%%%%%%%%%%%%%%%
%%%%%%%%%%%%%%%%%%%%%%%%%%%%%%%%%%%%%%%%%%%%%%%%%%%%%%%%%%

Next, similar to \cite{chamon2017greedy}, we formulate \ref{eq:probf} as a  set function maximization task. Let $f(S) = \mathrm{Tr}(\P-\bar{\S}_S)$. Then,  \ref{eq:probf} can equivalently be written as
%
\begin{equation}\label{eq:probsub}
\begin{aligned}
& \underset{S}{\text{max}}
\quad f(S)
& \text{s.t.}\quad S\subseteq\mathcal{N},\quad |S| \leq k.
\end{aligned}
\end{equation}

In Proposition \oldref{thm:p} below, by applying the matrix inversion lemma 
\cite{bellman1997introduction} we establish that $f(S)$ is monotone and weakly submodular. 
Moreover, we derive an efficient recursion to find the marginal gain of adding a new node 
to the sampling set $S$. 
%
\begin{proposition}\label{thm:p}
\textit{$f(S) = \mathrm{Tr}(\P-\bar{\S}_S)$ is a weak submodular, monotonically increasing set function, $f(\emptyset)=0$, and for all $j \in \mathcal{N}\backslash S$
%
\begin{equation}\label{eq:mg}
f(S\cup \{j\})-f(S) = \frac{\u_j^\top\bar{\S}_S^{2}\u_j}{\sigma^{2}+\u_j^\top\bar{\S}_S\u_j},\: \text{ and }
\end{equation}
%
\begin{equation}\label{eq:upf}
\bar{\S}_{S \cup\{j\}} = \bar{\S}_S-\frac{\bar{\S}_{S}\u_{j}\u_{j}^\top\bar{\S}_{S}}{\sigma^2+\u_{j}^\top\bar{\S}_{S}\u_{j}}.
\end{equation}}
%
\end{proposition}
%
Proposition \oldref{thm:p} enables efficient construction of the sampling set in an iterative fashion. To further reduce the computational cost, we propose a randomized greedy algorithm that performs the task of sampling set selection in the following way.  Starting with $S = \emptyset$, at iteration $(i+1)$ of the algorithm, a subset $R$ of size $s$ is sampled uniformly at random and without replacement from $\mathcal{N} \backslash S$. 
The marginal gain of each node in $R$ is found using \ref{eq:mg}, and the one corresponding to the highest marginal gain is added to $S$. Then, the algorithm employs the recursive relation \ref{eq:upf} to update $\bar{\S}_S$ for the subsequent iteration. This procedure is repeated until some stopping criteria, such as condition on the cardinality of $S$ is met. Regarding $s$, we follow the suggestion in \cite{mirzasoleiman2014lazier} and set $s=\frac{N}{k}\log\frac{1}{\epsilon}$, where $e^{-k}\leq \epsilon<1$ is a predetermined parameter that controls trade-off between the computational cost and MSE of the reconstructed signal; randomized greedy algorithm with smaller $\epsilon$ produces sampling solutions with lower MSE while the one with larger $\epsilon$ requires lower computational costs. Note that if $\epsilon = e^{-k}$, the randomized greedy algorithm in each iteration considers all the available nodes and hence  matches the greedy scheme proposed in \cite{chamon2017greedy}. However, as we illustrate in our simulation studies, the proposed randomized greedy algorithm is significantly faster than the method in \cite{chamon2017greedy} for larger $\epsilon$ while returning essentially the same sampling solution. The randomized greedy algorithm is formalized as Algorithm 2.

%=================================== ALGORITHM 2
\renewcommand\algorithmicdo{}	% removes "DO" from for loops
\begin{algorithm}[t]
\caption{Randomized Greedy Algorithm for Graph Sampling}
\label{alg:greedy}
\begin{algorithmic}[1]
    \STATE \textbf{Input:}  $\P$, $\U$, $k$, $\epsilon$.
    \STATE \textbf{Output:} Subset $S\subseteq \mathcal{N} $ with $|S|=k$.
    \STATE Initialize $S =  \emptyset$, $\bar{\S}_{S}=\P$.
	\WHILE{$|S|<k$}\vspace{0.1cm}
			\STATE Choose $R$ by sampling $s=\frac{N}{k}\log{(1/\epsilon)}$ indices uniformly at random from $\mathcal{N}\backslash S$
            \STATE $j_s = \argmax_{j\in R} \frac{\u_j^\top\bar{\S}_S^{2}\u_j}{\sigma^{2}+\u_j^\top\bar{\S}_S\u_j}$\vspace{0.1cm}
            \STATE $\bar{\S}_{S \cup\{j_s\}} = \bar{\S}_S-\frac{\bar{\S}_{S}\u_{j}\u_{j}^\top\bar{\S}_{S}}{\sigma^2+\u_{j}^\top\bar{\S}_{S}\u_{j}}$\vspace{0.1cm}
            \STATE Set $S \leftarrow S\cup \{j_s\}$\vspace{0.1cm}
	\ENDWHILE
	\RETURN $S$.
\end{algorithmic}
\end{algorithm}
%===================================

\noindent{\textbf{Performance guarantees.}} Here we analyze performance of the randomized greedy algorithm. First, Theorem \oldref{thm:exp} below states that if $f(S)$ is characterized by a bounded maximum element-wise curvature, Algorithm 2 returns a sampling subset yielding an MSE that is on average within a multiplicative factor of the MSE associated with the optimal sampling set.
%
\begin{theorem}\label{thm:exp}
\textit{Let $\mathcal{C}_{max}$ be the maximum element-wise curvature of $f(S) = \mathrm{Tr}(\P-\bar{\S}_S)$, the objective function in problem \ref{eq:probsub}. Let $\alpha =(1-e^{-\frac{1}{c}}-\frac{\epsilon^\beta}{c})$, where $c=\max\{1,{\cal C}_{\max}\}$, $e^{-k}\leq\epsilon<1$, and $\beta = 1+\max\{0,\frac{s}{2N}-\frac{1}{2(N-s)}\}$. Let $S_{rg}$ be the sampling set returned by the randomized greedy algorithm and let $O$ denote the optimal solution of \ref{eq:probf}. Then,}
%
\begin{equation}\label{eq:expbound}
\E\left[\mathrm{Tr}(\bar{\S}_{S_{rg}})\right]\leq \alpha \mathrm{Tr}(\bar{\S}_{O}) + (1-\alpha) \mathrm{Tr}(\P).
\end{equation}
%
\end{theorem}
%
The proof of Theorem \oldref{thm:exp} relies on the argument that if $s = \frac{N}{k}\log\frac{1}{\epsilon}$, then with high probability the random subset $R$ in each iteration of Algorithm 2 contains at least one node from $O$.

Next, we study the performance of the randomized greedy algorithm using the tools of probably approximately correct (PAC) learning theory \cite{valiant1984theory,valiant2013probably}. The randomization of Algorithm 2 can be interpreted as approximating the marginal gains of the nodes selected by the greedy scheme proposed in \cite{chamon2017greedy}. More specifically, for the $i^{th}$ iteration it holds that $f_{j_{rg}}(S_{rg}) = \eta_i f_{j_{g}}(S_{g})$, where subscripts $rg$ and $g$ refer to the sampling sets and nodes selected by the randomized greedy (Algorithm 2) and greedy algorithm in \cite{chamon2017greedy}, respectively, and $0<\eta_i\leq 1$ for all $i \in [k]$ are random variables. 
In view of this argument and by employing the Bernstein inequality \cite{tropp2015introduction}, we obtain 
Theorem \oldref{thm:pac} which states that the randomized greedy algorithm selects a near-optimal sampling 
set with high probability.
%
\begin{theorem}\label{thm:pac}
\textit{Instate the notation and hypotheses of Theorem \oldref{thm:exp}. Assume $\{\eta_i\}_{i=1}^k$ are independent and let $C=0.088$. Then, with probability at least $1-e^{-Ck}$ it holds that}
%
\begin{equation}\label{eq:pacbound}
\mathrm{Tr}(\bar{\S}_{S_{rg}})\leq (1-e^{-\frac{1}{2c}}) \mathrm{Tr}(\bar{\S}_{O}) + e^{-\frac{1}{2c}} \mathrm{Tr}(\P).
\end{equation}
%
\end{theorem}
%
Indeed, in simulation studies (see Section \oldref{sec:sim}) we empirically verify the results of Theorems \oldref{thm:exp} and \oldref{thm:pac} and illustrate that Algorithm 2 performs favorably compared to the competing schemes both on average and for each individual sampling task. Before moving on to these numerical studies, 
in Theorem \oldref{thm:curv} we show that the maximum element-wise curvature of $f(S) = \mathrm{Tr}(\P-\bar{\S}_S)$ is bounded, even for non-stationary graph signals. 
%
\begin{theorem}\label{thm:curv}
\textit{Let $\mathcal{C}_{max}$ be the maximum element-wise curvature of $f(S) = \mathrm{Tr}(\P-\bar{\S}_S)$. Then, it holds that}
\begin{equation}\label{eq:curvbound}
\mathcal{C}_{\max} \leq \frac{\lambda_{\max}^2(\P)}{\lambda_{\min}^2(\P)}\left(1+\frac{\lambda_{\max}(\P)}{\sigma^2}\right)^3.
\end{equation}
\end{theorem}
An implication of Theorem \oldref{thm:curv} is a generalization of a result in \cite{chamon2017greedy} for stationary signals. There, it has been shown that if $\x$ is stationary and $\P = \sigma_\x^2\I_N$ for some $\sigma_\x^2>0$, then the curvature of the MSE objective is bounded. However, Theorem \oldref{thm:curv} holds even in the scenarios where the signal is non-stationary and $\P$ is non-diagonal.
\vspace{-0.2cm}
%%%%%%%%%%%%%%%%%%%%%%%%%%%%%%%%%%%%%%%%%%%%%%%%%%%%%%%%%%%%%%%%%%
%%%%%%%%%%%%%%%%%%%%%%%%%%%%%%%%%%%%%%%%%%%%%%%%%%%%%%%%%%%%%%%%%%
\section{Simulation Results}\label{sec:sim}
We study the recovery of simulated noisy signals supported on synthetic and real-world graphs to assess performance of the proposed sampling algorithms in terms of MSE and running time. To this end, we first 
consider an undirected Erd\H{o}s-R\'enyi random graph $\mathcal{G}$ of size $N=100$ and edge probability 0.2 \cite{newman2010networks}. Bandlimited graph signals $\x = \U\bar{\x}_K$ are generated by taking $\mathbf{U}$ as the first $k=30$ eigenvectors of the graph adjacency matrix. The non-zero frequency components $\bar{\x}_K$ are drawn from a zero-mean, multivariate Gaussian distribution with covariance matrix $\P$ which is selected uniformly at random from the set of positive semi-definite (PSD) matrices.  Zero-mean Gaussian noise $\n$ with covariance $\sigma^2 = 10^{-2}\I_N$ is added to $\x$. Algorithms \oldref{alg:sdp} and \oldref{alg:greedy} are run to recover the signal for different sampling set sizes. We compare the MSE performance of the proposed schemes with the state-of-the-art greedy algorithm \cite{chamon2017greedy} and the random sampling approaches in \cite{chen2016signal}. For the randomized greedy algorithm we use $\epsilon = 0.1$ and $\epsilon = 0.01$.  Fig. \oldref{fig:rand} (top) depicts the MSE versus $k$ (sample size), where the results are obtained 
by averaging over $100$ Monte-Carlo simulations. As the figure indicates, Algorithms \oldref{alg:sdp} and \oldref{alg:greedy} outperform the random sampling schemes of \cite{chen2016signal} and perform nearly as well as the greedy sampling algorithm \cite{chamon2017greedy}. While not shown here for the sake of clarity of the presentation, similar patterns were also observed for other workhorse random graphs, e.g., preferential attachment and Barb\'asi--Albert models \cite{newman2010networks}.
%\begin{figure}[t]
%\vspace{-0.3cm}
%\centering
%    \includegraphics[width=0.45\textwidth]{Figures/mserand.eps}
%    \caption{Erd\H{o}s-R\'enyi graph. MSE comparison of different sampling schemes. }
%\label{fig:rand}
%\vspace{-0.1cm}
%\end{figure}

\begin{figure}[t]
	\vspace{-0.3cm}
	\begin{minipage}[b]{\linewidth}
		\centering
		\includegraphics[width=0.8\textwidth]{Figures/mserand.eps}
		%\centerline{(a)}\medskip
	\end{minipage}
	%
	\begin{minipage}[b]{\linewidth}
		\centering
		\includegraphics[width=0.8\textwidth]{Figures/hist.eps}
		%\centerline{(b)}\medskip
	\end{minipage}
	%	
	\vspace{-0.5cm}
	\caption{Erd\H{o}s-R\'enyi graph. Comparison of different schemes in terms of (top) MSE as a function of the size of the sampling set; and (bottom) histogram of MSE values for $100$ realizations and fixed sampling set size. }
	\label{fig:rand}
	\vspace{-0.3cm}
\end{figure}

Next, we study the performance of the proposed schemes for each individual sampling tasks (each Monte-Carlo realizations),  for the setting where $N=10$ and $k=4$. 
Bandlimited graph signals are generated as before except that this time we take $\mathbf{U}$ as the first $4$ eigenvectors of the adjacency matrix. Fig. \oldref{fig:rand} (bottom) depicts superimposed MSE histograms of Algorithms \oldref{alg:sdp} and \oldref{alg:greedy} as well as the greedy sampling scheme \cite{chamon2017greedy} for 100 realizations per method and fixed $|S|=4$. As the figure illustrates, the proposed SDP relaxation and randomized greedy schemes perform well and are comparable with the greedy approach.
%\begin{figure}[t]
%\vspace{-0.3cm}
%\centering
%    \includegraphics[width=0.45\textwidth]{Figures/hist.eps}
%    \caption{Erd\H{o}s-R\'enyi graph. MSE histograms of different sampling schemes.}
%\label{fig:hist}
%\vspace{-0.1cm}
%\end{figure}

Finally, we test Algorithm \oldref{alg:greedy} on the Minnesota road network\footnote{https://sparse.tamu.edu/Gleich/minnesota} with $N=2642$ nodes in order to showcase scalability of the proposed graph sampling method. To that end, Bandlimited graph signals are generated by taking the first $k=600$ eigenvectors of the graph Laplacian matrix, where the non-zero frequency components are drawn from a zero-mean, multivariate Gaussian distribution with randomly chosen PSD covariance matrix $\P$. The signals are corrupted with additive white Gaussian noise with $\sigma^{2}=10^{-2} \mathbf{I}_{N}$. 
%Then, performance of Algorithm \oldref{alg:greedy} and algorithm in \cite{chamon2017greedy} are averaged over $10$ Monte-Carlo simulations. 
As expected, Figs.~\oldref{fig:min} (top) and (bottom) depict trends of decreasing MSE and increasing running time versus $|S|$, respectively. The results are averaged over $1000$ Monte-Carlo simulations run on a commercial laptop with an Intel Core i$7$ processor at $3.1$ GHz. Remarkably, the proposed randomized greedy procedure achieves an order-of-magnitude speedup over the state-of-the-art algorithm in \cite{chamon2017greedy} while showing only a marginal degradation in the MSE performance.
%
\begin{figure}[t]
	\vspace{-0.3cm}
	\begin{minipage}[b]{\linewidth}
		\centering
		\includegraphics[width=0.8\textwidth]{Figures/mseMin.eps}
		%\centerline{(a)}\medskip
	\end{minipage}
	%
	\begin{minipage}[b]{\linewidth}
		\centering
		\includegraphics[width=0.8\textwidth]{Figures/timeMin.eps}
		%\centerline{(b)}\medskip
	\end{minipage}
	%
	\vspace{-0.5cm}	
	\caption{Minnesota road network. (top) MSE and (bottom) running time comparison of different sampling schemes as a function of the size of the sampling set.}
	\label{fig:min}
	\vspace{-0.3cm}
\end{figure}
%%%%%%%%%%%%%%%%%%%%%%%%%%%%%%%%%%%%%%%%%%%%%%%%%%%%%%%%%%%%%%%%%%%%%%%%%%%%
%%%%%%%%%%%%%%%%%%%%%%%%%%%%%%%%%%%%%%%%%%%%%%%%%%%%%%%%%%%%%%%%%%%%%%%%%%%%
%%%%%%%%%%%%%%%%%%%%%%%%%%%%%%%%%%%%%%%%%%%%%%%%%%%%%%%%%%%%%%%%%%%%%%%%%%%%
%\begin{figure}[t]
%\vspace{-0.3cm}
%\centering
%    \includegraphics[width=0.45\textwidth]{Figures/mseMin.eps}
%    \caption{Minnesota road network. MSE comparison of different sampling schemes.}
%\label{fig:minmse}
%\vspace{-0.1cm}
%\end{figure}
%\begin{figure}[t]
%\vspace{-0.3cm}
%\centering
%    \includegraphics[width=0.45\textwidth]{Figures/timeMin.eps}
%    \caption{Minnesota road network. Running time comparison of different sampling schemes.}
%\label{fig:mintime}
%\vspace{-0.4cm}
%\end{figure}
%%%%%%%%%%%%%%%%%%%%%%%%%%%%%%%%%%%%%%%%%%%%%%%%%%%%%%%%%%%%%%%%%%%%%%%%%%%%
%%%%%%%%%%%%%%%%%%%%%%%%%%%%%%%%%%%%%%%%%%%%%%%%%%%%%%%%%%%%%%%%%%%%%%%%%%%%
%%%%%%%%%%%%%%%%%%%%%%%%%%%%%%%%%%%%%%%%%%%%%%%%%%%%%%%%%%%%%%%%%%%%%%%%%%%%
% \begin{figure*}[!htb]
% 	\minipage{0.32\textwidth}
% 	\includegraphics[width=\linewidth]{Figures/MSE_erdos.eps}
% 	\caption{blah}\label{fig:MSE_erdos_avg}
% 	\endminipage\hfill
% 	\minipage{0.32\textwidth}
% 	\includegraphics[width=\linewidth]{Figures/MSE_erdos.eps}
% 	\caption{blah}\label{fig:blah}
% 	\endminipage\hfill
% 	\minipage{0.32\textwidth}%
% 	\includegraphics[width=\linewidth]{Figures/MSE_erdos.eps}
% 	\caption{blah}\label{fig:MSE_erdos}
% 	\endminipage
% \end{figure*}
% \begin{figure}[h]
% 	\centering    
% 	\includegraphics[width=1\linewidth]{Figures/minnesota_MSE.eps}
% 	\caption{....}
% 	\label{fig:minnesota_MSE}
% 	\vspace{-0.5cm}
% \end{figure} 
% \begin{figure}[h]
% 	\centering    
% 	\includegraphics[width=1\linewidth]{Figures/minnesota_run_time.eps}
% 	\caption{....}
% 	\label{fig:minnesota_run_time}
% 	\vspace{-0.5cm}
% \end{figure} 
\vspace{-0.2cm}
%%%%%%%%%%%%%%%%%%%%%%%%%%%%%%%%%%%%%%%%%%%%%%%%%%%%%%%%%%%%%%%%%%
%%%%%%%%%%%%%%%%%%%%%%%%%%%%%%%%%%%%%%%%%%%%%%%%%%%%%%%%%%%%%%%%%%
\section{Conclusion} \label{sec:concl}
In this paper, we proposed a novel algorithm for clustering high dimensional data lying on a union of subspaces. The proposed algorithm, referred to as accelerated sparse subspace clustering (ASSC), employs a computationally efficient variant of the orthogonal least-squares algorithm to construct a similarity matrix under the assumption that each data point can be written as a sparse linear combination of other data points in the subspaces. ASSC then performs spectral clustering on the similarity matrix to find the clustering solution. We analyzed the performance of the proposed scheme and provided a theorem stating that if the subspaces are independent, the similarity matrix generated by ASSC is subspace-preserving. In simulations, we demonstrated that the proposed algorithm is orders of magnitudes faster than the BP-based SSC scheme \cite{elhamifar2009sparse,elhamifar2013sparse} and essentially delivers the same or better clustering solution. The results also show that ASSC outperforms the state-of-the-art OMP-based method \cite{dyer2013greedy,you2015sparse}, especially in scenarios where the data points across different subspaces are similar. 

As part of the future work, it would be of interest to extend our results and analyze performance of ASSC in the general setting where the subspaces are arbitrary and not necessarily independent. Moreover, it would be beneficial to develop distributed implementations for further acceleration of ASSC.

%%%%%%%%%%%%%%%%%%%%%%%%%%%%%%%%%%%%%%%%%%%%%%%%%%%%%%%%%%%%%%%%%%
%%%%%%%%%%%%%%%%%%%%%%%%%%%%%%%%%%%%%%%%%%%%%%%%%%%%%%%%%%%%%%%%%%
\clearpage
\bibliographystyle{ieeetr}\small
\bibliography{refs}
\end{document}
