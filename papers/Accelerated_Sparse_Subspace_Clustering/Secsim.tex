%%%%%%%%%%%%%%%%%%%%%%%%%%%%%%%%%%%%%%%%%%%%%%%%%%%%%%%%%%%%%%%%%%%%%%%%
%%%%%%%%%%%%%%%%%%%%%%%%%%%%%%%%%%%%%%%%%%%%%%%%%%%%%%%%%%%%%%%%%%%%%%%
\begin{figure*}[]
\begin{subfigure}[]{0.24\textwidth}
  \centering
    \includegraphics[width=1\textwidth]{spr-t0}\quad\caption{\footnotesize Subspace preserving rate}
    \end{subfigure}
    \begin{subfigure}[]{.24\textwidth}
  \centering
    \includegraphics[width=1\textwidth]{spe-t0}\quad\caption{\footnotesize  Subspace preserving error}
    \end{subfigure}
    \begin{subfigure}[]{.24\textwidth}
  \centering
    \includegraphics[width=1\textwidth]{ca-t0}\quad\caption{\footnotesize  Clustering accuracy}
\end{subfigure}
    \begin{subfigure}[]{.24\textwidth}
  \centering
    \includegraphics[width=1\textwidth]{rt-t0}\quad\caption{\footnotesize Running time (sec)}
\end{subfigure}
\caption{\it Performance comparison of ASSC, SSC-OMP \cite{dyer2013greedy,you2015sparse}, and SSC-BP \cite{elhamifar2009sparse,elhamifar2013sparse} {\color{black}{on synthetic data with no perturbation}}. The points are drawn from $5$ subspaces of dimension $6$ in ambient dimension $9$. Each subspace contains the same number of points and the overall number of points is varied from $250$ to $5000$.}
\vspace{-0.1cm}
\end{figure*}
%%%%%%%%%%%%%%%%%%%%%%%%%%%%%%%%%%%%%%%%%%%%%%%%%%%%%%%%%%%%%%%%%%%%%%%%
%%%%%%%%%%%%%%%%%%%%%%%%%%%%%%%%%%%%%%%%%%%%%%%%%%%%%%%%%%%%%%%%%%%%%%%
\begin{figure*}[]
\begin{subfigure}[]{0.24\textwidth}
  \centering
    \includegraphics[width=1\textwidth]{spr-t1}\quad\caption{\footnotesize Subspace preserving rate}
    \end{subfigure}
\begin{subfigure}[]{0.24\textwidth}
  \centering
    \includegraphics[width=1\textwidth]{spe-t1}\quad\caption{\footnotesize  Subspace preserving error}
    \end{subfigure}
    \begin{subfigure}[]{.24\textwidth}
  \centering
    \includegraphics[width=1\textwidth]{ca-t1}\quad\caption{\footnotesize Clustering accuracy}
    \end{subfigure}
    \begin{subfigure}[]{.24\textwidth}
  \centering
    \includegraphics[width=1\textwidth]{rt-t1}\quad\caption{\footnotesize Running time (sec)}
\end{subfigure}
\caption{\it Performance comparison of ASSC, SSC-OMP \cite{dyer2013greedy,you2015sparse}, and SSC-BP \cite{elhamifar2009sparse,elhamifar2013sparse} {\color{black}{on synthetic data with perturbation terms $Q\sim \mathcal{U}(0,1)$}}. The  points are drawn from $5$ subspaces of dimension $6$ in ambient dimension $9$. Each subspace contains the same number of points and the overall number of points is varied from $250$ to $5000$.}
\vspace{-0.3cm}
\end{figure*}
%%%%%%%%%%%%%%%%%%%%%%%%%%%%%%%%%%%%%%%%%%%%%%%%%%%%%%%%%%%%%%%%%%%%%%%%
%%%%%%%%%%%%%%%%%%%%%%%%%%%%%%%%%%%%%%%%%%%%%%%%%%%%%%%%%%%%%%%%%%%%%%%
To evaluate performance of the ASSC algorithm, we compare it to that of the BP-based \cite{elhamifar2009sparse,elhamifar2013sparse} and OMP-based \cite{dyer2013greedy,you2015sparse} SSC schemes, referred to as SSC-BP and SSC-OMP, respectively. For SSC-BP, two implementations based on ADMM and interior point methods are available by the authors of \cite{elhamifar2009sparse,elhamifar2013sparse}. The interior point implementation of SSC-BP is more accurate than the ADMM implementation while the ADMM implementation tends to produce sup-optimal solution in a few iterations. However, the interior point implementation is very slow even for relatively small problems. Therefore, in our simulation studies we use the ADMM implementation of SSC-BP that is provided by the authors of \cite{elhamifar2009sparse,elhamifar2013sparse}. {\color{black}{Our scheme is tested for $L = 1$ and $L=2$}}. We consider the following two scenarios: (1) A random model where the subspaces are with high probability near-independent; and (2) The setting where we used hybrid dictionaries \cite{soussen2013joint} to generate similar data points across different subspaces which in turn implies the independence assumption no longer holds. In both scenarios, we randomly generate $n = 5$ subspaces, each of dimension $d = 6$, in an ambient space of dimension $D = 9$. Each subspace contains $N_i$ sample points  where we vary $N_i$ from $50$ to $1000$; therefore, the total number of data points, $N = \sum_{i=1}^n N_i$, is varied from $250$ to $5000$. The results are averaged over $20$ independent instances. For scenario (1), we generate data points by uniformly sampling from the unit sphere. For the second scenario, after sampling a data point, we add a perturbation term $Q\mathbf{1}_D$ where $Q\sim \mathcal{U}(0,1)$. 

In addition to comparing the algorithms in terms of their clustering accuracy and running time, we use the following metrics defined in \cite{elhamifar2009sparse,elhamifar2013sparse} that quantify the subspace preserving property of the representation matrix $\C$ returned by each algorithm: \textit{Subspace preserving rate} defined as the fraction of points whose representations are  subspace-preserving, \textit{Subspace preserving error} {\color{black}{defined as the fraction of $\ell_1$ norms of the representation coefficients associated with points from other subspaces, i.e., $\frac{1}{N}\sum_{j}{(\sum_{i\in O}{|\C_{ij}}|\slash \|\c_j\|_1)}$}} where $O$ represents the set of data points from other subspaces.
% \begin{itemize}
% \item \textit{Subspace preserving rate:} The fraction of points whose representations are 
% subspace-preserving.
% \item \textit{Subspace preserving error:} {\color{blue}{The fraction of $\ell_1$ norms of the representation 
% coefficients associated with points from other subspaces, i.e., 
% $\frac{1}{N}\sum_{j}{(1-\sum_{i\in O}{\C_{ij}})\slash \|\c_j\|_1}$}} where $O$ represents the 
% set of data points from other subspaces.
% \end{itemize}

The results for the scenario (1) and (2) are illustrated in Fig. 1 and Fig. 2, respectively. As we see in Fig. 1, ASSC is nearly as fast as SSC-OMP and orders of magnitude faster than SSC-BP while ASSC achieves better subspace preserving rate, subspace preserving error, and clustering accuracy compared to competing schemes. Regarding the second scenario, we observe that the performance of SSC-OMP is severely deteriorated while ASSC still outperforms both SSC-BP and SSC-OMP in terms of accuracy. Further, similar to the first scenario, running time of ASSC is similar to that of SSC-OMP while both methods are much faster that SSC-BP. Overall as Fig. 1 and Fig. 2 illustrate, ASSC algorithm, especially with $L=2$, is superior to other schemes and is essentially as fast as the SSC-OMP method.

