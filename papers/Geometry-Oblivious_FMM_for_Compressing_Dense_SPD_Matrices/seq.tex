% To transform the high level discussion of near-far pruning into algorithms, 
% we separate the whole
% method into two phases: (1) an one-time \textbf{compression}
% (\algref{a:compress}),
% and (2) an \textbf{evaluation} (\algref{a:evaluate}) for each $w$ instance.
% These algorithms are described in terms of binary tree traversals: 
% postorder \textbf{POST}, preorder \textbf{PRE},
% any order \textbf{ANY} and \textbf{LEAF} (only traverse leaf nodes).
% The computation that occurs in each tree
% node ($\alpha$ or $\beta$) is a task listed in 
% \tabref{tab:tasks}.
% 
% \begin{algorithm}[!t]
% \caption{{} \texttt{Compress}($K$)}
% \begin{algorithmic}
%   \STATE \texttt{\bf for each} randomize projection tree
%   \STATE \gap \texttt{\bf PRE} \texttt{SPLI($\alpha$)} \# random projection tree 
%   \STATE \gap \texttt{\bf LEAF} \texttt{ANN($\alpha$)} \#\textbf{RAW} on \texttt{SPLI}
%   \STATE \texttt{\bf PRE} \texttt{SPLI($\alpha$)} \# metric ball tree 
%   \STATE \texttt{NearNear()} \#\textbf{RAW} on \#\texttt{SPLI} 
%   \STATE \texttt{\bf ANY} \texttt{Kba($\alpha$)} \#\textbf{RAW} on \texttt{NearNear}
%   \STATE \texttt{\bf LEAF} \texttt{NearFar($\beta$,root)} \#\textbf{RAW} on \texttt{NearNear}
%   \STATE \texttt{\bf POST} \texttt{FarFar($\alpha$)} \#\textbf{RAW} on \texttt{NearFar}
%   \STATE \texttt{\bf POST} \texttt{SKEL($\alpha$)} \#\textbf{RAW} on \texttt{SPLI}
%   \STATE \texttt{\bf ANY} \texttt{COEF($\alpha$)} \#\textbf{RAW} on \texttt{SKEL}
%   \STATE \texttt{\bf ANY} \texttt{SKba($\alpha$)} \#\textbf{RAW} \texttt{FarFar}
%   and \texttt{SKEL}
% \end{algorithmic}
% \label{a:compress}
% \end{algorithm}
% 
% 
% \begin{table}[!t]
% \centering
% {
% \begin{tabular}{|rlr|}
% \hline 
% Task & Operations & \texttt{FLOPS} \\
% \hline 
% \texttt{SPLI($\alpha$)} & split $\alpha$ into $\lc$ and $\rc$ \algref{a:split} & $\lvert \alpha \rvert$ \\
% \hline 
% \texttt{ANN($\alpha$)} & update $\MA{N}_{\alpha}$ with \texttt{KNN($K_{\alpha\alpha}$)} & $dm^2$ \\ 
% \hline 
% \texttt{SKEL($\alpha$)} & $\sk{\alpha}$ in \algref{a:skeletonize} & $2s^3+2m^3$ \\ 
% \hline 
% \texttt{COEF($\alpha$)} & $P_{\sk{\alpha} \alpha}$ and $P_{\sk{\alpha} [\sk{\lc} \sk{\rc}]}$ in \algref{a:skeletonize} & $s^3$ \\ 
% \hline 
% \texttt{N2S($\alpha$)} & \texttt{\bf if} $\alpha$ is leaf \texttt{\bf then} $\sk{w_{\alpha}} = P_{\sk{\alpha}\alpha}w_{\alpha}$ & $2msr$ \\
%              & \texttt{\bf else} $\sk{w_{\alpha}} =
% P_{\sk{\alpha}[\sk{\lc}\sk{\rc}]}[ \sk{w_{\lc}}; \sk{w_{\rc}}]$ & $2s^2r$\\
% \hline
% \texttt{SKba($\beta$)}  & $\forall\alpha\in Far(\beta)$, $K_{\sk{\beta}\sk{\alpha}} = K(\sk{\beta}, \sk{\alpha})$ & $ds^2\lvert Far(\beta) \rvert$ \\ 
% \hline 
% \texttt{S2S($\beta$)} & $\sk{u_{\beta}} =
%   \sum_{\alpha\in Far(\beta)} K_{\sk{\beta}\sk{\alpha}}\sk{w_{\alpha}}$ & $2s^2r\lvert Far(\beta) \rvert$\\ 
% \hline 
% \texttt{S2N($\beta$)} & \texttt{\bf if} $\alpha$ is leaf \texttt{\bf then} $u_{\beta} = P_{\sk{\beta}\beta}^{T}\sk{u_{\beta}}$ & $2msr$ \\ 
%                       & \texttt{\bf else} $[\sk{u_{\lc}};\sk{u_{\rc}}] += P_{\sk{\beta}[\sk{\lc}\sk{\rc}]}^{T}\sk{u_{\beta}}$ & $2s^2r$ \\
% \hline
% \texttt{Kba($\beta$)}  & $\forall\alpha\in Near(\beta)$, $K_{\beta\alpha} = K(\beta, \alpha)$ & $dm^2\lvert Near(\beta) \rvert$ \\ 
% \hline  
% \texttt{L2L($\beta$)} & $u_{\beta} += \sum_{\alpha\in Near(\beta)}K_{\beta\alpha}w_{\alpha}$ & $2m^2r\lvert Near(\beta) \rvert$\\ 
% \hline 
% \end{tabular}
% }
% \caption{Tasks and their costs in \texttt{FLOPS}.
%   \texttt{SPLI} (tree splitting), \texttt{ANN} (all nearest-neighbor), \texttt{SKEL} (skeletonization), 
%   \texttt{COEF} (interpolation) \texttt{SKba} and 
%   \texttt{Kba} (caching submatrices) appear in the setup phase.
%   Interactions \texttt{N2S} (nodes to skeletons), \texttt{S2S} (skeletons to skeletons)
%   , \texttt{S2N} (skeletons to nodes) and \texttt{L2L} (leaves to leaves) appear in 
% the evaluation phase.}
% \label{tab:tasks}
% \end{table} 
% 
% 
% \textbf{Compression.}
% In this paper we apply a black-box geometry-oblivious heuristic
% to define \emph{distance} between pairs of rows (and columns).
% This \emph{distance} allows us to define neighbors $\MA{N}$ for
% each index, and partition $K$ with a metric ball tree~\cite{}.
% Given the capability of accessing any $K_{ij}$ element, 
% we first explain how such \emph{distance} is defined by only using
% raw elements from $K$; then we present the compression algorithm,
% which follows~\cite{} but in the flavor of \emph{out-of-order} using
% task scheduling.
% 
% 
% %\begin{enumerate}
% %\item hierarchical binary partitioning of the input matrix
% %\item a {\it distance} metric for neighbor sampling
% %\end{enumerate}
% %%Given $N^2$ entries in a dense system matrix it seems to be unavoidable to construct algorithms of complexity less than $\mathcal{O}(N^2)$; for several application problems [insert cites], however, low rank approximations can be applied to some blocks in the matrix. 
%For several application problems a suitable hierarchical subdivision structure can be found by a spatial tree in the physical problem domain ; in high-dimensions ASKIT... make use of randomized projection trees. 


Kernel summation methods - be it analytic, semi-analytic or algebraic - usually compute low rank approximations based on the far-field in the physical problem domain. Analytic and semi-analytic approaches fail in high dimensions as the number of required check and seed points grow exponentially.
In many applications (e.g. data analysis) the high dimensional feature space can be reduced to a low intrinsic dimension; for such kernel summations an algebraic skeletonization-based approach using a randomized projection tree has been introduced \cite{march-xiao-yu-biros15}. Near- and Far-Field separation is based on euclidean neighbor information; i.e. the near iteraction of each row $K_{i:}$ is defined by the
neighbor entries $\MA{N}_i$, and the far interaction is 
computed based on a nested binary partition of $K$.

Any symmetric positive-definite (SPD) matrix can be described as a Gramian matrix of some set of vectors $\phi$.
Given a matrix $K$ (e.g. from a kernel function) being symmetric positive definite, we can assume that there exists a set of vectors $\phi_i \in \mathcal{V}$ which define this matrix by a scalar product,  $$K_{ij}=\mathcal{K} (x_i,x_j)=<\phi_i,\phi_j>$$ 
%short version
%We consider a given symmetric positive-definite (SPD)  matrix $K$ in terms of an underlying Gramian set of vectors as  $$K_{ij}=\mathcal{K} (x_i,x_j)=<\phi_i,\phi_j>$$ 

%Note that we do 	not have direct access to the underlying vector set of $\phi_i$, but only to the scalar product between each other.

Hence, we use the Gramian space in defining a hierarchical subdivision where entries $i$,$j$ are considered near in terms of 
\begin{itemize}
\item[a.] the $L_2$ distance $\norm{\phi_i-\phi_j}_2^2=K_{ii}+K_{jj}-2K_{ij}$
\item[b.] the spanned angle $\sphericalangle(\phi_i,\phi_j)=arccos(\frac{K_{ij}}{K_{ii}K{jj}})$
\end{itemize}

This tree-like partition permutes $K$ into a hierarchical matrix whose off-diagonal blocks are potentially low-rank.
%We first define distance metrics for each column and row based on the symmetric factorization of $K$.

For doing so, we consider a direction spanned by the two farthest points $\phi_\alpha$ and $\phi_\beta$ and split indices $i$ binary by
\begin{itemize}
\item[a.] a $L_2$ based splitting $<\phi_i,\phi_\alpha-\phi_\beta>$
\item[b.] a $d-1$-dimensional two-fold cone spanned in direction $\phi_\alpha-\phi_\beta$, and its opposite direction $\phi_\beta-\phi_\alpha$, i.e. \\ $\frac{1}{\norm{\phi_i}}\norm{<\phi_i,\phi_\alpha-\phi_\beta>}$
\end{itemize}


% 
% 
% 
% Any symmetric positive-definite (SPD) matrix can be described as 
% a \emph{Gramian matrix} of some set of vectors $\{\phi_i\}$.
% Given a matrix $K$ (e.g. from a kernel function $K_{ij}=\MA{K}(x_i,x_j)$,
% where $x_{i} \in \mathbb{R}^{d}$) being symmetric 
% positive definite, we can assume that there exists a set of vectors 
% $\phi_i \in \mathcal{V}$ which define this matrix by an inner product,  
% $K_{ij}=\mathcal{K} (x_i,x_j)=\phi_i^{T}\phi_j$.
% Hence, we can define \emph{distance} $d_{ij}$ between $\phi_i$ and $\phi_j$
% in the Gramian space. We use two metrics:
% \begin{itemize}
% \item the $L^2$ distance $\|\phi_i-\phi_j\|_2^2=K_{ii}+K_{jj}-2K_{ij}$, or
% \item the $\sin{}$ similarity $1- \K_{ij}^2 / (K_{ii}K_{jj}) $.
% \end{itemize}
% These \emph{distances} define hierarchical subdivisions.
% 
% \begin{algorithm}[!t]
%   \caption{{} $[\lc,\rc]=\texttt{metricSplit}(\alpha)$}
% \begin{algorithmic}
%   \STATE $p = argmax( \{\sum_{j\in\texttt{c}}d_{ij} / \lvert \texttt{c} \rvert
%   \lvert i \in \alpha \})$
%   \STATE $q = argmax( \{ d_{ip} \lvert i \in \alpha \})$
%   \STATE $[\lc,\rc]=medianSplit(\{ d_{ip} - d_{iq} \lvert i \in \alpha\})$
% \end{algorithmic}
% \label{a:split}
% \end{algorithm}
% 
% 
% Starting from all indicies (\texttt{root}), we use a metric ball 
% tree to recursively partition indicies into two
% children $\lc$ and $\rc$. 
% This task is described as \texttt{SPLI($\alpha$)}
% in \tabref{tab:tasks} using \algref{a:split}. 
% Let's define an approximate centroid by $\hat{\phi_\alpha}= \frac{1}{\lvert \texttt{c}\rvert}\sum_{i\in\texttt{c}}\phi_{i}$,
% where $\texttt{c} \subset \alpha$ contains a constant number of samples, in order to avoid accessing $\MA{O}(N)$ entries.
% With this approximate centroid, we first find $p$ as the most far away
% index from the centroid in $\MA{O}(\lvert c \rvert\lvert\alpha\rvert)$, and we find $q$
% corresponding to $p$ in $\MA{O}(\lvert\alpha\rvert)$.
% 
% Next, we compute a \emph{distance} for each index $i$ towards $p$ and $q$, i.e. $d_{ip} - d_{iq}$. 
% Then we perform a median split on this metric $d_{ip} - d_{iq}$.
% This results in an even split, and $p$, $q$ will be in a different child.
% \texttt{SPLI($\alpha$)} stops when $\lvert \alpha \rvert \leq m$, a user defined leaf node size.
% %To avoid $\MA{O}(N^2)$ search on $d_{pq}$, an approximate centroid is defined as $\sum_{i\in\texttt{c}}\phi_{i} / \lvert \texttt{c} \rvert$, where $\texttt{c} \subset \alpha$ contains constant number of samples.
% 
% 
% The neighbors $\MA{N}$ are defined using the same metric $d_{ij}$.
% We employ an iterative neighbor search~\cite{} in \algref{a:compress} 
% that creates a binary 
% tree using randomized projection tree. That is, instead of searching for $p$ and
% $q$ in \algref{a:split}, we randomly select $p$ and $q$ from $\alpha$.
% In each iteration, exact neighbor search
% is only performed in the leaf level as \texttt{ANN($\alpha$)}
% shown in \tabref{tab:tasks}. 
% In this case, while searching for $k$ neighbors, we can restrict the
% candidate numbers to $\lvert \alpha \rvert$ (usually $2k$).
% Due to the randomness, these $2k$ candidates will be different 
% each time, and $\MA{N}$ may graduatelly converge\footnote{
% \scriptsize The convergence speed is affected by different factors; typically this is related to the intrinsic dimension of $K$.}.
% 
% 
% \textbf{Discussion.}
% Notice that in traditional FMM the \emph{distance} metric measures the 
% \emph{difficulty} of approximating the element. 
% 
% %This tree-like partition permutes $K$ into a hierarchical matrix whose 
% %off-diagonal blocks are potentially low-rank.
% %%We first define distance metrics for each column and row based on the symmetric factorization of $K$.
% %
% %For doing so, we consider a direction spanned by the two farthest 
% %points $\phi_\alpha$ and $\phi_\beta$ and split indices $i$ binary by
% %\begin{itemize}
% %\item[a.] a $L^2$ based splitting $<\phi_i,\phi_\alpha-\phi_\beta>$
% %\item[b.] a $d-1$-dimensional two-fold cone spanned in direction 
% %  $\phi_\alpha-\phi_\beta$, and its opposite direction 
% %  $\phi_\beta-\phi_\alpha$, 
% %  i.e. \\ $\frac{1}{\norm{\phi_i}}\lvert<\phi_i,\phi_\alpha-\phi_\beta>\rvert$
% %\end{itemize}
% 
% 
% 
% 
% %This tree-like partition permutes $K$ into a hierarchical matrix whose off-diagonal blocks are potentially low-rank.
% %We first define distance metrics for each column and row based on the symmetric factorization of $K$.
% %We then use these metrics to find neighbors and define the binary tree. 
% 
% %
% % define distance metrics here
% %
% 
% % section is reworded. please comment if unhappy
% %We then use these metrics to find neighbors; neighbors are computed 
% %with randomized projection tree based
% %algorithms. For each iteration, we select a random direction,
% %and split a tree node $\alpha$
% %into two children $\lc$ and $\rc$
% %\begin{itemize}
% %\item[a.] by a fictive orthogonal hyper-plane defined by the median
% % \item[b.] a fictive hyper-cone defined by a (roughly) equal-sized split 
% %\end{itemize}
% %An exhaustive neighbor search \texttt{ANN} is performed in each leaf node
% %$\alpha$ to update its neighbor list $\MA{N}_{\alpha}$.
% %When a pre-computed neighbor list is provided, this step is skipped.
% 
% % obsolete
% %The permutation for $K$ is computed with a ball tree. Given the distance metric, the two most far away points are select to form the projection direction. Again, a tree node $\alpha$ is split with the corresponding orthogonal hyper-plane. With this tree-based permutation, we can now compute the low-rank approximations for the off-diagonal blocks.
% 
% 
% \textbf{Low-rank approximations.}
% We now explain how skeleton $\sk{\alpha}$ is computed by the nested Interpolative Decomposition (ID) 
% in this work.
% Let $\alpha$ be the points in a tree node, and $S=\{1,...,N\}\backslash \alpha$
% be the set complement. The skeletonization of $\alpha$ is a rank-$s$
% approximation of its off-diagonal blocks $K_{S\alpha}$ using ID, 
% written as
% \begin{equation}
% K_{S\alpha} \approx K_{S\sk{\alpha}}P_{\sk{\alpha}\alpha}.
% \end{equation}
% Here $\sk{\alpha} \subset \alpha$ is the \emph{skeleton} of $\alpha$.
% $K_{S\sk{\alpha}}$ contains $s$ columns basis, and $P_{\sk{\alpha}\alpha} \in
% \mathbb{R}^{s\times \lvert \alpha \rvert}$ contains all interpolation
% coefficients.
% We use the first $s$ pivots of a rank-revealing QR factorization (\texttt{GEQP3}) 
% on $K_{S\sk{\alpha}}$ to select $\alpha$.
% With QR factorization of $K_{S\sk{\alpha}}$, we can compute
% $P_{\sk{\alpha}\alpha} = K_{S\sk{\alpha}}^{\dagger}K_{S\alpha}$ 
% with a triangular solver (\texttt{TRSM}).
% This scheme however results in $\bigO(d N^2\ppl)$ complexity for the
% overall factorization. We can turn it to a $\bigO(d \log N \ppl)$
% scheme by sampling a small subset $S'$ of $S$ and using it instead of
% $S$. With $\MA{N}(\alpha)$, we can perform importance sampling as shown 
% in~\cite{march-xiao-biros-e15}.
% The approximation rank $\ns$ is chosen such that
% $\sigma_{\ns+1}(K_{S' \alpha}) < \idtol$,
% where $\idtol$ is user-specified and $\sigma$ are the singular
% values estimated by the diagonal of the rank-revealing QR.
% 
% \begin{algorithm}[!t]
% \caption{{} [$\sk{\alpha}, P_{\sk{\alpha} \alpha}$]=\texttt{Skeleton}($\alpha$)}
% \begin{algorithmic}
%   \STATE \texttt{\bf if} $\alpha$ is leaf \texttt{\bf then return} [$\sk{\alpha}, P_{\sk{\alpha} \alpha}$] = {\tt ID}$(\alpha)$;
%   \STATE $[\sk{\lc},]=\texttt{Skeleton}(\lc)$; $[\sk{\rc},]=\texttt{Skeleton}(\rc)$;
%   \STATE \texttt{\bf return} [$\sk{\alpha}, P_{\sk{\alpha} [\sk{\lc} \sk{\rc}]
%   }$] = {\tt ID}($[\sk{\lc} \sk{\rc}]$);
% \end{algorithmic}
% \label{a:skeletonize}
% \end{algorithm}
% 
% \algref{a:skeletonize} computes skeletonization for all tree nodes with a 
% postorder traversal. As we have discussed in \secref{s:hier}, 
% for a non-leaf node $\alpha$, we use nested basis from children.
% Thus, instead of selecting 
% $\sk{\alpha}$ from $\alpha$, a greedy algorithm is used to only sub-select
% from children's skeleton $\sk{\lc}\cup\sk{\rc}$. Subsequently, 
% %instead of creating the whole coefficient matrix
% %$P_{\sk{\alpha}\alpha}$, we only compute the interpolation coefficients
% %$P_{\sk{\alpha}[\sk{\lc}\sk{\rc}]}$. 
% we have the \emph{telescoping} relationship shown in \eqref{e:telescoping}.
% Overall, \algref{a:skeletonize} has
% two tasks for each tree node $\alpha$ shown in \tabref{tab:tasks}: 
% (1) \texttt{SKEL($\alpha$)} selects $\sk{\alpha}$ (critical path) and
% (2) \texttt{COEF($\alpha$)} computes $P_{\sk{\alpha}[\sk{\lc}\sk{\rc}]}$ 
% (depending on $\sk{\alpha}$).
% Since the parallelism diminishes in \algref{a:skeletonize} during the 
% traversal, a proper schedule is required to improve the efficiency. 
% 
% \textbf{Evaluation.}
% While partial matrix-multiplication $u_{\beta} = K_{\beta\alpha}w_{\alpha}$
% can be interpreted as interaction between tree nodes $\beta$ and $\alpha$,
% We approximate $K_{\beta\alpha}$ with both column and row skeletons as
% $P_{\sk{\beta}\beta}^{T}K_{\sk{\beta}\sk{\alpha}}P_{\sk{\alpha}\alpha}$.
% Notice that due to the nested relation of $\alpha \subset \sk{\lc} \cup \sk{\rc}$,
% $P_{\sk{\alpha}\alpha}$ and $P_{\sk{\beta}\beta}$
% can be \emph{telescoped} recursively as
% \begin{equation}
% P_{\sk{\alpha}\alpha} =
% P_{\sk{\alpha}[\sk{\lc}\sk{\rc}]}
% \begin{bmatrix}
%   P_{\sk{\lc}\lc} & \\
%                   & P_{\sk{\rc}\rc} \\
% \end{bmatrix}.
% % \label{e:telescope}
% \end{equation}
% Recursive interpolation from both sides (left and right) is the key to achieve
% fast evaluation (less than quadratic). Still we need to first decide
% which pair of $\beta$ and $\alpha$ can be approximated and how to compute
% all pairs of interaction efficiently.
% 
% In \algref{a:nearfar} and \algref{a:farfar}, we decide which pair of
% interaction can be approximated and which cannot by traversing the tree
% top-down. 
% We collect all neighbors
% of a leaf node $\beta$ (including $\beta$ itself) named as $\MA{N}(\beta)$, 
% and we invoke \texttt{NearFar}($\beta$,\texttt{root}) for each $\beta$. 
% While traversing to $\alpha$,
% we say $\beta$ can prune $\alpha$ if none of $\MA{N}(\beta)$ appears in $\alpha$.
% We append $\alpha$ to $Far(\beta)$, which denotes all interactions that can be
% approximated with $\beta$. 
% Otherwise, we recurse to visit the children $\lc$ and $\rc$ of $\alpha$.
% While reaching the leaf level, if $\alpha$ still contains neighbors of $\beta$,
% then we append $\alpha$ to $Near(\beta)$, which denotes all interactions that
% cannot be approximated.
% \algref{a:nearfar} generates a list of prunable interactions for all leaf
% nodes. To generate prunable interaction lists for inner nodes, we merge
% two prunable lists from children to find the intersection.
% \algref{a:farfar} performs a post-order traversal, merging the far interaction
% lists from children, then removing duplications.
% 
% 
% % \begin{algorithm}
% % \caption{{} \texttt{Node2Skel}($\alpha$)}
% % \begin{algorithmic}
% % \STATE \texttt{\bf if} $\alpha$ is leaf \texttt{\bf then} $\sk{w_{\alpha}} = P_{\sk{\alpha}\alpha}w_{\alpha}$;
% % \texttt{\bf else} $\sk{w_{\alpha}} = P_{\sk{\alpha}[\sk{\lc}\sk{\rc}]}[ \sk{w_{\lc}}; \sk{w_{\rc}}]$;
% % \end{algorithmic}
% % \label{a:n2s}
% % \end{algorithm}
% % 
% % 
% % \begin{algorithm}
% % \caption{{} \texttt{Skel2Skel}()}
% % \begin{algorithmic}
% %   \STATE \texttt{\bf for each} $\beta$ \texttt{\bf do} $\sk{u_{\beta}} =
% %   \sum_{\alpha\in Far(\beta)} K_{\sk{\beta}\sk{\alpha}}\sk{w_{\alpha}}$;
% % \end{algorithmic}
% % \label{a:n2n}
% % \end{algorithm}
% % 
% % \begin{algorithm}
% % \caption{{} \texttt{Skel2Node}($\beta$)}
% % \begin{algorithmic}
% % \STATE \texttt{\bf if} $\beta$ is leaf \texttt{\bf then} $u_{\beta} = P_{\sk{\beta}\beta}^{T}\sk{u_{\beta}} + \sum_{\alpha\in Near(\beta)}K_{\beta\alpha}w_{\alpha}$;
% % \STATE \texttt{\bf else} $[\sk{u_{\lc}};\sk{u_{\rc}}] +=
% % P_{\sk{\beta}[\sk{\lc}\sk{\rc}]}^{T}\sk{u_{\beta}}$;
% % \end{algorithmic}
% % \label{a:n2s}
% % \end{algorithm}
% 
% 
% \begin{algorithm}
% \caption{{} \texttt{Evaluate}($u$,$w$)}
% \begin{algorithmic}
% \STATE \textbf{POST} \texttt{N2S($\alpha$)}
% \STATE \textbf{ANY} \texttt{S2S($\beta$)} \#\textbf{RAW} on \texttt{N2S}
% \STATE \textbf{PRE} \texttt{S2N($\beta$)} \#\textbf{RAW} on \texttt{S2S}
% \STATE \textbf{ANY} \texttt{L2L($\beta$)} \#\textbf{RAR} on \texttt{S2N}
% \end{algorithmic}
% \label{a:evaluate}
% \end{algorithm}
% 
% With the near and far interaction lists in hand, approximating the 
% matrix-multiplication of $K$ is described in \algref{a:evaluation}
% as a three-step process. 
% 
% \texttt{Node2Skel} performs a postorder traversal
% and computes the skeleton weight $\sk{w_{\alpha}}=P_{\sk{\alpha}\alpha}w_{\alpha}$ 
% for each tree node by multiplying the coefficient matrix on the left. 
% Recall from \eqref{e:telescope} that for a non-leaf node,
% $P_{\sk{\alpha}\alpha}$ can be \emph{telescoped} fast. 
% Thus, $\sk{w_{\alpha}}$ can be computed as
% $P_{\sk{\alpha}[\sk{\lc}\sk{\rc}]}[\sk{w_{\sk{\lc}}}\sk{w_{\sk{\rc}}}]$
% in the postorder traversal, because skeleton weights of both children have
% been computed. \texttt{Skel2Skel} traverses each node and compute the skeleton potentials
% $\sk{u_{\beta}} = \sum_{\alpha\in Far(\beta)}
% K_{\sk{\beta}\sk{\alpha}}\sk{w_{\alpha}}$.
% Finally \texttt{Skel2Node} performs a preorder traveral to accumulate
% all skeleton potentials. While traversing to leaf nodes, the aggregate
% potential is the summation of the skeleton potentials
% $P_{\sk{\beta}\beta}^{T}\sk{u_{\beta}}$
% and all direct interactions  
% $\sum_{\alpha\in Near(\beta)}K_{\beta\alpha}w_{\alpha}$.
% 
% \textbf{Complexity.} 
% While the worst case estimate of \algref{a:evalaute} is $\MA{O}(N^2)$
% (no approximation takes place), the best case appears when each 
% $Near(\alpha)$ only contains $\alpha$ itself.
% For a complete binary tree with $\MA{O}(N/m)$ leaf nodes, there are also 
% $\MA{O}(N/m)$ inner nodes.
% The cost for \texttt{Node2Skel} is $\MA{O}(sm(N/m)+2s^2(N/m))$.
% The cost for \texttt{Skel2Skel} is $\MA{O}(2s^2(N/m))$.
% Finally the cost for \texttt{Skel2Node} is
% $\MA{O(m^2(N/m)+sm(N/m)+2s^2(N/m))}$.
% When $s$ and $m$ are constant in $N$, the asympotoptic work 
% is $O(N)$ per right hand side.
% 
