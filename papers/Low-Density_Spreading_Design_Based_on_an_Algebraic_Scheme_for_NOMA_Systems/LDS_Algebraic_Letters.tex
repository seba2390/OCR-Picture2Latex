
%% bare_jrnl_comsoc.tex
%% V1.4b
%% 2015/08/26
%% by Michael Shell
%% see http://www.michaelshell.org/
%% for current contact information.
%%
%% This is a skeleton file demonstrating the use of IEEEtran.cls
%% (requires IEEEtran.cls version 1.8b or later) with an IEEE
%% Communications Society journal paper.
%%
%% Support sites:
%% http://www.michaelshell.org/tex/ieeetran/
%% http://www.ctan.org/pkg/ieeetran
%% and
%% http://www.ieee.org/

%%*************************************************************************
%% Legal Notice:
%% This code is offered as-is without any warranty either expressed or
%% implied; without even the implied warranty of MERCHANTABILITY or
%% FITNESS FOR A PARTICULAR PURPOSE! 
%% User assumes all risk.
%% In no event shall the IEEE or any contributor to this code be liable for
%% any damages or losses, including, but not limited to, incidental,
%% consequential, or any other damages, resulting from the use or misuse
%% of any information contained here.
%%
%% All comments are the opinions of their respective authors and are not
%% necessarily endorsed by the IEEE.
%%
%% This work is distributed under the LaTeX Project Public License (LPPL)
%% ( http://www.latex-project.org/ ) version 1.3, and may be freely used,
%% distributed and modified. A copy of the LPPL, version 1.3, is included
%% in the base LaTeX documentation of all distributions of LaTeX released
%% 2003/12/01 or later.
%% Retain all contribution notices and credits.
%% ** Modified files should be clearly indicated as such, including  **
%% ** renaming them and changing author support contact information. **
%%*************************************************************************


% *** Authors should verify (and, if needed, correct) their LaTeX system  ***
% *** with the testflow diagnostic prior to trusting their LaTeX platform ***
% *** with production work. The IEEE's font choices and paper sizes can   ***
% *** trigger bugs that do not appear when using other class files.       ***                          ***
% The testflow support page is at:
% http://www.michaelshell.org/tex/testflow/



\documentclass[journal,comsoc]{IEEEtran}
%\documentclass[12pt,peerreview,onecolumn]{IEEEtran}
\IEEEoverridecommandlockouts
%
% If IEEEtran.cls has not been installed into the LaTeX system files,
% manually specify the path to it like:
% \documentclass[journal,comsoc]{../sty/IEEEtran}
\usepackage{cite}
\usepackage{amsmath,amssymb,amsfonts}
\usepackage{algorithmic}
\usepackage{graphicx}
\usepackage{textcomp}
\usepackage{xcolor}

\usepackage[T1]{fontenc}% optional T1 font encoding

\usepackage[hyphens]{url}
\usepackage{hyperref}
\hypersetup{breaklinks=true}

\graphicspath{{myfigures/}}
\usepackage{epstopdf}					% convert eps figure automatically to pdf for pdflatex
\epstopdfsetup{suffix=}	
\usepackage{amsfonts}
% or
\usepackage{amssymb}
\usepackage{multirow}
\usepackage{bm}
\usepackage{cite}
\newtheorem{lem}{Lemma}
\newtheorem{theorem}{Theorem}[section]
\newcommand{\nid}{\noindent}
\newcommand{\argmin}{\operatornamewithlimits{argmin}}
\newcommand{\argmax}{\operatornamewithlimits{argmax}}

\newtheorem{defin}{Definition}
%\theoremstyle{definition}
\newtheorem{definition}{Definition}[section]
%\DeclareMathOperator*{\argmax}{arg\,max}
%\DeclareMathOperator*{\argmin}{arg\,min}
%\DeclareMathOperator*{\argmax}{argmax}

\usepackage{color}
% Some very useful LaTeX packages include:
% (uncomment the ones you want to load)


% *** MISC UTILITY PACKAGES ***
%
%\usepackage{ifpdf}
% Heiko Oberdiek's ifpdf.sty is very useful if you need conditional
% compilation based on whether the output is pdf or dvi.
% usage:
% \ifpdf
%   % pdf code
% \else
%   % dvi code
% \fi
% The latest version of ifpdf.sty can be obtained from:
% http://www.ctan.org/pkg/ifpdf
% Also, note that IEEEtran.cls V1.7 and later provides a builtin
% \ifCLASSINFOpdf conditional that works the same way.
% When switching from latex to pdflatex and vice-versa, the compiler may
% have to be run twice to clear warning/error messages.






% *** CITATION PACKAGES ***
%
%\usepackage{cite}
% cite.sty was written by Donald Arseneau
% V1.6 and later of IEEEtran pre-defines the format of the cite.sty package
% \cite{} output to follow that of the IEEE. Loading the cite package will
% result in citation numbers being automatically sorted and properly
% "compressed/ranged". e.g., [1], [9], [2], [7], [5], [6] without using
% cite.sty will become [1], [2], [5]--[7], [9] using cite.sty. cite.sty's
% \cite will automatically add leading space, if needed. Use cite.sty's
% noadjust option (cite.sty V3.8 and later) if you want to turn this off
% such as if a citation ever needs to be enclosed in parenthesis.
% cite.sty is already installed on most LaTeX systems. Be sure and use
% version 5.0 (2009-03-20) and later if using hyperref.sty.
% The latest version can be obtained at:
% http://www.ctan.org/pkg/cite
% The documentation is contained in the cite.sty file itself.






% *** GRAPHICS RELATED PACKAGES ***
%
\ifCLASSINFOpdf
  % \usepackage[pdftex]{graphicx}
  % declare the path(s) where your graphic files are
  % \graphicspath{{../pdf/}{../jpeg/}}
  % and their extensions so you won't have to specify these with
  % every instance of \includegraphics
  % \DeclareGraphicsExtensions{.pdf,.jpeg,.png}
\else
  % or other class option (dvipsone, dvipdf, if not using dvips). graphicx
  % will default to the driver specified in the system graphics.cfg if no
  % driver is specified.
  % \usepackage[dvips]{graphicx}
  % declare the path(s) where your graphic files are
  % \graphicspath{{../eps/}}
  % and their extensions so you won't have to specify these with
  % every instance of \includegraphics
  % \DeclareGraphicsExtensions{.eps}
\fi
% graphicx was written by David Carlisle and Sebastian Rahtz. It is
% required if you want graphics, photos, etc. graphicx.sty is already
% installed on most LaTeX systems. The latest version and documentation
% can be obtained at: 
% http://www.ctan.org/pkg/graphicx
% Another good source of documentation is "Using Imported Graphics in
% LaTeX2e" by Keith Reckdahl which can be found at:
% http://www.ctan.org/pkg/epslatex
%
% latex, and pdflatex in dvi mode, support graphics in encapsulated
% postscript (.eps) format. pdflatex in pdf mode supports graphics
% in .pdf, .jpeg, .png and .mps (metapost) formats. Users should ensure
% that all non-photo figures use a vector format (.eps, .pdf, .mps) and
% not a bitmapped formats (.jpeg, .png). The IEEE frowns on bitmapped formats
% which can result in "jaggedy"/blurry rendering of lines and letters as
% well as large increases in file sizes.
%
% You can find documentation about the pdfTeX application at:
% http://www.tug.org/applications/pdftex





% *** MATH PACKAGES ***
%
\usepackage{amsmath}
% A popular package from the American Mathematical Society that provides
% many useful and powerful commands for dealing with mathematics.
% Do NOT use the amsbsy package under comsoc mode as that feature is
% already built into the Times Math font (newtxmath, mathtime, etc.).
% 
% Also, note that the amsmath package sets \interdisplaylinepenalty to 10000
% thus preventing page breaks from occurring within multiline equations. Use:
\interdisplaylinepenalty=2500
% after loading amsmath to restore such page breaks as IEEEtran.cls normally
% does. amsmath.sty is already installed on most LaTeX systems. The latest
% version and documentation can be obtained at:
% http://www.ctan.org/pkg/amsmath


% Select a Times math font under comsoc mode or else one will automatically
% be selected for you at the document start. This is required as Communications
% Society journals use a Times, not Computer Modern, math font.
\usepackage[cmintegrals]{newtxmath}
% The freely available newtxmath package was written by Michael Sharpe and
% provides a feature rich Times math font. The cmintegrals option, which is
% the default under IEEEtran, is needed to get the correct style integral
% symbols used in Communications Society journals. Version 1.451, July 28,
% 2015 or later is recommended. Also, do *not* load the newtxtext.sty package
% as doing so would alter the main text font.
% http://www.ctan.org/pkg/newtx
%
% Alternatively, you can use the MathTime commercial fonts if you have them
% installed on your system:
%\usepackage{mtpro2}
%\usepackage{mt11p}
%\usepackage{mathtime}


%\usepackage{bm}
% The bm.sty package was written by David Carlisle and Frank Mittelbach.
% This package provides a \bm{} to produce bold math symbols.
% http://www.ctan.org/pkg/bm





% *** SPECIALIZED LIST PACKAGES ***
%
%\usepackage{algorithmic}
% algorithmic.sty was written by Peter Williams and Rogerio Brito.
% This package provides an algorithmic environment fo describing algorithms.
% You can use the algorithmic environment in-text or within a figure
% environment to provide for a floating algorithm. Do NOT use the algorithm
% floating environment provided by algorithm.sty (by the same authors) or
% algorithm2e.sty (by Christophe Fiorio) as the IEEE does not use dedicated
% algorithm float types and packages that provide these will not provide
% correct IEEE style captions. The latest version and documentation of
% algorithmic.sty can be obtained at:
% http://www.ctan.org/pkg/algorithms
% Also of interest may be the (relatively newer and more customizable)
% algorithmicx.sty package by Szasz Janos:
% http://www.ctan.org/pkg/algorithmicx




% *** ALIGNMENT PACKAGES ***
%
%\usepackage{array}
% Frank Mittelbach's and David Carlisle's array.sty patches and improves
% the standard LaTeX2e array and tabular environments to provide better
% appearance and additional user controls. As the default LaTeX2e table
% generation code is lacking to the point of almost being broken with
% respect to the quality of the end results, all users are strongly
% advised to use an enhanced (at the very least that provided by array.sty)
% set of table tools. array.sty is already installed on most systems. The
% latest version and documentation can be obtained at:
% http://www.ctan.org/pkg/array


% IEEEtran contains the IEEEeqnarray family of commands that can be used to
% generate multiline equations as well as matrices, tables, etc., of high
% quality.




% *** SUBFIGURE PACKAGES ***
%\ifCLASSOPTIONcompsoc
%  \usepackage[caption=false,font=normalsize,labelfont=sf,textfont=sf]{subfig}
%\else
%  \usepackage[caption=false,font=footnotesize]{subfig}
%\fi
% subfig.sty, written by Steven Douglas Cochran, is the modern replacement
% for subfigure.sty, the latter of which is no longer maintained and is
% incompatible with some LaTeX packages including fixltx2e. However,
% subfig.sty requires and automatically loads Axel Sommerfeldt's caption.sty
% which will override IEEEtran.cls' handling of captions and this will result
% in non-IEEE style figure/table captions. To prevent this problem, be sure
% and invoke subfig.sty's "caption=false" package option (available since
% subfig.sty version 1.3, 2005/06/28) as this is will preserve IEEEtran.cls
% handling of captions.
% Note that the Computer Society format requires a larger sans serif font
% than the serif footnote size font used in traditional IEEE formatting
% and thus the need to invoke different subfig.sty package options depending
% on whether compsoc mode has been enabled.
%
% The latest version and documentation of subfig.sty can be obtained at:
% http://www.ctan.org/pkg/subfig




% *** FLOAT PACKAGES ***
%
%\usepackage{fixltx2e}
% fixltx2e, the successor to the earlier fix2col.sty, was written by
% Frank Mittelbach and David Carlisle. This package corrects a few problems
% in the LaTeX2e kernel, the most notable of which is that in current
% LaTeX2e releases, the ordering of single and double column floats is not
% guaranteed to be preserved. Thus, an unpatched LaTeX2e can allow a
% single column figure to be placed prior to an earlier double column
% figure.
% Be aware that LaTeX2e kernels dated 2015 and later have fixltx2e.sty's
% corrections already built into the system in which case a warning will
% be issued if an attempt is made to load fixltx2e.sty as it is no longer
% needed.
% The latest version and documentation can be found at:
% http://www.ctan.org/pkg/fixltx2e


%\usepackage{stfloats}
% stfloats.sty was written by Sigitas Tolusis. This package gives LaTeX2e
% the ability to do double column floats at the bottom of the page as well
% as the top. (e.g., "\begin{figure*}[!b]" is not normally possible in
% LaTeX2e). It also provides a command:
%\fnbelowfloat
% to enable the placement of footnotes below bottom floats (the standard
% LaTeX2e kernel puts them above bottom floats). This is an invasive package
% which rewrites many portions of the LaTeX2e float routines. It may not work
% with other packages that modify the LaTeX2e float routines. The latest
% version and documentation can be obtained at:
% http://www.ctan.org/pkg/stfloats
% Do not use the stfloats baselinefloat ability as the IEEE does not allow
% \baselineskip to stretch. Authors submitting work to the IEEE should note
% that the IEEE rarely uses double column equations and that authors should try
% to avoid such use. Do not be tempted to use the cuted.sty or midfloat.sty
% packages (also by Sigitas Tolusis) as the IEEE does not format its papers in
% such ways.
% Do not attempt to use stfloats with fixltx2e as they are incompatible.
% Instead, use Morten Hogholm'a dblfloatfix which combines the features
% of both fixltx2e and stfloats:
%
% \usepackage{dblfloatfix}
% The latest version can be found at:
% http://www.ctan.org/pkg/dblfloatfix




%\ifCLASSOPTIONcaptionsoff
%  \usepackage[nomarkers]{endfloat}
% \let\MYoriglatexcaption\caption
% \renewcommand{\caption}[2][\relax]{\MYoriglatexcaption[#2]{#2}}
%\fi
% endfloat.sty was written by James Darrell McCauley, Jeff Goldberg and 
% Axel Sommerfeldt. This package may be useful when used in conjunction with 
% IEEEtran.cls'  captionsoff option. Some IEEE journals/societies require that
% submissions have lists of figures/tables at the end of the paper and that
% figures/tables without any captions are placed on a page by themselves at
% the end of the document. If needed, the draftcls IEEEtran class option or
% \CLASSINPUTbaselinestretch interface can be used to increase the line
% spacing as well. Be sure and use the nomarkers option of endfloat to
% prevent endfloat from "marking" where the figures would have been placed
% in the text. The two hack lines of code above are a slight modification of
% that suggested by in the endfloat docs (section 8.4.1) to ensure that
% the full captions always appear in the list of figures/tables - even if
% the user used the short optional argument of \caption[]{}.
% IEEE papers do not typically make use of \caption[]'s optional argument,
% so this should not be an issue. A similar trick can be used to disable
% captions of packages such as subfig.sty that lack options to turn off
% the subcaptions:
% For subfig.sty:
% \let\MYorigsubfloat\subfloat
% \renewcommand{\subfloat}[2][\relax]{\MYorigsubfloat[]{#2}}
% However, the above trick will not work if both optional arguments of
% the \subfloat command are used. Furthermore, there needs to be a
% description of each subfigure *somewhere* and endfloat does not add
% subfigure captions to its list of figures. Thus, the best approach is to
% avoid the use of subfigure captions (many IEEE journals avoid them anyway)
% and instead reference/explain all the subfigures within the main caption.
% The latest version of endfloat.sty and its documentation can obtained at:
% http://www.ctan.org/pkg/endfloat
%
% The IEEEtran \ifCLASSOPTIONcaptionsoff conditional can also be used
% later in the document, say, to conditionally put the References on a 
% page by themselves.




% *** PDF, URL AND HYPERLINK PACKAGES ***
%
%\usepackage{url}
% url.sty was written by Donald Arseneau. It provides better support for
% handling and breaking URLs. url.sty is already installed on most LaTeX
% systems. The latest version and documentation can be obtained at:
% http://www.ctan.org/pkg/url
% Basically, \url{my_url_here}.




% *** Do not adjust lengths that control margins, column widths, etc. ***
% *** Do not use packages that alter fonts (such as pslatex).         ***
% There should be no need to do such things with IEEEtran.cls V1.6 and later.
% (Unless specifically asked to do so by the journal or conference you plan
% to submit to, of course. )


% correct bad hyphenation here
\hyphenation{op-tical net-works semi-conduc-tor}


\begin{document}
%
% paper title
% Titles are generally capitalized except for words such as a, an, and, as,
% at, but, by, for, in, nor, of, on, or, the, to and up, which are usually
% not capitalized unless they are the first or last word of the title.
% Linebreaks \\ can be used within to get better formatting as desired.
% Do not put math or special symbols in the title.
\title{Low-Density Spreading Design Based on an Algebraic Scheme for NOMA Systems}
%
%
% author names and IEEE memberships
% note positions of commas and nonbreaking spaces ( ~ ) LaTeX will not break
% a structure at a ~ so this keeps an author's name from being broken across
% two lines.
% use \thanks{} to gain access to the first footnote area
% a separate \thanks must be used for each paragraph as LaTeX2e's \thanks
% was not built to handle multiple paragraphs
%

%\author{Author~1, Author~2, Author~3, Author~4, Author~5, and Author~6
      % <-this % stops a space
%\thanks{M. Shell was with the Department
%of Electrical and Computer Engineering, Georgia Institute of Technology, Atlanta,
%GA, 30332 USA e-mail: (see http://www.michaelshell.org/contact.html).}% <-this % stops a space
%\thanks{J. Doe and J. Doe are with Anonymous University.}% <-this % stops a space
%\thanks{Manuscript received Oct. 15, 2020;}}
\author{Goldwyn~Millar, Michel~Kulhandjian,~\IEEEmembership{Senior Member,~IEEE,
}Ayse~Alaca, Saban~Alaca,  Claude~D'Amours,~\IEEEmembership{Member,~IEEE,
}and Halim~Yanikomeroglu,~\IEEEmembership{Fellow,~IEEE
}
%Goldwyn Millar, Michel Kulhandjian, Saban Alaca, Ayse Alaca, Claude D’Amours, and Halim Yanikomeroglu
      % <-this % stops a space
%\thanks{M. Shell was with the Department
%of Electrical and Computer Engineering, Georgia Institute of Technology, Atlanta,
%GA, 30332 USA e-mail: (see http://www.michaelshell.org/contact.html).}% <-this % stops a space
%\thanks{J. Doe and J. Doe are with Anonymous University.}% <-this % stops a space
%\thanks{Authors' affiliation}

\thanks{M. Kulhandjian and C. D'Amours are with the School of Electrical Engineering, \& Computer Science, University of Ottawa, Ottawa, Canada, e-mail: mkk6@buffalo.edu, cdamours@uottawa.ca.}% <-this % stops a space
\thanks{G. Millar, A. Alaca and S. Alaca are with the School of Mathematics and Statistics, Carleton University, Ottawa, Canada, e-mail:
goldwynmillar@cmail.carleton.ca, \{aysealaca,sabanalaca\}@cunet.carleton.ca.}
\thanks{H. Yanikomeroglu is with the Department of Systems \& Computer Engineering, Carleton University, Ottawa, Canada, e-mail: halim@sce.carleton.ca.}
\thanks{Manuscript received Dec. 03, 2021;}
}

% note the % following the last \IEEEmembership and also \thanks - 
% these prevent an unwanted space from occurring between the last author name
% and the end of the author line. i.e., if you had this:
% 
% \author{....lastname \thanks{...} \thanks{...} }
%                     ^------------^------------^----Do not want these spaces!
%
% a space would be appended to the last name and could cause every name on that
% line to be shifted left slightly. This is one of those "LaTeX things". For
% instance, "\textbf{A} \textbf{B}" will typeset as "A B" not "AB". To get
% "AB" then you have to do: "\textbf{A}\textbf{B}"
% \thanks is no different in this regard, so shield the last } of each \thanks
% that ends a line with a % and do not let a space in before the next \thanks.
% Spaces after \IEEEmembership other than the last one are OK (and needed) as
% you are supposed to have spaces between the names. For what it is worth,
% this is a minor point as most people would not even notice if the said evil
% space somehow managed to creep in.



% The paper headers
\markboth{IEEE Wireless Communications Letters,~Vol.~1, No.~1, Dec.~2021}%
{Shell \MakeLowercase{\textit{et al.}}: Bare Demo of IEEEtran.cls for IEEE Communications Society Journals}
% The only time the second header will appear is for the odd numbered pages
% after the title page when using the twoside option.
% 
% *** Note that you probably will NOT want to include the author's ***
% *** name in the headers of peer review papers.                   ***
% You can use \ifCLASSOPTIONpeerreview for conditional compilation here if
% you desire.




% If you want to put a publisher's ID mark on the page you can do it like
% this:
%\IEEEpubid{0000--0000/00\$00.00~\copyright~2015 IEEE}
% Remember, if you use this you must call \IEEEpubidadjcol in the second
% column for its text to clear the IEEEpubid mark.



% use for special paper notices
%\IEEEspecialpapernotice{(Invited Paper)}




% make the title area
\maketitle

% As a general rule, do not put math, special symbols or citations
% in the abstract or keywords.
%\vspace{-0.5cm}
\begin{abstract}
In this paper, a code-domain nonorthogonal
multiple access (NOMA) technique based on an algebraic design is studied. We propose an improved low-density spreading (LDS) sequence design based on projective geometry. In terms of its bit error rate (BER) performance, our proposed improved LDS code set outperforms the existing LDS designs over the frequency-nonselective Rayleigh fading and additive white Gaussian noise (AWGN) channels. We demonstrated that achieving the best BER depends on the minimum distance.  

\end{abstract}

% Note that keywords are not normally used for peerreview papers.
\begin{IEEEkeywords}
Nonorthogonal multiple access (NOMA), low-density spreading (LDS), sparse code multiple access (SCMA).
\end{IEEEkeywords}



% For peer review papers, you can put extra information on the cover
% page as needed:
% \ifCLASSOPTIONpeerreview
% \begin{center} \bfseries EDICS Category: 3-BBND \end{center}
% \fi
%
% For peerreview papers, this IEEEtran command inserts a page break and
% creates the second title. It will be ignored for other modes.
\IEEEpeerreviewmaketitle


\section{Introduction}
% The very first letter is a 2 line initial drop letter followed
% by the rest of the first word in caps.
% 
% form to use if the first word consists of a single letter:
% \IEEEPARstart{A}{demo} file is ....
% 
% form to use if you need the single drop letter followed by
% normal text (unknown if ever used by the IEEE):
% \IEEEPARstart{A}{}demo file is ....
% 
% Some journals put the first two words in caps:
% \IEEEPARstart{T}{his demo} file is ....
% 
% Here we have the typical use of a "T" for an initial drop letter
% and "HIS" in caps to complete the first word.
%\IEEEPARstart{T}{he} faster than Nyquist (FTN) ...  \cite{anderson2013faster}

\IEEEPARstart{I}{n} previous generations of wireless communications, orthogonal multiple access (OMA) techniques were predominant. In OMA systems, users are assigned resources that are orthogonal to one another, such as orthogonal frequency division multiple access (OFDMA), orthogonal time division multiple access (OTDMA) or code division multiple access (CDMA) where the spreading signatures were mutually orthogonal. Ideally, in OMA systems the presence of multiple users does not cause interference to any of the users that occupy the channel. However, the capacity of an OMA system is limited by the number of available orthogonal resources \cite{Dai2018}.

Future wireless networks are required to support a wide range of use cases. One such use case is to provide communication capabilities to a massive number of low-power Internet-of-Things (IoT) devices \cite{Dai2018}. Due to the limitations of OMA, supporting a large number of users over a common channel while achieving the required level of service quality may not be possible. In rank-deficient cases, where the number of active communication devices exceeds the number of orthogonal resources, nonorthogonal multiple access (NOMA) systems are proposed \cite{Dai2018}. In NOMA systems, users are assigned multiple orthogonal resources and the resources are assigned to multiple users. When users transmit simultaneously, each user may experience some multiple-access interference (MAI) if the same resource is used by two or more transmitters simultaneously. For mitigating the MAI, many researchers have proposed a sparse allocation of these resources so as to take advantage of efficient sparse signal processing techniques. For example, the message passing algorithm (MPA) can be used to iteratively perform multiuser detection (MUD) in a NOMA system using sparsely assigned resources. NOMA techniques can be categorized into power-domain multiplexed NOMA (PDM-NOMA) and code domain multiplexed NOMA (CDM-NOMA) \cite{Dai2018}. A few of the strong contenders of CDM-NOMA are low-density spread CDMA (LDS-CDMA) \cite{Hoshyar2008}, low-density spread orthogonal frequency-division multiplexing (LDS-OFDM) \cite{Razavi2012}, and sparse code multiple access (SCMA) \cite{Nikopour2013}.

A number of studies have been undertaken to design spreading code sets for sparse spreading based NOMA \cite{Hoshyar2006, Song2017, Jiang2019}. In \cite{Hoshyar2006} the authors proposed an LDS structure based on LDPC codes, where the user's symbols are arranged in such a way that the interference seen by each user on each chip is different, while in \cite{JVan2009}, the authors designed the spreading sequences based on an LDPC indicator matrix. In general, signatures having a unity scalar magnitude are designed by maximizing their minimum Euclidean distance. Similar to the minimum distance criterion based LDS code design of \cite{JVan2009}, the authors of \cite{Song2017} consider the maximization of the minimum Euclidean distance for QAM constellations. Notably, they design signature matrices that have factor graphs exhibiting very few short cycles and large superposed signal constellation distances. In \cite{Jiang2019}, the authors optimize the degree distribution of the LDS signature matrix.

Combinatorial structures with balanced incomplete block design (BIBD) can allow a large number of users employing few resources. As an example, authors in \cite{Claude1995} proposed a multiple tone frequency shift keying (MT-FSK) waveform design based on BIBD. A specific highly structured BIBD called the Steiner triple system (STS) is well studied for low density parity-check (LDPC) constructions \cite{Ivanov2013}. Steiner designs used as sparse codes have better interference properties, which provide higher user/bandwidth efficiency and variable code rates. Motivated by this, Wu \emph{et al.} \cite{Atkin2018} proposed a STS-based LDS signature set design, whose incidence matrix supports superposition based multiuser communications. By using algebraic code construction methods, the authors in \cite{Mheich2019, Xudong2021} proposed a power-imbalanced LDS design of nonzero entries for a given factor graph with the aid of Eisenstein integers\footnote{Eisenstein integers are complex number of the form $z = a+b \omega$, where $a,b \in \mathbb{N}$ and $\omega = \frac{-1 + i\sqrt{3}}{2} = e^{\frac{2\pi i}{3}}$ is a primitive cube root of unity.}. Compared to the design of conventional antipodal spreading sequences for classic CDMA, designing the LDS sequences for NOMA systems is more complicated, since the design should be implemented under the sparsity constraint of the signature matrix. In the literature, there is a little work on the design of optimal signature matrix designs that maximize the minimum Euclidean distance. 

In this paper, we study an LDS design constructed using lines and quadrics from certain finite projective planes. More explicitly, our new contributions are summarized as follows:
      \vspace{-0.0cm}
      \begin{enumerate}%[label=(\arabic*)]
      \item We propose a novel LDS design based on algebraic scheme. Explicitly, our design constructs the incidence matrices by applying Singer's Theorem.   
       \item We demonstrate that our proposed code sets achieve TSC asymptotically by comparing with the widely known Welch bound. Furthermore, we provide a proof that the maximum minimum Euclidean distance of the column vectors is $\sqrt{2}$.
      \end{enumerate}

The rest of the paper is organized as follows. In Section \ref{sec:prelimiaries}, we discuss the preliminaries. In Section \ref{lines and quadrics}, we introduce the facts from projective geometry that are required for our construction. We provide the actual design of our spreading codes in Section \ref{proposedLDS}, along with a theoretical analysis of the properties of our code matrices. \iffalse We find that the cross-correlation properties of the code matrices we propose are either optimal or asymptotically optimal.  Next, we provide our proposed detection method for AWGN channels in Section \ref{detection}. \fi After illustrating our simulation results in Section \ref{simulation}, our conclusions are drawn in Section \ref{conclusion}.
\iffalse
	The following notations are used in this paper. All boldface lower
case letters indicate column vectors and boldface upper case letters indicate
matrices, $()^T$ denotes transpose operation, \iffalse $\mathsf{sgn}$ denotes the sign function, $| . |$ is the scalar magnitude, $||\cdot||_p$ denotes ${\ell}_p$ norm, $|| \cdot || \triangleq || \cdot ||_2$ is vector norm $\mathbb{E} \{ \cdot \}$ denotes expected value, \fi $p$ denotes a prime number, $q = p^r$ denotes a power of $p$, $\mathbb{F}_q$ denotes the finite field of size $q$, and $\mathbb{F}_q^*$ denotes the multiplicative group of $\mathbb{F}_q$. \fi
\section{System Model}
First of all, perfect chip synchronization among all the transmitters is assumed. In our multiple-access system the users' symbols are multiplexed after spreading them using the LDS codes. Mathematically, we can formulate the system model as\vspace{-0.0cm}
		\begin{eqnarray}
	\label{systemModelAWGN} \mathbf{y} &=& \sum_{k = 1}^K \mathbf{c}_k d_k x_k + \mathbf{n} \nonumber \\  &=& \mathbf{C}\mathbf{D}\mathbf{x} + \mathbf{n},
	\end{eqnarray}
	\nid where $K$ is the number of the users, $d_k$ is the $k$-th user's amplitude, $x_k \in \mathcal{X}_k$ is the $k$-th user's symbol to be transmitted from the constellation alphabet, $\mathcal{X}_k$, $\mathbf{C}= [\mathbf{c}_1, \mathbf{c}_2, \dots, \mathbf{c}_K] \in \mathbb{C}^{L \times K}$ is the column-normalized LDS code matrix, $||\mathbf{c}_k|| = 1$ for $1 \leq k \leq K$, $\mathbf{n}\in \mathbb{C}^{L \times 1}$ is an $L$-dimensional complex-valued AWGN vector with variance of $\sigma^2$ and $\mathbf{D}$ is a diagonal matrix hosting the users' amplitudes. We assume that the constellation alphabet of each user is identical, i.e., $\mathcal{X}_k = \mathcal{X}$, $\forall k$ and the cardinality of the constellation is $M = | \mathcal{X} |$.

\vspace{-0.2cm}
\section{Preliminaries}
\label{sec:prelimiaries}
\iffalse
In the present work, chip synchronization among all the transmitters is assumed. This provides a worst-case estimate of the performance of what is in reality a fully asynchronous system, which only requires chip synchronization between the source transmitter and the target receiver.
 
 The spreading sequence $\mathbf{c}_k \in \mathbb{C}^{L \times 1}$ is considered to be $s$-sparse, when $s$ coefficients are non-zero and $L-s$ are zeros, with the non-zero coefficients located in $\mathcal{I}_k \subset \{1,2, ..., L\}$. In the scope of LDS design $\mathbf{c}_k$ can be considered sparse if the cardinality obeys $|\mathcal{I}_k| \leq L/2$. However, the sparsity metric is also discussed in the next section.
 \subsection{AWGN Channel}
 Mathematically, we can formulate the system model as\vspace{-0.0cm}
 \begin{eqnarray}
 \label{systemModelAWGN} \mathbf{y} &=& \sum_{k = 1}^K \mathbf{c}_k d_k x_k + \mathbf{n}, \\  &=& \mathbf{C}\mathbf{D}\mathbf{x} + \mathbf{n},
 \end{eqnarray}
 \nid where $d_k$ is the $k$-th user's amplitude, $x_k \in \mathcal{X}_k$ is the $k$-th user's symbol to be transmitted from the constellation alphabet, $\mathcal{X}_k$ \iffalse$x_i \in \{\pm1\}$ is the $i$-th user's BPSK signal\fi, $\mathbf{C}= [\mathbf{c}_1, \mathbf{c}_2, \dots, \mathbf{c}_K] \in \mathbb{C}^{L \times K}$ is the column normalized LDS code matrix, $||\mathbf{c}_k|| = 1$ for $1 \leq k \leq K$, $\mathbf{n}\in \mathbb{C}^{L \times 1}$ is an $L$-dimensional complex-valued AWGN vector with variance of $\sigma^2$ and $\mathbf{D}$ is a diagonal matrix with users' amplitude, which is given as 
 \begin{equation}
 \mathbf{D} = \begin{bmatrix}
 d_1 & 0 & \cdots & 0 \\
 0 & d_2 & 0 & \vdots \\
 \vdots & 0 & \ddots & 0  \\
 0 & \hdots & 0 & d_k
 \end{bmatrix}.
 \end{equation}
 Assume that the constellation alphabet of each users are the same, $\mathcal{X}_k = \mathcal{X}$, $\forall k$ and the cardinality of the constellation is $M = | \mathcal{X} |$. The block diagram of transmitter model of an LDS system in AWGN channel is shown in Fig.
 \fi  \iffalse 
 \subsection{Cross-correlation}
 Let $\rho_{v,\lambda} = \mathbf{c}_v \cdot \mathbf{c}_{\lambda}$ (where $\cdot$ denotes the standard Euclidean complex inner product). We say that $\rho_{v,\lambda}$ is the \emph{cross-correlation} of $\mathbf{c}_v$ and $\mathbf{c}_{\lambda}.$ \iffalse The cross-correlation is an important measure for determining the extent to which sequences in a code matrix interfere with one another.
 \fi
 Let $c_{\text{max}} = \text{max}_{v \neq \lambda}|\rho_{v,\lambda}|.$ We say that $c_{\text{max}}$ is the \emph{maximum cross-correlation} of spreading matrix $\mathbf{C}.$ The following theorem, due to L. R. Welch, gives a lower bound on the maximum cross-correlation of $\mathbf{C}.$
 \begin{theorem} \cite[Proof of Inner Products Theorem, case $k = 1$]{Welch1974} 
 \[c_{\text{max}} \geq \sqrt{\frac{1}{K-1}\left(\frac{K}{L}-1\right)},\]
 \end{theorem}
\noindent where $L$ is the length of the spreading columns and $K$ is the number of columns (users) of the matrix $\mathbf{C}$.
Let the total squared correlation (TSC) of $\mathbf{C}$ be defined as $\textsf{TSC}(\mathbf{C}) = \sum_{v = 1}^K\sum_{\lambda = 1}^K |\rho_{v,\lambda}|^2$. The following result, which is implicit in the work of L. R. Welch \cite{Welch1974}, gives a lower bound on the total squared correlation of $\mathbf{C}.$
\begin{theorem} \cite[Proof of Inner Products Theorem]{Welch1974} 
\[\textsf{TSC}(\mathbf{C}) \geq \frac{K^2}{L}.\]
\end{theorem}
\vspace{-0.9cm}  \fi
\subsection{Desiderata}
\iffalse We now list some properties which are useful for a spreading matrix $\mathbf{C}$ to possess (and which the spreading matrices we construct in this paper do, at least to some extent, possess). \fi

We are interested in ``overloaded'' spreading matrices, i.e., matrices for which $K > L$. Additionally, in order to achieve good LDS spreading sets, we would like for the maximum cross-correlation and the total squared correlation of our matrices to be low; ideally, these quantities should be as close to the Welch bounds as possible.
\iffalse 
We require the sequence vectors in our matrices to be sparse. \iffalse (in the sense specified at the beginning of Section \ref{sec:transmission})\fi It is desirable \iffalse (and it would facilitate the implementation of our scheme in practice) \fi to have nonzero entries that come from a small alphabet in our sequence vectors (e.g., $\lbrace \text{exp}(j\frac{2\pi}{T}t)\rbrace$, $t =0,1,\dots, T-1$, etc.). Although, it is important to note that when one does restrict one's attention to vectors with entries from a restricted alphabet, it is not always possible to attain the Welch bounds. \fi
\iffalse
Finally, for many applications, we want to maximize the minimum Euclidean distance between the sequence vectors in our spreading matrices.\fi
\vspace{-0.0cm}
\section{Background from projective geometry}
\label{lines and quadrics}
\subsection{Definitions and Basic Facts}
\iffalse
For proofs of the results stated in this subsection, see any introductory projective geometry textbook (for example, \cite{Beutelspacher1998}).

A \emph{projective plane} $\mathcal{P}$ is a geometry that satisfies the following axioms.\\
(i) Any two points are incident with exactly one line.\\
(ii) Any two lines intersect one another in a unique point.\\
(iii) There exist four points, no three of which are collinear.

A projective plane with a finite point set is a \emph{finite projective plane}. It can be shown that for any finite projective plane $\mathcal{P}$, there exists a number $k$ such that each line in the plane is incident with exactly $k+1$ points, each point in the plane is incident with exactly $k+1$ lines, and also such that $\mathcal{P}$ contains exactly $k^2+k+1$ points and $k^2+k+1$ lines. The number $k$ is called the \emph{order} of $\mathcal{P}$. \fi

Let $\mathcal{P}$ be a finite projective plane of order $k$. Enumerate the points and lines of $\mathcal{P}$ using the integers from the set $\lbrace 0,1,...,k^2+k \rbrace$. Then the \emph{incidence matrix} of $\mathcal{P}$ relative to this enumeration is the $(k^2+k+1)\times(k^2+k+1)$ matrix $B = [b_{ij}]$ such that $b_{ij}=1$ if point $i$ is on line $j$ and $b_{ij}=0$ otherwise.

The smallest nontrivial projective plane is the projective plane of order $2$, which is also known as the Fano plane. Relative to a certain enumeration of its points and lines, the incidence matrix for the Fano plane is as follows:
\begin{equation}
\mathbf{I}_{7}=\begin{bmatrix} 0 & 0 & 0 & 1 & 0 & 1 & 1 \\
1 & 0 & 0 & 0 & 1 & 0 & 1 \\
1 & 1 & 0 & 0 & 0 & 1 & 0 \\
0 & 1 & 1 & 0 & 0 & 0 & 1 \\
1 & 0 & 1 & 1 & 0 & 0 & 0 \\
0 & 1 & 0 & 1 & 1 & 0 & 0 \\
0 & 0 & 1 & 0 & 1 & 1 & 0 \\
\end{bmatrix}.
\end{equation}
Incidence matrices of projective planes will be our starting point for our construction of overloaded code matrices for LDS. \iffalse We will construct our code matrices by modifying and adding to the column vectors in these incidence matrices. \fi

\iffalse
\subsection{The Vector Space Construction}
Proofs of the claim made in this subsection can also be found in any standard projective geometry textbook (such as \cite{Beutelspacher1998}).

Let $q$ be a prime power. Let $V = \mathbb{F}_q^3$ be the $3$-dimensional vector space developed over $\mathbb{F}_q$. Let $P(V)$ be the incidence structure comprised of points and lines defined as follows. The points have the form 
$\overline{p}\setminus \lbrace \mathbf{0} \rbrace$ where $\overline{p}$ is a one-dimensional subspace of $V$. The lines have the form $\overline{\ell}\setminus \lbrace \mathbf{0} \rbrace$, where $\overline{\ell}$ is a two-dimensional subspace of $V$. We say that a point $\overline{p} \setminus \lbrace \mathbf{0} \rbrace$ lies on a line $\overline{\ell} \setminus \lbrace \mathbf{0} \rbrace$ if $\overline{p} \subset \overline{\ell}.$ It turns out that $P(V)$ is a projective plane of order $q$.\fi

\subsection{Singer's Theorem}
\label{Singer}
In this section, we discuss a well-known theorem first proven by J. Singer in 1938 \cite{Singer1938}. \iffalse This result is usually stated in terms of combinatorial objects called cyclic difference sets, but we will state the result in terms of incidence matrices of projective planes.  The equivalence between the standard version of the result and the version given here will be obvious to anyone who studies the proof of the result given in \cite{Singer1938}.

A square matrix is said to be \emph{circulant} if every row in the matrix can be obtained from the row above it by a right cyclic shift. Equivalently, a matrix is circulant if every column can be obtained from the column that precedes it by a downward cyclic shift. The incidence matrix, $\mathbf{I}_L$, of the Fano plane given earlier in this document is a circulant matrix. As we shall see, Singer's Theorem \cite{Singer1938} implies that it is always possible to enumerate the points and lines of a projective plane so that the resulting incidence matrix is circulant.

Recall that the multiplicative group of a finite field is cyclic. In other words, for any finite field $\mathbb{F}_q$, there exists $\alpha \in \mathbb{F}_q^*$ such that $\mathbb{F}_q^* = \langle \alpha \rangle.$ We say that $\alpha$ is a \emph{generator} (or, a \emph{primitive element}) of $\mathbb{F}_q^*.$

Recall also the trace map from a field extension onto a ground field. In the special case with which we shall be concerned, $\text{Tr}:\mathbb{F}_{q^3} \to \mathbb{F}_q$ and is defined by the rule that for $\gamma \in \mathbb{F}_q,$ $\text{Tr}(\gamma) = \gamma + \gamma^q + \gamma^{q^2}.$ It can be shown that the trace map is a linear transformation from the vector space of $\mathbb{F}_{q^3}$ over $\mathbb{F}_q$ onto the vector space of $\mathbb{F}_q$ over $\mathbb{F}_q$. Consequently, kernel of the trace, $\text{Ker(Tr)}$, is a two dimensional subspace of $\mathbb{F}_{q^3}$ over $\mathbb{F}_q$.
\renewcommand*{\thefootnote}{\fnsymbol{footnote}}
The following result (a special case of Singer's theorem) can be proven by making use of a cleverly chosen map from $\mathbb{F}_q^3$ to $\mathbb{F}_{q^3}$. \fi
\begin{theorem} \cite{Singer1938} Let $q$ be a prime power, let $V = \mathbb{F}_q^3,$ and let $\alpha$ be a generator of $\mathbb{F}_{q^3}$. Then there exists a labelling of $P(V)$ such that the resulting incidence matrix is circulant. The first column of this matrix is obtained as follows: we set the entry in row $i$, col $0$ to be $1$ if $\text{Tr}(\alpha^i)\iffalse \footnote{The trace function, $\text{Tr}(\cdot)$, is the usual trace function of the Galois extension $\mathbb{F}_{q^{(t+1)}}/\mathbb{F}_q$. }\fi = 0$ and to be $0$ otherwise.
\end{theorem}

\iffalse 
As the next example illustrates, in addition to revealing an interesting structural feature of incidence matrices, Singer's theorem also gives us an easy way to construct these matrices. \fi
\begin{equation}
\setcounter{MaxMatrixCols}{20}
\mathbf{I}_{13}= \begin{bmatrix} 1 & 0 & 0 & 0 & 1 & 0 & 0 & 0 & 0 & 0 & 1 & 0 & 1 \\
1 & 1 & 0 & 0 & 0 & 1 & 0 & 0 & 0 & 0 & 0 & 1 & 0 \\
0 & 1 & 1 & 0 & 0 & 0 & 1 & 0 & 0 & 0 & 0 & 0 & 1 \\
1 & 0 & 1 & 1 & 0 & 0 & 0 & 1 & 0 & 0 & 0 & 0 & 0 \\
0 & 1 & 0 & 1 & 1 & 0 & 0 & 0 & 1 & 0 & 0 & 0 & 0 \\
0 & 0 & 1 & 0 & 1 & 1 & 0 & 0 & 0 & 1 & 0 & 0 & 0 \\
0 & 0 & 0 & 1 & 0 & 1 & 1 & 0 & 0 & 0 & 1 & 0 & 0 \\
0 & 0 & 0 & 0 & 1 & 0 & 1 & 1 & 0 & 0 & 0 & 1 & 0 \\
0 & 0 & 0 & 0 & 0 & 1 & 0 & 1 & 1 & 0 & 0 & 0 & 1 \\
1 & 0 & 0 & 0 & 0 & 0 & 1 & 0 & 1 & 1 & 0 & 0 & 0 \\
0 & 1 & 0 & 0 & 0 & 0 & 0 & 1 & 0 & 1 & 1 & 0 & 0 \\
0 & 0 & 1 & 0 & 0 & 0 & 0 & 0 & 1 & 0 & 1 & 1 & 0\\
0 & 0 & 0 & 1 & 0 & 0 & 0 & 0 & 0 & 1 & 0 & 1 & 1
\end{bmatrix} \label{I13}
\end{equation}
We will use Singer's Theorem to construct the incidence matrix of $P(\mathbb{F}_3^3)$. Note that $x^3+2x+1$ is irreducible over $\mathbb{F}_3.$ and therefore the Galois field $\mathbb{F}_{3^3}$ can be constructed as $\mathbb{F}_{3^3} = \mathbb{F}_3[x]/\langle x^3+2x+1 \rangle$. In fact, it turns out that $x^3+2x+1$ is a primitive polynomial over $\mathbb{F}_3,$ which means that $x+\langle x^3+2x+1\rangle$ is a generator of $\mathbb{F}_{3^3}^*$ (i.e., $\alpha = x + \langle x^3+2x+1 \rangle$). The cyclic group $\mathbb{F}_{3^3}^*/\mathbb{F}_{3}^*$ induces an automorphism group of $P(\mathbb{F}_3^3)$ which acts sharply transitively on points and hyperplanes \cite{Arasu1995}.

In order to use Singer's theorem to construct our incidence matrix, we need to be able to compute $\text{Tr}(\alpha^i)$ for $0 \leq i \leq (3^2+3),$ i.e., $0 \leq i \leq 12$, as described in \cite{Arasu1995}. This can be accomplished by first constructing a dictionary between the additive and multiplicative representations of the elements of $\mathbb{F}_{27}^*.$ \iffalse  We shall repress the tedious and standard computations needed to construct this dictionary and simply state the \fi Here are the values we found for the trace function: $\text{Tr}(\alpha^0) = 0,$ $\text{Tr}(\alpha^1) = 0,$ $\text{Tr}(\alpha^2) = 2,$ $\text{Tr}(\alpha^3) = 0,$ $\text{Tr}(\alpha^4) = 2,$ $\text{Tr}(\alpha^5) = 1,$ $\text{Tr}(\alpha^6) = 2,$ $\text{Tr}(\alpha^7) = 2,$ $\text{Tr}(\alpha^8) = 1,$ $\text{Tr}(\alpha^9) = 0,$ $\text{Tr}(\alpha^{10}) = 2,$ $\text{Tr}(\alpha^{11}) = 2,$ and $\text{Tr}(\alpha^{12}) = 2$. Therefore, the incidence matrix for $P(\mathbb{F}_3^3)$ is as shown in (\ref{I13}).
\subsection{Quadrics}
\label{quadrics}
\iffalse We shall need to make use of objects from projective geometry called non-degenerate quadrics. \fi
\iffalse 
Recall that a hypersurface $V^n_{d-1}$, of order $n$, in $PG(d, F)$ is the set of points denoted as a vector $\mathbf{x} = [x_0, x_1, \dots,x_d]^T$ satisfying
\begin{equation}
    f(\mathbf{x}) =0,
\end{equation}
\noindent where $PG(d, F)$ is a projective space of dimension $d$ over the field $F$\iffalse, $F$ is any field (e.g., $F = \mathbb{F}, \mathbb{C}$ etc.)\fi, and $f(\cdot)$ is a non-zero homogeneous polynomial of degree $n$. When the order $n$ is equal to $2$, a hypersurface is called a quadric denoted by the symbol $Q$. \fi \iffalse As an example let $\mathbf{Q}$ be a symmetric matrix of size $d+1$ over $F$, the set of points $\mathbf{x} \in PG(d, F)$ satisfy the equation $\mathbf{x}^T \mathbf{Q} \mathbf{x} =0$. In the case of $d = 2$, the set of points in projective plane generally is called curve but when the order $n = 2$, it is called conic. The rank of a quadratic is defined as the rank of its defining matrix $\mathbf{Q}$. For odd $q$, the quadric $Q$ is non-degenerate if and only if $\mathbf{Q}$ is non-singular \cite{Hirschfeld1998}.  If $d$ is even, the number of points on a non-degenerate quadric of $PG(d, F)$ is $(q^d-1)/(q-1)$ \cite{Hirschfeld1998}, which is the same as the cardinality of the hyperplanes. \fi

Let $Q$ be a non-degenerate quadric in $P(\mathbb{F}_q^3)$. Define the incidence vector $\mathbf{g}_Q$ of $Q$ as follows. Let the entry in the $i$-th row equal $1$ if $\alpha^i \in Q$ and $0$ if $\alpha^i \not\in Q$.
\iffalse 
The following theorem provides a straightforward construction of the non-degenerate quadrics in projective planes. It is a version of a result that was \iffalse first\fi proven by Arasu \emph{et al.}\! in\! \cite{Arasu1995}. \fi
\begin{theorem} \cite{Arasu1995} Let $q$ be a prime power, and let $r \in (\mathbb{Z}/(q^2+q+1)\mathbb{Z})^*.$ If $q$ is even, let $r$ be such that $r^{-1} = q+1$; if $q$ is odd, let $r$ be such that $r^{-1} = 2$. Then there exists a non-degenerate quadric $Q$ in $P(\mathbb{F}_q^3)$ whose incidence vector $\mathbf{g}_Q$ is obtained as follows. Let $\mathbf{g}_{\ell}$ be the first column of the incidence matrix of $P(\mathbb{F}_q^3)$ (obtained using Singer's construction). Then the $ri$-th row entry of $\mathbf{g}_Q$ (where $ri$ is reduced modulo $q^2+q+1$) is the same as the $i$-th row entry of $\mathbf{g}_{\ell}$.
\end{theorem} \iffalse
Note that, in particular, this result implies that the non-degenerate quadric under consideration has the same number of points as a line in the projective plane. We shall use the result of Arasu \emph{et al.} to construct incidence vectors of non-degenerate quadrics in $P(\mathbb{F}_2^3)$ and in $P(\mathbb{F}_3^3)$. \fi

Let $q=2$, we have $r^{-1} = 2+1 = 3$ where $r = 3^{-1} = 5$. The first column of $\mathbf{I}_7$ is $[0, 1, 1, 0, 1, 0, 0]^T$. \iffalse The first column of the incidence matrix of the Fano plane provided earlier is $[0, 1, 1, 0, 1, 0, 0]^T$.\fi Hence, $\mathbf{g}_Q = [ 0,0,0,1,0,1,1]^T$.
\iffalse
\[ \begin{bmatrix} 0 \\
1\\
1\\
0\\
1\\
0\\
0 \end{bmatrix}. \]
Hence, 
\[I_Q = \begin{bmatrix} 0\\
0\\
0\\
1\\
0\\
1\\
1 \end{bmatrix}. \]\fi
When $q = 3$, we have $r^{-1}=2$ where $r = 2^{-1} = 7$. The first column in the incidence matrix of $P(\mathbb{F}_3^3)$ we derived earlier is $[1,1,0,1,0,0,0,0,0,1,0,0,0]^T$. So, $\mathbf{g}_Q = [ 1,0,0,0,0,0,0,1,1,0,0,1,0]^T$.
\iffalse \[ \begin{bmatrix} 1\\
1\\
0\\
1\\
0\\
0\\
0\\
0\\
0\\
1\\
0\\
0\\
0 \end{bmatrix}. \]
So, 
\[I_Q = \begin{bmatrix} 1\\
0\\
0\\
0\\
0\\
0\\
0\\
1\\
1\\
0\\
0\\
1\\
0 \end{bmatrix}.\] 
We conclude this section by citing a theorem that gives us a complete description of the sizes of the intersections between lines and quadrics in finite projective planes of the form $P(\mathbb{F}_q^3)$ \cite{Beutelspacher1998} (see \cite{Beutelspacher1998} for a proof)\fi. 
\begin{theorem} \label{intersectthm} Let $q$ be a prime power, and let $Q$ be a non-degenerate quadric in $P(\mathbb{F}_q^3)$. Then the lines of $P(\mathbb{F}_q^3)$ intersect $Q$ in sets of sizes $0,$ $1,$ and $2$ and with multiplicities $A,$ $B,$ and $C$, respectively, where 
\[A = \frac{q^2-q}{2}, \hspace{0.1in} B = q+1, \hspace{0.1in} \text{and} \hspace{0.1in} C = \frac{q^2+q}{2}.\]
\end{theorem}
\vspace{-0.4cm}
\section{Proposed LDS code design}
\label{proposedLDS}
\subsection{LDS Construction}
\label{contruction}
\iffalse We propose the following procedure for constructing overloaded code matrices for LDS system. \fi
In the following section, we describe the proposed algorithm as shown in Table \ref{LDSAlgebraicDesign}. \iffalse, which is used for designing the LDS code matrix $\mathbf{C}$. We present our proposed algorithm in Table \ref{LDSAlgebraicDesign}. \fi
    \begin{table}[h]
		\vspace{-0.0cm} \caption {}%{Noiseless Decoding Algorithm (NDA)}
		\centering  %
		%\begin{center}
		\begin{tabular}{l}
			\hline \hline \rule{0pt}{3ex} 
			\nid \textbf{LDS design algorithm}  \\
			\hline \rule{0pt}{3ex} 
			\nid \textbf{{Input}:} $L$;  \\
			\hspace{0.1cm} 1: \hspace{0.0cm} Construct incidence matrix, $\mathbf{I}_L$ \\
			\hspace{0.1cm} 2: \hspace{0.0cm} Compute $\mathbf{g}_Q$  \\ 
			\hspace{0.1cm} 3: \hspace{0.0cm}  Generate vector $\mathbf{g}_Q^{\prime}$ from $\mathbf{g}_Q$ \\%\footnotemark  \\
			\hspace{0.1cm} 4: \hspace{0.0cm} $\mathbf{C}\gets [\mathbf{I}_L \: \mathbf{g}_Q \: \mathbf{g}_Q^{\prime}]$ \\ 
			\hspace{0.1cm} 5:  \hspace{0.0cm} Negate and normalize $ \mathbf{C}$ \\
			\nid \textbf{{Output}:} $\mathbf{{C}}$ \\
			\hline
		\end{tabular}\vspace{-0.0cm}
		%\end{center}
		\label{LDSAlgebraicDesign}
	\end{table}
	%\end{center}
%\footnotetext{Randomly generate $\mathbf{C} \in \mathbb{C}^{L \times K}$, where $||\mathbf{c}_i|| =1, \forall \: \: 1 \leq i \leq K$.}
	 %= \argmin_{\substack{  \mathbf{c} \in \mathcal{C}^{L\times 1} \\ \mathbf{y}_i =\mathbf{C}\mathbf{x}_i  }} d_L (\mathbf{y}_i, \mathbf{y}_j).
	
	%d_{min} (\mathbf{C}) = \argmin_{\substack{\mathbf{x}_i,\mathbf{x}_j \in \{\pm 1\}^{K\times1} \notin \{0\}^{K\times 1} \\ \mathbf{y}_i =\mathbf{C}\mathbf{x}_i,\mathbf{y}_j =\mathbf{C}\mathbf{x}_j}} d_L (\mathbf{y}_i, \mathbf{y}_j).
	\vspace{-0.0cm}
\iffalse 	
Begin by constructing the incidence matrix of the projective plane $P(\mathbb{F}_q^3)$. Next, adjoin the incidence vector of the non-degenerate quadric $Q$ given by the construction of Arasu \emph{et al.} \cite{Arasu1995} to this incidence matrix \iffalse (on the right)\fi. Now, identify every line $\ell$ that intersects $Q$ in two points. Then, in the columns  corresponding to such lines, we negate one of the entries corresponding to either of the points of intersection. Thus, we obtain a sparse code matrix overloaded by one column in which the dot product of any two distinct columns is at most $1$.

To add another vector to our matrix, we can proceed as follows. Perform a downward cyclic shift on $\mathbf{g}_Q$ to obtain a vector $\mathbf{g}_Q^{\prime}$ that has a suitably small number of nonzero entries in common with $\mathbf{g}_Q$. It follows from the circulant structure of the incidence matrix for $P(\mathbb{F}_q^3)$ that the dot product of $\mathbf{g}_Q^{\prime}$ with the columns of the incidence matrix also equal either $0$, $1$, or $2$. If possible, at this point, the next step is to negate entries of the incidence matrix so that the dot product of $\mathbf{g}_Q^{\prime}$ with its columns is less than or equal to $1$. The final step in the process is to normalize the vectors in the resulting matrix. \fi

First, we generate the incidence matrix $\mathbf{I}_L$ of the projective plane $P(\mathbb{F}_q^3)$. Next, append $\mathbf{g}_Q$ vector to incidence matrix $\mathbf{I}_L$. Now, identify every line $\ell$ that intersects $Q$ in two points. Then, in the columns  corresponding to such lines, we negate one of the entries corresponding to either of the points of intersection. \iffalse Note the dot product of any two distinct columns of the resultant matrix is at most $1$.\fi Perform a downward cyclic shift on $\mathbf{g}_Q$ to obtain a vector $\mathbf{g}_Q^{\prime}$ that has a suitably small number of nonzero entries in common with $\mathbf{g}_Q$. It follows from the circulant structure of the incidence matrix $\mathbf{I}_L$ that the dot product of $\mathbf{g}_Q^{\prime}$ with the columns of the $\mathbf{I}_L$ also equal either $0$, $1$, or $2$. The next step is to negate entries of the incidence matrix so that the dot product of $\mathbf{g}_Q^{\prime}$ with its columns is less than or equal to $1$. \iffalse The final step in the process is to normalize the vectors in the resulting matrix.  In this document, we use the procedure described above to construct code matrices that are overloaded by two columns. But, it may be possible to push the procedure even further and construct code matrices that are more overloaded.\fi For example, the LDS code construction of size of $7\times 9$ are given as follows: \iffalse We begin by adjoining $\mathbf{g}_Q$ to the incidence matrix, $\mathbf{I}_L$, of the Fano plane and negating entries of the columns of the Fano plane so that the dot product of any two columns in the resulting matrix is less than or equal to $1$. We thereby obtain the following matrix:\fi
\[ \begin{bmatrix} 0 & 0 & 0 & 1 & 0 & 1 & 1 & 0\\
1 & 0 & 0 & 0 & 1 & 0 & 1 & 0\\
1 & 1 & 0 & 0 & 0 & 1 & 0 & 0\\
0 & -1 & -1 & 0 & 0 & 0 & 1 & 1\\
1 & 0 & 1 & 1 & 0 & 0 & 0 & 0\\
0 & 1 & 0 & 1 & -1 & 0 & 0 & 1\\
0 & 0 & 1 & 0 & 1 & 1 & 0 & 1 \end{bmatrix}. \]
Next, we obtain a downward cyclic shift (by one position) on $\mathbf{g}_Q$ to obtain the vector $\mathbf{g}_Q^{\prime} = [1,0,0,0,1,0,1]^T$. \iffalse
\[I_Q^{\prime} = \begin{bmatrix} 1\\
0\\
0\\
0\\
1\\
0\\
1 \end{bmatrix}. \]\fi
Note that $\mathbf{g}_Q \cdot \mathbf{g}_Q^{\prime} = 1.$ In the last step of the construction process adjoin $\mathbf{g}_Q^{\prime}$ to the matrix, negating entries so that the dot product of any two columns is less than or equal to $1$, and normalize all of the columns. Therefore, LDS code of size $7\times 9$ is obtained as follows:
\[ (1/\sqrt{3} )\begin{bmatrix} 0 & 0 & 0 & 1 & 0 & 1 & 1 & 0 & 1\\
1 & 0 & 0 & 0 & 1 & 0 & 1 & 0 & 0\\
1 & 1 & 0 & 0 & 0 & 1 & 0 & 0 & 0\\
0 & -1 & -1 & 0 & 0 & 0 & 1 & 1 & 0\\
1 & 0 & -1 & -1 & 0 & 0 & 0 & 0 & 1\\
0 & 1 & 0 & 1 & -1 & 0 & 0 & 1 & 0\\
0 & 0 & 1 & 0 & 1 & -1 & 0 & 1 & 1 \end{bmatrix}. \]
\iffalse The same process can be applied to transform the incidence matrix that is given at the end of Section \ref{quadrics} into a $13 \times 15$ LDS code matrix (via the incidence vector for the quadric $Q$ derived near the end of Section \ref{Singer}). The resulting matrix is given as \fi
Using our proposed algorithm in Table \ref{LDSAlgebraicDesign}, we can construct LDS code set having size of $13 \times 15$ as follows:
\setcounter{MaxMatrixCols}{20}
\[ (\tiny 1/2)\begin{bmatrix} 1 & 0 & 0 & 0 & 1 & 0 & 0 & 0 & 0 & 0 & 1 & 0 & 1 & 1 & 0 \\
1 & 1 & 0 & 0 & 0 & 1 & 0 & 0 & 0 & 0 & 0 & 1 & 0 & 0 & 1 \\
0 & 1 & 1 & 0 & 0 & 0 & 1 & 0 & 0 & 0 & 0 & 0 & 1 & 0 & 0 \\
1 & 0 & 1 & 1 & 0 & 0 & 0 & 1 & 0 & 0 & 0 & 0 & 0 & 0 & 0 \\
0 & 1 & 0 & 1 & 1 & 0 & 0 & 0 & 1 & 0 & 0 & 0 & 0 & 0 & 0 \\
0 & 0 & 1 & 0 & 1 & 1 & 0 & 0 & 0 & 1 & 0 & 0 & 0 & 0 & 0 \\
0 & 0 & 0 & 1 & 0 & 1 & 1 & 0 & 0 & 0 & 1 & 0 & 0 & 0 & 0 \\
0 & 0 & 0 & 0 & -1 & 0 & 1 & -1 & 0 & 0 & 0 & 1 & 0 & 1 & 0 \\
0 & 0 & 0 & 0 & 0 & -1 & 0 & 1 & -1 & 0 & 0 & 0 & -1 & 1 & 1 \\
-1 & 0 & 0 & 0 & 0 & 0 & 1 & 0 & 1 & -1 & 0 & 0 & 0 & 0 & 1 \\
0 & 1 & 0 & 0 & 0 & 0 & 0 & 1 & 0 & 1 & 1 & 0 & 0 & 0 & 0 \\
0 & 0 & 1 & 0 & 0 & 0 & 0 & 0 & 1 & 0 & -1 & -1 & 0 & 1 & 0 \\
0 & 0 & 0 & 1 & 0 & 0 & 0 & 0 & 0 & 1 & 0 & -1 & 1 & 0 & 1
\end{bmatrix}. \]

\subsection{Analysis}
\label{Analysis}
Let us consider the $(q^2+q+1) \times (q^2+q+2)$ matrices obtained by performing the first stage of the process outlined in Section \ref{contruction}. As for our future LDS code design development, we are working on deriving some theoretical results for the code sets having greater number of users by performing the second step (or, even by performing further iterations) of our proposed construction process. Therefore, the theoretical results of such overloaded matrices will be necessarily built upon our analysis of proposed code matrices that have lower number of users.\iffalse (less overloaded cases).\fi

\iffalse Let us consider the maximum cross-correlation of the vectors in our code matrices. Let $\mathbf{C}$ be a $L \times K$ code matrix. Suppose that each column vector in $\mathbf{C}$ has exactly $s$ nonzero entries, each of which is either $1/\sqrt{s}$ or $-1/\sqrt{s}$. Furthermore, assume that $K>L$. Thus $\mathbf{C}$ cannot be an orthogonal matrix: indeed, the maximum cross-correlation of the columns of $\mathbf{C}$ must be greater than or equal to $1/s$. \fi

Consider a $(q^2+q+1)\times (q^2+q+2)$ matrix obtained by performing the first stage of the procedure outlined in \ref{Singer}. Since any two lines in a projective plane intersect one another in exactly $1$ point, it follows that the cross-correlation of any two columns indexing lines in the plane equals $\pm 1/(q+1).$ Due to the fact that a quadric intersects a line in $0$, $1$, or $2$ points and the way we assign signs to some of the entries in our construction, it follows that the cross-correlation of $\mathbf{g}_Q$ with a column indexing a line is less than or equal to $1/(q+1)$ in absolute value. Hence, the maximum cross-correlation of our code matrix is $1/(q+1)$. Therefore, relative to the size of the alphabet used to construct our sequences and the number of nonzero entries appearing in each row and column of the matrix, the maximum cross-correlation of our code matrix is \emph{optimal}. The vectors in our code matrix have length $q^2+q+1$ and contain only $q+1$ nonzero entries. \iffalse The sparseness condition where the number of zeros must be at least half of the length of the vector is definitely satisfied. \fi The larger $q$ results in more sparseness in our proposed vectors.

To consider the total squared (TSC) correlation criteria, we examine how close our code matrices come to the Welch bound, 
\[\textsf{TSC} \geq \frac{K^2}{L} = \frac{(q^2+q+2)^2}{q^2+q+1} = \mathcal{O}(q^2).\]
Since our vectors have unit length, the correlations of the vectors with themselves contribute $q^2+q+1$ to the TSC. Consider the contributions to the TSC provided by pairs of columns corresponding to lines. Any two lines intersect in exactly one point. \iffalse Therefore, any given pair contributes $1/(q+1)^2$ to the TSC. There are 
\[{q^2+q+1 \choose 2} = \frac{(q^2+q+1)(q^2+q)}{2}\]
such pairs of lines. Altogether, correlations of pairs of columns corresponding to lines contribute 
\[\frac{(q^2+q+1)(q^2+q)}{2(q+1)^2}\]
to the TSC.\fi

Finally, consider the contribution of the correlations of $\mathbf{g}_Q$ with the columns indexing lines. For the columns indexing lines that do not intersect $Q$ or that intersect $Q$ in $2$ points, the contribution of this cross-correlation to the TSC is $0$. By Theorem \ref{intersectthm}, there are exactly $q+1$ columns indexing lines that intersect $Q$ in $1$ point. The cross-correlations of $\mathbf{g}_Q$ with these lines contribute exactly $(q+1) \cdot {1}/{(q+1)^2} = {1}/{(q+1)}$
\iffalse \[(q+1) \cdot \frac{1}{(q+1)^2} = \frac{1}{q+1}\] \fi
to the TSC. Hence, 
\[\textsf{TSC} = (q^2+q+2) + \frac{(q^2+q+1)(q^2+q)}{2(q+1)^2} + \frac{1}{q+1}\] \iffalse \[=  \frac{3}{2}q^2 + \mathcal{O}(q) = \mathcal{O}(q^2). \]\fi
Therefore, TSC of our code matrices asymptotically equal to the Welch bound.
\iffalse Thus, the TSC of our code matrices seems to be slightly larger than the Welch bound. However, it is asymptotically equal to the Welch bound. \fi

\iffalse We conclude this subsection by considering the maximum minimum Euclidean distance of the column vectors in our code matrices. \fi It is clear from the discussion about cross-correlation given above that the Hamming distance between any two vectors in the $(q^2+q+1)\times (q^2+q+2)$ code matrix constructed using our procedure is either $2(q+1)$, $2q$, or $2(q-1)$. When the Hamming distance is $2(q+1)$, the Euclidean distance is $\sqrt{{2(q+1)}/{(q+1)}} = \sqrt{2}$.
\iffalse \[\sqrt{\frac{2(q+1)}{q+1}} = \sqrt{2}.\] \fi
When the Hamming distance is $2q$, the Euclidean distance is either $\sqrt{{2q}/{(q+1)}}$ or $\sqrt{{(2q+4)}/{(q+1)}}$.
\iffalse \[\sqrt{\frac{2q}{q+1}} \: \: \text{or} \: \:  \sqrt{\frac{2q+4}{q+1}}.\]\fi
Note we have signed certain elements in the columns, when the Hamming distance is $2(q-1)$, the Euclidean distance is 
\[\sqrt{\frac{2(q-1)+4}{q+1}} = \sqrt{\frac{2(q+1)}{q+1}} = \sqrt{2}.\]
Therefore, the maximum minimum Euclidean distance of the column vectors in our code matrix is $\sqrt{2}$. Note that this Euclidean distance is not equivalent to the superimposed vectors minimum Euclidean distance.
\iffalse
\section{Multiuser Detection}
\label{detection}
\fi
\vspace{-0.2cm}
\section{Comparisons with other Algebraic LDS designs}
\label{simulation}
In this section, we assess the performance of our proposed LDS code sets against the LDS sets in \cite{Mheich2019}, for quadrature amplitude modulation (QAM) signaling. We observed that when the non-zero positions of the LDS code sets generated according to \cite{Mheich2019} are different than those of the proposed LDS code sets, the bit error rate (BER) performance of the former deteriorates substantially. Therefore, while generating the LDS codes of sizes $7 \times 9$ and $13 \times 15$ based on \cite{Mheich2019}, the non-zero entries are kept in the same positions as those in the proposed LDS sets; however, the values of the non-zero entries are obtained by following the exact procedure outlined in \cite{Mheich2019}.
%\vspace{-.4cm}
\begin{figure}[h]
\vspace{-.4cm}
	\centering
	\includegraphics[width=0.53\textwidth]{__uncodedRayleighFading04.eps}
	\caption{Uncoded BER performance of LDS and SCMA in frequency-nonselective Rayleigh fading.}\label{uncoded}
	\vspace{-.2cm}
\end{figure}
It should also be noted that, the proposed LDS code sets are not necessary uniquely decodable (UD) under QAM signaling; as such, the BER performance will have an error floor under the maximum-likelihood (ML) detection over the additive white Gaussian noise (AWGN) channel.
\begin{figure}[h]
\vspace{-.2cm}
	\centering
	\includegraphics[width=0.53\textwidth]{__CodedAWGN_03.eps}
	\caption{BER Performance of LDS and SCMA with Turbo Coding in AWGN.}\label{CodedAWGN}
	\vspace{-.1cm}
\end{figure}
Due to this reason we observed that the proposed LDS codes perform much better over the Rayleigh fading channels as opposed to AWGN channel as shown in Figs. \ref{uncoded} - \ref{CodedRayleigh}. In addition, the BER hinges not only on the minimum distance criterion, but also on the average Gaussian separability margin \cite{MichelHanzo2021}.
 Simulations were performed over the AWGN and Rayleigh fading channels with the fading rate of the symbol duration. We can apply transmitter precoding scheme for frequency selective channels \cite{MichelHanzo2021}. \iffalse The greater performance gain was achieved in Figs. \ref{CodedAWGN} - \ref{CodedRayleigh} in terms of BER when using the long-term evolution (LTE) turbo channel encoder, with the interleaver, which is based on the quadratic permutation polynomial scheme. \fi  \iffalse described in \cite{Zarrinkoub2014}.\fi In our simulations, we utilized LDS spreading codes of sizes $7 \times 9$ and $13 \times 15$, SCMA \cite{Altera5G} of size $4 \times 6$ and the corresponding information rates in case of 4QAM are $\eta_{LDS}= c_r \cdot b_s\cdot 9/7=0.86 $, and $\eta_{LDS}=c_r \cdot b_s\cdot 15/13=0.77$ bits/s/Hz, where $b_s=2$ bits/symbol and $c_r=1/3$ code rate, respectively.
 Therefore, the corresponding unrestricted Shannon limits are calculated by using the upper bound $\log_2{(1+\gamma \beta)}$ as $E_b/N_o = (2^{\eta_{LDS}}-1)/\eta_{LDS}$, $E_b/No = 0.947 \:\:(-0.24\: \text{dB})$ and  $E_b/No = 0.916 \:\:(-0.38 \:\text{dB})$ for $\eta_{LDS}=0.86$ and $\eta_{LDS}=0.77$ where $\gamma$ denotes signal-to-noise ratios (SNR) and $\beta =K/L$ denotes the overload factor, respectively.
 \begin{figure}[h]
\vspace{-.4cm}
	\centering
	\includegraphics[width=0.53\textwidth]{__CodedRayleigh_11.eps}
	\caption{BER Performance of LDS and SCMA with Turbo Coding in Frequency-nonselective Rayleigh Fading.}\label{CodedRayleigh}
	\vspace{-.2cm}
\end{figure}
Fig. \ref{uncoded} shows that when no error control coding is used, SCMA performs better than LDS using our proposed LDS codes at high $E_b/N_0$ in Rayleigh fading. However, we also note that the BER performance of our proposed LDS scheme performs better at low $E_b/N_0$. When Turbo coding with the rate of $1/3$, generator polynomials of $1+x+x^2$, $1+x^2+x^3$, a feedback connection polynomial of $1+x+x^2$ and interleaver is used; however, our proposed LDS technique outperforms SCMA in Rayleigh fading as demonstrated in Fig. \ref{CodedRayleigh}.  This is because the energy per code bit to single sided noise spectral density ratio at the input to the decoder is low. At a BER of $10^{-3}$, LDS using our proposed spreading codes provides approximately a $3\:\text{dB}$ improvement over SCMA in Rayleigh fading. In all of our simulations, we used message passage algorithm (MPA) detector for SCMA and probabilistic data association (PDA) \cite{Pattipati2004} multiuser detector for all LDS codes. The reason we used PDA detector as its performance is similar to other best low-complexity detectors. In contrast to the PDA, the MPA does not need to perform any matrix inversion, but its complexity increases exponential by both with the size of the symbol alphabet $M$ and number of non-zero positions of the spreading waveform $d_f$. Fortunately matrix inversion required by PDA can be carried out quite efficiently with the aid of the Sherman–Morrison–Woodbury formula at an overall complexity order of $\mathcal{O}(K^3)$. 
\vspace{-.3cm}
\section{Conclusion}
\label{conclusion}
In this paper, we conceived an improved low-density spreading (LDS) sequence design based on an algebraic scheme. We developed a novel LDS construction based on projective geometry. In terms of its bit error rate (BER) performance, our proposed improved LDS code set outperforms the existing LDS designs over the frequency-nonselective Rayleigh fading and additive white Gaussian noise (AWGN) channels. We demonstrated that achieving the best BER depends on the minimum distance. Our future research will consider developing a theory for the overloaded cases that has greater number of users than the proposed construction. 

\bibliographystyle{IEEEtran}
\bibliography{IEEEabrv,michelbib_file}
\end{document}


