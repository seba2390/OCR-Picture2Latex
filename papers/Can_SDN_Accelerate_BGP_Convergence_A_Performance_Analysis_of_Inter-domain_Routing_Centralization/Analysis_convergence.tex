In this section we derive results for the control plane convergence time, i.e., the time needed after a routing change till \textit{every} AS in the network has updated and established the final (i.e., shortest, conforming to routing policies) paths.

The control-plane convergence time is equal to the maximum of the $T_{SD}$ times over all the SD-paths. Due to the involved order statistics, proceeding similarly to Section~\ref{sec:data-plane}, would lead to complex computations and loose bounds. Hence, in this section, we proceed to an approximate analysis that allows us to provide useful insights for the effects of routing centralization on the BGP convergence time. %is a trade-off between accuracy and analytical tractability.

% Hence, to be able to obtain an analytical understanding (which is the main goal of this study) also on the control-plane convergence, in this section, we proceed to an approximate analysis that is a trade-off between accuracy and analytical tractability.

Specifically, we first narrow the Assumption~\ref{assumption:t-bgp}, by assuming that the renewal process for the BGP update times $T_{bgp}$ is a Poisson process; this allows to study the problem using a Markovian framework. %\red{The Poisson assumption is commonly used in the literature to approximate and study propagation processes on complex networks (e.g., epidemics, mobile networks, computer viruses).} 
Our experiments and measurements in the real Internet (Appendix~\ref{sec:distr-t-bgp}), support the selection of the Poisson assumption for the times $T_{bgp}$.

%We next derive exact results for the control-plane convergence time in a full-mesh network, and then generalize them and derive approximations for sparse networks based on a random graph model.

%While in the previous section, we studied the time needed to establish data-plane connectivity over an SD-path, the control-plane convergence time is equal to the maximum of the $T_{SD}$ times over all the SD-paths (with the same source node S, i.e., the announcing AS) in the network. Since for each routing change we need to consider $N-1$ SD-paths, whose BGP update propagation times $T_{SD}$ are correlated as well, the calculation of the control-plane convergence time becomes a very complex task, analytically intractable. Furthermore, an approach similar to Section~\ref{}, would lead to loose bounds, due to the involved order statistics (max of $N-1$ non-iid times).
%
%Therefore, to be able to obtain an analytical understanding (which is the main goal of this study) also on the effects on the control-plane convergence, in this section, we proceed to an approximate analysis that is a trade-off between accuracy and analytical tractability. Specifically, we first narrow the Assumption~\ref{assumption:t-bgp}, by assuming that the renewal process for the BGP update times $T_{bgp}$ is a Poisson process; this allows to study the problem using a Markovian framework. The Poisson assumption is commonly used in the literature to approximate and study propagation processes on complex networks (e.g., epidemics, mobile networks, computer viruses). We next derive exact results for the control-plane convergence time in a full-mesh network, and then generalize them and derive approximations for sparse networks based on a random graph model.

\begin{assumption}[BGP updates - Poisson process]\label{assumption:t-bgp-poisson}
The times $T_{bgp}$ are iid random variables, drawn from an exponential distribution with rate $\lambda = \frac{1}{\mu_{bgp}}$, and mean value $E[T_{bgp}] =~\mu_{bgp}$.
\end{assumption}

Under Assumption~\ref{assumption:t-bgp-poisson}, we can build a \textit{transient} Markov Chain  to model the propagation of BGP updates, where each state denotes the set of nodes that have updated the paths in their RIBs. However, analysing such a Markov chain is still very complex, since the state space contains $2^{N}-1$ states, and the transition rates depend on the topology of the network, which cannot be known exactly in most practical cases.

%, set of nodes in the SDN cluster, and node that initiated the routing change. In most practical cases, the exact topology of the whole network cannot be known exactly (see also discussion of Section~\ref{}), or even if it was known, the solution could be found only through numerical calculations (due to the aforementioned complexity).

%To this end, we first consider the case of a full-mesh network, which can be described by a much simpler Markov chain, and compute the control-plane convergence time as a function of the network size $N$, SDN cluster size $k$, and rate $\lambda$. Then, we generalize the results, and derive approximations for sparse topologies, which are of higher practical interest.

To this end, we first consider the case of a full-mesh network (a common approach in related literature~\cite{Kotronis-Routing-Centralization-ComNets-2015,Labozitz-Delayed-convergence-CCR-2000,convergence-properties-BGP_ComNets-2011}), which can be described by a much simpler Markov chain, and compute the control-plane convergence time as a function of the network size $N$, SDN cluster size $k$, and rate $\lambda$ (Section~\ref{sec:full-mesh}). Then, we generalize the results, and derive approximations for sparse topologies, which are of higher practical interest (Section~\ref{sec:poisson}). Simulation results show that the insights stemming from our analysis are valid also for the (much more complex) Internet AS-graph (Section~\ref{sec:control-plane-validation}).


\subsection{Analysis: Full-Mesh Topology}\label{sec:full-mesh}
%Analysis of BGP convergence in full-mesh graphs is common in related literature~\cite{Kotronis-Routing-Centralization-ComNets-2015,Labozitz-Delayed-convergence-CCR-2000,convergence-properties-BGP_ComNets-2011}, since it makes easier to obtain initial insights, and is also motivated by the recent trends on the Internet evolution (i.e., peering connections increase, flattening of Internet~\cite{Gregori-Impact-IXPs-ComCom-2011}).

In a full-mesh network, every pair of nodes has a direct connection, and, thus, the shortest path (i.e., BGP path) to each node is the direct path of size $d=1$. Hence, every node receives the BGP update from the source node. Moreover, since all nodes in the SDN cluster are informed the time any of them receives the BGP update ($T_{sdn}\ll T_{bgp}$, or $T_{sdn}\rightarrow 0$), the SDN cluster can be considered as a single node.

As a result, a Markov Chain as this in Fig.~\ref{fig:mc-steps} can be used to model the propagation of BGP updates. Each time a node (a single AS or the SDN cluster) receives the BGP update, the Markov chain moves to the next state. %\red{We start from the moment/state (time $t=0$ / state $0$) just before the routing change takes place. If the routing change is initiated by a node in the SDN cluster, the source node is the SDN cluster and $k$ nodes have the updated BGP paths. Otherwise, a single node (the source) knows about the routing change. We say that there is control-plane convergence, and denote it with the state $C$, when all nodes have the updated paths in their RIBs.} 
We start from the moment/state (time $t=0$ / state $0$) just before the routing change takes place. Control-plane convergence is achieved at state $C$, when all nodes have the updated paths in their RIBs. 

%Since in the Markov chain of Fig.~\ref{fig:mc-nodes}, to move from state $0$ to state $N$, requires exactly $N-k+1$ transitions, we can build -for simplicity- an equivalent Markov chain as in Fig.~\ref{fig:mc-steps}, where each state corresponds to the number of transitions (or, \textit{steps}) in the Markov chain of Fig.~\ref{fig:mc-nodes}, and the absorbing state $C$ corresponds to control-plane convergence (i.e., to state $N$ of Fig.~\ref{fig:mc-nodes}). The correspondence between the transition rates in the two Markov chains, can be done if we know when the first node in SDN cluster received the BGP update (i.e., the transition from a state $i$ to the state $k+i-1$ in the Markov chain of Fig.~\ref{fig:mc-nodes}). Denoting as $x$ the step that the first node in the SDN cluster receives the BGP update, we can express the transition rates of the Markov chain of Fig.~\ref{fig:mc-steps} as



%In a full-mesh network, every pair of nodes has a direct connection, and, thus, the shortest path (i.e., BGP path) to each node is the direct path of size $d=1$. As a result, a Markov Chain as this in Fig.~\ref{fig:mc-nodes} can be used to model the propagation of BGP updates. The states correspond to the number of nodes with the updated route paths in their RIBs, and transitions between states denote the propagation of the BGP update (from the source to a neighbor node). We start from the moment/state (time $t=0$ / state $0$) just before the routing change takes place. If the routing change is initiated by a node in the SDN cluster, a transition to state $k$ takes place, since all nodes in the SDN cluster are immediately informed from the controller about the update. Otherwise, a transition to state $1$ (i.e., denoting only the source node) takes place. When a node $i$ receives the BGP update from a neighbor, the number of nodes with the updated information increases by $1$ (or $k$), in the case node $i$ does not (or does) belong to the SDN cluster. We say that there is control-plane convergence, when all nodes have the updated paths in their RIBs (i.e., in state $N$). 
%
%Since in the Markov chain of Fig.~\ref{fig:mc-nodes}, to move from state $0$ to state $N$, requires exactly $N-k+1$ transitions, we can build -for simplicity- an equivalent Markov chain as in Fig.~\ref{fig:mc-steps}, where each state corresponds to the number of transitions (or, \textit{steps}) in the Markov chain of Fig.~\ref{fig:mc-nodes}, and the absorbing state $C$ corresponds to control-plane convergence (i.e., to state $N$ of Fig.~\ref{fig:mc-nodes}). The correspondence between the transition rates in the two Markov chains, can be done if we know when the first node in SDN cluster received the BGP update (i.e., the transition from a state $i$ to the state $k+i-1$ in the Markov chain of Fig.~\ref{fig:mc-nodes}). Denoting as $x$ the step that the first node in the SDN cluster receives the BGP update, we can express the transition rates of the Markov chain of Fig.~\ref{fig:mc-steps} as
%\begin{equation}
%\lambda_{i}^{'} = \left\{
%\begin{tabular}{ll}
%$\lambda_{i,i+1}+\lambda_{i,i+k}$		&$, i\leq x$\\
%$\lambda_{k+i-1, k+i}$					&$, i>x$
%\end{tabular}
%\right.
%\end{equation}

To calculate the transition rates $\lambda_{i}^{'}$, we first define the following quantities.
\begin{definition}[bgp-eligible nodes \& bgp-degree]\label{def:bgp-degree}~\\
$-$ A \texttt{bgp-eligible} node is a node the (a) has not received the BGP update, and (b) lies on a BGP (shortest) path where the previous node has the updated route in its RIB.~\\ %We denote, at a step $i$, the set of bgp-eligible nodes as $\mathcal{D}(i)$.
$-$ The \texttt{bgp-degree} at step $i$, $D(i)$, is the number of nodes that are bgp-eligible nodes.%, i.e., $D(i) = |\mathcal{D}(i)|$.
\end{definition}
Under the above definition, the time to move from a step/state $i$ to the next step/state, is the time needed till the \textit{first} of the bgp-eligible nodes receives the update. Under Assumption~\ref{assumption:t-bgp-poisson}, it follows that this time is the minimum of $D(i)$ iid random variables exponentially distributed with rate $\lambda$. Therefore the transition time is also exponentially distributed with rate (i.e., the transition rate)
\begin{equation}\label{eq:transition-rate-lambda-prime}
\lambda_{i}^{'} = \lambda\cdot D(i)
\end{equation}


Now, in a full-mesh network, bgp-eligible nodes are all the nodes that have not received the BGP update (since all nodes are directly connected to the source node). We denote as $n(i)$ the number of nodes that have received the BGP update at step $i$. From the above discussion it follows $n(i)$ depends on which step the SDN cluster received the BGP update. Denoting as $x$ the state/step that the first node in the SDN cluster receives the BGP update, we can write
\begin{equation}\label{eq:n(i)}
n(i|x) = \left\{
\begin{tabular}{ll}
$i$	& $, i\leq x$ \\
$i+k-1$	& $, i>x$
\end{tabular}
\right.
\end{equation}
and the bgp-degree is easily shown to be given by Lemma~\ref{thm:Dix-full-mesh}.
\begin{lemma}\label{thm:Dix-full-mesh}
The \bgp $\Dix$, $i\in[1,N-k], x\in[0,N-k]$, in a full-mesh network topology is given by
\begin{equation}
\Dix = N-n(i|x)
\end{equation}
\end{lemma}


Up to this point, we have calculated the transition rates of the Markov chain of Fig.~\ref{fig:mc-steps} conditionally on $x$ (see, \eq{eq:transition-rate-lambda-prime} and Lemma~\ref{thm:Dix-full-mesh}). To compute the control-plane convergence time, we need also the probabilities $P_{sdn}(x)$ that the SDN cluster receives the BGP update at step $x$. In the following lemma, we derive the expression for the probabilities $P_{sdn}(x)$.
%To compute the control-plane convergence time, we need to calculate also the probabilities $P_{sdn}(x)$ that the SDN cluster receives the BGP update at step $x$.
%Proceeding recursively, we derive the following lemma that gives the probability $P_{sdn}(x)$.

\begin{lemma}\label{thm:P-sdn}
The probability that the SDN cluster receives the update at step $x$ is given by
\begin{equation}\label{eq:P-sdn}
P_{sdn}(x) = \frac{k}{N-x}\cdot \prod_{j=0}^{x-1}\left(1-\frac{k}{N-j}\right)
\end{equation}
\end{lemma}
\begin{proof}
The proof is given in Appendix~\ref{sec:proof-of-thm-P-sdn}.
\end{proof}




%
%\begin{figure}
%\subfigure[Markov Chain (number of nodes)]{\includegraphics[width=\linewidth]{./figures/MC_nb_of_nodes1.eps}\label{fig:mc-nodes}}
%\subfigure[Markov Chain (number of transitions, or \textit{steps})]{\includegraphics[width=\linewidth]{./figures/MC_steps.eps}\label{fig:mc-steps}}
%\caption{Markov Chains where the states correspond to (a) the number of nodes that have updated BGP routes, and (b) the number of transitions, or \textit{steps}, of the BGP update dissemination process.}
%\label{fig:markov-chains}
%\end{figure}


\begin{figure}
\includegraphics[width=\linewidth]{./figures/MC_steps.eps}
\caption{Markov Chain for the BGP update dissemination process.}\label{fig:mc-steps}
\end{figure}


Now, using Lemmas~\ref{thm:Dix-full-mesh} and~\ref{thm:P-sdn}, we proceed and derive the following result for the distribution of the control-plane convergence time $T_{c}$. Specifically, Lemma~\ref{thm:MGF-Tc} gives a closed form expression for the moment generating function (MGF)\footnote{
We remind that the MGF of a random variable $X$ is defined as $M_{X}(\theta) = E[e^{\theta\cdot X}]$, $\theta\in\mathbb{R}$, and completely characterizes a random variable (equivalently to its distribution), since all the moments of $X$ can be calculated from its MGF.% as $E[X^{n}] = \left. \frac{d^{n}M_{X}(\theta)}{(d\theta)^{n}}\right|_{\theta=0}$
} of the time $T_{c}$.

%We remind that the MGF of a random variable $X$ is defined as
%\begin{equation}
%M_{X}(\theta) = E[e^{\theta\cdot X}]~~,~~~~\theta\in\mathbb{R}
%\end{equation}
%and completely characterizes a random variable (it is considered equivalent to its distribution), since all the moments of the variable $X$ can be calculated from its MGF as
%\begin{equation}\label{eq:MGF-moments}
%E[X^{n}] = \left. \frac{d^{n}M_{X}(\theta)}{(d\theta)^{n}}\right|_{\theta=0}
%\end{equation}


\begin{lemma}\label{thm:MGF-Tc}
The moment generating function (MGF) $M_{T_{c}}(\theta)$ of the BGP convergence time $T_{c}$ is given by
\begin{equation}
M_{T_{c}}(\theta) = \sum_{x=0}^{N-k}\prod_{i=1}^{N-k}\left(1-\frac{\theta}{\lambda\cdot D(i|x)}\right)^{-1} \cdot P_{sdn}(x)
\end{equation}
\end{lemma}
\begin{proof}
The proof is given in Appendix~\ref{sec:proof-of-thm-MGF-Tc}.
\end{proof}


Using the above lemma, and applying the property
\begin{equation}\label{eq:MGF-moments}
E[X^{n}] = \left. \frac{d^{n}M_{X}(\theta)}{(d\theta)^{n}}\right|_{\theta=0}
\end{equation}
we can calculate the moments of $T_{c}$. The following theorem gives the mean value (first moment) %\footnote{Higher moments can be calculated in a similar way.}
 of $T_{c}$ as a function of $D(i|x)$ (Lemma~\ref{thm:Dix-full-mesh}) and $P_{sdn}(x)$ (Lemma~\ref{thm:P-sdn}), or, equivalently, as a function of the parameters $N$, $k$, and $\lambda$. %\underline{Note:} Higher moments can be calculated in a similar way, and result to closed form expressions as well. %Here, we focus on the first two moments that are of higher practical interest.
\begin{theorem}\label{thm:expectation-and-variance-Tc}
The expectation of the BGP convergence time $T_{c}$ is
\begin{equation}
E[T_{c}] = \frac{1}{\lambda}\cdot\sum_{x=0}^{N-k}\sum_{i=1}^{N-k}\frac{1}{ D(i|x)}\cdot P_{sdn}(x)
\end{equation}
%The expectation (i.e., first moment) and second moment of the BGP convergence time $T_{c}$ is
%\begin{align}
%E[T_{c}] 	& = \frac{1}{\lambda}\cdot\sum_{x=0}^{N-k}\sum_{i=1}^{N-k}\frac{1}{ D(i|x)}\cdot P_{sdn}(x)	\\
%E[T_{c}^{2}] & = \frac{1}{\lambda^{2}}\sum_{x=0}^{N-k}\left(\sum_{i=1}^{N-k}\frac{1}{\left(D(i|x)\right)^{2}}
%				+ \left(\sum_{i=1}^{N-k}\frac{1}{D(i|x)} \right)^{2}\right)\cdot P_{sdn}(x)
%\end{align}
\end{theorem}


The methodology in the proof of Lemma~\ref{thm:MGF-Tc} can be applied to derive useful expressions for other quantities that are of practical interest, and allow us to obtain a better understanding of the effects of routing centralization on control-plane convergence. For example, the following corollary quantifies the speed of the control-plane convergence process.%(whose proof is omitted due to space limitation)


%\begin{definition}
%$\ell$-Partial BGP Convergence Time, $T_{\ell}$, is the time needed till $\ell$ ($\ell\leq N$) ASes have the final BGP updates.
%\end{definition}


\begin{corollary}\label{thm:l-partial-Tc}
The expectation of the $\ell$-Partial BGP Convergence Time, $T_{\ell}$, i.e., the time needed till $\ell$ ($\ell\leq N$) nodes have the final BGP updates,is given by
\begin{equation}
E[T_{\ell}] = \frac{1}{\lambda}\cdot \sum_{x=0}^{N-k} \sum_{i=1}^{M(\ell,x)}\frac{1}{D(i|x)}\cdot P_{sdn}(x)
\end{equation}
where
\begin{equation}
M(\ell,x) = \left\{
\begin{tabular}{ll}
$\ell-1$		&		, $~~~~~0<\ell\leq x+1$	\\
$x$			&		, $x+1<\ell\leq x+k$	\\
$\ell-k$		&		, $x+k<\ell\leq N$	\\
\end{tabular}
\right.
\end{equation}
\end{corollary}


%%% REMOVED COROLLARY %%%mew 
%
%\begin{corollary}\label{thm:convergenve-in-out-cluster}
%The expectation of the BGP convergence time for a routing change initiated by an AS \underline{in} the SDN cluster is
%\begin{equation}\label{eq:Tc-AS-in-SDN}
%E[T_{c}|AS\in SDN] = \frac{1}{\lambda}\cdot \sum_{i=1}^{N-k} \frac{1}{D(i|0)}
%\end{equation} 
%and for a routing change initiated by an AS \underline{outside} the SDN cluster is
%\begin{align}\label{eq:Tc-AS-notin-SDN}
%E[T_{c}|AS\notin SDN] = \frac{N\cdot E[T_{c}] - k\cdot E[T_{c}|AS\in SDN]}{N-k}
%\end{align} 
%%or
%%\begin{align}
%%E[T_{c}|AS\notin SDN] = \frac{1}{\lambda}\cdot \sum_{x=1}^{N-k}\sum_{i=1}^{N-k} \frac{1}{D(i|x)}\cdot P_{sdn}(x)
%%\end{align} 
%\end{corollary}






\subsection{Analysis: Sparse Topologies}\label{sec:poisson}
%As mentioned earlier, computing the control-plane convergence for an arbitrary topology is very complex. For instance, applying the methodology of Section~\ref{sec:full-mesh}, in \eq{eq:mgf-conditional-expectation} the terms in the product are not independent, since the set of bgp-eligible nodes at a step $i$ depends on the exact paths $\mathcal{P}$ that the BGP updates have been propagated. Hence, the expectation needs to be taken over all $S\in\mathcal{P}$ (with $|\mathcal{P}|\sim O\left(2^{N}\right)$), and we need to keep track of all $D(i|x,S\in\mathcal{P})$ and $P_{sdn}(x|S\in\mathcal{P})$.


%To avoid an intractable exact analysis, in this section, we derive approximations for sparse networks. To this end, we assume a Poisson (or, Erdos-Renyi) random graph $G(N,p)$, to capture the sparseness of a topology. However, as we show in the validation Section~\ref{sec:control-plane-validation}, our results describe well effects of routing centralization also in more generic/realistic topologies, like power-law or small-world graphs.
As mentioned earlier, computing the control-plane convergence for an arbitrary topology is very complex. For instance, applying the methodology of Section~\ref{sec:full-mesh}, the set of bgp-eligible nodes at a step $i$ depends on the exact paths $\mathcal{P}$ that the BGP updates have been propagated. Hence, we need to consider all $S\in\mathcal{P}$ (with $|\mathcal{P}|\sim O\left(2^{N}\right)$), and we need to keep track of all $D(i|x,S\in\mathcal{P})$ and $P_{sdn}(x|S\in\mathcal{P})$. However, approximating sparse topologies with a Poisson (or, Erdos-Renyi) random graph $G(N,p)$, we derive expressions for the BGP convergence time in the following result. As we show in the validation Section~\ref{sec:control-plane-validation}, our approximations describe well effects of routing centralization also in more generic/realistic topologies, like power-law graphs or the Internet AS-graph.






%
%Although the Poisson random graphs cannot be used always as an accurate approximation of sparse networks, our results provide insights that describe well the performance of more generic topologies, as we show in the simulations Section~\ref{}. Similar methodologies can be applied for more generic models of network topologies, e.g., with arbitrary degree distributions using the configuration model~\cite{}, however, this comes with an increased complexity in analysis. Here, we test our predictions (derived on Poisson graphs) against results on networks of more complex/realistic topologies, and show that our results can still capture the main effects of routing centralization.

%As mentioned earlier, computing the control-plane convergence for an arbitrary topology is very complex. For instance, applying the methodology of the previous section, in \eq{eq:mgf-conditional-expectation} the terms in the product are not independent, since the number of bgp-eligible nodes at a step $i$ depends on the exact paths $\mathcal{P}$ that the BGP updates have been propagated; hence, the expectation needs to be taken over all the possible combinations (which grow exponentially with the size of the network, $|\{p\in\mathcal{P}\}|\sim O\left(2^{N}\right)$) and thus we would need to keep track of the respective values of $D(i|x,p\in\mathcal{P})$ and $P_{sdn}(x|p\in\mathcal{P})$.
%
%Since an exact analysis is not tractable, in this section, we derive approximations for sparse networks. Specifically, we assume a Poisson (or, Erdos-Renyi) random graph topology, and derive the expected bgp-degree, which can be substituted in the results of Section~\ref{} to predict approximately the performance in a sparse network.
%
%Although the Poisson random graphs cannot be used always as an accurate approximation of sparse networks, our results provide insights that describe well the performance of more generic topologies, as we show in the simulations Section~\ref{}. Similar methodologies can be applied for more generic models of network topologies, e.g., with arbitrary degree distributions using the configuration model~\cite{}, however, this comes with an increased complexity in analysis. Here, we test our predictions (derived on Poisson graphs) against results on networks of more complex/realistic topologies, and show that our results can still capture the main effects of routing centralization.

%To avoid the aforementioned complexity, we approximate the propagation process with its expected behavior. I.e., we consider an average path, derive the values $D(i|x)$ and $P_{sdn}(x)$  and use them in the results of the previous section. The intuition behind this approach can be found on the \textit{Delta method}, in which the expectation of a function of a random variable is approximated by the function of the expectation of the random variable.
%
%\begin{result}\label{thm:Dix-poisson}~\\
%$-$ The expectation of the \bgp $\Dix$, $i\in[1,N-k], x\in[0,N-k]$, in a Poisson graph network topology is
%\begin{equation}
%E[\Dix] = \left(N-n(i|x)\right) \cdot \left(1-(1-p)^{n(i|x)}\right)
%\end{equation}
%$-$ The results of Lemma~\ref{thm:MGF-Tc}, Theorem~\ref{thm:expectation-and-variance-Tc}, and Corollary~\ref{thm:l-partial-Tc}, with $E[\Dix]$ (instead of $D(i|x)$), approximate the control-plane convergence time in a Poisson graph network topology.
%\end{result}

\begin{result}\label{thm:Dix-poisson} Lemma~\ref{thm:MGF-Tc}, Theorem~\ref{thm:expectation-and-variance-Tc}, and Corollary~\ref{thm:l-partial-Tc}, with $E[\Dix]$ (instead of $D(i|x)$), approximate the control-plane convergence time in a Poisson graph network topology; where $E[\Dix]$ is the expectation of the \bgp $\Dix$, $i\in[1,N-k], x\in[0,N-k]$, in a Poisson graph 
\begin{equation}
E[\Dix] = \left(N-n(i|x)\right) \cdot \left(1-(1-p)^{n(i|x)}\right)
\end{equation}
\end{result}
\begin{proof}
The proof is given in Appendix~\ref{sec:proof-of-thm-Dix-poisson}.
\end{proof}




\subsection{Simulation Results and Implications}\label{sec:control-plane-validation}
We evaluate the accuracy of our theoretical results in various simulation scenarios, including also scenarios where the assumptions for (i) exponential $f_{bgp}(t)$, and (ii) full-mesh or Poisson graph networks, do not hold. %We discuss the main findings and present a subset of results.

In scenarios of full-mesh networks, where the times $T_{bgp}$ are exponentially distributed, our theoretical expressions of Section~\ref{sec:full-mesh} predict the simulation results for the expected convergence time $E[T_{c}]$ with very high accuracy. 

For the validation of the theoretical expressions in sparse networks (Section~\ref{sec:poisson}), we simulate various sparse topologies, like Poisson, Barabasi-Albert (power low), and Newman-Watts-Strogatz (small world) graphs. Although the theoretical results are derived under the Poisson graph assumption, our simulations show that they can predict the performance with similar accuracy in the all the topologies we tested. 

%\red{Moreover, a main observation is that the accuracy of the approximation of Result~\ref{thm:Dix-poisson} is higher for well-connected (sparse) graphs, with (shortest) paths of smaller lengths. This is in accordance with the findings of Section~\ref{sec:data-plane}, where we show that the variability of the data-plane connectivity time increases with the path length.}

In Fig.~\ref{fig:Tell-vs-k} we present a representative subset of our results that demonstrate how the routing centralization can decrease the BGP convergence time. We plot the partial convergence time, normalized over the scenario without centralization, i.e. ,$\frac{E[T_{\ell}|k]}{E[T_{\ell}|k=0]}$. We consider three cases, $\ell=100$ (or, $0.1\cdot N$) in Fig.~\ref{fig:ell-100}, $\ell=500$ (or, $0.5\cdot N$) in Fig.~\ref{fig:ell-500}, and $\ell=N=1000$ that corresponds to the control-plane convergence in Fig.~\ref{fig:ell-N}. 

A first observation is that our results can capture well the relative changes\footnote{The accuracy of the theoretical results (approximations), when we consider the \textit{actual} -not normalized- values, is lower.} in the (partial) convergence time, not only for scenarios with exponential $f_{bgp}(t)$ (as we assume in our analysis), but also for scenarios with uniform $f_{bgp}(t)$.

In Fig.~\ref{fig:ell-N}, we can see that the control-plane convergence time does not significantly improve as the SDN cluster size $k$ increases. For instance, even for $k=500$ (i.e., $50\%$ of the nodes belong to the SDN cluster), the decrease in the convergence time is less than $30\%$. This comes to verify the results of~\cite{Kotronis-Routing-Centralization-ComNets-2015}, which showed that significant gains can be achieved only for high values ($>50\%$) of SDN penetration.

However, when it comes to the partial control-plane convergence (Figs.~\ref{fig:ell-100} and~\ref{fig:ell-500}), the effects of routing centralization are higher. The time needed till $10\%$  of the nodes ($\ell=100$ - Fig.~\ref{fig:ell-100}) to receive the updated routing information, decreases quickly; e.g., to $0.5$ of its no-SDN ($k=0$) value, only with $k=100$ nodes ($10\%$) participating in the SDN cluster. 

This reveals an important aspect, relating to the effects of routing centralization, which has not been shown in previous works (e.g.,~\cite{Kotronis-Routing-Centralization-ComNets-2015}): although the control-plane convergence can significantly improve only if a high percentage ($>50\%$) of nodes cooperate, we can have very large gains in the \textit{partial convergence} even with small sizes of SDN clusters.


In Fig.~\ref{fig:ell-caida-betweenness} we present simulation results on the Internet AS-graph\footnote{For scalability issues, we did not consider here stub ASes and ASes with less than 3 neighbors, resulting in a reduced Internet graph with $N=11527$.}, where the top betweenness centrality nodes form the SDN cluster. Despite the fact that the simulated scenario deviates from our assumptions, our main theoretical findings are still valid: centralization can significantly accelerate the connectivity time with a large percentage of ASes (i.e., $\ell$-partial convergence, see, e.g., curves for $\ell = 0.1\cdot N$ and $\ell = 0.5\cdot N$), while the time needed till every AS has received the updated routes (i.e., total convergence $E[T_{c}]$) improves more slowly with the SDN cluster size $k$. Moreover, we can see that the efficiency of inter-domain centralization is quite impressing; with only $k=50$ central nodes in the SDN cluster, the time needed to establish updated paths with half of the Internet nodes ($\ell=0.5\cdot N$) is $50\%$ less than in the case without centralization.




\begin{figure}
\centering
\subfigure[$\ell = 100$]{\includegraphics[width=0.9\linewidth]{./figures/fig_ET_ell_vs_k_ell100_ba_N1000_k5.eps}\label{fig:ell-100}}
\subfigure[$\ell = 500$]{\includegraphics[width=0.9\linewidth]{./figures/fig_ET_ell_vs_k_ell500_ba_N1000_k5.eps}\label{fig:ell-500}}
\subfigure[$\ell = 1000 (=N)$]{\includegraphics[width=0.9\linewidth]{./figures/fig_ET_ell_vs_k_ell_N_ba_N1000_k5.eps}\label{fig:ell-N}}
\caption{Partial convergence time, normalized over the no SDN scenario, $\frac{E[T_{\ell}|k]}{E[T_{\ell}|k=0]}$ (y-axis), vs. size of SDN cluster $k$ (x-axis). Simulation scenarios: Barabasi-Albert topology with $N=1000$ and average node degree $10$; $T_{bgp}\sim exponential(\lambda=1)$ (black line - squares) and $T_{bgp}\sim uniform(0,2)$ (blue line - circles).}
\label{fig:Tell-vs-k}
\end{figure}



\begin{figure}
\centering
\includegraphics[width=\linewidth]{./figures/fig_T_ell_CAIDA_dataset_betweenness.eps}
\caption{Partial convergence time, normalized over the no SDN scenario, $\frac{E[T_{\ell}|k]}{E[T_{\ell}|k=0]}$ (y-axis), vs. size of SDN cluster $k$ (x-axis). Simulation scenarios on the Internet AS-graph. Nodes in the SDN cluster are selected with decreasing \textit{betweenness centrality}.}
\label{fig:ell-caida-betweenness}
\end{figure}



%
%\begin{figure}
%\centering
%\subfigure[SDN cluster: random]{\includegraphics[width=\linewidth]{./figures/fig_T_ell_CAIDA_dataset.eps}\label{fig:ell-caida-random}}
%\subfigure[SDN cluster: betweenness]{\includegraphics[width=\linewidth]{./figures/fig_T_ell_CAIDA_dataset_betweenness.eps}\label{fig:ell-caida-betweenness}}
%\caption{Partial convergence time, normalized over the no SDN scenario, $\frac{E[T_{\ell}|k]}{E[T_{\ell}|k=0]}$ (y-axis), vs. size of SDN cluster $k$ (x-axis). Simulation scenarios on the Internet AS-graph. Nodes in the SDN cluster are selected (i) \textit{randomly} and (ii) with decreasing \textit{betweenness centrality}.}
%\label{fig:Tell-vs-k-caida}
%\end{figure}