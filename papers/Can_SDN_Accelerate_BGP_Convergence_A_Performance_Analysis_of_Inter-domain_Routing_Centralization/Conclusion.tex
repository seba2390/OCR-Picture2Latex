In this paper, we analytically studied the effects of inter-domain SDN on the time needed for establishing connectivity and convergence after a routing change. We proposed a probabilistic model, and derived results for the expected data-plane connectivity time (lower/upper bounds) and control-plane convergence time (exact predictions and approximations).

Our results can be used to quickly evaluate the effects of different network parameters, like network size, topology, path lengths, or number of SDN nodes, on the routing performance. Hence, they can complement previous system-oriented studies and facilitate future research. Moreover, our methodology and results can be a useful tool for studying important problems relating to routing changes in the Internet. Finally, they can be applied in practical design problems, like selecting the nodes to participate in the SDN cluster based on performance criteria (i.e., which node can have the highest impact), or for network economics purposes (e.g., detecting the potential incentives for an AS to participate in inter-domain routing centralization).

%
%the performance for example our theoretical predictions show that that the control-plane convergence for not very sparse graphs (e.g., Poisson graph with edge probability $p-0.5$) has negligible differences from a full-mesh topology. A similar observation has been made in~\cite{} where rrealistic emulations have been used. Hence a researcher, with our theory at hands, could spot the most appropriate scenarios to test their system or perform a realistic (and resource demanding) validation (e.g., emulations with real software or sth else)
%
%
%selection policies for the SDN cluster. This is not only once, since routing centralization can be used in a per application basis, and only for specific purposes, and not as a continuous solution. E.g., cooperation of different ASes to automatically mitigate BGP prefix hijacking attacks, cooperation of ASes for QoS path-stitching services, etc. Our results can be used both for selecting nodes based on performance, i.e., which node can have the highest impact, or for network economics purposes, e.g., what are the incentives for an AS to participate in an SDN cluster, i.e., how much its own performance will be improved or how much it can gain by its peers, etc.