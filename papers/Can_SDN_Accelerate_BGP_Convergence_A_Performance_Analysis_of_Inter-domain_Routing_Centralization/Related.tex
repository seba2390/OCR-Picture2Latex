Inter-domain SDN is a new research area that attracts increasing attention~\cite{Gupta-SDX-CCR-2014, Kotronis-CXP-SOSR-2016,Lin-Seamless-Internetworking-Demo-Sigcomm-2013,Thai-Decoupling-BGP-Conext-2012,Rothenberg-Revisiting-RCP-HotSDN-2012,Bennesby-Innovating-IDrouting-AINA-2014,Kotronis-Routing-Centralization-ComNets-2015}. In~\cite{Gupta-SDX-CCR-2014} authors propose and implement SDX, a software-defined component for IXPs, which increases the capabilities on routing control. Another IXP-based system that enables novel services for establishing QoS route paths is described in~\cite{Kotronis-CXP-SOSR-2016}. In~\cite{Lin-Seamless-Internetworking-Demo-Sigcomm-2013} a solution for incremental deployment of inter-domain SDN, which is seamless to traditional IP networks, is proposed, and~\cite{Thai-Decoupling-BGP-Conext-2012} contributes in this direction by proposing an SDN-based methodology for decoupling BGP policy control from routing. \cite{Rothenberg-Revisiting-RCP-HotSDN-2012} proposes an SDN-based architecture to enhance inter-domain routing, and~\cite{Bennesby-Innovating-IDrouting-AINA-2014} proposes an component to enable inter-domain SDN. Finally, authors in~\cite{Kotronis-Routing-Centralization-ComNets-2015} build a realistic emulator, and use it to investigate the effects of routing centralization on BGP convergence time.  


%outsourcing NF~\cite{Gibb-Outsourcing-NF-HotSDN-2012,Lakshminarayanan-RaaS-report-2006}

The slow convergence of BGP has been extensively studied through measurements in~\cite{Labozitz-Delayed-convergence-CCR-2000,Kushman-Can-Hear-CCR-2007,Oliveira-Quantifying-Path-Exploration-ToN-2009}. It has been shown that BGP can take several minutes to converge after a routing change, and this can cause severe packet losses~\cite{Labozitz-Delayed-convergence-CCR-2000} and performance degradation~\cite{Kushman-Can-Hear-CCR-2007}. 

Finally, analytic approaches for the BGP convergence can be found in~\cite{convergence-properties-BGP_ComNets-2011,Labozitz-Delayed-convergence-CCR-2000,stability-inter-domain-Infocom-2009}. In~\cite{convergence-properties-BGP_ComNets-2011}, a probabilistic model and automata theory is used to study the BGP convergence (probability of convergence, and convergence time). \cite{Labozitz-Delayed-convergence-CCR-2000} studies analytically the BGP convergence with respect to the number of exchanged messages, while~\cite{stability-inter-domain-Infocom-2009} performs a worst-case analysis of BGP convergence 


%based on automata theory and numerical methods and asymptotic results for convergence time 

%investigate how ASes can collaborate to improve the security of BGP, e.g., based on routing (BGP paths) information~\cite{collaborative-bgp-security-IFIPnet-2016}

%~\cite{BGP-high-SDN-techrep}


%uses a probabilistic model and automata theory to study the BGP convergence (probability of convergence, and convergence time) ~\cite{convergence-properties-BGP_ComNets-2011} based on automata theory and numerical methods and asymptotic results for convergence time 
%
%~\cite{Labozitz-Delayed-convergence-CCR-2000} also studies analytically the BGP convergence time, but as a function of the number of exchanged messages.
%
%or worst-case analysis~\cite{stability-inter-domain-Infocom-2009} of BGP convergence