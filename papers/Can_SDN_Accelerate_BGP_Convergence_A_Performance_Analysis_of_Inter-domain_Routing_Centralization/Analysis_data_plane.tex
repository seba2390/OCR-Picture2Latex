\subsection{Analysis}


A source node ``S'' announces a new IP prefix, and SD-path is the final path from S to a ``destination'' node D; see, e.g., Fig.~\ref{fig:sd-path}. Theorem~\ref{thm:sd-path-d-k} bounds the expectation of the time $T_{SD}$ needed to establish data-plane connectivity in the path.

\begin{figure}
\includegraphics[width=\linewidth]{./figures/SD_path.eps}
\caption{\textit{SD path} of size $d$. The node $n_{0}$ initiates the routing change; nodes $n_{i}$ and $n_{j}$ belong to the SDN cluster.}
\label{fig:sd-path}
\end{figure}


%
%\begin{theorem}
%The expectation of the time $T_{SD}$ needed for data plane connectivity in a path from the source node S, which initiates the routing change, to a destination node D, which is at distance $d$ AS-hops from S, is bounded as follows
%\begin{equation}
%\frac{d}{1+\frac{(d+1)\cdot k}{N}}  \leq \frac{E[T_{SD}|d]}{E[T_{bgp}]} \leq (d+1)\cdot \left(1-\frac{k}{N}\right)
%\end{equation}
%\end{theorem}

\begin{theorem}\label{thm:sd-path-d-k}
The expectation of the time $T_{SD}$ in a path of length $d$ with $k^{'} \in [0,d+1]$ nodes in the SDN cluster, is bounded as follows
\begin{equation}
\hspace{-0.08cm}LB(d,k^{'})\cdot E[T_{bgp}] \leq E[T_{SD}|d,k^{'}] \leq UB(d,k^{'})\cdot E[T_{bgp}]
\end{equation}
where 
\begin{equation}
LB(d,k^{'})= \left\{
\begin{tabular}{ll}
$0$				&, $d\leq k^{'} \leq d+1$\\
$\frac{d}{k^{'}+1}$	&, $0\leq k^{'}< d$
\end{tabular}
\right.
\end{equation}
and
\begin{equation}
UB(d,k^{'})= \left\{
\begin{tabular}{ll}
$d-k^{'}+1$	&, $2\leq k^{'}\leq d+1$\\
$d$				&, $0\leq k^{'} <2$
\end{tabular}
\right.
\end{equation}
\end{theorem}
\begin{proof}
The proof is given in Appendix~\ref{sec:proof-of-thm-sd-path-d-k}.
\end{proof}

We provide the intuition behind the proof of Lemma~\ref{thm:sd-path-d-k} in relation to Fig.~\ref{fig:sd-path}. When a node in the SD-path that belongs to the SDN cluster receives the BGP update (e.g., node $n_{i}$), then every other node in the SDN cluster (e.g., node $n_{j}$) is informed about the update, sometimes even before its preceding node(s) (e.g., $n_{j-1}$). Hence, the BGP update can propagate on \textit{different sections} of the SD-path \textit{simultaneously} (e.g,. from $n_{i}$ up to $n_{j-1}$, and -at the same time- from $n_{j}$ to $n_{d}$). The length of these SD-path sections (which determine the BGP update propagation time) depend on the positions of the SDN nodes on the path. The bounds are derived based on the ``best'' and ``worst'' possible positions of the SDN nodes on the SD-path.


%Theorem~\ref{thm:sd-path-d-k} provides bounds for the expected time needed to establish data plane connectivity between two ASes, after one of them has initiated a routing change. Given $d$ and $k^{'}$, one can calculate the range within which the data plane connectivity time is expected to be.

% the length $d$ of the path over which the BGP updates propagate (i.e., the shortest path conforming to routing policies) between the two ASes, and the number of ASes on the path that belong to the SDN cluster, one can calculate the range within which the data plane connectivity time is expected to be.


%\vspace{\baselineskip}
\subsection{Network Topology and Routing Centralization}\label{sec:topo-and-sdn}
Based on Theorem~\ref{thm:sd-path-d-k} we can calculate the average time $E[T_{SD}|d]$ over all paths of the same size $d$ (or, equivalently, for an average path of size $d$), using the property of conditional expectation:%\footnote{The same quantity, gives also the expectation of $T_{SD}$, when the exact set of nodes belonging to the SDN cluster (and, thus, $k^{'}$) is not known.}
\begin{equation}\label{eq:Tsd-conditional-k}
E[T_{SD}|d] = \sum_{i=0}^{d+1} E[T_{SD}|d,k^{'}=i] \cdot P\{k^{'}=i|d\}
\end{equation}
where $P\{k^{'}=i|d\}$ denotes the probability that $i$ nodes (out of the total $d+1$ nodes on the path) belong to the SDN cluster.

\textbf{Topology-independent SDN cluster.} If the SDN cluster is formed independently of the network topology, the quantity $k^{'}$ follows an \textit{hypergeometric distribution} with parameters $N$ (population size), $k$ (number of successes in the population), and $d+1$ (number of draws), and probability mass function
\begin{equation}\label{eq:hypergeometric-distribution}
P\{k^{'}=i|d\} = \frac{\binom{k}{i}\cdot \binom{N-k}{d+1-i}}{\binom{N}{d+1}}
\end{equation}
%and mean value
%\begin{equation}
%E[k^{'}] = \frac{(d+1)\cdot k}{N}
%\end{equation}



\textbf{Topology-related SDN cluster.} On the other hand, if the participation of ASes in the SDN cluster is related to the topology, e.g., because ASes are explicitly selected based on topological characteristics (e.g.,  centrality), or the incentives of cooperation are inherently related to their connectivity (e.g., SDN deployment on tier-1 ISPs, or IXPs~\cite{Gupta-SDX-CCR-2014,Kotronis-CXP-SOSR-2016}), then $k^{'}$ might not be captured accurately by \eq{eq:hypergeometric-distribution}. Therefore, the actual distribution $P\{d,k^{'}\}$ needs to be calculated; however, this might be a difficult (or infeasible) task. 

Alternatively, in certain cases, the distribution $P\{k^{'}=i|d\}$ could be approximated with variations of the standard hypergeometric distribution that are able to take into account the fact that different nodes appear in shortest paths with different probabilities. For instance the \textit{Fisher's noncentral hypergeometric distribution} can be used to consider biased selection of ASes for the SDN cluster: let $\omega_{i}$ be the betweenness centrality~\cite{Newman:Networks-book} of a node $n_{i}$, and $\omega_{sdn}$ and $\omega_{bgp}$ the averages among the nodes in the respective sets, i.e., 
\begin{equation*}
\omega_{sdn} = \frac{\sum_{n_{i}\in SDN} \omega_{i}}{|\{n_{i}:n_{i}\in SDN\}|}~,~~\omega_{bgp} = \frac{\sum_{n_{i}\notin SDN} \omega_{i}}{|\{n_{i}:n_{i}\notin SDN\}|}
\end{equation*}
Denoting $\omega = \frac{\omega_{sdn}}{\omega_{bgp}}$, the probability $P\{k^{'}=i\}$ is approximately given by
\begin{equation}\label{eq:fisher-distribution}
P\{k^{'}=i|d\} = \frac{\binom{k}{i}\cdot \binom{N-k}{d+1-i}\cdot \omega^{i}}{\sum_{j = 0}^{d+1}  \binom{k}{j}\cdot \binom{N-k}{d+1-j}\cdot \omega^{j}}
\end{equation}

In the above distribution, the higher the betweenness centrality of the ASes in the SDN cluster, the more skewed towards the higher values of $k^{'}$ the distribution $P\{k^{'}|d\}$ is, and, thus, the lower the delay $T_{SD}$.


\textbf{Internet AS-topology vs. SDN cluster.} We now focus on the Internet topology, which is of higher interest, and apply our -generic- theoretical results to investigate the effects of routing centralization.

We first build the Internet AS graph from a large experimentally collected dataset~\cite{AS-relationships-dataset} (consisting of $N=55567$ ASes), and infer routing policies over existing links based on the Gao-Rexford conditions~\cite{stable-internet-routing-TON-2001} (this is the most common approach in related literature; more details can be found in Appendix~\ref{sec:simulations-internet}). We consider about $10^{6}$ different SD-paths, from which we calculate the path length distribution $P\{d\}$ (see Fig.~\ref{fig:path-length-distribution}), and the betweenness centrality for each node.

We consider different scenarios with variable SDN cluster size $k = 1, ..., N$, where the set of nodes in the SDN cluster are selected (a) \textit{randomly}, or (b) based on their \textit{betweenness centrality} (i.e., the top $k$ nodes with the highest betweenness centrality values). From Theorem~\ref{thm:sd-path-d-k}, we calculate the lower and upper bounds for the average $T_{SD}$ time over all path lengths, i.e., $E[T_{SD}] = \sum_{d}E[T_{SD}|d]\cdot P\{d\}$, where $E[T_{SD}|d]$ is given by \eq{eq:Tsd-conditional-k}, and $P\{k^{'}|d\}$ from \eq{eq:hypergeometric-distribution} or \eq{eq:fisher-distribution} for the aforementioned cases (a) and (b), respectively.

In Fig.~\ref{fig:bounds-random-betweenness-CAIDA}, we present the lower (LB) and upper (UB) bounds for $E[T_{SD}]$ for different SDN cluster sizes $k$, normalized over the case without routing centralization ($k=0$). When the nodes in the SDN cluster are selected \textit{randomly}, i.e., independently of the topology, a significant decrease in the average connectivity time can be achieved only when at least $20\%$ (around $k=10000$) of the nodes participate in the SDN cluster (note the log scale of the x-axis). This observation, which is in accordance with previous findings~\cite{Kotronis-Routing-Centralization-ComNets-2015}, is a rather grim message for the efficiency of (a randomly deployed) inter-domain routing centralization, since even if a few hundreds or thousands of ASes were willing to cooperate, the gains would be marginal.

On the contrary, as is shown in Fig.~\ref{fig:bounds-random-betweenness-CAIDA}, when the SDN cluster consists of ASes with high \textit{betweenness centrality}, with only a few tens of nodes the average delay can decrease up to $50\%$. This new insight (compared to previous understanding of the effects of routing centralization) brings optimism for the feasibility of inter-domain centralization: even if deployed incrementally, e.g., starting from a few tier-1 ISPs\footnote{Large ISPs are central in the Internet topology, with high betweenness centrality. For example, the top-10 ASes with the highest betweenness centrality values belong to the list of the top-50 ASes with the largest number of ASes in customer cone~\cite{as-rank-website}}, the Internet can immediately see significant performance improvements.
%suggests that inter-domain centralization can be a feasible approach


%Our theoretical results are generic and apply to arbitrary topologies. However, it is of higher interest to investigate the effects of routing centralization on the Internet topology. To this end, 


%
%; in our case the weights of the nodes in the SDN cluster $\omega_{sdn}$ and those outside it $\omega_{bgp}$ can correspond to the average betweenness centrality among the ASes in the respective sets, i.e., $\omega_{sdn} = \frac{1}{|\{AS:AS\in SDN\}|}\sum_{AS\in SDN} BC_{AS}$, $\omega_{bgp} = \frac{1}{|\{AS:AS\notin SDN\}|}\sum_{AS\notin SDN} BC_{AS}$, and $\omega = \frac{\omega_{sdn}}{\omega_{bgp}}$. The higher the betweenness centrality of the ASes in the SDN cluster, the more skewed towards the higher values of $k^{'}$ the distribution $P\{k^{'}\}$ is, and thus, the lower the delay $T_{SD}$.
%\begin{equation}\label{eq:fisher-distribution}
%P\{k^{'}=i\} = \frac{\binom{k}{i}\cdot \binom{N-k}{d+1-i}\cdot \omega^{i}}{\sum_{j = 0}^{d+1}  \binom{k}{j}\cdot \binom{N-k}{d+1-j}\cdot \omega^{j}}
%\end{equation}

%More accurate calculations (i.e., considering betweenness centralities per AS, rather than the averages) can be done using a multivariate generalization of Fisher's noncentral hypergeometric distribution; however, this would introduce a significant computation complexity. 
%
%
%
%\begin{itemize}
%\item Given a network topology, how to select which ASes to ``recruit'' in the SDN cluster? Is it better to have many SDN-nodes in a few, e.g., the longest paths, or one SDN-node in as many paths as possible?
%\item Which set of ASes, if included in the SDN cluster, leads to better performance? or, if the size of SDN is incremented by 1, which is the next AS to be included that lead to highest increase in performance?
%\end{itemize}



%from online control plane information services/databases~\cite{ripe-stat,Cyclops,routeviews}, or through simple ping/traceroute-based measurements~\cite{ip-to-asn-mapping,traixroute}, or from known analytic results and methods for calculating the distribution of shortest paths in networks based on the (joint) degree distribution~\cite{shortest-paths-real-networks,shortest-paths-random-networks-exploration,analytic-shortest-paths-random-networks}.



\begin{figure}
\centering\includegraphics[width=\linewidth]{./figures/fig_Tsd_random_betweenness_CAIDA.eps}
\caption{Bounds for the average data-plane connectivity time, normalized over the no SDN scenario, i.e., $\frac{E[T_{SD}|k]}{E[T_{SD}|k=0]}$, in the Internet AS-graph. Upper (UB) and lower (LB) bounds enclose the colored areas: nodes in the SDN cluster are selected (i) \textit{randomly} (light grey area) and (ii) with decreasing \textit{betweenness centrality} (dark grey area).}
\label{fig:bounds-random-betweenness-CAIDA}
\end{figure}


\begin{figure}
\centering
\includegraphics[width=\linewidth]{./figures/fig_path_length_distribution.eps}
\caption{Path length distribution on the Internet AS-topology.}
\label{fig:path-length-distribution}
\end{figure}


\subsection{Simulation Results and Implications}
To validate our theoretical results, we conduct simulations on scenarios with varying (a) \textit{network topologies}: synthetic graphs such as full-mesh, Poisson graph, Barabasi-Albert (power low graph), Newman-Watts-Strogatz (small world graph), as well as, the real Internet AS-graph; (b) \textit{SDN cluster sizes}: $k=0,...,N$; and (c) \textit{distributions} $f_{bgp}(t)$: exponential with rate $\lambda=1$ and uniform in $[0,2]$, both with $\mu_{bgp}=1$. In the following we present a subset of representative results, and discuss some important observations.
%(a) network topologies: full-mesh, Poisson graph, regular, star, Barabasi-Albert (power low graph), Newman-Watts-Strogatz (small world graph); 

The average values of $T_{SD}$ in the simulations, are \textit{always} within the bounds of Theorem~\ref{thm:sd-path-d-k} for all pairs $\{d,k^{'}\}$ in every scenario we tested.

In Fig.~\ref{fig:bounds} we compare the simulation results for $E[T_{SD}|d]$ (average over all $k^{'}$) against the theoretical bounds, which are calculated from \eq{eq:Tsd-conditional-k} by using the expressions of \eq{eq:hypergeometric-distribution} (topology-independent SDN cluster) and Theorem~\ref{thm:sd-path-d-k}. For both cases of $f_{bgp}(t)$ %(exponential in Fig.~\ref{fig:bounds-exp} and uniform in Fig.~\ref{fig:bounds-uni})
, the bounds are very tight for $k=50$, when only a small fraction ($5\%$) of the nodes belong to the SDN cluster (top plots). For larger SDN cluster sizes ($k=200$, or $20\%$; bottom plots), the bounds are still very tight for small path lengths (e.g., $d<4$), while the range \textit{[lower bound, upper bound]} increases with $d$.  In summary, the accuracy of the bounds increases for smaller $k$ or $d$.%; \blue{the explanation for this can be found in the proof of Theorem~\ref{thm:sd-path-d-k}, where the uncertainty under which we consider the inequalities, decreases when $k$ or $d$ is small.}

For $k=200$ and $d=7$ (rightmost points in bottom plots), while the simulated value lies in the middle of the two bounds in the exponential $f_{bgp}(t)$ case (Fig.~\ref{fig:bounds-exp}), it is closer to the upper bound in the uniform $f_{bgp}(t)$ case (Fig.~\ref{fig:bounds-uni}). Among all the scenarios we tested, we did not observe any tendency of the values to be closer either to the upper or lower bound. This is an indication that there is probably a limit on how much tighter bounds can be derived.


In Table~\ref{table:bounds}, we show how the times $T_{SD}$ change for increasing SDN cluster size $k$. Comparing the two cases, $d=2$ and $d=5$, we can see that the effect of the routing centralization is higher for longer paths. The simulated data-plane connectivity times decrease more and faster for $d=5$, and this is captured also by the relative changes of the theoretical bounds. 

Similar behavior is observed also in Fig.~\ref{fig:t-sd-betweenness-caida}, in simulation scenarios on the Internet topology where the SDN cluster comprises nodes with high betweenness centrality. For $k=10$, paths of length $d=3$, $d=6$, and $d=9$, see a relative decrease on the average connectivity time of about $10\%$, $20\%$, and $40\%$, respectively. The corresponding values for $k=50$ are about $25\%$, $40\%$, and $60\%$ (i.e., almost double than $k=10$), while for larger SDN cluster sizes ($k>50$) the extra gain is small.

These findings (Table~\ref{table:bounds} and Fig.~\ref{fig:t-sd-betweenness-caida}) demonstrate that ASes which have (on average) longer paths to other ASes, e.g., stub networks or small ISPs at the edge of the Internet, would see a higher benefit from routing centralization than central ASes (e.g., tier-1 ISPs) or well connected ASes such as CDNs~\cite{Chiu-One-Hop-Away-IMC-2015}. Hence, the \textit{node closeness centrality}~\cite{Newman:Networks-book} can be used as a metric to evaluate (or rank) the improvement in the performance of ASes: the lower the closeness centrality, the higher the benefit from routing centralization.

The above observation sheds light on an interesting trade-off related to which nodes participate to the SDN cluster and which nodes benefit from routing centralization. As shown in Section~\ref{sec:topo-and-sdn}, nodes with high betweenness centrality improve more the performance if they participate in the SDN cluster (see, e.g., Fig.~\ref{fig:bounds-random-betweenness-CAIDA}). However, their own gain is smaller since they are central nodes in the network (betweenness and closeness centrality are positively correlated measures). As a result, incentives -other than performance- might be needed for attracting central ASes to cooperate for routing centralization. For instance, tier-1 ISPs could deploy inter-domain centralization in order to offer new services (related to the improved BGP convergence performance) to their customers.



\begin{figure}
\centering
\subfigure[$T_{bgp}\sim exponential(\lambda=1)$]{\includegraphics[width=0.45\linewidth]{./figures/fig_ET_vs_d_poisson_N1000_p0.005.eps}\label{fig:bounds-exp}}
\hspace{0.05\linewidth}
\subfigure[$T_{bgp}\sim uniform(0,2)$]{\includegraphics[width=0.45\linewidth]{./figures/fig_ET_vs_d_uniformTimes_poisson_N1000_p0.005.eps}\label{fig:bounds-uni}}
\caption{Data-plane connectivity time $E[T_{SD}|d]$ (y-axis), vs. size of network cluster $k$ (x-axis). Simulation scenarios: Poisson graph network topology of size $N=1000$ and $p=0.005$, with (a) $T_{bgp}\sim exponential(\lambda=1)$ and (b) $T_{bgp}\sim uniform(0,2)$.}
\label{fig:bounds}
\end{figure}

\begin{table*}
\centering
\caption{Data-plane connectivity time normalized over the no SDN scenario, $\frac{E[T_{SD}|k]}{E[T_{SD}|k=0]}$.}
\label{table:bounds}
\begin{tabular}{|c|cccc|}
\hline
{Upper bound / \textbf{Simulation} / Lower bound}		& {$k=20$}								& {$k=50$}		& {$k=100$}		& {$k=200$}\\
\hline
{$d=2$}				& {99.9\% / \textbf{99.2\%} / 97.0\%}	& {99.6\% / \textbf{97.7\%} / 92.5\%}	& {98.6\% / \textbf{92.9\%} / 85.1\%}	& {94.4\% / \textbf{85.1\%} / 70.4\%}	\\
{$d=5$}				& {99.9\% / \textbf{97.8\%} / 94.2\%}	& {99.3\% / \textbf{93.9\%} / 86.2\%}	& {97.4\% / \textbf{86.4\%} / 74.5\%}	& {90.1\% / \textbf{75.6\%} / 56.4\%}	\\
\hline
\end{tabular}
\end{table*}




\begin{figure}
\centering
\includegraphics[width=\linewidth]{./figures/fig_Tsd_betweenness_CAIDA.eps}
\caption{Data-plane connectivity time normalized over the no SDN scenario $\frac{E[T_{SD}|d,k]}{E[T_{SD}|d,k=0]}$ (y-axis) vs. size of network cluster $k$ (x-axis). Curves correspond to the averages for three different path lengths (i) $d=3$, (ii) $d=6$, and (iii) $d=9$, in simulation scenarios over the Internet AS-graph, with $T_{bgp}\sim exponential(\lambda=1)$, and nodes in the SDN cluster selected with decreasing \textit{betweenness centrality}.}
\label{fig:t-sd-betweenness-caida}
\end{figure}