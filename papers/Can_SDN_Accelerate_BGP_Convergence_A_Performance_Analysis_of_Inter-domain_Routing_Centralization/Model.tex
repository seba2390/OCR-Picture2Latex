\subsection{Network}
We consider a network, e.g., the whole Internet or a part of it, that consists of $N$ autonomous systems (ASes). We represent each AS as a \textit{single node} that operates as a BGP router; this abstraction that is common in related literature~\cite{Labozitz-Delayed-convergence-CCR-2000,Kotronis-Routing-Centralization-ComNets-2015}, allows to hide the details of the intra-AS structure and functionality, and focus on inter-domain routing. When two ASes are connected (transit, peering, etc., relation), we consider that a link exists between the corresponding routers, over which data traffic and BGP messages can be exchanged. %In reality, an AS can be a very large network composed of hundreds or thousands switches/routers, and extend over a large area. However, the  abstraction of our model allows to hide the details of the intra-AS structure and functionality, whose effects on inter-domain routing is less important.

%We consider a network, e.g., the whole Internet or a part of it, that consists of $N$ autonomous systems (ASes). Since we are interested in the inter-domain routing, we represent each AS as a \textit{single node} that operates as a BGP router, which is common when studying inter-domain routing~\cite{Labozitz-Delayed-convergence-CCR-2000,Kotronis-Routing-Centralization-ComNets-2015}. When two ASes are connected (transit, peering, etc., relation), we consider that a link exists between the corresponding routers, over which data traffic and BGP messages can be exchanged. In reality, an AS can be a very large network composed of hundreds or thousands switches/routers, and extend over a large area. However, the  abstraction of our model allows to hide the details of the intra-AS structure and functionality, whose effects on inter-domain routing is less important.



\subsection{SDN Cluster}
%\subsection{Inter-domain Routing Centralization}

ASes can be ISPs, enterprises, CDNs, IXPs, etc., belong to different administrative entities, and span a wide range of topological, operational, economic, etc., characteristics. As a result, not all ASes should be expected to be willing to cooperate for and/or participate in an inter-domain centralization effort. Routing centralization is envisioned to begin from a group of a few ASes cooperating with each other, e.g., at an IXP location~\cite{Gupta-SDX-CCR-2014,Kotronis-CXP-SOSR-2016}; then, more ASes could be attracted (performance or economics related incentives) to join the group, or form another group.
%a centralized inter-domain routing service

%ASes in the Internet belong to different administrative entities (e.g., ISPs, enterprises, CDNs, IXPs), spanning a wide range of topological, operational, financial, etc., characteristics. Hence, not all ASes should be expected to be willing to cooperate in order to establish a globally centralized inter-domain routing system. Routing centralization is more probable to begin from a group of a few ASes cooperating with each other, e.g., at an IXP location~\cite{}; then, more ASes could be attracted (due to incentives related to performance or financial factors) to join the group, or form another group.


To this end, we assume that $k\in[1,N]$ ASes, i.e., a fraction of the entire network, cooperate in order to centralize their inter-domain routing. In the remainder, we refer to the set of these $k$ ASes, as the \textit{SDN cluster}\footnote{Although we use the term \textit{SDN}, our framework does not require necessarily that routing centralization is implemented on an SDN architecture.}. To avoid delving into system-specific issues of the centralization implementation, we assume the following setup, which captures main characteristics of several proposed solutions(e.g.,~\cite{Kotronis-Routing-Centralization-ComNets-2015,Rothenberg-Revisiting-RCP-HotSDN-2012,fibbing-sigcomm-2015}), and is generic enough to accommodate future solutions: ASes in the SDN cluster exchange routing information with a central entity, which we call \textit{multi-domain SDN controller}. The multi-domain SDN controller might be an SDN controller that directly controls the BGP routers of the ASes (e.g., as in~\cite{Kotronis-Routing-Centralization-ComNets-2015}), or a central server that only provides information or sends BGP messages to the ASes (e.g., similar to~\cite{fibbing-sigcomm-2015}). %\blue{In the latter case, the ASes in the SDN cluster would not need to grant access to their routers, or disclose all their routing policies.}


%there exists a \textit{multi-domain SDN controller}, which is connected to (and controls) the BGP routers of all the ASes in the SDN cluster. \blue{[say that we do not necessarily mean a centralized SDN controller that controls all the routers of the members; such a service could be implemented as a centralized service (e.g., a server running some routing software) that provides information to the participants. E.g., the central server provides information about the routing changes (in a far part of the Internet) to the intra-domain SDN controller / BGP routers / etc. of its member. Then the member, which is controls its own routers, decides based on this information. In this way, no routing policies need to be disclosed etc.]}





\subsection{BGP Updates}
Each node has a routing table (Routing Information Base, RIB), in which each entry contains an IP Prefix, and the corresponding AS-path (i.e., sequence of ASes) through which this prefix can be reached. RIBs are built from the information received by the neighbor ASes: upon a routing change, the ``source'' AS (e.g., the AS that originates a prefix) sends BGP updates to its neighbors to notify them about the change; when an AS receives a BGP update, it calculates the needed updates (if any) for its RIB, and sends BGP updates to its own neighbors. In this way, BGP updates propagate over the entire network, and paths to prefixes are built in a distributed way.

%Each AS/router keeps a routing table (Routing Information Base, RIB) with information about how to route traffic to different IP prefixes. Each entry in the table includes an IP Prefix, and the corresponding AS-path, i.e., the sequence of ASes through which this prefix can be reached. The routing tables are built based on the information received by the neighbor ASes: upon a routing change, the ``source'' AS (e.g., the AS that announced a prefix) sends BGP messages/updates to its neighbors to notify them about the change; when an AS receives a BGP update, it calculates the needed updates (if any) for its RIB, and sends BGP messages to its own neighbors. In this way the BGP updates propagate over the whole network, and (shortest) paths to prefixes are built in a distributed way.


Let us assume that an AS receives a BGP update at time $t_{1}$ and forwards it to a neighbor AS at time $t_{2}$. We call \textit{BGP update time}, and denote $T_{bgp}$, the time between the reception of a BGP update in an AS and its forwarding to a neighbor AS, i.e., $T_{bgp} = t_{2}-t_{1}$. The BGP update times may vary a lot among different ASes and/or connections, since they depend on a number of parameters: routers' hardware/software (e.g., time to process BGP data and update RIB) and/or configuration (e.g., MRAI timers), intra-AS network structure (e.g., number of routers, topology) and/or operation (e.g,. iBGP configuration, intra-AS SDN), etc. 

%Knowing all these parameters and for every AS in the network is not possible. Even if we knew them, the involved complexity would not allow us to perform a tractable analysis for the propagation of the routing information. To this end, we follow a probabilistic approach to model the BGP update times. Specifically, we make the following assumption.

Knowing all these parameters for every AS is not possible, and using (upper) bounds for $T_{bgp}$ would not lead to practical conclusions~\cite{Labozitz-Delayed-convergence-CCR-2000}. Thus, to be able to perform a useful analysis, we follow a probabilistic approach, and model the BGP update times as follows.


\begin{assumption}[BGP updates - renewal process]\label{assumption:t-bgp}
The BGP update times $T_{bgp}$ are independent and identically distributed random variables, drawn from an arbitrary distribution $f_{bgp}(t)$, with $E[T_{bgp}] = \mu_{bgp}$.
%$f_{bgp}:[0,+\infty)\rightarrow [0,1]$, where $\int_{0}^{+\infty} f_{bgp}(x)dx=1$, and $E[T_{bgp}] = \mu_{bgp}$.
\end{assumption}
Under Assumption~\ref{assumption:t-bgp}, BGP update times are given by a renewal process. The model is very generic, since it allows to use any valid function $f_{bgp}(t)$, and thus describe a wide range of scenarios with different parameters. Real measurements can be used to make a realistic selection for the distribution $f_{bgp}(t)$, as we show in Appendix~\ref{sec:distr-t-bgp}; however, a detailed study for fitting the $f_{bgp}(t)$ is beyond the scope of this paper.


%\blue{[say since bgp update times are mostly related to intra-domain characteristics, the independence assumption is not far from reality(???)]}
%\blue{[see comment about autocorrelation - I'd suppose it's not a problem for a single bgp update]}



\subsection{Inter-domain SDN Routing}
Routing information in the SDN cluster propagates in a centralized way, through the multi-domain SDN controller. When an AS in the SDN cluster receives a BGP update from a neighbor AS (not in the SDN cluster), it forwards the update to the SDN controller. The SDN controller, which is aware of the topology in the SDN cluster and the connections/paths to external ASes, informs every AS in the SDN cluster about the needed changes in the routing paths. The ASes that receive the updated routes from the controller, notify their non-SDN neighbors using the standard BGP mechanism. 

%The propagation of the routing information among the ASes in the SDN cluster takes place in a centralized way, through the multi-domain SDN controller. When an AS belonging to the SDN cluster receives a BGP update from an neighbor AS (not in the SDN cluster), forwards the update to the SDN controller. The SDN controller, which is aware of the topology in the SDN cluster and the connections/paths to external ASes, calculates the needed changes in the routing tables and informs every AS in the SDN cluster. The ASes that receive the updated routes from the controller, notify their non-SDN neighbors using the standard BGP mechanism.


Let $t_{1}$ be the time that the first AS belonging to the SDN cluster receives a BGP update from a non-SDN neighbor, and $t_{2}$ the time till \textit{all} ASes in the SDN cluster have been informed (by the controller) for the BGP updates. We denote as $T_{sdn}$ the time needed for all the SDN cluster to be informed after a member has received a BGP update, i.e., $T_{sdn} = t_{2}-t_{1}$. The times $T_{sdn}$ would depend on the system implementation. However, it was shown that system designs can achieve $T_{sdn}\ll T_{bgp}$~\cite{supercharge-me-2015}. Hence, in the remainder -for the sake of presentation- we assume that $T_{sdn}\rightarrow 0$. Nevertheless, our results can be easily modified for arbitrary $T_{sdn}$ (even for cases with $E[T_{sdn}] > E[T_{bgp}]$), without this affecting the main conclusions of the study.

%Thus, we assume -for simplicity- that $T_{sdn}\rightarrow 0$. 


%For instance, in~\cite{Kotronis-Routing-Centralization-ComNets-2015} $T_{sdn}$ is not more than a few seconds, whereas the default value for MRAI timers in Cisco routers is $30sec.$.

%\blue{[Say that (a) T-{sdn} is similarly given by a distribution, (b) in some case it might be much smaller than T-bgp, (c) in the remainder we assume (for the sake of presentation) that T-sdn << T-bgp, (d) however, all our results can be easily modified for arbitrary T-sdn (e.g., even for T-sdn >> T-bgp, (e) maybe cite the TechRep where you mention what are the needed modifications, (f) the main conclusions/insights are not affected by considering Tsdn << T-bgp], (g) remove the argument of MRAI timers}

%This time can be expected to be much lower than the BGP updating process, thus, for simplicity, we assume here that $T_{sdn}=0$.
%This time can be expected to be in the order of few seconds~\cite{}, and much lower than the BGP updating process (cf., the default value for MRAI timers is Cisco routers is $30sec.$), thus, for simplicity, we assume here that $T_{sdn}=0$.

\subsection{Preliminaries and Problem Statement}
\noindent In our analysis, we consider the following setup: 

Every node in the network knows at least one (BGP) path to every other node.

A node initiates a routing change that affects the inter-domain routing (e.g., node $n_{0}$ in Fig.~\ref{fig:sd-path}). This could be an announcement or withdrawal of an IP prefix, an interruption of an AS connection (e.g., a link is down), etc. Here, we consider that a node, which we call the ``source node'', announces a new IP prefix; this routing change affects the entire network, every node will install a path for this prefix in its RIB upon the reception of the BGP update.

Nodes in the SDN cluster, receive route information from the SDN controller, and add an entry in their RIB for the prefix to the source node; even if the path is not established in the node preceding in this path (e.g., in Fig.~\ref{fig:sd-path} node $n_{j}$ might receive the update before node $n_{j-1}$). In this case only the node in the SDN cluster knows how to route traffic to the new prefix, therefore, if the SDN node sends traffic to the new prefix, this would not necessarily reach the source-node. The connectivity will be established when every AS in the path has been informed about the BGP update.

BGP updates do not propagate backwards in the path; this would create loops or longer paths, which are discarded or not preferred by BGP.

We call ``SD-path'' the final path, i.e., the shortest conforming to the routing policies, between the source node (``S'') and another node (``destination'', or ``D'').

In the remainder of the paper we investigate the effects of routing centralization on: (a) the data-plane connectivity between the source node (``S'') and any node (``D'') in the network, i.e., the time needed till all nodes in an SD-path have installed the updated BGP paths after a routing change (Section~\ref{sec:data-plane}); and (b) the control-plane convergence, i.e, the time needed till the entire network has established the final paths corresponding to the routing change (Section~\ref{sec:control-plane}).
 
%\begin{itemize}
%\item every AS in the network knows at least a path to reach every other AS
%\item an AS announces a new prefix
%\item an AS in the SDN cluster when it receives the BGP update, adds an entry in its RIB for the prefix to the source-AS; even if the path is not established in the ASes preceding in this path (in this case the SDN AS knows how to route traffic to the new prefix, although other ASes do not. Hence if the SDN AS sends traffic this would not necessarily reach the source-AS. The connectivity will be established when every AS in the path has been informed about the BGP update.
%\item no AS sends BGP updates to its neighbors backwards in the path, since this would create loops or longer paths (which are discarded or not preferred by BGP)
%\item an SD-path is the final path (i.e., the shortest conforming to the routing policies) between the nodes S and D
%\end{itemize}

For ease of reference, we summarize the notation in Table~\ref{table:important-notation}.
%For ease of reference, we summarize the main notation we use in the paper in Table~\ref{table:important-notation}.
\begin{table}
\centering
\caption{Important Notation}
\label{table:important-notation}
\begin{tabular}{|l|l|l|}
\hline
{$N$}	& {network size (total \# of nodes)}&{}\\
\hline
{$k$}	& {SDN cluster size (total \# of SDN-nodes)}&{}\\
\hline
{$T_{bgp}$}	& {BGP update time}&{}\\
\hline
{$f_{bgp}(t)$}	& {distribution of BGP update times}&{Assumption~\ref{assumption:t-bgp}}\\
\hline
{$d$}	& {path length}&{}\\
\hline
{$k^{'}$}	& {\# of SDN-nodes \textit{on a path}}&{}\\
\hline
{$T_{SD}$}	& {data-plane connectivity time in a SD-path}&{Theorem~\ref{thm:sd-path-d-k}}\\
\hline
{$T_{c}$}	& {BGP convergence time}&{Theorem~\ref{thm:expectation-and-variance-Tc}}\\
\hline
{$T_{\ell}$}	& {$\ell$-partial BGP convergence time}&{Corollary~\ref{thm:l-partial-Tc}}\\
\hline
\end{tabular}
\end{table}