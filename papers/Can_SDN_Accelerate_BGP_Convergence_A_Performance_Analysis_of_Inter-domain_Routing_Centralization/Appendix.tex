\section{Distribution of BGP update times $T_{bgp}$}\label{sec:distr-t-bgp}
To investigate if and how well our modeling assumptions can describe the BGP update times in the Internet, we compare them against real measurement data. 

We conducted experiments in the Internet using the PEERING testbed~\cite{Schlinker-PEERING-HotNets-2014}, which owns IP prefixes and ASNs, peers with networks in different locations around the world, and allows users to make real BGP announcements. In our experiments/measurements, we follow a similar methodology as in~\cite{ARTEMIS-Demo-Sigcomm-2016}: we (i) announce a /24 prefix from a site of the PEERING testbed, and (ii) use publicly available control-plane monitoring services (route collectors and looking glass servers)~\cite{bgpmon, ripe-ris,periscope} to measure the time needed till different ASes receive our announcements.

We collected BGP updates, as seen from the monitors, from $M=40$ ASes. We repeated the experiments $14$ times; each time making a BGP announcement either from the PEERING site at an IXP at Amsterdam (NL), or at an ISP at Los Angeles (US). From each received BGP update $i$, we consider (a) $T_{SD}(i)$, the time needed till the BGP update $i$ received by the monitor (i.e., timestamp of the BGP update $i$ minus the timestamp of our BGP announcement), and (b) $d(i)$, the length of the AS-path included in the BGP update $i$. 

We group the times $T_{SD}(i)$ by the respective path lengths $d(i)$, and plot the distribution (CCDF) of the measured times $T_{SD}$ in Fig.~\ref{fig:measurements-poisson-assumption} for two example cases with $d=2$ and $d=5$.

Then, we fit the real data with a distribution $f_{bgp}(t)$ (cf. Section~\ref{sec:model}), where we select $f_{bgp}(t)\sim exponential(\lambda)$ in order to test the validity of (the stronger) Assumption~\ref{assumption:t-bgp-poisson}. We estimate the \textit{average} BGP update time from the measured data as $\hat{E}[T_{bgp}] = \frac{\sum_{i}T_{SD}(i)}{\sum_{i} d(i)}$ and set the rate $\lambda = \frac{1}{\hat{E}[T_{bgp}]}$. 

We generate from $f_{bgp}(t)$ a large number of times $T_{SD}$ for paths of length $d=2$ and $d=5$, calculate their CCDFs, and compare them against the real data in Fig.~\ref{fig:measurements-poisson-assumption}. As we can observe, there is a good match between the generated and real data. This indicates that Assumption~\ref{assumption:t-bgp-poisson} is a realistic and reasonable assumption, and, thus, emphasizes the practicality of our theoretical and simulation findings in real settings

%and thus the insights stemming from our analysis and simulations will remain valid in real settings.


%\begin{equation}
%\hat{E}[T_{bgp}] = \frac{\sum_{i}T_{SD}(i)}{\sum_{i} d(i)}
%\end{equation}
%and set the rate $\lambda = \frac{1}{\hat{E}[T_{bgp}]}$.


\begin{figure}
\centering
\subfigure[$d = 2$]{\includegraphics[width=0.49\linewidth]{./figures/fig_CDF_SDtime_real_vs_exp_path_length2.eps}\label{fig:measurements-poisson-assumption-path-length-2}}
\subfigure[$d = 5$]{\includegraphics[width=0.49\linewidth]{./figures/fig_CDF_SDtime_real_vs_exp_path_length5.eps}\label{fig:measurements-poisson-assumption-path-length-5}}
\caption{CCDF of the times $T_{SD}$ for SD-paths of length (a) $d=2$ and (b) $d=5$. Comparison of times $T_{SD}$ from \textit{measurements} in the Internet (where we found $E[T_{bgp}] = 6.27$), and times $T_{SD}$ generated from our model with $f_{bgp}(t)\sim$ \textit{exponential distribution}$\left(\lambda=\frac{1}{E[T_{bgp}]}\right)$.}
\label{fig:measurements-poisson-assumption}
\end{figure}












%%%%% commented text %%%%
%\begin{comment}


%%%%%%%%%%%%%%%%%%%%%%%%%%%%%%%%%%%%%%%%%%%
%%%%%%%%%%%%%%%%%%%%%%%%%%%%%%%%%%%%%%%%%%%
%%%%%%%%%%%%%%%%%%%%%%%%%%%%%%%%%%%%%%%%%%%
\section{Proof of Theorem~\ref{thm:sd-path-d-k}}\label{sec:proof-of-thm-sd-path-d-k}


\begin{proof}
Let us assume a SD-path of length $d$ and denote the ASes/nodes in the path as $n_{0}, n_{1}, ..., n_{d}$, where $n_{0}\equiv S$ and $n_{d}\equiv D$. The total number of ASes on the SD-path is $d+1$  (including nodes S and D). Let us denote as $k^{'}$, $0\leq k^{'} \leq d+1$ the number of these nodes that belong also to the SDN cluster. 

If none of the nodes comprising the SD-path belong to the SDN cluster (i.e., $k^{'} = 0$), the BGP updates propagate from $n_{0}\equiv S$ to $n_{1}$, then from $n_{1}$ to $n_{2}$, etc., till they reach the destination node $n_{d}\equiv D$. Therefore, the time $T_{SD}$ is equal to
\begin{equation}\label{eq:Tsd-sum-d-Ti_i+1}
T_{SD} = T_{n_{0},n_{1}} + T_{n_{1},n_{2}} + ... + T_{n_{d-1},n_{d}} = \sum_{i = 0}^{d-1}T_{n_{i},n_{i+1}}
\end{equation}
and since the times $T_{n_{i},n_{i+1}}$ are iid random variables, i.e., $T_{n_{i},n_{i+1}}\sim f_{bgp}(t)$, (Assumption~\ref{assumption:t-bgp}), the expectation of $T_{SD}$ is
\begin{equation}\label{eq:Tsd-sum-d-Tbgp}
E[T_{SD}|d,k^{'}=0] = \sum_{i = 0}^{d-1}E\left[T_{n_{i},n_{i+1}}\right] = d\cdot E[T_{bgp}]
\end{equation}

Now assume that the node $n_{j}$, $j=1,...,d$, is the only node on the SD-path that belongs in the SDN cluster (i.e., $k^{'}=1$). Let $T_{1} = \sum_{i=1}^{j-1}T_{n_{i},n_{i+1}}$ be the time needed for the update to propagate from $n_{0}\equiv S$ to $n_{j}$, and $T_{2} = \sum_{i=j}^{d-1}T_{n_{i},n_{i+1}}$ the time needed for the update to propagate from $n_{j}$ to the destination $n_{d} \equiv D$.

%The BGP update will start propagating from $n_{0}\equiv S$ to $n_{1}$, then from $n_{1}$ to $n_{2}$, etc., and it reaches at time $T_{1} = \sum_{i=0}^{j-1}T_{n_{i},n_{i+1}}$ at node $n_{j}$. When $n_{j}$ is informed about the BGP update, it forwards the update to the next node in the path $n_{j+1}$; the update propagates further in the path till it reaches the destination node. Let $T_{2} = \sum_{i=j}^{d-1}T_{n_{i},n_{i+1}}$ be the time needed for the update to propagate from $n_{j}$ to the destination.

The node $n_{j}$ is first informed about the BGP update at time $T^{'}\leq T_{1}$: either from the previous node in the path ($T^{'}=T_{1}$), or at an earlier time ($T^{'}<T_{1}$) from the SDN cluster, if the SDN cluster has received (through another path) the BGP update earlier.
%\footnote{The BGP updates propagate only towards one direction on a path (from node S to D). Hence, even if $n_{j}$ receives the update (from the SDN controller) earlier than node $n_{j-1}$, the update is not forwarded to (or accepted from) $n_{j-1}$ since this would create a loop or lead to a non-shortest path.}. 

Therefore, the total time needed for all the nodes in the SD-path to receive the BGP update can be expressed as
\begin{equation}\label{eq:Tsd-definition}
T_{SD} = max\{T_{1} , T^{'}+T_{2}\}
\end{equation}

\vspace{\baselineskip}
\noindent\underline{Lower Bound:}

To derive the lower bound of the expectation of $T_{SD}$, we take the expectations on \eq{eq:Tsd-definition} and proceed as follows.
\begin{align}
E[T_{SD}|d,k^{'}=1] 
&= E\left[max\{T_{1} , T^{'}+T_{2}\}\right] \\
&\geq E\left[max\{T_{1} , T_{2}\}\right] \label{eq:Tprime-0}\\
&\geq max\left\{E[T_{1}] , E[T_{2}]\right\} \label{eq:Expectation-vs-Max}\\
&= max\left\{E[T_{bgp}]\cdot d_{1} , E[T_{bgp}]\cdot d_{2}\right\} \label{eq:Expectations-times-ETbgp}\\
&= E[T_{bgp}]\cdot max\left\{ d_{1} ,  d_{2}\right\} \\
&\geq E[T_{bgp}]\cdot \displaystyle\min_{d_{1},d_{2}}\left\{max\left\{ d_{1} ,  d_{2}\right\}\right\} \label{eq:min-d1-d2}
\end{align}
%\\& = E[T_{bgp}]\cdot \frac{d}{2} \label{eq:d1-d2-dover2}
which gives
\begin{equation}\label{eq:d1-d2-dover2}
E[T_{SD}|d,k^{'}=1] ~~\geq~~ \left\{
\begin{tabular}{lc}
$\displaystyle 0$				&, $d=1$\\
$\displaystyle E[T_{bgp}]\cdot \frac{d}{2}$	&, $d>1$
\end{tabular}
\right.
\end{equation}
where 
\begin{itemize}
\item \eq{eq:Tprime-0} follows since $T^{'}\geq 0$ ($T^{'}=0$ denotes the event that the SDN cluster receives the BGP update immediately after the routing change takes place).

\item The inequality of \eq{eq:Expectation-vs-Max} follows since the times $T_{1}$ and $T_{2}$ are independent random variables, and thus it holds
\begin{align*}
P\{max\{T_{1},T_{2}\}\leq t\} 	&= P\{T_{1}\leq t\}\cdot P\{T_{2}\leq t\}~~~\Rightarrow\\
P\{max\{T_{1},T_{2}\}\leq t\} 	&\leq P\{T_{i}\leq t\}~~~\Rightarrow\\
%1- P\{max\{T_{1},T_{2}\}\leq t\}	&\geq 1-P\{T_{i}\leq t\}~~~\Rightarrow\\
P\{max\{T_{1},T_{2}\}> t\} 		&\geq P\{T_{i}> t\},~~~~\forall i=\{1,2\}
\end{align*}
and for a positive r.v. $X$ it also holds that $E[X]=\int_{0}^{\infty}P\{X>x\}dx$, and thus taking the integral in the above inequality it follows
\begin{align*}
\int_{0}^{\infty}P\{max\{T_{1},T_{2}\}> t\}dt 	&\geq \int_{0}^{\infty}P\{T_{i}> t\}dt~\Rightarrow\\
E[max\{T_{1},T_{2}\}] 						&\geq E[T_{i}],~~~~\forall i=\{1,2\}
\end{align*}
or, equivalently, $E[max\{T_{1},T_{2}\}] \geq max\left\{E[T_{i}]\right\}$.

\item The expectations $E[T_{i}], ~i=\{1,2\}$ are substituted in \eq{eq:Expectations-times-ETbgp} with $E[T_{bgp}]\cdot d_{i}$ since $T_{i}$ is the sum of $d_{i}$ iid r.v. with expected value $E[T_{bgp}]$.

\item In \eq{eq:min-d1-d2} we consider all the possible combinations of $d_{1}$ and $d_{2}$ (under the condition $d_{1}+d_{2}=d$), whose max value is minimized when $d_{1}=d_{2}=\frac{d}{2}$ (\eq{eq:d1-d2-dover2}).
\end{itemize}

Now, if there are $k^{'}$ nodes in the SD-path that belong to the SDN cluster, proceeding similarly to the above case $k^{'}=1$ leads to the following generic inequality
\begin{equation}
E[T_{SD}|d,k^{'}] ~~\geq~~ \left\{
\begin{tabular}{lc}
$ 0$				&, $d\leq k^{'}$\\
$ E[T_{bgp}]\cdot \frac{d}{k^{'}+1}$	&, $d>k^{'}$
\end{tabular}
\right.
\end{equation}
which gives the lower bound of Theorem~\ref{thm:sd-path-d-k}.
%\begin{align}
%E[T_{SD}|d,k^{'}] = E[T_{bgp}]\cdot \frac{d}{k^{'}+1} \label{eq:d1-d2-dover2}
%\end{align}


\vspace{\baselineskip}
\noindent\underline{Upper Bound:}

For $k^{'}=0$, the expectation of $T_{SD}$ is given by \eq{eq:Tsd-sum-d-Tbgp}. For $k^{'}=1$, since $T^{'}\leq T_{1}$, we can use \eq{eq:Tsd-definition} and write
\begin{align}
E[T_{SD}|d,k^{'}=1] 
&= E\left[max\{T_{1} , T^{'}+T_{2}\}\right] \\
&\leq E\left[max\{T_{1} , T_{1}+T_{2}\}\right] \\
&= d\cdot E[T_{bgp}] \label{eq:upper-bound-k=1}
\end{align}
where the last equality follows from \eq{eq:Tsd-sum-d-Tbgp}.

In the case of $k^{'}>1$, it is probable that, after the SDN cluster is informed about the routing change, the BGP update propagates simultaneously on more than one sections on the SD-path. For example, in Fig.~\ref{fig:sd-path}, after the SDN cluster is informed ($n_{i}$ and $n_{j}$ receive the update at the same time), the BGP update will propagate \textit{simultaneously} in the sub-paths $n_{i}\rightarrow ...\rightarrow n_{j-1}$ and $n_{j}\rightarrow ...\rightarrow n_{d}$. This, accelerates the propagation process, and, thus, decreases the time $T_{SD}$. 

It is easy to see, that the smaller decrease (on average) on $T_{SD}$, will take place when the $k^{'}$ nodes that belong to the SDN cluster are located consecutively on the SD-path. Without loss of generality, let assume that the first $k^{'}$ nodes $n_{0},...,n_{k^{'}-1}$ are the nodes that belong to the SDN cluster, and denote the time $T_{SD}$ for this (worst) case as $T_{SD}^{max}$. Then, the time $T_{SD}^{max}$ is given by
\begin{align}
T_{SD}^{max} 	&= \sum_{i=0}^{k^{'}-2}T_{n_{i},n_{i+1}} + \sum_{i=k^{'}-1}^{d-1}T_{n_{i},n_{i+1}} = \sum_{i=k^{'}-1}^{d-1}T_{n_{i},n_{i+1}}
\end{align}
since $\sum_{i=0}^{k^{'}-2}T_{n_{i},n_{i+1}} = T_{sdn}\equiv 0$. The expectation of $T_{SD}^{max}$ is derived similarly to \eq{eq:Tsd-sum-d-Ti_i+1} and \eq{eq:Tsd-sum-d-Tbgp}, i.e., 
\begin{equation}\label{eq:T-sd-max}
E[T_{SD}^{max}] = E\left[\sum_{i=k^{'}-1}^{d-1}T_{n_{i},n_{i+1}}\right] = \left(d-(k^{'}-1)\right)\cdot E[T_{bgp}]
\end{equation}

Combining \eq{eq:Tsd-sum-d-Tbgp}, \eq{eq:upper-bound-k=1}, and \eq{eq:T-sd-max}, gives the upper bound of Theorem~\ref{thm:sd-path-d-k}.
\end{proof}





%%%%%%%%%%%%%%%%%%%%%%%%%%%%%%%%%%%%%%%%%%%
%%%%%%%%%%%%%%%%%%%%%%%%%%%%%%%%%%%%%%%%%%%
%%%%%%%%%%%%%%%%%%%%%%%%%%%%%%%%%%%%%%%%%%%

\section{Proof of Lemma~\ref{thm:P-sdn}}\label{sec:proof-of-thm-P-sdn}

\begin{proof}
Considering all the cases for which node initiates the routing change, the probability that the source node belongs to the SDN cluster (and thus $x=0$) is
\begin{equation}
P_{sdn}(0)\equiv P_{sdn}(x=0) = \frac{k}{N}
\end{equation}
If the source node does not belong to the SDN cluster, then at step $1$ there are $N-1$ bgp-eligible nodes, of which $k$ belong to the SDN cluster. This gives
\begin{equation}
P_{sdn}(1~|x>0) = \frac{k}{N-1}
\end{equation}
and, consequently,
\begin{align*}
P_{sdn}(1) = P_{sdn}(1~|x>0)\cdot P_{sdn}(x>0) = \textstyle \frac{k}{N-1}\cdot \left(1-\frac{k}{N}\right)
\end{align*}
%\begin{align*}
%P_{sdn}(1) &= P_{sdn}(1~|x>0)\cdot P_{sdn}(x>0) \\
%			& = P_{sdn}(x=1|x>0)\cdot \left(1-P_{sdn}(x=0)\right) \\
%			& = \textstyle \frac{k}{N-1}\cdot \left(1-\frac{k}{N}\right)
%\end{align*}
Proceeding recursively, we derive \eq{eq:P-sdn} that gives the probability $P_{sdn}(x)$.
\end{proof}



%%%%%%%%%%%%%%%%%%%%%%%%%%%%%%%%%%%%%%%%%%%
%%%%%%%%%%%%%%%%%%%%%%%%%%%%%%%%%%%%%%%%%%%
%%%%%%%%%%%%%%%%%%%%%%%%%%%%%%%%%%%%%%%%%%%
\section{Proof of Theorem~\ref{thm:MGF-Tc}}\label{sec:proof-of-thm-MGF-Tc}

\begin{proof}
The convergence time is $T_{c}$ is calculated by the sum of the transition times of the Markov Chain of Fig.~\ref{fig:mc-steps}, i.e., 
\begin{equation}
T_{c} = T_{1,2} + T_{2,3} + ... + T_{N-k,C} = \sum_{i =1}^{N-k} T_{i,i+1}
\end{equation}
where we denote $T_{N-k,N-k+1}\equiv T_{N-k,C}$. Hence, the MGF of $T_{c}$ is expressed as
\begin{align}
M_{T_{c}}(\theta) 
& = E\left[e^{\theta\cdot \sum_{i =1}^{N-k} T_{i,i+1}}\right] \\
& = E\left[\prod_{i =1}^{N-k} e^{\theta\cdot T_{i,i+1}}\right] \label{eq:mgf-expectation-of-product}\\
& = \sum_{x=0}^{N-k} E\left[\prod_{i =1}^{N-k} e^{\theta\cdot T_{i,i+1}} \Big| x\right] \cdot P_{sdn}(x) \label{eq:mgf-conditional-expectation}\\
& = \sum_{x=0}^{N-k} \prod_{i =1}^{N-k} E\left[ e^{\theta\cdot T_{i,i+1}} \Big| x\right] \cdot P_{sdn}(x) \label{eq:mgf-independency}\\
& = \sum_{x=0}^{N-k} \prod_{i =1}^{N-k} \left(1-\frac{\theta}{\lambda \cdot D(i|x)}\right)^{-1} \cdot P_{sdn}(x) \label{eq:mgf-exponential-variable}
\end{align}
where 
\begin{itemize}
\item In \eq{eq:mgf-conditional-expectation} we consider the conditional expectation, given that the SDN cluster receives the update at step~$x$.

\item  \eq{eq:mgf-independency} follows from the fact that the times $T_{i,i+1}$ are independent under a given $x$; due to Assumption~\ref{assumption:t-bgp-poisson}, they depend only on the number of infected nodes, which is determined by the step $i$ and the value of $x$. 

\item We derive \eq{eq:mgf-exponential-variable}, since $T_{i,i+1}$ is an exponential random variable with rate $\lambda_{i,i+1}^{'} = \lambda \cdot D(i|x)$, and the MGF of an exponential r.v. with rate $\mu$ is given by $\left(1-{\theta}/{\mu}\right)^{-1}$. 
\end{itemize}
\end{proof}


%%%%%%%%%%%%%%%%%%%%%%%%%%%%%%%%%%%%%%%%%%%
%%%%%%%%%%%%%%%%%%%%%%%%%%%%%%%%%%%%%%%%%%%
%%%%%%%%%%%%%%%%%%%%%%%%%%%%%%%%%%%%%%%%%%%
\section{Proof of Result~\ref{thm:Dix-poisson}}\label{sec:proof-of-thm-Dix-poisson}

\begin{proof}
To derive the MGF of $T_{c}$ we apply the methodology in the proof of Lemma~\ref{thm:MGF-Tc}; here, we highlight only the key points and differences from the full-mesh case. 
\begin{align}
&M_{T_{c}}(\theta) 
 = E\left[\prod_{i =1}^{N-k} e^{\theta\cdot T_{i,i+1}}\right] \label{eq:mgf-expectation-of-product-poisson}\\
& = \sum_{S\in\mathcal{P}}\sum_{x=0}^{N-k} E\left[\prod_{i =1}^{N-k} e^{\theta\cdot T_{i,i+1}} \Big| x,S\right] \cdot P\{x,S\} \label{eq:mgf-conditional-expectation-poisson}\\
& = \sum_{x=0}^{N-k}\sum_{S\in\mathcal{P}} E\left[\prod_{i =1}^{N-k} e^{\theta\cdot T_{i,i+1}} \Big| x,S\right] \cdot P\{S\}\cdot P_{sdn}(x) \label{eq:mgf-conditional-expectation-independent-Psdn-poisson}\\
& = \sum_{x=0}^{N-k}\sum_{S\in\mathcal{P}} \prod_{i =1}^{N-k} \left(1-\frac{\theta}{\lambda \cdot D(i|x,S)}\right)^{-1} \cdot P\{S\}\cdot P_{sdn}(x) \label{eq:mgf-conditional-expectation-product-poisson}\\
& = \sum_{x=0}^{N-k} E\left[ \prod_{i =1}^{N-k} \left(1-\frac{\theta}{\lambda \cdot D(i|x,S)}\right)^{-1} \right]\cdot P_{sdn}(x) \label{eq:mgf-expectation-over-S-poisson}\\
& \approx \sum_{x=0}^{N-k} \prod_{i =1}^{N-k} \left(1-\frac{\theta}{\lambda \cdot E_{\mathcal{P}}[D(i|x)]}\right)^{-1} \cdot P_{sdn}(x) \label{eq:mgf-conditional-expectation-product-average-D-poisson}
\end{align}
where
\begin{itemize}
\item After expressing the MGF in \eq{eq:mgf-expectation-of-product-poisson}, we apply the conditional expectation property to write \eq{eq:mgf-conditional-expectation-poisson}, where $x$ is the step that the SDN cluster received the BGP update, $S$ is the set of nodes that have the BGP update, and with $P\{x,S\}$ we denote the respective joint probability.
\item Since we assume the SDN cluster to be formed independently of the topology, it holds (for any topology) that the variables $x$ and $S$ are independent. Hence, $P\{x,S\} = P\{x\}\cdot P\{S\}$, where $P\{x\}\equiv P_{sdn}(x)$ and its value is given by Theorem~\ref{thm:P-sdn}. Also, we can reorder the summations over $x$ and $S$, which gives \eq{eq:mgf-conditional-expectation-independent-Psdn-poisson}.
\item \eq{eq:mgf-conditional-expectation-product-poisson} follows by making similar arguments as in the proof of Lemma~\ref{thm:MGF-Tc}, and can be written as \eq{eq:mgf-expectation-over-S-poisson}, where the expectation is taken over the set $S\in\mathcal{P}$. 
\item Since the expectation in \eq{eq:mgf-expectation-over-S-poisson} is difficult to compute (see above discussion), we approximate it with the \textit{Delta method}~\cite{Oehlert1992}. In the Delta method the expectation of a function (i.e., the product in \eq{eq:mgf-expectation-over-S-poisson}) of a random variable (i.e., $D(i|x,S)$) is approximated by the function of the expectation of the random variable (i.e., $E_{\mathcal{P}}[D(i|x)]$).
\end{itemize}

From \eq{eq:mgf-conditional-expectation-product-average-D-poisson}, it can be seen that the approximation of $M_{T_{c}}(\theta)$ is given by an expression as in Lemma~\ref{thm:MGF-Tc}, where $D(i|x)$ is replaced by $E_{\mathcal{P}}[D(i|x)]$. Moreover, it is easy to see that all the consequent results for the full-mesh network can be similarly modified for the Poisson graph case.

Now, we need only to calculate the expected bgp-degree $E_{\mathcal{P}}[D(i|x)]$: Let assume that we are at step $i$, and $n(i)$ nodes (see \eq{eq:n(i)}) have received the BGP updates; we denote the set of these nodes as $S_{i}$. A node $m\notin S_{i}$ is connected with a node $j\in S_{i}$ with probability $P(m,j)=p$ (by the definition of a Poisson graph). Hence, the probability that $m$ is a bgp-eligible node (i.e., is connected with \textit{any} of the nodes $j\in S_{i}$, where $|S_{i}| = n(i)$), is given by
\begin{align}
P(m,S_{i}) = 1- (1-p)^{n(i)}
\end{align}

Finally, we note that there are $N-n(i)$ nodes without the update, with each of them being a bgp-eligible node with any of the nodes $j\in S_{i}$ with (equal) probability $P(m,S_{i})$. As a result, the total number of bgp-eligible nodes (or, as defined in Def.~\ref{def:bgp-degree}, the \textit{bgp-degree} $D(i)$) is a binomially distributed random variable, whose expectation is given by 
\begin{equation}
E[D(i)] = (N-n(i))\cdot (1-(1-p)^{n(i)})
\end{equation}
\end{proof}





%%%%%%%%%%%%%%%%%%%%%%%%%%%%%%%%%%%%%%%%%%%
%%%%%%%%%%%%%%%%%%%%%%%%%%%%%%%%%%%%%%%%%%%
%%%%%%%%%%%%%%%%%%%%%%%%%%%%%%%%%%%%%%%%%%%
\section{Internet Topology and Routing Policies}\label{sec:simulations-internet}
To approximate the routing system of the Internet, we use a methodology similar to many previous works~\cite{Let-the-market-BGP-sigcomm-2011,how-secure-goldberg-ComNet-2014,Jumpstarting-BGP-sigcomm-2016,RPKI-deployment-2016}. We first build the Internet topology graph from a large experimentally collected dataset~\cite{AS-relationships-dataset}, and infer routing policies over existing links based on the Gao-Rexford conditions~\cite{stable-internet-routing-TON-2001}. 


\subsection{Building the Internet Topology}
We build the Internet topology graph from the AS-relationship dataset of CAIDA~\cite{AS-relationships-dataset}, which is collected based on the methodology of~\cite{AS-relationships-IMC-2013} and enriched with many extra peering (p2p) links~\cite{multilateral-peering-conext-2013}. The dataset contains a list of AS pairs with a peering link, which is annotated based on their relationship as \textit{c2p} (\textit{customer to provider}) or \textit{p2p} (\textit{peer to peer}). 


\subsection{Selecting Routing Policies}
When an AS learns a new route for a prefix (or, announces a new prefix), it updates its routing table and, if required, sends BGP updates to its AS neighbors. The update and export processes are defined by its routing policies. Similarly to previous works~\cite{Let-the-market-BGP-sigcomm-2011,how-secure-goldberg-ComNet-2014,Jumpstarting-BGP-sigcomm-2016,RPKI-deployment-2016}, we select the routing policies based on the Gao-Rexford conditions that guarantee BGP convergence and stability~\cite{stable-internet-routing-TON-2001}:
\begin{description}
\item[C.1] Paths learned from customers are preferred to paths learned from peers or providers. Paths learned from peers are preferred to paths learned from providers.
\item[C.2] Between paths that are equivalent with respect to \textbf{C.1}, shorter paths (in number of AS-hops) are preferred.
\item[C.3] Between paths that are equivalent with respect to \textbf{C.1} and \textbf{C.2}, the path learned from the AS neighbor with the highest \textit{local preference} is preferred.
\item[C.4] Paths learned from customers, are advertised to all AS neighbors. Paths learned from peers or providers, are advertised only to customers.
\end{description}

In practice, the local preferences (see, \textbf{C.3}) are selected by an AS based on factors related to its intra-domain topology, business agreements, etc. Since it is not possible to know and emulate the real policies for every AS, we assign randomly the local preferences. 






%\end{comment}