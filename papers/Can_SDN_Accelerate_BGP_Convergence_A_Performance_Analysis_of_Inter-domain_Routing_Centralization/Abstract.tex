The Internet is composed of Autonomous Systems (ASes) or domains, i.e., networks belonging to different administrative entities. Routing between domains/ASes is realised in a distributed way, over the Border Gateway Protocol (BGP). Despite its global adoption, BGP has several shortcomings, like slow convergence after routing changes, which can cause packet losses and interrupt communication even for several minutes. To accelerate convergence, inter-domain routing centralization approaches, based on Software Defined Networking (SDN), have been recently proposed. Initial studies show that these approaches can significantly improve performance and routing control over BGP. In this paper, we complement existing system-oriented works, by analytically studying the gains of inter-domain SDN. We propose a probabilistic framework to analyse the effects of centralization on the inter-domain routing performance. We derive bounds for the time needed to establish data plane connectivity between ASes after a routing change, as well as predictions for the control-plane convergence time. Our results provide useful insights (e.g., related to the penetration of SDN in the Internet) that can facilitate future research. We discuss applications of our results, and demonstrate the gains through simulations on the Internet AS-topology.
