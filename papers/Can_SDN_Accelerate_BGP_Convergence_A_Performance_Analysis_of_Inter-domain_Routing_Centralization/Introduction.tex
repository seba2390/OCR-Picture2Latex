The Border Gateway Protocol (BGP) is globally used, since the early days of the Internet, to route  traffic between \textit{Autonomous Systems} (ASes) or \textit{domains}, i.e., networks belonging to different administrative entities. BGP is a distributed, shortest path vector protocol, over which ASes exchange routing information with their neighbors, and establish route paths. 

Although BGP is known to suffer from a number of issues related to security~\cite{Kent-secure-BGP-JSAC-2000,Subramanian-listen-whisper-NSDI-2004}, or slow convergence~\cite{Labozitz-Delayed-convergence-CCR-2000,Kushman-Can-Hear-CCR-2007,Oliveira-Quantifying-Path-Exploration-ToN-2009}, deployment of other protocols or modified versions of BGP is difficult, due to its widespread use, and the entailed political, technical, and economic challenges. Hence, any advances and proposed solutions, should be seamless to BGP.%, in order to have chances to become reality.

Taking this into account, it has been proposed recently that Software Defined Networking (SDN) principles could be applied to improve BGP and inter-domain routing~\cite{Gupta-SDX-CCR-2014,Kotronis-CXP-SOSR-2016,Thai-Decoupling-BGP-Conext-2012,Rothenberg-Revisiting-RCP-HotSDN-2012,Bennesby-Innovating-IDrouting-AINA-2014,Lin-Seamless-Internetworking-Demo-Sigcomm-2013}. The SDN paradigm has been successfully applied in enterprise (i.e., \textit{intra}-AS) networks, like LANs, data centers, or WANs (e.g., Google). However, its application to inter-domain routing (i.e., between different ASes) has to overcome many challenges, like the potential unwillingness of some ASes to participate in the routing centralization. For instance, a small ISP might not have incentives (due to the high investment costs) to change its network configuration. This led previous works on inter-domain SDN to consider (a) partial deployment, only by a fraction of ASes, and (b) interoperability with BGP.

The proposed solutions have demonstrated that bringing SDN to inter-domain routing can indeed improve the convergence performance of BGP~\cite{Kotronis-Routing-Centralization-ComNets-2015}, offer new routing capabilities~\cite{Gupta-SDX-CCR-2014}, or lay the groundwork for new services and markets~\cite{Gibb-Outsourcing-NF-HotSDN-2012,Kotronis-CXP-SOSR-2016}. However, most of previous works are system-oriented: they propose new systems or architectures, and focus on design or implementation aspects. Hence, despite some initial evaluations (e.g., experiments, emulations, simulations) we still lack a clear understanding about the interplay between inter-domain centralization and routing performance.
%Moreover, performance evaluation in these works is usually done through experiments, emulations, or simulations. Although, these methods enable an accurate evaluation, they are time and resource demanding, are not scalable, and are not sufficient to provide generic answers about performance.

To this end, in this paper, we study \textit{in an analytic way} the effects of centralization on the performance of inter-domain routing. We focus on the potential improvements on the (slow) BGP convergence, a long-standing issue that keeps on concerning industry and researchers~\cite{survey-bgp-nanog}. Our goal is to complement previous (system-oriented) works, obtain an analytic understanding, and answer questions such as: \textit{``To what extent can inter-domain centralization accelerate BGP convergence? How many ASes need to cooperate (partial deployment) for a significant performance improvement? Is the participation of certain ASes more crucial? Will all ASes experience equal performance gains?''} Specifically, our contributions are:

%The question is how much? Can it improve always? For whom? Experiments/emulations/simulations give some indications or inittial results, but are not able to give annswers for generic cases. Also, using only sims etc., that are time and resources demaninding, makes a generic evaluation (comprising sensitivity analysis, etc.) a heavy task. There is a scaling problem.



%To this end, we attack the problem in an analytic way, which has not be done before. We build a model and conduct an analysis to study what are the effects of routing centralization on BGP convergence, or the establishment of connectivity after a routing change.   Our goals are to obtain analytical understanding and provide insighits about the effects of the network oarameters, e.g., SDN participation, routing paths, AS topology, on the BGP convergence

%Contributions:
\begin{itemize}
\item We propose a model (Section~\ref{sec:model}) and methodology (Sections~\ref{sec:data-plane} and~\ref{sec:control-plane}) for the performance analysis of inter-domain routing centralization. To our best knowledge, we are the first to employ a probabilistic approach to study the performance of inter-domain SDN. %We deem our approach can be used in the future for analyzing different/further aspects of inter-domain SDN.

\item We analyse the time that the network needs to establish connectivity after a routing change. In particular, we derive upper and lower bounds for the time needed to achieve data-plane connectivity between two ASes (Section~\ref{sec:data-plane}), and exact expressions and approximations for the time till control-plane convergence over the entire network (Section~\ref{sec:control-plane}). Our results are given by closed-form expressions, as a function of network parameters, like network size, path lengths, and number of~SDN~nodes.

\item Based on the theoretical expressions, as well as on extensive simulation results, we provide insights for potential gains of centralization, inter-domain SDN deployment strategies, network economics, etc.


% demonstrate the interplay between network parameteres provide insights on the effects of routing centralization, related to potential performance improvements by SDN, topological characteristics of the SDN ASes, network economics, etc. %We demonstrate the applicability of our predictions to real setting, through an extensive set of simulations.
\end{itemize}

We believe that our study can be useful in a number of directions. Research in inter-domain SDN can be accelerated and facilitated, since a fast performance evaluation with our results can precede and limit the volume of required emulations/simulations. The probabilistic framework we propose can be used as the basis (and be extended and/or modified) to study other problems or aspects relating to inter-domain routing, e.g., BGP prefix hijacking, or anycast. Finally, the provided insights can be taken into account in the design of protocols, systems, architectures, pricing policies, etc.
