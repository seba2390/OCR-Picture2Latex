%%%%%%%%%%%%%%%%%%%%%%%%%%%%%%%%%%%%%%%%%%%%%%%%%%%%%%%%%%%%%%%%%%%%%%%%
%
%
%\documentclass[letterpaper,11pt]{article}
\documentclass{natureprintstyle}
%\documentclass{nature} % preprint

\usepackage{psfig}
%

\usepackage{lscape}
\usepackage{amsmath}
\usepackage{amssymb}
\usepackage{color}
\usepackage{url}
\usepackage{ulem}
\usepackage{multirow}
\usepackage{graphics,graphicx}
\usepackage[affil-it]{authblk}
\usepackage{blindtext}
\usepackage{subfig}
\usepackage{hyperref}

\bibliographystyle{naturemag}
\usepackage{astjnlabbrev-nature} 

\def\etal{{et al.}}
\def\eg{{\em e.g.,~}}
\def\cf{{\em c.f.,~}}
\def\ie{{\em i.e.,~}}
\def\chandra{{\it Chandra}}
\def\nustar{{\it NuSTAR}}
\def\spitzer{{\it Spitzer}}
\def\wise{{\it WISE}}
\def\xmm{{\it XMM-Newton}}
\def\micron{$\mu$m}
\def\degr{$^o$}
\def\arcsec{''}

\def\spose#1{\hbox to 0pt{#1\hss}}
\def\simlt{\mathrel{\spose{\lower 3pt\hbox{$\mathchar"218$}}
     \raise 2.0pt\hbox{$\mathchar"13C$}}}
\def\simgt{\mathrel{\spose{\lower 3pt\hbox{$\mathchar"218$}}
     \raise 2.0pt\hbox{$\mathchar"13E$}}}

\def\plotone#1{\centering \leavevmode
\epsfxsize=0.8\columnwidth \epsfbox{#1}}
\def\plottwo#1#2{\centering \leavevmode
\epsfxsize=.45\columnwidth \epsfbox{#1} \hfil
\epsfxsize=.45\columnwidth \epsfbox{#2}}
\def\plotfiddle#1#2#3#4#5#6#7{\centering \leavevmode
\vbox to#2{\rule{0pt}{#2}}
\special{psfile=#1 voffset=#7 hoffset=#6 vscale=#5 hscale=#4 angle=#3}}





\setlength{\textwidth}{6.5in} 
\setlength{\textheight}{9in}
\setlength{\topmargin}{-0.0625in} 
\setlength{\oddsidemargin}{0in}
\setlength{\evensidemargin}{0in} 
\setlength{\headheight}{0in}
\setlength{\headsep}{0in} 
\setlength{\hoffset}{0in}
\setlength{\voffset}{0in}
%\setlength{\parindent}{0cm}

%\makeatletter
%\renewcommand*{\thanks}[1]{%
% \footnotemark
%  \protected@xdef\@thanks{\@thanks
%    \protect\footnotetext[\arabic{footnote}]{#1}}%
%}
%\makeatother

\makeatletter
\renewcommand{\section}{\@startsection%
{section}{1}{0mm}{-\baselineskip}%
{0.5\baselineskip}{\normalfont\Large\bfseries}}%
\makeatother



\title{Nuclear obscuration in active galactic nuclei}

%% Notice placement of commas and superscripts and use of &
%% in the author list

\author{Cristina Ramos Almeida$^{1,2}$ \& Claudio Ricci$^{3,4,5}$\thanks{*The authors' order is purely alphabetical since they both have contributed equally to the Review. 
e-mail: cra@iac.es and cricci@astro.puc.cl}}

%\smallskip

\begin{document}
\pagestyle{plain}
\pagenumbering{arabic}
%\noindent

\twocolumn[
  \begin{@twocolumnfalse}

\let\newpage\relax\maketitle


\begin{affiliations}
 \item Instituto de Astrof\' isica de Canarias, Calle V\' ia L\'actea, s/n, E-38205, La Laguna, Tenerife, Spain.
 \item Departamento de Astrof\' isica, Universidad de La Laguna, E-38206, Tenerife, Spain.
 \item Instituto de Astrof\' isica, Facultad de F\' isica, Pontificia Universidad Cat\'olica de Chile, Casilla 306, Santiago 22, Chile.
 \item Kavli Institute for Astronomy and Astrophysics, Peking University, Beijing 100871, China.
 \item Chinese Academy of Sciences South America Center for Astronomy and China-Chile Joint Center for Astronomy, Camino El Observatorio 1515, Las Condes, Santiago, Chile.
\end{affiliations}

%   \begin{center} 
%\bfseries\uppercase{%
%%
%\huge Nuclear obscuration in active galactic nuclei
%%
%The Nuclear environment of accreting supermassive black holes
%%
%}
%\end{center}
%     

\bigskip




    \begin{abstract}
        The material surrounding accreting supermassive black holes connects the active galactic nucleus (AGN) with its host galaxy and, besides being responsible for feeding the black hole, 
       provides important information on the feedback that nuclear activity produces on the galaxy. In this Review we summarize our current understanding of the close environment of accreting 
       supermassive black holes obtained from studies of local AGN carried out in the infrared and X-ray band. The structure of this circumnuclear material is complex, clumpy and dynamical, 
       and its covering factor depends on the accretion properties of the AGN. From the infrared point of view, this obscuring material is a transition zone between the broad- and 
       narrow-line region, and at least in some galaxies, it consists of two structures: an equatorial disk/torus and a polar component. In the X-ray regime, the obscuration is produced 
       by multiple absorbers on various spatial scales, mostly associated with the torus and the broad-line region. In the next decade the new generation of infrared and X-ray facilities will 
       greatly contribute to our understanding of the structure and physical properties of nuclear obscuration in AGN.
       %{\color {blue} See if you like it!} I do!
      \bigskip
    \end{abstract}
  \end{@twocolumnfalse}
  ]
{
  \renewcommand{\thefootnote}%
    {\fnsymbol{footnote}}
  \footnotetext[1]{\small The authors' order is purely alphabetical since they both have contributed equally to this Review. 
  
  e-mail: cra@iac.es and cricci@astro.puc.cl} 
}

Over the past decades several pieces of observational evidence have shown that supermassive black holes (SMBHs; $M_{\rm\,BH}\sim 10^{6-9.5}M_{\odot}$) 
are found at the center of almost all massive galaxies, and that those SMBHs play an important role in the evolution of their host galaxies\cite{Kormendy:2013uf} 
during a phase in which they are accreting material and are observed as active galactic nuclei (AGN). Indeed, different modes of AGN feedback are expected to be 
key processes shaping the environment of SMBHs. In particular, quasar-induced outflows might be capable of regulating black hole and galaxy growth\cite{DiMatteo05}. 
For instance, they are required by semi-analytical models of galaxy formation for quenching star formation in massive galaxies\cite{Croton06}. However, directly 
studying the influence of nuclear activity on galaxy evolution is difficult because of the completely different timescales involved\cite{Hickox14,Schawinski:2015cs}. 
{Therefore, to directly probe the AGN--host galaxy connection we need to look at the structure and kinematics of the parsec-scale dust and gas surrounding the 
accreting SMBHs.}


%Accretion onto SMBHs can be very efficient, as it is the case for Seyfert galaxies and quasars, %or radiatively inefficient, which seems to happen in a significant number of low-ionization %nuclear emission-line regions and radio galaxies\cite{Ho08,Netzer15}. 
%Studies carried out in the mid-IR (MIR) and X-ray regime are a p

AGN radiate across the entire electromagnetic spectrum, from the radio {and up to gamma-rays}. A large fraction of their emission is produced in the accretion disk and emitted in the optical and ultraviolet (UV) bands.
%This primary emission can be characterized in AGN with low extinction, but  However, characterizing this emission in AGN is extremely difficult since those wavelengths are severely affected by the extinction produced by dust in their host galaxies. Fortunately, 
 {A significant proportion of these optical/UV photons are reprocessed i) by dust located beyond the sublimation radius and re-emitted in the infrared (IR), and ii) by a corona of hot electrons close to the accretion disk that up-scatters them in the X-ray band\cite{Haardt:1994bq} and illuminates the surrounding material. Thus, studying the IR and X-ray emission and absorption of AGN is key to characterize nuclear regions of AGN.} 
 %and at the same time avoid the severe extinction that affects optical and UV wavelengths.
%The X-ray radiation is then reprocessed by the surrounding material.

%A common origin for the IR and X-ray emission explains the strong 1:1 correlation between the nuclear mid-IR %(MIR) and hard X-ray AGN luminosities\cite{Krabbe01,Lutz:2004gf,Asmus15}, and also the MIR and X-ray bumps %generally observed in the nuclear spectral energy distributions (SEDs) of AGN\cite{Prieto10}. 
%\section*{The X-ray emission of AGN}

 


\section*{AGN structure}

%Accreting SMBHs radiate across the entire electromagnetic spectrum, from the radio to the X-rays. A strong portion of their emission is produced in the accretion disk and is emitted in the optical and UV bands. This optical and UV radiation is then reprocessed by dust and it can be observed in the infrared (IR), and by a corona of hot electrons and up-scattered in the X-ray band\cite{Haardt:1994bq}. 

 %Observational evidence from the X-rays and IR indicates that the strong AGN continuum source must be absorbed %by small-scale obscuring material over a wide solid angle\cite{Rowan77,Lawrence82,Maiolino98}. 
%Indeed, a strong 1:1 correlation between the mid-IR (MIR) and X-ray AGN luminosity has been found by a large number of studies carried out in the past decade\cite{Krabbe01,Lutz:2004gf,Asmus15}. 
%X-rays are a reasonable isotropic measure of the AGN bolometric luminosity, and the MIR emission is mostly due to dust reprocessing of the optical/UV photons produced by the accretion of material onto the SMBH.  


AGN are classified as type-1 and type-2 depending on the presence or not of broad components (full width at half maximum; FWHM$\geq$2000 km~s$^{-1}$) in the permitted lines of their optical spectra. Those broad lines are produced in a sub-pc scale dust-free region known as the broad-line region (BLR). On the other hand the narrow lines (FWHM$<$1000 km~s$^{-1}$) that are ubiquitous in the spectra of AGN --excluding beamed AGN-- are produced in the narrow-line region (NLR). {In the case of moderately luminous AGN, such as Seyfert galaxies, the NLR extends from $\sim$10 pc to $\sim$1 kpc\cite{Capetti96}. }

%It should be remarked that some AGN that are not recognized as such in the optical have been discovered in the past decade\cite{Cocchia:2007kq}.} {\color{blue} This sentence requires a more detailed explanation if we want to add it. Considering that this is a particular type of AGN, I would avoid to mention anything about XBONGs, what do you say?}
%The discovery of a highly polarized H$\alpha$ broad component in the radio galaxy 3C\,234\cite{Antonucci84}, with position angle perpendicular to the radio axis\cite{Antonucci84},led to the development of the AGN unified model\cite{Antonucci93,Urry95}.
% The polarized flux spectra of type-2 AGN reveal highly polarized broad lines

{The discovery of a highly polarized H$\alpha$ broad component in the radio galaxy 3C\,234 with position angle perpendicular to the radio axis\cite{Antonucci84} led to the development of the AGN unified model\cite{Antonucci93,Urry95}. %The polarized flux spectra of type-2 AGN reveal highly polarized broad lines with position angles perpendicular to the radio axis\cite{Antonucci84}. 
These observations can be explained if the central engine is surrounded by a dusty toroidal structure, dubbed the torus, which blocks the direct emission from the BLR in type-2 AGN, and scatters the photons producing the observed polarized spectra.} This toroidal structure, of 0.1--10 pc in size {as constrained from mid-IR (MIR) imaging\cite{Packham05,Radomski08} and interferometry\cite{Burtscher13}, and more recently from sub-millimiter observations\cite{Imanishi16,Garcia16,Gallimore16}}, also collimates the AGN radiation, hence producing the bi-conical shapes of their NLRs known as ionization cones\cite{Malkan98}. 
In summary, from the very center to host-galaxy scales the main AGN structures are the accretion disk and the corona, the BLR, the torus and the NLR, as shown in Figure~\ref{fig1}.


\begin{figure*}
\centering
\includegraphics[width=14cm]{f1.eps}
\caption{Sketch of the main AGN structures seen along the equatorial and polar direction. From the center to host-galaxy scales:
SMBH, accretion disk and corona, BLR, torus and NLR. Different colours indicate different compositions or
densities.}
\label{fig1}
\end{figure*}




{Another structure inferred from radio observations is the sub-pc scale maser disk: a compact concentration of clouds orbiting the SMBH and emitting in the 22 GHz maser line\cite{Greenhill96}. The maser disk is generally assumed to be co-spatial with the torus, although it is not clear whether it corresponds to its innermost part or to a geometrically thin disk which inflates in the outer part\cite{Masini16}. %In many cases the maser emission traces a warped edge-on disk and a nuclear outflow that can be collimated by the warp, as it is the case of the Circinus galaxy\cite{Greenhill03}.
%Also in the radio, a high proportion of AGN show jets of highly collimated accelerated particles that emerge in opposite directions. %Although only $\sim$10\% of the AGN population is considered radio-loud or jetted, the majority of radio-quiet active galaxies are not radio-silent and they show small-scale radio jets. %In the following, we will only refer to the radio-quiet AGN population.
}  

%\color{red} Here we should add the cartoon we discussed if we finally decide to include it. \color{black}
%\color{blue} I think it would be great to have it in the review :). Maybe we can see if it is possible to add a physical scale as well?\color{black}

%X-ray emission is ubiquitous in AGN, and is widely believed to be produced by the Comptonization of optical/UV photons from the accretion disk in . 

%The level of absorption in the X-rays is typically parametrized in terms of the line-of-sight column density ($N_{\rm\,H}$).
%The accretion disk and the corona (i.e. the X-ray source) are compact and located within a few gravitational radii\cite{Zoghbi:2012jk} from the SMBH. 


{X-ray emission is ubiquitous in AGN, and is produced in a compact source located within a few gravitational radii\cite{Zoghbi:2012jk} from an accretion disk.} The study of the reprocessed and absorbed X-ray radiation can provide important information on the structure and physical properties of the circumnuclear material. 
%X-ray obscuration is a very common feature in AGN\cite{Risaliti:1999dw}, . 
{The level of low-ionization absorption in the X-rays} is typically parametrized in terms of the line-of-sight column density ($N_{\rm\,H}$), and AGN are considered to be obscured if $N_{\rm\,H}\geq 10^{22}\rm\,cm^{-2}$. While obscuration strongly depletes the X-ray flux at $E<10$\,keV due to the photo-electric effect, emission in the hard X-ray band ($E\gtrsim10\rm\,keV$) is less affected by obscuration. {Therefore, observations carried out using hard X-ray satellites such as {\it NuSTAR}, {\it Swift}/BAT, {\it Suzaku}/PIN, {\it INTEGRAL} IBIS/ISGRI, and {\it BeppoSAX}/PDS can probe even some of the most elusive accretion events.} Recent hard X-ray surveys have {contributed to} significantly improve our understanding of AGN obscuration, {showing that $\sim 70\%$ of all local AGN are obscured\cite{Burlon:2011dk,Ricci:2015tg}}. {While nuclear obscuration is mostly associated with dust within the torus at IR wavelengths, it can also be related to dust-free gas in the case of the X-rays.} Indeed, it is likely that X-ray obscuration is produced by multiple absorbers on various spatial scales. {This might include dust beyond the sublimation radius, and dust-free gas within the BLR and the torus\cite{Risaliti07,Maiolino:2010fu}}. This explains observations showing that, in general, the columns of material implied in the X-ray absorption are found to be comparable to or larger than those inferred from nuclear IR observations\cite{Ramos09,Burtscher16}. 


Early X-ray studies revealed that, while most type-1 AGN are unobscured, type-2 AGN are usually obscured\cite{Awaki:1991rw}, supporting the unification model. {A clear example is NGC\,1068, the archetypal type-2 AGN, which has been shown to be obscured by material optically-thick to photon-electron scattering (Compton-thick or CT, $N_{\rm\,H}\geq 1.5\times 10^{24}\rm\,cm^{-2}$), which depletes most of the X-ray flux\cite{Matt:1997qy,Bauer:2015si}.} {Nevertheless for some objects with no broad optical lines no X-ray obscuration has been found\cite{Panessa:2002if}.} 
%An example is NGC\,3147, for which simultaneous optical and X-ray observations ruled out intrinsic flux variability as the source of discrepancy between the two classifications\cite{Bianchi:2012mi}. 
{Interestingly, many of these objects have low accretion rates, which would be unable to sustain the dynamical obscuring environment (i.e., the BLR and the torus) observed in typical AGN\cite{Nicastro:2000cq,Elitzur:2009hh}, explaining the lack of X-ray obscuration and broad optical lines.} On the other hand, studies of larger samples of objects have reported tantalizing evidence of a significant AGN population showing broad optical lines and column densities $N_{\rm\,H}\geq 10^{21.5}\rm\,cm^{-2}$ in the X-rays\cite{Merloni:2014wq}. {This has been explained considering that some obscuration is related to dust-free gas within the sublimation region associated to the BLR\cite{Davies:2015rw}.}


%values of the Eddington ratio. This would be in agreement with the idea that the BLR and the torus are related to outflows from the accreting system. Those objects with low 
%These objects seem to be more common at high luminosities\cite{Davies:2015rw}, 
%, and is observed in the X-rays only through its reflected emission


%Since the dust-reprocessed radiation is re-emitted in the IR, this range is key to characterize nuclear obscuration in AGN, and in particular, the torus. 
%However, due to its small size [0.1--10 pc according to constraints based on mid-infrared (MIR) imaging\cite{Packham05,Radomski08} and interferometry\cite{Burtscher13}], until the arrival of the Atacama Large Millimeter/submillimeter Array (ALMA) it was not possible to directly image the torus. 

{The boundary between the BLR and the torus} is set by the dust sublimation temperature. The sublimation region has been resolved\cite{Kishimoto09,Weigelt12} in the near-IR (NIR) using the Very Large Telescope Interferometer (VLTI) 
%using the Astronomical Multi-BEam combineR (AMBER) 
and the Keck Interferometer. {From these interferometric observations it has been found that the inner torus radius scales with the AGN luminosity\cite{Kishimoto11} as $r\propto L^{1/2}$, as previously inferred from optical-to-IR time lag observations\cite{Suganuma06} and also from theoretical considerations\cite{Barvainis87}.} 
%and it provides a promising standard ruler for cosmology based on atomic physics only\cite{Honig17}.}

{The torus radiates the bulk of its energy at MIR wavelengths, although recent interferometry results might complicate this scenario\cite{Honig12,Honig13,Lopez16}. 
From both IR and X-ray observations it has been shown that the nuclear dust is distributed in clumps\cite{Ramos09,Markowitz:2014oq}, and further constraints on the torus 
size and geometry have been provided by MIR interferometry\cite{Burtscher13,Lopez16}. The MIR-emitting dust is compact and sometimes it appears not 
as a single component but as two or three\cite{Tristram14}.} 
%{\color{red} I have the feeling that this new paragraph (except the first sentence) might be a bit of a repetition of what we say later, no?} {\color{blue} It is indeed, but R2 wanted to see a summary of the state-of-the-art...I have summarized it a bit.}

Thanks to the unprecedented angular resolution afforded by the Atacama Large Millimeter/submillimeter Array (ALMA), {recent observations have, for the first time, imaged the dust emission, the molecular gas distribution, and the kinematics from a 7--10 pc diameter disk that represents the sub-mm counterpart of the putative torus of NGC\,1068\cite{Imanishi16,Garcia16,Gallimore16} (see Figure \ref{alma}). As the sub-mm range probes the coolest dust within the torus, this molecular/dusty disk extends is twice larger than the warmer compact MIR sources detected by the VLTI in the nucleus of NGC\,1068\cite{Lopez14} and the pc-scale ionized gas and maser disks imaged in the mm regime\cite{Gallimore96,Gallimore97} which correspond to the torus innermost part. The highest angular resolution ALMA images available to date (0.07\arcsec$\times$0.05\arcsec) reveal a compact molecular gas distribution showing non-circular motions and enhanced turbulence superposed to the slow rotation pattern of the disk\cite{Garcia16}. This is confirmed by deeper ALMA observations at the same frequency\cite{Gallimore16} which permit to disentangle the low-velocity compact CO emission ($\pm$70 km~s$^{-1}$ relative to the systemic velocity) from the higher-velocity CO emission ($\pm$400 km~s$^{-1}$), which the authors interpreted as a bipolar outflow almost perpendicular to the disk.} 

Furthermore, from the left panel of Figure \ref{alma} it is clear that the torus is not an isolated structure. Instead, it is connected physically and dynamically with the circumnuclear disk (CND) of the galaxy\cite{Garcia16} ($\sim$300 pc$\times$200 pc). Indeed, previous NIR integral field spectroscopy data of NGC\,1068 revealed molecular gas streams from the CND into the nucleus\cite{Muller09}. CNDs appear to be ubiquitous in nearby AGN and they constitute the molecular gas reservoirs of accreting SMBHs\cite{Hicks13}. 


\begin{figure*}
\centering
\includegraphics[width=16cm]{figure-ngc1068-review-cris.eps}
\caption{{ALMA maps of the dust continuum and molecular gas in the nucleus of NGC\,1068\cite{Garcia16}.} (a) ALMA natural (NA)-weighted map of the dust continuum emission at 
432 $\mu$m in the circumnuclear disk of NGC\,1068. (b) Close-up of the dust continuum emission shown in the left panel. (c) Overlay of the continuum emission contours shown 
in panel (b) on the CO(6--5) emission from the torus. The red-filled ellipses at the bottom left corner of panels (b) and (c) represent the ALMA beam size at 694 GHz. The dashed
lines highlight the AGN location.}
\label{alma}
\end{figure*}

\section*{Dust and gas spectral properties}

%In this section we describe how X-ray and IR spectroscopy can help us to shed light on the characteristics of dust and gas.
%{\color{blue} I think the title it's kind of self-explanatory, don't you think?}

\subsection*{X-ray tracers of circunmnuclear material.}

The X-ray emitting plasma irradiates the surrounding material giving rise to several {\it reflection} features, the most important of which are the Fe\,K$\alpha$ line at 6.4\,keV and a 
Compton {\it hump} that peaks at $\sim 30\rm\,keV$\cite{Matt:1991ly}. While the Fe\,K$\alpha$ line can be produced by material with column densities as low as 
$N_{\rm\,H}\simeq 10^{21-23}\rm\,cm^{-2}$, the Compton hump can only be created by the reprocessing of X-ray photons in CT material. The Compton hump is a common feature in the X-ray spectra of AGN, showing that CT material is almost omnipresent in AGN. It is however still unclear what fraction of the Compton hump arises {from the accretion disk and what from material associated to the BLR or the torus.}

The narrow Fe\,K$\alpha$ line (FWHM$\simeq 2000\rm\,km\,s^{-1}$)\cite{Shu:2010tg} is an almost ubiquitous feature in the X-ray spectra of AGN\cite{Nandra:1994ly}, and its energy is consistent with this feature originating in lowly-ionised material\cite{Shu:2010tg}. Its origin is still under discussion, and it could be related to the torus\cite{Shu:2010tg}, to the BLR\cite{Bianchi:2008sf}, or to an intermediate region between the two\cite{Gandhi:2015zp}. {The flux of the narrow Fe\,K$\alpha$ line (compared to the intrinsic X-ray flux) is generally weaker in type-2 AGN than in type-1, and it is depleted in CT AGN with respect to less obscured AGN\cite{Ricci:2014ek}.} This would be in agreement with the idea that the circumnuclear material is axisymmetric, as predicted by the unified model, and pointing to the torus or its immediate surroundings as the region where the bulk of this line is produced. In CT material some of the Fe\,K$\alpha$ photons are bound to be down scattered, giving rise to the Compton shoulder. While the shape of this feature carries important information on the geometry and physical characteristics of the material surrounding the SMBH\cite{Matt:2002eu}, the spectral resolution of current facilities has not permitted to study it in detail.

%{I changed this, but what do you mean when you say that the material cannot be anisotropic?} \color{blue} Well, this is what I told you about the obscuring material being more isotropic than originally thought. In the context of the clumpy torus and/or polar dust the anisotropy is really weak. \color{black}
% {OK, but that would still be anisotropic, since the optical depth of the material is different in different directions. In any case it should be fine like this :)}


%\color{red} (I changed anisotropic by axisymmetric because emission is isotropic/anisotropic, not the material. Also, this way we do not contradict the rest of the text, as we mention several times the weak anisotropy of the MIR and X-ray emission)\color{black}


\subsection*{Infrared tracers of circunmnuclear material.}

High angular resolution observations obtained with ground-based 8--10 m--class telescopes and with the  {\it Hubble Space Telescope} have been fundamental to characterize 
the nuclear IR spectral energy distributions (SEDs) of AGN\cite{Alonso03,Ramos09,Prieto10,Asmus14}. In general, 
while the subarcsecond resolution NIR SEDs
(1--8 \micron) of nearby type-1 AGN are {bluer} than those of type-2s, the MIR slope (8--20 \micron) is practically identical for the two types\cite{Levenson09,Prieto10,Ramos11,Asmus14}, 
indicating that the MIR emission is more isotropic than expected from a smooth torus\cite{Pier92,Pier93}. {The wavelength dependency of the IR anisotropy has been also studied 
at higher redshift using an isotropically selected sample of quasars and radio galaxies\cite{Honig11}. Longward of 12 \micron~the anisotropy is very weak, 
and the emission becomes practically isotropic at 15 \micron.} This weak MIR anisotropy explains the strong 1:1 correlation between the MIR and hard X-ray luminosities 
found for both type-1 and type-2 AGN\cite{Krabbe01,Lutz:2004gf,Asmus15}. 

Another MIR spectral characteristic used to study nuclear obscuration and commonly associated with the torus is the 9.7\,$\mu$m silicate feature. 
It generally appears in emission in type-1 AGN and in absorption in type-2 AGN, although there are exceptions\cite{Roche91,Mason09}. The amount of extinction that can be inferred from the silicate feature strength shows a correlation, although with large scatter, with the column densities derived from the X-rays\cite{Shi06}. {In general,} large columns correspond to silicate absorption and small columns to silicate emission. High angular resolution MIR spectroscopy of face-on and isolated AGN {revealed shallow silicate features in type-2 AGN\cite{Roche06,Alonso16}. A clumpy distribution of dust naturally produces these shallow absorption features, but another interpretation to explain this and other ``anomalous'' dust properties in AGN such as the reduced E$_{B-V}$/N$_H$ and A$_V$/N$_H$ ratios is a dust distribution dominated by large grains\cite{Maiolino01}.} 

%{\it On the other hand, deep silicate features are often observed in the nuclear MIR spectra of interacting or edge-on galaxies, but those are produced by dust located on galaxy scales\cite{Levenson07,Alonso11,Gonzalez13}.}  {I think we could remove this part, since is focussed on the galactic absorption. It might also help us to get rid of a few references (jeez I wish we had 30 pages and 2000 references available :D)}

%Indeed, in some nearby AGN kpc-scale dust lanes appear connected to pc-scale dust that contributes to some degree to the nuclear obscuration and could have an influence on the optical classification\cite{Prieto14}.

%\color{blue} In the near-IR (NIR), integral field spectroscopy (IFS) studies of nearby AGN have revealed gas inflows and outflows on nuclear scales\cite{Davies14,Storchi14} (inner $\sim$100 pc), indicating that the torus is a dynamical structure. This is also supported by the recent ALMA observations of NGC\,1068, which show complex kinematics of the molecular gas in the torus\cite{Garcia16,Gallimore16}. On the other hand, nuclear inflows appear associated with the presence of thick molecular gas disks on scales of 50--100 pc, according to sub-mm interferometry and NIR IFS observations of Seyfert galaxies\cite{Sani12,Hicks13,Davies14}.  \color{black}


\section*{Torus models}

%I added X-ray to the first paragraph to connect a bit more 
%{\color{blue} Good idea!}
{As a result of the small size of the torus, neither ground-based single-dish telescopes nor X-ray satellites are able to resolve it even in the most nearby AGN.}
As a consequence, different sets of IR and X-ray torus models were developed aiming to reproduce the observed SEDs and to put indirect constraints on the torus properties. Pioneering work in modelling the dusty torus\cite{Pier92,Pier93} in the IR assumed a uniform dust density distribution for the sake of simplicity. However, it was known from the very beginning that a smooth dust distribution cannot survive in the hostile AGN vicinity\cite{Krolik88}. Instead, the dust has to be arranged in dense and compact clumps. Observationally, X-ray variability studies have provided further support to a clumpy distribution of the obscuring material\cite{Markowitz:2014oq,Marinucci:2016eu}.

%\cite{Pier93} and \cite{Granato94} reproduced
%IR observations of nearby Seyfert galaxies with $\sim$100 pc scale tori. However, hard X-ray observations showed that about
%half of nearby type-2 Seyferts are Compton-thick (i.e., obscured by a column density higher than 1.5$\times10^{24}~cm^{-2}$; 
%\cite{Risaliti99}). For these highly obscured sources the torus dimensions are expected to be of a few parsecs only, because otherwise 
%the dynamical mass of the obscuring material would be too large to be realistic \cite{Risaliti99}.


In order to solve the discrepancies between IR observations and the first smooth torus models (e.g. shallow silicate features, relatively {blue} IR SEDs in type-2 AGN, and small torus sizes), 
more sophisticated models were developed in the last decade. Roughly, two different sets of models can be distinguished. On the one hand, {\it physical models} aim to consider processes such as AGN and supernovae feedback, inflowing material, and disk maintenance\cite{Schartmann08,Wada02,Wada12}. On the other hand, {\it Geometrical/ad-hoc models} attempt to reproduce the IR SED by assuming a certain geometry and dust composition\cite{Nenkova08a,Nenkova08b,Honig10,Stalevski12,Siebenmorgen15}. The two types of models have advantages and disadvantages. Physical models are {potentially more realistic but it is more difficult to compare them with observations, and they generally have to assume extreme conditions to work, such as very massive star clusters or disks, or combine multiple effects like star formation, feedback and radiation with high Eddington ratios. However, much progress has been made since the first physical torus models were developed, and many of these problems are currently being solved.} On the other hand, geometrical models can be easily compared with observations but they face the problem of large degeneracies and dynamical instability. Nevertheless, much has been learned in the last years from comparing models and observations, and what is important is to be aware of the model limitations when interpreting the results\cite{Feltre12}. 


Geometrical torus models are particularly useful for performing statistical analysis {using galaxy samples and, for example, deriving trends in the torus parameters between 
type-1 and 2 AGN. This can be done by evaluating the joint posterior distributions using Bayesian analysis\cite{Ramos11}, or a hierarchical Bayesian approach 
to derive information about the global distribution of the torus parameters for a given subgroup\cite{Ichikawa15}. Individual source fitting with geometrical 
torus models should be used when additional constraints from observables such as the ionization cones opening angle and/or orientation are considered as a priori information in the fit.} In particular, clumpy torus models have made significant progress in accounting for the IR emission of different AGN samples\cite{Mor09,Ramos09,Honig10,Alonso11,Lira13}. Examples of torus parameters that can be derived from SED modeling and compared with independent observations are e.g. the torus width ($\sigma$), which is the complement of the ionization cone half-opening angle angle; the torus outer radius (R$_o$), which can be compared with interferometry constraints; and the covering factor, which depends on the number of clumps (N$_0$) and $\sigma$ (see Figure \ref{torus}). These IR covering factors can be compared with those derived from the modeling of X-ray spectra (see next section for further details). 

%are the number of clumps and their optical depth, which can be constrained with high angular resolution 8--13 \micron~spectroscopy\cite{Alonso11,Ramos14,Ichikawa15}. The torus width, inclination, and the radial distribution of the clouds are sensitive to the NIR-to-MIR slope\cite{Ramos14}, and the torus size can only be constrained if fluxes at $\lambda>$20 \micron~are considered\cite{Fuller16}.



\begin{figure*}
\centering
\includegraphics[width=16cm]{torus_cf.eps}
\caption{{Sketches of two clumpy tori with different covering factors.} Smaller covering factor tori have larger photon escape probabilities associated, while larger covering factor tori are more likely to result in a type-2 AGN classification from our line of sight. Torus parameters such as the number of clumps (N$_0$), angular width ($\sigma$), optical depth of the clumps ($\tau_V$) and inclination ($i$) are labelled. R$_d$ is the dust sublimation radius and R$_o$ the outer radius of the torus.
Figure adapted from \cite{Ramos14} and based on the clumpy torus scheme\cite{Nenkova08a}.}
\label{torus}
\end{figure*}


It is worth noting that we do not only learn from what the models can fit, but also from what they cannot fit. For example, the NIR SEDs of some type-1 Seyferts and 
Palomar-Green quasars (PG quasars) show a $\sim$3 $\mu$m bump that cannot be reproduced with clumpy torus models only, revealing either the presence of nuclear hot dust that is not accounted for in the models or NLR flux contaminating the nuclear measurements\cite{Mor09,Alonso11}. Note, however, that in the case of the more sophisticated two-phase torus models\cite{Stalevski12}, the low-density diffuse interclump dust accounts for the NIR excess in some cases\cite{Lira13,Roseboom13}. {The NIR bump is also reproduced by recently available radiative transfer models\cite{Honig17} that assume an inflowing disk which is responsible for the NIR peak and an outflowing wind that produces the bulk of the MIR emission. Although successful in reproducing recent MIDI interferometric observations of nearby AGN\cite{Lopez16}, the number of free parameters is even larger than in clumpy torus models.}

Another example of IR SEDs that clumpy models cannot reproduce are those of {low-luminosity AGN (LLAGN) with L$_{\rm\,bol} < 10^{42}$ erg~s$^{-1}$, which show a {bluer MIR spectrum (5-35 \micron)} than those of LLAGN with L$_{\rm\,bol} \ge 10^{42}$ erg~s$^{-1}$} \cite{Gonzalez15}. This could be indicating that the torus disappears at low bolometric luminosities\cite{Elitzur:2009hh}.

In the X-rays, torus spectral models are calculated from Monte Carlo simulations of reprocessed and absorbed X-ray radiation\cite{Ikeda:2009hb,Murphy:2009hb,Brightman:2011fe,Liu:2014ff,Furui:2016qf}, and currently consider geometries more simplified than the IR models. Typical parameters obtained from these models are the column density of the torus, its covering factor and the inclination angle with respect to the system. {The two most commonly used models adopt homogeneous toroidal\cite{Murphy:2009hb,Brightman:2011fe} geometries,} and are used to study the most heavily obscured sources, for which the obscuring material acts as a sort of coronagraph, permitting to clearly observe the reprocessed X-ray radiation. These models have allowed to significantly improve the constraints on the properties of the most obscured AGN\cite{Balokovic:2014dq,Annuar:2015wd,Koss:2016fv}, and in some cases to separate the characteristics of the material responsible for the reprocessed emission and those of the obscurer\cite{Yaqoob:2012wu,Bauer:2015si}. 


\section*{Covering factor of the obscuring material}
\label{covering}
%The typical covering factor and column density of the nuclear obscuring material are still widely discussed, %as well as the interplay between these quantities and the physical parameters of the accreting system. 

The covering factor is the fraction of sky covered by the obscuring material, as seen from the accreting SMBH, and it is one of the main elements regulating the intensity of the reprocessed X-ray and IR radiation. In the last decade different trends with luminosity and redshift have been found by studying different AGN samples at different wavelengths. 

Both in the IR and X-rays the covering factor can be inferred from spectral modelling, as outlined in the previous section. 
Two additional methods are often used: i) in the IR the ratio between the MIR and the AGN bolometric luminosity is used as a proxy of the {torus reprocessing efficiency. The fraction of the optical/UV and X-ray radiation reprocessed by the torus and observed 
in the MIR is proportional to its covering factor.} ii) In the X-rays the covering factor of the gas and dust surrounding the SMBH can be estimated using a statistical argument and studying the absorption properties of large samples of AGN. The compactness of the X-ray corona implies that the value of the column density obtained from X-ray spectroscopy of single objects provides information only along an individual line-of-sight. Studying large samples of objects {allows us to probe} random inclination angles, therefore providing a better understanding of the average characteristics of the obscuring material. In fact the probability of seeing an AGN as obscured is proportional to the covering factor of the gas and dust. Therefore the fraction of sources with column density within a certain range can be used as a proxy of the mean covering factor within that $N_{\rm\,H}$ interval. 


%{\color{red} I like the new start of this section! What do you think about moving these two sentences to the 2nd paragraph of the AGN structure section? I think it also fits there and by doing that you would be able to connect the previous paragraph with the description of Figure 3.} Awesome idea

The left panel of Figure\,\ref{fig:NHdistribution} shows the intrinsic column density distribution of local AGN selected in the hard X-ray band and corrected for selection effects. Following the statistical argument outlined above, the $N_{\rm\,H}$ distribution provides important insights on the average structure of the gas and dust, and it can be used to infer the covering factors of different layers of obscuring material, assuming that the column density increases for larger inclination angles (see right panel of Figure\,\ref{fig:NHdistribution}). 
%
%{\color{blue} I think we should also mention here the 70\% that you derive for the total covering factor, and the dependence of Nh of the inclination. It sounds very important to me.} {\color{red} I will check this tomorrow, it got a bit late tonight. Btw I did not hear anything back from Mislav, if he does not reply by tomorrow morning I guess we can move on to do our own figure.}
%{\color{blue} Now, this is personal curiosity: did you check the covering factor for type-1 vs type-2?} 
%{\color{red} The thing is that from this approach (or from comparing the luminosity functions of type1 and type2 AGN) you are getting the average covering factor of the obscuring material from the ratio between obscured and unobscured AGN, so you do not know the covering factor for single objects.} {\color{blue} I wasn't talking about individual objects, but about the population of type-1 vs type-2 in your BAT sample, for example. I guess if you only select type-2s you are biased to larger obscurations and viceversa and you cannot do the exercise you do in Figure 3.}
%
The existence of a significant population of AGN observed through CT column densities was suggested by early X-ray observations of nearby AGN\cite{Risaliti:1999dw}, which found that some of the nearest objects (Circinus, NGC\,1068, and NGC\,4945) are obscured by column densities $\geq 10^{24}\rm\,cm^{-2}$. Recent hard X-ray studies have shown that $\sim 20-27\%$ of all AGN in the local Universe are CT \cite{Burlon:2011dk,Ricci:2015tg}, and a similar percentage is found also at higher redshift \cite{Ueda:2014ix,Brightman:2014zp,Buchner:2015ve,Lanzuisi:2015qr}.




\begin{figure*}[t!]
\centering
\begin{minipage}{.49\textwidth}
\centering
\subfloat[]{\includegraphics[width=8.2cm]{NH_one_Theta_normalized.eps}}\end{minipage}
\begin{minipage}{.49\textwidth}
\centering
\subfloat[]{\includegraphics[height=6.5cm]{f4b.eps}}\end{minipage}
%
%
 \begin{minipage}{1\textwidth}
  \caption{{Structure of the obscuring material in local AGN.} (a) Intrinsic column density distribution of local hard X-ray selected AGN\cite{Ricci:2015tg}, 
  showing that the average covering factor of the obscuring material is $70\%$. (b) Schematic representation of the structure of the obscuring material as inferred from 
  the intrinsic $N_{\rm\,H}$ distribution shown in the left panel. $\theta_{\rm\,i}$ is the inclination angle relative to the observer. 
    }\label{fig:NHdistribution}
     \end{minipage}
\end{figure*}

%Studies of the spectral shape of the Cosmic X-ray background (CXB) also showed that a significant fraction of CT AGN are needed to reproduce the peak of the CXB\cite{Ueda:2003nx,Ballantyne:2006kx,Gilli:2007qf,Treister:2009hb,Akylas:2012uq}.
%\color{red}
%*And the same for this one. The first part is already said in the first section, and I don't think it is really necessary to talk about the CXB and high-redshift AGN in a review about nuclear obscuration. This way we could get rid of a lot of references. \color{blue}


%; a possible increase of the fraction of CT sources to $\sim 40\%$ at $z=1-4$ has been found studying AGN in the {\it Chandra} Deep Field South\cite{Brightman:2012dk}. 


%The MIR/X-ray correlation has also been used to select CT AGN candidates, exploiting the fact that in these objects most of the X-ray flux is depleted by the obscuring material.


Studies of X-ray selected samples of AGN have shown that the fraction of obscured Compton-thin ($N_{\rm\,H}=10^{22-24}\rm\,cm^{-2}$) sources decreases with luminosity\cite{La-Franca:2005kl,Akylas:2006gd,Burlon:2011dk,Merloni:2014wq,Ueda:2014ix} (see Figure\,\ref{fig:fobs_L}). A similar behavior has been observed for the Compton-thick material in a large sample of {\it Swift}/BAT AGN\cite{Ricci:2015tg}, as well as from the parameters derived from the broad-band X-ray spectroscopic analysis of a sample of local CT AGN using a physical torus model\cite{Brightman:2015fv}. 
This trend has been interpreted as being due to the decrease of the covering factor of the obscuring material with the luminosity, and it has been also reported by several studies carried out in the IR using the ratio between the MIR and the bolometric luminosity\cite{Maiolino:2007ye,Treister:2008kc,Lusso:2013vf,Stalevski:2016hl}. Furthermore, the fraction of obscured AGN at a given X-ray luminosity also increases with redshift\cite{Ueda:2014ix}, suggesting that the circumnuclear material of AGN might also evolve with Cosmic time.
Nevertheless, recent works carried out in the IR\cite{Stalevski:2016hl} and X-rays\cite{Sazonov:2015ys} have argued that if the anisotropy of the circumnuclear material is properly accounted for, the decrease of the covering factor with luminosity would be significantly weaker.

%} {\color{green} Since this sentence is in between covering factor-luminosity stuff, I suggest to move it to the beginning of this paragraph, to then continue with covering-factor-luminosity.}
% If you don' mind I would prefer leaving it here, since the idea is that you are looking at redshift evolution in luminosity bins, otherwise one would get confused with luminosity dependence and flux limits.

The intensity of the reprocessed X-ray radiation depends on the covering factor of the obscuring material, which would lead to expect a direct connection between reflection features and X-ray luminosity. While the relation between the Compton hump and the luminosity is still unclear, a decrease of the equivalent width of the Fe\,K$\alpha$ line with the luminosity (i.e. the X-ray Baldwin effect\cite{Iwasawa:1993ez}) has been observed in both unobscured\cite{Bianchi:2007os,Shu:2010tg} and obscured\cite{Ricci:2014ek} AGN. Moreover, the slope of the X-ray Baldwin effect can be reproduced by the relation between the covering factor and the luminosity found by X-ray studies\cite{Ricci:2013hi}, also suggesting a relation between the two trends. 


%


%Two classes of AGN have been found not to follow the relation between covering factor and luminosity discussed above: red quasars\cite{Gregg:2002zm} and Hot Dust Obscured Galaxies\cite{Eisenhardt:2012ve}. These objects show very high luminosities  and column densities $>10^{22}\rm\,cm^{-2}$, and are not rare oddities, since it has been shown that about half of the most luminous ($>10^{47}\rm\,erg\,s^{-1}$) AGN are obscured\cite{Assef:2015ly}.






The relation between the obscuring material and the AGN luminosity has been often explained as being a form of {\it feedback}, with the radiation field of the AGN 
cleaning out its circumnuclear environment\cite{Fabian:2006sp}. 
Interestingly, the decrease of the covering factor with luminosity does not extend to the highest bolometric luminosities ($10^{46-48}\rm\,erg\,s^{-1}$), 
where about half of the AGN population seem to be obscured\cite{Assef:2015ly}. We note that these obscured AGN are not necessarily type-2 AGN in the optical range.
In the low-luminosity regime, evidence for a decrease of the fraction of obscured sources for X-ray luminosities $<10^{42}\rm\,erg\,s^{-1}$ has been 
observed\cite{Burlon:2011dk} (Figure\,\ref{fig:fobs_L}). This result is supported by MIR observations of nearby AGN, which claim that the torus disappears 
at luminosities below the before-mentioned limit\cite{Gonzalez15}. As discussed in previous sections, these results can be explained if low-luminosity AGN 
fail to sustain the AGN internal structures\cite{Elitzur:2006ec}. {This idea would also explain the observed decrease of the Fe\,K$\alpha$ intensity with 
respect to the X-ray flux at low-luminosities found using {\it Suzaku}\cite{Kawamuro:2016lq}.} 

\begin{figure}[t!]
%\centering
\includegraphics[width=8.5cm]{f16.eps}
%
%
  \caption{{Evolution of the covering factor of the obscuring material with luminosity.} Relation between the fraction of obscured Compton-thin sources ($10^{22}<N_{\rm\,H}<10^{24}\rm\,cm^{-2}$) and the 14--195\,keV luminosity for AGN detected by {\it Swift}/BAT\cite{Burlon:2011dk}.  
    }
\label{fig:fobs_L}
\end{figure}

The modelling of IR SEDs has also provided important insights on the unification scheme. {Using clumpy torus models, it has been claimed} that the covering factors of type-2 
tori are larger (i.e., more clumps and broader tori) than those of type-1 AGN\cite{Ramos11,Ichikawa15,Alonso11,Mateos16}. %Differences in the covering factor explain, for example, the detection of edge-on ionization cones in several type-1 AGN\cite{Muller11,Fischer13}.
{This would imply,} first, that the observed differences between type-1 and type-2 AGN are not due to orientation effects only, as proposed in the simplest unification model, but also to {the dust covering factor}. 
Second, that the torus is not identical for all AGN of the same luminosity. Therefore, the classification of an AGN as a type-1 or type-2 is probabilistic\cite{Elitzur12} (see Figure \ref{torus}). 
The larger the covering factor, the smaller the escape probability of an AGN-produced photon. Although these results are model-dependent, they reflect the observed differences between the IR 
SEDs of type-1 and type-2 AGN. It is noteworthy that in the radio-loud regime it has been found that the radio core dominance parameter (the ratio of pc-scale jets to radio lobes emission) 
agrees with the optical classification of a subsample of 3CR radio galaxies at z$\geq$1\cite{Marin2016}. This would be indicating a small dispersion of torus opening angle and inclination for 
luminous type-1 and type-2 radio-loud AGN. Unfortunately, this analysis is not possible in radio-quiet AGN.

While models reproducing at the same time both the X-ray and MIR spectral properties of the reprocessed radiation are still missing, the values of the covering factors obtained by using 
MIR\cite{Alonso11,Lira13,Ichikawa15} and X-ray torus models are consistent\cite{Brightman:2015hb}, albeit with large uncertainties and for a handful of AGN only. 

%Besides, in the context of the clumpy models, for every combination of covering factor and inclination there is always a finite probability of having a direct view to the nucleus\cite{Nenkova08b}. 




%Recent results have however shown that the observed
%Using an IR torus model, it has been shown that the fact that both the accretion disk and the torus emit anisotropically should be considered when computing the covering factors using this approach\cite{Stalevski:2016hl}. {Taking into account this anisotropy the dependence of the covering factor of the obscuring material with the luminosity is flattened\cite{Stalevski:2016hl}.}


%Two classes of AGN have been found not to follow the relation between covering factor and luminosity discussed above: red quasars\cite{Gregg:2002zm} and Hot Dust Obscured Galaxies\cite{Eisenhardt:2012ve}. These objects show very high luminosities ($10^{45-48}\rm\,erg\,s^{-1}$) and column densities $>10^{22}\rm\,cm^{-2}$, and are not rare oddities, since it has been shown that about half of the most luminous ($>10^{47}\rm\,erg\,s^{-1}$) AGN are obscured\cite{Assef:2015ly}. Interestingly, both red quasars\cite{Urrutia:2008vn} and Hot Dust Obscured Galaxies\cite{Fan:2016sf} are preferentially found in mergers. The enhanced fraction of obscured sources at high luminosities could be interpreted in the framework of an evolutionary sequence\cite{Sanders:1988kl}, in which gas-rich merger of galaxies trigger star formation and heavily obscured accretion onto the SMBH. Eventually the feedback from the AGN clears out the environment and the source is observed as an unobscured quasar. Recent studies have shown that AGN in mergers tend to be obscured\cite{Satyapal:2014th,Kocevski:2015zr,Koss:2016fv}, suggesting that indeed galaxy interactions can affect the close environment of SMBHs. 

 



%\cite{Stern:2014kx}




%In particular, it has been proposed that the main physical mechanism influencing the covering factor of the obscuring material is radiation pressure. 
%This would imply that the covering factor would mostly depend on accretion efficiency rather than luminosity. 
%Unfortunately the lack of a large sample of AGN with good constraints on the black hole mass for both obscured and unobscured AGN has so far not allowed to confirm this idea. 



%Moreover, recent {\it NuSTAR} observations have confirmed that the low-luminosity AGN in NGC\,7213 lacks a Compton hump, which could be explained by the absence of a molecular torus\cite{Ursini:2015vn}.






%This result is independently supported by X-ray observations. Highly obscured type-2 AGN have a stronger X-ray reflection component than less obscured type-1 and type-2 AGN\cite{Ricci11}. If the reflector is the torus, the more obscured the AGN the larger the covering factor. 




\section*{Variability of the line-of-sight obscuring material}
\label{variability}

\begin{figure*}[t!]
\centering
\subfloat[]{\includegraphics[width=6.9cm]{f_risalitiN1365.eps}}
\subfloat[]{\includegraphics[width=8.6cm]{plot_excess.eps}}
\caption{{Examples of absorption variability in the X-ray spectrum of nearby AGN.} (a) Absorption variability observed in the X-ray spectrum of NGC\,1365\cite{Risaliti07}. The three {\it Chandra} observations were carried out two days from each other. The spectrum of the second observation clearly shows the flat continuum and prominent Fe\,K$\alpha$ line typical of heavily obscured AGN. (b) {\it NuSTAR} observations of NGC\,1068\cite{Marinucci:2016eu}, showing an excess above 20\,keV in August 2014 that can be explained by a cloud moving away from the line-of-sight, allowing for the first time to observe the nuclear X-ray continuum of this source.}
\label{fig:variableNH}
\end{figure*}



Studies carried out in the X-rays have found variations of the column densities of the obscuring material for several dozens AGN, confirming the idea that the obscuring material is clumpy and not homogeneous, and very dynamic. In about ten objects\cite{Risaliti07,Guainazzi:2002mz,Piconcelli:2007bh}, these variations were found to be very extreme, with the line-of-sight obscuration going from CT to Compton-thin (and vice versa) on timescales of hours to weeks (see left panel of Figure\,\ref{fig:variableNH}). This is consistent with the absorber originating in the BLR. Due to the strong changes in their spectral shapes, these objects are called {\it changing-look AGN}. For the archetypal of these objects, NGC\,1365, it has been found that the obscuring clouds have a cometary shape\cite{Maiolino:2010fu}, with a high-density head and an elongated structure with a lower density. Even objects such as Mrk\,766\cite{Risaliti:2011jl}, which are usually unobscured, have been found to show eclipses produced by highly-obscuring material. For this object, highly-ionized blueshifted iron absorption lines (Fe\,XXV and Fe\,XXVI) were also detected, showing that the absorbing medium is outflowing with velocities ranging from 3,000 to 15,000\,$\rm\,km\,s^{-1}$. Interestingly, due to the mass loss from the cometary tail, clouds would be expected to be destroyed within a few months\cite{Maiolino:2010fu}. This suggests that the BLR must be very dynamic, with gas clouds being created and dissipating continuously. The origin of these clouds is still unclear, but it has been suggested that they might be created in the accretion disk\cite{Elitzur:2009hh}.


Evidence for a clumpy obscurer on scales larger than the BLR have also been found. 
%In the past few years several pieces of evidence have confirmed that, as showed by analysis of IR SEDs, also on scales larger than the BLR the material is clumpy. 
A study carried out using the {\it Rossi X-ray Timing Explorer} has recently discovered variation of the absorbing material (with $N_{\rm\,H}\sim 10^{22-23}\rm\,cm^{-2}$) on timescales of months to years  for several objects\cite{Markowitz:2014oq}. For seven AGN the distance of the obscuring clouds locates them between the outer side of the BLR and up to ten times the distance of the BLR, suggesting that they are associated with clumps in the torus. Recent {\it NuSTAR} observations of NGC\,1068\cite{Marinucci:2016eu} found a $\sim$30\% flux excess above 20\,keV in August 2014 with respect to previous observations (right panel of Figure\,\ref{fig:variableNH}). The lack of variability at lower energies permitted to conclude that the transient excess was due to a temporary decrease of the column density, caused by a clump moving out of the line-of-sight, which enabled to observe the primary X-ray emission for the first time. {In the MIR, results from dust reverberation campaigns using data from the {\it Spitzer Space Telescope} are consistent with the presence of clumps located in the inner wall of the torus\cite{Vazquez15}}.



%For Centaurus\,A the clump was found to have a size of  $1.4-2.4 \times 10^{15}$\,cm\cite{Rivers:2011rz}

Therefore, we know from absorption variability studies that both the BLR and the torus are not homogeneous structures, but clumpy and dynamic regions which might be generated as part of an outflowing wind. 

\section*{Gas and dust in the polar region}


In the last decade, MIR interferometry has represented a major step forward in the characterization of nuclear dust in nearby AGN. VLTI/MIDI interferometry of 23 AGN has revealed that a large part of the MIR flux is concentrated on scales between 0.1 and 10 pc\cite{Tristram09,Burtscher13}. Besides, for the majority of the galaxies studied, two model components are needed to explain the observations\cite{Burtscher13}, instead of a single toroidal/disk structure. Moreover, for some of these sources one of these two components appears elongated in the polar direction. Detailed studies of four individual sources performed with MIDI\cite{Honig12,Honig13,Tristram14,Lopez14} have shown further evidence for this nuclear polar component (see Figure \ref{polar}). More recently, a search for polar dust in the MIDI sample of 23 AGN has been carried out\cite{Lopez16}, and this feature has been found in one more galaxy. Thus, up to date, evidence for a diffuse MIR-emitting polar component has been found in five AGN, including both type-1 and type-2 sources (Circinus, NGC\,424, NGC\,1068, NGC\,3783, and NGC\,5506). This polar component appears to be brighter in the MIR than the more compact equatorial structure, and {it has been interpreted as an outflowing dusty wind driven by radiation pressure\cite{Honig12}. Indeed, radiation-driven hydrodynamical models\cite{Wada12,Wada16} taking into account both AGN and supernovae feedback can reproduce geometrically thick pc-scale disks and also polar emission, although these features are rather transient in nature even when averaged over time. %It is also possible that the polar component is just optically-thin dust at the base of the ionization cones, but the MIR spectra of the type-2 galaxies observed with MIDI do not show the silicate feature in emission which is characteristic of optically thin dust. Indeed, it is seen in modest absorption with the exception of the galaxy NGC\,424\cite{Honig12}.
} 

\begin{figure*}
\centering
\subfloat[]{\includegraphics[width=7.7cm]{Tristram2013.eps}}
\subfloat[]{\includegraphics[width=7.9cm]{NGC3783_contribution.eps}}
\caption{{Interferometry results from VLTI/MIDI observations of two nearby AGN.} (a) 3-component model that reproduces the observed MIDI visibilities obtained for the Circinus galaxy\cite{Tristram14}. 
This includes a polar component containing most of the MIR emission, a disk component, and 
an unresolved component. (b) Relative contributions of the disk and polar components derived from the model that better reproduces the MIDI observations of the type-1 AGN NGC\,3783\cite{Honig13}.}
\label{polar}
\end{figure*}

%\begin{figure*}
%\centering
%\includegraphics[width=14cm]{wada.eps}
%\caption{Simulated 12 \micron~images at inclinations of 75\degr~and 90\degr~for the case of the Circinus %galaxy. The cold disk appears as 
%a clumpy dark lane, and the polar component is also inhomogeneous and originates from hot dust in the bipolar %outflow\cite{Wada16}.}
%\label{wada}
%\end{figure*}

 
It is noteworthy that this polar component was first detected in NGC\,1068 using high angular resolution MIR observations ($\le$0.5\arcsec) obtained with {single-dish 4--10 m-class telescopes\cite{Cameron93,Bock00,Mason06}. These observations revealed that
the point source was only responsible for $\sim$30--40\% of the 8--24.5 \micron~emission, and the remaining 60--70\% was emitted by dust in the ionization cones. Thus, the bulk of the nuclear MIR flux comes from polar dust within the central 70 pc of NGC\,1068\cite{Mason06}.} More recently, a similar result has been reported for 18 active galaxies observed with 8 m-class telescopes in the MIR\cite{Asmus16}. The resolved emission is
elongated in the polar direction (i.e. NLR dust), it represents at least 40\% of the MIR flux, and it scales with the [O IV] flux. This is in line with the results from MIR interferometry\cite{Honig12,Honig13}, which indicate that the bulk of the MIR emission comes from a diffuse polar component, while the NIR flux would be dominated by a compact disk (see right panel of Figure \ref{polar}). The difference between single-dish telescopes and MIR interferometry studies is the scale of the IR-emitting regions probed.
%In the case of NGC\,1068 the extent of the polar component is $\sim$14 pc\cite{Lopez14}. %{Considering the results discussed here, do we have to rule out the traditional view of the torus and incorporate new components to the models? More observations are required to test the ubiquity of the polar component.}

{The existence of non-nuclear reflecting material in obscured AGN has been confirmed in the X-ray regime by {\it Chandra} studies of the Fe\,K$\alpha$ line, which showed that part of the emission originates from an extended region.} This has been found for some of the nearest heavily obscured AGN, such as NGC\,1068\cite{Young:2001jw}, for which $\sim$30\% of the Fe\,K$\alpha$ emission has been found to originate in material located at $\gtrsim$140 pc\cite{Bauer:2015si} and it seems to be aligned with the NLR. {Similarly, the 0.3--2\,keV radiation has also been found to coincide with the NLR in obscured AGN\cite{Bianchi:2006kq}.}


If proved to be common features in a significant fraction of AGN, co-existing compact disks/tori and polar dust components {should be incorporated in the models\cite{Honig17}} and could explain the observed NIR and MIR bumps seen in the SEDs of some type-1\cite{Mor09,Alonso11,Ichikawa15} and type-2 AGN\cite{Lira13}. Besides, the polar emission has been proposed as an alternative scenario to explain {the weak MIR anisotropy observed in active galaxies\cite{Honig11},} which is responsible for the strong 1:1 X-ray/MIR correlation slopes found for type-1 and type-2 AGN\cite{Gandhi:2009pd,Ichikawa:2012uo,Asmus15}. However, as explained in previous sections, a toroidal clumpy distribution also explains the weak MIR anisotropy\cite{Levenson09} and more sophisticated clumpy models account for the NIR excess of the nuclear SEDs, {either including a polar component in addition to the torus\cite{Honig17} or not\cite{Stalevski12}.}


%We should keep in mind that MIDI interferometry results are model-dependent, as the number of baselines is very limited and, with the exception of the Circinus galaxy, are subject to uncertainties. The large angular resolution afforded by MIDI in the MIR can be also reached with ALMA in the mm and sub-mm regimes, without suffering from the limited number of baselines. In fact, ALMA has already provided the first images of the sub-mm counterpart of the dusty torus/disk in NGC\,1068. Using band-9 observations ($\sim$430 \micron) with a spatial resolution of $\sim$4 pc, two different studies independently reported sizes of $\sim$7--12 pc for the major axis of the disk\cite{Garcia16,Gallimore16}. There is a small contribution from polar dust to the 430 \micron~emission (see Figure \ref{alma}), but much less important than the disk\cite{Garcia16}. Besides, as it can be seen from Figure \ref{alma}, the polar component is not detected in molecular gas, but only in dust continuum emission. 


%The dynamics of the molecular gas in the torus/disk detected with ALMA show turbulence and strong non-circular motions superposed on a slow
%disk rotation pattern \cite{Garcia16,Gallimore16}. These complex kinematics and the evidence for a no edge-on inclination of the disk can 
%be explained by gas participating in the bipolar outflow roughly perpendicular to the disk \cite{Garcia14,Garcia16,Gallimore16} or by a 
%particular type of disk instability \cite{Garcia16}.



%In Mrk\,3 and NGC\,4945 the Fe\,K$\alpha$ line is extended up to $\sim$300 pc\cite{Guainazzi:2012tg} and $180\rm\,pc \times 90\rm\,pc$\cite{Marinucci:2012zt} respectively, but the orientation of this extended emission is not clear. \color{blue} I changed this last sentence a bit, but check if you are ok with that. If you don't like the last part, I suggest to remove it, because by commenting here on the Mrk3 and NGC 4945 we are implying that there is extended polar emission in the Fe Ka, and I don't think is the case. \color{black}

%\color{blue} But is this extended emission in the polar direction? \color{red} Yes, in both NGC1068 and Mrk3 the FeKalpha is extended in an $\sim$ axisymmetric fashion, which looks similar to the polar region, while for NGC4945 the extended emission is more like a big box.  \color{black} \color{blue} In that case it should be said in the text, because now we are just saying that the line is produced in an extended region. \color{black}


\section*{Current picture and the future of IR and X-ray studies of nuclear obscuration in AGN}



In the past 10--15 years, studies of AGN in the IR and X-rays have provided important information on the characteristics of the nuclear environment of accreting SMBHs, 
showing that its nature is extremely complex and dynamic. {The obscuring structure is is compact, clumpy, and not isolated, but connected with the host galaxy via gas inflow/outflows.} 
From the IR point of view, it is a transition zone between the dust-free BLR clouds and the NLR and, at least in some galaxies, it consists of two structures: an equatorial disk/torus 
and a polar component. This polar component would be part of the outflowing dusty wind predicted by radiation-driven hydrodynamical models. In the case of the X-rays, the obscuration 
is produced by multiple absorbers on various spatial scales, but mostly associated with the torus and the BLR. The covering factor of the obscuring material depends on the luminosity 
of the system and possibly on the redshift, and it is important to take these dependencies into account to explain observations of both high- and low-luminosity AGN. The covering 
factor should also be considered in our current view of AGN unification, as the classification of an AGN as type-1 or type-2 does not depend on orientation only, but also on the 
AGN-produced-photon escape probability. 


In the next decade the new generation of IR and X-ray facilities will {contribute greatly to our} understanding of the structure and physical properties of the nuclear material, and to shed light on relevant open question such as: what is the relationship between the physical parameters of the accreting system and the circumnuclear material? Are the torus and BLR produced by outflows in the accretion disk? Is the polar dust ubiquitous?, and how much does it contribute to the IR emission? 


{In the X-ray regime, {\it NuSTAR}, {\it XMM-Newton}, {\it Chandra}, {\it Swift} and {\it INTEGRAL} will continue carrying out broad-band X-ray observations of AGN, providing tighter constraints on the most obscured accretion events and on the characteristics of the circumnuclear material through studies of the reprocessed X-ray radiation.} The recently launched X-ray satellite {\it ASTROSAT}\cite{Singh:2014pd}, thanks to its large effective area and broad band X-ray coverage, will be ideal to study absorption variability, and will improve our understanding of the properties of the BLR clouds. {\it eROSITA} (\href{http://adsabs.harvard.edu/abs/2012arXiv1209.3114M}{Merloni et al. 2012}), on board the {\it Spectrum-Roentgen-Gamma} satellite, will carry out a deep survey of the entire X-ray sky in the 0.5--10\,keV range, and is expected to detect tens of thousands of obscured AGN. {This will certainly improve} our understanding of the relation between obscuration and the accretion and host galaxy properties.

On longer timescales, {\it Athena} (\href{http://adsabs.harvard.edu/abs/2013arXiv1306.2307N}{Nandra et al. 2013}), and before that the successor of {\it Hitomi}, will enable studies of reflection features in AGN with an exquisite level of detail, exploiting the energy resolution of a few eV of micro-calorimeters. High-resolution spectroscopy studies of AGN {will make possible to disentangle} the different components (arising in the BLR, torus, NLR) of the Fe\,K$\alpha$ line, and to set tighter constraints on the properties of the circumnuclear material using the Compton shoulder\cite{Odaka:2016fv}.
NASA recently selected the {\it Imaging X-ray Polarimetry Explorer}\cite{Weisskopf:2016qd} (IXPE) mission to be launched in the next decade. X-ray polarimetry will open a new window in the study of the close environment of AGN, since the reprocessed X-ray radiation is bound to be polarised. 
%{\color{green} Shall we move this sentence to the end? it looks like a nice wrap-up.} 

To date, IR interferometry has provided constraints on the size and distribution of nuclear dust for about 40 AGN. Now, a second generation of interferometers for the VLTI are coming online. In the NIR, GRAVITY\cite{Eisenhauer11} will be able to observe $\sim$20 nearby AGN with unprecedented sensitivity and high spectral resolution, allowing to estimate reliable SMBH masses and put constraints on the geometry of the BLR. In the MIR, MATISSE\cite{Lopez14b} will combine the beams of up to 4 VLTI telescopes {to produce images that will serve to analyze} the dust emission at 300--1500 K in the central 0.1-–5 pc of the closest AGN. In the NIR and MIR, the {\it James Webb Space Telescope\cite{Gardner2006}} ({\it JWST}) will represent a revolution in terms of sensitivity and wavelength coverage. %The angular resolutions afforded by its instruments will not be larger than those achieved with 8--10 m ground-based telescopes and the Hubble Space Telescope ({\it Hubble}), but 
Faint low-luminosity and high-redshift AGN will be accessible at subarcsecond resolution from 0.6 to 28 \micron~for the first time. 
%In particular, the number of high angular resolution MIR studies of LLAGN is reduced\cite{Mason12,Mason13,Fernandez12,Asmus14}, and for some of them there is debate on whether the nuclear MIR emission is produced by dust or just by non-thermal emission related to the jet. The JWST will be crucial in clarifying this, and also in confirming/discarding the torus disappearance at the lowest bolometric luminosities\cite{Elitzur:2006ec,Gonzalez15}. 
Finally, in the sub-mm regime ALMA will continue providing the first images of the nuclear obscurer in nearby AGN. In Cycle 4 and later, {ALMA will fully resolve} the gas kinematics 
from galaxy scales to the area of influence of the SMBH in nearby AGN. This will serve to characterize the inflowing/outflowing material in the nucleus and its connection with the 
host galaxy, leading to a better understanding of the feeding/feedback mechanisms in AGN.


With the advent of all the facilities described above, in order to fully exploit the wealth of data that will be available in the next decade, it will be necessary for the community 
to develop physical AGN spectral models that could self-consistently reproduce reprocessed X-ray radiation and MIR emission, ideally considering polarization as well. 

\section*{Acknowledgements}

The authors acknowledge Almudena Alonso-Herrero, Poshak Gandhi, Nancy A. Levenson, Marko Stalevski and the referees for useful comments that helped to improve this Review. 
CRA acknowledges the Ram\'on y Cajal Program of the Spanish Ministry of Economy and Competitiveness through project RYC-2014-15779 and the Spanish Plan Nacional de Astronom\' ia y 
Astrof\' isica under grant AYA2016-76682-C3-2-P. CR acknowledges financial support from the China-CONICYT fellowship program, FONDECYT 1141218 and Basal-CATA PFB--06/2007.
This work is sponsored by the Chinese 
Academy of Sciences (CAS), through a grant to the CAS South America Center for Astronomy (CASSACA) in Santiago, Chile. \\

\noindent{Correspondence should be addressed to the two authors.}

\section*{Author contributions}
The two authors contributed equally to this work. They both decided the concept of the Review and provided/adapted the figures that appear on it. 
CR and CRA wrote the X-ray and IR part of the text, respectively, and worked together to put them in common.  

%\bibliography{review}

\begin{thebibliography}{100}
\expandafter\ifx\csname url\endcsname\relax
  \def\url#1{\texttt{#1}}\fi
\expandafter\ifx\csname urlprefix\endcsname\relax\def\urlprefix{URL }\fi
\providecommand{\bibinfo}[2]{#2}
\providecommand{\eprint}[2][]{\url{#2}}


\bibitem{Kormendy:2013uf}
\bibinfo{author}{{Kormendy}, J.} \& \bibinfo{author}{{Ho}, L.~C.}
\newblock {Coevolution (Or Not) of Supermassive Black Holes and Host Galaxies}.
\newblock \textit{\bibinfo{journal}{Ann. Rev. Astron. Astrophys.}} \textbf{\bibinfo{volume}{51}},
  \bibinfo{pages}{511--653} (\bibinfo{year}{2013}).

\bibitem{DiMatteo05}
\bibinfo{author}{{Di Matteo}, T.}, \bibinfo{author}{{Springel}, V.} \&
  \bibinfo{author}{{Hernquist}, L.}
\newblock {Energy input from quasars regulates the growth and activity of black
  holes and their host galaxies}.
\newblock \textit{\bibinfo{journal}{\nat}} \textbf{\bibinfo{volume}{433}},
  \bibinfo{pages}{604--607} (\bibinfo{year}{2005}).

\bibitem{Croton06}
\bibinfo{author}{{Croton}, D.~J.} \textit{et~al.}
\newblock {The many lives of active galactic nuclei: cooling flows, black holes
  and the luminosities and colours of galaxies}.
\newblock \textit{\bibinfo{journal}{\mnras}} \textbf{\bibinfo{volume}{365}},
  \bibinfo{pages}{11--28} (\bibinfo{year}{2006}).

\bibitem{Hickox14}
\bibinfo{author}{{Hickox}, R.~C.} \textit{et~al.}
\newblock {Black Hole Variability and the Star Formation-Active Galactic
  Nucleus Connection: Do All Star-forming Galaxies Host an Active Galactic
  Nucleus?}
\newblock \textit{\bibinfo{journal}{\apj}} \textbf{\bibinfo{volume}{782}},
  \bibinfo{pages}{9} (\bibinfo{year}{2014}).

\bibitem{Schawinski:2015cs}
\bibinfo{author}{{Schawinski}, K.}, \bibinfo{author}{{Koss}, M.},
  \bibinfo{author}{{Berney}, S.} \& \bibinfo{author}{{Sartori}, L.~F.}
\newblock {Active galactic nuclei flicker: an observational estimate of the
  duration of black hole growth phases of $\sim10^{5}$ yr}.
\newblock \textit{\bibinfo{journal}{\mnras}} \textbf{\bibinfo{volume}{451}},
  \bibinfo{pages}{2517--2523} (\bibinfo{year}{2015}).

\bibitem{Haardt:1994bq}
\bibinfo{author}{{Haardt}, F.}, \bibinfo{author}{{Maraschi}, L.} \&
  \bibinfo{author}{{Ghisellini}, G.}
\newblock {A model for the X-ray and ultraviolet emission from Seyfert galaxies
  and galactic black holes}.
\newblock \textit{\bibinfo{journal}{\apjl}} \textbf{\bibinfo{volume}{432}},
  \bibinfo{pages}{L95--L99} (\bibinfo{year}{1994}).

\bibitem{Capetti96}
\bibinfo{author}{{Capetti}, A.}, \bibinfo{author}{{Axon}, D.~J.},
  \bibinfo{author}{{Macchetto}, F.}, \bibinfo{author}{{Sparks}, W.~B.} \&
  \bibinfo{author}{{Boksenberg}, A.}
\newblock {Radio Outflows and the Origin of the Narrow-Line Region in Seyfert
  Galaxies}.
\newblock \textit{\bibinfo{journal}{\apj}} \textbf{\bibinfo{volume}{469}},
  \bibinfo{pages}{554} (\bibinfo{year}{1996}).

\bibitem{Antonucci84}
\bibinfo{author}{{Antonucci}, R.~R.~J.}
\newblock {Optical spectropolarimetry of radio galaxies}.
\newblock \textit{\bibinfo{journal}{\apj}} \textbf{\bibinfo{volume}{278}},
  \bibinfo{pages}{499--520} (\bibinfo{year}{1984}).

\bibitem{Antonucci93}
\bibinfo{author}{{Antonucci}, R.}
\newblock {Unified models for active galactic nuclei and quasars}.
\newblock \textit{\bibinfo{journal}{Ann. Rev. Astron. Astrophys.}} \textbf{\bibinfo{volume}{31}},
  \bibinfo{pages}{473--521} (\bibinfo{year}{1993}).

\bibitem{Urry95}
\bibinfo{author}{{Urry}, C.~M.} \& \bibinfo{author}{{Padovani}, P.}
\newblock {Unified Schemes for Radio-Loud Active Galactic Nuclei}.
\newblock \textit{\bibinfo{journal}{Publ. Astron. Soc. Pac.}} \textbf{\bibinfo{volume}{107}},
  \bibinfo{pages}{803} (\bibinfo{year}{1995}).

\bibitem{Packham05}
\bibinfo{author}{{Packham}, C.} \textit{et~al.}
\newblock {The Extended Mid-Infrared Structure of the Circinus Galaxy}.
\newblock \textit{\bibinfo{journal}{\apjl}} \textbf{\bibinfo{volume}{618}},
  \bibinfo{pages}{L17--L20} (\bibinfo{year}{2005}).

\bibitem{Radomski08}
\bibinfo{author}{{Radomski}, J.~T.} \textit{et~al.}
\newblock {Gemini Imaging of Mid-Infrared Emission from the Nuclear Region of
  Centaurus A}.
\newblock \textit{\bibinfo{journal}{\apj}} \textbf{\bibinfo{volume}{681}},
  \bibinfo{pages}{141--150} (\bibinfo{year}{2008}).

\bibitem{Burtscher13}
\bibinfo{author}{{Burtscher}, L.} \textit{et~al.}
\newblock {A diversity of dusty AGN tori. Data release for the VLTI/MIDI AGN
  Large Program and first results for 23 galaxies}.
\newblock \textit{\bibinfo{journal}{\aap}} \textbf{\bibinfo{volume}{558}},
  \bibinfo{pages}{A149} (\bibinfo{year}{2013}).

\bibitem{Imanishi16}
\bibinfo{author}{{Imanishi}, M.}, \bibinfo{author}{{Nakanishi}, K.} \&
  \bibinfo{author}{{Izumi}, T.}
\newblock {ALMA 0.1-0.2 arcsec Resolution Imaging of the NGC 1068 Nucleus:
  Compact Dense Molecular Gas Emission at the Putative AGN Location}.
\newblock \textit{\bibinfo{journal}{\apjl}} \textbf{\bibinfo{volume}{822}},
  \bibinfo{pages}{L10} (\bibinfo{year}{2016}).

\bibitem{Garcia16}
\bibinfo{author}{{Garc{\'{\i}}a-Burillo}, S.} \textit{et~al.}
\newblock {ALMA Resolves the Torus of NGC 1068: Continuum and Molecular Line
  Emission}.
\newblock \textit{\bibinfo{journal}{\apjl}} \textbf{\bibinfo{volume}{823}},
  \bibinfo{pages}{L12} (\bibinfo{year}{2016}).

\bibitem{Gallimore16}
\bibinfo{author}{{Gallimore}, J.~F.} \textit{et~al.}
\newblock {High-velocity Bipolar Molecular Emission from an AGN Torus}.
\newblock \textit{\bibinfo{journal}{\apjl}} \textbf{\bibinfo{volume}{829}},
  \bibinfo{pages}{L7} (\bibinfo{year}{2016}).

\bibitem{Malkan98}
\bibinfo{author}{{Malkan}, M.~A.}, \bibinfo{author}{{Gorjian}, V.} \&
  \bibinfo{author}{{Tam}, R.}
\newblock {A Hubble Space Telescope Imaging Survey of Nearby Active Galactic
  Nuclei}.
\newblock \textit{\bibinfo{journal}{\apjs}} \textbf{\bibinfo{volume}{117}},
  \bibinfo{pages}{25--88} (\bibinfo{year}{1998}).

\bibitem{Greenhill96}
\bibinfo{author}{{Greenhill}, L.~J.}, \bibinfo{author}{{Gwinn}, C.~R.},
  \bibinfo{author}{{Antonucci}, R.} \& \bibinfo{author}{{Barvainis}, R.}
\newblock {VLBI Imaging of Water Maser Emission from the Nuclear Torus of NGC
  1068}.
\newblock \textit{\bibinfo{journal}{\apjl}} \textbf{\bibinfo{volume}{472}},
  \bibinfo{pages}{L21} (\bibinfo{year}{1996}).

\bibitem{Masini16}
\bibinfo{author}{{Masini}, A.} \textit{et~al.}
\newblock {NuSTAR observations of water megamaser AGN}.
\newblock \textit{\bibinfo{journal}{\aap}} \textbf{\bibinfo{volume}{589}},
  \bibinfo{pages}{A59} (\bibinfo{year}{2016}).

\bibitem{Zoghbi:2012jk}
\bibinfo{author}{{Zoghbi}, A.}, \bibinfo{author}{{Fabian}, A.~C.},
  \bibinfo{author}{{Reynolds}, C.~S.} \& \bibinfo{author}{{Cackett}, E.~M.}
\newblock {Relativistic iron K X-ray reverberation in NGC 4151}.
\newblock \textit{\bibinfo{journal}{\mnras}} \textbf{\bibinfo{volume}{422}},
  \bibinfo{pages}{129--134} (\bibinfo{year}{2012}).

\bibitem{Burlon:2011dk}
\bibinfo{author}{{Burlon}, D.} \textit{et~al.}
\newblock {Three-year Swift-BAT Survey of Active Galactic Nuclei: Reconciling
  Theory and Observations?}
\newblock \textit{\bibinfo{journal}{\apj}} \textbf{\bibinfo{volume}{728}},
  \bibinfo{pages}{58} (\bibinfo{year}{2011}).

\bibitem{Ricci:2015tg}
\bibinfo{author}{{Ricci}, C.} \textit{et~al.}
\newblock {Compton-thick Accretion in the Local Universe}.
\newblock \textit{\bibinfo{journal}{\apjl}} \textbf{\bibinfo{volume}{815}},
  \bibinfo{pages}{L13} (\bibinfo{year}{2015}).

\bibitem{Risaliti07}
\bibinfo{author}{{Risaliti}, G.} \textit{et~al.}
\newblock {Occultation Measurement of the Size of the X-Ray-emitting Region in
  the Active Galactic Nucleus of NGC 1365}.
\newblock \textit{\bibinfo{journal}{\apjl}} \textbf{\bibinfo{volume}{659}},
  \bibinfo{pages}{L111--L114} (\bibinfo{year}{2007}).

\bibitem{Maiolino:2010fu}
\bibinfo{author}{{Maiolino}, R.} \textit{et~al.}
\newblock {``Comets'' orbiting a black hole}.
\newblock \textit{\bibinfo{journal}{\aap}} \textbf{\bibinfo{volume}{517}},
  \bibinfo{pages}{A47} (\bibinfo{year}{2010}).

\bibitem{Ramos09}
\bibinfo{author}{{Ramos Almeida}, C.} \textit{et~al.}
\newblock {The Infrared Nuclear Emission of Seyfert Galaxies on Parsec Scales:
  Testing the Clumpy Torus Models}.
\newblock \textit{\bibinfo{journal}{\apj}} \textbf{\bibinfo{volume}{702}},
  \bibinfo{pages}{1127--1149} (\bibinfo{year}{2009}).

\bibitem{Burtscher16}
\bibinfo{author}{{Burtscher}, L.} \textit{et~al.}
\newblock {On the relation of optical obscuration and X-ray absorption in
  Seyfert galaxies}.
\newblock \textit{\bibinfo{journal}{\aap}} \textbf{\bibinfo{volume}{586}},
  \bibinfo{pages}{A28} (\bibinfo{year}{2016}).

\bibitem{Awaki:1991rw}
\bibinfo{author}{{Awaki}, H.}, \bibinfo{author}{{Koyama}, K.},
  \bibinfo{author}{{Inoue}, H.} \& \bibinfo{author}{{Halpern}, J.~P.}
\newblock {X-ray implications of a unified model of Seyfert galaxies}.
\newblock \textit{\bibinfo{journal}{Publ. Astron. Soc. Jpn.}} \textbf{\bibinfo{volume}{43}},
  \bibinfo{pages}{195--212} (\bibinfo{year}{1991}).

\bibitem{Matt:1997qy}
\bibinfo{author}{{Matt}, G.} \textit{et~al.}
\newblock {Hard X-ray detection of NGC 1068 with BeppoSAX.}
\newblock \textit{\bibinfo{journal}{\aap}} \textbf{\bibinfo{volume}{325}},
  \bibinfo{pages}{L13--L16} (\bibinfo{year}{1997}).

\bibitem{Bauer:2015si}
\bibinfo{author}{{Bauer}, F.~E.} \textit{et~al.}
\newblock {NuSTAR Spectroscopy of Multi-component X-Ray Reflection from NGC
  1068}.
\newblock \textit{\bibinfo{journal}{\apj}} \textbf{\bibinfo{volume}{812}},
  \bibinfo{pages}{116} (\bibinfo{year}{2015}).

\bibitem{Panessa:2002if}
\bibinfo{author}{{Panessa}, F.} \& \bibinfo{author}{{Bassani}, L.}
\newblock {Unabsorbed Seyfert 2 galaxies}.
\newblock \textit{\bibinfo{journal}{\aap}} \textbf{\bibinfo{volume}{394}},
  \bibinfo{pages}{435--442} (\bibinfo{year}{2002}).

\bibitem{Nicastro:2000cq}
\bibinfo{author}{{Nicastro}, F.}
\newblock {Broad Emission Line Regions in Active Galactic Nuclei: The Link with
  the Accretion Power}.
\newblock \textit{\bibinfo{journal}{\apjl}} \textbf{\bibinfo{volume}{530}},
  \bibinfo{pages}{L65--L68} (\bibinfo{year}{2000}).

\bibitem{Elitzur:2009hh}
\bibinfo{author}{{Elitzur}, M.} \& \bibinfo{author}{{Ho}, L.~C.}
\newblock {On the Disappearance of the Broad-Line Region in Low-Luminosity
  Active Galactic Nuclei}.
\newblock \textit{\bibinfo{journal}{\apjl}} \textbf{\bibinfo{volume}{701}},
  \bibinfo{pages}{L91--L94} (\bibinfo{year}{2009}).

\bibitem{Merloni:2014wq}
\bibinfo{author}{{Merloni}, A.} \textit{et~al.}
\newblock {The incidence of obscuration in active galactic nuclei}.
\newblock \textit{\bibinfo{journal}{\mnras}} \textbf{\bibinfo{volume}{437}},
  \bibinfo{pages}{3550--3567} (\bibinfo{year}{2014}).

\bibitem{Davies:2015rw}
\bibinfo{author}{{Davies}, R.~I.} \textit{et~al.}
\newblock {Insights on the Dusty Torus and Neutral Torus from Optical and X-Ray
  Obscuration in a Complete Volume Limited Hard X-Ray AGN Sample}.
\newblock \textit{\bibinfo{journal}{\apj}} \textbf{\bibinfo{volume}{806}},
  \bibinfo{pages}{127} (\bibinfo{year}{2015}).

\bibitem{Kishimoto09}
\bibinfo{author}{{Kishimoto}, M.} \textit{et~al.}
\newblock {Exploring the inner region of type 1 AGNs with the Keck
  interferometer}.
\newblock \textit{\bibinfo{journal}{\aap}} \textbf{\bibinfo{volume}{507}},
  \bibinfo{pages}{L57--L60} (\bibinfo{year}{2009}).

\bibitem{Weigelt12}
\bibinfo{author}{{Weigelt}, G.} \textit{et~al.}
\newblock {VLTI/AMBER observations of the Seyfert nucleus of NGC 3783}.
\newblock \textit{\bibinfo{journal}{\aap}} \textbf{\bibinfo{volume}{541}},
  \bibinfo{pages}{L9} (\bibinfo{year}{2012}).

\bibitem{Kishimoto11}
\bibinfo{author}{{Kishimoto}, M.} \textit{et~al.}
\newblock {The innermost dusty structure in active galactic nuclei as probed by
  the Keck interferometer}.
\newblock \textit{\bibinfo{journal}{\aap}} \textbf{\bibinfo{volume}{527}},
  \bibinfo{pages}{A121} (\bibinfo{year}{2011}).

\bibitem{Suganuma06}
\bibinfo{author}{{Suganuma}, M.} \textit{et~al.}
\newblock {Reverberation Measurements of the Inner Radius of the Dust Torus in
  Nearby Seyfert 1 Galaxies}.
\newblock \textit{\bibinfo{journal}{\apj}} \textbf{\bibinfo{volume}{639}},
  \bibinfo{pages}{46--63} (\bibinfo{year}{2006}).

\bibitem{Barvainis87}
\bibinfo{author}{{Barvainis}, R.}
\newblock {Hot dust and the near-infrared bump in the continuum spectra of
  quasars and active galactic nuclei}.
\newblock \textit{\bibinfo{journal}{\apj}} \textbf{\bibinfo{volume}{320}},
  \bibinfo{pages}{537--544} (\bibinfo{year}{1987}).

\bibitem{Honig12}
\bibinfo{author}{{H{\"o}nig}, S.~F.} \textit{et~al.}
\newblock {Parsec-scale Dust Emission from the Polar Region in the Type 2
  Nucleus of NGC 424}.
\newblock \textit{\bibinfo{journal}{\apj}} \textbf{\bibinfo{volume}{755}},
  \bibinfo{pages}{149} (\bibinfo{year}{2012}).

\bibitem{Honig13}
\bibinfo{author}{{H{\"o}nig}, S.~F.} \textit{et~al.}
\newblock {Dust in the Polar Region as a Major Contributor to the Infrared
  Emission of Active Galactic Nuclei}.
\newblock \textit{\bibinfo{journal}{\apj}} \textbf{\bibinfo{volume}{771}},
  \bibinfo{pages}{87} (\bibinfo{year}{2013}).

\bibitem{Lopez16}
\bibinfo{author}{{L{\'o}pez-Gonzaga}, N.}, \bibinfo{author}{{Burtscher}, L.},
  \bibinfo{author}{{Tristram}, K.~R.~W.}, \bibinfo{author}{{Meisenheimer}, K.}
  \& \bibinfo{author}{{Schartmann}, M.}
\newblock {Mid-infrared interferometry of 23 AGN tori: On the significance of
  polar-elongated emission}.
\newblock \textit{\bibinfo{journal}{\aap}} \textbf{\bibinfo{volume}{591}},
  \bibinfo{pages}{A47} (\bibinfo{year}{2016}).

\bibitem{Markowitz:2014oq}
\bibinfo{author}{{Markowitz}, A.~G.}, \bibinfo{author}{{Krumpe}, M.} \&
  \bibinfo{author}{{Nikutta}, R.}
\newblock {First X-ray-based statistical tests for clumpy-torus models: eclipse
  events from 230 years of monitoring of Seyfert AGN}.
\newblock \textit{\bibinfo{journal}{\mnras}} \textbf{\bibinfo{volume}{439}},
  \bibinfo{pages}{1403--1458} (\bibinfo{year}{2014}).

\bibitem{Tristram14}
\bibinfo{author}{{Tristram}, K.~R.~W.} \textit{et~al.}
\newblock {The dusty torus in the Circinus galaxy: a dense disk and the torus
  funnel}.
\newblock \textit{\bibinfo{journal}{\aap}} \textbf{\bibinfo{volume}{563}},
  \bibinfo{pages}{A82} (\bibinfo{year}{2014}).

\bibitem{Lopez14}
\bibinfo{author}{{L{\'o}pez-Gonzaga}, N.}, \bibinfo{author}{{Jaffe}, W.},
  \bibinfo{author}{{Burtscher}, L.}, \bibinfo{author}{{Tristram}, K.~R.~W.} \&
  \bibinfo{author}{{Meisenheimer}, K.}
\newblock {Revealing the large nuclear dust structures in NGC 1068 with
  MIDI/VLTI}.
\newblock \textit{\bibinfo{journal}{\aap}} \textbf{\bibinfo{volume}{565}},
  \bibinfo{pages}{A71} (\bibinfo{year}{2014}).

\bibitem{Gallimore96}
\bibinfo{author}{{Gallimore}, J.~F.}, \bibinfo{author}{{Baum}, S.~A.},
  \bibinfo{author}{{O'Dea}, C.~P.}, \bibinfo{author}{{Brinks}, E.} \&
  \bibinfo{author}{{Pedlar}, A.}
\newblock {H 2O and OH Masers as Probes of the Obscuring Torus in NGC 1068}.
\newblock \textit{\bibinfo{journal}{\apj}} \textbf{\bibinfo{volume}{462}},
  \bibinfo{pages}{740} (\bibinfo{year}{1996}).

\bibitem{Gallimore97}
\bibinfo{author}{{Gallimore}, J.~F.}, \bibinfo{author}{{Baum}, S.~A.} \&
  \bibinfo{author}{{O'Dea}, C.~P.}
\newblock {A direct image of the obscuring disk surrounding an active galactic
  nucleus}.
\newblock \textit{\bibinfo{journal}{\nat}} \textbf{\bibinfo{volume}{388}},
  \bibinfo{pages}{852--854} (\bibinfo{year}{1997}).

\bibitem{Muller09}
\bibinfo{author}{{M{\"u}ller S{\'a}nchez}, F.} \textit{et~al.}
\newblock {Molecular Gas Streamers Feeding and Obscuring the Active Nucleus of
  NGC 1068}.
\newblock \textit{\bibinfo{journal}{\apj}} \textbf{\bibinfo{volume}{691}},
  \bibinfo{pages}{749--759} (\bibinfo{year}{2009}).

\bibitem{Hicks13}
\bibinfo{author}{{Hicks}, E.~K.~S.} \textit{et~al.}
\newblock {Fueling Active Galactic Nuclei. I. How the Global Characteristics of
  the Central Kiloparsec of Seyferts Differ from Quiescent Galaxies}.
\newblock \textit{\bibinfo{journal}{\apj}} \textbf{\bibinfo{volume}{768}},
  \bibinfo{pages}{107} (\bibinfo{year}{2013}).

\bibitem{Matt:1991ly}
\bibinfo{author}{{Matt}, G.}, \bibinfo{author}{{Perola}, G.~C.} \&
  \bibinfo{author}{{Piro}, L.}
\newblock {The iron line and high energy bump as X-ray signatures of cold
  matter in Seyfert 1 galaxies}.
\newblock \textit{\bibinfo{journal}{\aap}} \textbf{\bibinfo{volume}{247}},
  \bibinfo{pages}{25--34} (\bibinfo{year}{1991}).

\bibitem{Shu:2010tg}
\bibinfo{author}{{Shu}, X.~W.}, \bibinfo{author}{{Yaqoob}, T.} \&
  \bibinfo{author}{{Wang}, J.~X.}
\newblock {The Cores of the Fe K{$\alpha$} Lines in Active Galactic Nuclei: An
  Extended Chandra High Energy Grating Sample}.
\newblock \textit{\bibinfo{journal}{\apjs}} \textbf{\bibinfo{volume}{187}},
  \bibinfo{pages}{581--606} (\bibinfo{year}{2010}).

\bibitem{Nandra:1994ly}
\bibinfo{author}{{Nandra}, K.} \& \bibinfo{author}{{Pounds}, K.~A.}
\newblock {GINGA Observations of the X-Ray Spectra of Seyfert Galaxies}.
\newblock \textit{\bibinfo{journal}{\mnras}} \textbf{\bibinfo{volume}{268}},
  \bibinfo{pages}{405} (\bibinfo{year}{1994}).

\bibitem{Bianchi:2008sf}
\bibinfo{author}{{Bianchi}, S.} \textit{et~al.}
\newblock {A broad-line region origin for the iron K{$\alpha$} line in NGC
  7213}.
\newblock \textit{\bibinfo{journal}{\mnras}} \textbf{\bibinfo{volume}{389}},
  \bibinfo{pages}{L52--L56} (\bibinfo{year}{2008}).

\bibitem{Gandhi:2015zp}
\bibinfo{author}{{Gandhi}, P.}, \bibinfo{author}{{H{\"o}nig}, S.~F.} \&
  \bibinfo{author}{{Kishimoto}, M.}
\newblock {The Dust Sublimation Radius as an Outer Envelope to the Bulk of the
  Narrow Fe Kalpha Line Emission in Type 1 AGNs}.
\newblock \textit{\bibinfo{journal}{\apj}} \textbf{\bibinfo{volume}{812}},
  \bibinfo{pages}{113} (\bibinfo{year}{2015}).

\bibitem{Ricci:2014ek}
\bibinfo{author}{{Ricci}, C.} \textit{et~al.}
\newblock {Iron K{$\alpha$} emission in type-I and type-II active galactic
  nuclei}.
\newblock \textit{\bibinfo{journal}{\mnras}} \textbf{\bibinfo{volume}{441}},
  \bibinfo{pages}{3622--3633} (\bibinfo{year}{2014}).

\bibitem{Matt:2002eu}
\bibinfo{author}{{Matt}, G.}
\newblock {The iron K{$\alpha$} Compton shoulder in transmitted and reflected
  spectra}.
\newblock \textit{\bibinfo{journal}{\mnras}} \textbf{\bibinfo{volume}{337}},
  \bibinfo{pages}{147--150} (\bibinfo{year}{2002}).

\bibitem{Alonso03}
\bibinfo{author}{{Alonso-Herrero}, A.}, \bibinfo{author}{{Quillen}, A.~C.},
  \bibinfo{author}{{Rieke}, G.~H.}, \bibinfo{author}{{Ivanov}, V.~D.} \&
  \bibinfo{author}{{Efstathiou}, A.}
\newblock {Spectral Energy Distributions of Seyfert Nuclei}.
\newblock \textit{\bibinfo{journal}{\aj}} \textbf{\bibinfo{volume}{126}},
  \bibinfo{pages}{81--100} (\bibinfo{year}{2003}).

\bibitem{Prieto10}
\bibinfo{author}{{Prieto}, M.~A.} \textit{et~al.}
\newblock {The spectral energy distribution of the central parsecs of the
  nearest AGN}.
\newblock \textit{\bibinfo{journal}{\mnras}} \textbf{\bibinfo{volume}{402}},
  \bibinfo{pages}{724--744} (\bibinfo{year}{2010}).

\bibitem{Asmus14}
\bibinfo{author}{{Asmus}, D.}, \bibinfo{author}{{H{\"o}nig}, S.~F.},
  \bibinfo{author}{{Gandhi}, P.}, \bibinfo{author}{{Smette}, A.} \&
  \bibinfo{author}{{Duschl}, W.~J.}
\newblock {The subarcsecond mid-infrared view of local active galactic nuclei -
  I. The N- and Q-band imaging atlas}.
\newblock \textit{\bibinfo{journal}{\mnras}} \textbf{\bibinfo{volume}{439}},
  \bibinfo{pages}{1648--1679} (\bibinfo{year}{2014}).

\bibitem{Levenson09}
\bibinfo{author}{{Levenson}, N.~A.} \textit{et~al.}
\newblock {Isotropic Mid-Infrared Emission from the Central 100 pc of Active
  Galaxies}.
\newblock \textit{\bibinfo{journal}{\apj}} \textbf{\bibinfo{volume}{703}},
  \bibinfo{pages}{390--398} (\bibinfo{year}{2009}).

\bibitem{Ramos11}
\bibinfo{author}{{Ramos Almeida}, C.} \textit{et~al.}
\newblock {Testing the Unification Model for Active Galactic Nuclei in the
  Infrared: Are the Obscuring Tori of Type 1 and 2 Seyferts Different?}
\newblock \textit{\bibinfo{journal}{\apj}} \textbf{\bibinfo{volume}{731}},
  \bibinfo{pages}{92} (\bibinfo{year}{2011}).

\bibitem{Pier92}
\bibinfo{author}{{Pier}, E.~A.} \& \bibinfo{author}{{Krolik}, J.~H.}
\newblock {Infrared spectra of obscuring dust tori around active galactic
  nuclei. I - Calculational method and basic trends}.
\newblock \textit{\bibinfo{journal}{\apj}} \textbf{\bibinfo{volume}{401}},
  \bibinfo{pages}{99--109} (\bibinfo{year}{1992}).

\bibitem{Pier93}
\bibinfo{author}{{Pier}, E.~A.} \& \bibinfo{author}{{Krolik}, J.~H.}
\newblock {Infrared Spectra of Obscuring Dust Tori around Active Galactic
  Nuclei. II. Comparison with Observations}.
\newblock \textit{\bibinfo{journal}{\apj}} \textbf{\bibinfo{volume}{418}},
  \bibinfo{pages}{673} (\bibinfo{year}{1993}).

\bibitem{Honig11}
\bibinfo{author}{{H{\"o}nig}, S.~F.}, \bibinfo{author}{{Leipski}, C.},
  \bibinfo{author}{{Antonucci}, R.} \& \bibinfo{author}{{Haas}, M.}
\newblock {Quantifying the Anisotropy in the Infrared Emission of Powerful
  Active Galactic Nuclei}.
\newblock \textit{\bibinfo{journal}{\apj}} \textbf{\bibinfo{volume}{736}},
  \bibinfo{pages}{26} (\bibinfo{year}{2011}).

\bibitem{Krabbe01}
\bibinfo{author}{{Krabbe}, A.}, \bibinfo{author}{{B{\"o}ker}, T.} \&
  \bibinfo{author}{{Maiolino}, R.}
\newblock {N-Band Imaging of Seyfert Nuclei and the Mid-Infrared-X-Ray
  Correlation}.
\newblock \textit{\bibinfo{journal}{\apj}} \textbf{\bibinfo{volume}{557}},
  \bibinfo{pages}{626--636} (\bibinfo{year}{2001}).

\bibitem{Lutz:2004gf}
\bibinfo{author}{{Lutz}, D.}, \bibinfo{author}{{Maiolino}, R.},
  \bibinfo{author}{{Spoon}, H.~W.~W.} \& \bibinfo{author}{{Moorwood}, A.~F.~M.}
\newblock {The relation between AGN hard X-ray emission and mid-infrared
  continuum from ISO spectra: Scatter and unification aspects}.
\newblock \textit{\bibinfo{journal}{\aap}} \textbf{\bibinfo{volume}{418}},
  \bibinfo{pages}{465--473} (\bibinfo{year}{2004}).

\bibitem{Asmus15}
\bibinfo{author}{{Asmus}, D.}, \bibinfo{author}{{Gandhi}, P.},
  \bibinfo{author}{{H{\"o}nig}, S.~F.}, \bibinfo{author}{{Smette}, A.} \&
  \bibinfo{author}{{Duschl}, W.~J.}
\newblock {The subarcsecond mid-infrared view of local active galactic nuclei -
  II. The mid-infrared-X-ray correlation}.
\newblock \textit{\bibinfo{journal}{\mnras}} \textbf{\bibinfo{volume}{454}},
  \bibinfo{pages}{766--803} (\bibinfo{year}{2015}).

\bibitem{Roche91}
\bibinfo{author}{{Roche}, P.~F.}, \bibinfo{author}{{Aitken}, D.~K.},
  \bibinfo{author}{{Smith}, C.~H.} \& \bibinfo{author}{{Ward}, M.~J.}
\newblock {An atlas of mid-infrared spectra of galaxy nuclei}.
\newblock \textit{\bibinfo{journal}{\mnras}} \textbf{\bibinfo{volume}{248}},
  \bibinfo{pages}{606--629} (\bibinfo{year}{1991}).

\bibitem{Mason09}
\bibinfo{author}{{Mason}, R.~E.} \textit{et~al.}
\newblock {The Origin of the Silicate Emission Features in the Seyfert 2 Galaxy
  NGC 2110}.
\newblock \textit{\bibinfo{journal}{\apjl}} \textbf{\bibinfo{volume}{693}},
  \bibinfo{pages}{L136--L140} (\bibinfo{year}{2009}).

\bibitem{Shi06}
\bibinfo{author}{{Shi}, Y.} \textit{et~al.}
\newblock {9.7 {$\mu$}m Silicate Features in Active Galactic Nuclei: New
  Insights into Unification Models}.
\newblock \textit{\bibinfo{journal}{\apj}} \textbf{\bibinfo{volume}{653}},
  \bibinfo{pages}{127--136} (\bibinfo{year}{2006}).

\bibitem{Roche06}
\bibinfo{author}{{Roche}, P.~F.} \textit{et~al.}
\newblock {Mid-infrared, spatially resolved spectroscopy of the nucleus of the
  Circinus galaxy}.
\newblock \textit{\bibinfo{journal}{\mnras}} \textbf{\bibinfo{volume}{367}},
  \bibinfo{pages}{1689--1698} (\bibinfo{year}{2006}).

\bibitem{Alonso16}
\bibinfo{author}{{Alonso-Herrero}, A.} \textit{et~al.}
\newblock {A mid-infrared spectroscopic atlas of local active galactic nuclei
  on sub-arcsecond resolution using GTC/CanariCam}.
\newblock \textit{\bibinfo{journal}{\mnras}} \textbf{\bibinfo{volume}{455}},
  \bibinfo{pages}{563--583} (\bibinfo{year}{2016}).

\bibitem{Maiolino01}
\bibinfo{author}{{Maiolino}, R.}, \bibinfo{author}{{Marconi}, A.} \&
  \bibinfo{author}{{Oliva}, E.}
\newblock {Dust in active nuclei. II. Powder or gravel?}
\newblock \textit{\bibinfo{journal}{\aap}} \textbf{\bibinfo{volume}{365}},
  \bibinfo{pages}{37--48} (\bibinfo{year}{2001}).

\bibitem{Krolik88}
\bibinfo{author}{{Krolik}, J.~H.} \& \bibinfo{author}{{Begelman}, M.~C.}
\newblock {Molecular tori in Seyfert galaxies - Feeding the monster and hiding
  it}.
\newblock \textit{\bibinfo{journal}{\apj}} \textbf{\bibinfo{volume}{329}},
  \bibinfo{pages}{702--711} (\bibinfo{year}{1988}).

\bibitem{Marinucci:2016eu}
\bibinfo{author}{{Marinucci}, A.} \textit{et~al.}
\newblock {NuSTAR catches the unveiling nucleus of NGC 1068}.
\newblock \textit{\bibinfo{journal}{\mnras}} \textbf{\bibinfo{volume}{456}},
  \bibinfo{pages}{L94--L98} (\bibinfo{year}{2016}).

\bibitem{Schartmann08}
\bibinfo{author}{{Schartmann}, M.} \textit{et~al.}
\newblock {Three-dimensional radiative transfer models of clumpy tori in
  Seyfert galaxies}.
\newblock \textit{\bibinfo{journal}{\aap}} \textbf{\bibinfo{volume}{482}},
  \bibinfo{pages}{67--80} (\bibinfo{year}{2008}).

\bibitem{Wada02}
\bibinfo{author}{{Wada}, K.} \& \bibinfo{author}{{Norman}, C.~A.}
\newblock {Obscuring Material around Seyfert Nuclei with Starbursts}.
\newblock \textit{\bibinfo{journal}{\apjl}} \textbf{\bibinfo{volume}{566}},
  \bibinfo{pages}{L21--L24} (\bibinfo{year}{2002}).

\bibitem{Wada12}
\bibinfo{author}{{Wada}, K.}
\newblock {Radiation-driven Fountain and Origin of Torus around Active Galactic
  Nuclei}.
\newblock \textit{\bibinfo{journal}{\apj}} \textbf{\bibinfo{volume}{758}},
  \bibinfo{pages}{66} (\bibinfo{year}{2012}).

\bibitem{Nenkova08a}
\bibinfo{author}{{Nenkova}, M.}, \bibinfo{author}{{Sirocky}, M.~M.},
  \bibinfo{author}{{Ivezi{\'c}}, {\v Z}.} \& \bibinfo{author}{{Elitzur}, M.}
\newblock {AGN Dusty Tori. I. Handling of Clumpy Media}.
\newblock \textit{\bibinfo{journal}{\apj}} \textbf{\bibinfo{volume}{685}},
  \bibinfo{pages}{147--159} (\bibinfo{year}{2008}).

\bibitem{Nenkova08b}
\bibinfo{author}{{Nenkova}, M.}, \bibinfo{author}{{Sirocky}, M.~M.},
  \bibinfo{author}{{Nikutta}, R.}, \bibinfo{author}{{Ivezi{\'c}}, {\v Z}.} \&
  \bibinfo{author}{{Elitzur}, M.}
\newblock {AGN Dusty Tori. II. Observational Implications of Clumpiness}.
\newblock \textit{\bibinfo{journal}{\apj}} \textbf{\bibinfo{volume}{685}},
  \bibinfo{pages}{160--180} (\bibinfo{year}{2008}).

\bibitem{Honig10}
\bibinfo{author}{{H{\"o}nig}, S.~F.} \textit{et~al.}
\newblock {The dusty heart of nearby active galaxies. I. High-spatial
  resolution mid-IR spectro-photometry of Seyfert galaxies}.
\newblock \textit{\bibinfo{journal}{\aap}} \textbf{\bibinfo{volume}{515}},
  \bibinfo{pages}{A23} (\bibinfo{year}{2010}).

\bibitem{Stalevski12}
\bibinfo{author}{{Stalevski}, M.}, \bibinfo{author}{{Fritz}, J.},
  \bibinfo{author}{{Baes}, M.}, \bibinfo{author}{{Nakos}, T.} \&
  \bibinfo{author}{{Popovi{\'c}}, L.~{\v C}.}
\newblock {3D radiative transfer modelling of the dusty tori around active
  galactic nuclei as a clumpy two-phase medium}.
\newblock \textit{\bibinfo{journal}{\mnras}} \textbf{\bibinfo{volume}{420}},
  \bibinfo{pages}{2756--2772} (\bibinfo{year}{2012}).

\bibitem{Siebenmorgen15}
\bibinfo{author}{{Siebenmorgen}, R.}, \bibinfo{author}{{Heymann}, F.} \&
  \bibinfo{author}{{Efstathiou}, A.}
\newblock {Self-consistent two-phase AGN torus models. SED library for
  observers}.
\newblock \textit{\bibinfo{journal}{\aap}} \textbf{\bibinfo{volume}{583}},
  \bibinfo{pages}{A120} (\bibinfo{year}{2015}).

\bibitem{Feltre12}
\bibinfo{author}{{Feltre}, A.}, \bibinfo{author}{{Hatziminaoglou}, E.},
  \bibinfo{author}{{Fritz}, J.} \& \bibinfo{author}{{Franceschini}, A.}
\newblock {Smooth and clumpy dust distributions in AGN: a direct comparison of
  two commonly explored infrared emission models}.
\newblock \textit{\bibinfo{journal}{\mnras}} \textbf{\bibinfo{volume}{426}},
  \bibinfo{pages}{120--127} (\bibinfo{year}{2012}).

\bibitem{Ichikawa15}
\bibinfo{author}{{Ichikawa}, K.} \textit{et~al.}
\newblock {The Differences in the Torus Geometry between Hidden and Non-hidden
  Broad Line Active Galactic Nuclei}.
\newblock \textit{\bibinfo{journal}{\apj}} \textbf{\bibinfo{volume}{803}},
  \bibinfo{pages}{57} (\bibinfo{year}{2015}).

\bibitem{Mor09}
\bibinfo{author}{{Mor}, R.}, \bibinfo{author}{{Netzer}, H.} \&
  \bibinfo{author}{{Elitzur}, M.}
\newblock {Dusty Structure Around Type-I Active Galactic Nuclei: Clumpy Torus
  Narrow-line Region and Near-nucleus Hot Dust}.
\newblock \textit{\bibinfo{journal}{\apj}} \textbf{\bibinfo{volume}{705}},
  \bibinfo{pages}{298--313} (\bibinfo{year}{2009}).

\bibitem{Alonso11}
\bibinfo{author}{{Alonso-Herrero}, A.} \textit{et~al.}
\newblock {Torus and Active Galactic Nucleus Properties of Nearby Seyfert
  Galaxies: Results from Fitting Infrared Spectral Energy Distributions and
  Spectroscopy}.
\newblock \textit{\bibinfo{journal}{\apj}} \textbf{\bibinfo{volume}{736}},
  \bibinfo{pages}{82} (\bibinfo{year}{2011}).

\bibitem{Lira13}
\bibinfo{author}{{Lira}, P.} \textit{et~al.}
\newblock {Modeling the Nuclear Infrared Spectral Energy Distribution of Type
  II Active Galactic Nuclei}.
\newblock \textit{\bibinfo{journal}{\apj}} \textbf{\bibinfo{volume}{764}},
  \bibinfo{pages}{159} (\bibinfo{year}{2013}).

\bibitem{Ramos14}
\bibinfo{author}{{Ramos Almeida}, C.} \textit{et~al.}
\newblock {Investigating the sensitivity of observed spectral energy
  distributions to clumpy torus properties in Seyfert galaxies}.
\newblock \textit{\bibinfo{journal}{\mnras}} \textbf{\bibinfo{volume}{439}},
  \bibinfo{pages}{3847--3859} (\bibinfo{year}{2014}).

\bibitem{Roseboom13}
\bibinfo{author}{{Roseboom}, I.~G.} \textit{et~al.}
\newblock {IR-derived covering factors for a large sample of quasars from
  WISE-UKIDSS-SDSS}.
\newblock \textit{\bibinfo{journal}{\mnras}} \textbf{\bibinfo{volume}{429}},
  \bibinfo{pages}{1494--1501} (\bibinfo{year}{2013}).

\bibitem{Honig17}
\bibinfo{author}{{H{\"o}nig}, S.~F.} \& \bibinfo{author}{{Kishimoto}, M.}
\newblock {Dusty Winds in Active Galactic Nuclei: Reconciling Observations with
  Models}.
\newblock \textit{\bibinfo{journal}{\apjl}} \textbf{\bibinfo{volume}{838}},
  \bibinfo{pages}{L20} (\bibinfo{year}{2017}).

\bibitem{Gonzalez15}
\bibinfo{author}{{Gonz{\'a}lez-Mart{\'{\i}}n}, O.} \textit{et~al.}
\newblock {Nuclear obscuration in LINERs. Clues from Spitzer/IRS spectra on the
  Compton thickness and the existence of the dusty torus}.
\newblock \textit{\bibinfo{journal}{\aap}} \textbf{\bibinfo{volume}{578}},
  \bibinfo{pages}{A74} (\bibinfo{year}{2015}).

\bibitem{Ikeda:2009hb}
\bibinfo{author}{{Ikeda}, S.}, \bibinfo{author}{{Awaki}, H.} \&
  \bibinfo{author}{{Terashima}, Y.}
\newblock {Study on X-Ray Spectra of Obscured Active Galactic Nuclei Based on
  Monte Carlo Simulation - An Interpretation of Observed Wide-Band Spectra}.
\newblock \textit{\bibinfo{journal}{\apj}} \textbf{\bibinfo{volume}{692}},
  \bibinfo{pages}{608--617} (\bibinfo{year}{2009}).

\bibitem{Murphy:2009hb}
\bibinfo{author}{{Murphy}, K.~D.} \& \bibinfo{author}{{Yaqoob}, T.}
\newblock {An X-ray spectral model for Compton-thick toroidal reprocessors}.
\newblock \textit{\bibinfo{journal}{\mnras}} \textbf{\bibinfo{volume}{397}},
  \bibinfo{pages}{1549--1562} (\bibinfo{year}{2009}).

\bibitem{Brightman:2011fe}
\bibinfo{author}{{Brightman}, M.} \& \bibinfo{author}{{Nandra}, K.}
\newblock {An XMM-Newton spectral survey of 12 {$\mu$}m selected galaxies - I.
  X-ray data}.
\newblock \textit{\bibinfo{journal}{\mnras}} \textbf{\bibinfo{volume}{413}},
  \bibinfo{pages}{1206--1235} (\bibinfo{year}{2011}).

\bibitem{Liu:2014ff}
\bibinfo{author}{{Liu}, Y.} \& \bibinfo{author}{{Li}, X.}
\newblock {An X-Ray Spectral Model for Clumpy Tori in Active Galactic Nuclei}.
\newblock \textit{\bibinfo{journal}{\apj}} \textbf{\bibinfo{volume}{787}},
  \bibinfo{pages}{52} (\bibinfo{year}{2014}).

\bibitem{Furui:2016qf}
\bibinfo{author}{{Furui}, S.} \textit{et~al.}
\newblock {X-Ray Spectral Model of Reprocess by Smooth and Clumpy Molecular
  Tori in Active Galactic Nuclei with the Framework MONACO}.
\newblock \textit{\bibinfo{journal}{\apj}} \textbf{\bibinfo{volume}{818}},
  \bibinfo{pages}{164} (\bibinfo{year}{2016}).

\bibitem{Balokovic:2014dq}
\bibinfo{author}{{Balokovi{\'c}}, M.} \textit{et~al.}
\newblock {The NuSTAR View of Nearby Compton-thick Active Galactic Nuclei: The
  Cases of NGC 424, NGC 1320, and IC 2560}.
\newblock \textit{\bibinfo{journal}{\apj}} \textbf{\bibinfo{volume}{794}},
  \bibinfo{pages}{111} (\bibinfo{year}{2014}).

\bibitem{Annuar:2015wd}
\bibinfo{author}{{Annuar}, A.} \textit{et~al.}
\newblock {NuSTAR Observations of the Compton-thick Active Galactic Nucleus and
  Ultraluminous X-Ray Source Candidate in NGC 5643}.
\newblock \textit{\bibinfo{journal}{\apj}} \textbf{\bibinfo{volume}{815}},
  \bibinfo{pages}{36} (\bibinfo{year}{2015}).

\bibitem{Koss:2016fv}
\bibinfo{author}{{Koss}, M.~J.} \textit{et~al.}
\newblock {A New Population of Compton-thick AGNs Identified Using the Spectral
  Curvature above 10 keV}.
\newblock \textit{\bibinfo{journal}{\apj}} \textbf{\bibinfo{volume}{825}},
  \bibinfo{pages}{85} (\bibinfo{year}{2016}).

\bibitem{Yaqoob:2012wu}
\bibinfo{author}{{Yaqoob}, T.}
\newblock {The nature of the Compton-thick X-ray reprocessor in NGC 4945}.
\newblock \textit{\bibinfo{journal}{\mnras}} \textbf{\bibinfo{volume}{423}},
  \bibinfo{pages}{3360--3396} (\bibinfo{year}{2012}).

\bibitem{Risaliti:1999dw}
\bibinfo{author}{{Risaliti}, G.}, \bibinfo{author}{{Maiolino}, R.} \&
  \bibinfo{author}{{Salvati}, M.}
\newblock {The Distribution of Absorbing Column Densities among Seyfert 2
  Galaxies}.
\newblock \textit{\bibinfo{journal}{\apj}} \textbf{\bibinfo{volume}{522}},
  \bibinfo{pages}{157--164} (\bibinfo{year}{1999}).

\bibitem{Ueda:2014ix}
\bibinfo{author}{{Ueda}, Y.}, \bibinfo{author}{{Akiyama}, M.},
  \bibinfo{author}{{Hasinger}, G.}, \bibinfo{author}{{Miyaji}, T.} \&
  \bibinfo{author}{{Watson}, M.~G.}
\newblock {Toward the Standard Population Synthesis Model of the X-Ray
  Background: Evolution of X-Ray Luminosity and Absorption Functions of Active
  Galactic Nuclei Including Compton-thick Populations}.
\newblock \textit{\bibinfo{journal}{\apj}} \textbf{\bibinfo{volume}{786}},
  \bibinfo{pages}{104} (\bibinfo{year}{2014}).

\bibitem{Brightman:2014zp}
\bibinfo{author}{{Brightman}, M.} \textit{et~al.}
\newblock {Compton thick active galactic nuclei in Chandra surveys}.
\newblock \textit{\bibinfo{journal}{\mnras}} \textbf{\bibinfo{volume}{443}},
  \bibinfo{pages}{1999--2017} (\bibinfo{year}{2014}).

\bibitem{Buchner:2015ve}
\bibinfo{author}{{Buchner}, J.} \textit{et~al.}
\newblock {Obscuration-dependent Evolution of Active Galactic Nuclei}.
\newblock \textit{\bibinfo{journal}{\apj}} \textbf{\bibinfo{volume}{802}},
  \bibinfo{pages}{89} (\bibinfo{year}{2015}).

\bibitem{Lanzuisi:2015qr}
\bibinfo{author}{{Lanzuisi}, G.} \textit{et~al.}
\newblock {Compton thick AGN in the XMM-COSMOS survey}.
\newblock \textit{\bibinfo{journal}{\aap}} \textbf{\bibinfo{volume}{573}},
  \bibinfo{pages}{A137} (\bibinfo{year}{2015}).

\bibitem{La-Franca:2005kl}
\bibinfo{author}{{La Franca}, F.} \textit{et~al.}
\newblock {The HELLAS2XMM Survey. VII. The Hard X-Ray Luminosity Function of
  AGNs up to z = 4: More Absorbed AGNs at Low Luminosities and High Redshifts}.
\newblock \textit{\bibinfo{journal}{\apj}} \textbf{\bibinfo{volume}{635}},
  \bibinfo{pages}{864--879} (\bibinfo{year}{2005}).

\bibitem{Akylas:2006gd}
\bibinfo{author}{{Akylas}, A.}, \bibinfo{author}{{Georgantopoulos}, I.},
  \bibinfo{author}{{Georgakakis}, A.}, \bibinfo{author}{{Kitsionas}, S.} \&
  \bibinfo{author}{{Hatziminaoglou}, E.}
\newblock {XMM-Newton and Chandra measurements of the AGN intrinsic absorption:
  dependence on luminosity and redshift}.
\newblock \textit{\bibinfo{journal}{\aap}} \textbf{\bibinfo{volume}{459}},
  \bibinfo{pages}{693--701} (\bibinfo{year}{2006}).

\bibitem{Brightman:2015fv}
\bibinfo{author}{{Brightman}, M.} \textit{et~al.}
\newblock {Determining the Covering Factor of Compton-thick Active Galactic
  Nuclei with NuSTAR}.
\newblock \textit{\bibinfo{journal}{\apj}} \textbf{\bibinfo{volume}{805}},
  \bibinfo{pages}{41} (\bibinfo{year}{2015}).

\bibitem{Maiolino:2007ye}
\bibinfo{author}{{Maiolino}, R.} \textit{et~al.}
\newblock {Dust covering factor, silicate emission, and star formation in
  luminous QSOs}.
\newblock \textit{\bibinfo{journal}{\aap}} \textbf{\bibinfo{volume}{468}},
  \bibinfo{pages}{979--992} (\bibinfo{year}{2007}).

\bibitem{Treister:2008kc}
\bibinfo{author}{{Treister}, E.}, \bibinfo{author}{{Krolik}, J.~H.} \&
  \bibinfo{author}{{Dullemond}, C.}
\newblock {Measuring the Fraction of Obscured Quasars by the Infrared
  Luminosity of Unobscured Quasars}.
\newblock \textit{\bibinfo{journal}{\apj}} \textbf{\bibinfo{volume}{679}},
  \bibinfo{pages}{140--148} (\bibinfo{year}{2008}).

\bibitem{Lusso:2013vf}
\bibinfo{author}{{Lusso}, E.} \textit{et~al.}
\newblock {The Obscured Fraction of Active Galactic Nuclei in the XMM-COSMOS
  Survey: A Spectral Energy Distribution Perspective}.
\newblock \textit{\bibinfo{journal}{\apj}} \textbf{\bibinfo{volume}{777}},
  \bibinfo{pages}{86} (\bibinfo{year}{2013}).

\bibitem{Stalevski:2016hl}
\bibinfo{author}{{Stalevski}, M.} \textit{et~al.}
\newblock {The dust covering factor in active galactic nuclei}.
\newblock \textit{\bibinfo{journal}{\mnras}} \textbf{\bibinfo{volume}{458}},
  \bibinfo{pages}{2288--2302} (\bibinfo{year}{2016}).

\bibitem{Sazonov:2015ys}
\bibinfo{author}{{Sazonov}, S.}, \bibinfo{author}{{Churazov}, E.} \&
  \bibinfo{author}{{Krivonos}, R.}
\newblock {Does the obscured AGN fraction really depend on luminosity?}
\newblock \textit{\bibinfo{journal}{\mnras}} \textbf{\bibinfo{volume}{454}},
  \bibinfo{pages}{1202--1220} (\bibinfo{year}{2015}).

\bibitem{Iwasawa:1993ez}
\bibinfo{author}{{Iwasawa}, K.} \& \bibinfo{author}{{Taniguchi}, Y.}
\newblock {The X-ray Baldwin effect}.
\newblock \textit{\bibinfo{journal}{\apjl}} \textbf{\bibinfo{volume}{413}},
  \bibinfo{pages}{L15--L18} (\bibinfo{year}{1993}).

\bibitem{Bianchi:2007os}
\bibinfo{author}{{Bianchi}, S.}, \bibinfo{author}{{Guainazzi}, M.},
  \bibinfo{author}{{Matt}, G.} \& \bibinfo{author}{{Fonseca Bonilla}, N.}
\newblock {On the Iwasawa-Taniguchi effect of radio-quiet AGN}.
\newblock \textit{\bibinfo{journal}{\aap}} \textbf{\bibinfo{volume}{467}},
  \bibinfo{pages}{L19--L22} (\bibinfo{year}{2007}).

\bibitem{Ricci:2013hi}
\bibinfo{author}{{Ricci}, C.} \textit{et~al.}
\newblock {Luminosity-dependent unification of active galactic nuclei and the
  X-ray Baldwin effect}.
\newblock \textit{\bibinfo{journal}{\aap}} \textbf{\bibinfo{volume}{553}},
  \bibinfo{pages}{A29} (\bibinfo{year}{2013}).

\bibitem{Fabian:2006sp}
\bibinfo{author}{{Fabian}, A.~C.}, \bibinfo{author}{{Celotti}, A.} \&
  \bibinfo{author}{{Erlund}, M.~C.}
\newblock {Radiative pressure feedback by a quasar in a galactic bulge}.
\newblock \textit{\bibinfo{journal}{\mnras}} \textbf{\bibinfo{volume}{373}},
  \bibinfo{pages}{L16--L20} (\bibinfo{year}{2006}).

\bibitem{Assef:2015ly}
\bibinfo{author}{{Assef}, R.~J.} \textit{et~al.}
\newblock {Half of the Most Luminous Quasars May Be Obscured: Investigating the
  Nature of WISE-Selected Hot Dust-Obscured Galaxies}.
\newblock \textit{\bibinfo{journal}{\apj}} \textbf{\bibinfo{volume}{804}},
  \bibinfo{pages}{27} (\bibinfo{year}{2015}).

\bibitem{Elitzur:2006ec}
\bibinfo{author}{{Elitzur}, M.} \& \bibinfo{author}{{Shlosman}, I.}
\newblock {The AGN-obscuring Torus: The End of the ``Doughnut'' Paradigm?}
\newblock \textit{\bibinfo{journal}{\apjl}} \textbf{\bibinfo{volume}{648}},
  \bibinfo{pages}{L101--L104} (\bibinfo{year}{2006}).

\bibitem{Kawamuro:2016lq}
\bibinfo{author}{{Kawamuro}, T.}, \bibinfo{author}{{Ueda}, Y.},
  \bibinfo{author}{{Tazaki}, F.}, \bibinfo{author}{{Terashima}, Y.} \&
  \bibinfo{author}{{Mushotzky}, R.}
\newblock {Study of Swift/Bat Selected Low-luminosity Active Galactic Nuclei
  Observed with Suzaku}.
\newblock \textit{\bibinfo{journal}{\apj}} \textbf{\bibinfo{volume}{831}},
  \bibinfo{pages}{37} (\bibinfo{year}{2016}).

\bibitem{Mateos16}
\bibinfo{author}{{Mateos}, S.} \textit{et~al.}
\newblock {X-Ray Absorption, Nuclear Infrared Emission, and Dust Covering
  Factors of AGNs: Testing Unification Schemes}.
\newblock \textit{\bibinfo{journal}{\apj}} \textbf{\bibinfo{volume}{819}},
  \bibinfo{pages}{166} (\bibinfo{year}{2016}).

\bibitem{Elitzur12}
\bibinfo{author}{{Elitzur}, M.}
\newblock {On the Unification of Active Galactic Nuclei}.
\newblock \textit{\bibinfo{journal}{\apjl}} \textbf{\bibinfo{volume}{747}},
  \bibinfo{pages}{L33} (\bibinfo{year}{2012}).

\bibitem{Marin2016}
\bibinfo{author}{{Marin}, F.} \& \bibinfo{author}{{Antonucci}, R.}
\newblock {A Robust Derivation of the Tight Relationship of Radio Core
  Dominance to Inclination Angle in High Redshift 3CRR Sources}.
\newblock \textit{\bibinfo{journal}{\apj}} \textbf{\bibinfo{volume}{830}},
  \bibinfo{pages}{82} (\bibinfo{year}{2016}).

\bibitem{Brightman:2015hb}
\bibinfo{author}{{Brightman}, M.}
\newblock {Determining the torus covering factor in Compton-thick AGN with
  NuSTAR}.
\newblock In \bibinfo{editor}{{Gandhi}, P.} \& \bibinfo{editor}{{Hoenig},
  S.~F.} (eds.) \textit{\bibinfo{booktitle}{TORUS2015}} (\bibinfo{year}{2015}).

\bibitem{Guainazzi:2002mz}
\bibinfo{author}{{Guainazzi}, M.}, \bibinfo{author}{{Matt}, G.},
  \bibinfo{author}{{Fiore}, F.} \& \bibinfo{author}{{Perola}, G.~C.}
\newblock {The Phoenix galaxy: UGC 4203 re-birth from its ashes?}
\newblock \textit{\bibinfo{journal}{\aap}} \textbf{\bibinfo{volume}{388}},
  \bibinfo{pages}{787--792} (\bibinfo{year}{2002}).

\bibitem{Piconcelli:2007bh}
\bibinfo{author}{{Piconcelli}, E.}, \bibinfo{author}{{Bianchi}, S.},
  \bibinfo{author}{{Guainazzi}, M.}, \bibinfo{author}{{Fiore}, F.} \&
  \bibinfo{author}{{Chiaberge}, M.}
\newblock {XMM-Newton broad-band observations of NGC 7582: N$\{$H$\}$
  variations and fading out of the active nucleus}.
\newblock \textit{\bibinfo{journal}{\aap}} \textbf{\bibinfo{volume}{466}},
  \bibinfo{pages}{855--863} (\bibinfo{year}{2007}).

\bibitem{Risaliti:2011jl}
\bibinfo{author}{{Risaliti}, G.} \textit{et~al.}
\newblock {X-ray absorption by broad-line region clouds in Mrk 766}.
\newblock \textit{\bibinfo{journal}{\mnras}} \textbf{\bibinfo{volume}{410}},
  \bibinfo{pages}{1027--1035} (\bibinfo{year}{2011}).

\bibitem{Vazquez15}
\bibinfo{author}{{Vazquez}, B.} \textit{et~al.}
\newblock {Spitzer Space Telescope Measurements of Dust Reverberation Lags in
  the Seyfert 1 Galaxy NGC 6418}.
\newblock \textit{\bibinfo{journal}{\apj}} \textbf{\bibinfo{volume}{801}},
  \bibinfo{pages}{127} (\bibinfo{year}{2015}).

\bibitem{Tristram09}
\bibinfo{author}{{Tristram}, K.~R.~W.} \textit{et~al.}
\newblock {Parsec-scale dust distributions in Seyfert galaxies. Results of the
  MIDI AGN snapshot survey}.
\newblock \textit{\bibinfo{journal}{\aap}} \textbf{\bibinfo{volume}{502}},
  \bibinfo{pages}{67--84} (\bibinfo{year}{2009}).

\bibitem{Wada16}
\bibinfo{author}{{Wada}, K.}, \bibinfo{author}{{Schartmann}, M.} \&
  \bibinfo{author}{{Meijerink}, R.}
\newblock {Multi-phase Nature of a Radiation-driven Fountain with Nuclear
  Starburst in a Low-mass Active Galactic Nucleus}.
\newblock \textit{\bibinfo{journal}{\apjl}} \textbf{\bibinfo{volume}{828}},
  \bibinfo{pages}{L19} (\bibinfo{year}{2016}).

\bibitem{Cameron93}
\bibinfo{author}{{Cameron}, M.} \textit{et~al.}
\newblock {Subarcsecond Mid-Infrared Imaging of Warm Dust in the Narrow-Line
  Region of NGC 1068}.
\newblock \textit{\bibinfo{journal}{\apj}} \textbf{\bibinfo{volume}{419}},
  \bibinfo{pages}{136} (\bibinfo{year}{1993}).

\bibitem{Bock00}
\bibinfo{author}{{Bock}, J.~J.} \textit{et~al.}
\newblock {High Spatial Resolution Imaging of NGC 1068 in the Mid-Infrared}.
\newblock \textit{\bibinfo{journal}{\aj}} \textbf{\bibinfo{volume}{120}},
  \bibinfo{pages}{2904--2919} (\bibinfo{year}{2000}).

\bibitem{Mason06}
\bibinfo{author}{{Mason}, R.~E.} \textit{et~al.}
\newblock {Spatially Resolved Mid-Infrared Spectroscopy of NGC 1068: The Nature
  and Distribution of the Nuclear Material}.
\newblock \textit{\bibinfo{journal}{\apj}} \textbf{\bibinfo{volume}{640}},
  \bibinfo{pages}{612--624} (\bibinfo{year}{2006}).

\bibitem{Asmus16}
\bibinfo{author}{{Asmus}, D.}, \bibinfo{author}{{H{\"o}nig}, S.~F.} \&
  \bibinfo{author}{{Gandhi}, P.}
\newblock {The Subarcsecond Mid-infrared View of Local Active Galactic Nuclei.
  III. Polar Dust Emission}.
\newblock \textit{\bibinfo{journal}{\apj}} \textbf{\bibinfo{volume}{822}},
  \bibinfo{pages}{109} (\bibinfo{year}{2016}).

\bibitem{Young:2001jw}
\bibinfo{author}{{Young}, A.~J.}, \bibinfo{author}{{Wilson}, A.~S.} \&
  \bibinfo{author}{{Shopbell}, P.~L.}
\newblock {A Chandra X-Ray Study of NGC 1068. I. Observations of Extended
  Emission}.
\newblock \textit{\bibinfo{journal}{\apj}} \textbf{\bibinfo{volume}{556}},
  \bibinfo{pages}{6--23} (\bibinfo{year}{2001}).

\bibitem{Bianchi:2006kq}
\bibinfo{author}{{Bianchi}, S.}, \bibinfo{author}{{Guainazzi}, M.} \&
  \bibinfo{author}{{Chiaberge}, M.}
\newblock {The soft X-ray/NLR connection: a single photoionized medium?}
\newblock \textit{\bibinfo{journal}{\aap}} \textbf{\bibinfo{volume}{448}},
  \bibinfo{pages}{499--511} (\bibinfo{year}{2006}).

\bibitem{Gandhi:2009pd}
\bibinfo{author}{{Gandhi}, P.} \textit{et~al.}
\newblock {Resolving the mid-infrared cores of local Seyferts}.
\newblock \textit{\bibinfo{journal}{\aap}} \textbf{\bibinfo{volume}{502}},
  \bibinfo{pages}{457--472} (\bibinfo{year}{2009}).

\bibitem{Ichikawa:2012uo}
\bibinfo{author}{{Ichikawa}, K.} \textit{et~al.}
\newblock {Mid- and Far-infrared Properties of a Complete Sample of Local
  Active Galactic Nuclei}.
\newblock \textit{\bibinfo{journal}{\apj}} \textbf{\bibinfo{volume}{754}},
  \bibinfo{pages}{45} (\bibinfo{year}{2012}).

\bibitem{Singh:2014pd}
\bibinfo{author}{{Singh}, K.~P.} \textit{et~al.}
\newblock {ASTROSAT mission}.
\newblock In \textit{\bibinfo{booktitle}{Space Telescopes and Instrumentation
  2014: Ultraviolet to Gamma Ray}}, vol. \bibinfo{volume}{9144} of
  \textit{\bibinfo{series}{\procspie}}, \bibinfo{pages}{91441S}
  (\bibinfo{year}{2014}).

\bibitem{Odaka:2016fv}
\bibinfo{author}{{Odaka}, H.}, \bibinfo{author}{{Yoneda}, H.},
  \bibinfo{author}{{Takahashi}, T.} \& \bibinfo{author}{{Fabian}, A.}
\newblock {Sensitivity of the Fe K{$\alpha$} Compton shoulder to the geometry
  and variability of the X-ray illumination of cosmic objects}.
\newblock \textit{\bibinfo{journal}{\mnras}} \textbf{\bibinfo{volume}{462}},
  \bibinfo{pages}{2366--2381} (\bibinfo{year}{2016}).

\bibitem{Weisskopf:2016qd}
\bibinfo{author}{{Weisskopf}, M.~C.} \textit{et~al.}
\newblock {The Imaging X-ray Polarimetry Explorer (IXPE)}.
\newblock In \textit{\bibinfo{booktitle}{Society of Photo-Optical Instrumentation
  Engineers (SPIE) Conference Series}}, vol. \bibinfo{volume}{9905} of
  \textit{\bibinfo{series}{\procspie}}, \bibinfo{pages}{990517}
  (\bibinfo{year}{2016}).

\bibitem{Eisenhauer11}
\bibinfo{author}{{Eisenhauer}, F.} \textit{et~al.}
\newblock {GRAVITY: Observing the Universe in Motion}.
\newblock \textit{\bibinfo{journal}{The Messenger}}
  \textbf{\bibinfo{volume}{143}}, \bibinfo{pages}{16--24}
  (\bibinfo{year}{2011}).

\bibitem{Lopez14b}
\bibinfo{author}{{Lopez}, B.} \textit{et~al.}
\newblock {An Overview of the MATISSE Instrument - Science, Concept and Current
  Status}.
\newblock \textit{\bibinfo{journal}{The Messenger}}
  \textbf{\bibinfo{volume}{157}}, \bibinfo{pages}{5--12}
  (\bibinfo{year}{2014}).

\bibitem{Gardner2006}
\bibinfo{author}{Gardner, J.~P.} \textit{et~al.}
\newblock The James Webb Space Telescope.
\newblock \textit{\bibinfo{journal}{Space Sci. Rev.}}
  \textbf{\bibinfo{volume}{123}}, \bibinfo{pages}{485--606}
  (\bibinfo{year}{2006}).

\end{thebibliography}




%
%
%

\end{document}
%
