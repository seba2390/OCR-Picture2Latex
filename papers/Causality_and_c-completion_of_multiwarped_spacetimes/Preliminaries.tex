\section{The causal structure of doubly warped spacetimes}
\label{sec:chronologicalrelation}
%By a {\em \multiwarped spacetime} we understand a lorentzian manifold $(V,g)$ where
%\begin{equation}
%  \label{eq:1}
%  V:= \R \times M_{1} \times M_{2}\quad\hbox{and}\quad
%g=-dt^{2}+\alpha_{1}g_{1}+\alpha_{2}g_{2},
%\end{equation}
%being $(M_{i},g_{i})$ a Riemannian manifold for $i=1,2$.
%
%\smallskip

In this section we are going to characterize the chronological and causal relations in doubly warped spacetimes. First, recall that a {\em \multiwarped spacetime} is a multiwarped spacetime $(V,g)$ as in (\ref{eqqq}) with two fibers ($n=2$), that is,

\begin{equation}
  \label{eq:1-aux}
  V:= (a,b)\times M_{1} \times M_{2}\quad\hbox{and}\quad
g=-dt^{2}+\alpha_{1}g_{1}+\alpha_{2}g_{2}.
\end{equation}
%where $\alpha_i:\R\rightarrow (0,\infty)$ are positive smooth functions and $(M_{i},g_{i})$ are Riemannian manifolds for $i=1,2$.
%\begin{rem}\label{rem:infinito}
%Since the structures studied in this paper are invariant by conformal transformations, our analysis implicitly covers the case where the temporal interval $(a,b)$ is a proper subset of $\R$. In fact, otherwise, multiply the metric by an appropriate positive function in order to obtain another doubly warped spacetime in the same conformal class with $(a,b)=\R$.
%\end{rem}




%\begin{rem}\label{rem:infinito}
%  As we are going to be interested in causal structures, there is no loss of generality if we assume that $(a,b)\equiv \R$. In fact, given a metric as in \eqref{eq:1-aux}, let us consider $\beta:(a,b)\rightarrow (0,\infty)$ a function satisfying that:
%  \[
%\int_{c}^b \frac{1}{\sqrt{\beta(t)}}dt=\infty=\int_{a}^c\frac{1}{\sqrt{\beta(t)}}dt
%    \]
%    for some $c\in (a,b)$. Then,
%    \[
%      \begin{array}{rl}
%        {\mathfrak g}=& -dt^{2}+\alpha_{1}g_{1}+\alpha_{2}g_{2}\\ =& \beta \left(-\left(\frac{dt}{\sqrt{\beta}}\right)^2+ \frac{\alpha_1}{\beta}g_1+\frac{\alpha_2}{\beta}g_2\right)\\ = & \beta \left(-ds^2+\tilde{\alpha}_1g_1+\tilde{\alpha}_2g_2   \right) \\ =&\beta\, g,
%      \end{array}
%      \]
%      where $s\in \R$ as, from definition, $ds=dt/\sqrt{\beta}$. Moreover,
%
%      \[
%\int_{0}^{\infty} \frac{1}{\sqrt{\tilde{\alpha}_i(s)}}ds=\int_{c}^{a} \frac{1}{\sqrt{\alpha_i(t)}}dt, \qquad \int_{0}^{-\infty} \frac{1}{\sqrt{\tilde{\alpha}_i(s)}}ds=\int_{c}^{b} \frac{1}{\sqrt{\alpha_i(t)}}dt, \quad\hbox{ for $i=1,2$}.
%        \]
%        In conclusion, the metric $\mathfrak{g}$ is conformal to $g$, which is also a {\multiwarped} model where the time component belongs to the entire $\R$ and with warping functions preserving some integral condition. Therefore, unless stated otherwise (essentially, on the main results), we will always work considering {\multiwarped} models as follow:
%
%        \begin{equation}
%  \label{eq:1}
%  V:= \R \times M_{1} \times M_{2}\quad\hbox{and}\quad
%g=-dt^{2}+\alpha_{1}g_{1}+\alpha_{2}g_{2},
%\end{equation}
%
%\end{rem}
%}


Take $(t^e,x^e)\in V$ and $x^o\in M:=M_1\times M_2$. Denote by $C(x^o,x^e)$ the set of smooth curves in $M$ connecting $x^o$ with $x^e$. Given $c=(c_1,c_2)\in C(x^o,x^e)$, consider the unique future-directed lightlike curve $\rho:[s^o,s^e]\rightarrow V$ with $\rho(s)=(\tau_{c,t^e}(s),c(s))$ and $\tau_{c,t^e}(s^e)=t^e$. From the metric expression in (\ref{eq:1-aux}), the component $\tau_{c,t^e}(s)$ is determined by the Cauchy problem
\[
-\dot{\tau}_{c,t^e}^2+\alpha_1(\tau_{c,t^e})g_{1}(\dot{c}_1,\dot{c}_1)+\alpha_2(\tau_{c,t^e})g_{2}(\dot{c}_2,\dot{c}_2)=0,\qquad
    \tau_{c,t^e}(s^e)=t^e.
  \]
Consider the functional
\[\J_{x^o,(t^e,x^e)}: C(x^o,x^e) \rightarrow (a,b), \quad c \mapsto \tau_{c,t^e}(s^o).\]
A direct computation shows that $(t^o,x^o)\ll (t^e,x^e)$ if, and only if, there exists $c\in C(x^o,x^e)$ such that $t_o<\J_{x^o,(t^e,x^e)}(c)$. This property suggests the following definition for the {\em departure time function}:
\[
T:M\times \left((a,b)\times M\right)\rightarrow (a,b),\qquad T(x^{o},(t^{e},x^{e})):= {Sup}_{C}\J_{x^o,(t^e,x^e)}
\]
(compare with \cite[Section 2.9]{Perlick2004} and \cite[Section 4]{FS2}). By construction, this function characterizes the chronological relation in $(V,g)$, as follows:
\begin{equation}\label{e0}
(t^{o},x^{o}) \ll (t^{e},x^{e}) \;\; \Longleftrightarrow \;\;
t^{o}<T(x^{o},(t^{e},x^{e})).
\end{equation}
In particular, the chronological past of a given point $(t^e,x^e)$ is given by
\[
I^-\left((t^e,x^e) \right):=\{(t,x)\in (a,b)\times M: t<T(x,(t^e,x^e)) \}.
  \]
Given a future-directed timelike curve $\gamma(t)=(t,c(t))$, $t\in [\omega,\Omega)$, and a point $x\in M$, the transitivity of the chronological relaction $\ll$ ensures that the function $T(x,\gamma(t))$ is increasing on $t$. Hence, the chronological past of $\gamma$ can be written as
  \[
I^-(\gamma)=\{(s,x)\in (a,b)\times M: s<b_c(x):=lim_{t\rightarrow b}T(x,\gamma(t))\}.
    \]
%    \begin{rem}
%      The choice of $b_c$ to denote such a limit came motivated from the studies of the causal boundary for standard static  spacetimes, that can be considered as... In fact, as it can be deduced from CITA REQUERIDA, $T(x,(t^*,y))=t^*-d(x,y)$, and so, when...\footnote{Simplemente concretar lo anterior al caso estático estandar, a modo de ejemplo.}.
%    \end{rem}

Next, let us characterize the departure time function, and so, the chronological relations (recall (\ref{e0})), in terms of some integral conditions involving the warping functions $\alpha_i$ and the Riemannian distances $d_i$ associated to the fibers $(M_i,g_i)$, $i=1,2$.
To this aim, let us consider a future-directed causal curve $\gamma:  I \rightarrow V$,
$\gamma(s)=(t(s),c_{1}(s),c_{2}(s))$. From the metric expression in \eqref{eq:1-aux}:
\[
\frac{dt}{ds}(s)=\sqrt{-D+\frac{\mu_{1}}{\alpha_{1}\circ
t}+\frac{\mu_{2}}{\alpha_{2}\circ t}}(s),
\]
where $D:=g(d\gamma/ds,d\gamma/ds)\leq 0$ and
$\mu_{i}:=(\alpha_{i}\circ t)^2 g_{i}(dc_{i}/ds,dc_{i}/ds)$,
$i=1,2$. From the Inverse Function Theorem, previous formula translates into
\[
\frac{ds}{dt}(t)=\left(-(D\circ s)+\frac{\mu_{1}\circ s}{\alpha
_{1}}+\frac{\mu_{2}\circ s}{\alpha _{2}}\right)^{-1/2}.
\]
Therefore, if we denote $t^{o}=t(s^{o})$, $t^{e}=t(s^{e})$, we
deduce
\begin{equation}\label{eq:3}
\begin{array}{c}
\hbox{length}\left(c_{i}\mid_{[s^{o},s^{e}]}\right)=\int_{s^{o}}^{s^{e}}\sqrt{g_{i}(\dot
c_{i}, \dot c_{i})} ds=\int_{t^o}^{t^e}\sqrt{g_{i}(\dot c_{i}, \dot
c_{i})} \frac{ds}{dt} dt \qquad\qquad\qquad\quad \\
\;\quad\qquad\qquad\qquad =\int_{t^o}^{t^e}\frac{\sqrt{\mu_{i}\circ
s}}{\alpha_i(t)}\left(-(D\circ s)+\frac{\mu_{1}\circ
s}{\alpha_{1}(t)}+\frac{\mu_{2}\circ s}{\alpha_{2}(t)}\right)^{-1/2}dt
\qquad\hbox{for}\;\; i=1,2.
\end{array}
\end{equation}
In the particular case of being $\gamma$ a lightlike geodesic we have: (i) $D=0$ (lightlike character of $\gamma)$, (ii) $\mu_i\circ s$ are constants and (iii) $c_i$ are (pre-)geodesics on the corresponding Riemannian manifold $(M_i,g_i)$ (geodesic character of $\gamma$). So, from (\ref{eq:3}), one deduces (see \cite[Theorem 2]{FS} for details):
\begin{prop}\label{thm:characluzgeodesics}
  Let $(V,g)$ be a {\multiwarped} spacetime as in (\ref{eq:1-aux}) with (weakly) convex fibers (i.e., satisfying that any pair of points can be joined by some minimizing geodesic). Consider two distinct points $(t^o,x_1^o,x_2^o),(t^e,x_1^e,x_2^e)\in V$ with $t^o<t^e$. Then, the following statements are equivalent:
  \begin{itemize}
  \item[(a)] There exists a lightlike geodesic joining $(t^o,x_1^o,x_2^o)$ and $(t^e,x_1^e,x_2^e)$.
  \item[(b)] There exist $\mu_1,\mu_2 \ncambios{\geq}0$ with $\mu_1+\mu_2=1$ such that
    \[
\Integral{t^o}{t^{e}}{\mu_{i}}{i}{\mu_{k}}=d_{i}(x^{o}_{i},x^{e}_{i})\qquad\hbox{for}\;\;
i=1,2;
      \]

  \end{itemize}

\end{prop}


%This equality will be important in the forthcoming result, since it allows to explicitly compute the length of the projection of a causal curve over any of the Riemannian fibers.
%    \begin{lemma}
%\label{lightlikecurve}
%Let $(t^{o},x_1^o,x_2^o)$, $(t^e,x_1^{e},x_2^e)$  be two points of a \multiwarped spacetime $(V,g)$ as in \eqref{eq:1} such that $(x_1^o,x_2^o) \neq (x_1^{e},x_2^e)$. Assume the existence of positive values $\mu_{1},\mu_{2} \in\R$, with $\mu_{1}+\mu_{2}=1$, such that
%\begin{equation}\label{eq:2}
%\Integral{t^o}{t^e}{\mu_{i}}{i}{\mu_{k}}>d_{i}(x_{i}^o,x_{i}^{e}) \qquad\hbox{for}\;\;
%i=1,2.
%\end{equation}
%Then $(t^{o},x_1^o,x_2^o) \ll (t^{e},x_1^e,x_2^e)$, i.e., there exists a timelike future-directed curve between $(t^{o},x_1^o,x_2^o)$ and $(t^{e},x_1^e,x_2^e)$.
%\end{lemma}
%
%\begin{proof} For any $\epsilon>0$, let us denote
%  \[
%L^\epsilon_i:=\Integral{t^o+\epsilon}{t^e}{\mu_{i}}{i}{\mu_{k}}.
%    \]
% Take $\epsilon>0$ small enough so that the inequalities in \eqref{eq:2} still hold for $t_0+\epsilon$ instead of $t_0$. There exist curves $y_i:[s^0,s^e]\rightarrow M_i$, with $y_{i}(s^{o})=x_{i}^{o}$ and $y_{i}(s^{e})=x^{e}_{i}$, such that $length(y_{i})=L^\epsilon_{i}$, $i=1,2$. Next, consider the lightlike curve
%$\rho(s)=(\tau(s),\overline{y}_{1}(s),\overline{y}_{2}(s))$,
%with $\overline{y}_{i}$ reparametrizations of $y_{i}$,
%constructed by
%requiring
%\[
%\left\{\begin{array}{l}\dot{\tau}=\sqrt{\sum_{i=1}^{2}\frac{\mu_{i}}{\alpha_{i}\circ\tau}}
%\\ \tau(s^{e})=t^{e}
%\end{array}\right.,\qquad
%\left\{\begin{array}{l}g_{i}(\dot{\overline{y}}_{i},\dot{\overline{y}}_{i})=\frac{\mu_{i}}{(\alpha_{i}\circ
%\tau)^{2}} \\
%\overline{y}_{i}(s^{e})=x^{e}_{i}\end{array}\right.
%\qquad\hbox{for}\;\; i=1,2.
%\]
%Then, by applying \eqref{eq:3} to the lightlike curve $\rho$ (thus, $D=0$), we deduce
%\[
%\hbox{length}(\overline{y}_{i}\mid_{[\tau^{-1}(t^o+\epsilon),s^{e}]})=\Integral{t^o+\epsilon}{t^{e}}{\mu_{i}}{i}{\mu_{k}}
%%\int^{t^{e}}_{T'_{m}}\sqrt{\mu_{i}}\alpha_{i}^{-1}\left(\sum_{j=1}^{n}\frac{\mu_{j}}{\alpha_{j}}\right)^{-1/2}dt
%=L^\epsilon_{i}=\hbox{length}(y_{i}).
%\]
%Therefore, $\rho(s)$ is a lightlike curve joining $(t^o+\epsilon,x^{o})$ with
%$(t^{e},x^{e})$, and so, these points are causally related. Since $(t^0+\epsilon,x^0)\ll (t^0,x^0)$, it directly follows that $(t^0,x^0)\ll (t^{e},x^{e})$.
%\end{proof}
%
%In order to complete the characterization of the chronological relation, we will use the following characterization of the departure time function:
%
%%Assume just for a second that we have already obtained such an equivalence. Given $x^0,x^e\in M$ and $t^e$, the departure time function $T(x^0,(t^e,x^e))$ determines the exact point where the line $(t,x^0)$ with $t\in \R$ leaves the past of $(t^e,x^e)$. In particular, for all $t^0<T(x^0,(t^e,x^e))$ we can obtain $\mu_1,\mu_2$ so the inequalities on \eqref{eq:2} are satisfied. Therefore, at least intuitively, the departure time function should be in the edge of satisfying such inequalities, being expected that it satisfies a relation as in \eqref{eq:2} (substituing $t^0$ with $T(x^0,(t^e,x^e))$) but with equalities. The next result is a formalization of this intuitive idea:
%
%\begin{prop}
%\label{p0}
%Let $(V,g)$ be a \multiwarped spacetime and $(x^{o},(t^{e},x^{e}))\in M \times V$, with $x^{o}=(x_1^o,x_2^o)\neq (x_1^e,x_2^e)=x^{e}$. If $T=T(x^{o},(t^{e},x^{e}))>-\infty$ then $\varsigma=T$ is the unique real value satisfying
%\begin{equation}
%\label{e*}
%\Integral{\varsigma}{t^{e}}{\mu_{i}}{i}{\mu_{k}}=d_{i}(x^{o}_{i},x^{e}_{i}),\qquad
%i=1,2,
%\end{equation}
%for some (unique) constants $\mu_{1},\mu_{2} \geq 0$ with
%$\mu_{1}+\mu_{2}=1$.
%\end{prop}
%
%\begin{proof}
% First, let us show that if $\varsigma=T'$
%satisfies (\ref{e*}) then necessarily $T'\leq T$. To this aim,
%take any sequence $\epsilon_{m} \searrow 0$ and define for $i=1,2$
%\[
%L_{i,m}:=\Integral{T'_{m}}{t^{e}}{\mu_{i}}{i}{\mu_{k}}
%\qquad
%\hbox{with}\;\; T'_{m}:=T'-\epsilon_{m}.
%\]
%From the choice for $T'_m$, we have that $L_{i,m}>d_{i}(x_{i}^o,x_{i}^{e})$ for all $i$. So, Lemma \ref{lightlikecurve} ensures that
%$(T'_{m},x^o) \ll (t^e,x^e)$ for all $m$. Therefore, from the definition of $T(x^o,(t^e,x^e))$, $T'-\epsilon_{m}<T$ for all $m$, and then, $T' \leq T$.
%
%\medskip
%%From the hypothesis for $T'$, there exist curves $y_{i,m}(s)$ in
%%$M_{i}$ with lengths $L_{i,m}$ such that
%%$y_{i,m}(s^{o})=x_{i}^{o}$ and $y_{i,m}(s^{e})=x^{e}_{i}$.
%%Consider the lightlike curves
%%$\rho_{m}(s)=(\tau(s),\overline{y}_{1,m}(s),..., \overline{y}_{n,m}(s))$,
%%with $\overline{y}_{i,m}$ reparametrizations of $y_{i,m}$,
%%constructed by
%%requiring
%%\[
%%\left\{\begin{array}{l}\dot{\tau}=\sqrt{\sum_{i=1}^{2}\frac{\mu_{i}}{\alpha_{i}\circ\tau}}
%%\\ \tau(s^{e})=t^{e}
%%\end{array}\right.,\qquad
%%\left\{\begin{array}{l}g_{i}(\dot{\overline{y}}_{i,m},\dot{\overline{y}}_{i,m})=\frac{\mu_{i}}{(\alpha_{i}\circ
%%\tau)^{2}} \\
%%\overline{y}_{i,m}(s^{e})=x^{e}_{i}\end{array}\right.
%%\qquad\hbox{for}\;\; i=1,2.
%%\]
%%Then,
%%\[
%%\hbox{length}(\overline{y}_{i,m}\mid_{[\tau^{-1}(T'_{m}),s^{e}]})=\Integral{T'_{m}}{t^{e}}{\mu_{i}}{i}{\mu_{k}}
%%%\int^{t^{e}}_{T'_{m}}\sqrt{\mu_{i}}\alpha_{i}^{-1}\left(\sum_{j=1}^{n}\frac{\mu_{j}}{\alpha_{j}}\right)^{-1/2}dt
%%=L_{i,m}=\hbox{length}(y_{i,m}).
%%\]
%%Therefore, $\rho_{m}(s)$ joins $(T'_{m},x^{o})$ with
%%$(t^{e},x^{e})$, and so, the points $(T'_{m-1},x^{o})$,
%%$(t^{e},x^{e})$ are chronologically related for all $m$. According
%%to (\ref{e0}), this implies $T'<T+\epsilon_{m-1}$ for all $m$, and
%%thus, $T'\leq T$.
%
%Next, let us prove that some value $\varsigma=T'$
%verifying (\ref{e*}) always exists, and necessarily $T'\geq T$.
%From (\ref{e0}), $(t^{o},x^{o})\ll (t^{e},x^{e})$ for any
%$-\infty<t^{o}<T(x^{o},(t^{e},x^{e}))$. Let $\gamma:[t^0,t^e]\rightarrow V$,
%$\gamma(t)=(t,c_{1}(t), c_{2}(t))$, be a timelike curve such
%that $\gamma(t^{o})=(t^{o},x^{o})$ and $\gamma(t^{e})=(t^{e},x^{e})$.
%Consider real curves $\overline{c}_{i}$, $i=1,2$, such that
%\[
%\left\{\begin{array}{l} 0\leq\dot{\overline{c}}_{i}(t)\leq
%\sqrt{g_{i}(\dot{c}_{i}(s),\dot{c}_{i}(s))} \\
%\overline{c}_{i}(t^{o})=0 \\
%\overline{c}_{i}(t^{e})=d_{i}(x_{i}^{o},x_{i}^{e})
%\end{array}\right. \qquad\hbox{for}\;\; i=1,2.
%\]
%Then, $\overline{\gamma}(s)=(t,\overline{c}_{1}(t),\overline{c}_{2}(t))$
%becomes a future directed timelike curve in the globally hyperbolic \multiwarped
%spacetime
%$V'=(\R \times \R^{2},-dt^{2}+\alpha_{1}dx_{1}^{2}+\alpha_{2}dx_{2}^{2})$ joining $\overline{\gamma}(t^o)=\point{t^o}{0}{0}$ with $\overline{\gamma}(t^e)=\point{t^e}{d_{1}(x_{1}^o,x_{1}^{e})}{d_{2}(x_{2}^o,x_{2}^e)}$, i.e., \[\overline{\gamma}(t^o)=(t^o,0,0)\ll (t^e,d_{1}(x_{1}^o,x_{1}^{e}),d_{2}(x_{2}^o,x_{2}^{e})=\overline{\gamma}(t^e).\] Now consider $T'$ such that $(T',0,0)\leq \overline{\gamma}(t^e)$ but $(T',0,0)\not\ll \overline{\gamma}(t^e)$. From
%Avez and Seifert's result, there exists some lightlike geodesic in $\R\times \R^2$
%joining both points. Now, \cite[Theorem 2]{FS} (see also \cite[Lemma 3]{FS}) applied to the lightlike geodesic in $V'$ ensures that there exist unique positive constants $\mu_1,\mu_2$ with $\mu_1+\mu_2=1$ and such that
%
%\[
%\Integral{T'}{t^{e}}{\mu_{i}}{i}{\mu_{k}}=|d_{i}(x^{o}_{i},x^{e}_{i})-0|=d_{i}(x^{o}_{i},x^{e}_{i}) \qquad\hbox{for}\;\;
%i=1,2.
%  \]
%Finally, observe that $t^o<T'$ for all $t^o<T$, so in particular $T\leq T'$, which concludes the proof.
%
%% . \cambios{Denote by $\lambda(s)=(\tau(s),c_{1}(s),c_{2}(s))$ ($s \in [0,1]$) the
%% unique lightlike geodesic, and recall that a geodesic in a \multiwarped satisfies the following: (a) each $\mu_{i}=\alpha_{i}(\tau(s))^{2}g_{i}(c_{i}',c_{i}')$ is constant and (b) $c_{i}(s)$ is a pregeodesic in $(\mathbb{R},+dx_{i}^{2})$ for all $i$, see \cite{FS}[Eqns. (5), (6)]. Applying a change of variables in the following integral equalities by using the fact that $\frac{d\tau}{ds}=\sqrt{\sum_{i=1}^{2} \frac{\mu_{i}}{\alpha_{i} \circ \tau}}$ leads to:
%
%% \begin{equation}
%% \label{e7}
%% \hbox{length}(c_{i})=\int_{0}^{1}\sqrt{g_{i}(c_{i}',c_{i}')}dr=\Integral{T'}{t^{e}}{\mu_{i}}{i}{\mu_{k}}=d_{i}(x^{o}_{i},x^{e}_{i})\qquad\hbox{for
%% all}\;\; i
%% %\int^{t^{e}}_{T'}\sqrt{\mu_{i}}\alpha_{i}^{-1}\left(\sum_{j=1}^{n}\frac{\mu_{j}}{\alpha_{j}}\right)^{-1/2}dt=d_{i}(x^{o}_{i},x^{e}_{i})\qquad\hbox{for
%% %all}\;\; i
%% \end{equation}
%% }
%% \cambios{Note that $\mu_{1}$ and $\mu_{2}$ are unique for $T'$, in fact, if $\mu_{1}'$ and $\mu_{2}'$ are different to $\mu_{1}$ and $\mu_{2}$ then we have two possibilities: (1) $\mu_{1}'<\mu_{1}$ and $\mu_{2}<\mu_{2}'$ or (2) $\mu_{1}<\mu_{1}'$ and $\mu_{2}'<\mu_{2}$. Any of the previous cases imply that for some $i_{0} \in \{1,2\}$ the following strict inequality is obtained:
%% \[
%% \Integral{T'}{t^e}{\mu_{i_{0}}'}{i_{0}}{\mu_{k}'}>\Integral{T'}{t^e}{\mu_{i_{0}}}{i_{0}}{\mu_{k}}=d_{i_{0}}(x_{i_0}^o,x_{i_0}^e).
%% \]
%% So, $\mu_{1}$ and $\mu_{2}$ are the unique constants satisfying equation (\ref{e7}).
%% }
%%\footnote{For the uniqueness, and the forthcoming
%%property $(t^{o},x^{o})\not\ll (t^{e},x^{e})$, just apply [Subl.
%%3.4.2, PhD]: if $\mu_{i},\overline{\mu}_{i}\geq 0$,
%%$\mu_{1}+\cdots +\mu_{n}=1=\overline{\mu}_{1}+\cdots
%%+\overline{\mu}_{n}$ and $\overline{\mu}_{n}<\mu_{n}$ then there
%%exists some $i_{0}\in \{1,\ldots,n-1\}$ such that
%%$\overline{\mu}_{i_{0}}>\mu_{i_{0}}$ (and thus,
%%$\overline{\mu}_{n}/\overline{\mu}_{i_{0}}<\mu_{n}/\mu_{i_{0}}$)
%%and $\overline{\mu}_{j}/\overline{\mu}_{i_{0}}\leq
%%\mu_{j}/\mu_{i_{0}}$ $\forall j\neq n$.}
%% \cambios{Moreover, necessarily $T'\geq T$. In fact, if $T'<T$ then for any ${t^o}' \in (T',T)$ it cannot happen that
%% $({t^o}',x^{o}) \ll (t^{e},x^{e})$ since the same process to construct $T'$ will imply the existence of  $T''\in ({t^{o}}',T')$ such that $(T'',0)$ is causally but no
%% timelike related to $(t^{e},d_{1}(x_{1}^o,x_{1}),d_{2}(x_{2}^o,x_{2}^{e}))$ in the globally hyperbolic \multiwarped spacetime $(\R \times \R^{2},\hat{g})$, this will contradict the achronality of $\partial I_{\hat{g}}^{+}((t^{e},d_{1}(x_{1}^o,x_{1}),d_{2}(x_{2}^o,x_{2}^{e})))$ because $(T',0) \ll (T'',0)$ and both points live in the boundary. Therefore, for any ${t^{o}}' \in (T',T)$ we have that $({t^o}',x^{o}) \not \ll (t^{e},x^{e})$ and this is a contradiction to the condition over $T$ given in (\ref{e0}). Therefore, $T' \geq T$. The first part of the proof proves that $T \geq T'$ for any $T'$ satisfying equation (\ref{e*}), therefore $T$ is the only point satisfying equation (\ref{e*}) for unique constants $(\mu_{1},\mu_{2})$ with $\mu_{1}+\mu_{2}=1$. }
%\end{proof}
%
%\begin{rem}\label{rem:3}
% It is worth mentioning how in previous proof we have moved from the \multiwarped model $V$ to the globally hyperbolic \multiwarped model $V'$. This trick allow us to work in globally hyperbolic models which are complete, and so, where $\overline{I^\pm(p)}=J^\pm(p)$, being possible to obtain the integral condition more easily.\footnote{Jony: Este remark lo puse pensando en que este truco iba a volver a usarse más adelante, pero finalmente (al menos para el borde causal) no ha hecho falta. Mirar si vale la pena dejarlo.}
%\end{rem}
We are now in conditions to establish the characterization of the chronological relation.
\begin{prop}\label{c0}
Let $(V,g)$ be a \multiwarped spacetime as in (\ref{eq:1-aux}), and $(t^{o},x^{o}), (t^{e},x^{e})\in V$ with $x^{o}\neq
x^{e}$. The following conditions are equivalent:
\begin{itemize}

\item[(i)]  $(t^{o},x^{o})\ll (t^{e},x^{e})$; or, equivalently, $t^o<T(x^o,(t^e,x^e))$ (recall (\ref{e0}));
\item[(ii)] $T(x^o,(t^e,x^e))$ is the unique real value $T\in (a,b)$
with $t^{o}<T<t^{e}$ such that, for some (unique) positive constants $\mu_{1},\mu_2 \ncambios{\geq}
0$, with $\mu_{1}+\mu_{2}=1$, it satisfies
\begin{equation}\label{ee2}
\Integral{T}{t^{e}}{\mu_{i}}{i}{\mu_{k}}=d_{i}(x^{o}_{i},x^{e}_{i})\qquad\hbox{for}\;\;
i=1,2;
%\int_{T}^{t^{e}}\sqrt{\mu_{i}}\alpha_{i}^{-1}\left(\sum_{j=1}^{n}\frac{\mu_{j}}{\alpha_{j}}\right)^{-1/2}dt=d_{i}(x^{o}_{i},x^{e}_{i})\qquad\hbox{for}\;\;
%i=1,...,n;
\end{equation}

\item[(iii)] there exist strictly positive constants $\mu'_{1},\mu'_{2}> 0$, with $\mu'_1+\mu'_2=1$,
%(with
%$\mu'_{1}+\cdots +\mu'_{n}=1$)
such that
\begin{equation}\label{ee2''}
\Integral{t^{o}}{t^{e}}{\mu_{i}'}{i}{\mu_{k}'}>
d_{i}(x^{o}_{i},x^{e}_{i})\qquad\hbox{for $i=1,2$}.
%\int_{t^{o}}^{t^{e}}\sqrt{\mu'_{i}}\alpha_{i}^{-1}\left(\sum_{j=1}^{n}\frac{\mu'_{j}}{\alpha_{j}}\right)^{-1/2}dt\geq
%d_{i}(x^{o}_{i},x^{e}_{i})\qquad\hbox{for $i=1,...,n$},
\end{equation}
%with equality in the $i$-th inequality if and only if
%$\mu'_{i}=0$.
\end{itemize}
\end{prop}
\begin{proof}
%\footnote{He hecho cambios en esta prueba. Hay que chequear que todo es correcta.}
The implication $(ii)\Rightarrow (iii)$ is trivial unless some $\mu_i$ is equal to $0$. So, assume for instance that $\mu_1=0$ (and so, $\mu_2=1$). Then, \eqref{ee2} becomes
\[
\left\{
  \begin{array}{l}
    0=d_1(x_1^o,x_1^e)\\
    \\
    \displaystyle \int_{T}^{t^e}\frac{1}{\sqrt{\alpha_2(s)}}ds=d_2(x_2^o,x_2^e).
  \end{array}
\right.
  \]
  By continuity, we can modify slightly $\mu_1$, $\mu_2$, to obtain strictly positive $\mu'_1,\mu'_2$, with $\mu'_1+\mu'_2=1$, such that
 \[
    \left\{
      \begin{array}{l}\displaystyle\Integral{t^{o}}{t^{e}}{\mu_{1}'}{1}{\mu_{k}'}>0= d_1(x_1^o,x_1^e)\\
      \\
      \displaystyle\Integral{t^{o}}{t^{e}}{\mu_{2}'}{2}{\mu_{k}'}> d_2(x_2^o,x_2^e),
      \end{array}\right.
    \]
as desired.


For the implication $(iii) \Rightarrow (i)$, denote
  \[
L^\epsilon_i:=\Integral{t^o+\epsilon}{t^e}{\mu_{i}'}{i}{\mu_{k}'},\quad\hbox{for $\epsilon>0$.}
    \]
 Take $\epsilon>0$ small enough so that $t^o+\epsilon<t^e$ and the inequalities in \eqref{ee2''} still hold for $t^o+\epsilon$ instead of $t^o$. Since $L_i^{\epsilon}>d_i(x_i^o,x_i^e)$, there exist curves $y_i:[s^o,s^e]\rightarrow M_i$, with $y_{i}(s^{o})=x_{i}^{o}$ and $y_{i}(s^{e})=x^{e}_{i}$, such that $length(y_{i})=L^\epsilon_{i}$, $i=1,2$. Consider the lightlike curve
$\rho(s)=(\tau(s),\overline{y}_{1}(s),\overline{y}_{2}(s))$,
with $\overline{y}_{i}$ reparametrizations of $y_{i}$,
constructed by
requiring
\[
\left\{\begin{array}{l}\dot{\tau}=\sqrt{\sum_{i=1}^{2}\frac{\mu_{i}'}{\alpha_{i}\circ\tau}}
\\ \tau(s^{e})=t^{e}
\end{array}\right.,\qquad
\left\{\begin{array}{l}g_{i}(\dot{\overline{y}}_{i},\dot{\overline{y}}_{i})=\frac{\mu_{i}'}{(\alpha_{i}\circ
\tau)^{2}} \\
\overline{y}_{i}(s^{e})=x^{e}_{i}\end{array}\right.
\qquad\hbox{for}\;\; i=1,2.
\]
Then, by applying \eqref{eq:3} to the lightlike curve $\rho$ (in particular, $D=0$), we deduce
\[
\hbox{length}(\overline{y}_{i}\mid_{[\tau^{-1}(t^o+\epsilon),s^{e}]})=\Integral{t^o+\epsilon}{t^{e}}{\mu_{i}'}{i}{\mu_{k}'}
%\int^{t^{e}}_{T'_{m}}\sqrt{\mu_{i}}\alpha_{i}^{-1}\left(\sum_{j=1}^{n}\frac{\mu_{j}}{\alpha_{j}}\right)^{-1/2}dt
=L^\epsilon_{i}=\hbox{length}(y_{i}).
\]
Therefore, $\rho(s)$ is a lightlike curve joining $(t^o+\epsilon,x^{o})$ with
$(t^{e},x^{e})$, and so, these points are causally related. Since $(t^o,x^o)\ll (t^o+\epsilon,x^o)$, necessarily $(t^o,x^o)\ll (t^{e},x^{e})$.

Finally, for the implication $(i) \Rightarrow (ii)$, let us show first that if $T$
satisfies (\ref{ee2}) then $T\leq T(x^o,(t^e,x^e))$. So, assume that (\ref{ee2}) holds. %Up to a small modification of $\mu_i$, $i=1,2$, we can suppose that both coefficients are strictly positive.
Take any sequence $\epsilon_{m} \searrow 0$ and define
\[
L_{i,m}:=\Integral{T-\epsilon_m}{t^{e}}{\mu_{i}}{i}{\mu_{k}}\quad\hbox{i=1,2.}
\]
We have that $L_{i,m}>d_{i}(x_{i}^o,x_{i}^{e})$ for all $i$. The implication (iii)$\Rightarrow$(i), which has been proved before, ensures that
$(T-\epsilon_{m},x^o) \ll (t^e,x^e)$ for all $m$. Therefore, from the definition of $T(x^o,(t^e,x^e))$, $T-\epsilon_{m}<T(x^o,(t^e,x^e))$ for all $m$, and then, $T \leq T(x^o,(t^e,x^e))$.

%From the hypothesis for $T'$, there exist curves $y_{i,m}(s)$ in
%$M_{i}$ with lengths $L_{i,m}$ such that
%$y_{i,m}(s^{o})=x_{i}^{o}$ and $y_{i,m}(s^{e})=x^{e}_{i}$.
%Consider the lightlike curves
%$\rho_{m}(s)=(\tau(s),\overline{y}_{1,m}(s),..., \overline{y}_{n,m}(s))$,
%with $\overline{y}_{i,m}$ reparametrizations of $y_{i,m}$,
%constructed by
%requiring
%\[
%\left\{\begin{array}{l}\dot{\tau}=\sqrt{\sum_{i=1}^{2}\frac{\mu_{i}}{\alpha_{i}\circ\tau}}
%\\ \tau(s^{e})=t^{e}
%\end{array}\right.,\qquad
%\left\{\begin{array}{l}g_{i}(\dot{\overline{y}}_{i,m},\dot{\overline{y}}_{i,m})=\frac{\mu_{i}}{(\alpha_{i}\circ
%\tau)^{2}} \\
%\overline{y}_{i,m}(s^{e})=x^{e}_{i}\end{array}\right.
%\qquad\hbox{for}\;\; i=1,2.
%\]
%Then,
%\[
%\hbox{length}(\overline{y}_{i,m}\mid_{[\tau^{-1}(T'_{m}),s^{e}]})=\Integral{T'_{m}}{t^{e}}{\mu_{i}}{i}{\mu_{k}}
%%\int^{t^{e}}_{T'_{m}}\sqrt{\mu_{i}}\alpha_{i}^{-1}\left(\sum_{j=1}^{n}\frac{\mu_{j}}{\alpha_{j}}\right)^{-1/2}dt
%=L_{i,m}=\hbox{length}(y_{i,m}).
%\]
%Therefore, $\rho_{m}(s)$ joins $(T'_{m},x^{o})$ with
%$(t^{e},x^{e})$, and so, the points $(T'_{m-1},x^{o})$,
%$(t^{e},x^{e})$ are chronologically related for all $m$. According
%to (\ref{e0}), this implies $T'<T+\epsilon_{m-1}$ for all $m$, and
%thus, $T'\leq T$.

Next, it is sufficient to prove that some value $T$
verifying (\ref{ee2}) always exists, and necessarily $T\geq T(x^o,(t^e,x^e))$. Let $t'<T(x^o,(t^e,x^e))$, and thus, $(t',x^{o})\ll (t^{e},x^{e})$.
% We know that $(t^{o},x^{o})\ll (t^{e},x^{e})$.
Let $\gamma:[t',t^e]\rightarrow V$,
$\gamma(t)=(t,c_{1}(t), c_{2}(t))$, be a timelike curve such
that $\gamma(t')=(t',x^{o})$ and $\gamma(t^{e})=(t^{e},x^{e})$.
Consider real curves $\overline{c}_{i}$, $i=1,2$, such that
\[
\left\{\begin{array}{l} 0\leq\dot{\overline{c}}_{i}(t)\leq
\sqrt{g_{i}(\dot{c}_{i}(t),\dot{c}_{i}(t))} \\
\overline{c}_{i}(t')=0 \\
\overline{c}_{i}(t^{e})=d_{i}(x_{i}^{o},x_{i}^{e})
\end{array}\right. \qquad\hbox{for}\;\; i=1,2.
\]
Then, $\overline{\gamma}(t)=(t,\overline{c}_{1}(t),\overline{c}_{2}(t))$
becomes a future directed timelike curve in the globally hyperbolic \multiwarped
spacetime with convex fibers
$V'=(\ncambios{(a,b)} \times \R^{2},-dt^{2}+\alpha_{1}dx_{1}^{2}+\alpha_{2}dx_{2}^{2})$ joining $\overline{\gamma}(t')=\point{t'}{0}{0}$ with $\overline{\gamma}(t^e)=\point{t^e}{d_{1}(x_{1}^o,x_{1}^{e})}{d_{2}(x_{2}^o,x_{2}^e)}$, i.e., \[\overline{\gamma}(t')=(t',0,0)\ll (t^e,d_{1}(x_{1}^o,x_{1}^{e}),d_{2}(x_{2}^o,x_{2}^{e}))=\overline{\gamma}(t^e).\]
Consider $T>t'$ such that $(T,0,0)\leq \overline{\gamma}(t^e)$ but $(T,0,0)\not\ll \overline{\gamma}(t^e)$. From
Avez and Seifert's result, there exists some lightlike geodesic in $V'$
joining both points. Now, from Prop. \ref{thm:characluzgeodesics} applied to this lightlike geodesic, there exist unique positive constants $\ncambios{\mu_1,\mu_2 \geq}0$, with $\mu_1+\mu_2=1$, such that
\[
\Integral{T}{t^{e}}{\mu_{i}}{i}{\mu_{k}}=|d_{i}(x^{o}_{i},x^{e}_{i})-0|=d_{i}(x^{o}_{i},x^{e}_{i}) \qquad\hbox{for}\;\;
i=1,2.
  \]
Finally, since $t'<T$ for all $t'<T(x^o,(t^e,x^e))$, necessarily $T(x^o,(t^e,x^e))\leq T$, which concludes the proof.

\end{proof}

%\begin{rem}
%  Condition (iii) in previous theorem can be replaced by the following one:
%  {\em
%  \begin{itemize}
%  \item[(iii')] there exist constants $\mu'_{1},\mu'_{2}\geq 0$, with $\mu'_1+\mu'_2=1$,
%%(with
%%$\mu'_{1}+\cdots +\mu'_{n}=1$)
%such that
%\begin{equation}\label{e2''}
%\Integral{t^{0}}{t^{e}}{\mu_{i}'}{i}{\mu_{k}'}\geq
%d_{i}(x^{0}_{i},x^{e}_{i})\qquad\hbox{for $i=1,2$},
%%\int_{t^{o}}^{t^{e}}\sqrt{\mu'_{i}}\alpha_{i}^{-1}\left(\sum_{j=1}^{n}\frac{\mu'_{j}}{\alpha_{j}}\right)^{-1/2}dt\geq
%%d_{i}(x^{o}_{i},x^{e}_{i})\qquad\hbox{for $i=1,...,n$},
%\end{equation}
%with equality in the $i$-th inequality if and only if
%${\mu'}_{i}=0$.
%  \end{itemize}}
%  Now, the implication $(ii)\Rightarrow (iii')$ is straightforward. However, (iii) presents more clearly the open character of the chronological relation, which will be much more practical for the forthcoming sections.
%\end{rem}

Let us consider now the characterization of the causal relation (see \cite[Theorem 2(2)]{FS}).
 %In a first approach, one can think that it follows just by replacing the strict inequalities in condition (iii) of Thm. \ref{c0} by regular ones. However, this procedure forgets the necessity to include a convexity condition on each fiber, in order to ensure the existence of some lightlike geodesic connecting the two points.
 \begin{defi}
A Riemannian manifold $(N,h)$ is $L$-{\em convex} if any pair of points $p,q\in N$ with $d_h(p,q)<L$ can be joined by a minimizing geodesic.
\end{defi}
%Now, we can establish the announced characterization about the causal relation (see)
\begin{prop}
\label{p2'}
Let $(V,g)$ be a {\multiwarped} spacetime as in (\ref{eq:1-aux}) whose fibers $(M_i,g_i)$ are $L_i$-convex for $i=1,2$. Consider two points $\point{t^{o}}{x^o_{1}}{x^o_2}, \point{t^{e}}{x_1^{e}}{x_2^e} \in V$, with $t^o \leq t^e$, satisfying $d(x_i^o,x_i^e)<L_i$, $i=1,2$. Then, the following conditions are equivalent:
\begin{itemize}
\item[(i)] the points are causally related,
$\point{t^{o}}{x_1^{o}}{x_2^o} \leq \point{t^{e}}{x_1^{e}}{x_2^e}$;

\item[(ii)] there exists a causal geodesic joining $\point{t^{o}}{x_1^{o}}{x_2^o}$ with $\point{t^{e}}{x_1^{e}}{x_2^e}$;

\item[(iii)] there exist constants $\mu'_{1},\mu'_{2}\geq 0$, $\mu'_1+\mu'_2=1$,
%(with
%$\mu'_{1}+\cdots +\mu'_{n}=1$)
such that
\begin{equation}
\label{e2'''}
\Integral{t^{o}}{t^{e}}{\mu'_{i}}{i}{\mu'_{k}} \geq
d_{i}(x^{o}_{i},x^{e}_{i})\qquad\hbox{for}\;\;
i=1,2.
\end{equation}
\end{itemize}
Moreover, if the equalities hold in (\ref{e2'''}), then there is a lightlike and no timelike geodesic joining the points.
\end{prop}

%The proof of previous proposition, that can be found on \cite[Theorem 2]{FS}, relies on the following characterization of geodesics in doubly warped spacetimes (see \cite[(5) and (6)]{FS}): a curve $\gamma:I\rightarrow \R\times M_1\times M_2$ with $\gamma(t)=(\tau(t),c_1(t),c_2(t))$ is a geodesic if, and only if,
%
%\begin{equation}
%  \label{eq:31}
%  \begin{array}{rl}
%  \displaystyle \frac{d^2 \tau}{d t^2}=& \displaystyle -\left(\sum_{i=1}^2 \frac{\mu_i}{(\alpha_i\circ \tau)^{3/2}} \frac{d(\sqrt{\alpha_i})}{d\tau}\circ \tau \right)\\  & \\
%\displaystyle  \frac{D}{dt}\frac{d\,c_i }{dt}=& \displaystyle -\frac{2}{\sqrt{\alpha\circ \tau}} \frac{d(\sqrt{\alpha\circ \tau})}{dt} \frac{d\,c_i}{dt}, \quad \hbox{i=1,2}.
%  \end{array}
%\end{equation}\footnote{Nueva ecuacion sacada de otro artículo, verificar que no he metido la pata...}where $D/dt$ denotes the covariant derivative associated to each $g_i$ along $c_i$ and $\mu_i$ is the constant $(\alpha_i^2\circ \tau)g_i(d\,c_i/dt,d\,c_i/dt)$.


% \newpage


% \subsection{Chronological relation in multiwarped spacetimes}

% Let $(V,g)$ be a doublywarped spacetime, \cambios{in order to study
% the causal boundary of this kind of spacetimes we need to characterize the chronological relation $\ll$ in $(V,g)$.}
% For every piecewise smooth curve $c:[s^{o},s^{e}] \rightarrow M$
% with endpoints $c(s^{o})=x^{o}, c(s^{e}) =x^{e}$, consider the
% unique future-directed lightlike curve $\gamma(t)=(t,c(s(t))),
% t\in [T,t^{e}]$, being $s(t)$ and $T=T[c]$ determined by
% $g(\dot\gamma, \dot\gamma )\equiv 0, s(T)=s^{o},
% s(t^{e})=s^{e}$\footnote{If such a curve $\gamma$ does not exist
% (i.e. $s(t)>s^{o}$ for all $t<t^{e}$), just define
% $T=T[c]:=-\infty$.}. Let $C \equiv C(x^{o},x^{e})$ be the set of
% all such curves $c=c(s)$, and consider the functional
% $$\J: C \rightarrow \mathbb{R}, \quad c \mapsto T[c].$$
% Define a function $T: M \times (\R\times M) \rightarrow \R$ in the
% following way:
% $$(x^{o},(t^{e},x^{e})) \mapsto T(x^{o},(t^{e},x^{e})):= {Sup}_{C}\J.$$
% Then, one easily has:
% %\footnote{With this notation, the relation
% %between function $T$ and the {\em (time) arrival} map $\delta:
% %V\times M\rightarrow\R$ is $\delta((T,x^{o}),x^{e})=t^{e}-T$, with
% %$T=T(x^{o},(t^{e},x^{e}))$. (Apply the continuity of $\delta$.)}
% \begin{equation}\label{e0}
% (t^{o},x^{o}) \ll (t^{e},x^{e}) \;\; \Longleftrightarrow \;\;
% t^{o}<T(x^{o},(t^{e},x^{e})).
% \end{equation}
% %\footnote{When property (\ref{e0}) is applied to a causal curve
% %$\gamma(s)=(t(s),x(s))$, it translates into: $(t^{o},x^{o})\in
% %I^{-}[\gamma]\Leftrightarrow t^{o}<b_{\gamma}(x^{o})$, where
% %$b_{\gamma}(x^{o}):=\lim_{s}(t(s)-\delta(x^{o},(t(s),x(s))))$ can
% %be interpreted as the {\em Busemann function} associated to
% %$\gamma$.}
% %Notice that function $\delta$ is always finite and continuous, and
% %essentially the same function is obtained if past-directed causal
% %curves are taken.

% Let $\gamma:  I \rightarrow V$,
% $\gamma(s)=(t(s),x_{1}(s),x_{2}(s))$ be a future-directed causal
% curve. From the expression of $g$ in previous section, it is
% \[
% \frac{dt}{ds}(s)=\sqrt{-D+\frac{c_{1}}{\alpha_{1}\circ
% t}+\frac{c_{2}}{\alpha_{2}\circ t}}(s),
% \]
% where $D:=g(d\gamma/ds,d\gamma/ds)\leq 0$ and
% $c_{i}:=(\alpha_{i}^{2}\circ t)g_{i}(dx_{i}/ds,dx_{i}/ds)$,
% $i=1,2$. Then, from the Inverse Function Theorem:
% \[
% \frac{ds}{dt}(t)=\left(-(D\circ s)+\frac{c_{1}\circ s}{\alpha
% _{1}}+\frac{c_{2}\circ s}{\alpha _{2}}+...+\frac{c_{n}\circ s}{\alpha
% _{n}}\right)^{-1/2}(t)
% \]
% Therefore, if we denote $t^{o}=t(s^{o})$, $t^{e}=t(s^{e})$, we
% deduce
% \[
% \begin{array}{c}
% \hbox{length}\left(x_{i}\mid_{[s^{o},s^{e}]}\right)=\int_{s^{o}}^{s^{e}}\sqrt{g_{i}(\dot
% x_{i}, \dot x_{i})} ds=\int_{t^o}^{t^e}\sqrt{g_{i}(\dot x_{i}, \dot
% x_{i})} \frac{ds}{dt} dt \qquad\qquad\qquad\quad \\
% \;\quad\qquad\qquad\qquad =\int_{t^o}^{t^e}\frac{\sqrt{c_{i}\circ
% s}}{\alpha_i}\left(-(D\circ s)+\frac{c_{1}\circ
% s}{\alpha_{1}}+\frac{c_{2}\circ s}{\alpha_{2}}\right)^{-1/2}dt
% \qquad\hbox{for}\;\; i=1,2.
% \end{array}
% \]

% \ncambios{Last equality will be usefull in the Lemma below, because it allows to compute the length of the spatial components of a causal curve in terms of
% $\mu_{i}$'s and the warping functions.}

% \cambios{
% \begin{lemma}
% \label{lightlikecurve}
% Let $(t^{o},x^o)$ and $(t^e,x^{e})$ in $V$ \multiwarped spacetime with $x^o \neq x^{e}$. If there exists $(\mu_{1},\mu_{2}) \in (0,1)^{2}$ with $\mu_{1}+\mu_{2}=1$ and satisfying the following integral conditions:
% \[
% \Integral{t^o}{t^e}{\mu_{i}}{i}{\mu_{k}}>d_{i}(x_{i}^o,x_{i}^{e}) \qquad\hbox{for}\;\;
% i=1,2.
% \]
% Then $(t^{o},x^o) \ll (t^e,x^{e})$, i.e., there exists a timelike future directed curve between $(t^{o},x^o)$ and $(t^e,x^{e})$.
% \end{lemma}

% {\bf Proof:}

% The integral conditions and the continuity of the lower limit of the integral imply that there exists $\epsilon>0$ such that   $L_{i}^{\epsilon}:=\Integral{t^o+\epsilon}{t^e}{\mu_{i}}{i}{\mu_{k}}$ satisfies $L_{i}^{\epsilon}>d_{i}(x_{i}^o,x_{i}^{e})$ for all $i$, then, this last condition implies
% that there exists curves $y_{i}:[s^o,s^e] \rightarrow M_{i}$ such that $length(y_{i})=L_{i}^{\epsilon}$,
% $y_{i}(s^{o})=x_{i}^{o}$ and $y_{i}(s^{e})=x^{e}_{i}$. Consider the following lightlike curve
% $\rho(s)=(\tau(s),\overline{y}_{1}(s),\overline{y}_{2}(s))$,
% with $\overline{y}_{i}$ reparametrizations of $y_{i}$,
% constructed by
% requiring
% \[
% \left\{\begin{array}{l}\dot{\tau}=\sqrt{\sum_{i=1}^{2}\frac{\mu_{i}}{\alpha_{i}\circ\tau}}
% \\ \tau(s^{e})=t^{e}
% \end{array}\right.,\qquad
% \left\{\begin{array}{l}g_{i}(\dot{\overline{y}}_{i},\dot{\overline{y}}_{i})=\frac{\mu_{i}}{(\alpha_{i}\circ
% \tau)^{2}} \\
% \overline{y}_{i}(s^{e})=x^{e}_{i}\end{array}\right.
% \qquad\hbox{for}\;\; i=1,2.
% \]
% Then,
% \[
% \hbox{length}(\overline{y}_{i}\mid_{[\tau^{-1}(t^o+\epsilon),s^{e}]})=\Integral{t^o+\epsilon}{t^{e}}{\mu_{i}}{i}{\mu_{k}}
% %\int^{t^{e}}_{T'_{m}}\sqrt{\mu_{i}}\alpha_{i}^{-1}\left(\sum_{j=1}^{n}\frac{\mu_{j}}{\alpha_{j}}\right)^{-1/2}dt
% =L_{i}^{\epsilon}=\hbox{length}(y_{i}).
% \]
% Therefore, $\rho(s)$ joins $(t^o+\epsilon,x^{o})$ with
% $(t^{e},x^{e})$, and so, the points $(t^{o},x^{o})$,
% $(t^{e},x^{e})$ are chronologically related since $(t^o,x^o) \ll (t^{o}+\epsilon,x^o) \leq (t^e,x^e)$.
% \begin{flushright}
% $\spadesuit$
% \end{flushright}
% \medskip

% \ncambios{Luis:
% Previous lemma can be extended to the following cases: (1) some $\mu_{i}$ is equal to zero or (2) some $x_{i}^{o}$ is equal to $x_{i}^{e}$. The construction of the causal curve can be carried out as in the proof of previous lemma with the difference that we will be working in the Riemannian manifold $(M_{j},g_{j})$ in which $\mu_{j}\neq 0$ or $x_{j}^{o} \neq x_{j}^{e}$ and we will obtain a curve with $i$ component equal to a constant point $x_{i}^{o}=x_{i}^{e}$.
% }
% \medskip

% Next result will give an analytic characterization of $T(x^{o},(t^{e},x^e))$ defined before:
% }

% \begin{prop}
% \label{p0}
% Let $(V,g)$ be a \multiwarped spacetime and $(x^{o},(t^{e},x^{e}))\in M \times V$, with $x^{o}\neq x^{e}$. If $T=T(x^{o},(t^{e},x^{e}))>-\infty$ then $\varsigma=T$ is the unique value satisfying
% \begin{equation}
% \label{e*}
% \Integral{\varsigma}{t^{e}}{\mu_{i}}{i}{\mu_{k}}=d_{i}(x^{o}_{i},x^{e}_{i}) \qquad\hbox{for}\;\;
% i=1,2
% \end{equation}
% for some (unique) constants $\mu_{1},\mu_{2} \geq 0$ with
% $\mu_{1}+\mu_{2}=1$.
% \end{prop}
% {\it Proof.} First, we are going to show that if $\varsigma=T'$
% satisfies (\ref{e*}), then necessarily $T'\leq T$. To this aim,
% take any sequence $\epsilon_{m} \searrow 0$ and define
% \[
% L_{i,m}:=\Integral{T'_{m}}{t^{e}}{\mu_{i}}{i}{\mu_{k}}
% \qquad
% \hbox{for}\;\; i=1,2 \quad\hbox{with}\;\; T'_{m}:=T'-\epsilon_{m}.
% \]
% \cambios{From the hypothesis for $T'$, we have that $L_{i,m}>d_{i}(x_{i}^o,x_{i}^{e})$ for all $i$, then Lemma \ref{lightlikecurve} implies that
% $(T'_{m},x^o) \ll (t^e,x^e)$ for all $m$, then, $T'-\epsilon_{m}<T$ for all $m$ and therefore $T' \leq T$.
% }
% \medskip
% %From the hypothesis for $T'$, there exist curves $y_{i,m}(s)$ in
% %$M_{i}$ with lengths $L_{i,m}$ such that
% %$y_{i,m}(s^{o})=x_{i}^{o}$ and $y_{i,m}(s^{e})=x^{e}_{i}$.
% %Consider the lightlike curves
% %$\rho_{m}(s)=(\tau(s),\overline{y}_{1,m}(s),..., \overline{y}_{n,m}(s))$,
% %with $\overline{y}_{i,m}$ reparametrizations of $y_{i,m}$,
% %constructed by
% %requiring
% %\[
% %\left\{\begin{array}{l}\dot{\tau}=\sqrt{\sum_{i=1}^{2}\frac{\mu_{i}}{\alpha_{i}\circ\tau}}
% %\\ \tau(s^{e})=t^{e}
% %\end{array}\right.,\qquad
% %\left\{\begin{array}{l}g_{i}(\dot{\overline{y}}_{i,m},\dot{\overline{y}}_{i,m})=\frac{\mu_{i}}{(\alpha_{i}\circ
% %\tau)^{2}} \\
% %\overline{y}_{i,m}(s^{e})=x^{e}_{i}\end{array}\right.
% %\qquad\hbox{for}\;\; i=1,2.
% %\]
% %Then,
% %\[
% %\hbox{length}(\overline{y}_{i,m}\mid_{[\tau^{-1}(T'_{m}),s^{e}]})=\Integral{T'_{m}}{t^{e}}{\mu_{i}}{i}{\mu_{k}}
% %%\int^{t^{e}}_{T'_{m}}\sqrt{\mu_{i}}\alpha_{i}^{-1}\left(\sum_{j=1}^{n}\frac{\mu_{j}}{\alpha_{j}}\right)^{-1/2}dt
% %=L_{i,m}=\hbox{length}(y_{i,m}).
% %\]
% %Therefore, $\rho_{m}(s)$ joins $(T'_{m},x^{o})$ with
% %$(t^{e},x^{e})$, and so, the points $(T'_{m-1},x^{o})$,
% %$(t^{e},x^{e})$ are chronologically related for all $m$. According
% %to (\ref{e0}), this implies $T'<T+\epsilon_{m-1}$ for all $m$, and
% %thus, $T'\leq T$.

% Next, we are going to prove that some value $\varsigma=T'$
% verifying (\ref{e*}) always exists, and necessarily $T'\geq T$.
% From (\ref{e0}), $(t^{o},x^{o})\ll (t^{e},x^{e})$ for any
% $-\infty<t^{o}<T(x^{o},(t^{e},x^{e}))$. Let
% $\rho(s)=(\tau(s),y_{1}(s), y_{2}(s))$ be a timelike curve such
% that $\rho(s^{o})=(t^{o},x^{o})$ and $\rho(s^{e})=(t^{e},x^{e})$.
% Consider $2$ curves $\overline{y}_{i}$ in $\R$ such that
% \[
% \left\{\begin{array}{l} 0\leq\dot{\overline{y}}_{i}(s)\leq
% \sqrt{g_{i}(\dot{y}_{i}(s),\dot{y}_{i}(s))} \\
% \overline{y}_{i}(s^{o})=0 \\
% \overline{y}_{i}(s^{e})=d_{i}(x_{i}^{o},x_{i}^{e})
% \end{array}\right. \qquad\hbox{for}\;\; i=1,2.
% \]
% Then, the curve
% $\overline{\rho}(s)=(\tau(s),\overline{y}_{1}(s),\overline{y}_{2}(s))$
% is a future directed timelike curve in the globally hyperbolic \multiwarped
% spacetime
% $(\R \times \R^{2},-dt^{2}+\alpha_{1}dx_{1}^{2}+\alpha_{2}dx_{2}^{2})$ joining $\overline{\rho}(s^o)=\point{t^o}{0}{0}$ with $\overline{\rho}(s^e)=\point{t^e}{d_{1}(x_{1}^o,x_{1}^{e})}{d_{2}(x_{2}^o,x_{2}^e)}$.
% Therefore, there exists some unique $T' \in (t^{o},t^{e})$ such that
% $(T',\overline{y}(s^{o}))\leq\overline{\rho}(s^{e})$ but
% $(T',\overline{y}(s^{o}))\not\ll \overline{\rho}(s^{e})$. From
% Avez and Seifert's result, there exists some lightlike geodesic
% joining $(T',\overline{y}(s^{o}))$ with $\overline{\rho}(s^{e})$. \cambios{Denote by $\lambda(s)=(\tau(s),c_{1}(s),c_{2}(s))$ ($s \in [0,1]$) the
% unique lightlike geodesic, and recall that a geodesic in a \multiwarped satisfies the following: (a) each $\mu_{i}=\alpha_{i}(\tau(s))^{2}g_{i}(c_{i}',c_{i}')$ is constant and (b) $c_{i}(s)$ is a pregeodesic in $(\mathbb{R},+dx_{i}^{2})$ for all $i$, see \cite{FS}[Eqns. (5), (6)]. Applying a change of variables in the following integral equalities by using the fact that $\frac{d\tau}{ds}=\sqrt{\sum_{i=1}^{2} \frac{\mu_{i}}{\alpha_{i} \circ \tau}}$ leads to:

% \begin{equation}
% \label{e7}
% \hbox{length}(c_{i})=\int_{0}^{1}\sqrt{g_{i}(c_{i}',c_{i}')}dr=\Integral{T'}{t^{e}}{\mu_{i}}{i}{\mu_{k}}=d_{i}(x^{o}_{i},x^{e}_{i})\qquad\hbox{for
% all}\;\; i
% %\int^{t^{e}}_{T'}\sqrt{\mu_{i}}\alpha_{i}^{-1}\left(\sum_{j=1}^{n}\frac{\mu_{j}}{\alpha_{j}}\right)^{-1/2}dt=d_{i}(x^{o}_{i},x^{e}_{i})\qquad\hbox{for
% %all}\;\; i
% \end{equation}
% }
% \cambios{Note that $\mu_{1}$ and $\mu_{2}$ are unique for $T'$, in fact, if $\mu_{1}'$ and $\mu_{2}'$ are different to $\mu_{1}$ and $\mu_{2}$ then we have two possibilities: (1) $\mu_{1}'<\mu_{1}$ and $\mu_{2}<\mu_{2}'$ or (2) $\mu_{1}<\mu_{1}'$ and $\mu_{2}'<\mu_{2}$. Any of the previous cases imply that for some $i_{0} \in \{1,2\}$ the following strict inequality is obtained:
% \[
% \Integral{T'}{t^e}{\mu_{i_{0}}'}{i_{0}}{\mu_{k}'}>\Integral{T'}{t^e}{\mu_{i_{0}}}{i_{0}}{\mu_{k}}=d_{i_{0}}(x_{i_0}^o,x_{i_0}^e).
% \]
% So, $\mu_{1}$ and $\mu_{2}$ are the unique constants satisfying equation (\ref{e7}).
% }
% %\footnote{For the uniqueness, and the forthcoming
% %property $(t^{o},x^{o})\not\ll (t^{e},x^{e})$, just apply [Subl.
% %3.4.2, PhD]: if $\mu_{i},\overline{\mu}_{i}\geq 0$,
% %$\mu_{1}+\cdots +\mu_{n}=1=\overline{\mu}_{1}+\cdots
% %+\overline{\mu}_{n}$ and $\overline{\mu}_{n}<\mu_{n}$ then there
% %exists some $i_{0}\in \{1,\ldots,n-1\}$ such that
% %$\overline{\mu}_{i_{0}}>\mu_{i_{0}}$ (and thus,
% %$\overline{\mu}_{n}/\overline{\mu}_{i_{0}}<\mu_{n}/\mu_{i_{0}}$)
% %and $\overline{\mu}_{j}/\overline{\mu}_{i_{0}}\leq
% %\mu_{j}/\mu_{i_{0}}$ $\forall j\neq n$.}
% \cambios{Moreover, necessarily $T'\geq T$. In fact, if $T'<T$ then for any ${t^o}' \in (T',T)$ it cannot happen that
% $({t^o}',x^{o}) \ll (t^{e},x^{e})$ since the same process to construct $T'$ will imply the existence of  $T''\in ({t^{o}}',T')$ such that $(T'',0)$ is causally but no
% timelike related to $(t^{e},d_{1}(x_{1}^o,x_{1}),d_{2}(x_{2}^o,x_{2}^{e}))$ in the globally hyperbolic \multiwarped spacetime $(\R \times \R^{2},\hat{g})$, this will contradict the achronality of $\partial I_{\hat{g}}^{+}((t^{e},d_{1}(x_{1}^o,x_{1}),d_{2}(x_{2}^o,x_{2}^{e})))$ because $(T',0) \ll (T'',0)$ and both points live in the boundary. Therefore, for any ${t^{o}}' \in (T',T)$ we have that $({t^o}',x^{o}) \not \ll (t^{e},x^{e})$ and this is a contradiction to the condition over $T$ given in (\ref{e0}). Therefore, $T' \geq T$. The first part of the proof proves that $T \geq T'$ for any $T'$ satisfying equation (\ref{e*}), therefore $T$ is the only point satisfying equation (\ref{e*}) for unique constants $(\mu_{1},\mu_{2})$ with $\mu_{1}+\mu_{2}=1$. }
% \begin{flushright}
% $\spadesuit$
% \end{flushright}



% %before implies $(t^{o},x^{o})\not\ll (t^{e},x^{e})$ for any
% %$T'<t^{o}<T$, which contradicts (\ref{e0})}. $\square$
% %Consider $n$ {\em complete} metrics
% %$\overline{g}_{1},\ldots,\overline{g}_{n}$ on
% %$M_{1},\ldots,M_{n}$, resp., such that $\overline{g}_{i}\geq
% %g_{i}$ and $\overline{g}_{i}$ agrees $g_{i}$ on the range of
% %$y_{i}$, $i=1,\ldots,n$. (As the range of $y_{i}$ is compact, the
% %metrics $\overline{g}_{i}$ can be constructed by a standard
% %partition of unity argument.) Now, the corresponding warped metric
% %$\overline{g}$ obtained by replacing $g_{i}$ by $\overline{g}_{i}$
% %in $g$ satisfy:
% %\begin{itemize}

% %\item[(a)] $\overline{g}$ is globally hyperbolic, because each
% %$\overline{g}_{i}$ is complete, \item[(b)] as
% %$\overline{g}_{i}=g_{i}$ on $y_{i}$, then $\gamma(s^{*})\in
% %\overline{I}^{+}((t^{o},x^{o}))$, where $\overline{I}^{+}(z)$
% %denotes the set of points which can be joined with $z$ by a
% %future-pointing $\overline{g}$-timelike curve, \item[(c)] by Avez
% %and Seifert's result, there exists a $\overline{g}^{1,2}$-timelike
% %geodesic joining the two points, \item[(d)] by previous
% %implication, there exist $\mu_{1},\ldots,\mu_{n}$ such that
% %inequalities (\ref{e2}) hold, putting in the right hand side of
% %each inequality the distance $\overline{d}_{i}$ associated to
% %$\overline{g}_{i}$, and \item[(e)] as $\overline{g}_{i}\geq
% %g_{i}$, the corresponding distances also satisfy
% %$\overline{d}_{i}\geq d_{i}$, and the desired inequalities are
% %obtained. % \begin{flushright}
% %$\spadesuit$
% %\end{flushright}
% %\end{itemize}
% \vspace{1mm}


% Now, we can establish the following useful characterizations of
% the chronologically relation in \multiwarped spacetimes:
% \begin{thm}\label{c0}
% Let $(V,g)$ be a multiwarped spacetime and $(t^{o},x^{o}), (t^{e},x^{e})\in V$ with $x^{o}\neq
% x^{e}$. The following conditions are equivalent\footnote{Habra que comentar en algun sitio que los resultados
% anteriores, y, por tanto, los siguientes, son trivialmente
% extensibles al caso $(t^o,x^o)\in \R\times \overline{M}_C$. de
% hecho, esto ya se esta usando en la Proposicion \ref{r} (2).}:
% \begin{itemize}

% \item[(i)] The points are chronologically related,
% $(t^{o},x^{o})\ll (t^{e},x^{e})$;
% \item[(ii)] there exist some
% unique $t^{o}<T<t^{e}$ and (unique) constants $\mu_{1},\mu_2 \geq
% 0$ with $\mu_{1}+\mu_{2}=1$ such that
% \begin{equation}\label{e2}
% \Integral{T}{t^{e}}{\mu_{i}}{i}{\mu_{k}}=d_{i}(x^{o}_{i},x^{e}_{i})\qquad\hbox{for}\;\;
% i=1,...,n;
% %\int_{T}^{t^{e}}\sqrt{\mu_{i}}\alpha_{i}^{-1}\left(\sum_{j=1}^{n}\frac{\mu_{j}}{\alpha_{j}}\right)^{-1/2}dt=d_{i}(x^{o}_{i},x^{e}_{i})\qquad\hbox{for}\;\;
% %i=1,...,n;
% \end{equation}

% \item[(iii)] there exist constants $\mu'_{1},\mu'_{2}\geq 0$
% %(with
% %$\mu'_{1}+\cdots +\mu'_{n}=1$)
% such that
% \begin{equation}\label{e2''}
% \Integral{t^{o}}{t^{e}}{\mu_{i}'}{i}{\mu_{k}'}\geq
% d_{i}(x^{o}_{i},x^{e}_{i})\qquad\hbox{for $i=1,...,n$},
% %\int_{t^{o}}^{t^{e}}\sqrt{\mu'_{i}}\alpha_{i}^{-1}\left(\sum_{j=1}^{n}\frac{\mu'_{j}}{\alpha_{j}}\right)^{-1/2}dt\geq
% %d_{i}(x^{o}_{i},x^{e}_{i})\qquad\hbox{for $i=1,...,n$},
% \end{equation}
% with equality in the $i$-th inequality if and only if
% $\mu'_{i}=0$.
% \end{itemize}
% \end{thm}
% {\it Proof.} The equivalence $(i) \Leftrightarrow (ii)$ is direct
% from (\ref{e0}) and Prop. \ref{p0}. \cambios{The implication $(iii) \Rightarrow (i)$ is given by Lemma \ref{lightlikecurve}.
% Finally, the implication $(ii)\Rightarrow (iii)$ is
% trivial.
% \begin{flushright}
% $\spadesuit$
% \end{flushright}
% }
% \medskip

% \cambios{We can ensure that we can choose $\mu_{i}'$s non-zero in part $(iii)$ of previous result. This will be of help when we deal with the analytic characterization of
% the chronological relation.

% \begin{cor}
% \label{munonzero}
% Let $(V,g)$ be a multiwarped spacetime. If $(t^o,x^o)$ and $(t^{e},x^e)$ satisfies $x^o \neq x^e$ then the following statements are equivalent:
% \begin{enumerate}
% \item $(t^{o},x^o) \ll (t^e,x^e)$.
% \item There exist constants $\mu_{1},\mu_{2}>0$ with $\sum_{i=1}^{2} \mu_{i}=1$ and such that:
% \begin{equation}
% \label{e3}
% \Integral{t^o}{t^e}{\mu_{i}}{i}{\mu_{k}}>d_{i}(x^{o}_{i},x^{e}_{i})\qquad\hbox{for $i=1,2$}.
% %\int_{t^{o}}^{t^{e}}\sqrt{\mu_{i}}\alpha_{i}^{-1}\left(\sum_{j=1}^{n}\frac{\mu_{j}}{\alpha_{j}}\right)^{-1/2}dt >
% %d_{i}(x^{o}_{i},x^{e}_{i})\qquad\hbox{for $i=1,...,n$},
% \end{equation}
% \end{enumerate}
% \end{cor}

% {\bf Proof:}
% \smallskip
% Suppose that $(t^{o},x^o) \ll (t^e,x^e)$, then Prop. \ref{c0} (iii) implies the existence of $\mu'_{1},\mu'_{2}\geq 0$
% such that
% \begin{equation}
% \Integral{t^{o}}{t^{e}}{\mu_{i}'}{i}{\mu_{k}'}\geq
% d_{i}(x^{o}_{i},x^{e}_{i})\qquad\hbox{for $i=1,2$},
% %\int_{t^{o}}^{t^{e}}\sqrt{\mu'_{i}}\alpha_{i}^{-1}\left(\sum_{j=1}^{n}\frac{\mu'_{j}}{\alpha_{j}}\right)^{-1/2}dt\geq
% %d_{i}(x^{o}_{i},x^{e}_{i})\qquad\hbox{for $i=1,...,n$},
% \end{equation}
% with equality in the $i$-th inequality if and only if $\mu'_{i}=0$. Suppose, without lose of generality, that equality is achieved in the equation involving $\mu_{1}'$, then $\mu_{1}'=0$ and $\mu_{2}'=1$, since $x^o \neq x^e$ then $d_{2}(x_{2}^o,x_{2}^e)>0$ and $\int_{t^o}^{t^e}\frac{1}{\sqrt{\alpha_{2}}}ds>d_{2}(x_{2}^o,x_{2}^e)$. So,
% we can modify $(\mu_{1},\mu_{2})=(0,1)$ to obtain $(\mu_{1},\mu_{2})$ with $\mu_{1} \neq 0 \neq \mu_{2}$, $\mu_{1}<1$, $\mu_{2}>0$ and satisfying:
% \[
% \Integral{t^{o}}{t^{e}}{\mu_{i}}{i}{\mu_{k}} >
% d_{i}(x^{o}_{i},x^{e}_{i})\qquad\hbox{for $i=1,2$}.
% \]
% \medskip

% That $(2)$ implies $(1)$ is given by Lemma \ref{lightlikecurve}.
% \begin{flushright}
% $\spadesuit$
% \end{flushright}
% }

% Finally, and as useful result for the study of the causal hierarchy in the next subsection, we will give a characterization of the causal relation $\leq$ in
% \multiwarped spacetimes when each Riemannian manifold $(M_{i},g_{i})$ is $L_{i}$-weakly convex (see \cite[Thm. 2]{FS}.), where a Riemannian manifold is $L$-weakly convex if for any $x_{0}$ and $x_{1}$ such that $d(x_0,x_1) \leq L$ there exists a minimizing geodesic.
% %where $L_{i}$-weakly convex means that any pair of points $x^{o},x^{e} \in M_{i}$ with $d_{i}(x_{i}^{o},x_{i}^{e})<L_{i}$ can be joined by a minimizing geodesic.

% \begin{prop}
% \label{p2'}
% Let $V=\mathbb{R}\times_{\alpha_1}M_1 \times_{\alpha_2}M_2$ be a multiwarped spacetime whose fibers $(M_i,g_i)$ are $L_i$-weakly convex for $i=1,2$. Assume also that $\point{t^{o}}{x^o_{1}}{x^o_2}, \point{t^{e}}{x_1^{e}}{x_2^e} \in V$ and satisfying $d(x_i^o,x_i^e)<L_i$, $i=1,2$, with $t^o \leq t^e$. The following conditions are equivalent:
% \begin{itemize}
% \item[(i)] the points are causally related,
% $\point{t^{o}}{x_1^{o}}{x_2^o} \leq \point{t^{e}}{x_1^{e}}{x_2^e}$;

% \item[(ii)] there exists a causal geodesic joining $\point{t^{o}}{x_1^{o}}{x_2^o}$ with $\point{t^{e}}{x_1^{e}}{x_2^e}$;

% \item[(iii)] there exist constants $\mu'_{1},\ldots,\mu'_{n}\geq 0$
% %(with
% %$\mu'_{1}+\cdots +\mu'_{n}=1$)
% such that
% \begin{equation}
% \label{e2'''}
% \Integral{t^{o}}{t^{e}}{\mu'_{i}}{i}{\mu'_{k}} \geq
% d_{i}(x^{o}_{i},x^{e}_{i})\qquad\hbox{for $i=1,2$.}
% \end{equation}
% \end{itemize}
% Moreover if the equality holds in all equations, then there is a lightlike and no timelike geodesic joining the points.
% \end{prop}











%%% Local Variables:
%%% mode: latex
%%% TeX-master: "DoublyWarpedBoundary2017.tex"
%%% End:
