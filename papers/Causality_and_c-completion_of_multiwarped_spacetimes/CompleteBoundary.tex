\section{The total c-completion of doubly warped spacetimes}
\label{sec:totalcompletion}
%\subsection{Total c-boundary as a point set}

We are now in conditions to construct the (total) c-completion of doubly warped spacetimes by merging appropriately the future and past c-boundaries obtained in previous section.

To this aim, first we need to determine the S-relation between indecomposable sets. So, let $\gamma:[\omega,\Omega)\rightarrow V$, $\gamma(t)=(t,c_{1}(t),c_{2}(t))$, be an inextensible future-directed timelike curve.
%Finally, we are going to study the point set, causal and topological structure for the total c-completion of {\multiwarped} spacetimes from the corresponding structures for the future (and past) partial boundaries obtained in previous sections. Of course, these structures will depend on integral conditions in both directions (i.e., from $0$ to $\pm\infty$) involving the warping functions $\alpha_i$, $i=1,2$.
%
%\begin{align}
%  \int^{0}_{-\infty}\frac{1}{\sqrt{\alpha_i(s)}}ds<\infty, \qquad \hbox{for $i=1,2$.}\hspace{1.5cm}\label{eq:20}\\
%   \int_{-\infty}^{0}\frac{1}{\sqrt{\alpha_1(s)}}ds<\infty \qquad \hbox{and}\qquad \int^{0}_{-\infty}\frac{1}{\sqrt{\alpha_2(s)}}ds=\infty. \label{eq:21}
%\end{align}
%
%\smallskip
Clearly, if $\Omega=b$ then $\uparrow I^{-}(\gamma)=\emptyset$, and there are no IFs S-related to $I^{-}(\gamma)$. So, we will focus on the case $\Omega<b$.

\begin{prop}
\label{tip}
Let $(V,g)$ be a \multiwarped spacetime and consider a future-directed (resp. past-directed) timelike curve $\gamma$ with associated endpoint $(\Omega^+,x_1^*,x_2^*) \in (a,b) \times M_{1}^{C} \times M_{2}^{C}$  (resp. $(\Omega^-,y_1^*,y_2^*) \in (a,b) \times M_{1}^{C} \times M_{2}^{C}$). Then
\begin{equation}
\begin{aligned}
        \uparrow I^{-}(\gamma) &=\{(t,x_1,x_2) \in V\; \mid\; \exists \,  \mu_{1},\mu_{2} > 0\;\; \hbox{{\rm such that}} \\
        & \Integral{\Omega^+}{t}{\mu_{i}}{i}{\mu_{k}}> d_{i}(x_{i},x_{i}^*),\; i=1,2.\}%\quad\hbox{pointwise and topologically.}
\end{aligned}
\end{equation}

\begin{equation*}
\begin{aligned}
(\hbox{resp.}\;\;\downarrow I^{+}(\gamma) &=\{(t,x_1,x_2) \in V\; \mid\; \exists \,  \mu_{1},\mu_{2} > 0\;\; \hbox{{\rm such that}} \\
                & \Integral{t}{\Omega^-}{\mu_{i}}{i}{\mu_{k}}> d_{i}(x_{i},y^{*}_{i}),\; i=1,2\}).%\quad\hbox{pointwise and topologically.}
\end{aligned}
\end{equation*}
As consequence, if $P\in\hat{V}$ and $F\in\check{V}$ are associated to $(\Omega^+,x^*_1,x^*_2)$ and $(\Omega^-,y^*_1,y^*_2)$ in $(a,b) \times M_{1}^{C} \times M_{2}^{C}$, resp, then the following equivalence holds:
\[
P \sim_{S} F\quad \Longleftrightarrow\quad
\Omega^{-}=\Omega^{+}\;\;\hbox{and}\;\; x^*_{i}=y^*_{i} \in M_{i}^{C},\; i=1,2.
\]
\end{prop}
\begin{proof} Assume that $\gamma:[\omega,\Omega^+) \rightarrow V$, $\gamma(t)=(t,c_1(t),c_2(t))$, is a future-directed causal curve with associated endpoint $(\Omega^+,x_1^*,x_2^*) \in (a,b) \times M_{1}^{C} \times M_{2}^{C}$ (for the past is analogous). We need to show that $\uparrow I^-(\gamma)=A_{(\Omega,x_1^*,x_2^*)}$, where
\begin{equation*}
\begin{aligned}
A_{(\Omega^+,x_1^{*},x_2^*)}&:=\{(r,x_1,x_2) \in V \mid \exists \,  \mu_{1},\mu_{2} > 0\;\; \hbox{{\rm such that}} \\
        &
        \int_{\Omega^+}^{r}\frac{\sqrt{\mu_{i}}}{\alpha_i(s)}\left(\sum_{k=1}^2\frac{\mu_k}{\alpha_k(s)}\right)^{-1/2}dt > d_{i}(x_{i},x_{i}^*),\; i=1,2\}.%\quad\hbox{pointwise and topologically.}
\end{aligned}
\end{equation*}
For the inclusion to the right, take $(r,x_1,x_2) \in \uparrow I^{-}(\gamma)$ and $\epsilon>0$ small enough so that $(r-\epsilon,x_1,x_2)\in \uparrow I^-(\gamma)$ (recall that the common future is open). For any sequence $\{t_{n}\}_{n}\nearrow \Omega^+$ we have $\gamma(t_n) \ll (r-\epsilon,x_1,x_2)$ for all $n$. From Prop. \ref{c0} there exist constants $\mu^n_{1},\mu^n_{2}>0$, with $\mu^n_{1}+\mu_{2}^n=1$ for all $n$, such that
\[
\int_{t_n}^{r-\epsilon}\frac{\sqrt{\mu^n_{i}}}{\alpha_i(s)}\left(\sum_{k=1}^2\frac{\mu_k^n}{\alpha_k(s)}\right)^{-1/2}dt > d_{i}(x_i,c_{i}(t_n))\quad i=1,2.
\]
Then, by the standard limit process, we deduce the following inequalities:
\[
\int_{\Omega^+}^{r-\epsilon}\frac{\sqrt{\mu^*_{i}}}{\alpha_i(s)}\left(\sum_{k=1}^2\frac{\mu_k^*}{\alpha_k(s)}\right)^{-1/2}dt \geq d_{i}(x_{i},x_{i}^{*}),\quad i=1,2,
\] where $\mu_i^*$ is the limit (up to a subsequence) of $\{\mu_i^n\}$. Now observe that some of previous inequalities become strict if we replace $r-\epsilon$ by $r$. So, a small variation of $\mu_1^*$ and $\mu_2^*$ if necessary (concretely, if one of these constants is zero), provides
% If, either $\mu^*_i=0$ for some $i=1,2$ or some inequality is actually an equality, a small variation of $\mu^*_{1},\mu^*_{2}$ ensures the existence of
positive constants $\mu_{1}',\mu_{2}'>0$ satisfying
\[
\int_{\Omega^+}^{r}\frac{\sqrt{\mu'_{i}}}{\alpha_i(s)}\left(\sum_{k=1}^2\frac{\mu'_k}{\alpha_k(s)}\right)^{-1/2}dt > d_{i}(x_{i},x_{i}^{*}),\quad i=1,2.
\]
In particular, $(r,x_1,x_2)\in A_{(\Omega,x_1^{*},x_2^*)}$, and so, $\uparrow I^{-}(\gamma) \subset A_{(\Omega,x_1^{*},x_2^*)}$.

\smallskip

 For the inclusion to the left, assume that $(r,x_1,x_2) \in A_{(\Omega,x_1^*,x_2^*)}$. By the continuity of both, the integral with respect to the lower limit of integration and the distance function, and the convergence of $\gamma(t)=(t,c_1(t),c_2(t))$ to
$(\Omega,x_1^{*},x_2^*)$, we deduce that
\[
\int_{t}^{r}\frac{\sqrt{\mu_{i}}}{\alpha_i(s)}\left(\sum_{k=1}^2\frac{\mu_k}{\alpha_k(s)}\right)^{-1/2} > d_{i}(x_{i},c_{i}(t))\quad\hbox{for large $t$.}
\]
So, from Prop. \ref{c0}, $\gamma(t) \ll  (r,x_1,x_2)$ for  all $t$, which implies $(r,x_1,x_2) \in \uparrow I^{-}(\gamma)$.

\smallskip

For the last assertion, assume that $P$ is associated to $(\Omega^+,x^*_1,x^*_2)\in (a,b)\times M_1^C\times M_2^C$. From the first part of this proposition, $\uparrow P= I^+(\sigma)$, where $\sigma$ is a past-directed timelike curve converging to $(\Omega^+,x^*_1,x^*_2)$. So, $F=I^+(\sigma)$ is the unique maximal IF into the common future of $P$. Reasoning analogously we deduce that $P$ is the unique maximal IP into the common past of $F$. In conclusion, $P$ is $S$-related just with the indecomposable future set $F$, and vice versa.
\end{proof}

%Now, we are ready to give the following proper characterization of the S-relation between terminal sets.




% \noindent {\it Proof:} From Prop. \ref{tip}, if $P$ and $F$ are associated to the same point $(\Omega,x_{1}^{*},x_{2}^{*}) \in \mathbb{R} \times M_{1}^{c} \times M_{2}^{c}$ then $P=\downarrow F$ and $F=\uparrow P$, and thus, $P \sim_{s} F$.

% Next, let us show that $P \sim_{S} F$ implies $(\Omega^{+},x^{P})=(\Omega^{-},x^{F})$. Since $\uparrow P=I^{+}(\gamma_{P})$ for some inextensible past directed causal curve $\gamma_{P}$ with past endpoint $(\Omega^{+},x^{P})$ and $\downarrow F=I^{-}(\gamma_{F})$ for some inextensible future directed causal curve $\gamma_{F}$ with future endpoint $(\Omega^{-},x^{F})$, necessarily $\uparrow P $ and $\downarrow F$ are indecomposable past and future sets, respectively. On the other hand, since $P \subset \downarrow F=I^{-}(\gamma_{F})$ and $F \subset \uparrow P=I^{+}(\gamma_{P})$, the S-relation
% $P \sim_{S} F$ implies that
% $P=I^{-}(\gamma_{F})$ and $F=I^{+}(\gamma_{P})$; in fact, otherwise maximality of $P$ and $F$ as an IP and IF is not satisfied. Finally, since $P=I^{-}(\gamma_{F})$ and $F=I^{+}(\gamma_{P})$ are univocally determined by $(\Omega^{+},x^{P})$ and $(\Omega^{-},x^{F})$ (see Prop. \ref{pp1}), then $(\Omega^{+},x_{P})=(\Omega^{-},x_{F})$. $\Box$
From this result it is clear that $\overline{V}$ is simple as a point set (see Defn. \ref{simpletop}). On the other hand, if we define
\[
\partial^{\B}V:=\hat{\partial}^{\B}V\cup\check{\partial}^{\B}V,
\]
the following identification is deduced:
\[
\overline{V}\setminus \partial^{\B} V\leftrightarrow (a,b) \times M_1^C\times M_2^C.
  \]
  In particular, $\partial V\setminus \partial^{\B} V$ can be identified with a cone with base $(M_1^C\times M_2^C)\setminus (M_1\times M_2)$. Moreover, if we assume that both $M_1^C,M_2^C$ are locally compact, Prop. \ref{topcurvasfinitas} ensures that previous bijection is a homeomorphism. Particularly, this proves that, given $(P,F)\in \overline{V}\setminus \partial^{\B}V$,
    \[
P\in \hat{L}(\{P_n\}) \iff F\in \check{L}(\{F_n\})
      \]
      for any sequence $\{(P_n,F_n)\}_n\in \overline{V}$. Hence, $\overline{V}\setminus \partial^{\B}V$ is also simple topologically.
      %Prop. \ref{topcurvasfinitas} permits to extend previous bijection to a homeomorphism by assuming that $M_1^C$ and $M_2^C$ are both locally compact.\footnote{J.L.: Revisar esta afirmacion.}

  Finally, the following lemma ensures that the line over each point $(x_1^*,x_2^*)\in (M_1^C\times M_2^C)\setminus (M_1\times M_2)$ is timelike:
 \begin{lemma}\label{causalstructurenoinf}
     If $(P,F),(P',F')\in \partial V\setminus \partial^{\B}V$, with $(P,F)\equiv (\Omega,x^*_1,x^*_2), (P',F')\equiv (\Omega',x^*_1,x^*_2)$ in $(a,b)\times M_1^C\times M_2^C$, satisfy that $a<\Omega<\Omega'<b$ then $(P,F)\ll (P',F')$.
  \end{lemma}
  \begin{proof}
     Take $t=(\Omega+\Omega')/2$ and $\mu_1=\mu_2=1/2$. For $i=1,2$, consider $y_i$ close enough to $x_i^*$ so that
    \[
\left\{\begin{array}{l}
\displaystyle\Integral{t}{\Omega'}{\mu_{i}}{i}{\mu_{k}}>
             d_{i}(y_i,x_i^*) \\

\displaystyle\Integral{\Omega}{t}{\mu_{i}}{i}{\mu_{k}}>
             d_{i}(y_i,x_i^*),

\end{array}\right.
\quad i=1,2.
      \]
    From Prop. \ref{pastofcurve} (and its past analogous) we deduce that $(t,y_1,y_2)\in F\cap P'$, as desired.
  \end{proof}

  \smallskip

  The $S$-relation described in Prop. \ref{tip} implies that each pair $(P,F)\in\overline{V}$ is determined by any of its non-empty components, that is, $\overline{V}$ is simple as a point set. Even more, from Prop. \ref{topcurvasfinitas} and the definition of the chronological limit (see \eqref{eq:29} and \eqref{limcrono}), $\overline{V}$ is topologically simple as well (recall Defn. \ref{simpletop}); concretely, if $(P,F)\in \overline{V}$, $P\neq\emptyset$, and $\sigma=\{(P_n,F_n)\}_n\subset \overline{V}$, then $(P,F)\in L_{chr}(\sigma)$ if, and only if, $P\in \hat{L}_{chr}(\{P_n\}_n)$. Therefore, in order to determine the, pointwise and topological, structure of the (total) $c$-boundary, it suffices to study the partial boundaries. Consequently, we will describe $\overline{V}$ in two different ways, according to our convenience, namely:
  \[
\overline{V}= (a,b) \times M_1^C\times M_2^C\cup\hat{\partial}^{\B}V\cup\check{\partial}^{\B}V=\hat{V}\cup \check{\partial}^{\B}V=\hat{\partial}^{\B}V\cup \check{V}.
    \]
     Restricting conveniently, the open sets of $\overline{V}$ containing a pair $(P,F)$ can be viewed as: (i) open sets in $(a,b) \times M_1^C\times M_2^C$ if $P\neq\emptyset\neq F$, (ii) open sets in $\hat{V}$ if $F=\emptyset$ or (iii) open sets in $\check{V}$ if $P=\emptyset$.

\smallskip

It rests to determine the causal structure of $\overline{V}$. This is contained in the following result, which summarizes all the information about the (total) c-completion of doubly warped spacetimes:
  \begin{thm}\label{thm:main}
    Let $(V,g)$ be a {\multiwarped} spacetime as in \eqref{eq:1-aux}. Then, there exists a homeomorphism
    \[
\overline{V}\setminus \partial^{\B}V \leftrightarrow (a,b) \times M_1^C\times M_2^C,
      \]
    where each line $\{(t,x_1^*,x_2^*): t\in (a,b),\; (x_1^*,x_2^*)\in M_1^C\times M_2^C\}$ is timelike. Moreover:
      %$\partial^{\infty}V$ has the following structure:
     \begin{itemize}
      \item[(i)] If \eqref{eq:7} and \eqref{eqq:7} hold, then $\partial^{\B} V$ is homeomorphic to a couple of spacelike copies of $M_1^C\times M_2^C$. As consequence, we have the following homeomorphism:
        \begin{equation}
          \label{eq:18}
         \overline{V}\equiv [a,b]\times M_1^C\times M_2^C\quad\hbox{pointwise and topologically.}
        \end{equation}

        \item[(ii)] If \eqref{eq:7} and \eqref{eqq:9} hold, then $\partial^{\B} V$ has a copy of $M_1^C\times M_2^C$ for the future, with spatial causal character; and a copy of $M_1^C\times \left({\cal B}(M_2)\cup \{\infty\}\right)$ for the past.  This second set can be seen as a cone with base $M_1^C\times \partial_{\cal B}(M_2)$ generated by horismotic lines over each pair $(x_1^*,[b_{c_2}])$ ending at the point $(x_1^*,\infty)$. As consequence, we have the following homeomorphism
          \[
              \overline{V}\equiv\left\{\begin{array}{l} \hat{V}\cup \check{\partial}^{\B}V \leftrightarrow \left((a,b]\times M_1^C\times M_2^C\right) \cup \left(M_1^C\times \left({\cal B}(M_2)\cup \{\infty\}\right)\right) \\ \hat{\partial}^{\B} V\cup \check{V} \leftrightarrow \left(\{b\}\times M_1^C\times M_2^C\right) \cup \left(M_1^C\times \left(B(M_2)\cup \{\infty\} \right) \right).
            \end{array}\right.
            \]


          \item[(iii)] If \eqref{eq:9} and \eqref{eqq:7} hold we have a structure analogous to (ii), but interchanging the roles of future and past.

            \item[(iv)] If \eqref{eq:9} and \eqref{eqq:9} hold, then $\partial^{\B} V$ has two copies of the space $M_1^C\times \left({\cal B}(M_2)\cup \{\infty\}\right)$, one for the future and the other one for the past, formed by horismotic lines over each point $(x_1^*,[b_{c_2}])\in M_1^C\times \partial_{\cal B}(M_2)$ ending at the point $(x_1^*,\infty)$. As consequence,
              \[
                  \overline{V}\equiv \left\{\begin{array}{l} \hat{V}\cup \check{\partial}^{\B}V \leftrightarrow \left(M_1^C\times \left(B(M_2)\cup \{\infty\} \right)\right) \cup \left(M_1^C\times \left({\cal B}(M_2)\cup \{\infty\}\right)\right) \\
                  \hat{\partial}^{\B} V\cup \check{V} \leftrightarrow \left( M_1^C\times \left({\cal B}(M_2)\cup \{\infty\}\right)\right) \cup  \left(M_1^C\times \left(B(M_2)\cup \{\infty\}\right)\right).
                \end{array}\right.
            \]
      \end{itemize}
    \end{thm}
    \begin{proof}
As we have argued before, the first assertion about the point set topological and causal structure of $\overline{V}\setminus \partial^{\ncambios{b}}V$ is a direct consequence of \cambios{Props. \ref{structuraparcialsininfinito'}, \ref{topcurvasfinitas} (and its past analogous)}, \ref{tip} and Lemma \ref{causalstructurenoinf}. So, we will focus on the rest of assertions.

  \begin{itemize}
  \item[(i)] The point set and topological structure are straightforward from Thms. \ref{futurestructurefiniteconditions} and \ref{pfuturestructurefiniteconditions}. So, we only need to prove that $\partial^{\B}V=\hat{\partial}^{\B}V\cup \check{\partial}^{\B}V$ is spacelike. Take $(P,\emptyset),(P',\emptyset)\in \partial^{\B}V$ two different boundary points (for TIFs is completely analogous). By using the identification in \eqref{eq:18}, we can assume that $(P,\emptyset)\equiv (b,x_1^*,x_2^*)$ and $(P',\emptyset)\equiv (b,y_1^*,y_2^*)$ with $(x_1^*,x_2^*)\neq (y_1^*,y_2^*)$. From the proof of Prop. \ref{structuraparcialsininfinito} (recall also Rem. \ref{rem:1}) it follows both, $P\not \subset P'$ and $P'\not \subset P$, thus $(P,\emptyset)$ and $(P',\emptyset)$ are neither timelike nor lightlike related, i.e., they are spatially related.

  \item[(ii)] The point set and topological structure are deduced from Thm. \ref{futurestructurefiniteconditions} and Thm. \ref{pfuturecomploneinfinite}. For the causal structure, let us take two points $(P,\emptyset),(P',\emptyset)\in \partial^{\B}V$ over the same point $(x_1^*,[b_{c}])\in M_1^C \times \partial_{\cal B} M_2$. Hence, we can make the identifications $(P,\emptyset)\equiv (x_1^*,b_{c_1})$ and $(P',\emptyset)\equiv (x_1^*,b_{c_2})$ with $b_{c_1}-b_{c_2}={\cal K}$, ${\cal K}$ constant. If we assume that ${\cal K}>0$, then $b_{c_1}\geq b_{c_2}$, and so, $P'\subset P$ (recall Lemma \ref{lemma:aux5}), i.e., both points are lightlike related (the case with ${\cal K}<0$ is completely analogous).
  \end{itemize}

Finally, assertions (iii) and (iv) are easily deduced from (i) and (ii).

\end{proof}

\begin{rem} {\rm (1) Of course, the four cases considered in previous theorem do not cover all the possibilities compatible with the finiteness of at most one warping integral (since the finite warping integral may not be necessarily the last one). However, the structure of the c-completion for these additional cases are easily deducible from our approach, and can be considered an easy exercise for the reader.

(2) In order to simplify the exposition, we have considered along this paper multiwarped spacetimes with just two fibers. Nevertheless, the corresponding results for the general case of $n$ fibers can be easily deduced by the reader (see, for instance, Section \ref{sec:applications}).}
\end{rem}

%\subsection{Multiwarped models.}
%\label{sec:multiwarped}
%Up to now, we analyzed the case of just two fibers...
%It is worth noting that, even if we have considered the doubly warped model along the paper, it is possible to obtain analogous results for the corresponding multiwarped models. Concretely, let us assume that $(V,g)$ is a multiwarped model where $\mathcal{V}=(a,b)\times \Sigma_1\times \dots\times \Sigma_n\times \Sigma_{n+1}$ and
%\begin{equation}
%  \label{eq:32}
%\mathfrak{g}=-dt^2+\alpha_1g_1+ \dots + \alpha_ng_n+\alpha_{n+1} g_{n+1},
%\end{equation}
%
%In this models, the characterization of the chronological relation, which is the key for the studies in this paper, reads as (compare with Prop. \ref{c0})
%\begin{prop}
%  Let $(V,g)$ be a multiwarped spacetime as in (\ref{eq:32}), and $(t^{o},x^{o}), (t^{e},x^{e})\in V$ with $x^{o}, x^{e}\in \prod_{i=1}^{n+1}\Sigma_i$ and $x^{o}\neq
%x^{e}$. The following conditions are equivalent:
%\begin{itemize}
%
%\item[(i)]  $(t^{o},x^{o})\ll (t^{e},x^{e})$; or, equivalently, $t^o<T(x^o,(t^e,x^e))$\footnote{Observe that the departure time function do not depend on the specific expression for the metric.} (recall (\ref{e0}));
%\item[(ii)] the departure time function $T(x^o,(t^e,x^e))$ is the unique real value $T\in \R$
%with $t^{o}<T<t^{e}$ such that for some (unique) constants $\mu_{1},\dots,\mu_{n+1} \geq
%0$, with $\sum_{i=1}^{n+1}\mu_{i}=1$, it satisfies
%\begin{equation}\label{e2}
%\Integralm{T}{t^{e}}{\mu_{i}}{i}{\mu_{k}}=d_{i}(x^{o}_{i},x^{e}_{i})\qquad\hbox{for}\;\;
%i=1,\dots,n+1;
%\end{equation}
%
%\item[(iii)] there exist positive constants $\mu'_{1},\dots,\mu'_{n}> 0$, with $\sum_{i=1}^{n+1}\mu'_i=1$,
%such that
%\begin{equation}\label{e2''}
%\Integralm{t^{o}}{t^{e}}{\mu_{i}'}{i}{\mu_{k}'}>
%d_{i}(x^{o}_{i},x^{e}_{i})\qquad\hbox{for $i=1,\dots,n$}.
%\end{equation}
%\end{itemize}
%\end{prop}
%
%Then, the rest of results follow analogously, just taking into account that the integral conditions \eqref{eq:7} and \eqref{eq:9} correspond respectively with
%
%\begin{equation}
%  \label{eq:33}
%  \int_{0}^{\infty}\frac{1}{\sqrt{\alpha_i(s)}}ds<\infty, \qquad \hbox{for $i=1,\dots,n+1$}
%  \end{equation}
%and
%\begin{equation}
%  \label{eq:34}
%  \left\{\begin{array}{l}
%    \displaystyle\int_{0}^{\infty}\frac{1}{\sqrt{\alpha_i(s)}}ds<\infty, \qquad \hbox{for $i=1,\dots,n$}\\
%\\
%   \displaystyle \int_{0}^{\infty}\frac{1}{\sqrt{\alpha_{n+1}(s)}}ds=\infty.
%  \end{array}\right.
%\end{equation}
%while the corresponding integrals for the past  \eqref{eq:40} and \eqref{eq:42} correspond with
%
%\begin{equation}
%  \label{eq:44}
%    \int^{0}_{-\infty}\frac{1}{\sqrt{\alpha_i(s)}}ds<\infty, \qquad \hbox{for $i=1,\dots,n+1$}
%\end{equation}
%and
%\begin{equation}
%  \label{eq:45}
%  \left\{\begin{array}{l}
%    \displaystyle\int^{0}_{-\infty}\frac{1}{\sqrt{\alpha_i(s)}}ds<\infty, \qquad \hbox{for $i=1,\dots,n$}\\
%\\
%   \displaystyle \int^{0}_{-\infty}\frac{1}{\sqrt{\alpha_{n+1}(s)}}ds=\infty.
%  \end{array}\right.
%\end{equation}
%
%
%Concretely, we arrive to the following main result for multiwarped models (compare with Thm. \ref{thm:main} assuming $M_1=\Sigma_1\times\dots\times \Sigma_n$ and $M_2=\Sigma_{n+1}$):
%
%
%\begin{thm}\label{thm:main2}
%   Let $(V,g)$ be a multiwarped spacetime as in \eqref{eq:32}. Then,
%    \[
%\overline{V}\setminus \partial^{\infty}V\equiv \cambios{(a,b)}\times \left(\prod_{i=1}^n  \Sigma_i^C\right)\times \Sigma_{n+1}^C\quad\hbox{pointwise and topologically.}
%      \]
%     Moreover, the line $\{(t,x_1^*,\dots,x_n^*,x_{n+1}^*): t\in\R\}$ is timelike.
%
%      \cambios{The rest of boundary points, that is,}  $\partial^{\infty}V$ has the following structure:
%     \begin{itemize}
%      \item[(i)] If \eqref{eq:33} and \eqref{eq:44} hold, then $\partial^{\infty} V$ correspond to two copies of $\prod_{i=1}^n  \Sigma_i^C\times \Sigma_{n+1}^C$ with spatial causal character. Moreover,
%        \begin{equation}
%         \overline{V}\equiv \cambios{[a,b]}\times \left(\prod_{i=1}^n  \Sigma_i^C\right)\times \Sigma_{n+1}^C\quad\hbox{pointwise and topologically.}
%        \end{equation}
%
%        \item[(ii)] If \eqref{eq:33} and \eqref{eq:45} hold, then $\partial^{\infty} V$ has a copy of $\prod_{i=1}^n  \Sigma_i^C\times \Sigma_{n+1}^C$ for the future, with spatial causal character; and a copy of $\prod_{i=1}^n  \Sigma_i^C\times \left({\cal B}(\Sigma_{n+1})\cup \{\infty\}\right)$ for the past.  This second set can be seen as a cone with base $\prod_{i=1}^n  \Sigma_i^C\times \partial_{\cal B}(\Sigma_{n+1})$ formed by \cambios{horismotic} lines over each pair $(x_1^*,\dots,x_n^*,[b_{c_{n+1}}])$ \cambios{ending at the point $(x_1^*,\dots,x_n^*,\infty)$}. Moreover,
%          \[
%            \begin{array}{rl}
%              \overline{V}\equiv &\hat{V}\cup \check{\partial}^{\infty}V\equiv \left(\cambios{(a,b]}\times \prod_{i=1}^n  \Sigma_i^C\times \Sigma_{n+1}^C \right) \cup \left(\prod_{i=1}^n  \Sigma_i^C\times \left({\cal B}(\Sigma_{n+1})\cup \{\infty\}\right)\right)\\
%
%              \equiv & \hat{\partial}^{\infty} V\cup \check{V} \equiv \left(\cambios{\{b\}}\times \prod_{i=1}^n  \Sigma_i^C\times \Sigma_{n+1}^C\right) \cup \left(\prod_{i=1}^n  \Sigma_i^C\times B(\Sigma_{n+1})\right)
%
%            \end{array}
%            \]
%            both pointwise and topologically.
%
%          \item[(iii)] If \eqref{eq:34} and \eqref{eq:44} hold we have a structure completely analogous to (ii), but interchanging the roles of future and past.
%
%            \item[(iv)] If \eqref{eq:34} and \eqref{eq:45} hold, then $\partial^\infty V$ has two copies of the space $\prod_{i=1}^n  \Sigma_i^C\times \left({\cal B}(\Sigma_{n+1})\cup \{\infty\}\right)$; one for the future and the other for the past, formed by horismotic lines over each point $(x_1^*,\dots,x_n^*,[b_{c_{n+1}}])$  in $\prod_{i=1}^n  \Sigma_i^C\times \partial_{\cal B}(\Sigma_{n+1})$ ending at the point $(x_1^*,\dots,x_n^*,\infty)$. Moreover,
%              \[
%                \begin{array}{rl}
%                  \overline{V}\equiv & \hat{V}\cup \check{\partial}^{\infty}V\equiv \left(\prod_{i=1}^n  \Sigma_i^C\times \left(B(\Sigma_{n+1})\cup \{\infty\} \right)\right) \cup \left(\prod_{i=1}^n  \Sigma_i^C\times \left({\cal B}(\Sigma_{n+1})\cup \{\infty\}\right)\right)\\
%                  \equiv & \hat{\partial}^{\infty} V\cup \check{V} \equiv \left( \prod_{i=1}^n  \Sigma_i^C\times \left({\cal B}(\Sigma_{n+1})\cup \{\infty\}\right)\right) \cup  \left(\prod_{i=1}^n  \Sigma_i^C\times \left(B(\Sigma_{n+1})\cup \{\infty\}\right)\right)
%                \end{array}
%            \]
%
%
%
%      \end{itemize}
%    \end{thm}


\section{Some examples of interest}%\footnote{Jony: TRABAJO EN PROCESO!!!}
\label{sec:applications}

In this section we are going to apply our results to compute the c-completion of some spacetimes of physical interest. Concretely, we will consider some Kasner models, the intermediate region of Reissner-Nordstr\"om and de Sitter models with (non necessarily compact) internal spaces.

% \begin{rem}
%   CUIDADO: El principal problema que veo en esta sección es que mostramos pocos ejemplos que sólo se resuelvan con los resultados de este paper. Fijémonos que:
%   \begin{itemize}
%   \item En el caso del Kasner, sólo estamos incluyendo un caso más.
%   \item En el caso de Reissner-Nordstr\"om tenemos los correspondientes bordes conformes, luego podríamos calcularlo con un paper anterior.
%     \item En el caso de los inflacionarios, el resultado puede deducirse de lo hecho por Harris.
%   \end{itemize}
% Por ello, veo NECESARIO incluir una sección adicional con el producto de de-Sitter con espacios tanto compactos como no compactos (estos últimos motivados con alguno de los paper que busqué el otro día, cuidado aquí...).
%
%\end{rem}
    \subsection*{Kasner models}
{\em Generalized Kasner models} are multiwarped  spacetimes $(V,g)$ where $V=(0,\infty)\times \R^{n}$ and
\begin{equation}
  \label{eq:35}
g=-dt^2+t^{2p_1}dx_1^2+\dots +t^{2p_{n}}dx_{n}^2,\quad\hbox{$(p_1,\ldots,p_n)\in\R^n$}.
\end{equation}
These models are solutions to the vacuum Einstein equations if $(p_1,\dots,p_{n})\in \R^{n}$ belongs to the so-called {\em Kasner sphere}, i.e., if it satisfies
  \[
\sum_{i=1}^{n}p_i=1=\sum_{i=1}^{n}p^2_i.
\]
Even if this condition does not fall under the hypotheses of our results, this does not cover all the cases of interest, and so, we are not going to assume it.

As far as we know, the c-boundary of these models can be faced in two different ways. On the one hand, by using Harris' result (Thm. \ref{thm:harris}); taking into account that the fibers are complete, this result gives a full description of the future c-boundary when $p_i>1$ for all $i$, and provides some partial information in the other cases. On the other hand, these models have been studied
 by Garc\'ia-Parrado and Senovilla in \cite{GS03} by using the isocausal relation. They essentially prove that, depending on the values of the constants $p_1,\ldots,p_n$, the corresponding Kasner model is isocausal to a particular Robertson-Walker model whose c-boundary is well-known. This may be useful, since, although the c-boundary of isocausal spacetimes may be different (see \cite{0264-9381-28-17-175016}), they can share some qualitative properties (see \cite{FHSIso2}).

 \smallskip


Of course, Thm. \ref{futurestructurefiniteconditions} parallels Harris' result for Kasner models when $p_i>1$ for all $i$. However, now we can go a step further and give a complete description of the c-boundary when
% Our aim here is to show that our result does not just  determine the causal boundary of Kasner models by using Thm. \ref{thm:main2} and compare it with the one obtained by García-Parrado and Senovilla. Observe that not all Kasner models will fall under the hypothesis of such a theorem. For instance, if the vector $(p_1,\dots,p_n)$ belongs to the Kasner sphere, they satisfy in particular that $|p_i|<1$, and so, the warping functions in \eqref{eq:35} will not satisfy the required integral conditions.

% \smallskip

% For this, we will distinguish two main cases:

% \begin{itemize}
% \item Let us assume that $p_i>1$ for all $i$. Then, we have that

%   \begin{equation}
%     \label{eq:36}
%    \int_1^{\infty}\frac{1}{t^{p_i}}dt<\infty \qquad \hbox{and}\qquad \int_0^{1}\frac{1}{t^{p_i}}dt=\infty,
%   \end{equation}


%     for all $i$, and so we fall under the hypothesis of Thm. \ref{thm:main2} (i). Therefore, as all the fibres are complete, it follows that:

%     \[
% \overline{V}\equiv [0,\infty]\times \R^{n},\qquad \partial V\equiv \left(\{0\}\times \R^{n} \right)\cup \left(\{\infty\}\times \R^{n} \right)
%       \]
% COMPARAR CON EL OTRO!!\footnote{Aquí hay un error, pero parece que puede solucionarse del siguiente modo: usamos la relación isocausal para pasar del Kasner a un Robertson-Walker. Con ello, puedo probar que dos curvas generan el mismo pasado y jugar con la distancia definida ahí....PENSAR CON CALMA}
    % \item For the second case we will make two assumptions. First, we will assume that there exists $1\leq k \leq n$ such that $p_{k+1}=p_j$ for all $k+1\leq j\leq n$. In particular, the Kasner metric becomes:
%      \[
%-dt^2+\sum_{i=1}^k t^{p_i}dx_i^2 + t^{p_{k+1}}\left(\sum_{i=k+1}^n dx^2_i \right),
%        \]
%  where
\[
p_i>1\;\;\hbox{for $1\leq i\leq k$,}\quad p_i=q\;\;\hbox{for $k+1\leq i\leq n$}\quad\hbox{and}\quad \int_1^{\infty}\frac{1}{t^{q}}dt=\infty.
\]
In this case we can write
\[
V=(0,\infty)\times\R^k\times\R^{n-k},\qquad g=-dt^2+\sum_{i=1}^{k}t^{2p_i}dx_i^2+t^{2q}\left(\sum_{i=k+1}^ndx_i^2 \right),
\]
% where
% \[
% \alpha_i(t)=t^{2 p_i},\;\; i=1,\ldots,k,\qquad \alpha_{k+1}(t)=t^{2q}.
% \]
In particular,
\[
\int_{1}^{\infty}\frac{dt}{t^{p_i}}<\infty,\;\; i=1,\ldots,k,\qquad\int_{1}^{\infty}\frac{dt}{t^{q}}=\infty.
\]
Therefore, the spacetime falls under the hypotheses of (the obvious multiwarped version of) Thm. \ref{futurecomploneinfinite} (essentially, with $M_1=\R^k$ and $M_2=\R^{n-k}$), which provides
%the warping functions satisfy \eqref{eq:45}, and thus, (the obvious multiwarped version of) Thm. \ref{futurecomploneinfinite} provides
the following homeomophism:
          \[
\hat{V}\;\leftrightarrow\; \left((0,\infty)\times \R^n\right)\cup \left(\R^k\times \left({\cal B}(\R^{n-k})\cup \{\infty\}\right)\right).
            \]
            So, taking into account that (see, for instance, \cite[Section 5.1]{H2})
            \[
{\cal B}(\R^{n-k})\equiv \R\times \mathbb{S}^{n-k-1},
              \]
              we immediately deduce that
              \[
\hat{\partial} V\leftrightarrow \R^k \times \left(\left(\R\times \mathbb{S}^{n-k-1}\right)\cup \{\infty\} \right).
                \]




%  This result provides a point set description of the causal completion of $(V,g)$ when the integrals $\int_{0}^{+\infty} \frac{1}{\sqrt{\alpha_{i}}}ds$ and $\int_{-\infty}^{0} \frac{1}{\sqrt{\alpha_{i}}}ds$ are finite for all $i=1,2$:
% \[
% \overline{V} \equiv (\mathbb{R}\cup \{+\infty\} \cup \{-\infty\}) \times M_{1}^{c} \times M_{2}^{c}.
% \]
% The point set structure of the c-boundary is then:
% \[
% \partial V \equiv (\mathbb{R} \times \partial_{c} M_{1} \times M_{2}^{c}) \cup (\mathbb{R} \times M_{1}^{c} \times \partial_{c} M_{2}) \cup ((\{-\infty\} \cup \{+\infty\}) \times M_{1}^{c} \times M_{2}^{c}).
% \]
%  Here, $(\{+\infty\} \cup \{-\infty\}) \times M_{1}^{c} \times M_{2}^{c}$ denotes the set of spatial boundary points, that is, pairs of the form $(P,\emptyset)$ and $(\emptyset,F)$, that are identified with points of the form $(\pm \infty,x_{1}^{\pm},x_{2}^{\pm})$ for some univocally determined $(x_{1}^{\pm},x_{2}^{\pm}) \in M_{1}^{c} \times M_{2}^{c}$ (recall Prop \ref{spatialboundaries}). The rest elements of the boundary correspond with timelike boundary points, that is, pairs of the form $(P,F)$ with $P \neq \emptyset \neq F$ and $P \sim_{S} F$, that are identified with a univocally determined point  $(\Omega,x_{1}^{*},x_{2}^{*}) \in \R \times M_{1}^{c} \times M_{2}^{c}$ (recall Prop. \ref{Srelatedtipstifs}).

%     \subsection*{C-completion on Multidimensional Inflationary Models with negative curvature}

% In \cite{doi:10.1063/1.532366}, Mignemi and Schmidt give a classification for multiwarped spacetimes with constant curvature, depending on the dimension of the model and the number of fibres $n$. Observe that the case where $n=1$ falls into the Robertson-Walker models, whose c-completion is well-known (see \cite{AF} for instance), so we will focus on $n\geq  2$. Moreover, let us restrict to the case of doubly warped models with {\em negative} constant curvature. Then, according to the classification given in \cite{doi:10.1063/1.532366}, $(V,g)$ falls into one of the following categories:

% \begin{itemize}
% \item[(a)] $D={\rm dim}(V)=3$, $V=\R\times \R\times \R$ and
%   \[
%     g=-dt^2+\left(sinh^2(t)dx^2 + cosh^2(t)dy^2\right).
%   \]

% \item[(b)] $D={\rm dim}(V)\geq 4$, $V=\R\times \R\times \mathbb{S}^{D-2}$ and
%   \[
% g=-dt^2+\left(sinh^2(t)dx^2 + cosh^2(t)g_{\mathbb{S}^{D-2}}\right)
%     \]

%     or

%     \[
%       g=-dt^2+\left(sinh^2(t)g_{\mathbb{S}^{D-2}} + cosh^2(t)dx^2\right).
%       \]
% \item[(c)] $D={\rm dim}(V)\geq 4$, $V=\R\times \mathbb{S}^{k}\times \mathbb{S}^{j}$ with $k+j=D-1$ and
%   \[
% g=-dt^2+\left(sinh^2(t)g_{\mathbb{S}^k} + cosh^2(t)g_{\mathbb{S}^{j}}\right)
%     \]
% \end{itemize}
% Of course, in previous classification $(\mathbb{S}^l,g_{\mathbb{S}^l})$ denotes the $l$-dimensional sphere attached with its standard Riemannian metric. As we can see, in all previous cases it follows that:

%   \[
% \int_0^{\pm\infty} \frac{1}{|sinh(t)|}dt<\infty,\qquad \hbox{and}\qquad \int_0^{\pm\infty} \frac{1}{|cosh(t)|}dt<\infty
%     \]
%     so we are in the conditions of Thm. \ref{thm:main} (i). Moreover, in all the cases the fibres (that we will denote $M_1$ and $M_2$ as usual) are complete. In conclusion, the c-completion of any inflationary multidimensional model with negative constant curvature is of the form:
%     \[
% \overline{V}=\left(\R\cup \{\pm\infty\}\right)\times M_1\times M_2.
%       \]
%  and its c-boundary is formed by two copies of the product of its fibres (one for the future and one for the past) with spatial causality.


\subsection*{The intermediate  Reissner-Nordstr\"om}
%\footnote{Jony: Hablar con José Luis sobre el cambio de orientación...}
%Next, we consider the intermediate region of Reissner-Nordstr\"om spacetime. In this case, Thm. \ref{thm:harris} is not useful to determine the future c-completion.
The Reissner-Nordstr\"om model is a spacetime $(V,g)$, where $V=\R\times \R\times \mathbb{S}^2$ and
\[
g=-\left(1-\frac{2m}{r}+\frac{q^{2}}{r^{2}}\right)dt^{2}+\left(1-\frac{2m}{r}+\frac{q^{2}}{r^{2}}\right)^{-1}dr^{2}+r^{2}(d\theta^{2}+sin^{2}\theta d\phi^{2}).
\]
 This metric degenerates at the zeros of the function $f(r)=(1-2m/r+q^2/r^2)$, which depend on the parameters $m$ (mass) and $q$ (charge). For our purposes we will require that $q\leq m$, which ensures the zeros $r^{\pm}=m\left( 1\pm\sqrt{1-q^2/m^2}\right)$ for $f$. The {\em intermediate region} of the Reissner-Nordstr\"om  is the spacetime $(V_I,g)$, where $V_I=\R\times (r^-,r^+)\times \mathbb{S}^2$.

Taking into account that $f(r)<0$ on $(r^-,r^+)$, the metric $g$ can be rewritten on $V_I$ as
\begin{equation}
  \label{eq:37}
g= -f(r)dt^2 + \frac{1}{f(r)}dr^2 + r^2 d\sigma^2=-d\tau^2 + r(\tau)^2d\sigma^2-F(\tau)dt^2,
\end{equation}
where
\[
d\tau:=-\frac{dr}{\sqrt{f(r)}}=\frac{dr}{\sqrt{-1+2m/r-q^2/r^2}} \qquad \hbox{and}\qquad F(\tau)=f(r(\tau)).
\]
Note that $\tau$ ranges in a finite interval $(a,b)$, and so, $(V_I,g)$ clearly corresponds with the standard form of a doubly warped spacetime where $V_I=(a,b)\times\mathbb{S}^2\times \R$.
%\footnote{El cambio de orden en las fibras es para que coincida con la forma del teorema...}.
In order to proceed with the analysis of the c-completion of $(V_I,g)$, we need to distinguish two cases: $q\neq 0$ and $q=0$.\footnote{Since the Penrose's diagram of Reissner-Nordstr\"om is well-known (see, for instance, \cite{hawking1975large}), the c-completion of $(V_I,g)$ can be also studied by applying \cite[Thm. 4.32]{FHSFinalDef}.}
%
%determine the c-completion of the intermediate region of the Reissner-Nordstr\"om, we have to study the integral conditions for the corresponding warping functions $\alpha_1(\tau)=r^2(\tau)$ and $\alpha_2(\tau)=-F(\tau)$, which will depend on the value of $q$. In the next subsections we distinguish two cases.
%
%\begin{rem} {\em  By using \cite[Thm. 4.32]{FHSFinalDef}, recalling the Penrose's diagram of the Reissner-Nordstr\"om spacetime (see, for instance, \cite{hawking1975large}).
%}
%\end{rem}


\subsubsection*{Intermediate Reissner-Nordstr\"om with charge, $q\neq 0$.}

%\footnote{CUIDADO AQUÍ: Hay un cambio en la orientación, cuando $r$ crece, nos movemos en la dirección pasada y viceversa...}
In this case, the warping integrals satisfy, for $a<c<b$,
 \begin{align}
   \int_{a}^{c}\frac{1}{\sqrt{\alpha_1(\tau)}}d\tau = \int_{r^-}^{r(c)}\frac{1}{r\sqrt{-1+\frac{2m}{r}-\frac{q^2}{r^2}}}dr<\infty\label{eq:38a} \\
   \int_{c}^{b}\frac{1}{\sqrt{\alpha_1(\tau)}}d\tau=\int^{r^+}_{r(c)}\frac{1}{r\sqrt{-1+\frac{2m}{r}-\frac{q^2}{r^2}}}dr<\infty\label{eq:38b}
  \end{align}
and
  \begin{align}
    \int_{a}^{c}\frac{1}{\sqrt{\alpha_2(\tau)}}d\tau=\int_{r^-}^{r(c)}\frac{1}{-1+\frac{2m}{r}-\frac{q^2}{r^2}}dr=\infty \label{eq:38}\\
    \int_{c}^{b}\frac{1}{\sqrt{\alpha_2(\tau)}}d\tau=\int^{r^+}_{r(c)}\frac{1}{-1+\frac{2m}{r}-\frac{q^2}{r^2}}dr=\infty.\label{eq:38c}
  \end{align}
  So, from Thm. \ref{thm:main} (iv) (with $M_1=\mathbb{S}^2$ and $M_2=\R$), we deduce the homeomorphisms
  \[
    \begin{array}{c}
      \overline{V}\leftrightarrow \left((a,b)\times \mathbb{S}^2\times \R\right) \cup (\mathbb{S}^2\times \left(\left(\R\times \{z^-, z^+\}\right)\cup \{i^+\} \right))\cup (\mathbb{S}^2\times \left(\left(\R\times \{z^-, z^+\}\right)\cup \{i^-\}\right)),\\
      \\
\partial V\leftrightarrow (\mathbb{S}^2\times \left(\left(\R\times \{z^-, z^+\}\right)\cup \{i^+\} \right))\cup (\mathbb{S}^2\times \left(\left(\R\times \{z^-, z^+\}\right)\cup \{i^-\}\right)),
    \end{array}
    \]
    where we have used that ${\cal B}(\R)\equiv \R\times \{z^-,z^+\}$, being $z^-$ and $z^+$ the two asymptotic directions (left and right) of $\R$.
    %\footnote{J.L.: ¿Es esto realmente relevante? JONY: No especialmente, puede quitarse si se quiere.}In particular, the c-completion of the intermediate region $(V_I,g)$ with $q\neq 0$ coincides with the c-completion of the spacetime ${\mathbb L}^2\times {\mathbb S}^2$.
  %, and such identification extends to the chronological and topological level.

  \subsubsection*{Interior Schwarzschild, $q=0$.}

When $q=0$, $f(r)$ has only one zero, we can identify $(r^-,r^+)\equiv (0,2M)$, and the intermediate region of Reissner-Nordstr\"om coincides with the interior region of Schwarzschild. In this case, the warping integrals (\ref{eq:38a}), (\ref{eq:38b}) and (\ref{eq:38c}) still hold, but \eqref{eq:38} transforms into
  \[
\int_{a}^{c}\frac{1}{\sqrt{\alpha_2(\tau)}}d\tau = \int_{0}^{r(c)}\frac{1}{-1+\frac{2m}{r}}dr<\infty.
  \]
  So, from Thm. \ref{thm:main} (iii), we deduce the homeomorphism
  %applies, and so, the past c-boundary has only spatial points. More precisely,
  \[
\overline{V}\leftrightarrow \left([a,b)\times \mathbb{S}^2\times \R\right) \cup \left( \mathbb{S}^2\times \left(\R\times \{z^-,z^+\} \right)\right)
    \]
  and thus,\footnote{The usual time-orientation on Reissner-Nordstr\"om makes the vector field $\partial_r$ past-directed in the intermediate region. So, in formula (\ref{d}), the roles of the future and past c-boundaries are interchanged with respect to the (a priori) expected ones.}
 \begin{equation}\label{d}
  \partial V\equiv \hat{\partial} V\cup \check{\partial} V \leftrightarrow  \left(\{a\}\times \mathbb{S}^2\times \R   \right) \cup \left(\mathbb{S}^2\times \left( \left(\R\times \{z^-,z^+\}\right)\cup \{i^+\} \right)\right).
  \end{equation}

\subsection*{De Sitter models with (non-necessarily compact) internal spaces}

Motivated by the relevance for the problem of the dS/CFT correspondence, finally we study the c-boundary of warped products of de Sitter models with general Riemannian manifolds.

Recall that {\em de Sitter spacetime} can be seen as a Robertson-Walker spacetime $(M,g_{M})$, where
\[
M=\R\times \mathbb{S}^l,\qquad g_{M}=-dt^2 + cosh(t)^2 g_{\mathbb{S}^l}.
  \]
Consider the doubly warped spacetime $(V,g)$ obtained as the product of de Sitter space $(M,g_{M})$ and a Riemannian manifold $(F,g_{F})$, i.e.,
  \[V=\R\times \mathbb{S}^l\times F,\qquad g=-dt^2+cosh^2(t)g_{\mathbb{S}^{l}}+g_{F}.
    \]
  The first warping function $\alpha_1(t)=\cosh(t)^2$ satisfies the finite integral conditions for both, the future and the past directions, meanwhile the second one $\alpha_2(t)\equiv 1$ does not. Therefore, from Thm. \ref{thm:main} (iv) (with $M_1=\mathbb{S}^l$ and $M_2=F$), we deduce the following homeomorphism for the c-boundary of $(V,g)$:
\[
\partial {V}\equiv \hat{\partial} V \cup \check{\partial} V \leftrightarrow  \left(\mathbb{S}^l\times \left({\cal B}(F)\cup \{i^+\}\right) \right)\,  \cup \, \left(\mathbb{S}^l\times \left({\cal B}(F)\cup \{i^-\}\right) \right).
\]
  In particular, if $(F,g_{F})$ is compact, then ${\cal B}(F)$ is empty, and the c-boundary becomes (compare with the last assertion on Thm. \ref{thm:harris}):
\[
  \partial V\leftrightarrow \left(\mathbb{S}^l\times \{i^+\}) \right)\,  \cup \, \left(\mathbb{S}^l\times \{i^-\} \right).
  \]

%
%
%   to the product of the de Sitter space with any Riemannian manifold, we obtain:
  % and with warping function satisfying the finite integral condition.  In such cases, as both warping functions satisfy nice  integral conditions and the fibres are complete, it is possible to make  use of Thm. \ref{thm:harris}. However, there exist cases where we need to eliminate such a restriction on the warping function (for instance, on the dS/CFT correspondence, see \cite{}), and Thm. \ref{thm:main} becomes essential. For instance, we can obtain the following general result:

%\begin{prop}
%  Let $(V,g)$ be the multiwarped spacetime obtained as the product of de Sitter space $(M,g_{M})$ and a Riemannian manifold $(F,g_{F})$, i.e.,
%  \[V=\R\times \mathbb{S}^l\times F,\qquad g=-dt^2+cosh^2(t)g_{\mathbb{S}^{l}}+g_{F}.
%    \]
% Then, the c-boundary of $(V,g)$ becomes
%  \[
%      \partial {V}\equiv \hat{\partial} V \cup \check{\partial} V \equiv  \left(\mathbb{S}^l\times \left({\cal B}(F)\cup \{i^+\}\right) \right)\,  \cup \, \left(\mathbb{S}^l\times \left({\cal B}(F)\cup \{i^-\}\right) \right).
%    \]
%
%\end{prop}
%Therefore, the c-completion is expressed in terms of the proper Busemann boundary ${\cal B}(F)$ of the Riemannian manifold $(F,g_{F})$. In particular, if $(F,g_{F})$ is compact, then ${\cal B}(F)$ is empty, and we obtain (compare with the last assertion on Thm. \ref{thm:harris}):
%\begin{cor}
%  If $(F,g_{F})$ is compact, then the c-boundary of $(V,g)$ is
%\[
%  \partial V\equiv \left(\mathbb{S}^l\times \{i^+\}) \right)\,  \cup \, \left(\mathbb{S}^l\times \{i^-\} \right).
%  \]
%\end{cor}

% \subsubsection*{Generalized Kasner models}
% Generalized Kasner models correspond with Multiwarped spacetimes $(V,g)$ where $V=(0,\infty)\times \R^{n}$ and
% \begin{equation}
%   \label{eq:35}
% g=-dt^2+t^{2p_1}dx_1^2+\dots +t^{2p_{n}}dx_{n}^2
% \end{equation}

%   with $p_i$ arbitrary constants. These models are solutions of the vacuum Einstein equations if the vector $(p_1,\dots,p_{n})\in \R^{n}$ belongs to the so-called {\em Kasner sphere}, that is, if its components satisfy:

%   \[
% \sum_{i=1}^{n}p_i=1=\sum_{i=1}^{n}p^2_i.
% \]

% The causal boundary of such models was studied by Garc\'ia-Parrado and Senovilla in \cite{GS03} by means of the isocausal relation. Essentially they prove that, depending on the constants $p_i$, the corresponding Kasner model is isocausal to a particular Robertson-Walker model with known causal completion. However, as it was proved in \cite{0264-9381-28-17-175016}, such a procedure presents some drawback, as the causal boundary of isocausal models may differ (even if they seem to share some qualitative properties in general cases, see \cite{FHSIso2}).


% Our aim here is to determine the causal boundary of Kasner models by using Thm. \ref{thm:main2} and compare it with the one obtained by García-Parrado and Senovilla. Observe that not all Kasner models will fall under the hypothesis of such a theorem. For instance, if the vector $(p_1,\dots,p_n)$ belongs to the Kasner sphere, they satisfy in particular that $|p_i|<1$, and so, the warping functions in \eqref{eq:35} will not satisfy the required integral conditions.

% \smallskip

% For this, we will distinguish two main cases:

% \begin{itemize}
% \item Let us assume that $p_i>1$ for all $i$. Then, we have that

%   \begin{equation}
%     \label{eq:36}
%    \int_1^{\infty}\frac{1}{t^{p_i}}dt<\infty \qquad \hbox{and}\qquad \int_0^{1}\frac{1}{t^{p_i}}dt=\infty,
%   \end{equation}


%     for all $i$, and so we fall under the hypothesis of Thm. \ref{thm:main2} (i). Therefore, as all the fibres are complete, it follows that:

%     \[
% \overline{V}\equiv [0,\infty]\times \R^{n},\qquad \partial V\equiv \left(\{0\}\times \R^{n} \right)\cup \left(\{\infty\}\times \R^{n} \right)
%       \]
% COMPARAR CON EL OTRO!!\footnote{Aquí hay un error, pero parece que puede solucionarse del siguiente modo: usamos la relación isocausal para pasar del Kasner a un Robertson-Walker. Con ello, puedo probar que dos curvas generan el mismo pasado y jugar con la distancia definida ahí....PENSAR CON CALMA}
%     \item For the second case we will make two assumptions. First, we will assume that there exists $1\leq k \leq n$ such that $p_{k+1}=p_j$ for all $k+1\leq j\leq n$. In particular, the Kasner metric becomes:
%       \[
% -dt^2+\sum_{i=1}^k t^{p_i}dx_i^2 + t^{p_{k+1}}\left(\sum_{i=k+1}^n dx^2_i \right).
%         \]
%         Secondly, we will assume that $p_i>1$ for $1\leq i\leq k$, so they also satisfy the integral conditions \eqref{eq:36}; but for $p_{k+1}$ we have:

%         \[
%    \int_1^{\infty}\frac{1}{t^{p_{k+1}}}dt=\infty \qquad \hbox{and}\qquad \int_0^{1}\frac{1}{t^{p_{k+1}}}dt<\infty,
%           \]

% \end{itemize}




%  This result provides a point set description of the causal completion of $(V,g)$ when the integrals $\int_{0}^{+\infty} \frac{1}{\sqrt{\alpha_{i}}}ds$ and $\int_{-\infty}^{0} \frac{1}{\sqrt{\alpha_{i}}}ds$ are finite for all $i=1,2$:
% \[
% \overline{V} \equiv (\mathbb{R}\cup \{+\infty\} \cup \{-\infty\}) \times M_{1}^{c} \times M_{2}^{c}.
% \]
% The point set structure of the c-boundary is then:
% \[
% \partial V \equiv (\mathbb{R} \times \partial_{c} M_{1} \times M_{2}^{c}) \cup (\mathbb{R} \times M_{1}^{c} \times \partial_{c} M_{2}) \cup ((\{-\infty\} \cup \{+\infty\}) \times M_{1}^{c} \times M_{2}^{c}).
% \]
%  Here, $(\{+\infty\} \cup \{-\infty\}) \times M_{1}^{c} \times M_{2}^{c}$ denotes the set of spatial boundary points, that is, pairs of the form $(P,\emptyset)$ and $(\emptyset,F)$, that are identified with points of the form $(\pm \infty,x_{1}^{\pm},x_{2}^{\pm})$ for some univocally determined $(x_{1}^{\pm},x_{2}^{\pm}) \in M_{1}^{c} \times M_{2}^{c}$ (recall Prop \ref{spatialboundaries}). The rest elements of the boundary correspond with timelike boundary points, that is, pairs of the form $(P,F)$ with $P \neq \emptyset \neq F$ and $P \sim_{S} F$, that are identified with a univocally determined point  $(\Omega,x_{1}^{*},x_{2}^{*}) \in \R \times M_{1}^{c} \times M_{2}^{c}$ (recall Prop. \ref{Srelatedtipstifs}).



% \begin{prop}
% If $(P,F) \in \overline{V}$ and $(P,F) \in L(\{(P_{m},F_{m})\}_{m})$ then $\{(\Omega_{m},x_{m}^{*})\}_{m}$ converges to $(\Omega,x^{*})$, where $(P,F)$ and each $(P_{m},F_{m})$ are identified with $(\Omega,x^{*})$ and $(\Omega_{m},x_{m}^{*})$ in $(\mathbb{R} \cup \{+\infty\} \cup \{-\infty\}) \times \overline{M}_{1}^{c}\times \cdots \times \overline{M}_{n}^{c}$, respectively.
% \end{prop}



% EL SIGUIENTE RESULTADO (MUY ESQUEMÁTICO) RESUME TODA LA ESTRUCTURA DE LOS ESPACIOTIEMPOS {\multiwarped}.

% \begin{thm}
% Let $(V,g)$ be a {\multiwarped} spacetime. $\overline{V}$ has the following structure.
%   \begin{enumerate}
%   \item Point set structure:

%   \begin{equation}
%     \label{eq:17}
% \overline{V}\setminus \partial^\infty V\equiv \R\times M_1^C\times M_2^C,
% \end{equation}
% with the product topology. Moreover, each point on $\left(M_1^C\times M_2^C\right)\setminus M_1\times M_2$ generates a timelike line over the boundary.

% For $\partial^\infty V$ we have:
% \begin{itemize}
% \item If \eqref{eq:7}, it is formed by two (spatial) copies of $M_1^C\times M_2^C$, one for the future and one for the past. So,
%   \[
% \overline{V}equiv\left(\R\cup \{\pm\}\right)\times M_1^C\times M_2^C
%     \]
% and previous equivalence extends topologically.
% \item If \eqref{eq:9}, it is formed by two copies of $M_1^C\times {\cal B}(M_2)$, one for the future and one for the past.  Each pair $x_1\in M_1^C$ and $[b_{x_2}]\in \partial_{{\cal B}(M_2)} M_2$ defines a lightlike line over the future and past boundary.
%   \end{itemize}

% \end{enumerate}
% \end{thm}



%\begin{proposition}
%Let $\gamma(t)=(t,c(t))$ with $\gamma(t) \rightarrow (\Omega,x^{*})$ be a future directed causal curve with $\Omega < \infty$
%and $x^* \in \bar{M}^{C}$ then
%$$\uparrow I^{-}(\gamma)=\{(t,q)\in V \mid \exists\ \mu_{1}^{q},...,\mu_{n}^{q}>0 {\text{ such that }}
%\int_{\Omega}^{t} \frac{\sqrt{\mu_{i}^{q}}}{\alpha_{i}}(\Sigma_{k}\frac{\mu_{k}^{q}}{\alpha_{k}})^{-1/2}>d_{i}(q_{i},x_i^*)\}$$
%\end{proposition}
%
%Let $(t_q,q) \in V$ such that for the future directed causal curve $\gamma(t)=(t,c(t))$ we have $\gamma(t) \ll (t_{q},q)$ for all
%$t \in [a,\Omega)$, then by the characterization of the chronological relation we have the existence of
%$(\mu_{1}^q(t),...,\mu_{n}^q(t)) \in (0,1)^n$ with $\Sigma_{k} \mu_{k}^q(t)=1$ and satisfying the inequalities:
%$$\int_{t}^{t^{q}} \frac{\mu_{i}^q(t)}{\alpha_{i}}\left(\Sigma_{k}\frac{\mu_{k}^q(t)}{\alpha_{k}}\right)^{-1/2}d\tau>d_{i}(q_{i},c_{i}(t)) \ \forall i,$$
%take any $\{t_{l}\}$ such that $t_{l} \rightarrow \Omega$ and consider the point $(\mu_{1}^{q},...,\mu_{n}^{q})$ limit point of some
%subsequence $\{(\mu_{1}^{q}(t_{l}),...,\mu_{n}^{q}(t_{l}))\}_{l}$ that satisfies the integral inequalities showed above, by using the convergence
%of those sequences we have:
%$$\int_{\Omega}^{t^{q}} \frac{\mu_{i}^q}{\alpha_{i}}\left(\Sigma_{k}\frac{\mu_{k}^q}{\alpha_{k}}\right)^{-1/2}d\tau \geq d_{i}(q_{i},x_{i}^{*}) \ \forall i,$$
%therefore $$\uparrow I^{-}[\gamma]=I^{+}[\{(t,q)\in V \mid \text{ there exists } \mu_{1}^{q},...,\mu_{n}^{q} \geq 0 \text{ such that }
%\int_{\Omega}^{t} \frac{\sqrt{\mu_{i}^{q}}}{\alpha_{i}}(\Sigma_{k}\frac{\mu_{k}^{q}}{\alpha_{k}})^{-1/2} \geq d_{i}(q_{i},x_i^*)\}]$$
%
%
%
%
%since this $\mu_{i}^{q}$ are taken as a limit of $\mu_{i}^{q}(t_{l})$ then it can happen that some $\mu_{i}^{q}$ are equal to zero, but not all of them are by
%the condition over the sum of $\mu_{i}^{q}$. Therefore for
%

%%% Local Variables:
%%% mode: latex
%%% TeX-master: "DoublyWarpedBoundary2017"
%%% End:
