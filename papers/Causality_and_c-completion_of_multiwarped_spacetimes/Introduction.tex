\section{Introduction}

The holographic principle \cite{tHooft:1993dmi,doi:10.1063/1.531249} states that the information of a particular space can be thought as encoded on a lower-dimensional boundary of the space, thus considering the original space as an hologram of the latter. One of the best understood examples of such a principle is the AdS/CFT correspondence, or Maldacena duality \cite{Mal}, where a dual description between the string theory on the bulk space (typically, the product of anti-de Sitter AdS$_n$ by a round sphere $\mathbb{S}^m$, or by another compact manifold) and a Quantum Field Theory without gravity on the  boundary of the initial space is achieved. Currently, there is a growing interest in the study of the realization of such a holographic principle with other bulk spaces \cite{PhysRevD.80.124008,GHODSI201079,1126-6708-2009-04-019}, particularly de Sitter spacetime dS$_{n}$ \cite{Witten:2001kn,1126-6708-2001-10-034,0264-9381-34-1-015009,Gibbons:1984kp}.



Two problems arise here. On the one hand, which {\em boundary} must we consider to formulate the holographic principle? In the original approach to the AdS/CFT correspondence, it is used the conformal boundary. However, this boundary presents important limitations generated by its {\em ad hoc} character: there is  no general formalism ensuring when a reasonably general spacetime has an intrinsic and unique conformal boundary. In fact, Bernstein, Maldacena and Nastase \cite{BMN} put forward different problems when the conformal boundary on plane waves is considered, and some years later, Marolf and Ross \cite{MR1} showed that, indeed, the conformal boundary is not available for non-conformally flat plane waves. This makes the alternative {\em causal boundary} a more suitable construction for the holographic principle, since it is intrinsic, conformally invariant and can be computed systematically.

On the other hand, anti-de Sitter spacetime is embedded in a string theory by making a warping product with a compact manifold. Due to the compactness of the latter, it is not difficult to obtain the causal boundary of the product from the causal boundary of AdS$_n$ (see for instance \cite{AF}). However, if de Sitter spacetime is considered, the no-go theorems (first due to Gibbons \cite{Gibbons:1984kp} and Maldacena, Nuñez \cite{doi:10.1142/S0217751X01003937}) ensure that there is no way to embed it in a string theory by a product with a compact manifold. There exist several ways to circumvent these no-go theorems, for instance, by considering warped product with non-compact Riemannian manifolds, but this complicates significatively the computation of the boundary.

\smallskip

 These problems motivate the systematic study of the causal boundary for the so-called {\em multiwarped spacetimes}, a class of spacetimes wide enough to cover the situations described above. A {\em multiwarped spacetime} $(V,g)$ can be written as $V=(a,b)\times M_1\times \dots \times M_n$, $-\infty\leq a<b\leq\infty$, and
\begin{equation}\label{eqqq}
g=-dt^2 + \alpha_1 g_1+\dots+\alpha_n g_n,
\end{equation}
where $\alpha_i:(a,b)\rightarrow\R$ are positive smooth functions and $(M_i,g_i)$ are Riemannian manifolds, for all $i=1,\ldots,n$.

As far as we know, the unique result in the literature about the causal boundary of these spacetimes is due to Harris \cite{H}:
\begin{thm}(Harris, 2008)\label{thm:harris}
Let $(V,g)$ be a multiwarped spacetime as above, and assume that (for some $c\in (a,b)$) the first $k$ warping functions, $1\leq i\leq k$, obey $\int_{c}^b(\alpha_i(s))^{-1/2}ds<\infty$, and the rest, $k+1\leq i\leq n$, obey $\int_{c}^b(\alpha_i(s))^{-1/2}ds=\infty$. Then the following hold:
  \begin{itemize}
  \item[(a)] If some Riemannian factor $M_i$ is incomplete, the future causal boundary $\hat{\partial} V$ has timelike-related elements.
  \item[(b)] If $M_i$ is incomplete for some $i\geq k+1$, the future causal boundary $\hat{\partial} V$ has null-related elements.
  %i.e., $P\subset P'$ with $P\neq P'$.
    \item[(c)] If neither of those occur, then $V$ has only spacelike future boundaries.
  \end{itemize}
In the last case, $\hat{\partial} V$ is homeomorphic to $M^0=M_1\times \dots \times M_k$. Furthermore, the future causal completion $\hat{V}$ is homeomorphic to $\left( (a,b]\times M^0\times M'\right)/\sim$, where $M'=M_{k+1}\times \dots \times M_n$ and $\sim$ is the equivalence relation defined by $(b,x^0,x')\sim (b,x^0,y')$ for any $x^0\in M^0$ and $x',y'\in M'$; $\hat{\partial} V$ appears there as $\{b\}\times M^0\times \{*\}$.
\end{thm}
\noindent This result covers the case of warped products of anti-de Sitter with compact manifold, however it does not provide a complete description of the boundary when the product of de Sitter with non-compact manifolds is considered.


% Sin embargo, como podemos ver en el resultado anterior, aún quedan ejemplos de espacio-tiempos multialabeados con gran interés físico que no pueden cubrirse con el teorema anterior.

% Uno de estos ejemplos de interés físico podemos encontrarlo en las teorías de quantum gravity, concretamente cuando queremos obtener un análogo a la conocida AdS/CFT correspondencia para espacios de de Sitter. La correspondencia original AdS/CFT (véase \cite{}) permite obtener un principio holográfico en el que encontrar una dualidad entre la teoría de cuerdas definida sobre el espacio-tiempo Anti-de Sitter y teorías conformes sobre el correspondiente borde del espacio. Para ello, es necesario considerar un warped product de AdS con una variedad Riemanniana compacta.

% Sin embargo, cuando intentamos obtener un análogo para la dS/CFT correspondence, aparece un problema al intentar embeber el espacio-tiempo de de Sitter en la string theory, the so called no-go theorems \cite{}. Essentially, such a theorems ensures that there is no way to embbed de de Sitter spacetime in string theory by means of a compact manifold. Como método para evitar dichos problemas, diversos autores han optado por considerar non-compact internal spaces. Esto hace necesario obtener el borde de espacio-tiempos de la forma $dS^d\times F$ with $F$ non compact, and Thm. \ref{thm:harris} do not give a full description on them.

 \smallskip

 The aim of this paper is twofold. First, we develop a systematic study of the causal structure and global causality properties of multiwarped spacetimes. Then, we use this approach to describe in full detail the causal boundary of these spacetimes by considering some mild integral hypothesis on the warping functions. Our main results for the future causal boundary, Thms. \ref{futurestructurefiniteconditions} and \ref{futurecomploneinfinite}, not only include the cases covered by Thm. \ref{thm:harris} for the future causal boundary, but also some additional ones. Concretely, we are able to remove the completeness condition on the Riemannian factors, and we also include the case where just one warping integral is infinite ($k+1=n$). Moreover, we consider the total c-boundary in Section \ref{sec:totalcompletion}, i.e. the construction obtained when the future and past causal boundaries are merged, concluding in Theorem \ref{thm:main}.
 Finally, we also discuss some relevant examples where our results are applicable.

The paper is organized as follows. In Section \ref{sec:preliminaries} we consider some preliminaries about the c-completion of spacetimes, focusing on the particular case of Robertson-Walker models, which we are going to use later. In Section \ref{sec:chronologicalrelation} we establish characterizations for the chronological and causal relations in doubly warped spacetimes. Then, in Section \ref{sec:causalladder}, we determine the position of these spacetimes into the causal ladder.
After that, in Sections \ref{sec:futurecompletion}, \ref{ss6} and \ref{sec:totalcompletion}, we use the machinery developed before to make a systematic study of the c-boundary of doubly warped spacetimes.
%, which culminates with Thms. \ref{thm:main} and \ref{thm:main2}.
Finally, in Section \ref{sec:applications}, we discuss the applicability of our results by considering several examples of interest: Kasner models, intermediate Reissner--Nordstr\"om, and de Sitter models with general internal spaces.

\section{Preliminaries}
\label{sec:preliminaries}
\subsection{The c-completion of spacetimes}
%\footnote{CUIDADO: Esta sección está adaptada del Computability...}
The {\em causal completion} of spacetimes is a conformally invariant construction which consists of adding {\em
ideal points} to a strongly causal spacetime in such a way that any timelike
curve in the original spacetime acquires some endpoint in the new
space \cite{GKP}. The {\em c-completion}, which is the concrete formalization of the causal completion that we are going to adopt in this paper, requires some preliminary notions.

Let $(V,g)$ be a spacetime. We say that a non-empty
%\footnote{OJO con esto!! Errores en articulos previos.}
subset $P\subset V$ is a {\em past set} if it coincides
with its past; i.e. $P=I^{-}(P):=\{p\in V: p\ll q\;\hbox{for
some}\; q\in P\}$. The {\em common past} of $S\subset V$ is
defined by $\downarrow S:=I^{-}(\{p\in V:\;\; p\ll q\;\;\forall
q\in S\})$. From construction, the past and common past sets are
open. When a past set $P$ cannot be written as the union of two proper
past sets, we say that $P$
%\footnote{Y con esto!! Errores en articulos previos.},
%both of which are also past sets,
is an {\em indecomposable past} set, {\em IP}. The indecomposable past sets can be classified in two major classes. On the one hand, the IPs which
coincide with the past of some point of the spacetime,
$P=I^{-}(p)$, $p\in V$, are called {\em proper indecomposable past
  sets}, {\em PIP}. On the other hand, the IPs which are obtained as the past of inextendible timelike curve $\gamma$, $P=I^-(\gamma)$, are called {\em terminal indecomposable past sets}, {\em TIPs}. The dual
notions, {\em future set}, {\em common future}, {\em IF}, {\em
TIF} and {\em PIF}, are defined just by interchanging the roles of
past and future in previous definitions.

In order to construct the {\em future} and {\em past c-completions}, first
we have to identify each {\em event} $p\in V$ with its PIP,
$I^{-}(p)$, and PIF, $I^{+}(p)$. To achieve this, we need to restrict our attention on {\em
distinguishing} spacetimes. On the other hand, in order to obtain consistent topologies for the
c-completions, we need to focus on a somewhat more restrictive class
of spacetimes, the {\em strongly causal ones} (see Defn. \ref{ant}). These are
characterized by the fact that the PIPs and PIFs constitute a
sub-basis for the topology of the manifold $V$.

%if any of its points is
%characterized by its past and future, {\em strongly causal} if it
%does not admit neither closed nor ``almost closed'' causal curves,
%

Once the events $p\in V$ have been identified with their corresponding PIPs, we define the {\em future c-boundary} $\hat{\partial}V$ of $V$
as the set of all the TIPs in $V$, and  {\em the future
c-completion} $\hat{V}$ as the set of all the IPs:
\[
V\equiv \hbox{PIPs},\qquad \hat{\partial}V\equiv
\hbox{TIPs},\qquad\hat{V}\equiv \hbox{IPs}.
\]
Analogously, each $p\in V$ can be identified with its corresponding PIF,
$I^+(p)$. The {\em past c-boundary} $\check{\partial}V$ of $V$ is
defined as the set of all the TIFs in $V$, and  {\em the past
c-completion} $\check{V}$ is the set of all the IFs:
\[
V\equiv \hbox{PIFs},\qquad \check{\partial}V\equiv
\hbox{TIFs},\qquad\check{V}\equiv \hbox{IFs}.
\]

In order to merge the future and past c-boundaries together to form the (total) c-boundary, the so-called S-relation comes into
play \cite{Sz}.
%First, we will identify $V$ with the subset of
%$\hat{V}\times \check{V}$ formed by all the pairs
%$(I^-(p),I^+(p))$.
Let $\hat{V}_{\emptyset}:=\hat{V}\cup \{\emptyset\}$ (resp.
$\check{V}_{\emptyset}:=\check{V}\cup \{\emptyset\}$), and define the
S-relation $\sim_S$ in $\hat{V}_{\emptyset}\times
\check{V}_{\emptyset}$ as follows: First, in the case $(P,F)\in \hat{V}\times
\check{V}$, \be \label{eSz}  P\sim_S F \Longleftrightarrow \left\{
\begin{array}{l}
P \quad \hbox{is included and is a maximal IP into} \quad
\downarrow F
 \\
F \quad \hbox{is included and is a maximal IF into} \quad \uparrow
P.
\end{array} \right.
\end{equation}
Here, {\em maximal} means that no other $P'\in\hat{V}$ (resp.
$F'\in \check{V}$) satisfies the stated property and contains
strictly $P$ (resp. $F$). As it was proved by Szabados in
\cite{Sz}, $I^-(p) \sim_S I^+(p)$ for all $p\in V$, and these are
the unique S-relations (according to our definition (\ref{eSz}))
involving proper indecomposable sets. In the case $(P,F)\in
\hat{V}_{\emptyset}\times \check{V}_{\emptyset}\setminus
\{(\emptyset,\emptyset)\}$, we also include \be \label{eSz2} P\sim_S
\emptyset, \quad \quad (\hbox{resp.} \; \emptyset \sim_S F )\ee if
$P$ (resp. $F$) is a (non-empty, necessarily terminal)
indecomposable past (resp. future) set that  is not S-related by
(\ref{eSz}) to any other indecomposable set (notice that
$\emptyset$ is never S-related to itself).

Now, we are in conditions to introduce
the notion of c-completion at the point set level, according to \cite{FHSFinalDef}:
\begin{defi}\label{d1}
The {\em c-completion} $\overline{V}$ of a strongly causal spacetime $V$ is formed by all
the pairs $(P,F)\in
\hat{V}_{\emptyset}\times\check{V}_{\emptyset}$ with
$P\sim_{S} F$. The {\em c-boundary} $\partial V$ is defined as
$\partial V:=\overline{V}\setminus V$, under the identification $V\equiv
\{(I^{-}(p),I^{+}(p)): p\in V\}$.
\end{defi}

The chronological relation $\ll$ of the spacetime is extended to
the c-completion  in the following way: Two points $(P,F),
(P',F')\in \overline{V}$ are {\em chronologically related},
$(P,F)\overline{\ll} (P',F')$, if $F\cap P'\neq\emptyset$.
The situation is remarkably more complicated if one tries to define the extension $\overline{\leqslant}$ of the causal relation $\leqslant$. However, in the particular case of the spacetimes treated in this paper, the following criterium suffices (see the discussion in
\cite[Sect. 6.4]{FHSBuseman}, and references therein, for further details). Given two points $(P,F),
(P',F')\in \overline{V}$ with, either $P\neq\emptyset$ or $F'\neq
\emptyset$:
\[
P\subset P'\;\;\hbox{and}\;\; F'\subset F \Rightarrow
(P,F)\overline{\leqslant} (P',F').
\]
Moreover, we will
say that two different pairs in $\overline{V}$ are {\em
horismotically related} if they are causally but not
chronologically related.

\smallskip

Finally, the topology of the spacetime is also extended to the
c-completion by means of the so-called {\em chronological topology} ({\em chr. topology}, for short). This is a {\em sequential} topology defined
in terms of the following
{\em limit operator} $L$ for $\overline{V}$ (see \cite[Section 2]{FHSHaus} for an introduction to sequential topologies): given a sequence
$\sigma=\{(P_{n},F_{n})\}_n\subset\overline{V}$,

\begin{equation}
  \label{eq:29}
(P,F)\in L(\sigma)\iff\left\{ \begin{array}{ccc} P\in \hat{L}(\{P_n\}_n) & \hbox{whenever} &  P\neq \emptyset\\ F\in \check{L}(\{F_n\}_n) & \hbox{whenever} & F\neq \emptyset, \end{array}\right.
\end{equation}
 where
\begin{equation}\label{limcrono}
\begin{array}{c}
\hat{L}(\{P_{n}\}_n):=\{P'\in\hat{V}: P'\subset {\mathrm
LI}(\{P_{n}\}_n)\;\;\hbox{and}\;\; P'\;\;\hbox{is a maximal IP into}\;\; {\mathrm LS}(\{P_{n}\}_n)\} \\
\check{L}(\{F_{n}\}_n):=\{F'\in\check{V}: F'\subset {\mathrm
LI}(\{F_{n}\}_n)\;\;\hbox{and}\;\; F'\;\;\hbox{is a maximal IF
into}\;\; {\mathrm LS}(\{F_{n}\}_n)\}
\end{array}
\end{equation}
(LI and LS are the usual point set inferior and superior limits of
sets).
%:
%i.e. LI$(\{A_{n}\})\equiv
%\liminf(A_{n}):=\cup_{n=1}^{\infty}\cap_{k=n}^{\infty}A_{k}$ and
%LS$(\{A_{n}\})\equiv
%\limsup(A_{n}):=\cap_{n=1}^{\infty}\cup_{k=n}^{\infty}A_{k}$.}
%Then, one can check that a topology is defined on $\overline{V}$
%as follows:
% \[
% %\label{charactopolo}
% \hbox{$C$ is {\em closed} $\Leftrightarrow$ $L(\sigma)\subset C$
% for any sequence $\sigma\subset C$.}
% \]
Concretely, the {\em closed sets} for the chr. topology are those subsets $C\subset V$ satisfying that $L(\sigma)\subset C$ for any sequence $\sigma\subset C$. Note that a topology on the future (resp. past) c-completion
$\hat{V}$ (resp. $\check{V}$) can be defined in a similar way,
just by using the limit operator $\hat{L}$ (resp. $\check{L}$)
instead of $L$. In this case, the resulting topology, which also
extends the topology of the spacetime, is called the {\em future}
(resp. {\em past}) {\em chronological topology}.
%(Fig.
%\ref{intfig3}).
\begin{rem}\label{propsimplepunt} {\rm We emphasize
the following natural properties about the chronological topology:
\begin{itemize}
\item[(1)] The chronological topology (as well as the future and
past ones) is sequential and $T_1$ (see \cite[Prop. 3.39 and 3.21]{FHSFinalDef}), but may
be non-Hausdorff.

\item[(2)] Clearly, if $(P,F)\in L(\{(P_{n},F_{n})\}_n)$ then
$\{(P_{n},F_{n})\}_n$ converges to $(P,F)$. When the  converse
happens, $L$ is called {\em of first order} (see \cite[Section
3.6]{FHSFinalDef}).

\item[(3)] Given a pair $(P,F)\in \partial V$, any timelike curve
defining $P$ (or $F$) converges to $(P,F)$ with the chronological
topology (see \cite[Th. 3.27]{FHSFinalDef}).
\end{itemize}
%(3) As a simple consequence of (2), if the chronological topology is Hausdorff, then $\partial V$ is simple as point set.
}
\end{rem}

\smallskip

There are several subtleties involving the definition of the c-boundary which are essentially associated to the following facts: on one hand, a TIP (or TIF) may not determine a unique pair in the c-boundary; on the other hand, the topology does not always agree with the $S$-relation, in the sense that, for $S$-related elements as above:


% These definitions for the c-boundary construction involve some
% particular %\footnote{\br Referee 2\er }
% subtleties, which are essentially associated to the following two
% facts: first, a TIP (or TIF) may not determine a unique pair in
% the c-boundary, and, second, the topology does not always agree
% with the S-relation, in the sense that, for S-related elements as
% above
$$P\in \hat{L}(\{P_n\}_n)\not\Leftrightarrow F\in \check{L}(\{F_n\}_n).$$ This
makes natural to consider the following special cases:
\begin{defi}\label{simpletop}
A spacetime $V$ has a c-completion $\overline{V}$ which is {\em
simple as a point set} if each TIP (resp. each TIF) determines a
unique pair in $\partial V$.
Moreover, the c-completion
%\footnote{\br {\bf !! OJO Tal vez
%mejor decir que la c-completion es simple, al menos cuando
%hablamos tambien de la topologia. Ok} \er}
is {\em simple} if it is simple as a point set and also {\em
topologically simple}, i.e. $(P,F)\in L(\{(P_{n},F_{n})\}_n)$ holds when
either $P\in \hat{L}(\{P_{n}\}_n)$ or $F\in \check{L}(\{F_{n}\}_n)$.
\end{defi}
% \footnote{Usar este concepto en la última sección...}

% \begin{remark}
% {\em The previous definition is slightly redundant. Even though
% the notion of {\em simple} comprises two levels, simplicity as a
% point set and as a topological set, really it is equivalent to the
% second level. In fact, if the c-completion is topologically simple
% and we assume by contradiction the existence of, say,
% $(P,F_1),(P,F_2)\in \partial V$ with $F_1\neq F_2$, then the
% constant sequence $\{(P,F_1)\}$ converges to $(P,F_1)$ and not to
% $(P,F_2)$ (as the c-completion is always $T_1$) in contradiction
% with topological simplicity. }
% \end{remark}

\subsection{Case of interest: Generalized Robertson-Walker model}\label{sec:Robertson}

Let us restrict our attention to the future c-completion of Robertson-Walker models. In order to obtain it, we will reproduce the study developed in \cite[Section 3]{FHSIso2} adapted to this particular setting.

\smallskip

Let $(V,g)$ be a {\em Generalized Robertson-Walker model}, that is, $V=(a,b)\times M$
% \footnote{Even though in \cite{FHSIso2} the temporal component of the spacetime is assumed to be $\R$, the results can be easily adapted to the case of a general interval $(a,b)$.}
and
\[
g=-dt^2+\alpha g_M,
\]
where $\alpha:(a,b)\rightarrow (0,\infty)$ is a positive smooth function and $(M,g_{M})$ is a Riemannian manifold. This spacetime will be denoted by $(a,b) \times_{\alpha} M$ for short. Assume that the warping function $\alpha$ satisfies the following integral condition:
\begin{equation}
  \label{eq:47}
  \int_{\C}^{b}\frac{1}{\sqrt{\alpha(s)}}ds=\infty,\quad a<\C<b.
\end{equation}


\begin{rem}
  The only difference between the spacetime model studied in \cite{FHSIso2} and the one considered here is that  the temporal component $\R$ and the integral conditions

  \[
   \int^{\infty}_{0}\frac{1}{\sqrt{\alpha(s)}}ds=\int^{0}_{-\infty}\frac{1}{\sqrt{\alpha(s)}}ds=\infty
    \]
  has been replaced by a general interval $(a,b)$ and just the integral condition \eqref{eq:47}.
Nevertheless, the results established in this section are easily deducible by simple adaptations of the corresponding proofs in \cite{FHSIso2}. We leave the details to the reader interested on the subject.
\end{rem}

The chronological relation can be characterized in terms of the warping function $\alpha$ and the distance $d$ associated to $(M,g_M)$ as follows:
\begin{equation}
  \label{eq:26}
  \begin{array}{rl}
    (t^o,x^o)\ll (t^e,x^e) \iff   & d(x^o,x^e)<\displaystyle\int_{t^o}^{t^e}\frac{1}{\sqrt{\alpha(s)}}ds \\ & \\
    \iff & \displaystyle \int_{\C}^{t^o}\frac{1}{\sqrt{\alpha(s)}}ds<\int_{\C}^{t^e}\frac{1}{\sqrt{\alpha(s)}}ds-d(x^o,x^e).
  \end{array}
\end{equation}
%This characterization of the chronological relation is key for the study of the structure of the future completion.
Take a future-directed timelike curve $\gamma:[\omega,\Omega)\rightarrow V$, which can be expressed without loss of generality as $\gamma(t)=(t,c(t))$. The function
\begin{equation}
  \label{eq:25}
t\mapsto \int_{\C}^{t}\frac{1}{\sqrt{\alpha(s)}}ds-d(\cdot,c(t))
\end{equation}
is increasing with $t$ (see \cite[Prop. 3.1]{FHSIso2} for details). In particular, from \eqref{eq:26},
  \begin{equation}
    \label{eq:27}
  \begin{array}{rl}
      P=& I^-(\gamma)\\ = & \left\{(t^o,x^o)\in V:\displaystyle\int_{\C}^{t^o}\frac{1}{\alpha(s)}ds<\lim_{t^e\rightarrow \Omega}\left(\int_{\C}^{t}\frac{1}{\sqrt{\alpha(s)}}ds-d(x^o,c(t)) \right)\right\}.
    \end{array}
  \end{equation}
    Therefore, $P=P(b_{c})$, where $b_{c}(\cdot):=\lim_{t\rightarrow \Omega}\left(\displaystyle\int_{\C}^{t}\frac{1}{\sqrt{\alpha(s)}}ds-d(\cdot,c(t)) \right)$ is the {\em Busemann function associated to the curve $c$} and
    \begin{equation}
      \label{eq:28}
     P(f):=\{ (t^o,x^o)\in V:\int_{\C}^{t^o}\frac{1}{\sqrt{\alpha(s)}}ds<f(x^o)\}.
   \end{equation}

      Summarizing, the future c-completion $\hat{V}$, i.e. the set of all IPs, can be identified with the set of all Busemann functions on $M$. So, if we denote by $B(M)$ the set of all finite Busemann functions, it follows that
      \[
\hat{V}\equiv B(M)\cup \{\infty\},
        \]
where $\infty$ represents the constantly infinite Busemann function, which is associated to the TIP $P(\infty)=V=i^+$.

        Next, let us write $\hat{V}=\left(\hat{V}\setminus \hat{\partial}^{\B}V\right) \cup \hat{\partial}^{\ncambios{\B}}V$, where $\hat{\partial}^{\ncambios{\B}}V$ denotes the TIPs obtained from inextensible future-directed timelike curves with divergent timelike component ($\Omega=b$). The finite Busemann functions associated to these curves are called {\em proper}, and the set of all of them is denoted by ${\cal B}(M)$. So,
        \[
\hat{\partial}^{\ncambios{\B}}V\equiv {\cal B}(M)\cup \{\infty\}.
          \]

%           \ncambios{In order to rewrite these sets in a more appealing way, let us define an action $B(M)\times (0,\infty)\rightarrow B(M)$ given by $(b_{c},k)\rightarrow b_{c}+k$. This action is well defined whenever  $b_{c}+k$ is again a Busemann function. In order to show this, and for each $t$ on the domain of $c$, let us denote by $s_k(t)$ the value ensuring that
%             \[
% \int_{t}^{s_k(t)}\frac{1}{\sqrt{\alpha(s)}}ds=K
%               \]
%               (recall \eqref{eq:47}). Then,
%               \[
%                 \begin{array}{rl}
%                   b_{c}(\cdot)+k = &\lim_t \left(\int_\C^t \frac{1}{\sqrt{\alpha(s)}}ds-d(\cdot,c(t))\right)+k     \\
%                   = & \lim_t \left(\int_\C^t \frac{1}{\sqrt{\alpha(s)}}ds-d(\cdot,c(t))\right) + \int_{t}^{s_k(t)}\frac{1}{\sqrt{\alpha(s)}}ds\\
%                   = & \lim_t \left(\int_\C^{s_k(t)} \frac{1}{\sqrt{\alpha(s)}}-d(\cdot,c(t))\right)ds\\
%                   = & b_{\tilde{c}}(\cdot)
%                 \end{array}
%                 \]
%   for $\tilde{c}(t)=c(s_{k}^{-1}(t))$. Then define
%           }

        In order to rewrite this set in a more appealing way, consider the quotient space
        \[
 \partial_{\cal B} M:={\cal B}(M)/\R
            \]
            where  two Busemann functions are $\R$-related if they differ only by a constant.
            Then, we can write
            \[
\hat{\partial}^{\ncambios{\B}} V\equiv \left(\R\times\partial_{\cal B} M \right)\cup \{\infty\},
              \]
              and so, we can see the future c-boundary as a cone with base $\partial_{\cal B} M$ and apex $\{\infty\}$. This picture is reinforced by the fact that the generatrix lines of the cone are shown to be horismotic, that is, each couple of points on the same generatrix line are horismotically related (see \cite[Section 3]{FHSIso2}).

              %is particularly appealing recalling the causal structure, as its generatrices are horismotic lines.
              %Topologically however, the situation is quite more technical in general (see \cite[Chapter 6]{FHSBuseman}).

       The remaining set $\left(\hat{V}\setminus \hat{\partial}^{\ncambios{\B}}V\right)$ is formed by IPs obtained as the past of future-directed timelike curves $\gamma:[\omega,\Omega)\rightarrow V$, $\gamma(t)=(t,c(t))$, with $\Omega<b$. It can be proved that, in this case, $c(t)\rightarrow x^*\in M^C$, where $M^C$ denotes the Cauchy completion of $(M,g_{M})$, and so,

         \begin{equation}
           \label{eq:46}
b_{c}(\cdot)=d_{(\Omega,x^*)}(\cdot):=\int_{\C}^{\Omega}\frac{1}{\sqrt{\alpha(s)}}ds-d(\cdot,x^*)
         \end{equation}
(see \cite[(3.7)]{FHSIso2}). In conclusion, we have the following identification
       \begin{equation}
            \label{eq:19}
       \hat{V}\setminus \hat{\partial}^{\ncambios{\B}}V\equiv (a,b) \times M^C,
          \end{equation}
          which implies,
          \[
            %\begin{array}{rl}
\hat{V}=
%&
\left(\hat{V}\setminus \hat{\partial}^{\ncambios{\B}}V\right)\cup \hat{\partial}^{\ncambios{\B}}V
%\\
%\equiv
%&
%\left(\R\times M^C\right) \cup {\cal B}(M)
%\\
\equiv
%&
\left((a,b)\times M^C\right) \cup \left(\R\times \partial_{\cal B} M \right) \cup \{\infty\}.
            %\end{array}
            \]
            and
            \[
            \hat{\partial}V=((a,b) \times\partial^C M)\cup (\R\times \partial_{{\cal B}}M)\cup \{\infty\}.
            \]

Note that, given $P= P(b_{c})$ and $P_n=P(b_{c_n})$,
              \begin{equation}
                \label{eq:50}
                \begin{array}{c}
                P\subset {\rm LI}(\{P_n\}_n) \iff b_{c}\leq \liminf_n(\{b_{c_n}\}_n)\\
                \left(\hbox{resp. }P\subset {\rm LS}(\{P_n\}_n) \iff b_{c}\leq \limsup_n(\{b_{c_n}\}_n)\right).
               \end{array}
              \end{equation}
              This property joined to the identification between $\hat{V}$ and $B(M)$ described above, suggests to translate
            the future chronological topology on $\hat{V}$ into a (sequential) topology on $B(M)$, which is also called {\em future chronological topology}
            (see \cite[Section 3.3]{FHSIso2}. The limit operatior for this topology, also denoted by $\hat{L}$, is defined as follows: 
            \begin{equation}
              \label{eq:22}
              f\in \hat{L}(\{f_n\}_n) \iff \left\{
                \begin{array}{l}
                  (a)\;\; f\leq {\liminf}_n f_n \hbox{ and}\\
                  (b)\;\; \forall g\in B(M) \hbox{ with $f\leq g\leq \limsup_n f_n,$ it is $g=f$.}
                \end{array}\right.
            \end{equation}

%             \ncambios{Sin embargo, y dado que no estamos considerando las mismas condiciones integrales que en ..., algunas consideraciones deben ser hechas. Por ejemplo, es claro considerando ... que si $b_{c}\subset \liminf_n(\{b_{c_n}_n\})$ entonces $P(b_{c})\subset {\rm LI}(\{P(b_{c_n})\})$, sin embargo la otra implicación precisa de algunas observaciones.

%               La principal dificultad de la otra implicación reside en el siguiente hecho: tomemos $r\in \R$ y $x_0\in M$ con $r<b_{c}(x_0)$. En el caso que exista $t^0$ tal que
%               \[
% r=\int_\C^{t^0}\frac{1}{\sqrt{\alpha(s)}}ds
%               \]
%             entonces deducimos que $(t^0,x^0)\in P(b_{c})$ (recall ...). Sin embargo la existencia de dicho $t^0$ no está asegurada para cualquier valor de $r$. Los casos donde no existe dicho $t^0$, no es posible moverse hacia el correspondiente modelo de Robertson-Walker, que es donde tenemos la hipótesis. Para solventar esto, la prueba puede dividirse en dos pasos: primero, vamos a tomar $x_0=c(t^1)$ para un cierto $t^1$. En estos casos, al saber que $(t^0,c(t^0))\in P(b_{c})$ tenemos que necesariamente

%             \[
% \int_\C^{t^1}\frac{1}{\sqrt{\alpha(s)}}ds<b_c(c(t^1))
%               \]
%               En estos casos, el estudio para los valores $r$ tales que $r<b_{c}(c(t^1))$ puede restringirse a aquellos con $r>\int_{\C}^{t^0}1/\sqrt{\alpha(s)}ds$ y, dada la condición ...., siempre va a existir $t^0$ como en .... De este modo el par $(t^0,c(t^1))\in P(b_{c})$, pudiendo aplicar la hipótesis y .... para determinar que $r<b_{c_n}(c(t^1))$ para $n$ suficientemente grande.

%               Cuando ahora consideramos $x^0$ general, de la desigualdad $r<b_{c}(x^0)$ podemos encontrar un valor $t^0$ suficientemente grande de modo que
%               \begin{equation}
%                 \label{eq:49}
%                 r< \int_{\C}^{t^0}\frac{1}{\sqrt{\alpha(s)}}ds-d(x^0,c(t^0))
%               \end{equation}
%               Por otro lado, tenemos que $(t^0,c(t^0))\in P(b_{c})$, por lo que
%               \[
% \int_{\C}^{t^0}\frac{1}{\sqrt{\alpha(s)}}ds<b_{c}(c(t^0))
%                 \]

%                 Now, by using previous case it follows that

%                 \[
% \int_{\C}^{t^0}\frac{1}{\sqrt{\alpha(s)}}ds< b_{c_n}(c(t^0))
%                   \]
% for $n$ big enough. By using \eqref{eq:49} on previous inequality (plus the triangular inequality) follows that:
%                 \[
% r<b_{c_n}(c(t^0))+d(x^0,c(t^0))\leq b_{c_n}(x^0)  \right)
%                   \]
%                   as desired.}

%                   \smallskip

            The following result establishes the relation between this topology and the pointwise topology on $B(M)$ (see \cite[Prop. 3.2]{FHSIso2} and \cite[Prop. 5.29]{FHSBuseman}):
            \begin{prop}\label{prel:PropToponefibre}
              Consider $\{f_n\}_n\subset B(M)$ a sequence which converges pointwise to a function $f\in B(M)$. Then, $f$ is the unique future chronological limit of $\{f_n\}_n$. In particular, if $\{f_n\}_n=\{d_{(\Omega^n,x^n)}\}_n$ with $\Omega^n\rightarrow \Omega$ and $x^n\rightarrow x (\in M^C)$, then $f=d_{(\Omega,x)}\in \hat{L}(\{f_n\}_n)$ is the unique future chronological limit of $\{f_n\}_n$.

              Moreover, if $M^C$ is locally compact, then the following converse follows: if $f=d_{(\Omega,x)}\in \hat{L}(\{f_n\}_n)$, then for $n$ big enough $\{f_n\}_n=\{d_{(\Omega^n,x^n)}\}_n$ for some $\Omega^n\in \R$ and $x^n\in M^C$ satisfying that $\Omega^n\rightarrow \Omega$ and $x^n\rightarrow x (\in M^C)$.
            \end{prop}

            \smallskip

            Finally, note that the study of the past c-completion is very similar, just with some minor changes. First, the following integral condition is imposed:
              \[
\int_{a}^\C \frac{1}{\sqrt{\alpha(s)}}ds=\infty.
                \]
              Given a past-directed timelike curve $\gamma:[\omega,-\Omega)\rightarrow V$, $\gamma(t)=(-t,c(t))$, then $I^+(\gamma)=F(-b^-_{c})$ where
            \[
F(f):=\{ (t^o,x^o)\in V:\int_{\C}^{t^o}\frac{1}{\sqrt{\alpha(s)}}ds>f(x^o)\}.
              \]
              Moreover, the backward Busemann functions are written now as
              \[
              b^-_{c}(\cdot):=\lim_{t\rightarrow -\Omega}\left(\int^{\C}_{-t}\frac{1}{\sqrt{\alpha(s)}}ds -d(\cdot,c(t))\right). 
              \]
              The space of finite backward Busemann functions coincide with $B(M)$, so there is a natural bijection between the future and past c-completions. Moreover,  when $a<-\Omega$, then $c(t)\rightarrow x^*\in M^C$ and the backward Busemann function becomes (compare with \eqref{eq:46})
              \begin{equation}
                \label{eq:48}
                b^-_{c}(\cdot)=d^-_{(\Omega,x^*)}(\cdot)=\int^{\C}_{\Omega}\frac{1}{\sqrt{\alpha(s)}}ds-d(\cdot,x^*).
              \end{equation}
In conclusion, one deduces
\[
\check{V}\setminus \check{\partial}^{a}V\equiv (a,b) \times M^C,
  \]
  and then,
  \[\begin{array}{rl}
      \check{V} \equiv & B(M)\cup \{-\infty\}   \\
         \equiv  & \left( (a,b) \times M^C  \right) \cup \left({\cal B}(M)\cup \{-\infty\} \right) \\ \equiv & \left( (a,b) \times M^C  \right) \cup \left(\R\times \partial_{\cal B}(M)\right) \cup \{-\infty\}.
  \end{array}
  \]




%%%Local Variables:
%%% mode: latex
%%% TeX-master: "DoublyWarpedBoundary2017"
%%% End:
