\section{The future c-completion of doubly warped spacetimes}
\label{sec:futurecompletion}
 In this section we are going to study the point set and topological structure of the future c-completion of doubly warped spacetimes.

 Let $\gamma: [\omega,\Omega) \rightarrow V$, $\Omega\leq b$ be a future-directed timelike
curve in $V$. We can reparametrize this curve by using the standard parameter $t$ for the temporal component,
$\gamma(t)=(t,c_{1}(t),c_{2}(t))$. So, from \eqref{eq:3},
\begin{equation}
  \label{eq:4}
  \hbox{length}(c_{i}\mid_{[\omega,\Omega)})
  %=\int_{\alpha}^{\Omega}\sqrt{\mu_{i}\circ
%s}\cdot\alpha_{i}^{-1}\left(-(D\circ s)+\frac{\mu_{1}\circ
%s}{\alpha_{1}}+\frac{\mu_{2}\circ
%s}{\alpha_{2}}\right)^{-1/2}ds
  \leq\int_{\omega}^{\Omega}\frac{ds}{\sqrt{\alpha_{i}(s)}}.
\end{equation}
Next, assume that
%In order to determine $I^-(\gamma)$, there are two main cases to discuss, depending on the finite or infinite value of $\Omega$. In the first case this study can be developed independently of the warping functions. However, in the second case, these functions will play an essential role, determining in part the structure of the future boundary.
%First, note that from \eqref{eq:3}, and for any $\Omega$ (perhaps infinity),
%\begin{equation}
%  \label{eq:4}
%  \hbox{length}(x_{i}\mid_{[0,\Omega)})=\int_{\alpha}^{\Omega}\sqrt{c_{i}\circ
%s}\cdot\alpha_{i}^{-1}\left(-(D\circ s)+\frac{c_{1}\circ
%s}{\alpha_{1}}+\frac{c_{2}\circç
%s}{\alpha_{2}}\right)^{-1/2}\leq\int_{\alpha}^{\Omega}\frac{1}{\sqrt{\alpha_{i}(s)}}ds,
%\end{equation}
%First, let us assume that
$\Omega<b$. Then, the integral in (\ref{eq:4}) is finite. Hence, $\hbox{length}(c_i)<\infty$, and so, $c_{i}(t)\rightarrow x_i^*$ for some $x_i^*\in M_i^C$, where $M_i^C$ denotes the Cauchy completion of the Riemannian manifold $(M_i,g_i)$, $i=1,2$. If, in addition, $x_i^*\in M_i$ for $i=1,2$, the past of $\gamma$ is clearly determined by the triple $(\Omega,x_1^*,x_2^*)$. The following result shows that this is also true if $x_i^*$ belongs to the Cauchy boundary $\partial^C M_i$ for some $i=1,2$.

\begin{prop}\label{pastofcurve}
  Let $\gamma:[\omega,\Omega)\rightarrow V$, $\Omega<b$, be a future-directed timelike curve with $\gamma(t)=(t,c_1(t),c_2(t))$. Then, $\gamma(t)\rightarrow (\Omega, x_1^*,x_2^*)\in (a,b)\times M_1^C\times M_2^C$ for some $(x_1^*,x_2^*)\in M_1^C\times M_2^C$. Moreover, $(t^o,x_1^o,x_2^o)\in I^-(\gamma)$ if, and only if, there exist constants $\mu_{1},\mu_{2}>0$ with $\mu_{1}+\mu_{2}=1$ and
%(with
%$\mu'_{1}+\cdots +\mu'_{n}=1$)
such that
\begin{equation}
  \label{eq:5}
\Integral{t^{o}}{\Omega}{\mu_{i}}{i}{\mu_{k}} >
d_{i}(x^{o}_{i},x^{*}_{i})\qquad\hbox{for $i=1,2$.}
\end{equation}
%In particular, the past of $\gamma$ is determined by the triple $(\Omega,x_1^*,x_2^*)$.
\end{prop}

\begin{proof}
As argued above, the first assertion is a direct consequence of \eqref{eq:4}. So, we only need to focus on the last assertion.

For the implication to the right, assume that $(t^o,x_1^o,x_2^o)\in I^-(\gamma)$. Since the chronological past $I^-(\gamma)$ is an open set, we can take $\epsilon >0$ small enough so that $(t^o+\epsilon,x_1^o,x_2^o)\in I^-(\gamma)$. Consider an increasing sequence $\{t_n\}\subset [\omega,\Omega)$ with $t_n\nearrow \Omega$ and $(t^o+\epsilon,x_1^o,x_2^o)\ll \gamma(t_n)$ for all $n$. For each $n$, Thm. \ref{c0} ensures the existence of constants $\mu_1^n,\mu_2^n>0$, with $\mu^n_1+\mu^n_2=1$, such that:
  \begin{equation}\label{eq:6}
\Integral{t^{o}+\epsilon}{t_n}{\mu^n_{i}}{i}{\mu^n_{k}} >
d_{i}(x^{o}_{i},c_{i}(t_n))\qquad\hbox{for $i=1,2$.}
    \end{equation}
    Observe that $\{c_i(t_n)\}_n\rightarrow x_i^*\in M_i^C$ for $i=1,2$, and so, from the continuity of the distance function $d_i(x_i^o,\cdot)$ on $M_i^C$, necessarily $\{d_i(x_i^o,x_i(t_n))\}_n\rightarrow d_i(x_i^o,x_i^*)$. Even more, since $\{\mu^n_i\}_n\subset [0,1]$, we can assume that $\{\mu^n_i\}_n$ converges (up to a subsequence) to, say, $\mu_i^*$, $i=1,2$, with $\mu_1^*+\mu_2^*=1$. Hence,
    \[
\left\{\frac{\sqrt{\mu^n_{i}}}{\alpha_{i}(s)}\left(\sum_{k=1}^2 \frac{\mu_{k}^n}{\alpha_{k}(s)}\right)^{-1/2} \right\}_n\rightarrow \frac{\sqrt{\mu^*_{i}}}{\alpha_{i}(s)}\left(\sum_{k=1}^2 \frac{\mu_{k}^*}{\alpha_{k}(s)}\right)^{-1/2}\quad\hbox{pointwise on $[t^o,\Omega]$.}
    \]
    Arguing as in the proof of Thm. \ref{causi}, we observe that these functions are bounded by the integrable function $g:[t^o,\Omega]\rightarrow\R$, $g(t)=\alpha_i(t)^{-1/2}$, so the Dominated Convergence Theorem ensures that
    %Recalling now that all previous functions are bounded by the (Lebesgue) integrable function $g:[t^o,\Omega]\rightarrow \R$,
    %$g(t)=\frac{1}{\sqrt{\alpha_i(t)}}$, the Dominated Convergence Theorem\footnote{JONY: Este argumento aparece dos veces...} ensures that:
   \[
\left\{\Integral{t^{o}+\epsilon}{t_n}{\mu^n_{i}}{i}{\mu^n_{k}}\right\}_n\rightarrow \Integral{t^{o}+\epsilon}{\Omega}{\mu^*_{i}}{i}{\mu^*_{k}}.
    \]
    In conclusion, by taking limits in \eqref{eq:6}, we arrive to
    \[
\Integral{t^{o}+\epsilon}{\Omega}{\mu^*_{i}}{i}{\mu^*_{k}}\geq d_i(x_i^o,x_i^*)\qquad\hbox{for $i=1,2$.}
      \]
      In order to conclude the implication, it rests to show that, if $t^o+\epsilon$ is replaced by $t^o$, all previous inequalities are strict. In principle, the only way to avoid this conclusion is by assuming that some $\mu_i^*$ is equal to zero. If, say, $\mu_1^*=0$ (and so, $\mu_2^*=1$), then (i) $d_1(x_1^o,x_1^*)=0$ and (ii)
      \[
\int_{t^o}^{\Omega}\frac{1}{\sqrt{\alpha_2(s)}}ds>d_2(x_2^o,x_2^*).
        \]
        Reasoning as in the proof of Prop. \ref{c0}, a small modification of $\mu_1^*,\mu_2^*$ provides new constants $\mu_1,\mu_2>0$, with $\mu_1+\mu_2=1$, such that
\[
          \left\{\begin{array}{ll}\displaystyle
\Integral{t^{o}}{\Omega}{\mu_{1}}{i}{\mu_{k}}>0= d_1(x_1^o,x_1^*)\\
                   \displaystyle\Integral{t^{o}}{\Omega}{\mu_{2}}{i}{\mu_{k}}> d_i(x_2^o,x_2^*),

            \end{array}
          \right.
          \]
          and we are done.

          \smallskip

          For the converse, assume that \eqref{eq:5} holds for some $(t^o,x_1^o,x_2^o)$ and some constants $\mu_1,\mu_2>0$, with $\mu_1+\mu_2=1$, and let us prove that $(t^o,x^o_1,x^o_2)\in I^-(\gamma)$. Recalling that the inequalities in \eqref{eq:5} are strict and $\gamma(t)=(t,c_1(t),c_2(t))\rightarrow (\Omega,x_1^*,x_2^*)$, there exists some $t^e\in (a,b)$ big enough such that
 \[\Integral{t^{o}}{t^e}{\mu_{i}}{i}{\mu_{k}} >
d_{i}(x^{o}_{i},c_{i}(t^e))\qquad\hbox{for $i=1,2$.}
\]
Hence, from Prop. \ref{c0}, $(t^o,x^o_1,x^o_2)\ll \gamma(t^e)$, as required.

%For the final assertion, recall that...\footnote{Jony: En este último repaso me he dado cuenta que se me olvidó meter aquí la prueba de la ultima frase. Observar que, con lo anterior, queda claro que dos curvas temporales con la tripla $(\Omega,x_1^*,x_2^*)$ determinan el mismo pasado, teniendo que probar que si tienen triplas diferentes generan el pasados diferentes. Si el primer elemento (la parte temporal) es diferente, es muy fácil de probar. En el caso que sea uno de los otros dos, se razona como en la prueba de  }

\end{proof}

We have just proved that the chronological past of a future-directed timelike curve $\gamma$ defined on a finite interval $[\omega,\Omega)$, $\Omega<b$, is determined by its future limit point $(\Omega,x^*_1,x^*_2)$, in the sense that any other future-directed timelike curve $\gamma'$ with the same future limit point has the same chronological past. Next, we are going to prove that if $\gamma'$ is another future-directed timelike curve converging to another triple, then it generates a different past.

\begin{prop}\label{structuraparcialsininfinito}
  Let $(V,g)$ be a {\multiwarped} spacetime as in (\ref{eq:1-aux}). If $\gamma^i:[\omega^i,\Omega^i)\rightarrow V$, $i=1,2$ satisfy $\gamma^i(t)\rightarrow p_i:=(\Omega^i,x_1^i,x_2^i)\in (a,b)\times M_1^C\times M_2^C$ with $p_1\neq p_2$, then $I^-(\gamma^1)\neq I^-(\gamma^2)$.
\end{prop}
\begin{proof}
    The conclusion easily follows if, say, $\Omega^1<\Omega^2$, since in this case $\gamma^2(t)\in I^-(\gamma^2)\setminus I^-(\gamma^1)$ whenever $\Omega^1<t<\Omega^2$. So, we will assume that $\Omega^1=\Omega^2(=:\Omega)$ and, say, $d_1(x_1^1,x_1^2)>0$. Let $t^o$ be close enough to $\Omega<\infty$ so that (recall that $c_1^1(t)\rightarrow x_1^1$)
 \[
\int_{t^o}^{\Omega}\frac{1}{\sqrt{\alpha_1(s)}}ds<\frac{d_1(x_1^1,x_1^2)}{3}\quad\hbox{and}\quad d_1(c_1^1(t^o),x_1^2)>\frac{d_1(x_1^1,x_1^2)}{3},
    \]
and define $q=\gamma^1(t^o)\in I^-(\gamma^1)$.   % $(x^o_1,x_2^o)\in M$ such that $q=(t^o,x^o_1,x_2^o)\in I^-(\gamma_1)$ and $d_1(x^o_1,x^2_1)>d_1(x^1_1,x^2_1)/3$.
    Then, $q\not\in I^-(\gamma^2)$, since, otherwise, from Prop. \ref{pastofcurve},
    \[
\Integral{t^{o}}{\Omega}{\mu_{1}}{1}{\mu_{k}} >
d_{1}(c^{1}_{1}(t^o),x^{2}_{1})\quad\hbox{for some $\mu_1,\mu_2>0$}.
      \]
      But this is not possible since, from the choice of $t^o$,
      \[
d_1(c^1_1(t^o),x^2_1)>\frac{d_1(x^1_1,x^2_1)}{3}>\int_{t^o}^{\Omega}\frac{1}{\sqrt{\alpha_1(s)}}ds>\Integral{t^{o}}{\Omega}{\mu'_{1}}{1}{\mu'_{k}}      \]
   for any positive constants $\mu'_1,\mu'_2$, with $\mu'_1+\mu'_2=1$. In conclusion, $I^-(\gamma^1)\neq I^-(\gamma^2)$ if $p_1\neq p_2$, and the conclusion follows.
\end{proof}
\begin{rem}\label{rem:1} In the proof of previous result the key property is the finite value of the integral $\int_{t^o}^{\Omega}\alpha_i(s)^{-1/2}ds<\infty$, not the finite value of $\Omega$. Of course, the second imply the first, but the same argument can be reproduced if only the first holds.
%
%In previous result condition $\int_{t^o}^{\Omega}\alpha_i(s)^{-1/2}ds<\infty$ has been used in two different ways:
%(i) to ensure that $c_i(t)$ converges to some point $x_i^*\in M_i^C$, $i=1,2$; (ii) to show that any couple of future-directed timelike curves $\gamma^1,\gamma^2:[\alpha,\Omega)\rightarrow V$ whose $i$-components converge to different points in $M_i^C$ have different past.
  \end{rem}
\noindent Props. \ref{pastofcurve} and \ref{structuraparcialsininfinito} together establish a natural bijection between the space $(a,b) \times M_1^C\times M_2^C$ and the set $\hat{V}\setminus \hat{\partial}^{\ncambios{\B}} V$, where $\hat{\partial}^{\ncambios{\B}} V$ denotes the set of TIPs determined by future-directed timelike curves with divergent temporal component ($\Omega=b$). More precisely:

\begin{prop}\label{structuraparcialsininfinito'}
  Let $(V,g)$ be a {\multiwarped} spacetime as in (\ref{eq:1-aux}). Then, there exists a bijection
  \begin{equation}\label{v}
\hat{V}\setminus \hat{\partial}^{\ncambios{\B}} V\; \leftrightarrow\; (a,b)\times M_1^C\times M_2^C,
    \end{equation}
    which maps each indecomposable past set $P\in\hat{V}\setminus \hat{\partial}^{\ncambios{\B}}V$ to the limit point $(\Omega,x^*_1,x^*_2)\in (a,b)\times M_1^C\times M_2^C$ of any future-directed timelike curve generating $P$.
    \end{prop}

% \begin{rem}\label{rem:2}
% For future references,  observe that, in the last part of the proof, we have shown that two future-directed timelike curves $\gamma^j:[\alpha,\Omega)\rightarrow V$ with $x^j_i(t)\rightarrow x_i^j\in M_i^C$ and satisfying that $x_i^1\neq x_i^2$ for some $i=1,2$ have different pasts. Moreover, it is remarkable that such a proof only relies on the fact that $\int_{0}^{\Omega}\frac{1}{\sqrt{\alpha_i(s)}}ds<\infty$.\footnote{MEJORAR ESTE TEXTO!!}
% \end{rem}

\smallskip

Next, we are going to extend the point set structure obtained above to a topological level. We will consider $(a,b)\times M_1^C\times M_2^C$ attached with the product topology. The first result shows the continuity of bijection (\ref{v}) in the left direction:

\begin{prop}\label{prop:topbuenadir}
Let $P_n,P\in \hat{V}\setminus \hat{\partial}^{\ncambios{\B}}V$ with $P_n\equiv (\Omega_n,x_1^n,x_2^n)$ and $P\equiv (\Omega,x_1^*,x_2^*)$, where we are assuming that the triplets belong to $(a,b)\times M_1^C\times M_2^C$. If $(\Omega_n,x_1^n,x_2^n)\rightarrow (\Omega,x_1^*,x_2^*)$, then $P\in \hat{L}(\{P_n\}_n)$.
\end{prop}
\begin{proof}
   First, recall the analytic characterization of the IPs $P$ and $P_n$ provided by Prop. \ref{pastofcurve}: a point $(t,x_1,x_2)\in V$ belongs to $ P$ (resp. $P_n$) if, and only if, there exist positive constants $\mu_1,\mu_2$ ($\mu_1^n,\mu_2^n$) with $\mu_1+\mu_2=1$ ($\mu_1^n+\mu_2^n=1$) and satisfying that, for $i=1,2$:
  \[
    \begin{array}{c}
      \displaystyle\Integral{t}{\Omega}{\mu_{i}}{i}{\mu_{k}}>d_i(x_i,x_i^*) \\
   \left( \displaystyle  \Integral{t}{\Omega_n}{\mu^n_{i}}{i}{\mu^n_{k}}>d_i(x_i,x_i^n)  \right)
    \end{array}
    \]
Second, note that, from the hypotheses, the continuity of the distance map, and the Dominated Convergence Theorem, the following two limits hold: $d_i(x_i,x_i^n)\rightarrow d_i(x_i,x_i^*)$ and
    \[
\Integral{t}{\Omega_n}{\mu_{i}}{i}{\mu_{k}}\rightarrow \Integral{t}{\Omega}{\mu_{i}}{i}{\mu_{k}}\quad\hbox{for any $\mu_1,\mu_2>0$.}
    \]
These two properties directly imply both, $P\subset {\rm LI}(\{P_n\}))$ and $P$ is maximal into ${\rm LS}(\{P_n\})$, i.e., $P\in \hat{L}(\{P_n\})$.
 \end{proof}

In order to prove the continuity of bijection (\ref{v}) in the right direction, we need to impose local compactness on the Cauchy completion, since, otherwise, there exist counterexamples (see, for instance, \cite[Example 4.9]{FHSBuseman})

\begin{prop}\label{topcurvasfinitas}
   Let $(V,g)$ be a doubly warped spacetime as in (\ref{eq:1-aux}) with $M_1^C$ and $ M_2^C$ locally compact. If $\{P_n\}_n$ is a sequence of IPs converging to some IP, $P\equiv (\Omega,x_1^*,x_2^*)\in (a,b) \times M_1^C\times M_2^C$, then $P_n\equiv (\Omega^n,x_1^n,x_2^n)\in (a,b) \times M_1^C\times M_2^C$ for $n$ big enough, and $(\Omega^n,x_1^n,x_2^n)\rightarrow (\Omega,x_1^*,x_2^*)$ with the product topology. As consequence, the bijection (\ref{v}) becomes a homeomorphism.
\end{prop}

\begin{proof}
 The proof follows essentially in the same fashion as \cite[Prop. 5.24]{FHSBuseman}.

  Since the Cauchy completion $M_1^C\times M_2^C$ is locally compact, there exists a pre-compact neighbourhood $U$ of $P\equiv(\Omega,x_1^*,x_2^*)$. Let $\{p^n_m\}_m,\{p_m\}_m\subset V$ be future chains generating $P_n$ and $P$, resp. We can assume without restriction that $\{p_m\}\subset U$.
  It suffices to show the existence of $n_0$ and a map $\mathfrak{m}:\N\rightarrow\N$ such that $p_m^n\in U$ for all $n\geq n_0$ and $m\geq \mathfrak{m}(n)$. In fact, in this case, the temporal component of the sequence $\{p^n_m\}_m$ will not diverge as $m\rightarrow\infty$, and so, $P_n$ can be identified with some $(\Omega^n,x_1^n,x_2^n)\in (a,b) \times M_1^C\times M_2^C\cap\overline{U}$. Moreover, since the result is valid for any pre-compact open set $U$, and $(\Omega,x_1^*,x_2^*)$ admits a countable local neighbourhood basis given by pre-compact open sets, necessarily $(\Omega^n,x_1^n,x_2^n)\rightarrow (\Omega_1^*,x_1^*,x_2^*)$.

In order to prove the statement in previous paragraph, assume by contradiction that, up to subsequences, $p^n_m$ is not contained in $U$ for all $m$ and $n$. Since $P\subset {\mathrm LI}(\{P_n\}_n)$, for each $m\in \N$ there exists $n_0$ such that $p_m\in P_n$ for all $n\geq n_0$. Consider a strictly increasing sequence $\{\mathfrak{n}(m)\}_m$ such that $p_m\in P_{\mathfrak{n}(m)}$. Denote by  $\gamma_m$ the future-directed timelike curve from $p_m$ to some point of a future chain generating $P_{\mathfrak{n}(m)}$. Each $\gamma_m $ intersects the boundary of $\overline{U}$ at a point, say, $(s^m,y_1^m,y_2^m)$. Since $U$ is pre-compact, its boundary is compact and we can assume (up to a subsequence) that $(s^m,y_1^m,y_2^m)\rightarrow (s^*,y^*_1,y^*_2)$ for some $(s^*,y^*_1,y^*_2)\in (a,b) \times M_1^C\times M_2^C$. Let us denote by $P'$ the indecomposable set associated to $q=(s^*,y^*_1,y^*_2)$ which is necesssarily different from $P$ (as $q$ belong to the boundary of $U$); and by $\{q_m\}_m$ a future chain generating $P'$. Next, we are going to show that $P'$ breaks the maximality of $P$ into ${\mathrm LS}(\{P_n\})$, in contradiction with $P\in \hat{L}(\{P_n\}_n)$.

Let us show that $P'\subset {\mathrm LS}(\{P_n\})$. First recall that,  for each $m\in \N$, the set $I^+(q_m)$ is an open set containing $q$:
% \footnote{Esto es evidente si $q\in M$, pero no lo parece tanto cuando $q\in \overline{M}$. ¿Incluimos un lemma auxiliar?}.
in fact, this is straightforward if $q\in (a,b)\times M_1\times M_2$; otherwise, it suffices to realize that the characterization of the chronological relation given in Prop. \ref{c0} (which is an open property) extends to the set $(a,b)\times M_1^C\times M_2^C$ (see Prop. \ref{pastofcurve}).
In particular, since $\{(s^k,y_1^k,y_2^k)\}\rightarrow q$, it follows that $(s^k,y_1^k,y_2^k)\in I^+(q_m)$ for $k$ big enough. But, from construction, $(s^k,y_1^k,y_2^k)\in P_{\mathfrak{n}(k)}$, so $q_m\in P_{\mathfrak{n}(k)}$ for $k$ big enough. Therefore, $P'\subset {\mathrm LS}(\{P_{\mathfrak{n}(m)}\}_m)$.

It rests to show that $P\subsetneq P'$; that is, any point $p_m$ of the future chain generating $P$ is contained in $P'$. From construction, $p_m=(t^m,x_1^m,x_2^m)\ll p_k \ll (s^k,y_1^k,y_2^k)$ for all $k> m$. Let $\epsilon>0$ be small enough so that $p_m^\epsilon=(t^m+\epsilon,x_1^m,x_2^m)\ll p_{m+1}$, and thus, $p_m^\epsilon\ll (s^k,y_1^k,y_2^k)$ for all $k>m$. From Prop. \ref{c0}, there exist positive constants $\mu_1^k$ and $\mu_2^k$, with $\mu_1^k+\mu_2^k=1$, such that:
\[
\int_{t^m+\epsilon}^{s^k} \frac{\sqrt{\mu^k_i}}{\alpha_{i}(s)}\left(\sum_{l=1}^{2} \frac{\mu^k_l}{\alpha_{k}(s)} \right)^{-1/2}ds>
d_{i}(x^{m}_{i},y^{k}_{i})\qquad\hbox{for $i=1,2$.}
  \]
 But $\{(s^k,y^k_1,y^k_2)\}\rightarrow (s^*,y_1^*,y_2^*)$. By continuity, and up to a subsequence, there exist positive constants $\mu_1^*,\mu_2^*$, with $\mu_1^*+\mu_2^*=1$, such that:
  \[
\int_{t^m+\epsilon}^{s^*} \frac{\sqrt{\mu^*_i}}{\alpha_{i}(s)}\left(\sum_{l=1}^{2} \frac{\mu^*_l}{\alpha_{k}(s)} \right)^{-1/2}ds \geq
d_{i}(x^{m}_{i},y^{*}_{i})\qquad\hbox{for $i=1,2$.}
    \]
    Now if we replace in previous expression $t^m+\epsilon$ by $t^m$, at least one of previous inequalities becomes strict. Then, reasoning as in the proof of Prop. \ref{c0}, we arrive to

      \[
\int_{t^m}^{s^*} \frac{\sqrt{\mu'_i}}{\alpha_{i}(s)}\left(\sum_{l=1}^{2} \frac{\mu'_l}{\alpha_{k}(s)} \right)^{-1/2}ds >
d_{i}(x^{m}_{i},y^{*}_{i})\qquad\hbox{for $i=1,2$,}
      \]
      for some slightly modified constants $\mu'_i$ from $\mu^*_i$. Therefore, the point $p_m$ belongs to $P'$ (recall Prop. \ref{pastofcurve}). Since this argument works for any point of the sequence $\{p_m\}_m$ generating $P$, the inclusion $P\subsetneq P'$ follows.

    \smallskip

For the last assertion, observe that previous argument gives the continuity of bijection \eqref{v} to the right direction, while Prop. \ref{prop:topbuenadir} ensures the continuity to the left one.

  \end{proof}

 Next, we analyze the case $\Omega=b$. In this case, the finiteness/infiniteness of the integrals associated to the warping functions becomes crucial, so we will consider several subsections to discuss it.
%the past of such a curves (and so, the structure of $\hat{\partial}^{\infty}V$) under some simple conditions on such warping functions.

\subsection{Finite warping integrals}

First, we consider the case when the integrals associated to the warping functions are both finite:
% \cambios{The results will be given in full generality, so we will consider doubly warped models as in \eqref{eq:1-aux}. Then, the integral conditions on the warping functions should read as:}
\begin{equation}
  \label{eq:7}
  \int_{\C}^{b}\frac{1}{\sqrt{\alpha_i(s)}}ds<\infty, \qquad \hbox{$i=1,2$}\quad\hbox{for some $\C\in (a,b)$.}
\end{equation}
In this case, the following result provides the point set and topological structure of the future c-boundary:
\begin{thm}\label{futurestructurefiniteconditions}
  Let $(V,g)$ be a {\multiwarped} spacetime as in (\ref{eq:1-aux}), and assume that the integral conditions \eqref{eq:7} hold. Then, there exists a bijection
  \begin{equation}
    \label{eq:8}
    \hat{V}\; \leftrightarrow \; (a,b]\times M_1^C\times M_2^C
  \end{equation}
  which maps each IP $P\in \hat{V}$ to the limit point $(\Omega,x_1,x_2)\in (a,b]\times M_1^C\times M_2^C$ of any future-directed timelike curve generating $P$. Moreover, if $M_1^C,M_2^C$ are locally compact, this bijection becomes an homeomorphism.
\end{thm}
\begin{proof}
%\cambios{As we recall in Rem. \ref{rem:infinito}, we can consider directly that $(a,b)\equiv \R$}.  Let us begin with the point set structure.
For the first assertion, we only need to prove the corresponding bijection between $\hat{\partial}^{\ncambios{\B}} V$ and $\{b\}\times M_1^C\times M_2^C$ (recall Prop. \ref{structuraparcialsininfinito'}). But this follows from the same arguments as in the proofs of Props. \ref{pastofcurve} and \ref{structuraparcialsininfinito} (recall \eqref{eq:7} and Remark \ref{rem:1}).
%since the key point in those arguments was the finiteness of the corresponding warping functions \eqref{eq:7}.
%
%
%with $\Omega$ replaced by $\infty$, since
%
%
%and taking into account the warping integral hypothesis \eqref{eq:7} (see Remark \ref{rem:1}).

For the second assertion, the continuity to the left of bijection (\ref{eq:8}) follows as in Prop. \ref{prop:topbuenadir}, just taking into account that the integral condition \eqref{eq:7} must be used in order to apply the Dominated Convergence Theorem. For the continuity to the right, assume that $P\in \hat{L}(\{P_n\}_n)$, with $P=I^-(\gamma)$, $P_n=I^-(\gamma_n)$, and being $\gamma:[\omega,\Omega)\rightarrow V$, $\gamma_n:[\omega_n,\Omega_n)\rightarrow V$ future-directed timelike curves. Let $(\Omega,x_1^*,x_2^*)$ and $(\Omega_n,x_1^n,x_2^n)$ be the limit points in $(a,b]\times M_1^C\times M_2^C$ of $\gamma$ and $\gamma_n$, resp. We need to prove that $(\Omega_n,x_1^n,x_2^n)\rightarrow (\Omega,x_1^*,x_2^*)$. Observe that, if $\Omega<b$, then the result follows from Prop. \ref{topcurvasfinitas}, so we will focus on the case $\Omega=b$.

First, note that $\Omega_n\rightarrow b$. In fact, otherwise, there exists $\Omega^*<b$ and a subsequence $\{\Omega_{n_k}\}$ such that $\Omega_{n_k}<\Omega^*$ for all $k\in \N$. But, in this case, $P_{n_k}$ will not contain any point $\gamma(t)$ with $t>\Omega^*$, and so, $P\not\subset  {\mathrm LI}(\{P_n\})$.

Assume by contradiction that, say, $\{x_1^n\}_n$ does not converge to $ x_1^*$. Then, up to a subsequence, there exists $\epsilon_0>0$ such that $d_1(x_1^n,x_1^*)>\epsilon_0$. Take $t^0$ big enough so that

  \[
\int_{t^o}^{b}\frac{1}{\sqrt{\alpha_1(s)}}ds<\frac{\epsilon_0}{3}.
    \]
    Take $(x_1^o,x_2^o)\in M_1\times M_2$ such that $q=(t^o,x_1^o,x_2^o)\in I^-(\gamma)=P$ with $d_1(x_1^o,x_1^*)<\epsilon_0/3$. It suffices to show that $q$ does not belong to $P_n$ for any $n$, since, in this case, we arrive to a contradiction with $P\subset {\mathrm LI}(P_n)$.  So, assume that $q\in P_n$ for all $n$. From  Prop. \ref{pastofcurve}, there exists some $\mu^n_1,\mu^n_2>0$ such that
    \[
\Integral{t^{o}}{\Omega^n}{\mu^n_{1}}{1}{\mu^n_{k}} >
d_{1}(x^{o}_{1},x^{n}_{1}).
      \]
      This is in contradiction with the fact that, for any pair of positive constants $\mu'_1,\mu_2'>0$ with $\mu_1'+\mu'_2=1$,
      \[
d_1(x^o_1,x^n_1)>\frac{2}{3} d_1(x^*_1,x^n_1)> \frac{2}{3}\epsilon_0>\int_{t^o}^{b}\frac{1}{\sqrt{\alpha_1(s)}}ds>\Integral{t^{o}}{b}{\mu'_{1}}{1}{\mu'_{k}}.
        \]
%for any pair of positive constants $\mu'_1,\mu_2'$ with $\mu_1'+\mu'_2=1$.
%Hence, $q$ does not belong to $P_n$ for any $n$, in contradiction with $P\in {\mathrm LI}(P_n)$. In conclusion, $\{x^n_1\}_n\rightarrow x_1^*$, as required.
%\footnote{JONY: Observar que al probar  que la sucesión $\{x_1^n\}$ converge a $x_1^*$, se está usando el mismo argumento que en la prueba de la Prop. \ref{structuraparcialsininfinito}. Probablemente se debería poder extraer el argumento para simplemente llamarlo en los dos lados...}
\end{proof}

\subsection{One infinite warping integral}

Let us consider now the case when just one of the warping integrals is infinite, say:
\begin{equation}
  \label{eq:9}
 \int_{\C}^{b}\frac{1}{\sqrt{\alpha_1(s)}}ds<\infty \qquad \hbox{and}\qquad \int_{\C}^{b}\frac{1}{\sqrt{\alpha_2(s)}}ds=\infty.
\end{equation}

%\cambios{In general, we will assume conditions as
%\begin{equation}
%  \label{eq:9}
% \int_{c}^{b}\frac{1}{\sqrt{\alpha_1(s)}}ds<\infty \qquad \hbox{and}\qquad \int_{c}^{b}\frac{1}{\sqrt{\alpha_2(s)}}ds=\infty,
%\end{equation}
%but, as we recall in Rem. \ref{rem:infinito}, there is no loss of generality if we assume in \eqref{eq:9} that $c=0$ and $b=\infty$.
%}

%Now, if we consider a future-directed timelike curve $\gamma:[\alpha,\infty)\rightarrow V$, $\gamma(t)=(t,c_1(t),c_2(t))$, the curve $c_2$ is not forced to converge to some point of the Cauchy completion $M_2^C$, which will provide a different point set structure for $\hat{\partial}^{\infty}V$.
%
%Let us analyze the possible structure of $\hat{\partial}^{\infty}V$: For each $x_1\in M_1$, the fiber $\R\times \{x_1\}\times M_2$ is isometric to the Generalized Robertson-Walker $(\R\times M_2,\tilde{g}_2)$, where $\tilde{g}_2=-dt^2+\alpha_2(t)g_2$ (from now on denoted by $\R\times_{\alpha_2}M_2$). According to Section \ref{sec:Robertson}, the point set and topological structure of $\hat{\partial}^{\infty}V$ is determined by the corresponding proper Busemann completion ${\cal B}(M_2)$: two different Busemann functions in $M_2$ generate two different indecomposable sets in $\R\times_{\alpha_2} M_2$, and so, two different indecomposable sets in $V$. Moreover, the finite integral conditions ensure that these indecomposable sets in $V$ will depend also on the selected point in $M_1$, that is, the same Buseman function in ${\cal B}(M_2)$ will generate different indecomposable sets in $V$, at least one for each $x_1\in M_1$.
%
%\smallskip
%
%  In the next subsections we will formalize these ideas by considering, not only points of $M_1$, but also of $M_1^C$. Moreover, we will show that the structure of $\hat{\partial}^{\infty}V$ depicted in previous paragraph extends naturally to the topological level, giving a complete characterization of the future c-completion. These results are particularly technical, being necessary to present them gradually. We will focus first on the point set structure.

\subsubsection{Point set structure}

The first integral in condition \eqref{eq:9} plus \eqref{eq:4} ensures that any future-directed timelike curve $\gamma:[\omega,b)\rightarrow V$,  $\gamma(t)=(t,c_1(t),c_2(t))$, satisfies that $c_1(t)\rightarrow x_1^*\in M_1^C$. Moreover, the second integral ensures that the associated Generalized Robertson-Walker spacetime $((a,b) \times M_2,-dt^2+\alpha_2g_2)$ corresponds with the model studied in Section \ref{sec:Robertson}. In particular, since the curve $\sigma(t)=(t,c_2(t))$ is also a future-directed timelike in that spacetime, we can consider the Busemann function $b_{c_2}\in B(M_2)\cup \{\infty\}$.

Next, our aim is to show that the chronological past of $\gamma$ is determined by both, $x_1^*\in M_1^C$ and the Busemann function $b_{c_2}\in B(M_2)\cup \{\infty\}$. Let us begin with the following result:
\begin{prop}\label{prop:conddiferbordedif}
  Let $(V,g)$ be a {\multiwarped} spacetime and assume that the integral conditions \eqref{eq:9} are satisfied. Consider two future-directed timelike curves $\gamma^i:[\omega,b)\rightarrow V$, $\gamma^i(t)=(t,c_1^i(t),c_2^i(t))$, with $c_1^i(t)\rightarrow x_1^i\in M_i^C$, $i=1,2$. If $(x_1^1,b_{c_1})\neq (x_1^2,b_{c_2})$ then $I^-(\gamma^1)\neq I^-(\gamma^2)$.
\end{prop}
\begin{proof} If $x_1^1\neq x_1^2$, we can reason as in the proof of Prop. \ref{structuraparcialsininfinito} (taking $\Omega=\Omega'=b$ and $x_1^1\neq x_1^2$; Remark \ref{rem:1} and the first integral condition in \eqref{eq:9}). So, it suffices to consider the case $b_{c_2^1}\neq b_{c_2^2}$.

Let $\sigma_i(t)=(t,c_2^i(t))$, $i=1,2$, be two future-directed timelike curves on the Generalized Robertson-Walker spacetime $\left( (a,b)\times M_2,-dt^2+\alpha_2 g_2\right)$. Since $b_{c_2^1}\neq b_{c_2^2}$, necessarily $I^{-}(\sigma_1)\neq I^{-}(\sigma_2)$. Assume, for instance, that  $(t^0,y_2)\in I^{-}(\sigma_2)\setminus I^{-}(\sigma_1)$ (the other case is analogous). Then, taking into account the characterization in \eqref{eq:27}, it follows that

  % Assume, for instance, that $b_{c_2^1}(y_2)<b_{c_2^2}(y_2)$ for some $y_2\in M_2$, and let us show that $I^-(\gamma^2)\not\subset I^-(\gamma^1)$. \cambios{NECESITO VER CÓMO METER QUE AMBAS FUNCIONES PUEDEN SER POSITIVAS...}
  % From (\ref{eq:9}) and the continuity of the integral with respect to the superior limit of integration, there exists $t^o$ such that\footnote{J.L.: Aqui parece que se necesita $b_{c_2^i}(y_2)>0$ $i=1,2$.}
  \begin{equation}\label{eq:b}
b_{c_2^1}(y_2)<\int_{\C}^{t^o}\frac{1}{\sqrt{\alpha_2(s)}}ds <b_{c_2^2}(y_2).
\end{equation}
From the first inequality in (\ref{eq:b}),
    \[
      \begin{array}{l}
  (b_{c^1_2}(y_2)=)\lim_{t\rightarrow b} \left(\int_\C^{t}\frac{1}{\sqrt{\alpha_2(s)}}ds-d_2(y_2,c_2^1(t))\right)\leq\int_\C^{t^o}\frac{1}{\sqrt{\alpha_2(s)}}ds\Rightarrow\\ \Rightarrow  \lim_{t\rightarrow b}\left( \int_{t^o}^{t}\frac{1}{\sqrt{\alpha_2(s)}}ds-d_2(y_2,c_2^1(t))\right)\leq 0.
     \end{array}
 \]
 Therefore, since the function $t\mapsto \left(\int_{t^o}^{t}\frac{1}{\sqrt{\alpha_2(s)}}ds-d_2(y_2,c_2^1(t))\right)$ is increasing, we deduce
      \begin{equation}\label{b}
\int_{t^o}^{t}\frac{1}{\sqrt{\alpha_2(s)}}ds<d_2(y_2,c_2^1(t))\quad\hbox{for all $t$.}
        \end{equation}

Let us show the existence of $x_1^o\in M_1$ such that $q=(t^o,x_1^o,y_2)\in I^-(\gamma^2)$. From the inequality
        \[
\int_\C^{t^o}\frac{1}{\sqrt{\alpha_2(s)}}ds <b_{c_2^2}(y_2)=  \lim_{t\rightarrow b}\left(\int_\C^t\frac{1}{\sqrt{\alpha_2(s)}}ds-d_2(y_2,c_2^2(t))\right),
          \]
        there exists $t'>t^o$ big enough such that
        \[
          \int_{t^o}^{t'}\frac{1}{\sqrt{\alpha_2(s)}}ds> d_2(y_2,c_2^2(t')).
        \]
        From continuity, we can find positive constants $\mu_1,\mu_2$, with $\mu_1+\mu_2=1$, such that
        \[
\left\{
  \begin{array}{l}
    \displaystyle\Integral{t^0}{t'}{\mu_2}{2}{\mu_k}>d_2(y_2,c_2^2(t'))\\
    \displaystyle \Integral{t^0}{t'}{\mu_1}{1}{\mu_k}>0.
  \end{array}
\right.
          \]
          So, if we take $x_1^o$ close enough to $c^2_1(t')$ so that
          \[d_1(x_1^o,c^2_1(t'))<\Integral{t^o}{t'}{\mu_1}{1}{\mu_k},\] Prop. \ref{c0} ensures that $(t^o,x_1^o,y_2)\ll \gamma^2(t')$, and thus, $q=(t^o,x_1^o,y_2)\in I^-(\gamma^2)$.

          \smallskip

          On the other hand, for any pair of positive constants $\mu_1,\mu_2>0$ with $\mu_1+\mu_2=1$, necessarily
        \[
\Integral{t^{o}}{t}{\mu_{2}}{2}{\mu_{k}}<\int_{t^o}^t \frac{1}{\sqrt{\alpha_2(s)}}ds< d_2(y_2,c_2^1(t))\quad\hbox{for all $t>t^0$,}
          \]
          where (\ref{b}) has been used in the last inequality. Therefore, from Prop. \ref{c0}, $q\not\ll \gamma^1(t)$ for all $t>t^o$, and thus, $q\not\in I^-(\gamma^1)$.
\end{proof}

  %Proposition \ref{prop:conddiferbordedif} establishes that $\hat{\partial}^\infty V$ contains a set identifiable with the product space $M_1^C\times {\cal B}(M_2)$. Now, we are going to prove that, indeed, there are no additional points in $\hat{\partial}^\infty V$. To this aim, first we will focus on the first spatial component of the curve.

\begin{lemma}\label{lemma:aux3}
Let $\gamma:[\omega,\Omega)\rightarrow V$, $\gamma(t)=(t,c_1(t),c_2(t))$ be a future-directed timelike curve with $c_1(t)\rightarrow x_1^*\in M_1^C$. If $\sigma=\{(t_n,x_1^n,c_2(t_n))\}_n\subset V$ satisfies $\{t_n\}_n\rightarrow \Omega$ and $x_1^n\rightarrow x_1^*$, then $I^-(\gamma)\subset {\mathrm LI}(\{I^-(t_n,x_1^n,c_2(t_n))\}_n)$.
%$P=I^-(\gamma)$ with $\gamma:[\alpha,\Omega)\rightarrow V$, $\gamma(t)=(t,c_1(t),c_2(t))$ and $c_1(t)\rightarrow x_1^*\in M_1^C$. Consider a sequence $\sigma=\{(t_n,x_1^n,c_2(t_n))\}_n\subset V$ with $\{t_n\}_n\rightarrow \Omega$\footnote{Cambiado estrictamente creciente a convergencia, cuidado...} and $x_1^n\rightarrow x_1^*$. Then, $P\subset {\mathrm LI}(\{I^-(t_n,x_1^n,c_2(t_n))\}_n)$.
\end{lemma}
\begin{proof}
Assume by contradiction the existence of some point $q=(t^o,x_1^o,x_2^o)\in I^-(\gamma)$ such that $q\not\ll (t_n,x^n_1,c_2(t_n))$ for infinitely many $n$. From the open character of the chronological relation, we can assume that $x_1^o\neq x_1^*$. Moreover, for $\epsilon>0$ small enough, it follows that $q_{\epsilon}=(t^o+\epsilon,x_1^o,x_2^o)\in I^{-}(\gamma)$.

Assume that, up to a subsequence, $q_{\epsilon}\ll \gamma(t_n)$ for all $n$. From Prop. \ref{c0}, there exist positive constants $\mu_1^n,\mu_2^n>0$, with $\mu_1^n+\mu_2^n=1$, such that
  \[
\Integral{t^{o}+{\epsilon}}{t_n}{\mu^n_{i}}{i}{\mu^n_{k}} >
d_{i}(x^{o}_{i},c_{i}(t_n))\qquad\hbox{for $i=1,2$.}
    \]
We can assume without restriction that $\{\mu_i^n\}_n$ converges to some point $\mu_i^*$, $i=1,2$. Since $q_{\epsilon}\not\in I^-((t_n,x^n_1,c_2(t_n)))$ (recall that $q\not\ll (t_n,x_1^n,c_2(t_n))$), necessarily
   \[
\left(d_{1}(x^{o}_{1},c_{1}(t_n))< \right)\Integral{t^{o}+{\epsilon}}{t_n}{\mu^n_{1}}{1}{\mu^n_{k}}\leq d_1(x_1^o,x^n_1),
      \]
      the last inequality by Prop. \ref{c0}. From the hypothesis, the first and third element in previous expression converge to $d_1(x_1^o,x_1^*)>0$. Moreover, from \eqref{eq:9}, the integral in the middle is also finite. Hence,
      \begin{equation}\label{eq:c}
\left\{\Integral{t^{o}+\epsilon}{t_n}{\mu^n_{1}}{1}{\mu^n_{k}}\right\}_n\rightarrow \Integral{t^{o}+\epsilon}{\Omega}{\mu^*_{1}}{1}{\mu^*_{k}}=d_1(x_1^o,x_1^*)<\infty.
        \end{equation}
        %and thus,
%        \[
%\Integral{t^{o}+\epsilon}{\infty}{\mu^*_{1}}{1}{\mu^*_{k}}=d_1(x_1^o,x_1^*).
%          \]
          In particular, since $x_1^o\neq x_1^*$, necessarily $\mu_1^*\neq 0$, and so,
        \begin{equation}\label{eq:cc}
\Integral{t^{o}}{\Omega}{\mu^*_{1}}{1}{\mu^*_{k}}>d_1(x_1^o,x_1^*).
          \end{equation}
          Finally, from (\ref{eq:c}) and (\ref{eq:cc}),
        \[
\int_{t^o}^{t^n}\frac{\sqrt{\mu_1^n}}{\alpha_1}\left(\sum_{k=1}^{2}\frac{\mu_k^n}{\alpha_k}\right)^{-1/2}ds>d_1(x_1^o,x^n_1)\quad\hbox{for $n$ big enough,}
          \]
which implies that $q=(t^o,x_1^o,x_2^o)\in I^-((t_n,x^n_1,c_2(t_n)))$ for $n$ big enough, a contradiction.
\end{proof}

\noindent This Lemma has the following direct consequence:

\begin{lemma}\label{lemma:aux1}
 Let $\gamma^i:[\omega,b)\rightarrow V$, $\gamma^i(t)=(t,c_1^i(t),c_2(t))$, $i=1,2$, be future-directed timelike curves. If $c_1^i(t)\rightarrow x_1^*\in M^C_1$, $i=1,2$, then $I^-(\gamma^1)=I^-(\gamma^2)$.
\end{lemma}
\begin{proof}
 Let us focus on the inclusion to the right (the other one is analogous). Consider the sequence $\sigma=\{(t_n,c_1^2(t_n),c_2(t_n))\}_n$, where $\{t_n\}_n\nearrow \infty$. For any $p\in I^-(\gamma^1)$, Lemma \ref{lemma:aux3} ensures the existence of $n_0$ such that $p\in I^-((t_n,c_1^2(t_n),c_2(t_n)))\subset I^-(\gamma^2)$ for all $n\geq n_0$, as desired.
\end{proof}

%These results provide some freedom in the choice of the curve $\gamma$ generating the TIP $P$, whenever the first spatial component is converging to an appropriate point. The following result provides a similar property, but now for the second spatial component. The difference is that now this freedom is restricted to the corresponding Busemann function.

\begin{lemma}\label{lemma:aux2}

Let $\gamma^i:[\omega,b)\rightarrow V$, $\gamma^i(t)=(t,c_1(t),c_2^i(t))$, $i=1,2$, be future-directed timelike curves. If $b_{c_2^1}=b_{c_2^2}$, then $I^-(\gamma^1)=I^-(\gamma^2)$.
%Let $\gamma^j:[\alpha,\infty)\rightarrow V$ be curves in $V$ with $\gamma^j(t)=(t,c_1^j(t),c_2^j(t))$ for $j=1,2$. If $c_1^1=c_1^2(=c)$ and $b_{c_2^1}=b_{c_2^2}$, then $I^-(\gamma^1)=I^-(\gamma^2)$.
\end{lemma}
\begin{proof}
Since the first warping integral is finite (recall (\ref{eq:9})), the spatial component $c_1$ admits some limit point $x_1^*\in M_1^C$. Assume by contradiction that, say, $q=(t^o,x_1^o,x_2^o)\in I^-(\gamma^2)\setminus I^-(\gamma^1)$. It is not a restriction to additionally assume that $x_1^o\neq x_1^*$. Let $\epsilon>0$ small enough such that $q_\epsilon=(t^o+\epsilon,x_1^o,x_2^o)\in I^-(\gamma^2)\setminus I^-(\gamma^1)$. Since $q_\epsilon\in I^-(\gamma^2)$, there exists an increasing sequence $\{t_n\}\nearrow b$ such that $q_\epsilon\ll \gamma^2(t_n)$ for all $n$. Then, from  Prop. \ref{c0}, there exist positive constants $\mu_1^n, \mu_2^n >0$, with $\mu_1^n+\mu_2^n=1$, for each $n$, such that
  \begin{equation}\label{eq*}
    \left\{\begin{array}{l}
    \displaystyle  \Integral{t^o+\epsilon}{t_n}{\mu^n_{1}}{1}{\mu^n_{k}}>
             d_{1}(x^o_{1},c_{1}(t_n))\\
\displaystyle\Integral{t^o+\epsilon}{t_n}{\mu^n_{2}}{2}{\mu^n_{k}}>
             d_{2}(x^o_{2},c^2_{2}(t_n)).
    \end{array}\right.
    \end{equation}
    It is not a restriction to assume that each sequence $\{\mu^n_i\}_n$ is convergent to $\mu_i^*$ for $i=1,2$. Next, we are going to prove that the sequences can be chosen satisfying $\mu_{1}^{*} \neq 1$:


   \smallskip

{\em Claim. The sequences $\{\mu^n_i\}_n$  can be chosen so that $\mu_1^*\neq 1$ (and thus, $\mu_2^*\neq 0$).}


  \smallskip


{\em Proof of the Claim.} Let us prove that, if we have a sequence $\{t_n\}_n$ such that $q=(t^o,x_1^o,x_2^o)\ll \gamma(t_n)=(t_n,c_1(t_n),c_2(t_n)) $, we can find sequences $\{\mu_i^n\}_n$, with $\mu_1^n+\mu_2^n=1$, which converge, up to a subsequence, to $\mu_1^*\neq 1$ and $\mu_2^*\neq 0$, such that
\begin{equation}
  \label{eq:30}
    \Integral{t^o}{t_n}{\mu^n_i}{i}{\mu^n_{k}}-
             d_{i}(x^o_{i},c_{i}(t_n))>0\quad \hbox{for $i=1,2$.}
\end{equation}
Observe that Prop. \ref{c0} ensures the existence of such convergent sequences $\{\mu_i^n\}_n$  without the statement about the limits. Assume that $\mu_1^*= 1$. By using standard arguments (that is, working with the point $q_{\epsilon}=(t^o+\epsilon,x_1^o,x_2^o)$ as before, and recalling that $\mu_1^n\geq \frac{1}{2}$ for $n$ big enough), we can take limits on \eqref{eq:30} preserving the strict inequality. So,
\[
\lim_{n\rightarrow \infty}\left(\Integral{t^o}{t_n}{\mu^n_1}{1}{\mu^n_{k}}-
             d_{1}(x^o_{1},c_{1}(t_n))\right)=\int_{t^o}^{b}\frac{1}{\sqrt{\alpha_1(s)}}ds-
             d_{1}(x^o_{1},x_1^*)>0
           \]
where $c_1(t_n)\rightarrow x_1^*$. Now take $\overline{\mu}_2^*>0$ small enough such that $\overline{\mu}_1^*=1-\overline{\mu}_2^*>0$ and such that
           \[
\Integral{t^o}{b}{\overline{\mu}_1^*}{1}{\overline{\mu}^*_k}-d_{1}(x^o_{1},x_1^*)>0
             \]
Now, define $\overline{\mu}_1^n=\mu_1^n-\overline{\mu}_2^*$ and $\overline{\mu}_2^n=\mu_2^n+\overline{\mu}_2^*$. As $\mu_{1}^{n} \rightarrow 1$,  we have that $\overline{\mu}_{1}^{n}>0$ for large $n$ and that
$\overline{\mu}_{1}^{n} \rightarrow \overline{\mu_{1}^{*}}$, therefore by  the Dominated Convergence Theorem (recall the integral condition for $\alpha_1$) we have:
\[
\Integral{t^o}{b}{\overline{\mu}_1^*}{1}{\overline{\mu}^*_k}=lim_{n} \Integral{t^o}{t_{n}}{\overline{\mu}_1^n}{1}{\overline{\mu}^n_k},
\]
Hence,
\[
  lim_{n}\left(\Integral{t^o}{t_{n}}{\overline{\mu}_1^n}{1}{\overline{\mu}^n_k}-d_{1}(x_{1}^{o},c_{1}(t_{n}))\right)
  % =\Integral{t^0}{\infty}{\overline{\mu_1^*}}{1}{\overline{\mu^*_k}}-d_{1}(x_{1}^o,x_{1}^*)
  >0,
\]
and so  for large $n$ we can take $\overline{\mu}_1^n$ and $\overline{\mu}_2^n$ satisfying
\[
\Integral{t^o}{t_n}{\overline{\mu}^n_1}{1}{\overline{\mu}^n_{k}}-
             d_{1}(x^o_{1},c_{1}(t_n))>0.
  \]
  Moreover, as $\overline{\mu}_1^n<\mu_1^n$ and $\overline{\mu}_2^n>\mu_2^n$, it easily follows that:

  \[
\Integral{t^o}{t_n}{\overline{\mu}^n_2}{2}{\overline{\mu}^n_{k}}>\Integral{t^o}{t_n}{\mu^n_2}{2}{\mu^n_{k}}\left(>d_2(x_2^0,c_2(t_n))\right).
    \]
In conclusion, equation \eqref{eq:30} is also true with the sequences $\{\overline{\mu}_i^n\}_n$ and $\{\overline{\mu}_1^n\}_n\rightarrow \overline{\mu}_{1}^{*}= 1-\overline{\mu}_{2}^{*}\neq 1$, which proves the claim.
%... and taking limits on the first previous inequality we deduce that
  %  \[
%\int_{t^0}^{\infty}\frac{1}{\sqrt{\alpha(s)}}ds=
  %           d_{1}(x^0_{1},x_{1}^*),
   %   \]
    %  (we cannot have strict inequality as, in such a case, we can perturb the sequence $\{\mu_1^n\}_n$ such that $\mu_1^*\neq 1$). But this is not possible as, in such a case,

  %    \[
%d_{1}(x^0_{1},x_{1}^*)=\int_{t^0}^{\infty}\frac{1}{\sqrt{\alpha(s)}}ds>\Integral{t^0}{\infty}{\mu'_1}{1}{\mu'_k}
  %      \]
    %  for any pair of positive constant sequences $\mu'_1,\mu'_2$ with $\mu'_1+\mu'_2=1$. But this is a contradiction with....


    \smallskip


  Continuing with the proof of the lemma, note that $\gamma^1$ and $\gamma^2$ share the same first spatial component $c_1$, the first integral condition (\ref{eq*}) coincides for both curves. Therefore, since $q_\epsilon\not\in I^-(\gamma^1)$, necessarily (recall Prop. \ref{c0}):
    \begin{equation}
      \label{eq:11}
d_2(x^o_2,c^1_2(t_n))\geq \Integral{t^o+\epsilon}{t_n}{\mu^n_{2}}{2}{\mu^n_{k}}\left(>
             d_{2}(x^o_{2},c^2_{2}(t_n))\right).
    \end{equation}
    Moreover, from the hypothesis, $b_{c_2^1}(x_2^o)=b_{c_2^2}(x_2^o)$. So, from the definition of Busemann function,
    \begin{equation}\label{x}
    \lim_{n}\left(d_2(x_2^o,c_2^1(t_n))-d_2(x_2^o,c_2^2(t_n))\right)=0.
    \end{equation}
    From \eqref{eq:11} and (\ref{x})
    \begin{equation}
      \label{eq:12}
\lim_n \left(\Integral{t^o+\epsilon}{t_n}{\mu^n_{2}}{2}{\mu^n_{k}}-d_2(x_2^o,c_2^1(t_n))\right)=0.
    \end{equation}

On the other hand, from the claim, the sequence of positive constants $\{\mu_2^n\}_n$ does not converge to $0$, so there exists ${\cal K}>0$ such that
   \begin{equation}
     \label{eq:13}
     \Integral{t^o}{t^o+\epsilon}{\mu_2^n}{2}{\mu_k^n}>{\cal K}>0\quad\hbox{for $n$ big enough.}
   \end{equation}
        So, putting together \eqref{eq:12} and \eqref{eq:13} we obtain that
\[
\lim_n \left(\Integral{t^o}{t_n}{\mu^n_{2}}{2}{\mu^n_{k}}-d_2(x_2^o,c_2^1(t_n))\right)>0,
  \]
  and thus,
  \[
\Integral{t^o}{t_n}{\mu^n_{2}}{2}{\mu^n_{k}}>d_2(x_2^o,c_2^1(t_n))\quad\hbox{for $n$ big enough.}
    \]
    From Prop. \ref{c0}, necessarily $q\in I^-(\gamma^1)$, a contradiction.


\end{proof}
\noindent As a direct consequence of Lemmas \ref{lemma:aux1} and \ref{lemma:aux2}, we obtain:
\begin{prop}\label{samecondsamepast}
Let $\gamma^i:[\omega,b)\rightarrow V$, $\gamma^i(t)=(t,c_1^i(t),c_2^i(t))$, $i=1,2$, be future-directed timelike curves. If $c_1^i(t)\rightarrow x_1^*\in M_1^C$, $i=1,2$, and $b_{c_2^1}=b_{c_2^2}$, then $I^-(\gamma^1)=I^-(\gamma^2)$.
\end{prop}
\begin{proof}
Let $c_1:[\omega,b)\rightarrow M_1$ be a curve with $c_1(t)\rightarrow x_1^*$ such that the curves $\overline{\gamma}^i:[\omega,\infty)\rightarrow V$, $\overline{\gamma}^i(t)=(t,c_1(t),c_2^i(t))$, $i=1,2$, are future-directed timelike. From Lemma \ref{lemma:aux1}, $I^-(\gamma^i)=I^-(\overline{\gamma}^i)$, $i=1,2$. But $\overline{\gamma}^1$, $\overline{\gamma}^2$ share the same first spatial components, and their second spatial components define the same Busemann function. Hence, Lemma \ref{lemma:aux2} ensures that $I^-(\overline{\gamma}^1)=I^-(\overline{\gamma}^2)$, as required.
\end{proof}

%Putting together Props. \ref{structuraparcialsininfinito}, \ref{prop:conddiferbordedif}
%and \ref{samecondsamepast} we obtain the following bijection:
%%\begin{equation}
%%  \label{eq:14}
%%\hat{V}\;\leftrightarrow\; \left(\R\times M_1^C\times M_2^C\right)\cup \left(\{\infty\}\times M_1^C\times {\cal B}(M_2)\right),
%%\end{equation}
%%which maps each indecomposable past set $P\in \hat{V}$ to the limit point of any curve generating it, which is either $(\Omega,x_1^*,x_2^*)\in \R\times M_1^C\times M_2^C$ if $\Omega<\infty or $(\Omega,x_1^*,b_{c_2})\in \{\infty\}\times M_1^C\times {\cal B}(M_2)$ if $\Omega=\infty$.
%
%%In particular, any future indecomposable set $P=I^-(\gamma)$ with $\gamma:[\alpha,\Omega)\rightarrow V$ and $\gamma(t)=(t,c_1(t),c_2(t))$ is determined by three elements: the limit $\Omega\in \R\cup \{\infty\}$ of the temporal component; the limit $x_1^*\in M_1^C$ of the first spatial component; and, (a) the limit $x_2^*\in M_2^C$ of the second spatial component if $\Omega<\infty$ or (b) the busemann function $b_{c_2}$ if $\Omega=\infty$.
%Even more, we can even give a unified treatment to this last case by recalling that, when $\Omega<\infty$, then the associated Busemann function $b_{c_2}\equiv d_{(\Omega,x_2^*)}$, so it codifies both, the limits of the temporal component and the second spatial component.

\noindent Summarizing, if we put together Props. \ref{structuraparcialsininfinito'}, \ref{prop:conddiferbordedif}
and \ref{samecondsamepast}, we deduce the following point set structure for the future c-completion of $(V,g)$:
\begin{thm}\label{futurecomploneinfinite}
  Let $(V,g)$ be a  {\multiwarped} spacetime as in \eqref{eq:1-aux}, and assume that the integral conditions \eqref{eq:9} hold. Then, there exists a bijection
 \begin{equation}
   \label{eq:10}
     \hat{V}\; \leftrightarrow \;  M_1^C\times \left(B(M_2)\cup \{\infty\}\right)\;\equiv\;
       \left( (a,b)\times M_1^C\times M_2^C\right) \cup M_{1}^{C} \times \left({\cal B}(M_2)\cup \{\infty\}\right).
     %,\qquad \hat{V}\setmin us \hat{\partial}^\infty V\equiv \cambios{(a,b)}\times M_1^C\times M_2^C  \\
%
%     \\
%\cambios{
%     \begin{array}{rl}
%       \hat{\partial}^{\infty}V\equiv  & M_1^C\times \left({\cal B}(M_2)\cup \{\infty\}\right)\\ \equiv & M_1^C\times \left(\left(\R\times \partial_{\cal B}M_2\right) \cup \{\infty\}   \right)
%     \end{array}
%}
 % \end{array}
  \end{equation}
 This bijection maps each indecomposable past set $P=I^-(\gamma)\in \hat{V}$, where $\gamma:[\omega,\Omega)\rightarrow V$, $\gamma(t)=(t,c_1(t),c_2(t))$, is any curve generating $P$, to a pair $(x_1^*,b_{c_2})$, where $x_1^*\in M_1^C$ is the limit point of the curve $c_1$. If $\Omega<b$, then $b_{c_2}=d_{(\Omega,x_2^*)}$, where $x_2^*$ is the limit point of $c_2$ (see \eqref{eq:46}), and thus, $P$ can be also identified with the limit point $(\Omega,x_1^*,x_2^*)$ of $\gamma$ (recall Prop. \ref{structuraparcialsininfinito'}).

  % to the limit point of any curve generating it, which is either $(\Omega,x_1^*,x_2^*)\in \R\times M_1^C\times M_2^C$ if $\Omega<\infty$ or $(\Omega,x_1^*,b_{c_2})\in \{\infty\}\times M_1^C\times {\cal B}(M_2)$ if $\Omega=\infty$.
\end{thm}


%This last property becomes specially interesting in order to analyze the topology, since it makes unnecessary to distinguish between indecomposable sets associated to finite or infinite $\Omega$.
%Concretely, we will not conceive the future completion as in \eqref{eq:14}, but in this alternative way:
%\begin{equation}
%  \label{eq:15}
%  \hat{V}\equiv M_1^C\times B(M_2),
%\end{equation}
%that is, any future indecomposable set $P$ will be identified to a pair $(x_1^*,b_{c_2^2})\in M_1^C\times B(M_2)$, and so $P\equiv (x_1^*,b_{c_2^2})$.


\subsubsection{Topological Structure}

 Next, we are going to extend previous study to a topological level, showing that the bijection obtained above is actually a homeomorphism when the corresponding product topology on $M_1^C\times (B(M_2)\cup \{\infty\})$ is considered.

 To this aim, we only need to prove the following equivalence:
%In this last subsection we show that the identification \eqref{eq:15} can be extended to the topological level, assumed that $M_1^C\times B(M_2)$ is endowed with the product topology. To this aim, we are going to prove the following equivalence:
given $P\equiv (x_1^*,b_{c_2})\in \hat{V}$ and $\{P_n\}_n\equiv \{(x_1^n,b_{c_2^n})\}_n\subset \hat{V}$,
\begin{equation}\label{equ}
P\in \hat{L}(\{P_n\}_n)\iff  x_1^n\rightarrow x^*_1\;\;\hbox{and}\;\; b_{c_2}\in \hat{L}(\{b_{c_2^n}\}_n).
  \end{equation}
  Under the hypothesis of $M_1^C$ and $M_2^C$ being locally compact, the equivalence (\ref{equ}) for the case $b_{c_2}\equiv d_{(\Omega,x_2)}$ is already proved in Prop. \ref{topcurvasfinitas}. In fact, if $P_n=I^-(\gamma^n)$ with $\gamma^n:[\omega,\Omega_n)\rightarrow V$, then $\Omega_n<b$ for $n$ big enough. In particular, $b_{c_2^n}\equiv d_{(\Omega_n,x^n_2)}$ with $x_2^n\in M_2^C$ (see \eqref{eq:46}). Moreover, Prop. \ref{topcurvasfinitas} implies that $(\Omega_n,x_1^n,x_2^n)\rightarrow (\Omega,x^*_1,x^*_2)$. Hence, $\{d_{(\Omega_n,x_2^n)}\}_n$ converges pointwise to $d_{(\Omega,x_2)}$, and thus, $d_{(\Omega,x_2)}\in \hat{L}(\{d_{(\Omega_n,x_2^n)}\}_n)$ (see Prop. \ref{prel:PropToponefibre}). So, to finish the proof of (\ref{equ}), we can focus just on the case $b_{c_2}\in {\cal B}(M_2)$.

  We begin with some preliminary results.
  \begin{lemma}\label{lemma:aux5}
    Let $P,P'\in \hat{V}$ and $\{P_n\}_n\subset \hat{V}$, and assume that $P\equiv (x_1,b_{c_2}), P'\equiv (x_1',b_{c'_2})$ and $P_n\equiv (x_1^n,b_{c_2^n})$ belong to $M_1^C\times \left(B(M_2)\cup \{\infty\}\right)$ for all $n$ (recall the identification in \eqref{eq:10}). Then, the following statements hold:
    \begin{itemize}
    \item[(i)] If $x_1=x_1'$, then
        \[
b_{c_2}\leq b_{c_2'} \iff P\subset P'.
        \]
    \item[(ii)] If $x_1^n\rightarrow x_1$, then
      \[
P\subset {\mathrm LI}(\{P_n\}_n) \iff b_{c_2}\leq {\mathrm lim\, inf}_n (\{b_{c_2^n}\}_n).
        \]
    \end{itemize}
  \end{lemma}
  \begin{proof} Let $\gamma:[\omega,\Omega)\rightarrow V$, $\gamma':[\omega',\Omega')\rightarrow V$ and $\gamma^n:[\omega^n,\Omega^n)\rightarrow V$ be future-directed timelike curves generating $P,P'$ and $P_n$, resp.

    (i) First, let us prove the implication to the left. Assume that $\gamma(t)=(t,c_1(t),c_2(t))$ and $\gamma'(t)=(t,c_1'(t),c_2'(t))$ satisfy that $c_{1}(t) \rightarrow x_{1}$, $c_{1}'(t) \rightarrow x_{1}$ and $b_{c_{2}}$, $b_{c_{2}'}$ are their Busemann functions. Consider the future-directed timelike curves $\sigma(t)=(t,c_2(t))$ and $\sigma'(t)=(t,c_2'(t))$ in the Generalized Robertson-Walker spacetime $$\left( (a,b)\times M_2,-dt^2+\alpha_2g_2\right).$$ Since $P\subset P'$, necessarily $P(b_{c_2})=I^-(\sigma)\subset I^-(\sigma')=P(b_{c_2'})$, and thus, $b_{c_2}\leq b_{c_2'}$ (recall \eqref{eq:27} and \eqref{eq:28}).

For the implication to the right, assume that $x_1=x_1'$ and $b_{c_2}\leq b_{c'_2}$. It suffices to show the existence of a sequence $\sigma=(t_n,y_1^n,c_2(t_n))$ with $\{t_n\}_n\nearrow \Omega$, satisfying $\{y_1^n\}_n\rightarrow x_1$ and $\sigma\subset P'$. In fact, in this case, Lemma \ref{lemma:aux3} ensures that $P\subset {\rm LI}(\sigma)$ and, taking into account that $\sigma\subset P'$, necessarily $P\subset P'$.

To this aim, take $\{t_n\}_n\nearrow \Omega$ and observe that, by hypothesis, $b_{c_2}\leq b_{c'_2}$. So, in the Generalized Robertson-Walker spacetime $\left( (a,b)\times M_2,-dt^2+{\alpha_2}g_2\right)$, the inclusion $P(b_{c_2})\subset P(b_{c'_2})$ holds (recall equations \eqref{eq:27} and \eqref{eq:28}). In particular, since the future-directed timelike curves $\sigma(t)=(t,c_2(t))$ and $\sigma'(t)=(t,c'_2(t))$ satisfy $I^-(\sigma)=P(b_{c_2})$ and $ I^-(\sigma')=P(b_{c'_2})$, there exists a sequence $\{s_n\}_n$, with $\{s_n\}_n\nearrow \Omega'$, such that
$\sigma(t_n)=(t_n,c_2(t_n))\ll (s_n,c'_2(s_n))=\sigma'(s_n)$. Let us show that $\sigma=\{(t_n,c_1'(s_n),c_2(t_n))\}$ is the required sequence. From construction and the fact that $(t_n,c_1'(s_n),c_2(t_n))\ll (s_n,c_1'(s_n),c'_2(s_n))$ in $V$ for all $n$, necessarily $\sigma\subset P'$. Moreover, since $\{s_n\}_n\nearrow \Omega'$, necessarily $c_1'(s_n)\rightarrow x_1'=x_1$, as desired.

\smallskip

(ii) For the implication to the right, assume that $P\subset {\mathrm LI}(\{P_n\}_n)$ and let us show that $b_{c_2}\leq \liminf (\{b_{c_2^n}\}_n)$. Denote by $\sigma(t)=(t,c_2(t))$ and $\sigma_n(t)=(t,c_2^n(t))$ future-directed timelike curves in the Generalized Robertson Walker model \[\left((a,b)\times M_2,-dt^2+ \alpha_2g_2\right).\] Since $P\subset {\mathrm LI}(\{P_n\}_n)$, necessarily
  \[
P(b_{c_2})=I^-(\sigma)\subset {\mathrm LI}(\{I^-(\sigma_n)\}_n)={\mathrm LI}(\{P(b_{c_2^n})\}_n)
    \]
(where we are considering past sets in the associated Generalized Robertson Walker model), and the conclusion follows from \eqref{eq:50}.

% . Our aim is to show that $b_{c_2}\leq \liminf (\{b_{c_2^n}\}_n))$, or equivalently, that if we have $x_2^o\in M_2$ and $r\in\R$ such that

% \begin{equation}
%   \label{eq:23}
% r< b_{c_2}(x_2^o)=\lim_{t\rightarrow \Omega} \left(\int_{\C}^{t}\frac{1}{\sqrt{\alpha_2(s)}}ds - d(x_2^o,c_2(t))\right),
% \end{equation}
% then
%   \[
% r< b_{c^n_2}(x_2^0)\quad\hbox{for $n$ big enough.}
%     \]
% Take $t^o$ and $x_2^o$ satisfying \eqref{eq:23}. For $t>t^o$ big enough it follows that:
%   \[
% \int_{\C}^{t^o}\frac{1}{\sqrt{\alpha_2(s)}}ds<\int_{\C}^{t}\frac{1}{\sqrt{\alpha_2(s)}}ds - d(x_2^0,c_2(t))\Rightarrow \int_{t^o}^{t}\frac{1}{\sqrt{\alpha(s)}}ds>d_2(x_2^o,c_2(t)).
% \]
%  Since this last inequality is strict, we can take positive constants $\mu_1,\mu_2>0$, with $\mu_1+\mu_2=1$, and $x_1^o\in M_1$ close enough to $c_1(t)$, such that
% \[
% \Integral{t^o}{t}{\mu_i}{i}{\mu_k}>d_i(x_i^o,c_i(t)) \qquad \hbox{for $i=1,2$.}
%   \]
%   From Prop. \ref{c0}, $q=(t^o,x_1^o,x_2^o)\in I^-(\gamma)=P$. Since $P\subset {\mathrm LI}(\{P_n\})$, there exists $n_0$ such that $q\in P_n$ for $n\geq n_0$. Again from Prop. \ref{c0}, there exist positive constants $\mu_1^n,\mu_2^n>0$, with $\mu_1^n+\mu_2^n=1$, and $t_n\in [\omega,\Omega_n)$, such that
%   \[
% \Integral{t^o}{t_n}{\mu^n_i}{i}{\mu^n_k}>d_i(x_i^o,c_i^n(t_n)) \qquad \hbox{for $i=1,2$}.
%     \]
% In particular, for $i=2$,
%     \[
%       \begin{array}{c}
% \displaystyle\int_{t^o}^{t_n}\frac{1}{\sqrt{\alpha_2(s)}}ds > \Integral{t^o}{t^n}{\mu^n_2}{2}{\mu^n_k}>  d_2(x_2^o,c_2^n(t_n))\Rightarrow\\  \\ \Rightarrow \displaystyle \int_{\C}^{t^o}\frac{1}{\sqrt{\alpha_2(s)}}ds< \int_{\C}^{t_n}\frac{1}{\sqrt{\alpha_2(s)}}ds-d_2(x_2^o,c_2^n(t_n)).
%       \end{array}
%       \]
%       Now, from the increasing character of this last term, as described in \eqref{eq:25}, it follows that
%       \[
% \int_{\C}^{t^o}\frac{1}{\sqrt{\alpha_2(s)}}<b_{c_2^n}(x_2^o)\quad\hbox{for $n\geq n_0$}.
%         \]
%         In conclusion, $b_{c_2}\leq {\mathrm lim\, inf}_n( \{b_{c_2^n}\}_n)$, as desired.

        \smallskip

        For the implication to the left, assume that $b_{c_2}\leq {\mathrm lim\, inf}_n (\{b_{c_2^n}\}_n)$ and let us prove that $P\subset {\mathrm LI}(\{P_n\}_n)$. Let $\{t_k\}\nearrow \Omega$ be an arbitrary sequence. For each $k$, and from the timelike character of $\gamma$, we have $(t_k,c_2(t_k))\ll (t,c_2(t))$ in the Generalized Robertson-Walker spacetime $\left( (a,b)\times M_2,-dt^2+{\alpha_2}g_2\right)$ for all $t>t_k$. From \eqref{eq:26} and the increasing character of \eqref{eq:25},

      \[
\int_\C^{t_k}\frac{1}{\sqrt{\alpha_2(s)}}ds < b_{c_2}(c_2(t_k))=\lim_{t\rightarrow \Omega} \left(\int_\C^t \frac{1}{\sqrt{\alpha_2(s)}}ds - d_2(c_2(t_k),c_2(t))\right).
        \]
Since $b_{c_2}\leq {\mathrm lim\,inf}(\{b_{c_2^n}\}_n)$, there exists an increasing sequence $\{n_k\}_k$ such that
\begin{equation}
  \label{eq:16}
\int_\C^{t_k}\frac{1}{\sqrt{\alpha_2(s)}}ds<b_{c^n_2}(c_2(t_k))=\lim_{r\rightarrow \Omega_n} \left(\int_\C^{r}\frac{1}{\sqrt{\alpha_2(s)}}ds- d_2(c_2(t_k),c_2^n(r))\right)\quad\hbox{$\forall$ $n\geq n_k$.}
\end{equation}
  For each $n_k\leq n<n_{k+1}$, consider $r_n\in [\omega_{n},\Omega_n)$ such that
  \begin{equation}
    \label{eq:24}
\int_\C^{t_k}\frac{1}{\sqrt{\alpha_2(s)}}ds< \int_\C^{r_n}\frac{1}{\sqrt{\alpha_2(s)}}ds-d_2(c_2(t_k),c_2^n(r_n)),\qquad d_{1}(c_1^n(r_n),x_1^n)<\frac{1}{2^n},
  \end{equation}
    (for the first inequality recall \eqref{eq:16}; for the second one, recall that $c_1^n(t)\rightarrow x_1^n$). From the first inequality, it follows that
    \[
    (t_k,c_2(t_k))\ll (r_n,c_2^n(r_n))\quad\hbox{for $n_k\leq n< n_{k+1}$ and all $k$.}
    \]
    However, since $\{(t_k,c_2(t_k))\}$ is a chronological chain, previous chronological relation is true for all $n\geq n_k$: in fact, if $n\geq n_k$, there exists $k'(\geq k)$ such that $n_{k'}\leq n < n_{k'+1}$. As we have noted before $(t_{k'},c_2(t_{k'}))\ll (r_n,c_2^n(r_n)))$ but, taking into account $(t_k,c_2(t_k))\ll (t_{k'},c_2(t_{k'}))$, necessarily $(t_{k},c_2(t_{k}))\ll (r_n,c_2^n(r_n)))$.

    Next, define the sequence $\sigma=\{(l_n,c^n_1(r_n),c_2(l_n))\}_n$, where $l_n:=t_k$ if $n_{k}\leq n< n_{k+1}$. Since $\{t_k\}_k\nearrow \Omega$, necessarily $\{l_n\}_n\rightarrow \Omega$. Moreover, since $(t_k,c_2(t_k))\ll (r_n,c_2^n(r_n))$,
    \[(l_n,c^n_1(r_n),c_2(l_n))=(t_k,c_1^n(r_n),c_2(t_k))\ll (r_n,c_1^n(r_n),c^n_2(r_n))=\gamma^n(r_n),\] hence $(l_n,c^n_1(r_n),c_2(l_n))\in P_n$ for all $n$. Finally, note that $\sigma$ satisfies the conditions of Lemma \ref{lemma:aux3}, as $\{l_n\}_n\rightarrow \Omega$ and $c_1^n(r_n)\rightarrow x_1$ (recall that $c_1^n(t)\rightarrow x_1^n$, $x_1^n\rightarrow x_1$ from hypothesis and the second inequality in \eqref{eq:24}). Therefore,
\[
    P\subset {\mathrm LI}(\{I^-(l_n,c^n_1(r_n),c_2(l_n))\}_n)\subset {\mathrm LI}(\{P_n\}),
    \]
    as desired.
  \end{proof}

  %After previous technical result we are able to prove the equivalence between topologies. Let us begin with:


  \begin{prop}\label{prop:topcharac} Let $P\in \hat{V}$ and $\{P_n\}_n\subset \hat{V}$, and assume that $P\equiv (x_1,b_{c_2})$ and $P_n\equiv (x_1^n,b_{c_2^n})$ (in $M_1^C\times \left(B(M_2)\cup \{\infty\} \right)$) for all $n$. Then,
$P\in \hat{L}(\{P_n\}_n)$ if, and only if, $x_1^n\rightarrow x_1$ and $b_{c_2}\in \hat{L}(\{b_{c_2^n}\}_n)$.
  \end{prop}
  \begin{proof}
    For the implication to the right, and reasoning as in the proof of Thm. \ref{futurestructurefiniteconditions}, it follows that $x_1^n\rightarrow x_1$ (recall the finite warping integral in \eqref{eq:9} and Remark \ref{rem:1}). Hence, we will focus on $b_{c_2}\in \hat{L}(\{b_{c_2^n}\})$. From Lemma \ref{lemma:aux5} and the fact that $P\in \hat{L}(\{P_n\}_n)$, necessarily $b_{c_2}\leq \liminf(\{b_ {c_2^n}\}_n)$. So, $b_{c_2}\in \hat{L}(\{b_{c_2^n}\})$ follows if we prove that $b_{c_2}$ is maximal into $\limsup(\{b_{c^n_2}\}_n)$. Consider any $b_{\overline{c}_2}$ such that $b_{c_2}\leq b_{\overline{c}_2}\leq {\mathrm lim\,sup}(\{b_{c^n_2}\}_n)$, and consider the associated past set $\overline{P}\equiv (x_1,b_{\overline{c}_2})$. Up to a subsequence, we can assume that $b_{\overline{c}_2}\leq {\mathrm lim\,inf}(\{b_{c^n_2}\}_n)$. From Lemma \ref{lemma:aux5}, $P\subset \overline{P}$ and $\overline{P}\subset {\mathrm LI}(\{P_{n}\}_n)$. But $P$ is maximal into the superior limit of the sequence $\{P_n\}_n$, so necessarily $P=\overline{P}$. From Prop. \ref{prop:conddiferbordedif} we have that $b_{c_2}=b_{\overline{c}_2}$ so the maximal character of $b_{c_2}$ into $\limsup(\{b_{c_2^n}\}_n)$ is obtained.

    \smallskip

    For the implication to the left, first note that $P\subset {\mathrm LI}(\{P_n\}_n)$ (recall Lemma \ref{lemma:aux5} and the definition of $\hat{L}$ for Busemann functions \eqref{eq:22}). So, we only need to focus on the maximal character of $P$ into ${\mathrm LS}(\{P_n\})$. Take $\overline{P}$ an indecomposable past set with $P\subset \overline{P}$ and maximal into ${\mathrm LS}(\{P_n\})$, and let us prove that $P=\overline{P}$. Assume that $\overline{P}\equiv (\overline{x}_1,b_{\overline{c}_2})$. Up to a subsequence, we can also assume that $\overline{P}\subset {\mathrm LI}(\{P_n\})$, hence $\overline{P}\in \hat{L}(\{P_n\})$. Hence, from previous part, $x_1^n\rightarrow \overline{x}_1$. But, by hypothesis, $x_1^n\rightarrow x_1$, obtaining that $x_1=\overline{x}_1$. Once this is observed, Lemma \ref{lemma:aux5} ensures both, $b_{c_2}\leq b_{\overline{c}_2}$ and $b_{\overline{c}_2}\leq \limsup(\{b_{c_2^n}\})$. Since $b_{c_2}\in \hat{L}(\{b_{c_2^n}\})$, necessarily $b_{c_2}=b_{\overline{c}_2}$, and so, $P=\overline{P}$ (recall Prop. \ref{samecondsamepast}).
  \end{proof}

\noindent Summarizing, we are in conditions to deduce the following result:

\begin{thm}\label{futurecomploneinfinite}
  Let $(V,g)$ be a  {\multiwarped} spacetime as in \eqref{eq:1-aux}, and assume that the integral conditions in \eqref{eq:9} are satisfied. If $M_1^C$ and $M_2^C$ are locally compact, the bijection (\ref{eq:10}) becomes a homeomorphism.
\end{thm}
\begin{proof}
  From Prop. \ref{topcurvasfinitas}, the bijection between $\hat{V}\setminus \hat{\partial}^{\B} V$ and $(a,b) \times M_1^C\times M_2^C$ is a homeomorphism if we assume that $M_1^C$ and $M_2^C$ are locally compact. From Prop. \ref{prop:topcharac}, the homeomorphism can be extended to the bijection (\ref{eq:10}).
\end{proof}

\section{The past c-completion of doubly warped spacetimes}\label{ss6}

Obviously, similar arguments provide the corresponding results for the past c-completion:

\begin{thm}\label{pfuturestructurefiniteconditions}
  Let $(V,g)$ be a {\multiwarped} spacetime as in (\ref{eq:1-aux}), and assume that the integral conditions
  \begin{equation}
  \label{eqq:7}
  \int_{a}^{\C}\frac{1}{\sqrt{\alpha_i(s)}}ds<\infty, \qquad \hbox{$i=1,2$}\quad\hbox{for some $\C\in (a,b)$.}
\end{equation}
 hold. Then, there exists a bijection
  \begin{equation}
    \label{eqq:8}
    \check{V}\; \leftrightarrow \; [a,b) \times M_1^C\times M_2^C
  \end{equation}
  which maps each IF $F\in \check{V}$ to the limit point $(\Omega,x_1,x_2)\in [a,b)\times M_1^C\times M_2^C$ of any past-directed timelike curve generating $F$. Moreover, if $M_1^C$ and $M_2^C$ are locally compact, then this bijection becomes a homeomorphism.
\end{thm}


\begin{thm}\label{pfuturecomploneinfinite}
  Let $(V,g)$ be a  {\multiwarped} spacetime as in \eqref{eq:1-aux}, and assume that the integral conditions
  \begin{equation}
  \label{eqq:9}
 \int_{a}^{\C}\frac{1}{\sqrt{\alpha_1(s)}}ds<\infty \qquad \hbox{and}\qquad \int_{a}^{\C}\frac{1}{\sqrt{\alpha_2(s)}}ds=\infty,
\end{equation}
hold. Then, there exists a bijection
 \begin{equation}
   \label{eqq:10}
     \check{V}\; \leftrightarrow\;  M_1^C\times \left(B(M_2)\cup \{-\infty\}\right)
     \equiv  \left( (a,b)\times M_1^C\times M_2^C\right) \cup M_{1}^{C} \times \left({\cal B}(M_2)\cup \{\infty\}\right).
  \end{equation}
  This bijection maps each indecomposable future set $F=I^+(\gamma)\in \check{V}$, where $\gamma:[\omega,-\Omega)\rightarrow V$, $\gamma(t)=(-t,c_1(t),c_2(t))$, is any curve generating $F$, to a pair $(x_1^*,b^-_{c_2})$, where $x_1^*\in M_1^C$ is the limit point of the curve $c_1$. If $-\Omega>a$, then  $b^-_{c_2}=d^-_{(\Omega,x_2^*)}$, where $x_2^*$ is the limit point of $c_2$ (see \eqref{eq:48}), and thus, $F$ can be also identified with the limit point $(\Omega,x_1^*,x_2^*)$ of $\gamma$.
  % to the limit point of any past-directed timelike curve generating it, which is either $(\Omega,x_1^*,x_2^*)\in \R\times M_1^C\times M_2^C$ if $\Omega>-\infty$ or $(\Omega,x_1^*,b_{c_2})\in \{-\infty\}\times M_1^C\times {\cal B}(M_2)$ if $\Omega=-\infty$. Moreover, if $M_1^C$ and $M_2^C$ are locally compact, the bijection (\ref{eq:10}) becomes a homeomorphism.
\end{thm}

%.....................................................
%
%
%Obviously, all previous results have a past analog, which we briefly review as follows:
%
%\begin{thm}\label{pastcompletion}
%  Let $(V,g)$ be a {\multiwarped} model as in \eqref{eq:1}. Then,
%
%  \begin{equation}
%    \label{eq:39}
%    \check{V}\setminus \check{\partial}^{\infty} V\equiv \R\times M_1^C\times M_2^C,
%  \end{equation}
%  where $\check{\partial}^{\infty}V$ denotes the TIFs determined by past-directed timelike curves with divergent temporal component. Moreover,
%  \begin{itemize}
%  \item If
%    \begin{equation}
%      \label{eq:40}
%      \int^{0}_{-\infty}\frac{1}{\sqrt{\alpha_i(s)}}ds<\infty, \qquad \hbox{for $i=1,2$.}
%    \end{equation}
%    then the past causal boundary and completion has the following structure
%    \begin{equation}
%      \label{eq:41}
%      \check{V}\equiv [a,b)\times M_1^C\times M_2^C.
%    \end{equation}
%That is, any indecomposable future set $F\in \check{V}$ can be labelled by a triple $(\Omega,x_1,x_2)\in [a,b)\times M_1^C\times M_2^C$. Moreover, if $M_1^C,M_2^C$ are locally compact, then previous identifications can be extended to the topological level.
%  \item If
%    \begin{equation}
%      \label{eq:42}
%       \int^{0}_{-\infty}\frac{1}{\sqrt{\alpha_1(s)}}ds<\infty \qquad \hbox{and}\qquad \int^{0}_{-\infty}\frac{1}{\sqrt{\alpha_2(s)}}ds=\infty.
%    \end{equation}
%    then we obtain
%    \begin{equation}
%      \label{eq:43}
%      \begin{array}{c}
%     \check{V}\equiv M_1^C\times \left(B(M_2)\cup \{-\infty\}\right),\qquad \check{V}\setminus \check{\partial}^\infty V\equiv \cambios{(a,b)}\times M_1^C\times M_2^C\\
%
%     \\
%\cambios{
%     \begin{array}{rl}
%       \check{\partial}^{\infty}V\equiv  & M_1^C\times \left({\cal B}(M_2)\cup \{-\infty\}\right)\\ \equiv & M_1^C\times \left(\left(\R\times \partial_{\cal B}M_2\right) \cup \{-\infty\}   \right)
%     \end{array}
%}
%\end{array}
%    \end{equation}
%    and if $M_1^C,M_2^C$ are locally compact, previous identifications extend to the topological level.
%  \end{itemize}
%\end{thm}




%%% Local Variables:
%%% mode: latex
%%% TeX-master: "DoublyWarpedBoundary2017"
%%% End:
