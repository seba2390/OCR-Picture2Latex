\NeedsTeXFormat{LaTeX2e}
%\documentclass[a4paper,10pt]{scrartcl}
%\documentclass[a4paper,10pt]{scrartcl}
%\documentclass[oneside,draft,letterpaper,12pt]{amsart}
\documentclass[a4paper,reqno,11pt]{article}
%\usepackage{jheppub}
\usepackage[a4paper,bindingoffset=0.2in,%
left=1in,right=1in,top=1in,bottom=1in,%
           footskip=.25in]{geometry}
\usepackage{graphicx}
\renewcommand{\sc}{\scshape}
\setlength{\textwidth}{\paperwidth}
% \usepackage{cite}
\usepackage{authblk}
\addtolength{\textwidth}{-2.5in}
%\calclayout
\usepackage{amssymb}
\usepackage{amsthm}
\usepackage{amsmath}
% \usepackage[english]{babel}
\usepackage{color}
\usepackage[utf8]{inputenc}
%\usepackage{refcheck}
\usepackage{tikz}
\usepackage[all]{xy}
\usepackage[bookmarksnumbered,colorlinks]{hyperref}
\usepackage{dsfont}
\usepackage[normalem]{ulem}

\usepackage{float} % % Figuras


\usepackage[euler-digits]{eulervm}
\usepackage{enumerate}
\def\br#1\er{\textcolor{red}{#1}} %
\hyphenation{Lo-ren-tzian}
\def\soutE#1{\sout{#1}} %to see the overlines
\providecommand{\bold}{\mathbf}
\providecommand{\scr}{\mathcal}

\newcommand{\eps}{\varepsilon}
\newcommand{\wk}{\rightharpoonup}
\newcommand{\then}{\Longrightarrow}
\newcommand{\g}{\langle\cdot,\cdot\rangle }
\newcommand{\m}{{\cal M}}
\newcommand{\mo}{{\cal M}_0}
\newcommand{\I}{{\cal G}}
\newcommand{\cat}{{\mathrm cat}}
\newcommand{\arcsinh}{{\mathrm arcsinh}}
\newcommand{\dimo}{{\bf Proof.\ }}
\newcommand{\cvd}{\ \rule{0.5em}{0.5em}}
\newcommand{\be}{\begin{equation}}
\newcommand{\ee}{\end{equation}}
\newcommand{\<}{\langle}
\renewcommand{\>}{\rangle}

\newcommand{\N}{{\mathbb N}}
\newcommand{\Z}{{\mathbb Z}}
\newcommand{\R}{{\mathbb R}}
\newcommand{\LL}{{\mathbb L}}


\newcommand{\M}{{\cal M}}
\newcommand{\V}{{\cal V}}
\newcommand{\C}{\mathfrak{c}}
\newcommand{\J}{{\cal J}}

\newcommand{\noi}{\noindent}
\newcommand{\ben}{\begin{enumerate}}
\newcommand{\een}{\end{enumerate}}
\newcommand{\bit}{\begin{itemize}}
\newcommand{\eit}{\end{itemize}}
\newcommand{\edoc}{\end{document}}
\newcommand{\sm}{\smallskip}
\newcommand{\np}{\newpage}

\newcommand{\vs}{\vspace}
\newcommand{\onto}{{\rightarrow}}
%\newcommand{\R}{\mathds R}
%\newcommand{\N}{\mathds N}
%\newcommand{\Z}{\mathds Z}
\newcommand{\Ver}{\mathcal V}
\newcommand{\Hor}{\mathcal H}
\newcommand{\Ddt}{\frac{\mathrm D}{\mathrm dt}}
\newcommand{\germ}{\mathrm{germ}}
\newcommand{\relpri}{\sim_{G_0}}
%\newcommand{\cambios}{\color{red}}
\newcommand{\lev}[2]{\overline{#1}_{#2}}
\newcommand{\opciones}[2]{#2}
\newcommand{\lcrono}{L_{chr}}
\newcommand{\lcronof}{\hat{L}_{chr}}
\newcommand{\tcrono}{\tau_{chr}}
\newcommand{\cambios}[1]{{\color{black} #1}}
\newcommand{\ncambios}[1]{{\color{black} #1}}

\newcommand{\multiwarped}{doubly warped~}
\newcommand{\timelike}{timelike}
\newcommand{\lightlike}{lightlike}
\newcommand{\spatial}{spatial}
\newcommand{\unique}{{univocally determined}}
\newcommand{\Integral}[5]{\int_{#1}^{#2} \frac{\sqrt{#3}}{\alpha_{#4}(s)}\left(\sum_{k=1}^{2} \frac{#5}{\alpha_{k}(s)} \right)^{-1/2}ds}
% Para hacer el cambio al multiwarped
\newcommand{\Integralm}[5]{\int_{#1}^{#2} \frac{\sqrt{#3}}{\alpha_{#4}}\left(\sum_{k=1}^{n+1} \frac{#5}{\alpha_{k}} \right)^{-1/2}ds}

\newcommand{\point}[3]{(#1,#2,#3)}
\newcommand{\B}{b}
%\frac{#3}{\alpha_{1}}+\frac{#5}{\alpha_{2}}   \sum_{k} \frac{#5}{\alpha_{k}}

%\title{Spacetimes coverings and C-boundary}
%\author[L. A. Ak\'e, J. Herrera]{Luis Alberto Ak\'e \and J\'onatan Herrera}
%\address{Universidade Federal de Santa Catarina, Brazil}
%\address{Universidad de M\'alaga, Spain}
%\date{5 May 2016}
\title{Causality and c-completion of multiwarped spacetimes}
%\author[L. A. Ak\'e, J. Herrera]{Luis Alberto Ak\'e \and J\'onatan Herrera}
%\address{Universidade Federal de Santa Catarina, Brazil}
%\address{Universidad de M\'alaga, Spain}
%\date{5 May 2016}
\date{}
\author[1]{Luis Alberto Aké}
\author[1]{José Luis Flores}
\author[2]{Jónatan Herrera}
\affil[1]{\small Departamento de Álgebra, Geometría y Topología, Facultad de Ciencias\\ Universidad de Málaga, Campus Teatinos, 29071 Málaga, Spain}
\affil[2]{\small Departamento de Matemáticas, Edificio Albert Einstein\\ Universidad de Córdoba, Campus de Rabanales, 14071 Córdoba, Spain}
\affil[ ]{\textit{Corresponding author: jonatanhf@gmail.com}}


% \author[L. Ak\'e]{Luis Alberto Ak\'e}
% \address{Departamento de \'Algebra, Geometr\'{\i}a y Topolog\'{\i}a,  Universidad de M\'alaga
% \hfill\break\indent
% \hfill\break\indent
% Facultad de Ciencias, Campus Universitario de Teatinos,
% \hfill\break\indent 29080 M\'alaga, Spain}
% \email{luisake@uma.es}

% % % % Information Jonatan

%\author[J. Herrera]{J\'onatan Herrera}
%\address{Department of Mathematics,
%	Universidade Federal de Santa Catarina,
%	\hfill\break\indent 88.040-900 Florian\'{o}polis-SC, Brazil.}
%\email{jonatanhf@gmail.es}

%\abstract{We consider the relation between the c-completion of a Lorentz manifold $V$ and its quotient $M=V/G$, where $G$ is an isometry group acting freely and properly discontinuously.
%  First, we consider the future causal completion case, characterizing virtually when such a quotient is well behaved with the future chronological topology and improving the existing results on the literature.  Secondly, we show that under some general assumptions, there exists a homeomorphism and chronological isomorphism between both, the c-completion of $M$ and some adequate quotient of the c-completion of $V$ defined by $G$. Our results are optimal, as we show in several examples. Finally, we give a practical application by considering isometric actions over Robertson-Walker spacetimes, including in particular the Anti-de Sitter model.}
%\arxivnumber{1605.03128}

\begin{document}
\newtheorem{thm}{Theorem}[section]
\newtheorem{prop}[thm]{Proposition}
\newtheorem{lemma}[thm]{Lemma}
\newtheorem{cor}[thm]{Corollary}
\newtheorem{conv}[thm]{Convention}
\theoremstyle{definition}
\newtheorem{defi}[thm]{Definition}
\newtheorem{notation}[thm]{Notation}
\newtheorem{exe}[thm]{Example}
\newtheorem{conj}[thm]{Conjecture}
\newtheorem{prob}[thm]{Problem}
\newtheorem{rem}[thm]{Remark}


%\theoremstylx
\maketitle
\usetikzlibrary{matrix}



\begin{abstract}
In this paper a systematic study of the causal structure and global causality properties of multiwarped spacetimes is developed. This analysis is used to make a detailed description of the causal boundary of these spacetimes. Some applications of our results in examples of physical interest, for instance, in the context of Maldacena's conjecture, are considered.
%
%A discussion about the implications of these results to the
%
%The aim of this paper is twofold: first, it will provide a general framework for the study of causal structures in multiwarped models $(\mathcal{V},\mathfrak{g})$; then, we will describe in full detail the causal boundary of $(\mathcal{V},\mathfrak{g})$ consider some mild hypothesis on the integrals of the warping functions. We will present some relevant examples where our results are applicable.
\end{abstract}

%  \maketitle
%  \flushbottom

\tableofcontents

\section{Introduction}
\label{sec:Introduction}


The goal in top-$\size$ recommendation is to recommend to each
consumer a small set of $\size$ items from a large collection of
items~\cite{cremonesi2010performance}.  For example, Netflix may want
to recommend $\size$ appealing movies to each consumer.  Collaborative
Filtering (CF)~\cite{herlocker2002empirical,lee2012comparative} is a
common top-$\size$ recommendation method.  CF infers user interests by
analyzing partially observed user-item interaction data, such as user
ratings on movies or historical purchase
logs~\cite{kanagal2012supercharging}. The main assumption in CF is that
users with similar interaction patterns have similar interests.


Standard CF methods for top-$\size$ recommendation focus on making  suggestions  that accurately reflect the user's preference history. However, as  observed in previous work,  CF recommendations are generally biased toward  popular items, leading to a rich get richer effect~\cite{vargas2014improving,steck2011item}.  The major reasons for this are \textit{popularity bias} and \textit{sparsity} of CF interaction data (detailed in Section~\ref{sec:related-work}). In a nutshell, to maintain  accuracy, recommendations are generated from the dense regions of the data,  where the popular items lie.  

However,  accurately suggesting popular items, may not be satisfactory for the consumers. For example, in Netflix, an accuracy-focused movie recommender may recommend ``Star Wars: The Force Awakens'' to users who have seen ``Star Wars: Rogue One''.  But, those users are probably already aware of ``The Force Awakens''. Considering additional factors, such as novelty of recommendations,  can lead to more effective suggestions~\cite{cremonesi2010performance,Castells2015,zhang2008avoiding,ziegler2005improving,zhang2012auralist}. 
%Second, accuracy-focused models typically achieve a   overall item-space coverage across their recommendations,  whereas high item-space coverage helps providers of the items increase revenue
%, users satisfaction since they are  likely already aware of or can find these items on their own.  

Focusing on popular items also adversely affects the satisfaction of  the providers of the items. This is because  accuracy-focused models typically achieve a  low overall item space coverage across their recommendations, whereas   high item space coverage helps providers of the items increase their revenue~\cite{vargas2014improving,Castells2015,adomavicius2011maximizing,anderson2006thelongtail, yin2012challenging,adomavicius2012improving}.
%accuracy-focused models typically achieve a

In contrast to the relatively small number of popular items, there are copious  {\it long-tail\/} items that have fewer observations (e.g., ratings) available. More precisely,  using the Pareto  principle (i.e.,~the $80/20$ rule),  long-tail items can be defined as items that generate the lower $20\%$ of observations~\cite{yin2012challenging}. Experimentally we found that these items correspond to almost $85\%$ of the items in several datasets (Sections~\ref{sec:Notation} and \ref{sec:Experiments}). %Table~\ref{tab:DatasetStatsticsSmall})


As previously shown, one way to improve the novelty of top-$\size$ sets is to recommend interesting long-tail items~\cite{cremonesi2010performance,ge2010beyond}.  The intuition  is that since they have fewer observations available,  they are more likely to be unseen~\cite{Kaminskas:2016:DSN:3028254.2926720}.  
 %For example, in online commerce,  newly added items are long-tail items that are yet to be discovered.  
Moreover, long-tail item promotion also results in higher overall coverage of the item space%, which increases profits for providers of the items
~\cite{vargas2014improving,Castells2015,zhang2008avoiding,zhang2012auralist,adomavicius2011maximizing,anderson2006thelongtail,yin2012challenging,jambor2010optimizing}. Because long-tail promotion reduces accuracy~\cite{steck2011item}, there are trade-offs to be explored.


%original submitted to ICDE
%This work studies three aspects of top-$\size$ recommendation: accuracy, novelty, and item-space coverage, and examines their trade-offs. In most previous work, predictions of a base recommendation system are re-ranked to handle their trade-offs~\cite{adomavicius2012improving,jambor2010optimizing,zhang2013personalize,wang2009portfolio}. Due to performance considerations, however, these techniques are not customized per user. For example,  parameters that balance the trade-off between novelty and accuracy are cross-validated at a global level.  This can be detrimental since users have varying preferences for  objectives such as long-tail novelty. We explore how to  automatically infer  user  preference for long-tail novelty, and how to leverage  it to correct  the popularity bias in standard recommender models. Our work does not rely on any additional contextual data, although such data, if available, can help promote newly-added long-tail items~\cite{agarwal2009regression,Saveski:2014:ICR:2645710.2645751}.

This work studies three aspects of top-$\size$ recommendation: accuracy, novelty, and item space coverage, and examines their trade-offs. In most previous work, predictions of a base recommendation algorithm are \textit{re-ranked} to handle these trade-offs~\cite{adomavicius2012improving,jambor2010optimizing,zhang2013personalize,wang2009portfolio}. The re-ranking models are computationally efficient but suffer from two drawbacks. First, due to performance considerations,  parameters that balance the trade-off between novelty and accuracy  are not customized per user. Instead they are cross-validated at a global level.  This can be detrimental since users have varying preferences for  objectives such as long-tail novelty. Second,  the re-ranking methods are often limited to a specific base recommender  that may be sensitive to dataset density. 
As a result, the datasets are pruned and the problem is studied in dense settings~\cite{adomavicius2012improving,ho2014likes}; but real world  scenarios are often sparse~\cite{kanagal2012supercharging,liu2017experimental}.   
% Because  dataset density can impact the performance of most base recommenders (like R-SVD), which in turn affects the performance of the re-ranking model, 

\iffalse
We address these limitations by directly inferring  user  preference for long-tail novelty  from interaction data.  This  allows us to customize the re-ranking  per user, and design a \textit{generic} framework, which resolves the second problem. In particular, since the long-tail novelty preferences are estimated independently of any base  recommender model, we can  plug-in an appropriate base recommender w.r.t. the dataset sparsity.% including ones that are more suitable for sparse settings.  

Modelling  user  preference for  long-tail novelty using only item popularity statistics, e.g., the average popularity of rated items as in~\cite{jugovac2017efficient}, disregards additional information like whether the user found the item interesting and the long-tail preferences of other users  of the items. \iffalse To incorporate them, we introduce the notion of  \emph{item long-tail importance}. Both  user long-tail preferences and item long-tail importance are dependent:  a user has high preference for discovering long-tail items if she is interested in important long-tail items, and an item that is associated with many of these kinds of users is likely to be more important.  We propose a joint optimization framework to directly learn,  from interaction data, both the users' long-tail preferences and the  items' long-tail importance. \fi
We propose an optimization approach that  incorporates  this information and  directly learns,  from interaction data, the users' long-tail novelty preferences.

Next, we use these learned preferences  to design a  top-$\size$ recommendation framework thats is generic, and provides customized balance between accuracy, novelty, and coverage. We refer to it as framework as GANC.  Using GANC, we design a novel algorithm, {\it Ordered Sampling-based Locally Greedy (OSLG)\/}, that relies on the learned long-tail novelty preferences  to scalably correct for popularity bias. Our work does not rely on any additional contextual data, although such data, if available, can help promote newly-added long-tail items~\cite{agarwal2009regression,Saveski:2014:ICR:2645710.2645751}. In summary:
\fi

We address the first limitation by directly inferring  user  preference for long-tail novelty  from interaction data.   Estimating these  preferences  using only item popularity statistics, e.g., the average popularity of rated items as in~\cite{jugovac2017efficient}, disregards additional information, like whether the user found the item interesting or the long-tail preferences of other users  of the items. We propose an approach that  incorporates  this information and  learns the users' long-tail novelty preferences from interaction data.

This approach allows us to customize the re-ranking  per user, and  design a \textit{generic} re-ranking framework, which resolves the second limitation of prior work. In particular, since the long-tail novelty preferences are estimated independently of any base recommender, we can  plug-in an appropriate one w.r.t. different factors, such as the dataset sparsity.

Our top-$\size$ recommendation framework, \textbf{GANC}, is \textbf{G}eneric, and provides customized balance between \textbf{A}ccuracy, \textbf{N}ovelty, and \textbf{C}overage. % Moreover, based on the learned long-tail novelty preferences, we also design a novel algorithm, {\it Ordered Sampling-based Locally Greedy (OSLG)\/}, that relies on the learned long-tail novelty preferences  to scalably correct for popularity bias. 
Our work does not rely on any additional contextual data, although such data, if available, can help promote newly-added long-tail items~\cite{agarwal2009regression,Saveski:2014:ICR:2645710.2645751}. In summary:

%Consider  the following toy example:
\vspace{-0.2cm}
\begin{table}[htb]
\centering
\scriptsize
%\small
\begin{tabular}{ccccccc} 
%\toprule
%&\multirow{2}{*}{}&\multicolumn{7}{c}{Ratings}\\
& & \cellcolor{blue!35}$w_1$ &\cellcolor{blue!18} $w_2$ & $\dots$ &\cellcolor{blue!8} $w_{89}$  &\cellcolor{blue!8} $w_{99}$   
\\
&   &$i_1$&$i_2$&$\dots$&$i_{89}$&$i_{90}$\\ 
\cmidrule(r){3-7} 	 
%\midrule
\cellcolor{red!35}$\theta_1$  &$u_1 $   &5 &   & $\dots$ &  &   \\
\cellcolor{red!28}$\theta_2$  &$u_2$     &5 &    & $\dots$ &  &  \\
 $\theta_3=?$  &$\bf u_3$  &5 &  &   $\dots$ &  &  \\
\cellcolor{red!10}$\theta_4$ & $u_4$  &  &5   & $\dots$ & &\\ 
\cellcolor{red!10}$\theta_5$ & $u_5$  &  & 5  & $\dots$ & &\\ 
$\theta_6=?$  & $\bf u_6$ & &5  &      $\dots$& &  \\ 
 & & $\hdots$  &$\hdots$   &$\hdots$   &$\hdots$   &$\hdots$  \\
%\midrule 
\cmidrule(r){3-7} 	 
\multicolumn{2}{c}{item pop.}  & 3  & 3  & $\dots$ &50&60\\  
%\bottomrule
%$ f_i$    &3  &3  &1  &3  &1  &2  \\  \hline
\end{tabular}
%#.
\caption{Simplified user-item interaction data. The user long-tail novelty preference ($\theta_u$), item long-tail importance weight ($w_i$) are highlighted. Darker colors indicate larger values. } \label{tab:example}
\end{table} 
\vspace{-0.2cm}
\begin{example}  
In Table~\ref{tab:example}, we are interested in estimating $\theta_3$ and $\theta_6$,  the long-tail preference of users $u_3$ and $u_6$ who have each rated a single movie. Additional ratings for other users  are not included here.  Considering only rating information, we observe $i_1$ and $i_2$ are  equally popular $|\mathcal{U}_{i_1}^{\trainset}| = |\mathcal{U}_{i_2}^{\trainset}|=3$, and $r_{31}=5$ and $r_{62}=5$. Using Eq.~\ref{eq:tfidf-risk}  we have $\theta_3 = \theta_6$. However, if we were given the long-tail preferences of the each item's user set, specifically that $u_1$ and $u_2$ have high long-tail preference (darker red), while $u_4$ and $u_5$ have lower long-tail preference (lighter red), we could conclude $i_1$ is a more important long-tail item compared to $i_2$ (indicated by a darker blue shade for $w_1$), and we expect  $\theta_3 \geq \theta_6$.

% On the other hand, if we knew that $u_4$ and $u_5$ have lower long-tail preference, we could conclude $i_2$ is a  less significant long-tail item. Therefore, However, if we  consider the long-tail preferences of other users, we may reason differently.    We need another variable $w_i$ which captures this information. 
%we would conclude that $u_3$ has higher long-tail preference compared to $u_6$, since the users $i_1$ is a more prominent long-tail item. 

% Relying only  on item popularity information, we would  conclude   $u_3$ and $u_6$ have equal long-tail preference, since $i_1$ and $i_2$ are  equally popular. However, considering  the second column,  long-tail preference of users,  long-tail importance for each item,  which captures the long-tail preference of its users. Since  that  both users of $i_1$ have high long-tail preference while  the users of $i_2$ have lower preference,  we may conclude $i_1$ is a more important long-tail item compared to $i_2$. Therefore, $u_3$'s long-tail preference should be at least as large as $u_6$'s preference. Specifically, consider two  items $i_1$ and $i_2$, with the following rating data: $i_1=\{u_1:5, u_2:5, u_3:5 \}$, $i_2=\{u_4:5, u_5:5, u_6:5\}$.  

%Table~\ref{tab:example} shows  simplified rating data. We want an estimate of the long-tail preference of $u_3$ and $u_6$, who have each  rated a single movie.  Relying only  on movie popularity information, we would  conclude   $u_3$ and $u_6$ have similar long-tail preference, since $m_1$ and $m_2$ are  equally popular. However, considering the long-tail preferences of other users of those movies, we may reason differently: since $u_1$ and $u_2$ have high long-tail preference, and $u_4$ and $u_5$ have low long-tail preference, $m_1$ is a more prominent long-tail item compared to $m_2$. Therefore, it is likely that $u_3$ has higher long-tail preference compared to $u_6$.considering the long-tail preferences of other users of those movies, we may reason differently.  For example, 
\label{ex:running}
\end{example}



%------------------------------

\iffalse
\begin{example}
Table~\ref{tab:example} shows rating data for a simplified system. %Note the user-item interaction matrix is sparse.
For this example, we define popular movies as those that have received  three or more ratings; $\{m_1, m_2, m_4\}$ are popular and  $\{m_3, m_5, m_6\}$ are niche movies. We observe $u_1$ and $u_3$  have rated relatively popular movies (risk-averse) while $u_2$ and $u_4$ have rated niche movies (risk-loving). 
\label{ex:running}
\end{example}

\begin{table}[htb]
\centering
\scriptsize
\begin{tabular}{ccccccc} 
\toprule
			&$m_1$ &$m_2$   &$m_3$    &$m_4$   &$m_5$ &$m_6$  \\ \hline 
$u_1 $ &5  &4  & - &-  &-  &-   \\
$u_2$  &-  &-  &-  &-  &5  &5   \\
$u_3$  &-  &4  &-  &5  &-  &-   \\
$u_4$  &-  &-  &3  &-  &-  &4   \\ 
$u_5$  &5  &-  &-  &3  &-  &-   \\ 
$u_6$  &4  &2  &-  &4  &-  &-   \\ 
\bottomrule
%$ f_i$    &3  &3  &1  &3  &1  &2  \\  \hline
\end{tabular}
\caption{User-Movie rating data} \label{tab:example}
\end{table}

It is essential to consider consumer characteristics in designing recommender systems so that they promote long-tail items to the right group of users and spread demand evenly between hit and niche items.  

\fi





%------------------------------
\iffalse
\begin{table}[htb]
\centering
\scriptsize
\begin{tabular}{ccccccc} 
\toprule
			&$m_1$ &$m_2$   &$m_3$    &$m_4$   &$m_5$ &$m_6$  \\ \hline 
$u_1 $ &\textbf{5}  & \textbf{4}  &\textcolor{gray}{ 1.2} &-  &-  &-   \\
$u_2$  &-  &-  &-  &-  & \textbf{5}  &\textbf{5}   \\
$u_3$  &-  &\textbf{4}  &-  &\textbf{5}  &-  &-   \\
$u_4$  &-  &-  &\textbf{3}  &-  &-  &\textbf{4}   \\ 
$u_5$  &\textbf{5}  &-  &-  &\textbf{3}  &-  &-   \\ 
$u_6$  &\textbf{4}  &\textbf{2}  &-  &\textbf{4}  &-  &-   \\ 
\bottomrule
%$ f_i$    &3  &3  &1  &3  &1  &2  \\  \hline
\end{tabular}
\caption{User-Movie rating data} \label{tab:example}
\end{table}
% $\mathcal{P}^1= \{ \mathcal{P}_1^1 \{i_1,i_2,i_3\}, \mathcal{P}_2^1:\{i_2,i_3,i_5\}  \}$
 %$\mathcal{P}^2= \{ \mathcal{P}_1^2: \{i_1,i_2,i_3\}, \mathcal{P}_2^2:\{i_2,i_5,i_6\}  \}$
 %$\mathcal{P}^3= \{ \mathcal{P}_1^3: \{i_7,i_8,i_9\}, \mathcal{P}_2^3:\{i_{10},i_{11},i_{12}\}  \}$
\begin{table}[htb]
\centering
\tiny
\begin{tabular}{ccc} 
\toprule
		&$u_1$&$u_2$  \\ \hline 
$\mathcal{P}^1 $ & $\{i_1,i_2,i_3\}$ & $\{i_2,i_3,i_5\} $ \\
$\mathcal{P}^2$ & $\{i_1,i_2,i_3\}$ & $\{i_2,i_5,i_6\} $ \\
$\mathcal{P}^3$ & $\{i_7,i_8,i_9\}$ & $\{i_{10},i_{11},i_{12} \}$ \\
\bottomrule
%$ f_i$    &3  &3  &1  &3  &1  &2  \\  \hline
\end{tabular}
\caption{Top-$\size$ allocations to users.} \label{tab:paretoExamples}
\end{table}
\fi


\iffalse
When considering long-tail items, it is important to consider consumers' willingness  to explore niche or unpopular items and their propensity towards similar items. In particular, they can be characterized by their  {\it risk degree\/} and {\it focusing degree\/}, respectively.  We compute these estimates  based on historical rating information. The following example further describes these notions in the context of movie rating data. 

\begin{example}  
Table~\ref{tab:example} shows rating data for a simplified system with $6$ users, $6$ movies, and $3$ genres. $m_i^{j}$ implies that movie $m_i$ belongs to genre $j$. Note the user-item interaction matrix is sparse. 
  For this setting, we define popular movies as those that have received  three or more ratings; $\{m_1, m_2, m_4\}$ are popular and  $\{m_3, m_5, m_6\}$ are niche movies. We now profile the users according to their risk and focusing degree. E.g., $u_1$ has rated relatively popular movies belonging to the same genre (risk-averse, high focusing degree); $u_2$ has rated niches movies in the same genre (risk-loving, high focusing degree); $u_3$ has rated popular movies in two different genres (risk-averse, low focusing degree), and $u_4$ has rated niches movies in two different genres (risk-loving, low focusing degree). 
\label{ex:running}
\end{example}
\begin{table}[htb]
\centering
\tiny
\begin{tabular}{ccccccc} 
\toprule
			&$m_1^{1}$ &$m_2^{1}$   &$m_3^{2}$    &$m_4^{3}$   &$m_5^{3}$ &$m_6^{3}$  \\ \hline 
$u_1 $ &5  &4  &-  &-  &-  &-   \\
$u_2$  &-  &-  &-  &-  &5  &5   \\
$u_3$  &-  &4  &-  &5  &-  &-   \\
$u_4$  &-  &-  &3  &-  &-  &4   \\ 
$u_5$  &5  &-  &-  &3  &-  &-   \\ 
$u_6$  &4  &2  &-  &4  &-  &-   \\ 
\bottomrule
%$ f_i$    &3  &3  &1  &3  &1  &2  \\  \hline
\end{tabular}
\caption{User-Movie rating data} \label{tab:example}
\end{table}
It is essential to consider these consumer characteristics in designing recommender systems so that they promote long-tail items to the right group of users and spread demand evenly between the hit and niche items.  
\fi
\iffalse
\begin{center}
\begin{figure*}[tp]
%\scalebox{0.5}{%
\resizebox{1\textwidth}{!}{%
%\small%\addtolength{\tabcolsep}{5pt}% below sums to 8
\begin{tabularx}{1.5\textwidth}{>{\hsize=2.5\hsize}X>{\hsize=2.5\hsize}X>{\hsize=0.5\hsize}X>{\hsize=0.5\hsize}X>{\hsize=0.5\hsize}X>{\hsize=0.5\hsize}X>{\hsize=0.5\hsize}X>{\hsize=0.5\hsize}X}
    \multirow{12}{*}{\includegraphics[scale=0.3]{codeForExample/popularity-movie.png}} & \multirow{12}{*}{\includegraphics[scale=0.3]{codeForExample/scatterplot.png}} & & & & & & \\
%   & &               &       &       &       &       &       \\
    & &\multicolumn{1}{l|}{}               &$m_1^{g1}$   	&$m_2^{g1}$    	&$m_3^{g2}$    &$m_4^{g2}$      &$m_5^{g3}$    \\ \cline{3-8}%\hline
    & &\multicolumn{1}{l|}{u1}          &5  &5  &-  &-   &-  \\
    & &\multicolumn{1}{l|}{u2}    		&-  &-  &4  &4  &5  \\
    & &\multicolumn{1}{l|}{u3}   			&1  &2  &1  &-  &-   \\
    & &\multicolumn{1}{l|}{u4}     		&1  &-  &-  &-  &-  \\
    & &               &       &       &       &       &       \\
    & &               &       &       &       &       &       \\
    & &               &       &       &       &       &       \\
    & &               &       &       &       &       &	\\
    \\
\end{tabularx}}
\caption{User-Movie interaction data a) Popularity-Movie histogram b)Movie genres/clusters c) User-Movie rating data} \label{fig:example}
\end{figure*}
\end{center}
\fi



%We propose a novel approach that allows us to  promote long-tail items in a targeted manner, thereby improving the novelty of top-$\size$ sets, the overall item-space coverage across recommendations, while maintaining reasonable levels of accuracy.

%Next, we integrate these learned preferences  in a generic  top-$\size$ recommendation framework to provide customized balance between accuracy and coverage.

%sequentially make recommendations, while adjusting its parameters with regard to the set of top-$\size$ recommendations made so far. However, since  sequential parameter updates  cause  scalability issues, we propose a sampling based algorithm. This variant of our framework, called {\it Ordered Sampling-based Locally Greedy (OSLG)\/},  allows us to  correct for the popularity bias in recommendations with regard to individual user long-tail preferences. 

%ICDE submission
%Our framework differs with  prior work in the following aspects:  unlike~\cite{adomavicius2011maximizing,adomavicius2012improving,zhang2013personalize,ho2014likes},  the long-tail preference personalization in our framework is learned rather than optimized using cross-validation or parameter tuning. In other words, our personalization method is independent of the underlying base  recommendation models.  Moreover, our framework is  generic. This enables us to  plug-in several base recommenders, and evaluate their  effectiveness without requiring  extensive tuning for the accuracy and coverage trade-off. 


%\vspace{-2.8pt}
\begin{itemize}

\item  We examine various measures for estimating user long-tail novelty preference in Section~\ref{sec:lt-pref} and formulate an optimization problem  to directly learn users' preferences for long-tail  items from interaction data in Section~\ref{sec:learning-lt-pref}. %In addition, we introduce several heuristics for measuring the user preference for less common items from historical rating data.% 

\item  We integrate the user preference estimates into GANC %, a generic re-ranking framework that provides customized balance between accuracy, novelty, and coverage 
(Section~\ref{sec:RiskbasedReranking}), and  introduce {\it Ordered Sampling-based Locally Greedy (OSLG)\/}, a scalable algorithm that relies  on user long-tail preferences to correct the popularity bias (Section~\ref{sec:optimizationAlgorithm}).
%We introduce OSLG, a scalable algorithm that relies  on user long-tail preferences to  maximize item space coverage \textcolor{red}{while maintaining acceptable levels of accuracy} (Section~\ref{sec:optimizationAlgorithm}).

\item   We conduct an extensive empirical study and evaluate performance from  accuracy, novelty, and coverage perspectives (Section~\ref{sec:Experiments}).  We use five  datasets with varying density and difficulty levels. %:  Netflix, MovieTweetings, and MovieLens (100K, 1M, 10M). 
  In contrast to most related work,  our evaluation considers realistic settings that include a large number of infrequent  items and users. %This enables us to study the impact of  data density on the performance trade-offs of several  state of the art top-$\size$ recommendation algorithms. %   %,  and use the all-items ranking protocol~\cite{steck2013evaluation,vargas2014improving}, where performance is measured using all items with train data. to evaluate the performance of several  state of the art top-$\size$ recommendation algorithms 
 
\item Our empirical results confirm that the performance of re-ranking models is impacted by the underlying   base recommender and the dataset density. Our generic approach enables us to easily incorporate a suitable base recommender to devise an effective solution for both dense and sparse settings. In dense settings, we use the same base recommender as existing re-ranking approaches, and we outperform them in accuracy and coverage metrics. For sparse settings, we plug-in a more suitable base recommender, and devise an effective solution that is competitive with existing top-$\size$ recommendation methods in accuracy and novelty. 

%Directly estimating the long-tail novelty preferences allows us to customize re-ranking per user, and  devise a generic framework.   
 
\end{itemize}

Section~\ref{sec:related-work} describes related work. Section~\ref{sec:conclusion} concludes.


\section{The causal structure of doubly warped spacetimes}
\label{sec:chronologicalrelation}
%By a {\em \multiwarped spacetime} we understand a lorentzian manifold $(V,g)$ where
%\begin{equation}
%  \label{eq:1}
%  V:= \R \times M_{1} \times M_{2}\quad\hbox{and}\quad
%g=-dt^{2}+\alpha_{1}g_{1}+\alpha_{2}g_{2},
%\end{equation}
%being $(M_{i},g_{i})$ a Riemannian manifold for $i=1,2$.
%
%\smallskip

In this section we are going to characterize the chronological and causal relations in doubly warped spacetimes. First, recall that a {\em \multiwarped spacetime} is a multiwarped spacetime $(V,g)$ as in (\ref{eqqq}) with two fibers ($n=2$), that is,

\begin{equation}
  \label{eq:1-aux}
  V:= (a,b)\times M_{1} \times M_{2}\quad\hbox{and}\quad
g=-dt^{2}+\alpha_{1}g_{1}+\alpha_{2}g_{2}.
\end{equation}
%where $\alpha_i:\R\rightarrow (0,\infty)$ are positive smooth functions and $(M_{i},g_{i})$ are Riemannian manifolds for $i=1,2$.
%\begin{rem}\label{rem:infinito}
%Since the structures studied in this paper are invariant by conformal transformations, our analysis implicitly covers the case where the temporal interval $(a,b)$ is a proper subset of $\R$. In fact, otherwise, multiply the metric by an appropriate positive function in order to obtain another doubly warped spacetime in the same conformal class with $(a,b)=\R$.
%\end{rem}




%\begin{rem}\label{rem:infinito}
%  As we are going to be interested in causal structures, there is no loss of generality if we assume that $(a,b)\equiv \R$. In fact, given a metric as in \eqref{eq:1-aux}, let us consider $\beta:(a,b)\rightarrow (0,\infty)$ a function satisfying that:
%  \[
%\int_{c}^b \frac{1}{\sqrt{\beta(t)}}dt=\infty=\int_{a}^c\frac{1}{\sqrt{\beta(t)}}dt
%    \]
%    for some $c\in (a,b)$. Then,
%    \[
%      \begin{array}{rl}
%        {\mathfrak g}=& -dt^{2}+\alpha_{1}g_{1}+\alpha_{2}g_{2}\\ =& \beta \left(-\left(\frac{dt}{\sqrt{\beta}}\right)^2+ \frac{\alpha_1}{\beta}g_1+\frac{\alpha_2}{\beta}g_2\right)\\ = & \beta \left(-ds^2+\tilde{\alpha}_1g_1+\tilde{\alpha}_2g_2   \right) \\ =&\beta\, g,
%      \end{array}
%      \]
%      where $s\in \R$ as, from definition, $ds=dt/\sqrt{\beta}$. Moreover,
%
%      \[
%\int_{0}^{\infty} \frac{1}{\sqrt{\tilde{\alpha}_i(s)}}ds=\int_{c}^{a} \frac{1}{\sqrt{\alpha_i(t)}}dt, \qquad \int_{0}^{-\infty} \frac{1}{\sqrt{\tilde{\alpha}_i(s)}}ds=\int_{c}^{b} \frac{1}{\sqrt{\alpha_i(t)}}dt, \quad\hbox{ for $i=1,2$}.
%        \]
%        In conclusion, the metric $\mathfrak{g}$ is conformal to $g$, which is also a {\multiwarped} model where the time component belongs to the entire $\R$ and with warping functions preserving some integral condition. Therefore, unless stated otherwise (essentially, on the main results), we will always work considering {\multiwarped} models as follow:
%
%        \begin{equation}
%  \label{eq:1}
%  V:= \R \times M_{1} \times M_{2}\quad\hbox{and}\quad
%g=-dt^{2}+\alpha_{1}g_{1}+\alpha_{2}g_{2},
%\end{equation}
%
%\end{rem}
%}


Take $(t^e,x^e)\in V$ and $x^o\in M:=M_1\times M_2$. Denote by $C(x^o,x^e)$ the set of smooth curves in $M$ connecting $x^o$ with $x^e$. Given $c=(c_1,c_2)\in C(x^o,x^e)$, consider the unique future-directed lightlike curve $\rho:[s^o,s^e]\rightarrow V$ with $\rho(s)=(\tau_{c,t^e}(s),c(s))$ and $\tau_{c,t^e}(s^e)=t^e$. From the metric expression in (\ref{eq:1-aux}), the component $\tau_{c,t^e}(s)$ is determined by the Cauchy problem
\[
-\dot{\tau}_{c,t^e}^2+\alpha_1(\tau_{c,t^e})g_{1}(\dot{c}_1,\dot{c}_1)+\alpha_2(\tau_{c,t^e})g_{2}(\dot{c}_2,\dot{c}_2)=0,\qquad
    \tau_{c,t^e}(s^e)=t^e.
  \]
Consider the functional
\[\J_{x^o,(t^e,x^e)}: C(x^o,x^e) \rightarrow (a,b), \quad c \mapsto \tau_{c,t^e}(s^o).\]
A direct computation shows that $(t^o,x^o)\ll (t^e,x^e)$ if, and only if, there exists $c\in C(x^o,x^e)$ such that $t_o<\J_{x^o,(t^e,x^e)}(c)$. This property suggests the following definition for the {\em departure time function}:
\[
T:M\times \left((a,b)\times M\right)\rightarrow (a,b),\qquad T(x^{o},(t^{e},x^{e})):= {Sup}_{C}\J_{x^o,(t^e,x^e)}
\]
(compare with \cite[Section 2.9]{Perlick2004} and \cite[Section 4]{FS2}). By construction, this function characterizes the chronological relation in $(V,g)$, as follows:
\begin{equation}\label{e0}
(t^{o},x^{o}) \ll (t^{e},x^{e}) \;\; \Longleftrightarrow \;\;
t^{o}<T(x^{o},(t^{e},x^{e})).
\end{equation}
In particular, the chronological past of a given point $(t^e,x^e)$ is given by
\[
I^-\left((t^e,x^e) \right):=\{(t,x)\in (a,b)\times M: t<T(x,(t^e,x^e)) \}.
  \]
Given a future-directed timelike curve $\gamma(t)=(t,c(t))$, $t\in [\omega,\Omega)$, and a point $x\in M$, the transitivity of the chronological relaction $\ll$ ensures that the function $T(x,\gamma(t))$ is increasing on $t$. Hence, the chronological past of $\gamma$ can be written as
  \[
I^-(\gamma)=\{(s,x)\in (a,b)\times M: s<b_c(x):=lim_{t\rightarrow b}T(x,\gamma(t))\}.
    \]
%    \begin{rem}
%      The choice of $b_c$ to denote such a limit came motivated from the studies of the causal boundary for standard static  spacetimes, that can be considered as... In fact, as it can be deduced from CITA REQUERIDA, $T(x,(t^*,y))=t^*-d(x,y)$, and so, when...\footnote{Simplemente concretar lo anterior al caso estático estandar, a modo de ejemplo.}.
%    \end{rem}

Next, let us characterize the departure time function, and so, the chronological relations (recall (\ref{e0})), in terms of some integral conditions involving the warping functions $\alpha_i$ and the Riemannian distances $d_i$ associated to the fibers $(M_i,g_i)$, $i=1,2$.
To this aim, let us consider a future-directed causal curve $\gamma:  I \rightarrow V$,
$\gamma(s)=(t(s),c_{1}(s),c_{2}(s))$. From the metric expression in \eqref{eq:1-aux}:
\[
\frac{dt}{ds}(s)=\sqrt{-D+\frac{\mu_{1}}{\alpha_{1}\circ
t}+\frac{\mu_{2}}{\alpha_{2}\circ t}}(s),
\]
where $D:=g(d\gamma/ds,d\gamma/ds)\leq 0$ and
$\mu_{i}:=(\alpha_{i}\circ t)^2 g_{i}(dc_{i}/ds,dc_{i}/ds)$,
$i=1,2$. From the Inverse Function Theorem, previous formula translates into
\[
\frac{ds}{dt}(t)=\left(-(D\circ s)+\frac{\mu_{1}\circ s}{\alpha
_{1}}+\frac{\mu_{2}\circ s}{\alpha _{2}}\right)^{-1/2}.
\]
Therefore, if we denote $t^{o}=t(s^{o})$, $t^{e}=t(s^{e})$, we
deduce
\begin{equation}\label{eq:3}
\begin{array}{c}
\hbox{length}\left(c_{i}\mid_{[s^{o},s^{e}]}\right)=\int_{s^{o}}^{s^{e}}\sqrt{g_{i}(\dot
c_{i}, \dot c_{i})} ds=\int_{t^o}^{t^e}\sqrt{g_{i}(\dot c_{i}, \dot
c_{i})} \frac{ds}{dt} dt \qquad\qquad\qquad\quad \\
\;\quad\qquad\qquad\qquad =\int_{t^o}^{t^e}\frac{\sqrt{\mu_{i}\circ
s}}{\alpha_i(t)}\left(-(D\circ s)+\frac{\mu_{1}\circ
s}{\alpha_{1}(t)}+\frac{\mu_{2}\circ s}{\alpha_{2}(t)}\right)^{-1/2}dt
\qquad\hbox{for}\;\; i=1,2.
\end{array}
\end{equation}
In the particular case of being $\gamma$ a lightlike geodesic we have: (i) $D=0$ (lightlike character of $\gamma)$, (ii) $\mu_i\circ s$ are constants and (iii) $c_i$ are (pre-)geodesics on the corresponding Riemannian manifold $(M_i,g_i)$ (geodesic character of $\gamma$). So, from (\ref{eq:3}), one deduces (see \cite[Theorem 2]{FS} for details):
\begin{prop}\label{thm:characluzgeodesics}
  Let $(V,g)$ be a {\multiwarped} spacetime as in (\ref{eq:1-aux}) with (weakly) convex fibers (i.e., satisfying that any pair of points can be joined by some minimizing geodesic). Consider two distinct points $(t^o,x_1^o,x_2^o),(t^e,x_1^e,x_2^e)\in V$ with $t^o<t^e$. Then, the following statements are equivalent:
  \begin{itemize}
  \item[(a)] There exists a lightlike geodesic joining $(t^o,x_1^o,x_2^o)$ and $(t^e,x_1^e,x_2^e)$.
  \item[(b)] There exist $\mu_1,\mu_2 \ncambios{\geq}0$ with $\mu_1+\mu_2=1$ such that
    \[
\Integral{t^o}{t^{e}}{\mu_{i}}{i}{\mu_{k}}=d_{i}(x^{o}_{i},x^{e}_{i})\qquad\hbox{for}\;\;
i=1,2;
      \]

  \end{itemize}

\end{prop}


%This equality will be important in the forthcoming result, since it allows to explicitly compute the length of the projection of a causal curve over any of the Riemannian fibers.
%    \begin{lemma}
%\label{lightlikecurve}
%Let $(t^{o},x_1^o,x_2^o)$, $(t^e,x_1^{e},x_2^e)$  be two points of a \multiwarped spacetime $(V,g)$ as in \eqref{eq:1} such that $(x_1^o,x_2^o) \neq (x_1^{e},x_2^e)$. Assume the existence of positive values $\mu_{1},\mu_{2} \in\R$, with $\mu_{1}+\mu_{2}=1$, such that
%\begin{equation}\label{eq:2}
%\Integral{t^o}{t^e}{\mu_{i}}{i}{\mu_{k}}>d_{i}(x_{i}^o,x_{i}^{e}) \qquad\hbox{for}\;\;
%i=1,2.
%\end{equation}
%Then $(t^{o},x_1^o,x_2^o) \ll (t^{e},x_1^e,x_2^e)$, i.e., there exists a timelike future-directed curve between $(t^{o},x_1^o,x_2^o)$ and $(t^{e},x_1^e,x_2^e)$.
%\end{lemma}
%
%\begin{proof} For any $\epsilon>0$, let us denote
%  \[
%L^\epsilon_i:=\Integral{t^o+\epsilon}{t^e}{\mu_{i}}{i}{\mu_{k}}.
%    \]
% Take $\epsilon>0$ small enough so that the inequalities in \eqref{eq:2} still hold for $t_0+\epsilon$ instead of $t_0$. There exist curves $y_i:[s^0,s^e]\rightarrow M_i$, with $y_{i}(s^{o})=x_{i}^{o}$ and $y_{i}(s^{e})=x^{e}_{i}$, such that $length(y_{i})=L^\epsilon_{i}$, $i=1,2$. Next, consider the lightlike curve
%$\rho(s)=(\tau(s),\overline{y}_{1}(s),\overline{y}_{2}(s))$,
%with $\overline{y}_{i}$ reparametrizations of $y_{i}$,
%constructed by
%requiring
%\[
%\left\{\begin{array}{l}\dot{\tau}=\sqrt{\sum_{i=1}^{2}\frac{\mu_{i}}{\alpha_{i}\circ\tau}}
%\\ \tau(s^{e})=t^{e}
%\end{array}\right.,\qquad
%\left\{\begin{array}{l}g_{i}(\dot{\overline{y}}_{i},\dot{\overline{y}}_{i})=\frac{\mu_{i}}{(\alpha_{i}\circ
%\tau)^{2}} \\
%\overline{y}_{i}(s^{e})=x^{e}_{i}\end{array}\right.
%\qquad\hbox{for}\;\; i=1,2.
%\]
%Then, by applying \eqref{eq:3} to the lightlike curve $\rho$ (thus, $D=0$), we deduce
%\[
%\hbox{length}(\overline{y}_{i}\mid_{[\tau^{-1}(t^o+\epsilon),s^{e}]})=\Integral{t^o+\epsilon}{t^{e}}{\mu_{i}}{i}{\mu_{k}}
%%\int^{t^{e}}_{T'_{m}}\sqrt{\mu_{i}}\alpha_{i}^{-1}\left(\sum_{j=1}^{n}\frac{\mu_{j}}{\alpha_{j}}\right)^{-1/2}dt
%=L^\epsilon_{i}=\hbox{length}(y_{i}).
%\]
%Therefore, $\rho(s)$ is a lightlike curve joining $(t^o+\epsilon,x^{o})$ with
%$(t^{e},x^{e})$, and so, these points are causally related. Since $(t^0+\epsilon,x^0)\ll (t^0,x^0)$, it directly follows that $(t^0,x^0)\ll (t^{e},x^{e})$.
%\end{proof}
%
%In order to complete the characterization of the chronological relation, we will use the following characterization of the departure time function:
%
%%Assume just for a second that we have already obtained such an equivalence. Given $x^0,x^e\in M$ and $t^e$, the departure time function $T(x^0,(t^e,x^e))$ determines the exact point where the line $(t,x^0)$ with $t\in \R$ leaves the past of $(t^e,x^e)$. In particular, for all $t^0<T(x^0,(t^e,x^e))$ we can obtain $\mu_1,\mu_2$ so the inequalities on \eqref{eq:2} are satisfied. Therefore, at least intuitively, the departure time function should be in the edge of satisfying such inequalities, being expected that it satisfies a relation as in \eqref{eq:2} (substituing $t^0$ with $T(x^0,(t^e,x^e))$) but with equalities. The next result is a formalization of this intuitive idea:
%
%\begin{prop}
%\label{p0}
%Let $(V,g)$ be a \multiwarped spacetime and $(x^{o},(t^{e},x^{e}))\in M \times V$, with $x^{o}=(x_1^o,x_2^o)\neq (x_1^e,x_2^e)=x^{e}$. If $T=T(x^{o},(t^{e},x^{e}))>-\infty$ then $\varsigma=T$ is the unique real value satisfying
%\begin{equation}
%\label{e*}
%\Integral{\varsigma}{t^{e}}{\mu_{i}}{i}{\mu_{k}}=d_{i}(x^{o}_{i},x^{e}_{i}),\qquad
%i=1,2,
%\end{equation}
%for some (unique) constants $\mu_{1},\mu_{2} \geq 0$ with
%$\mu_{1}+\mu_{2}=1$.
%\end{prop}
%
%\begin{proof}
% First, let us show that if $\varsigma=T'$
%satisfies (\ref{e*}) then necessarily $T'\leq T$. To this aim,
%take any sequence $\epsilon_{m} \searrow 0$ and define for $i=1,2$
%\[
%L_{i,m}:=\Integral{T'_{m}}{t^{e}}{\mu_{i}}{i}{\mu_{k}}
%\qquad
%\hbox{with}\;\; T'_{m}:=T'-\epsilon_{m}.
%\]
%From the choice for $T'_m$, we have that $L_{i,m}>d_{i}(x_{i}^o,x_{i}^{e})$ for all $i$. So, Lemma \ref{lightlikecurve} ensures that
%$(T'_{m},x^o) \ll (t^e,x^e)$ for all $m$. Therefore, from the definition of $T(x^o,(t^e,x^e))$, $T'-\epsilon_{m}<T$ for all $m$, and then, $T' \leq T$.
%
%\medskip
%%From the hypothesis for $T'$, there exist curves $y_{i,m}(s)$ in
%%$M_{i}$ with lengths $L_{i,m}$ such that
%%$y_{i,m}(s^{o})=x_{i}^{o}$ and $y_{i,m}(s^{e})=x^{e}_{i}$.
%%Consider the lightlike curves
%%$\rho_{m}(s)=(\tau(s),\overline{y}_{1,m}(s),..., \overline{y}_{n,m}(s))$,
%%with $\overline{y}_{i,m}$ reparametrizations of $y_{i,m}$,
%%constructed by
%%requiring
%%\[
%%\left\{\begin{array}{l}\dot{\tau}=\sqrt{\sum_{i=1}^{2}\frac{\mu_{i}}{\alpha_{i}\circ\tau}}
%%\\ \tau(s^{e})=t^{e}
%%\end{array}\right.,\qquad
%%\left\{\begin{array}{l}g_{i}(\dot{\overline{y}}_{i,m},\dot{\overline{y}}_{i,m})=\frac{\mu_{i}}{(\alpha_{i}\circ
%%\tau)^{2}} \\
%%\overline{y}_{i,m}(s^{e})=x^{e}_{i}\end{array}\right.
%%\qquad\hbox{for}\;\; i=1,2.
%%\]
%%Then,
%%\[
%%\hbox{length}(\overline{y}_{i,m}\mid_{[\tau^{-1}(T'_{m}),s^{e}]})=\Integral{T'_{m}}{t^{e}}{\mu_{i}}{i}{\mu_{k}}
%%%\int^{t^{e}}_{T'_{m}}\sqrt{\mu_{i}}\alpha_{i}^{-1}\left(\sum_{j=1}^{n}\frac{\mu_{j}}{\alpha_{j}}\right)^{-1/2}dt
%%=L_{i,m}=\hbox{length}(y_{i,m}).
%%\]
%%Therefore, $\rho_{m}(s)$ joins $(T'_{m},x^{o})$ with
%%$(t^{e},x^{e})$, and so, the points $(T'_{m-1},x^{o})$,
%%$(t^{e},x^{e})$ are chronologically related for all $m$. According
%%to (\ref{e0}), this implies $T'<T+\epsilon_{m-1}$ for all $m$, and
%%thus, $T'\leq T$.
%
%Next, let us prove that some value $\varsigma=T'$
%verifying (\ref{e*}) always exists, and necessarily $T'\geq T$.
%From (\ref{e0}), $(t^{o},x^{o})\ll (t^{e},x^{e})$ for any
%$-\infty<t^{o}<T(x^{o},(t^{e},x^{e}))$. Let $\gamma:[t^0,t^e]\rightarrow V$,
%$\gamma(t)=(t,c_{1}(t), c_{2}(t))$, be a timelike curve such
%that $\gamma(t^{o})=(t^{o},x^{o})$ and $\gamma(t^{e})=(t^{e},x^{e})$.
%Consider real curves $\overline{c}_{i}$, $i=1,2$, such that
%\[
%\left\{\begin{array}{l} 0\leq\dot{\overline{c}}_{i}(t)\leq
%\sqrt{g_{i}(\dot{c}_{i}(s),\dot{c}_{i}(s))} \\
%\overline{c}_{i}(t^{o})=0 \\
%\overline{c}_{i}(t^{e})=d_{i}(x_{i}^{o},x_{i}^{e})
%\end{array}\right. \qquad\hbox{for}\;\; i=1,2.
%\]
%Then, $\overline{\gamma}(s)=(t,\overline{c}_{1}(t),\overline{c}_{2}(t))$
%becomes a future directed timelike curve in the globally hyperbolic \multiwarped
%spacetime
%$V'=(\R \times \R^{2},-dt^{2}+\alpha_{1}dx_{1}^{2}+\alpha_{2}dx_{2}^{2})$ joining $\overline{\gamma}(t^o)=\point{t^o}{0}{0}$ with $\overline{\gamma}(t^e)=\point{t^e}{d_{1}(x_{1}^o,x_{1}^{e})}{d_{2}(x_{2}^o,x_{2}^e)}$, i.e., \[\overline{\gamma}(t^o)=(t^o,0,0)\ll (t^e,d_{1}(x_{1}^o,x_{1}^{e}),d_{2}(x_{2}^o,x_{2}^{e})=\overline{\gamma}(t^e).\] Now consider $T'$ such that $(T',0,0)\leq \overline{\gamma}(t^e)$ but $(T',0,0)\not\ll \overline{\gamma}(t^e)$. From
%Avez and Seifert's result, there exists some lightlike geodesic in $\R\times \R^2$
%joining both points. Now, \cite[Theorem 2]{FS} (see also \cite[Lemma 3]{FS}) applied to the lightlike geodesic in $V'$ ensures that there exist unique positive constants $\mu_1,\mu_2$ with $\mu_1+\mu_2=1$ and such that
%
%\[
%\Integral{T'}{t^{e}}{\mu_{i}}{i}{\mu_{k}}=|d_{i}(x^{o}_{i},x^{e}_{i})-0|=d_{i}(x^{o}_{i},x^{e}_{i}) \qquad\hbox{for}\;\;
%i=1,2.
%  \]
%Finally, observe that $t^o<T'$ for all $t^o<T$, so in particular $T\leq T'$, which concludes the proof.
%
%% . \cambios{Denote by $\lambda(s)=(\tau(s),c_{1}(s),c_{2}(s))$ ($s \in [0,1]$) the
%% unique lightlike geodesic, and recall that a geodesic in a \multiwarped satisfies the following: (a) each $\mu_{i}=\alpha_{i}(\tau(s))^{2}g_{i}(c_{i}',c_{i}')$ is constant and (b) $c_{i}(s)$ is a pregeodesic in $(\mathbb{R},+dx_{i}^{2})$ for all $i$, see \cite{FS}[Eqns. (5), (6)]. Applying a change of variables in the following integral equalities by using the fact that $\frac{d\tau}{ds}=\sqrt{\sum_{i=1}^{2} \frac{\mu_{i}}{\alpha_{i} \circ \tau}}$ leads to:
%
%% \begin{equation}
%% \label{e7}
%% \hbox{length}(c_{i})=\int_{0}^{1}\sqrt{g_{i}(c_{i}',c_{i}')}dr=\Integral{T'}{t^{e}}{\mu_{i}}{i}{\mu_{k}}=d_{i}(x^{o}_{i},x^{e}_{i})\qquad\hbox{for
%% all}\;\; i
%% %\int^{t^{e}}_{T'}\sqrt{\mu_{i}}\alpha_{i}^{-1}\left(\sum_{j=1}^{n}\frac{\mu_{j}}{\alpha_{j}}\right)^{-1/2}dt=d_{i}(x^{o}_{i},x^{e}_{i})\qquad\hbox{for
%% %all}\;\; i
%% \end{equation}
%% }
%% \cambios{Note that $\mu_{1}$ and $\mu_{2}$ are unique for $T'$, in fact, if $\mu_{1}'$ and $\mu_{2}'$ are different to $\mu_{1}$ and $\mu_{2}$ then we have two possibilities: (1) $\mu_{1}'<\mu_{1}$ and $\mu_{2}<\mu_{2}'$ or (2) $\mu_{1}<\mu_{1}'$ and $\mu_{2}'<\mu_{2}$. Any of the previous cases imply that for some $i_{0} \in \{1,2\}$ the following strict inequality is obtained:
%% \[
%% \Integral{T'}{t^e}{\mu_{i_{0}}'}{i_{0}}{\mu_{k}'}>\Integral{T'}{t^e}{\mu_{i_{0}}}{i_{0}}{\mu_{k}}=d_{i_{0}}(x_{i_0}^o,x_{i_0}^e).
%% \]
%% So, $\mu_{1}$ and $\mu_{2}$ are the unique constants satisfying equation (\ref{e7}).
%% }
%%\footnote{For the uniqueness, and the forthcoming
%%property $(t^{o},x^{o})\not\ll (t^{e},x^{e})$, just apply [Subl.
%%3.4.2, PhD]: if $\mu_{i},\overline{\mu}_{i}\geq 0$,
%%$\mu_{1}+\cdots +\mu_{n}=1=\overline{\mu}_{1}+\cdots
%%+\overline{\mu}_{n}$ and $\overline{\mu}_{n}<\mu_{n}$ then there
%%exists some $i_{0}\in \{1,\ldots,n-1\}$ such that
%%$\overline{\mu}_{i_{0}}>\mu_{i_{0}}$ (and thus,
%%$\overline{\mu}_{n}/\overline{\mu}_{i_{0}}<\mu_{n}/\mu_{i_{0}}$)
%%and $\overline{\mu}_{j}/\overline{\mu}_{i_{0}}\leq
%%\mu_{j}/\mu_{i_{0}}$ $\forall j\neq n$.}
%% \cambios{Moreover, necessarily $T'\geq T$. In fact, if $T'<T$ then for any ${t^o}' \in (T',T)$ it cannot happen that
%% $({t^o}',x^{o}) \ll (t^{e},x^{e})$ since the same process to construct $T'$ will imply the existence of  $T''\in ({t^{o}}',T')$ such that $(T'',0)$ is causally but no
%% timelike related to $(t^{e},d_{1}(x_{1}^o,x_{1}),d_{2}(x_{2}^o,x_{2}^{e}))$ in the globally hyperbolic \multiwarped spacetime $(\R \times \R^{2},\hat{g})$, this will contradict the achronality of $\partial I_{\hat{g}}^{+}((t^{e},d_{1}(x_{1}^o,x_{1}),d_{2}(x_{2}^o,x_{2}^{e})))$ because $(T',0) \ll (T'',0)$ and both points live in the boundary. Therefore, for any ${t^{o}}' \in (T',T)$ we have that $({t^o}',x^{o}) \not \ll (t^{e},x^{e})$ and this is a contradiction to the condition over $T$ given in (\ref{e0}). Therefore, $T' \geq T$. The first part of the proof proves that $T \geq T'$ for any $T'$ satisfying equation (\ref{e*}), therefore $T$ is the only point satisfying equation (\ref{e*}) for unique constants $(\mu_{1},\mu_{2})$ with $\mu_{1}+\mu_{2}=1$. }
%\end{proof}
%
%\begin{rem}\label{rem:3}
% It is worth mentioning how in previous proof we have moved from the \multiwarped model $V$ to the globally hyperbolic \multiwarped model $V'$. This trick allow us to work in globally hyperbolic models which are complete, and so, where $\overline{I^\pm(p)}=J^\pm(p)$, being possible to obtain the integral condition more easily.\footnote{Jony: Este remark lo puse pensando en que este truco iba a volver a usarse más adelante, pero finalmente (al menos para el borde causal) no ha hecho falta. Mirar si vale la pena dejarlo.}
%\end{rem}
We are now in conditions to establish the characterization of the chronological relation.
\begin{prop}\label{c0}
Let $(V,g)$ be a \multiwarped spacetime as in (\ref{eq:1-aux}), and $(t^{o},x^{o}), (t^{e},x^{e})\in V$ with $x^{o}\neq
x^{e}$. The following conditions are equivalent:
\begin{itemize}

\item[(i)]  $(t^{o},x^{o})\ll (t^{e},x^{e})$; or, equivalently, $t^o<T(x^o,(t^e,x^e))$ (recall (\ref{e0}));
\item[(ii)] $T(x^o,(t^e,x^e))$ is the unique real value $T\in (a,b)$
with $t^{o}<T<t^{e}$ such that, for some (unique) positive constants $\mu_{1},\mu_2 \ncambios{\geq}
0$, with $\mu_{1}+\mu_{2}=1$, it satisfies
\begin{equation}\label{ee2}
\Integral{T}{t^{e}}{\mu_{i}}{i}{\mu_{k}}=d_{i}(x^{o}_{i},x^{e}_{i})\qquad\hbox{for}\;\;
i=1,2;
%\int_{T}^{t^{e}}\sqrt{\mu_{i}}\alpha_{i}^{-1}\left(\sum_{j=1}^{n}\frac{\mu_{j}}{\alpha_{j}}\right)^{-1/2}dt=d_{i}(x^{o}_{i},x^{e}_{i})\qquad\hbox{for}\;\;
%i=1,...,n;
\end{equation}

\item[(iii)] there exist strictly positive constants $\mu'_{1},\mu'_{2}> 0$, with $\mu'_1+\mu'_2=1$,
%(with
%$\mu'_{1}+\cdots +\mu'_{n}=1$)
such that
\begin{equation}\label{ee2''}
\Integral{t^{o}}{t^{e}}{\mu_{i}'}{i}{\mu_{k}'}>
d_{i}(x^{o}_{i},x^{e}_{i})\qquad\hbox{for $i=1,2$}.
%\int_{t^{o}}^{t^{e}}\sqrt{\mu'_{i}}\alpha_{i}^{-1}\left(\sum_{j=1}^{n}\frac{\mu'_{j}}{\alpha_{j}}\right)^{-1/2}dt\geq
%d_{i}(x^{o}_{i},x^{e}_{i})\qquad\hbox{for $i=1,...,n$},
\end{equation}
%with equality in the $i$-th inequality if and only if
%$\mu'_{i}=0$.
\end{itemize}
\end{prop}
\begin{proof}
%\footnote{He hecho cambios en esta prueba. Hay que chequear que todo es correcta.}
The implication $(ii)\Rightarrow (iii)$ is trivial unless some $\mu_i$ is equal to $0$. So, assume for instance that $\mu_1=0$ (and so, $\mu_2=1$). Then, \eqref{ee2} becomes
\[
\left\{
  \begin{array}{l}
    0=d_1(x_1^o,x_1^e)\\
    \\
    \displaystyle \int_{T}^{t^e}\frac{1}{\sqrt{\alpha_2(s)}}ds=d_2(x_2^o,x_2^e).
  \end{array}
\right.
  \]
  By continuity, we can modify slightly $\mu_1$, $\mu_2$, to obtain strictly positive $\mu'_1,\mu'_2$, with $\mu'_1+\mu'_2=1$, such that
 \[
    \left\{
      \begin{array}{l}\displaystyle\Integral{t^{o}}{t^{e}}{\mu_{1}'}{1}{\mu_{k}'}>0= d_1(x_1^o,x_1^e)\\
      \\
      \displaystyle\Integral{t^{o}}{t^{e}}{\mu_{2}'}{2}{\mu_{k}'}> d_2(x_2^o,x_2^e),
      \end{array}\right.
    \]
as desired.


For the implication $(iii) \Rightarrow (i)$, denote
  \[
L^\epsilon_i:=\Integral{t^o+\epsilon}{t^e}{\mu_{i}'}{i}{\mu_{k}'},\quad\hbox{for $\epsilon>0$.}
    \]
 Take $\epsilon>0$ small enough so that $t^o+\epsilon<t^e$ and the inequalities in \eqref{ee2''} still hold for $t^o+\epsilon$ instead of $t^o$. Since $L_i^{\epsilon}>d_i(x_i^o,x_i^e)$, there exist curves $y_i:[s^o,s^e]\rightarrow M_i$, with $y_{i}(s^{o})=x_{i}^{o}$ and $y_{i}(s^{e})=x^{e}_{i}$, such that $length(y_{i})=L^\epsilon_{i}$, $i=1,2$. Consider the lightlike curve
$\rho(s)=(\tau(s),\overline{y}_{1}(s),\overline{y}_{2}(s))$,
with $\overline{y}_{i}$ reparametrizations of $y_{i}$,
constructed by
requiring
\[
\left\{\begin{array}{l}\dot{\tau}=\sqrt{\sum_{i=1}^{2}\frac{\mu_{i}'}{\alpha_{i}\circ\tau}}
\\ \tau(s^{e})=t^{e}
\end{array}\right.,\qquad
\left\{\begin{array}{l}g_{i}(\dot{\overline{y}}_{i},\dot{\overline{y}}_{i})=\frac{\mu_{i}'}{(\alpha_{i}\circ
\tau)^{2}} \\
\overline{y}_{i}(s^{e})=x^{e}_{i}\end{array}\right.
\qquad\hbox{for}\;\; i=1,2.
\]
Then, by applying \eqref{eq:3} to the lightlike curve $\rho$ (in particular, $D=0$), we deduce
\[
\hbox{length}(\overline{y}_{i}\mid_{[\tau^{-1}(t^o+\epsilon),s^{e}]})=\Integral{t^o+\epsilon}{t^{e}}{\mu_{i}'}{i}{\mu_{k}'}
%\int^{t^{e}}_{T'_{m}}\sqrt{\mu_{i}}\alpha_{i}^{-1}\left(\sum_{j=1}^{n}\frac{\mu_{j}}{\alpha_{j}}\right)^{-1/2}dt
=L^\epsilon_{i}=\hbox{length}(y_{i}).
\]
Therefore, $\rho(s)$ is a lightlike curve joining $(t^o+\epsilon,x^{o})$ with
$(t^{e},x^{e})$, and so, these points are causally related. Since $(t^o,x^o)\ll (t^o+\epsilon,x^o)$, necessarily $(t^o,x^o)\ll (t^{e},x^{e})$.

Finally, for the implication $(i) \Rightarrow (ii)$, let us show first that if $T$
satisfies (\ref{ee2}) then $T\leq T(x^o,(t^e,x^e))$. So, assume that (\ref{ee2}) holds. %Up to a small modification of $\mu_i$, $i=1,2$, we can suppose that both coefficients are strictly positive.
Take any sequence $\epsilon_{m} \searrow 0$ and define
\[
L_{i,m}:=\Integral{T-\epsilon_m}{t^{e}}{\mu_{i}}{i}{\mu_{k}}\quad\hbox{i=1,2.}
\]
We have that $L_{i,m}>d_{i}(x_{i}^o,x_{i}^{e})$ for all $i$. The implication (iii)$\Rightarrow$(i), which has been proved before, ensures that
$(T-\epsilon_{m},x^o) \ll (t^e,x^e)$ for all $m$. Therefore, from the definition of $T(x^o,(t^e,x^e))$, $T-\epsilon_{m}<T(x^o,(t^e,x^e))$ for all $m$, and then, $T \leq T(x^o,(t^e,x^e))$.

%From the hypothesis for $T'$, there exist curves $y_{i,m}(s)$ in
%$M_{i}$ with lengths $L_{i,m}$ such that
%$y_{i,m}(s^{o})=x_{i}^{o}$ and $y_{i,m}(s^{e})=x^{e}_{i}$.
%Consider the lightlike curves
%$\rho_{m}(s)=(\tau(s),\overline{y}_{1,m}(s),..., \overline{y}_{n,m}(s))$,
%with $\overline{y}_{i,m}$ reparametrizations of $y_{i,m}$,
%constructed by
%requiring
%\[
%\left\{\begin{array}{l}\dot{\tau}=\sqrt{\sum_{i=1}^{2}\frac{\mu_{i}}{\alpha_{i}\circ\tau}}
%\\ \tau(s^{e})=t^{e}
%\end{array}\right.,\qquad
%\left\{\begin{array}{l}g_{i}(\dot{\overline{y}}_{i,m},\dot{\overline{y}}_{i,m})=\frac{\mu_{i}}{(\alpha_{i}\circ
%\tau)^{2}} \\
%\overline{y}_{i,m}(s^{e})=x^{e}_{i}\end{array}\right.
%\qquad\hbox{for}\;\; i=1,2.
%\]
%Then,
%\[
%\hbox{length}(\overline{y}_{i,m}\mid_{[\tau^{-1}(T'_{m}),s^{e}]})=\Integral{T'_{m}}{t^{e}}{\mu_{i}}{i}{\mu_{k}}
%%\int^{t^{e}}_{T'_{m}}\sqrt{\mu_{i}}\alpha_{i}^{-1}\left(\sum_{j=1}^{n}\frac{\mu_{j}}{\alpha_{j}}\right)^{-1/2}dt
%=L_{i,m}=\hbox{length}(y_{i,m}).
%\]
%Therefore, $\rho_{m}(s)$ joins $(T'_{m},x^{o})$ with
%$(t^{e},x^{e})$, and so, the points $(T'_{m-1},x^{o})$,
%$(t^{e},x^{e})$ are chronologically related for all $m$. According
%to (\ref{e0}), this implies $T'<T+\epsilon_{m-1}$ for all $m$, and
%thus, $T'\leq T$.

Next, it is sufficient to prove that some value $T$
verifying (\ref{ee2}) always exists, and necessarily $T\geq T(x^o,(t^e,x^e))$. Let $t'<T(x^o,(t^e,x^e))$, and thus, $(t',x^{o})\ll (t^{e},x^{e})$.
% We know that $(t^{o},x^{o})\ll (t^{e},x^{e})$.
Let $\gamma:[t',t^e]\rightarrow V$,
$\gamma(t)=(t,c_{1}(t), c_{2}(t))$, be a timelike curve such
that $\gamma(t')=(t',x^{o})$ and $\gamma(t^{e})=(t^{e},x^{e})$.
Consider real curves $\overline{c}_{i}$, $i=1,2$, such that
\[
\left\{\begin{array}{l} 0\leq\dot{\overline{c}}_{i}(t)\leq
\sqrt{g_{i}(\dot{c}_{i}(t),\dot{c}_{i}(t))} \\
\overline{c}_{i}(t')=0 \\
\overline{c}_{i}(t^{e})=d_{i}(x_{i}^{o},x_{i}^{e})
\end{array}\right. \qquad\hbox{for}\;\; i=1,2.
\]
Then, $\overline{\gamma}(t)=(t,\overline{c}_{1}(t),\overline{c}_{2}(t))$
becomes a future directed timelike curve in the globally hyperbolic \multiwarped
spacetime with convex fibers
$V'=(\ncambios{(a,b)} \times \R^{2},-dt^{2}+\alpha_{1}dx_{1}^{2}+\alpha_{2}dx_{2}^{2})$ joining $\overline{\gamma}(t')=\point{t'}{0}{0}$ with $\overline{\gamma}(t^e)=\point{t^e}{d_{1}(x_{1}^o,x_{1}^{e})}{d_{2}(x_{2}^o,x_{2}^e)}$, i.e., \[\overline{\gamma}(t')=(t',0,0)\ll (t^e,d_{1}(x_{1}^o,x_{1}^{e}),d_{2}(x_{2}^o,x_{2}^{e}))=\overline{\gamma}(t^e).\]
Consider $T>t'$ such that $(T,0,0)\leq \overline{\gamma}(t^e)$ but $(T,0,0)\not\ll \overline{\gamma}(t^e)$. From
Avez and Seifert's result, there exists some lightlike geodesic in $V'$
joining both points. Now, from Prop. \ref{thm:characluzgeodesics} applied to this lightlike geodesic, there exist unique positive constants $\ncambios{\mu_1,\mu_2 \geq}0$, with $\mu_1+\mu_2=1$, such that
\[
\Integral{T}{t^{e}}{\mu_{i}}{i}{\mu_{k}}=|d_{i}(x^{o}_{i},x^{e}_{i})-0|=d_{i}(x^{o}_{i},x^{e}_{i}) \qquad\hbox{for}\;\;
i=1,2.
  \]
Finally, since $t'<T$ for all $t'<T(x^o,(t^e,x^e))$, necessarily $T(x^o,(t^e,x^e))\leq T$, which concludes the proof.

\end{proof}

%\begin{rem}
%  Condition (iii) in previous theorem can be replaced by the following one:
%  {\em
%  \begin{itemize}
%  \item[(iii')] there exist constants $\mu'_{1},\mu'_{2}\geq 0$, with $\mu'_1+\mu'_2=1$,
%%(with
%%$\mu'_{1}+\cdots +\mu'_{n}=1$)
%such that
%\begin{equation}\label{e2''}
%\Integral{t^{0}}{t^{e}}{\mu_{i}'}{i}{\mu_{k}'}\geq
%d_{i}(x^{0}_{i},x^{e}_{i})\qquad\hbox{for $i=1,2$},
%%\int_{t^{o}}^{t^{e}}\sqrt{\mu'_{i}}\alpha_{i}^{-1}\left(\sum_{j=1}^{n}\frac{\mu'_{j}}{\alpha_{j}}\right)^{-1/2}dt\geq
%%d_{i}(x^{o}_{i},x^{e}_{i})\qquad\hbox{for $i=1,...,n$},
%\end{equation}
%with equality in the $i$-th inequality if and only if
%${\mu'}_{i}=0$.
%  \end{itemize}}
%  Now, the implication $(ii)\Rightarrow (iii')$ is straightforward. However, (iii) presents more clearly the open character of the chronological relation, which will be much more practical for the forthcoming sections.
%\end{rem}

Let us consider now the characterization of the causal relation (see \cite[Theorem 2(2)]{FS}).
 %In a first approach, one can think that it follows just by replacing the strict inequalities in condition (iii) of Thm. \ref{c0} by regular ones. However, this procedure forgets the necessity to include a convexity condition on each fiber, in order to ensure the existence of some lightlike geodesic connecting the two points.
 \begin{defi}
A Riemannian manifold $(N,h)$ is $L$-{\em convex} if any pair of points $p,q\in N$ with $d_h(p,q)<L$ can be joined by a minimizing geodesic.
\end{defi}
%Now, we can establish the announced characterization about the causal relation (see)
\begin{prop}
\label{p2'}
Let $(V,g)$ be a {\multiwarped} spacetime as in (\ref{eq:1-aux}) whose fibers $(M_i,g_i)$ are $L_i$-convex for $i=1,2$. Consider two points $\point{t^{o}}{x^o_{1}}{x^o_2}, \point{t^{e}}{x_1^{e}}{x_2^e} \in V$, with $t^o \leq t^e$, satisfying $d(x_i^o,x_i^e)<L_i$, $i=1,2$. Then, the following conditions are equivalent:
\begin{itemize}
\item[(i)] the points are causally related,
$\point{t^{o}}{x_1^{o}}{x_2^o} \leq \point{t^{e}}{x_1^{e}}{x_2^e}$;

\item[(ii)] there exists a causal geodesic joining $\point{t^{o}}{x_1^{o}}{x_2^o}$ with $\point{t^{e}}{x_1^{e}}{x_2^e}$;

\item[(iii)] there exist constants $\mu'_{1},\mu'_{2}\geq 0$, $\mu'_1+\mu'_2=1$,
%(with
%$\mu'_{1}+\cdots +\mu'_{n}=1$)
such that
\begin{equation}
\label{e2'''}
\Integral{t^{o}}{t^{e}}{\mu'_{i}}{i}{\mu'_{k}} \geq
d_{i}(x^{o}_{i},x^{e}_{i})\qquad\hbox{for}\;\;
i=1,2.
\end{equation}
\end{itemize}
Moreover, if the equalities hold in (\ref{e2'''}), then there is a lightlike and no timelike geodesic joining the points.
\end{prop}

%The proof of previous proposition, that can be found on \cite[Theorem 2]{FS}, relies on the following characterization of geodesics in doubly warped spacetimes (see \cite[(5) and (6)]{FS}): a curve $\gamma:I\rightarrow \R\times M_1\times M_2$ with $\gamma(t)=(\tau(t),c_1(t),c_2(t))$ is a geodesic if, and only if,
%
%\begin{equation}
%  \label{eq:31}
%  \begin{array}{rl}
%  \displaystyle \frac{d^2 \tau}{d t^2}=& \displaystyle -\left(\sum_{i=1}^2 \frac{\mu_i}{(\alpha_i\circ \tau)^{3/2}} \frac{d(\sqrt{\alpha_i})}{d\tau}\circ \tau \right)\\  & \\
%\displaystyle  \frac{D}{dt}\frac{d\,c_i }{dt}=& \displaystyle -\frac{2}{\sqrt{\alpha\circ \tau}} \frac{d(\sqrt{\alpha\circ \tau})}{dt} \frac{d\,c_i}{dt}, \quad \hbox{i=1,2}.
%  \end{array}
%\end{equation}\footnote{Nueva ecuacion sacada de otro artículo, verificar que no he metido la pata...}where $D/dt$ denotes the covariant derivative associated to each $g_i$ along $c_i$ and $\mu_i$ is the constant $(\alpha_i^2\circ \tau)g_i(d\,c_i/dt,d\,c_i/dt)$.


% \newpage


% \subsection{Chronological relation in multiwarped spacetimes}

% Let $(V,g)$ be a doublywarped spacetime, \cambios{in order to study
% the causal boundary of this kind of spacetimes we need to characterize the chronological relation $\ll$ in $(V,g)$.}
% For every piecewise smooth curve $c:[s^{o},s^{e}] \rightarrow M$
% with endpoints $c(s^{o})=x^{o}, c(s^{e}) =x^{e}$, consider the
% unique future-directed lightlike curve $\gamma(t)=(t,c(s(t))),
% t\in [T,t^{e}]$, being $s(t)$ and $T=T[c]$ determined by
% $g(\dot\gamma, \dot\gamma )\equiv 0, s(T)=s^{o},
% s(t^{e})=s^{e}$\footnote{If such a curve $\gamma$ does not exist
% (i.e. $s(t)>s^{o}$ for all $t<t^{e}$), just define
% $T=T[c]:=-\infty$.}. Let $C \equiv C(x^{o},x^{e})$ be the set of
% all such curves $c=c(s)$, and consider the functional
% $$\J: C \rightarrow \mathbb{R}, \quad c \mapsto T[c].$$
% Define a function $T: M \times (\R\times M) \rightarrow \R$ in the
% following way:
% $$(x^{o},(t^{e},x^{e})) \mapsto T(x^{o},(t^{e},x^{e})):= {Sup}_{C}\J.$$
% Then, one easily has:
% %\footnote{With this notation, the relation
% %between function $T$ and the {\em (time) arrival} map $\delta:
% %V\times M\rightarrow\R$ is $\delta((T,x^{o}),x^{e})=t^{e}-T$, with
% %$T=T(x^{o},(t^{e},x^{e}))$. (Apply the continuity of $\delta$.)}
% \begin{equation}\label{e0}
% (t^{o},x^{o}) \ll (t^{e},x^{e}) \;\; \Longleftrightarrow \;\;
% t^{o}<T(x^{o},(t^{e},x^{e})).
% \end{equation}
% %\footnote{When property (\ref{e0}) is applied to a causal curve
% %$\gamma(s)=(t(s),x(s))$, it translates into: $(t^{o},x^{o})\in
% %I^{-}[\gamma]\Leftrightarrow t^{o}<b_{\gamma}(x^{o})$, where
% %$b_{\gamma}(x^{o}):=\lim_{s}(t(s)-\delta(x^{o},(t(s),x(s))))$ can
% %be interpreted as the {\em Busemann function} associated to
% %$\gamma$.}
% %Notice that function $\delta$ is always finite and continuous, and
% %essentially the same function is obtained if past-directed causal
% %curves are taken.

% Let $\gamma:  I \rightarrow V$,
% $\gamma(s)=(t(s),x_{1}(s),x_{2}(s))$ be a future-directed causal
% curve. From the expression of $g$ in previous section, it is
% \[
% \frac{dt}{ds}(s)=\sqrt{-D+\frac{c_{1}}{\alpha_{1}\circ
% t}+\frac{c_{2}}{\alpha_{2}\circ t}}(s),
% \]
% where $D:=g(d\gamma/ds,d\gamma/ds)\leq 0$ and
% $c_{i}:=(\alpha_{i}^{2}\circ t)g_{i}(dx_{i}/ds,dx_{i}/ds)$,
% $i=1,2$. Then, from the Inverse Function Theorem:
% \[
% \frac{ds}{dt}(t)=\left(-(D\circ s)+\frac{c_{1}\circ s}{\alpha
% _{1}}+\frac{c_{2}\circ s}{\alpha _{2}}+...+\frac{c_{n}\circ s}{\alpha
% _{n}}\right)^{-1/2}(t)
% \]
% Therefore, if we denote $t^{o}=t(s^{o})$, $t^{e}=t(s^{e})$, we
% deduce
% \[
% \begin{array}{c}
% \hbox{length}\left(x_{i}\mid_{[s^{o},s^{e}]}\right)=\int_{s^{o}}^{s^{e}}\sqrt{g_{i}(\dot
% x_{i}, \dot x_{i})} ds=\int_{t^o}^{t^e}\sqrt{g_{i}(\dot x_{i}, \dot
% x_{i})} \frac{ds}{dt} dt \qquad\qquad\qquad\quad \\
% \;\quad\qquad\qquad\qquad =\int_{t^o}^{t^e}\frac{\sqrt{c_{i}\circ
% s}}{\alpha_i}\left(-(D\circ s)+\frac{c_{1}\circ
% s}{\alpha_{1}}+\frac{c_{2}\circ s}{\alpha_{2}}\right)^{-1/2}dt
% \qquad\hbox{for}\;\; i=1,2.
% \end{array}
% \]

% \ncambios{Last equality will be usefull in the Lemma below, because it allows to compute the length of the spatial components of a causal curve in terms of
% $\mu_{i}$'s and the warping functions.}

% \cambios{
% \begin{lemma}
% \label{lightlikecurve}
% Let $(t^{o},x^o)$ and $(t^e,x^{e})$ in $V$ \multiwarped spacetime with $x^o \neq x^{e}$. If there exists $(\mu_{1},\mu_{2}) \in (0,1)^{2}$ with $\mu_{1}+\mu_{2}=1$ and satisfying the following integral conditions:
% \[
% \Integral{t^o}{t^e}{\mu_{i}}{i}{\mu_{k}}>d_{i}(x_{i}^o,x_{i}^{e}) \qquad\hbox{for}\;\;
% i=1,2.
% \]
% Then $(t^{o},x^o) \ll (t^e,x^{e})$, i.e., there exists a timelike future directed curve between $(t^{o},x^o)$ and $(t^e,x^{e})$.
% \end{lemma}

% {\bf Proof:}

% The integral conditions and the continuity of the lower limit of the integral imply that there exists $\epsilon>0$ such that   $L_{i}^{\epsilon}:=\Integral{t^o+\epsilon}{t^e}{\mu_{i}}{i}{\mu_{k}}$ satisfies $L_{i}^{\epsilon}>d_{i}(x_{i}^o,x_{i}^{e})$ for all $i$, then, this last condition implies
% that there exists curves $y_{i}:[s^o,s^e] \rightarrow M_{i}$ such that $length(y_{i})=L_{i}^{\epsilon}$,
% $y_{i}(s^{o})=x_{i}^{o}$ and $y_{i}(s^{e})=x^{e}_{i}$. Consider the following lightlike curve
% $\rho(s)=(\tau(s),\overline{y}_{1}(s),\overline{y}_{2}(s))$,
% with $\overline{y}_{i}$ reparametrizations of $y_{i}$,
% constructed by
% requiring
% \[
% \left\{\begin{array}{l}\dot{\tau}=\sqrt{\sum_{i=1}^{2}\frac{\mu_{i}}{\alpha_{i}\circ\tau}}
% \\ \tau(s^{e})=t^{e}
% \end{array}\right.,\qquad
% \left\{\begin{array}{l}g_{i}(\dot{\overline{y}}_{i},\dot{\overline{y}}_{i})=\frac{\mu_{i}}{(\alpha_{i}\circ
% \tau)^{2}} \\
% \overline{y}_{i}(s^{e})=x^{e}_{i}\end{array}\right.
% \qquad\hbox{for}\;\; i=1,2.
% \]
% Then,
% \[
% \hbox{length}(\overline{y}_{i}\mid_{[\tau^{-1}(t^o+\epsilon),s^{e}]})=\Integral{t^o+\epsilon}{t^{e}}{\mu_{i}}{i}{\mu_{k}}
% %\int^{t^{e}}_{T'_{m}}\sqrt{\mu_{i}}\alpha_{i}^{-1}\left(\sum_{j=1}^{n}\frac{\mu_{j}}{\alpha_{j}}\right)^{-1/2}dt
% =L_{i}^{\epsilon}=\hbox{length}(y_{i}).
% \]
% Therefore, $\rho(s)$ joins $(t^o+\epsilon,x^{o})$ with
% $(t^{e},x^{e})$, and so, the points $(t^{o},x^{o})$,
% $(t^{e},x^{e})$ are chronologically related since $(t^o,x^o) \ll (t^{o}+\epsilon,x^o) \leq (t^e,x^e)$.
% \begin{flushright}
% $\spadesuit$
% \end{flushright}
% \medskip

% \ncambios{Luis:
% Previous lemma can be extended to the following cases: (1) some $\mu_{i}$ is equal to zero or (2) some $x_{i}^{o}$ is equal to $x_{i}^{e}$. The construction of the causal curve can be carried out as in the proof of previous lemma with the difference that we will be working in the Riemannian manifold $(M_{j},g_{j})$ in which $\mu_{j}\neq 0$ or $x_{j}^{o} \neq x_{j}^{e}$ and we will obtain a curve with $i$ component equal to a constant point $x_{i}^{o}=x_{i}^{e}$.
% }
% \medskip

% Next result will give an analytic characterization of $T(x^{o},(t^{e},x^e))$ defined before:
% }

% \begin{prop}
% \label{p0}
% Let $(V,g)$ be a \multiwarped spacetime and $(x^{o},(t^{e},x^{e}))\in M \times V$, with $x^{o}\neq x^{e}$. If $T=T(x^{o},(t^{e},x^{e}))>-\infty$ then $\varsigma=T$ is the unique value satisfying
% \begin{equation}
% \label{e*}
% \Integral{\varsigma}{t^{e}}{\mu_{i}}{i}{\mu_{k}}=d_{i}(x^{o}_{i},x^{e}_{i}) \qquad\hbox{for}\;\;
% i=1,2
% \end{equation}
% for some (unique) constants $\mu_{1},\mu_{2} \geq 0$ with
% $\mu_{1}+\mu_{2}=1$.
% \end{prop}
% {\it Proof.} First, we are going to show that if $\varsigma=T'$
% satisfies (\ref{e*}), then necessarily $T'\leq T$. To this aim,
% take any sequence $\epsilon_{m} \searrow 0$ and define
% \[
% L_{i,m}:=\Integral{T'_{m}}{t^{e}}{\mu_{i}}{i}{\mu_{k}}
% \qquad
% \hbox{for}\;\; i=1,2 \quad\hbox{with}\;\; T'_{m}:=T'-\epsilon_{m}.
% \]
% \cambios{From the hypothesis for $T'$, we have that $L_{i,m}>d_{i}(x_{i}^o,x_{i}^{e})$ for all $i$, then Lemma \ref{lightlikecurve} implies that
% $(T'_{m},x^o) \ll (t^e,x^e)$ for all $m$, then, $T'-\epsilon_{m}<T$ for all $m$ and therefore $T' \leq T$.
% }
% \medskip
% %From the hypothesis for $T'$, there exist curves $y_{i,m}(s)$ in
% %$M_{i}$ with lengths $L_{i,m}$ such that
% %$y_{i,m}(s^{o})=x_{i}^{o}$ and $y_{i,m}(s^{e})=x^{e}_{i}$.
% %Consider the lightlike curves
% %$\rho_{m}(s)=(\tau(s),\overline{y}_{1,m}(s),..., \overline{y}_{n,m}(s))$,
% %with $\overline{y}_{i,m}$ reparametrizations of $y_{i,m}$,
% %constructed by
% %requiring
% %\[
% %\left\{\begin{array}{l}\dot{\tau}=\sqrt{\sum_{i=1}^{2}\frac{\mu_{i}}{\alpha_{i}\circ\tau}}
% %\\ \tau(s^{e})=t^{e}
% %\end{array}\right.,\qquad
% %\left\{\begin{array}{l}g_{i}(\dot{\overline{y}}_{i,m},\dot{\overline{y}}_{i,m})=\frac{\mu_{i}}{(\alpha_{i}\circ
% %\tau)^{2}} \\
% %\overline{y}_{i,m}(s^{e})=x^{e}_{i}\end{array}\right.
% %\qquad\hbox{for}\;\; i=1,2.
% %\]
% %Then,
% %\[
% %\hbox{length}(\overline{y}_{i,m}\mid_{[\tau^{-1}(T'_{m}),s^{e}]})=\Integral{T'_{m}}{t^{e}}{\mu_{i}}{i}{\mu_{k}}
% %%\int^{t^{e}}_{T'_{m}}\sqrt{\mu_{i}}\alpha_{i}^{-1}\left(\sum_{j=1}^{n}\frac{\mu_{j}}{\alpha_{j}}\right)^{-1/2}dt
% %=L_{i,m}=\hbox{length}(y_{i,m}).
% %\]
% %Therefore, $\rho_{m}(s)$ joins $(T'_{m},x^{o})$ with
% %$(t^{e},x^{e})$, and so, the points $(T'_{m-1},x^{o})$,
% %$(t^{e},x^{e})$ are chronologically related for all $m$. According
% %to (\ref{e0}), this implies $T'<T+\epsilon_{m-1}$ for all $m$, and
% %thus, $T'\leq T$.

% Next, we are going to prove that some value $\varsigma=T'$
% verifying (\ref{e*}) always exists, and necessarily $T'\geq T$.
% From (\ref{e0}), $(t^{o},x^{o})\ll (t^{e},x^{e})$ for any
% $-\infty<t^{o}<T(x^{o},(t^{e},x^{e}))$. Let
% $\rho(s)=(\tau(s),y_{1}(s), y_{2}(s))$ be a timelike curve such
% that $\rho(s^{o})=(t^{o},x^{o})$ and $\rho(s^{e})=(t^{e},x^{e})$.
% Consider $2$ curves $\overline{y}_{i}$ in $\R$ such that
% \[
% \left\{\begin{array}{l} 0\leq\dot{\overline{y}}_{i}(s)\leq
% \sqrt{g_{i}(\dot{y}_{i}(s),\dot{y}_{i}(s))} \\
% \overline{y}_{i}(s^{o})=0 \\
% \overline{y}_{i}(s^{e})=d_{i}(x_{i}^{o},x_{i}^{e})
% \end{array}\right. \qquad\hbox{for}\;\; i=1,2.
% \]
% Then, the curve
% $\overline{\rho}(s)=(\tau(s),\overline{y}_{1}(s),\overline{y}_{2}(s))$
% is a future directed timelike curve in the globally hyperbolic \multiwarped
% spacetime
% $(\R \times \R^{2},-dt^{2}+\alpha_{1}dx_{1}^{2}+\alpha_{2}dx_{2}^{2})$ joining $\overline{\rho}(s^o)=\point{t^o}{0}{0}$ with $\overline{\rho}(s^e)=\point{t^e}{d_{1}(x_{1}^o,x_{1}^{e})}{d_{2}(x_{2}^o,x_{2}^e)}$.
% Therefore, there exists some unique $T' \in (t^{o},t^{e})$ such that
% $(T',\overline{y}(s^{o}))\leq\overline{\rho}(s^{e})$ but
% $(T',\overline{y}(s^{o}))\not\ll \overline{\rho}(s^{e})$. From
% Avez and Seifert's result, there exists some lightlike geodesic
% joining $(T',\overline{y}(s^{o}))$ with $\overline{\rho}(s^{e})$. \cambios{Denote by $\lambda(s)=(\tau(s),c_{1}(s),c_{2}(s))$ ($s \in [0,1]$) the
% unique lightlike geodesic, and recall that a geodesic in a \multiwarped satisfies the following: (a) each $\mu_{i}=\alpha_{i}(\tau(s))^{2}g_{i}(c_{i}',c_{i}')$ is constant and (b) $c_{i}(s)$ is a pregeodesic in $(\mathbb{R},+dx_{i}^{2})$ for all $i$, see \cite{FS}[Eqns. (5), (6)]. Applying a change of variables in the following integral equalities by using the fact that $\frac{d\tau}{ds}=\sqrt{\sum_{i=1}^{2} \frac{\mu_{i}}{\alpha_{i} \circ \tau}}$ leads to:

% \begin{equation}
% \label{e7}
% \hbox{length}(c_{i})=\int_{0}^{1}\sqrt{g_{i}(c_{i}',c_{i}')}dr=\Integral{T'}{t^{e}}{\mu_{i}}{i}{\mu_{k}}=d_{i}(x^{o}_{i},x^{e}_{i})\qquad\hbox{for
% all}\;\; i
% %\int^{t^{e}}_{T'}\sqrt{\mu_{i}}\alpha_{i}^{-1}\left(\sum_{j=1}^{n}\frac{\mu_{j}}{\alpha_{j}}\right)^{-1/2}dt=d_{i}(x^{o}_{i},x^{e}_{i})\qquad\hbox{for
% %all}\;\; i
% \end{equation}
% }
% \cambios{Note that $\mu_{1}$ and $\mu_{2}$ are unique for $T'$, in fact, if $\mu_{1}'$ and $\mu_{2}'$ are different to $\mu_{1}$ and $\mu_{2}$ then we have two possibilities: (1) $\mu_{1}'<\mu_{1}$ and $\mu_{2}<\mu_{2}'$ or (2) $\mu_{1}<\mu_{1}'$ and $\mu_{2}'<\mu_{2}$. Any of the previous cases imply that for some $i_{0} \in \{1,2\}$ the following strict inequality is obtained:
% \[
% \Integral{T'}{t^e}{\mu_{i_{0}}'}{i_{0}}{\mu_{k}'}>\Integral{T'}{t^e}{\mu_{i_{0}}}{i_{0}}{\mu_{k}}=d_{i_{0}}(x_{i_0}^o,x_{i_0}^e).
% \]
% So, $\mu_{1}$ and $\mu_{2}$ are the unique constants satisfying equation (\ref{e7}).
% }
% %\footnote{For the uniqueness, and the forthcoming
% %property $(t^{o},x^{o})\not\ll (t^{e},x^{e})$, just apply [Subl.
% %3.4.2, PhD]: if $\mu_{i},\overline{\mu}_{i}\geq 0$,
% %$\mu_{1}+\cdots +\mu_{n}=1=\overline{\mu}_{1}+\cdots
% %+\overline{\mu}_{n}$ and $\overline{\mu}_{n}<\mu_{n}$ then there
% %exists some $i_{0}\in \{1,\ldots,n-1\}$ such that
% %$\overline{\mu}_{i_{0}}>\mu_{i_{0}}$ (and thus,
% %$\overline{\mu}_{n}/\overline{\mu}_{i_{0}}<\mu_{n}/\mu_{i_{0}}$)
% %and $\overline{\mu}_{j}/\overline{\mu}_{i_{0}}\leq
% %\mu_{j}/\mu_{i_{0}}$ $\forall j\neq n$.}
% \cambios{Moreover, necessarily $T'\geq T$. In fact, if $T'<T$ then for any ${t^o}' \in (T',T)$ it cannot happen that
% $({t^o}',x^{o}) \ll (t^{e},x^{e})$ since the same process to construct $T'$ will imply the existence of  $T''\in ({t^{o}}',T')$ such that $(T'',0)$ is causally but no
% timelike related to $(t^{e},d_{1}(x_{1}^o,x_{1}),d_{2}(x_{2}^o,x_{2}^{e}))$ in the globally hyperbolic \multiwarped spacetime $(\R \times \R^{2},\hat{g})$, this will contradict the achronality of $\partial I_{\hat{g}}^{+}((t^{e},d_{1}(x_{1}^o,x_{1}),d_{2}(x_{2}^o,x_{2}^{e})))$ because $(T',0) \ll (T'',0)$ and both points live in the boundary. Therefore, for any ${t^{o}}' \in (T',T)$ we have that $({t^o}',x^{o}) \not \ll (t^{e},x^{e})$ and this is a contradiction to the condition over $T$ given in (\ref{e0}). Therefore, $T' \geq T$. The first part of the proof proves that $T \geq T'$ for any $T'$ satisfying equation (\ref{e*}), therefore $T$ is the only point satisfying equation (\ref{e*}) for unique constants $(\mu_{1},\mu_{2})$ with $\mu_{1}+\mu_{2}=1$. }
% \begin{flushright}
% $\spadesuit$
% \end{flushright}



% %before implies $(t^{o},x^{o})\not\ll (t^{e},x^{e})$ for any
% %$T'<t^{o}<T$, which contradicts (\ref{e0})}. $\square$
% %Consider $n$ {\em complete} metrics
% %$\overline{g}_{1},\ldots,\overline{g}_{n}$ on
% %$M_{1},\ldots,M_{n}$, resp., such that $\overline{g}_{i}\geq
% %g_{i}$ and $\overline{g}_{i}$ agrees $g_{i}$ on the range of
% %$y_{i}$, $i=1,\ldots,n$. (As the range of $y_{i}$ is compact, the
% %metrics $\overline{g}_{i}$ can be constructed by a standard
% %partition of unity argument.) Now, the corresponding warped metric
% %$\overline{g}$ obtained by replacing $g_{i}$ by $\overline{g}_{i}$
% %in $g$ satisfy:
% %\begin{itemize}

% %\item[(a)] $\overline{g}$ is globally hyperbolic, because each
% %$\overline{g}_{i}$ is complete, \item[(b)] as
% %$\overline{g}_{i}=g_{i}$ on $y_{i}$, then $\gamma(s^{*})\in
% %\overline{I}^{+}((t^{o},x^{o}))$, where $\overline{I}^{+}(z)$
% %denotes the set of points which can be joined with $z$ by a
% %future-pointing $\overline{g}$-timelike curve, \item[(c)] by Avez
% %and Seifert's result, there exists a $\overline{g}^{1,2}$-timelike
% %geodesic joining the two points, \item[(d)] by previous
% %implication, there exist $\mu_{1},\ldots,\mu_{n}$ such that
% %inequalities (\ref{e2}) hold, putting in the right hand side of
% %each inequality the distance $\overline{d}_{i}$ associated to
% %$\overline{g}_{i}$, and \item[(e)] as $\overline{g}_{i}\geq
% %g_{i}$, the corresponding distances also satisfy
% %$\overline{d}_{i}\geq d_{i}$, and the desired inequalities are
% %obtained. % \begin{flushright}
% %$\spadesuit$
% %\end{flushright}
% %\end{itemize}
% \vspace{1mm}


% Now, we can establish the following useful characterizations of
% the chronologically relation in \multiwarped spacetimes:
% \begin{thm}\label{c0}
% Let $(V,g)$ be a multiwarped spacetime and $(t^{o},x^{o}), (t^{e},x^{e})\in V$ with $x^{o}\neq
% x^{e}$. The following conditions are equivalent\footnote{Habra que comentar en algun sitio que los resultados
% anteriores, y, por tanto, los siguientes, son trivialmente
% extensibles al caso $(t^o,x^o)\in \R\times \overline{M}_C$. de
% hecho, esto ya se esta usando en la Proposicion \ref{r} (2).}:
% \begin{itemize}

% \item[(i)] The points are chronologically related,
% $(t^{o},x^{o})\ll (t^{e},x^{e})$;
% \item[(ii)] there exist some
% unique $t^{o}<T<t^{e}$ and (unique) constants $\mu_{1},\mu_2 \geq
% 0$ with $\mu_{1}+\mu_{2}=1$ such that
% \begin{equation}\label{e2}
% \Integral{T}{t^{e}}{\mu_{i}}{i}{\mu_{k}}=d_{i}(x^{o}_{i},x^{e}_{i})\qquad\hbox{for}\;\;
% i=1,...,n;
% %\int_{T}^{t^{e}}\sqrt{\mu_{i}}\alpha_{i}^{-1}\left(\sum_{j=1}^{n}\frac{\mu_{j}}{\alpha_{j}}\right)^{-1/2}dt=d_{i}(x^{o}_{i},x^{e}_{i})\qquad\hbox{for}\;\;
% %i=1,...,n;
% \end{equation}

% \item[(iii)] there exist constants $\mu'_{1},\mu'_{2}\geq 0$
% %(with
% %$\mu'_{1}+\cdots +\mu'_{n}=1$)
% such that
% \begin{equation}\label{e2''}
% \Integral{t^{o}}{t^{e}}{\mu_{i}'}{i}{\mu_{k}'}\geq
% d_{i}(x^{o}_{i},x^{e}_{i})\qquad\hbox{for $i=1,...,n$},
% %\int_{t^{o}}^{t^{e}}\sqrt{\mu'_{i}}\alpha_{i}^{-1}\left(\sum_{j=1}^{n}\frac{\mu'_{j}}{\alpha_{j}}\right)^{-1/2}dt\geq
% %d_{i}(x^{o}_{i},x^{e}_{i})\qquad\hbox{for $i=1,...,n$},
% \end{equation}
% with equality in the $i$-th inequality if and only if
% $\mu'_{i}=0$.
% \end{itemize}
% \end{thm}
% {\it Proof.} The equivalence $(i) \Leftrightarrow (ii)$ is direct
% from (\ref{e0}) and Prop. \ref{p0}. \cambios{The implication $(iii) \Rightarrow (i)$ is given by Lemma \ref{lightlikecurve}.
% Finally, the implication $(ii)\Rightarrow (iii)$ is
% trivial.
% \begin{flushright}
% $\spadesuit$
% \end{flushright}
% }
% \medskip

% \cambios{We can ensure that we can choose $\mu_{i}'$s non-zero in part $(iii)$ of previous result. This will be of help when we deal with the analytic characterization of
% the chronological relation.

% \begin{cor}
% \label{munonzero}
% Let $(V,g)$ be a multiwarped spacetime. If $(t^o,x^o)$ and $(t^{e},x^e)$ satisfies $x^o \neq x^e$ then the following statements are equivalent:
% \begin{enumerate}
% \item $(t^{o},x^o) \ll (t^e,x^e)$.
% \item There exist constants $\mu_{1},\mu_{2}>0$ with $\sum_{i=1}^{2} \mu_{i}=1$ and such that:
% \begin{equation}
% \label{e3}
% \Integral{t^o}{t^e}{\mu_{i}}{i}{\mu_{k}}>d_{i}(x^{o}_{i},x^{e}_{i})\qquad\hbox{for $i=1,2$}.
% %\int_{t^{o}}^{t^{e}}\sqrt{\mu_{i}}\alpha_{i}^{-1}\left(\sum_{j=1}^{n}\frac{\mu_{j}}{\alpha_{j}}\right)^{-1/2}dt >
% %d_{i}(x^{o}_{i},x^{e}_{i})\qquad\hbox{for $i=1,...,n$},
% \end{equation}
% \end{enumerate}
% \end{cor}

% {\bf Proof:}
% \smallskip
% Suppose that $(t^{o},x^o) \ll (t^e,x^e)$, then Prop. \ref{c0} (iii) implies the existence of $\mu'_{1},\mu'_{2}\geq 0$
% such that
% \begin{equation}
% \Integral{t^{o}}{t^{e}}{\mu_{i}'}{i}{\mu_{k}'}\geq
% d_{i}(x^{o}_{i},x^{e}_{i})\qquad\hbox{for $i=1,2$},
% %\int_{t^{o}}^{t^{e}}\sqrt{\mu'_{i}}\alpha_{i}^{-1}\left(\sum_{j=1}^{n}\frac{\mu'_{j}}{\alpha_{j}}\right)^{-1/2}dt\geq
% %d_{i}(x^{o}_{i},x^{e}_{i})\qquad\hbox{for $i=1,...,n$},
% \end{equation}
% with equality in the $i$-th inequality if and only if $\mu'_{i}=0$. Suppose, without lose of generality, that equality is achieved in the equation involving $\mu_{1}'$, then $\mu_{1}'=0$ and $\mu_{2}'=1$, since $x^o \neq x^e$ then $d_{2}(x_{2}^o,x_{2}^e)>0$ and $\int_{t^o}^{t^e}\frac{1}{\sqrt{\alpha_{2}}}ds>d_{2}(x_{2}^o,x_{2}^e)$. So,
% we can modify $(\mu_{1},\mu_{2})=(0,1)$ to obtain $(\mu_{1},\mu_{2})$ with $\mu_{1} \neq 0 \neq \mu_{2}$, $\mu_{1}<1$, $\mu_{2}>0$ and satisfying:
% \[
% \Integral{t^{o}}{t^{e}}{\mu_{i}}{i}{\mu_{k}} >
% d_{i}(x^{o}_{i},x^{e}_{i})\qquad\hbox{for $i=1,2$}.
% \]
% \medskip

% That $(2)$ implies $(1)$ is given by Lemma \ref{lightlikecurve}.
% \begin{flushright}
% $\spadesuit$
% \end{flushright}
% }

% Finally, and as useful result for the study of the causal hierarchy in the next subsection, we will give a characterization of the causal relation $\leq$ in
% \multiwarped spacetimes when each Riemannian manifold $(M_{i},g_{i})$ is $L_{i}$-weakly convex (see \cite[Thm. 2]{FS}.), where a Riemannian manifold is $L$-weakly convex if for any $x_{0}$ and $x_{1}$ such that $d(x_0,x_1) \leq L$ there exists a minimizing geodesic.
% %where $L_{i}$-weakly convex means that any pair of points $x^{o},x^{e} \in M_{i}$ with $d_{i}(x_{i}^{o},x_{i}^{e})<L_{i}$ can be joined by a minimizing geodesic.

% \begin{prop}
% \label{p2'}
% Let $V=\mathbb{R}\times_{\alpha_1}M_1 \times_{\alpha_2}M_2$ be a multiwarped spacetime whose fibers $(M_i,g_i)$ are $L_i$-weakly convex for $i=1,2$. Assume also that $\point{t^{o}}{x^o_{1}}{x^o_2}, \point{t^{e}}{x_1^{e}}{x_2^e} \in V$ and satisfying $d(x_i^o,x_i^e)<L_i$, $i=1,2$, with $t^o \leq t^e$. The following conditions are equivalent:
% \begin{itemize}
% \item[(i)] the points are causally related,
% $\point{t^{o}}{x_1^{o}}{x_2^o} \leq \point{t^{e}}{x_1^{e}}{x_2^e}$;

% \item[(ii)] there exists a causal geodesic joining $\point{t^{o}}{x_1^{o}}{x_2^o}$ with $\point{t^{e}}{x_1^{e}}{x_2^e}$;

% \item[(iii)] there exist constants $\mu'_{1},\ldots,\mu'_{n}\geq 0$
% %(with
% %$\mu'_{1}+\cdots +\mu'_{n}=1$)
% such that
% \begin{equation}
% \label{e2'''}
% \Integral{t^{o}}{t^{e}}{\mu'_{i}}{i}{\mu'_{k}} \geq
% d_{i}(x^{o}_{i},x^{e}_{i})\qquad\hbox{for $i=1,2$.}
% \end{equation}
% \end{itemize}
% Moreover if the equality holds in all equations, then there is a lightlike and no timelike geodesic joining the points.
% \end{prop}











%%% Local Variables:
%%% mode: latex
%%% TeX-master: "DoublyWarpedBoundary2017.tex"
%%% End:


\section{Position into the causal ladder}
\label{sec:causalladder}
In order to have an idea of the goodness of the causality of doubly warped spacetimes, next we are going to
determine their position into the causal ladder. As we will see, this depends on the warping functions integrals and the convexity character of their Riemannian fibers.

Let us consider first a brief remainder of the main levels of the causal ladder. Each level corresponds with a causality condition which is strictly more restrictive than the previous one:
\begin{defi}\label{ant} A spacetime $(V,g)$ is
\begin{itemize}
\item {\em non-totally vicious} if $p\not\ll p$ for some $p\in V$.

\item {\em chronological} if it does not contain closed timelike curves.

\item {\em causal} if it does not contain closed causal curves.

\item {\em distinguishing} if whenever $I^+(p)=I^+(q)$ and $I^-(p)=I^-(q)$, necessarily $p=q$.

\item{\em strongly causal} if it does not contain ``nearly closed'' causal curves, i.e. for any open neighborhood $U$ of $p$ there exists some open neighborhood $V$ with $p\in V\subset U$ such that any timelike segment with extreme points in $V$ is contained in $U$.

\item {\em stably causal} if there exists some causal Lorentzian metric $g'$ on $V$ with $g<g'$, i.e., such that $g'(v,v)<0$ for any $v\in TV\setminus \{0\}$ with $g(v,v)\leq 0$. This is equivalent to the existence of some {\em global time function}, i.e., a function defined on the whole spacetime $(V,g)$ which is strictly increasing along each future-directed causal curve.

\item {\em causally continuous} if it is distinguishing and the set valued functions $I^{+}(\cdot)$ and $I^{-}(\cdot)$ are outer continuous (say, $I^{+}(\cdot)$ is {\em outer continuous at some} $p\in V$ if, for any compact subset $K\subset I^{+}(p)$ there exists an open neighborhood $U\ni p$ such that $K\subset I^{+}(q)$ for all $q\in U$). This is equivalent to being distinguishing and {\em reflecting}, %(see \cite[Defn. 3.59]{MS}),
i.e. for any pair of events $p,q \in V$, $I^{+}(q) \subset I^{+}(p)$ if and only if $I^{-}(p) \subset I^{-}(q)$.

\item {\em causally simple} if it is causal and $J^{\pm}(p)$ are closed sets for any $p\in V$.

\item {\em globally hyperbolic} if it is causal and $J^{+}(p)\cap J^{-}(q)$ are compact for any $p,q \in V$.
\end{itemize}
\end{defi}
It is direct from the very basic structure of \multiwarped spacetimes (\ref{eq:1-aux}) that $t:V \rightarrow (a,b)$ is a global time function (see \cite[Lemma 3.55]{beem}). Therefore, any \multiwarped spacetime is stably causal. The approach developed in previous section will allow to show that any \multiwarped spacetime is causally continuous as well. In fact:

\begin{thm}
Any \multiwarped spacetime $(V,g)$ as in (\ref{eq:1-aux}) is causally continuous.
\end{thm}

\begin{proof}  Since $(V,g)$ is stably causal, it is also distinguishing. So, it suffices to show that $(V,g)$ is reflecting. Let $\point{t^o}{x_{1}^o}{x_{2}^{o}}, \point{t^{e}}{x_{1}^{e}}{x_{2}^{e}} \in V$ be such that
$I^{+}(\point{t^{e}}{x_{1}^{e}}{x_{2}^{e}}) \subset I^{+}(\point{t^{o}}{x_{1}^{o}}{x_{2}^{o}})$, and let us prove that $I^{-}(\point{t^{o}}{x_{1}^{o}}{x_{2}^{o}})
\subset I^{-}(\point{t^{e}}{x_{1}^{e}}{x_{2}^{e}})$
(the converse is analogous). Consider the sequence $\{\point{t^{e}+1/n}{x_{1}^{e}}{x_{2}^{e}}\}_{n} \subset I^{+}(\point{t^{e}}{x_{1}^{e}}{x_{2}^{e}})$ %\cambios{(for simplicity, we will assume from here that $t^e+1<b$, otherwise we will take $n\geq n_0$ so $t^e+1/n_0<b$)}
and note that, by the hypothesis, this sequence also belongs to $I^{+}(\point{t^{o}}{x_{1}^{o}}{x_{2}^{o}})$.
Therefore, from Prop. \ref{c0}, there exist constants $\mu_{1}^{n},\mu_{2}^{n}>0$, with $\mu_{1}^{n} + \mu_{2}^{n}=1$, satisfying the following inequalities:
\begin{equation}
\label{eq2'}
\Integral{t^{o}}{t^{e}+1/n}{\mu_{i}^{n}}{i}{\mu_{k}^{n}}
> d_{i}(x_{i}^{o},x_{i}^{e})\qquad\hbox{for $i=1,2$}.
\end{equation}
Up to a subsequence, we can assume that $\{\mu_{i}^{n}\}_{n}$ converges to $\mu_i$, for all $i$, with $0 \leq \mu_{1},\mu_{2}\leq 1$ and $\mu_{1}+\mu_{2}=1$. Moreover,
\begin{equation}
\label{eq3'}
\left\{\sqrt{\mu_{i}^{n}}\alpha_{i}(s)^{-1}\left(\sum_{k=1}^2\mu_{k}^{n} \alpha_{k}(s)^{-1}\right)^{-1/2}\right\}_n \longrightarrow\sqrt{\mu_{i}}\alpha_{i}(s)^{-1}\left(\sum_{k=1}^2\mu_{k}\alpha_{k}(s)^{-1}\right)^{-1/2}
\end{equation}
uniformly on $[t^o,t^e+1]$. Therefore, from (\ref{eq2'}) and (\ref{eq3'}), we deduce
\[
\Integral{t^{o}}{t^{e}}{\mu_{i}}{i}{\mu_{k}} \geq d_{i}(x_{i}^{o},x_{i}^{e}),\qquad\hbox{for $i=1,2$.}
\]
If we consider $\point{t^{o}-1/n}{x_{1}^{o}}{x_{2}^o}$, and modify slightly $(\mu_{1},\mu_{2})$, by continuity we obtain new coefficients $(\mu_{1}',\mu_{2}')$, with $\mu'_{1},\mu'_{2}>0$ and $\mu'_{1}+\mu'_{2}=1$, such that
\[
\Integral{t^{o}-1/n}{t^e}{\mu'_{i}}{i}{\mu'_{k}}
>
d_{i}(x^{o}_{i},x^{e}_{i})\qquad\hbox{for $i=1,2$.}
\]
Again from Prop. \ref{c0}, we have $\point{t^o-1/n}{x_1^o}{x_2^o} \ll \point{t^e}{x_1^e}{x_2^e}$
for all $n$. So, taking into account that $I^-(\point{t^o}{x_1^o}{x_2^o})=\cup_{n \in {\mathbb N}}I^-(\point{t^o-1/n}{x_1^o}{x_2^o})$, we deduce the inclusion $I^-(\point{t^o}{x_1^o}{x_2^o} )\subset
I^-(\point{t^e}{x_1^e}{x_2^e})$, as required.
\end{proof}
\begin{thm}
\label{causi}
A \multiwarped spacetime $(V,g)$ as in (\ref{eq:1-aux}) is causally simple if and only if $(M_i,g_i)$ is $L_i$-convex for $L_i=\int_{a}^{b}\frac{1}{\sqrt{\alpha_{i}(s)}}ds$, $i=1,2$.
\end{thm}

\begin{proof} For the implication to the right, assume that $(V,g)$ is causally simple. We will prove that
$(M_1,g_1)$ is $L_1$-convex (the proof for the second fiber is analogous). Let $x^o_{1},x^e_{1} \in M_{1}$ with $0<d_1(x^o_{1},x^e_{1})<L_1$. Since
$\int_{a}^{b}\frac{1}{\sqrt{\alpha_1(s)}}ds=L_1 > d_1(x^o_{1},x^e_{1})$, there exists $a<\C_1<\C_2<b$ such that
\begin{equation}
\label{eq6'}
\int_{\C_1}^{\C_2} \frac{1}{\sqrt{\alpha_{1}(s)}}ds>d_{1}(x^o_{1},x^e_{1}).
\end{equation}
Fix $x_{2} \in M_{2}$ and consider
the points $\point{\C_1}{x_{1}^o}{x_{2}}$ and $\point{\C_2}{x_{1}^e}{x_{2}}$. Inequality (\ref{eq6'}) and Prop. \ref{c0} imply that
$\point{\C_2}{x_{1}^e}{x_{2}} \in I^{+}(\point{\C_1}{x_{1}^o}{x_{2}})$.
Since $\point{\C_1}{x_{1}^e}{x_{2}} \not \in I^+(\point{\C_1}{x_{1}^o}{x_{2}})$, there exists
$t^e \in \R$ such that $\point{t^e}{x^e_{1}}{x_{2}} \in \partial I^{+}(\point{\C_1}{x^o_{1}}{x_{2}})$,
i.e.,
\[
\begin{array}{c}
\point{t^e}{x^e_{1}}{x_{2}} \in \overline{I^{+}(\point{\C_1}{x^o_{1}}{x_{2}})} \setminus I^{+}(\point{\C_1}{x^o_{1}}{x_{2}}) \qquad\qquad\qquad \\ \qquad\qquad\qquad\qquad\qquad\qquad=J^{+}(\point{\C_1}{x^o_{1}}{x_{2}})\setminus I^{+}(\point{\C_1}{x^o_{1}}{x_{2}}),
\end{array}
\]
where, in the equality, we have used that $(V,g)$ is causally simple. Therefore, there exists a null geodesic
$\gamma(s)=(t(s),c_{1}(s),c_{2}(s))$ connecting $\point{\C_1}{x^o_{1}}{x_{2}}$ with $\point{t^e}{x^e_{1}}{x_{2}}$.
% (\cite[Cor. 4.14]{beem})\footnote{¿Es esta dita realmente necesaria?}
From Prop. \ref{p2'} there exist constants $\mu'_1, \mu'_2\geq 0$ such that the following inequalities hold:
%In particular,
%$x_{1}(s)$ is a pregeodesic in $(M_{1},g_{1})$. Since $\gamma(s)$ is a future directed
%null curve, the pregeodesics $x_{1}(s)$ can be parametrized using $t$ as a parameter; so, the following inequalities hold
%satisfies
\[
\begin{array}{c}
0<d_{1}(x^o_{1},x^e_{1})\leq \displaystyle\Integral{\C_1}{t^e}{\mu'_{1}}{1}{{\mu'}_{k}}=\hbox{length}_1(c_{1}),
\\ 0=d_{2}(x_{2},x_{2}) \leq \displaystyle\Integral{\C_1}{t^e}{\mu'_{2}}{2}{{\mu'}_{k}}=\hbox{length}_2(c_{2}).
%\\ \vdots \\
%0=d_{n}(x_{n},x_{n}) \leq \Integral{-n_{0}}{t^e}{\mu_{n}}{1}{\mu_{k}}   ={\mathrm length}_n(x_{n}),
%
%
%\int_{-n_{0}}^{t^e}
%\sqrt{\mu_{n}}\alpha_{n}^{-1}\left(\sum_{j=1}^n\mu_{j}\alpha_{j}^{-1}\right)^{-1/2}={\mathrm length}_n(x_{n}),
\end{array}
\]
%where $\mu_{i}=\alpha_{i}^{2}(t(s))g_{i}(c_{i}',c_{i}') \in \mathbb{R}$ and $c_{i}(s)$ are pregeodesics in $(M_{i},g_{i})$ for  $i=1,2$ (see \eqref{eq:31}).
So, taking into account that $$\point{t^e}{x^e_{1}}{x_{2}} \not\in I^+(\point{\C_1}{x^o_{1}}{x_{2}}),$$ the second inequality in the first line must be an equality (recall Prop. \ref{c0}). In conclusion, $c_1(s)$ is a reparametrization of a minimizing geodesic of $(M_1,g_1)$, as required.

\smallskip

For the implication to the left, assume that $(M_i,g_i)$ is $L_i$-convex for $L_i=\int_{a}^{b}\frac{1}{\sqrt{\alpha_i(s)}}ds$, $i=1,2$. In order to prove that $(V,g)$ is causally simple,
take $\point{t^{e}}{x_{1}^{e}}{x_{2}^{e}} \in \overline{J^{+}(\point{t^{o}}{x_{1}^{o}}{x_{2}^{o}})}=\overline{I^{+}(\point{t^{o}}{x_{1}^{o}}{x_{2}^{o}})}$. Then, $I^+(\point{t^{e}}{x_{1}^{e}}{x_{2}^{e}}) \subset I^+(\point{t^{o}}{x_{1}^{o}}{x_{2}^{o}})$, and thus, $\point{t^{o}}{x_{1}^{o}}{x_{2}^{o}} \ll \point{t^{e}+1/n}{x_{1}^{e}}{x_{2}^{e}}$ for all $n$.
From Prop. \ref{c0}, there exist constants $0 <\mu_{1}^{n},\mu_{2}^{n} < 1$, with $\mu_{1}^{n}+\mu_{2}^{n}=1$ for all $n$, such that
\[
\Integral{t^o}{t^{e}+1/n}{\mu_{i}^{n}}{i}{\mu_{k}^{n}}
> d_{i}(x_{i}^{o},x_{i}^{e}),\quad i=1,2.
\]
Since $\{\mu_{i}^{n}\}_{n}$ converges (up to subsequence) to some $\mu_{i} \in [0,1]$ for $i=1,2$, with $\mu_{1}+\mu_{2}=1$, we have:
\[
\frac{\sqrt{\mu_{i}^{n}}}{\alpha_{i}(s)}\left(\sum_{k=1}^2 \frac{\mu_{k}^{n}}{\alpha_{k}(s)}\right)^{-1/2}\longrightarrow
\frac{\sqrt{\mu_{i}}}{\alpha_{i}(s)}\left(\sum_{k=1}^2 \frac{\mu_{k}}{\alpha_{k}(s)}\right)^{-1/2}\quad\hbox{uniformly on $[t^o,t^e+1]$.}
\]
Recalling now that all previous functions are bounded by the (Lebesgue) integrable function $g:[t^o,t^e+1]\rightarrow \R$,
    $g(t)=\alpha_i(t)^{-1/2}$, the Dominated Convergence Theorem ensures that:
\[
\Integral{t^o}{t^e}{\mu_{i}}{i}{\mu_{k}}
=\lim_{n\rightarrow\infty} \Integral{t^{o}}{t^{e}+1/n}{\mu_{i}^{n}}{i}{\mu_{k}^n} \geq d_{i}(x_{i}^{o},x_{i}^{e}).
\]
%\footnote{OJO!!!}with equality in the $i$-equation if and only if $\mu_{i,n}=0$.
In particular,
\[
d_{i}(x_{i}^{o},x_{i}^{e})<\Integral{a}{b}{\mu_{i}}{i}{\mu_{k}} \le \int_{a}^{b}\frac{1}{\sqrt{\alpha_i(s)}}ds=L_i,\quad i=1,2.
\]
So, taking into account that $(M_i,g_i)$ are $L_i$-convex for $i=1,2$ we have that Prop. \ref{p2'} implies $\point{t^{e}}{x_{1}^{e}}{x_2^e} \in J^{+}(\point{t^{o}}{x_{1}^{o}}{x_2^o})$, as required.
\end{proof}

The following example shows the tight character of Thm. \ref{causi}, in the sense that there may exist causally simple warped spacetimes with non-convex fiber (the extension to the case of two fibers is straightforward). In fact:
\begin{exe}

 \begin{figure}
\centering
\ifpdf
  \setlength{\unitlength}{1bp}%
  \begin{picture}(377.33, 110.89)(0,0)
  \put(0,0){\includegraphics{fig1.pdf}}
  \put(83.67,100.21){\fontsize{9.42}{11.71}\selectfont $x_0$}
  \put(286.41,100.21){\fontsize{9.42}{11.71}\selectfont $x_1$}
  \put(165.89,99.40){\fontsize{9.42}{11.71}\selectfont $T_1$}
  \put(166.29,78.44){\fontsize{9.42}{11.71}\selectfont $T_2$}
  \put(166.70,50.22){\fontsize{9.42}{11.71}\selectfont $T_n$}
  \put(47.67,69.16){\fontsize{13.76}{14.11}\selectfont $H_0$}
  \put(300.28,67.84){\fontsize{13.76}{14.11}\selectfont $H_1$}
  \end{picture}%
\else
  \setlength{\unitlength}{1bp}%
  \begin{picture}(377.33, 110.89)(0,0)
  \put(0,0){\includegraphics{fig1}}
  \put(83.67,100.21){\fontsize{9.42}{11.71}\selectfont $x_0$}
  \put(286.41,100.21){\fontsize{9.42}{11.71}\selectfont $x_1$}
  \put(165.89,99.40){\fontsize{9.42}{11.71}\selectfont $T_1$}
  \put(166.29,78.44){\fontsize{9.42}{11.71}\selectfont $T_2$}
  \put(166.70,50.22){\fontsize{9.42}{11.71}\selectfont $T_n$}
  \put(47.67,69.16){\fontsize{13.76}{14.11}\selectfont $H_0$}
  \put(300.28,67.84){\fontsize{13.76}{14.11}\selectfont $H_1$}
  \end{picture}%
\fi
%   \begin{figure}[H]
% \centering
% \ifpdf
%   \setlength{\unitlength}{1bp}%
%   \begin{picture}(454.19, 131.80)(20,0)
%   \put(0,0){\includegraphics{fig1.pdf}}
%   \put(100.04,120.06){\fontsize{11.77}{13.33}\selectfont $x_0$}
%  \put(70.04,80.06){\fontsize{14.77}{15.33}\selectfont $H_0$}
%  \put(345.36,120.06){\fontsize{11.77}{13.33}\selectfont $x_1$}
%  \put(365.36,80.06){\fontsize{14.77}{15.33}\selectfont $H_1$}
%   \put(199.54,119.08){\fontsize{11.77}{13.33}\selectfont $T_1$}
%   \put(200.02,93.72){\fontsize{11.77}{13.33}\selectfont $T_2$}
%   \put(200.51,59.58){\fontsize{11.77}{13.33}\selectfont $T_n$}
%   \end{picture}%
% \else
%   \setlength{\unitlength}{1bp}%
%   \begin{picture}(454.19, 131.80)(0,0)
%   \put(0,0){\includegraphics{fig1}}
%   \put(100.04,120.06){\fontsize{9.77}{11.33}\selectfont $x_0$}

%   \put(345.36,120.06){\fontsize{7.77}{9.33}\selectfont $x_1$}
%   \put(199.54,119.08){\fontsize{7.77}{9.33}\selectfont $T_1$}
%   \put(200.02,93.72){\fontsize{7.77}{9.33}\selectfont $T_2$}
%   \put(200.51,59.58){\fontsize{7.77}{9.33}\selectfont $T_n$}
%   \end{picture}%
% \fi
  \caption{\label{fig:1} Both hemispheres $H_0$ and $H_1$ are connected by a sequence of immersed tubes $\{T_n\}_n$, where a length-minimizing curve connecting the north pole $x_0$ of $H_0$ to the north pole $x_1$ of $H_1$ through $T_n$ has bigger length than a length-minimizing curve connecting the same points through $T_{n+1}$. This picture is based on \cite[Figure 1]{Bartolo2002}.}
\end{figure}


\footnote{We are thankful to Prof. Miguel Sánchez for bringing this example to our attention.}In \cite[Section 2.1]{Bartolo2002} the authors construct a Riemannian manifold $(M,g)$ containing two points $x_0, x_1\in M$ such that any geodesic $\gamma\subset M$ connecting them satisfies ${\mathrm length}(\gamma)>d(x_0,x_1)$. The example basically consists of two open hemispheres $H_0$, $H_1$ in $\R^3$ connected by a sequence of immersed tubes $(T_n)_n$ of decreasing lengths, and such that any curve joining the corresponding north poles $x_0$ and $x_1$ through $T_n$ is longer than a minimizing curve joining them through $T_{n+1}$ (see Figure \ref{fig:1}). It is assumed also that the lengths of these tubes converge to a number which is strictly positive. In particular, $x_{0}$ and $x_{1}$ cannot be joined by a minimizing geodesic, and thus,
$(M,g)$ is not convex. However, there exists some $\delta>0$ such that $(M,g)$ is $L$-convex for any $L\leq \delta$. Consider now the warped spacetime $V=\R\times_{\alpha}M$ with $\alpha:\R\rightarrow (0,\infty)$ satisfying $\int_{-\infty}^{+\infty}1/\sqrt{\alpha(s)}ds=L\leq\delta$. From Thm. \ref{causi}, $V$ is causally simple.
%However, as commented above, the fiber $(M,g)$ is not weakly convex, since $x_{0}$ and $x_{1}$ cannot be joined by a minimizing geodesic.
\end{exe}

Finally, for the sake of completeness, we include the following simple consequence of \cite[Th. 3.68]{beem}, whose implication to the left is reproved here by using the techniques developed in this paper:

\begin{thm}
A \multiwarped spacetime $(V,g)$ as in (\ref{eq:1-aux}) is globally hyperbolic if and only if $(M_{i},g_{i})$, $i=1,2$, are complete Riemannian manifolds.
\end{thm}

\begin{proof}
%For the implication to the right, assume for instance that $(M_1,g_1)$ is not complete. From \cite{beem}[Th. 3.68], the Lorentzian warped product $H\times_{\alpha_1} M_1$, with $H \equiv (\R,-dt^2)$ and $M_1 \equiv (M_1,g_1)$, is not globally hyperbolic. Since $H\times_{\alpha_1}M_1$ is causal, it must contain some noncompact causal diamond $J^+((t^o,x_1^o))\cap J^-((t^e,x_1^e))$. Then, the causal diamond $J^+(\point{t^o}{x_{1}^o}{x_{2}}) \cap J^-(\point{t^e}{x_1^e}{x_2})$, for any $x_2\in M_2$, is not compact either; in fact, note that
%\[
%J^+(\point{t^o}{x_{1}^o}{x_{2}}) \cap J^-(\point{t^e}{x_1^e}{x_2})=\pi \circ j_{x_2}(J^+(\point{t^o}{x_1^o}{x_2})\cap J^-(\point{t^e}{x_1^e}{x_2}),
%\]
%where $j_{x_2}:\mathbb{R} \times M_{1} \rightarrow \mathbb{R} \times M_{1} \times M_{2}$, with $j_{x_2}(t,x_{1})=(t,x_{1},x_2)$, and the natural projection $\pi:\mathbb{R} \times M_{1}\times M_{2} \rightarrow \mathbb{R} \times M_{1}$ are both, causal relations preserving and continuous maps.
%\medskip
%\footnote{Jony: Sólo se está probando un lado. El otro aparece probado en el .tex, pero comentado para que no aparezca. ¿Por qué?}
Assume that $(M_{i},g_{i})$, $i=1,2$, are complete. Since $(V,g)$ is causally continuous, and thus, causal, it suffices to prove that any causal diamond is sequentially compact (and thus, compact). Let $\{\point{t^n}{x_1^n}{x_2^n}\}_n$ be a sequence in $J^+(\point{t^o}{x_1^o}{x_2^o})\cap J^-(\point{t^e}{x_1^e}{x_2^e})$. Since the fibers are complete, they are convex, and so, we can apply Prop. \ref{p2'}. Hence, there exist constants $0\leq\mu_{1}^{n},\mu_{2}^{n}\leq 1$, $0\leq\overline{\mu}_{1}^{n},\overline{\mu}_{2}^{n}\leq 1$ with $\mu_{1}^{n}+\mu_{2}^{n}=1=\overline{\mu}_{1}^{n}+\overline{\mu}_{2}^{n}$ for all $n$, such that
\[
  \begin{array}{l}
\displaystyle\Integral{t^{o}}{t^n}{\mu_{i}^{n}}{i}{\mu_{k}^{n}} \geq d_{i}(x_{i}^{o},x_{i}^{n})   \\
\displaystyle\Integral{t^{n}}{t^e}{\overline{\mu}_{i}^{n}}{i}{\overline{\mu}_{k}^{n}} \geq d_{i}(x_{i}^{n},x_{i}^{e}),
  \end{array}\quad i=1,2.
\]
In particular, the following inequalities hold for all $n$:
\[
t^o\leq t^n\leq t^e\quad\hbox{and}\quad\int_{t^{o}}^{t^e}\frac{1}{\sqrt{\alpha_i(s)}}ds \geq d_{i}(x_{i}^{o},x_{i}^{n}),\quad i=1,2.
\]
That is, $t^n \in [t^o,t^e]$ and $x_i^{n} \in \overline{B}_{r_i}(x_i^o)$, $r_i:=\int_{t^{o}}^{t^e}\alpha_i^{-1/2}ds$, $i=1,2$, for large $n$.
But, $[t^o,t^e]$ and $\overline{B}_{r_i}(x_i^o)$, $i=1,2$, are compact sets (recall that $(M_{i},g_{i})$, $i=1,2$, are complete). So,
%there some subsequence
%$\{(x_{1}^{m},x_{2}^{m})\}_{m}$ to some point $(x_{1},x_{2}) \in \overline{B}_{r_{1}}(x_{1}^o) \times \overline{B}_{r_{2}}(x_{2}^o)$ and also $\{t_{m}\}_{m}$ converges to some $t' \in [t^o,t^e]$.
%So,
up to a subsequence, $\{\point{t^n}{x_1^n}{x_2^n}\}_m$ converges to some point $\point{t^*}{x_{1}}{x_{2}}\in V$, which necessarily lies into the (closed) causal diamond $J^+(\point{t^o}{x_1^o}{x_2^o}) \cap J^-(\point{t^e}{x_1^e}{x_2^e})$. In conclusion, the causal diamond is sequentially compact, and so, $(V,g)$ is globally hyperbolic.
\end{proof}

%\cambios{Therefore \multiwarped spacetimes $(V,g)$ are causally continuous, stably causal, strongly causal, distinguishing, causal, chronological and non-totally vicious. And with special conditions over the Riemannian fibers, such as $L_{i}$-weakly convexity and completeness, we obtain that \multiwarped spacetimes are causally simple and globally hyperbolic.}

%%% Local Variables:
%%% mode: latex
%%% TeX-master: "DoublyWarpedBoundary2017.tex"
%%% End:


\section{The future c-completion of doubly warped spacetimes}
\label{sec:futurecompletion}
 In this section we are going to study the point set and topological structure of the future c-completion of doubly warped spacetimes.

 Let $\gamma: [\omega,\Omega) \rightarrow V$, $\Omega\leq b$ be a future-directed timelike
curve in $V$. We can reparametrize this curve by using the standard parameter $t$ for the temporal component,
$\gamma(t)=(t,c_{1}(t),c_{2}(t))$. So, from \eqref{eq:3},
\begin{equation}
  \label{eq:4}
  \hbox{length}(c_{i}\mid_{[\omega,\Omega)})
  %=\int_{\alpha}^{\Omega}\sqrt{\mu_{i}\circ
%s}\cdot\alpha_{i}^{-1}\left(-(D\circ s)+\frac{\mu_{1}\circ
%s}{\alpha_{1}}+\frac{\mu_{2}\circ
%s}{\alpha_{2}}\right)^{-1/2}ds
  \leq\int_{\omega}^{\Omega}\frac{ds}{\sqrt{\alpha_{i}(s)}}.
\end{equation}
Next, assume that
%In order to determine $I^-(\gamma)$, there are two main cases to discuss, depending on the finite or infinite value of $\Omega$. In the first case this study can be developed independently of the warping functions. However, in the second case, these functions will play an essential role, determining in part the structure of the future boundary.
%First, note that from \eqref{eq:3}, and for any $\Omega$ (perhaps infinity),
%\begin{equation}
%  \label{eq:4}
%  \hbox{length}(x_{i}\mid_{[0,\Omega)})=\int_{\alpha}^{\Omega}\sqrt{c_{i}\circ
%s}\cdot\alpha_{i}^{-1}\left(-(D\circ s)+\frac{c_{1}\circ
%s}{\alpha_{1}}+\frac{c_{2}\circç
%s}{\alpha_{2}}\right)^{-1/2}\leq\int_{\alpha}^{\Omega}\frac{1}{\sqrt{\alpha_{i}(s)}}ds,
%\end{equation}
%First, let us assume that
$\Omega<b$. Then, the integral in (\ref{eq:4}) is finite. Hence, $\hbox{length}(c_i)<\infty$, and so, $c_{i}(t)\rightarrow x_i^*$ for some $x_i^*\in M_i^C$, where $M_i^C$ denotes the Cauchy completion of the Riemannian manifold $(M_i,g_i)$, $i=1,2$. If, in addition, $x_i^*\in M_i$ for $i=1,2$, the past of $\gamma$ is clearly determined by the triple $(\Omega,x_1^*,x_2^*)$. The following result shows that this is also true if $x_i^*$ belongs to the Cauchy boundary $\partial^C M_i$ for some $i=1,2$.

\begin{prop}\label{pastofcurve}
  Let $\gamma:[\omega,\Omega)\rightarrow V$, $\Omega<b$, be a future-directed timelike curve with $\gamma(t)=(t,c_1(t),c_2(t))$. Then, $\gamma(t)\rightarrow (\Omega, x_1^*,x_2^*)\in (a,b)\times M_1^C\times M_2^C$ for some $(x_1^*,x_2^*)\in M_1^C\times M_2^C$. Moreover, $(t^o,x_1^o,x_2^o)\in I^-(\gamma)$ if, and only if, there exist constants $\mu_{1},\mu_{2}>0$ with $\mu_{1}+\mu_{2}=1$ and
%(with
%$\mu'_{1}+\cdots +\mu'_{n}=1$)
such that
\begin{equation}
  \label{eq:5}
\Integral{t^{o}}{\Omega}{\mu_{i}}{i}{\mu_{k}} >
d_{i}(x^{o}_{i},x^{*}_{i})\qquad\hbox{for $i=1,2$.}
\end{equation}
%In particular, the past of $\gamma$ is determined by the triple $(\Omega,x_1^*,x_2^*)$.
\end{prop}

\begin{proof}
As argued above, the first assertion is a direct consequence of \eqref{eq:4}. So, we only need to focus on the last assertion.

For the implication to the right, assume that $(t^o,x_1^o,x_2^o)\in I^-(\gamma)$. Since the chronological past $I^-(\gamma)$ is an open set, we can take $\epsilon >0$ small enough so that $(t^o+\epsilon,x_1^o,x_2^o)\in I^-(\gamma)$. Consider an increasing sequence $\{t_n\}\subset [\omega,\Omega)$ with $t_n\nearrow \Omega$ and $(t^o+\epsilon,x_1^o,x_2^o)\ll \gamma(t_n)$ for all $n$. For each $n$, Thm. \ref{c0} ensures the existence of constants $\mu_1^n,\mu_2^n>0$, with $\mu^n_1+\mu^n_2=1$, such that:
  \begin{equation}\label{eq:6}
\Integral{t^{o}+\epsilon}{t_n}{\mu^n_{i}}{i}{\mu^n_{k}} >
d_{i}(x^{o}_{i},c_{i}(t_n))\qquad\hbox{for $i=1,2$.}
    \end{equation}
    Observe that $\{c_i(t_n)\}_n\rightarrow x_i^*\in M_i^C$ for $i=1,2$, and so, from the continuity of the distance function $d_i(x_i^o,\cdot)$ on $M_i^C$, necessarily $\{d_i(x_i^o,x_i(t_n))\}_n\rightarrow d_i(x_i^o,x_i^*)$. Even more, since $\{\mu^n_i\}_n\subset [0,1]$, we can assume that $\{\mu^n_i\}_n$ converges (up to a subsequence) to, say, $\mu_i^*$, $i=1,2$, with $\mu_1^*+\mu_2^*=1$. Hence,
    \[
\left\{\frac{\sqrt{\mu^n_{i}}}{\alpha_{i}(s)}\left(\sum_{k=1}^2 \frac{\mu_{k}^n}{\alpha_{k}(s)}\right)^{-1/2} \right\}_n\rightarrow \frac{\sqrt{\mu^*_{i}}}{\alpha_{i}(s)}\left(\sum_{k=1}^2 \frac{\mu_{k}^*}{\alpha_{k}(s)}\right)^{-1/2}\quad\hbox{pointwise on $[t^o,\Omega]$.}
    \]
    Arguing as in the proof of Thm. \ref{causi}, we observe that these functions are bounded by the integrable function $g:[t^o,\Omega]\rightarrow\R$, $g(t)=\alpha_i(t)^{-1/2}$, so the Dominated Convergence Theorem ensures that
    %Recalling now that all previous functions are bounded by the (Lebesgue) integrable function $g:[t^o,\Omega]\rightarrow \R$,
    %$g(t)=\frac{1}{\sqrt{\alpha_i(t)}}$, the Dominated Convergence Theorem\footnote{JONY: Este argumento aparece dos veces...} ensures that:
   \[
\left\{\Integral{t^{o}+\epsilon}{t_n}{\mu^n_{i}}{i}{\mu^n_{k}}\right\}_n\rightarrow \Integral{t^{o}+\epsilon}{\Omega}{\mu^*_{i}}{i}{\mu^*_{k}}.
    \]
    In conclusion, by taking limits in \eqref{eq:6}, we arrive to
    \[
\Integral{t^{o}+\epsilon}{\Omega}{\mu^*_{i}}{i}{\mu^*_{k}}\geq d_i(x_i^o,x_i^*)\qquad\hbox{for $i=1,2$.}
      \]
      In order to conclude the implication, it rests to show that, if $t^o+\epsilon$ is replaced by $t^o$, all previous inequalities are strict. In principle, the only way to avoid this conclusion is by assuming that some $\mu_i^*$ is equal to zero. If, say, $\mu_1^*=0$ (and so, $\mu_2^*=1$), then (i) $d_1(x_1^o,x_1^*)=0$ and (ii)
      \[
\int_{t^o}^{\Omega}\frac{1}{\sqrt{\alpha_2(s)}}ds>d_2(x_2^o,x_2^*).
        \]
        Reasoning as in the proof of Prop. \ref{c0}, a small modification of $\mu_1^*,\mu_2^*$ provides new constants $\mu_1,\mu_2>0$, with $\mu_1+\mu_2=1$, such that
\[
          \left\{\begin{array}{ll}\displaystyle
\Integral{t^{o}}{\Omega}{\mu_{1}}{i}{\mu_{k}}>0= d_1(x_1^o,x_1^*)\\
                   \displaystyle\Integral{t^{o}}{\Omega}{\mu_{2}}{i}{\mu_{k}}> d_i(x_2^o,x_2^*),

            \end{array}
          \right.
          \]
          and we are done.

          \smallskip

          For the converse, assume that \eqref{eq:5} holds for some $(t^o,x_1^o,x_2^o)$ and some constants $\mu_1,\mu_2>0$, with $\mu_1+\mu_2=1$, and let us prove that $(t^o,x^o_1,x^o_2)\in I^-(\gamma)$. Recalling that the inequalities in \eqref{eq:5} are strict and $\gamma(t)=(t,c_1(t),c_2(t))\rightarrow (\Omega,x_1^*,x_2^*)$, there exists some $t^e\in (a,b)$ big enough such that
 \[\Integral{t^{o}}{t^e}{\mu_{i}}{i}{\mu_{k}} >
d_{i}(x^{o}_{i},c_{i}(t^e))\qquad\hbox{for $i=1,2$.}
\]
Hence, from Prop. \ref{c0}, $(t^o,x^o_1,x^o_2)\ll \gamma(t^e)$, as required.

%For the final assertion, recall that...\footnote{Jony: En este último repaso me he dado cuenta que se me olvidó meter aquí la prueba de la ultima frase. Observar que, con lo anterior, queda claro que dos curvas temporales con la tripla $(\Omega,x_1^*,x_2^*)$ determinan el mismo pasado, teniendo que probar que si tienen triplas diferentes generan el pasados diferentes. Si el primer elemento (la parte temporal) es diferente, es muy fácil de probar. En el caso que sea uno de los otros dos, se razona como en la prueba de  }

\end{proof}

We have just proved that the chronological past of a future-directed timelike curve $\gamma$ defined on a finite interval $[\omega,\Omega)$, $\Omega<b$, is determined by its future limit point $(\Omega,x^*_1,x^*_2)$, in the sense that any other future-directed timelike curve $\gamma'$ with the same future limit point has the same chronological past. Next, we are going to prove that if $\gamma'$ is another future-directed timelike curve converging to another triple, then it generates a different past.

\begin{prop}\label{structuraparcialsininfinito}
  Let $(V,g)$ be a {\multiwarped} spacetime as in (\ref{eq:1-aux}). If $\gamma^i:[\omega^i,\Omega^i)\rightarrow V$, $i=1,2$ satisfy $\gamma^i(t)\rightarrow p_i:=(\Omega^i,x_1^i,x_2^i)\in (a,b)\times M_1^C\times M_2^C$ with $p_1\neq p_2$, then $I^-(\gamma^1)\neq I^-(\gamma^2)$.
\end{prop}
\begin{proof}
    The conclusion easily follows if, say, $\Omega^1<\Omega^2$, since in this case $\gamma^2(t)\in I^-(\gamma^2)\setminus I^-(\gamma^1)$ whenever $\Omega^1<t<\Omega^2$. So, we will assume that $\Omega^1=\Omega^2(=:\Omega)$ and, say, $d_1(x_1^1,x_1^2)>0$. Let $t^o$ be close enough to $\Omega<\infty$ so that (recall that $c_1^1(t)\rightarrow x_1^1$)
 \[
\int_{t^o}^{\Omega}\frac{1}{\sqrt{\alpha_1(s)}}ds<\frac{d_1(x_1^1,x_1^2)}{3}\quad\hbox{and}\quad d_1(c_1^1(t^o),x_1^2)>\frac{d_1(x_1^1,x_1^2)}{3},
    \]
and define $q=\gamma^1(t^o)\in I^-(\gamma^1)$.   % $(x^o_1,x_2^o)\in M$ such that $q=(t^o,x^o_1,x_2^o)\in I^-(\gamma_1)$ and $d_1(x^o_1,x^2_1)>d_1(x^1_1,x^2_1)/3$.
    Then, $q\not\in I^-(\gamma^2)$, since, otherwise, from Prop. \ref{pastofcurve},
    \[
\Integral{t^{o}}{\Omega}{\mu_{1}}{1}{\mu_{k}} >
d_{1}(c^{1}_{1}(t^o),x^{2}_{1})\quad\hbox{for some $\mu_1,\mu_2>0$}.
      \]
      But this is not possible since, from the choice of $t^o$,
      \[
d_1(c^1_1(t^o),x^2_1)>\frac{d_1(x^1_1,x^2_1)}{3}>\int_{t^o}^{\Omega}\frac{1}{\sqrt{\alpha_1(s)}}ds>\Integral{t^{o}}{\Omega}{\mu'_{1}}{1}{\mu'_{k}}      \]
   for any positive constants $\mu'_1,\mu'_2$, with $\mu'_1+\mu'_2=1$. In conclusion, $I^-(\gamma^1)\neq I^-(\gamma^2)$ if $p_1\neq p_2$, and the conclusion follows.
\end{proof}
\begin{rem}\label{rem:1} In the proof of previous result the key property is the finite value of the integral $\int_{t^o}^{\Omega}\alpha_i(s)^{-1/2}ds<\infty$, not the finite value of $\Omega$. Of course, the second imply the first, but the same argument can be reproduced if only the first holds.
%
%In previous result condition $\int_{t^o}^{\Omega}\alpha_i(s)^{-1/2}ds<\infty$ has been used in two different ways:
%(i) to ensure that $c_i(t)$ converges to some point $x_i^*\in M_i^C$, $i=1,2$; (ii) to show that any couple of future-directed timelike curves $\gamma^1,\gamma^2:[\alpha,\Omega)\rightarrow V$ whose $i$-components converge to different points in $M_i^C$ have different past.
  \end{rem}
\noindent Props. \ref{pastofcurve} and \ref{structuraparcialsininfinito} together establish a natural bijection between the space $(a,b) \times M_1^C\times M_2^C$ and the set $\hat{V}\setminus \hat{\partial}^{\ncambios{\B}} V$, where $\hat{\partial}^{\ncambios{\B}} V$ denotes the set of TIPs determined by future-directed timelike curves with divergent temporal component ($\Omega=b$). More precisely:

\begin{prop}\label{structuraparcialsininfinito'}
  Let $(V,g)$ be a {\multiwarped} spacetime as in (\ref{eq:1-aux}). Then, there exists a bijection
  \begin{equation}\label{v}
\hat{V}\setminus \hat{\partial}^{\ncambios{\B}} V\; \leftrightarrow\; (a,b)\times M_1^C\times M_2^C,
    \end{equation}
    which maps each indecomposable past set $P\in\hat{V}\setminus \hat{\partial}^{\ncambios{\B}}V$ to the limit point $(\Omega,x^*_1,x^*_2)\in (a,b)\times M_1^C\times M_2^C$ of any future-directed timelike curve generating $P$.
    \end{prop}

% \begin{rem}\label{rem:2}
% For future references,  observe that, in the last part of the proof, we have shown that two future-directed timelike curves $\gamma^j:[\alpha,\Omega)\rightarrow V$ with $x^j_i(t)\rightarrow x_i^j\in M_i^C$ and satisfying that $x_i^1\neq x_i^2$ for some $i=1,2$ have different pasts. Moreover, it is remarkable that such a proof only relies on the fact that $\int_{0}^{\Omega}\frac{1}{\sqrt{\alpha_i(s)}}ds<\infty$.\footnote{MEJORAR ESTE TEXTO!!}
% \end{rem}

\smallskip

Next, we are going to extend the point set structure obtained above to a topological level. We will consider $(a,b)\times M_1^C\times M_2^C$ attached with the product topology. The first result shows the continuity of bijection (\ref{v}) in the left direction:

\begin{prop}\label{prop:topbuenadir}
Let $P_n,P\in \hat{V}\setminus \hat{\partial}^{\ncambios{\B}}V$ with $P_n\equiv (\Omega_n,x_1^n,x_2^n)$ and $P\equiv (\Omega,x_1^*,x_2^*)$, where we are assuming that the triplets belong to $(a,b)\times M_1^C\times M_2^C$. If $(\Omega_n,x_1^n,x_2^n)\rightarrow (\Omega,x_1^*,x_2^*)$, then $P\in \hat{L}(\{P_n\}_n)$.
\end{prop}
\begin{proof}
   First, recall the analytic characterization of the IPs $P$ and $P_n$ provided by Prop. \ref{pastofcurve}: a point $(t,x_1,x_2)\in V$ belongs to $ P$ (resp. $P_n$) if, and only if, there exist positive constants $\mu_1,\mu_2$ ($\mu_1^n,\mu_2^n$) with $\mu_1+\mu_2=1$ ($\mu_1^n+\mu_2^n=1$) and satisfying that, for $i=1,2$:
  \[
    \begin{array}{c}
      \displaystyle\Integral{t}{\Omega}{\mu_{i}}{i}{\mu_{k}}>d_i(x_i,x_i^*) \\
   \left( \displaystyle  \Integral{t}{\Omega_n}{\mu^n_{i}}{i}{\mu^n_{k}}>d_i(x_i,x_i^n)  \right)
    \end{array}
    \]
Second, note that, from the hypotheses, the continuity of the distance map, and the Dominated Convergence Theorem, the following two limits hold: $d_i(x_i,x_i^n)\rightarrow d_i(x_i,x_i^*)$ and
    \[
\Integral{t}{\Omega_n}{\mu_{i}}{i}{\mu_{k}}\rightarrow \Integral{t}{\Omega}{\mu_{i}}{i}{\mu_{k}}\quad\hbox{for any $\mu_1,\mu_2>0$.}
    \]
These two properties directly imply both, $P\subset {\rm LI}(\{P_n\}))$ and $P$ is maximal into ${\rm LS}(\{P_n\})$, i.e., $P\in \hat{L}(\{P_n\})$.
 \end{proof}

In order to prove the continuity of bijection (\ref{v}) in the right direction, we need to impose local compactness on the Cauchy completion, since, otherwise, there exist counterexamples (see, for instance, \cite[Example 4.9]{FHSBuseman})

\begin{prop}\label{topcurvasfinitas}
   Let $(V,g)$ be a doubly warped spacetime as in (\ref{eq:1-aux}) with $M_1^C$ and $ M_2^C$ locally compact. If $\{P_n\}_n$ is a sequence of IPs converging to some IP, $P\equiv (\Omega,x_1^*,x_2^*)\in (a,b) \times M_1^C\times M_2^C$, then $P_n\equiv (\Omega^n,x_1^n,x_2^n)\in (a,b) \times M_1^C\times M_2^C$ for $n$ big enough, and $(\Omega^n,x_1^n,x_2^n)\rightarrow (\Omega,x_1^*,x_2^*)$ with the product topology. As consequence, the bijection (\ref{v}) becomes a homeomorphism.
\end{prop}

\begin{proof}
 The proof follows essentially in the same fashion as \cite[Prop. 5.24]{FHSBuseman}.

  Since the Cauchy completion $M_1^C\times M_2^C$ is locally compact, there exists a pre-compact neighbourhood $U$ of $P\equiv(\Omega,x_1^*,x_2^*)$. Let $\{p^n_m\}_m,\{p_m\}_m\subset V$ be future chains generating $P_n$ and $P$, resp. We can assume without restriction that $\{p_m\}\subset U$.
  It suffices to show the existence of $n_0$ and a map $\mathfrak{m}:\N\rightarrow\N$ such that $p_m^n\in U$ for all $n\geq n_0$ and $m\geq \mathfrak{m}(n)$. In fact, in this case, the temporal component of the sequence $\{p^n_m\}_m$ will not diverge as $m\rightarrow\infty$, and so, $P_n$ can be identified with some $(\Omega^n,x_1^n,x_2^n)\in (a,b) \times M_1^C\times M_2^C\cap\overline{U}$. Moreover, since the result is valid for any pre-compact open set $U$, and $(\Omega,x_1^*,x_2^*)$ admits a countable local neighbourhood basis given by pre-compact open sets, necessarily $(\Omega^n,x_1^n,x_2^n)\rightarrow (\Omega_1^*,x_1^*,x_2^*)$.

In order to prove the statement in previous paragraph, assume by contradiction that, up to subsequences, $p^n_m$ is not contained in $U$ for all $m$ and $n$. Since $P\subset {\mathrm LI}(\{P_n\}_n)$, for each $m\in \N$ there exists $n_0$ such that $p_m\in P_n$ for all $n\geq n_0$. Consider a strictly increasing sequence $\{\mathfrak{n}(m)\}_m$ such that $p_m\in P_{\mathfrak{n}(m)}$. Denote by  $\gamma_m$ the future-directed timelike curve from $p_m$ to some point of a future chain generating $P_{\mathfrak{n}(m)}$. Each $\gamma_m $ intersects the boundary of $\overline{U}$ at a point, say, $(s^m,y_1^m,y_2^m)$. Since $U$ is pre-compact, its boundary is compact and we can assume (up to a subsequence) that $(s^m,y_1^m,y_2^m)\rightarrow (s^*,y^*_1,y^*_2)$ for some $(s^*,y^*_1,y^*_2)\in (a,b) \times M_1^C\times M_2^C$. Let us denote by $P'$ the indecomposable set associated to $q=(s^*,y^*_1,y^*_2)$ which is necesssarily different from $P$ (as $q$ belong to the boundary of $U$); and by $\{q_m\}_m$ a future chain generating $P'$. Next, we are going to show that $P'$ breaks the maximality of $P$ into ${\mathrm LS}(\{P_n\})$, in contradiction with $P\in \hat{L}(\{P_n\}_n)$.

Let us show that $P'\subset {\mathrm LS}(\{P_n\})$. First recall that,  for each $m\in \N$, the set $I^+(q_m)$ is an open set containing $q$:
% \footnote{Esto es evidente si $q\in M$, pero no lo parece tanto cuando $q\in \overline{M}$. ¿Incluimos un lemma auxiliar?}.
in fact, this is straightforward if $q\in (a,b)\times M_1\times M_2$; otherwise, it suffices to realize that the characterization of the chronological relation given in Prop. \ref{c0} (which is an open property) extends to the set $(a,b)\times M_1^C\times M_2^C$ (see Prop. \ref{pastofcurve}).
In particular, since $\{(s^k,y_1^k,y_2^k)\}\rightarrow q$, it follows that $(s^k,y_1^k,y_2^k)\in I^+(q_m)$ for $k$ big enough. But, from construction, $(s^k,y_1^k,y_2^k)\in P_{\mathfrak{n}(k)}$, so $q_m\in P_{\mathfrak{n}(k)}$ for $k$ big enough. Therefore, $P'\subset {\mathrm LS}(\{P_{\mathfrak{n}(m)}\}_m)$.

It rests to show that $P\subsetneq P'$; that is, any point $p_m$ of the future chain generating $P$ is contained in $P'$. From construction, $p_m=(t^m,x_1^m,x_2^m)\ll p_k \ll (s^k,y_1^k,y_2^k)$ for all $k> m$. Let $\epsilon>0$ be small enough so that $p_m^\epsilon=(t^m+\epsilon,x_1^m,x_2^m)\ll p_{m+1}$, and thus, $p_m^\epsilon\ll (s^k,y_1^k,y_2^k)$ for all $k>m$. From Prop. \ref{c0}, there exist positive constants $\mu_1^k$ and $\mu_2^k$, with $\mu_1^k+\mu_2^k=1$, such that:
\[
\int_{t^m+\epsilon}^{s^k} \frac{\sqrt{\mu^k_i}}{\alpha_{i}(s)}\left(\sum_{l=1}^{2} \frac{\mu^k_l}{\alpha_{k}(s)} \right)^{-1/2}ds>
d_{i}(x^{m}_{i},y^{k}_{i})\qquad\hbox{for $i=1,2$.}
  \]
 But $\{(s^k,y^k_1,y^k_2)\}\rightarrow (s^*,y_1^*,y_2^*)$. By continuity, and up to a subsequence, there exist positive constants $\mu_1^*,\mu_2^*$, with $\mu_1^*+\mu_2^*=1$, such that:
  \[
\int_{t^m+\epsilon}^{s^*} \frac{\sqrt{\mu^*_i}}{\alpha_{i}(s)}\left(\sum_{l=1}^{2} \frac{\mu^*_l}{\alpha_{k}(s)} \right)^{-1/2}ds \geq
d_{i}(x^{m}_{i},y^{*}_{i})\qquad\hbox{for $i=1,2$.}
    \]
    Now if we replace in previous expression $t^m+\epsilon$ by $t^m$, at least one of previous inequalities becomes strict. Then, reasoning as in the proof of Prop. \ref{c0}, we arrive to

      \[
\int_{t^m}^{s^*} \frac{\sqrt{\mu'_i}}{\alpha_{i}(s)}\left(\sum_{l=1}^{2} \frac{\mu'_l}{\alpha_{k}(s)} \right)^{-1/2}ds >
d_{i}(x^{m}_{i},y^{*}_{i})\qquad\hbox{for $i=1,2$,}
      \]
      for some slightly modified constants $\mu'_i$ from $\mu^*_i$. Therefore, the point $p_m$ belongs to $P'$ (recall Prop. \ref{pastofcurve}). Since this argument works for any point of the sequence $\{p_m\}_m$ generating $P$, the inclusion $P\subsetneq P'$ follows.

    \smallskip

For the last assertion, observe that previous argument gives the continuity of bijection \eqref{v} to the right direction, while Prop. \ref{prop:topbuenadir} ensures the continuity to the left one.

  \end{proof}

 Next, we analyze the case $\Omega=b$. In this case, the finiteness/infiniteness of the integrals associated to the warping functions becomes crucial, so we will consider several subsections to discuss it.
%the past of such a curves (and so, the structure of $\hat{\partial}^{\infty}V$) under some simple conditions on such warping functions.

\subsection{Finite warping integrals}

First, we consider the case when the integrals associated to the warping functions are both finite:
% \cambios{The results will be given in full generality, so we will consider doubly warped models as in \eqref{eq:1-aux}. Then, the integral conditions on the warping functions should read as:}
\begin{equation}
  \label{eq:7}
  \int_{\C}^{b}\frac{1}{\sqrt{\alpha_i(s)}}ds<\infty, \qquad \hbox{$i=1,2$}\quad\hbox{for some $\C\in (a,b)$.}
\end{equation}
In this case, the following result provides the point set and topological structure of the future c-boundary:
\begin{thm}\label{futurestructurefiniteconditions}
  Let $(V,g)$ be a {\multiwarped} spacetime as in (\ref{eq:1-aux}), and assume that the integral conditions \eqref{eq:7} hold. Then, there exists a bijection
  \begin{equation}
    \label{eq:8}
    \hat{V}\; \leftrightarrow \; (a,b]\times M_1^C\times M_2^C
  \end{equation}
  which maps each IP $P\in \hat{V}$ to the limit point $(\Omega,x_1,x_2)\in (a,b]\times M_1^C\times M_2^C$ of any future-directed timelike curve generating $P$. Moreover, if $M_1^C,M_2^C$ are locally compact, this bijection becomes an homeomorphism.
\end{thm}
\begin{proof}
%\cambios{As we recall in Rem. \ref{rem:infinito}, we can consider directly that $(a,b)\equiv \R$}.  Let us begin with the point set structure.
For the first assertion, we only need to prove the corresponding bijection between $\hat{\partial}^{\ncambios{\B}} V$ and $\{b\}\times M_1^C\times M_2^C$ (recall Prop. \ref{structuraparcialsininfinito'}). But this follows from the same arguments as in the proofs of Props. \ref{pastofcurve} and \ref{structuraparcialsininfinito} (recall \eqref{eq:7} and Remark \ref{rem:1}).
%since the key point in those arguments was the finiteness of the corresponding warping functions \eqref{eq:7}.
%
%
%with $\Omega$ replaced by $\infty$, since
%
%
%and taking into account the warping integral hypothesis \eqref{eq:7} (see Remark \ref{rem:1}).

For the second assertion, the continuity to the left of bijection (\ref{eq:8}) follows as in Prop. \ref{prop:topbuenadir}, just taking into account that the integral condition \eqref{eq:7} must be used in order to apply the Dominated Convergence Theorem. For the continuity to the right, assume that $P\in \hat{L}(\{P_n\}_n)$, with $P=I^-(\gamma)$, $P_n=I^-(\gamma_n)$, and being $\gamma:[\omega,\Omega)\rightarrow V$, $\gamma_n:[\omega_n,\Omega_n)\rightarrow V$ future-directed timelike curves. Let $(\Omega,x_1^*,x_2^*)$ and $(\Omega_n,x_1^n,x_2^n)$ be the limit points in $(a,b]\times M_1^C\times M_2^C$ of $\gamma$ and $\gamma_n$, resp. We need to prove that $(\Omega_n,x_1^n,x_2^n)\rightarrow (\Omega,x_1^*,x_2^*)$. Observe that, if $\Omega<b$, then the result follows from Prop. \ref{topcurvasfinitas}, so we will focus on the case $\Omega=b$.

First, note that $\Omega_n\rightarrow b$. In fact, otherwise, there exists $\Omega^*<b$ and a subsequence $\{\Omega_{n_k}\}$ such that $\Omega_{n_k}<\Omega^*$ for all $k\in \N$. But, in this case, $P_{n_k}$ will not contain any point $\gamma(t)$ with $t>\Omega^*$, and so, $P\not\subset  {\mathrm LI}(\{P_n\})$.

Assume by contradiction that, say, $\{x_1^n\}_n$ does not converge to $ x_1^*$. Then, up to a subsequence, there exists $\epsilon_0>0$ such that $d_1(x_1^n,x_1^*)>\epsilon_0$. Take $t^0$ big enough so that

  \[
\int_{t^o}^{b}\frac{1}{\sqrt{\alpha_1(s)}}ds<\frac{\epsilon_0}{3}.
    \]
    Take $(x_1^o,x_2^o)\in M_1\times M_2$ such that $q=(t^o,x_1^o,x_2^o)\in I^-(\gamma)=P$ with $d_1(x_1^o,x_1^*)<\epsilon_0/3$. It suffices to show that $q$ does not belong to $P_n$ for any $n$, since, in this case, we arrive to a contradiction with $P\subset {\mathrm LI}(P_n)$.  So, assume that $q\in P_n$ for all $n$. From  Prop. \ref{pastofcurve}, there exists some $\mu^n_1,\mu^n_2>0$ such that
    \[
\Integral{t^{o}}{\Omega^n}{\mu^n_{1}}{1}{\mu^n_{k}} >
d_{1}(x^{o}_{1},x^{n}_{1}).
      \]
      This is in contradiction with the fact that, for any pair of positive constants $\mu'_1,\mu_2'>0$ with $\mu_1'+\mu'_2=1$,
      \[
d_1(x^o_1,x^n_1)>\frac{2}{3} d_1(x^*_1,x^n_1)> \frac{2}{3}\epsilon_0>\int_{t^o}^{b}\frac{1}{\sqrt{\alpha_1(s)}}ds>\Integral{t^{o}}{b}{\mu'_{1}}{1}{\mu'_{k}}.
        \]
%for any pair of positive constants $\mu'_1,\mu_2'$ with $\mu_1'+\mu'_2=1$.
%Hence, $q$ does not belong to $P_n$ for any $n$, in contradiction with $P\in {\mathrm LI}(P_n)$. In conclusion, $\{x^n_1\}_n\rightarrow x_1^*$, as required.
%\footnote{JONY: Observar que al probar  que la sucesión $\{x_1^n\}$ converge a $x_1^*$, se está usando el mismo argumento que en la prueba de la Prop. \ref{structuraparcialsininfinito}. Probablemente se debería poder extraer el argumento para simplemente llamarlo en los dos lados...}
\end{proof}

\subsection{One infinite warping integral}

Let us consider now the case when just one of the warping integrals is infinite, say:
\begin{equation}
  \label{eq:9}
 \int_{\C}^{b}\frac{1}{\sqrt{\alpha_1(s)}}ds<\infty \qquad \hbox{and}\qquad \int_{\C}^{b}\frac{1}{\sqrt{\alpha_2(s)}}ds=\infty.
\end{equation}

%\cambios{In general, we will assume conditions as
%\begin{equation}
%  \label{eq:9}
% \int_{c}^{b}\frac{1}{\sqrt{\alpha_1(s)}}ds<\infty \qquad \hbox{and}\qquad \int_{c}^{b}\frac{1}{\sqrt{\alpha_2(s)}}ds=\infty,
%\end{equation}
%but, as we recall in Rem. \ref{rem:infinito}, there is no loss of generality if we assume in \eqref{eq:9} that $c=0$ and $b=\infty$.
%}

%Now, if we consider a future-directed timelike curve $\gamma:[\alpha,\infty)\rightarrow V$, $\gamma(t)=(t,c_1(t),c_2(t))$, the curve $c_2$ is not forced to converge to some point of the Cauchy completion $M_2^C$, which will provide a different point set structure for $\hat{\partial}^{\infty}V$.
%
%Let us analyze the possible structure of $\hat{\partial}^{\infty}V$: For each $x_1\in M_1$, the fiber $\R\times \{x_1\}\times M_2$ is isometric to the Generalized Robertson-Walker $(\R\times M_2,\tilde{g}_2)$, where $\tilde{g}_2=-dt^2+\alpha_2(t)g_2$ (from now on denoted by $\R\times_{\alpha_2}M_2$). According to Section \ref{sec:Robertson}, the point set and topological structure of $\hat{\partial}^{\infty}V$ is determined by the corresponding proper Busemann completion ${\cal B}(M_2)$: two different Busemann functions in $M_2$ generate two different indecomposable sets in $\R\times_{\alpha_2} M_2$, and so, two different indecomposable sets in $V$. Moreover, the finite integral conditions ensure that these indecomposable sets in $V$ will depend also on the selected point in $M_1$, that is, the same Buseman function in ${\cal B}(M_2)$ will generate different indecomposable sets in $V$, at least one for each $x_1\in M_1$.
%
%\smallskip
%
%  In the next subsections we will formalize these ideas by considering, not only points of $M_1$, but also of $M_1^C$. Moreover, we will show that the structure of $\hat{\partial}^{\infty}V$ depicted in previous paragraph extends naturally to the topological level, giving a complete characterization of the future c-completion. These results are particularly technical, being necessary to present them gradually. We will focus first on the point set structure.

\subsubsection{Point set structure}

The first integral in condition \eqref{eq:9} plus \eqref{eq:4} ensures that any future-directed timelike curve $\gamma:[\omega,b)\rightarrow V$,  $\gamma(t)=(t,c_1(t),c_2(t))$, satisfies that $c_1(t)\rightarrow x_1^*\in M_1^C$. Moreover, the second integral ensures that the associated Generalized Robertson-Walker spacetime $((a,b) \times M_2,-dt^2+\alpha_2g_2)$ corresponds with the model studied in Section \ref{sec:Robertson}. In particular, since the curve $\sigma(t)=(t,c_2(t))$ is also a future-directed timelike in that spacetime, we can consider the Busemann function $b_{c_2}\in B(M_2)\cup \{\infty\}$.

Next, our aim is to show that the chronological past of $\gamma$ is determined by both, $x_1^*\in M_1^C$ and the Busemann function $b_{c_2}\in B(M_2)\cup \{\infty\}$. Let us begin with the following result:
\begin{prop}\label{prop:conddiferbordedif}
  Let $(V,g)$ be a {\multiwarped} spacetime and assume that the integral conditions \eqref{eq:9} are satisfied. Consider two future-directed timelike curves $\gamma^i:[\omega,b)\rightarrow V$, $\gamma^i(t)=(t,c_1^i(t),c_2^i(t))$, with $c_1^i(t)\rightarrow x_1^i\in M_i^C$, $i=1,2$. If $(x_1^1,b_{c_1})\neq (x_1^2,b_{c_2})$ then $I^-(\gamma^1)\neq I^-(\gamma^2)$.
\end{prop}
\begin{proof} If $x_1^1\neq x_1^2$, we can reason as in the proof of Prop. \ref{structuraparcialsininfinito} (taking $\Omega=\Omega'=b$ and $x_1^1\neq x_1^2$; Remark \ref{rem:1} and the first integral condition in \eqref{eq:9}). So, it suffices to consider the case $b_{c_2^1}\neq b_{c_2^2}$.

Let $\sigma_i(t)=(t,c_2^i(t))$, $i=1,2$, be two future-directed timelike curves on the Generalized Robertson-Walker spacetime $\left( (a,b)\times M_2,-dt^2+\alpha_2 g_2\right)$. Since $b_{c_2^1}\neq b_{c_2^2}$, necessarily $I^{-}(\sigma_1)\neq I^{-}(\sigma_2)$. Assume, for instance, that  $(t^0,y_2)\in I^{-}(\sigma_2)\setminus I^{-}(\sigma_1)$ (the other case is analogous). Then, taking into account the characterization in \eqref{eq:27}, it follows that

  % Assume, for instance, that $b_{c_2^1}(y_2)<b_{c_2^2}(y_2)$ for some $y_2\in M_2$, and let us show that $I^-(\gamma^2)\not\subset I^-(\gamma^1)$. \cambios{NECESITO VER CÓMO METER QUE AMBAS FUNCIONES PUEDEN SER POSITIVAS...}
  % From (\ref{eq:9}) and the continuity of the integral with respect to the superior limit of integration, there exists $t^o$ such that\footnote{J.L.: Aqui parece que se necesita $b_{c_2^i}(y_2)>0$ $i=1,2$.}
  \begin{equation}\label{eq:b}
b_{c_2^1}(y_2)<\int_{\C}^{t^o}\frac{1}{\sqrt{\alpha_2(s)}}ds <b_{c_2^2}(y_2).
\end{equation}
From the first inequality in (\ref{eq:b}),
    \[
      \begin{array}{l}
  (b_{c^1_2}(y_2)=)\lim_{t\rightarrow b} \left(\int_\C^{t}\frac{1}{\sqrt{\alpha_2(s)}}ds-d_2(y_2,c_2^1(t))\right)\leq\int_\C^{t^o}\frac{1}{\sqrt{\alpha_2(s)}}ds\Rightarrow\\ \Rightarrow  \lim_{t\rightarrow b}\left( \int_{t^o}^{t}\frac{1}{\sqrt{\alpha_2(s)}}ds-d_2(y_2,c_2^1(t))\right)\leq 0.
     \end{array}
 \]
 Therefore, since the function $t\mapsto \left(\int_{t^o}^{t}\frac{1}{\sqrt{\alpha_2(s)}}ds-d_2(y_2,c_2^1(t))\right)$ is increasing, we deduce
      \begin{equation}\label{b}
\int_{t^o}^{t}\frac{1}{\sqrt{\alpha_2(s)}}ds<d_2(y_2,c_2^1(t))\quad\hbox{for all $t$.}
        \end{equation}

Let us show the existence of $x_1^o\in M_1$ such that $q=(t^o,x_1^o,y_2)\in I^-(\gamma^2)$. From the inequality
        \[
\int_\C^{t^o}\frac{1}{\sqrt{\alpha_2(s)}}ds <b_{c_2^2}(y_2)=  \lim_{t\rightarrow b}\left(\int_\C^t\frac{1}{\sqrt{\alpha_2(s)}}ds-d_2(y_2,c_2^2(t))\right),
          \]
        there exists $t'>t^o$ big enough such that
        \[
          \int_{t^o}^{t'}\frac{1}{\sqrt{\alpha_2(s)}}ds> d_2(y_2,c_2^2(t')).
        \]
        From continuity, we can find positive constants $\mu_1,\mu_2$, with $\mu_1+\mu_2=1$, such that
        \[
\left\{
  \begin{array}{l}
    \displaystyle\Integral{t^0}{t'}{\mu_2}{2}{\mu_k}>d_2(y_2,c_2^2(t'))\\
    \displaystyle \Integral{t^0}{t'}{\mu_1}{1}{\mu_k}>0.
  \end{array}
\right.
          \]
          So, if we take $x_1^o$ close enough to $c^2_1(t')$ so that
          \[d_1(x_1^o,c^2_1(t'))<\Integral{t^o}{t'}{\mu_1}{1}{\mu_k},\] Prop. \ref{c0} ensures that $(t^o,x_1^o,y_2)\ll \gamma^2(t')$, and thus, $q=(t^o,x_1^o,y_2)\in I^-(\gamma^2)$.

          \smallskip

          On the other hand, for any pair of positive constants $\mu_1,\mu_2>0$ with $\mu_1+\mu_2=1$, necessarily
        \[
\Integral{t^{o}}{t}{\mu_{2}}{2}{\mu_{k}}<\int_{t^o}^t \frac{1}{\sqrt{\alpha_2(s)}}ds< d_2(y_2,c_2^1(t))\quad\hbox{for all $t>t^0$,}
          \]
          where (\ref{b}) has been used in the last inequality. Therefore, from Prop. \ref{c0}, $q\not\ll \gamma^1(t)$ for all $t>t^o$, and thus, $q\not\in I^-(\gamma^1)$.
\end{proof}

  %Proposition \ref{prop:conddiferbordedif} establishes that $\hat{\partial}^\infty V$ contains a set identifiable with the product space $M_1^C\times {\cal B}(M_2)$. Now, we are going to prove that, indeed, there are no additional points in $\hat{\partial}^\infty V$. To this aim, first we will focus on the first spatial component of the curve.

\begin{lemma}\label{lemma:aux3}
Let $\gamma:[\omega,\Omega)\rightarrow V$, $\gamma(t)=(t,c_1(t),c_2(t))$ be a future-directed timelike curve with $c_1(t)\rightarrow x_1^*\in M_1^C$. If $\sigma=\{(t_n,x_1^n,c_2(t_n))\}_n\subset V$ satisfies $\{t_n\}_n\rightarrow \Omega$ and $x_1^n\rightarrow x_1^*$, then $I^-(\gamma)\subset {\mathrm LI}(\{I^-(t_n,x_1^n,c_2(t_n))\}_n)$.
%$P=I^-(\gamma)$ with $\gamma:[\alpha,\Omega)\rightarrow V$, $\gamma(t)=(t,c_1(t),c_2(t))$ and $c_1(t)\rightarrow x_1^*\in M_1^C$. Consider a sequence $\sigma=\{(t_n,x_1^n,c_2(t_n))\}_n\subset V$ with $\{t_n\}_n\rightarrow \Omega$\footnote{Cambiado estrictamente creciente a convergencia, cuidado...} and $x_1^n\rightarrow x_1^*$. Then, $P\subset {\mathrm LI}(\{I^-(t_n,x_1^n,c_2(t_n))\}_n)$.
\end{lemma}
\begin{proof}
Assume by contradiction the existence of some point $q=(t^o,x_1^o,x_2^o)\in I^-(\gamma)$ such that $q\not\ll (t_n,x^n_1,c_2(t_n))$ for infinitely many $n$. From the open character of the chronological relation, we can assume that $x_1^o\neq x_1^*$. Moreover, for $\epsilon>0$ small enough, it follows that $q_{\epsilon}=(t^o+\epsilon,x_1^o,x_2^o)\in I^{-}(\gamma)$.

Assume that, up to a subsequence, $q_{\epsilon}\ll \gamma(t_n)$ for all $n$. From Prop. \ref{c0}, there exist positive constants $\mu_1^n,\mu_2^n>0$, with $\mu_1^n+\mu_2^n=1$, such that
  \[
\Integral{t^{o}+{\epsilon}}{t_n}{\mu^n_{i}}{i}{\mu^n_{k}} >
d_{i}(x^{o}_{i},c_{i}(t_n))\qquad\hbox{for $i=1,2$.}
    \]
We can assume without restriction that $\{\mu_i^n\}_n$ converges to some point $\mu_i^*$, $i=1,2$. Since $q_{\epsilon}\not\in I^-((t_n,x^n_1,c_2(t_n)))$ (recall that $q\not\ll (t_n,x_1^n,c_2(t_n))$), necessarily
   \[
\left(d_{1}(x^{o}_{1},c_{1}(t_n))< \right)\Integral{t^{o}+{\epsilon}}{t_n}{\mu^n_{1}}{1}{\mu^n_{k}}\leq d_1(x_1^o,x^n_1),
      \]
      the last inequality by Prop. \ref{c0}. From the hypothesis, the first and third element in previous expression converge to $d_1(x_1^o,x_1^*)>0$. Moreover, from \eqref{eq:9}, the integral in the middle is also finite. Hence,
      \begin{equation}\label{eq:c}
\left\{\Integral{t^{o}+\epsilon}{t_n}{\mu^n_{1}}{1}{\mu^n_{k}}\right\}_n\rightarrow \Integral{t^{o}+\epsilon}{\Omega}{\mu^*_{1}}{1}{\mu^*_{k}}=d_1(x_1^o,x_1^*)<\infty.
        \end{equation}
        %and thus,
%        \[
%\Integral{t^{o}+\epsilon}{\infty}{\mu^*_{1}}{1}{\mu^*_{k}}=d_1(x_1^o,x_1^*).
%          \]
          In particular, since $x_1^o\neq x_1^*$, necessarily $\mu_1^*\neq 0$, and so,
        \begin{equation}\label{eq:cc}
\Integral{t^{o}}{\Omega}{\mu^*_{1}}{1}{\mu^*_{k}}>d_1(x_1^o,x_1^*).
          \end{equation}
          Finally, from (\ref{eq:c}) and (\ref{eq:cc}),
        \[
\int_{t^o}^{t^n}\frac{\sqrt{\mu_1^n}}{\alpha_1}\left(\sum_{k=1}^{2}\frac{\mu_k^n}{\alpha_k}\right)^{-1/2}ds>d_1(x_1^o,x^n_1)\quad\hbox{for $n$ big enough,}
          \]
which implies that $q=(t^o,x_1^o,x_2^o)\in I^-((t_n,x^n_1,c_2(t_n)))$ for $n$ big enough, a contradiction.
\end{proof}

\noindent This Lemma has the following direct consequence:

\begin{lemma}\label{lemma:aux1}
 Let $\gamma^i:[\omega,b)\rightarrow V$, $\gamma^i(t)=(t,c_1^i(t),c_2(t))$, $i=1,2$, be future-directed timelike curves. If $c_1^i(t)\rightarrow x_1^*\in M^C_1$, $i=1,2$, then $I^-(\gamma^1)=I^-(\gamma^2)$.
\end{lemma}
\begin{proof}
 Let us focus on the inclusion to the right (the other one is analogous). Consider the sequence $\sigma=\{(t_n,c_1^2(t_n),c_2(t_n))\}_n$, where $\{t_n\}_n\nearrow \infty$. For any $p\in I^-(\gamma^1)$, Lemma \ref{lemma:aux3} ensures the existence of $n_0$ such that $p\in I^-((t_n,c_1^2(t_n),c_2(t_n)))\subset I^-(\gamma^2)$ for all $n\geq n_0$, as desired.
\end{proof}

%These results provide some freedom in the choice of the curve $\gamma$ generating the TIP $P$, whenever the first spatial component is converging to an appropriate point. The following result provides a similar property, but now for the second spatial component. The difference is that now this freedom is restricted to the corresponding Busemann function.

\begin{lemma}\label{lemma:aux2}

Let $\gamma^i:[\omega,b)\rightarrow V$, $\gamma^i(t)=(t,c_1(t),c_2^i(t))$, $i=1,2$, be future-directed timelike curves. If $b_{c_2^1}=b_{c_2^2}$, then $I^-(\gamma^1)=I^-(\gamma^2)$.
%Let $\gamma^j:[\alpha,\infty)\rightarrow V$ be curves in $V$ with $\gamma^j(t)=(t,c_1^j(t),c_2^j(t))$ for $j=1,2$. If $c_1^1=c_1^2(=c)$ and $b_{c_2^1}=b_{c_2^2}$, then $I^-(\gamma^1)=I^-(\gamma^2)$.
\end{lemma}
\begin{proof}
Since the first warping integral is finite (recall (\ref{eq:9})), the spatial component $c_1$ admits some limit point $x_1^*\in M_1^C$. Assume by contradiction that, say, $q=(t^o,x_1^o,x_2^o)\in I^-(\gamma^2)\setminus I^-(\gamma^1)$. It is not a restriction to additionally assume that $x_1^o\neq x_1^*$. Let $\epsilon>0$ small enough such that $q_\epsilon=(t^o+\epsilon,x_1^o,x_2^o)\in I^-(\gamma^2)\setminus I^-(\gamma^1)$. Since $q_\epsilon\in I^-(\gamma^2)$, there exists an increasing sequence $\{t_n\}\nearrow b$ such that $q_\epsilon\ll \gamma^2(t_n)$ for all $n$. Then, from  Prop. \ref{c0}, there exist positive constants $\mu_1^n, \mu_2^n >0$, with $\mu_1^n+\mu_2^n=1$, for each $n$, such that
  \begin{equation}\label{eq*}
    \left\{\begin{array}{l}
    \displaystyle  \Integral{t^o+\epsilon}{t_n}{\mu^n_{1}}{1}{\mu^n_{k}}>
             d_{1}(x^o_{1},c_{1}(t_n))\\
\displaystyle\Integral{t^o+\epsilon}{t_n}{\mu^n_{2}}{2}{\mu^n_{k}}>
             d_{2}(x^o_{2},c^2_{2}(t_n)).
    \end{array}\right.
    \end{equation}
    It is not a restriction to assume that each sequence $\{\mu^n_i\}_n$ is convergent to $\mu_i^*$ for $i=1,2$. Next, we are going to prove that the sequences can be chosen satisfying $\mu_{1}^{*} \neq 1$:


   \smallskip

{\em Claim. The sequences $\{\mu^n_i\}_n$  can be chosen so that $\mu_1^*\neq 1$ (and thus, $\mu_2^*\neq 0$).}


  \smallskip


{\em Proof of the Claim.} Let us prove that, if we have a sequence $\{t_n\}_n$ such that $q=(t^o,x_1^o,x_2^o)\ll \gamma(t_n)=(t_n,c_1(t_n),c_2(t_n)) $, we can find sequences $\{\mu_i^n\}_n$, with $\mu_1^n+\mu_2^n=1$, which converge, up to a subsequence, to $\mu_1^*\neq 1$ and $\mu_2^*\neq 0$, such that
\begin{equation}
  \label{eq:30}
    \Integral{t^o}{t_n}{\mu^n_i}{i}{\mu^n_{k}}-
             d_{i}(x^o_{i},c_{i}(t_n))>0\quad \hbox{for $i=1,2$.}
\end{equation}
Observe that Prop. \ref{c0} ensures the existence of such convergent sequences $\{\mu_i^n\}_n$  without the statement about the limits. Assume that $\mu_1^*= 1$. By using standard arguments (that is, working with the point $q_{\epsilon}=(t^o+\epsilon,x_1^o,x_2^o)$ as before, and recalling that $\mu_1^n\geq \frac{1}{2}$ for $n$ big enough), we can take limits on \eqref{eq:30} preserving the strict inequality. So,
\[
\lim_{n\rightarrow \infty}\left(\Integral{t^o}{t_n}{\mu^n_1}{1}{\mu^n_{k}}-
             d_{1}(x^o_{1},c_{1}(t_n))\right)=\int_{t^o}^{b}\frac{1}{\sqrt{\alpha_1(s)}}ds-
             d_{1}(x^o_{1},x_1^*)>0
           \]
where $c_1(t_n)\rightarrow x_1^*$. Now take $\overline{\mu}_2^*>0$ small enough such that $\overline{\mu}_1^*=1-\overline{\mu}_2^*>0$ and such that
           \[
\Integral{t^o}{b}{\overline{\mu}_1^*}{1}{\overline{\mu}^*_k}-d_{1}(x^o_{1},x_1^*)>0
             \]
Now, define $\overline{\mu}_1^n=\mu_1^n-\overline{\mu}_2^*$ and $\overline{\mu}_2^n=\mu_2^n+\overline{\mu}_2^*$. As $\mu_{1}^{n} \rightarrow 1$,  we have that $\overline{\mu}_{1}^{n}>0$ for large $n$ and that
$\overline{\mu}_{1}^{n} \rightarrow \overline{\mu_{1}^{*}}$, therefore by  the Dominated Convergence Theorem (recall the integral condition for $\alpha_1$) we have:
\[
\Integral{t^o}{b}{\overline{\mu}_1^*}{1}{\overline{\mu}^*_k}=lim_{n} \Integral{t^o}{t_{n}}{\overline{\mu}_1^n}{1}{\overline{\mu}^n_k},
\]
Hence,
\[
  lim_{n}\left(\Integral{t^o}{t_{n}}{\overline{\mu}_1^n}{1}{\overline{\mu}^n_k}-d_{1}(x_{1}^{o},c_{1}(t_{n}))\right)
  % =\Integral{t^0}{\infty}{\overline{\mu_1^*}}{1}{\overline{\mu^*_k}}-d_{1}(x_{1}^o,x_{1}^*)
  >0,
\]
and so  for large $n$ we can take $\overline{\mu}_1^n$ and $\overline{\mu}_2^n$ satisfying
\[
\Integral{t^o}{t_n}{\overline{\mu}^n_1}{1}{\overline{\mu}^n_{k}}-
             d_{1}(x^o_{1},c_{1}(t_n))>0.
  \]
  Moreover, as $\overline{\mu}_1^n<\mu_1^n$ and $\overline{\mu}_2^n>\mu_2^n$, it easily follows that:

  \[
\Integral{t^o}{t_n}{\overline{\mu}^n_2}{2}{\overline{\mu}^n_{k}}>\Integral{t^o}{t_n}{\mu^n_2}{2}{\mu^n_{k}}\left(>d_2(x_2^0,c_2(t_n))\right).
    \]
In conclusion, equation \eqref{eq:30} is also true with the sequences $\{\overline{\mu}_i^n\}_n$ and $\{\overline{\mu}_1^n\}_n\rightarrow \overline{\mu}_{1}^{*}= 1-\overline{\mu}_{2}^{*}\neq 1$, which proves the claim.
%... and taking limits on the first previous inequality we deduce that
  %  \[
%\int_{t^0}^{\infty}\frac{1}{\sqrt{\alpha(s)}}ds=
  %           d_{1}(x^0_{1},x_{1}^*),
   %   \]
    %  (we cannot have strict inequality as, in such a case, we can perturb the sequence $\{\mu_1^n\}_n$ such that $\mu_1^*\neq 1$). But this is not possible as, in such a case,

  %    \[
%d_{1}(x^0_{1},x_{1}^*)=\int_{t^0}^{\infty}\frac{1}{\sqrt{\alpha(s)}}ds>\Integral{t^0}{\infty}{\mu'_1}{1}{\mu'_k}
  %      \]
    %  for any pair of positive constant sequences $\mu'_1,\mu'_2$ with $\mu'_1+\mu'_2=1$. But this is a contradiction with....


    \smallskip


  Continuing with the proof of the lemma, note that $\gamma^1$ and $\gamma^2$ share the same first spatial component $c_1$, the first integral condition (\ref{eq*}) coincides for both curves. Therefore, since $q_\epsilon\not\in I^-(\gamma^1)$, necessarily (recall Prop. \ref{c0}):
    \begin{equation}
      \label{eq:11}
d_2(x^o_2,c^1_2(t_n))\geq \Integral{t^o+\epsilon}{t_n}{\mu^n_{2}}{2}{\mu^n_{k}}\left(>
             d_{2}(x^o_{2},c^2_{2}(t_n))\right).
    \end{equation}
    Moreover, from the hypothesis, $b_{c_2^1}(x_2^o)=b_{c_2^2}(x_2^o)$. So, from the definition of Busemann function,
    \begin{equation}\label{x}
    \lim_{n}\left(d_2(x_2^o,c_2^1(t_n))-d_2(x_2^o,c_2^2(t_n))\right)=0.
    \end{equation}
    From \eqref{eq:11} and (\ref{x})
    \begin{equation}
      \label{eq:12}
\lim_n \left(\Integral{t^o+\epsilon}{t_n}{\mu^n_{2}}{2}{\mu^n_{k}}-d_2(x_2^o,c_2^1(t_n))\right)=0.
    \end{equation}

On the other hand, from the claim, the sequence of positive constants $\{\mu_2^n\}_n$ does not converge to $0$, so there exists ${\cal K}>0$ such that
   \begin{equation}
     \label{eq:13}
     \Integral{t^o}{t^o+\epsilon}{\mu_2^n}{2}{\mu_k^n}>{\cal K}>0\quad\hbox{for $n$ big enough.}
   \end{equation}
        So, putting together \eqref{eq:12} and \eqref{eq:13} we obtain that
\[
\lim_n \left(\Integral{t^o}{t_n}{\mu^n_{2}}{2}{\mu^n_{k}}-d_2(x_2^o,c_2^1(t_n))\right)>0,
  \]
  and thus,
  \[
\Integral{t^o}{t_n}{\mu^n_{2}}{2}{\mu^n_{k}}>d_2(x_2^o,c_2^1(t_n))\quad\hbox{for $n$ big enough.}
    \]
    From Prop. \ref{c0}, necessarily $q\in I^-(\gamma^1)$, a contradiction.


\end{proof}
\noindent As a direct consequence of Lemmas \ref{lemma:aux1} and \ref{lemma:aux2}, we obtain:
\begin{prop}\label{samecondsamepast}
Let $\gamma^i:[\omega,b)\rightarrow V$, $\gamma^i(t)=(t,c_1^i(t),c_2^i(t))$, $i=1,2$, be future-directed timelike curves. If $c_1^i(t)\rightarrow x_1^*\in M_1^C$, $i=1,2$, and $b_{c_2^1}=b_{c_2^2}$, then $I^-(\gamma^1)=I^-(\gamma^2)$.
\end{prop}
\begin{proof}
Let $c_1:[\omega,b)\rightarrow M_1$ be a curve with $c_1(t)\rightarrow x_1^*$ such that the curves $\overline{\gamma}^i:[\omega,\infty)\rightarrow V$, $\overline{\gamma}^i(t)=(t,c_1(t),c_2^i(t))$, $i=1,2$, are future-directed timelike. From Lemma \ref{lemma:aux1}, $I^-(\gamma^i)=I^-(\overline{\gamma}^i)$, $i=1,2$. But $\overline{\gamma}^1$, $\overline{\gamma}^2$ share the same first spatial components, and their second spatial components define the same Busemann function. Hence, Lemma \ref{lemma:aux2} ensures that $I^-(\overline{\gamma}^1)=I^-(\overline{\gamma}^2)$, as required.
\end{proof}

%Putting together Props. \ref{structuraparcialsininfinito}, \ref{prop:conddiferbordedif}
%and \ref{samecondsamepast} we obtain the following bijection:
%%\begin{equation}
%%  \label{eq:14}
%%\hat{V}\;\leftrightarrow\; \left(\R\times M_1^C\times M_2^C\right)\cup \left(\{\infty\}\times M_1^C\times {\cal B}(M_2)\right),
%%\end{equation}
%%which maps each indecomposable past set $P\in \hat{V}$ to the limit point of any curve generating it, which is either $(\Omega,x_1^*,x_2^*)\in \R\times M_1^C\times M_2^C$ if $\Omega<\infty or $(\Omega,x_1^*,b_{c_2})\in \{\infty\}\times M_1^C\times {\cal B}(M_2)$ if $\Omega=\infty$.
%
%%In particular, any future indecomposable set $P=I^-(\gamma)$ with $\gamma:[\alpha,\Omega)\rightarrow V$ and $\gamma(t)=(t,c_1(t),c_2(t))$ is determined by three elements: the limit $\Omega\in \R\cup \{\infty\}$ of the temporal component; the limit $x_1^*\in M_1^C$ of the first spatial component; and, (a) the limit $x_2^*\in M_2^C$ of the second spatial component if $\Omega<\infty$ or (b) the busemann function $b_{c_2}$ if $\Omega=\infty$.
%Even more, we can even give a unified treatment to this last case by recalling that, when $\Omega<\infty$, then the associated Busemann function $b_{c_2}\equiv d_{(\Omega,x_2^*)}$, so it codifies both, the limits of the temporal component and the second spatial component.

\noindent Summarizing, if we put together Props. \ref{structuraparcialsininfinito'}, \ref{prop:conddiferbordedif}
and \ref{samecondsamepast}, we deduce the following point set structure for the future c-completion of $(V,g)$:
\begin{thm}\label{futurecomploneinfinite}
  Let $(V,g)$ be a  {\multiwarped} spacetime as in \eqref{eq:1-aux}, and assume that the integral conditions \eqref{eq:9} hold. Then, there exists a bijection
 \begin{equation}
   \label{eq:10}
     \hat{V}\; \leftrightarrow \;  M_1^C\times \left(B(M_2)\cup \{\infty\}\right)\;\equiv\;
       \left( (a,b)\times M_1^C\times M_2^C\right) \cup M_{1}^{C} \times \left({\cal B}(M_2)\cup \{\infty\}\right).
     %,\qquad \hat{V}\setmin us \hat{\partial}^\infty V\equiv \cambios{(a,b)}\times M_1^C\times M_2^C  \\
%
%     \\
%\cambios{
%     \begin{array}{rl}
%       \hat{\partial}^{\infty}V\equiv  & M_1^C\times \left({\cal B}(M_2)\cup \{\infty\}\right)\\ \equiv & M_1^C\times \left(\left(\R\times \partial_{\cal B}M_2\right) \cup \{\infty\}   \right)
%     \end{array}
%}
 % \end{array}
  \end{equation}
 This bijection maps each indecomposable past set $P=I^-(\gamma)\in \hat{V}$, where $\gamma:[\omega,\Omega)\rightarrow V$, $\gamma(t)=(t,c_1(t),c_2(t))$, is any curve generating $P$, to a pair $(x_1^*,b_{c_2})$, where $x_1^*\in M_1^C$ is the limit point of the curve $c_1$. If $\Omega<b$, then $b_{c_2}=d_{(\Omega,x_2^*)}$, where $x_2^*$ is the limit point of $c_2$ (see \eqref{eq:46}), and thus, $P$ can be also identified with the limit point $(\Omega,x_1^*,x_2^*)$ of $\gamma$ (recall Prop. \ref{structuraparcialsininfinito'}).

  % to the limit point of any curve generating it, which is either $(\Omega,x_1^*,x_2^*)\in \R\times M_1^C\times M_2^C$ if $\Omega<\infty$ or $(\Omega,x_1^*,b_{c_2})\in \{\infty\}\times M_1^C\times {\cal B}(M_2)$ if $\Omega=\infty$.
\end{thm}


%This last property becomes specially interesting in order to analyze the topology, since it makes unnecessary to distinguish between indecomposable sets associated to finite or infinite $\Omega$.
%Concretely, we will not conceive the future completion as in \eqref{eq:14}, but in this alternative way:
%\begin{equation}
%  \label{eq:15}
%  \hat{V}\equiv M_1^C\times B(M_2),
%\end{equation}
%that is, any future indecomposable set $P$ will be identified to a pair $(x_1^*,b_{c_2^2})\in M_1^C\times B(M_2)$, and so $P\equiv (x_1^*,b_{c_2^2})$.


\subsubsection{Topological Structure}

 Next, we are going to extend previous study to a topological level, showing that the bijection obtained above is actually a homeomorphism when the corresponding product topology on $M_1^C\times (B(M_2)\cup \{\infty\})$ is considered.

 To this aim, we only need to prove the following equivalence:
%In this last subsection we show that the identification \eqref{eq:15} can be extended to the topological level, assumed that $M_1^C\times B(M_2)$ is endowed with the product topology. To this aim, we are going to prove the following equivalence:
given $P\equiv (x_1^*,b_{c_2})\in \hat{V}$ and $\{P_n\}_n\equiv \{(x_1^n,b_{c_2^n})\}_n\subset \hat{V}$,
\begin{equation}\label{equ}
P\in \hat{L}(\{P_n\}_n)\iff  x_1^n\rightarrow x^*_1\;\;\hbox{and}\;\; b_{c_2}\in \hat{L}(\{b_{c_2^n}\}_n).
  \end{equation}
  Under the hypothesis of $M_1^C$ and $M_2^C$ being locally compact, the equivalence (\ref{equ}) for the case $b_{c_2}\equiv d_{(\Omega,x_2)}$ is already proved in Prop. \ref{topcurvasfinitas}. In fact, if $P_n=I^-(\gamma^n)$ with $\gamma^n:[\omega,\Omega_n)\rightarrow V$, then $\Omega_n<b$ for $n$ big enough. In particular, $b_{c_2^n}\equiv d_{(\Omega_n,x^n_2)}$ with $x_2^n\in M_2^C$ (see \eqref{eq:46}). Moreover, Prop. \ref{topcurvasfinitas} implies that $(\Omega_n,x_1^n,x_2^n)\rightarrow (\Omega,x^*_1,x^*_2)$. Hence, $\{d_{(\Omega_n,x_2^n)}\}_n$ converges pointwise to $d_{(\Omega,x_2)}$, and thus, $d_{(\Omega,x_2)}\in \hat{L}(\{d_{(\Omega_n,x_2^n)}\}_n)$ (see Prop. \ref{prel:PropToponefibre}). So, to finish the proof of (\ref{equ}), we can focus just on the case $b_{c_2}\in {\cal B}(M_2)$.

  We begin with some preliminary results.
  \begin{lemma}\label{lemma:aux5}
    Let $P,P'\in \hat{V}$ and $\{P_n\}_n\subset \hat{V}$, and assume that $P\equiv (x_1,b_{c_2}), P'\equiv (x_1',b_{c'_2})$ and $P_n\equiv (x_1^n,b_{c_2^n})$ belong to $M_1^C\times \left(B(M_2)\cup \{\infty\}\right)$ for all $n$ (recall the identification in \eqref{eq:10}). Then, the following statements hold:
    \begin{itemize}
    \item[(i)] If $x_1=x_1'$, then
        \[
b_{c_2}\leq b_{c_2'} \iff P\subset P'.
        \]
    \item[(ii)] If $x_1^n\rightarrow x_1$, then
      \[
P\subset {\mathrm LI}(\{P_n\}_n) \iff b_{c_2}\leq {\mathrm lim\, inf}_n (\{b_{c_2^n}\}_n).
        \]
    \end{itemize}
  \end{lemma}
  \begin{proof} Let $\gamma:[\omega,\Omega)\rightarrow V$, $\gamma':[\omega',\Omega')\rightarrow V$ and $\gamma^n:[\omega^n,\Omega^n)\rightarrow V$ be future-directed timelike curves generating $P,P'$ and $P_n$, resp.

    (i) First, let us prove the implication to the left. Assume that $\gamma(t)=(t,c_1(t),c_2(t))$ and $\gamma'(t)=(t,c_1'(t),c_2'(t))$ satisfy that $c_{1}(t) \rightarrow x_{1}$, $c_{1}'(t) \rightarrow x_{1}$ and $b_{c_{2}}$, $b_{c_{2}'}$ are their Busemann functions. Consider the future-directed timelike curves $\sigma(t)=(t,c_2(t))$ and $\sigma'(t)=(t,c_2'(t))$ in the Generalized Robertson-Walker spacetime $$\left( (a,b)\times M_2,-dt^2+\alpha_2g_2\right).$$ Since $P\subset P'$, necessarily $P(b_{c_2})=I^-(\sigma)\subset I^-(\sigma')=P(b_{c_2'})$, and thus, $b_{c_2}\leq b_{c_2'}$ (recall \eqref{eq:27} and \eqref{eq:28}).

For the implication to the right, assume that $x_1=x_1'$ and $b_{c_2}\leq b_{c'_2}$. It suffices to show the existence of a sequence $\sigma=(t_n,y_1^n,c_2(t_n))$ with $\{t_n\}_n\nearrow \Omega$, satisfying $\{y_1^n\}_n\rightarrow x_1$ and $\sigma\subset P'$. In fact, in this case, Lemma \ref{lemma:aux3} ensures that $P\subset {\rm LI}(\sigma)$ and, taking into account that $\sigma\subset P'$, necessarily $P\subset P'$.

To this aim, take $\{t_n\}_n\nearrow \Omega$ and observe that, by hypothesis, $b_{c_2}\leq b_{c'_2}$. So, in the Generalized Robertson-Walker spacetime $\left( (a,b)\times M_2,-dt^2+{\alpha_2}g_2\right)$, the inclusion $P(b_{c_2})\subset P(b_{c'_2})$ holds (recall equations \eqref{eq:27} and \eqref{eq:28}). In particular, since the future-directed timelike curves $\sigma(t)=(t,c_2(t))$ and $\sigma'(t)=(t,c'_2(t))$ satisfy $I^-(\sigma)=P(b_{c_2})$ and $ I^-(\sigma')=P(b_{c'_2})$, there exists a sequence $\{s_n\}_n$, with $\{s_n\}_n\nearrow \Omega'$, such that
$\sigma(t_n)=(t_n,c_2(t_n))\ll (s_n,c'_2(s_n))=\sigma'(s_n)$. Let us show that $\sigma=\{(t_n,c_1'(s_n),c_2(t_n))\}$ is the required sequence. From construction and the fact that $(t_n,c_1'(s_n),c_2(t_n))\ll (s_n,c_1'(s_n),c'_2(s_n))$ in $V$ for all $n$, necessarily $\sigma\subset P'$. Moreover, since $\{s_n\}_n\nearrow \Omega'$, necessarily $c_1'(s_n)\rightarrow x_1'=x_1$, as desired.

\smallskip

(ii) For the implication to the right, assume that $P\subset {\mathrm LI}(\{P_n\}_n)$ and let us show that $b_{c_2}\leq \liminf (\{b_{c_2^n}\}_n)$. Denote by $\sigma(t)=(t,c_2(t))$ and $\sigma_n(t)=(t,c_2^n(t))$ future-directed timelike curves in the Generalized Robertson Walker model \[\left((a,b)\times M_2,-dt^2+ \alpha_2g_2\right).\] Since $P\subset {\mathrm LI}(\{P_n\}_n)$, necessarily
  \[
P(b_{c_2})=I^-(\sigma)\subset {\mathrm LI}(\{I^-(\sigma_n)\}_n)={\mathrm LI}(\{P(b_{c_2^n})\}_n)
    \]
(where we are considering past sets in the associated Generalized Robertson Walker model), and the conclusion follows from \eqref{eq:50}.

% . Our aim is to show that $b_{c_2}\leq \liminf (\{b_{c_2^n}\}_n))$, or equivalently, that if we have $x_2^o\in M_2$ and $r\in\R$ such that

% \begin{equation}
%   \label{eq:23}
% r< b_{c_2}(x_2^o)=\lim_{t\rightarrow \Omega} \left(\int_{\C}^{t}\frac{1}{\sqrt{\alpha_2(s)}}ds - d(x_2^o,c_2(t))\right),
% \end{equation}
% then
%   \[
% r< b_{c^n_2}(x_2^0)\quad\hbox{for $n$ big enough.}
%     \]
% Take $t^o$ and $x_2^o$ satisfying \eqref{eq:23}. For $t>t^o$ big enough it follows that:
%   \[
% \int_{\C}^{t^o}\frac{1}{\sqrt{\alpha_2(s)}}ds<\int_{\C}^{t}\frac{1}{\sqrt{\alpha_2(s)}}ds - d(x_2^0,c_2(t))\Rightarrow \int_{t^o}^{t}\frac{1}{\sqrt{\alpha(s)}}ds>d_2(x_2^o,c_2(t)).
% \]
%  Since this last inequality is strict, we can take positive constants $\mu_1,\mu_2>0$, with $\mu_1+\mu_2=1$, and $x_1^o\in M_1$ close enough to $c_1(t)$, such that
% \[
% \Integral{t^o}{t}{\mu_i}{i}{\mu_k}>d_i(x_i^o,c_i(t)) \qquad \hbox{for $i=1,2$.}
%   \]
%   From Prop. \ref{c0}, $q=(t^o,x_1^o,x_2^o)\in I^-(\gamma)=P$. Since $P\subset {\mathrm LI}(\{P_n\})$, there exists $n_0$ such that $q\in P_n$ for $n\geq n_0$. Again from Prop. \ref{c0}, there exist positive constants $\mu_1^n,\mu_2^n>0$, with $\mu_1^n+\mu_2^n=1$, and $t_n\in [\omega,\Omega_n)$, such that
%   \[
% \Integral{t^o}{t_n}{\mu^n_i}{i}{\mu^n_k}>d_i(x_i^o,c_i^n(t_n)) \qquad \hbox{for $i=1,2$}.
%     \]
% In particular, for $i=2$,
%     \[
%       \begin{array}{c}
% \displaystyle\int_{t^o}^{t_n}\frac{1}{\sqrt{\alpha_2(s)}}ds > \Integral{t^o}{t^n}{\mu^n_2}{2}{\mu^n_k}>  d_2(x_2^o,c_2^n(t_n))\Rightarrow\\  \\ \Rightarrow \displaystyle \int_{\C}^{t^o}\frac{1}{\sqrt{\alpha_2(s)}}ds< \int_{\C}^{t_n}\frac{1}{\sqrt{\alpha_2(s)}}ds-d_2(x_2^o,c_2^n(t_n)).
%       \end{array}
%       \]
%       Now, from the increasing character of this last term, as described in \eqref{eq:25}, it follows that
%       \[
% \int_{\C}^{t^o}\frac{1}{\sqrt{\alpha_2(s)}}<b_{c_2^n}(x_2^o)\quad\hbox{for $n\geq n_0$}.
%         \]
%         In conclusion, $b_{c_2}\leq {\mathrm lim\, inf}_n( \{b_{c_2^n}\}_n)$, as desired.

        \smallskip

        For the implication to the left, assume that $b_{c_2}\leq {\mathrm lim\, inf}_n (\{b_{c_2^n}\}_n)$ and let us prove that $P\subset {\mathrm LI}(\{P_n\}_n)$. Let $\{t_k\}\nearrow \Omega$ be an arbitrary sequence. For each $k$, and from the timelike character of $\gamma$, we have $(t_k,c_2(t_k))\ll (t,c_2(t))$ in the Generalized Robertson-Walker spacetime $\left( (a,b)\times M_2,-dt^2+{\alpha_2}g_2\right)$ for all $t>t_k$. From \eqref{eq:26} and the increasing character of \eqref{eq:25},

      \[
\int_\C^{t_k}\frac{1}{\sqrt{\alpha_2(s)}}ds < b_{c_2}(c_2(t_k))=\lim_{t\rightarrow \Omega} \left(\int_\C^t \frac{1}{\sqrt{\alpha_2(s)}}ds - d_2(c_2(t_k),c_2(t))\right).
        \]
Since $b_{c_2}\leq {\mathrm lim\,inf}(\{b_{c_2^n}\}_n)$, there exists an increasing sequence $\{n_k\}_k$ such that
\begin{equation}
  \label{eq:16}
\int_\C^{t_k}\frac{1}{\sqrt{\alpha_2(s)}}ds<b_{c^n_2}(c_2(t_k))=\lim_{r\rightarrow \Omega_n} \left(\int_\C^{r}\frac{1}{\sqrt{\alpha_2(s)}}ds- d_2(c_2(t_k),c_2^n(r))\right)\quad\hbox{$\forall$ $n\geq n_k$.}
\end{equation}
  For each $n_k\leq n<n_{k+1}$, consider $r_n\in [\omega_{n},\Omega_n)$ such that
  \begin{equation}
    \label{eq:24}
\int_\C^{t_k}\frac{1}{\sqrt{\alpha_2(s)}}ds< \int_\C^{r_n}\frac{1}{\sqrt{\alpha_2(s)}}ds-d_2(c_2(t_k),c_2^n(r_n)),\qquad d_{1}(c_1^n(r_n),x_1^n)<\frac{1}{2^n},
  \end{equation}
    (for the first inequality recall \eqref{eq:16}; for the second one, recall that $c_1^n(t)\rightarrow x_1^n$). From the first inequality, it follows that
    \[
    (t_k,c_2(t_k))\ll (r_n,c_2^n(r_n))\quad\hbox{for $n_k\leq n< n_{k+1}$ and all $k$.}
    \]
    However, since $\{(t_k,c_2(t_k))\}$ is a chronological chain, previous chronological relation is true for all $n\geq n_k$: in fact, if $n\geq n_k$, there exists $k'(\geq k)$ such that $n_{k'}\leq n < n_{k'+1}$. As we have noted before $(t_{k'},c_2(t_{k'}))\ll (r_n,c_2^n(r_n)))$ but, taking into account $(t_k,c_2(t_k))\ll (t_{k'},c_2(t_{k'}))$, necessarily $(t_{k},c_2(t_{k}))\ll (r_n,c_2^n(r_n)))$.

    Next, define the sequence $\sigma=\{(l_n,c^n_1(r_n),c_2(l_n))\}_n$, where $l_n:=t_k$ if $n_{k}\leq n< n_{k+1}$. Since $\{t_k\}_k\nearrow \Omega$, necessarily $\{l_n\}_n\rightarrow \Omega$. Moreover, since $(t_k,c_2(t_k))\ll (r_n,c_2^n(r_n))$,
    \[(l_n,c^n_1(r_n),c_2(l_n))=(t_k,c_1^n(r_n),c_2(t_k))\ll (r_n,c_1^n(r_n),c^n_2(r_n))=\gamma^n(r_n),\] hence $(l_n,c^n_1(r_n),c_2(l_n))\in P_n$ for all $n$. Finally, note that $\sigma$ satisfies the conditions of Lemma \ref{lemma:aux3}, as $\{l_n\}_n\rightarrow \Omega$ and $c_1^n(r_n)\rightarrow x_1$ (recall that $c_1^n(t)\rightarrow x_1^n$, $x_1^n\rightarrow x_1$ from hypothesis and the second inequality in \eqref{eq:24}). Therefore,
\[
    P\subset {\mathrm LI}(\{I^-(l_n,c^n_1(r_n),c_2(l_n))\}_n)\subset {\mathrm LI}(\{P_n\}),
    \]
    as desired.
  \end{proof}

  %After previous technical result we are able to prove the equivalence between topologies. Let us begin with:


  \begin{prop}\label{prop:topcharac} Let $P\in \hat{V}$ and $\{P_n\}_n\subset \hat{V}$, and assume that $P\equiv (x_1,b_{c_2})$ and $P_n\equiv (x_1^n,b_{c_2^n})$ (in $M_1^C\times \left(B(M_2)\cup \{\infty\} \right)$) for all $n$. Then,
$P\in \hat{L}(\{P_n\}_n)$ if, and only if, $x_1^n\rightarrow x_1$ and $b_{c_2}\in \hat{L}(\{b_{c_2^n}\}_n)$.
  \end{prop}
  \begin{proof}
    For the implication to the right, and reasoning as in the proof of Thm. \ref{futurestructurefiniteconditions}, it follows that $x_1^n\rightarrow x_1$ (recall the finite warping integral in \eqref{eq:9} and Remark \ref{rem:1}). Hence, we will focus on $b_{c_2}\in \hat{L}(\{b_{c_2^n}\})$. From Lemma \ref{lemma:aux5} and the fact that $P\in \hat{L}(\{P_n\}_n)$, necessarily $b_{c_2}\leq \liminf(\{b_ {c_2^n}\}_n)$. So, $b_{c_2}\in \hat{L}(\{b_{c_2^n}\})$ follows if we prove that $b_{c_2}$ is maximal into $\limsup(\{b_{c^n_2}\}_n)$. Consider any $b_{\overline{c}_2}$ such that $b_{c_2}\leq b_{\overline{c}_2}\leq {\mathrm lim\,sup}(\{b_{c^n_2}\}_n)$, and consider the associated past set $\overline{P}\equiv (x_1,b_{\overline{c}_2})$. Up to a subsequence, we can assume that $b_{\overline{c}_2}\leq {\mathrm lim\,inf}(\{b_{c^n_2}\}_n)$. From Lemma \ref{lemma:aux5}, $P\subset \overline{P}$ and $\overline{P}\subset {\mathrm LI}(\{P_{n}\}_n)$. But $P$ is maximal into the superior limit of the sequence $\{P_n\}_n$, so necessarily $P=\overline{P}$. From Prop. \ref{prop:conddiferbordedif} we have that $b_{c_2}=b_{\overline{c}_2}$ so the maximal character of $b_{c_2}$ into $\limsup(\{b_{c_2^n}\}_n)$ is obtained.

    \smallskip

    For the implication to the left, first note that $P\subset {\mathrm LI}(\{P_n\}_n)$ (recall Lemma \ref{lemma:aux5} and the definition of $\hat{L}$ for Busemann functions \eqref{eq:22}). So, we only need to focus on the maximal character of $P$ into ${\mathrm LS}(\{P_n\})$. Take $\overline{P}$ an indecomposable past set with $P\subset \overline{P}$ and maximal into ${\mathrm LS}(\{P_n\})$, and let us prove that $P=\overline{P}$. Assume that $\overline{P}\equiv (\overline{x}_1,b_{\overline{c}_2})$. Up to a subsequence, we can also assume that $\overline{P}\subset {\mathrm LI}(\{P_n\})$, hence $\overline{P}\in \hat{L}(\{P_n\})$. Hence, from previous part, $x_1^n\rightarrow \overline{x}_1$. But, by hypothesis, $x_1^n\rightarrow x_1$, obtaining that $x_1=\overline{x}_1$. Once this is observed, Lemma \ref{lemma:aux5} ensures both, $b_{c_2}\leq b_{\overline{c}_2}$ and $b_{\overline{c}_2}\leq \limsup(\{b_{c_2^n}\})$. Since $b_{c_2}\in \hat{L}(\{b_{c_2^n}\})$, necessarily $b_{c_2}=b_{\overline{c}_2}$, and so, $P=\overline{P}$ (recall Prop. \ref{samecondsamepast}).
  \end{proof}

\noindent Summarizing, we are in conditions to deduce the following result:

\begin{thm}\label{futurecomploneinfinite}
  Let $(V,g)$ be a  {\multiwarped} spacetime as in \eqref{eq:1-aux}, and assume that the integral conditions in \eqref{eq:9} are satisfied. If $M_1^C$ and $M_2^C$ are locally compact, the bijection (\ref{eq:10}) becomes a homeomorphism.
\end{thm}
\begin{proof}
  From Prop. \ref{topcurvasfinitas}, the bijection between $\hat{V}\setminus \hat{\partial}^{\B} V$ and $(a,b) \times M_1^C\times M_2^C$ is a homeomorphism if we assume that $M_1^C$ and $M_2^C$ are locally compact. From Prop. \ref{prop:topcharac}, the homeomorphism can be extended to the bijection (\ref{eq:10}).
\end{proof}

\section{The past c-completion of doubly warped spacetimes}\label{ss6}

Obviously, similar arguments provide the corresponding results for the past c-completion:

\begin{thm}\label{pfuturestructurefiniteconditions}
  Let $(V,g)$ be a {\multiwarped} spacetime as in (\ref{eq:1-aux}), and assume that the integral conditions
  \begin{equation}
  \label{eqq:7}
  \int_{a}^{\C}\frac{1}{\sqrt{\alpha_i(s)}}ds<\infty, \qquad \hbox{$i=1,2$}\quad\hbox{for some $\C\in (a,b)$.}
\end{equation}
 hold. Then, there exists a bijection
  \begin{equation}
    \label{eqq:8}
    \check{V}\; \leftrightarrow \; [a,b) \times M_1^C\times M_2^C
  \end{equation}
  which maps each IF $F\in \check{V}$ to the limit point $(\Omega,x_1,x_2)\in [a,b)\times M_1^C\times M_2^C$ of any past-directed timelike curve generating $F$. Moreover, if $M_1^C$ and $M_2^C$ are locally compact, then this bijection becomes a homeomorphism.
\end{thm}


\begin{thm}\label{pfuturecomploneinfinite}
  Let $(V,g)$ be a  {\multiwarped} spacetime as in \eqref{eq:1-aux}, and assume that the integral conditions
  \begin{equation}
  \label{eqq:9}
 \int_{a}^{\C}\frac{1}{\sqrt{\alpha_1(s)}}ds<\infty \qquad \hbox{and}\qquad \int_{a}^{\C}\frac{1}{\sqrt{\alpha_2(s)}}ds=\infty,
\end{equation}
hold. Then, there exists a bijection
 \begin{equation}
   \label{eqq:10}
     \check{V}\; \leftrightarrow\;  M_1^C\times \left(B(M_2)\cup \{-\infty\}\right)
     \equiv  \left( (a,b)\times M_1^C\times M_2^C\right) \cup M_{1}^{C} \times \left({\cal B}(M_2)\cup \{\infty\}\right).
  \end{equation}
  This bijection maps each indecomposable future set $F=I^+(\gamma)\in \check{V}$, where $\gamma:[\omega,-\Omega)\rightarrow V$, $\gamma(t)=(-t,c_1(t),c_2(t))$, is any curve generating $F$, to a pair $(x_1^*,b^-_{c_2})$, where $x_1^*\in M_1^C$ is the limit point of the curve $c_1$. If $-\Omega>a$, then  $b^-_{c_2}=d^-_{(\Omega,x_2^*)}$, where $x_2^*$ is the limit point of $c_2$ (see \eqref{eq:48}), and thus, $F$ can be also identified with the limit point $(\Omega,x_1^*,x_2^*)$ of $\gamma$.
  % to the limit point of any past-directed timelike curve generating it, which is either $(\Omega,x_1^*,x_2^*)\in \R\times M_1^C\times M_2^C$ if $\Omega>-\infty$ or $(\Omega,x_1^*,b_{c_2})\in \{-\infty\}\times M_1^C\times {\cal B}(M_2)$ if $\Omega=-\infty$. Moreover, if $M_1^C$ and $M_2^C$ are locally compact, the bijection (\ref{eq:10}) becomes a homeomorphism.
\end{thm}

%.....................................................
%
%
%Obviously, all previous results have a past analog, which we briefly review as follows:
%
%\begin{thm}\label{pastcompletion}
%  Let $(V,g)$ be a {\multiwarped} model as in \eqref{eq:1}. Then,
%
%  \begin{equation}
%    \label{eq:39}
%    \check{V}\setminus \check{\partial}^{\infty} V\equiv \R\times M_1^C\times M_2^C,
%  \end{equation}
%  where $\check{\partial}^{\infty}V$ denotes the TIFs determined by past-directed timelike curves with divergent temporal component. Moreover,
%  \begin{itemize}
%  \item If
%    \begin{equation}
%      \label{eq:40}
%      \int^{0}_{-\infty}\frac{1}{\sqrt{\alpha_i(s)}}ds<\infty, \qquad \hbox{for $i=1,2$.}
%    \end{equation}
%    then the past causal boundary and completion has the following structure
%    \begin{equation}
%      \label{eq:41}
%      \check{V}\equiv [a,b)\times M_1^C\times M_2^C.
%    \end{equation}
%That is, any indecomposable future set $F\in \check{V}$ can be labelled by a triple $(\Omega,x_1,x_2)\in [a,b)\times M_1^C\times M_2^C$. Moreover, if $M_1^C,M_2^C$ are locally compact, then previous identifications can be extended to the topological level.
%  \item If
%    \begin{equation}
%      \label{eq:42}
%       \int^{0}_{-\infty}\frac{1}{\sqrt{\alpha_1(s)}}ds<\infty \qquad \hbox{and}\qquad \int^{0}_{-\infty}\frac{1}{\sqrt{\alpha_2(s)}}ds=\infty.
%    \end{equation}
%    then we obtain
%    \begin{equation}
%      \label{eq:43}
%      \begin{array}{c}
%     \check{V}\equiv M_1^C\times \left(B(M_2)\cup \{-\infty\}\right),\qquad \check{V}\setminus \check{\partial}^\infty V\equiv \cambios{(a,b)}\times M_1^C\times M_2^C\\
%
%     \\
%\cambios{
%     \begin{array}{rl}
%       \check{\partial}^{\infty}V\equiv  & M_1^C\times \left({\cal B}(M_2)\cup \{-\infty\}\right)\\ \equiv & M_1^C\times \left(\left(\R\times \partial_{\cal B}M_2\right) \cup \{-\infty\}   \right)
%     \end{array}
%}
%\end{array}
%    \end{equation}
%    and if $M_1^C,M_2^C$ are locally compact, previous identifications extend to the topological level.
%  \end{itemize}
%\end{thm}




%%% Local Variables:
%%% mode: latex
%%% TeX-master: "DoublyWarpedBoundary2017"
%%% End:


%\input{PartialBoundaryTop}

\section{The total c-completion of doubly warped spacetimes}
\label{sec:totalcompletion}
%\subsection{Total c-boundary as a point set}

We are now in conditions to construct the (total) c-completion of doubly warped spacetimes by merging appropriately the future and past c-boundaries obtained in previous section.

To this aim, first we need to determine the S-relation between indecomposable sets. So, let $\gamma:[\omega,\Omega)\rightarrow V$, $\gamma(t)=(t,c_{1}(t),c_{2}(t))$, be an inextensible future-directed timelike curve.
%Finally, we are going to study the point set, causal and topological structure for the total c-completion of {\multiwarped} spacetimes from the corresponding structures for the future (and past) partial boundaries obtained in previous sections. Of course, these structures will depend on integral conditions in both directions (i.e., from $0$ to $\pm\infty$) involving the warping functions $\alpha_i$, $i=1,2$.
%
%\begin{align}
%  \int^{0}_{-\infty}\frac{1}{\sqrt{\alpha_i(s)}}ds<\infty, \qquad \hbox{for $i=1,2$.}\hspace{1.5cm}\label{eq:20}\\
%   \int_{-\infty}^{0}\frac{1}{\sqrt{\alpha_1(s)}}ds<\infty \qquad \hbox{and}\qquad \int^{0}_{-\infty}\frac{1}{\sqrt{\alpha_2(s)}}ds=\infty. \label{eq:21}
%\end{align}
%
%\smallskip
Clearly, if $\Omega=b$ then $\uparrow I^{-}(\gamma)=\emptyset$, and there are no IFs S-related to $I^{-}(\gamma)$. So, we will focus on the case $\Omega<b$.

\begin{prop}
\label{tip}
Let $(V,g)$ be a \multiwarped spacetime and consider a future-directed (resp. past-directed) timelike curve $\gamma$ with associated endpoint $(\Omega^+,x_1^*,x_2^*) \in (a,b) \times M_{1}^{C} \times M_{2}^{C}$  (resp. $(\Omega^-,y_1^*,y_2^*) \in (a,b) \times M_{1}^{C} \times M_{2}^{C}$). Then
\begin{equation}
\begin{aligned}
        \uparrow I^{-}(\gamma) &=\{(t,x_1,x_2) \in V\; \mid\; \exists \,  \mu_{1},\mu_{2} > 0\;\; \hbox{{\rm such that}} \\
        & \Integral{\Omega^+}{t}{\mu_{i}}{i}{\mu_{k}}> d_{i}(x_{i},x_{i}^*),\; i=1,2.\}%\quad\hbox{pointwise and topologically.}
\end{aligned}
\end{equation}

\begin{equation*}
\begin{aligned}
(\hbox{resp.}\;\;\downarrow I^{+}(\gamma) &=\{(t,x_1,x_2) \in V\; \mid\; \exists \,  \mu_{1},\mu_{2} > 0\;\; \hbox{{\rm such that}} \\
                & \Integral{t}{\Omega^-}{\mu_{i}}{i}{\mu_{k}}> d_{i}(x_{i},y^{*}_{i}),\; i=1,2\}).%\quad\hbox{pointwise and topologically.}
\end{aligned}
\end{equation*}
As consequence, if $P\in\hat{V}$ and $F\in\check{V}$ are associated to $(\Omega^+,x^*_1,x^*_2)$ and $(\Omega^-,y^*_1,y^*_2)$ in $(a,b) \times M_{1}^{C} \times M_{2}^{C}$, resp, then the following equivalence holds:
\[
P \sim_{S} F\quad \Longleftrightarrow\quad
\Omega^{-}=\Omega^{+}\;\;\hbox{and}\;\; x^*_{i}=y^*_{i} \in M_{i}^{C},\; i=1,2.
\]
\end{prop}
\begin{proof} Assume that $\gamma:[\omega,\Omega^+) \rightarrow V$, $\gamma(t)=(t,c_1(t),c_2(t))$, is a future-directed causal curve with associated endpoint $(\Omega^+,x_1^*,x_2^*) \in (a,b) \times M_{1}^{C} \times M_{2}^{C}$ (for the past is analogous). We need to show that $\uparrow I^-(\gamma)=A_{(\Omega,x_1^*,x_2^*)}$, where
\begin{equation*}
\begin{aligned}
A_{(\Omega^+,x_1^{*},x_2^*)}&:=\{(r,x_1,x_2) \in V \mid \exists \,  \mu_{1},\mu_{2} > 0\;\; \hbox{{\rm such that}} \\
        &
        \int_{\Omega^+}^{r}\frac{\sqrt{\mu_{i}}}{\alpha_i(s)}\left(\sum_{k=1}^2\frac{\mu_k}{\alpha_k(s)}\right)^{-1/2}dt > d_{i}(x_{i},x_{i}^*),\; i=1,2\}.%\quad\hbox{pointwise and topologically.}
\end{aligned}
\end{equation*}
For the inclusion to the right, take $(r,x_1,x_2) \in \uparrow I^{-}(\gamma)$ and $\epsilon>0$ small enough so that $(r-\epsilon,x_1,x_2)\in \uparrow I^-(\gamma)$ (recall that the common future is open). For any sequence $\{t_{n}\}_{n}\nearrow \Omega^+$ we have $\gamma(t_n) \ll (r-\epsilon,x_1,x_2)$ for all $n$. From Prop. \ref{c0} there exist constants $\mu^n_{1},\mu^n_{2}>0$, with $\mu^n_{1}+\mu_{2}^n=1$ for all $n$, such that
\[
\int_{t_n}^{r-\epsilon}\frac{\sqrt{\mu^n_{i}}}{\alpha_i(s)}\left(\sum_{k=1}^2\frac{\mu_k^n}{\alpha_k(s)}\right)^{-1/2}dt > d_{i}(x_i,c_{i}(t_n))\quad i=1,2.
\]
Then, by the standard limit process, we deduce the following inequalities:
\[
\int_{\Omega^+}^{r-\epsilon}\frac{\sqrt{\mu^*_{i}}}{\alpha_i(s)}\left(\sum_{k=1}^2\frac{\mu_k^*}{\alpha_k(s)}\right)^{-1/2}dt \geq d_{i}(x_{i},x_{i}^{*}),\quad i=1,2,
\] where $\mu_i^*$ is the limit (up to a subsequence) of $\{\mu_i^n\}$. Now observe that some of previous inequalities become strict if we replace $r-\epsilon$ by $r$. So, a small variation of $\mu_1^*$ and $\mu_2^*$ if necessary (concretely, if one of these constants is zero), provides
% If, either $\mu^*_i=0$ for some $i=1,2$ or some inequality is actually an equality, a small variation of $\mu^*_{1},\mu^*_{2}$ ensures the existence of
positive constants $\mu_{1}',\mu_{2}'>0$ satisfying
\[
\int_{\Omega^+}^{r}\frac{\sqrt{\mu'_{i}}}{\alpha_i(s)}\left(\sum_{k=1}^2\frac{\mu'_k}{\alpha_k(s)}\right)^{-1/2}dt > d_{i}(x_{i},x_{i}^{*}),\quad i=1,2.
\]
In particular, $(r,x_1,x_2)\in A_{(\Omega,x_1^{*},x_2^*)}$, and so, $\uparrow I^{-}(\gamma) \subset A_{(\Omega,x_1^{*},x_2^*)}$.

\smallskip

 For the inclusion to the left, assume that $(r,x_1,x_2) \in A_{(\Omega,x_1^*,x_2^*)}$. By the continuity of both, the integral with respect to the lower limit of integration and the distance function, and the convergence of $\gamma(t)=(t,c_1(t),c_2(t))$ to
$(\Omega,x_1^{*},x_2^*)$, we deduce that
\[
\int_{t}^{r}\frac{\sqrt{\mu_{i}}}{\alpha_i(s)}\left(\sum_{k=1}^2\frac{\mu_k}{\alpha_k(s)}\right)^{-1/2} > d_{i}(x_{i},c_{i}(t))\quad\hbox{for large $t$.}
\]
So, from Prop. \ref{c0}, $\gamma(t) \ll  (r,x_1,x_2)$ for  all $t$, which implies $(r,x_1,x_2) \in \uparrow I^{-}(\gamma)$.

\smallskip

For the last assertion, assume that $P$ is associated to $(\Omega^+,x^*_1,x^*_2)\in (a,b)\times M_1^C\times M_2^C$. From the first part of this proposition, $\uparrow P= I^+(\sigma)$, where $\sigma$ is a past-directed timelike curve converging to $(\Omega^+,x^*_1,x^*_2)$. So, $F=I^+(\sigma)$ is the unique maximal IF into the common future of $P$. Reasoning analogously we deduce that $P$ is the unique maximal IP into the common past of $F$. In conclusion, $P$ is $S$-related just with the indecomposable future set $F$, and vice versa.
\end{proof}

%Now, we are ready to give the following proper characterization of the S-relation between terminal sets.




% \noindent {\it Proof:} From Prop. \ref{tip}, if $P$ and $F$ are associated to the same point $(\Omega,x_{1}^{*},x_{2}^{*}) \in \mathbb{R} \times M_{1}^{c} \times M_{2}^{c}$ then $P=\downarrow F$ and $F=\uparrow P$, and thus, $P \sim_{s} F$.

% Next, let us show that $P \sim_{S} F$ implies $(\Omega^{+},x^{P})=(\Omega^{-},x^{F})$. Since $\uparrow P=I^{+}(\gamma_{P})$ for some inextensible past directed causal curve $\gamma_{P}$ with past endpoint $(\Omega^{+},x^{P})$ and $\downarrow F=I^{-}(\gamma_{F})$ for some inextensible future directed causal curve $\gamma_{F}$ with future endpoint $(\Omega^{-},x^{F})$, necessarily $\uparrow P $ and $\downarrow F$ are indecomposable past and future sets, respectively. On the other hand, since $P \subset \downarrow F=I^{-}(\gamma_{F})$ and $F \subset \uparrow P=I^{+}(\gamma_{P})$, the S-relation
% $P \sim_{S} F$ implies that
% $P=I^{-}(\gamma_{F})$ and $F=I^{+}(\gamma_{P})$; in fact, otherwise maximality of $P$ and $F$ as an IP and IF is not satisfied. Finally, since $P=I^{-}(\gamma_{F})$ and $F=I^{+}(\gamma_{P})$ are univocally determined by $(\Omega^{+},x^{P})$ and $(\Omega^{-},x^{F})$ (see Prop. \ref{pp1}), then $(\Omega^{+},x_{P})=(\Omega^{-},x_{F})$. $\Box$
From this result it is clear that $\overline{V}$ is simple as a point set (see Defn. \ref{simpletop}). On the other hand, if we define
\[
\partial^{\B}V:=\hat{\partial}^{\B}V\cup\check{\partial}^{\B}V,
\]
the following identification is deduced:
\[
\overline{V}\setminus \partial^{\B} V\leftrightarrow (a,b) \times M_1^C\times M_2^C.
  \]
  In particular, $\partial V\setminus \partial^{\B} V$ can be identified with a cone with base $(M_1^C\times M_2^C)\setminus (M_1\times M_2)$. Moreover, if we assume that both $M_1^C,M_2^C$ are locally compact, Prop. \ref{topcurvasfinitas} ensures that previous bijection is a homeomorphism. Particularly, this proves that, given $(P,F)\in \overline{V}\setminus \partial^{\B}V$,
    \[
P\in \hat{L}(\{P_n\}) \iff F\in \check{L}(\{F_n\})
      \]
      for any sequence $\{(P_n,F_n)\}_n\in \overline{V}$. Hence, $\overline{V}\setminus \partial^{\B}V$ is also simple topologically.
      %Prop. \ref{topcurvasfinitas} permits to extend previous bijection to a homeomorphism by assuming that $M_1^C$ and $M_2^C$ are both locally compact.\footnote{J.L.: Revisar esta afirmacion.}

  Finally, the following lemma ensures that the line over each point $(x_1^*,x_2^*)\in (M_1^C\times M_2^C)\setminus (M_1\times M_2)$ is timelike:
 \begin{lemma}\label{causalstructurenoinf}
     If $(P,F),(P',F')\in \partial V\setminus \partial^{\B}V$, with $(P,F)\equiv (\Omega,x^*_1,x^*_2), (P',F')\equiv (\Omega',x^*_1,x^*_2)$ in $(a,b)\times M_1^C\times M_2^C$, satisfy that $a<\Omega<\Omega'<b$ then $(P,F)\ll (P',F')$.
  \end{lemma}
  \begin{proof}
     Take $t=(\Omega+\Omega')/2$ and $\mu_1=\mu_2=1/2$. For $i=1,2$, consider $y_i$ close enough to $x_i^*$ so that
    \[
\left\{\begin{array}{l}
\displaystyle\Integral{t}{\Omega'}{\mu_{i}}{i}{\mu_{k}}>
             d_{i}(y_i,x_i^*) \\

\displaystyle\Integral{\Omega}{t}{\mu_{i}}{i}{\mu_{k}}>
             d_{i}(y_i,x_i^*),

\end{array}\right.
\quad i=1,2.
      \]
    From Prop. \ref{pastofcurve} (and its past analogous) we deduce that $(t,y_1,y_2)\in F\cap P'$, as desired.
  \end{proof}

  \smallskip

  The $S$-relation described in Prop. \ref{tip} implies that each pair $(P,F)\in\overline{V}$ is determined by any of its non-empty components, that is, $\overline{V}$ is simple as a point set. Even more, from Prop. \ref{topcurvasfinitas} and the definition of the chronological limit (see \eqref{eq:29} and \eqref{limcrono}), $\overline{V}$ is topologically simple as well (recall Defn. \ref{simpletop}); concretely, if $(P,F)\in \overline{V}$, $P\neq\emptyset$, and $\sigma=\{(P_n,F_n)\}_n\subset \overline{V}$, then $(P,F)\in L_{chr}(\sigma)$ if, and only if, $P\in \hat{L}_{chr}(\{P_n\}_n)$. Therefore, in order to determine the, pointwise and topological, structure of the (total) $c$-boundary, it suffices to study the partial boundaries. Consequently, we will describe $\overline{V}$ in two different ways, according to our convenience, namely:
  \[
\overline{V}= (a,b) \times M_1^C\times M_2^C\cup\hat{\partial}^{\B}V\cup\check{\partial}^{\B}V=\hat{V}\cup \check{\partial}^{\B}V=\hat{\partial}^{\B}V\cup \check{V}.
    \]
     Restricting conveniently, the open sets of $\overline{V}$ containing a pair $(P,F)$ can be viewed as: (i) open sets in $(a,b) \times M_1^C\times M_2^C$ if $P\neq\emptyset\neq F$, (ii) open sets in $\hat{V}$ if $F=\emptyset$ or (iii) open sets in $\check{V}$ if $P=\emptyset$.

\smallskip

It rests to determine the causal structure of $\overline{V}$. This is contained in the following result, which summarizes all the information about the (total) c-completion of doubly warped spacetimes:
  \begin{thm}\label{thm:main}
    Let $(V,g)$ be a {\multiwarped} spacetime as in \eqref{eq:1-aux}. Then, there exists a homeomorphism
    \[
\overline{V}\setminus \partial^{\B}V \leftrightarrow (a,b) \times M_1^C\times M_2^C,
      \]
    where each line $\{(t,x_1^*,x_2^*): t\in (a,b),\; (x_1^*,x_2^*)\in M_1^C\times M_2^C\}$ is timelike. Moreover:
      %$\partial^{\infty}V$ has the following structure:
     \begin{itemize}
      \item[(i)] If \eqref{eq:7} and \eqref{eqq:7} hold, then $\partial^{\B} V$ is homeomorphic to a couple of spacelike copies of $M_1^C\times M_2^C$. As consequence, we have the following homeomorphism:
        \begin{equation}
          \label{eq:18}
         \overline{V}\equiv [a,b]\times M_1^C\times M_2^C\quad\hbox{pointwise and topologically.}
        \end{equation}

        \item[(ii)] If \eqref{eq:7} and \eqref{eqq:9} hold, then $\partial^{\B} V$ has a copy of $M_1^C\times M_2^C$ for the future, with spatial causal character; and a copy of $M_1^C\times \left({\cal B}(M_2)\cup \{\infty\}\right)$ for the past.  This second set can be seen as a cone with base $M_1^C\times \partial_{\cal B}(M_2)$ generated by horismotic lines over each pair $(x_1^*,[b_{c_2}])$ ending at the point $(x_1^*,\infty)$. As consequence, we have the following homeomorphism
          \[
              \overline{V}\equiv\left\{\begin{array}{l} \hat{V}\cup \check{\partial}^{\B}V \leftrightarrow \left((a,b]\times M_1^C\times M_2^C\right) \cup \left(M_1^C\times \left({\cal B}(M_2)\cup \{\infty\}\right)\right) \\ \hat{\partial}^{\B} V\cup \check{V} \leftrightarrow \left(\{b\}\times M_1^C\times M_2^C\right) \cup \left(M_1^C\times \left(B(M_2)\cup \{\infty\} \right) \right).
            \end{array}\right.
            \]


          \item[(iii)] If \eqref{eq:9} and \eqref{eqq:7} hold we have a structure analogous to (ii), but interchanging the roles of future and past.

            \item[(iv)] If \eqref{eq:9} and \eqref{eqq:9} hold, then $\partial^{\B} V$ has two copies of the space $M_1^C\times \left({\cal B}(M_2)\cup \{\infty\}\right)$, one for the future and the other one for the past, formed by horismotic lines over each point $(x_1^*,[b_{c_2}])\in M_1^C\times \partial_{\cal B}(M_2)$ ending at the point $(x_1^*,\infty)$. As consequence,
              \[
                  \overline{V}\equiv \left\{\begin{array}{l} \hat{V}\cup \check{\partial}^{\B}V \leftrightarrow \left(M_1^C\times \left(B(M_2)\cup \{\infty\} \right)\right) \cup \left(M_1^C\times \left({\cal B}(M_2)\cup \{\infty\}\right)\right) \\
                  \hat{\partial}^{\B} V\cup \check{V} \leftrightarrow \left( M_1^C\times \left({\cal B}(M_2)\cup \{\infty\}\right)\right) \cup  \left(M_1^C\times \left(B(M_2)\cup \{\infty\}\right)\right).
                \end{array}\right.
            \]
      \end{itemize}
    \end{thm}
    \begin{proof}
As we have argued before, the first assertion about the point set topological and causal structure of $\overline{V}\setminus \partial^{\ncambios{b}}V$ is a direct consequence of \cambios{Props. \ref{structuraparcialsininfinito'}, \ref{topcurvasfinitas} (and its past analogous)}, \ref{tip} and Lemma \ref{causalstructurenoinf}. So, we will focus on the rest of assertions.

  \begin{itemize}
  \item[(i)] The point set and topological structure are straightforward from Thms. \ref{futurestructurefiniteconditions} and \ref{pfuturestructurefiniteconditions}. So, we only need to prove that $\partial^{\B}V=\hat{\partial}^{\B}V\cup \check{\partial}^{\B}V$ is spacelike. Take $(P,\emptyset),(P',\emptyset)\in \partial^{\B}V$ two different boundary points (for TIFs is completely analogous). By using the identification in \eqref{eq:18}, we can assume that $(P,\emptyset)\equiv (b,x_1^*,x_2^*)$ and $(P',\emptyset)\equiv (b,y_1^*,y_2^*)$ with $(x_1^*,x_2^*)\neq (y_1^*,y_2^*)$. From the proof of Prop. \ref{structuraparcialsininfinito} (recall also Rem. \ref{rem:1}) it follows both, $P\not \subset P'$ and $P'\not \subset P$, thus $(P,\emptyset)$ and $(P',\emptyset)$ are neither timelike nor lightlike related, i.e., they are spatially related.

  \item[(ii)] The point set and topological structure are deduced from Thm. \ref{futurestructurefiniteconditions} and Thm. \ref{pfuturecomploneinfinite}. For the causal structure, let us take two points $(P,\emptyset),(P',\emptyset)\in \partial^{\B}V$ over the same point $(x_1^*,[b_{c}])\in M_1^C \times \partial_{\cal B} M_2$. Hence, we can make the identifications $(P,\emptyset)\equiv (x_1^*,b_{c_1})$ and $(P',\emptyset)\equiv (x_1^*,b_{c_2})$ with $b_{c_1}-b_{c_2}={\cal K}$, ${\cal K}$ constant. If we assume that ${\cal K}>0$, then $b_{c_1}\geq b_{c_2}$, and so, $P'\subset P$ (recall Lemma \ref{lemma:aux5}), i.e., both points are lightlike related (the case with ${\cal K}<0$ is completely analogous).
  \end{itemize}

Finally, assertions (iii) and (iv) are easily deduced from (i) and (ii).

\end{proof}

\begin{rem} {\rm (1) Of course, the four cases considered in previous theorem do not cover all the possibilities compatible with the finiteness of at most one warping integral (since the finite warping integral may not be necessarily the last one). However, the structure of the c-completion for these additional cases are easily deducible from our approach, and can be considered an easy exercise for the reader.

(2) In order to simplify the exposition, we have considered along this paper multiwarped spacetimes with just two fibers. Nevertheless, the corresponding results for the general case of $n$ fibers can be easily deduced by the reader (see, for instance, Section \ref{sec:applications}).}
\end{rem}

%\subsection{Multiwarped models.}
%\label{sec:multiwarped}
%Up to now, we analyzed the case of just two fibers...
%It is worth noting that, even if we have considered the doubly warped model along the paper, it is possible to obtain analogous results for the corresponding multiwarped models. Concretely, let us assume that $(V,g)$ is a multiwarped model where $\mathcal{V}=(a,b)\times \Sigma_1\times \dots\times \Sigma_n\times \Sigma_{n+1}$ and
%\begin{equation}
%  \label{eq:32}
%\mathfrak{g}=-dt^2+\alpha_1g_1+ \dots + \alpha_ng_n+\alpha_{n+1} g_{n+1},
%\end{equation}
%
%In this models, the characterization of the chronological relation, which is the key for the studies in this paper, reads as (compare with Prop. \ref{c0})
%\begin{prop}
%  Let $(V,g)$ be a multiwarped spacetime as in (\ref{eq:32}), and $(t^{o},x^{o}), (t^{e},x^{e})\in V$ with $x^{o}, x^{e}\in \prod_{i=1}^{n+1}\Sigma_i$ and $x^{o}\neq
%x^{e}$. The following conditions are equivalent:
%\begin{itemize}
%
%\item[(i)]  $(t^{o},x^{o})\ll (t^{e},x^{e})$; or, equivalently, $t^o<T(x^o,(t^e,x^e))$\footnote{Observe that the departure time function do not depend on the specific expression for the metric.} (recall (\ref{e0}));
%\item[(ii)] the departure time function $T(x^o,(t^e,x^e))$ is the unique real value $T\in \R$
%with $t^{o}<T<t^{e}$ such that for some (unique) constants $\mu_{1},\dots,\mu_{n+1} \geq
%0$, with $\sum_{i=1}^{n+1}\mu_{i}=1$, it satisfies
%\begin{equation}\label{e2}
%\Integralm{T}{t^{e}}{\mu_{i}}{i}{\mu_{k}}=d_{i}(x^{o}_{i},x^{e}_{i})\qquad\hbox{for}\;\;
%i=1,\dots,n+1;
%\end{equation}
%
%\item[(iii)] there exist positive constants $\mu'_{1},\dots,\mu'_{n}> 0$, with $\sum_{i=1}^{n+1}\mu'_i=1$,
%such that
%\begin{equation}\label{e2''}
%\Integralm{t^{o}}{t^{e}}{\mu_{i}'}{i}{\mu_{k}'}>
%d_{i}(x^{o}_{i},x^{e}_{i})\qquad\hbox{for $i=1,\dots,n$}.
%\end{equation}
%\end{itemize}
%\end{prop}
%
%Then, the rest of results follow analogously, just taking into account that the integral conditions \eqref{eq:7} and \eqref{eq:9} correspond respectively with
%
%\begin{equation}
%  \label{eq:33}
%  \int_{0}^{\infty}\frac{1}{\sqrt{\alpha_i(s)}}ds<\infty, \qquad \hbox{for $i=1,\dots,n+1$}
%  \end{equation}
%and
%\begin{equation}
%  \label{eq:34}
%  \left\{\begin{array}{l}
%    \displaystyle\int_{0}^{\infty}\frac{1}{\sqrt{\alpha_i(s)}}ds<\infty, \qquad \hbox{for $i=1,\dots,n$}\\
%\\
%   \displaystyle \int_{0}^{\infty}\frac{1}{\sqrt{\alpha_{n+1}(s)}}ds=\infty.
%  \end{array}\right.
%\end{equation}
%while the corresponding integrals for the past  \eqref{eq:40} and \eqref{eq:42} correspond with
%
%\begin{equation}
%  \label{eq:44}
%    \int^{0}_{-\infty}\frac{1}{\sqrt{\alpha_i(s)}}ds<\infty, \qquad \hbox{for $i=1,\dots,n+1$}
%\end{equation}
%and
%\begin{equation}
%  \label{eq:45}
%  \left\{\begin{array}{l}
%    \displaystyle\int^{0}_{-\infty}\frac{1}{\sqrt{\alpha_i(s)}}ds<\infty, \qquad \hbox{for $i=1,\dots,n$}\\
%\\
%   \displaystyle \int^{0}_{-\infty}\frac{1}{\sqrt{\alpha_{n+1}(s)}}ds=\infty.
%  \end{array}\right.
%\end{equation}
%
%
%Concretely, we arrive to the following main result for multiwarped models (compare with Thm. \ref{thm:main} assuming $M_1=\Sigma_1\times\dots\times \Sigma_n$ and $M_2=\Sigma_{n+1}$):
%
%
%\begin{thm}\label{thm:main2}
%   Let $(V,g)$ be a multiwarped spacetime as in \eqref{eq:32}. Then,
%    \[
%\overline{V}\setminus \partial^{\infty}V\equiv \cambios{(a,b)}\times \left(\prod_{i=1}^n  \Sigma_i^C\right)\times \Sigma_{n+1}^C\quad\hbox{pointwise and topologically.}
%      \]
%     Moreover, the line $\{(t,x_1^*,\dots,x_n^*,x_{n+1}^*): t\in\R\}$ is timelike.
%
%      \cambios{The rest of boundary points, that is,}  $\partial^{\infty}V$ has the following structure:
%     \begin{itemize}
%      \item[(i)] If \eqref{eq:33} and \eqref{eq:44} hold, then $\partial^{\infty} V$ correspond to two copies of $\prod_{i=1}^n  \Sigma_i^C\times \Sigma_{n+1}^C$ with spatial causal character. Moreover,
%        \begin{equation}
%         \overline{V}\equiv \cambios{[a,b]}\times \left(\prod_{i=1}^n  \Sigma_i^C\right)\times \Sigma_{n+1}^C\quad\hbox{pointwise and topologically.}
%        \end{equation}
%
%        \item[(ii)] If \eqref{eq:33} and \eqref{eq:45} hold, then $\partial^{\infty} V$ has a copy of $\prod_{i=1}^n  \Sigma_i^C\times \Sigma_{n+1}^C$ for the future, with spatial causal character; and a copy of $\prod_{i=1}^n  \Sigma_i^C\times \left({\cal B}(\Sigma_{n+1})\cup \{\infty\}\right)$ for the past.  This second set can be seen as a cone with base $\prod_{i=1}^n  \Sigma_i^C\times \partial_{\cal B}(\Sigma_{n+1})$ formed by \cambios{horismotic} lines over each pair $(x_1^*,\dots,x_n^*,[b_{c_{n+1}}])$ \cambios{ending at the point $(x_1^*,\dots,x_n^*,\infty)$}. Moreover,
%          \[
%            \begin{array}{rl}
%              \overline{V}\equiv &\hat{V}\cup \check{\partial}^{\infty}V\equiv \left(\cambios{(a,b]}\times \prod_{i=1}^n  \Sigma_i^C\times \Sigma_{n+1}^C \right) \cup \left(\prod_{i=1}^n  \Sigma_i^C\times \left({\cal B}(\Sigma_{n+1})\cup \{\infty\}\right)\right)\\
%
%              \equiv & \hat{\partial}^{\infty} V\cup \check{V} \equiv \left(\cambios{\{b\}}\times \prod_{i=1}^n  \Sigma_i^C\times \Sigma_{n+1}^C\right) \cup \left(\prod_{i=1}^n  \Sigma_i^C\times B(\Sigma_{n+1})\right)
%
%            \end{array}
%            \]
%            both pointwise and topologically.
%
%          \item[(iii)] If \eqref{eq:34} and \eqref{eq:44} hold we have a structure completely analogous to (ii), but interchanging the roles of future and past.
%
%            \item[(iv)] If \eqref{eq:34} and \eqref{eq:45} hold, then $\partial^\infty V$ has two copies of the space $\prod_{i=1}^n  \Sigma_i^C\times \left({\cal B}(\Sigma_{n+1})\cup \{\infty\}\right)$; one for the future and the other for the past, formed by horismotic lines over each point $(x_1^*,\dots,x_n^*,[b_{c_{n+1}}])$  in $\prod_{i=1}^n  \Sigma_i^C\times \partial_{\cal B}(\Sigma_{n+1})$ ending at the point $(x_1^*,\dots,x_n^*,\infty)$. Moreover,
%              \[
%                \begin{array}{rl}
%                  \overline{V}\equiv & \hat{V}\cup \check{\partial}^{\infty}V\equiv \left(\prod_{i=1}^n  \Sigma_i^C\times \left(B(\Sigma_{n+1})\cup \{\infty\} \right)\right) \cup \left(\prod_{i=1}^n  \Sigma_i^C\times \left({\cal B}(\Sigma_{n+1})\cup \{\infty\}\right)\right)\\
%                  \equiv & \hat{\partial}^{\infty} V\cup \check{V} \equiv \left( \prod_{i=1}^n  \Sigma_i^C\times \left({\cal B}(\Sigma_{n+1})\cup \{\infty\}\right)\right) \cup  \left(\prod_{i=1}^n  \Sigma_i^C\times \left(B(\Sigma_{n+1})\cup \{\infty\}\right)\right)
%                \end{array}
%            \]
%
%
%
%      \end{itemize}
%    \end{thm}


\section{Some examples of interest}%\footnote{Jony: TRABAJO EN PROCESO!!!}
\label{sec:applications}

In this section we are going to apply our results to compute the c-completion of some spacetimes of physical interest. Concretely, we will consider some Kasner models, the intermediate region of Reissner-Nordstr\"om and de Sitter models with (non necessarily compact) internal spaces.

% \begin{rem}
%   CUIDADO: El principal problema que veo en esta sección es que mostramos pocos ejemplos que sólo se resuelvan con los resultados de este paper. Fijémonos que:
%   \begin{itemize}
%   \item En el caso del Kasner, sólo estamos incluyendo un caso más.
%   \item En el caso de Reissner-Nordstr\"om tenemos los correspondientes bordes conformes, luego podríamos calcularlo con un paper anterior.
%     \item En el caso de los inflacionarios, el resultado puede deducirse de lo hecho por Harris.
%   \end{itemize}
% Por ello, veo NECESARIO incluir una sección adicional con el producto de de-Sitter con espacios tanto compactos como no compactos (estos últimos motivados con alguno de los paper que busqué el otro día, cuidado aquí...).
%
%\end{rem}
    \subsection*{Kasner models}
{\em Generalized Kasner models} are multiwarped  spacetimes $(V,g)$ where $V=(0,\infty)\times \R^{n}$ and
\begin{equation}
  \label{eq:35}
g=-dt^2+t^{2p_1}dx_1^2+\dots +t^{2p_{n}}dx_{n}^2,\quad\hbox{$(p_1,\ldots,p_n)\in\R^n$}.
\end{equation}
These models are solutions to the vacuum Einstein equations if $(p_1,\dots,p_{n})\in \R^{n}$ belongs to the so-called {\em Kasner sphere}, i.e., if it satisfies
  \[
\sum_{i=1}^{n}p_i=1=\sum_{i=1}^{n}p^2_i.
\]
Even if this condition does not fall under the hypotheses of our results, this does not cover all the cases of interest, and so, we are not going to assume it.

As far as we know, the c-boundary of these models can be faced in two different ways. On the one hand, by using Harris' result (Thm. \ref{thm:harris}); taking into account that the fibers are complete, this result gives a full description of the future c-boundary when $p_i>1$ for all $i$, and provides some partial information in the other cases. On the other hand, these models have been studied
 by Garc\'ia-Parrado and Senovilla in \cite{GS03} by using the isocausal relation. They essentially prove that, depending on the values of the constants $p_1,\ldots,p_n$, the corresponding Kasner model is isocausal to a particular Robertson-Walker model whose c-boundary is well-known. This may be useful, since, although the c-boundary of isocausal spacetimes may be different (see \cite{0264-9381-28-17-175016}), they can share some qualitative properties (see \cite{FHSIso2}).

 \smallskip


Of course, Thm. \ref{futurestructurefiniteconditions} parallels Harris' result for Kasner models when $p_i>1$ for all $i$. However, now we can go a step further and give a complete description of the c-boundary when
% Our aim here is to show that our result does not just  determine the causal boundary of Kasner models by using Thm. \ref{thm:main2} and compare it with the one obtained by García-Parrado and Senovilla. Observe that not all Kasner models will fall under the hypothesis of such a theorem. For instance, if the vector $(p_1,\dots,p_n)$ belongs to the Kasner sphere, they satisfy in particular that $|p_i|<1$, and so, the warping functions in \eqref{eq:35} will not satisfy the required integral conditions.

% \smallskip

% For this, we will distinguish two main cases:

% \begin{itemize}
% \item Let us assume that $p_i>1$ for all $i$. Then, we have that

%   \begin{equation}
%     \label{eq:36}
%    \int_1^{\infty}\frac{1}{t^{p_i}}dt<\infty \qquad \hbox{and}\qquad \int_0^{1}\frac{1}{t^{p_i}}dt=\infty,
%   \end{equation}


%     for all $i$, and so we fall under the hypothesis of Thm. \ref{thm:main2} (i). Therefore, as all the fibres are complete, it follows that:

%     \[
% \overline{V}\equiv [0,\infty]\times \R^{n},\qquad \partial V\equiv \left(\{0\}\times \R^{n} \right)\cup \left(\{\infty\}\times \R^{n} \right)
%       \]
% COMPARAR CON EL OTRO!!\footnote{Aquí hay un error, pero parece que puede solucionarse del siguiente modo: usamos la relación isocausal para pasar del Kasner a un Robertson-Walker. Con ello, puedo probar que dos curvas generan el mismo pasado y jugar con la distancia definida ahí....PENSAR CON CALMA}
    % \item For the second case we will make two assumptions. First, we will assume that there exists $1\leq k \leq n$ such that $p_{k+1}=p_j$ for all $k+1\leq j\leq n$. In particular, the Kasner metric becomes:
%      \[
%-dt^2+\sum_{i=1}^k t^{p_i}dx_i^2 + t^{p_{k+1}}\left(\sum_{i=k+1}^n dx^2_i \right),
%        \]
%  where
\[
p_i>1\;\;\hbox{for $1\leq i\leq k$,}\quad p_i=q\;\;\hbox{for $k+1\leq i\leq n$}\quad\hbox{and}\quad \int_1^{\infty}\frac{1}{t^{q}}dt=\infty.
\]
In this case we can write
\[
V=(0,\infty)\times\R^k\times\R^{n-k},\qquad g=-dt^2+\sum_{i=1}^{k}t^{2p_i}dx_i^2+t^{2q}\left(\sum_{i=k+1}^ndx_i^2 \right),
\]
% where
% \[
% \alpha_i(t)=t^{2 p_i},\;\; i=1,\ldots,k,\qquad \alpha_{k+1}(t)=t^{2q}.
% \]
In particular,
\[
\int_{1}^{\infty}\frac{dt}{t^{p_i}}<\infty,\;\; i=1,\ldots,k,\qquad\int_{1}^{\infty}\frac{dt}{t^{q}}=\infty.
\]
Therefore, the spacetime falls under the hypotheses of (the obvious multiwarped version of) Thm. \ref{futurecomploneinfinite} (essentially, with $M_1=\R^k$ and $M_2=\R^{n-k}$), which provides
%the warping functions satisfy \eqref{eq:45}, and thus, (the obvious multiwarped version of) Thm. \ref{futurecomploneinfinite} provides
the following homeomophism:
          \[
\hat{V}\;\leftrightarrow\; \left((0,\infty)\times \R^n\right)\cup \left(\R^k\times \left({\cal B}(\R^{n-k})\cup \{\infty\}\right)\right).
            \]
            So, taking into account that (see, for instance, \cite[Section 5.1]{H2})
            \[
{\cal B}(\R^{n-k})\equiv \R\times \mathbb{S}^{n-k-1},
              \]
              we immediately deduce that
              \[
\hat{\partial} V\leftrightarrow \R^k \times \left(\left(\R\times \mathbb{S}^{n-k-1}\right)\cup \{\infty\} \right).
                \]




%  This result provides a point set description of the causal completion of $(V,g)$ when the integrals $\int_{0}^{+\infty} \frac{1}{\sqrt{\alpha_{i}}}ds$ and $\int_{-\infty}^{0} \frac{1}{\sqrt{\alpha_{i}}}ds$ are finite for all $i=1,2$:
% \[
% \overline{V} \equiv (\mathbb{R}\cup \{+\infty\} \cup \{-\infty\}) \times M_{1}^{c} \times M_{2}^{c}.
% \]
% The point set structure of the c-boundary is then:
% \[
% \partial V \equiv (\mathbb{R} \times \partial_{c} M_{1} \times M_{2}^{c}) \cup (\mathbb{R} \times M_{1}^{c} \times \partial_{c} M_{2}) \cup ((\{-\infty\} \cup \{+\infty\}) \times M_{1}^{c} \times M_{2}^{c}).
% \]
%  Here, $(\{+\infty\} \cup \{-\infty\}) \times M_{1}^{c} \times M_{2}^{c}$ denotes the set of spatial boundary points, that is, pairs of the form $(P,\emptyset)$ and $(\emptyset,F)$, that are identified with points of the form $(\pm \infty,x_{1}^{\pm},x_{2}^{\pm})$ for some univocally determined $(x_{1}^{\pm},x_{2}^{\pm}) \in M_{1}^{c} \times M_{2}^{c}$ (recall Prop \ref{spatialboundaries}). The rest elements of the boundary correspond with timelike boundary points, that is, pairs of the form $(P,F)$ with $P \neq \emptyset \neq F$ and $P \sim_{S} F$, that are identified with a univocally determined point  $(\Omega,x_{1}^{*},x_{2}^{*}) \in \R \times M_{1}^{c} \times M_{2}^{c}$ (recall Prop. \ref{Srelatedtipstifs}).

%     \subsection*{C-completion on Multidimensional Inflationary Models with negative curvature}

% In \cite{doi:10.1063/1.532366}, Mignemi and Schmidt give a classification for multiwarped spacetimes with constant curvature, depending on the dimension of the model and the number of fibres $n$. Observe that the case where $n=1$ falls into the Robertson-Walker models, whose c-completion is well-known (see \cite{AF} for instance), so we will focus on $n\geq  2$. Moreover, let us restrict to the case of doubly warped models with {\em negative} constant curvature. Then, according to the classification given in \cite{doi:10.1063/1.532366}, $(V,g)$ falls into one of the following categories:

% \begin{itemize}
% \item[(a)] $D={\rm dim}(V)=3$, $V=\R\times \R\times \R$ and
%   \[
%     g=-dt^2+\left(sinh^2(t)dx^2 + cosh^2(t)dy^2\right).
%   \]

% \item[(b)] $D={\rm dim}(V)\geq 4$, $V=\R\times \R\times \mathbb{S}^{D-2}$ and
%   \[
% g=-dt^2+\left(sinh^2(t)dx^2 + cosh^2(t)g_{\mathbb{S}^{D-2}}\right)
%     \]

%     or

%     \[
%       g=-dt^2+\left(sinh^2(t)g_{\mathbb{S}^{D-2}} + cosh^2(t)dx^2\right).
%       \]
% \item[(c)] $D={\rm dim}(V)\geq 4$, $V=\R\times \mathbb{S}^{k}\times \mathbb{S}^{j}$ with $k+j=D-1$ and
%   \[
% g=-dt^2+\left(sinh^2(t)g_{\mathbb{S}^k} + cosh^2(t)g_{\mathbb{S}^{j}}\right)
%     \]
% \end{itemize}
% Of course, in previous classification $(\mathbb{S}^l,g_{\mathbb{S}^l})$ denotes the $l$-dimensional sphere attached with its standard Riemannian metric. As we can see, in all previous cases it follows that:

%   \[
% \int_0^{\pm\infty} \frac{1}{|sinh(t)|}dt<\infty,\qquad \hbox{and}\qquad \int_0^{\pm\infty} \frac{1}{|cosh(t)|}dt<\infty
%     \]
%     so we are in the conditions of Thm. \ref{thm:main} (i). Moreover, in all the cases the fibres (that we will denote $M_1$ and $M_2$ as usual) are complete. In conclusion, the c-completion of any inflationary multidimensional model with negative constant curvature is of the form:
%     \[
% \overline{V}=\left(\R\cup \{\pm\infty\}\right)\times M_1\times M_2.
%       \]
%  and its c-boundary is formed by two copies of the product of its fibres (one for the future and one for the past) with spatial causality.


\subsection*{The intermediate  Reissner-Nordstr\"om}
%\footnote{Jony: Hablar con José Luis sobre el cambio de orientación...}
%Next, we consider the intermediate region of Reissner-Nordstr\"om spacetime. In this case, Thm. \ref{thm:harris} is not useful to determine the future c-completion.
The Reissner-Nordstr\"om model is a spacetime $(V,g)$, where $V=\R\times \R\times \mathbb{S}^2$ and
\[
g=-\left(1-\frac{2m}{r}+\frac{q^{2}}{r^{2}}\right)dt^{2}+\left(1-\frac{2m}{r}+\frac{q^{2}}{r^{2}}\right)^{-1}dr^{2}+r^{2}(d\theta^{2}+sin^{2}\theta d\phi^{2}).
\]
 This metric degenerates at the zeros of the function $f(r)=(1-2m/r+q^2/r^2)$, which depend on the parameters $m$ (mass) and $q$ (charge). For our purposes we will require that $q\leq m$, which ensures the zeros $r^{\pm}=m\left( 1\pm\sqrt{1-q^2/m^2}\right)$ for $f$. The {\em intermediate region} of the Reissner-Nordstr\"om  is the spacetime $(V_I,g)$, where $V_I=\R\times (r^-,r^+)\times \mathbb{S}^2$.

Taking into account that $f(r)<0$ on $(r^-,r^+)$, the metric $g$ can be rewritten on $V_I$ as
\begin{equation}
  \label{eq:37}
g= -f(r)dt^2 + \frac{1}{f(r)}dr^2 + r^2 d\sigma^2=-d\tau^2 + r(\tau)^2d\sigma^2-F(\tau)dt^2,
\end{equation}
where
\[
d\tau:=-\frac{dr}{\sqrt{f(r)}}=\frac{dr}{\sqrt{-1+2m/r-q^2/r^2}} \qquad \hbox{and}\qquad F(\tau)=f(r(\tau)).
\]
Note that $\tau$ ranges in a finite interval $(a,b)$, and so, $(V_I,g)$ clearly corresponds with the standard form of a doubly warped spacetime where $V_I=(a,b)\times\mathbb{S}^2\times \R$.
%\footnote{El cambio de orden en las fibras es para que coincida con la forma del teorema...}.
In order to proceed with the analysis of the c-completion of $(V_I,g)$, we need to distinguish two cases: $q\neq 0$ and $q=0$.\footnote{Since the Penrose's diagram of Reissner-Nordstr\"om is well-known (see, for instance, \cite{hawking1975large}), the c-completion of $(V_I,g)$ can be also studied by applying \cite[Thm. 4.32]{FHSFinalDef}.}
%
%determine the c-completion of the intermediate region of the Reissner-Nordstr\"om, we have to study the integral conditions for the corresponding warping functions $\alpha_1(\tau)=r^2(\tau)$ and $\alpha_2(\tau)=-F(\tau)$, which will depend on the value of $q$. In the next subsections we distinguish two cases.
%
%\begin{rem} {\em  By using \cite[Thm. 4.32]{FHSFinalDef}, recalling the Penrose's diagram of the Reissner-Nordstr\"om spacetime (see, for instance, \cite{hawking1975large}).
%}
%\end{rem}


\subsubsection*{Intermediate Reissner-Nordstr\"om with charge, $q\neq 0$.}

%\footnote{CUIDADO AQUÍ: Hay un cambio en la orientación, cuando $r$ crece, nos movemos en la dirección pasada y viceversa...}
In this case, the warping integrals satisfy, for $a<c<b$,
 \begin{align}
   \int_{a}^{c}\frac{1}{\sqrt{\alpha_1(\tau)}}d\tau = \int_{r^-}^{r(c)}\frac{1}{r\sqrt{-1+\frac{2m}{r}-\frac{q^2}{r^2}}}dr<\infty\label{eq:38a} \\
   \int_{c}^{b}\frac{1}{\sqrt{\alpha_1(\tau)}}d\tau=\int^{r^+}_{r(c)}\frac{1}{r\sqrt{-1+\frac{2m}{r}-\frac{q^2}{r^2}}}dr<\infty\label{eq:38b}
  \end{align}
and
  \begin{align}
    \int_{a}^{c}\frac{1}{\sqrt{\alpha_2(\tau)}}d\tau=\int_{r^-}^{r(c)}\frac{1}{-1+\frac{2m}{r}-\frac{q^2}{r^2}}dr=\infty \label{eq:38}\\
    \int_{c}^{b}\frac{1}{\sqrt{\alpha_2(\tau)}}d\tau=\int^{r^+}_{r(c)}\frac{1}{-1+\frac{2m}{r}-\frac{q^2}{r^2}}dr=\infty.\label{eq:38c}
  \end{align}
  So, from Thm. \ref{thm:main} (iv) (with $M_1=\mathbb{S}^2$ and $M_2=\R$), we deduce the homeomorphisms
  \[
    \begin{array}{c}
      \overline{V}\leftrightarrow \left((a,b)\times \mathbb{S}^2\times \R\right) \cup (\mathbb{S}^2\times \left(\left(\R\times \{z^-, z^+\}\right)\cup \{i^+\} \right))\cup (\mathbb{S}^2\times \left(\left(\R\times \{z^-, z^+\}\right)\cup \{i^-\}\right)),\\
      \\
\partial V\leftrightarrow (\mathbb{S}^2\times \left(\left(\R\times \{z^-, z^+\}\right)\cup \{i^+\} \right))\cup (\mathbb{S}^2\times \left(\left(\R\times \{z^-, z^+\}\right)\cup \{i^-\}\right)),
    \end{array}
    \]
    where we have used that ${\cal B}(\R)\equiv \R\times \{z^-,z^+\}$, being $z^-$ and $z^+$ the two asymptotic directions (left and right) of $\R$.
    %\footnote{J.L.: ¿Es esto realmente relevante? JONY: No especialmente, puede quitarse si se quiere.}In particular, the c-completion of the intermediate region $(V_I,g)$ with $q\neq 0$ coincides with the c-completion of the spacetime ${\mathbb L}^2\times {\mathbb S}^2$.
  %, and such identification extends to the chronological and topological level.

  \subsubsection*{Interior Schwarzschild, $q=0$.}

When $q=0$, $f(r)$ has only one zero, we can identify $(r^-,r^+)\equiv (0,2M)$, and the intermediate region of Reissner-Nordstr\"om coincides with the interior region of Schwarzschild. In this case, the warping integrals (\ref{eq:38a}), (\ref{eq:38b}) and (\ref{eq:38c}) still hold, but \eqref{eq:38} transforms into
  \[
\int_{a}^{c}\frac{1}{\sqrt{\alpha_2(\tau)}}d\tau = \int_{0}^{r(c)}\frac{1}{-1+\frac{2m}{r}}dr<\infty.
  \]
  So, from Thm. \ref{thm:main} (iii), we deduce the homeomorphism
  %applies, and so, the past c-boundary has only spatial points. More precisely,
  \[
\overline{V}\leftrightarrow \left([a,b)\times \mathbb{S}^2\times \R\right) \cup \left( \mathbb{S}^2\times \left(\R\times \{z^-,z^+\} \right)\right)
    \]
  and thus,\footnote{The usual time-orientation on Reissner-Nordstr\"om makes the vector field $\partial_r$ past-directed in the intermediate region. So, in formula (\ref{d}), the roles of the future and past c-boundaries are interchanged with respect to the (a priori) expected ones.}
 \begin{equation}\label{d}
  \partial V\equiv \hat{\partial} V\cup \check{\partial} V \leftrightarrow  \left(\{a\}\times \mathbb{S}^2\times \R   \right) \cup \left(\mathbb{S}^2\times \left( \left(\R\times \{z^-,z^+\}\right)\cup \{i^+\} \right)\right).
  \end{equation}

\subsection*{De Sitter models with (non-necessarily compact) internal spaces}

Motivated by the relevance for the problem of the dS/CFT correspondence, finally we study the c-boundary of warped products of de Sitter models with general Riemannian manifolds.

Recall that {\em de Sitter spacetime} can be seen as a Robertson-Walker spacetime $(M,g_{M})$, where
\[
M=\R\times \mathbb{S}^l,\qquad g_{M}=-dt^2 + cosh(t)^2 g_{\mathbb{S}^l}.
  \]
Consider the doubly warped spacetime $(V,g)$ obtained as the product of de Sitter space $(M,g_{M})$ and a Riemannian manifold $(F,g_{F})$, i.e.,
  \[V=\R\times \mathbb{S}^l\times F,\qquad g=-dt^2+cosh^2(t)g_{\mathbb{S}^{l}}+g_{F}.
    \]
  The first warping function $\alpha_1(t)=\cosh(t)^2$ satisfies the finite integral conditions for both, the future and the past directions, meanwhile the second one $\alpha_2(t)\equiv 1$ does not. Therefore, from Thm. \ref{thm:main} (iv) (with $M_1=\mathbb{S}^l$ and $M_2=F$), we deduce the following homeomorphism for the c-boundary of $(V,g)$:
\[
\partial {V}\equiv \hat{\partial} V \cup \check{\partial} V \leftrightarrow  \left(\mathbb{S}^l\times \left({\cal B}(F)\cup \{i^+\}\right) \right)\,  \cup \, \left(\mathbb{S}^l\times \left({\cal B}(F)\cup \{i^-\}\right) \right).
\]
  In particular, if $(F,g_{F})$ is compact, then ${\cal B}(F)$ is empty, and the c-boundary becomes (compare with the last assertion on Thm. \ref{thm:harris}):
\[
  \partial V\leftrightarrow \left(\mathbb{S}^l\times \{i^+\}) \right)\,  \cup \, \left(\mathbb{S}^l\times \{i^-\} \right).
  \]

%
%
%   to the product of the de Sitter space with any Riemannian manifold, we obtain:
  % and with warping function satisfying the finite integral condition.  In such cases, as both warping functions satisfy nice  integral conditions and the fibres are complete, it is possible to make  use of Thm. \ref{thm:harris}. However, there exist cases where we need to eliminate such a restriction on the warping function (for instance, on the dS/CFT correspondence, see \cite{}), and Thm. \ref{thm:main} becomes essential. For instance, we can obtain the following general result:

%\begin{prop}
%  Let $(V,g)$ be the multiwarped spacetime obtained as the product of de Sitter space $(M,g_{M})$ and a Riemannian manifold $(F,g_{F})$, i.e.,
%  \[V=\R\times \mathbb{S}^l\times F,\qquad g=-dt^2+cosh^2(t)g_{\mathbb{S}^{l}}+g_{F}.
%    \]
% Then, the c-boundary of $(V,g)$ becomes
%  \[
%      \partial {V}\equiv \hat{\partial} V \cup \check{\partial} V \equiv  \left(\mathbb{S}^l\times \left({\cal B}(F)\cup \{i^+\}\right) \right)\,  \cup \, \left(\mathbb{S}^l\times \left({\cal B}(F)\cup \{i^-\}\right) \right).
%    \]
%
%\end{prop}
%Therefore, the c-completion is expressed in terms of the proper Busemann boundary ${\cal B}(F)$ of the Riemannian manifold $(F,g_{F})$. In particular, if $(F,g_{F})$ is compact, then ${\cal B}(F)$ is empty, and we obtain (compare with the last assertion on Thm. \ref{thm:harris}):
%\begin{cor}
%  If $(F,g_{F})$ is compact, then the c-boundary of $(V,g)$ is
%\[
%  \partial V\equiv \left(\mathbb{S}^l\times \{i^+\}) \right)\,  \cup \, \left(\mathbb{S}^l\times \{i^-\} \right).
%  \]
%\end{cor}

% \subsubsection*{Generalized Kasner models}
% Generalized Kasner models correspond with Multiwarped spacetimes $(V,g)$ where $V=(0,\infty)\times \R^{n}$ and
% \begin{equation}
%   \label{eq:35}
% g=-dt^2+t^{2p_1}dx_1^2+\dots +t^{2p_{n}}dx_{n}^2
% \end{equation}

%   with $p_i$ arbitrary constants. These models are solutions of the vacuum Einstein equations if the vector $(p_1,\dots,p_{n})\in \R^{n}$ belongs to the so-called {\em Kasner sphere}, that is, if its components satisfy:

%   \[
% \sum_{i=1}^{n}p_i=1=\sum_{i=1}^{n}p^2_i.
% \]

% The causal boundary of such models was studied by Garc\'ia-Parrado and Senovilla in \cite{GS03} by means of the isocausal relation. Essentially they prove that, depending on the constants $p_i$, the corresponding Kasner model is isocausal to a particular Robertson-Walker model with known causal completion. However, as it was proved in \cite{0264-9381-28-17-175016}, such a procedure presents some drawback, as the causal boundary of isocausal models may differ (even if they seem to share some qualitative properties in general cases, see \cite{FHSIso2}).


% Our aim here is to determine the causal boundary of Kasner models by using Thm. \ref{thm:main2} and compare it with the one obtained by García-Parrado and Senovilla. Observe that not all Kasner models will fall under the hypothesis of such a theorem. For instance, if the vector $(p_1,\dots,p_n)$ belongs to the Kasner sphere, they satisfy in particular that $|p_i|<1$, and so, the warping functions in \eqref{eq:35} will not satisfy the required integral conditions.

% \smallskip

% For this, we will distinguish two main cases:

% \begin{itemize}
% \item Let us assume that $p_i>1$ for all $i$. Then, we have that

%   \begin{equation}
%     \label{eq:36}
%    \int_1^{\infty}\frac{1}{t^{p_i}}dt<\infty \qquad \hbox{and}\qquad \int_0^{1}\frac{1}{t^{p_i}}dt=\infty,
%   \end{equation}


%     for all $i$, and so we fall under the hypothesis of Thm. \ref{thm:main2} (i). Therefore, as all the fibres are complete, it follows that:

%     \[
% \overline{V}\equiv [0,\infty]\times \R^{n},\qquad \partial V\equiv \left(\{0\}\times \R^{n} \right)\cup \left(\{\infty\}\times \R^{n} \right)
%       \]
% COMPARAR CON EL OTRO!!\footnote{Aquí hay un error, pero parece que puede solucionarse del siguiente modo: usamos la relación isocausal para pasar del Kasner a un Robertson-Walker. Con ello, puedo probar que dos curvas generan el mismo pasado y jugar con la distancia definida ahí....PENSAR CON CALMA}
%     \item For the second case we will make two assumptions. First, we will assume that there exists $1\leq k \leq n$ such that $p_{k+1}=p_j$ for all $k+1\leq j\leq n$. In particular, the Kasner metric becomes:
%       \[
% -dt^2+\sum_{i=1}^k t^{p_i}dx_i^2 + t^{p_{k+1}}\left(\sum_{i=k+1}^n dx^2_i \right).
%         \]
%         Secondly, we will assume that $p_i>1$ for $1\leq i\leq k$, so they also satisfy the integral conditions \eqref{eq:36}; but for $p_{k+1}$ we have:

%         \[
%    \int_1^{\infty}\frac{1}{t^{p_{k+1}}}dt=\infty \qquad \hbox{and}\qquad \int_0^{1}\frac{1}{t^{p_{k+1}}}dt<\infty,
%           \]

% \end{itemize}




%  This result provides a point set description of the causal completion of $(V,g)$ when the integrals $\int_{0}^{+\infty} \frac{1}{\sqrt{\alpha_{i}}}ds$ and $\int_{-\infty}^{0} \frac{1}{\sqrt{\alpha_{i}}}ds$ are finite for all $i=1,2$:
% \[
% \overline{V} \equiv (\mathbb{R}\cup \{+\infty\} \cup \{-\infty\}) \times M_{1}^{c} \times M_{2}^{c}.
% \]
% The point set structure of the c-boundary is then:
% \[
% \partial V \equiv (\mathbb{R} \times \partial_{c} M_{1} \times M_{2}^{c}) \cup (\mathbb{R} \times M_{1}^{c} \times \partial_{c} M_{2}) \cup ((\{-\infty\} \cup \{+\infty\}) \times M_{1}^{c} \times M_{2}^{c}).
% \]
%  Here, $(\{+\infty\} \cup \{-\infty\}) \times M_{1}^{c} \times M_{2}^{c}$ denotes the set of spatial boundary points, that is, pairs of the form $(P,\emptyset)$ and $(\emptyset,F)$, that are identified with points of the form $(\pm \infty,x_{1}^{\pm},x_{2}^{\pm})$ for some univocally determined $(x_{1}^{\pm},x_{2}^{\pm}) \in M_{1}^{c} \times M_{2}^{c}$ (recall Prop \ref{spatialboundaries}). The rest elements of the boundary correspond with timelike boundary points, that is, pairs of the form $(P,F)$ with $P \neq \emptyset \neq F$ and $P \sim_{S} F$, that are identified with a univocally determined point  $(\Omega,x_{1}^{*},x_{2}^{*}) \in \R \times M_{1}^{c} \times M_{2}^{c}$ (recall Prop. \ref{Srelatedtipstifs}).



% \begin{prop}
% If $(P,F) \in \overline{V}$ and $(P,F) \in L(\{(P_{m},F_{m})\}_{m})$ then $\{(\Omega_{m},x_{m}^{*})\}_{m}$ converges to $(\Omega,x^{*})$, where $(P,F)$ and each $(P_{m},F_{m})$ are identified with $(\Omega,x^{*})$ and $(\Omega_{m},x_{m}^{*})$ in $(\mathbb{R} \cup \{+\infty\} \cup \{-\infty\}) \times \overline{M}_{1}^{c}\times \cdots \times \overline{M}_{n}^{c}$, respectively.
% \end{prop}



% EL SIGUIENTE RESULTADO (MUY ESQUEMÁTICO) RESUME TODA LA ESTRUCTURA DE LOS ESPACIOTIEMPOS {\multiwarped}.

% \begin{thm}
% Let $(V,g)$ be a {\multiwarped} spacetime. $\overline{V}$ has the following structure.
%   \begin{enumerate}
%   \item Point set structure:

%   \begin{equation}
%     \label{eq:17}
% \overline{V}\setminus \partial^\infty V\equiv \R\times M_1^C\times M_2^C,
% \end{equation}
% with the product topology. Moreover, each point on $\left(M_1^C\times M_2^C\right)\setminus M_1\times M_2$ generates a timelike line over the boundary.

% For $\partial^\infty V$ we have:
% \begin{itemize}
% \item If \eqref{eq:7}, it is formed by two (spatial) copies of $M_1^C\times M_2^C$, one for the future and one for the past. So,
%   \[
% \overline{V}equiv\left(\R\cup \{\pm\}\right)\times M_1^C\times M_2^C
%     \]
% and previous equivalence extends topologically.
% \item If \eqref{eq:9}, it is formed by two copies of $M_1^C\times {\cal B}(M_2)$, one for the future and one for the past.  Each pair $x_1\in M_1^C$ and $[b_{x_2}]\in \partial_{{\cal B}(M_2)} M_2$ defines a lightlike line over the future and past boundary.
%   \end{itemize}

% \end{enumerate}
% \end{thm}



%\begin{proposition}
%Let $\gamma(t)=(t,c(t))$ with $\gamma(t) \rightarrow (\Omega,x^{*})$ be a future directed causal curve with $\Omega < \infty$
%and $x^* \in \bar{M}^{C}$ then
%$$\uparrow I^{-}(\gamma)=\{(t,q)\in V \mid \exists\ \mu_{1}^{q},...,\mu_{n}^{q}>0 {\text{ such that }}
%\int_{\Omega}^{t} \frac{\sqrt{\mu_{i}^{q}}}{\alpha_{i}}(\Sigma_{k}\frac{\mu_{k}^{q}}{\alpha_{k}})^{-1/2}>d_{i}(q_{i},x_i^*)\}$$
%\end{proposition}
%
%Let $(t_q,q) \in V$ such that for the future directed causal curve $\gamma(t)=(t,c(t))$ we have $\gamma(t) \ll (t_{q},q)$ for all
%$t \in [a,\Omega)$, then by the characterization of the chronological relation we have the existence of
%$(\mu_{1}^q(t),...,\mu_{n}^q(t)) \in (0,1)^n$ with $\Sigma_{k} \mu_{k}^q(t)=1$ and satisfying the inequalities:
%$$\int_{t}^{t^{q}} \frac{\mu_{i}^q(t)}{\alpha_{i}}\left(\Sigma_{k}\frac{\mu_{k}^q(t)}{\alpha_{k}}\right)^{-1/2}d\tau>d_{i}(q_{i},c_{i}(t)) \ \forall i,$$
%take any $\{t_{l}\}$ such that $t_{l} \rightarrow \Omega$ and consider the point $(\mu_{1}^{q},...,\mu_{n}^{q})$ limit point of some
%subsequence $\{(\mu_{1}^{q}(t_{l}),...,\mu_{n}^{q}(t_{l}))\}_{l}$ that satisfies the integral inequalities showed above, by using the convergence
%of those sequences we have:
%$$\int_{\Omega}^{t^{q}} \frac{\mu_{i}^q}{\alpha_{i}}\left(\Sigma_{k}\frac{\mu_{k}^q}{\alpha_{k}}\right)^{-1/2}d\tau \geq d_{i}(q_{i},x_{i}^{*}) \ \forall i,$$
%therefore $$\uparrow I^{-}[\gamma]=I^{+}[\{(t,q)\in V \mid \text{ there exists } \mu_{1}^{q},...,\mu_{n}^{q} \geq 0 \text{ such that }
%\int_{\Omega}^{t} \frac{\sqrt{\mu_{i}^{q}}}{\alpha_{i}}(\Sigma_{k}\frac{\mu_{k}^{q}}{\alpha_{k}})^{-1/2} \geq d_{i}(q_{i},x_i^*)\}]$$
%
%
%
%
%since this $\mu_{i}^{q}$ are taken as a limit of $\mu_{i}^{q}(t_{l})$ then it can happen that some $\mu_{i}^{q}$ are equal to zero, but not all of them are by
%the condition over the sum of $\mu_{i}^{q}$. Therefore for
%

%%% Local Variables:
%%% mode: latex
%%% TeX-master: "DoublyWarpedBoundary2017"
%%% End:

%\input{biblio2}

\section*{Acknowledgments}

The authors are partially supported by the Spanish Grant MTM2016-78807-C2-2-P (MINECO and FEDER funds). L. Aké also acknowledges a grant funded by the Consejo Nacional de Ciencia y Tecnolog\'ia (CONACyT), M\'exico.

\bibliographystyle{unsrt}
\bibliography{biblio2}



%\section{SIMULATION RESULTS}
\label{sec:examples}
This section presents simulation results of the proposed method implemented on the unicycle model example.
Each semidefinite program was prepared using a custom software toolbox and the modeling tool YALMIP \cite{lofberg2004yalmip}.
The programs are run with commercial solver MOSEK on a machine with $1$ TB availabe memory. 

\subsection{FRS Computation}
We computed the FRS for a 3$^\text{rd}$ order Taylor-expanded Dubins car as the low-fidelity model $f_s$.
Trajectories produced by this model were tracked by the unicycle model from Equation \eqref{eq:big_dyn} as the high-fidelity model $f$.
The vehicle's representation as an initial distribution $X_0 \subset X_s$, was a rectangle of length $0.2$ [m] in $x$ and width $0.1$ [m] in $y$, at $0^\circ$ initial heading, and centered at $x=-0.75$ and $y=0$.
This is the same vehicle representation shown in all previous figures.

% The error function $g$, illustrated in Figure \ref{fig:error_dynamics}, was given by:
% \begin{equation}
% \label{eq:g_definition}
% g(t,x_s) = \begin{bmatrix}
% v_\text{err}\cdot(1 - \frac{1}{2}\theta^2)  \\
% v_\text{err}\cdot(\theta - \frac{1}{6}\theta^3) \\
% \dot{\theta}_\text{err}
% \end{bmatrix}
% \end{equation}
% where $v_\text{err} = (t-1)^2$ and $\dot{\theta}_\text{err} = (t-1)^4$.
We chose $\tau_\text{stop} = \tau_\text{plan} = 0.5$ [s], so $T = 1$ [s].
The stopping time can be seen in Figure \ref{fig:error_dynamics}. 
The FRS computation took 79 hours and used a maximum of 150 GB of memory 
%on a server with 1 TB of available memory and 18 processors each running at 1.2 GHz.

\subsection{Set Intersection and Trajectory Planning}

We used the precomputed FRS for safe trajectory planning in $1000$ simulated trials in MATLAB on the aforementioned machine.
For each trial, the vehicle began at the same initial location and heading, surrounded by $1-10$ randomized obstacles and a randomly-located goal to reach.
%If the planning time took more than $\tau_\text{plan}$, the simulation paused until the computation was complete. 
%In practice, if $\tau_\text{plan}$ was exceeded the vehicle could begin braking to ensure safety.
The vehicle's initial speed, and the desired speed to maintain for the duration of the trial, were randomly chosen between $0.25$ and $0.75$ [m/s].
% The trials ran in 12.7 hours.
% Prior to running these trials, several example trials were run on a laptop with a 2.3 GHz processor and 16 GB of RAM.
% The trials run on the server were individually no faster than running on the laptop, because the set intersection optimization is a single-core process that uses very little memory. 
% Therefore, the server did not provide any significant decrease in the implemented planning time.


Obstacles were represented as line segments between $0.1$ and $0.2$[m] in length, with random location and orientation.
The obstacles were always placed between the vehicle and the goal.
We checked for crashes conservatively for each trial, by inspecting if any obstacle was within a circle circumscribing the rectangular vehicle at any point of the vehicle's trajectory. 
Using this method, \emph{no crashes were detected in any trial}.
Out of all the trials, $82\%$ reached the goal, and $15\%$ performed an emergency braking maneuver (by setting $v_\text{des} = 0$). 
The remaining 3\% hit a simulation iteration limit.
Examples of the vehicle's path from a randomly-generated trial and from two constructed emergency braking cases are shown in Figure \ref{fig:example_trial}.


\begin{figure}
\centering
\includegraphics[width=1\columnwidth]{running_examples.pdf}
\caption{The top subplot shows an example result out of the $1000$ trials.
This trial used eight randomly-generated obstacles.
The vehicle begins on the left at $x = -0.75$ and reaches a randomly-generated goal near $(2.5, 0.5)$, plotted as a blue circle.
Every $\tau_\text{plan} = 0.5$[s], the vehicle replans its trajectory, shown by an asterisk plotted on the global trajectory in blue.
The bounding box of the vehicle at each planning step is shown as a grey rectangle. In the bottom-left subplot, an obstacle was constructed between the vehicle and the goal, forcing an emergency braking maneuver. In the bottom-right subplot, an obstacle was constructed with a hole that would allow the vehicle to pass, but the set intersection result is overly conservative, resulting in a braking maneuver.}
\label{fig:example_trial}
\end{figure}

Currently, our implementation cannot consistently achieve $\tau_\text{plan} = 0.5$ [s].
Consequently, instead of replanning and driving simultaneously, we pause time every 0.5 [s] of the simulation to guarantee that the vehicle can finish replanning.
In a physical implementation, if $\tau_\text{plan}$ is exceeded, then the vehicle must emergency brake; recall that a safe braking trajectory is always available.
As shown in Figure \ref{fig:planning_time_vs_Nobs}, $\tau_\text{plan}$ scales linearly with the number of obstacles.
%Methods for reducing the set intersection to meet $\tau_\text{plan}$ will be presented in future work.

\begin{figure}
\centering
\includegraphics[scale=0.45,trim={1cm 6cm 1cm 7cm},clip]{planning_time_vs_Nobs.pdf}
\caption{The mean set intersection time (top) and trajectory optimization time (bottom) versus the number of obstacles. Over the $1000$ trials, each number of obstacles from $1$ to $10$ was used for $100$ trials. Notice that set intersection takes up to $3$[s], and scales with the number of obstacles. On the other hand, the trajectory optimization takes around $80$ [ms] and has low correlation with number of obstacles.}
\label{fig:planning_time_vs_Nobs}
\end{figure}

% \begin{figure}
% \centering
% \includegraphics[scale=0.5,trim={1cm 8cm 1cm 8cm},clip]{example_trial_bluecar.pdf}
% \caption{An example result out of the 1000 trials.
% This trial used eight randomly-generated obstacles.
% The vehicle begins on the left at $x = -0.75$ and reaches a randomly-generated goal near $(2.5, 0.5)$, plotted as a blue circle.
% Every $\tau_\text{plan} = 0.5$ [s], the vehicle replans its trajectory, shown by an asterisk plotted on the global trajectory in blue.
% The bounding box of the vehicle at each planning step is shown as a grey rectangle.}
% \label{fig:example_trial}
% \end{figure}

% \begin{figure}
% \centering
% \includegraphics[scale=0.4,trim={1cm 7cm 1cm 7cm},clip]{example_emergency_brake.pdf}
% \caption{An example of a forced emergency braking situation. The vehicle cannot find a path to the desired location (plotted as a blue circle), so it brakes.}
% \label{fig:example_emergency_brake}
% \end{figure}

% \begin{figure}
% \centering
% \includegraphics[scale=0.4,trim={1cm 7cm 1cm 7cm},clip]{example_overly_conservative.pdf}
% \caption{An example of an unnecessary emergency braking situation. The vehicle cannot find a path to the desired location despite an obviously-safe path existing, because the FRS is overly conservative.}
% \label{fig:example_overly_conservative}
% \end{figure}


%\section{Applications}\label{sec:applications}

In this section, we present applications of our results to two SDPs: Max-Cut and matrix completion, both of which are important problems in the learning domain and have been studied extensively. Interest has grown to develop efficient solvers for these SDPs~\citep{arora2007combinatorial, pmlr-v65-mei17a, hardt2013understanding, bandeira2016low}.
%General SDP solvers such as interior point or ellipsoid method can lead to slow algorithms for these problems. Instead, over the years, specialized efficient algorithms have been developed for these problems \citep{arora2007combinatorial, hardt2013understanding}. 

This work differs from previous efforts in at least two ways. First, we aim to demonstrate that Burer--Monteiro-style approaches, which are often used in practice, can indeed lead to provably efficient algorithms for general SDPs. We believe that building upon this work, it should be possible to improve the time-complexity guarantees of such factorization-based algorithms. Second, we note that several problems formulated as SDPs in fact necessitate low-rank solutions, for example because of memory concerns (as is the case in matrix completion),  and factorization approaches provide a natural means to control rank. % For such problem, existing methods do not provide low rank solutions while our method is guaranteed to give a low-rank solution. 

%We note that both of these problems have been extensively studied and for both of them there exist highly specialized algorithms that are highly efficient~\cite{arora2007combinatorial,hardt2013understanding}. Our results here do not beat them; rather our goal here is to demonstrate that the Burer-Monteiro approach can successfully solve these SDPs in polynomial time. In practice, this approach is much faster than other generic SDP solvers such as interior point method and ellipsoid method, and in addition returns low rank solutions.
\subsection{Max-Cut}

We first consider the popular Max-Cut problem which finds applications in clustering related problems. In a seminal paper, \cite{goemans1995improved} defined the following SDP to solve the Max-Cut problem: $\min_{X\in \Rnn} \ip{C}{X}, \mbox{s.t. } X_{ii} = 1 \; \forall \; 1 \leq i \leq n, X \succeq 0 $, where $n$ is the number of vertices in the given graph and $C$ is its adjacency matrix. Since the constraint set also satisfies $\trace{X}=n$, we consider the following penalized, non-convex version of the problem.
% \begin{align*}	& \min_{X\in \Rnn} \ip{C}{X} \\
%	& \mbox{s.t. } X_{ii} = 1 \; \forall \; 1 \leq i \leq n \\
%	& \qquad X \succeq 0,
%\end{align*}
\begin{align}
	\widehat{L}_{\mu}(U) \defeq \ip{C+G}{U\trans{U}} + \mu\left(\left(\ip{I}{U\trans{U}}-n\right)^2 + \sum_{i=1}^{n}\left(\ip{e_i \trans{e_i}}{U\trans{U}}-1\right)^2\right),\label{eqn:maxcut}
\end{align}
where $G$ is a random symmetric Gaussian matrix.  Let $\widehat F_{\mu}(UU^T) = \widehat L_{\mu} (U)$. After some simplifying computations, we have the following corollary of Theorem~\ref{thm:optimal_approx_compact}.
\begin{corollary}\label{cor:maxcut}
There exists an absolute numerical constant $c_1$ such that the following holds. With probability greater than $1-\delta$,
every $(\eps, \gamma)$-SOSP $U$ of the perturbed Max-Cut problem $\widehat{L}_{\mu}(U)$~\eqref{eqn:maxcut} with:
\begin{align*}\epsilon \leq \frac{1}{c_1} \left(\frac{\gamma \sigma_G^2}{\mu n}\right)^{2/3},~~ \text{ and } ~~  k = \tilde{\Omega} \left( \sqrt{n \log\left(\frac{\mu^2 \sqrt{n}}{\sigma_G}\right)}\right),
\end{align*}
satisfies $	\widehat{F}_{\mu}(UU^T) - \widehat{F}_{\mu}(X^*) \leq \gamma \sqrt{\epsilon} \trace{X^*} +\frac{1}{2} \epsilon \frob{U}$, where $X^*$ is a global optimum of $\widehat{F}_{\mu}(X)$.
%\begin{align*}
%	\widehat{L}_{\mu}(U) - \widehat{L}_{\mu}(U^*) \leq \gamma \sqrt{\epsilon} \trace{U^* \trans{U^*}} + \epsilon \frob{U}.
%\end{align*}
\end{corollary}
The above result states that for the penalized version of the perturbed Max-Cut SDP, the Burer--Monteiro approach finds an approximate global optimum as soon as the factorization rank $k = \tilde{\Omega}(\sqrt{n})$. Existing results for Max-Cut using this approach either only handle exact SOSPs~\citep{boumal2016non}, or require $k=n+1$~\citep{boumal2016globalrates}, or require $k$ that is dependent on $\frac{1}{\eps}$~\citep{pmlr-v65-mei17a}. Moreover, complexity per iteration scales only linearly with the number of edges in the graph. %However, current analysis is required to set $\epsilon$ to be fairly small which can lead to a super-linear algorithm; we leave further tightening of dependence on $\epsilon$ for future work.


\subsection{Matrix Completion}
In this section we specialize our results for the matrix completion problem \cite{candes2009exact}. The goal of a matrix completion problem is to find a low-rank matrix $M$ using only a small number of its entries, with applications in recommender systems. To ensure that the computed matrix is low-rank and generalizes well, one typically imposes nuclear-norm regularization which leads to the following SDP: 

\begin{minipage}{0.2\linewidth}
	\begin{align*}
	\min &\quad \trace{W_1} + \trace{W_2}\\ \text{s. t. }&\quad X_{ij} =M_{ij}, (i,j) \in \calS \\  &\quad \begin{bmatrix}W_1 & X \\ X^T & W_2\end{bmatrix} \succeq 0.
	\end{align*}
\end{minipage}
\begin{minipage}{0.05\linewidth}
	\begin{align*}
		\equiv \\
	\end{align*} \break
\end{minipage}
\begin{minipage}{0.6\linewidth}
	\begin{align*}
	\min & \quad \ip{I}{Z} \nonumber \\ \text{s. t. }&\quad \frac{1}{2}\ip{e_{i+n}e_{j+n}^T + e_{j+n} e_{i+n}^T}{Z} = M_{ij}, (i,j) \in \calS \nonumber \\  &\quad Z \succeq 0.
	\end{align*}
\end{minipage}
%\begin{align*} \min &\quad \trace{W_1} + \trace{W_2}\\ \text{s. t. }&\quad X_{ij} =M_{ij}, (i,j) \in \Omega \\  &\quad \begin{bmatrix}W_1 & X \\ X^T & W_2\end{bmatrix} \succeq 0.
%\end{align*} 
\noindent Here $\calS$ is the set of observed indices of $M$ and $Z\defeq \begin{bmatrix}W_1 & X \\ X^T & W_2\end{bmatrix}$. % by $Z$, we can rewrite the above SDP as \begin{align} \min & \quad \ip{Z}{I} \nonumber \\ \text{s. t. }&\quad \frac{1}{2}\ip{e_{i+n}e_{j+n}^T + e_{j+n} e_{i+n}^T}{Z} = M_{ij}, (i,j) \in \Omega \nonumber \\  &\quad Z \su,cceq 0. \label{eq:mc_sdp}\end{align}
Let
\begin{align}
	\widehat{L}_{\mu}(U) = \ip{I+G}{UU^T} + \mu \sum_{i=1}^m \left(\frac{1}{2}\ip{e_{i+n}e_{j+n}^T + e_{j+n} e_{i+n}^T}{UU^T} - M_{ij} \right)^2  \label{eq:matcomp}
\end{align}
be the corresponding penalty objective.  Let $\widehat F_{\mu}(UU^T) = \widehat L_{\mu} (U)$. The objective is positive definite with $\lambda_1(C)=\lambda_n(C)=1$. Also, since $\calA$ is a sub-sampling operator, $\|\calA\| \leq 1$. Finally, for $\eps^2 \leq \frac{\mu}{2}\sqrt{\sum_{(i,j) \in \calS} M_{ij}^2}$, the residues are bounded by: \begin{align*} B&=\|\calA\| \max \left \{ \left( \frac{2\eps} {\lambda_n(C)}\right)^2, \frac{2\mu}{\lambda_n(C)} \|\vb\|_2^2 \right \}+\|\vb\|_2 \leq \max  3\mu \sqrt{\sum_{(i,j) \in \calS} M_{ij}^2}. \end{align*}


\noindent Applying Theorem~\ref{thm:optimal_approx} for this setting gives the following corollary.
\begin{corollary}\label{cor:mc_optimal}There exists an absolute numerical constant $c_2$ such that the following holds. With probability greater than $1-\delta$,
every $(\eps, \gamma)$-SOSP $U$ of the perturbed matrix completion problem $\widehat{L}_{\mu}(U)$~\eqref{eq:matcomp} with:
\begin{align*}\sG \leq \frac{1}{4\sqrt{n \log(n/ \delta)}},~~ \eps \leq \frac{1}{c_2}\left(\frac{\gamma \abs{\calS} \sigma_G^2 }{ n  \mu }\right)^{\sfrac{2}{3}}, ~\text{ and }~ k = \tilde{\Omega} \left( \sqrt{ \abs{\calS}   \log\left(\frac{\mu^2 \sqrt{n} \sqrt{\sum_{(i,j) \in \calS} M_{ij}^2}}{\sigma_G}\right) } \right),\end{align*} satisfies $\widehat{F}_{\mu}(UU^T)  - \widehat F_{\mu}(X^*)  \leq \gamma \sqrt{\epsilon} \trace{X^*} + \frac{1}{2} \eps \|U\|_F$, where $X^*$ is a global optimum of $\widehat{F}_{\mu}(X)$.
\end{corollary}
\noindent This result shows that for the  matrix completion problem with $m$ observations, for rank $\tilde{\Omega}(\sqrt{m})$, any approximate local minimum of the factorized and penalized problem is an approximate global minimum. 

Most of the existing results on matrix completion either require strong distribution assumptions on $\calS$ and incoherence assumptions on $M$ to recover a low-rank solution \citep{candes2009exact, jain2013low}. The standard nuclear norm minimization algorithms are not guaranteed to converge to low-rank solutions without these assumptions,  which implies that the entire matrix would need to be stored for prediction which is infeasible in practice. Similarly,  generalization error bounds \citep{foygel2011concentration} as well as differential privacy guarantees  depend on recovery of a low-rank solution.

Our result guarantees finding a rank -$\tilde{\Omega}(\sqrt{m})$ solution without any statistical assumptions on the sampling or the matrix. The tradeoff is our results do not guarantee finding a lower (potentially a constant) rank solution, even if one exists for a given problem. 


%\subsection{Normalized Cut}
%
%In this section we will consider the problem of computing the Normalized cut of a given graph $\calG$ \citep{shi2000normalized}. Given a graph $G$ with $n$ vertices and $e$ edges, the normalized cut is the partition of vertices into two sets $S$ and $S^c$ that minimizes, $$\frac{cut(S,S^c)}{D(S)}+ \frac{cut(S,S^c)}{D(S^c)}. $$ Here $cut(S,S^c)$ is the number of edges between $S$ and $S^c$. $D(S)$ is the sum of degree of vertices in $S$. This problem is NP-hard in the worst case. 
%
%Let $\widehat{X}$ be a $n+1 \times n+1$ matrix, with the following structure, $\widehat{X} = \begin{bmatrix}  X & 0_{n \times 1} \\ 0_{1 \times n} & x \end{bmatrix}$. \citet{bie2006fast} proposed the following SDP relaxation to find the normalized cut.
%
% \begin{align*} \underset{\widehat{X}}{\minimize} &\quad \ip{\widehat{X}}{L}\\ \text{subject to } &\quad \widehat{X} \succeq 0 \\
%&\quad \ip{\widehat{X}}{A_i} =0, i \in [n]\\
%&\quad \ip{\widehat{X}}{A_{n+1}}=-1 \\
%\end{align*}
%Here $L$ is the graph Laplacian divided by twice the number of edges, $2e$. $A_i$ ($i \in [n]$) is a matrix with $1$ at $(i,i)$ and $-1$ at $(n+1,n+1)$ entry, with rest of the entries begin $0$s. $\displaystyle A_{n+1} =\begin{bmatrix} dd^T/(4e^2) & 0_{n \times 1} \\0_{1 \times n} & -1 \end{bmatrix}$, where $d$ is the vector of degrees of the vertices.
%
%
%




%
%\providecommand{\href}[2]{#2}\begingroup\raggedright\begin{thebibliography}{10}
%
%\bibitem{Mal}
%J.~Maldacena, \emph{The large {$N$} limit of superconformal field theories and
%  supergravity [ {MR}1633016 (99e:81204a)]},  in \emph{Trends in theoretical
%  physics, {II} ({B}uenos {A}ires, 1998)}, vol.~484 of \emph{AIP Conf. Proc.},
%  pp.~51--63.
%\newblock Amer. Inst. Phys., Woodbury, NY, 1999.
%
%\bibitem{GpScqg05}
%A.~Garc{\'{\i}}a-Parrado and J.~M.~M. Senovilla, \emph{Causal structures and
%  causal boundaries},
%  \href{http://dx.doi.org/10.1088/0264-9381/22/9/R01}{\emph{Classical Quantum
%  Gravity} {\bf 22} (2005) R1--R84}.
%
%\bibitem{H3}
%S.~G. Harris, \emph{Boundaries on spacetimes: causality, topology, and group
%  actions}, \href{http://dx.doi.org/10.1007/s10711-007-9168-2}{\emph{Geom.
%  Dedicata} {\bf 126} (2007) 255--274}.
%
%\bibitem{S}
%M.~S{\'a}nchez, \emph{Causal boundaries and holography on wave type
%  spacetimes},
%  \href{http://dx.doi.org/10.1016/j.na.2009.02.101}{\emph{Nonlinear Anal.} {\bf
%  71} (2009) e1744--e1764}.
%
%\bibitem{FHSFinalDef}
%J.~L. Flores, J.~Herrera and M.~S{\'a}nchez, \emph{On the final definition of
%  the causal boundary and its relation with the conformal boundary},
%  {\emph{Adv. Theor. Math. Phys.} {\bf 15} (2011) 991--1057}.
%
%\bibitem{FHSHaus}
%J.~L. Flores, J.~Herrera and M.~S{\'a}nchez, \emph{Hausdorff separability of
%  the boundaries for spacetimes and sequential spaces},
%  \href{http://dx.doi.org/10.1063/1.4939485}{\emph{J. Math. Phys.} {\bf 57}
%  (2016) 022503, 25}.
%
%\bibitem{Chrusciel}
%P.~T. Chru{\'s}ciel, \emph{Conformal boundary extensions of {L}orentzian
%  manifolds}, {\emph{J. Differential Geom.} {\bf 84} (2010) 19--44}.
%
%\bibitem{BMN}
%D.~Berenstein, J.~Maldacena and H.~Nastase, \emph{Strings in flat space and pp
%  waves from {$\scr N=4$} super {Y}ang {M}ills},
%  \href{http://dx.doi.org/10.1088/1126-6708/2002/04/013}{\emph{J. High Energy
%  Phys.} (2002) No. 13, 30}.
%
%\bibitem{MR1}
%D.~Marolf and S.~F. Ross, \emph{Plane waves: to infinity and beyond!},
%  \href{http://dx.doi.org/10.1088/0264-9381/19/24/302}{\emph{Classical Quantum
%  Gravity} {\bf 19} (2002) 6289--6302}.
%
%\bibitem{MR}
%D.~Marolf and S.~F. Ross, \emph{A new recipe for causal completions},
%  \href{http://dx.doi.org/10.1088/0264-9381/20/18/314}{\emph{Classical Quantum
%  Gravity} {\bf 20} (2003) 4085--4117}.
%
%\bibitem{FS2}
%J.~L. Flores and M.~S{\'a}nchez, \emph{The causal boundary of wave-type
%  spacetimes}, \href{http://dx.doi.org/10.1088/1126-6708/2008/03/036}{\emph{J.
%  High Energy Phys.} (2008) 036, 43}.
%
%\bibitem{Me}
%G.~Mess, \emph{Lorentz spacetimes of constant curvature},
%  \href{http://dx.doi.org/10.1007/s10711-007-9155-7}{\emph{Geometriae Dedicata}
%  {\bf 126} (2007) 3--45}.
%
%\bibitem{BTZ}
%M.~Ba\~nados, C.~Teitelboim and J.~Zanelli, \emph{The black hole in
%  three-dimensional space-time},
%  \href{http://dx.doi.org/10.1103/PhysRevLett.69.1849}{\emph{Phys. Rev. Lett.}
%  {\bf 69} (1992) 1849--1851}.
%
%\bibitem{HP}
%S.~W. Hawking and D.~N. Page, \emph{Thermodynamics of black holes in anti-de
%  {S}itter space}, {\emph{Comm. Math. Phys.} {\bf 87} (1982/83) 577--588}.
%
%\bibitem{BHTZ}
%M.~Ba\~nados, M.~Henneaux, C.~Teitelboim and J.~Zanelli, \emph{Geometry of the
%  $2+1$ blackhole},
%  \href{http://dx.doi.org/10.1103/PhysRevD.48.1506}{\emph{Phys. Rev. D.} {\bf
%  48} (1993) 1506--1525}.
%
%\bibitem{Wit1}
%E.~Witten, \emph{Anti-de {S}itter space, thermal phase transition and
%  confinement in gauge theories},
%  \href{http://dx.doi.org/10.1142/S0217751X01004451}{\emph{Internat. J. Modern
%  Phys. A} {\bf 16} (2001) 2747--2769}.
%
%\bibitem{Wit2}
%E.~Witten, \emph{Anti-de {S}itter space, thermal phase transition and
%  confinement in gauge theories},
%  \href{http://dx.doi.org/10.1142/S0217751X01004451}{\emph{Internat. J. Modern
%  Phys. A} {\bf 16} (2001) 2747--2769}.
%
%\bibitem{AF}
%V.~Ala{\~n}a and J.~L. Flores, \emph{The causal boundary of product
%  spacetimes}, \href{http://dx.doi.org/10.1007/s10714-007-0492-5}{\emph{Gen.
%  Relativity Gravitation} {\bf 39} (2007) 1697--1718}.
%
%\bibitem{Got}
%J.~R. Gott, III, \emph{Closed timelike curves produced by pairs of moving
%  cosmic strings: exact solutions},
%  \href{http://dx.doi.org/10.1103/PhysRevLett.66.1126}{\emph{Phys. Rev. Lett.}
%  {\bf 66} (1991) 1126--1129}.
%
%\bibitem{H}
%S.~G. Harris, \emph{Discrete group actions on spacetimes: causality conditions
%  and the causal boundary},
%  \href{http://dx.doi.org/10.1088/0264-9381/21/4/032}{\emph{Classical Quantum
%  Gravity} {\bf 21} (2004) 1209--1236}.
%
%\bibitem{FHSIso2}
%J.~Flores, J.~Herrera and M.~S{\'a}nchez, \emph{Computability of the causal
%  boundary by using isocausality}, {\emph{Classical and Quantum Gravity} {\bf
%  30} (2013) 075009}.
%
%\bibitem{GKP}
%R.~Geroch, E.~H. Kronheimer and R.~Penrose, \emph{Ideal points in space-time},
%  \href{http://dx.doi.org/10.1098/rspa.1972.0062}{\emph{Proceedings of the
%  Royal Society A: Mathematical, Physical and Engineering Sciences} {\bf 327}
%  (apr, 1972) 545--567}.
%
%\bibitem{BS}
%R.~Budic and R.~K. Sachs, \emph{Causal boundaries for general relativistic
%  space times},
%  \href{http://dx.doi.org/http://dx.doi.org/10.1063/1.1666812}{\emph{Journal of
%  Mathematical Physics} {\bf 15} (1974) 1302--1309}.
%
%\bibitem{H1}
%S.~G. Harris, \emph{Universality of the future chronological boundary},
%  \href{http://dx.doi.org/10.1063/1.532582}{\emph{J. Math. Phys.} {\bf 39}
%  (1998) 5427--5445}.
%
%\bibitem{H2}
%S.~G. Harris, \emph{Topology of the future chronological boundary: universality
%  for spacelike boundaries},
%  \href{http://dx.doi.org/10.1088/0264-9381/17/3/303}{\emph{Classical Quantum
%  Gravity} {\bf 17} (2000) 551--603}.
%
%\bibitem{Ra}
%I.~R{\'a}cz, \emph{Causal boundary of space-times},
%  \href{http://dx.doi.org/10.1103/PhysRevD.36.1673}{\emph{Phys. Rev. D (3)}
%  {\bf 36} (1987) 1673--1675}.
%
%\bibitem{Sz}
%L.~B. Szabados, \emph{Causal boundary for strongly causal spacetimes},
%  {\emph{Classical and Quantum Gravity} {\bf 5} (1988) 121}.
%
%\bibitem{Sz2}
%L.~B. Szabados, \emph{Causal boundary for strongly causal spacetimes. {II}},
%  {\emph{Classical Quantum Gravity} {\bf 6} (1989) 77--91}.
%
%\bibitem{Wald}
%R.~Wald, \emph{General Relativity}.
%\newblock University of Chicago Press, 1984.
%
%\bibitem{Flores}
%J.~L. Flores, \emph{The causal boundary of spacetimes revisited},
%  \href{http://dx.doi.org/10.1007/s00220-007-0345-9}{\emph{Comm. Math. Phys.}
%  {\bf 276} (2007) 611--643}.
%
%\bibitem{MS}
%E.~Minguzzi and M.~S{\'a}nchez, \emph{The causal hierarchy of spacetimes},  in
%  \emph{Recent developments in pseudo-{R}iemannian geometry}, ESI Lect. Math.
%  Phys., pp.~299--358.
%\newblock Eur. Math. Soc., Z\"urich, 2008.
%\newblock \href{http://dx.doi.org/10.4171/051-1/9}{DOI}.
%
%\bibitem{FHSBuseman}
%J.~L. Flores, J.~Herrera and M.~S{\'a}nchez, \emph{Gromov, {C}auchy and causal
%  boundaries for {R}iemannian, {F}inslerian and {L}orentzian manifolds},
%  \href{http://dx.doi.org/10.1090/S0065-9266-2013-00680-6}{\emph{Mem. Amer.
%  Math. Soc.} {\bf 226} (2013) vi+76}.
%
%\end{thebibliography}
%\endgroup
%
%%\bibliography{biblio2}
%%\bibliographystyle{JHEP}
%
%%\input{referencias}

\end{document}

%%% Local Variables:
%%% mode: latex
%%% TeX-master: t
%%% End:
