\NeedsTeXFormat{LaTeX2e}
%\documentclass[a4paper,10pt]{scrartcl}
%\documentclass[a4paper,10pt]{scrartcl}
%\documentclass[oneside,draft,letterpaper,12pt]{amsart}
\documentclass[a4paper,reqno,11pt]{article}
%\usepackage{jheppub}
\usepackage[a4paper,bindingoffset=0.2in,%
left=1in,right=1in,top=1in,bottom=1in,%
           footskip=.25in]{geometry}
\usepackage{graphicx}
\renewcommand{\sc}{\scshape}
\setlength{\textwidth}{\paperwidth}
% \usepackage{cite}
\usepackage{authblk}
\addtolength{\textwidth}{-2.5in}
%\calclayout
\usepackage{amssymb}
\usepackage{amsthm}
\usepackage{amsmath}
% \usepackage[english]{babel}
\usepackage{color}
\usepackage[utf8]{inputenc}
%\usepackage{refcheck}
\usepackage{tikz}
\usepackage[all]{xy}
\usepackage[bookmarksnumbered,colorlinks]{hyperref}
\usepackage{dsfont}
\usepackage[normalem]{ulem}

\usepackage{float} % % Figuras


\usepackage[euler-digits]{eulervm}
\usepackage{enumerate}
\def\br#1\er{\textcolor{red}{#1}} %
\hyphenation{Lo-ren-tzian}
\def\soutE#1{\sout{#1}} %to see the overlines
\providecommand{\bold}{\mathbf}
\providecommand{\scr}{\mathcal}

\newcommand{\eps}{\varepsilon}
\newcommand{\wk}{\rightharpoonup}
\newcommand{\then}{\Longrightarrow}
\newcommand{\g}{\langle\cdot,\cdot\rangle }
\newcommand{\m}{{\cal M}}
\newcommand{\mo}{{\cal M}_0}
\newcommand{\I}{{\cal G}}
\newcommand{\cat}{{\mathrm cat}}
\newcommand{\arcsinh}{{\mathrm arcsinh}}
\newcommand{\dimo}{{\bf Proof.\ }}
\newcommand{\cvd}{\ \rule{0.5em}{0.5em}}
\newcommand{\be}{\begin{equation}}
\newcommand{\ee}{\end{equation}}
\newcommand{\<}{\langle}
\renewcommand{\>}{\rangle}

\newcommand{\N}{{\mathbb N}}
\newcommand{\Z}{{\mathbb Z}}
\newcommand{\R}{{\mathbb R}}
\newcommand{\LL}{{\mathbb L}}


\newcommand{\M}{{\cal M}}
\newcommand{\V}{{\cal V}}
\newcommand{\C}{\mathfrak{c}}
\newcommand{\J}{{\cal J}}

\newcommand{\noi}{\noindent}
\newcommand{\ben}{\begin{enumerate}}
\newcommand{\een}{\end{enumerate}}
\newcommand{\bit}{\begin{itemize}}
\newcommand{\eit}{\end{itemize}}
\newcommand{\edoc}{\end{document}}
\newcommand{\sm}{\smallskip}
\newcommand{\np}{\newpage}

\newcommand{\vs}{\vspace}
\newcommand{\onto}{{\rightarrow}}
%\newcommand{\R}{\mathds R}
%\newcommand{\N}{\mathds N}
%\newcommand{\Z}{\mathds Z}
\newcommand{\Ver}{\mathcal V}
\newcommand{\Hor}{\mathcal H}
\newcommand{\Ddt}{\frac{\mathrm D}{\mathrm dt}}
\newcommand{\germ}{\mathrm{germ}}
\newcommand{\relpri}{\sim_{G_0}}
%\newcommand{\cambios}{\color{red}}
\newcommand{\lev}[2]{\overline{#1}_{#2}}
\newcommand{\opciones}[2]{#2}
\newcommand{\lcrono}{L_{chr}}
\newcommand{\lcronof}{\hat{L}_{chr}}
\newcommand{\tcrono}{\tau_{chr}}
\newcommand{\cambios}[1]{{\color{black} #1}}
\newcommand{\ncambios}[1]{{\color{black} #1}}

\newcommand{\multiwarped}{doubly warped~}
\newcommand{\timelike}{timelike}
\newcommand{\lightlike}{lightlike}
\newcommand{\spatial}{spatial}
\newcommand{\unique}{{univocally determined}}
\newcommand{\Integral}[5]{\int_{#1}^{#2} \frac{\sqrt{#3}}{\alpha_{#4}(s)}\left(\sum_{k=1}^{2} \frac{#5}{\alpha_{k}(s)} \right)^{-1/2}ds}
% Para hacer el cambio al multiwarped
\newcommand{\Integralm}[5]{\int_{#1}^{#2} \frac{\sqrt{#3}}{\alpha_{#4}}\left(\sum_{k=1}^{n+1} \frac{#5}{\alpha_{k}} \right)^{-1/2}ds}

\newcommand{\point}[3]{(#1,#2,#3)}
\newcommand{\B}{b}
%\frac{#3}{\alpha_{1}}+\frac{#5}{\alpha_{2}}   \sum_{k} \frac{#5}{\alpha_{k}}

%\title{Spacetimes coverings and C-boundary}
%\author[L. A. Ak\'e, J. Herrera]{Luis Alberto Ak\'e \and J\'onatan Herrera}
%\address{Universidade Federal de Santa Catarina, Brazil}
%\address{Universidad de M\'alaga, Spain}
%\date{5 May 2016}
\title{Causality and c-completion of multiwarped spacetimes}
%\author[L. A. Ak\'e, J. Herrera]{Luis Alberto Ak\'e \and J\'onatan Herrera}
%\address{Universidade Federal de Santa Catarina, Brazil}
%\address{Universidad de M\'alaga, Spain}
%\date{5 May 2016}
\date{}
\author[1]{Luis Alberto Aké}
\author[1]{José Luis Flores}
\author[2]{Jónatan Herrera}
\affil[1]{\small Departamento de Álgebra, Geometría y Topología, Facultad de Ciencias\\ Universidad de Málaga, Campus Teatinos, 29071 Málaga, Spain}
\affil[2]{\small Departamento de Matemáticas, Edificio Albert Einstein\\ Universidad de Córdoba, Campus de Rabanales, 14071 Córdoba, Spain}
\affil[ ]{\textit{Corresponding author: jonatanhf@gmail.com}}


% \author[L. Ak\'e]{Luis Alberto Ak\'e}
% \address{Departamento de \'Algebra, Geometr\'{\i}a y Topolog\'{\i}a,  Universidad de M\'alaga
% \hfill\break\indent
% \hfill\break\indent
% Facultad de Ciencias, Campus Universitario de Teatinos,
% \hfill\break\indent 29080 M\'alaga, Spain}
% \email{luisake@uma.es}

% % % % Information Jonatan

%\author[J. Herrera]{J\'onatan Herrera}
%\address{Department of Mathematics,
%	Universidade Federal de Santa Catarina,
%	\hfill\break\indent 88.040-900 Florian\'{o}polis-SC, Brazil.}
%\email{jonatanhf@gmail.es}

%\abstract{We consider the relation between the c-completion of a Lorentz manifold $V$ and its quotient $M=V/G$, where $G$ is an isometry group acting freely and properly discontinuously.
%  First, we consider the future causal completion case, characterizing virtually when such a quotient is well behaved with the future chronological topology and improving the existing results on the literature.  Secondly, we show that under some general assumptions, there exists a homeomorphism and chronological isomorphism between both, the c-completion of $M$ and some adequate quotient of the c-completion of $V$ defined by $G$. Our results are optimal, as we show in several examples. Finally, we give a practical application by considering isometric actions over Robertson-Walker spacetimes, including in particular the Anti-de Sitter model.}
%\arxivnumber{1605.03128}

\begin{document}
\newtheorem{thm}{Theorem}[section]
\newtheorem{prop}[thm]{Proposition}
\newtheorem{lemma}[thm]{Lemma}
\newtheorem{cor}[thm]{Corollary}
\newtheorem{conv}[thm]{Convention}
\theoremstyle{definition}
\newtheorem{defi}[thm]{Definition}
\newtheorem{notation}[thm]{Notation}
\newtheorem{exe}[thm]{Example}
\newtheorem{conj}[thm]{Conjecture}
\newtheorem{prob}[thm]{Problem}
\newtheorem{rem}[thm]{Remark}


%\theoremstylx
\maketitle
\usetikzlibrary{matrix}



\begin{abstract}
In this paper a systematic study of the causal structure and global causality properties of multiwarped spacetimes is developed. This analysis is used to make a detailed description of the causal boundary of these spacetimes. Some applications of our results in examples of physical interest, for instance, in the context of Maldacena's conjecture, are considered.
%
%A discussion about the implications of these results to the
%
%The aim of this paper is twofold: first, it will provide a general framework for the study of causal structures in multiwarped models $(\mathcal{V},\mathfrak{g})$; then, we will describe in full detail the causal boundary of $(\mathcal{V},\mathfrak{g})$ consider some mild hypothesis on the integrals of the warping functions. We will present some relevant examples where our results are applicable.
\end{abstract}

%  \maketitle
%  \flushbottom

\tableofcontents

\IEEEraisesectionheading{\section{Introduction}}

\IEEEPARstart{V}{ision} system is studied in orthogonal disciplines spanning from neurophysiology and psychophysics to computer science all with uniform objective: understand the vision system and develop it into an integrated theory of vision. In general, vision or visual perception is the ability of information acquisition from environment, and it's interpretation. According to Gestalt theory, visual elements are perceived as patterns of wholes rather than the sum of constituent parts~\cite{koffka2013principles}. The Gestalt theory through \textit{emergence}, \textit{invariance}, \textit{multistability}, and \textit{reification} properties (aka Gestalt principles), describes how vision recognizes an object as a \textit{whole} from constituent parts. There is an increasing interested to model the cognitive aptitude of visual perception; however, the process is challenging. In the following, a challenge (as an example) per object and motion perception is discussed. 



\subsection{Why do things look as they do?}
In addition to Gestalt principles, an object is characterized with its spatial parameters and material properties. Despite of the novel approaches proposed for material recognition (e.g.,~\cite{sharan2013recognizing}), objects tend to get the attention. Leveraging on an object's spatial properties, material, illumination, and background; the mapping from real world 3D patterns (distal stimulus) to 2D patterns onto retina (proximal stimulus) is many-to-one non-uniquely-invertible mapping~\cite{dicarlo2007untangling,horn1986robot}. There have been novel biology-driven studies for constructing computational models to emulate anatomy and physiology of the brain for real world object recognition (e.g.,~\cite{lowe2004distinctive,serre2007robust,zhang2006svm}), and some studies lead to impressive accuracy. For instance, testing such computational models on gold standard controlled shape sets such as Caltech101 and Caltech256, some methods resulted $<$60\% true-positives~\cite{zhang2006svm,lazebnik2006beyond,mutch2006multiclass,wang2006using}. However, Pinto et al.~\cite{pinto2008real} raised a caution against the pervasiveness of such shape sets by highlighting the unsystematic variations in objects features such as spatial aspects, both between and within object categories. For instance, using a V1-like model (a neuroscientist's null model) with two categories of systematically variant objects, a rapid derogate of performance to 50\% (chance level) is observed~\cite{zhang2006svm}. This observation accentuates the challenges that the infinite number of 2D shapes casted on retina from 3D objects introduces to object recognition. 

Material recognition of an object requires in-depth features to be determined. A mineralogist may describe the luster (i.e., optical quality of the surface) with a vocabulary like greasy, pearly, vitreous, resinous or submetallic; he may describe rocks and minerals with their typical forms such as acicular, dendritic, porous, nodular, or oolitic. We perceive materials from early age even though many of us lack such a rich visual vocabulary as formalized as the mineralogists~\cite{adelson2001seeing}. However, methodizing material perception can be far from trivial. For instance, consider a chrome sphere with every pixel having a correspondence in the environment; hence, the material of the sphere is hidden and shall be inferred implicitly~\cite{shafer2000color,adelson2001seeing}. Therefore, considering object material, object recognition requires surface reflectance, various light sources, and observer's point-of-view to be taken into consideration.


\subsection{What went where?}
Motion is an important aspect in interpreting the interaction with subjects, making the visual perception of movement a critical cognitive ability that helps us with complex tasks such as discriminating moving objects from background, or depth perception by motion parallax. Cognitive susceptibility enables the inference of 2D/3D motion from a sequence of 2D shapes (e.g., movies~\cite{niyogi1994analyzing,little1998recognizing,hayfron2003automatic}), or from a single image frame (e.g., the pose of an athlete runner~\cite{wang2013learning,ramanan2006learning}). However, its challenging to model the susceptibility because of many-to-one relation between distal and proximal stimulus, which makes the local measurements of proximal stimulus inadequate to reason the proper global interpretation. One of the various challenges is called \textit{motion correspondence problem}~\cite{attneave1974apparent,ullman1979interpretation,ramachandran1986perception,dawson1991and}, which refers to recognition of any individual component of proximal stimulus in frame-1 and another component in frame-2 as constituting different glimpses of the same moving component. If one-to-one mapping is intended, $n!$ correspondence matches between $n$ components of two frames exist, which is increased to $2^n$  for one-to-any mappings. To address the challenge, Ullman~\cite{ullman1979interpretation} proposed a method based on nearest neighbor principle, and Dawson~\cite{dawson1991and} introduced an auto associative network model. Dawson's network model~\cite{dawson1991and} iteratively modifies the activation pattern of local measurements to achieve a stable global interpretation. In general, his model applies three constraints as it follows
\begin{inlinelist}
	\item \textit{nearest neighbor principle} (shorter motion correspondence matches are assigned lower costs)
	\item \textit{relative velocity principle} (differences between two motion correspondence matches)
	\item \textit{element integrity principle} (physical coherence of surfaces)
\end{inlinelist}.
According to experimental evaluations (e.g.,~\cite{ullman1979interpretation,ramachandran1986perception,cutting1982minimum}), these three constraints are the aspects of how human visual system solves the motion correspondence problem. Eom et al.~\cite{eom2012heuristic} tackled the motion correspondence problem by considering the relative velocity and the element integrity principles. They studied one-to-any mapping between elements of corresponding fuzzy clusters of two consecutive frames. They have obtained a ranked list of all possible mappings by performing a state-space search. 



\subsection{How a stimuli is recognized in the environment?}

Human subjects are often able to recognize a 3D object from its 2D projections in different orientations~\cite{bartoshuk1960mental}. A common hypothesis for this \textit{spatial ability} is that, an object is represented in memory in its canonical orientation, and a \textit{mental rotation} transformation is applied on the input image, and the transformed image is compared with the object in its canonical orientation~\cite{bartoshuk1960mental}. The time to determine whether two projections portray the same 3D object
\begin{inlinelist}
	\item increase linearly with respect to the angular disparity~\cite{bartoshuk1960mental,cooperau1973time,cooper1976demonstration}
	\item is independent from the complexity of the 3D object~\cite{cooper1973chronometric}
\end{inlinelist}.
Shepard and Metzler~\cite{shepard1971mental} interpreted this finding as it follows: \textit{human subjects mentally rotate one portray at a constant speed until it is aligned with the other portray.}



\subsection{State of the Art}

The linear mapping transformation determination between two objects is generalized as determining optimal linear transformation matrix for a set of observed vectors, which is first proposed by Grace Wahba in 1965~\cite{wahba1965least} as it follows. 
\textit{Given two sets of $n$ points $\{v_1, v_2, \dots v_n\}$, and $\{v_1^*, v_2^* \dots v_n^*\}$, where $n \geq 2$, find the rotation matrix $M$ (i.e., the orthogonal matrix with determinant +1) which brings the first set into the best least squares coincidence with the second. That is, find $M$ matrix which minimizes}
\begin{equation}
	\sum_{j=1}^{n} \vert v_j^* - Mv_j \vert^2
\end{equation}

Multiple solutions for the \textit{Wahba's problem} have been published, such as Paul Davenport's q-method. Some notable algorithms after Davenport's q-method were published; of that QUaternion ESTimator (QU\-EST)~\cite{shuster2012three}, Fast Optimal Attitude Matrix \-(FOAM)~\cite{markley1993attitude} and Slower Optimal Matrix Algorithm (SOMA)~\cite{markley1993attitude}, and singular value decomposition (SVD) based algorithms, such as Markley’s SVD-based method~\cite{markley1988attitude}. 

In statistical shape analysis, the linear mapping transformation determination challenge is studied as Procrustes problem. Procrustes analysis finds a transformation matrix that maps two input shapes closest possible on each other. Solutions for Procrustes problem are reviewed in~\cite{gower2004procrustes,viklands2006algorithms}. For orthogonal Procrustes problem, Wolfgang Kabsch proposed a SVD-based method~\cite{kabsch1976solution} by minimizing the root mean squared deviation of two input sets when the determinant of rotation matrix is $1$. In addition to Kabsch’s partial Procrustes superimposition (covers translation and rotation), other full Procrustes superimpositions (covers translation, uniform scaling, rotation/reflection) have been proposed~\cite{gower2004procrustes,viklands2006algorithms}. The determination of optimal linear mapping transformation matrix using different approaches of Procrustes analysis has wide range of applications, spanning from forging human hand mimics in anthropomorphic robotic hand~\cite{xu2012design}, to the assessment of two-dimensional perimeter spread models such as fire~\cite{duff2012procrustes}, and the analysis of MRI scans in brain morphology studies~\cite{martin2013correlation}.

\subsection{Our Contribution}

The present study methodizes the aforementioned mentioned cognitive susceptibilities into a cognitive-driven linear mapping transformation determination algorithm. The method leverages on mental rotation cognitive stages~\cite{johnson1990speed} which are defined as it follows
\begin{inlinelist}
	\item a mental image of the object is created
	\item object is mentally rotated until a comparison is made
	\item objects are assessed whether they are the same
	\item the decision is reported
\end{inlinelist}.
Accordingly, the proposed method creates hierarchical abstractions of shapes~\cite{greene2009briefest} with increasing level of details~\cite{konkle2010scene}. The abstractions are presented in a vector space. A graph of linear transformations is created by circular-shift permutations (i.e., rotation superimposition) of vectors. The graph is then hierarchically traversed for closest mapping linear transformation determination. 

Despite of numerous novel algorithms to calculate linear mapping transformation, such as those proposed for Procrustes analysis, the novelty of the presented method is being a cognitive-driven approach. This method augments promising discoveries on motion/object perception into a linear mapping transformation determination algorithm.




\section{Preliminaries}
Throughout, we denote by $k$ a separably closed field. Our references for algebraic groups are~\cite{Borel-AG-book},~\cite{Borel-Tits-Groupes-reductifs},~\cite{Conrad-pred-book},~\cite{Humphreys-book1}, and~\cite{Springer-book}. 

Let $H$ be a (possibly non-connected) affine algebraic group. We write $H^{\circ}$ for the identity component of $H$. We write $[H,H]$ for the derived group of $H$. A reductive group $G$ is called \emph{simple} as an algebraic group if $G$ is connected and all proper normal subgroups of $G$ are finite. We write $X_k(G)$ and $Y_k(G)$ ($X(G)$ and $Y(G)$) for the set of $k$-characters and $k$-cocharacters ($\overline k$-characters and $\overline k$-cocharacters) of $G$ respectively. For $\overline k$-characters and $\overline k$-cocharacters $G$ we simply say characters and cocharacters of $G$. 

Fix a maximal $k$-torus $T$ of $G$ (such a $T$ exists by~\cite[Cor.~18.8]{Borel-AG-book}). Then $T$ splits over $k$ since $k$ is separably closed. Let $\Psi(G,T)$ denote the set of roots of $G$ with respect to $T$. We sometimes write $\Psi(G)$ for $\Psi(G,T)$. Let $\zeta\in\Psi(G)$. We write $U_\zeta$ for the corresponding root subgroup of $G$. We define $G_\zeta := \langle U_\zeta, U_{-\zeta} \rangle$. Let $\zeta, \xi \in \Psi(G)$. Let $\xi^{\vee}$ be the coroot corresponding to $\xi$. Then $\zeta\circ\xi^{\vee}:\overline k^{*}\rightarrow \overline k^{*}$ is a $k$-homomorphism such that $(\zeta\circ\xi^{\vee})(a) = a^n$ for some $n\in\mathbb{Z}$.
Let $s_\xi$ denote the reflection corresponding to $\xi$ in the Weyl group of $G$. Each $s_\xi$ acts on the set of roots $\Psi(G)$ by the following formula~\cite[Lem.~7.1.8]{Springer-book}:
$
s_\xi\cdot\zeta = \zeta - \langle \zeta, \xi^{\vee} \rangle \xi. 
$
\noindent By \cite[Prop.~6.4.2, Lem.~7.2.1]{Carter-simple-book} we can choose $k$-homomorphisms $\epsilon_\zeta : \overline k \rightarrow U_\zeta$  so that 
$
n_\xi \epsilon_\zeta(a) n_\xi^{-1}= \epsilon_{s_\xi\cdot\zeta}(\pm a)
            \text{ where } n_\xi = \epsilon_\xi(1)\epsilon_{-\xi}(-1)\epsilon_{\xi}(1).  \label{n-action on group}
$




The next result~\cite[Prop.~1.12]{Uchiyama-Nonperfect-pre} shows complete reducibility behaves nicely under central isogenies. In this paper we do not specify the isogeny type of $G$. (Our argument works for $G$ of any isogeny type anyway.) Note that if $k$ is algebraically closed, the centrality assumption for $f$ is not necessary in Proposition~\ref{isogeny}. 
\begin{defn}
Let $G_1$ and $G_2$ be reductive $k$-groups. A $k$-isogeny $f:G_1\rightarrow G_2$ is \emph{central} if $\textup{ker}\,df_1$ is central in $\mathfrak{g_1}$ where $\textup{ker}\,df_1$ is the differential of $f$ at the identity of $G_1$ and $\mathfrak{g_1}$ is the Lie algebra of $G_1$. 
\end{defn}
\begin{prop}\label{isogeny}
Let $G_1$ and $G_2$ be reductive $k$-groups. Let $H_1$ and $H_2$ be subgroups of $G_1$ and $G_2$ be subgroups of $G_1$ and $G_2$ respectively. Let $f:G_1 \rightarrow G_2$ be a central $k$-isogeny. 
\begin{enumerate}
\item{If $H_1$ is $G_1$-cr over $k$, then $f(H_1)$ is $G_2$-cr over $k$.}
\item{If $H_2$ is $G_2$-cr over $k$, then $f^{-1}(H_2)$ is $G_1$-cr over $k$.} 
\end{enumerate}
\end{prop}


The next result~\cite[Thm.~1.4]{Bate-cocharacterbuildings-Arx} is used repeatedly to reduce problems on $G$-complete reducibility to those on $L$-complete reducibility where $L$ is a Levi subgroup of $G$. 

\begin{prop}\label{G-cr-L-cr}
Suppose that a subgroup $H$ of $G$ is contained in a $k$-defined Levi subgroup of $G$. Then $H$ is $G$-cr over $k$ if and only if it is $L$-cr over $k$. 
\end{prop}


We recall characterizations of parabolic subgroups, Levi subgroups, and unipotent radicals in terms of cocharacters of $G$~\cite[Prop.~8.4.5]{Springer-book}. These characterizations are essential to translate results on complete reducibility into the language of GIT; see~\cite{Bate-geometric-Inventione},~\cite{Bate-uniform-TransAMS} for example. 

\begin{defn}
Let $X$ be a affine $k$-variety. Let $\phi : \overline k^*\rightarrow X$ be a $k$-morphism of affine $k$-varieties. We say that $\displaystyle\lim_{a\rightarrow 0}\phi(a)$ exists if there exists a $k$-morphism $\hat\phi:\overline k\rightarrow X$ (necessarily unique) whose restriction to $\overline k^{*}$ is $\phi$. If this limit exists, we set $\displaystyle\lim_{a\rightarrow 0}\phi(a) = \hat\phi(0)$.
\end{defn}

\begin{defn}\label{R-parabolic}
Let $\lambda\in Y(G)$. Define
$
P_\lambda := \{ g\in G \mid \displaystyle\lim_{a\rightarrow 0}\lambda(a)g\lambda(a)^{-1} \text{ exists}\}, $\\
$L_\lambda := \{ g\in G \mid \displaystyle\lim_{a\rightarrow 0}\lambda(a)g\lambda(a)^{-1} = g\}, \,
R_u(P_\lambda) := \{ g\in G \mid  \displaystyle\lim_{a\rightarrow0}\lambda(a)g\lambda(a)^{-1} = 1\}. 
$
\end{defn}
Then $P_\lambda$ is a parabolic subgroup of $G$, $L_\lambda$ is a Levi subgroup of $P_\lambda$, and $R_u(P_\lambda)$ is the unipotent radical of $P_\lambda$. If $\lambda$ is $k$-defined, $P_\lambda$, $L_\lambda$, and $R_u(P_\lambda)$ are $k$-defined~\cite[Sec.~2.1-2.3]{Richardson-conjugacy-Duke}. Any $k$-defined parabolic subgroups and $k$-defined Levi subgroups of $G$ arise in this way since $k$ is separably closed. It is well known that $L_\lambda = C_G(\lambda(\overline k^*))$. Note that $k$-defined Levi subgroups of a $k$-defined parabolic subgroup $P$ of $G$ are $R_u(P)(k)$-conjugate~\cite[Lem.~2.5(\rmnum{3})]{Bate-uniform-TransAMS}. Let $M$ be a reductive $k$-subgroup of $G$. Then, there is a natural inclusion $Y_k(M)\subseteq Y_k(G)$ of $k$-cocharacter groups. Let $\lambda\in Y_k(M)$. We write $P_\lambda(G)$ or just $P_\lambda$ for the parabolic subgroup of $G$ corresponding to $\lambda$, and $P_\lambda(M)$ for the parabolic subgroup of $M$ corresponding to $\lambda$. It is clear that $P_\lambda(M) = P_\lambda(G)\cap M$ and $R_u(P_\lambda(M)) = R_u(P_\lambda(G))\cap M$. 

Recall the following geometric characterization for complete reducibility via GIT~\cite{Bate-geometric-Inventione}. Suppose that a subgroup $H$ of $G$ is generated by $n$-tuple ${\bf h}=(h_1,\cdots, h_n)$ of $G$, and $G$ acts on ${\bf h}$ by simultaneous conjugation. 
\begin{prop}\label{geometric}
A subgroup $H$ of $G$ is $G$-cr if and only if the $G$-orbit $G\cdot {\bf h}$ is closed. 
\end{prop}
Combining Proposition~\ref{geometric} and a recent result from GIT~\cite[Thm.~3.3]{Bate-uniform-TransAMS} we have
\begin{prop}\label{unipotentconjugate}
Let $H$ be a subgroup of $G$. Let $\lambda\in Y(G)$. Suppose that ${\bf h'}:=\lim_{a\rightarrow 0}\lambda(a)\cdot {\bf h}$ exists. If $H$ is $G$-cr, then ${\bf h'}$ is $R_u(P_\lambda)$-conjugate to ${\bf h}$. 
\end{prop}
%We also use a rational version of Proposition~\ref{unipotentconjugate}; see~\cite[Cor.~5.1]{Bate-cocharacter-Arx},~\cite[Thm.~9.3]{Bate-cocharacter-Arx}:

%\begin{prop}\label{rationalonjugacy}
%Let $H$ be a subgroup of $G$. Let $\lambda\in Y_k(G)$. Suppose that ${\bf h'}:=\lim_{a\rightarrow 0}\lambda(a)\cdot {\bf h}$ exists. If $H$ is $G$-cr over $k$, then ${\bf h'}$ is $R_u(P_\lambda)(k)$-conjugate to ${\bf h}$. 
%\end{prop}   


\section{Position into the causal ladder}
\label{sec:causalladder}
In order to have an idea of the goodness of the causality of doubly warped spacetimes, next we are going to
determine their position into the causal ladder. As we will see, this depends on the warping functions integrals and the convexity character of their Riemannian fibers.

Let us consider first a brief remainder of the main levels of the causal ladder. Each level corresponds with a causality condition which is strictly more restrictive than the previous one:
\begin{defi}\label{ant} A spacetime $(V,g)$ is
\begin{itemize}
\item {\em non-totally vicious} if $p\not\ll p$ for some $p\in V$.

\item {\em chronological} if it does not contain closed timelike curves.

\item {\em causal} if it does not contain closed causal curves.

\item {\em distinguishing} if whenever $I^+(p)=I^+(q)$ and $I^-(p)=I^-(q)$, necessarily $p=q$.

\item{\em strongly causal} if it does not contain ``nearly closed'' causal curves, i.e. for any open neighborhood $U$ of $p$ there exists some open neighborhood $V$ with $p\in V\subset U$ such that any timelike segment with extreme points in $V$ is contained in $U$.

\item {\em stably causal} if there exists some causal Lorentzian metric $g'$ on $V$ with $g<g'$, i.e., such that $g'(v,v)<0$ for any $v\in TV\setminus \{0\}$ with $g(v,v)\leq 0$. This is equivalent to the existence of some {\em global time function}, i.e., a function defined on the whole spacetime $(V,g)$ which is strictly increasing along each future-directed causal curve.

\item {\em causally continuous} if it is distinguishing and the set valued functions $I^{+}(\cdot)$ and $I^{-}(\cdot)$ are outer continuous (say, $I^{+}(\cdot)$ is {\em outer continuous at some} $p\in V$ if, for any compact subset $K\subset I^{+}(p)$ there exists an open neighborhood $U\ni p$ such that $K\subset I^{+}(q)$ for all $q\in U$). This is equivalent to being distinguishing and {\em reflecting}, %(see \cite[Defn. 3.59]{MS}),
i.e. for any pair of events $p,q \in V$, $I^{+}(q) \subset I^{+}(p)$ if and only if $I^{-}(p) \subset I^{-}(q)$.

\item {\em causally simple} if it is causal and $J^{\pm}(p)$ are closed sets for any $p\in V$.

\item {\em globally hyperbolic} if it is causal and $J^{+}(p)\cap J^{-}(q)$ are compact for any $p,q \in V$.
\end{itemize}
\end{defi}
It is direct from the very basic structure of \multiwarped spacetimes (\ref{eq:1-aux}) that $t:V \rightarrow (a,b)$ is a global time function (see \cite[Lemma 3.55]{beem}). Therefore, any \multiwarped spacetime is stably causal. The approach developed in previous section will allow to show that any \multiwarped spacetime is causally continuous as well. In fact:

\begin{thm}
Any \multiwarped spacetime $(V,g)$ as in (\ref{eq:1-aux}) is causally continuous.
\end{thm}

\begin{proof}  Since $(V,g)$ is stably causal, it is also distinguishing. So, it suffices to show that $(V,g)$ is reflecting. Let $\point{t^o}{x_{1}^o}{x_{2}^{o}}, \point{t^{e}}{x_{1}^{e}}{x_{2}^{e}} \in V$ be such that
$I^{+}(\point{t^{e}}{x_{1}^{e}}{x_{2}^{e}}) \subset I^{+}(\point{t^{o}}{x_{1}^{o}}{x_{2}^{o}})$, and let us prove that $I^{-}(\point{t^{o}}{x_{1}^{o}}{x_{2}^{o}})
\subset I^{-}(\point{t^{e}}{x_{1}^{e}}{x_{2}^{e}})$
(the converse is analogous). Consider the sequence $\{\point{t^{e}+1/n}{x_{1}^{e}}{x_{2}^{e}}\}_{n} \subset I^{+}(\point{t^{e}}{x_{1}^{e}}{x_{2}^{e}})$ %\cambios{(for simplicity, we will assume from here that $t^e+1<b$, otherwise we will take $n\geq n_0$ so $t^e+1/n_0<b$)}
and note that, by the hypothesis, this sequence also belongs to $I^{+}(\point{t^{o}}{x_{1}^{o}}{x_{2}^{o}})$.
Therefore, from Prop. \ref{c0}, there exist constants $\mu_{1}^{n},\mu_{2}^{n}>0$, with $\mu_{1}^{n} + \mu_{2}^{n}=1$, satisfying the following inequalities:
\begin{equation}
\label{eq2'}
\Integral{t^{o}}{t^{e}+1/n}{\mu_{i}^{n}}{i}{\mu_{k}^{n}}
> d_{i}(x_{i}^{o},x_{i}^{e})\qquad\hbox{for $i=1,2$}.
\end{equation}
Up to a subsequence, we can assume that $\{\mu_{i}^{n}\}_{n}$ converges to $\mu_i$, for all $i$, with $0 \leq \mu_{1},\mu_{2}\leq 1$ and $\mu_{1}+\mu_{2}=1$. Moreover,
\begin{equation}
\label{eq3'}
\left\{\sqrt{\mu_{i}^{n}}\alpha_{i}(s)^{-1}\left(\sum_{k=1}^2\mu_{k}^{n} \alpha_{k}(s)^{-1}\right)^{-1/2}\right\}_n \longrightarrow\sqrt{\mu_{i}}\alpha_{i}(s)^{-1}\left(\sum_{k=1}^2\mu_{k}\alpha_{k}(s)^{-1}\right)^{-1/2}
\end{equation}
uniformly on $[t^o,t^e+1]$. Therefore, from (\ref{eq2'}) and (\ref{eq3'}), we deduce
\[
\Integral{t^{o}}{t^{e}}{\mu_{i}}{i}{\mu_{k}} \geq d_{i}(x_{i}^{o},x_{i}^{e}),\qquad\hbox{for $i=1,2$.}
\]
If we consider $\point{t^{o}-1/n}{x_{1}^{o}}{x_{2}^o}$, and modify slightly $(\mu_{1},\mu_{2})$, by continuity we obtain new coefficients $(\mu_{1}',\mu_{2}')$, with $\mu'_{1},\mu'_{2}>0$ and $\mu'_{1}+\mu'_{2}=1$, such that
\[
\Integral{t^{o}-1/n}{t^e}{\mu'_{i}}{i}{\mu'_{k}}
>
d_{i}(x^{o}_{i},x^{e}_{i})\qquad\hbox{for $i=1,2$.}
\]
Again from Prop. \ref{c0}, we have $\point{t^o-1/n}{x_1^o}{x_2^o} \ll \point{t^e}{x_1^e}{x_2^e}$
for all $n$. So, taking into account that $I^-(\point{t^o}{x_1^o}{x_2^o})=\cup_{n \in {\mathbb N}}I^-(\point{t^o-1/n}{x_1^o}{x_2^o})$, we deduce the inclusion $I^-(\point{t^o}{x_1^o}{x_2^o} )\subset
I^-(\point{t^e}{x_1^e}{x_2^e})$, as required.
\end{proof}
\begin{thm}
\label{causi}
A \multiwarped spacetime $(V,g)$ as in (\ref{eq:1-aux}) is causally simple if and only if $(M_i,g_i)$ is $L_i$-convex for $L_i=\int_{a}^{b}\frac{1}{\sqrt{\alpha_{i}(s)}}ds$, $i=1,2$.
\end{thm}

\begin{proof} For the implication to the right, assume that $(V,g)$ is causally simple. We will prove that
$(M_1,g_1)$ is $L_1$-convex (the proof for the second fiber is analogous). Let $x^o_{1},x^e_{1} \in M_{1}$ with $0<d_1(x^o_{1},x^e_{1})<L_1$. Since
$\int_{a}^{b}\frac{1}{\sqrt{\alpha_1(s)}}ds=L_1 > d_1(x^o_{1},x^e_{1})$, there exists $a<\C_1<\C_2<b$ such that
\begin{equation}
\label{eq6'}
\int_{\C_1}^{\C_2} \frac{1}{\sqrt{\alpha_{1}(s)}}ds>d_{1}(x^o_{1},x^e_{1}).
\end{equation}
Fix $x_{2} \in M_{2}$ and consider
the points $\point{\C_1}{x_{1}^o}{x_{2}}$ and $\point{\C_2}{x_{1}^e}{x_{2}}$. Inequality (\ref{eq6'}) and Prop. \ref{c0} imply that
$\point{\C_2}{x_{1}^e}{x_{2}} \in I^{+}(\point{\C_1}{x_{1}^o}{x_{2}})$.
Since $\point{\C_1}{x_{1}^e}{x_{2}} \not \in I^+(\point{\C_1}{x_{1}^o}{x_{2}})$, there exists
$t^e \in \R$ such that $\point{t^e}{x^e_{1}}{x_{2}} \in \partial I^{+}(\point{\C_1}{x^o_{1}}{x_{2}})$,
i.e.,
\[
\begin{array}{c}
\point{t^e}{x^e_{1}}{x_{2}} \in \overline{I^{+}(\point{\C_1}{x^o_{1}}{x_{2}})} \setminus I^{+}(\point{\C_1}{x^o_{1}}{x_{2}}) \qquad\qquad\qquad \\ \qquad\qquad\qquad\qquad\qquad\qquad=J^{+}(\point{\C_1}{x^o_{1}}{x_{2}})\setminus I^{+}(\point{\C_1}{x^o_{1}}{x_{2}}),
\end{array}
\]
where, in the equality, we have used that $(V,g)$ is causally simple. Therefore, there exists a null geodesic
$\gamma(s)=(t(s),c_{1}(s),c_{2}(s))$ connecting $\point{\C_1}{x^o_{1}}{x_{2}}$ with $\point{t^e}{x^e_{1}}{x_{2}}$.
% (\cite[Cor. 4.14]{beem})\footnote{¿Es esta dita realmente necesaria?}
From Prop. \ref{p2'} there exist constants $\mu'_1, \mu'_2\geq 0$ such that the following inequalities hold:
%In particular,
%$x_{1}(s)$ is a pregeodesic in $(M_{1},g_{1})$. Since $\gamma(s)$ is a future directed
%null curve, the pregeodesics $x_{1}(s)$ can be parametrized using $t$ as a parameter; so, the following inequalities hold
%satisfies
\[
\begin{array}{c}
0<d_{1}(x^o_{1},x^e_{1})\leq \displaystyle\Integral{\C_1}{t^e}{\mu'_{1}}{1}{{\mu'}_{k}}=\hbox{length}_1(c_{1}),
\\ 0=d_{2}(x_{2},x_{2}) \leq \displaystyle\Integral{\C_1}{t^e}{\mu'_{2}}{2}{{\mu'}_{k}}=\hbox{length}_2(c_{2}).
%\\ \vdots \\
%0=d_{n}(x_{n},x_{n}) \leq \Integral{-n_{0}}{t^e}{\mu_{n}}{1}{\mu_{k}}   ={\mathrm length}_n(x_{n}),
%
%
%\int_{-n_{0}}^{t^e}
%\sqrt{\mu_{n}}\alpha_{n}^{-1}\left(\sum_{j=1}^n\mu_{j}\alpha_{j}^{-1}\right)^{-1/2}={\mathrm length}_n(x_{n}),
\end{array}
\]
%where $\mu_{i}=\alpha_{i}^{2}(t(s))g_{i}(c_{i}',c_{i}') \in \mathbb{R}$ and $c_{i}(s)$ are pregeodesics in $(M_{i},g_{i})$ for  $i=1,2$ (see \eqref{eq:31}).
So, taking into account that $$\point{t^e}{x^e_{1}}{x_{2}} \not\in I^+(\point{\C_1}{x^o_{1}}{x_{2}}),$$ the second inequality in the first line must be an equality (recall Prop. \ref{c0}). In conclusion, $c_1(s)$ is a reparametrization of a minimizing geodesic of $(M_1,g_1)$, as required.

\smallskip

For the implication to the left, assume that $(M_i,g_i)$ is $L_i$-convex for $L_i=\int_{a}^{b}\frac{1}{\sqrt{\alpha_i(s)}}ds$, $i=1,2$. In order to prove that $(V,g)$ is causally simple,
take $\point{t^{e}}{x_{1}^{e}}{x_{2}^{e}} \in \overline{J^{+}(\point{t^{o}}{x_{1}^{o}}{x_{2}^{o}})}=\overline{I^{+}(\point{t^{o}}{x_{1}^{o}}{x_{2}^{o}})}$. Then, $I^+(\point{t^{e}}{x_{1}^{e}}{x_{2}^{e}}) \subset I^+(\point{t^{o}}{x_{1}^{o}}{x_{2}^{o}})$, and thus, $\point{t^{o}}{x_{1}^{o}}{x_{2}^{o}} \ll \point{t^{e}+1/n}{x_{1}^{e}}{x_{2}^{e}}$ for all $n$.
From Prop. \ref{c0}, there exist constants $0 <\mu_{1}^{n},\mu_{2}^{n} < 1$, with $\mu_{1}^{n}+\mu_{2}^{n}=1$ for all $n$, such that
\[
\Integral{t^o}{t^{e}+1/n}{\mu_{i}^{n}}{i}{\mu_{k}^{n}}
> d_{i}(x_{i}^{o},x_{i}^{e}),\quad i=1,2.
\]
Since $\{\mu_{i}^{n}\}_{n}$ converges (up to subsequence) to some $\mu_{i} \in [0,1]$ for $i=1,2$, with $\mu_{1}+\mu_{2}=1$, we have:
\[
\frac{\sqrt{\mu_{i}^{n}}}{\alpha_{i}(s)}\left(\sum_{k=1}^2 \frac{\mu_{k}^{n}}{\alpha_{k}(s)}\right)^{-1/2}\longrightarrow
\frac{\sqrt{\mu_{i}}}{\alpha_{i}(s)}\left(\sum_{k=1}^2 \frac{\mu_{k}}{\alpha_{k}(s)}\right)^{-1/2}\quad\hbox{uniformly on $[t^o,t^e+1]$.}
\]
Recalling now that all previous functions are bounded by the (Lebesgue) integrable function $g:[t^o,t^e+1]\rightarrow \R$,
    $g(t)=\alpha_i(t)^{-1/2}$, the Dominated Convergence Theorem ensures that:
\[
\Integral{t^o}{t^e}{\mu_{i}}{i}{\mu_{k}}
=\lim_{n\rightarrow\infty} \Integral{t^{o}}{t^{e}+1/n}{\mu_{i}^{n}}{i}{\mu_{k}^n} \geq d_{i}(x_{i}^{o},x_{i}^{e}).
\]
%\footnote{OJO!!!}with equality in the $i$-equation if and only if $\mu_{i,n}=0$.
In particular,
\[
d_{i}(x_{i}^{o},x_{i}^{e})<\Integral{a}{b}{\mu_{i}}{i}{\mu_{k}} \le \int_{a}^{b}\frac{1}{\sqrt{\alpha_i(s)}}ds=L_i,\quad i=1,2.
\]
So, taking into account that $(M_i,g_i)$ are $L_i$-convex for $i=1,2$ we have that Prop. \ref{p2'} implies $\point{t^{e}}{x_{1}^{e}}{x_2^e} \in J^{+}(\point{t^{o}}{x_{1}^{o}}{x_2^o})$, as required.
\end{proof}

The following example shows the tight character of Thm. \ref{causi}, in the sense that there may exist causally simple warped spacetimes with non-convex fiber (the extension to the case of two fibers is straightforward). In fact:
\begin{exe}

 \begin{figure}
\centering
\ifpdf
  \setlength{\unitlength}{1bp}%
  \begin{picture}(377.33, 110.89)(0,0)
  \put(0,0){\includegraphics{fig1.pdf}}
  \put(83.67,100.21){\fontsize{9.42}{11.71}\selectfont $x_0$}
  \put(286.41,100.21){\fontsize{9.42}{11.71}\selectfont $x_1$}
  \put(165.89,99.40){\fontsize{9.42}{11.71}\selectfont $T_1$}
  \put(166.29,78.44){\fontsize{9.42}{11.71}\selectfont $T_2$}
  \put(166.70,50.22){\fontsize{9.42}{11.71}\selectfont $T_n$}
  \put(47.67,69.16){\fontsize{13.76}{14.11}\selectfont $H_0$}
  \put(300.28,67.84){\fontsize{13.76}{14.11}\selectfont $H_1$}
  \end{picture}%
\else
  \setlength{\unitlength}{1bp}%
  \begin{picture}(377.33, 110.89)(0,0)
  \put(0,0){\includegraphics{fig1}}
  \put(83.67,100.21){\fontsize{9.42}{11.71}\selectfont $x_0$}
  \put(286.41,100.21){\fontsize{9.42}{11.71}\selectfont $x_1$}
  \put(165.89,99.40){\fontsize{9.42}{11.71}\selectfont $T_1$}
  \put(166.29,78.44){\fontsize{9.42}{11.71}\selectfont $T_2$}
  \put(166.70,50.22){\fontsize{9.42}{11.71}\selectfont $T_n$}
  \put(47.67,69.16){\fontsize{13.76}{14.11}\selectfont $H_0$}
  \put(300.28,67.84){\fontsize{13.76}{14.11}\selectfont $H_1$}
  \end{picture}%
\fi
%   \begin{figure}[H]
% \centering
% \ifpdf
%   \setlength{\unitlength}{1bp}%
%   \begin{picture}(454.19, 131.80)(20,0)
%   \put(0,0){\includegraphics{fig1.pdf}}
%   \put(100.04,120.06){\fontsize{11.77}{13.33}\selectfont $x_0$}
%  \put(70.04,80.06){\fontsize{14.77}{15.33}\selectfont $H_0$}
%  \put(345.36,120.06){\fontsize{11.77}{13.33}\selectfont $x_1$}
%  \put(365.36,80.06){\fontsize{14.77}{15.33}\selectfont $H_1$}
%   \put(199.54,119.08){\fontsize{11.77}{13.33}\selectfont $T_1$}
%   \put(200.02,93.72){\fontsize{11.77}{13.33}\selectfont $T_2$}
%   \put(200.51,59.58){\fontsize{11.77}{13.33}\selectfont $T_n$}
%   \end{picture}%
% \else
%   \setlength{\unitlength}{1bp}%
%   \begin{picture}(454.19, 131.80)(0,0)
%   \put(0,0){\includegraphics{fig1}}
%   \put(100.04,120.06){\fontsize{9.77}{11.33}\selectfont $x_0$}

%   \put(345.36,120.06){\fontsize{7.77}{9.33}\selectfont $x_1$}
%   \put(199.54,119.08){\fontsize{7.77}{9.33}\selectfont $T_1$}
%   \put(200.02,93.72){\fontsize{7.77}{9.33}\selectfont $T_2$}
%   \put(200.51,59.58){\fontsize{7.77}{9.33}\selectfont $T_n$}
%   \end{picture}%
% \fi
  \caption{\label{fig:1} Both hemispheres $H_0$ and $H_1$ are connected by a sequence of immersed tubes $\{T_n\}_n$, where a length-minimizing curve connecting the north pole $x_0$ of $H_0$ to the north pole $x_1$ of $H_1$ through $T_n$ has bigger length than a length-minimizing curve connecting the same points through $T_{n+1}$. This picture is based on \cite[Figure 1]{Bartolo2002}.}
\end{figure}


\footnote{We are thankful to Prof. Miguel Sánchez for bringing this example to our attention.}In \cite[Section 2.1]{Bartolo2002} the authors construct a Riemannian manifold $(M,g)$ containing two points $x_0, x_1\in M$ such that any geodesic $\gamma\subset M$ connecting them satisfies ${\mathrm length}(\gamma)>d(x_0,x_1)$. The example basically consists of two open hemispheres $H_0$, $H_1$ in $\R^3$ connected by a sequence of immersed tubes $(T_n)_n$ of decreasing lengths, and such that any curve joining the corresponding north poles $x_0$ and $x_1$ through $T_n$ is longer than a minimizing curve joining them through $T_{n+1}$ (see Figure \ref{fig:1}). It is assumed also that the lengths of these tubes converge to a number which is strictly positive. In particular, $x_{0}$ and $x_{1}$ cannot be joined by a minimizing geodesic, and thus,
$(M,g)$ is not convex. However, there exists some $\delta>0$ such that $(M,g)$ is $L$-convex for any $L\leq \delta$. Consider now the warped spacetime $V=\R\times_{\alpha}M$ with $\alpha:\R\rightarrow (0,\infty)$ satisfying $\int_{-\infty}^{+\infty}1/\sqrt{\alpha(s)}ds=L\leq\delta$. From Thm. \ref{causi}, $V$ is causally simple.
%However, as commented above, the fiber $(M,g)$ is not weakly convex, since $x_{0}$ and $x_{1}$ cannot be joined by a minimizing geodesic.
\end{exe}

Finally, for the sake of completeness, we include the following simple consequence of \cite[Th. 3.68]{beem}, whose implication to the left is reproved here by using the techniques developed in this paper:

\begin{thm}
A \multiwarped spacetime $(V,g)$ as in (\ref{eq:1-aux}) is globally hyperbolic if and only if $(M_{i},g_{i})$, $i=1,2$, are complete Riemannian manifolds.
\end{thm}

\begin{proof}
%For the implication to the right, assume for instance that $(M_1,g_1)$ is not complete. From \cite{beem}[Th. 3.68], the Lorentzian warped product $H\times_{\alpha_1} M_1$, with $H \equiv (\R,-dt^2)$ and $M_1 \equiv (M_1,g_1)$, is not globally hyperbolic. Since $H\times_{\alpha_1}M_1$ is causal, it must contain some noncompact causal diamond $J^+((t^o,x_1^o))\cap J^-((t^e,x_1^e))$. Then, the causal diamond $J^+(\point{t^o}{x_{1}^o}{x_{2}}) \cap J^-(\point{t^e}{x_1^e}{x_2})$, for any $x_2\in M_2$, is not compact either; in fact, note that
%\[
%J^+(\point{t^o}{x_{1}^o}{x_{2}}) \cap J^-(\point{t^e}{x_1^e}{x_2})=\pi \circ j_{x_2}(J^+(\point{t^o}{x_1^o}{x_2})\cap J^-(\point{t^e}{x_1^e}{x_2}),
%\]
%where $j_{x_2}:\mathbb{R} \times M_{1} \rightarrow \mathbb{R} \times M_{1} \times M_{2}$, with $j_{x_2}(t,x_{1})=(t,x_{1},x_2)$, and the natural projection $\pi:\mathbb{R} \times M_{1}\times M_{2} \rightarrow \mathbb{R} \times M_{1}$ are both, causal relations preserving and continuous maps.
%\medskip
%\footnote{Jony: Sólo se está probando un lado. El otro aparece probado en el .tex, pero comentado para que no aparezca. ¿Por qué?}
Assume that $(M_{i},g_{i})$, $i=1,2$, are complete. Since $(V,g)$ is causally continuous, and thus, causal, it suffices to prove that any causal diamond is sequentially compact (and thus, compact). Let $\{\point{t^n}{x_1^n}{x_2^n}\}_n$ be a sequence in $J^+(\point{t^o}{x_1^o}{x_2^o})\cap J^-(\point{t^e}{x_1^e}{x_2^e})$. Since the fibers are complete, they are convex, and so, we can apply Prop. \ref{p2'}. Hence, there exist constants $0\leq\mu_{1}^{n},\mu_{2}^{n}\leq 1$, $0\leq\overline{\mu}_{1}^{n},\overline{\mu}_{2}^{n}\leq 1$ with $\mu_{1}^{n}+\mu_{2}^{n}=1=\overline{\mu}_{1}^{n}+\overline{\mu}_{2}^{n}$ for all $n$, such that
\[
  \begin{array}{l}
\displaystyle\Integral{t^{o}}{t^n}{\mu_{i}^{n}}{i}{\mu_{k}^{n}} \geq d_{i}(x_{i}^{o},x_{i}^{n})   \\
\displaystyle\Integral{t^{n}}{t^e}{\overline{\mu}_{i}^{n}}{i}{\overline{\mu}_{k}^{n}} \geq d_{i}(x_{i}^{n},x_{i}^{e}),
  \end{array}\quad i=1,2.
\]
In particular, the following inequalities hold for all $n$:
\[
t^o\leq t^n\leq t^e\quad\hbox{and}\quad\int_{t^{o}}^{t^e}\frac{1}{\sqrt{\alpha_i(s)}}ds \geq d_{i}(x_{i}^{o},x_{i}^{n}),\quad i=1,2.
\]
That is, $t^n \in [t^o,t^e]$ and $x_i^{n} \in \overline{B}_{r_i}(x_i^o)$, $r_i:=\int_{t^{o}}^{t^e}\alpha_i^{-1/2}ds$, $i=1,2$, for large $n$.
But, $[t^o,t^e]$ and $\overline{B}_{r_i}(x_i^o)$, $i=1,2$, are compact sets (recall that $(M_{i},g_{i})$, $i=1,2$, are complete). So,
%there some subsequence
%$\{(x_{1}^{m},x_{2}^{m})\}_{m}$ to some point $(x_{1},x_{2}) \in \overline{B}_{r_{1}}(x_{1}^o) \times \overline{B}_{r_{2}}(x_{2}^o)$ and also $\{t_{m}\}_{m}$ converges to some $t' \in [t^o,t^e]$.
%So,
up to a subsequence, $\{\point{t^n}{x_1^n}{x_2^n}\}_m$ converges to some point $\point{t^*}{x_{1}}{x_{2}}\in V$, which necessarily lies into the (closed) causal diamond $J^+(\point{t^o}{x_1^o}{x_2^o}) \cap J^-(\point{t^e}{x_1^e}{x_2^e})$. In conclusion, the causal diamond is sequentially compact, and so, $(V,g)$ is globally hyperbolic.
\end{proof}

%\cambios{Therefore \multiwarped spacetimes $(V,g)$ are causally continuous, stably causal, strongly causal, distinguishing, causal, chronological and non-totally vicious. And with special conditions over the Riemannian fibers, such as $L_{i}$-weakly convexity and completeness, we obtain that \multiwarped spacetimes are causally simple and globally hyperbolic.}

%%% Local Variables:
%%% mode: latex
%%% TeX-master: "DoublyWarpedBoundary2017.tex"
%%% End:


\section{The future c-completion of doubly warped spacetimes}
\label{sec:futurecompletion}
 In this section we are going to study the point set and topological structure of the future c-completion of doubly warped spacetimes.

 Let $\gamma: [\omega,\Omega) \rightarrow V$, $\Omega\leq b$ be a future-directed timelike
curve in $V$. We can reparametrize this curve by using the standard parameter $t$ for the temporal component,
$\gamma(t)=(t,c_{1}(t),c_{2}(t))$. So, from \eqref{eq:3},
\begin{equation}
  \label{eq:4}
  \hbox{length}(c_{i}\mid_{[\omega,\Omega)})
  %=\int_{\alpha}^{\Omega}\sqrt{\mu_{i}\circ
%s}\cdot\alpha_{i}^{-1}\left(-(D\circ s)+\frac{\mu_{1}\circ
%s}{\alpha_{1}}+\frac{\mu_{2}\circ
%s}{\alpha_{2}}\right)^{-1/2}ds
  \leq\int_{\omega}^{\Omega}\frac{ds}{\sqrt{\alpha_{i}(s)}}.
\end{equation}
Next, assume that
%In order to determine $I^-(\gamma)$, there are two main cases to discuss, depending on the finite or infinite value of $\Omega$. In the first case this study can be developed independently of the warping functions. However, in the second case, these functions will play an essential role, determining in part the structure of the future boundary.
%First, note that from \eqref{eq:3}, and for any $\Omega$ (perhaps infinity),
%\begin{equation}
%  \label{eq:4}
%  \hbox{length}(x_{i}\mid_{[0,\Omega)})=\int_{\alpha}^{\Omega}\sqrt{c_{i}\circ
%s}\cdot\alpha_{i}^{-1}\left(-(D\circ s)+\frac{c_{1}\circ
%s}{\alpha_{1}}+\frac{c_{2}\circç
%s}{\alpha_{2}}\right)^{-1/2}\leq\int_{\alpha}^{\Omega}\frac{1}{\sqrt{\alpha_{i}(s)}}ds,
%\end{equation}
%First, let us assume that
$\Omega<b$. Then, the integral in (\ref{eq:4}) is finite. Hence, $\hbox{length}(c_i)<\infty$, and so, $c_{i}(t)\rightarrow x_i^*$ for some $x_i^*\in M_i^C$, where $M_i^C$ denotes the Cauchy completion of the Riemannian manifold $(M_i,g_i)$, $i=1,2$. If, in addition, $x_i^*\in M_i$ for $i=1,2$, the past of $\gamma$ is clearly determined by the triple $(\Omega,x_1^*,x_2^*)$. The following result shows that this is also true if $x_i^*$ belongs to the Cauchy boundary $\partial^C M_i$ for some $i=1,2$.

\begin{prop}\label{pastofcurve}
  Let $\gamma:[\omega,\Omega)\rightarrow V$, $\Omega<b$, be a future-directed timelike curve with $\gamma(t)=(t,c_1(t),c_2(t))$. Then, $\gamma(t)\rightarrow (\Omega, x_1^*,x_2^*)\in (a,b)\times M_1^C\times M_2^C$ for some $(x_1^*,x_2^*)\in M_1^C\times M_2^C$. Moreover, $(t^o,x_1^o,x_2^o)\in I^-(\gamma)$ if, and only if, there exist constants $\mu_{1},\mu_{2}>0$ with $\mu_{1}+\mu_{2}=1$ and
%(with
%$\mu'_{1}+\cdots +\mu'_{n}=1$)
such that
\begin{equation}
  \label{eq:5}
\Integral{t^{o}}{\Omega}{\mu_{i}}{i}{\mu_{k}} >
d_{i}(x^{o}_{i},x^{*}_{i})\qquad\hbox{for $i=1,2$.}
\end{equation}
%In particular, the past of $\gamma$ is determined by the triple $(\Omega,x_1^*,x_2^*)$.
\end{prop}

\begin{proof}
As argued above, the first assertion is a direct consequence of \eqref{eq:4}. So, we only need to focus on the last assertion.

For the implication to the right, assume that $(t^o,x_1^o,x_2^o)\in I^-(\gamma)$. Since the chronological past $I^-(\gamma)$ is an open set, we can take $\epsilon >0$ small enough so that $(t^o+\epsilon,x_1^o,x_2^o)\in I^-(\gamma)$. Consider an increasing sequence $\{t_n\}\subset [\omega,\Omega)$ with $t_n\nearrow \Omega$ and $(t^o+\epsilon,x_1^o,x_2^o)\ll \gamma(t_n)$ for all $n$. For each $n$, Thm. \ref{c0} ensures the existence of constants $\mu_1^n,\mu_2^n>0$, with $\mu^n_1+\mu^n_2=1$, such that:
  \begin{equation}\label{eq:6}
\Integral{t^{o}+\epsilon}{t_n}{\mu^n_{i}}{i}{\mu^n_{k}} >
d_{i}(x^{o}_{i},c_{i}(t_n))\qquad\hbox{for $i=1,2$.}
    \end{equation}
    Observe that $\{c_i(t_n)\}_n\rightarrow x_i^*\in M_i^C$ for $i=1,2$, and so, from the continuity of the distance function $d_i(x_i^o,\cdot)$ on $M_i^C$, necessarily $\{d_i(x_i^o,x_i(t_n))\}_n\rightarrow d_i(x_i^o,x_i^*)$. Even more, since $\{\mu^n_i\}_n\subset [0,1]$, we can assume that $\{\mu^n_i\}_n$ converges (up to a subsequence) to, say, $\mu_i^*$, $i=1,2$, with $\mu_1^*+\mu_2^*=1$. Hence,
    \[
\left\{\frac{\sqrt{\mu^n_{i}}}{\alpha_{i}(s)}\left(\sum_{k=1}^2 \frac{\mu_{k}^n}{\alpha_{k}(s)}\right)^{-1/2} \right\}_n\rightarrow \frac{\sqrt{\mu^*_{i}}}{\alpha_{i}(s)}\left(\sum_{k=1}^2 \frac{\mu_{k}^*}{\alpha_{k}(s)}\right)^{-1/2}\quad\hbox{pointwise on $[t^o,\Omega]$.}
    \]
    Arguing as in the proof of Thm. \ref{causi}, we observe that these functions are bounded by the integrable function $g:[t^o,\Omega]\rightarrow\R$, $g(t)=\alpha_i(t)^{-1/2}$, so the Dominated Convergence Theorem ensures that
    %Recalling now that all previous functions are bounded by the (Lebesgue) integrable function $g:[t^o,\Omega]\rightarrow \R$,
    %$g(t)=\frac{1}{\sqrt{\alpha_i(t)}}$, the Dominated Convergence Theorem\footnote{JONY: Este argumento aparece dos veces...} ensures that:
   \[
\left\{\Integral{t^{o}+\epsilon}{t_n}{\mu^n_{i}}{i}{\mu^n_{k}}\right\}_n\rightarrow \Integral{t^{o}+\epsilon}{\Omega}{\mu^*_{i}}{i}{\mu^*_{k}}.
    \]
    In conclusion, by taking limits in \eqref{eq:6}, we arrive to
    \[
\Integral{t^{o}+\epsilon}{\Omega}{\mu^*_{i}}{i}{\mu^*_{k}}\geq d_i(x_i^o,x_i^*)\qquad\hbox{for $i=1,2$.}
      \]
      In order to conclude the implication, it rests to show that, if $t^o+\epsilon$ is replaced by $t^o$, all previous inequalities are strict. In principle, the only way to avoid this conclusion is by assuming that some $\mu_i^*$ is equal to zero. If, say, $\mu_1^*=0$ (and so, $\mu_2^*=1$), then (i) $d_1(x_1^o,x_1^*)=0$ and (ii)
      \[
\int_{t^o}^{\Omega}\frac{1}{\sqrt{\alpha_2(s)}}ds>d_2(x_2^o,x_2^*).
        \]
        Reasoning as in the proof of Prop. \ref{c0}, a small modification of $\mu_1^*,\mu_2^*$ provides new constants $\mu_1,\mu_2>0$, with $\mu_1+\mu_2=1$, such that
\[
          \left\{\begin{array}{ll}\displaystyle
\Integral{t^{o}}{\Omega}{\mu_{1}}{i}{\mu_{k}}>0= d_1(x_1^o,x_1^*)\\
                   \displaystyle\Integral{t^{o}}{\Omega}{\mu_{2}}{i}{\mu_{k}}> d_i(x_2^o,x_2^*),

            \end{array}
          \right.
          \]
          and we are done.

          \smallskip

          For the converse, assume that \eqref{eq:5} holds for some $(t^o,x_1^o,x_2^o)$ and some constants $\mu_1,\mu_2>0$, with $\mu_1+\mu_2=1$, and let us prove that $(t^o,x^o_1,x^o_2)\in I^-(\gamma)$. Recalling that the inequalities in \eqref{eq:5} are strict and $\gamma(t)=(t,c_1(t),c_2(t))\rightarrow (\Omega,x_1^*,x_2^*)$, there exists some $t^e\in (a,b)$ big enough such that
 \[\Integral{t^{o}}{t^e}{\mu_{i}}{i}{\mu_{k}} >
d_{i}(x^{o}_{i},c_{i}(t^e))\qquad\hbox{for $i=1,2$.}
\]
Hence, from Prop. \ref{c0}, $(t^o,x^o_1,x^o_2)\ll \gamma(t^e)$, as required.

%For the final assertion, recall that...\footnote{Jony: En este último repaso me he dado cuenta que se me olvidó meter aquí la prueba de la ultima frase. Observar que, con lo anterior, queda claro que dos curvas temporales con la tripla $(\Omega,x_1^*,x_2^*)$ determinan el mismo pasado, teniendo que probar que si tienen triplas diferentes generan el pasados diferentes. Si el primer elemento (la parte temporal) es diferente, es muy fácil de probar. En el caso que sea uno de los otros dos, se razona como en la prueba de  }

\end{proof}

We have just proved that the chronological past of a future-directed timelike curve $\gamma$ defined on a finite interval $[\omega,\Omega)$, $\Omega<b$, is determined by its future limit point $(\Omega,x^*_1,x^*_2)$, in the sense that any other future-directed timelike curve $\gamma'$ with the same future limit point has the same chronological past. Next, we are going to prove that if $\gamma'$ is another future-directed timelike curve converging to another triple, then it generates a different past.

\begin{prop}\label{structuraparcialsininfinito}
  Let $(V,g)$ be a {\multiwarped} spacetime as in (\ref{eq:1-aux}). If $\gamma^i:[\omega^i,\Omega^i)\rightarrow V$, $i=1,2$ satisfy $\gamma^i(t)\rightarrow p_i:=(\Omega^i,x_1^i,x_2^i)\in (a,b)\times M_1^C\times M_2^C$ with $p_1\neq p_2$, then $I^-(\gamma^1)\neq I^-(\gamma^2)$.
\end{prop}
\begin{proof}
    The conclusion easily follows if, say, $\Omega^1<\Omega^2$, since in this case $\gamma^2(t)\in I^-(\gamma^2)\setminus I^-(\gamma^1)$ whenever $\Omega^1<t<\Omega^2$. So, we will assume that $\Omega^1=\Omega^2(=:\Omega)$ and, say, $d_1(x_1^1,x_1^2)>0$. Let $t^o$ be close enough to $\Omega<\infty$ so that (recall that $c_1^1(t)\rightarrow x_1^1$)
 \[
\int_{t^o}^{\Omega}\frac{1}{\sqrt{\alpha_1(s)}}ds<\frac{d_1(x_1^1,x_1^2)}{3}\quad\hbox{and}\quad d_1(c_1^1(t^o),x_1^2)>\frac{d_1(x_1^1,x_1^2)}{3},
    \]
and define $q=\gamma^1(t^o)\in I^-(\gamma^1)$.   % $(x^o_1,x_2^o)\in M$ such that $q=(t^o,x^o_1,x_2^o)\in I^-(\gamma_1)$ and $d_1(x^o_1,x^2_1)>d_1(x^1_1,x^2_1)/3$.
    Then, $q\not\in I^-(\gamma^2)$, since, otherwise, from Prop. \ref{pastofcurve},
    \[
\Integral{t^{o}}{\Omega}{\mu_{1}}{1}{\mu_{k}} >
d_{1}(c^{1}_{1}(t^o),x^{2}_{1})\quad\hbox{for some $\mu_1,\mu_2>0$}.
      \]
      But this is not possible since, from the choice of $t^o$,
      \[
d_1(c^1_1(t^o),x^2_1)>\frac{d_1(x^1_1,x^2_1)}{3}>\int_{t^o}^{\Omega}\frac{1}{\sqrt{\alpha_1(s)}}ds>\Integral{t^{o}}{\Omega}{\mu'_{1}}{1}{\mu'_{k}}      \]
   for any positive constants $\mu'_1,\mu'_2$, with $\mu'_1+\mu'_2=1$. In conclusion, $I^-(\gamma^1)\neq I^-(\gamma^2)$ if $p_1\neq p_2$, and the conclusion follows.
\end{proof}
\begin{rem}\label{rem:1} In the proof of previous result the key property is the finite value of the integral $\int_{t^o}^{\Omega}\alpha_i(s)^{-1/2}ds<\infty$, not the finite value of $\Omega$. Of course, the second imply the first, but the same argument can be reproduced if only the first holds.
%
%In previous result condition $\int_{t^o}^{\Omega}\alpha_i(s)^{-1/2}ds<\infty$ has been used in two different ways:
%(i) to ensure that $c_i(t)$ converges to some point $x_i^*\in M_i^C$, $i=1,2$; (ii) to show that any couple of future-directed timelike curves $\gamma^1,\gamma^2:[\alpha,\Omega)\rightarrow V$ whose $i$-components converge to different points in $M_i^C$ have different past.
  \end{rem}
\noindent Props. \ref{pastofcurve} and \ref{structuraparcialsininfinito} together establish a natural bijection between the space $(a,b) \times M_1^C\times M_2^C$ and the set $\hat{V}\setminus \hat{\partial}^{\ncambios{\B}} V$, where $\hat{\partial}^{\ncambios{\B}} V$ denotes the set of TIPs determined by future-directed timelike curves with divergent temporal component ($\Omega=b$). More precisely:

\begin{prop}\label{structuraparcialsininfinito'}
  Let $(V,g)$ be a {\multiwarped} spacetime as in (\ref{eq:1-aux}). Then, there exists a bijection
  \begin{equation}\label{v}
\hat{V}\setminus \hat{\partial}^{\ncambios{\B}} V\; \leftrightarrow\; (a,b)\times M_1^C\times M_2^C,
    \end{equation}
    which maps each indecomposable past set $P\in\hat{V}\setminus \hat{\partial}^{\ncambios{\B}}V$ to the limit point $(\Omega,x^*_1,x^*_2)\in (a,b)\times M_1^C\times M_2^C$ of any future-directed timelike curve generating $P$.
    \end{prop}

% \begin{rem}\label{rem:2}
% For future references,  observe that, in the last part of the proof, we have shown that two future-directed timelike curves $\gamma^j:[\alpha,\Omega)\rightarrow V$ with $x^j_i(t)\rightarrow x_i^j\in M_i^C$ and satisfying that $x_i^1\neq x_i^2$ for some $i=1,2$ have different pasts. Moreover, it is remarkable that such a proof only relies on the fact that $\int_{0}^{\Omega}\frac{1}{\sqrt{\alpha_i(s)}}ds<\infty$.\footnote{MEJORAR ESTE TEXTO!!}
% \end{rem}

\smallskip

Next, we are going to extend the point set structure obtained above to a topological level. We will consider $(a,b)\times M_1^C\times M_2^C$ attached with the product topology. The first result shows the continuity of bijection (\ref{v}) in the left direction:

\begin{prop}\label{prop:topbuenadir}
Let $P_n,P\in \hat{V}\setminus \hat{\partial}^{\ncambios{\B}}V$ with $P_n\equiv (\Omega_n,x_1^n,x_2^n)$ and $P\equiv (\Omega,x_1^*,x_2^*)$, where we are assuming that the triplets belong to $(a,b)\times M_1^C\times M_2^C$. If $(\Omega_n,x_1^n,x_2^n)\rightarrow (\Omega,x_1^*,x_2^*)$, then $P\in \hat{L}(\{P_n\}_n)$.
\end{prop}
\begin{proof}
   First, recall the analytic characterization of the IPs $P$ and $P_n$ provided by Prop. \ref{pastofcurve}: a point $(t,x_1,x_2)\in V$ belongs to $ P$ (resp. $P_n$) if, and only if, there exist positive constants $\mu_1,\mu_2$ ($\mu_1^n,\mu_2^n$) with $\mu_1+\mu_2=1$ ($\mu_1^n+\mu_2^n=1$) and satisfying that, for $i=1,2$:
  \[
    \begin{array}{c}
      \displaystyle\Integral{t}{\Omega}{\mu_{i}}{i}{\mu_{k}}>d_i(x_i,x_i^*) \\
   \left( \displaystyle  \Integral{t}{\Omega_n}{\mu^n_{i}}{i}{\mu^n_{k}}>d_i(x_i,x_i^n)  \right)
    \end{array}
    \]
Second, note that, from the hypotheses, the continuity of the distance map, and the Dominated Convergence Theorem, the following two limits hold: $d_i(x_i,x_i^n)\rightarrow d_i(x_i,x_i^*)$ and
    \[
\Integral{t}{\Omega_n}{\mu_{i}}{i}{\mu_{k}}\rightarrow \Integral{t}{\Omega}{\mu_{i}}{i}{\mu_{k}}\quad\hbox{for any $\mu_1,\mu_2>0$.}
    \]
These two properties directly imply both, $P\subset {\rm LI}(\{P_n\}))$ and $P$ is maximal into ${\rm LS}(\{P_n\})$, i.e., $P\in \hat{L}(\{P_n\})$.
 \end{proof}

In order to prove the continuity of bijection (\ref{v}) in the right direction, we need to impose local compactness on the Cauchy completion, since, otherwise, there exist counterexamples (see, for instance, \cite[Example 4.9]{FHSBuseman})

\begin{prop}\label{topcurvasfinitas}
   Let $(V,g)$ be a doubly warped spacetime as in (\ref{eq:1-aux}) with $M_1^C$ and $ M_2^C$ locally compact. If $\{P_n\}_n$ is a sequence of IPs converging to some IP, $P\equiv (\Omega,x_1^*,x_2^*)\in (a,b) \times M_1^C\times M_2^C$, then $P_n\equiv (\Omega^n,x_1^n,x_2^n)\in (a,b) \times M_1^C\times M_2^C$ for $n$ big enough, and $(\Omega^n,x_1^n,x_2^n)\rightarrow (\Omega,x_1^*,x_2^*)$ with the product topology. As consequence, the bijection (\ref{v}) becomes a homeomorphism.
\end{prop}

\begin{proof}
 The proof follows essentially in the same fashion as \cite[Prop. 5.24]{FHSBuseman}.

  Since the Cauchy completion $M_1^C\times M_2^C$ is locally compact, there exists a pre-compact neighbourhood $U$ of $P\equiv(\Omega,x_1^*,x_2^*)$. Let $\{p^n_m\}_m,\{p_m\}_m\subset V$ be future chains generating $P_n$ and $P$, resp. We can assume without restriction that $\{p_m\}\subset U$.
  It suffices to show the existence of $n_0$ and a map $\mathfrak{m}:\N\rightarrow\N$ such that $p_m^n\in U$ for all $n\geq n_0$ and $m\geq \mathfrak{m}(n)$. In fact, in this case, the temporal component of the sequence $\{p^n_m\}_m$ will not diverge as $m\rightarrow\infty$, and so, $P_n$ can be identified with some $(\Omega^n,x_1^n,x_2^n)\in (a,b) \times M_1^C\times M_2^C\cap\overline{U}$. Moreover, since the result is valid for any pre-compact open set $U$, and $(\Omega,x_1^*,x_2^*)$ admits a countable local neighbourhood basis given by pre-compact open sets, necessarily $(\Omega^n,x_1^n,x_2^n)\rightarrow (\Omega_1^*,x_1^*,x_2^*)$.

In order to prove the statement in previous paragraph, assume by contradiction that, up to subsequences, $p^n_m$ is not contained in $U$ for all $m$ and $n$. Since $P\subset {\mathrm LI}(\{P_n\}_n)$, for each $m\in \N$ there exists $n_0$ such that $p_m\in P_n$ for all $n\geq n_0$. Consider a strictly increasing sequence $\{\mathfrak{n}(m)\}_m$ such that $p_m\in P_{\mathfrak{n}(m)}$. Denote by  $\gamma_m$ the future-directed timelike curve from $p_m$ to some point of a future chain generating $P_{\mathfrak{n}(m)}$. Each $\gamma_m $ intersects the boundary of $\overline{U}$ at a point, say, $(s^m,y_1^m,y_2^m)$. Since $U$ is pre-compact, its boundary is compact and we can assume (up to a subsequence) that $(s^m,y_1^m,y_2^m)\rightarrow (s^*,y^*_1,y^*_2)$ for some $(s^*,y^*_1,y^*_2)\in (a,b) \times M_1^C\times M_2^C$. Let us denote by $P'$ the indecomposable set associated to $q=(s^*,y^*_1,y^*_2)$ which is necesssarily different from $P$ (as $q$ belong to the boundary of $U$); and by $\{q_m\}_m$ a future chain generating $P'$. Next, we are going to show that $P'$ breaks the maximality of $P$ into ${\mathrm LS}(\{P_n\})$, in contradiction with $P\in \hat{L}(\{P_n\}_n)$.

Let us show that $P'\subset {\mathrm LS}(\{P_n\})$. First recall that,  for each $m\in \N$, the set $I^+(q_m)$ is an open set containing $q$:
% \footnote{Esto es evidente si $q\in M$, pero no lo parece tanto cuando $q\in \overline{M}$. ¿Incluimos un lemma auxiliar?}.
in fact, this is straightforward if $q\in (a,b)\times M_1\times M_2$; otherwise, it suffices to realize that the characterization of the chronological relation given in Prop. \ref{c0} (which is an open property) extends to the set $(a,b)\times M_1^C\times M_2^C$ (see Prop. \ref{pastofcurve}).
In particular, since $\{(s^k,y_1^k,y_2^k)\}\rightarrow q$, it follows that $(s^k,y_1^k,y_2^k)\in I^+(q_m)$ for $k$ big enough. But, from construction, $(s^k,y_1^k,y_2^k)\in P_{\mathfrak{n}(k)}$, so $q_m\in P_{\mathfrak{n}(k)}$ for $k$ big enough. Therefore, $P'\subset {\mathrm LS}(\{P_{\mathfrak{n}(m)}\}_m)$.

It rests to show that $P\subsetneq P'$; that is, any point $p_m$ of the future chain generating $P$ is contained in $P'$. From construction, $p_m=(t^m,x_1^m,x_2^m)\ll p_k \ll (s^k,y_1^k,y_2^k)$ for all $k> m$. Let $\epsilon>0$ be small enough so that $p_m^\epsilon=(t^m+\epsilon,x_1^m,x_2^m)\ll p_{m+1}$, and thus, $p_m^\epsilon\ll (s^k,y_1^k,y_2^k)$ for all $k>m$. From Prop. \ref{c0}, there exist positive constants $\mu_1^k$ and $\mu_2^k$, with $\mu_1^k+\mu_2^k=1$, such that:
\[
\int_{t^m+\epsilon}^{s^k} \frac{\sqrt{\mu^k_i}}{\alpha_{i}(s)}\left(\sum_{l=1}^{2} \frac{\mu^k_l}{\alpha_{k}(s)} \right)^{-1/2}ds>
d_{i}(x^{m}_{i},y^{k}_{i})\qquad\hbox{for $i=1,2$.}
  \]
 But $\{(s^k,y^k_1,y^k_2)\}\rightarrow (s^*,y_1^*,y_2^*)$. By continuity, and up to a subsequence, there exist positive constants $\mu_1^*,\mu_2^*$, with $\mu_1^*+\mu_2^*=1$, such that:
  \[
\int_{t^m+\epsilon}^{s^*} \frac{\sqrt{\mu^*_i}}{\alpha_{i}(s)}\left(\sum_{l=1}^{2} \frac{\mu^*_l}{\alpha_{k}(s)} \right)^{-1/2}ds \geq
d_{i}(x^{m}_{i},y^{*}_{i})\qquad\hbox{for $i=1,2$.}
    \]
    Now if we replace in previous expression $t^m+\epsilon$ by $t^m$, at least one of previous inequalities becomes strict. Then, reasoning as in the proof of Prop. \ref{c0}, we arrive to

      \[
\int_{t^m}^{s^*} \frac{\sqrt{\mu'_i}}{\alpha_{i}(s)}\left(\sum_{l=1}^{2} \frac{\mu'_l}{\alpha_{k}(s)} \right)^{-1/2}ds >
d_{i}(x^{m}_{i},y^{*}_{i})\qquad\hbox{for $i=1,2$,}
      \]
      for some slightly modified constants $\mu'_i$ from $\mu^*_i$. Therefore, the point $p_m$ belongs to $P'$ (recall Prop. \ref{pastofcurve}). Since this argument works for any point of the sequence $\{p_m\}_m$ generating $P$, the inclusion $P\subsetneq P'$ follows.

    \smallskip

For the last assertion, observe that previous argument gives the continuity of bijection \eqref{v} to the right direction, while Prop. \ref{prop:topbuenadir} ensures the continuity to the left one.

  \end{proof}

 Next, we analyze the case $\Omega=b$. In this case, the finiteness/infiniteness of the integrals associated to the warping functions becomes crucial, so we will consider several subsections to discuss it.
%the past of such a curves (and so, the structure of $\hat{\partial}^{\infty}V$) under some simple conditions on such warping functions.

\subsection{Finite warping integrals}

First, we consider the case when the integrals associated to the warping functions are both finite:
% \cambios{The results will be given in full generality, so we will consider doubly warped models as in \eqref{eq:1-aux}. Then, the integral conditions on the warping functions should read as:}
\begin{equation}
  \label{eq:7}
  \int_{\C}^{b}\frac{1}{\sqrt{\alpha_i(s)}}ds<\infty, \qquad \hbox{$i=1,2$}\quad\hbox{for some $\C\in (a,b)$.}
\end{equation}
In this case, the following result provides the point set and topological structure of the future c-boundary:
\begin{thm}\label{futurestructurefiniteconditions}
  Let $(V,g)$ be a {\multiwarped} spacetime as in (\ref{eq:1-aux}), and assume that the integral conditions \eqref{eq:7} hold. Then, there exists a bijection
  \begin{equation}
    \label{eq:8}
    \hat{V}\; \leftrightarrow \; (a,b]\times M_1^C\times M_2^C
  \end{equation}
  which maps each IP $P\in \hat{V}$ to the limit point $(\Omega,x_1,x_2)\in (a,b]\times M_1^C\times M_2^C$ of any future-directed timelike curve generating $P$. Moreover, if $M_1^C,M_2^C$ are locally compact, this bijection becomes an homeomorphism.
\end{thm}
\begin{proof}
%\cambios{As we recall in Rem. \ref{rem:infinito}, we can consider directly that $(a,b)\equiv \R$}.  Let us begin with the point set structure.
For the first assertion, we only need to prove the corresponding bijection between $\hat{\partial}^{\ncambios{\B}} V$ and $\{b\}\times M_1^C\times M_2^C$ (recall Prop. \ref{structuraparcialsininfinito'}). But this follows from the same arguments as in the proofs of Props. \ref{pastofcurve} and \ref{structuraparcialsininfinito} (recall \eqref{eq:7} and Remark \ref{rem:1}).
%since the key point in those arguments was the finiteness of the corresponding warping functions \eqref{eq:7}.
%
%
%with $\Omega$ replaced by $\infty$, since
%
%
%and taking into account the warping integral hypothesis \eqref{eq:7} (see Remark \ref{rem:1}).

For the second assertion, the continuity to the left of bijection (\ref{eq:8}) follows as in Prop. \ref{prop:topbuenadir}, just taking into account that the integral condition \eqref{eq:7} must be used in order to apply the Dominated Convergence Theorem. For the continuity to the right, assume that $P\in \hat{L}(\{P_n\}_n)$, with $P=I^-(\gamma)$, $P_n=I^-(\gamma_n)$, and being $\gamma:[\omega,\Omega)\rightarrow V$, $\gamma_n:[\omega_n,\Omega_n)\rightarrow V$ future-directed timelike curves. Let $(\Omega,x_1^*,x_2^*)$ and $(\Omega_n,x_1^n,x_2^n)$ be the limit points in $(a,b]\times M_1^C\times M_2^C$ of $\gamma$ and $\gamma_n$, resp. We need to prove that $(\Omega_n,x_1^n,x_2^n)\rightarrow (\Omega,x_1^*,x_2^*)$. Observe that, if $\Omega<b$, then the result follows from Prop. \ref{topcurvasfinitas}, so we will focus on the case $\Omega=b$.

First, note that $\Omega_n\rightarrow b$. In fact, otherwise, there exists $\Omega^*<b$ and a subsequence $\{\Omega_{n_k}\}$ such that $\Omega_{n_k}<\Omega^*$ for all $k\in \N$. But, in this case, $P_{n_k}$ will not contain any point $\gamma(t)$ with $t>\Omega^*$, and so, $P\not\subset  {\mathrm LI}(\{P_n\})$.

Assume by contradiction that, say, $\{x_1^n\}_n$ does not converge to $ x_1^*$. Then, up to a subsequence, there exists $\epsilon_0>0$ such that $d_1(x_1^n,x_1^*)>\epsilon_0$. Take $t^0$ big enough so that

  \[
\int_{t^o}^{b}\frac{1}{\sqrt{\alpha_1(s)}}ds<\frac{\epsilon_0}{3}.
    \]
    Take $(x_1^o,x_2^o)\in M_1\times M_2$ such that $q=(t^o,x_1^o,x_2^o)\in I^-(\gamma)=P$ with $d_1(x_1^o,x_1^*)<\epsilon_0/3$. It suffices to show that $q$ does not belong to $P_n$ for any $n$, since, in this case, we arrive to a contradiction with $P\subset {\mathrm LI}(P_n)$.  So, assume that $q\in P_n$ for all $n$. From  Prop. \ref{pastofcurve}, there exists some $\mu^n_1,\mu^n_2>0$ such that
    \[
\Integral{t^{o}}{\Omega^n}{\mu^n_{1}}{1}{\mu^n_{k}} >
d_{1}(x^{o}_{1},x^{n}_{1}).
      \]
      This is in contradiction with the fact that, for any pair of positive constants $\mu'_1,\mu_2'>0$ with $\mu_1'+\mu'_2=1$,
      \[
d_1(x^o_1,x^n_1)>\frac{2}{3} d_1(x^*_1,x^n_1)> \frac{2}{3}\epsilon_0>\int_{t^o}^{b}\frac{1}{\sqrt{\alpha_1(s)}}ds>\Integral{t^{o}}{b}{\mu'_{1}}{1}{\mu'_{k}}.
        \]
%for any pair of positive constants $\mu'_1,\mu_2'$ with $\mu_1'+\mu'_2=1$.
%Hence, $q$ does not belong to $P_n$ for any $n$, in contradiction with $P\in {\mathrm LI}(P_n)$. In conclusion, $\{x^n_1\}_n\rightarrow x_1^*$, as required.
%\footnote{JONY: Observar que al probar  que la sucesión $\{x_1^n\}$ converge a $x_1^*$, se está usando el mismo argumento que en la prueba de la Prop. \ref{structuraparcialsininfinito}. Probablemente se debería poder extraer el argumento para simplemente llamarlo en los dos lados...}
\end{proof}

\subsection{One infinite warping integral}

Let us consider now the case when just one of the warping integrals is infinite, say:
\begin{equation}
  \label{eq:9}
 \int_{\C}^{b}\frac{1}{\sqrt{\alpha_1(s)}}ds<\infty \qquad \hbox{and}\qquad \int_{\C}^{b}\frac{1}{\sqrt{\alpha_2(s)}}ds=\infty.
\end{equation}

%\cambios{In general, we will assume conditions as
%\begin{equation}
%  \label{eq:9}
% \int_{c}^{b}\frac{1}{\sqrt{\alpha_1(s)}}ds<\infty \qquad \hbox{and}\qquad \int_{c}^{b}\frac{1}{\sqrt{\alpha_2(s)}}ds=\infty,
%\end{equation}
%but, as we recall in Rem. \ref{rem:infinito}, there is no loss of generality if we assume in \eqref{eq:9} that $c=0$ and $b=\infty$.
%}

%Now, if we consider a future-directed timelike curve $\gamma:[\alpha,\infty)\rightarrow V$, $\gamma(t)=(t,c_1(t),c_2(t))$, the curve $c_2$ is not forced to converge to some point of the Cauchy completion $M_2^C$, which will provide a different point set structure for $\hat{\partial}^{\infty}V$.
%
%Let us analyze the possible structure of $\hat{\partial}^{\infty}V$: For each $x_1\in M_1$, the fiber $\R\times \{x_1\}\times M_2$ is isometric to the Generalized Robertson-Walker $(\R\times M_2,\tilde{g}_2)$, where $\tilde{g}_2=-dt^2+\alpha_2(t)g_2$ (from now on denoted by $\R\times_{\alpha_2}M_2$). According to Section \ref{sec:Robertson}, the point set and topological structure of $\hat{\partial}^{\infty}V$ is determined by the corresponding proper Busemann completion ${\cal B}(M_2)$: two different Busemann functions in $M_2$ generate two different indecomposable sets in $\R\times_{\alpha_2} M_2$, and so, two different indecomposable sets in $V$. Moreover, the finite integral conditions ensure that these indecomposable sets in $V$ will depend also on the selected point in $M_1$, that is, the same Buseman function in ${\cal B}(M_2)$ will generate different indecomposable sets in $V$, at least one for each $x_1\in M_1$.
%
%\smallskip
%
%  In the next subsections we will formalize these ideas by considering, not only points of $M_1$, but also of $M_1^C$. Moreover, we will show that the structure of $\hat{\partial}^{\infty}V$ depicted in previous paragraph extends naturally to the topological level, giving a complete characterization of the future c-completion. These results are particularly technical, being necessary to present them gradually. We will focus first on the point set structure.

\subsubsection{Point set structure}

The first integral in condition \eqref{eq:9} plus \eqref{eq:4} ensures that any future-directed timelike curve $\gamma:[\omega,b)\rightarrow V$,  $\gamma(t)=(t,c_1(t),c_2(t))$, satisfies that $c_1(t)\rightarrow x_1^*\in M_1^C$. Moreover, the second integral ensures that the associated Generalized Robertson-Walker spacetime $((a,b) \times M_2,-dt^2+\alpha_2g_2)$ corresponds with the model studied in Section \ref{sec:Robertson}. In particular, since the curve $\sigma(t)=(t,c_2(t))$ is also a future-directed timelike in that spacetime, we can consider the Busemann function $b_{c_2}\in B(M_2)\cup \{\infty\}$.

Next, our aim is to show that the chronological past of $\gamma$ is determined by both, $x_1^*\in M_1^C$ and the Busemann function $b_{c_2}\in B(M_2)\cup \{\infty\}$. Let us begin with the following result:
\begin{prop}\label{prop:conddiferbordedif}
  Let $(V,g)$ be a {\multiwarped} spacetime and assume that the integral conditions \eqref{eq:9} are satisfied. Consider two future-directed timelike curves $\gamma^i:[\omega,b)\rightarrow V$, $\gamma^i(t)=(t,c_1^i(t),c_2^i(t))$, with $c_1^i(t)\rightarrow x_1^i\in M_i^C$, $i=1,2$. If $(x_1^1,b_{c_1})\neq (x_1^2,b_{c_2})$ then $I^-(\gamma^1)\neq I^-(\gamma^2)$.
\end{prop}
\begin{proof} If $x_1^1\neq x_1^2$, we can reason as in the proof of Prop. \ref{structuraparcialsininfinito} (taking $\Omega=\Omega'=b$ and $x_1^1\neq x_1^2$; Remark \ref{rem:1} and the first integral condition in \eqref{eq:9}). So, it suffices to consider the case $b_{c_2^1}\neq b_{c_2^2}$.

Let $\sigma_i(t)=(t,c_2^i(t))$, $i=1,2$, be two future-directed timelike curves on the Generalized Robertson-Walker spacetime $\left( (a,b)\times M_2,-dt^2+\alpha_2 g_2\right)$. Since $b_{c_2^1}\neq b_{c_2^2}$, necessarily $I^{-}(\sigma_1)\neq I^{-}(\sigma_2)$. Assume, for instance, that  $(t^0,y_2)\in I^{-}(\sigma_2)\setminus I^{-}(\sigma_1)$ (the other case is analogous). Then, taking into account the characterization in \eqref{eq:27}, it follows that

  % Assume, for instance, that $b_{c_2^1}(y_2)<b_{c_2^2}(y_2)$ for some $y_2\in M_2$, and let us show that $I^-(\gamma^2)\not\subset I^-(\gamma^1)$. \cambios{NECESITO VER CÓMO METER QUE AMBAS FUNCIONES PUEDEN SER POSITIVAS...}
  % From (\ref{eq:9}) and the continuity of the integral with respect to the superior limit of integration, there exists $t^o$ such that\footnote{J.L.: Aqui parece que se necesita $b_{c_2^i}(y_2)>0$ $i=1,2$.}
  \begin{equation}\label{eq:b}
b_{c_2^1}(y_2)<\int_{\C}^{t^o}\frac{1}{\sqrt{\alpha_2(s)}}ds <b_{c_2^2}(y_2).
\end{equation}
From the first inequality in (\ref{eq:b}),
    \[
      \begin{array}{l}
  (b_{c^1_2}(y_2)=)\lim_{t\rightarrow b} \left(\int_\C^{t}\frac{1}{\sqrt{\alpha_2(s)}}ds-d_2(y_2,c_2^1(t))\right)\leq\int_\C^{t^o}\frac{1}{\sqrt{\alpha_2(s)}}ds\Rightarrow\\ \Rightarrow  \lim_{t\rightarrow b}\left( \int_{t^o}^{t}\frac{1}{\sqrt{\alpha_2(s)}}ds-d_2(y_2,c_2^1(t))\right)\leq 0.
     \end{array}
 \]
 Therefore, since the function $t\mapsto \left(\int_{t^o}^{t}\frac{1}{\sqrt{\alpha_2(s)}}ds-d_2(y_2,c_2^1(t))\right)$ is increasing, we deduce
      \begin{equation}\label{b}
\int_{t^o}^{t}\frac{1}{\sqrt{\alpha_2(s)}}ds<d_2(y_2,c_2^1(t))\quad\hbox{for all $t$.}
        \end{equation}

Let us show the existence of $x_1^o\in M_1$ such that $q=(t^o,x_1^o,y_2)\in I^-(\gamma^2)$. From the inequality
        \[
\int_\C^{t^o}\frac{1}{\sqrt{\alpha_2(s)}}ds <b_{c_2^2}(y_2)=  \lim_{t\rightarrow b}\left(\int_\C^t\frac{1}{\sqrt{\alpha_2(s)}}ds-d_2(y_2,c_2^2(t))\right),
          \]
        there exists $t'>t^o$ big enough such that
        \[
          \int_{t^o}^{t'}\frac{1}{\sqrt{\alpha_2(s)}}ds> d_2(y_2,c_2^2(t')).
        \]
        From continuity, we can find positive constants $\mu_1,\mu_2$, with $\mu_1+\mu_2=1$, such that
        \[
\left\{
  \begin{array}{l}
    \displaystyle\Integral{t^0}{t'}{\mu_2}{2}{\mu_k}>d_2(y_2,c_2^2(t'))\\
    \displaystyle \Integral{t^0}{t'}{\mu_1}{1}{\mu_k}>0.
  \end{array}
\right.
          \]
          So, if we take $x_1^o$ close enough to $c^2_1(t')$ so that
          \[d_1(x_1^o,c^2_1(t'))<\Integral{t^o}{t'}{\mu_1}{1}{\mu_k},\] Prop. \ref{c0} ensures that $(t^o,x_1^o,y_2)\ll \gamma^2(t')$, and thus, $q=(t^o,x_1^o,y_2)\in I^-(\gamma^2)$.

          \smallskip

          On the other hand, for any pair of positive constants $\mu_1,\mu_2>0$ with $\mu_1+\mu_2=1$, necessarily
        \[
\Integral{t^{o}}{t}{\mu_{2}}{2}{\mu_{k}}<\int_{t^o}^t \frac{1}{\sqrt{\alpha_2(s)}}ds< d_2(y_2,c_2^1(t))\quad\hbox{for all $t>t^0$,}
          \]
          where (\ref{b}) has been used in the last inequality. Therefore, from Prop. \ref{c0}, $q\not\ll \gamma^1(t)$ for all $t>t^o$, and thus, $q\not\in I^-(\gamma^1)$.
\end{proof}

  %Proposition \ref{prop:conddiferbordedif} establishes that $\hat{\partial}^\infty V$ contains a set identifiable with the product space $M_1^C\times {\cal B}(M_2)$. Now, we are going to prove that, indeed, there are no additional points in $\hat{\partial}^\infty V$. To this aim, first we will focus on the first spatial component of the curve.

\begin{lemma}\label{lemma:aux3}
Let $\gamma:[\omega,\Omega)\rightarrow V$, $\gamma(t)=(t,c_1(t),c_2(t))$ be a future-directed timelike curve with $c_1(t)\rightarrow x_1^*\in M_1^C$. If $\sigma=\{(t_n,x_1^n,c_2(t_n))\}_n\subset V$ satisfies $\{t_n\}_n\rightarrow \Omega$ and $x_1^n\rightarrow x_1^*$, then $I^-(\gamma)\subset {\mathrm LI}(\{I^-(t_n,x_1^n,c_2(t_n))\}_n)$.
%$P=I^-(\gamma)$ with $\gamma:[\alpha,\Omega)\rightarrow V$, $\gamma(t)=(t,c_1(t),c_2(t))$ and $c_1(t)\rightarrow x_1^*\in M_1^C$. Consider a sequence $\sigma=\{(t_n,x_1^n,c_2(t_n))\}_n\subset V$ with $\{t_n\}_n\rightarrow \Omega$\footnote{Cambiado estrictamente creciente a convergencia, cuidado...} and $x_1^n\rightarrow x_1^*$. Then, $P\subset {\mathrm LI}(\{I^-(t_n,x_1^n,c_2(t_n))\}_n)$.
\end{lemma}
\begin{proof}
Assume by contradiction the existence of some point $q=(t^o,x_1^o,x_2^o)\in I^-(\gamma)$ such that $q\not\ll (t_n,x^n_1,c_2(t_n))$ for infinitely many $n$. From the open character of the chronological relation, we can assume that $x_1^o\neq x_1^*$. Moreover, for $\epsilon>0$ small enough, it follows that $q_{\epsilon}=(t^o+\epsilon,x_1^o,x_2^o)\in I^{-}(\gamma)$.

Assume that, up to a subsequence, $q_{\epsilon}\ll \gamma(t_n)$ for all $n$. From Prop. \ref{c0}, there exist positive constants $\mu_1^n,\mu_2^n>0$, with $\mu_1^n+\mu_2^n=1$, such that
  \[
\Integral{t^{o}+{\epsilon}}{t_n}{\mu^n_{i}}{i}{\mu^n_{k}} >
d_{i}(x^{o}_{i},c_{i}(t_n))\qquad\hbox{for $i=1,2$.}
    \]
We can assume without restriction that $\{\mu_i^n\}_n$ converges to some point $\mu_i^*$, $i=1,2$. Since $q_{\epsilon}\not\in I^-((t_n,x^n_1,c_2(t_n)))$ (recall that $q\not\ll (t_n,x_1^n,c_2(t_n))$), necessarily
   \[
\left(d_{1}(x^{o}_{1},c_{1}(t_n))< \right)\Integral{t^{o}+{\epsilon}}{t_n}{\mu^n_{1}}{1}{\mu^n_{k}}\leq d_1(x_1^o,x^n_1),
      \]
      the last inequality by Prop. \ref{c0}. From the hypothesis, the first and third element in previous expression converge to $d_1(x_1^o,x_1^*)>0$. Moreover, from \eqref{eq:9}, the integral in the middle is also finite. Hence,
      \begin{equation}\label{eq:c}
\left\{\Integral{t^{o}+\epsilon}{t_n}{\mu^n_{1}}{1}{\mu^n_{k}}\right\}_n\rightarrow \Integral{t^{o}+\epsilon}{\Omega}{\mu^*_{1}}{1}{\mu^*_{k}}=d_1(x_1^o,x_1^*)<\infty.
        \end{equation}
        %and thus,
%        \[
%\Integral{t^{o}+\epsilon}{\infty}{\mu^*_{1}}{1}{\mu^*_{k}}=d_1(x_1^o,x_1^*).
%          \]
          In particular, since $x_1^o\neq x_1^*$, necessarily $\mu_1^*\neq 0$, and so,
        \begin{equation}\label{eq:cc}
\Integral{t^{o}}{\Omega}{\mu^*_{1}}{1}{\mu^*_{k}}>d_1(x_1^o,x_1^*).
          \end{equation}
          Finally, from (\ref{eq:c}) and (\ref{eq:cc}),
        \[
\int_{t^o}^{t^n}\frac{\sqrt{\mu_1^n}}{\alpha_1}\left(\sum_{k=1}^{2}\frac{\mu_k^n}{\alpha_k}\right)^{-1/2}ds>d_1(x_1^o,x^n_1)\quad\hbox{for $n$ big enough,}
          \]
which implies that $q=(t^o,x_1^o,x_2^o)\in I^-((t_n,x^n_1,c_2(t_n)))$ for $n$ big enough, a contradiction.
\end{proof}

\noindent This Lemma has the following direct consequence:

\begin{lemma}\label{lemma:aux1}
 Let $\gamma^i:[\omega,b)\rightarrow V$, $\gamma^i(t)=(t,c_1^i(t),c_2(t))$, $i=1,2$, be future-directed timelike curves. If $c_1^i(t)\rightarrow x_1^*\in M^C_1$, $i=1,2$, then $I^-(\gamma^1)=I^-(\gamma^2)$.
\end{lemma}
\begin{proof}
 Let us focus on the inclusion to the right (the other one is analogous). Consider the sequence $\sigma=\{(t_n,c_1^2(t_n),c_2(t_n))\}_n$, where $\{t_n\}_n\nearrow \infty$. For any $p\in I^-(\gamma^1)$, Lemma \ref{lemma:aux3} ensures the existence of $n_0$ such that $p\in I^-((t_n,c_1^2(t_n),c_2(t_n)))\subset I^-(\gamma^2)$ for all $n\geq n_0$, as desired.
\end{proof}

%These results provide some freedom in the choice of the curve $\gamma$ generating the TIP $P$, whenever the first spatial component is converging to an appropriate point. The following result provides a similar property, but now for the second spatial component. The difference is that now this freedom is restricted to the corresponding Busemann function.

\begin{lemma}\label{lemma:aux2}

Let $\gamma^i:[\omega,b)\rightarrow V$, $\gamma^i(t)=(t,c_1(t),c_2^i(t))$, $i=1,2$, be future-directed timelike curves. If $b_{c_2^1}=b_{c_2^2}$, then $I^-(\gamma^1)=I^-(\gamma^2)$.
%Let $\gamma^j:[\alpha,\infty)\rightarrow V$ be curves in $V$ with $\gamma^j(t)=(t,c_1^j(t),c_2^j(t))$ for $j=1,2$. If $c_1^1=c_1^2(=c)$ and $b_{c_2^1}=b_{c_2^2}$, then $I^-(\gamma^1)=I^-(\gamma^2)$.
\end{lemma}
\begin{proof}
Since the first warping integral is finite (recall (\ref{eq:9})), the spatial component $c_1$ admits some limit point $x_1^*\in M_1^C$. Assume by contradiction that, say, $q=(t^o,x_1^o,x_2^o)\in I^-(\gamma^2)\setminus I^-(\gamma^1)$. It is not a restriction to additionally assume that $x_1^o\neq x_1^*$. Let $\epsilon>0$ small enough such that $q_\epsilon=(t^o+\epsilon,x_1^o,x_2^o)\in I^-(\gamma^2)\setminus I^-(\gamma^1)$. Since $q_\epsilon\in I^-(\gamma^2)$, there exists an increasing sequence $\{t_n\}\nearrow b$ such that $q_\epsilon\ll \gamma^2(t_n)$ for all $n$. Then, from  Prop. \ref{c0}, there exist positive constants $\mu_1^n, \mu_2^n >0$, with $\mu_1^n+\mu_2^n=1$, for each $n$, such that
  \begin{equation}\label{eq*}
    \left\{\begin{array}{l}
    \displaystyle  \Integral{t^o+\epsilon}{t_n}{\mu^n_{1}}{1}{\mu^n_{k}}>
             d_{1}(x^o_{1},c_{1}(t_n))\\
\displaystyle\Integral{t^o+\epsilon}{t_n}{\mu^n_{2}}{2}{\mu^n_{k}}>
             d_{2}(x^o_{2},c^2_{2}(t_n)).
    \end{array}\right.
    \end{equation}
    It is not a restriction to assume that each sequence $\{\mu^n_i\}_n$ is convergent to $\mu_i^*$ for $i=1,2$. Next, we are going to prove that the sequences can be chosen satisfying $\mu_{1}^{*} \neq 1$:


   \smallskip

{\em Claim. The sequences $\{\mu^n_i\}_n$  can be chosen so that $\mu_1^*\neq 1$ (and thus, $\mu_2^*\neq 0$).}


  \smallskip


{\em Proof of the Claim.} Let us prove that, if we have a sequence $\{t_n\}_n$ such that $q=(t^o,x_1^o,x_2^o)\ll \gamma(t_n)=(t_n,c_1(t_n),c_2(t_n)) $, we can find sequences $\{\mu_i^n\}_n$, with $\mu_1^n+\mu_2^n=1$, which converge, up to a subsequence, to $\mu_1^*\neq 1$ and $\mu_2^*\neq 0$, such that
\begin{equation}
  \label{eq:30}
    \Integral{t^o}{t_n}{\mu^n_i}{i}{\mu^n_{k}}-
             d_{i}(x^o_{i},c_{i}(t_n))>0\quad \hbox{for $i=1,2$.}
\end{equation}
Observe that Prop. \ref{c0} ensures the existence of such convergent sequences $\{\mu_i^n\}_n$  without the statement about the limits. Assume that $\mu_1^*= 1$. By using standard arguments (that is, working with the point $q_{\epsilon}=(t^o+\epsilon,x_1^o,x_2^o)$ as before, and recalling that $\mu_1^n\geq \frac{1}{2}$ for $n$ big enough), we can take limits on \eqref{eq:30} preserving the strict inequality. So,
\[
\lim_{n\rightarrow \infty}\left(\Integral{t^o}{t_n}{\mu^n_1}{1}{\mu^n_{k}}-
             d_{1}(x^o_{1},c_{1}(t_n))\right)=\int_{t^o}^{b}\frac{1}{\sqrt{\alpha_1(s)}}ds-
             d_{1}(x^o_{1},x_1^*)>0
           \]
where $c_1(t_n)\rightarrow x_1^*$. Now take $\overline{\mu}_2^*>0$ small enough such that $\overline{\mu}_1^*=1-\overline{\mu}_2^*>0$ and such that
           \[
\Integral{t^o}{b}{\overline{\mu}_1^*}{1}{\overline{\mu}^*_k}-d_{1}(x^o_{1},x_1^*)>0
             \]
Now, define $\overline{\mu}_1^n=\mu_1^n-\overline{\mu}_2^*$ and $\overline{\mu}_2^n=\mu_2^n+\overline{\mu}_2^*$. As $\mu_{1}^{n} \rightarrow 1$,  we have that $\overline{\mu}_{1}^{n}>0$ for large $n$ and that
$\overline{\mu}_{1}^{n} \rightarrow \overline{\mu_{1}^{*}}$, therefore by  the Dominated Convergence Theorem (recall the integral condition for $\alpha_1$) we have:
\[
\Integral{t^o}{b}{\overline{\mu}_1^*}{1}{\overline{\mu}^*_k}=lim_{n} \Integral{t^o}{t_{n}}{\overline{\mu}_1^n}{1}{\overline{\mu}^n_k},
\]
Hence,
\[
  lim_{n}\left(\Integral{t^o}{t_{n}}{\overline{\mu}_1^n}{1}{\overline{\mu}^n_k}-d_{1}(x_{1}^{o},c_{1}(t_{n}))\right)
  % =\Integral{t^0}{\infty}{\overline{\mu_1^*}}{1}{\overline{\mu^*_k}}-d_{1}(x_{1}^o,x_{1}^*)
  >0,
\]
and so  for large $n$ we can take $\overline{\mu}_1^n$ and $\overline{\mu}_2^n$ satisfying
\[
\Integral{t^o}{t_n}{\overline{\mu}^n_1}{1}{\overline{\mu}^n_{k}}-
             d_{1}(x^o_{1},c_{1}(t_n))>0.
  \]
  Moreover, as $\overline{\mu}_1^n<\mu_1^n$ and $\overline{\mu}_2^n>\mu_2^n$, it easily follows that:

  \[
\Integral{t^o}{t_n}{\overline{\mu}^n_2}{2}{\overline{\mu}^n_{k}}>\Integral{t^o}{t_n}{\mu^n_2}{2}{\mu^n_{k}}\left(>d_2(x_2^0,c_2(t_n))\right).
    \]
In conclusion, equation \eqref{eq:30} is also true with the sequences $\{\overline{\mu}_i^n\}_n$ and $\{\overline{\mu}_1^n\}_n\rightarrow \overline{\mu}_{1}^{*}= 1-\overline{\mu}_{2}^{*}\neq 1$, which proves the claim.
%... and taking limits on the first previous inequality we deduce that
  %  \[
%\int_{t^0}^{\infty}\frac{1}{\sqrt{\alpha(s)}}ds=
  %           d_{1}(x^0_{1},x_{1}^*),
   %   \]
    %  (we cannot have strict inequality as, in such a case, we can perturb the sequence $\{\mu_1^n\}_n$ such that $\mu_1^*\neq 1$). But this is not possible as, in such a case,

  %    \[
%d_{1}(x^0_{1},x_{1}^*)=\int_{t^0}^{\infty}\frac{1}{\sqrt{\alpha(s)}}ds>\Integral{t^0}{\infty}{\mu'_1}{1}{\mu'_k}
  %      \]
    %  for any pair of positive constant sequences $\mu'_1,\mu'_2$ with $\mu'_1+\mu'_2=1$. But this is a contradiction with....


    \smallskip


  Continuing with the proof of the lemma, note that $\gamma^1$ and $\gamma^2$ share the same first spatial component $c_1$, the first integral condition (\ref{eq*}) coincides for both curves. Therefore, since $q_\epsilon\not\in I^-(\gamma^1)$, necessarily (recall Prop. \ref{c0}):
    \begin{equation}
      \label{eq:11}
d_2(x^o_2,c^1_2(t_n))\geq \Integral{t^o+\epsilon}{t_n}{\mu^n_{2}}{2}{\mu^n_{k}}\left(>
             d_{2}(x^o_{2},c^2_{2}(t_n))\right).
    \end{equation}
    Moreover, from the hypothesis, $b_{c_2^1}(x_2^o)=b_{c_2^2}(x_2^o)$. So, from the definition of Busemann function,
    \begin{equation}\label{x}
    \lim_{n}\left(d_2(x_2^o,c_2^1(t_n))-d_2(x_2^o,c_2^2(t_n))\right)=0.
    \end{equation}
    From \eqref{eq:11} and (\ref{x})
    \begin{equation}
      \label{eq:12}
\lim_n \left(\Integral{t^o+\epsilon}{t_n}{\mu^n_{2}}{2}{\mu^n_{k}}-d_2(x_2^o,c_2^1(t_n))\right)=0.
    \end{equation}

On the other hand, from the claim, the sequence of positive constants $\{\mu_2^n\}_n$ does not converge to $0$, so there exists ${\cal K}>0$ such that
   \begin{equation}
     \label{eq:13}
     \Integral{t^o}{t^o+\epsilon}{\mu_2^n}{2}{\mu_k^n}>{\cal K}>0\quad\hbox{for $n$ big enough.}
   \end{equation}
        So, putting together \eqref{eq:12} and \eqref{eq:13} we obtain that
\[
\lim_n \left(\Integral{t^o}{t_n}{\mu^n_{2}}{2}{\mu^n_{k}}-d_2(x_2^o,c_2^1(t_n))\right)>0,
  \]
  and thus,
  \[
\Integral{t^o}{t_n}{\mu^n_{2}}{2}{\mu^n_{k}}>d_2(x_2^o,c_2^1(t_n))\quad\hbox{for $n$ big enough.}
    \]
    From Prop. \ref{c0}, necessarily $q\in I^-(\gamma^1)$, a contradiction.


\end{proof}
\noindent As a direct consequence of Lemmas \ref{lemma:aux1} and \ref{lemma:aux2}, we obtain:
\begin{prop}\label{samecondsamepast}
Let $\gamma^i:[\omega,b)\rightarrow V$, $\gamma^i(t)=(t,c_1^i(t),c_2^i(t))$, $i=1,2$, be future-directed timelike curves. If $c_1^i(t)\rightarrow x_1^*\in M_1^C$, $i=1,2$, and $b_{c_2^1}=b_{c_2^2}$, then $I^-(\gamma^1)=I^-(\gamma^2)$.
\end{prop}
\begin{proof}
Let $c_1:[\omega,b)\rightarrow M_1$ be a curve with $c_1(t)\rightarrow x_1^*$ such that the curves $\overline{\gamma}^i:[\omega,\infty)\rightarrow V$, $\overline{\gamma}^i(t)=(t,c_1(t),c_2^i(t))$, $i=1,2$, are future-directed timelike. From Lemma \ref{lemma:aux1}, $I^-(\gamma^i)=I^-(\overline{\gamma}^i)$, $i=1,2$. But $\overline{\gamma}^1$, $\overline{\gamma}^2$ share the same first spatial components, and their second spatial components define the same Busemann function. Hence, Lemma \ref{lemma:aux2} ensures that $I^-(\overline{\gamma}^1)=I^-(\overline{\gamma}^2)$, as required.
\end{proof}

%Putting together Props. \ref{structuraparcialsininfinito}, \ref{prop:conddiferbordedif}
%and \ref{samecondsamepast} we obtain the following bijection:
%%\begin{equation}
%%  \label{eq:14}
%%\hat{V}\;\leftrightarrow\; \left(\R\times M_1^C\times M_2^C\right)\cup \left(\{\infty\}\times M_1^C\times {\cal B}(M_2)\right),
%%\end{equation}
%%which maps each indecomposable past set $P\in \hat{V}$ to the limit point of any curve generating it, which is either $(\Omega,x_1^*,x_2^*)\in \R\times M_1^C\times M_2^C$ if $\Omega<\infty or $(\Omega,x_1^*,b_{c_2})\in \{\infty\}\times M_1^C\times {\cal B}(M_2)$ if $\Omega=\infty$.
%
%%In particular, any future indecomposable set $P=I^-(\gamma)$ with $\gamma:[\alpha,\Omega)\rightarrow V$ and $\gamma(t)=(t,c_1(t),c_2(t))$ is determined by three elements: the limit $\Omega\in \R\cup \{\infty\}$ of the temporal component; the limit $x_1^*\in M_1^C$ of the first spatial component; and, (a) the limit $x_2^*\in M_2^C$ of the second spatial component if $\Omega<\infty$ or (b) the busemann function $b_{c_2}$ if $\Omega=\infty$.
%Even more, we can even give a unified treatment to this last case by recalling that, when $\Omega<\infty$, then the associated Busemann function $b_{c_2}\equiv d_{(\Omega,x_2^*)}$, so it codifies both, the limits of the temporal component and the second spatial component.

\noindent Summarizing, if we put together Props. \ref{structuraparcialsininfinito'}, \ref{prop:conddiferbordedif}
and \ref{samecondsamepast}, we deduce the following point set structure for the future c-completion of $(V,g)$:
\begin{thm}\label{futurecomploneinfinite}
  Let $(V,g)$ be a  {\multiwarped} spacetime as in \eqref{eq:1-aux}, and assume that the integral conditions \eqref{eq:9} hold. Then, there exists a bijection
 \begin{equation}
   \label{eq:10}
     \hat{V}\; \leftrightarrow \;  M_1^C\times \left(B(M_2)\cup \{\infty\}\right)\;\equiv\;
       \left( (a,b)\times M_1^C\times M_2^C\right) \cup M_{1}^{C} \times \left({\cal B}(M_2)\cup \{\infty\}\right).
     %,\qquad \hat{V}\setmin us \hat{\partial}^\infty V\equiv \cambios{(a,b)}\times M_1^C\times M_2^C  \\
%
%     \\
%\cambios{
%     \begin{array}{rl}
%       \hat{\partial}^{\infty}V\equiv  & M_1^C\times \left({\cal B}(M_2)\cup \{\infty\}\right)\\ \equiv & M_1^C\times \left(\left(\R\times \partial_{\cal B}M_2\right) \cup \{\infty\}   \right)
%     \end{array}
%}
 % \end{array}
  \end{equation}
 This bijection maps each indecomposable past set $P=I^-(\gamma)\in \hat{V}$, where $\gamma:[\omega,\Omega)\rightarrow V$, $\gamma(t)=(t,c_1(t),c_2(t))$, is any curve generating $P$, to a pair $(x_1^*,b_{c_2})$, where $x_1^*\in M_1^C$ is the limit point of the curve $c_1$. If $\Omega<b$, then $b_{c_2}=d_{(\Omega,x_2^*)}$, where $x_2^*$ is the limit point of $c_2$ (see \eqref{eq:46}), and thus, $P$ can be also identified with the limit point $(\Omega,x_1^*,x_2^*)$ of $\gamma$ (recall Prop. \ref{structuraparcialsininfinito'}).

  % to the limit point of any curve generating it, which is either $(\Omega,x_1^*,x_2^*)\in \R\times M_1^C\times M_2^C$ if $\Omega<\infty$ or $(\Omega,x_1^*,b_{c_2})\in \{\infty\}\times M_1^C\times {\cal B}(M_2)$ if $\Omega=\infty$.
\end{thm}


%This last property becomes specially interesting in order to analyze the topology, since it makes unnecessary to distinguish between indecomposable sets associated to finite or infinite $\Omega$.
%Concretely, we will not conceive the future completion as in \eqref{eq:14}, but in this alternative way:
%\begin{equation}
%  \label{eq:15}
%  \hat{V}\equiv M_1^C\times B(M_2),
%\end{equation}
%that is, any future indecomposable set $P$ will be identified to a pair $(x_1^*,b_{c_2^2})\in M_1^C\times B(M_2)$, and so $P\equiv (x_1^*,b_{c_2^2})$.


\subsubsection{Topological Structure}

 Next, we are going to extend previous study to a topological level, showing that the bijection obtained above is actually a homeomorphism when the corresponding product topology on $M_1^C\times (B(M_2)\cup \{\infty\})$ is considered.

 To this aim, we only need to prove the following equivalence:
%In this last subsection we show that the identification \eqref{eq:15} can be extended to the topological level, assumed that $M_1^C\times B(M_2)$ is endowed with the product topology. To this aim, we are going to prove the following equivalence:
given $P\equiv (x_1^*,b_{c_2})\in \hat{V}$ and $\{P_n\}_n\equiv \{(x_1^n,b_{c_2^n})\}_n\subset \hat{V}$,
\begin{equation}\label{equ}
P\in \hat{L}(\{P_n\}_n)\iff  x_1^n\rightarrow x^*_1\;\;\hbox{and}\;\; b_{c_2}\in \hat{L}(\{b_{c_2^n}\}_n).
  \end{equation}
  Under the hypothesis of $M_1^C$ and $M_2^C$ being locally compact, the equivalence (\ref{equ}) for the case $b_{c_2}\equiv d_{(\Omega,x_2)}$ is already proved in Prop. \ref{topcurvasfinitas}. In fact, if $P_n=I^-(\gamma^n)$ with $\gamma^n:[\omega,\Omega_n)\rightarrow V$, then $\Omega_n<b$ for $n$ big enough. In particular, $b_{c_2^n}\equiv d_{(\Omega_n,x^n_2)}$ with $x_2^n\in M_2^C$ (see \eqref{eq:46}). Moreover, Prop. \ref{topcurvasfinitas} implies that $(\Omega_n,x_1^n,x_2^n)\rightarrow (\Omega,x^*_1,x^*_2)$. Hence, $\{d_{(\Omega_n,x_2^n)}\}_n$ converges pointwise to $d_{(\Omega,x_2)}$, and thus, $d_{(\Omega,x_2)}\in \hat{L}(\{d_{(\Omega_n,x_2^n)}\}_n)$ (see Prop. \ref{prel:PropToponefibre}). So, to finish the proof of (\ref{equ}), we can focus just on the case $b_{c_2}\in {\cal B}(M_2)$.

  We begin with some preliminary results.
  \begin{lemma}\label{lemma:aux5}
    Let $P,P'\in \hat{V}$ and $\{P_n\}_n\subset \hat{V}$, and assume that $P\equiv (x_1,b_{c_2}), P'\equiv (x_1',b_{c'_2})$ and $P_n\equiv (x_1^n,b_{c_2^n})$ belong to $M_1^C\times \left(B(M_2)\cup \{\infty\}\right)$ for all $n$ (recall the identification in \eqref{eq:10}). Then, the following statements hold:
    \begin{itemize}
    \item[(i)] If $x_1=x_1'$, then
        \[
b_{c_2}\leq b_{c_2'} \iff P\subset P'.
        \]
    \item[(ii)] If $x_1^n\rightarrow x_1$, then
      \[
P\subset {\mathrm LI}(\{P_n\}_n) \iff b_{c_2}\leq {\mathrm lim\, inf}_n (\{b_{c_2^n}\}_n).
        \]
    \end{itemize}
  \end{lemma}
  \begin{proof} Let $\gamma:[\omega,\Omega)\rightarrow V$, $\gamma':[\omega',\Omega')\rightarrow V$ and $\gamma^n:[\omega^n,\Omega^n)\rightarrow V$ be future-directed timelike curves generating $P,P'$ and $P_n$, resp.

    (i) First, let us prove the implication to the left. Assume that $\gamma(t)=(t,c_1(t),c_2(t))$ and $\gamma'(t)=(t,c_1'(t),c_2'(t))$ satisfy that $c_{1}(t) \rightarrow x_{1}$, $c_{1}'(t) \rightarrow x_{1}$ and $b_{c_{2}}$, $b_{c_{2}'}$ are their Busemann functions. Consider the future-directed timelike curves $\sigma(t)=(t,c_2(t))$ and $\sigma'(t)=(t,c_2'(t))$ in the Generalized Robertson-Walker spacetime $$\left( (a,b)\times M_2,-dt^2+\alpha_2g_2\right).$$ Since $P\subset P'$, necessarily $P(b_{c_2})=I^-(\sigma)\subset I^-(\sigma')=P(b_{c_2'})$, and thus, $b_{c_2}\leq b_{c_2'}$ (recall \eqref{eq:27} and \eqref{eq:28}).

For the implication to the right, assume that $x_1=x_1'$ and $b_{c_2}\leq b_{c'_2}$. It suffices to show the existence of a sequence $\sigma=(t_n,y_1^n,c_2(t_n))$ with $\{t_n\}_n\nearrow \Omega$, satisfying $\{y_1^n\}_n\rightarrow x_1$ and $\sigma\subset P'$. In fact, in this case, Lemma \ref{lemma:aux3} ensures that $P\subset {\rm LI}(\sigma)$ and, taking into account that $\sigma\subset P'$, necessarily $P\subset P'$.

To this aim, take $\{t_n\}_n\nearrow \Omega$ and observe that, by hypothesis, $b_{c_2}\leq b_{c'_2}$. So, in the Generalized Robertson-Walker spacetime $\left( (a,b)\times M_2,-dt^2+{\alpha_2}g_2\right)$, the inclusion $P(b_{c_2})\subset P(b_{c'_2})$ holds (recall equations \eqref{eq:27} and \eqref{eq:28}). In particular, since the future-directed timelike curves $\sigma(t)=(t,c_2(t))$ and $\sigma'(t)=(t,c'_2(t))$ satisfy $I^-(\sigma)=P(b_{c_2})$ and $ I^-(\sigma')=P(b_{c'_2})$, there exists a sequence $\{s_n\}_n$, with $\{s_n\}_n\nearrow \Omega'$, such that
$\sigma(t_n)=(t_n,c_2(t_n))\ll (s_n,c'_2(s_n))=\sigma'(s_n)$. Let us show that $\sigma=\{(t_n,c_1'(s_n),c_2(t_n))\}$ is the required sequence. From construction and the fact that $(t_n,c_1'(s_n),c_2(t_n))\ll (s_n,c_1'(s_n),c'_2(s_n))$ in $V$ for all $n$, necessarily $\sigma\subset P'$. Moreover, since $\{s_n\}_n\nearrow \Omega'$, necessarily $c_1'(s_n)\rightarrow x_1'=x_1$, as desired.

\smallskip

(ii) For the implication to the right, assume that $P\subset {\mathrm LI}(\{P_n\}_n)$ and let us show that $b_{c_2}\leq \liminf (\{b_{c_2^n}\}_n)$. Denote by $\sigma(t)=(t,c_2(t))$ and $\sigma_n(t)=(t,c_2^n(t))$ future-directed timelike curves in the Generalized Robertson Walker model \[\left((a,b)\times M_2,-dt^2+ \alpha_2g_2\right).\] Since $P\subset {\mathrm LI}(\{P_n\}_n)$, necessarily
  \[
P(b_{c_2})=I^-(\sigma)\subset {\mathrm LI}(\{I^-(\sigma_n)\}_n)={\mathrm LI}(\{P(b_{c_2^n})\}_n)
    \]
(where we are considering past sets in the associated Generalized Robertson Walker model), and the conclusion follows from \eqref{eq:50}.

% . Our aim is to show that $b_{c_2}\leq \liminf (\{b_{c_2^n}\}_n))$, or equivalently, that if we have $x_2^o\in M_2$ and $r\in\R$ such that

% \begin{equation}
%   \label{eq:23}
% r< b_{c_2}(x_2^o)=\lim_{t\rightarrow \Omega} \left(\int_{\C}^{t}\frac{1}{\sqrt{\alpha_2(s)}}ds - d(x_2^o,c_2(t))\right),
% \end{equation}
% then
%   \[
% r< b_{c^n_2}(x_2^0)\quad\hbox{for $n$ big enough.}
%     \]
% Take $t^o$ and $x_2^o$ satisfying \eqref{eq:23}. For $t>t^o$ big enough it follows that:
%   \[
% \int_{\C}^{t^o}\frac{1}{\sqrt{\alpha_2(s)}}ds<\int_{\C}^{t}\frac{1}{\sqrt{\alpha_2(s)}}ds - d(x_2^0,c_2(t))\Rightarrow \int_{t^o}^{t}\frac{1}{\sqrt{\alpha(s)}}ds>d_2(x_2^o,c_2(t)).
% \]
%  Since this last inequality is strict, we can take positive constants $\mu_1,\mu_2>0$, with $\mu_1+\mu_2=1$, and $x_1^o\in M_1$ close enough to $c_1(t)$, such that
% \[
% \Integral{t^o}{t}{\mu_i}{i}{\mu_k}>d_i(x_i^o,c_i(t)) \qquad \hbox{for $i=1,2$.}
%   \]
%   From Prop. \ref{c0}, $q=(t^o,x_1^o,x_2^o)\in I^-(\gamma)=P$. Since $P\subset {\mathrm LI}(\{P_n\})$, there exists $n_0$ such that $q\in P_n$ for $n\geq n_0$. Again from Prop. \ref{c0}, there exist positive constants $\mu_1^n,\mu_2^n>0$, with $\mu_1^n+\mu_2^n=1$, and $t_n\in [\omega,\Omega_n)$, such that
%   \[
% \Integral{t^o}{t_n}{\mu^n_i}{i}{\mu^n_k}>d_i(x_i^o,c_i^n(t_n)) \qquad \hbox{for $i=1,2$}.
%     \]
% In particular, for $i=2$,
%     \[
%       \begin{array}{c}
% \displaystyle\int_{t^o}^{t_n}\frac{1}{\sqrt{\alpha_2(s)}}ds > \Integral{t^o}{t^n}{\mu^n_2}{2}{\mu^n_k}>  d_2(x_2^o,c_2^n(t_n))\Rightarrow\\  \\ \Rightarrow \displaystyle \int_{\C}^{t^o}\frac{1}{\sqrt{\alpha_2(s)}}ds< \int_{\C}^{t_n}\frac{1}{\sqrt{\alpha_2(s)}}ds-d_2(x_2^o,c_2^n(t_n)).
%       \end{array}
%       \]
%       Now, from the increasing character of this last term, as described in \eqref{eq:25}, it follows that
%       \[
% \int_{\C}^{t^o}\frac{1}{\sqrt{\alpha_2(s)}}<b_{c_2^n}(x_2^o)\quad\hbox{for $n\geq n_0$}.
%         \]
%         In conclusion, $b_{c_2}\leq {\mathrm lim\, inf}_n( \{b_{c_2^n}\}_n)$, as desired.

        \smallskip

        For the implication to the left, assume that $b_{c_2}\leq {\mathrm lim\, inf}_n (\{b_{c_2^n}\}_n)$ and let us prove that $P\subset {\mathrm LI}(\{P_n\}_n)$. Let $\{t_k\}\nearrow \Omega$ be an arbitrary sequence. For each $k$, and from the timelike character of $\gamma$, we have $(t_k,c_2(t_k))\ll (t,c_2(t))$ in the Generalized Robertson-Walker spacetime $\left( (a,b)\times M_2,-dt^2+{\alpha_2}g_2\right)$ for all $t>t_k$. From \eqref{eq:26} and the increasing character of \eqref{eq:25},

      \[
\int_\C^{t_k}\frac{1}{\sqrt{\alpha_2(s)}}ds < b_{c_2}(c_2(t_k))=\lim_{t\rightarrow \Omega} \left(\int_\C^t \frac{1}{\sqrt{\alpha_2(s)}}ds - d_2(c_2(t_k),c_2(t))\right).
        \]
Since $b_{c_2}\leq {\mathrm lim\,inf}(\{b_{c_2^n}\}_n)$, there exists an increasing sequence $\{n_k\}_k$ such that
\begin{equation}
  \label{eq:16}
\int_\C^{t_k}\frac{1}{\sqrt{\alpha_2(s)}}ds<b_{c^n_2}(c_2(t_k))=\lim_{r\rightarrow \Omega_n} \left(\int_\C^{r}\frac{1}{\sqrt{\alpha_2(s)}}ds- d_2(c_2(t_k),c_2^n(r))\right)\quad\hbox{$\forall$ $n\geq n_k$.}
\end{equation}
  For each $n_k\leq n<n_{k+1}$, consider $r_n\in [\omega_{n},\Omega_n)$ such that
  \begin{equation}
    \label{eq:24}
\int_\C^{t_k}\frac{1}{\sqrt{\alpha_2(s)}}ds< \int_\C^{r_n}\frac{1}{\sqrt{\alpha_2(s)}}ds-d_2(c_2(t_k),c_2^n(r_n)),\qquad d_{1}(c_1^n(r_n),x_1^n)<\frac{1}{2^n},
  \end{equation}
    (for the first inequality recall \eqref{eq:16}; for the second one, recall that $c_1^n(t)\rightarrow x_1^n$). From the first inequality, it follows that
    \[
    (t_k,c_2(t_k))\ll (r_n,c_2^n(r_n))\quad\hbox{for $n_k\leq n< n_{k+1}$ and all $k$.}
    \]
    However, since $\{(t_k,c_2(t_k))\}$ is a chronological chain, previous chronological relation is true for all $n\geq n_k$: in fact, if $n\geq n_k$, there exists $k'(\geq k)$ such that $n_{k'}\leq n < n_{k'+1}$. As we have noted before $(t_{k'},c_2(t_{k'}))\ll (r_n,c_2^n(r_n)))$ but, taking into account $(t_k,c_2(t_k))\ll (t_{k'},c_2(t_{k'}))$, necessarily $(t_{k},c_2(t_{k}))\ll (r_n,c_2^n(r_n)))$.

    Next, define the sequence $\sigma=\{(l_n,c^n_1(r_n),c_2(l_n))\}_n$, where $l_n:=t_k$ if $n_{k}\leq n< n_{k+1}$. Since $\{t_k\}_k\nearrow \Omega$, necessarily $\{l_n\}_n\rightarrow \Omega$. Moreover, since $(t_k,c_2(t_k))\ll (r_n,c_2^n(r_n))$,
    \[(l_n,c^n_1(r_n),c_2(l_n))=(t_k,c_1^n(r_n),c_2(t_k))\ll (r_n,c_1^n(r_n),c^n_2(r_n))=\gamma^n(r_n),\] hence $(l_n,c^n_1(r_n),c_2(l_n))\in P_n$ for all $n$. Finally, note that $\sigma$ satisfies the conditions of Lemma \ref{lemma:aux3}, as $\{l_n\}_n\rightarrow \Omega$ and $c_1^n(r_n)\rightarrow x_1$ (recall that $c_1^n(t)\rightarrow x_1^n$, $x_1^n\rightarrow x_1$ from hypothesis and the second inequality in \eqref{eq:24}). Therefore,
\[
    P\subset {\mathrm LI}(\{I^-(l_n,c^n_1(r_n),c_2(l_n))\}_n)\subset {\mathrm LI}(\{P_n\}),
    \]
    as desired.
  \end{proof}

  %After previous technical result we are able to prove the equivalence between topologies. Let us begin with:


  \begin{prop}\label{prop:topcharac} Let $P\in \hat{V}$ and $\{P_n\}_n\subset \hat{V}$, and assume that $P\equiv (x_1,b_{c_2})$ and $P_n\equiv (x_1^n,b_{c_2^n})$ (in $M_1^C\times \left(B(M_2)\cup \{\infty\} \right)$) for all $n$. Then,
$P\in \hat{L}(\{P_n\}_n)$ if, and only if, $x_1^n\rightarrow x_1$ and $b_{c_2}\in \hat{L}(\{b_{c_2^n}\}_n)$.
  \end{prop}
  \begin{proof}
    For the implication to the right, and reasoning as in the proof of Thm. \ref{futurestructurefiniteconditions}, it follows that $x_1^n\rightarrow x_1$ (recall the finite warping integral in \eqref{eq:9} and Remark \ref{rem:1}). Hence, we will focus on $b_{c_2}\in \hat{L}(\{b_{c_2^n}\})$. From Lemma \ref{lemma:aux5} and the fact that $P\in \hat{L}(\{P_n\}_n)$, necessarily $b_{c_2}\leq \liminf(\{b_ {c_2^n}\}_n)$. So, $b_{c_2}\in \hat{L}(\{b_{c_2^n}\})$ follows if we prove that $b_{c_2}$ is maximal into $\limsup(\{b_{c^n_2}\}_n)$. Consider any $b_{\overline{c}_2}$ such that $b_{c_2}\leq b_{\overline{c}_2}\leq {\mathrm lim\,sup}(\{b_{c^n_2}\}_n)$, and consider the associated past set $\overline{P}\equiv (x_1,b_{\overline{c}_2})$. Up to a subsequence, we can assume that $b_{\overline{c}_2}\leq {\mathrm lim\,inf}(\{b_{c^n_2}\}_n)$. From Lemma \ref{lemma:aux5}, $P\subset \overline{P}$ and $\overline{P}\subset {\mathrm LI}(\{P_{n}\}_n)$. But $P$ is maximal into the superior limit of the sequence $\{P_n\}_n$, so necessarily $P=\overline{P}$. From Prop. \ref{prop:conddiferbordedif} we have that $b_{c_2}=b_{\overline{c}_2}$ so the maximal character of $b_{c_2}$ into $\limsup(\{b_{c_2^n}\}_n)$ is obtained.

    \smallskip

    For the implication to the left, first note that $P\subset {\mathrm LI}(\{P_n\}_n)$ (recall Lemma \ref{lemma:aux5} and the definition of $\hat{L}$ for Busemann functions \eqref{eq:22}). So, we only need to focus on the maximal character of $P$ into ${\mathrm LS}(\{P_n\})$. Take $\overline{P}$ an indecomposable past set with $P\subset \overline{P}$ and maximal into ${\mathrm LS}(\{P_n\})$, and let us prove that $P=\overline{P}$. Assume that $\overline{P}\equiv (\overline{x}_1,b_{\overline{c}_2})$. Up to a subsequence, we can also assume that $\overline{P}\subset {\mathrm LI}(\{P_n\})$, hence $\overline{P}\in \hat{L}(\{P_n\})$. Hence, from previous part, $x_1^n\rightarrow \overline{x}_1$. But, by hypothesis, $x_1^n\rightarrow x_1$, obtaining that $x_1=\overline{x}_1$. Once this is observed, Lemma \ref{lemma:aux5} ensures both, $b_{c_2}\leq b_{\overline{c}_2}$ and $b_{\overline{c}_2}\leq \limsup(\{b_{c_2^n}\})$. Since $b_{c_2}\in \hat{L}(\{b_{c_2^n}\})$, necessarily $b_{c_2}=b_{\overline{c}_2}$, and so, $P=\overline{P}$ (recall Prop. \ref{samecondsamepast}).
  \end{proof}

\noindent Summarizing, we are in conditions to deduce the following result:

\begin{thm}\label{futurecomploneinfinite}
  Let $(V,g)$ be a  {\multiwarped} spacetime as in \eqref{eq:1-aux}, and assume that the integral conditions in \eqref{eq:9} are satisfied. If $M_1^C$ and $M_2^C$ are locally compact, the bijection (\ref{eq:10}) becomes a homeomorphism.
\end{thm}
\begin{proof}
  From Prop. \ref{topcurvasfinitas}, the bijection between $\hat{V}\setminus \hat{\partial}^{\B} V$ and $(a,b) \times M_1^C\times M_2^C$ is a homeomorphism if we assume that $M_1^C$ and $M_2^C$ are locally compact. From Prop. \ref{prop:topcharac}, the homeomorphism can be extended to the bijection (\ref{eq:10}).
\end{proof}

\section{The past c-completion of doubly warped spacetimes}\label{ss6}

Obviously, similar arguments provide the corresponding results for the past c-completion:

\begin{thm}\label{pfuturestructurefiniteconditions}
  Let $(V,g)$ be a {\multiwarped} spacetime as in (\ref{eq:1-aux}), and assume that the integral conditions
  \begin{equation}
  \label{eqq:7}
  \int_{a}^{\C}\frac{1}{\sqrt{\alpha_i(s)}}ds<\infty, \qquad \hbox{$i=1,2$}\quad\hbox{for some $\C\in (a,b)$.}
\end{equation}
 hold. Then, there exists a bijection
  \begin{equation}
    \label{eqq:8}
    \check{V}\; \leftrightarrow \; [a,b) \times M_1^C\times M_2^C
  \end{equation}
  which maps each IF $F\in \check{V}$ to the limit point $(\Omega,x_1,x_2)\in [a,b)\times M_1^C\times M_2^C$ of any past-directed timelike curve generating $F$. Moreover, if $M_1^C$ and $M_2^C$ are locally compact, then this bijection becomes a homeomorphism.
\end{thm}


\begin{thm}\label{pfuturecomploneinfinite}
  Let $(V,g)$ be a  {\multiwarped} spacetime as in \eqref{eq:1-aux}, and assume that the integral conditions
  \begin{equation}
  \label{eqq:9}
 \int_{a}^{\C}\frac{1}{\sqrt{\alpha_1(s)}}ds<\infty \qquad \hbox{and}\qquad \int_{a}^{\C}\frac{1}{\sqrt{\alpha_2(s)}}ds=\infty,
\end{equation}
hold. Then, there exists a bijection
 \begin{equation}
   \label{eqq:10}
     \check{V}\; \leftrightarrow\;  M_1^C\times \left(B(M_2)\cup \{-\infty\}\right)
     \equiv  \left( (a,b)\times M_1^C\times M_2^C\right) \cup M_{1}^{C} \times \left({\cal B}(M_2)\cup \{\infty\}\right).
  \end{equation}
  This bijection maps each indecomposable future set $F=I^+(\gamma)\in \check{V}$, where $\gamma:[\omega,-\Omega)\rightarrow V$, $\gamma(t)=(-t,c_1(t),c_2(t))$, is any curve generating $F$, to a pair $(x_1^*,b^-_{c_2})$, where $x_1^*\in M_1^C$ is the limit point of the curve $c_1$. If $-\Omega>a$, then  $b^-_{c_2}=d^-_{(\Omega,x_2^*)}$, where $x_2^*$ is the limit point of $c_2$ (see \eqref{eq:48}), and thus, $F$ can be also identified with the limit point $(\Omega,x_1^*,x_2^*)$ of $\gamma$.
  % to the limit point of any past-directed timelike curve generating it, which is either $(\Omega,x_1^*,x_2^*)\in \R\times M_1^C\times M_2^C$ if $\Omega>-\infty$ or $(\Omega,x_1^*,b_{c_2})\in \{-\infty\}\times M_1^C\times {\cal B}(M_2)$ if $\Omega=-\infty$. Moreover, if $M_1^C$ and $M_2^C$ are locally compact, the bijection (\ref{eq:10}) becomes a homeomorphism.
\end{thm}

%.....................................................
%
%
%Obviously, all previous results have a past analog, which we briefly review as follows:
%
%\begin{thm}\label{pastcompletion}
%  Let $(V,g)$ be a {\multiwarped} model as in \eqref{eq:1}. Then,
%
%  \begin{equation}
%    \label{eq:39}
%    \check{V}\setminus \check{\partial}^{\infty} V\equiv \R\times M_1^C\times M_2^C,
%  \end{equation}
%  where $\check{\partial}^{\infty}V$ denotes the TIFs determined by past-directed timelike curves with divergent temporal component. Moreover,
%  \begin{itemize}
%  \item If
%    \begin{equation}
%      \label{eq:40}
%      \int^{0}_{-\infty}\frac{1}{\sqrt{\alpha_i(s)}}ds<\infty, \qquad \hbox{for $i=1,2$.}
%    \end{equation}
%    then the past causal boundary and completion has the following structure
%    \begin{equation}
%      \label{eq:41}
%      \check{V}\equiv [a,b)\times M_1^C\times M_2^C.
%    \end{equation}
%That is, any indecomposable future set $F\in \check{V}$ can be labelled by a triple $(\Omega,x_1,x_2)\in [a,b)\times M_1^C\times M_2^C$. Moreover, if $M_1^C,M_2^C$ are locally compact, then previous identifications can be extended to the topological level.
%  \item If
%    \begin{equation}
%      \label{eq:42}
%       \int^{0}_{-\infty}\frac{1}{\sqrt{\alpha_1(s)}}ds<\infty \qquad \hbox{and}\qquad \int^{0}_{-\infty}\frac{1}{\sqrt{\alpha_2(s)}}ds=\infty.
%    \end{equation}
%    then we obtain
%    \begin{equation}
%      \label{eq:43}
%      \begin{array}{c}
%     \check{V}\equiv M_1^C\times \left(B(M_2)\cup \{-\infty\}\right),\qquad \check{V}\setminus \check{\partial}^\infty V\equiv \cambios{(a,b)}\times M_1^C\times M_2^C\\
%
%     \\
%\cambios{
%     \begin{array}{rl}
%       \check{\partial}^{\infty}V\equiv  & M_1^C\times \left({\cal B}(M_2)\cup \{-\infty\}\right)\\ \equiv & M_1^C\times \left(\left(\R\times \partial_{\cal B}M_2\right) \cup \{-\infty\}   \right)
%     \end{array}
%}
%\end{array}
%    \end{equation}
%    and if $M_1^C,M_2^C$ are locally compact, previous identifications extend to the topological level.
%  \end{itemize}
%\end{thm}




%%% Local Variables:
%%% mode: latex
%%% TeX-master: "DoublyWarpedBoundary2017"
%%% End:


%\input{PartialBoundaryTop}

\section{The total c-completion of doubly warped spacetimes}
\label{sec:totalcompletion}
%\subsection{Total c-boundary as a point set}

We are now in conditions to construct the (total) c-completion of doubly warped spacetimes by merging appropriately the future and past c-boundaries obtained in previous section.

To this aim, first we need to determine the S-relation between indecomposable sets. So, let $\gamma:[\omega,\Omega)\rightarrow V$, $\gamma(t)=(t,c_{1}(t),c_{2}(t))$, be an inextensible future-directed timelike curve.
%Finally, we are going to study the point set, causal and topological structure for the total c-completion of {\multiwarped} spacetimes from the corresponding structures for the future (and past) partial boundaries obtained in previous sections. Of course, these structures will depend on integral conditions in both directions (i.e., from $0$ to $\pm\infty$) involving the warping functions $\alpha_i$, $i=1,2$.
%
%\begin{align}
%  \int^{0}_{-\infty}\frac{1}{\sqrt{\alpha_i(s)}}ds<\infty, \qquad \hbox{for $i=1,2$.}\hspace{1.5cm}\label{eq:20}\\
%   \int_{-\infty}^{0}\frac{1}{\sqrt{\alpha_1(s)}}ds<\infty \qquad \hbox{and}\qquad \int^{0}_{-\infty}\frac{1}{\sqrt{\alpha_2(s)}}ds=\infty. \label{eq:21}
%\end{align}
%
%\smallskip
Clearly, if $\Omega=b$ then $\uparrow I^{-}(\gamma)=\emptyset$, and there are no IFs S-related to $I^{-}(\gamma)$. So, we will focus on the case $\Omega<b$.

\begin{prop}
\label{tip}
Let $(V,g)$ be a \multiwarped spacetime and consider a future-directed (resp. past-directed) timelike curve $\gamma$ with associated endpoint $(\Omega^+,x_1^*,x_2^*) \in (a,b) \times M_{1}^{C} \times M_{2}^{C}$  (resp. $(\Omega^-,y_1^*,y_2^*) \in (a,b) \times M_{1}^{C} \times M_{2}^{C}$). Then
\begin{equation}
\begin{aligned}
        \uparrow I^{-}(\gamma) &=\{(t,x_1,x_2) \in V\; \mid\; \exists \,  \mu_{1},\mu_{2} > 0\;\; \hbox{{\rm such that}} \\
        & \Integral{\Omega^+}{t}{\mu_{i}}{i}{\mu_{k}}> d_{i}(x_{i},x_{i}^*),\; i=1,2.\}%\quad\hbox{pointwise and topologically.}
\end{aligned}
\end{equation}

\begin{equation*}
\begin{aligned}
(\hbox{resp.}\;\;\downarrow I^{+}(\gamma) &=\{(t,x_1,x_2) \in V\; \mid\; \exists \,  \mu_{1},\mu_{2} > 0\;\; \hbox{{\rm such that}} \\
                & \Integral{t}{\Omega^-}{\mu_{i}}{i}{\mu_{k}}> d_{i}(x_{i},y^{*}_{i}),\; i=1,2\}).%\quad\hbox{pointwise and topologically.}
\end{aligned}
\end{equation*}
As consequence, if $P\in\hat{V}$ and $F\in\check{V}$ are associated to $(\Omega^+,x^*_1,x^*_2)$ and $(\Omega^-,y^*_1,y^*_2)$ in $(a,b) \times M_{1}^{C} \times M_{2}^{C}$, resp, then the following equivalence holds:
\[
P \sim_{S} F\quad \Longleftrightarrow\quad
\Omega^{-}=\Omega^{+}\;\;\hbox{and}\;\; x^*_{i}=y^*_{i} \in M_{i}^{C},\; i=1,2.
\]
\end{prop}
\begin{proof} Assume that $\gamma:[\omega,\Omega^+) \rightarrow V$, $\gamma(t)=(t,c_1(t),c_2(t))$, is a future-directed causal curve with associated endpoint $(\Omega^+,x_1^*,x_2^*) \in (a,b) \times M_{1}^{C} \times M_{2}^{C}$ (for the past is analogous). We need to show that $\uparrow I^-(\gamma)=A_{(\Omega,x_1^*,x_2^*)}$, where
\begin{equation*}
\begin{aligned}
A_{(\Omega^+,x_1^{*},x_2^*)}&:=\{(r,x_1,x_2) \in V \mid \exists \,  \mu_{1},\mu_{2} > 0\;\; \hbox{{\rm such that}} \\
        &
        \int_{\Omega^+}^{r}\frac{\sqrt{\mu_{i}}}{\alpha_i(s)}\left(\sum_{k=1}^2\frac{\mu_k}{\alpha_k(s)}\right)^{-1/2}dt > d_{i}(x_{i},x_{i}^*),\; i=1,2\}.%\quad\hbox{pointwise and topologically.}
\end{aligned}
\end{equation*}
For the inclusion to the right, take $(r,x_1,x_2) \in \uparrow I^{-}(\gamma)$ and $\epsilon>0$ small enough so that $(r-\epsilon,x_1,x_2)\in \uparrow I^-(\gamma)$ (recall that the common future is open). For any sequence $\{t_{n}\}_{n}\nearrow \Omega^+$ we have $\gamma(t_n) \ll (r-\epsilon,x_1,x_2)$ for all $n$. From Prop. \ref{c0} there exist constants $\mu^n_{1},\mu^n_{2}>0$, with $\mu^n_{1}+\mu_{2}^n=1$ for all $n$, such that
\[
\int_{t_n}^{r-\epsilon}\frac{\sqrt{\mu^n_{i}}}{\alpha_i(s)}\left(\sum_{k=1}^2\frac{\mu_k^n}{\alpha_k(s)}\right)^{-1/2}dt > d_{i}(x_i,c_{i}(t_n))\quad i=1,2.
\]
Then, by the standard limit process, we deduce the following inequalities:
\[
\int_{\Omega^+}^{r-\epsilon}\frac{\sqrt{\mu^*_{i}}}{\alpha_i(s)}\left(\sum_{k=1}^2\frac{\mu_k^*}{\alpha_k(s)}\right)^{-1/2}dt \geq d_{i}(x_{i},x_{i}^{*}),\quad i=1,2,
\] where $\mu_i^*$ is the limit (up to a subsequence) of $\{\mu_i^n\}$. Now observe that some of previous inequalities become strict if we replace $r-\epsilon$ by $r$. So, a small variation of $\mu_1^*$ and $\mu_2^*$ if necessary (concretely, if one of these constants is zero), provides
% If, either $\mu^*_i=0$ for some $i=1,2$ or some inequality is actually an equality, a small variation of $\mu^*_{1},\mu^*_{2}$ ensures the existence of
positive constants $\mu_{1}',\mu_{2}'>0$ satisfying
\[
\int_{\Omega^+}^{r}\frac{\sqrt{\mu'_{i}}}{\alpha_i(s)}\left(\sum_{k=1}^2\frac{\mu'_k}{\alpha_k(s)}\right)^{-1/2}dt > d_{i}(x_{i},x_{i}^{*}),\quad i=1,2.
\]
In particular, $(r,x_1,x_2)\in A_{(\Omega,x_1^{*},x_2^*)}$, and so, $\uparrow I^{-}(\gamma) \subset A_{(\Omega,x_1^{*},x_2^*)}$.

\smallskip

 For the inclusion to the left, assume that $(r,x_1,x_2) \in A_{(\Omega,x_1^*,x_2^*)}$. By the continuity of both, the integral with respect to the lower limit of integration and the distance function, and the convergence of $\gamma(t)=(t,c_1(t),c_2(t))$ to
$(\Omega,x_1^{*},x_2^*)$, we deduce that
\[
\int_{t}^{r}\frac{\sqrt{\mu_{i}}}{\alpha_i(s)}\left(\sum_{k=1}^2\frac{\mu_k}{\alpha_k(s)}\right)^{-1/2} > d_{i}(x_{i},c_{i}(t))\quad\hbox{for large $t$.}
\]
So, from Prop. \ref{c0}, $\gamma(t) \ll  (r,x_1,x_2)$ for  all $t$, which implies $(r,x_1,x_2) \in \uparrow I^{-}(\gamma)$.

\smallskip

For the last assertion, assume that $P$ is associated to $(\Omega^+,x^*_1,x^*_2)\in (a,b)\times M_1^C\times M_2^C$. From the first part of this proposition, $\uparrow P= I^+(\sigma)$, where $\sigma$ is a past-directed timelike curve converging to $(\Omega^+,x^*_1,x^*_2)$. So, $F=I^+(\sigma)$ is the unique maximal IF into the common future of $P$. Reasoning analogously we deduce that $P$ is the unique maximal IP into the common past of $F$. In conclusion, $P$ is $S$-related just with the indecomposable future set $F$, and vice versa.
\end{proof}

%Now, we are ready to give the following proper characterization of the S-relation between terminal sets.




% \noindent {\it Proof:} From Prop. \ref{tip}, if $P$ and $F$ are associated to the same point $(\Omega,x_{1}^{*},x_{2}^{*}) \in \mathbb{R} \times M_{1}^{c} \times M_{2}^{c}$ then $P=\downarrow F$ and $F=\uparrow P$, and thus, $P \sim_{s} F$.

% Next, let us show that $P \sim_{S} F$ implies $(\Omega^{+},x^{P})=(\Omega^{-},x^{F})$. Since $\uparrow P=I^{+}(\gamma_{P})$ for some inextensible past directed causal curve $\gamma_{P}$ with past endpoint $(\Omega^{+},x^{P})$ and $\downarrow F=I^{-}(\gamma_{F})$ for some inextensible future directed causal curve $\gamma_{F}$ with future endpoint $(\Omega^{-},x^{F})$, necessarily $\uparrow P $ and $\downarrow F$ are indecomposable past and future sets, respectively. On the other hand, since $P \subset \downarrow F=I^{-}(\gamma_{F})$ and $F \subset \uparrow P=I^{+}(\gamma_{P})$, the S-relation
% $P \sim_{S} F$ implies that
% $P=I^{-}(\gamma_{F})$ and $F=I^{+}(\gamma_{P})$; in fact, otherwise maximality of $P$ and $F$ as an IP and IF is not satisfied. Finally, since $P=I^{-}(\gamma_{F})$ and $F=I^{+}(\gamma_{P})$ are univocally determined by $(\Omega^{+},x^{P})$ and $(\Omega^{-},x^{F})$ (see Prop. \ref{pp1}), then $(\Omega^{+},x_{P})=(\Omega^{-},x_{F})$. $\Box$
From this result it is clear that $\overline{V}$ is simple as a point set (see Defn. \ref{simpletop}). On the other hand, if we define
\[
\partial^{\B}V:=\hat{\partial}^{\B}V\cup\check{\partial}^{\B}V,
\]
the following identification is deduced:
\[
\overline{V}\setminus \partial^{\B} V\leftrightarrow (a,b) \times M_1^C\times M_2^C.
  \]
  In particular, $\partial V\setminus \partial^{\B} V$ can be identified with a cone with base $(M_1^C\times M_2^C)\setminus (M_1\times M_2)$. Moreover, if we assume that both $M_1^C,M_2^C$ are locally compact, Prop. \ref{topcurvasfinitas} ensures that previous bijection is a homeomorphism. Particularly, this proves that, given $(P,F)\in \overline{V}\setminus \partial^{\B}V$,
    \[
P\in \hat{L}(\{P_n\}) \iff F\in \check{L}(\{F_n\})
      \]
      for any sequence $\{(P_n,F_n)\}_n\in \overline{V}$. Hence, $\overline{V}\setminus \partial^{\B}V$ is also simple topologically.
      %Prop. \ref{topcurvasfinitas} permits to extend previous bijection to a homeomorphism by assuming that $M_1^C$ and $M_2^C$ are both locally compact.\footnote{J.L.: Revisar esta afirmacion.}

  Finally, the following lemma ensures that the line over each point $(x_1^*,x_2^*)\in (M_1^C\times M_2^C)\setminus (M_1\times M_2)$ is timelike:
 \begin{lemma}\label{causalstructurenoinf}
     If $(P,F),(P',F')\in \partial V\setminus \partial^{\B}V$, with $(P,F)\equiv (\Omega,x^*_1,x^*_2), (P',F')\equiv (\Omega',x^*_1,x^*_2)$ in $(a,b)\times M_1^C\times M_2^C$, satisfy that $a<\Omega<\Omega'<b$ then $(P,F)\ll (P',F')$.
  \end{lemma}
  \begin{proof}
     Take $t=(\Omega+\Omega')/2$ and $\mu_1=\mu_2=1/2$. For $i=1,2$, consider $y_i$ close enough to $x_i^*$ so that
    \[
\left\{\begin{array}{l}
\displaystyle\Integral{t}{\Omega'}{\mu_{i}}{i}{\mu_{k}}>
             d_{i}(y_i,x_i^*) \\

\displaystyle\Integral{\Omega}{t}{\mu_{i}}{i}{\mu_{k}}>
             d_{i}(y_i,x_i^*),

\end{array}\right.
\quad i=1,2.
      \]
    From Prop. \ref{pastofcurve} (and its past analogous) we deduce that $(t,y_1,y_2)\in F\cap P'$, as desired.
  \end{proof}

  \smallskip

  The $S$-relation described in Prop. \ref{tip} implies that each pair $(P,F)\in\overline{V}$ is determined by any of its non-empty components, that is, $\overline{V}$ is simple as a point set. Even more, from Prop. \ref{topcurvasfinitas} and the definition of the chronological limit (see \eqref{eq:29} and \eqref{limcrono}), $\overline{V}$ is topologically simple as well (recall Defn. \ref{simpletop}); concretely, if $(P,F)\in \overline{V}$, $P\neq\emptyset$, and $\sigma=\{(P_n,F_n)\}_n\subset \overline{V}$, then $(P,F)\in L_{chr}(\sigma)$ if, and only if, $P\in \hat{L}_{chr}(\{P_n\}_n)$. Therefore, in order to determine the, pointwise and topological, structure of the (total) $c$-boundary, it suffices to study the partial boundaries. Consequently, we will describe $\overline{V}$ in two different ways, according to our convenience, namely:
  \[
\overline{V}= (a,b) \times M_1^C\times M_2^C\cup\hat{\partial}^{\B}V\cup\check{\partial}^{\B}V=\hat{V}\cup \check{\partial}^{\B}V=\hat{\partial}^{\B}V\cup \check{V}.
    \]
     Restricting conveniently, the open sets of $\overline{V}$ containing a pair $(P,F)$ can be viewed as: (i) open sets in $(a,b) \times M_1^C\times M_2^C$ if $P\neq\emptyset\neq F$, (ii) open sets in $\hat{V}$ if $F=\emptyset$ or (iii) open sets in $\check{V}$ if $P=\emptyset$.

\smallskip

It rests to determine the causal structure of $\overline{V}$. This is contained in the following result, which summarizes all the information about the (total) c-completion of doubly warped spacetimes:
  \begin{thm}\label{thm:main}
    Let $(V,g)$ be a {\multiwarped} spacetime as in \eqref{eq:1-aux}. Then, there exists a homeomorphism
    \[
\overline{V}\setminus \partial^{\B}V \leftrightarrow (a,b) \times M_1^C\times M_2^C,
      \]
    where each line $\{(t,x_1^*,x_2^*): t\in (a,b),\; (x_1^*,x_2^*)\in M_1^C\times M_2^C\}$ is timelike. Moreover:
      %$\partial^{\infty}V$ has the following structure:
     \begin{itemize}
      \item[(i)] If \eqref{eq:7} and \eqref{eqq:7} hold, then $\partial^{\B} V$ is homeomorphic to a couple of spacelike copies of $M_1^C\times M_2^C$. As consequence, we have the following homeomorphism:
        \begin{equation}
          \label{eq:18}
         \overline{V}\equiv [a,b]\times M_1^C\times M_2^C\quad\hbox{pointwise and topologically.}
        \end{equation}

        \item[(ii)] If \eqref{eq:7} and \eqref{eqq:9} hold, then $\partial^{\B} V$ has a copy of $M_1^C\times M_2^C$ for the future, with spatial causal character; and a copy of $M_1^C\times \left({\cal B}(M_2)\cup \{\infty\}\right)$ for the past.  This second set can be seen as a cone with base $M_1^C\times \partial_{\cal B}(M_2)$ generated by horismotic lines over each pair $(x_1^*,[b_{c_2}])$ ending at the point $(x_1^*,\infty)$. As consequence, we have the following homeomorphism
          \[
              \overline{V}\equiv\left\{\begin{array}{l} \hat{V}\cup \check{\partial}^{\B}V \leftrightarrow \left((a,b]\times M_1^C\times M_2^C\right) \cup \left(M_1^C\times \left({\cal B}(M_2)\cup \{\infty\}\right)\right) \\ \hat{\partial}^{\B} V\cup \check{V} \leftrightarrow \left(\{b\}\times M_1^C\times M_2^C\right) \cup \left(M_1^C\times \left(B(M_2)\cup \{\infty\} \right) \right).
            \end{array}\right.
            \]


          \item[(iii)] If \eqref{eq:9} and \eqref{eqq:7} hold we have a structure analogous to (ii), but interchanging the roles of future and past.

            \item[(iv)] If \eqref{eq:9} and \eqref{eqq:9} hold, then $\partial^{\B} V$ has two copies of the space $M_1^C\times \left({\cal B}(M_2)\cup \{\infty\}\right)$, one for the future and the other one for the past, formed by horismotic lines over each point $(x_1^*,[b_{c_2}])\in M_1^C\times \partial_{\cal B}(M_2)$ ending at the point $(x_1^*,\infty)$. As consequence,
              \[
                  \overline{V}\equiv \left\{\begin{array}{l} \hat{V}\cup \check{\partial}^{\B}V \leftrightarrow \left(M_1^C\times \left(B(M_2)\cup \{\infty\} \right)\right) \cup \left(M_1^C\times \left({\cal B}(M_2)\cup \{\infty\}\right)\right) \\
                  \hat{\partial}^{\B} V\cup \check{V} \leftrightarrow \left( M_1^C\times \left({\cal B}(M_2)\cup \{\infty\}\right)\right) \cup  \left(M_1^C\times \left(B(M_2)\cup \{\infty\}\right)\right).
                \end{array}\right.
            \]
      \end{itemize}
    \end{thm}
    \begin{proof}
As we have argued before, the first assertion about the point set topological and causal structure of $\overline{V}\setminus \partial^{\ncambios{b}}V$ is a direct consequence of \cambios{Props. \ref{structuraparcialsininfinito'}, \ref{topcurvasfinitas} (and its past analogous)}, \ref{tip} and Lemma \ref{causalstructurenoinf}. So, we will focus on the rest of assertions.

  \begin{itemize}
  \item[(i)] The point set and topological structure are straightforward from Thms. \ref{futurestructurefiniteconditions} and \ref{pfuturestructurefiniteconditions}. So, we only need to prove that $\partial^{\B}V=\hat{\partial}^{\B}V\cup \check{\partial}^{\B}V$ is spacelike. Take $(P,\emptyset),(P',\emptyset)\in \partial^{\B}V$ two different boundary points (for TIFs is completely analogous). By using the identification in \eqref{eq:18}, we can assume that $(P,\emptyset)\equiv (b,x_1^*,x_2^*)$ and $(P',\emptyset)\equiv (b,y_1^*,y_2^*)$ with $(x_1^*,x_2^*)\neq (y_1^*,y_2^*)$. From the proof of Prop. \ref{structuraparcialsininfinito} (recall also Rem. \ref{rem:1}) it follows both, $P\not \subset P'$ and $P'\not \subset P$, thus $(P,\emptyset)$ and $(P',\emptyset)$ are neither timelike nor lightlike related, i.e., they are spatially related.

  \item[(ii)] The point set and topological structure are deduced from Thm. \ref{futurestructurefiniteconditions} and Thm. \ref{pfuturecomploneinfinite}. For the causal structure, let us take two points $(P,\emptyset),(P',\emptyset)\in \partial^{\B}V$ over the same point $(x_1^*,[b_{c}])\in M_1^C \times \partial_{\cal B} M_2$. Hence, we can make the identifications $(P,\emptyset)\equiv (x_1^*,b_{c_1})$ and $(P',\emptyset)\equiv (x_1^*,b_{c_2})$ with $b_{c_1}-b_{c_2}={\cal K}$, ${\cal K}$ constant. If we assume that ${\cal K}>0$, then $b_{c_1}\geq b_{c_2}$, and so, $P'\subset P$ (recall Lemma \ref{lemma:aux5}), i.e., both points are lightlike related (the case with ${\cal K}<0$ is completely analogous).
  \end{itemize}

Finally, assertions (iii) and (iv) are easily deduced from (i) and (ii).

\end{proof}

\begin{rem} {\rm (1) Of course, the four cases considered in previous theorem do not cover all the possibilities compatible with the finiteness of at most one warping integral (since the finite warping integral may not be necessarily the last one). However, the structure of the c-completion for these additional cases are easily deducible from our approach, and can be considered an easy exercise for the reader.

(2) In order to simplify the exposition, we have considered along this paper multiwarped spacetimes with just two fibers. Nevertheless, the corresponding results for the general case of $n$ fibers can be easily deduced by the reader (see, for instance, Section \ref{sec:applications}).}
\end{rem}

%\subsection{Multiwarped models.}
%\label{sec:multiwarped}
%Up to now, we analyzed the case of just two fibers...
%It is worth noting that, even if we have considered the doubly warped model along the paper, it is possible to obtain analogous results for the corresponding multiwarped models. Concretely, let us assume that $(V,g)$ is a multiwarped model where $\mathcal{V}=(a,b)\times \Sigma_1\times \dots\times \Sigma_n\times \Sigma_{n+1}$ and
%\begin{equation}
%  \label{eq:32}
%\mathfrak{g}=-dt^2+\alpha_1g_1+ \dots + \alpha_ng_n+\alpha_{n+1} g_{n+1},
%\end{equation}
%
%In this models, the characterization of the chronological relation, which is the key for the studies in this paper, reads as (compare with Prop. \ref{c0})
%\begin{prop}
%  Let $(V,g)$ be a multiwarped spacetime as in (\ref{eq:32}), and $(t^{o},x^{o}), (t^{e},x^{e})\in V$ with $x^{o}, x^{e}\in \prod_{i=1}^{n+1}\Sigma_i$ and $x^{o}\neq
%x^{e}$. The following conditions are equivalent:
%\begin{itemize}
%
%\item[(i)]  $(t^{o},x^{o})\ll (t^{e},x^{e})$; or, equivalently, $t^o<T(x^o,(t^e,x^e))$\footnote{Observe that the departure time function do not depend on the specific expression for the metric.} (recall (\ref{e0}));
%\item[(ii)] the departure time function $T(x^o,(t^e,x^e))$ is the unique real value $T\in \R$
%with $t^{o}<T<t^{e}$ such that for some (unique) constants $\mu_{1},\dots,\mu_{n+1} \geq
%0$, with $\sum_{i=1}^{n+1}\mu_{i}=1$, it satisfies
%\begin{equation}\label{e2}
%\Integralm{T}{t^{e}}{\mu_{i}}{i}{\mu_{k}}=d_{i}(x^{o}_{i},x^{e}_{i})\qquad\hbox{for}\;\;
%i=1,\dots,n+1;
%\end{equation}
%
%\item[(iii)] there exist positive constants $\mu'_{1},\dots,\mu'_{n}> 0$, with $\sum_{i=1}^{n+1}\mu'_i=1$,
%such that
%\begin{equation}\label{e2''}
%\Integralm{t^{o}}{t^{e}}{\mu_{i}'}{i}{\mu_{k}'}>
%d_{i}(x^{o}_{i},x^{e}_{i})\qquad\hbox{for $i=1,\dots,n$}.
%\end{equation}
%\end{itemize}
%\end{prop}
%
%Then, the rest of results follow analogously, just taking into account that the integral conditions \eqref{eq:7} and \eqref{eq:9} correspond respectively with
%
%\begin{equation}
%  \label{eq:33}
%  \int_{0}^{\infty}\frac{1}{\sqrt{\alpha_i(s)}}ds<\infty, \qquad \hbox{for $i=1,\dots,n+1$}
%  \end{equation}
%and
%\begin{equation}
%  \label{eq:34}
%  \left\{\begin{array}{l}
%    \displaystyle\int_{0}^{\infty}\frac{1}{\sqrt{\alpha_i(s)}}ds<\infty, \qquad \hbox{for $i=1,\dots,n$}\\
%\\
%   \displaystyle \int_{0}^{\infty}\frac{1}{\sqrt{\alpha_{n+1}(s)}}ds=\infty.
%  \end{array}\right.
%\end{equation}
%while the corresponding integrals for the past  \eqref{eq:40} and \eqref{eq:42} correspond with
%
%\begin{equation}
%  \label{eq:44}
%    \int^{0}_{-\infty}\frac{1}{\sqrt{\alpha_i(s)}}ds<\infty, \qquad \hbox{for $i=1,\dots,n+1$}
%\end{equation}
%and
%\begin{equation}
%  \label{eq:45}
%  \left\{\begin{array}{l}
%    \displaystyle\int^{0}_{-\infty}\frac{1}{\sqrt{\alpha_i(s)}}ds<\infty, \qquad \hbox{for $i=1,\dots,n$}\\
%\\
%   \displaystyle \int^{0}_{-\infty}\frac{1}{\sqrt{\alpha_{n+1}(s)}}ds=\infty.
%  \end{array}\right.
%\end{equation}
%
%
%Concretely, we arrive to the following main result for multiwarped models (compare with Thm. \ref{thm:main} assuming $M_1=\Sigma_1\times\dots\times \Sigma_n$ and $M_2=\Sigma_{n+1}$):
%
%
%\begin{thm}\label{thm:main2}
%   Let $(V,g)$ be a multiwarped spacetime as in \eqref{eq:32}. Then,
%    \[
%\overline{V}\setminus \partial^{\infty}V\equiv \cambios{(a,b)}\times \left(\prod_{i=1}^n  \Sigma_i^C\right)\times \Sigma_{n+1}^C\quad\hbox{pointwise and topologically.}
%      \]
%     Moreover, the line $\{(t,x_1^*,\dots,x_n^*,x_{n+1}^*): t\in\R\}$ is timelike.
%
%      \cambios{The rest of boundary points, that is,}  $\partial^{\infty}V$ has the following structure:
%     \begin{itemize}
%      \item[(i)] If \eqref{eq:33} and \eqref{eq:44} hold, then $\partial^{\infty} V$ correspond to two copies of $\prod_{i=1}^n  \Sigma_i^C\times \Sigma_{n+1}^C$ with spatial causal character. Moreover,
%        \begin{equation}
%         \overline{V}\equiv \cambios{[a,b]}\times \left(\prod_{i=1}^n  \Sigma_i^C\right)\times \Sigma_{n+1}^C\quad\hbox{pointwise and topologically.}
%        \end{equation}
%
%        \item[(ii)] If \eqref{eq:33} and \eqref{eq:45} hold, then $\partial^{\infty} V$ has a copy of $\prod_{i=1}^n  \Sigma_i^C\times \Sigma_{n+1}^C$ for the future, with spatial causal character; and a copy of $\prod_{i=1}^n  \Sigma_i^C\times \left({\cal B}(\Sigma_{n+1})\cup \{\infty\}\right)$ for the past.  This second set can be seen as a cone with base $\prod_{i=1}^n  \Sigma_i^C\times \partial_{\cal B}(\Sigma_{n+1})$ formed by \cambios{horismotic} lines over each pair $(x_1^*,\dots,x_n^*,[b_{c_{n+1}}])$ \cambios{ending at the point $(x_1^*,\dots,x_n^*,\infty)$}. Moreover,
%          \[
%            \begin{array}{rl}
%              \overline{V}\equiv &\hat{V}\cup \check{\partial}^{\infty}V\equiv \left(\cambios{(a,b]}\times \prod_{i=1}^n  \Sigma_i^C\times \Sigma_{n+1}^C \right) \cup \left(\prod_{i=1}^n  \Sigma_i^C\times \left({\cal B}(\Sigma_{n+1})\cup \{\infty\}\right)\right)\\
%
%              \equiv & \hat{\partial}^{\infty} V\cup \check{V} \equiv \left(\cambios{\{b\}}\times \prod_{i=1}^n  \Sigma_i^C\times \Sigma_{n+1}^C\right) \cup \left(\prod_{i=1}^n  \Sigma_i^C\times B(\Sigma_{n+1})\right)
%
%            \end{array}
%            \]
%            both pointwise and topologically.
%
%          \item[(iii)] If \eqref{eq:34} and \eqref{eq:44} hold we have a structure completely analogous to (ii), but interchanging the roles of future and past.
%
%            \item[(iv)] If \eqref{eq:34} and \eqref{eq:45} hold, then $\partial^\infty V$ has two copies of the space $\prod_{i=1}^n  \Sigma_i^C\times \left({\cal B}(\Sigma_{n+1})\cup \{\infty\}\right)$; one for the future and the other for the past, formed by horismotic lines over each point $(x_1^*,\dots,x_n^*,[b_{c_{n+1}}])$  in $\prod_{i=1}^n  \Sigma_i^C\times \partial_{\cal B}(\Sigma_{n+1})$ ending at the point $(x_1^*,\dots,x_n^*,\infty)$. Moreover,
%              \[
%                \begin{array}{rl}
%                  \overline{V}\equiv & \hat{V}\cup \check{\partial}^{\infty}V\equiv \left(\prod_{i=1}^n  \Sigma_i^C\times \left(B(\Sigma_{n+1})\cup \{\infty\} \right)\right) \cup \left(\prod_{i=1}^n  \Sigma_i^C\times \left({\cal B}(\Sigma_{n+1})\cup \{\infty\}\right)\right)\\
%                  \equiv & \hat{\partial}^{\infty} V\cup \check{V} \equiv \left( \prod_{i=1}^n  \Sigma_i^C\times \left({\cal B}(\Sigma_{n+1})\cup \{\infty\}\right)\right) \cup  \left(\prod_{i=1}^n  \Sigma_i^C\times \left(B(\Sigma_{n+1})\cup \{\infty\}\right)\right)
%                \end{array}
%            \]
%
%
%
%      \end{itemize}
%    \end{thm}


\section{Some examples of interest}%\footnote{Jony: TRABAJO EN PROCESO!!!}
\label{sec:applications}

In this section we are going to apply our results to compute the c-completion of some spacetimes of physical interest. Concretely, we will consider some Kasner models, the intermediate region of Reissner-Nordstr\"om and de Sitter models with (non necessarily compact) internal spaces.

% \begin{rem}
%   CUIDADO: El principal problema que veo en esta sección es que mostramos pocos ejemplos que sólo se resuelvan con los resultados de este paper. Fijémonos que:
%   \begin{itemize}
%   \item En el caso del Kasner, sólo estamos incluyendo un caso más.
%   \item En el caso de Reissner-Nordstr\"om tenemos los correspondientes bordes conformes, luego podríamos calcularlo con un paper anterior.
%     \item En el caso de los inflacionarios, el resultado puede deducirse de lo hecho por Harris.
%   \end{itemize}
% Por ello, veo NECESARIO incluir una sección adicional con el producto de de-Sitter con espacios tanto compactos como no compactos (estos últimos motivados con alguno de los paper que busqué el otro día, cuidado aquí...).
%
%\end{rem}
    \subsection*{Kasner models}
{\em Generalized Kasner models} are multiwarped  spacetimes $(V,g)$ where $V=(0,\infty)\times \R^{n}$ and
\begin{equation}
  \label{eq:35}
g=-dt^2+t^{2p_1}dx_1^2+\dots +t^{2p_{n}}dx_{n}^2,\quad\hbox{$(p_1,\ldots,p_n)\in\R^n$}.
\end{equation}
These models are solutions to the vacuum Einstein equations if $(p_1,\dots,p_{n})\in \R^{n}$ belongs to the so-called {\em Kasner sphere}, i.e., if it satisfies
  \[
\sum_{i=1}^{n}p_i=1=\sum_{i=1}^{n}p^2_i.
\]
Even if this condition does not fall under the hypotheses of our results, this does not cover all the cases of interest, and so, we are not going to assume it.

As far as we know, the c-boundary of these models can be faced in two different ways. On the one hand, by using Harris' result (Thm. \ref{thm:harris}); taking into account that the fibers are complete, this result gives a full description of the future c-boundary when $p_i>1$ for all $i$, and provides some partial information in the other cases. On the other hand, these models have been studied
 by Garc\'ia-Parrado and Senovilla in \cite{GS03} by using the isocausal relation. They essentially prove that, depending on the values of the constants $p_1,\ldots,p_n$, the corresponding Kasner model is isocausal to a particular Robertson-Walker model whose c-boundary is well-known. This may be useful, since, although the c-boundary of isocausal spacetimes may be different (see \cite{0264-9381-28-17-175016}), they can share some qualitative properties (see \cite{FHSIso2}).

 \smallskip


Of course, Thm. \ref{futurestructurefiniteconditions} parallels Harris' result for Kasner models when $p_i>1$ for all $i$. However, now we can go a step further and give a complete description of the c-boundary when
% Our aim here is to show that our result does not just  determine the causal boundary of Kasner models by using Thm. \ref{thm:main2} and compare it with the one obtained by García-Parrado and Senovilla. Observe that not all Kasner models will fall under the hypothesis of such a theorem. For instance, if the vector $(p_1,\dots,p_n)$ belongs to the Kasner sphere, they satisfy in particular that $|p_i|<1$, and so, the warping functions in \eqref{eq:35} will not satisfy the required integral conditions.

% \smallskip

% For this, we will distinguish two main cases:

% \begin{itemize}
% \item Let us assume that $p_i>1$ for all $i$. Then, we have that

%   \begin{equation}
%     \label{eq:36}
%    \int_1^{\infty}\frac{1}{t^{p_i}}dt<\infty \qquad \hbox{and}\qquad \int_0^{1}\frac{1}{t^{p_i}}dt=\infty,
%   \end{equation}


%     for all $i$, and so we fall under the hypothesis of Thm. \ref{thm:main2} (i). Therefore, as all the fibres are complete, it follows that:

%     \[
% \overline{V}\equiv [0,\infty]\times \R^{n},\qquad \partial V\equiv \left(\{0\}\times \R^{n} \right)\cup \left(\{\infty\}\times \R^{n} \right)
%       \]
% COMPARAR CON EL OTRO!!\footnote{Aquí hay un error, pero parece que puede solucionarse del siguiente modo: usamos la relación isocausal para pasar del Kasner a un Robertson-Walker. Con ello, puedo probar que dos curvas generan el mismo pasado y jugar con la distancia definida ahí....PENSAR CON CALMA}
    % \item For the second case we will make two assumptions. First, we will assume that there exists $1\leq k \leq n$ such that $p_{k+1}=p_j$ for all $k+1\leq j\leq n$. In particular, the Kasner metric becomes:
%      \[
%-dt^2+\sum_{i=1}^k t^{p_i}dx_i^2 + t^{p_{k+1}}\left(\sum_{i=k+1}^n dx^2_i \right),
%        \]
%  where
\[
p_i>1\;\;\hbox{for $1\leq i\leq k$,}\quad p_i=q\;\;\hbox{for $k+1\leq i\leq n$}\quad\hbox{and}\quad \int_1^{\infty}\frac{1}{t^{q}}dt=\infty.
\]
In this case we can write
\[
V=(0,\infty)\times\R^k\times\R^{n-k},\qquad g=-dt^2+\sum_{i=1}^{k}t^{2p_i}dx_i^2+t^{2q}\left(\sum_{i=k+1}^ndx_i^2 \right),
\]
% where
% \[
% \alpha_i(t)=t^{2 p_i},\;\; i=1,\ldots,k,\qquad \alpha_{k+1}(t)=t^{2q}.
% \]
In particular,
\[
\int_{1}^{\infty}\frac{dt}{t^{p_i}}<\infty,\;\; i=1,\ldots,k,\qquad\int_{1}^{\infty}\frac{dt}{t^{q}}=\infty.
\]
Therefore, the spacetime falls under the hypotheses of (the obvious multiwarped version of) Thm. \ref{futurecomploneinfinite} (essentially, with $M_1=\R^k$ and $M_2=\R^{n-k}$), which provides
%the warping functions satisfy \eqref{eq:45}, and thus, (the obvious multiwarped version of) Thm. \ref{futurecomploneinfinite} provides
the following homeomophism:
          \[
\hat{V}\;\leftrightarrow\; \left((0,\infty)\times \R^n\right)\cup \left(\R^k\times \left({\cal B}(\R^{n-k})\cup \{\infty\}\right)\right).
            \]
            So, taking into account that (see, for instance, \cite[Section 5.1]{H2})
            \[
{\cal B}(\R^{n-k})\equiv \R\times \mathbb{S}^{n-k-1},
              \]
              we immediately deduce that
              \[
\hat{\partial} V\leftrightarrow \R^k \times \left(\left(\R\times \mathbb{S}^{n-k-1}\right)\cup \{\infty\} \right).
                \]




%  This result provides a point set description of the causal completion of $(V,g)$ when the integrals $\int_{0}^{+\infty} \frac{1}{\sqrt{\alpha_{i}}}ds$ and $\int_{-\infty}^{0} \frac{1}{\sqrt{\alpha_{i}}}ds$ are finite for all $i=1,2$:
% \[
% \overline{V} \equiv (\mathbb{R}\cup \{+\infty\} \cup \{-\infty\}) \times M_{1}^{c} \times M_{2}^{c}.
% \]
% The point set structure of the c-boundary is then:
% \[
% \partial V \equiv (\mathbb{R} \times \partial_{c} M_{1} \times M_{2}^{c}) \cup (\mathbb{R} \times M_{1}^{c} \times \partial_{c} M_{2}) \cup ((\{-\infty\} \cup \{+\infty\}) \times M_{1}^{c} \times M_{2}^{c}).
% \]
%  Here, $(\{+\infty\} \cup \{-\infty\}) \times M_{1}^{c} \times M_{2}^{c}$ denotes the set of spatial boundary points, that is, pairs of the form $(P,\emptyset)$ and $(\emptyset,F)$, that are identified with points of the form $(\pm \infty,x_{1}^{\pm},x_{2}^{\pm})$ for some univocally determined $(x_{1}^{\pm},x_{2}^{\pm}) \in M_{1}^{c} \times M_{2}^{c}$ (recall Prop \ref{spatialboundaries}). The rest elements of the boundary correspond with timelike boundary points, that is, pairs of the form $(P,F)$ with $P \neq \emptyset \neq F$ and $P \sim_{S} F$, that are identified with a univocally determined point  $(\Omega,x_{1}^{*},x_{2}^{*}) \in \R \times M_{1}^{c} \times M_{2}^{c}$ (recall Prop. \ref{Srelatedtipstifs}).

%     \subsection*{C-completion on Multidimensional Inflationary Models with negative curvature}

% In \cite{doi:10.1063/1.532366}, Mignemi and Schmidt give a classification for multiwarped spacetimes with constant curvature, depending on the dimension of the model and the number of fibres $n$. Observe that the case where $n=1$ falls into the Robertson-Walker models, whose c-completion is well-known (see \cite{AF} for instance), so we will focus on $n\geq  2$. Moreover, let us restrict to the case of doubly warped models with {\em negative} constant curvature. Then, according to the classification given in \cite{doi:10.1063/1.532366}, $(V,g)$ falls into one of the following categories:

% \begin{itemize}
% \item[(a)] $D={\rm dim}(V)=3$, $V=\R\times \R\times \R$ and
%   \[
%     g=-dt^2+\left(sinh^2(t)dx^2 + cosh^2(t)dy^2\right).
%   \]

% \item[(b)] $D={\rm dim}(V)\geq 4$, $V=\R\times \R\times \mathbb{S}^{D-2}$ and
%   \[
% g=-dt^2+\left(sinh^2(t)dx^2 + cosh^2(t)g_{\mathbb{S}^{D-2}}\right)
%     \]

%     or

%     \[
%       g=-dt^2+\left(sinh^2(t)g_{\mathbb{S}^{D-2}} + cosh^2(t)dx^2\right).
%       \]
% \item[(c)] $D={\rm dim}(V)\geq 4$, $V=\R\times \mathbb{S}^{k}\times \mathbb{S}^{j}$ with $k+j=D-1$ and
%   \[
% g=-dt^2+\left(sinh^2(t)g_{\mathbb{S}^k} + cosh^2(t)g_{\mathbb{S}^{j}}\right)
%     \]
% \end{itemize}
% Of course, in previous classification $(\mathbb{S}^l,g_{\mathbb{S}^l})$ denotes the $l$-dimensional sphere attached with its standard Riemannian metric. As we can see, in all previous cases it follows that:

%   \[
% \int_0^{\pm\infty} \frac{1}{|sinh(t)|}dt<\infty,\qquad \hbox{and}\qquad \int_0^{\pm\infty} \frac{1}{|cosh(t)|}dt<\infty
%     \]
%     so we are in the conditions of Thm. \ref{thm:main} (i). Moreover, in all the cases the fibres (that we will denote $M_1$ and $M_2$ as usual) are complete. In conclusion, the c-completion of any inflationary multidimensional model with negative constant curvature is of the form:
%     \[
% \overline{V}=\left(\R\cup \{\pm\infty\}\right)\times M_1\times M_2.
%       \]
%  and its c-boundary is formed by two copies of the product of its fibres (one for the future and one for the past) with spatial causality.


\subsection*{The intermediate  Reissner-Nordstr\"om}
%\footnote{Jony: Hablar con José Luis sobre el cambio de orientación...}
%Next, we consider the intermediate region of Reissner-Nordstr\"om spacetime. In this case, Thm. \ref{thm:harris} is not useful to determine the future c-completion.
The Reissner-Nordstr\"om model is a spacetime $(V,g)$, where $V=\R\times \R\times \mathbb{S}^2$ and
\[
g=-\left(1-\frac{2m}{r}+\frac{q^{2}}{r^{2}}\right)dt^{2}+\left(1-\frac{2m}{r}+\frac{q^{2}}{r^{2}}\right)^{-1}dr^{2}+r^{2}(d\theta^{2}+sin^{2}\theta d\phi^{2}).
\]
 This metric degenerates at the zeros of the function $f(r)=(1-2m/r+q^2/r^2)$, which depend on the parameters $m$ (mass) and $q$ (charge). For our purposes we will require that $q\leq m$, which ensures the zeros $r^{\pm}=m\left( 1\pm\sqrt{1-q^2/m^2}\right)$ for $f$. The {\em intermediate region} of the Reissner-Nordstr\"om  is the spacetime $(V_I,g)$, where $V_I=\R\times (r^-,r^+)\times \mathbb{S}^2$.

Taking into account that $f(r)<0$ on $(r^-,r^+)$, the metric $g$ can be rewritten on $V_I$ as
\begin{equation}
  \label{eq:37}
g= -f(r)dt^2 + \frac{1}{f(r)}dr^2 + r^2 d\sigma^2=-d\tau^2 + r(\tau)^2d\sigma^2-F(\tau)dt^2,
\end{equation}
where
\[
d\tau:=-\frac{dr}{\sqrt{f(r)}}=\frac{dr}{\sqrt{-1+2m/r-q^2/r^2}} \qquad \hbox{and}\qquad F(\tau)=f(r(\tau)).
\]
Note that $\tau$ ranges in a finite interval $(a,b)$, and so, $(V_I,g)$ clearly corresponds with the standard form of a doubly warped spacetime where $V_I=(a,b)\times\mathbb{S}^2\times \R$.
%\footnote{El cambio de orden en las fibras es para que coincida con la forma del teorema...}.
In order to proceed with the analysis of the c-completion of $(V_I,g)$, we need to distinguish two cases: $q\neq 0$ and $q=0$.\footnote{Since the Penrose's diagram of Reissner-Nordstr\"om is well-known (see, for instance, \cite{hawking1975large}), the c-completion of $(V_I,g)$ can be also studied by applying \cite[Thm. 4.32]{FHSFinalDef}.}
%
%determine the c-completion of the intermediate region of the Reissner-Nordstr\"om, we have to study the integral conditions for the corresponding warping functions $\alpha_1(\tau)=r^2(\tau)$ and $\alpha_2(\tau)=-F(\tau)$, which will depend on the value of $q$. In the next subsections we distinguish two cases.
%
%\begin{rem} {\em  By using \cite[Thm. 4.32]{FHSFinalDef}, recalling the Penrose's diagram of the Reissner-Nordstr\"om spacetime (see, for instance, \cite{hawking1975large}).
%}
%\end{rem}


\subsubsection*{Intermediate Reissner-Nordstr\"om with charge, $q\neq 0$.}

%\footnote{CUIDADO AQUÍ: Hay un cambio en la orientación, cuando $r$ crece, nos movemos en la dirección pasada y viceversa...}
In this case, the warping integrals satisfy, for $a<c<b$,
 \begin{align}
   \int_{a}^{c}\frac{1}{\sqrt{\alpha_1(\tau)}}d\tau = \int_{r^-}^{r(c)}\frac{1}{r\sqrt{-1+\frac{2m}{r}-\frac{q^2}{r^2}}}dr<\infty\label{eq:38a} \\
   \int_{c}^{b}\frac{1}{\sqrt{\alpha_1(\tau)}}d\tau=\int^{r^+}_{r(c)}\frac{1}{r\sqrt{-1+\frac{2m}{r}-\frac{q^2}{r^2}}}dr<\infty\label{eq:38b}
  \end{align}
and
  \begin{align}
    \int_{a}^{c}\frac{1}{\sqrt{\alpha_2(\tau)}}d\tau=\int_{r^-}^{r(c)}\frac{1}{-1+\frac{2m}{r}-\frac{q^2}{r^2}}dr=\infty \label{eq:38}\\
    \int_{c}^{b}\frac{1}{\sqrt{\alpha_2(\tau)}}d\tau=\int^{r^+}_{r(c)}\frac{1}{-1+\frac{2m}{r}-\frac{q^2}{r^2}}dr=\infty.\label{eq:38c}
  \end{align}
  So, from Thm. \ref{thm:main} (iv) (with $M_1=\mathbb{S}^2$ and $M_2=\R$), we deduce the homeomorphisms
  \[
    \begin{array}{c}
      \overline{V}\leftrightarrow \left((a,b)\times \mathbb{S}^2\times \R\right) \cup (\mathbb{S}^2\times \left(\left(\R\times \{z^-, z^+\}\right)\cup \{i^+\} \right))\cup (\mathbb{S}^2\times \left(\left(\R\times \{z^-, z^+\}\right)\cup \{i^-\}\right)),\\
      \\
\partial V\leftrightarrow (\mathbb{S}^2\times \left(\left(\R\times \{z^-, z^+\}\right)\cup \{i^+\} \right))\cup (\mathbb{S}^2\times \left(\left(\R\times \{z^-, z^+\}\right)\cup \{i^-\}\right)),
    \end{array}
    \]
    where we have used that ${\cal B}(\R)\equiv \R\times \{z^-,z^+\}$, being $z^-$ and $z^+$ the two asymptotic directions (left and right) of $\R$.
    %\footnote{J.L.: ¿Es esto realmente relevante? JONY: No especialmente, puede quitarse si se quiere.}In particular, the c-completion of the intermediate region $(V_I,g)$ with $q\neq 0$ coincides with the c-completion of the spacetime ${\mathbb L}^2\times {\mathbb S}^2$.
  %, and such identification extends to the chronological and topological level.

  \subsubsection*{Interior Schwarzschild, $q=0$.}

When $q=0$, $f(r)$ has only one zero, we can identify $(r^-,r^+)\equiv (0,2M)$, and the intermediate region of Reissner-Nordstr\"om coincides with the interior region of Schwarzschild. In this case, the warping integrals (\ref{eq:38a}), (\ref{eq:38b}) and (\ref{eq:38c}) still hold, but \eqref{eq:38} transforms into
  \[
\int_{a}^{c}\frac{1}{\sqrt{\alpha_2(\tau)}}d\tau = \int_{0}^{r(c)}\frac{1}{-1+\frac{2m}{r}}dr<\infty.
  \]
  So, from Thm. \ref{thm:main} (iii), we deduce the homeomorphism
  %applies, and so, the past c-boundary has only spatial points. More precisely,
  \[
\overline{V}\leftrightarrow \left([a,b)\times \mathbb{S}^2\times \R\right) \cup \left( \mathbb{S}^2\times \left(\R\times \{z^-,z^+\} \right)\right)
    \]
  and thus,\footnote{The usual time-orientation on Reissner-Nordstr\"om makes the vector field $\partial_r$ past-directed in the intermediate region. So, in formula (\ref{d}), the roles of the future and past c-boundaries are interchanged with respect to the (a priori) expected ones.}
 \begin{equation}\label{d}
  \partial V\equiv \hat{\partial} V\cup \check{\partial} V \leftrightarrow  \left(\{a\}\times \mathbb{S}^2\times \R   \right) \cup \left(\mathbb{S}^2\times \left( \left(\R\times \{z^-,z^+\}\right)\cup \{i^+\} \right)\right).
  \end{equation}

\subsection*{De Sitter models with (non-necessarily compact) internal spaces}

Motivated by the relevance for the problem of the dS/CFT correspondence, finally we study the c-boundary of warped products of de Sitter models with general Riemannian manifolds.

Recall that {\em de Sitter spacetime} can be seen as a Robertson-Walker spacetime $(M,g_{M})$, where
\[
M=\R\times \mathbb{S}^l,\qquad g_{M}=-dt^2 + cosh(t)^2 g_{\mathbb{S}^l}.
  \]
Consider the doubly warped spacetime $(V,g)$ obtained as the product of de Sitter space $(M,g_{M})$ and a Riemannian manifold $(F,g_{F})$, i.e.,
  \[V=\R\times \mathbb{S}^l\times F,\qquad g=-dt^2+cosh^2(t)g_{\mathbb{S}^{l}}+g_{F}.
    \]
  The first warping function $\alpha_1(t)=\cosh(t)^2$ satisfies the finite integral conditions for both, the future and the past directions, meanwhile the second one $\alpha_2(t)\equiv 1$ does not. Therefore, from Thm. \ref{thm:main} (iv) (with $M_1=\mathbb{S}^l$ and $M_2=F$), we deduce the following homeomorphism for the c-boundary of $(V,g)$:
\[
\partial {V}\equiv \hat{\partial} V \cup \check{\partial} V \leftrightarrow  \left(\mathbb{S}^l\times \left({\cal B}(F)\cup \{i^+\}\right) \right)\,  \cup \, \left(\mathbb{S}^l\times \left({\cal B}(F)\cup \{i^-\}\right) \right).
\]
  In particular, if $(F,g_{F})$ is compact, then ${\cal B}(F)$ is empty, and the c-boundary becomes (compare with the last assertion on Thm. \ref{thm:harris}):
\[
  \partial V\leftrightarrow \left(\mathbb{S}^l\times \{i^+\}) \right)\,  \cup \, \left(\mathbb{S}^l\times \{i^-\} \right).
  \]

%
%
%   to the product of the de Sitter space with any Riemannian manifold, we obtain:
  % and with warping function satisfying the finite integral condition.  In such cases, as both warping functions satisfy nice  integral conditions and the fibres are complete, it is possible to make  use of Thm. \ref{thm:harris}. However, there exist cases where we need to eliminate such a restriction on the warping function (for instance, on the dS/CFT correspondence, see \cite{}), and Thm. \ref{thm:main} becomes essential. For instance, we can obtain the following general result:

%\begin{prop}
%  Let $(V,g)$ be the multiwarped spacetime obtained as the product of de Sitter space $(M,g_{M})$ and a Riemannian manifold $(F,g_{F})$, i.e.,
%  \[V=\R\times \mathbb{S}^l\times F,\qquad g=-dt^2+cosh^2(t)g_{\mathbb{S}^{l}}+g_{F}.
%    \]
% Then, the c-boundary of $(V,g)$ becomes
%  \[
%      \partial {V}\equiv \hat{\partial} V \cup \check{\partial} V \equiv  \left(\mathbb{S}^l\times \left({\cal B}(F)\cup \{i^+\}\right) \right)\,  \cup \, \left(\mathbb{S}^l\times \left({\cal B}(F)\cup \{i^-\}\right) \right).
%    \]
%
%\end{prop}
%Therefore, the c-completion is expressed in terms of the proper Busemann boundary ${\cal B}(F)$ of the Riemannian manifold $(F,g_{F})$. In particular, if $(F,g_{F})$ is compact, then ${\cal B}(F)$ is empty, and we obtain (compare with the last assertion on Thm. \ref{thm:harris}):
%\begin{cor}
%  If $(F,g_{F})$ is compact, then the c-boundary of $(V,g)$ is
%\[
%  \partial V\equiv \left(\mathbb{S}^l\times \{i^+\}) \right)\,  \cup \, \left(\mathbb{S}^l\times \{i^-\} \right).
%  \]
%\end{cor}

% \subsubsection*{Generalized Kasner models}
% Generalized Kasner models correspond with Multiwarped spacetimes $(V,g)$ where $V=(0,\infty)\times \R^{n}$ and
% \begin{equation}
%   \label{eq:35}
% g=-dt^2+t^{2p_1}dx_1^2+\dots +t^{2p_{n}}dx_{n}^2
% \end{equation}

%   with $p_i$ arbitrary constants. These models are solutions of the vacuum Einstein equations if the vector $(p_1,\dots,p_{n})\in \R^{n}$ belongs to the so-called {\em Kasner sphere}, that is, if its components satisfy:

%   \[
% \sum_{i=1}^{n}p_i=1=\sum_{i=1}^{n}p^2_i.
% \]

% The causal boundary of such models was studied by Garc\'ia-Parrado and Senovilla in \cite{GS03} by means of the isocausal relation. Essentially they prove that, depending on the constants $p_i$, the corresponding Kasner model is isocausal to a particular Robertson-Walker model with known causal completion. However, as it was proved in \cite{0264-9381-28-17-175016}, such a procedure presents some drawback, as the causal boundary of isocausal models may differ (even if they seem to share some qualitative properties in general cases, see \cite{FHSIso2}).


% Our aim here is to determine the causal boundary of Kasner models by using Thm. \ref{thm:main2} and compare it with the one obtained by García-Parrado and Senovilla. Observe that not all Kasner models will fall under the hypothesis of such a theorem. For instance, if the vector $(p_1,\dots,p_n)$ belongs to the Kasner sphere, they satisfy in particular that $|p_i|<1$, and so, the warping functions in \eqref{eq:35} will not satisfy the required integral conditions.

% \smallskip

% For this, we will distinguish two main cases:

% \begin{itemize}
% \item Let us assume that $p_i>1$ for all $i$. Then, we have that

%   \begin{equation}
%     \label{eq:36}
%    \int_1^{\infty}\frac{1}{t^{p_i}}dt<\infty \qquad \hbox{and}\qquad \int_0^{1}\frac{1}{t^{p_i}}dt=\infty,
%   \end{equation}


%     for all $i$, and so we fall under the hypothesis of Thm. \ref{thm:main2} (i). Therefore, as all the fibres are complete, it follows that:

%     \[
% \overline{V}\equiv [0,\infty]\times \R^{n},\qquad \partial V\equiv \left(\{0\}\times \R^{n} \right)\cup \left(\{\infty\}\times \R^{n} \right)
%       \]
% COMPARAR CON EL OTRO!!\footnote{Aquí hay un error, pero parece que puede solucionarse del siguiente modo: usamos la relación isocausal para pasar del Kasner a un Robertson-Walker. Con ello, puedo probar que dos curvas generan el mismo pasado y jugar con la distancia definida ahí....PENSAR CON CALMA}
%     \item For the second case we will make two assumptions. First, we will assume that there exists $1\leq k \leq n$ such that $p_{k+1}=p_j$ for all $k+1\leq j\leq n$. In particular, the Kasner metric becomes:
%       \[
% -dt^2+\sum_{i=1}^k t^{p_i}dx_i^2 + t^{p_{k+1}}\left(\sum_{i=k+1}^n dx^2_i \right).
%         \]
%         Secondly, we will assume that $p_i>1$ for $1\leq i\leq k$, so they also satisfy the integral conditions \eqref{eq:36}; but for $p_{k+1}$ we have:

%         \[
%    \int_1^{\infty}\frac{1}{t^{p_{k+1}}}dt=\infty \qquad \hbox{and}\qquad \int_0^{1}\frac{1}{t^{p_{k+1}}}dt<\infty,
%           \]

% \end{itemize}




%  This result provides a point set description of the causal completion of $(V,g)$ when the integrals $\int_{0}^{+\infty} \frac{1}{\sqrt{\alpha_{i}}}ds$ and $\int_{-\infty}^{0} \frac{1}{\sqrt{\alpha_{i}}}ds$ are finite for all $i=1,2$:
% \[
% \overline{V} \equiv (\mathbb{R}\cup \{+\infty\} \cup \{-\infty\}) \times M_{1}^{c} \times M_{2}^{c}.
% \]
% The point set structure of the c-boundary is then:
% \[
% \partial V \equiv (\mathbb{R} \times \partial_{c} M_{1} \times M_{2}^{c}) \cup (\mathbb{R} \times M_{1}^{c} \times \partial_{c} M_{2}) \cup ((\{-\infty\} \cup \{+\infty\}) \times M_{1}^{c} \times M_{2}^{c}).
% \]
%  Here, $(\{+\infty\} \cup \{-\infty\}) \times M_{1}^{c} \times M_{2}^{c}$ denotes the set of spatial boundary points, that is, pairs of the form $(P,\emptyset)$ and $(\emptyset,F)$, that are identified with points of the form $(\pm \infty,x_{1}^{\pm},x_{2}^{\pm})$ for some univocally determined $(x_{1}^{\pm},x_{2}^{\pm}) \in M_{1}^{c} \times M_{2}^{c}$ (recall Prop \ref{spatialboundaries}). The rest elements of the boundary correspond with timelike boundary points, that is, pairs of the form $(P,F)$ with $P \neq \emptyset \neq F$ and $P \sim_{S} F$, that are identified with a univocally determined point  $(\Omega,x_{1}^{*},x_{2}^{*}) \in \R \times M_{1}^{c} \times M_{2}^{c}$ (recall Prop. \ref{Srelatedtipstifs}).



% \begin{prop}
% If $(P,F) \in \overline{V}$ and $(P,F) \in L(\{(P_{m},F_{m})\}_{m})$ then $\{(\Omega_{m},x_{m}^{*})\}_{m}$ converges to $(\Omega,x^{*})$, where $(P,F)$ and each $(P_{m},F_{m})$ are identified with $(\Omega,x^{*})$ and $(\Omega_{m},x_{m}^{*})$ in $(\mathbb{R} \cup \{+\infty\} \cup \{-\infty\}) \times \overline{M}_{1}^{c}\times \cdots \times \overline{M}_{n}^{c}$, respectively.
% \end{prop}



% EL SIGUIENTE RESULTADO (MUY ESQUEMÁTICO) RESUME TODA LA ESTRUCTURA DE LOS ESPACIOTIEMPOS {\multiwarped}.

% \begin{thm}
% Let $(V,g)$ be a {\multiwarped} spacetime. $\overline{V}$ has the following structure.
%   \begin{enumerate}
%   \item Point set structure:

%   \begin{equation}
%     \label{eq:17}
% \overline{V}\setminus \partial^\infty V\equiv \R\times M_1^C\times M_2^C,
% \end{equation}
% with the product topology. Moreover, each point on $\left(M_1^C\times M_2^C\right)\setminus M_1\times M_2$ generates a timelike line over the boundary.

% For $\partial^\infty V$ we have:
% \begin{itemize}
% \item If \eqref{eq:7}, it is formed by two (spatial) copies of $M_1^C\times M_2^C$, one for the future and one for the past. So,
%   \[
% \overline{V}equiv\left(\R\cup \{\pm\}\right)\times M_1^C\times M_2^C
%     \]
% and previous equivalence extends topologically.
% \item If \eqref{eq:9}, it is formed by two copies of $M_1^C\times {\cal B}(M_2)$, one for the future and one for the past.  Each pair $x_1\in M_1^C$ and $[b_{x_2}]\in \partial_{{\cal B}(M_2)} M_2$ defines a lightlike line over the future and past boundary.
%   \end{itemize}

% \end{enumerate}
% \end{thm}



%\begin{proposition}
%Let $\gamma(t)=(t,c(t))$ with $\gamma(t) \rightarrow (\Omega,x^{*})$ be a future directed causal curve with $\Omega < \infty$
%and $x^* \in \bar{M}^{C}$ then
%$$\uparrow I^{-}(\gamma)=\{(t,q)\in V \mid \exists\ \mu_{1}^{q},...,\mu_{n}^{q}>0 {\text{ such that }}
%\int_{\Omega}^{t} \frac{\sqrt{\mu_{i}^{q}}}{\alpha_{i}}(\Sigma_{k}\frac{\mu_{k}^{q}}{\alpha_{k}})^{-1/2}>d_{i}(q_{i},x_i^*)\}$$
%\end{proposition}
%
%Let $(t_q,q) \in V$ such that for the future directed causal curve $\gamma(t)=(t,c(t))$ we have $\gamma(t) \ll (t_{q},q)$ for all
%$t \in [a,\Omega)$, then by the characterization of the chronological relation we have the existence of
%$(\mu_{1}^q(t),...,\mu_{n}^q(t)) \in (0,1)^n$ with $\Sigma_{k} \mu_{k}^q(t)=1$ and satisfying the inequalities:
%$$\int_{t}^{t^{q}} \frac{\mu_{i}^q(t)}{\alpha_{i}}\left(\Sigma_{k}\frac{\mu_{k}^q(t)}{\alpha_{k}}\right)^{-1/2}d\tau>d_{i}(q_{i},c_{i}(t)) \ \forall i,$$
%take any $\{t_{l}\}$ such that $t_{l} \rightarrow \Omega$ and consider the point $(\mu_{1}^{q},...,\mu_{n}^{q})$ limit point of some
%subsequence $\{(\mu_{1}^{q}(t_{l}),...,\mu_{n}^{q}(t_{l}))\}_{l}$ that satisfies the integral inequalities showed above, by using the convergence
%of those sequences we have:
%$$\int_{\Omega}^{t^{q}} \frac{\mu_{i}^q}{\alpha_{i}}\left(\Sigma_{k}\frac{\mu_{k}^q}{\alpha_{k}}\right)^{-1/2}d\tau \geq d_{i}(q_{i},x_{i}^{*}) \ \forall i,$$
%therefore $$\uparrow I^{-}[\gamma]=I^{+}[\{(t,q)\in V \mid \text{ there exists } \mu_{1}^{q},...,\mu_{n}^{q} \geq 0 \text{ such that }
%\int_{\Omega}^{t} \frac{\sqrt{\mu_{i}^{q}}}{\alpha_{i}}(\Sigma_{k}\frac{\mu_{k}^{q}}{\alpha_{k}})^{-1/2} \geq d_{i}(q_{i},x_i^*)\}]$$
%
%
%
%
%since this $\mu_{i}^{q}$ are taken as a limit of $\mu_{i}^{q}(t_{l})$ then it can happen that some $\mu_{i}^{q}$ are equal to zero, but not all of them are by
%the condition over the sum of $\mu_{i}^{q}$. Therefore for
%

%%% Local Variables:
%%% mode: latex
%%% TeX-master: "DoublyWarpedBoundary2017"
%%% End:

%\input{biblio2}

\section*{Acknowledgments}

The authors are partially supported by the Spanish Grant MTM2016-78807-C2-2-P (MINECO and FEDER funds). L. Aké also acknowledges a grant funded by the Consejo Nacional de Ciencia y Tecnolog\'ia (CONACyT), M\'exico.

\bibliographystyle{unsrt}
\bibliography{biblio2}



%\section{Discussion on Approximation \textit{vs} Stability and Recovery}\label{sec:approx-stability}


In the world of approximation algorithms, for a maximization problem for which an algorithm outputs $S$ and the optimum is $S^*$, what we typically try to prove is that
$w(S)\ge w(S^*)/\alpha$, even in the worst case; this \textit{approximation inequality} means that the algorithm at hand is an $\alpha$-approximation, so it is a \textit{good} algorithm. Though one might be quick to say that recovery of $\alpha$-stable instances immediately follows from the approximation inequality, this is not true because of the intersection $S\cap S^*$; if we have no intersection, then recovery indeed follows. 

What the research on stability and exact recovery suggests, is that we should try to understand if some of our already known approximation algorithms have the stronger property $w(S\setminus S^*)\ge w(S^*\setminus S)/\alpha$ or at least if they have it on stable instances. We refer to the latter as the \textit{recovery inequality}. This would directly imply an exact recovery result for $\alpha$-stable instances because we could $\alpha$-perturb only the $S\setminus S^*$ part of the input and get: 
\[
\noindent w(S\setminus S^*)\ge w(S^*\setminus S)/\alpha \implies \alpha\cdot w(S\setminus S^*) +w(S\cap S^*) \ge w(S^*\setminus S) +w(S\cap S^*) = w(S^*)
\] thus violating the fact we were given an $\alpha$-stable instance, unless $S\setminus S^* = \emptyset$.

This would mean that the algorithm successfully retrieved $S^*$ and could potentially explain why many approximation algorithms behave far better in practice than in theory. Furthermore, from a theory perspective, it would mean that many results from the well-studied area of approximation algorithms could be translated in terms of stability and recovery.

As a concluding remark, we want to point out that even though one might think that an $\alpha$-approximation algorithm needs at least $\alpha$-stability for recovery, this is not true as the somewhat counterintuitive result from \cite{balcan2015k} tells us: asymmetric $k$-center cannot be approximated to any constant factor, but can be solved optimally on 2-stable instances. This was the
first problem that is hard to approximate to any constant factor in the worst case, yet can be optimally
solved in polynomial time for 2-stable instances. The other direction (having an $\alpha$-approximation algorithm that cannot recover arbitrarily stable instances) is also true. These findings suggest that there are interesting connections between stability, exact recovery and approximation.


%\section{Applications}\label{sec:applications}

Here we will see many problems \cite{peng2016approximate} that fall into the framework of $p$-extendible systems and thus, we directly have recovery results on their stable instances. Some of the problems might be hard, like Weighted Independent Set, whereas others may be easy (i.e. in $P$), having exact algorithms, however the greedy is extremely simple and fast compared to those. 





%
%\providecommand{\href}[2]{#2}\begingroup\raggedright\begin{thebibliography}{10}
%
%\bibitem{Mal}
%J.~Maldacena, \emph{The large {$N$} limit of superconformal field theories and
%  supergravity [ {MR}1633016 (99e:81204a)]},  in \emph{Trends in theoretical
%  physics, {II} ({B}uenos {A}ires, 1998)}, vol.~484 of \emph{AIP Conf. Proc.},
%  pp.~51--63.
%\newblock Amer. Inst. Phys., Woodbury, NY, 1999.
%
%\bibitem{GpScqg05}
%A.~Garc{\'{\i}}a-Parrado and J.~M.~M. Senovilla, \emph{Causal structures and
%  causal boundaries},
%  \href{http://dx.doi.org/10.1088/0264-9381/22/9/R01}{\emph{Classical Quantum
%  Gravity} {\bf 22} (2005) R1--R84}.
%
%\bibitem{H3}
%S.~G. Harris, \emph{Boundaries on spacetimes: causality, topology, and group
%  actions}, \href{http://dx.doi.org/10.1007/s10711-007-9168-2}{\emph{Geom.
%  Dedicata} {\bf 126} (2007) 255--274}.
%
%\bibitem{S}
%M.~S{\'a}nchez, \emph{Causal boundaries and holography on wave type
%  spacetimes},
%  \href{http://dx.doi.org/10.1016/j.na.2009.02.101}{\emph{Nonlinear Anal.} {\bf
%  71} (2009) e1744--e1764}.
%
%\bibitem{FHSFinalDef}
%J.~L. Flores, J.~Herrera and M.~S{\'a}nchez, \emph{On the final definition of
%  the causal boundary and its relation with the conformal boundary},
%  {\emph{Adv. Theor. Math. Phys.} {\bf 15} (2011) 991--1057}.
%
%\bibitem{FHSHaus}
%J.~L. Flores, J.~Herrera and M.~S{\'a}nchez, \emph{Hausdorff separability of
%  the boundaries for spacetimes and sequential spaces},
%  \href{http://dx.doi.org/10.1063/1.4939485}{\emph{J. Math. Phys.} {\bf 57}
%  (2016) 022503, 25}.
%
%\bibitem{Chrusciel}
%P.~T. Chru{\'s}ciel, \emph{Conformal boundary extensions of {L}orentzian
%  manifolds}, {\emph{J. Differential Geom.} {\bf 84} (2010) 19--44}.
%
%\bibitem{BMN}
%D.~Berenstein, J.~Maldacena and H.~Nastase, \emph{Strings in flat space and pp
%  waves from {$\scr N=4$} super {Y}ang {M}ills},
%  \href{http://dx.doi.org/10.1088/1126-6708/2002/04/013}{\emph{J. High Energy
%  Phys.} (2002) No. 13, 30}.
%
%\bibitem{MR1}
%D.~Marolf and S.~F. Ross, \emph{Plane waves: to infinity and beyond!},
%  \href{http://dx.doi.org/10.1088/0264-9381/19/24/302}{\emph{Classical Quantum
%  Gravity} {\bf 19} (2002) 6289--6302}.
%
%\bibitem{MR}
%D.~Marolf and S.~F. Ross, \emph{A new recipe for causal completions},
%  \href{http://dx.doi.org/10.1088/0264-9381/20/18/314}{\emph{Classical Quantum
%  Gravity} {\bf 20} (2003) 4085--4117}.
%
%\bibitem{FS2}
%J.~L. Flores and M.~S{\'a}nchez, \emph{The causal boundary of wave-type
%  spacetimes}, \href{http://dx.doi.org/10.1088/1126-6708/2008/03/036}{\emph{J.
%  High Energy Phys.} (2008) 036, 43}.
%
%\bibitem{Me}
%G.~Mess, \emph{Lorentz spacetimes of constant curvature},
%  \href{http://dx.doi.org/10.1007/s10711-007-9155-7}{\emph{Geometriae Dedicata}
%  {\bf 126} (2007) 3--45}.
%
%\bibitem{BTZ}
%M.~Ba\~nados, C.~Teitelboim and J.~Zanelli, \emph{The black hole in
%  three-dimensional space-time},
%  \href{http://dx.doi.org/10.1103/PhysRevLett.69.1849}{\emph{Phys. Rev. Lett.}
%  {\bf 69} (1992) 1849--1851}.
%
%\bibitem{HP}
%S.~W. Hawking and D.~N. Page, \emph{Thermodynamics of black holes in anti-de
%  {S}itter space}, {\emph{Comm. Math. Phys.} {\bf 87} (1982/83) 577--588}.
%
%\bibitem{BHTZ}
%M.~Ba\~nados, M.~Henneaux, C.~Teitelboim and J.~Zanelli, \emph{Geometry of the
%  $2+1$ blackhole},
%  \href{http://dx.doi.org/10.1103/PhysRevD.48.1506}{\emph{Phys. Rev. D.} {\bf
%  48} (1993) 1506--1525}.
%
%\bibitem{Wit1}
%E.~Witten, \emph{Anti-de {S}itter space, thermal phase transition and
%  confinement in gauge theories},
%  \href{http://dx.doi.org/10.1142/S0217751X01004451}{\emph{Internat. J. Modern
%  Phys. A} {\bf 16} (2001) 2747--2769}.
%
%\bibitem{Wit2}
%E.~Witten, \emph{Anti-de {S}itter space, thermal phase transition and
%  confinement in gauge theories},
%  \href{http://dx.doi.org/10.1142/S0217751X01004451}{\emph{Internat. J. Modern
%  Phys. A} {\bf 16} (2001) 2747--2769}.
%
%\bibitem{AF}
%V.~Ala{\~n}a and J.~L. Flores, \emph{The causal boundary of product
%  spacetimes}, \href{http://dx.doi.org/10.1007/s10714-007-0492-5}{\emph{Gen.
%  Relativity Gravitation} {\bf 39} (2007) 1697--1718}.
%
%\bibitem{Got}
%J.~R. Gott, III, \emph{Closed timelike curves produced by pairs of moving
%  cosmic strings: exact solutions},
%  \href{http://dx.doi.org/10.1103/PhysRevLett.66.1126}{\emph{Phys. Rev. Lett.}
%  {\bf 66} (1991) 1126--1129}.
%
%\bibitem{H}
%S.~G. Harris, \emph{Discrete group actions on spacetimes: causality conditions
%  and the causal boundary},
%  \href{http://dx.doi.org/10.1088/0264-9381/21/4/032}{\emph{Classical Quantum
%  Gravity} {\bf 21} (2004) 1209--1236}.
%
%\bibitem{FHSIso2}
%J.~Flores, J.~Herrera and M.~S{\'a}nchez, \emph{Computability of the causal
%  boundary by using isocausality}, {\emph{Classical and Quantum Gravity} {\bf
%  30} (2013) 075009}.
%
%\bibitem{GKP}
%R.~Geroch, E.~H. Kronheimer and R.~Penrose, \emph{Ideal points in space-time},
%  \href{http://dx.doi.org/10.1098/rspa.1972.0062}{\emph{Proceedings of the
%  Royal Society A: Mathematical, Physical and Engineering Sciences} {\bf 327}
%  (apr, 1972) 545--567}.
%
%\bibitem{BS}
%R.~Budic and R.~K. Sachs, \emph{Causal boundaries for general relativistic
%  space times},
%  \href{http://dx.doi.org/http://dx.doi.org/10.1063/1.1666812}{\emph{Journal of
%  Mathematical Physics} {\bf 15} (1974) 1302--1309}.
%
%\bibitem{H1}
%S.~G. Harris, \emph{Universality of the future chronological boundary},
%  \href{http://dx.doi.org/10.1063/1.532582}{\emph{J. Math. Phys.} {\bf 39}
%  (1998) 5427--5445}.
%
%\bibitem{H2}
%S.~G. Harris, \emph{Topology of the future chronological boundary: universality
%  for spacelike boundaries},
%  \href{http://dx.doi.org/10.1088/0264-9381/17/3/303}{\emph{Classical Quantum
%  Gravity} {\bf 17} (2000) 551--603}.
%
%\bibitem{Ra}
%I.~R{\'a}cz, \emph{Causal boundary of space-times},
%  \href{http://dx.doi.org/10.1103/PhysRevD.36.1673}{\emph{Phys. Rev. D (3)}
%  {\bf 36} (1987) 1673--1675}.
%
%\bibitem{Sz}
%L.~B. Szabados, \emph{Causal boundary for strongly causal spacetimes},
%  {\emph{Classical and Quantum Gravity} {\bf 5} (1988) 121}.
%
%\bibitem{Sz2}
%L.~B. Szabados, \emph{Causal boundary for strongly causal spacetimes. {II}},
%  {\emph{Classical Quantum Gravity} {\bf 6} (1989) 77--91}.
%
%\bibitem{Wald}
%R.~Wald, \emph{General Relativity}.
%\newblock University of Chicago Press, 1984.
%
%\bibitem{Flores}
%J.~L. Flores, \emph{The causal boundary of spacetimes revisited},
%  \href{http://dx.doi.org/10.1007/s00220-007-0345-9}{\emph{Comm. Math. Phys.}
%  {\bf 276} (2007) 611--643}.
%
%\bibitem{MS}
%E.~Minguzzi and M.~S{\'a}nchez, \emph{The causal hierarchy of spacetimes},  in
%  \emph{Recent developments in pseudo-{R}iemannian geometry}, ESI Lect. Math.
%  Phys., pp.~299--358.
%\newblock Eur. Math. Soc., Z\"urich, 2008.
%\newblock \href{http://dx.doi.org/10.4171/051-1/9}{DOI}.
%
%\bibitem{FHSBuseman}
%J.~L. Flores, J.~Herrera and M.~S{\'a}nchez, \emph{Gromov, {C}auchy and causal
%  boundaries for {R}iemannian, {F}inslerian and {L}orentzian manifolds},
%  \href{http://dx.doi.org/10.1090/S0065-9266-2013-00680-6}{\emph{Mem. Amer.
%  Math. Soc.} {\bf 226} (2013) vi+76}.
%
%\end{thebibliography}
%\endgroup
%
%%\bibliography{biblio2}
%%\bibliographystyle{JHEP}
%
%%\input{referencias}

\end{document}

%%% Local Variables:
%%% mode: latex
%%% TeX-master: t
%%% End:
