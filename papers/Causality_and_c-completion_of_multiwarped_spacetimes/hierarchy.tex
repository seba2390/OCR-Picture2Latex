\section{Position into the causal ladder}
\label{sec:causalladder}
In order to have an idea of the goodness of the causality of doubly warped spacetimes, next we are going to
determine their position into the causal ladder. As we will see, this depends on the warping functions integrals and the convexity character of their Riemannian fibers.

Let us consider first a brief remainder of the main levels of the causal ladder. Each level corresponds with a causality condition which is strictly more restrictive than the previous one:
\begin{defi}\label{ant} A spacetime $(V,g)$ is
\begin{itemize}
\item {\em non-totally vicious} if $p\not\ll p$ for some $p\in V$.

\item {\em chronological} if it does not contain closed timelike curves.

\item {\em causal} if it does not contain closed causal curves.

\item {\em distinguishing} if whenever $I^+(p)=I^+(q)$ and $I^-(p)=I^-(q)$, necessarily $p=q$.

\item{\em strongly causal} if it does not contain ``nearly closed'' causal curves, i.e. for any open neighborhood $U$ of $p$ there exists some open neighborhood $V$ with $p\in V\subset U$ such that any timelike segment with extreme points in $V$ is contained in $U$.

\item {\em stably causal} if there exists some causal Lorentzian metric $g'$ on $V$ with $g<g'$, i.e., such that $g'(v,v)<0$ for any $v\in TV\setminus \{0\}$ with $g(v,v)\leq 0$. This is equivalent to the existence of some {\em global time function}, i.e., a function defined on the whole spacetime $(V,g)$ which is strictly increasing along each future-directed causal curve.

\item {\em causally continuous} if it is distinguishing and the set valued functions $I^{+}(\cdot)$ and $I^{-}(\cdot)$ are outer continuous (say, $I^{+}(\cdot)$ is {\em outer continuous at some} $p\in V$ if, for any compact subset $K\subset I^{+}(p)$ there exists an open neighborhood $U\ni p$ such that $K\subset I^{+}(q)$ for all $q\in U$). This is equivalent to being distinguishing and {\em reflecting}, %(see \cite[Defn. 3.59]{MS}),
i.e. for any pair of events $p,q \in V$, $I^{+}(q) \subset I^{+}(p)$ if and only if $I^{-}(p) \subset I^{-}(q)$.

\item {\em causally simple} if it is causal and $J^{\pm}(p)$ are closed sets for any $p\in V$.

\item {\em globally hyperbolic} if it is causal and $J^{+}(p)\cap J^{-}(q)$ are compact for any $p,q \in V$.
\end{itemize}
\end{defi}
It is direct from the very basic structure of \multiwarped spacetimes (\ref{eq:1-aux}) that $t:V \rightarrow (a,b)$ is a global time function (see \cite[Lemma 3.55]{beem}). Therefore, any \multiwarped spacetime is stably causal. The approach developed in previous section will allow to show that any \multiwarped spacetime is causally continuous as well. In fact:

\begin{thm}
Any \multiwarped spacetime $(V,g)$ as in (\ref{eq:1-aux}) is causally continuous.
\end{thm}

\begin{proof}  Since $(V,g)$ is stably causal, it is also distinguishing. So, it suffices to show that $(V,g)$ is reflecting. Let $\point{t^o}{x_{1}^o}{x_{2}^{o}}, \point{t^{e}}{x_{1}^{e}}{x_{2}^{e}} \in V$ be such that
$I^{+}(\point{t^{e}}{x_{1}^{e}}{x_{2}^{e}}) \subset I^{+}(\point{t^{o}}{x_{1}^{o}}{x_{2}^{o}})$, and let us prove that $I^{-}(\point{t^{o}}{x_{1}^{o}}{x_{2}^{o}})
\subset I^{-}(\point{t^{e}}{x_{1}^{e}}{x_{2}^{e}})$
(the converse is analogous). Consider the sequence $\{\point{t^{e}+1/n}{x_{1}^{e}}{x_{2}^{e}}\}_{n} \subset I^{+}(\point{t^{e}}{x_{1}^{e}}{x_{2}^{e}})$ %\cambios{(for simplicity, we will assume from here that $t^e+1<b$, otherwise we will take $n\geq n_0$ so $t^e+1/n_0<b$)}
and note that, by the hypothesis, this sequence also belongs to $I^{+}(\point{t^{o}}{x_{1}^{o}}{x_{2}^{o}})$.
Therefore, from Prop. \ref{c0}, there exist constants $\mu_{1}^{n},\mu_{2}^{n}>0$, with $\mu_{1}^{n} + \mu_{2}^{n}=1$, satisfying the following inequalities:
\begin{equation}
\label{eq2'}
\Integral{t^{o}}{t^{e}+1/n}{\mu_{i}^{n}}{i}{\mu_{k}^{n}}
> d_{i}(x_{i}^{o},x_{i}^{e})\qquad\hbox{for $i=1,2$}.
\end{equation}
Up to a subsequence, we can assume that $\{\mu_{i}^{n}\}_{n}$ converges to $\mu_i$, for all $i$, with $0 \leq \mu_{1},\mu_{2}\leq 1$ and $\mu_{1}+\mu_{2}=1$. Moreover,
\begin{equation}
\label{eq3'}
\left\{\sqrt{\mu_{i}^{n}}\alpha_{i}(s)^{-1}\left(\sum_{k=1}^2\mu_{k}^{n} \alpha_{k}(s)^{-1}\right)^{-1/2}\right\}_n \longrightarrow\sqrt{\mu_{i}}\alpha_{i}(s)^{-1}\left(\sum_{k=1}^2\mu_{k}\alpha_{k}(s)^{-1}\right)^{-1/2}
\end{equation}
uniformly on $[t^o,t^e+1]$. Therefore, from (\ref{eq2'}) and (\ref{eq3'}), we deduce
\[
\Integral{t^{o}}{t^{e}}{\mu_{i}}{i}{\mu_{k}} \geq d_{i}(x_{i}^{o},x_{i}^{e}),\qquad\hbox{for $i=1,2$.}
\]
If we consider $\point{t^{o}-1/n}{x_{1}^{o}}{x_{2}^o}$, and modify slightly $(\mu_{1},\mu_{2})$, by continuity we obtain new coefficients $(\mu_{1}',\mu_{2}')$, with $\mu'_{1},\mu'_{2}>0$ and $\mu'_{1}+\mu'_{2}=1$, such that
\[
\Integral{t^{o}-1/n}{t^e}{\mu'_{i}}{i}{\mu'_{k}}
>
d_{i}(x^{o}_{i},x^{e}_{i})\qquad\hbox{for $i=1,2$.}
\]
Again from Prop. \ref{c0}, we have $\point{t^o-1/n}{x_1^o}{x_2^o} \ll \point{t^e}{x_1^e}{x_2^e}$
for all $n$. So, taking into account that $I^-(\point{t^o}{x_1^o}{x_2^o})=\cup_{n \in {\mathbb N}}I^-(\point{t^o-1/n}{x_1^o}{x_2^o})$, we deduce the inclusion $I^-(\point{t^o}{x_1^o}{x_2^o} )\subset
I^-(\point{t^e}{x_1^e}{x_2^e})$, as required.
\end{proof}
\begin{thm}
\label{causi}
A \multiwarped spacetime $(V,g)$ as in (\ref{eq:1-aux}) is causally simple if and only if $(M_i,g_i)$ is $L_i$-convex for $L_i=\int_{a}^{b}\frac{1}{\sqrt{\alpha_{i}(s)}}ds$, $i=1,2$.
\end{thm}

\begin{proof} For the implication to the right, assume that $(V,g)$ is causally simple. We will prove that
$(M_1,g_1)$ is $L_1$-convex (the proof for the second fiber is analogous). Let $x^o_{1},x^e_{1} \in M_{1}$ with $0<d_1(x^o_{1},x^e_{1})<L_1$. Since
$\int_{a}^{b}\frac{1}{\sqrt{\alpha_1(s)}}ds=L_1 > d_1(x^o_{1},x^e_{1})$, there exists $a<\C_1<\C_2<b$ such that
\begin{equation}
\label{eq6'}
\int_{\C_1}^{\C_2} \frac{1}{\sqrt{\alpha_{1}(s)}}ds>d_{1}(x^o_{1},x^e_{1}).
\end{equation}
Fix $x_{2} \in M_{2}$ and consider
the points $\point{\C_1}{x_{1}^o}{x_{2}}$ and $\point{\C_2}{x_{1}^e}{x_{2}}$. Inequality (\ref{eq6'}) and Prop. \ref{c0} imply that
$\point{\C_2}{x_{1}^e}{x_{2}} \in I^{+}(\point{\C_1}{x_{1}^o}{x_{2}})$.
Since $\point{\C_1}{x_{1}^e}{x_{2}} \not \in I^+(\point{\C_1}{x_{1}^o}{x_{2}})$, there exists
$t^e \in \R$ such that $\point{t^e}{x^e_{1}}{x_{2}} \in \partial I^{+}(\point{\C_1}{x^o_{1}}{x_{2}})$,
i.e.,
\[
\begin{array}{c}
\point{t^e}{x^e_{1}}{x_{2}} \in \overline{I^{+}(\point{\C_1}{x^o_{1}}{x_{2}})} \setminus I^{+}(\point{\C_1}{x^o_{1}}{x_{2}}) \qquad\qquad\qquad \\ \qquad\qquad\qquad\qquad\qquad\qquad=J^{+}(\point{\C_1}{x^o_{1}}{x_{2}})\setminus I^{+}(\point{\C_1}{x^o_{1}}{x_{2}}),
\end{array}
\]
where, in the equality, we have used that $(V,g)$ is causally simple. Therefore, there exists a null geodesic
$\gamma(s)=(t(s),c_{1}(s),c_{2}(s))$ connecting $\point{\C_1}{x^o_{1}}{x_{2}}$ with $\point{t^e}{x^e_{1}}{x_{2}}$.
% (\cite[Cor. 4.14]{beem})\footnote{¿Es esta dita realmente necesaria?}
From Prop. \ref{p2'} there exist constants $\mu'_1, \mu'_2\geq 0$ such that the following inequalities hold:
%In particular,
%$x_{1}(s)$ is a pregeodesic in $(M_{1},g_{1})$. Since $\gamma(s)$ is a future directed
%null curve, the pregeodesics $x_{1}(s)$ can be parametrized using $t$ as a parameter; so, the following inequalities hold
%satisfies
\[
\begin{array}{c}
0<d_{1}(x^o_{1},x^e_{1})\leq \displaystyle\Integral{\C_1}{t^e}{\mu'_{1}}{1}{{\mu'}_{k}}=\hbox{length}_1(c_{1}),
\\ 0=d_{2}(x_{2},x_{2}) \leq \displaystyle\Integral{\C_1}{t^e}{\mu'_{2}}{2}{{\mu'}_{k}}=\hbox{length}_2(c_{2}).
%\\ \vdots \\
%0=d_{n}(x_{n},x_{n}) \leq \Integral{-n_{0}}{t^e}{\mu_{n}}{1}{\mu_{k}}   ={\mathrm length}_n(x_{n}),
%
%
%\int_{-n_{0}}^{t^e}
%\sqrt{\mu_{n}}\alpha_{n}^{-1}\left(\sum_{j=1}^n\mu_{j}\alpha_{j}^{-1}\right)^{-1/2}={\mathrm length}_n(x_{n}),
\end{array}
\]
%where $\mu_{i}=\alpha_{i}^{2}(t(s))g_{i}(c_{i}',c_{i}') \in \mathbb{R}$ and $c_{i}(s)$ are pregeodesics in $(M_{i},g_{i})$ for  $i=1,2$ (see \eqref{eq:31}).
So, taking into account that $$\point{t^e}{x^e_{1}}{x_{2}} \not\in I^+(\point{\C_1}{x^o_{1}}{x_{2}}),$$ the second inequality in the first line must be an equality (recall Prop. \ref{c0}). In conclusion, $c_1(s)$ is a reparametrization of a minimizing geodesic of $(M_1,g_1)$, as required.

\smallskip

For the implication to the left, assume that $(M_i,g_i)$ is $L_i$-convex for $L_i=\int_{a}^{b}\frac{1}{\sqrt{\alpha_i(s)}}ds$, $i=1,2$. In order to prove that $(V,g)$ is causally simple,
take $\point{t^{e}}{x_{1}^{e}}{x_{2}^{e}} \in \overline{J^{+}(\point{t^{o}}{x_{1}^{o}}{x_{2}^{o}})}=\overline{I^{+}(\point{t^{o}}{x_{1}^{o}}{x_{2}^{o}})}$. Then, $I^+(\point{t^{e}}{x_{1}^{e}}{x_{2}^{e}}) \subset I^+(\point{t^{o}}{x_{1}^{o}}{x_{2}^{o}})$, and thus, $\point{t^{o}}{x_{1}^{o}}{x_{2}^{o}} \ll \point{t^{e}+1/n}{x_{1}^{e}}{x_{2}^{e}}$ for all $n$.
From Prop. \ref{c0}, there exist constants $0 <\mu_{1}^{n},\mu_{2}^{n} < 1$, with $\mu_{1}^{n}+\mu_{2}^{n}=1$ for all $n$, such that
\[
\Integral{t^o}{t^{e}+1/n}{\mu_{i}^{n}}{i}{\mu_{k}^{n}}
> d_{i}(x_{i}^{o},x_{i}^{e}),\quad i=1,2.
\]
Since $\{\mu_{i}^{n}\}_{n}$ converges (up to subsequence) to some $\mu_{i} \in [0,1]$ for $i=1,2$, with $\mu_{1}+\mu_{2}=1$, we have:
\[
\frac{\sqrt{\mu_{i}^{n}}}{\alpha_{i}(s)}\left(\sum_{k=1}^2 \frac{\mu_{k}^{n}}{\alpha_{k}(s)}\right)^{-1/2}\longrightarrow
\frac{\sqrt{\mu_{i}}}{\alpha_{i}(s)}\left(\sum_{k=1}^2 \frac{\mu_{k}}{\alpha_{k}(s)}\right)^{-1/2}\quad\hbox{uniformly on $[t^o,t^e+1]$.}
\]
Recalling now that all previous functions are bounded by the (Lebesgue) integrable function $g:[t^o,t^e+1]\rightarrow \R$,
    $g(t)=\alpha_i(t)^{-1/2}$, the Dominated Convergence Theorem ensures that:
\[
\Integral{t^o}{t^e}{\mu_{i}}{i}{\mu_{k}}
=\lim_{n\rightarrow\infty} \Integral{t^{o}}{t^{e}+1/n}{\mu_{i}^{n}}{i}{\mu_{k}^n} \geq d_{i}(x_{i}^{o},x_{i}^{e}).
\]
%\footnote{OJO!!!}with equality in the $i$-equation if and only if $\mu_{i,n}=0$.
In particular,
\[
d_{i}(x_{i}^{o},x_{i}^{e})<\Integral{a}{b}{\mu_{i}}{i}{\mu_{k}} \le \int_{a}^{b}\frac{1}{\sqrt{\alpha_i(s)}}ds=L_i,\quad i=1,2.
\]
So, taking into account that $(M_i,g_i)$ are $L_i$-convex for $i=1,2$ we have that Prop. \ref{p2'} implies $\point{t^{e}}{x_{1}^{e}}{x_2^e} \in J^{+}(\point{t^{o}}{x_{1}^{o}}{x_2^o})$, as required.
\end{proof}

The following example shows the tight character of Thm. \ref{causi}, in the sense that there may exist causally simple warped spacetimes with non-convex fiber (the extension to the case of two fibers is straightforward). In fact:
\begin{exe}

 \begin{figure}
\centering
\ifpdf
  \setlength{\unitlength}{1bp}%
  \begin{picture}(377.33, 110.89)(0,0)
  \put(0,0){\includegraphics{fig1.pdf}}
  \put(83.67,100.21){\fontsize{9.42}{11.71}\selectfont $x_0$}
  \put(286.41,100.21){\fontsize{9.42}{11.71}\selectfont $x_1$}
  \put(165.89,99.40){\fontsize{9.42}{11.71}\selectfont $T_1$}
  \put(166.29,78.44){\fontsize{9.42}{11.71}\selectfont $T_2$}
  \put(166.70,50.22){\fontsize{9.42}{11.71}\selectfont $T_n$}
  \put(47.67,69.16){\fontsize{13.76}{14.11}\selectfont $H_0$}
  \put(300.28,67.84){\fontsize{13.76}{14.11}\selectfont $H_1$}
  \end{picture}%
\else
  \setlength{\unitlength}{1bp}%
  \begin{picture}(377.33, 110.89)(0,0)
  \put(0,0){\includegraphics{fig1}}
  \put(83.67,100.21){\fontsize{9.42}{11.71}\selectfont $x_0$}
  \put(286.41,100.21){\fontsize{9.42}{11.71}\selectfont $x_1$}
  \put(165.89,99.40){\fontsize{9.42}{11.71}\selectfont $T_1$}
  \put(166.29,78.44){\fontsize{9.42}{11.71}\selectfont $T_2$}
  \put(166.70,50.22){\fontsize{9.42}{11.71}\selectfont $T_n$}
  \put(47.67,69.16){\fontsize{13.76}{14.11}\selectfont $H_0$}
  \put(300.28,67.84){\fontsize{13.76}{14.11}\selectfont $H_1$}
  \end{picture}%
\fi
%   \begin{figure}[H]
% \centering
% \ifpdf
%   \setlength{\unitlength}{1bp}%
%   \begin{picture}(454.19, 131.80)(20,0)
%   \put(0,0){\includegraphics{fig1.pdf}}
%   \put(100.04,120.06){\fontsize{11.77}{13.33}\selectfont $x_0$}
%  \put(70.04,80.06){\fontsize{14.77}{15.33}\selectfont $H_0$}
%  \put(345.36,120.06){\fontsize{11.77}{13.33}\selectfont $x_1$}
%  \put(365.36,80.06){\fontsize{14.77}{15.33}\selectfont $H_1$}
%   \put(199.54,119.08){\fontsize{11.77}{13.33}\selectfont $T_1$}
%   \put(200.02,93.72){\fontsize{11.77}{13.33}\selectfont $T_2$}
%   \put(200.51,59.58){\fontsize{11.77}{13.33}\selectfont $T_n$}
%   \end{picture}%
% \else
%   \setlength{\unitlength}{1bp}%
%   \begin{picture}(454.19, 131.80)(0,0)
%   \put(0,0){\includegraphics{fig1}}
%   \put(100.04,120.06){\fontsize{9.77}{11.33}\selectfont $x_0$}

%   \put(345.36,120.06){\fontsize{7.77}{9.33}\selectfont $x_1$}
%   \put(199.54,119.08){\fontsize{7.77}{9.33}\selectfont $T_1$}
%   \put(200.02,93.72){\fontsize{7.77}{9.33}\selectfont $T_2$}
%   \put(200.51,59.58){\fontsize{7.77}{9.33}\selectfont $T_n$}
%   \end{picture}%
% \fi
  \caption{\label{fig:1} Both hemispheres $H_0$ and $H_1$ are connected by a sequence of immersed tubes $\{T_n\}_n$, where a length-minimizing curve connecting the north pole $x_0$ of $H_0$ to the north pole $x_1$ of $H_1$ through $T_n$ has bigger length than a length-minimizing curve connecting the same points through $T_{n+1}$. This picture is based on \cite[Figure 1]{Bartolo2002}.}
\end{figure}


\footnote{We are thankful to Prof. Miguel Sánchez for bringing this example to our attention.}In \cite[Section 2.1]{Bartolo2002} the authors construct a Riemannian manifold $(M,g)$ containing two points $x_0, x_1\in M$ such that any geodesic $\gamma\subset M$ connecting them satisfies ${\mathrm length}(\gamma)>d(x_0,x_1)$. The example basically consists of two open hemispheres $H_0$, $H_1$ in $\R^3$ connected by a sequence of immersed tubes $(T_n)_n$ of decreasing lengths, and such that any curve joining the corresponding north poles $x_0$ and $x_1$ through $T_n$ is longer than a minimizing curve joining them through $T_{n+1}$ (see Figure \ref{fig:1}). It is assumed also that the lengths of these tubes converge to a number which is strictly positive. In particular, $x_{0}$ and $x_{1}$ cannot be joined by a minimizing geodesic, and thus,
$(M,g)$ is not convex. However, there exists some $\delta>0$ such that $(M,g)$ is $L$-convex for any $L\leq \delta$. Consider now the warped spacetime $V=\R\times_{\alpha}M$ with $\alpha:\R\rightarrow (0,\infty)$ satisfying $\int_{-\infty}^{+\infty}1/\sqrt{\alpha(s)}ds=L\leq\delta$. From Thm. \ref{causi}, $V$ is causally simple.
%However, as commented above, the fiber $(M,g)$ is not weakly convex, since $x_{0}$ and $x_{1}$ cannot be joined by a minimizing geodesic.
\end{exe}

Finally, for the sake of completeness, we include the following simple consequence of \cite[Th. 3.68]{beem}, whose implication to the left is reproved here by using the techniques developed in this paper:

\begin{thm}
A \multiwarped spacetime $(V,g)$ as in (\ref{eq:1-aux}) is globally hyperbolic if and only if $(M_{i},g_{i})$, $i=1,2$, are complete Riemannian manifolds.
\end{thm}

\begin{proof}
%For the implication to the right, assume for instance that $(M_1,g_1)$ is not complete. From \cite{beem}[Th. 3.68], the Lorentzian warped product $H\times_{\alpha_1} M_1$, with $H \equiv (\R,-dt^2)$ and $M_1 \equiv (M_1,g_1)$, is not globally hyperbolic. Since $H\times_{\alpha_1}M_1$ is causal, it must contain some noncompact causal diamond $J^+((t^o,x_1^o))\cap J^-((t^e,x_1^e))$. Then, the causal diamond $J^+(\point{t^o}{x_{1}^o}{x_{2}}) \cap J^-(\point{t^e}{x_1^e}{x_2})$, for any $x_2\in M_2$, is not compact either; in fact, note that
%\[
%J^+(\point{t^o}{x_{1}^o}{x_{2}}) \cap J^-(\point{t^e}{x_1^e}{x_2})=\pi \circ j_{x_2}(J^+(\point{t^o}{x_1^o}{x_2})\cap J^-(\point{t^e}{x_1^e}{x_2}),
%\]
%where $j_{x_2}:\mathbb{R} \times M_{1} \rightarrow \mathbb{R} \times M_{1} \times M_{2}$, with $j_{x_2}(t,x_{1})=(t,x_{1},x_2)$, and the natural projection $\pi:\mathbb{R} \times M_{1}\times M_{2} \rightarrow \mathbb{R} \times M_{1}$ are both, causal relations preserving and continuous maps.
%\medskip
%\footnote{Jony: Sólo se está probando un lado. El otro aparece probado en el .tex, pero comentado para que no aparezca. ¿Por qué?}
Assume that $(M_{i},g_{i})$, $i=1,2$, are complete. Since $(V,g)$ is causally continuous, and thus, causal, it suffices to prove that any causal diamond is sequentially compact (and thus, compact). Let $\{\point{t^n}{x_1^n}{x_2^n}\}_n$ be a sequence in $J^+(\point{t^o}{x_1^o}{x_2^o})\cap J^-(\point{t^e}{x_1^e}{x_2^e})$. Since the fibers are complete, they are convex, and so, we can apply Prop. \ref{p2'}. Hence, there exist constants $0\leq\mu_{1}^{n},\mu_{2}^{n}\leq 1$, $0\leq\overline{\mu}_{1}^{n},\overline{\mu}_{2}^{n}\leq 1$ with $\mu_{1}^{n}+\mu_{2}^{n}=1=\overline{\mu}_{1}^{n}+\overline{\mu}_{2}^{n}$ for all $n$, such that
\[
  \begin{array}{l}
\displaystyle\Integral{t^{o}}{t^n}{\mu_{i}^{n}}{i}{\mu_{k}^{n}} \geq d_{i}(x_{i}^{o},x_{i}^{n})   \\
\displaystyle\Integral{t^{n}}{t^e}{\overline{\mu}_{i}^{n}}{i}{\overline{\mu}_{k}^{n}} \geq d_{i}(x_{i}^{n},x_{i}^{e}),
  \end{array}\quad i=1,2.
\]
In particular, the following inequalities hold for all $n$:
\[
t^o\leq t^n\leq t^e\quad\hbox{and}\quad\int_{t^{o}}^{t^e}\frac{1}{\sqrt{\alpha_i(s)}}ds \geq d_{i}(x_{i}^{o},x_{i}^{n}),\quad i=1,2.
\]
That is, $t^n \in [t^o,t^e]$ and $x_i^{n} \in \overline{B}_{r_i}(x_i^o)$, $r_i:=\int_{t^{o}}^{t^e}\alpha_i^{-1/2}ds$, $i=1,2$, for large $n$.
But, $[t^o,t^e]$ and $\overline{B}_{r_i}(x_i^o)$, $i=1,2$, are compact sets (recall that $(M_{i},g_{i})$, $i=1,2$, are complete). So,
%there some subsequence
%$\{(x_{1}^{m},x_{2}^{m})\}_{m}$ to some point $(x_{1},x_{2}) \in \overline{B}_{r_{1}}(x_{1}^o) \times \overline{B}_{r_{2}}(x_{2}^o)$ and also $\{t_{m}\}_{m}$ converges to some $t' \in [t^o,t^e]$.
%So,
up to a subsequence, $\{\point{t^n}{x_1^n}{x_2^n}\}_m$ converges to some point $\point{t^*}{x_{1}}{x_{2}}\in V$, which necessarily lies into the (closed) causal diamond $J^+(\point{t^o}{x_1^o}{x_2^o}) \cap J^-(\point{t^e}{x_1^e}{x_2^e})$. In conclusion, the causal diamond is sequentially compact, and so, $(V,g)$ is globally hyperbolic.
\end{proof}

%\cambios{Therefore \multiwarped spacetimes $(V,g)$ are causally continuous, stably causal, strongly causal, distinguishing, causal, chronological and non-totally vicious. And with special conditions over the Riemannian fibers, such as $L_{i}$-weakly convexity and completeness, we obtain that \multiwarped spacetimes are causally simple and globally hyperbolic.}

%%% Local Variables:
%%% mode: latex
%%% TeX-master: "DoublyWarpedBoundary2017.tex"
%%% End:
